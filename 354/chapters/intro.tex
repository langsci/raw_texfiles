\chapter{Introduction}\label{chapter: intro}


In this grammar, what we call {\iaIA} is the koine variety of spoken Eastern Armenian  that   developed       in Tehran, Iran over the last few centuries.  It has a substantial community of speakers in California.  This variety or lect is called `Persian Armenian' [pɒɻskɒhɒjeɻen] or `{\iaIA}' [iɻɒnɒhɒjeɻen] by members of the community (romanized as `Parksahayeren' and `Iranahayeren'). A speaker of this dialect (or a person descended from this community) is called a `Persian Armenian' [pɒɻskɒhɒj] or `{\iaIA}' [iɻɒnɒhɒj] (romanized as `Parskahay'  and `Iranahay').  The name is a compound of the term for Persian or Iranian, plus the compound linking vowel /-ɒ-/, and then the word for Armenian (Table \ref{tab:Intro:Name}). 

\begin{table}
	\caption{Name of the language and of the ethnic group }\label{tab:Intro:Name}
	
	\begin{tabular}{llll}
		\lsptoprule
		& Armenian & Persian Armenian & {\iaIA} \\\midrule
		Person & hɒj & pɒɻsk-ɒ-hɒj & iɻɒn-ɒ-hɒj		\\
		& \armenian{հայ}& \armenian{պարսկահայ}& \armenian{իրանահայ}\\\addlinespace
		Language & hɒjeɻen & pɒɻsk-ɒ-hɒjeɻen & iɻɒn-ɒ-hɒjeɻen		\\
		& \armenian{հայերէն}& \armenian{պարսկահայերէն}& \armenian{իրանահայերէն}\\\addlinespace
		Roots: && pɒɻsik `Persian' &iɻɒn `Iran' \\
		& &\armenian{պարսիկ} & \armenian{Իրան}\\
		\lspbottomrule
	\end{tabular}
\end{table} 

Persian Armenian  is the more conventional name for the language. It reflects the historic name for Persia used in the Armenian language, \textit{Parskastan} \armenian{Պարսկաստան}, which is still widely used by Armenians in Iran today, and the fact that the Armenian community and their dialects existed prior to the creation of the modern state of Iran. But in recent years, some circles within the  community have shifted to preferring the term ``Iranian Armenian.'' They feel that using the name ``Persian Armenian'' creates the wrong sense that either a) the Armenian variety is closely related genetically to the Persian language, or b) that these Armenians are ethnically Persian. Out of respect to this newer sentiment in the community, we use the English name “{\iaIA}”  (\iaAbbre) in this grammar to refer to this dialect. 

The present book is not a comprehensive grammar of the language. It occupies a gray zone between being a simple sketch vs. a sizable   grammar. We try to clarify the basic aspects of the language, such as its phoneme inventory, noticeable morphophonological processes, various inflectional paradigms, and some peculiar aspects of its syntax.  We likewise provide a sample text of {\iaIA} speech (\chapref{chapter:text}).  Many aspects of this variety seem  to be identical to {\seaSEA}, so we tried to focus more on those aspects of {\iaIA} which differ from that variety.  Readers are encouraged to consult \citeauthor{DumTragut-2009-ArmenianReferenceGrammar}'s (\citeyear{DumTragut-2009-ArmenianReferenceGrammar}) reference grammar of {\seaSEA} if needed. 

The introduction provides a basic typological sketch of the language (\S\ref{section: intro: overview}). We then discuss the origin of the {\iaIA} community and its demographics in \S\ref{section: intro: migration}. The community displays triglossia and we discuss the community's basic sociolinguistics in \S\ref{section: intro: socio}. We discuss how we carried out our fieldwork in \S\ref{section: intro: fieldwork} and our annotation system in \S\ref{section: intro: ortho transc gloss}. 

At the time of writing this grammar, we have made recordings of some but not all of the examples in the grammar. We have created an online archive.  We are currently holding it on GitHub, but we plan to transfer it to a more dedicated archive in the future.\footnote{\url{https://github.com/jhdeov/iranian_armenian}}  The archive consists of the following items:

\begin{itemize}\raggedright
	\item some recorded elicitations
	\item original sound files that are used in the figures in the  phonology chapter (\chapref{section phono})
	\item complete verb conjugation classes from the verb morphology chapter (\chapref{chapter: verb})
	\item the sample text from \chapref{chapter:text}
\end{itemize} 

Elicitation records were made over either Zoom, Audacity, or text messaging services (Telegram and Facebook Messenger); the recording medium does have some effects on the acoustic signal \citep{SankerWeber-2021-DontDoThiSHomeEffectRecordingDevicesAcousticAnalyssis}. The elicitations and  sample text were transcribed with Praat TextGrids \citep{boersma-2001-praat}, and then broken up with Praat scripts \citep{DiCanio-ScriptFileDivison}.  






\section{Overview of {\iaIA}}\label{section: intro: overview}


When providing a basic typological sketch of this variety, it is wise to first explain how {\iaIA} relates to other Armenian varieties. Armenian is an independent branch of the Indo-European language family. Its earliest attested ancestor is Classical Armenian of the \~{}5\textsuperscript{th} century.  The modern  varieties of Armenian  are conventionally divided into two branches: Western and Eastern. There are two   standardized dialects  that are mutually intelligible after significant exposure: {\swaSWA} ({\swaAbbre}) and {\seaSEA} ({\seaAbbre}), which we sometimes  call {\swaSW} and {\seaSE}. Both branches have dozens of   extinct, endangered, or viable non-standard varieties (\citealt{Adjarian-1909-ClassificationArmenianDialect, Adjarian-1911-DialectologyBook, GreppinKhachaturian-1986-HandbookArmenianDialectology}, \citealt[\S1.1]{Vaux-1998-ArmenianPhono}, \citealt{Baronian-2017-TwoProblemsArmenianPhono, Dolatian-prep-Adjarian}). 


Geographically, the dividing line between the two branches roughly corresponds with the Turkey-Armenia border. Dialects that developed and were spoken in the Ottoman Empire are part of the Western group, while dialects that developed in the Persian and Russian Empires constitute the Eastern branch.  {\iaIA} is part of this Eastern branch. The variety likely developed from a common ancestor between {\seaSE} and {\iaIA}. Whereas {\seaSE}  (as spoken in Yerevan) is a more conservative descendant of this ancestor, {\iaIA} has developed various innovations that we discuss in this grammar.  Despite these innovations, speakers of   {\iaIA} report  feeling that   {\iaIA} is a dialect of {\seaSE}. 

In terms of its segmental and suprasegmental phonology, {\iaIA} for the most part resembles   {\seaSEA}. Like {\seaSE} and unlike {\swaSW}, {\iaIA} has a three-way laryngeal contrast for stops and affricates, e.g., /b, p, pʰ/ as in Table \ref{tab:Intro:3wayvoicing} (\S\ref{section:phono:segmental:laryngeal cons}) \citep{Hacopian-2003-ThreeWayVOTCOntrastArmenian}. It has a two-way rhotic contrast between a trill /r/ and a retroflex approximant /ɻ/ (\S\ref{section:phono:segmental:rhotic}). It has a relatively simple vowel inventory of /ɒ, e, i, o, u, ə/, and it includes /æ/ as a marginal phoneme, mostly for Iranian loanwords (\S\ref{section:phono:segmental:vowel}).\largerpage

\begin{table}[h]
\caption{Illustrating the three-way laryngeal contrast in {\seaSE} and {\iaIA}, but not {\swaSW}} \label{tab:Intro:3wayvoicing}
\begin{tabular}{lllll}
	\lsptoprule 
	& {\iaAbbre} & {\seaAbbre} & {\swaAbbre} & \\\midrule
	`word' & \textbf{b}ɒr& \textbf{b}ɑr& \textbf{pʰ}ɑɾ & \armenian{բառ}\\  
	`cheese'   & \textbf{p}ɒniɻ& \textbf{p}ɑniɾ& \textbf{b}ɑniɾ & \armenian{պանիր} \\
	`elephant' &  \textbf{pʰ}iʁ &  \textbf{pʰ}iʁ &  \textbf{pʰ}iʁ& \armenian{փիղ}\\ 
	\lspbottomrule
\end{tabular} 
\end{table}


In terms of differences,   the {\iaIA} segments  /ɻ, ɒ/ correspond to {\seaSE} /ɾ, ɑ/, while /æ/ does not exist in {\seaSE}. These differences are likely due to contact with Persian. A significant area of difference is in question intonation:  {\iaIA} has adopted the intonation patterns of Persian when forming questions (\S\ref{section:phono:suprasegmental:intonation}). 

For morphophonology (\chapref{section:morphophono}),   {\iaIA} has grammaticalized as obligatory some processes that are optional or variable    in {\seaSE}. These involve allomorphy of the definite article (\S\ref{section:morphophono:allomorphy: det}), and a process of liquid deletion in periphrasis (\S\ref{section:morphophono:auxiliary}). Liquid deletion is a type of phonosyntactic  or syntax-sensitive phonological process (or arguably syntax-sensitive allomorphy). The liquid of the perfective converb suffix \textit{-el} or \textit{-eɻ} is deleted if the suffix does not precede the auxiliary.


For morphology, {\iaIA} has agglutinative and suffixal inflection. There is no grammatical gender. Nouns inflect for case, number, and determiners (definite, possessive), with some residue of irregular inflection. Nominal morphology is largely the same in {\seaSE} and {\iaIA}   (\chapref{chapter:noun}). 

For verbal morphology (\chapref{chapter: verb}), {\iaIA} verbs are divided into different conjugation classes based on the type of theme vowel, presence of valency suffixes (causative, passive, inchoative), and any irregularities in inflection (root suppletion, affix allomorphy, etc.).   {\iaIA} uses synthetic inflection for some parts of the verbal paradigm, but it is largely periphrastic. Like {\seaSE} and unlike {\swaSW}, {\iaIA} forms the present indicative by using a converb and an inflected auxiliary, while {\swaSW} uses a synthetic form instead (Table \ref{tab:Intro:SynthPeriph}).  

\vfill
\begin{table}[H]
	\caption{Illustrating periphrastic vs. synthetic verbal inflection across the dialects} \label{tab:Intro:SynthPeriph}
	%		\begin{tabular}{|l|ll|l|}
		%		\hline & {\iaIA} & {\seaSE} & {\swaSW}  
		%		\\
		%		\hline 
		%		`I like' & siɻ-um e-m& siɾ-um e-m& ɡə-siɾ-e-m 
		%		\\
		%		& \multicolumn{2}{l|}{like-{\impfcvb} {\auxgloss}-1{\sg}}& {\ind}-like-{\thgloss}-1{\sg}
		%		\\
		%		& \multicolumn{2}{l|}{\armenian{սիրում եմ}}
		%		& 
		%		\\ \hline 
		%	\end{tabular}
	\begin{tabular}{llll}
		\lsptoprule 
		{\iaAbbre}& siɻ-um & e-m &\armenian{սիրում եմ} \\
		{\seaAbbre} & siɾ-um & e-m &\armenian{սիրում եմ} \\
		& like-{\impfcvb} & {\auxgloss}-1{\sg} & \\\addlinespace
		{\swaAbbre}& ɡə-siɾ-e-m &  & \armenian{կը սիրեմ}\\
		&	{\ind}-like-{\thgloss}-1{\sg} & & \\\addlinespace
		& \multicolumn{2}{l}{`I like.'}& \\ \lspbottomrule
	\end{tabular}
\end{table} 
\vfill
\pagebreak


Compared to {\seaSE}, {\iaIA} has developed some significant changes in verbal inflection. The suffix /-m/ is a 1SG agreement marker for present verbs in {\seaSEA}, but this suffix has been generalized to mark the 1SG for any possible tense in {\iaIA} (\S\ref{section:verb:aux:past}). Compare  the various tenses of   `to read' in Table \ref{tab:intro:overviewVerbChange}.  And in the past perfective or aorist, {\iaIA} has developed extensive changes in what suffixes are used to mark the past and perfective/aorist morphemes  (\S\ref{section:verb:synthesis:perf}). In brief,  {\seaSEA} uses the morpheme template /-t͡sʰ-i/ for most verb classes, such as A-Class `to read' and E-Class `to sing', while it uses  /-$\emptyset$-ɑ/ for irregulars like `to eat'. Note the presence of theme vowels before /-t͡sʰ-i/, and the absence of theme vowels before  /-$\emptyset$-ɑ/.  In contrast, {\iaIA} has generalized the   /-$\emptyset$-ɒ/    pattern and uses this template for many types of regular verb classes, such as `they sang' but not `they read'.   


\begin{table}
	\caption{Illustrating changes in  verbal inflection across {\seaSE} and {\iaIA}   } \label{tab:intro:overviewVerbChange}
	\resizebox{\textwidth}{!}{%
		\begin{tabular}{lllll}
			\lsptoprule Tense & Verb & {\seaAbbre} & {\iaAbbre} & %{\swaSW} & 
			\\\midrule
			Sbjv. Pres.  1SG & `to read'& kɑɾtʰ-ɑ-m&    kɒɻtʰ-ɒ-m&read-{\thgloss}-1{\sg}\\
			&	& \armenian{կարդամ}& \armenian{կարդամ}&% \armenian{կարդացի} & 
			\\
			Sbjv. Past     1SG & `to read' & kɑɾtʰ-ɑj-i-$\emptyset$& kɒɻtʰ-ɒj-i-m&   read-{\thgloss}-{\pst}-1{\sg}\\
			&	&  \armenian{կարդայի}&\armenian{կարդայիմ}&% \armenian{կարդացի} & 
			\\
			\midrule
			Past Pfv.   3PL & `to read' & kɑɾtʰ-ɑ-t͡sʰ-i-n& kɒɻtʰ-ɒ-t͡sʰ-i-n&   read-{\thgloss}-{\aorperf}-{\pst}-3{\pl}
			\\
			&	& \armenian{կարդացին}& \armenian{կարդացին}&% \armenian{կարդացի} & 
			\\
			Past Pfv.   3PL & `to sing'  & jeɾkʰ-e-t͡sʰ-i-n&  jeɻkʰ-$\emptyset$-$\emptyset$-ɒ-n &  sing-{\thgloss}-{\aorperf}-{\pst}-3{\pl}
			\\
			&	& \armenian{երգեցին}&\armenian{երգան}& % \armenian{երգեցի} & 
			\\
			Past Pfv.   3PL &`to eat'&  keɾ-$\emptyset$-$\emptyset$-ɑ-n
			&keɻ-$\emptyset$-$\emptyset$-ɒ-n    
			& eat-{\thgloss}-{\aorperf}-{\pst}-3{\pl}
			\\
			&	& \armenian{կերան}& \armenian{կերան}&% \armenian{կերայ} & 
			\\
			\lspbottomrule
		\end{tabular}}
\end{table} 

In terms of syntax (\chapref{chapter:syntax}), we have not been able to carry out an extensive study of {\iaIA}. Based on intuitions of our speakers, it seems that {\seaSE} and {\iaIA}  have relatively few significant syntactic differences. Like {\seaSEA}, {\iaIA} is primarily an SOV language but with free word order. One important area of commonality is that  the copula is   a mobile auxiliary in {\seaSE} and {\iaIA} but not in {\swaSW}  \citep{KahnemuyipourMegerdoomian-2011-secondcliticvP}. The auxiliary is added to focused words in {\seaSE} and {\iaIA} (\tabref{tab:Intro:MobileClitic}).   

\begin{table}
	\caption{Mobile clitic in {\seaSE} and {\iaIA} but not {\swaSW} } \label{tab:Intro:MobileClitic}
	\begin{tabular}{lllll}
		\lsptoprule 
		{\iaAbbre}& \textbf{mɒɻjɒ-n}& ɒ &uɻɒχ & \armenian{Մարիան ա ուրախ։}\\
		{\seaAbbre} & \textbf{mɑɾjɑ-n}& e &uɾɑχ& \armenian{Մարիան է ուրախ։}\\
		& Maria-{\defgloss} & {\auxgloss} & happy& \\ \addlinespace
		{\swaAbbre}& \textbf{mɑɾjɑ-n}  &uɾɑχ& e& \armenian{Մարիան  ուրախ է։} \\
		& Maria-{\defgloss}  & happy &  {\auxgloss}&\\ \addlinespace  
		& \multicolumn{2}{l}{`MARIA is happy.'} & & \\ 
		\lspbottomrule
	\end{tabular}
\end{table} 

There are some syntactic differences that we have noted. Due to contact with Persian, {\iaIA} can use the second person possessive suffix \textit{-t} to act as an object clitic. No such use is attested for the other persons.  There are other minor innovations in relative clause formation, again mostly due to Persian contact. 


In terms of its lexicon, we have not found any major differences between {\seaSE} and {\iaIA}. Because of contact and sometimes bilingualism with Persian, {\iaIA} speakers tell us that they often use Persian words for some concepts, such as for various plants or spices. The community has likewise borrowed  some Persian phrases and turned them into Armenian phrases, i.e., calques.  

For example, the following phrases in Table \ref{tab:Intro:PersianCalque} are common phrases in Persian; they are syntactically complex predicates made up of a word and light verb.\footnote{For the Persian borrowing [pʰæχʃ], NK felt that this word meant nothing outside of the context of the calqued phrase in Table \ref{tab:Intro:PersianCalque}. So we are not sure if this word should be translated as `broadcast' or not. } Armenian speakers have adopted these phrases and just replaced the light verb with an Armenian equivalent. These phrases are known even by young members of the California diaspora who speak {\iaIA} but not Persian.\footnote{Persian IPA     is taken from Wiktionary, verified by Koorosh Ariyaee.}



\begin{table}
	\caption{Calqued phrases from Persian to {\iaIA}}\label{tab:Intro:PersianCalque}
	\begin{tabular}{llllll}
		\lsptoprule   
		& \multicolumn{3}{l}{Persian}& \multicolumn{2}{l}{ {\iaIA}} \\
		\midrule  `to take a nap' & t͡ʃʰoɾt& zædæn& \textarab{چرت زدن }  & t͡ʃʰoɾtʰ  &χəpʰel 
		\\
		& nap &hit && nap& hit\\\addlinespace
		`to broadcast' & pʰæxʃ &kærdæn & \textarab{پخش كردن } &  pʰæχʃ& ɒnel\\
		& broadcast &do& & X & do\\\addlinespace
		`to shower' & duʃ&ɡeɾeftæn & \textarab{دوش گرفتن } & duʃ& bərnel\\
		& shower &catch && shower &catch
		\\ \lspbottomrule
	\end{tabular}
\end{table} 




Unfortunately due to lack of time and resources, we haven't been able to carry out an extensive study of such phrases in {\iaIA}. See \citet{Afsheen-Blog} and our sample text  (\chapref{chapter:text}) for  more examples of calques and borrowed words.\largerpage[2]


Finally, {\iaIA} is under-described as a language. To our knowledge, the only manuscript that even has data on this variety is    \citet{ShakibiBonyadi-1995-ShortSurveyArmenianLanguageTehrani}. This manuscript provides some sample paradigms, and a large glossary of {\iaIA}. However, this document seems to actually describe a type of code switching or mixing  between {\iaIA} and {\seaSEA}.  For example, that manuscript  uses some {\iaIA} features like the 1SG suffix \textit{-m}, but it also uses more {\seaSEA} features like using the Eastern style of marking the past perfective.\footnote{\citet{ShakibiBonyadi-1995-ShortSurveyArmenianLanguageTehrani} do not represent the three-way laryngeal contrast for stops and affricates. We suspect that this is because this manuscript seems to have developed without using linguistic sources on   Armenian (which would state that there is such a distinction), and that the authors of this manuscript likely don't speak Armenian. } As we discuss later, {\seaSE} and {\iaIA}  are two registers of Armenian as spoken by the {\iaIA} community in a type of diglossia.



\section{Migration history and dialect classification}\label{section: intro: migration}


Armenians have  had a long historical presence in Persia or Iran. We briefly review this history in order to later illustrate the   sociolinguistic situation of the modern community. 

Ethnic Armenians have been in contact with Persian or Iranian culture since antiquity, since at least the 6\textsuperscript{th} century BCE (\citealt[421]{Dekmejian-1997-ArmenianDiaspora}, \citealt[1]{Hovhannisian-2021-ArmenianCommunityPersianIranIntro}). Because of this historic contact, there has been extensive language contact between Armenian and Iranian languages, particularly Parthian and Middle Persian   \citep[\S1]{Meyer-2017-dissIranianArmenianContact5ThCentury}.  There have been villages or areas in modern-day Iran with historically large Armenian populations, especially in Northwest Iran or Iranian Azerbaijan such as Tabriz. These   villages,  towns, and districts developed their own dialects or Armenian varieties. These varieties   differ significantly from {\seaSEA} and from (Tehrani) {\iaIA}.  


An incomplete list of some   area-specific   varieties include  Maku \citep{Katvalyan-2018-Maku}, Maragha \citep{Adjarian-1926-MaraghaDialect},  New Julfa \citep{Adjarian-1940-NewJulfaDialect,Vaux-prep-NewJulfa}, Salmast \citep{Vaux-Salmast}, and Urmia/Khoy \citep{Asatryan-1962-KhoyUrmiaDialect}. For an overview of these dialects, see  \citet[85]{Martirosyan-2019-ArmenianDialectsBigVersionRussianJournal,Martirosyan-2019-Armeniandialects}.  These dialects constitute the historical region of ``Persian Armenia'', called [pɑɾskɑhɑjkʰ] \armenian{Պարկսահայք} in {\seaSEA}  \citep{Martirosyan-prep-LinguCulturalPersianArmenian}.   For an overview of the migration patterns of these dialects, see \citet{HaykanushMesropian-Blog}.  For lists and historical overviews of  past and present Armenian villages and districts, see   \citet{EncyclopediaIranica-ModernArmeniansIran} and  \citet{Ghougassian-2021-ArmenianRuralSettlementNewJulfa}. For in-depth historical and anthropological overviews of the Armenian community in Iran, see \citet{Chaqueri-1998-ArmeniansIran},  \citet{sanasarian-2000-religiousMinoritiesIran},  and \citet{Barry-2017-ReGhettoArmenianTehran,barry-2018-armenianArmeniansIran}. There is likewise recent work on language signage in Armenian-populated areas \citep{RezaeiTadayyon-2018-LinguisticLandscapeIsfahanIranJulfa}.  

In terms of demographics, the ancestors of most  modern {\iaIA}s entered Iran via mass migrations (\citealt[19]{Kouymjian-1997-ArmeniaFromCiliciaShah}, \citealt[3]{Hovhannisian-2021-ArmenianCommunityPersianIranIntro}). In the 1600s, Shah Abbas I of Persia forced the mass migration of ethnic Armenians from historical Eastern Armenia, especially  from modern-day Nakhchivan or Nakhijevan (\armenian{Նախիջեւան}). The number of these Armenians is estimated as 400,000 being deported to Iran in 1604, of  which 300,000 individuals survived by 1606 \citep[314]{Ghougassian-2021-ArmenianRuralSettlementNewJulfa}.   These Armenians then settled in different regions of Iran, especially in Tabriz and in the New Julfa quarter of Isfahan, which had been constructed specifically for their resettlement \citep[9]{Hovhannisian-2021-ArmenianCommunityPersianIranIntro}. Other areas where Armenians were settled in Safavid times included Peria (Fereydan), Chaharmahal, and Buurvari, while thereafter New Julfan trade networks gave rise to Armenian communities in other urban centers throughout Persia and as far as Astrakhan, India, Burma/Myanmar, and Java. 

Over time, large numbers of Armenians then moved to Tehran sometime in the 19\textsuperscript{th} and early 20\textsuperscript{th} centuries,  drawn by better prospects for wealth and social mobility. \citep[6]{Hovhannisian-2021-ArmenianCommunityPersianIranIntro}. Then in the mid to late 20\textsuperscript{th} century, particularly around the time of the Islamic Revolution (1979), large numbers of Armenians emigrated from Tehran to elsewhere around the globe, especially to   Los Angeles county in Southern California. 

In terms of contemporary population size, it is difficult to get clear numbers \citep{Iskandaryan-2019-ArmeniancommunityIranIssuesEMigration}.\footnote{To illustrate this, see the inconsistent population estimates on the Wikipedia page for Iranian Armenians: {\url{https://en.wikipedia.org/wiki/Iranian_Armenians}}}  Some sources estimate that the Armenian population of Tehran reached a peak of 50,000 people in the late decades of the 20\textsuperscript{th} century \citep[6]{Hovhannisian-2021-ArmenianCommunityPersianIranIntro}.  The US government gives larger numbers. \citet[101]{CurtisHooglund-2008-IranCountryStudy}  estimate  that   the size of the Armenian population in Iran was around 350,000 in 1979 (prior to the revolution). Emigration then led to a population count of 300,000 in 2000. They report that   65\% of the population lived in Tehran, around 195,000. 

As for the {\iaIA} diaspora, {\iaIA}s are a culturally significant subset of the Armenian population in California \citep{Bakalian-2017-ArmenianArmenian}.  The US census lists 47,197 individuals in California who report themselves as Armenians  born in Iran \citep{uscensusbureau:2015-foreignborn}. For more in-depth socio-economic, demographic, and anthropological studies of the California population, see \citet{DerMartirosian-2021-EconomicSocialIntegrationArmenianIranianSouthernCalifornia} and 
\citet{Fittante-2017-ButWhyGlendaleHistoryArmenian,Fittante-2018-ArmeniansGlendale,Fittante-2019-ConstructivistTheoriesPolitical}.






Because of these complicated demographic changes, it is possible that modern Tehrani {\iaIA} developed as an offshoot of {\seaSEA}. The Tehrani variety had some degree of contact with the varieties of other Armenian villages in Iran over the centuries. Over time, as Armenians moved within Iran to Tehran, the Tehrani community levelled their speech to form modern-day Tehrani {\iaIA}. This   modern variety is what we refer to as {\iaIA}. This is the variety that is spoken and acquired by Armenian children in Tehran, and in the large {\iaIA} diaspora. 

Because {\iaIA} is a spoken vernacular, there are only scant records of it. Within Armenian philology, the earliest reference we have found for Tehrani {\iaIA} is in the introduction chapter of  Adjarian 1940 \citep{Adjarian-1940-NewJulfaDialect}, which is a  grammar of New Julfa Armenian (translated into English in \citealt{Vaux-prep-NewJulfa}).  For that grammar, Adjarian collected data from native speakers on a visit to New Julfa in 1919. That variety is spoken  primarily in the New Julfa district of Isfahan. He contrasts New Julfa Armenian with what he calls ``Persian Armenian'' or ``Perso-Armenian'' which he says is spoken in the northern regions of Iran, including Tehran. He doesn't provide any data on this dialect but he states that  this Perso-Armenian lect is socially predominant  and close to Yerevan Armenian. We suspect that what he calls Perso-Armenian is the direct ancestor of modern Tehrani {\iaIA}.

Based on conventional dialectological work in Armenian \citep{Adjarian-1911-DialectologyBook}, the ancestor of {\seaSEA} is often assumed to be the dialect of Old Yerevan Armenian \citep{Dolatian-prep-Adjarian}, though the exact genetic relationship is complicated \citep{SayeedVaux-2017-EvolutionArmenian}. Tehrani   {\iaIA} may have developed as a subdialect of 16\textsuperscript{th} century Yerevan Armenian, or a koine that arose via mingling Yerevan and {\seaAbbre} with other migrant communities (like Julfa Armenians) and the pre-existing Armenian dialects of Iran (such as in Maragha, Khoy, and others). Based on migration patterns from Armenia to Iran, Tehrani {\iaAbbre}   may be viewed as a daughter of the 16\textsuperscript{th} century dialects of Nakhichevan and Atropatene (Iranian Azerbaijan, Atrpatakan), having differentiated further within Iran over the centuries, and more recently having been subjected to prescriptive influences from {\seaAbbre} and modern Yerevan Armenian through education and literary and broadcast media. Moreover, we believe that the koineization of multiple Iranian Armenian dialects in Tehran during the 20\textsuperscript{th} century was compounded by improved schooling in {\seaAbbre} and increased cultural output from Yerevan. This led to leveling the more salient features of the lect and has in turn brought Tehrani {\iaIA} (the Tehran koine) closer to {\seaAbbre} than to other local Iranian Armenian dialects like New Julfa Armenian. This is reflected in the tendency for some older speakers to employ more Persianisms and retain dialectical forms from their hometowns throughout Iran. Because of the close contact between {\iaIA} and {\seaAbbre}, {\iaAbbre} speakers likewise report perceiving that {\iaAbbre} is a dialect of {\seaAbbre}. Adjarian himself discusses some difficulties in classifying {\iaAbbre}, while using the name of Perso-Armenian \citep[\S1]{Adjarian-1940-NewJulfaDialect}, where he says that Perso-Armenian is related to Tabriz and Astrakhan Armenian.



\section{Sociolinguistics of the {\iaIA} community }\label{section: intro: socio}

The Tehran community is diglossic or triglossic \citep{Nercissians-1988-lifeCultureArmeniansIran,Nercissians-2012-lifeCultureArmeniansIran}. Armenians learn and speak Persian with non-Armenians, and code switching is common \citep{GhiasianRezayi-2014-PersianArmenianCodeswitching,GhiasianRazaei-2014-StudyPersianMultifunctionalDiscourseMarkerArmenianPersianBilinguals}. Within the Armenian community, children acquire {\iaIA} at home. This variety is spoken as an informal register. In Armenian schools, children learn {\seaSEA}. The community uses {\seaSEA} as a formal register  in literature, newspapers, written communications, and formal speech. We discuss each code in \S\ref{section: intro: socio: characteristics}, and then discuss the social stigmatization of the spoken vernacular with respect to    {\seaSEA} (\S\ref{section: intro: socio: stigma}). 

Whereas the {\iaIA} community in Tehran is diglossic or triglossic, the {\iaIA} diaspora is much less so. For the diaspora  in California,   families may speak {\iaIA} at home, but not necessarily {\seaSE} or Persian. The relative rarity of transmitting Persian to the youth makes sense because it is not a lingua franca among Armenians in the US. As for the Armenian registers,  {\seaSEA} is the formal register, while {\iaIA} is the informal register. Thus the children of such communities acquire {\iaIA} at home. Some but not all diaspora children attend Armenian schools where they    acquire {\seaSE}. 

\subsection{Characteristics  of the three codes}\label{section: intro: socio: characteristics}

For Persian,  \citet[\S6.7]{zamir-1982-variationStandardPersianSociolinguisticStudy} reports that Tehrani Armenians spoke a distinctive   dialect of Persian. Their dialect involved various phonological changes. For example, standard Persian /æ/ was pronounced as /ɒ/ by speakers of this dialect \citep[370]{zamir-1982-variationStandardPersianSociolinguisticStudy}; the history of /æ/ is discussed more in \S\ref{section:phono:segmental:vowel}.  Afsheen Sharifzadeh  (AS) and others report  that this Persian dialect     died out over the last few decades \citep[154]{Barry-2017-Monologue}. This dialect  is now more characteristic of the current generation's grandparents or great-grandparents, i.e., people who were adults around the time of \citet{zamir-1982-variationStandardPersianSociolinguisticStudy}'s study. 

The modern community still has some level of awareness of this old  dialect however; for example, the phonological accent of this old dialect is   satirized in the work of {\iaIA} comedian  Gilbert Sinanian (Gibo Hopar).\footnote{\url{https://www.facebook.com/gibohopar/}} For the modern community, speakers seem to use the same dialect of Persian as non-Armenians but with some noticeable phonological features. For example, \citet[220]{barry-2018-armenianArmeniansIran} reports: 

\begin{quote}
	Furthermore, the Armenian accent is not simply something of which {\iaIA}s are self conscious; Muslim Iranians recognise it also. Two Iranian students in Melbourne stated that the Armenian accent in Persian is easily recognisable in its intonation.
\end{quote}

There are similar reports on Armenian-accented Persian among   Isfahan Armenians \citep{Rezaei-Farnia-2023-ArmenianLanguageIdentityIranCaseIranianArmeniansIsfahan}. Though  the exact linguistic features of this accent are unclear to us.  In AS’s experience, some members of the {\iaIA} speech community utilize the {\iaAbbre} approximant /ɻ/ when speaking Persian, to an extent that is easily recognizable and stereotypical of their speech among Iranians \citep[220]{barry-2018-armenianArmeniansIran}. 


As for the informal register of {\iaIA}, this variety is natively acquired at home by speakers in the Tehran community. Outside of Tehran, various people have told us that the Tehrani variety is known in other Armenian-populated towns and villages in Iran. For example,  \citet[64]{Nercissians-2001-bilingualismDiglossiaEthnicMinorityTehran} explicitly states that ``there is a clearly prestigious Tehrani dialect for Armenian.'' Specifically, spoken Tehrani {\iaIA} is more prestigious than the spoken vernacular of other towns and villages, such as Isfahan, Tabriz, and so on. 

It   seems that other Armenian varieties in Iran are dying out and being replaced by Tehrani {\iaIA}. For example, in AS's travels through Iran, he's found that many young people in New Julfa (Isfahan)  no longer speak the New Julfa variety of their ancestors. Instead, the current generation speaks the Tehrani variety. The parents of this generation speak Tehrani and the local New Julfa vernacular; while the grandparents of this generation speak only the New Julfa vernacular. 

Because of the prestige and language shifts, AS suggests that  Tehrani {\iaIA} has become a spoken koine or lingua franca among Armenians in Iran. The social prominence of Tehrani has likewise spread throughout the Iranian Armenian community in Los Angeles. Here,  Varand Nikolaian (\citeyear{Nikolaian-2016-DialectsArmenianIraniandiasporastudyhierarchicalinteractionIranianArmeniandialects}, p.c.) reports   that the Tehrani variety is quite prominent among Iranian Armenians. In Los Angeles, Iranian Armenians from Isfahan, Tabriz, and other areas often shift to speaking  Tehrani {\iaIA} when talking to   Iranian Armenians from other villages or towns. Some people likewise feel ashamed of their own local vernacular and have shifted to using Tehrani {\iaIA} even in their own homes. 

As for the formal register,   it's more accurate to say that the formal register is {\seaSEA} with an \textit{{\iaIA}  accent}. That is, the community would say a  {\seaSEA} sentence but use {\iaIA} phonology, such as using the rounded {\iaIA} /ɒ/ instead of unrounded {\seaSE} /ɑ/. 


\subsection{Social stigmatization of the spoken vernacular}\label{section: intro: socio: stigma}
As a last note on   sociolinguistics,  we must mention the social status of {\iaIA} with respect to {\seaSEA}. Because of the  diglossic situation in Tehran, the spoken vernacular of Tehrani {\iaIA}s   is often stigmatized as  ``wrong'', ``broken'', or  ``vulgar'' speech, especially by the older generation of speakers. For example, in the early 2000s, one of the present authors (Bert Vaux, BV)    gave a conference presentation at UCLA (University of California, Los Angeles) on  {\iaIA}. Before the conference, he received an aggressive email from a member of the {\iaIA} community in California. We repeat  parts of that email below, anonymized. We re-transcribe  Armenian words in IPA. Bolding is our own; Persian words are romanized in italics.   

\begin{quote}
	\begin{sloppypar}
	I am writing to you to express my deep concern about your thesis of the third  literary dialect (the Persian-Armenian). The examples you cite to prove your  findings, [ɡənɒt͡sʰim], [imɒt͡sʰɒm] (instead of [ɡənɒt͡sʰi, imɒt͡sʰɒ]), [me]      (instead of [mek]) \textbf{are all dialectal forms}, they are used in spoken language but  never, never, never in print. You mention the printed material before the  revolution. I have not seen one example with such \textbf{vulgar errors}. As to [lev] or [lɒf]  instead of [lɒv], this is truly \textbf{unheard of}. These are all spoken forms by  \textbf{not-very-educated people} in Iran and those who are here, and there are many. As to the  words \textit{havich}, \textit{xiar}, \textit{jafari}, \textit{xiarshur}, these are purely Persian words (not even  borrowings) and \textbf{nonexistent} in the spoken language let alone in the Persian  Armenian literary dialect which I think, such an animal does not exist at all... 
	\end{sloppypar}
	
	Please \textbf{check your sources} before coming to these conclusions. I  consider myself an educated {\iaIA}, who writes in Eastern Armenian  literary language (and there is \textbf{non} [\textit{sic}] \textbf{other variations}) and also speaks with some  dialectal forms but never mixes Persian words.  
	
	
	Your question of what form of literary language is/was taught in schools in  Iran. I am very much familiar with the textbooks used in Iran before the  revolution and after. The text, the syntax, the lexicon, and the grammar is that of  {\seaSEA} literary language. The same standards are used also  in the media. I beg you again, revisit your findings and conclusions. Your  presentation may \textbf{irritate} many {\iaIA}s. I was hoping you would speak  about a \textbf{distinct} dialect of {\iaIA}s, like the Maragha dialect (the \textit{er}  branch: [etɑs eɾ] meaning I am going) or the Gharadagh dialect that is close  to the Gharabagh dialect.
	
	[Correction by BV: No one uses /etɑs eɾ/. Khoy/Urmia/Salmast have /eɾtʰɑs em/ `I am going' and /eɾtʰɑs em eɾ/  `I was going'. Maragha uses /etʰæli im/ `I am going' and /etʰæli im eɾ/ `I was going'.]
\end{quote}


As is clear, the email shows that the spoken vernacular is extremely stigmatized by at least some members of higher social classes. The dialect is considered ``vulgar'', ``un-educated'', or even ``non-existent''. Paradoxically, the {\iaIA} community legitimizes Armenian varieties that are spoken in the more peripheral   areas of Iran. These varieties are deemed ``exotic'' and un-intelligible enough for Tehranis to consider them as legitimate languages. In contrast, the spoken language of the average Tehrani child or adult is erased. People pretend they don't speak this spoken vernacular, even though they do. 






\section{Fieldwork and language consultants}\label{section: intro: fieldwork}
This grammar is based on fieldwork that was done by each of the authors, at different times, and with different people. We go through each phase of fieldwork below.

The first phase of fieldwork was undertaken in the 1990s and early 2000s by Bert Vaux (BV). BV is a trained generative phonologist and is a native speaker of English.  He undertook fieldwork by collecting data from Armenian expatriates from Iran, especially in    Boston and Los Angeles. 

BV's  main consultant was  Karine Megerdoomian (KM, female), who was born and raised in Tehran   until the age of 13. There, she acquired {\iaIA}, {\seaSEA}, and Persian. After that, she moved  across Europe and North America until finally settling in the United States. KM is a trained generative syntactician and thus often gave meta-linguistic judgments as  a linguist-speaker.  At the time of BV's fieldwork, KM was in her early 30s.  

BV also elicited data from other {\iaIA} expatriates living in the US and Europe. One such consultant is  AP. AP is a male from Peria, which is in the province of Isfahan, Iran. His judgments were relayed to BV through AP's wife. 



The second phase was undertaken by  Afsheen Sharifzadeh (AS). AS is a self-trained linguist and is  a native speaker of Persian and English. His fieldwork was somewhat atypical.  He initially was interested in merely learning the Armenian culture and language.  He often visited the Armenian  community in Iran and would befriend {\iaIA} speakers. His exposure was some time in the late 2000s and early 2010s. Over time, he developed an advanced proficiency in {\seaSEA} and {\iaIA}. His data comes from his interactions with a wide community of {\iaIA} speakers, both in Tehran and in expatriate communities in the US. His main consultants were people in their early to late 20s. 

The third phase was undertaken by  Hossep Dolatian (HD). HD is a trained generative morphophonologist and is a native speaker of {\swaSWA}. He did fieldwork after discovering the data collected by BV and AS. He then undertook the task of synthesizing   their data and replicating it with speakers of {\iaIA} in California.  He did fieldwork in 2021 and his main consultant was  Nicole Khachikian (NK, female). Her parents and grandparents are from Tehran.  She was born and raised in the US outside of Los Angeles, but was often within the Iranian Armenian community of LA.  Her home languages were {\iaIA} and English. She does not know Persian. She learned aspects of {\seaSEA} both  by a) learning the spoken formal register of {\seaSEA} with the larger Armenian community in Los Angeles, and b) taking Armenian classes at university.  She was in her early 20s during HD's fieldwork. HD at times elicited data from KM, who was in her early 50s in 2021. Recordings were made remotely,  either with Praat \citep{boersma-2001-praat} over Zoom or with Audacity. HD's recording methodology is documented on the associated archive of this grammar.

For some data points, HD elicited material on {\seaSEA}   in order to show a contrast between {\seaSE} and {\iaIA}. Some other {\iaAbbre}-speaking linguists were also consulted at times. Elicitations were done with the following speakers:

\begin{itemize}
	\item Eastern Armenian
	\begin{itemize}
	\item 	Mariam Asatryan (MA): female; born and raised in Tsovasar, Armenia, age was around late 20s. 
	\item Victoria Khurshudyan (VK): female; born and raised in Goris, Armenia, age was around early 40s. 
	\item Vahagn Petrosyan (VP): male; born and raised in Yerevan, Armenia; age was around mid 30s. 
	\item    Arevik Torosyan (AT): female; born and raised in Yerevan, Armenia up until her late teens; age was around early 20s. 
\end{itemize}
\item Iranian Armenian
\begin{itemize}
 \item Anooshik Melikian (AM): female; born and raised in Tehran, Iran up until 2016; age was around early 50s.
	\item Garoun Engström (GE): female; born and raised in Uppsala, Sweden; age was around early 30s.
	
\end{itemize}
\end{itemize}

As is clear, the three linguists did their fieldwork at different times and locations. However, we have found little to no discrepancies across these different pools of data. The main differences come from generational changes in the pronunciation of certain lexical items and morphemes, which we take note of. 

Furthermore, neither BV, AS, nor HD are native speakers of {\seaSEA} or {\iaIA}.  BV's  and AS's data come from speakers who can be considered bi-dialectal, which means the speakers are proficient in both {\iaIA} and {\seaSEA}. This is because their speakers were born and raised in Iran and thus were exposed to {\seaSEA} within the education system of the Armenian community. In contrast, HD's main consultants are mono-lectal and mainly speak {\iaIA}. Because HD's consultants grew up in the US, his speakers did not acquire {\seaSEA} within an educational system.  We have found only  minor   differences between the grammars of bi-dialectal vs. mono-lectal speakers when it comes to {\iaIA} judgments or pronunciations.

\section{Orthography, transcription, and glossing}\label{section: intro: ortho transc gloss}
The Armenian language is normally written in the Armenian script \citep{sanjian-1996-armenianAlphabet}. There are two orthographic conventions or spelling systems for Armenian: Classical and Reformed. The Classical system is the original system of writing the Armenian script. It is used for {\swaSW}. It was originally used for {\seaSEA} as well, but then a series of Soviet-era spelling reforms created the Reformed system. The Reformed system is used for {\seaSEA} as spoken in Armenia and large parts of the Diaspora. But in Iran, {\seaSEA} is still written with the Classical system. For an overview of these orthographic changes, see \citet[5--6, 12]{DumTragut-2009-ArmenianReferenceGrammar}.

For this grammar, we use the Reformed spelling to write {\seaSEA} examples. We use Classical spelling to write {\iaIA} examples out of respect to the community's orthographic customs. This is somewhat atypical because {\iaIA} is an unwritten vernacular. We have decided to provide orthographic forms to make future cross-dialectal work easier. Note that the orthographic script does not indicate all phonetic aspects of {\iaIA} pronunciation. All data is likewise transcribed in IPA. 

For our glossed sentences, we first provide an IPA transcription, then gloss, then translation, and then the orthographic representation.


For glossing, we follow the   Leipzig Glossing Rules, and we've added our own conventions for those morphosyntactic features that are absent from the Leipzig Glossing Rules. 

In this grammar,  we adopt a simple item-and-arrangement model of morphology \citep{Hockett-1942-SystemDescriptivePhono}. We try to segment  as many affixes as possible. We adopt the word “morph” as a theory-neutral term to denote the surface form of morphemes, i.e., to simply denote  morphological items \citep{Haspelmath-2020-MorphMinimalLinguisticForm}. We at times provide realization rules to more clearly show how certain inflectional features are marked in {\iaIA}; these rules should not be treated as explicit formal theoretical rules.  

Full morpheme segmentation and glosses are given for sentences and for morphological paradigms.  In the morphology section, we likewise segment zero morphemes. We generally avoid segmentation for the data in the phonology chapter in order to reduce clutter. Outside of the morphology chapter, we often segment the    3SG auxiliary (positive \textit{ɒ} and negative \textit{t͡ʃ-i}) as just   `({\neggloss})-{\auxgloss}' instead of  `({\neggloss})-{\auxgloss}.{\prs}.3{\sg}' to reduce clutter. 

For our bibliography, we do not romanize or transliterate Armenian entries. All Armenian entries are given in the Armenian alphabet, so that searching for those entries in the future (via library catalogs) is easier. Translations are provided to help preview the content of the entry. 
