\title{Absolute Komplexität in der Nominalflexion}  %look no further, you can change those things right here.
\subtitle{Althochdeutsch, Mittelhochdeutsch, Alemannisch und deutsche Standardsprache}
% % \BackTitle{Change your backtitle in localmetadata.tex} % Change if BackTitle != Title
\BackBody{Diese Arbeit quantifiziert und analysiert die absolute Komplexität in der Nominalflexion von 17 alemannischen Dialekten, der deutschen Standardsprache sowie des Mittel- und Althochdeutschen. Als Datengrundlage dienen Ortsgrammatiken. Die theoretische Grundlage bilden LFG (Lexical-Functional Grammar) und die inferentielle-realisierende Morphologie, wovon abgeleitet wird, was ein System komplexer bzw. simpler macht. Auf diesen Modellen basiert auch die eigens entwickelte Methode zur Messung der morphologischen Komplexität. Die Variation in der Komplexität der Nominalflexion wird anhand der folgenden Faktoren analysiert: Diachronie, Isolation, Kontakt, Standardisierung und Dialektgruppen.}
%\dedication{Change dedication in localmetadata.tex}
\typesetter{Felix Kopecky}
\proofreader{Oliver Schallert,
Lea Schäfer,
Timm Lichte,
Anne Kilgus,
Kilu von Prince,
Jean Nitzke,
Ann-Marie  Moser,
Katja Politt,
Thilo Weber,
Andreas Hölzl,
Simon Pröll,
Pia Bergmann}
\author{Raffaela Baechler}
\BookDOI{10.5281/zenodo.1067783}%ask coordinator for DOI
\renewcommand{\lsISBNdigital}{978-3-96110-022-4}
\renewcommand{\lsISBNhardcover}{978-3-96110-023-1}
\renewcommand{\lsISBNsoftcover}{000-0-000000-00-0}
\renewcommand{\lsISBNsoftcoverus}{000-0-000000-00-0}
\renewcommand{\lsSeries}{mi} % use lowercase acronym, e.g. sidl, eotms, tgdi
\renewcommand{\lsSeriesNumber}{2} %will be assigned when the book enters the proofreading stage
\renewcommand{\lsURL}{http://langsci-press.org/catalog/book/134} % contact the coordinator for the right number


\setlength{\csspine}{25.0559784mm} % Please calculate: Total Page Number (excluding cover, usually (Total Page - 3)) * 0.0572008 mm
\setlength{\bodspine}{20mm} % Please calculate: Total Page Number (excluding cover) * BODFACTOR + BODABS

\renewcommand{\lsLanguageIndexTitle}{Sprachregister}	% This can be changed according to the language.
\renewcommand{\lsSubjectIndexTitle}{Sachregister}
\renewcommand{\lsNameIndexTitle}{Personenregister}

\defbibheading{references}{\chapter{Literaturverzeichnis}}