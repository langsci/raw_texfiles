\chapter{Hypothesen und Forschungsdesign}\label{3}

Dieses Kapitel steckt den Rahmen der vorliegenden Arbeit ab. In \sectref{3.1} wird argumentiert, weshalb die Komplexität der Nominalflexion und welche Arten von Komplexität gemessen werden. In \sectref{3.2} werden die Hypothesen vorgestellt sowie anhand welcher Varietäten diese Hypothesen überprüft werden. Die Charakteristika der untersuchten Varietäten werden in \sectref{3.3} präsentiert.

\section{Absolute Komplexität in der Nominalflexion}\label{3.1}

Das Ziel dieses Kapitels ist es, zuerst zu erklären, weshalb hier die  der nominalen Flexionsmorphologie gemessen wird (und z.\,B.\ nicht die verbale Flexionsmorphologie). In einem zweiten Schritt wird erläutert, welche Arten der Komplexität untersucht werden, und zwar auf der Basis der Einteilung von \citet{Miestamo2008} sowie \citet{Rescher1998} und \citet{Sinnemäki2011}. Diese Arten der Komplexität wurden in \sectref{2.2.5} eingeführt.

\subsection{Nominalflexion}\label{3.1.1}

In der vorliegenden Arbeit soll die Komplexität der nominalen Flexionsmorphologie gemessen werden, und zwar in den folgenden Wortarten: \isi{Substantive}, starke und schwache \isi{Adjektive}, \isi{Personalpronomen}, \isi{Interrogativpronomen}, bestimmter und \isi{unbestimmter Artikel}, einfaches \isi{Demonstrativpronomen} und \isi{Possessivpronomen}. Begründet wird diese Auswahl der Wortarten in \sectref{4.3.2}. Im Gegensatz zur verbalen Flexionsmorphologie bietet die nominale Flexionsmorphologie vor allem zwei Vorteile: Erstens ist die Variationsbreite zwischen den Varietäten größer und zweitens können innerhalb des nominalen Bereichs unterschiedliche Wortarten miteinander verglichen werden.\largerpage[1.5]

Die große Variationsbreite in der Nominalflexion kommt besonders dadurch zustande, dass es keinen diachronen Hauptprozess gibt, der in allen der hier untersuchten Varietäten gewirkt und gleiche Resultate verursacht hat.\footnote{Von einem der wichtigsten diachronen Prozesse in der Verbalflexion jedoch, nämlich vom Präteritumschwund, sind alle alemannischen Dialekte gleichermaßen betroffen. Eine Ausnahme davon bildete die Sprachinsel Saley/Salecchio, in der heute jedoch kein Alemannisch mehr gesprochen wird (herzlichen Dank an Oliver Schallert für diesen Hinweis).} Ein vermeintlicher prominenter Hauptprozess betrifft die Setzung des germanischen Initialakzents und die daraus resultierende Schwächung der Vollvokale im Nebenton, was zum Verlust von Kasusmarkierung (besonders am \isi{Substantiv} und \isi{Adjektiv}) geführt hat. Diese Hypothese könnte zutreffen, vergleicht man z.\,B.\ die althochdeutsche Kasusmorphologie (mit /\textit{a}/, /\textit{e}/, /\textit{i}/, /\textit{o}/ und /\textit{u}/ im Nebenton, \citealt[61]{Braune2004}) mit jener der nieder- und hochalemannischen Dialekte (mit /\textit{e}/, /\textit{i}/ und /\textit{ə}/ im Nebenton, \citealt[102--103]{Caro2011}): Während Althochdeutsch am \isi{Substantiv} morphologisch vier \isi{Kasus} unterscheidet (im Singular Maskulin und Neutrum mit dem Instrumental sogar fünf), sind in den meisten nieder- und hochalemannischen Dialekten alle \isi{Kasus} am \isi{Substantiv} zusammengefallen. Aber auch beide logisch möglichen Gegenbeispiele sind zu beobachten, die dagegen sprechen, dass die Setzung des Initialakzents, die Schwächung der Nebentonvokale und der Kasusverlust direkt miteinander zusammenhängen müssen, wie dies in vielen Sprachgeschichten zur deutschen Sprache suggeriert wird. Erstens weist der höchstalemannische Dialekt des Sensebezirks zwar die Vollvokale /\textit{a}/, /\textit{ə}/, /\textit{ɪ}/ und /\textit{ʊ}/ im Nebenton auf, die Kasusmarkierung am \isi{Substantiv} ist jedoch abgebaut \citep[116, 179–190]{Henzen1927}. Im Gegensatz dazu hat die deutsche Standardsprache nur Schwa im Nebenton (außer in Fremd- und Lehnwörtern), dafür ist im Gegensatz zum Sensebezirk die Markierung einiger \isi{Kasus} am \isi{Substantiv} erhalten (wenn auch nur sehr geringfügig) \citep[158--169]{Eisenberg2006}. Es kann also erstens festgehalten werden, dass die Setzung des Initialakzents zur Schwächung der Nebentonsilben führen kann, aber nicht muss.\footnote{Vgl. auch die kritische Diskussion dieser wenig hinterfragten Annahme in Caro Reina (2011: 103–105).} Weitere Beispiele dafür sind Finnisch, Ungarisch und Tschechisch, die ebenfalls einen Initialakzent haben, in denen jedoch die Nebensilben stabil sind.\footnote{Vielen Dank an Martin Kümmel für diesen Hinweis.} Zweitens kann die Ursache für die Nicht-Mar\-kie\-rung von \isi{Kasus} die Zentralisierung der Nebensilben sein. Ein Beispiel dafür ist die deutsche Standardsprache, ein Gegenbeispiel der höchstalemannische Dialekt des Sensebezirks. Dieser Dialekt hat zwar die Kasusmarkierung am \isi{Substantiv} abgebaut (trotz des Erhalts der Vollvokale in der Nebensilbe), markiert jedoch interessanterweise \isi{Kasus} am \isi{Adjektiv} (vgl. Paradigmen 7 und 27). Die Ursache für den Abbau der Kasusmarkierung am \isi{Substantiv} in diesem Dialekt ist also nicht in der Phonologie zu finden, sondern im Umbau im System der Kasusmarkierung, folglich innerhalb der Flexionsmorphologie.\largerpage[-2]\pagebreak

Der zweite Grund, weshalb die nominale und nicht die verbale Flexionsmorphologie untersucht wird, ist die Tatsache, dass die nominale Flexionsmorphologie unterschiedliche Wortarten aufweist. Synchron kann somit die Komplexität der verschiedenen Wortarten eines Sprachsystems miteinander verglichen werden (z.\,B.\ Komplexität des bestimmten und \isi{unbestimmten Artikels} in Varietät A), aber auch die Komplexität einer Wortart in den verschiedenen Sprachsystemen (z.\,B.\ die Komplexität des \isi{bestimmten Artikels} in Varietät A und Varietät B). Diachron sind in den deutschen Varietäten vor allem Determinierer grammatikalisiert worden. Man kann also fragen, ob diese neuen Wortarten die Flexionsmorphologie insgesamt komplexer gemacht haben oder ob Ausgleichstendenzen zwischen den Wortarten zu beobachten sind (wird z.\,B.\ die Artikelsetzung immer durch einen Abbau der Kasusmarkierung am \isi{Substantiv} kompensiert?). Die nominale Flexionsmorphologie bietet also sozusagen eine größere Angriffsfläche, über die verschiedenen Prozesse und Mechanismen innerhalb eines morphologischen Systems Antworten zu bekommen.

\subsection{Komplexität}\label{3.1.2}

In der vorliegenden Arbeit ist mit Komplexität stets absolute Komplexität gemeint. Welche linguistischen Phänomene für welche Sprecher-/Hörergruppen (L1/L2) schwierig oder aufwändig sind (= relative Komplexität), interessiert hier nicht. Im Fokus steht das linguistische System selbst. Die grobe Grundidee dahinter lautet, dass ein System umso komplexer ist, je mehr Elemente es aufweist. Die Einteilung in absolute und relative Komplexität wurde von \citet{Miestamo2008} eingeführt und wird in \sectref{2.2.5} erläutert. \citet{Rescher1998} liefert in seiner philosophischen Auseinandersetzung mit dem Konzept Komplexität eine detailliertere Taxonomie, welche von \citet{MiestamoSinnemäkiKarlsson2008} auf linguistische Phänomene übertragen wurde. Diese verschiedenen Arten an Komplexität werden in \sectref{2.2.5} ausgeführt. Hier soll nun dargestellt werden, welche Arten von Komplexität in der vorliegenden Arbeit untersucht werden. Dazu ist zuerst zu überlegen, welche Phänomene ein System mehr oder weniger komplex machen. Um dabei eine maximal mögliche Objektivität zu erreichen, ist ein theoriegeleitetes Vorgehen zu bevorzugen.

Die theoretischen Annahmen und wie diese zur Messung von Komplexität operationalisiert werden können, wird in \sectref{4.1} ausführlich vorgestellt. Für dieses Kapitel ist vorerst nur wichtig, dass die \is{Lexical-Functional Grammar (LFG)}Lexical-Functional Grammar (LFG) und die in\-fe\-ren\-tiel\-le-re\-a\-li\-sie\-ren\-de Morphologie die theoretische Grundlage bilden. \citet{Stump2001} schlägt zur Erfassung von Morphologie sogenannte \isi{Realisierungsregeln} (RR) vor, die ausschließlich die Form (und nicht die Funktion) definieren, und zwar nicht nur von \isi{Affixen}, sondern überhaupt die Form eines Wortes, also z.\,B.\ auch Wur\-zel-/Stamm\-mo\-di\-fi\-ka\-tio\-nen. Bezogen auf die Flexionsmorphologie sind RRs Instruktionen/Regeln zum Aufbau eines Paradigmas. Hinsichtlich der Komplexität gilt: Je mehr RRs ein System hat, desto komplexer ist dieses System. Basierend auf \citeauthor{Rescher1998}s \citeyearpar{Rescher1998} Klassifikation wird hier folglich generative Komplexität gemessen.

Welche linguistischen Phänomene die Flexionsmorphologie nun mehr oder weniger komplex machen, wird in \sectref{4.2} ausführlich dargestellt und von \sectref{4.1} (theoretische Grundlage und Operationalisierung dieser zur Messung von Komplexität) abgeleitet. Um aber zu verstehen, mit welcher Art von Komplexität wir uns befassen, werden die wichtigsten Punkte, die die Flexionsmorphologie komplexer machen, hier zusammengefasst. (in Klammern stehen die Arten der Komplexität):

\begin{enumerate}
\item 
Anzahl grammatischer Eigenschaften, die in der Flexion unterschieden und overt markiert werden, z.\,B.\ Anzahl \isi{Kasus}, \isi{Genera} etc. (konstitutionelle und taxonomische Komplexität).
\item 
Anzahl Allomorphe, z.\,B.\ Anzahl Pluralallomorphe (konstitutionelle und organisationelle Komplexität).
\item 
Mehrfachausdruck derselben Funktion, z.\,B.\ \textit{Wäld}-\textit{er} (Plural wird durch \isi{Umlaut} und \isi{Suffix} ausgedrückt; konstitutionelle und organisationelle Komplexität).
\item 
Bestimmte Art von \isi{Synkretismus}, nämlich wenn die Werte von mehr als zwei Features variieren; wird z.\,B.\ das \isi{Suffix} -\textit{er} der starken Adjektivflexion im Genitiv Feminin Singular und Genitiv Plural (keine Genusunterscheidung) suffigiert, so sind zwei RRs nötig (die Werte von \isi{Genus} und \isi{Numerus} variieren; organisationelle Komplexität).\footnote{Detailliert beschrieben wird dieser Fall von \isi{Synkretismus} in \sectref{4.1.3.3}.}
\end{enumerate}

Wie bereits erwähnt, wird hier generative (also epistemische) Komplexität gemessen, denn je mehr Instruktionen (d.h. RRs) gebraucht werden, um ein Flexionsparadigma zu produzieren, desto komplexer ist ein Flexionssystem. Betrachtet man jedoch, für welche Phänomene RRs benötigt werden (wie oben 1.–4. aufgelistet), so befassen wir uns mit den ontologischen Arten der Komplexität: Konstitutionelle, taxonomische und organisationelle Komplexität. Mit konstitutioneller Komplexität ist die Größe des Inventars gemeint, mit taxonomischer Komplexität die Anzahl kodierter Unterscheidungen in einem System. Je mehr Kategorien also unterschieden (1.) und je häufiger diese Kategorien overt markiert werden (2. und 3.), desto komplexer ist ein System. Unter der organisationellen Komplexität versteht man allgemein formuliert alles, was dem Eins-Zu-Eins-Verhältnis zwischen Form und Funktion widerspricht. Dazu gehört Allomorphie, d.h., eine Funktion kann durch verschiedene Formen ausgedrückt werden (z.\,B.\ unterschiedliche \isi{Suffixe} für Plural). Ein Extremfall von Allomorphie ist, wenn eine Funktion nicht nur unterschiedlich ausgedrückt werden kann, sondern wenn sie am selben Wort unterschiedlich markiert wird (z.\,B.\ Mehrfachausdruck von Plural, \textit{Wäld}-\textit{er}). Aber auch der umgekehrte Fall, d.h., Homonymie, kommt vor: Eine Form hat mehrere Funktionen. Dies betrifft den beschriebenen \isi{Synkretismus} (4.). Es liegt eine Form vor (-\textit{er}), die jedoch unterschiedliche Funktionen hat, die nicht einheitlich (d.h. durch eine RR) erfasst werden können.

In dieser Arbeit wird also unter Komplexität Folgendes verstanden:

\begin{itemize}
\item 
Kompositionelle Komplexität (konstitutionelle und taxonomische Komplexität): Je mehr Kategorien unterschieden werden und je größer das Inventar an Markierungsmöglichkeiten ist, desto komplexer ist das Flexionssystem.
\item 
Organisationelle Komplexität: Allomorphie und Homonymie führen zu einer höheren Komplexität.
\item 
Generative Komplexität: Kompositionelle und organisationelle Komplexität werden automatisch durch \isi{Realisierungsregeln} erfasst (wie diese in \sectref{4.1.3} definiert sind). Weil \isi{Realisierungsregeln} eine Art Instruktion zum Aufbau eines Paradigmas darstellen, wird also auch generative Komplexität gemessen.
\end{itemize}

\section{Hypothesen}\label{3.2}

Das Ziel dieses Kapitels ist, die Fragestellungen und Hypothesen vorzustellen sowie die Varietäten des Samples anhand derer diese Hypothese überprüft werden. Die Varietäten selbst werden im \sectref{3.3} beschreiben. Die Hypothesen basieren vorwiegend auf Trudgills Arbeiten, deren wichtigste Überlegungen und Resultate bereits im \sectref{2.2.3} vorgestellt wurden. Aus diesem Grund werden diese hier nur kurz wiederaufgegriffen und nicht detailliert diskutiert. Das Kapitel ist nach folgenden Faktoren gegliedert: \isi{Diachronie} (\sectref{3.2.1}), \isi{Dialektgruppe} (\sectref{3.2.2}), Kontakt (\sectref{3.2.3}), \isi{Standardvarietät} (\sectref{3.2.4}) und \isi{Isolation} (\sectref{3.2.5}).

\subsection{Diachronie}\label{3.2.1}

Es stellt sich hier die Frage, ob Sprachen im Laufe der Zeit tendenziell komplexer oder einfacher werden oder ob ihre Komplexität konstant bleibt. Alle drei Hypothesen sind in der Linguistik vertreten. Vertreter der \is{Equi-Complexity-Hypothese}\textit{Equi-Com\-ple\-xi\-ty-Hy\-po\-the\-se} nehmen an, dass zwar die Komplexität einzelner Ebenen der Grammatik variieren kann. Verrechnet man jedoch die Komplexität der verschiedenen linguistischen Beschreibungsebenen, sind die Sprachen immer gleich komplex. Ein oft genanntes Beispiel sind die angenommenen Ausgleichstendenzen zwischen der Morphologie und der Syntax: Was in der Morphologie nicht ausgedrückt wird, wird in der Syntax kodiert und umgekehrt. Dies impliziert also, dass Sprachen aus diachroner Perspektive immer gleich komplex bleiben. Eine weitere prominente Hypothese ist, dass Sprachen und vor allem ihre Morphologie kontinuierlich an Komplexität verlieren. Sie fußt wohl vorwiegend auf dem Wissen über alte indogermanische Sprachen, die im Vergleich zu modernen indogermanischen Sprachen eine relativ reiche Flexionsmorphologie aufweisen. Schließlich konnte vor allem die soziolinguistische Typologie zeigen, dass Sprachen auch diachrone Komplexifizierung aufweisen (vgl. \sectref{2.2.1}, \sectref{2.2.3} und \sectref{2.2.4}), d.h., dass Sprachen komplexer werden können. Zurzeit liegt noch keine Messung der Gesamtkomplexität einer Sprache vor, weil zahlreiche theoretische und methodische Fragen offenstehen (vgl. \sectref{2.1.2}). Es konnte jedoch nachgewiesen werden, dass einzelne Phänomene oder Teilsysteme synchron (im Vergleich mit anderen Sprachen/Varietäten) oder diachron komplexer sind. Es handelt sich dabei vorwiegend um Sprachen, die von einer Sprachgemeinschaft mit bestimmten Charakteristika gesprochen werden: klein, isoliert, mit einem dichten Netzwerk und wenig L2-Ler\-nern. Sprachen dieser Sprachgemeinschaften können eine höhere Komplexität aufweisen, die nicht auf Sprachkontakteffekte zurückgeführt werden kann. Die höhere Komplexität wird dadurch erklärt, dass es in diesen Sprachgemeinschaften einfacher ist, einen Wandel des phonologisch eher unnatürlichen Typs sowie den Zuwachs an morphologischen Kategorien durchzuführen und zu erhalten (\citealt[11]{Trudgill1996} und \citealt[109]{Trudgill2009}; ausführlich vorgestellt in \sectref{2.2.3}). Trudgill nennt dies spontane Komplexifizierung \citep[71]{Trudgill2011}. Einen zweiten Typ von Komplexifizierung bezeichnet Trudgill als \textit{Additive Borrowing} \citep[27]{Trudgill2011}. Es handelt sich dabei um Elemente oder Kategorien, die von der einen in die andere Sprache übernommen werden, ohne dass in der übernehmenden Sprache bereits existierende Elemente oder Kategorien substituiert werden. Dies kommt in Sprachen von Sprachgemeinschaften vor, die sich in einer langzeitigen koterritorialen Kontaktsituation befinden, in der folglich Kinder zweisprachig aufwachsen \citep[34]{Trudgill2011}.

In dieser Arbeit gehe ich \textit{erstens} davon aus, dass Sprachen bezüglich ihrer Komplexität aus diachroner Perspektive variieren (Zunahme und Abnahme von Komplexität), da die \is{Equi-Complexity-Hypothese}\textit{Equi-Com\-ple\-xi\-ty-Hy\-po\-the\-se} als allgemeine Tendenz verworfen werden kann (vgl. Diskussion \sectref{2.1.2}). Es soll also untersucht werden, wie sich Komplexität diachron entwickelt und wie diese unterschiedlichen Entwicklungen erklärt werden können. Dazu wird Althochdeutsch mit Mittelhochdeutsch verglichen sowie Alt-/Mittelhochdeutsch mit modernen alemannischen Dialekten und mit der deutschen Standardsprache. \textit{Zweitens} kann spontane Komplexifizierung in Dialekten beobachtet werden, die von einer kleinen, \isi{isolierten Sprachgemeinschaft} mit engen Netzwerken und wenig L2-Ler\-nern gesprochen wird. In diesem Sample trifft diese Definition am besten auf folgende Dialekte zu (zur Begründung, weshalb diese Dialekte als isoliert gelten, s. \sectref{3.3.3}): Issime (Höchstalemannisch), Visperterminen (Höchstalemannisch), Jaun (Höchstalemannisch), Vorarlberg (Hochalemannisch), Huzenbach (Schwäbisch), Münstertal (Oberrheinalemannisch, Elsass). \textit{Drittens} können \textit{Additive Borrowings} in bi- oder multilingualen Sprachgemeinschaften erwartet werden. Zu den bi- oder multilingualen Sprachgemeinschaften gehören in diesem Sample die Sprachinseldialekte Issime (Höchstalemannisch), Petrifeld (Schwäbisch) und Elisabethtal (Schwäbisch) wie auch die elsässischen Dialekte (Oberrheinalemannisch).

\subsection{Dialektgruppen}\label{3.2.2}

Die alemannischen Dialekte werden in die folgenden Gruppen eingeteilt: Höchstalemannisch, Hochalemannisch, Oberrheinalemannisch, Bodenseealemannisch und Schwäbisch, wobei die drei letzten Gruppen auch unter dem Begriff Niederalemannisch subsumiert werden können. Da für das Bodenseealemannische keine vollständige Beschreibung der nominalen Flexionsmorphologie vorliegt, wird es im weiteren Verlauf nicht berücksichtigt. Diese Einteilung der alemannischen Dialekte basiert vorwiegend auf phonologischen Phänomenen, z.\,B.\ Nasalausfall vor Frikativ (Höchst- vs. Hochalemannisch), \textit{k}{}-Verschiebung (Hoch- vs. Niederalemannisch), frühneuhochdeutsche Diphthongierung (Oberrheinalemannisch vs. Schwäbisch). Ergänzt wird diese Klassifikation durch morphologische, morphosyntaktische und lexikalische Eigenschaften, z.\,B.\ Flexion der prädikativen \isi{Adjektive} (Höchst- vs. Hochalemannisch), verbaler Einheitsplural -\textit{e}/-\textit{et} sowie die Lexeme \textit{Matte}/\textit{Wiese} ‘Wiese’ (Oberrheinalemannisch vs. Schwäbisch). Es stellt sich also die Frage, ob diese Einteilung der alemannischen Dialekte sich auch in der strukturellen Komplexität dieser Dialekte widerspiegelt. Somit könnte diese Klassifikation durch einen weiteren Faktor ergänzt werden.

Da die höchstalemannischen Dialekte eine reiche Flexionsmorphologie und die niederalemannischen Dialekte eine schlanke Flexionsmorphologie aufweisen, gehe ich von folgender Komplexitätshierarchie aus: Höchstalemannisch > Hochalemannisch > Niederalemannisch. Innerhalb des Niederalemannischen können weiter Oberrheinalemannisch und Schwäbisch miteinander verglichen werden. Da es für das badische Oberrheinalemannisch nur eine Grammatik gibt, die eine vollständige Beschreibung der nominalen Flexionsmorphologie enthält, macht es wenig Sinn, das badische Oberrheinalemannisch mit dem elsässischen Oberrheinalemannisch oder dem Schwäbischen zu vergleichen.

\subsection{Kontakt}\label{3.2.3}

\isi{Sprachkontakt} kann sowohl zur Komplexifizierung als auch zur Simplifizierung linguistischer Systeme führen. Im vorangehenden Kapitel wurde gezeigt, dass in Sprachen bi- oder multilingualer Sprachgemeinschaften \textit{Additive Borrowings} gefunden werden, wodurch die Komplexität einer Sprache erhöht wird. Wie bereits erwähnt wurde, trifft dies auf folgende Dialekte zu: Issime (Höchstalemannisch), Petrifeld (Schwäbisch) und Elisabethtal (Schwäbisch) wie auch auf die elsässischen Dialekte (Oberrheinalemannisch). Es müssen hier jedoch zwei Einschränkungen gemacht werden. Erstens entwickelte sich eine bilinguale Sprachgemeinschaft im Elsass erst nach der Französischen Revolution und vor allem nach dem 2. Weltkrieg. Zweitens ist über die Sprachkompetenz in der Kontaktsprache der Sprecher von Petrifeld und Elisabethtal (zur Zeit der Publikation der hier verwendeten \isi{Ortsgrammatiken}) nichts Genaues bekannt. Bei Issime hingegen handelt es sich um eine alte Sprachinsel (seit dem 13. Jh.) und Informationen zur Kontaktsituation und Sprachkompetenz sind vorhanden. Ausführlich dargestellt und diskutiert wird dies in \sectref{3.3.3} und \sectref{6.3.1}.

Simplifizierung hingegen kommt in Sprachen von großen Sprachgemeinschaften mit vielen Kontakten, losen Netzwerken und vielen L2-Ler\-nern vor (\citealt[146]{Trudgill2011}; ausführlich vorgestellt in \sectref{2.2.3}). In diesen Sprachen werden besonders Irregularitäten, Redundanzen und Opazität reduziert \citep[101]{Trudgill2009}. Um diese Hypothese der Simplifizierung zu überprüfen, lassen sich bezüglich der alemannischen Dialekte die Stadt- mit den Landdialekte vergleichen.\footnote{Standardvarietäten und Nicht-Stan\-dard\-va\-ri\-e\-tä\-ten werden in \sectref{3.2.4} verglichen, \isi{geografisch isolierte} und nicht isolierte Dialekte in \sectref{3.2.5}.} Es kann davon ausgegangen werden, dass die Sprachgemeinschaft einer Stadt größer ist und mehr Kontakte, eher lose Netzwerke und mehr L2-Ler\-ner aufweist, als dies auf dem Land der Fall ist. Aus dem hier untersuchten Sample (vgl. \tabref{table3.1}) werden für das Hochalemannische Bern und Zürich mit Vorarlberg verglichen, für das Schwäbische Stuttgart mit Bad Saulgau und Huzenbach, für das Oberrheinalemannische Colmar mit Münstertal, Elsass (Ebene) und Kaiserstuhl. Für Städte im höchstalemannischen Gebiet liegen keine \isi{Ortsgrammatiken} vor, die die Flexionsmorphologie aller untersuchten Wortarten beschreiben. Der Vergleich Stadt–Land wird innerhalb derselben \isi{Dialektgruppe} vorgenommen, da die Flexionsmorphologie mancher \isi{Dialektgruppen} eine höhere Komplexität aufweist als die Flexionsmorphologie anderer \isi{Dialektgruppen} (vgl. \sectref{6.2}).

\subsection{Standardvarietät}\label{3.2.4}

Bezüglich der Standardsprache widersprechen sich die Erwartungen, weil sowohl Simplifizierung als auch Komplexifizierung möglich sind. Laut \citet{Ferguson1959} ist zu erwarten, dass Standardvarietäten (\textit{High} \textit{Varieties}) eine höhere Komplexität aufweisen als Nicht-Stan\-dard\-va\-ri\-e\-tä\-ten (\textit{Low} \textit{Varieties}) (\citealt[333]{Ferguson1959}; vgl. \sectref{2.1.2}). Des Weiteren ist denkbar, dass Kategorien durch Kodifizierung besser konserviert werden können. Ein Beispiel hierfür ist der Erhalt des Genitivs in der deutschen Standardsprache, während dieser in den meisten Dialekten abgebaut wurde. Zudem können auch kulturelle Faktoren eine Rolle spielen. Beispielsweise zeigen sich die Varietäten im südlichen Teil des deutschsprachigen Raumes sehr progressiv, z.\,B.\ durch die Syn- und Apokope von \textit{e} in der Flexion \citep[275]{vonPolenz2013}. Grammatiker des 17. und 18. Jahrhunderts forderten jedoch, das -\textit{e} zur Unterscheidung von \isi{Numerus} am \isi{Substantiv} beizubehalten \citep[275]{vonPolenz2013}. Solche Bemühungen sind auch auf bestimmte kulturelle Kontexte zurückzuführen: „Nach der Reformationszeit, im Zusammenhang mit der stärkeren überregionalen Sprachvereinheitlichung, ist eine deutliche Tendenzwende zu beobachten: Schriftsteller, Korrektoren, Drucker und Sprachgelehrte bemühen sich um schreib- und und drucksprachliche Restitutionen von sprechsprachlich längst geschwundenen Flexionsendungen“ \citep[275]{vonPolenz2013}.

Vor dem Hintergrund der Diskussion im vorangehenden Kapitel ist jedoch zu erwarten, dass eine Standardsprache eine geringere Komplexität als ein Dialekt aufweist. Es wurde dargestellt, dass die Sprachen von großen Sprachgemeinschaften mit einem losen Netzwerk, vielen Kontakten und vielen L2-Ler\-nern dazu tendieren, strukturell einfacher zu werden. Noch besser als die in \sectref{3.2.3} erwähnten Stadtdialekte treffen Standardvarietäten auf diese Definition zu. Dies gilt besonders bezüglich der L2-Ler\-ner. Erstens lernt, wer Deutsch als Fremdsprache erwirbt, meistens zumindest die deutsche Standardsprache, wobei der Erwerb eines Dialektes bzw. einer Regionalsprache nicht ausgeschlossen ist. Dies trifft vorwiegend auf Deutschland und Österreich zu, in geringerem Maße auf die Schweiz (besonders bezogen auf den ungesteuerten Spracherwerb). Zweitens lernen viele Deutsch-Muttersprachler einen Dialekt oder eine Regionalsprache als L1, in der Schweiz gilt dies für alle Deutsch-Muttersprachler. In diesen Fällen kann die Standardsprache wohl nicht als Fremdsprache angesehen werden, trotzdem fängt der Erwerb der Standardsprache etwas später als der L1 an. Ein weiterer Grund, weshalb bei einer Standardsprache eine geringere Komplexität zu erwarten ist, liegt in der Natur der Standardisierung. Das Ziel eines Standards ist die Vereinheitlichung und Vereinfachung, wobei die Vereinfachung u.a. als Konsequenz aus der Vereinheitlichung resultieren kann. Bei der Standardisierung einer Sprache wird bestimmt, welche Varianten richtig oder falsch sind, was zu einer Variantenreduzierung führt \citep[231]{vonPolenz1999}. Welche Varianten als richtig und falsch klassifiziert werden, hängt von unterschiedlichen kultur- und wissenschaftsgeschichtlichen Faktoren ab, worauf hier nicht weiter eingegangen wird. Schließlich ist hier noch zu bemerken, dass Deutsch vor allem im 19. und 20. Jahrhundert stark standardisiert wurde, was vorwiegend politische und soziale Ursachen hat \citep[232]{vonPolenz1999}. In derselben Zeit konnte sich die deutsche Standardsprache auch aus sozialer und regionaler Sicht ausbreiten, u.a. aufgrund von „Industrialisierung, Verstädterung, Bevölkerungsmischung, […] Fernverkehr[…], […] Verschriftlichung des täglichen Lebens“ \citep[232]{vonPolenz1999}, aber auch aufgrund der Bildung von Nationalstaaten und der Alphabetisierung der Bevölkerung \citep[233]{vonPolenz1999}. Die deutsche Standardsprache hat folglich eine deutlich geringere diachrone Tiefe als die Dialekte.

Aus dieser Diskussion scheint es plausibler anzunehmen, dass die strukturelle Komplexität einer Standardsprache geringer ist als jene von Nicht-Stan\-dard\-spra\-chen. Dies schließt jedoch nicht den Erhalt von einzelnen Kategorien aus, die in den Dialekten abgebaut wurden (z.\,B.\ Genitiv). Um zu überprüfen, ob Standardsprachen weniger komplex sind als Nicht-Stan\-dard\-spra\-chen, wird die deutsche Standardsprache mit den hier untersuchten alemannischen Dialekten verglichen.

\subsection{Isolation}\label{3.2.5}

In den beiden vorangehenden Kapiteln wurde gezeigt, dass Sprachen innerhalb eines bestimmten Typs von Sprachgemeinschaft zur Simplifizierung tendieren. Sprachen hingegen, die von kleinen, stabilen Sprachgemeinschaften mit einem engen Netzwerk, wenig \isi{Sprachkontakt} und kaum L2-Ler\-nern (= \isi{sozial isoliert}, vgl. \sectref{2.2.3}) gesprochen werden, sind laut \citet{Trudgill2011} tendenziell komplexer \citep[146–147]{Trudgill2011}. Es handelt sich dabei um kleine, stabile Sprachgemeinschaften . Diese Sprachgemeinschaften können zusätzlich auch \isi{geografisch isoliert} sein, womit in den bisherigen Studien meistens Sprachen bzw. Sprachgemeinschaften in den Bergen gemeint sind (vgl. \sectref{2.2.4}). Sprachen solcher Sprachgemeinschaften tendieren erstens dazu, ihre strukturelle Komplexität in größerem Umfang zu erhalten, da der Sprachwandel langsamer abläuft \citep[103]{Trudgill2011}. Zweitens kann auch spontane Komplexifizierung beobachtet werden (\citealt[71]{Trudgill2011}, vgl. \sectref{3.2.1} zur \isi{Diachronie}).

Um diese Hypothese zu überprüfen, bietet das hier untersuchte Sample zwei Möglichkeiten (vgl. \tabref{table3.1}). \textit{Erstens} können die Landdialekte den Stadtdialekten gegenübergestellt werden, wofür bereits in \sectref{3.2.3} argumentiert wurde. \textit{Zweitens} werden Dialekte, die zusätzlich \isi{geografisch isoliert} sind, mit geografisch nicht isolierten Dialekten verglichen, d.h. Stadtdialekte und geografisch nicht isolierte Landdialekte vs. \isi{geografisch isolierte} Landdialekte.\footnote{Inwiefern welche Dialekte \isi{geografisch isoliert} bzw. nicht isoliert sind, wird in \sectref{3.3} dargestellt.} Dies geschieht innerhalb derselben \isi{Dialektgruppe}, weil die Flexionsmorphologie gewisser \isi{Dialektgruppen} komplexer ist als die Flexionsmorphologie anderer \isi{Dialektgruppen} (vgl. \sectref{6.2}). Für die vier hier untersuchten alemannischen \isi{Dialektgruppen} (Höchstalemannisch, Hochalemannisch, Oberrheinalemannisch, Schwäbisch) liegen vollständige Beschreibungen der nominalen Flexionsmorphologie von geografisch isolierten und nicht-isolierten Dialekten vor.\\

% \textbf{\tabref{table3.1}: Die untersuchten alemannischen Dialekte}\\

\begin{table}
\caption{Die untersuchten alemannischen Dialekte}\label{table3.1}
\begin{tabular}{lllc}
\lsptoprule
{DG} & {Dialekt} & {Stadt/Land} & {geogr. isoliert}\\\midrule
h-st\il{Höchstalemannisch} & Issime & Land & \ding{52}\\
& Visperterminen & Land & \ding{52}\\
& Jaun & Land & \ding{52}\\
& Sensebezirk & Land & \ding{55}\\
& Uri & Land & \ding{55}\\
hoch\il{Hochalemannisch} & Vorarlberg & Land & \ding{52}\\
& Zürich & Stadt & \ding{55}\\
& Bern & Stadt & \ding{55}\\
schw\il{Schwäbisch} & Huzenbach & Land & \ding{52}\\
& Bad Saulgau & Land & \ding{55}\\
& Stuttgart & Stadt & \ding{55}\\
& Petrifeld & Land & \ding{55}\\
& Elisabethtal & Land & \ding{55}\\
oberr\il{Oberrheinalemannisch} & Münstertal & Land & \ding{52}\\
& Elsass (Ebene) & Land & \ding{55}\\
& Colmar & Stadt & \ding{55}\\
& Kaiserstuhl & Land & \ding{55}\\
\lspbottomrule
\end{tabular}
\end{table}

\subsection{Zusammenfassung}\label{3.2.6}

In der Folge werden die Hypothesen nochmal zusammengefasst. Gleichzeitig wird gezeigt, welche Vergleiche zur Überprüfung der Hypothesen angestellt werden.\\

\noindent
Diachrone Zunahme und/oder Abnahme von Komplexität, d.h., wie sich Komplexität diachron entwickelt:
\begin{itemize}
\item Ahd. vs. Mhd.
\item Ahd./Mhd. vs. alemannische Dialekte und deutsche Standardsprache: Ob AHD oder MHD für den Vergleich mit einer bestimmten modernen Varietät gewählt wird, ist abhängig davon, auf welche Stufe diese Varietät zurückgeführt werden kann, vgl. \sectref{6.1.1}).
\end{itemize}

\noindent
\isi{Dialektgruppen} unterscheiden sich in ihrer Komplexität, und zwar:
\begin{itemize}
\item Höchstalemannisch ist komplexer als Hochalemannisch und Hochalemannisch ist komplexer als Niederalemannisch.
\end{itemize}

\noindent
Spontane Komplexifizierung wird in isolierten Dialekten erwartet:
\begin{itemize}
\item Ahd./Mhd. vs. isolierte alemannische Dialekte (diachroner Vergleich).
\item Nicht isolierte vs. isolierte alemannische Dialekte, innerhalb derselben \isi{Dialektgruppe} (synchroner Vergleich).
\end{itemize}

\noindent
\isi{Additive Borrowings} werden in mehrsprachigen Sprachgemeinschaften erwartet:
\begin{itemize}
\item Ahd./Mhd. und Alemannisch einer einsprachigen Sprachgemeinschaft vs. Alemannisch einer zwei- oder mehrsprachigen Sprachgemeinschaft.
\end{itemize}

\noindent
Wird ererbte Komplexität erhalten oder ist Simplifizierung festzustellen:
\begin{itemize}
\item Geografisch nicht isolierte Dialekte simplifizieren eher, während \isi{geografisch isolierte} Dialekte eher ererbte Komplexität erhalten (Vergleich innerhalb derselben \isi{Dialektgruppe}).
\item Stadtdialekte simplifizieren, Landdialekte bewahren ererbte Komplexität stärker (Vergleich innerhalb derselben \isi{Dialektgruppe}).
\item Deutsche Standardsprache simplifiziert, alemannische Dialekte erhalten ererbte Komplexität stärker.
\end{itemize}

\section{Varietäten}\label{3.3}

In diesem Kapitel werden alle analysierten Varietäten vorgestellt und die verwendeten (Orts-) Grammatiken aufgeführt. Bezüglich der alemannischen Dialekte wird zudem erörtert, ob es sich um einen Stadt- oder Landdialekt handelt sowie ob und weshalb der Dialekt \isi{geografisch isoliert} ist oder nicht. Es werden zuerst die älteren Stufen des Deutschen präsentiert (\sectref{3.3.1}), dann die deutsche Standardsprache (\sectref{3.3.2}) und schließlich die alemannischen Dialekte (\sectref{3.3.3}).

Es sei des Weiteren darauf hingewiesen, dass die Grammatiken sich in ihrer Art der Beschreibung unterscheiden. Für die Messung der Komplexität konnten die Paradigmen folglich nicht verbatim aus den Grammatiken übernommen werden. Vielmehr wurden die relevanten Informationen den Grammatiken entnommen und daraus neue Paradigmen nach einheitlichen Kriterien erstellt. Welche diese Kriterien sind und wie dabei genau vorgegangen wurde, wird in \sectref{4.3.2} und \sectref{5} erklärt.

\subsection{Ältere Stufen des Deutschen}\label{3.3.1}
\subsubsection{Althochdeutsch}

Unter Althochdeutsch versteht man die älteste Stufe des Deutschen. Der Beginn wird mit der 2. Lautverschiebung im späten 6. Jahrhundert angesetzt und das Ende mit der Abschwächung der Endsilbenvokale in der 2. Hälfte des 11. Jahrhunderts, wobei jedoch eine fortwährende schriftliche Überlieferung erst Ende des 8. Jahrhunderts beginnt \citep[1]{Braune2004}. Das Althochdeutsche verfügt über keine überdachende Koiné, sondern besteht aus dem Oberdeutschen (Alemannisch und Bairisch), dem Mitteldeutschen (Rhein- und Mittelfränkisch) und dem Ostfränkischen, das den Übergang zwischen Ober- und Mitteldeutsch bildet \citep[1, 6]{Braune2004}.

Als grammatische Beschreibung des Althochdeutschen werden die beiden\linebreak Bände der althochdeutschen Referenzgrammatik von Braune (2004, Laut- und Formenlehre) und von \citet[Syntax]{Schrodt2004} verwendet. Diese Grammatik bildet eine Art ‘Normalalthochdeutsch’ ab, basierend vorwiegend auf dem althochdeutschen Tatian, also auf einer ostfränkischen Handschrift aus dem 8. Jh. \citep[6]{Braune2004}. Es handelt sich dabei folglich um ein schriftbasiertes Normalalthochdeutsch, das in dieser Form nie existiert hat. Trotzdem gibt es etliche Gründe, weshalb diese Grammatik für die Analyse herangezogen werden kann. Erstens wäre jene Grammatik am idealsten, die einen althochdeutschen Ortsdialekt zu einem bestimmten Zeitpunkt beschreibt (z.\,B.\ St. Gallen um 830). Eine solche Beschreibung liegt jedoch zum gegenwärtigen Zeitpunkt nicht vor. Zweitens ist die Grammatik von \citet{Braune2004} und \citet{Schrodt2004} die ausführlichste und umfassendste Beschreibung des Althochdeutschen. Drittens beschreibt sie das Althochdeutsche zu einem bestimmten Zeitpunkt, nämlich das Althochdeutsche des 9. Jahrhunderts. Gibt es durch Sprachwandel bedingte \isi{Variation}, werden die älteren und jüngeren Varianten genannt. Viertens basiert diese Grammatik zwar auf einem ostfränkischen Dialekt, dialektale Besonderheiten werden jedoch ebenfalls angegeben \citep[6]{Braune2004}. Fünftens sind auch die \isi{Ortsgrammatiken} zu den alemannischen Dialekten nicht uneingeschränkt repräsentativ für den gesamten Ort, da meist nur wenige Sprecher befragt wurden. Diese Ortgrammatiken stellen folglich ebenfalls nur einen Ausschnitt dar und können deshalb mit der althochdeutschen Grammatik von \citet{Braune2004} und \citet{Schrodt2004} verglichen werden.

Da \citet{Braune2004} ältere und jüngere Formen sowie dialektale Charakteristika unterscheidet, ist noch zu klären, welche Varianten für die vorliegende Analyse verwendet werden. Werden ältere und jüngere Formen genannt, wird die ältere Form gewählt. Zum Beispiel: Instr.Sg.m./n. = -\textit{u}, Nom.Sg.f. = -\textit{u}, Dat.Sg.f. = -\textit{u}/-\textit{eru}/\textit{iru}, Dat.Sg.m/n. = -\textit{emu}/\textit{imu}, Nom.Sg.f. = -\textit{u}, Dat.Pl. = V\textit{m}. Werden ober- von mitteldeutschen Varianten oder alemannische von Varianten anderer Dialekte unterschieden, wird die oberdeutsche bzw. alemannische Variante bevorzugt. Zum Oberdeutschen, wozu das Alemannische, das Ostfränkische und das Bairische gehören, ist noch zu erwähnen, dass die beiden Dialekte in althochdeutscher Zeit deutlich weniger Unterschiede aufweisen als später \citep[7]{Braune2004}.

\subsubsection{Mittelhochdeutsch}

Das Mittelhochdeutsche beginnt mit der Schwächung der Endsilbenvokale (Mitte 11. Jh.) und endet neben anderen Phänomenen mit der frühneuhochdeutschen Diphthongierung und Monophthongierung (Mitte 14. Jh., wobei der Anfang der Diph- und Monophthongierung früher anzusetzen ist) \citep[18–21]{Paul2007}. Wie das Althochdeutsche ist auch das Mittelhochdeutsche dialektal gegliedert und weist keine überdachende\largerpage Varietät auf \citep[34]{Paul2007}. Zudem zeigen die mittelhochdeutschen Handschriften auch stilistische Unterschiede \citep[11]{Paul2007}. Folglich bildet die mittelhochdeutsche Referenzgrammatik von \citet{Paul2007} ebenfalls ein Konstrukt, das vorwiegend auf dem sogenannten klassischen Mittelhochdeutsch (ca. 1170–1250) basiert \citep[10]{Paul2007}. Aus denselben Gründen, die bereits für das Althochdeutsche dargestellt wurden, kann diese Grammatik trotzdem für diese Arbeit verwendet werden.

\subsection{Deutsche Standardsprache}\label{3.3.2}
\largerpage
Die Daten für die deutsche Standardsprache stammen aus der deskriptiven Grammatik von \citet{Eisenberg2006}, die eine vollständige Wort- und Satzgrammatik enthält. Es handelt sich dabei um die vorwiegend geschriebene Variante der Standardsprache. Eine umfassende Grammatik der gesprochenen Standardsprache gibt es meines Wissens nicht. Außerdem ist die gesprochene Standardsprache stets regional gefärbt, d.h., für die vorliegende Analyse bräuchte es nicht nur eine Beschreibung für den alemannischen Raum, sondern auch mindestens je eine für die unterschiedlichen Staaten des alemannischen Raumes. Umfassende und detaillierte Beschreibungen der nominalen Flexionsmorphologie dieser regionalen Varietäten der deutschen Standardsprache existieren jedoch nicht. Um klarer zu machen, um welche Unterschiede es sich u.a. handelt, wird hier ein Beispiel angefügt. In der Schweiz werden üblicherweise die vollen Formen des \isi{unbestimmten Artikels} benutzt (z.\,B.\ \textit{eine Birne}), während in weiten Teilen Deutschlands (teils auch im alemannischsprachigen Raum) die reduzierten Formen verwendet werden (z.\,B.\ \textit{ne Birne}). Dass die deutsche Standardsprache in der Schweiz oft als ʻSchriftspracheʼ bezeichnet wird, kommt also (aus unterschiedlichen Gründen) nicht von ungefähr: Die deutsche Standardsprache in der Schweiz orientiert sich relativ stark an der Schrift. Da es also regionale Unterschiede in der deutschen Standardsprache gibt, aber keine vollständigen und detaillierten Beschreibungen für das hier untersuchte Gebiet, wird die geschriebene deutsche Standardsprache in dieser Arbeit herangezogen, da es sich um jene \isi{Standardvarietät} handelt, die von allen Sprechern des alemannischen Raums verwendet wird, wenn auch von einigen von ihnen vorwiegend schriftlich.

Neben diesem praktischen gibt es jedoch noch einen soziolinguistischen\linebreak Grund, weshalb die geschriebene Standardsprache herbeigezogen wird. Die\linebreak Sprachgemeinschaft, die diese Standardsprache verwendet (schriftlich wie mündlich), passt genau auf \citeauthor{Trudgill2011}s \citeyearpar{Trudgill2011} Definition des einen Typs von Sprachgemeinschaft: große Sprachgemeinschaft mit losen Netzwerken, viel Kontakt und vielen L2-Ler\-nern. Die Standardsprache fungiert als Lingua franca innerhalb des deutschsprachigen Raumes, ist normalerweise Unterrichtssprache und jene Variante des Deutschen, die im Unterricht für Deutsch als Fremdsprache vermittelt wird.

\subsection{Alemannische Dialekte}\label{3.3.3}

Die alemannischen Dialekte werden in fünf Gruppen eingeteilt, wie bereits in \sectref{3.2.2} dargestellt wurde: Höchstalemannisch, Hochalemannisch, Bodenseeale-\linebreak mannisch, Oberrheinalemannisch und Schwäbisch. Da für das Bodenseealemannische keine Grammatik mit einer vollständigen Beschreibung der nominalen Flexionsmorphologie existiert, wird diese Gruppe hier ausgeschlossen. Das Oberrheinalemannische kann weiter in ein elsässisches und ein badisches Oberrheinalemannisch unterteilt werden, da auch in der Vergangenheit das elsässische Oberrheinalemannisch deutlich stärker und länger dem französischen Einfluss ausgesetzt war.

Welche Dialekte und Grammatiken ausgewählt wurden, hat unterschiedliche Gründe. Die Grammatiken betreffend war wichtig, dass sie die Flexionsmorphologie aller hier untersuchten Wortarten beschreiben: \isi{Substantive}, \isi{Adjektive}, \isi{Personalpronomen}, \isi{Interrogativpronomen}, einfaches \isi{Demonstrativpronomen}, \isi{Possessivpronomen}, bestimmter und \isi{unbestimmter Artikel} (vgl. \sectref{4.3.2}). Würde für einen Dialekt die Komplexität einer Wortart mehr oder weniger gemessen werden, würde dies die Resultate verzerren. Bezüglich der Dialekte sollte erstens möglichst das gesamte alemannische Gebiet abgedeckt werden, d.h. alle Gruppen und alle Staaten. Mit Ausnahme des Bodenseealemannischen konnte dies erreicht werden (vgl. \tabref{table3.2}). Zweitens verfügt idealerweise jede Gruppe über Stadt- und Landdialekte sowie über \isi{geografisch isolierte} und nicht isolierte Dialekte. Dies ist für das Hochalemannische, für das elsässische Oberrheinalemannische und für das Schwäbische gewährleistet. Für das badische Oberrheinalemannische konnte nur eine Grammatik gefunden werden, die die Flexionsmorphologie aller analysierten Wortarten beschreibt. Für das Höchstalemannische gibt es keine Grammatiken eines Stadtdialekts. Drittens sollen auch alemannische Sprachinseln einbezogen werden. Dazu konnten \isi{Ortsgrammatiken} zu den Dialekten von Issime (Höchstalemannisch) sowie von Petrifeld und Elisabethtal (beide Schwäbisch) gefunden werden. Was bei der Wahl der Grammatiken nicht berücksichtigt werden konnte, war, dass alle Grammatiken in derselben Zeit entstanden sind. Die älteste verwendete Grammatik stammt aus dem Jahr 1886, die jüngste aus dem Jahr 1999. Die meisten Grammatiken sind jedoch aus der ersten Hälfte des 20. Jh., wodurch die Vergleichbarkeit wieder besser gegeben ist.

\begin{table}
\caption{Die untersuchten alemannischen Dialekte und ihre sprachexternen Eigenschaften}\label{table3.2}
\small\begin{tabularx}{\textwidth}{lllcXX}
\lsptoprule
{DG} & {Dialekt} & \multicolumn{1}{p{1cm}}{Stadt\slash\newline Land} & \multicolumn{1}{p{.75cm}}{geogr.\newline isoliert} & {Staat} & {Quellen}\\\midrule
h-st\il{Höchstalemannisch} & Issime & Land & \ding{52} & Italien & \citet{Zürrer1999} \citet{Perinetto1981}\\
& Visperterminen & Land & \ding{52} & Schweiz & \citet{Wipf1911}\\
& Jaun & Land & \ding{52} & Schweiz & \citet{Stucki1917}\\
& Sensebezirk & Land & \ding{55} & Schweiz & \citet{Henzen1927}\\
& Uri & Land & \ding{55} & Schweiz & \citet{Clauß1929}\\
hoch\il{Hochalemannisch} & Vorarlberg & Land & \ding{52} & Österreich & \citet{Jutz1925}\\
& Zürich & Stadt & \ding{55} & Schweiz & \citet{Weber1987}\\
& Bern & Stadt & \ding{55} & Schweiz & \citet{Marti1985}\\
schw & Huzenbach & Land & \ding{52} & Deutschland & \citet{Baur1967}\\
& Bad Saulgau & Land & \ding{55} & Deutschland & \citet{Raichle1932}\\
& Stuttgart & Stadt & \ding{55} & Deutschland & \citet{Frey1975}\\
& Petrifeld & Land & \ding{55} & Rumänien & \citet{Moser1937}\\
& Elisabethtal & Land & \ding{55} & Georgien & \citet{Žirmunskij1928/29}\\
oberr\il{Oberrheinalemannisch} (Baden) & Kaiserstuhl & Land & \ding{55} & Deutschland & \citet{Noth1993}\\
oberr (Elsass) & Münstertal & Land & \ding{52} & Frankreich & \citet{Mankel1886}\\
 & Elsass (Ebene) & Land & \ding{55} & Frankreich & \citet{Beyer1963}\\
 & Colmar & Stadt & \ding{55} & Frankreich & \citet{Henry1900}\\
\lspbottomrule
\end{tabularx}
\end{table}

In der Folge werden alle 17 alemannischen Dialekte bzw. die Orte, in denen sie gesprochen werden, kurz vorgestellt. Dabei liegt der Fokus vor allem auf der Darstellung, inwiefern ein Dialekt bzw. Ort \isi{geografisch isoliert} ist und zu welchem Typ Sprachgemeinschaft nach  \citeauthor{Trudgill2011}s \citeyearpar{Trudgill2011} Definition der Dialekt gehört (vgl. \sectref{2.2.3}, \sectref{3.2.3} und \sectref{3.2.5}.). Die Reihenfolge entspricht jener in \tabref{table3.2}.

\subsubsection{Issime}

Issime ist eine der höchstalemannischen Walser Sprachinseln in den Alpen des Aostatals (Italien), die im 13. Jh. entstanden sind. Genauer liegt Issime im Lystal, einem Seitental des Aostatals, auf 953 m.ü.M \citep[25]{Zürrer1999} und hat 432 Einwohner \citep{Issime2013}. Der Ort kann also als topografisch isoliert gelten, denn zudem „liessen sich [die Walser Kolonisten] in unwirtlichen Höhen nieder“, wie z.\,B.\ an Steilhängen, Terrassen, schmalen Hangleisten etc. \citep[31]{Zürrer1999}. Der Kontakt der Sprachinseln im Aostatal zum Deutschwallis wurde erst Ende des 19. Jh. unterbrochen, als Schienen- und Straßennetze gebaut und die alten Saumpässe zum Wallis vernachlässigt wurden \citep[28]{Zürrer1999}. Dies gilt jedoch nicht für Issime, das schon immer kaum Beziehungen zum Wallis hatte. Vielmehr gab es in Issime eine Teilassimilation an die Umgebungsgesellschaft \citep[28]{Zürrer1999}. Des Weiteren war in Issime Deutsch nie Schriftsprache oder Sprache der Schule und der Kirche \citep[29--30]{Zürrer1999}. Dazu kommt, dass die Walser Sprachinseln nur sehr wenig Kontakt untereinander hatten \citep[28]{Zürrer1999}.

Der Kontakt zu den benachbarten romanischen Sprachen hat dazu geführt, dass die Bewohner von Issime mehrsprachig sind. Zu den im Aostatal gesprochenen romanischen Sprachen gehören Piemontese, Dialekte des Franco-Provençal, sowie die Standardvarietäten des Französischen und Italienischen (beide offizielle Sprachen der autonomen Region Aostatal), wobei vor allem Standardfranzösisch und Franco-Provençal eine große Rolle gespielt haben \citep[28]{Zürrer1999}. Der alemannische Dialekt wird nur innerhalb von Issime und nur unter Alemannisch-Muttersprachlern gesprochen \citep[37]{Zürrer1999}. Beispielsweise wird sogar mit den Gressoneyer Walsern (13km entferntes Dorf, das ebenfalls eine Walser\linebreak Sprachinsel ist) auf Italienisch oder Piemontesisch gesprochen \citep[37]{Zürrer1999}.

Dass sich der Dialekt von Issime in besonderem Maße von den übrigen Dialekten entfernt hat, zeigen auch Zürrers Ausführungen auf Grundlage eigner Erfahrungen. Im Rahmen der SDS-Aufnahmen in den 1960er Jahren führte Zürrer als Schweizer Hochalemannischsprecher die Interviews in Issime auf Französisch \citep[38]{Zürrer1999}.

Für Issime kann also festgehalten werden, dass es sich dabei um eine kleine, \isi{isolierte Sprachgemeinschaft} handelt, deren Sprache nur innerhalb des Dorfes und nur mit Muttersprachlern gesprochen wird. Durch die Mehrsprachigkeit sind jedoch auch \textit{Additive Borrowings} zu erwarten. Ein Fall ist tatsächlich unter den \isi{Personalpronomen} zu finden (vgl. \sectref{5.3.1}).

\subsubsection{Visperterminen}

Vispterterminen ist ein Dorf in den Walliser Alpen (Kanton Wallis, Schweiz), das auf 1378 m.ü.M. liegt und zur Zeit der Erhebung ca. 600 Einwohner hatte \citep[1]{Wipf1911}. Noch heute führt nur eine Straße nach Visperterminen, die kurz nach dem Ort endet. Das Ziel von Wipf war, den Dialekt „eines möglichst abgelegenen, noch nicht von dem großen Touristenstrome ergriffenen Walliser Dorfes“ zu beschreiben \citep[1]{Wipf1911}. Außerdem beschränkten sich die Kontakte außerhalb des Dorfes vorwiegend auf einige Einkäufe in Visp und Stalden (beide im Tal gelegen) und auf den Militärdienst in Sitten, Brig (beide Wallis) und Chur (Graubünden) \citep[1--2]{Wipf1911}. Wie Issime kann also auch Visperterminen zu den kleinen und \isi{isolierten Sprachgemeinschaften} gezählt werden. Im Gegensatz zu Issime jedoch wird Visperterminen Alemannisch von einer einsprachigen Sprachgemeinschaft gesprochen und ist vom Deutschen überdacht.

\subsubsection{Jaun}

Jaun ist ein Bergdorf in den Freiburger Voralpen (Kanton Freiburg, Schweiz) und liegt auf 1030 m.ü.M. \citep[1]{Stucki1917}. Zur Zeit der Erhebung hatte das Dorf 802 Einwohner \citep[10]{Stucki1917}. Jaun ist das vorletzte Dorf des Jauntals. Dahinter liegt nur Abläntschen, ein Bergdorf mit etwa 100 Einwohnern, das der Gemeinde Saanen (Kanton Bern) zugeteilt ist und dessen Dialekt zum westlichen Berner Oberländischen gehört \citep[2--3]{Stucki1917}. Verbunden mit Jaun war es durch ein „schlechtes Fahrsträßchen“ \citep[2]{Stucki1917}. Mit Ausnahme von sehr geringem Handel und der Post (eine Person aus Abläntschen holte täglich die Post in Jaun ab), gab es nur wenig Kontakt zwischen den beiden Dörfern. Die Bewohner von Abläntschen tätigten die Behördengänge in Saanen (Kanton Bern), wohin sie auch neben Zweisimmen (Kanton Bern) zu den großen Märkten gingen \citep[3]{Stucki1917}. Für die Bewohner von Jaun ist Bulle (französischsprachig, Kanton Freiburg) \citep[1]{Stucki1917} der Ort für Behördengänge und Märkte. Schließlich trennte die unterschiedliche Religion die beiden Dörfer: Während die Bewohner von Jaun mehrheitlich katholisch sind, sind die Bewohner von Abläntschen protestantisch, was zur damaligen Zeit einen engen Kontakt zwischen den Dörfern verhinderte \citep[3 ,9]{Stucki1917}. Mit zwei weiteren deutschsprachigen Dörfern ist Jaun durch Schotterwege oder Pfade über Pässe (Scheitelpunkt über 1500 m) verbunden: Boltigen (Simmental, Kanton Bern) und Plaffeien (Sensebezirk, Kanton Freiburg) \citep[2--3]{Stucki1917}. Es wäre anzunehmen, dass Jaun gerade mit Plaffeien viel Kontakt hatte: ähnlicher Dialekt, Kanton Freiburg, katholisch. Nach Plaffeien führte jedoch nur ein Pfad, der als „beschwerlich [und] durchweg schlecht unterhalten“ beschrieben wird \citep[3]{Stucki1917}, wohin noch heute nur ein Wanderweg führt. Die Dörfer, die im Tal vor Jaun liegen, wie auch der gesamte Bezirk Greyerz, zu dem Jaun gehört, sind französischsprachig. Seit 1875 (also 42 Jahre bevor Stuckis Grammatik erschien) verbindet „eine gute Fahrstraße“ das Dorf Jaun talabwärts mit den französischsprachigen Orten Charmey, Broc und vor allem Bulle \citep[1]{Stucki1917}. Jaun kann also definitiv als ein kleiner, isolierter Ort gelten, dessen Dialekt zur damaligen Zeit vorwiegend im Dorf selbst gesprochen wurde (die meisten Kontakte nach außen führten ins französischsprachige Gebiet) und der praktisch keine L2-Ler\-ner hat.

\subsubsection{Sensebezirk}\largerpage

Die Grammatik von \citet{Henzen1927} erfasst den Dialekt des Sensebezirks (hauptsächlich katholisch) sowie der neun katholischen Gemeinden der Pfarrei Gurmels im Seebezirk des Kantons Freiburg (Schweiz). Im weiteren Verlauf der Arbeit wird der Einfachheit halber auf dieses gesamte Gebiet mit \textit{Sensebezirk} referiert. Dieses Gebiet ist zwar klar begrenzt: Im Süden die Voralpen, im Westen das französischsprachige Gebiet, im Osten und Norden hochalemannisches berndeutsches Gebiet. Zum Berner Gebiet gab es im Gegensatz zu heute nur wenig Kontakte, weil dieses hauptsächlich protestantisch war \citep[2]{Henzen1927}. Die Mobilität innerhalb dieses Gebietes war jedoch relativ hoch, und zwar vor allem durch Heirat und durch die Kleinbauern, die ihren Wohnort oft wechseln mussten \citep[9]{Henzen1927}. Nur in einigen Gemeinden des voralpinen Oberlandes ist dies deutlich weniger ausgeprägt \citep[8--9]{Henzen1927}. Dies führte dazu, dass die Dialekte der einzelnen Dörfer keine sehr großen Unterschiede aufwiesen. \citet{Henzen1927} spricht von einem „Mischdialekt“ \citep[1]{Henzen1927}. Des Weiteren ist die Stadt Freiburg aus der Erhebung ausgeschlossen, einbezogen wurden nur Dörfer mit durchschnittlich 838 Einwohnern \citep[8]{Henzen1927}. Es handelt sich hierbei also um einen Landdialekt, der im Vergleich zu Issime, Visperterminen und Jaun jedoch nicht isoliert ist.

\subsubsection{Uri}\largerpage

Die Grammatik von \citet{Clauß1929} umfasst das Reußtal und seine Seitentäler (ohne Talschaft Urseren) im Kanton Uri (Schweiz). Es handelt sich dabei um ein in den Alpen zwischen Urnersee und Gotthard liegendes Tal \citep[1]{Clauß1929}. Laut \citet{Clauß1929} kann das untersuchte Gebiet linguistisch als weitgehend einheitlich angesehen werden. Ein geschlossenes Gebiet ist vor allem das Reußtal, die Dialekte der Seitentäler weisen, wenn überhaupt, vor allem phonologische Unterschiede auf \citep[11--12]{Clauß1929}. Bei den erhobenen Orten handelt es sich um Dörfer. Zwar wurde auch Altdorf, der Hauptort des Kantons Uri, einbezogen, der jedoch zum Zeitpunkt der Erhebung um 1920 nur ca. 4000 Einwohner zählte (Historisches Lexikon der Schweiz ~\citeyear{LexA2011}, ~\citetitle{LexA2011}).

Schließlich stellt sich noch die Frage nach der Isoliertheit. Seit mindestens der römischen Zeit wird der Gotthard als Verkehrsweg genutzt, wenn auch kein kontinuierlicher Verkehrsfluss nachgewiesen werden kann (Historisches Lexikon der Schweiz ~\citeyear{LexG2011},~\citetitle{LexG2011}). Um 1200 entstand die erste Brücke über die Schöllenen, zwischen 1166/1176 und 1230 wurde eine Kapelle auf der Passhöhe eingeweiht (Historisches Lexikon der Schweiz ~\citeyear{LexG2011},~\citetitle{LexG2011}). 1707 wurde der erste Tunnel der Alpen auf der Urner Seite gebaut (Historisches Lexikon der Schweiz ~\citeyear{LexG2011},~\citetitle{LexG2011}). Zwischen 1810 und 1831 wurde der Pass für Kutschen (im Winter für Postschlitten) fahrbar gemacht (Historisches Lexikon der Schweiz ~\citeyear{LexG2011},~\citetitle{LexG2011}). Folglich kann vor allem das Reußtal nicht als isoliert gelten Bei diesem Untersuchungsareal handelt es sich also um ein ländliches, nicht isoliertes Gebiet.

\subsubsection{Vorarlberg}

Im weiteren Verlauf dieser Arbeit wird dieses Gebiet \textit{Vorarlberg} genannt, obwohl nur der Süden Vorarlbergs (ohne die Walsertäler), aber zusätzlich das Fürstentum Liechtenstein von \citet{Jutz1925} erhoben wurden. Zum Untersuchungsgebiet des südlichen Vorarlberg gehören der Walgau (unteres Illtal), das Montafon (oberes Illtal) und das Klostertal \citep[3]{Jutz1925}. Größere Städte gab es zu Beginn des 20. Jh. in diesem Gebiet nicht. Bezüglich Kontakte/\isi{Isolation} beschreibt \citet{Jutz1925} vorwiegend das Ill- und Klostertal. Zwischen dem Walgau und dem Rheintal existierte eine alte Straße von Rankweil über Göfis \citep[6]{Jutz1925}, wobei \citet{Jutz1925} aber nicht spezifiziert, was er genau mit ‘alt’ meint. Er merkt jedoch an, dass der Dialekt des Walgau von jenem des Rheintals beeinflusst ist \citep[3]{Jutz1925}. Kaum Einfluss gab es von Feldkirch her, da der Verkehrsweg durch die Illschlucht erst relativ jung war \citep[5]{Jutz1925}. Die Dialekte im Montafon und im Klostertal waren kaum vom Rheintal beeinflusst \citep[3]{Jutz1925}. Vielmehr sieht \citet{Jutz1925} für das gesamte Gebiet einen wachsenden schriftsprachlichen Einfluss durch Schule, Kirche, Industrie und Tourismus, der jedoch vorwiegend den Wortschatz betraf \citep[4--5]{Jutz1925}. Allgemein kann jedoch das Ill- und Klostertal zur damaligen Zeit durch seine geografische Lage als abgeschieden gelten \citep[3]{Jutz1925}. Auch für dieses Gebiet kann also von einer eher kleinen, \isi{isolierten Sprachgemeinschaft} ausgegangen werden, und zwar vor allem auch im Vergleich zu den beiden anderen hochalemannischen Dialekten (Bern und Zürich), die hier analysiert werden. Wichtig ist in diesem Zusammenhang noch \citeauthor{Jutz1925}ʼ \citeyearpar{Jutz1925} Anmerkung, dass „der Grundzug der Maa. [= Mundarten] überall derselbe ist“ und dass auch höher gelegene Orte berücksichtigt werden, da diese einen archaischeren Dialekt aufweisen \citep[7, 9]{Jutz1925}. Gibt es Unterschiede zwischen den Gebieten, so sind diese markiert. In diesem Fall werden in dieser Arbeit die Varianten des Montafon und des Klostertals verwendet, da diese abgeschiedener als das Walgau liegen.

Schließlich ist noch zu erwähnen, dass in der vorliegenden Arbeit Vorarlberg zum Hochalemannischen gezählt wird. \citet{Jutz1925} ordnet dieses Gebiet dem Niederalemannischen zu, weil germ. \textit{k} im Anlaut nur bis zur Affrikata und nicht weiter zum Frikativ verschoben ist (\citealt[9]{Jutz1925}, vgl. auch VALTS, Bd. III, Karten 40–53). Da aber das Niederalemannische (z.\,B.\ Bodenseealemannisch und Oberrheinalemannisch) germ. \textit{k} im Anlaut nicht verschoben hat (SSA, Karten II 105.00–105.05), wird das von \citet{Jutz1925} untersuchte Gebiet zum Hochalemannischen gerechnet. Ein weiteres Abgrenzungskriterium sind die gerundeten Vordervokale, deren Isoglosse im äußersten Norden von Vorarlberg verläuft. \footnote{Herzlichen Dank an Oliver Schallert für diesen Hinweis.}

\subsubsection{Zürich}

Als die Grammatik von \citet{Weber1987} erschien, hatte die Stadt Zürich bereits 383.568 Einwohner \citep{Zürich2015}. Zürich liegt im Schweizer Mittelland, d.h., sie ist topografisch nicht isoliert. Diese Sprachgemeinschaft kann also zum Typ der großen Sprachgemeinschaften mit viel Kontakt gezählt werden. Der Dialekt von Zürich gehört zu den östlichen hochalemannischen Dialekten \citep[51–67]{Hotzenköcherle1984}. 

\subsubsection{Bern}

Auch die Sprachgemeinschaft der Stadt Bern kann zum Typ der großen Sprachgemeinschaften mit viel Kontakt gerechnet werden. Wie Zürich liegt auch Bern im Schweizer Mittelland, es gibt also keine topografischen Gegebenheiten, die Bern isolieren. In der Zeit der Publikation von \citeauthor{Marti1985}s \citeyearpar{Marti1985} Grammatik hatte die Stadt Bern 145.254 Einwohner (Zahlen für das Jahr 1980, \citealt{Bern2014}). Da diese Grammatik deutlich später erschienen ist als die meisten anderen Grammatiken, die für diese Arbeit berücksichtigt wurden, ist es wichtig, die Zahlen aus der ersten Hälfte des 20. Jh. für Bern zu betrachten: 90.937 Einwohner im Jahr 1910, 111.783 Einwohner im Jahr 1930. Die Sprachgemeinschaft kann also auch für die erste Hälfte des 20. Jh. zum anfangs genannten Typ gezählt werden. Der Dialekt von Bern ist wie jener von Zürich Teil des Hochalemannischen, und zwar des westlichen Hochalemannisch \citep[51--67]{Hotzenköcherle1984}.

\subsubsection{Huzenbach}\largerpage[1]

Huzenbach liegt zwischen 450 und 950 m.ü.M. im oberen Murgtal im Schwarzwald (Deutschland) und kann mit seinen 753 Einwohnern als Dorf in einer ländlichen Gegend gelten \citep{Baiersbronn2015}.\footnote{Ältere Daten zu den Einwohnern konnten leider nicht gefunden werden.} Huzenbach gehört zur Gemeinde Baiersbronn, von der es jedoch 12 km entfernt situiert ist. Obwohl heute eine Bundesstraße durch das Murgtal führt, war der obere Teil dieses Tals lange Zeit isoliert. Huzenbach liegt an keinem der vier älteren Hauptübergänge durch den Schwarzwald (z.\,B.\ Römerstraße) \citep[28–29]{Baur1967}. Erst am Ende des 18. Jhs. wurde das hintere mit dem vorderen Murgtal durch eine Straße nach Gernsbach verbunden \citep[29]{Baur1967}. Am Anfang des 20. Jhs. wurde das ganze Tal ans Eisenbahnnetz angeschlossen, was die Ansiedlung von Industrie und den Tourismus im Murg- und Kinzigtal förderte \citep[30]{Baur1967}. Im Vergleich zu den beiden schwäbischen Orten Bad Saulgau und Stuttgart kann Huzenbach folglich als klein und isoliert gelten.

\subsubsection{Bad Saulgau}

Bad Saulgau liegt im östlichen Teil des Landkreises Sigmaringen (Ba\-den-Würt\-tem\-berg, Deutschland), zwischen Biberach und dem Bodensee. Mit 17.080 Einwohnern\footnote{Vgl. Fußnote 6.} und einer mittleren Besiedlungsdichte \citep{Bund2014} gilt der Ort als halbstädtisch (vgl. \citealt{Eurostat2011}). Die Grammatik von \citet{Raichle1932} berücksichtigt neben Bad Saulgau auch etliche Dörfer in der Umgebung der Kleinstadt. Im Gegensatz zu Stuttgart handelt es sich also um eine eher ländliche Gegend, die jedoch im Vergleich mit Huzenbach nicht als isoliert gelten kann.

\subsubsection{Stuttgart}

Stuttgart ist die Landeshauptstadt von Ba\-den-Würt\-tem\-berg. Bei der Publikation der Grammatik von Frey im Jahr 1975 hatte Stuttgart 628.598 Einwohner \citep{Stuttgart2015}. Aufgrund ihrer Bevölkerungsdichte gilt sie als städtisch ( \citealt{Bund2014}, vgl. \citealt{Eurostat2011} ). Wie die Sprachgemeinschaften von Bern und Zürich kann auch jene von Stuttgart zu den großen Sprachgemeinschaften mit vielen Kontakten und losen Netzwerken gezählt werden.

\subsubsection{Petrifeld}

Petrifeld (rum. Petrești, ung. Mezőpetri) liegt im Sathmargebiet. Dieses gehörte zu Ungarn und ging 1919 an Rumänien \citep[14]{Moser1937}. In diesem Gebiet gab es 33 schwäbische Siedlungen, wovon drei nach 1919 zu Ungarn gehörten und 30 zu Rumänien \citep[15--16]{Moser1937}. Die Bevölkerungszahl pro Dorf lag zwischen 600 und 2000. Die schwäbischen Einwanderer waren überwiegend katholisch \citep[16]{Moser1937} und stammten aus dem Gebiet zwischen Donau, Iller und Bodensee \citep[Karte 1 und 2]{Moser1937}, also genau aus jenem Gebiet, in dem auch Bad Saulgau liegt. In den rumänischen Bezirken Sălai und Sătmar gaben 1927 ca. 6\% der Bewohner an, deutscher Herkunft zu sein \citep[15--16]{Moser1937}. Die schwäbischen Dörfer selbst waren jedoch vollständig oder vorwiegend deutsch \citep[16]{Moser1937}, wobei es aber große Unterschiede in der Kenntnis des Schwäbischen zwischen den Bewohnern der einzelnen Dörfer gab. Durchschnittlich führten ca. 2/3 der Bevölkerung an, Schwäbisch oder Standarddeutsch zu können \citep[17]{Moser1937}.

Die Grammatik von \citet{Moser1937} beschreibt den Dialekt von Petrifeld, der stellvertretend für das gesamte Sathmargebiet steht, denn die Dialekte aller Orte wiesen große Ähnlichkeiten auf \citep[23]{Moser1937}. Es kann also angenommen werden, dass es Kontakte zwischen den schwäbischsprachigen Dörfern gab. Detaillierte Infomationen zu diesen möglichen Kontakten liegen jedoch nicht vor. Petrifeld gehörte zur Westgruppe, welche eine geschlossenen Siedlungsgruppe bildete und von ungarischen und rumänischen Dörfern umgeben war \citep[17]{Moser1937}. Gegründet wurde Petrifeld 1740–1741 \citep[19]{Moser1937}. 1930 zählte das Dorf 1588 Bewohner, wobei folgende Muttersprachen angegeben wurden: 1276 Bewohner Deutsch, 264 Bewohner Ungarisch, 30 Bewohner Romani, 18 Bewohner Rumänisch \citep[61]{Varga2002}. Laut \citet{Moser1937} beeinflussten das Rumänische und Ungarische vor allem den Schwäbischen Wortschatz \citep[102]{Moser1937}, wobei die Lehnwörter zumeist phonologisch integriert wurden \citep[103]{Moser1937}. Einfluss übte besonders das Ungarische als Verwaltungs- und Handelssprache, weniger das Rumänische \citep[103]{Moser1937}.

Wir haben es hier also mit einer mehrheitlich ländlichen Gegend zu tun, die jedoch geografisch nicht als isoliert gelten kann. Da \citet{Moser1937} keine genauen Angaben macht, wie kompetent die deutschsprachigen Bewohner in der ungarischen und rumänischen Sprache waren, ist bezüglich möglicher \textit{Additive Borrowings} keine Voraussage möglich. Vielmehr wird sich zeigen, dass der Dialekt von Petrifeld zwar rumänische und ungarische Lehnwörter aufweist, jedoch keine \textit{Additive Borrowings} in der Flexionsmorphologie.

\subsubsection{Elisabethtal}

Die Grammatik von \citet{Žirmunskij1928/29} basiert auf den Dialekten der Dörfer Katharinenfeld (georg. Bolnissi) und Elisabethtal (georg. Asureti) \citep[38]{Žirmunskij1928/29}, die sich im heutigen Georgien befinden. Bei den Bewohnern handelte es sich um Schwaben aus dem Neckartal, ungefähr zwischen Stuttgart und Esslingen im Norden, Tübingen und Reutlingen im Süden \citep[56]{Žirmunskij1928/29}. Diese waren Pietisten, welche sich von der lutherischen Kirche getrennt hatten \citep[39]{Žirmunskij1928/29}. Neben religiösen waren es besonders wirtschaftliche Gründe, die sie zum Auswandern bewogen: große Bevölkerung, Krieg, relativ hohe Steuerlast, schlechte Ernten, Hungersnot etc. \citep[41]{Žirmunskij1928/29}. Die Auswanderung in die Südukraine und in den Südkaukasus fand zwischen 1816 und 1819 statt \citep[39]{Žirmunskij1928/29}. Wie viele Personen nach Katharinenfeld und Elisabethtal kamen, nennt \citet{Žirmunskij1928/29} nicht. Jedoch gibt er an, dass ins südkaukasische Gebiet 2629 Menschen kamen, die sich auf 7 Orte verteilten \citep[42]{Žirmunskij1928/29}. Es kann also festgehalten werden, dass es sich um kleine Dörfer gehandelt hat. Laut \citet{Schrenk1997} lebten in den 1860er Jahren 851 Personen in Elisabethtal \citep[202–203]{Žirmunskij1928/29}. Des Weiteren sind keine Details über den Kontakt zwischen den Kolonien in Georgien bekannt. Die Menschen lebten aber zumindest nicht völlig isoliert voneinander, da die Kolonien einen gemeinsamen Fonds für gemeinnützige Zwecke hatten, in den jede Gemeinde einbezahlte \citep[198]{Schrenk1997}. Des Weiteren verfügten sie über eine gemeinsame Synode \citep[199]{Schrenk1997}.

Laut \citet{Žirmunskij1928/29} werden diese schwäbischen Mundarten von der\linebreak deutschen geschriebenen Standardsprache beeinflusst \citep[58]{Žirmunskij1928/29}. Neben einigen russischen Lehnwörtern (vgl. \citealt[52]{Žirmunskij1928/29}) konnten in der Flexionsmorphologie keine \textit{Additive Borrowings} gefunden werden. Dasselbe wurde bereits für Petrifeld festgestellt. Im Gegensatz zu diesen beiden Sprachinseln gibt es jedoch im Dialekt von Issime ein \textit{Additive Borrowing} (vgl. \sectref{5.3.1}). Dies könnte dadurch erklärt werden, dass Issime (13. Jh.) eine deutlich ältere Sprachinsel ist als Petrifeld (18. Jh.) und Elisabethtal (19. Jh.).

Schließlich ist noch zu erwähnen, dass \citet{Žirmunskij1928/29} die Varianten sowohl für Elisabethtal als auch für Katharinenfeld angibt, wenn sich diese voneinander unterscheiden. In diesem Fall wird in der vorliegenden Arbeit die Variante von Elisabethtal aufgenommen, da die Mundart von Elisabethtal die am besten erhaltene war \citep[58]{Žirmunskij1928/29}. Wir können also festhalten, dass es sich bei Elisabethtal wie bei Petrifeld um ein Dorf in einer ländlichen Gegend handelte, welches geografisch nicht isoliert und wohl auch sozial zumindest nicht völlig isoliert war.

\subsubsection{Kaiserstuhl}

Die Grammatik von \citet{Noth1993} basiert auf dem Dialekt von Rotweil (heute: politische Gemeinde Vogtsburg-Oberrotweil) \citep[293]{Noth1993}. Rotweil liegt an der westlichen Seite des Kaiserstuhls (Ba\-den-Würt\-tem\-berg, Deutschland), ca. 25 km von Freiburg entfernt. Zum Rhein sind es etwa 5 km, der gleichzeitig die Staatsgrenze zwischen Frankreich und Deutschland bildet. Im weiteren Verlauf der Arbeit wird auf diesen Dialekt immer mit \textit{Kaiserstuhl} referiert. Da Rotweil nicht im Zentrum, sondern an der Peripherie des Kaiserstuhlgebirges liegt, kann Rotweil als nicht isoliert gelten. Die Gemeinde Vogtsburg-Oberrotweil hat 5737 Einwohner, weist eine geringe Besiedlungsdichte auf und kann folglich als ländlich charakterisiert werden (Statistisches Bundesamt).

\subsubsection{Münstertal}

Die Grammatik von \citet{Mankel1886} beschreibt den Dialekt des Münstertals, das in den Vogesen unweit von Colmar liegt (Elsass, Frankreich). Das Haupttal führt von Colmar bis nach Münster, dazwischen liegen etwa 15 km \citep[1]{Mankel1886}. Bei Münster gabelt sich das Tal in das sogenannte Großtal und Kleintal, welche 8–9 km lang sind \citep[1]{Mankel1886}. Beschrieben wird die Grammatik des Großtals, da dieses am meisten von den übrigen elsässischen Dialekten abweicht \citep[2]{Mankel1886}. Alle drei Täler, aber besonders das Großtal können als isoliert gelten, denn die Bewohner dieser Täler hatten aufgrund der Berge kaum Kontakt mit anderen Gebieten \citep[1]{Mankel1886}. \citet{Mankel1886} weist darauf hin, dass sich wegen dieser \isi{Isolation} die Dialekte „eigenartig ausgebildet“ haben \citep[1]{Mankel1886}. Übrigens führt heute zwar eine \textit{Route Nationale} durch das Haupt- und Kleintal, jedoch nur eine \textit{Route Départementale} durch das Großtal. Des Weiteren handelt es sich bei den Orten in allen drei Tälern um Dörfer. Die erhobenen Orte im Großtal hatten im Jahr 1886 zwischen 996 und 1091 Einwohner, Münster (Hauptort und größter Ort des Münstertals) 1886 wie auch heute ca. 5.000 Einwohner \citep{MotteVouloirSarrabezolles2015}. Die erhobenen Dörfer des Großtals sind: Mühlbach (frz. Muhlbach-sur-Munster), Breitenbach (frz. Breitenbach-Haut-Rhin), Metzeral und Sondernach \citep[2]{Mankel1886}.

Die Sprachgemeinschaft im Großtal kann also als klein, isoliert, mit wenig Kontakten und engen Netzwerken charakterisiert werden. Wichtig ist hier noch zu erwähnen, dass die Variante des Großtals in die vorliegende Arbeit übernommen wurde, wenn zwei Varianten für das Großtal und das Haupt-/Kleintal angegeben werden.

Neben dem Dialekt des Münstertals gehören zum untersuchten Sample auch der Dialekt von Colmar und jener der elsässischen Rheinebene, welche in den nachfolgenden Kapiteln vorgestellt werden. Da heutzutage die Sprecher elsässischer Dialekte mindestens bilingual sind, stellt sich die Frage, wie lange dies schon der Fall ist. Deswegen wird in der Folge ein Teil der elsässischen Sprachgeschichte skizziert, wobei die Ausführungen äußerst kurz gefasst sind, denn eine ausführliche Darstellung würde den Rahmen dieser Arbeit sprengen.

Während der Völkerwanderung siedelten sich Alemannen und Franken im heutigen Elsass an, das vorher romanischsprachiges (und evtl. keltisches) Gebiet war \citep[2]{Lösch1997}. Zum Königreich Frankreich gehörte das Elsass erst ab 1681. Außer dem Adel und Teilen des gehobenen Bürgertums, die Französisch konnten, wurden im Elsass weiterhin alemannische Dialekte gesprochen \citep[7]{Lösch1997}. Erst nach der Französischen Revolution kann eine Französisierung des Gebiets durch eine auch repressive Sprachpolitik festgestellt werden \citep[7--10]{Lösch1997}. Trotzdem hielten sich die elsässischen Dialekte wie auch die deutsche Standardsprache, wohl auch, weil es keine Schulpflicht gab und Messen auf Deutsch gehalten wurden \citep[10--11]{Lösch1997}. 1871–1918 gehörte das Elsass zum Deutschen Kaiserreich. Die Zeit bis 1914 ist von einer gewissen sprachlichen Liberalität geprägt, da es je nach Gemeinde deutsche und französische Schulen gab und auch Literatur und Zeitungen in beiden Sprachen vertrieben werden konnte \citep[12]{Lösch1997}. Während des 1. Weltkrieges durfte nur noch auf Deutsch unterrichtet werden \citep[16]{Lösch1997}. In der Zeit von 1871 bis 1918 war jedoch vor allem das Bürgertum zweisprachig, Bauern und Arbeiter vorwiegend deutschsprachig \citep[13]{Lösch1997}. 1918–1940 gehörte das Elsass wieder zu Frankreich. Bis 1927 wurde an den Schulen ausschließlich auf Französisch unterrichtet, ab 1927 war Deutsch „in eingeschränktem Maße in den Schulen wieder zugelassen“ \citep[18]{Lösch1997}. 1940 wurde das Elsass de facto dem Deutschen Reich einverleibt \citep[20]{Lösch1997}. Die Nationalsozialisten führten eine „Entfranzösisierungskampagne“ \citep[20]{Lösch1997}, woraus u.a. resultierte, dass Französischsprachige vertrieben wurden und Deutsch die einzige Unterrichtssprache war \citep[21]{Lösch1997}. 1944 wurde das Elsass „von amerikanischen und französischen Verbänden […] zurückerobert“ und gehörte nach Kriegsende wieder zu Frankreich \citep[21]{Lösch1997}. Französisch war wieder Sprache des Unterrichts und des öffentlichen Lebens, Deutsch spielte in den Medien kaum noch eine Rolle \citep[22]{Lösch1997}. Dadurch dehnte sich Französisch auch stärker in den privaten und familiären Bereich aus \citep[25]{Lösch1997}. Erst in den 1980er Jahren im Zuge der Regionalisierung Frankreichs änderte sich die Situation, „Schulen, Radio und Fernsehen sollten den Regionalsprachen eröffnet werden“ \citep[26]{Lösch1997}. Daraus kann geschlossen werden, dass sich eine deutsch-französische Zweisprachigkeit erst seit der Französischen Revolution allmählich ausbreitete. Diese doch eher kurze Zeit ist vielleicht auch ein Grund, weshalb in der nominalen Flexionsmorphologie der elsässischen Dialekte keine \textit{Additive Borrowings} gefunden werden können. Ähnliches wurde bereits in Bezug auf Petrifeld und Elisabethtal beobachtet. Demgegenüber hat Issime eine viel längere Geschichte der Mehrsprachigkeit (seit dem 13. Jh.).

\subsubsection{Elsass (Ebene)}

Die Grammatik von \citet{Beyer1963} beschreibt die nominale Flexionsmorphologie des gesamten Elsass (Frankreich). Gibt es zwischen Regionen Unterschiede, werden diese genannt und geografisch eingeordnet. In diesen Fällen wurden für die hier vorgestellte Auswertung jeweils die Varianten des Zentrums (im Gegensatz zu nördlichen und südlichen Varianten) gewählt, da dieses Gebiet ungefähr zwischen dem Kaiserstuhl und dem Münstertal liegt.

Welche Orte genau erhoben wurden, ist aus der sonst sehr detaillierten Beschreibung nicht ersichtlich. Der größte Teil des Gebiets liegt jedoch in der Rheinebene (weswegen im weiteren Verlauf darauf mit \textit{Elsass (Ebene)} referiert wird), in der neben den beiden größeren Städten Strasbourg und Mulhouse und mittleren Städten wie Colmar auch viele Dörfer liegen. Da 180 Orte erhoben wurden \citep[15]{Beyer1963}, ist also davon auszugehen, dass es sich bei der Mehrheit der Orte um Dörfer handelt. Bei diesem Gebiet handelt es sich folglich nicht um ein isoliertes, aber um ein eher ländliches Gebiet.

\subsubsection{Colmar}

Colmar ist eine Stadt im Elsass (Frankreich), die im oben beschriebenen Gebiet Elsass (Ebene) liegt. Im Gegensatz zum Gebiet Elsass (Ebene), für das vorwiegend der Dialekt der Dörfer erhoben wurde, handelt es sich bei Colmar um eine Stadt. Im Jahr 1901 (die hier verwendete Grammatik erschien 1900) hatte Colmar 36.844 Einwohner \citep{MotteVouloirSarrabezolles2015}. Die Sprachgemeinschaft in Colmar kann um 1900 als eine große, nicht \isi{isolierte Sprachgemeinschaft} mit vielen Kontakten und losen Netzwerken charakterisiert werden.