\chapter{Strukturelle Komplexität in der Linguistik}\label{2}

\section{Grundsätzliche Überlegungen zur linguistischen Komplexität}\label{2.1}

\subsection{Wissenschaftsgeschichtlicher Überblick}\label{2.1.1}

\citet[1--5]{Kusters2003} und \citet{Sampson2009} geben einen ausgezeichneten Überblick darüber, ob, und wenn ja, auf welche Art und Weise, die Komplexität der Sprache in der Linguistik des 19. und 20. Jh. eine Rolle gespielt hat. Deswegen werden hier nur die wichtigsten Punkte kurz zusammengefasst, was natürlich eine gewisse Verallgemeinerung als Konsequenz hat. Die Übersicht ist chronologisch organisiert.

Im philosophischen Idealismus und der Romantik des 19. Jh. wurde das vermeintlich hohe Niveau europäischer Kultur mit der vorgeblich hohen Komplexität europäischer Sprachen in Verbindung gebracht \citep[2]{Kusters2003}. \citet{Humboldt1836} bringt dies wie folgt auf den Punkt: „Die Sprache ist gleichsam die äußerliche Erscheinung des Geistes der Völker; ihre Sprache ist ihr Geist und ihr Geist ihre Sprache […]“ \citep[53]{Humboldt1836}. Der angenommene Zusammenhang zwischen Sprache und Denken kann ebenfalls mit Humboldt illustriert werden, der davon ausging, dass flektierende Sprachen sich für komplexe Gedanken besser eignen als andere Sprachtypen \citep[3]{Kusters2003}. Mit den Junggrammatikern gegen Ende des 19. Jh. hatten Denken und Kultur keine so große Bedeutung mehr in der Sprache, wie dies im Idealismus und in der Romantik der Fall war. Sprache wurde als Naturphänomen (vgl. Ausnahmslosigkeit der Lautgesetze) angesehen und die Verwandtschaften zwischen den Sprachen (vgl. Stammbaumtheorie) standen im Vordergrund \citep[3]{Kusters2003}. Auch die Strukturalisten interessierten sich (zumindest in erster Linie) nicht für die Komplexität in der Sprache. Vielmehr ging es darum, ein umfassendes Instrument zu haben, um alle Sprachen beschreiben zu können \citep[4]{Kusters2003}. In der generativen Grammatik spielte Komplexität nur eine geringe Rolle, da Sprache als Teil der menschlichen Biologie und nicht der Kultur gesehen wird \citep[6]{Sampson2009}. Deswegen steht die I-Language im Fokus der Betrachtung, die E-Language wird für die linguistische Analyse als eher unwichtig angesehen. Ziel ist es, Übereinstimmungen zwischen Sprachen und den Sprachen zugrunde liegenden Universalien herauszuarbeiten \citep[4]{Kusters2003}. Unterschiede zwischen den Sprachen werden durch Unterschiede in der Derivation erklärt (z.\,B.\ Transformationen, Move und Merge etc.). Kürzlich erschienene Bände zeigen jedoch, dass linguistische Komplexität vermehrt auch innerhalb von generativen Modellen analysiert wird (z.\,B.\ \citealt{Culicover2013}, \citealt{TrotzkeBayer2015}). Die Aufsätze in dem von \citet{TrotzkeBayer2015} herausgegebene Band beschäftigen sich vor allem mit syntaktischer Komplexität (z.\,B.\ Rekursion, Derivation, Einbettung etc.) und deren Erwerb, die Monografie von \citet{Culicover2013} mit dem Sprachwandel, Spracherwerb und Prozessierung und deren Zusammenhang bzw. Einfluss auf linguistische Komplexität.

Im 20. Jh. stand also die Komplexität der Sprache nicht im Fokus des Interesses. Vielmehr galt, verallgemeinernd ausgedrückt, dass sich Sprachen in ihrer Komplexität nicht unterscheiden, was als \is{Equi-Complexity-Hypothese}\textit{Equi-Com\-ple\-xi\-ty-Hy\-po\-the\-se} bezeichnet wird. Dies kann auch als Gegenreaktion auf die Annahmen aus dem 19. Jh. verstanden werden. Gleichzeitig sprachen besonders Variationslinguisten im 20. Jh. immer wieder von Komplexitätsunterschieden, z.\,B.\ \citeauthor{Ferguson1959}s \citeyearpar{Ferguson1959} Unterscheidung zwischen \textit{High} und \textit{Low Varieties}. Darauf wird im anschließenden Kapitel noch genauer eingegangen.

Im 21. Jh. ist ein wachsendes Interesse an struktureller Komplexität in der Sprache zu beobachten. Dazu sind bereits etliche Aufsatzsammlungen und Dissertationen entstanden, wie, um nur einige wenige zu nennen, \citet{Kusters2003}, \citet{MiestamoSinnemäkiKarlsson2008}, \citet{SampsonGilTrudgill2009}, \citet{Sinnemäki2011}, \citet{SzmrecsanyiKortmann2012}. Dabei stehen vor allem folgende Fragen im Zentrum: Sind alle Sprachen gleich komplex? Wird höhere Komplexität in einem Subsystem (z.\,B.\ Morphologie) durch niedrigere Komplexität in einem anderen Subsystem (z.\,B.\ Syntax) ausgeglichen? Wie kann Komplexität gemessen werden? Wie können Unterschiede in der Komplexität zwischen Sprachen erklärt werden? \citep{BaechlerSeiler2016a}. Warum die linguistische Komplexitätsforschung seit wenigen Jahren und relativ plötzlich so viel Aufwind erfahren hat, liegt vermutlich in der klaren, expliziten Entkoppelung der früher angenommenen Verbindung zwischen besonders komplex und besonders wertvoll, d.h., dass eine vor allem flexionsmorphologisch besonders komplexe Sprache auch besonders wertvoll sei. Versteht man jedoch eine Sprache als ein System von Regeln, die vorgeben, wie das Lexikon aufgebaut wird und wie mit diesem Lexikon Sätze generiert werden, müsste man vom Gegenteil ausgehen.\largerpage[2] Denn ein solches System zur Informationsverarbeitung wäre wohl dann besonders elegant, wenn es mit sehr wenigen Regeln (also möglichst geringer Komplexität) auskommt, folglich sehr effizient ist.

\subsection{Hypothese der gleichen Komplexität (\textit{Equi-Complexity-Hypothese})}\label{2.1.2}

Im Strukturalismus des 20. Jh. wurde generell angenommen – insofern es überhaupt diesbezügliche Äußerungen gibt – dass sich Sprachen in ihrer Gesamtkomplexität nicht signifikant unterscheiden. Diese Hypothese wird \is{Equi-Complexity-Hypothese}\textit{Equi-Com\-ple\-xi\-ty-Hy\-po\-the\-se} genannt. Ihr prominentester Vertreter ist wohl \citet{Hockett1958}, der feststellt, dass „[…] impressionistically it would seem that the total grammatical complexity of any language, counting both morphology and syntax, is about the same as that of any other. This is not surprising, since all languages have about equally complex jobs to do […]“ \citep[180]{Hockett1958}. Wenn man also die Komplexität aller Subsysteme messen würde, käme unter dem Strich heraus, dass alle Sprachen ungefähr gleich komplex sind. \citet{Hockett1958} sagt jedoch nichts dazu, welche die relevanten Subsysteme sind und wie deren Komplexität gemessen werden kann. Sehr verwundert und pointiert äußert sich \citet{Sampson2009} über den angenommenen Ausgleichsmechanismus in Sprachen:

\begin{quote}
If it really were so that languages varied greatly in the complexity of subsystem \textit{X}, varied greatly in the complexity of subsystem \textit{Y}, and so on, yet for all languages the totals from the separate subsystems added together could be shown to come out the same, then I would not agree with Hockett in finding this unsurprising. To me it would feel almost like magic. \citep[2--3]{Sampson2009}
\end{quote}

\noindent
Obwohl im 20. Jh. die \is{Equi-Complexity-Hypothese}\textit{Equi-Com\-ple\-xi\-ty-Hy\-po\-the\-se} dominierte, sprachen besonders Variationslinguisten immer wieder von Komplexitätsunterschieden zwi-\linebreak schen Sprachen. Der einflussreichste Vertreter ist hier u.a. wohl \citet{Ferguson1959}, der \textit{High Varieties} von \textit{Low Varieties} differenziert: „One of the most striking differences between H[igh] and L[ow] in the defining languages is in the grammatical structure: H has grammatical categories not present in L and has an inflectional system of nouns and verbs which is much reduced or totally absent in L“ \citep[333]{Ferguson1959}. Es bleibt aber offen, wie diese Unterschiede quantifiziert werden können. In der Folge sollen nun einige Studien vorgestellt werden, die zeigen, dass es Ausgleichstendenzen zwischen den Subsystemen innerhalb einer Sprache gibt, wie auch solche Studien, die das Gegenteil belegen. Diese Liste versteht sich als kurzer Überblick mit Fokus auf den Resultaten und erhebt keinen Anspruch auf Vollständigkeit.

Sowohl \citet{Juola2008} als auch \citet{EhretSzmrecsanyi2016} verwenden ein informationstheoretisches Maß (Kolmogorov-Komplexität und Ziv-Lem\-pel-Kom\-ple\-xi\-tät), um Komplexität zu messen, womit die Menge an Informationen und Redundanzen ermittelt werden kann \citep[93]{Juola2008}. Konkret wird Komplexität durch das Verzerren bestimmter Subsysteme und durch Komprimierung gemessen. Die Datengrundlage bilden große Textkorpora in sechs Sprachen: bei \citet{Juola2008} die Bibel, bei \citet{EhretSzmrecsanyi2016} das Markusevangelium. \citet{Juola2008} kann zeigen, dass sich Sprachen in ihrer Komplexität nicht wesentlich unterscheiden \citep[106]{Juola2008}, während die Resultate von \citet{EhretSzmrecsanyi2016} mit einer vergleichbaren Methode das Gegenteil demonstrieren \citep[78]{EhretSzmrecsanyi2016}. Beide Studien können aber mit genau derselben Messmethode einen Ausgleich zwischen morphologischer und syntaktischer Komplexität nachweisen (\citealt[104]{Juola2008}; \citealt[79--80]{EhretSzmrecsanyi2016}).

\citet{Sinnemäki2008} untersucht in 50 Sprachen die Markierung von Agens und Patiens, wozu vier Strategien existieren, aber nur die letzten drei als strukturelle Strategien berücksichtigt werden: Lexikon, Wortstellung, \textit{Head Marking} und \textit{Dependent Marking} \citep[68]{Sinnemäki2008}. Gemessen wird, wie oft eine bestimmte Kodierungsstrategie verwendet wird, um Agens und Patiens zu unterscheiden \citep[72]{Sinnemäki2008}. \citet{Sinnemäki2008} stellt fest, dass in einigen Subdomänen Ausgleichstendenzen zu beobachten sind, vor allem verallgemeinernd zwischen Wortstellung und \textit{Dependent Marking} \citep[84--85]{Sinnemäki2008}. Die meisten potentiellen Korrelationen jedoch, die geprüft wurden, waren äußerst gering oder inexistent, weshalb Trade-Offs als allgemeines Prinzip verworfen werden können \citep[84]{Sinnemäki2008}.

\citet{Shosted2006} überprüft in 32 Sprachen einen möglichen Ausgleich zwischen morphologischer und phonologischer Komplexität. Dazu zählt er die Anzahl\linebreak möglicher Silben einer Sprache und die Anzahl der Marker in der Verbflexion (als eine Art Synthesegrad) \citep[9–17]{Shosted2006}. Die Korrelation zwischen den Messresultaten ist leicht positiv, aber statistisch nicht signifikant \citep[1]{Shosted2006}.

Auch \citet{Nichols2009} konnte keine negativen Korrelationen zwischen verschiedenen Komponenten der Grammatik feststellen \citep[119]{Nichols2009}. Sie untersucht 68 Sprachen und fünf Komponenten: Phonologie (u.a. Anzahl der Qualitätsunterschiede der Vokale), Synthese (u.a. Anzahl am Verb markierter Kategorien), Klassifikation (u.a. Genuskongruenz), Syntax (u.a. Anzahl unterschiedlicher Abfolgen von nominalen und pronominalen Argumenten sowie dem Verb) und Lexikon (u.a. Anzahl Suppletivpaare in neun Paaren Vollverb/kausatives Verb) \citep[113]{Nichols2009}.

Einen Ausgleich zwischen Subsystemen der Grammatik konnten nur die informationstheoretisch basierten Methoden nachweisen. Diese Methoden ermöglichen zwar im Prinzip, gesamte Subsysteme zu messen.\largerpage Sie sind jedoch probabilistisch und erlauben keinen Einblick in die Details der Ausgleiche.\largerpage Dies wäre jedoch wichtig, um die eventuellen Ausgleichsmechanismen zu verstehen und um ein Ausgleichsmuster abstrahieren zu können, denn so wäre dann ein Vergleich dieser Muster von verschiedenen Sprachen möglich. Des Weiteren könnten diese Ausgleichsmuster und ihr Vergleich für die linguistische Theoriebildung nutzbar gemacht werden. In diese Richtung gehen eher Studien wie jene von \citet{Sinnemäki2008} und \citet{Nichols2009}. Das Problem dieser Methoden jedoch ist, dass immer nur ein Ausschnitt einer Sprache oder eines Subsystems analysiert werden kann. Dafür ermöglichen sie aber einen detaillierteren Einblick in die inneren Vorgänge der Sprachen.

Dies tritt in Konflikt mit der \is{Equi-Complexity-Hypothese}\textit{Equi-Com\-ple\-xi\-ty-Hy\-po\-the\-se}, die implizit verlangt, dass die Gesamtkomplexität einer Sprache gemessen werden soll, d.h. alle Subsysteme. Erstens gibt es aber keinen Konsens über die Taxonomie der linguistischen Subsysteme. Zweitens stellt sich die Frage, wie die Komplexität der einzelnen Subsysteme gemessen werden kann, sodass eine Verrechnung der Resultate die Gesamtkomplexität darstellt. \citet{Miestamo2008} bezeichnet dies als das Problem der Repräsentativität und als das Problem der Vergleichbarkeit:

\begin{quote}
The problem of representativity means that no metric can pay attention to all aspects of grammar that are relevant for measuring global complexity. Even if this were theoretically possible, it would be beyond the capacities of the mortal linguist to exhaustively count all grammatical details of the languages studied […]. The problem of comparability is about the difficulty of comparing different aspects of grammar in a meaningful way, and especially about the impossibility of quantifying their contributions to global complexity. \citep[30]{Miestamo2008}
\end{quote}

\noindent\largerpage
Beispielsweise stellt sich die Frage, wie viele Unterscheidungen im Aspektsystem gleich komplex sind und wie viele Unterscheidungen im Tempussystem \citep[7]{Miestamo2006}. Daraus kann geschlossen werden, dass (zumindest vorerst) die Messung einzelner Subsysteme oder Teile von Subsystemen zu bevorzugen ist. \citet{Miestamo2008} nennt dies \textit{Local Complexity} im Gegensatz zur \textit{Global} oder \textit{Overall} \textit{Complexity}, womit die Gesamtkomplexität einer Sprache gemeint ist \citep[29]{Miestamo2008}. Viel grundsätzlicher geht es darum, dass nur Vergleichbares verglichen werden kann. \citet{Miestamo2008} plädiert dafür, formale Aspekte der Grammatik (z.\,B.\ morphologische Systeme) von den funktionalen Aspekten (z.\,B.\ Kodierung von Tempus, Aspekt) zu unterscheiden und nur innerhalb dieser Bereiche einen Komplexitätsvergleich zwischen Sprachen vorzunehmen \citep[31]{Miestamo2008}. Ein gutes Beispiel dafür ist die oben zitierte Arbeit von \citet{Sinnemäki2008}. Dabei wird die Markierung von\largerpage Argumenten gemessen, wofür die Sprache unterschiedliche Mittel (z.\,B.\ morphologische und syntaktische) zur Verfügung hat.

\subsection{Hat Komplexität eine Funktion in der Sprache?}\label{2.1.3}

Die \is{Equi-Complexity-Hypothese}\textit{Equi-Com\-ple\-xi\-ty-Hy\-po\-the\-se}, die vorwiegend auf den Strukturalismus zurückgeht, nimmt an, dass alle Sprachen gleich komplex sind und höhere Komplexität in einem Subsystem durch niedrigere Komplexität in einem anderen Subsystem kompensiert wird. Dies impliziert, dass Komplexität eine Funktion innerhalb der Sprache erfüllt. In \sectref{2.1.1} wurde gezeigt, dass besonders in der ersten Hälfte des 19. Jhs. davon ausgegangen wurde, dass die strukturelle Komplexität einer Sprache mit der (u.a. kulturellen) Entwicklung ihrer Sprecher korreliert. Verallgemeinernd kann man also Folgendes festhalten. Sowohl im 19. Jh. als auch im 20. Jh. wurde angenommen, dass strukturelle Komplexität eine Funktion in der Sprache innehat. Im 19. Jh. wurde diese Annahme verwendet, um die angebliche Fortschrittlichkeit besonders Europas zu erklären. Im 20. Jh. ging man davon aus, dass die kommunikativen Anforderungen an eine Sprache immer dieselben sind, folglich alle Sprachen den gleichen Grad an Komplexität aufweisen müssen. Ein Beispiel dafür ist die bereits oben zitierte Aussage von \citet{Hockett1958}, der behauptet, dass Ausgleichstendenzen in der Komplexität von verschiedenen Subsystemen zu erwarten sind, „[…] since all languages have about equally complex jobs to do […]“ \citep[180]{Hockett1958}. Dem entgegnet \citet{Sampson2009} mit einer ganz grundsätzlichen Überlegung: „[…] but it seems to me very difficult to define the job which grammar does in a way that is specific enough to imply any particular prediction about grammatical complexity“ \citep[2]{Sampson2009}.

Es ist also sicher nicht möglich, jene Aufgaben umfassend zu beschreiben, die eine Sprache erfüllen muss. Noch grundlegender ist jedoch die Frage, ob eine komplexe Aufgabe zwangsläufig mit einer komplexen Grammatik gelöst werden muss: Weshalb sollten komplexe Konzepte auch nur durch eine komplexe Grammatik ausgedrückt werden können und umgekehrt, einfache Konzepte durch eine einfache Grammatik? Können wir doch mit derselben Sprache sowohl einfache wie auch schwierige Dinge auf eine mehr oder weniger komplexe Art formulieren. Folglich gibt es zwischen der Komplexität der Grammatik und der Komplexität der Dinge, die wir tun, keinen Zusammenhang. Dass Grammatik oft eben gerade nicht funktional ist, bringt \citet{Gil2009} wie folgt auf den Punkt:

\begin{quote}
These facts cast doubt on a central tenet of most functionalist approaches to language, in accordance with which grammatical complexity is there to enable us to communicate the messages we need to get across. In spite of overwhelming evidence showing that diachronic change can be functionally motivated, the fact remains that language is hugely dysfunctional. Just think of all the things that it would be wonderful to be able to say but for which no language comes remotely near to providing the necessary expressive tools. For example, it would be very useful to be able to describe the face of a strange person in such a way that the hearer would be able to pick out that person in a crowd or a police line-up. But language is completely helpless for this task, as evidenced by the various stratagems police have developed, involving skilled artists or, more recently, graphic computer programs, to elicit identifying facial information from witnesses – in this case a picture actually being worth much more than the proverbial thousand words. Yet paradoxically, alongside all the things we’d like to say but can’t, language also continually forces us to say things that we don’t want to say; this happens whenever an obligatorily marked grammatical category leads us to specify something we would rather leave unspecified. English, famously, forces third person singular human pronouns to be either masculine or feminine; but in many contexts we either don’t know the person’s gender or actually wish to leave it unspecified […]. \citep[32]{Gil2009}
\end{quote}

\noindent
Strukturelle Komplexität hat folglich nichts mit der Effizienz und Expressivität einer Sprache als Mittel zur Kommunikation zu tun \citep[2]{Miestamo2006}. Auch ist sie kein Symptom für Zivilisation und Fortschrittlichkeit. Vielmehr kann Grammatik als ein System gesehen werden, dessen \isi{Variation} in der Komplexität primär systeminterne Ursachen hat:

\begin{quote}
Rather than having evolved in order to enable us to survive, sail boats, and do all the other things that modern humans do, most contemporary grammatical complexity is more appropriately viewed as the outcome of natural processes of self-organization whose motivation is largely or entirely system-internal. In this respect, grammatical complexity may be no different from complexity in other domains, such as anthropological complexity, economics, biology, chemistry, and cosmology, which have been suggested to be governed by general laws of nature pertaining to the evolution of complex systems. \citep[32--33]{Gil2009}
\end{quote}

\noindent
Wenn Grammatik also ein sich selbst organisierendes System ist, dann besteht ebenfalls die Möglichkeit, dass auch die Komplexität der Grammatik verschiedener Sprachen unterschiedlich hoch ist. Die grammatischen Systeme von Sprachen mögen zwar untereinander mehr Ähnlichkeiten aufweisen als im Vergleich mit z.\,B.\ chemischen Systemen. Trotzdem gibt es keinen Grund, weshalb die grammatischen Systeme aller Sprachen gleich komplex sein sollten.

\section{Strukturelle Komplexität: Definitionen, Messmethoden und Einflussfaktoren}\label{2.2}%%%\rohead{\thesection\hspace{0.5em}Definitionen, Messmethoden und Einflussfaktoren}
\sectionmark{Definitionen, Messmethoden und Einflussfaktoren}

Dieser Abschnitt gibt einen Überblick über die Arbeiten, die Unterschiede in der strukturellen Komplexität u.a. durch sprachexterne Faktoren zu erklären versuchen. Alle bereits publizierten Untersuchungen zu diskutieren, würde den hiesigen Rahmen sprengen. Vielmehr soll ein Ausschnitt jener Studien vorgestellt werden, die sich mit den sprachexternen Faktoren beschäftigen, welche für die vorliegende Arbeit zentral sind: kleine, \isi{isolierte Sprachgemeinschaften} mit wenig \isi{Sprachkontakt} und einem engen Netzwerk vs. große, nicht \isi{isolierte Sprachgemeinschaften} mit viel \isi{Sprachkontakt} (vielen L2-Ler\-nern) und losen Netzwerken. Unterschiedliche Arbeiten dazu werden in den  \sectref{2.2.3} und \sectref{2.2.4} vorgestellt. Zuerst sollen jedoch die ersten publizierten Überlegungen zu möglichen Einflussfaktoren kurz eingeführt werden (\sectref{2.2.1}). Anschließend werden Arbeiten erörtert, in denen sprachexterne Faktoren herangezogen werden, die in der vorliegenden Arbeit nicht im Fokus stehen, die aber die Diskussion über den Zusammenhang zwischen struktureller Komplexität und sprachexternen Faktoren beeinflusst haben (\sectref{2.2.2}). Es handelt sich dabei um die Faktoren Alter, Geschlecht, Schicht, Region und \isi{Bevölkerungsgröße}. In diesem Überblick stehen folgende Fragen im Vordergrund: Was wird unter struktureller Komplexität verstanden und wie kann diese gemessen werden? Welche Sprachen/Varietäten und welche linguistischen Beschreibungsebenen werden untersucht? Wie werden die Unterschiede in der Komplexität von Sprachen erklärt? In einem abschließenden Abschnitt wird zuerst die mittlerweile etablierte Unterscheidung zwischen absoluter und relativer Komplexität vorgestellt. Anschließend wird eine detailliertere Unterscheidung von Komplexitätstypen eingeführt, die auf den Philosophen \citet{Rescher1998} zurückgeht und von \citet{MiestamoSinnemäkiKarlsson2008} für linguistische Phänomene adaptiert wurde. Zuletzt werden die in den vorangehenden Abschnitten \sectref{2.2.1}–\sectref{2.2.4} verwendeten Definitionen struktureller Komplexität zusammengefasst und kategorisiert (\sectref{2.2.5}).

\subsection{Erste Überlegungen zu möglichen Einflussfaktoren}\label{2.2.1}

Überlegungen zu möglichen sprachexternen Faktoren, die mit der \isi{Variation} in der strukturellen Komplexität zusammenhängen könnten, sind schon relativ alt. Einen Überblick über die ersten Hypothesen in diesem Zusammenhang bieten \citet{BaechlerSeiler2016a}, woran ich mich in der Folge orientiere.

Der vielleicht erste Linguist, der soziale und geografische Faktoren mit der Höhe struktureller Komplexität in Verbindung brachte, ist Roman \citet{Jakobson1929}. Er untersuchte das phonologische System ukrainischer Dialekte und stellte Folgendes fest: Zentrale, sich ausbreitende Dialekte, die von einer homogenen Sprachgemeinschaft gesprochen werden, haben ein kleineres Vokalinventar als die Dialekte an der Peripherie des Sprachgebiets \citep[73]{Jakobson1929}.

\begin{quote}
Cette différence est due, en premier lieu à la tendance conservatrice qui est caractéristique des parlers de la périphérie, et en second lieu à des différences fonctionnelles. Il n’est pas rare d’observer que la tendance à simplifier le système phonologique croît à mesure que grandit le rayon d’emploi d’un dialecte, avec la plus grande hétérogénéité des sujets parlant la langue généralisée. On n’a pas encore, en linguistique, prêté assez attention à la différence essentielle de structure et d’évolution qui existe entre les parlers gravitant vers le rôle de ϰoɩν\'η ou langue commune, et ceux d’usage purement local. \citep[73]{Jakobson1929}
\end{quote}

\begin{quote}
[Dieser Unterschied ist an erster Stelle durch die konservierende Tendenz bedingt, die charakteristisch für die Sprachen an der Peripherie ist, und an zweiter Stelle durch funktionale Unterschiede. Es ist nicht selten zu beobachten, dass die Tendenz, das phonologische System zu vereinfachen, zunimmt, je mehr das Areal, in dem der Dialekt gesprochen wird, größer wird, und mit einer heterogenen Sprachgemeinschaft, die die Lingua franca spricht. In der Linguistik hat man dem wesentlichen Unterschied in der Struktur und der Evolution noch nicht genügend Aufmerksamkeit\linebreak geschenkt, der zwischen den Sprachen existiert, die in Richtung Koiné oder Gemeinsprache tendieren, und jenen, die ausschließlich lokal verwendet werden.] [meine Übersetzung]
\end{quote}

\citet{Jakobson1929} beschreibt hier also geografische und soziale Faktoren, die auf die Funktion einer Sprache Einfluss haben, was wiederum auf den Sprachwandel wirkt. \citeauthor{Hymnes1975}’ \citeyearpar{Hymnes1975} Beobachtungen weisen in eine ähnliche Richtung wie jene von \citet{Jakobson1929}, wobei bei \citet{Hymnes1975} soziale Faktoren der Sprachgemeinschaft im Vordergrund stehen. Er stellt fest, dass kleine Sprachen mit einem engen Netzwerk eine höhere strukturelle Komplexität aufweisen: „This latter process may have something to do with the fact that the surface structures of languages spoken in small, cheek-by-jowl communities so often are markedly complex, and the surface structures of languages spoken over wide ranges less so“ \citep[50]{Hymnes1975}.

Auch \citet{Werner1975} beobachtet Ähnliches wie \citet{Jakobson1929}. Er vergleicht kleine und \isi{isolierte Sprachgemeinschaften}, die Jakobsons peripheren\linebreak Sprachen entsprechen, mit Sprachen mit viel Kontakt, die Jakobsons sich ausbreitenden Sprachen gleichkommen. Bezüglich der möglichen Korrelation zwischen diesen Sprachgemeinschaftstypen und der strukturellen Komplexität einer Sprache stellt \citet{Werner1975} dasselbe wie \citet{Jakobson1929} fest: 

\begin{quote}
Es ist – vermute ich – ganz allgemein ein Kennzeichen kleinerer, isolierter Sprachgemeinschaften, dass sie lange komplizierte Regelsysteme bewahren; großräumige Sprachkontakte und damit verbundene Interferenzen fordern dagegen die Analogien, wie sie ja auch von Kindern und Ausländern gerne gemacht werden. \citep[791]{Werner1975}
\end{quote}

Eine der ersten Arbeiten, in der explizit ein möglicher Zusammenhang zwischen sprachexternen Faktoren und struktureller Komplexität systematischer geprüft wird, ist \citeauthor{Braunmüller1984}s \citeyear{Braunmüller1984} erschienener Aufsatz. Er untersucht die Flexionsmorphologie des Isländischen, Färöischen und Friesischen. Dabei handelt es sich um Sprachen, die nicht nur wenig \isi{Sprachkontakt} aufweisen (mit Nicht-Mut\-ter\-sprach\-lern wird in einer anderen Sprache kommuniziert), sondern auch \isi{geografisch isoliert} sind: Alle drei Sprachen werden auf Inseln gesprochen, Friesisch zusätzlich an der Küste Schleswig-Holsteins. \citet{Braunmüller1984} zeigt, dass viele unterschiedliche phonologische Regeln gleichzeitig wirken und so Opazität im Paradigma verursachen. Die so entstandene paradigmatische Opazität wird nur sehr wenig durch Analogien ausgeglichen, was besonders bei kleinen und isolierten Sprachen vorkommt \citep[49]{Braunmüller1984}. Dies wird vor allem durch zwei Charakteristika dieser Sprachgemeinschaften begünstigt. Erstens findet der Kontakt mit Nicht-Mut\-ter\-sprach\-lern in einer anderen Sprache statt, sodass diese Sprachen nur sehr selten als L2 erlernt werden und sie nicht als Koiné dienen \citep[49]{Braunmüller1984}. Dies hat zur Konsequenz, dass Sprachen von kleinen und \isi{isolierten Sprachgemeinschaften} nur wenig durch andere Sprachen beeinflusst werden und kaum L2-Sim\-pli\-fi\-zie\-rungen aufweisen. Zweitens sind kleine und \isi{isolierte Sprachgemeinschaften} sehr homogen, weswegen die Sprache wenig strenger Normierung ausgesetzt, die in den Sprachwandel eingreifen könnte \citep[49]{Braunmüller1984}. \citet{Braunmüller1984} bringt also frühere Beobachtungen zu möglichen Einflussfaktoren zusammen: kleine Sprachgemeinschaften mit einem engen Netzwerk und hoher Homogenität, wenige L2-Ler\-ner, kleinräumiger Gebrauch der Sprache, geografische \isi{Isolation} und Peripheralität. 

\subsection{Sprachexterne Faktoren}\label{2.2.2}

In diesem Abschnitt sollen drei Studien vorgestellt werden, die den Zusammenhang zwischen sprachexternen Faktoren und struktureller Komplexität untersuchen. Bei \citet{Sampson2001} handelt es sich bei den sprachexternen Faktoren um Alter, Geschlecht, Schicht und Region, bei \citet{HayBauer2007} sowie bei \citet{Sinnemäki2009} um Bevölkerungsgrößen.

\citet{Sampson2009} prüft den hypothetischen Zusammenhang zwischen syntaktischer Komplexität und Alter, Geschlecht, Schicht sowie Region, wobei es sich also um demografische Daten handelt. Die Datengrundlage bildet sein eigens erstelltes Korpus ‘Christine’, das spontansprachliche Äußerungen enthält und auf dem British National Corpus basiert \citep[57--58]{Sampson2009}. Syntaktische Komplexität wird als die Einbettungstiefe definiert: „For a sentence to be ‘simple’ or ‘complex’ in traditional grammatical parlance refers to whether or not it contains subordinate clause(s)“ \citep[58]{Sampson2009}. Da in der Spontansprache der Anfang und das Ende von Sätzen nicht immer eindeutig bestimmt werden können, wird die Einbettungstiefe auf der Ebene des Wortes definiert: „The present research treats degree of embedding as a property of individual words. Each word is given a score representing the number of nodes in the CHRISTINE ‘lineage’ of that word […] which are labeled with clause categories“ \citep[59]{Sampson2001}. In einem ersten Durchgang fließen die Äußerungen aller Informanten in die Analyse ein, in einem zweiten Durchgang nur die Äußerungen jener Informanten, die älter als 16 Jahre sind. Zwischen den fünf untersuchten Regionen (Süd- und Nordengland, Wales, Schottland, Nordirland) gibt es keinen signifikanten Unterschied in der Komplexität, unabhängig davon, ob die unter 16-Jährigen mit eingeschlossen sind oder nicht \citep[62, 64, 66]{Sampson2009}. Dasselbe gilt bezüglich der sozialen Schicht \citep[64]{Sampson2009}; die Resultate werden aber annähernd signifikant, wenn die unter 16-Jährigen ausgeschlossen werden (von mehr zu weniger komplex): ausgebildet-handwerklich, leitende und technische Berufe, ausgebildet-nicht handwerklich, teils ausgebildet/ohne Ausbildung \citep[67]{Sampson2009}. Dieses doch eher erstaunliche Ergebnis die soziale Schicht betreffend wird dadurch erklärt, dass die Angaben zur Schicht in diesen Daten am wenigsten verlässlich sind \citep[67]{Sampson2009}. Frauen produzieren etwas häufiger komplexe Sätze (signifikant) als Männer \citep[65]{Sampson2009}. Werden jedoch nur die über 16-Jährigen berücksichtigt, ist das Resultat nicht mehr signifikant \citep[66]{Sampson2009}. Das signifikante Ergebnis wird als Problem der Datengrundlage gedeutet, denn in diesem Sample sind mehr Männer als Frauen \citep[66--67]{Sampson2009}. Die verschiedenen Altersgruppen unterscheiden sich signifikant in der syntaktischen Komplexität, wobei mehr komplexe Äußerungen produziert werden, je älter die Informanten sind \citep[65]{Sampson2009}. Von der Annahme ausgehend, dass sich die Informanten bis 13 Jahre noch im Spracherwerbsprozess befinden, werden in einem weiteren Test die unter 13-Jährigen ausgeschlossen (\citealt{Sampson2009}: 67–70). Auch in diesem Test zeigt sich, dass die syntaktische Komplexität mit dem Alter zunimmt \citep[70]{Sampson2009}. \citet{Sampson2009} interpretiert, „that (while the evidence is not overwhelming) increase in average grammatical complexity of speech appears to be a phenomenon that does \textit{not} terminate at puberty, but continues throughout life“ \citep[70]{Sampson2009}.

\citet{Sinnemäki2009} misst die Komplexität in der Markierung von Agens und Patiens in 50 Sprachen und prüft den Zusammenhang zwischen Komplexität und \isi{Bevölkerungsgröße}. Zur Markierung von Agens und Patiens stehen drei Strategien zur Verfügung: \textit{Head Marking} (am Verb), \textit{Dependent Marking} (an Agens oder Patiens) und die Wortstellung \citep[130]{Sinnemäki2009}. Komplexität wird als die Verletzung des Eine-Bedeutung-Eine-Form-Prinzips definiert. Von diesem Prinzip gibt es zwei Abweichungen, welche die Komplexität erhöhen: Verletzung der Ökonomie und Verletzung der Distinktheit \citep[132]{Sinnemäki2009}. Wird das Prinzip der Ökonomie verletzt, wird zu viel markiert, d.h., die Sprache verwendet mehr als eine Strategie, um Agens und Patiens zu markieren, welche nicht komplementär verteilt sind \citep[133]{Sinnemäki2009}. Wird das Prinzip der Distinktheit verletzt, wird zu wenig markiert, d.h., die Sprachen „use only one strategy but in limited contexts, or allow a lot of syncretism in the head or dependent marking paradigms“ \citep[133]{Sinnemäki2009}. \citet{Sinnemäki2009} listet und zählt die Sprachen, die dem Eine-Bedeutung-Eine-Form-Prinzip entsprechen, und jene Sprachen, die davon abweichen. Die Resultate können wie folgt zusammengefasst werden. Erstens nimmt die Anzahl der Sprachen, die dem Eine-Bedeutung-Eine-Form-Prinzip entsprechen, parallel zur Größe der Sprachgemeinschaft zu \citep[135]{Sinnemäki2009}. Zweitens tendieren Sprachen, die von weniger als 10.000 Sprechern gesprochen werden, dazu, entweder Ökonomie oder Distinktheit zu verletzen. Dabei nimmt die Anzahl der Sprachen, die Ökonomie oder Distinktheit verletzen, ab, wenn die Anzahl der Sprecher zunimmt \citep[135]{Sinnemäki2009}. Da es problematisch ist zu bestimmen, was eine große oder kleine Sprachgemeinschaft ist, wird mit demselben Sample ein weiterer Test gemacht. Dazu werden Grenzwerte bestimmt: Angefangen wird bei 250 Sprechern, der nächste Grenzwert ist das Doppelte des vorangehenden Grenzwertes usw. \citep[135]{Sinnemäki2009}. Es kann festgestellt werden, dass Sprachen, die von einer Sprachgemeinschaft gesprochen werden, die kleiner oder gleich groß wie ein Grenzwert ist, eher Distinktheit oder Ökonomie verletzen. Des Weiteren tendieren Sprachen, die von einer Sprachgemeinschaft gesprochen werden, die größer als ein Grenzwert ist, dem Eine-Bedeutung-Eine-Form-Prinzip zu entsprechen \citep[136--138]{Sinnemäki2009}. Zusammengefasst kann also festgehalten werden, dass Sprachen einer kleineren Sprachgemeinschaft in der Markierung von Agens und Patiens höhere Komplexität aufweisen als Sprachen einer größeren Sprachgemeinschaft.

\citet{HayBauer2007} prüfen einen Zusammenhang zwischen der Größe des Phoneminventars einer Sprache und der Größe der Sprachgemeinschaft. Das Sample beträgt 216 Sprachen \citep[388]{HayBauer2007}. Die Größe des Phoneminventars wird durch die Anzahl folgender Phonemkategorien gemessen: Basismonophthonge (unterscheiden sich in ihrer Qualität), Extramonophthonge (Unterscheidung der Länge und Nasalierung), Diphthonge, Obstruenten und Sonoranten \citep[389]{HayBauer2007}. Getestet wird ein Zusammenhang zwischen diesen einzelnen Kategorien und der \isi{Bevölkerungsgröße} wie auch zwischen der Gesamtgröße des Phoneminventars und der \isi{Bevölkerungsgröße}. Es wird eine positive Korrelation sowohl zwischen jeder Phonemkategorie und der \isi{Bevölkerungsgröße} als auch zwischen der Gesamtgröße des Phoneminventars und der \isi{Bevölkerungsgröße} festgestellt \citep[389--390]{HayBauer2007}. Wichtig ist hier auch die Beobachtung, dass das Vokalinventar und das Konsonanteninventar jedoch nicht miteinander korrelieren \citep[391]{HayBauer2007}. Des Weiteren handelt es sich hier um eine statistische Tendenz. Es gibt also auch Sprachen, die dieser Tendenz zuwiderlaufen, wie z.\,B.\ das Färöische, das von relativ wenigen Sprechern gesprochen wird, aber ein großes Phoneminventar aufweist \citep[390]{HayBauer2007}. In einem weiteren Test wird untersucht, ob diese Tendenz verursacht wird von einem Zusammenhang zwischen Sprachfamilien und der \isi{Bevölkerungsgröße} \citep[391]{HayBauer2007}. Tatsächlich hat die Sprachfamilie einen Einfluss, z.\,B.\ haben die indgermanischen Sprachen die größten Phoneminventare, während die austronesischen Sprachen die kleinsten Phoneminventare aufweisen \citep[392]{HayBauer2007}. Die \isi{Bevölkerungsgröße} ist jedoch neben der Sprachfamilie ein zusätzlicher, signifikanter Prädiktor für die Größe des Phoneminventars \citep[392]{HayBauer2007}. Weshalb Sprachen mit kleinen Sprachgemeinschaften auch kleine Phoneminventare aufweisen, kann nicht definitiv erklärt werden. \citet{HayBauer2007} diskutieren verschiedene Erklärungsversuche, auf die hier nicht weiter eingegangen werden kann. Eine Hypothese soll hier aber kurz vorgestellt werden. \citet{Trudgill2004a} geht davon aus, dass kleine Sprachgemeinschaften entweder große oder kleine Phoneminventare aufweisen, während große Sprachgemeinschaften eher mittlere Phoneminventare favorisieren (\citealt{Trudgill2004a}: 317, zitiert aus \citealt[396]{HayBauer2007}). Große Phoneminventare werden dadurch erklärt, dass kleine Sprachgemeinschaften die Fähigkeit haben „to encourage continued adherence to norms from one generation to another, however complex they may be“ \citep[317]{Trudgill2004a}, zitiert aus \citealt[396]{HayBauer2007}). Kleine Phoneminventare dagegen werden auf das gemeinsame Wissen kleiner Sprachgemeinschaften zurückgeführt: „initial small community size […] would have led in turn to tight social networks, which would have implied large amounts of shared background information – a situation in which communication with relatively low level of phonological redundancy would have been relatively tolerable“ (\citealt[720]{Trudgill2002}: 720, zitiert aus \citealt[396]{HayBauer2007}). Den Zusammenhang zwischen phonologischer Komplexität, d.h. kleinem Inventar, und kleiner Sprachgemeinschaft illustriert \citet{Trudgill2011} anhand des Hawaiischen (basierend auf \citealt{Maddieson1984}), das fünf Vokale, acht Konsonanten, eine CVCV-Sil\-ben\-struk\-tur und nur 162 mögliche Silben hat:

\begin{quote}
My suggestion is that possessing only a small number of available syllables – and therefore a relatively small amount of redundancy – may, other things being equal, lead to greater communicative and/or cognitive difficulty because of a lack of contrastive possibilities. I suggest that while this lack of contrastive possibilities is entirely unproblematical for native speakers, languages such as Hawai’ian will cause difficulties for non-natives. Languages with very small phoneme inventories cause problems of \textit{memory load} for foreign learners – they are L2 difficult. […] The problem lies in the relative lack of distinctiveness between one vocabulary item and another, due to the necessarily high proportion of usage of possible syllables […]”. \citep[124]{Trudgill2011}
\end{quote}

\subsection{Soziale Isolation vs. Sprachkontakt}\label{2.2.3}

In diesem Abschnitt werden einige Studien vorgestellt, die strukturelle Komplexität in Sprachgemeinschaften mit viel und wenig \isi{Sprachkontakt} untersuchen. Dabei werden Sprachgemeinschaften mit wenig \isi{Sprachkontakt} als \isi{isolierte Sprachgemeinschaften} bezeichnet. Es handelt sich hier nur um einen exemplarischen Ausschnitt, in dem die wichtigsten Richtungen aufgezeigt werden sollen.

Peter Trudgill hat eine Vielzahl an Arbeiten veröffentlicht, die sich mit dem Zusammenhang zwischen struktureller Komplexität und dem Typ der Sprachgemeinschaft beschäftigen. In \citet{BaechlerSeiler2016a} wurden die wichtigsten Resultate zusammengefasst, was hier in leicht abgeänderter Form übernommen wird. Trudgill unterscheidet vor allem zwei Typen von Sprachgemeinschaften. Der eine Typ wird wie bei \citet{Braunmüller1984} durch geringe \isi{Bevölkerungsgröße}, geografische und vor allem soziale \isi{Isolation} charakterisiert \citep{Trudgill1992}. Mit sozialer \isi{Isolation} sind wenig \isi{Sprachkontakt} und wenige L2-Ler\-ner gemeint. Zusätzlich berücksichtigt \citet{Trudgill2011} aber auch interne Charakteristika einer Sprachgemeinschaft: enge soziale Netzwerke, hohe soziale Stabilität und eine große Menge gemeinsamen Wissens \citep[146]{Trudgill2011}. Folglich gibt es auf der einen Seite kleine, \isi{geografisch isolierte}, stabile Sprachgemeinschaften mit einem engen Netzwerk und wenig \isi{Sprachkontakt}, welche hier \textit{isolierte Sprachgemeinschaften} genannt werden, und auf der anderen Seite große, geografisch nicht isolierte, sich verändernde Sprachgemeinschaften mit losen Netzwerken und viel \isi{Sprachkontakt}, welche hier als \textit{nicht isoliert} gekennzeichnet werden. Diese zwei Typen von Sprachgemeinschaften bilden natürlich die beiden Pole, die Ausprägung der Charakteristika (z.\,B.\ hohe vs. geringe Bevölkerungszahl) wie auch die Charakteristika selbst können in unterschiedlicher Kombination auftreten \citep[147]{Trudgill2011}. Einfachheitshalber wird hier jedoch weiterhin von den beiden Extremen, d.h. von isolierten und nicht \isi{isolierten Sprachgemeinschaften} gesprochen. Zusammengefasst geht Trudgill davon aus, dass \isi{isolierte Sprachgemeinschaften} dazu tendieren, höhere strukturelle Komplexität aufzuweisen und nicht \isi{isolierte Sprachgemeinschaften} dazu, geringere strukturelle Komplexität zu zeigen. Die Gründe dafür sieht Trudgill im Sprachwandel, dessen Mechanismen und Ergebnisse abhängig vom Typ Sprachgemeinschaft sind, was hier kurz ausgeführt werden soll. Trudgill nimmt an, dass \isi{isolierte Sprachgemeinschaften} höhere strukturelle Komplexität haben, weil in diesen Sprachgemeinschaften die Wahrscheinlichkeit groß ist, „to find not only the preservation of complexity but also an increase in complexity, i.e. irregularity, opacity, syntagmatic redundancy, and non-borrowed morphological categories“ \citep[64]{Trudgill2011}. Sowohl der Erhalt von Komplexität als auch die Zunahme an Komplexität können durch Charakteristika des Sprachwandels erklärt werden, der durch die Struktur der Sprachgemeinschaft beeinflusst ist. Der Erhalt von Komplexität liegt besonders an drei Faktoren. Erstens ist es einfacher für kleine Sprachgemeinschaften mit wenig \isi{Sprachkontakt} und engen sozialen Netzwerken „to enforce and reinforce the learning and use of irregularities“ \citep[204]{Trudgill1992} und deshalb „to enforce and reinforce the learning and use of complexities by children and adolescents“ \citep[13]{Trudgill1996}. Zweitens unterscheidet sich das Tempo des Sprachwandels in Abhängigkeit vom Typ der Sprachgemeinschaft: „In small, isolated, stable communities, linguistic change will be slower“ \citep[103]{Trudgill2011}. Drittens sind kleine und \isi{isolierte Sprachgemeinschaften} weniger vom Sprachwandel betroffen bzw. die „[g]eographically peripheral varieties which have been least subject to dialect contact most strongly resist [...] language change leading to simplification“ \citep[6]{Trudgill1996}. Wenn jedoch Sprachwandel stattfindet, so sind kleine, \isi{isolierte Sprachgemeinschaften} mit engen Netzwerken in höherem Maße fähig einerseits, „to push through, enforce and sustain changes of a less natural or usual phonological type“ \citep[11]{Trudgill1996}, und andererseits „[to] promote the spontaneous growth of morphological categories“ \citep[109]{Trudgill2009}. Allgemein kann folglich davon ausgegangen werden, dass in diesen Sprachgemeinschaften der Sprachwandel eher Komplexifizierung verursacht \citep[103]{Trudgill2011}. \citet{Trudgill2011} nennt diesen Typ der Komplexifizierung „spontaneous, non-additive complexification“ \citep[71]{Trudgill2011}. Es kann also festgehalten werden, dass kleine, \isi{geografisch isolierte}, stabile Sprachgemeinschaften mit einem engen Netzwerk und wenig \isi{Sprachkontakt} dazu tendieren, Komplexität zu erhalten wie auch zu erhöhen. Im Gegensatz dazu tendieren große, geografisch nicht isolierte, sich verändernde Sprachgemeinschaften mit losen Netzwerken und viel \isi{Sprachkontakt} dazu, ihre Grammatik zu vereinfachen, „because high irregularity, low transparency, and high levels of redundancy make for difficulties of learning and remembering for adolescent and adult learner-speakers“ \citep[101]{Trudgill2009}. Es kann aber noch eine weitere Art des Kontakts beobachtet werden, nämlich „long-term co-territorial contact situations involving child bilingualism“ \citep[34]{Trudgill2011}. In dieser Kontaktsituation kommt Komplexifizierung in der Form von \textit{Additive Borrowings} vor \citep[27]{Trudgill2011}. Dabei werden neue Elemente oder Kategorien von der einen in die andere Sprache übernommen, ohne dass in der übernehmenden Sprache bereits existierende Elemente oder Kategorien ersetzt werden \citep[27]{Trudgill2011}.

\citet{Kusters2003} untersucht in seiner Dissertation den möglichen Zusammenhang zwischen dem Wandel der Komplexität in der Verbflexion und dem Wandel einer Sprachgemeinschaft von Typ 1 zu Typ 2. Typ 1 hat vor allem L1-Ler\-ner, ist sprecher-orientiert, d.h., der Hörer kann vermuten, was gesagt wird, und Sprecher sowie Hörer verfügen über ein großes gemeinsames Wissen. Des Weiteren hat die Sprache in Gemeinschaften des Typs 1 eine symbolische Funktion: Ausdruck von Identität, ästhetische Funktion etc. \citep[41]{Kusters2003}. Typ-2-Sprachgemeinschaften sind genau das Gegenteil: viele L2-Ler\-ner, kommunikative Funktion (primäres Ziel ist der Austausch von Informationen) und hörer-orientiert, d.h. die Bedürfnisse des Hörers (z.\,B.\ klare, explizite Artikulation) stehen im Vordergrund \citep[41]{Kusters2003}. Die Komplexität der Verbflexion wird definiert als die Phänomene, die für Außenstehende schwierig sind \citep[403]{Kusters2003}. Genauer werden drei Prinzipien herangezogen: Ökonomie, Transparenz und Isomorphie. Werden diese Prinzipien verletzt, gilt die Verbflexion als komplexer: Das Prinzip der Ökonomie ist verletzt, wenn eine Sprache viele Flexionskategorien aufweist; das Prinzip der Transparenz ist verletzt durch Fusion, Fission, Allomorphie und Homonymie; das Prinzip der Isomorphie ist verletzt bei nicht gleichbleibender Affixabfolge in verschiedenen Domänen \citep[403]{Kusters2003}. Untersucht werden Sprachen aus vier typologisch unterschiedlichen Sprachfamilien: Arabische Varietäten, skandinavische Sprachen, Quechua und Swahili \citep[403]{Kusters2003}. Für jede dieser Sprachfamilien gibt es Varietäten, deren Sprachgemeinschaft eher dem Typ 1 entspricht und solche, deren Sprachgemeinschaft eher dem Typ 2 entspricht. So kann geprüft werden, ob die Komplexität in der Verbflexion abnimmt, wenn die Sprachgemeinschaft sich von Typ 1 zu Typ 2 wandelt \citep[45]{Kusters2003}. Kurz zusammengefasst kann \citet{Kusters2003} zeigen, dass eine Sprache ökonomischer wird, wenn ihre Sprachgemeinschaft von einem Typ 1 zu einem Typ 2 wird: „[T]he number of categories and category combinations indeed decreases correspondingly in these communities“ \citep[357]{Kusters2003}. Dies hat zur Folge, dass diese Sprachen auch transparenter werden \citep[357]{Kusters2003}. Im Gegensatz dazu scheint Isomorphie keine große Rolle zu spielen \citep[357]{Kusters2003}. Es kann also ein Zusammenhang zwischen dem Typ einer Sprachgemeinschaft und den Phänomenen, die in der Verbflexion vorkommen, also der Komplexität der Verbflexion, beobachtet werden.

\citet{McWhorter2001} vergleicht in seinem ausführlichen und viel beachteten Aufsatz die strukturelle Komplexität einer Kreolsprache (Saramaccaans) mit zwei alten Sprachen, wobei die eine (Tzes) eher einen synthetischen, die andere (Lahu) eher einen analytischen Sprachbau aufweist. Komplexität wird durch Überspezifikation definiert, d.h. die overte Markierung phonetischer, morphologischer, syntaktischer und semantischer Unterscheidungen, die über die kommunikativen Notwendigkeiten hinausgehen \citep[125]{McWhorter2001}. Bestimmt wird der Grad an Überspezifikation mit der Beschreibungslänge eines Systems: „[S]ome grammars might be seen to require lengthier descriptions in order to characterize even the basics of their grammar than others“ \citep[134--135]{McWhorter2001}. \citet{McWhorter2001} geht dabei aber qualitativ und nicht quantitativ vor, d.h., er macht eine Analyse zu mehr oder weniger Markierung, es werden jedoch für die verschiedenen linguistischen Ebenen keine Zahlen ermittelt und am Schluss miteinander verrechnet. Untersucht wird Überspezifikation in vier Bereichen. Erstens ist das phonologische System komplexer, je mehr markierte Phoneme das System aufweist \citep[135]{McWhorter2001}. Mit markiert sind selten vorkommende Phoneme gemeint (z.\,B.\ Klicks, hintere gerundete Vokale etc.), wobei es hier nicht um komplexe Artikulation, sondern um Implikation geht: Eine Sprache mit markierten Phonemen hat auch unmarkierte und ist somit komplexer als eine Sprache mit unmarkierten Phonemen \citep[135]{McWhorter2001}. Zweitens wird die Syntax komplexer, je mehr Regeln verarbeitet werden müssen, wie beispielsweise Asymmetrien zwischen Matrix- und Nebensatz (z.\,B.\ Verb\-zweit-/Verb\-end\-stel\-lung), Ergativ/Absolutiv und Nominativ/Akkusativ in derselben Sprache usw. \citep[136]{McWhorter2001}. Drittens wird die Grammatik komplexer, wenn feine semantische und/oder pragmatische Unterscheidungen overt kodiert werden und dieser Ausdruck grammatikalisiert ist \citep[136]{McWhorter2001}. Ein Beispiel hierfür gibt Koasati (Muskogee-Sprache, Nordamerika), das spezielle pronominale \isi{Affixe} hat, die nur mit stativen Verben verwendet werden \citep[137]{McWhorter2001}. Viertens führt die Flexionsmorphologie generell zu Überspezifikation. Hier werden vor allem drei Bereiche hervorgehoben. A) Morphophonologie und Suppletion: Die Morphophonologie verursacht Prozesse, die phonetisch nur wenig vorausgesagt werden können (z.\,B.\ \isi{Umlaut}) und Suppletion \citep[137]{McWhorter2001}. B) Allomorphie in der Flexion und arbiträre Allomorphie: Zu Ersterem gehören Unterscheidungen zwischen Substantiv- oder Verbklassen, die von der Flexion kodiert werden; zum zweiten Flexion, die mit der \isi{Wurzel} gelernt und gespeichert werden muss (z.\,B.\ Aspektpaare im Russischen) \citep[138]{McWhorter2001}. C) Die Markierung von Kongruenz erhöht ebenfalls die strukturelle Komplexität \citep[138]{McWhorter2001}. In einem ersten Schritt vergleicht \citet{McWhorter2001} die Kreolsprache Saramaccaans, die Ende des 17. Jh. entstanden ist, mit der synthetischen Sprache Tsez, eine nakho-daghestanische Sprache aus dem Nordkaukasus. Dabei beobachtet er, dass Tsez häufiger als Saramaccaans Überspezifikation zeigt und erklärt dies wie folgt: „[B]y virtue of the fact that they were born as pidgins, and thus stripped of almost all features unnecessary to communication, and since then have not existed as natural languages for a long enough time for diachronic drift to create the weight of ʻornamentʼ that encrusts older languages“ \citep[125]{McWhorter2001}. Man könnte also annehmen, dass der Unterschied in der Komplexität zwischen analytischen Sprachen und Kreolsprachen nicht so groß ausfällt \citep[143--144]{McWhorter2001}. Dazu wird Saramaccaans mit Lahu verglichen. Das Resultat des Vergleichs zeigt genau das Gegenteil von dem, was man erwarten würde: „It is demonstrated that this complexity differential remains robust even when creoles are compared with older languages lacking inflection“ \citep[125]{McWhorter2001}. Zusammengefasst kann man also festhalten, dass die strukturelle Komplexität einer Sprache mit der Geschichte ihrer Entstehung und ihres Wandels zusammenhängt.

\citet{SzmrecsanyiKortmann2009} untersuchen die morphologische/morphosyntaktische Komplexität vieler unterschiedlicher englischer Varietäten, die in folgende Typen eingeteilt werden können: traditionelle L1-Va\-rie\-tä\-ten mit wenig \isi{Sprachkontakt} (z.\,B.\ East Anglia English), L1-Va\-rie\-tä\-ten mit viel \isi{Sprachkontakt} (z.\,B.\ Australian English), Englisch-basierte Pidgins und Kreols (z.\,B.\ Tok Pisin) und L2-Va\-rie\-tä\-ten (z.\,B.\ Hong Kong English) \citep[64–65]{SzmrecsanyiKortmann2009}. Die strukturelle Komplexität wird anhand von vier Parametern definiert und gemessen: Ornamentale Regeln, Schwierigkeiten beim L2-Er\-werb, Grammatizität/Redundanz, Irregularitäten \citep[64–65]{SzmrecsanyiKortmann2009}. Ornamentale Regeln sind solche, die Unterscheidungen und Asymmetrien hinzufügen, ohne dass sie einen kommunikativen oder funktionalen Vorteil bringen, z.\,B.\ \isi{Genus} \citep[68]{SzmrecsanyiKortmann2009}. Die Schwierigkeit beim L2-Er\-werb wird durch den Grad gemessen, zu dem eine bestimmte Varietät nicht jene Phänomene aufweist, „that L2 acquisition research has shown to recur in interlanguage varieties“ \citep[69]{SzmrecsanyiKortmann2009}. Dazu zählen beispielsweise die fehlende Markierung der Vergangenheit bei regelmäßigen Verben oder die fehlende Inversion. Die Merkmale dieser zwei Parameter sind binär, d.h., entweder kommen sie vor oder nicht (Liste in \citealt[69–71]{SzmrecsanyiKortmann2009}). Die Komplexität der vier Typen von Varietäten wird folglich durch das durchschnittliche Vorkommen dieser Merkmale berechnet. Die Datengrundlage hierfür bildet der \textit{World Atlas of Morphosyntactic \isi{Variation} in English} \citep[65--66]{SzmrecsanyiKortmann2009}. Die Datengrundlage für die Parameter Grammatizität und Irregularität sind verschiedene Korpora, die jedoch keine Pidgins und Kreols beinhalten \citep[67]{SzmrecsanyiKortmann2009}. Dies ermöglicht, die strukturelle Komplexität durch die \isi{Textfrequenz} zu messen. Grammatizität/Redundanz beinhaltet die synthetische Grammatizität, d.h. die gebundenen grammatischen Morpheme und die analytische Grammatizität, d.h. die freien grammatischen Morpheme \citep[71--72]{SzmrecsanyiKortmann2009}. Für die synthetische Grammatizität wird ein Syntheseindex berechnet: Die Datenbasis bilden 15.000 Tokens und der Syntheseindex ist die Prozentzahl der gebundenen grammatischen Morpheme pro 1.000 Tokens \citep[72]{SzmrecsanyiKortmann2009}. Auf die gleiche Weise wird der Analyseindex berechnet und die Komplexität der Grammatizität bildet die Summe des Synthese- und Analyseindex \citep[72]{SzmrecsanyiKortmann2009}. Die Irregularität wird ebenfalls durch die \isi{Textfrequenz} der gebundenen grammatischen Morpheme gemessen, jedoch getrennt nach regelmäßigen und unregelmäßigen Allomorphen \citep[74]{SzmrecsanyiKortmann2009}. Genauer wird ein Transparenzindex ermittelt, und zwar durch den Anteil (in Prozent) der regelmäßigen Allomorphe an allen gebundenen grammatischen Allomorphen \citep[74]{SzmrecsanyiKortmann2009}. Es sollen nun kurz die wichtigsten Resultate zusammengefasst werden. Am meisten ornamentale Regeln haben die traditionellen L1-Va\-rie\-tä\-ten, gefolgt von L1-Kon\-takt\-va\-rie\-tä\-ten, Pi\-dgins/Kreols, L2-Va\-rie\-tä\-ten, d.h. „[…] ornamental complexity is clearly a function of the degree of contact […]“ \citep[69]{SzmrecsanyiKortmann2009}. Die Ergebnisse des Parameters L2-Schwie\-rig\-kei\-ten zeigen Folgendes (von am meisten zu am wenigsten L2-Merk\-ma\-le): Pidgins/Kreols > L1-Kon\-takt\-va\-rie\-tä\-ten > traditionelle L1-Va\-rie\-tä\-ten > L2-Va\-rie\-tä\-ten. Diese Resultate sind zu erwarten, jedoch mit Ausnahme der L2-Va\-rie\-tä\-ten, wofür eine Erklärung im Parameter Grammatizität gefunden wird \citep[71]{SzmrecsanyiKortmann2009}. Den höchsten Grammatizitätsindex zeigen die traditionellen L1-Va\-rie\-tä\-ten, danach kommen die L1-Kon\-takt\-va\-rie\-tä\-ten und schließlich die L2-Va\-rie\-tä\-ten \citep[73]{SzmrecsanyiKortmann2009}. Erstens zeigt dies, dass „a history of contact and adult language learning can eliminate certain types of redundancy“ \citep[73]{SzmrecsanyiKortmann2009}. Zweitens ziehen L2-Spre\-cher nicht die einfacheren den komplexeren Merkmale vor, sondern bevorzugen Null-Markierung. Dies erklärt, weshalb die L2-Va\-rie\-tä\-ten besonders wenige Merkmale aufweisen, die als L2-Merk\-ma\-le identifiziert wurden \citep[73]{SzmrecsanyiKortmann2009}. Drittens ist schließlich besonders interessant, dass es zwischen dem Analyse- und dem Syntheseindex keine Ausgleichstendenzen gibt bzw. dass diese Indizes sogar positiv miteinander korrelieren \citep[74]{SzmrecsanyiKortmann2009}. Auch der vierte Parameter weist erwartete Resultate auf. Am transparentesten sind die L2-Va\-rie\-tä\-ten, gefolgt von den L1-Kon\-takt\-va\-rie\-tä\-ten und zuletzt von den traditionellen L1-Va\-rie\-tä\-ten \citep[75]{SzmrecsanyiKortmann2009}. Dies zeigt, dass ausgeprägter \isi{Sprachkontakt} (d.h. Spracherwerb von Erwachsenen), dazu führt, Unregelmäßigkeiten abzubauen \citep[75]{SzmrecsanyiKortmann2009}. Zusammengefasst erweist sich also, „that variety type is a powerful predictor of complexity variance“, weshalb \isi{Sprachkontakt} bezüglich des Grads struktureller Komplexität eine zentrale Rolle spielt \citep[76]{SzmrecsanyiKortmann2009}.

\citet{MaitzNémeth2014} übernehmen den Forschungsaufbau von \citet{SzmrecsanyiKortmann2009} und übertragen ihn auf deutsche Varietäten. Die untersuchten Varietäten sind Zimbrisch (traditionelle L1-Va\-rie\-tä\-t), die deutsche Standardsprache (L1-Kon\-takt\-va\-rie\-tät), Kiche Duits und Unserdeutsch (Pidgin/Kreol) \citep[7--9]{MaitzNémeth2014}. Strukturelle Komplexität wird durch die Parameter Synthetizität, Analytizität, Grammatizität und Irregularität definiert \citep[6--7]{MaitzNémeth2014}. Die Datengrundlage bilden gesprochensprachliche Korpora. Daraus werden 3.000 Wörter zufällig ausgewählt, welche in drei Subsamples à 1.000 Wörter eingeteilt sind \citep[9]{MaitzNémeth2014}. Synthetizität wird durch den Anteil der Wörter am Korpus gemessen (in Prozent), die durch gebundene grammatische Morpheme markiert sind \citep[10]{MaitzNémeth2014}. Dazu gehören die Kasus- und Numerusmarkierung am Nomen, Pronomen und Artikel, die Nu\-me\-rus- und Genusmarkierung und die Komparativformen des \isi{Adjektivs}, die Tempusmarkierung und die Markierung der Person am Verb \citep[10]{MaitzNémeth2014}. Analytizität wird durch den Anteil an Funktionswörtern am Korpus ermittelt, z.\,B.\ Determinierer, Konjunktionen, Auxiliarverben etc. \citep[11--12]{MaitzNémeth2014}. Die Grammatizität besteht aus der Summe der Synthetizität und der Analytizität \citep[12]{MaitzNémeth2014}. Schließlich wird die Irregularität anhand der durchschnittlichen \isi{Textfrequenz} von unregelmäßigen Markern berechnet, wozu z.\,B.\ Stammmodifikationen und Suppletion gehören \citep[12--13]{MaitzNémeth2014}. Die Parameter Synthetizität, Analytizität und Grammatizität zeigen dieselben Resultate: Die höchste Komplexität zeigt Zimbrisch (traditionelle L1-Va\-rie\-tät), gefolgt von der Standardsprache (L1-Kon\-takt\-va\-rie\-tät) und zuletzt von Kiche Duits/Unserdeutsch (Pidgin/Kreol), wobei der Unterschied nur zwischen Zimbrisch und der Standardsprache sowie zwischen Zimbrisch und Kiche Duits/Unserdeutsch signifikant ist, jedoch nicht zwischen der Standardsprache und Kiche Duits/Unserdeutsch \citep[11--13]{MaitzNémeth2014}. Gleiches gilt für den Parameter Irregularität, nur dass hier auch der Unterschied zwischen der Standardsprache und Kiche Duits/Unserdeutsch signifikant ist \citep[14]{MaitzNémeth2014}. Des Weiteren wird eine positive Korrelation zwischen Synthetizität und Analytizität gefunden, was gegen die oft angenommene Ausgleichstendenz zwischen synthetischen und analytischen Phänomenen spricht \citep[15]{MaitzNémeth2014}. Auch eine positive Korrelation zwischen Grammatizität und Irregularität kann beobachtet werden \citep[15]{MaitzNémeth2014}. Folglich ist ein hoher Grad an Irregularität typisch für morphosyntaktisch komplexe Varietäten \citep[17]{MaitzNémeth2014}. Es kann also festgehalten werden, dass \isi{Sprachkontakt}, d.h. Spracherwerb von Erwachsenen, zu Simplifizierungen in der Grammatik führt. \citet{MaitzNémeth2014} geben aber ebenfalls zu bedenken, dass auch weitere Faktoren die Morphologie vereinfachen können, wie beispielsweise phonologischer Wandel (z.\,B.\ Schwächung der Nebensilbe) \citep[20]{MaitzNémeth2014}. Gleichzeitig können jedoch ebenfalls Faktoren wirken, die Simplifizierung hemmen, wie z.\,B.\ die Attitüden gegenüber präskriptiven Normen \citep[21]{MaitzNémeth2014}.
 
Auch \citet{Schreier2016} geht davon aus, dass \isi{Sprachkontakt} die strukturelle Komplexität einer Sprache oder einer Varietät beeinflussen kann. Im Gegensatz jedoch zu den bis hier erörterten Studien kann er zeigen, dass Kontakt nicht nur zu Simplifizierung, sondern auch zu Komplexifizierung führt, und zwar in derselben Varietät zur selben Zeit \citep[139, 148-151]{Schreier2016}. Auch die (so\-zio-)lin\-gu\-is\-ti\-schen Faktoren differenziert er genauer, wobei es sich um folgende handelt: a) die Eigenschaften der Systeme, die miteinander in Kontakt stehen; b) das soziale Fundament des Kontakts, wozu auch demografische Daten und die Besiedlung gehören; c) die Intensität des Kontakts, wie z.\,B.\ die Stärke von Substrateffekten \citep[139]{Schreier2016}. Es geht vor allem darum, „to do justice to the multifaceted and interwoven (social and linguistic) phenomena of contact-induced change“ \citep[139]{Schreier2016}. Vor allem kann \citet{Schreier2016} zeigen, dass die Dichotomie – Varietäten mit viel Kontakt werden einfacher, während Varietäten mit wenig Kontakt komplexer werden – zu kurz greift. Als erstes wird anhand der Reduktion von Konsonantenclustern dargestellt, dass nicht die Intensität des Sprachkontakts entscheidend ist, sondern die systemischen und typologischen Unterschiede zwischen den Sprachen im Kontakt: Der starke Kontakt von zwei Varietäten, die beide komplexe Konsonantencluster aufweisen, hat kaum einen Effekt, während ein weniger starker Kontakt mit einer Varietät mit einfacheren Konsonantenclustern einen höheren Effekt hat \citep[144]{Schreier2016}. Dass starker Kontakt zu Simplifizierungen führt, trifft folglich nur auf unterschiedliche Systeme zu, nicht aber, wenn die Systeme große Ähnlichkeiten aufweisen \citep[144--145]{Schreier2016}. Zweitens zeigt \citet{Schreier2016}, dass das System der \isi{Personalpronomen} von Varietäten mit wenig Kontakt relativ regelmäßig, also wenig komplex ist und dass jenes von Varietäten mit viel Kontakt höhere Komplexität aufweist \citep[145--147]{Schreier2016}. Beispielsweise hat Tok Pisin in der 1. Person Plural zusätzlich die Unterscheidung inklusiv/exklusiv grammatikalisiert \citep[147]{Schreier2016}. Drittens ist festzustellen, dass Tristan da Cunha Englisch weder eine rein \textit{Low-Contact-} noch eine rein \textit{High-Contact-Varietät} ist und gleichzeitig Simplifizierung und Komplexifizierung aufweist. Je nachdem, welches Zeitfenster man sich anschaut, handelt es sich bei Tristan da Cunha Englisch um eine Varietät mit viel oder mit wenig Kontakt \citep[148]{Schreier2016}. Beispiele für Simplifizierung sind die Ausgleiche im Paradigma von \textit{sein} (\textit{is} vs. \textit{was} unabhängig von Person und \isi{Numerus}) oder die hohe Rate an Vereinfachung komplexer Konsonantencluster \citep[149--150]{Schreier2016}. Komplexifizierung zeigt sich z.\,B.\ darin, dass der dentale Frikativ vier Realisierungen hat: dentale Frikative, labiodentale Frikative und labiodentale Plosive sowie Sibilanten \citep[150]{Schreier2016}. Schließlich kann noch angefügt werden, dass Tristan da Cunha Englisch \isi{Archaismen} erhalten hat und gleichzeitig aber auch \isi{Innovationen} aufweist \citep[150--151]{Schreier2016}. Aufgrund dieser Resultate kritisiert \citet{Schreier2016} besonders drei Aspekte der bisherigen Modelle. Erstens ist es schwierig, Varietäten in die binäre Unterscheidung zwischen Varietäten mit viel Kontakt und wenig Kontakt einzuteilen \citep[151--152]{Schreier2016}. Zweitens berücksichtigen die Modelle die \isi{Diachronie} zu wenig, denn „isolation effects may override contact effects over time“ \citep[152]{Schreier2016}. Drittens können Komplexifizierung und Simplifizierung gleichzeitig auftreten \citep[153]{Schreier2016}. \citet{Schreier2016} schlägt ein Modell vor, das wenig Kontakt und viel Kontakt als die beiden Pole auf einem Kontinuum ansetzt \citep[153--154]{Schreier2016}. Dieses Modell trägt der Tatsache Rechnung, dass eine Varietät komplexe und einfache Phänomene gleichzeitig aufweisen kann.

\subsection{Geografische Isolation}\label{2.2.4} 

Im vorangehenden Kapitel wurden einige Studien skizziert, in denen mit dem Begriff \isi{Isolation}, wenn er verwendet wird, vorwiegend soziale \isi{Isolation} gemeint ist, d.h. Sprachgemeinschaften mit wenig \isi{Sprachkontakt}, wenig L2-Spre\-chern etc. Vor allem bei \citet{Jakobson1929}, \citet{Braunmüller1984} und \citet[u.a.]{Trudgill2011} impliziert \isi{Isolation} jedoch meistens auch geografische \isi{Isolation}. In diesem Kapitel sollen drei Untersuchungen vorgestellt werden, die sich explizit mit geografischer \isi{Isolation} beschäftigen, nämlich \citet{Nichols1992,Nichols2016} und \citet{Garzonio2016}. Obwohl diese Studien erst Ende des 20. Jh.\slash Anfang des 21. Jh. entstanden sind, weist bereits \citet{Jakobson1929} darauf hin, dass die ukrainischen Dialekte mit einem großen Vokalinventar sich eher an der Peripherie des ukrainischen Sprachgebiets befinden: „Cette différence [entre les systèmes de voyelles] est due, en premier lieu à la tendance conservatrice qui est caractéristique des parlers de la périphérie, et en second lieu à des différences fonctionnelles“ (\citealt[73]{Jakobson1929}, ausführliches Zitat in \sectref{2.2.1}).

Die wichtigsten Aspekte aus \citet{Nichols1992} für das Ziel dieser Arbeit wurden in \citet{BaechlerSeiler2016a} zusammengefasst. Die anschließenden Ausführungen folgen dieser Zusammenfassung. Nichols ist wahrscheinlich die erste Linguistin, die die mögliche Korrelation zwischen geografischer \isi{Isolation} und struktureller Komplexität explizit untersucht hat. Ihr wichtigstes Ziel ist, typologische Eigenschaften und Geografie in Beziehung zu setzen: „In describing the distribution of types and typological features we can often make active use of geography as a predictive factor. This means viewing the languages of a region as a population and demonstrating a correlation between the location or type of the region and the distribution of traits within the population or between populations“ \citep[12]{Nichols1992}. \citet{Nichols1992} unterscheidet u.a. zwischen \textit{Residual} und \textit{Spread Zones}, wobei ersterer als isoliertes Areal und letzterer als nicht isoliertes Areal bezeichnet werden kann. \textit{Spread Zones} sind wie folgt charakterisiert:

% \ea%1
%     \label{ex:key:1}
%     \gll\\
%         \\
%     \glt
%     \z
\begin{quote}


(1) Little genetic diversity

% \ea%2
%     \label{ex:key:2}
%     \gll\\
%         \\
%     \glt
%     \z

(2) Low structural diversity

% \ea%3
%     \label{ex:key:3}
%     \gll\\
%         \\
%     \glt
%     \z

(3) The language families present in the spread zone are shallow.

% \ea%4
%     \label{ex:key:4}
%     \gll\\
%         \\
%     \glt
%     \z

(4) Rapid spread of languages or language families and consequent language succession.

% \ea%5
%     \label{ex:key:5}
%     \gll\\
%         \\
%     \glt
%     \z

(5) Classic dialectal-geographical area with innovating center and conservative periphery.

% \ea%6
%     \label{ex:key:6}
%     \gll\\
%         \\
%     \glt
%     \z

(6) No net long-term increase in diversity. A spread zone is a long-lasting phenomenon, but it preserves little linguistic evidence of its history.

% \ea%7
%     \label{ex:key:7}
%     \gll\\
%         \\
%     \glt
%     \z

(7) The spreading language serves as a lingua franca for the entire area or a large part of it.


\citep[16--17]{Nichols1992}
\end{quote}

\noindent
Im Gegensatz zu den \textit{Spread Zones} weisen die \textit{Residual Zones} folgende typischen Charakteristika auf:

\begin{quote}


% \ea%1
%     \label{ex:key:1}
%     \gll\\
%         \\
%     \glt
%     \z

(1) High genetic density.

% \ea%2
%     \label{ex:key:2}
%     \gll\\
%         \\
%     \glt
%     \z

(2) High structural diversity.

% \ea%3
%     \label{ex:key:3}
%     \gll\\
%         \\
%     \glt
%     \z

(3) The language families [...] are deep.

% \ea%4
%     \label{ex:key:4}
%     \gll\\
%         \\
%     \glt
%     \z

(4) No appreciable spread of languages or families. No language succession.

% \ea%5
%     \label{ex:key:5}
%     \gll\\
%         \\
%     \glt
%     \z

(5) No clear center of innovation.

% \ea%6
%     \label{ex:key:6}
%     \gll\\
%         \\
%     \glt
%     \z

(6) Accretion of languages and long-term net increase in diversity.

% \ea%7
%     \label{ex:key:7}
%     \gll\\
%         \\
%     \glt
%     \z

(7) No lingua franca [...] for the entire area; local bilingualism or multilingualism is the main means of inter-ethnic communication.

\citep[21]{Nichols1992}
\end{quote}

Das wichtigste Resultat von Nicholsʼ Studie bezüglich der möglichen Korrelation zwischen \isi{Isolation} und struktureller Komplexität ist, dass „[r]esidual zones show relatively high complexity, equal to or greater than that of their respective continents. Spread zones show somewhat lower average complexity, equal to or lower than that of their respective continents“ \citep[192]{Nichols1992}. \citet{Nichols1992} nimmt aber auch an, dass es innovative und konservative Gegenden innerhalb der \textit{Residual} und \textit{Spread Zones} gibt. In einer \textit{spread Zone} kann ein innovatives Zentrum und eine konservative Peripherie gefunden werden \citep[17]{Nichols1992}. In bergigen Regionen, die als \textit{residual Zones} gelten können, gibt es \isi{Innovationen} eher an der Peripherie, d.h. im Tal oder in der Ebene, und \isi{Archaismen} im Zentrum, d.h. im Gebirge \citep[14]{Nichols1992}.

Dass \textit{Spread} und \textit{Residual} \textit{Zones} tatsächlich innovative und konservative Areale aufweisen, kann \citet{Nichols2016} in einer weiteren Untersuchung zeigen. Dazu werden Sprachen aus dem Ostkaukasus und aus der eurasischen Steppe analysiert. Bei den Sprachen aus dem Ostkaukasus handelt es sich um solche aus der nakho-daghestanischen Sprachfamilie \citep[120--121]{Nichols2016}. Aus der eurasischen Steppe werden 22 Sprachen aus folgenden Sprachfamilien herangezogen, wobei einige Sprachen keiner Sprachfamilie zugeordnet werden können (es werden nur die Sprachfamilien bzw. die isolierten Sprachen aufgezählt): Turksprachen, Mongolisch, Tungusisch, Tibetisch, Koreanisch, Japanisch, Ainu, Nivkh, Tschuk\-tschisch-Kam\-tscha\-da\-lisch, Es\-ki\-mo-A\-leu\-tisch, Jukagir, Jenissej, östliches Uralisch \citep[129]{Nichols2016}. Eingeteilt werden die Sprachen in zwei Kategorien, und zwar in sich ausbreitende Sprachen (mit vielen L2-Ler\-nern, Lingua Franca) und in isolierte, sich nicht ausbreitende Sprachen. Gemessen wird die strukturelle Komplexität durch die Größe des Inventars (d.h. die Anzahl der Elemente, z.\,B.\ die Anzahl der Phoneme, Flexionskategorien, Wortabfolgen etc.) und durch den Grad der Opazität (z.\,B.\ Suppletion, Sandhi, Mehrfachausdruck, Synthese etc.) \citep[118]{Nichols2016}. Die Größe des Inventars wird auf verschiedenen linguistischen Ebenen ermittelt, nämlich Phonologie, Synthese, Klassifizierer, Syntax und Lexikon, und Opazität durch \isi{Genus} und Deklination \citep[137]{Nichols2016}. Eine genaue Liste der untersuchten linguistischen Variablen ist aus \citet[137]{Nichols2016} zu entnehmen. Quantifiziert wird auf zwei Arten, und zwar je nachdem, ob es sich um eine binäre Variable handelt oder nicht. Auf der phonologischen Ebene kann beispielsweise die Anzahl der kontrastiven obstruenten Arten der Artikulation einfach gefunden werden, während Töne eine binäre Variable darstellen, die mit +1 für das Vorhandensein dieser Variable zu Buche schlägt und mit +0 für das nicht Vorhandensein \citep[137]{Nichols2016}. Für den Ostkaukasus kann \citet{Nichols2016} zeigen, dass Opazität mit isolierten Sprachen korreliert, während Transparenz und eine geringe \isi{Inventargröße} in sich ausbreitenden Sprachen zu beobachten sind \citep[129]{Nichols2016}. Da es sich um ein gebirgiges Gebiet handelt, kann \isi{Isolation} durch geografische Höhe quantifiziert werden \citep[129]{Nichols2016}. Diese Resultate sind aufgrund des soziolinguistischen Kontextes dieser Region zu erwarten, die kurz erörtert werden. Die Sprachen, die in der Ebene oder am Fuße eines Berges situiert sind, breiten sich bergaufwärts aus und weisen deshalb viele L2-Spre\-cher auf \citep[123--124]{Nichols2016}. In den Bergdörfern hingegen sind enge Netzwerke und kaum L2-Ler\-ner der isolierten Sprache zu beobachten \citep[123--124]{Nichols2016}. Neben den nakho-daghestanischen Sprachen des Ostkaukasus untersucht \citet{Nichols2016} auch unterschiedliche Sprachen der eurasischen Steppe. In dieser Gruppe weisen sich ausbreitende Sprachen eine geringe \isi{Inventargröße} auf, während isolierte Sprachen an der Peripherie des Gebiets einen hohen Grad an Opazität haben \citep[129--130]{Nichols2016}. Die eurasische Steppe kann eine lange Geschichte an Ausbreitung und Überlagerung unterschiedlicher Sprachen vorweisen. Sie hatten also immer viele L2-Ler\-ner, was zur Simplifizierung der Grammatik führt \citep[118--119]{Nichols2016}. Den Sprachen an der Peripherie (besonders im Norden) fehlen jüngste Ausbreitungsphasen, oder sie haben nie welche gehabt, d.h., diese Sprachen verfügten über längere Zeitabschnitte, in denen sich natürliche Komplexifizierung durch L1-Ü\-ber\-tra\-gung entwickeln konnte \citep[118--119]{Nichols2016}. Es kann also festgehalten werden, dass geringe \isi{Inventargröße} vor allem in sich ausbreitenden Sprachen gefunden werden kann, die zur Kommunikation zwischen verschiedenen Gesellschaften dienen, und hohe Opazität in isolierten Sprachen, die generell nicht als L2 gelernt werden \citep[132--133]{Nichols2016}.

\citet{Garzonio2016} untersucht die syntaktische Komplexität von Entscheidungsfragesätzen sowie w-Fragesätzen, und zwar im üblichen System nicht-alpiner norditalienischer Dialekte sowie in den folgenden alpinen ita\-lo-ro\-ma\-ni\-schen Dialekten: Monnese, Bellunese und Mendrisiotto \citep[95]{Garzonio2016}. Alle drei Dialekte liegen im alpinen Gebiet und können im Gegensatz zu vielen anderen norditalienischen Dialekten als isoliert gelten, wobei Monnese noch etwas isolierter ist als Bellunese und Mendrisiotto \citep[95, 102]{Garzonio2016}. Die syntaktische Komplexität wird anhand zweier Parameter gemessen: Komplexität der Derivation und die Anzahl freier Varianten für dieselbe Funktion (= Optionalität) \citep[98--99]{Garzonio2016}. Die Komplexität der Derivation wird anhand der Anzahl Move- und Mergeoperationen quantifiziert: je mehr Move- und Mergeoperationen, desto komplexer die Derivation \citep[98--99]{Garzonio2016}. Die Resultate sind in \tabref{table2.1} zusammengefasst.

% \textbf{Tabelle 2.1: Resultate syntaktischer Komplexität, Move/Merge \citep[111--112]{Garzonio2016}}\\

\begin{table}
\caption{Resultate syntaktischer Komplexität, Move/Merge \citep[111--112]{Garzonio2016} }\label{table2.1}
\begin{tabular}{lll}
\lsptoprule
\multicolumn{3}{l}{\mbox{Komplexität der Derivation:}}\\
 & Entscheidungsfragesätze & w-Fragesätze \\\midrule
\mbox{norditalienische Dialekte} & 1/1 & 2/1 \\
Monnese & 1/1 & 2/1 \\
Bellunese & 1/1 & 1/1 \\
Mendrisiotto & 0/1 & 1/1 \\
\midrule
\multicolumn{3}{l}{Optionalität:}\\
 & & w-Fragesätze   \\\midrule
\mbox{norditalienische Dialekte} & & 1  \\
Monnese & & 3  \\
Bellunese & & 1   \\
Mendrisiotto & & 5 (3)   \\
\lspbottomrule
\end{tabular}
\end{table}

Bezüglich der Optionalität zeigt sich, dass Monnese und Mendrisiotto deutlich komplexer sind. Dies hat vor allem damit zu tun, dass in diesen Dialekten das w-Element verdoppelt werden und dann links und/oder rechtsperipher stehen kann \citep[105, 108-111]{Garzonio2016}. Mendrisiotto weist zudem im Gegensatz zu den übrigen Varietäten des Samples in den w-Elementen drei verschiedene Formen auf: \textit{cusè}, \textit{cusa}, \textit{sa} \citep[109]{Garzonio2016}. Die Komplexität der Derivation betreffend sind die nicht-alpinen norditalienischen Dialekte und Monnese komplexer als Bellunese und Mendrisiotto. Garzonio kann jedoch zeigen, dass in Bellunese und Mendrisiotto die w-Fragesätze mit einer speziellen Semantik höhere Komplexität aufweisen, nämlich zwei Move- und zwei Mergeoperationen \citep[108, 111]{Garzonio2016}. Schließlich sind zwei interessante Beobachtungen hervorzuheben. Erstens hat Monnese eine \textit{tun}-Periphrase grammatikalisiert, die jedoch in der Derivation nicht zu höherer Komplexität führt \citep[103--104]{Garzonio2016}. \isi{Isolation} hat in diesem Fall also nicht höhere syntaktische Komplexität hervorgebracht, sondern eine andere Syntax \citep[113]{Garzonio2016}. Zweitens weist Medrisiotto in der Derivation die geringste, aber in der Optionalität die höchste Komplexität auf \citep[111--114]{Garzonio2016}. Zusammengefasst kann also Folgendes festgehalten werden: Erstens sind isolierte Dialekte nicht komplexer als nicht-isolierte bezüglich der derivationellen Komplexität, jedoch können sie einen hohen Grad an Optionalität aufweisen; zweitens variiert die derivationelle Komplexität zwischen verwandten Varietäten nur wenig; drittens können sich Fragesatztypen in ihrer Komplexität unterscheiden (am extremsten im Mendrisiotto) \citep[114]{Garzonio2016}.

\subsection{Arten von Komplexität}\label{2.2.5}

Hier soll nun zuerst die etablierte Unterscheidung zwischen absoluter und relativer Komplexität vorgestellt werden, die von \citet{Miestamo2008} vorgeschlagen wird. Danach wird eine detailliertere Einteilung der Typen struktureller Komplexität erörtert, die auf den Philosophen \citet{Rescher1998} zurückgeht und von \citet{MiestamoSinnemäkiKarlsson2008} für linguistische Phänomene adaptiert wurde. Schließlich werden die Definitionen struktureller Komplexität der vorangehenden Kapitel zusammengefasst und kategorisiert.

Neben lokaler und globaler Komplexität (vgl. \sectref{2.1.2}) unterscheidet \citet{Miestamo2008} zwischen absoluter und relativer Komplexität. Im relativen Ansatz wird Komplexität durch den Aufwand oder Schwierigkeit für den Sprecher definiert, d.h., „how difficult a phenomenon is to process (encode/decode) or learn. The more costly or difficult a linguistic phenomenon is, the more complex it is“ \citep[25]{Miestamo2008}. Es stellt sich also immer die Frage, für wen ein linguistisches Phänomen schwierig oder aufwändig ist, wie z.\,B.\ Hörer, Sprecher, L1- oder L2-Ler\-ner. Dabei ist zu beachten, dass ein linguistisches Phänomen für eine Gruppe (z.\,B.\ Hörer) Aufwand oder Schwierigkeiten verursacht, während dasselbe Phänomen für eine andere Gruppe (z.\,B.\ Sprecher) einfach ist \citep[25]{Miestamo2008}. Beschäftigt man sich mit absoluter Komplexität, steht ausschließlich das linguistische System im Vordergrund. Die Grundidee dahinter ist, je mehr Elemente ein System hat, desto komplexer ist es \citep[24]{Miestamo2008}. Um die Idee der Anzahl Elemente genereller zu fassen, greift \citet{Miestamo2008} auf Konzepte der \isi{Informationstheorie} zurück (z.\,B.\ Shannon-, Kolmogorov-Komplexität) und schlägt, basierend auf \citet{Dahl2004}, folgende Definition absoluter Komplexität vor:

\begin{quote}
[T]he complexity of a linguistic phenomenon may be measured in terms of the length of the description of that phenomenon; the longer a description a phenomenon requires, the more complex it is […] A less complex phenomenon can be compressed to a shorter description without losing information. On a high level of abstraction we may say that we are still dealing with the number of parts in a system, but these parts are now the elements that constitute the description of the system. \citep[24--25]{Miestamo2008}
\end{quote}

In Arbeiten zur strukturellen Komplexität sind bereits unzählige Übersichten zusammengestellt worden, die resümieren, was genau mit struktureller Komplexität gemeint ist und welche Arten struktureller Komplexität in Sprachen vorgefunden werden (z.\,B.\ \citealt[10--12]{SzmrecsanyiKortmann2012}). Im Gegensatz dazu gibt \citet{Rescher1998} in seiner philosophischen Auseinandersetzung mit dem Thema Komplexität eine breitere Definition und Klassifikation der Arten von Komplexität. Komplexität definiert er folgendermaßen:

\begin{quote}
Complexity is first and foremost a matter of the number and variety of an item’s constituent elements and of the elaborateness of their interrelational structure, be it organizational or operational. Any sort of system or process – anything that is a structured whole consisting of interrelated parts – will be to some extent complex. Accordingly, all manner of things can be more or less complex: natural objects (plants or river systems), physical artifacts (watches or sailboats), mind-engendered processes (languages or instructions), bodies of knowledge, and so on. \citep[1]{Rescher1998}
\end{quote}

Des Weiteren unterscheidet \citep[8--16]{Rescher1998} zwischen unterschiedlichen Typen von Komplexität. Diese Unterscheidung wurde von \citet{MiestamoSinnemäkiKarlsson2008} übernommen und für die Kategorisierung linguistischer Phänomene angepasst \citep[VIII–IX]{MiestamoSinnemäkiKarlsson2008}. Davon erstellt \citet{Sinnemäki2011} eine modifizierte Version, die in \figref{table2.2} wiedergegeben ist und welche weiter unten anhand der \figref{table2.3} genauer erörtert wird.

% \textbf{Tabelle 2.2: Arten der Komplexität (\citealt[9]{Rescher1998}; \citealt[23]{Sinnemäki2011}}\\

\begin{figure}
\caption{Arten der Komplexität (\citealt[9]{Rescher1998}; \citealt[23]{Sinnemäki2011})}\label{table2.2}
\noindent\fbox{\parbox{\textwidth}{\selectlanguage{english}\begin{enumerate}[label=\arabic*.,listparindent=0pt,leftmargin=.5cm,rightmargin=.5cm]
\item Epistemic modes
\begin{enumerate}[label=\Alph*.]
	\item Formulaic complexity
	\begin{enumerate}[label=\alph*.]
		\item Descriptive complexity: length of the account that must be given to provide an adequate description of a given system.
		\item Generative complexity: length of the set of instructions that must be given to provide a recipe for producing a given system.
		\item Computational complexity: amount of time and effort involved in resolving a problem.
	\end{enumerate}
\end{enumerate}
\item Ontological modes
	\begin{enumerate}[label=\Alph*.]
		\item Compositional complexity
		\begin{enumerate}[label=\alph*.]
			\item Constitutional complexity: number of constituent elements (e.g., in terms of the number of phonemes, morphemes, words, or clauses).
			\item Taxonomic complexity (or heterogeneity): variety of constituent elements, that is, the number of different kinds of components (e.g., tense-aspect distinctions, clause types).
		\end{enumerate}
		\item Structural complexity
		\begin{enumerate}[label=\alph*.]
			\item Organizational complexity: variety of ways of arranging components in different modes of interrelationship (e.g., phonotactic restrictions, variety of distinctive word orders).
			\item Hierarchical complexity: elaborateness of subordination relationships in the modes of inclusion and subsumption (e.g., recursion, intermediate levels in lexical-semantic hierarchies).
		\end{enumerate}
	\end{enumerate}
\item Functional complexity
	\begin{enumerate}[label=\Alph*.]
		\item Operational complexity: variety of modes of operation or types of functioning (e.g., cost-related differences concerning the production and comprehension of utterances).
		\item Nomic complexity: elaborateness and intricacy of the laws governing a phenomenon (e.g., anatomical and neurological constraints on speech production; memory restrictions).
	\end{enumerate}
\end{enumerate}}}
\end{figure}

Vergleicht man diese Arten von Komplexität und \citeauthor{Miestamo2008}s \citeyearpar{Miestamo2008} Einteilung in absolute und relative Komplexität, kann die epistemische und die ontologische Komplexität der absoluten Komplexität zugeteilt werden und die funktionale Komplexität der relativen Komplexität.

Abschließend soll nun versucht werden, die in diesem Kapitel diskutierten Arbeiten in die Kategorisierung der Komplexitätsarten von \citet{Rescher1998} und \citet{Sinnemäki2011} einzuordnen. Eine tabellarische Übersicht gibt \figref{table2.3}, die die Komplexitätsarten auslässt, welche in den erörterten Arbeiten keine Rolle spielen, da es das Ziel ist, einen Überblick zu bieten. Deswegen werden linguistische Konzepte und Prinzipien einzelnen Komplexitätsarten zugeordnet, was natürlich eine Verallgemeinerung zur Konsequenz hat.

Alle Studien mit Ausnahme von \citet{Sampson2001} und \citet{Garzonio2016} beruhen mehr oder weniger explizit auf der Grundidee der deskriptiven Komplexität: Je länger die Beschreibung der Grammatik ist, desto komplexer ist die Grammatik. Ausschließlich \citet{Garzonio2016} operationalisiert die generative Komplexität, um die Komplexität einer Sprache zu messen. Dazu zählt er die Anzahl von Move- und Mergeoperationen, wobei es sich um Instruktionen zur Ableitung (also Produktion) wohlgeformter Sätze handelt. Die ontologischen Arten bilden den Schwerpunkt in der bisherigen Beschäftigung mit linguistischer Komplexität. Diese werden unterteilt in kompositionelle und strukturelle Komplexität. Zur kompositionellen Komplexität gehören die Größe des Inventars (konstitutionelle Komplexität) und die Anzahl Unterscheidungen (und deren Kodierung), die in einem System gemacht werden (taxonomische Komplexität). Die \isi{Inventargröße} wird vorwiegend in der Phonologie untersucht, wie z.\,B.\ das \mbox{Phon-/}Pho\-nem\-in\-ven\-tar (\citealt{Jakobson1929}, \citealt{HayBauer2007}, \citealt{Nichols2016}, \citealt{Schreier2016}). Dazu gehört aber auch die Komplexität der Konsonantencluster bei \citet{Schreier2016} sowie die markierten Phoneme bei \citet{McWhorter2001}, da es hier um eine Implikation geht: Ein Sprachsystem, das markierte Phoneme hat, hat auch unmarkierte und folglich ein größeres Inventar. Zur \isi{Inventargröße} kommt bei \citet{Nichols2016} neben der Größe des Phoneminventars die Anzahl unterschiedlicher Präfixtypen dazu. Bei allen anderen Variablen, die \citet{Nichols2016} unter \isi{Inventargröße} subsummiert, handelt es sich nach \citeauthor{Rescher1998}s \citeyearpar{Rescher1998} und \citeauthor{Sinnemäki2011}s \citeyearpar{Sinnemäki2011} Einteilung eher um taxonomische Komplexität. Damit ist gemeint, dass je mehr Unterscheidungen (und Kodierungen dieser Unterscheidungen) gemacht werden, desto komplexer ist ein System. Beispielsweise ist ein System, das \isi{Genus} unterscheidet, komplexer als eines, das \isi{Genus} nicht unterscheidet, und je mehr \isi{Genera} ein System unterscheidet, desto komplexer ist es ebenfalls. Dazu gehören die meisten Variablen bei \citet{Nichols2016}, die ornamentalen Regeln bei \citet{SzmrecsanyiKortmann2009} sowie der Verstoß gegen das Ökonomieprinzip bei \citet{Kusters2003}. Auch \citeauthor{McWhorter2001}s \citeyearpar{McWhorter2001} Konzept der Überspezifikation kann zur taxonomischen Komplexität gerechnet werden und insbesondere sein dritter Parameter, nämlich die overt kodierten feinen semantischen und/oder pragmatischen Unterscheidungen. Schließlich zählen bei \citet{Schreier2016} die Grammatikalisierung von Exklusiv/Inklusiv in den \isi{Personalpronomen} wie auch die Ausgleiche im Verbalparadigma (Verlust von Person- und Numerusunterscheidung, also Simplifizierung) dazu. In der strukturellen Komplexität wird zwischen der organisationellen und der hierarchischen Komplexität unterschieden. Die organisationelle Komplexität beinhaltet jene Phänomene, die das 1-zu-1-Verhältnis zwischen Form und Bedeutung verletzen \citep[25]{Sinnemäki2011}. Bei \citet{Sinnemäki2009} ist das die Verletzung von Ökonomie und Distinktheit, d.h., eine Funktion (hier Markierung von Agens und Patiens) wird zu viel oder zu wenig markiert. Auch die Grammatizität bei \citet{MaitzNémeth2014} sowie \citet{SzmrecsanyiKortmann2009} könnte dazugezählt werden. Die Grammatizität wird durch die \isi{Textfrequenz} freier und gebundener grammatischer Morpheme gemessen. Damit können jedoch weder Aussagen gemacht werden, ob eine Funktion zu viel oder zu wenig kodiert wird, noch, welche und wie viele Funktionen kodiert werden. Das Konzept der Grammatizität ist also nur schwer in Reschers Kategorisierung einzuordnen. Des Weiteren gehören zur organisationelle Komplexität alle Phänomene, die unter dem Konzept Opazität/Transparenz zusammengefasst werden können: morphophonologische Regeln, Suppletion, Allomorphie, Homonymie, Fusion, Irregularität etc. Arbeiten dazu stammen von \citet{Braunmüller1984}, \citet{McWhorter2001}, \citet{Kusters2003}, \citet{SzmrecsanyiKortmann2009}, \citet{MaitzNémeth2014} sowie \citet{Nichols2016}. Zur organisationellen Komplexität wird auch die Anzahl unterschiedlicher Wortabfolgen gerechnet (bei \citealt{McWhorter2001}, \citealt{Nichols2016}). Schließlich untersucht \citet{Sampson2001} hierarchische Komplexität, indem die Einbettungstiefe gemessen wird. Wie bereits oben erwähnt, zählen die epistemische und die ontologische Komplexität zur absoluten Komplexität und die funktionale Komplexität zur relativen Komplexität. Zur funktionalen Komplexität gehört die operationelle Komplexität, womit u.a. der Aufwand und die Schwierigkeiten in der Produktion und Rezeption der Sprache durch unterschiedliche Gruppen (z.\,B.\ Sprecher, Hörer, L1- oder L2-Ler\-ner) gemeint sind. Dies spielt besonders bei \citet{Kusters2003} eine zentrale Rolle, bei \citet{SzmrecsanyiKortmann2009} betrifft dies den zweiten von vier Parametern. Die vorliegende Arbeit beschäftigt sich mit der generativen Komplexität, die genauer durch konstitutionelle, taxonomische und organisationelle Komplexität definiert ist. Genauer wird dies im anschließenden Kapitel erläutert.

%\textbf{Tabelle 2.3: Einordnung der diskutierten Arbeiten in die Komplexitätsarten nach \citet{Rescher1998} und \citet{Sinnemäki2011} }

\begin{figure}
\caption{Einordnung der diskutierten Arbeiten in die Komplexitätsarten nach \citet{Rescher1998} und \citet{Sinnemäki2011}}\label{table2.3}
\noindent\fbox{\parbox{\textwidth}{\selectlanguage{english}\begin{enumerate}[label=\arabic*.,listparindent=0pt,leftmargin=.5cm,rightmargin=.5cm]
\item Epistemic modes
\begin{enumerate}[label=\Alph*.]
	\item Formulaic complexity
	\begin{enumerate}[label=\alph*.]
		\item Descriptive complexity: all (except \citealt{Sampson2001} and \citealt{Garzonio2016} ) 
		\item Generative complexity: \citet{Garzonio2016}  
	\end{enumerate}
\end{enumerate}
\item Ontological modes
	\begin{enumerate}[label=\Alph*.]
	\item Compositional complexity
	\begin{enumerate}[label=\alph*.]
		\item Constitutional complexity: \citet{Jakobson1929}, \citet{McWhorter2001}, \citet{HayBauer2007}, \citet{Nichols2016}, \citet{Schreier2016} 
		\item Taxonomic complexity (or heterogeneity): \citet{McWhorter2001}, \citet{Kusters2003}, \citet{Sinnemäki2009}, \citet{SzmrecsanyiKortmann2009}, \citet{MaitzNémeth2014}, \citet{Nichols2016}, \citet{Schreier2016}  
	\end{enumerate}
	\item Structural complexity
	\begin{enumerate}[label=\alph*.]
		\item Organizational complexity: transparency/opacity: \citet{Braunmüller1984}, \citet{McWhorter2001}, \citet{Kusters2003}, \citet{Sinnemäki2009}, \citet{Nichols2016}; word order: \citet{McWhorter2001}, \citet{Nichols2016} 
		\item Hierarchical complexity: \citet{Sampson2001}  
	\end{enumerate}
\end{enumerate}
\item Functional complexity
	\begin{enumerate}[label=\Alph*.]
	\item Operational complexity: \citet{Kusters2003}, \citet{SzmrecsanyiKortmann2009} 
\end{enumerate}

\end{enumerate}}}
\end{figure}