\chapter{Zusammenfassung und Ausblick}\label{7}

Hier wird die gesamte Arbeit kurz zusammengefasst. Direkt auf die einzelnen zusammengefassten Resultate folgt ein Ausblick. \textbf{Ziel} der Arbeit ist, die absolute Komplexität in der Nominalflexion zu messen, und zwar in den folgenden Kategorien: \isi{Substantiv}, \isi{Adjektiv}, \isi{Interrogativpronomen}, \isi{Personalpronomen}, \isi{bestimmter Artikel}/ \isi{Demonstrativpronomen} (Det1) und \isi{unbestimmter Artikel}\slash Possessiv-\linebreak pronomen\is{Possessivpronomen} (Det2). Das untersuchte \textbf{Sample} besteht aus Althochdeutsch, Mittelhochdeutsch, der deutschen Standardsprache sowie 17 alemannischen Dialekten aus dem höchstalemannischen, hochalemannischen, oberrheinalemannischen\linebreak und schwäbischen Gebiet. Eine \textbf{Methode} zur Messung der Komplexität in der Nominalflexion wird auf der Basis von \is{Lexical-Functional Grammar (LFG)}LFG und der in\-fe\-ren\-tiel\-len-re\-a\-li\-sie\-ren\-den Morphologie (\citealt{Stump2001}, \citealt{AckermanStump2004}) entworfen. Dabei werden die Paradigmen ausschließlich durch \isi{Realisierungsregeln} definiert, und es gilt: Je mehr \isi{Realisierungsregeln} zur Definition der Paradigmen benötigt werden, desto komplexer ist die Flexionsmorphologie dieser Sprache. Mit dieser Methode sollte es prinzipiell möglich sein, die Komplexität in der Flexion jeder Sprache zu messen.\\

% \section{Fragestellung, Resultate und Ausblick}

\section{Diachronie} Verliert die Flexionsmorphologie diachron an Komplexität oder wird Komplexität aufgebaut? Sind Komplexitätsausgleiche zu beobachten (\is{Equi-Complexity-Hypothese}\textit{Equi-Com\-ple\-xi\-ty-Hy\-po\-the\-se})? Vergleicht man die modernen Varietäten mit ihrem diachronen Pendant, so ist Simplifizierung in den folgenden Kategorien festzustellen: Gesamtkomplexität, Substantiv, Adjektiv und Interrogativpronomen. Aber auch Komplexifizierung kann beobachtet werden, und zwar in den Kategorien Personalpronomen, Det1 und Det2, jedoch nur in einigen alemannischen Dialekten, nicht in der deutschen Standardsprache. In diesen Kategorien werden morphologische und morphosyntaktische Eigenschaften neu in der Flexion markiert. Schließlich kann keine Evidenz dafür gefunden werden, dass Komplexitätsausgleiche als allgemeine Tendenz in der Nominalflexion wirken. Komplexifizierung und Simplifizierung kommen in den unterschiedlichen Kategorien und Varietäten in allen möglichen Kombinationen vor.

Das sehr unterschiedliche diachrone Verhalten der Kategorien wirft viele neue Fragen auf, die hier kurz diskutiert werden sollen. Die folgenden Ausführungen erheben keinen Anspruch auf Vollständigkeit, sondern sollen nur erste Beobachtungen und Fragen diskutieren. Komplexifizierung ist nur im \isi{Personalpronomen}, Det1 sowie Det2 zu beobachten, und zwar werden morphologische, morphosyntaktische und morphosemantische Eigenschaften neu in der Flexion markiert.

\begin{description}
\item[Syntaktisch:] \isi{Betontheit} im \isi{Personalpronomen} und unterschiedliche Artikelform je nach syntaktischer Umgebung. Beides ist wahrscheinlich ursprünglich phonotaktisch bedingt.
\item[Semantisch:] \isi{Belebtheit} im \isi{Personalpronomen}, zusammengesetztes \isi{Personalpronomen} im Plural (= \isi{Additive Borrowing} in Issime).
\item[Morphologisch:] Klare Unterscheidung der Flexion der Wortarten \isi{unbestimmter Artikel} vs. \isi{Possessivpronomen} und \isi{bestimmter Artikel} vs. \isi{Demonstrativpronomen}, unterschiedliche Flexion je nach \isi{Possessivpronomen}.
\end{description}

Der Abbau von Komplexität ist in den Kategorien \isi{Substantiv} und \isi{Adjektiv} zu beobachten. Dies betrifft vor allem die Kasusmarkierung, wobei der Abbau der Kasusmarkierung auch in den anderen Kategorien (\isi{Personalpronomen}, Det1 und Det2) festgestellt wird. Bezüglich des \isi{Kasus} kann Folgendes festgehalten werden:

\begin{longdescription}
\item[Wegfall des \textbf{Genitivs} in den alemannischen Dialekten (außer isoliertes Höchstalemannisch):] Die Funktionen des Genitivs (Objekt, Partitiv, Possessiv) werden oft durch den Dativ oder Präpositionalkasus (meistens Präpositionen mit Dativ) ersetzt.
\item[Nominativ-Akkusativ-\isi{Synkretismen}:] Die \isi{Synkretismen} zwischen Nominativ und Akkusativ sind in allen Wortarten (außer \isi{Personalpronomen}) und allen Varietäten zu beobachten. Dieser \isi{Synkretismus} kommt schon im Althochdeutschen vor und ist im althochdeutschen \isi{Substantiv} die Regel (mit Ausnahme der schwachen Flexion). Es sei hier darauf hingewiesen, dass die (nord-)niederdeutschen Dialekte bei einer ähnlichen Ausgangssituation\linebreak wie die hochdeutschen Dialekte (4 \isi{Kasus} + Reste eines Instrumentals im Altsächsischen) einen anderen \isi{Synkretismus} zeigen, nämlich einen Zusammenfall von Akkusativ und Dativ (Genitiv ist ebenfalls weggefallen) \citep[66]{Berg2014}.
\item[Erhalt des Dativs:] Der Dativ ist gut erhalten in allen Varietäten und Wortarten (außer am \isi{Substantiv} in den meisten alemannischen Dialekten, s. nächster Punkt). In vielen alemannischen Varietäten ist der Dativ zusätzlich durch eine Präposition markiert (präpositionale Dativmarkierung, vgl. \citet{Seiler2003}).
\item[Vollständiger Verlust der Kasusmarkierung in der Substantivflexion (mit sehr wenigen Resten)]: Dies ist der Fall in allen alemannischen \isi{Dialektgruppen} außer im isolierten Höchstalemannisch, in der deutschen Standardsprache und im Mittelhochdeutschen.
\item[Abbau von Kasusallomorphie in der Substantivflexion:] Betrifft das isolierte Höchstalemannisch, die deutsche Standardsprache und Mittelhochdeutsch. Im Gegensatz zur Kasusallomorphie ist kaum oder keine Reduktion von Numerusallomorphie festzustellen (vgl. \tabref{table6.21}, \sectref{6.4.2}). Der Abbau von Kasusallomorphie bedeutet gleichzeitig die Reduktion der Anzahl von \isi{Flexionsklassen}. Davon sind die deutsche Standardsprache und Mittelhochdeutsch stärker betroffen als die isolierten alemannischen Dialekte, da der Abbau an Kasusallomorphie in diesen beiden Varietäten auch stärker ausfällt als in den isolierten alemannischen Dialekten. Hier ist zu berücksichtigen, dass die Reduktion von Kasusallomorphie im Mittelhochdeutschen und in der deutschen Standardsprache durch phonologischen Wandel (Nebensilbenschwächung) erklärt werden kann, jedoch nicht ausschließlich (z.\,B.\ Althochdeutsch \textit{zung\=un} Akk.Sg.+Pl. > Mittelhochdeutsch \textit{zungə} Akk.Sg., \textit{zungən} Akk.Pl.). Im Gegensatz dazu kann für das isolierte Höchstalemannisch nicht mit der Nebensilbenschwächung argumentiert werden, da Vollvokale im Nebenton erhalten sind. Die Reduktion von Kasusallomorphie führt also dazu, dass die Substantivflexion transparenter wird. Eine weitere Konsequenz betrifft das \isi{Genus}. Im Althochdeutschen gab es einen gewissen Zusammenhang zwischen \isi{Flexionsklasse} und \isi{Genus}. Dieser ist mit der Reduktion von Kasusallomorphie (und somit auch an \isi{Flexionsklassen}) in den modernen Varietäten klar weniger oder nicht mehr ersichtlich.
\end{longdescription}

Wir können also festhalten, dass der Kopf der DP nicht nur vorwiegend die Kasusmarkierung (wenn auch nicht allein und nicht immer) trägt, sondern auch zusätzlich neue Funktionen erhalten hat. Beim \isi{Personalpronomen} handelt es sich dabei um den Kopf der DP. Im Adjunkt der NP und besonders im Kopf der NP wird Markierung abgebaut, und zwar vor allem die Kasusmarkierung. Betrachtet man also die DP (syntagmatisch), wird Redundanz in der Kasusmarkierung reduziert. Des Weiteren wird durch die Reduktion von Kasusallomorphie (paradigmatisch) das System transparenter. Auf den ersten Blick hat der Wandel in der Flexion also zu weniger Redundanzen und höherer Transparenz geführt. Auf den zweiten Blick jedoch haben wir es hier mit viel komplexeren Vorgängen zu tun.

Erstens ist die Markierung des \textbf{Dativs} in den Determinierern und \isi{Adjektiven} (stärker in der starken als schwachen Flexion) erhalten, d.h., er wird weiterhin redundant markiert. Außerdem wird in vielen alemannischen Dialekten der Dativ zusätzlich mit einer Präposition markiert (präpositionale Dativmarkierung, vgl. \citealt{Seiler2003}). Redundanz wird also aufgebaut.

Zweitens ist der \isi{Synkretismus} zwischen \textbf{Nominativ und Akkusativ} in den meisten der hier untersuchten Varietäten in allen Wortarten vorhanden. Interessanterweise gerade auch im Höchstalemannischen, das Vollvokale im Nebenton hat. Der Zusammenfall von Nominativ und Akkusativ ist folglich nicht nur ein Nebeneffekt der Nebensilbenschwächung, sondern auch rein morphologischer Wandel spielt eine Rolle. Das heißt weiter, dass der \isi{Synkretismus} zwischen Nominativ und Akkusativ im \isi{Substantiv} nicht durch die Unterscheidung dieser \isi{Kasus} z.\,B.\ im Artikel kompensiert wird, da auch im Artikel die beiden \isi{Kasus} zusammenfallen sind. Es stellt sich also die Frage, ob dieser \isi{Synkretismus} z.\,B.\ durch die Wortstellung kompensiert wird. Jedoch ist sowohl in der deutschen Standardsprache als auch in den alemannischen Dialekten die Wortstellung nicht fest, d.h., auch ambige Sätze kommen vor (auch wenn es eine Tendenz gibt, dass in solchen Fällen das Subjekt dem Objekt vorangeht, vgl. u.a. \citealt{Ellsäßer2015}; zur festeren Abfolge in niederdeutschen Dialekten wohl infolge des Akkusativ-Dativ-\isi{Synkretismus} vgl. \citealt[320-321]{Berg2013}). Eine weitere Möglichkeit, das Subjekt vom Akkusativobjekt zu unterscheiden, ist durch Kongruenz mit dem Verb. Jedoch sind dann ambige Sätze immer noch möglich. Wie genau das System der Nominativ- und Akkusativunterscheidung bzw. der Subjekt- und Akkusativobjektunterscheidung funktioniert, ist folglich nicht klar und müsste im Detail und vergleichend untersucht werden. Neben einer synchronen Studie bietet sich hier vor allem eine diachrone Studie an. Es wäre ganz grundsätzlich zu überprüfen, ob und wie der Abbau in der Flexion durch eine festere Abfolge kompensiert wird. \citet{Allen2006} beispielsweise schlägt für das Englische vor, den Wandel nicht als eine Einbahnstraße zu sehen, sondern als zwei Wandel, die sich gegenseitig bedingen und beeinflussen: Der Abbau in der Flexion scheint eine festere Abfolge zu fördern, welche es wiederum erlaubt, die Flexion weiter abzubauen etc. \citep[209-216]{Allen2006}.

Während der Abbau der Akkusativmarkierung also in der Flexion nicht kompensiert wird (jedoch teils durch syntaktische Mittel), wird der Abbau der Dativmarkierung (am \isi{Substantiv}) stets kompensiert, und oft ist der Dativ sogar redundant markiert (Determinierer, \isi{Adjektive}, Präposition). Eine mögliche, eher funktionalistische Erklärung dafür wäre, dass das Akkusativobjekt durch syntaktische Mittel markiert wird, der Dativ jedoch nicht (mit seinen sehr unterschiedlichen Funktionen), weshalb dieser also durch Flexion von Nominativ und Akkusativ unterschieden werden muss. Dies erklärt jedoch noch lange nicht, wieso der Dativ redundant markiert wird (diachron erhaltene Dativmarkierung am Determinierer und \isi{Adjektiv} + Ausbau durch Präposition). Man betrachte in diesem Zusammenhang zum Beispiel die schwache Adjektivflexion im Singular von vielen alemannischen Dialekten, die die Opposition Nominativ/Akkusativ vs. Dativ durch das minimal mögliche Mittel kodieren, nämlich -ø vs. -ə. Die Markierung des Dativs erinnert an ein Zitat von \citet[31]{Gil2009}: „One cannot but wonder what all this complexity is for”. Sowohl aus synchroner als auch aus diachroner Sicht bleiben folglich unzählige Fragen bezüglich des Zusammenspiels von \isi{Kasus}, syntaktischen Funktionen, Theta-Rollen und Wortfolge bezüglich des Deutschen offen. Besonders eine Untersuchung der höchstalemannischen Dialekte bietet sich hier an, da diese untereinander relativ große Unterschiede aufweisen und Vollvokale im Nebenton haben (vgl. unten zur \isi{Isolation}). Auch wäre zu prüfen, ob sich ein Vergleich der höchstalemannischen Dialekte mit anderen stark flektierenden Sprachen (z.\,B.\ Isländisch, Färöisch, Övdalisch) lohnt.

Drittens kann gezeigt werden, dass der Link zwischen \textbf{Genus} und \isi{Flexionsklasse} gekappt wurde oder zumindest nicht mehr klar ist, was wohl gerade auch durch die Reduktion von Kasusallomorphie verursacht wurde. Damit stellt sich die Frage, weshalb \isi{Genus} nicht auch abgebaut wurde bzw. welche Funktion das Drei-Genus-System hat, das in allen hier untersuchten Varietäten erhalten ist. Dazu sollen hier nur einige Hypothesen aufgeführt werden. \isi{Genus} dient der Kongruenz innerhalb der DP. Somit sind auch sehr komplexe DPs möglich, wie z.\,B.\ \textit{der gestern am frühen Nachmittag vor dem Supermarkt von einem angetrunkenen Autofahrer verursachte sehr} \textit{schlimme Unfall}. In anderen germanischen Sprachen, wie z.\,B.\ im Englischen und im Norwegischen (Bokmål), ist \isi{Genus} entweder verloren oder stark reduziert. In diesen Sprachen sind solche komplexen DPs nicht möglich und müssen mit einem restriktiven Relativsatz übersetzt werden. Es sei hier jedoch darauf hingewiesen, dass auch in den alemannischen Dialekten komplexe DPs wie in der deutschen Standardsprache nicht möglich sind und durch restriktive Relativsätze ersetzt werden, obwohl das Drei-Genus-System erhalten ist. Des Weiteren hat \isi{Genus} auch eine deiktische Funktion. \citet{LosvanKemenadeerscheint} untersuchen den Rückgang der \isi{Demonstrativpronomen} in Spec,CP (=Vorfeld) vom Alt- zum Mittelenglischen. Kurz zusammengefasst zeigen sie u.a., dass der Genusverlust und die referentielle Funktion des deiktischen Systems im Englischen zusammenhängen. Wenn \isi{Genus} und somit Kongruenz verloren gehen, ist es in parataktischen Verbindungen schwieriger oder sogar unmöglich zu wissen, auf welche dem \isi{Demonstrativpronomen} vorangehende Phrase sich das \isi{Demonstrativpronomen} bezieht. Außerdem verliert das Vorfeld seine Funktion als diskursverbindende Position. Kompensiert wird dieser Verlust durch die Verwendung von \isi{Personalpronomen} bei Topikkontinuität und Relativpronomen bei Topikwechsel \citep[13-17]{LosvanKemenadeerscheint}. Im Deutschen hingegen sind \isi{Genus}, \isi{Demonstrativpronomen} im Vorfeld wie auch das Vorfeld als diskursverbindende Position erhalten.\\

\section{Dialektgruppen} Gibt es Gruppen von mehr oder weniger komplexen Dialekten, die der traditionellen Einteilung der alemannischen Dialekte entsprechen? Dies kann sowohl aus synchroner wie auch aus diachroner Sicht bestätigt werden. Synchron lässt sich allgemein feststellen, dass höchstalemannische Dialekte komplexer sind als hochalemannische, hochalemannische komplexer als schwäbische und schwäbische komplexer als oberrheinalemannische (mit ganz einzelnen Ausnahmen im Schwäbischen). Aus diachroner Perspektive sind zwei Resultate wichtig: Erstens wird gezählt, wie viele Dialekte welcher Dialektgruppe eine höhere Komplexität aufweisen als ihr diachrones Pendant. Dabei zeigt sich dieselbe Abfolge wie beim synchronen Vergleich: Die größte Anzahl von Dialekten mit diachronem Komplexitätsaufbau gibt es im Höchstalemannischen, gefolgt von Hochalemannisch, Schwäbisch und Oberrheinalemannisch. Zweitens resultiert auch aus der Innovationsrate (durchschnittliche Anzahl Innovationen in den Kategorien Personalpronomen, Det1 und Det2 pro Dialektgruppe) dieselbe Abfolge an innovativeren und weniger innovativen Dialektgruppen.\\

\section{Kontakt} Höhere Komplexität wird in lange währenden, koterritorialen Kontaktsituationen erwartet, in denen Kinder zwei- bzw. mehrsprachig aufwachsen. Denn in diesen Kontaktsituationen treten \textit{Additive Borrowings} auf. Mit einer niedrigeren linguistischen Komplexität ist in großen Sprachgemeinschaften mit losen Netzwerken, vielen Kontakten und vielen L2-Ler\-nern zu rechnen \citep{Trudgill2011}.

Ein \textit{Additive Borrowing} wird nur im Dialekt von Issime gefunden, jedoch weder in den elsässischen Dialekten noch in den Sprachinseln Petrifeld und Elisabethtal. Es zeigt sich, dass Issime die soziolinguistisch optimalen Voraussetzungen für \textit{Additive Borrowings} hat. Die Interpretation der anderen in Frage kommenden Dialekte ist schwierig, da synchrone und/oder diachrone Daten zum soziolinguistischen Kontext und zur Kompetenz in den Kontaktsprachen nicht gefunden werden können. Um genauere Aussagen machen zu können, bräuchte es folglich eine Studie, die vor allem unterschiedliche Sprachinseln vergleicht, für die die genannten Informationen vorhanden sind.

Zur Überprüfung der Hypothese der zweiten Kontaktsituation wird die Komplexität von Stadtdialekten mit jener von Landdialekten verglichen. Die Stadtdialekte von Colmar, Stuttgart und Zürich (jedoch nicht jener von Bern) sind fast in jeder Kategorie weniger komplex als ihre Landdialekte. Eine mögliche Erklärung dafür ist, dass in der Schweiz die Standardsprache nicht als Lingua Franca dient, diese also die Dialekte deutlich weniger beeinflusst und somit Stadtdialekte keine Inseln bilden, die sprachlich ausstrahlen. Um also die linguistische Rolle von Schweizer Städten besser zu verstehen, müssten Stadt- und Landdialekte innerhalb desselben Gebiets verglichen werden. Damit diese Gebiete bestimmt werden können, sollten z.\,B.\ geostatistische Analysen herangezogen werden.\\

\section{Standardvarietät} Da Standardsprachen in besonderem Maße von großen Sprachgemeinschaften mit losen Netzwerken, vielen Kontakten und L2-Ler\-nern gesprochen werden, ist für die deutsche Standardsprache eine niedrigere Komplexität zu erwarten \citep{Trudgill2011}. Des Weiteren kann die Vereinheitlichung während eines Standardisierungsprozesses zu Vereinfachungen führen, also zu niedrigerer Komplexität. Jedoch ist auch vorstellbar, dass in Standardsprachen Komplexität durch die Kodifikation besser erhalten werden kann.

Eine vergleichsweise hohe Komplexität hat die deutsche Standardsprache im \isi{Substantiv}, \isi{Adjektiv} und \isi{Interrogativpronomen}. Dies kann durch den partiellen Erhalt der Kasusmarkierung am \isi{Substantiv} und durch den Erhalt des Genitivs erklärt werden. Die geringste Komplexität von allen untersuchten Varietäten zeigt die deutsche Standardsprache in den Kategorien Det1, Det2 und \isi{Personalpronomen} sowie in der Gesamtkomplexität. Die Ursache dafür liegt vorwiegend darin, dass die deutsche Standardsprache keine komplexitätsaufbauenden \isi{Innovationen} wie die alemannischen Dialekte aufweist (mit Ausnahme der Grammatikalisierung des Artikels). Auch aus diachroner Perspektive lässt sich dasselbe beobachten, da die deutsche Standardsprache stets eine geringere Komplexität als Mittelhochdeutsch aufweist. Die deutsche Standardsprache funktioniert also in mannigfacher Hinsicht anders als die alemannischen Dialekte. Daraus ergibt sich eine wichtige Konsequenz für die Beschäftigung mit dem Sprachwandel im Deutschen: Zur Erforschung des Sprachwandels im Deutschen sollten immer (zumindest auch) Dialekte berücksichtigt werden. Denn nur so erhält man valide Ergebnisse zum Sprachwandel, die in der Theoriebildung berücksichtigt werden können. Dies kann dadurch begründet werden, dass bei der Standardisierung weitere sprachexterne Faktoren die Sprache beeinflussen, welche in den Dialekten keine Rolle spielen. Beispielsweise ist die Standardsprache präskriptiven Eingriffen ausgesetzt, wie z.\,B.\, dass nur die einfache Verneinung normiert wurde, während die doppelte Verneinung in vielen deutschen Dialekten durchaus üblich ist. Auch aus der Phonologie gibt es Nachweise, dass die deutsche Standardsprache gesondert betrachtet und Dialekte in Studien zum Sprachwandel herangezogen werden sollten. \citet{PröllKleiner2016} zeigen, dass die regionale \isi{Variation} des Glottisverschluss an der Silbengrenze im gesprochenen Gebrauchsstandard auf die „Konkurrenz verschiedener Silbenmodelle der ansässigen Basisdialekte und des niederdeutsch beeinflussten Standards“ zurückgeführt werden kann \citep[211]{PröllKleiner2016}. Denn die hochdeutsche Standardsprache verfügt über das phonologisch-prosodische System aus seinem niederdeutschen Substrat (Kontaktphänomen), was in der Entwicklung der deutschen Standardsprache zu einem „typologische[n] Bruch“ (von einer Silben- zu einer Wortsprache) geführt hat \citep[210]{PröllKleiner2016}. Eine Kontinuität zwischen Althochdeutsch und der deutschen Standardsprache ist folglich fragwürdig (vgl. auch Fußnote 23 in \citealt[210]{PröllKleiner2016}).\\

\section{Isolation} Es wird davon ausgegangen, dass in isolierten Varietäten ererbte Komplexität stärker erhalten wird, weil der Sprachwandel langsamer vonstattengeht. Gleichzeitig ist aber auch spontane Komplexifizierung zu beobachten, d.h. Sprachwandel, der zu höherer Komplexität führt. Unter Isolation wird hier nicht nur soziale Isolation (vgl. Kontakt), sondern auch geografische Isolation verstanden.

Vergleicht man die Komplexität der untersuchten Kategorien und der Gesamtkomplexität von isolierten und nicht isolierten Dialekten, ist die \isi{Variation} sehr groß. Ein deutlicheres Bild zeigt sich, wenn die Mechanismen +/−Kom\-ple\-xi\-fi\-zie\-rung und +/−Sim\-pli\-fi\-zie\-rung genauer betrachtet werden. Diesbezüglich werden für die Kategorien \isi{Substantiv} und \isi{Adjektiv} einerseits und für die Kategorien \isi{Personalpronomen}, Det1 und Det2 andererseits grundsätzlich unterschiedliche Ergebnisse gefunden. In den Kategorien \isi{Personalpronomen}, Det1 und Det2 sind die isolierten Dialekte weniger betroffen vom Sprachwandel, also sowohl von der Komplexifizierung als auch von der Simplifizierung. Folglich werden \isi{Archaismen} mit höherer Komplexität stärker erhalten. Dies bedeutet jedoch nicht, dass Komplexifizierung und Simplifizierung nicht auch in isolierten Dialekten vorkommen, jedoch in kleinerem Ausmaß als in den nicht isolierten Dialekten. Im Gegensatz dazu kann in den Kategorien \isi{Adjektiv} und \isi{Substantiv} kein Zusammenhang zwischen Komplexifizierung/Simplifizierung und einem Typ von Sprachgemeinschaft gefunden werden. Außerdem schließen sich Komplexifizierung und Simplifizierung weder ein noch aus.

Hier fällt also auf, dass Zahlen durchaus aufschlussreich sein können, da man mit diesen das sprachliche Material wie auch einzelne Phänomene gut in den Griff bekommen kann. Mit einem detaillierteren Blick auf die unterschiedlichen, sich auch widersprechenden Mechanismen in einem Sprachsystem erhöht sich jedoch der Erkenntnisgewinn deutlich. Gerade die Kombination der beiden Herangehensweisen erwies sich hier als besonders geeignet, um das Sprachsystem besser zu verstehen.

Schließlich stellt sich hier noch die Frage, ob das Höchstalemannische nicht als Ganzes als isoliert gelten müsste, wenn man dieses Gebiet mit den anderen alemannischen Gebieten vergleicht. Denn es handelt sich dabei um ein eher kleines und ausschließlich alpines Gebiet, das nach außen klar abgegrenzt ist, und zwar entweder durch Sprach- oder Konfessionsgrenzen. Dies wäre eine Erklärung dafür, weshalb der Unterschied in der durchschnittlichen Gesamtkomplexität zwischen den höchstalemannischen (isoliert und nicht isoliert) und den hochalemannischen Dialekten deutlich höher ausfällt als zwischen den hochalemannischen und den schwäbischen sowie zwischen den schwäbischen und den oberrheinalemannischen Dialekten.

Das Höchstalemannische ist jedoch noch in einer weiteren Hinsicht besonders interessant. Bezogen auf die diachronen Verlaufsmuster (+/−Sim\-pli\-fi\-zie\-rung und +/−Kom\-ple\-xi\-fi\-zie\-rung in den Kategorien \isi{Personalpronomen}, Det1 und Det2) können die hier untersuchten höchstalemannischen Dialekte in drei Gruppen geteilt werden:

\begin{longdescription}
\item[+Simplifizierung/+Komplexifizierung in den Dialekten Uri und Sensebezirk:] Abbau\linebreak von Komplexität (vor allem \isi{Kasus} und besonders im \isi{Substantiv}) und sehr innovativ (d.h. besonders viele komplexitätsaufbauende \isi{Innovationen}).
\item[-Simplifizierung/-Komplexifizierung in den Dialekten Issime und Visperterminen (Walser):] wenig Abbau von ererbter Komplexität und wenig komplexitätsaufbauende \isi{Innovationen}.
\item[-Simplifizierung/+Komplexifizierung im Dialekt Jaun:] wenig Abbau von ererbter\linebreak Komplexität und viele komplexitätsaufbauende \isi{Innovationen}.
\end{longdescription}


\citet{Trudgill2011} geht davon aus, dass isolierte Varietäten ererbte Komplexität stärker erhalten und dass sie auch Komplexität aufbauen (spontane Komplexifizierung). Wenn wir das höchstalemannische Gebiet als Ganzes als isoliert betrachten, ist es besonders aufschlussreich, dass alle Möglichkeiten, die Trudgill annimmt, im Höchstalemannischen auch zu finden sind: Erhalt ererbter Komplexität (Issime und Visperterminen), spontane Komplexifizierung (Sensebezirk und Uri) sowie die Kombination von beiden, also Erhalt ererbter Komplexität und spontane Komplexifizierung (Jaun). Dies zeigt auch, was Trudgill zumindest nicht explizit anspricht, dass in isolierten Varietäten (davon ausgehend, dass das ganze höchstalemannische Gebiet als isoliert gilt) nicht unbedingt beide Mechanismen auftreten, aber mindestens einer von beiden.

Schließlich kann hier auch gezeigt werden, dass die unterschiedlichen Verlaufsmuster diachronen Wandels deutlich aufschlussreicher und aussagekräftiger insbesondere für die soziolinguistische Typologie sind als die synchronen Komplexitätswerte, worauf auch \citet{Trudgill2011} immer wieder hinweist. Mit synchronen Komplexitätswerten beschäftigen sich vor allem die Typologie und die eher theoretisch orientierte Morphologie, wie z.\,B.\ \citet{Sinnemäki2011} und \citet{Camilleri2012}, um je nur ein Beispiel zu nennen. Die Analyse und der Vergleich des Wandels in der Komplexität von verschiedenen Varietäten (u.a. in Verbindung mit soziolinguistischen Faktoren) können die synchronen Komplexitätsunterschiede erklären, aber nicht umgekehrt. Es lohnt sich folglich gerade auch für die theoretische Morphologie, sich mit dem Wandel in der Komplexität zu beschäftigen.