\chapter{Einleitung}\label{1}

\epigraph{%
If it really were so that languages varied greatly in the complexity of subsystem \textit{X}, varied greatly in the complexity of subsystem \textit{Y}, and so on, yet for all languages the totals from the separate subsystems added together could be shown to come out the same, then I would not agree with Hockett in finding this unsurprising. To me it would feel almost like magic.\hfill\citep[2]{Sampson2009}}

\noindent Dieses Zitat illustriert, wie stark sich die Annahmen zur Komplexität in Sprachen zwischen dem 20. und dem 21. Jh. unterscheiden. Wenn man sich im 20. Jh. überhaupt mit eventuellen Komplexitätsunterschieden beschäftigt hat, ging man generell davon aus, dass unter dem Strich alle Sprachen gleich komplex sind (mit einigen Ausnahmen vor allem in der Variationslinguistik), was als \is{Equi-Complexity-Hypothese}\textit{Equi-Com\-ple\-xi\-ty-Hy\-po\-the\-se} bezeichnet wird. Dies kann sicher u.a. auch als Gegenreaktion auf Annahmen aus dem 19. Jh. interpretiert werden. Im 21. Jh. hingegen wurde die \is{Equi-Complexity-Hypothese}\textit{Equi-Com\-ple\-xi\-ty-Hy\-po\-the\-se} stark hinterfragt, woraus bereits zahlreiche Arbeiten vor allem aus der Typologie entstanden sind. Was jedoch bis jetzt fehlt, ist einerseits die Messung der Komplexität eines größeren Teilsystems und andererseits der Komplexitätsvergleich von eng verwandten Varietäten, die sich entweder synchron aus soziolinguistischer Perspektive oder diachron unterscheiden.

Eng verwandte Varietäten zu untersuchen, hat den Vorteil, dass diachrone Prozesse, wie z.\,B.\ Simplifizierung und Komplexifizierung, viel direkter betrachtet werden können \citep[22]{BaechlerSeiler2012}. Aufgrund der genetischen Verwandtschaft, die einen gemeinsamen historischen Ursprung impliziert, können diachrone Prozesse und ihre Wechselwirkungen innerhalb eines Systems A verglichen werden mit den diachronen Prozessen innerhalb eines anderen Systems B, das aber mit System A eng verwandt ist. Es stellt sich dann z.\,B.\ die Frage, weshalb zwei eng verwandte Varietäten trotz des gemeinsamen historischen Ursprungs (völlig) unterschiedliche Wege gehen.

Das Ziel dieser Arbeit ist, die absolute Komplexität der Nominalflexion in den folgenden Varietäten zu messen und zu vergleichen: Althochdeutsch, Mittelhochdeutsch, deutsche Standardsprache sowie siebzehn alemannische Dialekte aus dem höchstalemannischen, hochalemannischen, oberrheinalemannischen und schwäbischen Gebiet. Als Datengrundlage dienen \isi{Ortsgrammatiken}. Berücksichtigt wurden jene nominalen Wortarten, für die in allen zwanzig untersuchten Varietäten Beschreibungen existieren und die sich in ihrer Flexion unterscheiden: \isi{Substantiv}, \isi{Adjektiv}, \isi{Interrogativpronomen}, \isi{Personalpronomen}, einfaches \isi{Demonstrativpronomen}, \isi{Possessivpronomen}, bestimmter und \isi{unbestimmter Artikel}. Im Gegensatz zur Verbalflexion können also verschiedene Wortarten miteinander verglichen werden. Auch die Variationsbreite zwischen den Varietäten ist in der Nominalflexion grösser.

Die theoretischen Grundlagen bilden LFG (\is{Lexical-Functional Grammar (LFG)}Lexical-Functional Grammar) und die in\-fe\-ren\-tiel\-le-re\-a\-li\-sie\-ren\-de Morphologie. Aus diesen Modellen wird abgeleitet, was ein System komplexer bzw. simpler macht. Darauf auch die eigens entwickelte Methode zur Messung der Komplexität. Bei der Entwicklung der Messmethode standen drei Qualitätsmerkmale im Vordergrund: a) Die Messmethode muss so objektiv wie möglich sein; b) Die Messmethode muss auch kleinste Unterschiede von eng verwandten Varietäten messen können; c) Die Messmethode muss prinzipiell auf alle flektierenden Sprachen angewendet werden können. Im hier verwendeten Modell wird davon ausgegangen, dass Flexionsparadigmen durch sogenannte \isi{Realisierungsregeln} definiert sind. Folglich gilt, je mehr \isi{Realisierungsregeln} zur Definition der Flexionsparadigmen benötigt werden, desto komplexer ist das Flexionssystem. Schließlich soll überprüft werden, ob, und wenn ja, wie folgende Faktoren die \isi{Variation} in der Komplexität der Nominalflexion erklären können:

\begin{itemize}
\item 
Nimmt die Komplexität der Nominalflexion im Laufe der Zeit zu oder ab? Mit welchen soziolinguistischen Faktoren kann die Zu- oder Abnahme in Zusammenhang gebracht werden?
\item 
Unterscheiden sich die alemannischen \isi{Dialektgruppen} (Höchst-, Hoch-, Oberrheinalemannisch, Schwäbisch) in ihrer Komplexität?
\item 
Wie beeinflusst der Kontakt mit anderen Dialekten oder anderen Sprachen die Komplexität?
\item 
Wie beeinflussen die Standardisierung und die Kodifizierung die Komplexität?
\item 
Führt \isi{Isolation} (wenig Kontakt und geografische \isi{Isolation}) zu einer höheren oder niedrigeren Komplexität?
\end{itemize}

\noindent
Die vorliegende Arbeit ist wie folgt strukturiert. \chapref{2} gibt einen wissenschaftsgeschichtlichen Überblick über die Beschäftigung mit der linguistischen Komplexität (\sectref{2.1}) sowie über etliche Definitionen, Einflussfaktoren und Messmethoden, die zur linguistischen Komplexität vorgeschlagen und getestet wurden (\sectref{2.2}). \chapref{3} beantwortet die Fragen, weshalb hier gerade die Nominalflexion untersucht wird und was unter Komplexität verstanden wird (\sectref{3.1}). Des Weiteren werden die Hypothesen (\sectref{3.2}) und die untersuchten Varietäten (\sectref{3.3}) eingeführt. Im \chapref{4} werden die theoretischen Grundlagen vorgestellt (\sectref{4.1}), was ein System komplexer oder simpler macht (\sectref{4.2}) sowie die Methode zur Messung der absoluten Komplexität (\sectref{4.3}). \chapref{5} analysiert die Flexionsparadigmen und \isi{Realisierungsregeln} der hier untersuchten Varietäten. In \chapref{6} werden die Resultate vorgestellt, interpretiert und dadurch die Hypothesen überprüft. In \chapref{7} wird die vorliegende Arbeit kurz zusammengefasst, und zwar kombiniert mit einem Ausblick.

Außerdem folgen hier noch einige Anmerkungen zur Transkription und Verschriftlichung der untersuchten Varietäten. Die Daten basieren auf unterschiedlichen Ortgrammatiken und Beschreibungen der Varietäten, die verschiedene Transkriptionen (z.\,B.\ Teuthonista, Dieth etc.) verwenden und unterschiedlich präzise sind. Da diese Arbeit sich mit der nominalen Flexionsmorphologie beschäftigt, muss nur bezüglich der Flexionsaffixe auf phonetische Genauigkeit geachtet werden. Der einfacheren Lesbarkeit halber wird hier der Pho\-nem-Gra\-phem-Ent\-spre\-chung der deutschen Rechtschreibung gefolgt. Ausnahmen bilden Langvokale besonders in den alemannischen Dialekten, die mit einem Strich über dem betreffenden Vokal verschriftlicht werden (z.\,B.\ \=a). Des Weiteren gelten andere Regeln bezüglich der (vor allem auslautenden) e-Laute und a-Laute. Da gerade die untersuchten Dialekte im Auslaut unterschiedliche a- und e-Qualitäten unterscheiden, die verschiedene morphosyntaktische Eigenschaften kodieren, wird hier IPA gefolgt: [e], [ɛ], [æ], [a], [ɑ], [ɐ], [ə]. Aufgrund der unterschiedlichen Angaben in den \isi{Ortsgrammatiken} zur Lautqualität ist mit Ungenauigkeiten zu rechnen. Es wurde in dieser Arbeit jedoch darauf geachtet, dass, wenn in der \isi{Ortsgrammatik} zwei \isi{Affixe} unterschieden werden, dieser Unterschied auch in den Paradigmen aufgenommen wird. Dabei spielt die genaue Lautqualität der \isi{Affixe} keine ausschlaggebende Rolle, da hier die Flexionsmorphologie untersucht wird.

Bei der vorliegenden Monographie handelt es sich um eine überarbeitete Version meiner Dissertation, die von Prof.\ Dr.\ Guido Seiler und Prof.\ Dr.\ Martin Joachim Kümmel betreut und von der Albert-Ludwigs-Universität Freiburg angenommen wurde (promoviert am 23. Mai 2016). Schließlich möchte ich mich bei den folgenden Personen für ihre Unterstützung und die zahlreichen anregenden Diskussionen bedanken (alphabetisch gelistet): Ulrike Ackermann, Marco Angster, Manuela Baechler, Pia Bergmann, Antje Dammel, Jacopo Garzonio, Nikolay Khakimov, Martin Joachim Kümmel, Adriano Murelli, Johanna Nichols, Harald Noth, Simon Prentice, Simon Pröll, Javier Caro Reina, Lea Schäfer, Oliver Schallert, Guido Seiler, Peter Trudgill und Thilo Weber. Ein herzliches Dankeschön geht des Weiteren an Felix Kopecky für seine Unterstützung bezüglich Latex sowie an die LektorInnen.