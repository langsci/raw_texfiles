%% hyphenation points for line breaks
%% Normally, automatic hyphenation in LaTeX is very good
%% If a word is mis-hyphenated, add it to this file
%%
%% add information to TeX file before \begin{document} with:
%% %% hyphenation points for line breaks
%% Normally, automatic hyphenation in LaTeX is very good
%% If a word is mis-hyphenated, add it to this file
%%
%% add information to TeX file before \begin{document} with:
%% %% hyphenation points for line breaks
%% Normally, automatic hyphenation in LaTeX is very good
%% If a word is mis-hyphenated, add it to this file
%%
%% add information to TeX file before \begin{document} with:
%% %% hyphenation points for line breaks
%% Normally, automatic hyphenation in LaTeX is very good
%% If a word is mis-hyphenated, add it to this file
%%
%% add information to TeX file before \begin{document} with:
%% \include{localhyphenation}
\hyphenation{
affri-ca-te
affri-ca-tes
an-no-tated
com-ple-ments
com-po-si-tio-na-li-ty
non-com-po-si-tio-na-li-ty
Gon-zá-lez
out-side
Ri-chárd
se-man-tics
STREU-SLE
Tie-de-mann
}
\hyphenation{
affri-ca-te
affri-ca-tes
an-no-tated
com-ple-ments
com-po-si-tio-na-li-ty
non-com-po-si-tio-na-li-ty
Gon-zá-lez
out-side
Ri-chárd
se-man-tics
STREU-SLE
Tie-de-mann
}
\hyphenation{
affri-ca-te
affri-ca-tes
an-no-tated
com-ple-ments
com-po-si-tio-na-li-ty
non-com-po-si-tio-na-li-ty
Gon-zá-lez
out-side
Ri-chárd
se-man-tics
STREU-SLE
Tie-de-mann
}
\hyphenation{
affri-ca-te
affri-ca-tes
com-ple-ments
schwa-che
Va-ri-a-ti-on
Spra-chen
Spra-che
ge-ne-rell
Pro-zes-se
mit-ei-nan-der
die-sen
Re-a-li-sie-rungs-re-geln
schließ-lich
hö-he-ren
wis-sen-schafts-ge-schicht-li-chen
be-schäf-tigt
be-züg-lich
Un-ter-stüt-zung
Dam-mel
in-for-ma-ti-ons-the-o-re-ti-sches
Re-dun-dan-zen
ver-schie-den-en
er-mög-li-chen
hie-si-gen
vor-ge-stellt
Fak-to-ren
Ge-schlecht
Sprach-ge-mein-schaft
Sprach-ge-mein-schaf-ten
Ü-ber-set-zung
ge-prüft
un-ter-schied-li-che
A-na-lo-gi-en
Be-völ-ke-rungs-grö-ße
gro-ße
ba-sie-end
Aus-schnitt
ge-o-gra-fi-sche
Zu-sam-men-ge-fasst
Mar-kie-rung
mor-pho-syn-tak-ti-sche
Phä-no-me-ne
feh-len-de
a-na-ly-siert
Step-pe
Mon-go-lisch
ein-ge-teilt
de-fi-niert
Cha-rak-te-ri-sti-ka
Rhein-ebe-ne
Pos-ses-siv-pro-no-men
Bei-spie-le
Ne-ben-sil-be
ge-meint
mög-li-che
mög-li-cher
Kom-ple-xi-tät
ei-ner
nach-ge-wie-sen
be-stimm-ten
durch-zu-füh-ren
Mit-tel-hoch-deutsch
Höchst-a-le-man-nisch
O-ber-rhein-a-le-man-nisch
mul-ti-lin-gu-a-len
Bo-den-see-a-le-man-nisch
Nie-der-a-le-man-nisch
Ein-tei-lung
Schwä-bisch
Hoch-a-le-man-nisch
wei-ter
auf-weist
so-wohl
ei-nem
täg-li-chen
struk-tu-rel-le
lang-sa-mer
be-ob-ach-tet
ge-o-gra-fisch
lie-gen
stär-ker
wel-che
Ka-pi-teln
Ü-ber-gang
Bän-de
früh-neu-hoch-deut-sche
münd-lich
Fle-xi-ons-mor-pho-lo-gie
ge-spro-chen
Ge-mein-de
Gur-mels
bern-deut-sches
nie-der-a-le-man-ni-schen
Schwarz-wald
Be-völ-ke-rungs-dich-te
un-ga-ri-schen
deut-schen
von-ei-nan-der
Rot-weil
aus-führ-li-che
zu-rück-er-o-bert
fran-zö-sisch
Be-schrei-bung
mar-kiert
A-djek-tiv-fle-xi-on
ne-ben
Be-ob-ach-tung
zu-ge-wie-sen
e-ven-tu-el-le
Par-ti-zip
The-o-rie-bil-dung
de-fi-nie-ren
Wör-ter
mög-li-chen
a-dä-qua-ter
da-für
un-ter-schie-den
Me-cha-nis-men
An-satz
pe-ri-phras-ti-scher
Mo-dell
aus-schließ-lich
Ver-bin-dung-en
be-schrie-ben
ent-spre-chen
Wohl-ge-formt-heit
ab-ge-kürzt
U-ni-fi-ka-tion
vor-kommt
Be-schrän-kung
Syn-kre-tis-men
Syn-kre-tis-mus
Bei-spiel
Mas-ku-lin
un-ter-schied-li-che
zu-sam-men-fal-len
Ge-nus-syn-kre-tis-men
Stan-dard-spra-che
be-schreibt
in-for-mel-le
Ab-lei-ten
Hu-zen-bach
zu-sam-men-ge-nom-men
be-nö-tigt
Ver-tei-lung
ver-wand-te
Zel-len
wei-sen
die-sel-be
be-züg-lich
Vor-schlä-ge
Fin-kel
Stamm-for-men
Sum-me
re-gel-mä-ßi-gen
prä-zi-se
Aus-gleichs-me-cha-nis-men
o-ben
struk-tu-rel-ler
mög-lichst
Ka-te-go-rie
ge-zählt
ein-heit-li-ches
De-mon-stra-tiv-pro-no-men
des-we-gen
Pri-mär
Pri-mär-um-laut
Se-kun-där-um-laut
ver-or-ten
Diph-thon-gie-rung
Ak-ku-sa-tiv
ge-ge-ben
be-ob-ach-ten
na-sa-lier-ten
steu-ert
Alt-hoch-deut-schen
Haupt-ar-gu-ment
Mit-tel-hoch-deut-schen
Wur-zel
Wur-zeln
mo-vier-ten
wich-tig
Blö-cke
Ka-sus-mar-kie-rung
Ei-gen-schaf-ten
die-ser
wür-de
be-reits
Ge-nus
die-sem
ein-ge-gan-gen
müs-sen
ge-spro-che-nen
be-schrie-ben
Pa-ra-dig-men
bei-spiels-wei-se
ha-ben
Per-so-nal-pro-no-men
Gram-ma-ti-ka-li-sie-rung
Pa-ra-dig-ma
han-delt
Ut-rum
Alt-hoch-deutsch
wei-sen
Ur-in-do-ger-ma-ni-sche
Be-lebt-heit
In-ter-ro-ga-tiv-pro-no-mens
ko-diert
De-mon-stra-tiv-pro-no-mens
Mit-tel-hoch-deut-sche
die-sel-be
schwä-bi-sche
so-wie
Ka-pi-tel
Sprach-in-sel
un-be-stimm-ten
Pos-ses-siv-pro-no-mens
Ge-ni-tiv
Kai-ser-stuhl
mor-pho-syn-tak-ti-schen
vor-kom-men
Re-geln
mor-pho-lo-gi-schen
ge-glie-dert
un-be-stimm-te
er-klärt
No-mi-nal-fle-xi-on
un-ge-fähr
Kom-ple-xi-fi-zie-rung
Sim-pli-fi-zie-rung
Sub-stan-tiv
we-ni-ger
Di-a-lekt-grup-pe
Neu-trum
be-züg-lich
vor-han-den
deut-sche
ge-wor-fen
Ne-ben-sil-ben-schwä-chung
Alt-hoch-deut-sche
gram-ma-ti-ka-li-sier-ten
Ta-bel-len
De-ter-mi-nie-rer-ka-te-go-rie-en
Ge-samt-kom-ple-xi-tät
o-ber-rhein-a-le-man-ni-sche
ge-hö-ren
di-a-chron
Ta-bel-le
Fle-xi-ons-pa-ra-dig-ma
be-zie-hen
bräuch-te
auf-wach-sen
ü-ber-neh-men-den
aus-führ-lich
Mehr-spra-chig-keit
ge-nau-er
schwä-bi-schen
di-a-chrone
ver-gleichs-wei-se
Fle-xi-ons-klas-se
an-ge-nom-men
i-so-lier-ten
i-so-liert
hö-he-re
wel-chen
Ge-gen-teil
ge-fragt
da-von
Sprach-wan-del
Un-ter-ka-pi-teln
un-ab-hän-gig
Hoch-a-le-man-ni-schen
O-ber-rhein-a-le-man-ni-schen
Schwä-bi-sche
nie-dri-ge-ren
hö-he-rem
Ver-deut-li-chung
ver-schie-de-nen
ge-zeigt
Ka-pi-tels
di-a-chro-ner
plä-diert
un-ter-schied-li-chem
da-ran
kei-nem
i-so-lier-te
Neu-e-run-gen
ge-gen-ü-ber-ge-stellt
ent-spre-chen
Stan-dar-di-sie-rungs-pro-zess
Dy-na-mi-ken
Auf-tre-ten
be-stimm-ter
Zu-sam-men-hang
hoch-deut-sche
ver-stan-den
Sen-se-be-zirk
ins-be-son-de-re
Spen-cer
Frei-burg
Fri-bourg
Gruy-ter
Mann-heim
con-straint
Stutt-gar-ter
bün-dig
sued-oes-tli-chen
Schlach-ter
Nie-mey-er
sprach-wis-sen-schaft-li-cher
Kir-chen-sla-vi-schen
Mund-art
Karl-Ham-pus
Sprach-ge-schich-te
% Wörter, die falsch getrennt sind
%Kap. 1
un-ter-such-ten
ge-folgt
%Kap. 2
zwi-schen
bes-ser
Jung-gram-ma-ti-kern
mesch-li-chen
syn-tak-ti-scher
In-te-res-ses
Ver-zer-ren
Aus-gleich
be-reits
ver-schie-de-ner
Ar-bei-ten
he-te-ro-ge-nen
zwi-schen
Ho-mo-ge-ni-tät
A-na-ly-se
Grenz-wert
hier
Netz-werk
Cha-rak-te-ris-ti-ka
E-le-men-te
un-ter-schied-li-chen
Un-ter-spe-zi-fi-ka-tion
bei-spiels-wei-se
Ver-gleichs
or-na-men-ta-le
zwei-tens
Aus-gleichs-ten-den-zen
Sprach-kon-takts
Fle-xi-ons-ka-te-go-ri-en
O-pa-zi-tät
be-tref-fend
re-la-ti-ver
on-to-lo-gi-sche
Schwer-punkt
ta-xo-no-mi-sche
kom-ple-xer
hie-rar-chi-schen
Hö-rer
ge-ne-ra-ti-ven
%Kap. 3
ger-ma-ni-schen
höchst-a-le-man-ni-sche
A-djek-tiv-fle-xi-on
kons-ti-tu-tio-nel-ler
Ka-te-go-ri-en
Ver-gleich
Ge-ni-tivs
be-stimm-ten
fünf-tens
ent-hält
so-zio-lin-gu-is-ti-schen
ge-spro-che-nen
Kon-tak-te
trenn-te
stell-ver-tre-tend
Pe-tri-feld
Wei-te-ren
el-säs-si-scher
Gebiet
%Kap. 4
spe-zi-fi-ziert
ge-nannt
leicht
Mo-du-le
fran-zö-si-schen
mo-del-liert
mor-pho-lo-gi-sche
aus-ei-nan-der-ge-hal-ten
her-zu-lei-ten
be-steht
ver-ein-fach-te
un-ter-schied-li-chen
Ver-gleich-bar-keit
ent-we-der
Ge-ni-tiv
ab-ge-lei-tet
sy-ste-ma-tisch
Stamm-mo-di-fi-ka-ti-on
Le-xems
je-der
kor-pus-ba-sier-te
Ope-ra-ti-o-na-li-sie-rung
Kom-pri-mie-rung
Mess-me-tho-de
klein-ste
Wort-ar-ten
konn-te
%Kap. 5
exem-pla-risch
prak-ti-schen
aus-schließ-lich
spe-zi-fisch-sten
A-le-man-ni-schen
wes-halb
Wur-zel-al-ter-na-ti-o-nen
durch-aus
ver-wer-fen
be-trach-tet
zu-sätz-lich
Sys-te-ma-ti-sie-rungs-ar-beit
Ka-te-go-ri-sie-rung
Teil-a-na-ly-sen
um-lau-ten
Qua-li-tät
aus-lau-ten-de
syn-tak-ti-schen
je-weils
mo-der-nen
in-te-res-san-ter-wei-se
be-dingt
aus-lau-ten-de
er-rei-chen
E-li-sa-beth-tal
ab-zu-lei-ten
folgt
pho-no-lo-gi-sche
%Kap. 5
gleich
er-klärt
Ver-glei-che
ko-o-pe-ra-ti-ven
aus-fal-len
Pe-tri-feld
je-ner
Vor-be-rei-tung
mor-pho-lo-gisch
In-no-va-ti-ons-ra-te
durch-schnitt-li-chen
Durch-schnitt-li-che
In-no-va-ti-vi-tät
Mit-tel-sil-ben-zen-tra-li-sier-ung
ge-ra-de
Laut-qua-li-tät
or-ga-ni-siert
in-te-res-sier-ten
Qua-li-täts-un-ter-schie-de
Be-rei-che
an-nä-hernd
Di-stinkt-heit
Ü-ber-spe-zi-fi-ka-ti-on
ge-spei-chert
syn-te-thi-schen
in-te-res-sant
an-setzt
lin-gu-is-ti-schen
ent-wi-ckeln
die-nen
un-ter-schied-li-cher
phi-lo-so-phi-schen
Ta-xo-no-mie
ma-xi-mal
o-pe-ra-ti-o-na-li-siert
Mo-bi-li-tät
Un-ter-su-chungs-ge-biet
Schwei-zer
Frank-reichs
spe-zi-fi-ziert
Kon-sti-tu-en-ten-struk-tur
Funk-ti-ons-glei-chung
Pe-ri-phra-sen
re-a-li-sie-ren-de
steht
kon-kur-riert
re-a-li-sie-rung
zu-tref-fen
sys-te-ma-ti-schem
Syn-kre-tis-mus-paa-res
meh-re-ren
Me-tho-den
Kürsch-ner
Mo-di-fi-ka-ti-on
Iko-ni-zi-tät
Fle-xi-ons-klas-sen
ver-letzt
Me-tho-de
hier-hin
da-zu
ver-schie-de-ne
ver-an-schau-licht
dy-na-mi-sche
Mi-ni-ma-lis-mus
prin-zi-pi-el-len
prä-sen-tiert
sys-te-ma-ti-sche
Mess-me-tho-den
all-ge-mein-ste
hin-ge-wie-sen
grund-sätz-li-che
Fle-xi-ons-sys-stem
Ers-tens
Pri-mär-
ers-ten
folg-lich
öko-no-mi-scher
wo-durch
kon-kur-rie-ren
Blö-cken
Vis-per-ter-mi-nen
fal-len
Be-lebt-heits-hie-rar-chie
pos-se-si-ven
pho-no-lo-gisch
höchst-ale-man-ni-schen
vo-ka-lisch
Fol-ge
er-wei-tert
Mas-ku-lin
Be-reich
Fle-xi-ons-suf-fix
be-stimm-te
Hot-zen-kö-cher-le
be-ding-te
voll-stän-di-ges
Kon-takt-spra-chen
Sub-stan-tiv-
In-ter-ro-ga-tiv-pro-no-men
des-sen
Fle-xi-on
Szcze-pa-niak
be-stimmt
höchst-ale-man-ni-schen
Ver-all-ge-mei-ner-ung
Kom-ple-xi-täts-aus-gleich
De-ter-mi-nie-rern
De-ter-mi-nie-rer
hö-he-rer
wei-ten
grund-sätz-lich
Netz-wer-ken
Folg-lich
Durch-schnitts-wer-te
glei-cher
Letz-te-re
Be-tont-heit
Hoch-ale-man-ni-sche
fest-hal-ten
ge-nann-ten
Höchst-ale-man-ni-schen
pho-no-lo-gi-schen
No-mi-na-tiv
ein-fa-cher
Ers-tens
Be-reichs
Aus-gleichs-ten-denz
Gleich-zei-tig
fin-det
hoch-ale-man-ni-sche
ge-rings-te
dient
He-ran-ge-hens-wei-sen
Ver-laufs-mus-ter
er-klä-ren
the-o-re-ti-sche
An-tje
Adri-ano
Char-les
Eu-ro-pe
Aar-au
}