\author{Simona Sbranna}
\title{Prosody and interactional fluency of Italian learners of German}

\BackBody{This book explores the development of prosodic and interactional competence in second language acquisition, drawing on data from peer interactions by Italian learners of German in both German and their native language, Italian, as well as from German native speakers. Three key aspects of spoken interaction are examined across proficiency levels: prosodic marking of information status, turn-taking, and backchannels. The analysis of prosodic marking of information status reveals that learners mark givenness using distinct fundamental frequency patterns, as in their native language, but apply a reduction in prosodic strength typically found postfocally in native German, irrespective of its function. This suggests that learners perceive deaccentuation as a salient marker of native German, which they adopt during their learning. This book also presents a novel approach to quantifying interactional competence, showing that lower proficiency negatively affects the smoothness of interactional flow, resulting in reduced speech time and increased overall silence. Finally, it provides new insights into backchannel use in second language and cross-linguistic contexts. Results show a complex, non-arbitrary mapping between lexical type, turn-taking function, and intonation in both native languages. In second language speech, dyad-specific behaviour was found to have a stronger effect on backchannel frequency and duration than second language proficiency. Furthermore, learners tend to transfer preferred lexical backchannel types from their first language into their second language. Overall, this book offers a multidimensional persfpective on second language spoken interaction and lays the groundwork for future applications in language teaching and assessment.

The doctoral work, on which this book is based, was awarded the IPA PhD Thesis Award for the “Best PhD Thesis in the broad area of Phonetics, Speech Sciences, and Laboratory Phonology” in 2024.}
% \subtitle{Add subtitle here if it exists}
\renewcommand{\lsISBNdigital}{978-3-96110-530-4}
\renewcommand{\lsISBNhardcover}{978-3-98554-150-8}


\BookDOI{10.5281/zenodo.15726570}
\typesetter{Simona Sbranna, Hannah Schleupner, Sebastian Nordhoff}
\proofreader{Bruno Behling, Sebastian Nordhoff}
% \lsCoverTitleSizes{51.5pt}{17pt}
\renewcommand{\lsSeries}{eurosla}  
\renewcommand{\lsSeriesNumber}{8}
\renewcommand{\lsID}{515}
\renewcommand{\lsImpressumExtra}{The work presented in this book is based on the author’s doctoral dissertation, which was accepted by the Faculty of Arts and Humanities at the University of Cologne in 2023.}
