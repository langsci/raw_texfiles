\chapter{Prosodic marking of information status}
\graphicspath{{figures/plots-chapter-2}}
\label{chap:2}
This chapter contains an extensive experimental study dedicated to a specific aspect of L2 prosodic competence, the prosodic marking of information status.\footnote{The analysis of L1 Italian speech was previously published in \citet{SbrannaEtAl2023}. The analysis of L2 German speech, along with its comparison to learners’ native and target languages, was previously published in \citet{SbrannaEtAl2025}. This chapter is enriched with additional analyses and connections between individual sections, providing a comprehensive contribution to methodological approaches in phonetic SLA research, as well as insights into the pedagogical implications of the specific results.} This ability is the core of prosodic competence (a view also shared by the CEFR, \citealt{CouncilofEurope2020}), as it allows learners to highlight the relevant parts of their message, guiding the listener in the correct interpretation of their speech.

The objective of this study is to enrich the existing body of knowledge on L2 prosodic marking of information status by providing new evidence based on a continuous approach and unveiling phenomena which had not been observed yet. Previous research has largely relied on categorical approaches, which may not fully capture the dynamic nature of evolving systems such as L2s. Therefore, in this study I make use of an innovative method for phonetic analysis. To motivate and validate this new method, I also provide an alternative analysis using established measures to compare with and propose a possible categorical interpretation of the results, showing that methodological choices can influence our conclusions about SLA. In light of these reflections, I finally provide suggestions for future prosodic analyses and pedagogical applications in the field of SLA research.

Given the unavoidable technicalities present in this chapter, each section contains one or more summaries reporting the main findings in simple and concise form. These summaries can be read independently and sequentially to get a quick overview of the results.

\section{Background}
\label{sec:2.1}
\subsection{Information structure}
\label{sec:2.1.1}

To ensure the correct interpretation of their communicative intentions, speakers distribute information throughout discourse. The term “information structure” (IS) was introduced to describe this phenomenon by \citet{Halliday1967}. Halliday defines IS as the partitioning of a discourse into “information units”, which are distinct from syntactic constituents, but correlated with “tone groups”, i.e. with intonational phrasing. Another definition was offered by \citet{Chafe1976}, who described IS as information packaging, responding to the immediate and temporary communicative needs of speakers. Information packaging operates on the basis of interlocutors’ “common ground” (\citealt{Chafe1976}; \citealt{Karttunen1974}; \citealt{Krifka2008}; \citealt{Stalnaker1974}), which has been defined as “shared" (\citealt{ClarkHaviland1977}), “common" \citep{Lewis1979} or “mutual" \citep{Schiffer1972} knowledge that the parties to a conversation recognise as shared \citep{Stalnaker2002}. In other words, speakers organise their discourse based on the knowledge they assume their interlocutor is familiar with \citep{Prince1981}.

In Halliday’s theory, the IS of an utterance is based on two components, one more informative and one less informative, which are marked by linguistic means. Many terms have been used to describe this informational contrast, and there is no consensus on the different categories of information structure (\citealt{Büring2007}; \citealt{VonHeusinger1999}). In the present work, two terminological oppositions are relevant and will be described in detail, \textit{new} vs. \textit{given} and \textit{focus} vs. \textit{background}.

The terms “new" and “given" refer to the mental representations of discourse referents at two different levels (\citealt{Chafe1994}; \citealt{Lambrecht1994}): identifiability, i.e. the assumed listener’s knowledge of that referent \citep{Prince1981}; and degree of activation, i.e. the assumed listener’s consciousness of that referent at a certain moment in time during the discourse \citep{Chafe1994}. A constituent is given if it is present in the immediate common ground \citep{Krifka2008}, whereas in the converse case it is new. While the distinction between new and given was regarded as a dichotomy in earlier studies (\citealt{BrazilEtAl1980}; \citealt{Halliday1967}), it has since been constructed as a continuum along which different degrees of givenness reflect the degree of cognitive accessibility of the information at a certain moment in time (\citealt{Baumann2005}; \citealt{BaumannGrice2006}; \citealt{BaumannRiester2012}; \citealt{Chafe1994}; \citealt{Prince1981}).

Focus-background structure refers to the pragmatic partitioning of an utterance into informative and less informative parts based on speaker’s \textit{intentions}. In particular, the term ``focus" refers to the element of the sentence that conveys newsworthy information, while ``background" refers to less newsworthy elements (\citealt{Halliday1967}; \citealt{Kuno1978}). The domain of focus is denoted as \textit{broad} if it extends to a whole constituent or sentence and does not pragmatically single out a specific element (\citealt{Gussenhoven1983}; \citealt{Ladd2008}). In contrast, if the domain of focus corresponds only to a selected portion of the utterance, the type of focus is \textit{narrow} \citep{Ladd1980}. Narrow focus can also be contrastive or corrective, where the relative referent either contrasts with a referent in the previous context or serves as a correction (\citealt{Gussenhoven2008}; \citealt{KlassenEtAl2016}; \citealt{Krifka2008}; \citealt{Vander_klokEtAl2018}).

These two dimensions of IS, i.e. new-given and focus-background, are separate dimensions, but can intersect. Whilst referential givenness is orthogonal to focus, in that both given and new elements (representing the two poles on the givenness scale) can be focused, focus and newness are correlated (\citealt{KüglerCalhoun2020}). The present study is concerned with only two degrees of givenness: a) referentially and lexically given (with the referent being part of the immediately preceding context); b) referentially and lexically new (with the referent not being present in the immediately preceding context). Moreover, in this study newness correlates with contrastive focus (the new element is a contextually identifiable alternative of an element of the same category in the immediately preceding context). With this in mind, in the next section I will review previous research investigating prosodic encoding of both givenness and focus, both of which are relevant for the current study.

\subsection{Prosodic encoding of information structure}
\label{sec:2.1.2}
Languages differ in the linguistic means used to mark information structure – e.g. prosody, syntax, and lexical markers. In languages that make use of prosodic means to mark prominence, speakers prosodically attenuate shared or unimportant information and highlight new or important information. The highlighting function of prosody consists in making one element prominent compared to its neighbours, namely to make it ``stand out" from its context by virtue of its acoustic characteristics (\citealt{CangemiBaumann2020}; \citealt{Terken1991}). Different phonological and phonetic properties can contribute to prosodic prominence, e.g. accentuation, pitch accent type, phrase boundaries, or pitch register (for a review, see \citealt{KüglerCalhoun2020}).

In many stress-based systems, accentuation is used to prosodically realise prominence, reflected in a bundle of phonetic cues corresponding to change in F0, local increase in duration, enhanced overall energy and spectral properties on and around the stressed syllable (\citealt{BaumannWinter2018}; \citealt{Campbell1995}; \citealt{DImperio2000}; \citealt{Heldner2003}; \citealt{HermesRump1994}; \citealt{KochanskiEtAl2005}; \citealt{KüglerCalhoun2020}; \citealt{SluijterVanHeuven1996}; \citealt{Turk2012}). By contrast, deaccentuation is defined as the absence of accentuation where an accent would have been expected in a comparable all-new utterance, usually in final position (\citealt{Cruttenden1997}; \citealt{Ladd1980}), reflected in prosodic attenuation.

The majority of the studies on stress-based prosodic encoding of IS have been carried out within the Autosegmental-Metrical (AM) phonology framework (\citealt{BeckmanEtAl2005}; \citealt{Ladd2008}; \citealt{Pierrehumbert1980}) and predominantly on Germanic and Romance languages. According to AM, the main stressed syllable of the head of a prosodic unit is the most prominent one, to which the nuclear pitch accent is assigned – by default the final fully-fledged pitch accent of the prosodic unit. Therefore, focus is considered to be only indirectly marked by phonetic cues: these cues would, in fact, mark nuclear pitch accents, which, in turn, mark focus (\citealt{Büring2016}; \citealt{Calhoun2010}; \citealt{Ladd2008}; \citealt{Selkirk1995}). As a consequence, post-nuclear phonological heads are either deaccented, i.e. not associated with a pitch accent, or they present an accent with a compressed pitch register (\citealt{KüglerFéry2017}), with exceptions occurring only in specific discourse contexts.

However, even within stress-based systems, languages differ in the extent to which this strategy is applied and aligned with syntactical and lexical means. For example, some Romance languages (also stress-based systems) have been claimed to follow the default in situ stress-based pattern (i.e. the nuclear accent can be in any position in the utterance), while others appear to use in situ stress-based marking only in specific discourse contexts, such as contrastive focus, and no post-nuclear deaccentuation (e.g. Italian, Catalan and Madrid Spanish: \citealt{FrotaPrieto2015}; \citealt{VanrellFernández-Soriano2018}).

In the following section, I will review the different methodological approaches used to investigate the phonetics and phonology of prosodic marking of information status. 

\subsection{Categorical and continuous approaches}
\label{sec:2.1.3}
The long tradition of studies on West Germanic languages has yielded evidence for a close relationship between prosodic accentuation and information structure. In particular, focus and newness are considered to be prosodically indicated by the presence of a pitch accent, usually a nuclear one, with background and given elements usually deaccented (\citealt{Cruttenden2006}; \citealt{FérySamek-Lodovici2006}; \citealt{Halliday1967}; \citealt{Ladd1996}; \citealt{Terken1984}). The extensive evidence collected in favour of consistent associations between certain intonation patterns and their functions has often led to the simplistic assumption of a binary distinction between new and given information, and of a one-to-one relationship between accentuation and information status \citep{Halliday1967}.

The earlier studies on the intonation of West-Germanic languages influenced successive research questions on Romance languages. A clear example is the use of the term “re-accentuation” of given elements in early research on Romance languages, to contrast with the “deaccentuation” in West Germanic languages (\citealt{Cruttenden1993}; \citealt{SwertsEtAl1999}), implying that no accent is in some sense the subtraction of an accent that is assigned by default. However, \citet{Cruttenden1993,Cruttenden1997} himself observed that these consistent associations are, in fact, preferred patterns and that languages generally allow for alternative ways of expressing the same function. Nonetheless, the West-Germanic pattern is seen as the default.

Successive experimental evidence has shown that not only new but also given information can be accented (\citealt{BardAylett1999}; \citealt{RiesterPiontek2015}; \citealt{SchweitzerEtAl2009}; \citealt{TerkenHirschberg1994}) and that the relation between pragmatic functions and accentuation – and different pitch accent categories – is rather more complex (\citealt{GriceEtAl2017}; \citealt{KrahmerSwerts2001}; \citealt{MückeGrice2014}). For example, \citet{ChodroffCole2019} found in a study on American English that given information was mostly unaccented or conveyed through low pitch accents (L*), while contrastive information was mostly marked by high (H*) and rising accents (L+H*). However, the opposite relation was found as well: given items realised with high or rising accents and new and contrastive items realised with a low accent or no accent. Similar findings across languages led to the analysis of a probabilistic relation between form and function (for English: \citealt{Calhoun2010}; \citealt{ChodroffEtAl2019}; \citealt{ItoEtAl2004}; \citealt{Yoon2010}; for German: \citealt{BaumannRiester2013}; \citealt{DeRuiter2015}; \citealt{KurumadaRoettger2022}; \citealt{RöhrBaumann2010}).

The same research development from a rigid categorisation to a probabilistic mapping took place for Romance languages. Traditional focal typology used to categorise Romance languages as “non-plastic” languages, with rigid prominence patterns and obligatory word order modifications for expressing focus and information status, in contrast to “plastic” languages, such as West Germanic languages, which can flexibly modify prominence patterns in a sentence to accent focused or new information \citep{Vallduví1991}. Later research attested the use of both strategies in some Romance languages (e.g., Italian) and described it in terms of “preference”, even if these studies still employ a binary distinction between languages with deaccentuation for marking given elements, on the one hand, and languages with marked syntactical structures but limited deaccentuation on the other (\citealt{Cruttenden1993}; \citealt{Ladd1996,Ladd2008}). A different proposal is based on evidence from Spanish, Italian and English: \citet{FaceDImperio2005} show that differences across West-Germanic and Romance languages relate to their distribution, with English only rarely using word order, as compared to Spanish and Italian. Further, there is an interaction between the two strategies, with Spanish using either word order or intonation; and Italian requiring both, i.e. word order and intonation in utterance-final position but intonation only in utterance-medial position. The resulting complex picture cannot be accounted for with a simplistic binary distinction. Instead, a typological continuum with syntax (referred to as word order) and intonation at two opposite poles is proposed.

Recent research has raised awareness of the limitations of a purely categorical approach and the need to account for within- and cross-language variability. \citet{CangemiGrice2016} point out that a distributional approach to phonological categories can better account for the phonetic variability found in encoding. They compare categories to clusters in a multidimensional space which include substance – in Saussurian terms (\citealt{DeSaussure1916}) – along with form and function. Clusters can vary in their compactness and internal structure, possibly presenting sub-clusters (different variants to express the same function). Such an approach considers variability as an integral part of the categories and, consequently, as a source of information about their internal structure. As a result, related studies in intonational phonology have used approaches that explore the continuous dimensions of intonational categories (\citealt{CangemiGrice2016}; \citealt{GriceEtAl2017}; \citealt{RoessigEtAl2019}).

Findings on gradient and probabilistic mapping of form to function have generated a growing interest in the description of phonological pitch accent categories in terms of the acoustic correlates that are used to encode them in both production and perception.

Many studies have focussed on the relevant continuous modulations of the magnitude and the timing of events in the fundamental frequency contour with respect to landmarks in the segmental string (see \sectref{sec:2.8.1} for further details on the segmental anchoring hypothesis and autosegmental-metrical model) and, interestingly, found that the modulation of these phonetic parameters accounts for both variability in the mapping between form and function and inter- and intra-individual variability (\citealt{CangemiEtAl2016}; \citealt{CangemiGrice2016}; \citealt{CangemiEtAl2015}; \citealt{GriceEtAl2017}; \citealt{MückeGrice2014}). For example, \citet{GriceEtAl2017} investigated the mapping of different focal structures onto their prosodic realisation both in terms of pitch accent types and of continuous phonetic parameters contributing to pitch accent categorisation, i.e. F0 peak alignment, target height and tonal onglide. Results of the GToBI (German Tones and Break Indices: \citealt{GriceEtAl2005}) analysis showed that not all speakers use the same accent type to express the same function, but the distribution of continuous parameters revealed systematicity in the patterns for all speakers. For some speakers, the difference in modulation of these continuous parameters was clear enough to cause a shift in the transcription of accent type, while for other speakers the modulation was more subtle and did not lead to the transcription of a different accent category.

Perception studies further validate these results, as listeners accurately perceive the modulation of these parameters in marking a specific function, even when different speakers use different strategies (\citealt{CangemiEtAl2015}; \citealt{GriceEtAl2017}). Further evidence suggests that speakers and listeners differently weigh multiple dimensions to account for the same category (\citealt{CangemiEtAl2015}; \citealt{GriceEtAl2017}; \citealt{NiebuhrEtAl2011}). Thus, the subtle modulation of different phonetic parameters plays a distinctive role in both the expression and identification of functions, contradicting strictly categorical accounts.

In the next sections, I will review the state of the art on the prosodic marking of information status in both Italian and German (the native and the target language of the learners investigated in this study), and finally review the L2 literature, including a pioneering comparative study on Italian and German as L1 and L2.

\subsection{Prosodic marking of information structure in Italian}
\label{sec:2.1.4}
Research on the prosodic marking of information status in Italian offers a fragmentary picture due to differences in experimental design, making a comparison of results difficult. The picture is further complicated by the fact that there is no consensus about a standard intonation and rhythm in Italian, since it presents a great deal of variation across dialectological areas (\citealt{BertinettoLoporcaro2005}; \citealt{Canepari1980}; \citealt{Giordano2006}; \citealt{Lepschy1977}; \citealt{Magno-CaldognettoEtAl1978}; \citealt{Savino2012}; for a comprehensive review see \citealt{FivelaEtAl2015}; \citealt{Vietti2019}). 

Some studies have focused on the prosodic expression of givenness in Italian in light of the phenomenon of deaccentuation that is widely attested in West Germanic languages. These studies are mostly based on the observation that Italian seems to allow deaccentuation of entire syntactic constituents, i.e. full clauses or noun phrases, but blocks deaccentuation within syntactic constituents \citep{Ladd1996}.

Evidence shows that Italian can mark contrastive focus with an accent and then deaccent the following constituent in sentence-length utterances, as in West Germanic languages. In northern Italian varieties, an investigation of different focus structures in sentences composed of two phrases – a noun and a verb phrase – revealed that post-focal falling contours have reduced F0 range and duration as compared to final falls in a neutral condition (\citealt{FarnetaniZmarich1997}). In Tuscan Italian (\citealt{AvesaniEtAl1995}; \citealt{HirschbergAvesani1997}) and Neapolitan Italian (\citealt{DImperioHouse1997}), a low and flat F0 contour with no evidence of F0 movement on post-focal given elements was also found in both full clauses and simple phrases. Indeed, \citet{AvesaniEtAl2015} argue that Ladd’s observation should rather be interpreted as an indication that deaccenting of given elements can, in fact, occur in Italian. 

However, there are also reports claiming the opposite. In Tuscan Italian broadcast speech, textually given elements were found to be accented with an L* irrespective of their grammatical function and position in the sentence \citep{Avesani1997}. Similarly, in Roman Italian task-oriented dialogues, most coreferential given elements were found to be accented irrespective of their position in the discourse and prosodic unit, and only a few cases of post-focal deaccentuation are reported in sentences with fronted foci (\citealt{AvesaniVayra2005}). No relation was found between a specific pitch accent and givenness. Further investigations on Tuscan Italian have found that post-focal given elements occurring in a metrically strong position, i.e. as head of the prosodic domain, show a clear increase in duration, as well as in formant and spectral emphasis, but no F0 movement (\citealt{Bocci2013}; \citealt{BocciAvesani2011,BocciAvesani2015}). Thus, it has been suggested that the low and flat F0 contour found on post-focal given elements should be interpreted as an L* pitch accent, rather than as deaccentuation \citep{Bocci2013}.

The presence of an accent in the flat post-focal region is also suggested by another study on the production and perception of focus in short Neapolitan Italian sentences (\citealt{DImperio2002}). The author hypothesises that the flat postfocal region is a compressed, downstepped version of the non-salient H+L* phrase accent, which would also explain why it lacks salience in perception.

These studies, despite providing important experimental evidence for the prosodic expression of givenness in Italian, do not yield a clear picture, owing to differences in their design and materials (e.g., the variety of Italian under investigation, size of the data sample, data collection method, speech style analysed, and analysis). 

\subsection{Prosodic marking of information structure in German}
\label{sec:2.1.5}
In German, prosody is the main linguistic marker of information structure. The traditional assumption that new elements are accented and given ones deaccented (\citealt{Allerton1978}; \citealt{Cruttenden2006}) has been relativised by several studies (\citealt{Baumann2005}; \citealt{BaumannHadelich2003}; \citealt{KüglerFéry2017}; \citealt{Wagner1999}). These investigations still provide evidence that deaccentuation is the most appropriate and common way to mark givenness but show that different options are available. For example, \citet{BaumannRiester2013} found fewer deaccentuation cases than expected from the literature in a study on both read and spontaneous speech, and claim that there is a differential probability for an item to receive an accent. Therefore, the authors suggest a gradient scale of prosodic prominence, realised through a range of different accent types (including deaccentuation), mapping onto different degrees of activation of a referent (\citealt{Baumann2006}; \citealt{BaumannRiester2013}). In turn, different pitch accent types are realised through distinct modulations of continuous phonetic parameters.

The central phonetic cue used in German to mark information status is pitch excursion. \citet{FéryKügler2008} found a correspondence between information status and tonal scaling, with narrow focus raising the high tones of pitch accents and givenness lowering them in prenuclear position and cancelling them out in postnuclear position. Similar results are also reported for analyses of several phonetic cues contributing to prosodic marking of focus (e.g., pitch excursion, peak position, duration and accent type in \citealt{BaumannEtAl2006}; accent type, duration and articulatory gestures in \citealt{HermesEtAl2008}). Peak position appears to play a role as well. \citet{Kohler1991} found that different accent contours were perceived as corresponding to different meanings, i.e. late peaks (described as L+H*/L*+H) for emphasis or contrast, medial peaks (H*) for new information and early peaks (H+L*/H+!H*) for accessible or given information. In particular, a categorical distinction was only found between early and medial peaks, whereas there was a gradient difference between medial and late peaks. 

These studies provide a much clearer picture of prosodic marking of information status and focus in German as compared to the state of the art on Italian, clearly showing that there is a probabilistic and gradient relation between information status and accent type (including the absence of an accent).

\subsection{Information structure and prosody in L2}
\label{sec:2.1.6}
Since languages differ in how prosody is used to mark IS, one interesting research question concerns how L2 learners acquire and develop these patterns. To date, there are very few studies on the prosodic encoding of information structure in interlanguages, i.e. a system distinct from both the native and the target languages \citep{Selinker1972}, although it has important consequences for successful communication and the potential for research and educational application is therefore considerable. The little evidence collected so far is mostly on English as L2 and results are contradictory with regard to the effects of proficiency.

Learners’ prosodic encoding of IS has been found to differ from English native speech both in phonetic and phonological terms. From a phonetic point of view, differences in the use of peak alignment, pitch height and pitch movement have been found. For example, a delayed pitch peak on new information was found in Korean learners of English (\citealt{TrofimovichBaker2006}), probably due to an influence of L1 pitch alignment patterns. The interlanguage of Mandarin Chinese learners of English, instead, presented less difference in pitch excursion across new and given items compared to L1 English speakers \citep{Wennerstrom1998}. The same result is reported for Spanish learners of English, whose pitch range on focused constituents is narrower, without a clear differentiation from the adjacent syllables. Moreover, these learners were also found to produce a falling contour on both new and given elements, while native speakers differentiated the information status using a fall and a low rise respectively \citep{Verdugo2003}. A study on Malay learners of English showed that the phonetic details of L2 rises resembled those in the speakers’ L1 (\citealt{GutPillai2014}). From a phonological point of view, there seem to be a common tendency for non-native speakers to overaccentuate, regardless of IS (Austrian learners in \citealt{Grosser1997}; Spanish learners in \citealt{Verdugo2003}; learners of various L1 backgrounds in \citealt{Gut2009}; Malay learners in \citealt{GutPillai2014}). In interpreting this result, it is important to keep in mind that the target language is always English, in which accentuation is used to mark IS, whereas learners’ native languages might have a looser relation between accentuation and IS, and might follow other criteria (e.g. phonological ones) to distribute accents in the utterance, leading to the accentuation of given elements in the L2. This is the case for Spanish and Malay learners accentuating given elements in final position, where final position is taken to be the default, or unmarked case, i.e. “the pattern that is chosen when there is no compelling grammatical or contextual reason to choose some other” \citep[223]{Ladd2008}.

In a cross-linguistic study on a language pair not including English, i.e. Dutch and French (\citealt{RasierEtAl2010}), it was found that not all learners tend to overaccentuate, and that both Dutch learners of French and French learners of Dutch transfer their L1 features in prosodically encoding the IS of noun phrases. Specifically, Dutch learners tended to use the less common French “focus accent” (more similar to their own L1), in which only one element of the noun phrase is accented, and never the more common French “bridge accent”, in which both elements of the noun phrase are accented. French learners also applied their own L1 strategy to the L2 in not deaccenting contextually given information. The difference between learner groups is explained by the Markedness Differential Hypothesis \citep{Eckman1977} according to which marked structures (such as accentuation according to pragmatic contexts) are more difficult to learn that unmarked ones (such as accentuation in the default condition), so that Dutch learners have an advantage over French learners in this context. Another study conducted on languages other than English reached similar results (German and Dutch learners of Italian in \citealt{TurcoEtAl2015}). While in L1 Italian polarity contrasts was marked through a verum-focus accent (i.e., accent placed on the finite verb in verb second position used to emphasise the truth of the sentence as in \citealt{Höhle1992}) in a minority of cases, German learners of L2 Italian produced more verum-focus accents and transferred their L1 phonetic implementation by often deaccenting post-focal constituents instead of using post-nuclear pitch accents. Transfer was also found for Dutch learners of L2 Italian who preferred lexical markers as in their L1. From these findings based on L2s other than English, it appears that the influence of the L1 may better explain the consistent overaccentuation found in L2 studies on the prosodic encoding of IS (although there are few studies directly comparing learners’ L2 to their L1 productions to provide evidence for transfer) compared to the proposal that overaccentuation might be a universal tendency (\citealt{GutEtAl2013}; \citealt{GutPillai2014}).

Results relative to the role of proficiency are contradictory. Some studies show that transfer of L1 features tends to reduce as language proficiency increases. As an example, in a study comparing speakers of L1 Zulu with their L2 English, it was found that beginners did not mark information status within noun phrases through accents for contrastive or corrective focus as in their L1, while advanced learners tended to appropriately associate accentuation and information status in contrastive focus (\citealt{SwertsZerbian2010}). Likewise, advanced Japanese learners of English were able to map given information to deaccentuation and contrastive information to an L+H* accent in the same way as native English speakers in rating and production tasks. Differently, less proficient L2 learners associated given information with deaccentuation and contrastive information with L+H* in the rating task only \citep{Takeda2018}. In contrast to these studies, some other studies do not find an improvement in intonational competence corresponding to higher proficiency (\citealt{Bi2008}; \citealt{Verdugo2003}), so that the role of proficiency in the learning process remains unclear.

\subsection{A comparative research programme on Italian and German as L1 and L2}
\label{sec:2.1.7}
One pioneering research programme has allowed for the comparison of Italian and West-Germanic languages with an experimental design that brought to light differences ascribable to language structure: \citet{SwertsEtAl2002} and \citet{AvesaniEtAl2013,AvesaniEtAl2015} investigated the prosodic marking of information status in Tuscan Italian and Dutch, and in Tuscan Italian and German, respectively. Their findings support \citegen{Ladd1996} observation that Italian strongly disfavours deaccentuation within noun phrases as opposed to West-Germanic languages.

\citet{SwertsEtAl2002} used a card game to semi-spontaneously elicit noun phrases composed of two words (a noun and an adjective), which could be new, given or contrastive according to the context. For Italian, they report an F0 excursion on both words with a hat pattern stretching over the entire noun phrase regardless of the varying information structures, while in Dutch the F0 excursion correspond to the new element only. It was concluded that Italian fails to deaccent post-focal given elements within noun phrases. A following perception experiment reinforced this finding: Italian listeners could not reliably reconstruct the context in which the noun phrases were produced when listening to them in isolation. This result was replicated in a second perception experiment with the same data (\citealt{KrahmerSwerts2008}), which suggests that these utterances lack any other prosodic cues upon which listeners can rely to identify their information structure.

\citet{AvesaniEtAl2013,AvesaniEtAl2015} successfully replicated the production experiment by \citet{SwertsEtAl2002}, reporting that in Italian the second word of the noun phrase is always accented, independently of its pragmatic status, whereas the first word can lack an accent in some cases. Interestingly, a range of pitch accents was found for both the first and second words – H*, H+L* and L+H*, but not including the L* found in previous studies on Tuscan Italian \citep{Bocci2013}. The explanation given is that, in Italian, phonological constraints override the mapping between prosody and pragmatic functions, such as focus or information status. In detail, it is argued that the two words of the NP constitute an intonational phrase, whose metrical head at the rightmost position is the stressed syllable of the second word. The first word, being in pre-nuclear position, can optionally be accented, but does not have to be. In contrast, the metrical head (the second word of the NP) has to bear the nuclear accent and the presence of the nuclear accent in the rightmost strong metrical position cannot be overridden by syntactic or pragmatic requirements (\citealt{AvesaniEtAl2015}). In German, according to this view, deaccenting of the strongest metrical position of the intonational phrase is allowed and occurs when required by the information structure (i.e. in the case of given items).

\citet{AvesaniEtAl2013,AvesaniEtAl2015} also extended the investigation to L2 German spoken by Italian learners and L2 Italian spoken by German learners, with the same proficiency level in their respective L2s. They found that Germans successfully reproduced the Italian accentuation pattern, but Italians did not, as only a trace of deaccentuation of given post-focal elements was found in their L2 German (17\% of the cases; statistically significant). The authors explain this result with the ``Markedness Differential Hypothesis" (\citealt{Eckman1977}; as in \citealt{RasierHiligsmann2007} discussed in \sectref{sec:2.1.6}), whereby marked structures are more difficult to learn that unmarked ones, and also mention the ``Similar Differential Rate Hypothesis" (\citealt{MajorKim1996}), according to which marked structures are also acquired with a lower speed of learning. This means that Germans can take advantage of possessing both strategies of accentuation in their native language, i.e. the phonological and the pragmatic one. In contrast, Italian learners of German have a harder task: they have to learn that the distribution of prominences is not necessarily phonological in German and is instead associated with the pragmatic status of the referent. Therefore, the most difficult challenge will be to learn to deaccent the post-focal given referent in nuclear position.

Despite enriching our knowledge of prosodic marking in Italian and German, the studies reviewed above have certain limitations. They investigate a relatively small group of participants, making generalisation of the results difficult, in particular in light of individual differences. They also use a game in which noun phrases are elicited from alternating speakers, where the information status of adjectives and nouns differs across turns. The statements in alternating turns may not have created an engaging interaction between speakers, who may not have assumed the other player’s sentences as the context for their own productions, and, instead, speakers may have concentrated on their own list of statements. Finally, the authors focus on the categorical presence or absence of pitch accents and pitch accent type, and only provide limited information on continuous measures – \citet{SwertsEtAl2002} report the F0 excursion only. However, previous research on other languages has shown that a closer inspection of continuous phonetic parameters can provide essential information about the expression of pragmatic contrasts, raising the question as to whether a closer examination of these parameters might have revealed differences that were not captured in the categorical analysis.

In the following study, I attempt to overcome some of these limitations using a semi-spontaneous interactive board game to elicit different types of information structure in a more naturalistic interaction and with a larger sample of participants compared to previous studies. Moreover, I analyse the way speakers modulate continuous parameters to prosodically mark information status and propose a categorical interpretation of the results.

\section{Method}
\label{sec:2.2}
\subsection{Data elicitation}
\label{sec:2.2.1}
As discussed above, some previous studies on the prosodic marking of information status by Italian learners of German elicited data using a card game structured in form of statements between two participants. In that game, both participants receive an equal set of cards containing pictures of different types and varying colours. In alternating turns, one participant picks a card and names its content so that the other participant can align the corresponding card on a board. The two participants alternate the roles of \textit{instruction giver} and \textit{instruction follower}. The variation of picture type (noun) and colour (adjective) was designed to create contrastive information statuses in two successive noun phrases (NPs) produced by participants. However, this type of task presents some disadvantages: it does not favour interaction, so that participants might not assume the other’s turn as a context for their own statement and the production of alternating statements might become repetitive and create a list effect. These disadvantages might affect the prosody of participants’ realisations and interfere with the pragmatic conditions intended by the experimenters.

Therefore, in the present study special care has been taken in designing an elicitation game to 1) increase the degree of interaction and 2) avoid the risk of repetitiveness. Despite the difficulty in collecting such specific items in more spontaneous conversation, the design was oriented towards the best compromise possible between ecological validity and the elicitation of noun phrases under the intended pragmatic conditions.

To do so, I created a semi-spontaneous interactive board game to be played in pairs. Each participant received a differently randomised board containing 62 sequentially numbered squares. Each square had a flap which could be lifted to see what is underneath, that is images of various types (noun) and colours (adjective). One example board without the flaps is shown in \figref{fig:2.1}. All possibly occurring types and colours were listed in the instructions of the game (\figref{fig:2.2} and \figref{fig:2.3}) both in visual and written form, but the boards only contained the pictures to avoid interference from reading. Participants were also provided with an additional empty board, displaying only the numbered squares. The task was intended as a distraction, with the aim that participants would pay less attention to their speech. In the instructions, participants were informed that they would go through the table in sequentially alternating turns and that two items are important to win the game, golden apples and bombs. The latter destroys the opponent’s golden apples. The person who finds the most golden apples at the end of the game wins, provided that they have correctly transcribed the content of the other player’s board.

\begin{figure}
\includegraphics[height=.75\textheight]{figure_2_1_board-pdf.pdf}
% \includegraphics[height=.75\textheight]{figure_2_1_board-pdf.pdf-0.png}
\caption{Example board of the elicitation game, version for player A in German language. For the Italian example board version of the board game consult \citet{SbrannaEtAl2023}.}
\label{fig:2.1}
\end{figure}

Participants were instructed to sequentially lift the flaps of the table in alternating turns and communicate with a suggested script exemplified in the instructions. The game proceeds as follows: Player A starts uncovering the first picture by lifting the corresponding numbered flap (square number one) and asks player B if they have the picture they see, using a question in which they mention the type and colour of the image (e.g. “Do you have a yellow cow?”). To answer the question, Player B also uncovers the picture in square number one on their own board, and answers with yes or no, followed by the mentioning of the type and colour of the matching or mismatching image (e.g. “Yes, I have a [matching image]” or “No, I have a [mismatching image]”). This exchange constitutes a game turn. Player B would take the next turn and ask a question about the next square. This alternation of exchanges between players continues until all the squares are revealed. At the end of each turn, both players write on the board (with 62 empty spaces), what their opponent has on their board. At the end of the game, each participant counts how many golden apples they found that were not destroyed by the opponent’s bomb. The person with the most golden apples checks the correctness of their list (the empty board they filled up throughout the game) together with the opponent and wins or loses the game accordingly. This is done in order to keep participants alert and engaged in the game throughout.

Players’ answers about mismatching images by either type or colour, or both type and colour contain our target noun phrases with contrastive elements. The game also elicits yes-replies about matching images by both colour and type. These were inserted only to avoid the bias of a negative answer. Example 1 provides one exchange for each of the possible information structure conditions in the Italian and German versions, with the respective English translation. The questions serve as pragmatic context and the replies as carrier sentences of the noun phrases, marked in bold:

\ea
\ea Italian

A: \textit{Hai una \textbf{mano} \textbf{nera}?}

A: Do you have a black hand?\footnote{Note that in Italian the order of the elements in the NP is reversed, i.e. the noun is followed by the adjective, contrary to the English translation.}

B: \textit{No, ho una \textbf{mano} \textbf{lilla}.}

B: No, I have a lilac hand.

\ex German

A: \textit{Hast du eine \textbf{braune} \textbf{Blume}?}

A: Do you have a brown flower?

B: \textit{Nein, ich habe eine \textbf{braune} \textbf{Nonne}.}

B: No, I have a brown nun.
\z
\z

The pseudo-randomisation of the sequence of images in the boards followed the criterion that no two identical nouns or adjectives could occur at two subsequent turns, in order to avoid an unintended degree of activation of the elements deriving from the preceding turn, interfering with the desired prosodic realisations within the turn.



\begin{figure}
\includegraphics[width=\textwidth]{figure_2_2_Istruzioni IT-cropped.pdf}
\caption{Italian board game instructions with list of all occurring colours and objects.}
\label{fig:2.2}
\end{figure}

  
\begin{figure}
\includegraphics[width=.95\textwidth]{figure_2_3_Anleitung DE-cropped.pdf}
\caption{German board game instructions with list of all occurring colours and objects.}
\label{fig:2.3}
\end{figure}

One final clarification about the information structures of the target noun phrases is necessary. This study is concerned with the dimension of cognitive states (new and given, with all elements being equally accessible since they are listed in the instructions) although, given the context of elicitation (see Example 1), in our items this dimension overlaps with pragmatic functions, i.e. NN with broad focus, and the new element of GN and NG with contrastive focus. The latter case is defined as a contrastive and not a corrective focus to distinguish it from occasional spontaneous occurrences of explicit corrections in our corpus, e.g. instances elicited by a context question like “Have you said red apple?” (with “red” being in focus). Still, I consider contrastive and corrective focus as different degrees of contrast on a continuum and not as two distinct categories. It has also been claimed that it is unlikely that languages prosodically distinguish contrastive and corrective foci. Instead, speakers would increase prominence on contextually salient foci, which may be explicit contrasts or corrections (\citealt{BaumannEtAl2006}; \citealt{Calhoun2009}; \citealt{Féry2013}; \citealt{KüglerCalhoun2020}).

With this method, I elicited data from Italian and German native speakers (control groups) and Italian learners of German in both their L1 and L2 (\sectref{sec:1.3.1} and \sectref{sec:1.3.2} for details on participants and recording sessions, respectively).

\subsection{Corpus}
\label{sec:2.2.2}
The target noun phrases were derived from turns in the elicitation game. All NPs are composed of a disyllabic noun and a disyllabic adjective, both with penultimate stress, and correspond to three different types of information structures: given-new (GN), new-new (NN) and new-given (NG). Notice that in Italian the noun precedes the adjective, while in German the adjective precedes the noun. All target NPs are listed by information structure condition and language in \tabref{tab:2.1}. Each speaker uttered each noun phrase only once.

\begin{table}
\begin{tabular}{lll}
\lsptoprule
Condition & Italian & German\\
\midrule
GN & mano lilla & braune Welle\\
GN & nave nera & blaue Blume\\
GN & mela verde & graue Vase\\
GN & rana lilla & braune Nonne\\
GN & vela nera & blaue Birne\\
GN & luna verde & graue Dose\\
NN & nave lilla & blaue Welle\\
NN & mela nera & graue Blume\\
NN & mano verde & braune Vase\\
NN & vela lilla & blaue Nonne\\
NN & luna nera & graue Birne\\
NN & rana verde & braune Dose\\
NG & mela lilla & graue Welle\\
NG & mano nera & braune Blume\\
NG & nave verde & blaue Vase\\
NG & luna lilla & graue Nonne\\
NG & rana nera & braune Birne\\
NG & vela verde & blaue Dose\\
\lspbottomrule
\end{tabular}
\caption{Target noun phrases by condition for Italian and German.}
\label{tab:2.1}
\end{table}

I do not include in the analysis the functional elements which were merely intended to give the game a goal and avoid repetitiveness, i.e. the mentions of golden apples, bombs and noun phrases inserted in yes-reply carrier sentences, whereby both the noun and the adjective are contextually given. The reason for not including this latter given-given condition (GG) is that due to the semi-spontaneous nature of the game, many speakers simply answered “yes” to the context question without following the suggested script and using the noun phrase, in other words, I consider these productions driven by the script only. The first two exchanges were used as a training phase and, therefore, also excluded from the analysis.

It is important to mention that in some cases, speakers were so engaged in the game that they forgot the suggested script for the interaction. For this reason, some tokens were not realised as prescribed, resulting in a few missing items. However, these cases demonstrate that the game succeeded in engaging the speakers, resulting in more spontaneous behaviour than generally expected from a scripted task. \tabref{tab:2.2} contains the number of items for each condition and language group. For the sake of ecological validity, I decided not to exclude any items from the acoustic analysis based on subjective impression of what a “good” or “bad” item is.


\begin{table}
\begin{tabular}{lrrrr}
\lsptoprule
 & IT L1 & IT L1 & GE L2 & GE L1 \\
(Control) & (Learners) & (Learners) & (Target)\\
\midrule
Total items & 231 & 670 & 718 & 323\\
GN items & 69 & 222 & 240 & 108\\
NN items & 81 & 220 & 239 & 108\\
NG items & 81 & 228 & 239 & 107\\
\lspbottomrule
\end{tabular}
\caption{Amount of noun phrases collected by group and condition.}
\label{tab:2.2}
\end{table}


\subsection{Measurements}
\label{sec:2.2.3}
This study makes use of a different approach to the analysis than previously done. The continuous modulation of prosodic parameters – as compared to previous studies with a categorical approach – are analysed by using the open-source ProPer workflow \citep{AlbertEtAl2020}. This innovative method is based on two corresponding and interacting acoustic time series: F0, which measures the acoustic rate of oscillation of the perceived pitch, and periodic energy, which measures the acoustic strength of the pitch-bearing portions of the signal. 

The ProPer workflow derives the periodic energy curve from Praat’s signal processing objects (\citealt{BoersmaWeenink2021}), which is used in R (\citealt{R_core_team2021}) to 1) produce periograms, enriched visual representations of F0 trajectories modulated by periodic energy (\citealt{CangemiGrice2018}), and 2) calculate various metrics to account for the F0 movement and the prosodic strength of syllabic intervals. I employ periograms for visualisation and three ProPer metrics to quantify aspects of prosody and perform statistical inference: a) periodic energy mass, b) synchrony and c) ${\Delta}$F0. I will now explain these metrics in a way functional to the interpretation of the results. Please refer to \citet{SbrannaEtAl2023} and \citet{SbrannaEtAl2025} for technical details on their calculation.

Periodic energy correlates with sonority so that the fluctuations in the periodic energy curve are distributed around sonority peaks, i.e. syllable nuclei (see \citealt{CangemiGrice2018} for details and visual examples), tending to correspond in this way to syllabic intervals. The prosodic strength of each syllable is given by the periodic energy mass, from here on referred to as “mass”, reflected by the area under the periodic energy curve. Mass accounts for both duration and power of each syllable and is normalised relative to the other syllables within a given utterance, such that values above one indicate strong mass and values below one indicate weak mass (\figref{fig:2.4}a).

Two measurements account for the shape of F0 by calculating interactions between the periodic energy and F0 curves (\citealt{AlbertGrice2019}): synchrony and ${\Delta}$F0. Synchrony reflects the shape of the F0 contour within syllabic units (\figref{fig:2.4}b – akin to peak alignment, e.g. \citealt{LaddMennen2006}), while ${\Delta}$F0 reflects the shape of the F0 contour with respect to the previous syllabic unit (\figref{fig:2.4}c). For both metrics, positive values indicate a rising F0, while negative values signify a falling F0. To normalise these metrics, their relative values are used. Synchrony (measured in milliseconds) is calculated relative to the duration of the containing syllable to accurately represent the F0 slope in syllables of different lengths. ΔF0 (measured in Hertz) is calculated relative to the speaker's range to minimise speaker-specific paralinguistic effects, such as gender differences. Note that in our analysis, the value of ${\Delta}$F0 on the first syllable does not refer to the difference from the previous portion of the utterance, which is not taken into account, but rather the difference from the median F0 of that speaker. This value is useful to flag cases starting with a relatively high F0, resulting in a positive ${\Delta}$F0 value.\footnote{I use this method on the first syllable of the target NPs instead of ${\Delta}$F0 referring to the previous portion of the utterance because our data often display a pause between the carrier sentence (“No, I have a [. . .]”) and the target item, i.e. the noun phrase. Since there is not always analysable material preceding the target item, this choice allows me to present a unified measurement for ${\Delta}$F0 values in Syllable 1 across all data.} 

This workflow is applied to the acoustic analysis of the target noun phrases under different information structure conditions. 


\begin{figure}
\includegraphics[width=\textwidth]{figure_2_4_icons-cropped.pdf}
\caption{Integrated measures of F0 and periodic energy.}
\label{fig:2.4}
\end{figure}

\subsection{Bayesian analysis}
\label{sec:2.2.4}
Statistical inference was performed by fitting Bayesian hierarchical linear models using the Stan modelling language \citep{CarpenterEtAl2017} and the R package \textit{brms} \citep{Bürkner2016}. For each language group, the differences among conditions in synchrony, ${\Delta}$F0 and mass were tested as a function of factors \textsc{condition} (reference level \textsc{ng}), \textsc{syllable} (reference level \textsc{syllable 1}) and their interaction. As random effects, the models include random intercepts for \textsc{token} and \textsc{speaker}. For \textsc{speaker} the models also include by-speaker random slopes for \textsc{condition} and \textsc{syllable} and correlation terms between all random effect components. For models testing the differences across groups, the fixed effect \textsc{group} was added to \textsc{condition} and \textsc{syllable}, as well as a three-way interaction between them.

For the measurements of synchrony, ${\Delta}$F0, and mass, a normal distribution was used, and regularising priors for the intercept and the regression coefficient were defined based on theoretical reasons and observations on other datasets. Priors for synchrony are theoretically driven: The two centres, CoM and CoG, are both attracted to the centre of the interval, so the distance between them does not tend to exceed 25\% of the entire duration of the interval containing them. Priors for ${\Delta}$F0 are based on observations in multiple data sets, in which most of the ${\Delta}$F0 values reflecting the F0 change between syllables are below 50\% (of that speaker’s range). Higher values are possible but mostly do not exceed 70\%. Priors for mass are also based on observations in multiple data sets, in which the vast majority of values is found between 0.25--2.5.

The default settings of the brms package were retained for all other parameters. For relative synchrony, the intercept was set at µ = 0, δ = 30 and the regression coefficient at µ = 0, δ = 5; for relative ${\Delta}$F0, the intercept was set at µ = 0, δ = 50 and the regression coefficient at µ = 0, δ = 25, for mass, the intercept was set at µ = 0, δ = 3 and the regression coefficient at µ = 0, δ = 0.5. Four sampling chains for 4000 iterations with a warm-up period of 3000 iterations were run for all models. There was no indication of convergence issues (no divergent transitions after warm-up; all Rhat = 1.0), including from visual inspection of the posterior distributions.\footnote{The model’s assumption for \textrm{${\Delta}$}F0 were not fully satisfied since our posterior simulations are less leptokurtotic than our actual data. Still, the model does not show convergence problems.}

For all relevant contrasts (δ), I report the expected values under the posterior distribution and their 95\% credible intervals (CIs), i.e. the range within which an effect is expected to fall with a probability of 95\%. For the difference between each contrast, the posterior probability that a difference is bigger than zero (δ > 0) is also reported to ensure comparability with conventional null-hypothesis significance testing. In particular, it is assumed that there is (compelling) evidence for a hypothesis that states δ > 0 if zero is (by a reasonably clear margin) not included in the 95\% CI of δ and the posterior P(δ > 0) is close to one (cf. \citealt{FrankeRoettger2019}). 

All models, results and posteriors can be inspected in the accompanying RMarkdown file at the Open Science Framework (OSF) repository (\url{https://osf.io/9ca6m/}).

\section{Results: Italian L1}
\label{sec:2.3}
\subsection{Learners of German}
\label{sec:2.3.1}
\figref{fig:2.5} shows three representative example periograms (\sectref{sec:2.2.3}), along with the three acoustic metrics, one for each information structure condition as uttered by Italian learners of German in their L1 Italian. By visualising periograms, it can be observed that GN and NN are similar in that F0 is rising throughout the first syllable and reaches a peak on the second syllable, after which there is a fall. By contrast, NG reaches a peak already in the first syllable, which is where it starts falling. Thus, there are two intonation patterns that are distinguished through timing of the F0 fall: earlier when the final position features given information (NG) and later when the final position features new information (GN and NN). This distinction carries over to the transition between words, i.e. between the second and third syllables. The falling F0 trend ends earlier in NG, such that the transition from the second to the third syllable is mostly flat, while it is still falling in GN and NN. As for the values of mass, the syllables associated with stress (Syll1 and Syll3) tend to be stronger than the following unstressed syllables (Syll2 and Syll4, respectively), and the stressed syllables in new words of the NG and GN conditions tend to display stronger energy than the stressed syllables in adjacent given words.

\begin{figure}
\subfigure[GN. The token is \textit{mano lilla} (`lilac hand').]{
\includegraphics[width=.7\textwidth]{figure_2_5_a_periogram_GN_mano lilla.pdf}
}
\subfigure[NN. The token is \textit{rana verde} (`green frog').]{\includegraphics[width=.7\textwidth]{figure_2_5_b_periogram_NN_rana verde.pdf}
}
\subfigure[NG. The token is \textit{luna lilla} (`lilac moon').]{\includegraphics[width=.7\textwidth]{figure_2_5_c_periogram_NG_luna lilla.pdf}}
\caption{Example periograms displaying F0 and periodic energy for two-word noun phrases in three information structure conditions produced by L1 Italian learners.}
\label{fig:2.5}
\end{figure}

Aggregated data of synchrony, ${\Delta}$F0 and mass confirm these observations (\figref{fig:2.6}). Specifically:

\begin{itemize}
\item \textit{Synchrony}. Positive synchrony values indicating a rising F0 trajectory can be found almost exclusively on Syll1 of GN and NN. The earlier F0 fall in NG is reflected in negative synchrony values on Syll1 and Syll2, while the later F0 fall in GN and NN is reflected in negative synchrony values on Syll2 and Syll3.
\item \textit{${\Delta}$F0}. Positive ${\Delta}$F0 values on Syll2 and negative ${\Delta}$F0 values on Syll3 are indicative of the location of the F0 peak in Syll2 of GN and NN conditions. In contrast, negative ${\Delta}$F0 values on Syll2 in the NG condition indicate that the F0 peak is within Syll1.
\item \textit{Mass}. Mass values generally reflect that stressed syllables and new words promote stronger prosodic strength, as expected. Stressed syllables are reflected in the distinction between Syll1 (stressed) vs. Syll2 (unstressed) as well as Syll3 (stressed) vs. Syll4 (unstressed), while information status (given vs. new) is mostly reflected in the distinction between Syll1 vs. Syll3 of the GN and NG conditions. These trends are clearly apparent in the NG condition and, to a lesser extent, in the NN condition. Mass values in the initial three syllables of the GN condition display a wide distribution of mostly strong energy (values above one), thus appearing to attenuate the stress-related mass distinctions in the first word and the information status mass distinctions between the two stressed syllables (Syll1 vs. Syll3).
\end{itemize}

\begin{figure}
\includegraphics[width=.6\textwidth]{figure_2_6_Violins Italian L1 learners.pdf}
\caption{Aggregated values of synchrony, ${\Delta}$F0 and mass (on the y-axes) pooled across Italian L1 learners. The x-axis displays the four syllables of the noun phrases, with Syll1 and Syll2 being the noun and Syll3 and Syll4 the adjective. Information structure conditions are colour-coded: green for given-new (GN), blue for new-new (NN) and red for new-given (NG).}
\label{fig:2.6}
\end{figure}

These trends revealed to be robust by means of Bayesian analyses:

\begin{itemize}
\item \textit{Synchrony} on Syll1 presents lower values in NG as compared to GN (δ = 5.1, CI [4.13; 6.07], P (δ > 0) = 1) and NN (δ = 6, CI [5.01; 6.94], P (δ > 0) = 1), while synchrony on Syll3 presents higher values in NG as compared to GN (δ = 3.46, CI [2.49; 4.42], P (δ > 0) = 1) and NN (δ = 4.3, CI [3.26; 5.23], P (δ > 0) = 1).
\item \textit{${\Delta}$F0} on Syll2 is lower in NG as compared to GN (δ = 21.73, CI [17.64; 25.97], P(δ > 0) = 1) and NN (δ = 24.05, CI [19.93; 28.27], P(δ > 0) = 1), while ${\Delta}$F0 in Syll3 is higher in NG as compared to GN (δ = 15.46, CI [11.42; 19.37], P(δ > 0) = 1) and NN (δ = 19.86, CI [15.71; 23.74], P(δ > 0) = 1).
\item \textit{Mass} on Syll1 is higher in the NG condition than in GN (δ = 0.21, CI [0.17; 0.25], P(δ > 0) = 1) and NN (δ = 0.15, CI [0.11; 0.18], P(δ > 0) = 1). In addition, within NG values of mass are higher in Syll1 than in Syll3 (δ = 0.19, CI [0.13; 0.25], P(δ > 0) = 1).
\end{itemize}

Both the data and the models strongly support our claim that (Neapolitan) Italian prosodically marks post-focal given elements within NPs through the continuous modulation of acoustic parameters. In particular, NG is realised with an F0 peak early in the first word as opposed to GN and NN, which are realised with an F0 peak late in the first word. These F0 shapes are accompanied by distinct modulations of prosodic strength: along with the earlier F0 peak, NG displays stronger energy on the first syllable as compared to GN and NN. However, modulations are slight and do not result in overall different patterns across information structure conditions. As a consequence, no reduction in energy on the post-focal second word is found as typical of West-Germanic languages. 

\subsection{Monolingual control group}
\label{sec:2.3.2}
Since the Italian speakers commented above are learners of L2 German and recordings were performed at the Goethe Institute, which is a German-language-dominant setting, a group of monolingual Italian native speakers was recorded as well to control for a possible influence of the German language and setting on learners’ L1 Italian productions. By “monolingual” I mean that these Italian participants, despite some inevitable contact with foreign languages in the past years of their lives (such as in school contexts), were neither learners of, nor proficient or fluent in any foreign language and were not familiar with the Institute. To highlight the fact that there is no exposure to an L2 which could influence their native Italian prosody and that these participants are a different group of L1 Italian speakers than the one presented above (who are learners of L2 German), I refer to this group as the “monolingual control group”.

\figref{fig:2.7} shows the same patterns of synchrony, ${\Delta}$F0 and mass observed in the Italian learners’ dataset for the monolingual control group. Two F0 shapes distinguished by the location of the peak characterise the realisation of NG vs. GN and NN (earlier vs later in the first word). The stress pattern does not show differences from the learners’ group either and reflects the two expected effects in which stressed syllables and new words promote stronger prosodic energy. Information status contrasts are particularly apparent in NG with a proportional reduction in strength on the given element (Syll3) as compared to the new one (Syll1). Akin to the results for the learners’ group, this contrast is less clear in GN, with strong mass values across the first three syllables, reducing both stress and information status contrasts. Overall, there were no major differences between the Italian monolingual control group and the Italian learner group.

\begin{figure}
\includegraphics[width=.6\textwidth]{figure_2_7_Violins Italian monolinguals (control group).pdf}
\caption{Aggregated values of synchrony, ${\Delta}$F0 and mass (on the y-axes) pooled across Italian monolinguals (control group). The x-axis displays the four syllables of the noun phrases, with Syll1 and Syll2 being the noun and Syll3 and Syll4 the adjective. Information structure conditions are colour-coded: green for given-new (GN), blue for new-new (NN) and red for new-given (NG).}
\label{fig:2.7}
\end{figure}

The similarity observed between the two native Italian groups (learners vs. monolinguals) was confirmed using Bayesian models for each acoustic parameter, suggesting no robust difference across groups overall. The only between-group difference found in a position relevant for the two F0 contours concerns ${\Delta}$F0, which has higher values in the NG condition on Syll1 and lower values on Syll2, compared to the learners’ group. This shows that the control group uses a wider range of F0 values for the early peak contour, starting higher on Syll1 and reaching lower values on Syll2, which, in turn, is indicative of a steeper slope. However, this variability across groups might not be directly linked to the linguistic background of the speakers. Instead, it could also be ascribed to idiosyncratic differences, given that ${\Delta}$F0 values are normalised on each speaker’s range and not across speakers.

\subsection{Summary}
\label{sec:2.3.3}
To summarise, both the data and the models strongly support our claim that (Neapolitan) Italian does prosodically mark post-focal given elements within noun phrases through the continuous modulation of acoustic parameters, which did not emerge in previous categorical analyses of Italian. In particular, NG is realised with a different F0 shape as opposed to GN and NN, i.e. with an F0 peak early (vs. late) in the first word. These F0 shapes are accompanied by distinct modulations of prosodic strength: along with the earlier F0 peak, NG displays stronger energy on the first syllable as compared to GN and NN. Moreover, prosodic strength in the NG condition exhibits the expected stress patterns (stressed syllables are stronger than unstressed ones), as well as the expected information status pattern (despite both stressed syllables displaying values in the domain of strong energy, the stressed syllable of the new word is stronger than the stressed syllable of the given word).

\newpage
The minor differences found across the two groups did not generally result in different intonation or strength patterns. Overall, it can be confidently claimed that the two Italian native groups, i.e. the Italian learners of German and the monolingual Italian control group, produced prosodically very similar utterances. This result reassures us that the German-language dominant setting and the German linguistic knowledge of Italian learners of L2 German did not exert an influence on their native Italian speech. As a consequence, I will take into account all native Italian data, i.e. by both learners and monolinguals, when discussing native Italian speech itself (as in \sectref{sec:2.4}), while I will use the Italian spoken by learners themselves as a baseline for comparing their interlanguage to their native and target languages (in \sectref{sec:2.7}).

\section{Perceptual validation for Italian L1}
\label{sec:2.4}
The present finding that Italians use two different intonation contours to mark new vs. post-focal given information in the last position of a noun phrase diverges from previous results on Italian claiming that no prosodic marking of information status is realised within NPs in production. This was previously confirmed by a perception experiment in which Italian listeners could not match the NPs to their information structure from the acoustic signal and in absence of any other contextual cues (\citealt{KrahmerSwerts2008}). With this background in mind, a similar small-scale perception experiment was conducted to ensure the reliability of the elicited data. The aim is to verify the representativeness of the noun phrases elicited and to explore whether listeners make use of the prosodic distinctions found in production. Afterwards, an acoustic analysis of the items that successfully communicate the pragmatic function, i.e. correctly matched to their information structure, was carried out to see whether the trends identified in production are confirmed in perception.

\subsection{Procedure}
\label{sec:2.4.1}
The perception test included 754 output noun phrases of the production experiment in L1 Italian as auditory stimuli. From a total of 901 items, 147 items presenting disfluency phenomena (empty or filled pauses and final lengthening between the two words of the noun phrase) and extra-linguistic events (laughing, coughing, tongue clicks, microphone noises, etc.) were excluded to avoid interference with listeners’ judgment. The exclusion criteria listed were strictly followed and, in line with the choice made for the production experiment, no item was excluded on the basis of the experimenter’s subjective impression of a “good” or “bad” item.

The audio files serving as stimuli were normalised at -23LUFS (a standard reference level for loudness) and their sequence was randomised. To avoid a learning effect, no more than two items with the same information structure appeared sequentially. The online questionnaire was generated using SoSci Survey \citep{Leiner2019} and was made available to users via \url{www.soscisurvey.de}. At the beginning of the questionnaire, participants were explicitly requested to use headphones and to keep the volume level constant throughout the experiment.

The participants were three native speakers of Neapolitan Italian (the same variety as the Italian production data), who were trained phoneticians and not familiar with the present study. They were asked to listen to a sequence of elicited noun phrases composed of a noun and an adjective, and to indicate which element they perceived as new. A written example dialogue from the elicitation game for each information structure condition (reproduced here along with its English translation) was included in the instructions to clarify the context in which the items used as stimuli were produced:\\

\textit{Giocatore A: Hai una mela nera?}

Player A: Do you have a black apple? \medskip

\textit{Giocatore B \textbf{(condizione dato-nuovo)}: No, ho una mela rossa.}

Player B \textbf{(given-new condition)}: No, I have a red apple. \medskip

\textit{Oppure:}

or: \medskip

\textit{Giocatore B \textbf{(condizione nuovo-nuovo)}: No, ho una pera rossa.}

Player B \textbf{(new-new condition)}: No, I have a red pear. \medskip

\textit{Oppure:}

or: \medskip

\textit{Giocatore B \textbf{(condizione} \textbf{nuovo-dato)}: No, ho una pera nera.}

Player B \textbf{(new-given condition)}: No, I have a black pear.\\

For perceptual judgment, participants could choose among the following options: 1) noun, 2) adjective, 3) both noun and adjective or option 4) “I do not know”. The test was preceded by a training phase with four practice trials, followed by one single session that could be paused and resumed again at any time.

\subsection{Results}
\label{sec:2.4.2}
Recall that the analysis of the production data had revealed two distinctive trends across conditions, i.e. an early vs. late F0 peak on the first word for NG vs. GN and NN information structures, respectively. In other words, when the last element of the NP is post-focal and given, it seems to be marked by an alignment of the F0 peak within the first syllable, while the F0 peak is aligned later in the first word when the last element of the NP is new.

The results of the perception experiment are shown in \figref{fig:2.8}. The intended information structure in which the stimuli were elicited during the game is on the y-axis, while the ratings given by listeners are on the x-axis. From the percentages yielded by the matrix, it emerges that GN and NN items are frequently confused, whereas agreement is consistently more robust for NG items, reflecting the two F0 patterns found in production. In particular, the matrix shows that 53.8\% of GN stimuli ware recognised as such and 42.7\% ware matched to a NN information structure. Likewise, 50.3\% of NN stimuli were recognised as such and 45.9\% were matched to GN. For both GN and NN, judgements were thus nearly at chance level. This result from perception corresponds to the results of the acoustic analysis showing that GN and NN share a very similar prosodic contour that makes it difficult to reliably associate the stimuli to one of the two pragmatic conditions. Moreover, both GN and NN stimuli were matched to NG only in 1\% of cases, meaning that listeners could discriminate the shared contour with a later F0 peak on the first word very well and did not confuse it with NG. NG was correctly recognised in 64.1\% of cases and, only matched to GN and NN in a minority of cases (14.4\% and 20.8\%, respectively). An explanation for these mismatches is offered by the acoustic analysis reported in the following paragraph, showing that the two F0 contours can indeed be easily distinguished in perception.


\begin{figure}
\includegraphics[width=.75\textwidth]{figure_2_8_confusion matrix_green.pdf}
\caption{Ratings from the perception test pooled across listeners. The actual information structure of stimuli elicited in production is displayed on the y-axis. The information structure matched to the stimuli by listeners is displayed on the x-axis.}
\label{fig:2.8}
\end{figure}

\subsection{Acoustic analysis of correctly identified items}
\label{sec:2.4.3}
In total, 248 items were correctly matched to the intended information structure by all three listeners. Of these, 41 were GN, 36 NN and 171 NG noun phrases, confirming greater success in correctly identifying NG. The acoustic analysis of these items represents a complementary explanation for the results of the perceptual validation test, both for cases of correct and incorrect attribution. In particular, it offers a clearer overview of how far the acoustic parameters are differently modulated in the two F0 patterns to be perceptually discriminated and, in turn, shed light on the acoustic characteristics of the items which were not correctly matched to information structure.

\figref{fig:2.9} displays the values of synchrony and ${\Delta}$F0 for the stressed syllable of the last element in the NP (Syllable 3) of the correctly identified items and, for comparison, of all production data. Syll3 was chosen because both the distributions of synchrony and ${\Delta}$F0 show the effects of the different F0 peak alignments across conditions. Specifically, the F0 fall is already completed before Syll3 in the case of the early peak (NG), meaning that synchrony and ${\Delta}$F0 values should mainly be distributed around zero indicating no F0 movement at this location. In contrast, the F0 fall takes place on Syll3 in the case of the late peak (GN and NN) so that synchrony and ${\Delta}$F0 will be distributed in the domain of negative values, indicating a falling F0 within Syll3 and across Syll2 and Syll3, respectively. Indeed, this is the pattern found after filtering out items that were not reliably matched in the perception test, visible in the lower panel, where the differences in synchrony and ${\Delta}$F0 are strengthened. The more negative values for the NG condition (${\Delta}$F0 values below -20\% and synchrony values below -10) yielded by NG items realised with a late F0 peak and visible in the upper panel almost disappear in the lower panel. This means that those instances of NG which were not realised with an earlier F0 peak, were also not recognised as conveying NG information structure and, instead, fell under the minority of NG cases matched to GN and NN information structures. I speculate on the reasons for such occurrences in this corpus in the discussion.



\begin{figure}
\includegraphics[width=.6\textwidth]{figure_2_9_violins_sync_del_syll3-cropped.pdf}
\caption{Synchrony and ${\Delta}$F0 values at syllable three for all production data (upper panel) vs. only items that were correctly matched to the intended information status (lower panel). The x-axis displays syllable three, the last lexically accented syllable in the noun phrase. Information structure conditions are colour-coded: green for given-new (GN), blue for new-new (NN) and red for new-given (NG).}
\label{fig:2.9}
\end{figure}

The aggregated values of mass were also examined more closely (\figref{fig:2.10}). Unlike synchrony and ${\Delta}$F0, for which the trends in the large dataset were confirmed and strengthened in the perception-based reduced dataset, the values of mass for the reliably matched items reveal new patterns that were not evident before. A reduction in strength on the given element had already been observed within the NG condition (Syll1 > Syll3), while GN displayed mostly strong energy on both stressed syllables, thus appearing to attenuate information status distinctions between the two words of the noun phrase (visible in the upper panel). Instead, the perception-based dataset in the lower panel shows a similar attenuation of the given element also within the GN condition, where the first syllable, even if lexically stressed, presents mostly weak energy and, therefore, contrasts with the stressed syllable of the new word (Syll3) bearing strong mass, in particular.

As a result, the reduced dataset clearly shows, for both NG and GN, a proportional reduction of mass on the given element in comparison to the new element. However, the distribution of values on the given elements across GN and NG (Syll1 and Syll3, respectively) differs in the extent of their proportional reduction in energy compared to the respective new elements (Syll3/Syll1). In particular, the stressed syllable of the given word in NG (Syll3, corresponding to the nuclear position) still shows a distribution in the domain of strong mass values, while values of the stressed syllable of the given word in GN (Syll1, corresponding to a pre-nuclear position) are mostly distributed in the domain of weak mass values. This difference can also be seen in their mean values, represented by the black dots on the violins: below one on Syll1 in GN and above one on Syll3 in NG.

  

 

\begin{figure}
\includegraphics[width=.6\textwidth]{figure_2_10_violins perception mass.pdf}
\caption{Aggregated values of mass for all Italian L1 production data (upper panel) vs. only items that were correctly matched to the intended information status (lower panel). The x-axis displays the four syllables of the noun phrases, with Syll1 and Syll2 being the noun and Syll3 and Syll4 the adjective. Information structure conditions are colour-coded: green for given-new (GN), blue for new-new (NN) and red for new-given (NG).}
\label{fig:2.10}
\end{figure}

Differences in mass values for the correctly matched items was further tested by fitting a Bayesian hierarchical linear model on this reduced data set, which demonstrated that the observed patterns are statistically robust. In the GN condition, the given element (Syll1) is reduced compared to the new element (Syll3) (δ = 0.28, CI [0.17; 0.39], P(δ > 0) = 1). Moreover, at the same location (Syll1), the stressed syllable of the given element in GN is robustly weaker than the stressed syllable of the new element in NN (δ = 0.35, CI [0.28; 0.41], P(δ > 0) = 1).

This finding suggests that speakers may be using prosodic strength when decoding information status. In particular, mass might be especially helpful for discriminating GN from NN. Both conditions present highly similar F0 contours, but the mass distributions on Syll1 show opposite trends, i.e. weak mass when corresponding to given information status (first word in GN) and strong mass when corresponding to new information status (first word in NN). 

\subsection{Discussion}
\label{sec:2.4.4}
A small-scale perception study was conducted to verify the reliability of the production data. Overall, ratings suggest that listeners do make use of the distributed modulation of prosodic parameters used by speakers to mark post-focal given information and distinguish between different information structure conditions.

In line with the production data, ratings show that two F0 patterns can be identified in perception: a late F0 peak in the first word of the NP for GN and NN conditions, which share this highly similar F0 contour and are often confused with each other, and an early F0 peak on the first syllable for NG, which is correctly identified more often. A further acoustic analysis of the items correctly matched to their information structure has shown that the NG items which were matched to a GN or NN condition were realised with a later F0 peak alignment. A possible reason why these items are found in the corpus could be that marking information status prosodically was not strictly necessary for the successful transmission of the linguistic message in the context of our elicitation game. Therefore, in a minority of cases, speakers did not prosodically mark the new information shifted from the nuclear position in NG and, instead, realised an unmarked prosodic contour, in which new information corresponds to the nuclear position as well, as in GN and NN conditions. With regard to prosodic strength, the acoustic analysis of the correctly matched items confirmed the pattern found in production for NG, and revealed a new pattern for GN. In production, a proportional attenuation according to information status could be observed only for the NG condition, with both stressed syllables being strong, but with the one corresponding to the given word (Syll3) being less strong than the one corresponding to the new word (Syll1). For GN, a similar attenuation of the given element via mass was revealed in the perception-based, reduced dataset: looking at the items which where correctly matched to GN by all three raters, the GN items correctly identified are those with the lowest energy values on the given element (Syll1), with a distribution mostly shifted towards the domain of weak energy.

This result can be explained by the fact that the NG information structure is marked by an anticipated F0 peak which is apparently very salient and easily recognisable in perception. Therefore, any other cue would be redundant. Differently, GN and NN conditions, which share the new element in last position, but differ for the information status of the first element, are realised with a highly similar F0 contour so that listeners seem to rely on broad-range energy modulations, when present, to distinguish these two information structure categories.

The measure used to operationalise prosodic strength, i.e. periodic energy mass, results from a calculation accounting for duration and intensity. Thus, it can be taken as one cue to the postfocal attenuation of given information, which has previously been interpreted as deaccentuation when entire phrases are postfocal. Interpreting strong energy as a possible cue to accentuation means that these findings are in line with previous studies reporting some occurrences of prefocal deaccentuation in Italian (\citealt{AvesaniEtAl2015}). This is explained by the fact that the postfocal given position in NG corresponds to the metrical head of the noun phrase, which has to bear an accent. On the contrary, an accent corresponding to the prefocal given position in GN is optional (as suggested by \citealt{AvesaniEtAl2015}; \citealt{BocciAvesani2011}).

Finally, particularly relevant methodological implications need to be mentioned. Despite including noise in the acoustic analysis of our production data, consistent and robust patterns could be identified in production and confirmed in perception. This shows that it is possible to 1) reduce the experimenter’s control on the task output to favour a more spontaneous and interactive behaviour in the experimental setting (in our case the semi-spontaneous board game resulted in a few lost items due to some speakers’ spontaneous reactions to the events of the game, but allowed us to observe phenomena which were not previously identified); 2) have a more ecologically valid approach without excluding data considered not representative or stereotypical since they are a possible and real outcome; 3) reliably use periodic-energy-based metrics, which were shown to be robust to noise since the trends found in the entire corpus are confirmed and even more plainly manifested for the correctly matched items only. Of course, this being a small-scale experiment with expert listeners as raters, these results have to be taken as preliminary. Since this study is aimed at exploring whether the modulation of continuous prosodic parameters reveals patterns for the marking of postfocal givenness in Italian learners of German, I will not deal with this issue in greater depth within this context. However, the preliminary perceptual results warrant further investigation as well as large-scale replication with naive listeners to examine to which extent speakers robustly produce and listeners robustly attend to patterns of weak or strong mass on the first syllable of the NP as a strategy for discerning the GN and NN conditions.

\subsection{Summary}
\label{sec:2.4.5}
A production experiment and a perceptual validation test were conducted to explore whether and how Italian speakers (Neapolitan variety) use continuous prosodic parameters to mark information status within noun phrases. The present findings contrast with previous studies based on a categorical analysis of accentuation, in which it was concluded that Italians neither mark information status in production, nor decode it in perception. The analysis presented here, based on periodic-energy-related measures, has brought to light that Italian speakers (both learner and control groups) do mark post-focal given information within noun phrases. However, they do so early in the phrase, modulating the F0 shape on the first word, thus anticipating the upcoming given item. This result is further validated by the results of a perceptual evaluation, showing that NG information structure was the only one matched to prosodic contours presenting an early F0 fall on the stressed syllable of the first element of the NP.

\section{Results: German L1}
\label{sec:2.5}
\figref{fig:2.11} shows three example periograms (\sectref{sec:2.2.3}), along with the three acoustic metrics, for each of the three information structure conditions uttered by German native speakers, who represent the ideal target of the Italian learners of L2 German. These periograms show F0 and mass patterns which, according to the literature, are typical in this context, with an F0 peak and an increase in energy marking new or focused elements. In the GN and NG conditions, a peaking F0 movement corresponds to the stressed syllable of the new element (Syll3 in GN and Syll1 in NG), together with strong energy. In the NN condition (where both words of the noun phrase are new) two peaking F0 movements – the second lower than the first due to declination – are realised on the two stressed syllables (Syll1 and Syll3) and accompanied by increased energy.

\begin{figure}
\subfigure[GN. The token is \textit{blaue Birne} (Eng. `blue pear').]{
\includegraphics[width=.8\textwidth]{figure_2_11_a_periogram_blaue birne.pdf}
}
\subfigure[NN. The token is \textit{braune Vase} (Eng. `brown vase').]{
\includegraphics[width=.8\textwidth]{figure_2_11_b_periogram_braune Vase.pdf}
}
\subfigure[NG. The token is \textit{graue Nonne} (Eng. `grey nun').]{
\includegraphics[width=.8\textwidth]{figure_2_11_c_periogram_graue Nonne.pdf}
}
\caption{Example periograms displaying F0 and periodic energy for two-word noun phrases in three information structure conditions produced by L1 German speakers.}
\label{fig:2.11}
\end{figure}

In line with previous findings, these three example utterances show that German native speakers prosodically mark information status by means of an F0 peak and strong energy corresponding to the lexical accented syllable of the new or focused element. Conversely, the lexical accented syllable of the given element is attenuated by means of flat F0 and reduced energy.

At this point, a side note on variability for the native German corpus is in order (see \figref{fig:A3} in the Appendix for the variable L1 German by-speaker contours and compare to the very homogenous L1 Italian by-speaker contours in Figures \ref{fig:A1}--\ref{fig:A2}). The example periograms shown in \figref{fig:2.11} display the three most typical realisations described in the literature. However, alternative prosodic realisations were also found in the corpus due to speaker-specific preferences. The highest variability is encountered for GN and NN conditions (see example periograms in Figures \ref{fig:2.12} and \ref{fig:2.13}), while the pattern described above for NG (with the F0 peak located on the first word and a flat, low-energy post-focal region) was the most consistent across speakers. In particular, GN and NN can display a hat pattern (\figref{fig:2.13}), with high F0 stretching across the two elements of the NP and falling on the third syllable, i.e. after having reached the lexically stressed syllable of the second word.

\begin{figure}
\subfigure[GN with falling contour. The token is \textit{blaue Blume} (Eng. `blue flower').]{\includegraphics[width=.8\textwidth]{figure_2_12_a_periogram_blaue blume.pdf}}

\subfigure[NN with peak on the first word. The token is \textit{graue Birne} (Eng. `grey pear').]{\includegraphics[width=.8\textwidth]{figure_2_12_b_periogram_graue Birne.pdf}}

\subfigure[NN with peak on the second word. The token is \textit{graue Blume} (Eng. `grey flower').]{\includegraphics[width=.8\textwidth]{figure_2_12_c_periogram_graue Blume.pdf}}

\caption{Example periograms in L1 German displaying F0 and periodic energy for alternative contours in GN and NN conditions.}
\label{fig:2.12}
\end{figure}

\begin{figure}
\subfigure[GN. The token is \textit{graue Dose} (Eng. `grey can').]{\includegraphics[width=.8\textwidth]{figure_2_13_a_periogram_graue Dose.pdf}}

\subfigure[NN. The token is \textit{graue Blume} (Eng. `grey flower').]{\includegraphics[width=.8\textwidth]{figure_2_13_b_periogram_graue Blume.pdf}}

\caption{Example periograms in L1 German displaying F0 and periodic energy for the hat pattern in GN and NN conditions.}
\label{fig:2.13}
\end{figure}

Aggregated values of synchrony, ${\Delta}$F0 and mass (\figref{fig:2.14}) reflect the NG contour displayed in \figref{fig:2.11} and the GN/NN hat pattern shown in \figref{fig:2.13}:

\begin{itemize}
\item \textit{Synchrony.} In GN and NN, large distributions including negative and positive values on Syll1 and Syll3 reflect the variability described above, i.e. either Syll1 or Syll3 can present an F0 peak. Mean values near zero reflect the hat pattern across the two words until the second lexically stressed syllable, after which clearly negative distributions on Syll4 for both conditions indicate the F0 fall. In NG, mostly negative values are already found on Syll2, showing a falling movement that starts earlier than in GN and NN.
\item \textit{${\Delta}$F0.} In GN and NN, negative distributions on Syll4 signal that the F0 fall is realised across Syll3 and Syll4 (before this position, values around zero reflect a somewhat plateauing pattern). In NG the negative distribution on Syll3 signals an earlier location of the F0 fall across the two words, namely between Syll2 and Syll3.
\item \textit{Mass.} Mass patterns reflect stress-related distinctions only in the GN and NN conditions, with stressed syllables (Syll1 and Syll3) being strong and unstressed ones (Syll2 and Syll4) weak. In NG, stress-related distinctions on the second word are neutralised by weak mass on Syll3, which, in turn, reflects information status contrasts, as Syll3 is the stressed syllable of the given word. Information-status-related distinctions can be seen to a lesser extent in GN, where both Syll1 and Syll3 present strong mass, but Syll3 is relatively stronger than Syll1. Even if the contrast is not observable for NN (with both words having the same information status), it is worth noticing that distributions on both Syll1 and Syll3 are similar independently of their position in the noun phrase (nuclear or pre-nuclear), which might be a further indication that mass is primarily modulated according to information status.
\end{itemize}


\begin{figure}
\includegraphics[width=.6\textwidth]{figure_2_14_violins GL1.pdf}
\caption{Aggregated values of synchrony, ${\Delta}$F0 and mass (on the y-axes) pooled across German native speakers. The x-axis displays the four syllables of the noun phrases, with Syll1 and Syll2 being the adjective and Syll3 and Syll4 the noun. Information structure conditions are colour-coded: green for given-new (GN), blue for new-new (NN) and red for new-given (NG).}
\label{fig:2.14}
\end{figure}

These trends relative to synchrony, ${\Delta}$F0 and mass were found to be robust by means of Bayesian analyses:

\begin{itemize}
\item \textit{Synchrony} on Syll3 is lower in NG than in GN (δ = 2.4, CI [0.45; 4.71], P (δ > 0) = 0.98); no robust difference to NN was found. On Syll4, synchrony is higher in NG than both GN (δ = 6.23, CI [4.40; 8.06], P (δ > 0) = 1) and NN (δ = 7.03, CI [5.21; 8.84], P (δ > 0) = 1).
\item \textit{${\Delta}$F0} on Syll3 is lower in NG than in GN (δ = 13.55, CI [6.97; 20.09], P (δ > 0) = 1) and NN (δ = 8.74, CI [2.05; 15.14], P (δ > 0) = 0.99). On Syll4, ${\Delta}$F0 is higher in NG than GN (δ = 11.83, CI [7.05; 16.47], P (δ > 0) = 1) and NN (δ = 12.31, CI [7.26; 17.59], P (δ > 0) = 1).
\item \textit{Mass} on Syll3 is lower in NG than GN (δ = 0.37, CI [0.25; 0.50], P (δ > 0) = 0.99) and NN (δ = 0.31, CI [0.19; 0.44], P (δ > 0) = 0.99). Within GN Syll3 is higher than Syll1 (δ = 0.13, CI [0.01; 0.27], P (δ > 0) = 0.97), while the opposite is true for NG, with Syll1 being higher than Syll3 (δ = 0.69, CI [0.53; 0.86], P (δ > 0) = 1). In NN, mass is highly similar across Syll1 and Syll3 (δ = 0.01, CI [-0.11; 0.14], P (δ > 0) = 0.60).
\end{itemize}

The models confirm the patterns identified through data visualisation and, in line with previous findings, show that prosodic marking of information status in native German is tendentially achieved via high F0 and strong energy. On average, GN and NN present a similar intonation contour, whereas the contour for NG differs, reflected in distinct values of synchrony and ${\Delta}$F0 on Syll3 and Syll4. Moreover, stronger energy corresponding to new information status is mirrored in mass values across all conditions, with the stressed syllable of the new element being strong and relatively stronger than the given element within the same condition.

\subsection{Summary}

The results are in line with previous studies on L1 German, showing that prosodic marking of information status is achieved by different modulations of F0 and energy across new and given elements. Despite some variability, overall F0 starts falling after a new element, in the form of either an F0 peak on the new element (typically when the first element is new, i.e. in NG), or a high plateau until the new element (when the second element is new, i.e. in GN and NN). Energy patterns reflect information status contrasts across all conditions (i.e. stronger energy corresponding to the stressed syllable of new elements as compared to given elements), and stress contrasts only in GN and NN conditions (i.e. stressed syllables displaying strong energy and unstressed syllables weak energy). Stress-related distinctions do not hold for NG, showing that, in the presence of post-focal given material, information status marking overrides stress marking and prosodic attenuation takes place, in line with the phenomenon of deaccentuation widely described in the literature.

\section{Results: German L2}
\label{sec:2.6}

\figref{fig:2.15} displays three example periograms for the three information structure conditions uttered by Italian learners of German in their L2. The periograms show two different F0 patterns across conditions. In the GN and NN conditions, F0 is rising throughout the first syllable, reaches a peak on the second syllable and finally falls on the third syllable. In NG the peak is fully realised on the first syllable, where the falling movement takes place as well. As a consequence of the different location of the F0 peak within the first word, the contour shapes are dissimilar across the two words, that is between the second and the third syllable. This position displays a fall in the GN and NN conditions, while in NG it is quite flat, with the fall already completed beforehand. Therefore, the presence of new vs. given information in the final position of the noun phrase is characterised by two different F0 patterns distinguished by the position of the F0 peak: later in the first word in GN and NN, and earlier in the first word in NG.


\begin{figure}
\subfigure[GN. The token is \textit{graue Vase} (Eng. `grey vase').]{\includegraphics[width=.8\textwidth]{figure_2_15_a_periogram_graue Vase.pdf}}

\subfigure[NN. The token is \textit{braune Dose} (Eng. `brown can').]{\includegraphics[width=.8\textwidth]{figure_2_15_b_periogram_braune Dose.pdf}}

\subfigure[NG. The token is \textit{braune Birne} (Eng. `brown pear').]{\includegraphics[width=.8\textwidth]{figure_2_15_c_periogram_braune Birne.pdf}}

\caption{Example periograms displaying F0 and periodic energy for two-word noun phrases in three information structure conditions produced by Italian learners of German as L2.}
\label{fig:2.15}
\end{figure}

The aggregated data of synchrony, ${\Delta}$F0 and mass in \figref{fig:2.16} confirm these observations. In greater detail:

\begin{itemize}
\item \textit{Synchrony}. Positive distributions on Syll1 of GN and NN reflect a rising F0 trajectory. In contrast, predominantly negative values on Syll1 reflect the earlier F0 fall in NG. The effect of the later F0 fall in GN and NN is reflected in more negative synchrony values on Syll3 than in NG.
\item \textit{${\Delta}$F0}. Positive values on Syll2 and negative values on Syll3 are indicative of the F0 peak on Syll2 in GN and NN. In contrast, negative values already on Syll2 in NG indicate that the F0 peak is located within Syll1, resulting in less negative values on Syll3 as compared to GN and NN.
\item \textit{Mass}. All conditions present the same pattern and display strong energy on Syll1 and weak energy on Syll3. This pattern does not reflect stress-related distinctions as Syll3 is stressed, nor information status contrasts between NG and GN. Still, the two conditions do not seem to be totally equal and there is a subtle difference in the shape of the distributions on the stressed syllables (Syll1 and Syll3) across NG and GN. It seems that Syll1 in NG (stressed syllable of the new word) is stronger than in GN (stressed syllable of the given word), whereas Syll3 in NG is weaker (stressed syllable of the given word) than in GN (stressed syllable of the new word).
\end{itemize}


\begin{figure}
\includegraphics[width=.6\textwidth]{figure_2_16_violins L2.pdf}
\caption{Aggregated values of synchrony, ${\Delta}$F0 and mass (on the y-axes) pooled across German L2 learners. The x-axis displays the four syllables of the noun phrases, with Syll1 and Syll2 being the adjective and Syll3 and Syll4 the noun. Information structure conditions are colour-coded: green for given-new (GN), blue for new-new (NN) and red for new-given (NG).}
\label{fig:2.16}
\end{figure}

Results of the Bayesian analysis support these observations, including those of the more subtle modulations of mass values:

\begin{itemize}
\item \textit{Synchrony} on Syll1 is higher in GN (δ = 4.44, CI [3.39; 5.62], P (δ > 0) = 1) and NN (δ = 5.41, CI [4.22; 6.48], P (δ > 0) = 1) than in NG. In contrast, synchrony on Syll3 is lower in GN (δ = 1.62, CI [0.52; 2.78], P (δ > 0) = 0.99) and NN (δ = 2.20, CI [1.06; 3.36], P (δ > 0) = 1) compared to NG.
\item \textit{${\Delta}$F0} on Syll2 is higher in GN (δ = 19.52, CI [13.83; 25.73], P (δ > 0) = 1) and NN (δ = 20.12, CI [14.3; 26.66], P (δ > 0) = 1) than in NG. In contrast, ${\Delta}$F0 on Syll3 is lower in GN (δ = 7.15, CI [2.99; 11.66], P (δ > 0) = 0.99) and NN (δ = 8.20, CI [3.8; 12.47], P (δ > 0) = 1) than in NG.
\item \textit{Mass} on Syll1 is lower for GN (δ = 0.10, CI [0.05; 0.15], P (δ > 0) = 1) and NN (δ = 0.07, CI [0.02; 0.11], P (δ > 0) = 0.99) compared to NG. On Syll3 GN (δ = 0.05, CI [0.006; 0.10], P (δ > 0) = 0.98) is higher than NG.
\end{itemize}

Statistical results support the data showing two different F0 contours distinguished by the position of the F0 peak: within the first syllable in NG and later in the first word in GN and NN. One pattern of modulation for mass across all conditions was found, namely strong mass on the first word and weak mass on the second word, which pattern matches information status contrasts only in the NG condition.

\subsection{Proficiency levels}
\label{sec:2.6.1}
Since learners had different proficiency levels, it is interesting to assess whether and to which extent their ability to mark information status prosodically improves in the learning process, that is, across proficiency levels. To this end, learners were categorised into two main groups, i.e. beginner and advanced learners. In the previous section, it emerged that GN and NN present similar patterns for the three measures of synchrony, ${\Delta}$F0 and mass, in all language groups. Therefore, when exploring results across proficiency levels and, successively, across language groups, I will only comment on the difference between GN and NG and leave out the all-new condition to simplify the presentation of the results and increase clarity.

\figref{fig:2.17} shows that both proficiency groups produce the same patterns for F0 and mass modulation. Subtle differences across proficiency levels can be spotted, with beginners displaying less discrimination between conditions as compared to advanced learners. In particular:

\begin{itemize}
\item \textit{Synchrony}. In the beginner group, the NG distribution on Syll1 presents a similar amount of positive and negative values, whereas the distribution tends to be more negative in the advanced group, showing a better discrimination of the earlier F0 fall in the NG condition. On Syll3, while mean values for NG and GN appear to be similar across proficiency groups, the distribution of data is less spread out in the advanced compared to the beginner group.
\item \textit{${\Delta}$F0}. In the beginner group, values on Syll2 and Syll3 tend to be closer to zero for both conditions than in the advanced group, in which values for GN and NG tend towards opposite directions. As a result, beginners present lower values for GN and higher values for NG than advanced learners on Syll2, while on Syll3, beginners present higher values for GN than advanced learners. This shows that beginners use a reduced range when modulating F0 across syllables and conditions.
\item \textit{Mass}. The subtle contrasts regarding information status on Syll1 and Syll3 seem to be enhanced in the advanced group. In particular, for GN Syll1 is lower and Syll3 higher than in the beginner group.
\end{itemize}

\begin{figure}
\includegraphics[width=.8\textwidth]{figure_2_17_violins by proficiency.pdf}
\caption{Aggregated values of synchrony, ${\Delta}$F0 and mass (on the y-axes) pooled across German L2 learners by proficiency levels. Beginners are on the left and advanced learners on the right. The x-axis displays the four syllables of the noun phrases, with Syll1 and Syll2 being the adjective and Syll3 and Syll4 the noun. Information structure conditions are colour-coded: green for given-new (GN) and red for new-given (NG).}
\label{fig:2.17}
\end{figure}

These observations were confirmed to be statistically robust trends. A Bayesian analysis was used to test all relevant contrasts for the measures of synchrony, ${\Delta}$F0 and mass within groups (i.e. beginners and advanced, see bullet points in \sectref{sec:2.6}), and across groups (see correspondent bullet points above), in order to assess: 1) whether the contrasts among conditions hold true within both proficiency groups and 2) whether the two groups robustly differ from each other.

\begin{itemize}
\item \textit{Synchrony}. NG and GN are robustly different on Syll1 within both proficiency groups, confirming the general tendencies (for beginners: δ = 3.64, CI [2.21; 5.07], P (δ > 0) = 1; for advanced: δ = 5.6, CI [3.95; 7.14], P (δ > 0) = 1) and on Syll3 only within the advanced group (δ = 2.3, CI [0.70; 3.95], P (δ > 0) = 0.99). Across groups, there is compelling evidence only for a difference in GN on Syll1, which is higher in the advanced group (δ = 1.38, CI [-0.10; 3.05], P (δ > 0) = 0.95). However, Syll3 in GN shows a relatively high difference across groups (δ = 1.30, CI [-0.68; 3.27], P (δ > 0) = 0.90) and is lower in the advanced group.
\item \textit{${\Delta}$F0}. NG and GN are robustly different on Syll2 within both proficiency groups, confirming the general tendencies (for beginners: δ = 13.54, CI [6.02; 21.67], P (δ > 0) = 0.99; for advanced: δ = 26.95, CI [19.36; 34.74], P (δ > 0) = 1) and on Syll3 only within the advanced group (δ = 12.7, CI [6.08; 18.67], P (δ > 0) = 0.99). Across groups, there is strong evidence for a difference in NG on Syll2 (δ = 13.03, CI [3.11; 22.87], P (δ > 0) = 0.99) and in GN on Syll3 (δ = 7.37, CI [0.52; 14.2], P (δ > 0) = 0.98), which are lower in the advanced group.
\item \textit{Mass}. NG and GN are robustly different on Syll1 within both proficiency groups, confirming the general tendencies (for beginners: δ = 0.07, CI [0.01; 0.13], P (δ > 0) = 0.98; for advanced: δ = 0.14, CI [0 .07; 0.21], P (δ > 0) = 0.99) and on Syll3 only within the advanced group (δ = 0.08, CI [0.01; 0.16], P (δ > 0) = 0.99). Across proficiency levels, there is a relatively high difference in GN on Syll3 (δ = 0.07, CI [-0.01; 0.17], P (δ > 0) = 0.94), which is higher in the advanced group.
\end{itemize}

The models suggest that most of the between-group differences concern the modulation of synchrony and ${\Delta}$F0 values on the first three syllables. These slight variations across groups seem to suggest that with increasing proficiency, learners better discern the two pragmatic conditions, enhancing a distinct modulation of their phonetic details, although this does not result in different patterns in beginner as compared to advanced learners.

\subsection{Summary}
\label{sec:2.6.2}
In both their L2 and their native language, learners distinguish two intonation patterns through the modulation of the F0 falling movement on the first word, showing a transfer of L1 Italian patterns. As in their native language, they use an F0 peak aligned near the onset of the first syllable to mark the NG condition (i.e. earlier on the first word), whilst they realise the F0 peak across the first two syllables in the GN and NN conditions (i.e. later on the first word). However, learners use energy in a way that is different from both L1 Italian and L1 German. They produce the same pattern in all pragmatic conditions: the first syllable, which is lexically stressed, is strong while the rest of the noun phrase displays a reduction in energy. This shows that energy is never used according to lexical stress patterns and only in the case of NG signals information status differences within the noun phrase.

The analysis by proficiency level has shown that robust differences between beginner and advanced learners are present, but that they are limited and not consistent across measures, syllables, and conditions. Specifically, the advanced group seems to enhance the difference between their NG and GN realisations more as compared to beginners. In contrast, beginners consistently differentiate the two conditions less clearly on the second word by neutralising F0 and energy distinctions across conditions. Most importantly, these differences do not result in overall distinct patterns, and the two proficiency levels still show the same trends, which is an expected result considered that prosody is not explicitly thematised in L2 classrooms. For this reason, I will consider them as one single group in the next section, i.e. when comparing L2 German to L1 Italian and L1 German.

\section{Interlanguage compared to learners’ native and target languages}
\label{sec:2.7}
After having shown in detail the different strategies used by the three groups to prosodically mark information status, it is time to compare learners’ prosodic realisations of new-given (NG) and given-new (GN) information structures with their native baseline, Italian, and target language, German\footnote{In the previous chapters, it emerged that GN and NN present similar F0 and energy patterns in all language groups. Thus, to break down complexity, I will only comment on the difference between GN and NG and leave out the NN condition. All data are available and can be inspected in the OSF repository (\url{https://osf.io/9ca6m/}).} To do so, a preliminary discussion of the main cross-linguistic differences and similarities between the two native languages is necessary to highlight the relevant aspects for learners (see \citealt{RasierHiligsmann2007} for a review of models of second language acquisition). In light of this cross-linguistic comparison, I will then discuss learners’ productions.

\subsection{F0 contours}
\label{sec:2.7.1}
For a direct comparison, \figref{fig:2.18} displays the averaged F0 contours found for the realisation of NG and GN in the three different language groups, i.e. Italian L1, German L2 and German L1.


\begin{figure}
\includegraphics[width=\textwidth]{figure_2_18_averaged f0 contours.pdf}%dont change size; otherwise to small
\caption{Averaged F0 contours pooled across speakers for each language group. The y-axis shows F0 in semitones, while the x-axis shows normalised time aligned at the boundary between the two words of the noun phrase. Syllables of the noun phrase are numbered from one to four and syllable boundaries are marked by vertical black lines. The grey area around the contours represents the standard error and contours are colour-coded according to their information structure condition: green for given-new (GN) and red for new-given (NG).}
\label{fig:2.18}
\end{figure}

Comparing native Italian with native German, it is apparent that the two languages use different strategies to express the contrast between the two pragmatic conditions: L1 Italian prosodically differentiates the two conditions on the first word, while L1 German does so on the second word of the noun phrase. Specifically, in L1 Italian, NG shows an F0 fall early in the first syllable and GN on the second syllable so that in both cases the peak is located within the first word. In L1 German, in contrast, NG has an F0 fall on the second syllable with a peak on the first syllable and GN has a hat pattern, with the F0 fall on the third syllable and high F0 stretching across the first and the second word.\footnote{Even if averaged data for L1 German displays a hat pattern for GN, this does not mean that this contour is the one most used for this condition. Contours with a peak either on the first or on the second element contribute to the mean (see examples in \sectref{sec:2.5}). This can be seen by considering aggregated data for synchrony on Syll2, showing a tendency for slightly falling F0 (\figref{fig:2.14}) as well as by looking at by-speaker contours (\figref{fig:A3} in the Appendix).} Thus, from an Italian learner’s perspective, the timing of the F0 fall in German is later than in their native baseline.

Learners seem to transfer their L1 F0 shapes to their L2 realisations, as the intonation patterns in the L1 and L2 appear to be quite similar,\footnote{Notice that the difference in duration of the first syllable across languages depends on the underlying segmental material, i.e. a different syllabic structure. At the beginning of German NPs there are words like \textit{graue} and \textit{blaue} with two consonants in the onset and a diphthong as the nucleus, while in Italian all syllables are composed of a single consonant and a vowel.} even though they exploit a reduced F0 range as compared to their baseline (similar to the target language, which will be further discussed in \sectref{sec:2.7.3}). This means that learners still distinguish information status within noun phrases, but they do so using a different timing of the F0 fall on the first word and, therefore, do not match the target contours produced by German native speakers. In particular, in NG, learners’ F0 fall takes place too early in the first word compared to the target (on the first syllable instead of the second syllable) and, in GN, learners do not produce a hat pattern across the two words and the F0 fall occurs before the new element (on the second syllable instead of the third syllable). As a result, differences between learners’ realisations and their target language are evident on the first word in the NG condition and on the second word in the GN condition.

\figref{fig:2.19} provides greater detail on F0 modulation across the three language groups, showing again values of synchrony and ${\Delta}$F0 for the relevant locations in the noun phrases in light of the differences observed, i.e. Syll1 and Syll3 for synchrony and Syll2 and Syll3 for ${\Delta}$F0. Distributions and mean values of synchrony and ${\Delta}$F0 for the learner group are midway between their native and target languages (with the exception of synchrony on Syll1 in NG, displaying more negative values than either L1). A Bayesian analysis confirms the robustness of this observation:


\begin{itemize}
\item \textit{Synchrony}. On Syll1 in NG, values for the L2 group are lower than both in the native (δ = 1.09, CI [0.31; 1.81], P(δ > 0) = 0.99) and the target language (δ = 1.30, CI: [0.29; 2.19], P(δ > 0) = 0.99). On Syll3 in GN, values for the L2 group are higher than in the native (δ = 2.66, CI [1.83; 3.43], P(δ > 0) = 1) and lower than in the target language (δ = 6.02, CI: [5.02; 7.04], P(δ > 0) = 1).
\item \textit{${\Delta}$F0}. On Syll2 in NG, values for the L2 group are higher than in the native language (δ = 5.01, CI [2.3; 7.77], P (δ > 0) = 1), but still lower than in the target language (δ = 3.49, CI [0.03; 7.01], P (δ > 0) = 0.97). The same holds true for Syll3 in GN (difference to L1 Italian: δ = 13.89, CI [11.57; 16.24], P (δ > 0) = 1; difference to L1 German: δ = 16.56, CI [13.46; 19.81], P (δ > 0) = 1).
\end{itemize}


\begin{figure}
\includegraphics[width=\textwidth]{figure_2_19_violins sync del comp.pdf}%dont change size, otherwise to small
\caption{Aggregated values of synchrony and ${\Delta}$F0 for the relevant syllables across language groups. Synchrony values are displayed for syllables one and three (left panels), while ${\Delta}$F0 values are shown for syllables two and three (right panels). Information structure conditions are colour-coded and positioned on two separate rows: green for given-new (GN, upper row) and red for new-given (NG, bottom row).}
\label{fig:2.19}
\end{figure}

Despite the fact that learners seem to transfer their L1 contours to their L2, the models provide strong evidence for some differences in their continuous modulation. In particular, in learners’ interlanguage, there is a less steep slope on the first word in the NG condition and a narrower F0 range overall as compared to their native language, with the latter being a feature of native German as well. Still, learners’ F0 patterns mirror their Italian native productions and do not resemble the target ones.

\subsection{Mass}
\label{sec:2.7.2}
The previous measures give an overview of continuous parameters that quantify what is visible in the averaged contours. However, this provides little information about possible accentuation. Thus, I will now compare the use of periodic energy mass across language groups, as mass is derived from a calculation accounting for power and duration, two parameters which are often involved in accentuation together with F0 movement as described in \sectref{sec:2.1.2}. I will focus on the third syllable, as that is the one where deaccentuation should be realised by learners according to pragmatic condition in order to match the accentuation patterns of the target language.

Values of mass for the third syllable of the noun phrase are shown in \figref{fig:2.20}. Focussing first on native Italian and native German, the results provide evidence in line with the literature (\sectref{sec:2.1.4} for Italian, \sectref{sec:2.1.5} for German). L1 Italian displays strong mass on Syll3, both when the second word of the noun phrase is new and when it is given and post-focal, supporting the finding that the final word requires an accent, independent of pragmatic status. By contrast, in L1 German there is strong mass only on Syll3 when the second word of the noun phrase is new, whereas mass is weak on the post-focal given element, showing that, in line with previous studies reporting deaccentuation, the nuclear position is prosodically highlighted or attenuated according to information status.

In learners' L2, mass is clearly weak across conditions, with values similarly distributed below one. The values appear to be even more negatively distributed than in the NG condition by L1 German speakers, also displaying weak mass. As a result, learners do not seem to transfer mass patterns from their native language as they do for F0 and, instead, show prosodic attenuation as in the target language, which might be interpreted as an attempt to reproduce deaccentuation. However, in comparison with L1 German, learners’ mass on Syll3 is weak not only in NG, but also in GN, i.e. they appear to deaccent new information as well.

\begin{figure}
\includegraphics[width=.5\textwidth]{figure_2_20_violins mass comp-cropped.pdf}
\caption{Aggregated values of mass for syllable three across language groups. Information structure conditions are colour-coded and positioned on two separate rows: green for given-new (GN, upper row with Syll3 being a new item) and red for new-given (NG, bottom row with Syll3 being a given item).}
\label{fig:2.20}
\end{figure}

The robustness of the differences observed across language groups were confirmed by Bayesian models:

\begin{itemize}
\item \textit{Mass}. On Syll3 in both conditions, the L2 group presents lower values than both in the native (difference for NG: δ = 0.34, CI [0.29; 0.39], P (δ > 0) = 1; and GN: δ = 0.38, CI [0.33; 0.43], P (δ > 0) = 1) and target language (difference for NG: δ = 0.09, CI [0.03; 0.15], P (δ > 0) = 0.99; and GN: δ = 0.42, CI [0.35; 0.48], P (δ > 0) = 1).
\end{itemize}

Model results confirm that learners’ prosodic strength patterns diverge from their native Italian and tend to reproduce the native German for the NG condition, with the difference that 1) they attenuate the post-focal element even more than native speakers of German and 2) this same pattern is extended to GN as well, showing that they do not use prosodic strength to mark information status contrasts as in their native Italian.

\subsection{Discussion}
\label{sec:2.7.3}
The production of two-word noun phrases with different information status, new-given (NG) vs. given-new (GN), in L2 German spoken by Italian learners was compared to their native language and the target language.

Results show that Italian learners of German differentiate the two pragmatic conditions using two F0 shapes which highly resemble their native Italian ones, i.e. by producing the F0 peak within the first syllable in NG and later in the first word in GN. However, statistical results provide evidence against a complete transfer, showing a systematically different modulation of F0. Specifically, learners produce the anticipated peak in NG with a less steep slope and use a reduced F0 range across conditions. A narrower F0 range is characteristic of L1 German, but the hypothesis that learners intentionally compress their Italian native F0 range to approach the target language is contestable, as other L2 studies involving language pairs different from the one object of this study have made the same observation (Dutch learners of Greek in \citealt{Mennen1998}; Taiwan Mandarin learners of English in \citealt{ViscegliaEtAl2011}; Italian learners of English in \citealt{Urbani2012}; French learners of German and German learners of French in \citealt{ZimmererEtAl2014}; Chinese learners of Japanese in \citealt{ZhangXie2014}). This has been interpreted as a characteristic of interlanguages, and not necessarily as ascribable to transfer or learning effects, but possibly to insecurity in speaking a second language (\citealt{Mennen1998}; \citealt{ZhangXie2014}; \citealt{ZimmererEtAl2014}).

It was also found that, as in their L1, Italian learners of German do not make use of prosodic strength to distinguish the two pragmatic conditions. However, the pattern learners use is not present in their L1 and, instead, might be interpreted as a possible attempt to reproduce a pattern typical of German. In particular, learners strengthen the first word and weaken the second, similarly to L1 German productions in the NG condition. Hereby, learners enhance the attenuation of the second word even more than L1 German speakers, probably as a form of hypercorrection, which has already been documented in L2 phonetic acquisition (cf. \citealt{IversonSong2013}; \citealt{Kelly2022}; \citealt{Petrov2021}). Learners apply this energy pattern across all pragmatic conditions, possibly because in their L1 Italian, prosodic strength is not used to encode information structure (but only positional rules, accenting the final position independently of IS), and they may not be aware of the different function it serves in German. For these reasons, instead of using prosodic attenuation to convey information status, they seem to perceive and reproduce it as a salient feature of native German speech according to positional rules (deaccenting the nuclear position independently of IS).

\subsection{Summary}
\label{sec:2.7.4}
Italian learners of German produce F0 shapes that resemble those in their L1. However, they use a different energy pattern across all conditions, one that is similar to L1 German in the NG condition and can be interpreted as an attempt to reproduce target-like productions. Nevertheless, modelling the phonetic details reveals that learners’ modulations of F0 and energy are significantly different from both their native Italian baseline and the native German target. This indicates that although the patterns appear similar, they are phonetically implemented in different ways, i.e. learners do not fully transfer features from their L1 nor achieve complete alignment with their target language. Instead, they exhibit the expected characteristics of an interlanguage.

\section{Implications for phonological analysis}
\label{sec:2.8}
The majority of recent studies on the mapping between prosody and information status have been carried out within the Autosegmental-Metrical (AM) phonology framework. For this reason, a critical discussion of the present findings based on a phonetic periodic-energy-based approach in relation to the existing literature rooted in a phonological framework is necessary. To confront the different theoretical frameworks and validate the use of this new periodic-energy-based analysis in L2 research, I will compare periodic-energy-related measures to the established measures of alignment, scaling, duration, and intensity, and suggest a possible description of the contours found in terms of categorical pitch accent types. The goal is twofold: to critically discuss the advantages and disadvantages of a phonetic and phonological approach in L2 research and to validate the use of a periodic-energy-based method, especially for L2 analyses.

\subsection{Background}
\label{sec:2.8.1}
Most recent studies on the prosodic marking of information status have been conducted within the framework of AM phonology, using the ToBI (Tones and Break Indices) system for intonation transcription. The system was initially based on American English (\citealt{BeckmanPierrehumbert1986}; \citealt{BeckmanAyers1997}; \citealt{Beckman2005}; \citealt{Shattuck-HufnagelBrugos2006}) and has been fully adapted for German (GToBI: \citealt{GriceEtAl2005}) but only sketched for some Italian varieties \citep{GriceEtAl2005}.

This model sees F0 turning points as tonal targets joined by quasi-linear interpolation. These targets are phonologically represented as tones, with abstract discrete values: H for a high target, also called \textit{peak}, and L for a low target, also called \textit{valley}. In German ToBI (\citealt{GriceEtAl2005}), there are two further operators for H tones to describe the relative height of the target: \textit{downstep}, that is a lower H target compared to the previous H target and is transcribed with an exclamation mark (!H), and \textit{upstep}, that is a higher H target compared to the previous H target, and is transcribed with a caret (ˆH). The \textit{metrical} aspect of the model is reflected in the division of utterances into phrases and the assignment of relative prominence to the elements within the phrases, while the \textit{autosegmental} aspect refers to the fact that tonal and segmental features are considered to be independent and are indeed represented on different, parallel and autonomous levels. The elements on the different levels are then connected by associations at specific points, leading to the anchoring of the tune to the text on single syllables and/or at the edges of phrases. The resulting types of phonological categories are defined as pitch accents and edge or boundary tones, respectively. Pitch accents mark prominence and are typically associated with metrically strong syllables and transcribed with a star (e.g. L*; H*). Edge or boundary tones mark the edges of constituents and are associated with their periphery \citep{GriceEtAl2005}. They are transcribed with a ``-" for edges of minor phrases, also called \textit{intermediate phrases}, and with a ``\%" for edges of major phrases, also called \textit{intonation phrases}.

In the development of AM phonological systems, gradual phonetic parameters have been investigated for their contribution towards establishing phonological categories. Pitch accent categories are often determined on the basis of segmental anchoring of F0 turning points, which involves temporal alignment or synchronisation of F0 target peaks and valleys to landmarks in the segmental string (horizontal axis), such as the \textsc{CV} boundary of an accented syllable (i.e. the onset of the vowel, see \citealt{LaddMennen1998}). Alongside alignment, other gradual parameters typically used to describe pitch accent categories are scaling (vertical axis) – the relative F0 height of the tonal target encoded in accent type, H* being typically lower than L+H* – and pitch excursion – the direction (rising vs. falling) and extent (small vs. large) of the excursion resulting from the scaling and height of pitch and the alignment of a pitch peak or valley with a stressed syllable. Moreover, for German the \textit{tonal onglide} is of particular importance (\citealt{RitterGrice2015}). This describes the F0 movement from the preceding syllable to the tonal target on the accented syllable, often encoded in the leading tone of pitch accents: L+H* for a rising onglide from an L to an H target or H+L* for a falling onglide from an H to an L target.

In the following sections I present an alternative analysis of the collected data using the AM approach. This includes the use of alignment and scaling to explore F0 contours, duration and intensity to examine accentuation patterns, and providing a phonological interpretation using ToBI. The results are discussed in light of my findings based on a periodic-energy-based approach.

\subsection{Alignment and scaling}
\label{sec:2.8.2}
In the study I conducted, the measures of synchrony and ${\Delta}$F0 were used to analyse differences in F0 modulations used to mark information status contrasts. Within the AM framework, alignment and scaling are the two most widely used measures for investigating the phonetic details of tonal targets and describe the F0 contour. Alignment reflects the temporal coordination of tonal targets within the segmental string and scaling reflects their pitch level (for an extensive overview see Chapter 5 in \citealt{Ladd2008}).

To provide a comparative analysis of F0 shapes using these two measures (\citealt{AlbertGrice2019}), alignment is extracted as the time point corresponding to the highest F0 point of rising-falling F0 contours and normalised by token duration (thus, it is represented on a time scale from 0 to 1). Scaling is extracted in Hz as the highest F0 point of rising-falling F0 contours and normalised by each speaker’s range. Representing the alignment of F0 peaks realised within the target noun phrases implies that only rising-falling F0 shapes realised on and around the region of a syllable are accounted for, i.e. F0 contours that are simply rising or simply falling are not included in the analysis.

Figures \ref{fig:2.21}--\ref{fig:2.23} display values of normalised alignment and scaling of F0 peaks realised within the noun phrases under the three information structure conditions: GN (given-new), NN (new-new) and NG (new-given).

\figref{fig:2.21} shows alignment and scaling values for L1 Italian data, with all items visible in the upper panel and only items correctly matched to information status in the lower panel. As already observed in the periodic-energy based analysis (\sectref{sec:2.3}), NG forms a well-separated cluster of data points compared to the other two conditions, which, instead, overlap. This trend is clearer in the lower panel, after filtering out the items which were not identified as corresponding to the intended information structure in the perceptual rating task (\sectref{sec:2.4}). Thus, both methods show that two intonation contours with a different peak alignment within the first word distinguish the information status of the following word, that is, post-focal given corresponding to early alignment on the first word vs. new corresponding to late alignment on the first word.

\begin{figure}
\includegraphics[width=0.8\textwidth]{figure_2_21_Alignment and scaling (L1 Italian).pdf}
\caption{Alignment and Scaling for L1 Italian data (all data, on top, vs. correctly matched items only, on the bottom). Mean time-normalised boundaries of the four syllables of the noun phrases are marked by solid vertical lines. Syll1 and Syll2 form the noun and Syll3 and Syll4 form the adjective. Information structure conditions are colour-coded: green for given-new (GN), blue for new-new (NN) and red for new-given (NG).}
\label{fig:2.21}
\end{figure}

\figref{fig:2.22} shows values for alignment and scaling for L1 German, from which conclusions similar to those in the periodic-energy-based analysis can be derived. In NG, the F0 peak is consistently aligned between the first and the second syllable, that is, on the new word. GN and NN show instead more variability (\sectref{sec:2.5}), with a wide distribution of values showing that the peak can be located either late on the first word or on the second word. However, it is necessary to point out that this figure only shows a reduced subset of data relative to the specific F0 configuration of peaks (42\% of the occurrences). Other configurations found, such as the hat pattern or the double peak, cannot be represented using this measure without making an arbitrary decision as to where the peak is labelled (following \citealt{Welby2004}).


\begin{figure}
\includegraphics[width=\textwidth]{figure_2_22_Alignment and scaling (L1 German).pdf}
\caption{Alignment and Scaling for L1 German data. Mean time-normalised boundaries of the four syllables of the noun phrases are marked by solid vertical lines. Syll1 and Syll2 form the adjective and Syll3 and Syll4 form the noun. Information structure conditions are colour-coded: green for given-new (GN), blue for new-new (NN) and red for new-given (NG).}
\label{fig:2.22}
\end{figure}

The transfer of native F0 contours by Italian learners to their L2 German, assessed by a periodic-energy-based analysis (\sectref{sec:2.6}), is also visible using alignment and scaling. \figref{fig:2.23} shows that, similar to their Italian L1, distributions of data points in L2 German tend to form two clusters, separating the F0 peak of NG, which is aligned earlier in the first word, from the F0 peak of GN and NN, which is aligned later in the first word. Moreover, learners do realise the F0 peak on the second word for some GN and NN instances as in native German, but this is rarely the case as shown by the few green (GN) and blue (NN) data points on Syll3 and Syll4 (second word) as compared to the much more numerous ones on Syll1 and Syll2 (first word).

\begin{figure}
\includegraphics[width=\textwidth]{figure_2_23_Alignment and scaling (L2 German).pdf}
\caption{Alignment and Scaling for L2 German data. Mean time-normalised boundaries of the four syllables of the noun phrases are marked by solid vertical lines. Syll1 and Syll2 form the adjective and Syll3 and Syll4 form the noun. Information structure conditions are colour-coded: green for given-new (GN), blue for new-new (NN) and red for new-given (NG).}
\label{fig:2.23}
\end{figure}

To summarise, even using the established measures of alignment and scaling to quantify the F0 shape, the same F0 patterns identified via synchrony and ${\Delta}$F0 for all language groups clearly emerge. This confirms and validates our results based on periodic-energy-based measures and can be summarised as follows:

\begin{itemize}
\item In L1 Italian, the position of the F0 peak on the first word distinguishes the information status of the second word;
\item In L1 German, the F0 peak typically occurs on the new element. This is very consistent in NG and more variable in GN and NN information structures, for which other contour shapes are also realised;
\item In L2 German, Italian learners reproduce their L1 patterns (with very few exceptions of target-like F0 patterns).
\end{itemize}

\subsection{Duration and Intensity}
\label{sec:2.8.3}
Periodic energy mass, derived from a calculation accounting for duration and power, was used in the present study to analyse prosodic strength patterns as a cue to accentuation, while two well-established measures used to investigate power and length are intensity and duration. For a comparison with mass, the distributions of values of mean intensity (in decibels) and duration (in seconds) are presented for the lexically stressed syllables (Syll1 for the first word, Syll3 for the second word) in the GN and NG conditions for each language group.

Differences among conditions and syllables were statistically tested using Bayesian hierarchical linear models,\footnote{For each language group, the differences among conditions in duration and intensity were tested as a function of the factors \textsc{condition} (reference level \textsc{ng}), \textsc{syllable} (reference level \textsc{syllable 1}) and their interaction. As random effects, the models include random intercepts for \textsc{token} and \textsc{speaker}. I used regularising weakly informative priors \citep{Lemoine2019} for all models (for priors specifications, see the relative RMarkdown file at \url{https://osf.io/9ca6m/}) and ran three sampling chains for 3000 iterations with a warm-up period of 2000 iterations for each model.} with the output reported in parentheses throughout the description of the results.

In L1 Italian (\figref{fig:2.24}), distributions of intensity and duration across the noun phrase seem to each enhance a different word, yielding conflicting results. For both conditions, duration shows that Syll3 is longer than Syll1 (for GN: δ = 0.07, CI [0.07; 0.08], P (δ > 0) = 1; for NG: δ = 0.04, CI [0.03; 0.04], P (δ > 0) = 1), while intensity shows that Syll1 is stronger than Syll3 (for GN: δ = 2.57, CI [2.31; 2.84], P (δ > 0) = 1; for NG: δ = 3.70, CI [3.46; 3.97], P (δ > 0) = 1). Thus, duration appears to enhance the second word and intensity the first word. Moreover, no difference was found across information structure conditions on the second word. In this case, periodic energy mass is useful for overcoming the conflicting patterns, and to show that the new word is stronger than the given one not only within the NG condition, but also in GN (\sectref{sec:2.3.1}).

\begin{figure}%leave the file names of the graphics
\includegraphics[width=.6\textwidth]{figure_2_24_violins dur int IT-cropped.pdf}
\caption{Aggregated values of duration, intensity and mass (for comparison) pooled across Italian L1 speakers. The x-axis displays Syll1, the stressed syllable of the noun, and Syll3, the stressed syllable of the adjective. Information structure conditions are colour-coded: green for given-new (GN) and red for new-given (NG).}
\label{fig:2.24}
\end{figure}

In L1 German (\figref{fig:2.25}), intensity and duration across syllables and conditions follow far more similar trends than in L1 Italian, but these are mostly subtle modulations. Duration is slightly increased on new elements, but the trend is not robust (Syll1 in NG > Syll1 in GN: δ = 0.02, CI [-0.01; 0.06], P (δ > 0) = 0.91; Syll3 in NG < Syll3 in GN: δ = 0.01, CI [-0.01; 0.05], P (δ > 0) = 0.86). Intensity is decreased on the post-focal given element in the NG condition only (Syll3 < Syll1 in NG: δ = 3.83, CI [2.83; 4.70], P (δ > 0) = 1). In comparison to these results, mass seems to most clearly reveal an evident attenuation on the post-focal given element (\sectref{sec:2.5}).

\begin{figure}
\includegraphics[width=.6\textwidth]{figure_2_25_violins dur int GL1-cropped.pdf}
\caption{Aggregated values of duration, intensity and mass (for comparison) pooled across German L1 speakers. The x-axis displays Syll1, the stressed syllable of the adjective, and Syll3, the stressed syllable of the noun. Information structure conditions are colour-coded: green for given-new (GN) and red for new-given (NG).}
\label{fig:2.25}
\end{figure}

In L2 German (\figref{fig:2.26}), learners use duration in a unique way that does not resemble either L1 Italian or German, in attenuating the second word of the NP in both information structure conditions by almost halving its duration as compared to the first word (Syll1 > Syll3 in GN: δ = 0.22, CI [0.20; 0.24], P (δ > 0) = 1; in NG: δ = 0.23, CI [0.21; 0.25], P (δ > 0) = 1). Intensity contributes to the reduction of post-focal material analogously to their target language, but this is not comparable to the extreme modulation of duration values, which is exploited more by learners (Syll3 < Syll1 in NG: δ = 1.78, CI [1.39; 2.14], P (δ > 0) = 1). These trends are well summarised also by the measure of mass (\sectref{sec:2.6}), which clearly highlights the peculiar behaviour of learners.

\begin{figure}
\includegraphics[width=.6\textwidth]{figure_2_26_violins dur int GL2-cropped.pdf}
\caption{Aggregated values of duration, intensity and mass (for comparison) pooled across German L2 learners. The x-axis displays Syll1, the stressed syllable of the adjective, and Syll3, the stressed syllable of the noun. Information structure conditions are colour-coded: green for given-new (GN) and red for new-given (NG).}
\label{fig:2.26}
\end{figure}

Overall, mass patterns showed differences across and within conditions more clearly, supporting the interpretation that speakers modulate prosodic strength.

\subsection{A phonological interpretation}
\label{sec:2.8.4}
The present investigation, based on continuous measures, does not preclude a categorical analysis in terms of pitch accent types. The results yielded by periodic-energy-based measures as well as the established measures of alignment, scaling, duration, and intensity suggest that categorical decisions are made by speakers of all groups when prosodically marking IS. Therefore, I will suggest a possible phonological description in terms of pitch accent types using a ToBI analysis (\citealt{GriceEtAl2005} for German; \citealt{GriceEtAl2005} for Italian) for the example noun phrases proposed in \sectref{sec:2.3}, \sectref{sec:2.5} and \sectref{sec:2.6} for the three language groups. I will concentrate exclusively on the pitch accents within the target noun phrase, and I will leave out the boundary tones since they are (for these examples) exclusively low/falling (L-L\%). \tabref{tab:2.9} summarises the pitch accent types found in the following example noun phrases across language groups.

\begin{table}
\begin{tabular}{lrrrrrr}
\lsptoprule
 & \textbf{GN} & \textbf{GN} & \textbf{NN} & \textbf{NN} & \textbf{NG} & \textbf{NG}\\
& \textbf{(1st} & \textbf{(2nd} & \textbf{(1st} & \textbf{(2nd} & \textbf{(1st} & \textbf{(2nd}\\
& \textbf{word)} & \textbf{word)} & \textbf{word)} & \textbf{word)} & \textbf{word)} & \textbf{word)}\\
\midrule
\textbf{Italian} \textbf{L1} & (L+)H* & L* & L+H* & L* & H*+L & L*\\
\textbf{German} \textbf{L1} & Ø & L+H* & L+H* & (L+)H* & (L+)H* & Ø \\
& H* & H* & H* & (H+)!H* & & \\
\textbf{German} \textbf{L2} & (L+)H* & L* or Ø & (L+)H* & L* or Ø & H*+L & L* or Ø\\
\lspbottomrule
\end{tabular}
\caption{Pitch accent types found in the example noun phrases.}
\label{tab:2.9}
\end{table}


Regarding Italian, it can be concluded that, in line with previous results, prosodic marking of information status 1) is not realised through deaccentuation and 2) does not take place on the second word of the noun phrase. Indeed, in the case of the contrast between NG and NN, the acoustic properties of F0 on the first word allow for the interpretation of the information status on the second word. The difference in the acoustic properties of the first word could be prosodically represented as a difference in pitch accent type. As shown in \figref{fig:2.27}, the earlier F0 peak on the first word of NG may be described as H*+L, while the later F0 peak on the first word of GN and NN as (L+)H*. The trailing L tone is added in the former case to highlight that H is much closer to the syllable onset than in the latter case and that the fall occurs mainly on this syllable. The accent on the second word is similar in all three conditions and can be analysed as L*.

\begin{figure}
\subfigure[Given-New on \textit{mano lilla} (Eng. `lilac hand').]{\includegraphics[width=.8\textwidth]{figure-2-27-a-cropped.pdf}}

\subfigure[New-New on \textit{rana verde} (Eng. `green frog').]{\includegraphics[width=.8\textwidth]{figure-2-27-b-cropped.pdf}}

\subfigure[New-Given on \textit{luna lilla} (Eng. `lilac moon').]{\includegraphics[width=.8\textwidth]{figure-2-27-c-cropped.pdf}}

\caption{\label{fig:2.27} ToBI annotation for L1 Italian. The tiers from top to bottom contain: the ToBI labels for the target noun phrases, their information status, the transcription of the stressed and unstressed syllables of the noun phrases, and the orthographic transcription of the carrier sentence.}
\end{figure}


For German L1, the results are in line with the literature and show that the prosodic marking of information status is achieved by 1) the deaccentuation of post-focal given material and 2) the tendency to align an F0 peak with new or focussed elements. This alignment is generally interpreted as a H* or L+H* pitch accent. Therefore, in the case of NG, the acoustic properties of the first word of the noun phrase can be described with an (L)+H* pitch accent and the second word as deaccented. In GN and NN, there is no post-focal material, as the noun phrases are sentence-final. The typically expected realisations for GN and NN are displayed in \figref{fig:2.28} and can be described as L+H* for the second word in the GN condition, and L+H* for the first word and (L+)!H* for the second word in the NN condition. In contrast, the hat patterns displayed in \figref{fig:2.29} can be described as H* for the first word and H* or (H+)!H* for the second word.

\begin{figure}
\subfigure[Given-New on \textit{blaue Birne} (Eng. `blue pear').]{\includegraphics[width=.8\textwidth]{figure-2-28-a-cropped.pdf}}


\subfigure[New-New on \textit{braune Vase} (Eng. `brown vase').]{\includegraphics[width=.8\textwidth]{figure-2-28-b-cropped.pdf}}


\subfigure[New-Given on \textit{graue Nonne} (Eng. `grey nun').]{\includegraphics[width=.8\textwidth]{figure-2-28-c-cropped.pdf}}

\caption{\label{fig:2.28} ToBI annotation for L1 German. The tiers from top to bottom contain: the ToBI labels for the target noun phrases, their information status, the transcription of the stressed and unstressed syllables of the noun phrases, and the orthographic transcription of the carrier sentence.}
\end{figure}


\begin{figure}
\subfigure[Given-New on \textit{graue Dose} (Eng. `grey can').]{\includegraphics[width=.85\textwidth]{figure-2-29-a-cropped.pdf}}

\subfigure[New-New on \textit{graue Blume} (Eng. `grey flower').]{\includegraphics[width=.85\textwidth]{figure-2-29-b-cropped.pdf}}

\caption{\label{fig:2.29} ToBI annotation for L1 German (hat pattern). The tiers from top to bottom contain: the ToBI labels for the target noun phrases, their information status, the transcription of the stressed and unstressed syllables of the noun phrases, and the orthographic transcription of the carrier sentence.}
\end{figure}

The main difference to Italian is that in native German the acoustic properties contributing to marking information status are modulated in situ (\citealt{KüglerCalhoun2020}), so that both the first and the second word of the noun phrase are affected by changes in F0 and prosodic strength based on their information status. In Italian, however, changes in the acoustic properties of only the first word contribute to distinguishing the information status of the second and last word of the noun phrase. For German L2, learners prosodically mark the information status by 1) modulating the alignment of F0 on the first word, as in their native Italian, which may be reflected in two different pitch accents, described as H*+L on the first word of NG tokens and H* or L+H* on the first word of GN and NN tokens. Moreover, 2) they do not prosodically differentiate the second word, probably due to the influence of the phonological rules of their L1 Italian. The acoustic details provide evidence for prosodic attenuation across all conditions, akin to the native German post-focal condition. For instance, in \figref{fig:2.30}, the second word could be interpreted both as L* and as deaccented. In the post-focal condition, the flat F0 on the second word could be interpreted as deaccented by a German native speaker since the more salient marking of the first new word with an earlier F0 peak can lead to the perception of lower prominence on the following given word. However, the last words sound very similar across conditions.

\begin{figure}
\subfigure[Given-New on \textit{graue Vase} (Eng. `grey vase').]{\includegraphics[width=.8\textwidth]{figure-2-30-a-cropped.pdf}}

\subfigure[New-New on \textit{braune Vase} (Eng. `brown vase').]{\includegraphics[width=.8\textwidth]{figure-2-30-b-cropped.pdf}}

\subfigure[New-Given on \textit{braune Birne} (Eng. `brown pear').]{\includegraphics[width=.8\textwidth]{figure-2-30-c-cropped.pdf}}

\caption{\label{fig:2.30} ToBI annotation for L2 German. The tiers from top to bottom contain: the ToBI labels for the target noun phrases, their information status, the transcription of the stressed and unstressed syllables of the noun phrases, and the orthographic transcription of the carrier sentence.}
\end{figure}


\subsection{Summary and discussion}
\label{sec:2.8.5}
The results on alignment and scaling confirm the findings from the continuous measures regarding differences in F0 contours used to mark information status across all groups. This provides further evidence for the phenomena observed, independently of the method and framework used. These measures present some limitations, however, as alignment is specific to F0 configuration of peaks and other F0 shapes cannot be optimally captured (unless decisions are made as to where the peak is labelled in other F0 configurations). For this dataset, this resulted in the loss of data points, especially in L1 German. A further disadvantage is that alignment requires manual annotation of syllable boundaries, which inevitably implies a certain degree of arbitrariness. From this point of view, periodic energy measures of F0, synchrony and ${\Delta}$F0, have the advantage of being based on the signal itself, i.e. on periodic cycles roughly corresponding to syllables. This avoids annotator-specific decisions and permits a description of all F0 configurations in incorporating acoustic information about the underlying segmental material. Mass yielded clearer patterns than duration and intensity, as in some cases (Italian) the latter measures even produced conflicting results. Moreover, mass can be considered more ecologically valid than duration and intensity, as these are not experienced as separate entities in actual perception.

\hspace*{-3pt}Regarding a possible phonological interpretation, decision-making was straight-forward for the two native languages, but was more complicated in the case of the interlanguage, for which I proposed a two-fold interpretation, i.e. describing the post-focal element either as L* or as deaccented. There are several reasons for this. First, the nature of annotation itself – it is subjective, affected by expectations and the perception of meaning, as well as by the annotators’ native language (\citealt{CangemiGrice2016}). Indeed, annotators rely on F0, energy and tonal context when labelling phonological categories, reflecting their native perception. As a consequence, it is not always easy to make a decision as to which label to use, which, in turn, means that naive listeners, too, might find it hard to interpret the meaning. Moreover, it is very likely that this accentuation pattern is not consistent across all L2 noun phrases, since interlanguages are developing systems in which categories are constantly updated based on the linguistic input learners receive. Therefore, any potential category identified in L2 systems can only be taken as a snapshot of the system at that precise moment in time.

\section{Conclusion}
\label{sec:2.9}
The present study was inspired by a series of comparative cross-linguistic studies (\citealt{AvesaniEtAl2013}; \citealt{AvesaniEtAl2015}; \citealt{KrahmerSwerts2008};
\citealt{SwertsEtAl2002}). The authors of these studies performed a categorical analysis of presence or absence of pitch accents and pitch accent types. They found that, contrary to native speakers of West Germanic languages, Italians do not mark information status prosodically within noun phrases, either in their L1 or their L2 German, which was also confirmed in perception.

The goal of the current study was to find out whether a close inspection of continuous phonetic parameters would bring to light subtle modulations for prosodically differentiating information structure, which did not emerge in previous categorical analyses. To this aim, periodic-energy-related measures were used to analyse F0 shape and prosodic strength in two-word noun phrases under different information structure conditions, i.e. given-new (GN), new-new (NN) and new-given (NG), produced by L1 Italian (Neapolitan variety) learners of German, in both their native and second language. Moreover, every attempt was made to overcome some limitations of previous studies by collecting a larger sample of data and using a more interactive elicitation method.

Results for the German native group are in line with the literature, providing evidence for deaccentuation as a marker of post-focal given information. Findings on Italian learners of German, instead, contrast with previous results. Despite reports in the literature that L1 Italians fail to deaccent, results show that the Italian participants to the current study clearly and consistently mark focused new information when given information is postfocal through different F0 peak locations, both in their L1 and in their L2, and that this difference is perceptually valid. The different contours found can be described as different accent types, but, in learners’ native Italian, not by the absence of a pitch accent. In L2 German, learners prosodically attenuated the last element of a noun phrase at all proficiency levels, as in the post-focal given L1 German condition. However, attenuation is applied across all information structure conditions, meaning that prosodic strength is not used to mark different information status conditions. One possible interpretation of this result is that learners identify a reduction in prosodic strength, which can perceptually result in deaccentuation, as a salient marker of native German speech, but without identifying the relevant context. This might be due to negative interference from their L1 Italian, in which F0 peak location only (not prosodic strength) is used to discern information status contrasts within noun phrases.

One legitimate question which might arise is why these results on L1 Italian and L2 German have not emerged before. This could have two possible reasons: the methodological approach to data analysis and the elicitation method chosen, which I will now discuss.

The current study of continuous parameters revealed patterns which did not emerge previously in purely categorical analyses. Labelling phonological categories entails some degree of subjectivity due to the annotator-specific perception of meaning and expectations based on native language. As a result, different annotators can make different choices, and the individual-specific bias is even more problematic when labelling an L2. Thus, an investigation of the modulations of continuous acoustic parameters can offer a deeper understanding of linguistic phenomena by providing acoustic evidence for a categorical description. In other words, the two approaches can and should complement each other, especially when analysing complex and dynamic systems like interlanguages, where categories undergo a continuous process of restructuring based on the input and feedback that learners receive.

Another factor that may have contributed to this discrepancy with previous findings is the data collection method. Indeed, speech style and degree of spontaneity have a great influence on intonation (for the difference between read and spontaneous speech in Italian varieties and German, see \citealt{DeRuiter2015}; \citealt{SavinoRefice1997}). Previous studies have used an elicitation game structured in the form of alternating statements, with noun phrases containing contrastive information status categories. The production of alternating statements may not have created an engaging interaction for speakers, who may not have perceived the other player’s sentences as the context for their own productions, and, instead, may have concentrated on their own list of statements. In contrast, the design of the current elicitation game was intended to create real interaction between participants, as well as engagement in the task which, together with the lack of eye-contact, may have promoted the use of prosody for conveying different pragmatic meanings, as demonstrated by perceptual results.

For Italian learners of German, however, I also found a minority of NG items realised in the same way as GN and NN items, that is, without matching prosody to the information status of the last element. This can be explained through some limitations of this elicitation method. First and most obvious, in the context of the elicited noun phrases, prosody is not necessary for the correct interpretation of the sentence, as meaning can be conveyed by the lexicon alone. Secondly, speakers may not have paid attention to the question posed by the interlocutor, since game turns are repetitive in their structure, an effect that cannot be completely avoided despite being limited by the insertion of distractors in the design. Finally, speakers may have chosen different strategies for accomplishing a task. Instead of a listener-oriented strategy, where speakers try to make sure that the interlocutor can receive and interpret the message properly, some speakers may have applied a self-oriented strategy. A prerogative to win the board game was to correctly write down all the images named by the interlocutor. As a result, some speakers may have moved their attention from the interlocutor’s questions to the writing task and, consequently, produced their answers without having in mind the pragmatic context of the questions.

A further limitation of this study is that, although the interactiveness of the elicitation method was improved, the type of speech investigated here cannot be described as spontaneous. More spontaneous data collected with a map task \citep{AndersonEtAl1991} provided us with some rare, spontaneous examples of NG contours in Italian and L2 German (which can be inspected in the OSF repository at \url{https://osf.io/9ca6m/}). These resemble the one elicited with the semi-spontaneous board game, reassuring us that the data collected might indeed mirror non-scripted interactions. On the one hand, it is true that reliable evidence can only be drawn by systematic observation based on a large amount of data, which necessitates experimental control in the data collection process. On the other hand, future research should strive for the best possible compromise between spontaneity and systematic data collection, aiming to collect real spontaneous conversational speech in order to increase the ecological validity of research findings.

As for possible practical applications, the results of the present study are highly relevant for language pedagogy. I found that learners do not adjust the use of prosody dependent on information status, showing that they might not be aware of this function of prosody in the target language, and fail to convey the correct meaning by prosodic means. Indeed, they tend to use marked structures (i.e. prosodic attenuation typical of post-focal given material) also in the unmarked case, differently from what one could expect according to the Markedness Differential Hypothesis \citep{Eckman1977}, predicting more difficulty in learning marked structures (such as accentuation according to pragmatic contexts) than unmarked ones (such as accentuation in the default condition). This means that the implicit learning of prosody, i.e. without formal instructions, is not sufficient for learners to correctly acquire the target patterns, and that explicit training might help them to improve their communication skills in the L2. Some forms of prosodic training for Italian learners of German have been tested (\citealt{Dahmen2013}; \citealt{Missaglia2007}) and showed an improvement in learner performance in the both the segmental and suprasegmental domain. However, results on prosodic training across languages are contradictory, with some studies showing no robust (\citealt{BaillsEtAl2022b}; \citealt{Suter1976}) or only minor effects (\citealt{PurcellSuter1980}). Moreover, these studies are not directly comparable in terms of native and target languages as well as methodology, making the identification of the best approach for L2 classroom setting very difficult (e.g., hand gestures in \citealt{BaillsEtAl2022b}; imitation and repetition in \citealt{NicoraEtAl2018}; contour visualisation using Praat in \citealt{Smorenburg2015}; contour visualization using stylised contours in \citealt{NiebuhrEtAl2017}; computer- or robot-assisted techniques in \citealt{Bissiri2008}; \citealt{NiebuhrAlm2021}). Therefore, more research on efficient and practicable pedagogical techniques for teaching prosody in L2 classrooms is still necessary.

The present study has provided useful theoretical groundwork for the development of such pedagogical tools by individuating critical aspects of L2 prosody acquisition and exemplifying the prosodic strategy for marking information status in learners’ native and target languages. The analysis of learners’ native and target languages provides teachers with empirical knowledge on the pragmatic use of prosody in both phonological systems contributing to learners’ interlanguage, instead of a (complete) reliance on native speaker intuitions (\citealt{DerwingMunro2015}). The analysis of learners’ difficulties in acquiring the target prosodic features highlights the aspects of prosodic competence which should be explicitly thematised in teaching and be embedded in pedagogical tools to ensure their successful acquisition.
