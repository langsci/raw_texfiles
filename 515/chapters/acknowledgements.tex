\addchap{\lsAcknowledgementTitle}
It is finished. The time has come to thank the people who made the writing of this book possible and who have accompanied me on this journey.

First of all, I would like to thank my supervisors, Martine and Elina, for the time they took to plan, discuss, and improve the content of this book, and for the knowledge and skills they passed on to me. Your contribution to my professional development has been immense, but I have also learned much more that will be useful to me in life. Thank you for your roles as three-hundred-and-sixty-degree trainers.

Thanks to another supervisor, Renata, who is no longer with us, but without whom I would not be here writing these acknowledgements. Thank you for inspiring me and setting me on this path. Thank you for being my mentor with the love of a mother. I will always carry you with me, and this book is also for you. I hope you can be proud of that little girl you encouraged to fly out of the nest.

Thanks to my older colleagues, Francesco and Simon, who are in my eyes like older brothers (yes, we Italians always get attached to everyone like family members). Francesco, back in 2015, when a lost little girl was looking for advice for her B.A. thesis during her semester abroad and came across a researcher in Germany who went to work in a suit, was Neapolitan, and knew her thesis supervisor. Practically unreal, like a mirage in the desert. And Simon, in his course on second language acquisition back in 2016. Thanks to both of you for the long discussions, not only on professional topics, from which part of the thesis originated, but above all for the moments of connection between human beings, which are the ones most fondly etched in memory. Also, for Simon, who was also the language editor of this text, a special thanks for his patience and care devoted to this book.

Thanks also to Aviad and Christine for taking the time, over and over again, to give me advice and offer explanations. Your input was crucial in the development of certain aspects of this book, and I am immensely grateful for that. To Christine in particular, a special thanks for all the lunch breaks shared during the pandemic in the almost-empty laboratory. Thank you, too, for the discussions about life, they are always the best.

A special thank you to Caterina, who has unwittingly been a big sister. I could write a page just about you and still not do justice to your contribution to the development of this thesis and my growth. I have learned so much from you, and once again, the most important lessons have been about life. Thank you, because in you I have also found a friend and confidante.

I would like to thank Anna Clelia, Celine, and Harriet, student assistants and great companions, for their affection towards me. Thanks to all the other lab members, Stefan, Theo, Tabea, Lena, Alicia, Maria, Malin, and the guest researcher Vincenzo (another great friend) for the mutual help, meals, and shared chats. Your positive energy contributed to a pleasant and genuine working environment.

Thanks to a distant colleague and one of my dearest friends, Loredana. Thank you for your moral support, your advice, and for affectionately welcoming my outbursts in the most stressful moments. Having someone who shares the same journey and can fully understand has been an immense help. Thank you for being there and, whether close or distant, I hope we will still go a long way together.
Thank you to all my lifelong Italian friends, who have stayed by my side even if they do not see me very often: Pio, Susi, Daria, Francesca, Alfonso, Carolina, Laura, Sabrina, Grazia, Ilaria, Anna and Rossella. And thanks to the “new” friends (although for some it has already been eight years) who have made me feel at home in a foreign country with their affection and sincerity: Andre, Karen, Nele, Sivan and Giulia.

However, the greatest thank you goes to my family: my mum Santa, my dad Mimmo and my sister Noemi. Your support, despite the pain of being separated, has been an act of love and courage. Thank you for always encouraging and supporting me in my choices, even when you did not share them or suffered because of them. Thank you for passing on to me the values that allow me to act every day in the world, and for having taught me to always do my best and be the best version of myself. Thank you for your immense love; I feel it from kilometres away. I am truly blessed to have you as family, confidants, and best friends. I could not have asked for better. I love you all so much.

How could I forget to thank my German family, Gerhard? Thank you for accompanying me during these eight years and contributing so much to my thinking and philosophy of life. I have learned so much from you. Thank you for being my family here. When I felt lonely, I knew you were there and that I could count on you. That is an invaluable feeling for someone who lives far from their homeland. I hope I can give back at least some of the affection and selfless support I have received. I love you very much. Thank you.

And then there is you. This book also led me to you: my greatest and most unexpected gift. The man who cautiously knocked on the door of my heart and patiently waited for me to see him clearly. And when I did, I knew immediately that it was you. You were the one I was looking for. The man who awakened the joyful child in me. The man who knew how to make me fall in love with his gentle kindness. The man I want to share my path with. Thank you, my love, for making me feel safe and at home every day. Thank you for your love of few words but tender actions. Thank you for your gentleness and sensitivity. I love you so much. I thank this book for having brought us together.

Finally, I must thank one last person, Simona, myself, whom I often forget to thank. Thank you for your stubbornness, your strength, and your ability to always rise from the ashes. Know that I love you and am proud of you, whatever you do and wherever you go.