\chapter{Backchannels in L2 interactions}
\graphicspath{{figures/plots-chapter-4}}
\label{chap:4}
This chapter focuses on examining a particular ability of L2 interactional fluency, specifically the use of vocal feedback signals (henceforth “backchannels”) in two-party conversations.\footnote{The analysis provided in this chapter integrates work previously published in \citet{SbrannaEtAl2022,SbrannaEtAl2023,SbrannaEtAl2024}. Its inclusion in this book enables a discussion of the findings within the broader framework of interactional fluency and their potential pedagogical applications.}

Backchannels are very short lexical or non-lexical utterances used by a listener to signal acknowledgment to what the speaker is saying. Given their role, they support the ongoing turn of the interlocutor, and positively contribute to a smooth turn alternation and, eventually, structure in a dyadic conversation. While these small tokens are generally unnoticed in a conversation among native speakers, they can stand out in multicultural and multilingual settings if aspects of their realisation do not follow the native norm and potentially cause misunderstanding. 

Given the lack of a comprehensive study on backchannel production across languages and in L2, I propose a study with a within-subjects design in order to investigate several aspects of backchannel use across L1 and L2 and the possible relation between them. The goal is to highlight critical aspects of backchannel production in SLA. 

\section{Background}
\label{sec:4.1}
\subsection{Backchannel contribution to interactional fluency}
\label{sec:4.1.1}
As discussed in the previous chapter, one issue in SLA research has been the question of how to assess communicative competence in a quantitative and systematic way, while taking into account idiosyncratic and contextual factors impacting the L2 learning process and L2 oral performance. Fluency has been widely recognised as a central aspect in the assessment of L2 oral proficiency (\citealt{DeJong2016}), but most studies have focussed on individual measures of L2 fluency, while the majority of real-life oral performances are interactions and much less often monologues.

Fluency in dialogue is highly determined by the specific interaction mechanism that arises between the two parties of a conversation, so that along with individual factors, unique dyad-related factors play a fundamental role (for an extensive discussion on this topic see also \citealt{SbrannaEtAl2020}). For this reason, interaction has been described as a co-construction process both in research (\citealt{Hall1995}; \citealt{HeYoung1998}; \citealt{JacobyOchs1995}; \citealt{McCarthy2009}) and in the CEFR (\citealt{GoodierNorth2018}). The smoothness of a conversation is achieved, among other factors, through the rhythm of turn-taking (\citealt{SchegloffJefferson1974}). Indeed, smooth or disfluent transitions of turns can take place at turn-boundaries, and interlocutors have to appropriately foresee the end of the other party’s turn and react quickly and accordingly (\citealt{BögelsTorreira2015}; \citealt{Levinson2016}). Despite usually going unnoticed in conversation (\citealt{ShelleyGonzalez2013}), one important linguistic means that can facilitate turn transitions is the use of so-called ``backchannels".

Backchannels are very short lexical and non-lexical utterances, like \textit{okay} or \textit{mm-hm}, which have traditionally been described as non-intrusive tokens – that is, as not claiming a floor transfer – used to signal the listeners’ active engagement, showing acknowledgement and understanding (\citealt{Schegloff1982}; \citealt{Yngve1970}: 19). By supporting the ongoing turn of the interlocutor, backchannels positively contribute to fluency in social interactions (\citealt{Keeffe2013}) as they maintain flow and contribute to creating structure in a dyadic conversation (\citealt{LewisSwezey1982}; \citealt{SchegloffJefferson1974}; \citealt{Schegloff1982}).

On the other hand, backchannels can be potentially misleading in cross-cultural contexts where different culturally-shaped communicative conventions come into contact (\citealt{Cutrone2005,Cutrone2014}; \citealt{EbnerGrice2016}; \citealt{Li2006}). Research has indeed provided evidence for language- or variety-specific backchannel characteristics concerning duration, frequency, location, intonation and function, and these are a possible source of negative social implications in a communicational setting in which the interlocutors’ linguistic backgrounds diverge.

For these reasons, backchannels in L2 learning are extremely important. The CEFR \citep{FiguerasEtAl2009} lists the use of feedback expressions under passive competence already at the A2 level. However, backchannels are not explicitly thematised in most L2 classrooms, and it cannot be taken for granted that learners acquire appropriate backchannelling behaviour solely through exposure to the target language. Moreover, teachers are not always native speakers and input on this particular interactional feature might be completely absent from classroom settings.

Against this background, two possible manifestations of backchannels in interlanguage can be expected. On the one hand, it is possible to assume that backchannels go unnoticed in conversation, resulting in a transfer of features from the L1 to the L2. On the other hand, assuming that there is exposure, backchannels might be perceived by learners as salient features of foreign speech and receive an appropriate level of attention, which would favour an adaptation to target language patterns. In the latter case, a more target-like backchannel behaviour should be observed, especially at an advanced level, i.e. with more experience of and exposure to the target language. With these two scenarios in mind, I will explore the use of backchannels in L2 learning.

\subsection{Backchannel definitions and categorisations}
\label{sec:4.1.2}
In the literature, there is little agreement about the definition of backchannels (as noticed by \citealt{Lennon1990,Lennon2000}; \citealt{Rühlemann2007}; \citealt{Wolf2008} among others).

In his analysis of telephone conversations, \citet{Fries1952} was probably the first to recognise these ``signals of attention" that do not interrupt the speaker’s talk. Since then, other terms have been used to define this phenomenon, such as ``accompaniment signals" \citep{Kendon1967}, ``receipt tokens" \citep{Heritage1984}, ``minimal responses" \citep{Fellegy1995}, ``reactive tokens" \citep{ClancyEtAl1996}, ``response tokens" \citep{Gardner2001}, ``engaged listenership" \citep{Lambertz2011} and ``active listening responses" \citep{Simon2018}.

The term ``backchannel communication" was first coined by \citet{Yngve1970} to define the channel of communication used by the listener/recipient to give useful information to their interlocutor without claiming a turn, in opposition to the main channel used by the speaker holding the floor.

Initial investigations into backchannelling primarily focused on American English (\citealt{Duncan1974}; \citealt{DuncanFiske1977}; \citealt{Fries1952}; \citealt{Goodwin1986}; \citealt{Jefferson1983}; \citealt{Schegloff1982}; \citealt{Yngve1970}). These pioneering works sought to establish a definition of backchannels and proposed classifications of backchannel types grounded in either their pragmatic function or formal realisation.

\citet{Schegloff1982} noted that these short utterances were mainly used by the listener not only to acknowledge the interlocutor’s turn, but also to invite the primary speaker to carry on with his turn. For this reason, he defined the minimal utterances used in the specific contexts of an ongoing turn by the interlocutor as ``continuers". \citet{Jefferson1983} introduced the term ``acknowledgement tokens". Indeed, the term backchannel in its narrow use refers to tokens used to signal acknowledgement and understanding of what the interlocutor is saying, while inviting the main speaker to continue (\citealt{GravanoHirschberg2007}; \citealt{Hasegawa2014}).

In its broader use, the term ``backchannel" has also been matched to numerous other functions, and some attempts at establishing a function-based categorisation have been made. For example, \citet{Jefferson1983}, \citet{DrummondHopper1993} and later \citet{JurafskyEtAl1998}, \citet{Savino2010}, \citet{Savino2011}, \citet{Savino2014}, \citet{SavinoRefice2013} further distinguish acknowledgement tokens marking ``passive recipiency", as in the case of continuers, from those marking ``incipient speakership", signalling a listener’s intention to start a turn of their own. \citet{Senk1997} categorises backchannels according to the functions of continuer, understanding, agreement, support, strong emotional answer and minor additions. \citet{Kjellmer2009} recognises five functions of backchannels: regulative, supportive, confirmatory, attention-showing and empathetic. \citet{Tree2014} distinguish context-generic backchannels, used as continuers and promoting the production of new information, and context-specific backchannels, also called ``assessments" in previous studies \citep{Goodwin1986}, such as \textit{really} or \textit{wow}, eliciting further elaboration of what has just been said.

As far as their formal realisation is concerned, backchannels present a high degree of lexical variability, although they can also be non-lexical, e.g. realised through vocal noises (\citealt{WongPeters2007}), and non-verbal, making use of visual modalities such as facial expressions, head movements, and gestures (\citealt{Tree2014}), and responsive laughter \citep{Hasegawa2014}. Some structurally motivated proposals of classifications have been advanced to categorise backchannel lexical realisations. \citet{Tottie1991} classifies them into simple, double, and complex types. Simple backchannels are composed of one single utterance, e.g. \textit{yes}, double backchannels are repeated simple types, e.g. \textit{okay okay}, and complex backchannels are a combination of different simple types, such as \textit{okay yes right}. \citet{WongPeters2007} differentiate between minimal, lexical, and grammatical types. Minimal types are defined as non-lexical items that are semantically empty and items expressing polarity, e.g. \textit{mmhm}, \textit{yes}, and \textit{no}. Lexical types are considered to be all single words that are codified in dictionaries and show an increase in semantic weight, such as \textit{really}, \textit{right}, and \textit{good}. Finally, by grammatical types they mean predications in the form of short codified phrases, such as \textit{I see}, brief questions, repetitions, sentence completions, and commentaries.

As noticed by \citet{EdlundEtAl2010}, the variety of names and categorisations provided by previous studies are often vague and overlapping. Moreover, the labelling schemes adopted treat backchannels quite differently. Faced with these difficulties, the authors propose a more general unit called ``very short utterance" (VSU) to capture the large range of interactional dialogue phenomena commonly referred to as backchannels, feedback, and continuers.

Subsequent research broadened its scope beyond American English, examining other languages and revealing variations in backchannel usage across cultures and languages (\citealt{Berry1994}; \citealt{ClancyEtAl1996}; \citealt{Cutrone2005,Cutrone2014}; \citealt{Heinz2003}; \citealt{KraazBernaisch2022}; \citealt{Li2006}; \citealt{Nurjaleka2019}; \citealt{TaoThompson1991}; \citealt{Tottie1991}; \citealt{WardTsukahara2000}). These cross-cultural and cross-linguistic differences will be further explored in the following section.

\subsection{Backchannel use across languages and cultures}
\label{sec:4.1.3}
One focus in the field of backchannel research has been variation across languages and cultures. Differences in backchannel use have been identified regarding their frequency, duration, location, lexical types, functions, and intonation.

Because it is influenced by cultural norms, backchannelling has been found to vary even among varieties of the same language. \citet{Tottie1991} reports differences with regard to frequency and types across American and British English, showing that in American conversations there was an average of sixteen backchannels per minute, compared with just five backchannels per minute in British conversations. Similarly, differences were observed across Sri Lankan and Indian English in type, frequency, and function (\citealt{KraazBernaisch2022}).

Some studies report the impact of different backchannelling behaviour on the turn-taking system. For instance, in a cross-linguistic study on Spanish and North-American English, \citet{Berry1994} found that backchannels were more frequent and longer among Spanish speakers, resulting also in longer stretches of overlapping speech. In turn, American English speakers were shown to use more overlapping backchannels than Germans, as reported in a comparative study by \citet{Heinz2003}.

The differences found lead to the hypothesis of a potentially negative effect on communication in intercultural conversations. In a study on responsive tokens in English, Mandarin and Japanese, \citet{ClancyEtAl1996} observed that Japanese speakers produced the most frequent reacting tokens, placing them in the middle of the interlocutor’s speech. Mandarin speakers, in contrast, produced the fewest backchannels, and mostly at TRPs, i.e. at the end of the interlocutor’s turns. American English was in the middle between the two language groups in terms of frequency, and reacting tokens were placed both within an interlocutor’s turn and at TRPs, but preferably at grammatical competition points. The authors speculate that, in Japanese, backchannels are used as a form of emotional support and cooperation, whereas, on the opposite pole, Mandarin speakers might perceive Japanese backchannels as intrusive in comparison to their tendency not to interrupt the other speaker out of respect. American English speakers, likewise, might find Japanese speakers disruptive. However, the scarce reactions of Mandarin speakers would leave them wondering what their listeners are thinking (\citealt{ClancyEtAl1996}: 383). Similar hypotheses were tested in a study on backchannel intonation, in which \citet{HaEtAl2016} found differences across Vietnamese and German. While Vietnamese continuers are consistently level or falling, German equivalents are tendentially rising. Based on the results of a previous perception experiment \citep{Ha2012}, the authors hypothesise probable misunderstandings in intercultural dialogues. In Vietnamese, rising pitch as used by Germans might be interpreted as impolite. Conversely, for German natives, the level/falling pitch used by Vietnamese might cause irritation (\citealt{KoppGibbon2007}) and could be interpreted as showing disinterest, or as an attempt to end the interlocutor’s turn.

Given the observed differences across languages, the immediate next step in research was to put the consequences of this variation in intercultural conversations to the test and find out whether and to what extent differences in backchannel use can lead to miscommunication and/or have negative social implications. \citet{Li2006} conducted a study on Canadian and Chinese speakers in intra- and intercultural conversations and showed that backchannels facilitate content communication among speakers of the same language. But when Canadian speakers were paired with Chinese speakers, the opposite effect was observed, leading to the claim that backchannel responses can be misleading in intercultural conversations and cause miscommunication. It was also found that Chinese speakers produced the most backchannels and Canadians the fewest, but when crossed, speakers tended to produce a number of backchannels in between. In a follow-up study providing an analysis of backchannel types (\citealt{CuiWang2010}), it was found that, in intercultural conversations, both Canadian and Chinese speakers used other backchannels than in their respective native languages, showing some degree of speech convergence for both frequency and lexical type.

However, accommodation in intercultural conversations does not always take place automatically, with knowledge of language- and culture-specific conventions probably being essential. For example, a study reported that Japanese speakers did not adapt their active listening style in conversations with Americans, while Americans did, because “they clearly have the linguistic ability to do so” (\citealt{White1989}: 74), suggesting that language proficiency might play a role for accommodation to take place. A high level of L2 proficiency can, indeed, provide the speaker with diverse linguistic means which can be selected according to context and the flexibility to recognise and switch among linguistic conventions.

\subsection{Backchannels in L2 speech}
\label{sec:4.1.4}
To date, only relatively few studies have investigated backchannels produced by L2 learners. Their findings reinforce the assumptions made on the basis of intercultural studies, showing that L1 backchannel behaviour is generally carried over to the L2, which can cause miscommunication and misperception.

For example, \citet{Cutrone2005} examined the use of backchannels in dyadic interactions between Japanese EFL (English as Foreign Language) and British speakers. Differences were found in frequency, type and location, and affected intercultural communication negatively. The frequent backchannels used by the Japanese participants were interpreted as interruptions by the British speakers, and their interlocutors were perceived as impatient. In a follow-up study, \citet{Cutrone2014} reports that Japanese EFL speakers used a greater number of backchannels because it helped them to feel comfortable as listeners, showing a behaviour similar to the one reported for L1 Japanese by \citet{ClancyEtAl1996}.

\citet{WehrleGrice2019} also report on the negative effect of transfer on intercultural communication. In a pilot experiment, they compared the intonation of backchannels in L2 German spoken by Vietnamese and observed that Vietnamese learners produced twice as many non-lexical backchannels (e.g. \textit{mmhm}) with a flat intonation contour than German native speakers, showing a transfer from their L1. As noted earlier, in German, a flat intonation for backchannels can be perceived as disinterest and may cause offense (\citealt{EbnerGrice2016}).

Another study that hypothesises a transfer of backchannel features from the L1 to the L2 was conducted by \citet{CastelloGesuato2019}. They investigated the frequency and lexical types of ``expressions of convergence" (used synonymously with backchannels) in Chinese, Indian and Italian learners of English in a language examination setting. They found that Chinese learners used the most backchannels, and Indian learners used the least, while Italian learners showed a backchannel frequency in between these two groups. They also observed differences in the choice of backchannel types across groups, which was motivated by the influence of their own native language and culture.

A similar conclusion is reached by \citet{ShelleyGonzalez2013}, who analysed backchannel functions in informal interviews in four English as Second Language (ESL) speakers with different L1 backgrounds and one American native speaker of English. They identify four backchannel functions: continuers (the listener is paying attention and gives the floor back), acknowledgements (the listener agrees or understands), newsmakers (the listener communicates an emotional reaction) and change of activity (the listener signals to move toward a new topic). They report an effect of culture-specific preferences as they find differences in the backchannel functions used across the four speakers: Japanese and Saudi Arabian speakers were found to use continuers, acknowledgements, and change of activity tokens; the Taiwanese speaker limited their use to acknowledgements; and the Egyptian speaker used both continuers and acknowledgements; while the control American native speaker made use of the widest range of functions with continuers, acknowledgements, newsmakers, and change of activity tokens. These results are, however, difficult to interpret, as only one speaker is taken as representative of their language and culture, making it challenging to distinguish between language-specific and idiosyncratic factors.

Finally, there are studies showing that higher proficiency in the L2 implies a better ability to use backchannels. \citet{Galaczi2014} compared the frequency of backchannels and expressions of confirmation among learners of English with different proficiency levels. The results showed that intermediate learners provided less feedback than highly proficient learners, among which the “ability to act as supportive listeners through backchanneling and confirmations of comprehension was found to be more fully developed” \citep[570]{Galaczi2014}.

To summarise, previous research on various languages provides similar evidence 1) for miscomprehension and misperception of the interlocutor’s intentions due to a divergent use of backchannels from the native conventions, 2) for a transfer of native backchanneling behaviour to the L2, and 3) for proficiency as a positive factor for the improvement of learners’ L2 backchannelling ability.

At the same time, these studies have some limitations. Their results are not easily comparable as they differ considerably in design and methodology: how participants in the dialogue are matched, their status, their proficiency level in the language of the conversation, the setting of the dialogue, the method used for dialogue elicitation and aspects of backchannels analysed. Moreover, most studies have focussed on subjects with different L1 backgrounds, which is useful for detecting cultural-specific differences among groups of learners, but does not permit differentiation between transfer phenomena and cross-linguistic, speaker-specific characteristics.

Nevertheless, these findings have significant implications for the relevance of backchannels in language teaching environments. In order to better understand the mechanism behind cross-cultural backchannelling behaviour, it is important to shed light on how the backchannelling ability develops in interlanguages, with the goals of raising awareness in multicultural communicative contexts and improving L2 speakers’ interactional skills.

The aim of this study is to overcome some of the limitations mentioned by carrying out an in-depth analysis across languages and in an L2 with a homogeneous methodology. Using a within-subjects design, I investigate backchannel use across Italian learners’ L1 and L2 German and compare learners’ realisation of backchannels to a German native group to assess transfer phenomena and/or the acquisition of target-like backchannel features. In contrast to most previous studies, this investigation considers a broader range of backchannel features in an effort to provide a more comprehensive picture of the phenomenon, including frequency, length, formal structure, (non-)lexical type, pragmatic function and intonation. Particular attention will be paid to dyad-specific behaviour in order to differentiate idiosyncratic factors from actual transfer or acquisition of patterns.

\section{Method}
\label{sec:4.2}
For the purpose of this study, I will adopt the term ``very short utterances" (\citealt{EdlundEtAl2010}) as a loose definition for the wide variety of interactional dialogue phenomena providing feedback to the interlocutors. I define backchannels as a specific class of VSUs with an acknowledging function, that is, showing understanding and acceptance of the interlocutor’s turn. This investigation also includes the VSU class of positive replies realised with the same token types as backchannels, such as \textit{yes}, with the aim of assessing the impact of a different function on the distribution of lexical type and contour realisation. The criterion used to distinguish backchannels and positive replies is that backchannels are unsolicited, whereas positive replies are solicited by a yes-no or tag-question formulated by the primary speaker. To distinguish among these classes of VSUs, I will refer to ``backchannels" and ``acknowledgments" interchangeably, and to ``other VSUs" and ``positive replies" synonymously. Finally, given that previous studies (on Italian \citealt{Savino2010, Savino2011, Savino2014}) report an interaction between intonation, token type, and backchannel turn-taking function, acknowledgements will be further distinguished according to their turn-taking function.

\subsection{Corpus}
\label{sec:4.2.1}
Similar to the study on turn-taking contained in Chapter~\ref{chap:3}, the basis for this backchannel analysis consists of thirty-nine Map-Task dialogues collected in Italian L1, forty in German L2 spoken by the same Italian speakers and nineteen in German L1. Learners’ proficiency levels ranged from A2 to C1 on the CEFR scale. However, for the sake of determining potential effects of proficiency using two balanced groups, they were recategorised into two groups only: beginner (from A1 to B1 levels) and advanced learners (from B2 to C2 levels).\footnote{See \sectref{sec:3.2.2} for a description of the task and \sectref{sec:1.3} for details about participants, data collection, and learner proficiency levels.}

The resulting corpus includes a total of 2147 VSUs, of which 1745 were BCs and 402 other VSUs. 315 tokens (15\% of the extracted data) were excluded from prosodic analysis because they did not display the necessary amount of periodic energy to perform a prosodic analysis, e.g. items produced with creaky voice, or items with a voiced portion that was too short, such as \textit{sì}. Accordingly, 1572 BCs and 260 other VSUs underwent prosodic analysis. 

\tabref{tab:4.1} summarises the amount of tokens found for each category, language and proficiency group: total amount of VSUs (the totality of tokens independently of their function), the amount of backchannels (BCs), and the number of positive replies (other VSUs). The entries marked by the abbreviation PA refer to the amount of tokens which underwent prosodic analysis.

\begin{table}
\begin{tabular}{lrrrr}
\lsptoprule
 & IT L1 & GL2 (beginner) & GL2 (advanced) & GE L1\\
 \midrule
Total VSUs & 602 & 273 & 517 & 755\\
BCs & 496 & 223 & 422 & 603\\
Other VSUs & 106 & 50 & 95 & 152\\
Total VSUs (PA) & 414 & 213 & 427 & 722\\
BCs (PA) & 389 & 188 & 368 & 586\\
Other VSUs (PA) & 25 & 25 & 59 & 136\\
\lspbottomrule
\end{tabular}
\caption{VSU corpus size across language groups.}
\label{tab:4.1}
\end{table}

\subsection{Procedure and Metrics}
\label{sec:4.2.2}
All VSUs produced during the dialogues were annotated using Praat (\citealt{BoersmaWeenink2021}). After token annotation and extraction, the F0 trajectory of the extracted tokens was pre-processed through smoothing and manual correction of pitch points. The analysis of backchannels takes into account several aspects of their realisation, i.e. their frequency, length, lexical type, structure, function, and intonation. Moreover, for the aspects of type, function, and intonation, a comparison between BCs and positive replies is provided. \tabref{tab:4.2} summarises all aspects of BC and VSU realisation analysed.


\begin{table}
\subtable{
\begin{tabularx}{.8\textwidth}{Ql}
\lsptoprule
BC aspects & Operationalisation\\
\midrule
Frequency & BCs/minute \\
Length  &   Duration in ms\\
Type    &   Lexical/non-lexical realisation \\
Structure & Simple\\
& Repeated\\
& Complex\\
Function & Passive recipiency (PR)\\
& Incipient speakership (IS)\\
Intonation & Rising\\
& Level\\
& Falling\\
\lspbottomrule
\end{tabularx}
}
\subtable{
\begin{tabularx}{.8\textwidth}{Ql}
\lsptoprule
Other VSU aspects & Operationalisation\\
\midrule
Type & Lexical/non-lexical realisation \\
Function & Reply to tag question \\
& Reply to yes-no question\\
Intonation & Rising\\
& Level\\
& Falling\\
\lspbottomrule
\end{tabularx}
}
\caption{Aspects of BC and other VSU analysed and their operationalisations.}
\label{tab:4.2}
\end{table}


\textit{Frequency} is operationalised as backchannel rate per minute, while \textit{length} is their duration in milliseconds.

\textit{Type} encompasses both lexical and non-lexical forms. In this corpus, the most frequent lexical types were \textit{ja} and \textit{sì} (the German and Italian equivalents of ‘yes’, respectively ), \textit{genau} and \textit{esatto} (meaning ‘exactly’ in German and Italian, respectively), and \textit{okay}. The most common non-lexical type was \textit{mmhm}. These types constituted 92\% of the entire corpus. The category ``other" was used for less frequent token types.

Following a classification similar to \citet{Tottie1991}, \textit{structure} classifies the complexity of token form into simple, i.e. one single utterance such as \textit{yes}, repeated,\footnote{I do not use the word ``double" like \citet{Tottie1991} because simple types also occurred more than two times in succession in the present corpus.} i.e. repeated simple tokens such as \textit{okay okay}, and complex, i.e. combinations of different tokens, such as \textit{okay yes}.

Backchannels were also categorised based on their \textit{turn-taking function}, specifically passive recipiency (PR) and incipient speakership (IS) (\citealt{Savino2010,Savino2011,Savino2014}). Tokens that were produced without the speaker taking the floor and simply as signals to the primary speaker that they may continue, were labelled as acknowledgement tokens marking PR. When a speaker used backchannels to acknowledge the interlocutor’s turn but then took the floor by continuing to speak and causing a turn transition, these backchannels were labelled as marking IS. Interestingly, this corpus includes considerably more PR (1376) than IS (368) tokens. With regard to positive replies, the instances found in this corpus can fulfil functions of either an answer to a yes-no or to a tag question.

Finally, intonation was categorised as rising, flat or falling and measured in semitones (ST) with a reference value of 1 Hz. For each token, F0 points were extracted from two time points, one at the beginning and one at the end of the signal. Depending on the location of the first voiced sound, F0 points were sampled at 10\%-90\%, 20\%-80\% or 30\%-70\% of the token duration. Following \citet{Wehrle2023}, a rising intonation was defined as a difference greater than +1 ST between the initial and final F0 points, a level intonation was defined as a difference within +/- 1 ST, and a falling intonation was defined as a difference less than -1 ST.




\section{Results}
\label{sec:4.3}
\subsection{BC frequency}
\label{sec:4.3.1}

\figref{fig:4.1} illustrates the frequency of BCs per minute of dialogue across different groups. The graph reveals a similar rate of BC use among native speakers of German (5.82 BCs per minute) and Italian (5.14 BCs per minute). In contrast, learners exhibit a lower BC rate than that of both native groups. Beginner learners, in particular, produced the fewest BCs, at nearly half the rate of their target language (2.87 BCs per minute). Advanced learners, instead, show a BC rate closer to that of the native German speakers (4.75 BCs per minute). This result may lead to the conclusion that, as learners' proficiency and fluency improve, their backchanneling behaviour tends to approach the native German target. However, this observation is incomplete, as revealed by a more detailed analysis of the individual dyadic interactions.

\begin{figure}[p]
\includegraphics[width=.8\textwidth]{figure_4_1_BC rate by proficiency.pdf}
\caption{Backchannel frequency operationalised as rate per minute of dialogue. The number of BCs per minute is displayed on the y-axis. Language groups are shown on the x-axis and are colour-coded: blue for Italian learners’ native speech; aquamarine for beginner learners in L2 German; yellow for advanced learners in L2 German and red for the native German control group.}
\label{fig:4.1}
\end{figure}

\begin{figure}[p]
\includegraphics[width=.8\textwidth]{figure_4_2_BC rate proficiency by dyad.pdf}
\caption{Backchannel frequency by dyad operationalised as rate per minute of dialogue. The number of BCs per minute is displayed on the y-axis. Dyads are shown on the x-axis and language group is colour-coded: blue for Italian learners’ native speech; aquamarine for beginner learners in L2 German; yellow for advanced learners in L2 German and red for the native German control group. Italian learners of L2 German present two values corresponding to their L1 and L2 speech, distinguished by the colour of the square.}
\label{fig:4.2}
\end{figure}

\figref{fig:4.2} shows the crucial influence of dyad-specific behavior across all groups. Learners display a remarkably similar BC rate across L1 and L2. For instance, dyad BS presents nearly identical rate values in both languages (as shown by the overlapping squares). Moreover, the low BC rate in the beginner group is partly due to the peculiar behaviour of beginner dyad GS. Furthermore, the low backchannel rate observed in the beginner group can be partially attributed to the unique behaviour of dyad GS. The extremely low backchannel production in their L2 output is likely not solely due to their limited German proficiency, as they also produced no backchannels in their L1 (and only very few VSUs, which are not displayed in the graph). This suggests that their behaviour is dyad-specific. On the other extreme end, another beginner dyad, CV, presents the highest BC rate across all groups, which would not be expected from group-level results. This high degree of dyad variability is also evident within the native German speaker group. Notably, dyad EL's rate is very similar to that of the beginner dyad GS, indicating that very low backchannel frequency can also occur among native German speakers. This observation challenges the notion of a specific target backchannel rate for learners to achieve. Instead, it suggests that backchannel frequency is largely dependent on the specific dynamics of each dyadic interaction.


\subsection{BC length}
\label{sec:4.3.2}

\figref{fig:4.3} shows BC length, i.e. their duration in milliseconds (ms). Here, a difference emerges between the two native language groups: native Italian speakers produce longer BCs (475 ms, SD = 177 ms) than native German speakers (329 ms, SD = 133), suggesting a potential target for learners. The two learner groups show similar backchannel durations with no apparent effect of proficiency, falling between the values found for the two native languages (Beginners: 405 ms, SD = 155; Advanced: 409 ms, SD = 140). At first sight, this might suggest that during the learning process learners tend to approach the target, but eventually plateau. However, this interpretation is again incomplete, as demonstrated by an analysis of the individual dyadic interactions.

\begin{figure}[p]
\includegraphics[width=.8\textwidth]{figure_4_3_BC duration group.pdf}
\caption{Backchannel length operationalised as their duration in ms. Milliseconds are displayed on the y-axis. Language groups are shown on the x-axis and are colour-coded: blue for Italian learners’ native speech; aquamarine for beginner learners in L2 German; yellow for advanced learners in L2 German and red for the native German control group. Gray lines represent the standard error.}
\label{fig:4.3}
\end{figure}
\begin{figure}[p]
\includegraphics[width=.8\textwidth]{figure_4_4_BC duration by dyad.pdf}
\caption{Backchannel length by dyad operationalised as their duration in ms. Mean BC duration in milliseconds is displayed on the y-axis. Dyads are shown on the x-axis and language group is colour-coded: blue for Italian learners’ native speech; aquamarine for beginner learners in L2 German; yellow for advanced learners in L2 German and red for the native German control group. Italian learners of L2 German present two values corresponding to their L1 and L2 speech, distinguished by the colour of the square. The horizontal black line corresponds to the mean BC duration of the L1 German group pooled across all speakers.}
\label{fig:4.4}
\end{figure}

Similarly to frequency, by-dyad values for length displayed in \figref{fig:4.4} show that dyad-specific behaviour yields a better explanation for the results observed at least for half of the learner dyads. Indeed, especially advanced dyads (RC, CA, AA, RS, BS, AB, CR), but also some beginner ones (IF, CC, CV), present very similar length values across their L1 and L2. Moreover, it seems that the trend of reducing BC length in the L2 is especially present in dyads with very high length values in their L1 (around and above 500 ms), which is more often the case in beginner (AN, RM, GA, AC) than in advanced learners (CE, FF, MA). This might suggest that learners do perceive a difference in BC length across Italian and German and tend to shorten BCs in their L2, particularly when they perceive their native Italian backchannel length to be highly different from what they categorise as native German. However, this hypothesis is a mere speculation and should be tested on a larger dataset.



\subsection{BC structure}
\label{sec:4.3.3}
\figref{fig:4.5} shows the percentages of different BC structures across groups, enabling an exploration of the potential relationship between BC structure and length. Specifically, the goal is to check whether the higher proportion of repeated and complex BCs contributes to longer BCs in L1 Italian speakers as compared to L1 German speakers.

\begin{figure}
\includegraphics[width=.8\textwidth]{figure_4_5_BC structure by group.pdf}
\caption{Backchannel structure. Proportions of BC structures are shown in percentages on the x-axis. Language groups are shown on the y-axis and are each assigned one bar. Different BC structures are listed in the legend and are colour-coded: blue for single, green for repeated and yellow for complex BCs.}
\label{fig:4.5}
\end{figure}

Proportions of BC structures are highly similar across all groups, showing no particular tendency of Italians producing more repeated or complex BCs, which could explain the difference in BC length across L1s. On the contrary, German L1 speakers tended to produce slightly more repeated BCs than L1 Italian speakers (Italian L1: single = 93.8\%, repeated = 1.41\%, complex = 4.83\%; German L1: single = 92.5\%, repeated = 2.82\%, complex = 4.64\%).

\figref{fig:4.6} shows that no particular structure exhibits a dramatically different length that would explain the observed variations. For both L1 Italian and L1 German, simple BC length is very similar to the mean value of BC duration, and in both cases repeated and complex BCs are about 200–250 ms longer (Italian L1: single = 458 ms, repeated = 721 ms, complex = 741 ms; German L1: single = 314 ms, repeated = 498 ms, complex = 521 ms). Interestingly, learners shorten their simple BCs but do not proportionally reduce the length of their repeated and complex BCs, which remain similar to those produced in their L1 (Beginners: single = 389 ms, repeated = 720 ms, complex = 731 ms; Advanced: single = 395 ms, repeated = 719 ms, complex = 645 ms), with the exception of complex BCs in the advanced group.

\begin{figure}
\includegraphics[width=.8\textwidth]{figure_4_6_BC structure by duration.pdf}
\caption{Backchannel structure by length for each language group. Language groups are assigned a box each with mean BC length for each type of BC structure. Mean duration of BCs structures are shown on the y-axis. The different BC structures are displayed on the x-axis and are colour-coded: blue for single, green for repeated and yellow for complex BCs. Gray lines represent the standard error.}
\label{fig:4.6}
\end{figure}

Finally, the observed length differences cannot be traced back to differences in the syllabic structure of the lexical items used, i.e. to the production of a greater number of syllables. As will be discussed below, the BC types that differ between German and Italian are two monosyllabic types for ‘yes’ (\textit{ja} and \textit{sì}, respectively), and ‘exactly’,  which is disyllabic in German (\textit{genau}) and trisyllabic in Italian (\textit{esatto}). In turn, the Italian equivalent is used less frequently than the German one. Even excluding the category ``other" did not alter the results in a relevant way. This suggests that measures such as speech or articulation rate might explain the differences in BC length between the languages.

\subsection{BC type}
\label{sec:4.3.4}
\figref{fig:4.7} illustrates the proportions of BC types across groups. A comparison of the two native language groups reveals a divergence in preferred BC types. While both groups use \textit{mmhm} in similar proportions (37\% in L1 German and 28\% in L1 Italian), L1 Italian speakers prefer \textit{okay} (43\%) over \textit{sì} (23\%), whereas L1 German speakers show the opposite preference (\textit{ja} 36\%, \textit{okay} 20\%). Both groups use \textit{genau} and \textit{esatto} infrequently (5\% in L1 German and 2\% in L1 Italian).

Learners show varying proportions of BC types. Beginners more closely resemble their target language than advanced learners (beginners: \textit{ja} 37\%, \textit{mmhm} 32\%, \textit{okay} 29\%; advanced: \textit{ja} 27\%, \textit{mmhm} 30\%, \textit{okay} 41\%). However, this result warrants further investigation to determine the influence of dyad-specific preferences. Additionally, L2 learners may be transferring their L1 BC type preferences, as the type \textit{genau} is completely absent in the beginners' data and constitutes only 0.47\% of BC occurrences in the advanced learner group. A possible explanation is that its Italian equivalent is rarely used, suggesting that learners require more L2 experience and exposure to begin using this type of BC when speaking German.

\begin{figure}[t]
\includegraphics[width=.8\textwidth]{figure_4_7_BC type by group.pdf}
\caption{Backchannel types. Proportions of BC types are shown in percentages on the x-axis. Language groups are shown on the y-axis and are each assigned a bar. The most-used BC types are listed in the legend and are colour-coded. The category “other” refers to types that rarely occurred.}
\label{fig:4.7}
\end{figure}

\begin{figure}[p]
\includegraphics[width=\textwidth]{figure_4_8_BC type by dyad.pdf}
\caption{Backchannel type by dyad. Proportions of BC types are shown as percentages on the x-axis. Dyads arranged by language group are shown on the y-axis and are assigned one bar each. The most frequently used BC types are listed in the legend and are colour-coded. The category “other” refers to types that rarely occur.}
\label{fig:4.8}
\end{figure}
\figref{fig:4.8},\footnote{Remember that dyad GS did not produce any BCs in their L1 Italian (and only one VSU), while the Italian L1 file for dyad ME turned out to be damaged and was therefore not analysable.} which displays the choice of BC type by dyad, reveals a tendency for L1 dyad-specific patterns to be replicated in the L2, particularly among advanced learners (compare L1 and L2 in IF, CV, AN for beginner and AB, RS, CR, CA, BS, AA for advanced learners). This observation could be explained by individual preferences. However, it is also possible that, as learners become more comfortable in the L2, they unconsciously approach their own L1 spontaneous speech style. Essentially, with greater proficiency, L2 speech patterns seem to stabilise, whereas less proficient learners show more variability (independently of the model pattern being that of the native or the target language).

Finally, an evident difference between L1 Italian and L1 German concerns the proportions across types in a by-dyad comparison. It seems that the choice of BC types within L1 German is more consistent across dyads, whereas it is more variable and dyad-dependent in L1 Italian.



\subsection{BC type by function}
\label{sec:4.3.5}
\figref{fig:4.9} shows how the proportions of BC types vary depending on their function: passive recipiency (PR, non-turn-initiating) or incipient speakership (IS, turn-initiating).

For PR, L1 German and L1 Italian speakers show a similar behaviour, primarily using \textit{okay}, \textit{ja/sì} and \textit{mmhm}. Both L1s display roughly the same amount of \textit{mmhm} (25\% and 22\%, respectively), but L1 German speakers prefer \textit{ja} (43\%) over \textit{okay} (22\%), while L1 Italian speakers show the opposite pattern, with \textit{okay} (41\%) being preferred over \textit{sì} (23\%). In both languages, \textit{genau} (6\%) and \textit{esatto} (2\%) are rarely used for this function.

However, the two language groups diverge considerably in their preferred backchannel types for IS. Italian speakers almost exclusively use \textit{okay} (76\%), while German speakers use \textit{genau} (20\%) more frequently than its Italian counterpart \textit{esatto} (2\%). Interestingly, the non-lexical \textit{mmhm}, frequently used for PR, is only occasionally used for IS in both languages (7\% in German and 3\% in Italian). Its non-lexical nature likely leads speakers to perceive \textit{mmhm} as non-intrusive, making it unsuitable for signaling an intention to take a turn. Instead, it encourages the other speaker to continue, potentially characterising \textit{mmhm} as a prototypical continuer. 

Italian learners of German appear to transfer their L1 PR backchannel preferences to their L2, with the exception of two instances of \textit{genau} produced by advanced learners. For IS, advanced learners use \textit{ja} more often than its Italian equivalent \textit{sì} (29\%), but their most frequent choice remains \textit{okay} (65\%), mirroring their L1 pattern. Notably, none of the learners use the most typical German type, \textit{genau}, for this function. Beginners produce too few instances of IS backchannels (17 items only) to allow for firm conclusions about their type choices to be drawn. Their limited L2 proficiency may lead them to avoid actively taking turns, preferring to let their interlocutor lead the interaction. 

\begin{figure}
\includegraphics[width=.8\textwidth]{figure_4_9_BC type by function & proficiency.pdf}
\caption{Backchannel types by function. Proportions of BC types are shown in percentages on the x-axis. Functions are shown on the y-axis and are assigned one bar each: PR for passive recipiency and IS for incipient speakership. The most-used BC types are listed in the legend and are colour-coded. The category “other” refers to types that rarely occurred.}
\label{fig:4.9}
\end{figure}

\subsection{Other VSU type by function}
\label{sec:4.3.6}
\begin{figure}[b]
\includegraphics[width=.8\textwidth]{figure_4_10_VSU type by class.pdf}
\caption{Very short utterance types by class across language groups. Proportions of VSU types are shown in percentages on the x-axis. Classes are shown on the y-axis and are assigned one bar each: replies to yes-no questions, replies to tag questions and acknowledgements (BCs, for comparison). The most-used VSU types are listed in the legend and are colour-coded. The category “other” refers to types that rarely occurred.}
\label{fig:4.10}
\end{figure}

\figref{fig:4.10} illustrates the choice of BCs (acknowledgements) and other VSUs (replies to yes-no and tag questions) by function and across language groups. The bars representing acknowledgments correspond to those in \figref{fig:4.7} and are repeated here allowing for a direct comparison.

The two native languages vary greatly regarding the distribution of types across the two reply types. In response to yes-no questions, Italians predominantly use \textit{sì} (80\%), a preference mirrored by both beginner and advanced learners (96\% for both groups). German speakers, however, use a wider range of responses, with only a few instances of \textit{mmhm} (12\%), more predominant \textit{ja} (50\%) and many \textit{genau} (36\%) items. For tag replies, Italians seem to equally prefer \textit{sì} and \textit{mmhm} (33\% each), with occasional use of \textit{okay} (11\%). German speakers primarily use \textit{ja} (57\%), followed by \textit{mmhm} and \textit{genau} (23\% and 20\%, respectively). L2 learners reproduce the preference for \textit{ja} (beginners: 50\%; advanced: 64\%), but their overall pattern is closer to their native Italian,  with some occurrences of \textit{okay} and almost no use of \textit{genau}, which appears only once in each of the two reply classes among advanced learners. The type \textit{genau} appears to be a hallmark of L1 German, used in both yes-no and tag replies (36\% and 20\%, respectively), whereas this is not the case in L1 Italian and the interlanguage.

Finally, comparing the two replies to acknowledgements, it is evident that the choice of type changes in terms of proportions across classes and functions, suggesting a relation between type choice and function expressed.

\subsection{BC intonation}
\begin{figure}[b]
\includegraphics[width=.8\textwidth]{figure_4_11_BC intonation by function & proficiency.pdf}
\caption{BC contours by function across language groups. Values above zero represent items with a falling contour; values below zero represent items with a rising contour. Cyan diamonds represent mean values.}
\label{fig:4.11}
\end{figure}

\label{sec:4.3.7}
In this section, BC intonation contours are explored in relation to their function and type. \figref{fig:4.11} shows a tendency for PR backchannels to be expressed with rising intonation and IS backchannels with falling intonation, consistent with previous results (\citealt{Savino2010,Savino2011,Savino2014}; \citealt{Wehrle2023}). This pattern holds across all language groups under investigation, but distributions of values suggest some differences. Italian learners of German, both in their native Italian and their L2 German, seem to avoid flat intonation contours (values around zero), unlike native German speakers. Moreover, L2 German shows a considerable influence from native Italian patterns, but with higher variability, as is typical of an interlanguage. Finally, a small proportion of PR backchannels is expressed with a falling contour in all language groups.

\begin{figure}[b]
\includegraphics[width=.8\textwidth]{figure_4_12_BC prosody function type proficiency.pdf}
\caption{BC contours by type and function across language groups. Values above zero represent items with a falling contour; values below zero represent items with a rising contour. Cyan diamonds represent mean values.}
\label{fig:4.12}
\end{figure}
To assess whether this latter result can be explained by other variables, \figref{fig:4.12} and \figref{fig:4.13} display BC intonation contours by function and lexical type in a continuous and categorical fashion, respectively, so that proportions can be related to the amount of data for each type and function. It becomes clear that this apparent relationship between contour and pragmatic function is more nuanced, as intonation is also dependent on word choice. Indeed, two types, \textit{mmhm} and \textit{genau}, exhibit preferred contours regardless of function. \textit{Mmhm} is typically rising across all language groups, while \textit{genau} is predominantly falling in L1 German (with 11\% level in PR) and never rising. In contrast, the Italian equivalent \textit{esatto} shows similar proportions of rising (29\%), falling (43\%) and level contours (29\%) in the PR condition, indicating no specific contour association with this word in Italian. Learners transfer this variability to the corresponding \textit{genau} in their L2, using both rising and falling contours. However, these observations are based on limited data for \textit{esatto} in L1 Italian (seven PR tokens) and \textit{genau} in L2 German (two PR tokens produced by advanced learners, one rising, one falling). As previously noted, \textit{esatto} is not common in L1 or L2 Italian, unlike the more frequent use of \textit{genau} in L1 German.


\begin{figure}[t]
\includegraphics[width=.8\textwidth]{figure_4_13_BC contour categories by type and function.pdf}
 \caption{\label{fig:4.13} Categorical classification of BC contours by type and function across language groups. Proportions of BC contour categories are shown in percentages on the x-axis. BC types are displayed on the left of the y-axis and are each assigned a bar. Upper boxes refer to types used with an incipient speakership function (IS), while bottom boxes refer to types used with a passive recipiency function (PR).}
\end{figure}

The lexical type \textit{okay} shows a similar distribution of rising and falling contours for PR in native Italian (41\% falling, 11\% level and 48\% rising contours) and in L2 German (beginners: 51\% falling, 3\% level and 46\% rising; advanced: 35\% falling, 10\% level and 55\% rising). Native German speakers, however, prefer falling contours for \textit{okay} even when used with a PR function (61\% falling, 12\% level and 27\% rising). This variability in contours for \textit{okay} used with a PR function might play the biggest role in explaining the broad range of values observed for PR in \figref{fig:4.11}, across groups. When expressing IS, \textit{okay} instead tends to show falling contours across all language groups (L1 Italian: 84\%; L2 beginner learners: 57\%; L2 advanced learners: 67\%; L1 German: 60\% falls and 28\% levels).

Finally, \textit{ja} and \textit{sì} best illustrate the contour-function relation shown in \figref{fig:4.11} for Italian speakers. When expressing PR, Italian speakers predominantly use \textit{sì} with a rising contour in both their L1 and L2 (65\% in Italian; 74\% and 67\% in L2 German by beginner and advanced learners respectively), while native German speakers still prefer falling contours (61\%), similar to the case of \textit{okay}. When expressing IS, L1 German aligns with the general trend of using falling and level contours for this function (60\% and 28\% respectively). In L1 Italian, the limited number of instances (six tokens, three rising, three falling) prevents firm conclusions, and more tokens may yield different results. As reported in \sectref{sec:4.2.1}, a prosodic analysis of many \textit{sì} tokens was not possible due to the shortness of their voiced portion. However, L2 German has more \textit{ja} tokens (130 for PR and 21 for IS across proficiency groups), and given learners’ tendency to mostly transfer their L1 intonation patterns, it could be hypothesised that their L2 mirrors the L1 also in the case of \textit{sì} used with an IS function. This would support the observation that Italian speakers tend to use a rising contour for \textit{sì} in PR (74\% rises and 3\% levels by beginner learners; 67\% rises and 7\% levels by advanced learners) and a falling contour in IS (57\% falls and 29\% levels by beginner learners; 64\% falls and 36\% levels by advanced learners).

Overall, learners’ patterns of contour-type-function relations closely resemble those of their native language, indicating a transfer from their L1. Moreover, proficiency level appears to have no relevant impact, which is particularly evident in the categorical analysis of the PR function (\figref{fig:4.13}), where more data points yield more reliable results.

\subsection{Other VSU intonation}
\label{sec:4.}
A final observation is related to other classes of VSUs using the same lexical types. \figref{fig:4.14} shows contours of types used as positive replies to yes-no and tag questions in Italian L1, German L2 and German L1. Interlanguage data is combined across proficiency levels, since no major differences were observed in the analysis reported in the previous section. As stated above, the Italian \textit{sì} used for PR has both falling and rising contours in equal proportion. However, in yes-no replies, \textit{sì} is predominantly falling in Italian (86\% of 15 items). L1 German also primarily uses falling contours for both tag and yes-no replies (53\% for both, based on 17 tag and 51 yes-no tokens). Moreover, the tendency for \textit{mmhm} to have rising and \textit{genau} to have falling contours is consistent across all groups. Further analysis on positive replies is not possible, since the amount of data for these token classes is very limited and does not allow for reliable observations. Nevertheless, this preliminary view provides further evidence for the observation that the function of some lexical types influences their prosodic realisation.

\begin{figure}
\includegraphics[width=.8\textwidth]{figure_4_14_VSU contours by type and function.pdf}
\caption{Other VSU contours by type and function across language groups. Values above zero represent items with a falling contour; values below zero represent items with a rising contour. Cyan diamonds represent mean values. Functions of the replies are shown on the y-axis: replies to yes-no questions (yn) and replies to tag questions (tag).}
\label{fig:4.14}
\end{figure}

\section{Conclusion}
\label{sec:4.4}
This contribution offered an in-depth analysis of BCs and other VSUs in Italian and German, as well as in the L2 German spoken by Italian learners. The analysis covered BC frequency, duration, structure, lexical (and non-lexical) type, function, and prosodic realisation. Other VSUs sharing lexical types with BCs but serving different functions (e.g. positive replies to yes-no and tag questions) were also included for comparative purposes. Dyad-specific variability was considered because: 1) individual conversational behaviour depends not only on idiosyncratic factors and speakers-specific speech style but also, crucially, on the unique dynamics that arise from the interaction between the two specific parties in the conversation; and 2) this helps distinguish genuine acquisition or transfer of patterns from dyad-specific behaviour.

It was found that German and Italian are quite similar regarding BC frequency. In L2, analysing only group-level data could have falsely suggested that target-like patterns of BC frequency are achieved along with increasing proficiency. Instead, a by-dyad analysis revealed similar behaviour in learners’ L1 and L2, suggesting that dyad-specific patterns are more important than proficiency levels when examining the rate of BCs produced. Moreover, the by-dyad variability observed even within the group of native German speakers challenges the notion of a fixed target frequency for learners to acquire.

Cross-linguistic differences exist for BC duration. Italian speakers produce longer BCs than German speakers, a difference not explained by an analysis of BC structure. Learners' BC durations fall between those of the two native languages. However, in this case, too, dyad-specific behaviour better explains learners’ L2 patterns in at least half of the dyads, and proficiency appears to have no impact. Intriguingly, Italian dyads with very long BC durations were the same ones that tended to shorten them in their L2. Possibly, a larger perceived difference between native and target language norms might encourage an adaptation to the target. This hypothesis requires further investigation.

Regarding lexical choice, distinct language-specific relationships between type and function were observed, suggesting this could be a learning target for L2 learners. It was found that learners tend to favour BC types shared with their native Italian over German-specific ones like \textit{genau}. A by-dyad analysis revealed that advanced learners, in particular, tend to use BC types in proportions similar to their L1. One possible explanation is that higher proficiency allows learners to transfer their L1 spontaneous speaking style to the L2, resulting in a more consistent output compared to the highly variable production of beginners. Only advanced learners used German-specific BCs, albeit infrequently, indicating a positive effect of increased target language exposure. Moreover, beginners produced very few BCs with the function of actively taking a turn, possibly due to their lower proficiency and consequent preference to let their interlocutor lead the conversation.

An even more complex, non-arbitrary mapping between lexical type, function, and intonation was observed in both languages. Overall, PR acknowledgements tended to be produced with rising contours, and IS acknowledgements with falling contours, across all groups. However, when BC type was taken into account, it emerged that this is not a one-to-one relation. A possible example of this function-contour relation is the case of \textit{ja} and \textit{sì}, but limited L1 Italian data prevented a reliable analysis of this trend. Hypothesising that learners transfer their L1 patterns to the L2, it could be assumed that this tendency exists in L1 Italian as well, confirming that the prosody of some types is influenced by their function. On the other hand, the intonation contour was found to be highly variable for certain types, as in the case of \textit{okay}, which presented both rising and falling contours for PR. The types \textit{mmhm} and \textit{genau} exhibited mostly unidirectional, type-specific prosody regardless of function. Other VSUs, i.e. positive replies to yes-no and tag questions, provided further evidence for the influence of function and/or type on the prosodic realisation.

This study used peer interactions, matching learners with learners and natives with natives. Hence, it is only possible to speculate on Italian learners’ backchannel use in a conversation with native German speakers. An interesting direction for future research therefore involves studying mixed dyads to evaluate the potential negative effects of learners’ differing BC use on the success and perception of the communicative exchange. Larger corpora are needed to confirm the robustness of the observed trends, since this study’s limited sample of token types and functions did not allow for reliable statistical testing. Finally, future research on backchannel intonation could provide a more fine-grained analysis, such as by using periodic energy measures. These metrics, being based on periodic cycles roughly corresponding to syllables, would require experimenters to make methodological decisions to define backchannel structure in greater detail, since the overlap between syllables and backchannels can be less stable than expected in such short utterances.

Despite these limitations, this study offers some fundamental suggestions for further investigations. First, preferential co-occurrences appear to exist between different aspects of BCs (e.g., lexical type, function, and intonation), warranting further investigation of these relationships. Second, these results suggest that dyad-specific patterns seem more predictive of some aspects of L2 backchannel production than proficiency. Therefore, using learners’ L1 as a baseline and examining dyad-specific behaviour is important to distinguish individual variability from the transfer or acquisition of patterns. Finally, consistent with the literature, this study found both cross-linguistic similarities in BC use and language-specific aspects that are not correctly reproduced in the L2. This suggests an incomplete acquisition of target-like backchannelling behaviour in the L2. Therefore, comparative studies of diverse language pairs, like the present one, can raise awareness of culture- and language-specific conventions. These findings should be addressed in L2 pedagogy in order to improve learners’ intercultural communication skills.
