\chapter{Introduction}
\label{chap:1}

This book is concerned with prosody and fluency phenomena of interactional competence (IC) in second language acquisition (SLA).\footnote{The terms \textit{learning} and \textit{acquisition} have traditionally been used with different meanings. The former has been used to describe a conscious process occurring in formal settings under formal instruction (i.e. L2 classrooms), while the latter has been used to describe a subconscious or unconscious process of assimilation occurring in naturalistic settings (i.e. language immersion through direct exposure, such as in the country where the L2 is the dominant language used for daily communication). However, recent research does not tend to differentiate between acquisition and learning, and the term acquisition, which I also adopt, is the most widespread. This is rooted in the ``language socialisation paradigm" \citep{Watson-Gegeo2004} refusing the dichotomous distinction of the two processes (\citealt{BaraschJames1994}; \citealt{Ellis1989}). The classroom is seen as an inherent part of social life in many cultures and, therefore, should be considered as an integral part of a naturalistic setting (\citealt{Watson-GegeoNielsen2003}). This claim is supported by research findings indicating that the outcome of the two settings can be highly similar (\citealt{Rivers1994}; \citealt{Willett1995}), with important implications for language teaching, which should aim for more naturalistic methods and materials.} 

The particular skills investigated are a) prosodic marking of information status, b) floor management via turn-taking and c) backchanneling (i.e. vocal feedback signals), all in dyadic interactions. These abilities, which might seem unrelated, are actually complementary to each other. While the study on prosody will be concerned with learners’ ability to master melodic aspects of the interaction, i.e. the intonational meaning \citep{Wennerstrom2006}, the study of turn-taking and backchannels with a turn-regulating function will be focused on learners’ ability to manage the rhythm of the interaction and ensure a fluent and smooth conversational tempo. For these reasons, prosody and fluency phenomena have often been investigated together (\citealt{TrofimovichBaker2007}; \citealt{TsengEtAl2005}; \citealt{KallioEtAl2023}; \citealt{Ashby2016}). In this book, however, I approach fluency aspects with a novel perspective, that is, within the specific context of the coordinated collaborative interaction, as opposed to the tradition of studies which has regarded fluency from an individual perspective (\citealt{TavakoliSkehan2005}; \citealt{Segalowitz2010}; \citealt{Kormos2006}). In line with this approach, the abilities of turn-taking and backchanneling contributing to the fluency of conversation will be grouped together under the broader definition of interactional competence (IC, \citealt{Young2014}).

Despite being crucially involved in oral communication, prosodic and interactional competence still receive limited attention, both within L2 research (for prosody: \citealt{DerwingMunro2015}; \citealt{Gut2009}; \citealt{Mennen2004}; for IC: \citealt{Cekaite2007}; \citealt{Galaczi2014}; \citealt{YamamotoEtAl2015}) and the applied field of language teaching (for prosody: \citealt{Aronsson2014}; \citealt{PiccardoNorth2017}; for IC: \citealt{Campbell-Larsen2022}; \citealt{Cohen2005}; \citealt{VanCompernolleSoria2020}). In fact, the poor mastery of prosodic and interactional competence has been reported to negatively impact learners’ communicative abilities and perception. On the communicative side, a poor control of these abilities might slow down the processing of the information and reduce communication efficacy (\citealt{SørensenEtAl2019}; \citealt{SongIverson2018}), or even cause communication breakdowns (\citealt{SbrannaEtAl2020}). On the perceptual side, studies report a negative impact on the perception of learners’ proficiency (\citealt{VanOsEtAl2020}; \citealt{TrofimovichBaker2007}), comprehensibility and intelligibility (\citealt{MunroDerwing1999}; \citealt{Kang2010}; \citealt{Hahn2004}) by native speakers, arousal of stereotypes \citep{Nakane2007}, and even stigmatisation (\citealt{Munro2003}; \citealt{Piske2012}).

This book aims at shedding light on the acquisition of important prosodic and interactional skills belonging to oral communication which are normally not explicitly addressed in classroom settings. By identifying learners’ difficulties in acquiring the native norm, this work lays the groundwork for improving learners’ L2 communicative skills and findings are addressed to language pedagogy for future interventions.

In this introduction, I discuss prosodic and interactional competence in relation to the theory and practice of language learning and teaching, providing a framework for their acquisition and the motivation to investigate them in the L2 classroom. Additionally, I outline the book by briefly presenting the three studies conducted and the corpora used throughout.

\section{Communicative competence in oral performance}
\label{sec:1.1}
Early models of proficiency focussed mainly on grammar, under the influence of the post-Chomskyan interest in formal and abstract aspects of language (\citealt{Campbell-Larsen2015}; \citealt{Kramsch1986}). However, it has long been recognised that L2 learners would not be able to take part in spontaneous spoken interactions if they relied solely on their lexicogrammatical competence. Alongside the knowledge of lexis and grammar, so-called \textit{communicative competence} is crucial, described by \citet[277]{Hymes1972} as knowledge about “when to speak, when not, and as to what to talk about and with whom, when, where, in what manner”. In other words, communicative competence is the ability to use language in a socio-culturally appropriate way according to the context and the goal of the communicative exchange. In line with this view, \citet{CanaleSwain1980} argued that socio-cultural and strategic competence are an integral part of communicative competence.

The emphasis on the use of language rather than theoretical thereof knowledge was the origin of Communicative Language Teaching as a methodology for language teaching (CLT, \citealt{Savignon1991}). Within this framework, the ultimate goal of L2 learning is no longer mastering vocabulary and grammar structures per se, but the ability to apply them appropriately in the communicative context.

This view is embraced by the Common European Framework of Reference for languages (CEFR, \citealt{CouncilOfEurope2001}), an international standard designed to describe language ability for European languages, which is globally popular especially for its application to English.\footnote{There are various frameworks with the aim of describing language proficiency. Some examples are the American Council on the Teaching of Foreign Languages Proficiency Guidelines (ACTFL), the Canadian Language Benchmarks (CLB) and the Interagency Language Roundtable scale (ILR).}
The CEFR builds on an action-oriented approach and claims to represent a shift away from the idea of learning as a linear progression through abstract language structures, towards real-life tasks and purposefully selected notions and functions (\citealt{GoodierNorth2018}). In other words, the Framework sees language as a means of communication and not as a subject of study. According to these principles, it describes learners as “social agents” (\citealt[9]{CouncilOfEurope2001}) who, as members of society, constantly need to accomplish communicative tasks in various contexts and under different circumstances. Considering language as a vehicle to socialisation implies attributing high relevance to interaction and putting the co-construction of meaning at the centre of the learning process (\citealt{GoodierNorth2018}).

The CEFR provides a schematic description of communicative language competence \citep{FiguerasEtAl2009}.\footnote{After a recent revision (consult \citealt{GoodierNorth2018}), minor changes were applied to the overall structure. Strategic competence was moved from communicative language competence (left with three main components: linguistic, socio-linguistic and pragmatic competence) and recategorised as a component of overall proficiency, so that communicative strategies are now specified according to the four different communicative activities.} There, communicative competence is depicted as a compound competence formed by several components: linguistic competence, intended as lexical and grammatical knowledge, together with cognitive organisation and accessibility; sociocultural competence, including the knowledge of register appropriateness, degree of formality, rules of politeness and the knowledge of linguistic rituals specific to a community; pragmatic and strategic competence, referring to the functional skills necessary to arrange the message and manage the conversation according to interactional schemata. These components are differently involved in the four language activities learners can be faced with: production, reception, mediation and interaction. Among these language activities, spoken interaction is the most direct and common form of communication in our daily life — nowadays probably comparable only to online written communication in its daily frequency. Therefore, a central role should be ensured for all the skills involved in oral interactive communication in language teaching and testing. Within this framework, the skills investigated in this book — prosodic marking of information status, floor management via turn-taking and backchannel use — are considered necessary for L2 learners to effectively take part in an oral exchange (as demonstrated by the fact that they are listed under “spoken interaction” in the above-mentioned CEFR scheme). In the CEFR, prosodic marking of information status is an ability belonging to “phonological control”, floor management can be synonymously used with “turn taking” and backchannels or feedback expressions are means of “cooperating”, but also contribute to “spoken fluency”.

\subsection{The theory relating to prosodic competence}
\label{sec:1.1.1}
The correct pronunciation of a foreign language implies being able to produce both L2 segments, i.e. consonants and vowels, and prosody, i.e. intonation and rhythmic features at a suprasegmental level. Prosody is especially crucial in oral communication, as it plays a fundamental role in conveying and interpreting not only linguistic information, such as discourse chunking, information status (e.g. whether a referent is new, given, or contrastive), or the disambiguation of syntactically ambiguous sentences, but also paralinguistic information indicating a speaker’s identity and attitude.

Starting with the CLT approach, L2 prosody has gradually received more attention in SLA research (\citealt{Busà2008}; \citealt{Lengeris2012}) and evidence has been collected regarding its crucial role in the production and perception of meaning. Studies have demonstrated that prosodic features deviating from the native norm can affect listeners’ judgements of accentedness, comprehensibility and intelligibility (\citealt{Hahn2004}; \citealt{Jilka2000}; \citealt{Kang2010}; \citealt{RubinPickering2010}; \citealt{MunroDerwing2001}; \citealt{TrofimovichBaker2006}) more than deviations in the production of segments (\citealt{MunroWiebe1998}; \citealt{Munro1995}; \citealt{MunroDerwing1999}; \citealt{GordonDarcy2022}). A non-target-like use of intonation can even lead to misperception and negative evaluation by native speakers (\citealt{Munro2003}; \citealt{MunroDerwing2020}; \citealt{LeVelleLevis2014}).

In the past, the objective of pronunciation teaching was to eliminate any foreign accent, setting unsustainable learning goals. Nowadays, teaching L2 pronunciation aims at improving learners’ communicative effectiveness and success. Exactly for this reason, the description of phonological competence in the early version of the CEFR (\citealt{CouncilOfEurope2001}) attracted a lot of criticisms for its implicit negative connotation of foreign accents and unrealistic learning goals, as well as for referring to the concepts of stress, intonation, pronunciation, accent and intelligibility in an unclear way (\citealt{PiccardoNorth2017}). Following a recent revision, intelligibility has been adopted as a criterion for defining the progression through proficiency levels, and prosody has been awarded the attention it deserves. The description of prosodic ability in the current version mentions, in particular, the importance of learning how to prosodically highlight newsworthy information:

\begin{quote}
The focus is on the ability to effectively use prosodic features to convey meaning in an increasingly precise manner. Key concepts operationalised in the scale include the following: control of stress, intonation and/or rhythm; ability to exploit and/or vary stress and intonation to highlight his/her particular message. (\citealt{GoodierNorth2018}: 135)
\end{quote}

Therefore, the use of prosody to mark information status is officially recognised as an essential skill to be acquired by L2 learners in order to communicate effectively by expressing even the more subtle shades of meaning.

\subsection{The theory relating to interactional competence}
\label{sec:1.1.2}

Interactional competence is a concept that was developed to stand in contrast to communicative competence. As pointed out by \citet{Young2014}, the position of \citet{Hymes1972} and \citet{CanaleSwain1980} still considers communicative competence as a testable and quantifiable individual characteristic, reflecting the assumption that competence resides in the individual. Even though the notion of communicative competence was useful for moving language teaching away from its narrow focus on lexicogrammar competence, it does not take into account the interactive factor of communication. Therefore, the expression \textit{interactional competence} was coined to refer to what a speaker does when interacting with other individuals, differently from the concept of communicative competence, focussing on what a speaker knows, i.e. the knowledge of an individual \citep{Young2011}. Research in conversation analysis has contributed to this conceptualisation with descriptions of interaction as a joint creation of discourse between interlocutors (\citealt{JacobyOchs1995}). Moreover, interactional competence has been claimed to be context-specific to the extent that it emerges in varied interactive practices to which participants contribute with the appropriate linguistic and pragmatic resources (\citealt{Hall1993}; \citealt{HeYoung1998}; \citealt{Hall1995}; \citealt{Young2011}; as backed up by later research, i.e. \citealt{Tavakoli2016}; \citealt{Witton-Davies2014}).

In the CEFR, interaction is described as a language activity in which: 

\begin{quote}
at least two individuals participate in an oral and/or written exchange in which production and reception alternate and may, in fact, overlap in oral communication. Not only may two interlocutors be speaking and yet listening to each other simultaneously. Even where turn-taking is strictly respected, the listener is generally already forecasting the remainder of the speaker’s message and preparing a response. Learning to interact thus involves more than learning to receive and to produce utterances (\citealt[14]{CouncilOfEurope2001}).
\end{quote}

In its face-to-face form, interaction requires several skills simultaneously. Learners’ productive and receptive skills come into play, as well as additional abilities which allow speakers to monitor the development of the conversation and constantly adjust to it in real-time. To manage the floor, speakers use turn-taking strategies and during the exchange, they co-operate and build the conversation together. Speakers also need to be able to provide feedback, cope with unexpected misunderstandings, ask for clarification, or repair communication breakdowns. Thus, interlocutors constantly evaluate the on-going process to be able to react appropriately. The convergence of all these variables in real time and face-to-face results in a high degree of complexity, to which a second language learner might need to get accustomed before these processes become as automatic as in their first language (L1 – for the concept of automatization in L2 see \citealt{Kormos2006}). This definition focuses especially on the temporal or rhythmical aspect of face-to-face interaction. Therefore, I will refer to this specific aspect of general interactional competence as \textit{interactional fluency} (which pertains to the fluency of the co-constructed conversational rhythm by the interlocutors, see \citealt{Peltonen2024,Peltonen2020}).

The specific abilities related to turn-taking and feedback signals, which essentially contribute to interactional fluency, are described in the CEFR as follows:

\begin{quote}
Taking the floor (Turntaking) is concerned with the ability to take the discourse initiative. <…> Key concepts operationalised in the scale include the following: initiating, maintaining and ending conversation; intervening in an existing conversation or discussion, often using a prefabricated expression to do so, or to gain time to think (\citealt{GoodierNorth2018}: 140) 
\end{quote}

\begin{quote}
Cooperating concerns collaborative discourse moves intended to help a discussion develop. Key concepts operationalised in the scale include the following: confirming comprehension (lower levels); ability to give feedback and relate one’s own contribution to that of previous speakers (higher levels); <…> (\citealt{GoodierNorth2018}: 101)
\end{quote}

Research has also provided experimental evidence for the centrality of these abilities. Turn-taking, to which turn-regulating backchannels contribute, is the foundation of the interaction, representing its empty (i.e. content-free) rhythmic structure, and results from a form of cooperation (see \citealt{Wehrle2023} for an extensive discussion). Temporally coordinated collaborative activities, i.e. manual labour, dancing or music-making (see e.g. \citealt{HawkinsEtAl2013}), are greatly attested in human beings. Similarly, tightly synchronised and regulated communicative vocal and/or gestural turn-taking is attested across different species (\citealt{PikaEtAl2018}; \citealt{RavignaniEtAl2019}; \citealt{TakahashiEtAl2016}). However, human turn-taking appears particularly remarkable for several reasons: 1. it is executed with extreme precision and flexibility, 2. it involves the simultaneous prediction, planning and production of utterances, and 3. it is the means through which human language, and to a certain extent human culture, are learned and transmitted \citep{Schegloff2006}. For these reasons, turn-taking and turn-regulating backchannels are two skills to optimally start with when investigating interactional phenomena. This applies even more to SLA research, where very few studies have been carried out on turn-taking and backchannels despite them being primary and perceptually salient abilities in interactions. Indeed, speakers across different native languages have been found to be highly sensitive to turn-timing and tend to avoid very long gaps and overlaps in favour of smooth alternations of turns (\citealt{LevinsonTorreira2015}; \citealt{StiversEtAl2009}) because less smooth transitions can cause misunderstandings. For example, an excessively long gap may be interpreted by the recipient as some sort of problem from the interlocutor’s side, such as a difficulty in answering affirmatively (\citealt{RobertsEtAl2011}), or uncertainty in reacting to the turn \citep{Levinson1983}. Feedback expressions, or backchannels, can be used to signal comprehension and invite the interlocutor to continue speaking, or start their own turn. Furthermore, interactive practices are often bound to culture, which might represent an additional obstacle for learners. Therefore, mastering turn-taking and backchanneling conventions is necessary for L2 learners to smoothly manage the floor by correctly interpreting and expressing any intention of taking or relinquishing a turn.

\subsection{The gap between the theory and practice of language teaching and learning}
\label{sec:1.1.3}
\largerpage

Despite the centrality of spoken interaction being recognised by the CEFR and the CLT approach prevailing in language teaching, prosodic and interactional skills are still neglected in L2 classrooms. Moreover, SLA research extensively reports that learners transfer prosodic (e.g., \citealt{RasierHiligsmann2007}; \citealt{GoadWhite2019}; \citealt{NavaZubizarreta2008}; \citealt{AustinEtAl2022}; \citealt{ZhangEtAl2023}) and interactional features (e.g., \citealt{Reinhardt2022}; \citealt{SbrannaEtAl2020}; \citealt{Handley2024}; \citealt{ClingwallEtAl2024}) from their L1 to their L2, showing that there is a discrepancy between the theory and practice of language teaching and learning, which has an impact on the successful acquisition of the relevant abilities by L2 learners. 

One reason for this is the lack of a shared domain between L2 research and pedagogy, which impedes the exchange of knowledge about the results of empirical studies in laboratory settings (prevalent as compared to those in naturalistic settings, see \citealt{LoewenSato2018}) and the concrete application of teaching methods in classroom settings. This gap represents the greatest limitation to the application of research findings to teaching practice. As a result, teachers are left to rely on their own native speaker intuition when it comes to teaching both prosody (\citealt{DerwingMunro2015}) and interactional schemata (\citealt{Campbell-Larsen2022}; \citealt{TavakoliHunter2018}). This is problematic because teachers generally have only an unconscious competence of these aspects of their native language and culture, so that their intuitions about the pragmatic use of their L1 may not always be accurate \citep{Cohen2005}. Support for this claim comes from teachers reporting that they do not to teach pronunciation due to a lack of confidence, skills and knowledge \citep{Macdonald2002} and from the lack of a common understanding of what fluency is or how it can be fostered in class (\citealt{Morrison2018}; \citealt{Tavakoli2023}). Very importantly, many L2 teachers are not native speakers and cannot rely on L1 intuition, in which case a conscious and systematic knowledge of prosody and interactional behaviour is indispensable. For these reasons, it is necessary to provide a solid foundation for applied teaching methods by explicitly addressing research findings on language- and culture-specific aspects of language. Having an empirically-based understanding of these aspects of a language would allow native and non-native teachers to integrate them into their L2 teaching practice.

The aim of this book is to contribute to the empirically-based knowledge of SLA research. Even if these studies are based on a specific population of learners, their findings have broader implications for language research and pedagogy, encouraging replication with other language pairs and the application to teaching practices.

\section{Outline of the book}
\label{sec:1.2}

The book includes three studies, the topics of which deal with melodic (i.e. prosody) and temporal aspects of L2 interactions (i.e. turn-taking and backchannels which contribute to turn alternation). 

\chapref{chap:2} is dedicated to prosodic competence and contains an extensive study on the prosodic marking of information status, which is mentioned in the CEFR as a key ability to master in order to transmit and interpret the partitioning of the message into new or important vs. given or less important information. Previous research comparing West-Germanic languages and Italian has shown that the prosodic marking of information status within noun phrases (NPs) involves a different distribution of pitch accents, i.e. modulations of the fundamental frequency (F0) that are realised on the accented syllable of a word and make it sound prominent (\citealt{AvesaniEtAl2015}; \citealt{SwertsEtAl2002}). L1 speakers of West-Germanic languages deaccent post-focal given information, whereas L1 Italians seem to always accent the second word of the noun phrase, regardless of information status. These studies report a discrete classification of accentuation or deaccentuation and a categorical description of the pitch accent type. However, such a categorical description alone can entail missing important information, since speakers may modulate continuous parameters to prosodically mark information structure. I explore these modulations within noun phrases produced by Italian learners of L2 German in the absence of explicit prosodic training using an innovative method based on periodic energy, which reflects pitch perception. Results contrast with previous findings on Italian and show that Italian speakers do differentiate information status within NPs using an information-status-specific timing of F0 movements on the first word, both in production and perception. Comparing the strategies developed in the L2 to learners’ native and target languages shows that learners transfer their L1 strategy to the L2, irrespective of their proficiency level, by differentiating information status through a modulation of F0 alignment. However, contrary to productions in their native language and akin to the post-focal given German condition, learners prosodically attenuate the second word of the NPs across the board, probably perceiving the native German deaccentuation as a salient feature of the language. I further show that the method used provides similar or, in some cases, clearer results than well-established measures for acoustic analysis.

\chapref{chap:3} and \chapref{chap:4} are devoted to skills belonging to interactional competence and comprise two studies on turn-taking and backchannels, respectively. These are both underexplored areas of L2 research but core skills that complement each other in spoken interaction.

\chapref{chap:3} contains the study on turn-taking. This study is inspired by the discrepancy between the theory of the CEFR, rooted in the social paradigm, and the reality of many L2 assessment practices that are still based on grammar and the lexicon. The ability to converse is often only impressionistically evaluated. Thus, I propose a method for the quantification and visualisation of L2 interaction management across different levels of L2 proficiency, suggesting it as a possible starting point for the assessment of L2 interactional competence. The analysis based on the quantification of speech time, silence, overlap, backchannels and dialogue duration shows that the smoothness of the turn-taking system is affected by L2 proficiency, with more overall silence and less speech time for beginners. No effect of lexical competence was found on learners’ conversational patterns, which indicates the independence of the two types of competence. With the same method, I also propose a preliminary cross-linguistic comparison between native Italian and German interactions (learners’ native and target language respectively), pointing out that Germans might have a more careful approach to task completion, with longer speech time and dialogue duration. Thus, the method proposed appears to be useful to capture differences in turn-taking practices across L2 proficiency levels, as well as languages.

\chapref{chap:4} is dedicated to backchannels. This study is complementary to the one on turn-taking, as I take into account backchannels with turn-regulating functions, the production and interpretation of which contributes to a smooth turn alternation. Previous research reports that backchannels have positive social implications, signalling engagement in a conversation. Nevertheless, their language-specificity represents an obstacle for L2 learners, who might face miscommunications and misperceptions in case of an inappropriate use of backchannels in intercultural conversations. This study aims at investigating whether learners acquire target-like backchannelling behaviour even in the absence of explicit instructions, since this aspect of interactional competence is generally not thematized in L2 classroom settings. I present an in-depth study across native Italian and German and in L2 German, accounting for several backchannel features. In both L1s, findings show a complex, non-arbitrary mapping between lexical type, function and intonation. Backchannel frequency was found to be similar across languages, while backchannel duration is longer in German. The learners’ proficiency seems to only play a role in the lexical choice of backchannels, whilst dyad-specific patterns appear to account for the frequency and duration of backchannels in the L2 better than proficiency.

\chapref{chap:5} concludes the book. It begins with a summary of the three studies, followed by a discussion of the findings in the context of second language research and teaching. Finally, the limitations of the present investigations are examined and concordant directions for future research are suggested.

\section{Corpora}
\label{sec:1.3}

Since participants and the data collection method are shared between the three studies, all information relative to these aspects will be given in this introductory chapter and referred to throughout the book.

\subsection{Participants}
\label{sec:1.3.1}

40 Italian native speakers learning German, 14 Italian monolingual\footnote{By “monolingual” I mean that, despite having been exposed to some foreign languages in school, at the moment of the recording they were neither proficient in nor active speakers of any foreign language, due to the lack of exposure and exercise, as opposed to the learners of German.} native speakers and 18 German native speakers were recorded while performing dyadic task-based conversations.

All Italian speakers had grown up in the dialectal area of Naples with parents of the same origin. Thus, variation resulting from their native linguistic substratum can be ruled out. Learners were either students at the Goethe Institute in Naples (aged between 23 and 65, mean = 33; median = 30; SD = 12.29; 6 females, 4 males), or at the Department of Literary, Linguistic and Comparative Studies (It.: Dipartimento di Studi Letterari, Linguistici e Comparati) at the L’Orientale University of Naples (aged between 19 and 25, mean = 21; median = 20; SD = 1.2; 27 females, 3 males), with German as a foreign language as one of their main subjects.\footnote{Twenty-four learners had spent time in a German-speaking country for varying durations (ranging from one to ten months), either for a short language course or an exchange program at a partner university. However, the impact of time spent abroad is neither easily quantifiable nor uniform across individuals and depends on factors such as the amount of exposure to and use of the foreign language. For instance, some exchange students may struggle to establish regular contact with locals or choose not to enrol in a German language course. That said, because immersion in a foreign language generally contributes to overall proficiency, I did not treat this variable separately.} Italian monolingual speakers were students of subjects other than foreign languages at different universities in Naples (aged between 19 and 24, mean = 22; median = 22; SD = 1.80; 10 females, 4 males). They reported to have been exposed to English, or in some cases French and Spanish, in school. However, at the moment of the recording they only had a passive reminiscence of these languages, thus at proficiency levels which should not be assumed to affect their native language.

Native German participants came from different dialect areas,\footnote{Due to the limitations imposed by the pandemic it was not possible to strictly select German native speakers from the same dialect area. In particular, they were born in places above the Benrath line \citep{Wenker1877} and between the latter and the Speyer line \citep{Paul2013}. Thirteen speakers were from North Rhine-Westphalia, two from Lower Saxony and three from Hesse. However, nobody reported a mastery of the dialectal variety of their place of origin.} but had been living in Cologne for at least three years at the moment of the recording, and were students of subjects other than languages at the University of Cologne (aged between 22 and 27, mean = 24; median = 24.5; SD = 2.49; 11 females, 7 males).

No subject reported to have ever received specific phonetic and/or interactional training, nor to suffer from any speech or hearing problem. As colours were involved in one elicitation game, I also made sure that no participant was colour-blind.

Learners’ proficiency levels were established on the basis of the language courses they were attending at the time of the recordings and ranged from A2 to C1 CEFR levels, in which the notations “A”, “B”, “C” for proficiency levels correspond to beginner, intermediate and advanced levels of competence in German. The proficiency groups resulting from the data collection were unbalanced in number, with only six A- and four C-level learners. Thus, for the sake of a more reliable statistical analysis, learners were recategorised into two proficiency groups, each with a similar number of participants. Nevertheless, speaker- or dyad-specific variability is discussed in the three studies whenever it was found to be relevant for the interpretation of the results on the group level. I defined learners with A2 and B1 levels as beginners and learners with B2 and C1 levels as advanced. This division is not only based on the midpoint of the CEFR proficiency scale, but also on the structure of the reference levels themselves. Indeed, the gap between the abilities required by the B1 (also called “Threshold”) and B2 (also called “Vantage”) levels is greater than the one between C1 and B2, which makes it a suitable demarcation line for recategorising proficiency levels into two groups only. All dyads of learners, alongside their corresponding CEFR and recategorised levels of L2 German, are listed in \tabref{tab:key:1}.\footnote{Due to participant availability, dyads 11 to 14 are mixed, i.e. they are composed of one learner at B1 level and one learner at B2 level. Mixed dyads were recategorised based on the proficiency level of the instruction giver (a role in the Map Task, which generally leads the conversation), except for ME. I did so as the giver of this dyad was following a B1.2 course, whereby the notation “.2” is used to describe CEFR proficiency levels with a greater degree of detail and stands for an advanced mastering of the level it accompanies. Moreover, this participant had recently spent 9 months in Germany. As a result, his performance, i.e. fluency, vocabulary, grammar structures, was comparable to those of B2 learners.}

\begin{table}
\begin{tabular}{lcr}
\lsptoprule
{Dyad ID} & {CEFR level} & {Recategorised level}\\
\midrule
1  IF & A1 & Beginner\\
2  CV & A2 & Beginner\\
3  AR & A2 & Beginner\\
4  RM & B1 & Beginner\\
5  CC & B1 & Beginner\\
6  AA & B1 & Beginner\\
7  AC & B1 & Beginner\\
8  AN & B1 & Beginner\\
9  GS & B1 & Beginner\\
10  GA & B1 & Beginner\\
11  CA & B1-B2 & Advanced\\
12  RS & B1-B2 & Advanced\\
13  CR & B1-B2 & Advanced\\
14  ME & B1-B2 & Advanced\\
15  AB & B2 & Advanced\\
16  CE & B2 & Advanced\\
17  MA & B2 & Advanced\\
18  RC & B2 & Advanced\\
19  FF & C1 & Advanced\\
20  BS & C1 & Advanced\\
\lspbottomrule
\end{tabular}
\caption{Proficiency level of learner dyads.}
\label{tab:key:1}
\end{table}

\subsection{Data collection}
\label{sec:1.3.2}

Mono recordings of uncompressed WAV files at 44.1 kHz sample-rate and 16-bit depth were collected using headset microphones (AKG C 544 L) connected through an audio interface (Alesis iO2 Express). Each recording session included two conversational games: a semi-spontaneous interactive board game to elicit the prosodic marking of information status, and, subsequently, a conversational game, the Map Task \citep{AndersonEtAl1991}. The two conversational games are explained in the respective studies.

Three groups of participants took part in the experiment and were recorded in pairs: Italian learners of German, who were recorded in their native Italian and L2 German, Italian monolingual native speakers, and German monolingual native speakers. Italian learners of German could self-select their partner for the recordings with the only requirement that they had the same or a similar proficiency level of L2 German, i.e. they were classmates at the university or the Goethe Institute. L1 Italian and German monolingual speakers were also preferably matched following the criterion of self-selection and, in a minority of cases, based on the participants’ schedule. The Italian paired learners were mostly classmates and already knew each other before the recording session; Italian monolingual pairs were mostly matched according to their availability and the majority of them did not know each other before the recording, while native German pairs were all self-selected and speakers already knew each other.

In each recording session, two participants sat at two opposite sides of a table. Eye-contact and signal interference between the two microphones were prevented using an acoustic insulator dividing panel which was opaque, making it impossible for participants to see each other and the other person’s materials in order to maximise oral communication. Controlling this parameter was particularly crucial for the studies focussing on verbal interactional skills of learners, since backchannels and turn-taking can be conveyed via non-verbal cues like eye gaze, head movements, gestures and facial expressions (which are visual communicative channels in oral face-to-face interaction, see \citealt{KimEtAl2024}; \citealt{McDonoughEtAl2024}, but not the focus of this research). All pairs first played the board game and then performed the Map Task. Both tasks were introduced by written instruction and participants were given the chance to ask clarification questions before the beginning of the tasks. In the case of learners, the tasks were first completed in Italian and then repeated in German. This fixed order was chosen to prevent potential issues arising from misunderstandings of instructions in a foreign language and to avoid introducing additional variability due to task order, which might have obscured proficiency effects. Before carrying out the same tasks in their second language, learners watched video instructions explained by a German native speaker to help them get into the language mode and reduce L1 bias. Finally, all speakers were provided with a sociolinguistic questionnaire. Learners also completed the German version of LexTALE (\citealt{LemhöferBroersma2012}), an online test for L2 lexical competence.

Italian participants were recorded at the Goethe Institute in Naples. The recording session lasted approximately 90 minutes for learners since they performed the tasks in both their L1 and L2, whereas Italian monolinguals took 45 minutes. German native speakers were recorded at the University of Cologne and recording sessions lasted 45 minutes.

Finally, some ethical aspects are in order. The study was conducted in compliance with ethical standards for research involving human participants. All participants were provided with and signed an informed consent prior to their participation in the study, acknowledging their voluntary involvement and understanding of the research purpose. The consent process ensured participants were aware of their rights to withdraw at any time without repercussions. Participants received financial compensation for their time. Data processing adhered to the principles of GDPR compliance, ensuring the protection and confidentiality of personal data. Any identifying information was anonymised during data analysis and storage.

\subsection{ Data use across the studies}
\label{sec:1.3.3}

For each of the two games used for data elicitation, the resulting corpus consists of twenty dialogues in Italian and twenty dialogues in German as an L2 by Italian learners, seven dialogues in Italian L1 by monolinguals and nine dialogues by German native speakers.

Considering that during classes, exposure to the target prosody is proportionally greater than exposure to culture-specific interaction mechanisms, prosodic features might be implicitly acquired by repeatedly listening to the native teacher and recorded listening exercises, but no spontaneous dyadic interaction between two natives can be observed in a classroom setting (apart from exceptional cases of guests and exchange students) for implicit learning to take place. Therefore, in \chapref{chap:2} dedicated to prosodic competence, the Italian monolingual group will be used as a control group for learners’ Italian prosodic realisations since the learners’ native prosody might have been influenced by the German-language dominant setting of the recordings, i.e. the Goethe Institute.

In \chapref{chap:3} and \chapref{chap:4}, dealing with interactional competence, I will not take into account the Italian monolingual control group. It is reasonable to suppose that learners’ L2 style of interaction will not be different from their native one, since spoken interaction in L2 classrooms is mostly exercised among students themselves, and there is not enough foreign exposure to favour the acquisition of a target-like conversational style.

The test for L2 lexical competence is discussed in \chapref{chap:3}, in which I compare lexical competence to overall communicative competence.
