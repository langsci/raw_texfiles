\chapter{Conclusion}
\label{chap:5}
The goal of this book was to provide a theoretical and methodological foundation for understanding learners’ development of spoken interactional skills in the classroom context, with the ultimate aim of advancing the applied field of second language teaching. Its motivation was driven by the principles of the Common European Framework of Reference for Languages (CEFR, \citealt{CouncilOfEurope2001}), which describes learners as ``social agents", who constantly need to accomplish communicative tasks as members of society. Hence, language competence is considered as communicative competence, emphasising its interactional aspects rather than its mere linguistic aspects, such as grammar and the lexicon.

Considering the centrality of spoken interaction for daily communication and the descriptors of learners’ skills in the CEFR, I carried out three studies on crucial but neglected abilities in second language teaching and learning: prosodic competence – specifically, prosodic highlighting of important or new information in the message – and interactional competence – specifically, two aspects of interactional fluency: turn-taking and vocal feedback signals (backchannels). These studies were carried out on Italian learners of German as compared to German native speakers. However, the results have broader theoretical and methodological implications for second language acquisition, showing that much joint work by researchers and pedagogues is still needed to enable concrete and beneficial applications for L2 learners, irrespective of the native and target languages.

After summarising the relevance and findings of the three studies,\footnote{The following summaries offer a brief and conclusive overview of the studies conducted. Detailed recapitulations and discussions are available in \sectref{sec:2.9} for prosodic marking of information status, \sectref{sec:3.6} for turn-taking, and \sectref{sec:4.4} for backchannels.} I will discuss their implications for second language acquisition research and language teaching, and, finally, indicate some limitations and suggest future directions.

\section{Summary}
\subsection{Prosodic marking of information status}
In \chapref{chap:2}, I investigated Italian learners’ ability to prosodically mark information status within noun phrases in L2 German. Previous studies report that Italian speakers do not seem to mark post-focal given information in noun phrases, which is instead always accented, and that they transfer this prosodic behaviour to their L2 German. These previous studies have also supported their results with a perception experiment in which Italian listeners could not reconstruct the context of noun phrases from their prosodic realisations only, suggesting that no prosodic cues were available for interpreting the information status of the elicited noun phrases. However, these studies approached the question in terms of a categorical presence or absence of accentuation, overlooking continuous information relating to the modulation of prosodic cues.

I investigated such modulation using an innovative method based on periodic energy and F0 measurements to quantify prosodic aspects related to prosodic strength and F0 contours. Furthermore, I tried to overcome some limitations of previous studies regarding sample size and elicitation method by collecting a larger data sample with an interactive elicitation game.

Results stand in contrast to previous findings with regard to both Italian learners’ L1 and L2. The Italian learners of the present study marked post-focal information within noun phrases in their L1 but did so by using distinct F0 modulations on the first word of the noun phrase, instead of prosodically attenuating the post-focal given element, as in West-Germanic languages. Although they tend to transfer their native intonation contour when speaking German, Italian learners are able to reproduce the typically German post-focal reduction of prosodic strength. However, learners apply it in all pragmatic conditions across all proficiency levels (including beginners), suggesting that they might identify this cue to deaccentuation as, instead, a salient marker of native German. Its application irrespective of function can be interpreted either as a form of hypercorrection or as negative transfer from their native language, in which prosodic strength does not mark information status contrasts. These findings were confirmed by an additional analysis from an Autosegmental-Metrical perspective, using established measurements for the continuous investigation of F0 contours and prosodic strength, and a categorical interpretation of the results in terms of pitch accent type.

The current study revealed L1 and L2 prosodic patterns that were not apparent in previous analyses and made a step toward greater ecological validity by employing an innovative periodic-energy-based method for phonetic analysis and an interactional data collection approach adapted to the need for controlled data. Its results are relevant to L2 teaching, as they identify critical aspects of L2 prosody acquisition that should be addressed in pedagogical contexts.

\subsection{Turn-taking}
In \chapref{chap:3}, I discussed the absence of a standardised instrument for the quantification of interactional competence in L2 and proposed a workflow as a possible starting point for an objective assessment of L2 interactional competence. The proposed method includes quantification and visualisation tools of the L2 based on temporal measures. Its informativeness with regard to interaction management in an L2 across different proficiency levels was tested on goal-oriented cooperative dialogues performed by learner pairs matched by proficiency.

Specifically, I adapted a visualisation tool to display the dynamics of floor management in dyadic interactions operationalised as proportions of conversational activities: the time spent speaking for each interlocutor, the total amount of silence, the time of overlapping turns and backchannels (to distinguish the special status of the latter, as they do not constitute turns in themselves). In addition, the total duration of the dialogue was also taken into account. The extracted data were used to assess differences in the proportion of these metrics across proficiency levels and explore which were the best predictors of learner proficiency in terms of similarity to their native baseline.

Overall, the results suggested that silence and speech time of the instruction giver in the task robustly distinguish beginner and advanced learners and that higher proficiency corresponds with less overall silence and more speech time. Conversely, at low levels of proficiency, the cognitive difficulties of speaking an L2 can lead to less fluent interactions, with half of the conversation consisting of silence.

A qualitative by-dyad analysis integrates the by-group results, suggesting that from B2 level, learners present more similar patterns across their L1 and L2; this is an observation which has to be tested further on a larger data sample. Finally, some cross-linguistic/cross-cultural differences in the total duration of the dialogue and speech time of the instruction follower emerged from an exploratory analysis of native German interactions as compared to native Italian ones. This preliminary result suggests that there was a higher degree of diligence by and collaboration between native German speakers in the completion of the task.

These quantification and visualisation tools, based on temporal turn-taking metrics, demonstrate that reliably identifiable conversational activities are related to L2 proficiency, opening up possibilities for follow-up research in the field of interactional competence assessment, focused on implementing content-related information.


\subsection{Backchannels}
In \chapref{chap:4}, I carried out an in-depth analysis of backchannels across native Italian and German and in L2 German spoken by Italian learners. Previous research on backchannelling across languages and in L2 have been conducted only regarding some aspects of backchannel use and on limited language pairs. In order to provide a fully comprehensive view of the phenomenon in a new language pair, the various aspects of backchannels analysed here were: frequency of occurrence, length, lexical vs non-lexical type, structure (single- or multi-word), turn-taking function and prosodic realisation.

From a cross-linguistic perspective, the two native languages were found to have a similar BC frequency, but different BC length not ascribable to their structure (i.e. how many BCs are concatenated into a single BC production), with Italians producing longer backchannels than Germans. A complex, non-arbitrary, language-specific mapping between lexical type, function and intonation was found in both languages. Overall, there is a preference for producing non-turn-initial acknowledgements with a rising contour and turn-initial acknowledgements with a falling contour. Nevertheless, for some types the function seems to be overridden by the (non-)lexical type, so that the prosodic outcome is mostly independent of the function expressed. For learners, dyad-specific variability disentangled apparent cases of a progressive acquisition of target features, such as BC frequency and length, showing that learners do not approximate an ideal L2 target, but tend to instead approach their own L1 baseline. In the case of BC frequency, learners’ closer approximation to their L1 baseline results from higher proficiency in the L2, probably as an effect of improved overall fluency. Learners also tended to reproduce their L1 patterns in mapping lexical type, function, and intonation. Only the advanced learners were found to use typically German backchannels, although they did not always match the corresponding intonation contour.

In sum, L2 speakers showed similar backchanneling behaviour in their native language and in the L2, apart from a reduced frequency compared to both native languages. This transfer of native features to the L2 points to possible challenges in intercultural communication and remains to be explored further.

\section{Implications for second language research and teaching}

The results of these three studies are relevant not only for Italian learners of German. There is a great deal of literature dealing with cross-linguistic differences in the prosodic marking of information status, and a typological categorisation has also been proposed. The study presented in this book has shown that new methods and designs can enrich the existing knowledge, bringing to light overlooked linguistic phenomena and encouraging further research on this topic, especially taking into account the improved ecological validity of the experimental procedure used here and the application of various methodological approaches. Despite findings showing that the L2 learners under study do not operate a complete transfer of prosodic patterns as pointed out by previous studies (\citealt{SwertsEtAl2002}; \citealt{AvesaniEtAl2015}; \citealt{AvesaniEtAl2013}), these learners still show the interference of the L1 phonological rules. Thus, they do not implicitly acquire the correct foreign prosodic patterns in L2 classrooms, missing the link between form and function. This result is in line with other studies on prosodic marking of information status in L2 reporting cases of phonetic and phonological transfer from the L1, or realisations otherwise deviant from the target norm (see \sectref{sec:2.1.6}). Given that the relation between prosodic realisation (form) and meaning expressed (function) differs across languages and that learners are not able to master it solely from the input, the results presented here point to the necessity of creating pedagogical tools applicable to L2 classroom settings. This aim would require a joint effort by pedagogists and phoneticians, as most previous studies (see \sectref{sec:2.9} for a detailed discussion on prosodic training techniques) have been conducted in the field of phonetics, lacking a strong pedagogical framework, which makes it difficult to identify practicable pedagogical techniques for teaching prosody in L2 classrooms.

Turn-taking and backchannelling conventions are interactional aspects of communication which are not only language-specific, but also culture-specific. Most studies on intercultural interaction have been conducted on cultures and languages belonging to different continents. However, few studies have been conducted on intercultural communication within Europe, probably assuming that the long-established contact among inhabitants of these countries would reduce differences and favour adaptation. The present results on both turn-taking and backchanneling behaviour suggest some language- and culture-specific interactional conventions, which differ even between geographically close language communities among which there is well-established and long-lasting contact (for the specific case of South Italians and Germans, the historical factor of immigration has also contributed to bringing the two cultures closer together). These differences in interactional features and conversational patterns might seem more subtle than those found between some previously explored language pairs, such as American English–Japanese, Canadian English–Chinese or Vietnamese–German, but they are likely to still be noticeable by listeners and can potentially cause misunderstandings and/or misperceptions. Indeed, listeners’ high sensitivity to both backchanneling and turn-timing is widely attested. Thus, research investigating language- and culture-specific interactional behaviour can have beneficial pedagogical applications in the field of intercultural communication, and can increase awareness, comprehension and an acceptance of differences in multicultural societies. The findings of the studies presented here on aspects of interactional competence show that these learners do not achieve a target-like reproduction of interactional cues from the sole exposure to the target language in L2 classrooms, which points to the necessity of specific pedagogical tools for interactional skills. The TBLT framework would lend itself well to the case of interactional abilities, as it has shown positive effects on learners’ interactional competence in previous studies (\citealt{Pérez2016}; \citealt{Waluyo2019}; \citealt{FangEtAl2021}; see also \citealt{ZieglerBryfonski2016} for the theoretical framework).

Currently, both prosodic and interactional skills do not generally receive enough attention in the field of language teaching. In the past, this was due to a predominantly lexicogrammatical approach and a focus on written exercises in both teaching and assessment settings. Nowadays, even within the relatively recent shift towards the communicative approach in L2 pedagogy, which emphasises the use of language rather than its theoretical knowledge, there is often poor empirical understanding of these aspects, preventing them from being discussed and highlighted explicitly in classrooms. Indeed, realizing the central role of these skills in communication is not straightforward from a non-specialist and naïve perspective, as these abilities might not be seen as primary when compared to syntax and vocabulary in the elaboration of a linguistic message. However, especially given the renewed attention to spoken communication, they play a central role.

Prosody is essential for transmitting linguistic and paralinguistic information, as well as subtle shades of meaning, thus ensuring the correct interpretation of the message. A sufficiently smooth turn-taking system between interlocutors is necessary for avoiding the misinterpretation of long silences or long overlaps, together with possible negative culture-specific interpretations. Finally, backchannels are recognised as having a positive social value in conversation by manifesting attention and interest towards the primary speaker, as well as contributing to smoother turn transitions by signalling a possible desire to take the floor. Empirical evidence shows that unconscious acquisition only partially takes place in L2 classroom settings, and that it does not provide sufficiently good results. As discussed in the introduction of this book, teachers generally do not receive a training on prosodic and interactional competence, and even if native speakers of the language serve as instructors, they might not have clear intuitions about their unconscious use of the language. It is also important to remember that not all teachers are native speakers and might not be completely aware of subtle uses of prosody and cultural-specific interactional conventions. Finally, in some educational settings, language classes can be large, so that learners tend to exercise their communicative abilities among each other during practice sessions and only rarely speak with the teacher.

Due to these factors, the amount and quality of the input received in L2 classroom settings may not be enough to guarantee the implicit learning of prosodic and interactional target patterns. In our globalised and technologically advanced world, in which communication is enabled in real-time and face-to-face independently of geographical distance, applied L2 research with the aim of increasing cross-linguistic and cross-cultural awareness is highly necessary, as is the development of effective didactic tools for L2 learning. This objective requires joint work from both researchers and teachers, bridging the gap between the two fields.

By highlighting the differences between two languages (German and Italian), the three studies presented in this book provide groundwork for pedagogical tools with a contrastive approach, that is, based on this specific language pair as the native and target languages of the learners. At the same time, individuating strategies of prosodic marking of information status, turn-taking and backchannelling relative to each language, results could also be taken as a starting point for developing training materials for learners of Italian and German with different L1 backgrounds.

\section{Limitations and future directions}
Although the findings of this study are of some significance, there are some limitations to their generalisability.

Firstly, as interlanguages are complex systems to which the existing knowledge contributes, results on these particular learners are to be considered in the light of the participants’ specific native and target language. Furthermore, the high regional linguistic variability present in Italy does not necessarily permit a generalisation of the findings to other Italian learners of German, especially with regards to phonology. 

Secondly, learners were categorised into two main proficiency groups to allow for more reliable statistical testing, since the sample contained more intermediate than beginner and advanced learners (according to the CEFR classification). To investigate the process of second language acquisition and critically discuss it in relation to the CEFR and its descriptors, a larger sample for each proficiency level is required. In this way, it would be possible to observe if learners fulfil the skill descriptors related to each level and, if not, intervene with apposite and efficient pedagogical tools. Ideally, a replication of these studies based on the CEFR proficiency categories could clarify the nature and degree of the variability in learners’ interactional behaviour found among groups. A longitudinal study could provide further evidence for the development of L2 skills within each dyad.

Thirdly, two interactive, task-oriented data collection methods were used: a semi-scripted conversational board game and a Map Task. Despite the effort made to design a more interactive board game as compared to previous elicitation methods, and the inclusion of the more spontaneous Map Task, the ecological validity of the experimental design can still be improved. A challenge for future research lies in finding the best compromise possible between the investigation of fully spontaneous speech and systematic data collection, in order to provide robust observations based on a sample which is more representative of real-life communication and of sufficient external validity. Concordantly, it has to be remembered that eye contact was blocked during recordings in order to foreground communication in the vocal channel, thereby eliminating the analysis, if not the use, of all possible non-verbal signals which might have contributed to the interaction in crucial ways. Therefore, a similar investigation with a multimodal approach would enrich our understanding of real-life, face-to-face conversations by investigating how the different channels complement each other and how they interact (for eye-gaze and vocal feedback, see \citealt{SpaniolEtAl2023}; \citealt{SbrannaGaze2025}; for a multimodal approach in L2 interaction, see \citealt{TsunemotoEtAl2022}; \citealt{McDonoughEtAl2020}).

Finally, the data collected for the analysis presented in this book only included conversations among peers (learners with learners and natives with natives) to reproduce and investigate second language acquisition in a classroom setting. However, a second language is learnt not only for use as a lingua franca in intercultural contexts, i.e. as a common language among non-natives, but also to be able to communicate with native speakers. Thus, mixed dyads of learners and natives should be investigated as well, since L1-L2 and L2-L2 conversations have shown differences in interactional features (e.g., \citealt{Shibata2023}; \citealt{KunitzYeh2023}; \citealt{JungCrossley2024}). Such conversations could clarify which deviations from the native norms have the most detrimental effect on communication. This assessment could be based on features causing communication breakdowns and/or ratings of the perceived success of the communication. Findings of such a study would highlight the central aspects on which educational tools should be based, with the ultimate goal of improving learners’ communicative skills and the ease of interaction.

Despite these limitations, the findings presented in this book enrich the existent body of knowledge in SLA studies and hopefully raise awareness about the extent of implicit learning in L2 classrooms. Future studies should ideally address the gap between research and pedagogy to benefit learners and improve intercultural communication.
