\chapter{Turn-taking fluency in L2 interactions}
\graphicspath{{figures/plots-chapter-3}}
\label{chap:3}
This chapter includes a study dedicated to a specific ability of L2 interactional fluency (\citealt{Peltonen2024,Peltonen2020}), that is turn-taking in dyadic interactions.\footnote{The analysis provided in this chapter is an extended and enriched version of previous work published in \citet{SbrannaEtAl2020}.}

I problematise the discrepancy between the theory of CEFR (\citealt{CouncilofEurope2020}) that describes learners as “social agents", emphasising the interactional aspect of SLA, and the practice of most assessment realities, focused on the quantification of grammar and lexical competence, while neglecting interactional aspects. An obstacle to assessing learners in interactions is the complexity of the communicative phenomena involved in this process and the shared responsibility of participants in the co-construction of meaning. 

With this exploratory study, I try to break down the complexity of interaction by starting from its very foundation: the management of the conversational floor. I test a method for visualisation and quantification of turn-taking fluency across L2 proficiency levels by extracting reliable and testable metrics. This method could represent a valid starting point to build upon and develop a more complete tool for a quantifiable assessment of interactional competence.

\section{Background}
\label{sec:3.1}
\subsection{Assessing interactional competence}
\label{sec:3.1.1}
The assessment of learner proficiency often tends to focus mainly on grammar and the lexicon, neglecting interactional aspects. Many language test formats only involve a written form, such as the cloze format, where some words of a text are replaced with gaps to be filled in by learners. These kinds of test are often used with the explanation that they show correlations to all receptive and productive abilities: reading, listening, writing and speaking (\citealt{CouncilOfEurope2001}). Nevertheless, they mainly put to test grammar and vocabulary knowledge, leaving out the full range of pragmatic and strategic resources required for oral interaction.

\citet{BachmanPalmer1996}, in “Language Testing Practice”, mention interactiveness as one of the fundamental characteristics a good quality language proficiency test should have: reliability, construct validity, authenticity and interactiveness. Reliability refers to the consistency of the measurement, and thereby of the results given by the test. Construct validity indicates the possibility of interpreting the score of the test as a valid indication of global language proficiency. Authenticity defines to what extent the task given to learners in test circumstances corresponds to real-life tasks they would perform using the L2. Finally, interactiveness refers to the degree of involvement of learners’ different abilities in accomplishing the task, i.e. the extent to which a test involves various learners’ skills, which include general language knowledge, metacognitive strategies and strategic competence for planning and dealing with unexpected difficulties, topical knowledge, and affective schemata, which refers to learners’ emotional response to the task. According to the authors, tasks with a high level of interactiveness are role-playing and long conversations, as they require learners to draw on all these abilities.

Some possible reasons for neglecting highly interactive tasks in language proficiency testing may be practical. \citet{HeYoung1998} point out that having learners interviewed by native or highly proficient speakers can create certain difficulties. First, such interviewers have to be available; secondly, the interviews need to be carried out for a reasonable length of time to allow the interviewer to elicit enough linguistic data from the learner so that these data can be considered representative of the learner’s global knowledge. Hence, such testing would require more assessors, be time-consuming and would consequently be more expensive than a test format such as the cloze test, which can optimise time for testing and correction. Even when an assessment of the L2 speaking ability is conducted, the quantification of the skills involved in interaction may turn out to be extremely time-consuming and complex to synthesise. As a result, interactional competence is often subject only to a qualitative evaluation based on illustrative scales, \footnote{An example from the illustrative scale for overall spoken interaction (C2 level): “I can take part effortlessly in any conversation or discussion and have a good familiarity with idiomatic expressions and colloquialisms. I can express myself fluently and convey finer shades of meaning precisely. If I do have a problem I can backtrack and restructure around the difficulty so smoothly that other people are hardly aware of it.” (\citealt{GoodierNorth2018}: 168).} the interpretation of which may include a certain degree of subjectivity. Another problematic aspect of L2 oral proficiency assessment in non-experimental environments is the method used, i.e. oral proficiency interviews where a native speaker tries to elicit linguistic information from learners using a script representative of real-life language settings. Although these interviews try to simulate ordinary conversation, they are limited by various constraints that can affect learners’ oral performance (\citealt{HeYoung1998}): interviews take place in an institutional setting; speech activities are predetermined; and participants have different statuses, two different L1s and cultural backgrounds, as well as two different proficiency levels of the language used during the interviews. Moreover, being an interaction among a tester and a learner, the attention is totally on the learner’s production, their fluency and accuracy, and not on the interactional patterns they are able to create, maintain and manage during the exchange with the tester. Consequently, 1) interviews risk not providing a snapshot of learners’ skills that would be representative of a more spontaneous and less formal style, such as peer conversations, and 2) risk focussing on learner fluency and accuracy, instead of on more specifically interactional abilities.

In line with this tendency in non-experimental environments, research has mainly focused on the measurement of \textit{individual} fluency – how fluent learner speech is \textit{within} turns – rather than \textit{interactional} fluency – how fluent the conversation is \textit{across} interlocutors’ turns – \citep{Peltonen2017} with the aim of proposing quantification methods for L2 speaking abilities. Fluency is consistently mentioned as a fundamental component of learners’ oral performance in various assessment traditions, and its correlation with general L2 proficiency has been demonstrated by several studies (\citealt{DeJongEtAl2013,DeJongEtAl2015}; \citealt{SegalowitzFreed2004}). Therefore, the following paragraphs present a brief review of theories and findings on fluency as a measure of L2 speaking proficiency.

\subsection{Fluency in L2 studies}
\label{sec:3.1.2}
Fluency has been identified as one of the main aspects that ensure the success of a speaking performance (\citealt{DeJong2016}). One of its first definitions can be traced back to \citet{Fillmore1979}. He defines fluency as a measure of how well a language is spoken, in other words, the skill of using L2 knowledge efficiently, and enumerates four dimensions of fluency, including both quantitative and qualitative aspects: the ability to speak at length with few breaks; the ability to speak in a coherent, reasoned, and semantically dense way; the ability to talk appropriately according to context; and the ability to be creative and imaginative in speech production.

\citet{Lennon1990} distinguishes between a broad and a narrow definition of fluency. In the broad sense, fluency encompasses overall language proficiency, including grammatical accuracy and vocabulary knowledge. In contrast, fluent speech in its narrow sense focuses specifically on the ease and automaticity of speech production and is defined as “unimpeded by silent pauses and hesitations, filled pauses (“ers” and “erms”), self-corrections, repetitions, false starts, and the like” (\citealt{Lennon1990}: 390). \citet{TavakoliHunter2018}  built upon these concepts by exploring how teachers perceive and teach fluency. Specifically, they argue that many teachers interpret fluency primarily in its broad sense, while remaining largely unfamiliar with its narrow sense.

\citet{Starkweather1987} instead suggests four dimensions of fluency mainly related to physical aspects of speech: continuity, rate, rhythm, and effort. In other words, fluent speech should present few discontinuities, have a regular rhythm and a fast rate, and not require too much cognitive and physical effort \citep{Zmarich2017}.

In his model of fluency, \citet{Logan2015} adds two additional dimensions: naturalness, i.e. how much speech resembles that uttered by a typical speaker with regard to continuity, rate, rhythm and effort; and stability, i.e. how similar a speaker’s performances are over time if subject to repeated measurements.

\citet{Segalowitz2010} focuses on L2 fluency from a dynamical systems perspective. He argues that fluency is strongly linked to the social context in which the speech performance takes place and distinguishes three aspects: utterance fluency, cognitive fluency, and perceived fluency. L2 utterance fluency refers to the fluidity observable in a speech sample and quantifiable by temporal measures, among which the author mentions syllable rate, duration and rate of hesitations, filled and silent pauses, breakdown fluency (pausing phenomena) and repair fluency (false starts, corrections, repetitions). Indeed, most studies calculating temporal measures for fluency follow the classification in the sub-components breakdown, repair, and speed fluency (\citealt{TavakoliSkehan2005}; \citealt{HuenschTracy-Ventura2017}; \citealt{TavakoliNakatsuharaHunter2020}; \citealt{LahmannSteinkraussSchmid2017}). L2 cognitive fluency refers to the fluidity of the cognitive processes underlying speech production, such as processing skills (declarative and procedural knowledge), efficiency and speed of semantic retrieval, and cognitive load in working memory. Some measures of cognitive fluency have been found to correlate with L2 proficiency, e.g. reaction time and switch cost among competing tasks. Reaction time and its coefficient of variability have been used to operationalise the efficiency of semantic retrieval (\citealt{SegalowitzFreed2004}), while switch cost has been used as an indicator of linguistic attention, which refers to attention shifting guided by connections among grammatical elements within utterances (\citealt{DuncanSegalowitzPhillips2014}).

Such a systemic understanding of fluency is also assumed by \citeapo{Kormos2006} psycholinguistic model, in which different cognitive processes underlie the three above-mentioned sub-components of fluency. In particular, breakdown measures are related to learner effort. For instance, final-clause pauses reflect a learner’s conceptualization and planning of the message, and mid-clause pauses represent the time taken by learners to encode and formulate linguistic information, while repair measures signal the monitoring of the speech output and consequently the amount of attention required for speaking the L2. Finally, speed-related measures provide information on the degree of automatization in all the above processes.

Another important aspect that should be taken into account is the fact that both utterance and cognitive fluency are specific to each person. Still, individual variability in an L1 can only partially explain individual variability in an L2 (\citealt{DeJongEtAl2013}) since disfluency is also characterised by L2-specific features, such as a higher cognitive load. Therefore, it may be good scientific practice to consider L1 fluency measures as a baseline (as in \citealt{DeJongEtAl2015}; \citealt{SaitoEtAl2018}) to get a clearer picture of L2-specific fluency measures by partialling out the variables that are not specifically related to L2 disfluency phenomena \citep{Segalowitz2010}.

Finally, perceived L2 fluency indicates subjective listeners’ ratings on how fluent a speaker is. One disadvantage is that, being subjective, perceived fluency is only moderately informative about utterance fluency and cannot explain all the variance in objective measures. However, it is helpful to gain an understanding of which cues are relevant to native listeners when judging L2 speech fluency in relation to L2 proficiency. Moreover, a listener’s judgment of their interlocutor’s fluency can affect the interaction and influence both speakers’ fluency.

\subsection{Fluency measures and operationalisations}
\label{sec:3.1.3}
Research on L2 fluency has focused on individuating which objective measures can best explain L2 fluency judgments and has mainly concentrated on temporal features. A categorisation of aspects of fluency comparable to the already-mentioned, more recent triad of “breakdown, repair and speed” (\citealt{TavakoliSkehan2005}) was already proposed in one pioneering study. \citet{Riggenbach1991} classified the features which can characterise a judgement as fluent or non-fluent in non-native speech into hesitation and repair phenomena as well as rate and amount of speech. The study also includes an analysis of interactive features contributing to the turn-taking alternation, such as overlaps, pauses between turns, and collaborative completions. Hesitation phenomena and speech rate were found to be significantly correlated with ratings of L2 fluency, with hesitation placement and the resulting discourse chunking playing a central role. In contrast, results related to repair phenomena appeared to be less clear, probably due to the small amount of data. The same holds for interactive features, which showed high variability due to the idiosyncratic nature of interactions, which vary according to many linguistic and non-linguistic factors. Furthermore, this pragmatic-oriented analysis is reported to be extremely time-consuming and, indeed, studies on turn-taking fluency are relatively scarce.

Later on, many studies confirmed that perceived fluency by both native (\citealt{BoskerEtAl2013}; \citealt{DerwingEtAl2004}; \citealt{KormosDénes2004}; \citealt{PréfontaineKormosJohnson2016}; \citealt{Rossiter2009}; \citealt{SaitoEtAl2018}; \citealt{SuzukiKormos2020}) and L2 listeners \citep{MagneEtAl2019} is closely related to speed of delivery and pausing phenomena.

As reported in \citet{SuzukiKormos2020}, a recent approach in research on fluency differentiates three independent dimensions of breakdown fluency – pause frequency, duration, and location – and all of them have been demonstrated to independently contribute to fluency. Moreover, concerning pause location, mid-clause pauses have been found to have a more distinctive role than clause-final pauses (\citealt{MagneEtAl2019}; \citealt{SaitoEtAl2018}; \citealt{SuzukiKormos2020}). One possible reason, following \citeapo{Kormos2006} model, would be that mid-clause pauses, being associated with the time required for linguistic encoding, are more representative of proficiency than clause-final pauses, which are associated with content planning.


\begin{table}[p]
\caption{\label{tab:3.1a}The most frequently used temporal and interactional measures in research on L2 fluency: Speed of speech measures.}
\footnotesize
\begin{tabularx}{\textwidth}{lQl@{}}
\lsptoprule
{} & Formula & {Reference}\\
\midrule
Speech rate & number of syllables / total time & \citet{Kormos2006} in: \citet{DeJong2016}*\\
Pruned speech rate & number of syllables - number of disfluent syllables / total time & \citet{DeJong2016}\\
Phonation time ratio & speaking time / total time & \citet{Kormos2006} in: \citet{DeJong2016}*\\
Articulation rate & number of syllables / speaking time & \citet{Kormos2006} in: \citet{DeJong2016}\\
Mean length of syllables & speaking time / number of syllables & \citet{DeJong2016}*\\
Mean length of run & number of silent pauses / number of syllables & \citet{Kormos2006}*\\
\lspbottomrule
\end{tabularx}
\end{table}

\begin{table}[p]
\footnotesize
\caption{\label{tab:3.1b}The most frequently used temporal and interactional measures in research on L2 fluency: Breakdown fluency measures}
\begin{tabularx}{\textwidth}{Q@{}QQ}
\lsptoprule
{} & Formula & {Reference}\\
\midrule
\textit{Frequency} & ~ & ~\\
\midrule
Number of pauses & number of pauses / total time or speaking time & \citet{SaitoEtAl2018} for references*\\
Number of silent pauses & number of silent pauses / total time or speaking time & \citet{Kormos2006} in: \citet{DeJong2016}*\\
Mean length of utterance & total speaking time / number of utterances & \citet{DeJong2016}\\
Number of filled pauses & number of filled pauses / total time or speaking time & \citet{Kormos2006} in: \citet{DeJong2016}\\
\midrule
\textit{Duration} & ~ & ~\\
\midrule
Duration of silent pauses & pausing time / number of silent pauses & \citet{Kormos2006} in: \citet{DeJong2016}\\
\midrule
\textit{Location} & ~ & ~\\
\midrule
Number of clause-medial pauses & number of clause-medial pauses / total time or speaking time & \citet{SaitoEtAl2018}*\\
Number of clause-final pauses & number of clause-final pauses / total time or speaking time & \citet{SaitoEtAl2018}*\\
\lspbottomrule
\end{tabularx}
\end{table}


However, there are several differences in methodology across these studies, in particular regarding rating methods, the task used for data collection, and the operationalisation of measures. \tabref{tab:3.1a}--\ref{tab:3.1d}\footnote{Measures marked by a star have been found to be significant predictors of L2 learner fluency. Notably, in a more recent framework by \citet{TavakoliNakatsuharaHunter2020}, articulation rate is identified as the principal measure of the speed dimension, while measures incorporating information about pauses alongside production speed are regarded as composite measures, combining dimensions of speed and breakdown.} summarise the most frequently listed temporal measures in literature reviews on fluency, as well as some interactional measures. Moreover, they differ considerably in sample size.

In particular, \citet{DerwingEtAl2004} argue that tasks used in experimental and assessment settings can have a considerable impact on learners’ L2 fluency, depending on the amount of freedom accorded to the participants. For example, picture narrative and description impose a given range of lexical and syntactic structures, while a monologue or a conversation with free choice of topic allow learners to have much more control on the content and the expressions used to deliver it. For this reason, a wider range of speaking tasks should be employed, in particular more open and interactive ones, which are more representative of our daily use of spoken language. Furthermore, especially in experimental settings, assuming L1 fluency measures as a baseline for each speaker’s L2 fluency can help explain some idiosyncratic differences by controlling for non-linguistic factors possibly affecting learner performance \citep{Segalowitz2016}, such as contextual factors (e.g. attitude towards the task and the interlocutor), but also participant-specific ones (e.g. personality and motivation).


\begin{table}[t]
\footnotesize
\caption{\label{tab:3.1c}The most frequently used temporal and interactional measures in research on L2 fluency: Repair fluency  measures}
\begin{tabularx}{\textwidth}{lQl}
\lsptoprule
{} & Formula & {Reference}\\
\midrule
Number of repetitions & number of repetitions / total time or speaking time & \citet{DeJong2016}*\\
Number of repairs & number of corrections and restarts / total time or speaking time & \citet{DeJong2016}*\\
\lspbottomrule
\end{tabularx}
\end{table}

\begin{table}[t]
\footnotesize
\caption{\label{tab:3.1d}The most frequently used temporal and interactional measures in research on L2 fluency: Turn-taking fluency measures}
\begin{tabularx}{\textwidth}{lQQ}
\lsptoprule
{} & Formula & {Reference}\\
\midrule
Pause & within speaker silence & \citet{HeldnerEdlund2010}\\
Gap & between speakers silence & \citet{HeldnerEdlund2010}\\
Overlap & turn-changing and non-turn-changing & \citet{HeldnerEdlund2010}\\
Backchannel & not taking the turn & \citet{Riggenbach1991} \\
Collaborative completion & attempt to complete a sentence or a phrase of the other & \citet{Riggenbach1991}; \citet{Peltonen2017}\\
Other-repetitions & repetition of part of the turn of the other & \citet{Peltonen2017}\\
\lspbottomrule
\end{tabularx}
\end{table}

Finally, it should be noted that the static perspective provided by averaged measures of learners’ performance offers only a limited view. Fluency, instead, is a dynamic phenomenon that fluctuates both between speakers (inter-speaker) and within the same speaker (intra-speaker) over time. A long tradition of research has focussed on how conceptual planning impacts fluency, providing evidence for the alternation between fluent and disfluent temporal cycles during oral performance (\citealt{HendersonGoldman-EislerSkarbek1966}; \citealt{Goldman-Eisler1967}; \citealt{Butterworth1975}; \citealt{Beattie1980}). In this context, low-fluency sequences are believed to correspond to the conceptualisation phase of speech production, while high-fluency sequences align with the less cognitively demanding phases of formulation and articulation. Currently, there is limited research on the dynamics of L2 fluency (\citealt{DeJong2023}) probably due to the fact that it is very time-consuming (\citealt{RobertsKirsner2000}). I contribute to the field by introducing a visualisation method to explore these dynamics, even though fluency dynamics are not the primary focus of this research.

\subsection{Interactional fluency visualisation methods}
\label{sec:3.1.4}
Most of the studies mentioned in the previous paragraphs have focused mainly on the concept of fluency as an individual phenomenon (with the exceptions of \citealt{Riggenbach1991}; \citealt{Tavakoli2016}; \citealt{Peltonen2017}, while \citealt{Peltonen2022} and \citealt{Sato2014} also considered interactional fluency, but from an assessment perspective). Mostly using monologic tasks for data collection, their main concern was to define if and to what extent a speaker is fluent, the perception from a listener’s perspective, and implications for L2 assessment (see also more recent work by \citealt{Götz2013}; \citealt{Kahng2014}; \citealt{PeltonenLintunen2016}).

Nevertheless, researchers agree that the circumstances in which learner-spoken performances take place are fundamental for making assumptions about the performance itself. Therefore, the measures above only provide incomplete information about L2 fluency if considered out of context. This is especially true in the case of interactions, which are collaborative in nature. For example, it has been suggested that speed of delivery and pausing behaviour are accommodated to the interlocutor during the interaction (\citealt{KousidisDorran2009}), so that a jointly achieved harmonisation of tempo occurs. This phenomenon has been depicted through the metaphor of interactional “flow” \citep{McCarthy2009}. Moreover, being ruled by the turn-taking system (\citealt{SacksSchegloffJefferson1974}), interactions more thoroughly put to the test automaticity in L2 speech. Indeed, turn-boundaries (also called \textit{transition relevance places} – TRP – in conversational analysis) are the places in which smooth or disfluent transition of turns can take place and which require the interlocutor to appropriately anticipate the end of their turn to be able to quickly react (\citealt{BögelsTorreira2015}; \citealt{Levinson2016}). For these reasons, judgements of fluency based on a single speaker and ignoring the interlocutor’s contribution to the conversation would lack the interactive perspective and important information about learners’ abilities to co-create fluency, keeping in mind that interaction is described as a co-creation process in the CEFR (\citealt{PiccardoEtAl2018}: 81). 

In the CEFR, both dimensions of individual and interactional fluency have their own descriptors in the assessment scale (\citealt[28–29]{CouncilOfEurope2001}), with fluency including aspects of individual fluency (such as speed, pausing, and repair fluency) and interaction encompassing aspects of interactional competence, “thus also containing elements that could be considered indicators of interactional fluency” \citep[30]{Peltonen2017}. In particular, turn-taking management is the only ability which is exclusively mentioned under spoken interaction, whereas many fluency-related abilities are shared across communicative activities. Turn-taking appears to be the distinguishing ability of interactional competence, hence, a good starting point when examining L2 interactions.

L2 interactions have been approached from several perspectives: using conversational analysis to explore learners’ interactional practices (\citealt{DoehlerBerger2015,DoehlerBerger2018}) and, in L2 assessment studies, examining interactional cues that contribute to create cohesion (e.g., feedback in \citealt{Galaczi2014}; \citealt{MayEtAl2020}, and collaborative completions and other-repetitions in \citealt{Peltonen2017}). However, very little quantitative research has been conducted on the timing of turn-taking in L2 (\citealt{SørensenEtAl2021}).

Studies focusing on conversational speech rhythm have developed similar methods for capturing the speech activity performed in dyadic exchanges. Visualisation methods of this kind were pioneered by \citet{Chapple1939} in the field of anthropology and applied later in several domains and to different research objectives in the field of speech studies (for a review see \citealt{CangemiEtAl2023}). This visualisation method has been used to visualise the timing organisation of the interaction, categorised into classes of activity. The most basic version would be to represent speech vs. silence for each of the participant to the interaction, but other classes of activities can be flexibly introduced based on the specific research question. The horizontal axis of the plot displays a time window of one minute, whereas the vertical axis reproduces time passing throughout the interaction. The speaking activity is then represented on this graphic scaffold by colour-coded bars, whose length represents the duration of each speaker’s turn. When the two different bars are on top of each other, speakers speak at the same time, causing an overlap, whilst when bars end and a white space follows, speakers are silent. The colour coding can be then used to identify other classes of activity.

Some examples of application include speech activity patterns in telephone conversations \citep{Campbell2007}, the interplay of laughing and speaking (\citealt{TrouvainTruong2013}), human-machine interaction \citep{GilmartinEtAl2018}, psychiatric interviews (i.e. the effects of disease on interaction patterns in \citealt{CangemiEtAl2023}), cross-linguistic comparison of turn-taking patterns (\citealt{DingemanseLiesenfeld2022}) and even multimodal communication (\citealt{RühlemannPtak2023}; \citealt{SpaniolEtAl2023}). 

Despite being content-free, this kind of chart displaying the duration and timing of specific classes of interest has three main benefits. First, it serves as an “eye-opener” (\citealt[4]{TrouvainTruong2013}) helping researchers to evaluate their intuitions by means of visual exploration and comparison of data through a close and analytical reading of speech activities. Secondly, in the case of speech activity, the annotation can be performed largely in an automatic way and manual verification, if required, consists in excluding intervals containing vegetative vocal activities (e.g. coughing) and undesired noises. This enables a quick analysis of large amounts of interactional data. Finally, the data extracted to create the plot provide the material for quantitative analysis and statistical testing. 

In the present study, this type of visualisation is tested on L2 interactional data to assess its representative power and usefulness in capturing differences in oral interaction management by learners with different levels of L2 proficiency. Even though some hypothesis testing will be conducted for demonstrative purposes only (extending a previous investigation carried out in \citealt{SbrannaEtAl2020}), due to its preliminary nature, the main goal of this study is to explore the potential of this visualisation and quantification method when applied to L2 data. Specifically, the aim is to evaluate whether these tools can help fill the gap in quantification methods for L2 interactional competence and serve as possible groundwork for developing a more complete instrument for a standardised assessment of L2 oral performance.

\section{Method}
\label{sec:3.2}
\subsection{Corpus}
\label{sec:3.2.1}
The corpus consists of thirty-nine task-based dialogues by forty Italian learners of German, nineteen performed in their native language\footnote{The file for dyad ME corresponding to the dialogue in Italian language was found to be damaged and could not be analysed.} – Italian, Neapolitan variety – and twenty in L2 German categorized at different proficiency levels ranging from A2 to C1 on the CEFR scale. For the purpose of hypothesis testing, learners had been recategorised into two homogeneous groups based on their proficiency levels, i.e. beginner (from A1 to B1 levels) and advanced learners (from B2 to C2 levels). However, some qualitative considerations on the development of their conversational patterns across their actual CEFR proficiency levels will also be accounted for and discussed.\footnote{All details on participants, data collection, and learner proficiency levels are presented in detail in \sectref{sec:1.3}.}

The corpus also includes nine dialogues performed by eighteen German native speakers, which will enable an exploratory cross-linguistic comparison of native conversational schemata, i.e. across Italian and German as L1s, but, crucially, not as a target for L2 learners. 

Following the suggestion of \citet{Segalowitz2016}, learner L1 will be used as a baseline against which to assess L2 interactional patterns. This represents a relatively novel approach in fluency research, which has so far primarily applied Segalowitz’s idea to monologic data. As mentioned in \sectref{sec:1.3.3}, it is implausible to suppose that the learners’ way of interacting in L2 German would be similar to an L1 German speaker, since at the moment of the recording, participants were living in Italy, studying in Italy and talking to an Italian interlocutor with whom they shared the same L1 and culture. There is not enough foreign exposure to and experience with a German native conversational style which could favour a possible native target, as in L2 classrooms, conversational skills are mostly practiced among learners themselves. Moreover, using the L1 as a baseline was informative in a previous pilot study (\citealt{SbrannaEtAl2020}), which showed that with increasing proficiency, the interaction in the L2 approached the same interactional pattern learners used in their own L1. 

\subsection{Elicitation method}
\label{sec:3.2.2}
To investigate interactional competence in SLA in classroom settings (discussed is in this and the following Chapter), I collected data using a task, inspired by the Task-Based Language Teaching approach (TBLT; \citealt{Long1985,Long2015}; see also \citealt{GassMackey2014}) which is increasingly being applied in L2 classrooms. This approach is rooted in communicative language teaching principles and consists in engaging learners in meaningful and authentic tasks that require the use of the target language and are designed to be relevant to learners' needs and interests. TBLT views language as a tool for accomplishing communicative goals rather than as a set of grammar rules and vocabulary items. Thus, learners are encouraged to learn by using the language to accomplish tasks rather than simply practice isolated language components.

For this reason, semi-spontaneous speech data were elicited using the Map Task (\citealt{AndersonEtAl1991}; see \citealt{GriceSavino2003} for set up, map layout and instructions), which matches the goal-oriented cooperation task mentioned in the CEFR, i.e. (\citealt{PiccardoEtAl2018}: 88). 

In this task, participants are provided with two maps (\figref{fig:3.1} for Italian and \figref{fig:3.2} for German); one speaker receives a map with a route drawn across landmarks – the instruction giver – and has to describe the route to the other participant – the instruction follower – whose map only features landmarks. The goal is to co-operate so that the instruction follower can reproduce the route on their map thanks to the instructions given by the partner. Some landmarks are different across maps, but participants only discover this during the task, which creates unexpected problem-solving situations. 

To carry out the task, participants sat opposite each other and eye-contact was prevented using an opaque dividing panel, in order to maximise the use of the verbal channel for signalling turn-taking and providing feedback (discussed in \chapref{chap:4}).

\begin{figure}
\includegraphics[width=\textwidth]{sbranna-img047.png}
\caption{Italian version of Map Task}
\label{fig:3.1}
\end{figure}

\begin{figure}
\includegraphics[width=\textwidth]{sbranna-img048.png}
\caption{German version of Map Task}
\label{fig:3.2}
\end{figure}

This task was chosen for two reasons. First, it can be performed at every proficiency level, since learners should address the topic of grammar and vocabulary knowledge related to road indications at a beginner level according to the CEFR. In addition, it presents a fair degree of openness thanks to the unforeseen unmatched landmarks, which increase the degree of spontaneity in the interaction.

Native dyads performed the task once. Learners repeated it in both their native and second language and kept the same role (either instruction giver, or follower) across the two languages to prevent cross-language differences from being attributable to factors regarding their role in the task.

\subsection{Metrics}
\label{sec:3.2.3}
As a proxy for interactional competence, I explored the degree of fluency of the interaction co-created by participants through the turn-taking system. The interactional flow was operationalised by quantifying the percentage of time of five classes of conversational activities: 1) and 2) how long each of the two speakers (giver and follower) takes the floor, 3) how long participants’ turns overlap, 4) the amount of backchannels not initiating a turn and 5) the amount of total silence in the conversation. I distinguish backchannels which are not turn-initial from overlap since these tend to overlap with the interlocutor’s speech, but do not represent a genuine overlap between turns (i.e. they do not represent the intention to take the floor and, instead, fulfil a function supporting the interlocutor’s speech). Nonetheless, I will not focus on backchannels in this study, since the following separate study in \chapref{chap:4} is entirely dedicated to a comprehensive analysis of this conversational phenomenon. These metrics are extracted by a Praat script sampling the classes of conversational activities at a regular time interval of 0.1 seconds across the whole dialogue duration, and thus, dialogue duration is quantified as the total amount of time samples extracted using the Praat script.

\subsection{Procedure for tools generation}
\label{sec:3.2.4}
The annotation and extraction procedure is composed of three steps. After having extracted each of the two channels from the stereo recordings, a first step consisted of the automatic labelling of interpausal units in Praat using the function of silent interval detection with a minimum duration of 200 ms (\citealt{GleitmanEtAl2007}; \citealt{GriffinBock2000}; \citealt{LevinsonTorreira2015}; \citealt{SchnurEtAl2006}; \citealt{WesselingVanSon2005}). Secondly, boundaries were manually checked and corrected to make sure that interpausal units were correctly identified, since some voiceless consonants were automatically labelled as silence. Finally, the annotation text files (Praat TextGrid files) related to the two audio channels of each dialogue were used as input files for a Praat script \citep{CangemiEtAl2023}, which generated a figure depicting each speaker’s contribution to the interaction as it develops over time (Figures \ref{fig:3.3}, \ref{fig:3.4}, \ref{fig:3.5}, \ref{fig:3.6}).

In this conversation chart, each horizontal bar corresponds to an interpausal unit uttered by one of the two speakers involved in the task. Speakers are colour-coded according to their role in the dialogue: red for the instruction giver and blue for the instruction follower. Whenever the two speakers overlap, the differently coloured bars are on top of each other. Backchannels are depicted in green, to distinguish them from actual turn overlap. Time unfolds from top to bottom – minutes –, and from left to right – seconds – so that the interaction can be followed as on a written page in a left-to-right writing system.

Figures \ref{fig:3.3} and \ref{fig:3.4} display a low-proficiency dyad performing the task in their L1 and L2, respectively. The first striking difference is the total length; in the L2, these speakers need roughly 150\% of the time they need in L1 to conclude the task. Moreover, differently from the smooth L1 pattern, the flow of interaction in L2 appears much more fragmented, with shorter turns and more frequent and longer pauses, especially during the turn of an individual speaker. This observation is in line with the systemic perspective of fluency mentioned in the background \citep{Kormos2006}, according to which utterance fluency measures mirror learners’ cognitive fluency. Indeed, in the case of this dyad with a low proficiency of German (A1 level, beginner), it is not surprising to find lengthened within-speaker pauses in the L2 as compared to their interactional behaviour in the L1, reflecting greater cognitive effort.

 

\begin{figure}
\includegraphics[width=\textwidth]{Fig 3.3.pdf}
\caption{Visualisation of the interactional flow in L1 Italian (dyad with low L2 proficiency). Color code identifies speakers according to their role in the Map Task: red – instruction giver, blue – instruction follower. Green bars mark backchannels.}
\label{fig:3.3}
\end{figure}


\begin{figure}
\includegraphics[width=\textwidth]{Fig 3.4.pdf}
\caption{Visualisation of the interactional flow in L2 German (dyad with low L2 proficiency). Color code identifies speakers according to their role in the Map Task: red – instruction giver, blue – instruction follower. Green bars mark backchannels.}
\label{fig:3.4}
\end{figure}

Figures \ref{fig:3.5} and \ref{fig:3.6} show the interactional patterns of a dyad with high proficiency of German (C1 level, advanced). In this case, it is difficult to identify at first sight which dialogue was carried out in the foreign language since the two interactional patterns look very similar. It can still be noticed that turns are slightly more fragmented in the L2, especially for the instruction follower, and that two highly proficient speakers also need a little more time to complete the task in their L2 compared to their L1. Yet, the latter difference is extremely slight across languages and may be due to other factors generating variability in the total duration of the interaction.

\begin{figure}
\includegraphics[width=\textwidth]{Fig 3.5.pdf}
\caption{Visualisation of the interactional flow in L1 Italian (dyad with high L2 proficiency). Color code identifies speakers according to their role in the Map Task: red – instruction giver, blue – instruction follower. Green bars mark backchannels.}
\label{fig:3.5}
\end{figure}


\begin{figure}
\includegraphics[width=\textwidth]{Fig 3.6.pdf}
\caption{Visualisation of the interactional flow in L2 German (dyad with high L2 proficiency). Color code identifies speakers according to their role in the Map Task: red – instruction giver, blue – instruction follower. Green bars mark backchannels.}
\label{fig:3.6}
\end{figure}

This visualisation tool is particularly helpful for observing single interactions with a high degree of detail. However, larger amounts of data require a synthetic representation that is easier to interpret. Therefore, in addition to this figure, the Praat script derives a table used to generate pie plots (example in \figref{fig:3.7}) in R (\citealt{R_core_team2013}) from the extracted data. The five sections of the pie plots use the same colour-coding as the conversation charts to show the percentages of speech uttered by each speaker. The radius of the circle represents the total duration of the interaction: the bigger the pie, the longer the interaction. While the conversation chart is useful for observing the time-aligned development of the interaction, this pie plot summarises and quantifies the partition of the interaction into the five classes of conversational activities and its total duration. Applied to L2 data, this plot is helpful for understanding to which extent high-proficiency L2 interactions resemble L1 conversational patterns more than low-proficiency L2 interactions in terms of changes in the proportions of conversational activities across L1 and L2. Finally, the extracted metrics can be used for hypothesis testing as discussed in the following paragraph.
 

\begin{figure}
\includegraphics[width=.8\textwidth]{Figure_3_7_Example pie plot-cropped.pdf}
\caption{Example of pie plot summarising the five classes of conversational activities. The radius corresponds to dialogue duration.}
\label{fig:3.7}
\end{figure}

\subsection{Bayesian analysis}
\label{sec:3.2.5}
I statistically tested whether changes in the proportions of classes of conversational activities can predict learner proficiency. According to our expectation, with increasing proficiency, learners should approach their own native conversational patterns when speaking in their L2.

Bayesian hierarchical linear models were fitted using the Stan modelling language \citep{CarpenterEtAl2017} and the package \textit{brms} \citep{Bürkner2016}. For each language group, the differences in speech time, silence, overlap and dialogue duration were tested as a function of the factor \textsc{proficiency} (reference level is \textsc{italian l1}).

For the classes of conversational activities (i.e. speech time, silence, overlap), proportion values were taken into account. Therefore, a zero-inflated beta distribution was used and priors for the intercept and the regression coefficient were defined based on data exploration. For speech time of giver, the intercept was set at µ = 0, δ = 0.3 and the regression coefficient at µ = 0, δ = 0.55; for speech time of follower, the intercept was set at µ = 0, δ = 0.3 and the regression coefficient at µ = 0, δ = 0.1; for silence, the intercept was set at µ = 0, δ = 0.4 and the regression coefficient at µ = 0, δ = 0.2; for overlap, the intercept was set at µ = 0, δ = 0.06 and the regression coefficient at µ = 0, δ = 0.035. In all models, I used a beta distribution with α = 1 and β = 1 for the alpha parameter (i.e. the probability of an observation being 0 or 1), a beta distribution with α = 1 and β = 1 for the gamma parameter (i.e. if the probability an observation is 0 or 1, the probability being 1), a gamma distribution with k = .01, and θ = .01 for the phi (precision) parameter. The default settings of the brms package were retained for all other parameters. For total duration of dialogue, the total amount of time samples extracted by Praat were considered. Therefore, a lognormal distribution was used and priors for the intercept and the regression coefficient were defined based on data exploration. The intercept was set at µ = 0, δ = 5000 and the regression coefficient at µ = 0, δ = 10000. The default settings of the brms package were retained for all other parameters. Three sampling chains for 4000 iterations with a warm-up period of 3000 iterations were run for all models. There was no indication of convergence issues (no divergent transitions after warm-up; all Rhat = 1.0).

The expected values under the posterior distribution and their 95\% credible intervals (CIs) are reported for all relevant contrasts (δ), i.e. the range within which an effect is expected to fall with a probability of 95\%. For the difference between each contrast, the posterior probability that a difference is bigger than zero (δ > 0) is also reported to ensure comparability with conventional null-hypothesis significance testing. In particular, it is assumed that there is (compelling) evidence for a hypothesis that states δ > 0 if zero is (by a reasonably clear margin) not included in the 95\% CI of δ and the posterior P(δ > 0) is close to one (cf. \citealt{FrankeRoettger2019}).

All models, results and posteriors can be inspected in the accompanying RMarkdown file in the OSF repository (\url{https://osf.io/9ca6m/}).

\section{Quantitative analysis}
\label{sec:3.3}
Data visualisation (\sectref{sec:3.2.4}) using conversational charts has suggested that a higher proficiency level in L2 enables a degree of smoothness in managing the interactional flow that is closer to the one learners show in their native language, possibly due to an enhanced automatization of the cognitive processes required to speak a foreign language, while the pie plot summarises the changing of the interactional patterns with increasing command of the L2. 

To test informativeness, four explorative pie plots representing dialogues performed by two dyads with different proficiency in L1 and L2 are displayed in \figref{fig:3.8}. On the left, there is a dyad with low L2 proficiency and on the right, a dyad with high L2 proficiency (same speakers as in the conversation charts, Figures~\ref{fig:3.3}--\ref{fig:3.4} and Figures~\ref{fig:3.5}--\ref{fig:3.6} respectively). The high-proficiency dyad (advanced – C1 level) presents two very similar patterns of interaction across languages. The ratio of time speaking between the giver and the follower remains approximately 3:1 when they repeat the task in the L2. In contrast, the low-proficiency dyad (beginner – A1 level) presents two very different interactional patterns. In the L1, the ratio of time speaking between the giver and the follower is 3:1, whereas in the L2 the ratio changes to 2:1, with the giver speaking less in the L2 than in the L1.\footnote{An overall dominance of the instruction giver is expected due to the nature of the task itself.} Furthermore, more than half of the conversation consists of silence. Commonalities across proficiency levels seem to be relative to a longer total duration of the L2 dialogue, and a reduced amount of overlap in the L2, whereas in both dyads the amount of speech time of the follower remains unchanged across languages.

\begin{figure}
\includegraphics[width=.8\textwidth]{Figure_3_8_Exploratory data-cropped.pdf}
\caption{Pie plots summarising conversational activities for a low-proficiency dyad (IF, on the left) and a high-proficiency dyad (BS, on the right). Plots in the top row display dialogues in the L1 and plots below dialogues in the L2. The radius of each pie corresponds to dialogue duration.}
\label{fig:3.8}
\end{figure}

Statistical testing of these observations confirms that speech time of the giver as well as silence are good predictors of learner proficiency in terms of similarity to their native baseline. For both metrics, beginners are robustly different from their L1 Italian, in contrast to advanced learners. Specifically, beginner instruction givers speak notably less in their L2 than in their L1 (δ = -0.16, CI [-0.29; -0.04], P (δ > 0) = 0.99), and silence in their interactions is remarkably more prevalent than in the L1 (δ = 0.23, CI [0.08; 0.39], P (δ > 0) = 1). In contrast, within the advanced group, these two parameters show no robust differences between L1 and L2 (for giver’s speech time: δ = -0.1, CI [-0.23; 0.04], P (δ > 0) = 0.88; for silence: δ = 0.13, CI [-0.02; 0.28], P (δ > 0) = 0.93). As observed in data exploration, speech time of the follower did not show robust variation across L1 and L2, for any proficiency level. The same holds true for overlap, in contrast with preliminary observations. Finally, the total duration of the dialogue was found to be robustly longer in L2 than in L1, independently of learner proficiency (for beginner learners: δ = 1894.26, CI [920.15; 2979.22], P (δ > 0) = 1; for advanced learners: δ = 1381.85, CI [512.03; 2436.16], P (δ > 0) = 1). Therefore, the three latter conversational metrics did not turn out to be good indicators of the Italian learners’ L2 proficiency levels based on their oral performance.

Data averaged across dyads (\figref{fig:3.9}) visually support these results in showing that silence gradually decreases and speech time of the giver gradually increases across the two proficiency levels as compared to the L1. To examine the whole corpus, \figref{fig:3.10} depicts the pie plots for all learners’ interactions displayed by increasing proficiency in both their L1 and L2.\footnote{All by-dyad pie plots are presented singularly, with the percentages relative to the conversational activities, in the Appendix, for the sake of improved visualisation. L1 Italian and L2 German dyads are displayed in Figures~\ref{fig:A4}--\ref{fig:A23}, while L1 German in Figures~\ref{fig:A24}--\ref{fig:A32}.} It can be noticed that:

\begin{itemize}
\item L2 interactional patterns in the beginner group are highly variable across L1 and L2 (with the exception of dyads CV, AR and CC);
\item L2 interactional patterns in the advanced group start consistently resembling those produced in L1, especially from dyad RS on;
\item Amount of silence is consistently higher in beginner L2 interaction compared to their L1, whereas in the advanced group some dyads present very similar silence values across L1 and L2 (see CE, RC, BS and FF).
\end{itemize}

\begin{figure}
\includegraphics[width=.8\textwidth]{Figure_3_9_Averaged proportions of conversational activities-cropped.pdf}
\caption{Averaged pie plots summarising conversational activities for beginner and advanced learners, and their native Italian baseline. Radii correspond to average dialogue duration.}
\label{fig:3.9}
\end{figure}


\begin{figure}
\includegraphics[width=.8\textwidth]{Figure_3_10_All learner pies.pdf}
\caption{Pie plots summarising conversational activities for all beginner and advanced learners in both their L1 Italian and L2 German. Plots in the upper row display dialogues in L2 and plots in the bottom row dialogues in L1. Pairs of letters identify dyads. The radius of each pie corresponds to dialogue duration.}
\label{fig:3.10}
\end{figure}

\section{Qualitative analysis}
\label{sec:3.4}
For a statistical analysis, larger and homogeneous groups of learners were necessary, thus a regrouping into two main proficiency groups was performed. Nevertheless, changes in interactional patterns might be already visible at the more fine-grained CEFR levels. Thus, after having demonstrated the application of the method to L2 data, I will now deepen the discussion of the results obtained, going beyond a by-group analysis and providing a qualitative analysis of learners' improvement across CEFR levels, with a special focus on dyad-specific behaviour. Additionally, based on the CEFR levels, I will investigate whether the results related to interactional competence align with those from the lexical competence test, specifically examining whether the development of various abilities within overall communicative competence occurs in parallel or not.

\subsection{Procedure}
\label{sec:3.4.1}
To investigate changes in interactional patterns at the more fine-grained CEFR levels, I used a measure that summarises learners’ interactions across their L1 and L2 for each dyad, i.e. the difference between L1 and L2 ratios of time speaking of the giver on the follower. The reason is that interaction performance should not be analysed or evaluated in isolation, i.e. by speaker, as it is a form of co-creation of the discourse by both interlocutors. 

Before performing this calculation, the time proportion of backchannels was reassigned to the relative speaker’s speech time in order to obtain the absolute speech time for each speaker. Then, the difference between L1 and L2 ratios of speaking time between the giver and the follower was calculated, so that each data point presented in the results represents a dyad and not a single speaker. During data exploration, it was noticed that the difference between the L1 and L2 ratios of speech time between the giver and the follower is around 0 for high-proficiency learners and different from 0 for less proficient learners. Therefore, as a result of the difference between these two ratios, points approaching 0 will indicate less difference in the interactional behaviour of learners across languages.

\subsection{Results}
\label{sec:3.4.2}
Results are presented in two versions: the first one including all data points (\figref{fig:3.11}) and the second one excluding an outlier for the sake of improved visualisation (the golden data point in \figref{fig:3.11} is not present in \figref{fig:3.12}).

\begin{figure}
\includegraphics[width=.8\textwidth]{Figure_3_11_Speech time difference across L1 and L2.pdf}
\caption{Difference between L1 and L2 ratios of time speaking of the giver on the follower. The graph includes all data points. The x-axis displays the proficiency level, the y-axis shows the values resulting from the formula. The more points approach 0, the less difference there is between learners’ interactional behaviour in L1 and L2.}
\label{fig:3.11}
\end{figure}

\begin{figure}
\includegraphics[width=.8\textwidth]{Figure_3_12_Speech time difference across L1 and L2 (zoomed in).pdf}
\caption{Difference between L1 and L2 ratios of time speaking of the giver on the follower. The graph does not include one outlier for the sake of a better visualisation. The x-axis displays the proficiency level, the y-axis shows the values resulting from the formula. The more points approach 0, the less difference there is between learners’ interactional behaviour in L1 and L2.}
\label{fig:3.12}
\end{figure}

The graph reveals minor variation among dyads. Two out of the three dyads with A1 and A2 proficiency (considered as “beginner” levels in the CEFR) are not as far away from 0 as some B1 and B1-B2 dyads (considered as “intermediate” levels in the CEFR), showing that the difference in learners’ interactional behaviour in beginners can be less than in intermediate learners. Indeed, while proficiency assessment is mostly based on testing grammatical and lexical resources, these are not the only factors contributing to conversational rhythm. Many other linguistic and extralinguistic factors can play a role in goal-oriented conversation, such as personality, engagement in the task, relationship between speakers, and not to forget the skill to strategically draw on the few resources beginners have to reach the goal of the interaction. These two A-level dyads might be an example of the latter case. However, with few samples for the A-level of proficiency, this result cannot be considered informative about a general trend.

The third A-level dyad, i.e. the golden outlier, shows a particular behaviour: the follower only utters a few sentences towards the end of the dialogue in both the L1 and L2. In such cases, using the L1 as a baseline for learners’ interactional behaviour is particularly beneficial. The fact that the follower does not contribute to the conversation in L2 German would have generally been associated with poor command of the L2, but this speaker behaves exactly the same in L1.

On the other hand, the variability displayed in B1 and mixed B1-B2 levels of proficiency tends to be reduced in the more advanced B2 and C1 levels (C-levels are considered as “advanced” levels in the CEFR), which show a narrower range of values. This observation is in line with the statistical analysis, revealing that learners with a higher proficiency, in particular from a B2 level, produce a highly similar interactional pattern in their L1 and L2. In other words, less cognitive load, time for information retrieval and formulation, and attention required in the L2 due to a higher degree of automatization allow learners to achieve an increased degree of smoothness in terms of interactional flow. Still, for the C1 level, too, there is a limited number of dyads. Therefore, a more conspicuous and homogeneously distributed number of samples across proficiency groups would be required to confidently test the trend observed across the CEFR proficiency levels.

As mentioned in \sectref{sec:1.3.2}, learners also took part in an online test for lexical competence, i.e. the German version of LexTALE (\citealt{LemhöferBroersma2012}). \figref{fig:3.13} explores the relation between lexical competence and the interactional pattern. The range of values on the y-axis is reduced for the sake of a clearer visualisation, cutting out the golden outlier as in the previous figure. The radius of the circles represents the score obtained in the test for vocabulary knowledge by the giver, who is the one leading the task and contributing the most to the conversation. The larger the circle, the higher the lexical score the giver received. \citet[341]{Broersma2012} report a correspondence of A1, A2 and B1 levels to scores below 59, B2 level to scores between 60 and 80, and C1 and C2 levels to scores between 80 and 100. Based on their data, the lexical competence of these learners does not seem to increase in parallel with their overall proficiency level as we have no score above 65. Regarding the relation of the lexical competence score to interactional patterns, the graph does not show a clear trend. In the B1, B1-B2 and B2 groups, both very low and high lexical scores are near to zero, suggesting that lexical competence does not seem to be a factor determining how different the interactional patterns are in L1 and L2. This provides some evidence for the previous statement that in a goal-oriented cooperation task, skills other than merely linguistic ones come into play, and a good strategical competence can compensate for the low level of L2 vocabulary and grammar knowledge.

\begin{figure}
\includegraphics[width=.8\textwidth]{Figure_3_13_Relation between lexical competence and interactional pattern.pdf}
\caption{Relation between lexical competence and interactional pattern. The x-axis displays the proficiency level, the y-axis shows the difference between L1 and L2 ratios of time speaking of the giver on the follower. The radius of the circles represents the score for L2 lexical competence of instruction givers.} 
\label{fig:3.13}
\end{figure}

\section{Cross-linguistic comparison}
\label{sec:3.5}
After examining the turn-taking behaviour of Italian learners, an overview of L1 German conversational management patterns and their comparison to L1 Italian is appropriate. Exploring potential cross-linguistic differences may reveal critical aspects for learners that should be addressed in SLA.

Averaged values for the five classes of conversational activities (\figref{fig:3.14}) show that German L1 dialogues have a much longer duration than those performed by Italian native speakers. Indeed, one common strategy among L1 German dyads was to check the drawn path at the end of the task once again, which shows careful consideration regarding the correct execution of the task. Furthermore, silence is slightly reduced in favour of speech time of the follower, which suggests a more interactive and collaborative strategy than in Italian L1 conversations, despite the predetermined roles.

A closer look at by-dyad data (\figref{fig:3.15})\footnote{All by-dyad plots are presented individually in the Appendix for improved visualisation of percentage values (Figures~\ref{fig:A4}--\ref{fig:A32}).} confirms the large difference in total dialogue duration. Only one German L1 dyad, SI, presents a dialogue duration which falls within the range of Italian L1 duration values. Moreover, there seems to be more consistency across German dyads in the proportions of classes of conversational activities than in Italian dyads, even if this result might be due to the smaller sample. For this same reason, these observations have to be taken as preliminary. They suggest some cross-linguistic differences which might be interpreted as being culture-specific interactional conventions and are worth being investigated further.

\begin{figure}
\includegraphics[width=.8\textwidth]{Figure_3_14_Proportions of conversational activities in L1 by group-cropped.pdf}
\caption{\label{fig:3.14}Averaged pie plots summarising conversational activities for native German and Italian speakers. Radii correspond to average dialogue duration.}
\end{figure}

\begin{figure}
\includegraphics[width=.8\textwidth]{Figure_3_15_It & German dyads.pdf}
\caption{Pie plots summarising conversational activities for all L1 Italian and L1 German dyads. Plots in the upper line display dialogues in L1 German and plots in the bottom line dialogues in L1 Italian. The radius of each pie corresponds to dialogue duration.}
\label{fig:3.15}
\end{figure}

\section{Conclusion}
\label{sec:3.6}
In this study, I have problematised the absence of a standardised instrument for the quantification of interactional competence in L2. To open up new perspectives for L2 assessment, I presented visualisation tools and a quantification method that can extract reliable and testable metrics for interactional aspects of communication, i.e. speaking time of participants, silence, overlap, backchannels and total duration of the dialogue. The informativeness of this method with regard to learners’ L2 proficiency was tested on a corpus of L1 and L2 interactions by conducting data exploration and subsequent hypothesis testing.

Results show that more proficient learners better maintain the natural interactional rhythm they have in their L1 in the L2, which is especially clear in the speech time metrics for instruction giver and the total amount of silence. A further qualitative analysis suggests that improvements might be already visible at the more fine-grained CEFR levels, but this observation requires further testing on a larger scale considering that this corpus only includes a few samples for the A- and C-levels of the proficiency scale, in contrast to the more conspicuous B-level group.

Lexical competence did not seem to influence learners’ interactional behaviour, which suggests that mastering the lexicon does not automatically ensure a higher degree of success in oral interactions. Moreover, lexical scores did not seem to lead to the corresponding levels of general L2 competence. Indeed, the learning process is not linear, and there is no discrete order for L2 knowledge acquisition \citep{Nava2010}, so that different skills can improve at different speeds. Since open interactional tasks test learners’ L2 abilities in a more comprehensive way, an enhancement of these kind of tasks in L2 experimental and testing settings can help obtain a clearer picture of learners’ L2 general proficiency, and possibly shed light on the interplay among different skills.

In addition, a preliminary comparison of L1 German and L1 Italian conversations suggests some likely cross-linguistic/-cultural differences in the total duration of dialogues as well as the speech time of the instruction follower, which seem to suggest a more careful and collaborative approach by L1 German compared to L1 Italian dyads. In order to relate these results to language pedagogy, the differences in interactional conventions across languages/cultures deserve being investigated further on a larger scale.

The method proposed has considerable potential for the analysis of L2 oral interactions. It permits an immediate comparison of learners’ interactional behaviour in the L1 with their performance in the L2, both in a detailed and synthetic way throughout the different stages of the learning process, which can be complemented by statistical testing. Moreover, being performed mostly in an automated way, this is a labour-saving analysis. For these reasons, this method of visualisation and quantification of oral interactions could represent a starting point for quantifying L2 interactional competence based on interactional fluency in a standardised way.

At the same time, there are some limitations that need to be considered. First, this analysis offers only a partial, structural view of the interaction as it is content-free and based on temporal measures only. The missing verbal content is crucial for understanding the underlying reason generating the metrics. As an example, the verbal content is necessary for clarifying the different nature of silence and discourse chunking across the L1 and the L2. Consider silence following hesitative feedback by the interlocutor signaling a lack of understanding. The primary speaker might want to give time to the interlocutor to explain the nature of their hesitation with the unclear content, or take time to reformulate their own message in a different way. Only in the latter case can the amount of silence be interpreted as a measure of cognitive load and related to a less complete mastery of the language. Nevertheless, the implementation of other content-related information is possible by integrating the conversational activities with categories taken from the pragmatic and strategic skills indicated in the CEFR interaction scale (e.g. asking for clarification, compensating, cooperating, monitoring and repair), which would add more layers of information and complexity. Another option would be to integrate content-related information at turn transitions, i.e. turn sequences in which speakers alternate in occupying the floor, to assess the strategies learners use to coordinate the interaction throughout the learning process.

Secondly, there are non-verbal forms of communication (e.g. eye gaze, gesture, posture) which also contribute to the interactional patterns (see, for example,  \citealt{Kosmala2024}), but were not captured by the present data collection method optimised for verbal communication. Thus, the results should be interpreted within the context of the data collection setting. A future integration of non-verbal communicative features into this tool may be possible if learners consent to being video-recorded. The time investment necessary for such an analysis should be also tested in order to evaluate the practicality of its application.

Lastly, I have proposed a by-dyad quantification method which, together with the suggestion of collecting data through spontaneous peer conversations, contrasts with the need to assess each learner’s L2 competence individually. In real-world scenarios, such as language certification settings, learners’ L2 competence is typically assessed on an individual basis to provide a final score. However, in interactions, both interlocutors share responsibility for co-constructing the communicative exchange. Therefore, further development of this tool, which was designed for dyadic exchanges, should account for each speaker’s contribution to the interaction.

Overall, this study shows that it is possible to reliably identify conversational activities related to language proficiency, opening up possibilities for future implementations based on the linguistic resources contributing to these activities. One linguistic feature that plays a significant role in interactional fluency and might enrich these implementations with content-based data is the use of feedback as a measure of active listening, which will be the focus of the next chapter.
