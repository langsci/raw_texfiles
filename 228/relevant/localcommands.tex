%%%%%%%%%%%%%%%%%%%%%%%%%%%%%%%%
%%	Font Styles & Formatting	%%
%%%%%%%%%%%%%%%%%%%%%%%%%%%%%%%%

\newcommand{\ve}{\textit}			%vernacular	/ target language text
\newcommand{\tbr}{\textbf}		%Bold red text
\newcommand{\tcb}{\textcolor{black}}	%black colour
\renewcommand{\it}{\textit}	%italics
\newcommand{\tsc}{\textsc}	%small caps
\newcommand{\tbf}{\textbf}	%boldface

\newcommand{\sub}{\textsubscript}				%subscript
\newcommand{\su}{\textsuperscript}			%superscript
\renewcommand{\sf}[1]{\textsf{\textcolor{black}{#1}}}							%sans serif

%%%%%%%%%%%%%%%%%%%%%%%%%%%%%%%%
%%	Symbols and Characters		%%
%%%%%%%%%%%%%%%%%%%%%%%%%%%%%%%%

\newcommand{\tl}{\char`~}		%middle tilde ~
\newcommand{\x}{×} 					%times symbol ×
\newcommand{\rt}{√} 				%root symbol
\newcommand{\Q}{\textquotesingle}	%straight apostrophe
\newcommand{\<}{〈} 			%left angle (for infixes)
\renewcommand{\>}{〉} 		%right angle (for infixes)
\newcommand{\ra}{→} 		%->
\newcommand{\da}{↓} 		%downwards arrow
\newcommand{\la}{←} 		%<-
\newcommand{\lra}{↔} 		%<->
\newcommand{\0}{∅} 			%zero symbol
%\newcommand{\hand}{\raisebox{-0.3ex}{\Large ☞}} %OT pointing hand
\newcommand{\hand}{☞} 	%OT pointing hand
\renewcommand{\ast}{⁎} 	%centered asterisk
\newcommand{\gap}{\textunderscore} 	%underscore
\newcommand{\A}{α}
\newcommand{\sA}{\sub{α}}
\newcommand{\sB}{\sub{β}}
\newcommand{\br}[1]{$\overline{\textrm{#1}}$}	%Overbar (for N' etc.)

\newcommand{\vp}{\vphantom}			%space equal to height of argument
\newcommand{\hp}{\hphantom}			%space equal to width of argument
\newcommand{\lh}[1]{\leavevmode{\hp{#1}}} %space equal to width of argument for free translationa

\newcommand{\sarc}[1]{{#1}̯} %non-syllabic, combining inverted breve below i.e. u̯
\renewcommand{\j}{ʤ}				%dezh digraph
\newcommand{\tS}{ʧ}					%tesh digraph
\newcommand{\ny}{ɲ}					%palatal nasal -- enya = latin small letter n with left hook
%\newcommand{\B}{\addfontfeature{StylisticSet=20} β} %voiced bilabial fricative
\newcommand{\B}{β}	%voiced bilabial fricative
\newcommand{\tbrtb}[2]{\it{\tbf{#1}͡\tbf{#2}}} %tiebar over two boldface, italic, maroon letters -- for some reason the tiebar over a boldface character in Linux Libertine displays as  breve over the first character

%%%%%%%%%%%%%%%%%%%%%%%%%%%%%%%%%%%%%%%%%%%%%%%%%%%%
%% Tables %% Tables %% Tables %% Tables %% Tables %%
%%%%%%%%%%%%%%%%%%%%%%%%%%%%%%%%%%%%%%%%%%%%%%%%%%%%

\newcommand{\mc}{\multicolumn}									%multicolumn
\newcommand{\cgr}{\cellcolor{gray!14}}					%grey cellcolor
\newcommand{\wg}{\rowcolors{1}{white}{gray!14}} %white/grey/white alternating rowcolors
\newcommand{\gw}{\rowcolors{1}{gray!14}{white}} %grey/white/grey alternating rowcolors
\newcommand{\rcl}{\rowcolor{gray!14}}						%single grey row
\newcolumntype{g}{>{\columncolor{gray!14}}l}		%grey column
\newcommand{\stl}[1]{\setlength{\tabcolsep}{#1}}	%reduce column width in tables

%%%%%%%%%%%%%%%%%%%%%%%%%%%%%%%%
%%	Cross					References	%%
%%%%%%%%%%%%%%%%%%%%%%%%%%%%%%%%

\newcommand{\qf}[1]{(\ref{#1})}					%cross reference bracketed examples
\newcommand{\trf}[1]{Table \ref{#1}}		%cross reference Table
\newcommand{\frf}[1]{Figure \ref{#1}}		%cross reference Figure
\newcommand{\srf}[1]{\S\ref{#1}} 				%cross reference section
\newcommand{\prf}[1]{page \pageref{#1}} %cross reference page number

%\newcommand{\eax}[1]{\ea{\textnormal{#1}}}
\newcommand{\txrf}[1]{\hfill{\ttfamily\small {#1}}}			%Amarasi example information
\newcommand{\xrf}[1]{{\ttfamily\small {#1}}}						%Amarasi example information
\newcommand{\brac}[1]{\normalfont\textcolor{black}{[\sub{#1}}}  %square bracket with argument i.e. [NP
\newcommand{\bracr}[1]{\normalfont\textcolor{black}{]\sub{#1}}} %right square bracket with argument i.e. ]NP


\newcommand{\appref}[1]{Appendix \ref{#1}}
\newcommand{\fnref}[1]{Appendix \ref{#1}}
\newcommand{\regel}[1]{#1}
\newcommand{\vernacular}[1]{\emph{#1}}
\newcommand{\gloss}[1]{#1}

\newcounter{exx}
