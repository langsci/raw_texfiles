\title{Metathesis and unmetathesis in Amarasi} 
\BackBody{
This book provides a complete analysis of synchronic CV -> VC metathesis in Amarasi,
a language of western Timor. Metathesis and unmetathesis realise a paradigm of parallel
forms, pairs of which occur to complement each other throughout the language.
Metathesis in Amarasi is superficially associated with a bewildering array of dis-
parate phonological processes including: vowel deletion, consonant deletion, consonant
insertion and multiple kinds of vowel assimilation, any of which can (and do) vary by
lect in their realisation. By proposing that Amarasi has an obligatory CVCVC foot in
which C-slots can be empty, all these phonological processes can be straightforwardly
derived from a single rule of metathesis and two associated phonological rules.
Three kinds of metathesis can be identified in Amarasi: (i) Before vowel initial encl-
itics, roots must undergo metathesis, responding to the need to create a phonological
boundary between a clitic host and enclitic. Such metathesis is phonologically condi-
tioned. (ii) Metathesis occurs within the syntax to signal attributive modification. Such a
metathesised form cannot occur at the end of a phrase and thus requires the presence of
an unmetathesised form to complete it syntactically. (iii) In the discourse an unmetathe-
sised form marks an unresolved event or situation. Such an unmetathesised form cannot
occur in isolation and requires a metathesised form to achieve resolution.
Metathesis in Amarasi is the central linguistic process around which linguistic struc-
tures are organised. Amarasi metatheses also reflect fundamental Timorese notions of
societal and cosmic organisation. Alongside weaving and other performed activities,
metathesis is an important linguistic marker of identity in a region obsessed with similar-
ities and differences between different groups. The complementarity of Amarasi metathe-
sis and unmetathesis within the syntax and within discourse reflects the Timorese divi-
sion of the world into a series of mutually dependent binary and complementary pairs.
As well as being the key which unlocks the structure of the language, metathesis is also
a reflection of the structure of Amarasi society and culture.
}
\dedication{For Chuck and Om Roni who laid the groundwork in analysing Amarasi.}
\typesetter{Owen Edwards}
\proofreader{Brett Reynolds,
Christian Döhler,
Cormac Anderson,
Jeroen van de Weijer,
Laura Arnold,
Ludger Paschen,
Sauvane Agnès,
Sebastian Nordhoff,
Steven Kaye,
Tom Bossuyt
}
\author{Owen Edwards}
\BookDOI{10.5281/zenodo.4548165} %version 1.1.
\renewcommand{\lsISBNdigital}{978-3-96110-222-8}
\renewcommand{\lsISBNhardcover}{978-3-96110-223-5}
\renewcommand{\lsSeries}{sidl}
\renewcommand{\lsSeriesNumber}{29}
\renewcommand{\lsID}{228}
\renewcommand{\lsYear}{2020}


 
 
 
 
  
