\section{Alternate approaches}\label{sec:AltApp}
In the final part of this chapter I consider
the ways alternate approaches would handle the Amarasi data.
There are no existing proposals in the literature which can
adequately handle all of the Amarasi data.

In \srf{sec:ProMor} I consider whether an approach framed within
prosodic morphology \citep{mccpr90,mccpr93} can explain the Amarasi data.
In particular, I consider whether the analysis of \cite{mcc00} for Rotuman metathesis
or that of \cite{he04} for Kwara'ae can be extended to Amarasi.
While such analyses can explain a small amount of the Amarasi data,
they cannot explain it all.
Furthermore, because there is no consistent single prosodic structure present
in Amarasi surface M\=/forms, an analysis framed within prosodic morphology
does not seem appropriate,

In \srf{sec:AltAppPhoMet} I show that Amarasi
metathesis cannot be analysed as phonologically conditioned
as has been proposed for Rotuman \citep{haki98,mcc00},
Kwara'ae \citep{he04}, and Luang \citep{tata15},
as well as Meto \citep{mccko96}.

Finally, in \srf{sec:AffCVMel} I show that it
is typologically implausible to analyse the
Amarasi data as a fundamentally concatenative process
involving affixation of a CV melody to the segmental information of a word.

\subsection{Prosodic morphology}\label{sec:ProMor}
One alternate approach to the data would be to analyse
Amarasi metathesis within the framework of prosodic morphology.
Such an analysis has been proposed by \citet{mcc00} for Rotuman metathesis.
Similarly, \cite{he04} proposes an analysis of synchronic metathesis in
Kwara'ae which is compatible with a prosodic morphological approach.
After a discussion of each of these analyses,
I show that neither can be adapted and/or extended to the Amarasi data.

\subsubsection{Rotuman}\label{sec:ProMorRot}
The forms and functions of metathesis in Rotuman
has been comprehensively summarised in \srf{sec:Rot}.
As discussed in \srf{sec:Rot},
a variety of different processes are used to form the
M\=/form from the U\=/form. These processes include
vowel deletion, metathesis, umlaut, and diphthongisation.
Which process applies depends on the quality of the penultimate
and final vowels of the U\=/form, as well as
whether the U\=/form is VCV{\#} final or VV{\#} final.
There are also a certain number of word shapes with no distinction between the two forms.
Examples of each of these processes extracted from \cite{ch40}
are given in Table \ref{tab:RotPhaAlt} in standard IPA transcription.

\begin{table}[h]
	\caption{Rotuman U-form and M-form alternations}\label{tab:RotPhaAlt}
	\centering
		\begin{tabular}{llll}\lsptoprule
				Process 			& U\=/form	 		& M\=/form			& gloss \\ \midrule
				deletion			&	\it{haŋa}		& \it{haŋ}		& `feed'	\\
											&	\it{tokiri}	& \it{tokir}	& `roll'	\\
				metathesis		&	\it{pure}		& \it{puer} 	& `rule, decide'	\\
											&	\it{hoʔa}		& \it{hoaʔ} 	& `take'	\\
				umlaut				&	\it{mori}		& \it{mør} 		& `orange (fruit)'	\\
											&	\it{ʔuli}		& \it{ʔyl} 		& `skin'	\\
		diphthongisation	&	\it{pupui}	& \it{pupŭi}	& `floor'	\\
											&	\it{foʔou}	& \it{foʔŏu}	& `new'	\\
					no change		&	\it{rii}		& \it{rii} 		& `house'	\\
											&	\it{ree}		& \it{ree}		& `do'	\\
											\lspbottomrule
		\end{tabular}
\end{table}

An analysis of the Rotuman data within the framework of
prosodic morphology and Optimality Theory \citep{prsm93}
is presented in \cite{mcc00}.
\citet[159]{mcc00} bases his analysis on the observation that
{``The incomplete phase [M\=/form] is identical to the complete phase [U\=/form],
except that the \emph{final foot} of the complete phase is realized as a
\emph{monosyllabic foot} in the incomplete phase.''}
Regarding words which form the M\=/form
via metathesis, such as \it{hoʔa} {\ra} \it{hoaʔ} `take',
\cite{mcc00} argues that the M\=/form consists of a single syllable,
as is consistent with more recent descriptions of Rotuman including \cite{be87} and \cite{va02}.\footnote{\label{fn:Problem1}
		It not certain that the M\=/form was
		monosyllabic when \citeauthor{ch40} conducted his fieldwork
		\citet[86]{ch40} states that
		``the stress seems to be levelled out, so to speak,
		in the inc[omplete] phase [M\=/form]. Thus: \it{fo}ra becomes \it{foar},
		which is pronounced almost, though perhaps not quite, as one syllable,
		the stress being evenly distributed {\ldots}''
		While \cite{mcc00} takes this statement to suggest the M\=/form
		is a monosyllabic word, it can also be taken
		to mean that the M\=/form is shorter in phonetic length
		even though it remains two syllables.}

Under \citeauthor{mcc00}'s analysis Rotuman is a weight sensitive language.
Because a (metathesised) monosyllable such as \it{hoaʔ} `take' is
consonant final, it has two morae and bears stress
as expected for a heavy syllable in a weight sensitive language.

\cite{mcc00} also draws upon the observation
by \cite{haki98} that in most cases the use of the two stems in Rotuman
is conditioned by the number of syllables of a following suffix or enclitic.
Suffixes and enclitics consisting of two or more syllables trigger the U\=/form,
while monosyllabic or non-syllabic suffixes and enclitics occur
with a stem in the M\=/form.\footnote{\label{fn:Problem2}
		As noted by \cite[162]{mcc00}, \citet{haki98} identify
		two zero suffixes which occur with the U\=/form
		and one monosyllabic suffix which occurs with the M\=/form.
		\cite{mcc00} does not offer any explanation for their
		aberrant behaviour beyond mentioning that \citet{haki98} analysed such
		forms as taking zero suffixes which bear moraic weight.}

\citet[156]{mcc00} then draws upon the principle of Foot Binarity,
whereby feet are required to consist of a minimum of either two syllables or two morae.
\citeauthor{mcc00} proposes that polysyllabic suffixes and enclitics
are prosodically external to the stem, as they are eligible to form independent feet.
Non-syllabic and monosyllabic suffixes, on the other hand cannot form feet and are thus bound to the stem.

\citet[163]{mcc00} represents forms like \emph{hoaʔ-kia}
`take-\tsc{transitive}' with the structure in \qf{hoaq-kia}.
The stem of this form is in the M\=/form
(a heavy syllable consisting of two morae)
with a polysyllabic suffix/enclitic attached.
\citeauthor{mcc00} represents forms like \emph{hoʔa-ŋe} `take-away'
with the structure in \qf{hoqa-nge}.
The stem of this form is in the U\=/form (a disyllable) with a monosyllabic suffix.
In both these diagrams `PrWd' stands for `prosodic word'.\footnote{
		\cite{mcc00} uses different examples to illustrate the prosodic
		structures of the U\=/form and M\=/form of words.
		I have selected these examples in order to illustrate
		clearly the difference between metathesised and unmetathesised forms.}

\begin{multicols}{2}
	\begin{exe}
			\ex{M\=/form and long suffix:}\label{hoaq-kia}
			\sn{\it{hoʔa} `take' + \it{-kia} `transitive'}
			\sna{\xy
				<0em,3cm>*\as{PrWd}="s1",<2.5em,3cm>*\as{PrWd}="s2",
				<0em,2cm>*\as{Ft}="c1",<2.5em,2cm>*\as{Ft}="c2",
				<-0.35em,1cm>*\as{}="g1",<0.35em,1cm>*\as{}="g2",<2.15em,1cm>*\as{}="g3",<2.85em,1cm>*\as{}="g4",
				<0em,1cm>*\as{hoaʔ}="p1",<2.5em,1cm>*\as{-kia}="p2",
				<0em,0cm>*\as{stem}="f1",<2.5em,0cm>*\as{sfx}="f2",
				"f1"+U;"p1"+D**\dir{-};"f2"+U;"p2"+D**\dir{-};
				"g1"+U;"c1"+D**\dir{-};"g2"+U;"c1"+D**\dir{-};"g3"+U;"c2"+D**\dir{-};"g4"+U;"c2"+D**\dir{-};
				"g1"+U;"g2"+U**\dir{-};"g3"+U;"g4"+U**\dir{-};
				"c1"+U;"s1"+D**\dir{-};"c2"+U;"s2"+D**\dir{-};
			\endxy}
			
			\ex{U\=/form and short suffix:}\label{hoqa-nge}
			\sn{\it{hoʔa} `take' + \it{-ŋe} `away'}
			\sna{\xy
				<1em,3cm>*\as{PrWd}="s1",
				<2em,2cm>*\as{Ft}="c2",
				<1.65em,1cm>*\as{}="g3",<2.35em,1cm>*\as{}="g4",
				<0em,1cm>*\as{hoʔa}="p1",<2em,1cm>*\as{-ŋe}="p2",
				<0em,0cm>*\as{stem}="f1",<2em,0cm>*\as{sfx}="f2",
				"f1"+U;"p1"+D**\dir{-};"f2"+U;"p2"+D**\dir{-};
				"g3"+U;"c2"+D**\dir{-};"g4"+U;"c2"+D**\dir{-};"g3"+U;"g4"+U**\dir{-};
				"p1"+U;"s1"+D**\dir{-};
				"c2"+U;"s1"+D**\dir{-};
			\endxy}
	\end{exe}
\end{multicols}

The constraint \tsc{Align-Head}-σ,
which requires stressed syllables to be word final,
is  crucial in \citeauthor{mcc00}'s analysis.
This constraint regulates prosodic words.
When a stem occurs with a long suffix, each prosodic word occurs in the M\=/form.
This means that both the stem and the
suffix/enclitic in \qf{hoaq-kia} would occur in the M\=/form.
When a stem occurs with a short suffix the whole prosodic word,
--- the stem combined with the suffix/enclitic ---
would occur in the M\=/form.

For Rotuman, the constraint \tsc{Align-Head}-σ
and the constraint \tsc{max} (prohibiting deletion)
are ranked more highly than the constraint \tsc{Linearity} (prohibiting metathesis).
Each of these three constraints is given in \qf{ex:ConRan} below.

\begin{exe}
	\ex{Constraint Ranking in \citet{mcc00}:}\label{ex:ConRan}
		\begin{xlist}
			\ex{\tsc{Align-Head}-σ: Align(H′(PrWd), R, PrWd, R)
			(The main-stressed syllable is final in every prosodic word)}\label{ex:Ali}
			\ex{\tsc{Max}: Every element of S\sub{1} has a correspondent in S\sub{2}
			(No deletion)}\label{ex:Max}
			\ex{\tsc{Linearity}: S\sub{1} is consistent with the precedence structure of S\sub{2} and vice versa
			(No metathesis)}\label{ex:Lin}
		\end{xlist}
\end{exe}

Metathesis results as it is ``the most faithful constraint mapping of a
/{\ldots}VCV/ input that still satisfies \tsc{Align-Head}-σ.''
\citet[174]{mcc00} gives an equivalent of the optimality tableau in \qf{ex:RotOT} below.

\begin{exe}
\ex{\rule{0pt}{0pt}{} \\[-3ex]
	\begin{tabular}{|rrl||c|c|c|} \hline
		\mc{3}{|c||}{Input:~/hoʔ\sub{1}a\sub{2}/} & \tsc{Align-Head}-σ & \tsc{max} & \tsc{Linearity} \\[0.5ex]
		\hline \hline a. & \hand & hoa\sub{2}ʔ\sub{1} & & & {\cgr}* \\
		\hline b. & & hoʔ\sub{1} & & *! & {\cgr} \\
		\hline c. & & hoʔ\sub{1}a\sub{2} & *! & {\cgr} & {\cgr} \\
		\hline \end{tabular}}\label{ex:RotOT}
\end{exe}

Under \citeauthor{mcc00}'s analysis Rotuman metathesis
(along with other processes which form the M\=/form)
occurs in order to create final stressed syllables
within the domain of the prosodic word.
I refer the reader to \cite{mcc00} for a full analysis
of the different phonological processes in Rotuman
and the ways in which these are handled.

\subsubsection{Kwara'ae}\label{sec:ProMorKwa}
The forms and functions of metathesis in Kwara'ae are summarised in \srf{sec:Kwa}.
In Kwara'ae the metathesised form of words is the form of words used in everyday normal speech
while the unmetathesised form is used in traditional songs,
for clarification, and when calling/yelling out.
Some examples of metathesis in Kwara'ae
are given in \qf{ex2:KwVCV->VVC} below.

\begin{exe}
	\ex{Metathesis in Kwara'ae \hfill\citep{he04}}\label{ex2:KwVCV->VVC}
	\sn{\begin{tabular}{rcll}
\rcl U\=/form				 &		& M\=/form				&  \\
		\it{ˈlo.\tbr{ʔi}} &\ra& \it{ˈlo\tbr{i̯ʔ}}	& `snake' \\
\rcl\it{ˈbu.\tbr{ri}} &\ra& \it{ˈbu̯\tbr{ir}}	& `behind' \\
		\it{ˈbo.\tbr{re}} &\ra& \it{ˈbo̯\tbr{er}}	& `although' \\
\rcl\it{ˈki.\tbr{ni}} &\ra& \it{ˈk\tbr{iˑn}} 	& `woman' \\
		\it{ˈde.\tbr{ŋe}} &\ra& \it{ˈd\tbr{ɛˑŋ}} 	& `shrimp' \\
\rcl\it{ˈke.\tbr{ta}.ˌla.\tbr{ku}} 	&\ra& \it{ˈke̯\tbr{at}.ˌla\tbr{u̯k}} & `my height' \\
		\it{da.ˈ\tbr{ro}.ʔa.ˌni.\tbr{da}}&\ra& \it{ˈda\tbr{o̯r}.ʔa.ˌni̯\tbr{ɛd}} & `to share them' \\
	\end{tabular}}
\end{exe}

According to the description in \citet{he04},
each Kwara'ae unmetathesised form
contains one or more stressed syllables with a single vowel
which are followed by an unstressed vowel.
Each metathesised form contains stressed syllables,
each of which is a heavy syllable containing either
a vowel and glide or a long vowel.

Based on this observation \cite{he04} analyses metathesis in Kwara'ae
as being conditioned by the placement of stress.
His analysis is framed under Optimality Theory
and is based on ranking the Stress to Weight Principle constraint (requiring
stressed syllables to be heavy) more highly than the constraint of Linearity,
which requires segments to occur in their underlying order.
Given an underlying form such as /salo/ `sky',
the metathesised form with a heavy syllable consisting of a vowel and glide
is selected rather than the unmetathesised form with two light syllables,
as shown in the optimality tableau in \qf{KwMetOT} below

\begin{exe}
		\ex{Metathesis in Kwara'ae \hfill \cite[53]{he04}}\label{KwMetOT}
	\sn{\begin{tabular}{|rrl||c|c|} \hline
		\mc{3}{|c||}{Input:~/salo/} & SWP & \tsc{Linearity} \\[0.5ex]
		\hline \hline a. & & ˈsao̯l & & {\cgr}* \\
		\hline b. & & ˈsa.lo & *! & {\cgr} \\
		\hline \end{tabular}
	}
\end{exe}

\subsubsection{Amarasi}
Although the analysis of \cite{mcc00} for Rotuman
and the analysis of \cite{he04} for Kwara'ae
are different in several details,
in each case metathesis is analysed as occurring in
order to create a monosyllabic form which bears stress.
Whatever the merits these analyses may have for Rotuman or Kwara'ae,
they cannot be extended to describe the Amarasi data.
This is because in most cases the stress
and number of syllables of a U\=/form and M\=/form are identical.

This is true of forms with different penultimate and final vowels
which undergo metathesis; such as \ve{hitu} [ˈhi.t̪ʊ] {\ra} \ve{hiut} [ˈhi.ʊt̪] `seven'.
It is true of words which undergo metathesis and deletion of a final consonant;
\ve{muʔit} [ˈmʊ.ʔit̪] {\ra} \ve{muiʔ} [ˈmʊ.iʔ] `animal'.
It is also true of words which undergo deletion of a final consonant
such as \ve{kuan} [ˈkʊ.ɐn] {\ra} \ve{kua} [ˈkʊ.ɐ] `village'.\footnote{
		It is also worth noting that while words which end in VV{\#},
		such as \ve{ai} [ʔa.i] `fire',
		do not have distinct M\=/forms and U\=/forms in Amarasi,
		the prosodic structure of such a word is identical
		when it occurs in either a U\=/form or M\=/form environment.}

The only cases in which such an analysis could be successfully
applied are those in which the M\=/form is, arguably, a reduced form.
This includes words in which the penultimate and final vowel
are identical, such as \ve{fini} [ˈfɪ.ni] {\ra} \ve{fiin} [ˈfiːn] `seed',
or those in which assimilation of /a/ occurs, such as \ve{nima} [ˈni.mɐ] {\ra} \ve{niim} [ˈniːm] `five'.
It also perhaps includes words with an initial phonetic diphthong,
such as \ve{nautus} [ˈnəw.t̪ʊs] {\ra} \ve{naut} [ˈnə.ʊt̪] `beetle'.

In cases in which the M\=/form
has a final sequence of two identical vowels,
such as \ve{fini} [ˈfɪ.ni] {\ra} \ve{fiin} [ˈfiːn] `seed',
there is a reduction in the number of phonetic syllables,
with a resultant change in stress from a light syllable to a heavy syllable.
In such instances alone an analysis along the lines of \cite{mcc00} or \cite{he04}
in which metathesis occurs to create a heavy stressed syllable could be proposed.
However, given the evidence for treating phonetically long vowels
as a sequence of two identical vowels (\srf{sec:DouVow}),
and the fact that Amarasi is otherwise not a weight sensitive language (\srf{sec:Str}),
this analysis is not consistent with other facts of the language.

Not only does such an analysis contradict other
facts of Amarasi, but it also cannot account for all the data,
as the M\=/form of words such as \ve{hitu} [ˈhit̪ʊ] {\ra} \ve{hiut} [ˈhi.ʊt̪] `seven'
simply do not contain a stressed heavy syllable.
Therefore, neither of the analyses proposed by \citet{mcc00} or \citet{he04}
can explain all the Amarasi data.
Furthermore, it is not at all clear that \emph{any} analysis
framed within prosodic morphology could account for the Amarasi data.

One approach, broadly in line with the 
notion that consonant-vowel metathesis is a reduction strategy,
would be to propose that the M\=/forms occur in order
to create a phonetically shorter form.
An M\=/form with a vowel sequence such as \ve{hiut} [ˈhi.ʊt̪] `seven'
is usually, though not always, phonetically shorter than the U\=/form \ve{hitu} [ˈhit̪ʊ]
in which the two vowels are separated by a consonant.\footnote{
		That the total length of a vowel sequence is usually shorter
		than the combined length of two vowels separated by a consonant
		is confirmed by an instrumental phonetic study.
		I marked vowels in Praat in three recorded texts from a single speaker
		and extracted their lengths with a script.
		(Words with a distinctive pause intonation were discarded.)
		This resulted in 255 measurements of both the vowels in forms such as \ve{hitu} `seven'
		(that is, the measurements of 510 vowels with the length
		of the penultimate and final vowels summed)
		and 243 lengths of vowel sequences in M\=/forms such as \ve{hiut} `seven'.
		The average length of two vowels separated by a consonant was 0.17 seconds
		and the average length of a vowel sequence was 0.13 seconds.
		A two tailed t-test showed that this difference was statistically significant (p > 0.001).
		The lengths of 385 vowel sequences in U\=/forms such as \ve{kuan} `village'
		were also extracted and had an average duration of 0.12 seconds.
		A two tailed t-test showed that the difference between vowel sequences
		in M\=/forms and U\=/forms is probably not statistically significant (p = 0.11).}
Apart from the fact that the phonetic length of a word in Amarasi
is primarily determined by speech speed, sentence stress, and pragmatics
rather than the phonotactic shape of a word,
this approach would leave unexplained forms in which the U\=/form
and M\=/form both have a vowel sequence, such as \ve{kuan} [ˈkʊ.ɐn] {\ra} \ve{kua} [ˈkʊ.ɐ] `village'.

Another possibility would be to propose M\=/forms
occur in order to create consonant-final forms.
Such a proposal does not explain why VVC{\#} U\=/forms
such as  \ve{kuan} [ˈkʊ.ɐn] {\ra} \ve{kua} [ˈkʊ.ɐ] `village' delete their final consonant
to form an M\=/form, nor why already consonant-final U\=/forms
such as  [ˈmʊ.ʔit̪] {\ra} \ve{muiʔ} [ˈmʊ.iʔ] `animal'
undergo metathesis in the M\=/form.

The diversity in the surface prosodic structures of M\=/forms in Amarasi
confounds an analysis framed within prosodic morphology.
While in Rotuman there is a comparable diversity in forms,
nearly all instances of the M\=/form
have been analysed as bi-moraic monosyllables,
thus an analysis framed within prosodic morphology is successful.

A prosodic morphology approach to the Amarasi data encounters
serious challenges which cannot obviously be overcome.
Instead, by adopting a process based model of morphology
and an obligatory CVCVC foot structure for U\=/forms,
the different processes in the formation of the Amarasi M\=/form
can be explained with a single process of metathesis
and a single morphemically conditioned rule of /a/ assimilation.
This analysis accounts for Amarasi M\=/forms
in a simple, consistent, unified way.

Finally, regarding Optimality Theory, which is
utilised in the analyses of \cite{mcc00} and \cite{he04},
the high level of opacity/ambiguity in the derivation of M\=/forms
--- including at least one derived environment effect (\srf{sec:AssOfA}) ---
indicates that standard Optimality Theory would not fare particularly well in Amarasi.
This is one of the reasons I have not employed it in this book.

\subsection{Phonologically conditioned metathesis}\label{sec:AltAppPhoMet}
Another possible approach to the Amarasi data would
be to analyse metathesis as phonologically conditioned.
This is, in fact, part of the analysis of Rotuman
given by \citet[168]{mcc00} who explicitly rejects
the idea that there is a metathesis morpheme and states:
``[\ldots] earlier accounts of the phase [U\=/form/M\=/form] difference are inadequate empirically,
since it does not make sense to talk of a phase morpheme or template.''\footnote{
		That \cite{mcc00} explicitly rejects the notion of a ``phase morpheme'' is surprising
		given that \cite{ch40} gives many examples in which the use of the
		U\=/form or M\=/form is the only marker of a semantic difference,
		as summarised in \srf{sec:RotFun}.
		\cite{mcc00} does not offer any analysis of
		such uses of the U\=/form and M\=/form.}

Another language with synchronic consonant-vowel metathesis
which has been analysed as phonologically conditioned
is Luang in the Timor region \citep{tata15}.
For Luang, \citet[24]{tata15} propose that metathesis
is one of several phonological processes which operates to join
adjacent words into a single rhythm unit with only one stressed syllable.

Whatever the case may be for Rotuman or Luang,
metathesis in Amarasi cannot be reduced
to a phonologically conditioned process.
As discussed in Chapters \ref{ch:SynMet} and \ref{ch:DisMet},
metathesis in Amarasi is the only phonological marking
of certain syntactic and/or discourse structures.

Amarasi nouns followed by cardinal and ordinal numerals
provide the clearest demonstration that Amarasi metathesis
is not phonologically conditioned.
When followed by a cardinal number, nouns occur in the U\=/form.
However, when followed by an ordinal number, nouns occur in the M\=/form.
Examples are given in Table \ref{tab:AmaNouNum1},
which shows the noun \ve{neno} `day' followed by cardinal and ordinal numbers.\footnote{
		The ordinal numbers in \ref{tab:AmaNouNum1}
		are those used for counting days and months,
		and are derived from the cardinal numbers
		with the addition of a glottal stop as a suffix or infix.}

\begin{table}[ht]
	\caption{Amarasi nouns and numerals}\label{tab:AmaNouNum1}
	\centering
		\begin{threeparttable}[b]
		\begin{tabular}{llll} \lsptoprule
	Underlying	&Phonetic	&	& gloss\\\midrule
	\ve{ne\tbr{no} meseʔ}	&[ˌnɛnɔˈmɛsɛʔ]	&\emb{neno-meseq.mp3}{\spk{}}{\apl}	&`a single day'\su{†}\\
	\ve{ne\tbr{on} meseʔ}	&[ˌnɛ.ɔnˈmɛsɛʔ]	&\emb{neon-meseq.mp3}{\spk{}}{\apl}	&`first day (Monday)'\su{‡}	\\
	\ve{ne\tbr{no} nua}	&[ˌnɛnɔˈnʊ.ɐ]	&\emb{neno-nua.mp3}{\spk{}}{\apl}	&`two days'	\\
	\ve{ne\tbr{on} nua-ʔ}	&[ˌnɛ.ɔˈnːʊ.ɐʔ]	&\emb{neon-nuaq.mp3}{\spk{}}{\apl}	&`second day (Tuesday)'	\\
	\ve{ne\tbr{no} teun}\su{\#}&[ˌnɛnɔˈt̪ɛ.ʊn]	&\emb{neno-teun.mp3}{\spk{}}{\apl}	&`three days' \\
	\ve{ne\tbr{on} tenu-ʔ}	&[ˌnɛ.ɔn̪ˈt̪ɛnʊʔ]	&\emb{neon-tenuq.mp3}{\spk{}}{\apl}	&`third day (Wednesday)'	\\
	\ve{ne\tbr{no} haa}	&[ˌnɛnɔˈhaː]	&\emb{neno-haa.mp3}{\spk{}}{\apl}	&`four days'	\\
	\ve{ne\tbr{on} haa-ʔ}	&[ˌnɛ.ɔnˈhaˑʔ]	&\emb{neon-haaq.mp3}{\spk{}}{\apl}	&`fourth day (Thursday)'	\\
	\ve{ne\tbr{no} niim}	&[ˌnɛnɔˈniˑm]	&\emb{neno-niim.mp3}{\spk{}}{\apl}	&`five days'	\\
	\ve{ne\tbr{on} nima-ʔ}	&[ˌnɛ.ɔˈnːimɐʔ]	&\emb{neon-nimaq.mp3}{\spk{}}{\apl}	&`fifth day (Friday)'	\\
	\ve{ne\tbr{no} nee}	&[ˌnɛnɔˈnɛː]	&\emb{neno-nee.mp3}{\spk{}}{\apl}	&`six days'	\\
	\ve{ne\tbr{on} ne\<ʔ\>e}	&[ˌnɛ.ɔˈnːɛʔɛ]	&\emb{neon-neqe.mp3}{\spk{}}{\apl}	&`sixth day (Saturday)'	\\
		\lspbottomrule
		\end{tabular}
			\begin{tablenotes}
				\item [†]
					The phrase \it{neno meseʔ} has the sense of `a single day'.
					The normal phrase for `one day' would be \ve{neeŋgw=ees},
					from \ve{neno} + \ve{=ees}.
				\item [‡]
					The normal reference for the phrases with ordinal numbers
					is to the days of the week, with \ve{neon meseʔ} `first day'
					being Monday and \ve{neon ne\<ʔ\>e} `sixth day' being Saturday.
					The normal phrase for Sunday is \ve{neno krei} `day + church'.
					The phrase \ve{neon hitu-ʔ} `seventh day' is attested twice
					in the Amarasi Bible translation in Genesis 2.
					This shows that the phrases with ordinal numbers
					are not just completely lexicalised phrases.
				\item [\#] The default form for cardinal numerals is the M\=/form.
			\end{tablenotes}
		\end{threeparttable}
\end{table}

There is no phonetic difference between each kind of phrase,
with the exception of the metathesis of the noun
and, where applicable, the addition of the glottal stop forming ordinal numbers.
In every phrase the noun has two syllables
and stress falls on the penultimate vowel of the numeral.
Compare especially the phrase \ve{neno meseʔ} `a single day'
with that of \ve{neon meseʔ} `first day (Monday)',
in which the only difference is in the metathesis of the noun.

Instead, the M\=/form marks that these phrases have different syntactic structures.
Ordinal numbers occur within the noun phrase
while cardinal numbers occur outside of the noun phrase
as the head of a number phrase.
The different syntactic structures of the phrases
\ve{neno meseʔ} and \ve{neon meseʔ}
are shown in \qf{ex2:NenoMeseq} and \qf{ex2:NeonMeseq} below.
See Chapter \ref{ch:SynMet}, especially \srf{sec:OrdNum}, for full details.

%\newpage
\begin{multicols}{2}
	\begin{exe} \let\eachwordone=\textnormal \let\eachwordtwo=\ve
		\ex{\glll {}				ˌnɛnɔ {} {} ˈmɛsɛʔ	{} \\
							\brac{NP} ne\tbr{no} \bracr{} \brac{Num} meseʔ \bracr{}\\
							{} day{{\textbackslash}\tbr{\tsc{u}}} {} {} one{{\textbackslash}\tsc{u}} {}\\
				\glt \lh{\brac{NP}} `a single day' }\label{ex2:NenoMeseq}
		\ex{\glll	{} ˌnɛ.ɔn ˈmɛsɛʔ {} \\
							\brac{NP} ne\tbr{on} meseʔ \bracr{}\\
							{} day{{\textbackslash}\tbr{\tsc{m}}} one{{\textbackslash}\tsc{u}} {}\\
				\glt \lh{\brac{NP}}`first day (i.e. Monday)' }\label{ex2:NeonMeseq}
	\end{exe}
\end{multicols}

Such data rule out an analysis of Amarasi metathesis
as phonologically conditioned,
unless we posit that different
syntactic structures are associated with different abstract
phonological structures with no phonological realisation.

\subsubsection{Metathesis conditioned by intonation}\label{sec:MetConInt}
Based on preliminary data on a north-eastern variety of Meto, \cite{mccko96}
raise the possibility that metathesis in Meto could be prosodically conditioned.\footnote{
		Thanks go to Patrick McConvell for providing me with his unpublished notes.
		Based on this material, the variety described appears to be Miomafo, Insana, or Beboki.}
They noted that they had two examples of a U\=/form verb with falling or low pitch,
and two examples of an M\=/form verb with rising or high pitch.

To test the hypothesis that metathesis in Amarasi could be conditioned by intonation,
I took a random selection of 80 U\=/form verbs and 80 M\=/form verbs
in different sentence positions from a number of natural texts.
The pitch of each verb was recorded as either rise, fall, high, mid, or low.
The results are summarised in \trf{tab:VerMetInt}.

\begin{table}[h]
	\caption{Verbal metathesis and intonation}\label{tab:VerMetInt}
	\centering
		\begin{tabular}{rll}\lsptoprule
						& U\=/form	& M\=/form	\\ \midrule
			rise	& 10			& 9				\\
			high	& 14			& 15			\\
			mid		& 14			& 5				\\
			low		& 2				& 10			\\
			fall	&	40			& 41			\\
			\lspbottomrule
		\end{tabular}
\end{table}

\trf{tab:VerMetInt} shows that the pitch of both U\=/form
and M\=/form verbs is very similar.
About half of both U\=/forms and M\=/forms have a falling pitch
and about a quarter have a rising or high pitch.
The only difference is in the frequency  of mid pitch and low pitch,
with M\=/forms occurring with a low pitch more frequently than U\=/forms
--- the opposite to what would be predicted by \citeauthor{mccko96}'s preliminary hypothesis.
%This difference could be a result of the difficulty in consistently
%distinguishing between these two pitches in running text.

\subsection{Affixation of consonant-vowel melody}\label{sec:AffCVMel}
A final possible analysis of Amarasi metathesis
would be to analyse it under an item and arrangement model of morphology
in which the consonant-vowel template itself is a kind of affix
which combines with to the segments of a word.
This is the analysis proposed by \citet[160f]{st94} for Rotuman metathesis
and would be similar to the analysis of Arabic morphology in \cite{mcc81}.

Under such an analysis,
each consonant of an Amarasi word would be ordered with respect to each other consonant
and each vowel would be ordered with respect to each other vowel,
but consonants and vowels would not be ordered with respect to one another.
An Amarasi word such as \ve{fatu {\tl} faut} `stone'
could then be represented as  either /ft,au/ or /au,ft/.
This segmental information then combines with the appropriate consonant-vowel melody.
This is shown in \qf{ex:fatu/faut} below
which makes explicit the concatenative nature of this analysis,
and in \qf{as:fatu/faut} with autosegmental notation.
Examples \qf{ex:ft,au+CVCV} and \qf{as:fatu} show U\=/forms
and examples \qf{ex:ft,au+CVVC} and \qf{as:faut} show M\=/forms.

\begin{multicols}{2}
	\begin{exe}
		\ex{\begin{xlist}
			\ex{/ft,au/ + CVCV {\ra} \ve{fatu}}\label{ex:ft,au+CVCV}
			\ex{/ft,au/ + CVVC {\ra} \ve{faut}}\label{ex:ft,au+CVVC}
		\end{xlist}}\label{ex:fatu/faut}
	\end{exe}
\end{multicols}
\begin{multicols}{2}
	\begin{exe}
		\ex{\begin{xlist}
			\exa{\xy
				<0em,2cm>*\as{f}="c1",<2em,2cm>*\as{t}="c2",
				<0em,1cm>*\as{C}="C1",<2em,1cm>*\as{C}="C2",<1em,1cm>*\as{V}="V1",<3em,1cm>*\as{V}="V2",
				<1em,0cm>*\as{a}="v1",<3em,0cm>*\as{u}="v2",
				"C1"+U;"c1"+D**\dir{-};"C2"+U;"c2"+D**\dir{-};"v1"+U;"V1"+D**\dir{-};"v2"+U;"V2"+D**\dir{-};
			\endxy}\label{as:fatu}
			\exa{\xy
				<0em,2cm>*\as{f}="c1",<3em,2cm>*\as{t}="c2",
				<0em,1cm>*\as{C}="C1",<3em,1cm>*\as{C}="C2",<1em,1cm>*\as{V}="V1",<2em,1cm>*\as{V}="V2",
				<1em,0cm>*\as{a}="v1",<2em,0cm>*\as{u}="v2",
				"C1"+U;"c1"+D**\dir{-};"C2"+U;"c2"+D**\dir{-};"v1"+U;"V1"+D**\dir{-};"v2"+U;"V2"+D**\dir{-};
			\endxy}\label{as:faut}
		\end{xlist}}\label{as:fatu/faut}
	\end{exe}
\end{multicols}

Analysing Amarasi metathesis as affixation
with different consonant-vowel melodies is possible.
Under such an analysis,
the selection of the appropriate melody
would be determined by morphosyntactic criteria.
While a concatenative analysis accurately describes the data,
there are two ways in which the process-based
analysis adopted in this chapter better fits the Amarasi data.

Firstly, the affixal analysis misses the generalisation
that the M\=/form is always derivable from the U\=/form
by reversal of the final CV sequence.
While morphological consonant-vowel metathesis is rare,
it \emph{is} cross-linguitically attested
(see Chapter \ref{ch:SynchMet}).
Under the concatenative analysis it is not
immediately clear why the derived M\=/form
does not involve other kinds of metathesis,
or other arbitrary substitutions.
Under the concatenative analysis
there is no principled reason that the M\=/form
should not be, for instance, VCVV yielding \ve{fatu} `stone' {\ra} \ve{*afuu}.

Secondly -- and closely linked to the first reason --
the affixal approach to metathesis
misses the cross-linguistic generalisation discussed
in \srf{sec:OriMorMet} and \srf{sec:For ch:SynchMet}
that processes of compensatory metathesis (see further \srf{sec:OriMetAma} below)
are located adjacent to a stressed syllable.
The placement of stress plays no role
in the derivation of M\=/forms under the affixal analysis.
The rule based approach, on the other hand,
achieves this straightforwardly by including the stressed
syllable as the constraining environment after which metathesis occurs.