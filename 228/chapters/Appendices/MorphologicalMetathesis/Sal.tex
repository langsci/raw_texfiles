\section{Salishan}\label{sec:Sal}
The Salishan languages are a family of languages spoken in the Pacific Northwest,
around the western border of the United States of America and Canada (see \frf{fig:LanWesAmeMorMet}).
Most Salishan languages are either critically endangered or have recently become extinct.
I discuss metathesis in three Salishan varieties, all of which belong to the Coast Salish group.
These varieties include two varieties of Straits Salish:
Saanich (\srf{sec:Saa}) and Klallam (\srf{sec:Kla}),
as well as a Central Salishan variety, Halkomelem (\srf{sec:Hal}).
All are spoken in the immediate vicinity of Southern Vancouver island.

In each of these Salishan varieties metathesis signals the so-called
actual aspect described by \citet[215]{thth69}
as an ``action or state in effect at a particular moment''.
\citeauthor{thth69} compare this actual aspect to the Slavic imperfective
as well as the English \it{be {\ldots}-ing} progressive.
I refer to this aspect as the imperfective \tsc{(ipfv)} throughout this section.

In each Salishan language metathesis is only one of a number of processes used to form the imperfective.
Other processes include reduplication, infixation,
glottalisation, apocope, and apophony (among others).
Which process applies can usually, though not always, be predicted based
on the phonological shape of the perfective stem.

\subsection{Saanich}\label{sec:Saa}
I begin my discussion of Salishan metathesis with Saanich,
a variety of Straits Salish.
Saanich metathesis is described in \cite{mo86,mo89}.
Several different processes operate in Saanich to form the imperfective aspect.
These processes include infixation, reduplication, and metathesis.
Which of these processes operates is determined by the shape of the stem,
with the goal being to achieve a CVCC word structure for the imperfective.
In addition to these processes
all non-initial sonorants are glottalised in imperfective forms.

Metathesis occurs in two environments.
Firstly, when the root contains no vowels
and is suffixed with a vowel-initial suffix,
metathesis of this vowel and the root final consonant
occurs to form the imperfective.
Examples are given in \qf{ex:CC-V->CVC} below,
with the `control transitive' suffix \it{-ət}.

\begin{exe}
	\ex{Saanich C\sub{1}C\sub{2}-V\sub{1}{\ldots} {\ra} C\sub{1}V\sub{1}C\sub{2}{\ldots} \hfill\citep[97]{mo89}}\label{ex:CC-V->CVC}
	\sn{\gw\begin{tabular}{llrcll}
			Root				&							&\tsc{pfv}									&			&\tsc{ipfv}							&\\
		\it{{\rt}q'p'}&`patch it'		&\it{xʷ-q'\tbr{p'-\'ə}t}			&{\ra}&\it{xʷ-q'\tbr{\'əp'}t}	&`patching it' \\
		\it{{\rt}sq'}	&`tear it'		&\it{s\tbr{q'-\'ə}t}	&{\ra}&\it{s\tbr{\'əq'}t}			&		`tearing it' \\
		\it{{\rt}sχ}	&`push it'		&\it{s\tbr{χ-\'ə}t}		&{\ra}&\it{s\tbr{\'əχ}t}			&`pushing it' \\
		\it{{\rt}ʃʧ'}	&`whip it'		&\it{ʃ\tbr{ʧ'-\'ə}t}	&{\ra}&\it{ʃ\tbr{\'əʧ'}t}			&`whipping it' \\
		\it{{\rt}tkʷ}	&`break it'		&\it{t\tbr{kʷ-\'ə}t}	&{\ra}&\it{t\tbr{\'əkʷ}t}			&`breaking it' \\
		\it{{\rt}tqʷ}	&`tighten it'	&\it{t\tbr{qʷ-\'ə}t}	&{\ra}&\it{t\tbr{\'əqʷ}t}			&`tightening it' \\
		\it{{\rt}t's}	&`break it'		&\it{t'\tbr{s-\'ə}t}	&{\ra}&\it{t'\tbr{\'əs}t}			&`breaking it' \\
%		\it{{\rt}θkʷ}	&`straighten it out'&\it{θ\tbr{kʷ\'ə}t}		&{\ra}&\it{θ\tbr{\'əkʷ}t}			&`straightening it out' \\
		\it{{\rt}θχ}	&`shove it'		&\it{θ\tbr{χ-\'ə}t}		&{\ra}&\it{θ\tbr{\'əχ}t}			&`shoving it' \\
	\end{tabular}}		
\end{exe}

Similarly CCəC roots form the imperfective
by metathesis of the second consonant with the following vowel.
Examples are given in \qf{ex:CCVC->CVCC} below.
Only stems containing the vowel [ə] undergo metathesis in Saanich.

\newpage
\begin{exe}
	\ex{Saanich C\sub{1}C\sub{2}ə\/C\sub{3} {\ra} C\sub{1}ə\/C\sub{2}C\sub{3} \hfill\citep[93,97]{mo86,mo89}}\label{ex:CCVC->CVCC}
	\sn{\stl{0.4em}\gw\begin{tabular}{llrcll}
			Root									&									&\tsc{pfv}									&			&\tsc{ipfv}					&\\
		\it{{\rt}t\su{θ}'ɬəkʷ'}	&`pinch'					&\it{t\su{θ}'\tbr{ɬ\'ə}kʷ'}	&{\ra}&\it{t\su{θ}'\tbr{\'əɬ}kʷ'}	&`pinching' \\
		\it{{\rt}t͜ɬ'pəχ}				&`scatter'				&\it{t͜ɬ'\tbr{p\'ə}χ}				&{\ra}&\it{t͜ɬ'\tbr{\'əp}χ}		&`scattering' \\
		\it{{\rt}t͜ɬ'kʷ'ət}			&`extinguish it'	&\it{t͜ɬ'\tbr{kʷ'\'ə}t}			&{\ra}&\it{t͜ɬ'\tbr{\'əkʷ'}t}	&`extinguishing it' \\
		\it{{\rt}θɬəqʷ}					&`pierce it'			&\it{θ\tbr{ɬ\'ə}qʷ}					&{\ra}&\it{θ\tbr{\'əɬ}qʷ}			&`piercing it' \\
	\end{tabular}}		
\end{exe}

With stems of other shapes, reduplication or infixation of /ʔ/ occurs.
The process of reduplication copies the first consonant of a CVC root and
places it after the first vowel.
Reduplication applies \it{``[{\ldots}] when stress is on the root and the root either
1) stands alone as a stem by itself or
2) is followed by a suffix beginning with a consonant.''} \citep[95]{mo89}.
Examples are given in \qf{ex:C1VC->C1VC1C} below.
Predictable schwas are transcribed with a breve [ə̆].

\begin{exe}
	\ex{Saanich C\sub{1}\'VC\sub{2} {\ra} C\sub{1}\'VC\sub{1}C\sub{2} \hfill\citep[95]{mo89}}\label{ex:C1VC->C1VC1C}
	\sn{\stl{0.4em}\gw\begin{tabular}{lllcll}
			Root					&							&\tsc{pfv}			&		&\tsc{ipfv}					&\\
		\it{{\rt}qen'}	&`it's stolen'&\it{sqén'}			&\ra&\it{qéqə̆n'}				&`he's stealing' \\
		\it{{\rt}t\su{θ}'eʔ}	&`be on top'	&\it{t\su{θ}'éʔ}			&\ra&\it{t\su{θ}'ét\su{θ}'ə̆ʔ}		&`riding (a horse)' \\
		\it{{\rt}qʷəl'}	&`say'				&\it{qʷ\'əl'}		&\ra&\it{qʷ\'əqʷə̆l'}		&`saying (sth.)' \\
		\it{{\rt}kʷul}	&`school'			&\it{s-kʷúl}		&\ra&\it{s-kʷúkʷə̆l'}		&`going to school' \\
		\it{{\rt}ɬikʷ'}	&`trip'				&\it{ɬíkʷ'-sən}	&\ra&\it{ɬíɬə̆kʷ'-sən'}	&`tripping' \\
	\end{tabular}}		
\end{exe}

In other cases a glottal stop is infixed after the first vowel.
This infixation can also be accompanied by other
various phonological processes such as apophony.
Examples of infixation which do not involve any additional complications
are given in \qf{ex:C1VC->C1VqC} below.

\begin{exe}
	\ex{Saanich C\sub{1}VC\sub{2}(VC) {\ra} C\sub{1}V\it{ʔ}C\sub{2}(VC) \hfill\citep[98]{mo89}}\label{ex:C1VC->C1VqC}
	\sn{\stl{0.4em}\gw\begin{tabular}{lllcll}
			Root								&							&\tsc{pfv}	&			&\tsc{ipfv}					&\\
	%	\it{{\rt}ʔit\su{θ}'}	&`get dressed'&\it{ʔít\su{θ}'əN}		&{\ra}&\it{ʔí\<ʔ\>t\su{θ}'əN'}		&`getting dressed' \\
		\it{{\rt}ʔeʧ'}		&`wipe it'&\it{ʔéʧ'-ət}		&{\ra}&\it{ʔé\<ʔ\>ʧ'-ət}		&`wiping it' \\
		\it{{\rt}ʔiɬən}		&`eat'		&\it{ʔíɬən}			&{\ra}&\it{ʔí\<ʔ\>ɬən'}			&`eating' \\
		\it{{\rt}ʧaqʷ'}		&`sweat'	&\it{ʧáqʷ'-əŋ}	&{\ra}&\it{ʧá\<ʔ\>qʷ'-əŋ'}	&`sweating' \\
		\it{{\rt}weqəs}		&`yawn'		&\it{wéqəs}			&{\ra}&\it{wé\<ʔ\>qəs}			&`yawning' \\
		\it{{\rt}xʷit}		&`jump'		&\it{xʷít-əŋ}		&{\ra}&\it{xʷí\<ʔ\>t-əŋ'}		&`jumping' \\
		\it{{\rt}ʔamət}		&`sleep'	&\it{ʔámət}			&{\ra}&\it{ʔá\<ʔ\>m'ət}			&`sleeping' \\
	\end{tabular}}		
\end{exe}

In Saanich metathesis is one of several
processes which occurs to form the imperfective.
Other processes include reduplication and infixation.
Which process operates is determined by the phonological shape of the stem,
with the goal of forming a CVCC word shape in the imperfective.

It may be possible at an abstract level to analyse surface metathesis in Saanich
as an artefact of other phonological processes.
This is particularly so given that Saanich metathesis only affects roots with schwa /ə/.
This is the approach taken by \cite{de74} for similar data in the
closely related language Lummi (discussed in \srf{sec:EpeApo}),
in which metathesis is analysed as resulting from
stress shift with subsequent deletion of unstressed vowels.

\subsection{Klallam}\label{sec:Kla}
Klallam is very closely related to Saanich
and the data on Klallam metathesis is similar to that in Saanich.
Metathesis in Klallam is described by \cite{thth69}.
As in Saanich, there are a number of process for forming the imperfective aspect in Klallam.
These processes include infixation of /ʔ/, metathesis, and reduplication.

Examples of verbs which form the imperfective by metathesis
are given in \qf{ex:KlaCVC->CCV} below.
All words are cited with the control suffix \it{-t}.
Predictable schwas are transcribed with a breve [ə̆].

\begin{exe}
	\ex{Klallam CCV {\ra} CVC \hfill\citep[216]{thth69}}\label{ex:KlaCVC->CCV}
	\sn{\gw\begin{tabular}{lrcll}
								&\tsc{pfv}									&			&\tsc{ipfv}							&\\
			`tie up'	&\it{q'\tbr{xʷí}-t}					&{\ra}&\it{q'\tbr{íxʷ}-t}			&`tying up'\\
			`scratch'	&\it{χ\tbr{ʧ'í}-t}					&{\ra}&\it{χ\tbr{íʧ'}-t}			&`scratching'\\
			`restrain'&\it{q\tbr{q'í}-t}					&{\ra}&\it{q\tbr{íq'}-t}			&`restraining'\\
			`shoot'		&\it{ʧ\tbr{kʷú}-t}					&{\ra}&\it{ʧ\tbr{úkʷ}-t}			&`shooting'\\
			`throw'		&\it{ʧ\tbr{ʃú}-t}						&{\ra}&\it{ʧ\tbr{ús}-t}				&`throwing'\\
			`shatter'	&\it{t'\tbr{ʦ\'ə}-t}				&{\ra}&\it{t'\tbr{\'əʦ}-t}		&`shattering'\\
			`grasp'		&\it{t͜ɬ'\tbr{kʷ\'ə}-t}			&{\ra}&\it{t͜ɬ'\tbr{\'əkʷ}-t}	&`grasping'\\
			`swallow'	&\it{ŋə̆\tbr{q'\'ə}-t}				&{\ra}&\it{ŋ\tbr{\'əq'}-t}		&`swallowing'\\
			`pick up'	&\it{mə̆\tbr{kʷ'\'ə}-t}			&{\ra}&\it{m\tbr{\'əkʷ'}-t}		&`picking up'\\
			`burn'		&\it{ʧ\tbr{qʷ\'ə}-t}				&{\ra}&\it{ʧ\tbr{\'əqʷ}-t}		&`burning'\\
			`tear'		&\it{ʧ\tbr{χ\'ə}-t}					&{\ra}&\it{ʧ\tbr{\'əχ}-t}			&`tearing'\\
			`chop'		&\it{q'}\it{\tbr{mˀ\'ə}-t}	&{\ra}&\it{q'\tbr{\'əmˀ}-t}		&`chopping'\\
			`bite'		&\it{ʦ'}\it{ə̆\tbr{ŋˀ\'ə}-t}	&{\ra}&\it{ʦ'\tbr{\'əŋˀ}-t}		&`biting'\\
			`put in water'&\it{mə̆\tbr{t\'ə}qʷ-t}	&{\ra}&\it{m\tbr{\'ət}qʷ-t}		&`putting in water'\\
			`pour'		&\it{kʷ\tbr{jˀ\'ə}-t}				&{\ra}&\it{kʷ\tbr{\'əjˀ}-t}		&`pouring'\\
	\end{tabular}}		
\end{exe}

Other verbs form the imperfective by infixation of the glottal stop after the first vowel.
Some examples are given in \qf{ex:KlaC1VC->C1VqC} below.

\newpage
\begin{exe}
	\ex{Klallam C\sub{1}VC\sub{2}(VC) {\ra} C\sub{1}VʔC\sub{2}(VC) \hfill\citep[216]{thth69}}\label{ex:KlaC1VC->C1VqC}
		\sn{\gw\begin{tabular}{llcll}
								&\tsc{pfv}			&			&\tsc{ipfv}					&\\
			`wipe'		&\it{ʔáʧ'-t}		&{\ra}&\it{ʔá\<ʔ\>ʧ'-t}		&`wiping' \\
			`nudge'		&\it{ʦ'út'-t}		&{\ra}&\it{ʦ'ú\<ʔ\>t'-t}		&`nudging' \\
			`make'		&\it{ʧáʧ-t}			&{\ra}&\it{ʧá\<ʔ\>ʧ-t}			&`making' \\
			`blow'		&\it{púxʷ-t}		&{\ra}&\it{pú\<ʔ\>xʷ-t}		&`blowing' \\
			`set fire'&\it{húnə̆-t}		&{\ra}&\it{hú\<ʔ\>nə̆-t}		&`setting fire' \\
	\end{tabular}}		
\end{exe}

Metathesis and glottal stop infixation
are the two most common ways of forming the imperfective in Klallam.
Another strategy is reduplication,
as seen in \it{jáʔ-t} {\ra} \it{jájəʔ-t} `prepare'.
(Reduplication also involves a change in the quality of the root vowel.)

There are also verbs which combine glottal stop infixation with
either reduplication or metathesis.
When metathesis and infixation are combined,
the glottal stop infix ends up after the first consonant.
Examples are given in \qf{ex:KlaCVC->CqCV}.

\begin{exe}
	\ex{Klallam C\sub{1}VC\sub{2} {\ra} C\sub{1}ʔC\sub{2}V \hfill\cite[216]{thth69}}\label{ex:KlaCVC->CqCV}
	\sn{\gw\begin{tabular}{lrcll}
								&\tsc{pfv}					&		&\tsc{ipfv}								&\\
			`beat'		&\it{qʷ'\tbr{úʧ}-t}	&\ra&\it{qʷ'ə̆\<ʔ\>\tbr{ʧú}-t}	&`beating'\\	
			`inflate'	&\it{s\tbr{új}ə̆-t}	&\ra&\it{sə̆\<ʔ\>\tbr{jú}-t}		&`inflating'\\
			`command'	&\it{sá-t}					&\ra&\it{sə̆\<ʔ\>á-t}					&`commanding'\\
	\end{tabular}}		
\end{exe}

In Klallam metathesis is one of at least three
strategies used to form the imperfective.
The fact that a variety of roots -- not only those with medial schwa --
undergo metathesis to form the imperfective
poses a challenge for analyses of the Klallam data
in which metathesis is viewed as an artefact of
other processes, such as epenthesis and vowel deletion,
as discussed by \cite[540]{blga98}.
Regarding such an analysis, \cite[217]{thth69} state:

\begin{quote}
This treatment [an analysis involving true metathesis]
has the advantage of not requiring the setting up of special
hypothetical base forms like *čukʷut [*ʧukʷut `shoot'],
with actual and non-actual forms derived by vowel deletion,
or positing special stress patterns inserting vowels in different positions with relation to root consonants.
The current popular tendency to resort to such abstractions
(even where they may be well motivated in historical-comparative terms)
is at variance with objective consideration
of the facts of particular language structures
and tends to obstruct our efforts to understand how languages change
and to obscure phenomena important in the consideration of typological similarities. %\citep[217]{thth69}
\end{quote}		

\subsection{Halkomelem}\label{sec:Hal}
My summary of metathesis in Halkomelem is based on that provided by \cite{ur11},
who describes the Hul'q'umi'num' (Vancouver Island) dialect.
As in the other Salishan languages discussed,
metathesis in Halkoemelem is one of several processes used to form the imperfective.
Other processes include vowel apophony, reduplication, and vowel deletion.
Which process applies is (mostly) determined by the phonological shape of the verb.

Metathesis occurs when the verb root contains two obstruents followed by a vowel.
Examples are given in \qf{ex:HalCCV{\ra}CVC} below.
As in Saanich, non-initial sonorants
are additionally glottalised in the imperfective.

\begin{exe}
	\ex{Halkomelem C\sub{1}C\sub{2}V {\ra} C\sub{1}VC\sub{2} \hfill (\citet{hu78} in \citealp[477f]{ur11})}\label{ex:HalCCV{\ra}CVC}
	\sn{\stl{0.4em}\gw\begin{tabular}{lrcll}
										&\tsc{pfv}					&		&\tsc{ipfv}					&\\
		`break it'			&\it{p\tbr{qʷá}-t}	&\ra&\it{p\tbr{áqʷ}-t}	&`breaking it' \\
		`break it'			&\it{t'\tbr{qʷ'á}-t}&\ra&\it{t'\tbr{áqʷ'}-t}&`breaking it' \\
		`pull it'				&\it{xʷ\tbr{kʷ'á}-t}&\ra&\it{xʷ\tbr{ákʷ'}-t}&`pulling it' \\
		`tear/split it'	&\it{s\tbr{q'é}-t}	&\ra&\it{s\tbr{éq'}-t}	&`tearing/splitting it' \\
	\end{tabular}}		
\end{exe}

\citet{ur11} compares metathesis to a process of stress shift and schwa insertion,
viewing metathesis as a specific instances of this latter process.
Examples of imperfectives formed by stress shift
and epenthesis are given in \qf{ex:HalCCV->C@C@}.

\begin{exe}
	\ex{Halkomelem C\sub{1}C\sub{2}V {\ra} C\sub{1}ə́C\sub{2}ə \hfill \citep[478]{ur11}}\label{ex:HalCCV->C@C@}
	\sn{\stl{0.4em}\gw\begin{tabular}{lrcll}
												&\tsc{pfv}				&		&\tsc{ipfv}					&\\
		`tell him/her'			&\it{ʦse-t}				&\ra&\it{ʦə́sə-t}				&`telling him/her' \\
		`put it near'				&\it{tse-t}				&\ra&\it{tə́sə-t}				&`putting it near' \\
		`count stitches'		&\it{kʷ'ʃáləs-t}	&\ra&\it{kʷ'ə́ʃəl'əs-t }	&`counting stitches' \\
		`slice out a piece	&\it{ɬʦ'áləs-t}		&\ra&\it{ɬə́ʦ'əl'əs-t}		&`slicing out a piece \\ \hhline{~}
		\hp{`}of weaving'		&									&		&										&\hp{`}of weaving' \\
	\end{tabular}}		
\end{exe}

When the verb begins with CVC where neither consonant is a laryngeal,
or if the verb begins with an obstruent followed by schwa,
the first CV is reduplicated as a prefix to form the perfective.
If the vowel of the reduplicant is not schwa, stress falls on this vowel
and other vowels are reduced to schwa.
If the vowel of the reduplicant is schwa, stress falls on the second vowel.

\newpage
\begin{exe}
	\ex{Halkomelem C\sub{1}V\sub{1}C\sub{2} {\ra} C\sub{1}V\sub{1}C\sub{1}əC\sub{2} \hfill\citep[474f]{ur11}}\label{ex:HalRed}
	\sn{\gw\begin{tabular}{lrcll}
									&\tsc{pfv}		&		&\tsc{ipfv} & \\
		`cut it' 			&\it{ɬíʦ'ət}	&\ra&\it{ɬíɬəʦ'ət}		&`cutting it'\\
		`fight' 			&\it{kʷíntəl}	&\ra&\it{kʷíkʷən'təl}	&`fighting'\\
		`topple down' &\it{jeq'}		&\ra&\it{jéj'əq'}			&`toppling down'\\
		`get near' 		&\it{təs}			&\ra&\it{tətə́s}				&`getting near'\\
		`break' 			&\it{t'əqʷ'}	&\ra&\it{t'ət'ə́qʷ' }	&`breaking'\\
		`stretched taut'&\it{θəkʷ'}	&\ra&\it{θəθ\'əkʷ'}		&`stretching'\\
	\end{tabular}}		
\end{exe}

When the root begins with a sonorant (L) followed by schwa,
the imperfective is reported to be formed by CV reduplication
with subsequent reduction of the initial sonorant to /h/.
Stress falls on the reduplicant and the following schwa is deleted,
resulting in surface metathesis when comparing the perfective and imperfective forms.
%Any non-schwas in the root are reduced to schwa after reduplication.
Examples are given in \qf{ex:HalRed2} below.

\begin{exe}
	\ex{L\sub{1}əC\sub{2} {\ra} h\'əL\sub{1}C\sub{2} \hfill(\citet{hupe95} in \citealp[475]{ur11})}\label{ex:HalRed2}
	\sn{\stl{0.3em}\gw\begin{tabular}{lrcll}
											&\tsc{pfv}				&		&\tsc{ipfv} & \\
		`fill it' 				&\it{l\'əʦ'ət}		&\ra&\it{h\'əl'ʦ't}			&`filling it'\\
		`pile hay' 				&\it{m\'əkʷels}		&\ra&\it{h\'əm'kʷəl's}	&`piling hay'\\
		`bounce a cradle' &\it{n\'əkʷəjəł}	&\ra&\it{h\'ən'kʷəjəł}	&`bouncing a cradle'\\
		`drift						&\it{w\'əqʷ'ətəm}	&\ra&\it{h\'əw'qʷ'ətəm'}&`drifting \\ \hhline{~}
		\hp{`}downstream'	&									&		&										&\hp{`}downstream' \\
	\end{tabular}}		
\end{exe}

The remaining two ways of forming the imperfective
are apophony and schwa deletion.
Both are found with tri-consonantal roots,
the latter only when the suffix is \it{-m}.
Examples are given in \qf{ex:HalApo} below.

\begin{exe}
	\ex{Halkomelem apophony/schwa deletion \hfill\citep[475f]{ur11}}\label{ex:HalApo}
	\sn{\gw\begin{tabular}{lrcll}
										&\tsc{pfv}						&		&\tsc{ipfv} 					& \\
		`slurp it'			&\it{ɬəp't\su{θ}'-t}	&\ra&\it{ɬep't\su{θ}'-t}	&`slurping it'\\
		`seek'					&\it{səwq'}						&\ra&\it{sew'q'}					&`seeking'\\
		`fall apart'		&\it{ʦ'át'əqʷ'əm}			&\ra&\it{ʦ'át'qʷ'əm'}			&`falling apart'\\
		`fall (leaves)'	&\it{t͜ɬ'épəχəm}				&\ra&\it{t͜ɬ'épχəm'}				&`falling (leaves)'\\
	\end{tabular}}		
\end{exe}

In Halkomelem metathesis is one of several processes used to form the imperfective.
Other processes include stress shift, reduplication, apophony, and apocope.
Which process applies is predictable based on the phonological shape of the root.
Metathesis affects roots which contain two obstruents.
%
%In the Salishan languages CV {\ra} VC metathesis is one of several
%processes used to form the imperfective from the perfective.
%The main similarities between the Salishan data and the Amarasi data
%is that in both instances metathesis is associated with a large number of other processes.
%In Amarasi these processes are best analysed as being triggered by metathesis,
%while in the Salishan languages these other processes may have given rise to metathesis (\srf{sec:OriMorMet}).