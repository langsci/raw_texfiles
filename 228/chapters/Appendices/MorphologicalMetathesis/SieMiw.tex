\section{Sierra Miwok}\label{sec:SieMiw}
Sierra Miwok is a language of central California
(see \frf{fig:LanWesAmeMorMet}) related to Ohlone (\srf{sec:Ohl}).
My summary of Sierra Miwok metathesis is based on the description in \cite{fr51}.
Like Ohlone, each verb in Sierra Miwok has multiple stems.
There are three derived stems in Sierra Miwok formed from one of
four different shapes of the primary (underlying) stem.
These different shapes are summarised in \trf{tab:SieMiwVerSte}.
on the next page.

\begin{table}[ht]
	\caption[Sierra miwok verb stems]{Sierra miwok verb stems\su{†} \citep[94f]{fr51}}\label{tab:SieMiwVerSte}
	\centering
		\begin{threeparttable}[b]
			\begin{tabular}{rccccl}\lsptoprule
					&Primary						&Second				&Third				&Fourth				&\\ \midrule
			I		&CVC\tbr{V}ː\tbr{C}	&CVCVCː				&CVCːVC				&CVC\tbr{CV} 	&\\
					&\it{tujaːŋ-}				&\it{tujaŋː-}	&\it{tujːaŋ-}	&\it{tujŋa-}	& `to jump'\\
			II	&CVC\tbr{CV}				&CVC\tbr{VC}ː	&CVCː\tbr{VC}	&CVCCV 				&\\
					&\it{wɨkt̪ɨ-}				&\it{wɨkɨt̪ː-}	&\it{wɨkːɨt̪-}	&\it{wɨkt̪ɨ-}	&`to burn'\\
			III	&CVCːV							&CVCVʔː				&CVCː\tbr{Vʔ}	&CVC\tbr{ʔV}	&\\
					&\it{hamːe}					&\it{hameʔː-}	&\it{hamːeʔ-}	&\it{hamʔe-}	&`to bury'\\
			IV	&CVːC								&CVCː					&CVCː\tbr{Vʔ}	&CVC\tbr{ʔV}	&\\
					&\it{luːʃ-}					&\it{luʃː-}		&\it{luʃːuʔ-}	&\it{luʃʔu-}	&`to win'\\ \lspbottomrule
			\end{tabular}
			\begin{tablenotes}
				\item [†] Stress in Sierra Miwok falls on the first heavy syllable;
									either VCC, VːC, or VCː \citep[7]{fr51}.
									Because it is predictable, I do not indicate its presence in this section.
			\end{tablenotes}
		\end{threeparttable}
\end{table}

The shape of each derived stem is consistent across all four verb classes,
with the exception of the second stem of class IV verbs.
The second derived stem has the shape CVCVCː,
the third stem CVCːVC, and the fourth stem CVCCV.

In all cases the final C-slot is filled by a glottal stop
when the root has only two consonants.
Similarly, when the root has only a single vowel the final V-slot is filled
by /u/ after back rounded vowels,
and by /ɨ/ after all other vowels.
Both these facts can be seen with the primary stem \it{luːʃ-} `to win'
with the CVCːVC third stem \it{luʃː\tbr{uʔ}-}
with the final vowel and consonant
occurring to fill the otherwise empty V-slot and C-slot.

Final consonant-vowel metathesis is found in three cases;
between the primary stem of verb class I and the fourth stem,
with VC {\ra} CV metathesis,
and between the primary stem of verb class II
and the second and third derived stems,
with CV {\ra} VC metathesis.
It is also possible to analyse the epenthetic glottal stop as undergoing metathesis
in the third and fourth stems of class III verbs and class IV verbs.

As in Mutsun, some cases of metathesis in Sierra Miwok 
are instances of phonologically conditioned metathesis.
Before a CC-initial suffix the fourth stem is used,
thereby avoiding a cluster of three consonants.
One example is the class I stem \it{pol\tbr{a}ː\tbr{ŋ}}
`to stagger' + \it{-jnɨ} \tsc{desiderative}
{\ra} \it{pol\tbr{ŋa}jnɨ} \citep[116]{fr51}.

There are also many instances of morphemically conditioned metathesis
with different suffixes of the same phonological shape occurring with different stems.
Such instances are extremely numerous and I do not provide examples here.

In addition there are also instances
in which metathesis alone serves a morphological function.
For instance, one nominalisation strategy
for class I verbs is to use the fourth stem.
Examples are given in \qf{ex:SieMiwVerMet} below,
in which nouns are cited with the \tsc{subjective} suffix \it{-ʔ}.

\newpage
\begin{exe}
	\ex{Sierra Miwok verbalising metathesis \hfill\citep[149]{fr51}}\label{ex:SieMiwVerMet}
	\sn{\gw\begin{tabular}{lrcll}
								&Verb												&			&Noun									&\\
		`to relate'	&\it{ʔut̪\tbr{eːn}-}	&{\ra}&\it{ʔut̪\tbr{ne}-ʔ}		&`myth, tale' \\
		`to tell'		&\it{koy\tbr{oːw}-}	&{\ra}&\it{koy\tbr{wo}-ʔ}		&`words, speech' \\
		`to run'		&\it{hɨw\tbr{aːt̪}-}	&{\ra}&\it{hɨw\tbr{t̪a}-ʔ}		&`race' \\
		`to play'		&\it{ʔaw\tbr{iːn}-}	&{\ra}&\it{ʔaw\tbr{ni}-ʔ}		&`game' \\
		`to live'		&\it{ʔu{\tS}ːu-}		&{\ra}&\it{ʔu{\tS}ʔu-ʔ}			&`health, well-being: year' \\
		`to come'		&\it{ʔɨnːɨ-}				&{\ra}&\it{ʔɨnʔɨ-ʔ-}				&`way, journey' \\
		`to eat'		&\it{ʔɨwːɨ-}				&{\ra}&\it{ʔɨwʔɨ-ʔ}					&`food' \\
	\end{tabular}}
\end{exe}

The processes in Mutsun Ohlone and Sierra Miwok have much in common,
as might be expected from related languages.
However, in Mutsun Ohlone VC {\ra} CV metathesis is a  verbaliser
while in Sierra Miwok the same process is a nominaliser.

%The Sierra Miwok data has two similarities to Amarasi.
%Firstly, metathesis in Sierra Miwok is phonologically conditioned
%in some contexts and morphological in others.
%Secondly, metathesis in Sierra Miwok interacts with empty C-slots and empty V-slots.
%In Sierra Miwok empty C-slots and V-slots are filled by default segments
%which then metathesise with each other or with specified segments.
%
%In Amarasi, empty C-slots also occur and metathesise with filled V-slots
%triggering processes such as consonant insertion and vowel assimilation.
%The way empty C-slots interact with Amarasi metathesis is discussed in \srf{sec:ThePhoRul}
%and evidence for positing empty C-slots is presented in \srf{sec:EmpCSlo}.
