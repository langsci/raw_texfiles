\section{Mutsun Ohlone (Costanoan)}\label{sec:Ohl}
Metathesis in the Mutsun variety of Southern Ohlone (a.k.a. Costanoan),
a now extinct language of central California (see \frf{fig:LanWesAmeMorMet}), is described in \cite{ok79}.
The same author also wrote a grammar of the language published as \citet{ok77}.
In both instances data was drawn from material gathered in the
early twentieth century from the last fluent speaker.

Verbs in Mutsun have two stems, called the primary stem and the derived stem.
The main difference between each stem is that the derived stem is consonant final,
while the primary stem can be either vowel final or consonant final.
\cite{ok79} identifies seven types of stems of which stem types II, IV, and VII show metathesis.
These three stem types are given in \trf{tab:MutPriDerVerSte}
(the fourth stem type is poorly attested in the data).
In all cases the derived stem is formed from the primary
stem by metathesis of the final VC sequence.

\begin{table}[h]
	\caption[Mutsun primary and derived verb stems]{Mutsun primary and derived verb stems \citep[125]{ok79}}\label{tab:MutPriDerVerSte}
	\centering
		\begin{tabular}{rrllll} \lsptoprule
					&Primary 							&Derived 						& \mc{2}{l}{Examples} 				&Gloss\\ \midrule
			II	&CVCV\sub{2}ːC\sub{3}	&CVCC\sub{3}V\sub{2}& \it{pasiːk-}	& \it{paski-}	&`to greet, visit'\\
			IV	&CVːCV\sub{2}C\sub{3}	&CVCC\sub{3}V\sub{2}& \it{liːwak-}	& 						&`to hide nearby'\\
			VII	&CVCːV\sub{2}C\sub{3}	&CVCC\sub{3}V\sub{2}& \it{liʧːej-}	& \it{liʧje-}	&`to stand'\\
			\lspbottomrule
		\end{tabular}
\end{table}

In most cases the use of each stem is either phonologically or morphemically conditioned.
A phonologically conditioned use is found before suffixes which begin with a consonant cluster
in which case VC {\ra} CV metathesis occurs to prevent a cluster of three consonants surfacing.
Examples include \it{pas\tbr{iːk}} `visit' + \it{-jni} `to come to'
{\ra} \it{pas\tbr{ki}jni} `come to visit'
and \it{liʧː\tbr{ej}} `to stand' + \it{-hte} \tsc{perfective} {\ra}
\it{liʧ\tbr{je}hte} `already assumed a standing position'.

Likewise, word-final consonant clusters are disallowed in Mutsun.
As a result derived stems are used before suffixes consisting of a single consonant.
One example is \it{sotː\tbr{er}} `stick out' + \it{-j} \tsc{imperative}
{\ra} \it{sot\tbr{re}j} `stick out [your foot]!' \citep[125]{ok79}.

However, there are also some CV(C) suffixes which only occur with primary stems
and other CV(C) suffixes which only occur with derived stems.
This is a case of morphemically conditioned metathesis (\srf{sec:MorpheConMet}),
in which metathesis is a partial exponent of the morphological
category signalled by the suffix.

Suffixes which take primary stems include the reciprocal suffix \it{-mu}
and the reflexive suffix \it{-pu},
as seen in \it{hiːwo} `scold (s.o.)' + \it{-mu} \tsc{recp} {\ra} \it{hiːwomu} `(they) quarrel' 
and \it{matːal} `face down' + \it{-pu} \tsc{refl} {\ra} \it{matːalpu} `put oneself face down'.
One suffix which takes the derived stem and thus triggers metathesis is \it{-nu} `positional causative',
as seen in \it{matː\tbr{al}} `face down' + \it{-nu} {\ra} \it{mat\tbr{la}nu}
`put (s.o.) face down (into a prone position)' \citep[126]{ok79}.

The morphological function of metathesis comes about because the derived stem
is used in isolation as a non-past tense
and there are a number of cognate nouns
which take the primary stem.
Examples are given in \qf{ex:MutDerMet} below.

\begin{exe}
	\ex{Mutsun derivational metathesis \hfill\citep[127]{ok79}}\label{ex:MutDerMet}
	\sn{\gw\begin{tabular}{llll}
		 					 			&Noun								&Verb							& \\
		 `a cough' 			&\it{toːh\tbr{er}}	&\it{toh\tbr{re}} & `to cough' \\
		 `flute' 				&\it{lulː\tbr{up}}	&\it{lul\tbr{pu}} & `to play the flute' \\
		 `goose' 				&\it{laːl\tbr{ak}}	&\it{lal\tbr{ka}} & `gather geese' \\
		 `nest'					&\it{heːs\tbr{en}}	&\it{hes\tbr{ne}} & `make a nest' \\
		 `pozole (stew)'&\it{pos\tbr{ol}}		&\it{pos\tbr{lo}} & `to make pozole (stew)' \\
	\end{tabular}}		
\end{exe}

Given that Mutsun is now extinct, it is hard to tell exactly how productive metathesis was.
However, the occurrence of the Spanish loanword \it{posol} `pozole (stew)'
with both metathesised  and unmetathesised forms indicates that metathesis was productive.
It is likely that VC {\ra} CV metathesis in Mutsun was used to derive verbs from nouns.\footnote{
		Mutsun \it{posol} is a loan from Spanish \it{pozole}, itself a loan from Nahuatl \it{pozolli}.
		The final /e/ of the Spanish form has been re-analysed in Mutsun as 
		the object case suffix; \it{posoːl-e} \citep[127, fn.14]{ok77}.}

%The main similarity between the Mutsun Ohlone data and the Amarasi data
%is in the distribution of metathesis.
%In both instances metathesis is phonological
%in some contexts and morphological in others.