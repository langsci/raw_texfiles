\subsection{Empty C-Slots}\label{sec:EmpCSlo}
In \srf{sec:Syl} I proposed that the Amarasi syllable
is CVC and in \srf{sec:TheFoo} that the foot
is obligatorily CVCVC with empty C-slots permitted.
In this section I provide evidence for the these empty C-slots in Amarasi.
Under certain conditions there are phonetic traces of actual consonants in these empty C-slots.

In this section I discuss seven situations
in which consonants surface in positions we might not otherwise expect.
The analysis I propose to account for this data is
to posit an obligatory CVCVC foot in which C-slots can be empty. 
The seven phenomena are summarised in \qf{ex:EviEmpCsloAma} below,
along with the location of the empty C-slot within the root they provide evidence for.

\begin{exe}
	\ex{Evidence for Empty C-slots in Amarasi:}\label{ex:EviEmpCsloAma}
		\begin{xlist}
			\exi{\srf{sec:NomInf}}{Glottal stop infixation \hfill(medial)}
			\exi{\srf{sec:EmpCSloConIns}}{Consonant insertion at clitic boundaries  \hfill(final)}
			\exi{\srf{sec:EmpCSloVowAssConIns}}{Vowel assimilation after consonant insertion  \hfill(medial)}
			\exi{\srf{sec:PhoJNatVoc}}{Distribution of native /\j/  \hfill(medial)}
			\exi{\srf{sec:GloStoIns2}}{Glottal stop insertion  \hfill(initial)}
			\exi{\srf{sec:WorFinConIns}}{Consonant insertion in other Meto varieties  \hfill(medial/final)}
			\exi{\srf{sec:NonEtyGloSto}}{Non-etymological glottal stops \hfill(medial)}
		\end{xlist}
\end{exe}