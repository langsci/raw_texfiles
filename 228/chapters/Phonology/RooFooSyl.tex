\subsection{Roots with a foot and syllable (Root {\ra} σ|Ft)}\label{sec:RooFooSyl}
Roots which consist of a foot preceded by a syllable
comprise 9{\%} (178/1,913) of lexical roots I have so far collected.
Given that sequences of three vowels do not occur in Amarasi (\srf{sec:VowSeq}),
such roots are maximally CVC|CVCVC
and minimally V|CVV.
Examples of a range of roots containing a foot and syllable are given in \trf{tab:WorFooSyl} below.

\begin{table}[h]
	\caption[Words with a foot and syllable]{Words with a foot and syllable\su{†}}\label{tab:WorFooSyl}
	\centering
		\begin{threeparttable}[b]
				\begin{tabular}{llllllr} \lsptoprule
				Structure								&Root					&Phonetic		&																	&gloss					&no.&\%\\ \midrule
				 (C)V\hp{C}|CVCV(C)			&\ve{mahataʔ}	&[mɐˈhɐt̪ɐʔ]	&{\emb{mahataq.mp3}{\spk{}}{\apl}}&`itchy'				&56	&31\%\\
				 (C)VC|CVCV(C)					&\ve{bankofaʔ}&[bɐŋˈkɔfɐʔ]&{\emb{bankofaq.mp3}{\spk{}}{\apl}}&`caterpillar'	&49	&28\%\\
				 (C)V\hp{C}|CV\hp{C}V(C)&\ve{sekau}		&[sɛˈkəw]		&{\emb{sekau.mp3}{\spk{}}{\apl}}	&`who?'					&40	&22\%\\
				 (C)VC|CV\hp{C}V(C)			&\ve{karpeo}	&[karˈpɛɔ]	&{\emb{karpeo.mp3}{\spk{}}{\apl}}	&`onion'				&28	&16\%\\
				 \lspbottomrule
				\end{tabular}
			\begin{tablenotes}
				\item [†] In addition to the structures given in this Table,
									I have collected five roots with a syllable and foot
									which begin with a consonant cluster, all names of trees:
									\ve{ʔbakʔuruʔ} `\it{Morinda citrifolia}',
									\ve{ʔnamee} `\it{Pipturus argenteus}',
									\ve{ʔriksusu} `\it{Wrightia pubescens}',
									\ve{ʔbabuʔi} `\it{Pipturis argenteus}',
									and \ve{ʔnankaʔi} `\it{Albizia chinensis}'.
			\end{tablenotes}
	\end{threeparttable}
\end{table}

Many roots of this shape are historic compounds or polymorphemic words.
Thus, \ve{mahataʔ} `itchy' is from
Proto-Malayo-Polynesian *gatəl > *hata with the property
circumfix \ve{ma-\ldots-ʔ} attached (\srf{sec:PropCir}).
Similarly, the first part of \ve{sekau} `who?'
is from Proto-Malayo-Polynesian *sai combined with \ve{kau} of unknown origin.
Likewise, the second part of \ve{karpeo} `garlic' is cognate
with Amanuban \ve{pio} and Molo \ve{peo} indicating that
Amarasi \ve{karpeo} `garlic' is a historic compound,
though the origin of initial \ve{kar} is unclear.\footnote{
		\citet[173]{mi72} gives Molo \ve{<kalapo>}
		(possibly with a final double vowel /kalapoo(ʔ)/)
		as `weeping paperbark' \it{Myrtus leucadendra},
		a kind of tree with white papery bark.
		The initial element of \ve{<kalapo>} may well
		be cognate with the initial element of Amarasi \ve{karpeo},
		based on a resemblance between the bark of this
		tree and the white papery outer skin of garlic bulbs.}