\subsection{Roots with a consonant cluster (root {\ra} C|Ft)}\label{sec:RooConClu}
Roots which consist of a single foot preceded by an extra consonant
are the second most common type of root in my corpus.
21{\%} (401/1,913) of lexical roots have this shape.
Such roots are maximally CCVCVC and minimally CCVV.
Examples of each possible shape are given in \trf{tab:WorSinFooExtCon}.\footnote{
		The root \ve{{\rt}ʔkaunuʔ} `bother' is the only
		Kotos Amarasi root in my corpus with both a consonant cluster
		and the initial sequence of two vowels assigned to a single V-slot.}

\begin{table}[ht]
	\centering\caption{Words with a single foot and an extra consonant}\label{tab:WorSinFooExtCon}
		{\begin{tabular}{llllllr}\lsptoprule
Structure							&Root					&Phonetic		&		&gloss							&no.	&\%\\\midrule
		C|CVCVC						&\ve{kbateʔ}	&[ˈkβat̪ɛʔ]	&{\emb{kbateq.mp3}{\spk{}}{\apl}}	&`k.o. edible grub'	&188	&47\%\\
		C|CVCV\hp{C}			&\ve{bkaʔu}		&[ˈb˺kaʔʊ]	&{\emb{bkaqu.mp3}{\spk{}}{\apl}}	&`fruit bat'				&139	&35\%\\
		C|CV\hp{C}V\hp{C}	&\ve{ʔsao}		&[ʔ̩ˈsaɔ]		&{\emb{qsao.mp3}{\spk{}}{\apl}}		&`viper'						&43		&11\%\\
		C|CV\hp{C}VC			&\ve{snaen}		&[ˈsnaɛn]		&{\emb{snaen.mp3}{\spk{}}{\apl}}	&`sand'							&30		&8\%\\
		\lspbottomrule
		\end{tabular}}
\end{table}

\trf{tab:WorSinFooExtCon} shows that there are
more CCVCVC roots than there are CCVCV roots.
This is unexpected given that for roots of a single foot
there are many more CVCV roots than CVCVC roots
(see \trf{tab:RooSinFoo} on \prf{tab:RooSinFoo}).

One reason for the larger number of CCVCVC roots is
because Amarasi has two circumfixes of the shape \ve{ʔ-{\ldots}-ʔ}:
a nominaliser and a verbal intensive.
In addition to productive uses of these affixes (see \srf{sec:NomQ--q}),
there are many roots with fossilised \ve{ʔ-{\ldots}-ʔ}.
There are 71 CCVCVC roots in my current database which
contains a putative fossil of this suffix;
constituting 38{\%} of all CCVCVC roots.
Two examples are \ve{ʔmukiʔ} `lime' and \ve{na-ʔsekeʔ} `force'.