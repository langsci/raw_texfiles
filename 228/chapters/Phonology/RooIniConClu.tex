\subsubsection{Root-initial consonant clusters}\label{sec:RooIniConClu}
The root-initial consonant clusters attested
in my corpus are given in \trf{tab:AmaRooIniConClu} below,
with consonants sorted by place of articulation.
The consonants \ve{\j} and \ve{gw} are not shown
as they do not occur in any clusters in Kotos Amarasi.

\begin{table}[ht]
	\centering
	\caption[Kotos Amarasi root-initial consonant clusters]
					{Kotos Amarasi root-initial consonant clusters\su{†}}\label{tab:AmaRooIniConClu}
		\begin{threeparttable}[b]
		\begin{tabular}{c|ccc
											ccc
											ccccc|l}\lsptoprule
			C\sub{1}{\da}	
					&	\ve{p}	&	\ve{b}	&	\ve{m}	&	\ve{f}	&	\ve{t}	&	\ve{n}	&	\ve{r}	&	\ve{s}	&	\ve{k}	&	\ve{ʔ}	&	\ve{h}	&	$\underscriptleftarrow{\textrm{C\sub{2}}}$\\ \midrule
	\ve{p}	&		&		&		&		&		&	\ve{pn}	&	\ve{pr}	&	\ve{ps}	&		&		&		&	\\
	\ve{b}	&		&		&		&		&	\ve{bt}	&	\ve{bn}	&	\ve{br}	&	\ve{bs}	&	\ve{bk}	&		&	\ve{bh}	&	\\
	\ve{m}	&		&		&		&	\ve{mf}	&	\ve{mt}	&	\ve{mn}	&	\ve{mr}	&	\ve{ms}	&		&		&		&	\\
	\ve{f}	&		&		&		&		&\ve{ft}	&	\ve{fn}	&	\ve{fr}	&		&		&		&		&	\\
	\ve{t}	&	\ve{tp}	&	\ve{tb}	&		&	\ve{tf}	&		&	\ve{tn}	&	\ve{tr}	&		&		&		&	\ve{th}	&	\\
	\ve{n}	&		&		&	\ve{nm}	&		&		&		&		&	\ve{ns}	&		&		&		&	\\
	\ve{r}	&		&		&		&		&		&		&		&		&		&		&		&	\\
	\ve{s}	&	\ve{sp}	&	\ve{sb}	&	\ve{sm}	&	\ve{sf}	&	\ve{st}	&	\ve{sn}	&	\ve{sr}	&		&	\ve{sk}	&		&		&	\\
	\ve{k}	&	\ve{kp}	&	\ve{kb}	&	\ve{km}	&	\ve{kf}	&	\ve{kt}	&	\ve{kn}	&	\ve{kr}	&	\ve{ks}	&		&		&	\ve{kh}	&	\\
	\ve{ʔ}	&	\ve{ʔp}	&	\ve{ʔb}	&	\ve{ʔm}	&	\ve{ʔf}	&	\ve{ʔt}	&	\ve{ʔn}	&	\ve{ʔr}	&	\ve{ʔs}	&	\ve{ʔk}	&		&	\ve{ʔh}	&	\\
	\ve{h}	&		&		&		&		&		&		&		&		&		&		&		&	\\\lspbottomrule
		\end{tabular}
			\begin{tablenotes}
				\item [†] \ve{b\j} occurs in Ro{\Q}is \ve{b\j ae} `cow' (Kotos \ve{bi\j ae})
									and \ve{fk} occurs in Tais Nonof \ve{fkuun} `stars' (Kotos \ve{kfuun}).
			\end{tablenotes}
		\end{threeparttable}
\end{table}

\newpage
While it is difficult to state general restrictions on the appearance of root
initial consonant clusters for which exceptions cannot be found,
the following preferences can be said to loosely hold.
Firstly, clusters of two identical consonants are disallowed root initially
(but are allowed word initially in polymorphemic words).
Secondly, homorganic clusters are disfavoured root initially.
In particular, sequences of two labial consonants are not found,
with the exception of the cluster /mf/.\footnote{
		This cluster occurs only in the word \ve{mfaun} `many'.}
Thirdly, most Amarasi root-initial clusters involve either
a sonority plateau or sonority rise on the sonority hierarchy:
liquid {\textgreater} nasal {\textgreater} fricative {\textgreater} plosive
(see \citealt[210f]{ble95} for an overview of the sonority sequencing principle and sonority hierarchy),
though, again, exceptions occur.

Apart from these three general restrictions,
other restrictions involve specific consonants.
The glottal stop never occurs as the second member of a cluster,
while the glottal fricative /h/ and the alveolar liquid /r/
do not occur as the first member of any consonant cluster.
The frequency of each attested root-initial cluster
is given in \trf{tab:AmaRooIniConCluFre} below.\footnote{
		The frequencies in \trf{tab:AmaRooIniConCluFre} include the 401 disyllables with an initial
		cluster, eight roots larger than a disyllable with an initial
		cluster, and two monosyllabic functors with an initial cluster.}

\begin{table}[ht]
	\centering\caption{Kotos Amarasi root-initial consonant cluster frequencies}\label{tab:AmaRooIniConCluFre}
		\begin{tabular}{c|ccccccccccc|l}\lsptoprule
			C\sub{1}{\da}
								&	\ve{p}	&	\ve{b}	&	\ve{m}	&	\ve{f}	&	\ve{t}	&	\ve{n}	&	\ve{r}	&	\ve{s}	&	\ve{k}	&	\ve{ʔ}	&	\ve{h}	&	$\underscriptleftarrow{\textrm{C\sub{2}}}$\\ \midrule
				\ve{p}	&		&		&		&		&		&	3	&	2	&	2	&		&		&		&	7	\\
				\ve{b}	&		&		&		&		&	1	&	7	&	8	&	1	&	1	&		&	1	&	19	\\
				\ve{m}	&		&		&		&	1	&	4	&	20	&	1	&	2	&		&		&		&	28	\\
				\ve{f}	&		&		&		&		&	1	&	4	&	5	&		&		&		&		&	10	\\
				\ve{t}	&	1	&	1	&		&	2	&		&	9	&	4	&		&		&		&	2	&	19	\\
				\ve{n}	&		&		&	1	&		&		&		&		&	1	&		&		&		&	2	\\
				\ve{r}	&		&		&		&		&		&		&		&		&		&		&		&		\\
				\ve{s}	&	6	&	5	&	3	&	1	&	3	&	9	&	9	&		&	15	&		&		&	51	\\
				\ve{k}	&	4	&	14	&	7	&	4	&	5	&	21	&	29	&	2	&		&		&	3	&	89	\\
				\ve{ʔ}	&	19	&	33	&	9	&	10	&	23	&	22	&	14	&	20	&	27	&		&	9	&	186	\\
				\ve{h}	&		&		&		&		&		&		&		&		&		&		&		&		\\\midrule
				tot.	&	30	&	53	&	20	&	18	&	37	&	95	&	72	&	28	&	43	&		&	15	&	411	\\\lspbottomrule
		\end{tabular}
\end{table}

\trf{tab:AmaRooIniConCluFre} shows that clusters in which the glottal stop
is the first consonant greatly outnumber any other cluster.
This might indicate that the putative glottal stop initial clusters
are better analysed as a separate series of glottalised or pre-glottalised phonemes.
Under this analysis, sequences such as [ʔb] would be analysed as /ˀb/ or /bˀ/.

Comparable phonemes are regionally attested.
Examples include Dhao and Hawu \citep{gr10}
as well as some of the Rote languages \citep{ta07}
in which voiced implosives occur.\footnote{
		Implosives contrast with plain voiced plosives in Dhao and Hawu,
		while in the Rote languages implosives
		do not contrast with plain voiced plosives.}
Similarly, Waima{\Q}a in eastern Timor has been described with a full series
of glottalised consonants \citep{habo02,hahi06}.\footnote{
		Given the discussion in \citet{do03},
		it may be possible to analyse the Waima{\Q}a glottalised
		consonants as underlying consonant clusters involving a glottal stop.}

Phonetically, [ʔC] clusters \emph{can} be realised
phrase initially as glottalised single segments.
Nonetheless, there are four facts which
support the consonant cluster analysis in Amarasi.

Firstly, phrase medially the glottal stop of such clusters
is realised as a distinct component prior to the following consonant.
One example is the word \ve{ʔbaʔa\=/f} `roots' in the phrase
\ve{hau \mbox{ʔbaʔa-f}} `tree roots' {\ra} [ˌhə̰w̰ʔˈbaʔɐf] {\emb{hau-qbaqaf.mp3}{\spk{}}{\apl}}.
Secondly, words which begin with [ʔC] behave like other words which begin with a consonant cluster
in usually requiring epenthetic [a] after consonant-final roots (\srf{sec:Epe}).
We would not expect epenthesis if such roots began with a single phoneme.
Thirdly, the first person singular prefix for one verb class
consists of a single glottal stop \ve{ʔ-} (\srf{sec:VerAgrPre}), as does the prefixal
component of the nominalising circumfix \ve{ʔ-{\ldots}-ʔ} (\srf{sec:NomQ--q}).
When these affixes attach to a stem, the resulting cluster
is realised in the same way as an equivalent root-initial cluster.
Fourthly, if [ʔC] sequences were underlyingly single segments,
they would have a restricted distribution compared to other segments:
they cannot occur as part of an initial cluster, root medially, or root finally.
These distributional facts are straightforwardly explained by
positing that these segments are clusters.
They simply fit into the phonotactics structures of Amarasi
which does not allow medial or final clusters or initial
clusters of more than two consonants.

The best analysis of [ʔC] sequences in Amarasi
is that they are clusters of a glottal stop followed by a consonant.
There are both historical and typological reasons that such
clusters are the most common consonant clusters in Amarasi.
Historically, many /ʔC/ clusters come from reduction of an initial prefix *ka-.
Similarly, clusters with /k/ as the first member also
often arise from reduction of *ka- \citep[387f]{ed18b}
and /kC/ clusters are the second most common kind of cluster.
Typologically, it has been proposed that ``\emph{The Austronesian languages,
especially in the Timor area, show ample evidence of
utilizing laryngeal gestures in some way in their phonologies.}'' \citep[216]{do03}.
The Amarasi glottal stop initial clusters thus fit into this typological profile.

Finally, it is worth noting that all root-initial glottal
stop initial clusters in Kotos Amarasi have
been simplified in Ro{\Q}is Amarasi through loss
of the glottal stop.
