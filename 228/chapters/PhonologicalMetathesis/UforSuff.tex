%\subsection{No insertion after U\=/form suffix \it{-ʔ}}
%An additional complication in the derivation of the M\=/form
%is only revealed when the data from vowel-initial enclitics is considered.
%As discussed in \srf{sec:Met ch:PhoMet} above, any final consonant
%of the host is retained after attachment of a vowel-initial enclitic.
%
%However, there are at least two words in my data which have a final
%glottal stop in the U\=/form which does not surface
%after a vowel-initial enclitic has been attached.
%These words are \ve{uaba-\tbr{ʔ}} `speech, language'
%and \ve{mabe-\tbr{ʔ}} `evening, time',
%neither of which retains this glottal stop before an enclitic:
%\ve{uaba-ʔ} `speech, language' + \ve{=ees} `one' {\ra} \ve{uab=ees} `one speech'
%and \ve{mabe-ʔ} `evening, time' + \ve{=ees} `one' {\ra} \ve{maeb=ees} `one evening'.
%I analyse the final glottal stop as a suffix redundantly marking the U\=/form.
%
%It must be emphasised here that the vast majority of final glottal stops
%are members of the root and are not U\=/form suffixes.
%There is thus a difference between examples such as
%\ve{ma\tbr{be}-\tbr{ʔ}} {\ra} \ve{ma\tbr{eb}=ees} `evening'
%in which the glottal stop is a suffix and does not occur in the M\=/form
%and examples such as \ve{sba\tbr{keʔ}} {\ra} \ve{sba\tbr{ekʔ}=ees} `branch'
%in which the glottal stop \emph{is} part of the root and
%\emph{does} occur in the M\=/form before enclitics.
%
%While the U\=/form suffix on \ve{mabe-ʔ} `evening'
%does not surface when a vowel-initial enclitic
%is attached, neither does consonant insertion occur
%as would be expected for vowel-final roots.
%This indicates that the enclitic is attached
%to the U\=/form of the stem with the glottal
%stop suffix blocking consonant insertion.
%After metathesis the suffix is removed.
%This analysis is somewhat ad-hoc.
%However, there does not currently seem to be an alternate analysis
%which accounts for the data.