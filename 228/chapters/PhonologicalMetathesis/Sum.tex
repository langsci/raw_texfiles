\section{Conclusion}
Metathesis before vowel-initial enclitics can be
analysed as phonologically conditioned.
When a vowel-initial enclitic is added to a stem
this triggers a number of phonological processes: metathesis,
consonant insertion, and vowel assimilation.

The first process is consonant insertion (\srf{sec:ConIns}).
Consonant insertion occurs because feet require an onset.
The next process is metathesis (\srf{sec:Met ch:PhoMet}).
Metathesis occurs before enclitics to create a crisp edge after an internal prosodic word.
Analysing metathesis as motivated by \tsc{Crisp Edge} is cucially depenent
on the analysis of intervocalic consonants as ambisyllabic (\srf{sec:Syl}).
The final process is vowel assimilation (\srf{sec:VowAss ch:PhoMet}),
under which any vowel which conditioned insertion of a consonant assimilates.
This occurs because after metathesis any such vowel shares
features with the inserted consonant across another C-slot.

In one environment Amarasi metathesis is phonologically conditioned.
It occurs to create a phonological boundary between two prosodic words.
However, as discussed in Chapter \ref{ch:SynchMet},
just because \emph{some} instances of metathesis in a language
are phonologically conditioned, does not mean \emph{all}
instances of metathesis in that language are phonologically conditioned.
In addition to phonologically conditioned metathesis,
Amarasi also has instances of metathesis which
cannot be accounted for by reference to phonology alone.
Amarasi has two kinds of morphological metathesis:
metathesis marking syntactic structures (Chapter \ref{ch:SynMet})
and metathesis marking discourse structures (Chapter \ref{ch:DisMet}).