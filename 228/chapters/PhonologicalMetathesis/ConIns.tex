\section{Consonant insertion}\label{sec:ConIns}
When a vowel-initial enclitic is attached to a vowel-final stem,
a consonant conditioned by the final vowel of the stem is inserted.
After the front vowels /i/ and /e/ the inserted consonant is /\j/.
After the back rounded vowels /u/ and /o/ the inserted consonant is /ɡw/.
Examples are given in \qf{ex:VV->VVC=V} below.
(Consonant insertion after /a/ is discussed in \srf{sec:CliHosFinA} below.)

\begin{exe}
	\ex{VV[+\A\tsc{place}]+=V {\ra} VVC[+\A\tsc{place}]=V} \label{ex:VV->VVC=V}
	\sn{\gw\begin{tabular}{rlllll}
		\ve{nii} 	&+&\ve{=ee}&{\ra}&\ve{nii\tbr{\j}=ee}	& `the pole' \\
		\ve{fee} 	&+&\ve{=ee}&{\ra}&\ve{fee\tbr{\j}=ee}	& `the wife' \\
		\ve{kfuu} &+&\ve{=ee}&{\ra}&\ve{kfuu\tbr{gw}=ee}& `the star' \\
		\ve{oo} 	&+&\ve{=ee}&{\ra}&\ve{oo\tbr{gw}=ee}	& `the bamboo' \\
	\end{tabular}}
\end{exe}

This consonant insertion takes place because
feet in Amarasi require an onset consonant.
The requirement for an onset is a very common cross-linguistically \citep{mcpr93,prsm93}.

The requirement for feet to begin with an onset consonant
is also the reason for glottal stop insertion (\srf{sec:GloStoIns2}),
as illustrated for the vowel-initial stem \ve{ukum} `cuscus' in \qf{as:uukm=ee} below.
Because feet require an initial consonant, a glottal stop is inserted in \qf{as:quukm=ee}.
The glottal stop is the default initial consonant.
(This representations in \qf{as:uukm=ee>quukm=ee}
have been simplified by removing the tiers showing
the prosodic words and morphemes.)

\begin{multicols}{2}
\begin{exe}\ex{\label{as:uukm=ee>quukm=ee}
	\begin{xlist}
	\exa{\xy
		<2.5em,3cm>*\as{\hp{\sub{\tsc{m}}}Ft\sub{\tsc{m}}}="Ft1",<7em,3cm>*\as{\hp{\sub{2}}Ft\sub{2}}="Ft2",
		<1.5em,2cm>*\as{\hp{\sub{1}}σ\sub{1}}="s1",<3.5em,2cm>*\as{\hp{\sub{2}}σ\sub{2}}="s2",<6em,2cm>*\as{\hp{\sub{3}}σ\sub{3}}="s3",<8em,2cm>*\as{\hp{\sub{4}}σ\sub{4}}="s4",
		<1em,1cm>*\as{C}="CV1",<2em,1cm>*\as{V}="CV2",<3em,1cm>*\as{V}="CV3",<4em,1cm>*\as{C}="CV4",<5em,1cm>*\as{C}="CV5",
		<6em,1cm>*\as{V}="CV6",<7em,1cm>*\as{C}="CV7",<8em,1cm>*\as{V}="CV8",<9em,1cm>*\as{C}="CV9",
		<1em,0cm>*\as{\0}="cv1",<2em,0cm>*\as{u}="cv2",<3em,0cm>*\as{u}="cv3",<4em,0cm>*\as{k}="cv4",<5em,0cm>*\as{m}="cv5",
		<6em,0cm>*\as{e}="cv6",<7em,0cm>*\as{ }="cv7",<8em,0cm>*\as{e}="cv8",<9em,0cm>*\as{ }="cv9",
		"cv1"+U;"CV1"+D**\dir{-};"cv2"+U;"CV2"+D**\dir{-};"cv3"+U;"CV3"+D**\dir{-};"cv4"+U;"CV4"+D**\dir{-};"cv5"+U;"CV5"+D**\dir{-};
		"cv6"+U;"CV6"+D**\dir{-};"cv7"+U;"CV7"+D**\dir{};"cv8"+U;"CV8"+D**\dir{-};"cv9"+U;"CV9"+D**\dir{};
		"CV1"+U;"s1"+D**\dir{-};"CV2"+U;"s1"+D**\dir{-};"CV3"+U;"s2"+D**\dir{-};"CV4"+U;"s2"+D**\dir{-};
		"CV5"+U;"s3"+D**\dir{-};"CV6"+U;"s3"+D**\dir{-};"CV7"+U;"s3"+D**\dir{-};"CV7"+U;"s4"+D**\dir{-};"CV8"+U;"s4"+D**\dir{-};"CV9"+U;"s4"+D**\dir{-};
		"s1"+U;"Ft1"+D**\dir{-};"s2"+U;"Ft1"+D**\dir{-};"s3"+U;"Ft2"+D**\dir{-};"s4"+U;"Ft2"+D**\dir{-};
		<1em,0.5cm>*\as{\tikz[red,thick,dashed,baseline=0.9ex]\draw (0,0) rectangle (0.4cm,1.5cm);}="box",
	\endxy}\label{as:uukm=ee}
	\exa{\xy
		<2.5em,3cm>*\as{\hp{\sub{\tsc{m}}}Ft\sub{\tsc{m}}}="Ft1",<7em,3cm>*\as{\hp{\sub{2}}Ft\sub{2}}="Ft2",
		<1.5em,2cm>*\as{\hp{\sub{1}}σ\sub{1}}="s1",<3.5em,2cm>*\as{\hp{\sub{2}}σ\sub{2}}="s2",<6em,2cm>*\as{\hp{\sub{3}}σ\sub{3}}="s3",<8em,2cm>*\as{\hp{\sub{4}}σ\sub{4}}="s4",
		<1em,1cm>*\as{C}="CV1",<2em,1cm>*\as{V}="CV2",<3em,1cm>*\as{V}="CV3",<4em,1cm>*\as{C}="CV4",<5em,1cm>*\as{C}="CV5",
		<6em,1cm>*\as{V}="CV6",<7em,1cm>*\as{C}="CV7",<8em,1cm>*\as{V}="CV8",<9em,1cm>*\as{C}="CV9",
		<1em,0cm>*\as{\tbr{ʔ}}="cv1",<2em,0cm>*\as{u}="cv2",<3em,0cm>*\as{u}="cv3",<4em,0cm>*\as{k}="cv4",<5em,0cm>*\as{m}="cv5",
		<6em,0cm>*\as{e}="cv6",<7em,0cm>*\as{ }="cv7",<8em,0cm>*\as{e}="cv8",<9em,0cm>*\as{ }="cv9",
		"cv1"+U;"CV1"+D**\dir{-};"cv2"+U;"CV2"+D**\dir{-};"cv3"+U;"CV3"+D**\dir{-};"cv4"+U;"CV4"+D**\dir{-};"cv5"+U;"CV5"+D**\dir{-};
		"cv6"+U;"CV6"+D**\dir{-};"cv7"+U;"CV7"+D**\dir{};"cv8"+U;"CV8"+D**\dir{-};"cv9"+U;"CV9"+D**\dir{};
		"CV1"+U;"s1"+D**\dir{-};"CV2"+U;"s1"+D**\dir{-};"CV3"+U;"s2"+D**\dir{-};"CV4"+U;"s2"+D**\dir{-};
		"CV5"+U;"s3"+D**\dir{-};"CV6"+U;"s3"+D**\dir{-};"CV7"+U;"s3"+D**\dir{-};"CV7"+U;"s4"+D**\dir{-};"CV8"+U;"s4"+D**\dir{-};"CV9"+U;"s4"+D**\dir{-};
		"s1"+U;"Ft1"+D**\dir{-};"s2"+U;"Ft1"+D**\dir{-};"s3"+U;"Ft2"+D**\dir{-};"s4"+U;"Ft2"+D**\dir{-};
		<1em,0.5cm>*\as{\tikz[red,thick,dashed,baseline=0.9ex]\draw (0,0) rectangle (0.4cm,1.5cm);}="box",
	\endxy}\label{as:quukm=ee}
	\end{xlist}}
\end{exe}
\end{multicols}

Instead of inserting a glottal stop,
empty C-slots at clitic boundaries
are usually filled by the features of the previous vowel spreading.
Before vowel-initial enclitics this results in either /\j/ or /ɡw/,
depending on the quality of the vowel which spreads.
The way this works is illustrated in \qf{as:niij=ee} below for
the words \ve{nii} {\ra} \ve{nii\j=ee} `pole' and \ve{fee} {\ra} \ve{fee\j=ee} `wife'.

Example \qf{as:niij=ee1} shows the structure of these words before metathesis.
The initial C-slot of the foot containing the enclitic is empty.
In order to provide this foot with an onset, the feature \tsc{[+front]} of the previous
V-slot spreads, resulting in the consonant /\j/ in \qf{as:niij=ee2}.
%Metathesis would then occur as described in \srf{sec:Met ch:PhoMet} above.

\begin{multicols}{2}
\begin{exe}\ex{\label{as:niij=ee}
	\begin{xlist}
	\exa{\xy
%		<3em,5.5cm>*\as{\hp{\sub{1}}PrWd\sub{1}}="PrWd1",<5em,6.5cm>*\as{\hp{\sub{2}}PrWd\sub{2}}="PrWd2",
		<3em,4.5cm>*\as{\hp{\sub{1}}Ft\sub{1}}="Ft1",<7em,4.5cm>*\as{\hp{\sub{2}}Ft\sub{2}}="Ft2",
		<2em,3.5cm>*\as{\hp{\sub{1}}σ\sub{1}}="s1",<4em,3.5cm>*\as{\hp{\sub{2}}σ\sub{2}}="s2",<6em,3.5cm>*\as{\hp{\sub{3}}σ\sub{3}}="s3",<8em,3.5cm>*\as{\hp{\sub{4}}σ\sub{4}}="s4",
		<1em,2.5cm>*\as{C}="CV1",<2em,2.5cm>*\as{V}="CV2",<3em,2.5cm>*\as{C}="CV3",<4em,2.5cm>*\as{V}="CV4",<5em,2.5cm>*\as{C}="CV5",
		<6em,2.5cm>*\as{V}="CV6",<7em,2.5cm>*\as{C}="CV7",<8em,2.5cm>*\as{V}="CV8",<9em,2.5cm>*\as{C}="CV9",
		<1em,1.5cm>*\as{n}="cv1",<2em,1.5cm>*\as{i}="cv2",<3em,1.5cm>*\as{ }="cv3",<4em,1.5cm>*\as{i}="cv4",<5em,1.5cm>*\as{ }="cv5",
		<6em,1.5cm>*\as{e}="cv6",<7em,1.5cm>*\as{ }="cv7",<8em,1.5cm>*\as{e}="cv8",<9em,1.5cm>*\as{ }="cv9",
		<1em,1cm>*\as{f}="cv1.2",<2em,1cm>*\as{e}="cv2.2",<3em,1cm>*\as{ }="cv3.2",<4em,1cm>*\as{e}="cv4.2",<5em,1cm>*\as{ }="cv5.2",
		<6em,1cm>*\as{e}="cv6.2",<7em,1cm>*\as{ }="cv7.2",<8em,1cm>*\as{e}="cv8.2",<9em,1cm>*\as{ }="cv9.2",
		<4em,0cm>*\as{\tsc{[+fr.]}}="f","f"+U;"cv4.2"+D**\dir{-};"f"+U;"cv5.2"+D**\dir{.};"cv5.2"+D;"CV5"+D**\dir{.};"cv4"+D;"cv5"+U**\dir{.};
%		<2.5em,0cm>*\as{\hp{\sub{1}}M\sub{1}}="m1",<7em,0cm>*\as{\hp{\sub{2}}M\sub{2}}="m2",<5.5em,0cm>*\as{=}="=",
%		"m1"+U;"cv1.2"+D**\dir{-};"m1"+U;"cv2.2"+D**\dir{-};"m1"+U;"cv3.2"+D**\dir{};"m1"+U;"cv4.2"+D**\dir{-};"m1"+U;"cv5.2"+D**\dir{};
%		"m2"+U;"cv6.2"+D**\dir{-};"m2"+U;"cv8.2"+D**\dir{-};
		"cv1"+U;"CV1"+D**\dir{-};"cv2"+U;"CV2"+D**\dir{-};"cv3"+U;"CV3"+D**\dir{};"cv4"+U;"CV4"+D**\dir{-};"cv5"+U;"CV5"+D**\dir{};
		"cv6"+U;"CV6"+D**\dir{-};"cv7"+U;"CV7"+D**\dir{};"cv8"+U;"CV8"+D**\dir{-};"cv9"+U;"CV9"+D**\dir{};
		"CV1"+U;"s1"+D**\dir{-};"CV2"+U;"s1"+D**\dir{-};"CV3"+U;"s1"+D**\dir{-};"CV3"+U;"s2"+D**\dir{-};"CV4"+U;"s2"+D**\dir{-};"CV5"+U;"s2"+D**\dir{-};
		"CV5"+U;"s3"+D**\dir{-};"CV6"+U;"s3"+D**\dir{-};"CV7"+U;"s3"+D**\dir{-};"CV7"+U;"s4"+D**\dir{-};"CV8"+U;"s4"+D**\dir{-};"CV9"+U;"s4"+D**\dir{-};
		"s1"+U;"Ft1"+D**\dir{-};"s2"+U;"Ft1"+D**\dir{-};"s3"+U;"Ft2"+D**\dir{-};"s4"+U;"Ft2"+D**\dir{-};
%		"Ft1"+U;"PrWd1"+D**\dir{-};"PrWd1"+U;"PrWd2"+D**\dir{-};"Ft2"+U;"PrWd2"+D**\dir{-};
		<4.5em,1.75cm>*\as{\tikz[red,thick,dashed,baseline=0.9ex]\draw (0,0) rectangle (0.8cm,2cm);}="box",
	\endxy}\label{as:niij=ee1}
	\exa{\xy
%		<3em,5.5cm>*\as{\hp{\sub{1}}PrWd\sub{1}}="PrWd1",<5em,6.5cm>*\as{\hp{\sub{2}}PrWd\sub{2}}="PrWd2",
		<3em,4.5cm>*\as{\hp{\sub{1}}Ft\sub{1}}="Ft1",<7em,4.5cm>*\as{\hp{\sub{2}}Ft\sub{2}}="Ft2",
		<2em,3.5cm>*\as{\hp{\sub{1}}σ\sub{1}}="s1",<4em,3.5cm>*\as{\hp{\sub{2}}σ\sub{2}}="s2",<6em,3.5cm>*\as{\hp{\sub{3}}σ\sub{3}}="s3",<8em,3.5cm>*\as{\hp{\sub{4}}σ\sub{4}}="s4",
		<1em,2.5cm>*\as{C}="CV1",<2em,2.5cm>*\as{V}="CV2",<3em,2.5cm>*\as{C}="CV3",<4em,2.5cm>*\as{V}="CV4",<5em,2.5cm>*\as{C}="CV5",
		<6em,2.5cm>*\as{V}="CV6",<7em,2.5cm>*\as{C}="CV7",<8em,2.5cm>*\as{V}="CV8",<9em,2.5cm>*\as{C}="CV9",
		<1em,1.5cm>*\as{n}="cv1",<2em,1.5cm>*\as{i}="cv2",<3em,1.5cm>*\as{ }="cv3",<4em,1.5cm>*\as{i}="cv4",<5em,1.5cm>*\as{\j}="cv5",
		<6em,1.5cm>*\as{e}="cv6",<7em,1.5cm>*\as{ }="cv7",<8em,1.5cm>*\as{e}="cv8",<9em,1.5cm>*\as{ }="cv9",
		<1em,1cm>*\as{f}="cv1.2",<2em,1cm>*\as{e}="cv2.2",<3em,1cm>*\as{ }="cv3.2",<4em,1cm>*\as{e}="cv4.2",<5em,1cm>*\as{\j}="cv5.2",
		<6em,1cm>*\as{e}="cv6.2",<7em,1cm>*\as{ }="cv7.2",<8em,1cm>*\as{e}="cv8.2",<9em,1cm>*\as{ }="cv9.2",
		<4.5em,0cm>*\as{\tsc{[+fr.]}}="f","f"+U;"cv4.2"+D**\dir{-};"f"+U;"cv5.2"+D**\dir{-};
%		<2.5em,0cm>*\as{\hp{\sub{1}}M\sub{1}}="m1",<7em,0cm>*\as{\hp{\sub{2}}M\sub{2}}="m2",<5.5em,0cm>*\as{=}="=",
%		"m1"+U;"cv1.2"+D**\dir{-};"m1"+U;"cv2.2"+D**\dir{-};"m1"+U;"cv3.2"+D**\dir{};"m1"+U;"cv4.2"+D**\dir{-};"m1"+U;"cv5.2"+D**\dir{};
%		"m2"+U;"cv6.2"+D**\dir{-};"m2"+U;"cv8.2"+D**\dir{-};
		"cv1"+U;"CV1"+D**\dir{-};"cv2"+U;"CV2"+D**\dir{-};"cv3"+U;"CV3"+D**\dir{};"cv4"+U;"CV4"+D**\dir{-};"cv5"+U;"CV5"+D**\dir{-};
		"cv6"+U;"CV6"+D**\dir{-};"cv7"+U;"CV7"+D**\dir{};"cv8"+U;"CV8"+D**\dir{-};"cv9"+U;"CV9"+D**\dir{};
		"CV1"+U;"s1"+D**\dir{-};"CV2"+U;"s1"+D**\dir{-};"CV3"+U;"s1"+D**\dir{-};"CV3"+U;"s2"+D**\dir{-};"CV4"+U;"s2"+D**\dir{-};"CV5"+U;"s2"+D**\dir{-};
		"CV5"+U;"s3"+D**\dir{-};"CV6"+U;"s3"+D**\dir{-};"CV7"+U;"s3"+D**\dir{-};"CV7"+U;"s4"+D**\dir{-};"CV8"+U;"s4"+D**\dir{-};"CV9"+U;"s4"+D**\dir{-};
		"s1"+U;"Ft1"+D**\dir{-};"s2"+U;"Ft1"+D**\dir{-};"s3"+U;"Ft2"+D**\dir{-};"s4"+U;"Ft2"+D**\dir{-};
%		"Ft1"+U;"PrWd1"+D**\dir{-};"PrWd1"+U;"PrWd2"+D**\dir{-};"Ft2"+U;"PrWd2"+D**\dir{-};
		<5em,1.75cm>*\as{\tikz[red,thick,dashed,baseline=0.9ex]\draw (0,0) rectangle (0.4cm,2cm);}="box",
	\endxy}\label{as:niij=ee2}
	\end{xlist}}
\end{exe}
\end{multicols}

The process is the same when /ɡw/ is inserted.
This is shown for \ve{kfuu} {\ra} \mbox{\ve{kfuugw=ee}} `star'
and \ve{oo} {\ra} \ve{oogw=ee} `bamboo' in \qf{as:kfuugw=ee} below.
In \qf{as:kfuugw=ee1} the initial C-slot of the second foot is empty.
As a result, the features \tsc{[+back,+round]} of the previous vowel spread
producing the consonant /ɡw/ in \qf{as:kfuugw=ee2}.
The initial empty C-slot of \ve{oo} `bamboo'
is also filled by a glottal stop in \qf{as:kfuugw=ee2}.

\begin{multicols}{2}
\begin{exe}\ex{\label{as:kfuugw=ee}
	\begin{xlist}
	\ex\raisebox{\dimexpr-\totalheight+5ex\relax}{\xy
%		<3em,5.75cm>*\as{\hp{\sub{1}}PrWd\sub{1}}="PrWd1",<5em,6.75cm>*\as{\hp{\sub{2}}PrWd\sub{2}}="PrWd2",
		<3em,4.75cm>*\as{\hp{\sub{1}}Ft\sub{1}}="Ft1",<7em,4.75cm>*\as{\hp{\sub{2}}Ft\sub{2}}="Ft2",
		<2em,3.75cm>*\as{\hp{\sub{1}}σ\sub{1}}="s1",<4em,3.75cm>*\as{\hp{\sub{2}}σ\sub{2}}="s2",<6em,3.75cm>*\as{\hp{\sub{3}}σ\sub{3}}="s3",<8em,3.75cm>*\as{\hp{\sub{4}}σ\sub{4}}="s4",
		<1em,2.75cm>*\as{C}="CV1",<2em,2.75cm>*\as{V}="CV2",<3em,2.75cm>*\as{C}="CV3",<4em,2.75cm>*\as{V}="CV4",<5em,2.75cm>*\as{C}="CV5",
		<6em,2.75cm>*\as{V}="CV6",<7em,2.75cm>*\as{C}="CV7",<8em,2.75cm>*\as{V}="CV8",<9em,2.75cm>*\as{C}="CV9",
		<1em,1.75cm>*\as{kf}="cv1",<2em,1.75cm>*\as{u}="cv2",<3em,1.75cm>*\as{ }="cv3",<4em,1.75cm>*\as{u}="cv4",<5em,1.75cm>*\as{ }="cv5",
		<6em,1.75cm>*\as{e}="cv6",<7em,1.75cm>*\as{ }="cv7",<8em,1.75cm>*\as{e}="cv8",<9em,1.75cm>*\as{ }="cv9",
		<1em,1.25cm>*\as{ }="cv1.2",<2em,1.25cm>*\as{o}="cv2.2",<3em,1.25cm>*\as{ }="cv3.2",<4em,1.25cm>*\as{o}="cv4.2",<5em,1.25cm>*\as{ }="cv5.2",
		<6em,1.25cm>*\as{e}="cv6.2",<7em,1.25cm>*\as{ }="cv7.2",<8em,1.25cm>*\as{e}="cv8.2",<9em,1.25cm>*\as{ }="cv9.2",
		<4em,0cm>*\as{{$\left[\hspace{-2mm}\begin{array}{l}\textrm{\tsc{+ba.}}\\\textrm{\tsc{+ro.}}\end{array}\hspace{-2mm}\right]$}}="f","f"+U;"cv4.2"+D**\dir{-};"f"+U;"cv5.2"+D**\dir{.};"cv5.2"+D;"CV5"+D**\dir{.};"cv4"+D;"cv5"+U**\dir{.};
%		<2.5em,0cm>*\as{\hp{\sub{1}}M\sub{1}}="m1",<7em,0cm>*\as{\hp{\sub{2}}M\sub{2}}="m2",<5.5em,0cm>*\as{=}="=",
%		"m1"+U;"cv1.2"+D**\dir{-};"m1"+U;"cv2.2"+D**\dir{-};"m1"+U;"cv3.2"+D**\dir{};"m1"+U;"cv4.2"+D**\dir{-};"m1"+U;"cv5.2"+D**\dir{};
%		"m2"+U;"cv6.2"+D**\dir{-};"m2"+U;"cv8.2"+D**\dir{-};
		"cv1"+U;"CV1"+D**\dir{-};"cv2"+U;"CV2"+D**\dir{-};"cv3"+U;"CV3"+D**\dir{};"cv4"+U;"CV4"+D**\dir{-};"cv5"+U;"CV5"+D**\dir{};
		"cv6"+U;"CV6"+D**\dir{-};"cv7"+U;"CV7"+D**\dir{};"cv8"+U;"CV8"+D**\dir{-};"cv9"+U;"CV9"+D**\dir{};
		"CV1"+U;"s1"+D**\dir{-};"CV2"+U;"s1"+D**\dir{-};"CV3"+U;"s1"+D**\dir{-};"CV3"+U;"s2"+D**\dir{-};"CV4"+U;"s2"+D**\dir{-};"CV5"+U;"s2"+D**\dir{-};
		"CV5"+U;"s3"+D**\dir{-};"CV6"+U;"s3"+D**\dir{-};"CV7"+U;"s3"+D**\dir{-};"CV7"+U;"s4"+D**\dir{-};"CV8"+U;"s4"+D**\dir{-};"CV9"+U;"s4"+D**\dir{-};
		"s1"+U;"Ft1"+D**\dir{-};"s2"+U;"Ft1"+D**\dir{-};"s3"+U;"Ft2"+D**\dir{-};"s4"+U;"Ft2"+D**\dir{-};
%		"Ft1"+U;"PrWd1"+D**\dir{-};"PrWd1"+U;"PrWd2"+D**\dir{-};"Ft2"+U;"PrWd2"+D**\dir{-};
		<4.5em,2cm>*\as{\tikz[red,thick,dashed,baseline=0.9ex]\draw (0,0) rectangle (0.8cm,2cm);}="box",
	\endxy}\label{as:kfuugw=ee1}
	\ex\raisebox{\dimexpr-\totalheight+5ex\relax}{\xy
%		<3em,5.75cm>*\as{\hp{\sub{1}}PrWd\sub{1}}="PrWd1",<5em,6.75cm>*\as{\hp{\sub{2}}PrWd\sub{2}}="PrWd2",
		<3em,4.75cm>*\as{\hp{\sub{1}}Ft\sub{1}}="Ft1",<7em,4.75cm>*\as{\hp{\sub{2}}Ft\sub{2}}="Ft2",
		<2em,3.75cm>*\as{\hp{\sub{1}}σ\sub{1}}="s1",<4em,3.75cm>*\as{\hp{\sub{2}}σ\sub{2}}="s2",<6em,3.75cm>*\as{\hp{\sub{3}}σ\sub{3}}="s3",<8em,3.75cm>*\as{\hp{\sub{4}}σ\sub{4}}="s4",
		<1em,2.75cm>*\as{C}="CV1",<2em,2.75cm>*\as{V}="CV2",<3em,2.75cm>*\as{C}="CV3",<4em,2.75cm>*\as{V}="CV4",<5em,2.75cm>*\as{C}="CV5",
		<6em,2.75cm>*\as{V}="CV6",<7em,2.75cm>*\as{C}="CV7",<8em,2.75cm>*\as{V}="CV8",<9em,2.75cm>*\as{C}="CV9",
		<1em,1.75cm>*\as{kf}="cv1",<2em,1.75cm>*\as{u}="cv2",<3em,1.75cm>*\as{ }="cv3",<4em,1.75cm>*\as{u}="cv4",<5em,1.75cm>*\as{ɡw}="cv5",
		<6em,1.75cm>*\as{e}="cv6",<7em,1.75cm>*\as{ }="cv7",<8em,1.75cm>*\as{e}="cv8",<9em,1.75cm>*\as{ }="cv9",
		<1em,1.25cm>*\as{ʔ}="cv1.2",<2em,1.25cm>*\as{o}="cv2.2",<3em,1.25cm>*\as{ }="cv3.2",<4em,1.25cm>*\as{o}="cv4.2",<5em,1.25cm>*\as{ɡw}="cv5.2",
		<6em,1.25cm>*\as{e}="cv6.2",<7em,1.25cm>*\as{ }="cv7.2",<8em,1.25cm>*\as{e}="cv8.2",<9em,1.25cm>*\as{ }="cv9.2",
		<4.5em,0cm>*\as{{$\left[\hspace{-2mm}\begin{array}{l}\textrm{\tsc{+ba.}}\\\textrm{\tsc{+ro.}}\end{array}\hspace{-2mm}\right]$}}="f","f"+U;"cv4.2"+D**\dir{-};"f"+U;"cv5.2"+D**\dir{-};
%		<2.5em,0cm>*\as{\hp{\sub{1}}M\sub{1}}="m1",<7em,0cm>*\as{\hp{\sub{2}}M\sub{2}}="m2",<5.5em,0cm>*\as{=}="=",
%		"m1"+U;"cv1.2"+D**\dir{-};"m1"+U;"cv2.2"+D**\dir{-};"m1"+U;"cv3.2"+D**\dir{};"m1"+U;"cv4.2"+D**\dir{-};"m1"+U;"cv5.2"+D**\dir{};
%		"m2"+U;"cv6.2"+D**\dir{-};"m2"+U;"cv8.2"+D**\dir{-};
		"cv1"+U;"CV1"+D**\dir{-};"cv2"+U;"CV2"+D**\dir{-};"cv3"+U;"CV3"+D**\dir{};"cv4"+U;"CV4"+D**\dir{-};"cv5"+U;"CV5"+D**\dir{-};
		"cv6"+U;"CV6"+D**\dir{-};"cv7"+U;"CV7"+D**\dir{};"cv8"+U;"CV8"+D**\dir{-};"cv9"+U;"CV9"+D**\dir{};
		"CV1"+U;"s1"+D**\dir{-};"CV2"+U;"s1"+D**\dir{-};"CV3"+U;"s1"+D**\dir{-};"CV3"+U;"s2"+D**\dir{-};"CV4"+U;"s2"+D**\dir{-};"CV5"+U;"s2"+D**\dir{-};
		"CV5"+U;"s3"+D**\dir{-};"CV6"+U;"s3"+D**\dir{-};"CV7"+U;"s3"+D**\dir{-};"CV7"+U;"s4"+D**\dir{-};"CV8"+U;"s4"+D**\dir{-};"CV9"+U;"s4"+D**\dir{-};
		"s1"+U;"Ft1"+D**\dir{-};"s2"+U;"Ft1"+D**\dir{-};"s3"+U;"Ft2"+D**\dir{-};"s4"+U;"Ft2"+D**\dir{-};
%		"Ft1"+U;"PrWd1"+D**\dir{-};"PrWd1"+U;"PrWd2"+D**\dir{-};"Ft2"+U;"PrWd2"+D**\dir{-};
		<5em,2cm>*\as{\tikz[red,thick,dashed,baseline=0.9ex]\draw (0,0) rectangle (0.525cm,2cm);}="box",
	\endxy}\label{as:kfuugw=ee2}
	\end{xlist}}
\end{exe}
\end{multicols}

The newly-inserted consonant in \qf{as:niij=ee} and \qf{as:kfuugw=ee}
is shared between the prosodic word containing the host
and the prosodic word containing the enclitic,
as illustrated in \qf{as:niij=ee,kfuugw=ee1} below.
This is resolved by metathesis, which yields the structure
in \qf{as:niij=ee,kfuugw=ee2} with a crisp edge after the internal prosodic word.

\begin{multicols}{2}
\begin{exe}\ex{\label{as:niij=ee,kfuugw=ee}
	\begin{xlist}
	\exa{\xy
		<3em,6.5cm>*\as{\hp{\sub{1}}PrWd\sub{1}}="PrWd1",<4.5em,7.5cm>*\as{\hp{\sub{2}}PrWd\sub{2}}="PrWd2",
		<3em,5.5cm>*\as{\hp{\sub{1}}Ft\sub{1}}="Ft1",<7em,5.5cm>*\as{\hp{\sub{2}}Ft\sub{2}}="Ft2",
		<2em,4.5cm>*\as{\hp{\sub{1}}σ\sub{1}}="s1",<4em,4.5cm>*\as{\hp{\sub{2}}σ\sub{2}}="s2",<6em,4.5cm>*\as{\hp{\sub{3}}σ\sub{3}}="s3",<8em,4.5cm>*\as{\hp{\sub{4}}σ\sub{4}}="s4",
		<1em,3.5cm>*\as{C}="CV1",<2em,3.5cm>*\as{V}="CV2",<3em,3.5cm>*\as{C}="CV3",<4em,3.5cm>*\as{V}="CV4",<5em,3.5cm>*\as{C}="CV5",
		<6em,3.5cm>*\as{V}="CV6",<7em,3.5cm>*\as{C}="CV7",<8em,3.5cm>*\as{V}="CV8",<9em,3.5cm>*\as{C}="CV9",
		<1em,2.5cm>*\as{n}="cv1",<2em,2.5cm>*\as{i}="cv2",<3em,2.5cm>*\as{ }="cv3",<4em,2.5cm>*\as{i}="cv4",<5em,2.5cm>*\as{\j}="cv5",
		<6em,2.5cm>*\as{e}="cv6",<7em,2.5cm>*\as{ }="cv7",<8em,2.5cm>*\as{e}="cv8",<9em,2.5cm>*\as{ }="cv9",
		<1em,2cm>*\as{f}="cv1.3",<2em,2cm>*\as{e}="cv2.3",<3em,2cm>*\as{ }="cv3.3",<4em,2cm>*\as{e}="cv4.3",<5em,2cm>*\as{\j}="cv5.3",
		<6em,2cm>*\as{e}="cv6.3",<7em,2cm>*\as{ }="cv7.3",<8em,2cm>*\as{e}="cv8.3",<9em,2cm>*\as{ }="cv9.3",
		<1em,1.5cm>*\as{kf}="cv1.4",<2em,1.5cm>*\as{u}="cv2.4",<3em,1.5cm>*\as{ }="cv3.4",<4em,1.5cm>*\as{u}="cv4.4",<5em,1.5cm>*\as{ɡw}="cv5.4",
		<6em,1.5cm>*\as{e}="cv6.4",<7em,1.5cm>*\as{ }="cv7.4",<8em,1.5cm>*\as{e}="cv8.4",<9em,1.5cm>*\as{ }="cv9.4",
		<1em,1cm>*\as{ʔ}="cv1.2",<2em,1cm>*\as{o}="cv2.2",<3em,1cm>*\as{ }="cv3.2",<4em,1cm>*\as{o}="cv4.2",<5em,1cm>*\as{ɡw}="cv5.2",
		<6em,1cm>*\as{e}="cv6.2",<7em,1cm>*\as{ }="cv7.2",<8em,1cm>*\as{e}="cv8.2",<9em,1cm>*\as{ }="cv9.2",
		<2.5em,0cm>*\as{\hp{\sub{1}}M\sub{1}}="m1",<7em,0cm>*\as{\hp{\sub{2}}M\sub{2}}="m2",<5em,0cm>*\as{=}="=",
		"m1"+U;"cv1.2"+D**\dir{-};"m1"+U;"cv2.2"+D**\dir{-};"m1"+U;"cv3.2"+D**\dir{ };"m1"+U;"cv4.2"+D**\dir{-};
		"m2"+U;"cv6.2"+D**\dir{-};"m2"+U;"cv8.2"+D**\dir{-};
		"cv1"+U;"CV1"+D**\dir{-};"cv2"+U;"CV2"+D**\dir{-};"cv3"+U;"CV3"+D**\dir{};"cv4"+U;"CV4"+D**\dir{-};"cv5"+U;"CV5"+D**\dir{-};
		"cv6"+U;"CV6"+D**\dir{-};"cv7"+U;"CV7"+D**\dir{};"cv8"+U;"CV8"+D**\dir{-};"cv9"+U;"CV9"+D**\dir{};
		"CV1"+U;"s1"+D**\dir{-};"CV2"+U;"s1"+D**\dir{-};"CV3"+U;"s1"+D**\dir{-};"CV3"+U;"s2"+D**\dir{-};"CV4"+U;"s2"+D**\dir{-};"CV5"+U;"s2"+D**\dir{-};
		"CV5"+U;"s3"+D**\dir{-};"CV6"+U;"s3"+D**\dir{-};"CV7"+U;"s3"+D**\dir{-};"CV7"+U;"s4"+D**\dir{-};"CV8"+U;"s4"+D**\dir{-};"CV9"+U;"s4"+D**\dir{-};
		"s1"+U;"Ft1"+D**\dir{-};"s2"+U;"Ft1"+D**\dir{-};"s3"+U;"Ft2"+D**\dir{-};"s4"+U;"Ft2"+D**\dir{-};
		"Ft1"+U;"PrWd1"+D**\dir{-};"PrWd1"+U;"PrWd2"+D**\dir{-};"Ft2"+U;"PrWd2"+D**\dir{-};
		<5em,2.25cm>*\as{\tikz[red,thick,dashed,baseline=0.9ex]\draw (0,0) rectangle (0.525cm,3cm);}="box",
	\endxy}\label{as:niij=ee,kfuugw=ee1}
	\exa{\xy
		<2.5em,6.5cm>*\as{\hp{\sub{1}}PrWd\sub{1}}="PrWd1",<4.5em,7.5cm>*\as{\hp{\sub{2}}PrWd\sub{2}}="PrWd2",
		<2.5em,5.5cm>*\as{\hp{\sub{\tsc{m}}}Ft\sub{\tsc{m}}}="Ft1",<7em,5.5cm>*\as{\hp{\sub{2}}Ft\sub{2}}="Ft2",
<1.5em,4.5cm>*\as{\hp{\sub{1}}σ\sub{1}}="s1",<3.5em,4.5cm>*\as{\hp{\sub{2}}σ\sub{2}}="s2",<6em,4.5cm>*\as{\hp{\sub{3}}σ\sub{3}}="s3",<8em,4.5cm>*\as{\hp{\sub{4}}σ\sub{4}}="s4",
		<1em,3.5cm>*\as{C}="CV1",<2em,3.5cm>*\as{V}="CV2",<3em,3.5cm>*\as{V}="CV3",<4em,3.5cm>*\as{C}="CV4",<5em,3.5cm>*\as{C}="CV5",
		<6em,3.5cm>*\as{V}="CV6",<7em,3.5cm>*\as{C}="CV7",<8em,3.5cm>*\as{V}="CV8",<9em,3.5cm>*\as{C}="CV9",
		<1em,2.5cm>*\as{n}="cv1",<2em,2.5cm>*\as{i}="cv2",<3em,2.5cm>*\as{i}="cv3",<4em,2.5cm>*\as{ }="cv4",<5em,2.5cm>*\as{\j}="cv5",
		<6em,2.5cm>*\as{e}="cv6",<7em,2.5cm>*\as{ }="cv7",<8em,2.5cm>*\as{e}="cv8",<9em,2.5cm>*\as{ }="cv9",
		<1em,2cm>*\as{f}="cv1.3",<2em,2cm>*\as{e}="cv2.3",<3em,2cm>*\as{e}="cv3.3",<4em,2cm>*\as{ }="cv4.3",<5em,2cm>*\as{\j}="cv5.3",
		<6em,2cm>*\as{e}="cv6.3",<7em,2cm>*\as{ }="cv7.3",<8em,2cm>*\as{e}="cv8.3",<9em,2cm>*\as{ }="cv9.3",
		<1em,1.5cm>*\as{kf}="cv1.4",<2em,1.5cm>*\as{u}="cv2.4",<3em,1.5cm>*\as{u}="cv3.4",<4em,1.5cm>*\as{ }="cv4.4",<5em,1.5cm>*\as{ɡw}="cv5.4",
		<6em,1.5cm>*\as{e}="cv6.4",<7em,1.5cm>*\as{ }="cv7.4",<8em,1.5cm>*\as{e}="cv8.4",<9em,1.5cm>*\as{ }="cv9.4",
		<1em,1cm>*\as{ʔ}="cv1.2",<2em,1cm>*\as{o}="cv2.2",<3em,1cm>*\as{o}="cv3.2",<4em,1cm>*\as{ }="cv4.2",<5em,1cm>*\as{ɡw}="cv5.2",
		<6em,1cm>*\as{e}="cv6.2",<7em,1cm>*\as{ }="cv7.2",<8em,1cm>*\as{e}="cv8.2",<9em,1cm>*\as{ }="cv9.2",
		<2em,0cm>*\as{\hp{\sub{1}}M\sub{1}}="m1",<7em,0cm>*\as{\hp{\sub{2}}M\sub{2}}="m2",<5em,0cm>*\as{=}="=",
		"m1"+U;"cv1.2"+D**\dir{-};"m1"+U;"cv2.2"+D**\dir{-};"m1"+U;"cv3.2"+D**\dir{-};
		"m2"+U;"cv6.2"+D**\dir{-};"m2"+U;"cv8.2"+D**\dir{-};
		"cv1"+U;"CV1"+D**\dir{-};"cv2"+U;"CV2"+D**\dir{-};"cv3"+U;"CV3"+D**\dir{-};"cv4"+U;"CV4"+D**\dir{};"cv5"+U;"CV5"+D**\dir{-};
		"cv6"+U;"CV6"+D**\dir{-};"cv7"+U;"CV7"+D**\dir{};"cv8"+U;"CV8"+D**\dir{-};"cv9"+U;"CV9"+D**\dir{};
		"CV1"+U;"s1"+D**\dir{-};"CV2"+U;"s1"+D**\dir{-};"CV3"+U;"s2"+D**\dir{-};"CV4"+U;"s2"+D**\dir{-};
		"CV5"+U;"s3"+D**\dir{-};"CV6"+U;"s3"+D**\dir{-};"CV7"+U;"s3"+D**\dir{-};"CV7"+U;"s4"+D**\dir{-};"CV8"+U;"s4"+D**\dir{-};"CV9"+U;"s4"+D**\dir{-};
		"s1"+U;"Ft1"+D**\dir{-};"s2"+U;"Ft1"+D**\dir{-};"s3"+U;"Ft2"+D**\dir{-};"s4"+U;"Ft2"+D**\dir{-};
		"Ft1"+U;"PrWd1"+D**\dir{-};"PrWd1"+U;"PrWd2"+D**\dir{-};"Ft2"+U;"PrWd2"+D**\dir{-};
		<4.4em,4cm>*\as{\tikz[red,thick,dashed,baseline=0.9ex]\draw (0,0) -- (0,6.5cm);}="line",
	\endxy}\label{as:niij=ee,kfuugw=ee2}
	\end{xlist}}
\end{exe}
\end{multicols}

\newpage
For words which contain a surface vowel sequence,
the C-slot affected by metathesis is empty.
As a result, metathesis has no discernible effect on the surface structure of such words.
However, in \srf{sec:VowAss ch:PhoMet} I show that we can still detect metathesis
for words in which the surface vowel sequence involves vowels of different qualities.

The reason vowel features spread at clitic
boundaries rather than simply inserting a glottal stop is not immediately clear.
One possibility is that the glottal stop is only inserted word initially.
However, this does not account for glottal stop insertion in examples
such as \ve{n-ita} `see' {\ra} \ve{na-\tbr{ʔ}ita-b} `show' (\srf{sec:GloStoInsVocPre}),
in which case the inserted glottal stop is not word-initial.
\label{WhyNotGlottal?}

Another possible reason could be due to the differing morphological structures.
Glottal stop insertion happens at affix boundaries
while vowel features spread at clitic boundaries.
However, this runs counter to the general principle
that the phonological structures of Amarasi are `blind'
to the morphological structures.

Another possible reason why vowel features spread at clitic
boundaries could be that the C-slot which is filled by /\j/ or /ɡw/
is the final C-slot of the previous foot.
This is the analysis I currently favour,
though it seems somewhat counter intuitive
given that the whole reason vowel features spread is to
provide the \emph{following} clitic with an onset.

Given the presence of glottal stop insertion
word initially in forms such as \ve{ukum} {\ra} [ˈ\tbr{ʔ}ʊkʊm] `cuscus'
and foot initially in words such as \ve{n-ita} `see' {\ra} \ve{na-\tbr{ʔ}ita-b} `show',
there does not currently seem to be a good phonological
reason why such glottal stop insertion does not also happen at clitic boundaries.
It may simply be a fact of Amarasi that at clitic
boundaries vowel features spread to produce /\j/ and /ɡw/.

\subsection{Location of the inserted consonant}\label{sec:LocInsCon}
Amarasi consonant insertion can be analysed as a result of
vowel features spreading into an adjacent empty C-slot.
However, this empty C-slot could logically originate with the foot containing the clitic host,
or the foot containing the enclitic.
There are at least three reasons for analysing this empty C-slot
as originating with the foot of the clitic host rather than the enclitic:

\begin{exe}
	\exi{i.}{It simplifies the analysis of consonant-final words.}
	\exi{ii.}{There are varieties of Meto in which consonant insertion
						occurs with no enclitic present (\srf{sec:WorFinConIns}).}
	\exi{iii.}{It can provide a reason vowel features spread to produce
						an onset rather than glottal stop insertion.}
\end{exe}

Regarding the first point above, if the empty C-slot originated with the enclitic, forms such as
\ve{muʔit}+\ve{=ee} `the animal' {\ra} \ve{muiʔt=ee} would be underlyingly
\ve{muʔit}+\ve{=Cee} and we would probably expect something like \ve{\tcb{*}muʔittee},
(cf. gemination in Seri, discussed by \citealt[631]{mast83}).
Additional rules would then have to be introduced to avoid such forms.

While the empty C-slot probably originates with the foot of the clitic host,
the consonant inserted in this C-slot
is not a member of the same morpheme as the clitic host.
Instead, it is an epenthetic segment which does not
belong to either the previous or following morpheme,
much like epenthetic glottal stops (\srf{sec:GloStoIns}).

Nonetheless, the syllabification of Amarasi words (\srf{sec:Syl})
means that when a vowel-initial enclitic is attached to a stem
the final C-slot of this stem is ambisyllabic,
occurring as the coda of the initial foot
and as the onset of the foot containing the enclitic.
As a result, it is a member of more than one prosodic word.
This dual membership is the reason why metathesis
is triggered before vowel-initial enclitics in Amarasi (\srf{sec:Met ch:PhoMet}).
Metathesis rearranges the phonotactic structure
of the host and enclitic such that after
metathesis this C-slot is the onset to only one prosodic word.
