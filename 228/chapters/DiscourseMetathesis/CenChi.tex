\section{Centre of chiasmus}\label{sec:CenChi}
Another use of discourse U\=/forms is to mark the centre of a chiasmus.
Chiasmus is a kind of inverted parallelism in which parallel pairs
are repeated on either side of another parallel pair.
A simple example of chiasmus in English is given in \qf{ex:Macbeth}
below from act 1, scene 1 of Shakespeare's Macbeth.

\begin{exe}
	\ex{\it{\xytext{\fbox{Fair\vp{f}}\xybarconnect[4][-]{6}&is&\fbox{\,foul}\xybarconnect[2][-]{2}&and&\fbox{\,foul}&is&\fbox{\,fair.}}}}\label{ex:Macbeth}
\end{exe}

In Amarasi a U\=/form can occur in the middle of chiasmus to signal
that the information before this U\=/form
is going to be repeated again, as illustrated in \qf{ex:ChiUfor} and \qf{ex:ChiUfor2} below.
There are 20 examples of U\=/forms marking chiasmus in my corpus.

\begin{exe}
	\ex{Chiastic U\=/forms:}\label{ex:ChiUfor}
		\begin{xlist}
			\exi{A.}{information\sub{1}}
			\sn{B. verb{\U}}
			\exi{A.}{information\sub{1}}
		\end{xlist}
\end{exe}

\begin{exe}
	\ex{\xytext{\fbox{information\sub{1}\vp{|}}\xybarconnect[2][-](D,D){2}\xybarconnect[2][<-]{1}&\fbox{U\=/form\vp{|}}\xybarconnect[2][->]{1}&\fbox{information\sub{1}\vp{|}}}}\label{ex:ChiUfor2}
\end{exe}

By using a U\=/form in such examples the speaker
signals non-resolution and puts the listener
in a mild state of suspense, communicating
roughly `This is unresolved. Pay attention'.
The listener would thus be prepared for something unexpected.
By repeating old information
instead of providing something new,
the speaker emphasises this repeated information.
The U\=/form is resolved by the information on either side of it.

At its most simple, such U\=/forms are preceded and followed
by an identical word or phrase.
This simple chiastic structure constitutes nearly all
instances of chiasmus with a central U\=/form in my corpus (18/20 instances).
One example is given \qf{ex:130925-1, 4.10} below,
in which the U\=/form \ve{n-moni} `lives' is both preceded and followed
by M\=/forms of the verb \ve{n-boʔis} `praises'.

\begin{exe}
	\ex{\xytext{\fbox{praise{\Mv}\vp{|}}\xybarconnect[2][-](D,D){2}\xybarconnect[2][<-]{1}&\fbox{live{\U}\vp{|}}\xybarconnect[2][->]{1}&\fbox{praise{\Mv}\vp{|}}}}
	\sn{\glll	n-boiʔs=ee mate-s. aam baab-f=ee n-\tbr{moni} =te, \hspace{10mm} n-boiʔs=ee. \tcb{???\footnotemark}\\
						n-boʔis=ee mate-s ama baba-f=ee n-moni =te {} n-boʔis=ee\\
						\n-praise={\eeV} die-{\at} father FZ/MB-{\F}={\ee} \n-live{\tbrU} ={\te} {} \n-praise={\eeV}\\
			\glt	`He praised him a lot (\emph{lit.} dead). While the uncle was alive, he praised him.'
						\txrf{130925-1, 4.10} {\emb{130925-1-04-10.mp3}{\spk{}}{\apl}}}\label{ex:130925-1, 4.10}
\end{exe}
\footnotetext{
		The final word/phrase of this sentence
		was not transcribed by Roni,
		who recorded and transcribed this text.
		Due to the faintness of the recording,
		I also cannot make out the final word/phrase of this sentence.
		My best guess is that it is \ve{na-hiin\j=ee} `he knew it/him'.}

A more complex example is given in \qf{ex:120923-2-, 4.29-4.34} below,
in which the material which surrounds the U\=/form
is repeated multiple times, including two repetitions
which are not identical but parallel.
The structure of this chiasmus is given in \qf{ex:Chi ex:120923-2-, 4.29-4.34}.

\begin{exe}
	\ex{Chiasmus of \qf{ex:120923-2-, 4.29-4.34}:}\label{ex:Chi ex:120923-2-, 4.29-4.34}
		\begin{xlist}
			\exi{A.}{\ve{na-pein=koo} = `gets you'}
			\exi{A.}{\ve{na-pein=koo} = `gets you'}
			\sn{\lh{A′.} B. \ve{t-soʔi} = `counts{\U}'}
			\sn{A′. \ve{n-naaʔ=koo} = `holds you'}
			\sn{A′. \ve{n-naaʔ=koo} = `holds you'}
			\exi{A.}{\ve{na-pein=koo} = `gets you'}
		\end{xlist}
\end{exe}

\begin{exe}
	\ex{Catching a thief in your garden: \txrf{120923-2} {\emb{120923-2-04-29-04-34.mp3}{\spk{}}{\apl}}}\label{ex:120923-2-, 4.29-4.34}
	\sn{\xytext{\fbox{gets you\vp{h}}\xybarconnect[4][-]{1}\xybarconnect[2][-](D,D){1}&\fbox{gets you\vp{h}}\xybarconnect[4][-]{4}\xybarconnect[2][-](D,D){2}&counts{\U}&\fbox{holds you}\xybarconnect[2][-]{1}&\fbox{holds you}\xybarconnect[2][-](D,D){1}&\fbox{gets you\vp{h}}}}
		\begin{xlist}
			\ex{\glll	\sf{karna} tuan=ee na-pein =koo, na-pein =koo \hspace{15mm} naa\j=ena =ma\\
								\sf{karna} tuan=ee na-peni =koo na-peni =koo {} nai=ena =ma\\
								because owner={\ee} \na-get{\M} ={\koo} \na-get{\M} ={\koo} {} already={\een} =and\\
					\glt	`Because the owner gets you, he's got you already and' \txrf{4.29}}
			\ex{\glll	t-\tbr{soʔi} =t iin n-naaʔ =koo,\\
								t-soʔi =te ini n-naʔa =koo\\
								\tg-count{\tbrU} ={\te} {\iin} \n-hold{\M} ={\koo}\\
					\glt	`(someone) counts while he holds you' \txrf{4.32}}
			\ex{\glll	{n}-{naaʔ} ={koo} na-heer=een ees reʔ \hspace{40mm} iin {na}-{pein} ={koo} reʔ ia.\\
								n-naʔa =koo na-hera=ena esa reʔ {} ini na-peni =koo reʔ ia\\
								\n-hold{\M} ={\koo} \na-tight={\een} one {\req} {} {\iin} \na-get{\M} ={\koo} {\req} {\ia}\\
					\glt	`He holds you tight like this, the one who's got you here.' \txrf{4.34}}
	\end{xlist}
\end{exe}

The U\=/forms in examples \qf{ex:130925-1, 4.10} and \qf{ex:120923-2-, 4.29-4.34}
have a dual function, marking both chiasmus
and dependent coordination (\srf{sec:DepCoo}).
In these cases the information which resolves the U\=/form is similar/identical
to that which precedes the U\=/form.

In addition to such examples in which there is only a single layer
on either side of the U\=/form,
there are at least two examples of a more complex chiastic structure
in which there is more than one layer surrounding the U\=/form,
as exemplified in \qf{ex:ComChi} and \qf{ex:ComChi2} below.

\begin{exe}
	\ex{Complex Chiasmus:}\label{ex:ComChi}
		\begin{xlist}
			\exi{A.}{information\sub{1}}
			\sn{B. information\sub{2}}
			\sn{{\lh{B. }}C. verb{\U}}
			\sn{B. information\sub{2}}
			\exi{A.}{information\sub{1}}
		\end{xlist}
	\ex{\xytext{\fbox{information\sub{1}}\xybarconnect[4][-](D,D){4}\xybarconnect[4][<-]{2}&\fbox{information\sub{2}}
							\xybarconnect[2][-](D,D){2}\xybarconnect[2][<-](U,UL){1}&\fbox{verb{\U}\vp{|}}\xybarconnect[2][->](UR,U){1}\xybarconnect[4][->]{2}&\fbox{information\sub{2}}&
							\fbox{information\sub{1}}}}\label{ex:ComChi2}
\end{exe}

The first of these examples is given in \qf{ex:130825-6, 2.57-3.01} below.
This example consists of an outer layer (`I just followed the target')
and an inner layer (`I couldn't offer') with the core U\=/form \ve{ʔ-nesi} `more'.
The chiastic structure of \qf{ex:130825-6, 2.57-3.01} is given in \qf{ex:Chi ex:130825-6, 2.57-3.01}.


\begin{exe}
	\ex{Chiasmus in \qf{ex:130825-6, 2.57-3.01}:}\label{ex:Chi ex:130825-6, 2.57-3.01}
		\begin{xlist}
			\exi{A.}{\it{I just followed the target}}
			\sn{B. \it{I couldn't offer}}
			\sn{\lh{B. }C. \it{any more}{\U}}
			\sn{B. \it{I couldn't offer}}
			\exi{A.}{\it{I just followed the target}}
		\end{xlist}
\end{exe}

\begin{exe}
	\ex{Donating money: \txrf{130825-6} {\emb{130825-6-02-57-03-01.mp3}{\spk{}}{\apl}}}\label{ex:130825-6, 2.57-3.01}
		\begin{xlist}
			\ex{\gll	au ʔ-tuin=aah neʔ \sf{target}.\\
								{\au} \q-follow=just {\reqt} target\\
					\glt	`I just followed the target.' \txrf{}}
			\ex{\glll	au ka= bisa ʔ-\sf{korban} a|ʔ-\tbr{nesi} =f.\\
								au ka= bisa ʔ-\sf{korban} {\a}ʔ-nesi =fa\\
								{\au} {\ka}= can \q-sacrifice {\a\q}-more{\tbrU} ={\fa}\\
					\glt	`I couldn't offer any more.' \txrf{2.57}}
			\ex{\gll	au ka= bisa ʔ-\sf{korban}.\\
								{\au} {\ka}= can \q-sacrifice \\
					\glt	`I couldn't offer.' \txrf{}}
			\ex{\gll	au ʔ-tuin=aah neʔ \sf{target}.\\
								{\au} \q-follow=just {\reqt} target\\
					\glt	`I just followed the target.' \txrf{3.01}}
	\end{xlist}
\end{exe}

A second example is given in \qf{ex:130825-6, 9.20-9.28} below,
with the chiastic structure summarised in \qf{ex:Chi ex:130825-6, 9.20-9.28}.
In this example the outer layer consists of the person \ve{Olpi},
the inner layer consists of the activity `went down to bathe',
and the U\=/form in the centre in \qf{ex:130825-6, 9.20b} is \ve{n-sae n-fani} `came back up'.
This core is also followed by an additional event in \qf{ex:130825-6, 9.20-9.28}.

\begin{exe}
	\ex{Chiasmus in \qf{ex:130825-6, 9.20-9.28}:}\label{ex:Chi ex:130825-6, 9.20-9.28}
		\begin{xlist}
			\exi{A.}{\it{Olpi}}
			\sn{B. \it{went down to bathe}}
			\sn{\lh{B. }C. \it{came back}{\U} \it{up}}
			\sn{\lh{B. }D. \it{handed me a towel and soap}}
			\sn{B. \it{I went down to bathe}}
			\exi{A.}{\it{Olpi}}
		\end{xlist}
\end{exe}

\begin{exe}
	\ex{The narrator and Olpi are down at the garden:\txrf{130825-6} {\emb{130825-6-09-20-09-28.mp3}{\spk{}}{\apl}}}\label{ex:130825-6, 9.20-9.28}
		\begin{xlist}
		\ex{\glll	Olpi n-saun na-niu =ma nsa--,\\
							Olpi n-sanu na-niu =ma\\
							Olpi \n-go.down{\M} \na-bathe =and\\
				\glt	`Olpi went down to bathe and' \txrf{9.20}}
		\ex{\glll	n-sae n-\tbr{fani} =t\\
							n-sae n-fani =te\\
							\n-go.up \n-back{\tbrU} ={\te}\\
				\glt	`when he came back up,' \txrf{9.22}}\label{ex:130825-6, 9.20b}
		\ex{\glll	n-nonaʔ =kau nehh, n-nonaʔ =kau nehh, \sf{handuk} =am \sf{sabu} \\
							n-nonaʔ =kau {} n-nonaʔ =kau {} \sf{handuk} =ma \sf{sabu} \\
							{\n}-hand ={\kau} {} {\n}-hand ={\kau} {} towel =and soap \\
				\glt	`he handed me handed me  a towel and soap.' \txrf{9.23}}\label{ex:130825-6, 9.20}
		\ex{\glll	ʔ-saun u-niu =t, \\
							ʔ-sanu u-niu =te \\
							{\q}-go.down{\M} {\qu}-bathe ={\te}\\
				\glt `I went down and bathed while' \txrf{9.27}}
		\ex{\glll	Olpi n-ait nehh, hap-- hapee\j=ii \\
							Olpi n-aiti {} {} hapei=ii \\
							Olpi {\n}-pick.up{\M} {} {} mobile.phone={\ii} \\
				\glt `Olpi picked up the mobile phone' \txrf{9.28}}
	\end{xlist}
\end{exe}

U\=/forms can mark the centre of a chiasmus.
By introducing a U\=/form the narrator
sets up the discourse as unresolved and introduces
the possibility of an unexpected event.
By then denying this possibility and
repeating the information which occurred before the U\=/form,
the narrator emphasises the repeated information.
A U\=/form in the centre of chiasmus is resolved by
the information on either side of it.
