\subsection{Morphological metathesis}\label{sec:MorMet}
Morphological metathesis is when metathesis is the
only realisation of a morphological category.
Morphological metathesis has been reported
for about a dozen languages worldwide, of which about half
are found in the greater Timor region, where Meto is also spoken.
Cases of metathesis in this region are discussed in the next section.

It is important to reiterate here that a unitary analysis of synchronic
metathesis is not always possible.
Thus, the existence of morphological metathesis in a particular
language does not mean that all cases of metathesis in that
language should be analysed as morphological processes.

Similarly, a single morphological process (of metathesis of any other kind)
in a single language can have different functions in different contexts.
One example is the English the suffix \it{-(e)s} with allomorphs /-əz/, /-z/ and /-s/.
This suffix is a plural marker on nouns and a third person agreement marker on verbs.
This situation is found with morphological metathesis in several languages
in which metathesis has different morphological functions in different contexts
and/or with different word classes.
This is the case for Rotuman (\srf{sec:RotFun}), Leti (\srf{sec:LetFun}),
Mambae (\srf{sec:MamFun}), Helong (\srf{sec:HelFun}), and Amarasi
(Chapter \ref{ch:SynMet} and \ref{ch:DisMet})