\subsection{Loan nominals}\label{sec:LoaNou}
More evidence that the use of M\=/forms is productive in Amarasi
comes from the behaviour of loanwords.
When one or more parts of an attributive phrase is a loanword,
the first nominal usually takes the expected M\=/form according
to the normal rules discussed in Chapter \ref{ch:StrMetAma}.

Two examples of nominal phrases involving assimilated loans
are given in \qf{ex:130921-1, 1.35} and \qf{ex:130825-6, 17.02} below.
In \qf{ex:130921-1, 1.35} the second part of the phrase \ve{rais pirsai-t} `matters of belief'
is a loan from Malay \it{percaya} `believe' (ultimately from Sanskrit \it{pratyeti}).
Similarly in \qf{ex:130825-6, 17.02} the second part of the nominal phrase \ve{amnaah bubur}
`porridge eater' is a loan from Malay \it{bubur} `porridge'.

\begin{exe}
	\ex{\glll hai mi-noniʔ n-ok, a|n-ma-toom n-ok \hspace{35mm} hiit \tbr{rais} \tbr{pirsai}-\tbr{t}.\\
						hai mi-noniʔ n-oka {\a}n-ma-toma n-oka {} hiti rasi pirsai-t\\
						{\hai} {\mi}-learn {\n}-{\ok} {\a\n}-{\mak}-about {\n}-{\ok} {} {\hiit} matter{\tbrM} believe-{\at}\\
			\glt `We learnt about matters to do with (our) belief.'
						\txrf{130921-1, 1.35} {\emb{130921-1-01-35.mp3}{\spk{}}{\apl}}}\label{ex:130921-1, 1.35}
	\ex{\glll au ka= \tbr{amna}-\tbr{ah} \tbr{bubur} =kau =fa!\\
						au ka= amna-ha-t bubur =kau =fa\\
						{\au} {\ka}= {\at}-eat{\tbrM} porridge{\U} ={\kau} ={\fa}\\
			\glt	`I don't eat porridge!'
						(\emph{lit.} `I'm not a porridge eater!')\txrf{130825-6, 17.02} {\emb{130825-6-17-02.mp3}{\spk{}}{\apl}}}\label{ex:130825-6, 17.02}
\end{exe}

In \qf{ex:130825-6, 13.28} below both elements of the nominal phrase \ve{oot dinas} `work car'
are loans. The first element \ve{oto} is from Dutch \it{auto} `car'
and \ve{dinas} is ultimately from Dutch \it{dienst} [diːnst].
The nominal \ve{dinas} is furthermore unassimilated,
as Amarasi does not have the phoneme /d/.
Nonetheless, the first nominal of this nominal phrase occurs in the expected M\=/form
and the second nominal also occurs in the M\=/form 
with consonant-vowel metathesis as expected before enclitics (see Chapter \ref{ch:PhoMet}).

\begin{exe}
	\ex{\glll	iin n-eik iin \tbr{oot} \sf{\tbr{diins}}=ii =m na-sae-baʔ =kau.\\
						ini n-eki ini oto \sf{dinas}=ii =ma na-sae-baʔ =kau\\
						{\iin} {\n}-bring {\iin} car{\tbrM} service{\Mv}={\ii} =and {\nat}-go.up-{\b} ={\kau}\\
			\glt	`He brought his work car and picked me up.' \txrf{130825-6, 13.28} {\emb{130825-6-13-28.mp3}{\spk{}}{\apl}}}\label{ex:130825-6, 13.28}
\end{exe}

In \qf{ex:130926-1, 0.45} the entire nominal phrase
\ve{kapaal desa} `village head' is a loan from Malay \it{kepala desa}.
Nonetheless, the first part is in the M\=/form,
resulting in metathesised \ve{kapaal} from \ve{kapala}.
Furthermore, neither part of this nominal phrase has been phonologically assimilated
with both of the non-native consonants /l/ and /d/ remaining unchanged.\footnote{
		The phonemes /d/ and /l/ are assimilated
		as /r/ in naturalised loans (\srf{sec:LoaConNat}).}

\begin{exe}
	\ex{\glll natun niim on \sf{\tbr{kapaal}} \sf{\tbr{desa}} n-ok ina \sf{staaf}=ein=ee.\\
						natun nima on \sf{kapala} \sf{desa} n-oka ina \sf{staaf}=ein=ee\\
						thousand five {\on} head{\tbrM} village{\U} {\n}-{\ok} {\iin} staff={\ein}={\ee}\\
			\glt	`Five thousand (goes) to the village head and his staff.'
						\xrf{130926-1, 0.45} {\emb{130926-1-00-45.mp3}{\spk{}}{\apl}}}\label{ex:130926-1, 0.45}
\end{exe}

Similarly in \qf{ex:130920-1, 4.20} below the nominal phrase
\ve{baas Indonesia} `Indonesian language'
is a loan from Kupang Malay \it{basa Indonesia}.
Nonetheless the first part of the nominal phrase surfaces in Amarasi
in the expected M\=/form with final consonant-vowel
metathesis of putative underlying \ve{basa}.

\begin{exe}
	\ex{\glll kaah, on reʔ natiʔ =te, siin n-nena =ha Uisneno iin kaibn=ii, n-eki =ha uab, \sf{\tbr{baas}} \sf{\tbr{Indonesia}}. \\
						kaah  on reʔ natiʔ =te sini n-nena =ha Uisneno ini kabin=ii  n-eki =ha uaba \sf{basa} \sf{Indonesia} \\
						{\kaah} like {\reqt} normal ={\te} {\siin} {\n}-hear =only God {\iin} word={\ii} {\n}-use =only speech{\M} language{\tbrM} Indonesia \\
			\glt	`Unlike normal, when they just hear God's word in Indonesian.'
						\txrf{130920-1, 4.20} {\emb{130920-1-04-20.mp3}{\spk{}}{\apl}}}\label{ex:130920-1, 4.20}
\end{exe}

\subsubsection{Loans without M\=/forms}\label{sec:LoaWithoutMfo}
Although many loan words are treated the same as native vocabulary
when they occur in an attributive nominal phrase,
there are some loanwords which are not.
Notably, consonant-final loanwords do not have M\=/forms in attributive phrases.

One example is given in \qf{ex:130907-3, 1.23} below
with the nominal phrase \ve{tukan hau} `carpenter',
in which the first nominal occurs in the U\=/form
rather than expected \ve{\tcb{*}tuuk hau}.
Amarasi \ve{tukan} is borrowed from Malay \it{tukang} `artisan'.
I have an additional five examples of the form \ve{tukan}
as the first nominal in an attributive phrase in my corpus,
three of \ve{tukan hau} as in example \qf{ex:130907-3, 1.23},
and two of \ve{tukan besi} `blacksmith' (cf. Malay \it{tukang besi}).\footnote{
		Another example is the nominal \ve{skora} {\tl} \ve{skoor} `school'.
		There is variation in as to whether the root is \ve{{\rt}skora} from which the M\=/form
		\ve{skoor} is regularly derived, or whether the root is consonant-final \ve{{\rt}skoor}
		for which no M\=/form can be derived (the expected M\=/form would be \ve{\tcb{*}skoo}).
		Such variation is even found in the speech of single speakers.
		This may be a case of borrowing from different sources;
		Dutch \it{school} /sχoːl/ > \ve{skoor} and
		Portuguese \it{escola} /ɛskɔla/ > \ve{skora}.
		(The form \ve{skora} could also be via intermediate Malay which has \it{sekolah} /səkolah/.)
		The verbal equivalent of this nominal normally has the U\=/form
		\ve{na-skora} `(s/he) studies' and the M\=/form \ve{na-skoor}.
		These forms could be borrowing from the Dutch verb \it{scholen} [sχoːlə].}

\begin{exe}
	\ex{\glll \sf{na,} au u-teenb=ii, au ʔ-ak of aiʔ he ʔ-bi \hspace{10mm} skoor tuukn=ii, \tbr{tukan} \tbr{hau}\\
						\sf{na} au u-tenab=ii {au} ʔ-ak of aiʔ he ʔ-bi {} skora tukan=ii tukan hau\\
						well {\au} \qu-think={\ii} {\au} \q-say sure or {\he} {\q}-{\bi} {} school artisan{\Mv}={\ii} artisan wood\\
			\glt	`Well, I thought I would surely be at the artisan school, carpentry.'\\
						\txrf{130907-3, 1.23} {\emb{130907-3-01-23.mp3}{\spk{}}{\apl}}}\label{ex:130907-3, 1.23}
\end{exe}

Despite the fact that consonant-final loan nominals
do not have M\=/forms before attributive modifiers,
they \emph{are} attested with M\=/forms before vowel-initial enclitics.
One example has already been given in \qf{ex:130907-3, 1.23}
in which the form \ve{tuukn=ii} {\la} \ve{tukan} + \ve{=ii} occurs.

This provides evidence that the metathesis before vowel-initial enclitics
is a different kind of metathesis to metathesis in nominal attributive phrases.
In Chapter \ref{ch:PhoMet} I analysed metathesis before vowel-initial enclitics
as an automatic phonologically conditioned process.
This phonological process applies to all words without regard to whether they are loans or not.

Morphological metathesis, on the other hand, has phonotactic
restrictions on the kinds of loans it applies to.
Consonant-final loans do not usually undergo morphological metathesis.
This phonotactic restriction also occurs among verbs in Amarasi.
As discussed in \srf{sec:PhoResMfrSVC}, consonant-final verbs
followed by an attributive modifier usually occur in the U\=/form.