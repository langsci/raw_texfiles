\subsection{Ordinal numbers}\label{sec:OrdNum}
Nominals followed by a cardinal number take U\=/forms in Amarasi,
while there is one set of ordinal numbers which induce M\=/forms on the head noun.
Nonetheless, both kinds of phrases have identical stress patterns.
and this provides some of the most unambiguous evidence
that metathesis before attributive modifiers in Amarasi
cannot be analysed as phonologically conditioned as has been proposed
for both Rotuman (\srf{sec:Rot}) and Leti (\srf{sec:Let}).

\begin{table}[ht]
	\caption{Amarasi numerals}\label{tab:AmaNum}
	\centering\setlength{\tabcolsep}{0.85em}
		\begin{threeparttable}[b]
			\begin{tabular}{llll} \lsptoprule
				No. &Cardinal					&Ordinal\su{†} & Ordinal\su{‡}\\ \midrule
					1 &\ve{=ees\tcb{,} meseʔ}
															&	\ve{meseʔ}	&\ve{noogw=ees}\\
					2 &\ve{nua}					&	\ve{nuaʔ}		&\ve{noo nua-n}\\
					3	&\ve{tenu}				&	\ve{tenuʔ}	&\ve{noo tenu-n}\\
					4	&\ve{haa}					&	\ve{haaʔ}		&\ve{noo haa-n}\\
					5	&\ve{nima}				&	\ve{nimaʔ}	&\ve{noo nima-n}\\
					6	&\ve{nee}					&	\ve{neʔe}		&\ve{noo neʔe-n}\\
					7	&\ve{hitu}				&	\ve{hituʔ}	&\ve{noo hitu-n}\\
					8	&\ve{fanu}				&	\ve{fanuʔ}	&\ve{noo fanu-n}\\
					9	&\ve{seo}					&	\ve{seʔo}		&\ve{noo seo-n}\\
					10&\ve{boʔ=ees}			&	\ve{boʔ}		&\ve{noo boʔ}\\ 
	%			\ve{}	&	\ve{}	& 10 \\
				\lspbottomrule
				\end{tabular}
			\begin{tablenotes}
				\item [†] Used for weekdays and months of the year (take M\=/forms).
				\item [‡] Used for more other purposes (take U\=/forms).
			\end{tablenotes}
		\end{threeparttable}
\end{table}

Amarasi has two sets of ordinal numbers.
One set is used specifically for days of the week and months of the year,
while the other set is used in other instances.
The ordinal numbers used for days of the week and months of the year
are mostly formed from the cardinal
numbers through addition of a glottal stop,
either as a suffix or as an infix and obligatorily occur with M\=/forms.
The general purpose ordinal numbers occur after \ve{noo} and take a suffix \ve{-n}
and occur with U\=/forms.\footnote{
		The form \ve{noo} is probably cognate with the word \ve{noʔo} `leaf' < Proto-Malayo-Polynesian *dahun.
		The suffix \ve{-n} is probably connected with the \tsc{3sg.gen} suffix \ve{-n}.}
The Amarasi cardinal and ordinal numbers are given in \trf{tab:AmaNum}.

The ordinal numbers used for counting days and months
are nominals and thus induce M\=/forms on the preceding nominal.
Four examples of an attributive ordinal number are given in \qf{ex:130920-1, 2.11}
and \qf{ex:120715-2, 0.37} below.
Phrasal stress is indicated in each example with an acute accent.
In both instances phrasal stress falls on the penultimate or final vowel of each intonation group.

\begin{exe}
\let\eachwordone=\textnormal
\let\eachwordtwo=\itshape
		\ex{\gllll [nɛɐn hɐʔ {ɐfi nɐ} \hp{=}t̪ɛː || am fɛrdi ka n-\'{ɔ}kɐ \hp{=}f]\\
							\hp{[}ne\tbr{an} haa-ʔ afi{\gap}naa =te, {} aam Ferdi ka= n-oka =f.\\
							\hp{[}neno{\footnotemark} haa-ʔ afi{\gap}naa =te {} ama Ferdi ka= n-oka =fa\\
							\hp{[}day{\tbrM} four-{\qnum} yesterday ={\te} {} father{\M} Ferdi {\ka}= {\n}-{\ok}{\U} ={\fa}\\
				\glt	\lh{[}`Thursday, yesterday, father Ferdi didn't join (us).'
							\txrf{130920-1, 2.11} {\emb{130920-1-02-11.mp3}{\spk{}}{\apl}}}\label{ex:130920-1, 2.11}
		\ex{\gllll [fʊn hit̪ʊ fʊn fanʊ kɐ \hp{=}t̪ fʊn s\'{ɛ}ʔɐ]\\
							\hp{[}fu\tbr{un} hitu-ʔ, fu\tbr{un} fanu-ʔ kah =t fu\tbr{un} se\<ʔ\>a.\\
							\hp{[}funan hitu-ʔ funan fanu-ʔ kah =te funan se\hspace{-0.5mm}\<ʔ\>\hspace{-0.5mm}o\\
							\hp{[}moon{\tbrM} seven{\U}-{\qnum} moon{\tbrM} eight-{\qnum} {\kaah} ={\te} moon{\tbrM} nine\hspace{-0.5mm}\<\qnum\> \\
				\glt	\lh{[}`July (or) August, if not September.' (\emph{lit.} `seventh moon,
										eighth moon if \hp{[}not ninth moon.')
							\txrf{120715-2, 0.37} {\emb{120715-2-00-37.mp3}{\spk{}}{\apl}}}\label{ex:120715-2, 0.37}
\end{exe}
\footnotetext{
		The rule of Kotos vowel height dissimilation (\srf{sec:HeiDisKor})
		has applied in \qf{ex:130920-1, 2.11},
		yielding metathesised \ve{nean} rather than `expected' \ve{neon}.}

Cardinal numbers do not induce M\=/forms on the nominal they follow.
%(discussed in more detail in \srf{sec:NumPhr}).
Two textual examples of a U\=/form nominal followed by a cardinal numeral are given 
in \qf{ex:130920-1, 3.29} and \qf{ex:130821-1, 1.18} below.
As in examples \qf{ex:130920-1, 2.11} and \qf{ex:120715-2, 0.37} above,
phrasal stress falls on the final or penultimate vowel of the intonation group.

\begin{exe}
\let\eachwordone=\textnormal
\let\eachwordtwo=\itshape
	\ex{\glll	[hɛj mrɛs \hp{=}sin nɛnɔ ħ\'{ɐ̤}]\\
						\hp{[}hai m-rees =siin \tbr{neno} haa.\\
						\hp{[}{\hai} {\m}-read{\M} ={\siin} day{\tbrU} four\\
			\glt	\lh{[}`We read them for four days.'
						\txrf{130920-1, 3.29} {\emb{130920-1-03-29.mp3}{\spk{}}{\apl}}}\label{ex:130920-1, 3.29}
	\ex{\gllll	[t̪ʊɐfɛs namajkɐ̰ nɔk \hp{=}kit̪ fʊnɐn nʊɐ =m \hspace{20mm} \hp{[}ɔf hi ɛs m\'{ɔ}kɐn]\\
						\hp{[}tuaf=ees na-maikaʔ n-ok =kiit \tbr{funan} nua =m {} \hp{[}of hii ees m-oka=n.\\
						\hp{[}tuaf=esa na-maikaʔ n-oka =kiti funan nua =ma {} \hp{[}of hii esa m-oka=n\\
						\hp{[}person={\es} {\na}-stay {\n}-{\ok} ={\kiit} moon{\tbrU} two =and {} \hp{[}later {\hii} {\es} {\m}-{\ok}={\einV}\\
			\glt	\lh{[}`One person is staying with us for two months and later you'll be with \hp{[}the ones with him.'
						\txrf{130821-1, 1.18} {\emb{130821-1-01-18.mp3}{\spk{}}{\apl}}}\label{ex:130821-1, 1.18}
\end{exe}

The examples in \qf{ex:130920-1, 2.11}--\qf{ex:130821-1, 1.18} above
all have very similar stress patterns.
The penultimate or final phonemic syllable of each intonation group bears stress,
and yet M\=/forms occur before ordinal numbers and U\=/forms before cardinal numbers.

As discussed previously in \srf{sec:AltAppPhoMet},
this behaviour is shown even more explicitly,
by the nominal \ve{neno} `day' followed by each of the cardinal and ordinal numbers 1--6,
as given in \trf{tab:AmaNouNum}.
The only phonological difference between each pair of phrases
is metathesis of the final syllable of the nominal \ve{neno} `day',
and (when applicable) the presence of a
glottal stop to form an ordinal number.

\begin{table}[ht]
	\caption{Amarasi nominals and numerals}\label{tab:AmaNouNum}
	\centering
		\begin{threeparttable}[b]
		\begin{tabular}{llll} \lsptoprule
	Underlying	&Phonetic	&	& gloss\\\midrule
	\ve{ne\tbr{no} meseʔ}	&[ˌnɛnɔˈmɛsɛʔ]	&\emb{neno-meseq.mp3}{\spk{}}{\apl}	&`a single day'\su{†}\\
	\ve{ne\tbr{on} meseʔ}	&[ˌnɛ.ɔnˈmɛsɛʔ]	&\emb{neon-meseq.mp3}{\spk{}}{\apl}	&`first day (Monday)'\su{‡}	\\
	\ve{ne\tbr{no} nua}	&[ˌnɛnɔˈnʊ.ɐ]	&\emb{neno-nua.mp3}{\spk{}}{\apl}	&`two days'	\\
	\ve{ne\tbr{on} nua-ʔ}	&[ˌnɛ.ɔˈnːʊ.ɐʔ]	&\emb{neon-nuaq.mp3}{\spk{}}{\apl}	&`second day (Tuesday)'	\\
	\ve{ne\tbr{no} teun}\su{\#}&[ˌnɛnɔˈt̪ɛ.ʊn]	&\emb{neno-teun.mp3}{\spk{}}{\apl}	&`three days' \\
	\ve{ne\tbr{on} tenu-ʔ}	&[ˌnɛ.ɔn̪ˈt̪ɛnʊʔ]	&\emb{neon-tenuq.mp3}{\spk{}}{\apl}	&`third day (Wednesday)'	\\
	\ve{ne\tbr{no} haa}	&[ˌnɛnɔˈhaː]	&\emb{neno-haa.mp3}{\spk{}}{\apl}	&`four days'	\\
	\ve{ne\tbr{on} haa-ʔ}	&[ˌnɛ.ɔnˈhaˑʔ]	&\emb{neon-haaq.mp3}{\spk{}}{\apl}	&`fourth day (Thursday)'	\\
	\ve{ne\tbr{no} niim}	&[ˌnɛnɔˈniˑm]	&\emb{neno-niim.mp3}{\spk{}}{\apl}	&`five days'	\\
	\ve{ne\tbr{on} nima-ʔ}	&[ˌnɛ.ɔˈnːimɐʔ]	&\emb{neon-nimaq.mp3}{\spk{}}{\apl}	&`fifth day (Friday)'	\\
	\ve{ne\tbr{no} nee}	&[ˌnɛnɔˈnɛː]	&\emb{neno-nee.mp3}{\spk{}}{\apl}	&`six days'	\\
	\ve{ne\tbr{on} ne\<ʔ\>e}	&[ˌnɛ.ɔˈnːɛʔɛ]	&\emb{neon-neqe.mp3}{\spk{}}{\apl}	&`sixth day (Saturday)'	\\
		\lspbottomrule
		\end{tabular}
			\begin{tablenotes}
				\item [†]
					The phrase \it{neno meseʔ} has the sense of `a single day'.
					The normal phrase for `one day' would be \ve{neeŋgw=ees},
					from \ve{neno} + \ve{=ees}.
				\item [‡]
					The normal reference for the phrases with ordinal numbers
					is to the days of the week, with \ve{neon meseʔ} `first day'
					being Monday and \ve{neon ne\<ʔ\>e} `sixth day' being Saturday.
					The normal phrase for Sunday is \ve{neno krei} `day + church'.
					The phrase \ve{neon hitu-ʔ} `seventh day' is attested twice
					in the Amarasi Bible translation in Genesis 2.
					This shows that the phrases with ordinal numbers
					are not just completely lexicalised phrases.
				\item [\#] The default form for cardinal numerals is the M\=/form.
			\end{tablenotes}
		\end{threeparttable}
\end{table}

While different prosodic patterns may have contributed
to the diachronic development of Amarasi metathesis,
this analysis is no longer possible for the synchronic data.
Metathesis is a morphological device used
to signal the presence of an attributive modifier.
Syntactic structures for the nominal and numeral phrases
\ve{neno meseʔ} `one day' and \ve{neon meseʔ} `first day (Monday)' from \trf{tab:AmaNouNum}
have already been given in \qf{ex:NenoMeseq} and \qf{ex:NeonMeseq}
at the beginning of this chapter.

%\begin{multicols}{2}
	%\begin{exe}
		%\ex{\gll \brac{NP} ne\tbr{no} \bracr{} \brac{NumP} meseʔ \bracr{}\\
							%{} day{\tbrU} {} {} one{\U} {}\\
				%\glt \lh{\brac{NP}} `one day'}\label{ex:NenoMeseq2}
		%\ex{\gll \brac{NP} ne\tbr{on} meseʔ \bracr{}\\
							%{} day{\tbrM} one{\U} {}\\
				%\glt \lh{\brac{NP}}`first day (i.e. Monday)'}\label{ex:NeonMeseq2}
	%\end{exe}
%\end{multicols}
%
%\begin{multicols}{2}
	%\begin{exe}
		%\ex{\begin{forest} where n children=0{tier=word}{}
			%[NumP,[NP,[\br{N},[N,[\ve{ne\tbr{no}}\\day{\tbrU},label={below:{\hspace{15mm}`one day'}},]]]]
			%[Num,[\ve{meseʔ}\\one{\U}]]]
		%\end{forest}}\label{tr:NenoMeseq2}
		%\ex{\begin{forest} where n children=0{tier=word}{}
			%[NP,[\br{N},[\br{N},[N,[\ve{ne\tbr{on}}\\day{\tbrM},label={below:{\hspace{17mm}`Monday'}},]]]
			%[N,[\ve{meseʔ}\\one{\U}]]]]
		%\end{forest}}\label{tr:NeonMeseq2}
	%\end{exe}
%\end{multicols}

\subsection{Summary}
M\=/forms are used in the nominal phrase in Amarasi for all non-final
nominals below the level of \br{N}.
M\=/forms are a construct form which mark the presence of a dependent modifier.
In \srf{sec:Poss}--\srf{sec:EquCla} below
I discuss nominal structures in
which M\=/forms conditioned by syntax do not occur.
These structures include possession (\srf{sec:Poss}),
modifiers which are not nominals (\srf{sec:OthNomMod})
and equative clauses (\srf{sec:EquCla}).
