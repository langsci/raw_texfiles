\subsection{Proper names}\label{sec:ProNam}
Combinations of two personal names, typically a first name and a family/clan name,
are usually treated as an attributive nominal phrase with the first name in the M\=/form.
One example is given in \qf{ex:130821-1, 6.03} below,
in which the name \ve{Tefaʔ} occurs in the M\=/form before \ve{Unus},
and in the \mbox{U\=/form} without a modifier.

\begin{exe}
	\ex{\begin{xlist}
		\ex{\glll {okeʔ =te} reʔ a-tupa-s reʔ ia n-teek=ee =t n-ak:\\
							{okeʔ =te} reʔ a-tupa-s reʔ ia n-teka=ee =te n-ak \\
							after.that {\req} {\at}-sleep-{\at} {\req} {\ia} {\n}-call={\eeV} ={\te} {\n-\ak} \\
				\glt	`After that the one who is asleep (dead) here they called her:'}
		\ex{\glll bi \tbr{Teef} \tbr{Unus}, aiʔ bi \tbr{Tefaʔ}.\\
							bi Tefaʔ Unus aiʔ bi Tefaʔ\\
							{\BI} Tefa{\Q}{\tbrM} Uunus{\U} or {\BI} Tefa{\Q}{\tbrU}\\
				\glt	`Tefa{\Q} Unus or (just) Tefa{\Q}.'
							\txrf{130821-1, 6.03} {\emb{130821-1-06-03.mp3}{\spk{}}{\apl}}}\label{ex:130821-1, 6.03}
	\end{xlist}}
\end{exe}

In example \qf{ex:130925-1, 2.02} below, the first time the person
is mentioned only his first name is given.
When the speaker clarifies who exactly this \ve{Tinus}
is by supplying a clan name in \qf{ex:130925-1, 2.02b},
the first name occurs in the M\=/form.

\begin{exe}
	\ex{\begin{xlist}
		\ex{\glll	reʔ au u-toon ia =t, naiʔ \tbr{Tinus} a|n-\sf{palaŋ} nua.\\
							reʔ au u-tona ia =te naiʔ Tinus {\a}n-\sf{palaŋ} nua\\
							{\req} {\au} {\qu}-tell {\ia} ={\te} {\naiq} Tinus{\tbrU} {\a\n}-crossbeam two\\
				\glt	`I told (him) this. Tinus trapped two (cows).'
							\txrf{130925-1, 2.02} {\emb{130925-1-02-02-02-04.mp3}{\spk{}}{\apl}}}\label{ex:130925-1, 2.02}
		\ex{\glll \tbr{Tiun} \tbr{Nuban} n-\sf{palaŋ} nua.\\
							Tinus Nuban n-\sf{palaŋ} nua\\
							Tinus{\tbrM} Nuban{\U} {\n}-crossbeam two\\
				\glt	`Tinus Nuban trapped two.' \txrf{2.04}}\label{ex:130925-1, 2.02b}
	\end{xlist}}
\end{exe}

A similar example is given in \qf{ex:130907-5} below,
in which the name \ve{Daʔi} `David' occurs in the U\=/form
when on its own, but in the M\=/form when the family name of the referent follows.

\begin{exe}
	\ex{\begin{xlist}
		\ex{\glll	n-ok naiʔ Manase, naiʔ \tbr{Daʔi}\\
							n-oka naiʔ Manase  naiʔ 			Daʔi\\
							{\n}-{\ok} {\naiq} Manasseh {\naiq} David{\tbrU}\\
				\glt	`With Manasseh, (and) David,'
							\txrf{130907-5, 0.21} {\emb{130907-5-00-21-00-29.mp3}{\spk{}}{\apl}}}\label{ex:130907-5, 0.21}
		\ex{\glll	\tbr{Daiʔ} \tbr{Saebeisʔ}=ii n-ok naiʔ Manase Bani.\\
							Daʔi Saebesiʔ=ii n-oka naiʔ Manase Bani\\
							David{\tbrM} Saebesi{\Q}{\Mv}={\ii} {\n}-{\ok} {\naiq} Manasseh Bani{\U}\\
				\glt	`David Saebesi{\Q} with Manasseh Bani.'
							\txrf{0.29}}\label{ex:130907-5, 0.29}
	\end{xlist}}\label{ex:130907-5}
\end{exe}

Note, however, that in example \qf{ex:130907-5, 0.29}
the name \ve{Manase} `Manasseh',
does \emph{not} occur in the M\=/form when the family name \ve{Bani} follows.
A search of my corpus reveals many other instances in which a first name followed
by a family name does not occur in the expected M\=/form.
A selection of other examples include: \ve{Paulus Oraʔ}, \ve{Harun Bani} and \ve{Saul Bani}.
In most such instances, the first name is a non-nativised Biblical name.\footnote{
		While the name \ve{Daʔi} in \qf{ex:130907-5, 0.21} is Biblical, it \emph{is} (semi-)nativised.
		The form \emph{<Da{\Q}i>} /daʔi/ is associated with Timor
		and has its origins on Rote island.
		It is perceived by Amarasi speakers to be a Timorese name.
		The Indonesian (but non-Timorese) form of the name \emph{David} is \emph{Daud}.}