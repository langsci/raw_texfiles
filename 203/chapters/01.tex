\chapter{Ideologies of the acrolect and English in Jamaica}\label{ch:1}
\section{The (Jamaican) acrolect}\label{sec:1.1}

The Jamaican language situation was described very early on by linguists such as \citet{LePage1960} and \citet{DeCamp1961,DeCamp1971} as a \isi{linguistic continuum}, formed primarily from the historical sociolinguistic contact of speakers of various \isi{Niger-Congo} languages and several dialects of English.  Similar continua have also been identified in \isi{Guyana}, \isi{Belize}, \isi{Trinidad} and \isi{Barbados} (see the discussion in \citealt[233--236]{Winford1997}).  The polar varieties of this theoretical construct, the continuum, are basilectal \isi{Creole} and “standard” English (discussed below), which have been treated as separate linguistic systems by some linguists (for example \citealt{Bailey1971} (Jamaica); \citealt{Devonish1978} (\isi{Guyana}), and more generally \citealt{Alleyne1980a} and \citealt{Winford1997}) and which are linked by a seamless range of intermediate lects identified as the \isi{mesolect}.  

These intermediate varieties have been explained in a number of ways.  They have been attributed to: \isi{decreolization} (\citealt{Bickerton1973}; \citealt{DeCamp1971}), due to the targeting of English by \isi{Creole} speakers with varying levels of access to the prestige or high variety; or to \isi{basilectalization} brought about by increasing numbers of West Africans arriving over time and acquiring approximations of approximations of English (\citealt{Mufwene1996,Mufwene2001})\footnote{\citet{Chaudenson2001} describes a similar process of basilectalization for French-lexicon Creoles.}; or to the social stratification of plantation communities from the earliest stages of \isi{language contact} \citep{Alleyne1980a}.  This continuum is depicted as 

\begin{quote}
... a continuous spectrum of speech varieties whose extremes are mutually unintelligible, but which also includes all possible intermediate varieties \citep[28]{DeCamp1971}, ranging from the speech of the most backward peasant or labourer to that of the well-educated urban professional (\citealt[82]{DeCamp1961}).
\end{quote}

It is to this sociolinguistic discussion that \citet[15--16]{Stewart1965} added the terms \textit{acrolect} and \textit{basilect}, (the label \textit{mesolect} came later).  In his description of “Washington Negro speech” he wrote,

\begin{quote}
I will refer to this topmost dialect in the local sociolinguistic hierarchy as \textit{acrolect} (from \textit{acro}- “apex” plus -\textit{lect} as in dialect).  In most cases what is meant by “\isi{Standard English}” is either \isi{acrolect} or something close to it.  At the other extreme is a kind of speech which I refer to hereafter as \textit{basilect} (from \textit{basi}- “bottom”). (...) In between \isi{basilect} and \isi{acrolect}, there are a number of other dialect strata, and it is in this middle range that the majority of adult Washington Negroes probably belongs.  
\end{quote}

\textit{Acrolect}, the focus of this dissertation, is the name for the upper end of the continuum and, while in many cases the term is not explicitly used, it has been defined in the literature in a number of clearly related ways.  General descriptions of \isi{acrolect} speak of,  

\begin{itemize}
\item the “topmost dialect in the local sociolinguistic hierarchy” (\citealt[15]{Stewart1965});   
\item the stereotype of the educated standard, focussed by institutions of education, the media, and white collar employment (\citealt[34--35]{LePage1988});
\item a “high, prestigious language or variety of a language” in \textit{any} speech community (\citealt[xvii]{McArthur1998}).
\end{itemize}

Acrolect here is essentially a community of speakers’ idea of the speech patterns of those positioned at the top and centre of their social space.  It is in particular speakers, in those perceived to be of relatively high social status, that we initially locate the \isi{acrolect} in a community and idealise their ways of speaking as “good” or necessary.  This then makes the \isi{acrolect} coterminous with “the standard” for members of communities with that type of metalinguistic labelling.  This view of the \isi{acrolect} is also reflected in many of the definitions applied to the Caribbean generally and Jamaica specifically:        

\begin{itemize}
\item the educated counterpart of the lexifier in the \isi{Creole} speech community (\citealt[xi (n1)]{Chaudenson2001});
\item the English of educated nationals of the Caribbean used in formal social contexts, bound to a common core of morphology and syntax shared with “Internationally Accepted English” (\citealt[lvi]{Allsopp1996});
\item the speech of those with high education, typically urban residence and the higher socio-economic status that comes with certain (non-manual) occupations (\citealt{DeCamp1961,DeCamp1971} who used the speech of Philip Sherlock and Norman Manley as examples).
\end{itemize}

Central to these views of the \isi{acrolect} is the speaker, as it is a particular set of speaker variables that are typically associated with and used to identify the \isi{acrolect} in specific communities.  

In another approach, the \isi{acrolect} has been defined by its structures or in relation to the structures of other varieties that coexist with it in the community:

\begin{itemize}
\item the \isi{local standard}, similar to other standard Englishes except in “a few phonological details and a handful of lexical items” (\citealt[24]{Bickerton1975});
\item the variety or varieties used in Jamaica that are structurally farthest removed from the \isi{Creole}, i.e.\ \isi{basilect} (\citealt[73]{Akers1981}; \citealt[342]{Bailey1971} for example). 
\end{itemize}

Finally, the \isi{acrolect} is seen as the outcome of \isi{Creole} speakers’ acquisition of an idealised superstrate that is (and was) the target in \isi{language contact}:    

\begin{itemize}
\item a non-native version, derived from spontaneous or guided learning, of the standard language (\citealt[67]{Escure1997}).    
\end{itemize}

We can abstract from the above that the \isi{acrolect} has been defined, 

\begin{enumerate}
\item in terms of its structure, and this includes not only its relationship to English, say, but also its relationship to the \isi{Creole} \isi{basilect};

\item as a consequence of \isi{language acquisition}, as it is suggested that the \isi{acrolect} is the learned English of Caribbean peoples who are not themselves \isi{vernacular} speakers of English;

\item sociolinguistically, in relation to its speakers and the situational contexts in which these speakers use language, i.e. the formal, public speech of those with high social status;

\item ideologically, an idea of English that is legitimised by institutions that construct norms by their own expectations and practice. 
 \end{enumerate}

\section{Sense 1: The acrolect as defined by structure}\label{sec:1.2}
The focus of research in Caribbean continuum situations has, understandably, been the \isi{Creole} \isi{basilect}, the less well-known variety to linguists.  This has generally meant that the speaker’s central role in the characterization of lects along the continuum has been overshadowed by the search for what \citet[341]{Bailey1971} called “neat, clearly defined patterns of behaviour” and the analysis of idealised basilectal linguistic systems.  This process of erasure or selective disattention to “often unruly forms of variation” \citep[23]{Kroskrity2000} is no doubt driven by the requirements of linguistic description (\citealt[33]{Milroy1999}); but it is also, ironically, a reflection of the way speakers themselves construct language and language use, particularly in the process of social differentiation (\citealt[38]{IrvineGal2000}).    Jamaicans, for example, tend to see the local language situation in terms of what we Jamaicans call Patwa and English, and to produce speech in a socio-functional relationship that would accurately be described as diglossic (\citealt[8]{Akers1981}; \citealt{Winford1985}, \citealt[256]{DevonishHarry2004}).  This has had consequences for the study of the continuum generally, but in particular the \isi{acrolect}, because the underlying system that generates acrolectal speech is thought by both the speaker and the linguist to be no different from that of other Englishes.  Therefore, much of the literature assumes, but has not verified, the features associated with the \isi{acrolect} (\citealt[73]{Akers1981}; \citealt[342]{Bailey1971}; \citealt[209\,n6]{Mufwene2001}; \citealt{Patrick1999}, \citealt[11]{Patrick2000}; \citealt{Winford1991}) even when making passing reference to “a \isi{local standard}”.  With few exceptions (see below), this \isi{local standard} has not itself been the focus of research as most linguists presumably already know the structural properties of English and are seeking to uncover those of the \isi{Creole}.  In this respect, these studies are following in an earlier ideological tradition, described by \citet{IrvineGal2000}, which in practice meant that

\begin{quote}
	“once a variety has been declared to belong to the ‘same’ language as another, already-described variety, there was no reason to investigate it...” (57).
\end{quote}

Additionally, and indeed \textit{because} this “\isi{local standard}” is putatively not so different, metropolitan \isi{Standard English} (\isi{MSE} hereafter) has been used not only to compare but to typologize structures found along the continuum\footnote{\citet[33 in particular]{Milroy1999} discusses the effect of a standard language ideology on the investigation and description of “less well-known languages”.}, particularly the morphology and syntax but also the phonology \citep[575]{Wells1982c}.  As Alleyne pointed out over two decades ago, \isi{Creole} languages are compared with Standard Written English of the most formal registers (\citeyear[16]{Alleyne1980a}).

A few examples can be used to illustrate the approach.  Firstly, the negative past particle \textit{neva,} in the sentence \textit{Jan neva tiif di moni} ‘John did not steal the money’, has been analysed syntactically as \isi{Creole} \citep{Bailey1966} or as non-acrolectal (\citealt[141]{Escure2000}; \citealt[217]{Schneider1998}) even though it is used as a negative preterite marker throughout the \isi{Jamaican continuum} (as in \textit{John never stole the money} or in the following: Speaker A -- \textit{Give Damian back his book}; Speaker B -- \textit{What? I never borrowed it}).  This type of analysis is not limited to contact situations like Jamaica.  As \citet[133]{Cheshire1999} points out, the construction \textit{never} + V-ed when used like this (in \isi{MSE}) is “...frequently label[led] (...) as ‘non-standard’, despite the fact that the contexts occur in educated speech and writing”.

Secondly, \citet{Meade2001}, in his study of phonological acquisition in Jamaican children, uses \isi{MSE} phonology to determine what he labels acrolectal phonological features.  Some of his Jamaican children are said to be acquiring at least one feature either late or at the mesolectal level (162) – in this case the \isi{interdental fricative} – because they continue to vary them with dental stops up to ages 4-6 in words that would have [ð] or [θ] in \isi{MSE}.  Both explanations of his results are based on an idea of \isi{acrolect} that reflects both an external model as well as an idealization of that model of English.  One must therefore ask what children in Jamaica are likely to actually hear around them from local models of English, and should that English be treated as “falling short”, i.e. non-acrolectal, when it does not resemble the external ideal.  Meade does compare the children’s output with that of their caregivers in his study and concludes that frequency of use by caregivers of a particular feature is crucial to the rate of acquisition of the children (161).  However, Meade treats interdental fricatives and their variants as if they pattern as they do in \isi{MSE}, i.e. he does not distinguish voiced from voiceless when describing the significance of the feature to Jamaican English which, as he puts it, is “ranked very high on the \isi{implicational scale} of the \isi{Creole} continuum” (162).  It may well be that the local use of interdental fricatives and their variants differs from that of \isi{MSE} (see \sectref{sec:3.6}).

Finally, \citet[8]{Akers1981} argues that the speech of British, American and \isi{Canadian} tourists provides \textit{one} prestige model for Jamaicans.  This suggests that for Akers these three phonetically different varieties are idealized as one ``English", notwithstanding the distinctiveness the speakers of the varieties seem to claim when labelling even their Standards.  Here, arguably, metropolitan varieties of spoken English are also being characterized as “Standard Written English of the most formal registers”, notwithstanding the problems of doing so at the phonological level.  

\citegen[73]{Akers1981} description of the \isi{acrolect} as a rhotic variety was based largely on his analysis of the \isi{Creole} \isi{basilect} as non-rhotic, and not on the actual production of acrolectal speakers.  His pool of informants consisted of 10 relatively young people (the oldest was 33 years old), only one of whom could have been characterized as having the kind of background typically associated with the acrolectal speaker, a female college graduate.  As such, his conception of the \isi{acrolect} is a sort of \textit{anti}-\isi{Creole} that stands in opposition to the \isi{basilect} structurally.  \citegen{Patrick2000} is too.  While he does not explicitly characterize the \isi{acrolect} in his analysis of variation in Veeton, he also proceeds like Akers, Meade and Wells when he states that,

\begin{quote} ... variables are here counted in their presence or absence so that maximum usage approximates to the acrolectal standard, i.e. absence of (KYA) [kja {\textasciitilde} k\textsuperscript{h}a] ... testing the assumption that higher status speakers favour acrolectal speech more often. \citep[11]{Patrick2000}\end{quote}

What seems to have emerged is a kind of circularity in the definition of the polar varieties on the (Jamaican) continuum that, typically, takes this form: a) Since the \isi{acrolect} is not unlike other standard Englishes, phonologically as in [k\textsuperscript{h}] (in a word like \textit{cat}) or morpho-syntactically as in \textit{did not take}; then b) the \isi{basilect} reflects structures that are maximally divergent from the \isi{acrolect} – [kj] or \textit{no ben} \textit{{\textasciitilde} neva tek}; as such, c) acrolectal speech is maximally non-basilectal and can therefore be expected to produce the forms [k\textsuperscript{h}] or \textit{did not} \textit{take}.  Moreover, this circularity extends to the identification of so-called basilectal speakers, so that samples of speech collected in the field are rejected when they are “too English” and therefore “noncreole” (\citealt[74]{Escure1997} commenting on the general practice of creolists), because the linguist already knows what is \textit{supposed} to be basilectal.

One way of accommodating the above conception of the \isi{acrolect}, is, on one hand, the catch-all term “upper-mesolectal” (\citealt[161--162]{Bickerton1975} for example) for those forms that do not pattern like those of \isi{MSE} but have features found elsewhere in the continuum.  \citet[17]{Patrick2002} tells us that,

\begin{quote}Speaky-spoky is associated with lower mesolectal and basilectal speakers because of their distance from the standard; and it is associated with ‘mistakes’, failed attempts to speak a metropolitan \isi{prestige variety} that is not native to these speakers. ({\itshape Acrolectal and upper mesolectal speakers are viewed as using Standard Jamaican English appropriately, and not making linguistic errors in its use}, despite the variation in their spoken and written speech) [My emphasis].\end{quote}

In this description “the standard” is \isi{MSE}, the prestige target variety, and not Standard Jamaican English (\isi{SJE} hereafter), the variety used appropriately by both acrolectal \textit{and} upper mesolectal speakers.  It is therefore not clear why this distinction is being made, though this could be interpreted as defining acrolectal and upper mesolectal as labels of presumed social group.  

The irony of this conception of the \isi{acrolect} is that Jamaican speech is being analysed in terms of a variety that is not actually spoken in the speech community and the structures used by Jamaicans are being defined in terms of norms external to the community of speakers.  Moreover, this speech is being analysed more in relation to written metropolitan structures than to those used in spoken \isi{MSE}.  When \citet[89]{LallaDCosta1990} write, for example, that in the 18\textsuperscript{th} century “basilectal features occur in the usage of all classes sampled, though to varying extents”, they beg the question of why a feature is labelled “basilectal” if it is also found elsewhere in the continuum of varieties.  There would seem to be no \textit{Jamaica}-based structural criteria for singling out as particularly ``basilectal" features attested to in all lects.

\section{Sense 2: The acrolect as the outcome of language acquisition}%1.3
The \isi{acrolect} has also been conceived of as the product of \isi{decreolization} \citep{Escure1997}, the outcome of (post) \isi{Creole} speakers’ acquisition of English as a result of access to education or increased exposure to an external linguistic model (in \isi{Belize}, British or American \isi{Standard English}) available in the society.  Escure takes issue with the traditional notion of \isi{decreolization} only because it suggests \textit{replacement of}, rather than \textit{addition to}, \isi{Creole} varieties.  I will use her analysis of \isi{acrolect}, as it is explicitly discussed by her and includes many of the issues that are relevant to the discussion in this book.  

\citet[67--68]{Escure1997} conceives the \isi{acrolect} to be:

\begin{quote}A non-native version of the standard language, which is acquired through spontaneous or guided learning, functions in formal contexts, and extends its speaker’s repertoire without necessarily leading to loss of the speaker’s \isi{vernacular}.  The \isi{acrolect} is not necessarily a dominant or prestigious dialect, although it is usually associated with education.  The definition of the term \textit{acrolect} implies that there is an available standard that provides a linguistic model instrumental in the formation of acrolects, even in the absence of any guided learning.\end{quote}

She goes on to add that the label “English” is:

\begin{quote}extremely confusing to Belizeans, primarily because English is an external standard, never physically present, although it is officially proclaimed to be the language of \isi{Belize}.  English is not spoken in \isi{Belize} by any particular group, excepting a handful of immigrants (...) The only direct access to native renderings of a standard English variety appears to be through radio programs, \isi{BBC} news and American Evangelical broadcasts.  \isi{Standard English}, then, has practically no spoken presence within the country of \isi{Belize}….\end{quote}

The above quotes suggest the following:

\begin{enumerate}[label=\alph*)]
\item Acrolect and standard are not the same – the former is \isi{non-native English}, the result of speakers targeting the latter.

\item Acrolect is the formal, educated \isi{Belizean} variety of English (EBE hereafter), but it is not \isi{Standard English} which is only spoken by immigrants and on foreign radio.

\item What distinguishes \isi{Standard English} and the \isi{acrolect} is the issue of being a \isi{native speaker} and, more particularly, the phonology.  In the absence of \isi{MSE} phonological patterns then no \isi{Belizean} is a \isi{native speaker} of (standard) English, though they may write it.  Elsewhere in the book (\citeyear[66]{Escure1997}) Escure offers a definition of standard that incorporates the ideas of educated and prestige varieties and says that the term English clearly denotes for Belizeans a local \isi{standard variety}, because they are not motivated to learn the remote (\isi{MSE}) standard (\citeyear[73]{Escure1997}).   
\end{enumerate}

As in the discussion above, of Sense 1, the properties of \isi{MSE} are used as reference for defining \isi{Standard English} in \isi{Belize}.  Escure, however, goes further by saying that no one who is \isi{Belizean} speaks \isi{Standard English} because what they produce, i.e. \isi{acrolect}, differs from \isi{MSE}:

\begin{quote} ...acrolects develop independently because the officially prescribed standard is not present in the country; there is simply no linguistic exposure to a consistent British or American standard of English, either in daily activities or in school contexts \citep[73]{Escure1997}.
\end{quote}

Nevertheless, Escure tells us that for Belizeans, what she calls the \isi{acrolect} \textit{is} standard English as the British or American standards are sociolinguistically remote, and therefore unimportant in the speech community.  Moreover, Belizeans feel that they should speak standard English as it is the \isi{official language} and indexes education (see also \citealt{Young1973}), and they therefore target what \textit{they} consider it to be, there being no spoken \isi{MSE} presence in either the school system or the society generally.  Escure also suggests that this local EBE is not the superstrate (\citeyear[74]{Escure1997}), even as it is the variety that lives in the community of speakers and is the variety of English Belizeans (will) target and\slash or acquire.  In this respect Escure is not unlike Patrick (quoted above), as both suggest that the target (for Belizeans and Jamaicans) is metropolitan English.  

In her discussion of acrolects as innovations, Escure questions why, in light of the goals and choices a community or individual might make in the acquisition of a language, “acrolectal speakers would choose, albeit unconsciously, to produce varieties distinct from the official standard” (\citeyear[66]{Escure1997}).  But EBE is the standard, the spoken variety of educated Belizeans that has its own phonological, lexical and idiomatic forms.  

Such an exonormative approach to defining the standard in Caribbean territories like \isi{Belize} and Jamaica is curious, given the history of English in the Caribbean.  It speaks as much to a view of what is “native” English as it does to who is to be called a \isi{native speaker} of English.  It is never suggested, for example, that \isi{General American} represents the \isi{acrolect}, imperfect targeting of British English by speakers because of ties of history, even though much of the U.S. population historically were Europeans acquiring\slash learning a foreign language (\citealt{Romaine2001} \textit{passim}).  Certainly, the percentage of L\textsubscript{1} English speakers was\slash is greater in the U.S. than in Jamaica or \isi{Belize}.  However, that suggests that the characterization of a variety as native is a quantitative matter – with an arbitrary decision made as to numbers of L\textsubscript{1} speakers – rather than an issue of acquisition \textit{per} se.    

Escure’s idea of \isi{acrolect} necessarily takes into account the historical social context of \isi{language contact}, but seems to ignore the existence of \isi{vernacular} speakers of English in continuum situations like \isi{Belize}.  \citet[148--151]{Judd1998} suggests that a \isi{vernacular} English speaking population in \isi{Belize} existed since at least the 1900's.  

And in Jamaica, \citet[98]{LallaDCosta1990} carefully show

\begin{quote}the existence, from the eighteenth century, of extensive variation (among speakers and within individual usage) in features of basilectal and acrolectal models of Jamaican speech [and that] (...) the data also confirm considerable mixing of such features.\end{quote} 

The anonymous author of \textit{Marly,} \textit{A} \textit{planter’s} \textit{life in} \textit{Jamaica}, (cited in \citealt[44]{LallaDCosta1989}) seems to make a distinction, as does \isi{Lady Nugent} (writing between 1801 and 1805), between white Creoles who have been educated in England and those wholly “educated” in Jamaica.\footnote{\citet[90]{LallaDCosta1990}, citing \citet{Long1774}, state that most \isi{Creole} white women were home tutored if educated.  This home tutoring was in many cases carried out by British governesses, though the average governess was “an impoverished and ill-educated female down on her luck” (\citealt[83]{Brereton1995}, citing \citealt{Carmichael1833}).}   The latter description applied mostly, but not exclusively, to women.  These reports, by speakers of metropolitan varieties, indicate use of (a local variety of) English, with occasional “lapses” into \isi{Creole} in “moments of excitement” or when “not on guard”.  The exposure to \isi{MSE} for some Jamaicans does not negate the \isi{vernacular} status of English in Jamaica.  It only suggests the nature of the variation to be found in Jamaican English.  As \citet{LallaDCosta1990} conclude from \citegen[90--91]{Long1774} description of late 18\textsuperscript{th} century Jamaica, 

\begin{quote}of the landowning upper class alone (...) we see (1) a British-educated creole élite speaking \isi{RP}; (2) others of the same socioeconomic class speaking a Jamaican version of \isi{RP}; (3) others, less educated and less exposed to urban influence, speaking \isi{RP} “with much difficulty” and creole with ease; and (4) monolingual, rural creole speakers, who were usually women and brought up on plantations. (...) Among blacks (...) (3) creole blacks (“the better sort”) acquiring some \isi{RP} and altering their JC accordingly; (4) locally educated free blacks and coloureds using the Jamaican form of \isi{RP} in schools such as Francis Williams’s; and (5) British educated blacks and coloureds, such as Francis Williams himself, speaking \isi{RP}.\end{quote}

\noindent Moreover, as Michele \citet[14]{Stewart2002} notes,

\begin{quote} many [free coloured men] attended Wolmer’s, the only secondary school in \isi{Kingston}, alongside Jews, and some were sent to public schools ... By 1788, free coloureds comprised 12.4\% of the population.  A few of the English educated men had been admitted to the highest ranks of society by the turn of the century.\footnote{While free coloured men are specifically mentioned here, publications like Mary Seacole’s autobiography, \textit{Wondeful Adventures of Mrs. Seacole in Many Lands}, refers to a class of what she called “\isi{Creole}” and “yellow” women who were also clearly English speakers \citep{Busby1992}.}\end{quote}

In addition \citet{Bryan1996} describes the emergence of the black middle class in 19\textsuperscript{th} century Jamaica, whose institutions included the Jamaica Union of Teachers, the Artisan’s Union and the Advocate newspaper.  

The \isi{acrolect} then, as a macrosociolinguistic phenomenon, cannot be learned English alone, as not only were local norms of English identified some two centuries ago, displaying the expected variation that factors such as \isi{gender}, ethnicity and education would influence, but there was also an idea of Jamaicanness, which for some excluded English born persons and the ascriptive label Englishman \citep[34]{Patterson1973}.  Nor can the model for those acquiring English have necessarily been exclusively the metropolitan variety. For most colonies then, contact with the metropole would have been limited to the élite few.  

I argue that Jamaican English is not \isi{non-native English}, as defined by \citet[31]{Kachru1982} (cf. \citealt[270]{Moag1982}).  And I have difficulties with the tacit view of the researchers described above (discussed in \sectref{sec:1.1}--\sectref{sec:1.2}) who use an external model to identify the \isi{acrolect}.  The term “non-native” is itself problematic, as it is not at all clear what it means.  It is applied to the English of SE Asia, India, the Caribbean and Africa – all of which have different historical contexts of development.  It is not, however, applied to the English of North America, Australia or New Zealand.  This suggests that ethnicity may be one of the considerations in categorizing regional varieties of English.  The distinction seems to be part of what \citet[107]{Mufwene2001} calls the “disfranchising of particular varieties as illegitimate offspring”.  English in the Caribbean is not really an external colonial language grafted on to indigenous, living languages\footnote{I am aware of the indigenous languages of places like \isi{Belize} and \isi{Guyana}.  I am therefore referring to the general context of \isi{language contact} between West Africans and Europeans.} as it is in Africa or Asia, even if it does not have the continuity that is argued for, say, North America (for example by \citealt{Fisher2001}, but cf. \citealt{Görlach1987}, \citealt{Mufwene2001}).  Jamaica was a settlement colony, its primary reliance on the sugar plantations producing a non-European majority.  And if non-native means an “institutionalized variety” (\citealt[38--40]{Kachru1982}), or a “localized form” \citep[24]{Strevens1982}, then English in Jamaica, the \isi{acrolect}, has nearly always had multiple status for different groups of speakers.  It has, however, been the \textit{vernacular} of a section of the population for over three centuries, and can only be called a learned variety for some, maybe even most, speakers.

The guided learning of English in the school system, which would serve to focus and legitimize an educated norm, has, for most Jamaicans with access, been “localized” (Moag Ibid.,278) since the mid 19\textsuperscript{th} century, if not before.  Bryan points out that,

\begin{quote}As far as the racial factor is concerned, most of the elementary school teachers were black and coloured ... Teachers clearly regarded themselves as an upwardly mobile section of the black population (\citealt[287--288]{Bryan1996}).\end{quote}

And \citegen[317]{Wilmot2002} description of early local government politics in Jamaica notes that

\begin{quote}In order to provide more prestige for this public meeting [in 1854], individual black and coloured teachers from the Beckford’s Free School in Spanish Town and the Wolmer’s School in \isi{Kingston}, respectively, addressed the gathering.\end{quote}

For high schools, a “limited number of elementary school children [were] given the opportunity to attend” (\citealt[209]{Miller1989}, citing B.\,H.\,Easter, the Director of Education, Jamaica 1946).  

These high schools, some of which had been administered by British (trained) teachers, were almost completely localized and feminized by the last decade of the \isi{colonial period}, the 1950's (\citealt{Miller1989}).  This produced what \citet[440]{LePage1968} called the gap between “what is supposed to be happening in the schools and what is actually happening”.

Local educated norms of English were identified in Jamaica at least a century ago, if not before.  The \isi{acrolect} was then being added to or altered, not created, as these norms became generalized for more educated Jamaicans, as fewer and fewer people sent their children to school in England.  The speakers of this Jamaican English would have increased the range of stylistic and social variation in the \isi{acrolect} and perhaps generated different models of acceptable English.  The \isi{acrolect} as an aspect of a linguistic and sociolinguistic continuum, however, must have existed and been contemporary historically with a creole \isi{basilect}.  

The issue here, then, is not the theoretical validity of \isi{decreolization}.  It is that without a clear idea of what the phonology or syntax of the \isi{acrolect} is, it becomes difficult to discuss what it is speakers are targeting when they acquire or learn English.  For example, \citet[275]{Rickford1987} gives instances of the speech of a \isi{Guyanese} teacher and of a barrister delivering his summation in a \isi{Guyanese} court.  In both cases he identifies aspects of the phonology of these speakers that he labels “non-standard” – [d {\textasciitilde} ð] as in ‘them’, [fʌŋ] ‘found’, non \isi{retroflexion} in words like \textit{culture} or \textit{render}.  And as Rickford goes on to caution, “their classification as “non-standard” in the \isi{Guyanese} context is open to question”.  I suggest that \isi{Guyanese} speakers are more likely to target these local features than \isi{RP} or GenAm. variants. 

Discussions of the \isi{acrolect}, in both Sense 1 and 2, represent English as if it was and is a thing apart, foreign to places like Jamaica and “a more or less well-defined and discrete” layer that has little relation to the continuum \citep[241]{Winford1997}.  If English is for its speakers locally defined, then continued analysis of that English in terms of \isi{MSE} will result in paradoxical conclusions.  Some of these conclusions reflect what \citet[50]{Kachru1982} calls “linguistic schizophrenia ... [an inability] to decide whether to accept a mythical non-native model [\isi{MSE}], or to recognize the local functioning model [\isi{acrolect}] instead”. 

\section{Sense 3: The acrolect as defined by its speakers}\label{sec:1.4}%1.4

A smaller set of work on the (Jamaican) continuum has looked at the \isi{acrolect}, in terms of its speakers and the contexts in which they use language.  The speaker variables that are generally used are \isi{socio-economic class}, residence and level of education, particularly the latter. DeCamp’s original characterization of the \isi{acrolect}, for example, was the speech of  “the well-educated urban professional” (\citeyear[82]{DeCamp1961}), although the \isi{implicational scales} he eventually produced were a structurally defined linear spectrum of features.  \citet{Beckford-Wassink1999a} also uses education, urban residence and social class to locate acrolect-dominant speakers.  The situational context of use is generally described as formal, as a distinction is made between \isi{formal speech} and the production in relaxed, informal interactions.

\citegen[lvi]{Allsopp1996} definition of Caribbean \isi{Standard English} in the introduction to the \textit{Dictionary of Caribbean English Usage} is typical, focussing on the two variables of education and formality - “the literate English of educated nationals of Caribbean territories and their spoken English such as is considered natural in formal social contexts”.

\citet{Thaxter1977}, in his study of what he calls Standard Jamaican English, defines it as, “the kind of language which most educated Jamaicans speak on formal occasions in business transactions, legal affairs, in educational institutions and in public life” (2).  \citet[48,182]{Miller1987} describes Acrolectal Jamaican English (\isi{AJE} hereafter) as “the actual spoken variety [of English] which the educated élite uses in formal circumstances”.

The focus on education may be explained in a number of ways, although, as \citet[179]{Pollard1998} points out, \isi{SJE} is the “first language of the man at the top of the social ladder”.  Firstly, it is the case that many, if not most, Jamaicans learn what Allsopp calls Internationally Acceptable English (\isi{IAE} hereafter) in school and English has been a compulsory exam always required by the school system.  Secondly, the school system is one of the means of legitimising and standardizing norms (\citealt[65]{Lippi-Green1997}).  It is in the classroom that, according to \citet[61--62]{Bourdieu1991}, “...mastery of the legitimate language may be acquired through familiarization, that is, by more or less prolonged exposure to the legitimate language...”.

And of course, education as a defining characteristic of “standard” speech is not peculiar to the \isi{Jamaican social context} and therefore does not entail external origin of the standard.  \citet[119--137]{McArthur1998} cites 52 examples of definitions of \isi{Standard English} from a variety of sources, and of these 18 mention educatedness of speaker as the important criterion.  

The school system then links a sociolinguistic notion of \isi{acrolect}, the speech of a social group called “the educated”, with an ideological one, the construct called “\isi{Standard English}” legitimated by institutions like itself.  By analysing the product of the school system, the researcher can identify the individuals in the speech community who have best adopted the sanctioned linguistic forms and been rewarded for doing so.  And this identification of speakers, using language in formal contexts, is important so as to minimize assumptions about the forms to be found in the \isi{acrolect}.  

The other important variable in such studies is formality.  Thaxter’s study (\citeyear{Thaxter1977}) sees formality as the style used in specific kinds of physical and psychological contexts (after \citealt[55]{Hymes1974}).  He recorded students of a Teacher’s College during classroom debates, speech training exercises, election speeches and votes of thanks.  F. Miller conducted interviews designed to elicit certain self-conscious styles of speech (after \citealt[79]{Labov1972}).  And the \isi{Guyanese} data presented by \citet[275--276]{Rickford1987}, discussed in the previous section, also illustrates speech in certain formal settings – the courtroom and the classroom – as produced by educated informants.  

In all these studies the assumption is that the informant’s attention to speech, because of where and when language is being used, is likely to produce greater use of prestige variants and the most appropriate variety for the context (\citealt{Labov1966,Labov1972}; \citealt{Wolfson1976}).  This is not unrelated to the issue of education and the idea of a legitimised variety, which is, in part, the focussing of a set of norms (\citealt{LePage1975}) and the reduction of variation in the language that is to be used in such circumstances.   

In his study of \isi{SJE}, Thaxter looked at productions by informants as well as their evaluations of speech along a series of dimensions that include perceptions of acceptability, suitability for classroom use, and preferences of identity.  Among the phonological variables he examined were –  [ð/θ {\textasciitilde} d/t], [ɪŋ {\textasciitilde} ɪn], CC\#\# {\textasciitilde} C\#\#, [ɔ {\textasciitilde} a], [o {\textasciitilde} uo] and [e {\textasciitilde} ie].  One of his conclusions is that,

\begin{quote}The speakers have a wider range of language usage in phonology than in morphology and syntax probably because they spent more time learning more morphological and syntactic than phonological features ... [and] probably because morphology and syntax decreolize faster than phonology (\citeyear[215]{Thaxter1977}).\end{quote}

What Thaxter explains as an aspect of the rate of \isi{decreolization}, can be explained in another way.  The “standard” that speakers learn in the school system is essentially a written variety of English, with prescriptive rules about “correct grammar”, i.e. morpho-syntactic use.  \citet[370]{Gupta2001} makes the important observation that, 

\begin{quote}The remarkable thing about \isi{Standard English}, especially the \isi{Standard English} found in edited, printed documents, is its uniformity across the world (...) Even our areas of uncertainty are shared.  Countries with contact varieties of English (such as Jamaica, Nigeria, Singapore) participate in this agreement on what \isi{Standard English} is like (...). We must not forget that the concept of Standard is very weak in lexis and phonology.  There is no Standard \isi{accent} of English (though there are prestige accents...).\end{quote}

The English of the classroom is pronunciation neutral, based as it is internationally on a \isi{literary standard}, with few prescriptions on segmental features and fewer, if any on prosody (see \citealt[7]{HoPlatt1993}; \citealt[66--67]{MilroyMilroy1985}; \citealt[118]{Trudgill1999}).  Moreover, \citet[147]{Cheshire1999} argues, at least for the curriculum in English schools, that it is stylistically neutral, as typically little distinction is made between formal and informal styles of speaking in the English taught in the classroom. 

What distinguishes the English of an educated Jamaican from an educated Australian or Nigerian is most clearly the phonological patterns of the speaker, his or her \isi{accent}.  One can identify features of morphology and syntax that are peculiar to communities, but these are relatively few in the educated speaker of formal English.  Rickford’s examples from \isi{Guyana} cited above led him to comment that the \isi{Guyanese} sociolinguistic continuum is gradient really only with respect to the phonology \citep[278--279]{Rickford1987}.  The standard\slash non-standard use of syntactic forms he found much more sharply distinguished socially. 

On the evaluative dimension, Thaxter’s informants found the “local” voice preferable to the “foreign” one, that of an \isi{RP} speaker; and the “librarian’s speech” the variety they would most encourage their pupils to use.  What distinguished “her” speech from the English person was the phonology, specifically a greater use of initial [d] in words such as “this” and the use of mid-vowel diphthongs in words like [fies] \textit{face} and [buot] \textit{boat}.  Following the tradition outlined in the previous sections, Thaxter labelled this phonology “mesolectal” (239).

Miller’s study (\citeyear{Miller1987}) of \isi{AJE} described aspects of the phonological and morpho-syntactic use of post-secondary educated, urban (\isi{Kingston}) dwellers.  She select\-ed and stratified her 24 informants based on their \isi{socio-economic class} (indexed by occupation and income), \isi{gender} and \isi{age} (18--35 and 45--65 years old).  Her conclusions, summarized in the final section of the dissertation, are confusing at times because Miller tends to use the terms “standard” and “prestige” interchangeably, assuming them to be the same thing.  For example, she contends that the typical pattern of \isi{female speech}, the use of a greater number of prestige forms, is not borne out in her study because they differ from metropolitan forms~–  

\begin{quote}On the whole, the females in this study use more non-standard phonological forms than the males ... Only UMC\slash UC [upper middle class\slash upper class] women exhibit the tendency to produce more prestigious pronunciation ... \hbox{}\hfill\hbox{(\citeyear[112--113]{Miller1987})} \end{quote}  

Miller does not present support for her assumptions about the prestige of \isi{MSE} phonological forms.  She does say later that the speech of women and younger informants is possibly “a pattern which is emerging where educated speakers are expressing themselves by using linguistic forms which are not part of a British or American model” (177).  We do not, however, know if any forms are \textit{entering} the \isi{acrolect}, and there is enough historical evidence to suggest that its phonology has displayed variation for some time.

The phonological variables F. Miller studied were i) TH Stopping (θ/ð {\textasciitilde} t/d), ii) [ɪŋ {\textasciitilde} ɪn], iii) Vowel laxing [tek {\textasciitilde} tɛk] or [onli {\textasciitilde} ɔnli], iv) [kj {\textasciitilde} k\textsuperscript{h}] and v) [jʊ {\textasciitilde} jǝ {\textasciitilde} i] in words like \textit{education}.  (See \chapref{ch:2} for a more detailed discussion of many of these phonological features). 

She found, in general, that

\begin{quote}The high frequency of occurrences of palatal realizations (44.3\%), a 30 percent occurrence of TH Stopping, a 50.4 percent realization of the [i] variant of the variable /jʊ/ etc. behoves us to realize that we are indeed dealing with ‘inherent variation’ (\citeyear[183--184]{Miller1987}).\end{quote} 

This inherent variation, she suggests, is due to substrate \isi{Creole} influence and its relationship to the emerging Jamaican norms of English – for Jamaicans see \isi{MSE} patterns as stilted, pedantic and unnatural even in formal circumstances – where aspects of “popular speech” (178) are used to signal this local English language identity.  

In a previous study of acrolectal varieties (\citealt{Irvine1988,Irvine1994}), I was interested in exploring the speech of informants who lived in two neighbouring affluent suburbs of \isi{Kingston}, one characterized by its high level of ethnic minority residents (Whites, Syrians, Chinese for example).  My 32 informants were all educated (post-secondary) and were born into households of relative affluence (indexed by father’s occupation).  The segmental phonological features looked at were post vocalic \isi{rhoticity} and [k\textsuperscript{h}a {\textasciitilde} kja], in addition to aspects of intonation and vowel quality.  It was found, for example, that the informants from the community with the greater number of ethnic minority residents were much more likely to produce [kja] than [k\textsuperscript{h}a], and women generally also followed this pattern.  The other community, with a predominantly black educated population, tended to use the [k\textsuperscript{h}a] variant more frequently, particularly the men.  It was not at all clear, however, whether this finding was to be explained by the influence\slash avoidance of the “\isi{Creole}” [kja] or by a distancing from the patterns of the other community.  

  One could take the extreme position that \isi{MSE}, and certainly the phonology, is not especially relevant to any synchronic discussion of acrolectal speech or variation in the continuum generally; indeed, its relevance diachronically is also somewhat problematic.  \citet{Winford1997}, in his discussion of Caribbean English continua, suggests that “synchronic style shifting from creole \isi{vernacular} to more formal standard usage has much in common with historical language shift from creole to more ‘standard’ or ‘acrolectal’ targets” (271).  But terms like “standard-like” to discuss the production of EBE speakers (268) do, in a sense, presuppose what is standard in the particular social context and therefore what is being targetted.

  It is necessary at this point therefore to try and distinguish use of the terms \isi{acrolect}, Jamaican English and Standard Jamaican English.  The first label, \isi{acrolect}, is more typical of studies that seek to describe formal properties of systems along the continuum (as in the previous sections).  In doing so an idealization of the speech at the apex of the continuum is assumed, while variation is explicitly acknowledged and disregarded in the analysis.  Beckford-Wassink uses the term \textit{acrolect-dominant} to describe the linguistic characteristics of certain types of speakers, i.e. urban, well educated and of relatively high income.  She does analyse variation and argues that no such speaker produces the \isi{acrolect} all the time, but it does represent the dominant lect in their repertoires.  Implicit in the label, however, is the assumption that forms which deviate from \isi{MSE} represent shift away from the \isi{acrolect} in speakers, even when produced in formal\slash self-conscious contexts.  Moreover, her later discussion of the changing indexicality of a form like [kja] as in \textit{cat} for example, which she suggests is “now welcome” \citep[31]{DyerBeckford2001} in the speech of the upwardly mobile, speaks to a presumption that non-\isi{MSE} forms are somehow \textit{becoming} part of the \isi{acrolect}.  Lalla and D’Costa’s comments cited earlier would suggest otherwise.  

  If \isi{acrolect} is meant to be the idealized variety at one extreme of the \isi{Jamaican continuum}, essentially a device to describe a \isi{basilect}, then it cannot be coterminous with “the \isi{local standard}”, as used in the discussions of some Creolists (see \sectref{sec:1.1}, 5).  The latter label suggests forms that are or have been normalised by speakers in the community as “correct”, notwithstanding the lack of corroboration for this process in many of the discussions.  A shift then “from \isi{Creole} to more ‘standard’ or ‘acrolectal’ targets” as Winford states above is problematic if, for example, a form is used along the continuum or if the discussion centres on, say, phonology.  The latter, \isi{acrolect}, suggests the linguists’ construct of Jamaican English, informed as it is by the linguist’s knowledge of formal written English; the former, \isi{local standard}, is an idea held by speakers of what that English should be like, particularly, but not exclusively, when spoken.  

I argue that the two should be the same.  Any discussion of language change, or targetting of forms, or prestige of forms must incorporate the speakers’ ideas of what is English or \isi{Creole} in their social context.  As \citet[119--120]{Shields1987} points out, 

\begin{quote}though some phonological features are shared by mainstream \isi{RP} and \isi{Jamaican Creole} these are often eliminated from the speech of [English speaking] informants because of their obvious associations with \isi{Creole}.\end{quote} 

\section{Sense 4: The acrolect as an idea of Jamaican English}%1.5

Two examples can be used to reinforce why it seems necessary to resituate the idea of \isi{acrolect} in the speaker and in the (Jamaican) social context.  Firstly, in Jamaica, the \textit{Revised primary curriculum} 1999 \citep[14]{Curriculum1999} states:

\begin{quote}The language programme seeks to sensitize pupils to the richness and variety of language.  A major objective is to assist them to acquire the target language \textit{Standard Jamaican} \textit{English}. [My emphasis]\end{quote}

For the Jamaican government, and specifically the \isi{Ministry of Education}, there is such a thing as a \isi{SJE}, the official and prescribed target for Jamaicans in the school system.  The current phonics text selected for use in primary\slash preparatory education in Jamaica contains the following lesson:

\begin{quote}The words which describe the pictures below have vowel pairs (sic) which make the same sound - \textit{bowl}    cow    towel    out    couch    round    mouth \citep[14]{Gbedemah1995}. [My emphasis]\end{quote}

This is the 7th edition of the textbook, and it is normalizing or has normalized the above pronunciation of bowl (on the model of other [aʊ] words) as \isi{SJE}.  An informal survey of most of my own students at the University of the West Indies (Mona, Jamaica) suggests that they do distinguish ‘bowl’ [baʊl] `a ceramic dish' from ‘bowl’ [bol] `the delivery of the ball in cricket'.

Secondly, the teachers at one preparatory (and therefore private) school in \isi{Kingston} consistently instruct children to say [brekfa:st\textsuperscript{h}] \textit{breakfast} and [lIt\textsuperscript{h}ǝl] \textit{little}, but seldom ever comment on the structure never + V-ed (for the negative preterite, as in \textit{never borrowed}) unless children say \textit{neva borrow}.  The teacher's preoccupation with passing on [lIt\textsuperscript{h}ǝl] and [brekfa:st\textsuperscript{h}] to her students cannot be explained as the targeting of \isi{MSE}, but is a response to an idea that \isi{Creole} speakers say [lɪkl] and “leave off word endings”.  Moreover, her actions provide us with insights into her idea of \isi{SJE}, one obviously connected to her idea of ``\isi{Creole}".  

As such, it is the sociolinguistic patterns within communities, fuelled by the ideas speakers hold about language and language users in their community, that explain linguistic behaviour (over time).  In that respect, language situations like that found in Jamaica are not likely to be different from other communities of speakers, where it is generally assumed that foreign norms are peripheral.  

\citet[191]{LePageTabouret1985} believe standard, as applied to language, to have two, sometimes indistinguishable, meanings,

\begin{quote}that of \textit{norm}, (...) and that of a prescriptive yardstick against which people and things are measured ... norm often becomes converted into a prescriptive standard used as a yardstick; through the education and examination systems as a test for admission to various occupational elites, and through social convention as a test for admission to social elites.\end{quote} 
Le Page mentions, in particular, the emergence of the Caribbean Examinations Council (CXC), the Caribbean Lexicography Project (realized in Allsopp’s \textit{Dictionary of} \textit{Caribbean English} \textit{Usage}), the civil service and educational institutions that operate as “focussing agencies” (\citealt[34--35]{LePage1988}) for these norms, thus converting them into yardsticks.  Institutions such as the electronic media or the education system, their effect on the requirements for employment, and the stereotypes that they reinforce do inform speakers’ judgements of standard usage.

In media, for example, the British Broadcasting Corporation (\isi{BBC}) has tended to choose a certain type of voice\slash broadcaster to sell a particular idea of Britain and the British, particularly in its historical role as \textit{the} model for the English language community.  \citet{Lippi-Green1997} cites the then (1924) Managing Director of the \isi{BBC} as explicitly supporting the prescriptive role of media:

\begin{quote}One hears the most appalling travesties of vowel pronunciation.  This is a matter in which broadcasting may be of immense assistance ... We have made a special effort to secure in our stations men who ... can be relied upon to employ the correct pronunciation of the English tongue (\citeyear[137]{Lippi-Green1997}).\end{quote}

She provides similar documentation from National Broadcasting Corporation (\isi{NBC}) urging use of \isi{General American}, as it is “most readily understood” \citep[138]{Lippi-Green1997}.  Pronunciation is the salient feature of language mentioned in directives from media like the \isi{BBC} or \isi{NBC}, not only for the hiring of staff, but also because broadcasters are expected to provide a model of “good” English diction for the public. 

According to \citet{McArthur1992}, the \isi{BBC} policy on language use in broadcasting continued to favour \isi{RP} up to the 1950's, using non-\isi{RP} speakers typically for weather, entertainment (gardening, sport, drama) and less serious topics.  The implicit message is clear – regional voices do not have the authority of the \isi{RP} Standard.  The more inclusive language policy of the 1960's did not extend to the World Service, the branch of the corporation aimed at external markets, which continued to prefer \isi{RP} speaking broadcasters (at least until 1990).  The British English heard by outsiders, as reflected in this distinction between local and foreign broadcast requirements, is still, to a great extent, \isi{RP}.  And \citet[59]{Sharwood-Smith1999} notes that this is itself the wish of the World Service target audience, who, like the bosses of \isi{NBC} in relation to \isi{American English}, hold to an ideology that \isi{RP} is the English “everyone understands”. 

In education, \citet[109]{Lippi-Green1997} cites documents published by the National Council of Teachers of English (USA), and \citet[147]{Cheshire1999} the National Curriculum for English (England and Wales), which suggest that language education policy is also founded on the belief that the school system can and does (re)direct pupils’ language in the appropriate way.  And appropriate means the legitimised “standard”, the variety

\begin{quote}
at the neutral (hegemonic) top and center of stratified society; any deviations can only be interpreted as marked variants that index (...) the producer as also being a correspondingly removed one \citep[412]{Silverstein1998}.
\end{quote}

Jamaican teachers and pupils are told in the curriculum that there is such a standard locally – \isi{SJE} – the target language for all Jamaicans in the education system.  However, examination of the material in the curriculum suggests that \isi{SJE} is, at least in morphology and syntax, little different from the \isi{IAE} in use in the school system of countries in which English is in general use.  There are, of course, Jamaicanisms that occur, even in formal written Jamaican English (\citealt{Christie1989,Christie1998a}; \citealt{Craig1982}; \citealt{Shields-Brodber1997}) but these are not generally accepted in exams.\footnote{Christie cites as examples \textit{Today, when the eyes of the world is on us ${\ldots}$} or \textit{The water entering the reservoirs were extremely muddy}.  Additionally, in formal Jamaican English (written and spoken) sentences such as \textit{Here we have yet another of the anti-drug rally being held} or \textit{A suspect charged with possession of firearm} are more and more common. In both cases, speakers seem to be assigning number based on a logic that a) another = one = rally (SINGULAR) and b) the suspect had one, therefore SINGULAR, firearm. In addition, the use of prepositions (as in \textit{I've been in school from I was six}) needs further investigation.  She points out, however, that many of these forms are not necessarily peculiar to Jamaica, but occur in \isi{MSE} also \citep[17--18]{Christie2003}.}   \isi{SJE} in texts is little different from \isi{MSE}, though students are drilled in the specific aspects of structure that are deemed to be problematic because of \isi{Creole}.  The syllabus for secondary schools is designed to equip the student with (among other things) “the ability to use the language with precision, clarity and grammatical correctness”.  As \citet[256]{Christie1989} notes, however, “it is not usually the case (...) that conscious attempts are made to depart from the traditional model, British \isi{Standard English}”.

The examples of teaching practice given at the start of this section, and the data from sociolinguistic studies of Jamaican English, suggest that phonology is therefore going to be the locus of any construct of a Jamaican norm of \isi{Standard English}.  It is the phonology of standard English that is simultaneously the carrier of national identity - transforming members of the English language community into participants in a speech community \citep[407]{Silverstein1998} -- and one of the indices of what \citet[228]{Bourdieu1984} called ``cultural capital'', aspects of behaviour that reproduce the local social structure and privilege the holders of the legitimised patterns of these behaviours.  Thaxter’s and Escure’s works (cited above) show that foreign models of spoken English are not used or targetted in either \isi{Belize} or Jamaica.  Arguably, they can tell the listener only that the speaker is foreign or pretentious.  But when a speaker of \isi{SJE} has options such as, \textit{standard} [standa:rd], [standʌd] or [standǝd]; or \textit{only} [onlɪ], [ʌnlɪ] and [ondlɪ​] in the social context, each will be associated with particular groups, length and type of schooling, and other such factors that can locate the speaker socially in the speech community for the hearer.  

Notions of standard, then, are fundamentally informed by perceptions of \textit{who} is using the language in the speech community.  \citet[235]{ThakerarGilesCheshire1982} report that listeners presume “standard” forms in speakers they \textit{expect} to use them.  They suggest in their discussion that, “it may well be that speech stereotypes exist such that high status speakers are expected to talk with a standard \isi{accent} (...) low status speakers are expected to talk with a more non-standard \isi{accent}”.  Informants in a previous study (they cite \citealt{ThakerarGiles1981}) heard more standard phonological forms in the speech of a voice they were told belonged to a man who was doing well in his university exams; the group who was told he was doing poorly, heard more non-standard forms in the same voice (236).  Accent is the term used to define the supposed difference in speakers, with an assumption that success in education is naturally associated with “proper” use of English.

\citegen{DyerBeckford2001} study shows how problematic the issue is in their discussion of Jamaican respondent’s judgements of other Jamaican’s speech.  19 informants were asked to listen to three guises of Jamaican speech and judge who spoke better English or used more Patois.  To summarize the data (25), the particular structures that occurred (usually once or twice) in two of the guises were as in \tabref{tab:1.1}.

\begin{table}
\begin{tabular}{lcc}
\lsptoprule
 & Speaker 1       & Speaker 2\\\midrule\relax
[kja]                    &      yes  &    yes \\\relax 
[k\textsuperscript{h}a]  &      no   &   yes  \\\relax
[tʌn] (turned)           &      yes  &    yes \\\relax
tur[nd] (turned)         &        no &     yes\\\relax
fi im (his)              &      no   &   yes  \\\relax
his                      &      yes  &    no  \\\relax
copula (attr.)           &      yes  &    no  \\\relax
noun + dem               &      yes  &    yes \\\relax
noun + s                 &    yes    &  no    \\\lspbottomrule
\end{tabular}
\caption{Features in two Jamaican guises (from \citealt[25]{DyerBeckford2001})\label{tab:1.1}}
\end{table}

Speaker 1 cannot be said to use fewer \isi{Creole} forms than speaker 2, but she was judged to be a Patois speaker by more listeners than Speaker 2.  Speaker 2 was also the only voice considered to be possibly someone from \isi{Kingston}, “stoosh” and “mixing” English and Patois.  Speaker 2, for example, was presumed by listeners to be the younger guise and therefore her use of \isi{Creole} forms did not suggest rural origin as it would in older speakers.  Moreover, Speaker 2 was the only one perceived to be a speaker of proper English by any of Beckford-Wassink’s informants.  It is, however, not clear if informants were interpreting the question, who “uses” Patois or English, as meaning a) “is likely to be a user of” or b) “is here using”.  The single instance of “turned” in Speaker 2 might have been enough to index the former in relation to English, notwithstanding whatever else the speakers said.  What may be important then is not how many English forms the speakers uses, but which English forms and which \isi{Creole} forms.\footnote{\citet{Muhleisen2002} reports that for young black Londoners “the actual competence is of secondary importance, it is the symbol that matters. (...) phatic uses of \isi{Creole} like the greeting “wha‘appen” is often enough to stake one’s claim of “talking black” and belonging to a certain group” (169).  In much the same way, interviews on Jamaican television often start out in English (thus establishing the speaker’s competence) and then move into more \isi{Creole} varieties.}  Additionally, given a particular social profile in the listener’s judgement, here in Beckford-Wassink’s data being perceived as young and urban, occasional use of a particular variant like [tʌn] \textit{turned} or [fɪ ɪm] \textit{his} may not index the same things as if those same variants are used by another type of speaker.  

What seems to be at issue here is that the expectation of a particular phonological pattern, informed in large part by the idea of the speaker held by the hearer, affects judgements of speech.  And these judgements, of standardness or intelligibility or group membership, are not necessarily altered by what is actually produced.  Indeed they colour perceptions in spite of it.  So one identifiable influence on a speaker’s construct of standard speech is the forms they perceive to be used by successful or high status persons in the community, notwithstanding the circularity inherent in the idea that standard pronunciation is more likely to be heard in speakers we expect to produce it.

\section{Aims and methods of the study}%1.6

  I wish to explore aspects of the phonology of educated Jamaicans, speaking in formal circumstances in the workplace, as I argue that \isi{SJE} is essentially going to be distinctive for its phonology.  The study will have two main sections.  In the first (Chapters \ref{ch:2}--\ref{ch:3}), the phonological characteristics of the staff of a government agency, called \isi{JAMPRO}, will be described and subject to a number of sociolinguistic correlations.  The aim of this first section is to provide a description of the use of select linguistic variables and their variants by a sample of educated Jamaicans.  I am especially interested in providing data on the extent to which these features are in use in actual spoken formal Jamaican English.  In the second section (\chapref{ch:4}), the agency in question will be explored as an example of one of the mechanisms for normalizing \isi{SJE} phonology, as it functions in the \isi{Jamaican social context} as a government marketing agency set up to promote Jamaica to both local and international investors.

  I take \isi{acrolect} to be the following:

\begin{itemize}
\item The \isi{acrolect} is the English normalized in Jamaica through public\slash formal use by speakers at the top of the sociolinguistic spectrum, which evolves with reference to that local sociolinguistic context.  The evolution of course is more constrained by the prescriptions of written English and the greater access to education.  

\item The \isi{acrolect}, \isi{SJE}, is therefore typically going to be phonologically distinctive from other varieties of \isi{Standard English}.  Moreover, the patterns that occur in the \isi{acrolect} or \isi{SJE} must be informed by those that occur in the speech perceived to be \isi{Creole}.  This is why some forms that occur in \isi{MSE} are avoided by \isi{SJE} speakers when they are also aspects of \isi{Creole}.

\item While some speakers in the \isi{Jamaican speech community} are \isi{vernacular} speakers of a variety of \isi{JE}, and many more are \isi{vernacular} speakers of a variety of JC or both, \isi{SJE} represents a situationally defined and defining variety that is sanctioned and reinforced by institutions and agencies of the state for formal\slash  public discourse.    
\end{itemize}

\section{Data collection}\label{sec:1.6.1}%1.6.1
Data was collected in 1994 at Jamaica Promotions (\isi{JAMPRO}).  \isi{JAMPRO} has a total staff complement of 188, in various offices in both urban and rural Jamaica.  The principal \isi{JAMPRO} building, from which all data was collected, is a five storey structure located in New \isi{Kingston} -- the financial{\textbackslash}business district in the \isi{Kingston} Metropolitan Area (KMA hereafter).   It housed 153 members of staff at the time data was collected.  The layout of the building is depicted in \figref{fig:1.1}.

\begin{figure}
\resizebox{!}{\textheight}{\begin{tikzpicture}
\node [rectangle split,draw,text width=\linewidth,rectangle split parts=6] {
FLOOR 5 -- President’s Office, Legal Services, Corporate Services\\
No.\ of staff, 12\\
President (Female)\\
Vice Presidents (1 Female, 1 Male)\\
Group Director (1 Male)\\
Directors (2 Female)\\
Officers (3 Female)\\
Secretaries (2 Female)\\
President’s Driver (1 Male)
\nodepart{two}
FLOOR 4 – Research, Documentation Centre, Library, Market Development, Policy \& Projects\\
No.\ of staff, 26\\
Group Directors (2 Female)\\
Directors (5 Female)\\
Officers (8 Female, 2 Male)\\
Secretaries (9 Female)
\nodepart{three}
FLOOR 3 – Manufacture, Tourism\slash Film, Productivity Centre\\
No.\ of staff, 29\\
Group Directors (2 Female)\\
Directors (3 Female, 1 Male)\\
Officers (10 Female, 6 Male)\\
Secretaries (7 Female)
\nodepart{four}
FLOOR 2 – Agriculture\slash Agribusiness, Public Relations\\
No.\ of staff, 28\\
Group Directors (1 Female, 1 Male)\\
Graff\\
Directors (3 Female, 2 Male)\\
Officers (10 Female, 1 Male)\\
Secretaries (9 Female)\\
Ancillary Staff (1 Male)
\nodepart{five}
FLOOR 1 -- Human Resources, Office Administration, Registry, Intl. Trade\\
No.\ of staff, 53\\
Group Directors, (1 Female, 3 Male)\\
Directors (2 Female)\\
Officers (9 Female, 7 Male)\\
Secretaries (13 Female)\\
Ancillary Staff (9 Female, 9 Male)
\nodepart{six}
LOBBY\slash RECEPTION\\
No.\ of staff, 5\\
Secretaries\slash Receptionists (4 Female)\\
Security (1 Male)};
\end{tikzpicture}}
\caption{\label{fig:1.1}JAMPRO, Layout of Company by Sex, Rank and Floor}
\end{figure}

Of the 153 staff members in the Head Office, 104 were interviewed.  Of this 104, 82 had their interviews recorded.  The results presented here reflect the organization as it was at the time of \isi{data collection}.  This sample of the \isi{JAMPRO} population was then grouped in a number of ways, based on information gathered from the questions asked of each of the 104 informants.   

All interviews were done in an office on location loaned to me by the company.  All informants were given appointments, usually a day before the actual recording session, and told that I was conducting research on the company itself and the morale of staff.  All were tape recorded with a notebook in plain sight in which I would, from time to time, make jottings.  

\citet{Wolfson1976} and \citet{Bell1984} both discuss the effect of the audience on speak\-er style.  Both advance a notion that speakers design their speech style in response both to the situational context and to their audience.  Wolfson focusses more narrowly on the interviewer’s effect on the subject, particularly when background, variety and\slash or \isi{gender} are seen to be different.  In addition, she argues that interviews, of the type conducted here with a question and answer format, constitute a specific speech event with its own norms of interaction and language use.  This is, however, precisely the situation that an employee of \isi{JAMPRO} is likely to face when dealing with clients – providing answers to questions about doing business in Jamaica.  These interviews were all conducted by me, a female, a stranger and one expressly from the local University.  Moreover, all informants knew they were part of a study and that they were being recorded.  The audience in these interactions is therefore a local one which includes not only me, the interviewer, but, potentially, others at the University.  I was interested in eliciting \isi{formal speech}, and was attempting to set up a situational and interactional context that would discourage informality.  

An informant can, of course, in contexts like this, “exploit available linguistic resources to construct for herself a particular persona or to construct an encounter as intimate, distant, friendly or otherwise” \citep[206]{MilroyGordon2003}.  The informant, like all speakers, is an active initiator of use of particular styles of speech \citep{RickfordMcNair1994}, and might choose to depart from the \isi{SJE} that would be considered, certainly among educated Jamaicans, unmarked in a formal, taped interaction with a stranger.  As such, this study of speech at \isi{JAMPRO} rests on the assumption that for most, if not all, informants, the speech produced during the interviews reflects their use of this unmarked variety.        

Each interview lasted, on average, 20--25 minutes.  Each interview was transcribed phonetically for subsequent analysis.  These are, of course, relatively short interviews, as I did not wish to give informants time to relax, but still wanted to record a reasonable sample of each speaker’s production.{\interfootnotelinepenalty=10000\footnote{\citet[188, 191]{Young1973} shows that speakers in \isi{Belize} used different frequencies of variants of the same variable in earlier and later parts of interviews.  Speakers who started with high frequencies of “\isi{Standard English}” variants, used fewer later on.}}   This, of course, means that for some variables the number of tokens recorded per speaker was small.  For many commonly occurring forms like the interdental fricatives or the vowel /e/ I was able to collect adequate data.  For example, on average each informant produced roughly 50 tokens of the \isi{voiced interdental fricative} variable; or 27 tokens of the mid-vowel variable.  However, for other variables the data set is small, and below the minimum recommended amount (10) for sociolinguistic analysis (\citealt[164]{MilroyGordon2003}, for example).    

  The collection of this data was therefore affected by two competing necessities.  The first, getting enough data for good statistical analysis, requires a fairly long interview with each subject.  The second, eliciting \isi{formal speech} from an informant who is neither relaxed nor becoming more accustomed to the interaction with me, required that interviews be kept short and business-like.  In my favour methodologically, the literature shows that judgements about the social place of a speaker in the speech community, informed by their language use, are likely to be based on the occurrence of a few forms rather than how many times that form occurs (see results in \citealt[57]{GraffLabovHarris1986}; and other matched-guise studies such as \citealt{DyerBeckford2001}).  Indeed, the use of a single token, which has significance to others in the speech community, can be enough to signal a claim to a particular identity \citep[167]{Bell2001}.  

It is not unreasonable for me to assume that in cases where a few tokens of a particular variable were collected from an informant, these not only reflect appropriate use for the context of interaction but are also the basis on which other Jamaicans can and do make judgements about the speaker.   

All informants, whether recorded or not, were interviewed using the same set of questions. None were required to fill out a questionnaire, nor were they shown the questions.  However, I asked all informants questions from a prepared script (Appendix~\ref{appendix1}). 

The issues that I wish to explore, through use of this social and linguistic data, are therefore:

\begin{itemize}
\item What is the construct that is \isi{SJE}, as reflected in the patterns of phonological use found in this sample of educated Jamaican speakers when in formal interaction?
\item How do the various groups identified differ in their use of these phonological variables?
\item Does \isi{JAMPRO} select speakers of a particular sociolinguistic type for high status positions?
\item Do speakers with mobility aspirations pattern the phonology of their successful colleagues?
\item Is \isi{JAMPRO}, in its practice and expectation, legitimizing certain speech patterns?
\item What is \isi{JAMPRO}’s role in reflecting or promoting a Jamaican \isi{Standard English}?
\end{itemize}
