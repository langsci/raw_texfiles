\chapter{Conclusion}
This study set out to answer a set of questions about English in Jamaica, and to discuss what is called the \isi{acrolect} or acrolectal in Jamaica.  One central point that has been made in this study is that the construct called the Jamaican \isi{linguistic continuum}, of which the \isi{acrolect} is an integral part, has to be defined and analysed in relation to local norms of language use and not an external \isi{MSE}.  Approaches that use \isi{MSE} as benchmark for what is standard in the \isi{Jamaican speech community}, I argue, can be problematic both in analysis of data and in some of the conclusions that follow from that analysis.  For example:

\begin{itemize}
\item  Forms are typologized as basilectal in Jamaica even though they are attested to in the speech of all Jamaicans generally, irrespective of social context of use; 
\item Forms are labelled (upper) mesolectal even when found in the formal\linebreak speech of educated Jamaicans or appear to be accepted as standard locally;
\item Conclusions are arrived at that suggest that no one in the speech community speaks \isi{Standard English}, even as members of the speech community hold to an idea that there is such a thing and that there are such speakers; 
\item The presence of non-\isi{MSE} forms in the speech of educated Jamaicans is described as an emerging or new trend even though such variation was identified some two centuries ago; 
\item Discussions about \isi{decreolization} and language change are carried out on the assumption of a vaguely defined “\isi{local standard}” as one target, even though in practice that \isi{local standard} is typically treated as either \isi{RP} or \isi{General American}.
\end{itemize}

This exploration of the \isi{acrolect} takes an \isi{endonormative approach} and seeks to explore the phonology of \isi{JE}, as used by 82 educated speakers using language in a formal context of interaction.  From the data collected, I define what is spoken \isi{SJE} or the Jamaican \isi{acrolect}.  Further, the speech of select groups of informants was singled out for analysis, most notably frontline \isi{JAMPRO} staff, in order to present an actual example of good spoken \isi{JE} as judged by other Jamaicans.  In so doing I wished to address the following questions:

\begin{itemize}
\item What patterns of use are to be found in this sample of educated Jamaican speakers when in formal interaction?
\item How do the various groups identified differ in their use of these phonological variables?
\item What is the relationship between acrolectal phonology and other varieties in the speech community? 
\item Does \isi{JAMPRO} select speakers of a particular sociolinguistic type for high status positions?
\item Is \isi{JAMPRO}, in its practice and expectation, legitimizing certain speech patterns?
\item What, based on all the above, constitutes an endonormatively determined set of phonological features for \isi{SJE}?
\end{itemize}

\section{The architecture of phonological variation in the Jamaican acrolect}\label{sec:5.1}

Two types of phonological variables were created for the purposes of analysis.  One type was a group of variables previously described in the literature as having \isi{Creole} and English variants (Groups A); the other was a less studied group with variant options that are not solely describable as \isi{Creole} vs English and which suggest another aspect of \isi{JE} variation (Group B).  The general patterns of use of these variables is shown in Tables~\ref{tab:5.1} and \ref{tab:5.2}, with reference to particular sociolinguistic correlations.      

\begin{table}
\resizebox{\textwidth}{!}{\begin{tabular}{lcccccc}
\lsptoprule
Group A & Education & Background & Gender & Age & Frontline & Significant\\ 
\midrule{}
[h {\textasciitilde} ${\emptyset}$] & * & * & * &  & * & 4\\{}
[ð {\textasciitilde} d] & * & * & * &  & * & 4\\{}
[θ {\textasciitilde} t] & * & * & * &  & * & 4\\{}
[ɔ {\textasciitilde} a] & * & * & * &  & * & 4\\{}
[o {\textasciitilde} uo] \shadecell &  \multicolumn{6}{c}{\shadecell} \\{}
[e {\textasciitilde} ie] & * & * & * & * & * & 5\\{}
[ɔr {\textasciitilde} uor] & * & * & * & * &  & 4\\{}
[er {\textasciitilde} ier] & * & * & * &  & * & 4\\{}
[k\textsuperscript{h}a {\textasciitilde} kja] \shadecell &  \multicolumn{6}{c}{\shadecell} \\{}
[k\textsuperscript{h}a: {\textasciitilde} kja:] & * & * &  &  & * & 3\\ 
\midrule
Significant & 8 & 8 & 7 & 2 & 7 & \\ 
\lspbottomrule
\end{tabular}}
\caption{Overview of Group A data indicating statistically significant associations\label{tab:5.1}}
\end{table}

\begin{table}
\caption{Overview of the Group B data including statistically significant associations\label{tab:5.2}}
\resizebox{\textwidth}{!}{\begin{tabular}{lcccccc}
\lsptoprule
Group B & Education & Background & Gender & Age & Frontline & Significant \\
\midrule
\itshape -tion & * &  &  &  & * & 2\\
\textit{culture} wds. & * &  &  &  &  & 1\\
\textit{party} wds. & * &  & * &  &  & 2\\
\textit{forty} wds. & * &  &  & * & * & 3\\
-ed before V &  & * &  &  &  & 1\\
-ed before C &  &  & * &  &  & 1\\
-nt before V & * &  &  &  &  & 1\\
-nt before C & * &  &  & * & * & 3\\
-st before V &  &  & * &  & * & 2\\
-st before C &  &  &  &  & * & 1\\\midrule
Significant & 6 & 1 & 3 & 2 & 5 & \\
\lspbottomrule
\end{tabular}}
\end{table}

  As shown in \tabref{tab:5.1}, there are some Group A variables that are statistically significant with most of the social categories used to stratify this population.  Most of them, for examples [h] in \textit{hot}, [ð {\textasciitilde} d] in \textit{them}, [θ {\textasciitilde} t] in \textit{thing}, [ɔ {\textasciitilde} a] in \textit{pot}, [ɔr {\textasciitilde} uor] in \textit{poor} and [e {\textasciitilde} ie] in \textit{face}, are aspects of education, \isi{parent’s background} and high job status in the same speakers means a high degree of similarity in the way these groups pattern in their use of Group A variants.  The \isi{Creole} variants of the first four are also the features that women tended to produce less of.  We can identify the standard and prestige variants of these as [h], [θ], [ð], [ɔ], [ɔr] and [e], the first four also identified as part of the prescriptive \isi{SJE} of the school curriculum.  Further, the patterns of use in the sample indicate that [k\textsuperscript{h}a\textbf{:}] \textit{calf} and [o] \textit{boat} are also the standard and prestige forms in \isi{JE}.  

Indeed, in the sample of speakers who received at least a secondary level education, there was very consistent use of the \isi{JE} variants of these variables and very little use of either [kja\textbf{:}] or [uo].

Further support for this is to be had in the hypercorrect productions of some informants for some of these variables.  This \isi{hypercorrection} does suggest that these variants, for example /h/:[h],  /ɔ/:[ɔ] and /θ/:[θ], can be said to characterize and define acrolectal speech. 

Among Group B variables, those that also presented variation among \isi{JE} options, the sociolinguistic variation that occurs is more diffuse than with Group A variables.  Notably, only level of education and frontline status distinguish speakers’ use of more than a few features.  The association of education, \isi{parent’s background} and status in the agency was manifest in Group A variation, where JC and \isi{JE} variants were typically the speakers’ options for use.  But with these Group B variables, the patterns of the highly educated were not necessarily a predictor of the patterns of speakers from more affluent backgrounds or high status positions at \isi{JAMPRO}.

   The variants [ʃɔn] \textit{–} \textit{tion}, and [tj]\slash [dj] \textit{culture} \textit{/} \textit{soldier} reinforce the point that \isi{MSE} forms cannot be used to identify acrolectal forms.  In both instances speakers here do not necessarily use the \isi{MSE} variant (schwa and the \isi{affricate}) but are varying with options that reflect the coexistence with \isi{Creole} and what speakers believe to be \isi{Creole} forms.  These are features that are produced in response to an idea of what is stereotyped \isi{Creole} (affricates), distancing speakers even when the \isi{Creole} form is like the non-\isi{local standard} that the literature has assumed the \isi{acrolect} to be.  As such, the \isi{acrolect} is not ``more or less well-defined and discrete" \citep[241]{Winford1997}, at least phonologically.  Forms that are perceived to be \isi{Creole}, such as [ʃan] \textit{–} \textit{tion} or [ʤ] \textit{treasure}, affect the productions of some speakers and as a consequence the kind of variation found in the \isi{acrolect}.  

However, the distribution of the phonological variants in the data presented here is more complex than merely a low frequency of \isi{Creole} forms or high frequency of prestige forms in the speech of formal \isi{JE} speaking informants.  And it is this distribution that turns out to be \textit{the} important feature of acrolectal phonological variation.   

There is one category of phonological variables that appears to be salient for producing \isi{SJE}.  Speakers, when producing these variables, show a remarkable conformity in variant use.  

These are: 

\begin{enumerate}[label=\alph*)]
\item the word initial \isi{glottal fricative} [h] in words like \textit{hand};
\item the \isi{voiceless interdental fricative} in words like \textit{thin}; 
\item the word initial \isi{velar stop} [k\textsuperscript{h}] before the long \isi{low central vowel} [a\textbf{:}] in words like \textit{calf};
\item\relax [ɔ] in words such as \textit{cotton};
\item post [ɔ] \isi{rhoticity}, specifically before [+ coronal] consonants in words like \textit{forty};
\item\relax [ɔr] in words such as \textit{poor};
\item\relax [o] in words such as \textit{boat};
\item the word final phonological stop cluster [nt] before a following vowel, as in \textit{hint} \textit{at.};
\item the word final \isi{morphophonemic cluster} when a past tense marker, as in \textit{looked}.
\end{enumerate}

This type of variable I label a \textit{load-bearing} phonological variable in the \isi{acrolect}, for they seem to function primarily to define the variety the speaker is using.  Without these variants being produced in significant quantities, the speaker will not be interpreted in the \isi{Jamaican speech community} as someone speaking English.  Moreover, it appears that it is use of these load-bearing variants, and not English variants \textit{per} \textit{se}, that defines someone as speaking \isi{SJE}.  

As an example, high status speakers typically produce voiceless interdental fricatives, so much so that qualitatively hypercorrect use of the variant occurred.  A different analysis, however, would have to be made of these same speakers’ use of [ð {\textasciitilde} d].  The voiced TH stop variant occurs in their speech in high frequencies, and occurs more in the speech of \isi{frontline staff} than those not selected to interact with the public.  The kind of asymmetrical patterning I find for this pair of variables does not seem to occur in spoken \isi{MSE}.  

This same kind of asymmetrical sociolinguistic pattern is found elsewhere in the data, and is evident in the mid-vowels, the velar stops and post vocalic \isi{rhoticity}.  [e {\textasciitilde} ie] is sensitive to social factors here and more frequently varied, but [o {\textasciitilde} uo] is not; [kja] is attested in all speakers, but [kja\textbf{:}] is seldom produced by anybody; and while speakers vary rhotic productions after [a], they generally are rhotic after [ɔ].  

It is important to point out, also, that this is the pattern of production in \textit{careful} speech.  This asymmetrical attention to variants produced of related variables is then fundamentally different to the patterns I can identify in the literature on metropolitan English varieties, and is not a function of \isi{style shift} in speakers.   Informants here, when producing speech in circumstances that would suggest care and attention to forms used, focus on \textit{one} variable of the pair and show some consensus on which variant to select.  At the same time, use of the other variable in the pair is typically less focussed and is characterized by much greater levels of variation.  

This suggests that use of the variants [ð] in \textit{them}, [e] in \textit{face}, [k\textsuperscript{h}] in \textit{cat} or the \isi{affricate} [ʧ] in \textit{culture}, does not necessarily define \isi{SJE}.  This makes these phonological features a different type from their load-bearing counterparts, because their production does not seem to be either necessary for or an indicator of speaking \isi{SJE}.   

The data also shows that speakers, in performing acts of identity or signalling \isi{group affiliation}, are more likely to manipulate use of these non load-bearing \isi{JE} variants for such purposes.  I argue that this is because their presence or absence is \textit{not} already assigned particular social meaning in the speech community, i.e. an inability to speak English, and therefore the speaker can manipulate these phonological variables in contexts of interaction that require \isi{SJE}.  

Younger informants used the diphthong [ie] in \textit{face} more frequently; in the same speakers there was very little use of the \isi{back diphthong} [uo] in \textit{boat}.  When compared with the oldest speakers, it \textit{appeared} that \isi{front diphthong} use was increasing.  However, Alleyne’s observations about the more widespread use of [ie] in Jamaica were made over two decades ago, and therefore I remain cautious about saying that its more frequent occurrence in younger speakers reflects a changing norm.  It is possible therefore, that \isi{front diphthong} use here distinguished younger speakers from older ones, and therefore is an aspect of signalling \isi{group affiliation}.  

Variation in \isi{rhoticity}, specifically after [a] in words such as \textit{party}, also correlated with \isi{age} as well as \isi{gender}.  Young men were more typically non-rhotic after [a], distinguishing them from older speakers and women’s patterns generally.  Males in this sample also more typically produced an \isi{affricate} in \textit{culture} type words, as distinct from the female pattern which showed higher frequencies of [tj] use.\largerpage[-1]

I argue that because the variables discussed above are not crucial to defining the variety being used, whether JC or \isi{JE}, the type of variation that occurs is going to be less focussed and therefore more a function of signalling group affiliations like \isi{gender}, or \isi{age}.  Variation on load-bearing variables will be constrained by their function of either indexing “speaking \isi{Creole}” or “speaking English” or the identities of “\isi{Creole} speaker” or “English speaker”.  But this is not the only type of social differentiation that is marked sociolinguistically.  

The question then becomes, how does one project membership in other types of social groups, particularly to an unfamiliar hearer?  One cannot manipulate variants to which social meaning is already widely assigned unless one is interacting with intimates.  When interacting with intimates, who are familiar with one’s background and who generally have information about one’s social position, use of JC features is typical, indeed it is expected.  However, with strangers, variation in one’s use of [h] or [uo] in \isi{JE} is likely to indicate something undesirable, such as “backwardness” or being “low class” or “uneducated”, particularly in a (work) context that favours the English speaking, middle class persona.  As such, the pattern of variation on load-bearing and non load-bearing variables is different, more uniform on the former than on the latter.

I wish to therefore to reformulate my initial classification of variables as Group A or B, and instead distinguish load-bearing variables from the other phonological variables used in this study.  The absence or low frequencies of the appropriate variant of a \isi{load-bearing variable} will be negatively evaluated in social contexts where use of \isi{SJE} is expected.  

I further propose that this asymmetrical attention to variants is possibly a function of the coexistence of two varieties for speakers, English and \isi{Creole}, in an ideological context where use of both is the ideal and not just use of “the standard”.  Models of language situations like Jamaica point to the structural relationship among varieties, i.e. the \isi{linguistic continuum}, or the functional relationship among varieties, i.e. \isi{diglossia}.  The variation that occurs here in this \isi{JAMPRO} data, suggests that these speakers do not merely shift from the L variety JC to the H variety \isi{SJE} when switching from informal to formal situations.  Rather, these are speakers who seem to signal a shift in both variety and function by manipulating \textit{some} variables, those load-bearing variables which are indexical of variety and which therefore signal use of the H code.  At the same time, the ability to use JC, the L code, is signalled by the presence of other non load-bearing variables because the ability to use both is ideal in the \isi{Jamaican social context}.

I can abstract four types of variables from my data, reflecting the general patterns of their distribution in the speech of this \isi{JAMPRO} sample.

\begin{figure}
\caption{Load-bearing and non load-bearing variables in JE}

\begin{tabularx}{\textwidth}{XXXX}
\lsptoprule
\multicolumn{4}{c}{Variables}\\
\multicolumn{1}{l}{Load-Bearing} & \multicolumn{2}{c}{Socially Indexical} & \multicolumn{1}{l}{Non Load-Bearing} \\\cmidrule(lr){1-1}\cmidrule(lr){2-3}\cmidrule(lr){4-4}
\multicolumn{1}{c}{Type 1}   & \multicolumn{1}{c}{Type 2} & \multicolumn{1}{c}{Type 3} & \multicolumn{1}{c}{Type 4}\\\midrule
/o/:[o]C & /o/:[ɔr] & /e/:[e {\textasciitilde} ie] & \\
         & /h/:[h] &  & \\
-ed      & -nt:[nt]V & -st:[s {\textasciitilde} st] & -n’t:[n {\textasciitilde} nt]\\
         & /ɔ/:[ɔ] &  & \\
         & /θ/:[θ] & /ð/:[ð {\textasciitilde}d] & \\
         & /k/:[k\textsuperscript{h}]a: &  & /k/:[k\textsuperscript{h} {\textasciitilde} kj]a\\
         & [ɔ]+rhotic & [a]+/\textminus rhotic & \\\midrule
\multicolumn{2}{l}{Not used in \isi{Jamaican Creole}\raisebox{.5ex}{\tikzmark{Fig51left}}} & \multicolumn{2}{r}{\raisebox{.5ex}{\tikzmark{Fig51right}}\,Used in \isi{Jamaican Creole}\begin{tikzpicture}[remember picture, overlay] \draw[<->] (pic cs:Fig51left) -- (pic cs:Fig51right);\end{tikzpicture}}\\
   \lspbottomrule
\end{tabularx}
\end{figure}

Load-bearing and non load-bearing variables can be further distinguished in terms of whether or not they function as indexing membership in or affiliation with particular social groups.  

\begin{itemize}
\item Type 1 load-bearing variables typically are produced by speakers with very similar patterns of use.  All speakers tended to produce -ed and [o] in \textit{goat} and, in fact these features were seldom statistically significant when subject to the test of association.
\item Type 4 non load-bearing variables are also produced with very similar patterns of use.  All speakers tended to vary the Type 4 variants [kj {\textasciitilde} k\textsuperscript{h}] in \textit{cat} or [nt {\textasciitilde} n {\textasciitilde} õ] in \textit{don’t} regardless of how the sample was stratified and analysed.  
\item Types 1 and 4 are not usually sensitive to sample stratification.  What differentiates Type 1 from Type 4 is that the generally used variant of a Type 1 \isi{load-bearing variable} is \textit{not} used in JC, while the variant used in a Type 4 non \isi{load-bearing variable} does occur in JC as well.
\item In contrast, Types 2 and 3 variables \textit{are} typically sensitive to sample stratification.  However, they (are used to) index different things in the speech community. 
\item Type 2 variables, which are load-bearing, typically can be correlated with social factors like level of education and social status and therefore also index an ability to speak \isi{SJE}.  
\item  Type 3 variables tend to correlate with factors like \isi{gender} or \isi{age} without \textit{necessarily} also signalling an (in)ability to use \isi{SJE}.
\end{itemize}

\citegen{DyerBeckford2001} matched-guise test results, in which a\linebreak speaker who clearly used \isi{Creole} forms was judged to be a likely speaker of \isi{JE}, can be explained in terms of her use of load-bearing and non load-bearing variables\footnote{This data is discussed in \chapref{ch:1}, \tabref{tab:1.1}.}.  The speaker judged as “uses English”, even though she “produced 16 of 19 forms with basilectal variant” (26) in the test, did produce – ed \textit{turned} and [k\textsuperscript{h}] (lexical item unclear).  

As there is nothing to suggest that the phonology of a system would have patterns of variation peculiar to it alone, which are not also evident in other areas of structure, it is very likely that use of morphosyntactic forms in \isi{JE} is structured in a similar way to the phonological variation abstracted from the data presented here. 

\section{An endonormative definition of SJE phonological features}\label{sec:5.2}

  The data from this \isi{JAMPRO} sample reveals that education, social background and \isi{gender} are the three social factors that have a significant association with speakers productions, a not unexpected finding.  Crucially, being female, university educated and from a relatively affluent background to a great extent reflects the type of employee \isi{JAMPRO} hires; \isi{senior management} explicitly says so.  The \isi{JAMPRO} data as a whole shows that typically the features that tend to occur in \isi{female speech} also tend to distinguish the select employees at \isi{JAMPRO}, i.e. the frontline cohort.  Female speech also patterns the variant use that characterizes those with more access to (longer) schooling.  This suggests either that the phonology of formal acrolectal varieties is essentially going to be a female\slash educated norm, or that women at \isi{JAMPRO} approximate more to the norms considered to be acrolectal in the speech community.  It is perhaps one reason why Senior Management at \isi{JAMPRO} articulates a preference for female employees when selecting from interviewed candidates.  

Generally, there are very few, if any, phonological features that are not shared with other varieties along the Jamaican \isi{linguistic continuum}.  What distinguishes acrolectal speech is the frequency of use of these features and what characterizes it is a low production of \textit{stigmatized} \isi{Creole} items.  This suggests that the \isi{acrolect}, and here I mean the spoken \isi{SJE}, will be described as either having [h], [θ] and the like or as not having the \isi{Creole} reflex of these features.  It cannot be described as being typified by an absence of [kja] or [ð] \textit{them} or even mid-vowel diphthongs, although it is very probable that if asked, speakers will describe an idealized \isi{SJE} that is devoid of these features. 

The \isi{acrolect}, as actual spoken \isi{SJE}, cannot therefore be located by identifying \isi{MSE} forms; nor can the features along the \isi{linguistic continuum} be characterized with reference to \isi{MSE} without leading to paradoxical conclusions or misinterpretations of data.  Indeed, an idealized \isi{acrolect} as “\isi{local standard}” can also be problematic.  This can be illustrated by looking again at the example of Meade’s work (\citeyear{Meade2001}), cited at the beginning of this study.  In his study \isi{MSE} phonology was used to determine what he labels acrolectal phonology.  But his data on use of dental fricatives and their variants makes no distinction between voiced and voiceless forms.  It is entirely possible, therefore, that his sample of Jamaican children with educated caregivers are being exposed to [ð {\textasciitilde} d].  The data I have would suggest that it is the pattern on the voiceless fricative, or other such load-bearing variables, that would need to be examined to determine a time frame for acquisition of \isi{JE} phonology.      

The speaker's notion that two varieties exist in Jamaica, JC and \isi{JE}, is therefore important.  The form English takes in the \isi{Jamaican social context}, particularly its pronunciation, is shaped in part by the idea speakers have of what \isi{Creole} is.  In an informal survey of 42 students at the University of the West Indies, when asked how they know someone is speaking \isi{good English}, most responded by saying ``because they are not" - dropping aitches, saying ‘mi’, ‘did’ a and so on, listing a number of stereotypical \isi{Creole} practices.  For them, English is, in part, negatively defined in relation to what \isi{Creole} is believed to be and is assessed in terms of the speaker's avoidance of those characteristic items.  \citet[12]{Wolfram1998} comment on similar definitions of the standard in the US.    But there are other items that are present in the \isi{acrolect} that have been typically associated with \isi{Creole} that speakers use as frequently as, if not more than, the variants which also appear in \isi{MSE}. 

\citet[9]{Patrick1999} observes, of his urban, mesolectal sample, that the speakers he interviewed all seem to have

\begin{quote}
	a rich and nuanced ability to vary their speech (...) but a very limited set of metalinguistic labels for it, essentially binary oppositions of which the bluntest instrument is the English\slash \isi{patois} distinction.
\end{quote}

He does not take the strong position of DeCamp, who described this belief as a persistent myth (\citeyear[350]{DeCamp1971}), but argues that this distinction, though psychologically powerful, should not form the basis of linguistic description.  But the importance of the speaker’s distinction should not be underestimated either.  It is central to the way variation patterns asymmetrically on a number of phonological variables.  Moreover, issues like \isi{language acquisition} and language change in the Jamaican context are going to be affected by what speakers think is \isi{Creole} and English, to the extent that they are trying to target the latter.  For them, targeting English is not reproducing foreign norms.\footnote{An early discussion of this preference for and use of local norms is found in Eersel’s study (\citeyear[320]{Eersel1971}) of educated Surinamese use of Dutch.}   Many speakers here for example, notably women and the secondary educated, are avoiding affricates, perceived to be \isi{Creole}, and producing [tj] in \textit{culture} type words.

\isi{JAMPRO} was specifically selected because of its role as a one of the Jamaican state’s promotion agencies, conducting its business in a context that it explicitly states is reserved for \isi{Standard English}.  \isi{JAMPRO} is one of many agencies that select receptionists, broadcasters, managers and the like and reject or pass others over.  In its practice of selecting some staff for frontline positions, Jamaicans who represent the agency in this milieu, \isi{JAMPRO} is implicitly favouring some patterns of speech over others and determining which features are acceptable or unacceptable as spoken \isi{SJE}.

The data here demonstrate that a number of social and linguistic factors inform their selection.  Frontline staff are typically highly educated, they come from households more likely to be JC\slash \isi{JE} speaking and they are perceived by others in their work environment to be selected because they can represent \isi{JAMPRO} well, socially and in the way they speak English.  Arguably, \isi{frontline staff} reflect the agency’s idea of “best speakerhood” \citep[286]{Silverstein1996} – at least of the upper end of the bidialectal\slash diglossic construct \isi{JAMPRO} says exists in Jamaica.  Their speech shows the pattern of variation uncovered in this study – little if any voiceless TH stopping, but freely varied [d {\textasciitilde} ð]; a virtual absence of [kja\textbf{:}], but frequent use of [kja]; little use of diphthongal variants for the back mid-vowel and variation between [e {\textasciitilde} ie], especially pre-rhotically.  In addition, \isi{frontline staff} are more likely to produce [ɔ] – whether in stressed syllables or in words ending – tion – and seldom \isi{h-drop}.    

I suggest that this frontline data presents an actual example of what “\isi{good English}” is in Jamaica.  Speakers who seek certain kinds of jobs will perhaps target features like [ʃɔn], or even [kja]; certainly they will identify [h] and [ɔ] as aspects of speech that the more successful use, not because they belong to British or American varieties of English but because in Jamaica these are features of speakers presented as having the right diction.  Speakers who seek certain kinds of jobs will carefully avoid voiceless TH stopping, [uo], [kja\textbf{:}] because these are ``\isi{Creole}", not because British or American speakers do not use them.  And they will perhaps freely vary word final [st {\textasciitilde} s] in a word like \textit{cast}, [d {\textasciitilde} ð] or [e {\textasciitilde} ie] because they do not necessarily index ``bad English".    

It is important therefore to separate discussion of the (historical) linguistic processes that may have accounted for most of the forms I have described above, from a speaker-driven endonormative discussion of the Jamaican \isi{acrolect}.  Undoubtedly, \isi{hypercorrection}, distancing from \isi{Creole} or \isi{Creole} influence, may explain the presence of most of the features in the production of informants.  But the idea speaker hold of \isi{SJE}, the \isi{acrolect}, as it occurs now will come from their perceptions of who is successful and how they speak and who is presented as having a good command of the English language in the \isi{Jamaican social context}.  It cannot be assumed, however implicitly, to be a foreign model of English because English is not a foreign language in Jamaica.  \citegen{Muhleisen2002} criticism that “CELCs [Caribbean English-lexicon creoles] are (...) never defined in isolation but always 'in comparison with' the high prestige language English” (74) is made more pointed if that English is itself situated outside of the speaker’s community.

It is not clear how we can proceed to discuss issues of \isi{decreolization}, or of language change generally, with assumptions about the \isi{acrolect}, or for that matter the \isi{basilect}, that are not based on the norms of speakers in their speech communities.  Jamaican English is, after all, a national variety (or set of varieties) in its own right.  The speakers in this speech community operate in a social context where their two languages, Jamaican English and \isi{Jamaican Creole}, have coexisted for centuries.  The two are perceived as distinct, but connected elements of being Jamaican.  The data presented here demonstrates both – the speaker’s differentiation and integration of \isi{JE} and JC in the Jamaican diasystem.
