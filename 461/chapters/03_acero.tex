\documentclass[output=paper,colorlinks,citecolor=brown]{langscibook}
\ChapterDOI{10.5281/zenodo.14017925}

\author{Lucía Madrigal Acero\affiliation{Universidad     Complutense de Madrid}}

\title[What can be used in Greek and Latin?]{What can be used in Greek and Latin? A comparative study of the support verbs χράομαι \textit{kʰraomai}   and \emph{utor}}


\abstract{In this contribution, I offer a comparative approach to support-verb constructions in Greek and Latin.  Despite their differences, both languages employ verbs meaning ‘to use’ as support verbs in combination with a vast set of nouns. The objectives of this contribution are: (i) to observe the semantic-syntactic domains in which these verbs operate; (ii) to analyse the properties and functions of these support-verb constructions, together with their distribution; and (iii) to compare these support-verb constructions in Greek and Latin. The conclusions are reinforced by a quantitative analysis of the data. I conclude that χράομαι \emph{kʰraomai} 'to use' and \emph{utor} 'to use' are both used as support verbs in Ancient Greek and Latin, and that they alternate with aspectual and causative support-verb extensions\is{support-verb extension}.
\bigskip


 \foreignlanguage{spanish}{En esta contribución, ofrezco un acercamiento comparativo a las
construcciones de verbo soporte en griego y latín. 
A pesar de
sus diferencias, ambas lenguas utilizan verbos con el significado de ‘usar’ 
como verbos
soporte en combinación con un gran número de sustantivos. Los objetivos de esta contribución son: (i) observar los dominios
semánticos y sintácticos en que operan estos verbos; (ii) analizar las
propiedades y funciones de estas construcciones de verbo soporte, así como su distribución; y (iii) compararlas 
en griego y latín. Las conclusiones
vienen apoyadas por un análisis cuantitativo de los datos. Concluyo que χράομαι
\emph{kʰraomai} y \emph{utor} se usan como verbos
soporte en griego antiguo y latín y que alternan con extensiones de verbo soporte
aspectuales y causativas.}}




\IfFileExists{../localcommands.tex}{
   \addbibresource{../localbibliography.bib}
   \usepackage{langsci-optional}
\usepackage{langsci-gb4e}
\usepackage{langsci-lgr}

\usepackage{listings}
\lstset{basicstyle=\ttfamily,tabsize=2,breaklines=true}

%added by author
% \usepackage{tipa}
\usepackage{multirow}
\graphicspath{{figures/}}
\usepackage{langsci-branding}

   
\newcommand{\sent}{\enumsentence}
\newcommand{\sents}{\eenumsentence}
\let\citeasnoun\citet

\renewcommand{\lsCoverTitleFont}[1]{\sffamily\addfontfeatures{Scale=MatchUppercase}\fontsize{44pt}{16mm}\selectfont #1}
  
   %% hyphenation points for line breaks
%% Normally, automatic hyphenation in LaTeX is very good
%% If a word is mis-hyphenated, add it to this file
%%
%% add information to TeX file before \begin{document} with:
%% %% hyphenation points for line breaks
%% Normally, automatic hyphenation in LaTeX is very good
%% If a word is mis-hyphenated, add it to this file
%%
%% add information to TeX file before \begin{document} with:
%% %% hyphenation points for line breaks
%% Normally, automatic hyphenation in LaTeX is very good
%% If a word is mis-hyphenated, add it to this file
%%
%% add information to TeX file before \begin{document} with:
%% \include{localhyphenation}
\hyphenation{
affri-ca-te
affri-ca-tes
an-no-tated
com-ple-ments
com-po-si-tio-na-li-ty
non-com-po-si-tio-na-li-ty
Gon-zá-lez
out-side
Ri-chárd
se-man-tics
STREU-SLE
Tie-de-mann
}
\hyphenation{
affri-ca-te
affri-ca-tes
an-no-tated
com-ple-ments
com-po-si-tio-na-li-ty
non-com-po-si-tio-na-li-ty
Gon-zá-lez
out-side
Ri-chárd
se-man-tics
STREU-SLE
Tie-de-mann
}
\hyphenation{
affri-ca-te
affri-ca-tes
an-no-tated
com-ple-ments
com-po-si-tio-na-li-ty
non-com-po-si-tio-na-li-ty
Gon-zá-lez
out-side
Ri-chárd
se-man-tics
STREU-SLE
Tie-de-mann
}
   \boolfalse{bookcompile}
   \togglepaper[3]%%chapternumber
}{}


\begin{document}
\emergencystretch 3em
\maketitle

\section{Introduction}\label{sec:ma:1}

Support-verb constructions (SVCs henceforth)\footnote{The dataset is accessible here: \url{http://dx.doi.org/10.5287/ora-n652gamyj}. The Greek and Latin
  texts have been taken from the \emph{Thesaurus Linguae Graecae} and the \emph{Corpus Corporum} databases.
  Translations are my own. The examples for SVCs with verbs other than \emph{utor} have
  been obtained from the \emph{Dictionary of Latin Collocations (DiCoLat)}
  \parencite{banos_dicolat_nodate}. Some examples for SVCs with verbs other than χράομαι
  \emph{kʰraomai} have been obtained from the \emph{Dictionary of Greek
    Collocations (DiCoGra)} \parencite{jimenez_lopez_dicogra_nodate}. The glosses follow
  the Leipzig Glossing Rules.} in Greek and Latin have been the subject of
several papers by the members of successive research projects in Spain
\parencites{banos_bases_2018,jimenez_lopez_support_2016,jimenez_lopez__2021,jimenez_martinez_colocaciones_2019,mendozar_cruz_construcciones_2020,tur_aproximacion_2020,banos_latin_2022},\footnote{The projects are: `Interacción
  del léxico y la sintaxis en griego antiguo y en latín: construcciones con verbo soporte
  diátesis y aspecto' (FFI2017-83310-C3-3-P, led by J.\,M. Baños); `Diccionario de
  Colocaciones Latinas en la Red (DiCoLat)' (led by J.\,M. Baños); and `Interacción del
  léxico y la sintaxis en griego antiguo y latín 2: Diccionario de Colocaciones Latinas
  (DiCoLat) y Diccionario de Colocaciones del Griego Antiguo (DiCoGrA)'
  (PID2021-125076NB-C42, led by J.\,M. Baños and M.D. Jiménez López).} Italy
\parencites{tronci_at_2017,pompei_verbi_2019},\footnote{The projects are: `Lessico e sintassi in greco
  antico e italiano' and `Strutture di frase con sintagmi preposizionali predicativi:
  greco antico, latino e italiano a confronto', both led by L. Tronci.} and the United
Kingdom
\parencites{fendel_greek_2021,fendel_support-verb_2023,fendel_taking_2024}.\footnote{The project is: `Giving
  gifts and doing favours: Unlocking Greek support-verb constructions' (ECF-2020-181, led
  by V. Fendel).} The comparative approach taken by some of these contributions
\parencites{banos_arrepentirse_2017,banos_vota_2018,lopez_martin_colocaciones_2019} has
proved productive, since SVCs are frequent in contexts with intense cultural and
linguistic exchange and are easily transferred from one language to another
\parencites{bowern_diachrony_2008,fendel_greek_2021}. The different frequencies of SVCs in
Greek and Latin texts have often been highlighted, i.e. Greek texts tend to contain
more occurrences of simplex verbs\is{simplex verb} than SVCs, whereas Latin texts show a significantly
higher proportion of support-verb constructions
\parencites[229]{banos_bellum_2015}[183]{jimenez_lopez_support_2016}. Nevertheless, the
two also share some similarities.

One of these similarities lies in the use that both languages make of χράομαι
\emph{kʰraomai} and \emph{utor} `to use' as support verbs with a
surprisingly wide collocative spectrum. Both are often combined with a range of nouns
which is difficult to synthesise in a few semantic or lexical labels. In fact, previous
papers on \emph{utor} have overlooked this function of the verb, thereby showing
astonishment at its wide range of objects (\cite{alonso_fernandez_estructuras_2010}, see
also Squeri (this volume)).

The objectives of this contribution are: (i) to analyse the properties and functions of
the SVCs with χράομαι \emph{kʰraomai} and \emph{utor} (Section
\ref{sec:ma:4}), together with their distribution by text type and author (Section
\ref{sec:ma:5}); (ii) to observe the semantic domains in which χράομαι
\emph{kʰraomai} and \emph{utor} operate (Section~\ref{sec:ma:6}); and
(iii) to compare these SVCs in Greek and Latin (Sections \ref{sec:ma:4}--\ref{sec:ma:7}).
However, my approach to SVCs is different from that of other contributors of this volume
(\sectref{sec:ma:2}).\footnote{Squeri (this
  volume) takes into account collocations with χράομαι
  \emph{kʰraomai} where the noun functions as an object complement,
  whereas I discard them, and Veteikis (this volume) takes into account
  collocations with adjectives, while I only include in my analysis verb + noun
  constructions.} To support my analysis, I have used data from two different corpora,
one for each language (Section~\ref{sec:ma:3}). In Section~\ref{sec:ma:8}, I provide a
summary of my conclusions. 

\section{Definition of support-verb constructions}\label{sec:ma:2}

Several different definitions for SVCs have been proposed in the literature. In addition,
support verbs (SVs henceforth) have been referred to differently in different languages and the
description of their characteristics diverges depending on the language being analysed
(\cite[27]{banos_latin_2022}). For example, the German concept of \emph{Funktionsverb}
`functional verb' is broader than the English \emph{light verb}, the French \emph{verbe
  support} and the Spanish \emph{verbo de apoyo}. In this contribution, I use the term
\emph{support verb} in the more restricted sense (\cite{vives_aspect_1984};
\cite{alonso_ramos_construcciones_2004}) and \emph{support-verb extension} in the broader
sense (\cite{banos_consilium_2014}), that is, collocations that have many characteristics
in common with SVCs, but also some distinct properties. The verbs referred to by these
expressions are different from auxiliaries in several different ways, but the more obvious
is perhaps that auxiliaries are typically used in combination with another verb
(\emph{cf}. verbal periphrases, \emph{e.g.} in \citealt{bentein_verbal_2016}). For the
purpose of data organisation, I consider SVCs to be a set of different types of verb-noun
collocations arranged around a prototype.

For a better understanding of this concept, it is necessary to start with a general
definition of \emph{collocation.}\footnote{That is the definition of \emph{collocation}
  that I propose in this paper. Since the appearance of this concept, it has been
  understood differently by different researchers. Initially, for instance, collocations
  were merely considered frequent word co-occurrences
  \parencites{firth_papers_1964}[276]{halliday_categories_1961}. However, it was later
  pointed out that the high co-occurrence of certain items in a sentence was in fact due
  to the lexical, syntactic, and semantic restrictions of a certain word, which acted as a
  marker for the higher probability of other items, i.e. arguments, prepositions,
  conjunctions, etc. (\cite{harris_theory_1976}).} Collocations are lexically restricted
word combinations that differ from free word combinations because they are fixed in the
linguistic norm, and from idioms\is{idiom} because they allow for syntactic modification
\parencites[66]{corpas_manual_1997}[20-21]{alonso_ramos_construcciones_2004}. In other
words, collocations are at a middle point of a \emph{continuum} between free constructions
and idioms.\is{idiom} In a free construction, all the words are chosen by the speaker according to their
meaning, and its semantics is a result of the combination of the meanings of all these
words. By contrast, the meaning of an idiom\is{idiom} does not result from the addition of the meanings of its
parts, but rather from social consensus, whereby a combination of words expresses a
meaning unrelated to that which the words convey separately. Collocations are partially
restricted word combinations: when a speaker wants to say that they have strolled or
walked for leisure, they might choose the noun \emph{walk} to build the sentence, but it
is the lexical restrictions of \emph{walk} that impose the use of the verb \emph{to take}
in \emph{I took a walk}. In other words, it is unidiomatic\is{idiom} to say *\emph{I grabbed a walk}
or *\emph{I did a walk}.

What characterises collocations is that one element (\emph{base word})\is{base word} is freely chosen by
the speaker, while the other (\emph{collocate})\is{collocate} is determined by the base word. For
instance, \emph{attention} is \emph{paid} in English, but \emph{gifted} in German
(\emph{Aufmerksamkeit schenken}), and \emph{made} in French (\emph{faire attention}). These
phrases mean the same in all three languages, but each one takes a different verb to
express the same idea. This means that the noun is the semantically chosen element in the
sentence, whereas the verb is lexically selected by the noun. That being said, there are
several different types of collocations (\cite{banos_construcciones_2018}). In some cases,
both elements --- the base and the collocate\is{collocate} --- retain their original meaning (lexical
collocations, such as \emph{to play guitar/piano}), whereas in others, one of the elements
undergoes some kind of semantic change, be it de-semanticization or alteration of its
original meaning (functional collocations, such as \emph{to give a hug}). Another
restriction relates to the lexical specificity of the verb (collocate)\is{collocate}: collocates may
indeed be very widely applicable with a wide set of bases (in general
collocations\emph{,} such as \emph{to have a dream}) or be restricted to a certain set of
bases (in specific collocations, such as \emph{to commit a crime}).

SVCs are necessarily functional collocations, but they may be either specific or general.
For instance, the verb \emph{to give} has a very vague or general meaning, e.g.
\emph{to give a hug}, but the verb \emph{to commit}, by contrast, may only be used in the
context of crime. This distinction is relevant because it affects the interpretation of
the data. If one of the characteristics that is typically used for the identification of
collocations is absolute frequency, but a characteristic of specific collocations is
lexical restriction, then there is a methodological caveat: not
all the collocations are equally frequent and therefore less frequent word combinations also
deserve a collocational analysis, even if they do not have a high absolute frequency.

SVCs are a type of verb-noun collocation which consists of a support verb and a
predicative noun. A complete definition of the concept is provided in
\citet[7]{mendozar_cruz_causatividad_2015}:

\begin{quote}

  {[}SVCs are{]} verb-noun phrases in which the predication is largely borne
  by the noun, an event noun, and in which the verb, devoid of its nuclear function,
  becomes a `predicator' of the noun, providing it, on the one hand, with the grammatical
  features which the noun lacks (tense, mood, voice, etc.) and, on the other hand, with the
  syntactic slots which are required for its semantic arguments (my
  translation).\footnote{Original text: `Sintagmas verbo-nominales en los que el peso de
    la predicación recae sobre el sustantivo, un nombre de evento, y donde el verbo,
    depuesto de su función nuclear, cumple el papel de «actualizador» del nombre,
    proporcionándole, por un lado, los accidentes gramaticales (tiempo, modo, voz, etc.)
    de los que la morfología nominal carece y, por otro, las posiciones sintácticas
    necesarias para la expresión de sus argumentos semánticos.'}

\end{quote}

This accounts for prototypical SVCs, that is, \citet{alonso_ramos_construcciones_2004}'s
\emph{construcciones con verbo de apoyo} or \citet{vives_aspect_1984}'s \emph{constructions
  à verbe support}. The nature of the nouns in these collocations has  been subject to
debate (\cite[115-129]{alonso_ramos_construcciones_2004}). Before
\citet{alonso_ramos_construcciones_2004}, the terms \emph{abstract},
\emph{de-verbal} or \emph{event} were used to describe them. However, none of these terms
account for the whole range of nouns that can be found taking part in SVCs: there are SVCs
with non-abstract, non-de-verbal, and un-eventive nouns (\emph{e.g.} \emph{to give
  ear}). \citet[115]{alonso_ramos_construcciones_2004} argues that any noun with actants
(≈ arguments) must be considered predicative. The difficulty here lies in the fact that
some nouns can be forced into an SVC and assigned actants despite them not originally
taking them (see Squeri (this volume)). This is the perspective I adopt
in this contribution.

These constructions are often identified and described by means of batteries of tests
\parencites{langer_linguistic_2004,jimenez_lopez_support_2016}. So, for instance, SVCs
have a higher absolute frequency as opposed to free constructions which are usually
less frequent. They can be easily replaced by a simplex verb\is{simplex verb} without having their meaning
majorly altered --- e.g. \emph{to give a hug} ≈ \emph{to hug} ---, even though they can be
used to add certain nuances that the simplex verb on its own cannot convey, such as
intensification or iteration (\cite{jimenez_lopez_support_2016}).\footnote{Contrast for
  instance \emph{He walked} ≠ \emph{He took several walks a week.} This iteration cannot
  be conveyed by the verb alone. If a speaker tried to communicate the same, they might
  utter something like \emph{He kept walking}, but that is a durative predicate, not an
  iterative one.} They can have the verb removed without majorly altering the meaning of
the sentence (nominalisation)\footnote{In other words, the semantics of the predicate are
  not altered if it is nominalised. Removing the verb implies deleting the grammatical
  information it conveys, such as tense, mood, etc., but the ensemble of words conveys the
  same meaning as the original sentence.} --- e.g. \emph{Mary gave a hug to Paul} ≈
\emph{Mary's hug to Paul} --- and, very importantly, they have a subject that is
co-referential with the first argument of the base noun. That is, in an example such as
\emph{Mary took a walk around Camden}, the subject of \emph{took} is the same entity as
the first argument (i.e. the Agent) of \emph{walk}.

However, less prototypical SVCs may behave differently and still have a noun predicated by
an SV. These are what I call SV-extension\is{support-verb extension} constructions (SVECs henceforth). \footnote{These
  less prototypical SVCs have already been addressed in the literature
  \parencites{anscombre_morphologie_1995,gross_predicats_1996,gross_pour_2004,gross_fonction_1998,banos_consilium_2014}.}
For instance, causative constructions\is{causativity} are incapable of complying with the
co-referentiality\is{co-referentiality} criterion because the subject of the verb is necessarily a Causer or a
Force, and the first argument of the noun is often a different entity. So, for instance,
CG φόβον ἔχω \emph{pʰobon ekʰo} `I have fear' is a
prototypical SVC because the subject of ἔχω \emph{ekʰo} `I have'
coincides with the Experiencer of φόβος \emph{pʰobos} `fear'. However,
in CG φόβον ποιέω \emph{pʰobon poieo} `I make/cause/provoke fear', the
Experiencer of φόβος \emph{pʰobos} is different from the entity which
causes it, that is, the subject of ποιέω \emph{poieo}.

These causative/non-causative pairs\is{causativity} are what have been called \emph{constructions
  inverses} (\cite{gross_cas_1982}) or \emph{converses} (\cite{gross_les_1989}) in the
literature. This can be exemplified with \citet{gross_cas_1982}'s case-study of Fr.
\emph{donner} `to give' and \emph{recevoir} `to receive', which convey opposed diathetical
meanings. Most importantly, G. Gross' paper reaches three conclusions crucial to this
contribution: (i) the notion of SV is broader than generally assumed and includes verbs
which are not entirely devoid of meaning; (ii) SVs have a vague meaning, which can be
deduced from the arrangement of its complements; and (iii) the meaning of an SV can also be
identified by comparing it with other SVs with which it alternates.

With regard to this last point, \citet{jimenez_lopez__2021} case study of CG γίγνομαι
\emph{gignomai} `to come to be' is most illustrative: she concludes that γίγνομαι
\emph{gignomai} + noun SVCs perform as the lexical passive of ποιέομαι \emph{poieomai} `to
make' + noun SVCs. In other words, the comparison between ποιέομαι \emph{poieomai} and
γίγνομαι \emph{gignomai} allows her to elucidate the meaning of γίγνομαι \emph{gignomai}
as an SV (see \citet{vives_cuesta__2021} for another case study). This is the
methodological approach I have taken in my attempt to establish the properties of CG
χράομαι \emph{kʰraomai} and Lat. \emph{utor}.

The same happens with aspectual\is{aspect} or perspectival SVECs.\footnote{The term
  \emph{perspective} refers to the noun which takes the subject position, which has
  pragmatic implications in the discourse. For instance, it is not the same to say CG ἔχω
  φόβον \emph{ekʰo} \emph{pʰobon} `I have/feel fear'
  as φόβος μ'ἔχει \emph{pʰobos m'ekʰei} `fear
  has/owns me/I am controlled by fear'.} When the noun is the subject of the verb, such as
in CG φόβος ἐμπίπτει \emph{pʰobos empiptei} `fear falls (upon
someone)/someone starts to feel fear', it is impossible to have co-referential
arguments. This phrase cannot be replaced by a simplex verb\is{simplex verb} because Greek, as far as I
know, does not have a verb to convey the meaning of `to start to feel fear'. Instead,
φόβος ἐμπίπτει \emph{pʰobos empiptei} would need to be replaced by a
different kind of periphrasis, \emph{e.g.} CG ἄρχομαι φοβεῖσθαι
\emph{arkʰomai pʰobeistʰai} `I start
to feel fear'. Since the verb is not entirely devoid of its
original meaning because it possesses lexical aspect\is{aspect}, it cannot be suppressed without any
semantic consequences: the noun alone does not convey the aspectual meaning of ἐμπίπτει
\emph{empiptei} `it falls/begins'. However, the close relationship of SVECs to SVCs seems
undeniable, particularly if we observe the characteristics of the nouns and how they
interact with the verbs they take, that is, their collocational patterns.\is{collocational pattern} For these
reasons, we consider SVECs a sub-type of SVCs which lie closer to free constructions on
the continuum from the latter to idioms.\is{idiom}

\section{Quantitative data}\label{sec:ma:3}

In the process of data collection, I have handled two corpora, one for  Greek 
--- 1,082,905 words in total --- and the other for Latin --- 2,534,029 words in total.
The Greek corpus has been searched by means of the \emph{Thesaurus Linguae Graecae} database \parencite{pantelia_thesaurus_nodate} and the Latin corpus
has been taken from the \emph{Corpus Corporum} (\emph{Latinitas Antiqua}) database
\parencite{roelli_corpus_nodate}, both of which allow for semi-automated
searches.\footnote{The Greek corpus includes the following works: Aeschylus (\emph{Persae,
    Septem contra Thebas}), Sophocles (\emph{Oedipus Tyrannus, Antigone}), Euripides
  (\emph{Medea, Electra}), Aristophanes (\emph{Acharnenses, Nubes, Vespae, Pax,
    Thesmophoriazusae}), Xenophon (\emph{Hellenica, Memorabilia, Anabasis, Cynegeticus}),
  Thucydides (\emph{Historiae}), Herodotus (\emph{Historiae}), Lysias (\emph{De caede
    Eratosthenis, Contra Simonem, In Eratosthenem, In Agoratum}), Demosthenes (\emph{De
    falsa legatione, Adversus Leptinem, In Midiam, Adversus Androtionem}), Andocides
  (\emph{De mysteriis, De reditu suo}), Plato (\emph{Euthyphro, Apologia Socratis, Phaedo,
    Symposium, Phaedrus, Gorgias, Ion, Respublica}) and Aristotle (\emph{Ethica
    Nicomachea, Historia animalium, Politica}). The Latin corpus includes all the works in
  the \emph{Corpus Corporum} by the following authors: Cicero, Caesar, Catullus, Martial,
  Livy, Plautus, Sallust, Tacitus and Terence.} In total, I have analysed 1,003 tokens of
CG χράομαι \emph{kʰraomai} --- 0.93‰ of the sample --- and 1,237 of
Lat. \emph{utor} --- 0.49‰ of the sample. Out of these occurrences, 457 --- 45.56\% of the
total tokens of χράομαι \emph{kʰraomai}--- included χράομαι
\emph{kʰraomai} as an SV, and 598 --- 48.34\% of the total tokens of
\emph{utor} --- included \emph{utor} as an SV. This means that, despite \emph{utor} --- be
it as a full verb or an SV --- being only half as frequent in Latin as χράομαι
\emph{kʰraomai} is in Greek --- on a rate of absolute frequency of 0.49‰
in Latin to 0.93‰ in Greek ---, both verbs are used as SVs with a similar frequency ---
48.34\% of the tokens of \emph{utor} and 45.56\% of the tokens of χράομαι
\emph{kʰraomai}. In the following sections, I compare both SVs to
explain their similarities and differences.

Three types of constructions have been discarded in this analysis. In the first one,
χράομαι \emph{kʰraomai} or \emph{utor} do not govern any complements at
all or govern a {[}/+human/{]} complement. So, for instance, \emph{utor} might be used in
the sense of `to get along (with someone)'. These cases cannot be accounted for as
SVCs, since one of the requirements for the existence of an SVC is the combination of the
verb with a predicative noun.

The second type of construction is where either χράομαι \emph{kʰraomai}
or \emph{utor} take a non-predicative object. So, for instance, in CG χράομαι ἵππῳ
\emph{kʰraomai ʰippo\textsuperscript{i}} `to use/ride
a horse,' the noun is not predicative, and therefore the construction is not considered an
SVC. However, certain nouns can be \emph{forced} into a predicative structure and may
acquire complements in the process, in which case the construction has been considered.
For instance, in CG χράομαι τροφῇ \emph{kʰraomai
  tropʰe\textsuperscript{i}} `to use food/to eat' an Agent is imposed
upon τροφῇ \emph{tropʰe\textsuperscript{i}} `food', which is
co-referential with the subject of χράομαι \emph{kʰraomai}. A different
analysis is not possible because χράομαι τροφῇ \emph{kʰraomai
  tropʰe\textsuperscript{i}} is never found with the sense of `to feed
someone else' due to the morphosyntactic characteristics of the verb.

Χράομαι \emph{kʰraomai} is a \emph{media tantum} verb, i.e., it
is only used in the middle voice. This has some syntactic implications, such as its
inability to function as a causative verb\is{causativity} or to be passivised. This, in turn, means that
the \emph{fed} entity is always the subject of χράομαι \emph{kʰraomai}.
This collocation is so relevant that an Athenian author indicates that, in Athens, χράομαι
\emph{kʰraomai} is sometimes used with the meaning `to eat', even when
τροφῇ \emph{tropʰe\textsuperscript{i}} is not explicitly mentioned (\iwi{Xenophon,
\emph{Memorabilia} 3.14.6}).\footnote{I understand this case as the result of semantic change in
  the verb after the collocation had become ubiquitous in language. On this type of
  semantic change, see \citet{jimenez_martinez_collocations_nodate}.} When χράομαι
\emph{kʰraomai} is used without τροφῇ
\emph{tropʰe\textsuperscript{i}}, it has been discarded because it
cannot be considered an SVC. However, the constructions with τροφῇ
\emph{tropʰe\textsuperscript{i }}are accounted for as SVCs because the
noun is made predicative. This is the procedure I have followed with all the data (see
\citet{madrigal_acero_dataset_2024}).

Thirdly, I have not considered as SVCs the predicates in which the base noun occupied the
position of a third argument --- an object complement --- rather than a second
argument.\footnote{In these cases, χράομαι \emph{kʰraomai} and
  \emph{utor} are translated as `to use something as something', \emph{e.g.} Lat.
  \emph{his testibus} {[}\ldots{]} \emph{uteretur} `that he uses them as witnesses'
  (\iwi{Caesar, \emph{Commentarii belli civilis}, 3,105,1}). These constructions seem to be very
  close to the basic meanings of these verbs: since desematicization is not very clear, I
  have opted to leave them out of my survey. However, there are examples of other SVs more
  similar to SVEs or \emph{Funktionsverben} where the base noun is the third argument of
  the verb, e.g. Lat. \emph{tenere aliquid memoria} `to remember something/to keep
  something in memory'.} This decision is based on the ambivalence of χράομαι
\emph{kʰraomai} and \emph{utor}: since both are clearly not as
de-semanticised as other SVs, such as ποιέομαι \emph{poieomai} or \emph{facio} `to make',
the boundaries between regular uses of these verbs and their uses as SVs are not always
clear. However, I have observed that, in the cases where the base noun is the third
argument rather than the second, it is the verb which conveys the predicative force of the
phrase, rather than the noun. For some examples, see \iwi{Plato, \emph{Euthyphro} 6e} and \iwi{Cicero,
\emph{In Q. Caecilium Nigrum oratio} 9}.

I have considered regular SVCs the instances in which the noun is in the genitive, rather
than the accusative, when it is introduced by nouns such as CG εἶδος \emph{eidos}
`kind', CG γένος \emph{genos} `type', Lat. \emph{copia} `abundance', Lat. \emph{genus}
`type', etc. This is what \citet[55-60]{koike_colocaciones_2001} calls \emph{complex
  collocations}\is{complex collocation}, that is, a combination of two collocations in a single phrase. For some
examples, see \iwi{Xenophon, \emph{Cynegeticus} 9.7}; \iwi{Aristotle, \emph{Politics} 1342a}; \iwi{Cicero,
\emph{Academici libri ab ipso Cicerone postea retractati} 2,16}; \iwi{Cicero, \emph{Pro A.
  Cluentio Habito oratio} 45}.

\section{Properties and functions of χράομαι \emph{kʰraomai} and
  \emph{utor}}\label{sec:ma:4}

As synonyms in languages with many common characteristics, CG χράομαι
\emph{kʰraomai} and Lat. \emph{utor} behave very similarly. However,
they also diverge in some points. In this section I review some of the most relevant
points to understand their behavior as SVs.

\subsection{Predicative frames}\label{sec:ma:4:1}

The predicative frame (PF henceforth) of Lat. \emph{utor} as a full verb has already been addressed
by \citet{alonso_fernandez_estructuras_2010}. In her paper, she suggests a single PF for
\emph{utor} due to the characteristics of the nouns which it takes as an
object.\footnote{In \citet{alonso_fernandez_estructuras_2010} paper, ``x'' means that slot
  can be filled by a noun without any lexical restrictions.}

\begin{center}
  \emph{utor}: {[}/+human/{]}\textsubscript{Agent/Experiencer}
  {[}/x/{]}\textsubscript{Instrument}
\end{center}

It is not reasonable to suggest a different PF for \emph{utor} + {[}/+abstract/{]} because
it is a metaphorical extension of its literal use with a {[}/+concrete/{]} object. This is
self-evident in cases of coordination with {[}/± abstract/{]} nouns, see
\xref{ex:ma:1}.\footnote{Although in this particular case the use of the abstract
  \emph{copia} `abundance' might facilitate the coordination of the objects, the base in
  the collocation \emph{aquae copia} `abundance of water' is in fact \emph{aquae} `water',
  which is a concrete noun.}

% 1

\ea\label{ex:ma:1}

\gll \emph{at} \textit{Caesaris} \emph{exercit-us} \emph{cum} \emph{optim-a} \textbf{\itshape
  ualetudin-e} \emph{summ-a=que} \textbf{\itshape aqu-ae} \textbf{\itshape copi-a}
\textbf{\itshape
    ute-ba-tur}, \textit{tum}\ldots{}\\
but Caesar-\textsc{gen.sg} army-\textsc{nom.sg} because best-\textsc{abl.sg}
health-\textsc{abl.sg} greatest-\textsc{abl.sg}=and water-\textsc{gen.sg} amount-\textsc{abl.sg} enjoyed-\textsc{impf-3sg} then\ldots{}\\
\glt `But Caesar's army, since it enjoyed the best health and the greatest amount of
water, then, \ldots' \\
\hspace*{\fill}(\iwi{Caesar, \emph{Commentarii belli civilis} 3.49.5})

\z

This is unusual behavior for an SV, which is expected to coordinate only objects showing
the same characteristics, for instance, predicative nouns can be coordinated with other
predicative nouns, but not with concrete nouns. This is the so-called zeugma test\is{zeugma test}, on which
there is disagreement in the literature (\cite{langer_linguistic_2004}). However,
\emph{utor} might allow these zeugmata precisely due to its single PF and the metaphorical
conceptualisation of the nouns. The same happens with χράομαι
\emph{kʰraomai}:

\begin{center}
  χράομαι \emph{kʰraomai}:
  {[}/+human/{]}\textsubscript{Agent/Experiencer} {[}/x/{]}\textsubscript{Instrument}
\end{center}

The same PF can be proposed for the Greek verb, which also takes
coordinated objects with different lexical characteristics, see (\ref{ex:ma:2}):

% 2

\ea\label{ex:ma:2}

\glll οὐ \textbf{σπονδ-ῇ} \textbf{χρέω-νται}, οὐκὶ \textbf{αὐλ-ῷ}, οὐ \textbf{στέμμα-σι}, οὐκὶ \textbf{οὐλ-ῇσι.}\\
 \textit{u} \textit{spond-e\textsuperscript{i}} \textit{kʰreo-ntai} \textit{uki} \textit{aul-o\textsuperscript{i}} \textit{u} \textit{stemma-si} \textit{uki} \textit{ul-e\textsuperscript{i}si}\\
\textsc{neg} libation-\textsc{dat.sg} use-3\textsc{pl} \textsc{neg} flute-\textsc{dat.sg} \textsc{neg} garlands-\textsc{dat.pl} \textsc{neg} barley.corns-\textsc{dat.pl}\\
\glt `Neither do they perform libations, or use flutes, garlands or barley-corns.' \\
\hspace*{\fill}(\iwi{Herodotus, \emph{Histories} 1.132.4})

\z

This can be explained from a cognitive perspective. Collocations constitute a single unit
or \emph{chunk} in the speaker's mind, whereas an object governed by a verb constitutes
two separate units, e.g. a prototypical transitive predicate. This, in turn,
implies that due to its more frequent use and its fixation in language, the noun that
participates in a collocation with Lat. \emph{utor} or CG χράομαι
\emph{kʰraomai} is more readily available in the speaker's mind than
other types of objects (\cite[271]{bybee_frequency_2001}). This availability is supported
by the preferential position given to Lat. \emph{ualetudine} `health' and CG σπονδῇ
\emph{sponde\textsuperscript{i}} `libation' in \xref{ex:ma:1} and \xref{ex:ma:2}: the nouns which take part in a
collocation appear first, whereas the prototypical objects appear afterwards.

\subsection{Batteries of tests for support-verb constructions}\label{sec:ma:4:2}

Regarding the battery of tests\is{battery of tests} proposed for SVCs
\parencites{langer_linguistic_2004,jimenez_lopez_support_2016}, such as frequency,
nominalisation, pronominalisation, etc., the collocations I have identified comply with
them (see Section~\ref{sec:ma:2}). The most important test is probably that for the
co-referentiality\is{co-referentiality} of the verb's subject and the first argument of the noun.

Surprisingly, this is the case in Greek even with meteorological nouns, see \xref{ex:ma:3a}.
Greek meteorological verbs can sometimes take a subject, and, for this reason, it is also
possible for SVCs with meteorological nouns to take a subject, which is co-referential
with the first argument of the noun --- ἡ γῆ \emph{ʰe ge} `the earth'. What is remarkable in this case is that, in Latin, \emph{utor tempestate} `I
face/fight against a storm', behaves differently from CG χρᾶται νιφετῷ
\emph{kʰratai nipʰetoⁱ} `it snows'. \emph{Utor} takes
a personal subject: \emph{nos} `us' in example \xref{ex:ma:3b}. Interestingly, the subject
in this case functions as an Experiencer, rather than an Agent, which aligns with
\emph{utor} being used as an SV when combined with emotion nouns, as I show in
Section~\ref{sec:ma:6} below. The function of Experiencer can also be attributed to ἡ γῆ
\emph{ʰe ge} `the earth' in \xref{ex:ma:3a} despite it not being
{[}+human{]}.

% 3

\ea\label{ex:ma:3}

\ea\label{ex:ma:3a}

\glll ὕ-εται γὰρ ἡ γ-ῆ αὕτ-η τοῦ χειμῶν-ος πάμπαν ὀλίγῳ, \textbf{νιφετ-ῷ} δὲ {τὰ πάντα} \textbf{χρᾶ-ται.}\\
 \textit{ʰy-etai} \textit{gar} \textit{ʰe} \textit{g-e} \textit{ʰaut-e} \textit{tu} \textit{kʰeimon-os} \textit{pampan} \textit{oligo\textsuperscript{i}}
\textit{nipʰet-o\textsuperscript{i}} \textit{de} \textit{{ta} \textit{panta}} \textit{kʰra-tai}\\
rain-\textsc{3sg} \textsc{conj} the.\textsc{nom.sg} land-\textsc{nom.sg} that-\textsc{nom.sg} the.\textsc{gen.sg} winter-\textsc{gen.sg} altogether a.little snow-\textsc{dat.sg}
\textsc{prt} always use-3\textsc{sg}\\
\glt `For it rains a little altogether in that region during the winter, but it always \textbf{snows}.' \\
\hspace*{\fill}(\iwi{Herodotus, \textit{Histories} 4.50.10})

\ex\label{ex:ma:3b}

\gll \emph{ita} \emph{usque} \emph{advers-a} \textbf{\itshape tempestat-e}
\textbf{\itshape us-i} \textbf{\itshape su-mus}\\
so continuously adverse-\textsc{abl.sg} storm-\textsc{abl.sg} used-\textsc{nom.pl} be-\textsc{1pl}\\
\glt `So continuously \textbf{did we} \textbf{face an adverse storm}.' \\ 
\hspace*{\fill}(\iwi{Terence, \emph{Hecyra} 423})

\z

\z

SVCs can be distinguished from idioms\is{idiom} by means of tests that look for morphological and
syntactic modifications. One of these is the allowance of number variation --- e.g.
  \textit{I took a walk} vs. \emph{I take walks regularly} --- or the possibility of adding
complements. For instance, a common idiom in English is \emph{to pull somebody's leg}. One
of the reasons this is an idiom is that sentences such as *\emph{We pulled Mary's legs} or
*\emph{Mary's leg that we pulled} are in fact unidiomatic (see \cite{melcuk_general_2023}
for this idiom). However, SVCs do admit pluralisation \xref{ex:ma:4a} and relativisation
\xref{ex:ma:4b}. These examples do not prove \emph{per se} that the phrases in bold are
SVCs, but they show that Lat. \emph{dirimere iras} `to put an end to rage' and CG τίθημι
νόμον \emph{titʰemi nomon} `to impose a law' are not idioms.


% 4

\ea\label{ex:ma:4}

\ea\label{ex:ma:4a}

\gll \emph{tum} \emph{Sabin-ae} \emph{mulieres,} \ldots{} \textbf{\itshape dirim-ere} \textbf{\itshape ir-as}\ldots{}\\
then Sabine-\textsc{nom.pl} women-\textsc{nom.pl} ~ finish-\textsc{inf.} wrath-\textsc{acc.pl}\\
\glt `Then the Sabines, \ldots{} put an end to {[}their{]} wrath \ldots' \\
\hspace*{\fill}(Livy, \emph{Ab
  Urbe condita} 1,13,2)

\ex\label{ex:ma:4b}

\glll ἐπειδὴ <δ'> ἀν-ε-γράφ-ησαν, \textbf{ἐ-θέ-μεθα} \textbf{νόμ-ον}, \textbf{ᾧ} \textbf{πάντ-ες} \textbf{χρῆ-σθε.}\\
 \textit{epeide} \textit{<d'>} \textit{an-e-grapʰ-esan}
\textit{e-tʰe-metʰa} \textit{nom-on} \textit{\textbf{ʰo\textsuperscript{i}}} \textit{\textbf{pant-es}} \textit{\textbf{kʰre-stʰe}}\\
after \textsc{prt} in-\textsc{pst-}write-\textsc{3pl.pass} \textsc{pst-}put-\textsc{1pl}
law-\textsc{acc.sg} \textsc{rel.dat.sg} \textsc{all-nom.pl} \textsc{use-2pl}\\
\glt `After they were engraved, we \textbf{established} a \textbf{law} \textbf{by which} you all \textbf{abide}.' \\ 
\hspace*{\fill}(Andocides, \emph{De mysteriis} 1.85)

\z

\z


Nevertheless, corpus linguistics requires a specific treatment of these tests, since it
remains a possibility that morphosyntactic variation in a phrase existed but is not
attested in the corpus (\cite{herring_methodologies_2000}). In these cases, I have
resorted to different criteria for the identification of SVCs: (i) Is a certain verb
employed as an SV with other nouns? (ii) What is the syntactic structure of the phrase?
This means that the data I address in Sections \ref{sec:ma:3} and \ref{sec:ma:5} is open
to a certain range of error, but some aspects of historical languages will forever remain
unknown to us.





\subsection{Alternation of χράομαι \emph{kʰraomai} and \emph{utor} with other
  verbs}\label{sec:ma:4:3}

In some contexts, χράομαι \emph{kʰraomai} and \emph{utor} behave as
prototypical SVs and hence alternate with certain SVEs. These
SVEs may be used to convey aspectual\is{aspect}, see \xxref{ex:ma:5}{ex:ma:6}, or diathetic, see
\xxref{ex:ma:7}{ex:ma:8}, information, and their contrast with χράομαι \emph{kʰraomai} and \emph{utor}
elucidates the syntactic and semantic nuances that they convey. In \xref{ex:ma:5} there is a
clear contrast between χρῆσθαι ἔργοις \emph{kʰrestʰai
  ergois} `to make representations' and ἀφεῖσθαι τῶν ἔργων
\emph{apʰeistʰai ton ergon} `to stop making
representations'.


% 5

\ea\label{ex:ma:5}

\glll διὰ τοῦτο χρὴ νέ-ους μὲν ὄντ-ας \textbf{χρῆ-σθαι} \textbf{τοῖς} \textbf{ἔργ-οις,} \textbf{πρεσβυτέρ-ους} \textbf{δὲ} γεν-ομέν-ους \textbf{τῶν} μὲν \textbf{ἔργ-ων} \textbf{ἀφεῖ-σθαι}\\
 \textit{dia} \textit{tuto} \textit{kʰre} \textit{ne-us} \textit{men} \textit{ont-as}
\textit{\textbf{kʰre-stʰai}} \textit{\textbf{tois}} \textit{\textbf{ergois}} \textit{\textbf{presbyter-us}} \textit{\textbf{de}} \textit{gen-omen-us} \textit{ton} \textit{men} \textit{erg-on} \textit{apʰei-stʰai}\\
due.to this must young-\textsc{acc.pl} \textsc{prt} be.\textsc{ptcp}-\textsc{acc.pl} use-\textsc{inf} the work-\textsc{dat.pl} older-\textsc{acc.pl} \textsc{prt}
become-\textsc{ptcp-acc.pl} the \textsc{prt} work-\textsc{gen.pl} leave-\textsc{inf}\\
\glt `For this reason, teenagers must \textbf{make} {[}musical{]} \textbf{representations} while they are young and \textbf{abandon them} when they grow older.' \\
\hspace*{\fill}(\iwi{Aristoteles, \emph{Politics} 1340b})

\z

In short, ἀφεῖσθαι \emph{apʰeistʰai} `to give up' has
a terminative aspect\is{aspect}, while χρῆσθαι \emph{kʰrestʰai}
`to use' does not. The same happens in \xref{ex:ma:6}. \emph{Utamur ira} `we are angry' is
neutral in aspect, whereas \emph{dirimere iras} `to put an end to anger' is terminative.

% 6

\ea\label{ex:ma:6}

\ea\label{ex:ma:6a}


\gll \emph{verum} \emph{es-se} \emph{inscit-i} \emph{cred-imus} \emph{ne} \emph{ut}
\emph{iust-a} \textbf{\textit{ut-amur}} \textbf{\textit{ir-a}}\\
true be-\textsc{inf} fool-\textsc{nom.pl} believe-\textsc{1pl} \textsc{conj.neg} \textsc{conj} rightful\textsc{-abl.sg} use-\textsc{1pl} anger-\textsc{abl.sg}\\
\glt `\ldots{} We fools believe that it is true, in order not to be angry rightfully.' \\
\hspace*{\fill}(\iwi{Plautus, \emph{Truculentus} 192})

\ex\label{ex:ma:6b}

\gll \emph{tum} \emph{Sabin-ae} \emph{mulieres,} \ldots{} \textbf{\itshape dirim-ere} \textbf{\itshape ir-as}\ldots{}\\
then Sabine-\textsc{nom.pl} women-\textsc{nom.pl} ~ finish-\textsc{inf.} wrath-\textsc{acc.pl}\\
\glt `Then the Sabines, \ldots{} put an end to {[}their{]} wrath \ldots' \\
\hspace*{\fill}(\iwi{Livy, \emph{Ab urbe condita} 1.13.2}) (=
example \ref{ex:ma:4a})

\z

\z

Examples \xxref{ex:ma:7}{ex:ma:8} illustrate another aspect of these alternations. While
ἐθέμεθα νόμον \emph{etʰemetʰa nomon} `to establish a
law' and \emph{quod {[}consilium{]} dederit} `{[}the advice{]} that he gave' are causative
SVECs\is{causativity}, the contrasting constructions with χράομαι \emph{kʰraomai} and
\emph{utor} are neutral from a diathetic perspective.

% 7

\ea\label{ex:ma:7}

\glll ἐπειδὴ <δ'> ἀν-ε-γράφ-ησαν, \textbf{ἐ-θέ-μεθα} \textbf{νόμ-ον}, \textbf{ᾧ} \textbf{πάντ-ες} \textbf{χρῆ-σθε.}\\
 \textit{epeide} \textit{<d'>} \textit{an-e-grapʰ-esan} \textit{e-tʰe-metʰa} \textit{nom-on} \textit{\textbf{ʰo\textsuperscript{i}}} \textit{\textbf{pant-es}}
\textit{\textbf{kʰre-stʰe}}\\
after \textsc{prt} on-\textsc{pst-}write-\textsc{3pl.pass} \textsc{pst-}put-\textsc{1pl}
law-\textsc{acc.sg} \textsc{rel.dat.sg} all-\textsc{nom.pl} use-\textsc{2pl}\\
\glt `After they were engraved, we \textbf{established} a \textbf{law} \textbf{by which} you all \textbf{abide}.' \\
\hspace*{\fill}(\iwi{Andocides, \emph{De mysteriis} 1.85}) (= example \ref{ex:ma:4b})

\z

% 8

\ea\label{ex:ma:8}

\gll \emph{is} \textbf{\itshape quod} \emph{mihi} \textbf{\itshape ded-erit} \emph{de}
\emph{hac} \emph{r-e} \textbf{\itshape consili-um,} \emph{id} \textbf{\itshape sequ-ar}\\
he \textsc{rel.acc.sg.n} me.\textsc{dat.sg} give-\textsc{3sg.prf.subj} about
this.\textsc{abl.sg} thing-\textsc{abl.sg} advice-\textsc{acc.sg.n} it follow-\textsc{1sg.prs.subj}\\
\glt `I will follow the advice that he gave me concerning this matter.' \\ 
\hspace*{\fill}(\iwi{Terence, \emph{Hecyra} 461})

\z

In some other contexts there is no apparent alternation other than the lexical specificity
of χράομαι \emph{kʰraomai} and \emph{utor} in contrast with a more
general SV. This means that they also behave as what has sometimes been called
\emph{appropriated} or \emph{specific} SVs, that is, less frequent and less desemanticised
SVs that are usually prescribed by the rules of style, see \xxref{ex:ma:9}{ex:ma:10}
(\citealt[100–107]{gross_pour_2004} \citealt{alonso_ramos_construcciones_2004}; see also
Section~\ref{sec:ma:2}). This is made clear by their alternation with more prototypical
SVs, such as ἔχω \emph{ekʰo} \xref{ex:ma:9a} and \emph{habere}
\xref{ex:ma:10a}. In short, ἔχω ὀργήν \emph{ekʰo orgen} ≈ χράομαι ὀργῇ
\emph{kʰraomai orge\textsuperscript{i}} `to have/use anger' or `to be
angry', see \xxref{ex:ma:9a}{ex:ma:9b}.

% 9
\largerpage[2]
\ea\label{ex:ma:9}

\ea\label{ex:ma:9a}

\glll \textbf{ὀργ-ὴν} γὰρ αὐτ-οῖς \ldots{} πολλ-ὴν \textbf{ἔχ-ει.}\\
 \textit{org-en} \textit{gar} \textit{aut-ois} ~ \textit{poll-en} \textit{ekʰ-ei}\\
anger-\textsc{acc.sg} \textsc{conj} they-\textsc{dat.pl} ~ much-\textsc{acc.f} have-\textsc{3sg}\\
\glt `For she is very angry with them.' \\
\hspace*{\fill}(\iwi{Aristophanes, \emph{Pax} 660})

\ex\label{ex:ma:9b}

\glll ὃς\ldots{} ἀντιστατ-έων τε καὶ \textbf{ὀργ-ῇ} \textbf{χρεώ-μενος} ἐς τ-όν σε ἥκιστα ἐ-χρ-ῆν\ldots{}\\
 \textit{ʰos} \textit{antistat-eon} \textit{te} \textit{kai} \textit{org-e\textsuperscript{i}}
\textit{kʰreo-menos} \textit{es} \textit{t-on} \textit{se} \textit{ʰekista} \textit{e-kʰr-en}\\
you.\textsc{nom.sg} rebel-\textsc{ptcp.nom.sg} and and anger-\textsc{dat.sg} use-\textsc{ptcp.nom.sg} towards he-\textsc{acc.sg} you.\textsc{acc.sg} least
\textsc{pst-}should-\textsc{3sg}\\
\glt `You\ldots, rebelling and \textbf{being angry} with whom you least should\ldots' \\
\hspace*{\fill}(\iwi{Herodotus, \emph{Histories} 3.52.4})
\z
\z\clearpage


Similarly, for \emph{honorem habere} ≈ \emph{honore uti} `to have/use honour' or `to hold an
honour', see \xxref{ex:ma:10a}{ex:ma:10b}.

% 10

\ea\label{ex:ma:10}

\ea\label{ex:ma:10a}

\gll \textbf{\itshape honos=que} \emph{e-i} \emph{a} \emph{popul-o} \textbf{\itshape
  hab-it-us} \textbf{\itshape est}, \emph{ut} \emph{in} \emph{camp-o} \emph{Marti-o} \emph{sepel-ire-tur.}\\
honour=and he-\textsc{dat.sg} from people-\textsc{abl.sg} have-\textsc{ptcp-nom.sg}
be.3\textsc{sg} that in field-\textsc{abl.sg} of.Mars-\textsc{abl.sg}
bury-\textsc{impf.subj-3sg.pass}\\
\glt `And he had the honour from the people to be buried in the Field of
Mars.' \\
\hspace*{\fill}(\iwi{Livy, \emph{Periochae} 106})

\ex\label{ex:ma:10b}

\gll \emph{neque} \emph{er-at} \emph{superior-e} \textbf{\itshape honor-e}
\textbf{\itshape us-us} \emph{qu-em} \emph{praefic-erem.}\\
and.not be.\textsc{impf-3sg} higer-\textsc{abl.sg} honour-\textsc{abl.sg}
used\textsc{.ptcp}-\textsc{nom.sg} \textsc{rel-acc.sg} appoint-\textsc{1sg.impf.subj}\\
\glt `And there was no one who had held a higher honour for me to appoint.' \\
\hspace*{\fill}(\iwi{Cicero,
\emph{Epistulae ad familiares} 2,15,4})

\z

\z

The fact that the verb in \emph{honorem habere} can be passivised in example
\xref{ex:ma:10a} is an indicator of morphological flexibility, hence an indicator that
this is an SVC rather than an idiom.\is{idiom} \emph{Utor} and χράομαι
\emph{kʰraomai} cannot be passivised because they are deponent verbs,
but that does not impede an analysis as SVs. As a matter of fact, the Greek middle voice
seems to be particularly compatible with the syntactic properties of SVCs, see
\citet{jimenez_lopez_support_2016}; \citet{jimenez_lopez__2021}. In this section, I have
proved that χράομαι \emph{kʰraomai} and \emph{utor} often behave either as specialised SVs or as the
diathetically neutral construction in a pair of \emph{constructions converses}.

\section{Distribution of support-verb constructions with χράομαι \emph{kʰraomai} and \emph{utor}}\label{sec:ma:5}

In Section~\ref{sec:ma:3}, I stated that χράομαι \emph{kʰraomai} is used
in the Greek corpus almost twice as frequently as \emph{utor} is in the Latin, with a
proportion of 0.93\% of the sample in Greek as compared to 0.49\% of the sample in Latin one. This clearly
affects the proportions that I discuss in this section, but what is probably more relevant
is the distribution by author of each SV. Since the total number of tokens of χράομαι
\emph{kʰraomai} or \emph{utor} is a deceiving figure, due to the
different sample sizes for each author --- for instance, Herodotus's \emph{Histories} are
considerably longer than any Greek tragedy~---, I have calculated normalised counts per
1,000 words (see Section~\ref{sec:ma:3} for the discussion on the forms that are
considered and discarded in my analysis).

\begin{figure}[htb]
  \centering
  \includegraphics[width=.95\textwidth]{figures/madrigal_fig1.png}
  \caption{Tokens of SV χράομαι \emph{kʰraomai} per
    1,000 words by author}
  \label{fig:ma:1}
\end{figure}

\figref{fig:ma:1} shows the somewhat even distribution of SVCs with χράομαι
\emph{kʰrao\-mai} throughout Greek prose with few exceptions. The poets
make very little or no use at all of this verb in their compositions. By contrast,
Andocides shows a preference for this kind of SVCs. One could hypothesise that this verb
might have been specialised for some legal contexts, given that the construction he uses
in most instances is νόμῳ χράομαι \emph{nomo\textsuperscript{i}
  kʰraomai} 'to use a law', but, in that case, why would Demosthenes and Lysias not use
it the same way? It is also possible that this is just a stylistic characteristic of
Andocides' prose: a recent paper proved that collocations in general are useful for the
identification of authorial identity (\cite{lopez_martin_use_2022}). Another author that
stands out from the rest is Herodotus, although not as much as Andocides. The collocation
he uses most frequently is also νόμῳ χράομαι \emph{nomo\textsuperscript{i}
  kʰraomai}.
  
  
  It seems clear that the data is also conditioned by the
content of the texts: since νόμῳ χράομαι \emph{nomo\textsuperscript{i}
  kʰraomai} is a very common collocation (17\% of the examples), the
authors which address topics related to the law and customs in general may display
disproportionately high figures, particularly when the sample size is smaller, as in the case of
Andocides. However, this is not an idiom\is{idiom}: the main evidence is that it admits number
variation, i.e., together with νόμῳ χράομαι \emph{nomo\textsuperscript{i}
  kʰraomai} I have found νόμοις (pl.) χράομαι \emph{nomois
  kʰraomai} `to use laws' (cf. \iwi{Thucydides, \emph{Histories},
6.54.6}, \iwi{Thucydides, \emph{Histories} 2.52.4} --- which also happens to be pronominalised
---, \iwi{Demosthenes, \emph{Adversum Leptinem} 20.91}, \iwi{Euripides, \emph{Medea} 538}, and
\iwi{Herodotus, \emph{Histories}. 4.26.1}). Another caveat is that Herodotus is the only
writer in the corpus who uses the Ionic dialect: a future research question could be how
this dialectal difference affects the use of SVCs by different authors.

\begin{figure}[htb]
  \centering
  \includegraphics[width=.95\textwidth]{figures/madrigal_fig2.png}
  \caption{Tokens of SV \emph{utor} per 1,000 words by author}
  \label{fig:ma:2}
\end{figure}

The Latin corpus shows more balance, to a certain extent (see \figref{fig:ma:2}).
The historians use \emph{utor} as an SV more frequently than the poets, with the sole exception of
Livy, who is on a par with the latter. A diachronic trend is quite apparent in Figure
\ref{fig:ma:2}: in the archaic texts, these SVCs are very rare, but they peak in the
classical period only to decline shortly thereafter.\footnote{This has been thoroughly
  analysed in a recent paper with abudant data, which shows that this is a general trend
  in Latin SVCs (\cite{banos_support_2023}).} As some researchers have already pointed
out, collocations are sometimes short-lived, and tend to rapid diachronic renewal
(\cite[48]{banos_bases_2018}). However, the distinction between prose and verse also
affects this distribution. It has already been proven that SVCs are not exclusively found
in prose, but rather that different SVCs are preferred in poetic texts
(\cite[38]{banos_bases_2018}). My data confirm \cite{banos_support_2023}'s conclusions for
Latin that SVCs are subject to rapid diachronic renewal and that differences in authorship
and literary genre also condition the choice of SVCs.

\section{Semantic-syntactic domains of χράομαι \emph{kʰraomai} and
  \emph{utor}}\label{sec:ma:6}

The wide range of nouns that take either χράομαι \emph{kʰraomai} or
\emph{utor} as SVs is too varied to fit under a few semantic or lexical labels (see full
list in \cite{madrigal_acero_dataset_2024}). There are nouns of thought (CG γνώμη
\emph{gnome} `opinion', Lat. \emph{consilium} `deliberation, counsel'), of speech (CG
βοή \emph{boe} `scream', Lat. \emph{verbum} `word'), of emotion (CG ὀργή \emph{orge}
`anger', Lat. \emph{timor} `fear'), etc. The classifications I attempted previously failed
to offer a comprehensive and complete view of the collocative patterns of χράομαι
\emph{kʰraomai} and \emph{utor.} This led me to a different approach,
which focuses on the SVs themselves rather than on external evidence in order to organise the
data.

Although more could be said on this, I have found two tendencies. Sometimes, χράομαι
\emph{kʰraomai} alternates with ἔχω \emph{ekʰo} `to
have'/ποιέομαι \emph{poieomai} `to make', which are used as SVs for states (ἔχω
\emph{ekʰo}) and actions (ποιέομαι \emph{poieomai}). In these cases,
χράομαι \emph{kʰraomai} conveys the same meaning as ἔχω
\emph{ekʰo}/ποιέομαι \emph{poieomai}, but it is less frequent than
either of them, which has led me to analyse χράομαι as a more lexically restricted variant
--- or specific SV --- as compared to ἔχω \emph{ekʰo}/ποιέομαι \emph{poieomai}, see
\xref{ex:ma:11}.

% 11

\ea\label{ex:ma:11}

\ea\label{ex:ma:11a}

\glll\textbf{ὀργ-ὴν} γὰρ αὐτ-οῖς\ldots{} πολλ-ὴν \textbf{ἔχ-ει.}\\
 \textit{org-en} \textit{gar} \textit{aut-ois} \textit{poll-en} \textit{ekʰ-ei}\\
anger-\textsc{acc.sg} \textsc{conj} they-\textsc{dat.pl} much-\textsc{acc.sg} have-\textsc{3sg}\\
\glt `For she is very angry with them.' \\
\hspace*{\fill}(\iwi{Aristophanes, \emph{Pax} 660}) (= example \ref{ex:ma:9a})

\ex\label{ex:ma:11b}

\glll ὁ Καμβύσ-ης \textbf{ὀργ-ὴν} \textbf{ποιη-σά-μεν-ος} ἐ-στρατεύ-ετο ἐπὶ τοὺς Αἰθίοπ-ας.\\
 \textit{ʰo} \textit{Kambys-es} \textit{org-en} \textit{poiesamenos} \textit{e-strateu-eto} \textit{epi} \textit{tus} \textit{Aitʰiop-as}\\
the Cambyses-\textsc{nom.sg} anger-\textsc{acc.sg} make-\textsc{aor-ptcp-nom.sg}
\textsc{pst-}march-\textsc{3sg.impf} upon the Ethiophians-\textsc{acc.pl}\\
\glt `Cambyses got angry and marched against the Ethiopians' \\
\hspace*{\fill}(\iwi{Herodotus, \emph{Histories} 3.25.3})

\ex\label{ex:ma:11c}

\glll ὃς\ldots{} ἀντιστατ-έων τε καὶ \textbf{ὀργ-ῇ} \textbf{χρεώ-μενος} ἐς τ-όν σε ἥκιστα ἐ-χρ-ῆν\ldots{}\\
 \textit{ʰos} \textit{antistat-eon} \textit{te} \textit{kai} \textit{org-e} \textit{kʰreo-menos} \textit{es} \textit{t-on} \textit{se} \textit{ʰekista} \textit{e-kʰr-en}\\
you.\textsc{nom.sg} rebel-\textsc{ptcp.nom.sg} and and anger-\textsc{dat.sg}
use-\textsc{ptcp.nom.sg} towards he-\textsc{acc.sg} you.\textsc{acc-sg} least \textsc{pst-}should-\textsc{3sg}\\
\glt `You\ldots, rebelling and \textbf{being angry} with whom you least should\ldots' \\
\hspace*{\fill}(\iwi{Herodotus, \emph{Histories} 3.52.4}) (= example \ref{ex:ma:9b})

\z

\z

However, when χράομαι \emph{kʰraomai} alternates with δίδωμι
\emph{didomi} `to give'/τίθημι \emph{titʰemi} `to put', which are
intrinsically causative\is{causativity}, χράομαι \emph{kʰraomai} is markedly
non-causative or neutral, as in \xref{ex:ma:12}. In this case, the pairs χράομαι
\emph{kʰraomai}/δίδωμι \emph{didomi} and χράομαι
\emph{kʰraomai}/τίθημι \emph{titʰemi} behave as
converse constructions.

% 12

\ea\label{ex:ma:12}

\glll ἐπειδὴ <δ'> ἀν-ε-γράφ-ησαν, \textbf{ἐ-θέ-μεθα} \textbf{νόμ-ον}, \textbf{ᾧ} \textbf{πάντ-ες} \textbf{χρῆ-σθε.}\\
 \textit{epeide} \textit{<d'>} \textit{an-e-grapʰ-esan} \textit{e-tʰe-metʰa} \textit{nom-on} \textit{\textbf{ʰo\textsuperscript{i}}} \textit{\textbf{pant-es}}
\textit{\textbf{kʰre-stʰe}}\\
after \textsc{prt} in-\textsc{pst-}write-\textsc{3pl.pass} \textsc{pst-}put-\textsc{1pl}
law-\textsc{acc.sg} \textsc{rel.dat.sg} \textsc{all-nom.pl} \textsc{use-2pl}\\
\glt `After they were engraved, we \textbf{established} a \textbf{law} \textbf{by which} you all \textbf{abide}.' \\
\hspace*{\fill}(\iwi{Andocides, \emph{De mysteriis} 1.85}) (= examples \ref{ex:ma:4b} and \ref{ex:ma:7})

\z

This distribution is rather similar in Latin: \emph{utor} behaves as a lexically
restricted variant of certain verbs (\emph{habere} `to have', \emph{facere} `to make'), see
\xref{ex:ma:13}, and as a diathetically neutral form in contrast with certain causative
extensions\is{support-verb extension} (\emph{dare} `to give', \emph{ferre} `to carry', \emph{facere} `to make'), see
\xref{ex:ma:14}. For instance, \emph{rationem habere} ≈ \emph{ratione uti} `to have/use
reason'; \emph{consilium dare} `to give advice' ↔ \emph{consilium uti} `to follow
advice';\footnote{\emph{Consilium} and its collocational pattern\is{collocational pattern} have been analysed in
  depth by \citet{banos_construcciones_2014}. This particular example is interesting
  because it could be analysed as a diathetic alternation like ποιέομαι \emph{poieomai}
  `to do'↔γίγνομαι \emph{gignomai} `to come to be', where γίγνομαι \emph{gignomai} is used
  as the lexical passive of ποιέομαι \emph{poieomai} (\cite{jimenez_lopez__2021}). The
  reason for this is that ποιέομαι \emph{poieomai} cannot be passivised because it is always used in the
  middle voice when it functions as an SV, which makes voice variations impossible.} but
\emph{facere} may fall in either category: \emph{verbum facere} ≈ \emph{verbum uti} `to
speak,' but also \emph{pacem facere} `to make peace' ↔ \emph{pace uti} `to enjoy peace.'

% 13

\ea\label{ex:ma:13}

\ea\label{ex:ma:13a}


\gll \textbf{\itshape hab-et} \textbf{\itshape honor-em} \emph{qu-em} \emph{pet-imus.}\\
have-\textsc{3sg} honour-\textsc{acc.sg} \textsc{rel-acc.sg} seek-\textsc{1pl}\\
\glt `It is in possession of the office we are trying to obtain.' \\
\hspace*{\fill}(\iwi{Cicero, \emph{In Quintum
  Caecilium Nigrum oratio} 5,20,2})

\ex\label{ex:ma:13b}

\gll \emph{neque} \emph{er-at} \emph{superior-e} \textbf{\itshape honor-e}
\textbf{\itshape
  us-us} \emph{qu-em} \emph{praefic-erem.}\\
and.\textsc{neg} be.\textsc{impf-3sg} higher-\textsc{abl.sg} honour-\textsc{abl.sg}
used\textsc{.ptcp}-\textsc{nom.sg} \textsc{rel-acc.sg} appoint-\textsc{1sg.impf.subj}\\
\glt `And there was no one who had held a higher honour for me to appoint.' \\
\hspace*{\fill}(\iwi{Cicero,
\emph{Epistulae ad familiares} 2,15,4}) (= example \ref{ex:ma:10})

\z

\z


% 14

\ea\label{ex:ma:14}

\ea\label{ex:ma:14a}

\gll \emph{qu-id} \textbf{\itshape d-as} \textbf{\itshape consil-i}?\\
what-\textsc{acc.sg} give-\textsc{2sg} suggestion-\textsc{gen.sg}\\
\glt `What do you suggest?' \\
\hspace*{\fill}(\iwi{Cicero, \emph{Epistulae ad familiares} 2,15,4})

\ex\label{ex:ma:14b}

\gll \emph{ergo} \textbf{\itshape ut-ar} \textbf{\itshape tu-o} \textbf{\itshape consili-o} \emph{neque} \emph{me} \emph{Arpin-um} \emph{h-oc} \emph{tempor-e} \emph{abd-am}\\
then use-\textsc{1sg} your-\textsc{abl.sg} suggestion-\textsc{abl.sg} and.\textsc{neg}
I.\textsc{acc.sg} Arpinum-\textsc{acc.sg} this-\textsc{abl.sg} time-\textsc{abl.sg}
hide-\textsc{1sg}\\
\glt `I will follow your advice and will not hide in Arpinum at the moment.' \\
\hspace*{\fill}(\iwi{Cicero,
\emph{Epistulae ad Atticum} 9,6,1})

\z

\z

To summarise, I propose a continuum of agentivity\is{agentivity} and metaphoricisation\is{metaphoricisation} (see Figure
\ref{fig:ma:3}). 


\begin{figure}[htb]
  \centering
  \includegraphics[width=\textwidth]{figures/madrigal_fig3.png}
  \caption{Agentivity continuum}
  \label{fig:ma:3}
\end{figure}


When the SVC is more agentive, χράομαι \emph{kʰraomai}
and \emph{utor} imply the manipulation of a physical object, which is closer to the basic
meaning of the verb. In an intermediate position there are constructions where we can
perceive the manipulation of an abstract reality which is metaphorically reconceptualised
as an object. Lastly, there are constructions either with a less prototypical Agent, or
without an Agent, which do not convey any kind of manipulation. In these latter cases,
such as with emotion nouns, χράομαι \emph{kʰraomai} and \emph{utor} are
closer to being a prototypical SV.



\section{Support verbs and loan words}\label{sec:ma:7}

There is a clear tendency to transmit SVCs from one language to another for the
translation of foreign concepts (\cite[172-173]{bowern_diachrony_2008}). I have found two
examples in which Cicero uses a collocation of \emph{utor} + Greek noun where the noun is
left untranslated, \emph{adiaphoria} `indifference' (\iwi{Cicero, \emph{Epistulae ad Atticum} 2,17,2}) and
\emph{ekteneia} `zeal' (\iwi{Cicero, \emph{Epistulae ad Atticum} 10,17,1}, but that seems hardly enough
evidence to suggest an influx of Greek upon Latin comparable to the stream of Chinese
words that entered the Japanese language in the shape of SVCs with the verb \emph{suru}
`to do' (\cite[172]{lanz_diachrony_2009}).

Cicero does not merely translate Greek oratory; instead, he looks to relay Greek ideas in
Latin (\iwi{Cicero, \emph{De optimo genere oratorum} 14}). His knowledge of Greek oratory might be a reasonable
explanation for his use of foreign words, but not for the abundance of SVCs in his prose.
In fact, it has already been argued that Latin uses them a lot more frequently than Greek
(\cite[186]{jimenez_lopez_support_2016}).


An analysis of the relationship between Greek and Latin SVCs and the directionality of the
influence of either language upon the other is yet to be undertaken. However, some surveys
on the influence of other languages on Greek and Latin have suggested that the increased
number of SVCs in certain texts is partly due to the interference of other languages
during their composition
\parencites{jimenez_lopez_colocaciones_2017,jimenez_lopez_colocaciones_2018,banos_arrepentirse_2017}.

\section{Conclusions}\label{sec:ma:8}

To sum up, I have identified the following similarities between χράομαι
\emph{kʰraomai} and \emph{utor}:

\begin{enumerate}
   \def\labelenumi{\alph{enumi}.}
 \item Type frequency. Although χράομαι \emph{kʰraomai} is more
   frequently used in Greek (0.93\% of the sample) than \emph{utor} in Latin (0.49\%),
   both are used with a similar frequency as SVs in roughly half of their instances
   (45.56\% of the instances of χράομαι \emph{kʰraomai} and 48.34\% of the instances of \emph{utor}), see
   Section~\ref{sec:ma:3}.
 \item Both share the same predicative frame (Section~\ref{sec:ma:4:1}), with a [+/human/]
   Agent or Experiencer as their first argument and an Instrument as their second argument.
 \item Both behave as SVs according to the most common batteries of tests for ancient
   languages (see \cite{jimenez_lopez_support_2016}), such as the possibility of
   pluralisation, relativisation, pronominalisation, etc. (Section~\ref{sec:ma:4:2}).
 \item Both alternate with aspectual\is{aspect} and causative SVEs\is{causativity} (Section~\ref{sec:ma:4:3}). In
   both cases, χράομαι \emph{kʰraomai} and \emph{utor} behave as neutral
   or non-marked alternatives to verbs that convey lexical aspect\is{aspect} or a causative
   diathesis. The functions of these collocations seem to be conditioned by the
   characteristics of the subject of the phrase (see \figref{fig:ma:3}). Where there
   is a more prototypical Agent, SVCs are closer to free constructions, even though I
   still consider them SVCs because the nouns they take have been made predicative by
   placing them in the collocation. Where there is a less prototypical Agent, such as the
   Experiencer that emotion nouns take, the construction is in fact a prototypical SVC.
 \item Both are prevalent in prose (Section~\ref{sec:ma:5}), but their chronological
   distribution and their use by author differs. In Latin, there seems to be a clear
   diachronic trend where SVCs with \emph{utor} peak during the Classical Period, whereas
   in Greek there does not seem to be such trend. Instead, Andocides and 
   Herodotus peak as the authors who markedly employ the most SVCs with χράομαι
   \emph{kʰraomai}.
 \item χράομαι \emph{kʰraomai} and \textit{utor} serve as stylistically specialised SVs (Section~\ref{sec:ma:4:3}) and
   alternate with diathetic and causative SVEs\is{causativity}, depending on the noun with which they are
   combined and the way they alternate with other SVs or SVEs. For the organisation of
   these functions, I have proposed a continuum of agentivity\is{agentivity} and metaphoricisation\is{metaphoricisation}
   (Section~\ref{sec:ma:6}).
\end{enumerate}

However, there are also some differences between Greek and Latin. There is a difference in the base nouns each verb takes. While 17\% of
the SVCs with χράομαι \emph{kʰraomai} have νόμος \emph{nomos} as the base,
\emph{utor} does not have such a strong preference for any single base. Other differences
depend directly on the lexical properties of the nouns in each language.

\section*{Abbreviations}
\begin{tabularx}{.45\textwidth}{lQ}
  Fr. & French \\
  Lat.& Latin \\
  PF & Predicate Frame \\
\end{tabularx}
\begin{tabularx}{.45\textwidth}{lQ}
  SVE & support-verb extension \\
  SVEC & support-verb-extension construction \\
\end{tabularx}%

\newpage
\section*{Acknowledgements}

This paper has been made possible thanks to the funding of \emph{Interacción del léxico y
  la sintaxis en griego antiguo y latín 2: Diccionario de Colocaciones Latinas (DiCoLat) y
  Diccionario de Colocaciones del Griego Antiguo (DiCoGra)} (PID2021-125076NB-C42) and the
Spanish Ministry of Universities (FPU21/01964). I thank J.\,M. Baños, M.D. Jiménez López, and
A. Vives Cuesta for their valuable remarks on a previous version of this paper.

 \sloppy
 \printbibliography[heading=subbibliography,notkeyword=this]

\end{document}

