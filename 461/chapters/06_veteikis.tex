\documentclass[output=paper,colorlinks,citecolor=brown]{langscibook}
\ChapterDOI{10.5281/zenodo.14017931}

\author{Tomas Veteikis\orcid{}\affiliation{Vilnius University}}

\title[Support-verb constructions and other periphrases in Aristotle’s \textit{Rhetoric}]{Support-verb constructions and other periphrases in Aristotle’s \textit{Rhetoric} (books 1 and 2)}
\abstract{This chapter discusses empirically periphrastic constructions from books 1 and 2 of Aristotle’s \textit{Rhetoric}, treated holistically as a multilayered corpus. Some, e.g., ποιεῖσθαι λόγον \textit{poieĩsthai lógon}, reflect the canonical features of support-verb constructions. The chapter illustrates the relationship between these constructions and the rhetorical strategies of alternating between brevity and expansion. Furthermore, the stylistic diversity of phrases and issues with their terminological conception are addressed. The chapter considers the concepts developed in Graeco-Roman rhetorical theory, such as \textit{períphrasis, makrología, brakhulogía}, and their alignment with modern views, and hypothesises that the term ‘periphrasis’, elaborated in ancient rhetoric, is descriptively adequate for a range of multi-word constructions. It also classifies phraseological material based on verb semantic role and introversion and extraversion categories, reinterpreting theories of valency change.


\bigskip


Šiame skyriuje aptariamos empiriškai atrinktos perifrastinės konstrukcijos iš Aristotelio \textit{Retorikos} I ir II knygų, traktuojamų holistiškai kaip daugiasluoksnis korpusas. Kai kurios, pavyzdžiui, ποιεῖσθαι λόγον \textit{poieĩsthai lógon}, atspindi kanoninius leksinių analitinių konstrukcijų bruožus. Čia siekiama parodyti šių konstrukcijų ryšį su retorinėmis suglaudimo ir išplėtojimo kaitaliojimo strategijomis, nagrinėjama stilistinė frazių įvairovė, jų terminologinės sampratos klausimai, aptariamos graikų-romėnų retorikos teorijoje išplėtotos sąvokos, tokios kaip \textit{períphrasis, makrología, brakhulogía}, jų atitikimas šiuolaikiniam požiūriui, taip pat keliama hipotezė, kad senovės retorikoje išplėtota sąvoka ‟perifrazė" tinkama apibūdinti įvairioms daugiažodėms konstrukcijoms. Skyriuje klasifikuojama frazeologinė medžiaga, remiantis veiksmažodžio semantine role ir introversijos bei ekstraversijos kategorijomis, naujai interpretuojant valentingumo kaitos teorijas.

}


\IfFileExists{../localcommands.tex}{
   \addbibresource{../localbibliography.bib}
   \usepackage{langsci-optional}
\usepackage{langsci-gb4e}
\usepackage{langsci-lgr}

\usepackage{listings}
\lstset{basicstyle=\ttfamily,tabsize=2,breaklines=true}

%added by author
% \usepackage{tipa}
\usepackage{multirow}
\graphicspath{{figures/}}
\usepackage{langsci-branding}

   
\newcommand{\sent}{\enumsentence}
\newcommand{\sents}{\eenumsentence}
\let\citeasnoun\citet

\renewcommand{\lsCoverTitleFont}[1]{\sffamily\addfontfeatures{Scale=MatchUppercase}\fontsize{44pt}{16mm}\selectfont #1}
  
   %% hyphenation points for line breaks
%% Normally, automatic hyphenation in LaTeX is very good
%% If a word is mis-hyphenated, add it to this file
%%
%% add information to TeX file before \begin{document} with:
%% %% hyphenation points for line breaks
%% Normally, automatic hyphenation in LaTeX is very good
%% If a word is mis-hyphenated, add it to this file
%%
%% add information to TeX file before \begin{document} with:
%% %% hyphenation points for line breaks
%% Normally, automatic hyphenation in LaTeX is very good
%% If a word is mis-hyphenated, add it to this file
%%
%% add information to TeX file before \begin{document} with:
%% \include{localhyphenation}
\hyphenation{
affri-ca-te
affri-ca-tes
an-no-tated
com-ple-ments
com-po-si-tio-na-li-ty
non-com-po-si-tio-na-li-ty
Gon-zá-lez
out-side
Ri-chárd
se-man-tics
STREU-SLE
Tie-de-mann
}
\hyphenation{
affri-ca-te
affri-ca-tes
an-no-tated
com-ple-ments
com-po-si-tio-na-li-ty
non-com-po-si-tio-na-li-ty
Gon-zá-lez
out-side
Ri-chárd
se-man-tics
STREU-SLE
Tie-de-mann
}
\hyphenation{
affri-ca-te
affri-ca-tes
an-no-tated
com-ple-ments
com-po-si-tio-na-li-ty
non-com-po-si-tio-na-li-ty
Gon-zá-lez
out-side
Ri-chárd
se-man-tics
STREU-SLE
Tie-de-mann
}
   \boolfalse{bookcompile}
   \togglepaper[23]%%chapternumber
}{}

\begin{document}
%extra emergency stretch to avoid remaining overfull hboxes.
\emergencystretch 3em

\maketitle


\section{Introduction}
Aristotle’s \textit{Rhetoric}\footnote{The dataset is accessible here: \url{http://dx.doi.org/10.5287/ora-n652gamyj}.}, like any ancient literary monument, is a `repository' of expressions which contains a sizable collection of compound words and phrases,\footnote{For the purposes of this article, we use the term \textit{phrases} to refer to all the lexical expressions longer than one word and not forming a sentence. For a similar use of the corresponding term in Lithuanian phraseology, see \citet[121--122]{Marcinkeviciene2010}.} some rather challenging to detect and translate into another language. This chapter reflects a significant effort to evaluate and classify the verb and complement constructions of an Ancient Greek text being translated into another language, with a focus on Ancient Greek rhetorical terminology.
%why is this in here?
However, cross-linguistic parallels (such as Greek “ποιεῖσθαι λόγον" \textit{poieĩsthai lógon} (lit. \~ “make a speech”) and its English or Lithuanian equivalents), as part of the greater phenomenon of translation issues, will not be treated here. Instead, this chapter focuses only on the nature and classification of single-language (Ancient Greek) constructions.
%end of comment
Particular attention in this chapter is paid to the identification of verbal constructions, termed light-verb constructions (LVCs henceforth) or support-verb constructions (SVCs henceforth),\footnote{The synonymity of these terms is not questioned here on the basis of the terminology available to us in the research materials, such as \citet{Langer2004}, \citet{Kovalevskaite-etal2020}, \citet{Fotopoulou-etal2021}. In this article, preference will be given to the term SVC, while LVC may appear sporadically in commenting on the literature where there is a preference for the latter term.} which are treated as part of a larger phenomenon —linguistic, rhetorical, or poetic variation.

Aimed at a synthesis of empirical research, the chapter combines two major theoretical approaches: the classical theory of style with its basic ‘idea that a thought can be formulated in several ways with different effects'\footnote{\citet[326]{deJonge2014}} and the modern theories and insights of verb valency, transitivity, and non-causal-causal alternations.\footnote{E.g. \citet{Lavidas2009}, \citet{Arkadiev-Pakerys2015}, \citet{Haspelmath2016}, \citet{Grossman-WitzlackMakarevich2019}.} Two thirds of Aristotle’s \textit{Rhetoric}, Books 1 and 2, dealing with so-called rhetorical invention, form the basis of the empirical study. This choice of the corpus of limited scope was due, \textit{inter alia}, to the large amount of heterogeneous material obtained over the course of the research. 

Even though the results’ breadth may appear constrained, they may nonetheless contribute to a perceptual testing of the methodology: once the phraseological principles of these two books are established, the third book can be evaluated in a similar framework. This study is distinguished by its limited use of automated processes: many of the word combinations were found in the corpus by way of a close reading and manual extraction. On this basis, a number of constructions pertinent to the study were then selected.

The content of the chapter is divided into the following sections: 1) introductory reflections on the text under discussion (\sectref{Section2Rhet}); 2) observations on the linkage of verb formations from the perspectives of modern linguistics and of the notions known from ancient Greek rhetorical and linguistic theory (\sectref{Section3Rhet}); 3) key points of empirical research and the classification of phraseological material (with a focus on verbal semantics) (\sectref{Section4Rhet}); 4) an overview of recent findings on SVCs and other periphrastic constructions in Aristotle’s treatise (Sections \ref{Section5Rhet} and \ref{Section6Rhet}); 5) a brief outline of the stylistic functions of verb-based periphrases found in the course of the study (\sectref{Section7Rhet}).

\section{Aristotle’s \textit{Rhetoric} as a source of Greek phraseology}\label{Section2Rhet}

Τέχνη ῥητορική \textit{Τékhnē rhētorikḗ} (as some manuscripts title it\footnote{See \citet[3]{Kassel1976} (in app. crit.)}), or simply \textit{Rhetoric}, a theoretical work on the art of persuasive speech, which, in Aristotle’s view, shares many similarities with dialectics, ethics, politics, and poetics,\footnote{On the relation of rhetoric to dialectics, ethics, and politics, cf. \iwi{Aristotle, \textit{Rhetoric} 1.2.7 1356a25-27}, and on the relationship between rhetoric and poetics, see \citet{Kirby1991} with references.} discusses the nature and components of this art, the means of persuasion, the arguments relevant to the three types of speech (deliberative, epideictic, and juridical), and describes ethical, emotional and stylistic factors of a persuasive speech. The content of the treatise is roughly divided into three unequal parts: the first two of the three books, which form the core of the author’s original vision, deal with rhetorical invention and theory of proofs, while the third book covers more practical issues of style and composition.

The \textit{Rhetoric} is an integral part of the \textit{Corpus Aristotelicum} and contains references to other works by this author, such as treatises on logical reasoning and dialectics, Ἀναλυτικά Πρότερα \textit{Analutikà Prótera}, Κατηγορίαι \textit{Katēgoríai}, and Τοπικά \textit{Topiká}. This study therefore can contribute to our understanding of Aristotle’s phraseology and, to some degree, to that of the textual aspects of the treatise in question (e.g. differences across copies), as well as intertextual ones (such as quotations and paraphrasing of other texts, both oral and written).

As a multi-layered text, Aristotle’s \textit{Rhetoric}, on the one hand, captures the rich and literarily charged phraseology of Greek spoken in the 4\textsuperscript{th} century BC, of which most modern readers, being non-native speakers, can only have a vague idea. This phraseology is essentially the phraseology of the Attic dialect of the 4\textsuperscript{th} century BC, strongly influenced by literary genres dominant in contemporary Athens, such as Attic drama (apart from the choral parts), rhetorical, philosophical, and historiographical prose, and used in colloquial form not only in Attica but also in interstate communication (including the Macedonian court, with which Aristotle was closely associated). It is uncertain how much this basic dialectal layer of the treatise was influenced by lexical and phrasal variation from other dialects (cf. Aristotle’s habit of exemplifying his stylistic points from Herodotus and Homer, the representatives of the literary Ionic and an epic dialectal mixture respectively\footnote{\citet[168]{MorpurgoDavies2002}}), but the impact of the stylistic features of Attic drama and oratory is undoubted.\footnote{Aristotle’s treatise on rhetoric is particularly rich in quotations from classical Athenian tragedy and from the speeches of the orators of Aristotle’s time (esp. Isocrates and his students).} This naturally prompts us to focus principally on the Attic dialect.

On the other hand, to quote Aristotle’s translator, ‘our knowledge of what Aristotle wrote is based on manuscripts copied by scribes from older manuscripts, which were in turn copied from still earlier ones, going back to Aristotle’s personal copy, with opportunity for mistakes at every stage in the transmission. The earliest existing evidence for the text dates from over a thousand years after Aristotle died' (\citealt[xii]{Kennedy2007}). Understanding the textual tradition prompts a nuanced interpretation of Aristotle’s phrasing. The decision to use a manuscript version that uses single-word formations and, \textit{inter alia}, compound words rather than two-word combinations, or vice versa, can influence the way in which we perceive the author on the whole — either as a producer of periphrastic formulations or of compound words.\footnote{So e.g. in \iwi{Aristotle, \textit{Rhetoric} 1.7.26, 1364b31}, one version has ἀβεβαιοτέρων \textit{abebaiotérōn}, another μὴ βεβαιοτέρων \textit{mḕ bebaiotérōn}, in \iwi{Aristotle, \textit{Rhetoric} 2.23.11, 1398b11}, we find either βλάσφημον ὄντα \textit{blásphēmon ónta} or βλασφημήσαντα \textit{blasphēmḗsanta}, in 1.12.4, 1372a20, we find either φίλοι ὦσι \textit{phíloi ō̃si} or φιλῶσιν \textit{philō̃si}n, in 2.4.26, 1381b28, either τοὺς φιλεῖν ἀγαθοὺς \textit{toùs phileĩn agathoùs} or φιλαγάθους \textit{philagáthous}. For these and other examples see app. crit. ad loc. in \citet{Kassel1976}.} As fascinating as this aspect of the study is, we will not delve into the details here because of constraints of time and space. Instead, we will just acknowledge that the material used in this study is based on one of the most widely used Greek editions, that of Ross (\citealt{Ross1959}), but it also takes one of the most thorough critical editions, that of \citet{Kassel1976}, into account.

\largerpage%long distance
We are thus dealing with a largely literary version of Greek that shares (\textit{cum variatione}) the characteristics of every document of the ancient tradition which has undergone a change over the course of written transmission. This linguistic form deserves an approach that finds parallels not only with the terms and linguistic phenomena of our time, but also with the terminology and descriptions of poetic and literary phenomena of the period in which the texts under study were written. In other words, in addition to the complex typology of different expressions developed by modern linguistics, it is worth recalling the discoveries and insights of ancient thinkers and stylists, and combining their terminology with the terms we use today, such as Multi-word Expressions (MWEs henceforth), SVCs, LVCs, Function-Verb Constructions (FVCs henceforth)\footnote{Or FVG (for \textit{Funktionsverbgefüge}) in German literature, e.g. \citet{Schutzeichel2014}.} or V-PCs (V-PP-Cs),\footnote{On verb-preposition constructions cf. \citet{Farrell2005}, \citet{Keizer2009}, cf. \citet[8]{Langer2004}.} etc. This chapter does not focus on this issue in detail, but offers some insights.

\section{Reflections on verbal constructions: Between the modern concept of support-verb constructions and ancient rhetorical tradition}\label{Section3Rhet}
\largerpage%long distance
The concepts just mentioned, especially multi-word expressions (MWEs henceforth) (i.e. phrasal units of great variety and certain ‘semantic opaqueness' and a universal phenomenon inherent to a variety of language sources)\footnote{For this kind of definition, cf. \citet{Rayson-etal2010} and a set of facts about MWEs available on the PARSEME network website (\url{https://typo.uni-konstanz.de/parseme/index.php/the-action}).} and SVCs (i.e. verb + noun combinations acting as predicates of a sentence)\footnote{\citet[382]{Fendel2023coptic}}, are central to this discussion, which focuses on their forms and functions within Aristotle’s \textit{Rhetoric}. In addition to that, it is also worth considering the issue of the relevance of concepts employed in modern linguistics and their compatibility with the old ones, as well as that of the commensurability of phenomena covered by the two families of concepts.

When it comes to multi-word phenomena, we believe that some ancient concepts could be used more widely both in modern linguistics and in the study of ancient languages. One of these is περίφρασις \textit{periphrasis} (from late Greek περι-φράζομαι \textit{peri-phrázomai}, ‘to express in a roundabout manner') with its Latin equivalent \textit{circumlocutio} (cf. \iwi{Quintilian, \textit{Institutio Oratoria} 8.6.61}; \iwi{Servius, \textit{Commentary on Vergil's Aeneid} 1.65: 17-19}) coined by the Graeco-Roman rhetoricians and grammarians. As attested in ancient literary critics, beginning with Dionysius of Halicarnassus (cf. v. περίφρασις \textit{períphrasis} in \citealt{Liddell-Scott-Jones1996}), it denotes the use of a longer phrase instead of a possible shorter form (e.g. a combination of words instead of one word). Despite the ramified use of the term in our time, it often retains a fairly universal meaning, applying to phenomena of various linguistic and stylistic categories (cf. \citealt{Haspelmath2000}). Even when discussing a specific linguistic phenomenon, such as verbal periphrasis, a hint of that broad meaning is retained (cf. Bentein’s examples of synthetic vs analytic forms with the latter being called both multi-word and ‘periphrastic' ones).\footnote{\citet[2]{Bentein2016}}

The breadth of the import of the term periphrasis parallels that of the term MWE, both of which are sometimes explicitly linked and have similar definitions (cf. the definition of MWE as ‘linguistic objects consisting of two or more words' and ‘a highly varied set of objects (from idioms to collocations, from formulae to expressions)', \citealt{Masini2019}). In the context of such juxtapositions, for texts written in an ancient language, it is natural to favour the terms originating from that language. On the other hand, given the complexity of the concept of MWE, it is useful to have an alternative short and inclusive synonym, as is the case with periphrasis.

Regarding SVCs, their connection to the concept of periphrasis has been noticed (cf. \citealt[183]{JimenezLopez2016}), but it has yet to be thoroughly investigated. Given the relative abundance of studies on periphrasis, such an enterprise would be valuable.

Although linguists have noted that the concept periphrasis can be employed at various degrees of strictness,\footnote{See e.g. \citet[654--655]{Haspelmath2000}, where periphrasis has 3 main definitions: ‘the use of longer, multi-word expressions in place of single words', ‘one of the canonical literary rhetorical figures', and ‘a situation in which a multi-word expression is used in place of a single word in an inflectional paradigm'.} a theoretical framework has also been developed to identify characteristics of a ‘canonical periphrastic construction' (e.g. the expression of the grammatical meaning, lexical applicability, regularity, recognizable syntactic relations, and head of a construction).\footnote{Cf. \citet[249--250]{Chumakina2011}; \citet[244]{Brown-etal2012}.} Compared to rhetorical periphrasis, linguistic periphrasis has been more intensively studied in several forms. Perhaps the best known of these are nominal (or ‘inflectional', filling of a cell of the inflectional paradigm; cf. \citealt{Chumakina2011, Chumakina-Corbett2012}) and verbal (or ‘participial') periphrasis, the latter extensively studied in \citet{Bentein2016}. However, there is still a lack of clarity concerning the applicability of this concept to other constructions, including SVCs. One of the reasons for this may be that linguistic research pays little attention to the rhetorical (persuasion-targeted) and poetic (creation-targeted) background of periphrasis. Therefore, we have to offer several considerations on this issue.

\largerpage
Periphrasis (a multi-word substitution of a single-word lexical unit) is a tool employed for pragmatic or stylistically motivated objectives rather than merely a lexical and grammatical category referring to the usage of a combination of words in place of the appropriate lexical meaning and morphological form. Its essence is well reflected in Lausberg's definition based on various references to it in the Graeco-Roman rhetorical tradition: periphrasis is ‘paraphrasing of one word by several words' (\citealt[§590]{Lausberg1998}). This definition refers to a wide variety of quantitative (several instead of one) and qualitative (different degrees of semantic equivalence) substitution, some of which are explicitly illustrated in examples of the late manuals of rhetoric. 


Thus, for example, Alexander Numeniu, a rhetorician of the 2\textsuperscript{nd} century AD, gives examples to show that periphrasis, originally a poetic (creation-targeted) device, has become a stylistic flourish in prose as well (\citealt[32]{Spengel1853}). Here, beside nominal expressions, such as βίη Ἡρακληείη \textit{bíē Hēraklēeíē} (lit. ‘strength of Heracles’) and μένος Ἀλκινόοιο \textit{ménos Alkinóoio} (lit. ‘might of Alcinous’) standing for nouns (Ἡρακλῆς \textit{Hēraklē̃s} and Ἀλκίνοος \textit{Alkínoos}), we see Thucydides’ phrase ‘τὴν μάθησιν ἐποιεῖσθε' \textit{tḕn máthēsin epoieĩsthe}, ‘you were doing learning’ with the rhetorician’s remark: ‘instead of ἐμανθάνετε \textit{emanthánete}’, which corresponds to the well-known type of SVCs with the verb ποιεῖσθαι \textit{poieĩsthai}.\footnote{On this popular type of analytic predicate (ποιοῦμαι \textit{poioũmai} + event noun), see e.g. Jiménez López and Baños and Pompei, Pompeo, and Ricci in this volume.} This and other support verbs appear in similar constructions in many classical Greek literary texts, but even a single multi-layered text like Aristotle’s \textit{Rhetoric}, which combines the author’s own expressions with those borrowed for paraphrasing or quotation, shows that such a phenomenon exists in both spoken and literary Greek. Two examples will suffice here, see (\ref{Ex1}) and (\ref{Ex2}): 

\ea\label{Ex1}
	\glll διὸ εἴρηται ‘θυμὸς δὲ μέγας ἐστὶ διοτρεφέων βασιλήων’  καὶ ‘ἀλλά τε καὶ μετόπισθεν \textbf{ἔχει} \textbf{κότον}·’ ἀγανακτοῦσι γὰρ διὰ τὴν ὑπεροχήν\\
	\textit{diò} \textit{eírētai} ‘\textit{thumòs} \textit{dè} \textit{mégas} \textit{estì} \textit{diotrephéōn} \textit{basilḗōn}’ \textit{kaì} ‘\textit{allá} \textit{te} \textit{kaì} \textit{metópisthen} \textit{ékhei} \textit{kóton};’ \textit{aganaktoũsi} \textit{gàr} \textit{dià} \textit{tḕn} \textit{huperokhḗn}\\
	therefore say.\textsc{prf.3sg} wrath.\textsc{nom.sg} but big.\textsc{pred-adj} be.\textsc{prs.3sg} Zeus-nurtured.\textsc{gen}	king.\textsc{gen.pl} and yet \textsc{prt} even afterwards have.\textsc{prs.3sg} grudge.\textsc{acc.sg} feel.irritation.\textsc{prs.3pl}	for/since by.reason.of \textsc{art}.\textsc{acc} supremacy.\textsc{acc.sg}\\
	\glt ‘Wherefore it has been said: ‘Great is the wrath of kings cherished by Zeus,’ (\iwi{Homer, \textit{Iliad} 2.196}) and ‘Yet it may be that even afterwards he cherishes his resentment,’ (\iwi{Homer, \textit{Iliad} 1.82}) for kings are resentful in consideration of their superior rank.' \\
 \hspace*{\fill}(\iwi{Aristotle, \textit{Rhetoric} 2.2.7, 1379a3-7}, translated by J. H. Freese).
\z

\ea\label{Ex2}
	\glll καὶ τὸ Πολυεύκτου εἰς ἀποπληκτικόν τινα Σπεύσιππον, τὸ μὴ δύνασθαι \textbf{ἡσυχίαν} \textbf{ἄγειν} ὑπὸ τῆς τύχης ἐν πεντεσυρίγγῳ νόσῳ δεδεμένον\\
	\textit{kaì} \textit{tò} \textit{Polueúktou} \textit{eis} \textit{apoplēktikón} \textit{tina} \textit{Speúsippon} \textit{tò} \textit{mḕ} \textit{dúnasthai} \textit{hēsukhían} \textit{ágein} \textit{hupò} \textit{tē̃s} \textit{túkhēs} \textit{en} \textit{pentesuríngōi} \textit{nósōi} \textit{dedeménon}\\
	and that.[saying] Polyeuctus.\textsc{gen.sg} in/towards apoplectic.\textsc{acc.sg} some Speusippus.\textsc{acc.sg} \textsc{art} \textsc{neg} be.able.\textsc{inf} stillness.\textsc{acc.sg} keep/observe.\textsc{prs.inf} by \textsc{art.gen} fortune.\textsc{gen} in five.holed.\textsc{dat} disease.\textsc{dat.sg} bind.\textsc{prf.ptcp.pass.acc.sg}\\
	\glt ‘And the saying of Polyeuctus upon a certain paralytic named Speusippus, that he could not \textbf{keep quiet}, although Fortune had bound him in a five-holed pillory of disease.’ \\
 \hspace*{\fill}(\iwi{Aristotle, \textit{Rhetoric} 3.10.7, 1411a21-23}, translated by J. H. Freese) 
\z

The phrase ἔχει κότον \textit{ékhei kóton} ‘holds wrath’, ‘cherishes resentment’ in example (\ref{Ex1}), as quoted from the Iliad, in Book 2 (\iwi{Aristotle, \textit{Rhetoric} 2.2.7}), for the sake of brevity, could be replaced by the epic verb κοτέει \textit{kotéei},\footnote{ Only other forms are attested in Homer, but cf. famous dictum in Hes. \textit{Op}. 25.} while another one, ἡσυχίαν ἄγειν \textit{hēsukhían ágein} (example \ref{Ex2}), paraphrased in Book 3 from an unknown speech by Polyeuctus, stands for ἡσυχάζειν \textit{hēsukházein}, which is quite a common verb for Aristotle himself and his contemporary writers.\footnote{As becomes clear from the entry for ἡσυχάζω \textit{hēsukházō} in \citet{Liddell-Scott-Jones1996} and a simple search for this verb in the \textit{Thesaurus Linguae Graecae}.} Both examples conform with Alexander’s definition of periphrasis, both are rather verbose or ‘macrological’ than the reverse, and both resemble a typical SVC definition (desemanticised verb of frequent use acting as the syntactic operator + verbal noun, functioning together as one predicate).

Although περίφρασις \textit{períphrasis} is absent from the extant rhetorical τέχναι \textit{tékhnai} of Aristotle’s time, some discussion of the phenomenon could be found in Aristotle’s \textit{Rhetoric} too, especially in his discussion of style in Book 3.\footnote{The greater part of this book of \textit{Rhetoric} (chapters 1--12) is devoted to the rhetorical aspect of λέξις \textit{léxis}, and the remainder (13--19) to that of τάξις \textit{táxis}.} Here, in the context of the treatment of so-called virtues of style, clarity, correctness (τὸ ἑλληνίζειν \textit{tò hellēnízein}), and propriety (τὸ πρέπον \textit{tò prépon}), we read a statement that must have been dear to Aristotle, both as a writer and as a teacher of a rhetorical doctrine:

\ea\label{Ex3}
    \glll ὅλως δὲ δεῖ εὐανάγνωστον εἶναι τὸ γεγραμμένον καὶ εὔφραστον: ἔστιν δὲ τὸ αὐτό \\ 
    \textit{hólōs} \textit{dè} \textit{deĩ} \textit{euanágnōston} \textit{eĩnai} \textit{tò} \textit{gegramménon} \textit{kaì} \textit{eúphraston}: \textit{éstin} \textit{dè} \textit{tò} \textit{autó} \\
    %VBF inserted 
    generally \textsc{prt} it.is.necessary easy.to.read be.\textsc{inf} the.\textsc{acc} write.\textsc{prf.ptcp.pass.acc} and easy.to.utter be.\textsc{prs.3sg} \textsc{prt} the.\textsc{nom} same.\textsc{nom}\\
    \glt ‘Generally speaking, that which is written should be easy to read or easy to utter, which is the same thing.' \\
    \hspace*{\fill}(\iwi{Aristotle, \textit{Rhetoric} 3.5.6, 1407b11-12}, translated by John H. Freese). 
\z

An anonymous scholion on this passage interprets the identity of the terms εὐανάγνωστον \textit{euanágnōston} and εὔφραστον \textit{eúphraston} as a measure of the text’s clarity. Despite Freese's translation ‘easy to utter', \textit{eúphrastos}, according to the meaning of the synonym εὐφραδής \textit{euphradḗs} in Liddell-Scott-Jones’ \textit{Greek-English Lexicon} (\citealt{Liddell-Scott-Jones1996}), and the etymology of the root -φραδ- \textit{phrad}-\footnote{The verb φράζειν \textit{phrázein} (according to Aristarchus, cf. \citealt{Liddell-Scott-Jones1996} s.v.) was not used by Homer in the sense ‘to say, tell'.} of the verb φράζειν \textit{phrázein}, the two terms mean rather ‘easy to understand', ‘easy to express', or ‘well expressed', ‘well explained'. Of course, there is not yet the term of periphrasis here, to be coined by later rhetoricians, but this already implies a search for terms that refer to different linguistic strategies of expressing thoughts.

\largerpage
In fact, there were at least two such strategies in Aristotle’s time with appropriate, albeit not well-established, terms for each: συντομία \textit{suntomía} ‘brevity', as used by Plato and Aristotle, or βραχυλογία \textit{brakhulogía}, as in the \textit{Rhetoric to Alexander} (\iwi{Aristotle, \textit{Rhetoric to Alexander} 6.3; cf. 22.5}), and possibly (though not surely)\footnote{It should be noted that in the texts of Aristotle’s contemporaries, where the words μακρολογεῖν \textit{makrologeĩn} μακρολογία \textit{makrología} are used, they do not have a strictly technical meaning of a linguistic nature (choice of words, expansion of the text by longer lexical-syntactic units); rather, they are used in a more general sense in terms of genre (rhetorical speech vs. dialogue) and content (richness vs. scarcity of the elements of some topic).} and μακρολογία \textit{makrología}, called ὄγκος \textit{ónkos} by \iwi{Aristotle, \textit{Rhetoric} 3.6.1, 1407b}. 

βραχυλογία \textit{brakhulogía} and μακρολογία \textit{makrología} are not systematically discussed in ancient theories of style and their meanings are usually reduced to asyndeton (\iwi{Quintilian, \textit{Institutio Oratoria} 9.3.50}) and redundancy (\iwi{Quintilian, \textit{Institutio Oratoria} 8.3.53}). In fact, the compounds βραχυλογεῖν \textit{brakhulogeĩn}, μακρολογεῖν \textit{makrologeĩn}, and their derivatives in Aristotle’s time also referred to a stylistic tactic of linguistic communication: βραχυλογία \textit{brakhulogía} was the principle of naming things concisely, μακρολογία \textit{makrología} was the opposite. The former was associated with the pointed questions and straight answers of dialectics, the latter with rhetorical speeches.\footnote{These principles are well expressed by Plato, especially in the dialogues devoted to sophistic topics, see \iwi{Plato, \textit{Protagoras} 335b8}, \iwi{Plato, \textit{Gorgias} 449c4-d6}, \iwi{Plato, \textit{Sophist} 268b1-9} etc. Aristotle himself mentions μακρολογία \textit{makrología} in \iwi{Aristotle, \textit{Rhetoric} 3.17.16, 1418b25}, referring more to a naturally occurring practice in which the speaker exaggerates his self-presentation than to a cleverly balanced or consciously extended rhetorical strategy.} 


It is not impossible in this context that Aristotle distinguished between the tactics of style not only as a theorist but also as a practitioner, language user (writer, imitator, creator, teacher).\footnote{On Aristotle’s careful construction of sentences and the application of the rhetorical figure \textit{hyperbaton} in a particular passage of the \textit{Rhetoric}, see \citet{Martin2001}, and on Aristotle’s experimental attitude to language and important inventions, see \citet{Allan2004}.} The frequent presence of both elliptical and amplificatory expressions in the text of his Τέχνη \textit{Tékhnē} reinforces this assumption. Example (\ref{Ex4}) shows a typical syntax of rather unpolished text which nevertheless shows signs of professional stylistic skills even in a text of esoteric nature.\footnote{On the esotericism of the Aristotelian Corpus and the ‘quite rough prose' of the \textit{Rhetoric}, cf. \citet{Poster1997} and \citet[3]{Kennedy2007}.}

\largerpage
\ea\label{Ex4}
	\glll ἔτι ὑφ’ ὧν τις οἴεται \textbf{εὖ} \textbf{πάσχειν} δεῖν· οὗτοι δ' εἰσὶν οὓς εὖ πεποίηκεν ἢ ποιεῖ, αὐτὸς ἢ δι’ αὐτόν τις ἢ τῶν αὐτοῦ τις, ἢ βούλεται ἢ ἐβουλήθη.\\
    \textit{éti} \textit{huph᾽} \textit{hō̃n} \textit{tis} \textit{oíetai} \textit{eũ} \textit{páskhein} \textit{deĩn}; \textit{hoũtoi} \textit{d᾽} \textit{eisìn} \textit{hoùs} \textit{eũ} \textit{pepoíēken} \textit{ḕ} \textit{poieĩ}, \textit{autòs} \textit{ḕ} \textit{di᾽} \textit{autón} \textit{tis} \textit{ḕ} \textit{tō̃n} \textit{autoũ} \textit{tis}, \textit{ḕ} \textit{boúletai} \textit{ḕ} \textit{eboulḗthē}”\\
	yet from whom.\textsc{gen.pl} someone thinks.\textsc{prs.3sg} well suffer.\textsc{prs.inf} there.is.need.\textsc{prs.inf} these.\textsc{nom.pl} and   be.\textsc{prs.3pl} whom well do.\textsc{prf.3sg} or do.\textsc{prs.3sg} himself or by.aid.of he.\textsc{acc.sg} someone or those.\textsc{gen.pl} he.\textsc{gen.sg} someone or wishes/desires.\textsc{prs.3sg} or wish.\textsc{aor.3sg}\\
	\glt ‘Further, [men are angry at slights from those]\footnote{Here we use square brackets to mark the ellipsis.} by whom they think they have a right to expect to be well treated; such are those on whom they have conferred or are conferring benefits, either themselves, or someone else for them, or one of their friends; and all those whom they desire, or did desire, to benefit' \\
 \hspace*{\fill}(\iwi{Aristotle, \textit{Rhetoric} 2.2.8, 1379a6-8}, translated by J. H. Freese).
\z

Here, ἔτι \textit{éti}, which is used in the same way as in the previous sentence, precedes the implied governing phrase προσήκειν οἴεται πολυωρεῖσθαι \textit{prosḗkein oíetai poluōreĩsthai} ‘he thinks it is proper for him to be highly esteemed', which is omitted, as is the genitive of the omitted phrase ὑπὸ τούτων \textit{hupò toútōn} ‘by these'. Extended speech is indicated by the following additional factors: the separation of subject and predicate by the particle ἤ \textit{ḗ}, the use of εὖ πάσχειν \textit{eũ páskhein} instead of something like one-word εὐπαθεῖν \textit{eupatheĩn} or εὐπραγεῖν \textit{euprageĩn},\footnote{εὐπαθεῖν \textit{eupatheĩn} is attested in Plato (esp. \iwi{Plato, \textit{Phaedrus} 247d4}, \iwi{Plato, \textit{Republic} 347c7}), and εὐπραγεῖν \textit{euprageĩn} in Aristotle (e.g. \iwi{Aristotle, \textit{Rhetoric} 2.9.7, 2.9.9})} and the use of the passive construction (ὑφ’ ὧν \textit{huph᾽ hō̃n}...) rather than the active.

All this shows that the lexical and syntactic material of Aristotle’s \textit{Rhetoric} can be seen as the result of the interplay of ‘brachylogical’ and ‘macrological’ strategies and that the MWEs (‘linguistic objects consisting of two or more words') can be hypothetically associated with the latter.

Since SVCs, like periphrases, imply the use of more than one word and, in some cases, the substitution of a single word (a lexical verb whose meaning is echoed by a noun of verbal derivation, the constituent of an SVC) by a longer phrase, as if transforming the meaning of that word in the combination of two, albeit of unequal semantic weight, it is conceivable to think of these terms as synonyms by virtue of this similarity: SVCs as a type of periphrasis (verbal or predicative), and periphrasis itself as a general name for multi-word combinations of a similar category in which the substitution of a shorter lexical unit by a longer expression is discernible. 


In this way, the tripartite typology of word combinations (e.g. \citealt{VanderMeer1998}, also in \citealt{Marcinkeviciene2010}) could be merged with the typology of periphrases, so that periphrases could also include collocations, idioms, and other word combinations (e.g. compositional phrases, CPs henceforth). If it is possible to name a sequence of word combinations according to the looseness of their syntactic, lexical, and semantic relationships (free combinations – collocations – idioms; cf. \citealt[88]{Marcinkeviciene2010}), some periphrases can be classified as freely formed, others as collocations, since they are already characterised by the suspension of word meaning and their frequent use (which does not, however, prohibit their formation in the form of paraphrases, especially in poetry), and the others as idioms —word combinations characterised by the greatest suspension of meaning.

\section{In search of support-verb constructions in Aristotle’s \textit{Rhetoric}: Key points of empirical research on multi-word expressions}\label{Section4Rhet}
\largerpage
What follows below is a brief description of the stages of empirical work of the author of the present chapter. This work roughly happened in three interwoven stages: 1) empirical collection of the material, 2) search for theoretical models to classify the results, and 3) counting and sorting the material. In the first stage, about 900 two-plus-word phrases were collected, of which 350 items were most similar either to verb-based collocations, or SVCs. To achieve this, some sort of sifting and exclusion was necessary: the so-called free word combinations were excluded, while collocation-like expressions and combinations of verb derivatives (participles, adjectives) with nouns were accepted. Not only verb + noun formations were taken into consideration, but, as our concern is with various periphrases, also verb combinations with other complements (esp. adjectives and adverbs).\footnote{Adjectives of neuter gender can frequently express the meaning of a noun (and so, in fact, substitute nouns), whereas the more common combinations of verbs and adverbs (in fact collocations) are found in grammars under the name of periphrases (cf. \citealt[§1438]{Smyth1920} on adverbs with ἔχειν \textit{ékhein} or διακεῖσθαι \textit{diakeĩsthai}).}

The second stage, which dealt with terminological questions of naming and classifying expressions, was by no means easier. There are still many ambiguities in this area (how many different types of word combinations and periphrases there are in general, how they differ from each other, whether periphrasis is morphologically primary (cf. \citealt[5]{Chumakina-Corbett2012}) or not, whether it belonging to an inflectional paradigm and having multiple exponents is a necessary prerequisite of periphrasis, etc.), but this does not prevent us from sticking to the favoured term (periphrasis): it is quite flexible and can serve as a general term for different constructs, including SVCs. 


On the other hand, the variety of SVCs and expressions similar to them need further clarification and subdivision (as is not the case currently), since even the examples of the periphrases given by the above-mentioned rhetorician Alexander Numenius (2nd c. AD), are of at least two different types, one with the same subject (τὴν μάθησιν ἐποιεĩσθε \textit{tḕn máthēsin epoieĩsthe} = ἐμανθάνετε \textit{emanthánete}, the subject being ὑμεĩς \textit{humeĩs}, ‘you’ (pl.), in both cases), and another with a change in the subject of the sentence (ἔννοιά ποθ' ἡμĩν ἐγένετο \textit{énnoiá poth' hēmĩn egéneto} = ἐνενοήσαμεν \textit{enenoḗsamen}). In this study, we would like to highlight that, while a noun may have a greater significance as the semantic head in the typology of SVCs, a particular verb’s semantic import may also play a role. 

\section{On verbs forming periphrastic constructions: The idea of extra- and introversive verbs}\label{Section5Rhet}

While the definitions of SVCs emphasise the reduction of the semantic role of the verb, our intuition is that some of the verbs’ fundamental morpho-semantic aspects or features can be retained, leading to different verb-noun combinations with the same noun.

One such primary retainable aspect relates to the valency properties of the verb, i.e. the ability or inability to handle one or more complements. This intuition is in line with several theoretical frameworks, first of all, with the grammatical theory of valency, based on verb centricity (verbs structure sentences by binding the specific elements (complements and actants) in the same way as atoms of chemical elements do), with Lucien Tesnière’s theory of actants (agents or persons accompanying a verb in the form of the nominative, the accusative, and the dative cases respectively)\footnote{See further \citet[100--109]{Tesniere2015}.} and verbal node with its theatrical metaphor (‘like a drama, it obligatorily involves a process and most often actors and circumstances', \citealt[97]{Tesniere2015}). Notably, even when not acting in their full lexical meaning, verbs that form SVCs retain their bivalence (+nominative, +accusative), and in combination with the complement they can also become/seem to become trivalent (cf. ἔχω \textit{ékhō} + accusative > χάριν ἔχω \textit{khárin ékhō} + dative).

The observations on the verbal node as a metaphorical drama (or verb-governor in dependency grammar) and research on verbal derivations and valency change (variety of cross-linguistic morpho-syntactic strategies in transitivity alternations) reflect a general paradigm comparable, from our point of view, with Aristotle’s rhetorical model of persuasion, consisting of a triad of factors in the process of rhetorical action (also full of alternating stylistic strategies): the speaker’s ἦθος \textit{ē̃thos} (moral nature), the hearer’s πάθος \textit{páthos} (emotional condition), and the λόγος \textit{lógos} (rational basis, logical validity) of the speech. 


Aristotle’s scheme, most explicitly stated in \iwi{Aristotle, \textit{Rhetoric} 1.2.3}, parallels the semantic and syntactic relations between the participants (or actors) of the sentence in their connection to verbs of different valencies.\footnote{In rhetoric, the activity of verbs is probably paralleled by the ὑπόκρισις \textit{hupókrisis}, which, depending on the characteristics of each situation and the characters of the actors, can be different, both highly static and dynamic.} The speaker, the messenger, as if the agent of the sentence, is the initiating actor who, through his self-presentation and speech (or act of predication comparable to the function of a verb in a sentence), affects one or more ‘actors’, one of whom is the product of the logical material, the λόγος \textit{lógos}, the meaningful text (parallel to the object of the sentence, which represents the great variety of things), and another, the listener (or group of listeners) is the reactive agent, the recipient of the affection or message (like the secondary objects of the sentence). 

However, every text (oral or written) is not just a collection of identical sentences with identical verb properties. Variation, or variability, is important for rhetorical success, and the possibilities of word derivation help to achieve it. In Greek, the possibilities of derivation, both synthetic and analytic, are rather vast.\footnote{For a significant account of the possibilities of derivation and compounding, or word formation in general, in ancient Greek and Aristotle’s contribution to the conceptualization of these processes, see e.g. \citet{Wouters-etal2014} and \citet{Vaahtera2014}.} From some studies on word derivation we have important terms coined that describe variations in verb valency: extraversion and introversion. According to Lehmann and Verhoeven, extraversion is the process by which an intransitive (or monovalent) verb becomes a transitive (or bivalent) verb, and the reverse process is called introversion \citep[468--469]{Lehmann-Verhoeven2006}.


A simplified example of derivational extraversion would be to change the intransitive exhortation ‘let’s gamble’ (cf. Lith. \textit{loškime}, and Gr. κυβεύωμεν \textit{kubeúōmen}) into a sentence where the same verb becomes transitive: ‘I gambled away all my money’ (cf. Lith. \textit{aš pralošiau visus savo pinigus}, and Gr. κατεκύβευσα ἅπαν τὸ ἀργύριον \textit{katekúbeusa hápan tò argúrion}\footnote{Cf. \iwi{Lysias, \textit{In Alcibiadem I} 27}: κατακυβεύσας τὰ ὄντα \textit{katakubeúsas tà ónta}.}). This example of extraversion shows the ability of language to derive a transitive verb from an intransitive verb by adding certain analytical adjuncts. The phenomenon is well attested across languages and the term ‘ambi-transitive’ or ‘labile’ is applied to such verbs (\citealt[57]{Arkadiev-Pakerys2015}, \citealt[68]{Lavidas2009}, \citealt[38]{Haspelmath2016}, etc.). This is a situational and context-dependent change, i.e. situational extraversion. 

It is important to note, though, that Aristotle’s \textit{Rhetoric} exhibits both situational valency (cf. the transitive πράττειν \textit{práttein} in πράττειν τὰ καλά \textit{práttein tà kalá} in \iwi{Aristotle, \textit{Rhetoric} 1.7.38, 2.12.12}, and the intransitive one κακῶς / εὖ πράττειν \textit{kakō̃s / eũ práttein} in \iwi{Aristotle, \textit{Rhetoric} 2.9.2, 2.9.4}), which is dependent on the production process of the phrases, and the internal valency, the latter innate to each verb. The premise of this observation is that most transitive verbs fall into two categories depending on their underlying meaning: introversive and extraversive. 

This intuition is based on the assumption that transitive verbs can be used to express the direction of an action in one of two ways: either inwards, i.e. towards the area that is closer to the main performer of the action, or outwards, i.e. towards a more open area that does not belong to the performer or is distant from him/her. When we say ‘he/she gives, sells, carries, strikes, draws’, if we do not add the reflexive form, we refer to actions that are distant from the performer, and we focus on the exterior object, a component of the world that does not belong to the performer (‘gives, sells’, thus ‘takes away from himself’, ‘carries, strikes’, thus ‘uses his strength instead of replenishing it’, ‘draws’, thus ‘puts the idea on display to be seen by others’). When we say ‘takes, buys, owns, feels, sees’, we are focusing on the performer's inner world. In a way, this classification of verbs is reminiscent of semantic classes such as action verbs and stative verbs, except that it primarily concerns the categorisation of transitive verbs.

Thus, based on these considerations, extraversive verbs are those transitive and ambi-transitive verbs which imply a transfer in attention to an external object (‘I make, give, send, say’ etc.), while introversive verbs suggest a change in emphasis from an exterior object and/or subject to the main subject (‘I feel, receive, get, hear’). This difference in verbs might also be a prerequisite for the ramification of the semantic or syntactic roles of the respective phrases and for the nuances of their translation.\footnote{For example, the extraverted phrase may be ‘exert pressure' and the introverted one ‘feel pressure' or the extraverted phrase could be ‘tell the truth', and the introverted one ‘know the truth'. So perhaps ἔχω χάριν \textit{ékhō khárin} = χαρίζομαι \textit{kharízomai} ‘I feel grateful’, χάριν δίδωμι \textit{khárin dídōmi} = χαρίζω \textit{kharízō} ‘I express/share my gratitude’?}

\section{Most frequent ‘support verbs’ and potential support-verb-construction-type periphrases in Aristotle}\label{Section6Rhet}

Among the 350 constructions,\footnote{This figure can be verified by summing up the number of constructions given in \tabref{tab:overview1}, \tabref{tab:overview2}, and the table provided as the dataset for this chapter, see n. 1.} selected from around 900 phrasal combinations, we identified the following most frequent extraversive verbs: \textbf{διδόναι} \textit{didónai} ‘to give’, \textbf{λέγειν} \textit{légein} ‘to say’, \textbf{ποιεῖν} \textit{poieĩn} ‘to make’ and \textbf{ποιεῖσθαι} \textit{poieĩsthai} ‘to make (for onself)’, \textbf{τιθέναι} \textit{tithénai} ‘to put’, and \textbf{φέρειν} \textit{phérein} ‘to bring’, ‘carry’.

Most of them correspond to English light verbs. They typically direct the action towards the object (accusativus rei) and/or the recipient of the benefit or harm, expressed by the dative case or its syntactic equivalents (πρός τινα \textit{prós tina}, εἴς τινα \textit{eís tina} etc.). Versions with prefixes, such as ἀποδιδόναι \textit{apodidónai}, ἐπιλέγειν \textit{epilégein}, ἐμποιεῖν \textit{empoieĩn}, διατιθέναι / διατίθεσθαι \textit{diatithénai / diatíthesthai}, κατασκευάζειν \textit{kataskeuázein}, and παρασκευάζειν \textit{paraskeuázein}, were also included in the analysis. However, verbs with objects in the dative and genitive cases (such as χρῆσθαι \textit{khrē̃sthai} + dative or τυγχάνειν \textit{tunkhánein} + genitive) were not thoroughly examined at this stage of the research, so they are not covered in the present discussion.

%
% \clearpage
% \begin{landscape}
% {\scriptsize
% \begin{xltabular}{\linewidth}{>{\hsize=0.9\hsize}X >{\hsize=0.5\hsize}X >{\hsize=1.5\hsize}X >{\hsize=1.7\hsize}X >{\hsize=0.7\hsize}X >{\hsize=0.7\hsize}X}
% \caption{Periphrases with extraversive verbs\label{Table1Arist}}\\
% 	\lsptoprule
% 	Phrasal formula & No. of tokens / no. of types & No. of repeated types (with morphological variations), and list of V+SO and V+CO & No. of unrepeated types (occurring only once), and list of SO and CO & No. of types with SO, and potential SVCs & No. of types with CO, incl. Acc. duplex\\
% 	\midrule
%  \endfirsthead
%  \caption[]{Periphrases with extraversive verbs (continued)} \\
%     \lsptoprule
% 	Phrasal formula & No. of tokens / no. of types & No. of repeated types (with morphological variations), and list of V+SO and V+CO & No. of unrepeated types (occurring only once), and list of SO and CO & No. of types with SO, and potential SVCs & No. of types with CO, incl. Acc. duplex\\
% 	\midrule
% \endhead
%     \midrule
%     \multicolumn{6}{r}{\tiny\textit{Continues on next page}}
% \endfoot
%     \lspbottomrule
% \endlastfoot
% 	\textbf{διδόναι, ἀποδιδόναι, ἀνταποδιδόναι} (\textit{didónai, apodidónai, antapodidónai}) + Acc. & \multicolumn{1}{c}{14 / 9} &  \hfil{3(3+0)\footnote{\tiny In the brackets, the first number indicates the amount of verb-controlled single objects, and the second number refers to complex objects and objects with attributes.}}\newline V+SO: 1) χάριν διδόναι / ἀνταποδιδόναι / ἀποδιδόναι (\textit{khárin didónai / antapodidónai / apodidónai}) (thrice in total: \iwi{Aristotle, \textit{Rhetoric} 1.1.10, 2.2.17, 2.2.23}); 2) δοῦναι δίκην \textit{doũnai díkēn} (twice: \iwi{Aristotle, \textit{Rhetoric} 1.12.1, 1.12.3}); 3) διδόναι / δοῦναι φυλακήν (\textit{didónai / doũnai phulakḗn}) (twice: \iwi{Aristotle, \textit{Rhetoric} 2.20.5} (\textit{bis})) & \hfil{6 (5+1)\footnote{\tiny The function of the numbers in the brackets is analogous to the one in the previous footnote.}}\newline SO: 1) τὰς κρίσεις \textit{tàs kríseis} (\iwi{Aristotle, \textit{Rhetoric}  1.2.5}), 2) τὰ δίκαια \textit{tà díkaia} (\iwi{Aristotle, \textit{Rhetoric} 2.23.12}), 3) [ὅρκους] [\textit{hórkous}] (omitted Acc.) (\iwi{Aristotle, \textit{Rhetoric} 1.15.2}), 4) τὴν πρόθεσιν \textit{tḕn próthesin} (\iwi{Aristotle, \textit{Rhetoric} 2.18.5}), 5) αἵρεσιν \textit{haíresin} (\iwi{Aristotle, \textit{Rhetoric} 2.24.9}). CO: 1) τὸ δίκαιον καὶ τό συμφέρον \textit{tò díkaion kaì tò sumphéron} (\iwi{Aristotle, \textit{Rhetoric} 1.1.7}) & \multicolumn{1}{c}{8 (3+5)\footnote{\tiny These brackets show the data from the third and fourth columns.}} & \multicolumn{1}{c}{1 (0+1)\footnote{\tiny The function of the numbers in the brackets is analogous to the one in the previous footnote.}} \\
% 	\textbf{λέγειν, εἰπεῖν (\textit{légein, eipeĩn})} + Acc. &  \multicolumn{1}{c}{29 / 19} & \hfil{4 (3+1)}\newline V+SO: 1) λέγειν / ἐπιλέγειν τήν αἰτίαν / τὰς αἰτίας / τὸ αἴτιον (\textit{légein / epilégein tḕn aitían / tàs aitías / tò aítion}) (five times in total: \iwi{Aristotle, \textit{Rhetoric} 1.2.11} (ἐρεῖν \textit{ereĩn}), \iwi{Aristotle, \textit{Rhetoric} 2.9.5} (τὰς εἰρημένας αἰτίας \textit{tàs eirēménas aitías}), \iwi{Aristotle, \textit{Rhetoric} 2.23.24} (twice: λέγειν τὴν αἰτίαν \textit{légein tḕn aitían} and λεχθέντος τοῦ αἰτίου \textit{lekhthéntos toũ aitíou}), \iwi{Aristotle, \textit{Rhetoric} 2.21.14} (ἐπιλέγειν \textit{epilégein}), & \hfil{15 (11+4)}\newline SO: 1) οὐδέν \textit{oudén} (\iwi{Aristotle, \textit{Rhetoric}  1.1.3}), 2) παραδείγματα \textit{paradeígmata} (\iwi{Aristotle, \textit{Rhetoric} 1.2.8}), 3) ὑποθήκας \textit{hupothḗkas} (\iwi{Aristotle, \textit{Rhetoric} 1.9.36}), 4) τὰ ψευδῆ \textit{tà pseudē̃} (\iwi{Aristotle, \textit{Rhetoric} 1.15.26}), 5) παράδοξον \textit{parádoxon} (\iwi{Aristotle, \textit{Rhetoric} 2.21.4}), 6) τὰς γνώμας \textit{tàs gnṓmas} (\iwi{Aristotle, \textit{Rhetoric} 2.21.13}), 7) φανερά \textit{phanerá} (\iwi{Aristotle, \textit{Rhetoric} 2.22.3}), 8) τὰ δίκαια \textit{tà díkaia} (\iwi{Aristotle, \textit{Rhetoric} 2.23.15}), 9) τὰ ἄδικα \textit{tà ádika} (\iwi{Aristotle, \textit{Rhetoric} 2.23.15}), & \multicolumn{1}{c}{14 (3+11)} & \multicolumn{1}{c}{5 (1+4)} \\
%     \hfill & \hfill & 2) (τὰ) ἐνθυμήματα λέγειν / ἐνθύμημα εἰπεῖν (\textit{(tà) enthumḗmata légein / enthúmēma eipeĩn}) (four times in total: \iwi{Aristotle, \textit{Rhetoric} 1.2.8, 1.2.14, 1.15.19, 1.2.21}), 3) τἀληθῆ \textit{talēthē̃} (twice: \iwi{Aristotle, \textit{Rhetoric} 1.15.26} (\textit{bis})) V+CO: 1) (τὰ) ἔξω τοῦ πράγματος λέγειν / τεχνολογεῖν \textit{(tà) éxō toũ prágmatos légein / tekhnologeĩn} (thrice in total: \iwi{Aristotle, \textit{Rhetoric}  1.1.5, 1.1.9, 1.1.10}) & 10) λόγον lógon (\iwi{Aristotle, \textit{Rhetoric} 2.20.5} (εἰπεῖν eipeĩn)), 11) τἀναντία \textit{tanantía} (\iwi{Aristotle, \textit{Rhetoric} 2.23.12}); CO: 1) [τοὺς ἐπαίνους καὶ τοὺς ψόγους \textit{toùs epaínous kaì toùs psógous}] (ex pass. οἱ ἔπαινοι καὶ οἱ ψόγοι λέγονται \textit{hoi épainoi kaì hoi psógoi légontai}) (\iwi{Aristotle, \textit{Rhetoric} 1.9.41}), 2) τὰ κοινὰ καὶ καθόλου \textit{tà koinà kaì kathólou} (\iwi{Aristotle, \textit{Rhetoric} 2.22.3}), 3) [τὰ] ἐξ ὧν ἴσασι καὶ τὰ ἐγγύς \textit{[tà] ex hō̃n ísasi kaì tà engús} (\iwi{Aristotle, \textit{Rhetoric} 2.22.3}), 4) δόξαν τινά \textit{dóxan tiná} (\iwi{Aristotle, \textit{Rhetoric} 2.26.4}) & \hfill & \hfill \\
% 	\textbf{ποιεῖν, ποιῆσαι, ἐμποιεῖν (\textit{poieĩn, poiē̃sai,  empoieĩn})} + Acc. & \multicolumn{1}{c}{29 / 26} & \hfil{2 (1+1)}\newline V+SO: 1) τἀυτὸ / τἀυτὰ ποιεῖν (\textit{tautò / tautà poieĩn}) (twice in total: \iwi{Aristotle, \textit{Rhetoric} 2.2.9; 2.2.16}); V+CO: 1) τοὺς λόγους ἠθικοὺς ποιεῖν (\textit{toùs lógous ēthikoùs poieĩn}) (thrice in total with variations in word order: \iwi{Aristotle, \textit{Rhetoric} 2.18.1; 2.18.2; 2.21.16}) & \hfil{24 (9+15)}\newline SO: 1) μεγάλα \textit{megála} (\iwi{Aristotle, \textit{Rhetoric} 1.7.32}), 2) ἡδύ \textit{hēdú} (\iwi{Aristotle, \textit{Rhetoric} 1.11.4}), 3) ὑπερβολήν \textit{huperbolḗn} (\iwi{Aristotle, \textit{Rhetoric} 1.11.20}), 4) [ἀγαθά] [\textit{agathá}] (\iwi{Aristotle, \textit{Rhetoric} 1.13.18}: ἀγαθῶν ὧν ἐποίησεν > [ποιῆσαι ἀγαθά] \textit{agathō̃n hō̃n epoíēsen} > [\textit{poiē̃sai agathá}]), 5) τἀναντία \textit{tanantía} (\iwi{Aristotle, \textit{Rhetoric} 2.2.17}), 6) τὸν ἔλεον \textit{tòn éleon} (\iwi{Aristotle, \textit{Rhetoric} 2.8.16}), 7) τὴν συκοφαντίαν \textit{tḕn sukophantían} (\iwi{Aristotle, \textit{Rhetoric} 2.24.10}), 8) τὴν ὀργήν \textit{tḕn orgḗn} (\iwi{Aristotle, \textit{Rhetoric} 2.1.9}), 9) ἡδονήν \textit{hēdonḗn} (\iwi{Aristotle, \textit{Rhetoric} 2.2.2}); & \multicolumn{1}{c}{10 (1+9)} & \multicolumn{1}{c}{16 (1+15)} \\
% 	\hfill & \hfill & \hfill & CO: 1) τ\textbf{ὸν κανόνα} στρεβλόν \textit{\textbf{tòn kanóna}}\footnote{\tiny The direct object (DO) is highlighted in a bolder font.} \textit{streblón} (\iwi{Aristotle, \textit{Rhetoric}  1.1.5}), 2) ὡς ἐλαχίστων κύριον \textbf{τὸν κριτήν} \textit{hōs elakhístōn kúrion \textbf{tòn kritḗn}} (\iwi{Aristotle, \textit{Rhetoric} 1.1.8}), 3) \textbf{τὸν κριτὴν} ποιόν τινα \textit{\textbf{tòn kritḕn} poión tina} (\iwi{Aristotle, \textit{Rhetoric} 1.1.9}), 4) ἀξιόπιστον \textbf{τὸν λέγοντα} \textit{axiópiston \textbf{tòn légonta}} (\iwi{Aristotle, \textit{Rhetoric} 1.2.4}), 5) \textbf{τὸν λέγοντα} ἔμφρονα \textit{\textbf{tòn légonta} émphrona} (\iwi{Aristotle, \textit{Rhetoric} 1.2.21}), 6) μὴ βραδυτέρας \textbf{τὰς κινήσεις} \textit{mḕ bradutéras \textbf{tàs kinḗseis}} (\iwi{Aristotle, \textit{Rhetoric} 1.5.13}), 7) πιστὰς ἢ ἀπίστους [\textbf{τὰς συνθήκας}] \textit{pistàs ḕ apístous [\textbf{tàs sunthḗkas}]} (\iwi{Aristotle, \textit{Rhetoric} 1.15.20}), 8) \textbf{τὸν νόμον} κύριον \textit{\textbf{tòn nómon} kúrion} (\iwi{Aristotle, \textit{Rhetoric} 1.15.21}), 9) βουλευτικοὺς [sc. \textbf{τοὺς ἀνθρώπους}] \textit{bouleutikoùs [sc. \textbf{toùs anthrṓpous}]} (\iwi{Aristotle, \textit{Rhetoric} 2.5.14}), 10) πρὸ ὄμμάτων [\textbf{τὰ κακά}] \textit{prò ommátōn [\textbf{tà kaká}]} (\iwi{Aristotle, \textit{Rhetoric} 2.8.13}), 11) μὴ ἐλεεινὰ \textbf{ἅπαντα} \textit{mḕ eleeinà \textbf{hápanta}} (\iwi{Aristotle, \textit{Rhetoric} 2.9.5}), & \hfill & \hfill \\
%     \hfill & \hfill & \hfill & 12) δίκαια \textbf{πολλά} \textit{díkaia \textbf{pollá}} 13) [\textbf{τοὺς δυναμένους}] σεμνοτέρους \textit{[\textbf{toùs dunaménous}] semnotérous} (Ross) : ἐμφανεστέρους \textit{emphanestérous} (Kassel) (\iwi{Aristotle, \textit{Rhetoric} 2.17.4}), (opp. ἀδικεῖν ἔνια \textit{adikeĩn énia}) (\iwi{Aristotle, \textit{Rhetoric} 1.12.31}), 14) \textbf{τὸν ἥττω λόγον} κρείττω \textit{\textbf{tòn hḗttō lógon} kreíttō} (\iwi{Aristotle, \textit{Rhetoric} 2.24.11}), 15) [\textbf{λόγους}] ὥσπερ καὶ παραβολάς \textit{[\textbf{lógous}] hṓsper kaì parabolás} (\iwi{Aristotle, \textit{Rhetoric} 2.20.7}) & \hfill & \hfill \\
% 	\textbf{κατασκευάζειν (\textit{kataskeuázein})} + Acc. & \multicolumn{1}{c}{3 / 3} & \multicolumn{1}{c}{0} & \hfil{3 (0+3)}\newline CO: 1) καὶ \textbf{αὑτὸν} ποιόν τινα καὶ τὸν κριτήν \textit{kaì \textbf{hautòn} poión tina kaì tòn kritḕn} [sc. ποιόν τινα / \textit{poión tina}] (\iwi{Aristotle, \textit{Rhetoric} 2.1.2}), 2) \textbf{ἑαυτὸν} τοιοῦτον \textit{\textbf{heautòn} toioũton} (\iwi{Aristotle, \textit{Rhetoric} 2.1.7}), 3) [\textbf{τοὺς ἀκροατὰς} \textit{\textbf{toùs akroatàs}}] τοιούτους \textit{toioútous} (\iwi{Aristotle, \textit{Rhetoric} 2.2.27}) & \multicolumn{1}{c}{0} & \multicolumn{1}{c}{3 (0+3)} \\
% 	\textbf{παρασκευάζειν (\textit{paraskeuázein})} + Acc. & \multicolumn{1}{c}{2 / 2} & \multicolumn{1}{c}{0} & \hfil{2 (0+2)} CO: 1)\newline \textbf{αὑτοὺς} τοιούτους \textit{\textbf{hautoùs} toioútous} (\iwi{Aristotle, \textit{Rhetoric} 2.3.17}), 2) \textbf{τοὺς κριτὰς} τοιούτους \textit{\textbf{toùs kritàs} toioútous} (\iwi{Aristotle, \textit{Rhetoric} 2.9.16}) & \multicolumn{1}{c}{0} & \multicolumn{1}{c}{2} \\
% 	\textbf{ποιεῖσθαι (\textit{poieĩsthai})} + Acc. & \multicolumn{1}{c}{9 / 8} & \hfil{1 (1+0)}\newline V+SO: 1) ποιεῖσθαι τὸν λόγον poieĩsthai tòn lógon (twice in total with variation in word order: \iwi{Aristotle, \textit{Rhetoric} 1.5.18, 2.18.1}) & \hfil{7 (3+4)}\newline SO: 1) τὰς πίστεις \textit{tàs písteis} (\iwi{Aristotle, \textit{Rhetoric} 1.2.8}), 2) τὴν κρίσιν \textit{tḕn krísin} (\iwi{Aristotle, \textit{Rhetoric} 2.1.4}), 3) τοὺς συλλογισμούς \textit{toùs sullogismoús} (\iwi{Aristotle, \textit{Rhetoric} 1.10.1})  CO: 1) τὰς πίστεις καὶ τοὺς λόγους t\textit{às písteis kaì toùs lógous} (\iwi{Aristotle, \textit{Rhetoric} 1.1.12}), 2) φίλον \textbf{γέροντα} \textit{phílon \textbf{géronta}} (\iwi{Aristotle, \textit{Rhetoric} 1.15.14}), 3) πολίτας \textbf{τοὺς μισθοφόρους} \textit{polítas \textbf{toùs misthophórous}} (\iwi{Aristotle, \textit{Rhetoric} 2.23.17}), 4) φυγάδας \textbf{τοὺς} [...] \textbf{διαπεπραγμένους} \textit{phugádas \textbf{toùs} [...] \textbf{diapepragménous}} (\iwi{Aristotle, \textit{Rhetoric} 2.23.17}) & \multicolumn{1}{c}{4 (1+3)} & \multicolumn{1}{c}{4 (0+4)}\\
% 	\textbf{πράττειν (\textit{práttein})} + Acc. & \multicolumn{1}{c}{5 / 4} & \hfil{1 (1+0)}\newline V+SO: 1) πράττειν τὰ καλά \textit{práttein tà kalá} (twice: \iwi{Aristotle, \textit{Rhetoric} 1.7.38, 2.12.12}) & \hfil{3 (1+2)}\newline SO: 1) τὰ συμφέροντα \textit{tà sumphéronta} (\iwi{Aristotle, \textit{Rhetoric} 2.12.12}).  CO: 1) τὰ συμφέροντα ἢ βλαβερά \textit{tà sumphéronta ḕ blaberá} (\iwi{Aristotle, \textit{Rhetoric} 1.3.6}), 2) πολλὰ δίκαια \textit{pollà díkaia} (\iwi{Aristotle, \textit{Rhetoric} 1.12.31}). & \multicolumn{1}{c}{2 (1+1)} & \multicolumn{1}{c}{2 (0+2)} \\
% 	\textbf{τιθέναι, θεῖναι (\textit{tithénai, theĩnai})} + Acc. & \multicolumn{1}{c}{3 / 1} & \hfil{1}\newline V+SO: 1) [νόμον θεῖναι (τεθηκέναι)] [\textit{nómon theĩnai (tethēkénai)}] (thrice: \iwi{Aristotle, \textit{Rhetoric} 1.1.7, 1.14.4, 1.15.11}, always in passive construction; hence the periphrasis is only reconstructed) & \multicolumn{1}{c}{0} & \multicolumn{1}{c}{1 (1+0)} & \multicolumn{1}{c}{0} \\
% 	\textbf{φέρειν, ἐνεγκεῖν (\textit{phérein, enenkeĩn})} + Acc. & \multicolumn{1}{c}{10 / 4} & \hfil{3 (3+0)}\newline V+SO: 1) πίστεις φέρειν \textit{písteis phérein} (twice: \iwi{Aristotle, \textit{Rhetoric} 1.7.40, 2.18.2}), 2) φέρειν τὰ ἐνθυμήματα (ἐνθυμήματα φέρειν) \textit{phérein tà enthumḗmata (enthumḗmata phérein)} (twice in total: \iwi{Aristotle, \textit{Rhetoric} 2.22.16, 2.26.3}), 3) ἔνστασιν (ἐνστάσεις) φέρειν (ἐνεγκεῖν) / \textit{énstasin (enstáseis) phérein (enenkeĩn)} (five times in total: \iwi{Aristotle, \textit{Rhetoric} 2.25.1, 2.25.3, 2.25.5, 2.25.8, 2.26.3}) & \hfil{1 (1+0)}\newline SO: 1) τεκμήριον \textit{tekmḗrion} (\iwi{Aristotle, \textit{Rhetoric} 1.2.17}) & \multicolumn{1}{c}{4 (3+1)} & \multicolumn{1}{c}{0} \\
% 	\midrule
% 	\textbf{Total} & {104 / 76} & {15} & {61} & {43} & {33}
% \end{xltabular}
% }\largerpage
% \end{landscape}

\begin{table}[b]
{\scriptsize
\begin{tabularx}{\linewidth}{Qrrrrr}
    \lsptoprule
& tokens/types
& repeated$^*$%\footnote{\tiny In the brackets, the first number indicates the amount of verb-controlled single objects, and the second number refers to complex objects and objects with attributes.}
& unrepeated$^*$%\footnote{\tiny The function of the numbers in the brackets is analogous to the one in the previous footnote.}
& types SO$^\dagger$%\footnote{\tiny These brackets show the data from the second and third columns.}
& types with CO$^\dagger$%\footnote{\tiny The function of the numbers in the brackets is analogous to the one in the previous footnote.}
\\
\midrule
\textbf{διδόναι, ἀποδιδόναι, ἀνταποδιδόναι} (\textbf{\textit{didónai, apodidónai, antapodidónai}}) + Acc.
& 14 / 9 &   3 (3+0) &  6 (5+1) & 8 (3+5) & 1 (0+1) \\
\tablevspace
\textbf{λέγειν, εἰπεῖν (\textit{légein, eipeĩn})} + Acc.
& 29 / 19 &  4 (3+1) &  {15 (11+4)} & 14 (3+11) & 5 (1+4) \\
\tablevspace
\textbf{ποιεῖν, ποιῆσαι, ἐμποιεῖν (\textit{poieĩn, poiē̃sai,  empoieĩn})} + Acc.
& 29 / 26 &  2 (1+1) &  {24 (9+15)} & 10 (1+9) & 16 (1+15) \\
\tablevspace
\textbf{κατασκευάζειν (\textit{kataskeuázein})} + Acc.
& 3 / 3 & 0 &  3 (0+3) & 0 & 3 (0+3) \\
\tablevspace
\textbf{παρασκευάζειν (\textit{paraskeuázein})} + Acc.
& 2 / 2 & 0 &  2 (0+2) & 0 & 2 \\
\tablevspace
\textbf{ποιεῖσθαι (\textit{poieĩsthai})} + Acc.
& 9 / 8 &  1 (1+0) &  7 (3+4) & 4 (1+3) & 4 (0+4)\\
\tablevspace
\textbf{πράττειν (\textit{práttein})} + Acc.
& 5 / 4 &  1 (1+0) &  3 (1+2) & 2 (1+1) & 2 (0+2) \\
\tablevspace
\textbf{τιθέναι, θεῖναι (\textit{tithénai, theĩnai})} + Acc.
& 3 / 1 &  1 & 0 & 1 (1+0) & 0 \\
\tablevspace
\textbf{φέρειν, ἐνεγκεῖν (\textit{phérein, enenkeĩn})} + Acc.
& 10 / 4 &  3 (3+0) &  1 (1+0) & 4 (3+1) & 0 \\
\midrule
\textbf{Total}
& 104 / 76 & 15 & 61 & 43 & 33\\
\lspbottomrule
\end{tabularx}
\caption{Overview}
\label{tab:overview1}
}
\legendbox{$^*$ In the brackets, the first number indicates the amount of verb-controlled single objects, and the second number refers to complex objects and objects with attributes.\\ $^\dagger$ These brackets show the data from the second and third columns.}
\end{table}


Of all the verbs mentioned, 104 tokens (constructions with direct objects) were found in the analysed corpus (76 different types). The count includes formations with the suppletive forms and verbal derivatives (e.g. \textit{adiectiva verbalia}) as well. \tabref{tab:overview1} shows a simplified characterisation of periphrases with extraversive verbs.  
\tabref{tab:overview1} serves as a numeric overview, relevant examples are provided in \tabref{tab:startexampledata} to \tabref{tab:endexampledata}.
For the sake of simplicity, all the morphological variations are counted as though they are reducible to a single phrasal formula (infinitive + accusative of the object), including verb tenses, verbal adjectives, participles, singular and plural forms of nominals. The individual columns indicate the number of repeated and non-repeated expressions, and for each verb (or group of verbs) two categories of objects are distinguished: a single object (SO henceforth), and a complex object (CO henceforth), where verb constructions with an SO are labelled with the abbreviation V + SO and constructions with a CO are labelled V + CO. When CO is an accusative duplex, the direct object (DO henceforth) is marked in bold.



Of all the verb + object (V+O) combinations, the most important feature that brings such a combination closer to the concept of an SVC (a periphrasis of the direct lexical verb) is when the verb has only a single object (V+SO). But the presence of variants with a complex object, CO (noun + adjective or pronoun, noun + noun joined with a conjunction, or accusative duplex), especially the repeated ones, such as (τὰ) ἔξω τοῦ πράγματος λέγειν / \textit{(tà) éxō toũ prágmatos légein} and τοὺς λόγους ἠθικοὺς ποιεῖν / \textit{toùs lógous ēthikoùs poieĩn}, encourages us to distinguish another category next to the SVC category, more ‘macrologic’ an expression than the SVC category.

It should be noted that some polysemous verbs, such as ποιεῖν \textit{poieĩn}, have synonyms (verbs with closely related meanings and similar causative functions) that can form analogous periphrases, or rather patterns of periphrasis, with some variability. For example, the expression ‘(by one’s own speech) to make a judge of a certain state of mind' occurs several times in Aristotle’s treatise (cf. ὅπως \textbf{τὸν κριτὴν} ποιόν τινα \textbf{ποιήσωσιν} / \textit{hópōs tòn kritḕn poión tina poiḗsōsin} (\iwi{Aristotle, \textit{Rhetoric} 1.1.9}), \textbf{κατασκευάζειν} τῷ λόγῳ \textbf{[τοὺς κριτὰς]} τοιούτους / \textit{kataskeuázein tō̃i lógōi [toùs kritàs] toioútous} (\iwi{Aristotle, \textit{Rhetoric} 2.2.27}), ἐὰν \textbf{τούς} τε \textbf{κριτὰς} τοιούτους \textbf{παρασκευάσῃ} ὁ λόγος / \textit{eàn toús te kritàs toioútous paraskeuásē̃i ho lógos} (\iwi{Aristotle, \textit{Rhetoric} 2.9.16})), and always with some difference: the verbs vary (ποιεῖν, κατασκευάζειν, παρασκευάζειν \textit{poieĩn, kataskeuázein, paraskeuázein}), as does the way the verb’s object is inflected (singular, plural, or naturally omitted), and the predicative object is also inflected differently (either the accusative of τοιοῦτος \textit{toioũtos} or a combination of pronouns denoting indefiniteness, ποιός τις \textit{poiós tis}).

The following tables also show the variability of the grammatical tense categories and the suppletive forms of the verbs involved in the periphrases (cf. λέγειν \textit{légein} and εἰπεῖν \textit{eipeĩn}, φέρειν \textit{phérein} and ἐνεγκεῖν \textit{enenkeĩn}, etc.), and thus the irregularity that prevents the conclusion of a fixed rule for certain word combinations.


\begin{table}
\footnotesize
\begin{tabularx}{\textwidth}{QQ}
        \lsptoprule
        Repeated types (with morphological variations), and list of V+SO and V+CO & Unrepeated types (occurring only once), and list of SO and CO     \\
        \midrule
        \textbf{V+SO:}\newline
        \textbf{1)} χάριν διδόναι / ἀνταποδιδόναι / ἀποδιδόναι (\textit{khárin didónai / antapodidónai / apodidónai}) (thrice in total: \iwi{Aristotle, \textit{Rhetoric} 1.1.10, 2.2.17, 2.2.23});\newline
        \textbf{2)} δοῦναι δίκην \textit{doũnai díkēn} (twice: \iwi{Aristotle, \textit{Rhetoric} 1.12.1, 1.12.3});\newline
        \textbf{3)} διδόναι / δοῦναι φυλακήν (\textit{didónai / doũnai phulakḗn}) (twice: \iwi{Aristotle, \textit{Rhetoric} 2.20.5} (\textit{bis})) &
  	\textbf{SO:}\newline
  	\textbf{1)} τὰς κρίσεις \textit{tàs kríseis} (\iwi{Aristotle, \textit{Rhetoric}  1.2.5}),\newline
  	\textbf{2)} τὰ δίκαια \textit{tà díkaia} (\iwi{Aristotle, \textit{Rhetoric} 2.23.12}),\newline
  	\textbf{3)} [ὅρκους] [\textit{hórkous}] (omitted Acc.) (\iwi{Aristotle, \textit{Rhetoric} 1.15.2}),\newline
  	\textbf{4)} τὴν πρόθεσιν \textit{tḕn próthesin} (\iwi{Aristotle, \textit{Rhetoric} 2.18.5}),\newline
  	\textbf{5)} αἵρεσιν \textit{haíresin} (\iwi{Aristotle, \textit{Rhetoric} 2.24.9}).\newline~\newline
  	\textbf{CO:}\newline
  	\textbf{1)} τὸ δίκαιον καὶ τό συμφέρον \textit{tò díkaion kaì tò sumphéron} (\iwi{Aristotle, \textit{Rhetoric} 1.1.7})\\
  	\lspbottomrule
\end{tabularx}
\caption{{διδόναι, ἀποδιδόναι, ἀνταποδιδόναι} (\textit{didónai, apodidónai, antapodidónai}) + Acc. }
\label{tab:startexampledata}
\end{table}


\begin{table}
\footnotesize
\begin{tabularx}{\textwidth}{QQ}
        \lsptoprule
        Repeated types (with morphological variations), and list of V+SO and V+CO & Unrepeated types (occurring only once), and list of SO and CO     \\
        \midrule
        \textbf{V+SO:}\newline
        \textbf{1)} λέγειν / ἐπιλέγειν τήν αἰτίαν / τὰς αἰτίας / τὸ αἴτιον (\textit{légein / epilégein tḕn aitían / tàs aitías / tò aítion}) (five times in total: \iwi{Aristotle, \textit{Rhetoric} 1.2.11} (ἐρεῖν \textit{ereĩn}), \iwi{Aristotle, \textit{Rhetoric} 2.9.5} (τὰς εἰρημένας αἰτίας \textit{tàs eirēménas aitías}), \iwi{Aristotle, \textit{Rhetoric} 2.23.24} (twice: λέγειν τὴν αἰτίαν \textit{légein tḕn aitían} and λεχθέντος τοῦ αἰτίου \textit{lekhthéntos toũ aitíou}), \iwi{Aristotle, \textit{Rhetoric} 2.21.14} (ἐπιλέγειν \textit{epilégein})\newline
        \textbf{2)} (τὰ) ἐνθυμήματα λέγειν / ἐνθύμημα εἰπεῖν (\textit{(tà) enthumḗmata légein / enthúmēma eipeĩn}) (four times in total: \iwi{Aristotle, \textit{Rhetoric} 1.2.8, 1.2.14, 1.15.19, 1.2.21}),\newline
        \textbf{3)} τἀληθῆ \textit{talēthē̃} (twice: \iwi{Aristotle, \textit{Rhetoric} 1.15.26} (\textit{bis})) V+\textbf{CO:} \textbf{1)} (τὰ) ἔξω τοῦ πράγματος λέγειν / τεχνολογεῖν \textit{(tà) éxō toũ prágmatos légein / tekhnologeĩn} (thrice in total: \iwi{Aristotle, \textit{Rhetoric}  1.1.5, 1.1.9, 1.1.10})
        &
        \textbf{SO:}\newline
        \textbf{1)} οὐδέν \textit{oudén} (\iwi{Aristotle, \textit{Rhetoric}  1.1.3}),\newline
        \textbf{2)} παραδείγματα \textit{paradeígmata} (\iwi{Aristotle, \textit{Rhetoric} 1.2.8}),\newline
        \textbf{3)} ὑποθήκας \textit{hupothḗkas} (\iwi{Aristotle, \textit{Rhetoric} 1.9.36}),\newline
        \textbf{4)} τὰ ψευδῆ \textit{tà pseudē̃} (\iwi{Aristotle, \textit{Rhetoric} 1.15.26}),\newline
        \textbf{5)} παράδοξον \textit{parádoxon} (\iwi{Aristotle, \textit{Rhetoric} 2.21.4}),\newline
        \textbf{6)} τὰς γνώμας \textit{tàs gnṓmas} (\iwi{Aristotle, \textit{Rhetoric} 2.21.13}),\newline
        \textbf{7)} φανερά \textit{phanerá} (\iwi{Aristotle, \textit{Rhetoric} 2.22.3}),\newline
        \textbf{8)} τὰ δίκαια \textit{tà díkaia} (\iwi{Aristotle, \textit{Rhetoric} 2.23.15}),\newline
        \textbf{9)} τὰ ἄδικα \textit{tà ádika} (\iwi{Aristotle, \textit{Rhetoric} 2.23.15})\newline
        \textbf{10)} λόγον lógon (\iwi{Aristotle, \textit{Rhetoric} 2.20.5} (εἰπεῖν eipeĩn)),\newline
        \textbf{11)} τἀναντία \textit{tanantía} (\iwi{Aristotle, \textit{Rhetoric} 2.23.12}); \newline
        \\
        &
        \textbf{CO:}\newline
        \textbf{1)} [τοὺς ἐπαίνους καὶ τοὺς ψόγους \textit{toùs epaínous kaì toùs psógous}] (ex pass. οἱ ἔπαινοι καὶ οἱ ψόγοι λέγονται \textit{hoi épainoi kaì hoi psógoi légontai}) (\iwi{Aristotle, \textit{Rhetoric} 1.9.41}),\newline
        \textbf{2)} τὰ κοινὰ καὶ καθόλου \textit{tà koinà kaì kathólou} (\iwi{Aristotle, \textit{Rhetoric} 2.22.3}),\newline
        \textbf{3)} [τὰ] ἐξ ὧν ἴσασι καὶ τὰ ἐγγύς \textit{[tà] ex hō̃n ísasi kaì tà engús} (\iwi{Aristotle, \textit{Rhetoric} 2.22.3}),\newline
        \textbf{4)} δόξαν τινά \textit{dóxan tiná} (\iwi{Aristotle, \textit{Rhetoric} 2.26.4})\\
  	\lspbottomrule
\end{tabularx}
\caption{{λέγειν, εἰπεῖν (\textit{légein, eipeĩn})} + Acc.}
\end{table}


\begin{table}
\footnotesize
\begin{tabularx}{\textwidth}{QQ}
        \lsptoprule
        Repeated types (with morphological variations), and list of V+SO and V+CO & Unrepeated types (occurring only once), and list of SO and CO     \\
        \midrule
        \textbf{V+SO:} \newline
        \textbf{1)} τἀυτὸ / τἀυτὰ ποιεῖν (\textit{tautò / tautà poieĩn}) (twice in total: \iwi{Aristotle, \textit{Rhetoric} 2.2.9; 2.2.16}); \newline
        V+\textbf{CO:} \newline
        \textbf{1)} τοὺς λόγους ἠθικοὺς ποιεῖν (\textit{toùs lógous ēthikoùs poieĩn}) (thrice in total with variations in word order: \iwi{Aristotle, \textit{Rhetoric} 2.18.1; 2.18.2; 2.21.16})
        &
         \textbf{SO:}\newline
        \textbf{1)} μεγάλα \textit{megála} (\iwi{Aristotle, \textit{Rhetoric} 1.7.32}), \newline
        \textbf{2)} ἡδύ \textit{hēdú} (\iwi{Aristotle, \textit{Rhetoric} 1.11.4}), \newline
        \textbf{3)} ὑπερβολήν \textit{huperbolḗn} (\iwi{Aristotle, \textit{Rhetoric} 1.11.20}), \newline
        \textbf{4)} [ἀγαθά] [\textit{agathá}] (\iwi{Aristotle, \textit{Rhetoric} 1.13.18}: ἀγαθῶν ὧν ἐποίησεν > [ποιῆσαι ἀγαθά] \textit{agathō̃n hō̃n epoíēsen} > [\textit{poiē̃sai agathá}]), \newline
        \textbf{5)} τἀναντία \textit{tanantía} (\iwi{Aristotle, \textit{Rhetoric} 2.2.17}), \newline
        \textbf{6)} τὸν ἔλεον \textit{tòn éleon} (\iwi{Aristotle, \textit{Rhetoric} 2.8.16}), \newline
        \textbf{7)} τὴν συκοφαντίαν \textit{tḕn sukophantían} (\iwi{Aristotle, \textit{Rhetoric} 2.24.10}), \newline
        \textbf{8)} τὴν ὀργήν \textit{tḕn orgḗn} (\iwi{Aristotle, \textit{Rhetoric} 2.1.9}), \newline
        \textbf{9)} ἡδονήν \textit{hēdonḗn} (\iwi{Aristotle, \textit{Rhetoric} 2.2.2});
        \\
        \lspbottomrule
    \end{tabularx}
\caption{{ποιεῖν, ποιῆσαι, ἐμποιεῖν (\textit{poieĩn, poiē̃sai,  empoieĩn})} + Acc.}
\end{table}
\begin{table}
\footnotesize
\begin{tabularx}{\textwidth}{QQ}
        \lsptoprule
        Repeated types (with morphological variations), and list of V+SO and V+CO & Unrepeated types (occurring only once), and list of SO and CO     \\
        \midrule
        None
        &
        \textbf{CO:} \newline
        \textbf{1)} τ\textbf{ὸν κανόνα} στρεβλόν \textit{\textbf{tòn kanóna}}\footnote{\tiny The direct object (DO) is highlighted in a bolder font.}
        \textit{streblón} (\iwi{Aristotle, \textit{Rhetoric}  1.1.5}), \newline
        \textbf{2)} ὡς ἐλαχίστων κύριον \textbf{τὸν κριτήν} \textit{hōs elakhístōn kúrion \textbf{tòn kritḗn}} (\iwi{Aristotle, \textit{Rhetoric} 1.1.8}), \newline
        \textbf{3)} \textbf{τὸν κριτὴν} ποιόν τινα \textit{\textbf{tòn kritḕn} poión tina} (\iwi{Aristotle, \textit{Rhetoric} 1.1.9}), \newline
        \textbf{4)} ἀξιόπιστον \textbf{τὸν λέγοντα} \textit{axiópiston \textbf{tòn légonta}} (\iwi{Aristotle, \textit{Rhetoric} 1.2.4}), \newline
        \textbf{5)} \textbf{τὸν λέγοντα} ἔμφρονα \textit{\textbf{tòn légonta} émphrona} (\iwi{Aristotle, \textit{Rhetoric} 1.2.21}), \newline
        \textbf{6)} μὴ βραδυτέρας \textbf{τὰς κινήσεις} \textit{mḕ bradutéras \textbf{tàs kinḗseis}} (\iwi{Aristotle, \textit{Rhetoric} 1.5.13}), \newline
        \textbf{7)} πιστὰς ἢ ἀπίστους [\textbf{τὰς συνθήκας}] \textit{pistàs ḕ apístous [\textbf{tàs sunthḗkas}]} (\iwi{Aristotle, \textit{Rhetoric} 1.15.20}), \newline
        \textbf{8)} \textbf{τὸν νόμον} κύριον \textit{\textbf{tòn nómon} kúrion} (\iwi{Aristotle, \textit{Rhetoric} 1.15.21}), \newline
        \textbf{9)} βουλευτικοὺς [sc. \textbf{τοὺς ἀνθρώπους}] \textit{bouleutikoùs [sc. \textbf{toùs anthrṓpous}]} (\iwi{Aristotle, \textit{Rhetoric} 2.5.14}), \newline
        \textbf{10)} πρὸ ὄμμάτων [\textbf{τὰ κακά}] \textit{prò ommátōn [\textbf{tà kaká}]} (\iwi{Aristotle, \textit{Rhetoric} 2.8.13}), \newline
        \textbf{11)} μὴ ἐλεεινὰ \textbf{ἅπαντα} \textit{mḕ eleeinà \textbf{hápanta}} (\iwi{Aristotle, \textit{Rhetoric} 2.9.5}),\newline
        \textbf{12)} δίκαια \textbf{πολλά} \textit{díkaia \textbf{pollá}}
        \textbf{13)} [\textbf{τοὺς δυναμένους}] σεμνοτέρους \textit{[\textbf{toùs dunaménous}] semnotérous} (Ross) : ἐμφανεστέρους \textit{emphanestérous} (Kassel) (\iwi{Aristotle, \textit{Rhetoric} 2.17.4}), (opp. ἀδικεῖν ἔνια \textit{adikeĩn énia}) (\iwi{Aristotle, \textit{Rhetoric} 1.12.31}),
        \textbf{14)} \textbf{τὸν ἥττω λόγον} κρείττω \textit{\textbf{tòn hḗttō lógon} kreíttō} (\iwi{Aristotle, \textit{Rhetoric} 2.24.11}),
        \textbf{15)} [\textbf{λόγους}] ὥσπερ καὶ παραβολάς \textit{[\textbf{lógous}] hṓsper kaì parabolás} (\iwi{Aristotle, \textit{Rhetoric} 2.20.7})\\
        \lspbottomrule
\end{tabularx}
\caption{{ποιεῖν, ποιῆσαι, ἐμποιεῖν (\textit{poieĩn, poiē̃sai,  empoieĩn})} + Acc. (continued from previous table)}
\end{table}


\begin{table}
\footnotesize
\begin{tabularx}{\textwidth}{QQ}
        \lsptoprule
        Repeated types (with morphological variations), and list of V+SO and V+CO & Unrepeated types (occurring only once), and list of SO and CO     \\
        \midrule
        None&\textbf{CO:} \newline
        \textbf{1)} καὶ \textbf{αὑτὸν} ποιόν τινα καὶ τὸν κριτήν \textit{kaì \textbf{hautòn} poión tina kaì tòn kritḕn} [sc. ποιόν τινα / \textit{poión tina}] (\iwi{Aristotle, \textit{Rhetoric} 2.1.2}),\newline
        \textbf{2)} \textbf{ἑαυτὸν} τοιοῦτον \textit{\textbf{heautòn} toioũton} (\iwi{Aristotle, \textit{Rhetoric} 2.1.7}),\newline
        \textbf{3)} [\textbf{τοὺς ἀκροατὰς} \textit{\textbf{toùs akroatàs}}] τοιούτους \textit{toioútous} (\iwi{Aristotle, \textit{Rhetoric} 2.2.27})\\
        \lspbottomrule
\end{tabularx}
\caption{{κατασκευάζειν (\textit{kataskeuázein})} + Acc.}
\end{table}


\begin{table}
\footnotesize
\begin{tabularx}{\textwidth}{QQ}
        \lsptoprule
        Repeated types (with morphological variations), and list of V+SO and V+CO & Unrepeated types (occurring only once), and list of SO and CO     \\
        \midrule
        None&\textbf{CO:} \newline
        \textbf{1)} \textbf{αὑτοὺς} τοιούτους \textit{\textbf{hautoùs} toioútous} (\iwi{Aristotle, \textit{Rhetoric} 2.3.17}),\newline
        \textbf{2)} \textbf{τοὺς κριτὰς} τοιούτους \textit{\textbf{toùs kritàs} toioútous} (\iwi{Aristotle, \textit{Rhetoric} 2.9.16})\\
        \lspbottomrule
\end{tabularx}
\caption{{παρασκευάζειν (\textit{paraskeuázein})} + Acc.}
\end{table}



\begin{table}
\footnotesize
\begin{tabularx}{\textwidth}{QQ}
        \lsptoprule
        Repeated types (with morphological variations), and list of V+SO and V+CO & Unrepeated types (occurring only once), and list of SO and CO     \\
        \midrule
        \textbf{V+SO:}\newline
        \textbf{1)} ποιεῖσθαι τὸν λόγον poieĩsthai tòn lógon (twice in total with variation in word order: \iwi{Aristotle, \textit{Rhetoric} 1.5.18, 2.18.1})
        &
        \textbf{SO:} \newline
        \textbf{1)} τὰς πίστεις \textit{tàs písteis} (\iwi{Aristotle, \textit{Rhetoric} 1.2.8}), \newline
        \textbf{2)} τὴν κρίσιν \textit{tḕn krísin} (\iwi{Aristotle, \textit{Rhetoric} 2.1.4}), \newline
        \textbf{3)} τοὺς συλλογισμούς \textit{toùs sullogismoús} (\iwi{Aristotle, \textit{Rhetoric} 1.10.1})\newline
        ~\newline
        \textbf{CO:} \newline
        \textbf{1)} τὰς πίστεις καὶ τοὺς λόγους t\textit{às písteis kaì toùs lógous} (\iwi{Aristotle, \textit{Rhetoric} 1.1.12}), \newline
        \textbf{2)} φίλον \textbf{γέροντα} \textit{phílon \textbf{géronta}} (\iwi{Aristotle, \textit{Rhetoric} 1.15.14}), \newline
        \textbf{3)} πολίτας \textbf{τοὺς μισθοφόρους} \textit{polítas \textbf{toùs misthophórous}} (\iwi{Aristotle, \textit{Rhetoric} 2.23.17}), \newline
        \textbf{4)} φυγάδας \textbf{τοὺς} [...] \textbf{διαπεπραγμένους} \textit{phugádas \textbf{toùs} [...] \textbf{diapepragménous}} (\iwi{Aristotle, \textit{Rhetoric} 2.23.17})\\
        \lspbottomrule
\end{tabularx}
\caption{{ποιεῖσθαι (\textit{poieĩsthai})} + Acc.}
\end{table}


\begin{table}
\footnotesize
\begin{tabularx}{\textwidth}{QQ}
        \lsptoprule
        Repeated types (with morphological variations), and list of V+SO and V+CO & Unrepeated types (occurring only once), and list of SO and CO     \\
        \midrule
        \textbf{V+SO:} \newline
        \textbf{1)} πράττειν τὰ καλά \textit{práttein tà kalá} (twice: \iwi{Aristotle, \textit{Rhetoric} 1.7.38, 2.12.12})
        &
        \textbf{SO:} \newline
        \textbf{1)} τὰ συμφέροντα \textit{tà sumphéronta} (\iwi{Aristotle, \textit{Rhetoric} 2.12.12}).  CO:
        \textbf{1)} τὰ συμφέροντα ἢ βλαβερά \textit{tà sumphéronta ḕ blaberá} (\iwi{Aristotle, \textit{Rhetoric} 1.3.6}),
        \textbf{2)} πολλὰ δίκαια \textit{pollà díkaia} (\iwi{Aristotle, \textit{Rhetoric} 1.12.31}).\\
        \lspbottomrule
\end{tabularx}
\caption{{πράττειν (\textit{práttein})} + Acc.}
\end{table}
%
%
\begin{table}
\footnotesize
\begin{tabularx}{\textwidth}{QQ}
        \lsptoprule
        Repeated types (with morphological variations), and list of V+SO and V+CO & Unrepeated types (occurring only once), and list of SO and CO     \\
        \midrule
        \textbf{V+SO:} \newline
        \textbf{1)} [νόμον θεῖναι (τεθηκέναι)] [\textit{nómon theĩnai (tethēkénai)}] (thrice: \iwi{Aristotle, \textit{Rhetoric} 1.1.7, 1.14.4, 1.15.11}, always in passive construction; hence the periphrasis is only reconstructed)&None\\
        \lspbottomrule
\end{tabularx}
\caption{{τιθέναι, θεῖναι (\textit{tithénai, theĩnai})} + Acc.}
\end{table}

\begin{table}
\footnotesize
\begin{tabularx}{\textwidth}{QQ}
        \lsptoprule
        Repeated types (with morphological variations), and list of V+SO and V+CO & Unrepeated types (occurring only once), and list of SO and CO     \\
        \midrule
        \textbf{V+SO:} \newline
        \textbf{1)} πίστεις φέρειν \textit{písteis phérein} (twice: \iwi{Aristotle, \textit{Rhetoric} 1.7.40, 2.18.2}),\newline
        \textbf{2)} φέρειν τὰ ἐνθυμήματα (ἐνθυμήματα φέρειν) \textit{phérein tà enthumḗmata (enthumḗmata phérein)} (twice in total: \iwi{Aristotle, \textit{Rhetoric} 2.22.16, 2.26.3}),\newline
        \textbf{3)} ἔνστασιν (ἐνστάσεις) φέρειν (ἐνεγκεῖν) / \textit{énstasin (enstáseis) phérein (enenkeĩn)} (five times in total: \iwi{Aristotle, \textit{Rhetoric} 2.25.1, 2.25.3, 2.25.5, 2.25.8, 2.26.3})
        &
        \textbf{SO:} \newline
        \textbf{1)} τεκμήριον \textit{tekmḗrion} (\iwi{Aristotle, \textit{Rhetoric} 1.2.17})\\
        \lspbottomrule
\end{tabularx}
\caption{{φέρειν, ἐνεγκεῖν (\textit{phérein, enenkeĩn})} + Acc.}
\label{tab:endexampledata}
\end{table}




The data in the tables are purposefully grouped by the repetition of words and the complexity of their complements: in addition to the low semantic weight of the verb, SVCs/LVCs are usually identified by the single non-composite complement (SO) and the repetitive use of the whole phrase (cf. column ‘Repeated types' in each table). In this way, phrases such as: χάριν διδόναι, \textit{khárin didónai}, δοῦναι δίκην, \textit{doũnai díkēn}, διδόναι φυλακήν, \textit{ didónai phulakḗn}, ποιεῖσθαι τὸν λόγον, \textit{poieĩsthai tòn lógon}, λέγειν τὴν αἰτίαν, légein tḕn \textit{aitían}, ἐνθυμήματα λέγειν, \textit{enthumḗmata légein}, νόμον θεῖναι, \textit{nómon theĩnai}
%Wyn, I have added the Greek here but can you please make it into the order Greek – transcription in italics, the author again didn't put the Greek and I really don't have time to sort this in every instance, so I am typing in the Greek to make this easier for you but obviously shouldn't look like long list in Greek and then long list in transcription, that's useless for the reader
seemingly fall within this category. 




Of course, some may be disqualified due to high variability\footnote{This creates an irregularity factor, and the phrase begins to resemble a free word combination, arbitrarily created by the speaker/writer for the occasion rather than taken from common usage. If one sees a full realisation of the lexical meaning of the verb rather than a partial one, disqualification is inevitable.} (such as the phrase λέγειν τὴν αἰτίαν \textit{légein tḕn aitían}, which attests the variants τὴν αἰτίαν ἐρεῖν, \textit{tḕn aitían ereĩn}, διὰ τὰς είρημένας αἰτίας, \textit{dià tàs eirēménas aitías}, λεχθέντος τοῦ αἰτίου, \textit{lekhthéntos toũ aitíou}), while other phrases, although occurring only once, can be considered SVCs because they are quite frequent in other texts or can be created by analogy (e.g. various phrases with the verbs ποιεῖν \textit{poieĩn}, ποιεῖσθαι \textit{poieĩsthai}, and λέγειν \textit{légein}) and serve as analytic counterparts for the corresponding simplex or compound words (cf. τὰ ψευδῆ λέγειν \textit{tà pseudē̃ légein} ‘to speak/tell lies' = ψευδολογεῖν \textit{pseudologeĩn} ‘to speak falsely' (cf. LSJ s.v.), τὰς γνώμας λέγειν t\textit{às gnṓmas légein} ‘to say maxims'  ≈ γνωμολογεῖν \textit{gnōmologeĩn} ‘to speak in maxims', ποιεῖν ἡδύ \textit{poieĩn hēdú} ‘to make pleasant/sweet' = ἡδύνειν \textit{hēdúnein} ‘to sweeten', τὴν ὀργὴν ἐμποιεῖν \textit{tḕn orgḕn empoieĩn} ‘to produce/cause anger' = ὀργίζειν \textit{orgízein} ‘to make angry', ‘to irritate', etc.). 


Some phrases with the same verbs, although used repeatedly, e.g. ταὐτὸ ποιεῖν \textit{tautò poieĩn} ‘to do the same thing' or πράττειν τὰ καλά \textit{práttein tà kalá} ‘to do/practice good [deeds]', are on the edge of SVCs because they have a non-noun complement. The bivalent/trivalent verbs ποιεῖν \textit{poieĩn} ‘to make/cause', κατασκευάζειν \textit{kataskeuázein} ‘to furnish', ‘to make/render', and παρασκευάζειν \textit{paraskeuázein} ‘to furnish', ‘to make/render', which govern the accusative duplex and in which a predicate adjective together with the verb can replace the causative verb, are also reminiscent of the SVC-like periphrases, esp. e.g. ποιεῖν στρεβλόν \textit{poieĩn streblón} ‘to make crooked/distorted' = στρεβλοῦν \textit{strebloũn} ‘to crook', ‘to distort', ποιεῖν σεμνότερον \textit{poieĩn semnóteron} ‘to make more solemn' ≈ σεμνοῦν \textit{semnoũn} ‘to make solemn', ‘to magnify', etc.

\clearpage
Among the introversive verbs, the following components of periphrases were found most frequently in Aristotle’s treatise: \textbf{ἔχειν} \textit{ékhein} ‘to have’, ‘to have the potential’, \textbf{λαμβάνειν} \textit{lambánein} ‘to take’, ‘to accept’, ‘to admit’ etc., \textbf{πάσχειν} \textit{páskhein} ‘to be treated’, ‘to suffer’, ‘to experience’, and \textbf{πράττειν} \textit{práttein} ‘to experience certain fortunes’, ‘to fare’. 

These verbs frequently direct the action towards the object (accusativus rei) and/or maintain the recipient of the profit or harm, expressed in the nominative case, although sometimes they can also be related to the subject-giver (ἔκ τινος \textit{ék tinos}, παρά τινος \textit{pará tinos}, ὑπό τινος \textit{hupó tinos}). There are 64 different constructions (types) with these verbs + DOs, which occur 83 times in the text under consideration. Their brief characteristics are shown in \tabref{tab:overview2}.
\tabref{tab:overview2} serves as a numeric overview, relevant examples are provided in \tabref{tab:startexampledata2} to \tabref{tab:endexampledata2}.

\begin{table}
	\caption{Periphrases with introversive verbs\label{tab:overview2}}
{\scriptsize
\begin{tabularx}{\linewidth}{Qrrrrr}
    \lsptoprule
& tokens/types
& repeated$^*$
& unrepeated$^*$
& types SO$^\dagger$
& types with CO$^\dagger$
\\
\midrule
\textbf{ἔχειν (\textit{ékhein})} + Acc. &  49 / 35 &  {9 (9+0)} &  {26 (18+8)} &  {27 (9+18)} &  {8 (0+8)}\\
\tablevspace
\textbf{λαμβάνειν, λαβεῖν (\textit{lambánein, labeĩn})} + Acc. &  27 / 23 & {3 (3+0)} &  {20 (6+14)} &  {9 (3+6)} &  {14 (0+14)} \\
\tablevspace
\textbf{πάσχειν, παθεῖν, πεπονθέναι (\textit{páskhein, patheĩn, peponthénai})} + Acc. &  {6 / 5} &  {1 (1+0)} &  {4 (4+0)} &  {5} &  {0} \\
\tablevspace
\textbf{πράττειν \textit{práttein*}} + Acc. &  {1 / 1} & {0} &  {1} &  {1} &  {0} \\
	\midrule
\textbf{total} &  {83 / 64} &  {13} &  {51} &  {42} &  {22} \\
 \lspbottomrule
\end{tabularx}
}
\legendbox{$^*$ In the brackets, the first number indicates the amount of verb-controlled single objects, and the second number refers to complex objects and objects with attributes.\\ $^\dagger$ These brackets show the data from the second and third columns.}
\end{table}


Tables \ref{tab:overview1} and \ref{tab:overview2} show an equal number of recurrent V+CO phrases (see column 3), but the table on introversive verbs does not contain any recurrent V+CO phrases, and on the whole only 2 out of 4 (50\%) of the introversive verbs have a one-time phrase of the latter type, while among the extraversive verbs, as many as 7 out of 9 (\sim78\%) do.

Some of the verbs mentioned of both kinds, but especially the introversive ones (those listed in \tabref{tab:overview2}), form adverbial, prepositional, and parenthetical constructions. The text under study has a total of 163 of such constructions (on this see the dataset, see n. 1), with the number of non-repeated constructions being 73; the leading type here is ἔχειν \textit{ékhein} + adverb, called explicitly a periphrasis by Smyth\footnote{\citet[§1438]{Smyth1920}: “An adverb with ἔχειν [\textit{ékhein}] or διακεῖσθαι [\textit{diakeĩsthai}] is often used as a periphrasis for an adjective with εἶναι [\textit{eĩnai}] or for a verb.”} (73 occurrences of 22 different phrases).


\begin{table}
\caption{\textbf{ἔχειν (\textit{ékhein})} + Acc. }\label{tab:startexampledata2}
\footnotesize
\begin{tabularx}{\textwidth}{QQ}
        \lsptoprule
        Repeated types% (with morphological variations), and list of V+SO and V+CO
        & Unrepeated types (occurring only once), and list of SO and CO     \\
        \midrule
\textbf{SO:}
\newline 1) ἔχειν διαφοράς \textit{ékhein diaphorás} / διαφορὰν ἔχειν \textit{diaphoràn ékhein} (twice in total: \iwi{Aristotle, \textit{Rhetoric} 1.1.11, 2.25.13}),
\newline 2) ἔχειν ἀγαθόν \textit{ékhein agathón} (twice: \iwi{Aristotle, \textit{Rhetoric} 1.2.10, 2.20.7}),
\newline 3) ἔχειν (τὰς) προτάσεις  \textit{ékhein (tàs) protáseis} (thrice: \iwi{Aristotle, \textit{Rhetoric} 1.3.7, 1.3.8, 1.4.13}),
\newline 4) ἔχειν μέγεθος \textit{ékhein mégethos} / μέγεθος ἔχειν \textit{mégethos ékhein} (twice: \iwi{Aristotle, \textit{Rhetoric} 1.7.32, 2.8.8}),
\newline 5) χάριν ἔχειν \textit{khárin ékhein} (thrice: \iwi{Aristotle, \textit{Rhetoric} 1.13.12, 2.7.1, 2.7.2}),
\newline 6) συγγνώμην ἔχειν \textit{sungnṓmēn ékhein} (twice: \iwi{Aristotle, \textit{Rhetoric} 1.13.16, 2.25.7}),
\newline 7) δύναμιν ἔχειν \textit{dúnamin ékhein} / ἔχειν δύναμιν \textit{ékhein dúnamin} (four times in total: \iwi{Aristotle, \textit{Rhetoric} 2.5.4, 2.5.5, 2.5.8, 2.5.17}),
\newline 8) λόγον ἔχειν (τινὸς) \textit{lógon ékhein (tinòs)} (twice: \iwi{Aristotle, \textit{Rhetoric} 2.6.14, 2.6.15}),
\newline 9) ἔχειν τὰ ἤθη \textit{ékhein tà ḗthē} / ἦθος ἔχειν \textit{ē̃thos ékhein} (thrice: \iwi{Aristotle, \textit{Rhetoric} 12.17.1, 2.17.5, 2.21.16}). \newline

&

\textbf{SO:}
\newline 1) ἐπιστήμην \textit{epistḗmēn} (\iwi{Aristotle, \textit{Rhetoric} 1.1.12}),
\newline 2) τὸ πιστόν \textit{tò pistón} (\iwi{Aristotle, \textit{Rhetoric} 1.15.26}),
\newline 3) τέχνας \textit{tékhnas} (\iwi{Aristotle, \textit{Rhetoric} 1.2.12}),
\newline 4) τὰς ἀρχάς \textit{tàs arkhás (tinos)} (\iwi{Aristotle, \textit{Rhetoric} 1.2.21}),
\newline 5) μοχθηρίαν \textit{mokhthērían} (\iwi{Aristotle, \textit{Rhetoric} 1.10.4}),
\newline 6) κακόν \textit{kakón} (\iwi{Aristotle, \textit{Rhetoric} 1.11.8}),
\newline 7) ἐπιθυμίαν \textit{epithumían} (\iwi{Aristotle, \textit{Rhetoric} 1.11.14}),
\newline 8) ἀπολογίαν \textit{apologían} (\iwi{Aristotle, \textit{Rhetoric} 1.12.7}),
\newline 9) πρόφασιν \textit{próphasin} (\iwi{Aristotle, \textit{Rhetoric} 1.12.23}),10) κότον \textit{kóton} (\iwi{Aristotle, \textit{Rhetoric} 2.2.7}),
\newline 11) τιμήν \textit{timḗn} (\iwi{Aristotle, \textit{Rhetoric} 2.2.6}),
\newline 12) τὴν ὑπουργίαν \textit{tḕn hupourgían} (\iwi{Aristotle, \textit{Rhetoric} 2.7.4}),
\newline 13) βοήθειαν \textit{boḗtheian} (\iwi{Aristotle, \textit{Rhetoric} 2.21.15}),
\newline 14) δόξας \textit{dóxas} (\iwi{Aristotle, \textit{Rhetoric} 2.21.15}),
\newline 15) ὠφέλειαν \textit{ōphéleian} (\iwi{Aristotle, \textit{Rhetoric} 2.21.16}),
\newline 16) δίκην \textit{díkēn} (\iwi{Aristotle, \textit{Rhetoric} 2.3.5}),
\newline 17) τὴν αἰτίαν \textit{tḕn aitían} (\iwi{Aristotle, \textit{Rhetoric} 2.24.4}),
\newline 18) ἔνστασιν \textit{énstasin} (\iwi{Aristotle, \textit{Rhetoric} 2.25.10}). \\
\lspbottomrule
\end{tabularx}
\end{table}
\begin{table}
\caption{\textbf{ἔχειν (\textit{ékhein})} + Acc. (continued from previous table)}
\footnotesize
\begin{tabularx}{\textwidth}{QQ}
        \lsptoprule
        Repeated types% (with morphological variations), and list of V+SO and V+CO
        & Unrepeated types (occurring only once), and list of SO and CO     \\
        \midrule
none &
\textbf{CO:}
\newline 1) οὐδέν, ὅ τι λέγωσιν (ἂν) \textit{oudén, hó ti légōsin (án)} (\iwi{Aristotle, \textit{Rhetoric} 1.1.4}),
\newline 2) ὅ τι ἀπολέσει \textit{hó ti apolései} (\iwi{Aristotle, \textit{Rhetoric} 1.12.8}),
\newline 3) κυριωτάτην πίστιν \textit{kuriōtátēn pístin} (\iwi{Aristotle, \textit{Rhetoric} 1.2.4}),
\newline 4) κοινὸν εἶδος \textit{koinòn eĩdos} (\iwi{Aristotle, \textit{Rhetoric} 1.9.35}),
\newline 5) τὸ ἡδὺ καὶ τὸ καλόν \textit{tò hēdù kaì tò kalón} (\iwi{Aristotle, \textit{Rhetoric} 1.12.27}),
\newline 6) δύναμιν μεγάλην \textit{dúnamin megálēn} (\iwi{Aristotle, \textit{Rhetoric} 2.5.2}),
\newline 7) μίαν χρῆσιν \textit{mían khrē̃sin} (\iwi{Aristotle, \textit{Rhetoric} 2.21.16}),
\newline 8) πλείω τῶν ὑπαρχόντων \textit{pleíō tō̃n huparkhóntōn} (\iwi{Aristotle, \textit{Rhetoric} 2.22.11})\\
        \lspbottomrule
\end{tabularx}
\end{table}

\begin{table}
\caption{\textbf{λαμβάνειν, λαβεῖν (\textit{lambánein, labeĩn})} + Acc. }
\footnotesize
\begin{tabularx}{\textwidth}{QQ}
        \lsptoprule
        Repeated types% (with morphological variations), and list of V+SO and V+CO
        & Unrepeated types (occurring only once), and list of SO and CO     \\
        \midrule
\textbf{SO:}
\newline 1) λαμβάνειν/λαβεῖν πίστεις lambánein/labeĩn písteis (thrice in total: \iwi{Aristotle, \textit{Rhetoric} 1.2.7} (aor.), \iwi{Aristotle, \textit{Rhetoric} 1.6.30} (adj.verb.), \iwi{Aristotle, \textit{Rhetoric} 1.8.7}),
\newline 2) ) λαβεῖν / λαμβάνειν προτάσεις labeĩn/ lambánein protáseis (twice: \iwi{Aristotle, \textit{Rhetoric} 1.3.9} (aor.), \iwi{Aristotle, \textit{Rhetoric} 1.9.2} (adj.verb.)),
\newline 3) λαμβάνειν /εἰληφέναι τιμωρίαν lambánein/ eilēphénai timōrían (twice: \iwi{Aristotle, \textit{Rhetoric} 2.3.13} (aor. pass.: ληφθεῖσα τιμωρία lēphtheĩsa timōría), \iwi{Aristotle, \textit{Rhetoric} 2.3.14} (pf.)).\footnote{As can be seen, there is some modification rather than a precise replication of the construction.} \newline
&
\textbf{SO:}
\newline 1) δίκην \textit{díkēn} (\iwi{Aristotle, \textit{Rhetoric} 1.14.2}),
\newline 2) [ὅρκους \textit{hórkous}] (omitted Acc.) (\iwi{Aristotle, \textit{Rhetoric} 1.15.27}),
\newline 3) τὰς αὐξήσεις \textit{tàs auxḗseis} (\iwi{Aristotle, \textit{Rhetoric} 2.19.26}),
\newline 4) συμφοράς \textit{sumphorás} (\iwi{Aristotle, \textit{Rhetoric} 2.23.20}),
\newline 5) [δόξας dóxas] (restored Acc. from pass. \textit{eilēmménai dóxai}) (\iwi{Aristotle, \textit{Rhetoric} 2.18.2}),
\newline 6) [τοὺς τόπους \textit{toùs tópous}] (from pass. \textit{eilēmménoi ... hoi tópoi}) (\iwi{Aristotle, \textit{Rhetoric} 2.22.16}).\\
\lspbottomrule
\end{tabularx}
\end{table}
\begin{table}
\caption{\textbf{λαμβάνειν, λαβεῖν (\textit{lambánein, labeĩn})} + Acc. (continued from previous table)}
\footnotesize
\begin{tabularx}{\textwidth}{QQ}
        \lsptoprule
        Repeated types% (with morphological variations), and list of V+SO and V+CO
        & Unrepeated types (occurring only once), and list of SO and CO     \\
        \midrule
\textbf{CO:}
\newline 1) τὰ στοιχεῖα καὶ τὰς προτάσεις \textit{tà stoikheĩa kaì tàs protáseis} (\iwi{Aristotle, \textit{Rhetoric} 1.2.22}),
\newline 2) τὰ στοιχεῖα περὶ ἀγαθοῦ καὶ συμφέροντος ἁπλῶς \textit{tà stoikheĩa perì agathoũ kaì sumphérontos haplō̃s} (\iwi{Aristotle, \textit{Rhetoric} 1.6.1}),
\newline 3) νοῦν καὶ φρόνησιν noũn \textit{kaì phrónēsin} (\iwi{Aristotle, \textit{Rhetoric} 1.7.3}),
\newline 4) \textit{toúnoma toũto} (\iwi{Aristotle, \textit{Rhetoric} 1.8.4}),
\newline 5) τὰ ὑπάρχοντα ἢ δοκοῦντα ὑπάρχειν \textit{tà hupárkhonta ḕ dokoũnta hupárkhein} (\iwi{Aristotle, \textit{Rhetoric} 2.22.8}),
\newline 6) τὸ τί ἐστι \textit{tò tí esti} (2.23.20),
\newline 7) \textit{tò kathólou} (\iwi{Aristotle, \textit{Rhetoric} 2.25.8}),
\newline 8) ψεῦδός τι \textit{pseũdós ti} (\iwi{Aristotle, \textit{Rhetoric} 2.26.4}),
\newline 9) τὰ σύνεγγυς τοῖς ὑπάρχουσιν ὡς ταὐτὰ ὄντα \textit{tà súnengus toĩs hupárkhousin hōs tautà ónta} (\iwi{Aristotle, \textit{Rhetoric} 1.9.28}),
\newline 10) τὰ ἀπὸ τύχης \textit{tà apò túkhēs} (\iwi{Aristotle, \textit{Rhetoric} 1.9.32}),
\newline 11) τὰ συμφέροντα καὶ τὰ ἡδέα \textit{tà sumphéronta kaì tà hēdéa} (\iwi{Aristotle, \textit{Rhetoric} 1.10.19}),
\newline 12) πόσα καὶ ποῖα \textit{pósa kaì poĩa} (\iwi{Aristotle, \textit{Rhetoric} 1.10.19}),
\newline 13) τὸ μετὰ τοῦτο ὡς διὰ τοῦτο \textit{tò metà toũto hōs dià toũto} (\iwi{Aristotle, \textit{Rhetoric} 2.24.8}),
\newline 14) τὴν Δημοσθένους πολιτείαν ... κακῶν αἰτίαν t\textit{ḕn Dēmosthénous politeían ... kakō̃n aitían} (\iwi{Aristotle, \textit{Rhetoric} 2.24.8}) \\
        \lspbottomrule
\end{tabularx}
\end{table}


\clearpage

\begin{table}
\caption{\textbf{πάσχειν, παθεῖν, πεπονθέναι (\textit{páskhein, patheĩn, peponthénai})} + Acc.}
\footnotesize
\begin{tabularx}{\textwidth}{QQ}
        \lsptoprule
        Repeated types% (with morphological variations), and list of V+SO and V+CO
        & Unrepeated types (occurring only once), and list of SO and CO     \\
        \midrule
\textbf{SO:}
\newline 1) πάσχειν κακά / κακόν \textit{páskhein kaká / kakón} (twice in total: \iwi{Aristotle, \textit{Rhetoric} 1.13.18, 2.3.14})\newline
&
\textbf{SO:}
\newline 1) ἀγαθά \textit{agathá} (\iwi{Aristotle, \textit{Rhetoric} 1.13.18}),
\newline 2) τὸ ἔσχατον \textit{tò éskhaton} (\iwi{Aristotle, \textit{Rhetoric} 2.3.16}),
\newline 3) \textit{anáxia} (\iwi{Aristotle, \textit{Rhetoric} 2.12.15}),
\newline 4) τὸ αὐτό \textit{tò autó} (\iwi{Aristotle, \textit{Rhetoric} 2.20.5})\\
        \lspbottomrule
\end{tabularx}
\end{table}

\begin{table}
\caption{\textbf{πράττειν \textit{práttein*}}}\label{tab:endexampledata2}
\footnotesize
\begin{tabularx}{\textwidth}{lQ}
        \lsptoprule
        Repeated types% (with morphological variations), and list of V+SO and V+CO
        & Unrepeated types (occurring only once), and list of SO and CO     \\
        \midrule
None &
\textbf{SO:}
\newline μεγάλα πράττειν \textit{megála práttein} (\iwi{Aristotle, \textit{Rhetoric} 2.10.2}) (“experience great things (great fortunes)”)\\
        \lspbottomrule
\end{tabularx}
\end{table}




%
% \begin{tabularx}{\linewidth}{Qrrrrr}
%     \lsptoprule
% & tokens/types
% & repeated
% & unrepeated
% & types SO
% & types with CO
% 	\midrule
% 	\textbf{ἔχειν (\textit{ékhein})} + Acc. & 49 / 35 & \hfil{9 (9+0)}\newline SO: 1) ἔχειν διαφοράς \textit{ékhein diaphorás} / διαφορὰν ἔχειν \textit{diaphoràn ékhein} (twice in total: \iwi{Aristotle, \textit{Rhetoric} 1.1.11, 2.25.13}), 2) ἔχειν ἀγαθόν \textit{ékhein agathón} (twice: \iwi{Aristotle, \textit{Rhetoric} 1.2.10, 2.20.7}), 3) ἔχειν (τὰς) προτάσεις  \textit{ékhein (tàs) protáseis} (thrice: \iwi{Aristotle, \textit{Rhetoric} 1.3.7, 1.3.8, 1.4.13}), 4) ἔχειν μέγεθος \textit{ékhein mégethos} / μέγεθος ἔχειν \textit{mégethos ékhein} (twice: \iwi{Aristotle, \textit{Rhetoric} 1.7.32, 2.8.8}), 5) χάριν ἔχειν \textit{khárin ékhein} (thrice: \iwi{Aristotle, \textit{Rhetoric} 1.13.12, 2.7.1, 2.7.2}), 6) συγγνώμην ἔχειν \textit{sungnṓmēn ékhein} (twice: \iwi{Aristotle, \textit{Rhetoric} 1.13.16, 2.25.7}), 7) δύναμιν ἔχειν \textit{dúnamin ékhein} / ἔχειν δύναμιν \textit{ékhein dúnamin} (four times in total: \iwi{Aristotle, \textit{Rhetoric} 2.5.4, 2.5.5, 2.5.8, 2.5.17}), 8) λόγον ἔχειν (τινὸς) \textit{lógon ékhein (tinòs)} (twice: \iwi{Aristotle, \textit{Rhetoric} 2.6.14, 2.6.15}), 9) ἔχειν τὰ ἤθη \textit{ékhein tà ḗthē} / ἦθος ἔχειν \textit{ē̃thos ékhein} (thrice: \iwi{Aristotle, \textit{Rhetoric} 12.17.1, 2.17.5, 2.21.16}). & \hfil{26 (18+8)}\newline SO: 1) ἐπιστήμην \textit{epistḗmēn} (\iwi{Aristotle, \textit{Rhetoric} 1.1.12}), 2) τὸ πιστόν \textit{tò pistón} (\iwi{Aristotle, \textit{Rhetoric} 1.15.26}), 3) τέχνας \textit{tékhnas} (\iwi{Aristotle, \textit{Rhetoric} 1.2.12}), 4) τὰς ἀρχάς \textit{tàs arkhás (tinos)} (\iwi{Aristotle, \textit{Rhetoric} 1.2.21}), 5) μοχθηρίαν \textit{mokhthērían} (\iwi{Aristotle, \textit{Rhetoric} 1.10.4}), 6) κακόν \textit{kakón} (\iwi{Aristotle, \textit{Rhetoric} 1.11.8}), 7) ἐπιθυμίαν \textit{epithumían} (\iwi{Aristotle, \textit{Rhetoric} 1.11.14}), 8) ἀπολογίαν \textit{apologían} (\iwi{Aristotle, \textit{Rhetoric} 1.12.7}), 9) πρόφασιν \textit{próphasin} (\iwi{Aristotle, \textit{Rhetoric} 1.12.23}),10) κότον \textit{kóton} (\iwi{Aristotle, \textit{Rhetoric} 2.2.7}), 11) τιμήν \textit{timḗn} (\iwi{Aristotle, \textit{Rhetoric} 2.2.6}), 12) τὴν ὑπουργίαν \textit{tḕn hupourgían} (\iwi{Aristotle, \textit{Rhetoric} 2.7.4}), 13) βοήθειαν \textit{boḗtheian} (\iwi{Aristotle, \textit{Rhetoric} 2.21.15}), 14) δόξας \textit{dóxas} (\iwi{Aristotle, \textit{Rhetoric} 2.21.15}), 15) ὠφέλειαν \textit{ōphéleian} (\iwi{Aristotle, \textit{Rhetoric} 2.21.16}), 16) δίκην \textit{díkēn} (\iwi{Aristotle, \textit{Rhetoric} 2.3.5}), 17) τὴν αἰτίαν \textit{tḕn aitían} (\iwi{Aristotle, \textit{Rhetoric} 2.24.4}), 18) ἔνστασιν \textit{énstasin} (\iwi{Aristotle, \textit{Rhetoric} 2.25.10}). CO: 1) οὐδέν, ὅ τι λέγωσιν (ἂν) \textit{oudén, hó ti légōsin (án)} (\iwi{Aristotle, \textit{Rhetoric} 1.1.4}), 2) ὅ τι ἀπολέσει \textit{hó ti apolései} (\iwi{Aristotle, \textit{Rhetoric} 1.12.8}), & \hfil{27 (9+18)} & \hfil{8 (0+8)}\\
%     \hfill & \hfill & \hfill & 3) κυριωτάτην πίστιν \textit{kuriōtátēn pístin} (\iwi{Aristotle, \textit{Rhetoric} 1.2.4}), 4) κοινὸν εἶδος \textit{koinòn eĩdos} (\iwi{Aristotle, \textit{Rhetoric} 1.9.35}), 5) τὸ ἡδὺ καὶ τὸ καλόν \textit{tò hēdù kaì tò kalón} (\iwi{Aristotle, \textit{Rhetoric} 1.12.27}), 6) δύναμιν μεγάλην \textit{dúnamin megálēn} (\iwi{Aristotle, \textit{Rhetoric} 2.5.2}), 7) μίαν χρῆσιν \textit{mían khrē̃sin} (\iwi{Aristotle, \textit{Rhetoric} 2.21.16}), 8) πλείω τῶν ὑπαρχόντων \textit{pleíō tō̃n huparkhóntōn} (\iwi{Aristotle, \textit{Rhetoric} 2.22.11}) & \hfill & \hfill \\
% 	\textbf{λαμβάνειν, λαβεῖν (\textit{lambánein, labeĩn})} + Acc. & 27 / 23 & \hfil{3 (3+0)}\newline SO: 1) λαμβάνειν/λαβεῖν πίστεις lambánein/labeĩn písteis (thrice in total: \iwi{Aristotle, \textit{Rhetoric} 1.2.7} (aor.), \iwi{Aristotle, \textit{Rhetoric} 1.6.30} (adj.verb.), \iwi{Aristotle, \textit{Rhetoric} 1.8.7}), 2) ) λαβεῖν / λαμβάνειν προτάσεις labeĩn/ lambánein protáseis (twice: \iwi{Aristotle, \textit{Rhetoric} 1.3.9} (aor.), \iwi{Aristotle, \textit{Rhetoric} 1.9.2} (adj.verb.)), 3) λαμβάνειν /εἰληφέναι τιμωρίαν lambánein/ eilēphénai timōrían (twice: \iwi{Aristotle, \textit{Rhetoric} 2.3.13} (aor. pass.: ληφθεῖσα τιμωρία lēphtheĩsa timōría), \iwi{Aristotle, \textit{Rhetoric} 2.3.14} (pf.)).\footnote{As can be seen, there is some modification rather than a precise replication of the construction.} & \hfil{20 (6+14)}\newline SO: 1) δίκην \textit{díkēn} (\iwi{Aristotle, \textit{Rhetoric} 1.14.2}), 2) [ὅρκους \textit{hórkous}] (omitted Acc.) (\iwi{Aristotle, \textit{Rhetoric} 1.15.27}), 3) τὰς αὐξήσεις \textit{tàs auxḗseis} (\iwi{Aristotle, \textit{Rhetoric} 2.19.26}), 4) συμφοράς \textit{sumphorás} (\iwi{Aristotle, \textit{Rhetoric} 2.23.20}), 5) [δόξας dóxas] (restored Acc. from pass. \textit{eilēmménai dóxai}) (\iwi{Aristotle, \textit{Rhetoric} 2.18.2}), 6) [τοὺς τόπους \textit{toùs tópous}] (from pass. \textit{eilēmménoi ... hoi tópoi}) (\iwi{Aristotle, \textit{Rhetoric} 2.22.16}). CO: 1) τὰ στοιχεῖα καὶ τὰς προτάσεις \textit{tà stoikheĩa kaì tàs protáseis} (\iwi{Aristotle, \textit{Rhetoric} 1.2.22}), 2) τὰ στοιχεῖα περὶ ἀγαθοῦ καὶ συμφέροντος ἁπλῶς \textit{tà stoikheĩa perì agathoũ kaì sumphérontos haplō̃s} (\iwi{Aristotle, \textit{Rhetoric} 1.6.1}), 3) νοῦν καὶ φρόνησιν noũn \textit{kaì phrónēsin} (\iwi{Aristotle, \textit{Rhetoric} 1.7.3}), 4) \textit{toúnoma toũto} (\iwi{Aristotle, \textit{Rhetoric} 1.8.4}), & \hfil{9 (3+6)} & \hfil{14 (0+14)} \\
%     \hfill & \hfill & \hfill & 5) τὰ ὑπάρχοντα ἢ δοκοῦντα ὑπάρχειν \textit{tà hupárkhonta ḕ dokoũnta hupárkhein} (\iwi{Aristotle, \textit{Rhetoric} 2.22.8}), 6) τὸ τί ἐστι \textit{tò tí esti} (2.23.20), 7) \textit{tò kathólou} (\iwi{Aristotle, \textit{Rhetoric} 2.25.8}), 8) ψεῦδός τι \textit{pseũdós ti} (\iwi{Aristotle, \textit{Rhetoric} 2.26.4}), 9) τὰ σύνεγγυς τοῖς ὑπάρχουσιν ὡς ταὐτὰ ὄντα \textit{tà súnengus toĩs hupárkhousin hōs tautà ónta} (\iwi{Aristotle, \textit{Rhetoric} 1.9.28}), 10) τὰ ἀπὸ τύχης \textit{tà apò túkhēs} (\iwi{Aristotle, \textit{Rhetoric} 1.9.32}), 11) τὰ συμφέροντα καὶ τὰ ἡδέα \textit{tà sumphéronta kaì tà hēdéa} (\iwi{Aristotle, \textit{Rhetoric} 1.10.19}), 12) πόσα καὶ ποῖα \textit{pósa kaì poĩa} (\iwi{Aristotle, \textit{Rhetoric} 1.10.19}), 13) τὸ μετὰ τοῦτο ὡς διὰ τοῦτο \textit{tò metà toũto hōs dià toũto} (\iwi{Aristotle, \textit{Rhetoric} 2.24.8}), 14) τὴν Δημοσθένους πολιτείαν ... κακῶν αἰτίαν t\textit{ḕn Dēmosthénous politeían ... kakō̃n aitían} (\iwi{Aristotle, \textit{Rhetoric} 2.24.8}) & \hfill & \hfill \\
% 	\textbf{πάσχειν, παθεῖν, πεπονθέναι (\textit{páskhein, patheĩn, peponthénai})} + Acc. & \hfil{6 / 5} & \hfil{1 (1+0)}\newline SO: 1) πάσχειν κακά / κακόν \textit{páskhein kaká / kakón} (twice in total: \iwi{Aristotle, \textit{Rhetoric} 1.13.18, 2.3.14}) & \hfil{4 (4+0)}\newline SO: 1) ἀγαθά \textit{agathá} (\iwi{Aristotle, \textit{Rhetoric} 1.13.18}), 2) τὸ ἔσχατον \textit{tò éskhaton} (\iwi{Aristotle, \textit{Rhetoric} 2.3.16}), 3) \textit{anáxia} (\iwi{Aristotle, \textit{Rhetoric} 2.12.15}), 4) τὸ αὐτό \textit{tò autó} (\iwi{Aristotle, \textit{Rhetoric} 2.20.5}) & \hfil{5} & \hfil{0} \\ \pagebreak
% 	\textbf{πράττειν \textit{práttein*}} + Acc. & \hfil{1 / 1} & \hfil{0} & \hfil{1}\newline SO: μεγάλα πράττειν \textit{megála práttein} (\iwi{Aristotle, \textit{Rhetoric} 2.10.2}) (“experience great things (great fortunes)”) & \hfil{1} & \hfil{0} \\
% 	\midrule
% 	\textbf{TOTAL} & \hfil{83 / 64} & \hfil{13} & \hfil{51} & \hfil{42} & \hfil{22} \\
% \end{tabularx}



However, the general weakening of the semantic function of the verb and the closeness of the syntactic-semantic link between the verb and the adverb are important features that suggest parallels between verb + adverb phrases and SVCs (e.g. between phrases such as εὖ ἔχειν \textit{eũ ékhein} and χάριν ἔχειν \textit{khárin ékhein}). Since some of these constructions undergo a semantic change in the properties of the verb (the meaning is or seems to be non-literal) and the overall meaning of the expression is perceived only in the light of some non-literal interpretation. Periphrases of this kind resemble idioms.\footnote{Idioms not \textit{in sensu lato}, as one finds in \citet{Mastronarde2013} (passim, see esp. examples with ἔχω \textit{ékhō} and πράττω \textit{práttō} and adverbs on pp. 103--104), but in a stricter sense as described in \citet{Everaert2010} and \citet{Bruening2020}.}




Combining the data in the two tables, the following 23 phrases fall more or less into the category of SVC-type periphrases (in alphabetical order of the verbs). As can be seen from this list, a large proportion of these have lexical verbs that correspond to them in their core meaning (only verbs that are rare or absent in Aristotle’s texts and in Attic dialect texts close to his time are marked with a question mark; to be sure, the significant details of these correspondences still need to be checked): 

\begin{enumerate}
    \item \textbf{χάριν διδόναι (ἀποδιδόναι, ἀνταποδιδόναι)} \textit{khárin didónai (apodidónai, antapodidónai)} (1+1+1=3) ‘to give/return favour' = χαρίζειν \textit{kharízein}, χαρίζεσθαι \textit{kharízesthai};
    \item \textbf{δοῦναι δίκην} \textit{doũnai díkēn} (3) ‘to give right satisfaction', ‘to suffer punishment' = ζημιοῦσθαι \textit{zēmioũsthai} (cf. \iwi{Aristotle, \textit{Rhetoric} 1.9.15});
    \item \textbf{ἔχειν διαφοράν} \textit{ékhein diaphorán (diaphorás)} (2) ‘to have difference(s)' = διαφέρειν \textit{diaphérein};
    \item \textbf{ἔχειν δύναμιν} \textit{ékhein dúnamin} (5) ‘to have power' = δύνασθαι \textit{dúnasthai};
    ἔχειν ἐπιστήμην \textit{ékhein epistḗmēn} (1) ‘to have knowledge' = ἐπίστασθαι \textit{epístasthai};
    \item \textbf{ἔχειν μέγεθος} \textit{ékhein mégethos} (2) ‘to have size, importance' = μεγεθοῦσθαι \textit{megethoũsthai} (?);
    \item \textbf{ἔχειν συγγνώμην} \textit{ékhein sungnṓmēn} (2) ‘to have compassion/forgiveness' = συγγιγνώσκειν \textit{sungignṓskein};
    \item \textbf{χάριν ἔχειν} \textit{khárin ékhein} (3) ‘to have gratitude' = χαρίζεσθαι \textit{kharízesthai};
    \item \textbf{λαμβάνειν τιμωρίαν} \textit{lambánein timōrían} (2) ‘to obtain retaliation' = τιμωρεῖσθαι \textit{timōreĩsthai};
    \item \textbf{λέγειν (εἰπεῖν) ἐνθυμήματα} \textit{légein (eipeĩn) enthumḗmata} (4) ‘to speak up enthymemes/pieces of reasoning' = ἐνθυμεῖσθαι \textit{enthumeĩsthai};
    \item \textbf{λέγειν ἔπαινον} \textit{légein épainon} (1) ‘to say a word of praise' = ἐπαινεῖν \textit{epaineĩn};
    \item \textbf{λέγειν τἀληθῆ} \textit{légein talēthē̃} (1) ‘to speak the truth' = ἀληθεύειν \textit{alētheúein};
    \item \textbf{λέγειν τὰ ψευδῆ} \textit{légein tà pseudē̃} (1) ‘to tell lies' = ψευδολογεῖν \textit{pseudologeĩn};
    \item \textbf{λέγειν τὰς γνώμας} \textit{légein tàs gnṓmas} (1) ‘to say maxims' = γνωμολογεῖν \textit{gnōmologeĩn};
    \item \textbf{λέγειν ὑποθήκας} \textit{légein hupothḗkas} (1) ‘to tell advice' = ὑποτιθέναι \textit{hupotithénai} / ὑποτίθεσθαι \textit{hupotíthesthai};
    \item \textbf{λέγειν ψόγον} \textit{légein psógon} (1) ‘to say a word of blame' = ψέγειν \textit{pségein};
    \item \textbf{ποιεῖσθαι τὰς πίστεις} \textit{poieĩsthai tàs písteis} (2) ‘to produce proofs/means of persuasion' = πιστοῦν \textit{pistoũn} (?);
    \item \textbf{ποιεῖσθαι τὴν κρίσιν} \textit{poieĩsthai tḕn krísin} (1) ‘to make a judgement' = κρίνειν \textit{krínein};
    \item \textbf{ποιεῖσθαι τὸν λόγον (λόγους)} \textit{poieĩsthai tòn lógon (lógous)} (2+1=3) ‘to make/give a speech' = λέγειν \textit{légein};
    \item \textbf{ποιεῖσθαι τοὺς συλλογισμούς} \textit{poieĩsthai toùs sullogismoús} (1) ‘to make syllogisms' = συλλογίζεσθαι \textit{sullogízesthai};
    \item \textbf{φέρειν ἐνθυμήματα} \textit{phérein enthumḗmata} (2) ‘to provide enthymemes / pieces of reasoning' = ἐνθυμεῖσθαι \textit{enthumeĩsthai};
    \item \textbf{φέρειν ἔνστασιν} \textit{phérein énstasin} (5) ‘to bring (forward) an objection' = ἐνιστασθαι \textit{enístasthai};
    \item \textbf{φέρειν πίστεις} \textit{phérein písteis} (2) ‘to provide proof/means of persuasion' = πιστοῦν \textit{pistoũn} (?).
\end{enumerate}


So far, two or three criteria have been used to distinguish these expressions: (1) in most of these, the verb has a more or less\footnote{ἔχειν \textit{ékhein} and ποιεῖσθαι \textit{poieĩsthai}, for example, are less specific because they do not imply a clear instrument and situation for the action, whereas λέγειν \textit{légein} and φέρειν \textit{phérein} hint either at the mental/linguistic/rhetorical world and the organs and instruments involved in the action, or at a dramatic change of situation.} reduced semantic role and acts as a syntactic operator to convey the basic concept referred to by the noun, while (2) the latter, with few exceptions (cf. δοῦναι δίκην \textit{doũnai díkēn}), retains its basic meaning; (3) the above list contains provisional one-word equivalents of the phrases, implying that they are possible periphrases, or phraseological alternations, of individual verbs. 


In addition, many of these expressions seem to be transformable into nominal phrases without changing the noun’s core meaning\footnote{On this important criterion for the identification of SVCs/LVCs, see e.g. \citet[190--191]{JimenezLopez2016} and \citet[8]{Kovalevskaite-etal2020}.} (e.g. ἀδικία δύναμιν ἔχουσα \textit{adikía dúnamin ékhousa} (\iwi{Aristotle, \textit{Rhetoric} 2.5.4}), ‘injustice that has pow-er' > *ἀδικίας δύναμις \textit{adikías dúnamis}, ‘the power of injustice'), but in reality it is very rare to find in the texts of Aristotle and his contemporaries the nominalisations equivalent to the phrases at hand. So there is still more to discover here, and the number of SVC-type periphrases may change after additional categorisation. % power manually hyphenated to force line break to avoid hbox issue.


A broader intertextual investigation is also needed to reveal whether there is any regularity, in that different verbs are used with the base noun for similar meanings (e.g. χάριν διδόναι \textit{khárin didónai} ‘to give/express favour' and χάριν ἔχειν \textit{khárin ékhein} ‘to have gratitude', ποιεῖσθαι τὰς πίστεις \textit{poieĩsthai tàs písteis} ‘to produce proofs' and φέρειν πίστεις \textit{phérein písteis} ‘to bring/provide proofs'). Similarly, the reason why the author prefers the periphrases ἔχειν συγγνώμην \textit{ekhein sungnṓmēn} and λέγειν ἔπαινον \textit{légein épainon} to the forms with ποιεῖσθαι \textit{poieĩsthai} recorded in other contemporary writings remains to be clarified.\footnote{Cf. \iwi{Herodotus, \textit{Histories} 2.110}: Δαρεῖον … λέγουσι … \textbf{συγγνώμην ποιήσασθαι} \textit{Dareĩon ... légousi ... \textbf{sungnṓmēn poiḗsasthai}}; \iwi{Lysias, \textit{Pro milite} 22}: ὑπὲρ τῶν περιφανῶν ἀδικημάτων \textbf{συγγνώμην ποιεῖσθε}... \textit{hupèr tō̃n periphanō̃n adikēmátōn \textbf{sungnṓmēn poieĩsthe}...}; \iwi{Plato, \textit{Politicus}. 286c5-7}: χρὴ δὴ μεμνημένους ἐμὲ καὶ σὲ τῶν νῦν εἰρημένων τόν τε ψόγον ἑκάστοτε καὶ \textbf{ἔπαινον ποιεῖσθαι} \textit{khrḕ dḕ memnēménous emè kaì sè tō̃n nũn eirēménōn tón te psógon hekástote kaì \textbf{épainon poieĩsthai}}. } 

The material under study contains the following most common nouns in SVC-type periphrases: ἐνθύμημα \textit{enthúmēma} (6) ‘enthymeme', ‘piece of reasoning', δύναμις \textit{dúnamis} (5) ‘power', ἔνστασις \textit{énstasis} (5) ‘objection', πίστις \textit{pístis} (4) ‘proof', λόγος \textit{lógos} (3) ‘speech', χάρις \textit{kháris} (3) ‘favour', ‘gratitude'. These are abstract nouns, and given the Aristotelian concept of rhetoric, which assigns specific weight to various forms of persuasion and psychological effect, some of them could be classified as part of his rhetorical ‘technolect’. Their verbal partners may vary (e.g. ἐνθύμημα \textit{enthúmēma} goes with λέγειν \textit{légein} and φέρειν \textit{phérein}, χάρις \textit{kháris} with διδόναι \textit{didónai} and ἔχειν \textit{ékhein}). Common objects include the neuter adjectives κακόν \textit{kakón} and ἀγαθόν \textit{agathón} representing either nouns or adverbs (i.e. typical derivatives of abstract adjectives). However, adverbial periphrases are more common here, the four following constructions being the most frequent: οὕτως ἔχειν \textit{hoútōs ékhein} (26), πῶς ἔχειν \textit{pō̃s ékhein} (17), εὖ ποιεῖν \textit{eũ poieĩn} (12) and εὖ πάσχειν \textit{eũ páskhein} (9) (40\% of the 163 adverbial and adverbial-like constructions and over 18\% of the 350 verbal phrases selected from the currently analysed portion of Aristotle’s text).

\section{On the stylistic function of the support-verb-construction-type periphrases}\label{Section7Rhet}

As already mentioned (see the discussion above of stylistic tactics of brachylogy and macrology), periphrases can be classified according to their stylistic function. They indicate the author’s taste and intentions (aesthetic or pragmatic): either he/she aims at artistic effect (\textit{ornatus}\footnote{On the functions of the periphrasis (esp. according to Quintilian’s theory), see \citet[§592, 269--270]{Lausberg1998}.}) or seeks to improve comprehensibility, maintain \textit{decorum} (e.g. avoiding \textit{verba obscena}), or put a spontaneously caught thought into words. Thus, the expressions we encounter have their different occasion-related backgrounds: some are easy to grasp, others unclear due to an irregular sentence structure; some are often repeated, others are rare, occasional, and experimental.

A noteworthy stylistic phenomenon is the switching back and forth between MWEs and their shorter equivalents, the mutual substitution of words and phrases to avoid monotony and tautology. A good example of this alternation or variation (μεταβολή \textit{metabolḗ} or ἐναλλαγή \textit{enallagḗ} in Greek rhetorical terms)\footnote{\citet[§509, 236]{Lausberg1998}: other names for ‘grammatical changes’, but actually more complex inversions: ἐναλλαγή, ἑτεροίωσις, ἀλλοίωσις, ὑπαλλαγή \textit{exallagḗ, heteroíōsis, alloíōsis, hupallagḗ, mutatio}.} is in \iwi{Aristotle, \textit{Rhetoric} 2.19}, see (\ref{Ex5}), which deals with the topic of the possible and the impossible. Here the expression δυνατός ἐστι \textit{dunatós esti} alternates with the verb δύναται \textit{dúnatai} or with its own semantic head, the adjective δυνατός \textit{dunatós}, omitting the copula: 


\ea\label{Ex5}
    \glll ἂν δὴ τὸ ἐναντίον \textbf{ᾖ} \textbf{δυνατὸν} ἢ εἶναι ἢ γενέσθαι, καὶ τὸ ἐναντίον δόξειεν ἂν \textbf{εἶναι} \textbf{δυνατόν}, οἷον \textbf{εἰ} \textbf{δυνατὸν} ἄνθρωπον ὑγιασθῆναι, καὶ νοσῆσαι. καὶ εἰ τὸ ὅμοιον \textbf{δυνατόν}, καὶ τὸ ὅμοιον [...] καὶ οὗ ἡ ἀρχὴ \textbf{δύναται} \textbf{γενέσθαι}, καὶ τὸ τέλος· οὐδὲν γὰρ γίγνεται οὐδ᾽ ἄρχεται γίγνεσθαι τῶν ἀδυνάτων [...] καὶ οὗ τὸ τέλος, καὶ ἡ ἀρχὴ \textbf{δυνατή} \\
    \textit{àn} \textit{dḕ} \textit{tò} \textit{enantíon} \textit{ē̃i} \textit{dunatòn} \textit{ḕ} \textit{eĩnai} \textit{ḕ} \textit{genésthai}, \textit{kaì} \textit{tò} \textit{enantíon} \textit{dóxeien} \textit{àn} \textit{eĩnai} \textit{dunatón}, \textit{hoĩon} \textit{ei} \textit{dunatòn} \textit{ánthrōpon} \textit{hugiasthē̃nai}, \textit{kaì} \textit{nosē̃sai}. \textit{kaì} \textit{ei} \textit{tò} \textit{hómoion} \textit{dunatón}, \textit{kaì} \textit{tò} \textit{hómoion} [...] \textit{kaì} \textit{hoũ} \textit{hē} \textit{arkhḕ} \textit{dúnatai} \textit{genésthai}, \textit{kaì} \textit{tò} \textit{télos}; [...] \textit{kaì} \textit{hoũ} \textit{tò} \textit{télos}, \textit{kaì} \textit{hē} \textit{arkhḕ} \textit{dunatḗ}\\
	if but \textsc{art}.\textsc{nom} contrary.thing.\textsc{nom} \textsc{cop}.\textsc{prs.sbjv.3sg} possible.\textsc{nom} either be.\textsc{prs.inf} or become.\textsc{aor.inf} and \textsc{art}.\textsc{nom} contrary.thing.\textsc{nom} seem.\textsc{aor.opt.3sg} \textsc{prt} \textsc{cop}.\textsc{inf} possible.\textsc{nom} for.instance if possible.\textsc{nom.sg.n} man.\textsc{acc.sg} cure.\textsc{aor.inf.pass} and fall.ill.\textsc{aor.inf.act} and if \textsc{art}.\textsc{nom} similar.thing.\textsc{nom} possible.\textsc{nom} so.and \textsc{art}.\textsc{nom} similar.thing.\textsc{nom} [...] and \textsc{rel}.\textsc{gen} \textsc{art}.\textsc{nom} beginning.\textsc{nom} be.possible.\textsc{prs.ind.3sg} become.\textsc{aor.inf} so.and \textsc{art}.\textsc{nom} end.\textsc{nom} [...] and \textsc{rel}.\textsc{gen} \textsc{art}.\textsc{nom} end.\textsc{nom} so.and \textsc{art}.\textsc{nom} beginning.\textsc{nom} possible.\textsc{nom}\\
	\glt ‘If of two contrary things it is possible that one should exist or come into existence, then it would seem that the other is equally possible; for instance, if a man can be cured, he can also be ill; […] Similarly, if of two like things the one is possible, so also is the other. […] Again, if the beginning is possible, so also is the end; […] And when the end is possible, so also is the beginning' \\
 \hspace*{\fill}(\iwi{Aristotle, \textit{Rhetoric} 2.19.1-2, 1392a8-12}; \iwi{Aristotle, \textit{Rhetoric} 2.19.5, 1392a15-19}, translation by J. H. Freese).
\z

\largerpage

Some further examples of the alternation of periphrases (boldfaced) and their one-word equivalents can be found in  \REF{exTable3Arist}.


% \begin{table}
% \caption{Periphrases and their one-word alternatives\label{Table3Arist}}
% \scriptsize
% \begin{tabularx}{\textwidth}{p{3cm}Q}
%     \lsptoprule
%     {Pair} & {Example} \\
% 	\midrule
%     \textbf{συγγνώμην ἔχειν} vs συγγινώσκειν\newline
%     \textit{\textbf{sungnṓmēn ékhein}} vs \textit{sunginṓskein} & \vspace*{-6mm}\eanoraggedright\label{Ex6} ἐφ' οἷς τε γὰρ δεῖ \textbf{συγγνώμην ἔχειν}, ἐπιεικῆ ταῦτα, καὶ τὸ τὰ ἁμαρτήματα καὶ τὰ ἀδικήματα μὴ τοῦ ἴσου ἀξιοῦν, μηδὲ τὰ ἁμαρτήματα καὶ τὰ ἀτυχήματα· [...] καὶ τὸ τοῖς ἀνθρωπίνοις συγγινώσκειν ἐπιεικές.\newline \textit{eph’ hoĩs te gàr deĩ \textbf{sungnṓmēn ékhein}, epieikē̃ taũta, kaì tò tà hamartḗmata kaì tà adikḗmata mḕ toũ ísou axioũn, mēdè tà hamartḗmata kaì tà atukhḗmata; [...] kaì tò toĩs anthrōpínois sunginṓskein epieikés.} (\iwi{Aristotle, \textit{Rhetoric}. 1.13.15-16, 1374b4-11}) \z \\
%     \textbf{εὖ ποιεῖν} vs (ἀντ)ευποιεῖν\newline \textit{\textbf{eũ poieĩn}} vs \textit{(ant)eupoieĩn} & \vspace*{-6mm} \eanoraggedright\label{Ex7} τὸ χάριν ἔχειν τῷ \textbf{ποιήσαντι εὖ} καὶ ἀντευποιεῖν τὸν \textbf{εὖ ποιήσαντα}\newline \textit{tò khárin ékhein tō̃i \textbf{poiḗsanti eũ} kaì anteupoieĩn tòn \textbf{eũ poiḗsanta}} (\iwi{Aristotle, \textit{Rhetoric} 1.13.12, 1374a23-24}) \z \\
%     συμφέρειν vs \textbf{βλαβερὸν εἶναι} \newline \textit{sumphérein} vs \textit{\textbf{blaberòn eĩnai}} & \vspace*{-6mm} \eanoraggedright\label{Ex8} οὐδὲν γὰρ κωλύει ἐνίοτε ταὐτὸ συμφέρειν τοῖς ἐναντίοις· ὅθεν λέγεται ὡς τὰ κακὰ συνάγει τοὺς ἀνθρώπους, ὅταν \textbf{ᾖ} ταὐτὸ \textbf{βλαβερὸν} ἀμφοῖν \newline \textit{oudèn gàr kōlúei eníote tautò sumphérein toĩs enantíois; hóthen légetai hōs tà kakà sunágei toùs anthrṓpous, hótan \textbf{ē̃i} tautò \textbf{blaberòn} amphoĩn.} (\iwi{Aristotle, \textit{Rhetoric} 1.6.20, 1362b37-1363a1})\z \\
%     ἀδικεῖν vs \textbf{δίκαια πράττειν/ποιεῖν} \newline \textit{adikeĩn} vs \textit{\textbf{díkaia práttein/poieĩn}} & \vspace*{-6mm} \eanoraggedright\label{Ex9} καὶ οὓς ἀδικήσαντες δυνήσονται πολλὰ \textbf{δίκαια πράττειν}, ὡς ῥᾳδίως ἰασόμενοι, ὥσπερ ἔφη Ἰάσων ὁ Θετταλὸς δεῖν ἀδικεῖν ἔνια, ὅπως δύνηται καὶ \textbf{δίκαια} πολλὰ \textbf{ποιεῖν} \newline \textit{kaì hoùs adikḗsantes dunḗsontai pollà \textbf{díkaia práttein}, hōs rhaͅdíōs iasómenoi, hṓsper éphē Iásōn ho Thettalòs deĩn adikeĩn énia, hópōs dúnētai kaì \textbf{díkaia} pollà \textbf{poieĩn}}. (\iwi{Aristotle, \textit{Rhetoric} 1.12.31, 1373a24-27})\z \\
%     \textbf{πράττειν κακῶς} vs κακοπραγεῖν \newline \textit{\textbf{práttein kakō̃s}} vs \textit{kakoprageĩn} & \vspace*{-6mm} \eanoraggedright\label{Ex10} δεῖ γὰρ ἐπὶ μὲν τοῖς ἀναξίως \textbf{πράττουσι κακῶς} συνάχθεσθαι καὶ ἐλεεῖν, τοῖς δὲ εὖ νεμεσᾶν·[...] ὁ μὲν γὰρ λυπούμενος ἐπὶ τοῖς ἀναξίως κακοπραγοῦσιν ἡσθήσεται ἢ ἄλυπος ἔσται ἐπὶ τοῖς ἐναντίως κακοπραγοῦσιν, οἷον τοὺς πατραλοίας καὶ μιαιφόνους, ὅταν τύχωσι τιμωρίας, οὐδεὶς ἂν λυπηθείη χρηστός \newline \textit{deĩ gàr epì mèn toĩs anaxíōs \textbf{práttousi kakō̃s} sunákhthesthai kaì eleeĩn, toĩs dè eũ nemesãn;[...] ho mèn gàr lupoúmenos epì toĩs anaxíōs kakopragoũsin hēsthḗsetai ḕ álupos éstai epì toĩs enantíōs kakopragoũsin, hoĩon toùs patraloías kaì miaiphónous, hótan túkhōsi timōrías, oudeìs àn lupētheíē khrēstós} (\iwi{Aristotle, \textit{Rhetoric} 2.9.2-4, 1386b12-29})\z \\
% \end{tabularx}
% \end{table}



\ea
{Periphrases and their one-word alternatives\label{exTable3Arist}}


\ea \label{Ex6}
    \textbf{συγγνώμην ἔχειν} vs συγγινώσκειν\\
    \textit{\textbf{sungnṓmēn ékhein}} vs \textit{sunginṓskein} \\\medskip
    ἐφ' οἷς τε γὰρ δεῖ \textbf{συγγνώμην ἔχειν}, ἐπιεικῆ ταῦτα, καὶ τὸ τὰ ἁμαρτήματα καὶ τὰ ἀδικήματα μὴ τοῦ ἴσου ἀξιοῦν, μηδὲ τὰ ἁμαρτήματα καὶ τὰ ἀτυχήματα· [...] καὶ τὸ τοῖς ἀνθρωπίνοις συγγινώσκειν ἐπιεικές.\\
    \textit{eph’ hoĩs te gàr deĩ \textbf{sungnṓmēn ékhein}, epieikē̃ taũta, kaì tò tà hamartḗmata kaì tà adikḗmata mḕ toũ ísou axioũn, mēdè tà hamartḗmata kaì tà atukhḗmata; [...] kaì tò toĩs anthrōpínois sunginṓskein epieikés.} 
    \hspace*{\fill}(\iwi{Aristotle, \textit{Rhetoric}. 1.13.15-16, 1374b4-11})

\ex \label{Ex7}
    \textbf{εὖ ποιεῖν} vs (ἀντ)ευποιεῖν\\
    \textit{\textbf{eũ poieĩn}} vs \textit{(ant)eupoieĩn} \\\medskip
τὸ χάριν ἔχειν τῷ \textbf{ποιήσαντι εὖ} καὶ ἀντευποιεῖν τὸν \textbf{εὖ ποιήσαντα}\\
\textit{tò khárin ékhein tō̃i \textbf{poiḗsanti eũ} kaì anteupoieĩn tòn \textbf{eũ poiḗsanta}} 
\hspace*{\fill}(\iwi{Aristotle, \textit{Rhetoric} 1.13.12, 1374a23-24})

\newpage
\ex \label{Ex8}
    συμφέρειν vs \textbf{βλαβερὸν εἶναι} \\
    \textit{sumphérein} vs \textit{\textbf{blaberòn eĩnai}} \\\medskip
    οὐδὲν γὰρ κωλύει ἐνίοτε ταὐτὸ συμφέρειν τοῖς ἐναντίοις· ὅθεν λέγεται ὡς τὰ κακὰ συνάγει τοὺς ἀνθρώπους, ὅταν \textbf{ᾖ} ταὐτὸ \textbf{βλαβερὸν} ἀμφοῖν \\
    \textit{oudèn gàr kōlúei eníote tautò sumphérein toĩs enantíois; hóthen légetai hōs tà kakà sunágei toùs anthrṓpous, hótan \textbf{ē̃i} tautò \textbf{blaberòn} amphoĩn.} 
    \hspace*{\fill}(\iwi{Aristotle, \textit{Rhetoric} 1.6.20, 1362b37-1363a1})
\ex \label{Ex9}
    ἀδικεῖν vs \textbf{δίκαια πράττειν/ποιεῖν} \\
 \textit{adikeĩn} vs \textit{\textbf{díkaia práttein/poieĩn}} \\\medskip
 καὶ οὓς ἀδικήσαντες δυνήσονται πολλὰ \textbf{δίκαια πράττειν}, ὡς ῥᾳδίως ἰασόμενοι, ὥσπερ ἔφη Ἰάσων ὁ Θετταλὸς δεῖν ἀδικεῖν ἔνια, ὅπως δύνηται καὶ \textbf{δίκαια} πολλὰ \textbf{ποιεῖν} \\
 \textit{kaì hoùs adikḗsantes dunḗsontai pollà \textbf{díkaia práttein}, hōs rhaͅdíōs iasómenoi, hṓsper éphē Iásōn ho Thettalòs deĩn adikeĩn énia, hópōs dúnētai kaì \textbf{díkaia} pollà \textbf{poieĩn}}. 
 \hspace*{\fill}(\iwi{Aristotle, \textit{Rhetoric} 1.12.31, 1373a24-27})
\ex \label{Ex10}
    \textbf{πράττειν κακῶς} vs κακοπραγεῖν \\
 \textit{\textbf{práttein kakō̃s}} vs \textit{kakoprageĩn} \\\medskip
 δεῖ γὰρ ἐπὶ μὲν τοῖς ἀναξίως \textbf{πράττουσι κακῶς} συνάχθεσθαι καὶ ἐλεεῖν, τοῖς δὲ εὖ νεμεσᾶν·[...] ὁ μὲν γὰρ λυπούμενος ἐπὶ τοῖς ἀναξίως κακοπραγοῦσιν ἡσθήσεται ἢ ἄλυπος ἔσται ἐπὶ τοῖς ἐναντίως κακοπραγοῦσιν, οἷον τοὺς πατραλοίας καὶ μιαιφόνους, ὅταν τύχωσι τιμωρίας, οὐδεὶς ἂν λυπηθείη χρηστός \\
 \textit{deĩ gàr epì mèn toĩs anaxíōs \textbf{práttousi kakō̃s} sunákhthesthai kaì eleeĩn, toĩs dè eũ nemesãn;[...] ho mèn gàr lupoúmenos epì toĩs anaxíōs kakopragoũsin hēsthḗsetai ḕ álupos éstai epì toĩs enantíōs kakopragoũsin, hoĩon toùs patraloías kaì miaiphónous, hótan túkhōsi timōrías, oudeìs àn lupētheíē khrēstós} 
 \hspace*{\fill}(\iwi{Aristotle, \textit{Rhetoric} 2.9.2-4, 1386b12-29})
 \z
\z

In examples \REF{Ex6}--\REF{Ex10}, the interchange is rather veiled, e.g. the periphrasis συγγνώμην ἔχειν \textit{sungnṓmēn ékhein} in (\ref{Ex6}) is replaced by the verb συγγινώσκειν \textit{sunginṓskein} only in the next sentence; the compound verb ἀντ-ευποιεῖν \textit{ant-eupoieĩn} in (\ref{Ex7}) echoes the phrase εὖ ποιήσαντα \textit{eũ poiḗsanta} (the prefix hides the equivalent of the periphrasis); the verb συμφέρειν \textit{sumphérein} in (\ref{Ex8}) corresponds to the nominal phrase βλαβερὸν εἶναι \textit{blaberòn eĩnai} of opposite meaning, which is interchangeable with the verb βλάπτειν \textit{bláptein} (antonym to συμφέρειν \textit{sumphérein}); similarly, the verb ἀδικεῖν \textit{adikeĩn} (with complement ἔνια \textit{énia}) in (\ref{Ex9}) parallels the opposite phrase δίκαια πολλὰ ποιεῖν \textit{díkaia pollà poieĩn}, while κακοπραγεῖν \textit{kakoprageĩn} mirrors πράττειν κακῶς \textit{práttein kakō̃s} in (\ref{Ex10}). All this shows that Aristotle actively employed not only analytic but also synthetic constructions, i.e., he alternated the tactics of macrology and brachylogy.

Periphrases with other verbs (less frequent or with non-accusative objects) were not considered, but some possible candidates for SVC-type and Verb-Prepo-sitional Phrase Construction (V-PC)-type periphrases were noted. A few examples can be seen in \tabref{Table4Arist}. % Verb-Prepositional manually hyphenated to avoid hbox issue

% \footnote{The phrase is intertextually connected with Isoc. 1.27: ἀγάπα τῶν ὑπαρχόντων ἀγαθῶν μὴ τὴν ὑπερβάλλουσαν κτῆσιν ἀλλὰ τὴν μετρίαν ἀπόλαυσιν \textit{agápa tō̃n huparkhóntōn agathō̃n mḕ tḕn tḕn huperbállousan ktē̃sin allà tḕn metrían apólausin}. Cf. also Aristotle’s paraphrase recorded in another treatise: διὸ καὶ τὸν βίον ἀγαπῶσι τὸν ἀπολαυστικόν \textit{diò kaì tòn bíon agapō̃si tòn apolaustikón} (\textit{Eth. Nic.} 1095b17 Bekker).}

% \footnote{In various texts of Aristotle’s contemporaries, only the combination of the verb and preposition πρός \textit{prós} is repeated (cf. Plat. \textit{Symp}. 188d2-3, \textit{Resp}. 526d9-e1 et al.), sometimes with a prefix (συν-τείνειν \textit{sun-teínein}), while the combination with ἀλήθειαν \textit{alḗtheian} is very rare (used by Aristotle himself only in the quoted passage and in \textit{Top}. 104b1-2, and never by his contemporaries).}

% \footnote{The verb κακαγγελεῖν \textit{kakangeleĩn} is attested once with Demosthenes, cf. Dem. \textit{De cor.} 267, as a quotation from an unidentified tragedy.}

% \footnote{Cf. Eur. \textit{Or}. 696: ὅταν γὰρ ἡβᾷ δῆμος εἰς ὀργὴν πεσών (\textit{hótan gàr hēbãͅ dē̃mos eis orgḕn pesṓn}) “when the people youthfully rave, drowning in anger”. Cf. also: Trag. incert. fr. 80, v.1-2 (\citealt{Nauck1889}):\\ εἴπερ γὰρ οὐδὲ τοῖς κακῶς δεδρακόσιν\\ ἀκουσίως δίκαιον εἰς ὀργὴν πεσεῖν\\ \textit{eíper gàr oudè toĩs kakō̃s dedrakósin}\\ \textit{akousíōs díkaion eis orgḕn peseĩn}\\ (“if it is not right to be angry with those who have done wrong involuntarily”).}

The variety of periphrases is of course not limited to the verbal periphrases mentioned in this chapter. At least three other types of periphrasis can be identified in the present text: 1) the verbal periphrasis \textit{sensu stricto},\footnote{Of the type γεγονώς εἰμι \textit{gegonṓs eimi} or γεγενημένοι ἦσαν \textit{gegenēménoi ē̃san}.} with disputed terminological purity, most thoroughly studied by Klaas Bentein (\citealt{Bentein2016});\footnote{ A couple of examples of such periphrases in Aristotle’s \textit{Rhetoric} include: \textbf{ἔστι} δ' ἀπὸ τύχης μὲν τὰ τοιαῦτα \textbf{γιγνόμενα} \textit{\textbf{ésti} d’ apò túkhēs mèn tà toiaũta \textbf{gignómena}} (\iwi{Aristotle, \textit{Rhetoric} 1.10.12, 1369a32}; cf. \citealt[92]{Bentein2016}) and καὶ ἐὰν μεῖζον κακὸν \textbf{πεπονθότες ὦσιν} \textit{kaì eàn meĩzon kakòn \textbf{peponthótes ō̃sin}}, (\iwi{Aristotle, \textit{Rhetoric} 2.3.14, 1380b14}; cf. \citealt[128 n.87]{Bentein2016}).} 2) a certain kind of elaborated periphrasis which replaces parts of the sentence and makes use of articular infinitives\footnote{On articular infinitives see \citet[§§2025--2037]{Smyth1920}. Aristotle’s \textit{Rhetoric} has no shortage of such periphrases, ranging from 2 to 10 words. A couple of examples of longer periphrases include: τò παρὰ μικρòν σώζεσθαι ἐκ τῶν κινδύνων \textit{tò parà mikròn sṓzesthai ek tō̃n kindúnōn} (\iwi{Aristotle, \textit{Rhetoric} 1.11.24, 1371b10-11}), τὸ τὰ ἁμαρτήματα καὶ τὰ ἀδικήματα μὴ τοῦ ἴσου ἀξιοῦν’ \textit{tò tà hamartḗmata kaì tà adikḗmata mḕ toũ ísou axioũn} (\iwi{Aristotle, \textit{Rhetoric} 1.13.16, 1374b4-5}), τὸ ἢ μηδὲν γεγενῆσθαι ἀγαθὸν ἢ γενομένων μὴ εἶναι ἀπόλαυσιν \textit{tò ḕ mēdèn gegenē̃sthai agathòn ḕ genoménōn mḕ eĩnai apólausin} (\iwi{Aristotle, \textit{Rhetoric} 2.8.11, 1386a15-16}).} with complements, and 3) combinations of verbal adjectives in -τός (-\textit{tós}), -τή (-\textit{tḗ}), -τόν (-\textit{tón}), or -τικός (-\textit{tikós}), -τική (-\textit{tikḗ}), -τικόν (-\textit{tikón}) with copular verbs.\footnote{The latter type, not examined by us at present, would be considered a ‘true periphrasis’ in Lausberg’s rhetorical terminology, as it avoids the mention of the \textit{verbum proprium}. The following is one example of such a periphrasis in \iwi{Aristotle, \textit{Rhetoric} 2.12.3, 1389a3-5}: οἱ μὲν οὖν νέοι τὰ ἤθη \textbf{εἰσὶν ἐπιθυμητικοί} [...] καὶ τῶν περὶ τὸ σῶμα ἐπιθυμιῶν μάλιστα \textbf{ἀκολουθητικοί εἰσι} τῇ περὶ τὰ ἀφροδίσια \textit{hoi mèn oũn néoi tà ḗthē \textbf{eisìn epithumētikoí} [...] kaì tō̃n perì tò sō̃ma epithumiō̃n málista \textbf{akolouthētikoí eisi} tē̃ͅ perì tà aphrodísia} ‘In terms of their character, the young are prone to desires [...]. Of the desires of the body they are most inclined to pursue that relating to sex' (translation by G. A. Kennedy).}

\clearpage
{\scriptsize
\begin{xltabular}{\textwidth}{XX}
\caption{Further SVC candidates\label{Table4Arist}} \\
    \lsptoprule
    \multicolumn{1}{c}{SVC-type periphrasis} & \multicolumn{1}{c}{V-PC-type periphrasis} \\
	\midrule
    οὐχ ἑνὸς σώματος \textbf{ἀγαπᾶν ἀπόλαυσιν} \textit{oukh henòs sṓmatos \textbf{agapãn apólausin}}\footnote{The phrase is intertextually connected with \iwi{Isocrates, \textit{Speech} 1.27}: ἀγάπα τῶν ὑπαρχόντων ἀγαθῶν μὴ τὴν ὑπερβάλλουσαν κτῆσιν ἀλλὰ τὴν μετρίαν ἀπόλαυσιν \textit{agápa tō̃n huparkhóntōn agathō̃n mḕ tḕn tḕn huperbállousan ktē̃sin allà tḕn metrían apólausin} ‘value not the excessive acquisition of the goods that accrue to you, but the moderate enjoyment of them'. Cf. also Aristotle’s paraphrase recorded in another treatise: διὸ καὶ τὸν βίον ἀγαπῶσι τὸν ἀπολαυστικόν \textit{diò kaì tòn bíon agapō̃si tòn apolaustikón} (\iwi{Aristotle, \textit{Nicomachaean Ethics} 1095b17} Bekker)  ‘therefore they value (are fond of) the life based on enjoyment'.} (\iwi{Aristotle, \textit{Rhetoric} 2.23.8, 1398a23}) > \textbf{ἀγαπᾶν απόλαυσιν \textit{agapãn apólausin}} ‘to be fond of enjoyment' [= ἀπολαύειν \textit{apolaúein}?] & \textbf{πρὸς ἀλήθειαν} ... \textbf{τείνει} ταῦτα \textit{\textbf{pròs alḗtheian} ... \textbf{teínei} taũta} (\iwi{Aristotle, \textit{Rhetoric} 1.7.40, 1365b15}) > \textbf{τείνειν πρὸς ἀλήθειαν} \textit{\textbf{teínein pròs alḗtheian}}\footnote{In various texts of Aristotle’s contemporaries, only the combination of the verb and preposition πρός \textit{prós} is repeated (cf. \iwi{Plato, \textit{Symposium} 188d2-3}, \iwi{Plato, \textit{Republic} 526d9-e1} et al.), sometimes with a prefix (συν-τείνειν \textit{sun-teínein}, ‘direct earnestly (to)', ‘tend/contribute (towards)'), while the combination with ἀλήθειαν \textit{alḗtheian} is very rare (used by Aristotle himself only in the quoted passage and in \iwi{Aristotle, \textit{Topica} 104b1-2}, and never by his contemporaries).} (“to point to the truth”)\\
    τοῖς \textbf{κακὰ ἀγγέλλουσιν} \textit{toĩs \textbf{kakà angéllousin}} (\iwi{Aristotle, \textit{Rhetoric} 2.2.20, 1379b20}) > \textbf{ἀγγέλλειν κακά} \textit{\textbf{angéllein kaká}} ‘to report bad news' [= \textbf{κακαγγελεῖν} \textbf{\textit{kakangeleĩn}}\footnote{The verb κακαγγελεῖν \textit{kakangeleĩn} ‘bring evil tidings' is attested once with Demosthenes, cf. \iwi{Demosthenes, \textit{De Corona} 267}, as a quotation from an unidentified tragedy.}?] & \textbf{πίπτειν, πεσεῖν, ἐμπίπτειν + εἰς + Acc.}/ \textbf{\textit{píptein, peseĩn, empíptein} + \textit{eis} + Acc.} \newline \textbf{πίπτει} ... ἡ αὔξησις \textbf{εἰς τοὺς ἐπαίνους} / \textit{\textbf{píptei} ... hē aúxēsis \textbf{eis toùs epaínous}} (\iwi{Aristotle, \textit{Rhetoric} 1.9.39, 1368a23}) > \textbf{πίπτειν εἰς τοὺς ἐπαίνους} / \textbf{píptein eis toùs epaínous} ‘to fall among forms of praise' [= \textbf{προσκεῖσθαι / προσεῖναι τοῖς ἐπαίνοις}? / \textbf{proskeĩsthai / proseĩnai toĩs epaínois}?] \newline οὐδὲ τοῖς κακῶς δεδρακόσιν ἀκουσίως δίκαιον \textbf{εἰς ὀργὴν πεσεῖν} / \textit{oudè toĩs kakō̃s dedrakósin akousíōs díkaion \textbf{eis orgḕn peseĩn}}”\footnote{Cf. \iwi{Euripides, \textit{Orestes} 696}: ὅταν γὰρ ἡβᾷ δῆμος εἰς ὀργὴν πεσών \textit{hótan gàr hēbãͅ dē̃mos eis orgḕn pesṓn} ‘when the people youthfully rave, drowning in anger'. Cf. also: \iwi{\textit{Tragicorum Graecorum Fragmenta} 80, v.1-2} (\citealt{Nauck1889}):\\ εἴπερ γὰρ οὐδὲ τοῖς κακῶς δεδρακόσιν\\ ἀκουσίως δίκαιον εἰς ὀργὴν πεσεῖν\\ \textit{eíper gàr oudè toĩs kakō̃s dedrakósin}\\ \textit{akousíōs díkaion eis orgḕn peseĩn} ‘if it is not right to be angry with those who have done wrong involuntarily'.} (\iwi{Aristotle, \textit{Rhetoric} 2.23.1, 1397a13-14}, quoted from unknown drama) ‘it is unjust to fall into anger at those who have unwillingly done wrong' > \textbf{εἰς ὀργὴν πίπτειν (πεσεῖν)} / \textit{\textbf{eis orgḕn píptein(peseĩn)}} [= \textbf{ὀργίζεσθαι, ἐξαγριοῦσθαι} / \textit{\textbf{orgízesthai, exagrioũsthai}}] \newline εἰς τὴν ἔλλειψιν ἐμπίπτει / eis tḕn élleipsin empíptei (\iwi{Aristotle, \textit{Rhetoric} 2.24.7, 1401b29}) ‘it... falls under the [the fallacy of] omission' > \textbf{εἰς τὴν ἔλλειψιν ἐμπίπτειν} / \textit{\textbf{eis tḕn élleipsin empíptein}} \\ προσῆκον εἶναι τῷδ᾽ \textbf{ὀφείλεσθαι χάριν} \textit{prosē̃kon eĩnai tō̃id᾽ \textbf{opheílesthai khárin}} (\iwi{Aristotle, \textit{Rhetoric} 2.23.1, 1397a16}, from an unknown drama) > \textbf{χάρις ὀφείλεται} \textit{\textbf{kháris opheíletai}} (pass. pro act.) > \textbf{χάριν ὀφείλειν} \textbf{\textit{khárin opheílein}} ‘owe gratitude' & \\
    \lspbottomrule
\end{xltabular}
}


\section{Conclusions}

Aristotle’s \textit{Rhetoric}, the source of the phraseology of the fourth-century BC Attic dialect studied in this chapter, is a complex, multi-layered text in which the language of Athens at the height of Athenian drama and oratory is intertwined with Aristotle’s scholarly vocabulary and rhetorical ‘technolect’, and with the phraseology of various dialectal varieties and genres of text, presented as quotations.

An empirical examination of two thirds of this source (Books 1 and 2, covering over 32,500 words) showed that it contains no less than 350 verb-based phrases with popular accusative-taking verbs, of which 23 are of the SVC type. The most important criteria for identifying this type of expressions are the role of the verb as a syntactic operator with a reduced meaning, the semantic dominance of the abstract noun or noun-like adjective, the existence of a one-word equivalent (of the type ποιεῖσθαι τὸν λόγον \textit{poieĩsthai tòn lógon} = λέγειν \textit{légein}), and the repetitiveness of the phrase. Other criteria are more difficult to verify due to the lack of textual evidence.

The set of 350 verb-based phrases also includes up to more than 150 verb-noun combinations with the same semantically flexible verbs, and more than 160 combinations with adverbs and complex complements. This contributes to the discussion on the concept of SVC, as it is hypothesised that a support verb can also be a seemingly lexically complete causative verb (such as ποιεῖν \textit{poieĩn} ‘to do, make') with an accusative duplex, or a subject-oriented transitive verb (such as ἔχειν \textit{ékhein} ‘to have'), that drastically changes meaning when used in combinations with adverbs.

Theoretical reflection on the terms and their corresponding phenomena has shown that the linguistic terms MWE, SVC, and others, which are applied universally to phraseological phenomena in various languages, can in principle also account for Ancient Greek phenomena. At the same time, concepts invented by users of Ancient Greek themselves, such as ‘periphrasis’, or epithets designating stylistic strategies (‘macrological’, ‘brachylogical’), etc., also prove to be descriptively adequate.

Periphrasis is a term that has survived from Graeco-Roman rhetoric into modern linguistics to describe the substitution of a short lexical unit (a word) by a longer one (a combination of two or more words). The description of the periphrasis by the second-century-AD rhetorician Alexander Numenius, with appropriate examples, matches well in its content with what is nowadays labeled SVC. Since the term ‘periphrasis’, defined more strictly in linguistic contexts with emphasis on its grammatical function (as a cell-filler for a grammatical paradigm) does not stand in contradiction with the original meaning of the concept, the substitution of one word by two or more words, it may be the key to a possible solution for the terminological problem of reconciling the MWEs and the various phraseological units: the use of the term periphrasis as a synonym for the MWE, provided that both indicate substitution or alternation.  

The idea of the dichotomy between the change of valency and the inherent meaning of verbs, inspired by the theories of valency and transitivity change and their possible parallel in Aristotle’s conception of the conditions of the effective speech (ἦθος, πάθος, λόγος \textit{ē̃thos, páthos, lógos}), supports a simplified dichotomous classification of transitive verbs into introversive and extraversive ones, which in turn may help in the future to better assess the nuances of the semantic contribution of verbs in periphrases (or MWEs) to the overall meaning of a phrase.

The author’s personal style, scientific interests, aesthetic and occasional preferences (represented by the ‘macrological’ and ‘brachylogical’ alternatives) undoubtedly affected the variety and alternation of phrases contained in Aristotle’s \textit{Rhetoric}. This stylistic flexibility demonstrates the expressive capability of the Greek language, as well as each author’s creative contribution to the overall phraseological ‘bank’ of the language.

% \pagebreak



\section*{Abbreviations}

\begin{tabularx}{.5\textwidth}{@{}lQ@{}}
%Wyn, please search up 'henceforth' and add all the abbreviations I identified and marked as such but have not yet put into here
    AM & Agent marker \\
    AS & Agent-role subject \\
    CO & Complex object \\
%\footnote{Consisting of a basic nominal unit (noun, pronoun, \textit{sim}.) with an attribute or two or more nominal units joined by a conjunction, or construction (esp. double accusative).} \\
    CP & Compositional phrase \\
    DO & Direct object  \\
    FVC & Function-verb construction \\
\end{tabularx}%
\begin{tabularx}{.5\textwidth}{@{}lQ@{}}
    LVC & Light-verb construction \\
    MWE & Multi-word expression \\
    SO & Single object \\
% \footnote{Consisting of one noun or nominal unit.} \\
    V+CO & Verb with a complex object \\
    V+SO & Verb with a single object \\
    \\
\end{tabularx}

\section*{Acknowledgements}

First and foremost, I would like to thank Victoria Beatrix Fendel, Leverhulme Trust Early Career Fellow at the Faculty of Classics, University of Oxford, and her team for organising the excellent conference on SVCs in the Greek corpora and the fruitful discussions on this topic in September 2023, as well as for the step-by-step encouragement and advice that helped me write and edit the chapter. I am especially grateful to the reviewers for their constructive criticism and comments, which encouraged me to recognise new nuances, and for their accurate observations of the weaknesses and strengths of my work. My special thanks go to Matthew Ireland for the distant introductory lesson on working with Overleaf projects. I would also like to thank my colleague from the Department of French Philology at Vilnius University, Antanas Keturakis, for a brief practical introduction to working with LaTeX. Finally, I would like to thank Dmitry Dundua, who agreed to convert my article into a LaTeX template, thus contributing to the overall quality of the present publication.


\sloppy
\printbibliography[heading=subbibliography,notkeyword=this]
\end{document}
