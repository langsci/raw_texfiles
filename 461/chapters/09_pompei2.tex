\documentclass[output=paper,colorlinks,citecolor=brown]{langscibook}
\ChapterDOI{10.5281/zenodo.14017937}
\author{Anna Pompei\affiliation{Roma Tre University} and         Flavia Pompeo\affiliation{Sapienza University of Rome} and Eleonora Ricci\affiliation{Sapienza University of Rome; Roma Tre University}}
\title{Analytical and synthetic verbs: The lightness degree of ποιέω \textit{poiéō}}
\abstract{This chapter focuses on the alternation between analytic constructions (e.g., παῖδας ποιοῦμαι \textit{paîdas poioûmai} ‘to beget children’) and equivalent synthetic verbs (e.g., παιδοποιέω \textit{paidopoiéō} ‘to beget children’). The synthetic forms are considered here as noun incorporations in synchrony, as the second element of the compound is a verb that can also occur as a free form. The analysis of data (from the 5th c. BC to the beginning of the 2nd c. AD) shows that the selection of either analytic or synthetic forms is made for (i) semantic reasons, i.e., the specificity of the noun, and (ii) textual reasons, i.e., the establishment of the referent in the discourse, closely related to the information structure. Moreover, the overlapping between support-verb constructions and incorporations only concerns so-called simple-event nominals\is{simple-event nominal}, whereas complex-event nominals, which are fully predicative, cannot be incorporated. Analytic constructions equivalent to non-eventive noun incorporations are usually not support-verb constructions.

\bigskip


Questo capitolo è incentrato sull’alternanza tra costruzioni analitiche, come παῖδας ποιοῦμαι \textit{paîdas poioûmai} ‘generare figli’, e forme sintetiche equivalenti, come παιδοποιέω \textit{paidopoiéō} ‘generare figli’. Le forme sintetiche sono qui considerate incorporazioni del nome in sincronia, in quanto il secondo elemento del composto è un verbo che può occorrere anche in forma libera. L’analisi dei dati (dal sec. V a.C. all’inizio del II d. C.) mostra che l’alternanza tra forme analitiche e sintetiche è determinata i) da ragioni semantiche, ossia dalla specificità del nome, nonché ii) da ragioni testuali di instaurazione del referente nel discorso, strettamente legate alla distribuzione dell’informazione. L’area di sovrapposizione tra costruzioni a verbo supporto e incorporazioni, inoltre, riguarda solo i cosiddetti simple-event nominals, mentre i complex-event nominals, pienamente eventivi, non risultano mai incorporati. Le costruzioni analitiche che equivalgono a incorporazioni di nomi non eventivi non sono, invece, costruzioni a verbo supporto.
}


\IfFileExists{../localcommands.tex}{
   \addbibresource{../localbibliography.bib}
   \usepackage{langsci-optional}
\usepackage{langsci-gb4e}
\usepackage{langsci-lgr}

\usepackage{listings}
\lstset{basicstyle=\ttfamily,tabsize=2,breaklines=true}

%added by author
% \usepackage{tipa}
\usepackage{multirow}
\graphicspath{{figures/}}
\usepackage{langsci-branding}

   
\newcommand{\sent}{\enumsentence}
\newcommand{\sents}{\eenumsentence}
\let\citeasnoun\citet

\renewcommand{\lsCoverTitleFont}[1]{\sffamily\addfontfeatures{Scale=MatchUppercase}\fontsize{44pt}{16mm}\selectfont #1}
  
   %% hyphenation points for line breaks
%% Normally, automatic hyphenation in LaTeX is very good
%% If a word is mis-hyphenated, add it to this file
%%
%% add information to TeX file before \begin{document} with:
%% %% hyphenation points for line breaks
%% Normally, automatic hyphenation in LaTeX is very good
%% If a word is mis-hyphenated, add it to this file
%%
%% add information to TeX file before \begin{document} with:
%% %% hyphenation points for line breaks
%% Normally, automatic hyphenation in LaTeX is very good
%% If a word is mis-hyphenated, add it to this file
%%
%% add information to TeX file before \begin{document} with:
%% \include{localhyphenation}
\hyphenation{
affri-ca-te
affri-ca-tes
an-no-tated
com-ple-ments
com-po-si-tio-na-li-ty
non-com-po-si-tio-na-li-ty
Gon-zá-lez
out-side
Ri-chárd
se-man-tics
STREU-SLE
Tie-de-mann
}
\hyphenation{
affri-ca-te
affri-ca-tes
an-no-tated
com-ple-ments
com-po-si-tio-na-li-ty
non-com-po-si-tio-na-li-ty
Gon-zá-lez
out-side
Ri-chárd
se-man-tics
STREU-SLE
Tie-de-mann
}
\hyphenation{
affri-ca-te
affri-ca-tes
an-no-tated
com-ple-ments
com-po-si-tio-na-li-ty
non-com-po-si-tio-na-li-ty
Gon-zá-lez
out-side
Ri-chárd
se-man-tics
STREU-SLE
Tie-de-mann
}
   \boolfalse{bookcompile}
   \togglepaper[7]%%chapternumber
}{}


\begin{document}
\emergencystretch 3em
\maketitle


\section{Introduction: analytical constructions, support verbs, and incorporations}\label{Section1}
This chapter examines the reasons for selecting either analytical verbal constructions (e.g., παῖδας ποιοῦμαι \textit{paîdas poioûmai} `to beget children', as in \REF{ex:1}) or synthetic verbs, such as instances of noun incorporation\is{incorporation} (e.g., παιδοποιέω \textit{paidopoiéō} `to beget children', as in \REF{ex:2}) in Ancient Greek.\footnote{The Greek texts considered in this article cover the period from the 5\textsuperscript{th} c. BC to the beginning of the 2\textsuperscript{nd} c. AD (Plutarch). They are quoted according to the editions in the \textit{Thesaurus Linguae Graecae} (henceforth \textit{TLG}) electronic corpus (\url{https://stephanus.tlg.uci.edu}); texts classified in the \textit{TLG} as \textit{Fragmenta} were excluded from the corpus. For the case study presented in \sectref{Section2}, a sub-corpus has been considered (\sectref{Section2/1}). English translations are based on the \textit{Loeb Classical Library}. For the sake of readability, glosses are limited to basic morphological information (singular number not indicated for nouns, adjectives, participles, and articles; active voice, indicative mood, and present tense not indicated for verbs).}

%\ea 
%\gll Φαίνεται	τοίνυν\\
%	 phaínetai	toínun\\
%	 
%\glt 
%\z

\ea \label{ex:1}
	\glll Φαίνεται	τοίνυν	οὐχ	ὁ ἐμὸς 	πατὴρ	πρῶτος ὦ 	ἄνδρες        Ἀθηναῖοι, 	λαβὼν 			τὴν		ἐμὴν 		μητέρα, ἀλλ’ ὁ 		Πρωτόμαχος,		καὶ	\textbf{παῖδας}		\textbf{ποιησάμενος}	   	   καὶ θυγατέρ’		ἐκδούς·\\
	\textit{phaínetai}		\textit{toínun}		\textit{oukh} 	\textit{ho}	 	\textit{emòs}		\textit{patḕr}		\textit{prôtos} 	\textit{ô}	\textit{ándres}	       \textit{Athēnaîoi}		\textit{labṑn}			\textit{tḕn}		\textit{emḕn}		\textit{mētéra}	\textit{all’}	  \textit{ho}		\textit{Prōtómakhos}		\textit{kaì}	\textit{paîdas}		\textit{poiēsámenos}	   	   \textit{kaì}     \textit{thugatér’} 		\textit{ekdoús}\\
	be.plain.\textsc{mid/pass}.\Tsg{}	now		\Neg{}	\Art{}.\Nom{}.\M{}	\Poss{}.\Nom{}.\M{}	father.\Nom{}.\M{}	first.\Nom{}.\M{}	oh	man.\Voc{}.\M{}.\Pl{} Athenian.\Voc{}.\M{}.\Pl{}	take.\Aor{}.\Ptcp{}.\Nom{}.\M{}	\Art{}.\Acc{}.\F{}	\Poss{}.\Acc{}.\F{}	mother.\Acc{}.\F{}	but	 \Art{}.\Nom{}.\M{}	Protomachus.\Nom{}.\M{}	and	son.\Acc{}.\M{}.\Pl{}	make.\Aor{}.\Ptcp{}.\Mid{}.\Nom{}.\M{}  and	 daughter.\Acc{}.\F{} 	give.\Aor{}.\Ptcp{}.\Nom{}.\M{}\\

\glt `Now it is plain, men, that it was not my father who first received my mother in marriage. No; it was Protomachus, and he had by her a son, and a daughter whom he gave in marriage' \\
\hspace*{\fill}(\iwi{Demosthenes, \textit{Speech} 57.43})
\z

\ea \label{ex:2}
\glll Οὐκοῦν	οὕτω		γε	οὐ	δεῖ		\textbf{παιδοποιεῖσθαι};\\
	\textit{oukoûn}	\textit{hoútō}		\textit{ge}	\textit{ou}	\textit{deî}		\textit{paidopoieîsthai}	\\
	then	in.that.case	\textsc{prt}	\Neg{}	need.\Tsg{}	child.make.\Inf{}.\textsc{mid/pass}	\\
\glt `In that case then, they ought not to have children?' \\
\hspace*{\fill}(\iwi{Xenophon, \textit{Memorabilia} 4.4.23})
\z

The specific aim of this chapter is twofold: (a) to identify the reasons for selecting either analytic constructions or synthetic verbs (\sectref{Section2}); (b) to verify whether analytic predicates are always support-verb constructions (SVCs henceforth) or not, and, in the latter case, to highlight the consequences in terms of their possible equivalence with synthetic verbs (\sectref{Section3}).

By SVCs we mean a type of complex predicate\is{complex predicate}, a notion that originates in syntactic approaches such as \textit{Lexical-Functional Grammar} \citep{Bresnan1982, Bresnan2001} and \textit{Relational Grammar} \citep{Perlmutter1974}. In the framework of the former, complex predicates are multi-headed predicates, i.e., they are ‟composed of more than one grammatical element [\ldots{}], each of which contributes part of the information ordinarily associated with a head" \citep[1]{Alsina1997}. This is, for instance, the position of \citet[49]{Butt2010}: she considers support verbs (SVs henceforth) complex predicates, and argues that ‟the term complex predicate designates a construction that involves two or more predicational elements (e.g., nouns, verbs and adjectives) which predicate as a single unit". From this perspective, SVs are not completely empty elements with respect to the dense meaning spectrum of the equivalent lexically full verb (e.g., SV \textit{give someone a kiss} vs. \textit{give him a ball}; cf. \citealt[326; 339--340]{Butt2001}). From the perspective of \textit{Relational Grammar}, different predicates may exist in a single clause (clause union) as long as they are placed in successive strata (rather than in the same stratum) in a multi-stratal structure (\citealt[45--59]{LaFauci2003} on Italian, \textit{inter alia}).\footnote{The multi-stratal approach of \textit{Relational Grammar} involves the positing of grammatical relations at various levels or strata. In particular, the predicative noun is considered the initiator-predicate of the construct in the lowest stratum \citep[45--59]{LaFauci2003}. In order to license the subject of a proposition, it needs the aid of a non-initiator-predicate (e.g., the Italian support verb \textit{fare} `to do' in \textit{fare un peccato} lit. `to do a sin' > `to commit a sin'): it is located in the successive stratum and makes the subject pertain to the whole SVC \citep[46]{LaFauci2003}.} 


These perspectives are significantly different from earlier approaches to SVs, which do not allow for two predicates in a clause \citep[55]{Gross1996}, as the predication is conceived as unique and completely noun-dependent \citep[167]{Grossintro2004}. For instance, \citet[117]{Jespersen1942} considers the verb in Verb + Noun (V + N henceforth) constructions, such as \textit{to have a swim}, \textit{to take a walk}, and \textit{to give a sigh} in English, as a ‟light verb", i.e., ‟an insignificant verb, to which the marks of person and tense are attached, before the really important idea" conveyed by a deverbal noun that usually expresses `the action or an isolated instance of the action'. Such an idea of verb lightness\is{support-verb lightness} highlights the semantic bleaching of the verb. 


In a similar way, in the French definition of \textit{verbe support} the morpho-syntactic function of the verb is indicated exclusively. Indeed, the verb is considered as mere support, encoding only grammatical categories such as Tense-Aspect-Mood and agreement features, but it does not predicate: it only ‟actualises" the predicative noun (\textit{prédicat nominal})---in which the whole predication stands---thus having the same function as verb endings \citep[167]{Grossintro2004}.\footnote{See \citet[1--6]{Pompei2023} for a review of different theoretical perspectives on the notion of SV.} Overall, we find ourselves in opposition to this narrow binary division between a predicative noun and an empty verb, as will be discussed in detail below (\sectref{Section3}).

Synthetic verbs such as παιδοποιεῖσθαι \textit{paidopoieîsthai} `to beget children' in \REF{ex:2} can be considered instances of incorporation\is{incorporation} \citep{Pompei2006,Pompei2012}, namely, a compounding process between a verb and another part of speech that yields a new verb \citep{Baker1988}. In particular, in \REF{ex:2} there is an instance of noun incorporation, i.e., a process of composition of a noun and a verb, which outputs a new verb form ([N+V]\textsubscript{V}) (\citealt[257, \textit{passim}]{Sapir1911}; \citealt{Mithun1984,Mithun1986,Mithun1997}). As is well known, this is a very productive process in polysynthetic languages, particularly in compositional ones \citep{Mattissen2023}: 
\largerpage[2]

\begin{exe}\label{ex:3}
	\ex 
        \settowidth \jamwidth{(Onondaga, Iroquoian)} 
	\begin{xlist}
		\ex \label{ex:3a}\gll \textit{Pet}	\textit{\textbf{waˀ-ha-htu-ˀt-aˀ}}	\textit{\textbf{neˀ}}	\textit{\textbf{o-hwist-aˀ}}\\
				Pat	past-3ms/3N-lost-caus-asp	the	pre-money-suf\\ \jambox{(Onondaga, Iroquoian)}
			\glt `Pat lost the money' \\
   \hspace*{\fill}(\citealt[76--77]{Baker1988})
		\ex \label{ex:3b}\gll \textit{Pet}	\textit{\textbf{waˀ-ha-hwist-a-htu-ˀt-aˀ}}\\
				Pat	past-3ms-money-ep-lost-caus-asp\\ \jambox{(Onondaga, Iroquoian)}
			\glt `Pat lost money' \\
   \hspace*{\fill}(\citealt[76--77]{Baker1988})
	\end{xlist}
\end{exe}
\il{Iroquoian}

There is an analytical form in \REF{ex:3a}, i.e., a verbal phrase made up of a verb form followed by a noun phrase that is its direct object; on the other hand, the synthetic verb form in \REF{ex:3b} is the result of the incorporation\is{incorporation} of the noun into the verb (in this case with the interposition of an epenthetic vowel). It is worth noting that the incorporated noun is a bare one, as in this case both the prefix---which is a gender marker---and the suffix---which marks the lexical category---are missing, as well as the article marking definiteness. 


Despite its wide productivity in polysynthetic languages, incorporation is not an exclusive phenomenon of this morphological type.\footnote{Incorporation can also occur in agglutinative languages, such as Japanese (e.g., \citealt[229]{Grimshaw1988}), and even in isolating ones, such as Chinese (\citealp{Luo2022}, \textit{inter alia}). As far as fusional languages are concerned, the equivalence between Latin synthetic and analytical verbs, such as \textit{belligero} $\sim$ \textit{bellum gero} `to wage war' and \textit{ludifico} $\sim$ \textit{ludos facere} `to make an object of sport, trifle with', have been studied by \citet{Banos2012,Banos2012a}.} As far as Ancient Greek is concerned, there are formations---such as παιδοποιέω \textit{paidopoiéō} `to beget children', σιτομετρέω \textit{sitometréō} `to measure the wheat/provisions', καρπολογέω \textit{karpologéō} `to gather fruit', λογογραφέω \textit{logographéō} `to write speeches'---which show the same features as incorporation\is{incorporation} in polysynthetic languages from the morpho-phonological, semantic, and pragmatic points of view \citep{Pompei2006}. Diachronically, these formations have usually been considered as formed by conversion from both nominal compounds (e.g. λογογραφέω \textit{logographéō} `to write speeches' < λογογράφος \textit{logográphos} `speech writer') and adjective ones (e.g. καρπολογέω \textit{karpologéō} `to gather fruit' < καρπολόγος \textit{karpológos} `gathering fruit'; cf. \citealp[301]{meissner2002nominal}).

Synchronically, some of these formations can be considered instances of effective noun incorporation\is{incorporation}, i.e., instances of composition (cf. \citealt{Pompei2012}, from a \textit{Construction Grammar} perspective). In particular, this is true in cases in which the second element is a verb that can also occur as a free form, as the comparison between \REF{ex:1} and \REF{ex:2} clearly shows with regard to ποιέω \textit{poiéō} `to do, make'. For this reason, our comparison between analytical and incorporated constructions will focus on this verb.\footnote{In this chapter we do not consider instances like σιτομετρέω \textit{sitometréō} < σῖτον μετρέω \textit{sîton metréō} `to measure the wheat/provisions', as these are examples of collocations in which the verb retains its lexically full meaning. On the other hand, according to \citet[205]{Jezek2016}, SVCs are ‟\textit{noun-oriented} collocations" on the noun, i.e., preferential combinations of a verb with a general meaning and a noun with a predicative value.}

\section{First research question: selecting analytical constructions or incorporations}\label{Section2}
We will consider the selection of the constructions παῖδας ποιοῦμαι \textit{paîdas poioûmai} `to beget children' and the equivalent incorporation\is{incorporation} as a case study to answer our first research question, i.e., what are the reasons for selecting either analytic constructions or synthetic verbs, like noun incorporations. In this section we are not specifically interested in the nature of the analytical constructions in question---i.e., whether they are SVCs or not---since the degree of predicativeness of the noun in the SVCs will be discussed below (\sectref{Section3/2}). However, some preliminary considerations can be made.

A battery of tests has been developed to recognise SVCs (\citealp{langer2004linguistic}, \textit{inter alia}). Of these, (i) the possibility of the SVC being replaced by a synthetic verb, see \REF{ex:4}, and (ii) the so-called ‟reduction test", see \REF{ex:5} (\citealt[39--43]{Gross1981}; \citealt[28]{giry1987predicats}), within a traditional perspective, are considered particularly significant in revealing the predicativeness of the noun, on the one hand, and the consequent emptiness of the SV, on the other: 


\begin{exe}
	\ex \label{ex:4}
	\begin{xlist}
		\ex \label{ex:4a} \textit{to give a slap $\sim$ to slap}
		\ex \label{ex:4b} \textit{to take a walk $\sim$ to walk}
	\end{xlist}
	\ex \label{ex:5}
	\begin{xlist}
		\ex \textit{John gave a slap to Mary $\rightarrow$ The slap that John gave to Mary $\rightarrow$ John's slap to Mary}
		\ex \textit{John took a walk $\rightarrow$ The walk that John took $\rightarrow$ John's walk}
	\end{xlist}
\end{exe}

The criterion of the substitution of an SVC by a synthetic verb, see \REF{ex:4}, is used to distinguish SVCs from other types of lexical combinations (e.g., ‟normal" collocations in which the verb retains its full lexical meaning). Indeed, it shows that the concept analytically conveyed is equivalent to that expressed through a single verbal form, usually in cases in which the synthetic verb and the noun are morphologically linked, as either the noun is deverbal (\textit{walk}) or the verb is denominal (\textit{slap}).\footnote{However, not all the unitary concepts present both forms of expression---analytical and synthetic---in all the languages \citep[192]{Jezek2004}. In English, for example, a synthetic form for \textit{to beget children} might be \textit{to procreate}, which is morphologically unrelated, or perhaps \textit{father}, which is related lexically, whereas in Italian \textit{fare figli} `to beget children' corresponds to the denominal verb \textit{figliare}, although this is mainly used in reference to animals (similar to the English \textit{to lamb} relating to sheep, \textit{to pup} to dogs, and so on).} On the other hand, \REF{ex:5} shows that the meaning of the noun does not seem to be affected by the deletion of the verb in SVCs. As for παῖδας ποιοῦμαι \textit{paîdas poioûmai} `to beget children'---with reference to its occurrence in \REF{ex:1}---we can observe that if Protomachus had children by the mother of Euxiteus, those children would actually be `Protomachus' children'.\footnote{Nevertheless, in this case it is not easy to establish if the reduction test actually applies, namely, if `Protomachus' children' derives a) from the sequence `the children that Protomachus begot' $\leftarrow$ `Protomachus begot children', or b) from the sequence `the children that Protomachus has' $\leftarrow$ `Protomachus has children', in addition to the possibility that c) the government of the argument `Protomachus' by `children' is simply due to the relational nature of kinship nouns. Note that the translation of παῖδας \textit{paîdas} `children' as `son' in \REF{ex:1}---which is commented upon here--- is how the item is rendered in the Loeb edition, even if the noun is plural in Greek.}

\subsection{Sub-corpus}\label{Section2/1}

The corpus for the case study on παῖδας ποιοῦμαι \textit{paîdas poioûmai} `to beget children' and παιδοποιοέω \textit{paidopoiéō} `to beget children' concerns the Classical period.\footnote{The corpus was created by \citep{Ricci2016} from the online edition of the \textit{Thesaurus Linguae Graecae} and it comprises all the authors from the Archaic period to the 4\textsuperscript{th} c. BC. However, no occurrences were found prior to the Classical period. The only possible exception is in \iwi{Septem Sapientes, \textit{Apophthegmata} 5.7 = Stobaeus, \textit{Flor}. LXVIII.34}, but since this instance is only documented by the indirect tradition in a fragment of Stobaeus, it was deemed more prudent to exclude it. Examples \REF{ex:1} and \REF{ex:2} are part of this corpus.} There are 10 occurrences of the analytic construction (\tabref{tab:A:occurrences-paîdas}), whilst there are 31 occurrences of incorporation\is{incorporation} (\tabref{tab:B:occurrences-paidopoiéō}): 

\begin{table}
	\caption{Occurrences of παῖδας ποιοῦμαι \textit{paîdas poioûmai} `to beget children'}
	\label{tab:A:occurrences-paîdas}
	\begin{tabular}{rrrrrr}
    \lsptoprule
		Isocrates & Xenophon & Plato & Demosthenes & Aristotle & \textbf{Total} \\
		2     & 1  & 4   & 2  & 1      & \textbf{10}  \\
    \lspbottomrule
    \end{tabular}
\end{table}

\begin{table}
    
	\caption{Occurrences of παιδοποιοέω \textit{paidopoiéō} `to beget children'}
	\label{tab:B:occurrences-paidopoiéō}
    \small
	\begin{tabular}{rrrrrr}
    \lsptoprule
		Euripides & Sophocles & Isocrates & Aristophanes & Andocides & Xenophon  \\
		2  & 1  & 1     & 1   & 1    & 9 \\
        Plato & Hippocrates & Demosthenes & Aeschines & & \textbf{Total} \\
        5   & 1   & 6  & 4 & & \textbf{31} \\
    \lspbottomrule
	\end{tabular}
\end{table}

%https://stephanus.tlg.uci.edu/lsj/05-general_abbreviations.html abbreviations LSJ

It is worth noting that all 10 occurrences of the analytical construction are in the middle-passive voice, and that the noun is always in the plural; only in one case does παῖδας \textit{paîdas} co-occur with the article, see (\ref{ex:11}) below.\footnote{In fact, the noun is singular in \iwi{Homer, \textit{Iliad} 9.495} although it is not an object so much as a predicative of the object. It has therefore not been included in the sample. The noun in the analytical form is always in the accusative, with the exception of one passage (\iwi{Isocrates, \textit{Speech} 4.42}), where the infinitive ποιήσασθαι \textit{poiḗsasthai} `to make' actually governs the pronominal forms τοὺς μὲν\ldots{} τοὺς δ' \textit{toùs mèn\ldots{} toùs d'} `some\ldots{} others', followed by the partitive τῶν παίδων \textit{tôn paídōn} `of the children'. On the preponderance of middle-passive forms in SVCs, see \citet{Marini2010} and \citet{Jimenez2011}.} The most frequent form is the infinitive (7 out of 10 occurrences; 70\%). On the other hand, out of 31 instances of incorporation\is{incorporation}, 26 (83.87\%) are in the middle-passive voice; 9 forms are participles (29.03\%), while 12 are infinitives (38.70\%).

\subsection{Semantic reasons}\label{Section2/2}
The findings of our sub-corpus show that the first reason for the selection of analytical constructions is semantic in nature. For instance, in \REF{ex:1} Euxiteus observes that his mother, before marrying his father, was married to Protomachus, who begot children with her, one of whom he gave in marriage. These children are thus Euxiteus' siblings and he is aware of their existence; hence, they are specific people. By specificity we mean the use of a Noun Phrase when the speaker knows which individual he is referring to (\citealt{Hawkins1978}; \citealt[259--261]{Lehmann1984}; \citealt[10]{vonHeusinger2002}; \citealt[45]{vonHeusinger2003}; \citealt[335--336]{Vester1989} on Latin). 


Therefore, in \REF{ex:1} the signifier παῖδας \textit{paîdas} `children' has a non-empty reference. Indeed, the logical value of existence in a possible world is linked to the notion of referentiality, which is equivalent to specificity from a semantic point of view \citep[293]{Givon1978}. By contrast, if a nominal is generic, the speaker does not have any commitment to the existence of its referent in a possible world. Instances of genericness are the cases of παιδο- \textit{paido-} as the first element of the incorporation\is{incorporation} παιδοποιέω \textit{paidopoiéō} `to beget children', in \REF{ex:2} and in all the other 30 occurrences in \tabref{tab:B:occurrences-paidopoiéō}. 


In fact, incorporated nouns are devoid not only of any determiner but also of the information conveyed by endings (i.e., number, grammatical gender, case), being downgraded to the root plus a readjustment vowel \citep{Pompei2006}: features such as gender, number, and definiteness are referential parameters \citep{vonHeusinger2003}. This lack of semantic referentiality---i.e., of specificity---in incorporated nouns is consistent with the main function of incorporation\is{incorporation} according to \citet{Mithun1984}, namely, to create ‟labels" to denote states of affairs that are conceptually unitary and worthy of being indicated by means of a single word. Therefore, the incorporated noun only serves to specify the meaning of the verb, i.e., to ‟qualify" the verb rather than to ‟refer" \citep[866]{Mithun1984}; it is not marked for referentiality/specificity \citep[859]{Mithun1984}.

However, in our corpus for this case study, the feature of specificity explains the selection of the analytical form in only two of the 10 occurrences (18\%), viz., the extract in (\ref{ex:1}), and in (\ref{ex:6}):

\ea \label{ex:6}
\glll [\ldots{}] τά 		τε 	ἄλλα 		καὶ 	\textbf{παῖδας}      	ἐν 	αὐτῇ \textbf{ἐποιήσω},		ὡς 	ἀρεσκούσης 		σοι 		τῆς 		πόλεως.\\
{} \textit{tá}		\textit{te} 	\textit{álla} 		\textit{kaì}	\textit{paîdas}	      	\textit{en}	\textit{autêi} \textit{epoiḗsō}		\textit{hōs}	\textit{areskoúsēs}		\textit{soi}		\textit{tês} 		\textit{póleōs}\\
{} \Art{}.\Acc{}.\N{}.\Pl{} and other.\Acc{}.\N{}.\Pl{} and child.\Acc{}.\M{}.\Pl{} in \Dem{}.\Dat{}.\F{} make.\Aor{}.\Mid{}.\Ssg{} as.if please.\Ptcp{}.\Gen{}.\F{} \Ssg{}.\Dat{} \Art{}.\Gen{}.\F{} city.\Gen{}.\F{}\\
\glt `[so you certainly preferred us and agreed to live in accordance with us;] and besides, you begat children in the city, showing that it pleased you' \\
\hspace*{\fill}(\iwi{Plato, \textit{Crito} 52c})
\z

In \REF{ex:6}, the subject of παῖδας ἐποιήσω \textit{paîdas epoiḗsō} `you begat children' is Socrates, who, condemned to die, is rebuked by Crito for accepting death rather than going into exile and saving his life. Socrates responds to Crito’s accusations with a prosopopoeia of the Laws: they (the Laws) address Socrates, reminding him of how he had agreed to live under those same Laws that have now condemned him to death, albeit having been raised and educated in Athens and also having fathered children there. Therefore, in this case the children are Socrates'. 

By contrast, in all the other occurrences, the noun of the analytical construction does not refer to specific entities. Indeed, it is always found in the plural, which is usually an indication of greater genericness \citep[225]{timberlake1975hierarchies}. This means that all the other occurrences of analytical constructions are not selected for semantic reasons. For instances, in \REF{ex:7} and \REF{ex:8} the noun παῖδας \textit{paîdas} `children' is clearly generic, as in these instances children do not exist at all, no act of generation having taken place:

\ea \label{ex:7}
\glll ἔτι 	  δὲ 	πρὸς 		τούτοις 	οὔτε 	γυναῖκα 	γήμας οὔτε	\textbf{παῖδας}		\textbf{ποιησάμενος} [\ldots{}]\\
\textit{éti}	  \textit{dè}	\textit{pròs}		\textit{toútois}		\textit{oúte}	\textit{gunaîka}	\textit{gḗmas} \textit{oúte}	\textit{paîdas}		\textit{poiēsámenos} [\ldots{}]\\
besides \textsc{prt} beyond \Dem{}.\Dat{}.\N{}.\Pl{} \Neg{} woman.\Acc{}.\F{} marry.\Aor{}.\Ptcp{}.\Nom{}.\M{} \Neg{} child.\Acc{}.\M{}.\Pl{} make.\Aor{}.\Ptcp{}.\Mid{}.\Nom{}.\M{} {}\\
\glt `Moreover, he did not marry and beget children' [\ldots{}] \\
\hspace*{\fill}(\iwi{Isocrates, \textit{Speech} 15.156.4})
\z

\ea \label{ex:8}
\glll σοῦ 	{δ',} 	ἔφη, 		ὦ 	Γαδάτα, 	ὁ 		Ἀσσύριος \textbf{παῖδας}		μέν, 	ὡς 	ἔοικε, 		\textbf{τὸ}		\textbf{ποιεῖσθαι} ἀφείλετο,		    οὐ	μέντοι 	   τό 		γε    φίλους	          δύνασθαι κτᾶσθαι 			ἀπεστέρησεν\\
\textit{soû}	\textit{d’}	\textit{éphē}		\textit{ô}	\textit{Gadáta}		\textit{ho}		\textit{Assúrios} \textit{paîdas}		\textit{mén} 	\textit{hōs}	\textit{éoike}		\textit{tò}		\textit{poieîsthai} \textit{apheíleto}		    \textit{ou}	\textit{méntoi}	   \textit{tó}		\textit{ge}    \textit{phílous}	          \textit{dúnasthai} \textit{ktâsthai} 			\textit{apestérēsen}\\
\Ssg{}.\Gen{} \textsc{prt} say.\Impf{}.\Tsg{} oh Gadatas.\Voc{}.\M{} \Art{}.\Nom{}.\M{} Assyrian.\Nom{}.\M{} child.\Acc{}.\M{}.\Pl{} \textsc{prt} as seem.\Prf{}.\Tsg{} \Art{}.\Acc{}.\N{} make.\Inf{}.\textsc{mid/pass} take.away.\Aor{}.\Mid{}.\Tsg{} \Neg{} at.any.rate \Art{}.\Acc{}.\N{} \textsc{prt}  friend.\Acc{}.\M{}.\Pl{} be.able.\Inf{}.\textsc{mid/pass} acquire.\Inf{}.\textsc{mid/pass} deprive.\Aor{}.\Tsg{}\\
\glt `‟From you, Gadatas,'' [Cyrus] went on, ``the Assyrian has, it seems, taken away the power of begetting children, but at any rate he has not deprived you of the ability of acquiring friends''' \\
\hspace*{\fill}(\iwi{Xenophon, \textit{Cyropedia} 5.3.19})
\z

To sum up, genericness is a compelling constraint for selecting instances of incorporation\is{incorporation}: specific nouns cannot be incorporated (see \REF{ex:1}) and \REF{ex:6}). When the conditions of use of παῖδας ποιοῦμαι \textit{paîdas poioûmai} `to beget children' are very similar to those of incorporation from a semantic point of view, as the noun is generic (see \REF{ex:7} and \REF{ex:8}), the reasons for the selection are not semantic (\sectref{Section2/3}).

\subsection{Textual reasons}\label{Section2/3}

When the conditions for the use of παῖδας ποιοῦμαι \textit{paîdas poioûmai} `to beget children' are not semantic in nature, they are textual. On this level of analysis, the meaning of the term \textit{referentiality} does not relate to the logical-semantic value of existence in a possible world, but to the establishment of a referent in the discourse, which may be a ‟manipulable noun" to use \citeauthor{HopperThompson1984}'s (\citeyear[711--713]{HopperThompson1984}) term. This means that the noun is a free form because it serves the text grounding. An interesting case is provided in \REF{ex:9}: 

\ea \label{ex:9}
\glll ἢ 	γὰρ 	οὐ 	χρὴ		\textbf{ποιεῖσθαι} 		\textbf{παῖδας} ἢ 	συνδιαταλαιπωρεῖν 	καὶ 	τρέφοντα 		καὶ 	παιδεύοντα.\\
\textit{ḕ}	\textit{gàr}	\textit{ou}	\textit{khrḕ}		\textit{poieîsthai}		\textit{paîdas} \textit{ḕ}	\textit{sundiatalaipōreîn}	\textit{kaì}	\textit{tréphonta}		\textit{kaì}	\textit{paideúonta}\\
either for \Neg{} ought.\Tsg{} make.\Inf{}.\textsc{mid/pass} child.\Acc{}.\M{}.\Pl{} or stay.by.\Inf{} and bring.up.\Ptcp{}.\Acc{}.\M{} and educate.\Ptcp{}.\Acc{}.\M{}\\
\glt `Either one ought not to beget children, or one ought to stay by them and bring them up and educate them' \\
\hspace*{\fill}(\iwi{Plato, \textit{Crito} 45d})
\z

In this case, the conditions of use of ποιεῖσθαι παῖδας \textit{poieîsthai paîdas} `to beget children' are really very similar to those of incorporation\is{incorporation} from a semantic point of view as the noun is generic. However, from the perspective of text grounding, it is necessary for παῖδας \textit{paîdas} `children' to be a free form in order to be taken up in the reference tracking, and in particular by the null argument of the verbs that follow, i.e., by zero anaphora. Conversely, incorporated nouns do not usually constitute the starting point for reference tracking: being decategorised, they are non-prototypical nouns, whence they do not introduce participants into the discourse, like all nouns that are not the head of a compound.\footnote{In fact, this is true for noun incorporation originating from lexical compounds---as occurs with Ancient Greek incorporation (\sectref{Section1})--- according to the recent classification proposed by \citet{Olthof2020}. She deals with a sample of 21 languages, taking into account the two parameters of the modifiability and referentiality of the incorporated noun. The latter parameter is defined in pragmatic terms within the \textit{Functional Discourse Grammar} framework \citep{HengeveldMackenzie2008}; the former is not pertinent to Ancient Greek, in which incorporated nouns cannot be modified. Cf. \citet{Pompei2022}  on the application of \citeauthor{Olthof2020}'s (\citeyear{Olthof2020}) model to Ancient Greek.}

In an anaphoric chain, reference tracking might take place through different strategies, such as pronouns (including null ones, as in \REF{ex:9}), copies or semi-copies of the head lexeme, paradigmatic relations, and so on. In \REF{ex:1}, for instance, there is a paradigmatic relation of hyponymy between θυγατέρα \textit{thugatéra} `daughter' and παῖδας \textit{paîdas} `children'. This means that textual reasons also apply when semantic reasons are present. 

When there are no reference tracking reasons, the selection of the analytical construction is, in any case, usually due to the need for παῖδας \textit{paîdas} `children' to occur as a free form to establish a referent---i.e., a Topic---in the discourse, perhaps as an element of a conjunct, see \REF{ex:6}, which may also be negative, see \REF{ex:7}, or of a correlation with a contrastive value, see \REF{ex:8}. Since in all these cases there is the establishment of a Topic, textual reasons might also be considered as due to Information Structure, sometimes not disjunct from stylistic requirements.\largerpage[1.5]\footnote{The notion of Topic concerns the Information Structure, an area of linguistics studied in particular by the \textit{Prague School}. The Topic is usually intended as the item that the sentence is about, as opposed to the Focus, which can be considered the information given about the Topic (\textit{inter alia} \citealt{Lambrecht1994}). In addition to the introduction of a new referent (new Topic), the Topic can also recall a referent already present in the text (Topic continuity; cf. \citealt{Givón1983}), and have a constrastive function (contrastive Topic; cf. \citealt{Buring1999}). As far as stylistic requirements are concerned, correlations in conjunction, see \REF{ex:6}, negative conjunction, see \REF{ex:7}, or opposition, see \REF{ex:8}, are, in a sense, also examples of isocolia. In \REF{ex:7}, for instance, there is a parallelism between οὔτε παῖδας ποιησάμενος \textit{oúte paîdas poiēsámenos} `not having begotten children' and οὔτε γυναῖκα γήμας \textit{oúte gunaîka gḗmas} `not having married' (a collocation for `taking a wife'). Similarly, in \REF{ex:1}, there is a parallelism between the analytic construction παῖδας ποιησάμενος \textit{paîdas poiēsámenos} `having begotten children' and the SV λαβών \textit{labṓn} `having taken (as a wife)', in addition to the hyperonymy relation regarding θυγατέρ' \textit{thugatér'} `daughter'. Thus, several textual reasons for selecting the analytical construction may be involved.} 


In \REF{ex:10}, for instance, there is a parallelism between ὅτι πλείστους ποιεῖσθαι παῖδας \textit{hóti pleístous poieîsthai paîdas} `have as many children as possible' and ὡς πλείστους εἶναι τοὺς Σπαρτιάτας \textit{hōs pleístous eînai toùs Spartiátas} `make the Spartiates as numerous as possible', in addition to the fact that ποιεῖσθαι παῖδας \textit{poieîsthai paîdas} `have children' constitutes a case of Topic continuity: in this instance, it conveys given semi-active information, in \citeauthor{Chafe1987}'s (\citeyear{Chafe1987}) terms---of which the Topic is the linguistic correlate---since the increase in the number of Spartiates implies the increase in births: 

\ea \label{ex:10}
\glll βουλόμενος 			γὰρ 	ὁ 		νομοθέτης ὡς 	πλείστους 			εἶναι 	τοὺς 		Σπαρτιάτας, προάγεται 			τοὺς         	πολίτας	ὅτι 	 πλείστους \textbf{ποιεῖσθαι} 		\textbf{παῖδας}·\\
\textit{boulómenos}			\textit{gàr}	\textit{ho}		\textit{nomothétēs} \textit{hōs}	\textit{pleístous}			\textit{eînai}	\textit{toùs}		\textit{Spartiátas}  \textit{proágetai}			\textit{toùs}	         	\textit{polítas}		\textit{hóti}	 \textit{pleístous} \textit{poieîsthai} 		\textit{paîdas}\\
desire.\Ptcp{}.\textsc{mid/pass}.\Nom{}.\M{} for \Art{}.\Nom{}.\M{} lawgiver.\Nom{}.\M{} as numerous.\Sup{}.\Acc{}.\M{}.\Pl{} be.\Inf{} \Art{}.\Acc{}.\M{}.\Pl{} Spartiates.\Acc{}.\M{}.\Pl{} induce.\Tsg{}.\textsc{mid/pass} \Art{}.\Acc{}.\M{}.\Pl{} citizen.\Acc{}.\M{}.\Pl{} as numerous.\Sup{}.\Acc{}.\M{}.\Pl{} make.\Inf{}.\textsc{mid/pass} child.\Acc{}.\M{}.\Pl{}\\
\glt `For the lawgiver desiring to make the Spartiates as numerous as possible holds out inducements to the citizens to have as many children as possible' \\
\hspace*{\fill}(\iwi{Aristotle, \textit{Politics} 1270b})
\z

Eventually, a case of Topic continuity is also quoted in \REF{ex:11}; this is the only case in which the noun παῖδας \textit{paîdas} `children' is definite:\footnote{Definiteness may be regarded as a property whereby the discourse referent can be identified with another, previously introduced, discourse item (\citealt[44--45]{vonHeusinger2003}, \textit{inter alia}). In this case, τοὺς παῖδας \textit{toùs paîdas} `the children' recalls the phrase παίδων γένεσιν \textit{paídōn génesin} `production of children' in \iwi{Plato, \textit{Laws} 783b}; therefore, it probably answers the need to re-establish the referent after many lines. }

\ea \label{ex:11}
\glll Καλῶς. 	ἔλθωμεν 		δ’ 	ἐπὶ 	τὰ 		νυμφικά, διδάξοντές 			τε 	αὐτοὺς 	πῶς 	χρὴ καὶ 	τίνα		τρόπον		\textbf{τοὺς}		\textbf{παῖδας} 	\textbf{ποιεῖσθαι} \\
\textit{Kalôs}		\textit{élthōmen}		\textit{d’}	\textit{epì}	\textit{tà}		\textit{vumphiká} \textit{didáksontés}			\textit{te}	\textit{autoùs}		\textit{pôs}	\textit{khrḕ} \textit{kaì}	\textit{tína}		\textit{trópon}		\textit{toùs}		\textit{paîdas} 		\textit{poieîsthai}\\
well come.\Aor{}.\Sbjv{}.\Fpl{} \textsc{prt} to \Art{}.\Acc{}.\N{}.\Pl{} nuptial.\Acc{}.\N{}.\Pl{} instruct.\Fut{}.\Ptcp{}.\Nom{}.\M{}.\Pl{} and \Dem{}.\Acc{}.\M{}.\Pl{} how ought.\Tsg{} and \textsc{q}.\Acc{}.\M{} manner.\Acc{}.\M{} \Art{}.\Acc{}.\M{}.\Pl{} child.\Acc{}.\M{}.\Pl{} make.\Inf{}.\textsc{mid/pass}\\
\\
\glt `Very good. Let us now come to the nuptials, so as to instruct them how and in what manner they ought to produce children' \\
\hspace*{\fill}(\iwi{Plato, \textit{Laws} 783d})
\z

Nevertheless, it is very difficult to gauge the reasons for the selection of the analytical construction in an instance such as παῖδας ποιεῖσθαι \textit{paîdas poieîsthai} `to beget children' in \REF{ex:12}:\footnote{Loeb's translation---which we follow (cf. fn. 1)---is a little perplexing here; one reviewer suggested `nor should anyone---man or woman---do so by night, when\ldots{}'.}

\ea \label{ex:12}
\glll μηδ’	αὖ	νύκτωρ	ὅταν		ἐπινοῇ		  τις		\textbf{παῖδας} \textbf{ποιεῖσθαι} 		ἀνὴρ		ἢ	καὶ	γυνή. \\
\textit{mēd’}	\textit{aû}	\textit{núktōr}		\textit{hótan}		\textit{epinoêi}		  \textit{tis}		\textit{paîdas} \textit{poieîsthai} 		\textit{anḕr}		\textit{ḕ}	\textit{kaì}	\textit{gunḗ}\\
\Neg{} so at.night when think.\Sbjv{}.\Tsg{} \Indf{}.\Nom{} child.\Acc{}.\M{}.\Pl{} make.\Inf{}.\textsc{mid/pass} man.\Nom{}.\M{} or also woman.\Nom{}.\F{}\\
\glt `[nor should anyone whatever taste of it at all, except for reasons of bodily training or health, in the daytime;] nor should anyone do so by night – be he man or woman – when proposing to procreate children' \\
\hspace*{\fill}(\iwi{Plato, \textit{Laws} 674b})
\z

In this passage, the circumstances in which it is forbidden to drink wine are listed. The choice of the analytical form might be due to the fact that the incorporation\is{incorporation} is generally used with regard to men, while in this instance the prohibition to drink wine in case of procreation is valid for men and women. Alternatively, the very co-occurrence of παῖδας \textit{paîdas} `children' with `man' and `woman' might have played a role in the choice of the free form, this being a sort of third element, i.e., a possible result of their union. Finally, the author's \textit{usus scribendi} should perhaps also be considered, since 4 of the 10 analytical forms (40\%) appear in Plato vs. 5 of the 31 instances of incorporation (16.13\%) do.\footnote{For the sake of comprehensiveness, in one of the two occurrences that have not been analysed in the text (\iwi{Demosthenes, \textit{Speech} 45.81}), παῖδας \textit{paîdas} `children' as a free form is due to the need to establish an object taken up by an object predicative (`after being allowed to beget children as brothers to your own masters'). In the other instance (\iwi{Isocrates, \textit{Speech} 4.42}), the occurrence of the noun is a free form because it is in the genitive case, having a partitive value with regard to the pronominal forms τοὺς μὲν\ldots{} τοὺς δ’ \textit{toùs mèn\ldots{} toùs d’} `some\ldots{} others' (see fn. 9).}

To sum up, regarding the first research question, the selection of an analytical construction is usually made for textual reasons, namely, the need to establish a referent in the discourse, which might possibly be ‟manipulable" in \citeauthor{HopperThompson1984}'s (\citeyear[711--713]{HopperThompson1984}) terms. By contrast, incorporated nouns do not perform such a function in Ancient Greek. In Information Structure terms, the occurrence of παῖδας \textit{paîdas} `children' as a free form usually has the function of (re-)establishing the Topic. The requirement of referentiality in discourse terms also applies in cases of the specificity of the noun; in other words, referentiality at the textual level can combine with referentiality at the logical-semantic one.

\section{Second research question: the nature of analytical constructions equivalent to incorporation}\label{Section3}

In order to establish the reasons for the selection of either analytical or synthetic constructions, our second research question is twofold: (i) to verify whether analytic constructions are always SVCs or not, and (ii), in the latter case, to clarify the differences between types, particularly in terms of the possible equivalence with instances of incorporation.\is{incorporation} 

The answer to the first part of the question is clear: analytical constructions are not always SVCs. Even if we only take into account the analytical constructions with ποιέω \textit{poiéō} `to do, make'---the focus of this article---in many of them the verb does not co-occur with predicative nouns (\sectref{Section3/2/1}).

As for the second part of the research question, when the verb ποιέω \textit{poiéō} `to do, make' co-occurs with predicative nouns, we need to examine the meaning of predicativeness in relation to a noun (\sectref{Section3/2}). This leads to interesting findings: nouns that occur in analytic constructions usually considered SVCs do not belong to the same type. Indeed, it is possible to identify two different cases: (i) nouns that acquire a full predicative value in co-occurrence with an SV (simple-event nominals\is{simple-event nominal}), and (ii) nouns that fully inherit the event structure of the verb from which they derive (complex-event nominals\is{complex-event nominal}) (\sectref{Section3/2/2}). Only the former type has equivalent instances of incorporation. A third type of noun comprises non-eventive nouns\is{non-eventive noun} that can sometimes acquire an eventive interpretation (\sectref{Section3/2/1}).

\subsection{Corpus and methodology}\label{Section3/1}
The data considered in this second part of the study were taken from the main corpus (described in \sectref{Section1}).\footnote{The \textit{Thesaurus Linguae Graecae} query also covered the Archaic period, although no occurrences of the forms in question were found. For this reason we consider our corpus as starting from the 5\textsuperscript{th} c. BC.} As for the methodology, firstly, the reverse dictionary of Ancient Greek by \citet{KretschmerLocker1977} was used to draw up the list of instances of incorporation\is{incorporation} in -ποιέω \textit{-poiéō} `to do, make'.


The instances of incorporation were then searched for in the \textit{Thesaurus Linguae Graecae} to find their occurrences, which amount to 74 in the period considered.\footnote{The quantitative data presented in this section are the results of an initial survey (the study of the data is part of a doctoral thesis in progress). On a morphological basis, in addition to instances of noun incorporation, 27 instances of incorporation with an adjective as the first element were also identified (e.g., ἁγιοποιέω \textit{hagiopoiéō} `to sanctify' and ἀγαθοποιέω \textit{agathopoiéō} `to do good, make good, do well') making for a total of 101 incorporations.} Subsequently, instances of noun incorporation were divided into two groups on a semantic basis, namely, instances of non-eventive\is{non-eventive noun} noun incorporation (58) and instances of eventive noun incorporation (16). Successively, the \textit{Thesaurus Linguae Graecae} was queried in order to identify the equivalent analytical constructions.

\subsection{Support-verb constructions, incorporation, and the predicativeness of nouns}\label{Section3/2}
Predicative nouns that occur in SVCs are not only and not always deverbal nouns. A seminal study on this topic was made by \citet{GrossKiefer1995}. In addition to nominalisations, i.e., deverbal nouns, \citeauthor{GrossKiefer1995} identify two further types of predicative non-deverbal nouns: those with the event reading in their lexical representation (e.g., French \textit{orage} `storm', \textit{coup} `blow', \textit{épidemie} `epidemic'), and those whose event interpretation is due to a conceptual shift to a dynamic reading (e.g., when \textit{film} stands for `the screening of the film'). Indeed, \citet[141]{Vendler1967} had noted that among nouns there are what he calls \textit{disguised nominals}: ‟Fires and blizzards, unlike tables, crystals or cows, can occur, begin and end, can be sudden or prolonged, can be watched and observed---they are, in a word, events and not objects". From the actional point of view, the fact that the referents of \textit{disguised nominals} ‟can occur" means that they are [+dynamic], i.e., events, as opposed to states; conversely, the fact that they can ‟begin and end" means that they have the feature [+durative]. 
 
\citet[58--59]{Grimshaw1990} defines non-deverbal nouns (e.g., \textit{race}, \textit{trip}, and \textit{exam}) as \textit{simple-event nominals}\is{simple-event nominal}. They differ from complex-event nominals\is{complex-event nominal}---i.e., nominalisations, which inherit the argument and event structure from the verb from which they derive---since the former cannot co-occur with the modifiers that are used to detect telicity (‟in-x-time") and atelicity (‟for-x-time": e.g., \textit{*Jack's trip in five hours / for five hours was interesting}), as opposed to the latter (see, e.g., the \textit{nomen actionis construction} in \textit{Caesar’s construction of the bridge in five months}). 


According to \citet[59]{Grimshaw1990}, this means that what characterises complex-event nominals\is{complex-event nominal} ‟is not a matter of temporal extent, but of an internal semantic analysis of the event provided by the event structures [\ldots{}]".\footnote{Indeed, \citet{Grimshaw1990} simple-event nominals\is{simple-event nominal} correspond to \citeauthor{Vendler1967}'s (\citeyear{Vendler1967}) disguised nominals: they can co-occur with ‟happening" verbs (e.g., \textit{The race will take place tomorrow}), with phasal verbs (e.g., \textit{The trip started badly}), and with prepositions having a similar function (e.g., \textit{during lunch}). On noun actionality, see also \citet{Simone2003}, and recently \citet{PompeiTorino}.} It is noteworthy that \citet[56]{Borer2013} observes that ‟‟simple" events are fully compatible, syntactically, with ‟complex" events, insofar as arguments and event modification are possible providing a light verb is present". Moreover, \citet[50--59]{Grimshaw1990} notes that simple-event nominals\is{simple-event nominal} behave like result nominals (see, e.g., the \textit{nomen rei construction} in \textit{*That construction in five months / for five months is horrible}) and she considers both as noun-like, unlike complex-event nominals\is{complex-event nominal}, which are verb-like. 

All these observations on the eventive nature and the degree of predicativeness of nouns are highly relevant in understanding their occurrence within SVCs and incorporation.\is{incorporation} Indeed, from the perspective of SVCs as complex predicates\is{complex predicate}, the semantic contribution of the verb is not null (which is in contrast to how it is considered in the binary conception of predicative noun vs. ‟light" verb/‟support" verb (\sectref{Section1})). Indeed, the contributions of the noun and the verb to predicativeness can be considered complementary and, in a sense, inversely proportional, on a continuum.


In the following sections, an attempt will be made to position the various analytical constructions (both effective SVCs (\sectref{Section3/2/2}) and not (\sectref{Section3/2/1})) and their possible equivalent instances of incorporation on this continuum\is{incorporation}, according to the different noun types (\sectref{Section4}).

\subsubsection{Analytical constructions and incorporation with non-eventive nouns}\label{Section3/2/1}

Non-eventive nouns\is{non-eventive noun} are mostly concrete nouns, which denote first-order entities in \citeauthor{Lyons1977}'s (\citeyear[443]{Lyons1977}) terms, namely, they do not have any degree of predicativeness.\footnote{In fact, besides instances in which the non-eventive noun is actually concrete (e.g., ἀνδριαντο- \textit{andrianto-} `statue', γεφυρο- \textit{gephuro-} `bridge', λυχνο- \textit{lukhno-} `lamp', οἰνο- \textit{oino-} `wine'), there are others in which it is abstract, albeit non-eventive (e.g., μελο- \textit{melo-} `lyric poem', θεσμο- \textit{thesmo-} `law', ὀνοματο- \textit{onomato-} `name'). Concrete nouns that can also acquire an eventive value---e.g., σῖτος \textit{sîtos} `grain, meal' ((\ref{ex:17}) below)---have been classified for now according to their basic concrete semantic value. From the perspective of the syntactic function that the incorporated noun would have in the equivalent analytical construction, in many cases it is that of the object predicative, exclusively (e.g., θεοποιέω \textit{theopoiéō} `deify') or in addition to that of the object (e.g., ἀρτο- \textit{arto-} `cake, loaf, bread'; cf. fn. 21).} We can exemplify this type firstly by means of the noun ἄρτος \textit{ártos} `cake, loaf of wheat-bread, bread', which is present in 16 analytical constructions (\tabref{tab:C:occurrences-arton-poieo}) and 9 instances of incorporation (\tabref{tab:D:occurrences-artopoioumai}):\footnote{Out of 16 occurrences of analytical construction, 15 have the verb in the active voice and 1 has it in the middle-passive voice; the noun is a plural accusative in 11 occurrences and a singular accusative in the remaining 5 occurrences; only in 2 instances does the plural ἄρτους \textit{ártous} co-occur with the article. As far as the 9 instances of incorporation are concerned, 6 are in the middle-passive voice, while 3 are in the passive voice. Two attestations of ἄρτον ποιεῖω \textit{árton poieîō} in Clemens Romanus---but more likely Pseudo-Clemens---(\iwi{Clemens Romanus, \textit{Homiliae} 2.32.3}, \iwi{Pseudo-Clemens, \textit{Epitome de gestis Petri} 33}) have been excluded from the count because of their uncertain attribution and dating.}

%table C
\begin{table}
	\caption{Occurrences of ἄρτον ποιέω \textit{árton poiéō} `to make bread'}
	\label{tab:C:occurrences-arton-poieo}
	\begin{tabular}{rrrr}
    \lsptoprule
		Herodotus & Xenophon & Hippocrates & Theophrastus \\
		2    & 1  & 4   & 2      \\
        Septuagint (LXX) & Josephus & Plutarch & \textbf{Total} \\
        5   & 1  & 1    & \textbf{16} \\
    \lspbottomrule
	\end{tabular}
\end{table}

%table D
\begin{table}
	\caption{Occurrences of ἀρτοποιοῦμαι \textit{artopoioûmai} `to make bread'}
	\label{tab:D:occurrences-artopoioumai}
	\begin{tabular}{rrrr}
    \lsptoprule
		Strabo & Josephus & Dioscorides Medicus & \textbf{Total} \\
		2    & 1  & 6    & \textbf{9}   \\
    \lspbottomrule
	\end{tabular}
\end{table}

An example of an analytical construction is given in \REF{ex:13} and one of incorporation in \REF{ex:14}:

\ea \label{ex:13}
\glll ἀρτοφαγέουσι		δὲ	ἐκ	τῶν		ὀλυρέων	         \textbf{ποιεῦντες} \textbf{ἄρτους},	τοὺς		ἐκεῖνοι 		κυλλήστις		ὀνομάζουσι.\\
\textit{artophagéousi}		\textit{dè}	\textit{ek}	\textit{tôn} 		\textit{oluréōn}	         \textit{poieûntes} \textit{ártous}		\textit{toùs}		\textit{ekeînoi}			\textit{kullḗstis}		\textit{onomázousi}\\
eat.bread.\Tpl{}		\textsc{prt}	from	\Art{}.\Gen{}.\F{}.\Pl{}	coarse.grain.\Gen{}.\F{}.\Pl{}    make.\Ptcp{}.\Nom{}.\M{}.\Pl{} loaf.\Acc{}.\M{}.\Pl{} 	\Rel{}.\Acc{}.\M{}.\Pl{}	\Dem{}.\Nom{}.\M{}.\Pl{}	cyllestis.\Acc{}.\F{}.\Pl{}	call.\Tpl{}\\
\glt `They eat bread, making loaves which they call ``cyllestis'' of coarse grain' \\
\hspace*{\fill}(\iwi{Herodotus, \textit{Histories} 2.77.4})
\z

\ea \label{ex:14}
\glll εἶτ’	\textbf{ἀρτοποιοῦνται}	σίτου			μικρὰ		καταμίξαντες·\\
\textit{eît’}	\textit{artopoioûntai}		\textit{sítou}			\textit{mikrà}		\textit{katamíksantes}\\
then	make.cake.\textsc{mid/pass}.\Tpl{}	grain.flour.\Gen{}.\M{}	a.bit.of 	mix.\Aor{}.\Ptcp{}.\Nom{}.\M{}.\Pl{} \\
\glt `[The vertebral bones serve as mortars in which fish, which have been previously dried in the sun, are pounded.] Of this, with the addition of flour, cakes are made' \\
\hspace*{\fill}(\iwi{Strabo, \textit{Geography} 15.2.2})
\z

In these occurrences, the meaning of the verb ποιέω \textit{poiéō} `to do, make' is its full lexical value, namely, `to create, realise'. This means that the verb is not an SV in this case, and the analytical constructions are not SVCs. 


As far as semantic roles are concerned, the basic meaning of the verb implies an Agent and an incremental Theme denoting the entry of a new entity into the state of existence and its development at all the stages of production, as in the case of `to make loaves' and `to make cakes' with dried fish by the Ichthyophagi in \REF{ex:13} and \REF{ex:14}, respectively.\footnote{This value is identified perfectly by \iwi{Plato, \textit{Symposium} 205b}, where the ποίησις \textit{poíēsis} `creation, production' is described as the cause of anything that passes from not being into being; we thank Adele Teresa Cozzoli for this suggestion. In truth, the incorporation\is{incorporation} often has the value of `to make [something] into bread' (e.g., `acorn flour' in \iwi{Strabo, \textit{Geography} 3.3.7}), where the incorporated noun is equivalent to the predicative object of the analytical form rather than its object.} In this case too, noun concreteness and genericness being equal, the choice of the analytical construction in \REF{ex:13} is for textual reasons, i.e., the requirement of a head noun for the relative pronoun, i.e., of a referent for the reference tracking.

An apparently similar case is the co-occurrence of the concrete noun σῖτος \textit{sîtos} `grain, food, allowance of grain' with the verb both in analytical constructions (\tabref{tab:E:occurrences-siton-poieo}), exemplified in \xxref{ex:15}{ex:16}, and in instances of incorporation (\tabref{tab:F:occurrences-sitopoieo}), exemplified in \xref{ex:17}:\footnote{As for the analytical constructions, 4 verbs out of 5 are active and 1 is middle-passive. The noun is always singular: in the accusative in 3 occurrences, in the genitive in 2 (once with the article) given that the object of the verb is actually the quantity of the bread (cf. \ref{ex:15}). There are 3 occurrences of incorporation in the active voice and 3 in the middle-passive.}


%table E
\begin{table}
	\caption{Occurrences of σῖτον ποιέω \textit{síton poiéō} `to make grain, bread, food'}
	\label{tab:E:occurrences-siton-poieo}
	\begin{tabular}{rrrrr}
    \lsptoprule
		Xenophon & Plato & Demosthenes & Aristotle & \textbf{Total} \\
		1  & 1   & 2  & 1      & \textbf{5}   \\
    \lspbottomrule
	\end{tabular}
\end{table}

%table F
\begin{table}
	\caption{Occurrences of σιτοποιέω \textit{sitopoiéō} `to make bread, food, a meal'}
	\label{tab:F:occurrences-sitopoieo}
	\begin{tabular}{rrrr}
    \lsptoprule
		Euripides & Xenophon & Dioscorides Medicus & \textbf{Total} \\
		1  & 3  & 2    & \textbf{6} \\
    \lspbottomrule
	\end{tabular}
\end{table}

\ea \label{ex:15}
\glll πλουτεῖς	εἰκότως,	ἐπειδὰν	\textbf{ποιῇς}			\textbf{σίτου}		μὲν μεδίμνους		πλέον		ἢ	χιλίους\\
\textit{plouteîs}	\textit{eikótōs}		\textit{epeidàn}	\textit{poiêis}			\textit{sítou}		\textit{mèn} \textit{medímnous}		\textit{pléon}		\textit{ḕ}	\textit{khilíous}\\
be.rich.\Ssg{}	naturally	as		make.\Sbjv{}.\Ssg{}	grain.\Gen{}.\M{}	\textsc{prt} medimnus.\Acc{}.\M{}.\Pl{}	more		than	thousand.\Acc{}.\M{}.\Pl{}\\
\glt `you [\ldots{}] are a rich man, naturally, for you make more than a thousand medimni of grain' \\
\hspace*{\fill}(\iwi{Demosthenes, \textit{Speech} 42.31})
\z

\ea \label{ex:16}
\glll [\ldots{}] ἄλλο            τι         ἢ          \textbf{σῖτόν}            τε         \textbf{ποιοῦντες}               καὶ         οἶνον καὶ       ἱμάτια                      καὶ           ὑποδήματα;\\
{} \textit{állo}              \textit{ti}         \textit{ḕ}          \textit{sîtón}             \textit{te}         \textit{poioûntes}                \textit{kai}         \textit{oînon} \textit{kaì}       \textit{himátia}                    \textit{kaì}           \textit{hupodḗmata}\\
{} other.\Acc{}.\N{}    thing.\Acc{}.\N{}    or       bread.\Acc{}.\M{}     and     make.\Ptcp{}.\Nom{}.\M{}.\Pl{}      and    wine.\Acc{}.\M{}  and      garments.\Acc{}.\N{}.\Pl{}        and             shoes.\Acc{}.\N{}.\Pl{}\\
\glt `Will they not make bread and wine and garments and shoes?' \\
\hspace*{\fill}(\iwi{Plato, \textit{Republic} 372a})
\z

\ea \label{ex:17}
\glll \textbf{σιτοποιεῖσθαί}		τε	γὰρ	 ἀνάγκη		ἀμφοτέρους,	κοιμᾶσθαί τε	ἀνάγκη		ἀμφοτέρους \\
\textit{sitopoieîsthai}		\textit{te}	\textit{gàr}	 \textit{anánkē}		\textit{amfotérous}	\textit{koimâsthai} \textit{te}	\textit{anánkē}			\textit{amfotérous}\\
meal.make.\Inf{}.\textsc{mid/pass}	and	for	 necessity.\Nom.\F{}	both.\Acc{}.\M{}.\Pl{}	sleep.\Inf{}.\textsc{mid/pass} and	necessity.\Nom{}.\F{}	both.\Acc{}.\M{}.\Pl{}\\
\glt `for instance, you must both eat, and you must both sleep' \\
\hspace*{\fill}(\iwi{Xenophon, \textit{Cyropedia} 1.6.36}).
\z

Interestingly, also in this case the analytical structure allows the noun to occur with a concrete, specific, and definite value, see \xref{ex:16}, although here the frequent metonymy of the substance (`grain') for the product (`bread') applies. The concrete meaning of `grain' is retained in \xref{ex:15}. Whereas in \xref{ex:16} the value of the verb is the full lexical one, in \xref{ex:15} it is to some extent bleached, as it means to `to harvest, put together' a quantity of cereal; this would also be the case with `grain' as an object. This value can perhaps be called effective, in the sense that an effect is produced, even if not through a process of concrete and direct realisation of an incremental Theme.\footnote{See \citet[140]{Pompei2023} on the use of this label with reference to the Italian \textit{fare rumore} `to make noise'. However, in this case there is the production of a state of affairs, unlike the instance in \xref{ex:15}.} 


In both examples the analytical constructions are not SVCs. As far as incorporation\is{incorporation} is concerned, the noun can also acquire a dynamic reading via an abstraction process, as happens in \xref{ex:17}, through the metalepsis `meal' < `food' < `bread' < `grain'. Indeed, according to \citet{GrossKiefer1995}, this would be a case of event interpretation due to a conceptual shift (\sectref{Section3/2}). In this instance, the value of the verb is completely bleached, and it only retains its event structure of a process in accordance with a noun that has acquired an event reading; nevertheless, this is an incorporation and not an analytical construction. It is therefore not an SVC.
% added to force footnotes 23-25 onto the correct page.
\interfootnotelinepenalty=10000
\largerpage[1.5]

If we now reconsider the analytical construction παῖδας ποιεῖσθαι \textit{paîdas poieî\-sthai} `beget children' (\sectref{Section2}), the noun in this case denotes a concrete entity of the first order in \citeauthor{Lyons1977}'s terms (\citeyear[443]{Lyons1977}). The verb clearly means `to create', although it only denotes the moment of the generation, or of the birth (in the case of the value `to bear children' for the woman), rather than all the development stages of an incremental Theme. From this perspective, we cannot consider this analytical construction an SVC, since, on the one hand, the verb is not lexically empty, and on the other, the noun is not eventive. In this sense, this analytical construction cannot be considered as a ‟noun-oriented" collocation (\sectref{Section1}, fn. 5). Also in this case, an effective value of the verb may be involved, as an effect is, in fact, produced.\footnote{Therefore, the result of the reduction test seems to be due to the relational nature of the noun. As for the possible equivalence with synthetic verbs, this is consistent with the equivalence with an incorporation (which denotes a conceptually unitary state of affairs).\is{incorporation}}

\subsubsection{Analytical constructions and incorporation with event nouns}\label{Section3/2/2}
The co-occurrence of the verb ποιέω \textit{poiéō} `to do, make' with event nouns is the instance in which instances of incorporation\is{incorporation} and SVCs overlap perfectly.\footnote{In truth, the list of eventive nouns currently includes some stative nouns (e.g., ἐλπιδο- \textit{elpido-} `hope', νοσο- \textit{noso-} `sickness'), that are non-eventive by definition\is{non-eventive noun}, not being dynamic (\sectref{Section3/2}), although they are durative like eventive nouns. In these cases, the verb always has a causative value.} However, simple-event nominals\is{simple-event nominal} in \citeauthor{Grimshaw1990}'s (\citeyear{Grimshaw1990}) terms (\sectref{Section3/2}) need to be distinguished from complex-event ones.

To exemplify simple-event nouns, we can consider the noun ἄριστον \textit{áriston} `(morning) meal, breakfast, lunch'; this contains the event reading in its lexical structure, which is not the case with σῖτος \textit{sîtos} `grain, food, allowance of grain', see \xref{ex:17}. Its occurrences in analytical constructions and instances of incorporation are listed in \tabref{tab:G:occurrences-ariston-poieo} and \tabref{tab:H:occurrences-aristopoioumai}, respectively.\footnote{In this case, the number of instances of incorporation (43) is far greater than the number of analytical constructions (6). In the latter, 5 out of 6 verbs are in the middle-passive with the meaning of `to have breakfast / lunch'; the only occurrence in the active (\iwi{NT Luke 14.12}) means `to make lunch' for guests. On the other hand, all 43 instances of incorporation are in the middle-passive voice and mean `to have breakfast / lunch' or `to make breakfast / lunch' for themselves. The noun is always in the accusative singular and only once co-occurs with the article.}
\interfootnotelinepenalty=100
%resetting footnote penalty to avoid strange behaviour later in the chapter.

\begin{table}[h]
	\caption{Occurrences of ἄριστον ποιέω \textit{áriston poiéō} `to make / have breakfast, lunch'}
	\label{tab:G:occurrences-ariston-poieo}
	\begin{tabular}{rrr}
    \lsptoprule
		Thucydides & Herodotus & Xenophon \\
		1   & 2    & 1   \\
        Hippocrates & New Testament (NT) & \textbf{Total} \\
        1   & 1   & \textbf{6} \\
    \lspbottomrule
	\end{tabular}
\end{table}

\begin{table}[h]
	\caption{Occurrences of ἀριστοποιοῦμαι \textit{aristopoioûmai} `to make / have breakfast, lunch'}
	\label{tab:H:occurrences-aristopoioumai}
	\begin{tabular}{rrrr}
    \lsptoprule
		Thucydides & Xenophon & Demosthenes & Polybios  \\
		6   & 17 & 2  & 5 \\
        Diodorus Siculus & Philo & Aristonicus & Josephus  \\
        1& 1   & 2          & 2  \\
        \multicolumn{2}{l}{Onosander (Onasander) Tacticus} & Plutarch & \textbf{Total} \\
        \multicolumn{2}{l}{3}    & 4 & \textbf{43} \\
    \lspbottomrule
	\end{tabular}
\end{table}

The structures are exemplified in \xref{ex:18} and \xref{ex:19}:


\ea \label{ex:18}
\glll ἱκανὸς		γάρ     ἐστι        καὶ	νυκτὶ 		ὅσαπερ	ἡμέρᾳ		χρῆσθαι, καὶ	ὅταν	σπεύδῃ,	\textbf{ἄριστον}	 καὶ 	δεῖπνον	\textbf{ποιησάμενος} ἅμα 		πονεῖσθαι.\\
\textit{hikanòs}	\textit{gár}      \textit{esti}         \textit{kaì}	\textit{nuktì} 		\textit{hósaper}	\textit{hēmérai}	\textit{chrêsthai} \textit{kaì}	\textit{hótan}	\textit{speúdēi},	\textit{áriston}		 \textit{kaì}	\textit{deîpnon}	\textit{poiēsámenos} \textit{háma}		\textit{poneîsthai}.\\
able.\Nom{}.\M{}.\Sg{}	for      be.\Tsg{}    and	night.\Dat{}.\F{}	as		day.\Dat{}.\F{}	use.\Inf{}.\textsc{mid/pass} and	when	hasten.\Sbjv{}.\Tsg{}	breakfast.\Acc{}.\N{}	 and	dinner.\Acc{}.\N{}	make.\Aor{}.\Ptcp{}.\Mid{}.\Nom{}.\M{}.\Sg{} together 	labour.\Inf{}.\textsc{mid/pass}\\
\glt `For he is able to make as good use of night as of day, and when he is in haste, to take breakfast and dinner together and go on with his labours' \\
\hspace*{\fill}(\iwi{Xenophon, \textit{Hellenica} 6.1.15})
\z

\ea \label{ex:19}
\glll ταῦτα		ποιήσαντες			\textbf{ἠριστοποιοῦντο}.\\
\textit{taûta}		\textit{poiḗsantes}			\textit{ēristopoioûnto}\\
\Dem{}.\Acc{}.\N{}.\Pl{} make.\Aor{}.\Ptcp{}.\Nom{}.\M{}.\Pl{} make.breakfast.\Impf{}.\textsc{mid/pass}.\Tpl{}\\
\glt `[When they had done all this,] they set about preparing breakfast' \\
\hspace*{\fill}(\iwi{Xenophon, \textit{Anabasis} 3.3.1})
\z

This noun has the same meaning both when incorporated, see \xref{ex:18}, and when occurring independently, see \xref{ex:19}, in an SVC. In \xref{ex:18}, we find the only independent occurrence of ἄριστον \textit{áriston} `breakfast' in Xenophon (vs. 17 instances of incorporation); this seems to be due to its coordination with δεῖπνον \textit{deîpnon} `dinner'.

Another interesting case of a simple-event noun is πόλεμος \textit{pólemos} `war, battle' which has been formally linked to πελεμίζω \textit{pelemízō} `to shake, tremble' (\citealt{BeekesBeek2010}: s.v. πόλεμος), but certainly cannot be considered a deverbal noun. \tabref{tab:I:occurrences-polemon-poieo} presents the occurrences in analytical constructions for the period under consideration. 


\begin{table}[htb!]
	\caption{Occurrences of πόλεμον ποιέω \textit{pólemon poiéō} `to provoke, make war'}
	\label{tab:I:occurrences-polemon-poieo}
	\begin{tabular}{rrrrr}
    \lsptoprule
		Thucydides  & Isocrates & Andocides & Xenophon   & Plato \\
		16   & 5     & 2    & 5    & 3 \\
        Septuagint (LXX) & D.   & Aeschines     & Polybios & Lysias  \\
        18    & 18   & 1            & 2        & 1  \\
		Diodorus Siculus & \multicolumn{2}{l}{Dionysius Halicarnassensis}  & Philo  & Strabo \\ 
		3    & \multicolumn{2}{l}{1} & 2    & 2   \\
        New Testament (NT) & Josephus   & Plutarch & \multicolumn{2}{l}{\textbf{Total}} \\
        4   & 1    & 3    & \multicolumn{2}{l}{\textbf{87}} \\
    \lspbottomrule
	\end{tabular}
\end{table}


\tabref{tab:J:occurrences-polemopoieo} presents the instances of incorporation.\is{incorporation} 


\begin{table}[htb!]
	\caption{Occurrences of πολεμοποιέω \textit{polemopoiéō} `to provoke, make war'}
	\label{tab:J:occurrences-polemopoieo}
	\begin{tabular}{rrrrrr}
    \lsptoprule
		Xenophon & Hippocrates & Diodorus Siculus & Philo & Plutarch & \textbf{Total} \\
		1  & 1   & 1    & 3   & 1    & \textbf{7}   \\
    \lspbottomrule
	\end{tabular}
\end{table}


They are exemplified in \xref{ex:20} and \xref{ex:21}, respectively:\footnote{Of the 87 occurrences of the analytical construction, 51 have an active verb, with a causative value, whereas 36 have a middle-passive verb, meaning `to make war' (on this cf. \citealt{JiménezLópez2012,Jimenez2016}). The noun is usually singular (82 instances, of which 35 co-occur with the article) with the exception of 5 occurrences (of which 3 co-occur with the article). By contrast, all the instances of incorporation\is{incorporation} are active forms, having both the meaning of `to make war' and `to provoke war'. Two attestations of the analytical construction have been excluded from the count, \iwi{\textit{Oracula Sibyllina} 1.9} and \iwi{\textit{Testamenta XII Patriarcharum} 7.5.10}, owing to their uncertain dating.}

\ea \label{ex:20}
\glll καὶ	τῇ		πόλει		ὠφελιμώτερον		ἔφη          εἶναι       πρὸς τοὺς 		ἐν	τῇ		χώρᾳ		σφῶν            ἐπιτειχίζοντας         \textbf{τὸν} \textbf{πόλεμον} 		\textbf{ποιεῖσθαι}                 ἢ            Συρακοσίους\\
\textit{kaì}	\textit{têi}		\textit{pólei}		\textit{ōphelimṓteron}		\textit{éphē}         \textit{eînai}       \textit{pròs} \textit{toùs} 		\textit{en}	\textit{têi}		\textit{chṓrai}		\textit{sphôn}           \textit{epiteichízontas}        \textit{tòn} \textit{pólemon}		\textit{poieîsthai}                  \textit{è}            \textit{Surakosíous}\\
and	\Art{}.\Dat{}.\F{}.\Sg{s}	city.\Dat{}.\F{}	profitable.\Compv{}.\Acc{}.\N{}	\Impf.\Tsg{}   be.\Inf{}       against \Art{}.\Acc{}.\M{}.\Pl{}	in	\Art{}.\Dat{}.\F{}	country.\Dat{}.\F{}	\Tpl{}.\Gen{}               fortify.\Ptcp{}.\Acc{}.\M{}.\Pl{}    \Art{}.\Acc{}.\M{} war.\Acc{}.\M{}		make.\Inf{}.\textsc{mid/pass}           than        Syracusan.\Acc{}.\M{}.\Pl{}\\
\glt `He also said that it would be more profitable for the state to carry on the war against those who       were building fortifications in Attica, than against the Syracusans' \\
\hspace*{\fill}(\iwi{Thucydides, \textit{Histories} 7.47.4})
\z

\ea \label{ex:21}
\glll εἴτε	προφάσει	χρώμενοι		ταύτῃ		τοῦ		ταράττειν	καὶ \textbf{πολεμοποιεῖν}.\\
\textit{eíte}	\textit{prophásei}	\textit{chrṓmenoi}		\textit{taútēi}		\textit{toû} 		\textit{taráttein}	\textit{kaì} \textit{polemopoieîn}\\
either	pretext.\Dat{}.\F{}	use.\Ptcp{}.\textsc{mid/pass}.\Nom{}.\M{}.\Pl{}	\Dem{}.\Dat{}.\F{}	\Art{}.\Gen.\N{}	disturb.\Inf{}	and make.war.\Inf{}\\
\glt `[It is uncertain whether\ldots{}] they used this pretext for raising disturbance and war' \\
\hspace*{\fill}(\iwi{Plutarch, \textit{Life of Otho} 3.2})
\z


In this case, the choice of SVCs is often due to the fact that the war is a specific and definite one, as in \xref{ex:20}. Moreover, instances of incorporation appear later, probably because of competition with the denominative verbs πολεμέω \textit{poleméō} `to battle, fight a war' and πολεμίζω \textit{polemízō} `to fight'.

It is also noteworthy that μάχη \textit{mákhē} `battle, combat' only occurs in SVCs as the equivalent incorporation does not exist.\footnote{On SVCs with ποιέω \textit{poiéō} `to do, make' and πόλεμος \textit{pólemos} `war, battle' or μάχη \textit{mákhē} `battle, combat', see \citet{JiménezLópez2012,Jimenez2016} and \citet{Banos2015}.} From our perspective, this is due to the fact that this is a complex-event nominal\is{complex-event nominal} relating to the verb μάχομαι \textit{mákhomai} `to fight'. In \citeauthor{Grimshaw1990}'s (\citeyear{Grimshaw1990}) terms, this means that the predication of `fighting' is denoted by the noun alone, which fully inherits the argument and event structures of the verb (\sectref{Section3/2}). Of course, it is possible that the incorporation did not develop precisely because of the existence of this verb, although it is interesting that it did develop in the case of πόλεμος \textit{pólemos} `war, battle', despite other existing verbal forms. Moreover, the same is true of all the other deverbal nouns (such as πλόος \textit{plóos} `navigation', φυλακή \textit{phulakhḗ} `watching, guarding', and so on). The alternation between SVCs with deverbal nouns and the synthetic verb from which they derive follows semantic and textual principles \citep{Tambasco2021} similar to those that we have seen for the selection of SVCs equivalent to instances of incorporation.\is{incorporation}

\section{Conclusions}\label{Section4}

In this chapter, a comparison between analytical constructions and instances of incorporation with ποιέω \textit{poiéō} `to do, make' has been made with a twofold aim: (a) to identify the reasons for selecting either analytical constructions or synthetic verbs, and (b) to verify whether analytic predicates are always SVCs.

The answer to the first research question is that the selection of analytical constructions is mainly due to textual reasons, i.e., the establishment of the referent in the discourse, which also has some consequences on the Information Structure, particularly on Topic (re-)establishment (\sectref{Section2/3}); secondarily, semantic reasons such as specificity can play a role (\sectref{Section2/2}).

As for the second research question, it is clear that only analytical constructions with eventive nouns can be considered SVCs (\sectref{Section3}). These fall into two types, namely, simple-event nominals\is{simple-event nominal}, and complex-event ones (\sectref{Section3/2}). The comparison with instances of incorporation\is{incorporation} can be made only when the eventive noun in the SVC is a simple-event nominal (\sectref{Section3/2/2}), in addition to cases of analytical constructions where ποιέω \textit{poiéō} `to do, make' co-occurs with non-eventive nouns\is{non-eventive noun} (\sectref{Section3/2/1}). Incorporated simple-event nominals are nouns with the event reading in their lexical representation (e.g., ἄριστον \textit{áriston} ‘(morning) meal, breakfast, lunch', πόλεμος \textit{pólemos} ‘war, battle'); besides, other nouns may acquire an event interpretation thanks to a conceptual shift to a dynamic reading (e.g., σῖτος \textit{sîtos} ‘grain, food, allowance of grain'). By contrast, analytical constructions made up of complex-event nominals\is{complex-event nominal} do not alternate with instances of incorporation, but only with the verb from which the noun derives.

These findings are illustrated in \figref{figtab:incorporations}.

% \vspace*{\fill} % % uncomment if you'd like to centre the figure
\begin{figure}
	\caption{The noun predicativeness---verb lightness continuum}
	\label{figtab:incorporations}
	\fittable{
	\begin{NiceTabular}{llllll}
		&                      & & Incorporation \\
		\\
		& Fully lexical ποιέω \textit{poiéō} &  & \Block{1-2}{Emptier ποιέω \textit{poiéō}} \\
		& Non-eventive nominals & & Simple-event nominals & Complex-event nominals  &  \\
		& (ἄρτος \textit{ártos}---παῖς \textit{paîs} &   & (ἄριστον \textit{áriston}---πόλεμος \textit{pólemos})  & (μάχη \textit{mákhē}---πλόος \textit{plóos} &  \\
        & ---σῖτος \textit{sîtos}) & & & ---φυλακή \textit{phulakhḗ}) & \\ 
        \\
		&                      &                  & \Block{1-2}{Support-verb constructions} \\
		\\
		&                      &             \Block{1-2}{Analytical verb constructions}  
%		&                      &             & \begin{tabular}[c]{@{}l@{}}Analytical verb \\ constructions\end{tabular}  
		\CodeAfter
		\begin{tikzpicture} [-LaTeX]
			\draw [<->] (2.5-|1) -- (2.5-|5) ;
			\draw [<->] (7.5-|3) -- (7.5-|6) ;
			\draw [<->] (9.5-|2) -- (9.5-|7) ;
		\end{tikzpicture}
    
	\end{NiceTabular}
	}
\end{figure}

%\section*{Abbreviations}
%\printglossaries
%\printglosses[title={Abbreviations}]

\section*{Abbreviations}
\begin{tabularx}{.5\textwidth}{@{}lQ@{}}
\textsc{compv} & comparative \\
NT & New Testament \\
\textsc{sup} & superlative \\
\end{tabularx}

\section*{Acknowledgements}
This study is a result of research conducted within the framework of the PRIN (Progetti di Rilevante Interesse Nazionale) ``VerbACxSS: on analytic verbs, complexity, synthetic verbs, and simplification. For accessibility'' (2020, Prot. 2020\-BJKB9M) funded by the Italian Ministero dell’Università e della Ricerca. 

\section*{Contributions}
The text is the result of cooperation between the three authors. Nevertheless, for academic purposes, Flavia Pompeo is responsible for Section 1, Anna Pompei for Sections 2, 3, 3.2, 3.2.1, 3.2.2, and Eleonora Ricci for Sections 2.1, 2.2, 2.3, 3.1. The final section (Section 4) is to be ascribed jointly to Anna Pompei and Flavia Pompeo. We thank Victoria Fendel and two anonymous reviewers for their useful comments and suggestions; any remaining errors are our own.

\sloppy
\printbibliography[heading=subbibliography,notkeyword=this]
\end{document}
