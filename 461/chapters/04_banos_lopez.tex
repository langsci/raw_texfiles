\documentclass[output=paper,colorlinks,citecolor=brown]{langscibook}
\ChapterDOI{10.5281/zenodo.14017927}

\author{José Miguel Baños\affiliation{Universidad Complutense         de Madrid} and         María Dolores Jiménez López\affiliation{Universidad de           Alcalá de Henares}}

\title[Support-verb constructions in the Gospels]{Support-verb constructions in the Gospels: A comparative study between Greek and Latin}


\abstract{In this article we analyse the data on the frequency of support-verb   constructions (SVCs) in the Gospels, both in their original Greek version and in the Latin   translation of the Vulgate. In the former case,   we identify the most frequent support verbs and highlight the   differences among the gospel writers. These differences also speak of their varying proficiency in Greek and are sometimes the   result of linguistic influences. The parallel analysis of the Latin text of the Vulgate allows us to   compare the use of SVCs in both languages and reflect on the translation criteria   employed. The evidence, in addition to   highlighting the reasonable tension between the literal translation of the source   language (Greek) and the naturalness of the target language (Latin), demonstrates the   existence of different translation criteria in each Gospel.
  \bigskip


\foreignlanguage{spanish}{En este trabajo analizamos los datos sobre la
    frecuencia de las construcciones con verbo soporte (CVS) en los evangelios, tanto en su versión original en griego como en la
    traducción latina de la Vulgata. Mostramos en el primer caso cuáles son los verbos
    soporte más frecuentes, así como las diferencias entre
    los evangelistas. Estas diferencias nos
    hablan también de su distinta competencia en la lengua griega y son resultado a veces
    de interferencias lingüísticas. El análisis paralelo del texto latino de la Vulgata permite
    comparar el uso de las CVS en ambas lenguas y reflexionar sobre los criterios de
    traducción empleados. Los
    datos estudiados, además de reflejar la lógica tensión entre la traducción literal de
    la lengua de partida (el griego) y la naturalidad de la lengua de llegada (el latín),
    revelan criterios de traducción distintos en cada evangelio.}}

    

\IfFileExists{../localcommands.tex}{
   \addbibresource{../localbibliography.bib}
   \usepackage{langsci-optional}
\usepackage{langsci-gb4e}
\usepackage{langsci-lgr}

\usepackage{listings}
\lstset{basicstyle=\ttfamily,tabsize=2,breaklines=true}

%added by author
% \usepackage{tipa}
\usepackage{multirow}
\graphicspath{{figures/}}
\usepackage{langsci-branding}

   
\newcommand{\sent}{\enumsentence}
\newcommand{\sents}{\eenumsentence}
\let\citeasnoun\citet

\renewcommand{\lsCoverTitleFont}[1]{\sffamily\addfontfeatures{Scale=MatchUppercase}\fontsize{44pt}{16mm}\selectfont #1}
  
   %% hyphenation points for line breaks
%% Normally, automatic hyphenation in LaTeX is very good
%% If a word is mis-hyphenated, add it to this file
%%
%% add information to TeX file before \begin{document} with:
%% %% hyphenation points for line breaks
%% Normally, automatic hyphenation in LaTeX is very good
%% If a word is mis-hyphenated, add it to this file
%%
%% add information to TeX file before \begin{document} with:
%% %% hyphenation points for line breaks
%% Normally, automatic hyphenation in LaTeX is very good
%% If a word is mis-hyphenated, add it to this file
%%
%% add information to TeX file before \begin{document} with:
%% \include{localhyphenation}
\hyphenation{
affri-ca-te
affri-ca-tes
an-no-tated
com-ple-ments
com-po-si-tio-na-li-ty
non-com-po-si-tio-na-li-ty
Gon-zá-lez
out-side
Ri-chárd
se-man-tics
STREU-SLE
Tie-de-mann
}
\hyphenation{
affri-ca-te
affri-ca-tes
an-no-tated
com-ple-ments
com-po-si-tio-na-li-ty
non-com-po-si-tio-na-li-ty
Gon-zá-lez
out-side
Ri-chárd
se-man-tics
STREU-SLE
Tie-de-mann
}
\hyphenation{
affri-ca-te
affri-ca-tes
an-no-tated
com-ple-ments
com-po-si-tio-na-li-ty
non-com-po-si-tio-na-li-ty
Gon-zá-lez
out-side
Ri-chárd
se-man-tics
STREU-SLE
Tie-de-mann
}
   \boolfalse{bookcompile}
   \togglepaper[4]%%chapternumber
}{}



\begin{document}
\emergencystretch 3em
\maketitle

\section{Introduction}\label{sec:bj:1}

\is{Vulgate}

As part of a much broader study on the whole of the New Testament, this chapter
aims to present and analyse general data on the use of support-verb constructions (SVCs)
in the Gospels, both in the original Greek version and the Latin translation of the
Vulgate.\footnote{The dataset is accessible here: \url{https://doi.org/10.21950/E98VTJ}. The Greek and Latin texts are aligned for examples from the synoptic gospels such that the gloss applies to both.}

The structure is as follows: firstly (Section~\ref{sec:bj:2}), we will define the concept of
support-verb construction used in the collection of the data and identify the main
support verbs in Greek. Next (Section~\ref{sec:bj:3}), we will examine the frequency of SVCs in
the four Gospels in the original Greek version, paying particular attention to the
internal differences among the gospel writers. Finally (Section~\ref{sec:bj:4}), we will focus
on the analysis of the Vulgate, highlighting different degrees of literalness in the Latin
translation of the Greek SVCs, which we will illustrate primarily through collocations
containing the nouns συμβούλιον \emph{symboúlion} `counsel' and χρείαν \emph{chreían}
`need'. By way of summary (Section~\ref{sec:bj:5}), we will present the main
conclusions of the article and indicate some avenues for research.

In order to facilitate the comparison between the Greek texts and their Latin version, in each example we have tried to align word for word. Obviously, alignment has not always been possible: sometimes the word order does not match in both languages, as in (\ref{ex:bj:6a}), or a Greek synthetic predicate (e.g., in (\ref{ex:bj:2b}) ἐράπισαν \textit{erápisan} 'strike' is translated into Latin by an analytic predicate (\textit{palmas in faciem ei dederunt}).

\is{collocation}

\section{The concept of support-verb construction}\label{sec:bj:2}

The term support-verb construction (SVC henceforth) is employed in this study to refer to a type of
complex predicate formed by a verb and a predicative or eventive noun with its own
argument structure. The noun serves as the base that selectively chooses the support
verb(s) with which it combines, supplying the relevant semantic content and, consequently,
determining the semantic functions of the participants in the construction. The verb, on
the other hand, provides the grammatical categories (person, number, tense, mood, voice)
and the syntactic positions into which the participants of the event are inserted.

This framework allows us to approach SVCs broadly. Thus, we consider prototypical SVCs,
i.e. those collocations in which (i)~the verb has a general or vague meaning (light
verbs), (ii)~its subject is co-referential with the first semantic argument of the noun,
and often (iii)~equivalent to a synthetic predicate (cf.~\cite{Langer2004},
\cite{JiménezLópez2016}), as illustrated by examples \xxref{ex:bj:1a}{ex:bj:1b} and
\xxref{ex:bj:2a}{ex:bj:2b}.\footnote{We follow the edition of
  \citet{Nestle-AlandNestleAland2012} for the Greek text of the Gospels. The Latin text of
  the Vulgate follows the edition of \citet{WeberGryson2007}. 
  %whose abbreviations of the biblical texts are also used here. 
  The English translations are taken from
  \citet{EnglishStandardVersion2007}.}

\is{collocation}

\ea\label{ex:bj:1}

\ea\label{ex:bj:1a}

\gllll πᾶς ὁ \textbf{ποιῶν} \textbf{τὴν} \textbf{ἁμαρτίαν} δοῦλός ἐστιν {[}τῆς ἁμαρτίας{]}\\
 \textit{pâs} \textit{ho} \textit{poiôn} \textit{tḕn} \textit{hamartían} \textit{doûlos} \textit{estin} \textit{\emph{[}tês} \textit{hamartías\emph{]}}\\
everyone the practice the sin slave be {[}the sin{]}\\
\emph{qui} ~ \textbf{\itshape facit} ~ \textbf{\itshape peccatum} \emph{servus} \emph{est} ~ \emph{peccati}\\
\glt `everyone who practices sin is a slave to sin.' \\
\hspace*{\fill}(\iwi{NT John 8.34}) \\

\ex\label{ex:bj:1b}

\gllll ῥαββί, τίς \textbf{ἥμαρτεν}\ldots;\\
 \textit{rabbí,} \textit{tís} \textit{hḗmarten\ldots?}\\
Rabbi who sin\\
\emph{Rabbi}, \emph{quis} \textbf{\itshape peccavit}\ldots?\\
\glt `Rabbi, who \emph{sinned}\ldots?' \\
\hspace*{\fill}(\iwi{NT John 9.2}) \\

\z

\z

\ea\label{ex:bj:2}

\ea\label{ex:bj:2a}

\gllll καὶ \textbf{ἐδίδοσαν} αὐτῷ \textbf{ῥαπίσματα}\\
 \textit{kaì} \textit{edídosan} \textit{autôi} \textit{rapísmata}\\
and give him slaps\\
\emph{et} \textbf{\itshape dabant} \emph{ei} \textbf{\itshape alapas}\\
\glt `and struck him with their hands.' \\
\hspace*{\fill}(\iwi{NT John 19.3}) \\

\ex\label{ex:bj:2b}

\gllll ἐκολάφισαν αὐτόν, οἱ δὲ \textbf{ἐράπισαν}\\
 \textit{ekoláphisan} \textit{autón,} \textit{hoi} \textit{dè} \textit{erápisan}\\
buffet him these and strike\\
\emph{colaphis} {\emph{eum} \emph{ceciderunt}}, \emph{alii} \emph{autem} {\textbf{\itshape
    palmas} \emph{in} \emph{faciem} \emph{ei} \textbf{\itshape dederunt}}\\
\glt `they struck him. And some slapped him.' \\
\hspace*{\fill}(\iwi{NT Matthew 26.67}) \\

\z

\z

However, we also consider collocations in which the verb, possessing a fuller meaning,
contributes diathetic values  --- causative\is{diathesis (causative)}, passive\is{passive}, see \xref{ex:bj:3a} ---, aspectual
 --- inchoative\is{inchoative}, see \xref{ex:bj:3b}, terminative\is{terminative}, durative\is{durative} ---  or even intensive\is{intensive}, see
\xref{ex:bj:3c},\footnote{\citet[34]{Gross1998} introduces the concept of intensive
  variants of support verbs to refer to collocations such as \emph{jump for joy} (`to be
  very happy'), \emph{burn with desire} (`to desire very much') or, as in \xref{ex:bj:3c},
  \emph{fill with fear} (`to be very afraid'). In these, the verb semantically expresses
  an intensification of the event or experience denoted by the noun of the collocation.}
among others.

\is{collocation}

\ea\label{ex:bj:3}

\ea\label{ex:bj:3a}

\gllll καὶ ἐν σαββάτῳ \textbf{περιτέμνετε} ἄνθρωπον. εἰ \textbf{περιτομὴν}
\textbf{λαμβάνει} ἄνθρωπος ἐν σαββάτῳ\ldots{}\\
 \textit{kaì} \textit{en} \textit{Sabbátōi} \textit{peritémnete} \textit{ánthrōpon.} \textit{ei} \textit{peritomḕn} \textit{lambánei} \textit{ánthrōpos} \textit{en} \textit{Sabbátōi}\\
and on Sabbath circumcise man. if circumcision receive man on Sabbath\ldots{}\\
~ \emph{in} \emph{sabbato} \textbf{\itshape circumciditis} \emph{hominem}. \emph{Si}
\textbf{\itshape circumcisionem} \textbf{\itshape accipit} \emph{homo} \emph{in} \emph{sabbato}\ldots{}\\
\glt `you circumcise a man on the Sabbath. If on the Sabbath a man receives circumcision\ldots' \\
\hspace*{\fill}(\iwi{NT John 7.22-23})

\ex\label{ex:bj:3b}

\gllll καὶ ἐταράχθη Ζαχαρίας ἰδὼν καὶ \textbf{φόβος} \textbf{ἐπέπεσεν} ἐπ' αὐτόν\\
 \textit{kaì} \textit{etaráchthē} \textit{Zacharías} \textit{idṑn} \textit{kaì} \textit{phóbos} \textit{epépesen} \textit{ep'} \textit{autón}\\
and trouble Zechariah see and fear fall upon him\\
\emph{et} \emph{Zaccharias} {\emph{turbatus} \emph{est}} \emph{videns} \emph{et}
\textbf{\itshape timor} \textbf{\itshape inruit} \textit{super} \textit{eum}.\\
\glt `And Zechariah was troubled when he saw him, and fear fell upon him' \\
\hspace*{\fill}(\iwi{NT Luke 1.12}).

\ex\label{ex:bj:3c}

\gllll καὶ \textbf{ἔκστασις} \textbf{ἔλαβεν} ἅπαντας καὶ ἐδόξαζον τὸν θεὸν καì \textbf{ἐπλήσθησαν} \textbf{φόβου}\\
 \textit{kaì} \textit{éktasis} \textit{élaben} \textit{hápantas} \textit{kaì} \textit{edóxazon} \textit{tòn} \textit{theòn} \textit{kaì} \textit{eplḗsthēsan} \textit{phóbou}\\
and amazement take all and glorify the God and fill fear\\
\emph{et} \textbf{\itshape stupor} \textbf{\itshape adprehendit} \emph{omnes} \emph{et} \emph{magnificabant} ~  \emph{Deum} \emph{et} \textbf{\itshape {repleti} \textbf{\itshape
  sunt}} \textbf{\itshape timore}\\
\glt `And amazement seized them all, and they glorified God and were filled with awe.' \\
\hspace*{\fill}(\iwi{NT Luke 5.26})

\z

\z

In most SVCs the predicative or eventive noun is the direct object of the collocative verb, see
(\ref{ex:bj:1a}, \ref{ex:bj:2a}, \ref{ex:bj:3a}). However, this is not the only possible
syntactic construction. In our corpus, we also consider SVCs, such as φόβος ἐπέπεσεν
\emph{phóbos epépesen} in \xref{ex:bj:3b} and ἔκστασις ἔλαβεν \emph{ékstasis élaben}
in \xref{ex:bj:3c}, in which the noun is the subject. These collocations present the event from a
perspective which cannot be expressed by the corresponding synthetic predicate
 --- φοβεῖσθαι \emph{phobeîsthai} `to be afraid', ἐξιστάναι \emph{existánai} `to be
astonished' ---, since in these SVCs the subject is not the Experiencer but the eventive
noun itself
\parencites{BenedettiMarina-2010117,BenedettiMarina-2013672,Tur2019,JiménezLópez2024}.

\is{collocation}
\is{collocative verb}

In sum, the concept of SVC as employed in this study encompasses not only support verbs in a
narrow sense but also the so-called support-verb extensions\footnote{In previous studies
  \parencites{BañosJoséMiguel-2015206,BañosJiménezLópez2017a,BañosJiménezLópez2017b,BañosJoséMiguelJiménezLópezM.Dolores-2022707,JiménezLópezM.Dolores-2018285},
  the term \emph{verb-noun collocation} is used in the same sense. A list of different
  designations can be found in \citet[25--28]{HoffmannRoland-202223} and the state of the
  field in \citet{PompeiPiunnointro2023}.} (cf. \cite{Gross1981}, \cite{Vivès1983},
\cite{Cicalese1999}, \cite{JežekElisabetta-2004615}), as well as converse constructions
(\cite{Gross1989}, \cite{Mendózar2020}). This broad approach is, in our view, necessary,
as it allows the description of the full collocational pattern of a predicative noun and
of the motivations underlying the selection of the verbs with which it combines.

\subsection{The most frequent support verbs in Greek}\label{sec:bj:2:1}

Since it is not possible to present here a full list of the support verbs we have
considered, \tabref{tbl:bj:gospels} includes, as part of the results of our study, the
six most frequently used verbs in the Gospels. These represent approximately two-thirds of
both the total number of instances examined (521) and of the number of distinct SVCs (231)
in which they appear: ποιεῖν \emph{poieîn} `to do', γίγνεσθαι \emph{gígnesthai} `to
happen', εἶναι \emph{eînai} `to be', διδόναι \emph{didónai} `to give', ἔχειν
\emph{échein} `to have', and λαμβάνειν \emph{lambánein} `to take'.

\begin{table}
\caption{Support verbs in the Gospels}
  \label{tbl:bj:gospels}
\centering
\begin{tabularx}{.8\textwidth}{lrr}
\lsptoprule
SVs & n\textsuperscript{o} of distinct SVCs & Total n\textsuperscript{o}\\
& & of instances \\
\midrule
ἔχειν \emph{échein} & 26 & 83\\
ποιεῖν \emph{poieîn} & 26 & 75\\
διδόναι \emph{didónai} & 30 & 70\\
γίγνεσθαι \emph{gígnesthai} & 34 & 58\\
εἶναι \emph{eînai} & 23 & 44\\
λαμβάνειν \emph{lambánein} & 13 & 22\\
Total for the 6 verbs & 152 (65.80\%) & 352 (67.56\%)\\
Other verbs & 79 (34.20\%) & 169 (32.44\%)\\
Total & 231 & 521\\
\lspbottomrule
\end{tabularx}
\end{table}

The most frequent of these verbs is ἔχειν \emph{échein} (83 instances), due to the
frequency of certain SVCs --- χρείαν \emph{chreían} `need' (20 instances), ζωήν
\emph{zōḗn} `life' (15 instances), ἐξουσίαν \emph{exousían} `power, authority' (13
instances) ---, followed by ποιεῖν \emph{poieîn} (75 instances)  --- ἔργον \emph{érgon}
`deed' (15 instances), σημεῖον \emph{sēmeîon} `sign' (15 instances). Additionally,
γίγνεσθαι \emph{gígnesthai} (with 34 distinct SVCs) and διδόναι \emph{didónai} (with
30 distinct SVCs) exhibit the greatest variety of different SVCs.

These data are consistent with the fact that the same predicative noun may often select
several of these verbs as part of its collocational pattern to present the event from
different perspectives. Let us consider some representative examples.

Starting with the verb `to do', one of the support verbs \textit{par excellence} in many languages,
it is important to differentiate in classical Greek between ποιεῖσθαι \emph{poieîsthai}
`to do' in the middle voice, which behaves as a prototypical support verb in the narrowest
sense, and ποιεῖν \emph{poieîn} in the active voice, which is generally a causative
extension (\cite{JiménezLópez2012}). Although this distinction persists in the Gospels, as
shown by \xref{ex:bj:4a} and \xref{ex:bj:4b}, the active voice is often used in the New
Testament as a general support verb instead of the middle voice, as demonstrated in
\xref{ex:bj:1a} above (\cite[103--113]{JiménezLópezM.Dolores-2018285}). Other collocative uses
of ποιεῖν \emph{poieîn} in the active voice are those in which this verb denotes
accomplishment or fulfillment of an action, as in \xref{ex:bj:4c}.


\ea\label{ex:bj:4}

\ea\label{ex:bj:4a}

\gllll οἱ μαθηταὶ Ἰωάννου νηστεύουσιν πυκνὰ καὶ \textbf{δεήσεις} \textbf{ποιοῦνται}\\
 \textit{hoi} \textit{mathētaì} \textit{Iōánnou} \textit{nēsteúousin} \textit{pyknà} \textit{kaì} \textit{deḗseis} \textit{poioûntai}\\
the disciples John fast often and prayers do\\
~ \emph{discipuli} \emph{Iohannis} \emph{ieiunant} \emph{frequenter} \emph{et}
\textbf{\itshape obsecrationes} \textbf{\itshape faciunt}\\
\glt `The disciples of John fast often and offer prayers.' \\
\hspace*{\fill}(\iwi{NT Luke 5.33})

\ex\label{ex:bj:4b}

\gllll Ἡρῴδης τοῖς γενεσίοις αὐτοῦ \textbf{δεῖπνον} \textbf{ἐποίησεν} τοῖς μεγιστᾶσιν αὐτοῦ\\
 \textit{Hērṓidēs} \textit{toîs} \textit{genesíois} \textit{autoû} \textit{deipnon} \textit{epoíēsen} \textit{toîs} \textit{megistâsin} \textit{autoû}\\
Herod the birthday him banquet {bring about} the nobles his\\
\emph{Herodes} ~ \emph{natalis} \emph{sui} \textbf{\itshape cenam} \textbf{\itshape fecit}
~ \textit{principibus}\\
\glt `Herod on his birthday \textbf{gave a banquet} for his nobles.' \\
\hspace*{\fill}(\iwi{NT Mark 6.21})

\ex\label{ex:bj:4c}

\gllll ὃς {[}γὰρ{]} ἂν \textbf{ποιήσῃ} \textbf{τὸ} \textbf{θέλημα} τοῦ θεοῦ, οὗτος ἀδελφός μου\ldots{} ἐστίν\\
 \textit{hòs} \textit{{[}gàr{]}} \textit{àn} \textit{poiḗsēi} \textit{tò} \textit{thélēma} \textit{toû} \textit{theoû} \textit{hoûtos} \textit{adelphós} \textit{mou\ldots{}} \textit{estín}\\
who {[}for{]} \textsc{prt} do the will the God this brother my\ldots{} be\\
\emph{qui} \emph{enim} ~ \textbf{\itshape fecerit} ~ \textbf{\itshape voluntatem} ~ \emph{Dei} \emph{hic} \emph{frater} \emph{meus}\ldots{} \emph{est}\\
\glt `For whoever does the will of God, he is my brother.' \\
\hspace*{\fill}(\iwi{NT Mark 3.35})

\z

\z

Examples \xref{ex:bj:4b} and \xref{ex:bj:4c} also lead us to consider other parallel cases
as SVCs, such as \xref{ex:bj:5a} and \xref{ex:bj:5b}, where the verb γίγνεσθαι
\emph{gígnesthai} `to happen' is used to express the corresponding impersonal passive\is{passive} of
these collocations (\cite{JiménezLópezM.Dolores-2021724}). Γίγνεσθαι \emph{gígnesthai},
as well as εἶναι \textit{eînai}, function in these cases as typical support verbs,\footnote{We do not
  include, obviously, cases in which these verbs are used as a copula with an attribute or
  nominal predicate. On γίγνεσθαι \emph{gígnesthai} in the Gospels, see
  \citet{TronciLiana-2020488}.} denoting the occurrence of an event
\citep{GaatonDavid-2004326} in which the Agent is either irrelevant or relegated to a
secondary role, as demonstrated in \xxref{ex:bj:6a}{ex:bj:6b}. These verbs may alternate
when combined with nouns denoting inagentive processes or natural phenomena, as in
\xxref{ex:bj:6c}{ex:bj:6d}. It is worth noting that the Latin translation of the Greek
alternation in \xref{ex:bj:6a} and \xref{ex:bj:6b} involves in both cases the verb
\emph{fieri}.

\is{collocation}

\ea\label{ex:bj:5}

\ea\label{ex:bj:5a}

\gllll Καὶ \textbf{δείπνου} \textbf{γινομένου}\ldots{}\\
 \textit{Kaì} \textit{deípnou} \textit{ginoménou\ldots{}}\\
And supper happen\ldots{}\\
\emph{Et} \textit{\textbf{cena}} \textit{\textbf{facta}}\ldots{}\\
\glt `During supper\ldots' \\
\hspace*{\fill}(\iwi{NT John 13.2})

\ex\label{ex:bj:5b}

\gllll \textbf{γενηθήτω} \textbf{τὸ} \textbf{θέλημά} σου\\
 \textit{genēthētō} \textit{tò} \textit{thélēmá} \textit{sou}\\
{be done} the will your\\
\textbf{\itshape fiat} ~ \textbf{\itshape voluntas} \textit{tua}\\
\glt `Your will be done' \\
\hspace*{\fill}(\iwi{NT Matthew 6.10})

\z

\z

\ea\label{ex:bj:6}

\ea\label{ex:bj:6a}

\gllll Μὴ ἐν τῇ ἑορτῇ, μήποτε \textbf{ἔσται} \textbf{θόρυβος} τοῦ λαοῦ\\
 \textit{Mḕ} \textit{en} \textit{têi} \textit{heortêi} \textit{mḗpote} \textit{éstai} \textit{thórybos} \textit{toû} \textit{laoû}\\
Not in the feast never be uproar the people\\
\emph{non} \emph{in} \emph{die} \emph{festo} \emph{ne} \emph{forte} \textbf{\itshape
  tumultus} \textbf{\itshape fieret} \emph{populi}\\
\glt `Not during the feast, lest there be an uproar from the people' \\
\hspace*{\fill}(\iwi{NT Mark 14.2})

\ex\label{ex:bj:6b}

\gllll Μὴ ἐν τῇ ἑορτῇ, ἵνα μὴ \textbf{θόρυβος} \textbf{γένηται} ἐν τῷ λαῷ\\
 \textit{Mḕ} \textit{en} \textit{têi} \textit{heortêi} \textit{ína} \textit{mḕ} \textit{thórybos} \textit{génētai} \textit{en} \textit{tôi} \textit{laôi}\\
Not in the feast {in order that} not uproar happen in the people\\
\emph{non} \emph{in} ~ {\emph{die} \emph{festo}} {\emph{ne} \emph{forte}} ~ \textbf{\itshape
  tumultus} \textbf{\itshape fiat} \emph{in} ~  \emph{populo}\\
\glt `Not during the feast, lest there be an uproar among the people' \\
\hspace*{\fill}(\iwi{NT Matthew 26.5})

\ex\label{ex:bj:6c}

\gllll \textbf{ἐγένετο} \textbf{λιμὸς} μέγας ἐπὶ πᾶσαν τὴν γῆν\\
 \textit{egéneto} \textit{limòs} \textit{mégas} \textit{epì} \textit{pasan} \textit{tḕn} \textit{gên}\\
happen famine big over all the land\\
\textbf{\itshape facta est} \emph{\bfseries fames} \emph{magna} \emph{in} \emph{omni} ~ \emph{terra}\\
\glt `A great famine came over all the land' \\
\hspace*{\fill}(\iwi{NT Luke 4.25}).

\ex\label{ex:bj:6d}

\gllll \textbf{σεισμοί} τε μεγάλοι καὶ κατὰ τόπους \textbf{λιμοὶ} \textbf{καὶ} \textbf{λοιμοὶ} \textbf{ἔσονται}\\
 \textit{seismoí} \textit{te} \textit{megáloi} \textit{kaì} \textit{katà} \textit{tópous} \textit{limoì} \textit{kaì} \textit{loimoì} \textit{ésontai}\\
earthquakes \textsc{prt} big and in places famines and pestilences be\\
\textbf{\itshape terraemotus} ~ {\emph{magni} \textbf{\itshape erunt}} ~ \emph{per} \textbf{\itshape loca} {\textbf{\itshape et} \textbf{\itshape pestilentiae}}
\textbf{\itshape et} \textbf{\itshape fames}\\
\glt `There will be great earthquakes, and in various places famines and pestilences' \\
\hspace*{\fill}(\iwi{NT Luke 21.11}).

\z

\z

In a similar vein, the comparative analysis of the four Gospels allows the description of
the collocational pattern of certain highly frequent nouns, such as ἐντολή \emph{entolḗ}
`order, command'. The verb ἐντέλλεσθαι \emph{entéllesthai} `to command', see \xref{ex:bj:7a},
is used 9 times in the Gospels. However, John (and only he) also has recourse to various
SVCs which present the event from different perspectives: ἐντολὴν διδόναι \emph{entolḕn
  didónai}, see \xref{ex:bj:7b}, and, complementarily, ἐντολὴν λαμβάνειν \emph{entolḕn
  lambánein}, see \xref{ex:bj:7c}, and ἐντολὴν ἔχειν \emph{entolḕn échein}, see \xref{ex:bj:7d},
that is, `to give, receive, and have an order'. Moreover, an order is by definition a
command that must be obeyed, observed, and executed. Thus, the verb τηρεῖν \emph{tēreîn}
`to observe, keep', see \xref{ex:bj:7d},  also forms part of the combinatorial possibilities of ἐντολή
\emph{entolḗ}, expressing the fulfillment of the order, as well as the opposite: `to
break the commandment', ἀφιέναι \emph{aphiénai} (\iwi{NT Mark 7.8}) or παραβαίνειν
\emph{parabaínein} (\iwi{NT Matthew 15.3}).



\ea\label{ex:bj:7}

\ea\label{ex:bj:7a}

\gllll καθὼς \textbf{ἐνετείλατο} μοι ὁ πατήρ, οὕτως ποιῶ\\
 \textit{kathṑs} \textit{eneteílato} \textit{moi} \textit{ho} \textit{patḗr} \textit{hoútōs} \textit{poiô}\\
as command me the Father so do\\
\emph{sicut} {\textbf{\itshape mandatum} \textbf{\itshape dedit}} \emph{mihi} ~ \emph{Pater}, \emph{sic} \emph{facio}\\
\glt `I do as the Father has commanded me.' \\
\hspace*{\fill}(\iwi{NT John 14.31})

\ex\label{ex:bj:7b}

\gllll  ὁ πέμψας με πατὴρ αὐτός μοι \textbf{ἐντολὴν} \textbf{δέδωκεν}\\
 \textit{ho} \textit{pémpsas} \textit{me} \textit{patḕr} \textit{autós} \textit{moi} \textit{entolḕn} \textit{dédōken}\\
the sent me Father himself me commandment give\\
\emph{qui} \emph{misit} \emph{me}, \emph{Pater}, \emph{ipse} \emph{mihi} \textbf{\itshape mandatum} \textbf{\itshape dedit}\\
\glt `The Father who sent me has himself given me a commandment.' \\
\hspace*{\fill}(\iwi{NT John 12.49})

\ex\label{ex:bj:7c}

\gllll ταύτην \textbf{τὴν} \textbf{ἐντολὴν} \textbf{ἔλαβον} παρὰ τοῦ πατρός μου\\
 \textit{taútēn} \textit{tḕn} \textit{entolḕn} \textit{élabon} \textit{parà} \textit{toû} \textit{patrós} \textit{mou}\\
This the charge receive from the Father my\\
\emph{hoc} ~ \textbf{\itshape mandatum} \textbf{\itshape accepi} \emph{a} ~ \emph{Patre} \emph{meo}\\
\glt `This charge I have received from my Father.' \\
\hspace*{\fill}(\iwi{NT John 10.18})

\ex\label{ex:bj:7d}

\gllll ὁ \textbf{ἔχων} \textbf{τὰς} \textbf{ἐντολάς} μου καὶ \textbf{τηρῶν} αὐτὰς ἐκεῖνός ἐστιν ὁ ἀγαπῶν με\\
 \textit{ho} \textit{échōn} \textit{tàs} \textit{entolás} \textit{mou} \textit{kaì} \textit{tēròn} \textit{autàs} \textit{ekeìnós} \textit{estin} \textit{ho} \textit{agapôn} \textit{me}\\
the have the commandments my and keep them that be the love me\\
\emph{qui} \textbf{\itshape habet} ~ \textbf{\itshape mandata} \textbf{\itshape mea}
\emph{et} \textbf{\itshape servat} \emph{ea}, \emph{ille} \emph{est} \emph{qui} \emph{diligit} \emph{me}\\
\glt `Whoever has my commandments and keeps them, he it is who loves me.' \\
\hspace*{\fill}(\iwi{NT John 14.21})

\z

\z

In order not to prolong this discussion, let us consider one last example. Concerning the
meaning `to magnify, glorify' expressed by the synthetic predicate δοξάζειν
\emph{doxázein} in \xref{ex:bj:8a} and \xref{ex:bj:8d}, one finds the analytic
alternative δόξαν δοῦναι \emph{dóxan doûnai}, see \xref{ex:bj:8b}, but also other SVCs with
the same noun, which present the event from different perspectives: metaphorically,
`glory' is an `object' given, see \xref{ex:bj:8b}, but also received, see \xref{ex:bj:8c}, or
possessed, see \xxref{ex:bj:8d}{ex:bj:8e}.

\ea\label{ex:bj:8}

\ea\label{ex:bj:8a}

\gllll \textbf{ἐδόξαζον} τὸν θεὸν\\
 \textit{edóxazon} \textit{tòn} \textit{theòn}\\
glorify the God\\
\textbf{\itshape magnificabant} ~ \emph{Deum}\\
\glt `They glorified God.' \\
\hspace*{\fill}(\iwi{NT Luke 5.26})

\ex\label{ex:bj:8b}

\gllll \textbf{δοῦναι} \textbf{δόξαν} τῷ θεῷ\\
 \textit{doûnai} \textit{dóxan} \textit{tôi} \textit{theôi}\\
give praise the God\\
\emph{\bfseries darent} \emph{\bfseries gloriam} ~ \emph{Deo}\\
\glt `Give praise to God.' \\
\hspace*{\fill}(\iwi{NT Luke 17.18})

\ex\label{ex:bj:8c}

\gllll \textbf{δόξαν} παρὰ ἀνθρώπων οὐ \textbf{λαμβάνω}\\
 \textit{dóxan} \textit{parà} \textit{anthṓpōn} \textit{ou} \textit{lambánō}\\
glory from people not receive\\
\textbf{\itshape gloriam} \emph{ab} \emph{hominibus} \emph{non} \textbf{\itshape accipio}\\
\glt `I do not receive glory from people.' \\
\hspace*{\fill}(\iwi{NT John 5.41}).

\ex\label{ex:bj:8d}

\gllll καὶ νῦν \textbf{δόξασόν} με σύ, πάτερ, παρὰ σεαυτῷ \textbf{τῇ} \textbf{δόξῃ} \textbf{ᾗ} \textbf{εἶχον} πρὸ τοῦ τὸν κόσμον εἶναι παρὰ σοί\\
 \textit{kaì} \textit{nŷn} \textit{dóxasón} \textit{me} \textit{sý} \textit{páter} \textit{parà} \textit{seautôi} \textit{têi} \textit{dóxēi} \textit{hêi} \textit{eîchon} \textit{prò} \textit{toû} \textit{tòn} \textit{kósmon} \textit{eînai} \textit{parà} \textit{soí}\\
and now glorify me you Father near yourself the glory that have before the the world be near you\\
\emph{et} \emph{nunc} \textbf{\itshape clarifica} \emph{me} \emph{tu} \emph{Pater}, \emph{apud} \emph{temet} \emph{ipsum} \textbf{\itshape claritatem} \textbf{\itshape quam}
\textbf{\itshape habui} \emph{priusquam} ~ ~ \emph{mundus} \emph{esset} \emph{apud} \emph{te}\\
\glt `And now, Father, glorify me in your own presence with the glory that I had with you before the world existed.' \\
\hspace*{\fill}(\iwi{NT John 17.5})

\ex\label{ex:bj:8e}

\gllll τότε \textbf{ἔσται} \textbf{σοι} \textbf{δόξα} ἐνώπιον πάντων τῶν συνανακειμένων σοι\\
 \textit{tóte} \textit{éstai} \textit{soi} \textit{dóxa} \textit{enṓpion} \textit{pántōn} \textit{tōn} \textit{synanakeiménōn} \textit{soi}\\
then be you glory {face to face} all the {recline together at table} you\\
\emph{tum} \textbf{\itshape erit} \textbf{\itshape tibi} \textbf{\itshape gloria} {\textit{coram} \emph{simul}} ~ ~ \emph{discumbentibus} \vspace*{-2ex} \\
\glt `Then you will be honoured in the presence of all who sit at table with you.' \\
\hspace*{\fill}(\iwi{NT Luke 14.10})

\z

\z


\section{Support-verb constructions in Greek: the shared and exclusive SVCs in each Gospel}\label{sec:bj:3}

In accordance with \tabref{tbl:bj:gospels}, a total of 521 SVCs are
attested in the Gospels, distributed as follows: 76 in Mark, 117 in Matthew, 138 in Luke,
and 193 in John. However, these absolute figures need to be refined considering the
different length (number of words)\footnote{The number of words for each work is taken
  from the \textit{Thesaurus Linguae Graecae}.} of each Gospel. Thus, if we examine the relative frequency of SVCs
(number of SVCs per 1000 words), as shown in \tabref{tbl:bj:svc}, the synoptic Gospels\is{synoptic Gospels}
exhibit similar frequencies, as opposed to the Gospel of John, who is by far the author
that most frequently employs SVCs (almost twice as often as Matthew).




\begin{table}
  \caption{Number of examples with an SVC in the Gospels}
  \label{tbl:bj:svc}
  \centering
  \small
\begin{tabularx}{\textwidth}{Xrrrrr}
\lsptoprule
 & Mark & Matthew & Luke & John & Total\\
\midrule
nº of examples with an SVC & 76 & 116 & 136 & 193 & 521\\
nº of words & 11,299 & 18,338 & 19,451 & 15,635 & 64,723\\
nº of examples/1000 words & 6.72 & 6.32 & 6.99 & 12.34 & 8.04\\
\lspbottomrule
\end{tabularx}
\end{table}

This congruence among the three synoptic Gospels\is{synoptic Gospels} (Mark, Matthew, and Luke) is not \emph{a
  priori} surprising, as they essentially narrate the same events from the life of Jesus.
Likewise, one would expect the different aims and content of the Gospel of John to be also
reflected in the use of SVCs.

\newpage
However, this general impression will undergo considerable refinement upon a closer
analysis of the evidence. In fact, differences in SVC usage appear not only between John
and the synoptic Gospels but also between Mark, Matthew, and Luke, due to the different
nature and varying quality of the Greek they employ (\cite[vol.
IV]{MHTMoultonTurner1906}, \cite{PorterStanleyE-2014141}).\footnote{It is useful to bear in
  mind when comparing the three synoptic Gospels that the first published Gospel was that
  of Mark (hence it is cited first in the tables) and that both Matthew and Luke had the
  text of Mark in front of them and sometimes varied in the use of certain SVCs.}

These internal differences become more evident when comparing not only the total number of
occurrences of SVCs, but also the number of distinct SVCs used in each Gospel, regardless
of their frequency. Thus, the 521 examples correspond to 231 distinct SVCs. Some of these
are shared by multiple gospel writers, while others, as will be seen later, are exclusive
to a given text.\footnote{One should take into account the SVCs shared by multiple authors
  to understand why the figures in \tabref{tbl:bj:difsvc} total more than 231 cases.}
\tabref{tbl:bj:difsvc} presents the number of different SVCs attested in each Gospel.

\begin{table}
\caption{Number of distinct SVCs in the Gospels}
\label{tbl:bj:difsvc}
\centering
\begin{tabularx}{.9\textwidth}{Xrrrr}
\lsptoprule
 & Mark & Matthew & Luke & John\\
\midrule
nº of Greek words & 11,299 & 18,338 & 19,451 & 15,635\\
nº of distinct SVCs & 57 & 67 & 98 & 84\\
nº of SVCs /1000 & 5.04 & 3.65 & 5.04 & 5.37\\
\lspbottomrule
\end{tabularx}
\end{table}

In light of the above, Mark, Luke, and John employ, in relative terms, a similar number of
SVCs, whereas Matthew uses proportionally the lowest number of distinct SVCs.


\largerpage[-2]
However, it is necessary to delve even further into the data. Thus, out of the 231 SVCs
attested in the Gospels, 182 are exclusively used in one Gospel; that is, almost four out
of every five SVCs (78.79\%) are employed solely by one author.\footnote{Out of the 231
  SVCs, only 6 appear in all four Gospels; the most frequent is χρείαν ἔχειν
  \emph{chreían échein} `to need' (20 instances). There are only 7 SVCs common to Mark,
  Matthew, and Luke (e.g., πίστιν ἔχειν \emph{pístin échein} `to have faith') and another
  7 are shared by John and two of the three synoptic Gospels\is{synoptic Gospels}, such as θέλημα ποιεῖν
  \emph{thélēma poieîin} `to fulfill the will'. Two further gospel writers share the use
  of 29 SVCs.} \tabref{tbl:bj:uniqsvc} details the distribution of these 182 SVCs in
each Gospel.

\begin{table}
\caption{Number of SVCs unique to each Gospel}
\label{tbl:bj:uniqsvc}
\centering
\begin{tabularx}{.8\textwidth}{Xrrr}
\lsptoprule
 & nº of SVCs & nº of unique SVCs & \%\\
\midrule
Mark & 57 & 24 & 42.10\%\\
Matthew & 68 & 28 & 41.17\%\\
Luke & 99 & 69 & 69.69\%\\
John & 84 & 61 & 72.61\%\\
\lspbottomrule
\end{tabularx}
\end{table}

According to the data, the Gospel of John displays, in relative terms, the highest number
of unique SVCs: three out of every four SVCs used by John (72.61\%) do not appear in any
other Gospel. Among the synoptic Gospels\is{synoptic Gospels}, Luke employs proportionally the highest number
of unique SVCs (two out of every three), a frequency that decreases significantly in Mark
and Matthew.



This information is relevant, as it reveals the extent to which the use of SVCs can be
idiosyncratic. To mention a few illustrative cases, John employs σημεῖον ποιεῖν
\emph{sēmeîon poieîn} `to do signs', see \xref{ex:bj:9a}, in an exclusive manner and with
notable frequency (15 instances), while the synoptic Gospels\is{synoptic Gospels} use (7 instances) σημεῖον
διδόναι \emph{sēmeîon didónai}, see \xref{ex:bj:9b}. 

% =====

\ea\label{ex:bj:9}

\ea\label{ex:bj:9a}

\gllll Πολλὰ μὲν οὖν καὶ ἄλλα \textbf{σημεῖα} \textbf{ἐποίησεν} ὁ Ἰησοῦς\\
 \textit{Pollà} \textit{mèn} \textit{oûn} \textit{kaì} \textit{álla} \textit{sēmeîa} \textit{epoíēsen} \textit{ho} \textit{Iēsoûs}\\
Much \textsc{prt} \textsc{prt} and other signs do ~ Jesus\\
\emph{multa} ~ \emph{quidem} \emph{et} \emph{alia} \textbf{\itshape signa}
\textbf{\itshape fecit} ~ \emph{Iesus}\\
\glt `Now Jesus did many other signs.' \\
\hspace*{\fill}(\iwi{NT John 20.30})

\ex\label{ex:bj:9b}

\gllll καὶ \textbf{δώσουσιν} \textbf{σημεῖα} μεγάλα καὶ τέρατα\\
 \textit{kaì} \textit{dṓsousin} \textit{sēmeîa} \textit{megála} \textit{kaì} \textit{térata}\\
and give signs big and wonders\\
\emph{et} \textbf{\itshape dabunt} \textbf{\itshape signa} \emph{magna} \emph{et} \emph{prodigia}\\
\glt `{[}They{]} will perform great signs and wonders.' \\
\hspace*{\fill}(\iwi{NT Matthew 24.24})

\z

\z


A similar pattern is observed with ψυχὴν
τιθέναι \emph{psychḕn tithénai} `to lay down the life', see \xref{ex:bj:10a}, attested up to
6 times in John, while Mark and Matthew (2 instances) use ψυχὴν διδόναι \emph{psychḕn
  didónai}, see \xref{ex:bj:10b}:


\ea\label{ex:bj:10}

\ea\label{ex:bj:10a}

\gllll \textbf{τὴν} \textbf{ψυχήν} μου ὑπὲρ σοῦ \textbf{θήσω}\\
 \textit{tḕn} \textit{psychḗn} \textit{mou} \textit{hyper} \textit{soû} \textit{thḗsō}\\
the soul my for you put\\
~ \textbf{\itshape animam} \emph{meam} \emph{pro} \emph{te} \textbf{\itshape ponam}\\
\glt `I will lay down my life for you.' \\
\hspace*{\fill}(\iwi{NT John 13.37})

\ex\label{ex:bj:10b}

\gllll καὶ γὰρ ὁ υἱὸς τοῦ ἀνθρώπου\ldots{} ἦλθεν\ldots{} διακονῆσαι καὶ \textbf{δοῦναι} \textbf{τὴν} \textbf{ψυχὴν} αὐτοῦ λύτρον ἀντὶ πολλῶν\\
 \textit{kaì} \textit{gàr} \textit{ho} \textit{huiòs} \textit{toû} \textit{anthrṓpou} \textit{êlthen} \textit{diakonêsai} \textit{kaì} \textit{doûnai} \textit{tḕn} \textit{psychḕn} \textit{autoû} \textit{lýtron} \textit{antì} \textit{pollôn}\\
and for the son ~ man come serve and give the soul him {price paid} {instead of} many\\
\emph{Nam} \emph{et} ~ \emph{Filius} ~ \emph{hominis}\ldots{} \emph{venit}\ldots{} {\emph{ut}
  \emph{ministraret}} \emph{et} \textbf{\itshape daret} ~ \textbf{\itshape animam} \emph{suam} \emph{redemptionem} \emph{pro} \emph{multis}\\
\glt `For even the Son of Man came\ldots{} to serve, and to give his life as a ransom for many.' \\
\hspace*{\fill}(\iwi{NT Mark 10.45})

\z

\z

Other SVCs exclusive to John include λόγον τηρεῖν \emph{logon tēreîn} `to keep the word'
(8 instances), ἁμαρτίαν ἔχειν \emph{hamartian échein} `to have guilt' (4 instances), and
ἀγαπὴν ἔχειν \emph{agapḕn échein} `to have love' (3 instances). In addition to the
synthetic predicate μαρτυρεῖν \emph{martyreîn} `to give witness' (33 instances appear in John out of the total of 35 instances
in all the Gospels), John exclusively employs, on three occasions, μαρτυρίαν λαμβάνειν
\emph{martyrían lambánein} `to receive testimony', see \xref{ex:bj:11}, to express the reverse
perspective, placing the recipient of the testimony instead of the
one providing it in the subject position.

\ea\label{ex:bj:11}

\gllll ὃ ἑωράκαμεν \textbf{μαρτυροῦμεν}, καὶ \textbf{τὴν} \textbf{μαρτυρίαν} ἡμῶν οὐ \textbf{λαμβάνετε}\\
 \textit{ho} \textit{heōrákamen} \textit{martyroûmen} \textit{kaì} \textit{tḕn} \textit{martyrían} \textit{hēmôn} \textit{ou} \textit{lambánete}\\
what see {bear witness} and the witness our not receive\\
\emph{quod} \emph{vidimus}, \textbf{\itshape testamur}, \emph{et} ~ \textbf{\itshape
  testimonium} \emph{nostrum} \emph{non} \textbf{\itshape accipitis}\\
\glt `We speak of what we know, and bear witness to what we have seen, but you do not receive our testimony.' \\
\hspace*{\fill}(\iwi{NT John 3.11})

\z

Matthew uniquely employs (5 instances) the SVC συμβούλιον λαμβάνειν \emph{symboúlion
  lambánein} `to form a plan, to decide', where Mark uses συμβούλιον διδόναι
\emph{symboúlion} \emph{didónai} or συμβούλιον ποιεῖν \emph{symboúlion
  poieîn}.\footnote{\iwi{NT Mark 3.6} and \iwi{NT Mark 15.1}, respectively. For an analysis of the SVCs with
  συμβούλιον \emph{symboúlion}, cf. infra Section~\ref{sec:bj:4:1} and \citet{JiménezLópez2017}.} In
contrast to the systematic use of φονεύω \emph{phoneúō} `to commit murder' in the
synoptic Gospels (7 instances), Mark is the only one to employ the SVC φόνον ποιεῖν
\emph{phónon poieîn} (\iwi{NT Mark 15.7}). Additionally, alongside the synthetic predicate
τρέφειν \emph{tréphein} `to nourish' (5 instances), only Matthew (\iwi{NT Matthew 24.45}) has
recourse to τροφὴν διδόναι \emph{trophḕn didónai} `to give food'.




Finally, Luke is the only author who writes, on two occasions, φόρον διδόναι \emph{phónon
  didónai} `to pay tax', see \xref{ex:bj:12a}, whereas Mark and Matthew, see \xref{ex:bj:12b}, use
κῆνσον διδόναι \emph{kênson didónai} for the same episode:

\ea\label{ex:bj:12}

\ea\label{ex:bj:12a}

\gllll ἔξεστιν ἡμᾶς Καίσαρι \textbf{φόρον} \textbf{δοῦναι} ἢ οὔ;\\
 \textit{éxestin} \textit{hēmâs} \textit{Kaísari} \textit{phóron} \textit{doûnai} \textit{ḕ} \textit{oú?}\\
{it is possible} we Caesar tribute give or not\\
\emph{licet} \emph{nobis} \textbf{\itshape dare} \textbf{\itshape tributum} \emph{Caesari} \emph{an} \emph{non?}\\
\glt `Is it lawful for us to give tribute to Caesar, or not?' \\
\hspace*{\fill}(\iwi{NT Luke 20.22}. cf. also \iwi{NT Luke 23.2})

\ex\label{ex:bj:12b}

\gllll ἔξεστιν \textbf{δοῦναι} \textbf{κῆνσον} Καίσαρι ἢ οὔ;\\
 \textit{éxestin} \textit{doûnai} \textit{kênson} \textit{Kaísari} \textit{ḕ} \textit{oú?}\\
{it is possible} give tribute Caesar or not\\
\emph{licet} \textbf{\itshape censum} \textbf{\itshape dare} \emph{Caesari} \emph{an} \emph{non?}\\
\glt `Is it lawful to pay taxes to Caesar, or not?' \\
\hspace*{\fill}(\iwi{NT Mark 12.14}; cf. also \iwi{NT Matthew 22.17})

\z

\z

In order to gain a fuller understanding of the evidence presented here (along with other
findings yet to be discussed), a dedicated study of the unique collocations of each Gospel
writer from a diachronic perspective is required. It is thus crucial to investigate which
SVCs are already attested in literary texts from the archaic and classical periods, which
ones appear in koine writers\is{Koine} contemporaneous with the composition of the New Testament, or
if this usage is unique to the Greek of the Septuagint (LXX henceforth). This approach will allow an
assessment of the degree of continuity or innovation exhibited by each gospel writer in
employing these complex predicates.

\is{collocation}
\is{diachrony}

By way of example, 7 out of the 24 collocations exclusive to the Gospel of Mark are
already attested in classical times.\footnote{Specifically, ἀπώλειαν γίγνεσθαι
  \emph{apṓleian gígnesthai} `to be wasted', θόρυβον εἶναι \emph{thórybon eînai}
  `there be an uproar', λόγον λαμβάνειν \emph{logon lambánein} `to receive the word',
  λόγον παραδέχεσθαι \emph{logon paradéchesthai} `to accept the word', τρόμον ἔχειν
  \emph{trómon échein} `trembling overtakes someone', φέγγνος διδόναι \emph{phéngos
    didónai} `to give light', and φωνὴν ἀφιέναι \emph{phōnḕn aphiénai} `utter a cry'.}
Another 2 are found in the LXX, as well as in koine\is{Koine} literary
texts.\footnote{Specifically, ἁμαρτήματα ἀφιέναι \emph{hamartḗmata aphiénai} `to forgive
  sins' and φόνον ποιεῖν \emph{phónon poieîn} `to commite murder'.} The remaining, that
is, more than half of the unique SVCs, are attested for the first time in this author. A
similar comparative analysis of the rest of the Gospels will reveal the degree of
classicism or, conversely, innovation in the language of each author. It will also shed
light on potential interference\is{linguistic interference} from Aramaic, Hebrew or Latin within the multilingual
context in which the Gospels were written
\parencites{JanseMark-2007475,JanseMark-2014201,GeorgeCoulterH-2010201,RochetteBruno-2010755}[124--125]{Horrocks2010}.


Thus, for example, the collocation κῆνσον διδόναι \emph{kênson didónai} in Mark, see
\xref{ex:bj:12b}, is partially a Latinism (from \emph{censum}), which Luke corrects by
opting for the more natural-sounding Greek construction φόρον διδόναι \emph{phónon
  didónai}, see \xref{ex:bj:12a}, in line with the higher-quality Greek attributed to him
\parencites[vol. IV:
47--60]{MHTMoultonTurner1906}{PorterStanleyE-2014141}[98]{JiménezLópezM.Dolores-2018285}.
Luke, in turn, is the first to use ἐργασίαν διδόναι \emph{ergasían didónai} `to make an
effort' (\iwi{NT Luke 12.58}), considered a calque\is{calque} from the Latin \emph{operam dare}
\parencites[II, 1, 123]{Mayser1926}, just like συμβούλιον λαμβάνειν \emph{symboúlion
  lambánein}, which is exclusively used by Matthew and is a calque from \emph{consilium
  capere} (\cite[5--7]{BDFBlassFunk1961}, \cite[7]{Marucci1993}). On the other hand, the
combination συμβούλιον διδόναι \emph{symboúlion didónai} in Mark (\iwi{NT Mark 3.6}) is often
considered a Hebraism or Aramaism (\cite[852]{WestcottHort2007},
\cite[128]{ZerwickGrosvenor2008}, \cite{JiménezLópez2017}).

\is{linguistic interference}
\is{collocation}


Finally, the influence of Hebrew, indirectly evident in the Gospels primarily through
quotations and phraseology borrowed from the Greek of the LXX, explains, for
instance, Matthew's alternating use of ἀνομίαν ἐργάζεσθαι \emph{anomían ergázesthai}
(\iwi{NT Matthew 7.23}) and ἀνομίαν ποιεῖν \emph{anomían poieîn} (\iwi{NT Matthew 13.41}). This
alternation arises from the use of two different Hebrew support verbs in the Old
Testament,
{פָּעַל}
\emph{p̄āʿal}  and
{עָשָׂה}
\emph{ʿāśâ}, and their literal translation in the LXX as ἐργάζεσθαι
\emph{ergázesthai} and ποιεῖν \emph{poieîn}, respectively
\parencites{BañosJoséMiguelJiménezLópezM.Dolores-2022707,BañosJiménezLópez2024}.

%% hasta aquí

\section{Support-verb constructions in the Vulgate}\label{sec:bj:4}

In the Latin version of the Vulgate, a total of 644 SVCs are attested in the Gospels with
the following distribution: Mark 96 examples, Matthew 162, Luke 158, and John 238.
Considering the varying length of each Gospel, their relative frequency (number of SVCs
per 1000 words) is presented in \tabref{tbl:bj:vultate}. As expected in a Latin
translation which aimed to be literal, a proportion similar to the original Greek version
is observed (cf. \tabref{tbl:bj:svc}): the Gospel of John includes by far the highest
number of examples, while the three synoptic Gospels\is{synoptic Gospels} exhibit a comparable usage.



\begin{table}
  \caption{Number of examples with SVC in the Gospels (Vulgate)}
  \label{tbl:bj:vultate}
\centering
\small
\begin{tabularx}{\textwidth}{Xrrrrr}
\lsptoprule
 & Mark & Matthew & Luke & John & total\\
\midrule
nº of examples with an SVC & 96 & 162 & 148 & 238 & 644\\
nº of words & 12,076 & 19,521 & 20,728 & 16,576 & 68,901\\
nº of examples/1000 words & 7.95 & 8.30 & 7.14 & 14.36 & 9.35\\
\lspbottomrule
\end{tabularx}
\end{table}


According to the data in \tabref{tbl:bj:vultate}, the Gospels contain 9.35 SVCs per
1000 words. This figure is particularly striking when compared to the frequency of SVC
usage in the broader body of Latin literature.

\figref{fig:bj:graph} presents the data from
\citet{BañosJoséMiguel-2023295}\footnote{The study of \citet{BañosJoséMiguel-2023295}
  includes an analysis of SVC from 30 different literary works (or fragments thereof)
  displaying a comparable length (of approximately 4400--4600 words each). Among them was a
  fragment from the Gospel of Matthew (\iwi{NT Matthew 1-10.10}), with a relative frequency (8.71 SVCs per
  1000 words) similar to that in Table 5 (8.30) or the entire Gospel of Matthew.} on SVC
frequency in 30 Latin works, both in prose and verse, across various literary genres in a
comprehensive corpus from Plautus to the \emph{Historia Augusta}. We have incorporated the
data from the Gospels into this figure, arranging the works from the highest (leftmost edge of the figure) to the lowest (rightmost edge of the figure)
frequency of SVC usage:

\begin{figure}
  \centering
  \includegraphics[width=.9\textwidth]{figures/banos-jimenez-fig1.png}
  \caption{Frequency of SVCs from Plautus to the \emph{Gospels}}
  \label{fig:bj:graph}
\end{figure}

The Gospels are primarily narrative works, closely resembling historiographical texts,
which are the Latin literary genre that most employs SVCs, as illustrated in \figref{fig:bj:graph}.
However, out of the 30 Latin works examined, regardless of their content or literary
genre, the Gospels contain the lowest number of SVCs. This is due to their nature as
translations, and particularly, translations from Greek. On the one hand, these complex
predicates are generally used much less frequently in ancient Greek than in classical
Latin, constituting a fundamental distinguishing feature between the two classical
languages.\largerpage\footnote{Cf. \citet{BañosJoséMiguel-2015206}. Thus, for example, when comparing
  a corpus of similar size and content from Caesar and Xenophon (\cite{LópezMartín2019}),
  there are four SVCs in Caesar for every one found in Xenophon.} On the other hand,
considering that the Latin translation of the Vulgate aimed to be literal, one might
reasonably expect that if the source language (Greek) used few SVCs, this would be
reflected to a greater or lesser extent in the target language (Latin).

\is{Vulgate}

\subsection{Translation \emph{verbum e verbo} or \emph{sensum de sensu}?}\label{sec:bj:4:1}

However, this assumption of a literal translation must be qualified in view of the
evidence. Indeed, when comparing the Greek and Latin versions of the Gospels, it is
striking that the Vulgate contains many more SVCs (644 examples) than the original Greek
(521 examples).

This is largely because, given the more natural use of SVCs in Latin than in Greek, the
Vulgate often translates a Greek synthetic predicate with an SVC. To illustrate this
point, it is sufficient to compare the original Greek version of the passage on the
commandments in the three synoptic Gospels\is{synoptic Gospels} (`Do not murder, Do not commit adultery, Do not
steal, Do not bear false witness, Do not defraud') with its respective Latin translation:

% 13

\ea\label{ex:bj:13}

\ea\label{ex:bj:13a}

\gllll μὴ φονεύσῃς, μὴ μοιχεύσῃς, μὴ κλέψῃς, μὴ \textbf{ψευδομαρτυρήσῃς}, μὴ \textbf{ἀποστερήσῃς}\\
 \textit{mḕ} \textit{phoneúsēis} \textit{mḕ} \textit{moicheúsēis} \textit{mḕ} \textit{klépsēis} \textit{mḕ} \textit{pseudomartyrḗsēis} \textit{mḕ} \textit{aposterḗsēis}\\
not murder not {commit adultery} not steal not {bear false witness} not defraud\\
\emph{ne} \emph{adulteres}, \emph{ne} \emph{occidas}, \emph{ne} \emph{fureris}, \emph{ne} {\textbf{\itshape falsum} \textbf{\itshape testimonium} \textbf{\itshape dixeris}}
\emph{ne} {\textbf{\itshape fraudem} \textbf{\itshape feceris}}\\
\glt `Do not murder, Do not commit adultery, Do not steal, Do not bear false witness, Do not defraud.' \\
\hspace*{\fill}(\iwi{NT Mark 10.19})

\ex\label{ex:bj:13b}

\gllll Τὸ οὐ \textbf{φονεύσεις}, οὐ μοιχεύσεις, οὐ \textbf{κλέψεις}, οὐ \textbf{ψευδομαρτυρήσεις}\\
 \textit{tò} \textit{ou} \textit{phoneúsēis} \textit{ou} \textit{moicheúseis} \textit{ou} \textit{klépseis} \textit{ou} \textit{pseudomartyrḗseis}\\
the not murder not {commit adultery} not steal not {bear false witness}\\
~ \emph{non} {\textbf{\itshape homicidium} \textbf{\itshape facies}}, \emph{non} \emph{adulterabis}, \emph{non} {\textbf{\itshape facies} \textbf{\itshape furtum}},
\emph{non} {\emph{\bfseries falsum} \textbf{\itshape testimonium} \textbf{\itshape dices}}\\
\glt `You shall not murder, You shall not commit adultery, You shall not steal, You shall not bear false witness.' \\
\hspace*{\fill}(\iwi{NT Matthew 19.18})

\ex\label{ex:bj:13c}

\gllll  μὴ μοιχεύσῃς, μὴ φονεύσῃς, μὴ \textbf{κλέψῃς}, μὴ \textbf{ψευδομαρτυρήσῃ}\\
 \textit{mḕ} \textit{phoneúsēis} \textit{mḕ} \textit{moicheúseis} \textit{mḕ} \textit{klépseis} \textit{mḕ} \textit{pseudomartyrḗseis}\\
not murder not {commit adultery} not steal not {bear false witness}\\
\emph{non} \emph{occides}, \emph{non} \emph{moechaberis}, \emph{non} {\textbf{\itshape furtum} \textbf{\itshape facies}}, \textbf{\itshape non} {\textbf{\itshape falsum}
  \textbf{\itshape testimonium} \textbf{\itshape dices}}\\
\glt `Do not commit adultery, Do not murder, Do not steal, Do not bear false witness.' \\
\hspace*{\fill}(\iwi{NT Luke 18.20})

\z

\z

As can be seen, the three Greek Gospels express each commandment through the same
synthetic predicates, albeit with slight variations among them.\footnote{In addition to a
  change in the order of the first two commandments in Luke compared to Mark and Matthew,
  Mark adds a commandment  --- 'do not defraud' ---  which is absent from the versions of
  Matthew and Luke.} However, in the Vulgate these are sometimes translated as SVCs:
φονεύειν \emph{phoneúein} = \emph{homicidium facere} `to murder', κλέπτειν
\emph{kléptein} = \emph{furtum facere} `to steal', ψευδομαρτιρεῖν
\emph{pseudomartireîn} = \emph{falsum testimonium dicere} `to bear false witness', and
ἀποστερεῖν \emph{apostereîn} = \emph{fraudem facere} `to defraud'. Moreover, it seems
that there is no consistent approach to their translation, as the same Greek predicate is
sometimes translated into Latin synthetically and other times as an SVC: φονεύειν
\emph{phoneúein} = \emph{occidere} (Mark, Luke) / \emph{homicidium facere} (Matthew); κλέπτειν
\emph{kléptein} = \emph{furari} (Mark) / \emph{furtum facere} (Matthew, Luke).

In his revision of the earlier Latin translations of the Gospels (commonly known as
\emph{Vetus Latina}), carried out in AD 382 at the request of Pope Damasus, it seems that
St. Jerome did not strictly follow, in the case of the SVCs, the general principle which he had
laid out in his \iwi{\emph{Letter to Pammachius}} to explain his approach to translating Greek
texts:

\is{St. Jerome}
\is{Vetus Latina}

% 14

\eanoraggedright\label{ex:bj:14}

\emph{Ego enim non solum fateor, sed libera voce profiteor, ne in interpretatione
  Graecorum, absque Scripturis Sanctis, ubi et verborum ordo mysterium est, non verbum e
  verbo, sed sensum exprimere de sensu.} 
  
  
  `Truthfully, I admit it and also profess it
openly: in the translation of Greek texts --- \emph{apart from the Holy Scriptures}, where
even the order of the words is a mystery ---, I do not render word for word but sense for
sense' 
\\\hspace*{\fill}(\iwi{\textit{Epistula Hieronymi ad Damasum papam} 57.5-6}, italics our own).

\z

As can be seen, St. Jerome explicitly excludes the Bible (\emph{absque Scripturis
  Sanctis}) in his defense of the non-literal translation (\emph{non verbum e verbo}) of
Greek texts, since in his opinion the literalness of the sacred text, including word
order, must be respected. However, when it comes to the use of SVCs in the Gospels, he
does not strictly adhere, or only partially adheres, to this principle.

\is{St. Jerome}

In this regard, it is necessary to distinguish between two types of Latin SVCs in the
Vulgate (\cite[68--69]{BañosJoséMiguel-2015311}) based on their greater or lesser
literalness with respect to the original Greek:

\begin{enumerate}
\renewcommand{\labelenumi}{(\roman{enumi})}
\item Greek SVCs consistently translated as Latin SVCs, that is, \emph{verbum e
    verbo}. Specifically, 502 Latin SVCs follow this principle. This means that 77.95\% of
  the Latin SVCs in the Gospels are, in turn, translations of Greek SVCs.

\item However, on several occasions, a Latin SVC corresponds to a synthetic predicate in
  the Greek text, as in the examples discussed in \xref{ex:bj:13}. In such cases, a less literal
  translation is provided, more \emph{sensum de sensu}: 138 Latin SVCs (22.05\%) in the
  Vulgate, that is one out of five, do not have a parallel analytic correspondence in
  the original Greek text.
\end{enumerate}

In what follows, we will discuss the first type; in other words, how the Greek SVCs are
translated in the Vulgate. We will leave the second type, which presents numerous
variations and alternatives, for a future study.\footnote{Thus, a Greek synthetic
  predicate can be translated (i) with an SVC (εὐχαριστεῖν \emph{eucharisteîn} `to be
  thankful' = \emph{gratias agere}), (ii) with various SVCs (ἐπιμελεῖσθαι
  \emph{epimeleîsthai} `to take care of' = \emph{curam agere} and \emph{curam habere;}
  θανατοῦν \emph{thanatoûn} `to kill' = \emph{morte afficere} and \emph{morti tradere}), or
  (iii) interchangeably with a synthetic predicate and an SVC. To give three illustrative
  examples, μαρτυρεῖν \emph{martyreîn} `to bear witness' is translated as \emph{testari}
  (John), as well as \emph{testimonium perhibere} (John), \emph{testimonium dare} (Luke), or
  \emph{testimonio esse} (Matthew); μετανοεῖν \emph{metanoeîn} `to repent' as
  \emph{paenitere, paenitentiam agere} and \emph{paenitentiam habere}
  (\cite{BañosJiménezLópez2017a}); and μισεῖν \emph{miseîn} `to hate' as \emph{odisse,
    odio habere} and \emph{odio esse} (\cite{BañosJiménezLópez2017b}). The translations of
  types (ii) and (iii) sometimes reveal different translation criteria in each Gospel:
  \emph{morti tradere}, for example, is an exclusive translation of θανατοῦν
  \emph{thanatoûn} found only in the Gospel of Matthew; the same is true of \emph{odio
    habere}, which translates μισεῖν \emph{miseîn}, whereas the translators of Luke and
  John opt for \emph{odisse}.}

\is{St. Jerome}

\subsection{The Latin translation of Greek support-verb constructions}\label{sec:bj:4:2}

When the Greek text of the Gospels contains an SVC, St. Jerome remains faithful to the
principle of literal, word-for-word translation. Out of the 521 occurrences of Greek SVCs
in the Gospels, there are only 19 instances in which the Vulgate does not offer a
corresponding Latin SVC. In other words, only 3.65\% of the Greek SVCs are not translated
with Latin SVCs.

\is{Vulgate}

Let us take a closer look at these exceptions, drawing a link with other less literal
translations of Greek SVCs. We will distinguish for this purpose three types of examples
on a scale from less to more literal.

\begin{enumerate}
\renewcommand{\labelenumi}{(\roman{enumi})}

\item A Greek SVC is translated in the Vulgate as a synthetic predicate. This is the most
  exceptional case and only occurs with χρείαν ἔχειν \emph{chreían échein} `to need', an
  SVC to which we will return below, and which is translated with four different Latin
  verbs: \emph{desiderare} in \xref{ex:bj:15a}\emph{, egere} in \xref{ex:bj:15b}, \emph{debere} in \xref{ex:bj:15c}, and
  \emph{indigere} in \xref{ex:bj:15d}.

  % 15

\ea\label{ex:bj:15}

\ea\label{ex:bj:15a}

\gllll Τί ἔτι \textbf{χρείαν} \textbf{ἔχομεν} μαρτύρων;\\
 \textit{ti} \textit{éti} \textit{chreían} \textit{échomen} \textit{martýrōn}\\
what yet need have witnesses\\
\emph{Quid} \emph{adhuc} \textbf{\itshape desideramus} ~ \emph{testes?}\\
\glt `What further witnesses do we need?' \\
\hspace*{\fill}(\iwi{NT Mark 14.63})

\ex\label{ex:bj:15b}

\gllll Τί ἔτι \textbf{χρείαν} \textbf{ἔχομεν} μαρτύρων;\\
 \textit{ti} \textit{éti} \textit{chreían} \textit{échomen} \textit{martýrōn}\\
what yet need have witnesses\\
\emph{Quid} \emph{adhuc} \textbf{\itshape egemus} ~ \emph{testibus?}\\
\glt `What further witnesses do we need?' \\
\hspace*{\fill}(\iwi{NT Matthew 26.65})

\ex\label{ex:bj:15c}

\gllll Ἐγὼ \textbf{χρείαν} \textbf{ἔχω} ὑπὸ σοῦ βαπτισθῆναι\\
 \textit{egṑ} \textit{chreían} \textit{échō} \textit{hupò} \textit{soû} \textit{baptisthênai}\\
I need have by you {be baptized}\\
\emph{Ego} {\emph{a} \emph{te} \textbf{\itshape debeo}} ~ ~ ~ \emph{baptizari}\\
\glt `I need to be baptized by you.' \\
\hspace*{\fill}(\iwi{NT Matthew 3.14}).

\ex\label{ex:bj:15d}

\gllll Ὁ λελουμένος οὐκ \textbf{ἔχει} \textbf{χρείαν} εἰ μὴ τοὺς πόδας νίψασθαι\\
 \textit{ho} \textit{lelouménos} \textit{ouk} \textit{échei} \textit{chreían} \textit{ei} \textit{mḕ} \textit{toùs} \textit{pódas} \textit{nípsasthai}\\
the {be washed} not have need if not the feet wash\\
\emph{qui} {\emph{lotus} \emph{est,}} \emph{non} \textbf{\itshape indiget} ~ ~ ~ ~ ~ {\emph{ut} \emph{lavet}}\\
\glt `The one who has bathed does not need to wash, except for his feet.' (\iwi{NT John 13.10})

\z

\z

\item A Greek SVC is translated analytically, not as an SVC, but rather as a complex
  predicate with a verb + adverb, see \xref{ex:bj:16}, or a verb + adjective, see \xref{ex:bj:17}.
  Once again, χρείαν ἔχειν \emph{chreían échein} provides examples of both
  possibilities: \emph{necesse habere} in \xref{ex:bj:16a}, \emph{necessarium esse} in
  \xref{ex:bj:17a}, and \emph{necessarium habere} in \xref{ex:bj:17b}.

% 16

\ea\label{ex:bj:16}

\ea\label{ex:bj:16a}

\gllll Οὐ \textbf{χρείαν} \textbf{ἔχουσιν} οἱ ἰσχύοντες ἰατροῦ ἀλλ' οἱ κακῶς ἔχοντες\\
 \textit{ou} \textit{chreían} \textit{échousin} \textit{hoi} \textit{ischýontes} \textit{iatroû} \textit{all'} \textit{hoi} \textit{kakôs} \textit{échontes}\\
not need have the {be strong} physician but the badly have/be\\
\emph{Non} \textbf{\itshape necesse} \textbf{\itshape habent} ~  \emph{sani} \emph{medicum,} \emph{sed} \emph{qui} \emph{male} \emph{habent}\\
\glt `Those who are well have no need of a physician, but those who are sick.' \\
\hspace*{\fill}(\iwi{NT Mark 2.17})

\ex\label{ex:bj:16b}

\gllll  Ἀγρὸν ἠγόρασα καὶ \textbf{ἔχω} \textbf{ἀνάγκην} ἐξελθὼν ἰδεῖν αὐτóν\\
 \textit{agròn} \textit{ēgórasa} \textit{kaì} \textit{échō} \textit{anánkēn} \textit{exelthṑn} \textit{ideîn} \textit{autón}\\
field buy and have necessity {go out} see it\\
\emph{Villam} \emph{emi} \emph{et} \textbf{\itshape necesse} \textbf{\itshape habeo} {\emph{exire} \emph{et}} \emph{videre} \emph{illam}\\
\glt `I have bought a field, and I must go out and see it.' \\
\hspace*{\fill}(\iwi{NT Luke 14.18}).

\z

\z

% 17

\ea\label{ex:bj:17}

\ea\label{ex:bj:17a}

\gllll Ὁ κύριος αὐτοῦ \textbf{χρείαν} \textbf{ἔχει}\\
 \textit{ho} \textit{kýrios} \textit{autoû} \textit{chreían} \textit{échei}\\
the Lord it need have\\
~ \emph{Domino} ~ {\textbf{\itshape necessarius} \textbf{\itshape est}}\\
\glt `The Lord has need of it.' \\
\hspace*{\fill}(\iwi{NT Mark 11.3})

\ex\label{ex:bj:17b}

\gllll Ὁ κύριος αὐτοῦ \textbf{χρείαν} \textbf{ἔχει}\\
 \textit{ho} \textit{kýrios} \textit{autoû} \textit{chreían} \textit{échei}\\
the Lord it need have\\
\emph{Dominus} ~ \emph{eum} \textbf{\itshape necessarium} \textbf{\itshape habet}\\
\glt `The Lord has need of it.' \\
\hspace*{\fill}(\iwi{NT Luke 19.34}).

\ex\label{ex:bj:17c}

\gllll τί αὐτῇ \textbf{κόπους} \textbf{παρέχετε};\\
 \textit{tí} \textit{autêi} \textit{kópous} \textit{paréchete}\\
why her trouble supply\\
\emph{quid} \emph{illi} \textbf{\itshape molesti} \textbf{\itshape estis?}\\
\glt `Why do you trouble her?' \\
\hspace*{\fill}(\iwi{NT Mark 14.6})\footnote{The same translation of κόπον/κόπους παρέχειν \emph{kópon/kópous paréchein} as \emph{molestum esse} is repeated in \iwi{NT Matthew 26.10}, \iwi{NT Luke 11.7}, and \iwi{NT Luke 18.5}.}

\z

\z

\item A third way in which an SVC is not rendered by means of a strictly literal
  translation is when the text of the Vulgate, although using a Latin SVC, does not employ
  the expected support verb (γίγνεσθαι \emph{gígnesthai}/\emph{fieri}, διδόναι
  \emph{didónai}/\emph{dare}, εἶναι \emph{eînai}/\emph{esse}, ἔχειν
  \emph{échein}/\emph{habere}, ποιεῖν \emph{poi\-eîn}/\emph{facere}, etc.), but opts for
  a more suitable Latin verb or provides various translation alternatives.\footnote{Thus, the 44
    instances of SVCs with γίγνεσθαι \emph{gígnesthai} in the Gospels are translated
    into Latin as \emph{fieri}, except for two specific cases where the translator of Mark
    uses \emph{oriri} (\iwi{NT Mark 4.17}) and \emph{efficere} in the passive\is{passive} (\iwi{NT Mark 6.2}). In the case
    of SVCs with εἶναι \emph{eînai}, in the previously mentioned example \xref{ex:bj:6a}, the Vulgate
    uses \emph{fieri} instead of \emph{esse}, precisely due to its proximity with
    γίγνεσθαι \emph{gígnesthai}. Regarding ἔχειν \emph{échein}, when the predicative noun is the
    subject, Latin does not use \emph{habere} but \emph{invadere} (\iwi{NT Mark 16.8}). A similar
    example is \iwi{NT Luke 2.26}, where λαμβάνειν \textit{lambánein}, instead of its common translation as
    \emph{accipere}, is rendered as \emph{aprehendere}. Other examples of non-literal
    translation include \iwi{NT Mark 14.65} (ῥαπίσμασιν λαμβάνειν \emph{rhapísmasin lambánein} =
    \emph{alapis caedere} `to receive someone with blow, to slap'), \iwi{NT Luke 14.31} (συμβαλεῖν εἰς
    πόλεμον \emph{symbaleîn eis pólemon} = \emph{committere bellum} `to engage in war')
    and \iwi{NT John 3.21} (τὰ ἔργα εἰργασμένα \emph{tà érga eirgasména} = \emph{opera facta sunt}
    `to do works'), the only example in the Gospels where an SVC with ἐργάζεσθαι
    \emph{ergázesthai} is translated as \emph{facere} and not as \emph{operari}
    (\cite{BañosJoséMiguelJiménezLópezM.Dolores-2022707}, e.p.).}

\end{enumerate}

\is{linguistic interference}

  \is{Vulgate}

Since it is not possible to discuss all the examples of this kind, we will focus on those
  SVCs containing the nouns συμβούλιον \emph{symboúlion} and χρείαν \emph{chreían}, as
  they offer a greater variety of translations and, more importantly, can help illustrate
  three crucial aspects of the analysis of Greek SVCs and their Latin translations. From
  the perspective of the original Greek text, SVCs with συμβούλιον \emph{symboúlion} emphasise, on the one
  hand, the interferences between Aramaic\is{linguistic interference} (the native language of the gospel writers),
  Greek, and Latin in the multilingual context in which the Gospels were composed in the
  1st century AD. On the other hand, they reveal the varying proficiency of the gospel
  writers in Greek. From the perspective of the Vulgate, the multiple Latin translations
  of χρείαν ἔχειν \emph{chreían échein} seem to suggest the existence of different
  translation criteria in each Gospel.


\subsubsection{The translation of the support-verb constructions with συμβούλιον symboúlion}\label{sec:bj:4:2:1}

Thus, συμβούλιον \emph{symboúlion} (a calque\is{calque} from the Latin noun \emph{consilium}
`meeting, resolution, counsel') forms three different SVCs in the Gospels
(\cite{JiménezLópez2017}): συμβούλιον ποιεῖν \emph{symboúlion poieîn}, συμβούλιον
διδόναι \emph{symboúlion didónai}, and συμβούλιον λαμβάνειν \emph{symboúlion
  lambánein}.

The first one is translated literally in the Vulgate (\iwi{NT Mark 15.1}: \emph{consilium
  facientes}). However, the other two are approached differently. The sole instance of
\foreignlanguage{greek}{συμβούλιον} διδόναι \emph{symboúlion didónai} is
translated as \emph{consilium facere}, see \xref{ex:bj:18}, instead of \emph{dare}, and
συμβούλιον λαμβάνειν \emph{symboúlion lambánein}, a collocation unique to Matthew (5
instances), is once translated almost literally as \emph{consilium accipere} (\iwi{NT Matthew 28.12}),
but also more freely as \emph{consilium facere}, see \xref{ex:bj:19}, and, most
importantly,\footnote{Apart from example \xref{ex:bj:22}, cf. \iwi{NT Matthew 27.1} (συμβούλιον ἔλαβον
  \emph{symboúlion élabon} = \emph{consilium inierunt)} and \iwi{NT Matthew 27.7} (συμβούλιον λαβόντες
  \emph{symboúlion labóntes} = \emph{consilio inito}).} as \emph{consilium inire} in
\xref{ex:bj:20}:

\is{collocation}
\is{Vulgate}

% 18
% 19
% 20

\ea\label{ex:bj:18}

\gllll καὶ ἐξελθόντες οἱ Φαρισαῖοι εὐθὺς μετὰ τῶν Ἡρῳδιανῶν \textbf{συμβούλιον} \textbf{ἐδίδουν} κατ' αὐτοῦ ὅπως αὐτὸν ἀπολέσωσιν\\
 \textit{kaì} \textit{exelthóntes} \textit{hoi} \textit{Pharisaîoi} \textit{euthùs} \textit{meta} \textit{tôn} \textit{Herōidianôn} \textit{symboúlion} \textit{edídoun} \textit{kat'} \textit{autoû} \textit{hópōs} \textit{autòn} \textit{apolésōsin}\\
and {go out} the Pharisees immediately with the Herodians counsel give against him how him destroy\\
~ {\emph{Exeuntes} \emph{autem} \emph{statim}} ~ \emph{Pharisaei} ~ \emph{cum} ~ \emph{Herodianis}
\textbf{\itshape consilium} \textbf{\itshape faciebant} \emph{adversus} \emph{eum} \emph{quomodo} \emph{eum} \emph{perderent}\\
\glt `The Pharisees went out and immediately held counsel with the Herodians against him, how to destroy him.' \\
\hspace*{\fill}(\iwi{NT Mark 3.6}).

\z

\ea\label{ex:bj:19}

\gllll ἐξελθόντες δὲ οἱ Φαρισαῖοι \textbf{συμβούλιον} \textbf{ἔλαβον} κατ' αὐτοῦ ὅπως αὐτὸν ἀπολέσωσιν\\
  \textit{exelthóntes} \textit{dè} \textit{hoi} \textit{Pharisaîoi} \textit{symboúlion} \textit{élabon} \textit{kat'} \textit{autoû} \textit{hópōs} \textit{autòn} \textit{apolésōsin}\\
{go out} and the Pharisees counsel receive against him how him destroy\\
\emph{Exeuntes} \emph{autem} ~ \emph{Pharisaei} \textbf{\itshape consilium} \textbf{\itshape
  faciebant} \emph{adversus} \emph{eum,} \emph{quomodo} \emph{eum} \emph{perderent}\\
\glt `But the Pharisees went out and conspired against him, how to destroy him.' \\
\hspace*{\fill}(\iwi{NT Matthew 12.14}).

\z

\ea\label{ex:bj:20}

\gllll Τότε πορευθέντες οἱ Φαρισαῖοι \textbf{συμβούλιον} \textbf{ἔλαβον} ὅπως αὐτὸν παγιδεύσωσιν ἐν λόγῳ\\
 \textit{tóte} \textit{poreuthéntes} \textit{hoi} \textit{Pharisaîoi} \textit{symboúlion} \textit{élabon} \textit{hópōs} \textit{autòn} \textit{pagideúsōsin} \textit{en} \textit{lógōi}\\
then go the Pharisees counsel receive how him {lay a snare} in word\\
\emph{tunc} \emph{abuentes} ~ \emph{Pharisaei} \textbf{\itshape consilium}
\textbf{\itshape inierunt} \emph{ut} ~ {\emph{caperent} \emph{eum}}  \emph{in} \emph{sermone}\\
\glt `Then the Pharisees went and plotted how to entangle him in his words.' \\
\hspace*{\fill}(\iwi{NT Matthew 22.15})

\z

It is worth commenting briefly on this variety of seemingly synonymous
SVCs with the same noun, both in the original Greek and the Latin
translation.

(i) In the case of the Greek SVCs with συμβούλιον \emph{symboúlion}
(\cite{JiménezLópez2017}), as in fact in that of any other collocation, our starting point
is Mark, as he is the earliest gospel writer and reveals a higher degree of external
influence in the use of SVCs, undoubtedly reflecting his comparatively lower proficiency
in Greek.

\is{collocation}

Indeed, the SVC συμβούλιον διδόναι \emph{symboúlion didónai} in Mark, see \xref{ex:bj:20}, is
foreign to ancient Greek and, as mentioned above (Section~\ref{sec:bj:3}), is often considered a Hebraism or
Aramaism. Here it does not mean `to advise, counsel' (for which Greek regularly uses the
verb συμβουλεύειν \emph{symbouleúein} in the active voice) but rather `to form a plan,
deliberate, consult'. Perhaps for this reason Matthew, who has Mark's text in
\xref{ex:bj:18} at hand, corrects this unusual collocation by selecting a clearer Greek
expression for the same passage, συμβούλιον λαμβάνειν \emph{symboúlion lambánein}. This,
in turn, is a Latin loan\is{loan word} from \emph{consilium capere}, the prototypical SVC for expressing
the predicate `to form a plan, decide' in classical Latin
(\cite{BañosJoséMiguel-2014631}), namely, at the time when the Greek Gospels were written.

\is{Vulgate}

(ii) In the context of the Vulgate, there is a clear attempt to avoid a literal
translation of example \xref{ex:bj:20} in Mark (συμβούλιον διδόναι \emph{symboúlion
  didónai} \textbf{=} \emph{consilium dare}), since the Latin SVC conveys a different
meaning (`to counsel')\footnote{In \iwi{NT John 18.14}, \emph{consilium dare} is used precisely to
  translate συμβουλεύειν \emph{symbouleúein}.} than the one expressed by the original
Greek (`to deliberate'). Mark's text is thus translated as \emph{consilium facere}, an SVC
which is also employed as a literal translation of συμβούλιον ποιεῖν \emph{symboúlion
  poieîn} (\iwi{NT Mark 15.1}), συμβούλιον λαμβάνειν \emph{symboúlion lambánein}, see \xref{ex:bj:21}, and
συμβουλεύεσθαι \emph{symbouleúesthai} (\iwi{NT Matthew 26.4}) to express in all three cases the
predicate `to deliberate'.

Nevertheless, from a Latin perspective, the use of \emph{consilium facere} is striking, as
it is uncommon in classical Latin,\footnote{According to the data from DiCoLat
  (as of 30/11/2023), which includes the SVCs attested in the textual corpus of the \textit{Packard
  Humanities Institute} (PHI), there are two occurrences of \emph{consilium facere} in
  classical Latin: the first one (\iwi{Quintus Claudius Quadrigarius (2nd-1st c. BC), \textit{Historiae} fr 5}) is fragmentary; and the
  second (\iwi{Livy, \textit{Ab urbe condita} 35.42.8}), with a non-personal subject (\emph{fortuna vel ingenium)}, does
  not convey the same meaning as the biblical examples.} compared to the more frequent
\emph{consilium capere} and \emph{consilium inire}. Indeed, one would have expected
συμβούλιον λαμβάνειν \emph{symboúlion lambánein} to be translated as \emph{consilium
  capere}, an SVC which is nevertheless found nowhere in the Bible. This paradox
ultimately reflects the extent to which there might have been a diachronic renewal in the
use of these collocations over the three centuries that had elapsed between the original
Greek text and the Latin translation of the Vulgate. 


In the 1st century AD, Matthew
employed συμβούλιον λαμβάνειν \emph{symboúlion lambánein} under the influence of the
classical Latin SVC \emph{consilium capere}. However, when the Greek text was translated
into Latin three centuries later, \emph{consilium inire} had already displaced
\emph{consilium capere}\footnote{Indeed, according to the data from DiCoLat, despite the
  prevalence of \emph{capere} over \emph{inire} in classical Latin (129 vs 71 instances),
  both are used with a similar frequency in post-classical Latin (28/25), until \emph{inire}
  took precedence over \emph{capere} in late Latin, to the point that the latter is
  entirely absent from the Vulgate (Old and New Testaments).} as the prototypical
expression of the analytic predicate `to form a plan, to take a decision' and was
therefore given preference over the latter in the Gospel of Matthew (\iwi{NT Matthew 22.15}, \iwi{NT Matthew 27.1},
\iwi{NT Matthew 27.7}). 


In the meantime, a new SVC, \emph{consilium facere}, had emerged in biblical Latin
as a literal translation of συμβούλιον ποιεῖν \emph{symboúlion poieîn} (\iwi{NT Mark
15.1}),\footnote{\citet[126--127]{Burton2000} also mentions \emph{consilium capere}
  `instead of the standard VNCs [verb-noun collocations] \emph{consilium capere} and \emph{consilium inire}, as a
  literal translation of συμβούλιον ποιέω [\emph{symboúlion poiéō}]'. The SVC
  \emph{consilium facere} had already appeared in earlier versions of the \textit{Vetus Latina}, thus
  introducing an SVC which was foreign to Latin but was eventually generalised in the
  Vulgate.} but it also ended up being used to translate συμβούλιον διδόναι
\emph{symboúlion didónai}, see \xref{ex:bj:18}, συμβούλιον λαμβάνειν \emph{symboúlion
  lambánein}, see \xref{ex:bj:19}, and even συμβουλεύεσθαι \emph{symbouleúesthai} `to
deliberate' in a context, such as \xref{ex:bj:21} similar to that of \xxref{ex:bj:18}{ex:bj:20}:

\is{collocation}
\is{Vulgate}
\is{diachrony}

% 21

\ea\label{ex:bj:21}

\gllll καὶ \textbf{συνεβουλεύσαντο} ἵνα τὸν Ἰησοῦν δόλῳ κρατήσωσιν καὶ ἀποκτείνωσιν\\
 \textit{kaì} \textit{synebouleúsanto} \textit{hína} \textit{tòn} \textit{Iēsoûn} \textit{dólōi} \textit{kratḗsōsin} \textit{kaì} \textit{apokteìnōsin}\\
and deliberate {in order that} the Jesus ploy conquer and kill\\
\emph{et} {\textbf{\itshape consilium} \textbf{\itshape fecerunt}} \emph{ut} ~ \emph{Iesum} \emph{dolo} \emph{tenerent} \emph{et} \emph{occiderent}\\
\glt `and plotted together in order to arrest Jesus by stealth and kill him.' \\
\hspace*{\fill}(\iwi{NT Matthew 26.4})

\z

\subsubsection{The translations of χρείαν ἔχειν (\emph{chreían échein})}\label{sec:bj:4:2:2}

Equally interesting are the examples of χρείαν ἔχειν \emph{chreían échein} which, along
with other translation possibilities already discussed  ---  \emph{supra} \xref{ex:bj:15} to
\xref{ex:bj:17} ---, are also rendered with three SVCs in the Vulgate: \emph{necessitatem
  habere} in \xref{ex:bj:22a}, the most literal translation, which however gives rise to an SVC unknown
to classical Latin, as also happens with \emph{opus habere}, see \xref{ex:bj:22b}, and its classical
counterpart \emph{opus esse}, see \xref{ex:bj:22c}:

% 22

\ea\label{ex:bj:22}

\ea\label{ex:bj:22a}

\gllll οὐδέποτε ἀνέγνωτε τί ἐποίησεν Δαυὶδ ὅτε \textbf{χρείαν} \textbf{ἔσχεν} καὶ ἐπείνασεν αὐτὸς καὶ οἱ μετ' αὐτοῦ;\\
 \textit{oudépote} \textit{anégnōte} \textit{tí} \textit{epoíēsen} \textit{Dauìd} \textit{hóte} \textit{chreían} \textit{éschen} \textit{kaì} \textit{epeínasen} \textit{autòs} \textit{kaì} \textit{hoi} \textit{met'} \textit{autoù}\\
never read what do David when need have and {be hungry} himself and the with him\\
\emph{numquam} \emph{legistis} \emph{quid} \emph{fecerit} \emph{David} \emph{quando}
\textbf{\itshape necessitatem} \textbf{\itshape habuit} \emph{et} \emph{esuriit} \emph{ipse} \emph{et} \emph{qui} \emph{cum} \emph{eo} \emph{erant?}\\
\glt `Have you never read what David did, when he was in need and was hungry, he and those who were with him?' \\
\hspace*{\fill}(\iwi{NT Mark 2.25})

\ex\label{ex:bj:22b}

\gllll  Ὁ κύριος αὐτῶν \textbf{χρείαν} \textbf{ἔχει}\\
 \textit{ho} \textit{kýrios} \textit{autôn} \textit{chreían} \textit{échei}\\
the Lord them need have\\
\emph{Dominus} ~ \emph{his} \textbf{\itshape opus} \textbf{\itshape habet}\\
\glt `The Lord needs them.' \\
\hspace*{\fill}(\iwi{NT Matthew 21.3}). 


{[}Compare with \emph{necessarium esse} in \xref{ex:bj:17a} and \emph{necessarium habere} in \xref{ex:bj:17b} for the same passage in the other synoptic Gospels{]}.\is{synoptic Gospels}

\ex\label{ex:bj:22c}

\gllll Οὐ \textbf{χρείαν} \textbf{ἔχουσιν} οἱ ἰσχύοντες ἰατροῦ ἀλλ' οἱ κακῶς ἔχοντες\\
 \textit{ou} \textit{chreían} \textit{échousin} \textit{hoi} \textit{ischýontes} \textit{iatroû} \textit{all'} \textit{hoi} \textit{kakôs} \textit{échontes}\\
not need have the {be strong} physician but the badly have/be\\
\emph{Non} \textbf{\itshape est} \textbf{\itshape opus} ~ \emph{valentibus} \emph{medico,}
\emph{sed} ~ \emph{male} \emph{habentibus}\\
\glt `Those who are well have no need of a physician, but those who are sick.' \\
\hspace*{\fill}(\iwi{NT Matthew 9.12}) 


{[}Compare with \emph{necesse habere} in \xref{ex:bj:16a} for the same passage{]}.

\z

\z

The SVC χρείαν ἔχειν \emph{chreían échein} illustrates the three possible ways of
translating a Greek SVC into Latin discussed in the preceding pages: through various
simplex verbs, as seen in the examples in \xref{ex:bj:15}; through an analytic predicate of
the type verb + adverb, see \xref{ex:bj:16a}, or verb + adjective, see \xref{ex:bj:17a} and
\xref{ex:bj:17b}; and through the three SVCs cited in \xref{ex:bj:22}. In sum, χρείαν
ἔχειν \emph{chreían échein} is rendered through 10 different translations in the
Gospels: \emph{desiderare} in \xref{ex:bj:15a}, \emph{egere} in \xref{ex:bj:15b}, \emph{debere} in
\xref{ex:bj:15c}, \emph{indigere} in \xref{ex:bj:15d}, \emph{necesse habere} in \xref{ex:bj:16},
\emph{necessarium esse} in \xref{ex:bj:17a}, \emph{necessarium habere} in \xref{ex:bj:17b},
\emph{necessitatem habere} in \xref{ex:bj:22a}, \emph{opus habere} in \xref{ex:bj:22b}, and
\emph{opus esse} in \xref{ex:bj:22c}.

Although it would be worthwhile to analyse each of these translations individually\footnote{We will dedicate a specific study to the analysis of the various Latin
  translations. It is worth bearing in mind in this respect that χρείαν \emph{chreían}
  can be constructed absolutely (for instance, in the only example in which it is
  translated as \emph{necessitatem habere}, see \ref{ex:bj:22a}) or, more commonly, with an
  adnominal complement: either a noun in the genitive or, less frequently, an infinitive
  or a subordinate with ἵνα \emph{hína}. In addition, it will be necessary to determine,
  among other aspects, whether this variety of translations reflects a possible polysemy
  of the predicate in Greek, and analyse, from the point of view of Latin, the classicism
  of each possible translation, considering also translations previously attested in
  various versions of the \textit{Vetus Latina}.}, the existence of so many diverse translations for
the same Greek SVC, especially considering the almost inviolable principle (in 96\% of the
cases) that every Greek SVC should be translated with a corresponding Latin SVC, clearly
suggests, in our view, that there was no uniform approach to translating this SVC in the
Gospels, and that St. Jerome's subsequent revision in this respect was either superficial
or nonexistent.

\is{St. Jerome}

This is particularly evident in those passages of the synoptic Gospels\is{synoptic Gospels} which reproduce
Jesus' exact words --- words which are repeated in practically identical form in the
original Greek versions. One would expect that, as sacred words, these would be faithfully
replicated in their respective Latin versions. Nevertheless, the Vulgate does not strictly
adhere to the principle of literal translation. Each Gospel seems to be the work of a
different translator, who attempts to stay faithful to Jesus' words, yet achieves
different results which St Jerome respects and preserves.

\is{Vulgate}

Let us focus on the three most representative passages. In the first one, responding to
the Pharisees' muttering about him and his disciples eating at the house of the tax
collector Levi, Jesus replies in an almost identical manner (`it is not the healthy who
need a physician, but those who are sick') in all three Greek Gospels (\iwi{NT Mark 2.17} and \iwi{NT Matthew
9.12}: Οὐ χρείαν ἔχουσιν οἱ ἰσχύοντες ἰατροῦ \emph{Ou chreían échousin hoi ischýontes
  iatroû}; \iwi{NT Luke 5.31}: Οὐ χρείαν ἔχουσιν οἱ ὑγιαίνοντες ἰατροῦ \emph{Ou chreían échousin
  hoi hygiaínontes iatroû}). However, the Latin translation of Jesus' words is different:
\emph{non \textbf{necesse habent} sani medicum} (Mark), \emph{non \textbf{est opus}
  valentibus medico} (Matthew), and \emph{non \textbf{egent} qui sani sunt medico} (Luke).

In the second passage, just before his triumphant entry into Jerusalem, Jesus sends two
disciples to a village to bring him a donkey tied to a colt. He instructs them that should
anyone question them, they should simply reply, `The Lord needs it/them'. The wording in
Greek is the same in all three Gospels (repeated twice in Luke), with a slight variation
in number: Ὁ κύριος αὐτοῦ χρείαν ἔχει \emph{Ho kýrios autoû chreían échei} (\iwi{NT Mark 11.3}, \iwi{NT Luke
19.31}, \iwi{NT Luke 19.34}) / Ὁ κύριος αὐτῶν χρείαν ἔχει \emph{Ho kýrios autôn chreían échei} (\iwi{NT Matthew
21.3}). However, in the Vulgate, four different translations are provided: \emph{Domino
  \textbf{necessarius est}} (Mark), \emph{Dominus his \textbf{opus habet}} (Matthew),
\emph{Dominus operam eius \textbf{desiderat}} (\iwi{NT Luke 19.31}), and \emph{Dominus eum
  \textbf{necessarium habet}} (\iwi{NT Luke 19.34}).

\is{Vulgate}

Finally, when Jesus is arrested and brought to the house of the high priest Caiaphas, the
latter asks him whether he truly is the Messiah, the Son of God, to which Jesus responds,
`You have said it'. Caiaphas exclaims in shock: `What need do we have of any more
witnesses?' Once again, Caiaphas' words in Greek are almost the same in all three gospel
writers (Τί ἔτι χρείαν ἔχομεν μαρτύρων; \emph{Tí éti chreían échomen martýrōn?} in \iwi{NT Mark
14.63} and \iwi{NT Matthew 26.65}; Τί ἔτι ἔχομεν μαρτυρίας χρείαν; \emph{Tí éti échomen martyrías
  chreían?} in \iwi{NT Luke 22.71}). However, their Latin translations in the Vulgate differ:
\emph{quid adhuc \textbf{desideramus} testes}? (Mark), \emph{quid adhuc \textbf{egemus}
  testibus}? (Matthew) and \emph{quid adhuc \textbf{desideramus} testimonium}? (Luke).

\is{Vulgate}

In our opinion, these examples suggest that there is a different Latin translator behind
each Gospel, a perception that seems to be confirmed when considering all the translation
variants of χρείαν ἔχειν \emph{chreían échein} and their frequency in each gospel writer, as demonstrated in
\tabref{tbl:bj:6}.

\begin{table}
\caption{Different translation options of χρείαν ἔχειν \emph{chreían échein} in
  the Gospels}
\label{tbl:bj:6}
\centering
\small
\begin{tabularx}{\textwidth}{Xrrrr}
\lsptoprule
χρείαν ἔχειν \emph{chreían échein} & Mark & Matthew & Luke & John\\
\midrule
\emph{opus esse} &  & 6.8, 9.12 &  & 2.25, 13.29, 16.30\\
\emph{necessitatem habere} & 2.25 &  &  & \\
\emph{opus habere} &  & 21.3 &  & \\
\emph{necesse habere} & 2.17 & 14.16 &  & \\
\emph{necessarium esse} & 11.3 &  &  & \\
\emph{necessarium habere} &  &  & 19.34 & \\
\emph{desiderare} & 14.63 &  & 19.31, 22.71 & \\
\emph{debere} &  & 3.14 &  & \\
\emph{egere} &  & 26.65 & 5.31 & \\
\emph{indigere} &  &  & 9.11, 15.7 & 13.10\\
\lspbottomrule
\end{tabularx}
\end{table}

As can be observed, each Gospel translation has its own distinctive characteristics. The
translator of Mark employs two exclusive SVCs for χρείαν ἔχειν \emph{chreían échein},
\emph{necessitatem habere} in \xref{ex:bj:13a} and \emph{necessarium esse} in \xref{ex:bj:17a},
both of which are not attested in the other Gospels. The former, a result of extreme
literalness, is also unfamiliar in Latin.

\is{collocation}



The translator of the Gospel of Matthew also provides two unique translation alternatives:
\emph{opus habere} in \xref{ex:bj:22b}, an SVC attested only in late Latin and, more
specifically, in Christian Latin, and the verb \emph{debere} in \xref{ex:bj:15c}, a
surprising choice for a collocation like χρείαν ἔχειν \emph{chreían échein}, which
always expresses necessity in Greek. However, in this specific context (when Jesus
presents himself to John to be baptised) the Latin translator imbues it with an additional
sense of obligation.

On the other hand, the translator of Luke is the only one who avoids using a parallel
Latin SVC in all six instances in which χρείαν ἔχειν \emph{chreían échein} appears. Only
once does he use the analytic predicate \emph{necessarium habere}, see \xref{ex:bj:17b}, a choice that is
also unique to this Gospel. In the remaining five examples, he consistently employs
synthetic predicates: \emph{desiderare}, \emph{egere}, and \emph{indigere}.

Finally, the translator of John takes a radically different approach from that of Luke.
Except for one instance in which the verb \emph{indigere} is used, see \xref{ex:bj:15d}, in the
rest of the cases he uses \emph{opus esse}, which must have been the most natural
translation of χρείαν ἔχειν \emph{chreían échein} from the perspective of classical
Latin, had a uniform translation criterion been applied to this Greek SVC.

\is{Vetus Latina}
\is{St. Jerome}
\is{Vulgate}

Ultimately, we have four Gospels and four distinct translation principles. Faced with the
differences of these early translations (for all of them are found in manuscripts of the
\emph{Vetus Latina}), St. Jerome did not opt for a unifying criterion in his revision. This holds
true, at least, for the three passages in the synoptic Gospels\is{synoptic Gospels} just discussed, in which
Jesus' exact words are reproduced. Interestingly, his words remain the same across the
various synoptic Gospels in Greek but vary in the Vulgate version of each Gospel.

\section{Conclusions and prospects}\label{sec:bj:5}

By way of conclusion, the general data we have discussed regarding the use of SVCs in the
Gospels, both in the original Greek version and the Latin translation of the Vulgate,
allow us to draw some important conclusions and, at the same time, lay out new avenues for
research which we hope to address in future studies.

The frequent occurrence of collocative verbs in the original Greek text, such as ποιεῖν
\emph{poieîn} `to do', γίγνεσθαι \emph{gígnesthai} `to happen', εἶναι \emph{eînai}
`to be', διδόναι \emph{didónai} `to give', ἔχειν \emph{échein} `to have', or λαμβάνειν
\emph{lambánein} `to take', is partially due to the fact that they complement each other
and enrich the collocational pattern of many predicative nouns by expressing the same
event from perspectives which are different from those of the corresponding synthetic
predicates.

\is{collocative verb}

Although our analysis of Greek SVCs has primarily been based on a synchronic approach, we
have also noted the need for a diachronic focus. From a synchronic perspective, we have
highlighted some significant quantitative and qualitative differences among the four
gospel writers in the use of SVCs. John, for example, not only shows the highest frequency
of SVCs but also the highest number of unique SVCs, while the exact opposite situation is
observed in Matthew. These and other differences reveal, on the one hand, the
idiosyncratic nature of this type of collocations, and, on the other hand, the level and
quality of Greek employed by each writer. SVCs, situated halfway between lexicon and
syntax due to their degree of fixation, ultimately pose a challenge for second-language
users, such as the authors of the Gospels.\footnote{Most of the New Testament authors were
  L2 (second-language) Greek users, except perhaps Luke, who may have been an L1 (first-language)
  user (\cite[vol. IV]{MHTMoultonTurner1906}, \cite{PorterStanleyE-2014141}).\is{L2 users}} Their
study, therefore, can help shed light on the level of linguistic competence of each Gospel
writer.

\is{synchrony}
\is{collocation}
\is{diachrony}

To accomplish this, it is also important to adopt a diachronic perspective and
differentiate between those SVCs that are remnants of classical Greek, e.g.
\foreignlanguage{greek}{πορείαν} ποιεῖσθαι \emph{poreían poieîsthai} `to go, to walk' or
δεήσεις ποιεῖσθαι \emph{deḗseis poi\-eîsthai} `to pray, to make a prayer', and those
that represent innovations. The latter either reflect the renewal of these complex
predicates in koine\is{Koine} Greek (for example, the use of the active voice of the support verb
ποιεῖν \textit{poieîn} instead of the middle, as in φόνον ποιεῖν \emph{phónon poieîn} `to murder, to
commit murder' or κρίσιν ποιεῖν \emph{krísin poieîn} `to judge, to make a judgement') or
result from linguistic influences from other languages, such as Hebrew and Aramaic (e.g.
τὴν ἀνομίαν ἐργάζεσθαι \emph{tḕn anomían ergázesthai} `to commit iniquity, to act
lawlessly' or συμβούλιον διδόναι \emph{symboúlion didónai} `to deliberate, to form a
plan') or Latin: συμβούλιον λαμβάνειν \emph{symboúlion lambánein} \textasciitilde{}
\emph{consilium capere} `to form a plan, deliberate', κῆνσον διδόναι \emph{kênson
  didónai} \textasciitilde{} \emph{censum dare} `to tax, to pay tax', or ἐργασίαν
διδόναι \emph{ergasían didónai} \textasciitilde{} \emph{operam dare} `to make an
effort, to give attention to' are noteworthy in this regard. This diachronic perspective
and the linguistic influences on specific SVCs constitute areas that still require further
research.

Moreover, the analysis of the Latin text of the Vulgate has allowed us to compare the use
of these constructions in both languages and consider the translation principles at play.
It became clear in this respect that there is a tension between the desire for a literal
translation (when a Greek SVC finds a parallel translation in Latin) and the need for
linguistic naturalness in Latin (when a Latin SVC corresponds to a synthetic predicate in
Greek).

The quest for a literal translation of the original Greek text explains the limited use of
these complex predicates in the Vulgate compared to the whole body of Latin literature, a
phenomenon which is ultimately related to the lower frequency of the SVCs in Greek than in
Latin.

This principle of literal translation can clearly be seen in the way in which Greek SVCs
are almost always translated into Latin in a parallel fashion, occasionally creating
combinations (συμβούλιον ποιεῖν \emph{symboúlion poieîn} =\linebreak \emph{consilium
  facere}, χρείαν ἔχειν \emph{chreían échein} = \emph{neccesitatem habere}, \emph{opus
  habere}) which are uncharacteristic of classical Latin. The few exceptions in which the
Greek SVCs are not translated literally in the Vulgate are therefore particularly
significant. The two most interesting cases in this regard are the SVCs with συμβούλιον
\emph{symboúlion} and χρείαν \emph{chreían}. Their varied translations into Latin,
apart from highlighting linguistic influences, reveal the existence of different
translation criteria in each Gospel --- an aspect that merits further exploration. The study
of the Latin SVCs that correspond to synthetic predicates in Greek, with their multiple
variants and possibilities,\footnote{Cf. note 16.} can throw ample light on this matter.
This will be the focus of a future study.

%if we resolve in BibTeX file, just comment out abbreviations
\section*{Abbreviations}
\begin{tabularx}{.5\textwidth}{@{}lQ@{}}
% BDF & \citealt{BDFBlassFunk1961} \\
% ESV & \citealt{EnglishStandardVersion2007} \\
% MHT & \citealt{MHTMoultonTurner1906} \\
NT & New Testament \\
\end{tabularx}


\section*{Acknowledgements}
This study is part of the research project
  \emph{Interacción del léxico y la sintaxis en griego antiguo y latín 2:}
  \emph{Diccionario de Colocaciones Latinas} (\cite{banos_dicolat_nodate})
  \emph{Diccionario de Colocaciones del Griego Antiguo}
  (\cite{jimenez_lopez_dicogra_nodate}) (PID2021-125076NB-C42), funded by the Spanish
  Ministry of Science and Innovation.


\sloppy
 \printbibliography[heading=subbibliography,notkeyword=this]

\end{document}


