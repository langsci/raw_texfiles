\documentclass[output=paper,colorlinks,citecolor=brown]{langscibook}
\ChapterDOI{10.5281/zenodo.14017941}
\author{Victoria Beatrix Fendel\affiliation{University of Oxford}}
\title{Epilogue: Taking wing}
\abstract{This epilogue reflects on the shift in perspective between \textit{taking initiative} which we began with and \textit{taking wing} which we end on. It further sets out desiderata in the study of support-verb constructions, namely suitably annotated large-scale corpora, their coverage in authoritative lexicon resources, and their visibility in grammar books. It explains why and how support-verb constructions have so far-reaching an impact, using three poignant examples from Homer’s \textit{Odyssey} (epic), Thucydides’ \textit{Histories} (historiography), and Lysias’ courtroom speeches (oratory). The epilogue finishes by outlining four concrete avenues for further research, namely corpora, corpus-language annotation procedures, cooperation with educators, and collaboration between disciplines. 

\bigskip

Dieser Epilog zieht Bilanz in Bezug auf den Perspektivenwechsel betreffend Konstruktionen wie z.B. \textit{to take initiative} „die Initiative ergreifen“ im Gegensatz zu \textit{to take wing} „Flügel bekommen“ (metaphorisch), den wir durchlaufen haben. Er zeigt dabei Desiderata in der Forschung im Hinblick auf support-verb constructions auf, wie die Existenz von großen Korpora mit entsprechender Annotation, ihre Erfassung in einschlägigen lexikalischen Ressourcen sowie ihre Sichtbarmachung in Referenzgrammatiken. Anhand von drei aussagekräftigen Beispielen aus Homers Ilias (Epos), Thukydides Historien (Historiographie), and Lysias Gerichtsreden (Rhetorik) wird erklärt, wie und warum support-verb constructions einen so weitreichenden Einfluss haben. Der Epilog schließt mit vier konkreten Vorschlägen für künftige Forschung im Gebiet der support-verb constructions. Diese sind die Erstellung großer kommentierter Korpora, die Etablierung von Annotationsschemata und -verfahren, die  auf Korpussprachen abgestimmt sind, die Kooperation mit Lehrkräften, und eine stärkere Zusammenarbeit von Fachdisziplinen in diesem Rahmen. 

}


\IfFileExists{../localcommands.tex}{
   \addbibresource{../localbibliography.bib}
   % add all extra packages you need to load to this file

\usepackage{tabularx,multicol}
\usepackage{url}
\urlstyle{same}

\usepackage{listings}
\lstset{basicstyle=\ttfamily,tabsize=2,breaklines=true}

\usepackage{langsci-basic}
\usepackage{langsci-optional}
\usepackage{langsci-lgr}
\usepackage{langsci-osl}
% \usepackage{./langsci/styles/langsci-lgr}
% \usepackage{./langsci/styles/langsci-osl}
% \usepackage{langsci-gb4e}

\usepackage{tikz}
\usetikzlibrary{patterns,calc}
\pgfdeclarepatternformonly{south east lines}{\pgfqpoint{-0pt}{-0pt}}{\pgfqpoint{3pt}{3pt}}{\pgfqpoint{3pt}{3pt}}{
    \pgfsetlinewidth{0.6pt}
    \pgfpathmoveto{\pgfqpoint{0pt}{3pt}}
    \pgfpathlineto{\pgfqpoint{3pt}{0pt}}
    \pgfpathmoveto{\pgfqpoint{.2pt}{-.2pt}}
    \pgfpathlineto{\pgfqpoint{-.2pt}{.2pt}}
    \pgfpathmoveto{\pgfqpoint{3.2pt}{2.8pt}}
    \pgfpathlineto{\pgfqpoint{2.8pt}{3.2pt}}
    \pgfusepath{stroke}}
    
\usepackage{stmaryrd}
\usepackage{wasysym}
\usepackage{multirow}
\usepackage{caption}
\usepackage{subcaption}
\usepackage{mathrsfs}
\usepackage{qtree}

\usepackage{linguex}


   %pminos do not split footnotes
% \interfootnotelinepenalty=10000 %Footnote in Laporte chapters has to be split SN


%\DeclareIndexNameFormat{default}{%
%\nameparts{#1}%
%\usebibmacro{index:name}%
%{\index[names]}%
%{\namepartfamily}%
%{\namepartgiveni}%
% {}% L1
% {}% L2
%{\namepartprefix}% generates spurious space L3
%{\namepartsuffix}% generates spurious space L4
%}

%  {\DeclareIndexNameFormat{default}{%
%     \usebibmacro{index:name}{\index[names]}{#1}{#3}{#5}{#7}}}

%\DeclareIndexNameFormat{default}{%
%  \usebibmacro{index:name}{\sindex[nom]}{#1}{#3}{#5}{#7}}

%\DeclareIndexNameFormat{default}{%
%  \usebibmacro{index:name}{\sindex[person]}{#1}{#3}{#5}{#7}}
%\DeclareIndexNameFormat{default}{%
%\nameparts{#1} \usebibmacro{index:name}{\sindex[person]]}{\namepartfamily}{‌​\namepartgiven}{\nam‌​epartprefix}{\namepa‌​rtsuffix}}

%\newcommand{\smiley}{:)}

%\renewbibmacro*{index:name}[5]{%
%\usebibmacro{index:entry}{#1}%
%{\iffieldundef{usera}{}{\thefield{usera}\actualoperator}\mkbibindexname{#2}{#3}{#4}{#5}}}

% \newcommand{\noop}[1]{}

%remove for final
%\overfullrule=1mm

\newcommand{\tobi}[2]}}
\renewcommand{\S}[1]{\tobi{#1}{\textsc{*}}}

% this volume references
% puts: [this volume]
% already defined: \citetv
%\newcommand{\citepv}[1]{(\citeauthor{#1} \citeyear*{#1} [this volume])}
\newcommand{\citealtv}[1]{\citeauthor{#1} \citeyear*{#1} [this volume]}

%parentheses around example number
\newcommand{\pref}[1]{(\ref{#1})}

% in-text examples

\newcommand{\lnex}[1]{\textit{#1}} %target lang word
\newcommand{\lnlit}[1]{(lit.: `#1')} %literal reading
\newcommand{\lnlat}[1]{(#1)} % latinization
\newcommand{\lntrans}[1]{`#1'} %translation
\newcommand{\lnexl}[2]%
{\lnex{#1}{} \lnlat{#2}} % ex with latinization
\newcommand{\lnexlat}[3]{\lnex{#1}{} \lnlat{#2}{} \lntrans{#3}} % ex with latinization and tranl.

%ch01
\newcommand{\co}[1]{\mbox{\textbf{#1}}}

%ch09

\newcommand{\cyrbulg}[1]{\begin{otherlanguage*}{bulgarian}#1\end{otherlanguage*}}


%ch10
\newcommand{\nlp}{{\small NLP}}
\newcommand{\mwe}{{\small MWE}}
\newcommand{\rae}{{\small RAE}}
\newcommand{\lvc}{{\small LVC}}
\newcommand{\pos}{{\small P}o{\small S}}
%\newcommand{\todo}[1]{ \textcolor{red}{#1} }

%\renewcommand{\labelenumi}{\theenumi}
%\ainamefmt{{vv}{ll}{, ff}{, jj}} % fullname

\newcommand{\biberror}[1]{{\color{red}#1}}

\newcommand{\osenovaitem}{--~}
   %% hyphenation points for line breaks
%% Normally, automatic hyphenation in LaTeX is very good
%% If a word is mis-hyphenated, add it to this file
%%
%% add information to TeX file before \begin{document} with:
%% %% hyphenation points for line breaks
%% Normally, automatic hyphenation in LaTeX is very good
%% If a word is mis-hyphenated, add it to this file
%%
%% add information to TeX file before \begin{document} with:
%% %% hyphenation points for line breaks
%% Normally, automatic hyphenation in LaTeX is very good
%% If a word is mis-hyphenated, add it to this file
%%
%% add information to TeX file before \begin{document} with:
%% \include{localhyphenation}
\hyphenation{
    Beck-man
    Ngu-yen
    back-chan-nel
    back-chan-nels
    mo-not-o-nous
    ste-reo-typ-i-cal
}

\hyphenation{
    Beck-man
    Ngu-yen
    back-chan-nel
    back-chan-nels
    mo-not-o-nous
    ste-reo-typ-i-cal
}

\hyphenation{
    Beck-man
    Ngu-yen
    back-chan-nel
    back-chan-nels
    mo-not-o-nous
    ste-reo-typ-i-cal
}

   \boolfalse{bookcompile}
   \togglepaper[23]%%chapternumber
}{}

\begin{document}

%extra emergency stretch to resolve remaining overfull hboxes.
\emergencystretch 3em
\maketitle

The \textit{Oxford English Dictionary} (s.v. epilogue 3a) defines an epilogue in a theatrical context as “[a] speech or short poem addressed to the spectators by one of the actors after the conclusion of the play”. In this sense, this epilogue rather than \textit{taking stock} or \textit{drawing conclusions} takes wing in that it briefly comments on what we hope will come next. 


Purposefully, the \textit{proemium} was entitled \textit{taking initiative}, a support-verb construction that few would object to as the noun \textit{initiative} is eventive and encodes inchoativity by itself. Conversely, some may have objected to \textit{taking wing} being analysed as a support-verb construction early on when reading this volume, and some contributions in this volume do object (Ittzés [Chapter 1], Giouli [Chapter 2], and Pompei, Pompeo, and Ricci [Chapter 9]). We have pushed the boundaries with the chapters of this volume as regards approaches to support-verb constructions, corpora of Greek, and the interpretation of interfaces. As Squeri [Chapter 5] (similarly to \citealt{radimskyNomsPredicatifsNoms2011}) has shown, concrete nouns such as \textit{wing} can be reconceptualised as eventive in support-verb constructions. Support verbs can indicate aspect and voice (see Jiménez López and Baños [Chapter 4], Madrigal Acero [Chapter 3], and Vives Cuesta [Chapter 7]), even when morphologically functioning as clitics (Miyagawa [Chapter 10]). Crucially, we are not \textit{winging} it but \textit{taking wing}. What seems to be a formally related base-verb construction (see Veteikis [Chapter 6]) at first sight turns out to be semantically fundamentally different (see Ryan [Chapter 8]).


\section{Desiderata}

As support-verb constructions are highly susceptible to variation, we would need \textbf{diatopically, diastratically, and diachronically diverse corpora}, including those that are rather invisible in the current research landscape, \textbf{annotated for support-verb constructions}. Interest had focussed on three aspects which we have gone beyond. Firstly, instead of focussing only on a specific (small) range of support verbs (‘to do’, ‘to put’, ‘to have’, and ‘to give’), various chapters have discussed e.g. the verb ‘to use’. Secondly, instead of accepting only deverbal and non-deverbal eventive nouns as predicative nouns, several chapters questioned this approach and instead considered how nouns can be reconceptualised in support-verb constructions (Squeri [Chapter 5]) and how the polysemy of many nouns plays into their use in support-verb constructions (Pompei, Pompeo, and Ricci [Chapter 9]). Thirdly, instead of relying on a small range of very visible corpora including the Homeric epics \citep{bakkerLanguageHomer2020, vanseverenEhoVerbeSens1995, schutzeichelIndogermanischeFunktionsverbgefuege2014}, classical literary Attic, and New Testament corpora, we have included e.g. classical technical texts and later hagiographical corpora. 


Secondly, as support-verb constructions show significant lexical variability and can be collocations or idioms in Mel’čuk’s sense, they would need to be \textbf{integrated in dictionaries not as prose phrases or idioms but as a category in their own right}. For example, one of the better catalogued support-verb-construction families is that around δίκη \textit{dike}, shown in (\ref{ex:LSJ}). The reason for the support-verb-construction family around δίκη \textit{dikē} ‘judgement, penalty’ having found a place in the dictionary in the first instance is likely the idiomatic nature of its most frequent exponents, i.e. δίκην δίδωμι \textit{dikēn didōmi} ‘to pay the price for one's actions’ and δίκην λαμβάνω \textit{dikēn lambanō} ‘to exact punishment (from)’. 


\eanoraggedright\label{ex:LSJ}
Liddell-Scott-Jones s.v. δίκη \textit{dikē} IV.3


\glt the object or consequence of the action, atonement, satisfaction, penalty, δίκην ἐκτίνειν, τίνειν [\textit{dikēn ektinein}, \textit{tinein}], Hdt.9.94, S.Aj.113: adverbially in acc., τοῦ δίκην πάσχεις τάδε [\textit{tou dikēn paskʰeis tade}]; A.Pr.614; freq. δίκην or δίκας διδόναι [\textit{dikēn} or \textit{dikas didonai}] suffer punishment, i. e. make amends (but δίκας δ. [\textit{dikas d.}], in A.Supp.703 (lyr.), to grant arbitration); δίκας διδόναι τινί τινος [\textit{dikas didonai tini tinos}] Hdt.1.2, cf. 5.106; ἔμελλε τῶνδέ μοι δώσειν δίκην [\textit{emelle tōnde moi dōsein dikēn}] S.El.538, etc.; also ἀντί or ὑπέρ τινος [\textit{anti} or \textit{huper tinos}], Ar.Pl. 433, Lys.3.42; also δίκην διδόναι ὑπὸ θεῶν [\textit{dikēn didonai ʰupo tʰeōn}] to be punished by . . , Pl. Grg.525b; but δίκας ἤθελον δοῦναι [\textit{dikas ētʰelon doũnai}] they consented to submit to trial, Th.1.28; δίκας λαμβάνειν sts. = δ. διδόναι [\textit{dikas lambanein} sts. = \textit{d. didonai}], Hdt.1.115; δίκην ἀξίαν ἐλάμβανες [\textit{dikēn axian elambanes}] E.Ba.1312, Heracl.852; more freq. its correlative, inflict punishment, take vengeance, Lys.1.29, etc.; λαβεῖν δίκην παρά τινος [\textit{labein dikēn para tinos}] D.21.92, cf.9.2, etc.; so δίκην ἔχειν [\textit{dikēn ekʰein}] to have one's punishment, Antipho 3.4.9, Pl.R.529c (but ἔχω τὴν δ. [\textit{ekʰō tēn d.}] have satisfaction, Id.Ep.319e; παρά τινος [\textit{para tinos}] Hdt.1.45); δίκας or δίκην ὑπέχειν [\textit{dikas} or \textit{dikēn ʰupekʰein}] stand trial, Id.2.118, cf. S. OT552; δίκην παρασχεῖν [\textit{dikēn paraskʰein}] E.Hipp.50; θανάτου δίκην ὀφλεῖν ὑπό τινος [\textit{tʰanatou dikēn o$p^h$lein ʰupo tinos}] to incur the death penalty, Pl.Ap.39b; δίκας λαγχάνειν τινί [\textit{dikas lagkʰanein tini}] D.21.78; δίκης τυχεῖν παρά τινος [\textit{dikēs tukʰein para tinos}] ib.142; δίκην ὀφείλειν, ὀφλεῖν [\textit{dikēn o$p^h$eilein, o$p^h$lein}], Id.21.77, 47.63; ἐρήμην ὀφλεῖν τὴν δ. [\textit{erēmēn o$p^h$lein tēn d.}] Antipho 5.13; δίκην φεύγειν [\textit{dikēn $p^h$eugein}] try to escape it, be the defendant in the trial (opp. διώκειν [\textit{diōkein}] prosecute), D. 38.2; δίκας αἰτέειν [\textit{dikas aiteein}] demand satisfaction, τινός [\textit{tinos}] for a thing, Hdt.8.114; δ. ἐπιτιθέναι τινί [\textit{d. epiti-tʰenai tini}] Id.1.120; τινός [\textit{tinos}] for a thing, Antipho 4.1.5; δίκαι ἐπιφερόμεναι [\textit{dikai epi$p^h$eromenai}] Arist.Pol.1302b24; δίκας ἀφιέναι τινί [\textit{dikas a$p^h$ienai tini}] D.21.79; δίκας ἑλεῖν [\textit{dikas ʰelein}], v. ἔρημος [\textit{erēmos}] II; δίκην τείσασθαι [\textit{dikēn teisastʰai}], v. τίνω [\textit{tinō}] II; δὸς δὲ δίκην καὶ δέξο παρὰ Ζηνί [\textit{dos de dikēn kai dexo para Zēni}] h.Merc.312; δίκας διδόναι καὶ λαμβάνειν παρ’ ἀλλήλων [\textit{dikas didonai kai lambanein par’ allēlōn}], of communities, submit causes to trial, Hdt.5.83; δίκην δοῦναι καὶ λαβεῖν ἐν τῷ δήμῳ [\textit{dikēn dounai kai labein en tō dēmō}] X.Ath.1.18, etc.; δίκας δοῦναι καὶ δέξασθαι [\textit{dikas dounai kai dexastʰai}] submit differences to a peaceful settlement, Th.5.59.


(transcriptions and boldface were added, Liddell-Scott-Jones provides a full list to abbreviations used\footnote{\url{https://stephanus.tlg.uci.edu/lsj/05-general_abbreviations.html} (last accessed 23 April 2024).}, abbreviations are not resolved here)
\z


However, the distinction between support verbs and verbs of realisation is not made \citep{fendelHavenGotClue2023}, modifications (such as pluralisation or determiner phrases) triggering meaning changes are listed as exceptions (“but”), collocations and idioms (in Mel’čuk’s sense) are mixed indiscriminately \citep{fendelSupportverbConstructionsObjects2023, fendelNotSemilexicalAffixessubmitted}. The entry could be reorganised e.g. by drawing on the notion of support-verb-construction families and subdividing entries along the lines of Mel’čuk’s compositional vs. non-compositional semantic-lexemic phrasemes (collocations vs. idioms) \citep{melcuk_general_2023}. We would thus distinguish between active collocation, active idiom, passive collocation, passive idiom, aspectual collocation, aspectual idiom, etc. A further caveat regards the text type from which the examples referenced come as support-verb constructions are susceptible to pragmatic indexing. 


Thirdly, support-verb constructions sit at three interfaces, such that in addition to the lexical notions of collocation and idiom, the morphological notion of periphrasis and the syntactic notion of complex predicate have been discussed in this volume. They would need to be \textbf{integrated in grammar books}, similarly to what we find in Latin. \citet[74–77]{pinksterOxfordLatinSyntax2015} dedicates a subsection in his chapter on verb frames in Latin to support verbs. The situation is considerably different in Greek. While Kühner and Gerth’s classical \textit{Ausführliche Grammatik der griechischen Sprache} still has some brief, but insightful notes, shown in \ref{ex:KG}, the newer \textit{Cambridge Grammar of Classical Greek} \citep{vanemdeboasCambridgeGrammarClassical2019} does not account for support-verb constructions. 


\eanoraggedright\label{ex:KG}
\citealt[322]{kuehnerAusfuehrlicheGrammatikGriechischen1894}\footnote{Abbreviations are those used in Liddell-Scott-Jones, see \url{https://stephanus.tlg.uci.edu/lsj/05-general_abbreviations.html} (last accessed 23 April 2024).} 


Statt des einfachen Verbs bedienen sich die Griechen zuweilen einer Umschreibung durch den Akkusativ eines abstrakten Substantivs und die Verben ποιεῖσθαι [\textit{poieistʰai}], τίθεσθαι [\textit{titʰestʰai}], ἔχειν [\textit{ekʰein}], um den Verbalbegriff nachdrücklicher zu bezeichnen, wie συμβολὴν ποιεῖσθαι [\textit{sumbolēn poieistʰai}] Hdt. 6, 110. ὀργὴν π. [\textit{orgēn p.}]  3, 25. 7, 105. ἀπόπειραν π. [\textit{apopeiran p.}] 8, 10. πρόσοδον π. = προσιέναι [\textit{prosodon p. = prosienai}] 7, 223. λήθην π. = ἐπιλανθάνεσθαι [\textit{lētʰēn p. = epilantʰanestʰai}] 1, 127. σκῆψιν π. [\textit{skēpsin p.}] 5,30. μάθησιν ποεῖσθαι = μανθάνειν [\textit{matʰēsin poeistʰai = mantʰanein}] Th. 1, 68).


\bigskip

(my translation) ‘Instead of simplex verbs, the Greeks at times use periphrastic expressions with the accusative case of an abstract noun and verbs such as ποιεῖσθαι [\textit{poieistʰai}], τίθεσθαι [\textit{titʰestʰai}], ἔχειν [\textit{ekʰein}] in order to express the predication with more intensity, e.g. συμβολὴν ποιεῖσθαι [\textit{sumbolēn poieistʰai}] Hdt. 6, 110. ὀργὴν π. [\textit{orgēn p.}]  3, 25. 7, 105. ἀπόπειραν π. [\textit{apopeiran p.}] 8, 10. πρόσοδον π. = προσιέναι [\textit{prosodon p. = prosienai}] 7, 223. λήθην π. = ἐπιλανθάνεσθαι [\textit{lētʰēn p. = epilantʰanestʰai}] 1, 127. σκῆψιν π. [\textit{skēpsin p.}] 5,30. μάθησιν ποεῖσθαι = μανθάνειν [\textit{matʰēsin poeistʰai = mantʰanein}] Th. 1, 68).’

\z


Kühner and Gerth only include support verbs that are common across languages and that form active and stative predicates. Equivalence between the support-verb construction and the simplex verb related to the predicative noun is assumed with the only difference identified being “Nachdruck” (intensity).\footnote{The interest appears stylistic (similarly \citealt{aertsPeriphrasticaInvestigationUse1965} is primarily focussed on the inflexional and not the derivational morphology).}  The examples come primarily from Herodotus’ \textit{Histories}, an early historiographic text in the Ionic dialect, yet support-verb constructions are highly susceptible to diatopic variation \citep{fendelTakingStockGreek2024}.



\section{Relevance}
Support-verb constructions \textbf{permeate} all the corpora of Greek such that they cause issues in canonical or less canonical texts. Support-verb constructions are inherently \textbf{ambiguous} due to the polysemy of the constituent parts (e.g. \citealt{savaryLiteralOccurrencesMultiword2019}) such that they cause issues in any environment. Support-verb constructions sit at three \textbf{interfaces} such that they cause issue to everyone, notwithstanding whether they are interested in the syntax, semantics, or pragmatics of a text. This is illustrated below with three examples from well-known corpora, i.e. where contextual information should be able to aid the modern reader. In all three cases, the correct reading of the support-verb constructions has implications well beyond the sentence(s) quoted, e.g. for the reconstruction of the composition process, for the narratological structure of the narrative, or for the embedding of the text into its socio-political reality. 


Example one, (\ref{ex:Homer}), comes from Homer’s epics (pre 7th c. BC). The support-verb construction of interest is κακὸν εὑρίσκομαι \textit{kakon ʰeuriskomai} ‘to bring harm upon oneself', which is anaphorically resumed in the subsequent sentence by means of the noun phrase μέγα πῆμα \textit{mega pēma} ‘great harm’. The translation of West’s classical edition of the text and of the text containing Probert's editorial suggestion are provided with the example.


\ea\label{ex:Homer}
\glll ἐξ οὗ Κενταύροισι καὶ ἀνδράσι νεῖκος ἐτύχθη, \\
\textit{ex} \textit{ʰou} \textit{Kentaurioisi} \textit{kai} \textit{andrasi} \textit{veikos} \textit{etukʰtʰē} \\
    out.of \textsc{rel}.\textsc{gen} Centaurs.\textsc{dat} and men.\textsc{dat} battle.\textsc{nom} happen.\textsc{aor.ind.pass.3sg} \\
\glll οἷ δ᾽ αὐτῷ πρώτῳ \textbf{κακὸν} \textbf{ηὕρετο} οἰνοβαρείων. \\
\textit{ʰoi} \textit{d’} \textit{autō} \textit{prōtō} \textit{kakon} \textit{ʰēureto} \textit{oinobareiōn} \\
    they.\textsc{nom} \textsc{prt} he.\textsc{dat} first.\textsc{dat} evil.\textsc{acc} find.\textsc{aor.ind.mid.3sg} heavy.with.wine.\textsc{nom} \\
\glll ὣς καὶ σοὶ μέγα πῆμα πιφαύσκομαι […] \\
\textit{ʰōs} \textit{kai} \textit{soi} \textit{mega} \textit{pēma} \textit{piphauskomai} […] \\
    so also you.\textsc{dat} great.\textsc{acc} harm.\textsc{acc} foretell.\textsc{prs.ind.1sg} \\
\glt ‘Out of which arose the battle between centaurs and humans 
but he brought harm upon himself first, being heavy with wine.
In the same way I foretell great harm for you too […]’ (translation of the text as provided by \citealt[447–448]{West2017})


‘Ever since the battle between the centaurs and humans occurred, 
one who is heavy with wine brings harm first and foremost upon himself.
In the same way I foretell great harm for you too […]’ (translation of the text with τ᾽ \textit{t'} instead of δ᾽ \textit{d'} by \citealt{probertExOuKentayroisi2023})\footnote{On Probert’s reading, the support-verb construction appears in a gnomic phrase, a general rule, after which the discourse returns to the main line of events. The anaphoric noun phrase μέγα πῆμα \textit{mega pēma} ‘great harm' acts as the discursive link (cf. \citealt[278]{hallidayCohesionEnglish1976} on reiteration). While the syntactic nominalisation and the lexical noun are not formally related, they are functionally akin. πῆμα \textit{pēma} ‘harm' is a verbal noun from a root *\textit{pē-}, possibly also found in e.g. ταλαίπωρος \textit{talaipōros} ‘enduring hardship' \citep{beekesEtymologicalDictionaryGreek2010}.} \\

\hspace*{\fill}(\iwi{Homer, \textit{Odyssey} 21.303–305} (pre 7th c. BC))
\z

The support-verb construction in question is interesting for two reasons, first-ly since the predicative noun is a syntactic nominalisation rather than a lexical one, and secondly because the support verb is a verb that can appear in various argument frames. 


The syntactic nominalisation κακόν \textit{kakon} ‘evil' has to fill the object slot of the verb (εὑρίσκομαι \textit{ʰeuriskomai} ‘to find'), unlike in constructions with two accusatives (e.g. δίδωμι Χ μισθόν \textit{didōmi} X \textit{mistʰon} ‘to give X as salary’) or in constructions in which the verb could be read intransitively (e.g. ποιέω κακόν \textit{poieō kakon} ‘to act badly’). A support verb meaning ‘to find’ in Greek, as in English, can appear in various argument frames. (\ref{ex:BNC}) illustrates argument frames in English (see \textit{British National Corpus}):

\ea\label{ex:BNC}

‘to find’ in the \textit{British National Corpus}
\ea Paul finds fault with his parents. ≈ Paul blames his parents. [causative]
\ex Paul finds a compromise. ≈ Paul compromises. [active]
\ex Paul finds fame. ≈ Paul becomes famous. [stative] 
\ex Paul finds favour with his parents. ≈ Paul becomes liked by his parents. [passive]

\z
\z

εὑρίσκω/ομαι \textit{ʰeuriskō/omai} ‘to find’ would deserve a study of its own. A cursory look through the literary classical Attic \textit{ECF Leverhulme Corpus} reveals passages such as σπονδὰς εὑρίσκομαι \textit{spondas ʰeuriskomai} ‘to reach a truce’ (\iwi{Thucydides, \textit{Histories} 5.32.6}), contrasting with more frequent σπονδὰς ποιέομαι \textit{spondas poieomai} ‘to make a truce', and φιλίας εὑρίσκω \textit{$p^h$ilias ʰeuriskō} ‘to make friends’ (\iwi{Isocrates, \textit{Speech} 4.45}), akin to \iwi{Euripides, \textit{Electra} l. 650} (tragedy) εὑρίσκεις δὲ μητρὶ πῶς φόνον; \textit{ʰeuriskeis de mētri pōs $p^h$onon } ‘how are you bringing about the murder of the mother?’. The frames seem active and causative. Examples of passive and stative frames appear in the Liddell-Scott-Jones' entry for the verb (s.v. εὑρίσκω \textit{ʰeuriskō} ‘to find' IV middle voice). The passive ones come primarily from passages cited from tragedy and hence predisposed to fall into the category of ‘to suffer, get oneself into, find [something negative such as fate, pain, etc.]’. The stative ones include κλέος εὑρίσκομαι \textit{kleos ʰeuriskomai} ‘to find fame’ (\iwi{Pindar, \textit{Pythiae} 3.111} (lyric poetry),  ἐλπίδ’ ἔχω κλέος εὑρέσθαι \textit{elpid’ ekʰō kleos ʰeuriskestʰai} ‘I hope to gain/find fame’). The issue with the Liddell-Scott-Jones entry is the great variety of dialects, genres, registers, and periods of time evidenced by the examples. Corpus-based studies would be needed to gain a clear picture of the support-verb constructions with εὑρίσκω/ομαι \textit{ʰeuriskō/omai} ‘to find’ by dialect, genre, register, and period of time.

The impression gained is that at least in classical Greek, εὑρίσκω/ομαι \textit{ʰeuriskō/omai} ‘to find’ aligns with ποιέω/ομαι \textit{poieō/omai} ‘to act, to do, to make’ in that the middle ending has a transitivity-reducing function (stative and passive frames). However, this is not a hard-and-fast rule and verb lability allows for middle endings with causative semantics and active endings with reflexive semantics \citep{lavidasTransitivityAlternationsDiachrony2009} at times. Thus, without the syntactic (argument frame, esp. the indirect object), semantic (anaphoric resumption), and pragmatic (gnomic aorist and cue to return to main storyline) cues in (\ref{ex:Homer}), ambiguity abounds.

\largerpage[1.5]
Example two, (\ref{ex:Thuc}), comes from Thucydides’ \textit{Histories} (5th c. BC). The support-verb construction of interest is ἐκβολὴν ποιέομαι \textit{ekbolēn poieomai} which is co-ordina-ted with preceding ἔγραψα \textit{egrapsa} ‘I wrote’. A genitive λόγου \textit{logou} ‘word, plan' is bracketed between the predicative noun and the support verb. 


\ea\label{ex:Thuc}
\glll ἔγραψα δὲ αὐτὰ καὶ τὴν \textbf{ἐκβολὴν} τοῦ λόγου \textbf{ἐποιησάμην} διὰ τόδε, ὅτι (...) \\
\textit{egrapsa} \textit{de} \textit{auta} \textit{kai} \textit{tēn} \textit{ekbolēn} \textit{tou} \textit{logou} \textit{epoiēsamēn} \textit{dia} \textit{tode} \textit{ʰoti} (…) \\
    write.\textsc{aor.ind.act.1sg} \textsc{prt} they.\textsc{acc} and the.\textsc{acc} throwing.away.\textsc{acc} the.\textsc{gen} word/plan.\textsc{gen} make.\textsc{aor.ind.mid.1sg} due.to this.\textsc{acc} that \\

\glt ‘And I have made a digression to write of these matters for the reason that (…)’ \citep[165]{ForsterSmith1928}


‘I have written these things and discarded the plan due to the fact that (...)’ \citep{rustenTenEkvolenTou2020} \\

\hspace*{\fill}(\iwi{Thucydides, \textit{Histories} 1.97.2} (5th c. BC))
\z


The difference between the classical and Rusten’s readings of the passage boils down to (i) the semantics of the (polysemous) predicative noun (‘digressing’ or ‘tossing out’), (ii) the syntactic function of the genitive λόγου \textit{logou} ‘word, plan' (qualitative or objective), and (iii) the semantics of the (polysemous) noun λόγου \textit{logou} ‘narrative' or ‘plan'. \citet[233]{rustenTenEkvolenTou2020} argues that the support-verb construction is “a periphrasis for ἐξέβαλον τὸν λόγον” \textit{exebalon ton logon} meaning ‘to toss out’ (for reasons of consideration or rejection). This assumption entails that the genitive λόγου \textit{logou} is objective for him. \citet[234]{rustenTenEkvolenTou2020} further argues that multi-functional λόγος \textit{logos} does not refer to “a unit of narrative” in Thucydides, as it does in Herodotus. From this, \citet[251]{rustenTenEkvolenTou2020} concludes: “If 1.98–118 were a digression it would not have needed this preface. It is more than a digression like 88–96 (from which it is launched); it is instead a composition that nominally performs the mundane task (as does 5.25–116) of filling a gap in the record, but exploits it to reveal the terrible transformation of Athens from ξύμμαχος [\textit{xummakʰos} ‘ally’] to ἡγεμών [\textit{ʰēgemōn} ‘ruler’] to ἄρχων [\textit{arkʰōn} ‘sole ruler’], and to document the fully developed character of the newborn Athenian Empire.” Rusten’s new reading of the passage has far-reaching implications for the reconstruction of the composition process and the narratological structure of book 1 of the \textit{Histories}.
	
 
 Example three, (\ref{ex:Lysias}), comes from Lysias’ courtroom speeches (5th / 4th c. BC). The support-verb construction of interest is δίκην λαμβάνω \textit{dikēn lambanō} ‘to exact punishment' which is contrasted in a parallel structure (ὅταν \textit{ʰotan} … ἀλλ’ ὅταν \textit{all᾽ ʰotan} ‘whenever ... but whenever') with the simplex verb κολάζω \textit{kolazō} ‘to punish'.


\ea\label{ex:Lysias}
\glll οὐχ ὅταν τοὺς ἀδυνάτους εἰπεῖν κολάζητε, ἀλλ’ ὅταν παρὰ τῶν δυναμένων λέγειν \textbf{δίκην} \textbf{λαμβάνητε} \\
\textit{oukʰ} \textit{ʰotan} \textit{tous} \textit{adunatous} \textit{eipein} \textit{kolazēte}, \textit{all’} \textit{ʰotan} \textit{para} \textit{tōn} \textit{dunamenōn} \textit{legein} \textit{dikēn} \textit{lambanēte} \\
    \textsc{neg} when the.\textsc{acc} unable.\textsc{acc} speak.\textsc{aor.inf.act} punish.\textsc{prs.sbjv.act.2pl} but when from the.\textsc{gen} be.able.\textsc{prs.ptcp.act.gen} speak.\textsc{prs.inf.act} punishment.\textsc{acc} take.\textsc{prs.sbjv.act.2pl} \\

\glt ‘if instead of punishing unskilful speakers you exact requital from the skilful’ \citep[627]{lambLysiasLysiasEnglish1930}


‘not when you punish those who cannot speak/defend themselves, but when you collect punishment from those who are able to speak/defend themselves’ \citep[397]{fendelSupportverbConstructionsObjects2023} \\

\hspace*{\fill}(\iwi{Lysias, \textit{Speech} 30.23–24})
\z

In (\ref{ex:Lysias}), the relationship between the base-verb construction (κολάζω \textit{kolazō} ‘to punish' + accusative object) and the support-verb construction (δίκην λαμβάνω \textit{dikēn lambanō} ‘to exact punishment' + prepositional object with παρά \textit{para} ‘from' + genitive) can perhaps be described of one of hyponymy semantically speaking. 


The support-verb construction describes a specific type of punishing: “Suppose that simple punishment is the act of punishing someone without giving them the chance of defending themselves, i.e. using their rights within the legal framework, whereas punishment using the law (in the sense of ‘exacting justice’) means that the person to suffer the punishment is given the opportunity of a defence within the framework of the law. In the former case, the defendant will suffer punishment without any mediation; in the latter case, it is likely that the severity of the punishment and thus the impact on the one to be punished is mediated by the framework of the law (and the defendant’s defence)” \citep[397]{fendelSupportverbConstructionsObjects2023}. The different encoding of the object indicates the lower degree of affectedness of the object with the support-verb construction. Pointedly, in (\ref{ex:Lysias}), the object of the simplex verb is τοὺς ἀδυνάτους εἰπεῖν \textit{tous adunatous eipein} ‘those unable to speak’ and the object of the support-verb construction is τῶν δυναμένων λέγειν \textit{tōn dunamenōn legein} ‘those who are able to speak’. 


However, there is also a pragmatic index applied to the support-verb construction that the base-verb construction does not have. \citet[123]{benteinDimensionsSocialMeaning2019} considers linguistic indexes ““structures” (lexemes, affixes, diminutives, syntactic constructions, emphatic stress, etc.) that have become conventionally associated with a particular situational dimension, and that invoke that situational dimension whenever they are used \citep[411]{ochsResourcesSocializingHumanity1996}”. While the support-verb construction seems to index the legal framework, the base-verb construction is domain-unspecific.\footnote{The situation is in fact more complicated for δίκην δίδωμι \textit{dikēn didōmi} ‘to pay the price for one's action' and ‘to judge',  which due to its polysemy in different verb frames (akin to simplex verbs with verb profiles) adopts multiple meanings, only one of which is specifically pragmatically indexed \citep{FendelDiversity}.}  


The three passages illustrate (i) how support-verb constructions sit at three interfaces, (ii) how their correct reading can have far-reaching implications for the flow of the narrative, the reconstruction of the composition process, and the embedding of the text into the extra-linguistic reality, and (iii) how the polysemy of many nouns in Greek and the ambiguity inherent in support-verb constructions create a language barrier between us and the ancient native speakers, i.e., the texts. 


\section{Avenues}
The reader will have noticed that the chapters of this volume are suspiciously focussed around literary texts. This is no coincidence but it does in no way mean that support-verb constructions do not appear in papyrological and epigraphic material – in fact, they do in great variety (e.g. \citealt{fendelGreekEgyptEgyptian2021, fendelCopticInterferenceSyntax2022, fendelSupportverbConstructionsObjects2023} on bilingual letter archives, \citealt{fendelSupportverbConstructionsMagicalsubmitted} on the Magical papyri, \citealt{fendelTakingStockGreek2024} on structures with φροντίς \textit{$p^h$rontis} ‘care' and χρεία \textit{kʰreia} ‘need' in the documentary papyri, \citealt{FendelPPSVC} on support verb + prepositional phrase constructions in the documentary papyri).


However, papyrological and epigraphic corpora are less well prepared (as regards lemmatisation, part-of-speech tagging, etc.) than literary ones and often show a great amount of internal heterogeneity. Thus, the absence of chapters on papyrological and epigraphic data is in fact a data-driven issue. Identification and discovery of support-verb constructions is complicated at the best of times (e.g. \citealt{doucetNonContiguousWordSequences2004, sagMultiwordExpressionsPain2002}) and noisy datasets exacerbate the issue. Therefore, the first avenue for further work is a collaborative initiative such as the PARSEME Ancient Greek corpus in order to produce relevant datasets and make them openly available. 


In this context, the question of annotation guidelines arises, discussed e.g. by Giouli [Chapter 1]. Her el-PARSEME corpus applies a natural language processing annotation framework which is comparably narrow in the context of the chapters of this volume but has been tested on datasets in 20+ modern languages. However, this framework comes with a significant number of challenges when assessing corpus languages, as e.g. grammaticality judgements on transformations such as the deletion of the verb or the permissibility of pluralisation on the predicative noun cannot be obtained easily. The native speakers of corpus languages are the texts \citep{fleischmanMethodologiesIdeologiesHistorical2000}. Thus, a second avenue for further work is to synthesise annotation frameworks and consider not only language-specificity as regards pre-modern Greek but also the intricacies of working with a corpus language.


Support-verb constructions are currently seemingly shut into the ivory tower of academic research despite appearing everywhere and posing a challenge to everyone. Yet, language learners still stumble and fall. The PARSEME Ancient Greek working group actively recruits undergraduate students in order to bridge this gap.\footnote{\url{http://www.ancientgreekmwe.com/} (last accessed 23 April 2024).}  An excellent lexical resource has been introduced by Baños and Jiménez López [Chapter 4] in the form of the \textit{Diccionario de Colocaciones del Griego Antiguo}.\footnote{\url{https://dicogra.iatext.ulpgc.es/dicogra/} (last accessed 06 April 2024).} The key issue is that support-verb constructions are not consistently listed in authoritative resources, such as the Liddell-Scott-Jones. John Temple, for example, describes the situation as expressions “buried within articles”.\footnote{Note that his dictionary goes beyond support-verb constructions and is focussed on non-compositional expressions and assembled from the perspective of translation: \url{https://sites.google.com/view/classical-greek-idioms/home}.} Thus, a third avenue for further work is to enhance visibility of support-verb constructions for all those working with the corpora of Greek, e.g. by means of their integration into authoritative grammar books and dictionaries.


The PARSEME corpus shows the very fruitful collaboration between disciplines. This volume on a smaller scale focussed on the diachronic breadth of the corpora of Greek and thus brought together disciplines as far apart as comparative philology, dealing with the reconstructed proto-language, and natural language processing, dealing with large-scale internet corpora. A fourth avenue for further work is to foster collaboration between disciplines. Nobody knows everything but together we know a lot more than each on our own, especially with the sentiment of a dialogue between antiquity and our present \citep{vereeckWhyPlatoNeeds2023}. 


We started with Vergil and Homer, we end with Plato, in that the diversity of structures, approaches, and corpora has amply highlighted all the aspects of support-verb constructions that need and deserve further study. We now know how little we know or in the words of Plato’s Socrates, we know that we know nothing (\iwi{Plato, \textit{Apology} 22d}). 


%\begin{table}
%\caption{Frequencies of word classes}
%\label{tab:myname:frequencies}
% \begin{tabularx}{.8\textwidth}{X rrrr}
%  \lsptoprule
%            & nouns & verbs  & adjectives & adverbs\\
%  \midrule
%  absolute  &   12  &    34  &    23      & 13\\
%  relative  &   3.1 &   8.9  &    5.7     & 3.2\\
%  \lspbottomrule
% \end{tabularx}
%\end{table}


%\is{Cognition} %add "Cogntion" to subject index for this page

%\ea
%\gll cogito                           ergo      sum\\
%     think.\textsc{1sg}.\textsc{pres} therefore \textsc{cop}.\textsc{1sg}.\textsc{pres}\\
%\glt `I think therefore I am.'
%\z
%\il{Latin} %add "Latin" to language index for this page



%\section*{Abbreviations}
%\begin{tabularx}{.5\textwidth}{@{}lQ@{}}
%... & \\
%... & \\
%\end{tabularx}%
%\begin{tabularx}{.5\textwidth}{@{}lQ@{}}
%... & \\
%... & \\
%\end{tabularx}

\section*{Acknowledgements}
%repeat from intro
The project from which this volume arises \textit{Giving gifts and doing favours: Unlocking Greek support-verb constructions} (University of Oxford, 2020–-2024) was funded by the \textit{Leverhulme Trust} (grant n. ECF-2020-181).


%\section*{Contributions}
%John Doe contributed to conceptualization, methodology, and validation. 
%Jane Doe contributed to writing of the original draft, review, and editing.

\sloppy
\printbibliography[heading=subbibliography,notkeyword=this]
\end{document}
