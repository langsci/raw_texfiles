\documentclass[output=paper,colorlinks,citecolor=brown]{langscibook}
\ChapterDOI{10.5281/zenodo.14017935}
\author{Cressida Ryan\affiliation{University of Oxford}}
\title[Support the sinner not the sin]{Support the sinner not the sin: support-verb constructions and New Testament ethical frameworks}

\abstract{ In this chapter, I consider the development of support-verb constructions in New Testament Greek and the potential exegetical impact of philological developments.  I investigate to what extent ἁμαρτάνω \textit{hamartánō} ‘to sin' and the construction ποιῶ ἁμαρτίαν \textit{poiō hamartían} ‘I commit a sin' may be considered synonymous and explore how the use of a support-verb construction may have an exegetical impact of distancing sin from sinner. The noun becomes more frequently used, but remains less frequent than the verb.  In the New Testament, however, the ratio is 4:1. This increase in the use of the noun over the verb makes sin into a substantive, rather than a process.  In doing this, sin can be separated from sinner, made into something which can be removed from them and is not necessarily part of their identity.  This move to a support-verb construction with a noun is also evident with the related noun ἁμάρτημα \textit{hamártēma} ‘sin'.


\bigskip


En el presente artículo, se examina el desarrollo de las construcciones con verbo de apoyo en el Nuevo Testamento y el potencial impacto exegético de nuevos avances filológicos. 
Se estudia el grado en que se puede considerar ἁμαρτάνω \textit{hamartánō} ‘pecar' y la construcción ποιῶ ἁμαρτίαν \textit{poiō hamartían} ‘cometer un pecado’ como sinónimos, y se analiza cómo del uso de una construcción con verbo de apoyo puede tener el impacto exegético de separar el pecado del pecador. 
El uso del sustantivo gana frecuencia, pero sin superar al verbo. En el Nuevo Testamento, sin embargo, la proporción es de 4:1. Este aumento en el uso del nombre sobre el verbo hace que se trate el pecado como un sustantivo, más que como un proceso. 
De esta manera, el pecado puede separarse del pecador, como algo extraíble que no tiene que formar parte de su identidad. 
Esta tendencia a favor de las construcciones con verbo de apoyo y el sustantivo se aprecia también con el sustantivo relacionado ἁμάρτημα \textit{hamártēma} ‘pecado’. 
%En resumen, este capítulo hace un análisis basado en los datos de las construcciones con verbo de apoyo en el Nuevo Testamento, estableciendo una comparación con los otros géneros/dialectos griegos para explorar la relación exegética entre la filología y la teología dentro de un marco ético.
}

\IfFileExists{../localcommands.tex}{
   \addbibresource{../localbibliography.bib}
   \usepackage{langsci-optional}
\usepackage{langsci-gb4e}
\usepackage{langsci-lgr}

\usepackage{listings}
\lstset{basicstyle=\ttfamily,tabsize=2,breaklines=true}

%added by author
% \usepackage{tipa}
\usepackage{multirow}
\graphicspath{{figures/}}
\usepackage{langsci-branding}

   
\newcommand{\sent}{\enumsentence}
\newcommand{\sents}{\eenumsentence}
\let\citeasnoun\citet

\renewcommand{\lsCoverTitleFont}[1]{\sffamily\addfontfeatures{Scale=MatchUppercase}\fontsize{44pt}{16mm}\selectfont #1}
  
   %% hyphenation points for line breaks
%% Normally, automatic hyphenation in LaTeX is very good
%% If a word is mis-hyphenated, add it to this file
%%
%% add information to TeX file before \begin{document} with:
%% %% hyphenation points for line breaks
%% Normally, automatic hyphenation in LaTeX is very good
%% If a word is mis-hyphenated, add it to this file
%%
%% add information to TeX file before \begin{document} with:
%% %% hyphenation points for line breaks
%% Normally, automatic hyphenation in LaTeX is very good
%% If a word is mis-hyphenated, add it to this file
%%
%% add information to TeX file before \begin{document} with:
%% \include{localhyphenation}
\hyphenation{
affri-ca-te
affri-ca-tes
an-no-tated
com-ple-ments
com-po-si-tio-na-li-ty
non-com-po-si-tio-na-li-ty
Gon-zá-lez
out-side
Ri-chárd
se-man-tics
STREU-SLE
Tie-de-mann
}
\hyphenation{
affri-ca-te
affri-ca-tes
an-no-tated
com-ple-ments
com-po-si-tio-na-li-ty
non-com-po-si-tio-na-li-ty
Gon-zá-lez
out-side
Ri-chárd
se-man-tics
STREU-SLE
Tie-de-mann
}
\hyphenation{
affri-ca-te
affri-ca-tes
an-no-tated
com-ple-ments
com-po-si-tio-na-li-ty
non-com-po-si-tio-na-li-ty
Gon-zá-lez
out-side
Ri-chárd
se-man-tics
STREU-SLE
Tie-de-mann
}
   \boolfalse{bookcompile}
   \togglepaper[23]%%chapternumber
}{}


\begin{document}
\maketitle


\hypertarget{introduction}{%
\section{Introduction}\label{introductionCR}}

In this chapter\footnote{The dataset is accessible here: \url{http://dx.doi.org/10.5287/ora-dqjeo65n5}.}, I consider the development of support-verb constructions
in New Testament Greek and the potential exegetical impact of
philological developments. My key case study verb is ποιῶ \textit{poiō} ‘to make, do'. 
In 1 John, for example, both the verb ἁμαρτάνω \textit{hamartánō} ‘to sin' and the construction ποιῶ ἁμαρτίαν \textit{poiō hamartían} ‘to commit a sin' are used. 
I investigate to what extent these may be considered synonymous, and
explore how the use of a support-verb construction may have an
exegetical impact in terms of distancing sin from sinner. 
Support-verb constructions divorce the semantic and morphological roles of the verb used, and therefore allow for a different relationship between agent and action. 
This allows for the construction of Christian personhood\is{personhood} distinguishing between agent and action, sinner and sin, which has significant moral implications. 
There may also be a diachronic difference in how the gospels portray Jesus differentiating between the two, how epistles reflect on this, and how Christian ethics beyond the New Testament deal with the topic more broadly. 
In blending philological and theological approaches to the same material, I therefore consider the potential exegetical impact of improving our philological understanding of the New Testament.
Relatively little work has so far been done on support verb constructions in the New Testament, and this chapter therefore aims to add to both the philological discussion, and its application to New Testament exegesis.\footnote{Jiménez López has done some work in this area, but it does not deal with sin specifically (my focus here) and in part deals with the Latin translation of the New Testament, with which I deal with further in \citet{ryan_translating_2025}. See \citet{jimenez_lopez_colocaciones_2017, jimenez_lopez_colocaciones_2018, garcia2022translation}.}


\section{Definition}\label{definitionCR}
\largerpage
For the purpose of this chapter, I start with the simplicity of
Salkoff's definition of support-verb constructions (SVCs henceforth):
‟The principal feature of the support verb construction is that the
verbal slot in the sentence is occupied by the combination of a verb,
V\textsubscript{sup}, plus a noun, N\textsubscript{sup}" \citep[244]{salkoff_automatic_1990}. 
\citet[329]{nagy_t_full-coverage_2013} describe them as light verbs in multi-word
expressions, where the verb functions as the syntactic head while the
semantic head is the noun (see also \citealt{kamber_funktionsverbgefuge_2008} for the German background to the concept).
This splits process and product, a distinction which will be important
to this chapter. Stefan \citet{langer_formal_2005} makes this distinction clear in his work
on a general definition for SVCs which includes demonstrating the
semantic emptiness, potential interchangeability, and removability of
the verb.
\citet[275]{gross_lexicon-grammar_1984} encourages us to consider
phrasal lexical entries, that is nouns in their verbal contexts, and not
just individual words.
In this chapter, I examine the ramifications of
choosing an SVC over a simplex verb for the exegetical impact of the
text. 
\citet[363]{stroik_light_2001} argues that light verbs (his term for what I am calling support verbs) have
stronger phonetic and semantic justification than many SVC definitions
allow, at least in English; I aim to demonstrate that
with regards to sin in Judaeo-Christian thought, there is a relationship
between morphology / syntax and theology which is predicated on the
light verb enabling a particular more pragmatic relationship between
agent and action, rather than necessarily a phonetic or semantic
one.

I am working with a model of a periphrastic construction involving a
semantically empty verb with a deverbal noun carrying the semantic
weight, set against semantically equivalent verbs. 
% I don't understand the error code here. It is repeated several times.
My one modification would be that
I will also consider combinations where the N\textsubscript{sup} is
replaced by an adjective functioning substantively; this is particularly
relevant with the adjectives κακός \textit{kakos} ‘bad' and καλός \textit{kalos} ‘fine /
beautiful'. 
It is beyond the scope of this chapter to explore the use
of adjectives as substantives in the New Testament more generally, but
it is a frequent feature of New Testament Greek.\footnote{For the
  standard introduction to this given to many beginners, see \citet{duff_elements_2008}, chapter 5.}
In addition to the definition of an SVC, for
the purpose of this article there also needs to be a verb which could be
semantically equivalent, but potentially not pragmatically equivalent.
This chapter will consider what some of the pragmatic differences are, a
topic well-discussed by \citet[74]{cappelle_taking_2022}.

\largerpage
\section{My corpus and its limitations}\label{my-corpus-and-its-limitationsCR}

This chapter is confined to the use of SVCs in the New Testament.
Depending on the edition and means of counting, there are 138,162 words
in the Greek New Testament. 
This comprises 5,437 different words, only
319 of which occur more than 50 times, and account for around 80\% of
the total word count. 
3,465 are New Testament \textit{hapax legomena}, and 8 are full
corpus \textit{hapax legomena}.\footnote{In this chapter, my data are mainly
  drawn from the \textit{Thesaurus Linguae Graecae}. 
For the basic information about total word counts,
  however, I have used the standard Greek editions as made available in
  the \textit{Logos Bible software}.
The \textit{Thesaurus Linguae Graecae} gives a total word count for the Greek New
  Testament of 137,938, including 6,432 lemmata, which is significantly
  different to the usual figures quoted in New Testament studies. 
This
  is in part due to the texts used in the \textit{Thesaurus Linguae Graecae}, and the way in which it
  distinguishes and counts words. Of the 8 \textit{hapax legomena}, six are
  names, and only two are true New Testament \textit{hapax legomena}: οἰκουργός, -ὀν \textit{oikourgós,
  ‑ón} ‘homemaker' and πραϋπαθία, ‑ας, ἡ \textit{praüpathía, ‑as, hē} ‘gentleness of temper'. 
Despite
  its prolific word-building, very few of the words in the Greek New
  Testament remain unquoted elsewhere.} 
Given how relatively few
frequently used words there are in the New Testament, that 138,162-word
corpus is large enough to analyse in terms of patterns, with some
caveats.~

Any analysis of the New Testament must accept its significant
limitations as a corpus. 
It is an arbitrary collection of texts not
formally canonised until the councils of Hippo (AD 393) and Carthage (AD
397). 
It is constructed on theological grounds rather than linguistic
ones, and is written largely by authors whose first language was not
Greek (Luke is the major exception, with Luke-Acts accounting for
roughly 25\% of the whole corpus). 
The Greek may broadly reflect the
versions of contemporary vernaculars, but this is still an awkward
collection of texts with which to work on linguistic grounds. New
Testament linguistics faces many challenges when trying to extrapolate
general points about Greek from this relatively small and disparate
sample. 
The geographical, temporal, and linguistic backgrounds of the
writers are sufficiently diverse as to make it in many ways an
unrepresentative corpus on linguistic terms.\footnote{For a general
  introduction to New Testament Koine\is{Koine} as conceived in a great Greek
  context, see \citet{georgakopoulou_standard_2009}. \citet[147–152]{horrocks_greek_2010} deals in particular with New Testament Koine; see pp. 147 and 149 for his
  discussion of it as a standard language under the Roman
  administration in particular. I challenge some of the standardisation of New Testament Koine as a form in \citet{ryan_teaching_2024}.
  \citet[243]{tronci_aorist_2018} reiterates
  the point that many relevant linguistic analyses are synchronic, and
  the New Testament needs special attention as a corpus of
  linguistically disparate texts.}~~

As a simple example, the future tense is noticeably infrequent in the
New Testament, and therefore often not well-taught. 
One would not,
however, want to consider Greek a language without a way to express the
future, or the New Testament as a text wherein eschatology is
unimportant.\footnote{See \citet{ryan_teaching_2024} on the teaching of the future
  tense and the ideological impact of textbook design. 
In terms of the
  lack of frequency, there are, for example, only twelve future
  participles and five future infinitives in the New Testament.} 
The
future is talked about in different ways, including periphrastic phrases
which, being multi-word phrases themselves, begin to lead us into the
territory of SVCs.~

Although the corpus may be limited and awkward, both in size and nature, it does demonstrate some trends, and once it became canonised
as a closed corpus of religiously significant texts, the language in
which it was written underpinned the development of a new religion and
new forms of religious expression. 
By fossilising the New Testament to
preserve the text's religious importance, therefore, the techniques with
which it expresses some topics become significant in new ways. 
It is
this relationship between the development of the expressions and their
theological impact which I investigate in this chapter.~

\hypertarget{svcs-in-the-new-testament}{%
\section{Support-verb constructions in the New Testament}\label{svcs-in-the-new-testamentCR}}

Sometimes it is possible to see clear idiolectical differences between New
Testament authors, even in matters as simple as Mark's use of καί
\textit{kaí} ‘and' and John's use of οὖν \textit{oũn} ‘so, therefore'. In the case
of SVCs, however, the spread appears to be broader, governed by
contextual criteria beyond individual authorship. 
I demonstrate
how these criteria include the use of linguistic structures to sculpt a
new theological framework. 
This involves considering differences in the
locus of agency between various kinds of verbs, and support-verb
constructions.~

Of the 571 total uses of ποιῶ \textit{poiō} ‘to make, do' in the New Testament, 50 meet my criteria
for being interpreted as SVCs. 
These are a mix of active and middle
verbs, predominantly active (16 middle). 
They are found in all four
gospels and a further fourteen texts. 
A further 42 could be interpreted
as SVCs if the substantive use of adjectives is included, including 20
related to doing good or bad.
~These lead to 9--12\% of uses of ποιῶ \textit{poiō} ‘to make, do' in
the New Testament functioning as a support verb, according to my
definition. 
This is a considerable proportion of the uses of ποιῶ \textit{poiō} ‘to make, do' in the New
Testament, which is sufficiently significant to be worthy of further
investigation.~

\subsection{Choosing examples}\label{choosing-examples}

When searching for collocations, I considered only examples where the
verb was within five words of the noun. 
This allows for particles,
articles or other modifiers, whilst acknowledging that, in order to be
an SVC, the noun and verb needed to be in close proximity. 
I then
checked each example manually, to ensure that these were phrases and not
merely words in proximity but, for example, across sentence barriers.~

My key phrase in this article pertains to sin, but I also consider other
related terms and phrases, and ways in which the verb ποιῶ \textit{poiō} ‘to make, do' might be used
in an SVC. 
I do not, however, count examples such as `bearing fruit'
(ποιῶ καρπόν \textit{poiō karpón} ‘to bear fruit') as an SVC, as, although there is a verb (cf. καρποφορεῖ \textit{karpophoreĩ} at
\iwi{NT Matthew 13:2}), both the verb and the SVC are
only used eight times each in the New Testament, which would be too few
on which to base any argument.~
I outline the relevant numbers and
examples further below.

\subsection{The Septuagint as scene-setting}\label{the-septuagint-as-scene-settingCR}

ποιῶ \textit{poiō} ‘to make, do' is used along with ἁμαρτία \textit{hamartia} ‘sin' in order to form a mulit-word verb in the
Septuagint. 
Written around 300 years before the New Testament, it uses
an older form of Greek, which is itself Atticising, and therefore
occasionally archaic. 
The New Testament quotes the Septuagint directly,
paraphrases it, and remodels ideas from it, as well as being generally
influenced by it and the Jewish cultural language underlying it.
Elements of New Testament Greek can therefore display archaising
tendencies in keeping with the Septuagint, rather than being reflective of
their own linguistic context.

Multi-word verbs do have a role in the Hebrew of the Old Testament. 
One
might, therefore, consider that support verbs in the New Testament grow
in part from the Hebrew influence on the Septuagint, but this does not
seem to be the case. 
Most distinctive is the number of relative clauses
using ποιῶ \textit{poiō} ‘to make, do' to refer back to ἁμαρτία \textit{hamartia} ‘sin', in some senses a `split' SVC:
\iwi{OT Numbers 5:3}, \iwi{OT Deuteronomy 9:21}, \iwi{OT 3 Kings 16:19}, \iwi{OT 4 Kings 17:22}, \iwi{OT Psalms
8:13}, \iwi{OT Ezekiel 18:14}, and \iwi{OT Susanna 52:6}. While there are lots of
periphrastic phrases, particularly regarding the formulaic language of
sacrificing cows / burnt offerings, they are not SVCs. Only \iwi{OT Tobit 12:10}, in the \textit{Codex Sinaiticus}, fulfils my criteria for an
SVC (see \ref{ex1CR}).


\ea\label{ex1CR}
\glll οἱ \textbf{ποιοῦντες} \textbf{ἁμαρτίαν} καὶ \textbf{ἀδικίαν} πολέμιοί εἰσιν τῆς ἑαυτῶν ψυχῆς \\
%This is the only example label which worked. Something odd happened with all the others. I have inserted comments with each.
\textit{hoi}   \textit{poioũntes}  \textit{hamartían}  \textit{kaì}            \textit{adikían}    \textit{polémioí}   \textit{eisin}      \textit{tē̃s}        \textit{heautō̃n}   \textit{psukhē̃s}\\
    the.\textsc{nom} do.\textsc{prs.ptcp.nom} sin.\textsc{acc} and injustice.\textsc{acc} enemies.\textsc{nom} be.\textsc{prs.3pl} the\textsc{gen.sg} their.\textsc{gen.pl} souls.\textsc{gen.sg} \\

\glt `Those committing sin and injustice are enemies of their souls.' \\
\hspace*{\fill}(\iwi{OT Tobit 12:10})
\z


%\ea
%\gll cogito                           ergo      sum\\
   %  think.\textsc{1sg}.\textsc{pres} therefore \textsc{cop}.\textsc{1sg}.\textsc{pres}\\
%\glt `I think therefore I am.'
%\z

This pre-empts the similar relationship drawn between ἁμαρτία \textit{hamartia} ‘sin' and ἀδικία \textit{adikía} ‘unrighteousness'
discussed below, with particular reference to \iwi{NT 1 John}. 
It also follows
the other conventions seen in New Testament SVCs in this context, that
is, substantive participle of the light verb followed by the relevant
noun. 
A textual variation replaces οἱ ποιοῦντες ἁμαρτίαν \textit{hoi poioũntes hamartían} ‘those
committing a sin' with οἱ δὲ ἁμαρτάνοντες \textit{hoi dè hamartánont-es} ‘those sinning',
demonstrating the closeness of the relationship between the SVC and the
simplex verb in the minds of those copying out this text. % hamartánontes manually hyphenated to avoid hbox issue.

Verbs other than ποιῶ \textit{poiō} ‘to make, do' are also available for rendering description of
sin in the Septuagint. There are 25 examples where the verb ἁμαρτά-νω \textit{hamartánō} ‘to sin' and % ἁμαρτάνω manually hyphenated to avoid hbox issue.
the noun ἁμαρτία \textit{hamartia} ‘sin' are used within the same phrase.
22 of these, however,
are in subordinate clauses where the verb refers back to the noun in
fairly formulaic phrases, and 12/22 examples are in Leviticus (see (\ref{ex2CR})), further
limiting the construction to particular contexts.

\ea\label{ex2CR}  
\glll ὁ ἱερεὺς περὶ τῆς ἁμαρτίας αὐτοῦ, ἧς ἥμαρτεν\\
\textit{ho} \textit{hiereùs} \textit{perì} \textit{tē̃s} \textit{hamartías} \textit{autoũ}, \textit{hē̃s} \textit{hḗmarten}\\
    the.\textsc{nom} priest.\textsc{nom} about the.\textsc{gen} sin.\textsc{gen} he.\textsc{gen} \textsc{rel}.\textsc{gen} sin.\textsc{aor.ind.3sg} \\

\glt `The priest\ldots{} about his sin, sin which he had sinned.' \\
\hspace*{\fill}(\iwi{OT Leviticus 5:10=5:13})
\z

Indeed, 17/25 are from the Pentateuch, which very much suggests a
specific linguistic and theological context for the phrasing, linked
both to the Greek of those specific books, and to their significance
within Judaism. 
Only three are used (see (\ref{ex3CR}) to (\ref{ex5CR})) in any sense which could be called inflecting the topic (unnecessarily repeating multiple forms of a lexical root):


% should be Example 3
\ea\label{ex3CR} 
\glll Ὑμεῖς ἡμαρτήκατε ἁμαρτίαν μεγάλην\\
\textit{Humeĩs} \textit{hēmartḗkate} \textit{hamartían} \textit{megálēn}\\
    you.\textsc{nom} sin.\textsc{prf.ind.2pl} sin.\textsc{acc} great.\textsc{acc} \\

\glt `You have sinned a great sin' \\
\hspace*{\fill}(\iwi{OT Exodus 32:30})
\z

\ea\label{ex4CR}
\glll ἡμάρτηκεν ὁ λαὸς οὗτος ἁμαρτίαν μεγάλην\\
\textit{hēmártēken} \textit{ho} \textit{laòs} \textit{hoũtos} \textit{hamartían} \textit{megálēn} \\
    sin.\textsc{prf.ind.3sg} the.\textsc{nom} people.\textsc{nom} this.\textsc{nom} sin.\textsc{acc} great.\textsc{acc} \\

\glt `This people have sinned a great sin' \\
\hspace*{\fill}(\iwi{OT Exodus 32:31})
\z

\ea\label{ex5CR}  
\glll Ἁμαρτίαν ἥμαρτεν Ιερουσαλημ\\
\textit{Hamartían} \textit{hḗmarten} \textit{Ierousalēm}\\
    sin.\textsc{acc} sin.\textsc{aor.ind.3sg} Jerusalem.\textsc{nom}\\

\glt `Jerusalem sinned a sin' \\
\hspace*{\fill}(\iwi{OT Lamentations 8:1})
\z



Both Exodus examples use verbs in the perfect tense, delineating the
participants as sinners as much as the sin being committed. 
Both also
use the adjective `big', which may mean that the repetition is as much
about contributing to the sense of importance and enormity, not as a
linguistic trope. 
The example from Lamentations is again atypical, being
poetic, and anthropomorphising a town, Jerusalem. 
It does not seem,
therefore, as though this verb plus noun repetition is a standard
feature of the Septuagint, so much as being available for specific uses,
namely relative clauses and emphasis within the Pentateuch.

\subsection{Voice}\label{voice}

\citet{jimenez_lopez_support_2016} argues that SVCs use the middle voice of ποιῶ \textit{poiō} ‘to make, do'.\is{voice} 
In the New Testament, this is true, on my criteria, in
only 16/50 examples. 
The middle voice examples deal with memory, prayer,
nouns derived from βάλλω \textit{bállō} ‘to throw', causing an increase, or
making a journey. 
The examples are spread across authors (11/27 texts),
but are restricted to specific contexts. 
Eight are in the first chapter
of a text, and seven of those eight within the first four verses, in
phrases which seem to suggest formulaic idioms rather than free
linguistic choice (see (\ref{ex6CR})).\footnote{The full list is \iwi{NT Acts of the Apostles 1:1},
  \iwi{NT Ephesians 1:16}, \iwi{NT Philippians 1:4}, \iwi{NT 1 Timothy 2:1}, \iwi{NT 1 Thessalonians 1:2}, \iwi{NT 2
  Peter 1:10}, \iwi{NT 2 Peter 1:15}. Throughout this chapter I put the relevant verb form in bold with underline, and underline any nouns joined with it, so that readers less familiar with Greek can identify constructions. All translations from the New Testament in this chapter are my own. They are intended to support understanding of the Greek, not as elegant translations in their own right.}
  
\protectedex{
\ea\label{ex6CR} 
\glll Τὸν μὲν πρῶτον \textbf{λόγον} \textbf{ἐποιησάμην} περὶ πάντων\\
% I see the author guide says not to use bold and underline but this did seem the only way. I can remove them if needed.
\textit{Tòn} \textit{mèn} \textit{prō̃ton} \textit{lógon} \textit{epoiēsámēn} \textit{perì} \textit{pántōn}\\
    the.\textsc{acc} \textsc{prt} first.\textsc{acc} account.\textsc{acc} do.\textsc{aor.ind.1pl} about everything.\textsc{gen} \\

\glt `I made the first account about everything\ldots{}' \\
\hspace*{\fill}(\iwi{NT Acts of the Apostles 1:1})
\z
}

This example does not have an obvious corresponding verb apart from λέγω
\textit{légō} ‘to speak, say, recount, tell', which does not cover quite the
same remit. While it therefore meets my definition of an SVC in terms of
using ποιῶ \textit{poiō} ‘to make, do' as a semantically light verb along with a relevant noun, it
is missing the equivalent verb for this context. 
Given the novelty and
status Luke is trying to create for himself in this introduction,
however, the ease with which the phrase can be understood, and the
clearly ‟light" use of ποιῶ \textit{poiō} ‘to make, do', I would count it as an SVC, but
an example which demonstrates that there is a spectrum of usage in the
New Testament, and not a clear polarisation between SVCs and other
constructions.

More clearly under the category of SVCs with middle verbs are 1 Timothy
2:1 and Romans 1:9 (see (\ref{ex7CR}) and (\ref{ex8CR}) respectively).

\ea\label{ex7CR} 
\glll Παρακαλῶ οὖν πρῶτον πάντων \textbf{\textbf{ποιεῖσθαι}} \textbf{δεήσεις}, \textbf{προσευχάς}, \textbf{ἐντεύξεις}, \textbf{εὐχαριστίας}\\
\textit{Parakalō̃} \textit{oũn} \textit{prō̃ton} \textit{pántōn} \textit{poieĩsthai} \textit{deḗseis}, \textit{proseukhás}, \textit{enteúxeis}, \textit{eukharistías}\\
    urge.\textsc{prs.ind.1sg} \textsc{prt} first.\textsc{adv} all.\textsc{gen} do.\textsc{prs.inf.mid} prayers.\textsc{acc} entreaties.\textsc{acc} petitions.\textsc{acc} thanks.\textsc{acc} \\

\glt `So I urge you first of all to make prayers, entreaties, and petitions, and give thanks\ldots{}' \\
\hspace*{\fill}(\iwi{NT 1 Timothy 2:1})
\z


\ea\label{ex8CR} 
\glll ὡς ἀδιαλείπτως \textbf{μνείαν} ὑμῶν \textbf{ποιοῦμαι}\\
\textit{hōs} \textit{adialeíptōs} \textit{mneían} \textit{humō̃n} \textit{poioũmai}\\
    how unceasing.\textsc{adv} remembrance.\textsc{acc} you.\textsc{gen} do.\textsc{prs.ind.1sg} \\

\glt `\ldots how I unceasingly make a remembrance of you\ldots{}' \\
\hspace*{\fill}(\iwi{NT Romans 1:9})
\z


At first glance, therefore, it seems as though ποιῶ \textit{poiō} ‘to make, do' is used in typical
SVCs, in the middle voice\is{voice}, as we might expect, but infrequently, with
some variation. 
Voice in the New Testament is a contested topic,
remaining one of the key issues for debate among those dealing with New
Testament linguistics (see e.g. \citealt{tronci_aorist_2018, black_linguistics_2020}). ποιῶ \textit{poiō} ‘to make, do' used in the active voice as a support verb becomes
more usual as we move into later Greek, however, and its New Testament
use in this form is therefore not unexpected.\footnote{See \citet{cock__1981} on voice choice with ποιῶ \textit{poiō} ‘to make, do'. This is also linked to the
  phenomenon of aorist middle endings falling out of use / merging with
  aorist passive endings noted by \citet[103]{horrocks_greek_2010} and \citet[251–252]{tronci_aorist_2018}. Further work on this area can also be found in \citet{vives_cuesta_support-verb_2022}.} Given that ἁμαρτάνω \textit{hamartánō} ‘to sin' is only
used in the active voice\is{voice} in the New Testament, it also makes sense for
the replacement SVC to be expressed in the active voice, not least given
the necessarily transitive status of an SVC, and the potentially more
intransitive nature of the middle voice.\footnote{See \citet[245]{tronci_aorist_2018}
  on ἁμαρτάνω \textit{hamartánō} ‘to sin' as active only, and p. 249 on transitivity.} I explore some
potential ramifications of voice differences later in this chapter, but
at this point, it is enough to say that I do count active uses of ποιῶ \textit{poiō} ‘to make, do'
in the New Testament as eligible for forming SVCs, albeit demonstrating a
difference in the range of uses available in the active to the middle
voice.\footnote{\citet{jimenez_lopez__2021} also writes about γίγνομαι \textit{gígnomai} as the lexical
  passive of ποιῶ \textit{poiō} ‘to make, do' in support-verb constructions.
  There is only one example in the New Testament where γί(γ)νομαι \textit{gí(g)nomai} ‘to become' could
  be said to be taking this role with regard to sin, however, which is
  \iwi{NT Romans 7:13}. This is not a clear case, given the more predicative
  nature of the statement. In terms of committing sin, a passive
  expression using γί(γ)νομαι \textit{gí(g)nomai} ‘to become' is not found. This means that there
  remains an agent of sin throughout the language around ἁμαρτία \textit{hamartia} ‘sin' in the
  New Testament, but, I suggest, this agent is also held at a distance
  from the sin by the very form of the support-verb construction.
  The
  de-agentivisation talked about by Jiménez López is not needed, because
  the agency has already been reduced by the use of a support-verb
  construction.} This
means that, for the purposes of this chapter, ποιῶ ἁμαρτίαν \textit{poiō hamartían} ‘to commit a sin' is
considered an SVC. My specific context is that of committing a sin, and
the exegetical and ethical impact of using ποιῶ \textit{poiō} ‘to make, do' in this way.

\subsection{Putting ποιῶ \textit{poiō} ‘to make, do' as part of a support-verb construction  in context}\label{putting-ux3c0ux3bfux3b9ux1ff6-as-part-of-an-svc-in-contextCR}

Before turning to sin, however, I further define some of the aspects of
ποιῶ \textit{poiō} ‘to make, do' and related terms as SVCs and similar in the New Testament, notably
word order, negation, and the potential for plural head nouns. 
Word
order is relatively consistent in SVCs using ποιῶ \textit{poiō} ‘to make, do' in the New Testament.
In only four examples does the verb occur before the noun. 
Three of
those are in the formula πᾶς ὁ ποιῶν τὴν ἁμαρτίαν \textit{pãs ho poiō̃n tḕn hamartían} ‘everyone who
commits a sin' in John / \iwi{NT 1 John}, where πᾶς \textit{pãs} ‘everyone' +
article + participle is such a stylistic pattern that this formula seems
to override the SVC's internal syntax.\footnote{Examples include: \iwi{NT 1 John
  2:29} πᾶς ὁ ποιῶν τὴν δικαιοσύνην \textit{pãs ho poiō̃n tḕn dikaiosúnēn} ‘everyone who acts justly' --
  an SVC), \iwi{NT 1 John 3:4} Πᾶς ὁ ποιῶν τὴν ἁμαρτίαν καὶ τὴν ἀνομίαν ποιεῖ
  \textit{Pãs ho poiō̃n tḕn hamartían kaì tḕn anomían poieĩ} ‘Everyone who commits a sin also commits lawlessness', \iwi{NT 1 John
  4:7} καὶ πᾶς ὁ ἀγαπῶν ἐκ τοῦ θεοῦ γεγέννηται \textit{kaì pãs ho agapō̃n ek toũ theoũ gegénnētai} ‘Everyone who loves
  has been begotten from God', and \iwi{NT 1 John 5:1} Πᾶς ὁ πιστεύων
  ὅτι Ἰησοῦς ἐστιν ὁ Χριστὸς \textit{Pãs ho pisteúōn
  hóti Iēsoũs estin ho Khristòs} ‘Everyone who believes that Jesus is
  Christ', to give a representative sample from 1 John.} 
  The
other use is \iwi{NT 1 Timothy 2:1}, quoted above, where the verb governs a short
catalogue of nouns, which follow neatly in order. 
In all other examples,
the verb directly follows the noun; the only words which might intervene
are descriptions of the noun (e.g. possessive pronouns, prepositional
phrases, and adjectives), or negations of the verb.\footnote{See \citet[4] {fendel_i_2023} on this discontiguous aspect of SVCs.} In each of the negative
cases (\iwi{NT 1 John 3:9}, \iwi{NT 1 Peter 2:22}, \iwi{NT Romans 13:14}, the verb is negated with
the adverb (two veridical, one non-veridical), and not any of the more
complex syntactical elements described by \citet[7–8]{fendel_i_2023} in her work on
negating support verb constructions. 
This
strengthens the sense of the verbal phrase, with the noun syntactically
subordinated to the verb in the SVC, rather than the noun being negated.
None of these patterns are specific to the voice\is{voice} of the verb, however,
suggesting that the active and middle do work similarly in support-verb
constructions in the New Testament.


Only three of the New Testament SVCs with ποιῶ \textit{poiō} ‘to make, do' feature plural head nouns
(\iwi{NT 1 Timothy 2:1}, \iwi{NT James 5:15}, \iwi{NT Luke 5:33}).\footnote{On pluralising head
  nouns as a feature of SVCs, see \citet[4]{fendel_i_2023}.} 
One of these refers
to sin, the other two to prayers. Prayer is also referred to singularly
(\iwi{NT Philippians 1:4}), but in general, plural prayers standing as a
collective concept is not peculiar (`our thoughts and prayers are with
you'). Of the 18 uses of δέησις \textit{déēsis} ‘prayer' in the New Testament, 8
are plural, and the only example of δεήσεις \textit{déēseis} ‘prayers' not in an SVC is the \iwi{NT Letter to the Hebrews
5:7}, following on from a Septuagint quotation and so glossing archaising
Greek rather than reflecting natural New Testament Koine\is{Koine}.

The plural in James 5:15 may seem awkward (see (\ref{ex9CR})).

\ea\label{ex9CR} 
% Should be Example 9
\glll κἂν \textbf{ἁμαρτίας} ᾖ \textbf{πεποιηκώς}, ἀφεθήσεται αὐτῷ\\
\textit{kàn} \textit{hamartías} \textit{ē̃ͅ} \textit{pepoiēkṓs} \textit{aphethḗsetai} \textit{autō̃ͅ}\\
    even.if sins.\textsc{acc} be.\textsc{prs.sbjv.3sg} do.\textsc{prf.ptcp.nom} forgive.\textsc{fut.pass.3sg} he.\textsc{dat} \\

\glt `Even if he has committed sins, he will be forgiven' \\
\hspace*{\fill}(\iwi{NT James 5:15})
\z

The majority (111/173) of examples of ἁμαρτία \textit{hamartia} ‘sin' in the New
Testament are plural. 
The question might in fact be why all the rest of
the examples in SVCs are singular, accounting for 7/27, or nearly a
quarter of all the uses of ἁμαρτίαν \textit{hamartían} ‘sin' in the accusative
singular.\footnote{The other references are: \iwi{NT John 8:34}, \iwi{NT 2 Corinthians
  5:21} (x2), \iwi{NT 1 Peter 2:22}, \iwi{NT 1 John 3:4}, \iwi{NT 1 John 3:8}, \iwi{NT 1 John 3:9}.} 
There
may be something formulaic about the phraseology of committing a sin
developing in the New Testament, particularly as three of these phrases
occur within one chapter of one letter (\iwi{NT 1 John 3}). 
In addition, the use
of the singular makes sin specific, allowing for a clear example of an
individual instance of sin being committed by an individual person,
rather than as a general way of life. 
This begins to build a picture of a distinctive sinner committing distinctive sin, and not of general
ethical sweeps. Within the parameters of permissible variation outlined
by Fendel, however, there is very little relevant in New Testament SVCs.
The sample may be small compared with the size of the corpus, but the
construction seems to be relatively formulaic and context
specific \citep[4–5]{fendel_i_2023}. How, therefore, is it used with
reference to sin?

\section{Committing Sin}\label{committing-sinCR}

The verb ἁμαρτάνω \textit{hamartánō} ‘to sin' is attested 26,518 times in the \textit{Thesaurus Linguae Graecae} corpus. It
initially refers to a physical missing of a mark with a bow and arrow,
but by Christian times it refers to the process of sinning. 
The meaning
changes from literal mistake to metaphorical error to moral fault. 
In
the standard lexicon of Classical Greek, \emph{Liddell-Scott-Jones,} we find
`miss the mark\ldots{} fail of one's purpose\ldots{} go
wrong\ldots{} do wrong\ldots{} err\ldots{} sin' \citep{liddell_greek-english_1996}.
In Muraoko's lexicon of the Septuagint, this
becomes `act sinfully\ldots{} commit a sin\ldots{} fail to be
available', which already emphasises both the moral quality of the term
and its potential periphrastic expression \citep{muraoka_greek-english_2009}.
In
the standard New Testament lexicon, \textit{A Greek-English lexicon of the New Testament and other early Christian literature} (BDAG), we find `to commit
wrong, to sin', and only further down the entry any downgraded
reference to its earlier physical meaning \citep{arndt_greek-english_2000}.
As a physical term, its remit is very limited and so,
unsurprisingly, we find it used relatively infrequently. 
As it becomes
more metaphorical, its usage increases.\footnote{It is most commonly
  used by John Chrysostom, the fourth-century Early Church Father. 
  That
  is true, however, of most of the lemmata in this lexical group, and
  further work is needed to remove disproportionately over-represented
  authors such as Chrysostom from samples, not least because his much
  later date also means that his language represents a different phase
  in the development of Greek. 
  I discuss the diachronic lexical
  development of the Greek terms used in this chapter further in my
  forthcoming monograph \citep{ryan_translating_2025}, but further discussion of lexical
  aspects is largely beyond the scope of this chapter.}

The distribution of the verb begins to form more of a pattern when
considered in the light of its related nouns. 
The noun ἁμαρτία \textit{hamartia} ‘sin' has a
very different distribution. 
There are 44,868 examples attested in the
\textit{Thesaurus Linguae Graecae} corpus. 
The highest frequencies by author and text are again all
Christian contexts, notably John Chrysostom and the catena to the New
Testament. 
Overall, it is used 1.68 times for every use of the verb.

In what follows, I aim to demonstrate why the SVC formulation provides a
morpho-syntactic framework to carry a theological point demarcating
Christian ethics as different to other ethical systems, in
distinguishing the product of an action from its producer.


Homer does not use the noun at all. In all other pre-Christian authors I
have evaluated, the verb is more common than the noun. A few examples
are given in \tabref{tab:myname:frequenciesCR}.

\begin{table}
\caption{Ratio of uses of the noun ἁμαρτία \textit{hamartia} ‘sin' to the verb ἁμαρτάνω \textit{hamartánō} ‘to sin'
in 10 Greek authors}
\label{tab:myname:frequenciesCR}
 \begin{tabularx}{\textwidth}{rrrr}
  \lsptoprule
            Author & Century  & Genre & Noun : verb\\
  \midrule
Aeschylus~ & 5th BC & Tragedy & 0.31:1~ \\
Sophocles~ & 5th BC & Tragedy & 0.18:1~ \\
Euripides~ & 5th BC & Tragedy & 0.33:1~ \\
Plato~ & 5th - 4th BC & Philosophy & 0.16:1~ \\
Lysias~ & 5th - 4th BC & Forensic oratory & 0.07:1~ \\
Isocrates~ & 5th - 4th BC & Forensic oratory & 0.08:1~ \\
Demosthenes~ & 4th BC & Forensic oratory & 0.1:1~ \\
Aristotle~ & 4th BC & Philosophy & 0.49:1~ \\
Plutarch~ & 1st AD & Various but contemporary & 0.26:1~ \\
Lucian~ & 1st AD & Various but contemporary & 0.07:1~ \\
  \lspbottomrule
 \end{tabularx}
\end{table}


I chose these authors as representative of genres where wrongdoing is
discussed (drama, forensic oratory, philosophy). In the case of Lucian and Plutarch, they are roughly contemporaneous with the gospel writers, reflecting other varieties of Koine\is{Koine} used at the time.\footnote {See \citet{horrocks_greek_2010} for a broad categorisation of types of Koine.}
In addition, the older texts represent examples of the Atticising style which both the Septuagint and New Testament sometimes emulate. 
While there is
variation in the distribution, the verb remains more common, and there
is broad consistency between genres.


The distribution only inverts once we look at a Judaeo-Christian context. 
In
the New Testament, the noun is four times as common as the verb, which
reverses all the figures above, and is significantly different from the
whole corpus ratio of 1:1.68.\footnote{For reference, our top
  contributor John Chrysostom, uses ἁμαρτία \textit{hamartia} ‘sin' 1.46 times for every use of
  ἁμαρτάνω \textit{hamartánō}, so below the corpus average, but before the pre-Christian
  average.} There is a clear shift in emphasis from verb to noun.


I suggest that the increase in the use of the noun over the verb makes
sin into a thing, not a process. 
In so doing, sin can be separated from
sinner, made into something which can be removed from the agent. 
This
means the sin is not necessarily part of the sinner's identity, which
allows for a human personhood\is{personhood} that is not inherently sinful so much as
capable of committing sins. 
This leaves people as ultimately good
(God-created), but flawed, and so capable of sinning but of being
forgiven\is{forgiveness} and redeemed.
It also allows for Jesus to be human and yet
sinless, as sin is not inherently tied to human nature, but to human
action.

This may also partly inform the voice\is{voice} of the support verb. 
Given the
potential self-involvement of the middle voice, it may cast a
self-referentiality into sinning which would be at odds with the
distinction between sin and sinner. 
The balance of focus between sinner,
sin, and anyone sinned against is already obvious in the use of objects
with the different verbs. ἁμαρτάνω \textit{hamartánō} ‘to sin' can be directed towards a recipient;
people can be sinned against. 
About 1/5 uses in the New Testament take a
prepositional phrase, with seven examples of εἰς \textit{eis} ‘into', one of
ἐπί \textit{epí} ‘upon', and two of πρός \textit{prós} ‘towards'.\footnote{Note that
  πρός \textit{prós} ‘towards' only describes the difference between mortal and venial sin, in \iwi{NT 1
  John 5:16}, rather than sin against an individual.}

ποιῶ ἁμαρτίαν \textit{poiō hamartían} ‘to commit a sin', on the other hand, never includes a person sinned
against. 
This is partly due to the fact that the verb already has a
direct object (ἁμαρτίαν \textit{hamartían} ‘sin'), but a prepositional phrase could still have
been used. 
The focus is on the fact that someone is sinning, not that
sin might be causing a problem (e.g. see (\ref{ex10CR} to (\ref{ex12CR})).

\ea\label{ex10CR} 
\glll Πᾶς ὁ \textbf{ποιῶν} τὴν \textbf{ἁμαρτίαν} καὶ τὴν ἀνομίαν ποιεῖ\\
\textit{Pãs} \textit{ho} \textit{poiō̃n} \textit{tḕn} \textit{hamartían} \textit{kaì} \textit{tḕn} \textit{anomían} \textit{poieĩ}\\
    every.\textsc{nom} the.\textsc{nom} do.\textsc{prs.ptcp.nom} the.\textsc{acc} sin.\textsc{acc} and the.\textsc{acc} lawlessness.\textsc{acc} do.\textsc{prs.ind.3sg} \\

\glt `Everyone who commits a sin also commits lawlessness' \\
\hspace*{\fill}(\iwi{NT 1 John 3:4})
\z

\ea\label{ex11CR} 
\glll ὁ \textbf{ποιῶν} τὴν \textbf{ἁμαρτίαν} ἐκ τοῦ διαβόλου ἐστίν\\
\textit{ho} \textit{poiō̃}n \textit{tḕn} \textit{hamartían} \textit{ek} \textit{toũ} \textit{diabólou} \textit{estín}\\ 
    the.\textsc{nom} do.\textsc{prs.ptcp.nom} the.\textsc{acc} sin.\textsc{acc} from the.\textsc{gen} devil.\textsc{gen} be.\textsc{prs.ind.3sg} \\

\glt `The one who commits a sin comes from the devil' \\
\hspace*{\fill}(\iwi{NT 1 John 3:8})
\z

\ea\label{ex12CR}  
\glll Πᾶς ὁ γεγεννημένος ἐκ τοῦ θεοῦ \textbf{ἁμαρτίαν} οὐ \textbf{ποιεῖ}\\
\textit{Pãs} \textit{ho} \textit{gegennēménos} \textit{ek} \textit{toũ} \textit{theoũ} \textit{hamartían} \textit{ou} \textit{poieĩ}\\
    every.\textsc{nom} the.\textsc{nom} bear.\textsc{prf.ptcp.pass.nom} from the.\textsc{gen} god.\textsc{gen} sin.\textsc{acc} \textsc{neg} do.\textsc{prs.ind.3sg} \\
\largerpage
\glt `Everyone born of God does not commit sin' \\
\hspace*{\fill}(\iwi{NT 1 John 3:9})
\z

The transitivity of sinning is less marked in the SVC. 
As a move away
from the verb ἁμαρτάνω \textit{hamartánō} ‘to sin' and any object, it may also reflect aspects of God's omnipresence in the New Testament. 
Just as miracles are often
expressed in the passive with no agent (the so-called divine passive,
where God is the assumed agent)\footnote{For example, \iwi{NT Galatians 5:18},
  and \iwi{NT Ephesians 3:19}.}, so sin requires no expressed recipient as it is
ultimately always God against whom we are sinning. 
The production of sin
is the problem, not the consequence of the sin against any one person,
but against God in general. 
The construction ποιῶ ἁμαρτίαν \textit{poiō hamartían} ‘to commit a sin' appears to be
used specifically to focus attention on production, but not necessarily
agency. Where there is a third party affected by sin, the simplex verb is
used. 
The SVC is only used where the affected party is not referred to.
This makes what in Christian terms is a fundamentally relational
process, sinning against someone (certainly in Luke, where ¾ uses are
followed by εἰς \textit{eis} ‘into'), into an individualised one. It allows for reflection
on the space between causation and impact.

My reading of this distinction between SVC and simplex verb can be
demonstrated with some specific examples. 
Only 8 of the 173 uses of
ἁμαρτία \textit{hamartía} are within a five-word proximity of the verb ποιῶ \textit{poiō} ‘to make, do' to create a
meaningful phrase. 
Three of these are in the \iwi{NT 1 John
3} examples given above, a text where the act of sinning is a running
theme, echoing the use at \iwi{NT John 8:34}. 
10/43 uses of the verb ἁμαρτάνω \textit{hamartánō} ‘to sin' are
also used in 1 John, and four of these ten are in chapter 3, making 1
John the densest use of sin language in the New Testament.
In just the
first ten verses, there are six examples of πᾶς ὁ \textit{pãs ho} ‘the one who' + participle, and
another three with just the article and participle. 
There is a rhythm,
fluency, syllogistic undertone, potentially formulaic shape, and clear
stylistic unity to this passage, which focusses in on the process of sin
in relationship to God.

The ease with which Greek moves between lexically related items,
however, potentially undercuts my argument about the distinction between
sin and sinner. 
In \iwi{NT 1 John 3:7}, we read: ὁ \textbf{ποιῶν} τὴν
\textbf{δικαιοσύνην} δίκαιός ἐστιν, καθὼς ἐκεῖνος δίκαιός ἐστιν· \textit{ho poiō̃n tḕn dikaiosúnēn díkaiós estin, kathṑs ekeĩnos dí-kaiós estin;} ‘The one
who does something just is just, just as that one is just'. 
Here, action
and character are directly linked. 
A verse earlier, however, and sin has
been described in very different terms: πᾶς ὁ ἐν αὐτῷ μένων οὐχ
ἁμαρτάνει· πᾶς ὁ ἁμαρτάνων οὐχ ἑώρακεν αὐτὸν οὐδὲ ἔγνωκεν αὐτόν.
\textit{hamartánei; pãs ho hamartánōn oukh heṓraken autòn oudè égnōken autón} ‘Everyone who remains in him does not sin; everyone who sins has neither
seen him nor come to know him', \iwi{NT 1 John 3:6}. 
Here, the verb ἁμαρτάνω \textit{hamartánō} ‘to sin' is
used and not the SVC, and there is no equation with the character of the
person, but with what else the person has or has not done (remained /
seen / known). 
The relationship between the two verses points to a
difference between sin and other actions, but also to the lack of
availability of the SVC in the context where there is the potential for
the action to be equated with the character of the agent.

Differentiating New Testament ethics from its classical precursors also
result-ed in significant vocabulary coinage and repurposing. 
I now turn % resulted manually hyphenated to avoid hbox issue.
to consider my hypothesis about the impact of the increasing use of the
noun ἁμαρτία \textit{hamartia} ‘sin' in the context of  other words and phrases.

\hypertarget{a.-ux3c4ux3bf-ux1f01ux3bcux1f71ux3c1ux3c4ux3b7ux3bcux3b1-sin}{%
\subsection{το ἁμάρτημα
\textit{to hamártēma} ‘sin'}\label{a.-ux3c4ux3bf-ux1f01ux3bcux1f71ux3c1ux3c4ux3b7ux3bcux3b1-sinCR}}

The -μα \textit{-ma} suffix creates a noun representing the product of the
verb.\footnote{See \citet{long_language_1968} on this process in Sophocles for a
  particularly strong discussion of the phenomenon.} 
  Again, the word
becomes steadily moralised as it develops. 
In \textit{Liddell-Scott-Jones}, we find
`failure, fault', in Muroako `sinful act\ldots failure
to achieve an aim\ldots penalty incurred for committing a
sin\ldots slaughtered animal offered to atone', and in \textit{A Greek-English lexicon of the New Testament and other early Christian literature} (BDAG)
`sin, transgression' \citep {liddell_greek-english_1996, muraoka_greek-english_2009, arndt_greek-english_2000}. 
In terms
of Christian sin, therefore, this noun has two key uses. 
It
differentiates Christian ethics from the language of Aristotle, where
ἁμαρτία \textit{hamartia} ‘sin' has a very specific Greek cultural remit, and it firmly
represents sin as the consequence of action, divorcing the action from
the agent, and potentially from the process.

There are, however, only four examples of ἁμάρτημα \textit{hamártēma} ‘sin' in the New Testament
(out of 14,727 attested in the \textit{Thesaurus Linguae Graecae}), only one of which is used with ποιῶ \textit{poiō} ‘to make, do' (see (\ref{ex13CR})).

 {
\ea\label{ex13CR} 
\glll πᾶν \textbf{ἁμάρτημα} ὃ ἐὰν \textbf{ποιήσῃ} ἄνθρωπος ἐκτὸς τοῦ σώματός ἐστιν· ὁ δὲ πορνεύων εἰς τὸ ἴδιον σῶμα ἁμαρτάνει\\
\textit{pãn} \textit{hamártēma} \textit{hò} \textit{eàn} \textit{poiḗsēͅ} \textit{ánthrōpos} \textit{ektòs} \textit{toũ} \textit{sṓmatós} \textit{estin}; \textit{ho} \textit{dè} \textit{porneúōn} \textit{eis} \textit{tò} \textit{ídion} \textit{sō̃ma} \textit{hamartánei}\\
    every.\textsc{nom} sin.\textsc{nom} \textsc{rel}.\textsc{acc} if do.\textsc{aor.sbjv.3sg} man.\textsc{nom} outside.of the.\textsc{gen} body.\textsc{gen} be.\textsc{prs.ind.3sg} the.\textsc{nom} \textsc{prt} be.sexually.immoraly.\textsc{prs.ptcp.nom} against the.\textsc{acc} own.\textsc{acc} body.\textsc{acc} sin.\textsc{prs.ind.3sg} \\

\glt `Every sin which a man might commit is outside his body; but the one who is sexually immoral sins against his own body' \\
\hspace*{\fill}(\iwi{NT 1 Corinthians 6:18})
 
\z
}

The verb ποιῶ \textit{poiō} ‘to make, do' is only used in the relative clause to refer back to the
noun, rather than independently, and is counterbalanced by the verb
ἁμαρτάνω \textit{hamartánō} ‘to sin' in the second phrase. 
There seems to be some kind of
interchangeability between the two here, but we do not have enough
examples to be sure of the usage pattern.\footnote{The greatest
  frequency of ἁμάρτημα \textit{hamártēma} ‘sin' is again in John Chrysostom, with other
  Christian literature providing the next most frequent sources.} 
  The
relative lack of ἁμάρτημα \textit{hamártēma} ‘sin' may also be explained by the existence of an SVC;
an SVC achieves morpho-syntactically what ἁμάρτημα \textit{hamártēma} ‘sin' achieves lexically
when compared with ἁμαρτία \textit{hamartia} ‘sin'; within the whole corpus, there are
under 100 examples of ποιῶ ἁμάρτημα \textit{poiō hamártēma} ‘to commit a sin' as an SVC, depending on definition,
making it not an unusual construction, but not one the New Testament
needs to use to achieve its theological goals.

Similar to -μα \textit{-ma} nouns acting as products of verbs, -σις \textit{-sis} nouns give the
process of the verb in action.\footnote{Again, see \citet{long_language_1968} for a
  thorough discussion of Sophocles' manipulation of this form.} 
  A
further way to consider and contextualise the use of SVCs in
differentiating product from process is to look at the relative
distribution of ἁμάρτησις \textit{hamártēsis} ‘sin' and verbs used with it. 
Of the 238 attested
uses of ἁμάρτησις \textit{hamártēsis} ‘sin' found in the \textit{Thesaurus Linguae Graecae}, only nine predate the Christian era; it
is sufficiently uncommon not even to appear in \textit{Liddell-Scott-Jones}. 
There is only one
example in the New Testament (\iwi{NT Matthew 18:21}), after which it grows in
popularity. 
Almost none are used with ποιῶ \textit{poiō} ‘to make, do'; while other -σις \textit{-sis} nouns are
used in SVCs post-classically, ἁμάρτησις \textit{hamártēsis} ‘sin' is not, except in later
commentaries on Ecclesiastes, and Theophanes Continuatus.\footnote{\iwi{Olympiodorus
  Diaconus Scr. Eccl. Commentarii in Ecclesiasten vol.93 pg.569 line 21};
  \iwi{Maximus Confessor Theol. Scholia in Ecclesiasten (in catenis: catena
  trium patrum) 7:111}; and \iwi{Theophanes Continuatus Chronogr. et Hist.
  Chronographia (lib. 1--6) pg.27 line 17}.} 
  This suggests, at first
reading, that it is a thoroughly Christian (rather than biblical) way of
expressing moral wrongdoing, which sits at odds with the rest of the
argument I am making in divorcing product from process. 
It may be,
however, that the crucial link is not between product and process, but
between agent and action.~
It may also demonstrate the development of
Christian thought in progress, from a biblical concept where sin and
sinner need to be divorced, with morphology providing the mechanism, to
later works where the lexicon supplies an alternative route.

Adding weight to my argument that the agency behind sin is not located
in the sinner (but perhaps in the devil), the agent noun ἁμαρτητής
\textit{hamartētḗs} ‘sinner' does not appear in the New Testament at all; indeed, it
is only used twice, both in Georgius Gemistus, suggesting that this
conflation between sin and sinner is very much not a Greek concept, let
alone a New Testament one.\footnote{Neither does the related term
  κακότης \textit{kakótēs} ‘wrongdoer' -- 765 full corpus uses) appear in the New
  Testament.} 
  This distinction between agent and action has significant
consequences for the concept of personhood\is{personhood} developed in the New
Testament. 
This links into the use of adjectives as substantives,
reducing people to their characteristics (e.g. \iwi{NT Luke 14:13}, κάλει
πτωχούς, ἀναπείρους, χωλούς, τυφλούς \textit{kálei ptōkhoús, anapeírous, khōloús, tuphloús} ‘call the beggars, cripples, hungry
and blind people', and \iwi{NT Luke 14:21} for the list remodelled). 
Where this
link between characteristic and person is made in the case of
disability, it is not made in the case of ethical action.\footnote{See
  particularly the work of Isaac \citet{soon_disability_2021, soon_disabled_2023} on disability in the New
  Testament.} 
  What we do find, however, are compound
verbs which express ethical concepts akin to sin in different but
related words, using adjectives with ποιῶ \textit{poiō} ‘to make, do', and it is to these that I
finally turn.

\hypertarget{b.-ux1f00ux3b3ux3b1ux3b8ux3bfux3c0ux3bfux3b9ux1ff6-and-ux3baux3b1ux3baux3bfux3c0ux3bfux3b9ux1ff6}{%
\subsection{ἀγαθοποιῶ \textit{agathopoiō̃} ‘to do good' and
κακοποιῶ~ \textit{kakopoiō̃} ‘to do bad'}\label{b.-ux1f00ux3b3ux3b1ux3b8ux3bfux3c0ux3bfux3b9ux1ff6-and-ux3baux3b1ux3baux3bfux3c0ux3bfux3b9ux1ff6CR}}

There are ten examples of ἀγαθοποιῶ \textit{agathopoiō̃} ‘to do good' in the New Testament, a synthetic
verb which may be read as counterbalancing sin. 
 Four are in Luke, five in 1 Peter, and one in 3 John.\footnote{\iwi{NT Luke 6:9}, \iwi{NT Luke 6:33} (x2), \iwi{NT Luke 6:35}, \iwi{NT 1 Peter 2:14}, \iwi{NT 1 Peter 2:15}, \iwi{NT 1 Peter 2:20}, \iwi{NT 1 Peter 3:6}, \iwi{NT 1 Peter 3:17}, \iwi{NT 3 John 1:11}}
 The use of the verb, however, is syntactically notable. 
  Only
2/10 uses are in finite forms; 6/10 are in participial phrases, echoing
e.g. ποιῶν ἁμαρτίαν \textit{poiōn hamartían} ‘committing a sin' in \iwi{NT 1 John}. There are only three examples of the
negative equivalent, κακοποιῶ \textit{kakopoiō̃} ‘to do bad', in Mark, Luke, and 1 Peter, that is, in
very similar contexts.\footnote{\iwi{NT Mark 3:4}, \iwi{NT Luke 6:9}, \iwi{NT 1 Peter 3:17}.} 
In
Luke and 1 Peter they are in the same phrase as ἀγαθοποιῶ \textit{agathopoiō̃} ‘to do good' and in \iwi{NT Mark
3:4} it is set against the periphrastic or, I would argue, active SVC
ἀγαθὸν ποιῆσαι \textit{agathòn poiē̃sai} ‘to do good'. 
In addition, the phrases all pertain to suffering and
death, and seem to have a particular semantic context which is
distinctive from the other contexts I am considering.

There are, therefore, alternatives to the SVC ποιῶ ἁμαρτίαν \textit{poiō hamartían} ‘to commit a sin' available to
New Testament authors, but they mainly do not use them. 
Although some
uses of ποιῶ ἁμαρτίαν \textit{poiō hamartían} ‘to commit a sin' are formulaic, it also clearly functions as a
phrase in its own right, distinct from the verb ἁμαρτάνω \textit{hamartánō} ‘to sin'.

\hypertarget{conclusion}{%
\section{Conclusion}\label{conclusionCR}}

In this chapter, I have traced the shift in the language of sin and error
to become more substantive as it becomes more ethically laden. This
relationship between philology and theology demonstrates one of the ways
in which the linguistic and cultural contexts of the New Testament had a
profound effect on the development of Christian thought.\footnote{See,
  for example, \citet{atkinson_theology_1944, wallace_greek_1996, hart_new_2017} on the
  relationship between theology and philology, and \citet{conybeare_classical_2021} for a view on the other way around.} This work, as I
take it further, has the potential to explain differences in Christian
approaches to sin and forgiveness\is{forgiveness} in general. 
Forgiving the sinner is a
lot easier when the sin is a separate entity from them, the product of a
process carried out by a person, that is, two stages removed from the
person. 
This construction of a New Testament personhood\is{personhood} in which people
are fundamentally linked to but distinct from their actions and
attributes may be important in a range of other contexts. 
Similarly,
exposing the development of some branches of Christianity (notably
Catholicism) away from a biblical way of expressing things leads to the
chance to explore more thoroughly what the impact of \emph{ad fontes}
and \emph{sola scriptura} meant in the Reformation.\footnote{I explore
  this relationship between theology, philology, pedagogy, translation,
  and the development of Reformation thought further in \citet{ryan_translating_2025}.}
The language of the New Testament may not be a consistent dialect, but
it does reflect shifts in forms of expression which are as much
theologically as either culturally or linguistically driven. 
There may
not be a consensus among those working in linguistics about precisely
what constitutes an SVC, and whether any definition is replicable
between languages, but there is a clear and consistent pattern of change
within Greek. 
A shift from a predominantly one-word expression of sin
(ἁμαρτάνω \textit{hamartánō} ‘to sin') to a multi-word phrase which is not significantly modified
(ποιῶ ἁμαρτίαν \textit{poiō hamartían} ‘to commit a sin') is clearly discernible. Alternatives to ποιῶ ἁμαρτίαν \textit{poiō hamartían} ‘to commit a sin' do
not perform the same function, but the SVC holds a unique place in the
New Testament in laying out a framework wherein a sinner is not
inherently identified with their sin, either morphologically, or
semantically. 
A semantically light verb has allowed for a new form of
ethical precision.


\section*{Abbreviations}\label{abbreviationsCR}
\begin{tabularx}{.5\textwidth}{@{}lQ@{}}
NT & New Testament \\
OT & Old Testament \\
\end{tabularx}

% I don't know why but this has brought up the word References twice in the compiled document.
\sloppy
\printbibliography[heading=subbibliography,notkeyword=this]

\end{document}
