\documentclass[output=paper,colorlinks,citecolor=brown]{langscibook}
\ChapterDOI{10.5281/zenodo.14017919}


\title{Proemium: Taking initiative} 
\author{Victoria Beatrix Fendel\affiliation{University of Oxford, UK}}  

\abstract{This is a \textit{proemium} on rather than an introduction to structures, such as \textit{to have an idea} and \textit{to take into consideration}, which we label support-verb constructions. The \textit{proemium} briefly introduces the reader to past definitions and current approaches (esp. the \textit{Funktionsverbgefüge}, \textit{constructions à verbe support}, and light-verb-construction approaches) and the range of corpora, each representing a different variety of Greek, discussed in this volume. Varieties range from the proto-language to the modern language and thus span a period of more than 3,000 years. The \textit{proemium} provides an overview of the chapters of this volume organising them along the three interfaces that support-verb constructions sit at, the syntax-lexicon, the syntax-semantics, and the syntax-pragmatics interfaces. It closes with a note on practicalities including the bilingual abstracts the reader will observe. Within a heterogenous group (of constructions), we strive for \textit{in varietate unitas}.


\bigskip
Это \textit{proemium} или точнее введение в структуры типа «иметь идею» или «принимать во внимание», которые мы называем конструкциями с опорным глаголом. В proemium читателю кратко представлены определения из прошлых исследований и современные подходы (в особенности \textit{Funktionsverbgefüge}, \textit{constructions à verbe support} и подходы на основе легких глаголов) наряду с гаммой корпусов где каждый представляет собой разновидность греческого языка представленного в этой книге. Разновидности языка варьируют от протоязыка вплоть до современного языка, таким образом покрывая период более 3000 лет. Proemium предоставляет обзор глав этой книги, организуя их на основе трёх граней на которых расположены конструкции с опорным глаголом: грань синтаксиса и лексикона, синтаксиса и семантики, и синтаксиса и прагматики. В заключение приводится обсуждение практических аспектов, включая двуязычные аннотации замеченные ранее читателем. В гетерогенной группе (конструкций), мы стремимся к \textit{in varietate unitas}.}


\IfFileExists{../localcommands.tex}{
   \addbibresource{../localbibliography.bib}
   \usepackage{langsci-optional}
\usepackage{langsci-gb4e}
\usepackage{langsci-lgr}

\usepackage{listings}
\lstset{basicstyle=\ttfamily,tabsize=2,breaklines=true}

%added by author
% \usepackage{tipa}
\usepackage{multirow}
\graphicspath{{figures/}}
\usepackage{langsci-branding}

   
\newcommand{\sent}{\enumsentence}
\newcommand{\sents}{\eenumsentence}
\let\citeasnoun\citet

\renewcommand{\lsCoverTitleFont}[1]{\sffamily\addfontfeatures{Scale=MatchUppercase}\fontsize{44pt}{16mm}\selectfont #1}
  
   %% hyphenation points for line breaks
%% Normally, automatic hyphenation in LaTeX is very good
%% If a word is mis-hyphenated, add it to this file
%%
%% add information to TeX file before \begin{document} with:
%% %% hyphenation points for line breaks
%% Normally, automatic hyphenation in LaTeX is very good
%% If a word is mis-hyphenated, add it to this file
%%
%% add information to TeX file before \begin{document} with:
%% %% hyphenation points for line breaks
%% Normally, automatic hyphenation in LaTeX is very good
%% If a word is mis-hyphenated, add it to this file
%%
%% add information to TeX file before \begin{document} with:
%% \include{localhyphenation}
\hyphenation{
affri-ca-te
affri-ca-tes
an-no-tated
com-ple-ments
com-po-si-tio-na-li-ty
non-com-po-si-tio-na-li-ty
Gon-zá-lez
out-side
Ri-chárd
se-man-tics
STREU-SLE
Tie-de-mann
}
\hyphenation{
affri-ca-te
affri-ca-tes
an-no-tated
com-ple-ments
com-po-si-tio-na-li-ty
non-com-po-si-tio-na-li-ty
Gon-zá-lez
out-side
Ri-chárd
se-man-tics
STREU-SLE
Tie-de-mann
}
\hyphenation{
affri-ca-te
affri-ca-tes
an-no-tated
com-ple-ments
com-po-si-tio-na-li-ty
non-com-po-si-tio-na-li-ty
Gon-zá-lez
out-side
Ri-chárd
se-man-tics
STREU-SLE
Tie-de-mann
}
   \boolfalse{bookcompile}
   \togglepaper[1]%%chapternumber
}{}


\begin{document}
% LaTeX is struggling with overfull hboxes in some ways so let it add a bigger stretch if needed.
\emergencystretch 3em


\maketitle

\section{Approach(es)}
The Oxford English Dictionary defines \textit{proemium} (or proem) as 
‟[a]n introductory discourse at the beginning of a piece of writing; a preface, preamble". \citet[1]{melcukGeneralPhraseology2023} begins his \textit{General Phraseology} with the definition that ‟a preface is supposed to be no more than a polite greeting addressed to the reader and, therefore, to carry no, or very little, relevant information". Thus, this is a \textit{proemium} rather than an introduction. 
%Matthewyies has defined command for Victoriayies 


It begins like Vergil’s \textit{Aeneid} (1st c. BC) (\textit{arma virumque cano} ‘the weapons and the man I sing about’) – performative and declarative. The following paragraphs briefly outline the motivation and background for this volume, the timeframes and datasets taken into consideration, and the questions and issues that permeate the chapters of the volume. Less craftily than Vergil, this \textit{proemium} will need several paragraphs to provide a brief overview of the chapters of the volume. 


This volume arose from the conference \textit{Between lexicon and grammar? Support-verb constructions in the corpora of Greek} which took place at the Clarendon Institute, University of Oxford, United Kingdom on 5 to 6 September 2023. The conference was linked to the Leverhulme-funded project \textit{Giving gifts and doing favours: Unlocking Greek support-verb constructions} (grant n. ECF-2020-181, 2020–2024, University of Oxford). The project focusses on one corpus, literary classical Attic (prose, oratory, and historiography) shown in \tabref{tab:myname:ECF}:


\begin{table}
\small
\caption{ECF Leverhulme Corpus}
\label{tab:myname:ECF}
\begin{tabularx}{\textwidth}{>{\hsize=.3\hsize}X  X}
  \lsptoprule
  Historiography (\textbf{203,186~words}):    & Thucydides,~\emph{Histories}~vol.~1--5~(\textbf{98,945}); Xenophon,~\emph{Anabasis}~vol.~1--4~(\textbf{32,034}), \emph{Memorabilia}, vol.~1--4 (\textbf{36,465}), \emph{Hellenica} vol.~1--4 (\textbf{35,742});\\
  \midrule
  Oratory (\textbf{143,937~words}):          & Antiphon,~\emph{Speeches}~1--6~(\textbf{18,605}); Isocrates,~\emph{Speeches}~1--6 and~13
(\textbf{37,311}); Isaeus,~\emph{Speeches}~1--8~(\textbf{25,018}), Lysias,~\emph{Speeches} 1, 3, 7, 12, 14,
19, 22, 30, 31, 32 (\textbf{24,130}); Demosthenes,~\emph{Speeches} 1, 2, 3, 4, 6, 9, 18 (\textbf{38,873});\\
\midrule
  Prose (\textbf{145,497~words}):    & Plato,~\emph{Gorgias}~(\textbf{27,790}), \emph{Phaedrus}~(\textbf{17,271}), \emph{Republic},~vol.~1--3~(\textbf{28,688});
Aristotle,~\emph{Rhetoric}~(\textbf{44,312}), \emph{Politics},~vol.~1--3 (\textbf{27,436})\\
  \lspbottomrule
\end{tabularx}
\end{table}


The \textit{ECF Leverhulme corpus}\footnote{\url{https://ora.ox.ac.uk/objects/uuid:7ab3b631-6c04-42fe-ad80-617b7eaa74f9} (last accessed 08 April 2024).} \citep{fendelDiscourseCohesionXenophon2023} is implemented into Sketch Engine, an online corpus analysis tool, and forms the basis for the new PARSEME Ancient Greek corpus\is{PARSEME}. Annotation guidelines are available already (select the language label ‘GRC’ in the guidelines)\footnote{\url{https://parsemefr.lis-lab.fr/parseme-st-guidelines/1.3/index.php} (last accessed 02 April 2024).}, as are the working-group documents.\footnote{\url{http://www.ancientgreekmwe.com} (last accessed 02 April 2024).}


The project has approached this corpus primarily from a linguistic perspective with an interest in the morpho-syntax, semantics, and pragmatics of support-verb constructions. However, inevitably, there has been a lexical component. The syntax-lexicon interface\is{interface}, at which support-verb constructions are verbal multi-word expressions and complex predicates and can act as syntagms or words, is the starting point for this volume. 


Twenty years after \citet{GrossPontonx2004} \textit{Verbes supports: Nouvel état des lieux}, two recent edited volumes with a specific interest in corpus languages reflect the importance of the syntax-lexicon interface when examining support-verb constructions. \citet{banosCollocationsTheoreticalApplied2022} \textit{Collocations in theoretical and applied linguistics: from classical languages to Romance languages} focusses on the lexical characteristics of support-verb constructions and their diachronic development (see also \textit{Diccionario de Colocaciones del Griego Antiguo}\footnote{\url{https://dicogra.iatext.ulpgc.es/dicogra/} (last accessed 06 April 2024).}); \citet{PompeiPiunno2023} \textit{Light verb constructions as complex verbs: Features, typology, and function} focusses on the syntactic characteristics of support-verb constructions from a cross-linguistic perspective. The contributions below show amply that even considering the lexicon and syntax is a simplification of the fascinating diversity. 


Indeed, the first stumbling stone is the exact delimitation of the group of support-verb constructions, in other words their definition.\footnote{Each chapter provides the author's definition of the support-verb constructions for this reason.} Different approaches accept different degrees of internal heterogeneity of this group of constructions. There are three prominent approaches to structures such as δίκην δίδωμι \textit{dikēn didōmi} in (\ref{ex:plato1}) (repeated in (\ref{ex:Plato2}) below): 

%Plato example 
\ea\label{ex:plato1}
\glll τὸ \textbf{διδόναι} \textbf{δίκην} καὶ τὸ κολάζεσθαι δικαίως ἀδικοῦντα ἆρα τὸ αὐτὸ καλεῖς; \\ \textit{to} 		\textit{didonai} 	\textit{dikēn} 			\textit{kai} 	\textit{to} \textit{kolazes$t^{h}$ai} 		\textit{dikaiōs} 	\textit{adikounta} 		   \textit{ara} 		\textit{to} \textit{auto} 		         \textit{kaleis}? \\ the.\textsc{acc} 	give.\textsc{inf.act} punishment.\textsc{acc}  and 	the.\textsc{acc} punish.\textsc{inf.pass} 	just.\textsc{adv} 	wrong.\textsc{prs.ptcp.act.acc}  \textsc{prt}.\textsc{q}	the.\textsc{acc} same.\textsc{acc} 	 call.\textsc{prs.act.2sg}\\
\glt `Are you saying that ‘paying the price for one’s actions’ and ‘justly getting punished’ when one does wrong are the same?' \\
\hspace*{\fill}(\iwi{Plato, \textit{Gorgias} 476a} (CG))
\z


The first approach is the German research strain of \textit{Funktionsverbgefüge} `function-verb constructions'\is{function-verb construction} (with its sub-category of \textit{Nominalisierungs\-verb\-gefüge} `nominalisation-verb constructions') (\citealt{vonpolenzFunktionsverbenFunktionsverbgefugeUnd1987, kamberFunktionsverbgefugeEmpirischKorpusbasierte2008, storrerCorpusbasedInvestigationsGerman2009, deknopFunktionsverbgefuegeImFokus2020}, applied to early Greek by \citealt{schutzeichelIndogermanischeFunktionsverbgefuege2014}, and to classical Greek by \citealt{tronciSyntagmePrepositionnelGrec2016}, \citealt{tronciLexiconsyntaxInterfaceAncient2017}).
The focus lies on verb + prepositional phrase constructions, such as \textit{in Betracht ziehen} `to take into consideration' rather than verb + object constructions, such as \textit{Aufmerksamkeit schenken} `to pay attention'. Furthermore, the focus is on the verb (and preposition) in the construction rather than the noun, as Kamber’s concept of \textit{Umrahmte Schnittmengen}\is{prototype semantics} shows \citep[23]{kamberFunktionsverbgefugeEmpirischKorpusbasierte2008}. The latter is an attempt at creating sub-categories within a heterogenous group of constructions. 


The second approach is the English research strain of light-verb constructions\is{light-verb construction}. The term was coined by \citet{jespersenModernEnglishGrammar1954} and remains in use in much of English research literature \citep{buttStructureComplexPredicates1995, buttLightVerbJungle2010, buttDiachronicPertinacityLight2013}.\footnote{Light verbs combine with a nominal component to form the predicate of a sentence. They do not add voice, aspect, or polarity to the predicate phrase.} The term light verb has been repurposed extensively in language-contact studies\is{language contact} (\citealt[132]{bakkerMixedLanguagesAutonomous2003}, \citealt[134–139]{myers-scottonContactLinguisticsBilingual2002}, \citealt{reintgesCodemixingStrategiesCoptic2001}, \citealt[148]{ronanMobilizingLinguisticConcepts2012}, \citealt[203]{rutherfordBilingualismRomanEgypt2010}, applied to early Byzantine non-literary Greek by \citealt{fendelCopticInterferenceSyntax2022}) in order to refer to structures such as (\ref{ex:Narmouthis}) and (\ref{ex:Cypriot}):


%Narmouthis ostracon
\ea\label{ex:Narmouthis}
\settowidth \jamwidth{(Demotic)}
\gll \textit{w3ḥ} \textit{n3ḫe} \textit{p3} \textit{tmj} \textit{ir} διώκιν \textit{n.im=j} \\
\textsc{prf} people this village do prosecute.\textsc{prs.inf} \textsc{dom}=\textsc{1sg}\\ \jambox{(Demotic)}
\glt ‘the people of the village prosecuted me’ \\
\hspace*{\fill}(\iwi{Narmouthis ostracon n. 103} \citealt[203]{rutherfordBilingualismRomanEgypt2010})


%Cypric Greek
\ex\label{ex:Cypriot}
Cypriot Greek 


\ea κάνω/κάμνω ψώνια \textit{káno/kámno psonia} ‘to do shopping’


\ex κάνω/κάμνω γυμναστική \textit{káno/kámno gimnastiki} ‘to do gymnastics'


\ex κάνω/κάμνω τζόκινγκ \textit{káno/kámno jogging} ‘to do/go jogging’


\ex κάνω/κάμνω ζάπινγκ \textit{káno/kámno zapping} ‘to do zapping/to zap’ \\
\hspace*{\fill}(\citet[73]{fotiouCodeChoiceEnglishCypriot2010})
\z
\z
%comment to explain this is below

In language-contact settings\is{language contact}, the light verb used most commonly is the verb ‘to do’, as in (\ref{ex:Narmouthis}) and (\ref{ex:Cypriot}). A light verb, i.e. a verb that does not contribute aspects of meaning, is used to integrate a loan item into the morpho-syntactic frame of the target language.
\citet[73]{fotiouCodeChoiceEnglishCypriot2010} observes the parallel existence of ‟native compounds with 
\textit{káno/kámno} ['to do'], 
such as \textit{káno/kámno psonia} (do shopping)" alongside 
‟borrowings in the form of bilingual compound verbs, such as \textit{káno jogging} (do jogging)".
The same is true for the situation in Demotic, shown in (\ref{ex:Narmouthis}) \citep{funkDifferentialLoanCoptic2017, grossmanDialectalVariationLanguage2017, egediRemarksLoanVerb2017}, and  continued into later Coptic Egyptian\is{Coptic}.   


The term light verb has also been adopted in the natural language processing\is{Natural Language Processing} context, e.g. by the PARSEME initiative\is{PARSEME}. Their decision tree for LVCs (light-verb constructions\is{light-verb construction}) is reproduced in \figref{fig:LVSPAR}\footnote{\url{https://parsemefr.lis-lab.fr/parseme-st-guidelines/1.3/index.php?page=050_Cross-lingual_tests/020_Light-verb_constructions__LB_LVC_RB_} (last accessed 27 April 2024).}:


%parseme LVC tree
\begin{figure}
\caption{PARSEME LVC-specific decision tree}
\label{fig:LVSPAR}
\includegraphics[width=\textwidth]{figures/decisiontree.png}
\end{figure}


Any structures in which the verb adds properties, such as aspect (e.g. inchoative), voice (e.g. passive), polarity (e.g. contrastive negation), and the like to the predicate phrase are excluded. The testing starts from the noun, i.e. the semantic head, rather than the verb.


The third approach is the French research strain of \textit{constructions à verbe support} (support-verb constructions\is{support-verb construction}) that originated in the work of the \textit{Laboratoire d'Automatique Documentaire et Linguistique}\is{Laboratoire d’Automatique Documentaire et Linguistique} (esp. \citealt{grossFonctionSemantiqueVerbes1998}, applied to classical literary Greek by \citealt{jimenezlopezSupportVerbConstructions2016}). The verb plays a supporting role rather than being light. It can be used to add properties such as aspect, voice (\textit{constructions converses}), and polarity, see (\ref{ex:GrossSVC} a–c) \citep{giry-schneiderNominalisationsFrancaisOperateur1978, vivesAvoirPrendrePerdre1983, grossConstructionsConversesFrancais1989}, as well as for register-/genre-/style-related nuancing \citep{biberRegisterGenreStyle2009, melcukVerbesSupportsSans2004}, see (\ref{ex:GrossSVC}d): 


\ea\label{ex:GrossSVC}
Aspect, diathesis, polarity, and context \citep{grossFonctionSemantiqueVerbes1998}


\ea \textit{garder, prendre, perdre} (e.g. \textit{de l'importance}) ‘to keep, to take, to lose’ (durative, inchoative, terminative)


\ex \textit{donner} (e.g. \textit{une gifle}) ‘to give’ (causative)


\ex \textit{répéter la phrase} ‘repeat the sentence’ (repetition); \textit{montre du courage} ‘show courage’ (exteriorisation); \textit{abandonner, manquer} (e.g. \textit{l'énergie}) ‘to abandon, to lack’ (negation) \\


\ex \textit{passer} vs. \textit{signer une contrat} ‘to approve vs. sign a contract’ \\
\z
\z
%when I start a new level of identation, I use ea, otherwise I use ex, I can also do this betweeen examples if I am crazy as in (2) and (3) 

Support verbs contrast with \textit{verbes distributionnels} (such as \textit{manger} ‘to eat’) which fill the predicate slot in the syntactic structure on their own, as opposed to support verbs which need to combine with a predicative noun to fill the predicate slot. The group of support verbs contains a sub-class, the \textit{verbes supports appropriés}\is{verbes support appropriés} \citep{grossManuelAnalyseLinguistique2012}, such as Latin \textit{committere} ‘to commit’ with nouns referring to crimes \citep{roeschFacinusFacereFacinus2018}. 


While the volume adopts the term support-verb construction\is{support-verb construction} from the French tradition in its title, the contributors work with varying frameworks casting the net more or less wide. Depending on framework, a structure such as δίκην δίδωμι \textit{dikēn didōmi} in (\ref{ex:Plato2}) (repeated from above) would thus qualify as a lexical passive, a verbal idiomatic expression, or be excluded from the range of structures assessed entirely. 


%Plato example 
\ea\label{ex:Plato2}
\glll τὸ \textbf{διδόναι} \textbf{δίκην} καὶ τὸ κολάζεσθαι δικαίως ἀδικοῦντα ἆρα τὸ αὐτὸ καλεῖς; \\ \textit{to} 		\textit{didonai} 	\textit{dikēn} 			\textit{kai} 	\textit{to} \textit{kolazes$t^{h}$ai} 		\textit{dikaiōs} 	\textit{adikounta} 		   \textit{ara} 		\textit{to} \textit{auto} 		         \textit{kaleis}? \\ the.\textsc{acc} 	give.\textsc{inf.act} punishment.\textsc{acc}  and 	the.\textsc{acc} punish.\textsc{inf.pass} 	just.\textsc{adv} 	wrong.\textsc{prs.ptcp.act.acc}  \textsc{prt}.\textsc{q}	the.\textsc{acc} same.\textsc{acc} 	 call.\textsc{prs.act.2sg}\\
\glt `Are you saying that ‘paying the price for one’s actions’ and ‘justly getting punished’ when one does wrong are the same?' \\
\hspace*{\fill}(\iwi{Plato, \textit{Gorgias} 476a} (CG))
\z


If accepted as a support-verb construction\is{support-verb construction}, we would consider the nominal element (δίκην \textit{dikēn}) the predicative noun, the verbal element (δίδωμι \textit{didōmi}) the light/support/function verb, and the simplex verb which is functionally although not formally related (κολάζεσθαι \textit{kolazes$t^{h}$ai}) the base-verb construction\is{base-verb construction}. While some approaches and contributors consider the existence of a formally or functionally related base-verb construction a criterion to define support-verb constructions, others will dismiss this criterion on the basis that language is not redundant. 


Faced with the diversity of approaches and the magnitude of disagreements arising from them when working with as internally diverse a group of constructions as support-verb constructions, we still strive for \textit{in varietate unitas}.


\section{Corpora}
All the contributions in the volume take a corpus-based approach in order to lend empirical support to the observations made. Except for Giouli’s study of modern Greek, the contributions of the volume examine varieties of Greek that are only attested today in written form\is{corpus language}. The native speakers of these languages are the texts \citep[43]{fleischmanMethodologiesIdeologiesHistorical2000}. It is these native speakers that we question and interview. Like any native speakers, our texts represent idiosyncrasies (idiolects) along with geographically (dialect), societally (sociolect), or diachronically conditioned differences. 


The corpora considered in the present volume span over 2,000 years. For the core time periods, we adopt the following timeframes: Archaic Greek (AG) pre 5th c. BC; Classical Greek (CG) 5th/4th c. BC; Ptolemaic Greek (PG) 3rd–1st c. BC; Roman Greek (RG) 1st–3rd c. AD; Early Byzantine Greek (EBG) 4th–7th c. AD, Medieval Greek (MG) post 7th c. AD. If items are e.g. 4th–3rd c. BC, they are counted in PG; if items are e.g. 3rd–4th c. AD, they are counted in EBG. Both Giouli’s modern Greek corpus and Ittzés’ work on proto-Greek fall outside of these timeframes and constitute the edges of the volume’s coverage.


In the first footnote of each chapter, the reader will find the link to the dataset that the chapter is based, on except in two cases. Ittzés’ article on the proto-language does not have a dataset as it is based on internal and comparative reconstruction of a variety of the language that is unattested in written sources. Miyagawa examines Greek’s long-term contact language Coptic\is{language contact}.\footnote{Coptic\is{Coptic} is the final stage of the Egyptian language when written with the Coptic alphabet (from ca. AD 100 onwards) \citep{quackHowCopticScript2017}. This alphabet is an adaptation of the Greek alphabet \citep{fendelMissingPieceJigsaw2021}.} For ease of access and overview, all the datasets (corpora) that are examined by the contributions to the volume are listed below in chronological order: 

\begin{enumerate} 

\item	Squeri – Hippocratic Corpus (5th/4th c. BC) \url{http://dx.doi.org/10.5287/ora-n652gamyj};

\item Pompei, Pompeo, and Ricci – texts of the \textit{Thesaurus Linguae Graecae} excluding texts classified as \textit{Fragmenta} (5th c. BC – 2nd c. AD) \url{https://stephanus.tlg.uci.edu};

\item Veteikis – Aristotle’s Rhetoric (4th c. BC) \url{http://dx.doi.org/10.5287/ora-n652gamyj};


\item Baños and Jiménez López – the biblical corpora (the Septuagint, the Greek New Testament, the Vetus Latina, and Jerome’s Vulgate) (3rd c. BC to 4th c. AD) \url{https://doi.org/10.21950/E98VTJ};

\item Ryan – the New Testament (1st/2nd c. AD) \url{http://dx.doi.org/10.5287/ora-dqjeo65n5}; 

\item Madrigal Acero – selection of classical literary Attic and Ionic prose and verse (Aeschylus, Sophocles, Euripides, Aristophanes, Xenophon, Thucydides, Herodotus, Lysias, Demosthenes, Andocides, Plato, Aristotle) (5th/4th c. BC) and a selection of archaic, classical, and early imperial Latin prose and verse (Cicero, Caesar, Catullus, Martial, Livy, Plautus, Sallust, Tacitus, Terence) (2nd c. BC – 1st c. AD) \url{http://dx.doi.org/10.5287/ora-n652gamyj};


\item Vives Cuesta – selection of hagiographic texts: (a) New Testament (1st c. AD) (\textit{Evangelium secundum Matthaeum}, \textit{Evangelium secundum Lucam}, \textit{Epistula Pauli ad Corinthios} i–ii, \textit{Epistula Pauli ad Hebraeos}), (b) proto- and mezzo-byzantine hagiography (5th–9th c. AD) (\textit{Vita antiquior Sancti Danielis Stylitae} (BHG 489), \textit{Vita et martyrium sancti Anastasii Persae} (BHG 84), \textit{Martyrium antiquior sanctae Euphemiae} (BHG 619), \textit{Vita Stephani Iunioris} (BHG 1666), \textit{Vita Symeonis Stylitae senioris} (BHG 1683)), (c) metaphrastic hagiography (10th c. AD) (\textit{Passio sancti Anastasii Persae} (BHG 85), \textit{Passio sanctae Euphemiae} (BHG 620), \textit{Vita tertia Sancti Danielis Stylitae} (BHG 490), \textit{Vita Stephani Iunioris} (BHG 1667), \textit{Vita sancti Symeonis Stylitae} (BHG 1686)), (d) Comnene and late Byzantine hagiography (12th–14th c. AD) (\textit{Vita sancti Zotici} (BHG 2480), \textit{Vita Leontii Patriarchae Hierosolymorum} (BHG 985), \textit{Vita sancti Bartolomaei conditoris monasterii sancti Salvatoris Messanae} (BHG 235), \textit{Miracula sancti apostoli Marci} (BHG 1036m), \textit{Vita sancti Lazari} (BHG 980)) \url{http://dx.doi.org/10.5287/ora-n652gamyj};


\item Giouli – selection of news pieces, blogs, and Wikipedia articles from the web (manually collated) along with parliamentary debates and Wikinews articles (via the Greek Dependency Treebank \url{https://universaldependencies.org/treebanks/el_gdt/index.html}) (1453-present) \url{http://hdl.handle.net/11372/LRT-5124};


\bigskip
\item Miyagawa – Coptic Gospel of Thomas from the Nag Hammadi Codex II (4th/5th c. AD) (images) \url{http://gospel-thomas.net/x_facs.htm} and Coptic Letter to Aphthonia written by Besa (6th to 8th c. AD) \url{https://data.copticscriptorium.org/texts/besa_letters/to-aphthonia/}. 

\end{enumerate} 



The datasets are all available in open-access format and we hope that they will constitute the basis for many future studies building on the present authors’ work. 


\section{Interface(s)}
The contributions of this volume are diverse not only with regard to the definitions they apply and the native speakers they interview (the corpora they use) but also with regard to the perspectives they adopt on support-verb constructions. 


The multiple perspectives adopted are primarily caused by support-verb constructions sitting at three interfaces\is{interface}. 
\begin{itemize}
    \item The syntax-lexicon interface has found its way into the title of this volume, and Plato’s comment in (\ref{ex:Plato2}) quoted above illustrates the issue. Do we consider support-verb constructions lexemes to be listed in a dictionary (like the corresponding base verbs if available) or syntagms obeying the laws of the morpho-syntax? 
    \item The syntax-semantics interface is illustrated e.g. by Gross’ \textit{constructions converses}, which are lexical passives that if we believe Plato (\textit{Gorgias} 476d) include δίκην δίδωμι \textit{dikēn didōmi} in (\ref{ex:Plato2}). 
    \item The syntax-pragmatics interface has been touched upon with Gross’ register-/genre-/style-related options but is also visible in the patterns of negation with support-verb constructions in literary classical Attic, where considerations of intensity and contrast seem to determine the syntactic pattern used \citep{fendelHavenGotClue2023}. 
\end{itemize}

The volume is structured along these interfaces. The first section focusses on the outer edges of the corpora covered, whereas sections two to four each focus on one of the interfaces. 


The \textbf{first section} of the volume (Between too little and too much: the origins of data) contains the two contributions that form as regards empirical data the outer edges of the period this volume covers, Ittzés’ examination of the proto-language and Giouli’s account of the modern language. 


\textbf{Chapter 1 by Ittzés} examines traces in amongst others Greek that would suggest that support-verb constructions existed in Proto-Indo-European. Proto-Indo-European is the reconstructed proto-language from which the daughter languages branched off over time (for an accessible introduction, see e.g. \citealt{sihlerNewComparativeGrammar2008}). The Hellenic branch which Greek belongs to is only one of the branches that have been reconstructed. For example, Latin would be part of the Italic branch. Reconstruction of the proto-language is achieved either by comparative methods, i.e. comparing material from different branches in order to determine the moment when they went their separate ways (e.g. the Hellenic and Italic branches), or by internal reconstruction, i.e. comparing material from different stages of the language in one branch in order to determine the moment when subbranches split off (e.g. Mycenaean, the archaic and classical Greek dialects, etc. in the Hellenic branch). Given the reliance on reconstruction for the proto-language, Ittzés emphasises the need to rely on empirical provability (i.e. with data from the daughter languages) rather than theoretical possibility (based on reconstructed processes of development). In particular, he emphasises the need to rely on comparative data rather than overstate internal reconstruction, especially in the case of support-verb constructions which are susceptible to variation synchronically and diachronically. Ittzés critically examines as traces of support-verb constructions in the proto-language especially the so-called root extensions (\textit{Wurzelerweiterungen}) which would have become such due to univerbation and subsequent reanalysis. He applies a narrow definition of support-verb constructions, in that the verb does not add lexical semantics to the support-verb construction but only supplies verbal morphology. Thus, the verb is truly light and a function word. His specific interest lies with *$d^h$e$h_1$ ‘to put’ which underlies e.g. Greek τίθημι \textit{tí$t^h$ēmi} ‘to put’ and Latin \textit{facio} ‘to do’. While from a typological perspective, Ittzés argues that support-verb constructions existed in the proto-languages, he cautions that empirical evidence of specific exponents of the group of constructions are virtually absent because of the impossibility of corpus-based investigations. 


\textbf{Chapter 2 by Giouli} approaches support-verb construction from the perspective of natural language processing. Her corpus consists of modern Greek internet data including news pieces, blog posts, and Wikipedia articles but also parliamentary debates, thus covering a range of genres and registers. Her work is embedded in the context of the PARSEME initiative, which casts the net around support-verb constructions (light-verb constructions in their terminology) narrow and wide at the same time. Semantically, PARSEME only allows for constructions in which the verb does not contribute lexical semantics; syntactically, PARSEME allows for the predicative noun to appear in the subject, object, and prepositional complement slots. The initiative, whilst relying on ‟universally" applicable guidelines to determine what to annotate as support-verb constructions (light-verb constructions), acknowledges that these ‟universal" categories have language-specific realisations, of which Giouli introduces several for modern Greek. Unlike other contributions in this volume, in line with the natural language processing approaches, she applies a deterministic procedure, such that fuzzy lines, even if they exist during the annotation and evaluation stages, disappear in the result stage, i.e. every structure gets assigned a specific category (with \textit{light-verb construction (LVC)} being one of them). Giouli’s corpus, unlike the other corpora presented in this volume, is still continuously growing in the context of the PARSEME initiative. 


The \textbf{second section} of the volume (Between comparative concept and descriptive category: the syntax-semantics interface) taps into the difficulty that support-verb constructions have repeatedly been considered a comparative concept (\citealt[96]{savaryPARSEMEMultilingualCorpus2018} \citealt[29–31]{hoffmannLatinSupportverbConstructions2023}), i.e. ‟a concept created by comparative linguists for the specific purpose of crosslinguistic comparisons" \citep[665]{haspelmathComparativeConceptsDescriptive2010}. However, the instantiation of a comparative concept is language-specific, what \citet[664]{haspelmathComparativeConceptsDescriptive2010} terms descriptive categories. Madrigal Acero explores language specificity by means of a comparison of structures with the support verb ‘to use’ in classical Greek and Latin, whereas Jiménez López and Baños focus on the translation process of the post-classical New Testament. Both contributions square language-specific syntactic structures with across-language semantics\is{interface}. 


\textbf{Chapter 3 by Madrigal Acero} applies a comparative approach to the role that verbs meaning ‘to use’ (Greek χράομαι \textit{$k^{h}$raomai} and Latin \textit{utor}) play in support-verb constructions. The verb meaning ‘to use’ in Greek (χράομαι \textit{$k^{h}$raomai}) can be pragmatically motivated when alternating with a neutral option with ἔχω \textit{e$k^{h}$ō} ‘to have’ or ποιέομαι \textit{poieomai} ‘to do’; alternatively, it can be a diathetically motivated option when alternating with δίδωμι \textit{didōmi} ‘to give’ or τίθημι \textit{ti$t^{h}$ēmi} ‘to put’. The same applies to Latin \textit{utor} ‘to use’ which can be pragmatically motivated when alternating with \textit{facere} ‘to do’ or \textit{habere} ‘to have’ but can also be diathetically motivated when alternating with \textit{dare} ‘to give’, \textit{facere} ‘to make’, and \textit{ferre} ‘to bring’. Her approach in this way aligns with the framework of prototype semantics and support-verb-construction families surrounding predicative nouns (e.g. \textit{to provide help, to get help, to have help}) \citep{kamberFunktionsverbgefugeEmpirischKorpusbasierte2008}. Madrigal Acero’s corpus selection contains both Greek and Latin texts written in verse rather than prose. This allows her to disprove the often-assigned label of ‟prose phrases" for support-verb constructions.


\textbf{Chapter 4 by Baños and Jiménez López} examines the Greek and Latin biblical corpora (the Greek New Testament, the Septuagint, the \textit{Vetus Latina}, and Jerome’s Vulgate) (3rd c. BC to 4th c. AD) from a comparative perspective. They cast the net wide by including into the group of support-verb constructions (i) structures with the predicative noun in the subject slot, the direct-object slot, and the complement slot of a preposition, (ii) structures in which the support verb adds information about aspect, diathesis, and intensity, and (iii) structures in which the predicative noun takes the form of a syntactic nominalisation (e.g. Latin \textit{necessarium}). They show how the four gospels differ due to the writers’ idiosyncrasies (including due to their bilinguality) (cf. \citealt{hamersBilingualityBilingualism2000}), different translation practices (from Greek into Latin), and differences in natural language usage regarding support-verb constructions as opposed to simplex verbs in Latin and Greek. The chapter illustrates the language-specificity of support-verb constructions, e.g. with συμβούλιον διδόναι \textit{sumboulion didonai} ‘to deliberate’ as opposed to \textit{consilium dare} ‘to counsel’. While their primary focus is synchronic, succinct diachronic observations open up further avenues, e.g. regarding support-verb constructions with συμβούλιον \textit{sumboulion} ‘advice’. 



The \textbf{third section} of the volume (Between context and co-text: the syntax-pragmatics interface) turns to the syntax-pragmatics interface\is{interface}. Support-verb constructions are embedded in their structural (and semantic) co-text \citep[119]{crystalDictionaryLinguisticsPhonetics2008} but like any other item can also be pointing to the contextual setting in which the utterance containing the support-verb construction is embedded (cf. \citealt{benteinDimensionsSocialMeaning2019}).  Squeri investigates edge cases of support-verb constructions in the classical Hippocratic corpus of medical writings; Veteikis casts the net wide in the classical Aristotelian corpus on rhetoric; and Vives Cuesta argues for a morpho-syntactic distinction becoming a pragmatically motivated one in hagiographical writings. 


\textbf{Chapter 5 by Squeri} examines the classical Hippocratic corpus (5th/4th c. BC) of medical treatises. This technical register allows her to consider to what extent structures with χράομαι \textit{$k^{h}$raomai} ‘to use’ (+ dative case) are support-verb constructions that index a technical context. Squeri focusses on four predicative nouns κατάπλασμα \textit{kataplasma} ‘plaster’, κλυσμός \textit{klusmos} ‘douche’, κλύσμα \textit{klusma} ‘enema’, and πρόσθετον \textit{pros$t^{h}$eton} ‘vaginal suppository’. These are non-prototypical predicative nouns in that (i) functionally, they acquire an eventive meaning when used as predicative nouns in a support-verb construction, and (ii) formally, they are not deverbal event nouns (e.g. in -σι- \textit{-si-}). Squeri’s chapter explores to what extent such non-prototypical predicative nouns appear specifically in the technical writings of the Hippocratic corpus and to what extent there is a relationship between support-verb constructions and cognate-object structures. 


\textbf{Chapter 6 by Veteikis} examines the first two books of Aristotle’s \textit{Rhetoric} (4th c. BC). His interest lies with the stylistic value of support-verb constructions while acknowledging that in Aristotle’s \textit{Rhetoric} a technical register and the author’s idiolect play into the surface representation of the support-verb constructions observed. His approach is focussed on (i) support-verb-construction families, i.e. what support verbs appear with each predicative noun of interest and how support verbs modulate the event structure, and (ii) the relationship between support-verb constructions and base-verb constructions (i.e. simplex verbs that are formally or functionally related to the predicative noun of the support-verb construction), specifically with regard to the creation of discourse cohesion. Veteikis draws on the rhetorical definition of periphrasis heralded by the grammarian Quintilian (1st c. AD) and the rhetorician Numenius (2nd c. AD) and seeks to embed support-verb constructions into the catch area of this notion. He thus includes non-prototypical support verbs in his dataset, e.g. compound verbs and the verbs of saying and speaking. 

\textbf{Chapter 7 by Vives Cuesta} examines a large corpus of Byzantine hagiography spanning about 1000 years (5th to 14th c. AD). His interest lies with the support verb \textit{par excellence} ποιέω/ποιέομαι \textit{poieō/poieomai}. He finds that with an event noun referring to motion and/or movement (e.g. πορείαν/ἔκβασιν ποιέω \textit{poreian/ekbasin poieō} ‘to talk / escape’), the formally morpho-syntactic contrast between the active and middle voices of the verb was gradually replaced by a pragmatic contrast (similarly to what \citealt{benteinFiniteVsNonfinite2017} finds for verbal complementation patterns). Form-identical with the support verb is ποιέω \textit{poieō} as a verb of realisation, i.e. ‟indicat[ing] that the purpose for which the action exists has been achieved" (Vives Cuesta [this volume]), in θέλημα/λόγον/κέλευσιν ποιέω \textit{$t^h$elēma/logon/keleusin poieō} ‘to do/complete (somebody's) will/word/command’. These structures noticeably differ from support-verb constructions as the agent encoded by the support verb and that implied by the predicative noun are not co-referential. Finally, Vives Cuesta, in line with Gross’ approach, considers ἅπτομαι \textit{$^h$aptomai} ‘to touch upon’ and ἐμπίπτω \textit{empiptō} ‘to fall into’ aspectual and diathetic variants respectively of ποιέω/ποιέομαι \textit{poieō/poieomai} with the same predicative noun. These are related to commonly drawn upon conceptual metaphors. In the context of the Byzantine hagiographic works, the diachronic development of support-verb constructions must be set against the \textit{metaphrasis} tradition, which is akin to but different from, as Vives Cuesta emphasises, intralingual translation. Variation can index levels of speech.


The \textbf{fourth and final section} of the volume (Between analytic and synthetic: the syntax-lexicon interface) focusses on the support verb \textit{par excellence} ‘to do’. 
The debate on ‘to do’ is already far ranging. Proposals range from in favour to vehemently against grammaticalisation \citep{andersonAuxiliaryVerbConstructions2006, sladeDiachronyLightAuxiliary2013, ittzesLightVerbAuxiliary2022, croftMorphosyntaxConstructionsWorld2022} and from ‘to do’ becoming a derivational suffix to it retaining its lexical status \citep{buttLightVerbJungle2010, buttDiachronicPertinacityLight2013}.\footnote{Note that do-support as in English is a key driving force for the debate (see \citealt{ellegardAuxiliaryEstablishmentRegulation1953} on English, recently \citealt{Swinburne2024} on the Camuno dialect of Italian).} If we reject a lexical-grammatical continuum \citep{boyeGrammaticalizationConventionalizationDiscursively2023}, support-verb constructions are either lexemic or syntactic phrasemes\is{interface} \citep{melcukGeneralPhraseology2023}. Yet how do the fully developed systems of compounding \citep{tribulatoAncientGreekVerbinitial2015}, noun incorporation \citep{asrafMechanismNounIncoporation2021, pompeiTracceDiIncorporazione2006}, and enclisis \citep{SolticJanse2012} fit in? This is where the contributions of this volume pick up.

\textbf{Chapter 8 by Ryan} examines the exegetical implications of using the synthetic simplex verb ἁρμαρτάνω \textit{$^h$amartanō} ‘to sin’ as opposed to the analytic support-verb construction ἁμαρτίαν ποιέω \textit{$^h$amartian poieō} ‘to commit (a) sin’ in the New Testament corpus. In passing, derivatives such as the result nouns in -μα \textit{-ma}, event nouns in -σι- \textit{-si-}, and agent nouns in -της \textit{-tēs} built from the stem ἁρμαρτ- \textit{$^h$amart-} and the significance of their presence/absence in the New Testament corpus are considered. Ryan argues that the locus of agentivity shifts in the support-verb construction from the sinner (i.e. the subject of the simplex verb) to the sin (i.e. the semantic head of the support-verb construction). Sin may subsequently even be interpreted as separate or at least more distant from the sinner than when the process is expressed by means of a synthetic simplex verb. Crucially, the support-verb and base-verb constructions are neither semantically identical for Ryan as outlined nor pragmatically, in that the choice of the support-verb construction over the simplex verb is interpreted along the lines of a technical term motivated by the ethical framework into which the discourse is embedded. For Ryan, the support-verb construction is analytic. 

\textbf{Chapter 9 by Pompei, Pompeo, and Ricci} examines the difference between analytic and synthetic combinations with ποιέω/ποιέομαι \textit{poieō}/\textit{poieomai} ‘to do’. Crucially, their interest lies with pairs such as πολεμοποιέω \textit{polemopoieō} vs. πόλεμον ποιέω \textit{polemon poieō} rather than pairs like \textit{to make a decision} vs. \textit{to decide} in English (as Veteikis [Chapter 6] does). The authors consider what the reasons are behind the selection of an analytic as opposed to a synthetic construction and find that in addition to semantic differences, reasons of textual coherence and cohesion play a role (e.g. reference tracking). Furthermore, they distinguish between constructions that are built from event nouns (e.g. πόλεμος \textit{polemos} ‘war, battle’), nouns that have an eventive meaning in their lexical structure (e.g. ἄριστον \textit{ariston} ‘(morning) meal, breakfast, lunch’), and those nouns that are non-eventive (e.g. σῖτος \textit{sitos} ‘grain, food, allowance of grain’). Only the analytic constructions that contain a noun with an eventive meaning qualify as support-verb constructions, whereas those with a non-eventive noun and the verb meaning ‘to achieve, create’ do not qualify as support-verb constructions (compare by contrast Vives Cuesta [Chapter 7] and Baños and Jiménez López [Chapter 4]). Synthetic instances of noun incorporation (i.e. combinations with a non-eventive noun) appear with a disproportionate frequency in Plato’s writings, such that they may constitute an idiosyncrasy for personal, genre-, or register-related reasons.


\textbf{Chapter 10 by Miyagawa} examines Greek’s long-term contact language Coptic with a specific focus on texts dating from the 4th to 8th centuries. Greek and Coptic had existed for more than a millennium already by the fourth century AD and language-contact phenomena appear in the form of Coptic interference in Greek \citep{fendelCopticInterferenceSyntax2022} but also in the form of Greek interference in Coptic \citep{grossmanLanguageSpecificTransitivitiesContact2019}. One area that has received considerable debate is support-verb constructions when used to integrate Greek loan verbs into the predicate slot of the sentence \citep{reintgesCodemixingStrategiesCoptic2001, egediRemarksLoanVerb2017, funkDifferentialLoanCoptic2017, grossmanDialectalVariationLanguage2017, grossmanLanguageSpecificTransitivitiesContact2019, grossmanTransitiveVerbsLexical2023}. The crucial question relates to the status of the support verb, often the verb \coptic{ⲉⲓⲣⲉ} \textit{eire} ‘to do’, in such constructions – is it a derivational affix, an inflexional clitic, a (semi-)lexical verb, or something entirely different? Miyagawa discusses in detail the so-called prenominal state of the verb in the context of clitics, word segmentation, and (pseudo\nobreakdash-)noun incorporation. The support verb appears in this prenominal state, i.e. unstressed and often with a reduced vowel, when combined with a predicative noun, thus raising questions of cliticization or affixation (see also \citealt{grossmanTransitiveVerbsLexical2023}). However, this construction is not limited to support-verb constructions, but often considered in the context of (pseudo\nobreakdash-)noun incorporation of objects in Coptic. Miyagawa embeds the assessment of the status of the support verb (in the prenominal state) into a discussion of the degree of analyticity of the Coptic language from a typological perspective. The chapter thus offers a typological embedding for noun incoporation in Greek (see Chapter 9) and a critical assessment of the status of the support verb as lexical, grammatical, or both.   







\section{Practicalities}
The reader will observe that all the chapters of this volume are prefixed with an abstract in English and one in a pragmatically preferred/dominant language as defined by the author of each chapter \citep[23]{matrasLanguageContact2009}. In the past, research traditions on support-verb constructions have developed in language-specific settings and have been entrenched in the research landscape subsequently (see Section 1). We want to break with this and thus attempt to overcome language boundaries in a small way by providing multilingual abstracts.\footnote{Chapter 1 German, Chapter 2 Modern Greek, Chapter 3 Spanish, Chapter 4 Spanish, Chapter 5 Italian, Chapter 6 Lithuanian, Chapter 7 Spanish, Chapter 8 Spanish, Chapter 9 Italian, and Chapter 10 Japanese.} This \textit{proemium} began with an abstract in Russian, a morphology-rich language which formed the basis for Mel’čuk’s recent lexicographic treatment of support-verb constructions \citep{melcukGeneralPhraseology2023}. The \textit{epilogue} of this volume features an abstract in German, another morphology-rich language which forms the basis for the large \textit{Funktionsverbgefüge} ‘function-verb-construction’ research tradition. 

The reader will furthermore observe that transcription conventions in the present volume are corpus-specific. As no two chapters work on the same corpus, transcription conventions differ between chapters but are selected in order to be corpus appropriate, e.g. we do not want to transcribe modern Greek as if it were classical Attic. Throughout, the \textit{Leipzig Glossing Rules} are observed. Relevant abbreviations used are listed at the end of this \textit{Proemium}. The chapters only list chapter-specific abbreviations for simplicity. 


Synthesising the chapters of this volume and ensuring that they are comprehensible to a very interdisciplinary audience often felt like squaring a circle. We have attempted throughout to provide definitions of terms that are (sub\nobreakdash-)discipline-specific, such as laryngeals and Occam’s razor (Chapter 1 by Ittzés) to comparative philology, the F-score and Cohen’s kappa (Chapter 2 by Giouli) to natural language processing, metaphrasis and diglossia (Chapter 7 by Vives Cuesta) to Byzantine studies, and the prenominal state of the verb (Chapter 10 by Miyagawa) to Coptology. 


Furthermore, there are terms that adopt different meanings in different (sub\nobreakdash-)disciplines and we have endeavoured to define the relevant meaning when these terms are used. A prominent example is ‟periphrasis" (see e.g. \citealt{ledgewayPeriphrasisInflexionDiachrony2022, haspelmathPeriphrasis2000, aertsPeriphrasticaInvestigationUse1965}) (esp. Chapter 6 by Veteikis) and ‟verb of realisation" \citep{melcukVerbesSupportsSans2004, melcukGeneralPhraseology2023} (esp. Chapter 4 by Baños and Jiménez López and Chapter 7 by Vives Cuesta). The reader is made aware of this situation here in order to avoid confusion. 


Finally, the reader will observe that several chapters reflect an interest in the role of support-verb constructions in language-contact settings (e.g. Giouli’s code-mixing examples, Vives Cuesta’s intralingual translation, Baños and Jiménez López’ calques, and Madrigal Acero’s loans). This is an area that would deserve considerably more in-depth work but given the focus on the corpora of Greek in this volume, we only note this aspect in passing. 


\section{Thanks-giving}
The project from which this volume arose (\textit{Giving gifts and doing favours: Unlocking Greek support-verb constructions}, University of Oxford, 2020-–2024) has been kindly funded by the \textit{Leverhulme Trust}. In this context, the editor would like to acknowledge not only the overall funding but also the funding received for a fantastic Research Assistant, Wyn Shaw, who majorly aided the authors’ (and editor’s) typesetting of the volume.


In addition, there is a long list of people who supported and helped this volume come into existence. Matthew T. Ireland (Cambridge) headed up the computational magic and quietly made the impossible possible, Alexandre Loktionov (Cambridge) lent his language skills so as to diversify the range of languages in the abstracts, Agata Savary (Paris) as the invited speaker at the (September) conference aided all of us with her insightful discussion prompts, Philomen Probert (Oxford) mentored the editor over the last four years, and Michele Bianconi (Oxford) lent a helping hand in the various editorial storms. Many colleagues let the editor read pre-print copies of their work in the run-up to the conference and this edited volume, in particular Andreas Willi (Oxford), Klaas Bentein (Ghent), M. Dolores Jiménez López (Madrid), and José Miguel Baños (Madrid). Gregory Hutchinson (Oxford) \citep{hutchinsonRepetitionRangeAttention2017} and Jeffrey Rusten (New York) \citep{rustenTenEkvolenTou2020} pointed the editor to their work. Last but certainly not least, we wish to thank the twenty colleagues who lent their academic expertise as reviewers of the chapters to this volume.


As ‟a preface is supposed to be no more than a polite greeting addressed to the reader and, therefore, to carry no, or very little, relevant information" \citep[1]{melcukGeneralPhraseology2023}, this is the point where this \textit{proemium} should hand over to the contributors calling for inspiration and insight about debate and controversy, as Homer’s \textit{proemium} to his \textit{Iliad} (pre 7th c. BC, AG) μῆνιν ἄειδε θεὰ \textit{mēnin aeide $t^{h}$ea} ‘of the anger, sing, goddess’.


\section*{Abbreviations}
Leipzig Glossing Rules\is{Leipzig Glossing Rules}: \url{https://www.eva.mpg.de/lingua/pdf/Glossing-Rules.pdf} (only abbreviations used in this volume are listed and volume-specific abbreviations are marked with *.) 
\bigskip

\begin{table}
    \centering
\begin{tabularx}{.5\textwidth}{>{\hsize=0.5\hsize}X>{\hsize=1.5\hsize}X}
1 & first person \\
2 & second person \\
3 & third person \\
\textsc{abl} & ablative case \\
\textsc{acc} & accusative case \\
\textsc{adj} & adjective \\
\textsc{adv} & adverb(ial)  \\
*\textsc{aor} & aorist tense \\
\textsc{art} & article \\
\textsc{aux} & auxiliary \\
\textsc{caus} & causative \\
\textsc{comp} & complementizer \\ 
\textsc{cop} & copula \\
\textsc{dat} & dative case \\
\textsc{def} & definite \\
\textsc{dem} & demonstrative \\
*\textsc{dom} & differential object marker\\
\textsc{f} & feminine \\
\textsc{fut} & future \\
\textsc{gen} & genitive \\
\textsc{imp} & imperative \\
*\textsc{impers} & impersonal construction \\
*\textsc{impf} & imperfect tense \\
\textsc{ind} & indicative mood \\
\end{tabularx}%
\begin{tabularx}{.5\textwidth}{>{\hsize=0.5\hsize}X>{\hsize=1.5\hsize}X}
\textsc{indf} & indefinite \\
\textsc{inf} & infinitive mood \\
\textsc{m} & masculine \\
*\textsc{mid} & middle voice \\
\textsc{n} & neuter \\
\textsc{neg} & negation/negative \\
\textsc{nom} & nominative case \\
\textsc{obj} & object \\
*\textsc{opt} & optative mood \\
\textsc{pass} & passive voice \\
\textsc{pl} & plural \\
*\textsc{plp} & pluperfect tense \\
\textsc{poss} & possessive \\
\textsc{prf} & perfect tense \\
\textsc{prs} & present tense \\
*\textsc{prt} & particle (e.g. μέν \textit{men}) \\
\textsc{ptcp} & participle mood \\
\textsc{q} & question \\
\textsc{refl} & reflexive \\
\textsc{rel} & relative \\
\textsc{sbj} & subject \\
\textsc{sbjv} & subjunctive mood \\
\textsc{sg} & singular \\
\textsc{voc} & vocative case \\
\end{tabularx}%
\end{table}

%\begin{tabularx}{.5\textwidth}{@{}lQ@{}}
%... & \\
%... & \\
%\end{tabularx}

\section*{Acknowledgements}
%\section*{Contributions}
%John Doe contributed to conceptualization, methodology, and validation. 
%Jane Doe contributed to writing of the original draft, review, and editing.
The project from which this volume arises \textit{Giving gifts and doing favours: Unlocking Greek support-verb constructions} (University of Oxford, 2020-–2024) was funded by the \textit{Leverhulme Trust} (grant n. ECF-2020-181).



{\sloppy\printbibliography[heading=subbibliography,notkeyword=this]}
\end{document}

