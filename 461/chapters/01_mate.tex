\documentclass[output=paper,colorlinks,citecolor=brown]{langscibook}
\ChapterDOI{10.5281/zenodo.14017921}
\author[Máté Ittzés]{Máté Ittzés\affiliation{Eötvös Loránd University, Budapest }}
\title{Proto-Indo-European support verbs and support-verb constructions}

\abstract{This chapter argues that even if typological considerations make it very likely that the category of support-verb constructions did in fact exist in Proto-Indo-European and the support-verb use of roots such as *\textit{dʰeh\textsubscript{1}} ‘to put’ or *\textit{deh\textsubscript{3}} ‘to give’ may be assumed for the parent language with a sufficient degree of certainty, the reconstruction\is{reconstruction} of specific support-verb constructions will probably never be entirely successful. Apart from the almost complete lack of comparable constructions built of cognate elements\is{cognate} in the individual daughter languages it also runs counter to various theoretical and methodological principles of comparative historical linguistics.

\bigskip

In diesem Beitrag soll argumentiert werden, dass, auch wenn typologische Überlegungen es sehr wahrscheinlich machen, dass die Kategorie der Funktionsverbgefüge im Urindogermanischen tatsächlich existierte, und die Funktionsverbverwendung von Wurzeln wie *\textit{dʰeh\textsubscript{1}} ‘setzen’ oder *\textit{deh\textsubscript{3}} ‘geben’ für die Grundsprache mit hinreichender Sicherheit angenommen werden kann, die Rekonstruktion bestimmter Funktionsverbgefüge wahrscheinlich niemals völlig erfolgreich sein wird. Abgesehen von dem fast vollständigen Fehlen vergleichbarer und aus kognaten Elementen gebildeter Konstruktionen der indogermanischen Einzelsprachen läuft sie auch verschiedenen theoretischen und methodischen Prinzipien der vergleichenden historischen Sprachwissenschaft zuwider.}

%Hyphenation is not correct for German! Neither for other languages (also in the bibliography)!
%%Incorrect hyphenations can be resolved by adding the correct hyphenations to the local hyphenation file.
%Not only the hyphenation of words is problematic in the case of other languages, but sometimes the characters such as " ' - etc. cause problems at the end of the lines. See, e.g., Kloekhorst & Lubotsky 2014 in the bibliography: ... nī-|, ...
%%Issues such as the Kloekhorst & Lubotsky one can be resolved by putting the relevant text in a \mbox{} (I've done this in that case and am now looking for other cases).
%I'm waiting with the hyphenation issues until Vicoria finishes the copy-editing, which, as she wrote, may change a couple of things.



\IfFileExists{../localcommands.tex}{
   \addbibresource{../localbibliography.bib}
   \usepackage{langsci-optional}
\usepackage{langsci-gb4e}
\usepackage{langsci-lgr}

\usepackage{listings}
\lstset{basicstyle=\ttfamily,tabsize=2,breaklines=true}

%added by author
% \usepackage{tipa}
\usepackage{multirow}
\graphicspath{{figures/}}
\usepackage{langsci-branding}

   
\newcommand{\sent}{\enumsentence}
\newcommand{\sents}{\eenumsentence}
\let\citeasnoun\citet

\renewcommand{\lsCoverTitleFont}[1]{\sffamily\addfontfeatures{Scale=MatchUppercase}\fontsize{44pt}{16mm}\selectfont #1}
  
   %% hyphenation points for line breaks
%% Normally, automatic hyphenation in LaTeX is very good
%% If a word is mis-hyphenated, add it to this file
%%
%% add information to TeX file before \begin{document} with:
%% %% hyphenation points for line breaks
%% Normally, automatic hyphenation in LaTeX is very good
%% If a word is mis-hyphenated, add it to this file
%%
%% add information to TeX file before \begin{document} with:
%% %% hyphenation points for line breaks
%% Normally, automatic hyphenation in LaTeX is very good
%% If a word is mis-hyphenated, add it to this file
%%
%% add information to TeX file before \begin{document} with:
%% \include{localhyphenation}
\hyphenation{
affri-ca-te
affri-ca-tes
an-no-tated
com-ple-ments
com-po-si-tio-na-li-ty
non-com-po-si-tio-na-li-ty
Gon-zá-lez
out-side
Ri-chárd
se-man-tics
STREU-SLE
Tie-de-mann
}
\hyphenation{
affri-ca-te
affri-ca-tes
an-no-tated
com-ple-ments
com-po-si-tio-na-li-ty
non-com-po-si-tio-na-li-ty
Gon-zá-lez
out-side
Ri-chárd
se-man-tics
STREU-SLE
Tie-de-mann
}
\hyphenation{
affri-ca-te
affri-ca-tes
an-no-tated
com-ple-ments
com-po-si-tio-na-li-ty
non-com-po-si-tio-na-li-ty
Gon-zá-lez
out-side
Ri-chárd
se-man-tics
STREU-SLE
Tie-de-mann
}
   \boolfalse{bookcompile}
   \togglepaper[1]%%chapternumber
}{}



\begin{document}
\emergencystretch 3em

\maketitle

\section{Introduction: Proto-Indo-European support verbs and typological considerations}

According to the definition adopted in this chapter, support-verb constructions are Noun + Verb (N+V henceforth) constructions consisting of a so-called nominal host (for the term cf. \citealt[433]{Mohanan1997}), which embodies the lexical meaning of the expression and is the syntactic object argument of the verb, and a semantically reduced or bleached support verb, which conveys the grammatical information and no lexical semantics, filling together the predicate slot of the clause. The category of support-verb constructions itself is not homogeneous (cf. \citealt[21--18]{Kamber2008}; \citealt{Vincze2008} among countless others), but rather to be conceived of as a continuum that ranges from constructions behaving more like free syntagms to those that have more in common with idiomatic expressions.\is{idiomatic expression}

There are many tests in the secondary literature that are used to delimit these three categories. For the sake of simplicity, I will make use of the approach of \citet[288--294]{Vincze2008}, who argues that there are two tests that give grammatical results for support-verb constructions (or “semi-compositional constructions” in her terminology), but not for the other two neighbouring categories: 1. The test of \textit{variativity}: Is it possible to replace the whole construction with a derivationally related simple verb?; 2. The test of the \textit{omission of the verb}: Is it possible to recover the meaning of the construction when the verb is omitted?

Although the applicability of one of these tests alone is sufficient for a multi-word expression to be regarded as a support-verb construction, prototypical or core items, of which the nominal host is a verbal action noun, pass both. Consider as a prototypical example OIA \textit{praveśanaṃ cakre} \iwi{Mahābhārata (MBh)
%I didn't include the abbreviations for sources/original texts in the Abbreviations section. Shall I? Alternatively, shall I make a remark in a footnote that I use the standard abbreviations as it is used also in the relevant secondary literature (e.g. dictionaries) of the given language?
%Victoria https://dsal.uchicago.edu/dictionaries/soas/frontmatter/abbreviations.html seems to have them all 
2.4.1a} ‘entered; lit. made entering’, which is equivalent to the etymologically related simplex-verb form (i.e. \textit{praviveśa}) and the meaning of which could be fully reconstructed if the verb were omitted (i.e. the whole construction is in fact about \textit{praveśana-} ‘entering’).

The category of support-verb constructions seems to be a (near-)universal phenomenon, since it occurs in genetically unrelated languages all over the world. For instance, the studies of \citet{SchultzeBerndt2008,SchultzeBerndt2012}
% My intention was to have this: "The studies of E. Schultze-Berndt (2008; 2012) have shown...", but didn't manage to do it with the help of the LSP guidelines (https://langsci.github.io/guidelines/latexguidelines/LangSci-guidelines.pdf; p. 11).
%%Fixed
%OK, thanks. M.
have shown that so-called generalised action verbs (or `do-verbs') are used as support verbs in a large number of languages (her investigations cover Samoan, Hausa, Kalam, Yimas, Jaminjung, Ewe, Kham, Chantyal, German, English, and Moroccan Arabic), while Vincze’s frequency lists \citep[40--44]{Vincze2011} based on a corpus analysis in English and Hungarian have revealed that the most common support verbs, regardless of genetic affiliation, tend to be cross-linguistically the same verbs with a wide range of meanings. 


Furthermore, recent investigations (\citealt[72--74]{Butt2010}; \citealt[18--23]{ButtLahiri2013}) have emphasised that light verbs\footnote{The relationship between support verbs and light verbs is disputed. Some scholars claim that the two notions are identical (cf., e.g., \citealt{Mel’čuk2022}), while others, including myself, believe that light verbs constitute the larger category which includes support verbs.}
are not diachronically derived from full verbs via historical processes, such as semantic bleaching, but have existed beside form-identical full verbs at all stages and in all periods of human languages, even if their frequency might be subject to change, primarily increase, over time.\footnote{It has to be added, however, the Butt and Lahiri’s claims about light verbs are not universally accepted. See, e.g., the alternative views of \citealt{Hook1993}; \citealt{Slade2013}; \citealt{Hock2014}; \citealt{Ittzés2020/2021}.}

Accordingly, we may assume with a sufficient degree of certainty that support-verb constructions must have existed in Proto-Indo-European (PIE henceforth) as well and verbs with a general meaning, such as *\textit{dʰeh\textsubscript{1}} ‘to put, to set’, *\textit{deh\textsubscript{3}} ‘to give’ or *\textit{h\textsubscript{1}ei̯} ‘to go’ were indeed used as support verbs in the proto-language. Recent studies more or less agree that the PIE support verb \textit{par excellence} was the verb *\textit{dʰeh\textsubscript{1}}.\footnote{See, e.g., \citet[6]{Hackstein2002b}:
‟Es darf zunächst außer Zweifel stehen, daß die uridg. Wurzel *\textit{dʰeh\textsubscript{1}-} bereits grundsprachlich zur Bildung von Funktionsverbgefügen gedient hat."}
This assumption is made indeed plausible by the fact that the reflexes of *\textit{dʰeh\textsubscript{1}} are used as a support verb in several branches of the Indo-European language family (Old-Indo Aryan (OIA henceforth) √\textit{dhā}; Avestan (Av. henceforth) √\textit{dā}; Greek (Gr. henceforth) τίθημι \textit{ti$t^h$ēmi}; Latin (Lat. henceforth) \textit{facio}; Old High German (OHG henceforth) \textit{tuon}; Hittite (Hitt. henceforth) \textit{dai-}; Tocharian B (Toch. B henceforth) \textit{tā-}), although some of the daughter languages have apparently replaced it in this function over the course of time (cf., e.g., ποιέω/ποιέομαι \textit{poieō/poieomai} and √\textit{kṛ} as the most frequent support verbs in Greek and Old Indo-Aryan, respectively).\footnote{Since the most common support verbs of the daughter languages (i.e. Lat. \textit{facere} from PIE *\textit{dʰeh\textsubscript{1}k} (LIV: 139--140), certainly related to *\textit{dʰeh\textsubscript{1}} mentioned above, even if the origin of the *\textit{k} extension is disputed (on which see, e.g., \citealt[148--150]{Harðarson1993}; \citealt{Untermann1993}; \citealt{Korltandt2018}); OIA √\textit{kṛ} from PIE *\textit{k\textsuperscript{u̯}er}/*\textit{(s)k\textsuperscript{u̯}er} ‘to cut, to carve’ or/and *\textit{(s)ker} ‘to crop, to scrape, to scratch’ (LIV: 391--392; 556--557; LIVAdd: s.v. 1. *\textit{(s)ker}; VIA: 168--170; 259); Gr. ποιέω \textit{poieō} from PIE *\textit{k\textsuperscript{u̯}ei̯} ‘to collect, to stack’ (LIV: 378--379); Hitt. \textit{ie/a-} from PIE *\textit{h\textsubscript{x}eh\textsubscript{x}} (?) ‘to make, to do’ (EDHIL: 381--382)), with the exception of the Hittite verb, all have a primary, concrete meaning (on their semantics cf. the lemmata in LIV), it is possible that their use as semantically light support verbs is only a post-PIE development.}
Consider the following examples of support-verb constructions in a number of early attested Indo-European languages, which all involve a general `do'-verb (for the term cf. \citealt{SchultzeBerndt2008}) and an eventive noun, see (\ref{OldLatin}) to (\ref{OldHittite}).

\ea\label{OldLatin}
\settowidth \jamwidth{(Old Latin)}
\gll \textit{ubi}    \textit{\textbf{mentionem}}  \textit{ego}   \textit{\textbf{fecero}}  \textit{de}  \textit{filia}\\
when     mention.\textsc{acc}    \textsc{1sg}    do.\textsc{fut}.\textsc{prf}.\textsc{1sg}   about     daughter.\textsc{abl}\\ \jambox{(Old Latin)}
\glt `when I make mention of his daughter' \\
\hspace*{\fill}(\iwi{Plautus, \textit{Aulularia} 204})
\z
%I use the (optional) "." sign instead of (optional) ":" in the glosses. E.g. "mention.ACC" instead of "mention:ACC". Is this OK? Is this compatible with the other chapters? 


\ea
\settowidth \jamwidth{(Ancient Greek)}
\glll οὐκ	ἐξέχρησέ		σφι 		ἡ 	ἡμέρη 		\textbf{ναυμαχίην}    \textbf{ποιήσασθαι} \\ \textit{ouk}   \textit{exe$k^h$rēse}     \textit{s$p^h$i}    \textit{$^h$ē}    \textit{$^h$ēmerē}    \textit{nauma$k^h$iēn} \textit{poiēsas$t^h$ai} \\
\textsc{neg}     suffice.\textsc{aor}.\textsc{3sg}    they.\textsc{dat}  \textsc{art}  day.\textsc{nom}   see-fight.\textsc{acc} 
   make.\textsc{inf}.\textsc{aor}.\textsc{med}\\ \jambox{(Ancient Greek)}
\glt `There was not enough daylight left for them to fight the naval battle.' \\
\hspace*{\fill}(\iwi{Herodotus, \textit{Histories} 8.70.1})
\z

\ea
\settowidth \jamwidth{(Vedic Old Indo-Aryan)}
\gll \textit{\textbf{śruṣṭíṃ}}		\textit{\textbf{cakrur}}		\textit{bhṛ́gavo}		\textit{druhyávaś}		\textit{ca}\\
obedience.\textsc{acc}   do.\textsc{prf}.\textsc{3pl}   Bhṛgu.\textsc{nom}.\textsc{pl}     Druhyu.\textsc{nom}.\textsc{pl}   and\\ \jambox{(Vedic Old Indo-Aryan)}
\glt `The Bhṛgus and the Druhyus obeyed.' \\ 
\hspace*{\fill}(\iwi{R̥gveda (RV) 7.18.6c})
\z


\ea
\settowidth \jamwidth{(Old Avestan)}
\gll \textit{yōi}		\textit{mōi}		\textit{ahmāi}		\textit{\textbf{səraoṣ̌əm}}   \textit{\textbf{dąn}}			\textit{caiiascā}\\
who.\textsc{nom}.\textsc{pl}   \textsc{1sg}.\textsc{gen}   this.\textsc{dat}     readiness\_to\_listen.\textsc{acc}    give.\textsc{aor}.\textsc{sbjv}.\textsc{3pl}     whoever.\textsc{nom}.\textsc{pl}\\ \jambox{(Old Avestan)}
\glt `whoever are ready to listen to this [word] of mine' \\ 
\hspace*{\fill}(\iwi{Yasna (Y) 45.5c})
\z


\ea\label{OldHittite}
\settowidth \jamwidth{(Old Hittite)}
\gll \textit{takku}	\textit{āppatriwanzi}	\textit{kuišk}[\textit{i}			\textit{p}]\textit{aizzi}		\textit{ta}	\textit{\textbf{šullatar}}  \textit{\textbf{iezzi}}\\
if    seize.\textsc{inf}   someone.\textsc{nom}   go.\textsc{prs}.\textsc{3sg}   and     offense.\textsc{acc}    do.\textsc{prs}.\textsc{3sg}\\ \jambox{(Old Hittite)}
\glt `if someone goes to make a legal seizure and commits offense' \\
\hspace*{\fill}(\iwi{Keilschrifttexte aus Boghazköi (KBo) 6.26 i 28‒29})
\z


\section{Proto-Indo-European support-verb constructions: reconstructs or \textit{Transponat}s?}

In the last decades, there have been efforts to go beyond this general theoretical observation and reconstruct specific support-verb constructions (or `Funktionsverbgefüge') for PIE, a trend which is indicated in the first place by the publication of Marc Schutzeichel’s comprehensive monograph entitled \textit{Indogermanische Funktionsverbgefüge} (\citealt{Schutzeichel2014}) as well as several individual papers and articles.

However, if we have a look at the secondary literature, we can see that PIE support-verb constructions are posited most of the time on the basis of evidence from a single daughter language. To mention just one illustrative example, Olav Hackstein in his famous and often-cited 2002 article (\citealt{Hackstein2002b}) assumes the existence of a PIE support-verb construction *\textit{k\textsuperscript{u̯}oḱi dʰeh\textsubscript{1}} ‘to take into account, to consider; Acht geben’, the nominal host of which (*\textit{k\textsuperscript{u̯}oḱi}) is derived from the PIE root *\textit{k\textsuperscript{u̯}eḱ} ‘to see’ (cf., e.g., OIA √\textit{cakṣ} ‘to shine, to see’; OCS \textit{kažǫ} causative ‘to show, to remind of’; see LIV: 383–385). Nevertheless, his entire argumentation is based on the Tocharian B phrase \textit{keś tā-} ‘to judge, to consider’ alone (\textit{keś} ‘number’), which means that the alleged support-verb construction *\textit{k\textsuperscript{u̯}oḱi dʰeh\textsubscript{1}} is, strictly speaking, not a reconstruct based on comparative evidence, but – to use a term coined by Heiner Eichner – only a \textit{Transponat}\is{Transponat}. \textit{Transponat}s are ‟Formen, die nicht aufgrund von belegten Gleichungen in anderen altindogermanischen Sprachen rekonstruiert werden, sondern die eine einzelsprachliche Form mit den bekannten Lautgesetzen ins Indogermanische zurücktransponieren"\footnote{I.e. forms which are not reconstructed on the basis of documented equations in other Old Indo-European languages, but which transpose a single-language form back into Proto-Indo-European with the help of the known sound laws.} \citep[12]{Krisch1996}.

However, precisely due to the lack of comparative evidence, \textit{Transponat}s\is{Transponat} cannot claim certain PIE status, since it is entirely possible that such forms, be they independent lexemes or multi-word expressions, were created as innovations\is{innovation} only well after the break-up of PIE in the prehistory of the individual languages.

As far as support-verb constructions are concerned, this methodological consideration must be taken into account all the more seriously as languages may, and in fact very much tend to, create constantly new light-verb (including support-verb) constructions based on the analogy with earlier, potentially inherited, constructions or patterns, as emphasised by \citet{Bowern2008} in her important summarising article about the diachrony of complex predicates. This means that if we observe a particular support-verb construction in a single language, the default assumption must be that it was coined in the history of the individual language in question and we may not project it back out of hand to the parent language (PIE, in our case) or, for that matter, to a so-called transitional proto-language (such as, e.g., Proto-Indo-Iranian or Proto-Balto-Slavic).

Furthermore, although the methodological principle of \textit{Occam’s Razor} in linguistic reconstruction\is{reconstruction} may lean towards reducing (all else being equal) the number of independent developments in the daughter languages, the fact that the category of support-verb constructions is notoriously liable to proliferate suggests that even if we happen to have apparently related constructions in more than one daughter language, it cannot be excluded that they are independent innovative creations of the separate languages due to the analogy with other constructions rather than cognates\is{cognate} in the true sense of the word, which were inherited from their common proto-language.

Accordingly, the positing\footnote{As should be clear from what has been said so far, I deliberately avoid using the term ‟reconstruction"\is{reconstruction} in this context.}
of a PIE support-verb construction on the basis of the single Tocharian B phrase \textit{keś tā-} ‘to judge, to consider’ is to be rejected as being methodologically and theoretically unfounded.\footnote{Hackstein’s second Tocharian example, \textit{śāp tā‑} ‘to curse’ is even more evidently a late creation, as shown by its nominal member being a loanword from Old Indo-Aryan (\textit{śāpa‑} ‘curse, oath’; cf. \citealt{Adams2013}: s.v. \textit{śāp}).}      On the other hand, a potentially good example of an entire PIE construction reconstructed on the basis of comparative evidence may be the phrase ‘to give (lit. to place, to put) a name; to name’, which is attested in a relatively large set of Indo-European languages as consisting of etymologically cognate elements\is{cognate} (cf. \citealt[6]{Hackstein2002b}; \citealt[115--117]{Schutzeichel2014}).

\ea
Gr. ὄνομα τίθεσθαι \textit{onoma ti$t^h$es$t^h$ai}

OIA \textit{nā́ma} √\textit{dhā}

Lat. \textit{nomen facere}/\textit{indere}

Toch. B \textit{ñem tā-}

Hitt. \textit{lāman dai-}

SCr. \textit{ȉme djȅsti}


\z


The perfect equation of the above-mentioned constructions as well as their individual parts convincingly speaks in favour of a PIE reconstruction\is{reconstruction} *\textit{h\textsubscript{3}néh\textsubscript{3}mn̥}\footnote{The precise reconstruction of the PIE word for ‘name’ is irrelevant to our question. Beside the most plausible reconstruction mentioned above in the main text (cf. EDHIL: 282--285; EDG II: 1084--1085; \citealt[52--53]{Beek2011}) see also the alternative opinions by \citet{Stüber1997}; \citet[6]{Hackstein2002b} (both with initial *\textit{h\textsubscript{1}}).}   \textit{dʰeh\textsubscript{1}} ‘to give (lit. to place, to put) a name; to name’.\footnote{As one of my anonymous reviewers points out, it is important in the context of Proto-Indo-European textual or syntactic reconstruction to look at the exact nature of the collocations. Namely, if the combination of the members of a phrase is banal or unremarkable and does not have anything peculiarly Indo-European, its reconstruction for the parent language is questionable. If, however, the components of a collocation and their combination are unexpected or idiomatic\is{idiomatic expression}, its tracing back to Proto-Indo-European is more reasonable. On this argument see also \citet[72--76]{Matasović1996} (on Indo-European N-Adj phrases or formulas, in which the adjective is metaphoric and therefore “informative” or banal and thus “uninformative” with respect to the noun), \citet[78--80]{Matasović1996} (on the V-O type, i.e. formulas consisting of a transitive verb and its object); see also \citet[118--124]{Ittzés2017}. Since the combination of ‘name’ + ‘to place, to set’ is not (entirely) trivial, its reconstruction for PIE may indeed seem reasonable.}
However, it must be taken into account that the nominal element of this construction is not an abstract action noun, which means that it is, depending on one’s definition, either no support-verb construction at all or at least not a prototypical representative of the category.

Namely, as summarised by \citet[383]{Fendel2023}, ‟narrow definitions of support-verb constructions only accept deverbal formations in the predicative-noun slot", while ‟wider definitions will include any eventive noun". Under the latter view, even a non-deverbal concrete noun may form a support-verb construction if it is reconceptualised as eventive (cf. \citealt{Radimský2011}) or undergoes metaphorical extension.

Following the latter approach, one might in fact regard the noun *\textit{h\textsubscript{3}néh\textsubscript{3}mn̥} in the phrase *\textit{h\textsubscript{3}néh\textsubscript{3}mn̥ dʰeh\textsubscript{1}} as being reconceptualised as eventive (i.e. referring somehow to the process resulting in the given name) and take the whole phrase as a support-verb construction. However, it seems that neither of the two tests mentioned above yields a positive result when applied to this phrase.


Firstly, scholars who reconstruct an initial laryngeal\footnote{The so-called `laryngeals' (notated as *\textit{h\textsubscript{1}}, *\textit{h\textsubscript{2}}, *\textit{h\textsubscript{3}}) were probably fricatives in PIE phonology, but their exact phonetic reality is disputed (see \citealt[10--13]{Byrd2015} for a brief overview).}
*\textit{h\textsubscript{3}} in the ‘name’ word (cf. above) usually connect it to the PIE root *\textit{h\textsubscript{3}neh\textsubscript{3}}, which is reflected in Gr. ὄνομαι \textit{onomai} ‘to blame, to treat scornfully’ and Hittite \textit{ḫanna-\textsuperscript{i}} / \textit{ḫann}- ‘to sue, to judge’. Even though the original meaning of the PIE root could indeed be ‘to call (by name)’, whence Gr. ‘to call names’ > ‘to treat scornfully’ and Hitt. ‘to call to court > to sue’ (see EDHIL: 284), I do not think that in synchronic PIE the simplex verb *\textit{h\textsubscript{3}neh\textsubscript{3}}, which, as judged from its reflexes in the daughter languages, had already developed a special semantics, was still able to replace the putative support-verb construction *\textit{h\textsubscript{3}néh\textsubscript{3}mn̥ dʰeh\textsubscript{1}} ‘to give a name’. Secondly, in the case of omission of the verb the meaning of the construction is not recoverable either.

\section{Open-slot constructions and lexical substitutions}

A similar case with equally far-reaching methodological implications will be taken from another influential study of Olav \citet[96--101]{Hackstein2012}. Hackstein takes into account three collocations attested in the daughter languages: OHG \textit{wara tuon} ‘to pay attention/heed (to)’; Gr. (ἐπὶ) ἦρα φέρειν \textit{(epi) ēra $p^h$erein} ‘to bring help, to give a favour’ and Hitt. \textit{warri nāi-} ‘to bring as help’. As can be seen, the support verbs\footnote{\citet[96]{Hackstein2012} refers to them as light verbs.}
of the three attested constructions are etymologically unrelated (OHG \textit{tuon} < PIE *\textit{dʰeh\textsubscript{1}} ‘to put’; Gr. φέρειν \textit{$p^h$erein} < PIE *\textit{bʰer} ‘to bring’; Hitt. \textit{nāi-} < PIE *\textit{neh\textsubscript{x}i}\footnote{*\textit{neh\textsubscript{x}i} (actually *\textit{neHi}) is the form reconstructed by Hackstein himself. For other reconstructions\is{reconstruction} cf., e.g., LIV: 450--451 (*\textit{nei̯H}, i.e. *\textit{nei̯h\textsubscript{x}}); \citealt{KloekhorstLubotsky2014} (*\textit{(s)neh\textsubscript{1}}).}
‘to lead’), neither are the nominal hosts exact cognates\is{cognate}, but different nominal derivatives of the same root (OHG \textit{wara} < PIE *\textit{(s)u̯orh\textsubscript{3}-eh\textsubscript{2}}; Gr. ἦρα \textit{ēra} < PIE  *\textit{(s)u̯ērh\textsubscript{3}-}; Hitt. \textit{warri} < PIE *\textit{(s)u̯erh\textsubscript{3}-}; all ultimately from PIE *\textit{(s)u̯erh\textsubscript{3}}\footnote{Hackstein’s reconstruction (originally proposed in \citealt[123--131]{Hackstein2002a}) is not universally accepted. It is not even mentioned by LIVAdd. Note that Gr. ὁράω \textit{$^h$oraō} is derived from a root *\textit{ser} ‘aufpassen auf, beschützen’ by LIV: 534 and from *\textit{u̯er} ‘observe, note’ by EDG II: 1095--1096.}
‘to observe, to be attentive’; cf., e.g., Gr. ὁράω \textit{$^h$oraō} ‘to see’).\footnote{The connection of the Greek and Hittite phrases with the OHG one is not mentioned by \citet{GarcíaRamón2006}.}
What \citet[96]{Hackstein2012} posits for PIE on such evidence is a so-called ‟open slot construction" with the meaning ‘to pay heed to, to pay attention to’, in which the two slots could be filled by some nominal derivative of *\textit{(s)u̯erh\textsubscript{3}} and an optional transitive support verb with a motion-of-the object meaning.


\ea\label{openslot}
*\textit{(s)u̯erh\textsubscript{3}} ‘to perceive, to heed, to be attentive’ \\
\hspace{2cm} ↓ \\
\begin{math}
	\begin{Bmatrix}
		\text{nominal} & & \text{verb with} \\
		\text{derivative} & + & \text{motion of the} \\
		\text{of *\textit{(s)u̯erh\textsubscript{3}}} & & \text{object meaning}
	\end{Bmatrix}	
\end{math}
\z
%The previous figure is not what I want to have, but I couldn't produce it correctly. Please have a look at the manuscript to get an idea what it should look like.
%It would be nice if the curly brackets could be produced. If is needs too much extra struggling, it can remain of course as it is now. However, I think there should be as much space below the arrow as above anyway.
%% Fixed using a mathmode workaround.
%The figure is wonderful, thanks a lot!

Later on, \citet[100--101]{Hackstein2012} analyses          the Hittite            verb \textit{waritē}- (later \textit{weritē-}) ‘to be attentive, full of awe, to be afraid’ as well, which he interprets, following earlier accounts, as containing the reflex of PIE *\textit{dʰeh\textsubscript{1}} preceded by the same noun \textit{warri} being an incorporated object. If this is correct,\footnote{Note, however, the alternative etymology\is{etymology} of the first part of this verb by EDHIL: 1003--1004.}
then Hittite (\textit{wari *dai-} > \textit{waritē-}) also seems to offer evidence for the original use of the light (or support) verb *\textit{dʰeh\textsubscript{1}} in the open-slot construction\is{open-slot construction} in (\ref{openslot}). Nevertheless, I think that the derivational differences, i.e. non-cognateness\is{cognate}, of the nominal hosts of the above-mentioned three phrases and the fact that their support verbs themselves are partly etymologically unrelated point to their being independently created constructions of the daughter languages rather than inherited ones from the proto-language.\footnote{For the assumption of a formal variation of the nominal host cf., as a similar case, \citet[23]{Balles2009}, where the family of Gr. δολιχός \textit{doli$k^h$os}, Lat. \textit{longus}, etc. is traced back to a PIE support-verb construction *\textit{d(o)lh\textsubscript{1}(i/u/o)‑} (sic!) + *\textit{gʰeh\textsubscript{1}} ‘to reach length’. However, a form like *\textit{d(o)lh\textsubscript{1}(i/u/o)‑} is, in my view, not a meaningful PIE reconstruction.\is{reconstruction}}



Syntactic reconstruction\is{reconstruction} as such may aim at reconstructing either abstract syntactic configurations and rules of the proto-language (such as constituent order of various clause-types, agreement relations within the noun phrase, etc.) or individual syntactic units consisting of more than one word, i.e. phrases, in their material reality. Since the existence of support-verb constructions in human languages, as mentioned above, is probably a linguistic (near-)universal, statements about the mere existence of PIE support-verb constructions which can be described only in terms of their semantics without formal specification would not add much new to our knowledge about PIE as a natural human language. In my view, it is only the latter understanding of syntactic reconstruction which could in principle be meaningful in the case of support-verb constructions. Therefore, the fact that the formal aspects of the PIE construction ‘to pay heed to’ hypothesised by Hackstein must necessarily remain unspecified (‟open"), or at least underspecified, due to the absence of exactly cognate\is{cognate} nominal elements and support verbs makes its ‟reconstruction" for PIE, in my view, unfounded.

Instead of positing a formally un(der)specified construction for PIE (such as ‟nominal derivative of *\textit{(s)u̯erh\textsubscript{3}} + verb with motion-of-the-object meaning") one might also assume that one of the attested nominal derivatives and one of the attested support verbs are indeed the reflexes of the original constituents of the PIE support-verb construction\footnote{It remains, of course, to be seen which ones these were. As regards the support verb, many scholars would agree that it was *\textit{dʰeh\textsubscript{1}}.}  and the languages that do not have them underwent a process of innovation\is{innovation} usually called ‟lexical substitution"\is{lexical substitution} or ‟lexical renewal"\is{lexical renewal}\footnote{As far as the nominal host is concerned, in our case this would not mean the substitution by an etymologically unrelated lexeme, but only by a different derivative of the same root.} in their prehistory. At first sight, this assumption seems to be well-founded if we bear in mind that lexical substitutions in general  happen and are well attested in the history of various Indo-European languages and, which is more, it can be textually demonstrated in the case of the above-mentioned High German construction itself. Namely, as Hackstein describes in detail, the Old High German phrase \textit{wara tuon} got gradually replaced by the phrase \textit{wara niman} by the time of Middle High German (whence New High German (NHG henceforth) \textit{wahrnehmen}).

In my opinion, however, we can base our argumentation on the idea of lexical substitution\is{lexical substitution} neither in this particular example nor in any other case whenever we have to \textit{reconstruct} something for earlier, unattested linguistic stages and not merely \textit{describe} and analyse historically attested developments. It cannot be stressed enough that linguistic reconstruction\is{reconstruction} should always be based on cognates\is{cognate} which are actually attested in the daughter languages. While this \textit{caveat} is taken into account as a matter of fact in phonological, morphological, or lexical reconstruction,\footnote{Note as an example that there is no entry \textit{equus} in the etymological dictionary of the Romance languages (REW), even if it was the common word for ‘horse’ in Classical Latin, precisely because the ‘horse’ words of the Romance languages continue the Proto-Romance (Vulgar Latin) word \textit{caballus} (> It. \textit{cavallo}, Fr. \textit{cheval}, etc.) and provide no evidence whatsoever for the reconstruction of \textit{equus}. For similar reasons, the REW does not have an entry \textit{loquor} ‘to speak’ either, even if it was an extremely frequent verb in Classical Latin (cf. \citealt[11--12]{Herman2003}; \citealt[32--33]{Adamik2009}).}
it is often forgotten or deliberately ignored when it comes to syntactic reconstruction in the sense of ‟material" reconstruction\is{reconstruction} of syntactic units larger than single words. I consider it crucial that we should avoid referring to the notion of lexical substitution\is{lexical substitution} in making our reconstructions, since even though lexical substitution as such is a diachronic reality from the perspective of language change (i.e. when tracking attested historical processes ‟forwards"; cf. documented examples such as OHG \textit{wara tuon} above), its application when performing comparative reconstruction (i.e. when thinking ‟backwards") is not falsifiable and therefore to be avoided on methodological grounds.\footnote{My anonymous reviewer refers, in a similar vein, to the case of Gr. δωτῆρες ἐάων \textit{dōtēres eaōn} (\iwi{Homer, \textit{Odyssey} 8.325}; \iwi{Hesiod, \textit{Theogony} 46+}) vs. Ved. \textit{dātā́ vásūnām} (R̥gveda (RV)), built of cognate elements\is{cognate} and both meaning ‘givers of good’, and their later transformations or modernisations in Gr. πλουτοδόται \textit{ploutodotai} (\iwi{Hesiod, \textit{Works and Days} 126+}) and Skt. \textit{dātā … (a)rthasya} (\iwi{\textit{Mudrārākṣasa (Mudr.)} 5.19}) and points out that we would probably be unable to identify the latter ‟as, in some sense, the same expressions", were it not for the earlier, i.e. Vedic and Homeric/Hesiodic, forms. While I partly agree with this conclusion, I have to add that I am not convinced that the Vedic and Homeric/Hesiodic phrases must necessarily be regarded as the reflexes of a single Proto-Indo-European formula, since I can see nothing really idiomatic\is{idiomatic expression}, unexpected, or specifically Indo-European in a construction like ‘giver of good’ that would prevent us from considering them as later independent creations (cf. n. 9 above).}

It will have become clear by now that I firmly disagree with those who think that the method of “reconstructing” without having cognates\is{cognate} and not just etymologically loosely related elements can be applied in the case of PIE support-verb constructions. Furthermore, I think that it cannot be applied to entirely non-compositional multi-word expressions, i.e. idioms or phraseological units, either. I do not accept the opinion of \citet[79]{West2007}, who believes that ‟in looking for Indo-European idioms […] it is not necessary to limit ourselves to comparisons where all the terms stand in [an] etymological relationship. It is legitimate to adduce expressions that are semantically parallel, even if the vocabulary diverges, provided that they are distinctive enough to suggest a common origin".\footnote{For instance, Calvert Watkins, in his famous monograph on Indo-European poetics \citep[210--213]{Watkins1995}, referring to the notion of lexical substitution\is{lexical substitution}, goes so far as to posit a PIE formula *\textit{pah\textsubscript{2}- u̯ih\textsubscript{x}ro- peḱu-} \textsc{protect men} (and) \textsc{cattle}, even if literally none of the collocations collected by him from the daughter languages, contains the reflex of the root *\textit{pah\textsubscript{2}}- (i.e. *\textit{peh\textsubscript{2}} or *\textit{peh\textsubscript{2}(i̯)}; LIV: 460) and most of them involve different nouns as well.}          In my view, this approach cannot be applied to phraseological units either, and it works still less in the case of support-verb constructions, in which we do not even have the factor of sufficient distinctiveness.

\section{The univerbation hypothesis}

It is a matter of fact that incontestable examples of cognate\is{cognate} support-verb constructions are virtually lacking in the daughter languages. However, there is another relatively popular method in the secondary literature of tracking down PIE support-verb constructions, i.e. by assuming univerbation\is{univerbation}.

It is well known that several roots which can be reconstructed either for Proto-Indo-European itself or for some transitional proto-language show some phonetic addition in comparison to other synonymous roots. In Indo-European linguistics (cf., e.g., \citealt[100--101]{Szemerényi1996}), this apparently meaningless addition is called root extension\is{root extension} or root enlargement\is{root enlargement} (German ‟Wurzelerweiterung"). While root extensions\is{root extension} as such can be more or less clearly reconstructed from the formal point of view, it is difficult to determine what their specific function may originally have been before being obscured by the time of reconstructed Proto-Indo-European.\footnote{Recently, there have been attempts to clarify this problem. For instance, an entire workshop at the 15th ‟Fachtagung" of the \textit{Indogermanische Gesellschaft} (Vienna, September 2016) was dedicated to this topic.}                     Consider, for instance, the following two pairs of roots (on which see LIV: 179–180; 676–677; \citealt[14--15]{Hackstein2002b}; \citealt[38]{Balles2006}) in (\ref{pour}) and (\ref{strong}):

%I couldn't produce what I wanted. Please have a look at the manuscript to get an idea.
%%Fixed with tabbing environment
%Maybe Victoria can decide whether these should be as they are now, or according to the original alignment of the manuscript. For me, both seem to be acceptable.
\ea\label{pour}
\begin{tabbing}
*\textit{ǵʰeu̯} ‘to pour’ >    \= OIA √\textit{hu}, pres. \textit{juhóti} ‘to pour, to offer’ \\
%The following two lines should begin right below "OIA".
\> Gr. χέω \textit{$k^h$eō} ‘to pour’ \\
\> Toch. A, B \textit{ku-} ‘to pour’
\end{tabbing}

%footnote does not display VICTORIA
\begin{tabbing}
*\textit{ǵʰeu̯d}\footnote{The distribution of the data (i.e. the reflexes of the second root are attested only in the Italic and Germanic branches) indicates that the enlarged variant is in all probability an innovation\is{innovation} within the so-called North-West Indo-European block. In any case, it can be dated to a phase prior to the rise of the individual daughter languages.}          >	\= Lat. \textit{fundo} ‘to pour’ \\
%The following two lines should begin right below "Lat." Please have a look at the manuscript to have an idea.
\> Umbr. \textit{hondu} imperative ‘let him pour’ \\
\> Goth. \textit{giutan} ‘to pour’ \\
\> NHG \textit{giessen} ‘to pour’
\end{tabbing}
\z


\ea\label{strong}
\begin{tabbing}
*\textit{u̯elh\textsubscript{2}} ‘to be strong, powerful’ >	\= Lat. \textit{valeo} ‘to be strong, to be able’ \\
%The following two lines should begin right below "Lat.".
\> Toch. B \textit{walo} ‘king’ \\
\> OIr. \textit{follnadar} ‘to rule’ 
\end{tabbing}

\begin{tabbing}    
*\textit{u̯eldʰ} >	\= Lith. \textit{véldu} ‘to possess, to govern’ \\
%The following two lines should begin right below "Lith." Please have a look at the manuscript to have an idea.
\> Goth. \textit{waldan} ‘to rule’ \\
\> OCS \textit{vladǫ} ‘to rule’
\end{tabbing}
 
\z

The reason which makes this phenomenon relevant to our topic is that one of the most frequent root extensions\is{root extension}, *\textit{-dʰ-} (see (\ref{strong}))\footnote{See also *\textit{u̯erh\textsubscript{1}} ‘to say’ > Gr. fut. ἐρέω \textit{ereō}, perf. εἴρηκα \textit{eirēka} ‘to say’; Pal. \textit{wer-} ‘to say, to call’; Hitt. \textit{wer(iye)-} ‘to call, to name’ vs. *\textit{u̯erdʰ} in the nominal derivatives Lat. \textit{verbum} ‘word’; Goth. \textit{waurda} ‘word’; Lith. \textit{var̃das} ‘word’ (cf. LIV: 689‒690).} is now widely held to be the univerbated and grammaticalised\is{grammaticalisation} form of the originally independent light or support verb *\textit{dʰeh\textsubscript{1}}. For several scholars, this means that if we can reconstruct a root with the extension *\textit{-dʰ-} for PIE, it proves the former existence of a support-verb construction built with *\textit{dʰeh\textsubscript{1}} in an earlier phase of the proto-language. For instance, an enlarged root *\textit{u̯eldʰ} < *\textit{u̯elh\textsubscript{2}-dʰ}\footnote{As one of my reviewers points out, the reconstruction\is{reconstruction} of an earlier laryngeal in this form seems to be plausible after all on the basis of the Lithuanian acute intonation (a possibility mentioned but finally rejected by \citealt[472--473]{Kümmel2000}). Note, however, that the loss of the laryngeal here and in similar environments is not a trivial assumption for the PIE period (for a succinct overview of the PIE phonological rules targeting laryngeals cf. \citealt[25--27]{Byrd2015}). Since the so-called \textit{Lex Schmidt-Hackstein} probably operated in the environment *\textit{PH.CC} (cf. \citealt[134]{Byrd2015}) and not generally *\textit{CH.CC} as proposed by \citet{Hackstein2002b} himself (P = plosive/stop, H = laryngeal, C = consonant, and . = syllable boundary), the hypothesis that in the example mentioned above the laryngeal was lost already at the *\textit{u̯elh\textsubscript{2}-dʰh\textsubscript{1}} stage is questionable too. Thus, we would have to suppose that its loss was conditioned by the special circumstances of grammaticalisation\is{grammaticalisation} (cf. below).}
(root *\textit{u̯elh\textsubscript{2}} + root extension\is{root extension} *\textit{dʰ})\footnote{As my anonymous reviewer emphasises, there are some indications (ON preterite \textit{olla} without a reflex of the dental aspirate) that *\textit{-dʰ-} in this particular case has to be conceived of as a present formation rather than a root extension (cf. also LIV: 676) and similar considerations may apply to other instances of this formant across the Indo-European languages. The Indo-European dental-aspirate presents have recently been studied in detail by Z. Rothstein-Dowden, who mentions a number of difficulties related to the univerbation hypothesis\is{univerbation}, without entirely rejecting ‟a historical connection between the verbal formant *\textit{-dʰ-} and the root *\textit{dʰeh\textsubscript{1}} ‘put’" \citep[3--4 with n. 3]{RothsteinDowden2022}. I thank my reviewer for having brought Rothstein-Dowden’s dissertation to my attention.}                    could be analysed as resulting from the univerbation\is{univerbation} of an alleged support-verb construction *\textit{u̯elh\textsubscript{2}} (in this construction it would most probably be a root action noun) + support verb *\textit{dʰeh\textsubscript{1}} ‘lit. to do ruling’ (via the intermediate stage *\textit{u̯elh\textsubscript{2}-dʰh\textsubscript{1}}).

In most cases, the available data do not allow to decide with certainty, whether the alleged process of univerbation\is{univerbation} had taken place already in the proto-language or only later, independently, in the prehistory of the individual languages concerned. Nevertheless, the univerbation hypothesis implies that in spite of the problems mentioned above it is still possible to reconstruct support-verb constructions for (Pre-)Proto-Indo-European, at least by means of internal reconstruction.\is{reconstruction}

There are two fundamental questions concerning this hypothesis: firstly, whether the supposed process is theoretically possible and, secondly, whether it can be proven by empirical data. 

The answer for the first question is certainly a positive one, since the univerbation\is{univerbation} of support verbs (and light verbs in general) is a cross-linguistically well-attested phenomenon \citep[175--176]{Bowern2008}. A classic example is the emergence of the so-called German weak or dental preterite (cf., e.g., Goth. \textit{salbō-da} ‘anointed’; Eng. \textit{work-ed}; Germ. \textit{mach-te}), which probably originated in a support-verb construction with *\textit{dʰeh\textsubscript{1}} (\citealt{Hill2010}; \citealt[69--72]{Schutzeichel2014}).

It is also a matter of fact that the process of univerbation\is{univerbation}, similarly to other types of grammaticalisation\is{grammaticalisation}, is frequently accompanied by irregular sound changes and phonological reductions (often called ‟erosion") which are not observed under ‟normal" conditions. This fact might in principle account for the loss of the root-final laryngeals before the univerbated support verb even at a stage when the latter had already lost its final laryngeal (e.g. *\textit{u̯eldʰ} < *\textit{u̯elh\textsubscript{2}-dʰ}).

It is also worth mentioning in this context that there is a cross-linguistic generalisation that light verbs (including support verbs) are rather stable and more resistant to diachronic changes than auxiliaries. However, this is not meant to claim that light verbs are completely inert in this respect. For instance, there is an ongoing debate whether light verbs can grammaticalise\is{grammaticalisation} to become auxiliaries. Although some scholars (most notably \citealt{Butt2010} and \citealt{ButtLahiri2013}; cf. \citealt[174]{Bowern2008}) have argued that light verbs are never reanalysed as auxiliaries, I have demonstrated \citep{Ittzés2020/2021} that the history of the periphrastic perfect in Vedic Old Indo-Aryan is a typical example of precisely this kind of grammaticalisation\is{grammaticalisation} process (the supposed counterarguments presented by \citealt{ButtLahiri2023} do not seem valid to me).

As far as the second question, the empirical provability of the univerbation\is{univerbation} of a support verb is concerned, there seems to be at least one well-documented case which testifies to the univerbation of the root *\textit{dʰeh\textsubscript{1}} with a nominal element. I am referring to the famous PIE collocation *\textit{ḱréd} (or rather \textit{ḱréds}) \textit{dʰeh\textsubscript{1}} ‘to believe, to trust; lit. to place one’s heart\footnote{It is beyond doubt that the nominal member of the construction was originally some case form of the PIE word for ‘heart’: *\textit{ḱerd-/ḱr̥d-} (> HLuw. \textit{zārt-}; Lat. \textit{cor}, \textit{cord}-; Gr. κῆρ, καρδία \textit{kēr, kardia}; Arm. \textit{sirt}; Goth. \textit{haírtō}). However, its exact morphological evaluation is somewhat disputed, since apart from its widespread interpretation as an accusative singular form (as accepted above), it has also been suggested (\citealt[6--8]{Sandoz1973}; \citealt[583--584]{Tremblay2004}) to take it rather as an endingless locative (the meaning of the phrase being ‘to place sth. in one’s heart’). For recent detailed analyses of the construction cf. \citet[90--93]{Hackstein2012} (in relation to the issues of ‟colaescence" and univerbation\is{univerbation}); \citet{Weiss2019}.}                     (trust) in’, which is continued in the Indo-Iranian branch by a syntagmatic form\footnote{It has to be added that even Ved. \textit{śrád} had already more or less lost its syntactic autonomy and, as judged from its accentual behaviour and some properties of the argument structure, had become similar to local particles or preverbs (see \citealt[92]{Hackstein2012}). It is also worth mentioning that PIE *\textit{ḱr̥d}- (> PIIr. *\textit{ćr̥d}-) ‘heart’ as an independent noun seems to have been replaced in Proto-Indo-Iranian by a phonetically similar word: PIIr. *\textit{j́ʰr̥d}- > Ved. \textit{hŕ̥d}-; Av. \textit{zərəd}-. The exact relation of PIE *\textit{ḱr̥d}- to PIIr. *\textit{j́ʰr̥d}- is disputed (cf. EWAia II: 818; \citealt[271]{Weiss2019}).}                      (Ved. \textit{śrád} √\textit{dhā}, which is frequently attested, also with its components separated by intervening words, e.g., \textit{śrád asmai dhatta} ‘Trust in him!’ \iwi{R̥gveda (RV) 2.12.5d}; Av. \textit{zras=ča dāt̰} ‘and may she believe’ \iwi{Yašt (Yt) 9.26}), but by a simplex verb in the Italic (EDL: 141–142) and Celtic (EDPC: 221) languages as a result of univerbation\is{univerbation} (Lat. \textit{credo}; OIr. \textit{creitid}; MW \textit{credu}; MBr. \textit{crediff}, \textit{critim}; Corn. \textit{cresy}, \textit{krysi}, \textit{cregy}).

However illuminating this example may seem, there are some points which have to be borne in mind. Firstly, our data clearly show that the univerbation\is{univerbation} in this case did definitely not occur in the proto-language, but only in a much later period, certainly not earlier than the common Proto-Italo-Celtic period,\footnote{Possibly even much later, as \citet[274]{Weiss2019} assumes.}                        thus it can be referred to merely as a typological parallel to the hypothesised PIE (!) processes of univerbation of *\textit{dʰeh\textsubscript{1}}.

Secondly, in my view, it is questionable whether *\textit{ḱréd(s) dʰeh\textsubscript{1}} really has to be regarded as a support-verb construction at all. To be sure, as already mentioned above, the wide definition recognises the existence of support-verb constructions involving a non-deverbal concrete noun as the nominal host, if the latter is reconceptualised as eventive or undergoes metaphorical extension. However, similarly to *\textit{h\textsubscript{3}néh\textsubscript{3}mn̥ dʰeh\textsubscript{1}} treated above, the construction *\textit{ḱréd(s) dʰeh\textsubscript{1}} does not pass either of the two tests mentioned at the beginning of the chapter,\footnote{In fact, it passes the test of variativity even less than *\textit{h\textsubscript{3}néh\textsubscript{3}mn̥ dʰeh\textsubscript{1}} since there is no PIE root which would be derivationally connected to *\textit{ḱred-}/\textit{ḱr̥d-} ‘heart’ in any way.}            therefore it has to be taken in my understanding rather as a phraseological unit, i.e. an idiomatic\is{idiomatic expression} expression.\footnote{I maintain this claim even if it cannot be denied that, as one of my anonymous reviewers reminds me, support-verb constructions, too, may in principle involve some idiomatic components.}                    It follows that this example cannot be considered as a documented example of the univerbation\is{univerbation} of a genuine PIE support-verb construction belonging to the core of the category, even though the latter process seems to be cross-linguistically common, as Bowern points out (cf. above).

Similar considerations apply to the apparently parallel Indo-Iranian phrase *\textit{máns dʰaH} ‘to think of, to take note; lit. to set one’s mind’ (reflected by Avestan collocations, such as \textsuperscript{+}\textit{mǝ̄ṇg … dadē} \iwi{Yasna (Y) 28.4} ‘I take note of’ (cf. \citealt[178]{Peschl2022}) and by various nominal forms of both Vedic and Avestan (Ved. \textit{mandhātár}- ‘a thoughtful/devout person’, \textit{medhā́}- ‘intelligence, wisdom’, \textit{médhira}- ‘intelligent, wise’, Av. \textit{mazdā}- ‘wise/wisdom’, \textit{mązdra}- ‘wise’; see EWAia II: 313, 378)), except for the fact that, contrary to *\textit{ḱréd(s)}, *\textit{máns} is evidently a deverbal noun derived from the root *\textit{man} ‘to think’.

Some scholars (e.g., EDG II: 901; NIL: 493‒496 with n. 13; \citealt[281 n. 6]{Peschl2022}) have claimed that Greek μανθάνω \textit{man$t^h$anō} ‘to learn’ is a univerbated reflex of the same combination, but this is disputed (for an alternative view cf. \citealt[125]{Klingenschmitt1982}).\footnote{On the possible connection of the Indo-Iranian material with OCS \textit{mǫdrъ} ‘wise’ see, e.g., EWAia II: 378 with references; NIL: 496 with n. 16.}                    Remember, however, that even if it could be shown that already Proto-Indo-European did in fact have a construction *\textit{méns} (or *\textit{ménos}) \textit{dʰeh\textsubscript{1}} ‘to set one’s mind’, which was later univerbated either in the proto-language itself or separately in the daughter languages, it would still not count as an example of the univerbation\is{univerbation} of a prototypical PIE support-verb construction, since having in mind that *\textit{méns} (or *\textit{ménos}) is deverbal, but not an action noun, this phrase too would rather be classified as an idiomatic\is{idiomatic expression} unit (or a marginal support-verb construction at best).\footnote{Another example of this type is the phrase *\textit{g\textsuperscript{u̯}r̥h\textsubscript{2}- dʰeh\textsubscript{1}} ‘to offer (a) praise song(s)’ (cf. *\textit{g\textsuperscript{u̯}erh\textsubscript{2}} ‘to sing’ > OIA √\textit{gṝ} ‘to praise’; EWAia I: 468‒469; LIV: 210‒211), which is continued by OIA \textit{gíras} √\textit{dhā} ‘to offer praise songs’ and seems to be underlying Celtic *\textit{bardos} ‘singer, poet, bard’ (\citealt[37--38]{Balles2006}; see also below in n. 45 and 49).}

\section{Some case studies}

Since it is not possible to offer a comprehensive and exhaustive account of the entire scholarship on this topic, let us see now three representative case studies from the 2000s which hypothesise the univerbation\is{univerbation} of the original PIE support verb *\textit{dʰeh\textsubscript{1}} with some nominal element.

\subsection{PIE *\textit{bʰer(o) dʰeh\textsubscript{1}}?}

The first of them was formulated by \citet[240--241]{Janda2000}, who was followed by \citet[107--108]{Schutzeichel2014} in his afore-mentioned dissertation.

The Greek verb πέρθω \textit{per$t^h$ō} with the primary meaning ‘to loot, to capture; erbeuten’ is taken by Janda to be the reflex of PIE *\textit{bʰerdʰ} via the Proto-Greek devoicing of the PIE voiced aspirates and the phonological change called \textit{Grassmann’s law} (i.e. the regressive dissimilation of aspirates): PIE *\textit{bʰerdʰ} > PGr. *\textit{pʰertʰ} > Gr. πέρθ-ω \textit{per$t^h$-ō} (LIV: 77‒78 with n. 1; cf., on the other hand, EDG II: 1176 with question mark and the comment: ‟without a convincing etymology"\is{etymology}; GEW II: 512: ‟ohne überzeugende Etymologie"). Remember, however, that in the absence of any cognates\is{cognate} of this root in other IE languages,\footnote{Frisk (GEW II: 512) refers to Uhlenbeck’s suggestion to connect Gr. πέρθω \textit{per$t^h$ō} with OIA \textit{bardhaka-} ‘carpenter’ (note that the correct form of this noun is \textit{vardhaka-}; KEWA III: 157) and some Germanic words meaning ‘desk, plank’, but this hypothesis is semantically very doubtful.}        *\textit{bʰerdʰ} can in fact be regarded as nothing more than a \textit{Transponat}\is{Transponat}, the PIE status of which, if we stick to the methodological rigour of comparative linguistics, is entirely uncertain.

In a second step, the alleged PIE root *\textit{bʰerdʰ} is analysed by Janda as the univerbation\is{univerbation} of an original support-verb construction consisting of the support verb *\textit{dʰeh\textsubscript{1}} and what must be a deverbal action noun derived from *\textit{bʰer} ‘to carry, to bring’ (i.e. *\textit{bʰer dʰeh\textsubscript{1}} ‘lit. to do “carrying away”’). While Janda himself assumes that the nominal member of the phrase was a root noun *\textit{bʰer} (but why in its stem form? or was it a neuter noun with a zero accusative ending?), Schutzeichel posits it in the remarkable form “*\textit{bʰero}”, but fails to explain the reasons for his choice. Therefore, it is uncertain whether he assumes this to be the stem form of a thematic noun *\textit{bʰero-} (but why e-grade of the root?) or a peculiar case form of the root noun *\textit{bʰer-} (but which case?). To be sure, phonological attrition or erosion frequently accompanies grammaticalisation\is{grammaticalisation} and lexicalisation\is{lexicalisation} processes including univerbation\is{univerbation} (cf. \citealt[22--23]{Balles2006}) and thus it would not be impossible that an *\textit{o} was lost during the alleged univerbation, but I think that in our case its assumption, at least in its present form, is unfounded.

Furthermore, Janda seeks to underpin his hypothesis by referring to a Vedic Old Indo-Aryan phrase, which is built from etymologically related elements and therefore, according to him, supports the assumption of the earlier existence of the alleged support-verb construction *\textit{bʰer dʰeh\textsubscript{1}}, see (\ref{VedOIA}).

\ea \label{VedOIA}
\settowidth \jamwidth{(Vedic Old Indo-Aryan)}
\gll \textit{sá}	\textit{\textbf{no}}	\textit{vṛ́ṣā}    \textit{vṛ́ṣarathaḥ}		\textit{suśipra} \textit{vṛ́ṣakrato}		\textit{vṛ́ṣā}  \textit{vajrin}	\textit{\textbf{bháre}}  \textit{\textbf{dhāḥ}}\\
such.\textsc{nom}    \textsc{1pl.acc}   bull.\textsc{nom}   with\_a\_bullish\_chariot.\textsc{nom}  well-lipped.\textsc{voc}    with\_bullish\_will.\textsc{voc}    bull.\textsc{nom}    with\_the\_mace.\textsc{voc}    loot.\textsc{loc}    place.\textsc{aor.inj.act.2sg}\\ \jambox{(Vedic Old Indo-Aryan)}
\glt ‘As bull with a bullish chariot, well-lipped one, you with bullish will, as bull, you of the mace, set us up in loot.’ \\
\hspace*{\fill}(\iwi{R̥gveda (RV) 5.36.5cd}, translation following \citealt[II: 703]{JamisonBrereton2014})\footnote{I depart at a single point from Jamison’s version, i.e. in translating \textit{vṛ́ṣakrato} not as the attributive modifier of the predicative nominative \textit{vṛ́ṣā} (‟as bull with bullish 
will" 
in her translation), but as a vocative, which it certainly is. Geldner (followed by \citealt[108]{Schutzeichel2014}) takes the two lines as separate clauses. He regards \textit{no} in pāda c as the enclitic genitive form of the personal pronoun and supplies another \textit{nas} as an accusative in the second clause. His translation runs as follows: ‟Du bist unser Bulle mit dem Bullenwagen, du Schönlippiger. Du Bullenmutiger verhilf (uns) als Bulle [Anführer] zur Beute, o Keulenträger!" \citep[II: 36]{Geldner1951}.}
\z

It is, of course, undeniable that the individual members of the Vedic phrase \textit{bháre dhāḥ} are etymologically related to the PIE roots *\textit{bʰer} and *\textit{dʰeh\textsubscript{1}}, respectively. Nevertheless, apart from the obvious semantic discrepancies, the syntactic configuration of \textit{no … bháre dhāḥ} too is entirely different from that of the alleged support-verb construction *\textit{bʰer(o) dʰeh\textsubscript{1}}. Namely, in a support-verb construction such as the one hypothesised by Janda the nominal member, in our case *\textit{bʰer(o)}, should be the syntactic object argument of the support verb *\textit{dʰeh\textsubscript{1}}, while in the Vedic clause the direct object of the verb predicate is the pronominal clitic \textit{no} and \textit{bháre} is a locative expressing a goal.\footnote{I would like to point out that my argumentation concerning this particular example has nothing to do with the broader question whether non-accusative NP+V or Prepositional Phrase + Verb (PP + V henceforth) phrases in general should be acknowledged as belonging to the category of light-verb or support-verb constructions (as the \textit{Funktionsverbgefüge}-tradition claims: cf., e.g., Germ. \textit{zur Aufführung bringen}) or not.}      Thus, we have to conclude that no support-verb construction *\textit{bʰer(o) dʰeh\textsubscript{1}} may be reconstructed for Proto-Indo-European (or Pre-Proto-Indo-European) and the assumption of its erstwhile existence is in my view nothing more than unfounded speculation.

\subsection{PIE *\textit{k\textsuperscript{u̯}olh\textsubscript{1}im dʰeh\textsubscript{1}}?}

Similar considerations apply to the idea of \citet[21--22]{Balles2009}, who regards the Greek verb active κυλίνδω \textit{kulindō} ‘(trans.) to roll’, middle κυλίνδομαι \textit{kulindomai} ‘to be rolled, (intrans.) to roll’ as a thematic verb derived from an adjective *\textit{k\textsuperscript{u̯}olh\textsubscript{1}imdʰeh\textsubscript{1}-} or *\textit{k\textsuperscript{u̯}olh\textsubscript{1}imdʰh\textsubscript{1}o-} ‘rolling’ and ultimately traces it back to a PIE support-verb construction *\textit{k\textsuperscript{u̯}olh\textsubscript{1}im dʰeh\textsubscript{1}} ‘to make rolling(s), (intr.) to roll, to revolve’. The nominal host (*\textit{k\textsuperscript{u̯}olh\textsubscript{1}i-}) of the construction would be the action noun derived from the PIE root *\textit{k\textsuperscript{u̯}elh\textsubscript{1}} ‘to revolve, to turn around, to roll’ (cf. OIA √\textit{car} ‘to move, to go’; Av. √\textit{car} ‘to go’; Gr. πέλομαι \textit{pelomai} ‘to move, to become, to be’; Lat. \textit{colo} ‘to cultivate, to inhabit, to dwell’; HLuw. \textit{k(u)wali-} ‘[trans.] to turn’; LIV: 386–388). Since this derivation implies a disputed Greek sound change (*\textit{{NDʰ}}\footnote{In our case this would be preceded by the place assimilation *\textit{mdʰ} > *\textit{ndʰ}.}        > \textit{{ND}} ‟in bestimmten Kontexten"),\footnote{For this reason, \citet[128--129]{Schutzeichel2014} too considers Balles’ etymology\is{etymology} doubtful.}  Balles does not rule out the possibility of the support verb *\textit{deh\textsubscript{3}} ‘to give’ as an alternative.

However, there are some considerations which make the assumption of PIE *\textit{k\textsuperscript{u̯}olh\textsubscript{1}im dʰeh\textsubscript{1}} rather doubtful. Since PIE *\textit{k\textsuperscript{u̯}elh\textsubscript{1}} was a so-called\footnote{On the terminology see, e.g., \citet[25--29]{Gotō1987}; \citet[6--7]{Kümmel2000}.}      inattingent (i.e. no second actant is directly affected by the action) and syntactically intransitive verb, its derivative, the action noun *\textit{k\textsuperscript{u̯}olh\textsubscript{1}i-}, if it ever existed, must have had an intransitive semantics too (‘turning, revolving’ and not transitive ‘rolling sth, turning sth’). Accordingly, the alleged PIE support-verb construction *\textit{k\textsuperscript{u̯}olh\textsubscript{1}im dʰeh\textsubscript{1}} (or *\textit{deh\textsubscript{3}}) would have had to be equivalent to an intransitive simplex verb (cf. above: ‘to make rolling(s), [intr.] to roll, to revolve’), which means that the transitive active inflection of the Greek verb κυλίνδω \textit{kulindō} would have to be regarded as secondary to its intransitive middle κυλίνδομαι \textit{kulindomai}. Otherwise, we would have to suppose that the PIE support-verb construction expressed causativity (i.e. ‘to make a/the rolling [of sth./sb. else]; to roll sth./sb.’). Nevertheless, even if these considerations are left aside, the construction still only has the status of a \textit{Transponat}\is{Transponat} and its assumption for PIE is completely uncertain.\footnote{Beekes (EDG I: 800) regards κυλίνδω \textit{kulindō} as a borrowing from Pre-Greek and adds that ‟the word is hardly IE".}

\subsection{PIE *\textit{bʰsméh\textsubscript{2} dʰeh\textsubscript{1}}?}

\citet{Garnier2006} investigates the etymology\is{etymology} of Greek ψάμαθος \textit{psama$t^h$os} ‘dust, sand’ and traces it back to a PIE adjective *\textit{bʰsm̥‑h\textsubscript{2}‑dʰh\textsubscript{1}‑ó‑} ‘reduced to powder, pulverised’, which he then derives from an earlier phrase *\textit{bʰs‑m‑éh\textsubscript{2} dʰeh\textsubscript{1}} ‘to reduce to powder, to pulverate; lit. to make into powder’.\footnote{In Garnier’s opinion, *\textit{bʰs‑m‑éh\textsubscript{2}‑} ‘siltage, dust, rubbish’ is a so-called collective from *\textit{bʰos‑mó‑} ‘rubbing, sweeping’, a derivative of the PIE root *\textit{bʰes} ‘to crumble, to sweep’. He thinks that Proto-Germanic *\textit{samðaz} ‘sand’ has the same origin as the Greek noun, although it has undergone some additional analogical changes.}      Although Garnier himself refers to this syntagm as a periphrastic causative formation (with *\textit{dʰeh\textsubscript{1}} meaning `placer, mettre dans tel état' \citealt[82]{Garnier2006}) and not as a support-verb construction, later it is classified as such by \citet[109]{Schutzeichel2014}. In my opinion, the classification of Schutzeichel is incorrect and the alleged PIE phrase *\textit{bʰs‑m‑éh\textsubscript{2} dʰeh\textsubscript{1}}, if it ever existed, would have to be regarded as a copula-predicative construction, in which the verb *\textit{dʰeh\textsubscript{1}} functions as a factitive copula\is{factitive copula} (‘to make sth. into sth.’) and not as a support verb.

The function of *\textit{dʰeh\textsubscript{1}} in the collocation supposed by Garnier is thus equivalent to the use of OIA √\textit{kṛ} ‘to make, to do’ in various constructions (\citealt[41--44]{Ittzés2016} with references). Beside the very frequent double-accusative construction and the so-called \textit{cvi}-construction,\footnote{The \textit{cvi}-construction is a largely grammaticalised\is{grammaticalisation} analytic predicative construction of Old Indo-Aryan, consisting of an invariable and synchronically opaque nominal form in \textit{-ī} (occasionally -\textit{ū}), which is called \textit{cvi} by the 4th-century Indian grammarian Pāṇini, and one of the two copula verbs (√\textit{kṛ} ‘to make, to do’ or √\textit{bhū} ‘to become’): e.g., \textit{nava-} ‘new’ → \textit{navī} √\textit{kṛ} ‘to make new, to revive’; \textit{yuvan-} ‘new’ → \textit{yuvī} √\textit{bhū} ‘to become young’. For an exhaustive treatment, see \citet{Balles2006}.}
mention has to be made of the use of √\textit{kṛ} in combination with predicative instrumentals (cf. \citealt[245--247]{Balles2006}) and adverbs \citep{Hoffmann1976a}.\footnote{Several adverbs that are used predicatively as well go back to instrumental case forms themselves. On the instrumental origin of ‟Präverbien" in \textit{-ā}, see \citealt{Hoffmann1976b} (especially 353). In \textit{gúhā} ‘secretly’, note the adverbial accent shift as compared to instrumental singular \textit{guhā́} \iwi{R̥gveda (RV) 1.67.6b} of \textit{gúh-} ‘hiding place’ (\citealt[144]{Jasanoff2002-2003}; \citealt[116 n. 2]{Hoffmann1975}.).} 

With many predicative adverbs, the ‟Allerweltsverbum" or ‟passepartout" verb √\textit{kṛ} can be regarded as a colloquial replacement for other verbs with a richer meaning, such as √\textit{dhā} ‘to put, to place’ and a few more (cf. also \citealt[350 with n. 4]{Hoffmann1976b}). Consider, for instance, \textit{gúhā} √\textit{kṛ} ‘to hide, to conceal’ (\iwi{R̥gveda (RV) 4.18.5ab}) beside \textit{gúhā} (\textit{ni}+)√\textit{dhā} (\iwi{R̥gveda (RV) 3.56.2d}; \iwi{R̥gveda (RV) 10.5.2d}).\footnote{Note that √\textit{dhā} in such cases is not necessarily a synonym of √\textit{kṛ} as suggested by \citet[144--145]{Jasanoff2002-2003}, but might rather be interpreted as a verb with its full lexical meaning.}     Another illustrative example is \textit{āré} (‘far’) √\textit{kṛ} ‘to put away’ (\iwi{R̥gveda (RV) 8.61.16c}) beside \textit{āré} in combination with √\textit{dhā} (\iwi{R̥gveda (RV) 8.47.13d}), √\textit{bādh} ‘to press, to repel, to remove’ (\iwi{R̥gveda (RV) 9.66.19c}), or √\textit{yu} ‘to keep away, to ward off’ (\iwi{R̥gveda (RV) 10.63.12c}).

An instrumental origin is the most plausible explanation for the whole category of the Old Indo-Aryan \textit{cvi}-formation as well (\citealt[391--393]{Schindler1980}; \citealt[190--191]{Widmer2005}; \citealt[passim, esp. 287--292]{Balles2006}; cf. n. 42 above).

It is worth mentioning briefly in this context that PIE constructions consisting of a predicative instrumental and a (factitive) copula\is{factitive copula} are thought to be underlying also PIE stative-factitive pairs, such as the ones reflected in Latin \textit{caleo} ‘to be hot’ / \textit{calesco} ‘to grow hot’ vs. \textit{calefacio} ‘to make hot’, \textit{rubeo} ‘to be red’ / \textit{rubesco} ‘to turn red’ vs. \textit{rubefacio} ‘to make red’ etc. (see, first of all, \citealt{Jasanoff2002-2003}). Remember, however, that according to the definition adopted in this paper, the factitive member (*‘to make sth. [being with] hot[ness]’ etc.) of such putative PIE pairs was not a support-verb construction.

It has also been suggested (\citealt{MeierBrügger1980}; \citealt[475 n. 38]{Bader1986}; EDL: 61; EDG I: 43) that Gr. αἰσθάνομαι \textit{ais$t^h$anomai} ‘to perceive, apprehend’ and Lat. \textit{audio} ‘to hear’ also go back to a PIE phrase consisting of a predicative adverb followed by the root *\textit{dʰeh\textsubscript{1}}. The first member of the collocation is now generally thought to have been the adverb known from Ved. \textit{āvíṣ}, Av. \textit{āuuiš} ‘manifestly’; cf. also OCS \textit{(j)avě} ‘evidently’. I must add, however, that following this etymology\is{etymology} (*‘to make manifest’), I would expect the verb to mean something like ‘to show’ rather than ‘to perceive’. \citet[290]{MeierBrügger1980}, no doubt having in mind the deponency of the Greek verb αἰσθάνομαι \textit{ais$t^h$anomai}, gives the meaning of the original collocation as `\textit{sich} etwas offenbar machen' (emphasis mine), but even that implies, in my view, some intention on behalf of the subject, which is generally not characteristic of the process of perceiving or hearing. Furthermore, I have to stress that *\textit{dʰeh\textsubscript{1}} would not have functioned as a support verb in this phrase, therefore it is not immediately relevant to the present issue of the univerbation\is{univerbation} of support-verb constructions.

Finally, another related phenomenon, which has no support-verb construction origin, is the Latin adjective type in \textit{-idus}, which has been interpreted as the nominalisation (\textit{-idus} < *\textit{-idʰo-} < *\textit{-i(h\textsubscript{x})-dʰh\textsubscript{1}o-}) of a PIE syntagm consisting of the instrumental of \textit{i}-stem adjectival abstracts + *\textit{dʰeh\textsubscript{1}}: e.g. \textit{rubidus} ‘red, suffused with red < *(made with) red(ness)’ (\citealt[222--225]{Balles2006}; cf. \citealt{Nussbaum1999}; \citealt[13--14, 16--17]{Hackstein2002b}; \citealt{Balles2003}).

\section{The evaluation of the case studies}

In spite of the popularity of this kind of approach in recent scholarship, there are virtually no examples in which the univerbation\is{univerbation} of an earlier support-verb construction in one or more daughter languages could definitely be proven by means of the syntagmatic evidence surviving in others.\footnote{An exception to this is furnished by *\textit{ḱréd(s) dʰeh\textsubscript{1}}, but as I have argued above, it may be an idiomatic expression\is{idiomatic expression} rather than a support-verb construction in the strict sense. \citet[116]{Schutzeichel2014} claims that Vedic \textit{nāmadhā́-} ‘name-giver’ (cf. \citealt[254--255]{Scarlata1999}) is a univerbation of the PIE phrase *\textit{h\textsubscript{1}néh\textsubscript{3}mn̥ dʰeh\textsubscript{1}}, which survives as a syntagm in several daughter languages (cf. (\ref{openslot}) with initial *\textit{h\textsubscript{3}}), but this assumption is unnecessary. It could simply be a dependent determinative compound (\textit{tatpuruṣa} in the native Indian tradition) built according to the productive patterns of nominal composition (cf., e.g., \textit{somapā́-} ‘drinking soma’ etc.). The same applies to Celtic *\textit{bardos} ‘singer, poet’ beside OIA \textit{gíras} √\textit{dhā} ‘to offer praise songs’ from PIE *\textit{g\textsuperscript{u̯}r̥h\textsubscript{2}- dʰeh\textsubscript{1}-}.}       This is, of course, not to deny that there could be and are indeed cases in which the assumption of univerbation seems in fact to be the best solution (such as, e.g., the origin of the German weak preterit). However, we should remember that in such potential examples the univerbation\is{univerbation} must have taken place in all probability well after the break-up of the parent language and not within PIE or Pre-PIE itself.

As will have become clear, the application of the ‟univerbation hypothesis" when looking for PIE (or Pre-PIE) support-verb constructions has several pitfalls. Moreover, it seems to me improbable also on theoretical grounds that so many, if not all, PIE roots with an extension *\textit{-dʰ-} and so many lexemes of the daughter languages containing a potential reflex of PIE *\textit{dʰ} would ultimately go back to earlier support-verb constructions with *\textit{dʰeh\textsubscript{1}}.\footnote{Not to speak about other hypothesised univerbated support verbs, such as *\textit{gʰeh\textsubscript{1}} (cf. n. 16) or *\textit{deh\textsubscript{3}} (cf. above in the main text).}  Nevertheless, the typological considerations mentioned above make it reasonably certain that PIE did have support-verb constructions, among them obviously some (or possibly most) with the support verb *\textit{dʰeh\textsubscript{1}}. However, instead of positing an actually existing (Pre-)PIE support-verb construction in each and every case, I consider the following or a similar scenario theoretically more plausible (cf. \citealt[145--150]{Schutzeichel2014}).

Some support-verb constructions may actually have been univerbated at an early stage of the proto-language. The resulting formations may have been reanalysed\footnote{On reanalysis in general see, e.g., \citet[50--68]{HopperTraugott2003}.}  as stems containing a suffix-like extension added to what could be reinterpreted as a verbal root instead of the original nominal (root noun) host of a support-verb construction. Such extensions could then acquire a specific grammatical function and become a productive morpheme (e.g. *\textit{‑dʰ‑} as a factitive-causative (?)\footnote{However, this assumption seems to be incompatible with the observations of \citet[3 n. 3 and passim]{RothsteinDowden2022}, who argues that the dental-aspirate presents of PIE were originally intransitives.}            suffix), which may later have been added to other verbal roots with the same function. Finally, the original function of the suffix may have become opaque, which could result in the emergence of secondary roots with apparently meaningless enlargements. This means that several examples mentioned in the secondary literature have probably never been support-verb constructions at all, but were formed only at a later stage of the process just described. This means that, for instance, we had better not posit support-verb constructions such as *\textit{ǵʰeu̯ deh\textsubscript{3}} ‘lit. to give a pour(ing)’ merely on the basis of the ‟enlarged" root-variant *\textit{ǵʰeu̯d} beside *\textit{ǵʰeu̯} (cf. (\ref{pour}) above).

\section{The function of Proto-Indo-European support-verb constructions}

In my opinion, the main, but unfortunately inevitable shortcoming of all the studies that reconstruct PIE support-verb constructions is that due to the lack of original texts in PIE, not to mention native speakers with their own grammaticality judgements, nothing can be said with certainty about the function of these constructions within the language system of PIE and about their properties as compared to related simplex verbs. These could namely be detected only by means of corpus-based empirical investigations (cf. \citealt{Storrer2006} or \citealt{Kamber2008} with respect to German). 


Mainly on the basis of typological parallels from living languages, it is usually assumed, insofar as this question is dealt with at all (see, e.g., \citealt[37]{Balles2006}; \citealt[79]{Schutzeichel2014}), that support-verb constructions existed in the proto-language first of all as stylistic-pragmatic variants or technical terms.\footnote{To support this assumption, \citet[38]{Balles2006} also refers to the fact that the category of \textit{cvi}-constructions, which is in a certain sense similar to that of support-verb constructions (cf., however, above on their differences), included some agricultural terms too. She also mentions the PIE phrase *\textit{g\textsuperscript{u̯}r̥h\textsubscript{2}- dʰeh\textsubscript{1}-} (cf. n. 33 and 45 above), which ‟könnte ein Fachterminus für das Verfassen und Vortragen von Preisliedern auf eine Gottheit gewesen sein".}       The reason for this hypothesis is that, on the one hand, simplex verbs constituted an open word class with a fairly large number of elements in the proto-langauge and on the other hand, PIE formed denominative verbs and expressed various grammatical categories (such as aspect, \textit{Aktionsart}, tense, or mood) fundamentally by means of morphological devices, i.e. bound affixes, thus there seems to have been no need for support-verb constructions in such functions. Accordingly, support-verb constructions may have acquired the function of expressing aspect or \textit{Aktionsart} in the daughter languages only secondarily (cf. \citealt[38 n. 85]{Balles2006}; \citealt[79]{Schutzeichel2014}).

However, as I have argued in previous studies on support-verb constructions of Vedic Old Indo-Aryan \citep{Ittzés2013,Ittzés2016}, the existence of separate tense-aspect stems in a language does not necessarily mean that support-verb constructions may not have specific grammatical functions related to these categories, mainly in the context of suppletion\is{suppletion}. An illuminating example is the Vedic support-verb construction \textit{śruṣṭíṃ} √\textit{kṛ} ‘to obey; lit. to do obeying’ beside the simplex verb √\textit{śruṣ} ‘to obey’, which are in complementary distribution (the former is inflected in the aorist and perfect, the latter exclusively in the present-stem forms) and thus make up a suppletive paradigm\is{suppletive} in terms of the category of aspect (\citealt[107--108]{Ittzés2013}; \citealt[61--65]{Ittzés2016}).

\largerpage[-1]
Another example of the same phenomenon is \textit{vimócanaṃ} √\textit{kṛ} ‘to unyoke; lit. to do unyoking’, which is attested in Vedic with middle inflection of the support verb (\textit{vimócanaṃ kṛṇute} \iwi{R̥gveda (RV) 3.30.12d}).\footnote{With its single attestation, the support-verb construction \textit{vimócanaṃ kṛṇute} has to be considered as a nonce‑formation. However, since it apparently followed the same suppletive strategy\is{suppletive} as other similar constructions, it is in this sense not isolated in Early Vedic.}  This feature stands in contrast to the active-only inflection of the agentive-attingent, transitive simplex verb \textit{vi}+√\textit{muc} ‘to unyoke’.\footnote{The only real exception to this is \textit{ví mucadhvam} \iwi{R̥gveda (RV) 1.171.1d}. However, as I have demonstrated in \citet[114--116]{Ittzés2013}, this aorist imperative middle form is only metrically conditioned and therefore irrelevant to the evaluation of the support-verb construction \textit{vimócanaṃ kṛṇute}.}      As I have argued elsewhere \citep{Ittzés2013}, this support-verb construction probably supplies the missing (direct‑reflexive) middle of \textit{vi}+√\textit{muc} in Early Vedic, i.e. the two can be regarded as making up a suppletive paradigm\is{suppletive} with respect to verbal diathesis.\footnote{A further example is possibly furnished by the construction consisting of the support verb √\textit{kṛ} and the deverbal noun ˚\textit{héḍ/ḷana}‑ ‘angering, making sb. angry’ (a derivative of the causative \textit{heḍ/ḷáya}‑ ‘to make angry’ of the fientive‑inattingent, intransitive root √\textit{hīḍ/heḍ} ‘to be or get angry’), which is attested in the preventive prohibitive (on this notion, cf. \citealt{Hoffmann1967}) clause \textit{mā́ karma devahéḷanam} ‘let us not make the gods angry; lit. let us not do the angering of the gods’ \iwi{R̥gveda (RV) 7.60.8d}. It seems that in this case the support-verb construction was employed to supply the synthetic reduplicated causative aorist of the verb √\textit{hīḍ/heḍ} (*\textit{mā́ devā́ñ jīhiḷāma}; note that *\textit{mā́ heḷáyāma} would be inhibitive as per Hoffmann), which was apparently still absent from the verb’s paradigm in Early Vedic and was formed only later in Old Vedic (aorist stem \textit{jīhiḷa-}; cf. 3rd singular aorist indicative \textit{ájīhiḍat} \iwi{Atharvaveda (Śaunakīya recension) (AVŚ) 12.4.8b = Atharvaveda (Paippalādarecension) (AVP) 17.16.7b}, but with quite different semantics; see \citealt[351 n. 866]{Gotō1987}). On this example see \citet[343--345]{Ittzés2015} and (slightly revising the earlier account) \citet[108--111]{Ittzés2016} (also on possible counterarguments).}

Having in mind what has been said here on the status of support-verb constructions in the grammatical system of languages with a tense-aspect system, due to lack of relevant evidence, we necessarily have to remain agnostic about the functions of such constructions in the Proto-Indo-European parent language. They might have been merely stylistic or pragmatic variants of etymologically related simplex verbs, but they might have had some specific grammatical function in the language system.

\section{Conclusion}

To conclude, it seems to be fairly certain from a typological point of view that Proto-Indo-European did in fact have support-verb constructions consisting of verbal nouns (prototypically action nouns) and verbs of a rather broad lexical meaning, such as ‘to put, to set’, ‘to give’, ‘to go’, the most prominent of which was in all probability the root *\textit{dʰeh\textsubscript{1}}.

However, when it comes to reconstructing specific PIE support-verb constructions, we immediately have to face several serious issues, the most fundamental of which is the virtually complete lack, or at least extreme rarity, of comparable constructions built with cognate\is{cognate} elements in the daughter languages, which in my view would be a necessary prerequisite for the comparative reconstruction\is{reconstruction} of PIE support-verb constructions. In my view, the assumption of ‟open-slot constructions"\is{open-slot construction} for the proto-language or the application of the notion of ‟lexical substitution"\is{lexical substitution} in the reconstructions also have their own pitfalls and run counter to various theoretical and methodological principles of comparative historical linguistics. 


The nowadays popular approach based on what I would call the ‟univerbation hypothesis"\is{univerbation} also fails to produce solid and falsifiable results. Moreover, even if specific support-verb constructions could somehow be reconstructed for the proto-language, we would still be unable to discover their original function in the language system due to the impossibility of corpus-based empirical investigations.














\section*{Abbreviations}


%search up henceforth, V+N, V+PP, 

\begin{tabularx}{.5\textwidth}{@{}lQ@{}}
Arm. & Armenian \\
Av. & Avestan \\
C & Consonant \\
Corn & Cornish  \\
EDG & \citealt{Beekes2010} \\
EDL & \citealt{MichieldeVaan2008} \\
EDHIL & \citealt{Kloekhorst2008} \\
EDPC & \citealt{Matasović2009}  \\
Eng. & English \\
EWAia & \citealt{Mayrhofer1992–2001} \\
Fr. & French \\
Germ. & German \\
GEW & \citealt{Frisk1960-1972} \\
Goth. & Gothic \\
Gr. & (Ancient) Greek \\
H & Laryngeal \\
Hitt.& Hittite \\
HLuw.& Hieroglyphic Luwian \\
It. & Italian \\
KEWA & \citealt{Mayrhofer1956-1980} \\
Lat.& Latin \\
Lith.& Lithuanian \\
LIV & \citealt{LIV} \\ %Victoria added for testing
LIVAdd & \citealt{Kümmel2024} \\
\end{tabularx}%
\begin{tabularx}{.5\textwidth}{@{}lQ@{}}
MBr. & Middle Breton \\
MW & Middle Welsh \\
N & Noun \\
NHG & New High German \\
NIL & \citealt{NIL} \\
O & Object \\
OCS & Old Church Slavonic \\
OHG & Old High German \\
OIA & Old Indo-Aryan \\
OIr. & Old Irish \\
P & Plosive / stop \\
Pal.& Palaic \\
PGr.& Proto-Greek \\
PIE & Proto-Indo-European \\
PIIr. & Proto-Indo-Iranian \\
PP & Prepositional Phrase \\
REW & \citealt{MeyerLübke1935} \\
SCr. & Serbo-Croatian \\
Toch. & Tocharian \\
Umbr. & Umbrian \\
V & Verb \\
Ved. & Vedic (Old Indo-Aryan) \\
VIA & \citealt{Werba1997} \\
\end{tabularx}





\section*{Acknowledgements}
The writing of this paper was supported by the Hungarian Ministry of Culture and Innovation from the National Research, Development and Innovation Fund (project No. K 142535). I express my heartfelt thanks to Victoria Fendel and my anonymous reviewers for their comments and suggestions and to Wyn Shaw for their help in technical issues.



%\section*{Contributions}
%John Doe contributed to conceptualization, methodology, and validation. 
%Jane Doe contributed to writing of the original draft, review, and editing.

\sloppy
\printbibliography[heading=subbibliography,notkeyword=this]
\end{document}
