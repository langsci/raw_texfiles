\documentclass[output=paper,colorlinks,citecolor=brown]{langscibook}
\ChapterDOI{10.5281/zenodo.14017933}

\author{Alfonso Vives Cuesta\affiliation{Universidad de Valladolid / Instituto Bíblico y Oriental}}

\title[Support-verb constructions as level-of-speech markers]{Support-verb constructions as level-of-speech markers in a corpus of hagiographical   literature}


\abstract{This contribution traces the diachronic development of a specific type of   verbo-nominal collocations in a post-classical Greek corpus limited to   prototypical\linebreak support-verb constructions with ποιέω \emph{poieō} + eventive   noun. For this purpose, the chapter draws on an extensive corpus of Byzantine saints'   lives and adopts an eclectic methodology, which benefits from the developments in corpus   linguistics, sociolinguistics\is{sociolinguistics}, and Byzantine studies. In addition to stylistic and  register variation, it delves into the lexical and syntactic properties of some of these   collocations and pinpoints reasons for their development and renewal. The   study focusses on a wide selection of texts of the hagiographic genre, covering a wide   timespan (4th--14th centuries). It contributes to the better understanding of the   procedures of formal renewal and variation of support-verb constructions and   constructions with support-verb extensions in diachrony.
\bigskip

\foreignlanguage{spanish}{Esta contribución rastrea el desarrollo diacrónico de un tipo
  específico de colocaciones verbo-nominales en un corpus griego postclásico limitado a
  construcciones prototípicas de verbo soporte con ποιέω \emph{poieō} + sustantivo
  eventivo. Para ello, he compilado un extenso corpus de vidas de santos bizantinos y he
  adoptado una metodología ecléctica, que se beneficia de los desarrollos de la
  lingüística de corpus, la sociolingüística y la bizantinística. Además de la variación
  estilística y de registro, profundizo en las propiedades léxicas y sintácticas de
  algunas de estas colocaciones y voy a dar cuenta de su desarrollo
  %e intento señalar algunas razones de su desarrollo 
  y renovación formal. El estudio se centrará en una amplia selección de textos del género
  hagiográfico, abarcando un amplio espectro temporal (siglos \textsc{IV--XIV}). Con ello
  se espera obtener una mejor caracterización de los procedimientos de renovación y
  variación formales de las construcciones con verbo soporte y de las construcciones con
  extensión del verbo soporte en diacronía.} }




\IfFileExists{../localcommands.tex}{
   \addbibresource{../localbibliography.bib}
   % add all extra packages you need to load to this file

\usepackage{tabularx,multicol}
\usepackage{url}
\urlstyle{same}

\usepackage{listings}
\lstset{basicstyle=\ttfamily,tabsize=2,breaklines=true}

\usepackage{langsci-basic}
\usepackage{langsci-optional}
\usepackage{langsci-lgr}
\usepackage{langsci-osl}
% \usepackage{./langsci/styles/langsci-lgr}
% \usepackage{./langsci/styles/langsci-osl}
% \usepackage{langsci-gb4e}

\usepackage{tikz}
\usetikzlibrary{patterns,calc}
\pgfdeclarepatternformonly{south east lines}{\pgfqpoint{-0pt}{-0pt}}{\pgfqpoint{3pt}{3pt}}{\pgfqpoint{3pt}{3pt}}{
    \pgfsetlinewidth{0.6pt}
    \pgfpathmoveto{\pgfqpoint{0pt}{3pt}}
    \pgfpathlineto{\pgfqpoint{3pt}{0pt}}
    \pgfpathmoveto{\pgfqpoint{.2pt}{-.2pt}}
    \pgfpathlineto{\pgfqpoint{-.2pt}{.2pt}}
    \pgfpathmoveto{\pgfqpoint{3.2pt}{2.8pt}}
    \pgfpathlineto{\pgfqpoint{2.8pt}{3.2pt}}
    \pgfusepath{stroke}}
    
\usepackage{stmaryrd}
\usepackage{wasysym}
\usepackage{multirow}
\usepackage{caption}
\usepackage{subcaption}
\usepackage{mathrsfs}
\usepackage{qtree}

\usepackage{linguex}


   %pminos do not split footnotes
% \interfootnotelinepenalty=10000 %Footnote in Laporte chapters has to be split SN


%\DeclareIndexNameFormat{default}{%
%\nameparts{#1}%
%\usebibmacro{index:name}%
%{\index[names]}%
%{\namepartfamily}%
%{\namepartgiveni}%
% {}% L1
% {}% L2
%{\namepartprefix}% generates spurious space L3
%{\namepartsuffix}% generates spurious space L4
%}

%  {\DeclareIndexNameFormat{default}{%
%     \usebibmacro{index:name}{\index[names]}{#1}{#3}{#5}{#7}}}

%\DeclareIndexNameFormat{default}{%
%  \usebibmacro{index:name}{\sindex[nom]}{#1}{#3}{#5}{#7}}

%\DeclareIndexNameFormat{default}{%
%  \usebibmacro{index:name}{\sindex[person]}{#1}{#3}{#5}{#7}}
%\DeclareIndexNameFormat{default}{%
%\nameparts{#1} \usebibmacro{index:name}{\sindex[person]]}{\namepartfamily}{‌​\namepartgiven}{\nam‌​epartprefix}{\namepa‌​rtsuffix}}

%\newcommand{\smiley}{:)}

%\renewbibmacro*{index:name}[5]{%
%\usebibmacro{index:entry}{#1}%
%{\iffieldundef{usera}{}{\thefield{usera}\actualoperator}\mkbibindexname{#2}{#3}{#4}{#5}}}

% \newcommand{\noop}[1]{}

%remove for final
%\overfullrule=1mm

\newcommand{\tobi}[2]}}
\renewcommand{\S}[1]{\tobi{#1}{\textsc{*}}}

% this volume references
% puts: [this volume]
% already defined: \citetv
%\newcommand{\citepv}[1]{(\citeauthor{#1} \citeyear*{#1} [this volume])}
\newcommand{\citealtv}[1]{\citeauthor{#1} \citeyear*{#1} [this volume]}

%parentheses around example number
\newcommand{\pref}[1]{(\ref{#1})}

% in-text examples

\newcommand{\lnex}[1]{\textit{#1}} %target lang word
\newcommand{\lnlit}[1]{(lit.: `#1')} %literal reading
\newcommand{\lnlat}[1]{(#1)} % latinization
\newcommand{\lntrans}[1]{`#1'} %translation
\newcommand{\lnexl}[2]%
{\lnex{#1}{} \lnlat{#2}} % ex with latinization
\newcommand{\lnexlat}[3]{\lnex{#1}{} \lnlat{#2}{} \lntrans{#3}} % ex with latinization and tranl.

%ch01
\newcommand{\co}[1]{\mbox{\textbf{#1}}}

%ch09

\newcommand{\cyrbulg}[1]{\begin{otherlanguage*}{bulgarian}#1\end{otherlanguage*}}


%ch10
\newcommand{\nlp}{{\small NLP}}
\newcommand{\mwe}{{\small MWE}}
\newcommand{\rae}{{\small RAE}}
\newcommand{\lvc}{{\small LVC}}
\newcommand{\pos}{{\small P}o{\small S}}
%\newcommand{\todo}[1]{ \textcolor{red}{#1} }

%\renewcommand{\labelenumi}{\theenumi}
%\ainamefmt{{vv}{ll}{, ff}{, jj}} % fullname

\newcommand{\biberror}[1]{{\color{red}#1}}

\newcommand{\osenovaitem}{--~}
   %% hyphenation points for line breaks
%% Normally, automatic hyphenation in LaTeX is very good
%% If a word is mis-hyphenated, add it to this file
%%
%% add information to TeX file before \begin{document} with:
%% %% hyphenation points for line breaks
%% Normally, automatic hyphenation in LaTeX is very good
%% If a word is mis-hyphenated, add it to this file
%%
%% add information to TeX file before \begin{document} with:
%% %% hyphenation points for line breaks
%% Normally, automatic hyphenation in LaTeX is very good
%% If a word is mis-hyphenated, add it to this file
%%
%% add information to TeX file before \begin{document} with:
%% \include{localhyphenation}
\hyphenation{
    Beck-man
    Ngu-yen
    back-chan-nel
    back-chan-nels
    mo-not-o-nous
    ste-reo-typ-i-cal
}

\hyphenation{
    Beck-man
    Ngu-yen
    back-chan-nel
    back-chan-nels
    mo-not-o-nous
    ste-reo-typ-i-cal
}

\hyphenation{
    Beck-man
    Ngu-yen
    back-chan-nel
    back-chan-nels
    mo-not-o-nous
    ste-reo-typ-i-cal
}

   \boolfalse{bookcompile}
   \togglepaper[7]%%chapternumber
}{}


\begin{document}
\emergencystretch 3em
\maketitle


\section{Introduction}\label{sec:vc:1}

In the present chapter, I deal with ancient Greek support-verb constructions (SVCs
henceforth) in diachrony, focusing specifically on an extensive corpus of hagiographical
literature\is{hagiography}.\footnote{The dataset is accessible here: \url{http://dx.doi.org/10.5287/ora-n652gamyj}.} In the case of verbo-nominal
collocations, a basic distinction is generally accepted between functional collocations
(also called SVCs) and lexical collocations
\parencites[78]{KoikeKazumi-20011}[5]{BanosJoséMiguel-2014280}. In the former type of
collocation (e.g. \emph{take a walk}), the nominal base of the collocation is an abstract
noun that usually nominalises an event and therefore has its own argument structure; in
the latter (e.g. \emph{play the piano}), although the verb also has a figurative sense
(\emph{to play} here means \textit{to perform with the piano}), the base is a concrete noun.

With a few recent exceptions
\parencites{FendelVictoria-2021854,FendelVictoria-2023591,FendelVictoria-202382,VivesCuestaAlfonsoandMadrigalAceroLucía-2022404}
the diachronic examination of Ancient Greek SVCs remains a rather unexplored field of
study \parencite{Banos-2022595}. What I consider innovative in my approach to the topic is
the incorporation of historical sociolinguistics\is{sociolinguistics}, something I consider of paramount
importance in the linguistic approach to the study of post-classical Greek and Byzantine
learned literature.

To understand the synchronic and diachronic variability of SVCs inherent in the
development of Greek during the Byzantine millennium, we must start from the
sociolinguistic situation of \emph{diglossia} \parencite{ToufexisN-2008293}\is{diglossia}. In dealing
with it, most authors tend to speak of levels of \emph{style}, following
\citet{SevcenkoIgor-1981832}'s seminal article. However, there are reasons to believe that
the rewriting goes beyond a question of style and again involves changes in \emph{levels
  of speech} \parencites{HinterbergerMartin-2010837,HinterbergerMartin-202164}. It is
therefore closer to the definition of sociolinguistic\is{sociolinguistics} terms, such as sociolect or
diastratic variant.\footnote{Other authors, such as \citet{MarkopoulosTheodore-2009295},
  prefer to use the concept of register, which I believe does not do justice to the
  largely mimetic situation, resulting from a process of rewriting, which our texts
  present. As in \citet{VivesCuestaAlfonsoandMadrigalAceroLucía-2022404}, I have opted for
  the term \emph{levels of speech}\is{level of speech}, knowing that it competes with other terms such as
  \emph{register}, \emph{style} or even \emph{variation}. Style tends to refer to literary
  or rhetorical variation, while variation is too vague a term to comprehend the
  linguistic reality I deal with. However, the strictly linguistic characterisation
  of many phenomena invites us to opt for this terminology.}

A key issue that highlights the issues with defining levels of speech\is{level of speech} in diachrony
concerns linguistic variation \parencite{BenteinKlaas-2017395}. In the study of the social
mechanisms that govern linguistic change, studies applied to oral variants have been
remarkably predominant. However, based on the work of \citet[122]{RomaineSuzanne-1982785}
it can be argued that the socio-historical approach she develops is applicable to written
texts such as those under study here and, and on the other hand, they are reference texts for
understanding the development of many linguistic changes in Postclassical Greek, as Klaas
Bentein has shown in several studies
\parencites{BenteinKlaas-2017395,BenteinKlaas-201979,BenteinKlaas-2020124}.

The concept of levels of speech\is{level of speech} is used in the field of Byzantine studies to distinguish
linguistic variants that occur for sociolinguistic\is{sociolinguistics} reasons. At the heart of any study of
post-classical Greek is the question of which variants were in use and which were borrowed
from the learned language. Two or three levels can be distinguished in scholarly Greek,
depending on various variables. These levels are not airtight and were used creatively by
Byzantine authors \parencite{HinterbergerMartin-2014130}. To these levels of the learned
language, we must add the vernacular, which undoubtedly has a greater influence on the
lower registers of cultivated Greek. The identification of these different levels, which
interact with each other, is complex. The situation is further complicated by the lack of
a common terminology to define them.

Here, I make a distinction between `low' and `high' (\emph{koine}) levels and reserve the
term Atticism for cases where there is direct continuity with syntactic usages
attested in Classical Greek (CG henceforth) or New Testament Greek (NTG henceforth).
Recognising this general trend in the description of the New-Testament (NT henceforth) language does not necessarily
imply that all NT authors adopt the Atticist style in all its aspects. There are factors
such as free stylistic choices and bilingual interference due to the multilingual context
in the writings of the Gospels that should be considered in the study of each collocation
\parencites{BanosJoséMiguel-2015534,BanosJoséMiguelandMaríaDoloresJiménezLópezLópez-2017484}.\footnote{In
  a forthcoming paper, Baños and Jiménez López demonstrate the variability in the
  selection of SVCs when translating different collocations from the language of the
  \emph{Septuagint} version involving the noun καρπός \emph{karpos} (καρπόν φέρω
  \emph{karpon pʰero} `to bear fruit', δίδωμι \emph{didomi} `to give' or
  ποιέομαι \emph{poieomai} `to make'). The selection shows, on the one hand, the
  idiosyncratic character of this type of complex predicate and, on the other hand, how
  the literal translation of sacred texts becomes a means of creating new collocations in
  Greek, as well as semitisms that find continuity in the Gospels and form the basis
  for the lexical creation of new collocations through literary \emph{imitatio} operating
  in the genre of hagiography.}

Attempts to characterise sociolinguistic\is{sociolinguistics} variation in hagiographic texts\is{hagiography} have been rare.
The few that are available have focused on the comparison of different versions of the
same \emph{Vita} and on stylistic rather than linguistic aspects
\parencites{ZilliacusHerbert-1938377,SchifferElisabeth-1992532,SchifferElisabeth-1999709,FrancoLaura-2009510}.
To date, with the sole exception of \citet{ChurikNikolas-2019727}, we have not found a
reference that relates the functioning of SVCs to different levels of speech\is{level of speech} in Byzantine
Greek.

The kind of variation in diachrony which we are talking about has an important linguistic
exponent in the use of SVCs in contrastive contexts, such as those presented in passages like (\ref{ex:vc:1}),
where there is an alternation between the synthetic (ἔρχεσθαι πυκνά
\emph{erkʰes$t^h$ai pykna} `to go frequently') and  analytic forms
(ποιεῖσθαι τὰς προσευλεύσεις πυκνάς \emph{poieistʰai tas proseleuseis}
`to make frequent visits').

% 1 (a-b)

\ea\label{ex:vc:1}

\ea\label{ex:vc:1a}

\glll καὶ ἀπολύ-σας τοὺς γον-εῖς αὐτ-οῦ μετὰ εὐλογι-ῶν παρ-ή-γγειλ-εν μὴ \textbf{πυκνὰ} \textbf{ἔρχ-εσθαι} πρὸς αὐτ-όν\\
 \textit{kai} \textit{apoly-sas} \textit{t-us} \textit{gon-eis} \textit{aut-u} \textit{meta} \textit{eulogi-on} \textit{par-e-ngeil-en} \textit{me} \textit{pykn-a} \textit{erkʰ-estʰai} \textit{pros} \textit{aut-on}\\
and dismiss\textsc{-ptcp-nom.sg} the parents-\textsc{acc.pl} he-\textsc{gen.sg} with blessing-\textsc{gen.pl} next\textsc{-pst}.exhort.\textsc{-aor}-3\textsc{sg} \textsc{neg} often come-\textsc{inf} to he-\textsc{acc.sg}\\
\glt `And, bidding the parents farewell and blessing them, he asked them not \textbf{to visit him often}' \\
\hspace*{\fill}(\iwi{\emph{Vita antiquior Sancti Danielis Stylitae}~5.16})

\ex\label{ex:vc:1b}

\glll ἐντειλά-μενος δὲ τ-οῖς αὐτ-οῦ πατρ-άσιν ὁ τ-ῆς μον-ῆς προεστ-ὼς μὴ \textbf{πυκνὰς}
\textbf{ποιεῖσθαι} πρὸς τ-ὸν παῖδ-α τ-ὰς \textbf{προσελεύσ-εις,} χαίρ-οντας ἐκπέμπ-ει γον-εῖς τὸ καινό-τατον υἱοῦ στερο-μένους\\
 \textit{enteila-menos} \textit{de} \textit{t-ois} \textit{aut-u} \textit{patr-asin} \textit{ʰo} \textit{t-es} \textit{mon-es} \textit{proest-os} \textit{me} \textit{pykn-as} \textit{poiei-stʰai} \textit{pros} \textit{t-on} \textit{paid-a} \textit{t-as} \textit{proseleus-eis} \textit{$k^h$air-ontas}
\textit{ekpemp-ei} \textit{gon-eis} \textit{to} \textit{kaino-taton} \textit{ʰyi-u} \textit{stero-menus}\\
command.\textsc{ptcp} \textsc{prt} the.\textsc{dat} he-\textsc{gen.sg} parents-\textsc{dat.pl} The.\textsc{nom.sg} the-\textsc{gen.sg} monastery-\textsc{gen.sg} abbot\textsc{-ptcp-nom} \textsc{neg} frequent-\textsc{acc.pl} make-\textsc{inf} to the child-\textsc{acc.sg} the.\textsc{acc.pl} visit-\textsc{acc.pl} rejoice-\textsc{ptcp-acc.pl} send-\textsc{pres-3sg-act} parent-\textsc{acc.pl} the new-\textsc{sprl.nom.sg}
son-\textsc{gen.sg} leave-\textsc{ptcp-acc.pl}\\
\glt `The abbot of the monastery, asking the parents not \textbf{to make frequent visits} to the child, bids the parents, who are happy in most strange a way, since they were losing a son' \\
\hspace*{\fill}(\iwi{\emph{Vita sancti Symeonis Stylitae 5.23}}) \\

\z

\z

In what follows, I first present my own definition of the concept of SVC (\sectref{sec:vc:2}),
which follows that of the Spanish research projects led by Baños and Jiménez López
respectively (\emph{DiCoLat} \& \emph{DiCoGra}).\footnote{I am honoured to be involved in
  this Spanish project (\emph{Interacción del léxico y la sintaxis en griego antiguo y
    latín 2: Diccionario de Colocaciones Latinas. DiCoLat} y \emph{Diccionario de
    Colocaciones del Griego Antiguo. DiCoGrA}) which has developed extensive databases on
  Latin (\url{https://dicolat.iatext.ulpgc.es/}) and Greek collocations
  (\url{https://dicogra.iatext.ulpgc.es/}).} I then provide a brief overview of the corpus
compiled for my survey (\sectref{sec:vc:3}), the methodology used for the analysis
(\sectref{sec:vc:4}), several types of SVCs with motion nouns (\sectref{sec:vc:5:1}), an overview of
support-verb-extension constructions\is{support-verb-extension construction} (SVECs henceforth) (\sectref{sec:vc:5:2}), and edge cases
represented by verbs of realisation (\sectref{sec:vc:5:3}). Finally, I summarise my
conclusions (\sectref{sec:vc:6}).

\section{Definition of support-verb constructions}\label{sec:vc:2}

SVCs are considered a special kind of verbo-nominal collocations that are situated at the
interface between syntax and semantics.\footnote{The use of the term \emph{light verb}
  instead of \textit{support verb} continues to dominate the literature \citep{PompeiPiunnointro2023}. It focusses on the loss of semantic weight of the verb. The term \textit{light-verb construction}
  is widely used in language-contact studies
  \parencites{Myers-ScottonCarol-2002877,FendelVictoria-2021854,FendelVictoria-2023591}.
  This paper uses the term \emph{support verb}. We believe that this term has important
  theoretical advantages in semantic terms, but syntactically it may be too restrictive,
  as it reduces the descriptive scope to verbo-nominal collocations with the noun base as
  the direct object. \citet{KälviäinenNikolaos-2013215} carries out a statistical study in
  which he demonstrates the tendency for syntactic constructions to become increasingly
  complex in an irregular manner over the course of the Byzantine millennium.} Lexically, they are
considered verbal multi-word expressions, since support verbs are form-identical with the lexical
form of a verb when lexical and auxiliary forms coexist \parencite{BenteinKlaas-2013367}.
Lexical features of the components of the construction are its discontinuity, variability
\parencite{BooijGeert-2014823}, and ambiguity
\parencite[50]{SheinfuxLivnatandGreshlerTaliandMelnikNuritandWinterShuly-2019145}. SVs are
limited in their combinations and variability. Concrete examples of SVCs show their
untranslatable and language-specific character. For example, the same activity of `giving
a lecture' is expressed with different SVs in different languages: \emph{Elle \textbf{fait
    une présentation}} (French), \emph{Sie \textbf{hält eine Vorlesung}} (German) or
\emph{\textbf{está} \textbf{dando una conferencia}} (Spanish).

Syntactically, SVCs are complex predicates that typically (but not exclusively) take the
form of combinations of a verb and a predicative noun that fill the predicative frame of
an SV as ποιέω \emph{poieō} `to make' or δίδωμι \emph{didōmi} `to give', see
\xxref{ex:vc:2a}{ex:vc:2b}, both of which are exemplified here with the polysemous and
high-frequency noun λόγος \emph{logos} `word'
\parencite{VivesCuestaAlfonso-2021316}.\footnote{Synchronically, the syntactic status of
  collocations is ambiguous and may allow for a double analysis, according to whether the
  dependency is on the SV nucleus or on the predicative noun%
%   (\cite{JiménezLópezMaríaDolores-2024155})
.}

% 2 (a-b)

\ea\label{ex:vc:2}

\ea\label{ex:vc:2a}

\glll ὁ δὲ Σιτάλκ-ης \textbf{πρός} \textbf{τε} \textbf{τ-ὸν} \textbf{Περδίκκ-αν} \textbf{λόγους} \textbf{ἐποιεῖ-το}\\
 \textit{ʰo} \textit{de} \textit{Sitalk-es} \textit{pros} \textit{te} \textit{t-on} \textit{Perdikk-an} \textit{log-us} \textit{e-poiei-to}\\
the \textsc{prt} Sitalces.\textsc{nom.sg} to and the Perdikkas-\textsc{acc.sg} words-\textsc{acc.pl} \textsc{Pst.}made.\textsc{imp.3g.mid}\\
\glt `Sitalces spoke to Perdicas' \\
\hspace*{\fill}(\iwi{Thucydides, \textit{Histories} 2.101.2})

\ex\label{ex:vc:2b}

\glll τοῦτο δὲ ἀκού-σαντ-ες οἱ Ἕλλην-ες~ \textbf{λόγ-ον} \textbf{σφί-σι} \textbf{αὐτ-οῖσι} \textbf{ἐδίδο-σαν}\\
 \textit{tuto} \textit{de} \textit{aku-sant-es} \textit{ʰoi} \textit{$^h$ellen-es} \textit{log-on} \textit{sphi-si} \textit{aut-oisi} \textit{edido-san}\\
this \textsc{prt} Hearing-\textsc{ptcpl-nom.pl} the-\textsc{nom.pl}
Greeks-\textsc{nom.pl} word-\textsc{acc.sg} to.the-\textsc{dat.pl} to.them\textsc{-dat.pl} \textsc{pst-}gave-3\textsc{pl.act}\\
\glt `Upon hearing this, the Greeks \textbf{exchanged their arguments} among themselves' \\
\hspace*{\fill}(\iwi{Herodotus, \emph{Histories} 8.9.1})

\z

\z

The definition of SVCs is fraught with theoretical problems. Within the batteries of tests
used to identify SVCs \parencite{LangerStefan-2004588}, one that stands out is the
co-referentiality between the subject of the SV and the first argument of the predicative
noun, which always tends towards \emph{monoclausality} \citep{ButtMiriam-2010672}. In this respect, the
application of criteria commonly used to describe SVCs cross-linguistically has also
proved relevant in the analysis of Ancient Greek SVCs
\parencite{JiménezLópezMaríaDolores-2016731}: (a) the equivalence with a simplex verb; (b)
the reduction of SVCs to noun phrases; (c) the co-referentiality of the subject of the
verb and the first argument of the SVC noun; (d) noun variability, etc. From a
sociolinguistic\is{sociolinguistics} perspective, principle (a) should not be considered applicable, since the
simple and multi-word constructions can in no case incur the redundancy of being
considered pure synonyms. It is more accurate to think in terms of \emph{reallocation} or
\emph{nuancing} from a diachronic and variationist perspective.

To my knowledge, the most comprehensive theoretical introductions to the treatment of SVCs
applied to classical languages are \citet{Banos-2022595} and
\citet{PompeiAnnaandMereuLunellaandPiunnoValentina-2023608}, which provide an exhaustive
state of the art. After the first seminal approach by \citet{JespersenOtto-1942107}, the
first solid definition was given by \citet{PolenzvonPeter-1963634}, who defined the
verbs in question as \emph{Funktionsverben}. In all these treatments of the problem, the distinction
between SVCs and other periphrastic constructions dominates. In the context of the
\emph{Lexique-Grammaire} theory and the \emph{Laboratoire d'Automatique Documentaire et
  Linguistique} (LADL),
\citet{GrossGaston-1989673,GrossGaston-1996727,GrossGaston-2004226} developed an automated
database model that makes it possible to describe the syntactic properties of SVCs.
According to all these perspectives, the verb of an SVC only actualises the predicative
noun. On the other hand, the Meaning-Text Theory and its formalisation resource, i.e.,\largerpage
Lexical Functions \parencites{MelcukIgor-2004540,AlonsoRamosMargarita-2004672}, present a
type of analysis based on the collocational pattern and the selection of collocations
which consist of a predicative noun (the base) selecting a semantically empty verb
(collocative).

As far as our \emph{DiCoGra} research project is concerned, the proposal I apply to the
corpus is theoretically eclectic, although it is mainly dominated by the postulates of the
\emph{Lexique-Grammaire} and \textit{Lexical-Functional Grammar} (LFG henceforth) theories \parencite{Banos-2022595}.

In light of these theoretical developments, I propose the following definition of
SVCs:

\begin{quote}
A semi-compositional construction formed by a predicative noun dependent
on a semantically bleached verb, which is joined to the construction to
form a multi-word phrase. It is sometimes equivalent to a simplex verb.
\end{quote}

This definition corresponds to the function of these verbs, which act as an auxiliary or
syntactic support for the noun with which they are constructed, forming a specific type of
collocation. The verb has a very light semantic content and expresses the time, manner,
and aspect\is{aspect} of the event as a whole; the noun, which lexically selects the verb and is
usually presented as its direct object (DO henceforth), provides the arguments
(predicative frame) of the construction.

In addition to these functional SVCs with a genuine SV (ποιέομαι \emph{poieōmai}, ἔχω
\emph{ekʰo}, γίγνομαι \emph{gignomai}, δίδωμι \emph{didomi}, τίθημι
\emph{titʰemi} etc.), languages have several heavier verbs called
support-verb extensions\is{support-verb-extension construction} (SVEs henceforth) that convey an aspectual\is{aspect} or diathetic\is{diathesis} meaning
\parencites{VivesRobert-198467,GrossGaston-1989673,BanosJoséMiguel-2014280,BanosJoséMiguelandMaríaDoloresJiménezLópez-2018477}.
The range of SVs is language-specific, so that the mere existence of such SVECs shows the
diffuse character of the consideration of an SV as a concept.

From CG onwards, some verbs that preserve much of their lexical content can metaphorically\is{metaphor}
express diathetic\is{diathesis} (δέχομαι \emph{dekʰomai} `to accept') or aspectual\is{aspect}
(ἅπτομαι \emph{ʰaptomai} `to touch') content, see \xref{ex:vc:3}.

% 3 (a-b)

\ea\label{ex:vc:3}

\ea\label{ex:vc:3a}

\glll τ-ὸν μὲν τ-ῶν χρημάτ-ων \textbf{λόγ-ον} παρὰ τούτ-ων \textbf{λαμβάν-ειν}\\
 \textit{t-on} \textit{men} \textit{t-on} \textit{kʰremat-on} \textit{log-on} \textit{para} \textit{tut-on} \textit{lamban-ein}\\
the-\textsc{acc.sg} \textsc{prt} the-\textsc{gen.pl} money-\textsc{gen.pl} account-\textsc{acc.sg} from them-\textsc{gen.pl} take-\textsc{inf}\\
\glt `You must \textbf{demand} from your paymasters \textbf{an account} of their money' \\
\hspace*{\fill}(\iwi{Demosthenes, \textit{Speech} 8.47})

\ex\label{ex:vc:3b}

\glll καὶ ἅμα \textbf{λόγ-οι} πρὸς Λακεδαιμονί-ους περὶ τ-ῆς εἰρήν-ης \textbf{ἐ-γίγνο-ντο}\\
 \textit{kai} \textit{$^h$ama} \textit{log-oi} \textit{pros} \textit{Lakedaimoni-us} \textit{peri} \textit{t-es} \textit{eiren-es} \textit{e-gigno-nto}\\
and together words-\textsc{nom.pl} to Lacedaemonians-\textsc{acc.pl} about
the-\textsc{gen.sg} peace-\textsc{gen.sg} \textsc{pst.}be\textsc{-imp-3pl-mid}\\
\glt `And \textbf{negotiations} for peace \textbf{happened} at once with the Lacedaemonians' \\
\hspace*{\fill}(\iwi{Lysias, \textit{Speech} 13.5}) \parencite[231]{JiménezLópezMaríaDolores-2021150}

\z

\z


Linguistic change is expected to create semantic mechanisms of lexical
innovation (conceptual metaphors\is{metaphor} and metonymies) in the domain of SVECs\is{support-verb-extension construction}.

\section{Description of the dataset}\label{sec:vc:3}

As for the quantitative data of our corpus, we have also worked with the aim of studying
the chronological evolution of a broad literary genre ‒ Byzantine hagiography\is{hagiography} ‒ and the
inherent variations between versions of the same hagiographical text in its diachronic
evolution. Byzantine hagiography covers an entire literary spectrum. This makes it a
testing ground for the study of all kinds of diachronic variability
\parencite{BenteinKlaasJanse2021}.

According to \citet{BenteinKlaas-2013879}, in terms of level of speech\is{level of speech}, Byzantine
hagiographical literature is composed in a wide variety of registers, but always with the
avoidance of the most Attic styles. However, this statement must be qualified to some
extent, since the hagiographic texts of this period (4th to 14th centuries AD) and especially
during the 9th century can be classified as belonging to the high style
\parencite{SevcenkoIgor-1981832}. Through linguistic analysis of the texts, we have been
able to establish a clear picture of the sociolinguistic\is{sociolinguistics} development of the linguistic
style of Byzantine hagiography\is{hagiography}. There is an early period in which simpler, low-level
hagiographical texts were written alongside more rhetorically elaborate ones. In the
middle and even late Byzantine period, this would give way to a much larger proportion of
high-level \emph{Vitae}, often the product of rewriting, technically called
\emph{metaphrases} \parencite{HinterbergerMartin-2010837,HinterbergerMartin-2014621}.\is{metaphrasis}

As far as the chronology is concerned, because it is such a long period of time, I have
divided the corpus into four sub-periods which are related to the lifespan of
hagiographical literature in the Byzantine world:

\begin{enumerate}
\def\labelenumi{(\roman{enumi})}
\item \emph{New Testament Greek} (1\textsuperscript{st} century AD). According to
  \citet[61]{RicoChristopher-201078}, the NT is representative of a low koine (vernacular)
  language that was in contact with Semitic languages (Aramaic and Hebrew). However,
  traces of Atticism can also be found in the language of the NT.
\item
  \emph{Proto- and Mesobyzantine Greek}
  (4\textsuperscript{th}-9\textsuperscript{th} centuries AD). The
  hagiographic texts of this period (at least those of the first half)
  tend to be more classicising than the metaphrastic corpus, although we
  can also find some texts of a simpler style.
\item \emph{Metaphrastic hagiography} (10\textsuperscript{th}--11\textsuperscript{th}
  centuries AD): Under the label \emph{metaphrastic hagiography} there is room for a
  rewriting of texts to be understood as a synchronic intralingual translation (μετάφρασις
  \emph{metapʰrasis}) of the ancient versions of the same \emph{Vita.}
  Symeon Metaphrastes' rewriting technique consists essentially of making lexical and
  syntactic changes to introduce modifications at the level of language with respect to
  the older versions of the \emph{Vitae} and to establish a canonical text of reference
  for these works \parencites{HogelChristian-2002181,HøgelChristian-2021320}. Precisely,
  for this special literary status, the five \emph{Vitae} of the
  \emph{metaphrastic} period play a special role with regard to the variation of SVCs as markers
  of levels of speech\is{level of speech}.
\item \emph{Greek of the Comnene and Late Byzantine periods} (12th-14th centuries AD).
  Although the style of the hagiography of the Palaeologan period already shows certain
  demotic tendencies, it maintains the same high stylistic standards that characterise the
  canonisation of the work of the Metaphrastes
  \parencites{HinterbergerMartin-2014130,HinterbergerMartin-202164}.
\end{enumerate}


In accordance with this periodisation, all the works that have been collected in our
representative selection of the corpus are shown in table~\ref{tbl:vc:1}.\footnote{With
  slight modifications, this is the corpus of a Masters that I supervised
  \parencite{MadrigalAceroLucía-2022164}, and it also largely coincides with that of
  previous work \parencite[318--321]{VivesCuestaAlfonsoandMadrigalAceroLucía-2022404}. Not
  all the data are at the same descriptive level. In our dataset, we make a distinction
  between the main corpus and the control or reference corpus. In each of the selected
  periods, the texts are not necessarily grouped in chronological order. Links to other
  versions that rewrite earlier versions of the texts have conditioned the selection.}


% Table 1

\begin{table}[p]
  \caption{Corpus and abbreviations}
  \label{tbl:vc:1}
\small
\begin{tabularx}{\textwidth}{>{\raggedright\arraybackslash}X>{\raggedright\arraybackslash}X}
\lsptoprule
New Testament & \emph{Evangelium secundum Matthaeum}\\
\midrule
 & \emph{Evangelium secundum Lucam}\\\cmidrule(rl){2-2}
 & \emph{Epistula Pauli ad Corinthios i}\\\cmidrule(rl){2-2}
 & \emph{Epistula Pauli ad Corinthios ii}\\\cmidrule(rl){2-2}
 & \emph{Epistula Pauli ad Hebraeos}\\
\midrule
Proto- and Meso-byzantine hagiography & \emph{Vita antiquior Sancti Danielis Stylitae (BHG 489)}\\\cmidrule(rl){2-2}
 & \emph{Vita et martyrium sancti Anastasii Persae (BHG 84)}\\\cmidrule(rl){2-2}
 & \emph{Martyrium antiquior sanctae Euphemiae (BHG 619)}\\\cmidrule(rl){2-2}
 & \emph{Vita Stephani Iunioris (BHG 1666)}\\\cmidrule(rl){2-2}
 & \emph{Vita Symeonis Stylitae senioris (BHG 1683)}\\
\midrule
Metaphrastic hagiography & \emph{Passio sancti Anastasii Persae (BHG 85)}\\\cmidrule(rl){2-2}
 & \emph{Passio sanctae Euphemiae (BHG 620)}\\\cmidrule(rl){2-2}
 & \emph{Vita tertia Sancti Danielis Stylitae (BHG 490)}\\\cmidrule(rl){2-2}
 & \emph{Vita Stephani Iunioris (BHG 1667)}\\\cmidrule(rl){2-2}
 & \emph{Vita sancti Symeonis Stylitae (BHG 1686)}\\
\midrule
Comnene and Late Byzantine hagiography & \emph{Vita sancti Zotici (BHG 2480)}\\\cmidrule(rl){2-2}
 & \emph{Vita Leontii Patriarchae Hierosolymorum (BHG 985)}\\\cmidrule(rl){2-2}
 & \emph{Vita sancti Bartolomaei conditoris monasterii sancti Salvatoris Messanae (BHG 235)}\\\cmidrule(rl){2-2}
 & \emph{Miracula sancti apostoli Marci (BHG 1036m)}\\\cmidrule(rl){2-2}
 & \emph{Vita sancti Lazari (BHG 980)}\\
\lspbottomrule
\end{tabularx}
\end{table}
%can we force the start of this section underneath the table please, the bit of whitespace at the bottom of the page does not bother me
\section{Methodology}\label{sec:vc:4}

My practical methodology is the identification of the most frequent predicative nouns
(\emph{collocative pattern}) of ποιέω/ποιέομαι \emph{poieo}/\emph{poieomai} `to make' in
the corpus. The selection has been restricted to this verb precisely (a)~because of its
prototypical character in this type of construction; (b)~because of its very high
frequency of use in our corpus, which means that it offers a sufficiently representative
and comprehensive amount of data for our analysis; and (c) because of its syntactic
variability, represented by a wide range of constructions that show diachronic variation
and that do not occur with other support verbs.

In the selection, the nominal base is given priority, since in SVCs the meaning of a
general verb is specified by the meaning of the noun with which it interacts at the
syntagmatic level \parencite[29]{JezekElisabetta-201155}. In the analysis of our data, we
have chosen to include a broad notion of predicative noun, which includes all types of
predicative nouns that function as DO of ποιέω \emph{poieō}, and not only the \emph{nomina
  actionis} traditionally considered
\parencite{GarzónFontalvoEvelynandCristinaTur-2022234}. The SVCs already inventoried in
previous studies of the NT
\parencites{BanosJoséMiguelandMaríaDoloresJiménezLópezLópez-2017484,BanosJoséMiguelandMaríaDoloresJiménezLópezLópez-2017484}
are considered to be more sensitive to the type of semantic or syntactic variation that
this construction involves in the corpus, since many of the Saints' lives
reproduce traditional NT linguistic forms as their main intertextual source.

For CG, some authors \parencites{JiménezLópezMaríaDolores-2016731,FendelVictoria-2023591}
have proposed, with almost the same conclusions, an inventory of the most statistically
frequent SVs.\footnote{\citet{FendelVictoria-2023591}, for literary classical Attic,
  offers the most comprehensive set of verbs available, including the following verbs,
  some of which have already been the subject of monographs: ἄγω~\emph{ago} `to pass /
  spend', δέχομαι \emph{dekʰomai}
  `to receive', δίδωμι \emph{didomi} `to give', ἔχω~\emph{ekʰo} `to
  have', κομίζω~\emph{komidzo} `to give / receive', κτάομαι \emph{ktaomai}~`to gain',
  λαμβάνω \emph{lambano}~`to take / receive', παρέχω~\emph{parekʰo} `to
  give', πάσχω~\emph{paskʰo} `to suffer', ποιέομαι~\emph{poieomai} `to
  make', τίθημι \emph{titʰemi}~`to put',
  τυγχάνω~\emph{tynkʰano} `to get', φέρω \emph{pʰero}
  `to bring', χράομαι~\emph{kʰraomai} `to use'. We add γίγνομαι \emph{gignomai} `to become'
  \parencite{JiménezLópezMaríaDolores-2021150}.} In the dataset, I present the collocational patterns of ποιέω \emph{poieo}
formed by all the predicative nouns with which it is combined to form SVCs, as well as
quantitative information.

In total, I analysed 614 examples of ποιέω/ποιέομαι \emph{poieo}/\emph{poieōmai} + DO in
the main corpus. Of these, 211 (34.36 \%) used ποιέω/ποιέομαι \emph{poieo}/\emph{poieōmai}
as a candidate SV. 
The high distributional frequency of ποιέω/ποιέομαι \emph{poieo}/\emph{poieōmai} in the corpus as the main support verb is a key factor in considering the SVCs we analyse.
%This is a key issue and may be a condition for the identification of constructions in the data consulted but not all the cases listed are of the same nature. 
One of the effects of the high combinatorial frequency of two different lexical
items is the tendency for them to form sub-groups. The combinatorial freedom of items is
traditionally translated into the notion of ``collocational frequency''
\parencite{FendelVictoria-202382}. This phenomenon has consequences at the cognitive
level, in the way speakers process them mentally (\emph{analysability}), and at the level
of discourse cohesion (\emph{compositionality}). Indeed, constructions with ποιέω
\emph{poieo} `to make' tend to be productive and semantically compositional, so that
lexicalisation\is{lexicalisation} and other types of variation seem \textit{a priori}
unlikely.\footnote{\citet{KyriasoupoulouTitaandSfetsiouVasso-2003241} confirm that the verb
  κάνω \emph{kano} `to do' is still the most common collocative in Modern Greek SVCs.}
\largerpage
Finally, the study of the variability and discontinuity represented by these constructions
cannot be understood without recourse to the potential SVECs\is{support-verb-extension construction} attested with some
predicative nouns. The creation of new constructions or the appearance of metaphorical or
metonymic values associated with them demonstrates the productivity of the category and at
the same time constitutes a resource for creations at the different levels of speech\is{level of speech}, such
as dialectal variants of the \emph{to take a shower} / \emph{to have a shower} type
\parencite{OzbayAliSukru-2020683}.

\section{Types of support-verb constructions}\label{sec:vc:5}

The three case studies below have been selected to illustrate the diachronic variability
of SVCs in post-classical Greek. As for the most common noun bases in our corpus, I study
motion nouns (\sectref{sec:vc:5:1}), constructions with ποιέω \emph{poieo} `to make'
as a verb of realisation\is{verb
of realisation} (\sectref{sec:vc:5:2}), and finally a special type of SVECs\is{support-verb-extension construction}
expressing metaphorical content (\sectref{sec:vc:5:3}).

\subsection{Support-verb constructions with motion nouns}\label{sec:vc:5:1}

The type of nouns that ποιέω/ποιέομαι \emph{poieo/poieōmai} takes in my corpus are nouns
of motion. Indeed, this kind of collocation was also very widespread in the classical
period \citep{DePasqualeNoemi-2023335}. Examples \xxref{ex:vc:4a}{ex:vc:4b} are prototypical SVCs,
while \xref{ex:vc:5c} below is what is usually called an SVEC
\parencites{VivesRobert-198467,GrossGaston-1989673,BanosJoséMiguel-2014280}.

% 4

\ea\label{ex:vc:4}

\ea\label{ex:vc:4a}

\glll καὶ δι-ε-πορεύ-ετο κατὰ πόλ-εις καὶ κώμ-ας διδάσκ-ων καὶ \textbf{πορεί-αν} \textbf{ποιού-μεν-ος} εἰς Ἱεροσόλυμα\\
 \textit{kai} \textit{di-e-poreu-eto} \textit{kata} \textit{pol-eis} \textit{kai} \textit{kom-as} \textit{didask-on} \textit{kai} \textit{porei-an} \textit{poiu-men-os} \textit{eis} \textit{Hierosolyma}\\
and through-\textsc{pst-}crossed-3\textsc{sg-mid} through city-\textsc{acc.pl} and
village-\textsc{acc.pl} teach-\textsc{ptcp-nom.sg} and way-\textsc{acc.sg} make-\textsc{ptcp.mid}-\textsc{nom.sg} to Jerusalem\\
\newpage
\glt `And he passed through cities and villages teaching and \textbf{travelling} towards Jerusalem' \\
\hspace*{\fill}(\iwi{\emph{Evangelium secundum Lucam} 13.22})

\ex\label{ex:vc:4b}

\glll ἀποστέλλ-ει αὐτ-οὺς πρὸς τ-ὸν\ldots{} ἀρχιποιμέν-α τ-οῦ σὺν αὐτ-οῖς \textbf{ποιῆ-σαι} αὐτ-ὸν \textbf{τ-ὴν} \textbf{πορεί-αν} πρὸς τὸ\ldots{} μοναστήρι-ον\\
 \textit{apostell-ei} \textit{aut-us} \textit{pros} \textit{t-on} \textit{arkʰipoimen-a} \textit{t-u} \textit{syn} \textit{aut-ois} \textit{poie-sai} \textit{aut-on} \textit{t-en} \textit{porei-an} \textit{pros} \textit{to} \textit{monasteri-on}\\
send-\textsc{pres-3sg-act} they-\textsc{acc.pl} to the-\textsc{acc.sg} patriarch-\textsc{acc.sg} the-\textsc{gen.sg} with they-\textsc{dat.pl} make-\textsc{inf} he-\textsc{acc.sg} the-\textsc{acc.sg} way-\textsc{acc.sg} to the-\textsc{acc.sg}
monastery-\textsc{acc.sg}\\
\glt `He sends them to\ldots{} the patriarch, so that he would \textbf{make} with them \textbf{the journey} to the monastery' \\
\hspace*{\fill}(\iwi{\emph{Vita Stephani Iunioris} 42.12})

\z

\z

First, there is a diachronic continuity in their structure. SVCs with motion nouns
already show a prototypical character in CG, which is confirmed in our
corpus.\footnote{The motion nouns involved in SVCs expressing movement are derived from
  different verb classes that encode the main conceptual components of movement: basic
  motion verbs, caused motion verbs, manner verbs and Path + Manner verbs
  \parencite{DePasqualeNoemi-2023335}.} SVCs with motion nouns present a range of meanings
and functions, among which stylistic variation and the expression of connotative meanings
stand out \parencite{DePasqualeNoemi-2023335}. Connotative meanings tend to be associated
with a high level of speech\is{level of speech}, as they imply a reconceptualisation of the predicative noun,
precisely because they are part of an SVC.

However, as can be seen in \xxref{ex:vc:5b}{ex:vc:5c}, we observe the innovation of a type
of construction that occurs only very sporadically in CG.\footnote{For some motion nouns,
  such as ὁδός \emph{$^h$odos} `way' in ποιέω ὁδόν \emph{poieo $^h$odon} `marching' (\iwi{Herodotus,
  \emph{Histories} 1.211.1}), the loss of the diathetic\is{diathesis} distinction can be traced back to
  the beginning of the classical period \parencite{MariniEmmanuela-2010345}.} One of the
reasons for this syntactic variation in post-classical Greek is that, from the stage
represented by NT texts onwards, the progressive semantic bleaching and gradual decline of the
middle voice has affected the voice distinction between ποιέω/ποιέομαι
\emph{poieo}/\emph{poieōmai} in many SVCs, see \xxref{ex:vc:5a}{ex:vc:5c}. 

%moved here by Victoria example set 5
% 5

\ea\label{ex:vc:5}

\ea\label{ex:vc:5a}

\glll ἀλλ' ὁ \textbf{ποι-ῶν} τὸ \textbf{θέλη-μα} τ-οῦ πατρ-ός μου τοῦ ἐν τ-οῖς οὐραν-οῖς\\
 \textit{all'} \textit{ʰo} \textit{poi-on} \textit{to} \textit{$t^h$ele-ma} \textit{t-u} \textit{patr-os} \textit{m-u} \textit{t-u} \textit{en} \textit{t-ois} \textit{uran-ois}\\
but the-\textsc{nom} make-\textsc{ptcp.nom} the.\textsc{acc} will.\textsc{acc} the.\textsc{gen.sg} father-\textsc{gen} my-\textsc{gen.sg} the.\textsc{gen.sg} in
the\textsc{.dat.pl} heavens.\textsc{dat.pl}\\
\glt `But the one who \textbf{does the will} of my Father, who is in Heaven' \\
\hspace*{\fill}(\iwi{\emph{Evangelium secundum Matthaeum} 7.21})

\ex\label{ex:vc:5b}

\glll μήτηρ μου καὶ ἀδελφ-οί μου οὗτ-οί εἰ-σιν οἱ τ-ὸν \textbf{λόγ-ον} τ-οῦ θε-οῦ ἀκούοντ-ες καὶ \textbf{ποιοῦντ-ες}\\
 \textit{meter} \textit{m-u} \textit{kai} \textit{adelpʰ-oi} \textit{m-u} \textit{ʰut-oi} \textit{ei-sin}
\textit{ʰoi} \textit{t-on} \textit{log-on} \textit{t-u} \textit{$t^h$e-u} \textit{akuont-es} \textit{kai} \textit{poiunt-es}\\
mother my-\textsc{gen.sg} and brothers-\textsc{nom.pl} my-\textsc{gen.sg} these-\textsc{nom.pl} are-\textsc{3sg-atc} the.\textsc{nom.pl} the.\textsc{acc.sg} word-\textsc{acc.sg} the.\textsc{gen.sg} God-\textsc{gen.sg} hearing-\textsc{ptcp-nom.pl}
and doing\textsc{-ptcp-nom.pl}\\
\glt `My mother and my brothers are those who hear and \textbf{do God's word'} \\
\hspace*{\fill}(\iwi{\emph{Evangelium secundum Lucam} 8.21})

\ex\label{ex:vc:5c}

\glll τί λέγ-εις; \textbf{ποι-εῖς} τ-ὴν \textbf{κέλευσ-ιν} τ-οῦ βασιλ-έως ἢ ἐπιμέν-εις τ-οῖς αὐτ-οῖς;\\
 \textit{ti} \textit{leg-eis} \textit{poi-eis} \textit{t-en} \textit{keleus-in} \textit{t-u} \textit{basil-eos} \textit{e} \textit{epimen-eis} \textit{t-ois} \textit{aut-ois}\\
what say-\textsc{2sg-act} make-2\textsc{sg-act} the.\textsc{acc.sg} command-\textsc{acc.sg} the-\textsc{gen.sg} king-\textsc{gen.sg} or stay-\textsc{2sg-act} the.\textsc{dat.pl}
they-\textsc{dat.pl}\\
\glt `What do you say? Do you \textbf{do} the emperor's \textbf{command} or do you stay with them?' \\
\hspace*{\fill}(\iwi{\emph{Vita et martyrium sancti Anastasii Persa} 37.2})

\z

\z






One can hardly
observe a semantic contrast between the use of the active and middle voices, when
commenting on phrases such as ποιέω ἔκβασιν \emph{poieo ekbasin} `to escape'
(\iwi{\emph{Epistula Pauli ad Corinthios 1} 10.13}), ποιέω γάμους \emph{poieo gamus} `to make a
wedding feast' (\iwi{\emph{Evangelium secundum Matthaeum.} 22.2}) οr ποιέω δεῖπνον \emph{poieo
  deipnon} `to make supper' (\iwi{\emph{Evangelium secundum. Lucam} 14.16}).

%The ambiguity of these phrases makes it hard to distinguish between verbs or realization of and its use as a prototypical SV. 
The distinction between the uses of ποιέω \textit{poieo} as a verb of realisation\is{verb
of realisation} and its prototypical uses as a light verb are minimal or difficult to establish.
In my opinion, the general tendency towards analytic
constructions throughout the postclassical period may have contributed to the remarkable
increase in the use of SVCs
\parencites{HorrocksGeoffrey-2014183,HoltonDavidandManolesouIo-2010807}.\footnote{It is
  possible that the evolution of certain SVECs\is{support-verb-extension construction} expressing aspectual\is{aspect} or diathetic\is{diathesis} values
  follows a path partially parallel to that of certain auxiliary verbs that are
  constructed periphrastically such as θέλω, \emph{tʰelo} `to want',
  ἔχω, \emph{ekʰo} `to have', etc. in post-classical Greek
  \parencite{MarkopoulosTheodore-2009295}. However, we do not have enough data to speak in
  canonical terms of grammaticalisation \parencite{ButtMiriam-2010672}.} This kind of
choice, involving the selection of constructions appropriate to a learned register in
post-classical texts, is reminiscent of the stylistic tendency that
\citet{HorrocksGeoffrey-2020667} calls the ``creative use of syntax'', and which we find
especially in high-register Byzantine Greek. In fact, high-register Byzantine Greek was a
living language, used creatively by its practitioners, developing its own idiosyncrasies
and internal conventions in the process. It would not be inappropriate to compare it, for
example, with the highly specialised literary language of the early Greek Homeric
tradition, which retained many archaisms but allowed its authentic usage to evolve
alongside the constant incorporation of linguistic innovations inherent in the native
variants of each period.

Semi-lexicalised constructions\is{lexicalisation}, such as SVCs, are linguistic material in
which these evolutionary tendencies of the language can be observed most
clearly. The progressive blurring of the middle voice and the emergence
of SVCs with ποιέω \emph{poieo} `to make', as I have discussed, are
likely to have been additional factors to consider.

\subsection{Edge cases: verbs of realisation}\label{sec:vc:5:2}

In this section, I discuss some collocations with active ποιέω \emph{poieo} which,
although sometimes disregarded as not proper SVCs
\parencite[113--115]{AlonsoRamosMargarita-2004672}, have the syntactic behaviour of an SV
but, unlike prototypical SVs, are semantically complete.



As with SVECs\is{support-verb-extension construction}, they have certain combinatorial limitations. To some extent, the verbs of
realisation\is{verb
of realisation} project constructions that are midway between prototypical SVCs and SVECs.
However, whereas an SV simply reports the existence of the action denoted by the noun, a
verb of realisation indicates that the purpose for which the action exists has been
achieved \parencite[113--115]{AlonsoRamosMargarita-2004672}.\footnote{There is a real
  terminological issue with this type of verb. In addition to the more common term ``verbs
  of realisation'', the term can also be found in the literature as ``verbs of fulfillment''
  \parencite{MelcukIgor-2004540}.}

Unlike support verbs, which are semantically empty, realisation verbs\is{verb
of realisation} are full: roughly
speaking, they mean `to fulfil the requirement of something' and, like support verbs, they
produce collocations with their nominal bases. In their\linebreak syntactic-semantic
behaviour they are quite close to some of the SVECs\is{support-verb-extension construction} with diathetic\is{diathesis} or aspectual\is{aspect} functions
\parencite{MelcukIgor-2022555}. In my opinion, this semantic restriction is partly
aspectual, since the verb element implies a phase of the action after that of the SV and
the noun must therefore refer to a telic action \parencite{GrossMaurice-199846}. The absence of grammaticalisation of these constructions \citep{ButtMiriam-2010672} also explains why not all
the criteria for the formation of an SVEC are necessarily met, e.g. the non-strict
co-referentiality between noun and verb in \xref{ex:vc:5}.

\largerpage
We have identified borderline contexts that can lead to confusion as to whether the verb
is a true SVC, or a verb of realisation\is{verb
of realisation}, or even a causative verb. The canonical SVC with
the collocative ποιέομαι \emph{poieomai} + predicative noun is largely preserved and
reconstructed in the corpus of texts belonging to a high-level of speech\is{level of speech}, which, not by
chance, largely coincides with the metaphrastic versions\is{metaphrasis} of the \emph{Menologion} of
Symeon Metaphrastes and other late \emph{Vitae} of the Palaeologian era shown in
\xref{ex:vc:6}.\footnote{In situations of language contact, the metalanguage of
  cross-linguistic translation is expected to serve as a trigger for the creation of new
  SVCs
  \parencites{FendelVictoria-2021854,BanosJoséMiguelandMaríaDoloresJiménezLópez-2018477}.}

% 6

\ea\label{ex:vc:6}

\ea\label{ex:vc:6a}

\glll \ldots{} μηδέν-α \textbf{λόγ-ον} \textbf{ποιού-μεν-ος} τ-οῦ ταύτ-ας
ἀπωθεῖ-σθαι τολμῶ-ντ-ος αἱρεσιάρχ-ου βασιλ-έως\\
 ~ \textit{meden-a} \textit{log-o­n} \textit{poiu-men-os} \textit{t-u} \textit{taut-as} \textit{apo$t^h$ei-s$t^h$ai} \textit{tolmo-nt-os} \textit{$^h$airesiar$c^h$-u} \textit{basil-eos}\\
~ nobody-\textsc{acc.sg} word-\textsc{acc.sg} make-\textsc{ptcp-nom.sg} the-\textsc{gen.sg} these-\textsc{acc.pl} repel-\textsc{inf} dare-\textsc{ptcp-gen.sg}
heresiarch-\textsc{gen.sg} king-\textsc{gen.sg}\\
\glt `... without \textbf{paying attention} to the Emperor who dares to refuse them' \\
\hspace*{\fill}(\iwi{\emph{Vita Stephani Iunioris} 30.26})

\ex\label{ex:vc:6b}

\largerpage
\glll τ-ὸν δὲ κεκαρωμέν-ην \ldots, ἔχ-οντ-α τ-ὴν διάνοι-αν, \textbf{λόγ-ον} μὲν μηδέν-α τ-ῶν ἐκείν-ου \textbf{λόγ-ων} \textbf{ποιή-σα-σθαι}\\
 \textit{t-on} \textit{de} \textit{kekaromen-en} ~ \textit{ekʰonta} \textit{t-en} \textit{dianoi-an,} \textit{log-on} \textit{men} \textit{meden-a} \textit{t-on} \textit{ekein-u} \textit{log-on} \textit{poie-sa-stʰai}\\
the. \textsc{prt} stupefied-\textsc{ptcp-nom.sg} ~ have-\textsc{ptcp-acc} the-\textsc{acc.sg} thought-\textsc{acc} reason-\textsc{acc.sg} \textsc{part} no-one-\textsc{acc.sg} the-\textsc{gen.pl}
his-\textsc{gen.sg}  reason-\textsc{acc.sg} do-\textsc{aor-inf.mid}\\
\glt `He who falls into a deep stupor, ... even if he is mentally lucid, \textbf{makes no sense of any of his discourses'} \\
\hspace*{\fill}(\iwi{\emph{Vita sancti Lazari} 603.2.38})
\z
\z\clearpage


In this section, we have seen that when considering an
SVC, there are borderline cases that mean that it needs to be defined in
very vague terms.

\subsection{Support-verb-extension constructions and conceptual metaphors}\label{sec:vc:5:3}

Several explanations have been proposed for the motives underlying the lexical features
that characterise collocations. These explanations are generally based on the idea that
there is some semantic compatibility between the nominal base and the collocational verb,
although this compatibility has been understood in different ways.

One of the most typical and universal ways of creating and explaining the formal renewal
of SVCs is the conceptual metaphor \parencite{LakoffGeorge&JohnsonMark-1980803}.\is{metaphor} SVCs
represent a lexical domain in which many of their uses can be captured
\parencites{SalasJiménezG-2022210,SalasJiménezGuillermo-2024641}. Indeed, some
verbo-nominal collocations develop aspectual\is{aspect}, see \xref{ex:vc:7}, or diathetic\is{diathesis}, see \xref{ex:vc:8},
values, expressing different ranges of fixation and compositionality. The persistence of
these values in the development of post-classical Greek proves that any noun that can be
reconceptualised as eventive can be metaphorically extended by this kind of SVEC
\parencites{FedrianiChiara-2016339,TurCristina-2020406}.\is{support-verb-extension construction}

In this sense, the metaphor\is{metaphor} by which initiating an action is conceptualised as making
contact with an object, see \xxref{ex:vc:7a}{ex:vc:7b}, acquires an inchoative aspectual sense\is{aspect}:

% 7

\ea\label{ex:vc:7}

\ea\label{ex:vc:7a}

\glll ὥστε \textbf{πολέμ-ου} μὲν μηδ-ὲν ἔτι \textbf{ἅψα-σθαι} μηδε-τέρ-ους\\
 \textit{ʰoste} \textit{polem-u} \textit{men} \textit{med-en} \textit{eti} \textit{$^h$apsa-stʰai}
\textit{mede-ter-us}\\
so.that war-\textsc{gen} \textsc{prt} nothing-\textsc{acc} yet touch-\textsc{inf}
no.one-\textsc{du-acc}\\
\glt `So that neither the one nor the other \textbf{made war} {[}lit. touched war{]}' \\
\hspace*{\fill}(\iwi{Thucydides, \emph{Histories} 5.14.1})

\ex\label{ex:vc:7b}

\glll πρὸς λέοντ-α δορκ-ὰς \textbf{ἥ-πτ-ετο} \textbf{μάχ-ης}\\
 \textit{pros} \textit{leont-a} \textit{dork-as} \textit{$^h$e-pt-eto} \textit{mach-es}\\
against lion-\textsc{acc.sg} Gazelle.\textsc{nom.sg} \textsc{pst}-touch-\textsc{3sg}
battle-\textsc{gen-sg}\\
\glt `A gazelle \textbf{engaged in battle} against a lion' \\
\hspace*{\fill}(\iwi{\emph{Vita et martyrium sancti Anastasii Persa} 5 17.15})

\z

\z

Conversely, the SVECs\is{support-verb-extension construction} in \xref{ex:vc:8} correspond to the conceptual pattern by which an object
falling (ἐμπίπτω \emph{empipto} `to fall'% [\cite{JiménezLópezMaríaDolores-2024155}]
)
would serve to figuratively encode an inagentive or anticausative event:

% 8

\ea\label{ex:vc:8}

\ea

\glll ὀψὲ δέ ποτε βιασ-θεὶς ὑπὸ τ-ῶν πραγμάτ-ων \textbf{ἐν-ε-έπεσ-εν} \textbf{εἰς} \textbf{τ-ὸν} \textbf{νῦν} \textbf{δε-δηλωμέν-ον} \textbf{πόλεμ-ον}\\
 \textit{opse} \textit{de} \textit{pote} \textit{biast-ʰeis} \textit{ʰypo}
\textit{t-on} \textit{pragmat-on} \textit{en-e-pesen} \textit{eis} \textit{t-on} \textit{nyn} \textit{de-delomen-on} \textit{polem-on}\\
look \textsc{prt} ever force-\textsc{ptcp.pass} by the circumstances in-\textsc{past-}fell\textsc{-aor-3sg} into the-\textsc{acc.sg} now
\textsc{prf-}referred-\textsc{acc.sg} war-\textsc{acc.sg}\\
\glt `But later, forced by circumstances, h\textbf{e entered the war} {[}fell into the war{]} referred to' \\
\hspace*{\fill}(\iwi{Polybius, \textit{Histories} 14.12.4})

\ex

\glll πολλ-ῇ δὲ προθυμί-ᾳ περὶ τὴν ὁδοιπορί-αν χρωμέν-η \textbf{εἰς} \textbf{νόσ-ον} \textbf{ἑν-έ-πεσ-ε} μεταξὺ πορευομέν-η\\
\textit{poll-e\textsuperscript{i}} \textit{de} \textit{protʰymi-a\textsuperscript{i}} \textit{peri} \textit{t-en}
\textit{ʰoidopori-an} \textit{kʰromen-e} \textit{eis} \textit{nos-on} \textit{en-e-pes-e} \textit{metaxy}
\textit{poreuomen-e}\\
much-\textsc{dat.sg} \textsc{prt} courage-\textsc{dat.sg} about the.\textsc{acc.sg}
way-\textsc{acc.sg} useing\textsc{-nom.sg} into illness-\textsc{acc.sg}
in-\textsc{pst-}fell-\textsc{3sg-act} while walking\textsc{-ptcp-om.sg}\\
\glt `She \textbf{fell ill} while walking, having shown great eagerness while walking' \\
\hspace*{\fill}(\iwi{\emph{Vita
  et Miracula Sancti Artemii} 2.4.12})

\z

\z

The examples \xxref{ex:vc:7}{ex:vc:8} show the variability and discontinuity of SVCs in
post-classical Greek in terms of discourse levels. From a sociohistorical perspective, the
linguistic innovations involved in the survival or creation of new SVCs and SVECs\is{support-verb-extension construction} through
conceptual metaphors\is{metaphor} in written texts obey the logic of lexical change. The semantic
innovation induced by these metaphors confirms that the behaviour of support verbs forms a
distinct linguistic category that helps to represent the structure of the (sub-)event. By
observing the functioning of these metaphors, we can conclude that the formation of these
predicates can be detected through a formal renewal in the lexicon, thus rejecting, as
\citet{ButtMiriam-2010672} demonstrates, the possibility of explaining the changes on the
grounds of the strict rules associated with the canonical processes of grammaticalisation
\parencite{HopperPaulJ.andClossTraugottElizabeth-2003499}. The existence of SVCs that end
up being realised in compounds by univerbation of the type λογοποιέω \emph{logopoieō} `to
write speeches' (λόγον \emph{logon} + ποιέω \emph{poieō}) οr νομοθετέω
\emph{nomotʰeteo} `to make laws' (νόμον \emph{nomon} + τίθημι
\emph{titʰemi}) in post-classical Greek seems to be indicative of the
dissolution of compositionality \parencite{PompeiAnna-200619}. This is consistent with the
nature of lexical change that affects any kind of multi-word construction.\footnote{In
  this volume, Pompei \& Ricci give an account of the multiple phenomena that affect some
  of the collocations that undergo univerbation, configuring a typical case of nominal
  incorporation \parencite{VivesCuestaAlfonso-2012307}. In any case, we do not believe
  that these forms should be understood as authentic morphological compounds, since they
  do not meet the requirements of idiomaticity and lexicalisation\is{lexicalisation} that this type of
  nominal formation presupposes \parencite[30--33]{TribulatoOlga2015}.}

In all the cases studied above, we find the survival of SVCs introduced by ποιέομαι
\emph{poieomai} and other verbs (δίδωμι \emph{didomi}, λαμβάνω \emph{lambano}, γίγνομαι
\emph{gignomai}, etc,), combined with the same predicative nouns as these terms combined
with in CG. The frequency of the presence of these elements is significantly higher in
our so-called `metaphrastic' period.\is{metaphrasis} None of this can be a coincidence. Among other
possible explanations, we should not ignore the possibility that their survival is the
result of the actualisation of a practice of intralinguistic translation as
recently put forward by \citet[94]{LavidasNikolaos-2022877}:

\begin{quote}
  Intralingual translation, which is directly related to the diachrony of a language,
  describes the transfer of a text within one language due to the fact that the
  development of this language can be divided into two or more periods, for instance,
  ancient and modern, and can function as evidence of grammatical change.
\end{quote}

However, from the understanding of metaphrasis as a kind of intralingual translation\is{metaphrasis}, we
must be very careful in drawing conclusions. Lavidas is arguing in favour of a
`translation' into a modernised form of language.
Strictly speaking, it cannot be claimed that these are the kind of metaphrastic
transpositions of the 10th century.

In fact, such transpositions are adaptations of a more recent understanding and literary
aesthetic that can be called ``modern'', but in their formal expression Symeon Metaphrastes\is{metaphrasis}
chose a more conservative register than the authors of his model texts. It is only by
considering this limitation of the scope of the concept of ``intralingual translation''
that we can make generalisations about the functioning of syntactic or lexical variation
in this process of rewriting, in which the most avant-garde literary tendencies recover
linguistic uses of learned Greek. In this respect, it is striking that the generic term
for the Byzantine activity of rewriting (μετάφρασις \emph{metapʰrasis})
has among its basic meanings that of inter- and intralingual translation
\parencite{SignesCodonerJuan-2014739}. It is not surprising, therefore, that the main SVCs
that were in common use in earlier periods predominate in the periods when metaphrastic
activity\is{metaphrasis} was more widely cultivated by hagiographers\is{hagiography}.

\section{Conclusions}\label{sec:vc:6}

The SVCs form a heterogeneous group of productive multi-word expressions in classical and
post-classical Greek. Regarding this kind of constructions in the corpus studied
(Byzantine hagiography), I have detected a general evolution of the literary genre from a
popular (low) koine to a more learned (high) koine, which may have had some direct or
indirect influence on the higher frequency of occurrence and type of these collocations as
devices of intralingual translation which built new collocations.

However, this partial conclusion needs to be nuanced by the case studies of specific
predicative nouns, as we have previously done with εὐχή \textit{euche} and synonyms
\parencite{VivesCuestaAlfonsoandMadrigalAceroLucía-2022404}. The data analysed allow us to
verify trends in the general behaviour of these constructions which are compatible with
the rewriting procedures detected in Greek literature of the post-classical period,
especially in the texts called `metaphrastic'\is{metaphrasis}, which tend to recover classical linguistic
forms that were already fixed in earlier periods of the history of the language and from
which a certain variation in the distribution of the constructions can be explained. The
analysed data enables verification of trends in the general behaviour of these
constructions, which are compatible with the rewriting procedures detected in Greek
literature of the post-classical period. This is particularly evident in the texts
referred to as `metaphrastic', which aim to recover classical linguistic forms that were
already established in earlier periods of the language's history, and from which a certain
variation in the distribution of certain constructions can be explained.

Some SVCs existing in CG remain stable from a formal and syntactic point of view
in hagiographic texts\is{hagiography} of the high level of speech\is{level of speech}, as can be seen in the case of motion
nouns such as πορείαν/ἔκβασιν ποιέω \emph{poreian/ekbasin poieo}
(Section~\ref{sec:vc:5:1}), and partially in the borderline cases of the so-called verbs
of realisation\is{verb
of realisation} θέλημα/λόγον/κέλευσιν ποιέω \emph{$t^h$elema/logon/keleusin poieo}
(Section~\ref{sec:vc:5:2}), and even in SVECs\is{support-verb-extension construction} conceptualised by means of metaphors\is{metaphor} with
verbs such as ἅπτομαι \emph{$^h$aptomai} or ἐμπίπτω \emph{empipto}
(Section~\ref{sec:vc:5:3}). Within the corpus, the emergence of new verbo-nominal
collocations (SVCs or SVECs) is particularly noticeable in the metaphrastic reworking of
older \emph{Lives}.

In short,~there is a convergence of sociolinguistic\is{sociolinguistics} and purely linguistic~factors in the
life cycle of SVCs~in post-classical Greek. In future research, the scope of these general
statements can be refined by studying the diachronic evolution of particular SVCs from CG
to the end of the Byzantine period.

\section*{Abbreviations}\label{abbreviations}

\begin{tabularx}{.45\textwidth}{lQ}
DO & Direct Object \\
NT & New Testament \\
NTG & New Testament Greek \\
\end{tabularx}
\begin{tabularx}{.45\textwidth}{lQ}
SVE & support-verb extension \\
SVEC & support-verb-extension construction \\
\end{tabularx}

\section*{Acknowledgements}\label{acknowledgements}

This chapter has been supported by two national research projects: (1)~the I+D Excellence
Research Project (MINECO) \emph{Interacción del léxico y la sintaxis en griego antiguo y
  latín} 2: \emph{Diccionario de Colocaciones Latinas (DiCoLat)} y \emph{Diccionario de
  Colocaciones del Griego Antiguo (DiCoGrA)} (PID2021-125076NB-C42) and (2)~\emph{El autor
  bizantino III: metáfrasis, reescritura y recepción} (PID2019-105102GB-I00). I would like
to express my gratitude to Lucía Madrigal Acero (Universidad Complutense de Madrid) for
generously sharing her extensive database on the corpus of hagiographic texts and her MA
dissertation titled \emph{Estudio diacrónico de corpus de las colocaciones con ποιέω +
  nombre eventivo en griego posclásico} (2022), which I had the privilege of supervising.
I am grateful to Prof. Jiménez López for providing me with the handout of her paper on
Greek linguistics at the last National Congress of the Spanish Society of Classical
Studies (19.7.2023), entitled ``Entre el léxico y la sintaxis: las colocaciones
verbo-nominales en griego antiguo''. I would also like to express my gratitude to the
anonymous reviewers and the editor of this volume for their insightful comments on an
earlier version of this paper.



 \sloppy
 \printbibliography[heading=subbibliography,notkeyword=this]

\end{document}
