\documentclass[output=paper,colorlinks,citecolor=brown]{langscibook}
\ChapterDOI{10.5281/zenodo.14017929}
\author{Elena Squeri\affiliation{Sapienza - Università di Roma / UMR 8167: Orient et Méditerranée (équipe Médecine grecque et littérature technique)}} 
\title[χράομαι \textit{khraomai} as a support verb in the Hippocratic Corpus]{χράομαι \textit{khraomai} as a support verb in the medical jargon of the Hippocratic Corpus}


\abstract{This paper analyzes syntagms constituted by a potentially referential noun and the verb χράομαι \textit{khraomai} (‘to use’) in the Hippocratic writings of the 5th-4th c. BC, testing their potential inclusion among support-verb constructions. The survey starts with syntagms including deverbal nouns, which express both a therapeutic practice and the medical device it involves, and  then extends to nouns of foods and drinks, which combine with χράομαι \textit{khraomai} to express the actions of ‘eating' and ‘drinking'. The data suggest the inclusion among support-verb constructions of syntagms with referential nouns if they refer to a class of objects typically involved in the action expressed by verbs which act both transitively and intransitively. The choice of χράομαι \textit{khraomai} is explained both semantically and diaphasically.


\bigskip


Il contributo analizza i sintagmi costituiti da un nome potenzialmente referenziale e il verbo χράομαι \textit{khraomai} (‘usare') negli scritti ippocratici del V-IV sec. a.C., testandone la possibile inclusione fra le strutture a verbo supporto. La ricerca inizia analizzando i sintagmi contenenti nomi deverbali, che esprimono sia una pratica terapeutica, sia il presidio medico che vi è coinvolto, e si estende ai nomi di cibi e bevande, che si combinano con χράομαι \textit{khraomai} per esprimere le azioni di “mangiare” e “bere”. I dati suggeriscono l'inclusione fra le strutture a verbo supporto dei sintagmi con nomi referenziali, se essi fanno riferimento a una classe di oggetti tipicamente coinvolti in azioni espresse da verbi che possono essere sia transitivi sia intransitivi. La scelta di χράομαι \textit{khraomai} è spiegata semanticamente e dal punto di vista diafasico.}



\IfFileExists{../localcommands.tex}{
   \addbibresource{../localbibliography.bib}
   \usepackage{langsci-optional}
\usepackage{langsci-gb4e}
\usepackage{langsci-lgr}

\usepackage{listings}
\lstset{basicstyle=\ttfamily,tabsize=2,breaklines=true}

%added by author
% \usepackage{tipa}
\usepackage{multirow}
\graphicspath{{figures/}}
\usepackage{langsci-branding}

   
\newcommand{\sent}{\enumsentence}
\newcommand{\sents}{\eenumsentence}
\let\citeasnoun\citet

\renewcommand{\lsCoverTitleFont}[1]{\sffamily\addfontfeatures{Scale=MatchUppercase}\fontsize{44pt}{16mm}\selectfont #1}
  
   %% hyphenation points for line breaks
%% Normally, automatic hyphenation in LaTeX is very good
%% If a word is mis-hyphenated, add it to this file
%%
%% add information to TeX file before \begin{document} with:
%% %% hyphenation points for line breaks
%% Normally, automatic hyphenation in LaTeX is very good
%% If a word is mis-hyphenated, add it to this file
%%
%% add information to TeX file before \begin{document} with:
%% %% hyphenation points for line breaks
%% Normally, automatic hyphenation in LaTeX is very good
%% If a word is mis-hyphenated, add it to this file
%%
%% add information to TeX file before \begin{document} with:
%% \include{localhyphenation}
\hyphenation{
affri-ca-te
affri-ca-tes
an-no-tated
com-ple-ments
com-po-si-tio-na-li-ty
non-com-po-si-tio-na-li-ty
Gon-zá-lez
out-side
Ri-chárd
se-man-tics
STREU-SLE
Tie-de-mann
}
\hyphenation{
affri-ca-te
affri-ca-tes
an-no-tated
com-ple-ments
com-po-si-tio-na-li-ty
non-com-po-si-tio-na-li-ty
Gon-zá-lez
out-side
Ri-chárd
se-man-tics
STREU-SLE
Tie-de-mann
}
\hyphenation{
affri-ca-te
affri-ca-tes
an-no-tated
com-ple-ments
com-po-si-tio-na-li-ty
non-com-po-si-tio-na-li-ty
Gon-zá-lez
out-side
Ri-chárd
se-man-tics
STREU-SLE
Tie-de-mann
}
   \boolfalse{bookcompile}
   \togglepaper[23]%%chapternumber
}{}

\begin{document}
\emergencystretch 3em
\maketitle

\section{Support verb constructions as complex predicates}
Traditionally\footnote{The dataset is accessible here: \url{http://dx.doi.org/10.5287/ora-n652gamyj}. Hippocratic texts are quoted by mentioning the numbering, the page and the line in which they appear in the following critical editions: \textit{Affections}: \citet{PotterDiseases1988}; \textit{Diseases I}: \citet{PotterDiseases1988}; \textit{Diseases II}: \citet{Jouanna1983}; \textit{Diseases III}: \citet{Potter1980}; \textit{Diseases of Women I and II}: \citet{Potter2018}; \textit{Epidemics II}: \citet{Smith1994}; \textit{Epidemics V}: \citet{Jouanna2000}; \textit{Fractures}: \citet{Jouanna2022}; \textit{Internal Affections}: \citet{PotterAffections1988}; \textit{Nature of Man}: \citet{Jouanna2002}; \textit{Nature of Women}: \citet{Bourbon2008};  \textit{Places in Man}: \citet{Joly1978}; \textit{Regimen in Acute diseases}: \citet{Joly1972}; \textit{Regimen in Acute diseases (Appendix)}: \citet{Joly1972}; \textit{Sight}: \citet{Joly1978}; \textit{Ulcers}: \citet{Duminil1998}; \textit{Wounds in the Head}: \citet{Hanson1999} The number of the volume, page, and line of the traditional edition from \citet{Littre1839} are also added in round brackets. Unless otherwise stated, all translations have been proposed on the basis of those present in the Loeb collection. Some minor changes have been made in order to better highlight the syntactic structure which is the focus of this chapter.} the definition of “support verb construction” (SVC henceforth) is applied to those structures in which a predicative noun expresses a state, an event or a process thanks to its combination with a verb, which only supplies such grammatical information as tense, voice and person. The event is however only identified by the noun, which also activates and assigns the argument positions.\footnote{See among many others \citet{Gross1981}; \citet{Gross1989}; \citet{Gross2004}; \citet{Gross2017}; \citet{jezek2004types}; \citet{Langer2005}.}

According to this definition, SVCs could only involve nouns which refer to an action rather than an object. These are often labelled as “predicative nouns” (from the French definition of “noms” or “substantifs prédicatifs”, see \citealt{Gross1981} and \citealt{Gross1989}). \citet{Lyons1977}, however, speaks of “first order entities” for names of objects and “second order entities” for names of situations, while \citet{simone2003masdar} proposes “noms de procès” and \citet[49--54]{Grimshaw1990} labels nouns which are also argument-assigning as “complex event nominals”\is{event structure}.

The traditional definition of SVC also entails that the semantic meaning of the verb involved in the structure should have no influence on the meaning of the structure. This is why such verbs are called “light verbs”.\footnote{The definition of “light verb” is first found in \citet[117]{Jespersen1942}.}

In recent years, however, studies have come to a more flexible definition of SVCs. It has been observed that a single noun can combine with different support verbs (SV henceforth), whose meaning can be more or less “light”. The substitution of the verb which typically occurs in combination with a predicative noun may cause a shift in the aspect \citep[349–353]{gross_pour_2004} of the expressed action, but may also give particular nuances to the event, process or state expressed by the noun. \citet{jezek2004types} refers to such SV as “extensions de verbe support”. She compares such expressions as the Italian “dare una risposta” (‘give an answer'), in which ‘dare' is a standard SV, and “azzardare una risposta” (lit. ‘hazard an answer'), in which ‘azzardare' is an extension of the support verb. In English a similar opposition can be found between “to give an answer” and “to shoot an answer”.

It can therefore be argued that the semantic value of the verb involved in an SVC is not always completely bleached and cooperates with the noun in order to achieve a well-defined meaning.\footnote{\citet[115--117]{Pompei2017}, for instance, relies on this fact when stating that verbs involved in SVCs are not completely empty with respect to their predicative force, but can bring more or less semantic information to the structure, along with the noun.}

The exclusively predicative nature of the noun has been challenged as well, by noticing that not all nouns involved in SVCs are strictly predicative and argument-assigning, but that they become predicative once included in such structures \citep[168--171]{Bowern2008}. In SVCs such as ‘have a shower’ and ‘take a picture’ ‘shower’ and ‘picture’ express an action, but they can also refer to concrete objects (“I bought a new shower for my bathroom"; “They taped a picture of their cat on the door").\footnote{On the possibility that nouns which are neither deverbal nor predicative could hold some predicative force, see \citet{SimonePompei2007} and \textit{infra} § 7.}  This does not mean that nouns cannot be predicative by themselves, outside of an SVC, but that – according to the aforementioned view – some nouns are forced into predication when used in SVCs.

This is why SVCs have lately been included in the broader category of “complex predicates”\is{complex predicate}: multi-headed predicates, in which predication is shared by more than one element.\footnote{On the definition of “complex predicates”, see \citet[1]{AlsinaBresnanSells1997}; \citet[165]{Bowern2008}; \citet[49]{Butt2010}. On the inclusion of SVCs among complex predicates, see \citet{Bowern2008}; \citet{Butt2010}; \citet[xxiii-xxix]{PompeiMereu2019}.}  However, this redefinition of SVCs makes it harder to posit a clear limit between them and simple collocations.\footnote{\citet[186]{Jezek2004} defines SVCs as “un sous-type de collocation et plus précisément une collocation débalancée – du point de vue sémantique – vers le Nom” and proposes a continuum which goes from traditional SVCs to collocations, through SVCs with an extension of the SV. On the relationship between SVCs and collocations, see also \citet[195--198]{ElisabettaJežek2011}.}  Some scholars consider a solid proof for identifying an SVC its potential equivalence with a synthetic verb (‘to have a shower’ / ‘to shower’) (\citealt[169--170]{langer2004linguistic}; \citet[xxvi]{PompeiMereu2019}). However, it must be borne in mind that accepting this co-existence on the synchronic, diastratic and diaphasic levels would entail the acceptance of redundancy in language, something that is often excluded by linguistics.\footnote{On the problem, see \citet{JiménezLópez2011}; \citet[18]{Fendel2020}. \citet[120]{Pompei2017} observes that not all SVCs have a correspondent synthetic verb to which they are formally related. On the fact that the lack of a synthetic verb form cannot be considered proof for discarding the interpretation of an SVC as such, see also \citet[155]{Marini2010}.} Nevertheless, studies as the one conducted by \citet[159--164]{Marini2010} on the use of SVC with ποιέομαι \textit{poieomai} in Aristotle, showed that this author often used both forms in the same work, sometimes within a short distance of each other. \citet{JiménezLópez2011}, who ran the same study on Lysias, proposed that the choice to employ an SVC rather than a synthetic verb could be justified by the fact that SVCs are more prone to modification and restriction, since the predicative noun can easily be combined with a modifier. As far as the corpus under scrutiny here is concerned, it can be stated, along with \citet{Marini2010}, that both structures (i.e. the support-verb construction and the simplex verb derived from the same root as the predicative noun in the support-verb construction) can be found in the same work (see examples 5, 7 and 8) and that the predicative nouns in SVCs are not always combined with a modifier (see examples 16, 17 and 21).

\section {Corpus}
\largerpage
The first steps that have been made in the study of SVCs in Classical Greek (CG henceforth) focus on a small range of potential light verbs, mostly ποιέω \textit{poieō} (‘do’, ‘make’) and ἔχω \textit{ekhō} (‘have’).\footnote{On ποιέω \textit{poieō}, see \citet{Marini2010}; \citet{JiménezLópez2011,JiménezLópez2012,JiménezLópez2016}. On ἔχω \textit{ekhō}, see \citet{Vanséveren1995}; \citet{tronci2017lexicon}. See also \citet{jimenezlopez2021} on γίγνομαι \textit{gignomai}.} This paper aims at extending the investigation to the role that the verb χράομαι \textit{khraomai} (‘use’) may play in such constructions, even if it is not a typical light verb. In order to do so, a corpus study has been conducted on the medical writings of the \textit{Hippocratic corpus} (HC henceforth). 

The HC is a group of around sixty medical works of different length, subject and dating.\footnote{For an overview of the content and the dating of the treatises here analysed, see \citet{Craik2015}; \citet[529--590]{Jouanna2017}.} Some are more rhetorical\footnote{On the text of the HC that were supposed to be pronounced orally, see \citet{Jouanna1984}}, but many have a technical purpose. They discuss pathologies and injuries, and the way of healing them by different preparations and by a particular diet and lifestyle. This paper will take into account the more ancient ones, dated between the second half of the 5th and the second half of the 4th c. BC.
It has been claimed that complex predicates\is{complex predicate} and, among them, SVCs, may be created in specific areas of language and then analogically extended to other uses \citep{Bowern2008}. This research may therefore also be read in parallel with other studies of this same structure in other areas of CG (see Madrigal Acero in this volume), to evaluate whether there has truly been influence from one to another area and, if so, in what direction.

The works which compose the HC are also those in which the first efforts made by ancient physicians to create their own jargon\is{jargon} may be identified. For doing so, they created new words, but, most of the time, they just re-employed existing and common words, to refer to a more specific and sectorial meaning.\footnote{On the creation of the ancient medical jargon, see \citet{Benveniste1965}; \citet{Irigoin1980,Irigoin1983}; \citet{Skoda2004}; \citet{Schironi2013NEW}} This specialisation of meaning often concerned verbs \citep{Squeri2023}, from which many deverbal nouns were also created. CG deverbal nouns are inherently predicative, since they are derived from a verb expressing an action, but can, at the same time, refer either to the concrete product of that action, or to one of the referential\is{referentiality}\footnote{The adjective “referential” is used in this contribution for nouns which refer to a concrete and existing object, in the sense of \citet[293]{Givón1978}: «referentiality is a semantic property of nominals. It involves, roughly, the speaker’s intent to ‘refer to’ or ‘mean’ a nominal expression to have non-empty references – i.e. to ‘exist’ – within a particular universe of discourse». } elements that are involved in that action. Briefly, deverbal nouns can refer both  to an action (\ref{ex:Plato288}) and to one of its arguments (\ref{ex:Iliad29}).\footnote{On the possibility of deverbal nouns to refer to either an activity or an argument, see \citet{comriethompson2006} For CG, see \citet[31--34]{Civilleri2012}. \citet{Chantraine1933} did not make any general statement on the subject, but some considerations which go along these lines may be found in the chapters about deverbatives with the suffixes -μα \textit{-ma} \citep[183]{Chantraine1933} and -σις \textit{-sis} \citep[287--288]{Chantraine1933}.}

%MATTHEW, why does it not print '1' for the Plato example? 
\begin{exe}
\ex\label{ex:Plato288}
\glll Ἡ δέ γε φιλοσοφία κτῆσις  \\ 
\textit{hē} \textit{de} \textit{ge}  \textit{philosophia}  \textit{ktēsis}  \\
\textsc{art.nom.sg} \textsc{prt} \textsc{prt} philosophy.\textsc{nom.sg} acquiring.\textsc{nom.sg}  \\

\glll ἐπιστήμης. \\
\textit{epistēmēs}. \\
knowledge.\textsc{gen.sg} \\
\glt ‘Philosophy is an acquiring of knowledge.' \\
\hspace*{\fill}(\iwi{Plato, \textit{Euthydemus} 288d} (philosophy, dialogue))
\end{exe}

\begin{exe}
\ex\label{ex:Iliad29}
\glll οἱ δείξειας […] κτῆσιν ἐμὴν δμῶάς τε καὶ ὑψερεφὲς μέγα  \\ 
\textit{hoi} \textit{deixeias} […] \textit{ktēsin} \textit{emēn} \textit{dmōas} \textit{te} \textit{kai} \textit{hypserephes} \textit{mega}  \\
he.\textsc{dat.sg} show.\textsc{aor.opt.2sg} [...] possession.\textsc{acc.sg} my.\textsc{acc.sg} slaves.\textsc{acc.pl} and and high.roofed.\textsc{acc.sg} great.\textsc{acc.sg} \\

\glll δῶμα. \\
\textit{dōma} \\
house.\textsc{acc.sg} \\
\glt ‘Show him […] my possessions, my slaves, and my great high-roofed house.’ \\
\hspace*{\fill}(\iwi{Homer, \textit{Iliad} 19.333} (epic, poetry))
\end{exe}

This does not entail that CG only has SVCs with deverbal nouns, but that they are a good starting point to address the fact that some of the nouns involved in SVCs can refer to an object and, at the same time, act as predicative once supported by a verb.\footnote{On the involvement of CG deverbal nouns in SVCs, see \citet[160--164]{Marini2010} and \citet{JiménezLópez2011}.}
The next three sections take into account the combination of χράομαι (\textit{khraomai}) with four nouns that are derived from three verbs that Hippocratic medicine draws from common language and adapts to the expression of therapeutic practices\is{therapeutics}: κατάπλασμα \textit{kataplasma} (‘plaster'), κλυσμός \textit{klysmos} (‘lavage', ‘douche') and κλύσμα \textit{klysma} (‘lavage', ‘douche'), and πρόσθετον \textit{prostheton} (‘vaginal suppository').

\section{καταπλάσσω \textit{kataplassō} and κατάπλασμα \textit{kataplasma}}
καταπλάσσω \textit{kataplassō} expresses the therapeutic action\is{therapeutics} of plastering a part of the body with a curative substance. The common form is πλάσσω \textit{plassō}, whereas καταπλάσσω \textit{kataplassō} seems to have been created in the HC itself, since in the 5th c. BC the verb is attested only four times outside the HC: \iwi{Herodotus, \textit{Histories}, IV 75} (historiography, prose); \iwi{Aristophanes, \textit{Plutus} 721 and 724} (comedy, drama); \iwi{Aristophanes, \textit{Assemblywomen} 878} (comedy, drama). The first three, however, are used in reference to body and health care. 
In the HC the plastered substance may be expressed both in the accusative (\ref{ex:HCkataplasma1}) and the dative case (\ref{ex:HCkataplasma2}). The verb is however often employed with intransitive value\is{transitivity}, with the meaning of ‘treat with plasters’ (\ref{ex:HCkataplasma3}), a feature that it acquires as a consequence of its medical specialisation, which was not available for the simple form πλάσσω \textit{plassō}. More precisely, when employed with the meaning of ‘treat with plasters’, the verb acts as an “unergative”, i.e. as an intransitive verb whose subject is the agent that initiates the action, and not the patient, as happens in such other intransitives as “Jack fell”, normally called “unaccusatives”.\footnote{On the classification of intransitive verbs into unergatives and unaccusatives and their respective definition, see \citet{Perlmutter1978} and \citet{LevinHovav1995}, especially Ch. 1.} 
%MATTHEW, why is it printing '3' now for all three examples despite them having different labels, can you sort please? 
\begin{exe}
\ex\label{ex:HCkataplasma1}
\glll ἐπὶ δὲ ὑποχόνδρια λίνου σπέρμα  \\ 
\textit{Epi} \textit{de} \textit{hypokhondria} \textit{linou} \textit{sperma}  \\
to \textsc{prt} hypochondrium.\textsc{acc.pl} linen.\textsc{gen.sg} seed.\textsc{acc.sg}  \\

\glll καταπλάσσειν ἕως μαζῶν. \\
\textit{kataplassein} \textit{heōs} \textit{mazōn}. \\
apply.as.plaster.\textsc{prs.inf} up.to breasts.\textsc{gen.pl} \\
\glt ‘Apply linseed plasters to the hypochondrium up as far as the breasts.’ \\
\hspace*{\fill}(\iwi{HC \textit{Regimen in Acute Diseases (Appendix)} 33. 1, p. 84, 21-22 Joly (2, 464, 5 L.)})
\end{exe}

\begin{exe}
\ex\label{ex:HCkataplasma2}
\glll ἢν καταπλάσῃς γῇ {(Foes : τῇ θM)} κεραμίτιδι ἢ ἄλλῳ τῷ τοιούτῳ… \\ 
\textit{ēn} \textit{kataplasēs} \textit{gē} {(Foes : τῇ θM)} \textit{keramitidi} \textit{ē} \textit{allō} \textit{tō} \textit{toioutō}. \\
if plaster.\textsc{prs.sbjv.2sg} earth.\textsc{dat.sg} {(Foes : τῇ θM)} for.pottery.\textsc{dat.sg} or other.\textsc{dat.sg} \textsc{art.dat.sg} as.such.\textsc{dat.sg} \\
\glt ‘If you plaster the patient over with potter’s earth or some other such material…’ \\
\hspace*{\fill}(\iwi{HC \textit{Diseases I} 17, p. 138, 2-3 P. (6, 170, 20-21 L.)})
\end{exe}

\begin{exe}
\ex\label{ex:HCkataplasma3}
\glll χρὴ δὲ οὐδὲ τὰ ἐν τῷ  \\
\textit{khrē} \textit{de} \textit{oude} \textit{ta} \textit{en} \textit{tō}  \\
should.\textsc{prs.ind.impers} \textsc{prt} not \textsc{art.acc.pl} in \textsc{art.dat.sg}   \\

\glll μετώπῳ διὰ παντὸς τοῦ χρόνου \\
\textit{metōpō} \textit{dia} \textit{pantos} \textit{tou} \textit{khronou} \\
forehead.\textsc{dat.sg} through all.\textsc{gen.sg} \textsc{art.gen.sg} time.\textsc{gen.sg} \\

\glll καταπλάσσειν καὶ ἐπιδεῖν. \\
\textit{kataplassein} \textit{kai} \textit{epidein}. \\
treat.with.plasters.\textsc{prs.inf} and bandage.\textsc{prs.inf} \\
\glt ‘But even wounds on the forehead you should not treat with plasters and bandages continuously.’ \\
\hspace*{\fill}(\iwi{HC \textit{Wounds in the Head} 13, p. 78, 21-22 Hanson (3, 230, 7-8 L.)})
\end{exe}


It is certainly not by chance that κατάπλασμα \textit{kataplasma} is also attested as a sort of cognate dative object\is{cognate object}\is{dative object} of καταπλάσσω \textit{kataplassō}.\footnote{The relationship between κατάπλασμα \textit{kataplasma} and καταπλάσσω \textit{kataplassō} is opposite to that which typically exists between a verb and a cognate object, which is normally the nominal base of a denominal verb. On the use of -μα \textit{-ma} derivatives as COs in CG, see \citet[287]{horrocksstavrou2010}. This is however unsurprising, since πλάσσω \textit{plassō} is not originally a denominal verb as it is not originally unergative, but acquires this event frame only\is{event structure} as a consequence of its medical specialization.} According to \citet{HaleKeyserES1987}, \citet{halekeyser1993} cognate objects (CO henceforth)\is{cognate object} of unergative verbs are part of the logical structure of the action expressed by these verbs, since they represent the class of entities that must be involved in the action so that it can be referred to by that verb.\footnote{The logical structure of an action such as ‘John laughs’ should therefore be [John do LAUGH] (\citealt[48--50]{HaleKeyserES1987}; \citealt[§ 1]{halekeyser1993}). For cognate objects as the name of the prototypical product of an action, see also \citet{Massam1990}, who however argues against this analysis of the logical structure of unergatives. For the equivalence of structures with CO and SVCs, see \citet{mirto2007dream}, for English, and \citet[288--289]{horrocksstavrou2010}: 288--289 for AG.} In combination with the right SVs they can therefore evoke the same action of which they are a constitutive element, see (\ref{ex:HCkataplasma4}).

\begin{exe}
\ex\label{ex:HCkataplasma4}
\glll καταπλάσσειν τῶν καταπλασμάτων ὅ τι ἄν σοι δοκῇ συμφέρειν. \\
\textit{kataplassein} \textit{tōn} \textit{kataplasmatōn} \textit{ho} \textit{ti} \textit{an} \textit{soi} \textit{dokē} \textit{sympherein}. \\
apply.as.plaster.\textsc{prs.inf} \textsc{art.gen.pl} plaster.\textsc{gen.pl} \textsc{rel.sg.n} \textsc{indf.acc.n} \textsc{prt} you.\textsc{dat.sg} seem.\textsc{prs.sbjv.3sg} help.\textsc{prs.inf} \\
\glt ‘Apply plasters that you think may be beneficial.' \\
\hspace*{\fill}(\iwi{HC \textit{Sight} 9. 2, p. 171, 22-23 Joly (9, 160, 10-11 L.)})
\end{exe}

The fact that κατάπλασμα \textit{kataplasma} is the name of a class of objects that must be involved in a typical action ensures that, if combined as a predicative phrase with the dative object\is{dative object} of χράομαι \textit{khraomai}, it can define the action in which the dative object is involved. Examples (\ref{ex:HCkataplasma5}) and (\ref{ex:HCkataplasma6}) show the equivalence between the structure with καταπλάσσω \textit{kataplassō} and τῇ μάζῃ \textit{tē mazē} as a dative object, and the structure with χράομαι \textit{khraomai} combined with τῇ μάζῃ \textit{tē mazē} as a dative object and καταπλάσματι \textit{kataplasmati} as a predicative phrase.\footnote{On the possibility of the nominal parts of SVCs acting as predicative phrases, see \citet[122--123 and 127--128]{Pompei2017}.}

\begin{exe}
\ex\label{ex:HCkataplasma5}
\glll τὸ ἕλκος […] καταπλάσας τῇ μάζῃ ἐπιδῆσαι. \\ 
\textit{To} \textit{helkos} […] \textit{kataplasas} \textit{tē} \textit{mazē} \textit{epidēsai}.\\
\textsc{art.acc.sg} wound.\textsc{acc.sg} […] plaster.\textsc{aor.ptcp.nom.sg} \textsc{art.dat.sg} barley.meal.\textsc{dat.sg} bandage.\textsc{aor.inf} \\
\glt ‘Having applied a barley-meal as a plaster, bandage the wound.’ \\
\hspace*{\fill}(\iwi{HC \textit{Wounds in the Head} 14, p. 82, 16-17 Hanson (3, 240, 2-3 L.)})
\end{exe}

\begin{exe}
\ex\label{ex:HCkataplasma6}
\glll μοτώσαντα δὲ χρὴ \textbf{καταπλάσματι} \textbf{χρῆσθαι}, ὅσον ἄν περ χρόνον καὶ τῷ μοτῷ, μάζῃ. \\
\textit{Motōsanta} \textit{de} \textit{khrē} \textit{kataplasmati} \textit{khrēsthai}, \textit{hoson} \textit{an} \textit{per} \textit{khronon} \textit{kai} \textit{tō} \textit{motō}, \textit{maze}. \\
pack.\textsc{aor.ptcp.acc.sg} \textsc{prt} should.\textsc{prs.ind.impers} plaster.\textsc{dat.sg} use.\textsc{prs.inf} as.much.\textsc{acc.sg} \textsc{prt} \textsc{prt} time.\textsc{acc.sg} and \textsc{art.dat.sg} bandage.\textsc{dat.sg} barley.meal.\textsc{dat.sg} \\
\glt ‘After packing (sc. the wound) you must use as a plaster, for as long a time as the packing, a barley-cake.’ \\
\hspace*{\fill}(\iwi{HC \textit{Wounds in the Head} 14, p. 80, 20-22 Hanson (3, 236, 3-4 L.)})
\end{exe}

κατάπλασμα \textit{kataplasma} can therefore act both as a predicative noun and as the noun of a concrete medical device, as in the recipe in (\ref{ex:HCkataplasma7}), which describes plasters for lesions made of different herbal ingredients.\footnote{On the fact that -μα \textit{-ma} deverbatives may refer to a referential object\is{referentiality} or an instrument of an action, see \citet[ 159--168]{Civilleri2012}.}

\begin{exe}
\ex\label{ex:HCkataplasma7}
\glll Καταπλάσματα οἰδημάτων καὶ φλεγμασίης […]· ἡ ἑφθὴ φλόμος καὶ τῆς τριφύλλου τὰ φύλλα ὠμά καὶ τοῦ ἐπιπέτρου τὰ φύλλα ἑφθά καὶ τὸ πόλιον. \\ 
\textit{Kataplasmata} \textit{oidēmatōn} \textit{kai} \textit{phlegmasiēs} […]; \textit{hē} \textit{hephthē} \textit{phlomos} \textit{kai} \textit{tēs} \textit{trifyllou} \textit{ta} \textit{fylla} \textit{ōma} \textit{kai} \textit{tou} \textit{epipetrou} \textit{ta} \textit{fylla} \textit{hephtha} \textit{kai} \textit{to} \textit{polion}. \\
plaster.\textsc{acc.pl} swelling.\textsc{gen.pl} and inflammation.\textsc{gen.sg} […]: \textsc{art.nom.sg} boiled.\textsc{nom.sg} mullein.\textsc{nom.sg} and \textsc{art.gen.sg} clover.\textsc{gen.sg} \textsc{art.nom.pl} leaf.\textsc{nom.pl} raw.\textsc{nom.pl} and \textsc{art.gen.sg} rock.plant.\textsc{gen.sg} \textsc{art.nom.pl} leaf.\textsc{nom.pl} boiled.\textsc{nom.pl} and \textsc{art.nom.sg} hulwort.\textsc{nom.sg}\\
\glt ‘Plasters for swellings and for inflammation […]: boiled mullein, raw leaves of clover, boiled leaves of rock-plant, hulwort.' \\
\hspace*{\fill}(\iwi{HC \textit{Ulcers} 11. 1, p. 58, 16-19 Duminil (6, 410, 5-7 L.)})
\end{exe}

There is only one occurrence of χράομαι \textit{khraomai} with κατάπλασμα \textit{kataplasma} in the HC, but this structure is inherited by the medical tradition. For instance, this structure recurs 16 times in Galen. Interesting equivalences may be found between the prescriptions formulated with καταπλάσσω \textit{kataplassō} in the HC and those formulated with καταπλάσματι χράομαι \textit{kataplasmati khraomai} by later authors. The recipes proposed for fluxes in the gynecological writings of the HC and in the gynecological writings of Soranus (2nd c. AD) may thus be compared. In the HC the use of myrtle as a plaster is prescribed by using μυρσίνης φύλλα \textit{myrsinēs fylla} as the object of καταπλάσσω \textit{kataplassō}, see (\ref{ex:HCkataplasma8}), while in Soranus the use of the same ingredient in a plaster is specified by a prepositional phrase dependent upon καταπλάσμασι \textit{kataplasmasi}, the dative object\is{dative object} of χρῆσθαι \textit{khrēsthai}, see (\ref{ex:HCkataplasma9}). 

\begin{exe}
\ex\label{ex:HCkataplasma8}
\glll ἀκτῆς καὶ μυρσίνης φύλλα κατάπλασσε.\\ 
\textit{aktēs} \textit{kai} \textit{myrsinēs} \textit{fylla} \textit{kataplasse}. \\
elder.\textsc{gen.sg} and myrtle.\textsc{gen.sg} leaf.\textsc{acc.pl} apply.as.plaster.\textsc{prs.imp.2sg} \\
\glt ‘Apply plasters of elder and myrtle leaves.' \\
\hspace*{\fill}(\iwi{HC \textit{Diseases of Women II} 193 (3), p. 414, 4 Potter (8, 374, 16 L.)})
\end{exe}

\begin{exe}
\ex\label{ex:HCkataplasma9}
\glll τοῖς διὰ φοινίκων καὶ κυδωνίων καὶ μυρσίνης καταπλάσμασι καὶ κηρωταῖς χρῆσθαι. \\ 
\textit{tois} \textit{dia} \textit{phoinikōn} \textit{kai} \textit{kydōniōn} \textit{kai} \textit{myrsinēs} \textit{kataplasmasi} \textit{kai} \textit{kērōtais} \textit{khrēsthai.} \\
\textsc{art.dat.pl} with date.\textsc{gen.pl} and quince.\textsc{gen.pl} and myrtle.\textsc{gen.sg} plaster.\textsc{dat.pl} and cerate.\textsc{dat.pl} use.\textsc{prs.inf} \\
\glt ‘One should use plasters as well as cerates made of dates, quinces, and myrtle.' \\
\hspace*{\fill}(\iwi{Soranus \textit{Gynaecology} III 46, 1})
\end{exe}

\largerpage
\section{κλύζω \textit{klyzō}, κλυσμός \textit{klysmos} and κλύσμα \textit{klysma}}
The verb κλύζω \textit{klyzō} and its deverbal derivatives κλύσμα \textit{klysma} and κλυσμός \textit{klysmos} show similar behavior. κλύζω \textit{klyzō} is used in the Homeric epic to express the motion of “crashing” waves and, like most verbs of motion, is used intransitively\is{transitivity}, as an unaccusative.\footnote{\iwi{Homer, \textit{Iliad} 14.392-393} (epic, poetry): ἐκλύσθη δὲ θάλασσα ποτὶ κλισίας τε νέας τε / Ἀργείων \textit{eklysthē de thalassa poti klisias te neas te} crash.\textsc{aor.ind.pass.3sg} \textsc{prt} sea.\textsc{nom.sg} towards hut.\textsc{acc.pl} and ship.\textsc{acc.pl} and Argives.\textsc{gen.pl} ‘The sea crashed towards the huts and ships of the Argives’.} Hippocratic medicine rationalizes the power of water movement and starts using the verb to refer to the therapeutic practice\is{therapeutics} of purging with an enema, a lavage, or a douche \citep[ch. 5]{Squeri2023}. In the HC, κλύζω \textit{klyzō} is therefore used transitively with what is to be cleaned as a direct object and the purging liquid which is set in motion as a dative of instrument, see (\ref{ex:klyzw1}).

\begin{exe}
\ex\label{ex:klyzw1}
\glll κλύζειν τὰ ὦτα οἴνῳ γλυκεῖ. \\ 
\textit{klyzein} \textit{ta} \textit{ōta} \textit{oinō} \textit{glykei}. \\
make.a.lavage.\textsc{prs.inf} \textsc{art.acc.pl} ear.\textsc{acc.pl} wine.\textsc{dat.sg} sweet.\textsc{dat.sg} \\
\glt ‘Make a lavage to the ears with sweet wine.' \\
\hspace*{\fill}(\iwi{HC \textit{Diseases III} 2, p. 72, 1-2 Potter (7, 120, 9-10 L.)})
\end{exe}

However, as was the case with καταπλάσσω \textit{kataplassō}, in the HC κλύζω \textit{klyzō} may also be found in intransitive structures\is{transitivity}, with unergative value and the meaning of ‘make a lavage’, ‘make an enema’, see (\ref{ex:klyzw2}). 

\begin{exe}
\ex\label{ex:klyzw2}
\glll Τῷ Παρμενίσκου παιδὶ κωφότης.   \\ 
\textit{Tō} \textit{Parmeniskou} \textit{paidi} \textit{kōphotēs}.   \\
\textsc{art.dat.sg} Parmeniscus.\textsc{gen.sg} child.\textsc{dat.sg} deafness.\textsc{nom.sg}   \\

\glll Ξυνήνεγκε μὴ κλύζειν, διακαθαίρειν δὲ \\
\textit{Xynēnenke} \textit{mē} \textit{klyzein}, \textit{diakathairein} \textit{de} \\
help.\textsc{aor.ind.3sg} not make.a.lavage.\textsc{prs.inf} clean.\textsc{prs.inf} \textsc{prt} \\

\glll εἰρίῳ μοῦνον. \\
\textit{eiriō} \textit{mounon}. \\
wool.\textsc{dat.sg} only.\textsc{adv} \\
\glt ‘Parmeniscus’ child, deafness. It was helpful not to make any lavage, and only clean with wool instead.' \\
\hspace*{\fill}(\iwi{HC \textit{Epidemics V} 66. 1-2, p. 30, 8-10 Jouanna–Grmek (5, 244, 4-5 L.)})
\end{exe}

This standardized activity can also be referred to by the deverbal nouns κλυσμός \textit{klysmos}\footnote{\citet[146--147]{Chantraine1933} states that the suffix -μός \textit{-mos} tends to be employed for creating nouns referring to actions, rather than referring to objects. \citet[152]{Civilleri2012} observes however that «il tipo di processo denotato dai nomi in -μός \textit{-mos} è più definito e ciò ne favorisce la lessicalizzazione come nomi concreti».} and κλύσμα \textit{klysma}, which can be used as nouns referring to actions, see (\ref{ex:klyzw3}). \largerpage

\begin{exe}
\ex\label{ex:klyzw3}
\glll κλυσμῶν ἀπηλλάχθαι πάντων, πλὴν οἴνου καὶ ὕδατος. \\ 
\textit{klysmōn} \textit{apēllakhthai} \textit{pantōn}, \textit{plēn} \textit{oinou} \textit{kai} \textit{udatos}. \\
douche.\textsc{gen.pl} abstain.\textsc{prf.inf} all.\textsc{gen.pl} except wine.\textsc{gen.sg} and water.\textsc{gen.sg} \\
\glt ‘Abstain from any douche except of wine and water.' \\
\hspace*{\fill}(\iwi{HC \textit{Diseases of Women II} 115, p. 280, 9-10 Potter (8, 250, 14-15 L.)})
\end{exe}

The predicative nature of these nouns makes it possible to insert them in an SVC, and, again, the chosen verb is χράομαι \textit{khraomai}. Other than the equivalence with a synthetic verb, a typical test to prove that a structure is in fact an SVC is the possibility of the noun acting as a predicate by activating an argument structure which can codify the same information as that of the synthetic verb (\citealt[345--346]{Gross2004}; \citealt[181--182]{langer2004linguistic}; \citealt{JiménezLópez2011,JiménezLópez2012}). As observed by \citet{JiménezLópez2012}, when the synthetic verb is transitive\is{transitivity}, the equivalent SVC tends to codify the direct object as an objective genitive. This is exactly what happens between (\ref{ex:klyzw4}) and (\ref{ex:klyzw5}).\footnote{The form κατάκλυσμα \textit{kataklysma} is very rare. It is only employed in passage (\ref{ex:klyzw5}) of \textit{Nature of man}, in subsequent commentaries on this passage by Galen and in two passages of Oribasius (4th c. AD) and Stephanus (6th/7th c. AD).}


\begin{exe}
\ex\label{ex:klyzw4}
\glll τὴν κοιλίην κλύζειν χυλῷ \\ 
\textit{tēn} \textit{koiliēn} \textit{klyzein} \textit{khulō}  \\
\textsc{art.acc.sg} cavity.\textsc{acc.sg} make.an.enema.\textsc{prs.inf} juice.\textsc{dat.sg}  \\

\glll πτισάνης ἢ μέλιτι.  \\
\textit{ptisanēs} \textit{ē} \textit{meliti}. \\
barley.gruel.\textsc{gen.sg} or honey.\textsc{dat.sg} \\
\glt ‘Make an enema to her cavity with barley gruel or honey.' \\
\hspace*{\fill}(\iwi{HC \textit{Diseases of Women I} 26, p. 72, 24-25 Potter (8, 70, 16 L.)})
\end{exe}

\begin{exe}
\ex\label{ex:klyzw5}
\glll Τοῖσι δὲ ἐμέτοισι χρὴ καὶ τοῖσι \textbf{κατακλύσμασι} τῆς κοιλίης ὧδε \textbf{χρῆσθαι}. \\ 
\textit{Toisi} \textit{de} \textit{emetoisi} \textit{khrē} \textit{kai} \textit{toisi} \textit{kataklysmasi} \textit{tēs} \textit{koiliēs} \textit{hōde} \textit{khrēsthai}. \\
\textsc{art.dat.pl} \textsc{prt} emetic.\textsc{dat.pl} should.\textsc{prs.ind.impers} and \textsc{art.dat.pl} enema.\textsc{dat.pl} \textsc{art.gen.sg} cavity.\textsc{gen.sg} thus.\textsc{adv} use.\textsc{prs.inf} \\
\glt ‘Emetics and enemas for the cavity should be thus used.' \\
\hspace*{\fill}(\iwi{HC \textit{Nature of Man} 20, p. 212, 1-2 Jouanna (= \textit{Salubr}. 5; 6, 78, 3-4 L.)})
\end{exe}

However, example (\ref{ex:klyzw6}) shows a different structure, with κοιλίη \textit{koiliē} inserted in a prepositional phrase. \citet{langer2004linguistic} argues that a misalignment between the argument structure of the synthetic verb and that of the SVC may be evidence in favour of a slight difference in meaning between the two. \citet[174--175]{Marini2010} analyses the coding of the “indirect object” in a prepositional phrase as the result of the process of intransitivization\is{transitivity} which, according to her, is undergone by SVCs with ποιέομαι \textit{poiéomai}, as opposed to similar constructions with ποιέω \textit{poieō}. In this case, however, it must be borne in mind that κλύζω \textit{klyzō} is subject to a locative alternation\footnote{On locative alternations in general, see \citet[350--351]{Levin1993}. For Archaic and Classical Greek, see \citet[540--541]{DelaVilla2017}.}, since, while in the HC the target of the motion of the liquid substance is codified as a direct object, in Homer (but see also \iwi{Euripides, \textit{Hippolitus}, 653-654} [tragedy, drama]), it is originally inserted into a prepositional phrase (see note 26).\largerpage[1.5]\footnote{Note that the compound form διακλύζω \textit{diaklyzō} is also used in the HC with the substance injected as a lavage (the Theme) rather than with the part of the body which must be “cleaned” (the Target) as a direct object, showing the alternating nature of the verb: \iwi{HC \textit{Epidemics V} 67, p. 30, 14 Jouanna–Grmek (5, 244, 8 L.)}: Καστόριον καὶ πέπερι διακλυζομένη ὠφελεῖτο \textit{Kastorion kai peperi diakluzomenē ōpheleito}; castorium.\textsc{acc.sg} and pepper.\textsc{acc.sg} inject.\textsc{prs.ptcp.mid.nom.sg} help.\textsc{impf.ind.impers} ‘She got help when she injected castorium and pepper (\textit{scil}. in her mouth)’.}

Moreover, the statement of \citet{Marini2010} is not relevant with reference to SVCs with χράομαι \textit{khraomai}, which does not have an active counterpart as ποιέομαι \textit{poieomai} does.


\begin{exe}
\ex\label{ex:klyzw6}
\glll \textbf{κλυσμῷ} κατὰ κοιλίην \textbf{χρῆσθαι} διὰ τρίτης ἡμέρης. \\ 
\textit{klysmō} \textit{kata} \textit{koiliēn} \textit{khrēsthai} \textit{dia} \textit{tritēs} \textit{hēmerēs}. \\
enema.\textsc{dat.sg} to cavity.\textsc{acc.sg} use.\textsc{prs.inf} through third.\textsc{gen.sg} day.\textsc{gen.sg} \\
\glt ‘Make an enema for the cavity every other day.' \\
\hspace*{\fill}(\iwi{HC \textit{Regimen in Acute Diseases (Appendix)} 2. III 1, p. 69, 17 Joly (2, 398, 12 L.)})
\end{exe}


κλυσμός \textit{klysmos} is however also employed to refer more concretely to the liquid used in the therapy expressed by κλύζειν \textit{klyzein}, which thus becomes a κλυσμός \textit{klysmos}. In (\ref{ex:klyzw7}) the predicative force is held by κλύζειν \textit{klyzein} itself, and κλυσμός \textit{klysmos}, while acting as a sort of dative CO\is{dative object}, refers to a well quantified liquid substance. 

\begin{exe}
\ex\label{ex:klyzw7}
\glll Κλύζειν δέ, ἢν δέῃ,  \\ 
\textit{klyzein} \textit{de}, \textit{ēn} \textit{deē},  \\
make.a.douche.\textsc{prs.inf} \textsc{prt}, if be.necessary.\textsc{prs.sbjv.impers}, \\

\glll κλυσμῷ πλέον ἢ δυσὶ κοτύλαις. \\
\textit{klysmō} \textit{pleon} \textit{ē} \textit{dysi} \textit{kotylais}. \\
douche.\textsc{dat.sg} more.\textsc{adv} than two.\textsc{dat.pl} cotyle.\textsc{dat.pl} \\
\glt ‘Make a douche, if it is required, with a douche of more than two cotyles.' \\
\hspace*{\fill}(\iwi{HC \textit{Nature of Women} 33. 29, p. 46, 8-9 Bourbon (7, 370, 11-12 L.)})
\end{exe}

κλυσμός \textit{klysmos} and κλύσμα \textit{klysma} may be considered two COs of κλύζω \textit{klyzō} in its medical sense: they refer to the whole class of objects that must be involved in the therapeutic action\is{therapeutics} expressed by the verb. Once a liquid substance is employed in an action expressed by κλύζω \textit{klyzō}, it becomes a κλυσμός \textit{klysmos} or a κλύσμα \textit{klysma}. This is why one may posit an equivalence between the combination of the verb with a nominalized adjective in the dative case, see (\ref{ex:klyzw8}), and the combination of χράομαι \textit{khraomai} and κλυσμός \textit{klysmos} in the dative, combined with the same modifier\is{dative object}, see (\ref{ex:klyzw9}). 

\begin{exe}
\ex\label{ex:klyzw8}
\glll Ἢν ἑλκωθέωσι σφοδρῶς, αἷμα καὶ πῦον καθαίρεται […] κλύζειν \\ 
\textit{Ēn} \textit{helkōtheōsi} \textit{sphodrōs}, \textit{aima} \textit{kai} \textit{puon} \textit{kathaireitai} […] \textit{klyzein} \\
if ulcerate.\textsc{aor.sbjv.pass.3pl} vehemently.\textsc{adv} blood.\textsc{acc.sg} and pus.\textsc{acc.sg} clean.\textsc{prs.ind.pass.3sg} […] make.a.douche.\textsc{prs.inf}  \\

\glll δριμέσι καὶ μαλθακοῖσι καὶ στρυφνοῖσιν … \\
\textit{drimesi} \textit{kai} \textit{malthakoisi} \textit{kai} \textit{stryphnoisin} … \\
acrid.\textsc{dat.pl} and emollient.\textsc{dat.pl} and astringents.\textsc{dat.pl} \\
\glt ‘If (sc. the uterus) becomes very ulcerated, blood and pus will be discharged […] make a douche with acrid, emollient, astringent douches...' \\
\hspace*{\fill}(\iwi{HC \textit{Diseases of Women I} 65, p. 138, 22-28 Potter (8, 134, 9-14 L.)})
\end{exe}

\begin{exe}
\ex\label{ex:klyzw9}
\glll \textbf{κλύσματι} δὲ μαλθακῷ \textbf{χρησαμένῳ}  \\ 
\textit{klysmati} \textit{de} \textit{malthakō} \textit{khrēsamenō}  \\
enema.\textsc{dat.sg} \textsc{prt} emollient.\textsc{dat.sg} use.\textsc{aor.ptcp.dat.sg}  \\

\glll ἔληξεν ἡ ὀδύνη. \\
\textit{elēxen} \textit{hē} \textit{odynē}. \\
stop.\textsc{aor.ind.3sg} \textsc{art.nom.sg} pain.\textsc{nom.sg} \\
\glt ‘His pain was relieved when he used an emollient enema.' \\
\hspace*{\fill}(\iwi{HC \textit{Epidemics V} 73. 5, p. 33, 12-13 Jouanna–Grmek (5, 246, 19-20 L.)})
\end{exe}

It must however be noted that it is not mandatory for the noun in the SVC to be combined with a modifier: κλύσματι χρῆσθαι \textit{klysmati khrēsthai} can be used with the same meaning shown by κλύζειν \textit{klyzein} in its intransitive use\is{transitivity}, as shown in (\ref{ex:klyzw10}).

\begin{exe}
\ex\label{ex:klyzw10}
\glll ἢν δὲ ἡ γαστὴρ μὴ ὑποχωρέῃ,  \\ 
\textit{ēn} \textit{de} \textit{hē} \textit{gastēr} \textit{mē} \textit{hypokhōreē},  \\
if \textsc{prt} \textsc{art.nom.sg} cavity.\textsc{nom.sg} not withdraw.\textsc{prs.sbjv.3sg}  \\

\glll \textbf{κλύσματι} \textbf{χρῆσθαι} ἢ βαλάνῳ. \\
\textit{klysmati} \textit{khrēsthai} \textit{ē} \textit{balanō}. \\
enema.\textsc{dat.sg} use.\textsc{prs.inf} or suppository.\textsc{dat.sg} \\
\glt ‘If the belly does not pass anything, use an enema or a suppository.' \\
\hspace*{\fill}(\iwi{HC \textit{Affections} 14, p. 24, 11-12 Potter (6, 222, 2-3 L.)})
\end{exe}

\section{προστίθημι \textit{prostithēmi} and πρόσθετον \textit{prostheton}}

In the area of gynecology προστίθημι \textit{prostithēmi} develops the special meaning of ‘applying vaginal suppositories’, which are consequently referred to as πρόσθετα \textit{prostheta}.\footnote{The word is found both with proparoxytone and oxytone accentuation. I use the proparoxytone form, while reproducing the accentuation chosen by the editor in direct quotations from the Hippocratic text.} As happened with καταπλάσσω \textit{kataplassō} and κλύζω \textit{klyzō}, προστίθημι \textit{prostithēmi} may be used intransitively\is{transitivity} with this special sense.\footnote{Note that this kind of process, which I called ‘semantic specialisation’ in \citet{Squeri2023}, also involved the SVC προσέχω τὸν νοῦν \textit{prosekhō ton noun} (‘pay attention’), which, in the evolution from Classical to Modern Greek, became so standardised that now προσέχω \textit{prosekhō} alone can express this meaning.} To deal with a flux some fumigations and the application of suppositories are prescribed in (\ref{ex:prost1}).\footnote{The link of προστιθέναι \textit{prostithenai} with ὁκόσα ξηραίνει \textit{hokosa xērainei} as an anaphoric object (on which, see \citealt{Luraghi2003}) is very unlikely since in the whole corpus of the gynecological treatises a prescription for drying suppositories is never found. Suppositories were mostly used for purging, irritating, and emollient purposes.} This second action is, however, expressed by προστίθημι \textit{prostithēmi} alone. 

\begin{exe}
\ex\label{ex:prost1}
\glll Ἢν ῥόος ἐγγένηται […] ὑποθυμιῆν  \\ 
\textit{Ēn} \textit{rhoos} \textit{eggenētai} […] \textit{hypothymiēn}  \\
if flux.\textsc{nom.sg} develop.\textsc{aor.sbjv.3sg} […] fumigate.\textsc{prs.inf}  \\

\glll ὁκόσα ξηραίνει καὶ  \\
\textit{hokosa} \textit{xērainei} \textit{kai}  \\
\textsc{rel.indf.nom.pl} dry.\textsc{prs.ind.3sg} and  \\

\glll προστιθέναι. \\
\textit{prostithenai}. \\
apply.suppositories.in.the.vagina.\textsc{prs.inf} \\
\glt ‘If a flux occurs […] Fumigate from below with drying agents and apply vaginal suppositories.' \\
\hspace*{\fill}(\iwi{HC \textit{Nature of Women} 90. 1, p. 78, 12-14 Bourbon (7, 408, 18-20 L.)})
\end{exe}

This happens because, when used in gynecology, this verb has a predefined object in the logical structure of the action it expresses, a suppository, which is therefore called πρόσθετον \textit{prostheton} and acts as a sort of CO. This is why, exactly as has been observed for κλυσμός \textit{klysmos} and κλύσμα \textit{klysma} in the previous section, the direct combination of the verb with any form of nominalized modifier is equivalent to the use of the same modifier with πρόσθετον \textit{prostheton}. Examples (\ref{ex:prost2}) and (\ref{ex:prost3}) are parallels of the same clinical case in two gynecological writings.\footnote{On the presence of parallels between the writings \textit{Nature of Women} and \textit{Diseases of Women}, see \citet[xii–xvi]{Bourbon2008}.} In the first one προσθεῖναι \textit{prostheinai} is combined with the relative clause ἃ μὴ δήξεται \textit{ha mē dēxetai}, which apparently acts as an argument relative clause\footnote{On the classification of relative clauses in Ancient Greek, see \citet*[378--379]{CrespoContiMaquieira2003}.}, but, since the action has the class of suppositories as a predefined object, it actually narrows the type of suppositories to be applied to non-irritating ones.

\begin{exe}
\ex\label{ex:prost2}
\glll ἔπειτα πυριήσας τὰς ὑστέρας οἴνῳ […] προσθεῖναι ἃ μὴ δήξεται. \\ 
\textit{epeita} \textit{pyriēsas} \textit{tas} \textit{hysteras} \textit{oinō}  […] \textit{prostheinai} \textit{ha} \textit{mē} \textit{dēxetai}. \\
after.\textsc{adv} foment.\textsc{aor.ptcp.nom.sg} \textsc{art.acc.pl} uterus.\textsc{acc.pl} wine.\textsc{dat.sg} […] apply.\textsc{aor.inf} \textsc{rel.pl.n} not bite.\textsc{fut.ind.3sg} \\
\glt ‘After fomenting the uterus with wine […] apply non-irritating suppositories.' \\
\hspace*{\fill}(\iwi{HC \textit{Nature of Women} 14. 3, p. 18, 6-7 Bourbon (7, 332, 10-11 L.)})
\end{exe}

\begin{exe}
\ex\label{ex:prost3}
\glll ἔπειτα πυριῆσαι καὶ καταιονᾶν τὰς ὑστέρας τῷ σὺν τῇ δάφνῃ, καὶ προστιθέναι προσθετὸν καθαρτήριον ὃ μὴ  \\ 
\textit{epeita} \textit{pyriēsai} \textit{kai} \textit{kataionan} \textit{tas} \textit{hysteras} \textit{tō} \textit{syn} \textit{tē} \textit{daphnē}, \textit{kai} \textit{prostithenai} \textit{prostheton} \textit{kathartērion} \textit{ho} \textit{mē}  \\
after.\textsc{adv} foment.\textsc{prs.inf} and moisten.with.liquid.\textsc{prs.inf} \textsc{art.acc.pl} uterus.\textsc{acc.pl} \textsc{art.dat.sg} with \textsc{art.dat.sg} laurel.\textsc{dat.sg} and apply.\textsc{prs.inf} vaginal.suppository.\textsc{acc.sg} cleaning.\textsc{dat.sg} \textsc{rel.nom.sg} not  \\

\glll δήξεται. \\
\textit{dēxetai}. \\
bite.\textsc{fut.ind.3sg} \\
\glt ‘Then foment and moisten the uterus with a preparation of laurel, and apply a cleaning, non-irritating vaginal suppository.' \\
\hspace*{\fill}(\iwi{HC \textit{Diseases of Women II} 131, p. 312, 2-4 Potter (8, 278, 22-280, 1 L.)})
\end{exe}

Being the predefined object involved in a certain action, πρόσθετον \textit{prostheton} has thus both a referential\is{referentiality} and a predicative meaning, even though it never refers to the action itself, as κλυσμός \textit{klysmos} did in (\ref{ex:klyzw3}).\footnote{Greek deverbal nouns in -τον \textit{-ton} normally refer to concrete arguments of the action and not to the action itself \citep[180--181]{Civilleri2012}.} This is why, when combined as a predicative phrase with χράομαι \textit{khraomai} and its dative object\is{dative object}, πρόσθετον \textit{prostheton} is the element that defines the action to be realised, see (\ref{ex:prost4}).

\begin{exe}
\ex\label{ex:prost4}
\glll θερμῷ ὕδατι αἰονᾶν, καὶ φαρμάκοισι θερμαίνουσι \textbf{χρῆσθαι} \textbf{προσθετοῖσι}. \\ 
\textit{thermō} \textit{hydati} \textit{aionan}, \textit{kai} \textit{pharmakoisi} \textit{thermainousi} \textit{khrēsthai} \textit{prosthetoisi}.\\
hot.\textsc{dat.sg} water.\textsc{dat.sg} foment.\textsc{prs.inf} and medication.\textsc{dat.pl} warm.\textsc{prs.ptcp.dat.pl} use.\textsc{prs.inf} vaginal.suppository.\textsc{dat.pl} \\
\glt ‘Foment with hot water, and use warming medications as vaginal applications (or ‘as vaginal suppositories’).' \\
\hspace*{\fill}(\iwi{HC \textit{Places in Man} 47. 7, p. 78, 23-24 Joly (6, 346, 16-17 L.)})
\end{exe}

In (\ref{ex:prost5}) προσθέτοισι δριμέσι \textit{prosthetoisi drimesi} refers to the sharp suppositories as concrete therapeutic objects\is{therapeutics}, but, in combination with χράομαι \textit{khraomai}, the syntagm expresses the same action that, in parallel to this same passage in \textit{Nature of Women} in (\ref{ex:prost6}), is conveyed by the verb προστίθημι \textit{prostithēmi} in combination with τὰ δριμέα \textit{ta drimea} as a nominal adjective.  

\begin{exe}
\ex\label{ex:prost5}
\glll Ἢν δὲ ὑγρότερον ᾖ τὸ στόμα τῶν ὑστερέων […] προσθέτοισι δὲ \\ 
\textit{Ēn} \textit{de} \textit{hygroteron} \textit{ē} \textit{to} \textit{stoma} \textit{tōn} \textit{hystereōn} […] \textit{prosthetoisi} \textit{de} \\
if \textsc{prt} moist.\textsc{comp.nom.sg} be.\textsc{prs.sbjv.3sg} \textsc{art.nom.sg} mouth.\textsc{nom.sg} \textsc{art.gen.pl} uterus.\textsc{gen.pl} […] vaginal.suppository.\textsc{dat.pl} \textsc{prt}  \\

\glll δριμέσι χρήσθαι. \\
\textit{drimesi} \textit{khrēsthai}. \\
sharp.\textsc{dat.pl} use.\textsc{prs.inf} \\
\glt ‘If the mouth of a woman’s uterus is too moist […] employ sharp suppositories.' \\
\hspace*{\fill}(\iwi{HC \textit{Diseases of Women I} 18, p. 60, 7-9 Potter (8, 58, 3-4 L.)})
\end{exe}

\begin{exe}
\ex\label{ex:prost6}
\glll Ἢν ὑγρότερον τοῦ καιροῦ τὸ στόμα τῶν ὑστερέων ᾖ, προστιθέναι τὰ δριμέα. \\ 
\textit{Ēn} \textit{hygroteron} \textit{tou} \textit{kairou} \textit{to} \textit{stoma} \textit{tōn} \textit{hystereōn} \textit{ē}, \textit{prostithenai} \textit{ta} \textit{drimea}.\\
if moist.\textsc{comp.nom.sg} \textsc{art.gen.sg} due.measure.\textsc{gen.sg} \textsc{art.nom.sg} mouth.\textsc{nom.sg} \textsc{art.gen.pl} uterus.\textsc{gen.pl} be.\textsc{prs.sbjv.3sg} apply.\textsc{prs.inf} \textsc{art.acc.pl} sharp.\textsc{acc.pl} \\
\glt ‘If the mouth of a woman’s uterus is moister than it should be, apply sharp substances as a suppository (= apply sharp suppositories).' \\
\hspace*{\fill}(\iwi{HC \textit{Nature of Women} 24. 1, p. 25, 5-6 Bourbon (7, 342, 6-7 L.)})
\end{exe}

\section{Preliminary conclusions}
In the HC χράομαι \textit{khraomai} combines with deverbal nouns, which can sometimes refer to a therapeutic activity\is{therapeutics}, but mostly refer to the type of medical device involved in that activity. The structure seems to be equivalent to the use of the verbs from which the nouns are derived, both in their intransitive\is{transitivity}, see (\ref{ex:prost1}), and in their transitive uses. In this second case, the argument structure of the synthetic verb may appear with the noun, see (\ref{ex:klyzw4}), (\ref{ex:klyzw5}), (\ref{ex:klyzw6}), but, most of the time, the noun simply combines with those modifiers that are otherwise combined with the verb as neuter adjectives or as relative clauses with argument value. If one considers the entities signified by these nouns a predefined argument of the action expressed by the specialised sense of the verb from which they are derived,  any restriction applied to this class of entities, which recur as COs or as the nominal part of an SVC, equals a restriction on the action expressed by the verb. The modification of a CO or of the predicative noun in an SVCs is normally equivalent to the adverbial modification of the action signified by the synthetic verb.\footnote{For COs, see \citet[305]{HuddlestonPullum2002} for English, \citet[287]{horrocksstavrou2010} and \citet[103]{Bruno2011} for AG. For predicative nouns in SVCs, see, among many others, \citet[181--182]{langer2004linguistic}, for modern languages, \citet[156]{Marini2010} and \citet[197--198]{JiménezLópez2016} for AG.}

The syntagms in which χράομαι \textit{khraomai} is combined with κατάπλασμα \textit{kataplasma}, κλυσμός \textit{klysmos} and κλύσμα \textit{klysma}, and πρόσθετον \textit{prostheton}, however, do not involve abstract predicative nouns, which refer to actions that can be thought of as modified adverbially, but mostly concern concrete objects included in specific actions. Therefore, the equivalence is not between the adjectival modification of the CO or of the nominal part of the SVC and the adverbial modification of the synthetic verb, but between the adjectival restriction of the class represented by the CO or by the nominal part of the SVC and the combination of the verb with the same nominalized adjective. In this latter case, the restriction still applies to the class of objects whose involvement in the action expressed by the verb is mandatory.

Another question to be answered is that of the role that χράομαι \textit{khraomai} plays in the structure. Is it correct to consider it an SV or does it have its full meaning, by which it prescribes the ‘use of an instrument’? It is indeed true that, since these nouns refer to concrete medical devices, the verb could simply prescribe their use in medical practice. If one applies to example (\ref{ex:HCkataplasma9}) the so-called ‘zeugma test’\footnote{\citet[179]{langer2004linguistic}: *“he gives a lecture and a lot of money”.}, according to which a verb cannot be used with both light and full value when linked to two coordinated arguments, only one of which is predicative, a predicative value must be either given to κηρωταῖς \textit{kērōtais} or denied to καταπλάσμασι \textit{kataplasmasi}. 

Considering the objects referred to by these nouns instruments would also be in line with the fact that, as far as καταπλάσσω \textit{kataplassō} and κλύζω \textit{klyzō} are concerned, the substance that must be employed in the therapeutic action\is{therapeutics} is often codified in the dative\is{dative object}. However, this does not apply to προστίθημι \textit{prostithēmi} and to some uses of καταπλάσσω \textit{kataplassō}. Moreover, it must be noted that the medical devices referred to by these deverbal nouns become an instrument, but their use in the action requires their change of state, which takes place in the way prescribed by the verbal stem from which they are derived: καταπλάσματα \textit{kataplasmata} must be ‘spread over’ the body, κλυσμoί \textit{klysmoi} and κλύσματα \textit{klysmata} must be ‘injected’, and πρόσθετα \textit{prostheta} must be ‘applied’. This is not canonical for dative objects\is{dative object}, whose coding in the dative has the exact purpose of underlining how the object takes part in the action without undergoing any change of state \citep[66--67]{Luraghi2010}. It can therefore be provisionally noted that χράομαι \textit{khraomai} is not involved in this structure with its full meaning, which, however, is not completely bleached either. χράομαι \textit{khraomai} can therefore be considered an SV only by accepting the more flexible definition presented in Section 1, which assumes that SVCs are characterized by the sharing of the predicative power between the SV and the noun.

Further and stronger evidence in favour of the interpretation of χράομαι \textit{khraomai} as an SV will be given in the next section.

\section{Foods and drinks}
Hippocratic medicine considered diet and lifestyle an important factor to prevent and to cure certain diseases. The following of a diet is often expressed by the dative διαίτῃ \textit{diaitē} combined with χράομαι \textit{khraomai}. CG also has the synthetic verb διαιτάω \textit{diaitaō}, mostly used in the middle-passive form, to express the same action expressed by διαίτῃ χράομαι \textit{diaitē khraomai}, see (\ref{ex:food1}) and (\ref{ex:food2}). 

\begin{exe}
\ex\label{ex:food1}
\glll ἡσυχάζειν \textbf{διαίτῃ} μαλθακῇ \textbf{χρώμενον}  \\ 
\textit{hēsukhazein} \textit{diaitē} \textit{malthakē} \textit{khrōmenon} \\
rest.\textsc{prs.inf} diet.\textsc{dat.sg} emollient.\textsc{dat.sg} use.\textsc{prs.ptcp.acc.sg}  \\

\glll {(Cornarius : -ος θΜ)}. \\
{(Cornarius : -ος θΜ)}. \\
{(Cornarius : -ος θΜ)} \\
\glt ‘Have him rest and employ\footnote{In this example, as well as in examples (\ref{ex:food3}), (\ref{ex:food10}) and (\ref{ex:food12}), I decided to follow the choice made by the translators in the Loeb collection of translating χράομαι \textit{khraomai} as ‘employ', since it renders transparently the meaning of the verb.} an emollient diet.' \\
\hspace*{\fill}(\iwi{HC \textit{Diseases III} 2, p. 72, 7-8 Potter (7, 120, 15-16 L.)})
\end{exe}

\begin{exe}
\ex\label{ex:food2}
\glll Σκόπᾳ […] φλαύρως διαιτηθέντι ἡ κοιλίη ἀπελήφθη.\\ 
\textit{Skopa} […] \textit{phlaurōs} \textit{diaitēthenti} \textit{hē} \textit{koiliē} \textit{apelēphthē}.\\
Scopas.\textsc{dat.sg} […] badly.\textsc{adv} follow.a.diet.\textsc{aor.ptcp.pass.dat.sg} \textsc{art.nom.sg} cavity.\textsc{nom.sg} block.\textsc{aor.ind.pass.3sg} \\
\glt ‘Scopas […] from the following of a poor diet his bowels were seized.' \\
\hspace*{\fill}(\iwi{HC \textit{Epidemics II} 3, 11, p. 56, 12-14 Smith (5, 112, 9-10 L.)})
\end{exe}


δίαιτα \textit{diaita} is a noun referring to a process and is not referential\is{referentiality}. However, in order to be on a certain diet one needs to eat certain foods and drink certain drinks, see (\ref{ex:food3}). This assumption can justify, at least from the semantic point of view, the extension of the structure with χράομαι \textit{khraomai} to nonpredicative nouns such as ποτόν \textit{poton} (‘drink’) and σῖτος \textit{sitos} (‘food’). Such structures make more evident the role of χράομαι \textit{khraomai} as an SV, since the action thus expressed does not entail the “use” of its dative object\is{dative object}, which is however a concrete element which could ideally be involved in such an action (see \textit{infra} example \ref{ex:concl1}).

\begin{exe}
\ex\label{ex:food3}
\glll τοῖσι ποτοῖσι καὶ σίτοισι χρήσθω  \\ 
\textit{toisi} \textit{potoisi} \textit{kai} \textit{sitoisi} \textit{khrēsthō} \\
\textsc{art.dat.pl} drink.\textsc{dat.pl} and food.\textsc{dat.pl} use.\textsc{prs.imp.3sg}  \\

\glll μαλθακοῖσι. \\
\textit{malthakoisi}. \\
emollient.\textsc{dat.pl} \\
\glt ‘She should employ emollient drinks and food.' \\
\hspace*{\fill}(\iwi{HC \textit{Nature of Women} 25. 1, pp. 25, 17-26, 18 Bourbon (7, 342, 16 L.)})
\end{exe}

Here one can see a first step towards the use of the structure with χράομαι \textit{khraomai} as an SV in combination with nouns that do not predicate an event\is{event structure} or a process in any way. ποτόν \textit{poton} is a deverbal form from πίνω \textit{pinō} and retains some predicative force, but this does not apply to σῖτος \textit{sitos}. It is also clear that the action expressed by the SVC does not involve the employment of these substances as tools, but implies their change of state or, more precisely, their consumption. 

However, both ποτόν \textit{poton} and σῖτος \textit{sitos} can be considered nouns referring to a ‘class’ or to a ‘genus’ of substances: drinks and foods. Therefore, they are not fully referential either\is{referentiality} \citep[293--295]{Givón1978}.

ποτόν \textit{poton} is not the only deverbal noun referring to drinks used in this structure. Another  noun frequently combined with χράομαι \textit{khraomai} is ῥόφημα \textit{rhophēma}, which refers to a particular type of liquid gruel that was to be sipped by the patient. This is why it is derived from the verb ῥοφέω \textit{rhopheō} (‘sip’). It thus refers to an argument of the action expressed by ῥοφέω \textit{rhopheō}, while maintaining some predicative force. 

In (\ref{ex:food4}) ῥόφημα \textit{rhophēma} constitutes the nominal part of ῥοφήμασι χρεέσθω \textit{rhophēmasi khreesthō}, used to place the action of administering the gruel in a temporally ordered sequence, in which it precedes that of giving food. The same temporal collocation in a sequence can be observed in (\ref{ex:food5}), which, instead of the structure with χράομαι \textit{khraomai}, shows the use of the synthetic form ῥοφέω \textit{rhopheō}, employed as an unergative. 

\begin{exe}
\ex\label{ex:food4}
\glll τοῖσι \textbf{ῥοφήμασι} πρόσθεν \textbf{χρεέσθω} τοῦ  \\ 
\textit{toisi} \textit{rhophēmasi} \textit{prosthen} \textit{khreesthō} \textit{tou} \\
\textsc{art.dat.pl} gruel.\textsc{dat.pl} before use.\textsc{prs.imp.3sg} \textsc{art.gen.sg}  \\

\glll σίτου. \\
\textit{sitou}. \\
food.\textsc{gen.sg} \\
\glt ‘Let him use gruels before food.' \\
\hspace*{\fill}(\iwi{HC \textit{Internal Affections} 9, p. 100, 3-4 Potter (7, 188, 5 L.)})
\end{exe}

\begin{exe}
\ex\label{ex:food5}
\glll μηδὲ ῥοφεῖν μηδὲ πίνειν ταχὺ μετὰ τὸ λουτρόν. \\ 
\textit{mēde} \textit{rhophein} \textit{mēde} \textit{pinein} \textit{takhy} \textit{meta} \textit{to} \textit{loutron}.\\
nor sip.\textsc{prs.inf} nor drink.\textsc{prs.inf} right.\textsc{adv} after \textsc{art.acc.sg} bath.\textsc{acc.sg} \\
\glt ‘Gruels or drinks must not be taken soon after a bath.' \\
\hspace*{\fill}(\iwi{HC \textit{Regimen in Acute Diseases} 18. LXV. 3, p. 66, 2-3 Joly (2, 368, 2-3 L.)})
\end{exe}

Since they refer to a category of objects on which the action encoded in their own name must be performed, ποτόν \textit{poton} and ῥόφημα \textit{rhophēma} can also be used as predicative phrases, in connection with fully referential nouns\is{referentiality} which constitute the dative object of χράομαι \textit{khraomai}\is{dative object}. In example (\ref{ex:food6}), this structure is used to express the fact that the sipping of a πτισάνη \textit{ptisanē} (‘barley infusion') may result in excessive fullness: ῥοφήματι \textit{rhophēmati} defines the type of action in which the barley infusion is involved. In example (\ref{ex:food7}), the exact same action is expressed by the synthetic verb ῥοφεέτω \textit{rhopheetō}. 

\begin{exe}
\ex\label{ex:food6}
\glll Εἰ μέντοι \textbf{ῥοφήματι} \textbf{χρέοιτο} πτισάνῃ […] ἄγαν πλησμονῶδες ἂν εἴη. \\ 
\textit{Ei} \textit{mentoi} \textit{rhophēmati} \textit{khreoito} \textit{ptisanē} […] \textit{agan} \textit{plēsmonōdes} \textit{an} \textit{eiē}.\\
if however gruel.\textsc{dat.sg} use.\textsc{prs.opt.3sg} barley.infusion.\textsc{dat.sg} […] too.much.\textsc{adv} filling.\textsc{nom.sg} \textsc{prt} be.\textsc{prs.opt.3sg} \\
\glt ‘If, however, he uses a barley infusion as a gruel […] it will cause fullness.' \\
\hspace*{\fill}(\iwi{HC \textit{Regimen in Acute Diseases} 15. LVI. 3, p. 60, 22-23 Joly (2, 346, 6-7 L.)})
\end{exe}

\begin{exe}
\ex\label{ex:food7}
\glll μετὰ τὴν κάθαρσιν πτισάνης δύο τρυβλία ῥοφεέτω.\\ 
\textit{meta} \textit{tēn} \textit{katharsin} \textit{ptisanēs} \textit{duo} \textit{tryblia} \textit{rhopheetō}.\\
after \textsc{art.acc.sg} cleaning.\textsc{acc.sg} barley.infusion.\textsc{gen.sg} two bowl.\textsc{acc.pl} sip.\textsc{prs.imp.3sg} \\
\glt ‘After the cleaning, let him sip two bowls of barley infusion.' \\
\hspace*{\fill}(\iwi{HC \textit{Internal Affections} 13, p. 116, 13-14 Potter (7, 200, 13-14 L.)})
\end{exe}

The same happens with ποτόν \textit{poton}. Example (\ref{ex:food8}) prescribes the use of water as the drink for recovering from a fracture, and the avoidance of wine. This information is conveyed with χράομαι \textit{khraomai}, ὕδατι \textit{hydati} and οἴνῳ \textit{oinō} as dative objects\is{dative object}, and ποτῷ \textit{potō} as a predicative phrase. The same action is expressed in (\ref{ex:food9}) by the simple πίνω \textit{pinō}, with οἶνον \textit{oinon} and ὕδωρ \textit{hydōr} as direct objects.

\begin{exe}
\ex\label{ex:food8}
\glll \textbf{ποτῷ} δὲ \textbf{χρῆσθαι} ὕδατι, καὶ μὴ οἴνῳ.\\ 
\textit{potō} \textit{de} \textit{khrēsthai} \textit{hydati}, \textit{kai} \textit{mē} \textit{oinō}.\\
drink.\textsc{dat.sg} \textsc{prt} use.\textsc{prs.inf} water.\textsc{dat.sg}, and not wine.\textsc{dat.sg} \\
\glt ‘For drink use water and not wine.' \\
\hspace*{\fill}(\iwi{HC \textit{Fractures} 11, p. 21, 4 Jouanna–Anastassiou–Roselli (3, 458, 8-9 L.)})
\end{exe}

\begin{exe}
\ex\label{ex:food9}
\glll μηδ’ οἶνον πινέτω ἀλλὰ μάλιστα μὲν ὕδωρ. \\ 
\textit{mēd’} \textit{oinon} \textit{pinetō} \textit{alla} \textit{malista} \textit{men} \textit{hydōr}.\\
not wine.\textsc{acc.sg} drink.\textsc{prs.imp.3sg} but mostly.\textsc{adv} \textsc{prt} water.\textsc{acc.sg} \\
\glt ‘He should not drink wine, but preferably water. ' \\
\hspace*{\fill}(\iwi{HC \textit{Diseases II} 72. 2, p. 212, 7-8 Jouanna (7, 110, 10-11 L.)})
\end{exe}

σῖτος \textit{sitos} can be found in the exact same function. In example (\ref{ex:food10}), for instance, a diet based on barley cakes is prescribed by a structure with χράομαι \textit{khraomai}, μάζῃ \textit{mazē} as a dative object\is{dative object} and σίτῳ \textit{sitō} as a predicative phrase.

\begin{exe}
\ex\label{ex:food10}
\glll \textbf{σίτῳ} δὲ \textbf{χρήσθω} μάζῃ μαλθακῇ ἀτρίπτῳ.\\ 
\textit{sitō} \textit{de} \textit{khrēsthō} \textit{mazē} \textit{malthakē} \textit{atriptō}.\\
food.\textsc{dat.sg} \textsc{prt} use.\textsc{prs.imp.3sg} barley.cake.\textsc{dat.sg} soft.\textsc{dat.sg} unkneaded.\textsc{dat.sg} \\
\glt ‘As food let him employ soft unkneaded barley-cake.' \\
\hspace*{\fill}(\iwi{HC \textit{Internal Affections} 51, p. 244, 9-10 Potter (7, 294, 10-11 L.)})
\end{exe}

Unlike ποτόν \textit{poton} and ῥόφημα \textit{rhophēma}, however, σῖτος \textit{sitos} is not deverbal. Nevertheless – and this is a crucial point – it still refers to a category into which referential objects\is{referentiality} may fall on the basis of their involvement in a typical action: that of being eaten.\footnote{The importance of the semantic traits which give information about the typical action into which an object is involved or typically used has been underlined by \citet[76--81]{Pustejovsky1995}, who labelled them as part of the ‘telic quale’, one of the four main “sections" (the “qualia") in which semantic traits that are involved in generative transformations of meaning can be divided.} According to \citet{HaleKeyserES1987,halekeyser1993}, verbs like ‘eat’, which can be used either as unergatives (“John ate") or transitively (“John ate the bread")\is{transitivity} have a form of “internal” predefined object. This is why, if the action is mentioned without the need to better specify the type of food eaten, the verb can be used intransitively: if taken as a whole, the involvement of the class of ‘food’, that could recur as an argument, is already fully identified by the meaning of the verb. This may be the reason why, if used as a predicative phrase in an SVC with χράομαι \textit{khraomai}, a noun such as σῖτος \textit{sitos} identifies the action of eating, as ἐσθίω does in (\ref{ex:food11}): it refers to the class of objects that is part of the logical structure of this action.

\begin{exe}
\ex\label{ex:food11}
\glll Τὸ δὲ λοιπὸν τοῦ χρόνου  \\ 
\textit{To} \textit{de} \textit{loipon} \textit{tou} \textit{khronou}  \\
\textsc{art.acc.sg} \textsc{prt} remaining.\textsc{acc.sg} \textsc{art.gen.sg} time.\textsc{gen.sg}   \\

\glll διαιτάσθω μᾶζαν καὶ ἄρτον  \\
\textit{diaitasthō} \textit{mazan} \textit{kai} \textit{arton}  \\
follow.a.diet.\textsc{prs.imp.3sg} barley.cake.\textsc{acc.sg} and bread.\textsc{acc.sg}  \\

\glll ἐσθίων ἀμφότερα. \\
\textit{esthiōn} \textit{amphotera}. \\
eat.\textsc{prs.ptcp.nom.sg} both.\textsc{acc.pl} \\
\glt ‘From then on, let the regimen include eating both barley-cakes and bread.' \\
\hspace*{\fill}(\iwi{HC \textit{Internal Affections} 12, p. 114, 6-7 Potter (7, 198, 14 L.)})
\end{exe}

Example (\ref{ex:food4}) already proves that σῖτος \textit{sitos} holds some predicative power: it contains a prescription for the use of gruels ‘before food’ – πρόσθεν τοῦ σίτου \textit{prosthen tou sitou} – where the category of ‘food’ is used with a temporal value to refer to the action of eating. “Before food” stands for “before eating” and temporality is one of the characteristics that is taken into account to test the predicative force of a noun \citep[48--50]{SimonePompei2007}.

The HC,  however, contains some passages in which χράομαι \textit{khraomai} refers to the action of eating and drinking in combination with fully referential nouns\is{referentiality}, which refer to concrete foods and drinks, without σῖτος \textit{sitos} and ποτόν \textit{poton} – of which they are hyponyms – as predicative phrases.

\begin{exe}
\ex\label{ex:food12}
\glll οἴνῳ δὲ μέλανι χρήσθω, τοῖσι κρέασιν ὀπτοῖσι μᾶλλον ἢ ἑφθοῖσι.\\ 
\textit{oinō} \textit{de} \textit{melani} \textit{khrēsthō}, \textit{toisi} \textit{kreasin} \textit{optoisi} \textit{mallon} \textit{ē} \textit{hephthoisi}.\\
wine.\textsc{dat.sg} \textsc{prt} black.\textsc{dat.sg} use.\textsc{prs.imp.3sg}, \textsc{art.dat.sg} meat.\textsc{dat.pl} roasted.\textsc{dat.pl} more than boiled.\textsc{dat.pl} \\
\glt ‘She should employ dark wine, roasted meats in preference to boiled ones.' \\
\hspace*{\fill}(\iwi{HC \textit{Diseases of Women I} 11, p.  48, 23-24 Potter (8, 48, 5-7 L.)})
\end{exe}

Even if they are not predicative nouns, οἴνῳ \textit{oinō} and κρέασιν \textit{kreasin} are still the fundamental element for identifying the action expressed. Even though referential\is{referentiality}, they are part of superordinate classes of objects, as those of drinks and food. Being part of the logical structure of a certain action, this superordinate class still identifies the typical action in which its hyponyms are involved. Since the entities represented by these nouns are more specific than the whole class, however, were this action to be expressed by a synthetic verb, they should recur as arguments of the verb in a transitive structure\is{transitivity}, see (\ref{ex:food13}).

\begin{exe}
\ex\label{ex:food13}
\glll κρέας δὲ ἐσθιέτω ἀλέκτορος ὀπτὸν  \\ 
\textit{kreas} \textit{de} \textit{esthietō} \textit{alektoros} \textit{opton} \\
meat.\textsc{acc.sg} \textsc{prt} eat.\textsc{prs.imp.3sg} cock.\textsc{gen.sg} roasted.\textsc{acc.sg}  \\

\glll ἄναλτον, ἢ αἰγὸς ἑφθόν. \\
\textit{analton}, \textit{ē} \textit{aigos} \textit{hephthon}. \\
not.salted.\textsc{acc.sg} or goat.\textsc{gen.sg} boiled.\textsc{acc.sg} \\
\glt ‘He should eat roasted fowl meat without salt, or boiled goat meat.' \\
\hspace*{\fill}(\iwi{HC \textit{Internal Affections} 1, p. 72, 26-74, 1 Potter (7, 168, 8-9 L.)})
\end{exe}

The fact that a referential noun\is{referentiality} such as οἶνος \textit{oinos} can produce a predicative structure if combined with χράομαι \textit{khraomai} can also be proved by a ‘zeugma test’ (see supra). In (\ref{ex:food14}) χράομαι \textit{khraomai} is linked to the predicative noun λουτροῖσι \textit{loutroisi}, ‘baths’, with which it prescribes a therapy with baths, and to οἴνοισι γλυκέσιν \textit{oinoisi glykesin}. οἴνοισι \textit{oinoisi} must therefore hold the same predicative value as λουτροῖσι \textit{loutroisi}. 

\begin{exe}
\ex\label{ex:food14}
\glll Θεραπεύειν δὲ χρὴ τὰς πλευρίτιδας ὧδε […] \textbf{λουτροῖσί} τε \textbf{χρῆσθαι} θερμοῖσι καὶ οἴνοισι γλυκέσιν.\\ 
\textit{Therapeuein} \textit{de} \textit{khrē} \textit{tas} \textit{pleuritidas} \textit{hōde} […] \textit{loutroisi} \textit{te} \textit{khrēsthai} \textit{thermoisi} \textit{kai} \textit{oinoisi} \textit{glykesin}.\\
cure.\textsc{prs.inf} \textsc{prt} should.\textsc{prs.ind.impers} \textsc{art.acc.pl} pleurisy.\textsc{acc.pl} thus.\textsc{adv} […] bath.\textsc{dat.pl} and use.\textsc{prs.inf} hot.\textsc{dat.pl} and wine.\textsc{dat.pl} sweet.\textsc{dat.pl} \\
\glt ‘You must treat pleurisies as follows […] you must use warm baths and sweet wines.' \\
\hspace*{\fill}(\iwi{HC \textit{Diseases III} 16, p. 90, 9-11 Potter (7, 146, 13-15 L.)})
\end{exe}

\section{Conclusions}
\largerpage
χράομαι \textit{khraomai} is often linked with a dative object\is{dative object} which holds the predicative force of the verb phrase, being the element that identifies the type of action to be realised. The nouns that occur in that position are more or less close to the traditional definition of predicative noun. The closer ones are κατάπλασμα \textit{kataplasma}, κλυσμός \textit{klysmos} and κλύσμα \textit{klysma}, which refer both to the action expressed by the verb from which they are derived and to one of the arguments that takes part in that action. πρόσθετον \textit{prostheton}, ποτόν \textit{poton} and ῥόφημα \textit{rhophēma} are still deverbal nouns, but they only refer to one of the arguments of the action expressed by the verb from which they are derived. The action in which they are involved is however still inscribed in their own meaning, and this explains why they hold some predicative force. 

Moving further from the core of predicative nouns one finds σῖτος \textit{sitos}, which is not deverbal, but refers to a category which can be understood as semantically involved in the logical structure of the action of ‘eating’. This action is thus the one recalled by its combination with χράομαι \textit{khraomai}. Finally, this structure can also involve fully referential nouns\is{referentiality}, whose predicative force lies in the fact that they are hyponyms of a superordinate class of objects involved in the logical structure of a precise action, like those of `eating' and `drinking'.

The further one moves from deverbal and predicative nouns, the more χράομαι \textit{khraomai} deviates from its full value, since it does not express the action of `using as an instrument' the concrete referents of referential nouns\is{referentiality}, which would indeed be suitable for such an interpretation. If the verb maintained its full value in combination with nouns such as οἶνος \textit{oinos}, it would express the action of using wine as a tool, as it happens with water in example (\ref{ex:concl1}), which recommends the use of water while changing the dressing of a wound.

\begin{exe}
\ex\label{ex:concl1}
\glll Ἐν δὲ ἑκάστῃ τῶν ἐπιλυσίων ὕδατι πολλῷ θερμῷ χρέεσθαι.\\ 
\textit{En} \textit{de} \textit{hekastē} \textit{tōn} \textit{epilysiōn} \textit{hydati} \textit{pollō} \textit{thermō} \textit{khreesthai}.\\
at \textsc{prt} each.\textsc{dat.sg} \textsc{art.gen.pl} change.of.dressing.\textsc{gen.pl} water.\textsc{dat.sg} plenty.\textsc{dat.sg} warm.\textsc{dat.sg} use.\textsc{prs.inf} \\
\glt ‘At each change of dressing use plenty of warm water.' \\
\hspace*{\fill}(\iwi{HC \textit{Fractures} 10, p. 17, 21-18,1 Jouanna–Anastassiou–Roselli (3, 452, 4-5 L.)})
\end{exe}

The meaning of the expression ὕδατι…χρέεσθαι \textit{hydati…khreesthai} is far different from that activated by χράομαι \textit{khraomai} in examples such as (\ref{ex:food12}), in which it prescribes the ‘drinking’ of wine and not its use for other purposes. 

In the HC χράομαι \textit{khraomai} combines with deverbal nouns that refer to objects which can be conceived as therapeutic tools\is{therapeutics} as far as they are involved in the change of state prescribed by the verb from which they are derived. This link of χράομαι \textit{khraomai} with objects whose function as an instrument involves their change of state is then extended to other non-deverbal nouns, which refer either to a class of objects or to a member of such a class. This class is the one which typically undergoes a change of state in the logical structure of the action expressed by the synthetic verb which is equivalent to the SVC. The potential referentiality of these nouns\is{referentiality} rules out the use of ποιέω \textit{poieō}, which would take its full meaning, prescribing the ‘production’ of the object signified by the noun. χράομαι \textit{khraomai} is thus employed to express the interaction with these objects, realised by acting on them as is typical for the class to which they belong. 

%attempting to fix new hbox issue
\emergencystretch 3em 
It must also be noted that the choice of χράομαι \textit{khraomai} may also be in line with the medical purpose of underlining that these objects are functional to the healing of the patient as much as the employment of a therapeutic tool\is{therapeutics}. The use of this verb as an SV seems indeed to be far more frequent in the HC than in other writings (see also \textit{supra} ex. (\ref{ex:food14}): λουτροῖσί... χρῆσθαι, \textit{loutroisi ... khrēsthai}, ‘take baths'). \citet{JiménezLópez2011}, for instance, registers as standard the SVC δίαιταν ποιέομαι \textit{diaitan poieomai}, while the HC counts only two potential occurrences of this structure, compared to 25 occurrences of διαίτῃ χράομαι \textit{diaitē khraomai}.\footnote{\iwi{HC \textit{Regimen} 68, 198, 26-27 Joly (6, 602, 1-2 L.)}; \iwi{HC \textit{Diseases of Women} I 11, p. 48, 17 Potter (8, 46, 24-48, 1 L.)}.} While dealing with regimen, Hippocratic writings show a special tendency to express everyday practices, such as walking, with predicative nouns in combination with χράομαι \textit{khraomai}. Expressions such as περιπάτοις χράομαι \textit{peripatois khraomai} (‘take walks’) appear 20 times in the HC, while being almost absent from other writings of the Classical Period.\footnote{Only two occurrences of περιπάτῳ χράομαι \textit{peripatō khraοmai} can be found in \iwi{Xenophon, \textit{Oeconomicus} 11} [Socratic dialogue, prose])} This shift is certainly very interesting for studies focusing on changes induced on SVCs by register variation, but goes beyond the scope of this paper, whose focus is on Hippocratic SVCs involving nouns with a potentially referential meaning\is{referentiality}.





\section*{Abbreviations}
\begin{tabularx}{.5\textwidth}{@{}lQ@{}}
CO & Cognate object \\
\textsc{comp} & comparative (of adjectives) \\
HC & Hippocratic corpus \\
\end{tabularx}%







%\section*{Contributions}
%John Doe contributed to conceptualization, methodology, and validation. 
%Jane Doe contributed to writing of the original draft, review, and editing.

\sloppy
\printbibliography[heading=subbibliography,notkeyword=this]
\end{document}
