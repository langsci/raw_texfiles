%\chapter*{}
\renewcommand{\thesection}{\arabic{section}}


One of the best resources for the development of literacy and literary practices in a language is a comprehensive dictionary of the language informed by sound linguistic analysis. Dagaare has quite a lot of resources including grammars \citep{Bodomo1997, Bodomo2000, Dakubu2005}, many word lists and specialized dictionaries \citep{Durand1953, Bodomo2004cantonese}, readers \citep{Zakpaa1978} and numerous linguistic analyses in the form of theses, monographs and journal articles. However, there does not exist a comprehensive dictionary that is general and substantial enough for everyday use by native speakers and learners of the language.

The present dictionary therefore fills a major void in Dagaare studies. In this foreword, I provide some background information about the Dagaare language and Dagaare studies. I also present a sketch of the grammar of the language, based largely on my grammar sketch in \citet{Bodomo2000}.\footnote{I thank Lincom Europe for permission to use parts of that work in this grammar sketch.} All this is intended to assist the user to understand some basic aspects of the grammar towards understanding the various entries, including grammatical categories, and nominal and verbal categories and groupings. 

\section{The Dagaare language and Dagaare studies}\label{sec:dagaarestudies}
In terms of demographic and socio-political importance, Dagaare (also written variously as \textit{Dagaari}, \textit{Dagara}, and \textit{Birifor}) is one the prominent languages of West Africa, spoken in northwestern Ghana and adjacent areas of Burkina Faso and C\^{o}te d’Ivoire. It is broken down into various variants across these three countries: in Ghana it is mainly called \textit{Dagaare}, in Burkina Faso, it is mainly called \textit{Dagara} and in C\^{o}te d’Ivoire it is mainly called \textit{Birifor}. It is spoken by between two and three million people in all these countries, with far more speakers in Ghana than in the other two countries.
In terms of genetic relationship, it belongs to the Mabia (also known as Gur, although this term is dispreferred by many Dagaare-speaking linguists) group of languages of the Niger-Congo language family, along with other languages such as Dagbane, Waale, Gurenne, Mampruli, Kusaal, Buli, Kasem, Sisaali and Moore.

Dagaare is being vigorously studied and used both as a language of education at lower levels of the educational ladder and as a school subject at higher levels of the ladder. Indeed, in universities and colleges of higher education in Ghana and Burkina Faso, Dagaare programs have some of the highest enrollments. We can indeed talk of an emerging field of Dagaare Studies in the sense that, not only is there vibrancy at all these levels of education in teaching programs, Dagaare is also the subject of vigorous research by linguists and other scholars in West Africa (Ghana, Burkina Faso and C\^{o}te d'Ivoire) and  other parts of the world. Dagaare books and articles have appeared in some of the best linguistics journals (such as, \textit{Natural Language and Linguistic Theory}, \textit{Lingua}, and \textit{Studia Linguistica}) and some of the most prominent book publishing houses such as Stanford University CSLI Press, and Oxford University Press. There is even an existing (but currently dormant) \textit{Journal of Dagaare Studies}. What all this indicates is that a comprehensive user-friendly dictionary is long overdue in this area of study. A Daagare-English dictionary is therefore a most welcome resource. In the next section, I will introduce some of the major features of the structure of Dagaare towards understanding the dictionary entries in the book.


\section{Phonology}

\subsection{Orthography and sound system of Dagaare}

A note on the orthography of the language is necessary for an understanding of the written forms of the entries. The standard orthography, developed by the Catholic Church in Ghana and now widely used in educational institutions, is based on the Central dialect but there are several alternative orthographies (e.g.\  \citealt{Nakuma1999,Nakuma2002}). All of these are basically phonemic. Dagaare is a two-tone language, but tone is not marked in the standard orthography. Table \ref{tab:DagaareAlphabet} shows the standard Dagaare alphabet, which has 31 graphemes, comprising 24 single-letter graphemes (representing 17 consonants and 7 vowels), 6 digraphs and 1 trigraph (\tabref{tab:DagaareAlphabet}).

%\ex. \label{tab:alphabet} 
\begin{table}[t]
\begin{tabular}{llll}
A, a   & as in &  \textit{bàná lá wààná} &  ‘It is they who are coming’\\
B, b & as in & \textit{bàà} &  ‘dog’ \\ 
D, d & as in &  \textit{dúní} &  ‘knee’ \\ 
E, e & as in &  \textit{kpéré} &  ‘to cut up’\\ 
ɛ, ɛ & as in & \textit{gɛ̀rɛ́} &  ‘going’ \\ 
F, f & as in & \textit{fànfánè} &  ‘soap’ \\ 
G, g & as in & \textit{gánè} &  ‘book’ \\ 
GB, gb & as in &  \textit{gbɛ́rè} &  ‘leg’ \\ 
GY, gy & as in &  \textit{gyìlé} &  ‘xylophones’ \\ 
H, h & as in &  \textit{húólí} &  ‘to mock at someone’ \\ 
I, I & as in &  \textit{bìbììrí} &  ‘children’ \\ 
K, k & as in &  \textit{kànnè} &  ‘to read’ \\ 
KP, kp & as in & \textit{kpááré} &  ‘occiput’ \\ 
KY, ky & as in &  \textit{kyɛ́ngé} &  ‘to walk’ \\ 
L, l & as in & \textit{láá} &  ‘bowl’ \\ 
M, m & as in &  \textit{má} &  ‘mother’ \\ 
N, n & as in &  \textit{néɛ̀} &  ‘person’ \\ 
NG, ng & as in &  \textit{bòngó} &  ‘donkey’ \\ 
NY, ny & as in & \textit{nyɛ́} &  ‘to see’ \\ 
NGM, ngm & as in &  \textit{ngméǹ} &  ‘God’ \\ 
O, o & as in &  \textit{zòró} &  ‘running’\\ 
Ɔ, ɔ & as in &  \textit{sɔ́wɔ́lɔ́} &  ‘kind of dish’ \\ 
P, p & as in &  \textit{pɛ̀nnè} &  ‘to rest’ \\ 
R, r & as in &  \textit{pùrì} &  ‘to burst’ \\ 
S, s & as in &  \textit{sénsɛ́} &  ‘cakes’ \\ 
T, t & as in &  \textit{tùòrì} &  ‘to meet’ \\ 
U, u & as in &  \textit{dùndúló} &  ‘worms’ \\ 
V, v & as in &  \textit{vóóróng} &  ‘breath, life’ \\ 
W, w & as in &  \textit{wááó} &  ‘snake’ \\ 
Y, y & as in &  \textit{yánngáá} &  ‘grandchild’ \\ 
Z, z & as in &  \textit{zàgá} &  ‘pen’ \\ 
\end{tabular}
\caption{{Standard Dagaare Alphabet}}\label{tab:DagaareAlphabet}
\end{table}

\largerpage
\subsection{Consonants}

Dagaare has twenty-five consonants and two glides (semi-consonants) in underlying
representation. This is shown in \tabref{tab:Consonants}.
\begin{table}
    
\resizebox{\textwidth}{!}{
\begin{tabular}{l@{}ccccccccc}
\lsptoprule
&          & labio- &      &          &alveo-&         &    &labial-&  \\
& bilabial & dental& dental & alveolar &palatal&palatal & velar &velar& glottal \\
\midrule
Stops:  \\
 \hspace{.3pc}voiceless &p & & t &&&& k & kp & \textglotstop \\
\hspace{.3pc}voiced &b & &d&&&& g &gb& \\

\tablevspace
Affricates: \\
\hspace{.3pc}voiceless &&&&& ky [ʧ] &&&& \\
\hspace{.3pc}voiced &&&&&gy [dz]&&&&\\

\tablevspace
Fricatives: \\
\hspace{.3pc}voiceless &&f&& s&&&&& h \\
\hspace{.3pc}voiced &&v&& z  \\

\tablevspace
Lateral &&&&l  \\

\tablevspace
Nasals &m&&& n && \textltailn && \ng m & \ng \\
% \tablevspace
% implosive & \texthmlig &&&‘l &&&&& \texthth  \\

\tablevspace
Glides &&&&&&y [j]&& w      \\
\lspbottomrule
\end{tabular}}
\caption{Consonants}
    \label{tab:Consonants}
\end{table}



 





The glottal implosive counterparts of /h/, /l/ and /m/ are attested only in the Northern dialect of the language (spoken mostly in Burkina Faso and called Dagara); the Southern and Central dialects do not have them. Two additional consonants [r] and [ɣ] are found at surface level, occurring as allophones to /d/ and /g/ at initial positions and intervocalically, respectively.
Consider the following words:



\ea{ [dì] `to eat', [dìré] ‘eating’, [górí] ‘dowry’, [pɔ́ɣɔ́] ‘woman’}
\z


In these examples [d] and [g] occupy the word initial position. The sounds [r] and [ɣ], on the other hand, occur intervocalically. Voiceless plosives are usually aspirated when they occur in word initial position. The phonetic transcriptions: [tʰàllɪ̀] ‘to walk fast’ and [pʰàllɪ̀] ‘to weave’ illustrate aspiration.
As can be seen on the chart, some of the consonant phonemes such as /kp/, /gb/, and /\ng m/ have
a dual place of articulation. \largerpage What the term ``labial-velar'' means is that the sound involved is produced simultaneously with the velum and the lips as active organs of sound production.  These sounds are known as double articulations or co-articulations. Examples are /\ng mámá/ ‘calabashes’ and /gbɛ́rɪ́/ ‘cripple’.

\subsection{Vowels}
Nine oral vowel phonemes may be established for Dagaare and these are shown in   \figref{tabfig:Vowels}.

\begin{figure}
%
% \begin{tabular}{lllllllllllll}\lsptoprule
% Front & &&&&&&&&&&&Back \\\midrule
% High  \\
%  $+$\textsc{atr} &i & &  &&&&&  &  &&&u \\
%  $-$\textsc{atr} & & ɪ &&&&&&  &&&ʊ& \\
% \tablevspace
%
% Mid-High \\
%  $+$\textsc{atr} &&&e&&  &&&&&o&& \\
%
% \tablevspace
% Mid-Low \\
%  $-$\textsc{atr} &&&& ɛ&&&&&ɔ&&&  \\
% \tablevspace
% Low &&&&&&a&\hspace{1em}&&  \\\lspbottomrule
% \end{tabular}

\begin{tikzpicture}
\aeiouEO
\node at (0,2.5) {ɪ};
\node at (3,2.5) {ʊ};
\node at (-3.5,2.75) [align=left,text width=2cm] {high};
\node at (-3.5,2)    [align=left,text width=2cm] {mid-high};
\node at (-3.5,1)    [align=left,text width=2cm] {mid-low\vphantom{g}};
\node at (-3.5,0)    [align=left,text width=2cm] {low\vphantom{g}};
\node at (-2,3)   [align=left,text width=1cm,align=right] {$+$\textsc{atr}\vphantom{g}};
\node at (-2,2.5) [align=left,text width=1cm,align=right] {$-$\textsc{atr}\vphantom{g}};
\node at (-2,2)   [align=left,text width=1cm,align=right] {$+$\textsc{atr}\vphantom{g}};
\node at (-2,1)   [align=left,text width=1cm,align=right] {$-$\textsc{atr}\vphantom{g}};

\node at (0,3.5) (front) {Front};
\node at (3,3.5) (back) {Back};

\draw (front.south west)--(back.south east) ;


\end{tikzpicture}


\caption{Vowels}
    \label{tabfig:Vowels}

\end{figure}


\newpage
The + and -- signs in the chart show that the relevant distinctive features, High, Low, Round and \textsc{atr} (Advanced Tongue Root), are present or absent, respectively. These are then the basic vowel phonemes of Dagaare, but processes such as vowel lengthening, nasality, harmony and sequencing make the system a bit more complex. Each of the above  nine short vowels has a long counterpart, and the opposition of short and long vowels of the same
quality brings about differences in meaning, as is manifest in the following minimal pairs:



\ea /tɔ̌r/ ‘-self’ : /tɔ̀ɔ́r/ ‘far’\\
/kǔr/ ‘tortoise’ : /kùúr/ ‘hoe’\\
/bà/ ‘fix to the ground’ : /bàà/ ‘grow up’
\z



On the basis of this evidence, then, we may say that vowel length is phonemic in Dagaare. In addition, each of the nine oral vowels may be nasalized. 

An interesting type of co-occurrence restriction in Dagaare involves the distinctive feature [±\textsc{atr}]. In a Dàgáàrè phonological word, only vowels of the same \textsc{atr} value can occur. The following will briefly illustrate the point:


\ea \label{ex:vowelharmony}/dìré/ ‘eating’ : +\textsc{atr} vowels\\
/dɪ́rɛ́/ ‘taking’ : -\textsc{atr} vowels\\
/púò/ ‘farmland’ : +\textsc{atr} vowels\\
/pʊ́ɔ́/ ‘stomach’ : -\textsc{atr} vowels\\
\z

This co-occurrence restriction involving both contiguous and non-contiguous vowels is called vowel harmony.

\subsection{Tone}
Dagaare is a register tone language with two levels of tone, high and low, and a
downstepped high unit. Downstep is said to occur when in a given phonological unit the
second of two high tones (with relatively low tone in between them) happens not to be as
 high as the first high tone \citep{Kennedy1966}. However, downstep can also occur with two successive high tones. The data in 
(\ref{ex:tonecontrast}) are words showing high and low tonal contrasts:

\eabox{\label{ex:tonecontrast} \begin{tabular}{ll}
\textit{tú} ‘to dig' &\textit{tù} ‘to follow’  \\
\textit{dá} ‘to push (many items)’ & \textit{dà} ‘to buy’\\
\textit{nɔ́ŋ} ‘to massage’ &\textit{nɔ̀ŋ} ‘to like, love’
\end{tabular} 
}
\subsection{Phonological processes}

Various types of phonological processes occur in the language. I shall
outline three types here. These are vowel harmony, vowel assimilation, and vowel elision.

\subsubsection{Vowel harmony}
One of the most pervasive phonological processes is vowel harmony, as has been
described above. Besides vowel co-occurrence restriction based on [±\textsc{atr}], as in (\ref{ex:vowelharmony}), there is
also a minor vowel co-occurrence restriction based on roundness, [±round]. This happens
mainly with the imperfective suffix /rɔ/ with mid vowels which must be in total harmony
with root vowels in Central Dagaare. The following examples in (\ref{ex:imperfective}) illustrate this
phonological process.


\eabox{\label{ex:imperfective}
\begin{tabular}{lll}
zò ‘run’&  $\rightarrow$ & zò-ró ‘running’\\
ngmɛ̀ ‘beat, kick’  & $\rightarrow$  & ngmèɛ̀-rɛ́ ‘beating, kicking’\\
gbè ‘grind roughly’& $\rightarrow$ & gbìè-ré ‘grinding roughly’\\
dɔ̀ɔ̀ ‘squat’&  $\rightarrow$ &   dɔ̀ɔ̀-rɔ́ ‘squatting’\\
\end{tabular}
}


\subsubsection{Vowel assimilation}
While vowel harmony may involve both contiguous and non-contiguous vowels,
vowel assimilation occurs only with contiguous vowels, that is, vowels without any intervening consonants, though they may belong to different syllables or words. I illustrate this phonological process with the following quantifier NPs, involving a noun and a numeral, as shown in (\ref{ex:assimilation}).

\ea \label{ex:assimilation} 
\ea{ \gll wàà-ré áyì $\rightarrow$  wàà-rá áyì\\
yam-{\PL} two\\
\glt ‘two (tubers of) yam'}
\ex{ \gll bóóré átà $\rightarrow$ bóó-rá átà \\
goat-{\PL} three \\
\glt  ‘three goats’}
\ex {\gll kyúú-rì ànáárè $\rightarrow$ kyúú-rà ànáárè \\
month-{\PL} four \\
\glt ‘four months’}
\z\z 







\subsubsection{Final vowel elision/apocope}
Final vowel elision is a pervasive phonological process in Dagaare and other Gur languages (see \citealt{Kennedy1966}, \citealt{AnttilaBodomo1996}, \citealt{Bodomo1997} for Dagaare; \citealt{Rialland1985} for Moba; \citealt{Hyman1993} for Dagbani; and \citealt{Cahill1999} for Konni). In Dagaare,
there is the tendency for the final vowel that is often not a distinctive functional morpheme to
be dropped in various speech situations, such as fast speech, recitations and songs. In (\ref{ex:elision1}),
the final high and mid vowels are elided, thus giving us reduced forms of the words in which
they are found. The examples show broad phonemic transcriptions, not orthography.




\eabox{
\begin{xlist}
\exbox{{\label{ex:elision1} \begin{tabular}{llll} 
/mírì/&  $\sim$ &/míŕ/& ‘rope’\\
/wírí / & $\sim$& /wír/ &‘horse’\\
/bíírí/ & $\sim$& /bíír/& ‘children’\\
/pɪ́ɪ́rɪ̀ /&  $\sim$ &/pɪ́ɪ́r/& ‘sheep’\\
/bóórɪ́/&  $\sim$ &/bóór/& ‘goats’\\
/bírì/ & $\sim$ &/bîr/ &‘seed’\\
/dàgáárɪ̀/&  $\sim$ &/dàgáàr/& ‘Dagaare’\\
/zùmmú/  &$\sim$& /zǔm/& ‘fish’\\
/lɔ́ŋɔ́/  &$\sim$& /lɔ́ŋ/& ‘toad’\\
/sɔ́ŋɔ́/  &$\sim$& /sɔ́ŋ/ &‘wood carrier’\\
/pɔ́g-ɔ́/&  $\sim$ &/pɔ́g/ &‘woman’\\
/tɛ̀gɛ́/  &$\sim$ &/tɛ̌g/& ‘only’\\
\end{tabular}}
\ex {\label{ex:elision2}\begin{tabular}{llll} 
/pɔ́g-bɔ́/&  $\sim$ &*/pɔ́gb/ &\hspace{.7pc}‘women’\\\end{tabular}}
}
\end{xlist}
}

Sometimes, even vowels representing a distinctive functional morpheme, such as the singular
suffix marker in the word for `woman' in (\ref{ex:elision1}) can be elided, with the understanding that the
root word also stands for the singular form of the noun. The final vowel of the item \textit{pɔ́g-bɔ́}
cannot be elided probably because the consonant segment alone cannot indicate the plural
form of the word, and such syllable structure in coda position is forbidden in Dagaare.


\section{Morphology}

This section comprises two main subsections on nominal and verbal morphology, followed by a discussion of other words classes and morphological processes. Under
nominal morphology, I shall outline the main systems in the Dagaare noun phrase, including
number, definiteness/referentiality, and possession. I shall also outline other categories in the
noun phrase, including pronouns, adjectives, numerals. Under verbal morphology, I describe
parts of the verb, including its root, various inflectional and derivational affixes, as well as adverbs and pluractionality.  I then describe in a final section 
ideophones and additional morphological processes, namely reduplication and
compounding.

\subsection{Nominal morphology}
\subsubsection{ The noun}
\paragraph{Number/class}

Dagaare has an intricate system of number suffixes that divides its nouns into about ten classes. Each of the  nouns in  Table  \ref{tab:classstems}
represents a class of nouns.
\begin{table}
    
    \begin{tabularx}{\textwidth}{rXXXl}
    \lsptoprule
Class & Stem & Singular & Gloss &Plural\\\midrule
    1: & pɔ́g-& /pɔ́g-ɔ́/& ‘woman’& /pɔ́gɪ́bɔ́/  \\
2: &zí- &/zí-é/ &‘place’&/zíí-rí/ \\
3:& gyí-& /gyì-lì/& ‘xylophone’& /gyí-lé/\\
4: &pɪ́-& /pɪ́-rʊ́ʊ́/& ‘sheep’& /pɪ́ɪ́-rɪ́/\\
5: &zú-& /zû-Ø/& ‘head’ &/zú-rí/\\
6: &bí-& /bí-rì/ &‘seed’ &/bí-é/\\
7: &gán- &/gá-nɪ̀/ &‘book’ &/gá-má/\\
8: &gbɪ́ngbɪ́l-& /gbɪ́ngbɪ́-áà/& ‘drying spot’& /gbɪ́ngbɪ́l-lɪ́/\\
9: &dí-& /dí-íú/& ‘food’& (no plural)\\
10: & búúl- &(no singular) &‘porridge’& /búúl-úńg/\\\lspbottomrule
    \end{tabularx}
    \caption{Classes of nouns based on singular-plural pairings}
    \label{tab:classstems}
\end{table}

The basic class distinction in Dagaare then for number. One may be able to predict the
singular and plural suffixes of a noun depending on its class. Unlike in Bantu languages,
where linguists have developed a labeled system of classes, there is no recognized
system of labeled classes in the Mabia languages. In \citet{Bodomo1997}, an attempt is made to group Dagaare nouns into a labeled system of classes.  \tabref{tab:classstems} is
based on that classification. There is the need for a large-scale investigation of Mabia languages from a comparative perspective, with a view to establishing a system of labeled classes for the whole sub-group that covers quite a substantial landmass of the West Africa sub-continent (see also  \citealt{Mieheetal2007}). 

\paragraph{Definiteness/Referentiality}
Definiteness of nouns is expressed in Dagaare by placing the article \textit{à} before the noun. Hence:

\ea \ea[]{ \gll à bíé / à bíírí\\
 ‘the child’ / ‘the children’\\}
\ex[]{ \gll à pɔ́gɔ́ / à pɔ́gɔ́bɔ́ \\
 ‘the woman’ / ‘the women’\\}
\ex[]{ \gll à sàkúúrì / à sàkúè \\
 ‘the school’ / ‘the schools’ \\}\z \z


This is illustrated in (\ref{ex:definiteSentences}).

\ea\label{ex:definiteSentences} \ea{ \gll à bíé dé lá à áískèrìɪ̌m dì\\
{\DEF} child take.{\PFV} {\FOC} {\DEF} icecream eat.{\PFV}\\
\glt ‘The child has eaten up the ice cream.’}
\ex{\gll à bíírí gàá lá sàkúúrì zénɛ̀\\
{\DEF} child.{\PL} go.{\PFV} {\FOC} school today\\
\glt ‘The children have gone to school today.’}\z\z 



Indefiniteness is marked by a bare noun, as shown in (\ref{ex:indefiniteSentence1}), or by placing the article \textit{kàngá}
after the bare noun, as in (\ref{ex:indefiniteSentence2}):

\ea\label{ex:indefiniteSentence1}\gll bíé káá-ó wáá lá túó\\
child look-{\NMLZ} be {\FOC} bitter\\
\glt ‘It is difficult to look after a child.’ or
‘Taking care of a child is difficult.’\z 

\ea\label{ex:indefiniteSentence2}\gll pɔ́gɔ́ kàngá dà wá lá kyɛ́\\
woman certain {\PST} come {\FOC} here\\
\glt ‘A certain woman came here.’\z 



When we make reference to a specific item, the prenominal definite article \textit{à}, as seen
above, and a postnominal demonstrative article \textit{ná} are employed alongside the bare noun.
This use of the article to signal specificity is illustrated in (\ref{ex:indefiniteSentence3}).

\ea\label{ex:indefiniteSentence3}\gll à~ bíé ná dà wá lá kyɛ́\\
{\DEF} child that {\PST} come {\FOC} here\\
\glt ‘That child came here.’\z

\paragraph{Case}
Lexical nouns are not overtly marked for case in Dagaare. Even with pronouns, it is only in the first person singular that there is a morphological distinction between the nominative and accusative case forms, as shown in the list of pronouns in \S \ref{sec:pronouns}.

\paragraph{Possession}
The  associative relation is not overtly marked in Dagaare by any affixes. Instead, this is signaled by word order. In an associative construction (genitive/possessive) involving a pronoun, the pronoun occurs before the head noun. Where there are two nouns or noun phrases, the first one is the genitive (modifier) and the second the head. This is illustrated in (\ref{ex:possession}).

\ea \label{ex:possession}
\ea[]{\label{ex:possession1}\gll fò gánè\\
2{\SG} book\\
\glt ‘your book’}
\ex[]{\label{ex:possession2}\gll nààná gánè\\
Naana book\\
\glt ‘Naana’s book’}
\ex[]{\label{ex:possession3}\gll à dɔ̀ɔ̀-fáá ná gàn-vìlàà\\
{\DEF} man-bad that book-good\\
\glt ‘That bad man’s good book.’}
\z \z 


In (\ref{ex:possession1}), the pronoun occurs before the head noun; in (\ref{ex:possession2}) the first noun, \textit{Nààná}, is the
genitive and the second is the head noun. In (\ref{ex:possession3}) we have two noun phrases within the constellation of the genitive construction. The first phrase serves as the genitive to the
second phrase that is the head of the construction.

\subsubsection{Pronouns}\label{sec:pronouns}
Under pronouns, I will briefly discuss personal, demonstrative, reflexive, reciprocal,
relative, possessive, interrogative, and indefinite pronouns.

\paragraph{Personal pronouns}
\begin{table}[t]

    \begin{tabularx}{\textwidth}{Xlll}
\lsptoprule
& \multicolumn{2}{c}{Subject (Nom)}      & Object (Acc)\\
   & Weak form & Strong form       & \\
   \midrule
1st pers. {\SG} & ǹ& màá& mà\\
2nd pers. {\SG} &fò &fòó& fò\\
3rd pers. {\SG} &ò &ónó &ò\\
1st pers. {\PL} &tè &tènéè& tè\\
2nd pers. {\PL}  & yɛ̀ &yɛ̀néè &yɛ̀\\
3rd pers. {\PL} (human) &bà &báná &bà\\
3rd pers. {\PL} (non-human) &à &áná &à\\\lspbottomrule
    \end{tabularx}
    \caption{Dagaare personal pronouns}
    \label{tab:subjectpronouns}
\end{table}
In the Dagaare pronominal system, personal pronouns do not indicate gender
differences, as does English with he and she, and there is hardly any case marking. Only in
the first person singular pronominal paradigm is there a distinction between nominative and
accusative forms, \textit{ǹ} and \textit{ mà}, respectively.
One salient feature in the Dagaare system of pronouns is that, for subject pronouns,
we have a distinction between weak and strong forms, quite like the French, \textit{je} ‘I’ - \textit{moi} ‘me’,
\textit{tu} ‘you’ - \textit{toi} ‘you’ paradigm. Another important feature in this system is the distinction
between human and non-human forms for third person plural pronouns.  \tabref{tab:subjectpronouns}
shows a list of Dagaare personal pronouns.

 







\paragraph{Demonstrative pronouns}

The pronouns in (\ref{ex:dempronouns}) are a few of the demonstrative pronouns in Dagaare. As
with personal pronouns, there is a distinction between human and non-human forms for the
third person plural pronouns.

\eabox{ \label{ex:dempronouns} \begin{tabular}{ll}  
nyɛ̌& ‘this (one)’\\
ónɔ́ng& ‘that (one)’\\
bánàng &‘those (ones) - humans’\\
ánáng &‘those (ones) - non humans’\\
lɛ̌ &‘like that (one)’\\
nyɛ́ɛ̀ &‘like this (one)’
\end{tabular}
}


\paragraph{Reflexives}
The word \textit{mèngɛ́} or \textit{mèngɛ́ tɔ̌r} (singular) and \textit{mènné} or \textit{mènné tɔ̌r} (plural), used
after any of the personal pronouns above, expresses reflexivity in Dagaare. This is tabulated
in \tabref{tab:reflexives}.

\begin{table}
    
    \begin{tabularx}{\textwidth}{Xl}\lsptoprule
Weak reflexive pronouns &  Strong reflexive pronouns \\\midrule
ǹ mèngɛ́ (tɔ̌r) ‘myself’     & màá mèngɛ́ ‘me, myself’\\
fò mèngɛ́  ‘yourself’&fò~ó mèngɛ́ ‘you, yourself’\\
ò mèngɛ́  ‘him/herself’ &  ónɔ́ mèngɛ́ ‘s/he, him/herself’\\
tèmènné (tɔ̌r) ‘ourselves’ &    tènéè mènné! ‘we, ourselves'\\
yɛ̀ mènné ‘yourselves’& yɛ̀néè mènné ‘you, yourselves'\\
bà mènné ‘themselves’& báná mènné ‘they, themselves’ \\
à mènné ‘themselves’&áná mènné ‘they, themselves’\\\lspbottomrule
    \end{tabularx}
    \caption{Reflexives pronouns}
    \label{tab:reflexives}
\end{table}  


    

 
 
 

The constructions in (\ref{ex:reflexives}) illustrate how they are used.


\ea \label{ex:reflexives} \ea[]{ \gll ǹ tòńg lá à tómɔ́ ǹ mèngɛ́ tɔ̌r\\
1{\SG} work {\FOC} {\DEF} work 1{\SG} self self\\
\glt ‘I did the work by myself.’}
\ex[]{\gll màá mèngɛ́ tɔ̌r lá tòng à tómɔ́\\
1{\SG}.{\STR} self self {\FOC} work {\DEF} work\\
\glt ‘It is I that did the work all by myself.’} \z \z



\paragraph{Reciprocal pronouns}
Reciprocal pronouns in Dagaare include \textit{tɔ̌}, \textit{tɔ́ sòbá}, \textit{táá} and \textit{táábá}. However, the
most commonly used is \textit{táá}. Example \REF{ex:reciprocal} illustrates its syntactic position.

\ea[]{\label{ex:reciprocal}
\gll tè nɔ̀nɔ́ lá táá\\
1{\PL} love {\FOC} {\RECP}.{\PRON}\\
\glt ‘We love each other / one another.’}\z 

\paragraph{Relative pronoun}
Dagaare does not distinguish between the human and non-human form of relative
pronouns as the English ‘who’ (for humans) and ‘which’ (for non-humans). For both of these
the relative pronoun is \textit{nàng}. The following are example constructions illustrating the use of
the relative pronoun:

\ea \ea[]{ \gll à dɔ́ɔ́ ná náng wà\\
{\DEF} man {\COMP} {\REL}.{\PRON} come\\
\glt ‘the man who came’}
\ex[]{\gll à gánè ná náng lè\\
{\DEF} book {\COMP} {\REL}.{\PRON} fall\\
\glt ‘the book that fell’}\z\z

\paragraph{Possessive pronouns}
The words \textit{tóɔ́r} and \textit{dèń} ({\SG}) \textit{dèmé} ({\PL}), meaning ‘own’, combined with any of the
personal subject pronouns, express possession in Dagaare. This is illustrated below:

\eabox{ \label{ex:posspronouns} \begin{tabular}{ll} 
n tóɔ́r, děn, dèmé& ‘mine, my own’\\
fò tóɔ́r, děn, dèmé& ‘yours, your own’\\
ò tóɔ́r, děn, dèmé &‘his/hers, his/her own’\\
tè tóɔ́r, děn, dèmé& ‘ours, our own’\\
\end{tabular}
}
\begin{exe}
\sn \begin{tabular}{ll}
yɛ̀ tóɔ́r, děn, dèmé &‘yours , your own’\\
bà tóɔ́r, děn, dèmé &‘theirs, their own (humans)’\\
à tóɔ́r, děn, dèmé& ‘theirs, their own (non-humans)’\\
\end{tabular}
\end{exe}




\ea \ea[]{\gll  tè dèn lá\\
1{\PL} own {\FOC}\\
\glt ‘It is ours.’}
\ex[]{\gll tè dèmé lá\\
1{\PL} own.{\PL} {\FOC}\\
\glt ‘They are ours.’}\z\z

\paragraph{Interrogative pronouns}
The following is a list of interrogative pronouns in the language. Some appear in example sentences to show how they would be used in the language.

\eabox{ \label{ex:interrpronouns} \begin{tabular}{ll}  
bòńg, bòlúù& ‘what’\\
bóò& ‘which one, which of them’\\
bábóò, bábóbò& ‘which of them (humans)’\\
ábóò, ábòbò& ‘which of them (non-humans)’\\
àńg& ‘who (singular, humans)’\\
àńg mìné &‘who (plural, human)’\\
\end{tabular}
}

\ea \label{ex:interrpronounssents} \ea[]{\gll bòng lá ká ó dě wà nè ?\\
what {\FOC} {\COMP} s/he take come with\\
\glt ‘What did s/he bring?’}
\ex[]{\gll bóó lá zèng à bè ?\\
which {\FOC} sit {\DEF} there\\
\glt ‘Which of you/ which of them is sitting there?’}
\ex[]{\gll  bábóbò lá zèng à bè ?\\
which  {\FOC} sit {\DEF} there\\
\glt ‘Which of them are sitting there?’}
\ex[]{\gll  àńg lá àrè à bè ?\\
who {\FOC} stand {\DEF} there\\
\glt ‘Who is standing there?’}
\ex[]{\gll  àńg mìné lá àrè à bè ?\\
who {\PL} {\FOC} stand {\DEF} there\\
\glt ‘Who are standing there?’}\z\z

\paragraph{Indefinite pronouns}



Strictly speaking, Dagaare does not seem to have indefinite pronouns as, for
instance, the French \textit{quelqu’un}, ‘someone’. The situation is more like the English \textit{somebody},
\textit{someone}, etc., where a noun like `body' combines with the item \textit{kàngá}:

\ea \ea[]{\gll néɛ̀ kàngá wà-ɛ̀ lá\\
person {\INDF} come-{\PFV} {\FOC}\\
\glt ‘Someone has come.’}
\ex[]{\gll kàngá sóbɔ́ bé lá kyɛ̌\\
another person be {\FOC} here\\
\glt ‘Someone is here.’}\z\z

\subsubsection{Honorific system}
Dagaare does not have a system of high and low 2nd person pronouns to deploy
respectful address, as we find in some  European languages like French and German,
or as we find in Cantonese and some South-East Asian languages by the choice of classifiers.
However, like Cantonese \citep{Siew-Yue1993}, one way of deploying honour is to bestow a real or
fictitious relationship on people, rather than addressing them by their bare names. So words
like \textit{ǹ bǎ} ‘my father, friend’, \textit{ǹ béɛ́rè} ‘my big brother’ may be used to address people, even if
they are complete strangers, as a sign of respect or honour for them.

\subsubsection{Numerals}
The Dagaare numeral system is a mixed system of decimals and multiples of twenty,
so that 15 is ‘ten and five’, while 40 is ‘two twenties’. Both cardinal and ordinal numerals exist
in the language. \tabref{tab:numerals} shows a list of cardinal numerals from 1 to 10.
\begin{table}
    \begin{tabularx}{\textwidth}{rXXl}
    \lsptoprule
Numerals & Root &(+human count)& (-human count) \\\midrule
1 &-yeni&nényéni & bónyéní \\
2 &-yi&báyì&  áyì\\
3 &-ta&bátà& átà \\
4 & -naare&bànáárè& ànáárè\\
5 &-nuu&bànúú& ànúú\\
6 & -yoɔo&bàyòɔ̀ó& àyòɔ̀ó \\
7 & -yɔpoi&bàyɔ̀pòî& àyɔ̀pòî \\
8  &-nii&bànîî & ànîî \\
9  &-wae&bàwáé& àwáé \\
10 &&(nóbá) píé& (bómá) píé\\\lspbottomrule
\end{tabularx}
    \caption{Numerals}
    \label{tab:numerals}
\end{table}  



The table shows numeral roots and, depending on whether the noun is human or non-human, a
prefix or element precedes the root. Cardinal numerals follow the above regulation, shown in (\ref{ex:cardinalnumerals}), as do ordinal numerals, which add the element \textit{sòbɔ́}  following the numeral, as shown below in (\ref{ex:ordnumerals}). For multiplicative numerals, the element \textit{bo-} is added to the root of the numeral (from 1 to 9), as
shown in (\ref{ex:multnumerals}), although  from 10
onwards, people tend to prefer the expressions such as \textit{gbɛ́ɛ́~ píé}, ‘ten times’, etc., to using the element \textit{bo-}.

\eabox{ 
\begin{xlist}
\exbox{\label{ex:cardinalnumerals}\begin{tabular}{ll} 
dɔ́bɔ́ bànúú& ‘five men’\\
bóóré ànúú& ‘five goats’\\
\end{tabular}
}
\exbox{\label{ex:ordnumerals}\begin{tabular}{ll}
bátà sòbɔ́& ‘the third person’\\
átà sòbɔ́& ‘the third (thing)’ \\\end{tabular}}
\exbox{
{\label{ex:multnumerals}\begin{tabular}{ll} 
bóyì& ‘two times’\\
bótà &‘three times’\\
gbɛ́ɛ̀ píé& ‘ten times’\\
gbɛ́ɛ̀ lèzárè &‘twenty times'\\
\end{tabular}
}
}
\end{xlist}
}

\subsubsection{Adjectives}
An adjective may be used as a verbal predicate (\ref{ex:adjverbpred}), or a predicative with a copular
verb (\ref{ex:adjpredcop}), or in combination with a root of the noun it qualifies to form a compound (\ref{ex:adjcomp}).
Not all adjectives can be used in these three different ways.

\ea \ea[]{\label{ex:adjverbpred} \gll à pɔ́g-ɔ́ vèɛ̀lɛ́ lá\\
{\DEF} woman-{\SG} be.beautiful {\FOC}\\
\glt ‘The woman is beautiful.’}
\ex[]{ \label{ex:adjpredcop} \gll à pɔ́g-ɔ́ é lá vèlàà\\
{\DEF} woman-{\SG} be {\FOC} beautiful\\
\glt ‘The woman is beautiful.’}
\ex[]{\label{ex:adjcomp} \gll  pòg-vèlàà lá\\
woman-beautiful {\FOC}\\
\glt ‘That is a beautiful woman.’}\z\z

The comparative is expressed by the use of verbs of surpassing such
as \textit{gàngè} and \textit{zùò}, as in (\ref{ex:adjcomp2}).

\ea \label{ex:adjcomp2}\ea[]{ \gll à pɔ́g-ɔ́ nyɛ́ vèɛ̀lɛ́ lá gàngè à pɔ́g-ɔ́ ná\\
{\DEF} woman-{\SG} {\DEM} be-beautiful {\FOC} surpass {\DEF} woman-{\SG} {\DEM}\\
\glt ‘This woman is more beautiful than that woman.’}
\ex[]{\gll à pɔ́g-ɔ́ nyɛ́ é lá vèlàá zùò à pɔ̀g-ɔ̀ ná\\
{\DEF} woman-{\SG} {\DEM} be {\FOC} beautiful surpass {\DEF} woman-{\SG} {\DEM}\\
\glt ‘This woman is more beautiful than that woman.’}\z\z


The superlative  is expressed in Dagaare using the adjective, the surpass verb and the
intensifier \textit{à zàá} or \textit{ba zàá}, as in (\ref{ex:adjsuper}).

\ea \label{ex:adjsuper}\ea[]{ \gll à móngó nyɛ́ lá nòmɔ́ gàngè à zàá\\
{\DEF} mango {\DEM} {\FOC} sweet surpass 3{\PL}.{\NONHUMAN} {\INTENS}\\
\glt ‘This mango is the sweetest (of all)'}
\ex[]{\gll  à dɔ́ɔ́ nyɛ́ lá è bɛ́róng gàngè bà zàá\\
{\DEF} man {\DEM} {\FOC} be fat surpass 3{\PL}.{\HUMAN} {\INTENS}\\
\glt ‘This man is the fattest (of all).’}\z\z 

\subsection{Verbal morphology}
In this section we shall outline the verbal morphology of Dagaare, including the main
verb, inflectional and derivational affixes, adverbs and pluractionality. 

\subsubsection{ The verb}
\paragraph{Main verb}
Dagaare does not have what may be called an ``infinitive verb'', as is commonly found
in many Indo-European languages; so it is often better to talk of a dictionary entry or citation
form. In a sentence like: \textit{Ǹ yèlí ká “kúlɪ́” Ǹ bá yèlì ká “à zèngè}” ‘I say “go home”, I did
not say “sit down”’, the citation forms are \textit{à kúlí} or \textit{kúlí} and   \textit{à zèngè} or \textit{zèng}. These full forms may also have incomplete forms: \textit{kúlí}  $\sim$ \textit{kúl}, \textit{zèngè}  $\sim$ \textit{zèng}. It is these full forms
that would be featured as the key entry in any Dagaare dictionary.
In its conjugated form the main verb has three other forms marking perfective and
imperfective aspect. The following forms illustrate:





\eabox{{\label{ex:conjugations}\begin{tabular}{ll} 
\textit{tèɛ̀nè} &dictionary form\\
\textit{tèɛ̀nè}& perfective aspectual form\\
\textit{tèɛ̀n-ɛ́ɛ́}& perfective intransitive aspectual form\\
\textit{tèɛ̀n-nɛ́}& imperfective aspectual form'\\
\end{tabular}
}}


There are basically two aspectual forms in Dagaare: the perfective (completive) and the
imperfective (progressive). These are often expressed by suffixes. They respond to
phonological rules such as vowel harmony and assimilation, taking on the features of the root
of the verb. The perfective aspect, in turn, has two subtypes. The usual type is normally not
morphologically realized. The second type normally occurs when there is no object element
immediately after the verb, hence the term ‘perfective intransitive’ aspectual form. These
aspectual suffixes function in conjunction with the preverbal particles to express temporal,
aspectual, and polarity features. Basically then, the morphology of the main verb is one of a
root followed by a suffix.
The basic system of the Mabia verb is often labeled as aspect--the perfective and
imperfective aspect. It may also be called, according to \citet{Bendor-Samuel1971}, event and
process, punctiliar and linear, etc. In this basic system, the speaker sees the action as either
completed or not yet completed. This is irrespective of whether the action is viewed as being
in the past or not, as is shown in the following sentences:

\ea \ea[] {\gll Ò dà kúl-éé lá\\
3{\SG} {\PST} go.home-{\PFV}.{\INT} {\FOC}\\
\glt‘S/he went home.’}
\ex[] {\gll ò dà kúl-ó lá\\
3{\SG} {\PST} go.home-{\IMPF} {\FOC}\\
\glt ‘S/he was going home.’}
\ex[] {\gll ò kúl-ó lá\\
3{\SG} go.home-{\IMPF} {\FOC}\\
\glt ‘S/he is going home.’}\z\z 


In languages like Dagbane and Mampruli there is, in addition to this basic inflectional system, another inflectional positive imperative suffix \mbox{\textit{-ma}} which is added to the verb. This
is illustrated below in Dagbane and Mampruli:

\ea \gll ìsì-má\\
get.up-{\IMPV}\\
\glt ‘Get up!’ \z 


Further still, there are other verbal suffixes, \textit{-ya} in Dagbane and Mampruli and \textit{-ng} in
Dagaare, which serve to affirm or emphasize the verbal action, often known as a \textit{focus} or \textit{factitive} marker. This is also shown in (\ref{ex:dagbani}):

\ea\label{ex:dagbani} 
\ea{ 
Dagbane/Mampruli\\ 
\gll ò ìsì-yà\\
3{\SG} get.up-{\FOC}\\
\glt ‘S/he has gotten up.’}
\ex{Dagbane/Mampruli
\\ \gll ò kyàm-yá\\
3{\SG} walk-{\FOC}\\
\glt ‘S/he has gone.’}
\ex{Dagaare\\\gll ò kyɛ́ng-ɛ́ɛ́-ńg\\
3{\SG} walk-{\PFV}{\INTR}.-{\FOC}\\
\glt ‘S/he has walked.’}\z\z


In Dagaare and, possibly, in Mampruli and Dagbane, these focus/factitive/affirmative affixes
are in complementary distribution with the so-called postverbal \textit{lá}.

\paragraph{Verb root}

The most basic part of the verb is, of course, the root or stem. Most verb roots in
Dagaare are monosyllabic. An understanding of the relationship between verb roots and their
affixes is important for categorizing Dagaare verbs into classes. The following are examples
of Dagaare verb roots:

\eabox{ \begin{tabular}{ll}
nyu- &drink\\
zo- & run \\
vɔg- & turn up, open\\
\end{tabular}
}


In addition to these, there is a dictionary form, the equivalent of infinitive verbs in other
languages. This may also be called  the citation form. So, in citing any of the above
verbs, designated by their roots, we would say:

\eabox{ \begin{tabular}{ll}
à nyú  &to drink\\
à zó & to run \\
à vɔ̀gè &to turn up, open\\
\end{tabular}
}


These forms, excluding the \textit{à}, are the ones that would appear in dictionary entries.

\paragraph{Inflectional affixes}
As has been discussed, the main inflectional affixes of Dagaare express aspect.
There are three forms, one imperfective affix and two perfective
affixes. In addition to this is the affixal counterpart of the factitive or affirmative particle. I
shall use two of the verb roots above to illustrate the various inflectional affixes:

\eabox{
\begin{xlist}
\exbox{  
%\setlength{\tabcolsep}{10pt} 
\begin{tabular}{ll}
zo-  &verb root\\
 à zò  &dictionary form\\
 zò-Ø  & {\PFV}\\
 zò-e  & {\PFV}.{\INT}\\
 zò-ró  & {\IMPF}\\
 zò-é-ńg \hspace{1.5ex} &{\FOC} particle on {\PFV}\\
\end{tabular}
} 
\exbox{
%\setlength{\tabcolsep}{10pt} %
\begin{tabular}{ll} 
vɔg-   &verb root\\
 à vɔ̀gè  &dictionary form\\
 vɔ̀gè-Ø & {\PFV}\\
 vɔ̀g-ɛ̀ɛ́  & {\PFV}.{\INT}\\
 vɔ̀g(è)-rɔ́ & {\IMPF}\\
 vɔ̀g-ɛ̀ɛ̀-ńg  &{\FOC} particle on {\PFV}\\
\end{tabular}
}
 \end{xlist}
}

Imperative forms are homophonous with the perfective transitive forms.

\paragraph{Derivational affixes and changes}

Derivational affixes, unlike their inflectional counterparts, are not easily separable
from the verb root: their forms are not easily discernible, so we will discuss them under their
functions in the next section.

\paragraph{Functional systems of the verb}
An interesting aspect of Dagaare and other Mabia verbal systems is that verbs can
be classified into pairs or even several classes of oppositions depending on derivational
processes such as causativity, transitivity, reversivity and many others. \tabref{tab:verbOppositions} is an
attempt to illustrate this with a number of Dagaare verbs.

\begin{table}
    \begin{tabularx}{\textwidth}{r@{\qquad}lXll}
    \lsptoprule
1 &kó &‘kill’ &kpì &‘die’\\
2 &lɔ́ɔ́& ‘make fall’& lè& ‘fall’\\
3 &gáálè& ‘put to sleep’& gángè &‘put oneself to sleep’\\
4 &zèglè &‘seat’& zèng &‘sit’\\
5& túúlí& ‘make drink’ &nyú &‘drink’\\
6 &sù &‘feed’ &dì & ‘eat’\\
7& séǹg& ‘wake one up’ &ìrì &‘wake up (from sleep)’\\
8& yàglè& ‘hang’& yàgè& ‘take off’\\
9 &vɔ̀glè &‘put on (hat)’ &vɔ̀gè &‘remove (hat)’\\
10 &ùù &‘bury’ &ùnnì &‘exhume’\\
11 &léng &‘tie’ &lórì &‘untie’\\
12 &nyòglì &‘hold loosely’ &nyɔ́gè &‘hold’\\
13& sɔ́glè &‘to hide’ &sɔ́ɔ̀ &‘to be black’\\
14 &yíélì& ‘sing, say iter.' &yèlì& ‘say, speak’\\
\lspbottomrule
    \end{tabularx}
    \caption{Oppositions in verbal forms}
    \label{tab:verbOppositions}
\end{table}

The pairs of oppositions from 1 to 7 seem to illustrate causativity oppositions, with the
members to the left being the causatives.
While illustrating causativity the pairs from 1 to 4 also illustrate transitivity, with the
pairs to the left being the transitive verbs while those to the right are the intransitives.

Pairs from 8 to 11 illustrate the reversivity oppositions, while pair 12 may
illustrate what may be called the releasive opposition. Pair 14 seems to illustrate the repetitive
or iterative opposition between the two members.
In addition to these oppositions one may also find other oppositions. One good
example is the polarity opposition between the following Mampruli verbs: \textit{mi} ‘to know’; \textit{zi}
‘to not know’. This is illustrated in the following sentences where \textit{zi} is an inherent negative
verb:

\ea \ea[]{\gll ǹ mí\\
1{\SG} know\\
\glt ‘I know.’}
\ex[]{ \gll ǹ pá mí\\
1{\SG} {\NEG} know\\
\glt ‘I don’t know.’}
\ex[]{\gll  ǹ zì\\
1{\SG} {\NEG}+know\\
\glt ‘I don’t know.’}
\ex[]{ \gll *ǹ pá zí\\
1{\SG} {\NEG} {\NEG}+know\\
\glt ‘I don’t know.’}\z\z



Now, a glance at the table of oppositions shows that derivational affixation is not a
very developed phenomenon in Dagaare and other Mabia languages, certainly not as
developed as the derivational systems of Bantu. In \tabref{tab:verbOppositions}, there is only one
consistent suffix \textit{-li} in the pairs of oppositions. One cannot, however, say that it is any
particular derivational suffix, as the pairs of words in which it occurs cut across several
derivational classes. In this regard, there are no regular sequences of derivational affixes. As
can be seen, the rest of the morphological changes in these oppositions do not involve
affixation but rather internal vowel changes as in pairs 1 and 2 on the table.
This near lack of derivational morphology with respect to the verb is not surprising in
such languages where verb serialization is very productive. Within African languages there seems to be an interesting relation between verb serialization and verbal extensions; the
two possibly have complementary functions. It seems that languages with a rich verb
serialization system will necessarily have a poor verb derivational system and vice versa. This
is an interesting comparative research agenda, at least, within African linguistics.

\largerpage
\subsubsection{Adverbs}
The most recurrent morphophonological structure of adverbs in Dagaare is a 
reduplicated structure, as shown in (\ref{ex:adverbs1}), though there are some others deviating from this
recurrent structure, e.g.~(\ref{ex:adverbs2}). Example sentences are provided in (\ref{ex:adverbsexx}) to show how this main
group of adverbs is used.

\eabox{ 
\begin{xlist}
\exbox{
\label{ex:adverbs1} \begin{tabular}{ll} 
lígílígí& ‘quietly, silently’\\
 wíèú wíèú& ‘quickly, fast’\\
lɛ́mɛ́ lɛ́mɛ́ / lɛ̀mɛ̀lɛ̀mɛ̀& ‘very sweet’\\\end{tabular}
}
\exbox{\label{ex:adverbs2}\begin{tabular}{ll}
gyɔ̀s& ‘in an unrefined way’\\
 yɔ̀s& ‘in an unrefined way’\\
bàràtàtà& ‘in a huge way’\\
\end{tabular}
}
\end{xlist}
}


\eabox{ 
\label{ex:adverbsexx} 
\begin{xlist}
\exbox{\gll à dɔ́ɔ́ zìng-ɛ̀ɛ̀ lá líglíg lɛ́\\
{\DEF} man seat-{\PFV} {\FOC} quiet {\PART}\\
\glt ‘The man is sitting quietly/in a quiet way.’
}
\exbox{\gll  à bíé zò-ró lá wíéwíé\\
{\DEF} child run-{\IMPF} {\FOC} quickly/fast\\
\glt ‘The child is running fast.’
}
\exbox{\gll  à ànkàá é lá nòɔ́ lɛ́mɛ́lɛ́mɛ́ lɛ́\\
{\DEF} orange be {\FOC} sweet very-sweet {\PART}\\
\glt ‘The orange is very sweet.’
}
\end{xlist}
}

\subsubsection{Pluractional verbal constructions}
In Dagaare and some other languages, number is not exclusively a nominal system of
singular and plural nouns. Plurality of action can also be expressed in the verbal system.
Plurality here can mean that the action is repeated several times or that the same action affects
several entities. Pluraction is a quite complex morphological process in Dagaare that cannot
be discussed at length here, and I will simply provide some constructions in (\ref{ex:redup}) that illustrate
the syntactic restrictions involved in the expression of the phenomenon.
In (\ref{ex:redupA}), we have an action, cutting, that affects a couple of entities, the hands. As
such this plurality is expressed inside the verbal predicate. In (\ref{ex:redupB}), the action affects only
one entity, and as such, the plurality expressed within the verbal predicate is ungrammatical.
In (\ref{ex:redupC}), we have a normal case of the action affecting only one entity with no marked
plurality inside the predicate.

\ea \label{ex:redup} \ea[]{\label{ex:redupA}\gll ǹ núú-rì ngmàà-r-ɛ̀ɛ̀ lá\\
1{\SG} hand-{\PL} cut-{\PL}-{\PFV} {\FOC}\\
\glt ‘My hands are cut.’ $=$ ‘I am broke.’}
\ex[*]{\label{ex:redupB}\gll à nǔ ngmàà-r-ɛ̀ɛ̀ lá\\
1{\SG} hand cut-{\PL}-{\PFV} {\FOC}\\
\glt ‘My hand is cut.’}
\ex[]{\label{ex:redupC}\gll à mí-rì ngmàà-ɛ̀ lá\\
{\DEF} rope-{\SG} cut-{\PFV} {\FOC}\\
\glt ‘The rope is cut.’}
\ex[*]{\label{ex:redupD}\gll à mí-rì ngmàà-r-ɛ̀ɛ̀ lá\\
{\DEF} rope-{\SG} cut-{\PL}-{\PFV} {\FOC}\\
\glt ‘The rope is cut.’}
\ex[]{\label{ex:redupE}\gll à mí-è ngmàà-r-ɛ̀ɛ̀ lá\\
{\DEF} rope-{\PL} cut-{\PL}-{\PFV} {\FOC}\\
\glt ‘The ropes are cut.’}
\ex[]{\label{ex:redupF}\gll à mí-rì ngmààré ngmàà-r-ɛ̀ɛ̀ lá\\
{\DEF} rope-{\SG} cut cut-{\PL}-{\PFV} {\FOC}\\
\glt ‘The rope is cut at several places.’}\z\z



In (\ref{ex:redupD})-(\ref{ex:redupE}), we expect a normal case of an unacceptable plural affix within the predicate and
an acceptable case of the same expressed plurality, respectively. However, in (\ref{ex:redupF}), we are
faced with a case in which even though the entity is single, we still have an acceptable plural
affix within the verbal predicate. This needs explanation. 
One important observation is that
the verb is repeated, reduplicated. Reduplication is the next issue to explain in this
grammatical sketch. 
Pluraction seems to be definable, not just in terms of the number of
entities the predicate action affects, nor on the number of occurrences on the same entity in
the same place, but also on various parts it occurs on the same entity.



\subsection{Other word classes and morphological processes}
In this section, I discuss the additional word class of ideophones and the  general morphological processes of 
reduplication and compounding.


\subsubsection{Ideophones}
\citet[131--132]{Trask1993} defines an ideophone as ``[O]ne of a grammatically distinct
class of words, occurring in certain languages, which typically express either distinctive
sounds or visually distinctive types of action.'' Ideophones have a specific morphophonological structure in Dagaare that no other word class consistently exhibits.
First, it generally has a three-syllable structure. Second, the vowels of the first syllable are
copied on to the subsequent syllables. Third, and quite importantly, there is usually only one
tonal quality, either low or high, on the entire stretch of the three-syllable word. Fourth, each
ideophone can be produced either as a uniquely low tone lexeme or as a uniquely high tone
one, with a slight variation in meaning. The low-toned ones refer to heavier, longer, or fatter
entities, while the high-toned ones refer to lighter, shorter and thinner entities. These are
shown in (\ref{ex:ideophones}), with example sentences in (\ref{ex:ideophonesexx}).

\eabox{ \label{ex:ideophones} {\begin{tabular}{ll}  
gbànggbàlàng / gbángbáláng  & ‘of a long pole or thing falling down’\\
vàrkpàrà / várkpárá& ‘in a messy way’\\
gàrmànà / gármáná & ‘spread across a surface’\\
bɔ̀nggɔ̀lɔ̀ng / bɔ́nggɔ́ng& ‘of a fat and unwieldy mass’\\
bìlbàlàà / bílbáláá& ‘of a huge item lying down’\\
\end{tabular}}}


\ea
\label{ex:ideophonesexx} 
    \ea
    \gll  à lánggbáráà lá ká ó dé lɔ́ɔ́, gbàngbàràng\\
    {\DEF} hook {\TOP} {\COMP} 3{\SG} take throw-down, {\IDPH}\\
    \glt ‘It is the hook s/he has thrown down.’

    \ex
    \gll bíní lá ká ó nyɛ̀ bìng, vàrkpàrà\\
    excreta {\TOP} {\COMP} 3{\SG} shit put-down, {\IDPH}\\
    \glt ‘It is excreta s/he has shit.’

    \ex
    \gll à bíé bàl-ɛ̀ɛ̀ lá, à páà lè gàngè, gàrmànà\\
    {\DEF} child tire-{\PFV} {\FOC} {\DEF} then fall lie-down, {\IDPH}\\
    \glt ‘The child is tired and then is lying there.’

    \ex
    \gll nyɛ́ ò nàng páà zèng bɔ̀nggɔ̀lɔ̀ng lɛ́\\
    see 3{\SG} when then sit {\IDPH} {\PART}\\
    \glt ‘Just see how s/he is seated!’

    \ex
    \gll ò dé lá à dàngmáá lɔ́ɔ́, bìlbàlà\\
    3{\SG} take {\FOC} {\DEF} log throw-down {\IDPH}\\
    \glt ‘He threw down the log.’
    \z
\z

Another aspect of ideophones is that, besides having their unique morphophonology, which is
quite different from those of comparative word classes like adverbs, adjectives and verbs, they, again unlike these other word classes, do not seem to have independent semantics.
As can be seen from the above glosses and transliterations, it is hard to pin them down and
assign denotational, dictionary meanings to them. They depend on adjacent words and other
contexts for their meaning. This again makes this class of words unique in the language. It
seems that there are compelling reasons for setting up a word class of ideophones in
Dagaare. From a comparative point of view, these facts of ideophones raise important
empirical issues for discussing linguistic categorization.


\subsubsection{Reduplication}
Reduplication is a pervasive morphological process across the languages of the world. It
involves repetition or multiple occurrence of a morphological entity within a larger unit.
Formally, two types are recognized, partial and full reduplication. In the (\ref{ex:reduplication}), I illustrate full reduplication involving verbs, adjectives, adverbs, and partial
reduplication involving an ideophone.

\eabox{ \label{ex:reduplication} \begin{tabular}{ll}   zò zò (v.)& ‘run, run’\\
 wóg wóg (adj.)& ‘tall, tall; long, long’\\
 vɛ̀lvɛ̀lvɛ̀l (adv.)& ‘very long’\\
 vɛ̀nvɛ̀lvɛ̀ng (ideo.) & ‘description of a position \\
 &occupied by a long entity’\\
 \end{tabular}
 }

In a partial reduplication, not all the material of the segment being repeated is carried over,
hence we have \textit{vɛ̀ng} becoming \textit{vɛ̀} in the case of the partial reduplication.
A notion of reduplicative compounds exists in Dagaare. In this morphological process, a
compound is formed by repeating or bringing together certain segments if even they belong to
different word classes to form agentive nouns. This is illustrated in (\ref{ex:redupCompounds}).


\eabox{ \label{ex:redupCompounds} \begin{tabular}{llll} 
 gɔ́ngé& gɔ́nnè& $\rightarrow$ &gɔ́nggɔ́nɔ̀\\
‘to make noise’& ‘noise’&& ‘noise-maker’\\
&&&\\
dì & bóndìríì& $\rightarrow$ &bòndìdíre\\
‘to eat’& ‘food’&& ‘food eater, glutton’\\
\end{tabular}
}


This kind of phenomenon occurs in South-East Asian languages like Tagalog and Malay and should constitute a useful area of comparison between African and Asian languages. While the forms of reduplication are interesting in themselves, the way the
morphological process functions across languages is also interesting. In Dagaare, as may be
seen in (\ref{ex:reduplication}), it involves intensity and also a constant repetition of the action. But this is by no
means universal, as reduplication can also express less of something in some languages. In
some languages like Cantonese, it can express the diminutive. Again this should provide the
framework of useful comparisons between Asian and African languages.

\subsubsection{Compounding}
Compounding has already been discussed in some of the above morphological
processes. It is a very productive process in the language. Many new words and expressions are formed through compounding of several entities including noun+affix (\ref{ex:compoundingA}),
noun+adjective (\ref{ex:compoundingB}), noun + noun (\ref{ex:compoundingC}), verb + verb (\ref{ex:compoundingD}), verb + noun (+ affix) (\ref{ex:compoundingE}), and
even elements of a phrase (\ref{ex:compoundingF}).

\ea
\label{ex:compounding}
\begin{xlist}
\exbox{\label{ex:compoundingA}\begin{tabular}{p{1.1cm}@{+ }p{1.1cm}p{0.1cm}p{.1cm}p{7cm}}
 dɔ́ɔ́  &-léé& &$\rightarrow$& dɔ̀ɔ̀léé\\
 man & small&&& ‘boy’ \\\end{tabular}}\smallskip

\exbox{\label{ex:compoundingB}\begin{tabular}{p{1.1cm}@{+ }p{1.1cm}p{0.1cm}p{.1cm}p{7cm}}
 néɛ̀  &fáá &&$\rightarrow$& néngfáá\\
 person & bad&&& ‘bad fellow’ \\\end{tabular}}\smallskip

\exbox{\label{ex:compoundingC}\begin{tabular}{p{1.1cm}@{+ }p{1.1cm}p{0.1cm}p{.1cm}p{7cm}}
 dìé  &pɔ́gɔ́ &&$\rightarrow$& dìépɔ́gɔ́\\
 room & woman&&& ‘wife, housewife, homemaker’\\ \end{tabular}}\smallskip

\exbox{\label{ex:compoundingD}\begin{tabular}{p{1.1cm}@{+ }p{1.1cm}p{0.1cm}p{.1cm}p{7cm}}
 yɔ́  &tá &&$\rightarrow$ &yɔ́tá\\
 roam & reach&&& personal name (one who gets to destination)  \end{tabular}}\smallskip

\exbox{\label{ex:compoundingE}\begin{tabular}{p{1.1cm}@{+ }p{1.1cm}p{0.1cm}p{.1cm}p{7cm}}
 bɔ̀ng  &gánè &&$\rightarrow$ &gánbɔ́gnɔ́\\
 know & book&&& ‘literate’ \end{tabular}}

\exbox{\label{ex:compoundingF}\begin{tabular}{p{1.1cm}@{+ }p{.6cm}@{+ }p{0.7cm}p{.1cm}p{7cm}}
 kà  &á&pɔ́gè & $\rightarrow$& kààpɔ́gè\\
 {\COMP} & {\DEF}&get.target&& ‘maybe’ \end{tabular}}
\end{xlist}
\z


 
\section{Syntax}

In this part of the grammatical sketch, I provide brief comments about the structure of
the simple sentence including word order, voice, polarity, adpositions, and comparison. I also
provide illustrations of more complex sentence formation involving relativization, question
formation, small clause constructions, serialization, and serial verb nominalization. I end the
chapter with a survey of discourse particles, involving especially topic and focus in the
language.

\subsection{Sentence types}
Dagaare is basically an SVO language, although there are some complications when
we deal with much more complex sentences than the basic ones. It therefore means that
verbal elements in the most unmarked cases come after the subject NP and before the object
NP if there is any. There are three basic types of sentences in Dagaare: the verbless sentence,
the simple sentence, and complex sentences. Verbless sentences do not have an
overt verb in the structure. Here are two examples:

\ea \gll nààná lá\\
Naana {\FOC}\\
\glt ‘That is Naana.’\z 

\ea \gll á sákúúrì lá nyɛ́\\
{\DEF} school {\FOC} this\\
\glt ‘This is the school.’\z 

I now discuss the other two types of sentences in the following sections.

\subsection{The simple sentence}
One of the most distinctive aspects of general Mabia syntax is the presence of particles
between the subject NP at the beginning of a canonical declarative sentence and the main
verb. These particles, traditionally termed preverbal particles, express most of the temporal and modal aspects of the sentence. Another distinctive aspect is the presence of a
postverbal particle that expresses polarity and focus phenomena. The following schema
characterizes the Dagaare simple declarative sentence involving a transitive verb.

\ea  Subject NP |preverbal particles - main verb - postverbal particle| Object NP \z 

I now describe each of these parts of the simple sentence, illustrating with (\ref{ex:sentex}).

\ea \label{ex:sentex} \gll Dàkóráá dà nyúú-ró lá à kòɔ́ à dìè póɔ́\\
Dakoraa {\PST} drink-{\IMPF} {\FOC} {\DEF} water {\DEF} room \POSTP\\
\glt ‘Dakoraa was drinking the water in the room.’\z 

In (\ref{ex:sentex}), \textit{Dàkóráá} is the subject NP. No special affixes and particles mark the subject NP in
Dagaare, hence it is not an ergative language. Grammatical functions are marked mainly by
positional distribution in the sentence, with the subject occurring before the verb in a
canonical declarative sentence, as is the case with the sentence above. This is followed by the
preverbal particle \textit{dà}, which expresses the temporal features of the construction. We then
have the main verb, \textit{nyúú-rò}. The suffixal parts of the main verb express the aspectual
features of the construction. The postverbal particle is \textit{lá}; it functions as a focus particle. It
may also mark features of polarity. Finally \textit{à kòɔ́} is the object NP of the sentence. This is
followed by an adverbial phrase comprising an NP, \textit{à díè} and a locative postposition, \textit{póɔ́}.
The pervasive use of body part postpositions, rather than prepositions, is another distinctive
aspect of Dagaare and Gur syntax.
This basic declarative sentence provides the basis for a number of syntactic
alternations expressing aspect (perfective/imperfective), mood (imperative), polarity
(negation/positivity), voice (unaccusative) and for encoding more complex thought with constructions such as relativization, serialization, and serial verb nominalization.
In the following, I shall briefly illustrate these various syntactic alternations.

\subsubsection{Imperative}
This syntactic structure functions to compel (\ref{ex:impsent}) or exhort (\ref{ex:hortsent}) an interlocutor or other
participant in a speech situation. A special feature of the exhortative or the hortative is that it
imposes an inherent high tone on any preceding pronoun (\ref{ex:hortsent}).

\ea \label{ex:impsent} \gll nyú\\
drink\\
\glt ‘Drink!’\z

\ea \label{ex:hortsent} \gll ó nyú\\
3{\SG}.hortative drink\\
\glt ‘S/he should drink.’\z 

\subsubsection{ Negation}
Negation is expressed in Dagaare by preverbal particles. Negation has a relationship
to mood in the language. We may distinguish between a negative declarative particle as in (\ref{ex:negdecl})
and negative imperative particle as in (\ref{ex:negimp}).

\ea \label{ex:negdecl} \gll Dàkóráá dà bà nyú à kòɔ́\\
Dakoraa {\PST} {\NEG} drink {\DEF} water\\
\glt ‘Dakoraa did not drink the water.’\z

\ea \label{ex:negimp} \gll Tá nyú!\\
{\NEG} drink\\
\glt ‘Don’t drink!’\z 

The postverbal particle is mutually exclusive with the negative particle in such negation constructions, as shown in (\ref{ex:negla}).

\ea \label{ex:negla} \ea[]{\gll tè dà gáá lá dàá\\
1{\PL} {\PST} go {\FOC} market\\
\glt ‘We did go to the market.’}
\ex[]{\gll tè dà bá gàà dàá\\
1{\PL} {\PST} {\NEG} go market\\
\glt ‘We did not go to the market.’}
\ex[*]{\gll tè dà bá gáá lá dàá\\
1{\PL} {\PST} {\NEG} go {\FOC} market\\
\glt ‘We did not go to the market.’}\z\z

\iffalse 

\subsubsection{Unaccusativity}
A distinctive syntactic alternation in Dagaare is the unaccusative construction, as
shown in (\ref{ex:unaccusativeB}), its declarative counterpart being (\ref{ex:unaccusativeA}).

\ea \ea[]{\label{ex:unaccusativeA} \gll à bóó-ré nyú lá à kòɔ́\\
{\DEF} goat-{\PL} drink {\FOC} {\DEF} water\\
\glt ‘The goats have drunk the water.’}
\ex[]{\label{ex:unaccusativeB}\gll ɑ̀ kòɔ́ nyú-é lá\\
{\DEF} water drink-{\PFV} {\FOC}\\
\glt ‘The water has been drunk.’}\z\z 

The surface subject of this construction is an underlying object of the declarative counterpart
of the construction.
\fi 
\subsubsection{Comparative constructions}
Verbs/adjectives like \textit{gàngè} ‘more than’, \textit{sèɛ̀} ‘better than’, and \textit{zùò} ‘more than’, which
mean `over', `pass', etc., are used to express the comparison of different items within the
Dagaare sentence. Here are examples:

\ea \gll Bádɛ́ré é lá kpóǹg gàngè Àyúó\\
Badere be {\FOC} big pass Ayuo\\
\glt ‘Badere is bigger than Ayuo.’\z

\ea \gll  ò dà zó gàngé má lá\\
3{\SG} {\PST} run pass 3{\SG}.{\OBJ} {\FOC}\\
\glt ‘He ran faster than me.’\z

\ea \gll  fò dèmé zùó lá ǹ dèmé\\
2{\SG} own.{\PL} pass {\FOC} 1{\SG} own.{\PL}\\
\glt ‘Yours are more than mine.’\z



\subsection{ Complex sentences}
Complex sentences serve to encode the relationship between two or more ideas,
entities and events. The relationship may be one of conjunction, relational, sequential, or the
quest for information about another idea, entity or event.

\subsubsection{ Coordination and Subordination}
Several ways of conjoining sentences exist in Dagaare. For a more extensive
presentation, see \citet{Bodomo1997}. There are about seven coordinating conjunctions and four
subordinating conjunctions in the language. These are \textit{à}, \textit{àné}, \textit{kyɛ́}, \textit{kyɛ́ kà}, \textit{béé} and \textit{kyɛ́ béé} and \textit{ká}, \textit{kà}, \textit{ká kà} and \textit{nàng} respectively. The two sentences below in (\ref{ex:conj1}) and (\ref{ex:conj2})
illustrate one case each of coordinating and subordinating conjunctions respectively.

\ea \label{ex:conj1} \gll nààná dà yí lá sàkúúrì kúlí à tè zéng dì-ré bó-má\\
Naana {\PST} get.out {\FOC} school go.home {\CONJ} go sit eat-{\IMPF} thing-{\PL}\\
\glt ‘Naana went home from school and sat down eating food.’\z


\ea \label{ex:conj2} \gll à tè téngɛ́ nàá yèlì ká té kóɔ́-rɔ́ yágà\\
{\DEF} 1{\PL} country king say {\CONJ} 1{\PL} farm-{\IMPF} {\INTENS}\\
\glt ‘The king of our country says that we should farm a lot.’\z 

\subsubsection{ Relativization}
Relativization in Dagaare involves embedding one construction into another, usually
rendered by an overt relativizing element which occurs after the noun in question.

\ea \gll ɑ̀ dɔ́ɔ́ ná náng záà wà kyɛ̂ nyú lá à kòɔ́\\
{\DEF} man {\REL} {\REL} yesterday come here drink {\FOC} {\DEF} water\\
\glt ‘The man who came here yesterday drank the water.’\z 

The usual relativizing element is a complex, \textit{(ná) náng}, involving an optional first element.
The relativizing complex is neutral for human and non-human nouns.

\subsubsection{Question formation}
Dagaare presents some challenges to the so-called wh-parameter, which determines
whether wh-expressions, in the case of Dagaare, \textit{bòńg}-expressions \citep{Bodomo1997}, can or
cannot be moved to the front of the sentence. Consider the following sentences:


\ea \ea{\gll  ò bóɔ́-rɔ́ lá dííú\\
3{\SG} want-{\IMPF} {\FOC} food\\
\glt ‘S/he wants food.’}
\ex{\gll  bòng lá ká ó bóɔ́-rɔ̀\\
WH {\FOC} that 3{\SG} want-{\IMPF}\\
\glt ‘What does s/he want?’}
\ex{\gll ò bóɔ́-rɔ̀ lá bóng\\
3{\SG} want-{\IMPF} {\FOC} WH\\
\glt ‘What does he want?’}\z\z

\ea \ea{\gll ò gɛ̀-rɛ́ lá hɔ́ng kɔ́ng\\
3{\SG} go-{\IMPF} {\FOC} Hong Kong\\
\glt ‘S/he is going to Hong Kong.’}
\ex{\gll  yèng lá ká fó gɛ̀-rɛ́\\
WH {\FOC} that 2{\SG} go-{\IMPF}\\
\glt ‘Where are you going?’}\z\z 

\ea \ea{\gll  ǹ yúórí lá bòǹglàkyɛ́rè\\
1{\SG} name {\FOC} Bonlacher\\
\glt ‘My name is Bonlacher.’}
\ex{\gll  fò yúórí lá bóng\\
2{\SG} name {\FOC} WH\\
\glt ‘What is your name?’}
\ex[??]{\gll  bóng lá fò yúór̀\\
WH {\FOC} 2{\SG} name\\
\glt ‘What is your name?’}\z\z

In each of the above sentences, the (b.) and (c.) are questions inquiring about the second NP of
the (a.) sentences. The question word is freely found both at the beginning and at the end of the
question constructions. This seems to contravene the wh-parameter pervasive in the principles
and parameters approach to grammar. Such analyses claim that languages may be
parametrized to be wh-fronting or in situ languages. The above data shows that Dagaare seems to be at both sides of the parameter. It must, however, be observed that the (c.) question
construction is unnatural in the language.

\subsubsection{Small clause constructions}
Small clause constructions involve a complex of at least two clauses. While the first
clause may have overt verbal predicate elements, the second or subsequent clause may
not have an overtly expressed verbal predicate. This second or subsequent part is termed a
reduced clause or a small clause.  Here
are two examples.


\ea\label{ex:smallclause1} \gll tè bìng’ó lá nàá\\
1{\PL} put.3{\SG} {\FOC} king\\
\glt ‘We made him King.’\z

\ea \ea{\label{ex:smallclause2A}\gll ò tú lá à bògí z̀lùńg\\
3{\SG} dig.{\PFV} {\FOC} {\DEF} hole deep\\
\glt ‘S/he dug the hole deep.’}
\ex{\label{ex:smallclause2B}\gll ò tú lá à bògí sígí\\
3{\SG} dig.{\PFV} {\FOC} {\DEF} hole go.down\\
\glt ‘S/he dug the hole deep.’}\z\z

Alongside such small clause constructions, in the form of object depictive resultative predicates as we
have in (\ref{ex:smallclause1}) and (\ref{ex:smallclause2A}), Dagaare also has a serial verb constructions, as in (\ref{ex:smallclause2B}).
Serialization is the next typical syntactic construction to be discussed.

\subsubsection{Verb serialization}
Verb serialization (or the serial verb construction) is a productive syntactic process in
Dagaare and many other West African languages. Hardly any analysis of these languages
can be successfully undertaken without an understanding of the basic properties of this pervasive
grammatical construction. The following example illustrates a serial verb construction,
involving as many as four verbs within the construction.



\ea \gll Dàkóráá nà dé lá à kúúrí zá lɔ́ɔ́ èng à kòɔ̀ póɔ́\\
Dakoraa {\FUT} take {\FOC} {\DEF} stone throw fall put {\DEF} water inside\\
\glt ‘Dakoraa will throw the stone into the water.’\z

Many issues have been raised within the literature about these constructions and they continue
to be discussed (e.g. \citealt{Bodomo1993, Bodomo1997, Bodomo1997Pathfinders}). Some of these are whether the serial verb
construction consists of one clause or a series of clauses; whether we can arrive at a definitive
typology of serial verbs; whether the complex of verbs expresses a single event or not; and
how we can account for the fact that many of the verbs in the series share grammatical
functions such as subject and object. This syntactic process can be altered in several ways,
one being the serial verb nominalization.

\subsubsection{Serial verb nominalization}
In the serial verb nominalization, the object of a declarative serial verbal sentence
(\ref{ex:serialnomA}) gets ``moved'' to the beginning of the construction, and the second gets a nominal affix
attached to it (\ref{ex:serialnomB}). The whole construction is a nominalized phrase.

\ea \ea{\label{ex:serialnomA} \gll ò dà dé lá à kòɔ̀ nyú\\
3{\SG} {\PST} take {\FOC} {\DEF} water drink\\
\glt ‘He drank the water./He took the water and drank it./ He drank up the water.’}
\ex{\label{ex:serialnomB} \gll à kòɔ̀ dé nyú-ù\\
{\DEF} water take drink-{\NMLZ}\\
\glt ‘the drinking of the water.’}
\ex{\label{ex:serialnomC} \gll à kòɔ̀ dé nyú-ù bá è yèl-sòńg\\
{\DEF} water take drink-{\NMLZ} {\NEG} be deed-good\\
\glt ‘The drinking of the water is not a good thing (to do).’}\z\z 

This nominalized construction can form part of a sentence, and is indeed mostly the subject of
a more complete sentential construction (\ref{ex:serialnomC}). This construction is  unique in Dagaare
and a few other Mabia languages, but largely absent in the Kwa languages of West Africa, such
as Akan. This construction has not been widely examined in the linguistic literature, especially in the light of nominalization across languages. %\citet{OostendorpBodomo1993}
\citet{Bodomo1997, Bodomo1997Pathfinders} and \citet{Bodomo2004complex} provide a useful point of departure for such a cross linguistic
study.

\subsection{ Discourse phenomena}
In this section, I will first list and briefly comment on some expressive particles, which 
express various types of feelings on the part of the speaker during a discourse situation. The
various types of emotive feelings and the particles that encode them include the following:\bigskip



\noindent \textit{nàng}: expression of politeness, as in \ref{ex:nang}.

\ea \label{ex:nang}\gll  nàng dé à gánè nyɛ̀ kó má\\
{\PART} take {\DEF} book {\DEM}. give 1{\SG}\\
\glt ‘Please, give me this book.’\z


\noindent \textit{wɛ̀}, \textit{yáà}: pleading tone, as in (\ref{ex:part:we}) and softening particle, as in (\ref{ex:part:yaa}), in the sense of
extenuating a command or a potentially harsh tone:

\ea \label{ex:part:we}\gll  dé kò má wɛ̀\\
take give 1{\SG} {\PART}\\
\glt ‘Give it to me' $=$ `Why not give it to me?’\z

\ea  \label{ex:part:yaa} \gll  dé kò má yàà\\
take give 1{\SG} {\PART}\\
\glt ‘Give it to me, eh.’\z 


\noindent  \textit{àɪ́}, \textit{wɛ̀}: exasperation, as shown in (\ref{ex:part:ai}).


\ea \label{ex:part:ai} \gll  àɪ́ wɛ̀, ìrì à kyɛ̌\\
{\PART} {\PART} get.up {\DEF} here\\
\glt ‘Oh please, get up from here!’\z 

\noindent \textit{mɔ̀ɔ̀}, \textit{bée}, \textit{mɔ̀ɔ̀ béé}: question/interrogation in (\ref{ex:part:mooA}) and (\ref{ex:part:mooB}); exclamation in (\ref{ex:part:mooC}):


\ea \ea{\label{ex:part:mooA} \gll fò kúlò lá béé\\
2{\SG} go.home-{\IMPF} {\FOC} {\PART}\\
\glt ‘Are you going home?’}
\ex{\label{ex:part:mooB} \gll fò mɔ̀ɔ́ kòng gángè bèè\\
2{\SG} {\PART} {\NEG} lie.down {\PART}\\
\glt ‘Won’t you lie down /sleep overnight?’}
\ex{\label{ex:part:mooC}\gll yɛ́lɛ́ mɔ̀ɔ̀ béé\\
matter {\PART} {\PART}\\
\glt ‘Oh, what matter $=$ Oh, my God!’}\z\z 

\noindent \textit{wóówóóì}, \textit{ǹ sáá wóóI}, \textit{ǹ má wóóì}: pain, as in (\ref{ex:part:wooiA}), which is actually used in situations
of mourning; surprise, as in (\ref{ex:part:wooiB}).


\ea \ea{ \label{ex:part:wooiA} \gll ǹ sàà wóóí, ǹ sàà wóóí\\ 
1{\SG} father {\PART} 1{\SG} father {\PART}\\
\glt ‘Oh my father, oh my father.’}
\ex{ \label{ex:part:wooiB} \gll wóó, bòng lá ká ó déɛ̀ gángè lɛ̌\\
{\PART} why {\FOC} {\COMP} 3{\SG} just lie.down that.way\\
\glt ‘Oh, why is s/he just lying down that way?’}\z\z 

This list is not exhaustive but only just indicative of the particles and the feelings they encode
in discourse situations. This aspect of Dagaare still awaits a detailed study. It is interesting
both with respect to language internal considerations and to typological considerations.

A second aspect of discourse phenomena that I will briefly address involves two
particles and their variants which tend to express topic and focus phenomena in the language.

\subsubsection{ Topic}

The topic of a sentence or enunciation is that element on which the rest of the sentence
(the comment) refers to. Dagaare is much like English and many other languages in which
the topic is normally realized by the unmarked grammatical subject (\ref{ex:topicA}).

\ea \ea{\label{ex:topicA}\gll  à dɔ́ɔ́ kú-ló lá\\
{\DEF} man go.home-{\IMPF} {\FOC}\\
\glt ‘The man is going home.’}
\ex{\label{ex:topicB}\gll à dɔ́ɔ́ nyɛ́ éng (ò) kú-ló lá\\
{\DEF} man this {\TOP} (3{\SG}) go.home-{\IMPF} {\FOC}\\
\glt ‘As for this man, he is going home.’}\z\z


\ea \ea{\label{ex:topic2A}\gll ǹ nà nyú lá kòɔ́\\
1{\SG} {\FUT} drink {\FOC} water\\
\glt ‘I will drink water.’}
\ex{\label{ex:topic2B}\gll màà éǹg (ǹ) nà nyú lá kóɔ́\\
 1{\SG}.{\STR}. {\TOP} (1{\SG}) {\FUT} drink {\FOC} water\\
\glt ‘As for me, I will drink water.’}\z\z 

However, there are instances in which the grammatical subject is associated with an element,
\textsc{{\TOP}}, that topicalizes the whole NP entity. This NP entity is co-referential with an optional
resumptive subject pronoun, as indicated in (\ref{ex:topicB}). This is the case of a marked topic
construction in Dagaare. Dagaare has two types of alternate pronominal subjects in its
pronominal system--known as weak and strong pronouns \citep{Bodomo1997}. Unmarked
grammatical subjects use the weak subject pronouns, as in (\ref{ex:topic2A}). When pronominal subjects
are topicalized, it is the strong pronouns that are used, as exemplified in (\ref{ex:topic2B}).\bigskip

\subsubsection{ Focus}

The particle \textit{lá} and its variants expresses focus  among many other functions. \citet[105]{Trask1993} defines focus as ``[S]pecial prominence given to some element in a sentence
which represents the most important new information in that sentence or which is explicitly
contrasted with something else.'' Virtually any part of the sentence can be focused in
Dagaare. The sentence in (\ref{ex:focusA}) may be focused in various ways, or rather parts of it may be
emphasized in various ways. First, the sentence as a whole, in its neutral form, is marked with
the particle \textit{lá}, which acts as a default focus or assertive/factitive marker. It has 
been glossed variously as \textsc{{\FOC}}, \textsc{ass}, or \textsc{fact}. This is just a subset of the many ways the
particle \textit{lá} may be used in Dagaare
\citep{Bodomo1997, Bodomo1997Pathfinders,  Dakubu1998} and
other Mabia languages. Having illustrated the use of the focus particle with respect to the
whole sentence, we now illustrate argument focus, predicate focus and adjunct focus in the
Dagaare sentence.



\ea \ea{\label{ex:focusA} \gll Dàkóráá dà dì lá sááó à bágúó nyɛ́\\
Dakoraa {\PST} eat {\FOC}/{\ASSERT} saao {\DEF} morning {\DEM}\\
\glt ‘Dakoraa ate saao this morning.’}
\ex{\label{ex:focusB} \gll Dàkóráá là dà dì sááó à bágúó nyɛ́\\
Dakoraa {\FOC} {\PST} eat saao {\DEF} morning {\DEM}\\
\glt ‘It is DAKORAA that ate saao this morning.’}
\ex{\label{ex:focusC} \gll  sááó lá kà Dàkóráá dà dì à bágúó nyɛ́\\
saao {\FOC} {\COMP} Dakoraa {\PST} eat {\DEF} morning {\DEM}\\
\glt ‘It is SAAO that Dakoraa ate this morning.’}
\ex{\label{ex:focusD} \gll  bágúó nyɛ́ lá kà Dàkóráá dà dì sááó\\
morning {\DEM} {\FOC} {\COMP} Dakoraa {\PST} eat saao\\
\glt ‘It was THIS MORNING that Dakoraa ate saao.’}
\ex{\label{ex:focusE} \gll  Díí-ú lá kà Dàkóráá  dà dì à sááó\\
eat-{\NMLZ} {\FOC} {\COMP} Dakoraa {\PST} eat {\DEF} saao\\
\glt ‘Dakoraa ATE the saao.’ (What he did was eat the saao.)}\z\z 

Any argument position may be focused in Dagaare.
The structure in (\ref{ex:focusB}) illustrates subject
focus. The subject NP \textit{Dàkóráá} is being emphasized as the actor and not any other entity.
To focus the subject, the subject NP is preposed and the focus marker, \textsc{{\FOC}}, comes right after it.
Unlike other languages like Yoruba where a pronominal clitic may occupy the position from
which the NP subject is preposed, this does not happen in Dagaare. The structure
in (\ref{ex:focusC})
illustrates object focus. In this type of focus, a complementizer, {\COMP}, is introduced in the
structure. It comes right after the focus marker and before the subject/actor of the sentence.
The structure in (\ref{ex:focusD}) illustrates adjunct focus. The adverbial, \textit{à bágúó nyɛ́}, is focused.
The following features of focus obtain. First, the adverbial is preposed. Next, it is
immediately succeeded by the focus particle, and, finally, a complementizer, {\COMP}, follows
the focus marker. Predicates may be also focused. First, the verbal predicate is nominalized
and preposed, as shown in (\ref{ex:focusE}). However, unlike the situation with argument focus, the
preposed predicate still has a verbal copy in situ. As with object focus, however, the focus
particle and the complementizer occur after the focused predicate item.



From the above, a rule for focusing elements in a Dagaare sentence seems to
emerge. In each instance, the focus element \textit{lá} comes immediately after the focused element,
which is preposed. However, depending on the type of focus in question, we may or may not
have a copy of the focused (preposed) item in situ. Also, we may or may not have a
complementizer element right after the focus marker.\bigskip


\subsubsection{ Focus and Negation}


The above illustrates focus with respect to declarative, affirmative constructions in the
language. However, sentences with  negation may also be focused. I shall
illustrate this with the structure in (\ref{ex:focusnegA}), which is the negative version of (\ref{ex:focusA}). As can be
seen in (\ref{ex:focusnegB}), the focus element in the form of \textit{lá} cannot co-occur with the negative particle
\textit{dà} in the same sentence.

\ea \ea{\label{ex:focusnegA}\gll  Dàkóráá dà bá dì à sááó bágúó sángà\\
Dakoraa {\PST} {\NEG} eat {\DEF} saao morning period\\
\glt ‘Dakoraa did not eat the saao this morning.’}
\ex[*]{\label{ex:focusnegB} \gll Dàkóráá dà bá dì là à sááó bágúó sángà\\
Dakoraa {\PST} {\NEG} eat {\FOC} {\DEF} saao morning period\\
\glt ‘Dakoraa did not eat the saao this morning.’}
\ex{\label{ex:focusnegC} \gll Dàkóráá náá dà dì à sááó bágúó sángà\\
Dakoraa {\FOC}.{\NEG} {\PST} eat {\DEF} saao morning period\\
\glt ‘DAKORAA did not eat the saao this morning.’ (It was not him.)}
\ex{\label{ex:focusnegD}\gll   à sááó náá kà Dàkóráá dà dì bágúó sángà\\
{\DEF} saao {\FOC}.{\NEG} {\COMP} Dakoraa {\PST} eat morning {\DEM}\\
\glt ‘It is not the SAAO that Dakoraa ate this morning.’}

\ex{\label{ex:focusnegE} \gll Bágúó sángà náá kà Dàkóráá dà dì à sááó\\
morning period {\FOC}.{\NEG} {\COMP} Dakoraa {\PST} eat {\DEF} saao\\
\glt ‘It was not THIS MORNING that Dakoraa ate the saao.’}
\ex{\label{ex:focusnegF} \gll   Dííú náá kà Dàkóráá dà dì à sááó bágúó sángà\\
eat-{\NMLZ} {\FOC}.{\NEG} {\COMP} Dakoraa {\PST} eat {\DEF} saao morning period\\
\glt ‘Dakoraa did not EAT the saao in the morning.’}\z\z 



From the above it may be noticed that the negative focus particle \textit{náá}
has identical distribution with its positive counterpart \textit{lá} regarding the syntax of focus
constructions in Dagaare. Argument and predicate focus are deployed in the same way as for
positive focus in the language.



\sloppy

\printbibliography[title={References}]
%[heading=references]
