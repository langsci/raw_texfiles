% This file was converted to LaTeX by Writer2LaTeX ver. 1.0.2
% see http://writer2latex.sourceforge.net for more info
\documentclass[12pt]{article}
\usepackage[utf8]{inputenc}
\usepackage[T1]{fontenc}
\usepackage[english]{babel}
\usepackage{amsmath}
\usepackage{amssymb,amsfonts,textcomp}
\usepackage{array}
\usepackage{supertabular}
\usepackage{hhline}
\usepackage{hyperref}
\hypersetup{colorlinks=true, linkcolor=blue, citecolor=blue, filecolor=blue, urlcolor=blue}
% Text styles
\newcommand\textstyleListLabelxix[1]{\textup{#1}}
\newcommand\textstyleListLabell[1]{\textup{#1}}
\newcommand\textstyleListLabelxlix[1]{\textup{#1}}
\newcommand\textstyleListLabelli[1]{\textup{#1}}
\newcommand\textstyleListLabellii[1]{\textup{#1}}
\newcommand\textstyleappleconvertedspace[1]{#1}
\newcommand\textstyleListLabellv[1]{\textup{#1}}
\newcommand\textstyleListLabellvi[1]{\textup{#1}}
\makeatletter
\newcommand\arraybslash{\let\\\@arraycr}
\makeatother
\raggedbottom
% Paragraph styles
\renewcommand\familydefault{\rmdefault}
\newenvironment{styleStandard}{\setlength\leftskip{0cm}\setlength\rightskip{0cm plus 1fil}\setlength\parindent{0cm}\setlength\parfillskip{0pt plus 1fil}\setlength\parskip{0in plus 1pt}\writerlistparindent\writerlistleftskip\leavevmode\normalfont\normalsize\writerlistlabel\ignorespaces}{\unskip\vspace{0.111in plus 0.0111in}\par}
\newenvironment{stylelsAbstract}{\setlength\leftskip{0.5in}\setlength\rightskip{0.5in}\setlength\parindent{0in}\setlength\parfillskip{0pt plus 1fil}\setlength\parskip{0in plus 1pt}\writerlistparindent\writerlistleftskip\leavevmode\normalfont\normalsize\itshape\writerlistlabel\ignorespaces}{\unskip\vspace{0.111in plus 0.0111in}\par}
\newenvironment{stylelsSectioni}{\setlength\leftskip{0.25in}\setlength\rightskip{0in plus 1fil}\setlength\parindent{0in}\setlength\parfillskip{0pt plus 1fil}\setlength\parskip{0.1665in plus 0.016649999in}\writerlistparindent\writerlistleftskip\leavevmode\normalfont\normalsize\fontsize{18pt}{21.6pt}\selectfont\bfseries\writerlistlabel\ignorespaces}{\unskip\vspace{0.0835in plus 0.00835in}\par}
\newenvironment{stylelsSectionii}{\setlength\leftskip{0.25in}\setlength\rightskip{0in plus 1fil}\setlength\parindent{0in}\setlength\parfillskip{0pt plus 1fil}\setlength\parskip{0.222in plus 0.0222in}\writerlistparindent\writerlistleftskip\leavevmode\normalfont\normalsize\fontsize{16pt}{19.2pt}\selectfont\bfseries\writerlistlabel\ignorespaces}{\unskip\vspace{0.0835in plus 0.00835in}\par}
\newenvironment{stylelsSectioniii}{\setlength\leftskip{0.5717in}\setlength\rightskip{0in plus 1fil}\setlength\parindent{0in}\setlength\parfillskip{0pt plus 1fil}\setlength\parskip{0.0972in plus 0.00972in}\writerlistparindent\writerlistleftskip\leavevmode\normalfont\normalsize\fontsize{14pt}{16.8pt}\selectfont\bfseries\writerlistlabel\ignorespaces}{\unskip\vspace{0in plus 1pt}\par}
\newenvironment{stylelsTable}{\setlength\leftskip{0cm}\setlength\rightskip{0cm}\setlength\parindent{0cm}\setlength\parfillskip{0pt plus 1fil}\setlength\parskip{0.0201in plus 0.00201in}\writerlistparindent\writerlistleftskip\leavevmode\normalfont\normalsize\mdseries\writerlistlabel\ignorespaces}{\unskip\vspace{0in plus 1pt}\par}
\newenvironment{stylelsEnumerated}{\renewcommand\baselinestretch{1.0}\setlength\leftskip{0cm}\setlength\rightskip{0cm plus 1fil}\setlength\parindent{0cm}\setlength\parfillskip{0pt plus 1fil}\setlength\parskip{0in plus 1pt}\writerlistparindent\writerlistleftskip\leavevmode\normalfont\normalsize\writerlistlabel\ignorespaces}{\unskip\vspace{0.0972in plus 0.00972in}\par}
\newenvironment{stylelsLanginfo}{\renewcommand\baselinestretch{1.0}\setlength\leftskip{0.0783in}\setlength\rightskip{0in plus 1fil}\setlength\parindent{0in}\setlength\parfillskip{0pt plus 1fil}\setlength\parskip{0in plus 1pt}\writerlistparindent\writerlistleftskip\leavevmode\normalfont\normalsize\writerlistlabel\ignorespaces}{\unskip\vspace{0in plus 1pt}\par}
\newenvironment{stylelsConversationTranscript}{\renewcommand\baselinestretch{1.0}\setlength\leftskip{0cm}\setlength\rightskip{0cm plus 1fil}\setlength\parindent{0cm}\setlength\parfillskip{0pt plus 1fil}\setlength\parskip{0in plus 1pt}\writerlistparindent\writerlistleftskip\leavevmode\normalfont\normalsize\writerlistlabel\ignorespaces}{\unskip\vspace{0.0972in plus 0.00972in}\par}
% List styles
\newcommand\writerlistleftskip{}
\newcommand\writerlistparindent{}
\newcommand\writerlistlabel{}
\newcommand\writerlistremovelabel{\aftergroup\let\aftergroup\writerlistparindent\aftergroup\relax\aftergroup\let\aftergroup\writerlistlabel\aftergroup\relax}
\newcounter{listWWNumxxiileveli}
\newcounter{listWWNumxxiilevelii}[listWWNumxxiileveli]
\newcounter{listWWNumxxiileveliii}[listWWNumxxiilevelii]
\newcounter{listWWNumxxiileveliv}[listWWNumxxiileveliii]
\renewcommand\thelistWWNumxxiileveli{\arabic{listWWNumxxiileveli}}
\renewcommand\thelistWWNumxxiilevelii{\arabic{listWWNumxxiileveli}.\arabic{listWWNumxxiilevelii}}
\renewcommand\thelistWWNumxxiileveliii{\arabic{listWWNumxxiileveli}.\arabic{listWWNumxxiilevelii}.\arabic{listWWNumxxiileveliii}}
\renewcommand\thelistWWNumxxiileveliv{\arabic{listWWNumxxiileveli}.\arabic{listWWNumxxiilevelii}.\arabic{listWWNumxxiileveliii}.\arabic{listWWNumxxiileveliv}}
\newcommand\labellistWWNumxxiileveli{\thelistWWNumxxiileveli.}
\newcommand\labellistWWNumxxiilevelii{\thelistWWNumxxiilevelii.}
\newcommand\labellistWWNumxxiileveliii{\thelistWWNumxxiileveliii.}
\newcommand\labellistWWNumxxiileveliv{\thelistWWNumxxiileveliv.}
\newenvironment{listWWNumxxiileveli}{\def\writerlistleftskip{\addtolength\leftskip{0.0cm}}\def\writerlistparindent{}\def\writerlistlabel{}\def\item{\def\writerlistparindent{\setlength\parindent{-0cm}}\def\writerlistlabel{\stepcounter{listWWNumxxiileveli}\makebox[0cm][l]{\labellistWWNumxxiileveli}\hspace{0cm}\writerlistremovelabel}}}{}
\newenvironment{listWWNumxxiilevelii}{\def\writerlistleftskip{\addtolength\leftskip{0.0cm}}\def\writerlistparindent{}\def\writerlistlabel{}\def\item{\def\writerlistparindent{\setlength\parindent{-0cm}}\def\writerlistlabel{\stepcounter{listWWNumxxiilevelii}\makebox[0cm][l]{\labellistWWNumxxiilevelii}\hspace{0cm}\writerlistremovelabel}}}{}
\newenvironment{listWWNumxxiileveliii}{\def\writerlistleftskip{\addtolength\leftskip{0.0cm}}\def\writerlistparindent{}\def\writerlistlabel{}\def\item{\def\writerlistparindent{\setlength\parindent{-0cm}}\def\writerlistlabel{\stepcounter{listWWNumxxiileveliii}\makebox[0cm][r]{\labellistWWNumxxiileveliii}\hspace{0cm}\writerlistremovelabel}}}{}
\newenvironment{listWWNumxxiileveliv}{\def\writerlistleftskip{\addtolength\leftskip{0.0cm}}\def\writerlistparindent{}\def\writerlistlabel{}\def\item{\def\writerlistparindent{\setlength\parindent{-0cm}}\def\writerlistlabel{\stepcounter{listWWNumxxiileveliv}\makebox[0cm][l]{\labellistWWNumxxiileveliv}\hspace{0cm}\writerlistremovelabel}}}{}
\newcounter{listWWNumxleveli}
\newcounter{listWWNumxlevelii}[listWWNumxleveli]
\newcounter{listWWNumxleveliii}[listWWNumxlevelii]
\newcounter{listWWNumxleveliv}[listWWNumxleveliii]
\renewcommand\thelistWWNumxleveli{\arabic{listWWNumxleveli}}
\renewcommand\thelistWWNumxlevelii{\alph{listWWNumxlevelii}}
\renewcommand\thelistWWNumxleveliii{\roman{listWWNumxleveliii}}
\renewcommand\thelistWWNumxleveliv{\arabic{listWWNumxleveliv}}
\newcommand\labellistWWNumxleveli{\textstyleListLabelxix{\thelistWWNumxleveli.}}
\newcommand\labellistWWNumxlevelii{\thelistWWNumxlevelii.}
\newcommand\labellistWWNumxleveliii{\thelistWWNumxleveliii.}
\newcommand\labellistWWNumxleveliv{\thelistWWNumxleveliv.}
\newenvironment{listWWNumxleveli}{\def\writerlistleftskip{\addtolength\leftskip{0.0cm}}\def\writerlistparindent{}\def\writerlistlabel{}\def\item{\def\writerlistparindent{\setlength\parindent{-0cm}}\def\writerlistlabel{\stepcounter{listWWNumxleveli}\makebox[0cm][l]{\labellistWWNumxleveli}\hspace{0cm}\writerlistremovelabel}}}{}
\newenvironment{listWWNumxlevelii}{\def\writerlistleftskip{\addtolength\leftskip{0.0cm}}\def\writerlistparindent{}\def\writerlistlabel{}\def\item{\def\writerlistparindent{\setlength\parindent{-0cm}}\def\writerlistlabel{\stepcounter{listWWNumxlevelii}\makebox[0cm][l]{\labellistWWNumxlevelii}\hspace{0cm}\writerlistremovelabel}}}{}
\newenvironment{listWWNumxleveliii}{\def\writerlistleftskip{\addtolength\leftskip{0.0cm}}\def\writerlistparindent{}\def\writerlistlabel{}\def\item{\def\writerlistparindent{\setlength\parindent{-0cm}}\def\writerlistlabel{\stepcounter{listWWNumxleveliii}\makebox[0cm][r]{\labellistWWNumxleveliii}\hspace{0cm}\writerlistremovelabel}}}{}
\newenvironment{listWWNumxleveliv}{\def\writerlistleftskip{\addtolength\leftskip{0.0cm}}\def\writerlistparindent{}\def\writerlistlabel{}\def\item{\def\writerlistparindent{\setlength\parindent{-0cm}}\def\writerlistlabel{\stepcounter{listWWNumxleveliv}\makebox[0cm][l]{\labellistWWNumxleveliv}\hspace{0cm}\writerlistremovelabel}}}{}
\newcounter{listWWNumxxxvileveli}
\newcounter{listWWNumxxxvilevelii}[listWWNumxxxvileveli]
\newcounter{listWWNumxxxvileveliii}[listWWNumxxxvilevelii]
\newcounter{listWWNumxxxvileveliv}[listWWNumxxxvileveliii]
\renewcommand\thelistWWNumxxxvileveli{\arabic{listWWNumxxxvileveli}}
\renewcommand\thelistWWNumxxxvilevelii{\alph{listWWNumxxxvilevelii}}
\renewcommand\thelistWWNumxxxvileveliii{\roman{listWWNumxxxvileveliii}}
\renewcommand\thelistWWNumxxxvileveliv{\arabic{listWWNumxxxvileveliv}}
\newcommand\labellistWWNumxxxvileveli{\textstyleListLabell{\thelistWWNumxxxvileveli.}}
\newcommand\labellistWWNumxxxvilevelii{\thelistWWNumxxxvilevelii.}
\newcommand\labellistWWNumxxxvileveliii{\thelistWWNumxxxvileveliii.}
\newcommand\labellistWWNumxxxvileveliv{\thelistWWNumxxxvileveliv.}
\newenvironment{listWWNumxxxvileveli}{\def\writerlistleftskip{\addtolength\leftskip{0.0cm}}\def\writerlistparindent{}\def\writerlistlabel{}\def\item{\def\writerlistparindent{\setlength\parindent{-0cm}}\def\writerlistlabel{\stepcounter{listWWNumxxxvileveli}\makebox[0cm][l]{\labellistWWNumxxxvileveli}\hspace{0cm}\writerlistremovelabel}}}{}
\newenvironment{listWWNumxxxvilevelii}{\def\writerlistleftskip{\addtolength\leftskip{0.0cm}}\def\writerlistparindent{}\def\writerlistlabel{}\def\item{\def\writerlistparindent{\setlength\parindent{-0cm}}\def\writerlistlabel{\stepcounter{listWWNumxxxvilevelii}\makebox[0cm][l]{\labellistWWNumxxxvilevelii}\hspace{0cm}\writerlistremovelabel}}}{}
\newenvironment{listWWNumxxxvileveliii}{\def\writerlistleftskip{\addtolength\leftskip{0.0cm}}\def\writerlistparindent{}\def\writerlistlabel{}\def\item{\def\writerlistparindent{\setlength\parindent{-0cm}}\def\writerlistlabel{\stepcounter{listWWNumxxxvileveliii}\makebox[0cm][r]{\labellistWWNumxxxvileveliii}\hspace{0cm}\writerlistremovelabel}}}{}
\newenvironment{listWWNumxxxvileveliv}{\def\writerlistleftskip{\addtolength\leftskip{0.0cm}}\def\writerlistparindent{}\def\writerlistlabel{}\def\item{\def\writerlistparindent{\setlength\parindent{-0cm}}\def\writerlistlabel{\stepcounter{listWWNumxxxvileveliv}\makebox[0cm][l]{\labellistWWNumxxxvileveliv}\hspace{0cm}\writerlistremovelabel}}}{}
\newcounter{listWWNumxxxvleveli}
\newcounter{listWWNumxxxvlevelii}[listWWNumxxxvleveli]
\newcounter{listWWNumxxxvleveliii}[listWWNumxxxvlevelii]
\newcounter{listWWNumxxxvleveliv}[listWWNumxxxvleveliii]
\renewcommand\thelistWWNumxxxvleveli{\arabic{listWWNumxxxvleveli}}
\renewcommand\thelistWWNumxxxvlevelii{\alph{listWWNumxxxvlevelii}}
\renewcommand\thelistWWNumxxxvleveliii{\roman{listWWNumxxxvleveliii}}
\renewcommand\thelistWWNumxxxvleveliv{\arabic{listWWNumxxxvleveliv}}
\newcommand\labellistWWNumxxxvleveli{\textstyleListLabelxlix{\thelistWWNumxxxvleveli.}}
\newcommand\labellistWWNumxxxvlevelii{\thelistWWNumxxxvlevelii.}
\newcommand\labellistWWNumxxxvleveliii{\thelistWWNumxxxvleveliii.}
\newcommand\labellistWWNumxxxvleveliv{\thelistWWNumxxxvleveliv.}
\newenvironment{listWWNumxxxvleveli}{\def\writerlistleftskip{\addtolength\leftskip{0.0cm}}\def\writerlistparindent{}\def\writerlistlabel{}\def\item{\def\writerlistparindent{\setlength\parindent{-0cm}}\def\writerlistlabel{\stepcounter{listWWNumxxxvleveli}\makebox[0cm][l]{\labellistWWNumxxxvleveli}\hspace{0cm}\writerlistremovelabel}}}{}
\newenvironment{listWWNumxxxvlevelii}{\def\writerlistleftskip{\addtolength\leftskip{0.0cm}}\def\writerlistparindent{}\def\writerlistlabel{}\def\item{\def\writerlistparindent{\setlength\parindent{-0cm}}\def\writerlistlabel{\stepcounter{listWWNumxxxvlevelii}\makebox[0cm][l]{\labellistWWNumxxxvlevelii}\hspace{0cm}\writerlistremovelabel}}}{}
\newenvironment{listWWNumxxxvleveliii}{\def\writerlistleftskip{\addtolength\leftskip{0.0cm}}\def\writerlistparindent{}\def\writerlistlabel{}\def\item{\def\writerlistparindent{\setlength\parindent{-0cm}}\def\writerlistlabel{\stepcounter{listWWNumxxxvleveliii}\makebox[0cm][r]{\labellistWWNumxxxvleveliii}\hspace{0cm}\writerlistremovelabel}}}{}
\newenvironment{listWWNumxxxvleveliv}{\def\writerlistleftskip{\addtolength\leftskip{0.0cm}}\def\writerlistparindent{}\def\writerlistlabel{}\def\item{\def\writerlistparindent{\setlength\parindent{-0cm}}\def\writerlistlabel{\stepcounter{listWWNumxxxvleveliv}\makebox[0cm][l]{\labellistWWNumxxxvleveliv}\hspace{0cm}\writerlistremovelabel}}}{}
\newcounter{listWWNumxxxviileveli}
\newcounter{listWWNumxxxviilevelii}[listWWNumxxxviileveli]
\newcounter{listWWNumxxxviileveliii}[listWWNumxxxviilevelii]
\newcounter{listWWNumxxxviileveliv}[listWWNumxxxviileveliii]
\renewcommand\thelistWWNumxxxviileveli{\arabic{listWWNumxxxviileveli}}
\renewcommand\thelistWWNumxxxviilevelii{\alph{listWWNumxxxviilevelii}}
\renewcommand\thelistWWNumxxxviileveliii{\roman{listWWNumxxxviileveliii}}
\renewcommand\thelistWWNumxxxviileveliv{\arabic{listWWNumxxxviileveliv}}
\newcommand\labellistWWNumxxxviileveli{\textstyleListLabelli{\thelistWWNumxxxviileveli.}}
\newcommand\labellistWWNumxxxviilevelii{\thelistWWNumxxxviilevelii.}
\newcommand\labellistWWNumxxxviileveliii{\thelistWWNumxxxviileveliii.}
\newcommand\labellistWWNumxxxviileveliv{\thelistWWNumxxxviileveliv.}
\newenvironment{listWWNumxxxviileveli}{\def\writerlistleftskip{\addtolength\leftskip{0.0cm}}\def\writerlistparindent{}\def\writerlistlabel{}\def\item{\def\writerlistparindent{\setlength\parindent{-0cm}}\def\writerlistlabel{\stepcounter{listWWNumxxxviileveli}\makebox[0cm][l]{\labellistWWNumxxxviileveli}\hspace{0cm}\writerlistremovelabel}}}{}
\newenvironment{listWWNumxxxviilevelii}{\def\writerlistleftskip{\addtolength\leftskip{0.0cm}}\def\writerlistparindent{}\def\writerlistlabel{}\def\item{\def\writerlistparindent{\setlength\parindent{-0cm}}\def\writerlistlabel{\stepcounter{listWWNumxxxviilevelii}\makebox[0cm][l]{\labellistWWNumxxxviilevelii}\hspace{0cm}\writerlistremovelabel}}}{}
\newenvironment{listWWNumxxxviileveliii}{\def\writerlistleftskip{\addtolength\leftskip{0.0cm}}\def\writerlistparindent{}\def\writerlistlabel{}\def\item{\def\writerlistparindent{\setlength\parindent{-0cm}}\def\writerlistlabel{\stepcounter{listWWNumxxxviileveliii}\makebox[0cm][r]{\labellistWWNumxxxviileveliii}\hspace{0cm}\writerlistremovelabel}}}{}
\newenvironment{listWWNumxxxviileveliv}{\def\writerlistleftskip{\addtolength\leftskip{0.0cm}}\def\writerlistparindent{}\def\writerlistlabel{}\def\item{\def\writerlistparindent{\setlength\parindent{-0cm}}\def\writerlistlabel{\stepcounter{listWWNumxxxviileveliv}\makebox[0cm][l]{\labellistWWNumxxxviileveliv}\hspace{0cm}\writerlistremovelabel}}}{}
\newcounter{listWWNumxxxviiileveli}
\newcounter{listWWNumxxxviiilevelii}[listWWNumxxxviiileveli]
\newcounter{listWWNumxxxviiileveliii}[listWWNumxxxviiilevelii]
\newcounter{listWWNumxxxviiileveliv}[listWWNumxxxviiileveliii]
\renewcommand\thelistWWNumxxxviiileveli{\arabic{listWWNumxxxviiileveli}}
\renewcommand\thelistWWNumxxxviiilevelii{\alph{listWWNumxxxviiilevelii}}
\renewcommand\thelistWWNumxxxviiileveliii{\roman{listWWNumxxxviiileveliii}}
\renewcommand\thelistWWNumxxxviiileveliv{\arabic{listWWNumxxxviiileveliv}}
\newcommand\labellistWWNumxxxviiileveli{\textstyleListLabellii{\thelistWWNumxxxviiileveli.}}
\newcommand\labellistWWNumxxxviiilevelii{\thelistWWNumxxxviiilevelii.}
\newcommand\labellistWWNumxxxviiileveliii{\thelistWWNumxxxviiileveliii.}
\newcommand\labellistWWNumxxxviiileveliv{\thelistWWNumxxxviiileveliv.}
\newenvironment{listWWNumxxxviiileveli}{\def\writerlistleftskip{\addtolength\leftskip{0.0cm}}\def\writerlistparindent{}\def\writerlistlabel{}\def\item{\def\writerlistparindent{\setlength\parindent{-0cm}}\def\writerlistlabel{\stepcounter{listWWNumxxxviiileveli}\makebox[0cm][l]{\labellistWWNumxxxviiileveli}\hspace{0cm}\writerlistremovelabel}}}{}
\newenvironment{listWWNumxxxviiilevelii}{\def\writerlistleftskip{\addtolength\leftskip{0.0cm}}\def\writerlistparindent{}\def\writerlistlabel{}\def\item{\def\writerlistparindent{\setlength\parindent{-0cm}}\def\writerlistlabel{\stepcounter{listWWNumxxxviiilevelii}\makebox[0cm][l]{\labellistWWNumxxxviiilevelii}\hspace{0cm}\writerlistremovelabel}}}{}
\newenvironment{listWWNumxxxviiileveliii}{\def\writerlistleftskip{\addtolength\leftskip{0.0cm}}\def\writerlistparindent{}\def\writerlistlabel{}\def\item{\def\writerlistparindent{\setlength\parindent{-0cm}}\def\writerlistlabel{\stepcounter{listWWNumxxxviiileveliii}\makebox[0cm][r]{\labellistWWNumxxxviiileveliii}\hspace{0cm}\writerlistremovelabel}}}{}
\newenvironment{listWWNumxxxviiileveliv}{\def\writerlistleftskip{\addtolength\leftskip{0.0cm}}\def\writerlistparindent{}\def\writerlistlabel{}\def\item{\def\writerlistparindent{\setlength\parindent{-0cm}}\def\writerlistlabel{\stepcounter{listWWNumxxxviiileveliv}\makebox[0cm][l]{\labellistWWNumxxxviiileveliv}\hspace{0cm}\writerlistremovelabel}}}{}
\newcounter{listWWNumiileveli}
\newcounter{listWWNumiilevelii}[listWWNumiileveli]
\newcounter{listWWNumiileveliii}[listWWNumiilevelii]
\newcounter{listWWNumiileveliv}[listWWNumiileveliii]
\renewcommand\thelistWWNumiileveli{\arabic{listWWNumiileveli}}
\renewcommand\thelistWWNumiilevelii{\alph{listWWNumiilevelii}}
\renewcommand\thelistWWNumiileveliii{}
\renewcommand\thelistWWNumiileveliv{}
\newcommand\labellistWWNumiileveli{(\thelistWWNumiileveli)}
\newcommand\labellistWWNumiilevelii{\thelistWWNumiilevelii.}
\newcommand\labellistWWNumiileveliii{\thelistWWNumiileveliii}
\newcommand\labellistWWNumiileveliv{\thelistWWNumiileveliv}
\newenvironment{listWWNumiileveli}{\def\writerlistleftskip{\addtolength\leftskip{0.0cm}}\def\writerlistparindent{}\def\writerlistlabel{}\def\item{\def\writerlistparindent{\setlength\parindent{-0cm}}\def\writerlistlabel{\stepcounter{listWWNumiileveli}\makebox[0cm][l]{\labellistWWNumiileveli}\hspace{0cm}\writerlistremovelabel}}}{}
\newenvironment{listWWNumiilevelii}{\def\writerlistleftskip{\addtolength\leftskip{0.0cm}}\def\writerlistparindent{}\def\writerlistlabel{}\def\item{\def\writerlistparindent{\setlength\parindent{-0cm}}\def\writerlistlabel{\stepcounter{listWWNumiilevelii}\makebox[0cm][l]{\labellistWWNumiilevelii}\hspace{0cm}\writerlistremovelabel}}}{}
\newenvironment{listWWNumiileveliii}{\def\writerlistleftskip{\addtolength\leftskip{0.0cm}}\def\writerlistparindent{}\def\writerlistlabel{}\def\item{\def\writerlistparindent{\setlength\parindent{-0cm}}\def\writerlistlabel{\stepcounter{listWWNumiileveliii}\makebox[0cm][l]{\labellistWWNumiileveliii}\hspace{0cm}\writerlistremovelabel}}}{}
\newenvironment{listWWNumiileveliv}{\def\writerlistleftskip{\addtolength\leftskip{0.0cm}}\def\writerlistparindent{}\def\writerlistlabel{}\def\item{\def\writerlistparindent{\setlength\parindent{-0cm}}\def\writerlistlabel{\stepcounter{listWWNumiileveliv}\makebox[0cm][l]{\labellistWWNumiileveliv}\hspace{0cm}\writerlistremovelabel}}}{}
\newcounter{listWWNumxlileveli}
\newcounter{listWWNumxlilevelii}[listWWNumxlileveli]
\newcounter{listWWNumxlileveliii}[listWWNumxlilevelii]
\newcounter{listWWNumxlileveliv}[listWWNumxlileveliii]
\renewcommand\thelistWWNumxlileveli{\arabic{listWWNumxlileveli}}
\renewcommand\thelistWWNumxlilevelii{\alph{listWWNumxlilevelii}}
\renewcommand\thelistWWNumxlileveliii{\roman{listWWNumxlileveliii}}
\renewcommand\thelistWWNumxlileveliv{\arabic{listWWNumxlileveliv}}
\newcommand\labellistWWNumxlileveli{\textstyleListLabellv{\thelistWWNumxlileveli.}}
\newcommand\labellistWWNumxlilevelii{\thelistWWNumxlilevelii.}
\newcommand\labellistWWNumxlileveliii{\thelistWWNumxlileveliii.}
\newcommand\labellistWWNumxlileveliv{\thelistWWNumxlileveliv.}
\newenvironment{listWWNumxlileveli}{\def\writerlistleftskip{\addtolength\leftskip{0.0cm}}\def\writerlistparindent{}\def\writerlistlabel{}\def\item{\def\writerlistparindent{\setlength\parindent{-0cm}}\def\writerlistlabel{\stepcounter{listWWNumxlileveli}\makebox[0cm][l]{\labellistWWNumxlileveli}\hspace{0cm}\writerlistremovelabel}}}{}
\newenvironment{listWWNumxlilevelii}{\def\writerlistleftskip{\addtolength\leftskip{0.0cm}}\def\writerlistparindent{}\def\writerlistlabel{}\def\item{\def\writerlistparindent{\setlength\parindent{-0cm}}\def\writerlistlabel{\stepcounter{listWWNumxlilevelii}\makebox[0cm][l]{\labellistWWNumxlilevelii}\hspace{0cm}\writerlistremovelabel}}}{}
\newenvironment{listWWNumxlileveliii}{\def\writerlistleftskip{\addtolength\leftskip{0.0cm}}\def\writerlistparindent{}\def\writerlistlabel{}\def\item{\def\writerlistparindent{\setlength\parindent{-0cm}}\def\writerlistlabel{\stepcounter{listWWNumxlileveliii}\makebox[0cm][r]{\labellistWWNumxlileveliii}\hspace{0cm}\writerlistremovelabel}}}{}
\newenvironment{listWWNumxlileveliv}{\def\writerlistleftskip{\addtolength\leftskip{0.0cm}}\def\writerlistparindent{}\def\writerlistlabel{}\def\item{\def\writerlistparindent{\setlength\parindent{-0cm}}\def\writerlistlabel{\stepcounter{listWWNumxlileveliv}\makebox[0cm][l]{\labellistWWNumxlileveliv}\hspace{0cm}\writerlistremovelabel}}}{}
\newcounter{listWWNumxliiileveli}
\newcounter{listWWNumxliiilevelii}[listWWNumxliiileveli]
\newcounter{listWWNumxliiileveliii}[listWWNumxliiilevelii]
\newcounter{listWWNumxliiileveliv}[listWWNumxliiileveliii]
\renewcommand\thelistWWNumxliiileveli{\arabic{listWWNumxliiileveli}}
\renewcommand\thelistWWNumxliiilevelii{\alph{listWWNumxliiilevelii}}
\renewcommand\thelistWWNumxliiileveliii{\roman{listWWNumxliiileveliii}}
\renewcommand\thelistWWNumxliiileveliv{\arabic{listWWNumxliiileveliv}}
\newcommand\labellistWWNumxliiileveli{\textstyleListLabellvi{\thelistWWNumxliiileveli.}}
\newcommand\labellistWWNumxliiilevelii{\thelistWWNumxliiilevelii.}
\newcommand\labellistWWNumxliiileveliii{\thelistWWNumxliiileveliii.}
\newcommand\labellistWWNumxliiileveliv{\thelistWWNumxliiileveliv.}
\newenvironment{listWWNumxliiileveli}{\def\writerlistleftskip{\addtolength\leftskip{0.0cm}}\def\writerlistparindent{}\def\writerlistlabel{}\def\item{\def\writerlistparindent{\setlength\parindent{-0cm}}\def\writerlistlabel{\stepcounter{listWWNumxliiileveli}\makebox[0cm][l]{\labellistWWNumxliiileveli}\hspace{0cm}\writerlistremovelabel}}}{}
\newenvironment{listWWNumxliiilevelii}{\def\writerlistleftskip{\addtolength\leftskip{0.0cm}}\def\writerlistparindent{}\def\writerlistlabel{}\def\item{\def\writerlistparindent{\setlength\parindent{-0cm}}\def\writerlistlabel{\stepcounter{listWWNumxliiilevelii}\makebox[0cm][l]{\labellistWWNumxliiilevelii}\hspace{0cm}\writerlistremovelabel}}}{}
\newenvironment{listWWNumxliiileveliii}{\def\writerlistleftskip{\addtolength\leftskip{0.0cm}}\def\writerlistparindent{}\def\writerlistlabel{}\def\item{\def\writerlistparindent{\setlength\parindent{-0cm}}\def\writerlistlabel{\stepcounter{listWWNumxliiileveliii}\makebox[0cm][r]{\labellistWWNumxliiileveliii}\hspace{0cm}\writerlistremovelabel}}}{}
\newenvironment{listWWNumxliiileveliv}{\def\writerlistleftskip{\addtolength\leftskip{0.0cm}}\def\writerlistparindent{}\def\writerlistlabel{}\def\item{\def\writerlistparindent{\setlength\parindent{-0cm}}\def\writerlistlabel{\stepcounter{listWWNumxliiileveliv}\makebox[0cm][l]{\labellistWWNumxliiileveliv}\hspace{0cm}\writerlistremovelabel}}}{}
\setlength\tabcolsep{1mm}
\renewcommand\arraystretch{1.3}
\title{}
\author{UIC}
\date{2018-03-09}
\begin{document}
\title{\textsuperscript{The acquisition of discourse markers in the English-medium instruction context}}
\maketitle

\begin{styleStandard}
\textbf{Jennifer Ament \ \ }
\end{styleStandard}


\begin{styleStandard}
Universitat Internacional de Catalunya, Universitat Pompeu Fabra 
\end{styleStandard}


\begin{styleStandard}
\textbf{Júlia Barón Parés }
\end{styleStandard}


\begin{styleStandard}
Universitat Internacional de Catalunya, Universitat de Barcelona 
\end{styleStandard}


\begin{stylelsAbstract}
This study focuses on the effects of the context of learning on language acquisition by comparing the production of discourse markers (DMs) in oral output of English-medium instruction (EMI) students (N = 7) with non-EMI students (N = 9). Data were elicited through an oral discourse completion task and a conversation task. Four types of DMs were identified: cognitive, interpersonal, structural and referential. Quantitative analysis reveals that EMI students tend to produce longer responses and more structural markers, as opposed to control students, who use more referential markers. \ A qualitative interpretation of the data suggests that the EMI participants mark their discourse for their own as well as for their interlocutor’s benefit, specifically by using structural markers to ensure clear interpretation of utterances. The study further suggests that participation in an EMI program may lead to pragmatic benefits specifically in terms of the type and quality of DMs used, rather than of their frequency and overall variety. \ However, the study also indicates that this context alone may not be sufficient for the acquisition of all types of markers, and that there are many other factors at play in the acquisition of this pragmatic feature.
\end{stylelsAbstract}

\setcounter{listWWNumxxiileveli}{0}
\begin{listWWNumxxiileveli}
\item 
\begin{stylelsSectioni}
Introduction
\end{stylelsSectioni}

\end{listWWNumxxiileveli}
\begin{styleStandard}
This exploratory study examines the acquisition of discourse markers (DMs) in second language acquisition. The function of DMs as connectors in discourse makes them essential to smooth communication, as they facilitate the correct interpretation of an utterance and express the speakers’ intentions (Ariel 1998). Despite the attested importance of these markers, DMs are seldom addressed in second language classroom instruction (Vellenga 2004). Thus, learners are left with the difficult task of, firstly, interpreting, and secondly, integrating them effectively into their own speech. As this volume highlights, the context of learning plays an important role in second language acquisition, for it has been found that different contexts of learning foster the development of different language skills. General conclusions from research are that, for optimal language learning to occur, participation in more than one context is desirable (Pérez-Vidal 2014). More particularly, integrated content and language contexts, in which curricular subjects are taught through the medium of a second or foreign language, can lead to very positive outcomes in the domains of receptive skills, vocabulary, morphology, speaking, creativity, and motivation (Pérez-Cañado 2012). Regarding pragmatics, research shows that integrated context and language classes provide opportunities for incidental pragmatic learning (Taguchi 2015).
\end{styleStandard}

\begin{listWWNumxxiileveli}
\item 
\begin{stylelsSectioni}
Literature review
\end{stylelsSectioni}

\end{listWWNumxxiileveli}
\begin{styleStandard}
The rationale and motivation for the present study are that, firstly, there are scarce data regarding the acquisition of DMs by second language (L2) learners, despite their importance for successful communication. Secondly, the ever growing importance of English-medium instruction (EMI) in Europe today has both social and political consequences. Thus, knowing more about language acquisition in this setting can help inform higher education institutions across Europe regarding what types of linguistic gains can be expected from participation in EMI programs and what kind of language support is needed for students receiving their education through EMI. The literature review consists of two parts; First, an overview of EMI will be provided to contextualize the present study. In the second part, DMs are identified and classified according to their functions, and studies examining their acquisition are discussed.
\end{styleStandard}

\begin{listWWNumxxiileveli}
\item 
\setcounter{listWWNumxxiilevelii}{0}
\begin{listWWNumxxiilevelii}
\item 
\begin{stylelsSectionii}
English-medium instruction
\end{stylelsSectionii}

\end{listWWNumxxiilevelii}
\item 
\begin{stylelsSectioniii}
Policies
\end{stylelsSectioniii}

\end{listWWNumxxiileveli}
\begin{styleStandard}
While many factors have contributed to the rise of EMI across Europe, the Bologna process was perhaps the most impactful (The European Minister of Education 1999). This large-scale policy change, which sought to encourage the mobility of students and faculty within Europe, had widespread effects on language policies across the European Union. In efforts to become more competitive and attractive to both faculty and students from other countries, universities began to offer degrees either partially or completely taught through languages other than the official language of the country, most notably English (Coleman 2006; Llurda, Cots, \& Armengol 2013; Pérez-Vidal 2015). In fact, the number of EMI courses tripled from 1998 to 2008 in Europe (Wächter \& Maiworm 2008). The rapid implementation of EMI programs continues to rise to this day reaching nearly 6\% of all programs offered in Europe (Smit \& Dafouz 2012; Wächter \& Maiworm 2014).
\end{styleStandard}

\begin{listWWNumxxiileveli}
\item 
\begin{stylelsSectioniii}
Contextualization
\end{stylelsSectioniii}

\end{listWWNumxxiileveli}
\begin{styleStandard}
EMI can be defined as a context in which English is used as the language of instruction, in tertiary education, in non-English speaking countries (Hellekjaer \& Hellekjaer 2015). However, different regions and even individual universities have integrated EMI into their specific context in unique ways, thus making EMI somewhat of an umbrella term, for which specific realizations may differ across institutions. For instance, some regions have found it necessary to protect local languages, as was the case in the autonomous regions of Catalonia and the Basque Country in Spain. When introducing EMI programs in these two regions, the decision was made to implement trilingual policies with a view to protect and promote learning of regional languages (See Pérez-Vidal 2008; and Doiz, Lasagabaster \& Sierra 2014). Similarly, Nordic countries question if there is perhaps an over-reliance on English in academic contexts, and steps are being taken to protect national languages in research and education (Nordic Council of Ministers 2006). Thus, as demonstrated, program structure or intensity of EMI differs according to each community’s language needs. Some may have full EMI programs while others only a small percentage of EMI courses. Institutions differ as well according to what type of language support is offered to students, faculty and administration (regarding both English or national languages). Despite the differences in structure, when a course is offered through EMI, there are also some constants, such as a strong focus on content and little attention or support offered to aid language learning. Although EMI courses are now widespread, there is scarce research on how they are implemented in practice; only a handful of studies have been conducted, which reveal that lecturers do not focus on language, and that they may feel uncomfortable correcting errors as they are often non-native speakers of English themselves (Costa 2012). Lecturers are experts in their disciplinary fields and do not consider themselves language specialists; their aim, from their point of view, is therefore to educate students on their subject of expertise (Airey 2012; Unterberger 2012).
\end{styleStandard}

\begin{listWWNumxxiileveli}
\item 
\begin{stylelsSectioniii}
Research on EMI
\end{stylelsSectioniii}

\end{listWWNumxxiileveli}
\begin{styleStandard}
Within the European Union, a body of research on EMI from a second language acquisition perspective has begun to emerge. Much of such research has taken a qualitative approach, investigating such topics as lecturers’ and students’ attitudes and beliefs towards EMI (Kling 2013; Kuteeva \& Airey 2014). These studies seek to inform policymakers, program creators, language teachers, and professional development departments about how EMI is implemented in different institutions. Concerning content learning, research reveals that students find courses harder and the workload heavier when taught through English (Tazl 2011) and that EMI is not perceived as equal to first language instruction in terms of content delivery (Sert 2008). It has also been reported across a wide variety of contexts that students expect language gains when participating in EMI programs (Pecorari, Shaw, Irvine, \& Malmström 2011; Gundermann 2014; Lueg \& Lueg 2015; Margić \& Žeželić 2015). However, as mentioned above, there is hardly any language support provided to students during EMI degree programs, and language learning is not an explicit goal of such programs. Thus, investigating whether and how EMI leads to gains in linguistic competence is an area where more research is needed, and this chapter intends to offer a contribution in this direction.
\end{styleStandard}

\begin{listWWNumxxiileveli}
\item 
\setcounter{listWWNumxxiilevelii}{0}
\begin{listWWNumxxiilevelii}
\item 
\begin{stylelsSectionii}
Discourse Markers
\end{stylelsSectionii}

\end{listWWNumxxiilevelii}
\end{listWWNumxxiileveli}
\begin{styleStandard}
Discourse markers seem to play an important role both in first and second language acquisition, since they are constantly used by native speakers (NSs) and non-native speakers in interaction. As Yates (2011) points out, DMs help one interpret the speakers’ attitudes towards the content of their messages and they tend to carry socio-pragmatic meaning. What some studies in SLA have found is that foreign and second language learners tend to use a narrower variety of DMs than NSs do, and that they seem to be less aware of the multifunctional uses of DMs (Vanda 2007; Yates 2011). However, even if DMs seem to be key elements in interaction, defining and categorizing them is a complex issue, as the literature in the field has shown (see Fischer 2006 for review). First of all, different terms such as pragmatic markers, discourse particles, discourse connectives, conversational markers, among others have been used to refer to these different linguistic items which have specific cohesive functions and important interpretive roles in conversation. Secondly, the multifunctional nature of some DMs has not been reflected in most of the categorizations presented so far, since most of these elements tend to have different functions depending on the context and situation where they are produced. Thirdly, one of the most problematic issues may be the grammaticalization of some DMs, for some tend to overlap syntactically with subordinating conjunctions or coordinators, while some others may simply connect different parts of discourse. All these issues have contributed to categorizations that fail to completely and accurately describe what discourse markers really show in terms of structure and function (Fischer 2014). 
\end{styleStandard}

\begin{styleStandard}
Even if no clear definitions can be found in the literature, many studies have identified some common characteristics among DMs. Most of them seem to show flexibility, that is, they are flexible in terms of their placement and use in discourse; additionally, they also encode speakers’ intentions and interpersonal meanings (Carter \& McCarthy 2006); they also carry little semantic meaning (Schiffrin 1987), but at the same time are essential to the natural flow of speech, as well as to correct interpretation (Neary-Sundquist 2013). Another aspect which has been reported in various studies is that hearers seem to rely on DMs to interpret and follow discourse (Blakemore 1992; Aijmer 1996), so, as Leech \& Svartvik (1975: 156) suggest, by using DMs “in speech or in writing, you help people understand your message by signaling how one idea leads on from another. The words and phrases which have this connecting function are like ‘signposts’ on a journey”. Thus, these common characteristics are important elements for identifying discourse markers, which may have elusive referential meanings on the surface, but play important roles on different planes of communication (Schiffrin 1987).
\end{styleStandard}

\begin{styleStandard}
In an attempt to describe how DMs are used by non-native speakers, in the present study it was decided to follow existing categorizations in order to assess their adequacy for analyzing learners’ discourse. Therefore, following Maschler (1994) and Fung \& Carter (2007), the present study analyzes DMs according to four functional categories: \textit{cognitive, structural, referential, }and\textit{ interpersonal. }Each category serves several related functions. DMs in the \textit{cognitive} category are thought to provide information on the cognitive state of the speaker and instruct the hearer as to how to construct their mental representation of the ongoing discourse. \textit{Structural }DMs serve metalinguistic textual functions on how the flow of discourse is to be segmented. \textit{Referential} DMs mark relationships between the utterances before and after the DM; these relationships may be marked by conjunctions, and may be completely grammatically integrated while at the same time functioning pragmatically (Fischer 2014), DMs in this category seem to be more syntactically and textually bound than the other DM categories. The final category, \textit{interpersonal} DMs, are thought to be used to mark affective and social functions of spoken grammar, and indicate how the speaker feels towards the discourse statements (Anderson 2001). See Table 1 for a summary of the categories, functions and examples.
\end{styleStandard}

\begin{stylelsTable}
\textbf{\textit{Table 1.}}\textit{ Categorization of pragmatic markers according to functions}
\end{stylelsTable}

\begin{flushleft}
\tablehead{}
\begin{supertabular}{m{1.0927598in}m{1.9733598in}m{1.6170598in}}
\hline
\bfseries Category  &
\bfseries Functions &
\bfseries Example Items\\\hline
\mdseries Cognitive &
{\mdseries Denote thinking process}

{\mdseries Reformulation/Self-Correction}

{\mdseries Elaboration / Hesitation}

\mdseries Assessment of the listener’s knowledge about utterances &
{\mdseries \textit{Well, I think}}

{\mdseries \textit{In other words, I mean}}

{\mdseries \textit{It’s like / sort of, well}}

\mdseries \textit{Right? }\\\hline
\mdseries Structural &
{\mdseries Opening and closing of topics}

{\mdseries Sequencing topic shifts}

{\mdseries Summarizing options}

\mdseries Continuation of or return to topics &
{\mdseries \textit{Ok, right, well, now, }}

{\mdseries \textit{Anyways, so, then, next}}

{\mdseries \textit{And, so yeah}}

\mdseries \textit{Additionally, and so, and, plus }\\\hline
\mdseries Referential &
{\mdseries Cause / Contrast}

\mdseries Consequence / Digression  &
{\mdseries \textit{Because / But, although}}

\mdseries \textit{\ So, / Anyway}\\\hline
\mdseries Interpersonal &
{\mdseries Mark shared knowledge}

{\mdseries Indicate speaker attitudes }

\mdseries Show emotional response/interest and back channel  &
{\mdseries \textit{You see, you know}}

\mdseries \textit{\ Yes, of course, really, I agree , Great, Sure, ok, yeah }\\\hline
\end{supertabular}
\end{flushleft}
\begin{styleStandard}
In the present study the use of DMs and their relationship to pragmatic competence is also explored, since, as Müller\textit{ }(2005: 1) states “there is a general agreement that discourse markers contribute to the pragmatic meaning of utterances and thus play an important role in the pragmatic competence of the speaker”. Furthermore, Sankoff et al. (1997) note that a learner’s use of DMs may be a good indicator of the effect of L2 exposure on pragmatic competence. The present study thus intends to investigate how DMs may be acquired in EMI contexts.
\end{styleStandard}

\begin{listWWNumxxiileveli}
\item 
\setcounter{listWWNumxxiilevelii}{0}
\begin{listWWNumxxiilevelii}
\item 
\begin{stylelsSectionii}
The acquisition of pragmatic markers across contexts
\end{stylelsSectionii}

\end{listWWNumxxiilevelii}
\item 
\begin{stylelsSectioniii}
Study abroad
\end{stylelsSectioniii}

\end{listWWNumxxiileveli}
\begin{styleStandard}
A sojourn abroad has proven to be a positive learning environment for the development of pragmatic competence (Barron 2003; Schauer 2006). Cultural and linguistic immersion of this kind provides learners with increased opportunities to interact with NSs of the target language. Other benefits are that they are repeatedly exposed to daily routines, and have ample opportunities to practice a wide variety of communicative acts in many different social settings. These factors are believed to contribute to language learning. Research examining DMs show positive results: for example, Liu (2013; 2016) found that for Chinese students living in the United States both the increased exposure and increased socialization had significant positive effects on the frequency and variety of DMs produced. Similarly, Barron (2003) measured the use of pragmatic routines of 30 English-speaking learners of German through a written discourse completion task. She found that the exposure to input in the target language triggered important pragmatic development and more target like use of pragmatic routines. In a similar study on 128 international students who spent a study-abroad period in the United States, Sánchez-Hernández (2016; see also Chapter 10 of this volume) found parallel results, showing that there was a relationship between the degree of acculturation and acquisition of DMs. These studies demonstrate how increased exposure, socialization and acculturation through a study-abroad period have measurable effects on pragmatic development.
\end{styleStandard}

\begin{listWWNumxxiileveli}
\item 
\begin{stylelsSectioniii}
Classroom settings: formal instruction and integrated content and language classes
\end{stylelsSectioniii}

\end{listWWNumxxiileveli}
\begin{styleStandard}
While a classroom instructional setting does not offer the same variety of opportunities for learning or for practicing pragmatic skills as studying abroad can, it does show benefits of its own. Both instructed as well as incidental learning of pragmatics have been documented in previous studies (Nguyen, Pham \& Pham 2012; Bardovi-Harlig 2015). However, few studies report on the effects of explicit instruction on the acquisition of DMs; rather, most studies take language samples from classroom learners and report on their usage of DMs as learned incidentally. In the case of Bu (2013), oral data were gathered from interviews and recordings of English classroom discussions. She found that the Chinese learners in her study varied greatly regarding the types of DMs used when compared to NSs. She concluded that, while learners use many of the same DMs as English NSs, they do not employ them with the same functions as NSs do, and at times learners even give new and different functions to DMs that NSs never do (also found in Müller 2005). A study on Chinese learners of English by Liu (2013) also reported similar findings: Regarding the frequency of use of DMs in interviews, some markers were used significantly differently when compared to NSs. Those markers that learners used more frequently than the NSs were \textit{just}, \textit{sort of/kind of,} \textit{but}, \textit{well} and \textit{then}, compared to \textit{I think}, \textit{yeah/yes} and \textit{ah} which were used less frequently by the learners than by the NSs. The author argues that the difference between learners’ and natives’ use of DMs can be attributed to language transfer. 
\end{styleStandard}

\begin{styleStandard}
Among the studies investigating a larger range of DMs, Neary-Sundquist (2014) studied the relationship between proficiency level and pragmatic marker use. She reported that DM use rose with proficiency level, that lower proficiency learners used DMs much less frequently than NSs did, and that advanced learners reach NS levels for the frequency of use. With respect to the variety of markers used, she found that low-level learners overuse certain expressions while advanced learners make use of a larger variety of DMs. Another study comparing teenage learners in Hong Kong to a corpus of English NSs was conducted by Fung \& Carter (2007). Through the analysis of interactions between students, they found that learners use referential markers at high frequencies, while other categories are used more sparingly, and that NSs use DMs for a much wider variety of functions. The authors argue that the use of DMs by the participants reflects the input they receive through their formal instruction in English courses. Regarding integrated content and language settings, Nikula (2008) studied adolescents’ communication in content courses taught in English. She found this context to offer a wide variety of opportunities for pragmatic learning and practice. From classroom observation she reported students using DMs for a variety of pragmatic functions, such as mitigating or softening their communication acts. This gives evidence that content learning does allow learners to practice pragmatic routines, such as DMs. These studies show that although learners do not receive direct instruction on DM use, they can and do learn to use them implicitly, although more research is needed to know how the learning context specifically affects the acquisition of DMs.
\end{styleStandard}

\begin{listWWNumxxiileveli}
\item 
\begin{stylelsSectioni}
The study
\end{stylelsSectioni}

\end{listWWNumxxiileveli}
\begin{styleStandard}
The goal of the present research is to investigate the effect of the EMI setting on the acquisition and use of DMs, in order to inspire future studies on similar larger populations and to provide empirical evidence as to what kinds of pragmatic outcomes can be expected from the EMI setting. 
\end{styleStandard}

\begin{styleStandard}
The research questions addressed in the present study are as follows:
\end{styleStandard}

\setcounter{listWWNumxleveli}{0}
\begin{listWWNumxleveli}
\item 
\begin{stylelsEnumerated}
Do EMI and non-EMI learners use DMs at similar frequencies and distributions, according to the four functional categories of DMs? 
\end{stylelsEnumerated}

\item 
\begin{stylelsEnumerated}
Are there differences between the frequency and distribution of DMs between EMI and non-EMI learners across tasks, viz. an oral discourse completion task and a conversation task?
\end{stylelsEnumerated}

\end{listWWNumxleveli}
\setcounter{listWWNumxxiileveli}{0}
\begin{listWWNumxxiileveli}
\item 
\begin{stylelsSectioni}
Methodology
\end{stylelsSectioni}


\setcounter{listWWNumxxiilevelii}{0}
\begin{listWWNumxxiilevelii}
\item 
\begin{stylelsSectionii}
Participants
\end{stylelsSectionii}

\end{listWWNumxxiilevelii}
\end{listWWNumxxiileveli}
\begin{styleStandard}
Sixteen second-year Economics undergraduate students from a university in Catalonia, Spain, were recruited to participate in this study. Results from a language background questionnaire revealed that all participants were Spanish/Catalan bilinguals and that these languages were also the languages of their previous education. All participants reported English as a third language. Participants reported having an English certificate at a B2 level according to the common European framework of references for languages. 
\end{styleStandard}

\begin{styleStandard}
Participants were divided into two groups: an immersion group (henceforth, IM group) (\textit{N} = 7, age = 19) and a control group (henceforth CON) (\textit{N} = 9, age = 19). The IM group was enrolled in an International Business degree, which is taught completely through English. Participants in the CON group were enrolled in either Economics or Business Administration at the same university but had only one of their courses taught through English in the second year of study. Each degree program consists of 425 contact hours per academic year. For the IM group, all 425 hours are delivered through the English language, while the CON group had an exposure of 35 contact hours. Data were collected during the participants’ second year of study. Figure 1 illustrates the difference between groups in English contact hours per academic year.
\end{styleStandard}

\begin{styleStandard}

\end{styleStandard}

\begin{styleStandard}
\textbf{\textit{Figure 1}}\textit{. Exposure to EMI }
\end{styleStandard}

\begin{listWWNumxxiileveli}
\item 
\setcounter{listWWNumxxiilevelii}{0}
\begin{listWWNumxxiilevelii}
\item 
\begin{stylelsSectionii}
Data collection instruments
\end{stylelsSectionii}

\end{listWWNumxxiilevelii}
\end{listWWNumxxiileveli}
\begin{styleStandard}
Four instruments were used for data collection: two questionnaires (a language background questionnaire and a proficiency test) and two instruments for the elicitation of oral data (a conversation task, and an oral discourse completion task).
\end{styleStandard}

\begin{listWWNumxxiileveli}
\item 
\begin{stylelsSectioniii}
Language background and proficiency level questionnaires
\end{stylelsSectioniii}

\end{listWWNumxxiileveli}
\begin{styleStandard}
The language background questionnaire was designed to gather information on the participants’ language background and learning history to ensure homogeneity of the groups. The online Cambridge placement test was used to ensure a homogeneous group according to general English proficiency; any participant who did not score over a B2 level was not included in the sample.
\end{styleStandard}

\begin{listWWNumxxiileveli}
\item 
\begin{stylelsSectioniii}
Conversation task
\end{stylelsSectioniii}

\end{listWWNumxxiileveli}
\begin{styleStandard}
In order to gather spoken data through interaction, participants were asked to engage in conversation with another participant. Participants were asked three questions that required them to reflect on and discuss their motivations and attitudes towards English as a lingua franca, as well as towards their EMI courses (see appendix A).
\end{styleStandard}

\begin{listWWNumxxiileveli}
\item 
\begin{stylelsSectioniii}
Oral discourse completion task
\end{stylelsSectioniii}

\end{listWWNumxxiileveli}
\begin{styleStandard}
A ten-item oral discourse completion task was used to elicit DMs (see appendix B). Discourse completion tasks as a research tool are supported by Usó-Juan \& Martínez-Flor (2014), Parvaresh \& Tavakoli (2009), Kasper \& Rose (2002), and Hinkel (1997), and they are particularly valuable for eliciting DMs from L2 speakers (Roever 2009). However, they have been strongly debated in the literature, mainly because participants are often asked to write what they would say in a certain situation and this is considered an inaccurate representation of what they would actually say in real-time communication (Bardovi-Harlig 2015). 
\end{styleStandard}

\begin{styleStandard}
In order to address these concerns, an audio and visual discourse completion task was adopted. A video was created consisting of the researcher looking at the camera and recording the prompts for the ten discourse completion tasks, providing a pause of twenty-five seconds for the participants to respond before continuing to the next item. In this way, each item was orally contextualized and an interlocutor was provided to lower the cognitive load, thus enabling the participants to respond rapidly and as they would in an authentic interaction.
\end{styleStandard}

\begin{listWWNumxxiileveli}
\item 
\setcounter{listWWNumxxiilevelii}{0}
\begin{listWWNumxxiilevelii}
\item 
\begin{stylelsSectionii}
Procedure
\end{stylelsSectionii}

\end{listWWNumxxiilevelii}
\end{listWWNumxxiileveli}
\begin{styleStandard}
Participants completed the web questionnaire and placement test via email before attending the testing session. Recording of data took place in sound-proof booths. \ Each booth had a large window and was equipped with a microphone, headset and computer. The participants could see and hear the researcher outside of the booth and were given a series of instructions to set up Audacity, the program used to record their response. The oral discourse completion task was administered first, by playing the video simultaneously on the participants’ computer screens. Twenty-five seconds were given to respond to each prompt. 
\end{styleStandard}

\begin{styleStandard}
This was followed by the conversation task. For this task, participants were put into pairs in the booths, and recorded themselves. The researcher read each of the three questions out loud, the participants were told to include their opinions, personal experiences and anything else they felt they wanted to express in response to the statements. They had two minutes to discuss each question.
\end{styleStandard}

\begin{listWWNumxxiileveli}
\item 
\setcounter{listWWNumxxiilevelii}{0}
\begin{listWWNumxxiilevelii}
\item 
\begin{stylelsSectionii}
Data analysis
\end{stylelsSectionii}

\end{listWWNumxxiilevelii}
\end{listWWNumxxiileveli}
\begin{styleStandard}
Audio files were transcribed into Codes for the Human Analysis of Transcripts (CHAT) using computerized language analysis (CLAN) software (MacWhinney 2000). The researcher identified and tagged each DM used in both the oral discourse completion task and the conversation task and assigned it a code according to its functional category (cognitive, structural, referential, or interpersonal). The transcriptions were then checked by another researcher. A single researcher coded the transcriptions twice to ensure consistency. A further 25\% of the transcriptions were coded by a second researcher; and when there was a discrepancy, an agreement was reached through discussing the item and together deciding on how it should be coded. After coding, the frequency of use of each type of DM was calculated using CLAN. \ Tables 2 to 5 provide extracts from the data, giving three examples of each function. 
\end{styleStandard}

\begin{styleStandard}
\textbf{\textit{Table 2.}}\textit{ Exemplification from the data,}\textit{ according to function: Cognitive DMs}
\end{styleStandard}

\begin{flushleft}
\tablehead{}
\begin{supertabular}{m{4.91776in}}
\hline
\bfseries Examples from the data: Cognitive Function\\\hline
\setcounter{listWWNumxxxvileveli}{0}
\begin{listWWNumxxxvileveli}
\item 
\textit{Yeah I’ve tried it and it doesn’t fit me very well uhh I mean I would prefer another size or maybe another model that fits me better}\end{listWWNumxxxvileveli}
\setcounter{listWWNumxleveli}{0}
\begin{listWWNumxleveli}
\item 
\textit{Please I’m not really well here ahh could you leave me alone for a minute please? It’s like I’m a bit sick and I don’t feel well I need some loneliness to recover myself please.}\item 
\textit{Umm well ahh I’m not sure it looks nice but I wouldn’t wear it}\end{listWWNumxleveli}
\\\hline
\end{supertabular}
\end{flushleft}
\begin{styleStandard}
The markers were coded by taking into account the main function the DM was performing in the discourse, so that what may appear to be the same marker is, in fact, the marker performing different functions and therefore, would be coded accordingly. For example, the token of \textit{well} marks a cognitive function in example (\textit{iii}) in Table 2, and was so coded because we stipulated that in this context (i.e., utterance initial and occurring between two hesitation markers such as \textit{umm}) that it signals a cognitive function (in this case, hesitation), and perhaps an effort to hold the floor while the speaker searches for a word or formulates their utterance in their mind. Looking at examples (\textit{i}) and (\textit{ii}) in Table 2, \textit{I mean} functions to reformulate the message the speaker is conveying whereas \textit{it’s like }functions to signal an elaboration or exemplification of the previous utterance. 
\end{styleStandard}

\begin{styleStandard}
\textbf{\textit{Table 3}}\textit{. Exemplification from the data, according to function: Structural DMs}
\end{styleStandard}

\begin{flushleft}
\tablehead{}
\begin{supertabular}{m{4.91776in}}
\hline
\bfseries Examples from the data: Structural Function\\\hline
\setcounter{listWWNumxxxvleveli}{0}
\begin{listWWNumxxxvleveli}
\item 
\textit{Yes and, umm, for example you can go to London and then you can go to U. S. and it’s totally different so you can also}\end{listWWNumxxxvleveli}
\setcounter{listWWNumxleveli}{0}
\begin{listWWNumxleveli}
\item 
\textit{Participant 1: Umm ahh and the last ahh I would say that I see myself talking English in well, I hope to be in in United States or or somewhere}\item 
\textit{Participant 2: hmm so I think that aah I I chose to to have lessons in English because I wanted to improve my my level I wanted to to keep practising it}\item 
\textit{stop bothering me you know you’re annoying me and my friends so I would really appreciate that you left right now}\end{listWWNumxleveli}
\\\hline
\end{supertabular}
\end{flushleft}
\begin{styleStandard}
The structural markers were coded in the same manner, i.e. identifying the function of the marker in the discourse. For example, in example (\textit{i}) in Table 3 the structural marker \textit{and then} functions to show temporal sequence (going to one city and afterwards to another) and \ also indicates an implied contrast between the two cities (the way English is spoken contrasts greatly between the two cities). \ In example (\textit{ii}), \textit{and} functions to mark the summary of the speaker’s opinion on the matter being discussed and\textit{ so} functions to mark the beginning of the speaker’s turn as well as a slight shift in the topic, a shift from participant one’s opinion to participant two’s opinion. In example (\textit{iii}), \textit{so }serves to summarize the speakers’ message. 
\end{styleStandard}

\begin{styleStandard}
\textbf{\textit{Table 4.}}\textit{ Exemplification from the data, according to function: Referential DMs}
\end{styleStandard}

\begin{flushleft}
\tablehead{}
\begin{supertabular}{m{4.72056in}}
\hline
\bfseries Examples from the data: Referential Function\\\hline
\setcounter{listWWNumxxxviileveli}{0}
\begin{listWWNumxxxviileveli}
\item 
\textit{I would suggest you to do it on the weekend because we don’t have so much homework from university}\end{listWWNumxxxviileveli}
\setcounter{listWWNumxleveli}{0}
\begin{listWWNumxleveli}
\item 
\textit{I have tried on me but it doesn’t fit}\item 
\textit{It don’t fit me because it’s so small so I have to change}\end{listWWNumxleveli}
\\\hline
\end{supertabular}
\end{flushleft}
\begin{styleStandard}
The referential marker \textit{because} in example (\textit{i}) in Table 4 functions to introduce a reason or cause for suggesting the weekend for the party. Example (\textit{ii}) \textit{but} marks a contrast between the two parts of the utterance, and example (\textit{iii})\textit{ so} marks the causal or consequential relationship between the first part of the utterance and the second. \ 
\end{styleStandard}

\begin{styleStandard}
\textbf{Table 5}. Exemplification from the data according to function: Interpersonal DMs
\end{styleStandard}

\begin{flushleft}
\tablehead{}
\begin{supertabular}{m{4.72056in}}
\hline
\bfseries Examples from the data: Interpersonal Function\\\hline
\setcounter{listWWNumxxxviiileveli}{0}
\begin{listWWNumxxxviiileveli}
\item 
\textit{Well I kind of you know we don’t have that much of a relationship with Laura and things have gone pretty badly lately.}\end{listWWNumxxxviiileveli}
\setcounter{listWWNumxleveli}{0}
\begin{listWWNumxleveli}
\item 
\textit{Yeah absolutely I love it but it’s a little bit small for my size}\item 
\textit{Oh yeah I love it but you know what it is too small}\end{listWWNumxleveli}
\\\hline
\end{supertabular}
\end{flushleft}
\begin{styleStandard}
The use of the interpersonal marker \textit{you know} in example (\textit{i}) in Table 5 functions to align the speaker with their interlocutor and mark the shared knowledge that the speaker and the interlocutor have about the speaker and Laura not having a good relationship. Examples (\textit{ii}) \textit{yeah absolutely} and (\textit{iii}) \textit{oh yeah} are used to express the speaker’s attitudes and emotions towards what is being discussed. \ Below, data from each task is provided. 
\end{styleStandard}

\setcounter{listWWNumiileveli}{0}
\begin{listWWNumiileveli}
\item 
\begin{stylelsLanginfo}
Oral discourse completion task data: 
\end{stylelsLanginfo}

\end{listWWNumiileveli}
\begin{stylelsConversationTranscript}
1\ \  \textbf{\textit{Well}} I{\textquotesingle}m not sure about it \textbf{\textit{you know}} Laura 
\end{stylelsConversationTranscript}

\begin{stylelsConversationTranscript}
2\ \  it’s a very chaotic girl \textbf{\textit{and}} she’s always 
\end{stylelsConversationTranscript}

\begin{stylelsConversationTranscript}
3\ \  making noise maybe it’s not such a good 
\end{stylelsConversationTranscript}

\begin{stylelsConversationTranscript}
4\ \  idea inviting Laura if you feel to do it 
\end{stylelsConversationTranscript}

\begin{stylelsConversationTranscript}
5\ \  ahh go ahead \textbf{\textit{but}} in my opinion she 
\end{stylelsConversationTranscript}

\begin{stylelsConversationTranscript}
6\ \  shouldn’t be invited \textbf{\textit{you know.}}
\end{stylelsConversationTranscript}

\begin{styleStandard}
In (1), the speaker opens discourse and begins with the cognitive marker, \textit{well,} denoting mental processing. The participant begins to share her opinion and uses the interpersonal marker\textit{ you know }to mark and confirm shared knowledge, with her interlocutor. She lets her interlocutor know that she is continuing to add information to the same topic using the structural marker \textit{and}. Then in line 6, she uses the referential marker \textit{but} to show contrast between what the speaker and hearer feel and restates her opinion, she finishes her turn by closing with the cognitive marker \textit{you know} as an attempt to align with her interlocutor as well as to assess the interlocutor’s reception of her message. 
\end{styleStandard}

\begin{listWWNumiileveli}
\item 
\begin{stylelsLanginfo}
Conversation data: 
\end{stylelsLanginfo}

\end{listWWNumiileveli}
\begin{stylelsConversationTranscript}
1 \textbf{OK}, I start umm I’ve been learning English 
\end{stylelsConversationTranscript}

\begin{stylelsConversationTranscript}
2 all my life \textbf{\textit{and}}\textbf{\textit{ I think}} that I I would be 
\end{stylelsConversationTranscript}

\begin{stylelsConversationTranscript}
3 very competent and natural with English 
\end{stylelsConversationTranscript}

\begin{stylelsConversationTranscript}
4\ \  speakers, native ones, \textbf{\textit{but}} \textbf{\textit{I think}} that 
\end{stylelsConversationTranscript}

\begin{stylelsConversationTranscript}
5 I’ve always can improve.
\end{stylelsConversationTranscript}

\begin{styleStandard}
In this example the participant first uses the structural DM \textit{OK} to mark the opening of discourse and begin her turn. Then another structural DM \textit{and }is used in line 2, in order to indicate a continuation of the topic and to add information. It is followed by the interpersonal DM \textit{I think }which gives an indication of the speaker’s personal opinion towards the statement immediately following the marker. Then in line 4, a referential DM is used to contrast the information given after \textit{but} with the statement that precedes it. This is then immediately followed by an interpersonal DM \textit{I think} which expresses the speaker’s attitudes and beliefs towards the following statement.
\end{styleStandard}

\begin{styleStandard}
Below are examples of data from the IM group; P1, P2 etc. stand for Participant 1, 2 etc. \ 
\end{styleStandard}

\begin{listWWNumiileveli}
\item 
\end{listWWNumiileveli}
\begin{stylelsConversationTranscript}
P1:\ \ We are colleagues in the same class.
\end{stylelsConversationTranscript}

\begin{stylelsConversationTranscript}
P2:\ \ Yes.
\end{stylelsConversationTranscript}

\begin{stylelsConversationTranscript}
P1:\ \ \textbf{So} we probably agree.
\end{stylelsConversationTranscript}

\begin{stylelsConversationTranscript}
P2:\ \ Yes the same.
\end{stylelsConversationTranscript}

\begin{stylelsConversationTranscript}
P1:\textbf{\ \ }\textbf{\textit{And}}\textbf{ }how do you feel when you communicate with native English speakers?
\end{stylelsConversationTranscript}

\begin{stylelsConversationTranscript}
P2:\ \ \textbf{\textit{Well}} I don’t feel comfortable. 
\end{stylelsConversationTranscript}

\begin{styleStandard}
In (3), participant 1 uses \textit{so} to summarize opinions with her statement ‘\textit{so,} we probably agree’. Then she uses \textit{and} as a structural marker to signal a shift in the topic, from what the speaker feels to what participant two feels towards what is being discussed. \ 
\end{styleStandard}

\begin{listWWNumiileveli}
\item 
\end{listWWNumiileveli}
\begin{stylelsConversationTranscript}
P3:\ \ Yes, it’s difficult to reach the level of English that native speakers have, but I think that, umm it’s very important in, in your life to, to do so.\textbf{ }\textbf{\textit{So}}. 
\end{stylelsConversationTranscript}

\begin{stylelsConversationTranscript}
P4:\ \ Yeah, \textbf{and well}, ahh, in, ahh, the future, I, I want to go to, for example, Londres (London), to find, to will find a homework, ahh because it’s a nice city. 
\end{stylelsConversationTranscript}

\begin{styleStandard}
In (4), participant 3 uses \textit{so} as a structural DM at the end of her utterance to sum up her opinion and mark the end of her turn, which participant 4 correctly interprets and uptakes with an appropriate response. She goes on to use \textit{and} as a topic shift marker and \textit{well} to mark the introduction of a new topic.
\end{styleStandard}

\setcounter{listWWNumxxiileveli}{0}
\begin{listWWNumxxiileveli}
\item 
\begin{stylelsSectioni}
Results
\end{stylelsSectioni}

\end{listWWNumxxiileveli}
\begin{styleStandard}
This section first provides the results for research question 1 by presenting the findings from the discourse completion task and the conversation task together. Research question 2 is addressed in the second section by analyzing the two tasks separately. Before conducting inferential statistics, statistical assumptions were verified; for all but two of the variables, skewness and kurtosis values were out of normal distribution ranges. In addition, the sample size was small. It was thus decided to use non-parametric tests. Specifically, the Mann-Whitney test was carried out to detect any significant differences between the two groups of participants. Additionally, Cohen’s \textit{d} was calculated to determine effect sizes, using as a standardizer the pooled standard deviations of the two groups. The interpretation of the Cohen’s \textit{d} is as follows: \textit{d} values between 0 and .5 are considered small effect sizes, values between .5 and .8 are considered medium effect sizes, and over .8 are reflections of large effect sizes.
\end{styleStandard}

\begin{listWWNumxxiileveli}
\item 
\setcounter{listWWNumxxiilevelii}{0}
\begin{listWWNumxxiilevelii}
\item 
\begin{stylelsSectionii}
Differences in frequency and variety of DM use according to the four categories across both tasks
\end{stylelsSectionii}

\end{listWWNumxxiilevelii}
\end{listWWNumxxiileveli}
\begin{styleStandard}
To begin descriptive statistics were first calculated. Most notably the results reveal that, despite being given equal amounts of time to complete the tasks, participants in the IM group produced more words (\textit{M} = 847.86, \textit{SD} = 194), than the CON group (\textit{M }= 555.89, \textit{SD} = 166.59), which, according to the Mann-Whitney statistical test, proved to be significant, with a large effect size (\textit{U }= 8, \textit{p }= .013, \textit{d }= 1.614). \ Furthermore, and probably as a consequence of this, the IM group produced more DMs (\textit{M }= 104.43, \textit{SD} = 24.61) compared to those in the CON group (\textit{M} = 72.22, \textit{SD} = 17.27) which also proved to be a significant difference, with a large effect size (\textit{U }= 8, \textit{p }= .013, \textit{d }= 1.515). However, with respect to the ratio of DMs per 100 words, the CON group produced more than the IM group: IM (\textit{M} = 12.4, \textit{SD} = 1.47) versus CON (\textit{M }= 13.24, \textit{SD} = 1.42), although when tested for significance the result was not statistical (\textit{U }= 25, \textit{p }= .491). In order to assess the variety of DMs used, Guiraud’s index was calculated, dividing the number of DM types by the square root of the number of DM tokens; the difference between groups was not statistically significant. Guiraud’s index is a corrected version of the standard type/token ratio (TTR), which is less sensitive to variations in text length (Daller 2010). Table 6 reports descriptive and statistical results on the data from the two tasks together. \ Due to the significant difference in number of words spoken, it was decided to calculate all further tests based on the percentage of DMs produced with respect to the total number of words produced multiplied by one hundred.
\end{styleStandard}

\begin{flushleft}
\tablehead{}
\begin{supertabular}{m{1.0129598in}m{0.90735984in}m{0.99275976in}m{0.7316598in}m{0.81435984in}}
\multicolumn{5}{m{4.7740602in}}{\mdseries \textbf{\textit{Table 6.}}\textit{ Descriptive and statistical data for both tasks IM and CON groups}}\\\hline
\bfseries Group (N) &
\bfseries Words Spoken &
\bfseries DMs Spoken &
\bfseries DMs per 100 words  &
\bfseries Mean DM’ Guiraud Index\\\hline
{\mdseries \textit{IM (7)}}

 &
{\mdseries M = 847.86}

\mdseries SD = 194 &
\mdseries M = 104.43 SD = 24.61 &
\mdseries M = 12.4 SD = 1.47 &
{\mdseries M = 1.73}

\mdseries SD = .13\\\hline
\mdseries \textit{CON (9)} &
\mdseries M = 555.89 SD = 166.59 &
\mdseries M = 72.22 SD = 17.27 &
\mdseries M =13.24 SD = 1.42 &
{\mdseries M = 1.76}

\mdseries SD = .19\\\hline
\mdseries \textit{Mann-Whitney test \ } &
{\mdseries U = 8}

\mdseries p = .013 &
{\mdseries U = 8}

\mdseries p = .013 &
{\mdseries U = 25}

\mdseries p = .491 &
{\mdseries U = 18.5}

\mdseries p = .19\\\hline
\mdseries \textit{Cohen’s d} &
\mdseries d = 1.614 &
\mdseries d = 1.515 &
\mdseries d = - 0.58 &
\mdseries d = -0.20\\\hline
\end{supertabular}
\end{flushleft}
\begin{styleStandard}
In order to respond to research question 1 – \textit{Do EMI and non-EMI learners use DMs at similar frequencies and distributions, according to the four functional categories of DMs? }– Further analyses with respect to the four functional categories were carried out. Table 7 shows the mean ratios of DMs produced per participant according to each category across both tasks as well as the mean percentage of occurrence of each category of DM with respect to the total DMs produced. Regarding this distribution, when both tasks were analyzed together, the IM group produced a higher proportion of cognitive (IM = 12.72\%, CON = 11.85\%), structural (IM = 24.49\%, CON = 18.46\%) and interpersonal markers (IM = 37.62\%, CON = 36.92\%), while the CON group tended to produce a higher rate of referential markers (IM = 23.94\%, CON = 30.77\%). The CON group also produced more DM tokens and types per 100 words, which is reflected in the slightly larger value of the Guiraud Index. This may indicate that the use of DMs was both more frequent and more varied than compared to the IM group.
\end{styleStandard}

\begin{flushleft}
\tablehead{}
\begin{supertabular}{m{1.0372599in}m{0.56435984in}m{0.47615984in}m{0.45385984in}m{0.52335984in}m{0.5170598in}m{0.45185986in}}
\multicolumn{7}{m{4.49636in}}{\textbf{\textit{Table 7.}}\textit{ DMs used according to DM category IM and CON group both tasks }}\\\hline
 &
\multicolumn{4}{m{2.2539597in}}{\bfseries IM Group} &
\multicolumn{2}{m{1.0476599in}}{\bfseries CON Group}\\\hline
\bfseries DM Category  &
\bfseries \textit{Mean} &
\bfseries \textit{SD} &
\bfseries \textit{\% of all DMs} &
\bfseries \textit{Mean} &
\bfseries \textit{SD} &
\bfseries \textit{\% of all DMs}\\\hline
\textit{Cognitive} &
1.58 &
.46 &
12.72 &
1.56 &
.56 &
11.85\\\hline
\textit{Structural} &
3.02 &
.96 &
24.49 &
2.40 &
.67 &
18.46\\\hline
\textit{Referential} &
2.89 &
1.01 &
23.94 &
4.02 &
1.25 &
30.77\\\hline
\textit{Interpersonal} &
4.74 &
1.13 &
37.62 &
5.01 &
1.57 &
36.92\\\hline
\textit{DM frequency (tokens per 100w) \ } &
12.41 &
1.47 &
n/a &
13.24 &
1.42 &
n/a\\\hline
\textit{DM Variety (types per 100w) } &
3.07 &
.74 &
n/a &
3.97 &
.74 &
n/a\\\hline
\end{supertabular}
\end{flushleft}
\begin{styleStandard}
A Mann-Whitney test was carried out in order to detect any significant differences between the groups regarding these values per 100 words. Results show there was a significant difference in the production of referential markers, with a large effect size (\textit{U} = 7, \textit{p} = .010, \textit{d} = 1.097). Specifically, the CON group (\textit{M }= 4.33, \textit{SD }= 1.25) produced more referential DMs than the IM group (\textit{M }= 3.32, \textit{SD }= 1.01). Results for the remaining variables measured were not significant. The probability values for the differences and effect sizes are reported in Table 8. \ 
\end{styleStandard}

\begin{flushleft}
\tablehead{}
\begin{supertabular}{m{1.1601598in}m{0.9698598in}m{0.95945984in}m{0.9247598in}m{0.10045984in}}
\multicolumn{5}{m{4.42966in}}{\textbf{\textit{Table 8. }}\textit{Comparison of IM and CON Groups }}\\\hline
\bfseries Category of DM &
\bfseries \textit{Mann –Whitney Value} &
\bfseries \textit{p Value } &
\bfseries \textit{Cohen’s d } &
\\\hhline{----~}
\textit{Cognitive DMs } &
\textit{U = }31 &
\textit{p =}.958 &
\textit{d = }.039 &
\\\hline
\textit{Structural DMs } &
\textit{U = }20 &
\textit{p = }.223 &
\multicolumn{2}{m{1.1039598in}}{\textit{d = }.749}\\\hline
\textit{Referential DMs } &
\textit{U = 14} &
\textit{p = }.064 &
\multicolumn{2}{m{1.1039598in}}{\textit{d = -1.536}}\\\hline
\textit{Interpersonal DMs } &
\textit{U = }26 &
\textit{p = }.560 &
\multicolumn{2}{m{1.1039598in}}{\textit{d = }.197}\\\hline
\textit{DM frequency (tokens per 100w) \ } &
\textit{U = }25 &
\textit{p =} .491 &
\multicolumn{2}{m{1.1039598in}}{\textit{d = }.581}\\\hline
\textit{DM Variety (types per 100w) } &
\textit{U = }13 &
\textit{p = }.055 &
\multicolumn{2}{m{1.1039598in}}{\textit{d = -}1.211}\\\hline
\textit{Guiraud’s Index} &
\textit{U = 40} &
\textit{p =}.40 &
\multicolumn{2}{m{1.1039598in}}{\textit{d =} -0.195}\\\hline
\end{supertabular}
\end{flushleft}
\begin{styleStandard}
To summarize the results from research question 1, it was found that IM students spoke significantly more, and produced significantly more DMs in their texts, in absolute terms. However, looking at standardized values per 100 words, there were no significant differences detected between the groups. Regarding the distribution of the different categories of DMs, the CON group was found to produce a significantly higher ratio of referential DMs than the IM group.
\end{styleStandard}

\begin{listWWNumxxiileveli}
\item 
\setcounter{listWWNumxxiilevelii}{0}
\begin{listWWNumxxiilevelii}
\item 
\begin{stylelsSectionii}
Differences in frequency and variety of DM use in each task separately
\end{stylelsSectionii}

\end{listWWNumxxiilevelii}
\end{listWWNumxxiileveli}
\begin{styleStandard}
Separate analyses were run for each task in order to address research question 2 - \textit{Are there any differences between groups depending on the task, according to the four categories?-} Regarding the discourse completion task, descriptive statistics were calculated (see Table 9) and a Mann-Whitney test was then carried out to detect statistical significance (see Table 10). As in the previous section, all values discussed here are based on ratios per 100 words, given the significant differences in text length between the two groups. 
\end{styleStandard}

\begin{styleStandard}
The IM group (\textit{M }= 399.57, \textit{SD }= 84.72) produced significantly more words than the CON group, with a large effect size (\textit{M }= 287.33,\textit{ SD }= 100.34) (\textit{U }= 12,\textit{ p }= .039, \textit{d} = 1.209). According to the distribution of DMs, results show tendencies for the IM group to produce a higher rate of structural (IM = 20.00\%, CON = 15.73\%) and interpersonal markers (IM = 41.40\%, CON = 39.76\%) than the CON group, while the CON group appears to produce higher rates of cognitive (IM = 10.88\%, CON = 11.87\%) and referential markers (IM = 24.56\%, CON = 28.78\%) than the IM group. The only significant difference between the groups was detected in the referential marker category, with a large effect size (\textit{U }= 12, \textit{p }= .039,\textit{ d }= 1.34). 
\end{styleStandard}

\begin{styleStandard}
However, despite speaking more, the results show the IM group (\textit{M }= 10.01, \textit{SD }= 1.78) produced significantly fewer DMs per 100 words than the CON group (\textit{M }= 13.19, \textit{SD }= 2.30), with a large effect size (\textit{U }= 7.00, \textit{p }= .010, \textit{d }= 1.546). Furthermore, the CON group (\textit{M }= 3.49, \textit{SD }= 1.48) was found to produce a wider variety of DM types than the IM group (\textit{M }= 2.79, \textit{SD }= .49), although the result was not significant. Variety of types per 100 words is a measure that can be partially affected by text length (for example, longer texts will tend to have more repetitions of the same types). The Guiraud Index, which introduces a partial correction for these effects, is in fact slightly higher in the IM group (\textit{M =} 1.74, \textit{SD }= .20) than the CON group (\textit{M }= 1.54, \textit{SD }= .34), although this difference was not significant either. 
\end{styleStandard}

\begin{styleStandard}
In sum, significant differences were that the IM group spoke more than the CON group and that the CON group produced a higher frequency of DMs per 100 words, as well as a significantly higher proportion of referential DMs than the IM group.
\end{styleStandard}

\begin{flushleft}
\tablehead{}
\begin{supertabular}{m{1.1038599in}m{0.57335985in}m{0.5372598in}m{0.6011598in}m{0.6698598in}m{0.6698598in}m{0.48725984in}}
\multicolumn{7}{m{5.11506in}}{\textbf{\textit{Table 9.}}\textit{ Descriptive statistics for the oral discourse completion task }}\\\hline
\bfseries DM Category  &
\bfseries \textit{Mean} &
\bfseries \textit{SD} &
\bfseries \textit{\% of all DMs} &
\bfseries \textit{Mean} &
\bfseries \textit{SD} &
\bfseries \textit{\% of all DMs}\\\hline
\textit{Cognitive} &
1.06 &
.41 &
10.88 &
1.65 &
1.17 &
11.87\\\hline
\textit{Structural} &
2.04 &
1.00 &
20 &
1.94 &
1.09 &
15.73\\\hline
\textit{Referential} &
2.38 &
1.16 &
24.56 &
3.78 &
.91 &
28.78\\\hline
\textit{Interpersonal} &
4.19 &
1.12 &
41.40 &
5.38 &
2.09 &
39.76\\\hline
\textit{Mean words} &
399.57 &
84.72 &
n/a &
287.33 &
100.34 &
n/a\\\hline
\textit{DM frequency (tokens per 100w) \ } &
10.01 &
1.78 &
n/a &
13.19 &
2.30 &
n/a\\\hline
\textit{DM Variety (types per 100w) } &
2.79 &
.49 &
n/a &
3.49 &
1.48 &
n/a\\\hline
\textit{Guiraud’s Index} &
1.74 &
.20 &
n/a &
1.54 &
.34 &
n/a\\\hline
\end{supertabular}
\end{flushleft}
\begin{flushleft}
\tablehead{}
\begin{supertabular}{m{1.1775599in}m{1.1261599in}m{1.2997599in}m{1.2011598in}}
\multicolumn{4}{m{5.0408597in}}{\textbf{\textit{Table 10.}}\textit{ Comparison of groups discourse completion task }}\\\hline
\bfseries Category of DM &
\bfseries \textit{Mann –Whitney Value} &
\bfseries \textit{p Value } &
\bfseries \textit{Cohen’s d }\\\hline
\textit{Cognitive} &
\textit{U =} 21 &
\textit{p = }.266 &
\textit{d = -0.742}\\\hline
\textit{Structural} &
\textit{U =} 31 &
\textit{p = }.958 &
\textit{d = }.095\\\hline
\textit{Referential} &
\textit{U =} 12 &
\textit{p = }.039 &
\textit{d = }1.34\\\hline
\textit{Interpersonal} &
\textit{U =} 20 &
\textit{p = }.223 &
\textit{d = }.737\\\hline
\textit{Mean Words } &
\textit{U =} 12 &
\textit{p = }.039 &
\textit{d =} 1.209\\\hline
\textit{DM frequency (tokens per 100w) \ } &
\textit{U =} 7 &
\textit{p = }.010 &
\textit{d =} 1.546\\\hline
\textit{DM Variety (types per 100w) } &
\textit{U =} 23 &
\textit{p = }.401 &
\textit{d =-0.722 }\\\hline
\textit{Guiraud’s Index} &
\textit{U =} 19 &
\textit{p = }.186 &
\textit{d = }.752\\\hline
\end{supertabular}
\end{flushleft}
\begin{styleStandard}
Turning to the conversation task, descriptive statistics were calculated first (see Table 11), and secondly the data were analyzed statistically using the Mann-Whitney test (see Table 12). The descriptive statistics show that, during the conversation task, the IM group produced more words (\textit{M }= 448.28, \textit{SD }= 143.40), than the CON group (\textit{M }= 268.56, \textit{SD }= 84.14) a difference that proved to be statistically significant, with a large effect size (\textit{U }= 9, \textit{p }= .017, \textit{d }= 1.529). The IM group also showed a higher frequency of DM production overall (\textit{M }= 14.90, \textit{SD }= 3.12) compared to the CON group (\textit{M }= 13.08, \textit{SD }= 1.54), however, this difference failed to prove significant. The CON group produced a higher variety of DMs (\textit{M }= 4.44, \textit{SD }= 77) compared to the IM group (\textit{M }= 3.35, \textit{SD }= 1.42), and the difference was significant, with a large effect size (\textit{U} = 10, \textit{p} = .023, \textit{d} = .954). 
\end{styleStandard}

\begin{styleStandard}
Concerning the categories of DMs, a statistically significant difference was detected in the use of structural DMs, with a large effect size (\textit{U }= 13, \textit{p} = .050 \textit{d} = 1.100). Specifically, the IM group (\textit{M} = 4.11, \textit{SD }= 1.42) produced more structural DMs than the CON group (\textit{M }= 2.75, \textit{SD }= 1.02). Furthermore, the IM group tended to produce more cognitive and interpersonal DMs, and the CON group more referential DMs, although none of these differences were significant. 
\end{styleStandard}

\begin{styleStandard}
To summarize results from the conversation task, it was found that the IM group produced significantly more words and more structural DMs than the CON group and that the CON group produced a significantly higher variety of DMs than the IM group. 
\end{styleStandard}

\begin{flushleft}
\tablehead{}
\begin{supertabular}{m{1.1108599in}m{0.5379598in}m{0.5344598in}m{0.46225986in}m{0.6025598in}m{0.5038598in}m{0.5441598in}}
\multicolumn{7}{m{4.7685595in}}{\textbf{\textit{Table 11. }}\textit{Descriptive statistics for the conversation task }}\\\hline
 &
\multicolumn{3}{m{1.6921599in}}{\bfseries IM Group} &
\multicolumn{3}{m{1.8080599in}}{\bfseries CON Group }\\\hline
\bfseries DM Category &
\bfseries Mean &
\bfseries SD &
\bfseries \% of all DMs &
\bfseries Mean &
\bfseries SD &
\bfseries \% of all DMs\\\hline
\textit{Cognitive} &
1.95 &
.88 &
13.90 &
1.49 &
1.24 &
11.82\\\hline
\textit{Structural} &
4.11 &
1.42 &
27.35 &
2.75 &
1.02 &
21.41\\\hline
\textit{Referential} &
3.31 &
1.01 &
23.56 &
4.33 &
1.25 &
32.91\\\hline
\textit{Interpersonal} &
5.52 &
2.53 &
35.20 &
4.51 &
1.58 &
33.87\\\hline
\textit{Mean words} &
448.28 &
143.40 &
n/a &
268.56 &
84.14 &
n/a\\\hline
\textit{DM frequency (tokens per 100w) \ } &
14.90 &
3.12 &
n/a &
13.08 &
1.54 &
n/a\\\hline
\textit{DM Variety (types per 100w) } &
3.35 &
1.42 &
n/a &
4.44 &
.77 &
n/a\\\hline
\textit{Guiraud’s Index} &
1.72 &
.30 &
 &
1.98 &
.37 &
\\\hline
\end{supertabular}
\end{flushleft}
\begin{flushleft}
\tablehead{}
\begin{supertabular}{m{1.3462598in}m{1.0559598in}m{1.0045599in}m{1.1018599in}}
\multicolumn{4}{m{4.74486in}}{\textbf{\textit{Table 12.}}\textit{ Comparison of Groups Conversation Task }\textit{(ratios per 100 words)}}\\\hline
\bfseries Category of DM &
\bfseries \textit{Mann –Whitney Value} &
\bfseries \textit{p Value } &
\bfseries \textit{Cohen’s d }\\\hline
\textit{Cognitive} &
\textit{U =} 24 &
\textit{p }= .427 &
\textit{d =} .428\\\hline
\textit{Structural} &
\textit{U =} 13 &
\textit{p }= .050 &
\textit{d = }1.100\\\hline
\textit{Referential} &
\textit{U }= 14 &
\textit{p }= .064 &
\textit{d = -0}.898\\\hline
\textit{Interpersonal} &
\textit{U }= 24 &
\textit{p }=.427 &
\textit{d = }.479\\\hline
\textit{Mean Words } &
\textit{U }= 11 &
\textit{p }=.034 &
\textit{d = }1.529\\\hline
\textit{DM frequency (tokens per 100w) \ } &
\textit{U }= 21 &
\textit{p }=.226 &
\textit{d = }.740\\\hline
\textit{DM Variety (types per 100w) } &
\textit{U }= 10 &
\textit{p }=.023 &
\textit{d = }.954\\\hline
\textit{Guiraud’s Index} &
\textit{U =16} &
\textit{p =.10} &
\textit{d =}.772\\\hline
\end{supertabular}
\end{flushleft}
\begin{listWWNumxxiileveli}
\item 
\begin{stylelsSectioni}
Discussion
\end{stylelsSectioni}

\end{listWWNumxxiileveli}
\begin{styleStandard}
This study did not find many statistically significant differences between the two groups, both because of the limited sample size and also because the two groups were rather similar with respect to several dimensions. However, the significant differences that were found offer some interesting points for discussion.
\end{styleStandard}

\begin{styleStandard}
Regarding the first research question, findings show that students in an EMI program produced longer responses and dialogues. When calculating absolute scores, EMI students produced significantly longer stretches of speech and a significantly higher number of DMs. Both of these findings can be considered signs of increased oral fluency (Segalowitz \& Freed 2004). However, when text length was controlled for via calculation of standardized values per 100 words, the differences were not sustained. While this points out that the two groups produce similar frequencies of DMs in proportion to the total length of text produced, it still evidences an increased oral fluency among the EMI students.
\end{styleStandard}

\begin{styleStandard}
The second finding was that the non-EMI group had a very high frequency of use of referential markers. Previous research has suggested that this category might be easier or could be the first category of DMs to be acquired (Liu 2016). This is due to the main functions of referential DMs, namely to show cause and contrast, consequence and comparison. These markers are the type of DM most often addressed in the foreign language classroom due to their close relationship with syntax \ as well as their strong prevalence in written language (Fung \& Carter 2007). This contrasts with the other categories which appear more frequently or even exclusively as oral markers and have fewer text-dependent functions (Anderson 2001). While the EMI students did integrate referential markers into their speech, they did not use them quite as frequently as the non-EMI group did; on the contrary, they had a slightly more even distribution of use of DMs over the four categories, which may be an indication of the EMI group employing a more appropriate distribution of DMs across functions. It might be the case that EMI students were able to select other more appropriate markers while the CON group seemed to rely more on referential markers. These findings echo those reported in Fung \& Carter (2007), who found that L2 learners relied on referential DMs more than on the other DM categories. 
\end{styleStandard}

\begin{styleStandard}
Turning to the interpretation of the results in terms of \ the second research question, it was found that the non-EMI group produced a higher ratio of DMs to words during the discourse completion task. A possible explanation for this result may be that, due to the strict time limit during the discourse completion task, there may have been some cognitive competition as described by Skehan (1998), where some features are attended to at the expense of others. For example, in this case, providing a response within the time given may have been a difficult task for the non-EMI participants and, as a consequence, little attention might have been paid to how the message was delivered; in other words, they may have been more likely to repeat the same markers and utilize the same sentence structures to organize their discourse and convey their ideas to their interlocutors. This may be due to being unsure of how to continue a natural flow of conversation during the task. This interpretation would account for the difference in production of DMs between the groups. This effect of cognitive competition could be more prevalent in the non-EMI group, as they may speak English less often and might be less used to spontaneously using English, whereas the EMI students are accustomed to using English daily and thus might able to use DMs slightly more selectively.
\end{styleStandard}

\begin{styleStandard}
Additionally, the EMI participants were found to produce significantly longer texts, which as mentioned above can be interpreted as a sign of fluency, since they were able to produce longer responses than the non-EMI group was in the same amount of time. We suggest this could be due to the constant and frequent exposure to EMI classes. However, in future research one might compare the results to NS data to confirm if the ratio of DMs produced by the groups is similar or different from NS usage. 
\end{styleStandard}

\begin{styleStandard}
Regarding the significant difference between the use of referential DMs as measured on the discourse completion task, this trend was also found when analyzing the two tasks together and has been discussed above. It seems that the referential category is more closely linked to grammar and what is taught in L2 classrooms. The functions of referential DMs appear to be the most transparent in their meaning and use, and thus, may be slightly easier to incorporate into the one’s speech than the other DM categories (Liu 2016).
\end{styleStandard}

\begin{styleStandard}
Turning to the conversation task, in addition to producing significantly longer responses, which has already been discussed, EMI students were found to use significantly more structural markers than the non-EMI group during the conversation task. This could be a reflection of a slightly higher or more sensitive pragmatic competence in their ability to signpost discourse while engaged in conversation, as was found in Wei (2011), whose advanced learners were reported to use more structural markers to highlight information. Furthermore, the use of structural markers could be a sign of increased linguistic complexity. This finding aligns with those from Neary-Sunquist (2014), who reported that higher proficiency learners used DMs to support and enable their fluency, and that as proficiency increases, learners can allocate more attention not only on delivering their message but on how they would like their message to be received. However, in the present study, our participants had the same proficiency and they only differed in terms of amount of exposure to the target language. This leads us to suggest that the number of hours of exposure available through immersion programs (as was the case for the IM group in this study), may provide learners with more opportunities for communication and thus make them more aware of how they express themselves while speaking English.
\end{styleStandard}

\begin{styleStandard}
It was also found that the non-EMI group produced a wider variety of DMs overall compared to the EMI group during the conversation task. It seems that text length could be playing a strong role here. The EMI participants produced significantly longer responses on all tasks, and it is therefore much more likely that in a long text the same markers are used more than once. This interpretation is further supported by the non-significant findings of differences in variety found according to Guiraud’s Index. When text length was controlled for, the significant difference between the groups was not sustained.
\end{styleStandard}

\begin{styleStandard}
As mentioned in the literature review, the type of input in EMI is mainly via lectures (Hellekjaer \& Hellekjaer 2015), a context where academic language with a formal tone is primarily used. Lecturers must cue their interlocutors as to when they are opening a topic, changing, returning to, or continuing a topic, as well as mark progression while explaining processes. These functions are carried out by structural markers (Andersen 2001), thus, making them one of the most salient categories of DMs that EMI students are exposed to. This may be why EMI students integrate more structural markers into their speech than the non-EMI students. 
\end{styleStandard}

\begin{styleStandard}
Despite the explanations provided as possible reasons for the differences according to task, the results do not seem to point towards a clear relationship between task and DM use, as was also found by Neary-Sunquist (2013). This clearly points to the need for more research in this area.
\end{styleStandard}

\begin{listWWNumxxiileveli}
\item 
\begin{stylelsSectioni}
Conclusion
\end{stylelsSectioni}

\end{listWWNumxxiileveli}
\begin{styleStandard}
This preliminary study seems to provide evidence that the context of learning can make some difference in the learning of pragmatics. EMI students were found to produce significantly longer responses than the non-EMI group, including more words and more DMs in absolute terms, which is a sign of increased oral fluency. Furthermore, EMI students produced more structural DMs, which showed an effort on the behalf of these participants to produce more complex language and to signpost discourse clearly. The EMI students also had a more even distribution of use of DMs across categories. This could be a reflection of development in pragmatic competence: It seems as though the increased amount of time spent in EMI classrooms may lead learners to attend more to how they want their messages to be interpreted by their interlocutors. This pattern of use also reflects the type of input they receive, namely, academic lectures. Non-EMI students, on the other hand, produced more referential DMs, which seems to be the first category learned due to their transparent meanings, attention given to them in language classrooms as well as their prevalence in writing and formal speech (Fung \& Carter 2007; Neary-Sundquist 2014, Liu 2016). 
\end{styleStandard}

\begin{styleStandard}
This study aimed to shed some light on the incidental acquisition of DM in the EMI classroom and we have identified some trends. However, the study was conducted on a small number of participants and the findings should be taken as preliminary. It is, therefore, important to carry out more studies in this context with more participants to confirm the trends found here. 
\end{styleStandard}

\begin{stylelsSectioni}
Acknowledgments
\end{stylelsSectioni}


\begin{styleStandard}
The authors extend their gratitude to a number of researchers who offered their valuable insights comments and support during this project, Eloi Puig Mayenco, Andrew Lee, and Dr. Roy Lyster, as well as to the anonymous reviewers and editors of this monograph for their suggestions.
\end{styleStandard}

\begin{stylelsSectioni}
Funding
\end{stylelsSectioni}


\begin{styleStandard}
This study was supported by the Spanish Ministry of Science and Innovation (grant FFI2013-48640-C2-1-P). 
\end{styleStandard}

\begin{stylelsSectioni}
References
\end{stylelsSectioni}


\begin{styleStandard}
Aijmer, Karin. 1996. Swedish modal particles in a contrastive perspective. \textit{Language Sciences~}18\textit{ }(1). 393-427.
\end{styleStandard}


\begin{styleStandard}
Airey, John. 2012. I don’t teach language.\textstyleappleconvertedspace{~}The linguistic attitudes of physics lecturers in Sweden.\textit{ AILA Review}\textstyleappleconvertedspace{~}25. 64-79.
\end{styleStandard}


\begin{styleStandard}
Andersen, Gisle. 2001.\textstyleappleconvertedspace{~}\textit{Pragmatic markers and sociolinguistic variation: A relevance-theoretic approach to the language of adolescents}\textstyleappleconvertedspace{~}84. Amsterdam: Benjamins.
\end{styleStandard}


\begin{styleStandard}
Ariel, Mira. 1998. Discourse markers and form-function correlations.\textstyleappleconvertedspace{~In Jucker, Andreas \& Ziv, Yael (eds.). Discourse markers: descriptions and theory, 226-260. Amsterdam: Benjamins.}
\end{styleStandard}


\begin{styleStandard}
Bardovi-Harlig, Kathleen. 2015. Operationalizing conversation in studies of instructional effect in L2 pragmatics.\textstyleappleconvertedspace{~}\textit{System}\textstyleappleconvertedspace{ }48. 21-34.
\end{styleStandard}


\begin{styleStandard}
Barron, Anne. 2003. \ \textit{Acquisition in interlanguage pragmatics: Learning how to do things with words in a study abroad context}. Vol. 108. Amsterdam: Benjamins. 
\end{styleStandard}


\begin{styleStandard}
Blakemore, Diane. 1992.\textstyleappleconvertedspace{~}\textit{Understanding utterances: An introduction to pragmatics}. Oxford: Blackwell.
\end{styleStandard}


\begin{styleStandard}
Bu, Jiemin. 2013. A study of the acquisition of discourse markers by Chinese learners of English. \ \textit{International Journal of English Studies} 13 (1). 29-50. 
\end{styleStandard}


\begin{styleStandard}
Carter, Ronald \& McCarthy, Michael. 2006.\textstyleappleconvertedspace{~}\textit{Cambridge grammar of English: a comprehensive guide; spoken and written English grammar and usage}. Cambridge. Cambridge University Press. 
\end{styleStandard}


\begin{styleStandard}
Coleman, Jim. 2006. English-medium teaching in European higher education.\textstyleappleconvertedspace{~}\textit{Language teaching},\textstyleappleconvertedspace{~}39 (1). 1-14.
\end{styleStandard}


\begin{styleStandard}
Costa, Francesca. 2012. Focus on form in ICLHE lectures in Italy: Evidence from English-medium science lectures by native speakers of Italian. \textit{AILA Review} 25. 30-47. 
\end{styleStandard}


\begin{styleStandard}
Daller, Michael. 2010 Guiraud’s index of lexical richness. In: \textit{British Association of Applied Linguistics,} September 2010. Available from: http://eprints.uwe.ac.uk/11902
\end{styleStandard}


\begin{styleStandard}
Doiz, Aintzane, Lasagabaster, David. \& Sierra, J, M. 2014 Language friction and multilingual policies in higher education: the stakeholders{\textquotesingle} view, \textit{Journal of Multilingual and Multicultural Development }35 (4). 345-360.
\end{styleStandard}


\begin{styleStandard}
European Minister of Education. \ 1999. \textit{The declaration of Bologna}. The European minister of education June 19, 1999. 
\end{styleStandard}


\begin{styleStandard}
Fischer, Kerstin. (ed.) 2006.\textstyleappleconvertedspace{~}\textit{Approaches to discourse particles}.\textstyleappleconvertedspace{~}Amsterdam: Elsevier.
\end{styleStandard}


\begin{styleStandard}
Fischer, Kerstin. 2014. Discourse markers. In Schneider, Klaus, \& Barron, Anne (eds.) \textit{Pragmatics of discourse}, 271-294. Berlin: De Gruyter Mouton. 
\end{styleStandard}


\begin{styleStandard}
Fung, Loretta \& Carter, Ronald. 2007. Discourse markers and spoken English: Native and learner use in pedagogic settings.\textstyleappleconvertedspace{~}\textit{Applied linguistics},\textstyleappleconvertedspace{~}28 (3). 410-439.
\end{styleStandard}


\begin{styleStandard}
Gundermann, Susanne. 2014. \textit{EMI: Modelling the role of the Native speaker in a Lingua Franca Context.} Universität Freiburg. (Doctoral dissertation.) 
\end{styleStandard}


\begin{styleStandard}
Hellekjaer, Glen Ole. \& Hellekjaer, Anne-Inger. 2015. From tool to target language: Arguing the need to enhance language learning in English-medium instruction courses and programs. In Dimova, Slobodanka, Hultgren, Anna Kristina, \& Jensen, Christian (eds.), \textit{English-Medium Instruction in European Higher Education. }317-324. Berlin: Walter de Gruyter. 
\end{styleStandard}


\begin{styleStandard}
Hinkel, Eli. 1997. Appropriateness of advice: DCT and multiple choice data.\textstyleappleconvertedspace{~}\textit{Applied linguistics} 18 (1). 1-26.
\end{styleStandard}


\begin{styleStandard}
Kasper, Gabriele. \& Rose, Kenneth. 2002. \textit{Pragmatic }\textit{d}\textit{evelopment in a }\textit{s}\textit{econd }\textit{l}\textit{anguage}.\textstyleappleconvertedspace{~} Malden: Blackwell Publishing. \ 
\end{styleStandard}


\begin{styleStandard}
Kling Soren, Joyce. 2013. \textit{Teacher identity in English-Medium Instruction: Teacher cognitions from a Danish tertiary education context}. University of Copenhagen. (Doctoral dissertation.)
\end{styleStandard}


\begin{styleStandard}
Kuteeva, Maria \& Airey, John. 2014. Disciplinary differences in the use of English in higher education: reflections on recent language policy developments.\textstyleappleconvertedspace{~}\textit{Higher Education} 67 (5). 533-549.
\end{styleStandard}


\begin{styleStandard}
Leech, Geoffrey \& Svartvik, Jan. 1975.\textstyleappleconvertedspace{~}\textit{A communicative grammar of English}. London: Longman. 
\end{styleStandard}


\begin{styleStandard}
Liu, Binmei. 2013. Effect of first language on the use of English discourse markers by L1 Chinese speakers of English. \textit{Journal of Pragmatics }45. 149-172. 
\end{styleStandard}


\begin{styleStandard}
Liu, Binmei. 2016. Effect of L2 exposure: From a perspective of discourse markers. \textit{Applied Linguistics Review }7 (1). 73-98. 
\end{styleStandard}


\begin{styleStandard}
Llurda, Enric, Cots, Josep. M. \& Armengol, Lurdes. 2013. Expanding language borders in a bilingual institution aiming at trilingualism. In\textstyleappleconvertedspace{~Haberland, Hartmut., Lønsmann, Dorte, \& Preisler, Bent (eds), }\textit{Language alternation, language choice and language encounter in international tertiary education}, 203-222. Amsterdam: Springer Science \& Business Media
\end{styleStandard}


\begin{styleStandard}
Lueg, Klarissa \& Lueg, Ranier. 2015. Why do students choose English as a medium of instruction? A Bourdieusian perspective on the study strategies of non-native English speakers.\textstyleappleconvertedspace{~}\textit{Academy of Management Learning \& Education }14\textit{ }(1), 5-30.
\end{styleStandard}


\begin{styleStandard}
MacWhinney, Brian. 2000. \textit{The CHILDES project: Tools for }\textit{a}\textit{nalysing talk}. 3\textsuperscript{rd} Edition. Mahwah, NJ: Lawrence Erlbaum. 
\end{styleStandard}


\begin{styleStandard}
Margić, Branka. D., \& Žeželić, Tea. 2015. The implementation of English-medium instruction in Croatian higher education: Attitudes, expectations and concerns.\textstyleappleconvertedspace{~In Plo Alanstrué, Ramón \& Pérez-Llantada, Carmen. (eds), }\textit{English as a scientific and research language}, 311-332. Berlin: De Gruyter. 
\end{styleStandard}


\begin{styleStandard}
Maschler, Yael. 1994. Metalanguaging and discourse markers in bilingual conversation.\textstyleappleconvertedspace{~}\textit{Language in Society}\textstyleappleconvertedspace{~}23 (3). 325-366.
\end{styleStandard}


\begin{styleStandard}
Müller, Simone. 2005.\textstyleappleconvertedspace{~}\textit{Discourse markers in native and non-native English discourse}. Amsterdam: Benjamins.
\end{styleStandard}


\begin{styleStandard}
Neary-Sundquist, Colleen. 2013 Task type effects on pragmatic marker use by learners at varying proficiency levels. \textit{L2 Journal }5 (2). 1-21.
\end{styleStandard}


\begin{styleStandard}
Neary-Sundquist, Colleen. 2014 The use of pragmatic markers across proficiency levels in second language speech. \textit{Studies in Second Language Learning and Teaching }4\textit{ }(4). 637-663. 
\end{styleStandard}


\begin{styleStandard}
Neary-Sundquist, Colleen. 2013 Task type effects on pragmatic marker use by learners at varying proficiency levels. \textit{L2 Journal }5 (2). 1-21.
\end{styleStandard}


\begin{styleStandard}
Nguyen, Thi Thuy Minh., Pham, Thi Pham Hanh., \& Pham, Minh Tam. \ 2012. The relative effects of explicit and implicit form-focused instruction on the development of L2 pragmatic competence. \textit{Journal of Pragmatics }44.\textit{ }416-434. 
\end{styleStandard}


\begin{styleStandard}
Nikula, Tarja. 2008. Learning pragmatics in content-based classrooms. In Alcón Soler, Eva, \& Martínez Flor, Alicia. (eds.), \textit{Investigating pragmatics in foreign language learning, teaching and testing. }94-113. Clevedon, UK: Multilingual Matters. 
\end{styleStandard}


\begin{styleStandard}
Nordic Council of Ministers. 2006. \textit{Deklaration om nordisk språkpolitik} [Declaration on a Nordic language policy]. Copenhagen: Nordic Council of Ministers.
\end{styleStandard}


\begin{styleStandard}
Parvaresh, Vahid \& Tavakoli, Mansoor. 2009. Discourse completion tasks as elicitation tools: How convergent are they.\textstyleappleconvertedspace{~}\textit{The Social Sciences},\textstyleappleconvertedspace{~}4 (4). 366-373.
\end{styleStandard}


\begin{styleStandard}
Pecorari, Diane, Shaw, Philip, Irvine, Aileen, \& Malmström, Hans. 2011. English for academic purposes at Swedish universities: Teachers’ objectives and practices. \textit{Iberica }22. 55-78. 
\end{styleStandard}


\begin{styleStandard}
Pérez-Cañado, Maria Luisa. 2012. CLIL research in Europe: past, present, and future. \textit{International Journal of Bilingual Education and Bilingualism }15 (3). 315-341.
\end{styleStandard}


\begin{styleStandard}
Pérez-Vidal, Carmen. 2008. Política lingüística universitària catalana dins l’EEES a la Universitat Pompeu Fabra: El Pla d’Acció pel Multilingüisme. In Martí i Castell, Joan \& Mestres i Serra, Josep Maria. (eds.) \textit{El multilingüisme a les universitats en l’espai europeu d’educació superior} (Actes del Seminari del CUIMPB-CEL 2007), \ 115-141. Barcelona: Institut d{\textquotesingle}Estudis Catalans
\end{styleStandard}


\begin{styleStandard}
Pérez Vidal, Carmen. (ed.). 2014.\textstyleappleconvertedspace{~}\textit{Language acquisition in study abroad and formal instruction contexts.}\textstyleappleconvertedspace{~} Amsterdam: Benjamins.
\end{styleStandard}


\begin{styleStandard}
Pérez Vidal, Carmen. 2015. Languages for all in education: CLIL and ICLHE at the crossroads of multilingualism, mobility and internationalisation. In\textstyleappleconvertedspace{~Juan-Garau, Maria \& Salazar-Noguera, Joana, }\textit{Content-based language learning in multilingual educational environments,}\textstyleappleconvertedspace{~}31-50. Amsterdam: Springer International Publishing.
\end{styleStandard}


\begin{styleStandard}
Pérez Vidal, Carmen. (ed.). 2014.\textstyleappleconvertedspace{~}\textit{Language acquisition in study abroad and formal instruction contexts}\textstyleappleconvertedspace{~}(Vol. 13). John Benjamins.
\end{styleStandard}


\begin{styleStandard}
Pérez-Vidal, Carmen. 2008. Política lingüística universitària catalana dins l’EEES a la Universitat Pompeu Fabra: El Pla d’Acció pel Multilingüisme. In Martí i Castell, Joan \& Mestres i Serra, Josep Maria. (eds.) El multilingüisme a les universitats en l’espai europeu d’educació superior (Actes del Seminari del CUIMPB-CEL 2007) \ 115-141. 
\end{styleStandard}


\begin{styleStandard}
Roever, Carsten. 2009. Teaching and testing pragmatics.\textstyleappleconvertedspace{~In Long, Michael, \& Doughty, Catherine., J. (eds.), }\textit{The handbook of language teaching}, 560-577. Oxford U.K. \ John Wiley \& sons. 
\end{styleStandard}


\begin{styleStandard}
Sánchez-Hernández, Adriana. 2016. Pragmatic routines during study abroad programs: The impact of acculturation and intensity of interaction. (Paper presented at\textstyleappleconvertedspace{~the International conference on intercultural pragmatics and communication, Split. 10-12 June 2016). }
\end{styleStandard}


\begin{styleStandard}
\textstyleappleconvertedspace{Sankoff, Gillian., Thibault, Pierrette., Nagy, Naomi., Blondeau, Hélène., Fonollosa, Marie-Odile., \& Gagnon, Lucie. }\textstyleappleconvertedspace{1997. Variation in the use of discourse markers in a language contact situation. }\textstyleappleconvertedspace{\textit{Language Variation and Change }}\textstyleappleconvertedspace{9.}\textstyleappleconvertedspace{\textit{ }}\textstyleappleconvertedspace{191-217. }
\end{styleStandard}


\begin{styleStandard}
Schauer, Gila. A. 2006. Pragmatic awareness in ESL and EFL contexts: Contrast and development.\textstyleappleconvertedspace{~}\textit{Language Learning} 56 (2). 269-318.
\end{styleStandard}


\begin{styleStandard}
Schiffrin, Deborah. 1987. \textit{Discourse markers.} Cambridge: Cambridge University Press. 
\end{styleStandard}


\begin{styleStandard}
Segalowitz, Norman, \& Freed, Barbara. 2004 Context, contact, and cognition in oral fluency acquisition: Learning Spanish in at home and study abroad contexts.~\textit{Studies in second language acquisition}~26 (2) 173-199.
\end{styleStandard}


\begin{styleStandard}
Sert, Nehir. 2008. The language of instruction dilemma in the Turkish context. \ \textit{System }36. 156-171.
\end{styleStandard}


\begin{styleStandard}
Smit, Ute \& Dafouz, Emma. 2012. Integrating content and language in higher education: An introduction to English-medium policies, conceptual issues and research practices across Europe.\textstyleappleconvertedspace{~}\textit{AILA Review}\textstyleappleconvertedspace{~}25\textit{ }(1). 1-12..
\end{styleStandard}


\begin{styleStandard}
Skehan, Peter. 1998. \textit{A cognitive approach to language learning}. Oxford: Oxford University Press.
\end{styleStandard}


\begin{styleStandard}
Taguchi, Naoko. 2015. \ “Contextually” speaking: A survey of pragmatic learning abroad, in class, and online. \textit{System }48. 3-20 
\end{styleStandard}


\begin{styleStandard}
Tazl, Dietmar. 2011. English medium Masters’ programmes at an Austrian university of applied sciences: attitudes, experiences and challenges. \textit{Journal of English for Academic purposes }10 (4). 252-270. 
\end{styleStandard}


\begin{styleStandard}
Unterberger, Barbara. J. 2012 English medium programmes at Austrian business faculties. A status quo survey on national trends and a case study on programme design and delivery. \textit{AILA Review} 25, 80-100.
\end{styleStandard}


\begin{styleStandard}
Usó-Juan, E. \& Martínez-Flor, Alicia. 2014. Reorienting the assessment of the conventional expressions of complaining and apologising: From single-response to interactive DCTs.\textstyleappleconvertedspace{~}\textit{Iranian Journal of Language Testing} 4 (1). 113-136.
\end{styleStandard}


\begin{styleStandard}
Vanda, Koczogh Helga. 2007. Native speaker and non-native speaker discourse marker use.~\textit{Argumentum}~3. 46-53.
\end{styleStandard}


\begin{styleStandard}
Vellenga, Heidi. 2004. Learning pragmatics from ESL \& EFL textbooks: How likely?.\textstyleappleconvertedspace{~}\textit{Tesl-Ej}\textstyleappleconvertedspace{~}8 (2). 1-18. 
\end{styleStandard}


\begin{styleStandard}
Wächter, Bernd \& Maiworm, Friedhelm. 2008. \textit{English-taught programmes in European higher education:}\textstyleappleconvertedspace{\textit{~}}\textit{ACA Papers on International Cooperation in Education. }Bonn: Lemmens.
\end{styleStandard}


\begin{styleStandard}
Wächter, Bernd. \& Maiworm, Friedhelm. (eds). 2014.\textstyleappleconvertedspace{~}\textit{English-taught programmes in European higher education: The state of play in 2014}. Bonn: Lemmens.
\end{styleStandard}


\begin{styleStandard}
Wächter, Bernd \& Maiworm, Friedhelm. 2008. English-taught programmes in European higher education.\textstyleappleconvertedspace{~}\textit{ACA Papers on International Cooperation in Education. }Bonn: Lemmens.
\end{styleStandard}


\begin{styleStandard}
Wei, Ming. 2011. A comparative study of the oral proficiency of Chinese learners of English across task functions: A discourse marker perspective. \textit{Foreign Language Annals} 44\textit{ }(4). 674-691.
\end{styleStandard}


\begin{styleStandard}
Yates, Lynda. 2011. Interaction, language learning and social inclusion in early settlement. \textit{International Journal of Bilingual Education and Bilingualism} 14 (4). 457-471.
\end{styleStandard}


\begin{stylelsSectioni}
Appendix A: Conversation task
\end{stylelsSectioni}


\setcounter{listWWNumxlileveli}{0}
\begin{listWWNumxlileveli}
\item 
\begin{stylelsEnumerated}
Do you imagine yourself being a completely competent and natural speaker of English in the future? How do you feel when communicating with native speakers of English? What place do you see English having in your future?
\end{stylelsEnumerated}

\end{listWWNumxlileveli}
\setcounter{listWWNumxleveli}{0}
\begin{listWWNumxleveli}
\item 
\begin{stylelsEnumerated}
Why do you believe courses are taught in English in your University? Why did you enroll in a degree program that is taught in English? How do you feel about being taught in English by non-native speakers of English? 
\end{stylelsEnumerated}

\item 
\begin{stylelsEnumerated}
Do you enjoy communicating in English with other Non-Native English speakers? Can you share any of your experiences using English as an international language? 
\end{stylelsEnumerated}

\end{listWWNumxleveli}
\begin{stylelsSectioni}
Appendix B: Oral discourse completion task
\end{stylelsSectioni}


\setcounter{listWWNumxliiileveli}{0}
\begin{listWWNumxliiileveli}
\item 
\begin{stylelsEnumerated}
\textbf{Contextualization}: Your best friend is inviting you to her birthday party. You will definitely be able to make it whenever she suggests because she is such a good friend. (Suggestion non-face-threatening)
\end{stylelsEnumerated}

\end{listWWNumxliiileveli}
\setcounter{listWWNumxleveli}{0}
\begin{listWWNumxleveli}
\item 
\setcounter{listWWNumxlevelii}{0}
\begin{listWWNumxlevelii}
\item 
\begin{stylelsEnumerated}
\textbf{Researcher on video speaking directly to participant}: Hi, so I have just about everything for the party planned, which day do you think I should have it? 
\end{stylelsEnumerated}

\end{listWWNumxlevelii}
\item 
\begin{stylelsEnumerated}
\textbf{Contextualization}: Your friend wants to invite Laura to the birthday party, a girl that your friend knows you don’t get along with. Try to convince your friend to not invite Laura. \ (Suggestion face-threatening) 
\end{stylelsEnumerated}


\setcounter{listWWNumxlevelii}{0}
\begin{listWWNumxlevelii}
\item 
\begin{stylelsEnumerated}
\textbf{Researcher on video speaking directly to participant:} Oh yes, and by the way, I ran into Laura the other day, we went out for coffee. I know you’re not crazy about her, but I invited her to my birthday party. That will be ok, won’t it? 
\end{stylelsEnumerated}

\end{listWWNumxlevelii}
\item 
\begin{stylelsEnumerated}
\textbf{Contextualization}: Your friend is telling you all about her birthday plans; tell her what you think of them. (Opinion, non-face-threatening)
\end{stylelsEnumerated}


\setcounter{listWWNumxlevelii}{0}
\begin{listWWNumxlevelii}
\item 
\begin{stylelsEnumerated}
\textbf{Researcher on video speaking directly to participant:} As you know it’s my birthday coming up next week, and I have a few ideas about what I’d like to do. I thought about inviting everyone for dinner at my house, maybe everyone could bring a dish, then, afterwards we could go out and celebrate in this bar I know where you can drink and dance. 
\end{stylelsEnumerated}

\end{listWWNumxlevelii}
\item 
\begin{stylelsEnumerated}
\textbf{Contextualization}: You are shopping with a friend, they are trying on a hat that you think is very old-fashioned looking, and the colour (red) is terrible. You don’t like it at all. (opinion face-threatening) 
\end{stylelsEnumerated}


\setcounter{listWWNumxlevelii}{0}
\begin{listWWNumxlevelii}
\item 
\begin{stylelsEnumerated}
\textbf{Researcher on video speaking directly to participant:} Oh, I love just love hats, all kinds really. This red one is quite nice. What do you think, does it suit me? 
\end{stylelsEnumerated}

\end{listWWNumxlevelii}
\item 
\begin{stylelsEnumerated}
\textbf{Contextualization}: Your friends gave you a sweater as a gift. You don’t really like it and you want to return it. You need to ask your friend for the receipt so you can exchange it. (request, face threatening)
\end{stylelsEnumerated}


\setcounter{listWWNumxlevelii}{0}
\begin{listWWNumxlevelii}
\item 
\begin{stylelsEnumerated}
\textbf{Researcher on video speaking directly to participant:} So, have you had time to try on the sweater? Does it fit? We all hope you like it.
\end{stylelsEnumerated}

\end{listWWNumxlevelii}
\item 
\begin{stylelsEnumerated}
\textbf{Contextualization}: You are meeting your friend for a coffee and just missed the train; you’ll now be a few minutes late. (apology non-face threatening) 
\end{stylelsEnumerated}


\setcounter{listWWNumxlevelii}{0}
\begin{listWWNumxlevelii}
\item 
\begin{stylelsEnumerated}
\textbf{Researcher on video speaking directly to participant:} Hi, I am here waiting. Where are you? 
\end{stylelsEnumerated}

\end{listWWNumxlevelii}
\item 
\begin{stylelsEnumerated}
\textbf{Contextualization}: \ Your friend’s party started at 10. It is now 11 and you will not be able to go at all. You know she is going to be very disappointed. You call her and tell her. (apology, face threatening) 
\end{stylelsEnumerated}


\setcounter{listWWNumxlevelii}{0}
\begin{listWWNumxlevelii}
\item 
\begin{stylelsEnumerated}
\textbf{Researcher on video speaking directly to participant:} Hi, where are you? Are you on your way? 
\end{stylelsEnumerated}

\end{listWWNumxlevelii}
\item 
\begin{stylelsEnumerated}
\textbf{Contextualization}: Your friend has just picked you up in their car, and has all the windows down. You are cold and need to ask them to turn on the heat or roll up the windows. 
\end{stylelsEnumerated}


\setcounter{listWWNumxlevelii}{0}
\begin{listWWNumxlevelii}
\item 
\begin{stylelsEnumerated}
\textbf{Researcher on video speaking directly to participant:} Nothing, researcher provides interlocutor only. 
\end{stylelsEnumerated}

\end{listWWNumxlevelii}
\item 
\begin{stylelsEnumerated}
\textbf{Contextualization}: Your friend gets to the party and really looks great. You can tell that they cut their hair and have bought new clothes. You want to tell them how good they look. (Compliment, non-face threatening) 
\end{stylelsEnumerated}


\setcounter{listWWNumxlevelii}{0}
\begin{listWWNumxlevelii}
\item 
\begin{stylelsEnumerated}
\textbf{Researcher on video speaking directly to participant:} Nothing, researcher provides interlocutor only. 
\end{stylelsEnumerated}

\end{listWWNumxlevelii}
\item 
\begin{stylelsEnumerated}
\textbf{Contextualization}: You have been talking to this person at the party for a while and they are really starting to bother you. They keep making fun of your friends and you find it insulting, you find them offensive. You have tried to walk away, but they keep cornering you. You will have to tell them to leave you alone. (aggressive situation) 
\end{stylelsEnumerated}


\setcounter{listWWNumxlevelii}{0}
\begin{listWWNumxlevelii}
\item 
\begin{stylelsEnumerated}
\textbf{Researcher on video speaking directly to participant:} Nothing, researcher provides interlocutor only.
\end{stylelsEnumerated}

\end{listWWNumxlevelii}
\end{listWWNumxleveli}
\end{document}
