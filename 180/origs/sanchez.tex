% This file was converted to LaTeX by Writer2LaTeX ver. 1.0.2
% see http://writer2latex.sourceforge.net for more info
\documentclass[12pt]{article}
\usepackage[utf8]{inputenc}
\usepackage[T1]{fontenc}
\usepackage[english]{babel}
\usepackage{amsmath}
\usepackage{amssymb,amsfonts,textcomp}
\usepackage{array}
\usepackage{supertabular}
\usepackage{hhline}
\usepackage{hyperref}
\hypersetup{colorlinks=true, linkcolor=blue, citecolor=blue, filecolor=blue, urlcolor=blue}
\makeatletter
\newcommand\arraybslash{\let\\\@arraycr}
\makeatother
\raggedbottom
% Paragraph styles
\renewcommand\familydefault{\rmdefault}
\newenvironment{stylelsSourceline}{\renewcommand\baselinestretch{1.0}\setlength\leftskip{0.748in}\setlength\rightskip{0in plus 1fil}\setlength\parindent{0in}\setlength\parfillskip{0pt plus 1fil}\setlength\parskip{0in plus 1pt}\writerlistparindent\writerlistleftskip\leavevmode\normalfont\normalsize\itshape\writerlistlabel\ignorespaces}{\unskip\vspace{0in plus 1pt}\par}
\newenvironment{stylelsAbstract}{\setlength\leftskip{0.5in}\setlength\rightskip{0.5in}\setlength\parindent{0in}\setlength\parfillskip{0pt plus 1fil}\setlength\parskip{0in plus 1pt}\writerlistparindent\writerlistleftskip\leavevmode\normalfont\normalsize\itshape\writerlistlabel\ignorespaces}{\unskip\vspace{0.111in plus 0.0111in}\par}
\newenvironment{stylelsSectioni}{\setlength\leftskip{0.25in}\setlength\rightskip{0in plus 1fil}\setlength\parindent{0in}\setlength\parfillskip{0pt plus 1fil}\setlength\parskip{0.1665in plus 0.016649999in}\writerlistparindent\writerlistleftskip\leavevmode\normalfont\normalsize\fontsize{18pt}{21.6pt}\selectfont\bfseries\writerlistlabel\ignorespaces}{\unskip\vspace{0.0835in plus 0.00835in}\par}
\newenvironment{styleStandard}{\setlength\leftskip{0cm}\setlength\rightskip{0cm plus 1fil}\setlength\parindent{0cm}\setlength\parfillskip{0pt plus 1fil}\setlength\parskip{0in plus 1pt}\writerlistparindent\writerlistleftskip\leavevmode\normalfont\normalsize\writerlistlabel\ignorespaces}{\unskip\vspace{0.111in plus 0.0111in}\par}
\newenvironment{stylelsSectionii}{\setlength\leftskip{0cm}\setlength\rightskip{0cm plus 1fil}\setlength\parindent{0cm}\setlength\parfillskip{0pt plus 1fil}\setlength\parskip{0.222in plus 0.0222in}\writerlistparindent\writerlistleftskip\leavevmode\normalfont\normalsize\fontsize{16pt}{19.2pt}\selectfont\bfseries\writerlistlabel\ignorespaces}{\unskip\vspace{0.0835in plus 0.00835in}\par}
\newenvironment{stylelsEnumerated}{\renewcommand\baselinestretch{1.0}\setlength\leftskip{0cm}\setlength\rightskip{0cm plus 1fil}\setlength\parindent{0cm}\setlength\parfillskip{0pt plus 1fil}\setlength\parskip{0in plus 1pt}\writerlistparindent\writerlistleftskip\leavevmode\normalfont\normalsize\writerlistlabel\ignorespaces}{\unskip\vspace{0.0972in plus 0.00972in}\par}
\newenvironment{stylecaption}{\setlength\leftskip{0cm}\setlength\rightskip{0cm plus 1fil}\setlength\parindent{0cm}\setlength\parfillskip{0pt plus 1fil}\setlength\parskip{0.0835in plus 0.00835in}\writerlistparindent\writerlistleftskip\leavevmode\normalfont\normalsize\fontsize{10pt}{12.0pt}\selectfont\itshape\writerlistlabel\ignorespaces}{\unskip\vspace{0.0835in plus 0.00835in}\par}
\newenvironment{stylelsTable}{\setlength\leftskip{0cm}\setlength\rightskip{0cm}\setlength\parindent{0cm}\setlength\parfillskip{0pt plus 1fil}\setlength\parskip{0.0201in plus 0.00201in}\writerlistparindent\writerlistleftskip\leavevmode\normalfont\normalsize\mdseries\writerlistlabel\ignorespaces}{\unskip\vspace{0in plus 1pt}\par}
\newenvironment{stylelsLanginfo}{\renewcommand\baselinestretch{1.0}\setlength\leftskip{0.0783in}\setlength\rightskip{0in plus 1fil}\setlength\parindent{0in}\setlength\parfillskip{0pt plus 1fil}\setlength\parskip{0in plus 1pt}\writerlistparindent\writerlistleftskip\leavevmode\normalfont\normalsize\writerlistlabel\ignorespaces}{\unskip\vspace{0in plus 1pt}\par}
\newenvironment{stylelsSectioniii}{\setlength\leftskip{0.5717in}\setlength\rightskip{0in plus 1fil}\setlength\parindent{0in}\setlength\parfillskip{0pt plus 1fil}\setlength\parskip{0.0972in plus 0.00972in}\writerlistparindent\writerlistleftskip\leavevmode\normalfont\normalsize\fontsize{14pt}{16.8pt}\selectfont\bfseries\writerlistlabel\ignorespaces}{\unskip\vspace{0in plus 1pt}\par}
% List styles
\newcommand\writerlistleftskip{}
\newcommand\writerlistparindent{}
\newcommand\writerlistlabel{}
\newcommand\writerlistremovelabel{\aftergroup\let\aftergroup\writerlistparindent\aftergroup\relax\aftergroup\let\aftergroup\writerlistlabel\aftergroup\relax}
\newcounter{listWWNumxxiileveli}
\newcounter{listWWNumxxiilevelii}[listWWNumxxiileveli]
\newcounter{listWWNumxxiileveliii}[listWWNumxxiilevelii]
\newcounter{listWWNumxxiileveliv}[listWWNumxxiileveliii]
\renewcommand\thelistWWNumxxiileveli{\arabic{listWWNumxxiileveli}}
\renewcommand\thelistWWNumxxiilevelii{\arabic{listWWNumxxiileveli}.\arabic{listWWNumxxiilevelii}}
\renewcommand\thelistWWNumxxiileveliii{\arabic{listWWNumxxiileveli}.\arabic{listWWNumxxiilevelii}.\arabic{listWWNumxxiileveliii}}
\renewcommand\thelistWWNumxxiileveliv{\arabic{listWWNumxxiileveli}.\arabic{listWWNumxxiilevelii}.\arabic{listWWNumxxiileveliii}.\arabic{listWWNumxxiileveliv}}
\newcommand\labellistWWNumxxiileveli{\thelistWWNumxxiileveli.}
\newcommand\labellistWWNumxxiilevelii{\thelistWWNumxxiilevelii.}
\newcommand\labellistWWNumxxiileveliii{\thelistWWNumxxiileveliii.}
\newcommand\labellistWWNumxxiileveliv{\thelistWWNumxxiileveliv.}
\newenvironment{listWWNumxxiileveli}{\def\writerlistleftskip{\addtolength\leftskip{0.0cm}}\def\writerlistparindent{}\def\writerlistlabel{}\def\item{\def\writerlistparindent{\setlength\parindent{-0cm}}\def\writerlistlabel{\stepcounter{listWWNumxxiileveli}\makebox[0cm][l]{\labellistWWNumxxiileveli}\hspace{0cm}\writerlistremovelabel}}}{}
\newenvironment{listWWNumxxiilevelii}{\def\writerlistleftskip{\addtolength\leftskip{0.0cm}}\def\writerlistparindent{}\def\writerlistlabel{}\def\item{\def\writerlistparindent{\setlength\parindent{-0cm}}\def\writerlistlabel{\stepcounter{listWWNumxxiilevelii}\makebox[0cm][l]{\labellistWWNumxxiilevelii}\hspace{0cm}\writerlistremovelabel}}}{}
\newenvironment{listWWNumxxiileveliii}{\def\writerlistleftskip{\addtolength\leftskip{0.0cm}}\def\writerlistparindent{}\def\writerlistlabel{}\def\item{\def\writerlistparindent{\setlength\parindent{-0cm}}\def\writerlistlabel{\stepcounter{listWWNumxxiileveliii}\makebox[0cm][r]{\labellistWWNumxxiileveliii}\hspace{0cm}\writerlistremovelabel}}}{}
\newenvironment{listWWNumxxiileveliv}{\def\writerlistleftskip{\addtolength\leftskip{0.0cm}}\def\writerlistparindent{}\def\writerlistlabel{}\def\item{\def\writerlistparindent{\setlength\parindent{-0cm}}\def\writerlistlabel{\stepcounter{listWWNumxxiileveliv}\makebox[0cm][l]{\labellistWWNumxxiileveliv}\hspace{0cm}\writerlistremovelabel}}}{}
\newcounter{listWWNumxxxiileveli}
\newcounter{listWWNumxxxiilevelii}[listWWNumxxxiileveli]
\newcounter{listWWNumxxxiileveliii}[listWWNumxxxiilevelii]
\newcounter{listWWNumxxxiileveliv}[listWWNumxxxiileveliii]
\renewcommand\thelistWWNumxxxiileveli{\arabic{listWWNumxxxiileveli}}
\renewcommand\thelistWWNumxxxiilevelii{\alph{listWWNumxxxiilevelii}}
\renewcommand\thelistWWNumxxxiileveliii{\roman{listWWNumxxxiileveliii}}
\renewcommand\thelistWWNumxxxiileveliv{\arabic{listWWNumxxxiileveliv}}
\newcommand\labellistWWNumxxxiileveli{(\thelistWWNumxxxiileveli)}
\newcommand\labellistWWNumxxxiilevelii{\thelistWWNumxxxiilevelii.}
\newcommand\labellistWWNumxxxiileveliii{\thelistWWNumxxxiileveliii.}
\newcommand\labellistWWNumxxxiileveliv{\thelistWWNumxxxiileveliv.}
\newenvironment{listWWNumxxxiileveli}{\def\writerlistleftskip{\addtolength\leftskip{0.0cm}}\def\writerlistparindent{}\def\writerlistlabel{}\def\item{\def\writerlistparindent{\setlength\parindent{-0cm}}\def\writerlistlabel{\stepcounter{listWWNumxxxiileveli}\makebox[0cm][l]{\labellistWWNumxxxiileveli}\hspace{0cm}\writerlistremovelabel}}}{}
\newenvironment{listWWNumxxxiilevelii}{\def\writerlistleftskip{\addtolength\leftskip{0.0cm}}\def\writerlistparindent{}\def\writerlistlabel{}\def\item{\def\writerlistparindent{\setlength\parindent{-0cm}}\def\writerlistlabel{\stepcounter{listWWNumxxxiilevelii}\makebox[0cm][l]{\labellistWWNumxxxiilevelii}\hspace{0cm}\writerlistremovelabel}}}{}
\newenvironment{listWWNumxxxiileveliii}{\def\writerlistleftskip{\addtolength\leftskip{0.0cm}}\def\writerlistparindent{}\def\writerlistlabel{}\def\item{\def\writerlistparindent{\setlength\parindent{-0cm}}\def\writerlistlabel{\stepcounter{listWWNumxxxiileveliii}\makebox[0cm][r]{\labellistWWNumxxxiileveliii}\hspace{0cm}\writerlistremovelabel}}}{}
\newenvironment{listWWNumxxxiileveliv}{\def\writerlistleftskip{\addtolength\leftskip{0.0cm}}\def\writerlistparindent{}\def\writerlistlabel{}\def\item{\def\writerlistparindent{\setlength\parindent{-0cm}}\def\writerlistlabel{\stepcounter{listWWNumxxxiileveliv}\makebox[0cm][l]{\labellistWWNumxxxiileveliv}\hspace{0cm}\writerlistremovelabel}}}{}
\newcounter{listWWNumxxxileveli}
\newcounter{listWWNumxxxilevelii}[listWWNumxxxileveli]
\newcounter{listWWNumxxxileveliii}[listWWNumxxxilevelii]
\newcounter{listWWNumxxxileveliv}[listWWNumxxxileveliii]
\renewcommand\thelistWWNumxxxileveli{\alph{listWWNumxxxileveli}}
\renewcommand\thelistWWNumxxxilevelii{\alph{listWWNumxxxilevelii}}
\renewcommand\thelistWWNumxxxileveliii{\roman{listWWNumxxxileveliii}}
\renewcommand\thelistWWNumxxxileveliv{\arabic{listWWNumxxxileveliv}}
\newcommand\labellistWWNumxxxileveli{\thelistWWNumxxxileveli.}
\newcommand\labellistWWNumxxxilevelii{\thelistWWNumxxxilevelii.}
\newcommand\labellistWWNumxxxileveliii{\thelistWWNumxxxileveliii.}
\newcommand\labellistWWNumxxxileveliv{\thelistWWNumxxxileveliv.}
\newenvironment{listWWNumxxxileveli}{\def\writerlistleftskip{\addtolength\leftskip{0.0cm}}\def\writerlistparindent{}\def\writerlistlabel{}\def\item{\def\writerlistparindent{\setlength\parindent{-0cm}}\def\writerlistlabel{\stepcounter{listWWNumxxxileveli}\makebox[0cm][l]{\labellistWWNumxxxileveli}\hspace{0cm}\writerlistremovelabel}}}{}
\newenvironment{listWWNumxxxilevelii}{\def\writerlistleftskip{\addtolength\leftskip{0.0cm}}\def\writerlistparindent{}\def\writerlistlabel{}\def\item{\def\writerlistparindent{\setlength\parindent{-0cm}}\def\writerlistlabel{\stepcounter{listWWNumxxxilevelii}\makebox[0cm][l]{\labellistWWNumxxxilevelii}\hspace{0cm}\writerlistremovelabel}}}{}
\newenvironment{listWWNumxxxileveliii}{\def\writerlistleftskip{\addtolength\leftskip{0.0cm}}\def\writerlistparindent{}\def\writerlistlabel{}\def\item{\def\writerlistparindent{\setlength\parindent{-0cm}}\def\writerlistlabel{\stepcounter{listWWNumxxxileveliii}\makebox[0cm][r]{\labellistWWNumxxxileveliii}\hspace{0cm}\writerlistremovelabel}}}{}
\newenvironment{listWWNumxxxileveliv}{\def\writerlistleftskip{\addtolength\leftskip{0.0cm}}\def\writerlistparindent{}\def\writerlistlabel{}\def\item{\def\writerlistparindent{\setlength\parindent{-0cm}}\def\writerlistlabel{\stepcounter{listWWNumxxxileveliv}\makebox[0cm][l]{\labellistWWNumxxxileveliv}\hspace{0cm}\writerlistremovelabel}}}{}
\newcounter{listWWNumxxxiiileveli}
\newcounter{listWWNumxxxiiilevelii}[listWWNumxxxiiileveli]
\newcounter{listWWNumxxxiiileveliii}[listWWNumxxxiiilevelii]
\newcounter{listWWNumxxxiiileveliv}[listWWNumxxxiiileveliii]
\renewcommand\thelistWWNumxxxiiileveli{\alph{listWWNumxxxiiileveli}}
\renewcommand\thelistWWNumxxxiiilevelii{\alph{listWWNumxxxiiilevelii}}
\renewcommand\thelistWWNumxxxiiileveliii{\roman{listWWNumxxxiiileveliii}}
\renewcommand\thelistWWNumxxxiiileveliv{\arabic{listWWNumxxxiiileveliv}}
\newcommand\labellistWWNumxxxiiileveli{\thelistWWNumxxxiiileveli.}
\newcommand\labellistWWNumxxxiiilevelii{\thelistWWNumxxxiiilevelii.}
\newcommand\labellistWWNumxxxiiileveliii{\thelistWWNumxxxiiileveliii.}
\newcommand\labellistWWNumxxxiiileveliv{\thelistWWNumxxxiiileveliv.}
\newenvironment{listWWNumxxxiiileveli}{\def\writerlistleftskip{\addtolength\leftskip{0.0cm}}\def\writerlistparindent{}\def\writerlistlabel{}\def\item{\def\writerlistparindent{\setlength\parindent{-0cm}}\def\writerlistlabel{\stepcounter{listWWNumxxxiiileveli}\makebox[0cm][l]{\labellistWWNumxxxiiileveli}\hspace{0cm}\writerlistremovelabel}}}{}
\newenvironment{listWWNumxxxiiilevelii}{\def\writerlistleftskip{\addtolength\leftskip{0.0cm}}\def\writerlistparindent{}\def\writerlistlabel{}\def\item{\def\writerlistparindent{\setlength\parindent{-0cm}}\def\writerlistlabel{\stepcounter{listWWNumxxxiiilevelii}\makebox[0cm][l]{\labellistWWNumxxxiiilevelii}\hspace{0cm}\writerlistremovelabel}}}{}
\newenvironment{listWWNumxxxiiileveliii}{\def\writerlistleftskip{\addtolength\leftskip{0.0cm}}\def\writerlistparindent{}\def\writerlistlabel{}\def\item{\def\writerlistparindent{\setlength\parindent{-0cm}}\def\writerlistlabel{\stepcounter{listWWNumxxxiiileveliii}\makebox[0cm][r]{\labellistWWNumxxxiiileveliii}\hspace{0cm}\writerlistremovelabel}}}{}
\newenvironment{listWWNumxxxiiileveliv}{\def\writerlistleftskip{\addtolength\leftskip{0.0cm}}\def\writerlistparindent{}\def\writerlistlabel{}\def\item{\def\writerlistparindent{\setlength\parindent{-0cm}}\def\writerlistlabel{\stepcounter{listWWNumxxxiiileveliv}\makebox[0cm][l]{\labellistWWNumxxxiiileveliv}\hspace{0cm}\writerlistremovelabel}}}{}
\newcounter{listWWNumxxxivleveli}
\newcounter{listWWNumxxxivlevelii}[listWWNumxxxivleveli]
\newcounter{listWWNumxxxivleveliii}[listWWNumxxxivlevelii]
\newcounter{listWWNumxxxivleveliv}[listWWNumxxxivleveliii]
\renewcommand\thelistWWNumxxxivleveli{\arabic{listWWNumxxxivleveli}}
\renewcommand\thelistWWNumxxxivlevelii{\alph{listWWNumxxxivlevelii}}
\renewcommand\thelistWWNumxxxivleveliii{}
\renewcommand\thelistWWNumxxxivleveliv{}
\newcommand\labellistWWNumxxxivleveli{(\thelistWWNumxxxivleveli)}
\newcommand\labellistWWNumxxxivlevelii{\thelistWWNumxxxivlevelii.}
\newcommand\labellistWWNumxxxivleveliii{\thelistWWNumxxxivleveliii}
\newcommand\labellistWWNumxxxivleveliv{\thelistWWNumxxxivleveliv}
\newenvironment{listWWNumxxxivleveli}{\def\writerlistleftskip{\addtolength\leftskip{0.0cm}}\def\writerlistparindent{}\def\writerlistlabel{}\def\item{\def\writerlistparindent{\setlength\parindent{-0cm}}\def\writerlistlabel{\stepcounter{listWWNumxxxivleveli}\makebox[0cm][l]{\labellistWWNumxxxivleveli}\hspace{0cm}\writerlistremovelabel}}}{}
\newenvironment{listWWNumxxxivlevelii}{\def\writerlistleftskip{\addtolength\leftskip{0.0cm}}\def\writerlistparindent{}\def\writerlistlabel{}\def\item{\def\writerlistparindent{\setlength\parindent{-0cm}}\def\writerlistlabel{\stepcounter{listWWNumxxxivlevelii}\makebox[0cm][l]{\labellistWWNumxxxivlevelii}\hspace{0cm}\writerlistremovelabel}}}{}
\newenvironment{listWWNumxxxivleveliii}{\def\writerlistleftskip{\addtolength\leftskip{0.0cm}}\def\writerlistparindent{}\def\writerlistlabel{}\def\item{\def\writerlistparindent{\setlength\parindent{-0cm}}\def\writerlistlabel{\stepcounter{listWWNumxxxivleveliii}\makebox[0cm][l]{\labellistWWNumxxxivleveliii}\hspace{0cm}\writerlistremovelabel}}}{}
\newenvironment{listWWNumxxxivleveliv}{\def\writerlistleftskip{\addtolength\leftskip{0.0cm}}\def\writerlistparindent{}\def\writerlistlabel{}\def\item{\def\writerlistparindent{\setlength\parindent{-0cm}}\def\writerlistlabel{\stepcounter{listWWNumxxxivleveliv}\makebox[0cm][l]{\labellistWWNumxxxivleveliv}\hspace{0cm}\writerlistremovelabel}}}{}
\setlength\tabcolsep{1mm}
\renewcommand\arraystretch{1.3}
% footnotes configuration
\makeatletter
\renewcommand\thefootnote{\arabic{footnote}}
\makeatother
\title{}
\author{Jennifer Ament}
\date{2018-02-22}
\begin{document}
\title{\textsuperscript{Acculturation and pragmatic learning: International students in the US}}
\maketitle

\begin{stylelsSourceline}
Ariadna Sánchez Hernández,\textsuperscript{ }Leuphana University Lüneburg
\end{stylelsSourceline}


\begin{stylelsAbstract}
The present study explores the relationship between acculturation and the development of second language (L2) pragmatic competence during a semester-long study abroad (SA) program in the United States (US). Drawing on Schumann’s (1978) Acculturation theory of L2 acquisition, it was hypothesized that the degree to which SA participants acculturate socially and psychologically to the target language community would be related to the extent to which they acquire L2 pragmatic competence.~Twelve international students of three different nationalities – Brazilian, Turkish and Spanish – in their first semester of study at a US university completed a pre-test and a post-test version of a discourse completion task that measured their ability to produce speech acts, and of a sociocultural adaptation scale (Ward \& Kennedy, 1999) that measured their acculturation. Additionally, they participated in semi-structured interviews at the beginning and at the end of the stay that provided insights into their SA adaptation experiences.~An exploration of individual trajectories indicated that gains in pragmatic competence were promoted by acculturation development. On the one hand, pragmatic gains were related to social variables that included the integration strategy adopted and academic pressure. On the other hand, they were associated with affective factors such as social support from home-country peers. The reported findings bring new insights to the field of L2 pragmatics by examining the effects of acculturation. Ultimately, the results emphasize the importance of enhancing L2 learners’ social and affective adaptation during SA programs, so as to maximize their acculturation experiences and their subsequent L2 pragmatic learning.
\end{stylelsAbstract}

\setcounter{listWWNumxxiileveli}{0}
\begin{listWWNumxxiileveli}
\item 
\begin{stylelsSectioni}
Introduction
\end{stylelsSectioni}

\end{listWWNumxxiileveli}
\begin{styleStandard}
A main consequence of globalization is the increase of study abroad (SA) programs all over the world, which have even become mandatory for many university students. However, the traditional view that SA programs are the optimal context for learning\footnote{Acknowledging the difference between the terms \textit{acquisition} and \textit{learning} pointed out by Krashen (1985) – i.e. natural acquisition vs. acquisition that involves formal instruction respectively –, the present study follows the mainstream use of both terms as synonyms to refer to language development. } a second language (L2) is being challenged by studies reporting cases of unsuccessful adaptation experiences by international students (for a review, see Mitchell et al. 2015; 2017), and limited acquisition of some pragmatic features (for a review, see Taguchi \& Roever 2017: Chapter 7). This is not surprising if one considers that SA participants not only have to focus on improving their L2 proficiency, but also have to face the multiple challenges involved in the process of adapting to a new and unknown setting, while being expected to interact with people of diverse sociocultural backgrounds. Drawing on this idea, recent research (e.g., Taguchi et al. 2016; Sykes 2017; Taguchi 2017; Taguchi \& Roever 2017) points out that a problem in understanding SA outcomes is that there is a scarcity of empirical support for the relationship between intercultural and pragmatic competences, areas that have traditionally belonged to different domains (Psychology and Linguistics, respectively). 
\end{styleStandard}

\begin{styleStandard}
To address this problem, the present study explores the extent to which SA participants’ acculturation experiences are related to the development of their pragmatic competence in the SA context. Drawing on Schumann’s (1978; 1986) Acculturation theory of L2 acquisition, the study is based on the premise that the degree to which an individual acculturates to the target language society will determine his/her acquisition of the L2. According to Schumann (1978), in the process of adaptation in a new environment, different social variables (e.g., integration strategies, attitude towards the host culture) and affective factors (e.g., culture shock, motivation) are at play. While the Acculturation model has commonly been used in the general field of L2 acquisition (e.g., Hansen 1995; Lybeck 2002), its application in the field of L2 pragmatics still represents a research desideratum. There is not conclusive evidence as to whether acquisition of pragmatic ability during a stay abroad is related to students’ acculturation experiences.
\end{styleStandard}

\begin{listWWNumxxiileveli}
\item 
\begin{stylelsSectioni}
Literature review
\end{stylelsSectioni}


\setcounter{listWWNumxxiilevelii}{0}
\begin{listWWNumxxiilevelii}
\item 
\begin{stylelsSectionii}
Study abroad programs as a context for learning pragmatics
\end{stylelsSectionii}

\end{listWWNumxxiilevelii}
\end{listWWNumxxiileveli}
\begin{styleStandard}
Study abroad programs – that is, temporary educational sojourns in which a target language is used by the members of the community (Taguchi 2015a) – have typically been referred to as the optimal context for the acquisition of pragmatic competence. Mastering pragmatic competence in a L2 involves learning how to the use language appropriately to the context, the situation and the interlocutors. In other words, knowing “\textit{when} and \textit{where} to say something, \textit{what} to say, [and] to \textit{whom} to say it in a given social and linguistic context” (García 1989: 314). Pragmatic ability mainly involves the ability to perform speech acts, such as suggestions, requests, refusals, apologies, and compliments among others. In addition, it concerns the mastery of pragmatic features like implied meaning, pragmatic routines (i.e. formulaic language recurrently used by NSs in given situations), and managing interaction (i.e. turn-taking or conversation openings). \ While studying abroad, learners are likely to acquire these features as they have the potential to have rich exposure to the L2 outside of class, and plenty of opportunities to use the language in diverse social situations, with different interlocutors and for real-life purposes. Moreover, they continuously witness interactions among users of the L2 that provide them with valuable and authentic input. Taguchi (2015a) summarizes the main elements that make the SA context potentially optimal for pragmatic learning as follows:
\end{styleStandard}

\begin{styleStandard}
(1) Opportunities to observe local norms of interaction; (2) contextualized pragmatic practice and immediate feedback on that practice; (3) real-life consequences of pragmatic behavior; and (4) exposure to variation in styles and communicative situations (Taguchi 2015a: 4).
\end{styleStandard}

\begin{styleStandard}
Indeed, there is a burgeoning of studies in the field of interlanguage pragmatics (ILP) that have pointed out the advantage of the SA context for the development of different pragmatic features. These studies have typically carried out cross-sectional investigations, that is, comparing pragmatic ability among groups of L2 learners or with native speakers (NSs) and, to a lesser extent, they have conducted longitudinal studies that examine pragmatic development over time (see Alcón-Soler 2014, for a review of cross-sectional and longitudinal ILP findings in the SA context). All in all, these studies have reported that during SA, learners improve their pragmatic awareness, their production of speech acts, their use of pragmatic routines, and their comprehension of implied meaning. 
\end{styleStandard}

\begin{styleStandard}
Nevertheless, longitudinal ILP studies agree that the advantage of the SA context for pragmatic development is not straightforward. The process of acquiring L2 pragmatic competence is variable and non-linear, as it depends on (1) the pragmatic feature under study and (2) on different factors associated with the SA setting. For instance, SA seems to be beneficial for the acquisition of pragmatic routines, but there are mixed findings on its benefits for the ability to comprehend implied meaning and to produce certain speech acts (Taguchi \& Roever 2017). Indeed, not all speech acts present the same degree of difficulty. For instance, greetings, leave-takings and offers are acquired more quickly, and thus students learn them at earlier stages of immersion, while appropriate use of requests, refusals and invitations is achieved at a slower rate, and is thus more common of in longer SA sojourns (Barron 2003; Félix-Brasdefer 2004; Hassall 2006).
\end{styleStandard}

\begin{styleStandard}
ILP scholars have commonly classified predictors of pragmatic learning during SA into two main categories: external factors related to the context, and internal ones related to learners’ individual differences. The main external factors investigated in ILP research are length of stay and intensity of interaction with users of the L2, with studies reporting that amount of interaction is a better predictor of pragmatic development than length of stay (e.g., Bardovi-Harlig \& Bastos 2011; Bella 2011). That is, spending more time in the target language setting is not enough on its own to fully develop pragmatic competence, as L2 learners need to be willing to take advantage of the opportunities for interaction offered by the context. Nevertheless, a focus on the role of external factors does not seem to be enough to explain L2 pragmatic acquisition, since internal factors often interfere with the effect of contextual variables. Evidence of this fact is provided, for example, by Eslami \& Jin Ahn (2014), who explored how pragmatic development (measured in terms of the ability to respond to compliments) by Korean students in the US was influenced by two external factors (length of stay and intensity of interaction) and one internal variable (motivation), reporting that only motivation had a positive impact on pragmatic development. All in all, proficiency has been the most investigated internal predictor of pragmatic acquisition, with most research findings indicating that having a certain proficiency level enhances the acquisition of most pragmatic features, although lower-level students at times outperform higher-level ones depending on the pragmatic feature and on the context (for a review, see Xiao 2015).
\end{styleStandard}

\begin{styleStandard}
In sum, although most ILP investigations have revealed positive gains in pragmatic ability during SA, they have also reported that such pragmatic development is variable and non-linear, as it is influenced by different factors. Drawing on this idea, some scholars (e.g., Taguchi 2015a) have expressed the need for ILP studies to focus on the processes – rather than merely on the outcomes – of SA. This implies a call for more longitudinal research on the factors that influence the development of pragmatic ability over time. More particularly, in a recent monograph, Taguchi and Roever (2017) call for ILP studies to investigate new variables that have gained importance in the current era of globalization, such as intercultural competence, an umbrella term under which the concept of acculturation is embedded.
\end{styleStandard}

\begin{listWWNumxxiileveli}
\item 
\setcounter{listWWNumxxiilevelii}{0}
\begin{listWWNumxxiilevelii}
\item 
\begin{stylelsSectionii}
Acculturation and pragmatic learning
\end{stylelsSectionii}

\end{listWWNumxxiilevelii}
\end{listWWNumxxiileveli}
\begin{styleStandard}
Bridging the gap between internal and external factors that affect L2 pragmatic learning, the present study focuses on the variable of acculturation. Acculturation is a multifold phenomenon that is defined as “the process of cultural change that occurs when individuals from different cultural backgrounds come into prolonged, continuous, first-hand contact with each other” (Redfield et al. 1936: 146). It has been operationalized in terms of three main constructs: acculturation conditions (antecedent factors such as the characteristics of the sojourning and host cultures, of the sojourning group, and of the individuals), acculturation orientations (strategies of integration in the host society, such as assimilation, marginalization, separation, or integration), and acculturation outcomes, which include sociocultural adaptation (implying behavioral aspects, skills, attitudes, and cultural knowledge) and psychological adaptation (sojourners’ well-being and satisfaction) (see Arends-Tóth \& Van de Vijver 2006: Figure 1). 
\end{styleStandard}

\begin{styleStandard}
Different models have been proposed in an attempt to explore the influence of acculturation on the acquisition of an L2. Three major frameworks include the Inter-group model by Giles and colleagues (Beebe \& Giles 1984), the Socio-Educational model by Gardner (Gardner et al. 1983), and the Acculturation model by Schumann (1978; 1986) (see Ellis 1994: Chapter 3, for a review of each). The present study takes Schumann’s model as a reference to understand the process of acculturation to a new culture, as it accounts for acculturation conditions, orientations and outcomes. Moreover, it is the only model that has generated empirical evidence concerning the relationship between acculturation and pragmatic learning. 
\end{styleStandard}

\begin{styleStandard}
Kasper \& Rose (2002), in the first seminal book that provided a comprehensive review of L2 pragmatic development, present Schumann’s (1978) Acculturation model as the first theoretical framework that explained pragmatic development. According to Schumann, the degree to which an L2 learner acculturates to the new sociocultural community will influence the extent to which he/she learns the target language, acculturation being the first (but not the only one) in a list of factors that determines L2 acquisition. A main point of Schumann’s theory is that acculturation is determined by how close the sojourner is to the target language group in terms of sociocultural and psychological adaptation. Sociocultural adaptation refers to the degree to which a language learner achieves contact with the L2 group and becomes part of it; it thus depends on the individual’s skills with respect to integration and dealing with every-day situations. Psychological adaptation involves the degree to which a student is comfortable with the learning and the adaptation processes, and therefore implies emotional well-being and personal satisfaction. To determine the amount of acculturation with regards to these two aspects, Schumann (1986), distinguishes two sets of factors. Firstly, seven social factors, provided in (1), shape sociocultural adaptation: 
\end{styleStandard}

\setcounter{listWWNumxxxiileveli}{0}
\begin{listWWNumxxxiileveli}
\item 
\begin{styleStandard}
a. Social dominance of the target language group, in terms of political, cultural, technical and economic status, as perceived by the sojourning group.
\end{styleStandard}

\end{listWWNumxxxiileveli}
\setcounter{listWWNumxxxileveli}{1}
\begin{listWWNumxxxileveli}
\item 
\begin{styleStandard}
Integration strategy: assimilation, preservation or adaptation of sociocultural values.
\end{styleStandard}

\item 
\begin{styleStandard}
Enclosure: that is, the degree to which the two cultural groups share the same social facilities. 
\end{styleStandard}

\item 
\begin{styleStandard}
Cohesiveness and size of the sojourning group
\end{styleStandard}

\item 
\begin{styleStandard}
Cultural congruence between the two groups, regarding religion, general social practice, and other beliefs.
\end{styleStandard}

\item 
\begin{styleStandard}
Attitude towards the host culture. 
\end{styleStandard}

\item 
\begin{styleStandard}
Intended length of stay in the target language context.
\end{styleStandard}

\end{listWWNumxxxileveli}
\begin{styleStandard}
Secondly, the four affective factors in (2) determine psychological adaptation: 
\end{styleStandard}

\setcounter{listWWNumxxxiileveli}{0}
\begin{listWWNumxxxiileveli}
\item 
\begin{styleStandard}
a.\ \ Language shock: fear of appearing idiotic when speaking the L2. 
\end{styleStandard}

\end{listWWNumxxxiileveli}
\setcounter{listWWNumxxxiiileveli}{1}
\begin{listWWNumxxxiiileveli}
\item 
\begin{styleStandard}
Culture shock: feelings of rejection, anxiety, and disorientation by the sojourners while living amongst members of the target community.
\end{styleStandard}

\item 
\begin{styleStandard}
Motivation: according to Schumann, an integrative motivation is more likely to assist in SLA than instrumental one. 
\end{styleStandard}

\item 
\begin{styleStandard}
Ego permeability: which refers to the extent to which the identity can be flexible and can adapt.
\end{styleStandard}

\end{listWWNumxxxiiileveli}
\begin{styleStandard}
A few studies have drawn on Schumann’s (1978) assertion that the degree to which individuals acculturate will determine the degree to which they learn the L2. Most of them suggest that L2 acquisition, especially in terms of oral proficiency, is benefited by the learners’ process of acculturation (Hansen 1995; Lybeck 2002; Jiang et al. 2009). In the field of pragmatics, Schmidt’s (1983) and Dörnyei et al.’s (2004) investigations are the only ones, to the best of our knowledge, that have applied Schumann’s model to explain L2 pragmatic development.
\end{styleStandard}

\begin{styleStandard}
Schmidt (1983) conducted a case study of Wes, a 33-year-old Japanese male who immigrated to the US (Hawaii) without having previous formal instruction in English. Wes’ development with respect to acculturation and L2 acquisition were tracked over 3 years. Having the optimal sociocultural and psychological orientations, he increased his pragmatic ability but did not improve his grammatical competence. To assess pragmatic competence, Schmidt focused on directives, which include speech acts used to get the interlocutor do something; that is, orders, requests, and suggestions. At earlier stages of pragmatic development, Wes’ use of directives was characterized by a reliance on a small number of speech formulas that he only used in specific situations (for example, \textit{shall we go?}), and by transfer from Japanese sociopragmatic norms. Over time, he improved the appropriateness of meanings, pragmatic transfer was reduced, he became aware of the differences between languages, and he developed significant control of speech act strategies and formulas used in social interactions. Therefore, Schmidt’s (1983) study confirmed that acculturation leads to increased L2 pragmatic competence.
\end{styleStandard}

\begin{styleStandard}
Schmitt et al. (2004) analyzed quantitatively how acculturation affected the use of formulaic language learning. They quickly realized that acculturation was a complex phenomenon that demanded a qualitative in-depth analysis, which led them to conduct semi-structured interviews with a subset of 7 of the participants. This second investigation, conducted by Dörnyei et al. (2004), was a case study of 7 international students having spent 7 months in a British university, in which the authors explored the participants’ acculturation development in terms of sociocultural adaptation, measured through the social factors outlined in Schumann’s (1978) model\footnote{Dörnyei et al. (2004: 88) take Schumann’s (1978) Acculturation theory as a base, but they focus on the social aspects of the process. They define acculturation as “the extent to which learners succeeded in settling in and engaging with the host community, thereby taking advantage of the social contact opportunities available”. }. Four of the participants showed positive gains in their ability to use formulaic language, while three of them did not experience such gains. Research findings indicated a strong relationship between sociocultural adaptation and pragmatic learning. In particular, acquisition of formulaic language was mainly influenced by the variables of enclosure and the integration strategy adopted, as evident in the participants’ development of social networks. Indeed, most of the participants found it extremely hard to have meaningful contact with the L2 speakers outside of class. Successful learning of formulaic language depended on whether they could “beat the odds” and come out of the “international ghetto” (Dörnyei, Durrow \& Zahran. 2004: 105). This was evident in two of the participants that scored higher in the formulaic language test. The other two successful students had extraordinary motivation and language aptitude, and therefore these two aspects also played a key role in pragmatic learning. 
\end{styleStandard}

\begin{styleStandard}
Schumann’s (1978) Acculturation theory, however, has received little further empirical support, and it has faced some criticism (Ellis 1994; Zaker 2016). The main critique has been that it is difficult to assess some of the variables proposed by Schumann. Moreover, the framework disregards additional factors that may be better predictors of L2 learning, such as individual differences (e.g., cognitive abilities, learning style) and instruction (Mondy 2007; cited in Zaker 2016). According to Mondy (2007; cited in Zaker 2016), SA learners may acculturate successfully despite not having favorable conditions in the social and affective variables proposed by Schumann. 
\end{styleStandard}

\begin{styleStandard}
Other studies have explored the role of acculturation factors on pragmatic development without drawing on Schumann’s (1978) model. Overall, they have reported that pragmatic competence is determined by specific aspects such as identity (Siegal 1995), motivation (Eslami \& Ahn 2014), and cultural similarity (Bardovi-Harlig et al. 2008). Additionally, a recent line of studies has addressed the role of intercultural competence on the development of pragmatic ability (Taguchi 2015b; Taguchi et al. 2016). Nevertheless, the question still remains as to whether the development of pragmatic competence is determined by students’ sociocultural and psychological adaptation during SA. The current study directly addresses this question, and in doing so it fills in the gap that exists between studies on acculturation and on L2 pragmatic acquisition.
\end{styleStandard}

\setcounter{listWWNumxxiileveli}{0}
\begin{listWWNumxxiileveli}
\item 
\begin{stylelsSectioni}
Research questions
\end{stylelsSectioni}

\end{listWWNumxxiileveli}
\begin{styleStandard}
This investigation sheds new light on the development of L2 pragmatic competence in the SA context by exploring the influence of acculturation on the development of speech act performance by students in their first semester of participation in a SA program in the US. Two research questions guide the study:
\end{styleStandard}

\begin{stylelsEnumerated}
\textbf{RQ1.} Does a semester of study abroad afford gains in pragmatic competence, in terms of speech act production?
\end{stylelsEnumerated}


\begin{stylelsEnumerated}
\textbf{RQ2. }To what extent, if any, are gains in pragmatic competence related to students’ acculturation development, measured in terms of sociocultural and psychological adaptation?
\end{stylelsEnumerated}


\begin{listWWNumxxiileveli}
\item 
\begin{stylelsSectioni}
Method
\end{stylelsSectioni}


\setcounter{listWWNumxxiilevelii}{0}
\begin{listWWNumxxiilevelii}
\item 
\begin{stylelsSectionii}
Research design
\end{stylelsSectionii}

\end{listWWNumxxiilevelii}
\end{listWWNumxxiileveli}
\begin{styleStandard}
To address these research questions, a mixed-method case study approach was employed. This methodology differs from purely qualitative case-study ethnography as it integrates a quantitative research component, which in this case provided an objective assessment of pragmatic competence and of sociocultural adaptation. Additionally, both sociocultural and psychological adaptation were measured qualitatively though semi-structured interviews. Moreover, the current investigation was longitudinal and involved two data-collection points: at the beginning and at the end of a semester.
\end{styleStandard}

\begin{listWWNumxxiileveli}
\item 
\setcounter{listWWNumxxiilevelii}{0}
\begin{listWWNumxxiilevelii}
\item 
\begin{stylelsSectionii}
Participants
\end{stylelsSectionii}

\end{listWWNumxxiilevelii}
\end{listWWNumxxiileveli}
\begin{styleStandard}
Twelve international students at a public university in the US Midwest participated in the study. The sample is drawn from a larger-scale study that involved 122 international students. The group of 12 was selected as they had volunteered to take part in interviews. Table 1 summarizes the demographic information about the 12 informants.
\end{styleStandard}

\begin{stylecaption}
Table 1: Demographic information about case-study informants
\end{stylecaption}

\begin{center}
\tablehead{}
\begin{supertabular}{m{1.8cm}m{1.8cm}m{1.8cm}m{1.8cm}m{1.8cm}m{1.8cm}}
\hline
\bfseries Pseudonym &
\bfseries Age &
\bfseries Gender &
\bfseries Nationality &
\bfseries Proficiency &
\bfseries \ Living \ situation\\\hline
{\mdseries David}

{\mdseries Emma}

{\mdseries Mike}

{\mdseries Sean}

{\mdseries Lisa}

{\mdseries Jeff}

{\mdseries William}

{\mdseries Steven}

{\mdseries Jason}

{\mdseries Ethan}

{\mdseries Michelle }

\mdseries Mark &
{\mdseries 23}

{\mdseries 26}

{\mdseries 20}

{\mdseries 25}

{\mdseries 24}

{\mdseries 20}

{\mdseries 22}

{\mdseries 26}

{\mdseries 26}

{\mdseries 29}

{\mdseries 29}

\mdseries 27 &
{\mdseries M}

{\mdseries F}

{\mdseries M}

{\mdseries M}

{\mdseries F}

{\mdseries M}

{\mdseries M}

{\mdseries M}

{\mdseries M}

{\mdseries M}

{\mdseries F}

\mdseries M &
{\mdseries Brazilian}

{\mdseries Spanish}

{\mdseries Brazilian}

{\mdseries Turkish }

{\mdseries Spanish}

{\mdseries Brazilian}

{\mdseries Brazilian}

{\mdseries Turkish}

{\mdseries Brazilian}

{\mdseries Spanish}

{\mdseries Turkish}

\mdseries Turkish &
{\mdseries Beginner}

{\mdseries Advanced}

{\mdseries Advanced}

{\mdseries Advanced}

{\mdseries Intermediate}

{\mdseries Advanced}

{\mdseries Advanced}

{\mdseries Advanced}

{\mdseries Intermediate}

{\mdseries Advanced}

{\mdseries Intermediate}

\mdseries Intermediate &
{\mdseries With NNSs (Brazilian)}

{\mdseries With NNSs (Spanish)}

{\mdseries With NNSs (Brazilian)}

{\mdseries Change: NSs to NNSs (diverse nationality)}

{\mdseries With NNSs (Spanish)}

{\mdseries With NNSs (Brazilian)}

{\mdseries With NNSs (Brazilian)}

{\mdseries With NNSs (diverse nationality)}

{\mdseries With NNSs (Brazilian)}

{\mdseries With NSs}

{\mdseries With NSs}

\mdseries With NSs\\\hline
\end{supertabular}
\end{center}
\begin{styleStandard}
The group consisted of 3 females and 9 males, their mean age was 24.4 (ranging from 20 to 29), and they had 3 different nationalities: Brazilian (\textit{n} = 5), Turkish (\textit{n} = 4), and Spanish (\textit{n} = 3). All of them were in their first semester of study in the US, and their living arrangements were varied: most of them were living with non-native speakers (NNSs) (\textit{n} = 8), 3 of them were living with NSs and one student changed from living with NSs to living with NNSs. Their amount of English instruction received during the semester depended on their proficiency level, which was measured by scores in an initial entrance-exam Test of English as Foreign Language (TOEFL). Beginner students (\textit{n} = 1) enrolled in full-time English classes, intermediate students (\textit{n} = 4) took part-time classes (and therefore combined them with content classes), and advanced (\textit{n} = 7) learners took occasional and specialized English courses in addition to content classes.
\end{styleStandard}

\begin{listWWNumxxiileveli}
\item 
\setcounter{listWWNumxxiilevelii}{0}
\begin{listWWNumxxiilevelii}
\item 
\begin{stylelsSectionii}
Instruments
\end{stylelsSectionii}

\end{listWWNumxxiilevelii}
\end{listWWNumxxiileveli}
\begin{styleStandard}
Three main instruments were used in the study: a written DCT that measured students’ pragmatic knowledge quantitatively, a modified Sociocultural Adaptation Scale (SCAS, Ward \& Kennedy 1999) that assessed their sociocultural adaptation, and semi-structured interviews that revealed qualitative information about students’ acculturation in the US, in terms of sociocultural and psychological adaptation. Additionally, a background questionnaire was administered to collect demographic information and to control for variables such as age, proficiency, previous experience abroad, and nationality. 
\end{styleStandard}

\begin{styleStandard}
The written DCT was developed to elicit participants’ production of speech acts in high-imposition and low-imposition situations. The choice of the instrument was based on the suitability of DCTs for the elicitation of information about speech act production, as they allow the researcher to control the given conditions and to obtain simulated oral data (Félix-Brasdefer \& Hassler-Baker 2017). The selected speech acts were requests and refusals, chosen because of the importance of their appropriate use by L2 learners for successful communication with NSs. Requests and refusals are considered “face-threatening acts” (Brown \& Levinson 1978) and therefore inappropriate production of these could lead to unintended offense by the interlocutor. Table 2 displays the classification of the speech act situations included in the DCT. 
\end{styleStandard}

\begin{stylecaption}
Table 2: Description of the situations included in the DCT
\end{stylecaption}

\begin{flushleft}
\tablehead{}
\begin{supertabular}{m{1.8cm}m{1.8cm}m{1.8cm}}
\hline
\multicolumn{2}{m{3.8cm}}{\bfseries High-imposition situations} &
\\\hline
{\mdseries 1. Request}

{\mdseries 2. Request}

{\mdseries 3. Refusal}

\mdseries 4. Refusal &
\multicolumn{2}{m{3.8cm}}{{\mdseries Asking a professor for an extension of the deadline for an assignment}

{\mdseries Asking a professor to have the test on a different day }

{\mdseries Refusing to take summer classes}

\mdseries Refusing to help a lecturer carry some books to his/her office}\\\hline
\multicolumn{2}{m{3.8cm}}{\bfseries Low-imposition situations} &
\\\hline
{\mdseries 5. Request}

{\mdseries 6. Request}

{\mdseries 7. Refusal}

\mdseries 8. Refusal &
\multicolumn{2}{m{3.8cm}}{{\mdseries Asking a friend for a pen}

{\mdseries Asking a friend to have a ride to the supermarket}

{\mdseries Refusing an invitation to a party}

\mdseries Refusing to lend your notes to a classmate\ \ }\\\hline
\end{supertabular}
\end{flushleft}
\begin{styleStandard}
To determine the high- and low-imposition categories, Brown \& Levinson’s (1978) framework was employed by considering the social distance between the speaker and hearer, the social power of the interlocutors, and the degree of imposition. High-imposition situations included formal interactions between a student (the speaker) and a professor (the hearer), in which the social distance is large, and the social power of the hearer is higher. Low-imposition scenarios involved informal interactions between two students, in which the social distance is small, and the social power of the interlocutors is equal. The selection of the situations was made on the basis of previous studies that have used DCTs to explore the production of requests and refusals (Alcón-Soler 2008; Taguchi 2006; 2011; 2013; Martínez-Flor \& Usó-Juan 2011). The DCT was validated through a pilot study conducted the previous academic semester with 8 NSs and with 21 international students enrolled at the same university. This preliminary study aimed at checking whether the situations were understood correctly and whether they elicited the corresponding speech act. 
\end{styleStandard}

\begin{styleStandard}
Regarding the assessment of acculturation, a modified version of the Sociocultural Adaptation Scale (SCAS; Ward \& Kennedy 1999) was used to measure participants’ sociocultural adaptation in the US. Drawing on Berry (2003), we focused the quantitative analysis on sociocultural – rather than on psychological – adaptation, considering that affective factors are to a certain extent responsible for students’ sociocultural adaptation orientations. The SCAS is a five-point Likert-scale in which students are asked to rate from 1 (= very difficult) to 5 (= no difficulty) their level of adaptation to 29 items. In the original instrument high scores are associated with higher levels of difficulty (that is, less degree of acculturation). In this study, items were reversed from the original scale so that higher scores correspond with a positive adaptation. These items include 21 behavioral situations such as “finding food you enjoy” and “making friends,” and 7 cognitive aspects such as “seeing things from an American point of view.” The SCAS has been widely used in empirical studies given its strong psychometric properties (Celenk \& Van de Vijver 2011). In this study, the calculation of Cronbach alpha coefficient in the entire sample of 122 participants revealed a strong internal consistency of the scale ($\alpha $ = 0.937).
\end{styleStandard}

\begin{styleStandard}
In addition to this, acculturation was measured qualitatively through semi-structured interviews at the beginning and at the end of the semester, which revealed reasons for individual trajectories of sociocultural as well as of psychological adaptation during SA. The interviews were conducted in English, in the principal researcher’s office, and had a duration of 25 to 35 minutes. The questions formulated were related to students’ acculturation experiences following Schumann’s (1978) proposal of social and psychological acculturation variables (c.f. Section 2.2). Moreover, the semi-structured format of the interviews was advantageous for the elicitation of relevant topics that could explain acculturation but were not included in Schumann’s proposal. More particularly, the following themes were pre-selected: educational background and English experience in the at-home country; goal of SA program and expectations; outcomes of SA; sociocultural adjustment (academically and socially); overall well-being; English use (interaction with English speakers); pragmatic awareness; influence of instruction.
\end{styleStandard}

\begin{listWWNumxxiileveli}
\item 
\setcounter{listWWNumxxiilevelii}{0}
\begin{listWWNumxxiilevelii}
\item 
\begin{stylelsSectionii}
Data collection
\end{stylelsSectionii}

\end{listWWNumxxiilevelii}
\end{listWWNumxxiileveli}
\begin{styleStandard}
This is a longitudinal study that employed a pre-test/post-test design. Data collection took one semester (fall semester of 2014). For the pre-test, a day and time were established during the second week of the semester, during which participants were asked to complete the written instruments (the background questionnaire, the DCT, and the SCAS) and to participate in the interviews. Completion of the written instruments took place during L2 English classes, and the session lasted for approximately 30 minutes, in which participants read and signed the consent form (5 minutes), completed the background questionnaire (5 minutes), the SCAS (10 minutes), and the pragmatic test (10 minutes). The interview sessions were held at the main researcher’s office, each lasting between 25 and 35 minutes, and they were audio recorded by means of the software Audacity. The post-test data collection sessions took place during the week before the end of the semester, and they followed the same protocol used for the pre-test.
\end{styleStandard}

\begin{listWWNumxxiileveli}
\item 
\setcounter{listWWNumxxiilevelii}{0}
\begin{listWWNumxxiilevelii}
\item 
\begin{stylelsSectionii}
Data analysis
\end{stylelsSectionii}

\end{listWWNumxxiilevelii}
\end{listWWNumxxiileveli}
\begin{styleStandard}
The first type of data to code was quantitative information about pragmatic competence. Pragmatic knowledge was operationalized in terms of appropriateness of speech act production, and it was evaluated by means of NSs’ ratings in a holistic appropriateness scale designed by Taguchi (2011). The instrument is a 5-point Likert scale that ranges from (1) excellent to (5) very poor. It assesses an answer to a DCT situation in terms of 3 aspects of pragmatic competence: level of politeness, level of directness, and level of formality. Table 3 shows the rating scale used.
\end{styleStandard}

\begin{stylecaption}
Table 3: Appropriateness rating scale developed by Taguchi (2011: 459)
\end{stylecaption}

\begin{center}
\tablehead{}
\begin{supertabular}{m{1.0476599in}m{3.66846in}}
\hline
\mdseries 5. Excellent &
\mdseries Almost perfectly appropriate and effective in the level of directness, politeness and formality.\\\hline
\mdseries 4. Good &
\mdseries Not perfect but adequately appropriate in the level of directness, politeness, and formality. Expressions are a little off from target-like, but pretty good.\\\hline
\mdseries 3. Fair &
\mdseries Somewhat appropriate in the level of directness, politeness, and formality. Expressions are more direct or indirect than the situation requires. (e.g., What did you speak?)\\\hline
\mdseries 2. Poor  &
\mdseries Clearly inappropriate. Expressions sound almost rude or too demanding (e.g., You say that?)\\\hline
\mdseries 1. Very poor &
\mdseries Not sure if the target speech act is performed.\\\hline
\end{supertabular}
\end{center}
\begin{styleStandard}
Five NSs were trained in the rating of pragmatic appropriateness. The training contained information about the purpose of the data collection, the coding criteria, some examples of previous studies that have used the appropriateness scale of the study (Taguchi 2011; 2013), and practice with data from the pilot study (\textit{N} = 21). Drawing from Hudson et al. (1995), who proposed the use of an appropriateness rating scale to assess pragmatic production, the NSs were instructed not to consider grammaticality. Inter-rater reliability was \textit{r }=.83. The disagreements (17\% of the data) were discussed during a meeting session and agreements were reached.
\end{styleStandard}

\begin{styleStandard}
The second type of data to code was quantitative information about students’ sociocultural adaptation experiences. Answers from the SCAS were analyzed, revealing scores that ranged from 1 – which indicated poor sociocultural adaptation – to 5 – meaning high levels of adaptation. 
\end{styleStandard}

\begin{styleStandard}
Thirdly, the content of the semi-structured interviews was analyzed by eliciting different themes that allowed for the establishment of participant profiles based on development of their sociocultural and psychological adaptation experiences. Following Schumann’s (1986) proposal of acculturation variables, comments in the interviews were coded into 5 main sociocultural themes and 4 psychological ones. Sociocultural adaptation aspects included the integration strategy adopted (which involved the development of social networks), enclosure, cohesiveness and size of the sojourning group, cultural congruence, and changes in attitude towards the US culture. Intended length of residence and social dominance were not included in the analysis as there was homogeneity across the participants in these two aspects. Psychological adaptation aspects included culture shock, language shock, motivation, and ego permeability. Additionally, we considered that the semi-structured interviews could reveal further factors that accounted for students’ adaptation experiences, especially psychological factors (Schumann 1986). Finally, acculturation profiles of students were established by discerning whether sociocultural and psychological adaptation increased or decreased from the beginning to the end of the semester. To do so, we focused on answers given during the final interview, as well as on the change in answers given across time. \ 
\end{styleStandard}

\begin{styleStandard}
Once the three types of data were coded, a next step was to analyze them to answer the two research questions of the study. Firstly, gains in pragmatic production were calculated by means of a series of Wilcoxon Signed-Ranks tests. This non-parametric test, selected given the small sample size of participants, allowed for comparisons of average scores from the rating scales in the pre-test and the post-test. Secondly, gains in sociocultural adaptation were calculated through an additional Wilcoxon Signed-Ranks test, and the relationship between sociocultural adaptation and pragmatic development was measured through the non-parametric Spearman rho correlation. To analyze the qualitative data, different individual trajectories of sociocultural and psychological adaptation were discerned, and they were compared against individual trajectories of pragmatic learning. This allowed for the presentation and interpretation of case studies that illustrate patterns of associations between acculturation and pragmatic competence.
\end{styleStandard}

\begin{listWWNumxxiileveli}
\item 
\begin{stylelsSectioni}
Results
\end{stylelsSectioni}


\setcounter{listWWNumxxiilevelii}{0}
\begin{listWWNumxxiilevelii}
\item 
\begin{stylelsSectionii}
RQ1: Does a semester of study abroad afford gains in pragmatic competence, in terms of speech act production?
\end{stylelsSectionii}

\end{listWWNumxxiilevelii}
\end{listWWNumxxiileveli}
\begin{styleStandard}
The first research question of the study asked whether students improved their pragmatic competence in the SA context, pragmatic ability being operationalized in terms of appropriateness of speech act production, measured with a 5-point scale. To determine whether there were statistical differences between pre-test and post-test pragmatic performance, Wilcoxon Signed-Ranks tests were conducted for speech act production in high-imposition situations and in low-imposition situations, both including requests and refusals. Moreover, gains in overall speech act production were calculated. Table 4 displays pre-test and post-test means (\textit{M}), standard deviations (\textit{SD}), and differences – which indicate gains – for each of the three aspects. Differences were considered significant at \textit{p} {\textless} .05.
\end{styleStandard}

\begin{stylecaption}
Table 4: Descriptive statistics of production of speech acts
\end{stylecaption}

\begin{flushleft}
\tablehead{}
\begin{supertabular}{m{1.8cm}m{1.8cm}m{1.8cm}m{1.8cm}m{1.8cm}m{1.8cm}m{1.8cm}}
\hline
 &
\multicolumn{2}{m{3.8cm}}{\bfseries Pre-test} &
\multicolumn{2}{m{3.8cm}}{\bfseries Post-test} &
\multicolumn{2}{m{3.8cm}}{\bfseries Difference}\\\hline
 &
\bfseries \textit{M} &
\bfseries \textit{SD} &
\bfseries \textit{M} &
\bfseries \textit{SD} &
\bfseries Score &
\bfseries \%\\\hline
\mdseries High-imposition situations  &
\mdseries 3.12 &
\mdseries 0.74 &
\mdseries 3.56 &
\mdseries 0.79 &
\mdseries 0.44 &
\mdseries 10.94\\
\mdseries Low-imposition situations &
\mdseries 2.80 &
\mdseries 0.63 &
\mdseries 3.35 &
\mdseries 0.64 &
\mdseries 0.56* &
\mdseries 14.06*\\\hline
\mdseries Overall production of speech acts &
\mdseries 2.96 &
\mdseries 0.47 &
\mdseries 3.45 &
\mdseries 0.55 &
\mdseries 0.50* &
\mdseries 12.50*\\\hline
\end{supertabular}
\end{flushleft}
\begin{stylelsTable}
*Significant at p {\textless} .05
\end{stylelsTable}

\begin{styleStandard}
The statistical analysis revealed that participants improved their overall pragmatic competence during the first semester of immersion in the US (\textit{Z} = 2.31; \textit{p} = .021). More particularly, they significantly improved their appropriateness in the production of requests and refusals in low-imposition situations (\textit{Z} = 2.41; \textit{p} = .016), although they did not experience statistically-significant pragmatic gains in high-imposition situations (\textit{Z} = 1.69; \textit{p} = .09). To illustrate these findings, example (3) illustrates positive gains in pragmatic production in a low-imposition situation, and example (4) shows no gains in a high-imposition context. Both examples include answers in the pre- and post-tests to request situations by the same participant, Lisa, and they also show the agreed assessment of the responses by the raters in the appropriateness rating scale.
\end{styleStandard}

\setcounter{listWWNumxxxiileveli}{0}
\begin{listWWNumxxxiileveli}
\item 
\begin{stylelsLanginfo}
You are in class and you need to write something down, but you realize you forgot your pen at home. You tell the classmate sitting next to you:
\end{stylelsLanginfo}

\end{listWWNumxxxiileveli}
\setcounter{listWWNumxxxivleveli}{0}
\begin{listWWNumxxxivleveli}
\item 
\setcounter{listWWNumxxxivlevelii}{0}
\begin{listWWNumxxxivlevelii}
\item 
\begin{stylelsLanginfo}
Pre-test: \textit{I was wondering if you have a pen I could maybe borrow}
\end{stylelsLanginfo}

\end{listWWNumxxxivlevelii}
\end{listWWNumxxxivleveli}
\begin{stylelsSourceline}
\textup{Evaluation}: (3) fair. Somewhat appropriate in the level of directness, politeness, and formality. Expressions are more direct or indirect than the situation requires.
\end{stylelsSourceline}


\begin{listWWNumxxxivleveli}
\item 
\setcounter{listWWNumxxxivlevelii}{0}
\begin{listWWNumxxxivlevelii}
\item 
\begin{stylelsLanginfo}
Post-test:\textit{ Do you have a pen I could borrow? Please?}
\end{stylelsLanginfo}

\end{listWWNumxxxivlevelii}
\end{listWWNumxxxivleveli}
\begin{stylelsSourceline}
\textup{Evaluation}: (5) excellent. Almost perfectly appropriate and effective in the level of directness, politeness and formality
\end{stylelsSourceline}


\setcounter{listWWNumxxxiileveli}{0}
\begin{listWWNumxxxiileveli}
\item 
\begin{stylelsLanginfo}
You need to ask a professor for an extension of a deadline for turning in a paper. At the end of a class session you tell him: \ 
\end{stylelsLanginfo}

\end{listWWNumxxxiileveli}
\begin{listWWNumxxxiileveli}
\item 
\setcounter{listWWNumxxxiilevelii}{0}
\begin{listWWNumxxxiilevelii}
\item 
\begin{stylelsLanginfo}
Pre-test: \textit{Sorry, could I have an extension of the deadline for the Biology paper?}
\end{stylelsLanginfo}

\end{listWWNumxxxiilevelii}
\end{listWWNumxxxiileveli}
\begin{stylelsSourceline}
\textup{Evaluation}: (3) fair. Somewhat appropriate in the level of directness, politeness, and formality. Expressions are more direct or indirect than the situation requires.
\end{stylelsSourceline}


\begin{listWWNumxxxiileveli}
\item 
\setcounter{listWWNumxxxiilevelii}{0}
\begin{listWWNumxxxiilevelii}
\item 
\begin{stylelsLanginfo}
Post-test:\textit{ }\textit{Excuse me, can I have an extension of the deadline for the paper?}
\end{stylelsLanginfo}

\end{listWWNumxxxiilevelii}
\end{listWWNumxxxiileveli}
\begin{stylelsSourceline}
\textup{Evaluation}: (3) fair. Somewhat appropriate in the level of directness, politeness, and formality. Expressions are more direct or indirect than the situation requires.
\end{stylelsSourceline}


\begin{styleStandard}
As we may observe in example (3), Lisa improved her appropriateness in requesting a pen from a friend (that is, a low-imposition situation) by using the conventional request strategy “Do you have.... I can borrow?” \ which was more appropriate than the expression “I was wondering” and the mitigator “maybe”, which are more appropriately used in high-imposition situations. In contrast, example (4) shows that she did not improve her appropriateness in requesting an extension of a deadline from a professor (that is, a high-imposition situation). Although she provided different answers in the pre- and post-tests, both answers were rated as \textit{fair}, as she could have used a more polite and indirect strategy.
\end{styleStandard}

\begin{styleStandard}
The present findings in relation to research question 1 have pointed out that the SA context is beneficial for the production of speech acts, particularly in low-imposition situations that involve conversations with friends. This suggests that during the first semester of immersion, students were probably mostly exposed to informal situations and interactions with other students rather than to formal conversations with professors. Indeed, these results are in line with previous studies that have revealed the first months of immersion in the SA context enhance production of low-imposition speech acts to a higher extent than high-imposition speech acts of requests, refusals and opinions (Taguchi 2006; 2011; 2013).
\end{styleStandard}

\setcounter{listWWNumxxiileveli}{0}
\begin{listWWNumxxiileveli}
\item 
\setcounter{listWWNumxxiilevelii}{0}
\begin{listWWNumxxiilevelii}
\item 
\begin{stylelsSectionii}
RQ2: To what extent, fi any, does students’ acculturation development influence gains in pragmatic competence?
\end{stylelsSectionii}

\end{listWWNumxxiilevelii}
\end{listWWNumxxiileveli}
\begin{styleStandard}
he second research question of the study asked to what extent, if any, students’ acculturation development was related to the gains in pragmatic competence reported above. Acculturation was quantitatively operationalized in terms of sociocultural adaptation, while a qualitative analysis accounted for both sociocultural and psychological adaptation. 
\end{styleStandard}

\begin{styleStandard}
To determine whether the participants’ experienced gains in their sociocultural adaptation, a Wilcoxon Signed-Ranks test compared pre-test adaptation scores (\textit{M} = 3.80; \textit{SD} = 0.45; Min = 3.24; Max = 4.66) with post-test ones (\textit{M} = 4.04; \textit{SD} = 0.42; Min = 3.28; Max = 4.55). Results indicated a significant improvement (\textit{Z} = 1.778; \textit{p} = .075), and therefore confirmed that a semester of SA enhanced the students’ sociocultural adaptation. 
\end{styleStandard}

\begin{styleStandard}
To examine the relationship between the reported sociocultural adaptation gains and the pragmatic gains, a Spearman rho correlation was conducted, the significance level being established at \textit{p} {\textless} .10. the analysis revealing a positive correlation (\textit{r} = 0.604, \textit{p} = 0.038). Therefore, we may hypothesize that adaptation to the SA context may have played a key role in learners’ improvement of their ability to produce speech acts. Next, post-hoc Spearman rho correlation tests between adaptation gains and gains in speech act production in high-imposition and in low-imposition situations were calculated. The results revealed that sociocultural adaptation was significantly related to speech act production in high-imposition situations (\textit{r} = 0.527; \textit{p} = .079), but it was unrelated to speech act production in low-imposition situations (\textit{r} = 424; \textit{p} = .169). This finding suggests that students who improved their sociocultural adaptation during the semester were also likely to improve their pragmatic ability in high-imposition formal situations that involve interacting with a professor. It could also imply that students who improved their pragmatic ability in such situations were also likely to improve their adaptation to the new context. Nevertheless, gains in pragmatic ability in low imposition situations such as interacting with friends were unrelated to sociocultural adaptation. 
\end{styleStandard}

\begin{styleStandard}
Next, the association between acculturation and pragmatic development was explored qualitatively. The analysis focused on students’ acculturation development, which was operationalized in terms of sociocultural and psychological adaptation factors, as proposed by Schumann (1978). The participants’ answers in the interviews at the beginning and at the end of the semester were analyzed, and individual profiles in terms of sociocultural and psychological adaptation were established. Table 5 illustrates the 12 individual profiles by showing descriptive data about their pragmatic gains (in high- and low-imposition situations) – expressed in percentages –, their sociocultural adaptation gains (expressed in gain scores from the analysis of SCAS answers), and overall increase or decrease in sociocultural and psychological adaptation. The ordering of participants in the table is hierarchical, running from the student with the highest positive gains in overall pragmatic competence to the one with highest negative gains.
\end{styleStandard}

\begin{stylecaption}
Table 5: Descriptive information of the development of pragmatic competence and acculturation by 12 informants
\end{stylecaption}

\begin{flushleft}
\tablehead{}
\begin{supertabular}{m{1.8cm}m{1.8cm}m{1.8cm}m{1.8cm}m{1.8cm}m{1.8cm}m{1.8cm}m{1.8cm}m{1.8cm}m{1.8cm}m{1.8cm}}
\hline
 &
 &
\multicolumn{5}{m{9.8cm}}{\bfseries Pragmatic competence} &
\multicolumn{4}{m{7.8cm}}{\bfseries Acculturation}\\\hline
 &
\bfseries Participant &
\bfseries High imposition &
\bfseries Low imposition &
\bfseries Overall &
\multicolumn{3}{m{5.8cm}}{\bfseries \ \ \ \ Sociocultural \ \ \ \ \ \ \ \ \ \ \ } &
\multicolumn{2}{m{3.8cm}}{\bfseries Psychological} &
\\\hhline{----------~}
 &
{\mdseries David}

{\mdseries Emma}

{\mdseries Mike}

{\mdseries Sean}

{\mdseries Lisa}

{\mdseries Jeff}

{\mdseries William}

{\mdseries Steven}

{\mdseries Jason}

{\mdseries Ethan}

{\mdseries Michelle}

\mdseries Mark &
{\mdseries 2}

{\mdseries 1}

{\mdseries 1.75}

{\mdseries 0.5}

{\mdseries 0.5}

{\mdseries {}-0.25}

{\mdseries 0}

{\mdseries 0.75}

{\mdseries 0}

{\mdseries {}-0.25}

{\mdseries {}-0.5}

\mdseries {}-0.25 &
{\mdseries 1.75}

{\mdseries 1.75}

{\mdseries 0.5}

{\mdseries 1.25}

{\mdseries 0.5}

{\mdseries 1.25}

{\mdseries 0.75}

{\mdseries 0}

{\mdseries {}-0.25}

{\mdseries {}-0.25}

{\mdseries 0}

\mdseries {}-0.5 &
{\mdseries 1.9}

{\mdseries 1.4}

{\mdseries 1.2}

{\mdseries 0.8}

{\mdseries 0.5}

{\mdseries 0.5}

{\mdseries 0.4}

{\mdseries 0.4}

{\mdseries {}-0.1}

{\mdseries {}-0.2}

{\mdseries {}-0.3}

\mdseries {}-0.4 &
{\mdseries 0.97}

{\mdseries 0.89}

{\mdseries {}-0.04}

{\mdseries 0.48}

{\mdseries 0.6}

{\mdseries 0.41}

{\mdseries {}-0.18}

{\mdseries {}-0.01}

{\mdseries {}-0.08}

{\mdseries 0.07}

{\mdseries 0.03}

\mdseries {}-0.27 &
\multicolumn{2}{m{3.8cm}}{{\mdseries Increase}

{\mdseries Increase}

{\mdseries Decrease}

{\mdseries Increase}

{\mdseries Increase}

{\mdseries Increase}

{\mdseries Decrease}

{\mdseries Decrease}

{\mdseries Decrease}

{\mdseries Increase}

{\mdseries Increase}

\mdseries Decrease} &
{\mdseries Increase}

{\mdseries Increase}

{\mdseries Increase}

{\mdseries Decrease}

{\mdseries Increase}

{\mdseries Increase}

{\mdseries Increase}

{\mdseries Increase}

{\mdseries Decrease}

{\mdseries Decrease}

{\mdseries Decrease}

\mdseries Decrease &
 &
\\\hhline{---------~~}
 &
\mdseries AVERAGE &
\mdseries 0.44 &
\mdseries 0.56 &
\mdseries 0.5 &
\mdseries 0.24 &
\multicolumn{2}{m{3.8cm}}{} &
 &
 &
\\\hhline{---------~~}
\end{supertabular}
\end{flushleft}
\begin{styleStandard}
Table 5 shows that there were diverse individual trajectories of pragmatic learning and acculturation. Taking all the profiles into account, 3 patterns can be observed, which will guide the presentation of the qualitative findings in the next sections. Pattern 1 includes informants whose gains – either positive or negative – in pragmatic competence correspond with their gains in both overall sociocultural and psychological adaptation. Pattern 2 refers to participants whose pragmatic gains only correspond with their psychological adaptation gains. Finally, pattern 3 includes one informant whose gains in speech act production only correspond with his gains in sociocultural adaptation.
\end{styleStandard}

\begin{listWWNumxxiileveli}
\item 
\begin{stylelsSectioniii}
Pattern 1: Interplay of sociocultural and psychological adaptation and pragmatic gains
\end{stylelsSectioniii}

\end{listWWNumxxiileveli}
\begin{styleStandard}
The first category includes informants who have shown either positive or negative gains in pragmatic production and in both sociocultural and psychological outcomes of acculturation. David, Emma, Lisa and Jeff are gainers in this respect, while Jason, Mark and Michelle are non-gainers.
\end{styleStandard}

\begin{styleStandard}
On the one hand, David, Jeff, Emma and Lisa showed similar developmental paths in the three general aspects: gains in pragmatic competence, gains in sociocultural adaptation, and gains in psychological adaptation. Their sociocultural adaptation was mainly determined by the successful integration strategy adopted; that of assimilation of US sociocultural values. They were the only participants that consciously tried to interact beyond their home-country cohesive group (in this case, Brazilians and Spanish) and to assimilate the host values. David and Lisa’s integration can be largely attributed to making close US friends and, in the case of David, finding a NS girlfriend. Jeff and Emma’s successful integration and therefore pragmatic development were mainly due to their enrollment in clubs – theatre club in the case of Jeff, and a music band and volunteering program by Emma. At the same time, the four improved their psychological adaptation thanks to social support from their home-country peers. In this sense, both David and Jeff expressed that living with their “Brazilian family” was what made the experience great. Similarly, Lisa expressed that although she tried to spend most of her time with her US roommates in order to integrate into the community, she also felt she had a Spanish family, and indeed all of the Spanish students developed a close relationship. In the case of Emma, apart from social support, her improvement in psychological adaptation was primarily attributed to a reduction of language shock, which was a consequence of her integration into the L2 community.
\end{styleStandard}

\begin{styleStandard}
On the other hand, Jason, Mark and Michelle decreased their scores of pragmatic ability, as well as in sociocultural and psychological adaptation. In the three cases, the students were not able to integrate into the L2 society, and instead preserved their sociocultural values over the stay. This unsuccessful integration may be due to several main reasons: an increase in language shock and a consequent change in personality in the case of Jason, academic pressure in the case of Mark, and ego permeability (also a problem related to personality) in the case of Michelle. Jason developed a wide friendship network with his Brazilian peers and did not gain confidence to interact with Americans. In the post-test interview, he expressed that his anxiety to speak in English had increased, and that he was disappointed because he expected to improve his speaking ability thanks to coming to the US. Moreover, it seems that his language shock made him acquire a shy identity when using English, as he claimed to be extroverted and social with his Brazilian friends, but not able to interact with US students. Similarly, Michelle was aware that she was a shy person, and according to her, her introverted personality prevented her from interacting with NSs and from learning about their culture. As for Mark, his strong motivation to integrate into the society and to practice his English was evident: he enrolled in university clubs and also got in contact with NSs. Nevertheless, he regretted having a lot of pressure to pass a TOEFL test at the end of the semester so as to be able to continue in the SA program. As a consequence, both Michelle and Mark also increased their language shock and at the end of the semester reported being scared or ashamed of using their English at times.
\end{styleStandard}

\begin{listWWNumxxiileveli}
\item 
\begin{stylelsSectioniii}
Pattern 2: Interplay of psychological adaptation and pragmatic gains
\end{stylelsSectioniii}

\end{listWWNumxxiileveli}
\begin{styleStandard}
The second case involves participants whose pragmatic gains corresponded with their psychological adaptation gains, but not with their sociocultural ones. William, Mike and Steven experienced positive gains in speech act production and in psychological adaptation, while Ethan showed negative pragmatic development and a decrease in his psychological adaptation.
\end{styleStandard}

\begin{styleStandard}
William and Mike experienced similar sociocultural and psychological adaptation paths. They reported having a phenomenal semester thanks to the Brazilian friends they made. Therefore, social support from their home-country peers seemed to have enhanced their well-being and their psychological adaptation. Nevertheless, limiting their contact to Brazilians made their sociocultural adaptation decrease. This situation was more striking in William’s case, who openly admitted not making contacts outside his Brazilian peer group. William, however, claimed that he learned a great deal of English since sometimes the Brazilians spoke in English among themselves, so he attributes his language improvement to the meta-talk resulting from correcting home-country peers among each other. Mike did integrate to some extent into the L2 community, and apart from Brazilian colleagues, he made some friends from other cultures, mainly international students. In the case of Steven, his psychological adaptation increased during the semester thanks to making friends from different nationalities, and as a result of this positive experience his attitude towards the US improved. Nevertheless, he did not experience immersion in the US culture or made any NS friends. Steven’s increased well-being could explain his moderate but positive pragmatic gains. Additionally, the fact that he was mainly concerned with improving his academic English (a reflection of his instrumental motivation) could account for his development of speech act production in high-imposition situations. 
\end{styleStandard}

\begin{styleStandard}
Unlike William, Mike and Steven, Ethan did show gains in his sociocultural adaptation, mainly due to a progression towards a more positive attitude regarding US culture. Nevertheless, he experienced negative gains in his pragmatic competence as well as in his psychological adaptation. His decrease in psychological adaptation was mainly due to his lack of ego permeability. Ethan described himself as an introverted person whose preferred plan for a Saturday evening during the stay abroad was to play videogames with a Spanish peer, who became his best friend. He also admitted not trying very hard to integrate with Americans since his main motivation in the program was to improve academically, not socially or personally.
\end{styleStandard}

\begin{listWWNumxxiileveli}
\item 
\begin{stylelsSectioniii}
Pattern 3: Interplay of sociocultural adaptation and pragmatic gains
\end{stylelsSectioniii}

\end{listWWNumxxiileveli}
\begin{styleStandard}
The third category includes one student whose pragmatic gains corresponded to his sociocultural adaptation development, but not with his psychological one. This is the case of Sean, whose psychological adaptation decreased over the semester because his culture shock increased. His inability to cope with some cultural differences – particularly with the US custom of keeping dogs indoors and not removing shoes inside the house – led him to have arguments with his American roommates and change his living arrangements. Sean finally felt well-adapted to the setting when, by the end of the semester, he changed from having two US roommates to living with two international students, one from Saudi Arabia and one from Thailand. He particularly felt his English improved more when sharing accommodation with international students since they interacted frequently. Sean’s sociocultural adaptation improved during the semester abroad, which was mainly due to the fact that he felt more integrated into the L2 community once he had made real friends. Even if his friends were from other nationalities, going out with them gave him confidence to get closer to NSs.
\end{styleStandard}

\begin{styleStandard}
Summing up, a qualitative exploration of individual trajectories seems to indicate that gains in appropriateness of speech act production were somewhat related to overall acculturation gains. More particularly, the analysis highlighted the key role of the social variables of integration and of academic pressure, as well as the importance of the affective variables of social support from home-country peers and language shock in shaping the process of acculturation. We may also hypothesize that psychological adaptation might lead to pragmatic gains to a higher extent than sociocultural adaptation, as only in one case (Sean) did only sociocultural gains correspond with pragmatic gains, as opposed to the 4 cases in which only psychological adaptation was associated with pragmatic development (William, Mike, Steven and Ethan).
\end{styleStandard}

\begin{listWWNumxxiileveli}
\item 
\begin{stylelsSectioni}
Discussion
\end{stylelsSectioni}

\end{listWWNumxxiileveli}
\begin{styleStandard}
The current study investigated whether acculturation was related to the development of pragmatic competence in the SA context. More particularly, it sought to discern (1) whether students developed their appropriateness of pragmatic production during a semester of study in the US, and (2) whether and how their sociocultural and psychological adaptation were related to the reported pragmatic gains. The objectives of the study were formulated drawing on Schumann’s (1978) proposal that the degree to which an individual acculturates socially and affectively to the L2 setting would determine the extent to which he/she learns the L2.
\end{styleStandard}

\begin{styleStandard}
Firstly, the results corroborate previous longitudinal ILP studies that have reported that, despite the advantage of the SA context for pragmatic development, the process of learning pragmatic competence is variable and non-linear, as it is determined by different factors (e.g., Barron 2003; Félix-Brasdefer 2004; Taguchi \& Roever 2017). Overall, the findings revealed that during the first 4 months of immersion in the US, participants improved their ability to formulate requests and refusals, as measured on a DCT. Nevertheless, this improvement was influenced by type of situation, since learners showed higher gains in appropriateness of speech act production in low-imposition situations – that is, in interaction with friends – rather than in in high-imposition ones, which involved conversations with professors. This finding is, indeed, in line with a series of longitudinal investigations by Taguchi (2011; 2013) reporting that the development of high-imposition English speech acts (requests, refusals and opinions) takes place at later stages of the L2 immersion. It is hypothesized that low-imposition scenarios presented in this study (see Table 2) were encountered on campus more frequently than the high-imposition ones. This finding thus highlights the particularity of SA as an optimal setting for L2 learning given the opportunities it provides for interaction outside of class in different situations and for exposure to contextualized and authentic input (Taguchi 2015a).
\end{styleStandard}

\begin{styleStandard}
Moreover, the study has revealed that pragmatic development was related in various ways to students’ acculturation experiences. Therefore, it provides support for Schumann’s (1978, 1986) Acculturation model of L2 acquisition, and corroborates findings from previous studies that have provided empirical evidence for the relationship between acculturation and pragmatic acquisition (Schmidt 1983; Dörnyei et al. 2004). A quantitative analysis showed that sociocultural adaptation and pragmatic gains were significantly associated. Moreover, a qualitative exploration of individual trajectories of pragmatic learning and of acculturation showed that different factors related to sociocultural adaptation and psychological adaptation contributed to adaptation to the L2 setting and to subsequent learning of how to use the L2 appropriately as a function of interlocutor and situation. On the one hand, acculturation was determined by social variables that mainly included integration and academic pressure. On the other hand, it was influenced by affective factors, primarily social support from home-country peers and language shock. The most successful students in pragmatic learning were those who were able to integrate into the L2 community, and also those whose psychological adaptation increased thanks to the support from their home-country peers. In contrast, unsuccessful students were not able to integrate into the US society mainly because of a strong language shock and a high level of academic pressure. 
\end{styleStandard}

\begin{styleStandard}
Although academic pressure and social support from home-country peers were not a primary focus on this investigation, they were found to be related to acculturation in the US. This finding has two main implications. Firstly, it supports Schumann’s (1978) assertion that acculturation, rather than being a direct cause of L2 acquisition, is one of the main factors enhancing L2 learning. Secondly, it highlights the need to revise Schumann’s (1978) framework so as to account for language learning in the current era of globalization, which has resulted in a dramatic increase of student mobility worldwide (see Mitchell et al. 2015: Introduction; 2017: Chapter 7). For instance, nowadays it is common to have big groups of students from the same nationality, and even from the same home university, at the same SA site, which makes social support from home-country peers an inevitable element to take into account when investigating the SA setting. 
\end{styleStandard}

\begin{styleStandard}
All in all, findings from this study provide new insights into the investigation of outcomes of SA programs by reporting the relationship between social and psychological adaptation for the development of pragmatic ability. Therefore, the study has addressed the recent call for the need to investigate cultural and linguistic aspects of SA so as to have a more comprehensive understanding of the SA experience (Taguchi 2015b). To that end, future studies could draw from Schumann’s (1978) framework of Acculturation.
\end{styleStandard}

\begin{listWWNumxxiileveli}
\item 
\begin{stylelsSectioni}
Conclusion
\end{stylelsSectioni}

\end{listWWNumxxiileveli}
\begin{styleStandard}
This investigation focused on the interplay of acculturation and the acquisition of pragmatic competence in the SA context. More particularly it discussed ways in which students’ sociocultural and psychological adaptation during the first 4 months of immersion could be related to their ability to formulate requests and refusals appropriate to the given situation.
\end{styleStandard}

\begin{styleStandard}
The main limitations of the study concern the relatively small sample size, and the merely qualitative assessment of psychological adaptation. A qualitative analysis of 12 case studies was however the most suitable methodology for the purpose of the study since, as different scholars have noted (e.g., Dörnyei et al. 2004; Taguchi 2011), it allows for an in-depth exploration of the interplay between contextual factors and learners’ individual differences. Nevertheless, future research is encouraged to include a higher number of participants, as well as to administer quantitative measures of psychological adaptation or overall acculturation.
\end{styleStandard}

\begin{styleStandard}
Ultimately, the study makes a notable contribution in the field of ILP by bringing to the fore the relevance of the factor of acculturation. Moreover, the findings have important implications for the design of mobility programs, as they highlight the need to maximize students’ immersion experiences during SA programs, both at the social and at the affective level, so as to enhance their ability to use the L2 appropriately in the new sociocultural setting. 
\end{styleStandard}

\begin{stylelsSectioni}
References
\end{stylelsSectioni}


\begin{styleStandard}
Alcón-Soler, Eva. (ed.). 2008. \textit{Learning how to request in an instructed language learning context}. Berlin: Peter Lang.
\end{styleStandard}


\begin{styleStandard}
Alcón-Soler, Eva. 2014. Pragmatic learning and study abroad: Effects of instruction and length of stay. \textit{System} 48. 62-74.
\end{styleStandard}


\begin{styleStandard}
Arends-Tóth, Judith, \& Van de Vijver, Fons J. R. 2006. Issues in conceptualization and assessment of acculturation. In Bornstein, Marc H. \& Cote, Linda R. (eds.), \textit{Acculturation and parent-child relationships: Measurement and development}, 33-62. Mahwah, NJ:\textit{ }Lawrence Erlbaum.
\end{styleStandard}


\begin{styleStandard}
Bardovi-Harlig, Kathleen \& Bastos, Maria-Thereza. 2011. Proficiency, length of stay, and intensity of interaction, and the acquisition of conventional expressions in L2 pragmatics. \textit{Intercultural Pragmatics} 8(3). 347-384. 
\end{styleStandard}


\begin{styleStandard}
Bardovi-Harlig, Kathleen \& Rose, Marda \& Nickels, Edelmira. 2008. The influence of first language and level of development in the use of conventional expressions of thanking, apologizing and refusing. In Bowles, Melissa \& Foote, Rebecca \& Perpiñán, Silvia \& Bhatt, Rakesh (eds.), \textit{Selected proceedings of the 2007 second language research forum}, 113-130. Somerville, MA: Cascadilla Proceedings Project.
\end{styleStandard}


\begin{styleStandard}
Barron, Anne. 2003. \textit{Acquisition in interlanguage pragmatics: Learning how to do things with words in a study abroad context}. Amsterdam: Benjamins.
\end{styleStandard}


\begin{styleStandard}
Beebe, Leslie M. \& Giles, Howard. 1984. Speech accommodation theories: A discussion in terms of second language acquisition. \textit{International Journal of the Sociology of Language} 46. 5-32.
\end{styleStandard}


\begin{styleStandard}
Bella, Spyridoula. 2011. Mitigation and politeness in Greek invitation refusals: Effects of length of residence in the target community and intensity of interaction on non-native speakers’ performance. \textit{Journal of Pragmatics} 43. 1718–1740. 
\end{styleStandard}


\begin{styleStandard}
Berry, John W. \& Sam, David L. (1997). Acculturation and adaptation. In Berry, John W., Segall, Marshall \& Kagitcibasi, Cigdem (eds.), \textit{Handbook of cross-cultural psychology} (Vol 3), 291–326). Needham Heights, MA: Viacom.
\end{styleStandard}


\begin{styleStandard}
Berry, John W. 2003. Conceptual approaches to acculturation. In Chun, Kevin M. \& Balls Organista, Pamela \& Marín, Gerardo (eds.), \textit{Acculturation: Advances in theory, measurement and applied research}, 17\textbf{–}37. Washington, DC: American Psychological Association. 
\end{styleStandard}


\begin{styleStandard}
Brown, Penelope \& Levinson, Stephen C. 1987. \textit{Politeness: Some universals in language use}. Cambridge: Cambridge University Press.
\end{styleStandard}


\begin{styleStandard}
Celenk, Ozgur \& Van de Vijver, Fons J.R. 2011. Assessment of acculturation: Issues and overview of measures. \textit{Online Readings in Psychology and Culture} 8(1). 3-22.
\end{styleStandard}


\begin{styleStandard}
Dörnyei, Zoltán \& Durow, Valerie \& Zahran, Khawla. 2004. \ Individual differences and their effects on formulaic secquence acquisition. In Schmitt, Norbert (ed.), \textit{Formulaic sequences: Acquisition, processing and use}, 87-106. Amsterdam: Benjamins.
\end{styleStandard}


\begin{styleStandard}
Ellis, Rod. 1994. \textit{The study of second language acquisition}. Oxford: Oxford University Press.
\end{styleStandard}


\begin{styleStandard}
Eslami, Zohreh Rasekh \& Ahn, Soo-Jin. 2014. Motivation, amount of interaction, length of residence, and ESL learners’ pragmatic competence. \textit{Applied Research on English Language} 3(1). 9-28.
\end{styleStandard}


\begin{styleStandard}
Félix-Brasdefer, J. César. 2004. Interlanguage refusals: Linguistic politeness and length of residence in the target community. \textit{Language Learning} 54(4). 587-653.
\end{styleStandard}


\begin{styleStandard}
Félix-Brasdefer, J. César \& Hasler-Baker, Maria. 2017. Elicited data. In Barron, Anne \& Grundy, Peter \& Steen, Gerard \& Gu, Yueguo (eds.),~\textit{The Routledge handbook of pragmatics}, 27–38. London: Routledge.
\end{styleStandard}


\begin{styleStandard}
García, Carmen. 1989. Disagreeing and requesting by Americans and Venezuelans. \textit{Linguistics and Education} 1. 299- 322.
\end{styleStandard}


\begin{styleStandard}
Gardner, Richard C. \& Lalonde, Richard N. \& Pierson, Robert. 1983. The socio-educational model of second language acquisition: An investigation using LISREL causal modeling. \textit{Journal of Language and Social Psychology} 2. 1-15.
\end{styleStandard}


\begin{styleStandard}
Hansen, Doris. 1995. A study of the effect of the acculturation model on second language acquisition. In Eckman, Fred R. \& Highland, Diane \& Lee, Peter \& Mileham, Jean \& Weber, Rita R. (eds.), \textit{Second language acquisition theory and pedagogy}, 305-316. Hillsdale, NJ: Lawrence Erlbaum.
\end{styleStandard}


\begin{styleStandard}
Hassall, Timothy. 2006. Learning to take leave in social conversations: A diary study. In DuFon, Margaret A. \& Churchill, Eton (eds.),~\textit{Language learners in study abroad contexts, }31-58. Clevedon, UK: Multilingual Matters.
\end{styleStandard}


\begin{styleStandard}
Hudson, Thom \& Detmer, Emily \& Brown, James D. 1995. \textit{Developing prototypic measures of cross-cultural pragmatics} (Technical Report No.7). Honolulu, HI: University of Hawaii at Manoa, Second Language Teaching \& Curriculum Center.
\end{styleStandard}


\begin{styleStandard}
Jiang, Mei \& Green, Raymond J. \& Henley, Tracy B. \& Masten, William G. 2009. Acculturation in relation to the acquisition of a second language, \textit{Journal of Multilingual and Multicultural Development} 30(6). 481-492.
\end{styleStandard}


\begin{styleStandard}
Kasper, Gabriele \& Rose, Kenneth R. 2002. \textit{Pragmatic development in a second language}. Oxford, UK: Blackwell.
\end{styleStandard}


\begin{styleStandard}
Krashen, Stephen. 1985. \textit{The Input Hypothesis: Issues and implications}. Beverly Hills, CA: Laredo Publishing Company.
\end{styleStandard}


\begin{styleStandard}
Lybeck, Karen. 2002. Cultural identification and second language pronunciation of Americans in Norway. \textit{The Modern Language Journal }86. 174-91.
\end{styleStandard}


\begin{styleStandard}
Martínez-Flor, Alicia \& Usó-Juan, Esther. 2011. Research methodologies in pragmatics: eliciting refusals to requests. \textit{Estudios de Lingüística Inglesa Aplicada} 1. 47-87.
\end{styleStandard}


\begin{styleStandard}
Mitchell, Rosamond \& Tracy-Ventura, Nicole \& McManus, Kevin (eds.). 2015. Social interaction, identity and language learning during residence abroad.~\textit{Eurosla Monographs Series} 4. Amsterdam: The European Second Language Association.
\end{styleStandard}


\begin{styleStandard}
Mitchell, Rosamond \& Tracy-Ventura, Nicole \& McManus, Kevin. 2017.~\textit{Identity, social relationships and language learning during residence abroad.~}London: Routledge.
\end{styleStandard}


\begin{styleStandard}
Redfield, Robert \& Linton, Ralph \& Herskovits, Melville. 1936. Memorandum on the study of acculturation. \textit{American Anthropologist} 38. 149–152.
\end{styleStandard}


\begin{styleStandard}
Schmidt, Richard W. 1983. Interaction acculturation, and the acquisition of communicative competence: A case study of an adult. In Wolfson, Nessa \& Judd, Elliot (eds.), \textit{Sociolinguistics and language acquisition}, 137–174. Rowley, MA: Newbury House.
\end{styleStandard}


\begin{styleStandard}
Schmitt, Norbert \& Dörnyei, Zoltán \& Adolphs, Svenja \& Durow, Valerie. 2004. Knowledge and acquisition of formulaic sequences. In Schmitt, Norbert (ed.),\textit{ Formulaic sequences: Acquisition, processing and use,} 55-86. Amsterdam: Benjamins.
\end{styleStandard}


\begin{styleStandard}
Schumann, John H. 1978. \textit{The pidginization process: A model for second language acquisition.} Rowley, MA: Newbury House. 
\end{styleStandard}


\begin{styleStandard}
Schumann, John H. 1986. Research on the acculturation model for second language acquisition. \textit{Journal of Multilingual and Multicultural Development }7(5). 379-392.
\end{styleStandard}


\begin{styleStandard}
Siegal, Mary. 1995. Individual differences and study abroad: Women learning Japanese in Japan. In Freed, Barbara (ed.), \textit{Second language acquisition in a study abroad context}, 225-244. Amsterdam: Benjamins.
\end{styleStandard}


\begin{styleStandard}
Sykes, Julie. 2017. Technologies for teaching and learning intercultural competence and interlanguage pragmatics. In Sauro, Shannon \& Chapelle, Carol (eds.), \textit{The} \textit{handbook of technology and second language teaching and learning,} 118-133. Hoboken, NJ: John Wiley.
\end{styleStandard}


\begin{styleStandard}
Taguchi, Naoko. 2006. Analysis of appropriateness in a speech act of request in L2 English.~\textit{Pragmatics} 16. 513-535.
\end{styleStandard}


\begin{styleStandard}
Taguchi, Naoko. 2011. Pragmatic development as a dynamic, complex process: General patterns and case histories. \textit{The Modern Language Journal} 95(4). 605-627.
\end{styleStandard}


\begin{styleStandard}
Taguchi, Naoko. 2013. Individual differences and development of speech act production. \textit{Applied Research on English Language} 2(2). 1-16.
\end{styleStandard}


\begin{styleStandard}
Taguchi, Naoko. \ 2015a. Contextually speaking: A survey of pragmatic learning abroad, in class, and online. \textit{System} 48. 3-20.
\end{styleStandard}


\begin{styleStandard}
Taguchi, Naoko. 2015b. Cross-cultural adaptability and development of speech act production in study abroad.\textit{ International Journal of Applied Linguistics} 25(3). 343-365.
\end{styleStandard}


\begin{styleStandard}
Taguchi, Naoko. \ 2017. Interlanguage pragmatics: A historical sketch and future directions. In Barron, Anne \& Grundy, Peter \& Steen, Gerard \& Gu, Yueguo (eds.),~\textit{The Routledge handbook of pragmatics}, 153-167. London: Routledge.
\end{styleStandard}


\begin{styleStandard}
Taguchi, Naoko \& Roever, Carsten. 2017. \textit{Second language pragmatics}. Oxford: Oxford University Press.
\end{styleStandard}


\begin{styleStandard}
Taguchi, Naoko \& Xiao, Feng \& Li, Shuai. 2016. Effects of intercultural competence and social contact on speech act production in a Chinese study abroad context. \textit{The Modern Language Journal} 100(4). 1-22.
\end{styleStandard}


\begin{styleStandard}
Ward, Colleen \& Kennedy, Antony. 1999. The measurement of sociocultural adaptation. \textit{International Journal of Intercultural Relations} 23. 659-677.
\end{styleStandard}


\begin{styleStandard}
Xiao, Feng. 2015. Proficiency effect on L2 pragmatic competence. \textit{Studies in Second Language Learning and Teaching} 5(4). 557-581.
\end{styleStandard}


\begin{styleStandard}
Zaker, Alireza. 2016. The acculturation model of second language acquisition: Inspecting weaknesses and strengths. \textit{Indonesian EFL Journal }2(2). 80-87.
\end{styleStandard}

\end{document}
