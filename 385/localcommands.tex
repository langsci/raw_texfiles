\AffiliationsWithoutIndexing{}

\babelfont[armenian]{rm}[Scale=0.8, Script=Armenian]{DejaVuSans.ttf}

\newfontfamily\hy[Scale = 0.8, Script = Armenian]{DejaVuSans.ttf}
\newcommand{\armenian}[1]{{\hy #1}}
\newcommand{\zzzzzzzz}[1]{{\hy #1}}  % this is to force armenian bibliographies to be last + ordered in the bibliography

\newfontfamily\hyg[Scale=MatchLowercase]{NotoSerifArmenian-Regular.ttf}
\newcommand{\armeniang}[1]{{\hyg #1}} % for rare dialectological symbols

% \newtheorem{adjarianpage}{{Original page number}}
\newcounter{adjarianpage}
\newenvironment{adjarianpage}{

\noindent\centering\hrulefill\footnotesize~original page number \refstepcounter{adjarianpage}\theadjarianpage~\hrulefill}{}
\newcommand{\adjarianpaperpage}[1]{\footnotesize{---Original page number {#1}---}}

\newfontfamily\arabicfont[Script=Arabic,ItalicFont=*,Scale=1.4]{ScheherazadeRegOT_Jazm.ttf}
\newcommand{\textarab}[1]{{\RL{\arabicfont #1}}}


% conrefs
\newcommand{\translatorHD}[1]{[[\textit{#1}]]}

% conrefs
\newcommand{\paradigmExplanation}{%
	\translatorHD{Adjarian placed the entire paradigm of this verb into a single table.  We break it down with a morpheme segmentation and gloss. We contrast the dialectal data against SEA and/or SWA. The morpheme segmentation is my own, based on examining the entire paradigm and contrasting against SEA and/or SWA.}
 
	\translatorHD{Note that this verb is treated as the default type of verb. It is the reflex of the conjugation class that had a theme vowel /e/ in Classical Armenian. This conjugation class is also the default class in SEA and SWA. Philological work calls it the first class; a more mnemonic name is  the E-Class.}}

\newcommand{\paradigmExplanationAClass}{%
	\translatorHD{Adjarian placed the entire paradigm of this verb into a single table.  We break it down with a morpheme segmentation and gloss. We contrast against SEA and/or SWA. The morpheme segmentation is my own, based on examining the entire paradigm and contrasting against SEA and/or SWA.}
 
	\translatorHD{This verb is the reflex of the conjugation class that had a theme vowel /ɑ/ in Classical Armenian. Philological work calls it the third class; a more mnemonic name is  the A-Class. }}

\newcommand{\litoverview}{%
	\translatorHD{I do not translate or give bibliography entries for these sources because they are rather difficult to systematically track down. Adjarian does not usually state whether these manuscripts \textit{analyze} the dialect (such as a grammar or dictionary) or  simply \textit{use} the dialect (such as a play script).}
}


\newcommand{\sampleoverview}{%
	\translatorHD{I do not translate, gloss, or re-transcribe the text samples. Some of them are given in orthography, and not the Armenian dialectological transcription. Because I do not speak this dialect, then I cannot understand the text samples well enough to translate or annotate. In some cases, the printed letter was unclear so I rendered it as `X'.}}


\newcommand{\pl}{\textsc{pl}}
\newcommand{\dist}{\textsc{dist}}
\newcommand{\sg}{\textsc{sg}}
\newcommand{\nom}{\textsc{nom}}
\newcommand{\acc}{\textsc{acc}}
\newcommand{\gen}{\textsc{gen}}
\newcommand{\dat}{\textsc{dat}}
\newcommand{\abl}{\textsc{abl}}
\newcommand{\ins}{\textsc{ins}}
\newcommand{\locgloss}{\textsc{loc}}
\newcommand{\defgloss}{\textsc{def}}
\newcommand{\indf}{\textsc{indf}}
\newcommand{\obl}{\textsc{obl}}
\newcommand{\plposs}{\textsc{pl.poss}}
\newcommand{\prs}{\textsc{prs}}
\newcommand{\echo}{\textsc{echo}}
\newcommand{\pst}{\textsc{pst}}
\newcommand{\impf}{\textsc{impf}}
\newcommand{\perfect}{\textsc{perf}}
\newcommand{\perfective}{\textsc{pfv}}
\newcommand{\imp}{\textsc{imp}}
\newcommand{\opt}{\textsc{opt}}
\newcommand{\poss}{\textsc{poss}}
\newcommand{\infgloss}{\textsc{inf}}
\newcommand{\ind}{\textsc{ind}}
\newcommand{\impfcvb}{\textsc{impf.cvb}}
\newcommand{\perfcvb}{\textsc{perf.cvb}}
\newcommand{\aux}{\textsc{aux}}
\newcommand{\thgloss}{\textsc{th}}
\newcommand{\aor}{\textsc{aor}}
\newcommand{\deb}{\textsc{deb}}
\newcommand{\caus}{\textsc{caus}}
\newcommand{\pass}{\textsc{pass}}
\newcommand{\agr}{\textsc{agr}}
\newcommand{\fut}{\textsc{fut}}
\newcommand{\prox}{\textsc{prox}}
\newcommand{\proh}{\textsc{proh}}
\newcommand{\prog}{\textsc{prog}}
\newcommand{\cn}{\textsc{cn}}
\newcommand{\sbjv}{\textsc{sbjv}}
\newcommand{\neggloss}{\textsc{neg}}
\newcommand{\sptcp}{\textsc{sptcp}}
\newcommand{\q}{\textsc{q}}
\newcommand{\eptcp}{\textsc{eptcp}}
\newcommand{\rptcp}{\textsc{rptcp}}
\newcommand{\futcvb}{\textsc{fut.cvb}}
\newcommand{\vx}{\textsc{vx}}
\newcommand{\nx}{\textsc{nx}}
\newcommand{\inch}{\textsc{inch}}
\newcommand{\ptcp}{\textsc{ptcp}}
\newcommand{\lvgloss}{\textsc{lv}}
\newcommand{\kase}{\textsc{case}}
\newcommand{\narr}{\textsc{narr}}
\newcommand{\possFsg}{\textsc{poss.1sg}}	%possessive 1sg
\newcommand{\possSsg}{\textsc{poss.2sg}}	%possessive 1sg
\newcommand{\dimgloss}{\textsc{dim}}
\newcommand{\pro}{\textsc{pro}}


\newcommand{\adjbox}[5]{\noindent\parbox[t]{3cm}{\raggedright\small#1}\parbox[t]{3cm}{\raggedright\small#2}\parbox[t]{2.5cm}{\raggedright\small#3}\parbox[t]{2.5cm}{\raggedright\small#4}\parbox[t]{1.5cm}{\raggedright\small#5}\smallskip}
