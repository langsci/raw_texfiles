\author{Hossep Dolatian}
\title{Adjarian's {\itshape Armenian dialectology} (1911)}
\subtitle{Translation and commentary}

\renewcommand{\lsSeriesNumber}{4}
\renewcommand{\lsSeries}{loc}

\BackBody{Armenian is an Indo-European language that boasts a rich linguistic landscape comprising Classical Armenian (CA), Standard Western Armenian (SWA or WA), Standard Eastern Armenian (SEA or EA), and numerous non-standard dialects, many of which were tragically lost due to the Armenian Genocide. This book is an English translation and commentary of Hratchia Adjarian's seminal work \armenian{Հայ բարբառագիտութիւն} \textit{Armenian dialectology}, originally written in Armenian in 1911. Adjarian describes 31 non-standard Armenian varieties, offering insights into their linguistic structures and historical roots. To enhance accessibility and understanding, this translation unpacks implicit knowledge embedded in Adjarian's text, providing morpheme segmentation, glossing, and translations. This translation is tailored for three distinct audiences: linguists of non-Armenian, traditional Armenian dialectologists, and linguists of Armenian who were trained outside Armenia. This translation aims to bridge linguistic methodologies and facilitate deeper comprehension of Armenian dialectology. The translator supplements Adjarian's prose with commentary, ensuring clarity and accessibility across diverse readerships. This translation provides access to a linguistic landscape of Armenian before the genocide, with the hope of fostering broader scholarly engagement on Armenian dialects.}

\renewcommand{\lsID}{385}
\renewcommand{\lsISBNdigital}{978-3-96110-489-5}
\renewcommand{\lsISBNhardcover}{978-3-98554-118-8}
\BookDOI{10.5281/zenodo.14008766}
\typesetter{Sebastian Nordhoff}
\proofreader{Yasna}
