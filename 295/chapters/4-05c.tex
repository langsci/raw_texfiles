\chapter{Other verbal derivations} \label{chap:other.derivations}
 
\section{Autive} \label{sec:autobenefactive} 
\is{autive}
The autobenefactive-spontaneous or autive\footnote{I adopt the concise and elegant term ``autive'' from \citet{gong18these} instead of ``autobenefactive-spontaneous'' used in previous publications \citep{jacques15spontaneous}. } prefix \forme{nɯ-} is one of the most productive voice prefixes in Japhug. 

In this section, I first present the morphological properties of this prefix (allomorphy and position in the template), and then describe its three main functions: autobenefactive/self-affectedness, spontaneous and permansive (previously identified in \citealt{jacques15spontaneous}). Finally, I discuss cases of lexicalized autives and propose historical pathways between the autive and other derivations such as the vertitive and the anticausative.

The anticausative derivation in Japhug does not cause any stem alternation, unlike in other languages such as Bragbar \citep{zhangshuya20these}.
 
\subsection{The autive prefix and verb transitivity}  \label{sec:autoben.transitivity}
\is{autive!transitivity} \is{transitivity!autive}
Unlike most voice markers, the autive prefix \forme{nɯ-} neither increases nor decreases verb valency: whether the base verb is morphologically intransitive, transitive or labile, \forme{nɯ\trt{}} prefixation has no effect on any of the seven transitivity criteria (§\ref{sec:transitivity.morphology}).

For instance, transitive verbs with \forme{nɯ\trt{}} still have stem III alternation (§\ref{sec:stem3.form}, for instance \forme{-ndo} $\rightarrow$ \forme{-ndɤm} in \ref{ex:nAZo.tAnWndAm}) or the past tense \forme{-t} suffix (§\ref{sec:suffixes}, see \forme{tɤ-nɯ-ndo-t-a} in \ref{ex:aj.tAnWndota.me}).

 \begin{exe}
\ex \label{ex:nAZo.tAnWndAm}
\gll laʁjɯɣ nɤʑo tɤ-nɯ-ndɤm je  \\
staff \textsc{2sg} \textsc{imp}-\textsc{auto}-take[III] \textsc{sfp} \\
\glt `Take the staff.' (2005 khu) 
\end{exe}

 \begin{exe}
\ex \label{ex:aj.tAnWndota.me}
\gll nɤʑo nɤkinɯ, tɤ-tɯ-nɯrdoʁ, aj tɤ-nɯ-ndo-t-a me\\
\textsc{2sg} \textsc{filler} \textsc{aor}-2-pick.up  \textsc{1sg}  \textsc{aor}-\textsc{auto}-take-\textsc{tr}:\textsc{pst}-\textsc{1sg} not.exist:\textsc{fact}\\
\glt `It is you who collected [the fruits], I did not take them.' (lWlu 2002)
\end{exe}

Conversely, intransitive verbs taking the autive prefix present neither stem III alternation nor the \forme{-t} suffix. The autive derivation differs in this regard from the homophonous \forme{nɯ-} applicative prefix (§\ref{sec:applicative}).

The autive is compatible with valency-decreasing derivations, including passive (example \ref{ex:apAnWrku}, §\ref{sec:autoben.position}), reflexive (\ref{ex:tunWZGAraXtCAzi}) and antipassive (\ref{ex:lunWrAjinW} below and §\ref{sec:antipassive.compatibility}).

 \begin{exe}
\ex \label{ex:lunWrAjinW}
\gll lu-nɯ-rɤ-ji-nɯ tɕe, nɯ-kɤ-ndza nɯra ʑara pjɯ-nɯ-tɕɤt-nɯ pjɤ-ŋu \\
\textsc{ipfv}-\textsc{auto}-\textsc{apass}-plant-\textsc{pl} \textsc{lnk} \textsc{3pl}.\textsc{poss}-\textsc{obj}:\textsc{pcp}-eat \textsc{dem}:\textsc{pl} \textsc{3pl} \textsc{ipfv}-\textsc{auto}-take.out-\textsc{pl} \textsc{ifr}.\textsc{ipfv}-be \\
\glt  `(As they were receiving treatment), they were working in the fields, and earning their own food by themselves.' (25-khArWm) 
\japhdoi{0003644\#S61}
\end{exe}

\subsection{Position in the verbal template and allomorphy}  \label{sec:autoben.position}
 \is{autive!verbal template} \is{prefix!autive}

The position of the autive prefix in the template depends on the structure of the verb form. 

When occuring in combination with contracting derivational prefixes, the autive \forme{nɯ-} is inserted after the contracting vowel, as in (\ref{ex:apAnWrku}) with the passive \forme{-ɤ-} (§\ref{sec:passive}).

 \begin{exe}
\ex \label{ex:apAnWrku}
\gll ɕom kɯ scoʁ cʰɯ-kɤ-sɯ-βzu nɯ, tɯtsʰi ɯ-ŋgɯ nɯra a-pɯ-ɤ-nɯ-rku qʰe ɲɯ-ɲaʁ.  \\
iron \textsc{erg} ladle \textsc{ipfv}-\textsc{obj}:\textsc{pcp}-\textsc{caus}-make \textsc{dem} rice.gruel \textsc{3sg}.\textsc{poss}-in \textsc{dem}:\textsc{pl} \textsc{irr}-\textsc{ipfv}-\textsc{pass}-\textsc{auto}-put.in \textsc{lnk} \textsc{ipfv}-be.black \\
\glt `Ladles that are made of iron (iron ladles), whenever they are put in rice gruel, they blacken.' (30-Com) \japhdoi{0003736\#S77}
\end{exe}

Even when the contracting vowel \forme{a/ɤ-} is not a prefix but part of the verb root (§\ref{sec:contraction}), the \forme{nɯ-} prefix is inserted between the \forme{a-} and the rest of the verb stem.\footnote{There is only one counterexample where the \forme{nɯ-} prefix is not infixed, but it concerns a highly lexicalized derivation (§\ref{sec:autoben.lexicalized}). }

For example in (\ref{ex:pjAnWtAr.YWNu}) the \forme{nɯ-} is actually infixed inside the verb stem of \japhug{atɤr}{fall} (see also in example \ref{ex:pannWri} in §\ref{sec:autoben.spontaneous} the infixation within the suppletive stem II \forme{ari} of the verb \japhug{ɕe}{go}).  Although other homophonous prefixes exist in Japhug the autive is the only one that is infixable. All other \forme{nɯ-} prefixes (the applicative §\ref{sec:applicative}, the vertitive §\ref{sec:vertitive}, the denominal \forme{nɯ-} §\ref{sec:denom.nW}, and inflectional morphemes like the \textsc{3pl} possessive prefix §\ref{sec:possessive.paradigm} and the type A \textsc{westwards} orientation prefix §\ref{sec:kamnyu.preverbs}) occupy prefixal slots and undergo vowel contraction with the \forme{a-} of the verb stem (for instance, the applicative yields \forme{nɤ-} with contracting verb stems, §\ref{sec:allomorphy.applicative}).
 
 \begin{exe}
\ex \label{ex:pjAnWtAr.YWNu}
\gll nɯnɯ ɯ-ʁrɯ nɯnɯ tʰɯ-rgɤz tɕe tɕe nɯ ɯʑo pjɯ-ɤ<nɯ>tɤr ɲɯ-ŋu.  \\
\textsc{dem} \textsc{3sg}.\textsc{poss}-horn \textsc{dem} \textsc{aor}-be.old \textsc{lnk}   \textsc{lnk} \textsc{dem} \textsc{3sg} \textsc{ipfv}:\textsc{down}-<auto>fall \textsc{sens}-be \\
\glt `When its antlers age, they fall off by themselves.' (27-qartshAz)
\japhdoi{0003702\#S35}
\end{exe}


The irregular existential verbs \japhug{ɣɤʑu}{exist} and \japhug{maŋe}{not exist} (§\ref{sec:intr.person.irregular}) also take  the spontaneous marker as an infix rather than as a prefix as in (\ref{ex:GAnWZu}). 

\begin{exe}
\ex \label{ex:GAnWZu}
\gll pakuku ʑo ju-nɯɕe-nɯ tɕe nɯtɕu li ɣɤ<nɯ>ʑu ɕti. 	\\
 every.year \textsc{emph} \textsc{ipfv}-come.back-\textsc{pl} \textsc{lnk} there again <\textsc{auto}>exist:\textsc{sens} be.\textsc{aff}:\textsc{fact} \\
 \glt `They come back every year, and it is still there.' (20 grWBgrWB)
\japhdoi{0003554\#S51}
\end{exe}

In addition, like the inverse prefix (§\ref{sec:allomorphy.inv}), the autive is obligatorily infixed within the progressive \forme{asɯ-} (§\ref{sec:inner.prefixal.chain}, §\ref{sec:progressive.morphology}), as in example (\ref{ex:panWsWfkur}). 

 \begin{exe}
\ex \label{ex:panWsWfkur}
\gll <gaoyucheng> kɯ ɯ-laχtɕʰa [...] pɯ-a<nɯ>sɯ-fkur ɣɯ ɯ-ŋgɯ nɯtɕu ɯ-jaʁ cʰɤ-tsɯm ri,\\
\textsc{anthr} \textsc{erg} \textsc{3sg}.\textsc{poss}-thing { } \textsc{pst}.\textsc{ipfv}-<\textsc{auto}>\textsc{prog}-carry \textsc{dem} \textsc{3sg}.\textsc{poss}-inside \textsc{dem}:\textsc{loc} \textsc{3sg}.\textsc{poss}-hand \textsc{ifr}:\textsc{downstream}-take.away \textsc{lnk}\\
\glt `Gao Yucheng put his hand in the things (bag) that he was carrying on his back.' (150902 qixian-zh)
\japhdoi{0006258\#S130}
\end{exe}

In all other cases (and excluding lexicalized autives, §\ref{sec:autoben.lexicalized}), the autive \forme{nɯ-} occurs on the leftmost side of the verb stem, just before the reflexive (§\ref{sec:reflexive}), but after inflectional prefixes (including orientational, participial, inverse and person indexation prefixes), as shown by the forms \forme{tu-\textbf{nɯ}-ʑɣɤ-raχtɕɤz-i} `we treat ourselves' in (\ref{ex:tunWZGAraXtCAzi}) and \forme{a-tɤ-tɯ́-wɣ-\textbf{nɯ}-ndza} in (\ref{ex:atAtWwGnWndza}).

\begin{exe}
\ex \label{ex:tunWZGAraXtCAzi}
\gll  tɕe ji-kɤ-nɯ-raχtɕɤz ra jɤ-ɣe-nɯ, iʑora tu-nɯ-ʑɣɤ-raχtɕɤz-i,  \\
\textsc{lnk} \textsc{1pl}-\textsc{obj}:\textsc{pcp}-\textsc{auto}-cherish \textsc{pl} \textsc{aor}-come[II]-\textsc{pl} \textsc{1pl} \textsc{ipfv}-\textsc{auto}-\textsc{refl}-cherish-\textsc{1pl} \\
\glt `When people we cherish come, or when we [wish] to treat ourselves,' (30 macha)
\japhdoi{0003746\#S74}
\end{exe}

\begin{exe}
\ex \label{ex:atAtWwGnWndza}
\gll tu-tɯ́-wɣ-ndza ɯ-ɲɯ-sɯsɤm qʰe a-tɤ-tɯ́-wɣ-nɯ-ndza ɲɯ-ntsʰi \\
\textsc{ipfv}-2-\textsc{inv}-eat \textsc{qu}-\textsc{ipfv}-think[III] \textsc{lnk} \textsc{irr}-\textsc{pfv}-2-\textsc{inv}-\textsc{auto}-eat \textsc{sens}-be.better \\
\glt  `If it wants to eat you, let it eat you!'  (150901 dongguo xiansheng he lang-zh)
\japhdoi{0006336\#S94}
\end{exe}

When occurring in this leftmost slot, forms bearing the autive prefix can be ambiguous: for instance the surface form \forme{nɯ-ɕe} (from the verb \japhug{ɕe}{go}) can either be parsed as \textsc{3sg} Factual autive  (\textsc{auto}-go:\textsc{fact}), \textsc{3sg} Factual vertitive (§\ref{sec:vertitive}, (\textsc{vert}-go:\textsc{fact}) or \textsc{2sg} \textsc{westwards} imperative (§\ref{sec:imp.morphology}, \textsc{imp}:\textsc{west}-go).

In addition to its infixability, the autive prefix has the particularity of being (optionally) realized as geminated \forme{-n-} when it occurs before a verb stem or verb prefix in \forme{nɯ\trt}, as in (\ref{ex:nWtCu.kWnA.pjAnnWrga}).

\begin{exe}
\ex \label{ex:nWtCu.kWnA.pjAnnWrga}
\gll  tɤ-tɕɯ nɯnɯ, kɯ, nɯnɯtɕu kɯnɤ pjɤ-n-nɯ-rga ɕti. \\
\textsc{indef}.\textsc{poss}-son \textsc{dem} \textsc{erg} \textsc{dem}:\textsc{loc} also \textsc{ifr}:\textsc{ipfv}-\textsc{auto}-\textsc{appl}-like be.\textsc{aff}:\textsc{fact} \\
\glt `Even like that, the boy still loved her.' (140510 sanpian sheye-zh) \japhdoi{0003945\#S117}
\end{exe}

\subsection{Autobenefactive/self-affectedness function}  \label{sec:autoben.proper}
 \is{autive!autobenefactive} \is{autobenefactive}
The autobenefactive function of the \forme{nɯ-} prefix can be subdivided into four subcases: subject affectedness, beneficial, exclusive beneficial, and mild imperative.

First, the autive prefix frequently appears with transitive verbs when the object takes a possessive prefix coreferent with the transitive subject, especially in the case of body parts and other inalienably possessed nouns, to emphasize the fact that the agent is affected by his/her own action. For example, in (\ref{ex:pWnWXtCi}), the \textsc{2sg} possessive prefix on the body part \forme{nɤ-ku} is coreferent with the subject of the 2\textsc{sg}\fl{}3 imperative form \forme{pɯ-nɯ-χtɕi}.

\begin{exe}
\ex \label{ex:pWnWXtCi}
\gll  nɤ-ku pɯ-nɯ-χtɕi  \\
\textsc{2sg}.\textsc{poss}-head \textsc{imp}-\textsc{auto}-wash \\
\glt `Wash your head.' (elicited)
\end{exe}

The autive is not restricted to body part objects, but also appears with more abstract possessed objects such as \forme{ɯ-sroʁ} and \forme{ɯ-ʁrɯm} in (\ref{ex:konWri}) and (\ref{ex:WRrWm.pjAnWmto}).

\begin{exe}
\ex \label{ex:konWri}
\gll ɯʑo kɯ ɯ-sroʁ ko-nɯ-ri ɲɯ-ŋu \\
\textsc{3sg} \textsc{erg} \textsc{3sg}.\textsc{poss}-life \textsc{ifr}-\textsc{auto}-save \textsc{sens}-be \\
\glt `He saved his own life.' (140512 yufu yu mogui-zh)
\japhdoi{0003973\#S126}
\end{exe}

\begin{exe}
\ex \label{ex:WRrWm.pjAnWmto}
\gll tɯ-ci ɯ-ŋgɯ ɯ-ʁrɯm pjɯ-kɯ-ntɕʰɤr nɯ pjɤ-nɯ-mto tɕe,  \\
\textsc{indef}.\textsc{poss}-water \textsc{3sg}.\textsc{poss}-in \textsc{3sg}.\textsc{poss}-shade \textsc{ipfv}-\textsc{sbj}:\textsc{pcp}-shine \textsc{dem} \textsc{ifr}-\textsc{auto}-see \textsc{lnk} \\
\glt `She saw her [own] reflection in the water.' (140428 mu e guniang-zh)
\japhdoi{0003880\#S55}
\end{exe}

Example (\ref{ex:WjaR.Zo.tonWxtsWG}) presents a clear contrast between \forme{mɯ-to-xtsɯɣ} (no \forme{nɯ-} prefix) and \forme{to-nɯ-xtsɯɣ} (presence of \forme{nɯ-} prefix, coreference between possessor and transitive subject). It also illustrates that the subject affectedness expressed by the autive prefix is not necessarily beneficial, but can also be detrimental. 

\begin{exe}
\ex \label{ex:WjaR.Zo.tonWxtsWG}
\gll tɯrpa ci to-lɤt ri, si nɯ ɯ-taʁ mɯ-to-xtsɯɣ kɯ ɯʑo ɣɯ ɯ-jaʁ ʑo to-nɯ-xtsɯɣ. \\
axe once \textsc{ifr}-release \textsc{lnk} tree \textsc{dem} \textsc{3sg}.\textsc{poss}-on \textsc{neg}-\textsc{ifr}-hit \textsc{erg} \textsc{3sg} \textsc{gen} \textsc{3sg}.\textsc{poss}-hand \textsc{emph} \textsc{ifr}-\textsc{auto}-hit \\
\glt `He swung the axe but did not hit the tree, and instead hit his own arm.' (140430 jin e-zh)
\japhdoi{0003893\#S38}
\end{exe}

However, the  \forme{nɯ-} prefix prefix is optional in all four sentences: its presence is not required to express subject-object possessor coreference. In addition, non-coreference between subject and object possessor is possible on verbs  taking the autive prefix.

Second, the autive prefix can also be found on both intransitive and transitive verbs to focus on the pleasant character of an action for the subject, as in (\ref{ex:kunWrAZinW.pjANu}).
%{sec:gen.beneficiary}
 
\begin{exe}
\ex \label{ex:kunWrAZinW.pjANu}
\gll kɤtsa ra χsɯm nɯ kɯ-scɯ\redp{}scit ʑo ku-nɯ-rɤʑi-nɯ pjɤ-ŋu, \\
\textsc{coll}:family \textsc{pl} three \textsc{dem} \textsc{sbj}:\textsc{pcp}-\textsc{emph}\redp{}be.happy \textsc{emph} \textsc{ipfv}-\textsc{auto}-stay-\textsc{pl} \textsc{ifr}.\textsc{ipfv}-be \\
\glt `The three of them lived happily.' (Gesar 2003)
\end{exe}

 
Third, the autive \forme{nɯ-} can be used to express that the action only benefits the subject, to the exclusion of other referents. In (\ref{ex:tunWndzandZi}) for instance, the \forme{nɯ-} prefix on the verbs \forme{tu-nɯ-ndza-ndʑi} `they eat it' and \forme{ku-nɯ-tsʰi-ndʑi} `they drink it' express that the \textsc{3du} subject (the two elder sisters) performed these actions without sharing anything with the \textsc{1sg} referent.

\begin{exe}
\ex \label{ex:tunWndzandZi}
\gll nɤ-pi ni kɯ [...] qajɣi nɯra kɯ-mɯm ʑɤni tu-nɯ-ndza-ndʑi, ɯ-rkɯ kɤ-kɯ-ɕke ra aʑo ɲɯ́-wɣ-mbi-a-ndʑi, cʰa ra ʑɤni ku-nɯ-tsʰi-ndʑi, aʑo ɯ-ʁɟo ɲɯ́-wɣ-jtsʰi-a-ndʑi pɯ-ɕti      \\
\textsc{2sg}.\textsc{poss}-elder.sibling \textsc{du} \textsc{erg} {  } bread \textsc{dem}:\textsc{pl} \textsc{sbj}:\textsc{pcp}-tasty  \textsc{3du} \textsc{ipfv}-\textsc{auto}-eat-\textsc{du} \textsc{3sg}.\textsc{poss}-side \textsc{aor}-\textsc{sbj}:\textsc{pcp}-burn \textsc{pl}  \textsc{1sg} \textsc{aor}-\textsc{inv}-give-\textsc{1sg}-\textsc{du} alcohol \textsc{pl} \textsc{3du} \textsc{ipfv}-\textsc{auto}-drink-\textsc{du} \textsc{1sg} \textsc{3sg}.\textsc{poss}-diluted \textsc{aor}-\textsc{inv}-give.to.drink-\textsc{1sg}-\textsc{du} \textsc{pst}.\textsc{ipfv}-be.\textsc{aff}  \\
 \glt `Your two sisters (...) ate the tasty food and gave me the burned part of the bread, drank the alcohol and gave me diluted alcohol to drink.'  (The three sisters)
\end{exe} 


In these cases an emphatic pronoun referring to the beneficiary can be placed just before the verb (§\ref{sec:pronouns.emph}). It does not bear the genitive (§\ref{sec:gen.beneficiary}) or the ergative  even when it refers to the transitive subject, as \japhug{ʑɤni}{\textsc{3du}} in example (\ref{ex:tunWndzandZi}).  The distributive pronoun \japhug{ʑaka}{each his own} (§\ref{sec:distributive.pronouns}) can also be used in this function, as in (\ref{ex:kAnWBzu.mArtaRtCi}).

   \begin{exe}
\ex \label{ex:kAnWBzu.mArtaRtCi}
\gll tɕiʑo ʁnɯz ma maŋe-tɕi tɕe, ʑaka kɤ-nɯ-βzu mɤ-rtaʁ-tɕi \\
\textsc{1du} two apart.from not.exist:\textsc{sens}-\textsc{1du} \textsc{lnk} each \textsc{inf}-\textsc{auto}-do \textsc{neg}-be.enough:\textsc{fact}-\textsc{1du} \\
\glt `We are only two, we are not enough people to act separately.' (The three sisters)
\end{exe} 

The autive can also appear when the beneficiary is an oblique referent, for instance the genitively-marked \japhug{aʑɯɣ}{\textsc{1sg}:\textsc{gen}} (§\ref{sec:gen.beneficiary}) in (\ref{ex:aZWG.apWnWpe}), where the relative (between square brackets) containing the complement clause \forme{aʑɯɣ a-pɯ-nɯ-pe} `May it be beneficial to me! (at the expense of others)' is used to translate \ch{很自私}{hěn zìsī}{very selfish} in the original text.

   \begin{exe}
\ex \label{ex:aZWG.apWnWpe}
\gll [``aʑɯɣ a-pɯ-nɯ-pe" ɲɯ-kɯ-sɯso] ci pjɤ-ŋu \\
\textsc{3sg}:\textsc{gen} \textsc{irr}-\textsc{ipfv}-\textsc{auto}-be.good \textsc{ipfv}-\textsc{sbj}:\textsc{pcp}-think \textsc{indef} \textsc{ifr}.\textsc{ipfv}-be   \\
\glt `He was someone who was (always) thinking `May it be beneficial to me!'' (140430 jin e-zh)
\japhdoi{0003893\#S29}
\end{exe} 

Fourth, with Imperative and Irrealis forms (§\ref{sec:TAME.modal}), the autive prefix can convey a softened tone, expressing a friendly suggestion rather than an order, as in   (\ref{ex:GWtAnWXtWnW}) and (\ref{ex:CtAnWndze}) or even the wish that the addressee performs an action pleasant and beneficial to him/herself (\ref{ex:anWtWnWrWmaninW}).


\begin{exe}
\ex \label{ex:CtAnWndze}
\gll nɤʑo nɯnɯtɕu kɤndza kɯ-mɯm ɕ-tɤ-nɯ-ndze  \\
\textsc{2sg} \textsc{dem}:\textsc{loc} food \textsc{sbj}:\textsc{pcp}-be.tasty \textsc{tral}-\textsc{imp}-\textsc{auto}-eat[III] \\
\glt `Go and eat nice food there!' (140426 jiagou he lang-zh)
\japhdoi{0003804\#S64}
\end{exe}

\begin{exe}
\ex \label{ex:GWtAnWXtWnW}
\gll laχtɕʰa mɯtɕʰɯmɯrɯz tu, mɤʑɯ koxtɕinri tu tɕe, ɣɯ-tɤ-nɯ-χtɯ-nɯ \\ 
things all.kind exist:\textsc{fact} also silk.thread exist:\textsc{fact} \textsc{lnk} \textsc{cisl}-\textsc{imp}-\textsc{auto}-buy-\textsc{pl} \\
\glt `There are all kinds of things, there are silk threads, come and buy them!' (140504 baixue gongzhu-zh) \japhdoi{0003907\#S121}
\end{exe}

\begin{exe}
\ex \label{ex:anWtWnWrWmaninW}
\gll a-wɯ a-wi ra kutɕu nɯ-n-nɤʁaʁ-nɯ, a-nɯ-tɯ-nɯ-rɯmani-nɯ, tɕe kutɕu a-kɤ-tɯ-nɯ-rɤʑi-nɯ \\
\textsc{1sg}.\textsc{poss}-grandfather \textsc{1sg}.\textsc{poss}-grandmother \textsc{pl} \textsc{dem}.\textsc{prox}:\textsc{loc} \textsc{imp}-\textsc{auto}-have.a.good.time-\textsc{pl} \textsc{irr}-\textsc{pfv}-2-\textsc{auto}-recite.mantra-\textsc{pl} \textsc{lnk} \textsc{dem}.\textsc{prox}:\textsc{loc} \textsc{irr}-\textsc{pfv}-2-\textsc{auto}-stay-\textsc{pl} \\
\glt `Grandfathers and grandmothers, have a good time here, recite mantras and stay here (pleasantly).' (2003kandzwsqhaj)
\end{exe}
%ʑaka a-ɣɯ-tɤ-tɯ-nɯ-ru-nɯ ra

However, it should be pointed out that the autive has an almost opposite (mocking/defiance) function in imperatives in examples such as (\ref{ex:nWnWGAWu}) below in §\ref{sec:autoben.spontaneous}.
 
\subsection{Spontaneous function}  \label{sec:autoben.spontaneous} 
 \is{autive!spontaneous} \is{spontaneous}
The spontaneous function of the autive prefix includes six subcases: actions without external cause, self-volitional actions, non-volitional actions, casual action, concessive clauses,  and mocking imperatives.

First, the autive prefix expresses actions that are perceived by the speaker as occurring spontaneously by themselves, such as the growth of plants or animals (\ref{ex:YWkWnWBze}) or action taking place due to an unseen force (example \ref{ex:WZo.tonWYJW}, with an emphatic pronoun \forme{ɯʑo}, §\ref{sec:pronouns.emph}).

\begin{exe}
\ex \label{ex:YWkWnWBze}
\gll tɕe zrɯɣ nɯ tɕe, tsuku kɯ tɯ-pɤcʰaʁ ɯ-ŋgɯ tu-nɯ-ɬoʁ ŋu tu-ti-nɯ ŋu tɕe mɤ-xsi ma ɯʑo ɲɯ-kɯ-nɯ-βze ci ɲɯ-ɕti tɕe, 	\\
\textsc{lnk} louse \textsc{dem} \textsc{lnk} some \textsc{erg} \textsc{indef}.\textsc{poss}-navel \textsc{3sg}-inside \textsc{ipfv}-\textsc{auto}-come.out be:\textsc{fact} \textsc{ipfv}-say-\textsc{pl}  be:\textsc{fact} \textsc{lnk} \textsc{neg}-\textsc{genr}:A:know \textsc{lnk} \textsc{3sg} \textsc{ipfv}-\textsc{sbj}:\textsc{pcp}-\textsc{auto}-grow \textsc{indef} \textsc{sens}-be.\textsc{aff} \textsc{lnk} \\
\glt `The louse, some say that it comes from the navel, I don't know, it grows by itself.' (21 mdzadi) \japhdoi{0003578\#S52}
\end{exe}

\begin{exe}
\ex \label{ex:WZo.tonWYJW}
\gll si ɯ-rgɤm nɯ ɣɯ ɯ-fkaβ nɯ, ɯʑo to-nɯ-ɲɟɯ.  \\
wood \textsc{3sg}.\textsc{poss}-box \textsc{dem} \textsc{gen} \textsc{3sg}.\textsc{poss}-cover \textsc{dem} \textsc{3sg} \textsc{ifr}-\textsc{auto}-\textsc{acaus}:open \\
\glt `The cover of the box opened by itself.' (150906 toutao-zh)
\japhdoi{0006326\#S160}
\end{exe}

Second, with a volitional subject (in particular human), the autive can indicate an action performed of one's own will, without being forced by anything or anyone, as in (\ref{ex:pjWnWmtsaRa}), or without help from anybody else (`by oneself'), as in (\ref{ex:zYWnWrua}) and (\ref{ex:tunWGArRaR}).

\begin{exe}
\ex \label{ex:pjWnWmtsaRa}
\gll  aʑo pjɯ-kɯ-nɯ-βde-a-nɯ mɤ-ra, aʑo pjɯ-nɯ-mtsaʁ-a jɤɣ \\
\textsc{1sg} \textsc{ipfv}:\textsc{down}-2\fl{}1-throw-\textsc{1sg}-\textsc{pl} \textsc{neg}-be.needed:\textsc{fact} \textsc{1sg} \textsc{neg}-\textsc{ipfv}:\textsc{down}-\textsc{auto}-jump-\textsc{1sg} be.allowed:\textsc{fact} \\
\glt `You don't need to throw me in there, I will jump by myself (of my own free will).' (2011-05-nyima)
\end{exe}

\begin{exe}
\ex \label{ex:zYWnWrua}
\gll aʑo ʑo z-ɲɯ-nɯ-ru-a ɲɯ-ntsʰi \\
\textsc{1sg} \textsc{emph} \textsc{tral}-\textsc{ipfv}-\textsc{auto}-look-\textsc{1sg} \textsc{sens}-be.needed \\
\glt `I need to go and have a look by myself.' (140507 tangguowu-zh) \japhdoi{0003933\#S138}
\end{exe} 

\begin{exe}
	\ex \label{ex:tunWGArRaR}
	\gll tɕe nɯ ɯʑo tu-nɯ-ɣɤrʁaʁ qʰe nɯ ɯ-kɤ-ndza ɲɯ-nɯ-ɕar ŋu. \\
	\textsc{lnk} \textsc{dem} \textsc{3sg} \textsc{ipfv}-\textsc{auto}-hunt \textsc{lnk} \textsc{dem} \textsc{3sg}.\textsc{poss}-\textsc{obj}:\textsc{pcp}-eat \textsc{ipfv}-\textsc{auto}-look.for be:\textsc{fact} \\
	\glt `[The cat] hunts on its own and looks for its own food by itself.' (21-lWLU)
\japhdoi{0003576\#S46}
\end{exe} 


Third, the autive can also express an action occurring by mistake or against the volition of the subject: compare examples (\ref{ex:pWqrWta}) with autive and (\ref{ex:pWnWqrWta}) without it.

\begin{exe}
\ex 
\begin{xlist}
\ex \label{ex:pWqrWta}
\gll kʰɯtsa pɯ-qrɯ-t-a\\
bowl \textsc{aor}-break-\textsc{pst}:\textsc{tr}-\textsc{1sg} \\
\glt `I broke the bowl (possibly on purpose).' (elicited)
\ex \label{ex:pWnWqrWta}
\gll kʰɯtsa pɯ-nɯ-qrɯ-t-a\\
bowl \textsc{aor}-\textsc{auto}-break-\textsc{pst}:\textsc{tr}-\textsc{1sg} \\
\glt `I broke the bowl (by mistake).' (elicited)
\end{xlist} 
\end{exe}

The combination of inferential with autive (see also §\ref{sec:inf.1person}) can express that the action occurred against the volition of the subject and unbeknownst to him/her at the time when it happened (\ref{ex:chAtWnWrArJitnW.mWjtWnWsWXsAlnW}).

\begin{exe}
\ex \label{ex:chAtWnWrArJitnW.mWjtWnWsWXsAlnW}
\gll hehe a-ʑi ra cʰɤ-tɯ-nɯ-rɤrɟit-nɯ mɯ́j-tɯ-nɯ-sɯχsɤl-nɯ  \\
\textsc{interj} \textsc{1sg}.\textsc{poss}-lady \textsc{pl} \textsc{ifr}-2-\textsc{auto}-have.a.child-\textsc{pl} \textsc{neg}:\textsc{sens}-2-\textsc{auto}-realize-\textsc{pl}  \\
\glt `My lady, you had a child and did not notice it.' (2003 Kunbzang)
\end{exe}

Some verbs such as \japhug{jmɯt}{forget} or \japhug{ɕlɯɣ}{drop}, generally appears with the autive prefix in the corpus (in the case of \japhug{jmɯt}{forget} in 23 examples out of 28), as in  (\ref{ex:mAxsi2})  
 
\begin{exe}
 \ex \label{ex:mAxsi2}
 \gll mɤ-xsi ko, nɯra ɲɤ-nɯ-jmɯt-a  \\
\textsc{neg}-\textsc{genr}:know \textsc{sfp} \textsc{dem}:\textsc{pl} \textsc{ifr}-\textsc{auto}-forget-\textsc{1sg} \\
\glt `I don't know, I forgot those things.' (Conversation, 2013-12-24)
\end{exe}

Fourth, an extension of the spontaneous value of the prefix \forme{nɯ-} is the meaning `casually', `at one's will', `whatever' (corresponding to Chinese \ch{随便}{suíbiàn}{casually}), as in (\ref{ex:tunWtCAtnW}). 

\begin{exe}
\ex \label{ex:tunWtCAtnW}
\gll ``huaguniang" ra tu-nɯ-ti-nɯ ɲɯ-ŋu. nɯnɯra ʑara kɯ ɯ-rmi tu-nɯ-tɕɤt-nɯ ɲɯ-ŋu.  \\
name \textsc{pl}	\textsc{ipfv}-\textsc{auto}-say-\textsc{pl}	\textsc{sens}-be	\textsc{dem}:\textsc{pl}	\textsc{3pl}	\textsc{erg}	\textsc{3sg}.\textsc{poss}-name	\textsc{ipfv}-\textsc{auto}-take-\textsc{pl}	\textsc{sens}-be\\
\glt `They say `huaguniang', they call [this type of cows] like that.' (\textit{implied meaning}: they invented their name, it is not a real name; 28 qapar) \japhdoi{0003720\#S228}
\end{exe}

Fifth, as a further extension of the meaning `casually' seen above, the autive appears in the protasis of alternative (\ref{ex:pannWri}), scalar (\ref{ex:pWnnWtu.kWnA}) and universal (§\ref{sec:emph.redp}) concessive conditionals (§\ref{sec:concessive.conditional}, \citealt[298--300]{jacques14linking}). In these constructions, the result described in the apodosis takes place regardless of whether the condition in the protasis is fulfilled or not. The autive prefix expresses the fact that the resulting action is independent of the condition. In this function, it is often realized with a gemination as \forme{-nnɯ-}.

\begin{exe}
\ex  \label{ex:pannWri}
\gll tɕe tɯ-sɯm pɯ-a<nnɯ>ri nɤ ju-kɯ-ɕe, mɯ-pɯ-a<nnɯ>ri nɤ ju-kɯ-ɕe pɯ-ra \\
\textsc{lnk} \textsc{genr}.\textsc{poss}-mind  \textsc{aor}-<\textsc{auto}>go[II] \textsc{lnk} \textsc{ipfv}-\textsc{genr}:S/O-go \textsc{neg}-\textsc{aor}-<\textsc{auto}>go[II] \textsc{lnk} \textsc{ipfv}-\textsc{genr}:S/O-go \textsc{pst}.\textsc{ipfv}-be.needed \\
\glt `Whether one liked it or not, one had to go.' (14-siblings) \japhdoi{0003508\#S193}
\end{exe}

 \begin{exe}
\ex  \label{ex:pWnnWtu.kWnA}
\gll nɯ li ɯ-qa ɲɯ-βze ɲɯ-ɕti ma ɯ-mɯntoʁ pɯ-nnɯ-tu kɯnɤ, ɯ-rɣi ra kɤ-mto maŋe.  \\
\textsc{dem} again \textsc{3sg}.\textsc{poss}-foot \textsc{ipfv}-make[III] \textsc{sens}-be:\textsc{aff} \textsc{lnk} \textsc{3sg}.\textsc{poss}-flower \textsc{pst}.\textsc{ipfv}-\textsc{auto}-exist also \textsc{3sg}.\textsc{poss}-seed \textsc{pl} \textsc{inf}-see not.exist:\textsc{sens} \\
\glt `This one also grows by its root, as even if it has flowers, [I] have never seen its seeds.' (17 ndZWnW)
\japhdoi{0003524\#S149}
\end{exe}

Sixth, in the Imperative and the Irrealis, the spontaneous can be used to mock, or express defiance towards the addressee or another person, stating that all his/her actions will be in vain, as in (\ref{ex:nWnWGAWu}).\footnote{The autive prefix is however also used to express a mild imperative, as in examples (\ref{ex:CtAnWndze}) and (\ref{ex:GWtAnWXtWnW}) in §\ref{sec:autoben.proper}, depending on the context. }

\begin{exe}
\ex \label{ex:nWnWGAWu}
\gll  nɤʑo nɯ-nɯ-ɣɤwu ma, nɤ-kɯ-nɯɣ-mu me ma mɤ-ta-mbi \\
\textsc{2sg} \textsc{imp}-\textsc{auto}-cry \textsc{lnk} \textsc{2sg}.\textsc{poss}-\textsc{nmlz}:S/A-\textsc{appl}-be.afraid not.exist:\textsc{fact} \textsc{lnk} \textsc{neg}-1\fl{}2-give:\textsc{fact} \\
\glt `You can cry as much as you like, nobody is afraid of you, I won't give [my daughter] to you (in marriage).' (2002 qaCpa)
\end{exe}

The defying/mocking imperative usage clearly stems from the meaning `casually' of the autive prefix seen above -- in a Chinese translation of (\ref{ex:nWnWGAWu}), the imperative \forme{nɯ-nɯ-ɣɤwu} was translated as \ch{随便他哭}{suíbiàn tā kū}{let it cry} with the adverb \ch{随便}{suíbiàn}{casually}. In example (\ref{ex:atAnWti}), the verb \japhug{ti}{say} appears two times with the autive prefix, in the second instance with the irrealis, expressing both defiance and the `casually' meaning at the same time.

\begin{exe}
\ex \label{ex:atAnWti}
\gll  qajdo kɯ tɕʰi mɤ-nɯ-ti ɕti nɤ, a-tɤ-nɯ-ti ma ŋu cinɤ maʁ kɯ, \\
crow \textsc{erg} what \textsc{neg}-\textsc{auto}-say:\textsc{fact} be.\textsc{aff}:\textsc{fact} \textsc{sfp} \textsc{irr}-\textsc{pfv}-\textsc{auto}-say \textsc{lnk} be:\textsc{fact} not.even not.be:\textsc{fact} \textsc{sfp} \\
\glt `What wouldn't a crow say (a crow tells only lies), let it say what it wants, in any case none of it is true.'  (28-qajdoskAt)
\japhdoi{0003718\#S24}
\end{exe}
%{ex:atAtWwGnWndza}
 
\subsection{Permansive function}  \label{sec:autoben.permansive}
 \is{autive!permansive} \is{permansive}
In addition to the two previous functions, which are relatively straightforward for a middle marker, the autive prefix also presents an aspectual function. It expresses the continuity of an action or a state, like the adverb `still' in English. Two subcases must be distinguished.

First, the autive can mean that the action of the verb goes on despite the occurrence of another action which could have been expected to stop it (as in \ref{ex:pjAnWGAwu} and \ref{ex:anWmphWr}).  

\begin{exe}
\ex \label{ex:pjAnWGAwu}
\gll tɕʰeme nɯ ɲɤ-nɯkʰɤda ri, mɯ-pjɤ-pʰɤn, tɕʰeme nɯ pjɤ-nɯ-ɣɤwu ɕti, \\
girl \textsc{dem} \textsc{ifr}-convince \textsc{lnk} \textsc{neg}-\textsc{ifr}-be.efficient girl \textsc{dem} \textsc{ifr}.\textsc{ipfv}-\textsc{auto}-cry  be.\textsc{aff}:\textsc{fact} \\
\glt `She [tried to] comfort the girl, but it was for nothing, the girl kept on crying.' (zrAntCW 2003)
\end{exe} 

\begin{exe}
\ex \label{ex:anWmphWr}
\gll nɯnɯ pjɯ-ŋgra ɕɯŋgɯ tɕe tɕe ɲɯ-rom ɕti tɕe, ɯ-rɣi nɯnɯ tɕu a-nɯ-mpʰɯr ɕti \\
\textsc{dem} \textsc{ipfv}:\textsc{down}-\textsc{acaus}:make.fall before \textsc{lnk}  \textsc{lnk} \textsc{ipfv}-be.dry be:\textsc{aff}:\textsc{fact} \textsc{lnk} \textsc{3sg}.\textsc{poss}-seed \textsc{dem} \textsc{loc} \textsc{pass}-\textsc{auto}-wrap:\textsc{fact} be:\textsc{aff}:\textsc{fact} \\
\glt `Before [the flower] falls down, it dries up, and its seed is still wrapped in it.' (13-tCamu)
\japhdoi{0003498\#S56}
\end{exe} 

The autive is found in particular in sentences with the phrase \forme{nɯ kɯnɤ} `even like that, despite these circumstances', as in (\ref{ex:nW.kWnA.pjAnWrAma}) below and (\ref{ex:nWtCu.kWnA.pjAnnWrga}) in §\ref{sec:autoben.position} above.

\begin{exe}
\ex \label{ex:nW.kWnA.pjAnWrAma}
\gll  nɯ kɯnɤ pjɤ-nɯ-rɤma. \\
 \textsc{dem} also \textsc{ifr}.\textsc{ipfv}-\textsc{auto}-work \\ 
\glt `Even so, she kept on working.' (140520 ye tiane-zh) \japhdoi{0004044\#S375}
\end{exe} 
 
Second, the autive \forme{nɯ-} prefix can express that a state continues despite the fact that a long time has passed, as in (\ref{ex:anWrŋi}), or that it is maintained without change, as in (\ref{ex:YWnWNWNu}).

\begin{exe}
\ex \label{ex:anWrŋi}
\gll tɤtʰo nɯnɯ qartsɯmɤftɕar ʑo a<nɯ>rŋi ɕti \\
pine \textsc{dem} winter.and.summer \textsc{emph} <\textsc{auto}>be.blue:\textsc{fact} be.\textsc{aff}:\textsc{fact} \\
\glt `The pine [remains] green the whole year.' (07 tAtho)
\japhdoi{0003432\#S47}
\end{exe}
 

\begin{exe}
 \ex \label{ex:YWnWNWNu}
 \gll ʑmbɯlɯm cʰondɤre grɯβgrɯβ kɯ-fse tɤ-ɬoʁ tɕe χploʁχploʁ kɯ-pa tɕe ʑɯrɯʑɤri ɲɯ-kɯ-nɤwɤt nɯ ɲɯ-maʁ. tɤ-ɬoʁ jɤznɤ ɲɯ-xtɕi laʁma nɯ kɯ-fse ɲɯ-nɯ-ŋɯ\redp{}ŋu qʰe\\
type.of.mushroom \textsc{comit} Matsutake \textsc{sbj}:\textsc{pcp}-be.like \textsc{aor}-come.out \textsc{lnk} \textsc{idph(II):}spherical \textsc{sbj}:\textsc{pcp}-auxiliary \textsc{lnk} progressively \textsc{ipfv}-\textsc{sbj}:\textsc{pcp}-open.towards.the.exterior \textsc{dem} \textsc{sens}-not.be  \textsc{aor}-come.out when \textsc{sens}-be.small only \textsc{dem} \textsc{sbj}:\textsc{pcp}-be.like \textsc{sens}-\textsc{auto}-\textsc{emph}\redp{}be \textsc{lnk}\\
\glt `It is not like the (mushroom called) \forme{ʑmbɯlɯm} and the Matsutake, which are spherical when they come out and progressively open towards the exterior. It is just that it is small when it comes out, [otherwise] it is already like that.'  (24-zwArqhAjmAG)
\japhdoi{0003630\#S17}
 \end{exe}
 
The permansive reading of the autive prefix is only possible in non-perfective verb forms, in particular Factual, Imperfective, Past Imperfective and Sensory.
\is{permansive}
The permansive use of the autive \forme{nɯ-} is not without typological parallels. One of the clearest cases is the Russian pronominal element \forme{себе}, originally the dative form of \forme{себя} `oneself', and which alongside its autobenefactive value,  is used in certain contexts with a permansive value (`continue to ...').\footnote{I am indebted to Dmitry Nikolayev and Pavel Ozerov for pointing out this fact to me and suggesting the grammaticalization path proposed in this section, though I remain responsible for any error.}
 
This construction is not fully grammaticalized in Russian, but it is nevertheless a good parallel to the permansive value of the autive in Japhug. Since it is clear in Russian that the original meaning of this marker can only have been autobenefactive, not permansive or spontaneous, this fact suggests that the directionality of grammaticalization is more likely to be from autobenefactive to permansive in Japhug too. The following pathway in four stages can be proposed to account for this evolution; note that all four stages represent attested uses of Japhug autive.
\largerpage
\begin{enumerate}
\item `Do $X$ for/to oneself' (\textsc{autobenefactive}, examples \ref{ex:pWnWXtCi}, \ref{ex:konWri}, \ref{ex:tunWndzandZi} in §\ref{sec:autoben.proper}).  
\item `Do $X$ on one's own' (\ref{ex:zYWnWrua}, §\ref{sec:autoben.spontaneous}).
\item `Do $X$ on one's own, disregarding external conditions' (\ref{ex:pjWnWmtsaRa}, §\ref{sec:autoben.spontaneous}).
\item `Continue to do $X$, despite (adverse) external factors.' (\textsc{permansive})
\end{enumerate}
 
\subsection{Lexicalized autives} \label{sec:autoben.lexicalized}
The autive \forme{nɯ-} is lexicalized in a handful of verbs listed in \tabref{tab:autive.lexicalized}, whose meaning is not fully predictable from that of the base verb, and which present several morphological properties.

\begin{table} \small
\caption{Lexicalized autive verbs} \label{tab:autive.lexicalized}
\begin{tabular}{lllll}
\lsptoprule
Base verb & Derived verb \\
\midrule
\japhug{atɯɣ}{meet} & \japhug{nɤtɯɣ}{happen to be} \\
 \midrule
\japhug{βde}{throw} & \japhug{nɯβde}{lose} \\
\japhug{ta}{put}  &  \japhug{nɯta}{wear, take}  \\
 \japhug{stʰoʁ}{push} & \japhug{nɯstʰoʁ}{have sex} \\
 \japhug{kro}{share}  &  \japhug{nɯkro}{share among themselves}  \\
 \midrule
 \japhug{sɤndu}{exchange} &  \japhug{antsɤndu}{get exchanged (by mistake)} \\
 \lspbottomrule
\end{tabular}
\end{table}

The clearest example of lexicalized autive is the verb \japhug{nɤtɯɣ}{happen to be} (\ref{ex:YWnAtWG.pjANu}), which derives from \japhug{atɯɣ}{meet}. The autive \forme{nɯ-} prefix, which undergoes here vowel contraction (§\ref{sec:contraction}), occurs here in spontaneous action function (§\ref{sec:autoben.spontaneous}), but both the  meaning of the verb and the position of the prefix are anomalous: the autive is normally infixed rather than prefixed in the case if verbs in \forme{a-} (§\ref{sec:autoben.position}).
\largerpage
\begin{exe}
\ex \label{ex:YWnAtWG.pjANu}
\gll ndi na-ʑa-nɯ qʰe, tɕendi zɯ mɤ-kɯ-βɟɤt ɯ-rca ntsɯ ɲɯ-nɤtɯɣ pjɤ-ŋu. \\
west \textsc{aor}:\textsc{west}:3\fl{}3-start-\textsc{pl} \textsc{lnk} west \textsc{loc} \textsc{neg}-\textsc{sbj}:\textsc{pcp}-obtain \textsc{3sg}.\textsc{poss}-following always \textsc{ipfv}-happen.to.be.at \textsc{ifr}.\textsc{ipfv}-be \\
\glt `When they started [distributing food] from the west, he always happened to be among those who could not get anything.' (28-qajdoskAt) \japhdoi{0003718\#S148}
\end{exe}

The regular infixed autive, with predictable meaning, is also attested as in  (\ref{ex:nanWtWGndZi}), in the spontaneous function (§\ref{sec:autoben.spontaneous}).

\begin{exe}
\ex \label{ex:nanWtWGndZi}
\gll nɯni ʑɤni nɯ-a<nɯ>tɯɣ-ndʑi \\
\textsc{dem}:\textsc{du} \textsc{3du}
\textsc{aor}-<\textsc{auto}>meet-\textsc{du} \\
\glt `They met each other by themselves (it was not an arranged marriage).' (14-siblings) \japhdoi{0003508\#S313}
\end{exe}

The lexicalized autive \forme{nɤtɯɣ} can be prefixed by the sigmatic causative as \forme{znɤtɯɣ} `cause $X$ to be at' as in (\ref{ex:YWwGznAtWG}) or with the volitional meaning `make sure to happen to be at' (\ref{ex:ajAtWznAtWG}). This is an additional difference from the regular autive, which is placed further away from the stem than the causative prefix (§\ref{sec:autoben.position}).

\begin{exe}
\ex \label{ex:YWwGznAtWG}
\gll kɯki kɯ-ɤmtɕoʁ nɯnɯ ɯ-mci pjɯ-kɯ-ɣi ɯ-stu nɯ ɲɯ́-wɣ-z-nɤtɯɣ. \\
\textsc{dem}.\textsc{prox} \textsc{sbj}:\textsc{pcp}-be.pointy \textsc{dem} \textsc{3sg}.\textsc{poss}-saliva \textsc{ipfv}:\textsc{down}-\textsc{sbj}:\textsc{pcp}-come \textsc{3sg}.\textsc{poss}-direction \textsc{dem} \textsc{ipfv}-\textsc{caus}-happen.to.be.at \\
\glt `One turns the corner [of the bib] towards the direction of the drooling saliva [of the baby].' (vid-20140506043657)
\end{exe}

\begin{exe}
\ex \label{ex:ajAtWznAtWG}
\gll ɕɤr tɯtsʰot sqamnɯz ʑo tɕe a-jɤ-tɯ-z-nɤtɯɣ tɕe, \\
 night hour twelve \textsc{emph} \textsc{lnk}  \textsc{irr}-\textsc{pfv}-2-\textsc{caus}-happen.to.be.at \textsc{lnk} \\
\glt `You will have to make sure to be there at midnight.' (2003 qachGa)
\japhdoi{0003372\#S26}
\end{exe}
%nɯkro
 

The verbs \japhug{nɯβde}{lose} (\ref{ex:Wrte.YAnWBde2}), \japhug{nɯta}{wear, take} (in particular, `take as a wife in marriage' as in \ref{ex:WrZaB.akAnWte}), \japhug{nɯstʰoʁ}{have sex}\footnote{This transitive verb requires a male as subject. For a typological parallel, see the obscene sense of Latin \forme{comprimo} `press' (\citealt[182]{adams90latin}). } and \japhug{nɯkro}{share among themselves}, are additional cases of lexicalized autive derivations. The former \japhug{nɯβde}{lose} reflects the spontaneous function of the \forme{nɯ-} prefix, and the three other verbs the autobenefactive (`do for/on oneself') function.
  
\begin{exe}
\ex \label{ex:Wrte.YAnWBde2}
\gll jɤ-a<nɯ>ri ri, ɯ-rte ɲɤ-nɯβde \\
\textsc{aor}-<\textsc{vert}>go[II] \textsc{lnk} \textsc{3sg}.\textsc{poss}-hat \textsc{ifr}-lose\\
\glt `He went away, but lost his hat.' (2010-07-pear story)
\end{exe}

\begin{exe}
\ex \label{ex:WrZaB.akAnWte}
\gll nɯ kɯ ɯ-rʑaβ a-kɤ-nɯte ɲɯ-ntsʰi \\
\textsc{dem} \textsc{erg} \textsc{3sg}.\textsc{poss}-wife \textsc{irr}-\textsc{pfv}-take[III] \textsc{sens}-be.better \\
\glt `Let him take her as his wife.' (140513 shenqi de feitan-zh) \japhdoi{0003981\#S175}
\end{exe}

The causative forms of these  verbs, such as  \japhug{znɯta}{let X wear} (\ref{ex:pjWkWznWtaa}) and \japhug{znɯkro}{share X with} (\ref{ex:YWznWkrAm}) also have the sigmatic causative prefix placed before the lexicalized autive \forme{nɯ\trt}, unlike the expected order.  Note also that although the meaning of the autobenefactive \japhug{nɯkro}{share among themselves} from  \japhug{kro}{share}  appears to be compositional and predictable, the causative is both positionally and semantically irregular, as it works as a beneficiary applicative rather than as a true causative with this verb (a prototypical causative would be expected to mean `made/let X share among themselves', though it is easy to understand how such a meaning could have evolved to `share with X').

\begin{exe}
\ex \label{ex:pjWkWznWtaa}
\gll nɤ-rɟɤntɕʰa nɯra aʑo ci pjɯ-kɯ-z-nɯta-a tɕe \\
\textsc{2sg}.\textsc{poss}-ornament \textsc{dem}:\textsc{pl} \textsc{1sg} \textsc{indef} \textsc{ipfv}-2\fl{}1-\textsc{caus}-take-\textsc{1sg} \textsc{lnk} \\
\glt `Can you let me wear your ornaments?' (2014-kWlAG)
\end{exe}

\begin{exe}
\ex \label{ex:YWznWkrAm}
\gll ɯ-tɤrʁaʁɕa nɯra ɯ-zda ra ɲɯ-z-nɯ-krɤm, ɲɯ-mbi ntsɯ pjɤ-ŋu. \\
\textsc{3sg}.\textsc{poss}-wild.meat \textsc{dem}:\textsc{pl} \textsc{3sg}.\textsc{poss}-companion \textsc{pl} \textsc{ipfv}-\textsc{caus}-\textsc{auto}-share[III]:\textsc{fact} \textsc{ipfv}-give always \textsc{ifr}.\textsc{ipfv}-be \\
\glt `He would share the meat from his hunt with the others, give it to them.' (150902 hailibu-zh)
\japhdoi{0006316\#S5}
\end{exe}

The intransitive verb \japhug{antsɤndu}{get exchanged (by mistake)} is derived from the transitive verb \japhug{sɤndu}{exchange} (itself the causative of \japhug{andu}{be exchanged for}) by the combination of the passive \forme{a-} prefix (§\ref{sec:passive}) with the prefixal element \forme{-nt-}. Given the intrinsic non-volitional meaning of \forme{{antsɤndu}} (see examples \ref{ex:RnWzmWz.nantsAndu} and \ref{ex:YWtAntsAndunW}), it is likely that it originates from the autive in spontaneous action function (§\ref{sec:autoben.spontaneous}) as the reduced allomorph \forme{-n-}. The \forme{-t-} element between \forme{-n-} and the verb stem \forme{sɤndu} is an effect of internal sandhi \forme{-ns-} \fl{} \forme{-nts-} (a sound change reminiscent of Tibetan, \citealt{lifk33}), as prenasalized fricatives do not exist in Japhug (§\ref{sec:NC.clusters}).


\begin{exe}
\ex \label{ex:RnWzmWz.nantsAndu}
\gll kʰɤdi ʁnɯz-mɯz nɯ-antsɤndu mɤ-nɯ-sɯχsɤl \\
lady.seating.place two-kind \textsc{aor}-be.exchanged.by.mistake \textsc{neg}-\textsc{auto}-realize:\textsc{fact} \\
\glt `[Your king] does not realize that two [different sisters] have been exchanged at the lady seating place.' (2003 Kunbzang, 329; see §\ref{sec:orientation.kitchen} on the meaning of \forme{kʰɤdi})
\end{exe}

\begin{exe}
\ex \label{ex:YWtAntsAndunW}
\gll nɯʑora ɲɯ-tɯ-ɤntsɤndu-nɯ \\
\textsc{2pl} \textsc{sens}-2-be.exchanged.by.mistake-\textsc{pl} \\
\glt `You are the opposite way!' (conversation 16-04-12; context: I told \iai{Tshendzin} that my father and I cannot drive cars, while my mother and my wife can)
\end{exe}

The verb \forme{antsɤndu} can thus be analyzed as a double autive+passive derivation, with the expected placement of the autive after the passive (§\ref{sec:autoben.position}). This verb can be further causativized as \japhug{sɤntsɤndu}{exchange by mistake} and even receive a second autive prefix as in (\ref{ex:tonWsAntsAndutCi}).

\begin{exe}
\ex \label{ex:tonWsAntsAndutCi}
\gll tɕi-ŋga to-nɯ-sɯ-ɤntsɤndu-tɕi \\
\textsc{1du}.\textsc{poss}-clothes \textsc{ifr}-\textsc{auto}-\textsc{caus}-exchanged.by.mistake-\textsc{1du} \\
\glt `We exchanged our clothes by mistake.' (elicited)
\end{exe}

There are also examples of lexicalized autive \forme{nɯ-} prefixes that are located in their expected locus in the template, further away from the stem than the causative. This is the case for instance with the verb \japhug{nɯsɯkʰo}{rob, extort} (see example \ref{ex:YWkWnWsWkhoa}, §\ref{sec:sig.caus.lexicalized}), from the causative \japhug{sɯkʰo}{cause to give} of \japhug{kʰo}{give}. This verb has two \forme{sɤ-} antipassive forms (§\ref{sec:antipassive.sA}), the regular one \forme{sɤ-nɯ-sɯ-kʰo} (\textsc{apass}-\textsc{auto}-\textsc{caus}-give) `rob people', but also the variant \forme{nɯ-sɤ-sɯ-kʰo} (\textsc{auto}-\textsc{apass}-\textsc{caus}-give) with the autive prefix  \forme{nɯ-} moved further away from the stem than the antipassive \forme{sɤ-} (§\ref{sec:antipassive.sA}).

Additional possible examples of fossilized autive prefixes include \japhug{nɯɴɢɤt}{part ways} (§\ref{sec:anticausative.direction}) and \japhug{mɟa}{take} (§\ref{sec:antipassive.t}).

\subsection{Historical relationship with other derivations} \label{sec:autoben.historical}
Further semantic evolution of the autive prefix with some verbs has led to the development of distinct grammatical categories, the vertitive \forme{nɯ-} (§\ref{sec:vertitive}) which remained formally similar to the autive, and the anticausative (§\ref{sec:anticausative}), which underwent phonological reduction to a non-concatenative alternation: the shift from unvoiced stops and affricatives to their corresponding prenasalized voiced counterparts.
 
\section{Vertitive} \label{sec:vertitive} 
 \is{vertitive}
The vertitive \forme{nɯ-}  is exclusively attested with a restricted set of motion (§\ref{sec:motion.verbs}) and manipulation verbs (§\ref{sec:manipulation.verbs}), indicated in \tabref{tab:vertitive}, expressing that the motion is directed back to the point of origin.
 
\begin{table}[h]
\caption{The vertitive prefix \forme{nɯ-} in Japhug} \label{tab:vertitive}
\begin{tabular}{lllllllll}
\lsptoprule
Base verb   & Derived verb  & \\
\midrule
\japhug{ɕe}{go} & \japhug{nɯɕe}{go back}  (home) & \\
\japhug{ɣi}{come} & \japhug{nɯɣi}{come back}  (home) & \\
\japhug{pʰɣo}{flee} & \japhug{nɯpʰɣo}{flee back}  (home) & \\
\japhug{tsɯm}{take away} & \japhug{nɯtsɯm}{take back}  (home) & \\
\japhug{ɣɯt}{bring} & \japhug{nɯɣɯt}{bring back}  (home) & \\
\japhug{no}{drive} (cattle) & \japhug{nɯno}{drive back}  (home) & \\
\japhug{zɣɯt}{arrive} & \japhug{nɯzɣɯt}{arrive back}  (home) & \\
\lspbottomrule
\end{tabular}
\end{table}

The vertitive meaning developed out of the autobenefactive function of the autive `take for oneself' $\rightarrow$ `take to one's home' $\rightarrow$ `take back home'.  However, the two prefixes are synchronically distinct, and can be combined together, as in example (\ref{ex:kunWnWGinW}); note that in this case the autive is generally realized as \forme{-n-} (§\ref{sec:autoben.position}).


\begin{exe}
	\ex \label{ex:kunWnWGinW}
	\gll   tɕe ɲɯ-tɯ-nɤm qʰe, tɕe ʑara ku-n-nɯ-ɣi-nɯ ŋu ɕi ɕ-ku-tɯ-nɤm ra? \\
	\textsc{lnk} \textsc{ipfv}:\textsc{east}-2-chase[III] \textsc{lnk} \textsc{lnk} \textsc{3pl} \textsc{ipfv}:\textsc{east}-auto-come.back-\textsc{pl} be:\textsc{fact} \textsc{qu} \textsc{ipfv}:\textsc{east}-2-chase[III]\\
	\glt `(And your cows, are they (still) like that), you let them out of the pen [in the morning], and  they come back home on their own, or do you have to chase them home?' (taRrdo conversation)
\end{exe}

All vertitive verbs in \tabref{tab:vertitive} have corresponding homophonous autive forms. Example (\ref{ex:WrJAlkhAB.nWtCu.jonWtsWm}) shows the use of the vertitive form of \japhug{tsɯm}{take away}, while (\ref{ex:tWci.kW.thanWtsWm}) illustrates its autive form. It is clear in the case of (\ref{ex:tWci.kW.thanWtsWm}) that \forme{nɯ-tsɯm} cannot be interpreted as `take back' (since a river flows in one direction and does not take back floating objects to its source).

\begin{exe}
\ex \label{ex:WrJAlkhAB.nWtCu.jonWtsWm}
\gll iɕqʰa rɟɤlpu ɯ-tɕɯ nɯ kɯ tɤɕime nɯ, ɯʑo ɯ-rɟɤlkʰɤβ nɯtɕu jo-nɯ-tsɯm qʰe \\
the.aforementioned king \textsc{3sg}.\textsc{poss}-son \textsc{dem} \textsc{erg} girl \textsc{dem} \textsc{3sg} \textsc{3sg}.\textsc{poss}-kingdom \textsc{dem}:\textsc{loc} \textsc{ifr}-\textsc{vert}-take.away \textsc{lnk} \\
\glt `The prince took the girl back (\textit{vertitive}) to his kingdom.' (140504 baixue gongzhu-zh)
\japhdoi{0003907\#S231}
\end{exe}

\begin{exe}
\ex \label{ex:tWci.kW.thanWtsWm}
\gll aʑɯɣ nɯ-nɯβde-t-a nɯ ɯ́-ŋu tɯ-ci kɯ tʰɯ-a-nɯ-tsɯm nɯ ɯ́-ŋu\\
\textsc{1sg}:\textsc{gen} \textsc{aor}-lose-\textsc{pst}:\textsc{tr}-\textsc{1sg} \textsc{dem} \textsc{qu}-be:\textsc{fact} \textsc{indef}.\textsc{poss}-water \textsc{erg} \textsc{aor}-3\flobv{}-\textsc{auto}-take.away \textsc{dem} \textsc{qu}-be:\textsc{fact} \\
\glt `Is it the one that I lost? Is it the one that the water took away (\textit{spontaneous})?' (140427 bianfu jingji he shuiniao-zh)
\japhdoi{0003836\#S28}
\end{exe}

Unlike the regular autive, but similarly to lexicalized autive verbs (§\ref{sec:autoben.lexicalized}), the vertitive prefix is located closer to the verb stem than the causative, as shown by the form \japhug{znɯɕe}{let go back} in (\ref{ex:pjWtaznWCe.jAG}).

\begin{exe}
\ex \label{ex:pjWtaznWCe.jAG}
\gll  tɯ-ɣjɤn pjɯ-ta-z-nɯ-ɕe jɤɣ ri,  \\
one-time \textsc{ipfv}:\textsc{down}-1\fl{}2-\textsc{caus}-\textsc{vert}-go be.allowed:\textsc{fact} \textsc{lnk} \\
\glt `I can let you go back home one time.' (150901 changfamei-zh) \japhdoi{0006352\#S161}
\end{exe}

\section{Abilitative} \label{sec:abilitative} 
 \is{abilitative}
The abilitative \forme{sɯ-/z-} prefix occurs on transitive verb bases. It is formally identical to the sigmatic causative, and possibly historically derived from it (§\ref{sec:abilitative.origin}).  As in the case of the causative, the allomorph \forme{z-} is found on polysyllabic verb bases whose first syllable has a sonorant initial (§\ref{sec:caus.z}). Since the abilitative only occurs on transitive verbs, there is no equivalent of the \forme{sɯɣ-} allomorph (§\ref{sec:caus.sWG}).

This prefix derives verbs expressing the ability of the transitive subject to perform the action of the base verb, as in \japhug{sɯndza}{be able to eat} from \japhug{ndza}{eat} (homophonous with the causative \japhug{sɯndza}{make/let eat}, `eat with') in (\ref{ex:mWnWsWndzaj}) and (\ref{ex:nWmAkAsWndza}). It seems to be a productive derivation, but abilitative verbs are rare in the corpus, except for the highly frequent verb \japhug{znɤɕqa}{be able to endure/resist} from \japhug{nɤɕqa}{endure}, `resist'.

\begin{exe}
\ex \label{ex:mWnWsWndzaj}
\gll aʑo kɯnɤ nɯ́-wɣ-mbi-a, mɯ-nɯ-sɯ-ndza-j tɕe tɕendɤre <dong> ntsɯ pɯ-βzu-t-a. \\
\textsc{1sg} also \textsc{aor}-\textsc{inv}-give-\textsc{1sg}  \textsc{neg}-\textsc{aor}-\textsc{abil}-eat-\textsc{1pl} \textsc{lnk} \textsc{lnk} freeze always \textsc{aor}-make-\textsc{pst}:\textsc{tr}-\textsc{1sg} \\
\glt `She gave [some edible ferns] to me, we could not eat [all of it], so I froze it.' (conversation140510)
\end{exe}

\begin{exe}
\ex \label{ex:nWmAkAsWndza}
\gll nɯ-mɤ-kɤ-sɯ-ndza nɯnɯ nɯ-kʰo ɯ-ŋgɯ nɯtɕu ɯ-pɯ tu-nɯ-pa-nɯ ɲɯ-ŋgrɤl, \\
\textsc{3pl}.\textsc{poss}-\textsc{neg}-\textsc{obj}:\textsc{pcp}-\textsc{abil}-eat \textsc{dem} \textsc{3pl}.\textsc{poss}-room \textsc{3sg}.\textsc{poss}-in \textsc{dem}:\textsc{loc} \textsc{3sg}.\textsc{poss}-keep(1) \textsc{ipfv}-\textsc{auto}-keep(2) \textsc{sens}-be.usually.the.case \\
\glt `[The mice gathered food] and would keep in their room [the food] that they are not able to eat.' (150818 muzhi guniang-zh)
\japhdoi{0006334\#S273}
\end{exe}

The abilitative can indicate an intrinsic (in)ability (the quantity of food that one can ingest in examples \ref{ex:mWnWsWndzaj} and \ref{ex:nWmAkAsWndza}), or a possibility or impossibility due to adverse external circumstances over which the subject has no control, such as the absence of buyers in (\ref{ex:mWjsWntsGe2}), the shortage of food in (\ref{ex:mAsWXsundZi}), or the intellectual difficulty of the problem in (\ref{ex:WBlu.mWtosWtCAt}).

 \begin{exe}
\ex \label{ex:mWjsWntsGe2}
 \gll   sɤnɤmmtsʰu kɯ kɤ-ntsɣe cʰɤ-ɣɯt ri mɯ́j-sɯ-ntsɣe ndɤre, \\
  \textsc{anthr} \textsc{erg} \textsc{obj}:\textsc{pcp}-sell \textsc{ifr}:\textsc{downstream}-bring \textsc{lnk} \textsc{neg}:\textsc{sens}-\textsc{abil}-sell \textsc{lnk} \\
\glt `Bsod.nams.mtsho brought them [to Mbarkham] to sell, but could not sell it.' (conversation, 14.05.10)
 \end{exe}
 
  \begin{exe}
\ex \label{ex:mAsWXsundZi}
 \gll   tɕe li nɯ-kɤ-ndza ɲɤ-me qʰe tɕe nɤ li, tɤ-rɟit ra mɤ-sɯ-χsu-ndʑi pjɤ-ɕti qʰe \\
 \textsc{lnk} again \textsc{3pl}.\textsc{poss}-\textsc{obj}:\textsc{pcp}-eat \textsc{ifr}-not.exist \textsc{lnk} \textsc{lnk} \textsc{add} again \textsc{indef}.\textsc{poss}-child \textsc{pl} \textsc{neg}-\textsc{abil}-feed-\textsc{du} \textsc{ifr}.\textsc{ipfv}-be.\textsc{aff}:\textsc{fact} \textsc{lnk} \\
\glt `They ran out of food, and were about to be unable to feed the children.' (160701 poucet2) \japhdoi{0006155\#S56}
  \end{exe}
  

 \begin{exe}
\ex \label{ex:WBlu.mWtosWtCAt}
 \gll   tɕendɤre ɯ-βlu ʑaʑa ʑo mɯ-to-sɯ-tɕɤt tɕe ʑɯmkʰɤm ʑo cʰɤ-rɯ-sɯso pjɤ-ra. \\
\textsc{lnk}  \textsc{3sg}.\textsc{poss}-trick early \textsc{emph} \textsc{neg}-\textsc{ifr}-\textsc{abil}-take.out \textsc{lnk} a.long.time \textsc{emph} \textsc{ifr}-\textsc{apass}-think \textsc{ifr}.\textsc{ipfv}-be.needed \\
\glt `He could not find a solution at first, and had to think for a long time.' (140425 ajimide1)
\japhdoi{0003778\#S18}
 \end{exe}
 
 This derivation can also be used to indicate the acceptance or reluctance of the subject to do the action, as in (\ref{ex:mWpjAsWBde.abil}). 
 
\begin{exe}
\ex \label{ex:mWpjAsWBde.abil}
 \gll  qaɟy ɯ-me nɯ kɯ, tɕe li ɯ-pi nɯra wuma ʑo mɯ-pjɤ-sɯ-βde ɲɯ-ŋu  \\
 fish \textsc{3sg}.\textsc{poss}-daughter \textsc{dem} \textsc{erg} \textsc{lnk} again \textsc{3sg}.\textsc{poss}-elder.sibling \textsc{dem}.\textsc{pl} really \textsc{emph} \textsc{neg}-\textsc{ifr}-\textsc{abil}-throw \textsc{sens}-be \\
 \glt `The mermaid was reluctant to abandon her elder sisters.' (150819 haidenver-zh) \japhdoi{0006314\#S287}
\end{exe}  
 
 As shown by examples (\ref{ex:mWnWsWndzaj}) to (\ref{ex:mWpjAsWBde.abil}), abilitative verbs are only attested in negative form in the corpus. For some if not most abilitative verbs, the presence of a negative prefix is a requirement (§\ref{sec:abilitative}). Non-negative forms can be elicited for some of them (\ref{ex:aBlu.tAsWtCAta}).

\begin{exe}
\ex \label{ex:aBlu.tAsWtCAta}
 \gll  a-βlu tɤ-sɯ-tɕa-t-a \\
  \textsc{1sg}.\textsc{poss}-trick \textsc{aor}-\textsc{abil}-take.out-\textsc{pst}:\textsc{tr}-\textsc{1sg} \\
\glt `I succeeded in finding a solution.' (elicited)
\end{exe} 
 
\subsection{Lexicalized abilitatives} \label{sec:abilitative.lexicalized}
There are a two lexicalized abilitative verbs with the reduced allomorph \forme{s-}: \japhug{spʰɯt}{can cut}\footnote{In Chinese, this verb in negative form is translated as \ch{切不动}{qiēbúdòng}{cannot cut} or \ch{咬不动}{yǎobúdòng}{cannot tear by chewing} . } from \japhug{pʰɯt}{take out, cut}, `take out' as in (\ref{ex:mWjsphWt}) (see also \ref{ex:WndzrW.mWjsphWt}, §\ref{sec:obligatory.negative}) and the com\-ple\-ment-taking verb \japhug{spa}{be able} (§\ref{sec:spa.verb}) from  \japhug{pa}{do}. In addition, the isolated verb \japhug{jqu}{be able to lift} might also be a lexicalized abilitative with an irregular allomorph of the prefix (§\ref{sec:j.abilitative}).

 \begin{exe}
\ex \label{ex:mWjsphWt}
 \gll mbrɯtɕɯ ki mɯ́j-mtɕoʁ tɕe mɯ́j-spʰɯt \\
 knife \textsc{dem}.\textsc{prox} \textsc{neg}:\textsc{sens}-be.sharp \textsc{lnk} \textsc{neg}:\textsc{sens}-can.cut \\
 \glt `This knife is not sharp, it cannot cut.' (elicited)
\end{exe}

Both \japhug{spa}{be able} and its base verb  \japhug{pa}{do} have cognates in all Gyalrongic languages, including Tangut (\tangut{𘘭}{0385}{.wjị}{2.60} `be able to' and  \tangut{𘃡}{5113}{.wji}{1.10} `do', see \citealt[86;255-256]{jacques14esquisse}), Khroskyabs (Wobzi \forme{fsó} `savoir faire', \citealt[475]{lai17khroskyabs}) and Stau (\forme{vzə} with metathesis). Although West Gyalrongic languages lack an abilitative derivation, the existence of this cognate set demonstrates that this derivation goes back  to at least proto-Gyalrongic, and has been lost in West Gyalrongic except in this lexicalized form.
 
\subsection{Historical origin} \label{sec:abilitative.origin}
 \is{abilitative!origin}  \is{grammaticalization!abilitative}
Although the abilitative derivation goes back at least to the common ancestor of Japhug and Tangut, it is nevertheless likely that it derives from the sigmatic causative (\citealt[190]{jacques15causative}).

\is{reanalysis} 
Although abilitative and causative derivations share little semantic commonalities, there are nevertheless potentially ambiguous sentences, where a \forme{sɯ-} prefix can be interpreted either as causative or as abilitative with very similar meaning, differing only in perspective. These ambiguous clauses may have been the pivot constructions allowing a reanalysis from causative to abilitative.

The main ambiguous construction between causative and abilitative occurs with the presentive meaning of the sigmatic causative in negative form (§\ref{sec:sig.caus.negation}). For instance, in example (\ref{ex:mAkAsWrqoR}), the verb \forme{sɯ-rqoʁ} (from \japhug{rqoʁ}{hug}) can be analyzed as an abilitative `be able to hug'.


 \begin{exe}
\ex \label{ex:mAkAsWrqoR}
\gll tɯrme laʁnɯlaχsɯm kɯnɤ mɤ-kɤ-sɯ-rqoʁ kɯ-fse kɯ-jpum ɲɯ-βze cʰa  \\
people two.or.three also \textsc{neg}-\textsc{inf}-\textsc{abil}-hug \textsc{sbj}:\textsc{pcp}-be.like \textsc{sbj}:\textsc{pcp}-be.thick \textsc{ipfv}-grow can\textsc{:fact} \\
\glt  `[The fir] can grow so thick that two or three people cannot hug [its trunk].' (08-tWrgi)
\japhdoi{0003464\#S6}
   \end{exe}

However, it is also possible to construe the meaning in a different way: `The fir can grow so thick that it prevents even two or three people from hugging (its trunk)', with a causative interpretation. This interpretation is possible due to the ambiguity of the scope of the negation of the causative, which generates the preventive meaning `prevent, hinder' in negative forms (§\ref{sec:sig.caus.negation}), from which a modal meaning `not able to' can be derived, with the causee reanalyzed as the transitive subject of the \forme{sɯ-} prefixed verb. The reanalysis of the causee as transitive subject is made possible in non-finite clauses by the fact that the causer can optionally take the ergative (§\ref{sec:ditransitive.causative}) and that with the additive focus marker \japhug{kɯnɤ}{also, even} the ergative cannot surface anyway (§\ref{sec:kWnA}), so that the surface ambiguity between causee and causer can only be resolved by person indexation. 


This hypothesis is made more plausible by the fact that, as discussed above, abilitative verbs almost always occur in negative form.

Another potential pivot construction between causative and abilitative is found with the instrumental use of the sigmatic causative (§\ref{sec:sig.caus.instrumental}). In (\ref{ex:chWtWsWtsxWB}) for instance, the \forme{sɯ-} prefix is an instrumental causative (`sew with') but the context also invites an abilitative interpretation (`be able to sew').

\begin{exe}
\ex \label{ex:chWtWsWtsxWB}
\gll  kɯki taqaβ ki, ɯ-xso taqaβ nɯ maʁ, nɤkinɯ, nɯfse taqaβ nɯ maʁ tɕe, nɯ rcanɯ, tɕʰi nɯ-tɯ-sɯso-t cʰɯ-tɯ-nɯ-sɯ-tʂɯβ kʰɯ  \\
\textsc{dem}.\textsc{prox} needle \textsc{dem}.\textsc{prox} \textsc{3sg}.\textsc{poss}-common needle \textsc{dem} not.be:\textsc{fact} \textsc{filler} like.that needle \textsc{dem} not.be:\textsc{fact} \textsc{lnk} \textsc{dem} \textsc{unexp}:\textsc{deg} what \textsc{aor}-2-think-\textsc{pst}:\textsc{tr} \textsc{ipfv}-2-\textsc{auto}-\textsc{caus/abil}-sew be.possible:\textsc{fact} \\
\glt `This needle is no ordinary needle, it is not a simple needle like that, [with it] you will be able to sew whatever you like.' (140508 benling gaoqiang de si xiongdi-zh)
\japhdoi{0003935\#S99}
\end{exe}
   
\section{Distributed action} \label{sec:distributed.action} 
 \is{distributed action}  \is{reduplication!distributed action}
The distributed action derivation has a double morphological exponence, combining a prefix \forme{nɤ-} with the partially reduplicated stem of the base verb. Partial reduplication applies to the last syllable of the stem if polysyllabic (§\ref{sec:partial.redp}), and the replicant takes the vowel \forme{-ɯ} by default, except in a few cases studied in §\ref{sec.distributed.action.l} and §\ref{sec.distributed.action.oR}. The distributed action derivation is not compatible with verbs that already have reduplicated forms or a \forme{nɤ-} or \forme{nɯ-} prefix. For instance, \japhug{nɯqambɯmbjom}{fly} lacks a distributed action form; to express this meaning, a serial verb construction (§\ref{sec:svc.manner.other}) with the distributed action derivation  \japhug{nɤɕɯɕe}{go around} (from the motion verb \japhug{ɕe}{go}) is used instead, as in (\ref{ex:junWqambWmbjomndZi.junACWCendZi}).

\begin{exe}
\ex \label{ex:junWqambWmbjomndZi.junACWCendZi}
\gll  ɯ-ʁar ra ko-tsʰoʁ-nɯ tɕe tɕendɤre, rɟɤlpu nɯ chonɤ aʁɤndɯndɤt ju-nɯqambɯmbjom-ndʑi tɕe ju-nɤɕɯɕe-ndʑi ra to-kʰɯ ɲɯ-ŋu. \\
\textsc{3sg}.\textsc{poss}-wing \textsc{pl} \textsc{ifr}-attach-\textsc{pl} \textsc{lnk} \textsc{lnk} king \textsc{dem} \textsc{comit} everywhere \textsc{ipfv}-fly-\textsc{du} \textsc{lnk} \textsc{ipfv}-\textsc{distr}:go-\textsc{du} \textsc{pl} \textsc{ifr}-be.possible \textsc{sens}-be \\
\glt `They attached wings on her back, and she became able to fly around everywhere with the king.' (150818 muzhi guniang-zh)
\japhdoi{0006334\#S477}
\end{exe}

 
This derivation is found with both intransitive and transitive verbs, as shown by the examples in \tabref{tab:distributed.action}, and does not affect valency. It expresses a repeated action, often (with the base verb has a motional meaning) with aimless motion, distributed spatially and/or temporally. It is semantically very close to the Chinese construction \zh{X来X去}  \forme{X lái X qù}, and distributed action derivation can often be translated by this construction: for instance \japhug{nɤrɟɯrɟɯɣ}{run around} closely corresponds to Chinese \zh{跑来跑去}  \forme{pǎo lái pǎo qù}.


The distributed action derivation preserves stem alternation (§\ref{sec:stem2}). The verb \japhug{nɤtɯti}{tell around} has the stem II \forme{-nɤtɯtɯt} as expected from the base verb \japhug{ti}{say} (stem II \forme{-tɯt}), and \japhug{nɤɕɯɕe}{go around} has the stem II \forme{anɤrɯri} with a prefixed \forme{a-} like the stem II \forme{-ari} of the base verb \japhug{ɕe}{go}.

\begin{table}
\caption{Examples of distributed action derivations} \label{tab:distributed.action}
\begin{tabular}{lllll}
\lsptoprule
Base verb & Derived verb \\
\midrule
\japhug{ŋke}{walk} & \japhug{nɤŋkɯŋke}{walk around} \\
\japhug{rɟɯɣ}{run} & \japhug{nɤrɟɯrɟɯɣ}{run around} \\
\japhug{mtsaʁ}{jump} & \japhug{nɤmtsɯmtsaʁ}{jump around} \\
\japhug{ɕe}{go} & \japhug{nɤɕɯɕe}{go around} \\
\midrule
\japhug{ɕar}{search} & \japhug{nɤɕɯɕar}{search around}  \\
\japhug{ndo}{take} & \japhug{nɤndɯndo}{carry around}  \\
\japhug{ti}{say} & \japhug{nɤtɯti}{tell around} (or `say many times') \\
\japhug{ɕtʰɯz}{turn towards} & \japhug{nɤɕtʰɯɕtʰɯz}{turn in all directions} \\
\japhug{ʁndɯ}{hit} & \japhug{nɤʁndɯʁndɯ}{hit repeatedly} \\
\japhug{tʰu}{ask} & \japhug{nɤtʰɯtʰu}{ask around} \\
\japhug{βɟi}{chase} & \japhug{nɤβɟɯβɟi}{chase around} \\
\lspbottomrule
\end{tabular}
\end{table}

The distributed action derivation can be semantically very close to the repetition of the verb with the additive linker \forme{nɤ}, and both often occur in the same contexts. In the case of motion verbs, repetition with alternation between the \textsc{eastwards} and \textsc{westwards} orientations (§\ref{sec:centripetal.centrifugal}) has the same `goal-less action' function as the distribution action derivation. For instance, the meaning of example (\ref{ex:korJWG.nA.YArJWG}) could be expressed with the verb \japhug{nɤrɟɯrɟɯɣ}{run around}.

\begin{exe}
\ex \label{ex:korJWG.nA.YArJWG}
\gll ko-rɟɯɣ nɤ ɲɤ-rɟɯɣ ʑo qʰe, \\
\textsc{ifr}:\textsc{east}-run \textsc{lnk} \textsc{ifr}:\textsc{west}-run \textsc{emph} \textsc{lnk}  \\
\glt `She ran around.' (150901 changfamei-zh)
\japhdoi{0006352\#S104}
\end{exe}

However, in the case of allative motion verbs (§\ref{sec:motion.verbs}), distributed action derivation and verb repetition can have a different meaning. The former generally expresses the absence of a specific goal as in (\ref{ex:junACWCe.mWpjAjAG}), and often co-occurs with the adverb \japhug{aʁɤndɯndɤt}{everywhere} (see \ref{ex:junWqambWmbjomndZi.junACWCendZi} above). The distributed action verb \forme{nɤɕɯɕe} is only compatible with the unspecified orientation prefixes, and cannot occur with any of the other six orientations (§\ref{sec:tridimensional.preverb}).

\begin{exe}
\ex \label{ex:junACWCe.mWpjAjAG}
\gll ɯʑo-sɯso ju-nɤɕɯɕe mɯ-pjɤ-jɤɣ \\
\textsc{3sg}-as.wish \textsc{ipfv}-\textsc{distr}:go \textsc{neg}-\textsc{ifr}.\textsc{ipfv}-be.allowed \\
\glt `[The nightingale] was not allowed to [fly] around freely.' (140519 yeying-zh) \japhdoi{0004040\#S102}
\end{exe}


On the contrary, verb repetition with the same orientation prefix (\ref{ex:YACe.nA.YACe}) (unlike \ref{ex:korJWG.nA.YArJWG}) conveys an idea of continuous and lengthy motion in one specific direction, a meaning that would be incompatible with that of the corresponding distributed action verb \japhug{nɤɕɯɕe}{go around}.

\begin{exe}
\ex \label{ex:YACe.nA.YACe}
\gll tɕendɤre ɲɤ-ɕe nɤ ɲɤ-ɕe tɕe tɕendɤre, tɕendi tɕe tɕe, tɕendi tɕendi tɕe tɕendɤre mtsʰu ci ɲɤ-k-ɤtɯɣ-ci. \\
\textsc{lnk} \textsc{ifr}:\textsc{west}-go \textsc{add} \textsc{ifr}:\textsc{west}-go \textsc{lnk} \textsc{lnk} west \textsc{lnk} \textsc{lnk} west west \textsc{lnk} \textsc{lnk} lake \textsc{indef} \textsc{ifr}-\textsc{peg}-meet-\textsc{peg} \\
\glt `He went towards the west (for a long time). Very far in the west, he found a lake.' (28-smAnmi) \japhdoi{0004063\#S87}
\end{exe}

With non-motional oriented verbs such as \japhug{ɕtʰɯz}{turn towards} (§\ref{sec:orienting.verbs}), the distributed action derivation does not imply translational motion, but expresses the idea of turning one's aim/look/body part towards all directions, as with the verb \forme{nɤɕtʰɯɕtʰɯz} in (\ref{ex:WCna.YAnACthWCthWz}).

\begin{exe}
\ex \label{ex:WCna.YAnACthWCthWz}
\gll  ɯ-ɕna ra ɲɤ-nɤɕtʰɯɕtʰɯz tɕeri, nɤki, tɤ-pɤtso ra kʰri ɯ-pa kɤ-kɤ-sɯ-ɤnbaʁ nɯ pjɤ-sɯχsɤl matɕi ɯ-di pjɤ-mnɤm tɕe, \\
\textsc{3sg}.\textsc{poss}-nose \textsc{pl} \textsc{ifr}-\textsc{distr}:turn.towards \textsc{lnk} \textsc{filler} \textsc{indef}.\textsc{poss}-child \textsc{pl} bed \textsc{3sg}.\textsc{poss}-under \textsc{aor}-\textsc{obj}:\textsc{pcp}-\textsc{caus}-hide \textsc{dem} \textsc{ifr}-discover \textsc{lnk} \textsc{3sg}.\textsc{poss}-smell \textsc{ifr}.\textsc{ipfv}-have.a.smell \textsc{lnk} \\
\glt `[The ogre] pointed his nose in all directions, and discovered the children that had been hidden under the bed because of the smell.' (160704 poucet4-v2)
\japhdoi{0006097\#S5}
\end{exe}

With verbs of speech such as \japhug{ti}{say} and \japhug{tʰu}{ask}, the distributed action derivation (respectively \japhug{nɤtɯti}{tell around} and \japhug{nɤtʰɯtʰu}{ask around}) generally implies a repeated activity (generally at different places) directed towards many people as in (\ref{ex:mAnAtWtia}) and (\ref{ex:YAnAthWthu}). These verbs can select dative recipients as their base verbs (§\ref{sec:ditransitive.indirective}), but those are rarely overt and if overt only generic nouns such as \japhug{tɯrme}{person} are possible.

\begin{exe}
\ex \label{ex:mAnAtWtia}
\gll  nɤ-smɯlɤm nɯ aʑo a-rpɣo ɣɯ-tʰɯ-lɤt tɕe, tɕe mucin mɤ-nɤtɯti-a ma,
nɯ maʁ qʰe, tu-nɤtɯti-a ŋu \\
\textsc{2sg}.\textsc{poss}-prayer \textsc{dem} \textsc{1sg} \textsc{1sg}.\textsc{poss}-lap \textsc{cisl}-\textsc{aor}:\textsc{downstream}-release \textsc{lnk} \textsc{lnk} at.all \textsc{neg}-\textsc{distr}:say:\textsc{fact}-\textsc{1sg} \textsc{lnk} \textsc{dem} not.be:\textsc{fact} \textsc{lnk} \textsc{tral}-\textsc{distr}:say-\textsc{1sg} be:\textsc{fact} \\
\glt `(When the time will come to choose your husband), put your offering (prayer) on my lap, and I will not say anything to anybody, otherwise I will tell everybody [about it].' (2005 Kunbzang)
\end{exe}

\begin{exe}
\ex \label{ex:YAnAthWthu}
\gll tɕendɤre ʑɯ\redp{}ʑimkʰɤm ʑo aʁɤndɯndɤt ʑo ɲɤ-nɤtʰɯtʰu tɕe ``qala ŋoj nɯ-ari" ntsɯ to-ti pjɤ-nɤtʰɯtʰu tɕe, \\
\textsc{lnk} \textsc{emph}\redp{}long.time \textsc{emph} eveywhere \textsc{emph} \textsc{ifr}-\textsc{distr}:ask \textsc{lnk} rabbit where \textsc{aor}:\textsc{west}-go[II] always \textsc{ifr}-say \textsc{ifr}-\textsc{distr}:ask \textsc{lnk} \\
\glt `[The bear] asked around everywhere for a very long time where the rabbit had gone.' (2011-13-qala)
\end{exe}

However, \forme{nɤtɯti} can also have a temporally protracted action meaning `speak for an (overly) long time' without the implication of more than one recipient, as in (\ref{ex:kAnWsWGndzita}), a sentence describing the plight of an informant assailed with questions by a linguist.

\begin{exe}
\ex \label{ex:kAnWsWGndzita}
\gll kɤ-nɤtɯti kɯ a-rqo ʑo kɤ-nɯ-sɯɣ-ndzi-t-a \\
\textsc{ifr}-\textsc{distr}:say \textsc{erg} \textsc{1sg}.\textsc{poss}-throat \textsc{emph} \textsc{aor}-\textsc{auto}-\textsc{caus}-be.hoarse-\textsc{pst}:\textsc{tr}-\textsc{1sg} \\
\glt `My voice has become hoarse (I have made my voiced become hoarse) because of speaking again and again.' (elicited)
\end{exe}

With the verb \japhug{tsʰɤt}{try}, the distributed action derivation \forme{nɤtsʰɯtsʰɤt} `test/try again and again' expresses repetition, as in (\ref{ex:CtonAtshWtshAt}) with first syllable reduplication indicating iterative coincidence (§\ref{sec:iterative.coincidence}).

\begin{exe}
\ex \label{ex:CtonAtshWtshAt}
\gll tɕendɤre tɕʰeme nɯ ɕ-to-nɤtsʰɯtsʰɤt ri, tɯ\redp{}ta-nɤtsʰɯtsʰɤt ʑo nɯ tɕʰeme nɯ kɯ laʁnɤlaʁ ʑo tu-ste pɯ-ɕti ɲɯ-ŋu. \\
\textsc{lnk} girl \textsc{dem} \textsc{tral}-\textsc{ifr}-\textsc{distr}:try \textsc{lnk} \textsc{iter}\redp{}\textsc{aor}:3\flobv{}-\textsc{distr}:try \textsc{emph} \textsc{dem} girl \textsc{dem} \textsc{erg} \textsc{idph}(III):with.ease \textsc{emph} \textsc{ipfv}-do.like[III] \textsc{pst}.\textsc{ipfv}-be.\textsc{aff}:\textsc{fact} \textsc{sens}-be \\
\glt `He has gone and tested the girl again and again, and each time he tested her, she answered correctly with ease.' (2005 tAwa kWcqraR)
\end{exe}

%pjɯ-nɤsɲɯsɲu dynamic
%nɤɕkhɯɕkho

Stative verbs, including adjectives, are usually incompatible with the distributed action derivation. The resulting form would be formally identical to the tropative (§\ref{sec:tropative}) with emphatic reduplication (§\ref{sec:emph.redp}). The only adjective with such a derivation is \japhug{sɲu}{be mad} (from \tibet{སྨྱོ་}{smʲo}{be crazy}), which yields \forme{nɤsɲɯɲu} `be a little crazy, do crazy things (intermittently)', attested in (\ref{ex:pjWnAsYWsYu}).\footnote{
In this translated example, the form \forme{pjɯ-nɤsɲɯsɲu} corresponds to Chinese \ch{疯疯癫癫}{fēngfēngdiāndiān}{a little mad} in the original. However, it seems that the derivation turns the stative verb into a dynamic one. } 

\begin{exe}
\ex \label{ex:pjWnAsYWsYu}
\gll pjɯ-nɤsɲɯsɲu ntsɯ pjɤ-ɕti tɕe  \\
\textsc{ipfv}-\textsc{distr}:be.mad always \textsc{ifr}.\textsc{ipfv}-be.\textsc{aff}:\textsc{fact} \textsc{lnk} \\
\glt `He always did [all sorts of] crazy things.' (150829 jidian-zh) \japhdoi{0006338\#S12}
\end{exe}

The distributed action derivation also has the sense of `do $X$ in disorderly fashion': the form \forme{nɤfsɯfse} (from the similative verb \japhug{fse}{be like}) can be interpreted as `act foolishly', as in (\ref{ex:YWtWnAfsWfse}). From this use, \forme{nɤfsɯfse} has developed the extended meaning `be pretentious', like that of the auto-evaluative derivation (§\ref{sec:autoevaluative}).

\begin{exe}
\ex \label{ex:YWtWnAfsWfse}
\gll tɕʰi ɲɯ-tɯ-nɤme ŋu, tɕʰi ɲɯ-tɯ-nɤfsɯfse ɲɯ-ŋu ma, nɤki jɤ-pʰɣo ma \\
what \textsc{sens}-2-make[II] be:\textsc{fact} what \textsc{sens}-2-\textsc{distr}:be.like \textsc{sens}-be \textsc{lnk} \textsc{filler} \textsc{imp}-flee \textsc{lnk} \\
\glt `What are you doing, what kind of foolish act are you doing, flee!' (2012 Norbzang)
\japhdoi{0003768\#S57}
\end{exe}

In addition to regular distributed action verbs, there is one example with the prefix \forme{rɤ-}: \japhug{rɤβʑɯβʑar}{cut into many pieces} from \japhug{βʑar}{cut}.

\subsection{Lexicalized distributed action verbs} \label{sec:distributed.action.lexicalized} 
A certain number of verbs have a forms that is similar to a distributed action derivation, combining a \forme{nɤ-} prefix with partial reduplication of the stem, but have no corresponding base verb with exactly the same stem. The intransitive \japhug{nɤrɯra}{look around} is a particularly good candidate to be analyzed as a fossilized derivation of this type, as its meaning exactly fits that of a distributed action form of a verb meaning `look'. It might be an irregular derivation from \japhug{ru}{look at} (§\ref{sec:orienting.verbs}), with unexplained vowel alternation (note that the expected $\dagger$\forme{nɤrɯru} does not exist).\footnote{Khroskyabs and Western Gyalrongic in general have  \forme{-a} reduplication \citep{lai13fuyin}, but since proto-Gyalrong \forme{*-a} is fronted in Khroskyabs, this Japhug pattern cannot be directly cognate.}

The transitive verbs \japhug{nɤkʰɯkʰrɯt}{drag along}, \japhug{nɤɕɯɕi}{drag along} and \japhug{nɤmɯma}{stroke}\footnote{The verb \japhug{nɤmɯma}{stroke} is homophonous with the emphatic reduplicated form of the transitive verb \japhug{nɤma}{do}. } are possible candidates for being analyzed as lexicalized distributed action verbs. The verb \forme{nɤɕɯɕi} might be related to \japhug{rɤɕi}{pull}, but for the other two no known root exist in the language, and if verbs such as \forme{*kʰrɯt} and \forme{*ma} did exist at an earlier stage, they have been lost at least in the Kamnyu dialect. Note however that \forme{nɤɕɯɕi} and \forme{nɤkʰɯkʰrɯt} `drag along' are orientable manipulation verbs (§\ref{sec:manipulation.verbs}), and express an action occurring in one specific direction, as in (\ref{ex:lunAkhWkhrWta}); if these verbs are indeed ancient distributed action derivations, this derivation possibly has the sense of `protracted action' (as in \ref{ex:kAnWsWGndzita}  above) rather than distributed action in the proper sense.

\begin{exe}
\ex \label{ex:lunAkhWkhrWta}
\gll tɕe ki a-tɤɲi ki lu-nɤkʰɯkʰrɯt-a tɕe aʑo lu-mɤku-a ŋu tɕe, a-qʰu lɤ-ɣi je tɕe, a-tɤɲi ɯ-jrɯ\redp{}jroʁ ʑo lɤ-ɣi je tɕe, \\
\textsc{lnk} \textsc{dem}.\textsc{prox} \textsc{1sg}.\textsc{poss}-staff \textsc{dem}.\textsc{prox} \textsc{ipfv}:\textsc{upstream}-drag-\textsc{1sg} \textsc{lnk} \textsc{1sg} \textsc{ipfv}:\textsc{upstream}-be.first-\textsc{1sg} be:\textsc{fact} \textsc{lnk} \textsc{1sg}.\textsc{poss}-after \textsc{imp}:\textsc{upstream}-come \textsc{sfp} \textsc{lnk} \textsc{1sg}.\textsc{poss}-staff \textsc{3sg}.\textsc{poss}-trace\redp{}\textsc{perlative} \textsc{emph} \textsc{imp}:\textsc{upstream}-come \textsc{sfp} \textsc{lnk} \\
\glt `I am going up there first, dragging this staff$_i$ [of mine] along, come after me and follow its$_i$ trace.' (2005 Kunbzang)
\end{exe}

On the other hand \japhug{nɤmɯma}{stroke} is used to express touching or groping (eg, in the dark) without specific direction, as in (\ref{ex:YWwGnAmWma.rRom}).

\begin{exe}
\ex \label{ex:YWwGnAmWma.rRom}
\gll ɯ-mat nɯ ɲɯ-rko tɕe ɲɯ́-wɣ-nɤmɯma tɕe rʁom.\\
\textsc{3sg}.\textsc{poss}-fruit \textsc{dem} \textsc{sens}-be.hard \textsc{lnk} \textsc{ipfv}-\textsc{inv}-stroke \textsc{lnk} be.rough:\textsc{fact}\\
\glt `Its fruit is hard and rough to the touch.' (12-ndZiNgri)
\japhdoi{0003488\#S41}
\end{exe}

The intransitive verb \japhug{nɤpʰɯpʰɯ}{beg} could superficially seem to be a lexicalized distributed action verb, but it is better to analyze it as a denominal verb (§\ref{sec:denom.nW}) from the inalienable noun \japhug{ɯ-pʰɯpʰɯ}{alms}, which selects the person receiving the alms (rather than the one giving them) as possessor, as shown by (\ref{ex:nAphWphW.kAGWt}) (see also §\ref{sec:biactantial.ipn}).

\begin{exe}
\ex \label{ex:nAphWphW.kAGWt}
\gll nɤʑo nɤ-pʰɯpʰɯ kɤ-ɣɯt aj a-ʁa ku-me \\
\textsc{2sg} \textsc{2sg}.\textsc{poss}-alms \textsc{inf}-bring \textsc{1sg} \textsc{1sg}.\textsc{poss}-free.time \textsc{prs}-not.exist \\
\glt `I don't have time to give you alms.'(2003kandZislama)
\end{exe}

The verb \japhug{nɤstɯstu}{cause trouble to} (\ref{ex:nWkWnAstWstu}) is formally the distributed action derivation from the verb of similative \japhug{stu}{do like} (§\ref{sec:ditransitive.secundative}), with a synchronically unpredictable meaning.

\begin{exe}
\ex \label{ex:nWkWnAstWstu}
\gll ʑara-stɯsti tɕe nɯ-kɯ-nɤstɯstu maŋe, \\
\textsc{3pl}-alone \textsc{lnk} \textsc{3pl}.\textsc{poss}-\textsc{sbj}:\textsc{pcp}-cause.trouble not.exist:\textsc{sens} \\
\glt `When alone [without their young], [the wild yaks] have no [predators].' (20-RmbroN) \japhdoi{0003560\#S47}
\end{exe}

 
\subsection{Irregular partial reduplication}   \label{sec:irregular.reduplication}
\is{reduplication!irregular} \is{irregularity!reduplication}
Partial reduplication with \forme{Cɯ-} replicant (§\ref{sec:partial.redp}) is not the only way of forming distributed action verbs. Suffixal reduplication in \forme{-lV} (§\ref{sec.distributed.action.l}) and prefixal reduplication in \forme{Coʁ-} and \forme{Cɯm-} (§\ref{sec.distributed.action.oR}) are also attested.

\subsubsection{Replicant in \forme{-lV}} \label{sec.distributed.action.l}
 The verbs in \tabref{tab:distributed.action.l} take a suffixed syllable \forme{-le} or \forme{-lu} (\japhug{nɤmɲo}{watch} \fl{} \forme{nɤ-mɲo-l\textbf{e}}), or a syllable in \forme{\trt{}lV} whose rhyme replicates that of the verb stem (\japhug{mbɣaʁ}{turn over} \fl{} \forme{nɤ-mbɣ\textbf{aʁ}-l\textbf{aʁ}}). This exceptional example of rhyme reduplication in Japhug has important consequences for phonological analysis (§\ref{sec:W.i.closed.syllables}). A similar type of reduplication is observed in pattern IV ideophones (§\ref{sec:ideo.IV}).

Apart from \japhug{ɕar}{search}, which has two alternative distributed action derivations \forme{nɤɕɯɕar} and \forme{nɤɕarlar} (both `search around, search in all directions'), the other verbs only occur with \forme{-lV} replicant variant. The verb \japhug{mtɕɯr}{turn} has two variants \forme{nɤmtɕɯrlɯr} and \forme{nɤmtɕɯrlu}.


\begin{table}
\caption{Examples of distributed action derivations with \forme{-lV} replicant} \label{tab:distributed.action.l}
\begin{tabular}{lllll}
\lsptoprule
Base verb & Derived verb \\
\midrule 
\japhug{mbɣaʁ}{turn over} (vi)&\japhug{nɤmbɣaʁlaʁ}{turn over here and there} (vi)\\
\japhug{ndʐaβ}{fall/roll} & \japhug{nɤndʐaβlaβ}{roll again and again} (or in all directions) \\
\japhug{ndʑaʁ}{swim} &\japhug{nɤndʑaʁlaʁ}{swim around} \\
\japhug{mtɕɯr}{turn} (vi)&\japhug{nɤmtɕɯrlɯr}{turn in all directions} (vi)\\
&\japhug{nɤmtɕɯrlu}{turn in all directions} (vi)\\
\japhug{nɤmɲo}{watch} (vl)&\japhug{nɤmɲole}{watch the scenery} (vi)\\
\midrule
\japhug{tʂaβ}{cause to fall/roll} & \japhug{nɤntʂaβlaβ}{cause to roll in all directions} \\
\japhug{pɣaʁ}{turn over} (vt)& \japhug{nɤpɣaʁlaʁ}{turn over here and there} (vt) \\
\japhug{ɕar}{search} & \japhug{nɤɕarlar}{search around}  \\
\lspbottomrule
\end{tabular}
\end{table}

With the labile verb \japhug{nɤmɲo}{watch}, this derivation has an antipassivizing effect, since the derived verb \japhug{nɤmɲole}{watch the scenery} is strictly intransitive.

This derivation can be applied to both the base verbs \japhug{tʂaβ}{cause to fall/roll}and \japhug{pɣaʁ}{turn over} and their anticausatives \japhug{ndʐaβ}{fall/roll} and \japhug{mbɣaʁ}{turn over}, respectively.

\subsubsection{Replicant in \forme{Coʁ-} or \forme{Cɯm-}} \label{sec.distributed.action.oR}
A handful distributed action verbs take a replicant other than the regular \forme{-ɯ}. The transitive verb \japhug{mpʰɯr}{wrap} yields \forme{nɤmpʰoʁmpʰɯr} `preserve (something fragile) by wrapping under several layers' with \forme{-Coʁ-} reduplicant (this rare type of reduplication is also attested in some passive forms, §\ref{passive.redp}).

The intransitive verb \japhug{nɤʁaʁ}{have a good time} has the derived form \forme{nɤʁɯmʁaʁ} `play around' with \forme{-Cɯm-} replicant, the only example of this type of reduplication in Japhug.
%rɤmɯthu
\subsection{Compatibilities with other derivations} \label{sec:distributed.action.other}
The distributed action derivation is not attested on verbs with prefixal derivations, including the sigmatic causative. However, the \forme{-lV} variant (§\ref{sec.distributed.action.l}) is found with a handful of anticausativized verbs (§\ref{sec:anticausative.other.derivations}).

Further derivations on distributed action verbs are verb rare. The verb \japhug{nɤmɲole}{watch the scenery} for instance can be causativized with the sigmatic prefix \forme{z\trt}, as in (\ref{ex:chAznAmYole}).

\begin{exe}
\ex \label{ex:chAznAmYole}
\gll kɤntɕʰaʁ ra cʰɤ-z-nɤmɲole \\
street \textsc{pl} \textsc{ifr}-\textsc{caus}-\textsc{distr}:watch \\
\glt `He [took] him to do sightseeing in the city.' (140511 alading-zh) \japhdoi{0003953\#S53}
\end{exe}


\section{Auto-evaluative} \label{sec:autoevaluative}
\is{auto-evaluative} \is{reflexive!auto-evaluative}  \is{tropative!auto-evaluative}
Like the distributed action derivation (§\ref{sec:distributed.action}), the auto-evaluative derivation has a double exponence: the prefix \forme{znɤ-} and verb stem partial reduplication. It takes a stative intransitive verb as input, and derives an intransitive verb meaning `think of oneself as $X$, pretend to be $X$'(where $X$ stands for the meaning of the base verb, always expressing a positive characteristic), as in \tabref{tab:autoevaluative}. Although similar to reflexivized tropatives (§\ref{sec:tropative}), auto-evaluative verbs have a derogatory meaning and imply pretentiousness and vanity or bragging.

When the base verb already has an auto-evaluative meaning (\japhug{χpa}{be proud}), the auto-evaluative derivation only adds the derogatory nuance (\japhug{znɤχpɯχpa}{be arrogant}).

\begin{table}
\caption{Examples of auto-evaluative derivations}
\label{tab:autoevaluative}
\begin{tabular}{lllll}
\lsptoprule
Base verb & Derived verb \\
\midrule
\japhug{mpɕɤr}{be beautiful} & \japhug{znɤmpɕɯmpɕɤr}{think of oneself as beautiful} \\
\japhug{χɕu}{be strong} & \japhug{znɤχɕɯχɕu}{think of oneself as strong} \\
\japhug{χpa}{be proud} & \japhug{znɤχpɯχpa}{be arrogant} \\
\japhug{pe}{be good} & \japhug{znɤjpɯjpe}{be full of oneself} \\
\lspbottomrule
\end{tabular}
\end{table}

The verb \japhug{znɤjpɯjpe}{be full of oneself} (from \japhug{pe}{be good}) has an irregular form with an inserted \forme{-j-} element  (the expected form would be $\dagger$\forme{znɤpɯpe}), and its meaning is slightly lexicalized `be full of oneself' (from `think of oneself as good').

\begin{exe}
\ex \label{ex:mAZW.Zo.tonZnAjpWjpe}
\gll nɯ-rɟɤlpu nɯ kɯ nɯra, maka, ɯʑo tɤ-kɤ-fstɤt ɣɯ ɯ-rju nɯra pjɤ-mtsʰɤm tɕe, mɤʑɯ ʑo to-znɤjpɯjpe. \\
\textsc{3pl}.\textsc{poss}-king \textsc{dem} \textsc{erg} \textsc{dem}:\textsc{pl} at.all \textsc{3sg} \textsc{aor}-\textsc{obj}:\textsc{pcp}-flatter \textsc{gen} \textsc{3sg}.\textsc{poss}-word \textsc{dem}:\textsc{pl} \textsc{ifr}-hear \textsc{lnk} even.more \textsc{emph} \textsc{ifr}-be.full.of.oneself \\
\glt `Hearing these words of praise, the king became even more full of himself than before.' (150830 afanti-zh) \japhdoi{0006380\#S89}
\end{exe}

Other examples of lexicalized auto-evaluative verbs include \japhug{znɤlɯli}{play the coquette} (from \japhug{li}{be spoiled}) and \japhug{znɤŋɯŋu}{be arrogant}, a verb built from the root of the copula \japhug{ŋu}{be}, whose original meaning was `be right' (§\ref{sec:lexicalized.subject.participle}). %忘乎所以


%nɤ-kɤ-nɤfsɯfse ra mɤ-ra
%你不要装模作样

\section{Attenuative reduplication} \label{sec:attenuative}
\is{reduplication!attenuative}
In verbal derivation, reduplication occurs to express reciprocal (§\ref{sec:redp.reciprocal}), distributed action (§\ref{sec:distributed.action}) and emphasis (§\ref{sec:emph.redp}). Combination of \forme{a-} prefix with partial reduplication of the last syllable of the stem is also found with an attenuative meaning in a few stative verbs of colour (\tabref{tab:attenuative}). Note that the replicated syllables takes the vowel \ipa{ɤ} instead of \ipa{ɯ} in \forme{a-ɣɤ\redp{}ɣrum} `be whitish' and \forme{a-pɤ\redp{}pɣi} `be greyish', and the absence of \forme{w} in  \forme{a-ɣɤ\redp{}ɣrum} (on the cluster \forme{wxt\trt}, see §\ref{sec:wC.clusters}).

 \begin{table} 
 \caption{Examples of attenuative reduplication} \label{tab:attenuative}
\begin{tabular}{lllll}
\lsptoprule
Base verb & Attenuative \\
\midrule
\japhug{wɣrum}{be white} & \japhug{aɣrɤɣrum}{be whitish} \\
\japhug{pɣi}{be grey} & \forme{apɣɤpɣi}, \japhug{apɤpɣi}{be greyish} \\
\japhug{qarŋe}{be yellow} & \japhug{aqarŋɯrŋe}{be yellowish} \\
\lspbottomrule
\end{tabular}
\end{table}
 
Since the base verbs in \tabref{tab:attenuative} can also be subjected to emphatic reduplication, we find minimal pairs like and \forme{a-ɣrɤ\redp{}ɣrum} `be whitish' and \forme{a-qarŋɯ\redp{}rŋe} `be yellowish' (\ref{ex:aqarNWrNe})  vs. \forme{wɣrɯ\redp{}wɣrum} `be very white' (\ref{ex:kWwGrWwGrum}) and \forme{qarŋɯ\redp{}rŋe} `be very yellow' (\ref{ex:kWqarNWrNe}), respectively (§\ref{sec:emph.redp}).

\newpage
\begin{exe}
\ex \label{ex:aqarNWrNe}
\gll ɯ-mɯntoʁ nɯra aɣrɤɣrum ɯ-ŋgɯz kɯnɤ aqarŋɯrŋe kɯ-fse \\
\textsc{3sg}.\textsc{poss}-flower \textsc{dem}:\textsc{pl} be.whitish:\textsc{fact} \textsc{3sg}.\textsc{poss}-inside:\textsc{loc} also be.yellowish:\textsc{fact} \textsc{sbj}:\textsc{pcp}-be.like \\
\glt  `Its flower is whitish, with a taint of yellowish.' (`it is like yellowish inside the whitish colour') (16-CWrNgo) \japhdoi{0003518\#S195}
\end {exe}

\begin{exe}
\ex \label{ex:kWwGrWwGrum}
\gll  tɕeri qro nɯnɯ wuma ʑo wɣrum, kɯ-wɣrɯ\redp{}wɣrum ʑo ŋu. \\
\textsc{lnk} pigeon \textsc{dem} really \textsc{emph} be.white:\textsc{fact} \textsc{sbj}:\textsc{pcp}-\textsc{emph}\redp{}be.white \textsc{emph} be:\textsc{fact} \\
\glt `The pigeon is very white.' (24-qro) \japhdoi{0003626\#S3}
\end {exe}

\begin{exe}
\ex \label{ex:kWqarNWrNe}
\gll ɯ-mɯntoʁ rca wuma ʑo mpɕɤr tɕe kɯ-qarŋɯ\redp{}rŋe ʑo ŋu. \\
\textsc{3sg}.\textsc{poss}-flower \textsc{unexp}:\textsc{deg} really \textsc{emph} be.beautiful:\textsc{fact} \textsc{lnk} \textsc{sbj}:\textsc{pcp}-\textsc{emph}\redp{}be.yellow \textsc{emph} be:\textsc{fact} \\
\glt `Its flower is very beautiful, it is very yellow.' (15-babW) \japhdoi{0003512\#S101}
\end {exe}

\section{Fossil affixes and marginal derivations} \label{sec:marginal.derivations}


\subsection{Volitional \forme{mɯ-} prefix} \label{sec:volitional.mW}
\is{volitionality!derivation}
The intransitive dynamic verb \japhug{mɯnmu}{move} contains a prefix \forme{mɯ\trt}, as shown by the existence of the bare root \forme{nmu} in the verb \japhug{nmu}{shake (of earthquakes)} \citep{jacques17volitional}, which is only found in collocation with the noun \japhug{waɟɯ}{earthquake} (\ref{ex:waJW.YAnmu}).


\begin{exe}
\ex \label{ex:waJW.YAnmu}
\gll waɟɯ ɲɤ-nmu \\
earthquake \textsc{ifr}-shake \\
\glt `There was an earthquake.' (elicited)
\end{exe}

The Limbu labile verb \forme{|munt|} `move' \citep{michailovsky02dico} is probably cognate to the Japhug root (\citealt[212]{jacques17pkiranti}, see also §\ref{sec:historical.phono} on the absence of coda in Japhug), and since \forme{mɯnmu} means `move' in general (for both animate and inanimate beings), the restriction of the base verb \forme{nmu} to earthquakes specifically cannot be an archaism. Rather, the ancestor of the base verb \forme{nmu} must have had a more general meaning `move' when the proto-form from which \forme{mɯnmu} originates was derived from it. The exact meaning of the ancestor of \forme{nmu}, and the function of the \forme{mɯ-} prefix in this verb are uncertain, but the following scenario can be proposed: the \forme{n-} element in \forme{nmu} could be a frozen allomorph of the autive (§\ref{sec:autobenefactive}), and the original meaning of \forme{nmu} could have been `move (spontaneously, without external agency)', and the derived form \forme{mɯnmu} would thus have had the meaning `move (voluntarily)': in Japhug, this verb can refer to volitional actions, and it is used for instance with the imperative and prohibitive (for instance in \ref{ex:manWtWmWnmu}).

\begin{exe}
\ex \label{ex:manWtWmWnmu}
\gll ma-nɯ-tɯ-mɯnmu ma mɤ-pʰɤn \\
\textsc{neg}-\textsc{imp}-2-move \textsc{lnk} \textsc{neg}-be.efficient:\textsc{fact} \\
\glt `Don't move, otherwise it won't work!' (2002 qala)
\end{exe}

In this interpretation, the original function of the \forme{mɯ-} in this example would be deriving a volitional verb out of a non-volitional one (itself rendered non-volitional by the autobenefactive prefix). 

In other Gyalrong languages, only cognates of the derived verb \forme{mɯnmu} are attested. The Bragbar Situ form \forme{vərmô} `bouger' \citep{zhangshuya20these} suggests that \forme{mɯ-} originates from \forme{*wə-} with regressive nasalization from the last syllable.   This invalidates a possible comparison with the volitional \forme{*m-} prefix reconstructed by \citet[55]{bs14oc} in Old Chinese.

 
\subsection{Applicative \forme{-t} suffix} \label{sec:applicative.t}
\is{applicative!suffix}
Beside the productive prefixal \forme{nɯ-} applicative (§\ref{sec:applicative}), Japhug has vestigial traces of a \forme{-t} applicative suffix, better attested in Kiranti and West Himalayish languages (see \citealt{michailovsky85dental}, \citet{jacques15derivational.khaling} and \citealt{jacques16ssuffixes} for comparative studies of this suffix). Only two examples of this derivation exist in Japhug: \japhug{ɣɯt}{bring} and \japhug{mdɯt}{be resolved to, be determined to}.\footnote{More examples of \forme{-t} applicative are found in Situ \citep{linyj17space, zhangshuya20these}, though all involving verbs of motion. }

The verb of manipulation \japhug{ɣɯt}{bring} derives from the motion verb \japhug{ɣi}{come}; the vowel alternation is regular as pre-Japhug \forme{*i} changes to \ipa{ɯ} in closed syllables. With a motion verb such as `come', the effect of the applicative (\ref{ex:bring1}) is similar to a causative  (\ref{ex:bring2}). 

\begin{exe}
\ex \label{ex:bring1}
\glt `come with X' $\rightarrow$ `bring'
\ex \label{ex:bring2}
\glt `cause X to come' $\rightarrow$ `bring'
\end{exe}

The transitive verb \japhug{mdɯt}{be resolved to, be determined to} is historically related to the verb \japhug{mdɯ}{live up to}, and constitutes another example of the \forme{-t} applicative, though it is less immediately obvious than in the case of \japhug{ɣɯt}{bring} because each of the verbs has undergone semantic specialization after the derivation took place.

 
The verb \forme{mdɯ} is semi-transitive (§\ref{sec:semi.transitive}), and takes as its semi-object the lifespan; it can be applied to plants, animals and humans, as shown by examples (\ref{ex:chWmdW}) and (\ref{ex:chWmdWa}). It selects the \textsc{downstream} series of directional prefixes (§\ref{sec:vertical.preverbs.time}).

 \begin{exe}
\ex \label{ex:chWmdW}
\gll tɕe nɯŋa ɯʑo nɯnɯ, sqamŋu-xpa jamar cʰɯ-mdɯ ɲɯ-ŋgrɤl\\
\textsc{lnk} cow \textsc{3sg} \textsc{dem} fifteen-year about \textsc{ipfv}-live.up.to \textsc{sens}-be.usually.the.case \\
\glt `A cow can live up to fifteen years.' (05-qaZo)
\japhdoi{0003404\#S127}
\end{exe}

 \begin{exe}
\ex \label{ex:chWmdWa}
\gll ``nɤʑo nɯ kʰrɯtsu-xpa a-tʰɯ-tɯ-mdɯ ra nɤ" to-ti ɲɯ-ŋu. tɕe ``aʑo kɯnɤ kʰrɯtsu cʰondɤre tɯ-rʑaʁ nɯnɯ cʰɯ-mdɯ-a ra" to-ti \\
\textsc{2sg} \textsc{dem} ten.thousand-year \textsc{irr}-\textsc{pfv}-2-live.up.to be.needed:\textsc{fact} \textsc{sfp} \textsc{ifr}-say \textsc{sens}-be \textsc{lnk} \textsc{1sg} also  ten.thousand-year \textsc{comit} one-day \textsc{dem} \textsc{ipfv}-live.up.to-\textsc{1sg} be.needed:\textsc{fact} \textsc{ifr}-say \\
\glt `He said: `May you live ten thousand years! I want to live one thousand years and one more day.' (150830 afanti-zh)
\japhdoi{0006380\#S64}
\end{exe}

The meaning `live until/up to' is however a semantic innovation in Japhug: its Situ cognate \forme{mdə́} means `reach' as a motion verb. Japhug has restricted the meaning of this verb to a very specific context.

The verb \japhug{mdɯt}{be resolved to, be determined to} is morphologically transitive, and can take as its object an infinitive complement as in (\ref{ex:chWmdWta}). It shares with \japhug{mdɯ}{live up to} the \textsc{downstream} directional prefixes (\forme{cʰɯ-}).

 \begin{exe}
\ex \label{ex:chWmdWta}
\gll aʑo kɯrɯ-skɤt kɤ-βzjoz nɯ cʰɯ-mdɯt-a ʑo ɕti \\
\textsc{1sg} Tibetan-language \textsc{inf}-learn \textsc{dem} \textsc{ipfv}-be.determined \textsc{emph} be.\textsc{aff}:\textsc{fact} \\
\glt `I am determined to learn Tibetan/Gyalrong.' (elicited)
\end{exe}

The precise meaning of \forme{mdɯt}  is to be determined to do something that one has confidence they can realize. If one accepts the idea that the original meaning of Japhug \japhug{mdɯ}{live up to} was `reach' as in Situ, the meaning `be determined to' of the verb \forme{mdɯt} has the same relationship to that of the base verb as English `reach for' (`reach for the stars') to the verb `reach', with a conative interpretation `try/strive to reach'.  The addition of the suffix \forme{-t} turns the semi-transitive (morphologically intransitive) \forme{mdɯ} into a transitive verb whose A corresponds to the S of the base verb. This applicative derivation from a semi-transitive verb is not unique in Japhug; the transitive verb \japhug{nɯrga}{like} from the verb \japhug{rga}{like} with the \forme{nɯ-} applicative is another similar example (§\ref{sec:applicative}).

\subsection{Antipassive \forme{-t} suffix} \label{sec:antipassive.t}
\is{antipassive!suffix}
The semi-transitive verb \japhug{βɟɤt}{obtain}, 
as proposed by \citet[310]{gong18these}, is related to the orientable manipulation verb  \forme{mɟa} `take (from)', a meaning illustrated by example (\ref{ex:tonWmJa}). The two verbs differ by vowel alternation, \forme{-t} suffixation and preninital nasalization.
 
\begin{exe}
\ex \label{ex:tonWmJa}
\gll  nɯnɯra ɣɯ nɯ-rte nɯ ci ɯ-qʰu ci ʑo to-nɯ-mɟa. to-nɯ-mɟa qʰe jo-tsɯm qʰe ɯ-pi ra tɯkaka nɯ-ku ɯ-taʁ pjɤ-ta. \\
\textsc{dem}.\textsc{pl} \textsc{gen} \textsc{3pl}.\textsc{poss}-hat \textsc{dem} one \textsc{3sg}.\textsc{poss}-after one \textsc{emph} \textsc{ifr}:\textsc{up}-\textsc{auto}-take  \textsc{ifr}:\textsc{up}-\textsc{auto}-take \textsc{lnk} \textsc{ifr}-take.away \textsc{lnk} \textsc{3sg}.\textsc{poss}-elder.sibling \textsc{pl} each \textsc{3pl}.\textsc{poss}-head \textsc{3sg}.\textsc{poss}-on \textsc{ifr}:\textsc{down}-put \\
\glt `He took their crowns one after the other and put them on the heads of each of his brothers.' (160705 poucet5-v2) 
\japhdoi{0006163\#S19}
\end{exe}

The transitive verb \forme{mɟa} can also mean `get, obtain' when used with the \textsc{downwards} orientation, as in (\ref{ex:pjWGmJa.ma.mWpWkWcha}) (§\ref{sec:preverb.gain}). Apart from Tshobdun \forme{mɟê}, cognates in other Gyalrongic languages (such as Zbu \forme{vɟéʔ}) lack a nasal preinitial, suggesting that the Japhug and Tshobdun forms result from fusion with the autive prefix (§\ref{sec:autobenefactive}, \citealt[310]{gong18these}). This constitutes an interesting exclusive common innovation shared by Japhug and Tshobdun. 

\begin{exe}
\ex \label{ex:pjWGmJa.ma.mWpWkWcha}
\gll tɯ-sŋi kɤ-mdi tu-kɯ-rɤma, <gongfen> ʁnɯ-skɤrma pjɯ́-wɣ-mɟa ma mɯ-pɯ-kɯ-cʰa, pɯ-kɯ-xtɕi.  \\
one-day \textsc{inf}-complete \textsc{ipfv}-\textsc{genr}:S/O-work labour.point two-cent \textsc{ipfv}-\textsc{inv}-take apart.from \textsc{neg}-\textsc{pst}.\textsc{ipfv}-\textsc{genr}:S/O-can \textsc{pst}.\textsc{ipfv}-\textsc{genr}:S/O-be.small \\
\glt `Working a complete day, I could only get two cents of labour points, as I was young.' (2010-09)
\end{exe}

 
The verb \forme{βɟɤt}, meaning `obtain' (something that everyone is looking for) is semi-transitive (§\ref{sec:semi.transitive}), optionally taking a semi-object or a complement clause as in (\ref{ex:mWpjABJAt}) and (\ref{ex:mWpWkABJAt}), but conjugated intransitively, as shown by the Aorist \forme{pɯ-βɟɤt} (\textsc{aor}-obtain) `he got (it)' (with a A-type preverb, §\ref{sec:transitivity.morphology}).

\begin{exe}
\ex \label{ex:mWpjABJAt}
\gll kɤ-ndza maka mɯ-pjɤ-βɟɤt.   \\
\textsc{inf}-eat at.all \textsc{neg}-\textsc{ifr}:\textsc{down}-obtain \\
\glt `He did not obtain anything to eat.' (qajdoskAt 2002) \japhdoi{0003366\#S104}
\end{exe}

\begin{exe}
\ex \label{ex:mWpWkABJAt}
\gll tɤ-tɕɯ stu kɯ-wxti nɯnɯ kɯ [nɯnɯ tɕʰeme nɯ mɯ-pɯ-kɤ-βɟɤt] nɯ wuma ʑo pjɤ-nɤ-sɤɣdɯɣ  \\
\textsc{indef}.\textsc{poss}-son most \textsc{sbj}:\textsc{pcp}-be.big \textsc{dem} \textsc{erg} \textsc{dem} girl \textsc{dem} \textsc{neg}-\textsc{aor}-\textsc{inf}-obtain \textsc{dem} really \textsc{emph} \textsc{ifr}-\textsc{trop}-be.unpleasant \\
\glt `The elder boy was upset that he did not get the girl.' (140513 shenqi de feitan-zh) \japhdoi{0003981\#S209}
\end{exe}

The verb \japhug{βɟɤt}{obtain} is not synchronically derived from \japhug{mɟa}{take}: both verbs come from an etymon reflected by Zbu \forme{vɟéʔ} `prendre, obtenir, enlever' \citep[310]{gong18these} whose expected Japhug form would be *\forme{βɟa}. Based on the comparative evidence in \citet[310--311]{gong18these}, this lost verb was transitive and had the same argument structure as Japhug \forme{mɟa}. 
 
Hence, the \forme{-t} suffix in \japhug{βɟɤt}{obtain} used to remove morphological transitivity, turning the transitive subject into a intransitive subject, and the and the object into an (optional) semi-object. Given the fact that some \forme{-t} codas in Japhug originate from earlier *\forme{-s} (§\ref{sec:suffixes}), it is possible that this \forme{-t} suffix is related to the reflexive/middle suffix attested in Kiranti, Nungish and West-Himalayish (reflected for instance by Khaling \forme{-si}, \citealt{jacques16si}), which has antipassive functions in many languages, including possibly Old Chinese \citep{jacques21antipass}.
 
 \subsection{Other detransitive prefixes} \label{sec:CWmthu}
 The transitive verb \japhug{tʰu}{ask} (with indirective alignment, §\ref{sec:ditransitive.indirective}), in addition to the regular antipassives \japhug{rɤtʰu}{ask questions} and \japhug{sɤtʰu}{ask in marriage} (§\ref{sec:antipassive.function}), has two isolated intransitive derived forms: \japhug{rɤmɯtʰu}{ask around} and  \japhug{ɕɯmtʰu}{ask a lot of questions} (\ref{ex:nAtWCWmthu}), from which the compound \japhug{ɕɯmtʰuspoʁ}{child who likes to ask a lot of question} is derived (with \japhug{spoʁ}{have a hole} as second element).
 
  \begin{exe}
\ex \label{ex:nAtWCWmthu}
\gll nɤʑo ndɤre nɤ-tɯ-ɕɯmtʰu nɯ! \\
\textsc{2sg} \textsc{lnk} \textsc{2pl}.\textsc{poss}-\textsc{nmlz}:\textsc{deg}-ask.a.lot.of.questions \textsc{sfp} \\
\glt `You really [like to] ask a lot of questions!' (elicited)
\end{exe}

It is possible that the \forme{-m(ɯ)-} element in these complex prefixes is historically related to the \forme{amɯ-} reciprocal and distributed property prefixes (§\ref{sec:amW.reciprocal}, §\ref{sec:distributed.amW}).


\subsection{\forme{rɤ-} prefix} \label{sec:rA.non.apass}
Some verbs have \forme{rɤ-} prefixes that can neither be analyzed as antipassive (§\ref{sec:antipassive.rA}) nor as denominal (§\ref{sec:denom.rA}) derivations, at least synchronically, and whose function is not clearly identifiable.

The verbs \japhug{rɤwum}{tidy up} and \japhug{rɤtsʰɤt}{try} are clearly derived from \japhug{wum}{gather} and \japhug{tsʰɤt}{try}, respectively. However, they cannot be analyzed as antipassives, since they are morphologically transitive like their base verbs.

The verb \forme{rɤwum} can mean `tidy up' as in (\ref{ex:nWkha.ra.turAwum}), taking as object a place (house or room) or `collect (and put in order)' as in (\ref{ex:nWkAmbi.torAwum}).

\begin{exe}
\ex \label{ex:nWkha.ra.turAwum}
\gll  tɤɕime nɯ kɯ kʰa ku-rɤʑi tɕe, nɯ-ndzɤtsʰi ra tu-βze, nɯ-kʰa ra tu-rɤwum pjɤ-ŋu \\
girl \textsc{dem} \textsc{erg} house \textsc{ipfv}-stay \textsc{lnk} \textsc{3pl}.\textsc{poss}-food \textsc{pl} \textsc{ipfv}-make[III]  \textsc{3pl}.\textsc{poss}-house \textsc{pl} \textsc{ipfv}-tidy.up \textsc{ifr}.\textsc{ipfv}-be \\
\glt (140504 baixuegongzhu-zh)
\japhdoi{0003907\#S97}
\end{exe}

\begin{exe}
\ex \label{ex:nWkAmbi.torAwum}
\gll rgɤtpu nɯ kɯ nɯra rŋɯl nɯ-kɤ-mbi cʰo laχtɕʰa nɯ-kɤ-mbi nɯra to-rɤwum qʰe, \\
old.man \textsc{dem} \textsc{erg} \textsc{dem}:\textsc{pl} silver \textsc{aor}-\textsc{obj}:\textsc{pcp}-give \textsc{comit} thing \textsc{aor}-\textsc{obj}:\textsc{pcp}-give \textsc{dem}:\textsc{pl} \textsc{ifr}-collect \textsc{lnk} \\
\glt  The old man collected the money and the things that [the people] had given [them].' (150906 toutao-zh)
\japhdoi{0006326\#S154}
\end{exe}

These meanings can also be conveyed by the base verb \japhug{wum}{gather}, as in (\ref{ex:kAwum.taqurnW}). However, \forme{wum} has a much wider range of meaning, including `fold, close' (of umbrellas, wings, see \ref{ex:YWqAt.nA.kuwum}, §\ref{sec:centripetal.centrifugal}) and `take as (disciple)' (\ref{ex:naslama.kukWwuma}, §\ref{sec:essive.abs}). The derived verb \forme{rɤwum} thus has a more specific and restricted use than its base verb.

\begin{exe}
\ex \label{ex:kAwum.taqurnW}
\gll nɯra kɯ ɯ-paχɕi ra kɤ-wum ta-qur-nɯ tɕe, \\
\textsc{dem} \textsc{erg} \textsc{3sg}.\textsc{poss}-apple \textsc{pl} \textsc{inf}-gather \textsc{aor}:3\flobv{}-help-\textsc{pl} \textsc{lnk} \\
\glt `They helped him to pick up his apples.' (pear story-\iai{Tshendzin})
\end{exe}

The meaning difference between \japhug{rɤtsʰɤt}{try} and \japhug{tsʰɤt}{try} is more difficult to ascertain, as both verbs can occur with a nominal object as in (\ref{ex:tarAtshAt}). In the corpus, when the derived verb \forme{rɤtsʰɤt} has an overt object, it is always however a complement clause, and it can have the sense of `compare' as in (\ref{ex:CWrAtshAttCi}).

\begin{exe}
\ex \label{ex:tarAtshAt}
\gll ɯʑo kɯ tɯ-ŋga ta-rɤtsʰɤt/ta-tsʰɤt \\
\textsc{3sg} \textsc{erg} \textsc{indef}.\textsc{poss}-clothes \textsc{aor}:3\flobv{}-try \\
\glt `He tried the clothes.' (elicited)
\end{exe}

\begin{exe}
\ex \label{ex:CWrAtshAttCi}
\gll [a-mbro ɯ-jme ɲɯ-zri ɕi, nɤki, nɤʑo nɤ-kɤrme ɲɯ-zri nɯ] ɕɯ-rɤtsʰɤt-tɕi ra \\
\textsc{1pl}.\textsc{poss}-horse \textsc{3sg}.\textsc{poss}-tail \textsc{sens}-be.long \textsc{qu} \textsc{filler} \textsc{2sg} \textsc{2sg}.\textsc{poss}-hair \textsc{sens}-be.long \textsc{dem} \textsc{tral}-try:\textsc{fact}-\textsc{1du} be.needed:\textsc{fact} \\
\glt `Let us try whether my horse's tail is longer, or your hair (let us see which, of my horse's tail and your hair, is the longest).' (2003 Kunbzang)
\end{exe}

However, the base verb is also possible with exactly the same type of complement clauses, as in (\ref{ex:YWcha.Ci.mWjcha.totshAt}).

\begin{exe}
\ex \label{ex:YWcha.Ci.mWjcha.totshAt}
\gll <chengming> kɯ, nɯnɯ qartsʰi nɯnɯ to-tsʰɤt. tɕe ɲɯ-cʰa ɕi mɯ́j-cʰa nɯra to-tsʰɤt tɕe  \\
\textsc{anthr} \textsc{erg} \textsc{dem} cricket \textsc{dem} \textsc{ifr}-try \textsc{lnk} \textsc{sens}-can \textsc{qu} \textsc{neg}:\textsc{sens}-can \textsc{dem}:\textsc{pl} \textsc{ifr}-try \textsc{lnk} \\
\glt `Chengming tried the cricket, tried whether it would be victorious [in cricket fights] or not.'  (150904 cuzhi-zh)
\japhdoi{0006322\#S130}
\end{exe}
 
The intransitive verb \japhug{rɤmpɕɤr}{make up} (\ref{ex:pjAraXtCinW.torAmpCArnW}) is derived from the stative verb \japhug{mpɕɤr}{be beautiful}.

\begin{exe}
\ex \label{ex:pjAraXtCinW.torAmpCArnW}
\gll pɣa tʰamtɕɤt nɯ, [...] pjɤ-ra-χtɕi-nɯ, nɯ-ku ra pjɤ-sɤɕɤt-nɯ tɕe to-rɤmpɕɤr-nɯ ɲɯ-ŋu. \\
bird all \textsc{dem} {  } \textsc{ifr}-\textsc{apass}-wash-\textsc{pl} \textsc{3pl}.\textsc{poss}-head \textsc{pl} \textsc{ifr}-comb-\textsc{pl} \textsc{lnk} \textsc{ifr}-make.up-\textsc{pl} \textsc{sens}-be \\
\glt `All birds ... washed, combed their hair and dressed up.' (tulao de wuya-zh)
\end{exe}

It has the emphatic reduplicated form \forme{rɤmpɕoʁ\redp{}mpɕɤr} with a rare \forme{-oʁ} replicated syllable, and  often occurs in the causative form \forme{zrɤmpɕɤr} `help $X$ making up' or `make up/dress up using', as in (\ref{ex:tukWzrAmpCAr}).

\begin{exe}
\ex \label{ex:tukWzrAmpCAr}
\gll tu-kɯ-z-rɤmpɕɤr-tɕi ra \\
\textsc{ipfv}-2\fl{}1-\textsc{caus}-make.up-\textsc{1du} be.needed:\textsc{fact} \\
\glt `Help us dressing up and making up.' (140504 huiguniang-zh) \japhdoi{0003909\#S67}
\end{exe}
 
This use of the \forme{rɤ-} prefix could be analyzed as a quasi-reflexive `make oneself beautiful', reminiscent of the case of the antipassive \japhug{raχtɕi}{wash} (vi) (see \ref{ex:pjAraXtCinW.torAmpCArnW} and §\ref{sec:antipassive.reflexive}), though it cannot be analyzed as an antipassive or a reflexive since the base verb is intransitive.

The transitive verb \japhug{rɤɣrɯɣ}{cherish} comes from the intransitive \japhug{rɯɣ}{be precious} with a tropative meaning (`consider to be precious', §\ref{sec:tropative.other.construction}). The derivational prefix here \forme{rɤɣ\trt}, with an intrusive velar fricative (on which see §\ref{sec:caus.sWG} and the references therein).

%rɤɣlɤn tɯ-lɤn

\subsection{\forme{ɣɯ-/ɣɤ-} prefix  }  \label{sec:GA.non.denom}
A few verbs have \forme{ɣɯ-} or \forme{ɣɤ-} prefixes that can neither be analyzed as  facilitative \forme{ɣɤ-} (§\ref{sec:facilitative.GA}),  as causative (§\ref{sec:velar.causative}), as denominal  (§\ref{sec:denom.tr.GA}) nor as deideophonic derivations (§\ref{sec:voice.deideophonic}). 

First, the transitive verb \japhug{ɣɤtɕɤt}{select from}, more specifically `choose/select (someone) from a group of people' (in particular, as a leader),\footnote{This verb has a Tshobdun cognate \forme{wɐtʃet}$_{II}$  `select' \citep[209]{jackson19tshobdun}, and this derivation thus goes back at least to their common ancestor.
} is derived from \japhug{tɕɤt}{take out}. It selects as object the person that is chosen, and also takes an essive adjunct  (§\ref{sec:essive.abs}) describing the office/position of the chosen person. In example (\ref{ex:nWNgumdZWG.taGAtCAtnW}), the verb \forme{ɣɤtɕɤt} occurs in a finite relative clause (§\ref{sec:finite.relatives}) whose relativized element is the object. The noun \forme{nɯ-ŋgumdʑɯɣ} `their leader' inside the relative is the essive adjunct.

\begin{exe}
\ex \label{ex:nWNgumdZWG.taGAtCAtnW}
\gll tɕe nɯ <faliedong> nɯnɯ, [nɯnɯ nɯ-ŋgumdʑɯɣ ta-ɣɤtɕɤt-nɯ] kɯ <chake> nɯ ko-sɯ-βraʁ. \\
\textsc{lnk} \textsc{dem}  \textsc{anthr} \textsc{dem} \textsc{dem} \textsc{3pl}.\textsc{poss}-leader \textsc{aor}:3\fl{}3-choose-\textsc{pl} \textsc{erg}  \textsc{anthr} \textsc{dem} \textsc{ifr}-\textsc{caus}-attach \\
\glt `Feridun, the one they had chosen as their leader, had Zohak attached.' (140514 xiee de shewang-zh)
\japhdoi{0003994\#S99}
\end{exe}

Its reflexive form \japhug{ʑɣɤɣɤtɕɤt}{volunteer} (to go and go something) has a meaning that is not completely predictable from that of the base verb.

Second,  \japhug{ɣɤlɤt}{lock} is related to \japhug{lɤt}{release}, which can mean `lock' when occurring in collocation with the noun \japhug{sɤcɯ}{key} (\ref{ex:sAcW.malAt2}, §\ref{sec:lexicalized.oblique.participle}, §\ref{sec:passive}).

Third,  the verb of perception \japhug{ɣɯχsɤl}{realize} (§\ref{sec:tso.sWXsAl}) originates from \japhug{χsɤl}{be clear} (itself from \tibet{གསལ་}{gsal}{clear}), with a quasi-tropative meaning  `clearly perceive that $X$'.

It is possible that these three transitive verbs originally were denominal verbs from deverbal nouns such as the bare action nominals (§\ref{sec:bare.action.nominals}), but there is no evidence for the putative nouns from which these verbs could have been derived.

\subsection{\forme{a-} prefix  } \label{sec:a.non.passive.denominal}
In addition to the passive \forme{a-} prefix (§\ref{sec:passive}), the reduplicated reciprocal (§\ref{sec:redp.reciprocal}) and the denominal stative \forme{a-} (§\ref{sec:denom.a}) derivations, we find examples of \forme{a-} prefixes with unidentified function in the following examples.


The stative verb \japhug{amtɕoʁ}{be pointy} appears to be derived from \japhug{mtɕoʁ}{be sharp}. Both are stative verbs, and are obviously close semantically (both can take for instance \japhug{mbrɯtɕɯ}{knife} as intransitive subject, as in \ref{ex:mWjsphWt}, §\ref{sec:abilitative.lexicalized}). 

The verb \japhug{acʰɤt}{be $X$ years apart} is either plain intransitive, taking the number of years as subject (\ref{ex:kWmNupArme.achAt}), or semi-transitive, selecting the years as semi-object (\ref{ex:kWmNupArme.achAttCi})  (see also \ref{ex:RnWpArme.machAti}, §\ref{sec:raNri}).

 

\begin{exe}
\ex \label{ex:kWmNupArme.achAt}
\begin{xlist}
\ex 
\gll tɕiʑo tɕi-pɤrtʰɤβ kɯmŋu-pɤrme acʰɤt \\
\textsc{1du} \textsc{1du}.\textsc{poss}-between five-years differ.in.age:\textsc{fact} \\
\ex \label{ex:kWmNupArme.achAttCi}
\gll tɕiʑo kɯmŋu-pɤrme acʰɤt-tɕi \\
\textsc{1du} five-years differ.in.age:\textsc{fact}-\textsc{1du} \\
\glt `We have a five year difference.' (elicited)
\end{xlist}
\end{exe}

It could potentially be analyzed as a denominal verb in \forme{a-} (§\ref{sec:denom.a}) from a lost noun \forme{*cʰɤt} borrowed from Tibetan \tibet{ཁྱད་}{kʰʲɤt}{difference}. While the Tibetan origin of the root is beyond doubt, it is also possible that this verb is derived by the \forme{a-} prefix from the semi-transitive \japhug{cʰɤt}{differ by} (itself from Tibetan), which selects as semi-object not a characteristic other than age, for example the quantity of fern eaten in (\ref{ex:chAtnW}).
\largerpage
\begin{exe}
\ex \label{ex:chAtnW} 
\gll kɯ-dɤn kɯ-mɲi ci cʰɤt-nɯ ma nɯ kɯ-fse rcanɯ pakuku ʑo kɯ-dɤn ʑo ɲɯ-ɕar-nɯ.  \\
\textsc{sbj}:\textsc{pcp}-be.many \textsc{sbj}:\textsc{pcp}-be.few \textsc{indef} differ:\textsc{fact}-\textsc{pl} \textsc{lnk} \textsc{dem} \textsc{sbj}:\textsc{pcp}-be.like \textsc{unexp}:\textsc{deg} every.year \textsc{emph} \textsc{sbj}:\textsc{pcp}-be.many \textsc{emph} \textsc{ipfv}-search-\textsc{pl} \\
\glt `They differ in that some [eat fern] a lot or fewer, but they search a lot [of fern] every year.' (conversation 140510)
\end{exe}

The verb \japhug{cʰa}{can}, `be able', `be fine' has two derived forms in \forme{a-}: \japhug{acʰɯcʰa}{be capable} (see \ref{ex:hundred.and.eight}, §\ref{sec.hundred.plus}) with reduplication \japhug{acʰɤla}{be capable} with suffixed \forme{trt{}lV} replicated syllable (§\ref{sec.distributed.action.l}). The meaning of this derivation is both emphatic and antipassive-like: unlike the base verb \japhug{cʰa}{can}, which can take a complement clause as semi-object (§\ref{sec:inf.complementation}, §\ref{sec:TAM.finite}), \forme{acʰɯcʰa} and \forme{acʰɤla} are strictly intransitive stative verbs.
 

\subsection{Abilitative \forme{j-} prefix  } \label{sec:j.abilitative}
\is{abilitative}
The transitive verb \japhug{jqu}{be able to lift} (\ref{ex:mAjqu}) has an intrinsically abilitative meaning. This unusual property suggests that it may be the remnant of a lexicalized abilitative verb whose base \forme{*qu} `lift' was lost, and that the \forme{j-} preinitial was an abilitative prefix, possibly an irregular allomorph of the \forme{sɯ-} abilitative (§\ref{sec:abilitative}), reminiscent of the \forme{j-} allomorph of the sigmatic causative (§\ref{sec:caus.j}).

\begin{exe}
\ex \label{ex:mAjqu}
\gll  pɤjkʰu ɯ-ku mɤ-jqe \\
yet \textsc{3sg}.\textsc{poss}-head \textsc{neg}-be.able.to.lift[III]:\textsc{fact} \\
\glt  `[The baby] is not yet able to lift up his head.' (elicited)
\end{exe}

\subsection{Prenasalization} \label{sec:fossil.prenasalization}
\is{prenasalization}
In addition to the anticausative derivation (§\ref{sec:anticausative}), prenasalization alternation is found in an isolated pair of intransitive verbs: \japhug{sqlɯm}{collapse} (of the ground)\footnote{The verb \forme{sqlɯm} can for instance express the collapse of the ground under the weight of an object or person, as in (\ref{ex:WBrAsqlWm}) in §\ref
{sec:WBrA.functions}. } and \japhug{arɴɢlɯm}{be caved in}  (\ref{ex:YArNGlWm}).
  
\begin{exe}
\ex \label{ex:YArNGlWm}
\gll ɯ-tʰoʁ ɲɯ-ɤrɴɢlɯm \\
\textsc{3sg}.\textsc{poss}-ground \textsc{sens}-be.caved.in \\
\glt `The ground is caved in.' (elicited)
\end{exe} 

The form \forme{arɴɢlɯm} derives from \forme{sqlɯm} by addition of a prefix \forme{a-} and prenasalization of the uvular stop \ipa{q} to \ipa{ɴɢ}. In addition,  the \ipa{s} was rhotacized to \ipa{r} as result of voicing: the combination \forme{zɴɢ} is only attested across syllable boundaries in Japhug, as in the  plant name \forme{razɴɢu},\footnote{In this word, note also that a form such as $\dagger$\forme{rarɴɢu} would violate another phonotactic constraint (§\ref{sec:rhotic.dissimilation}). } never in an onset, and this example suggest that a sound change \forme{*sɴɢ} \fl{} \forme{rɴɢ} took place.

The combination of prenasalization with \forme{a-} in \forme{arɴɢlɯm} appears to have a resultative stative meaning, as opposed to the dynamic verb \forme{sqlɯm}. It is possible that the prenasalization reflects a trace of the autive prefix (§\ref{sec:autoben.spontaneous}).

\subsection{Comparative derivation?} \label{sec:mna.sna}
\is{comparative}
The pair of stative verbs \japhug{sna}{be good, be worthy} and \japhug{mna}{be better} are possibly historically related, sharing a common root \forme{-na} with different prefixes \forme{s-} and \forme{m-} (or \forme{*w\trt}, with nasalization, §\ref{sec:causative.m}, §\ref{sec:NC.clusters}) prefixes. If genuine, this etymological relationship is not synchronically obvious, and a detailed description of the synchronic meanings of these verbs is necessary.

The verb \forme{sna} can either mean `be kind, be generous' as in (\ref{ex:ataR.kusna}) (see also \ref{ex:kW.toZGApa}, §\ref{sec:semi.transitive}), `be pleasant' (example \ref{ex:tuwGnAkhAzNganW}, §\ref{sec:applicative.lexicalized}) or `be worthy, be fit to', as in (\ref{ex:WkWndza.mAtWsna}) and (\ref{ex:kAndza.mAsna}) (also \ref{ex:kWnA.mAsna}, §\ref{sec:kWnA}). In the third case, it selects a semi-object, which can either be a noun or participle (\ref{ex:WkWndza.mAtWsna}), or an infinitival complement clause (\ref{ex:kAndza.mAsna}).

 
\begin{exe}
\ex \label{ex:ataR.kusna}
\gll rɟɤlpu ri a-taʁ wuma ku-sna, \\
king also \textsc{1sg}.\textsc{poss}-on really \textsc{prs}-be.kind \\
\glt `The king is very kind with me.' (2002 qaCpa)
\end{exe}

\begin{exe}
\ex \label{ex:WkWndza.mAtWsna}
\gll lɤ-tɕɤt ma nɤʑo a-mtɕʰot ɯ-kɯ-ndza mɤ-tɯ-sna \\
\textsc{imp}:\textsc{upstream}-take.out \textsc{lnk} \textsc{2sg} \textsc{1sg}.\textsc{poss}-offering \textsc{3sg}.\textsc{poss}-\textsc{sbj}:\textsc{pcp}-eat \textsc{neg}-2-be.worthy:\textsc{fact} \\
\glt `Spit it out, you are not worthy of being the one eating my offering.' (2014 kWlAG)
\end{exe}

\begin{exe}
\ex \label{ex:kAndza.mAsna}
\gll tɯrme kɤ-ndza mɤ-sna\\
person \textsc{inf}-eat \textsc{neg}-be.fit:\textsc{fact} \\
\glt `It is unfit for people to eat.' (12-Zmbroko) \japhdoi{0003490\#S93}
\end{exe}

The range of meanings of \forme{mna} only partially overlaps with that of \forme{sna}. First, it can `feel better, heal', taking either the disease/wound (\ref{ex:nACqhe.WYWmna}) (see also example \ref{ex:tAkAzbGaR}, §\ref{sec:lexicalized.subject.participle}), or the person (or body part) afflicted by it (\ref{ex:WYWtWmna}) as subject. Note that in (\ref{ex:nACqhe.WYWmna}), the verb is in the \textsc{3sg} form, indexation as intransitive subject the noun \forme{nɤ-ɕqʰe} `your cough' rather than the \textsc{2sg}, showing that this noun cannot be analyzed as an essive adjunct (§\ref{sec:essive.abs}).

\begin{exe}
\ex \label{ex:nACqhe.WYWmna}
\gll nɤ-ɕqʰe ɯ-ɲɯ́-mna? \\
\textsc{2sg}.\textsc{poss}-cough \textsc{qu}-\textsc{sens}-be.better \\
\glt `Is your cough getting better?' (many attestations)
\end{exe}

\begin{exe}
\ex \label{ex:WYWtWmna}
\gll ɯ-ɲɯ́-tɯ-mna \\
\textsc{qu}-\textsc{sens}-2-be.better \\
\glt `Are you feeling better?' (smAnmi 2003.2)
\end{exe}

Second, \forme{mna} occurs in comparative constructions (§\ref{sec:comparison}) with the standard marker \forme{sɤz(nɤ)} (§\ref{sec:comparative}) or the comparee marker \forme{kɯ} (§\ref{sec:comparee.kW}), as in (\ref{ex:sAz.WYWmna}) (see also \ref{ex:tWZo.tWmtChi}, §\ref{sec:indef.genr.poss}). By contrast, \forme{sna} is not attested in comparative constructions in the whole corpus.

\begin{exe}
\ex \label{ex:sAz.WYWmna}
\gll  nɯfse pjɯ-kɤ-fɕɤt sɤz ɯ-ɲɯ́-mna? \\
like.that \textsc{ipfv}-\textsc{inf}-tell \textsc{comp} \textsc{qu}-\textsc{sens}-be.better \\
\glt `Is it better [to explain how to weave with video] than simply by telling it like that (without video)?' (vid-20140429090403) \japhdoi{0003776\#S161}
\end{exe}

The participle \forme{kɯ-mna} is lexicalized in the sense of `leader, chief' (example \ref{ex:tCaXpa.ra.GW.nWkWmna}, §\ref{sec:subject.participle.possessive}).

The data above suggest that \forme{mna} originally was a lexicalized comparative (`be better') of \forme{sna} (`be good'), and that these two verbs have undergone distinct semantic specialization (`be better' $\Rightarrow$ `feel better, heal' vs. `be good' $\Rightarrow$ `be worthy, be fit'). The morphological structure of these verbs is however elusive: the \forme{s-} element in \forme{sna} is possibly related to the proprietive \forme{sɤ-} (§\ref{sec:proprietive}), but the \forme{m-} prefix in \forme{mna} is isolated in Japhug. It could originate from \forme{*w-} with regressive nasalization from the \forme{n-} of the root.

\subsection{Reduplication} \label{sec:redp.voice}
\is{reduplication!derivation}
Reduplication occurs as a secondary exponent of several verbal derivations, including  Distributed action (§\ref{sec:distributed.action}), Reciprocal (§\ref{sec:redp.reciprocal}), and is also found sporadically with other derivations (for instance with the antipassive, §\ref{sec:antipassive.redp}).  
 
In addition, reduplication is also attested by itself as the only marking of a valency-changing derivation in the case of \japhug{rɯru}{guard}, which derives from the intransitive verb \japhug{ru}{look at}. The base verb \forme{ru} selects a goal in the dative or with locative marking (§\ref{sec:intr.goal}, §\ref{sec:orienting.verbs}), while \forme{rɯru} is transitive (as shown by stem alternation in \ref{ex:paRtshi.YWnWre}) and the entity taken care of by the subject is encoded as object. 


\begin{exe}
\ex \label{ex:paRtshi.YWnWre}
\gll tɕe paʁtsʰi ɲɯ-rɯre pjɤ-ŋu ri, \\
\textsc{lnk} hogwash \textsc{ipfv}-take.care[III]:\textsc{fact} \textsc{ifr}.\textsc{ipfv}-be \textsc{lnk} \\
\glt `While he was taking care of the hogwash....' (2014-kWlAG)
\end{exe}

The compound noun \japhug{βɣɤru}{miller} includes as second element the non-redupli\-ca\-ted variant of the same root \forme{-ru} with the same meaning as that of \japhug{rɯru}{guard} (the first element is from the noun \japhug{βɣa}{mill}, see §\ref{sec:object.verb.compounds}).

\subsection{Vowel alternation} \label{sec:pWrndeta}
\is{vowel!alternation}
%遭殃
The rare verb \japhug{rnde}{get into trouble} is only attested in the expression in (\ref{ex:pWrndeta}). The \forme{-t-} suffix shows that this verb is morphologically transitive (§\ref{sec:transitivity.morphology}), despite being unable to take an overt object. 

\begin{exe}
\ex \label{ex:pWrndeta}
\gll kɤ-rndu sɤznɤ pɯ-rnde-t-a  \\
\textsc{inf}-obtain \textsc{comit} \textsc{aor}-get.into.trouble-\textsc{pst}:\textsc{tr}-\textsc{1sg} \\
\glt `I not only did not obtain anything, but in addition I got into trouble.' (elicited)
\end{exe}

The co-occurrence of \forme{rnde} with \japhug{rndu}{obtain} in the expression exemplified by (\ref{ex:pWrndeta}) is probably not only a matter of euphony, but also of morphological relationship (a \textit{figura etymologica}). The verb \forme{rnde} is invariable, but its stem is identical to the stem III (§\ref{sec:stem3.form}) of \forme{rndu} as in (\ref{ex:kro.Zo.YWrnde}). 

\begin{exe}
\ex \label{ex:kro.Zo.YWrnde}
\gll ɯʑo stɯsti ʑo tu-ɕe qʰe, tɕe kʰro ʑo ɲɯ-rnde kʰi. \\
\textsc{3sg} alone \textsc{emph} \textsc{ipfv}:\textsc{up}-go \textsc{lnk} \textsc{lnk} a.lot \textsc{emph} \textsc{sens}-obtain[III] \textsc{hearsay} \\
\glt `She goes [up there in the mountain], and found a lot of mushrooms (they say).' (conversation 14-05-10)
\end{exe}

It is possible to suppose that the verb \forme{rnde} is a backformation from the stem III of \japhug{rndu}{obtain}. A hypothesis of this type is likely in the case of the defective intransitive verb \japhug{βze}{grow} which originates from \japhug{βzu}{make} (§\ref{sec:stem3.backformation}), due to the fact that the stem \forme{βze} never occurs in tenses such as the Aorist and that the verb \japhug{βzu}{make} is also used with the meaning `grow' with a dummy subject (§\ref{sec:transitive.dummy}).

In the case of \forme{rnde} however, the fact that this rare verb is mainly attested in Aorist forms, where a Stem III is not possible makes the backformation hypothesis less likely. In addition, although the meaning of \japhug{rnde}{get into trouble} is relatable to that of \japhug{rndu}{obtain} (`obtain/get problems/trouble'), it is not a simple narrowing of the meaning of \forme{rndu}, a verb all of whose attestations refer to positive events (see \ref{ex:kro.Zo.YWrnde} above, as well as \ref{ex:CpWrndutCi} in §\ref{sec:AM.volitionality}).\footnote{For the backformation hypothesis to work, one would need to suppose that at an earlier stage, \japhug{rndu}{obtain} with a non-overt object had a negative overtone `get/obtain it' \fl{} `get into trouble'. }

Alternatively, one could consider that \japhug{rnde}{get into trouble} originates from \japhug{rndu}{obtain} by a valency-neutral derivation. The vowel alternation could be accounted for by hypothesizing the existence of a \forme{*-j} suffix (as in the case of the Stem III, §\ref{sec:stem3.form}), whose exact function is difficult to describe in the absence of other examples.

 
 
