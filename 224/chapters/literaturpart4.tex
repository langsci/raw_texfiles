\WeitereLiteratur{

\begin{sloppypar}

\paragraph*{Einführung und Gesamtdarstellungen}

Einführungen in die syntaktische Analyse sind \zB \citet{Duerscheid2012}, \citet{WoellsteinEa1997}, \citet{Eroms2000}, \citet{Musan2009}.
Zum Feldermodell kann ausführlicher \citet{Woellstein2010} herangezogen werden.
Wenn eine formale Grundlage gewünscht wird, ist \citet{Mueller2008} einschlägig.
Eine Einordnung verschiedener Syntaxtheorien in einen größeren theoretischen Kontext bietet \citet{Mueller2010}.
Mit \citet{Engel2009} gibt es eine ausführliche Einführung in eine Theorie, in der die Dependenz das zentrale Konzept ist.
Eine aktuelle Einführung in die Bindungstheorie, die viele semantische Aspekte berücksichtigt, ist \citet{Buering2005}.

\paragraph*{Weiterführende Lesevorschläge}
\citet{Gallmann1996} zur Morphosyntax der deutschen Nominalphrase;
\citet{Fabriciushansen1993} zur Morphosyntax von Nominalisierungen;
\citet{Loetscher1981} zur Abfolge von Konstituenten im Mittelfeld;
\citet{Hoehle1986} zum Feldermodell;
\citet{Askedal1986} zu Stellungsfeldern;
\citet{Pittner2003} zu freien Relativsätzen;
\citet{PantherKoepcke2008} zu prototypischen Eigenschaften von unabhängigen Sätzen (auf Englisch und zum Englischen, aber wichtig);
\citet{SchaeferSayatz2016} zu nicht standardkonformen Konstruktionen mit \textit{obwohl} und \textit{weil} und zum Problem der unabhängigen Sätze (auf Englisch, aber zum Deutschen);
\citet{DekuthyMeurers2001} über Voranstellungen jenseits des hier Besprochenen (anspruchsvoll, auf Englisch);
\citet{Richter2002} zu Resultativprädikaten;
\citet{Dowty1991} zu thematischen Rollen;
\citet{Reis1982} zum Subjekt im Deutschen;
\citet{Askedal1990} zum Pronomen \textit{es};
\citet{Wegener1986} und \citet{Wegener1991} zum Dativ und zum indirekten Objekt;
\citet{Musan1999} zur Perfektbildung;
\citet{HentschelWeydt1995} und \citet{Leirbukt2013} zum Dativpassiv;
\citet{Reis2001} zur Syntax der Modalverben;
\citet{Askedal1991} zum Ersatzinfinitiv;
\citet{Bech1983} als einflussreiches Buch zu Infinitiven und Kohärenz;
\citet{Reis2005} zu Halbmodalen;
\citet{Askedal1988} zu Subjektinfinitiven.

\end{sloppypar}

}
