\title{Einführung in die grammatische Beschreibung des Deutschen}
\subtitle{Dritte, überarbeitete und erweiterte Auflage}
\BackTitle{Einführung in die grammatische Beschreibung des Deutschen}
\BackBody{\begin{sloppypar}
\textit{Einführung in die grammatische Beschreibung des Deutschen} ist eine Einführung in die Grammatik des gegenwärtigen Deutschen in den Bereichen Phonetik, Phonologie, Morphologie, Syntax und Graphematik.
Das Buch ist für alle geeignet, die sich für die Grammatik des Deutschen interessieren, vor allem aber für Studierende der Germanistik bzw.\ Deutschen Philologie, insbesondere auch für Lehramtsstudierende.
Im Vordergrund steht die Vermittlung grammatischer Erkenntnisprozesse und Argumentationsweisen auf Basis konkreten sprachlichen Materials.
Es wird kein spezielles theoretisches Modell angenommen, aber alle, die das Buch gelesen haben, sollten in der Lage sein, sowohl deskriptiv ausgerichtete Forschungsartikel als auch theorienahe Einführungen lesen zu können.
Das Buch enthält zahlreiche Übungsaufgaben, die im Anhang gelöst werden.

Die dritte Auflage behebt Tipp- und Stilfehler und bietet einige neue Vertiefungsblöcke sowie eine komplette Überarbeitung der Grafiken und Diagramme.
Ein Kapitel über Grammatik in Schule und Lehramtsstudium ergänzt das Buch.

\vspace{1\baselineskip}

\noindent\textbf{Roland Schäfer} ist Germanist und Linguist.
Er hat an der Phi\-lipps-\-Universität Marburg studiert und war wissenschaftlicher Mitarbeiter an der Georg-August Universität Göttingen und der Freien Universität Berlin.
Er hat Professuren in Göttingen (2011\slash 2012) und an der Freien Universität Berlin (2016 und seit 2018) vertreten.
Nach seiner Promotion in Göttingen im Jahr 2008 hat er 2018 eine Habilitationsschrift zum Thema \textit{Probabilistic German Morphosyntax} (eine Analyse von sogenannten \textit{Zweifelsfällen} im Rahmen der probabilistischen Grammatik) vorgelegt, auf Basis derer die Humboldt-Universität zu Berlin sein Habilitationsverfahren zur Erlangung der Venia legendi für germanistische und allgemeine Sprachwissenschaft eröffnet hat.
Seine aktuellen Forschungsschwerpunkte sind die probabilistische Morphosyntax und Graphematik des Deutschen, empirische und statistische Verfahren, Fachdidaktik und Lehramtsausbildung sowie die Korpuserstellung.
Von 2015 bis 2018 leitete er erfolgreich das selber eingeworbene DFG-Projekt \textit{Linguistische Web-Charakterisierung und Webkorpuserstellung} an der Freien Universität Berlin.
\end{sloppypar}
}

\dedication{%
\large Für Adrianna, Alma, Ariel, Block, Frau Brüggenolte, Chloe, Chopin, Christina, Doro, Edgar, Elena, Elin, Emma, den ehemaligen FCR Duisburg, Frida, Gabriele, Hamlet, Helmut Schmidt, Henry, Ian Kilmister, Ingeborg, Ischariot, Jean-Pierre, Johan, Juliette, Kiki, Kristine, Kurt, Lemmy, Liv, Marina, Martin, Mats, Mausi, Michelle, Nadezhda, Herrn Oelschlägel, Oma, Opa, Pavel, Philly, Sarah, Scully, Stig, Tania, Tante Klärchen, Tarek, Tatjana, Herrn Uhl, Ullis schreckhaften Hund, Vanessa und so.\\[\baselineskip]Wenn das schonmal klar sein würde.\\}

\typesetter{Roland Schäfer}

\proofreader{Thea Dittrich, Luise Rißmann, Ulrike Sayatz (in alphabetical order)}

\author{Roland Schäfer}

\BookDOI{10.5281/zenodo.1421660}
\renewcommand{\lsISBNdigital}{978-3-96110-116-0}
\renewcommand{\lsISBNhardcover}{978-3-96110-117-7}
\renewcommand{\lsISBNsoftcover}{978-3-96110-118-4}
\renewcommand{\lsISBNsoftcoverus}{978-1727793741}
\renewcommand{\lsSeries}{tbls}
\renewcommand{\lsSeriesNumber}{2}
\renewcommand{\lsURL}{http://langsci-press.org/catalog/book/224}

 
 
 
 
  
