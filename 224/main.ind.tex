\begin{theindex}

  \item Ablaut, \hyperpage{204}, \hyperpage{301}
  \item Adjektiv, \hyperpage{175, 176}, \hyperpage{184}, 
		\hyperpage{240}
    \subitem adjektival, \hyperpage{280}
    \subitem adverbial, \hyperpage{277}
    \subitem attributiv, \hyperpage{276}
    \subitem Flexion, \hyperpage{279}, \hyperpage{281}
    \subitem Komparation
      \subsubitem Flexion, \hyperpage{283}
      \subsubitem Funktion, \hyperpage{281}
    \subitem Kurzform, \hyperpage{276}
    \subitem prädikativ, \hyperpage{276}
    \subitem schwach, \hyperpage{278, 279}
    \subitem skalar, \hyperpage{281}
    \subitem stark, \hyperpage{278, 279}
    \subitem Stärke, \hyperpage{184}, \hyperpage{278}
    \subitem Valenz, \hyperpage{277}
  \item Adjektivphrase, \hyperpage{351}, \hyperpage{360}
  \item Adkopula, \hyperpage{187}
  \item Adverb, \hyperpage{187}
  \item Adverbialsatz, \hyperpage{385}, \hyperpage{388}, 
		\hyperpage{416, 417}
  \item Adverbphrase, \hyperpage{366}
  \item Affigierung, \hyperpage{212}
  \item Affix, \hyperpage{206}
  \item Affrikate, \hyperpage{90}, \hyperpage{99}
  \item Agens, \hyperpage{423}, \hyperpage{437}, \hyperpage{439, 440}
  \item Akkusativ, \hyperpage{196}, \hyperpage{198}, \hyperpage{251}, 
		\hyperpage{355}, \hyperpage{432}, \hyperpage{441}
    \subitem Doppel--, \hyperpage{442}
  \item Akronym, \hyperpage{502}
  \item Akzent, \hyperpage{153}
    \subitem Haupt--, \hyperpage{155}
    \subitem in Komposita, \hyperpage{155}
    \subitem metrisch vs.\ lexikalisch, \hyperpage{154}
    \subitem Neben--, \hyperpage{155}
    \subitem Präfixe und Partikeln, \hyperpage{156}
\enlargethispage{\baselineskip}
    \subitem Schreibung, \hyperpage{488}
    \subitem Stamm--, \hyperpage{155}
  \item Akzeptabilität, \hyperpage{6, 7}, \hyperpage{12}
  \item Allomorph, \hyperpage{214}
  \item Allophon, \hyperpage{162}
  \item Alphabet
    \subitem deutsch, \hyperpage{475}
    \subitem phonetisch, \hyperpage{95}
  \item Alveolar, \hyperpage{97}
  \item Ambiguität, \hyperpage{337}
  \item Anapher, \hyperpage{254}
  \item Angabe, \hyperpage{45}, \hyperpage{425}
    \subitem Akkusativ--, \hyperpage{442}
    \subitem Dativ--, \hyperpage{444}
    \subitem präpositional, \hyperpage{424}
  \item Antezedens, \hyperpage{255}
  \item Apostroph, \hyperpage{503}
  \item Approximant, \hyperpage{91}
  \item Artikel
    \subitem definit, \hyperpage{271}, \hyperpage{273}
    \subitem Flexionsklassen, \hyperpage{271}
    \subitem indefinit, \hyperpage{275}, \hyperpage{503}
    \subitem NP ohne, \hyperpage{359}
    \subitem Position, \hyperpage{351}
    \subitem possessiv, \hyperpage{275}
    \subitem vs.\ Pronomen, \hyperpage{268}
  \item Artikelfunktion, \hyperpage{269}
  \item Artikelwort, \hyperpage{268}, \hyperpage{344}, \hyperpage{351}
  \item Artikulationsart, \hyperpage{88}
  \item Artikulator, \hyperpage{88}
  \item Assimilation, \hyperpage{105}, \hyperpage{122}
  \item Ast, \hyperpage{337}
  \item Attribut, \hyperpage{351}

  \indexspace

  \item Baumdiagramm, \hyperpage{35}, \hyperpage{206}, \hyperpage{337}, 
		\hyperpage{347}, \hyperpage{376}
  \item Befragung, \hyperpage{20}
  \item Bewegung, \hyperpage{392}, \hyperpage{402}
  \item Bildungssprache, \hyperpage{50}, \hyperpage{52}, \hyperpage{54}
  \item Bindestrich, \hyperpage{500}
  \item Bindung, \hyperpage{448}
  \item Bindungstheorie, \hyperpage{449}
  \item Buchstabe, \hyperpage{81}
    \subitem konsonantisch, \hyperpage{475}
    \subitem vokalisch, \hyperpage{478}

  \indexspace

  \item Dativ, \hyperpage{198}, \hyperpage{262}, \hyperpage{442}
    \subitem Bewertungs--, \hyperpage{440}, \hyperpage{443}, 
		\hyperpage{445}
    \subitem frei, \hyperpage{425}, \hyperpage{443}
    \subitem Funktion, \hyperpage{252}
    \subitem Nutznießer--, \hyperpage{443}
    \subitem Pertinenz--, \hyperpage{443}
  \item Defektivität, \hyperpage{312}
  \item Dehnungsschreibung, \hyperpage{478}, \hyperpage{481}, 
		\hyperpage{506}
  \item Deixis, \hyperpage{254}
  \item Dekontextualisierung, \hyperpage{51}
  \item Dependenz, \hyperpage{341}
  \item Derivation, \hyperpage{236, 237}
    \subitem mit Worklassenwechsel, \hyperpage{240}
    \subitem ohne Wortklassenwechsel, \hyperpage{237}
  \item Deutschunterricht, \hyperpage{52}
  \item Diakritikon, \hyperpage{95}
  \item Dialekt, \hyperpage{17, 18}
  \item Diminutiv, \hyperpage{241}
  \item Diphthong, \hyperpage{101}
    \subitem Schreibung, \hyperpage{479}
    \subitem sekundär, \hyperpage{107}, \hyperpage{152}
  \item dritte Konstruktion, \hyperpage{457}

  \indexspace

  \item Eigenname, \hyperpage{263}
    \subitem Schreibung, \hyperpage{499}
  \item Einheit, \hyperpage{25}
  \item Einsilbler, \hyperpage{128}, \hyperpage{145}
  \item Elativ, \hyperpage{282}
  \item Ellipse, \hyperpage{333}
  \item Empirie, \hyperpage{19}
  \item Endrand
    \subitem Desonorisierung, \hyperpage{105}, \hyperpage{113}, 
		\hyperpage{116}, \hyperpage{475}
  \item Erbwort, \hyperpage{9}
  \item Ereigniszeitpunkt, \hyperpage{289}
  \item Ergänzung, \hyperpage{45, 46}, \hyperpage{425}
    \subitem Akkusativ--, \hyperpage{442}
    \subitem Dativ--, \hyperpage{444}
    \subitem fakultativ und obligatorisch, \hyperpage{41}
    \subitem Nominativ--, \hyperpage{429}
    \subitem PP--, \hyperpage{445}
    \subitem prädikativ, \hyperpage{427}
  \item Experiencer, \hyperpage{423}
  \item Experiment, \hyperpage{20}
  \item Extrasilbizität, \hyperpage{137}
    \subitem und Flexionssuffixe, \hyperpage{144}

  \indexspace

  \item Feldermodell, \hyperpage{395}
  \item Filtermethode, \hyperpage{181}
  \item Flexion, \hyperpage{178}, \hyperpage{196}, \hyperpage{211}, 
		\hyperpage{247}
  \item Form und Funktion, \hyperpage{52}, \hyperpage{61}, 
		\hyperpage{385}
  \item Fremdwort, \hyperpage{9}
  \item Frikativ, \hyperpage{90}
  \item Funktion
    \subitem systemintern und systemintern, \hyperpage{62}
  \item Funktionswort, \hyperpage{344}
  \item Futur, \hyperpage{293}
    \subitem Bedeutung, \hyperpage{290}
  \item Futurperfekt, \hyperpage{450}
    \subitem Bedeutung, \hyperpage{291}
  \item Fuß, \hyperpage{10}, \hyperpage{157}

  \indexspace

  \item Gaumensegel, \hyperpage{86}
  \item Gebrauchsschreibung, \hyperpage{472}, \hyperpage{488}, 
		\hyperpage{503}
  \item Gedankenstrich, \hyperpage{508}
  \item Generalisierung, \hyperpage{16}
  \item Genitiv, \hyperpage{262}
    \subitem Attributs--, \hyperpage{252}
    \subitem Funktion, \hyperpage{252}
    \subitem Objekts--, \hyperpage{355}
    \subitem postnominal, \hyperpage{353}, \hyperpage{355}
    \subitem pränominal, \hyperpage{351}, \hyperpage{355}, 
		\hyperpage{411}
    \subitem Subjekts--, \hyperpage{355}
    \subitem sächsisch, \hyperpage{504}
  \item Genus, \hyperpage{28}, \hyperpage{183}, \hyperpage{256}, 
		\hyperpage{266}
  \item Glottalplosiv, \hyperpage{96}, \hyperpage{115}, \hyperpage{158}
  \item Gradierungselement, \hyperpage{360}
  \item Grammatik, \hyperpage{7}, \hyperpage{34}
    \subitem als Kombinationssystem, \hyperpage{4}
    \subitem deskriptiv, \hyperpage{14}
    \subitem Ebene, \hyperpage{8}
    \subitem formbasiert, \hyperpage{5}
    \subitem präskriptiv, \hyperpage{14}, \hyperpage{59}
    \subitem Sprachsystem, \hyperpage{5}, \hyperpage{56}
    \subitem Unterricht, \hyperpage{53--55}, \hyperpage{59}, 
		\hyperpage{66}
  \item Grammatik-Werkstatt, \hyperpage{53}
  \item Grammatikalisierung, \hyperpage{243}, \hyperpage{496}
  \item Grammatikalität, \hyperpage{7}, \hyperpage{12}, 
		\hyperpage{325}
  \item Grammatikerfrage, \hyperpage{250}, \hyperpage{442}
  \item Graphematik, \hyperpage{9}, \hyperpage{80}, \hyperpage{83}, 
		\hyperpage{470}

  \indexspace

  \item Hilfsverb, \hyperpage{300}
  \item homorgan, \hyperpage{90}
  \item Häufigkeit, \hyperpage{11}

  \indexspace

  \item Idiosynkrasie, \hyperpage{249}
  \item Imperativ, \hyperpage{309}, \hyperpage{430}
    \subitem Satz, \hyperpage{405}
  \item Index, \hyperpage{255}
  \item Infinitiv, \hyperpage{31}, \hyperpage{512}
  \item Introspektion, \hyperpage{20}
  \item IPA, \hyperpage{95}
  \item Iterierbarkeit, \hyperpage{43}

  \indexspace

  \item Kante, \hyperpage{337, 338}
  \item Kasus, \hyperpage{172}, \hyperpage{200}, \hyperpage{250}
    \subitem Bedeutung, \hyperpage{43}, \hyperpage{251}
    \subitem Funktion, \hyperpage{196}
    \subitem Hierarchie, \hyperpage{250}
    \subitem oblique, \hyperpage{252}
    \subitem strukturell, \hyperpage{252}
  \item Kategorie, \hyperpage{26, 27}, \hyperpage{29}
  \item Kehlkopf, \hyperpage{85}
  \item Kern, \hyperpage{9}
    \subitem Wortschatz, \hyperpage{9}, \hyperpage{146}, 
		\hyperpage{474}, \hyperpage{489}
  \item Klammer, \hyperpage{508}
  \item Klitikon, \hyperpage{503}
  \item Knoten, \hyperpage{337}
    \subitem Mutter--, \hyperpage{338}
    \subitem Tochter--, \hyperpage{338}
    \subitem Wurzel--, \hyperpage{338}
  \item Kohärenz, \hyperpage{454}, \hyperpage{457}
    \subitem Schreibung, \hyperpage{512}
  \item Komma, \hyperpage{507}
  \item Komparativ, \hyperpage{283}
  \item Kompetenz, \hyperpage{329}
  \item Komplementierer, \hyperpage{185}, \hyperpage{367}, 
		\hyperpage{395}, \hyperpage{416}
  \item Komplementiererphrase, \hyperpage{367}
  \item Komplementsatz, \hyperpage{355}, \hyperpage{387}, 
		\hyperpage{398}, \hyperpage{414}, \hyperpage{430}, 
		\hyperpage{512}
  \item Kompositionalität, \hyperpage{4}, \hyperpage{222}
  \item Kompositum, \hyperpage{221}
    \subitem Ambiguität, \hyperpage{226}
    \subitem Determinativ--, \hyperpage{224}
    \subitem Fugenelement, \hyperpage{228}
    \subitem Pluralfuge, \hyperpage{230}
    \subitem Rektions--, \hyperpage{224}
    \subitem Schreibung, \hyperpage{500}
  \item Konditionalsatz, \hyperpage{417}
  \item Konditionierung, \hyperpage{214}
  \item Kongruenz, \hyperpage{39}
    \subitem Genus--, \hyperpage{276}
    \subitem Numerus--, \hyperpage{249}, \hyperpage{276}
    \subitem Possessor--, \hyperpage{270}
    \subitem Subjekt--Verb--, \hyperpage{297}, \hyperpage{457}
  \item Konjunktion, \hyperpage{189}, \hyperpage{344}, \hyperpage{348}, 
		\hyperpage{507}
  \item Konnektor, \hyperpage{398}
  \item Konnektorfeld, \hyperpage{398}
  \item Konsonant, \hyperpage{94}
    \subitem Schreibung, \hyperpage{475}
  \item Konstituente, \hyperpage{36}, \hyperpage{391}
    \subitem atomar, \hyperpage{336}
    \subitem mittelbar, \hyperpage{36}
    \subitem unmittelbar, \hyperpage{36}
  \item Konstituententest, \hyperpage{329}
  \item Kontrast, \hyperpage{113}
  \item Kontrolle, \hyperpage{460}
  \item Konversion, \hyperpage{232}, \hyperpage{497}
    \subitem im Deutschen, \hyperpage{234}
  \item Koordination, \hyperpage{249}, \hyperpage{348}
    \subitem Schreibung, \hyperpage{507}
  \item Koordinationstest, \hyperpage{332}
  \item Kopf
    \subitem Komposition, \hyperpage{223}
    \subitem Kopf-Merkmal-Prinzip, \hyperpage{343}
    \subitem Phrase, \hyperpage{342}
  \item Kopula, \hyperpage{187}, \hyperpage{276}, \hyperpage{301}, 
		\hyperpage{406}, \hyperpage{427}
    \subitem Satz, \hyperpage{406}
  \item Korpus, \hyperpage{11}, \hyperpage{21}
  \item Korreferenz, \hyperpage{255}
  \item Korrelat, \hyperpage{415}, \hyperpage{433}, \hyperpage{436}, 
		\hyperpage{460}
  \item Kurzwort, \hyperpage{245}, \hyperpage{502}

  \indexspace

  \item Labial, \hyperpage{98}
  \item Laryngal, \hyperpage{96}
  \item Lehnwort, \hyperpage{9}, \hyperpage{209}
  \item Lehramt, \hyperpage{59}
  \item Lexem, \hyperpage{214}
  \item Lexikon, \hyperpage{28}, \hyperpage{115, 116}
    \subitem Regel, \hyperpage{212}, \hyperpage{438}
    \subitem Unbegrenztheit, \hyperpage{209}
  \item Ligatur, \hyperpage{99}
  \item Lippen, \hyperpage{86}
  \item Liquid, \hyperpage{130}, \hyperpage{152}
  \item Lizenzierung, \hyperpage{42}
  \item Luftröhre, \hyperpage{84}
  \item Lunge, \hyperpage{84}

  \indexspace

  \item Majuskel, \hyperpage{474}, \hyperpage{488}, \hyperpage{497}, 
		\hyperpage{501}
  \item Markierungsfunktion, \hyperpage{199}, \hyperpage{217}
    \subitem lexikalisch, \hyperpage{202}
  \item Matrix, \hyperpage{384}
  \item Medium
    \subitem akustisch, \hyperpage{79}
    \subitem gestisch, \hyperpage{79}
    \subitem schriftlich, \hyperpage{470}
  \item Merkmal, \hyperpage{25, 26}, \hyperpage{32}, \hyperpage{47}
    \subitem Motivation, \hyperpage{33}
    \subitem statisch, \hyperpage{208}
  \item Merkmale, \hyperpage{103}
  \item Minimalpaar, \hyperpage{112}
  \item Minuskel, \hyperpage{474}
  \item Mitspieler, \hyperpage{422}
  \item Mittelfeld, \hyperpage{395}, \hyperpage{415}, \hyperpage{417}
  \item Modifizierer, \hyperpage{361}, \hyperpage{364}
  \item Monoflexion, \hyperpage{280}
  \item More, \hyperpage{147}
  \item Morph, \hyperpage{199}
  \item Morphem, \hyperpage{214}
  \item Morphologie, \hyperpage{9}, \hyperpage{198}
  \item Mundraum, \hyperpage{85}

  \indexspace

  \item Nachfeld, \hyperpage{398}, \hyperpage{413}, \hyperpage{417}
  \item Nasal, \hyperpage{92}
  \item Nasenhöhle, \hyperpage{86}
  \item Nebensatz, \hyperpage{31}, \hyperpage{185}, \hyperpage{415}, 
		\hyperpage{430}
    \subitem Funktion, \hyperpage{387}
    \subitem Schreibung, \hyperpage{512}
  \item Neutralisierung, \hyperpage{114}
  \item Nomen, \hyperpage{182}, \hyperpage{237}
    \subitem vs.\ Substantiv, \hyperpage{352}
  \item Nominalisierung, \hyperpage{354}
  \item Nominalphrase, \hyperpage{247}, \hyperpage{351}
  \item Nominativ, \hyperpage{251}, \hyperpage{429}, \hyperpage{432}
  \item Numerus, \hyperpage{29}, \hyperpage{172}, \hyperpage{181}, 
		\hyperpage{200}, \hyperpage{266}
    \subitem Nomen, \hyperpage{248}
    \subitem Verb, \hyperpage{287}, \hyperpage{304}

  \indexspace

  \item Oberfeldumstellung, \hyperpage{453, 454}
  \item Objekt, \hyperpage{197}
    \subitem direkt, \hyperpage{442}
    \subitem indirekt, \hyperpage{443}
    \subitem Infinitiv, \hyperpage{460}
    \subitem präpositional, \hyperpage{445}
    \subitem Satz, \hyperpage{414}
  \item Obstruent, \hyperpage{89}, \hyperpage{94}
  \item Obstruktion, \hyperpage{87}
  \item Orthographie, \hyperpage{80}, \hyperpage{472}

  \indexspace

  \item Palatal, \hyperpage{97}
  \item Palatoalveolar, \hyperpage{97}
  \item Paradigma, \hyperpage{31}, \hyperpage{172}, 
		\hyperpage{176, 177}
    \subitem Genus--, \hyperpage{33}
    \subitem Numerus--, \hyperpage{33}
  \item Parataxe und Hypotaxe, \hyperpage{63}, \hyperpage{385}
  \item Parenthese, \hyperpage{508}
  \item Partikel, \hyperpage{186}, \hyperpage{344}
  \item Passiv, \hyperpage{298}, \hyperpage{430}
    \subitem als Valenzänderung, \hyperpage{438}, \hyperpage{440}
    \subitem bekommen--, \hyperpage{440}
    \subitem unpersönlich, \hyperpage{437}
    \subitem werden--, \hyperpage{436}, \hyperpage{438}
  \item Perfekt, \hyperpage{293}
    \subitem Doppel--, \hyperpage{451}
    \subitem Semantik, \hyperpage{451}
  \item Performanz, \hyperpage{329}
  \item Peripherie, \hyperpage{9}
  \item Person
    \subitem Nomen, \hyperpage{253}
    \subitem Verb, \hyperpage{287}, \hyperpage{304}
  \item Phon, \hyperpage{161}
  \item Phonem, \hyperpage{161, 162}
  \item Phonetik, \hyperpage{80}
  \item Phonologie, \hyperpage{9}, \hyperpage{116}
  \item phonologischer Prozess, \hyperpage{115}
  \item Phonotaktik, \hyperpage{124}
  \item Phrase, \hyperpage{340}
  \item Phrasenschema, \hyperpage{347}
  \item Plosiv, \hyperpage{89}
  \item Pluraletantum, \hyperpage{249}
  \item Positiv, \hyperpage{283}
  \item Postposition, \hyperpage{364}
  \item Produktivität, \hyperpage{222}
  \item Pronomen, \hyperpage{184}
    \subitem anaphorisch, \hyperpage{254}
    \subitem definit, \hyperpage{271}
    \subitem deiktisch, \hyperpage{254}
    \subitem expletiv, \hyperpage{156}, \hyperpage{435}
    \subitem flektierend, \hyperpage{271}
    \subitem Flexion, \hyperpage{272}
    \subitem Flexionsklassen, \hyperpage{271}
    \subitem nicht-flektierend, \hyperpage{271}
    \subitem Personal--, \hyperpage{253}, \hyperpage{271}
    \subitem positional, \hyperpage{435, 436}
    \subitem possessiv, \hyperpage{270}
    \subitem reflexiv, \hyperpage{448}
    \subitem vs.\ Artikel, \hyperpage{268}
  \item Pronominaladverb, \hyperpage{194}
  \item Pronominalfunktion, \hyperpage{269}
  \item Pronominalisierungstest, \hyperpage{330}
  \item Prosodie, \hyperpage{152}
  \item Prädikat, \hyperpage{425}
    \subitem resultativ, \hyperpage{427}
  \item Prädikativ, \hyperpage{428}
  \item Prädikatsnomen, \hyperpage{427}
  \item Präfix, \hyperpage{206}
  \item Präposition, \hyperpage{184}
    \subitem flektierbar, \hyperpage{365}
    \subitem Wechsel--, \hyperpage{198}
  \item Präpositionalobjekt, \hyperpage{445}
  \item Präpositionalphrase, \hyperpage{364}
  \item Präsens, \hyperpage{293}, \hyperpage{306}
    \subitem Bedeutung, \hyperpage{289}
  \item Präsensperfekt, \hyperpage{450}
  \item Präteritalpräsens, \hyperpage{311}
  \item Präteritum, \hyperpage{293}, \hyperpage{306}
    \subitem Bedeutung, \hyperpage{290}
  \item Präteritumsperfekt, \hyperpage{293}, \hyperpage{450}
    \subitem Bedeutung, \hyperpage{291}
  \item Punkt, \hyperpage{509}

  \indexspace

  \item r-Vokalisierung, \hyperpage{107}
    \subitem Schreibung, \hyperpage{475}
  \item Rachen, \hyperpage{85}
  \item Rectum, \hyperpage{38}
  \item Referenzzeitpunkt, \hyperpage{291}
  \item Regel, \hyperpage{15}
  \item Regens, \hyperpage{38}
  \item Regularität, \hyperpage{3}, \hyperpage{5}, \hyperpage{15}
  \item Rektion, \hyperpage{38}, \hyperpage{46}
  \item Rekursion, \hyperpage{226, 227}, \hyperpage{373}
    \subitem in der Morphologie, \hyperpage{228}
    \subitem in der Syntax, \hyperpage{328}
  \item Relation, \hyperpage{37}
    \subitem syntaktisch, \hyperpage{37}
  \item Relativadverb, \hyperpage{411}
  \item Relativphrase, \hyperpage{409}
  \item Relativsatz, \hyperpage{351}, \hyperpage{388}, 
		\hyperpage{397, 398}, \hyperpage{409}
    \subitem Form und Funktion, \hyperpage{63}
    \subitem frei, \hyperpage{412}
  \item Rolle, \hyperpage{44}, \hyperpage{422}, \hyperpage{424}, 
		\hyperpage{458}
  \item Rückbildung, \hyperpage{242}

  \indexspace

  \item Satz, \hyperpage{383}
    \subitem Echofrage, \hyperpage{396}
    \subitem Entscheidungsfrage--, \hyperpage{404}
    \subitem Frage--, \hyperpage{396}
      \subsubitem eingebettet, \hyperpage{397}
    \subitem graphematisch, \hyperpage{510}
    \subitem Koordination, \hyperpage{509}
    \subitem Schreibung, \hyperpage{508}
    \subitem unabhängig, \hyperpage{386}
    \subitem Verb-Erst--, \hyperpage{368}, \hyperpage{397}, 
		\hyperpage{404}, \hyperpage{417}
    \subitem Verb-Letzt--, \hyperpage{368}, \hyperpage{397}
    \subitem Verb-Zweit--, \hyperpage{368}, \hyperpage{397}, 
		\hyperpage{402}
    \subitem w-Frage--, \hyperpage{17}, \hyperpage{396}, 
		\hyperpage{399}
  \item Satzglied, \hyperpage{250}, \hyperpage{335}, \hyperpage{426}
  \item Satzklammer, \hyperpage{395}
  \item Satzäquivalent, \hyperpage{188}
  \item Schreibprinzip
    \subitem Konstanz, \hyperpage{505}
    \subitem phonologisch, \hyperpage{478}
    \subitem Spatienschreibung, \hyperpage{495}
  \item Schriftsprache, \hyperpage{50, 51}
  \item Schwa, \hyperpage{101}
    \subitem Tilgung, \hyperpage{260}, \hyperpage{262}, \hyperpage{307}
  \item Schärfungsschreibung, \hyperpage{478}, \hyperpage{480}, 
		\hyperpage{483}
  \item Scrambling, \hyperpage{372}
  \item Segment, \hyperpage{83}, \hyperpage{111}
  \item Silbe, \hyperpage{124}, \hyperpage{126}
    \subitem Ambisyllabizität, \hyperpage{148}
    \subitem Anfangsrand, \hyperpage{127}, \hyperpage{148}
      \subsubitem komplex, \hyperpage{138}, \hyperpage{140}
    \subitem Endrand, \hyperpage{127}, \hyperpage{148}
      \subsubitem komplex, \hyperpage{140}, \hyperpage{145}
    \subitem extrametrisch, \hyperpage{157}
    \subitem Gelenk, \hyperpage{148}
      \subsubitem Endrand-Desonorisierung, \hyperpage{150}
    \subitem geschlossen, \hyperpage{146}
    \subitem Gewicht, \hyperpage{147}, \hyperpage{482}
    \subitem Grenze, \hyperpage{126}, \hyperpage{148}, \hyperpage{151}, 
		\hyperpage{482, 483}, \hyperpage{487}
    \subitem Kern, \hyperpage{127}
    \subitem Klatschmethode, \hyperpage{125}
    \subitem offen, \hyperpage{146}
    \subitem Rand, \hyperpage{138}
    \subitem Reim, \hyperpage{127}
    \subitem Schreibung, \hyperpage{481}
    \subitem Silbifizierung, \hyperpage{145}
  \item Simplex, \hyperpage{146}, \hyperpage{481}
  \item Singularetantum, \hyperpage{249}
  \item Sonorant, \hyperpage{94}
  \item Sonorität, \hyperpage{133, 134}
    \subitem Hierarchie, \hyperpage{133}
  \item Spatium, \hyperpage{495}, \hyperpage{501}
  \item Sprachbetrachtung, \hyperpage{51}, \hyperpage{54}, 
		\hyperpage{57}
  \item Sprache, \hyperpage{3}
  \item Spracherwerb, \hyperpage{50}, \hyperpage{57}
  \item Sprechzeitpunkt, \hyperpage{289}
  \item Spur, \hyperpage{393}, \hyperpage{402}, \hyperpage{415}
  \item Stamm, \hyperpage{202}
  \item Stammkonversion, \hyperpage{232}
  \item Standarddeutsch, \hyperpage{15}, \hyperpage{18}, \hyperpage{20}, 
		\hyperpage{22}
  \item Status, \hyperpage{297}, \hyperpage{308}, \hyperpage{373}, 
		\hyperpage{453}
  \item Stimmbänder, \hyperpage{85}
  \item Stimmhaftigkeit, \hyperpage{81}, \hyperpage{88}
  \item Stimmlippen, \hyperpage{85}
  \item Stimmton, \hyperpage{85}
  \item Stoffsubstantiv, \hyperpage{359}
  \item Struktur, \hyperpage{35}
  \item Strukturbedingung, \hyperpage{115}
  \item Subjekt, \hyperpage{197}, \hyperpage{426}, \hyperpage{429, 430}, 
		\hyperpage{458}
    \subitem Infinitiv, \hyperpage{460}
    \subitem Satz, \hyperpage{414}
  \item Substantiv, \hyperpage{33}, \hyperpage{176}, \hyperpage{183}, 
		\hyperpage{240}
    \subitem Großschreibung, \hyperpage{498}
    \subitem Kasusflexion, \hyperpage{262}
    \subitem Numerusflexion, \hyperpage{259}
    \subitem Plural, \hyperpage{259}
    \subitem s-Flexion, \hyperpage{502}
    \subitem schwach, \hyperpage{10}, \hyperpage{264}
    \subitem Stärke, \hyperpage{258}, \hyperpage{264}
    \subitem Subklassen, \hyperpage{258}, \hyperpage{266}
  \item Substantivierung, \hyperpage{497}
  \item Suffix, \hyperpage{206}
  \item Superlativ, \hyperpage{283}
  \item Suppletivität, \hyperpage{314}
  \item Symbolsystem, \hyperpage{3}
  \item Synkretismus, \hyperpage{34}
  \item Syntagma, \hyperpage{32}, \hyperpage{172}
  \item Syntax, \hyperpage{9}, \hyperpage{325}

  \indexspace

  \item Tempus, \hyperpage{182}, \hyperpage{289}
    \subitem analytisch, \hyperpage{372}, \hyperpage{449}
    \subitem einfach, \hyperpage{288, 289}
    \subitem Folge, \hyperpage{292}
    \subitem komplex, \hyperpage{292}
    \subitem synthetisch vs.\ analytisch, \hyperpage{293}
  \item Token, \hyperpage{11}
  \item Transkription, \hyperpage{95}, \hyperpage{104}
  \item Transparenz, \hyperpage{223}
  \item Typ, \hyperpage{11}

%  \indexspace
\newpage
  \item Umlaut, \hyperpage{203}
    \subitem Schreibung, \hyperpage{506}
  \item Univerbierung, \hyperpage{242}, \hyperpage{496}, 
		\hyperpage{498}
  \item Uvular, \hyperpage{96}

  \indexspace

  \item Valenz, \hyperpage{40}, \hyperpage{47}, \hyperpage{184}, 
		\hyperpage{341}, \hyperpage{424}, \hyperpage{437}, 
		\hyperpage{440}, \hyperpage{444}
    \subitem Adjektiv, \hyperpage{277}
    \subitem Substantiv, \hyperpage{354}
    \subitem Verb, \hyperpage{369}
  \item Variation, \hyperpage{18}, \hyperpage{20}
  \item Velar, \hyperpage{97}
  \item Verb, \hyperpage{176}, \hyperpage{182}, \hyperpage{238}, 
		\hyperpage{240}
    \subitem ditransitiv, \hyperpage{47}
    \subitem Experiencer--, \hyperpage{434}, \hyperpage{436}
    \subitem Finitheit, \hyperpage{182}, \hyperpage{296}
    \subitem Flexion
      \subsubitem finit, \hyperpage{307}
      \subsubitem Imperativ, \hyperpage{310}
      \subsubitem infinit, \hyperpage{308}
      \subsubitem unregelmäßig, \hyperpage{313}
    \subitem Flexionsklassen, \hyperpage{10}, \hyperpage{300}
    \subitem Futur, \hyperpage{449}
    \subitem gemischt, \hyperpage{313}
    \subitem Halbmodal--, \hyperpage{459}
    \subitem Hilfs--, \hyperpage{449}
    \subitem Indikativ, \hyperpage{303, 304}
    \subitem Infinitheit, \hyperpage{296}
    \subitem Infinitiv, \hyperpage{308}, \hyperpage{454}
      \subsubitem Ersatz--, \hyperpage{453, 454}
      \subsubitem zu--, \hyperpage{459}
    \subitem intransitiv, \hyperpage{47}, \hyperpage{438}
    \subitem Konjunktiv, \hyperpage{306}
      \subsubitem Flexion, \hyperpage{305}
      \subsubitem Form vs.\ Funktion, \hyperpage{305}
    \subitem Kontroll--, \hyperpage{459}
    \subitem Modal--, \hyperpage{301}, \hyperpage{457, 458}
      \subsubitem Flexion, \hyperpage{10}, \hyperpage{311}
    \subitem Partikel--, \hyperpage{406}
    \subitem Partizip, \hyperpage{308}, \hyperpage{454}
    \subitem Perfekt, \hyperpage{449}
    \subitem Person-Numerus-Suffixe, \hyperpage{304}
    \subitem Präfix-- vs.\ Partikel--, \hyperpage{309}
    \subitem Präsens, \hyperpage{303, 304}
    \subitem Präteritum, \hyperpage{303, 304}
    \subitem schwach, \hyperpage{303}, \hyperpage{306}
    \subitem stark, \hyperpage{304}, \hyperpage{306}
    \subitem Status, \hyperpage{449}, \hyperpage{455}, \hyperpage{457}
    \subitem Stärke, \hyperpage{302}, \hyperpage{313}
    \subitem transitiv, \hyperpage{47}, \hyperpage{437}
    \subitem unakkusativ, \hyperpage{438}
    \subitem unergativ, \hyperpage{438}, \hyperpage{440}
    \subitem Vokalstufe, \hyperpage{302}
    \subitem Voll--, \hyperpage{300}
    \subitem Wetter--, \hyperpage{434}, \hyperpage{436}
  \item Verbkomplex, \hyperpage{373}, \hyperpage{391}, \hyperpage{406}, 
		\hyperpage{454}
  \item Verbphrase, \hyperpage{370}, \hyperpage{391}, \hyperpage{455}
  \item Vergleichselement, \hyperpage{284}
  \item Verteilung, \hyperpage{111, 112}, \hyperpage{179}
    \subitem komplementär, \hyperpage{113}
  \item Vokal, \hyperpage{92}, \hyperpage{99}
    \subitem Gespanntheit, \hyperpage{117}, \hyperpage{147}, 
		\hyperpage{478}, \hyperpage{481}
    \subitem Höhe, \hyperpage{99}
    \subitem Lage, \hyperpage{99}
    \subitem Länge, \hyperpage{81}, \hyperpage{117}, \hyperpage{481}
    \subitem Rundung, \hyperpage{99}
    \subitem Schreibung, \hyperpage{478}
  \item Vokaltrapez, \hyperpage{100}, \hyperpage{108}, \hyperpage{119}, 
		\hyperpage{203}
  \item Vokativ, \hyperpage{310}
  \item Vorfeld, \hyperpage{17}, \hyperpage{186, 187}, \hyperpage{395}
  \item Vorfeldtest, \hyperpage{331}

  \indexspace

  \item Wackernagel-Position, \hyperpage{445}
  \item Wert, \hyperpage{25}
  \item Wort, \hyperpage{28}, \hyperpage{169}, \hyperpage{200}
    \subitem Bedeutung, \hyperpage{200}
    \subitem flektierbar, \hyperpage{29}, \hyperpage{181}
    \subitem graphematisch, \hyperpage{495}
    \subitem lexikalisch, \hyperpage{173}
    \subitem phonologisch, \hyperpage{145}, \hyperpage{160}
    \subitem prosodisch, \hyperpage{160}
    \subitem Stamm, \hyperpage{233}
    \subitem syntaktisch, \hyperpage{173}
  \item Wortbildung, \hyperpage{178}, \hyperpage{210}
    \subitem Komparation als --, \hyperpage{283}
  \item Wortformenkonversion, \hyperpage{232}
\newpage
  \item Wortklasse, \hyperpage{29}, \hyperpage{208}, \hyperpage{232}, 
		\hyperpage{237}
    \subitem morphologisch, \hyperpage{177}
    \subitem Schreibung, \hyperpage{496}
    \subitem semantisch, \hyperpage{174}
  \item Wortzeichen, \hyperpage{501}

  \indexspace

  \item Zahndamm, \hyperpage{86}
  \item Zeichen
    \subitem syntaktisch, \hyperpage{508}
    \subitem Wort--, \hyperpage{501}
  \item Zirkumfix, \hyperpage{206}
  \item zugrundeliegende Form, \hyperpage{115}, \hyperpage{152}
  \item Zunge, \hyperpage{86}
  \item Zweisilbler, \hyperpage{145}
  \item Zwerchfell, \hyperpage{84}
  \item Zähne, \hyperpage{86}
  \item Zäpfchen, \hyperpage{86}

\end{theindex}
