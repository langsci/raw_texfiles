%%%%%%%%%%%%%%%%%%%%%%%%%%%%%%%%%%%%%%%%%%%%%%%%%%%%
%%%                                              %%%
%%%     Language Science Press Master File       %%%
%%%         follow the instructions below        %%%
%%%                                              %%%
%%%%%%%%%%%%%%%%%%%%%%%%%%%%%%%%%%%%%%%%%%%%%%%%%%%%
 
% Everything following a % is ignored
% Some lines start with %. Remove the % to include them

\documentclass[output=covercreatespace% long|short|inprep              
%  	        ,draftmode  
		  ]{LSP/langsci}    
  
%%%%%%%%%%%%%%%%%%%%%%%%%%%%%%%%%%%%%%%%%%%%%%%%%%%%
%%%                                              %%%
%%%          additional packages                 %%%
%%%                                              %%%
%%%%%%%%%%%%%%%%%%%%%%%%%%%%%%%%%%%%%%%%%%%%%%%%%%%%

% put all additional commands you need in the 
% following files. If you do not know what this might 
% mean, you can safely ignore this section

\usepackage{url}
\usepackage{graphicx}
\usepackage{xspace}
\usepackage{multirow}
\usepackage{multicol}
\usepackage{lingmacros}
\usepackage{umoline}
\usepackage{setspace}
%\usepackage{tipa}
\usepackage{stmaryrd}
\usepackage{fancybox}
\usepackage{url}

\newcommand{\tdl}[1]{\textit{#1}}
%\newcommand{\myexe}[1]{{\small #1}}
\newcommand{\myexe}[1]{{\normalsize #1}}
\newcommand{\mysub}[1]{$_{\textnormal{\tiny{#1}}}$}
\newcommand{\mysout}[1]{\Midline{#1}\xspace}
\newcommand{\myemp}[1]{\textbf{\underline{#1}}}
\newcommand{\myref}[1]{(\ref{#1})\xspace}
\newcommand{\myS}[1]{\S\ref{#1}\xspace}
\newcommand{\mysec}[1]{\S\ref{#1}\xspace}
%\newcommand{\myurl}[1]{{\small\url{#1}}}
\newcommand{\myurl}[1]{\url{#1}}
\newcommand{\mypage}[1]{(p.\ \pageref{#1})}
\newcommand{\mypp}[1]{(\ref{#1},~p.\ \pageref{#1})~}
\newcommand{\xtab}{\xspace\xspace\xspace}
\newcommand{\lingo}{LinGO\xspace}
\newcommand{\lkb}{\textsc{\small LKB}\xspace}
\newcommand{\pet}{\textsc{\small PET}\xspace}
\newcommand{\ace}{\textsc{\small ACE}\xspace}
\newcommand{\agree}{\textit{agree}\xspace}
\newcommand{\itsdb}{\mbox{\textsf{\lbrack incr tsdb()\rbrack}\xspace}}
\newcommand{\logon}{\textsc{\small LOGON}\xspace}
\newcommand{\vs}{vs.\ }
\newcommand{\nun}{-(\textit{n})\textit{un}\xspace}
\newcommand{\onun}{(\textit{n})\textit{un}\xspace}
\newcommand{\ika}{\textit{i} / \textit{ka}\xspace}
\newcommand{\lul}{(\textit{l})\textit{ul}\xspace}
\newcommand{\wa}{\textit{wa}\xspace}
\newcommand{\ga}{\textit{ga}\xspace}



\title{Methods in prosody:\, \newlineCover A Romance language perspective}  
\author{Ingo Feldhausen\and Jan Fliessbach\lastand Maria del Mar Vanrell} 
\renewcommand{\lsSeriesNumber}{6}  
% \renewcommand{\lsCoverTitleFont}[1]{\sffamily\addfontfeatures{Scale=MatchUppercase}\fontsize{38pt}{12.75mm}\selectfont #1}

\renewcommand{\lsISBNdigital}{978-3-96110-104-7}
\renewcommand{\lsISBNhardcover}{978-3-96110-105-4}

\renewcommand{\lsSeries}{silp}          
\renewcommand{\lsSeriesNumber}{6}
\renewcommand{\lsURL}{http://langsci-press.org/catalog/book/183}
\renewcommand{\lsID}{183}
\renewcommand{\lsBookDOI}{10.5281/zenodo.1471564}

\typesetter{Jan Fliessbach\lastand Felix Kopecky}
\proofreader{Adrien Barbaresi, Amir Ghorbanpour, Aysel Saricaoglu, Brett Reynolds, Conor Pyle, Daniela Kolbe-Hanna, Jeroen van de Weijer, Sebastian Nordhoff\lastand Varun deCastro-Arrazola}

\BackBody{This book presents a collection of pioneering papers reflecting current methods in prosody research with a focus on Romance languages. The rapid expansion of the field of prosody research in the last decades has given rise to a proliferation of methods that has left little room for the critical assessment of these methods. The aim of this volume is to bridge this gap by embracing original contributions, in which experts in the field assess, reflect, and discuss different methods of data gathering and analysis. The book might thus be of interest to scholars and established researchers as well as to students and young academics who wish to explore the topic of prosody, an expanding and promising area of study.}

\usepackage{langsci-optional}
\usepackage{langsci-gb4e}
\usepackage{langsci-lgr}

\usepackage{listings}
\lstset{basicstyle=\ttfamily,tabsize=2,breaklines=true}

%added by author
% \usepackage{tipa}
\usepackage{multirow}
\graphicspath{{figures/}}
\usepackage{langsci-branding}

%% hyphenation points for line breaks
%% Normally, automatic hyphenation in LaTeX is very good
%% If a word is mis-hyphenated, add it to this file
%%
%% add information to TeX file before \begin{document} with:
%% %% hyphenation points for line breaks
%% Normally, automatic hyphenation in LaTeX is very good
%% If a word is mis-hyphenated, add it to this file
%%
%% add information to TeX file before \begin{document} with:
%% %% hyphenation points for line breaks
%% Normally, automatic hyphenation in LaTeX is very good
%% If a word is mis-hyphenated, add it to this file
%%
%% add information to TeX file before \begin{document} with:
%% \include{localhyphenation}
\hyphenation{
affri-ca-te
affri-ca-tes
an-no-tated
com-ple-ments
com-po-si-tio-na-li-ty
non-com-po-si-tio-na-li-ty
Gon-zá-lez
out-side
Ri-chárd
se-man-tics
STREU-SLE
Tie-de-mann
}
\hyphenation{
affri-ca-te
affri-ca-tes
an-no-tated
com-ple-ments
com-po-si-tio-na-li-ty
non-com-po-si-tio-na-li-ty
Gon-zá-lez
out-side
Ri-chárd
se-man-tics
STREU-SLE
Tie-de-mann
}
\hyphenation{
affri-ca-te
affri-ca-tes
an-no-tated
com-ple-ments
com-po-si-tio-na-li-ty
non-com-po-si-tio-na-li-ty
Gon-zá-lez
out-side
Ri-chárd
se-man-tics
STREU-SLE
Tie-de-mann
}

\newcommand{\sent}{\enumsentence}
\newcommand{\sents}{\eenumsentence}
\let\citeasnoun\citet

\renewcommand{\lsCoverTitleFont}[1]{\sffamily\addfontfeatures{Scale=MatchUppercase}\fontsize{44pt}{16mm}\selectfont #1}
   
\bibliography{localbibliography,sanghounsong} 

%%%%%%%%%%%%%%%%%%%%%%%%%%%%%%%%%%%%%%%%%%%%%%%%%%%%
%%%                                              %%%
%%%             Frontmatter                      %%%
%%%                                              %%%
%%%%%%%%%%%%%%%%%%%%%%%%%%%%%%%%%%%%%%%%%%%%%%%%%%%%
\begin{document}         
\maketitle                
\frontmatter
% %% uncomment if you have preface and/or acknowledgements

\currentpdfbookmark{Contents}{name} % adds a PDF bookmark
\tableofcontents
%\addchap{Preface}
\begin{refsection}

%content goes here
 
% \printbibliography[heading=subbibliography]
\end{refsection}


\addchap{Acknowledgments} 
%content goes here
The help and support of Martin Haspelmath and Sebastian Nordhoff in the preparation of this volume is gratefully acknowledged. 

We would also like to thank the authors of the chapters in this volume for their cooperation during the editing process and especially for their input to the reviewing of chapters by their peers. 

We especially thank the following additional external reviewers, %individuals, 
who contributed their time and expertise to provide independent peer review for the papers in this collection: Lisa Bonnici, Jason Brown, Elisabet Engdahl, Marieke Hoetjes, Beth Hume, Anne O'Keefe, Adam Schembri, Thomas Stolz, Andy Wedel and Shuly Wintner.
 

\addchap{Abbreviations}
\begin{tabular}{ll}
CR & Common Room     \\
NCR & non-Common Room        \\
SGH & Selwyn Girls' High     \\
The & BBs: The Blazer Brigade \\
\isi{The PCs} & The Palms Crew  \\
\end{tabular} 
\mainmatter         
 

%%%%%%%%%%%%%%%%%%%%%%%%%%%%%%%%%%%%%%%%%%%%%%%%%%%%
%%%                                              %%%
%%%             Chapters                         %%%
%%%                                              %%%
%%%%%%%%%%%%%%%%%%%%%%%%%%%%%%%%%%%%%%%%%%%%%%%%%%%%

\chapter{Introduction} \label{ch:1}
\section{Aims and scope} \label{sec:1aims}
The core problem to be dealt with in this book is the syntax of functional left peripheries in West Germanic. In particular, I will concentrate on how sentence types are marked at the leftmost edge of the clause and how the presence of multiple visible markers can be accounted for. Regarding syntactic structure, I adopt a minimalist framework (as proposed by \citealt{chomsky2001, chomsky2004, chomsky2008}, among others), according to which syntactic structures are derived by merge (external or internal). Further, in line with the principles of mainstream generative grammar, I assume that the derivation of structures is constrained by economy, and hence the number of projections, as well as of syntactic processes, is as minimal as possible.

The study of various issues associated with the left periphery of the clause has always been central in generative grammar and it continues to be one of the most well-researched areas of syntax. Among other functions, left peripheries are associated with defining the type of the clause, and they are also responsible for establishing connections between clauses that make them into complex sentences. Apart from purely syntactic concerns, left peripheries raise a number of questions that make this domain extremely relevant for the interfaces of syntax, referred to as PF (Perceptible Form or, more traditionally, Phonological Form\footnote{Since generative theory was initially limited to the study of oral languages, the term ``Phonological Form'' was established, and many properties of this interface reflect the properties of oral languages, even though sign languages also evidently have an interface connected to their perceptible form. In this sense, as proposed by \citet{sigurdsson2004}, the term ``Perceptible Form'' is more appropriate as it does not treat sign languages as secondary. See also \citet{vanderhulst2015} for the distinction of the two. In this book, I will restrict myself to examining selected oral languages, mostly from Germanic.}) and LF (Logical Form, indicating the semantic component) in standard generative grammar. The interaction with the interfaces becomes evident when considering issues related to the left periphery beyond clause typing proper: certain phrases appear to be located in the left periphery due to their specific information structural status. Apart from that, clausal ellipsis is also related to various functional heads (see \citealt{merchant2001}).

It is most probably this diversity of problems that led to a significant interest in the left periphery of the clause in generative grammar already in the 1970s, most notably in \citet{chomskylasnik1977}, followed by the well-known cartographic enterprise from the 1990s onwards, especially by \citet{rizzi1997, rizzi2004} and various analyses with more or less shared concerns: for example, \citet{sobin2002}, \citet{poletto2006}, \citet{bayerbrandner2008}, \citet{brandnerbraeuning2013}. I will both rely on these previous findings and critically evaluate them. In addition, while many questions have indeed been answered by previous accounts, there are several others that have remained unresolved and have not received an adequate explanation which would hold both cross-linguistically and specifically for West Germanic as well. In addition, I assume that any proposal should follow from general principles of the grammar rather than by applying construction-specific mechanisms. In other words, the specific configuration of the left periphery of one construction should be comparable to the left periphery of other clause types within a single model by applying predictable properties of the grammar. The aim of this work is to provide such an analysis and to enable a better understanding of functional left peripheries.

In the following, I will briefly provide an overview of the most important issues concerning functional left peripheries and clause typing in West Germanic, and then I will provide a concise outline of the problems to be dealt with in this book.

\section{Functional left peripheries} \label{sec:1functional}
Clauses can fulfil various functions in discourse; in canonical cases, the form of the clause is indicative of its discourse function. Consider the following examples:

\ea \label{clauses}
\ea Ralph is interested in poetry. \label{declarative}
\ex Is Ralph interested in poetry? \label{interrogative}
\z
\z

In (\ref{declarative}), we have a statement and the type of the clause is declarative. By contrast, (\ref{interrogative}) is a question and the type of the clause is interrogative. In the first case, a proposition (\textit{p}) is true; in the second case, the truth of the proposition is asked (\textit{p} or $\neg$\textit{p}). The two utterances differ in their form. The declarative sentence represents the neutral, unmarked word order in English, which is SVO: crucially, the subject (\textit{Ralph}) precedes the aspectual auxiliary (\textit{is}). In the interrogative clause, these two elements have exactly the opposite order: the aspectual auxiliary has been moved to the front of the clause.

In many cases, the form of an utterance is not indicative of its discourse function in a straightforward way. Consider the example in (\ref{window}):

\ea Could you open the window? \label{window}
\z

In this case, the speaker does not ask the addressee about the truth of the proposition but expresses a request: a simple \textit{yes} answer, which is satisfactory in (\ref{interrogative}), would not be pragmatically appropriate in (\ref{window}) if it is not accompanied by the speaker also opening the window. The pragmatic function of sentences is thus not in a one-to-one correspondence with the observed grammatical form; these issues are examined extensively in speech act theory, going back to the work of \citet{austin1962}. As the present book is concerned with the formal properties, especially the syntax of functional left peripheries and clause typing, these issues will not be addressed here.  

The two clauses in (\ref{clauses}) differ not only in their word order but also regarding their intonation: declarative clauses have falling intonation, while interrogative clauses have rising intonation. However, there are discrepancies in this respect as well; consider:

\ea Ralph is interested in poetry? \label{declquest}
\z

The example in (\ref{declquest}) is a declarative question: formally the clause is declarative but it has a rising (interrogative) intonation; regarding its function, it constitutes a special type of question which does not ask about the truth of a proposition but rather asks for confirmation or expresses surprise. Again, these cases will not be discussed in the present thesis as they are not immediately relevant to the specific syntactic issues to be examined.

The clauses in (\ref{clauses}) are main clauses. Clause types are identified in slightly different ways in embedded clauses such as (\ref{embedded}):

\ea \label{embedded}
\ea I think [\textbf{that} Ralph is interested in poetry]. \label{that}
\ex I wonder [\textbf{if} Ralph is interested in poetry]. \label{if}
\ex It is important [\textbf{for} Ralph to study Byron]. \label{for}
\z
\z

The highlighted complementisers determine the type of the embedded clause: (\ref{that}) and (\ref{for}) are declarative, while (\ref{if}) is interrogative. Apart from clause type, complementisers can also determine whether the clause is finite, as in (\ref{that}) and (\ref{if}), or non-finite, as in (\ref{for}). Finiteness, as determined by the C head, has an effect on whether the clause contains a tensed element (e.g. \textit{is} in (\ref{that}) and (\ref{if}) above) above or not (in which case, as in (\ref{for}), English uses the element \textit{to} and the infinitival form of the verb). The incompatibility of finite complementisers with a non-finite clause, and vice versa, is illustrated in (\ref{finnonfin}) below:

\ea \label{finnonfin}
\ea[*]{I think [for Ralph is interested in poetry].}
\ex[*]{It is important [that Ralph to study Byron].}
\z
\z

Likewise, the type of a complement clause must also be compatible with the lexical properties of the matrix verb: verbs like \textit{think} select for declarative complements, while verbs like \textit{wonder} select for interrogative complements. If these sectional restrictions are violated, the result is ungrammatical:

\ea
\ea[*]{I think [if Ralph is interested in poetry].}
\ex[*]{I wonder [that Ralph is interested in poetry].} 
\z
\z

In other words, it is evident that the left periphery of the clause has a dual function. On the one hand, it connects the clause to the matrix clause (in the case of embedded clauses) or to the discourse (in the case of root clauses). On the other hand, it has an impact on the internal properties of the clause itself.

Besides complementisers, the CP is known to host other elements as well, such as \textit{wh}-phrases in interrogative clauses:

\ea \label{wh}
\ea I wonder [\textbf{who} Mary will invite]. \label{who}
\ex I asked Louisa [\textbf{which city} she was travelling to]. \label{whichcity}
\z
\z

In (\ref{who}), the \textit{wh}-element consists of a single operator (\textit{who}), while in (\ref{whichcity}) the \textit{wh}-phrase is visibly phrase-sized, containing not only the operator \textit{which} but also a lexical element, the NP \textit{city}. This indicates that \textit{wh}-phrases can occupy only a phrase position, namely [Spec,CP], and not C. Further, since they also fulfil a role in the TP, that is, they are arguments, it is assumed in generative grammar that they undergo movement from a clause-internal position to the CP-domain. This is illustrated in (\ref{wonderasked}) below:

\ea \label{wonderasked}
\ea I wonder [\textbf{who} Mary will invite \sout{\textbf{who}}].
\ex I asked Louisa [\textbf{which city} she was travelling to \sout{\textbf{which city}}].
\z
\z

In line with current minimalist theory, I assume that movement involves the copying of the moved constituent: by default, the higher copy is realised phonologically at the PF interface, while PF eliminates lower copies of a movement chain. In English, \textit{wh}-elements move to the left periphery in interrogatives, leaving the higher copy in the CP overt. Operators moving to the left periphery thus differ from complementisers not only with respect to their relative position in the CP but also in that they land there via movement, while complementisers are base-generated in the left periphery.

Relative clauses also contain operator movement:

\ea
\ea This is the linguist [\textbf{who} Mary will invite].
\ex The candidate [\textbf{who} we voted for] has already left the city.
\z
\z

Relative clauses differ from interrogative clauses in that they modify a nominal head, referred to as the head noun, while embedded interrogatives are complements of a matrix predicate (and interrogative clauses can also be root clauses). Again, relative operators undergo leftward movement:

\ea
\ea This is the linguist [\textbf{who} Mary will invite \sout{\textbf{who}}].
\ex The candidate [\textbf{who} we voted for \sout{\textbf{who}}] has already left the city.
\z
\z

Such operators (both in interrogative and relative clauses, and beyond) move to the left periphery because they have a function regarding clause typing: cases like (\ref{wh}) are identifiable as interrogative clauses precisely because there are overt interrogative elements in the left periphery, there being no distinctive interrogative intonation or word order changes (such as subject--auxiliary inversion) in embedded clauses.

\section{The problems to be discussed} \label{sec:1problems}
\subsection{The model} \label{sec:1model}
In current minimalist theory, the Complementiser Phrase (CP) is responsible for typing clauses and for encoding finiteness in finite clauses. Apart from complementisers, as pointed out in \sectref{sec:1functional} above, various operators can appear in this domain. Consider:

\ea
\ea I wonder \textbf{if} Ralph has arrived. \label{englishifch1}
\ex I wonder \textbf{whether} Ralph has arrived. \label{englishwhetherch1}
\z
\z

In (\ref{englishifch1}), \textit{if} is a complementiser and it types the subordinate clause as interrogative. In (\ref{englishwhetherch1}), there is no overt complementiser but the operator \textit{whether} is present. In such cases, it is assumed that a zero complementiser types the clause (since the CP can be projected only by a C head, which in this case is not visible, though; see \citealt[137--138]{bacskaiatkari2020jcgl} for discussion), yet a sound model of the CP-periphery must also clarify the role of the overt operator  in (\ref{englishwhetherch1}).

On the other hand, the CP is not restricted to hosting a single overt element: depending on the particular construction and the dialect, multiple elements may appear in the CP-domain. This is illustrated by (\ref{englishdfcch1}) for non-standard English and by (\ref{norwegiandfcch1}) for Norwegian (\citealt[175]{bacskaiatkaribaudisch2018}):

\ea \label{dfcch1}
\ea[\%]{I wonder \textbf{which book that} Ralph is reading. \label{englishdfcch1}}
\ex[]{\gll Peter spurte \textbf{hvem} \textbf{som} likte bøker. \label{norwegiandfcch1}\\
           Peter asked.\textsc{3sg} who that liked books\\
\glt `Peter asked who liked books.'}
\z
\z

A proper formal account of the CP-domain must be able to condition when multiple overt elements are allowed and when not. Further, it must be clarified whether the appearance of several overt elements requires multiple CP projections, and in cases where it does, how word order restrictions can be modelled. The generation of multiple functional layers is in principle possible, yet it should be appropriately restricted to exclude the generation of superfluous layers that are empirically not motivated. This question is likewise relevant in cases involving a single overt C-element, since then the question arises whether and to what extent covert elements and phonologically not visible projections are present. 

Apart from the exact position of various elements in the CP, their function(s) must also be addressed. For instance, interrogative complementisers regularly mark finiteness as well. Consider:

\ea \label{ifwhetherch1}
\ea[]{I don't know \textbf{if} I should call Ralph. \label{iffinitech1}}
\ex[]{I don't know \textbf{whether} I should call Ralph. \label{whetherfinitech1}}
\ex[*]{I don't know \textbf{if} to call Ralph. \label{ifnonfinitech1}}
\ex[]{I don't know \textbf{whether} to call Ralph.  \label{whethernonfinitech1}}
\z
\z

In (\ref{iffinitech1}), the complementiser \textit{if} introduces a finite embedded interrogative clause, and as the ungrammaticality of (\ref{ifnonfinitech1}) shows, it is incompatible with a non-finite clause, suggesting that it encodes finiteness apart from the interrogative property, too. By contrast, the operator \textit{whether} is compatible with both a finite clause, see (\ref{whetherfinitech1}), and with a non-finite clause, see (\ref{whethernonfinitech1}), indicating that the overt marking of the interrogative property is not incompatible with a non-finite clause in English. Since \textit{whether} is not specified for finiteness, it should be clear that finiteness is specified by some other element in (\ref{whetherfinitech1}); the question is whether there is a separate element encoding finiteness in (\ref{iffinitech1}) as well and, if so, how the restriction of \textit{if} to finite clauses can be explained.

Finally, the function(s) of various left-peripheral elements must be clarified also because there are some non-trivial combinations in which elements seem to be largely similar, as in the non-standard German example in (\ref{alswiech1}) below:

\ea
\ea[\%]{\gll Ralf ist größer \textbf{als} \textbf{wie} Maria. \label{alswiech1}\\
Ralph is taller than as Mary\\
\glt `Ralph is taller than Mary.'}
\ex[]{\gll Ralf ist größer \textbf{als} Maria. \label{alsch1}\\
Ralph is taller than Mary\\
\glt `Ralph is taller than Mary.'}
\ex[\%]{\gll Ralf ist größer \textbf{wie} Maria. \label{wiech1}\\
Ralph is taller as Mary\\
\glt `Ralph is taller than Mary.'}
\ex[]{\gll Ralf ist so groß \textbf{wie} Maria. \label{wieequatch1}\\
Ralph is so tall as Mary\\
\glt `Ralph is as tall as Mary.'}
\z
\z

In (\ref{alswiech1}), the elements \textit{als} and \textit{wie} both seem to mark the comparative nature of the clause, whereby single \textit{als} is the comparative particle in Standard German comparatives, as shown in (\ref{alsch1}), and single \textit{wie} is the comparative particle in equatives, see (\ref{wiech1}), and in certain dialects also in comparatives, see (\ref{wieequatch1}). In such cases, the question is to what extent there is genuine doubling at hand and how it can be modelled.

A central issue for the theory regarding the above-mentioned constructions is how the various properties associated with clause typing are encoded in the syntax. The occurrence of multiple overt elements in the left periphery indicates some complexity and raises the question whether a single CP projection is sufficient or whether multiple projections are necessary. In this respect, cartographic approaches (starting from \citealt{rizzi1997}) have a relatively clear answer, inasmuch as they assume a designated projection (generated in narrow syntax) for each feature, which necessarily leads to multiple projections in the above cases. In turn, this kind of approach is prone to reducing analysis to description, as the observed surface patterns are restated as syntactic projections; the question in this regard is whether such models are tenable or at least favourable to more minimalist approaches. These questions will be addressed in \chapref{ch:2}.

\subsection{Embedded interrogative clauses} \label{sec:1interrogative}
In Standard English, Standard German and Standard Dutch, there is no overt complementiser with an overt interrogative operator. This is illustrated in (\ref{whothatch1}) for English embedded interrogatives:

\ea	I don't know \textbf{who (*that)} has arrived. \label{whothatch1}
\z

As can be seen, the complementiser \textit{that} is not permitted in Standard English in embedded constituent questions. This phenomenon is traditionally termed as the ``Doubly Filled COMP Filter'' (going back to the work of \citealt{chomskylasnik1977}). By contrast, there are languages and also many West Germanic varieties that allow such patterns, as in (\ref{dfcch1}) above. Further examples are given in (\ref{dfcintch1}) below from non-standard English (\citealt[331, ex. 1]{baltin2010}) and from non-standard Dutch (\citealt[32]{bacskaiatkaribaudisch2018}):

\ea \label{dfcintch1}
\ea[\%]{They discussed a certain model, but they didn't know \textbf{which model that} they discussed.}
\ex[\%]{\gll Peter vroeg \textbf{wie} \textbf{dat} er boeken leuk vindt. \label{dutchdfcch1}\\
Peter asked.\textsc{3sg} who that of.them books likeable finds\\
\glt `Peter asked who liked books.'}
\z
\z

Such patterns are often referred to as doubling patterns, indicating that there are two overt elements in a single CP: the \textit{wh}-phrase in the specifier and the complementiser in C. Note that this is not exceptional: the specifier of the CP and the C head can be both lexicalised overtly in main clauses, as in T-to-C movement in English interrogatives, and in V2 clauses in German and Dutch main clauses. Consider the examples for main clause interrogatives in Standard English:

\ea \label{ttocch1}
\ea	\textbf{Who saw} Ralph? \label{whosawch1}
\ex	\textbf{Who did} Ralph see? \label{whodidch1}
\z
\z

In this case, doubling in the CP involves a \textit{wh}-operator in [Spec,CP] and a verb in C. T-to-C movement is visible by way of \textit{do}-insertion in (\ref{whodidch1}), though not in (\ref{whosawch1}): in principle, one might analyse (\ref{whosawch1}) as not involving the movement of the verb to C, but the CP is clearly doubly filled in (\ref{whodidch1}).

Similarly, in German (and Dutch) V2 declarative clauses a verb moves to C, while another constituent moves to [Spec,CP] due to an [edge] feature (see \citealt{thiersch1978diss}, \citealt{fanselow2002, fanselow2004isis, fanselow2004}, \citealt{frey2005}, \citealt{denbesten1989}). Consider:

\ea \label{v2ch1}
\ea \gll \textbf{Ralf} \textbf{hat} morgen Geburtstag.\\
Ralph has tomorrow birthday\\
\glt `Ralph has his birthday tomorrow.'
\ex \gll \textbf{Morgen} \textbf{hat} Ralf Geburtstag.\\
tomorrow has Ralph birthday\\
\glt `Ralph has his birthday tomorrow.'
\z
\z

As can be seen, the fronted finite verb is preceded by a single constituent in each case, and since the first constituent is not a clause-typing operator in either case, it is evident that doubling in the CP in V2 clauses is independent of the interrogative property.

It is therefore clear that the ``Doubly Filled COMP Filter'' should be more restricted in its application domain. In principle, one could say that an operator and a complementiser with largely overlapping functions are not permitted to co-occur in standard West Germanic languages, or that the Doubly Filled COMP Filter should be seen as some kind of an economy principle. Still, the problem remains that the notion of the Doubly Filled COMP Filter implies that the C head and [Spec,CP] position would be filled without the Filter, and the Filter is responsible for ``deleting'' the content of C. 

Regarding this, at least two major questions arise. First, it should be clarified what requirement is responsible for filling C even in the presence of an overt operator in [Spec,CP], as in (\ref{dfcintch1}). Second, the question is what kinds of elements may appear in C: in particular, if elements other than complementisers can satisfy the requirement of filling C, then the deletion approach is probably mistaken.

In addition, there is a theoretical problem with the notion of the Filter, which arises from a merge-based, minimalist perspective, while it is less problematic in X-bar theoretic terms. X-bar theoretic notions can at best taken to be descriptive designators that are derived from more elementary principles, in the vein of \citet{kayne1994} and \citet{chomsky1995}.\footnote{Note that I will also use X-bar structures for representational purposes in this book.} Under this view, the position of an element (specifier, head, complement) is a result of its relative position when it is merged with another element, and which element is chosen to be the label. By contrast, the notion of the Doubly Filled COMP Filter, as applied to a CP (as in \citealt{baltin2010}), implies that a phrase is generated with designated, pre-given head and specifier positions, and that there are additional rules on whether and to what extent they can be actually ``filled'' by overt elements. In a merge-based account, there are no literally empty positions, as no positions are created independent of merge: zero heads and specifiers reflect elements that are either lexically zero or have been eliminated by some deletion process (e.g. as lower copies of a movement chain or via ellipsis). In other words, Doubly Filled COMP effects should be accounted for in a way other than referring to a pre-given XP. These questions will be addressed in \chapref{ch:3}.

\subsection{Relative clauses} \label{sec:1relative}
West Germanic languages show considerable variation in terms of elements introducing relative clauses. There are two major strategies: the relative pronoun strategy and the relative complementiser strategy. In present-day Standard English, both of these strategies are attested. Relative pronouns are illustrated in (\ref{whrelativebasic}) below:

\ea \label{whrelativebasic}
\ea I saw the woman \textbf{who} lives next door in the park. \label{whosubjectch1}
\ex The woman \textbf{who/whom} I saw in the park lives next door. \label{whoobjectch1}
\ex I saw the cat \textbf{which} lives next door in the park. \label{whichsubjectch1}
\ex The cat \textbf{which} I saw in the park lives next door. \label{whichobjectch1}
\z
\z

As can be seen, relative pronouns show partial case distinction and distinction with respect to whether the referent is human or non-human. In particular, \textit{who}/\textit{whom} is used with human antecedents, as with \textit{the woman} in (\ref{whosubjectch1}) and (\ref{whoobjectch1}); the form \textit{who} can appear both as nominative and as accusative, while the form \textit{whom} used for the accusative is restricted in its actual appearance (formal/marked). With non-human antecedents, such as \textit{the cat} in (\ref{whichsubjectch1}) and (\ref{whichobjectch1}), the pronoun \textit{which} is used, which shows no case distinction. Note that apart from human referents, \textit{who(m)} is possible with certain animals: these are the ``sanctioned borderline cases'' (see \citealt[41]{herrmann2005}, quoting \citealt{quirkgreenbaumleechsvartvik1985}). On the other hand, non-standard dialects allow \textit{which} with human referents, as illustrated in (\ref{boywhichch1}) below (\citealt[42, ex. 4a]{herrmann2005}):

\ea {[}\ldots] And the boy \textbf{which} I was at school with [\ldots] \label{boywhichch1}\\
(\textit{Freiburg English Dialect Corpus} Wes\_019)
\z

At any rate, English relative pronouns are formed on the \textit{wh}-base and no longer on the demonstrative base: note that this is historically not so, and the present-day complementiser \textit{that} was reanalysed from a pronoun, while the \textit{wh}-based relative operators appeared only in Middle English (\citealt{vangelderen2009}).

Accordingly, the complementiser \textit{that} constitutes the second major strategy:

\ea
\ea I saw the woman \textbf{that} lives next door in the park.
\ex The woman \textbf{that} I saw in the park lives next door.
\ex I saw the cat \textbf{that} lives next door in the park.
\ex The cat \textbf{that} I saw in the park lives next door.
\z
\z

The complementiser \textit{that} is not sensitive to case and to the human/non-human distinction, which follows from its status as a C head. 

Given the availability of two strategies, a number of questions arise regarding their distribution. First, while it seems logical that the two strategies can be combined, doubling, as mentioned above, is less likely to appear in relative clauses than in embedded interrogatives, which raises the question what restrictions apply here. Second, as also mentioned above, there seems to be a preference for the complementiser strategy in West Germanic varieties that have a choice in the first place: it should be investigated why this should be so and why relative operators still exist even in dialects that have the complementiser strategy. Third, apart from their syntagmatic distribution (combinability), the paradigmatic distribution of the two strategies must likewise be examined, that is, whether the individual strategies can relativise all functions and how potential differences correlate with the featural properties of the respective items. These questions will be addressed in \chapref{ch:4}.

\subsection{Embedded degree clauses} \label{sec:1degree}
Embedded degree clauses fall into two major types: degree equatives, also called comparatives expressing equality, as given in (\ref{astallch1}), and comparatives expressing inequality, as given in (\ref{tallerthanch1}):

\ea \label{comparisonch1}
\ea Ralph is as tall \textbf{as} Mary is.\label{astallch1}
\ex Ralph is taller \textbf{than} Mary is.\label{tallerthanch1}
\z
\z

In (\ref{astallch1}), the subclause introduced by \textit{as} expresses that the degree to which Mary is tall is the same as to which Ralph is tall, while in (\ref{tallerthanch1}) the subclause introduced by \textit{than} expresses that the degree to which Mary is tall is lower than the degree to which Ralph is tall.

The comparison constructions presented in (\ref{comparisonch1}) above are instances of degree comparison: there is one degree expressed in the matrix clause and another one expressed in the subclause. The matrix degree morpheme is \textit{as} in degree equatives and it selects an \textit{as}-clause, while the matrix degree morpheme in degree comparatives is -\textit{er} (or \textit{more}, which is actually a composite of -\textit{er} and \textit{much}, see \citealt{bresnan1973}, \citealt{bacskaiatkari2014diss, bacskaiatkari2018langsci}). However, it is possible to have comparison without degree; consider:

\ea \label{nondegreecomparisonch1}
\ea[]{Mary is tall, \textbf{as} is her mother. \label{tallasch1}}
\ex[]{Mary is glamorous \textbf{like} a film-star. \label{glamorouslikech1}}
\ex[]{Farmers have other concerns \textbf{than} the farm bill. \label{otherthanch1}}
\ex[\%]{Life in Italy is different \textbf{than} I expected. \label{differentthanch1}}
\z
\z

In these cases, there is obviously no matrix degree element. The sentences in (\ref{tallasch1}) and (\ref{glamorouslikech1}) express merely similarity with respect to the property denoted by the adjective; in (\ref{glamorouslikech1}), the subclause is introduced by \textit{like} and not by \textit{as}, a further difference from degree equatives. Given the availability of non-degree equatives, \citet[35]{jaeger2018} suggests that comparison constructions can be grouped into three major categories: non-degree equatives, degree equatives, and comparatives; these constitute a markedness hierarchy in this order (non-degree equatives being the least marked). However, constructions like (\ref{otherthanch1}) and (\ref{differentthanch1}) indicate that there is in fact a fourth category as well: these are non-degree comparatives expressing difference. This category seems not to be productive as the availability of the \textit{than}-clause is dependent on the presence of a particular element expressing difference in the matrix clause: the word \textit{other} or, at least in American English, the adjective \textit{different} are potential candidates.

While the patterns in (\ref{comparisonch1}) suggest a relatively simple left periphery consisting of a single CP at first sight, further data indicate that comparatives regularly demonstrate doubling, similarly to the German pattern given in (\ref{alswiech1}) above, which seems to be present at least underlyingly in comparatives proper in all cases, while equatives may indeed have a single CP in the subclause. Further, the left periphery of degree clauses is also relevant in terms of polarity marking. In English, both degree equatives and comparatives are negative polarity environments, as illustrated by the following examples containing the negative polarity items \textit{any} and \textit{ever}:

\ea \label{englishch1}
\ea Sophia is as nice as \textbf{any} other teacher in the school. \label{asanych1}
\ex Sophia is nicer than \textbf{any} other teacher in the school. \label{thananych1}
\ex Museums are as popular as \textbf{ever} before. \label{aseverch1}
\ex Museums are more popular than \textbf{ever} before. \label{thaneverch1}
\z
\z

Negative polarity items are licensed in other negative polarity contexts (cf. \citealt{klima1964}) such as interrogatives, clausal negation and conditionals, but not in affirmative clauses (\citealt[531, ex. 11]{seuren1973}):

\ea
\ea[*]{\textbf{Any} of my friends could \textbf{ever} solve those problems.}
\ex[]{Could \textbf{any} of my friends \textbf{ever} solve those problems?}
\ex[]{At no time could \textbf{any} of my friends \textbf{ever} solve those problems.}
\ex[]{If \textbf{any} of my friends \textbf{ever} solve those problems, I'll buy you a drink.}
\z
\z

While the data in (\ref{englishch1}) suggest that English is symmetrical regarding negative polarity across the two major types of comparison clauses, German shows an asymmetric pattern: comparatives but not equatives have negative polarity:

\ea \label{germanch1}
\ea[*]{\gll Museen sind so beliebt wie \textbf{jemals} zuvor. \label{wiejemalsch1}\\
museums are so popular how ever before\\
\glt `Museums are as popular as ever before.'}
\ex[]{\gll Museen sind beliebter als \textbf{jemals} zuvor. \label{alsjemalsch1}\\
museums are more.popular as ever before\\
\glt `Museums are more popular than ever before.'}
\z
\z

The data point to the conclusion that the role of the left periphery in comparatives extends to marking polarity, not in terms of designated projections but as part of the featural makeup of the individual projections that are present in the derivation anyway due to independent clause-typing and semantic properties. These issues will be investigated in \chapref{ch:5}.

\subsection{Information structure and ellipsis} \label{sec:1information}
Certain constituents may undergo topicalisation or focalisation involving movement to the left periphery of the clause. Consider the following examples taken from \citet[285, ex. 1 and 2]{rizzi1997}:

\ea \label{englishrizzich1}
\ea {[}Your book]\textsubscript{i}, you should give \textit{t}\textsubscript{i} to Paul (not to Bill). \label{topiccommentch1}
\ex {[}YOUR BOOK]\textsubscript{i} you should give \textit{t}\textsubscript{i} to Paul (not mine). \label{focuspresuppch1}
\z
\z

The construction in (\ref{topiccommentch1}) illustrates topicalisation, and the one in (\ref{focuspresuppch1}) focalisation. Apart from interpretive differences, they crucially differ in their intonation patterns: a topic is separated by a so-called ``comma intonation'' from the remaining part of the clause (the comment), while a focus bears focal stress and is thus prominent with respect to presupposed information (see \citealt[258]{rizzi1997}).

Such movement operations are clearly instances of A-bar movement, and since they are apparently not driven by clause-typing features either, they raise the question what triggers movement in the first place. The cartographic model proposed by \citet{rizzi1997}, adopted by others such as \citet{poletto2006}, proposes that leftward movement in these cases targets designated left-peripheral positions: TopP and FocP. Movement is driven by specific features making reference to information-structural properties: this operator-like feature agrees with the functional head (Top or Foc). In essence, this kind of movement is supposed to be similar to ordinary operator movement involving \textit{wh}-operators or relative operators. Such an assumption is problematic, though: while [wh] and [rel] features are lexically determined, [topic] and [focus] features are obviously not. Taking the examples in (\ref{englishrizzich1}) above, in both cases the entire phrase \textit{your book} is topicalised or focussed, and the phrase as such, being compositional, is not part of the lexicon. This indicates that features like [topic] and [focus] would have to be added during the derivation. In addition, even if one were to assume that a lexical element like \textit{Mary} can be equipped with information-structural features in the lexicon (contrary to generally accepted views about the lexicon and lexical features, cf. \citealt{neelemanszendroei2004} and \citealt{dendikken2006}), this would leave us with various lexical entries for \textit{Mary}: a neutral entry (not specified for any information-structural category), a focussed one, a topicalised one, not to mention possible fine-grained categories such as contrastive topic or aboutness topic. 

Moreover, foci (and topics) can occur in non-fronted positions. This is illustrated by the following examples taken from \citet[172, ex. 6c and 6d]{fanselowlenertova2011}, both answering the question \textit{What happened?}:

\ea
\ea \gll \textbf{Eine} \textbf{LAWINE} haben wir gesehen!\\
a.\textsc{f.acc} avalanche have.\textsc{1pl} we seen\\
\glt `We saw an AVALANCHE!'
\ex \gll Wir haben \textbf{eine} \textbf{LAWINE} gesehen!\\
we have.\textsc{1pl} a.\textsc{f.acc} avalanche seen\\
\glt `We saw an AVALANCHE!'
\z
\z

This kind of optionality obviously contrasts with the behaviour of ordinary \textit{wh}-movement (and relative operator movement) in German, which always targets the CP-domain. Note also that there are certain fronted elements in the German CP (occupying the ``first position'') that clearly do not correspond to information structural categories such as topic and focus. Consider the following examples from \citet[173, ex. 7a]{fanselowlenertova2011}:

\ea \gll \textbf{Wahrscheinlich} hat ein Kind einen Hasen gefangen.\\
probably has a.\textsc{n.nom} child a.\textsc{m.acc} rabbit caught.\textsc{ptcp}\\
\glt `A child has probably caught a rabbit.'
\z

In this case, the adverb \textit{wahrscheinlich} `probably' is a sentential adverb that evidently lacks a discourse function such as topic or focus.

These considerations indicate that movement is not always driven by lexical features; if so, this has consequences regarding the way functional left peripheries are organised.

As mentioned above, clausal ellipsis is also closely connected to the issue of functional left peripheries. The prototypical case for this is sluicing, demonstrated in (\ref{sluicech1}) below:

\ea\label{sluicech1} Someone phoned grandma but I don't remember \textbf{WHO} \sout{phoned grandma}.\z

The elliptical clause is embedded in a clause conjoined with another main clause: this clause (\textit{someone phoned grandma}) contains the antecedents for the elided elements in the elliptical clause. The elliptical clause contains a single remnant, the subject \textit{who}, which bears main stress: it contains non-given information. Ellipsis is licensed as all elided information is recoverable. The assumption regarding the implementation of ellipsis in grammar (\citealt[55--61]{merchant2001} and \citealt[670--673]{merchant2004}) is that there is an ellipsis feature, [E]. This is merged with a functional head (such as C) and the complement of this head is elided. The [E] feature is specified as having either an uninterpretable [wh] or an uninterpretable [Q] feature, ensuring that it occurs only in (embedded) questions. As shown by \citet{vancraenenbroeckliptak2006} and \citet{hoytteodorescu2012}, this particular syntactic condition is highly unsatisfactory as many languages allow canonical ellipsis processes such as sluicing also from non-interrogative projections, including relative clauses and projections hosting foci. Rather, it seems that the [E] feature is not tied to a specific projection or features; indeed, \citet{merchant2004} also proposes that a functional projection, FP, can be headed by [E] in fragment answers, illustrated in (\ref{phonegrandma}) below:

\begin{exe}
\ex \label{phonegrandma}
\begin{xlist} 
\exi{A:} Who phoned grandma?
\exi{B:} \textbf{Liz} \sout{phoned grandma}.
\end{xlist}
\end{exe}

In this case, the remnant (\textit{Liz}) is the subject and the rest of the clause is elided. Since in English the subject DP in declarative clauses is located in [Spec,TP] and not in [Spec,CP], the ellipsis mechanism assumed for sluicing (the [E] feature located in C) does not automatically carry over. As \citet{merchant2004} assumes, there is an unspecified FP projection hosting the remnant in its specifier, landing there by movement. In this vein, it seems that leftward movement can target functional projections due to reasons other than clause-typing. This raises the question whether such functional projections may not ultimately have a more substantial role in the architecture of a clause than merely enabling ellipsis.

Questions related to information structure and ellipsis, particularly regarding their relevance for the proposed model, will be addressed in \chapref{ch:6}.

\section{Methodology}
This book aims at examining the syntax of functional left peripheries in West Germanic from a generative perspective, applying the minimalist framework in the analysis of syntactic structure. The main focus lies on the analysis of English and German, and to a lesser extent on Dutch. As language variation is a central issue, other Germanic languages will also occasionally be considered, as well as other European languages (mostly Romance and Slavic, and to some extent also Greek and Uralic). Language comparison can help to understand the cross-linguistic status of the West Germanic patterns: beyond that, however, the present investigation cannot carry out a more detailed analysis of these languages. 

Since the clausal left periphery is a well-studied area of linguistics (see \sectref{sec:1aims}), part of the investigation is dedicated to the analysis of already known patterns, also taken to be a basis for further inquiries. In addition, however, the book presents novel empirical data gained via corpus studies, questionnaires, and grammaticality judgement experiments. Regarding this, it must be kept in mind that the individual West Germanic languages (and their varieties) under scrutiny differ considerably in terms of how accessible the relevant data are and to what extent they have been discussed in the literature. 

As far as historical data are concerned, the present investigation relies on parsed corpora to identify which patterns were used in the given periods and what their frequency is. Regarding English, the Michigan Corpus of Middle English Prose and Verse was used; in addition, I compiled a database on relative clauses in the King James Bible and its modernised version. Regarding German, the DDD Referenzkorpus Altdeutsch was used. For present-day dialect data, the SyHD atlas on Hessian dialects and the SynAlm database on Alemannic dialects have been used.

As part of my project (BA 5201/1) on functional left peripheries, I obtained data from various Germanic languages (Dutch, Swedish, Danish, Norwegian, Icelandic) via an online questionnaire; this allows for a direct comparison of the languages involved. For each language, two informants were gathered who translated sentences from English as well as answered specific questions about the combinability of certain elements. The questionnaire contains 147 questions altogether. The results have been published in an open access database under \citet{bacskaiatkaribaudisch2018} and will be referenced throughout this work.

Finally, the book also presents the results of a grammaticality judgement experiment (see \citealt{schuetze2016} on the methodology) on elliptical comparative clauses in German. This allows for a more fine-grained analysis than the mere grammaticality judgements available thus far in the literature.

\section{Previous work}
The present book builds on results gained in my research projects and partly published in earlier papers; these works will be referenced in the relevant chapters as well. In this section, I would like to point out how the present investigation relates to and differs from these articles, to provide better orientation for the reader in this respect.

\chapref{ch:2} summarises the most important principles regarding the proposed non-cartographic model. The basic ideas were spelt out in \citet{bacskaiatkari2018sardis} regarding data from South German dialects and some major concerns regarding the cartographic model were also expressed in terms of the proposal made by \citet{baltin2010}. In the present book, the scope of the investigation is naturally larger; in addition, this chapter contains a detailed critical review of the literature, pointing out additional problems that were not discussed before, in particular regarding the original cartographic proposal by \citet{rizzi1997, rizzi2004}.

\chapref{ch:3} discusses embedded interrogative clauses. The core part of this chapter was published in \citet{bacskaiatkari2020jcgl}, with a particular emphasis on the relation between Doubly Filled COMP patterns in German and V2 syntax. The present investigation has a wider empirical and theoretical scope. In the original study, results from a corpus study on Middle English \textit{whether} were discussed: this was based on a smaller sample from the two versions of the Wycliffe Bible. The present study includes the results from the entire text (for both versions). Regarding the theoretical scope, the present study includes a detailed critical study of alternative analyses of Doubly Filled COMP effects, in particular that of the original proposal made by \citet{chomskylasnik1977}, which was not discussed before. In addition, the present book contains a section on long movement.

\chapref{ch:4} examines relative clauses. A core part of the discussion is centred on a corpus study carried out on the King James Bible. Some implications regarding the subject/object asymmetries observed in the choice of relativisation strategies were discussed in \citet{bacskaiatkari2020nordlyd}. This previous study was based on a smaller data set: for the present study, the entire King James Bible was taken into account, using the parallel loci to relative clause introduced by \textit{who(m)}, \textit{which} and \textit{that} in the modernised version. The present book also discusses some statistical findings that were not included in the previous investigations at all. In addition, the present study connects the findings to the general non-cartographic approach, as well as to the dialectal German data and it also presents a detailed account of equative relative clauses, also connecting the findings to the proposal made by \citet{brandnerbraeuning2013}.

\chapref{ch:5} is dedicated to embedded degree clauses. Some of the findings regarding German historical data and their diachronic development were discussed in \citet{bacskaiatkari2021oup}. The present study is more extensive in this respect and it also places the discussion of these data into a cross-linguistic setting, showing that the polarity differences between equative and comparative clauses hold across languages. The analysis is also connected to the model proposed in this book, showing the importance of analysing multiple left-peripheral projections in a non-cartographic model. The proposed account relies on many insights of \citet{jaeger2018}, yet there are some important differences in the syntactic structure between the two models: this issue is also discussed in detail.

\chapref{ch:6} analyses ellipsis processes in embedded clauses, concentrating on elliptical comparatives in German. The key idea behind the proposed analysis for German was expressed in \citet{bacskaiatkari2017atoh}; however, that study was entirely based on classical, introspective grammaticality judgements in very specific context, explicitly targeted at measuring ambiguity. The present study includes the results of a grammaticality judgement experiment and it also relates the findings to the general theory of ellipsis and information structure.

\section{Roadmap}
This book is structured as follows. In \chapref{ch:2}, I will introduce the basic assumptions regarding the proposed model. Following this, the book offers in-depth analyses of the three major constructions that will be examined here: \chapref{ch:3} addresses embedded interrogatives, \chapref{ch:4} addresses relative clauses, and \chapref{ch:5} addresses embedded degree clauses. In \chapref{ch:6}, I show that the analysis can be extended beyond the scope of clause typing proper, connecting it to issues related to information structure and ellipsis.
 
\chapter{An evolving landscape}\label{chap2}

Around a decade ago, an important review article entitled ``Language evolution in the laboratory'' \citep{scott2010language} was published in \textit{Trends in Cognitive Sciences}. Its central message, in my opinion, was that it was becoming possible, at last, to approach at least certain aspects of language evolution in a scientific manner. This was a sharp departure from over a century of statements declaring that language evolution was a mystery.

Remnants of this old attitude still exist \citep{hauser2014mystery}; they typically invoke in a tedious fashion the 1866 ban on all discussion of the evolution of language imposed by the Linguistic Society of Paris; they also frequently cite Lewontin's pronouncement that we will never know why cognition evolved the way it did \citep{lewontin1998evolution}. But things have changed quite dramatically over the past two decades, so much so that it has become possible to contemplate ``controlled hypothesis-testing through experimentation'' \citep{motamedi2019evolving} in the domain of language evolution.

I still recall being told as a graduate student that the topic of language evolution was more a matter of science fiction than science, and that this was best left as a domain of study for after retirement. Today, some of the brightest students I know are actively engaged in this field, illustrating the massive progress made over the past 20 years, well attested in the Proceedings of the Evolang conference series, as well as in the creation of centers for the study of language evolution in Edinburgh and more recently Z\''{u}rich. The main change (still ongoing), to my mind, is the resistance to exploring hypotheses until they can be formulated in a way that can be put to the test. A change from `I think \textit{x}' to `I think \textit{x} and I can test \textit{x} doing \textit{y}.'

The efforts of members of the Centre for Language Evolution at the University of Edinburgh, led by Simon Kirby, have shown how combining the development of artificial languages (mini-grammars) in a laboratory setting \citep{kirby2008cumulative,kirby2015compression}, as well as agent-based modelling approaches controlling for biases that language users in the lab bring to the task in an unconscious manner \citep{thompson2016culture}, reveals how learnability and expressivity pressures shape grammars. Subsequent work from other centers (e.g. \cite{raviv2019larger,raviv2020language,raviv2021makes}) also experimentally demonstrates how communicative contexts impact grammar formation and the emergence of new languages. While it is often said that such work only addresses language change (`glossogeny'), and not language evolution proper (language phylogeny, the emergence of the modern language capacity),\footnote{Terminology introduced in \cite{hurford1990nativist}.} I do not find this dichotomy particularly useful, and believe that a continuum of cognitive biases that interact with changing communicative conditions from which language-readiness emerges, shaping the range of grammars acquired, is a more adequate stance (more on this in chapter \ref{chap4}).

The same year the review article by \cite{scott2010language} appeared, the first draft of the Neanderthal genome was published \citep{green2010draft}, starting a revolution that continues unabated to this day \citep{reich2018we}. As we will see later on, the successful retrieval of ancient DNA, from a few skeletal remains and now even cave sediments, and of ancient proteins, allows us to ask questions at an unprecedented level of resolution and dramatically changes what we mean by ``fossil record''. The debt we owe to Svante Pääbo and his collaborators is hard to overstate \citep{paabo2014neanderthal,meyer2012high,prufer2014complete,prufer2017high,mafessoni2020high,slon2017neandertal,vernot2021unearthing,zavala2021pleistocene,welker2016palaeoproteomic,welker2020dental}.

Yet this massive amount of data that is now accessible would be ``empty'' if it were not for the progress made in linking the genotype and the phenotype. In the domain of language, the work pioneered by Simon Fisher on \textit{FOXP2} is the gold standard \citep{lai2001forkhead}, and arguably one of the most significant achievements in the language sciences in the past twenty five years \citep{fisher2009foxp2,fisher2015genetics,fisher2019human,den2021molecular}. It has taught us that for all the intricacies and levels of analyses separating genes and behavior, careful work can illuminate central issues that Lenneberg could only dream of when he wrote his classic book, \textit{Biological Foundations of Language}, over fifty years ago \citep{Lenneberg1967biological}.

Equally important for the success of what is sometimes called ``evolinguistics'' is the dramatic shift of perspective that took place in the domain of comparative psychology. This is well-captured in the following passage from \cite{de2010towards}:

\largerpage
\begin{quote}	
Over the last few decades, comparative cognitive research has focused on the pinnacles of mental evolution, asking all-or-nothing questions such as which animals (if any) possess a theory of mind, culture, linguistic abilities, future planning, and so on. Research programs adopting this top-down perspective have often pitted one taxon against another, resulting in sharp dividing lines. Insight into the underlying mechanisms has lagged behind \ldots \clearpage


A dramatic change in focus now seems to be under way, however, with increased appreciation that the basic building blocks of cognition might be shared across a wide range of species. We argue that this bottom-up perspective, which focuses on the constituent capacities underlying larger cognitive phenomena, is more in line with both neuroscience and evolutionary biology.
\end{quote}

In the domain of language, calls for recognizing an ever broader ``community of descent'', to borrow a phrase from \cite{darwindescent}, are more and more frequent \citep{lattenkamp2018vocal}. Far from being rhetorical, these calls demonstrate how much one can learn about our kind by studying behavior in numerous species in accordance with Tinbergen's multi-level approach.

As Ernst Mayr was fond of saying, ``evolutionary biology [unlike physics] is a historical science, [where] one constructs a historical narrative, consisting of a tentative reconstruction of the particular scenario that led to the events one is trying to explain'' \citep{mayr2000darwin}. Narratives will continue to dominate evolutionary investigations into language, but crucially, thanks to the progress made in key areas that I singled out above, these narratives are enriched with, and constrained by, ``numbers''. Hypotheses can now be put to the test.

It becomes very apparent in this context that simple narratives, appealing as they may appear, are hopelessly misguided. Recalling the words of H. L. Mencken, ``For every complex problem there is an answer that is clear, simple, and wrong''. What more complex problem is there than the problem of language evolution?

Accordingly, the simple, clear, ``minimalist'', and influential evolutionary scenario advocated by Berwick and Chomsky in their book \textit{Why Only Us} \citep{berwick2016only} must be wrong.\footnote{If I am right, this has non-trivial ramifications for the minimalist program. Over the years, talk of optimization, efficiency, etc., which occupied center stage in the early days of the program, has been replaced by a focus on evolutionary considerations. If such considerations lead to an impasse, the program as a whole may indeed have been (at best) premature.} I have tried to say so on several occasions \citep{boeckx2017not,martins2019language,de2020evolutionary}. Very briefly: it is wrong because it disregards the comparative evidence (`only us'), it fails to appreciate the multi-level approach required to link genotype and phenotype (claiming that a single mutation yields the simple, atomic operation ``merge''), it keeps the discussion at the logical level, without attempting to even sketch a plausible path to testing it, and does not engage with the many lessons coming from the great discoveries in paleo-sciences over the past decade.

The reason I have spent time arguing against Berwick and Chomsky's narrative is not only because it was proposed by influential linguists, but because it is representative of a family of approaches that linguists remain attracted to: it presupposes that other animals don't have much to teach us about the core of our language faculty, because essentially they are non-linguistic creatures. The gap between them and us is a chasm. It also takes for granted that our language capacity is very recent in evolutionary terms, going back maybe 150 000 years. As such, so the claim goes, there was very little time to evolve a ``kludgy'' language organ (cf. \cite{marcus2009kluge}). Accordingly, a narrative must be developed that keeps the core language faculty essentially free of evolutionary tinkering.

Such a narrative (in many ways, the culmination of the minimalist program envisaged by Chomsky) clashes with recent attempts to attribute a significant portion of our ``modern'' language faculty to the last common ancestor shared with our closest extinct relatives \citep{dediu2013antiquity,dediu2018neanderthal}. It also clashes with mounting evidence for a complex, temporally very extended, mosaic-like evolution of our lineage \citep{scerri2018did,Bergstrom}. Also, it makes certain assumptions about how many changes can be favored by natural selection within a relatively short window of time which are not obviously true---indeed, very implausible \citep{de2020evolutionary}. Last, but not least, it grants too much power to linguistic theorizing. As argued in \cite{martins2019language}, it is fallacious to draw a direct correspondence between the formal structure of a computational operation and the biological changes that would lead to it.\footnote{In their reply to \cite{martins2019language}, \cite{berwick2019all} completely---and surprisingly---miss this point; see \cite{martinsboeckx_clar} for illustration.} It is what theoretical linguists would love to be able to do: it would make their theoretical work immediately relevant for evolutionary claims. But it is logically incorrect. This is precisely why, in my opinion, evolutionary considerations impact how we do theoretical linguistics, or how we see the import of that work. If there is no such direct correspondence, if the link between genotype and phenotype is very complex indeed, I do not see any alternative to painstakingly developing linking hypotheses that, we hope, progressively spell out what it means to say that our linguistic condition is part of our human (biological) condition.

I want to insist once more on the importance of debunking simple accounts like Berwick and Chomsky's. It may well be that there will be certain behaviors or artifacts or anatomical traits that we can confidently ascribe exclusively to members of our species that ``emerged'' recently. Right now this is being questioned, but I would not be surprised if we are left with a small set of recent ``\textit{sapiens}-exclusive'' properties (brain changes giving rise to our globular skull, use of complex symbiotic tools like the bow and arrow, and some aspects of figurative art are fairly good bets in my current opinion), but crucially, even if the evidence settles along these lines, it should not be used to argue for a recent cognitive revolution that matches a minimalist vision of the language faculty. Rather, such evidence will have to be integrated into the complex mosaic of language that evolution has constructed over an extended period of time.

This is certainly a major lesson I learned from thinking about Darwin's problem: Evolutionary considerations invalidate certain theoretical frameworks that fail to come to grips with the ``complex dynamical system'' nature of language. The next two chapters deal with other lessons that pertain to a broader range of approaches, and implicate a larger number of researchers: even those linguists that readily accept that the evolutionary trajectory of our language capacity was long and complex still subscribe to certain views that I think we would do well to abandon. I'll focus on three such views here. One is that somehow, there is at least one aspect of language (typically, some aspect of syntax) that makes our language capacity special, and that as a result forms some sort of barrier in a comparative setting. Another is the belief that linguistic theory matters and that one's theory of language evolution depends on one's theory of language. And third, the claim that because languages don’t leave fossils, the evidence for studying the evolution of language is too sparse. These three claims are incorrect.

\documentclass[output=paper,colorlinks,citecolor=brown]{langscibook}
\ChapterDOI{10.5281/zenodo.14282810}
\author{Carolin Ulmer\orcid{}\affiliation{Freie Universität Berlin}}
%\ORCIDs{}

\title{The expression of motion events in Haitian Creole}

\abstract{This paper investigates the expression of motion events in Haitian Creole. A bipartite typology has been proposed by \citet{Talmy_1991}, sorting languages into verb-framed and satellite-framed languages, depending on where they express the Path component of motion events. \citet{Slobin_2004} expanded the typology by a third type, equipollently-framed languages, to include verb-serializing languages which can express the Path as well as the Manner component in a serial verb construction. Creole languages have so far received little to no attention in regard to this typology. Creole languages are especially interesting because they were formed in a situation of language contact. The investigation of their morphosyntactic features can shed light on the question of which features of the languages involved are passed on and which are not. This can in turn offer clues for the study of the markedness of these features. The languages which were relevant to the formation of Haitian Creole, French and Kwa languages, present different patterns here. In French, verb-framed patterns are predominantly used, but in some cases Manner verbs constitute the main verb of the sentence \citep{Pourcel_Kopecka_2005}. In contrast, Kwa languages can use verb serializations to encode motion events (\citealt{Ameka_Essegbey_2013}; \citealt{LambertBrtire_2009}), a pattern not found in French. In this paper, I describe a small study conducted in Berlin, Germany, in 2017, investigating the expression of Motion events by four native speakers of Haitian Creole. They narrated a picture story and described drawings depicting different combinations of Manner and Path components. A wide range of different morphosyntactic structures encoding motion events was elicited. Verb-framed patterns were frequently used, as well as different Manner-Path verb serializations. Only a few satellite-framed constructions were elicited, but using different Manner verbs and Path-PPs. Further research will need to test the acceptability of different Manner and Path elements in the particular structures.}


\IfFileExists{../localcommands.tex}{
   \addbibresource{../localbibliography.bib}
   \usepackage{langsci-optional}
\usepackage{langsci-gb4e}
\usepackage{langsci-lgr}

\usepackage{listings}
\lstset{basicstyle=\ttfamily,tabsize=2,breaklines=true}

%added by author
% \usepackage{tipa}
\usepackage{multirow}
\graphicspath{{figures/}}
\usepackage{langsci-branding}

   
\newcommand{\sent}{\enumsentence}
\newcommand{\sents}{\eenumsentence}
\let\citeasnoun\citet

\renewcommand{\lsCoverTitleFont}[1]{\sffamily\addfontfeatures{Scale=MatchUppercase}\fontsize{44pt}{16mm}\selectfont #1}
  
   %% hyphenation points for line breaks
%% Normally, automatic hyphenation in LaTeX is very good
%% If a word is mis-hyphenated, add it to this file
%%
%% add information to TeX file before \begin{document} with:
%% %% hyphenation points for line breaks
%% Normally, automatic hyphenation in LaTeX is very good
%% If a word is mis-hyphenated, add it to this file
%%
%% add information to TeX file before \begin{document} with:
%% \include{localhyphenation}
\hyphenation{
affri-ca-te
affri-ca-tes
an-no-tated
com-ple-ments
com-po-si-tio-na-li-ty
non-com-po-si-tio-na-li-ty
Gon-zá-lez
out-side
Ri-chárd
se-man-tics
STREU-SLE
Tie-de-mann
}
\hyphenation{
affri-ca-te
affri-ca-tes
an-no-tated
com-ple-ments
com-po-si-tio-na-li-ty
non-com-po-si-tio-na-li-ty
Gon-zá-lez
out-side
Ri-chárd
se-man-tics
STREU-SLE
Tie-de-mann
}
   \boolfalse{bookcompile}
   \togglepaper[3]%%chapternumber
}{}

\begin{document}
\maketitle

\section{Introduction}

The expression of motion events in different languages has been of great interest to many linguists since \citet{Talmy_1991} proposed his typology of them, sorting languages into two types depending on whether they typically express the Path of motion in the main verb or a so-called satellite, some other element that is closely associated with the main verb. Romance languages are often cited as typical members of the first group, called verb-framed languages, whereas Germanic languages represent the second group, named satellite-framed languages. Later, \citet{Slobin_2004} proposed a third type that he calls equipollently-framed languages to describe languages expressing both the Path and the Manner component in a verb serialization, a structure that is found e.g. in Mandarin Chinese. Even though many languages have been investigated with regard to the morphosyntactic structures used to encode the different components of motion events, there is still no research on this question for Romance-based Creole languages. As Creole languages were formed in a situation of language contact, their investigation can show which features of the languages involved were passed on. The present paper looks at the morphosyntactic expression of Motion events in Haitian Creole. A small pilot study was conducted with four native speakers of Haitian Creole in Berlin, Germany. After a short sociolinguistic interview to determine their language ideologies and habits of language use (which were deemed necessary as the speakers all lived away from their home country and in a multilingual environment), the speakers completed two different tasks. First, they narrated a picture story about a little bird flying out of its cage and house to explore the outside world. After that, they provided descriptions for single pictures which were assembled in order to control for several combinations of Manner and Path components of motion events. The results show that Haitian Creole possesses a rich inventory of morphosyntactic structures to express motion events. A preference exists for the use of verb-framed constructions, but Manner-Path verb serializations were also used frequently. Satellite-framed constructions were rare, but do not seem to be totally ungrammatical.

The structure of the paper is as follows. \sectref{sec:3:2} first gives a definition of motion events and then describes the three types of motion event encoding mentioned above. Then, the concept of Manner salience, which describes the frequency with which Manner elements are used in different languages, is introduced. Sections 2.3 and 2.4 provide an insight into the expression of motion events in French and some African languages which were relevant to the formation of Haitian Creole. After that, a first light is shed on motion event encoding in Haitian Creole from previous works on the Haitian Creole language. \sectref{sec:3:3} then outlines the study conducted by the author and explains how the data was classified. \sectref{sec:3:4} gives a broad overview of the different structures that were elicited, which are then discussed in \sectref{sec:3:5}.


\section{Motion events}\label{sec:3:2}

In this study, a motion event is understood as defined by \citet[60--61]{Talmy_1991}:

\begin{quote}
    The basic motion event consists of one object (the “Figure”) moving or located with respect to another object (the reference object or “Ground”). It is analyzed as having four components: besides “Figure” and “Ground”, there are “Path” and “Motion”. The “Path” [...] is the course followed or site occupied by the Figure object with respect to the Ground object. […] In addition to these internal components a Motion event can have a “Manner” or a “Cause”, which we analyze as constituting a distinct external event.
\end{quote}


In sentence \REF{ex:3:1}, the components of the motion event are distributed as follows:

\ea\label{ex:3:1}
\gll Tom is running {down the stairs}.\\
     {}  {} Manner  Path\\
\z
		
\textit{Tom} represents the Figure, \emph{running} the Manner, \emph{down} the Path, and \emph{the stairs} the Ground of the motion event described.

Different lexicalization patterns are found in the languages of the world concerning the motion event component expressed in the verb. A first type encodes Motion and Manner in the verb, which is typical e.g. in English, as seen in sentence \REF{ex:3:1}. A second type, which is typically found in Romance languages like French or Spanish, encodes Motion and Path in verbs, such as  \emph{descendre/bajar} (‘go.down’) \citep[62--68]{Talmy_1985}. These differences lead to different patterns of encoding motion events, described in the following.

\subsection{Motion event typology}

In his 1991 paper, Talmy develops a typology of motion event encodings, sorting the languages of the world into two types depending on the morphosyntactic element in which they express the Path component: verb-framed and satellite-framed languages \citep[486--487]{Talmy_1991}. Verb-framed languages, such as Romance languages, express the Path component in the main verb of the sentence.

\ea Spanish \citep[69]{Talmy_1985} \\\label{ex:3:2}
\glll La botella salió de la cueva flotando.\\
      The bottle went.out of the cave floating \\
      {}  {}     Path     {} {} {} Manner\\
\z
	                     	
Other languages, such as English or German, express the Path component in a so-called satellite, a term defined by \citet[102]{Talmy_1985} as “immediate constituents of a verb root other than inflections, auxiliaries, or nominal arguments, [related] to the verb root as periphery (or modifiers) to a head”. These languages are therefore named satellite-framed languages. The English counterpart to \REF{ex:3:2} can be seen in \REF{ex:3:3}.

\ea\label{ex:3:3}
\gll  The bottle floated {out of the cave}.\\
      {}  {}      Manner Path\\
\z
						
\citet{Slobin_2004} revises this binary typology by adding a third type, equipollently-framed languages, suited to describe languages with serial verb constructions where both Path and Manner can be expressed in a verb, as illustrated by the Mandarin Chinese example in \REF{ex:3:4}.

\ea\label{ex:3:4}Mandarin Chinese\\
\glll  Hǎiōu cóng dòng l\u{i} fēi chū.\\
       seagull from hole in fly exit\\
       {}      {}   {}   {}  Manner Path\\
\glt `The seagull flew out of the hole.'
\z

Much work has followed the papers of \citet{Talmy_1991} and \citet{Slobin_2004}, classifying many different languages into the different patterns. Many of these works have used the so-called “Frog Stories” also used in \citet{Slobin_2004}. In sections 2.3 and 2.4, a short overview will be given of the work on motion event encoding in Kwa languages as well as French, which have been relevant to the formation of Haitian Creole.

\subsection{Manner salience}

\citet{Slobin_2004} takes a more detailed look at the expression of Manner, that is to say the frequency with which it is expressed in different languages. To this end, he compares the encoding of one certain event in the Frog Stories, namely an owl flying out of a knothole \citep[224--225]{Slobin_2004}. In the verb-framed languages Spanish, French, Italian, Turkish and Hebrew, virtually no Manner verbs are used, see \REF{ex:3:5} for French.

\ea\label{ex:3:5}French \citep[224]{Slobin_2004}\\
\glll D'un trou de l'arbre sort un hibou.\\
      from.a hole in the.tree  exits  an owl\\
      {}   {}  {} {}     Path {} {} \\ 
\z

Between different satellite-framed languages, more variation can be found regarding the expression of Manner. In German and English, Manner verbs are not used very frequently (only by about 17--32\% of the speakers) to express the motion event in question. This is due to the fact that often deictic verbs are used with Path satellites, as in the German example in \REF{ex:3:6}.

\ea\label{ex:3:6}German\\
\glll  Aus dem Astloch kommt eine Eule raus.\\
      from \textsc{art.def.dat} knothole comes  \textsc{art.indef} owl out \\
      {}   {}  {}     {}    {}   {}  Path\\
\z

In the equipollently-framed languages Mandarin Chinese and Thai, Manner is expressed more frequently (by 40\% of the Mandarin and 59\% of the Thai speakers). In the SF language Russian, the Manner component of this event is expressed by 100\% of the speakers. In all these cases, either a deictic (\emph{pri-letet} `fly here') or a Path-prefix (\emph{vy-letet} `fly out') is added to the Manner verb \emph{letet} `to fly'.

Slobin comes to the conclusion that the frequency with which the Manner component is expressed depends on the language type as well as the morphosyntactic possibilities to encode Manner. He proposes to align languages along a scale of Manner salience, where languages expressing Manner in the main verb typically have a high Manner salience, whereas languages where Manner is subordinate to Path typically have low Manner salience \citep[250]{Slobin_2004}.


\subsection{Motion events in French}

As already mentioned above, French is classified as a verb-framed language, encoding Path in the main verb and Manner in a gerund.

\ea\label{ex:3:7}French\\
\glll  Elle entrait à la  maison en.courant.\\
       \textsc{3sg.f} entered \textsc{prep} \textsc{art.def} house run.\textsc{ger} \\
       {} Path {} {} {} Manner\\
\z


An extensive study of motion event encoding in French can be found in \citet{Pourcel_Kopecka_2005}. They analyze a total of 1800 written and 594 oral descriptions of motion events and, on this basis, describe five different patterns frequently found in French \citep[145--149]{Pourcel_Kopecka_2005}. The most frequent is the verb-framed type, as already shown in \REF{ex:3:7}. Another frequent pattern is the coordination of two verb phrases, one containing a Manner verb and the other containing a Path verb:\footnote{Motion events expressed in a single phrase are the main interest of the study, but because the coordinated pattern is so frequent in the data, it is nevertheless listed here.}

\ea\label{ex:3:8}French \citep[145]{Pourcel_Kopecka_2005}\\
\glll  Il court dans une rue puis rentre dans une maison.\\
       He runs on a street then enters into a house \\
       {} Manner {} {}  Path  \\
\z

The authors find a third pattern which they call “reverse verb-framed pattern” because it is structurally identical with a verb-framed pattern but Manner and Path “switch places”, so that the Manner component is expressed in the main verb and the Path component in a gerund, see \REF{ex:3:9}.

\ea\label{ex:3:9}French \citep[145]{Pourcel_Kopecka_2005}\\
\glll Il court {en traversant} la rue.\\
      He runs crossing the street\\
      {} Manner {}  Path\\  
\z

The fourth type, in which Manner is expressed in the verb and Path in a PP, is also called reverse verb-framed pattern by the authors. This fourth type can also be described as a satellite-framed construction, see \REF{ex:3:10}.

\ea\label{ex:3:10}French \citep[145]{Pourcel_Kopecka_2005}\\
\glll    Il court dans le jardin.\\
         He runs into  the garden\\
         {} Manner {}  Path \\
\z

The fifth pattern is a hybrid type because the verbs here encode Path as well as Manner. There are two subtypes to this pattern: In the first, both elements are expressed in the verb, as in \REF{ex:3:11}; in the second, Path is expressed in an incorporated prefix of the verb, as in \REF{ex:3:12}. 

\ea\label{ex:3:11}French \citep[146]{Pourcel_Kopecka_2005}\\
\glll    Marc {a plongé dans} le lac.\\
         Marc {dived into} the lake\\
         {}    Manner.Path\\
\ex\label{ex:3:12}French \citep[149]{Pourcel_Kopecka_2005}\\
\glll   L'oiseau   s'est en-volé du nid.  \\
        {The bird} has away-flown from the nest  \\
        {}         {}  Path-Manner\\
\z

Besides the description of these five patterns, the authors show by using acceptability judgments that in French it is dispreferred to express the Manner component of a motion event as long as it is inferable from the context. Only when the Manner of motion is not typical for the Figure or Ground of the event, it is acceptable to express it \citep[148]{Pourcel_Kopecka_2005}. This finding is in line with the observation of \citet{Berthele_2013} that Manner is seldomly expressed in French motion event encodings.

\subsection{Motion events in Kwa languages}

Kwa languages form part of the Niger-Congo language family, members of which show a general tendency to lexicalize Path in verbs, such as \emph{enter}, \emph{pass}, or \emph{ascend} \citep[200--202]{Schaefer_Gaines_1997}. As for the expression of Manner, much variation is found between the members of this language family \citep[209]{Schaefer_Gaines_1997}.

A more detailed study on the expression of motion events has been carried out for two different Kwa languages, viz. Ewe \citep{Ameka_Essegbey_2013} and Fon \citep{LambertBrtire_2009}.

In Ewe, serial verb constructions combining a Path verb and a Manner verb can be used to express motion events, see \REF{ex:3:13}.

\ea\label{ex:3:13}
Ewe \citep[24]{Ameka_Essegbey_2013} \\
\gll    Devi-a tá yi xɔ-a me.\\
        child-\textsc{def} crawl go room-\textsc{def} in \\
\glt ‘The child crawled into the room.’
\z

It is possible to combine a Manner verb with more than one Path verb, each indicating movement in respect to a different ground object, see \REF{ex:3:14}.

\ea\label{ex:3:14}Ewe \citep[30--31]{Ameka_Essegbey_2013}\\
\gll  Kofi tá tó ve-a me do yi kpó-á dzí.\\
      Kofi crawl pass ditch-\textsc{def} in exit go hill-\textsc{def} top\\
\glt ‘Kofi crawled through the ditch and emerged  at the top of the hill.’
\z

In Fon, motion events can also be expressed using verb serialization, as in \REF{ex:3:15}.

\ea\label{ex:3:15}Fon \citep[14]{LambertBrtire_2009}\\
\gll xɛ̀ví ɔ̀ zɔ̀n gbɔ̀ tá nǔ é\\
     bird \textsc{def} fly pass head for \textsc{3SG} \\
\glt ‘The bird flew over his head.’
\z

As in Ewe, a Manner verb can be combined with more than one Path verb, as in \REF{ex:3:16}.

\ea Fon \citep[22]{LambertBrtire_2009}\label{ex:3:16}\\
\gll Cùkú ɔ́ lɔ̌n tɔ́n sín xɔ̀ mɛ̀ gbɔ̀n flɛ́tɛ́ ɔ́ nù.\\
     dog \textsc{def} jump exit from room in pass window \textsc{def} edge \\
\glt ‘The dog jumped out of the room through the   window.’
\z


Available for motion event verb serialization is a closed class of ten Path verbs \citep[9]{LambertBrtire_2009}. All of these can also be used outside of verb serializations, but not all Path verbs are available for serialization, like e.g. \emph{xá} ‘go.up’ \citep[16]{LambertBrtire_2009}. Similarly, not all Manner verbs are available for serialization, see \REF{ex:3:17}.

\ea\label{ex:3:17}Fon \citep[15]{LambertBrtire_2009}\\
\gll * yě dǔ-wè tɔ́n sìn xwé ɔ́ mɛ̀  \\
     {} \textsc{3pl} move-dance exit from house \textsc{def} in \\
\glt \phantom{*}‘They danced out of the house.’
\z

Whereas \citet[36]{Ameka_Essegbey_2013} classify Manner-Path verb serializations in Ewe as equipollently-framed constructions, \citet[19]{LambertBrtire_2009} argues that the Fon Manner-Path verb serializations are satellite-framed constructions with the Path verbs acting as satellites. She reaches that conclusion because certain inflectional elements can only appear in front of the Manner verb, which marks them, in her point of view, as the main verb of the sentence.

In fact, the question how Manner-Path verb serializations should be classified in the typology described above is controversial. It depends mainly on the question whether the verbs are co- or subordinated. The discussion of this problem is outside of the scope of the present study. More details on the topic can be found in \citet{Talmy_2009}.

\subsection{Motion events in Haitian Creole}
 
To my knowledge, no study has aimed at investigating the expression of motion events in Haitian Creole\footnote{In the following, all examples are from Haitian Creole, so this will not be indicated in the rest of the paper.} until now. Nonetheless, some insights can be obtained from the literature on Haitian Creole. The language possesses an inventory of Path verbs, many of French origin, like in the example in \REF{ex:3:18}.

\ea\label{ex:3:18} \citep[203]{Fattier_2013}\\
\gll Dlo antre anndan kay. \\
     water enter \textsc{loc} house \\
\glt ‘Water came into the house.’
\z

Besides that, of all French-based Creole languages, Haitian Creole is the one that exhibits the most serial verb constructions \citep[44]{Mutz_2017}. Many of those constructions found in the literature do not express motion events, but a few examples of Manner-Path verb serializations can be found, such as the one in \REF{ex:3:19}.

\ea\label{ex:3:19} \citep[244]{Valdman_2015}\\
\gll  Tidjo kouri ale lakay li. \\
      Tidjo run go home \textsc{poss.pron} \\
\glt ‘Tidjo ran over to his house.’
\z

The dissertation on Haitian Creole verb serialization by \citet{BucheliBerger_2009} does not offer examples of Manner-Path verb serialization, but lists the possible combinations of Manner and Path verbs, a shortened version of which is reproduced here in \tabref{tab:tab1_03} (on the following page).\footnote{Her results are derived from online research. Marked as possible are those combinations for which she could find examples online. If a combination is not marked as possible, this does not necessarily mean that it is impossible but simply that the author could not find an example for it during her research. No acceptability study was carried out. An anonymous reviewer of this paper notes that some combinations, especially the ones with \emph{tonbe}, sound strange to them.}

% Table 1:
\begin{table}
\resizebox{\linewidth}{!}{%
\begin{tabular}{lllllllll}
    \lsptoprule
    & \textit{al(e)} & \textit{vin(i)} & \textit{sòt(i)/} & \textit{antre} & \textit{rive} & \textit{monte/} & \textit{desann} & \textit{(re-) tounen} \\
    & ‘go’ & ‘come’ & \textit{sot(i)} & ‘go in’ & ‘arrive’ & \textit{moute} & ‘go down’ & ‘come back’ \\
     &  &  & ‘go out’ &  &  & ‘go up’ &  &  \\ \midrule
    \textit{kouri ‘run’} & + & + & + & + & + & + & + & + \\
    \textit{mache}  ‘march’ & {+} & {+} & {+} & {+} & {+} & {} & {} & {} \\
    \textit{naje}  ‘swim’& {+} & {} & {+} & {} & {} & {} & {} & {} \\
    \textit{woule} ‘roll’ & + & + & + &  &  &  & + &  \\
    \textit{koule} ‘flow’ &  &  & + &  &  &  & + &  \\
    \textit{vole} ‘fly’ & + & + & + &  &  &  & + &  \\
    \textit{glise} ‘glide’ &  &  & + &  &  &  & + &  \\
    \textit{tonbè} ‘fall’ &  &  & + &  & + &  &  &  \\ 
    \lspbottomrule
\end{tabular}}
\caption{Manner of Motion V1 + Path of Motion V2 in Haitian Creole after \citet[202]{BucheliBerger_2009}}
\label{tab:tab1_03}
\end{table}


\section{Study design}\label{sec:3:3}

The present study investigates the expression of motion events as presented above in Haitian Creole. The main purpose is to describe the morphosyntactic elements used to express the components Manner and Path and the preferences of their use. For this purpose, four Haitian Creole speakers living in Berlin, Germany, took part in interviews that consisted of three parts: an interview on their habits of language use and attitudes towards all their languages, a narration of a picture story, and descriptions of single pictures representing different motion events that were drawn by the author of this study. The entire interviews were held in Creole. One of the participants, P1, helped realize the other three interviews as well as transcribe and translate the language data recorded. She will henceforth be referred to as the main participant. More information on participants and tasks is given in the following sections.


\subsection{Participants}

The four participants were aged between 34 and 56. P1 is female; P2, P3 and P4 are male. P1 is a B.A. student, P2 is a mechanical engineer, P3 is a salesperson and photographer and P4 is a political scientist and educator in development cooperation. They were all born in Haiti and completed most of their education there. P1, P2 and P3 come from the area of Port-au-Prince, P4 moved there from the North of the country when he was ten years old. All four emigrated between 20 and 30 years of age. P2 and P4 regularly work in Haiti. The four participants all speak Haitian Creole, French, German, Spanish, and English. They learned Haitian Creole as their first language from their parents and later learned French in school. They received education almost entirely in French; only P1 had Creole language classes for one year. All four report they are able to converse fluently in Creole but have problems with writing, as they have never learned a norm. As the four participants all live in Germany, they speak German on a daily basis.\footnote{Of course, the fact that the four participants all live in a non-Creole-speaking country and use other languages on a daily basis could influence the Creole they speak causing it to be different from the Creole spoken in Haiti. Because the present study was carried out as an MA thesis, getting fieldwork data was not possible. Possible contact phenomena will not be investigated in the present study, but this has to be kept in mind when interpreting the results.} P1, P3 and P4 report that they speak French often, mostly with their family, especially with their children. P4 also speaks French (as well as Creole) at work. Creole is spoken with friends and family in Haiti and abroad, e.g. with their parents and siblings. P2 is the only participant that reports that he speaks Creole often, mainly with his children but also the rest of the family, as well as when working in Haiti. He is also the only one to name Creole as the language he finds most elegant; for the three others that language is French. When asked what the Creole language means to them, all four replied that it is an important part of their identity and their origin. P3 and P4 also say that they feel that Creole is the most important one of their languages.

\subsection{Tasks}

There were two tasks aiming at eliciting motion events, the narration of a picture story and the description of single pictures drawn for the purpose of this study. The picture story selected was \emph{Die Geschichte vom Vogel} (‘The bird story’) \citep[from][]{RettichRettich1972}.\footnote{Unfortunately, the image of the picture story cannot be reproduced here due to copyright reasons.} Even though the Frog Stories have been used to elicit motion events in many previous studies, they were not used here, first because they were considered difficult to narrate by the author of this study and her supervisor, and second because many of the Frog Story pictures do not contain motion events. The bird story is about a bird that flies out of its cage and then out of its house. Outside, he meets different animals that all chase him away. Finally, he flies back to his house and into his cage. The story was chosen because it contains many different motion events which could help determine how frequent the Manner component would be expressed in order to investigate the Manner salience of Haitian Creole.

The description of single pictures aimed at exploring the morphosyntactic elements that could be used to express different Manner-Path combinations. Therefore, seven Manner elements (\emph{run, swim, fly, jump, crawl, dance, roll}) and ten Path elements (\emph{out, away, to, into, up, down, along, past, after somebody, through}) were used to create a total of 48 motion events, as in \REF{ex:3:20}, which were then portrayed in simple pictures by the author of this study.

\ea\label{ex:3:20}
\gll    He runs out {of the burning house.}\\
        {} Manner Path\\
\z

% Figure 1:
\begin{figure}
    \includegraphics[width=0.6\linewidth]{figures/fig1_03.png}
    \caption{Depiction of the motion event ‘to run out of’}
    \label{fig:fig1_03}
\end{figure}

The combination of the 17 motion event components would have yielded more than 48 events, but it was decided not to overwhelm the participants with too many pictures. The 48 drawings were divided into two groups of 24, which were presented to two participants each. When dividing the pictures, the different Manner and Path components were divided as equally as possible between the two groups. Within the two groups, the pictures were arranged in such a way that two following pictures never contained a component already depicted in the previous picture. During the interview, the participants were told to describe what the person in the picture was doing.

\subsection{Data analysis}

The interviews were transcribed and translated into German by the main participant. Transcriptions and translations were later checked by the author of this paper and revised together.

At the beginning of the data analysis, the number of sentences was counted for the picture story narrations. Every unit containing a subject and (at least) one verb was counted as one sentence. Coordinated clauses were counted as two sentences, but subordinated clauses like complement clauses, relative clauses, causal clauses, temporal clauses and the like were counted as part of the matrix clause.

P1’s narration contains 26 sentences, P2’s contains 46 sentences, P3’s contains 55, and P4’s narration contains 42 sentences. In total, 169 sentences were elicited.

After counting the number of sentences, the number of motion events encoded was determined. Every sentence expressing directional motion was analyzed as a motion event encoding.

P1’s narration contains 18, P2’s 20, P3’s 29, and P4’s 27 motion event encodings. Altogether, the four participants encoded 94 motion events in their picture story narrations. With a total of 169 sentences, more than 50\% of the sentences contained a motion event encoding.

The motion events were then sorted by means of the morphosyntactic structure they used to encode different motion event components. They were sorted into six different categories: Path verbs only, see \REF{ex:3:21}, Path verbs with Ground-PP/NP, see \REF{ex:3:22}, Manner verbs only, see \REF{ex:3:23}, Manner verbs with path elements, see \REF{ex:3:24}, serial verb constructions, see \REF{ex:3:25}, and motion events without a motion verb, see \REF{ex:3:26}. The remaining cases were classified as “Other”.

\ea\label{ex:3:21}
\gll Li rantre.\\
     \textsc{3sg} enter.again \\
\glt ‘He goes back in.’
\ex \label{ex:3:22}
\gll Epi l antre nan kay la. \\
and \textsc{3sg} enter \textsc{loc} house \textsc{def} \\
\glt ‘He enters the house.’
\ex E zwazo a kouri. \\\label{ex:3:23}
    and bird \textsc{def} \textsc{run} \\
\glt ‘And the bird runs/flies fast.’
\ex \label{ex:3:24}
\gll {{\ob}…{\cb}} zwazo a vole sou do yon erison.\\
     {} bird \textsc{def} fly \textsc{loc} back \textsc{indef} hedgehog \\
\glt ‘The bird flies onto the back of a hedgehog.’
\ex \label{ex:3:25}
   \gll Li kouri retounen nan kay {[}kote li te ye a.{]}\\
    \textsc{3sg} run return \textsc{loc} house {[}\textsc{rel.pron} \textsc{3sg} \textsc{pst} \textsc{cop} \textsc{def}{]} \\
\glt `He goes back into the house where he was before.'
\ex \label{ex:3:26}
\gll Epi li kraze rak.\\
     then \textsc{3sg} destroy forest \\
\glt ‘Then he beats loose.’
\z

The results of the analysis will be given in the following section.

The first step of the picture description analysis was to determine the number of descriptions. As 48 pictures were described by two participants each, 96 descriptions should have been elicited, but as one of the participants failed to describe two of the pictures, only 94 descriptions were elicited. Some of the descriptions consist of a simple sentence, whereas others consist of a complex sentence or even more than one sentence. A total of 119 sentences were elicited in both tasks.

If more than one sentence was used for the description of a picture, they were counted and analyzed separately. The same holds for complex sentences if they contained more than one motion event, e.g. a sequence of two relative clauses, as in \REF{ex:3:27}, or sentences with \emph{pou} `in order to', as in  \REF{ex:3:28}.

\ea\label{ex:3:27}
\gll Yon zwazo k ap vole k ap pase bò kot yon pyebwa.\\
\textsc{indef} bird \textsc{rel.pron} \textsc{prog} fly \textsc{rel.pron} \textsc{prog} \textsc{pass} beside side \textsc{indef} tree\\
\glt ‘A bird which is   flying, who is passing next to a tree.’\footnote{Mostly P2, but also P4, described several of the pictures with utterances of the form NP + relative clause. Even though these do not constitute regular sentences, it is possible to analyze them as elliptic versions of sentences like \emph{This is [NP] who is moving} which are also found in some descriptions. They were therefore included in the analysis.}
\ex\label{ex:3:28}
\gll Yon mesye k ap naje sòti nan plaj pou l ale bò rivaj. \\
\textsc{indef} man \textsc{rel.pron} \textsc{prog} swim exit \textsc{loc} beach for \textsc{3sg} go beside coast\\
\glt ‘A man who is swimming away from the beach in   order to swim to the coast.’
\z

Sentences which did not express motion (13 of 119) were not analyzed.

In a few cases, modal verbs were used, see \REF{ex:3:29} and \REF{ex:3:30}. These were ignored for the analysis and the event encodings of the motions were treated as if they did not contain a modal verb.

\ea\label{ex:3:29}
\gll Yon gason ki vle monte sou yon tab.\\
      \textsc{indef} boy \textsc{rel.pron} want ascend \textsc{loc} \textsc{indef} table \\
\glt ‘A boy who wants to go up onto a table.’
\ex\label{ex:3:30} 
\gll Yon mesye ki dwe travèse dyagonal yon chanm. \\
     \textsc{indef} man \textsc{rel.pron} must cross diagonal \textsc{indef} room \\
\glt `A man who has to cross a room diagonally.'
\z

The motion events expressed in the picture descriptions were then sorted into eight categories, seven of which are equivalent to those for the picture story. A new category was established for this part of the data: Manner verbs with Ground elements, as exemplified in \REF{ex:3:31}.

\ea \label{ex:3:31}
\gll  Yon moun k ap rale kote yon mi.\\
     \textsc{indef} person \textsc{rel.pron} \textsc{prog} crawl beside \textsc{indef} wall \\
\glt `A person who is crawling next to a wall.'
\z

The results of the analysis are given in the following section.


\section{Results}\label{sec:3:4}

An overview of the results is given in \tabref{tab:tab2_03}. In the following subsections, the results are discussed in detail.

% Table 2:
\begin{table}
\begin{tabular}{l rr rr rr}
\lsptoprule
                 & \multicolumn{2}{c}{Picture} & \multicolumn{2}{c}{Single} & \\
                 & \multicolumn{2}{c}{story}   & \multicolumn{2}{c}{pictures} &\multicolumn{2}{c}{Total} \\\midrule
{Path verb only} &  10     & 10.6\% & 4      & 3.4\%  & 14     & 6.6\% \\
{Path verb + ground PP/NP} &  25       & 26.6\%   & 29       & 24.4\%   & 54       & 25.4\%  \\
{Manner verb only} & 18      & 19.1\%  & 19      & 16.0\%    & 37      & 17.4\% \\
{Manner verb + ground} &  0      & 0.0\%      & 13     & 10.9\% & 13     & 6.1\% \\
{manner verb + path element} &    2     &  2.1\% &  5     &  4.2\% &  7     &  3.2\%\\
{SVC} &  16     & 17.0\%   & 36     & 30.3\% & 52     & 24.4\%\\
{Motion event without motion verb} &   7      &  7.4\%  &  1      &  0.8\%  &  8      &  3.8\% \\
{Other} &   16      & 17.0\%    & 12      & 10.1\%  & 28      & 13.1\% \\
\midrule
{Total} &  94 & & 119 & & 213 & \\
\lspbottomrule
\end{tabular}
\caption{Motion event expressions in picture story narrations and single picture descriptions}
\label{tab:tab2_03}
\end{table}

Path verbs only were used ten times in the picture story narrations (10.6\% of all occurrences). In most cases, the Ground was mentioned in the preceding or following context but not in the same clause, see \REF{ex:3:32} and \REF{ex:3:33}.

\ea \label{ex:3:32}
\gll E li ouvri pòt kalòj la pou li kapab sòti.\\
     and \textsc{3sg} open door cage \textsc{def} for \textsc{3sg} able.to exit\\
\glt ‘And she opens the door of the cage so that he can go out.’
\ex \label{ex:3:33}
\gll Epi zwazo a tounen. L al nan menm kay la {[}...{]}\\
     and bird \textsc{def} return \textsc{3sg} go \textsc{loc} same house \textsc{def}\\
\glt ‘And he returns. He goes into the same house.’
\z

In one case, no ground is mentioned at all, see \REF{ex:3:34}.

\ea\label{ex:3:34}
\gll Papiyon an ale.\\
     butterfly \textsc{def} go\\
\glt ‘The butterfly goes/flies away.’
\z

At this point, the picture story shows a butterfly flying away from the bird. Therefore, \textit{ale} seems to express not simply `go' but `go away' here.

In the single picture descriptions, Path verbs only occurred in four motion events (3.4\%). As in the narrations, the Ground was usually mentioned in the context, see \REF{ex:3:35}.

\ea\label{ex:3:35}
\gll    Yon moun k ap rale kote yon mi.  L ap pase {[}…{]} \\
        \textsc{indef} person beside \textsc{rel.pron} \textsc{prog} crawl beside \textsc{indef} wall  \textsc{3sg} \textsc{prog} pass \\
\glt ‘A person is crawling next to a wall. He is passing {[}it{]}…’
\z

Again, there was one case where no Ground was mentioned at all, again with the verb \textit{ale}, which seems to mean `go away' \REF{ex:3:36}.

\ea\label{ex:3:36}
\gll Yon zwazo ki sòti nan kalòj pou ale.\\
     \textsc{indef} bird \textsc{rel.pron} exit \textsc{loc} cage for go \\
\glt ‘A bird who leaves the cage in order to go/fly away.’
\z

\subsection{Path verb + Ground NP/PP}

The most frequent pattern used to express motion events in the picture story is a Path verb with a Ground NP or PP, which was used in 25 cases (26.6\%). The verbs \emph{antre} ‘enter’, \emph{pase} ‘pass’, \emph{atèri} ‘land’, \emph{tonbe} ‘fall’, \emph{ale} ‘go’, \emph{poze} ‘sit down’, and \emph{sòti} ‘go out’ were used with PPs \REF{ex:3:37}.

\ea
\label{ex:3:37}
\gll Li atèri sou flè a.\\
     \textsc{3sg} land \textsc{loc} flower \textsc{def}\\
\glt ‘He lands on the flower.’
\z

The verbs \emph{jwenn} ‘reach’, \emph{suiv} ‘follow’, \emph{kite} ‘leave’, \emph{tounen} ‘come back’ were used with object NPs \REF{ex:3:38}.

\ea\label{ex:3:38}
\gll  Li kite do erison an.\\
      \textsc{3sg} leave back hedgehog \textsc{def}\\
\glt ‘He leaves the back of the hedgehog.’
\z

The verb \emph{rive} ‘arrive’ was used with a PP three times (by P2 and P3) and with an object NP once (by P1), see \REF{ex:3:39} and \REF{ex:3:40}. Because of the small number of occurrences, nothing can be said about whether this is simply due to individual preferences.

\ea\label{ex:3:39}
\gll  Lè l rive sou pyebwa {[}…{]} \\
      when \textsc{3sg} arrive on tree  \\
\glt ‘When he arrives on the tree…’
\ex\label{ex:3:40}
\gll    Zwazo a rive lakay li.\\
        bird \textsc{def} arrive home \textsc{poss.pron}\\
\glt ‘The bird arrives at his house.’
\z

In the single picture descriptions, Path verbs with Ground NPs or PPs present the second most frequent pattern with 29 occurrences (24.4\%).

Used with an object NP were the verbs \emph{depase, desann}, and \emph{kite} \REF{ex:3:41}.

\ea\label{ex:3:41}
\gll  Tidjo kite lekòl la. \\
      Tidjo leave school \textsc{def}\\
\glt ‘Tidjo leaves the school.’
\z

The verbs \emph{antre, rantre, pase} and \emph{al(e)} were used with PPs. \emph{Antre} and \emph{rantre} were used with  \emph{nan} ‘into’, \emph{al(e)} with \emph{bò} ‘next to’ and \emph{nan direksyon} ‘in the direction of’, and \emph{pase} also with \emph{bò}, see \REF{ex:3:42}.

\ea\label{ex:3:42}
\gll  L ap pase bò yon kay. \\
      \textsc{3sg} \textsc{prog} pass beside \textsc{indef} house \\
\glt ‘He is passing a house.’
\z

The verbs \emph{monte} and \emph{sòti} were used with both NPs and PPs. \emph{Sòti} was used with three different prepositions, \emph{nan, sou} and \emph{a travè}. See \REF{ex:3:43} for an example with \emph{a travè}, and \REF{ex:3:44} for the use with an NP.

\ea\label{ex:3:43}
\gll Sa se   yon   moun   ki       sòti {a travè} yon fenèt {[}…{]}\\
     \textsc{dem} \textsc{cop} \textsc{indef} person \textsc{rel.pron} exit through \textsc{indef} window\\
\glt ‘That is a person who leaves through a window.’ 
\ex\label{ex:3:44}
\gll    Yon timoun ki sòti lekòl. \\
        \textsc{indef} child \textsc{rel.pron} exit school \\
\glt ‘A child that leaves school.’
\z

As there are only a few occurrences for each verb, often just one but six at the most, it remains unclear whether the use with NP or PP attested here is a general preference of the verb or whether all verbs can appear with both.

\subsection{Manner verb only}

A Manner verb alone cannot, strictly speaking, express a motion event as it is defined above, but because they occur so frequently in both picture story narrations and single picture descriptions, they are taken into account here.

In the picture story task, in the 18 cases counted for this category (19.1\%), only three different Manner verbs were used, \emph{vole} ‘fly’, \emph{kouri} ‘run‘, and \emph{mache} ‘walk’. The last of the three is used only once where a hedgehog continues walking after the bird has landed on his back. The most frequently used of these verbs is \emph{vole}. This is not surprising when taking into account that the story is about a bird and also features other flying animals like owls or butterflies. \emph{Kouri} was used six times to describe the motion of the bird, see \REF{ex:3:45}.

\ea\label{ex:3:45}
\gll Lè chwèt la kouri dèyè zwazo a, sa k pase, zwazo a kouri.\\
     when owl \textsc{def} run behind bird \textsc{def} \textsc{def} \textsc{rel.pron} happen bird \textsc{def} run \\        
\glt ‘When the owl flies behind/after the bird, the bird runs/flies away fast.’
\z

As the story shows several instances of an animal chasing another animal (mostly the little bird) away, the cases where \emph{vole} and \emph{kouri} are used alone always describe a situation where the animal flees. Apparently, in these cases, a directed motion away from the place of action seems to be described. The Path ‘away’ seems to be inferable from the context and is therefore left out. \emph{Kouri} obviously does not express the Manner ‘run’ in these cases, but rather an accelerated manner of movement.

In the single picture descriptions, Manner verbs were used alone in 19 cases (16\% of all occurrences). In seven of these, no further information on the motion event was given, see \REF{ex:3:46}.

% (46)	Yon 	mesye  ki 	     ap 	     naje.
% 		INDEF	man	REL.PRON PROG  swim
% 		‘A man that is swimming.’
\ea\label{ex:3:46}
\gll  Yon mesye ki ap naje. \\
      \textsc{indef} man \textsc{rel.pron} \textsc{prog} swim \\
\glt ‘A man that is swimming.’
\z

These cases do not express a motion event as it is understood here, but an activity.

In the remaining twelve cases, more information on the motion event is given in the preceding or the following context. In five cases, the manner verb is followed by a construction with \emph{pou} ‘to’ in which Path is expressed, see \REF{ex:3:47}.

\ea\label{ex:3:47}
\gll  Tidjo ap naje pou l depase lòt la.\\
      Tidjo \textsc{prog} swim for \textsc{3sg} pass other \textsc{def} \\
\glt ‘Tidjo is swimming in order   to pass the other.'
\z

In three cases, information on the Path is given in the preceding or following sentence, see \REF{ex:3:48} and \REF{ex:3:49}. As the translations show, it is possible to interpret these cases as single complex motion events.

\ea\label{ex:3:48}
\gll Tidjo ap naje. Li kite il la.  \\
     Tidjo \textsc{prog} swim \textsc{3sg} leave island \textsc{def}  \\
\glt ‘Tidjo is swimming. He leaves the island./Tidjo is swimming away from the island.’

\ex\label{ex:3:49}
\gll Tidjo antre nan kay. {[}…{]} L ap danse.\\
     Tidjo enter \textsc{loc} house {} \textsc{3sg} \textsc{prog} dance  \\
\glt ‘Tidjo enters the house. He is dancing./Tidjo is dancing into the house.’
\z

In two cases, Path and Manner are expressed in two precedent relative clauses, see \REF{ex:3:50}. Again, it is possible to interpret this as a single complex motion event.

\ea\label{ex:3:50} 
\gll Yon zwazo k ap vole k ap pase bò yon mi. \\
     \textsc{indef} bird \textsc{rel.pron} \textsc{prog} fly \textsc{rel.pron} \textsc{prog} pass beside \textsc{indef} wall  \\
\glt ‘A bird who is flying, who is passing beside a wall.\slash A bird who is flying past a wall.'
\z

In one case, two main clauses, one containing a Manner and the other a Path verb, are combined with the conjunction \emph{pandan} ‘while’, see \REF{ex:3:51}.

\ea\label{ex:3:51}
\gll L ap danse pandan l ap monte mach eskalye a.\\
     \textsc{3sg} \textsc{prog} dance while \textsc{3sg} \textsc{prog} ascend  step stairs \textsc{def}\\
\glt ‘He dances while he goes up the stairs./He dances up the stairs.’
\z

All these examples seem to represent complex motion events whose components were not expressed in a single clause, meaning that they were not conflated into one event.

\subsection{Manner verb + Ground element}

Similar to the previous category, Manner verbs combined with a Ground element in the same clause do not constitute motion events as defined by \citet{Talmy_1985}, because the Path element obligatory for motion events is not encoded. Such a pattern does not occur in the picture story narrations, but there are 13 such occurrences in the single picture descriptions (10.9\%). Six of the seven manner verbs tested in this task were used in this pattern, see \REF{ex:3:52} for an example with \emph{naje} ‘swim’.

\ea\label{ex:3:52}
\gll    Yon moun k ap naje nan lanmè.\\
        \textsc{indef} person \textsc{rel.pron} \textsc{prog} swim \textsc{loc} sea\\
\glt ‘A person who is swimming in the sea.’
\z

\subsection{Manner verb + Path element}

The least frequent of the patterns is the use of a Manner verb in combination with a Path element. Such cases constitute instances of the satellite pattern described above.

In the picture story descriptions, two such cases (2.1\%) occurred with the verb \emph{vole}, once with the preposition \emph{sou} ‘onto’, see \REF{ex:3:53}, and once with \emph{deyò} ‘out of’, see \REF{ex:3:54}.

\ea\label{ex:3:53}
\gll {[}…{]} zwazo a vole sou do yon erison. \\
      {}  bird \textsc{def} fly \textsc{loc} back \textsc{indef} hedgehog \\
\glt ‘A bird flies onto the back of a hedgehog.’
\ex\label{ex:3:54}
\gll    Li vole deyò. \\
        \textsc{3sg} fly outside  \\
\glt ‘He flies outside.’
\z

In the single picture descriptions, this satellite-framed pattern is used in five cases (4.2\%), three of them with the verb \emph{vole}, see \REF{ex:3:55}, one with \emph{naje}, see \REF{ex:3:56}, and one with \emph{woule} ‘roll’.

\ea\label{ex:3:55}
\gll Yon zwazo k ap vole a travè nyaj yo. \\
     \textsc{indef} bird \textsc{rel.pron} \textsc{prog} fly through cloud \textsc{pl} \\
\glt ‘A bird that is flying through the clouds.’
\ex\label{ex:3:56}
\gll    Tidjo ap naje {[}…{]} sou lòt bò lak la. \\
        Tidjo \textsc{prog} swim  {} \textsc{loc} other side lake  \textsc{def}\\
\glt ‘Tidjo is swimming to the other side of the lake.’
\z

\subsection{Serial verb constructions}

Serial verb constructions are used frequently in both the picture story narrations and the single picture descriptions. As for the story narrations, they present the second most frequent pattern with 16 occurrences (17\%). Most of these consist of a serialization of two Path verbs: \emph{sòti kite} ‘leave go away’, \emph{al(e) poze} `go sit down', \emph{vin poze} `come sit down', \emph{al tonbe} ‘go fall’ and \emph{al antre} ‘go enter’. In two cases, a Manner verb is followed by a Path verb: \emph{vole poze} ‘fly sit down’ and \emph{kouri retounen} ‘run return’. One single verb serialization consists of three verbs: \emph{leve pran kouri} ‘get up take run’. \emph{Pran} most probably acts as an aspectual marker here, encoding inchoativity (see also \cite{Valdman_2015}: 231). Besides the fact that this serial verb construction consists of three verbs, it is also different from the others because the order of the Path and Manner verb is inverted here in regard to the other cases: the Path verb is the first, the Manner the last verb of the serialization.

\hspace*{-2.8pt}The different serial verb constructions occur either with a Ground NP, a Ground PP, or with no Ground element in the same clause. In the last case, Ground is mentioned in the surrounding context.

In four serial verb constructions, \emph{vole poze, vin poze, al(e) poze} and \emph{al tonbe}, the same kind of action is expressed: the bird flying to a certain place and coming to rest. It is not clear whether these are sequential or simultaneous verb serializations, the first meaning ‘he flies and then sits down’ and the latter meaning ‘he flies onto [the tree]’. The same problem exists with \emph{al antre} 'go enter', where it is unclear whether it is said that the bird first goes and then enters or whether he ‘goes into’. 

The case of \emph{sòti kite} is described by \citet[79--81]{BucheliBerger_2009} as a serialization of two verbs that are close to synonyms and which she interprets as a simultaneous serial verb construction expressing one simple motion. It is also possible that the verbs have different meanings here, expressing the Paths ‘out of’ and ‘away’, which would make this a sequential serial verb construction.

In \emph{kouri retounen}, which once again describes the motion of the verb, \emph{kouri} cannot be interpreted as expressing the Manner ‘to run’, but rather a fast way of moving.

\emph{Leve pran kouri} is probably a sequential serial verb construction, expressing that the bird first gets up and then flies away fast, where the Path `away' is left unexpressed and to be inferred from the context.

In the single picture descriptions, serial verb constructions were the pattern most frequently used by the participants to encode a motion event. With 36 occurrences in total (30.3\% of all occurrences), 31 different verb combinations can be described. The different internal structures of these serial verb constructions are summarized in \tabref{tab:tab3_03}.

% Table 3:
\begin{table}[!ht]
\centering
\resizebox{\linewidth}{!}{%
\begin{tabular}{lccccccc}
    \lsptoprule
    \multirow{2}{*}{Type of SVC} & \multirow{2}{*}{manner-path} & \multirow{2}{*}{manner-manner} & \multirow{2}{*}{path-path} & \multirow{2}{*}{path-Other} & \multicolumn{3}{c}{3 verbs} \\ \cmidrule{6-8}
    &  &  &  &  & MMP & PPP & PMP \\ \midrule
    Frequency & 19 & 2 & 1 & 6 & 1 & 1 & 1 \\ \lspbottomrule
\end{tabular}%
}
\caption{Frequency of different serial verb constructions in the single picture descriptions}
\label{tab:tab3_03}
\end{table}

By far the most common verb serializations are those consisting of a Manner and a Path verb. Six different Manner verbs were used in such constructions: \emph{kouri} ‘run’, \emph{naje} ‘swim’, \emph{vole} ‘fly’, \emph{rale} ‘crawl’, \emph{woule} ‘roll’, and \emph{glise} ‘glide/slide’ (used once instead of `roll'). Two of the Manner verbs investigated were not used in Manner-Path serializations, \emph{sote} ‘jump’ and \emph{danse} ‘dance’. Table \ref{tab:tab4_03} shows the different Path verbs that these Manner verbs were used with. Nothing can be said about the possibility to form serial verb constructions other than the ones attested in the data of this study.\footnote{An anonymous reviewer of this paper notes that more combinations than the ones attested in this study should be possible, especially with \emph{rale}. Also, all manner verbs should be able to combine with \emph{ale}.}

% Table 4:
\begin{table}[!ht]
\centering
\resizebox{\linewidth}{!}{%
\begin{tabular}{@{}lllllllllll}
    \lsptoprule
    & \textit{sòti} & \textit{kite} & \textit{antre} & \textit{monte} & \textit{desann} & \textit{pase} & \textit{travèse} & \textit{ale} & \textit{poze} & \textit{suiv} \\ \cmidrule{2-11}
    \textit{kouri} & + & + & + & + & + & + & + &  &  &  \\
    \textit{naje} & + &  &  &  &  &  &  &  &  &  \\
    \textit{vole} & + &  & + &  &  &  &  & + & + & + \\
    \textit{rale} & + &  &  & + &  &  &  &  &  &  \\
    \textit{woule} &  &  &  &  & + & + &  &  &  &  \\
    \textit{glise} & + &  &  &  & + &  &  &  &  & \\ \lspbottomrule
\end{tabular}%
}
\caption{MANNER-PATH-SVC in the single picture descriptions}
\label{tab:tab4_03}
\end{table}

These Manner-Path serial verbs also occur both with NPs and PPs, see \REF{ex:3:57} and \REF{ex:3:58}.

\ea\label{ex:3:57}
\gll Tipyè rale monte mòn nan. \\
     Tipyè crawl ascend mountain \textsc{def} \\
\glt ‘Tipyè crawls up the mountain.’
\ex\label{ex:3:58}
\gll Zwazo  a vole antre nan kizin nan. \\
     bird \textsc{def} fly enter \textsc{loc} kitchen \textsc{def} \\
\glt ‘The bird flies into the kitchen.’
\z

In one case, no ground element is given in the same clause, see \REF{ex:3:59}. As the Path verb is \emph{ale} in this case, the Path can probably again be interpreted as ‘away’ in this case.

\ea\label{ex:3:59}
\gll  Yon zwazo {[}…{]} k ap vole ale. \\
      \textsc{indef} bird   {} \textsc{rel.pron} \textsc{prog} fly go \\
\glt ‘A bird that is flying away.’
\z

In two cases, two Manner verbs were serialized, see \REF{ex:3:60} and \REF{ex:3:61}. In \REF{ex:3:60}, it is obvious that again \emph{kouri} cannot be interpreted as 'run', but most probably means that the motion takes place fast. In \REF{ex:3:61}, it is unclear what the exact meaning of \emph{vole} is. It could possibly be interpreted as having a figurative meaning expressing something like jumping into the water in a high arc. This hypothesis cannot be tested here.\footnote{An anonymous reviewer suggests that the two verbs act like synonyms here.}

\ea\label{ex:3:60}
\gll Tipyè kouri naje dèyè yon lòt.\\
     Tipyè run swim behind \textsc{indef} other \\
\glt ‘Tipye swims fast behind/after another.’
\ex\label{ex:3:61}
\gll Li vole sote nan dlo a.\\
     \textsc{3sg} fly jump \textsc{loc} water \textsc{def} \\
\glt ‘He jumps into the water (in a high arc).’
\z

The only Path-Path serialization in the single picture descriptions is \emph{sòti kite}, which also occurred in the picture story narrations.

In six cases, the participants used serial verb constructions consisting of a Path and a non-motion verb, like in \REF{ex:3:62}.

\ea\label{ex:3:62}
\gll   Yon moun ki ap antre kache nan yon gwòt. \\
      \textsc{indef} person \textsc{rel.pron} \textsc{prog} enter hide \textsc{indef} \textsc{loc} cave \\
\glt ‘A person who is going into a cave in order to hide.’
\z

In three of these six cases, the verb \emph{ale} was used, but cannot be interpreted as a Path verb, see \REF{ex:3:63}. It could be interpreted as an analytical future, but such a structure with this function has not been described for Haitian Creole (see for example, \cite{Valdman_2015,DeGraff_2007}). This structure cannot be further analyzed at this point.

\ea\label{ex:3:63}
    \gll {[Sa se Johana k ap kouri]} pou l al pran bis la  nan stasyon an.\\
          {}                        for \textsc{3sg} go take bus \textsc{def} \textsc{loc} station \textsc{def}\\
\glt ‘{[}That’s Johanna who is running{]} in order to take a bus at the station.’
\z

In three cases, three motion verbs are serialized. The first of them is a Manner-Manner-Path serialization, see \REF{ex:3:64}. As this sentence is about a swimming person, \emph{kouri} probably once again express a fast motion.

\ea\label{ex:3:64}
\gll  Li kouri naje kite il la. \\
      \textsc{3sg} run swim leave island \textsc{def} \\
\glt ‘He swims away from the island fast.’
\z

The second of these serializations consists of three Path verbs \REF{ex:3:65}. \emph{Rive} and \emph{jwenn} are close to synonyms and are therefore interpreted as expressing the same meaning here. Together with \emph{avanse} ‘move forward’ they probably present a sequential serial verb construction.

\ea\label{ex:3:65}
\gll  Li avanse rive jwenn demwazèl la. \\
      \textsc{3sg} advance arrive reach lady \textsc{def} \\
\glt ‘He advances towards and reaches the lady.’
\z

In the last of the three cases, we find a Path-Manner-Path verb serialization, see \REF{ex:3:66}. Interestingly, V1 and V3 are the same verb, \emph{sòti}. As this is the only case where we find such a structure in the data, nothing can be said about whether this is a common pattern of serial verb constructions in Haitian Creole.

\ea\label{ex:3:66}
\gll   Boul la sòti woule sòti nan bwàt katon.\\
       ball \textsc{def} exit roll exit \textsc{loc} box cardboard\\
\glt ‘The ball rolls out of the cardboard box.’
\z

Even though many of the serial verb constructions elicited in this study are single cases that need to be described separately, some patterns can also be found. One frequent strategy to express complex motion events in Haitian Creole is the use of a Manner-Path verb serialization, which in a few cases also occurred in serializations of three verbs. In other cases, two Path verbs or a Path and a non-motion verb were serialized to encode a motion event depicted in one of the drawings. 

\subsection{Motion events without a motion verb}

In the picture story narrations, seven cases occurred where a motion event was expressed without a motion verb (7.4\%). Five of these were uttered by P3, when he used the idiom \emph{kraze rak}, which has the meaning ‘to beat loose’. The two other cases were uttered by P2 using \emph{jwenn direksyon} and \emph{pran direksyon} to say that the bird was going into a certain direction, see \REF{ex:3:67}.

\ea\label{ex:3:67}
\gll  {{[}…{]}} kounye a la      li jwenn direksyon fenèt la \\
      {}        moment \textsc{def} \textsc{dem} \textsc{3sg} reach direction window \textsc{def}\\
\glt ‘Now he takes the direction of the window.’
\z

In the single picture descriptions, there is only one case where a motion event is expressed without a motion verb (0.8\%), also using \emph{direksyon}, see \REF{ex:3:68}.

\ea\label{ex:3:68}
\gll Li pran nan direksyon machin nan. \\
     \textsc{3sg} take \textsc{loc} direction car \textsc{def}\\
\glt ‘He takes the direction of the car.’
\z

\subsection{Other}

The remaining encodings of motion events had to be sorted into a separate category because it was not possible to analyze them as any of the other categories.

For the picture story narrations, all of the 16 cases sorted into this category (17\%) use the preposition \emph{dèyè} ‘behind’ or ‘after’, 13 with \emph{kouri}, see \REF{ex:3:69}, and 3 with \emph{pati}, see \REF{ex:3:70}. In all of these cases, one animal is chasing another.

\ea\label{ex:3:69}
\gll Chwèt la kouri dèyè zwazo  a pou l pran l.\\
     owl \textsc{def} run behind bird \textsc{def} \textsc{3sg} take \textsc{3sg}\\
\glt ‘The owl flies behind/after the bird in order to catch him.’
\ex\label{ex:3:70}
\gll Koukou a pati dèyè zwazo a. \\
     cuckoo \textsc{def} leave behind bird \textsc{def}\\
\glt ‘The cuckoo flies behind/after the bird.’
\z

The problem here is the preposition \emph{dèyè}: If it expresses ‘behind’, the PP can be analyzed to encode the Ground and locate the place where the motion is taking place. If it expresses ‘after’, it can be analyzed as expressing the Path of motion.

The main participant translated all of these cases with \emph{chase away}. This could be the implication of flying fast behind someone. In one case, however, P4 uses \emph{kouri dèyè} where the subject isn’t moving at all. The picture shows a group of birds sitting in a tree chasing away the little bird by screaming at him \REF{ex:3:71}.

\ea\label{ex:3:71}
\gll Zwazo sa yo genlè pa renmen  li. Yo kouri dèyè  li. E lè sa li vole.\\
     bird \textsc{dem} \textsc{pl} seem \textsc{neg} like  \textsc{3sg} \textsc{pl} run behind  \textsc{3sg} and when \textsc{dem} \textsc{3sg} fly\\        
\glt ‘These birds seem to not like him. They chase him away. And then he flies away.’
\z

Another problem is the semantics of \emph{pati}. P1 uses \emph{kouri dèyè} and \emph{pati dèyè} in a similar way. When the owl is chasing the little bird by flying after him, she uses \emph{pati dèyè}. The semantics of \emph{dèyè} are obscure here, because the Path ‘away from X’ is not relevant here. It is neither shown in the pictures nor expressed in the narration.

For the single picture descriptions, twelve descriptions had to be sorted into the category “Other” (10.1\%). Three cases were equivalent to those in the picture story narrations where \emph{dèyè} was used. In two other cases, hybrid verbs were used which could not be clearly identified as Manner or Path verbs as they contain both components: \emph{plonje} ‘dive into’ and \emph{tonbe} ‘fall’.

In \REF{ex:3:72}, a Path verb is combined with a further description of the Path.

\ea\label{ex:3:72}
\gll Yon mesye ki dwe travèse dyagonal {[}…{]} yon chanm.\\
     \textsc{indef} man \textsc{rel.pron} must cross diagonal {} \textsc{indef} room \\
\glt ‘A man who must cross a room diagonally.’
\z

In \REF{ex:3:73}, the Manner component is expressed in a PP, which is the only occurrence of this type.

\ea\label{ex:3:73}
\gll Tidjo ap monte mòn ak kat pat.\\
     Tidjo \textsc{prog} ascend mountain with four paws\\
\glt ‘Tidjo is crawling up the mountain.’
\z

The four remaining occurrences contain a gerund construction, all used by the same participant, P1. In two of these cases, a structure equivalent to the French structure V \emph{en} V.GER is used, see \REF{ex:3:74}. This structure has not been described for Haitian Creole.

\ea\label{ex:3:74}
\gll Li desann eskalye a an dansan. \\
     \textsc{3sg} descend stairs \textsc{def} \textsc{prep} dance.\textsc{ger}\\
\glt ‘He/She goes down the stairs dancing.’
\z

In the other two cases, the Path verb is followed by a gerund form of `to be', \emph{etan}, and then a full sentence consisting of subject, aspect marker and manner verb, see \REF{ex:3:75}. This structure has also not been described for Haitian Creole.\footnote{Both a native speaker present at the talk at the SPCL Meeting in Tampere as well as an anonymous reviewer of this
paper noted that this structure is very uncommon in Haitian Creole and most probably due to other language influence.} 

\ea\label{ex:3:75}
\gll Tipyè antre etan l ap danse nan chanmnam. \\
     Tipyè enter be.\textsc{ger} \textsc{3sg} \textsc{prog} dance \textsc{loc} room \textsc{def} \\
\glt ‘Tipyè is dancing into the room.’
\z


\section{Discussion}\label{sec:3:5}

In the previous section, the morphosyntactic patterns used to express motion events in the present study were described. Most commonly used were three different structures: a Path verb with a Ground NP or DP, a serial verb construction, or a Manner verb only. Path verbs with Ground elements are verb-framed structures in the sense of \citet{Talmy_1991}. In all of those cases, the Manner component of the event was not expressed. Serial verbs often consisted of a Manner and a Path verb, which could be labelled an equipollently-framed construction in the sense of \citet{Slobin_2004}. Other verb serializations were also found, mostly of two Path verbs, but also combinations of Path and non-motion verbs. These are also verb-framed constructions. The third most frequent pattern is the use of a Manner verb only, with no Path element encoded in the same clause. According to \citeauthor{Talmy_1985}'s (\citeyear{Talmy_1985}) definition, the latter is not a motion event. These cases are nevertheless taken into account here, primarily because of their relatively high frequency. Besides that, it is possible that at least some of them do, contrary to Talmy’s definition, express complex motion events, because the Path component is possibly left to be inferred in these cases but implicitly present. This was the case in some examples from the picture story narrations, where the participants uttered sentences like \emph{Zwazo a kouri/vole} ‘The bird flies (fast)’, meaning that the bird is flying away. Most of the uses of a sole Manner verb in the single picture descriptions are probably due to the fact that the depicted motion event was too difficult to recognize as such, as in \emph{swimming along the coast}. In these cases, the participants simply expressed a motion activity instead of the event.

Besides these three main patterns, three further but marginal patterns were found in the data: a Manner verb with a Path element, a Path verb alone and a motion event without any motion verb. Manner verbs with Path elements constitute satellite-framed constructions in the sense of \citet{Talmy_1991}, which were rare but are still attested in the present data. They occurred with few, but different Manner verbs and also different Path elements. When Path verbs were used alone, information on the Ground was usually given in the context. The expression of motion events with idioms or constructions like \emph{pran nan direksyon} ‘take a certain direction’ is most probably not typical for Haitian Creole but occurs in many languages.

The different patterns described above show that Haitian Creole possesses a rich inventory of morphosyntactic structures available to express motion events. It is therefore not classified here as a language of one of the three types described above, VF, SF and EF languages. All of these three patterns are found in the Haitian Creole data, VF and EF patterns being more frequent than SF patterns.

Some problematic cases were also described above which need further investigation. In the cases where \emph{dèyè} was used, it was not clear whether it expresses ‘behind’ or ‘after’ and it could therefore not be decided whether it constitutes a Ground or a Path element. This shows that clear semantic criteria to identify the components of motion events are needed. Such criteria could also help to further investigate hybrid verbs like \emph{plonje} ‘dive’ which are said to express both Manner and Path. Two other cases were also problematic as they presented completely different structures from the ones described earlier. The structures where a gerund of a Manner verb or of the verb `to be' were used, neither of which has been described for Haitian Creole. As the participants of this study all lived outside of Haiti and were using other languages such as German or French on a daily basis, it is possible that these structures are due to language contact, most probably with French. More research needs to be done in this area.

The second aim of this study was to investigate the Manner salience of Haitian Creole, that is the frequency with which the Manner component is encoded in motion event expressions in comparison with other languages. In the data described above, Manner was expressed either in a Manner-Path verb serialization or in a satellite construction.\footnote{As Manner verbs only and Manner verbs with ground expressions are not considered motion events as defined by \citet{Talmy_1985}, they are not included here. This leaves us with 76 motions event expressions in the picture story narrations and 85 motion event expressions in the single picture descriptions.} In the picture story, both Manner-Path verb serializations and satellite constructions were very rare, which means that Manner was often left unexpressed. As the story was about a bird, whose Manner of motion typically is to fly, it is not necessary to encode Manner in every motion event, as it can easily be inferred. In the single picture descriptions, Manner-Path verb serializations are used in 24.7\% (21 of 85) and satellite-framed constructions in 5.9\% (5 of 85) of the motion event expressions. With a total of 30.6\%, the frequency of Manner encodings is much higher here than in the picture story narrations. Considering the fact that all of the pictures showed a specific Manner component, this number is nevertheless rather small. Both the picture story as well as the single picture descriptions therefore indicate that Haitian Creole has low Manner salience.

In comparison to the French motion verb expressions described in the first part of the paper, some similarities and some differences can be observed. Just like French, Haitian Creole possesses a rather large inventory of Path verbs, most of which probably go back to their French counterparts. This is why VF constructions are common in both languages. However, their percentage is larger in French than in Haitian Creole, as the latter possesses another structure not available in French: verb serialization, particularly the serialization of a Manner and a Path verb. This is a feature that Haitian Creole shares with various African languages, Kwa languages in particular, which are said to have played a significant role in the formation of Haitian Creole. Just as in the Kwa languages described above, the Manner verb precedes the Path verb in the Haitian Creole motion verb serializations. Another interesting observation is the fact that in the present data, no serializations with the Manner verb \emph{danse} ‘to dance’ are attested, a Manner-Path serialization which is ungrammatical in Fongbe according to \citet{LambertBrtire_2009}. The (un)grammaticality of such serializations in Haitian Creole needs to be tested in a subsequent study. The preliminary result of the ongoing research on this question is that the morphosyntactic patterns used in Haitian Creole to express motion events seem to be a mixture of the patterns found in the languages that were relevant to its formation.

In further research, more data will be elicited. As some of the drawings used for the single picture descriptions proved to be difficult to interpret, these representations of motion events need to be revised. If possible, videos showing motion events would be preferred for data elicitation. In addition, acceptability judgments will be elicited to investigate which Manner and Path elements are possible in which pattern, mostly in serial verb constructions and satellite-framed constructions. 


\section*{Acknowledgements}
I am grateful for the ongoing support of the supervisor of my thesis, Prof. Judith Meinschaefer. Further thanks go to the native speaker who helped me transcribe and translate the Haitian Creole data. I would also like to thank two anonymous reviewers who have provided very helpful comments on this paper. All remaining errors are my own.

{\printbibliography[heading=subbibliography,notkeyword=this]}

\end{document}

\documentclass[output=paper,
modfonts
]{langscibook} 
\bibliography{localbibliography}

\usepackage{langsci-optional}
\usepackage{langsci-gb4e}
\usepackage{langsci-lgr}

\usepackage{listings}
\lstset{basicstyle=\ttfamily,tabsize=2,breaklines=true}

%added by author
% \usepackage{tipa}
\usepackage{multirow}
\graphicspath{{figures/}}
\usepackage{langsci-branding}



\newcommand{\sent}{\enumsentence}
\newcommand{\sents}{\eenumsentence}
\let\citeasnoun\citet

\renewcommand{\lsCoverTitleFont}[1]{\sffamily\addfontfeatures{Scale=MatchUppercase}\fontsize{44pt}{16mm}\selectfont #1}
  
 
\title{Language Shift}  

\author{%
 Andreia Caroline Karnopp\affiliation{University of Zurich}
}

% \chapterDOI{} %will be filled in at production
% \epigram{}

\abstract{
Language Shift has always existed. Conquests were the first cause of language shifts, then migrations prompted these types of changes, and today it is mainly language diffusion that triggers this language contact phenomenon. There are some promoting and/or retarding factors for shift, but not a single condition evokes the same patterns of language use in all language contact situations. For this reason, and because each language community should thus be considered and analyzed in isolation, this chapter discusses the most significant approaches, models, methods and examples of possible language choice patterns and trends, and finally, also addresses possible factors that may or may not boost language shift within a determined linguistic community.
}

\begin{document}
\maketitle

\section{Introduction}

Since \emph{Language Shift} (LS) is always preceded by language contact or collective multilingualism \parencite[320]{Ostler2011}, this social phenomenon is important to include in the discussions addressed within this book. Even if LS can happen at an individual level (\emph{Language Attrition} (LA)\footnote{LA is about ‘forgetting' an educational or extracurricularly learned L1, L2 or foreign language. It thus describes the loss of language skills by an individual and can in a way be considered as a reversal of language acquisition \parencite{Lambert1982}.}), it usually refers to the change in usage of a given language community from a language A to a language B in all situations and domains. This change in norm is usually observable as a bi- or multilingual period within one or across several generations.

LS is often described as a kind of `transitional phenomenon' \parencite[33]{Bohm2010} of changing language contact situations. It refers to a shift away from a `healthy' language state due to a `disorder' or a range of `disorders'\footnote{The term ‘disorder(s)' here refers to the fact that a previously rather stable speech contact situation -- above described as ‘healthy' -- can become unstable and cause a ‘disorder' as a result of various mostly related factors, and therefore change the habitual language use (clearly visible in stage B of Figure \ref{figure1} ).} of the affected languages. \emph{Language maintenance} (LM), in contrast, describes a relatively stable\footnote{LM is described here as ‘relatively stable', since long-lasting and intensive language contact can lead to interferences (e.g. borrowing-scale by \citealt{thomasonkaufman1988}), and further language contact phenomena.} language contact situation in which bilingual speakers, speaker groups or an entire linguistic community continue using the minority or heritage language\footnote{Alternative terms used in the literature are community language, (im)migrant language, ethnic language, and home language. Heritage language, however, is probably the most widely used term \parencite[23]{Pauwels2016}.}, despite the pressure of the majority and socially dominant language and other influencing factors. Consequently, the mentioned `disorders' can provoke different patterns of language use, which is why each language contact situation must be considered separately. What all situations have in common, however, is that LS affects only groups and communities which are in contact with a more dominant and more powerful social group. This is why LS is generally understood as ‘a barometer of inequality between linguistic minorities and the majority’ \parencite[613]{Heinrich2015}.

LS is not a recent phenomenon, but has been occurring throughout the history in different societies and in diverse places \parencite[4]{Puthuval2017}. \cite[326-328]{Ostler2011} supposes that LS started happening with the Neolithic revolution and the related establishment and settlement of humankind -- however, it can be assumed that these processes were already taking place before this period. Between 3000 BC and 1500 AD, dominant languages spread mainly through \emph{wars} and subsequent \emph{conquests} of rural societies. Since then, the languages of European conquest have prevailed -- e.g. Spanish in Latin America, Portuguese in Brazil or English in the USA. Dominant languages are therefore often associated with oversea explorations, invasions and migrations. Before the 20th century, \emph{migration} was the biggest risk for a group to be affected by LS. Nowadays, in contrast, the physical \emph{diffusion}\footnote{\emph{Language diffusion} (LD) often happens on an individual level and is promoted by e.g. \emph{cohabitation} (founding bilingual families) or by \emph{recruitment} (new employment, military, etc.) \parencite[323-324]{Ostler2011}.  In certain domains a LS can then progress quickly and widely. \emph{Linguistic diffusion}, on the other hand, refers to a shift of individual linguistic variants within a language -- which can also be caused by language contact -- on an individual or social level over a longer period of time. Both speakers and listeners can give different preferences to individual variants or even generate new variants \parencite{Gong2012}.} of so-called world languages plays an increasing role because young people tend to learn one of these rather than maintain their parents' minority language. 
\begin{figure}
\includegraphics[width=10cm]{Karnopp_pic1}
\caption{\emph{Language Shift Model}Abb.: L1 = first
  language; L2 = second language; HL = heritage language; ML = majority
  language; L1-HL = abandoned/heritage language; L1-ML* =
  target/majority language (can contain phonetic, morphologic,
  syntactic, semantic and prosodic traces from the HL.}
\label{figure1}
\end{figure}

%Both figures (Fig 1 and 2) were created by me especially for this chapter. How can I quote that?

As most of the LS literature, this chapter will mainly discuss the language use of minority groups -- migrant communities and territorial linguistic minorities, with a special focus on the former. Therefore the model in Figure \ref{figure1}, which is based on \cite{Fishman1964}'s three-generational model, illustrates the different phases typically leading to LS in migrant groups. Likewise \cite{Weinreich1953} assumes that at least three generations are necessary for LS to happen. \cite[6]{Ortman2008}, in contrast, point out that in many \emph{inter}generational analyses, the so-called `mother tongue shift' occurs mainly in the second and third generation. In this regard, LS can be understood here as a gradual and progressive process ① or as a reversible process ③ of the dynamics of a natural multilingual language community. On the other hand, LS is sometimes also analyzed as an outcome ② or a consequence of a language contact situation: The use of the languages changes across the three stages in Figure \ref{figure1}. A monolingual stage A is followed by a situation of language contact, caused, for instance, by migration. A bilingual transition period then follows, which is often \emph{diglossic}\footnote{\emph{Diglossia} is a special form of bilingualism of a language community in which a high and a low variety coexist. While \cite{Ferguson1959} distinguished between two variants of the same language (e.g. the case of German and Swiss German in the German-speaking part of Switzerland), \cite{Fishman1967} extended this definition to language contact situations of unrelated varieties (e.g. Hindi and Tamil in India).} and in which the collective language choice is \emph{variable} \parencite{Fasold1984}. This middle stage, which is at the heart of the progressive LS process ①, can last for one or more generations, and may affect an entire language community as follows. Three different types of bilingualism are distinguished for stage B: ⓐ supplementary, ⓑ complementary  (see LM), and ⓒ replacive bilingualism \parencite{Haugen1972}. Given that the preferred language can influence the language skills of every individual speaker (in LS situations the L1-HL, and in RLS situations the L1-ML), the three types of bilingualism can also co-occur within the same community. The bilingualism phase is then followed by another not necessarily purely monolingual stage C, as Fishman \parencite*{Fishman1964} idealistically showed. The target language can still contain traces of the L1-HL in the form of code mixing or code switching (see \emph{shift variety} in Section \ref{patterns}, and Chapter 3), 
%Crossreference
or even adopt new features from the heritage language and thus end up in a new variety or language (see Chapter 4).

\subsection*{LS as outcome ②}

Languages are social entities that need an associated society in order that their memory not be lost. This means that if speaker groups or societies, for example migrants, do not live in their home countries and lose contact with them, there is a good chance that they will shift more quickly to the L1-ML. Therefore, a language's survival depends on who speaks what, to whom and when \parencite{Fishman1964}.

The \emph{transmission} of an L1-HL to the following generations can be disturbed or inhibited during the three LS phases (see Figure \ref{figure1}) for various reasons, e.g. if a language community dies out, if it is conquered by another group who speaks a different language, or if it is eradicated (see Chapter 4)\footnote{It is important that we take into account here that LS does not end with a person's life or the life of a group, but rather represents a shift or a change from generation to generation \parencite[195]{Jagodic2011}.}. The speakers make or are forced to make a so-called ‘social choice' in order to better integrate themselves (or not) in the new society \parencite[325]{Ostler2011}. In other words, the preference for one of two or more contact languages automatically generates social closeness or distance. So if a bilingual speaker chooses the L1-ML, he automatically selects social proximity to the out-group and social distance to the in-group -- and vice versa. At this point in time the corresponding L1-HL is threatened if it is not spoken by another group or if it is not used for a specific purpose there is danger of becoming an \emph{endangered language}\footnote{Without adequate documentation, frequent use between L1-HL speakers in different situations and domains, and without transmission to the next generation, a language is \emph{endangered} and thus threatened with extinction (see EGIDS, the 13 levels of language endangerment/vitality proposed by \cite[2]{Brenzinger2003} based on Fishman's (\citeyear{Fishman1991}) 8-level GIDS -- see Section \ref{approaches}.}. The former mother tongue is then gradually replaced from generation to generation by the L1-ML (\emph{obsolescence} or \emph{language dead} at group level and \emph{attrition} at speaker level, e.g. \citealt{Crystal2000}).
According to \cite[7]{Nettle2000}, more than half of the more than 6000 languages spoken in the world are currently at such a stage. The process mainly affects small minority languages in Australia, the Pacific and in North and South America. Normally, such an advanced state of LS is irreversible, and has therefore achieved its \emph{morbid endpoint}\footnote{The difference between a \emph{morbid endpoint} of a language and \emph{language death} is that in the former case a language can still be spoken by other language communities.  On the other hand, the term \emph{language death} can be understood literally, because in this case a language is not spoken anymore, because it has been forgotten or simply not learned or passed on, and therefore no longer exists.} -- and not \emph{death} \parencite[18]{Pauwels2016}.


\subsection*{Reversing LS ③}

If a speech community sees a reason to take active steps to preserve an endangered (heritage) language, and if the language policy of a region or country supports these actions, an \emph{ongoing}\footnote{LS is to be viewed as a process in which different factors come together. This can be extremely dynamic. For this reason, and so that a model can be predictable at all, it must be flexible and able to take account of changing circumstances. Therefore I use the term \emph{ongoing}, like \cite[112]{Pauwels2016}.} LS-process can change direction and be reversed (\emph{Reversing Language Shift}, RLS), if it has not yet reached the \emph{morbid endpoint}. The heritage language can be documented by linguists and stored in archives, or get actively preserved and maintained through revitalization \parencite[315]{Ostler2011}. This reversal requires a new distribution of power between the language communities, which may lead to a different language policy issue. The idea of many supporters of minority languages is to teach it to the younger generation in school so as to enable them to use it regularly and pass it on to subsequent generation(s) \parencite[4]{Puthuval2017}. If an L1-HL plays a part in defining a sense of identity (see \emph{core-value}), if it hosts the community's culture and traditions, and if it is the basis of knowledge and experience, nowadays people or institutions often eagerly try to preserve that language. In this respect, language diversity, even though LS is the social norm \parencite[84]{Pauwels2016}, is still a universal phenomenon.

\subsection*{LS-process ①}

On the other hand, LS can also be understood as a process, process in the sense that  the dominant language spreads at the cost of the minority language \parencite[31]{Bohm2010}. Language `lives' and is associated with an \emph{ongoing} learning process that can lead to changes such as
variant formation (see Chapter 2, short- or long-term-accommodation),
%Crossreference
speaker-related language mixing (see Chapter 3, code-switching), 
%Crossreferencenew
new languages or varieties (see Chapter 4, Creoles and Pidgins) 
%Crossreference
and thus to long-term change (language shift or language change). Due to differences in individual settings, situations, speakers, etc., there is still \emph{no uniform and general definition} of the phenomenon LS. As \cite[19]{Pauwels2016} explains:

\begin{quote}
it may take one or more generations of speakers before the language is entirely abandoned. It also implies that the shifting away from the L1 does not occur simultaneously across all its users or functions and settings. The rate and speed of the shift process will vary from community to community. In some cases the process is relatively swift, within one or two generations, and in other contexts it will take much longer.
\end{quote}

The duration of the shift process therefore depends on various influencing factors (see Section \ref{factors}): While some language communities change their main language within only one generation, for instance, Dutch migrants in New Zealand (e.g. \citealt{VanRijk2017}), other migrant groups manage to maintain their L1-HL over several decades or centuries as the Amish  in the USA (e.g. \citealt{Sağlamel2013}) or the Swiss in Brazil (e.g. \citealt{Karnopp}).

Figure \ref{figure1} represents an overview of the phases typically involved in LS in a bilingual language community with language contact. However, this model does not hold for all settings and all contact situations with LS as an outcome. The transition period between a monolingual setting with language A and a monolingual setting with language B can be more multifaceted than depicted in Figure \ref{figure1} . The transition phase is discussed in more detail in the following subsections. Section \ref{approaches} discusses the main approaches to LS, presents theoretical models, and summarizes the methods typically used in LS research. Language choice patterns, which are key to the process of LS, will be discussed in section \ref{patterns}. Section \ref{factors} gives an overview of possible factors promoting LS. And finally, section \ref{discussion} summarizes the most important findings of the chapter and points out promising routes for future research.

\section{Approaches, models, and methods}
\label{approaches}

\noindent The various approaches to the study of language shift are best understood when we observe the transition period from the initial monolingual setting preceding the shift to the final ‘monolingual' setting following it. As Figure \ref{figure2} illustrates, the bilingual transition period of an LS involves not only factors regarding an individual speaker, but also often produces a situation where wider social, even societal phenomena become relevant. Furthermore, a finer differentiation between social levels (micro, meso, macro) is helpful in assessing the approaches, models, and methods presented in this section.

\begin{figure}
\includegraphics[width=10cm]{Karnopp_pic2}
\caption{Social model for LS-processes, based on Sasse \parencite*[63]{Sasse1992}.}
  \label{figure2}
\end{figure}

\subsection{Approaches and models}
\label{approachesmodels}
%(See Section \ref{approaches}).

LS and its counterpart, LM, both have a multidisciplinary nature. Since the beginning of the 20th century, they have attracted the attention of a number of scholars from a variety of disciplines -- including sociology, language sociology, anthropology, social psychology, sociolinguistics, contact linguistics, applied linguistics, demography, politics and history. If a language is to be considered in connection with its speakers and an entire society, this can sometimes lead to an interdisciplinary challenge -- because each discipline has its own questions and methods, it can sometimes be difficult to make them compatible with each other. Since LMLS studies are characterized by a wealth of approaches, models, and research methods, only a portion of the most influential ones are introduced in what follows (however, in Section \ref{factors} some of these will be taken up with regard to factors that can promote or retard LS).

Kloss initiated LM's systematic study for ethnic minorities in Germany. His key text (\citeyear{Kloss1966}) on language choice in correlation with a wide range of individual and group factors led to the generation of a quantitative taxonomic-typological model\footnote{In a taxonomic-typological model, language is named on the basis of types and systematically classified with regard to its structural and functional features.}. This was the first attempt to demonstrate the dynamics of LM and LS.
Based on Kloss' work, Haugen  (\citeyear{Haugen1972}) developed his concept of ‘language ecology'\footnote{In his model, Haugen used the ecosystem as a metaphor to show how languages behave in different language contact situations and how endangered languages can be preserved, similar to endangered species.} in migrant settings and expanded the field from Europe to North America. His \textit{descriptive} and explanatory model was the first to take into account the interaction between languages, their speakers, and the social environment.
Fishman's (\citeyear{Fishman1972}) study on ‘language use patterns' is one of the most important approaches in LMLS research. Using his famous question ‘who speaks what language to whom, and when', shift processes can since then be analyzed across a range of (originally) five main domains: \emph{family, education, employment, friendship,} and \emph{religion} -- although these may vary depending on each specific language contact situation. He assumes an ideal language contact situation in which all members are multilingual, regardless of the language competence of its individual speakers. If the analysis of an \emph{intra}group within domains and further factors is extended to an \emph{inter}group situation, language contact can be analyzed not only at the micro level but also at the macro level (\cite[335-336]{Werlen2004}, see also Figure \ref{figure2}).
Giles' et al. \parencite*{Giles1977} ‘ethnolinguistic vitality model'\footnote{ \cite[308]{Giles1977} understand ‘ethnolinguistic vitality' to be the distinctive and active collective behaviour of a minority group in \emph{inter}group relations.} includes socio-psychological factors as \emph{status, demography}, and \emph{institutional support}. To analyze the vitality perceptions of languages in contact within and between minority and dominant groups, language identity and language attitudes play a decisive role (see Section \ref{factors}).
The fourth milestone is Gal's \parencite*{Gal1979} study on language use in bilingual communities in the Austrian-Hungarian border. She was the first to take into account \textit{social and communicative networks}\footnote{Gal \parencite*{Gal1979} understands ‘social and communicative network' as the environment in which a speaker normally interacts in a given unit of time.} (see also \citealt{Dorian1980}). Language choice plays a very important role in this. However, if both contact languages are equally appropriate in a network, it is not possible to predict which language bilingual speakers will choose in which communicative situation. With her new qualitative approach in this field she was able to show that for certain language groups, at a specific historical moment, \emph{language choice can be variable}. 

Gal's pioneering study was followed by further research and enhanced concepts such as Bourdieu's \parencite*{Bourdieu1977, Bourdieu1982} \emph{linguistic markets} - stands as a metaphor for the place where linguistic exchange occurs and linguistic ‘capital' can be exchanged; Lieberson's \parencite*{Lieberson1980} distinction between \emph{age-grading} and \emph{age-cohort} analyses - linguistic changes on an individual level (former) or within an age group (latter); Smolicz's \parencite*{Smolicz1980} \emph{core-values} and their relationship to LM, with regard to the most important cultural and social values of a linguistic community; Tajfel's \parencite*{Tajfel1981} \emph{social identity theory} - which is intended to explain \emph{inter}group behaviour; Fishman's \parencite*{Fishman1991} discussion about RLS and the necessary redistribution of power within a community, as well as the promotion of the 8-level \textit{Graded Intergenerational Disruption scale} (GIDS) - an evaluative framework to identify endangered languages; and Edwards' \parencite*{Edwards1992} \emph{typology of language endangerment} - which includes factors for the viability of endangered languages.
%verylongandcomplexsentence

Although the approaches and models listed here are fundamental for a better understanding of the dynamics of LS processes, each of them also has individual weaknesses. In addition, they can only shed light on a specific part of the whole phenomenon. For example, Kloss' \parencite*{Kloss1966} clear-cut factors are not necessarily unequivocal indicators of LM for all migrant contexts. On the other hand, additional factors - not taken into account by Kloss - may also lead to the preservation of a heritage language (e.g. \citealt{Clyne1991}  in his research on migrants in Australia). Fishman's \parencite*{Fishman1972} model is based on a clear domain-by-domain shift, which is nowadays extended to further domains, as each language contact situation is unique and can therefore generate additional ‘exchange locations'. On the other hand, Fishman's domains can be inhibited by other language contact phenomena such as code-mixing and code switching
%Crossreference Chapter 2
. His proposal is thus better placed within an expanded \emph{domain continuum} -- from public to private domains -- by taking into account both the LS of a single speaker and the LS within the language community. Smolicz' \parencite*{Smolicz1980} core-value theory was also criticized by Clyne \parencite*{Clyne1991} due to its relative simplicity: the definition of ‘group' is problematic, the model is inapplicable to several group affiliations, and language attitudes can change, even if they normally are considered to be stable over a longer period of time (e.g. RLS). Newer approaches and models aim at \emph{hybridity} and being \emph{ongoing}, whereby abstract and episodic-concrete language material is to be made comparable and tested using various factor combinations. One of the first hybrid models for LS was published by Wei \parencite*{Wei2002} with his concept of ‘market, hierarchy, and network' that makes interaction strategies of individual speakers combinable with the community wide norms and values.

In summary, while in the initial phase of LS research the focus was on universal and abstract variables and systems ⑥, which were based on top-down approaches and aimed at the development of traditional-generative models at the macro level, Haugen's \parencite*{Haugen1972} descriptive approach was the trigger to catch the macro world by defining the micro level ④ of a social system. Since then, research approaches within the LS study show a preference for user/agent-based bottom-up models\footnote{For the differentiation of generative and usage-based models, see Langacker \parencite*{Langacker2000} and Prochazka \parencite*{Prochazka2017}.}. Even if the meso level ⑤ was not directly addressed here, it is crucial especially for the preservation of a minority language, since it deals with \emph{inter}generational \emph{transmission}. If the language is not transmitted to the next generation, it is forgotten and lost within the language community (e.g. \citealt{Gal1979}, \citealt{Brenzinger2003}). Starting from the meso level, LS can be viewed in two different ways: By default, an intergenerational LS is normally assumed, that is, a change between generations or age groups within a language community (see Figure \ref{figure2} ⑤⑥, \citealt{Lieberson1980}). The Lagged Generation Model by Myers et al. (2006 in \citealt{Ortman2008}) serves this purpose. However, since LS can also happen within a single speaker (see LA), the \emph{intra}generational change must also be taken into account -- e.g. analyzable  by the Period Cross Section Approach, discussed in Myers et al. (2006 in \citealt{Ortman2008}, see Figure \ref{figure2} ⑤④). 
In this regard, \cite[1423-1424]{Lutz2006} stated in her study on Latino youth in the USA that ‘the shift from Spanish to English as a usual language appears to occur as children progress through the school system’. 

\subsection{Data gathering methods}

Language data for LS research can be collected in various ways: through large-scale surveys and census data ⑥, or through observing language use by individuals through participatory observation, interviews, tests, and experiments ④\&⑤ \parencite[48]{Pauwels2016}. A distinction is also made between real-time and apparent-time methods. In a real-time study, the language of different age groups is observed over a longer period of time. Longitudinal studies can show a possible language change of a community as progress through time. Apparent time methods, often implemented as a one-shot case study, focus on the speech patterns of different age groups -- younger and older speakers -- in a specific moment in time and can indicate a language change in progress.

Using a questionnaire is the most commonly applied method for data collection in LMLS investigations. It can be applied to large-field studies, which tend to target quantitative data, or to smaller-field qualitative analyses. A challenge regarding all methods that employ interviews, in addition to the choice of informants, is the interviewer's role. In certain cases a bilingual interviewer or a member of the group under study may be preferred. This helps to ensure the authenticity of the linguistic data, and that trust and solidarity with the informant can be built up. The questionnaires themselves vary from closed-ended questions to multiple-choice questions, point scales to open questions \parencite[53-61]{Pauwels2016}. 

Surveys and census data are often used in longitudinal studies, providing objective data for comparison. Within the field of LS these data can be used to assess, for instance, number of speakers, geographical distribution, and sociodemographic profiles \parencite{Clyne1991}. However, surveys can be expensive and they address only a portion of the targeted group at a time. Censuses, on the other hand, are more exhaustive, but data regarding language use is often inaccurate or even subjective and therefore not especially valuable. Regarding this concern \cite[para. 16]{Buda1992} adds that:

\begin{quote}
respondents may not be fully conscious of their own language usage patterns, or may wish to portray them in a socially or culturally favorable light. Very often the respondent's assessment of his or her own language ability and usage represents more of what he or she would wish them to be, and less of what they really are.
\end{quote}

In the same vain, Pauwels \parencite*[66]{Pauwels2016} notes that a self-assessment of language skills is not comparable with accurately measured linguistic proficiency in reliability.

\section{Language choice patterns and trends}
\label{patterns}

The Fishman question -- who speaks what language to whom, and when -- can be further expanded with the question of \textit{how well} a language is spoken. Pivotal for answering these questions is \emph{language choice}, the selection of a language in a given communicative situation. As language choice patterns are variable and often difficult to generalize, LS can be seen as a long-term consequence of language choice \parencite[53]{Holmes2013}. \cite[68]{Fishman1965} suggests that language choice must first be analyzed in individual face-to-face meetings before approaching the ‘problem of the broader, underlying choice determinants on the level of larger group or cultural settings’. Concerning this, language choice patterns within a stable bilingual setting can be further applied to interpret less stable contact situations (see Figure \ref{figure1}). A domain analysis concerning the three social levels (see Figure \ref{figure2}) is useful to observe some general language choice patterns and trends. In what follows, the expanded Fishman question will be used as a framework for this more compassing analysis.

\subsection*{Who?}

The question \emph{who} speaks a specific contact language can be viewed, for example, in relation to \emph{age-related} patterns. Many studies (e.g. Gal \citeyear{Gal1979}, Wei \citeyear{Wei2002}, Karnopp \citeyear{Karnopp}) show that older speakers would rather maintain an L1-HL, while younger people often shift much faster to an L1-ML. This has to do with the fact that older migrants are usually more dependent on their heritage language. Learning the majority language is often more difficult for them -- if they learn it at all -- and their \emph{social networks} tend to the in-group. However, the use of L1-ML can also increase for older generations, for instance when they spend a lot of time within the out-group during their working years (see subsection ‘When?’). Likewise, younger generations may grow up in an L1-HL environment, by the latest within school or already before -- or by older siblings or through media -- they come into contact with the majority language (\emph{speech capital} ⑤). \cite[84-85]{Pauwels2016} notes in this connection, that if the second generation does not speak the heritage language as well as the first generation, and, additionally, if their language displays more contact phenomena, such as for instance code switching (see Chapter 3),
%Crossreference
LS progresses faster. 

On the other hand, \emph{gender-related patterns} can influence language choice, even if researchers do not agree on this. \cite[213-215]{Labov1990} therefore proposed the ‘gender paradox', which states that women can be both conservative and innovative in language use. But whether the female gender inspires or retards LS depends on her \emph{role relationship} and \emph{status} in a determined minority \parencite[86-88]{Pauwels2016}. Hence, a monolingual housewife who never had to or could learn the majority language, and also cares for her (old) parents, rather tends toward LM. In contrast, a bilingual woman who no longer lives in a migrant context may prefer L1-ML (LA), although a ‘healthy' bilingualism (LM) cannot be excluded here either.


\subsection*{To whom?}

In addition to the individual circumstances of each bilingual speaker, it is equally important to consider \emph{to whom} someone speaks one of two or more contact languages. Again this is closely related to a speaker's \emph{social network}, \emph{role relationship} and the conversational \emph{topics}\footnote{Along with domain analysis, these are further factors \cite{Fishman1964} considers in order to best define the language choice within a speech community.}: If a bilingual works in the countryside and only has contact with members of the in-group, an increased use of the L1-HL and thus LM is more likely. In contrast, a small in-group network and a low \emph{common routine} can boost LS. 

The \emph{home domain} also helps show how the role relationship within a family can change over generations, as the home is usually the last location to be affected by LS \parencite[616]{Heinrich2015}. If the \emph{speech capital}\footnote{\cite[18]{Bourdieu1977} defined \emph{speech capital} as the mastery of a language. Speech capital is closely related to cultural capital, since it is not  important to learn grammar and vocabulary, but equally vital for the speaker to identify with the culture’s \emph{language attitude} and \emph{prestige}.} ⑤ in families is low, or parents consider it `unfavorable' to transmit a language (\emph{transmission pattern}), this can have a negative effect on the language setting and use. \cite[84-89]{Pauwels2016} emphasizes that in migrant communities it is often the case that parents of the first generation use the L1-HL regularly among themselves, and with others the same age and older. In parent-child conversations there is a continuum of reciprocal to non-reciprocal use of the heritage language. The second generation therefore can learn the heritage language, but uses it less and are not as likely to \emph{transmit} it, especially in exogamy families. Nevertheless, the L1-HL can be used again more frequently when children have to look after their parents in old age.


\subsection*{What language and how well?}

A distinction between \emph{inter}- and \emph{intra}individual variation is useful at this point, since nobody speaks the same way all the time, and the speaker's choice among varieties -- languages or speech styles (\emph{language choice pattern} -- is usually linked to the corresponding social context in some way (\cite[12-17]{Gal1979}, see also Chapter 2
%Crossreference
). In any case, bi- or multilingual speakers normally know which of the two or more languages in contact to use with whom, and when (\emph{linguistic competence} and \emph{linguistic performance}\footnote{According to \cite[3]{Chomsky1965} every person has an unconscious grammatical knowledge of a language which is innate and allows them to understand and speak. Within this concept, he makes a fundamental distinction between \emph{competence} -- which includes knowledge of the speaker and listener of a language -- and \emph{performance} -- which describes the actual use of a language in specific conversational situations.}). Depending on the \emph{speech capital} ⑤ of the parents or older siblings, speakers in minority settings can unconsciously learn several languages simultaneously (\emph{bi- or multilingualism}).  In doing so, the \emph{competence} of each speaker may differ as follows: If both parents are bilingual -- or even speak only L1-HL --, there is an increased tendency toward LM in the home domain. The same applies to endogamy (\emph{endogamy pattern}). However, if both parents come from a different ethnic minority or one of them belongs to the majority society (\emph{exogamy pattern}), the probability of LS is considerably higher \parencite[89]{Pauwels2016}. \cite[65]{Holmes2013} thus affirms that ‘{[}m{]}arriage to a majority group member is the quickest way of ensuring shift to the majority group language for the children’. In contrast, an L1-HL can be sparsely spoken and transmitted to the next generation due to, for instance, negative \emph{attitude}, negative \emph{prestige}, lack of \emph{institutional support}, or  the support of other (world) languages of more \emph{economic interest} (\emph{diffusion pattern}). Often the heritage language is then only hesitantly used due to \emph{uncertainty}. In an environment with such a low \emph{linguistic starting point} ④, the children may not acquire the full competence of a minority language and thus the performance can contain inaccuracies (see \emph{semi-speaker} in Dorian \citeyear[87]{Dorian1980}). Concerning \cite[116]{Wei2002}, such speakers tend to use ‘linguistic innovations, structural changes, and new varieties of language’. 

On this account recent LS-studies not only focus on \emph{language choice patterns}, but also on how languages influence each other within their \emph{linguistic levels}. This can happen on a lexical as well as on a grammatical level, as \cite[35]{thomasonkaufman1988} showed with their ‘five level scale of borrowing'\footnote{\cite[37]{thomasonkaufman1988} define \emph{borrowing} as the ‘incorporation of foreign features into a group's native language: the native language is maintained but is changed by the addition of the incorporated features’.}. In their opinion, the \emph{contact-intensity}, and not the \emph{language structures}, determine possible outcomes of language contact. However, later studies also confirm the latter (e.g. Treffers \citeyear{Treffers1999}), because 

\begin{quote}
speakers in general are able to construct new word formation devices, new syntactic forms and generally are linguistically creative, in the well-documented Chomskyan sense, even if input of a specific structure is slight or lacking. \parencite[591]{Gal2008}
\end{quote}

During an \emph{ongoing} shift process, language contact effects usually happen between the middle and the last stage (see Figure \ref{figure1}). In my own research on the Swiss colony called Helvetia in São Paulo \parencite{Karnopp}, where the speakers today are strongly assimilated to the L1-ML, I was still able to identify some \emph{salient patterns}, as some informants showed a different, and for the region rather atypical, pronunciation of  some consonants in both contact languages (\emph{pronunciation pattern}). For instance, the common retroflex [ɻ] of the region is hardly used by the older bilingual generation, while the youngest generation uses it more than the surveyed young generation of the out-group, in order to differentiate themselves linguistically. Another finding are neologisms (\emph{word-formation pattern}), such as \emph{xeníssimo}, composed of the Swiss German adjective ‘scheen' and the Portuguese suffix ‘-issimo'. 

Sometimes even in the third and last LS phase (see Figure \ref{figure1}) there can still be some ‘remnants' of the former language contact situation. This is the case, for example, when bilinguals ‘create' new \emph{shift varieties}, recognized and adopted subsequently by the entire language community (see \emph{shift-induced change} in Thomason and Kaufman \citeyear[38]{thomasonkaufman1988}). In Ireland, for instance, comparatives are double marked in non-standard Irish English such as ‘working more harder’ \parencite[153]{Hickey2010}. According to the author, this shift can have two causes: either the form was taken from the Irish comparatives, formed by the particle \emph{níos} 'more' as well as the inflection of the adjective; or it comes from an older form of English, where this \emph{doubling pattern} also occurred.

\subsection*{When?}

For a better understanding of LMLS processes the study of interactional settings are central and imperative. A fundamental distinction is made between public and private domains (\emph{domain pattern}), which -- as I have already mentioned -- should be treated as a \emph{continuum}, as they are not always clearly delimitable. For example, if someone teaches at home, this domain is becoming both private and public. The \emph{labor market}, on the other hand, is considered to be a public space, although acquaintances and friendships between colleagues or business partners can also be cultivated here, which in turn produces more of a private character. The labor market thus has many facets, of which four possible language contact situations are shown below.

Minority members who work in a family business, for example farmers with a little village shop, may have a smaller \emph{social network} often limited to the in-group. If an L1-HL enjoys positive \emph{prestige} in such an environment, LM can be expected. However, if a migrant no longer lives within his language community because they moved to the next larger city for professional reasons, the tendency to use L1-ML in everyday life increases drastically. If the use of the heritage language furthermore decreases within the family domain, \emph{attrition} can be the consequence. It becomes even more challenging when a heritage language speaker works in a multi- or international company, where the linguistic exchange takes place exclusively in a \emph{lingua franca}, such as English or Spanish (\emph{diffusion pattern}). The probability that a minority language will ‘survive' under such circumstances is at this point rather low within the labor market -- but not impossible. Another complex language contact situation occurs at construction sites, where members of different ethnic groups work together. Language contact phenomena such as accommodation (see Chapter 2)
%Crossreference
or code switching (see Chapter 3)
%Crossreference
are common here, since many construction workers have often not (yet) properly learned the majority language. In order to still be able to communicate with each other, the L1-ML is drastically simplified and usually pronounced with a noticeable accent -- which leads to this variety being strongly stigmatized (negative \emph{prestige}). For mutual understanding to be possible, only common knowledge of the meaning and application of the words referring the construction are important. Today this ‘primitive language' is considered a \emph{learner variety} -- and not a pidgin, even if there are simplified structures in both of them (see Chapter 4)
%Crossreference
-- of migrant workers \parencite[129-135]{Riehl2014}.

More (rather) public domains proposed by \cite{Fishman1972} are \emph{education} and \emph{religion}. The school is not only a possible pivot point for learning (heritage) languages, but is also crucial for their revitalization and preservation (see RLS). Consequently, it is important to have, for instance, a supportive \emph{language policy} as well as for the minority group(s) to be \emph{interested in preserving and cultivating} these languages. For example, the Swiss descendants in Helvetia \parencite{Karnopp} built their own private school shortly after the founding of the colony, where their children could learn High German -- since this is the official standard language in German-speaking Switzerland -- as well as Swiss history and culture. After the nationwide ban on learning and using foreign languages, official German lessons were discontinued\footnote{High German courses were offered in the 1990s and have again been introduced since 2015. However, these efforts are only moderately fruitful and, in my opinion, do not lead to a language revival in Helvetia.}. Since the school was nationalized, in the 1980s, and had to open its doors to non-Swiss descendants, LM  was not possible anymore within this domain. Either way, \cite[95-96]{Pauwels2016} points out that when private schools consider the L1-HL of a language community, the programs usually focus only temporarily on bilingualism. Their aim is to prepare the students for the \emph{linguistic assimilation} toward L1-ML. However, in-group children often go to mainstream schools, where they only communicate in the majority language anyway.

On the other hand, the church is an important meeting place for believing (minority) groups, because faith unites and strengthens. The Helvetians have always been very Catholic, therefore they built their own church, and some of them still believe their harvest depends on the goodness of Saint Nicholas. Until the ban on foreign languages, the mass was said in High German. To this day, the Helvetians have maintained their tradition of exchanging greetings in front of the church after the official part of the service. However, what has changed is the language: the discussions have shifted from Swiss German to ‘regional' Portuguese -- with very little interferences like \emph{Giotä Sunnti} (trad. Have a good Sunday).

\section{Factors that can promote LS}
\label{factors}

\begin{quote}
Why is it that one minority group assimilates and its language dies, while another one maintains its linguistic and cultural identity? (Bradley 2002: 1, \emph{apud} Pauwels \citeyear[58]{Pauwels2016})
\end{quote}

Most studies on LS have repeatedly focused on identifying causes and factors which can promote or slow down the LS  process. On this basis, scholars have tried to generate a \emph{unique set of factors} that make LS predictable within every language community. However, certain factors may achieve differential effects, even in very similar contact situations. Kloss (\citeyear{Kloss1966})  noticed this early and suggested a typology in which he not only offered a set of \emph{clear-cut} factors (which clearly promote LM), but also \emph{ambivalent} factors (which can promote LM \emph{and} LS). The latter are \emph{linguistic attitude}\footnote{\emph{Linguistic attitudes} describe a positive or negative evaluation through social status of a language or variety in contact.} (speakers with negative feelings toward their L1-HL tend to LS), \emph{educational level} (speakers with little or no education tend to LM because their \emph{social network} is often smaller and more limited to in-group contacts, while a speaker with a higher degree may work outside the community and therefore has more  frequent contact to the majority), \emph{linguistic and cultural similarity} (contact languages from the same language family do or do not tend toward LS -- it depends on whether the desire for assimilation or differentiation is greater), \emph{numerical strength of the group} (language communities with a smaller number of L1-HL speaker tend to be more LS oriented, because they have little \emph{common routine}), and other sociocultural characteristics such as \emph{role of the family} (if an L1-HL is not used anymore for communication within a family -- in other words, low \emph{speech capital} -- and is no longer \emph{transmitted} to the next generation, the tendency is LS).

Thereupon Fishman  (\citeyear{Fishman1972}) presented his domain analysis for speech communities. Each of them contains domain-specific factors with regard to \emph{addressee} (to whom a specific language is spoken), \emph{setting} (in which environment a language in contact is used), and \emph{topic} (which subjects promote the language choice) -- discussed in greater detail earlier in this section. 

Giles' et. al. (\citeyear{Giles1977}) suggested three factors to define a minority group's vitality: \emph{status} (economic, social, sociohistorial, and language status), \emph{demography} (distribution and numbers of speakers), and \emph{institutional support} (formal and informal facilities). He explains that minority groups with a higher vitality (\emph{high attitude, high prestige, common routine}, etc.) tend to differentiate themselves from the dominant group, while those with a lower vitality show faster assimilation, and thus a faster LS.

In Gal's (\citeyear{Gal1979}) pioneering study, she considered social causes ⑥ such as \emph{urbanization} (LS often lasts longer in rural areas than in cities), \emph{industrialization} (new and better qualified jobs, achievable e.g. through higher education can also lead to LS), \emph{loss of isolation} (once rural regions have been taken over by political power, LS progresses), and different \emph{social and communicative networks} that can influence language use and language choice. \cite{Dorian1980} adds \emph{migration}, \emph{mobility of people} (the progress of means of transport and communication makes people much more flexible and enables them to move within a short time to another linguistic environment), and \emph{community size} (see \emph{numerical strength of the group} in Kloss \citeyear{Kloss1966}) to the list of factors pushing LS.

Smolicz's (\citeyear{Smolicz1980}) core-value theory states that \emph{symbolic group values} -- for instance language attitude and prestige, family cohesion, and religious- and cultural unity -- have a massive influence on LMLS and can thus convey a different identity\footnote{\emph{Identity} is a term very difficult to define because it is dynamic and changeable. It stands for, among other things, a correlation between ‘being me' and ‘belonging to the group'. Every human being has different ‘identities' which predominate depending on the situation or with whom one is interacting. Within language contact research ethnic/national (group membership, e.g. based on physical, religious or social factors), social (e.g. social stratification), geographical (e.g. language and dialect) and contextual (e.g. secret languages) identities can be relevant \parencite[172-173]{Riehl2014}.}. For example, if the L1-HL is handled as a \emph{core-value} within a language community, LM is more likely. In contrast, a negatively assessed heritage language of a single speaker (see \emph{attitude}) or the majority (see \emph{prestige}), makes LS more likely.

In more recent case and theoretical studies, further factors such as \emph{age} (in minority communities older people tend to be bilingual, while younger people sometimes hardly understand or speak the L1-HL), \emph{gender} (gender roles in the society -- see Section \ref{patterns}), \emph{language transmission} (if an L1-HL is not passed on to the next generation, younger people no longer speak the heritage language, which provokes LS), \emph{religion} (if in a bilingual colony the sermon is delivered in L1-ML, it is more likely that the majority language will be maintained in conversations after the church service -- see Section 3), \emph{marital status} (exogamy usually leads to LS), \emph{linguistic, social and ethnic identity} (identification with a group often supports assimilation, which can promote LS or LM, depending on the situation), \emph{language prestige} (if a bilingual language community appreciates more the L1-ML, LS is foreseeable), \emph{literacy} (if a contact language is not read or written, there is also a tendency toward LS) and \emph{media}\footnote{By the term \emph{media} I do not only mean written access (newspapers, magazines, etc.), but also digital media such as televisions and, above all, the Internet, computers, smartphones and tablets, which nowadays make contact with the home country easier and more accessible.} (low medial contact with the L1-HL can cause LS) are proposed to analyze LS stages (see e.g. Lutz \citeyear{Lutz2006}, Ortman \citeyear{Ortman2008}, Böhm \citeyear{Bohm2010}, Jagodic \citeyear{Jagodic2011}, Ostler \citeyear{Ostler2011}, Sağlamel \citeyear{Sağlamel2013}, Heinrich \citeyear{Heinrich2015}, Pauwels \citeyear{Pauwels2016}, Perez \citeyear{Perez2016}, Puthuval \citeyear{Puthuval2017}, Rijk \citeyear{VanRijk2017}, and Karnopp \citeyear{Karnopp}). \cite*{Holmes2013} therefore proposes a classification into economic, political, institutional, demographic, attitudinal, educational and socio-cultural factors. For her, these categories are the ones that can be held responsible for the speed of LS within a bilingual community.

As already introduced before, my current research \parencite{Karnopp} looks at the language contact situation related to the Swiss colony Helvetia in São Paulo, Brazil. Since its foundation in 1888, the \emph{language usage patterns} within the colony have undergone some fundamental changes (see also Section \ref{patterns}). Initially Helvetia was a \emph{language island} and therefore linguistically quite well isolated and shielded. The everyday language was the L1-HL -- Swiss German dialect from the Canton of Obwalden -- and when communication with the out-group was required, a translator was called in. After the First World War, the colony's own school had to hire Portuguese teachers and introduce the Portuguese language and other subjects related to Brazil like history and geography. Most of the Helvetians slowly became bilingual and could then communicate with the out-group (\emph{outside diglossia}). With the Second World War all foreign languages were banned in Brazil and everyone who continued to use them was fined or even arrested. These circumstances then led to \emph{inner diglossia}, which henceforth favored LS in all domains within the colony. Today only a few of the oldest generation surveyed still speak and understand the old Swiss German dialect -- many times with a light accent or interferences from Portuguese (see Section \ref{patterns}) -- and with this advanced linguistic assimilation to the Brazilian out-group LS reached its \emph{morbid endpoint} there.

In order to illustrate more precisely which major factors lead to this outcome within the Swiss colony in São Paulo, I defined fourteen main social and individual factors on the basis of the proposed social model (see Figure \ref{figure2}):
\begin{quote}
\textbf{At the micro level ④:} (1) rapid decrease of L1-HL usage in \emph{all domains} -- today the old Swiss German dialect has, even in the \emph{home domain}, a very low \emph{common routine}, (2) \emph{growing language diffusion} among young Helvetians, who would rather learn Spanish or English than High German or the dialect of their ancestors, (3) \emph{low linguistic attitudes and values} toward their heritage language -- because it is no longer needed for communication within the community and therefore considered useless.\end{quote}

\begin{quote}
\textbf{At the meso level ⑤:} (4) \emph{lack in transmitting} the L1-HL after the Second World War, (5) \emph{small group size} which is still decreasing today, (6) \emph{increased exogamy}, among other things to avoid hereditary diseases, (7) \emph{little contact with the homeland} because the Swiss relatives rarely speak Portuguese and less than 30 Helvetians speak Swiss German or High German.\end{quote}

\begin{quote}
\textbf{At the macro level ⑥:} (8) \emph{length of stay since arrival}, because the attachment to Switzerland tended to diminish due to (7) and \emph{low medial contact}, (9) \emph{low geographical concentration} due to resettlement to neighboring bigger cities, offering them more economic opportunities and security, (10) \emph{industrialization} and \emph{career change} away from the peasant lifestyle, (11) lack of \emph{isolation}, especially after the Second World War, due to \emph{political pressure}, (12) \emph{no institutional support} of the L1-HL, (13) \emph{no official literacy}\footnote{Since the standard language in German-speaking Switzerland is High German, it is the language taught in school and used in official contexts (\emph{medial diglossia}, see Glaser \citeyear{Glaser2014}). Moreover, a universal grammar for Swiss German dialects does not exist, because there is a continuum of dialect varieties (\emph{Dialektkontinuum}).} is available for the heritage dialect to date -- neither in Switzerland nor in Brazil, (14) \emph{low language prestige} -- because older bilinguals often have a (light) accent probably caused through language contact, and this is often criticized by younger Helvetians and the out-group.\end{quote}

To conclude, I would like to discuss in more detail the factor that is currently considered one of the biggest risks for LS: \emph{language diffusion}. With regard to globalization and the resulting convergence of languages has been increasingly discussed in recent years. Although heritage languages with a larger population can be supported by the language policy of a given region/country, their use in many migrant settings is diminishing. As I have shown above, mostly older people continue to speak a heritage language or are at least \emph{semi-speakers} \parencite{Dorian1980}, while younger people often have no possibility to learn it due to lack of preservation, neglected transmission, missing institutional support or insufficient revival will. Consequently, what can happen is that younger migrants learn other languages -- apart from the L1-ML-- which, for instance, can help them in \emph{economic terms} \parencite{Holmes2013}. In this regard, India, Pakistan and China show the growing importance of English as a world language. In these countries financial security is defined by speaking English. Only with the competence of this \emph{lingua franca} it is possible to obtain a high rank in the business world, where English determines all financial activities. In contrast to this, \cite[74]{Nawaz2012} explain that in India the less prestigious Punjabi does not guarantee any financial security and is
associated with the low and uneducated majority. Lastly, the high bilingualism rate, even if applied in clearly different \emph{domains}, can cause language contact phenomena such as accommodation (see Chapter 2)
%Crossreference
or code switching (see Chapter 3).
%Crossreference
If learning second languages other than the heritage language becomes more important within a bilingual group, individual factors such as \emph{attitude}, \emph{identity}, \emph{language loyalty}\footnote{\emph{Language loyalty} is a term used to describe a speaker's (conscious or unconscious) relationship with her or his mother tongue.}, and consequently \emph{language prestige} also come into play. For example:

\begin{quote}
They may feel shame when other people hear their language. They may believe that they can only know one language at a time. They may feel that the national language is the best language for expressing patriotism, the best way to get a job, the best chance at improving their children's future. \parencite{SIL}
\end{quote}

Lastly, and to return to my current research, even if the Helvetians appreciate their ancestors and their efforts, their L1-HL will probably not experience a revival there because it has lost its \emph{vitality} and its \emph{social network}. The Swiss dialect is  still considered important for cultural events such as yodeling, but useless in terms of language use, and therefore largely irrelevant within the colony. Consequently, the L1-HL no longer possesses importance as a \emph{core-value} in Helvetia, which is why the LS process will be completed soon.

\section{Discussion}
\label{discussion}

From what we have seen so far, it emerges that not only are languages ‘alive', but also that every language contact situation is \emph{dynamic} and thus \emph{different}. In the scenario of \emph{ongoing} LS, an individual speaker, a group or a whole language community \emph{can} \emph{choose} between an L1-HL and an L1-ML, although this usually happens unconsciously.  \cite[para. 8]{Buda1992} confirms this by arguing that ‘{[}t{]}he phenomenon of language shift takes place out of sight and out of mind’. RLS, on the other hand, certainly happens much more consciously, since it relies on the \emph{will} of  individuals, and of the whole language group, to reintegrate the heritage language into their \emph{social network} for specific purposes (see Figure \ref{figure1}).

LS can also happen when more than just two languages are in contact. In similar settings, \cite{Perez2016} observed that the shift commonly goes toward one of the more prestigious languages. Consequently the \emph{prestige}-factor is certainly one of the important determinants with regard do LMLS. However, in her study of the language contact situation in the Anglo-Paraguayan community New Australia, different circumstances led to the fact that the population did not shift toward one of the global languages (English or Spanish) -- which in turn contradicts the widespread tendency toward LD -- but rather chose to adopt the indigenous language Guarani. 

As this chapter shows, approaches, models and methods must be applicable to exceptional and constantly changing settings that take into consideration the \emph{inter}- and \emph{intra}group variation, but at the same time these methodological choices often compromise different research goals within all three social levels (see Figure \ref{figure2}). With the inclusion of my current research, I wish to reaffirm the fact that the dynamics influencing individual and social changes can be very different in each language contact situation. Therefore the tools that need to be developed for the study of LMLS have to be \emph{hybridized}. Ideally, this would be done by designing a framework that includes a universally applicable \emph{continuum}, from which every researcher would take only what they need for their research goal. The right path to designing this framework has already been taken by recognizing that there are \emph{no specific factors} that can be applied to all LMLS situations, because some factors may or may not promote different language-choice patterns. Now it is only a matter of implementation.

\printbibliography[heading=subbibliography,notkeyword=this]

\end{document}

\section{Introduction} 
Phasellus maximus erat ligula, accumsan rutrum augue facilisis in. Proin sit amet pharetra nunc, sed maximus erat. Duis egestas mi eget purus venenatis vulputate vel quis nunc. Nullam volutpat facilisis tortor, vitae semper ligula dapibus sit amet. Suspendisse fringilla, quam sed laoreet maximus, ex ex placerat ipsum, porta ultrices mi risus et lectus. Maecenas vitae mauris condimentum justo fringilla sollicitudin. Fusce nec interdum ante. Curabitur tempus dui et orci convallis molestie \citep{Chomsky1957}.

\begin{table}
\caption{Frequencies of word classes}
\label{tab:1:frequencies}
 \begin{tabular}{lllll} 
  \lsptoprule
            & nouns & verbs & adjectives & adverbs\\ 
  \midrule
  absolute  &   12 &    34  &    23     & 13\\
  relative  &   3.1 &   8.9 &    5.7    & 3.2\\
  \lspbottomrule
 \end{tabular}
\end{table}

Sed nisi urna, dignissim sit amet posuere ut, luctus ac lectus. Fusce vel ornare nibh. Nullam non sapien in tortor hendrerit suscipit. Etiam sollicitudin nibh ligula. Praesent dictum gravida est eget maximus. Integer in felis id diam sodales accumsan at at turpis. Maecenas dignissim purus non libero scelerisque porttitor. Integer porttitor mauris ac nisi iaculis molestie. Sed nec imperdiet orci. Suspendisse sed fringilla elit, non varius elit. Sed varius nisi magna, at efficitur orci consectetur a. Cras consequat mi dui, et cursus lacus vehicula vitae. Pellentesque sit amet justo sed lectus luctus vehicula. Suspendisse placerat augue eget felis sagittis placerat. 

\ea
\gll cogito                           ergo      sum\\  
     think.\textsc{1sg}.\textsc{pres} therefore \textsc{cop}.\textsc{1sg}.\textsc{pres}\\ 
\glt `I think therefore I am.'
\z

Sed cursus eros condimentum mi consectetur, ac consectetur sapien pulvinar. Sed consequat, magna eu scelerisque laoreet, ante erat tristique justo, nec cursus eros diam eu nisl. Vestibulum non arcu tellus. Nunc dignissim tristique massa ut gravida. Nullam auctor orci gravida tellus egestas, vitae pharetra nisl porttitor. Pellentesque turpis nulla, venenatis id porttitor non, volutpat ut leo. Etiam hendrerit scelerisque luctus. Nam sed egestas est. Suspendisse potenti. Nunc vestibulum nec odio non laoreet. Proin lacinia nulla lectus, eu vehicula erat vehicula sed. 


\section*{Abbreviations}
\section*{Acknowledgements}



\ChapterAndMark{Special and Nexal Negation} 
\label{ch:5}
\is{nexal negation!and special negation|(}
\is{scope of negation|(}
\is{special negation!and nexal negation|(}

The negative notion may belong logically either to one definite idea or to the combination of two ideas (what is here called the nexus).

\is{special negation!defined}
The first, or special, negation may be expressed either by some modification of the word, generally a \is{prefixes!negative|(}prefix, as in\label{sec:neg-prefix}
\is{grammaticalization}  

\phantom{a}

\begin{tabular}{@{}l@{}}
\textit{\emph{n}ever} (etc., see p.~\pageref{para:neveretc})\\

\textit{\emph{un}happy}\\

\textit{\emph{im}possible}, \textit{\emph{in}human}, \textit{\emph{in}competent}\\

\textit{\emph{dis}order}\\

\textit{\emph{non}-belligerent}\\
\end{tabular}

\phantom{a}

\is{adverbs!negative|(}
\noindent (see on these prefixes \chapref{ch:13})\is{prefixes!negative|)}---or else by the addition of \il{English!not@\textit{not}|(}\textit{not} (\textit{not happy}) or \il{English!no@\textit{no}}\textit{no} (\il{English!no longer@\textit{no longer}}\textit{no longer}). Besides there seem to be some words with inherent negative meaning though positive in form: compare pairs like \is{inherently negative meaning}

\phantom{a}

\begin{tabular}{@{}ll@{}}
\textit{absent}& \textit{present}\\

\textit{fail}& \textit{succeed}\\

\textit{lack}& \textit{have}\\

\textit{forget}& \textit{remember}\\

\textit{exclude}& \textit{include}\\
\end{tabular}

\phantom{a}

But though we naturally look upon the former in each of these pairs as the negative (\textit{fail} = `not succeed'), nothing hinders us from logically inverting the order (\textit{succeed} = `not fail'). These words, therefore, cannot properly be classed with such formally negative words as \textit{unhappy}, etc.

A simple example of negatived nexus is \textit{he doesn't come}: it is the combination of the two positive ideas \textit{he} and \textit{coming} which is negatived. If we say \textit{he doesn't come today}, we negative the combination of the two ideas \textit{he} and \textit{coming today}; compare, on the other hand, \textit{he comes, but not today}, where it is only the temporal idea \textit{today} that is negatived.
\is{nexal negation!defined}

Though the distinction between special and nexal negation is clear enough in principle, it is not always easy in practice to distinguish the two kinds, which accounts for some phenomena to be discussed in detail below. In the sentence \textit{he doesn't smoke cigars} it seems natural to speak of a negative nexus, but if we add \textit{only cigarettes}, we see that it is possible to understand it as `he smokes, but not cigars, only cigarettes'.

Similarly, it seems to be of no importance whether we look upon one notion only or the whole nexus as being negatived in \textit{she is not happy} (`she is [positive] not-happy' or `she is not [negative nexus] happy'); %  PE: The (outer) parentheses are ours; the (inner) brackets are OJ's. He used ( ) parentheses. Didn't we agree that parentheses within parentheses would remain ( ) and not become [ ]? ... Though come to think of it, this is an editorial interpolation (by OJ), so [ ] would be more appropriate.
thus also \textit{it is not possible to see it}, etc. In these cases, there is a tendency to attract \textit{not} to the verb: \textit{she isn't happy}, \textit{it isn't possible to see it}, but there is scarcely any difference between these expressions and \textit{she is unhappy}, \textit{it is impossible to see it}, though the latter are somewhat stronger. If, however, we add a subjunct like \textit{very}, we see a great difference between \textit{she isn't very happy} and \textit{she is very unhappy}.

\is{quantifiers!negatived|(}
The nexus is negatived in \refp{ex:05-01}.

\ea \label{ex:05-01}
\textit{Many of us didn't want the war}, but many others did\hfill(news 1917)
\z

\noindent which rejects the combination of the two ideas \textit{many of us} and \textit{want the war} and thus predicates something (though something negative) about \textit{many of us}. But in \textit{Not many of us wanted the war} we have a special negative belonging to \textit{many of us} and making that into \textit{few of us}; and about these it is predicated that they wanted the war. Cf. p.~\pageref{08-not-all}ff (in \chapref{ch:8}) below % PE: OJ simply specifies chapter VIII; no page number
on \textit{not all}, \textit{all {\dots} not}.

Note also the difference between \textit{the disorder was perfect} (\textit{order} negatived) and \textit{the order was not perfect} (nexus negatived, which amounts to the same thing as: \textit{perfect} negatived).

In a sentence like \textit{he won't kill me} it is the nexus (between the subject \textit{he} and the predicate \textit{will kill me}) that is negatived, even though it is possible by laying extra emphasis on one of the words seemingly to negative the corresponding notion; for \textit{\textsc{he} won't kill me} is not `not-he will kill me', nor is \textit{he won't \textsc{kill} me} `he will do the reverse of killing me', etc.\footnote{Jespersen notes in his Addenda that on this page ``or in some other place combinations like \textit{he regretted that \textsc{more} Englishmen did \textsc{not} come here} (news 1917) should have been mentioned''. \eds} % PE: ??? It's not necessary to have both this footnote and the single-sentence paragraph starting "Combinations" that now appears below. Or if simply deleting this footnote is somehow unsatisfactory, then lets simplify it a little and attach it to the single-sentence paragraph.

Cf. also the following passage from \citet[\href{https://archive.org/details/elementarylesson00jevo/page/174/mode/2up?q=\%22curious+to+observe\%22&view=theater}{175}]{jevons1893elementary}:

\begin{quote}
It is curious to observe how many and various may be the meanings attributable to the same sentence according as emphasis is thrown upon one word or another. Thus the sentence ``The study of Logic is not supposed to communicate a knowledge of many useful facts,'' may be made to imply that the study of Logic \textit{does} communicate such a knowledge although it is not supposed to; or that it communicates a knowledge of a \textit{few} useful facts; or that it communicates a knowledge of many \textit{useless} facts. 
\end{quote}

\is{nexal negation!tendency towards}
\is{position of negative}
\label{para:nexal-negation-tendency}There is a general tendency to use nexal negation wherever it is possible (though we shall later on see another tendency that in many cases counteracts this one); and as the (finite) verb is the linguistic bearer of a nexus, at any rate in all complete sentences, we therefore always find a strong tendency to attract the negative to the verb. We see this in the prefixed \textit{ne} in French as well as in Old English, and also in the suffixed \is{grammaticalization}\is{negation, suffixes}\il{English!n't@\textit{-n't}}\textit{-n't} in Modern English, which will be dealt with in \chapref{ch:11}, and in the suffixed \textit{ikke} in modern Norwegian, as in \textit{Er ikke (erke) det fint?} (`Is not (archaic \textit{not}) that nice?') and \textit{Vil-ikke De komme?} (`Will-not you come?'), where Danish has the older word-order \textit{Er det ikke fint?} (`Is it not nice?') and \textit{Vil De ikke komme?} (`Will you not come?').

\is{auxiliary verbs}
In Modern English the use or non-use of the auxiliary \textit{do} serves in many, but not of course in all, cases to distinguish between nexal and special negation; thus we have special negation in \refp{ex:05-02}.\label{p:nexal-sp}

\ea \label{ex:05-02}
He seems \textit{not certain} of his way\hfill(\href{https://archive.org/details/mrswarrensprofes00shawuoft/page/160/mode/2up?q=%22seems+not+certain%22}{Shaw, \textit{Profession} 160})
\z

Combinations like \refp{ex:05-03} should also be mentioned.

\ea \label{ex:05-03}
He regretted that \textit{more} Englishmen \textit{did not} come here\hfill(news 1917) % PE: ??? It's not necessary to have both this little paragraph and the footnote "Jespersen notes in his Addenda that....").
\z\il{English!not@\textit{not}|)}
\is{quantifiers!negatived|)}

\is{partitive constructions|(}
In French we have a distinction which is somewhat analogous to that between nexal and special negation, namely that between \is{grammaticalization}\textit{pas de} (`not any') and \textit{pas du} (`not some'): \textit{je ne bois pas de vin} (`I do not drink (any) wine'); \textit{ceci n'est pas du vin, c'est du vinaigre} (`this is not (some) wine, it's (some) vinegar'), see the full treatment in \citet[\href{https://www.nb.no/items/b6e32092abfe229e8854840d79878e30?page=113}{p.~87ff}]{storm1911storre}. Good examples are found in \refp{ex:05-04}, but see % PE: "but" (by itself) is straight from OJ; "see" is my (feeble) addition, just to make it read less oddly. 
\refp{ex:05-05}. 

\ea \label{ex:05-04}
 \gll ce n' était \textit{plus de} la poésie, ce n' était \textit{pas} \textit{de la} prose, ce était de la poésie, mise en prose\\
 this not was {more of} the poetry this not was not {of the} prose this was of the poetry put into prose\\
 \glt `it was no longer poetry, nor was it prose, but poetry put into prose'\\\hfill(\href{https://www.gutenberg.org/cache/epub/62021/pg62021-images.html}{Rolland, \textit{Buisson} 192})
\z

\ea \label{ex:05-05}
 \gll Il n' y a \textit{pas} \textit{d'} amour, \textit{pas} \textit{de} haine, \textit{pas} \textit{d'} amis, \textit{pas} \textit{d'} ennemis, \textit{pas} \textit{de} foi, \textit{pas} \textit{de} passion, \textit{pas} \textit{de} bien, \textit{pas} \textit{de} mal.\\
 it not there has not of love not of hate not of friends not of enemies not of faith not of passion not of good not of evil\\
 \glt `There is no love, no hate, no friends, no enemies, no faith, no passion, no good, no evil.'\hfill(\href{https://www.gutenberg.org/cache/epub/62021/pg62021-images.html}{ibid 197})
\z

With the partitive force of \textit{pas} with \textit{de} should be compared the well-known use of the genitive for the object in Russian negative sentences and with \textit{nět} (`there is not'), etc., also the use of the partitive case for the subject of a negative sentence in Finnish. % ??? PE: I've moved this down from its previous position, in front of the last two quotations. 
\is{partitive constructions|)}

In the case of a contrast we have a special negation; hence the separation of \textit{is} (with comparatively strong stress) and \il{English!not@\textit{not}|(}\textit{not} in \refp{ex:05-06}. \il{English!do@\textit{do}|(}\textit{Do} is not used in such sentences as \refp{ex:05-07}.\is{auxiliary verbs|(}

\ea \label{ex:05-06}
the remedy is, not to remand him into his dungeon, but to accustom him to the rays of the sun\hfill(\href{https://archive.org/details/essaysonmiltona05macagoog/page/n128/mode/2up?view=theater&q=remand}{Macaulay, \textit{Milton} 1.41})
\z

\ea \label{ex:05-07}
\ea
I came not to send peace, but a sword\hfill(\href{https://www.kingjamesbibleonline.org/1611_Matthew-10-34/}{AV \textit{Matthew} 10.34})
\ex
my ruin came not from too great individualism of life, but from too little\hfill(\href{https://archive.org/details/deprofundiswilde00wildiala/page/104/mode/2up?q=%22great+individualism+of+life%22&view=theater}{Wilde, \textit{Profundis} 135})
\ex
We meet not in drawing-rooms, but in the hunting-field\\\hfill(\href{https://www.gutenberg.org/files/30432/30432-h/30432-h.htm}{Dickinson, \textit{Symposium} 14})
\z
\z

Even in such contrasted statements, however, the negative is very often attracted to the verb, which then takes \textit{do}, the latter part being then equivalent to \textit{but we meet in the hunting-field} \refp{ex:05-08}.\is{auxiliary verbs|)}

\ea \label{ex:05-08}
we do not meet in the drawing-room, but in the hunting-field 
\z

\ea \label{ex:05-09}
\ea
I do not complain of your words, but of the tone in which they were uttered
\ex
I do not admire her face, but [I do admire] her voice
\ex
He didn't say that it was a shame, but that it was a pity
\ex
I did not come to curse thee, Guinevere\\\hfill(\href{https://en.wikisource.org/wiki/Idylls_of_the_King/Guinevere}{Tennyson, \textit{Guinevere}}; contrast not expressed)
\z
\z\il{English!do@\textit{do}|)}\il{English!not@\textit{not}|)}

In such cases, the Old English verb naturally had no \il{English!Old English!ne@\textit{ne}|(}\textit{ne} before it, see e.g. \refp{ex:05-15}. The exception in \refp{ex:05-18} may be accounted for by the Latin word-order \textit{non veni pacem mittere, sed gladium}. But in Ælfric we have \refp{ex:05-19}, where the meaning is `it happened not-unprovidentially', as shown by the indicative \textit{wæs} and by the necessity of the repetition \textit{hit getimode}. Cf. also the Middle English version: % PE: As is, this reads strangely. Can we simply skip "edited by Paues 56:"?
%Brett: sure % PE Done.
\refp{ex:05-20}.

\ea \label{ex:05-15}
\ea\il{English!Old English!nalles@\textit{nalles}}
\gll wen ic þæt ge for wlenco nalles for wræcsiðum ac for higeþrymmum Hroðgar \textit{sohton}\\
 expect I that you for pride {not at all} for {misery} but for {high spirits} Hrothgar {sought}\\
\glt `I expect that you did not seek Hrothgar out of dire straits but out of boldness and strength of heart'\hfill(\href{http://ebeowulf.uky.edu/ebeo4.0/CD/main.html}{\textit{Beowulf} 338})
\ex
\gll ðæt he nalæs to idelnesse, swa sume oðre, ac to gewinne, in ðæt mynster \textit{eode}\\
 that he not to idleness as some others but to labour into that monastery went\\
\glt `that he did not go into the monastery for idleness, as some others, but to labour'\hfill(\href{https://archive.org/details/oldenglishversio02bede/page/264/mode/2up?q=%22+pcet+he+nales+to+idelnesse%2C+swa+aume+o%27Sre%22&view=theater}{\textit{Bede} 4.3}) % Linked-to version uses þ rather than ð, and also uses macrons (or what look like macrons). What to do?
%Brett: The old link was https://archive.org/details/anglosaxonreader00wyatuoft/page/52/mode/2up?q=%22to+idelnesse%22&view=theater. The new one still has þ, but at least it doesn't have the diacritics. % Peter: Thank you. This is good enough, I think.
\ex
\gll ðe ic lufode na for galnesse ac for wisdome\\
 whom I loved not for wantonness but for wisdom\\
\glt `whom I did not love for lust but for wisdom'\hfill(\href{https://archive.org/details/anglosaxonversi00thorgoog/page/n34/mode/2up?q=%22for+galnesse%22&view=theater}{\textit{Apollonius} 255})
\z
\z

\ea \label{ex:05-18}
\gll ne com ic sybbe to sendanne, ac swurd\\
 not come I peace to send but sword\\
\glt `I did not come to send peace, but a sword'\hfill(\href{https://books.google.co.jp/books?id=twINAAAAIAAJ&newbks=1&newbks_redir=0&printsec=frontcover&pg=PA48&dq=%22ic+sybbe+to+sendanne%22&hl=en&redir_esc=y#v=onepage&q=%22ic%20sybbe%20to%20sendanne%22&f=false}{WG \textit{Matthew} 10.34})
\z

\ea \label{ex:05-19}
\gll Ne getimode þam apostole Thome unforsceawodlice, þæt he ungleafful wæs Cristes æristes, ac hit getimode þurh Godes forsceawunge\\ % PE: Restored "Cristes æristes", whose meaning I'd GUESS is "Christ's resurrection" (see https://bosworthtoller.com/790 )
%Brett: looks good
 not happened {to the} apostle Thomas {by chance} that he unbelieving was Christ's resurrection but it happened through God's providence\\
\glt `It did not happen to the apostle Thomas by chance that he doubted Christ's resurrection, but it occurred through God's providence'\\\hfill(\href{https://archive.org/details/homiliesanglosa00thorgoog/page/234/mode/2up?q=%22apostole+Thome+unforsceawodlice%22&view=theater}{Ælfric, \textit{Homilies} 1.234}) % Peter: Can we change "was unbelieving in" to "doubted"?
%Brett: done
\z\il{English!Old English!ne@\textit{ne}|)}

\ea \label{ex:05-20}\il{English!Middle English!ne@\textit{ne}}\il{English!Middle English!noȝt@\textit{noȝt}}
\gll For Crist ne sende noȝt me {for to} baptyze, bote {for-to} preche þe gospel\\
 for Christ not sent not me to baptize but to preach the gospel\\
\glt `For Christ sent me not to baptize, but to preach the Gospel'\\\hfill(\href{https://archive.org/details/fourteenthcentur00pauerich/fourteenthcentur00pauerich/page/56/mode/2up?q=%22me+for+to+baptyze%22&view=theater}{MEV \textit{1 Corinthians} 1.17})
\z

\il{English!not@\textit{not}|(}Other examples of constructions in which \textit{not} is referred to the verb instead of some other word \refp{ex:05-21}.%Brett: OJ has single quotes

\ea \label{ex:05-21}
\ea
I did not step into the well-known boat Without a cordial greeting (`I stepped {\dots} not without')\\\hfill(\href{https://en.wikisource.org/wiki/The_Prelude_(Wordsworth)/Book_IV}{Wordsworth, \textit{Prelude} 4.16})
\ex
Don't pay only the arrears, pay all you can. (`Pay, not only')\hfill(\href{https://archive.org/details/quisantanovel00hopegoog/page/n144/mode/2up?q=%22pay+only+the+arrears%22&view=theater}{Hope, \textit{Quisanté} 132})
\ex
it doesn't only concern myself\hfill(\href{https://archive.org/details/freelands00galsrich/page/288/mode/2up?q=%22concern+myself%22&view=theater}{Galsworthy, \textit{Freelands} 332})
\z
\z

Note also \refp{ex:05-24}, where the sentence \textit{we aren't here} in itself is a contradiction in terms. (Differently in \refp{ex:05-25}, where \textit{not} belongs more closely to what follows.)

\ea 
\ea \label{ex:05-24}
We aren't here to talk nonsense, but to act
\ex \label{ex:05-25}
We are here, not to retire till compelled to do so 
\z
\z

\is{auxiliary verbs}
When the negation is attracted to the verb (in the form \textit{n't}), it occasions a cleaving of \il{English!never@\textit{never}}\textit{never}, \textit{ever} thus standing by itself. In writing the verbal form is sometimes separated in an unnatural way: (\ref{ex:05-26}, representing the spoken \textit{Can't she ever~{\dots}}); and thus we get seemingly \il{English!not ever@\textit{not ever}|(}\textit{not ever} (`never', \ref{ex:05-27}, different from the old \textit{not ever} as in \href{https://archive.org/details/utopiasirthomas00robigoog/page/n355/mode/2up?q=%22not+euer%22&view=theater}{More, \textit{Utopia} 244}, which meant `not always'). Compare the rare \il{English!not any@\textit{not any}}\textit{not any} as in \refp{ex:05-36}.

\ea \label{ex:05-26}
\textit{Can she not ever} write herself?\hfill(\href{https://archive.org/details/alfredlordtenny05tenngoog/page/n250/mode/2up?q=%22can+she+not+ever%22&view=theater}{Hallam, letter})
\z

\ea \label{ex:05-27}
\ea
You shan't \textit{touch} those hostels ever again. Ever.\hfill(\href{https://archive.org/details/wifeofsirisaacha00well/page/422/mode/2up?view=theater&q=%22touch+those+hostels%22}{Wells, \textit{Wife} 422})
\ex
I suppose you don't ever write to him?\hfill(\href{https://archive.org/details/dollydialogues00hope_0/page/62/mode/2up?view=theater&q=%22ever+write+to+him%22}{Hope, \textit{Dialogues} 40})
\ex
I can't ever see that man again.\hfill(\href{https://archive.org/details/marriageofwillia0000mrsh_i0u5/page/284/mode/2up?q=%22can%27t+ever+see+that+man+again%22&view=theater}{Ward, \textit{Marriage} 242})
\ex
Don't you ever go down beneath the surface of things?\\\hfill(\href{https://archive.org/details/septimus00unkngoog/page/n253/mode/2up?q=%22you+ever+go+down%22&view=theater}{Locke, \textit{Septimus} 26})
\ex
so don't you ever be troubled about that\hfill(\href{https://archive.org/details/prodigalson00caingoog/page/n222/mode/2up?view=theater&q=%22don%27t+you+ever+be+troubled%22}{Caine, \textit{Prodigal} 219})
\z
\z

\ea \label{ex:05-27a}
\ea
let not euer The soule of Nero enter this firme bosome\\\hfill(\href{https://internetshakespeare.uvic.ca/doc/Ham_F1/scene/3.2/index.html#tln-2260}{Shakespeare, \textit{Hml} 3.2.411})
\ex
A light around my steps which would not ever fade\\\hfill(\href{https://archive.org/details/completepoeticalshel/page/78/mode/2up?view=theater&q=%22light+around+my+steps%22}{Shelley, \textit{Revolt} 4.34})
\ex
Do you not ever go?\hfill(\href{https://archive.org/details/dukeschildrennov00troluoft/page/172/mode/2up?q=%22do+you+not+ever%22&view=theater}{Trollope, \textit{Children} 2.40})
\ex
you shall not---not ever\hfill(\href{https://archive.org/details/widowershousesun00shaw/page/42/mode/2up?q=%22you+shall+not%22&view=theater}{Shaw, \textit{Houses} 40})
\z
\z\il{English!not ever@\textit{not ever}|)}

\ea \label{ex:05-36}\il{English!no@\textit{no}}
``Had any gentleman heard of a dauphin killed by small-pox?'' No; \il{English!not any@\textit{not any}}\textit{not any} gentleman \textit{had} heard of such a case.\hfill(\href{https://archive.org/details/miscellaneousess00dequuoft/page/78/mode/2up?q=%22heard+of+a+dauphin%22&view=theater}{Quincey, \textit{Murder}}) % PE: Quincey italicizes "had"; OJ (among his Addenda) italicizes both "not any" and "had"
\z

\is{scope of negation!ambiguity of}
\is{scope of negation!cause or reason and|(}
A special case of frequent occurrence is the rejection of something as the cause of or reason for something real, expressed in a negative form: `he is happy, not on account of his riches, but on account of his good health' expressed in this form \textit{he is not (isn't) happy on account of his riches, but on account of his good health}. It will easily be seen that \textit{I didn't go because I was afraid} is ambiguous (`I went and was not afraid', or, `I did not go, and was afraid'), and sentences like this are generally avoided by good stylists. In \refp{ex:05-37}, the clause gives the reason for the speaker not wanting to be patronized. Similarly \refp{ex:05-38}.\is{ambiguity!related to reason}

\ea \label{ex:05-37}
Don't patronize \textit{me}, Ma, because I can take care of myself \\\hfill(\href{https://archive.org/details/ourmutualfriendc0000char/page/258/mode/2up?q=%22patronise+me%22&view=theater}{Dickens, \textit{Friend} 348}) % In this edition at least, "patronise" has an "s".
\ex \label{ex:05-38}
I have not drunk deep of life because I have been unathirst\\\hfill(\href{https://archive.org/details/bwb_P8-BLX-259/page/150/mode/2up?view=theater&q=unathirst}{Locke, \textit{Morals} 151})
\z

\is{scope of negation!intonation and}
In the spoken language a distinction will usually be made between the two kinds of sentences by the tone, which rises on \textit{call} in \textit{I didn't call because I wanted to see her} (but for some other reason), while it falls on \textit{call} in \textit{I didn't call because I wanted to avoid her} (the reason for not calling). In (\ref{ex:05-39} \& \ref{ex:05-40}), the clause indicates the reason for the prohibition.

\ea \label{ex:05-39}
You mustn't come whining back to me, because I won't have you\\\hfill(\href{https://archive.org/details/runningwater00masouoft/page/104/mode/2up?q=%22whining+back+to+me%22&view=theater}{Mason, \textit{Water} 95})
\ex\label{ex:05-40}
We have not gagged our Press because we disliked our freedom, but because to this extent the Prussian has triumphed\hfill(\textit{Parable}) % From Addenda.
\z\il{English!not@\textit{not}|)}
\is{adverbs!negative|)}
\is{scope of negation!cause or reason and|)}

In other languages we have corresponding phenomena. G. Brandes's \refp{ex:05-41} is ambiguous; and when Ernst Møller writes \refp{ex:05-42}, I suppose that most readers will misunderstand it as if \textit{opløses} were to be taken in a positive sense; it would have been made clearer by a transposition: \refp{ex:05-43}. Also, \refp{ex:05-44}.

\ea \label{ex:05-41}
\gll [Napoleon] handlede ikke saadan, fordi han trængte til sine generaler\\
[Napoleon] acted not thus because he needed to his generals\\
\glt `Napoleon did not act like this, as he needed his generals'\\
`Napoleon did not act thus because he needed his generals [but for some other reason]'\hfill(\href{https://archive.org/details/napoleonoggarib00brangoog/page/n33/mode/2up?q=%22handlede+ikke+saadan%22&view=theater}{\textit{Napoleon} 52}) % PE: OJ cites Brandes writing in Tilskueren. This is a journal; in March '24, the issue was not online at kb.dk. In the linked-to book, Brandes writes not "Napoleon" but "Han" (i.e. "He").
%% SG: The second alternative is not a possible translation of the Danish sentence. The two possible readings are: (a) 'Napoleon did not act like this, as he needed his generals'; and (b) 'Napoleon did not act thus because he needed his generals [but for some other reason]'

\ex \label{ex:05-42}
\gll Men retningens magt opløses, som alt fremhævet, ikke fordi dens argumenter og læresætninger eftergås og optrævles; dens magt vil blive stående\\
 but movement.\DEF.\POSS{} power {is dissolved} as already emphasized not because its arguments and doctrines {are examined} and {are unraveled} its power will remain standing\\
\glt 
`But, as already emphasized, the power of the movement will not be dissolved because its arguments and doctrines are examined and unraveled; its power will continue to stand'\\\hfill(\textit{Inderstyre} 249, in speaking of ``Christian Science'') % ??S When checked in March '24 and again in September '24, not online at kb.dk or archive.org. Is there a copy somewhere that we can link to?
\ex \label{ex:05-43}
\gll Men som alt fremhævet opløses retningens magt ikke~{\dots}\\
 but as already emphasized {is dissolved} movement.\DEF.\POSS{} power not\\
\glt `But, as already emphasized, the power of the movement is not dissolved~{\dots}'
\ex \label{ex:05-44}
\gll Jeg elsker ikke mit sprog, fordi det er eller har været herligt og {skjønt {\dots}} jeg elsker det, fordi det er mine fædres og mit folks sprog\\
 I love not my language because it is or has been glorious and beautiful I love it because it is my fathers.\POSS{} and my people.\POSS{} language\\
\glt `I do not love my language because it is or has been glorious and beautiful {\dots} I love it because it is the language of my ancestors and my people'
\hfill(\href{https://books.google.com/books?id=XAJJAQAAMAAJ&pg=RA3-PA90&lpg=RA3-PA90&dq=madvig+%22Jeg+elsker+ikke+mit+sprog%22&source=bl&ots=rZlO8Lq4of&sig=ACfU3U0BK7Im87JxQP6To6D2Ey22Jl04eg&hl=en&sa=X&ved=2ahUKEwjCn5X5gZOFAxV4nK8BHWEgCx8Q6AF6BAgIEAM#v=onepage&q=madvig%20%22Jeg%20elsker%20ikke%20mit%20sprog%22&f=false}{Madvig, \textit{Kjönnet} 90})% From Addenda. % OJ attributes this to "Madvig Program 1857. 90." Perhaps a mistake? Anyway, it's on p. 90 of Om Kjönnet i Sprogene.
\z

\phantomsection \label{p:48}\il{English!not@\textit{not}|(}
\is{auxiliary verbs|(}
\is{adverbs!negative|(}
Not unfrequently \textit{not} is attracted to the verb in such a way that an adverb, which belongs to the whole proposition, is more or less awkwardly placed between words which should not properly be separated, as in \refp{ex:05-45}. The tendency to draw the auxiliary and \textit{not} together has, on the other hand, been resisted in % ??? PE: Originally: "on the other hand, been resisted in the following passages"; immediately followed by six quotations. (Yes, "passages" is a bit of a stretch.) But in view of the new arrangement, "the following passages" seems superfluous at best. If we keep it, then let's move "The tendency to draw ... the following word" downwards, so that it immediately precedes "You will of course not meet him..." 
\refp{ex:05-49}. In most of these, \textit{not} evidently is a special negative, belonging to the following word.

\ea \label{ex:05-45}
\ea
you \textit{are not probably} aware {\dots}\\(`probably you are not aware', or: `you are probably not aware')\\\hfill(\href{https://archive.org/details/dukeschildrennov00troluoft/page/38/mode/2up?q=%22are+not+probably+aware%22&view=theater}{Trollope, \textit{Children} 1.76})
\ex
were he at that moment Home Secretary and in the cabinet, he \textit{would not probably} be reading it\hfill(\href{https://archive.org/details/marriageofwillia0000mrsh_i0u5/page/268/mode/2up?q=%22were+he+at+that+moment+Home+Secretary%22&view=theater}{Ward, \textit{Marriage} 228})
\ex
Edward Manisty, however, \textit{was not apparently} consoled by her remarks\hfill(\href{https://archive.org/details/cu31924013567130/page/2/mode/2up?q=%22not+apparently+consoled%22&view=theater}{Ward, \textit{Eleanor} 2}) % OJ has "M."; this is restored to "Manisty"
\ex
This is a strong expression. Yet it \textit{is not perhaps} exaggerated.\\\hfill(news 1917)
\z
\z
\is{adverbs!negative|)}

\ea \label{ex:05-49}
\ea
You \textit{will of course not} meet him until he has spoken to me\\\hfill(\href{https://archive.org/details/widowershousesun00shaw/page/28/mode/2up?q=%22you+will+of+course%22&view=theater}{Shaw, \textit{Houses} 27})
\ex
he \textit{is clearly not} a prosperous man\hfill(\href{https://archive.org/details/doctorsdilemmatr00shawuoft/page/20/mode/2up?q=%22prosperous+man%22&view=theater}{Shaw, \textit{Dilemma} 21})
\ex
they \textit{had clearly not} been unfavourable to him\hfill(\href{https://archive.org/details/strangeadventure00blac/page/268/mode/2up?view=theater&q=%22unfavorable+to+him%22}{Black, \textit{Phaeton} 280}) % The edition linked to actually spells it "unfavorable", but presumably British editions spell it "unfavourable".
\ex
a fashionable music-master, whose blood \textit{was certainly not} Christian\\\hfill(\href{https://archive.org/details/marriageofwillia0000mrsh_i0u5/page/154/mode/2up?q=%22whose+blood+was%22&view=theater}{Ward, \textit{Marriage} 133}) % "fashionable" restored
\ex
It'\textit{s simply not} fair to other people \phantom{x} (`is simply unfair')\\\hfill(\href{https://archive.org/details/silverboxcomedyi00gals/page/54/mode/2up?q=%22simply+not+fair+to+other+people%22&view=theater}{Galsworthy, \textit{Box} 55})
\ex
the smashing up of the Burnet family {\dots} \textit{was disagreeably not} in the picture of these suppositions\hfill(\href{https://archive.org/details/wifeofsirisaacha00well/page/120/mode/2up?view=theater&q=%22smashing+up+of%22}{Wells, \textit{Wife} 120}) % OJ had removed "by the International Stores"
\z
\z

\is{redundancy of expression|(}
It has sometimes been said that the combination \textit{he cannot possibly come} is illogical; \textit{not} is here taken to the verb \textit{can}, while in Danish and German the negative is referred to \textit{possibly}: \refp{ex:05-55}. There is nothing illogical in either expression, but only redundance: % PE: OJ wrote "redundance", not "redundancy". Uncommon a century ago and rare now, "redundance" seems a legit word.
the notion of possibility is expressed twice, in the verb and in the adverb, and it is immaterial to which of these the negative notion is attached.

\ea \label{ex:05-55}
\ea
\gll han kan umuligt komme [Danish]\\
 he can impossibly come\\
\glt `he can't possibly come'
\ex
\gll er kann unmöglich kommen [German]\\
 he can impossibly come\\
\glt `he can't possibly come'
\z
\z
 % ??? PE Pretty sure that this pair should either (A) be glossed, etc, and labelled as Danish and German respectively, or (B) moved back into the paragraph. The latter is the less cumbersome.
%% SG: Shall I add glosses?
\is{redundancy of expression|)}

\is{adverbs!negative|(}
When \textit{not} is taken with some special word, it becomes possible to use the adverb \textit{still}, which is only found in positive sentences: \refp{ex:05-56} is different from \textit{the officers were not yet friendly} (\textit{not yet} nexal negative) insofar as the latter presupposes a change having occurred after that time, which the former does not. Cf. also \refp{ex:05-57}.

\ea \label{ex:05-56}
The officers were still not friendly\hfill(news 1917) 
\z

\ea \label{ex:05-57}
\ea
Although I wrote to him a fortnight ago, I have still not heard from him\hfill(letter 1899)
\ex \il{English!no@\textit{no}}my head is still in no good order \phantom{x}(`is still bad', slightly different from `is not yet well')\hfill(\href{https://archive.org/details/journaltostellae00swifuoft/page/502/mode/2up?q=%22no+good+order%22&view=theater}{J. Swift, \textit{Journal} 503})
\z
\z

\textit{Yet not} is rare: \refp{ex:05-59}.

\ea \label{ex:05-59}
Pekuah was yet not satisfied {\dots}\hfill(\href{https://archive.org/details/historyrasselas01johngoog/page/n117/mode/2up?q=%22was+yet+not+satisfied%22&view=theater}{Johnson, \textit{Rasselas} 112}) % "P." expanded to "Pekuah"
\z
\is{auxiliary verbs|)}

\il{English!not a/one@\textit{not a}/\textit{one}|(}
\label{para:kind-of-stronger-no}\textit{Not a} or \textit{not one} before a substantive (very often \textit{word}) is a kind of stronger \il{English!no@\textit{no}}\textit{no}; at any rate, the two words may be treated as belonging closely together, i.e. as an instance of special negative, the verb consequently taking no auxiliary \textit{do}; cf. \citet[\href{https://archive.org/details/jespersen-1954-a-modern-english-grammar-on-historical-principles-part-ii-syntax-first-volume/page/426/mode/2up?q=\%22not+one+word\%22&view=theater}{16.73}]{jespersenMEG2}, where many examples are given; see further \refp{ex:05-60}.

\ea \label{ex:05-60}
\ea
Say not a word of it\hfill(\href{https://archive.org/details/mansfieldpark00aust_1/page/364/mode/2up?q=%22say+not+a+word%22&view=theater}{Austen, \textit{Mansfield} 395})
\ex
The Face seemed to smile, but answered not a word\\\hfill(\href{https://archive.org/details/snowimageothertw0000hawt/page/44/mode/2up?q=%22face+seemed+to+smile%22&view=theater}{Hawthorne, \textit{Image} 46}) % "The" and "Face" both capitalized, as in the book
\ex
he mentioned not a word\hfill(\href{https://archive.org/details/returnofthenativ00harduoft/page/270/mode/2up?q=%22he+mentioned+not%22&view=theater}{Hardy, \textit{Return} 356})
\ex
she said not a word about their interview\hfill(\href{https://archive.org/details/grandbabylonhote00bennuoft/page/82/mode/2up?view=theater&q=%22said+not+a+word%22}{Bennett, \textit{Babylon} 66}) % OJ has "that", but the book linked to has "their"
\ex
he lost not an hour in breaking entirely with the murderer\\\hfill(\href{https://archive.org/details/returnofsherlock0000acon/page/152/mode/2up?view=theater&q=%22lost+not+an+hour%22}{Doyle, \textit{Return} 5.230}) % OJ omits "entirely"
\z
\z\il{English!not a/one@\textit{not a}/\textit{one}|)}

\il{English!not the least\textit{/}the slightest@\textit{not the least}/\textit{the slightest}|(}
In a similar way \textit{not} is attracted to \textit{the least}, \textit{the slightest}, and in recent usage \textit{at all}, as shown by the absence of the auxiliary \textit{do} \refp{ex:05-65}. Cf. \refp{ex:05-71}.

\ea \label{ex:05-65}
\ea
his Majesty took not the least notice of us\hfill(\href{https://archive.org/details/bim_eighteenth-century_the-works-of-j-s-dd-_swift-jonathan_1735_3/page/200/mode/2up?view=theater&q=%22Maje%C5%BFty+took+not+the+lea%C5%BFt+Notice%22}{J. Swift, \textit{Travels} 200}) % "Notice" capitalized in the original
\ex
my resignation of the wardenship need offer not the slightest bar to its occupation by another person\hfill(\href{https://archive.org/details/warden0000anth_w6p5/page/228/mode/2up?q=%22need+offer+not%22&view=theater}{Trollope, \textit{Warden} 243})
\ex
He rested but two hours and slept not at all\hfill(\href{https://archive.org/details/themother00phil/page/350/mode/2up?q=%22but+two+hours%22&view=theater}{Phillpotts, \textit{Mother} 350}) % Starts the sentence
\ex
an urgency that helped him not at all\hfill(\href{https://archive.org/details/loveandmrlewisha00welluoft/page/64/mode/2up?view=theater&q=%22urgency+that+helped%22}{Wells, \textit{Love} 65})

\ex
this explanation enlightened the Commandant not at all\\\hfill(\href{https://archive.org/details/majorvigoureux00quil/page/58/mode/2up?q=%22enlightened%22&view=theater}{Quiller-Couch, \textit{Major} 59})
\il{English!not the least\textit{/}the slightest@\textit{not the least}/\textit{the slightest}|)}
\ex
they talked not at all for a long time\hfill(\href{https://archive.org/details/freelands00galsrich/page/184/mode/2up?q=%22not+at+all+for+a+long+time%22&view=theater}{Galsworthy, \textit{Freelands} 209})
\z
\z

\ea \label{ex:05-71}
he {\dots} cared not the snap of one of his thin, yellow fingers\hfill(\href{https://archive.org/details/freelands00galsrich/page/362/mode/2up?q=%22cared+not+the+snap%22&view=theater}{ibid 415}) % OJ deleted " had done it hundreds of times before and"
\z

\is{auxiliary verbs|(}
Where we have a verb connected with an infinitive, it is often of great importance whether the negation refers to the nexus (main verb) or to the infinitive. In the earlier stages of the language, this was not always clear: \textit{he tried not to look that way} was ambiguous; now the introduction of \il{English!do@\textit{do}|(}\textit{do} as the auxiliary of a negative nexus has rendered a differentiation possible: \textit{he did not try to look that way}; \textit{he tried not to look that way}; and the (not yet recognized) placing of \textit{not} after \textit{to} serves to make the latter sentence even more unambiguous: \textit{he tried to not look that way}. The distinction is clear in \refp{ex:05-72}.

\ea \label{ex:05-72}
She \textit{did not wish} to reflect; she strongly \textit{wished not to} reflect\\\hfill(\href{https://archive.org/details/cu31924013586940/page/470/mode/2up?q=%22did+not+wish+to+reflect%22&view=theater}{Bennett, \textit{Wives} 2.187})
\z\il{English!do@\textit{do}|)}\is{auxiliary verbs|)}

Other examples with \textit{not} belonging to an infinitive: \refp{ex:05-73}--\refp{ex:05-73c}.

\ea \label{ex:05-73}
\ea
\textit{Try not to do} it again\hfill(\href{https://archive.org/details/personalhistory05dickgoog/page/n53/mode/2up?q=%22try+not+to+do+it%22&view=theater}{Dickens, \textit{David} 112})
\ex
\textit{Try not to associate} bodily defects with mental\hfill(\href{https://archive.org/details/personalhistory05dickgoog/page/n191/mode/2up?q=%22bodily%22&view=theater}{ibid 432})
\ex
the more he \textit{endeavoured not to think}, the more he thought\\\hfill(\href{https://archive.org/details/christmascarol0000char_h5c8/page/40/mode/2up?q=%22endeavored%22&view=theater}{Dickens, \textit{Carol} 20}) % Edition linked to has "endeavored"; I (PE) presume that this was exclusively for the US market.
\ex
the fool {\dots} who \textit{resolved not to go} into the water till he had learnt to swim\hfill(\href{https://archive.org/details/essaysonmiltona05macagoog/page/n128/mode/2up?view=theater&q=resolved}{Macaulay, \textit{Milton} 1.41}) % "in the old story" has been cut
\ex
Tommy \textit{deserved not to be} hated\hfill(\href{https://archive.org/details/intrusionspeggy02hopegoog/page/n46/mode/2up?q=%22deserved+not+to+be%22&view=theater}{Hope, \textit{Intrusions} 38})
\ex
if one were to live always among those bright colours, one would \textit{get not to see} them\hfill(\href{https://archive.org/details/strangeadventure00blac/page/58/mode/2up?view=theater&q=%22among+those+bright+colors%22}{Black, \textit{Phaeton} 61}) % The edition linked to actually spells it "colors", but presumably British editions spell it "colours".
\ex
I soon \textit{got not to care}\hfill(\href{https://archive.org/details/in.ernet.dli.2015.260707/page/n79/mode/2up?q=%22soon+got+not+to+care%22&view=theater}{Galsworthy, \textit{Justice} 91})
\ex
I may \textit{come not to feel} such unbearable shame as I do now\\\hfill(\href{https://archive.org/details/lovescrosscurren00swinuoft/page/142/mode/2up?q=%22may+come+not+to+feel%22&view=theater}{Swinburne, \textit{Cross-currents} 158})
\ex
I knew he'd \textit{come not to care} about the book-selling\\\hfill(\href{https://archive.org/details/historydavidgri02wardgoog/page/458/mode/2up?q=%22come+not+to+care%22&view=theater}{Ward, \textit{David} 3.132})
\z
\z

\ea \label{ex:05-73a}
\ea
I beseech you before you go, not perhaps to return, once more to let me press the hand\hfill(\href{https://archive.org/details/vanityfairanove03thacgoog/page/n129/mode/2up?q=%22beseech+you+before%22&view=theater}{Thackeray, \textit{Vanity} 200})
\ex
the Prime Minister himself was personally too much absorbed in the zeal of his cause not sometimes to run counter to the feelings {\dots} of men less earnest\hfill(\href{https://books.google.co.jp/books?id=9PjnkXA5jOkC&pg=PP5&dq=%22justin+mccarthy%22+%22our+own+times%22&hl=en&newbks=1&newbks_redir=0&sa=X&ved=2ahUKEwiE2YnrmaCFAxVsoa8BHcesDZwQ6AF6BAgKEAI#v=onepage&q=%22much%20absorbed%22&f=false}{McCarthy, \textit{History} 2.521}) % Replaced an omission with dots, restored "personally" and "less earnest", etc. According to OJ, MacCarthy [sic] writes: "the Prime-minister was too much absorbed in the zeal of his cause not sometimes to run counter to the feelings of men".
\z
\z

\ea \label{ex:05-73b}
I wished to not treat you to more tears\hfill(\href{https://digital.library.upenn.edu/women/carlyle/jwclam/lam301.html#24}{J. Carlyle, \textit{Letters} 3.24})
\z

\ea \label{ex:05-73c}
``I might not have gone,'' I mused. ``I might easily not have gone.''\\\hfill(\href{https://archive.org/details/dollydialogues00hope_0/page/152/mode/2up?view=theater&q=%22might+not+have+gone%22}{Hope, \textit{Dialogues} 94}; cf. p.~\pageref{p:48} \hyperref[p:48]{above} % PE: Let's add page number
%Brett: go for it  % PE: Done
and p.~\pageref{must-and-may}ff (\chapref{ch:8}) below) % PE: Restoring "I mused". OJ merely says chapter VIII, not the page number(s) within this.
\z

When \textit{do} cannot be used, it is not always easy to see whether \textit{not} belongs to the main verb or the infinitive, as in \refp{ex:05-86},\footnote{Jespersen's Addenda include the example \textit{Sylvia was determined \textsc{not to be} disappointed} (\href{https://archive.org/details/runningwater00masouoft/page/114/mode/2up?view=theater}{Mason, \textit{Water} 104}). \eds} % Want to insert \href{https://archive.org/details/runningwater00masouoft/page/114/mode/2up?view=theater&q=%22Sylvia+was+determined%22}{Mason, \textit{Water} 104} within that pair of parentheses, but doing so triggers errors
%Brett: fixed by removing text search/highlighting. % Peter: Strange! I must try to remember this. 
where, however, the next line shows that what is meant is `it was not my purpose to have seen you here', and not `it was my purpose not to have {\dots}'. This paraphrase further serves to show that in some cases word-order may remove any doubt as to the belonging of the negative, thus very often with a predicative; cf. also such frequent cases as \refp{ex:05-87}. And in the spoken language the use of \textit{wasn't} [wɔznt] in one case, and unstressed \textit{was} [wəz] followed by a strongly stressed \textit{not} in the other, will at once make the meaning clear of such sentences as the one first quoted here.\is{stress!effect on negation of}

\ea \label{ex:05-86}
My purpose was not to haue seene you heere\\\hfill(\href{https://internetshakespeare.uvic.ca/doc/MV_F1/scene/3.2/index.html#tln-1575}{Shakespeare, \textit{Merch} 3.2.230}) % "seene" with three E s
\z

\ea \label{ex:05-87}
He was beginning not to despise the day of small things\\\hfill(\href{https://archive.org/details/septimus00unkngoog/page/n221/mode/2up?q=%22was+beginning+not%22&view=theater}{Locke, \textit{Septimus} 232})
\z

\textit{Don't let us} is the idiomatic expression, where logically it would be preferable to say \textit{let us} with \textit{not} to the infinitive (an injunction not to {\dots}): \refp{ex:05-88}.

\ea \label{ex:05-88}
Do not let us, however, be too prodigal of our pity upon Pegasus\\\hfill(\href{https://archive.org/details/dli.ministry.14127/page/343/mode/2up?q=%22be+too+prodigal%22&view=theater}{Thackeray, \textit{Pendennis} 2.213}) % "upon Pegasus" restored
\z

In the old construction without \textit{do} we see the same attraction of \textit{not} to \textit{let}: (\ref{ex:05-89}, though the last two quotations show \textit{not} placed with the infinitive).

\ea \label{ex:05-89}
\ea
let not vs rent it\hfill(\href{https://www.kingjamesbibleonline.org/1611_John-19-24/}{AV \textit{John} 19.24})
\ex
let not my behaviour seem rude\hfill(\href{https://archive.org/details/bim_eighteenth-century_epicne-or-the-silent-_jonson-ben_1776/page/n29/mode/2up?q=%22behaviour%22&view=theater}{Jonson, \textit{Epicœne} 3.183})
\ex
let not the prospect of worldly lucre carry us beyond your judgment\\\hfill(\href{https://archive.org/details/in.ernet.dli.2015.219151/page/n195/mode/2up?q=%22worldly+lucre%22&view=theater}{Congreve, \textit{Love} 255}) % Original is bristling with capitals
\ex

And let not those Londoners whose eyes have been accustomed to {\dots} suppose that {\dots}\hfill(\href{https://archive.org/details/lifeadventuresofdickrich/page/466/mode/2up?q=%22those+Londoners+whose%22&view=theater}{Dickens, \textit{Nicholas} 443}) % "Londoners" restored
\ex
let not another dare suspect it\hfill(\href{https://archive.org/details/evanharringtonno00mererich/page/194/mode/2up?q=%22dare+suspect%22&view=theater}{Meredith, \textit{Harrington} 219})
\ex
let us not add guilt to our misfortunes\hfill(\href{https://quod.lib.umich.edu/cgi/t/text/pageviewer-idx?c=ecco;cc=ecco;idno=004771299.0001.000;node=004771299.0001.000:6.1;seq=189;page=root;view=text}{Goldsmith, \textit{Good-natur'd} 5})
\ex
let us not imagine evils which we do not feel\hfill(\href{https://archive.org/details/historyrasselas01johngoog/page/n105/mode/2up?q=%22let+us+not+imagine%22&view=theater}{Johnson, \textit{Rasselas} 101}) % "Evil" corrected to "evils"
\z
\z

While now \textit{not} is always in natural language placed before the infinitive it belongs to, there is a poetic or archaic way of placing it after the infinitive, as in \refp{ex:05-96}.
\pagebreak

\ea \label{ex:05-96}
\ea
one object which you might pass by, Might see and \textit{notice not}\\\hfill(\href{https://archive.org/details/poemsofwilliamwo00wor/page/50/mode/2up?q=%22one+object+which+you+might+pass+by%22&view=theater}{Wordsworth, \textit{Michael}})
\ex
a continuance of enduring thought. Which then I can \textit{resist not}\\\hfill(\href{https://archive.org/details/manfreddramaticp06byro/page/n11/mode/2up?q=continuance&view=theater}{Byron, \textit{Manfred} 1.1})
\ex
God bless you, my son, {\dots} and when he smiles on you, may the frown of man \textit{affect you not}!\hfill(\href{https://archive.org/details/christianstory00cainrich/page/70/mode/2up?q=%22frown%22&view=theater}{Caine, \textit{Christian} 69}) % Little changes to accord with the printed book
\z
\z\il{English!not@\textit{not}|)}

\is{scope of negation!resolving ambiguity of|(}
In other languages, difficulties like those mentioned in English are obviated in different ways. Thus in Greek \textit{mē} is used to negative an infinitive, while \textit{ou} is used with a finite verb. In Danish, a certain number of combinations like \textit{jeg beklager ikke at kunne hjælpe Dem} (`I am sorry I cannot help you') may be ambiguous, though less so in the spoken than in the printed form; but in some instances the colloquial use of a preposition shows where \textit{ikke} belongs; instead of the literary \textit{prøv ikke at se derhen} (`try not to look over there') it is usual to say either \textit{prøv ikke på at se derhen} (`don't try to look over there') or \textit{prøv på ikke at se derhen}. There is another colloquial way out of the difficulty, by means of the verbal phrase \textit{lade være} or rather \textit{la vær} (literally `leave be'): % ??? PE: What's in single quotes is expected to be a more or less idiomatic translation; but a problem for me is that I don't understand "leave be" in these examples. "Leave her be", OK; ?"Leave her be to look", pretty strange; *"Leave be to look", ungrammatical (for me).
%% SG: It's a conventionalized expression: lad være at se derhen, lit. 'leave be to look over there' (= 'do not look over there')
\textit{prøv at} (\textit{å}) \textit{la vær at} (\textit{å}) \textit{se derhen} (`Avoid looking over there'). Thus also \textit{du skal lade vær at se derhen} (`You must refrain from looking over there'), different from \textit{du skal ikke se derhen} (`You must not look over there').

In Latin, the place of \textit{non} before the main verb or before the infinitive will generally suffice to make the meaning clear. Similarly in French: \textit{il ne tâche pas de regarder} (`he doesn't try to look'); \textit{il tâche de ne pas regarder} (`he tries not to look'); \textit{il ne peut pas entendre} (`he cannot hear'); \textit{il peut ne pas entendre} (`he can not hear (as he chooses)') --- whence the possibility of saying \textit{non potest non amare} (`he cannot not love'); \textit{il ne peut pas ne pas aimer} corresponding to Danish \textit{han kan ikke lade være at elske}, English \textit{he cannot but love}, \textit{cannot help loving} (\textit{cannot choose but love}). Cf. \chapref{ch:8} below. % PE: OJ writes "Cf. below ch. VIII.". We could insert a comma before "chapter 8" to show that it's in apposition, but I've switched the order instead. Feel free to revert.
\is{scope of negation!resolving ambiguity of|)}

\label{para:not-to-sing}In this connexion, I must mention an interesting phenomenon frequent in Russian; I take my examples from Holger \citet[12]{pedersen1916russisk}; \textit{a pět' už ne stal} `but sing now he not began' % Peter: The bunch of English translations in this paragraph are OJ's own, and putting them into ( ) would complicate matters.
% PE: I have now (Sept '04) rather changed my mind about this. I'm putting ‘must/should’ in ( ).
which is explained as standing for the logical `not-to-sing he began', i.e. `he ceased to sing'; \textit{ne vélěno étogo dělat'} `order is not given to do this' instead of the logical `order is given not to do this', i.e. `it is prohibited to do this'. Similarly with \textit{dolžen} (`must/should'). But how comes it that the negative \textit{ne} is in such expressions attached to the wrong word? There is another way of viewing these sentences, if we take the negative to mean not the contradictory, but the contrary term: \textit{ne stal} `did the opposite of beginning', i.e. `ceased'; \textit{ne velmo} `the opposite of order, i.e. prohibition, is given'. And in \citet[\href{https://archive.org/details/vergleichendesl00vondgoog/page/399/mode/2up?q=\%22mitunter+wird+der+begriff\%22&view=theater}{400}]{vondrak1908vergleichende}, I find:

\begin{quote}
mitunter wird der Begriff des Verbs nicht durch \textit{ne} aufgehoben, sondern in sein Gegenteil verwandelt: [altkirchenslawisch] nenaviděti ‘hassen’ ([böhmisch] náviděti ‘lieben’), [serbisch] nèstati ‘verschwinden’
 % The abbreviations are Vondrák’s; should we have somebody proficient in German expand them? Also, OJ writes "ins gegenteil" but V writes "in sein Gegenteil" (of course we're following OJ's non-capitalizing)
%Brett: The abbreviations in the text refer to different Slavic languages: aksl. - Altkirchenslavisch (Old Church Slavonic) b. - Bulgarisch (Bulgarian) s. - Serbisch (Serbian) % Peter: Ah yes, obvious now that I think of it. I've inserted "Bulgarian". But hang on -- isn't it a little odd to bring up Bulgarian? If "b." were also the abbreviation of a word meaning roughly "derived from", it would be a lot less surprising. 
%Brett: The abbreviation "b." does indeed appear to stand for "bei" or possibly "beziehungsweise" rather than "Bulgarisch" (Bulgarian). This is evident from the context: "aksl. nenaviditi 'hassen' (b. náviditi 'lieben')" Here, it's showing the contrast between the negated form (nenaviditi - to hate) and its positive counterpart (náviditi - to love).
% Peter: Accordingly, I've deleted "Bulgarian" from "Bulgarian náviděti". (However, I haven't provided any alternative rendering of "b.".)

`sometimes the concept of the verb is not negatived by \textit{ne}, but transformed into its opposite: in Old Church Slavonic, \textit{nenaviděti} means ``to hate'' (Czech náviděti ``to love''),, and in Serbian, \textit{nèstati} means ``to disappear''' % PE: I have (unenthusiastically) changed "negated" to "negatived".
\end{quote}

This closely resembles a Greek idiom, see:

\begin{quote}
Einzelne Begriffe werden besonders durch \textit{ou} aufgehoben, ja zuweilen ins Gegenteil verwandelt, wie \textit{oú phēmi} nego, verneine {\dots} \textit{ouk axiô} verlange dass nicht, \textit{ouk eô} veto, verwehre, widerrate (auch erlaube nicht)

`Individual concepts are especially negatived by \textit{ou}, and sometimes even transformed into the opposite, as in \textit{oú phēmi} (\textit{nego} in Latin, `I deny') {\dots} \textit{ouk axiô} (`I do not deem worthy'), \textit{ouk eô} (\textit{veto} in Latin, `I forbid, I refuse, I advise against'---also `I do not permit')'\hfill\citep[\href{https://archive.org/details/griechischesprac00kr/page/296/mode/2up?view=theater}{§67 1.a.2}]{kruger1875griechische}
% PE: How about: `Individual concepts are especially negated by \textit{ou}, and sometimes even transformed into the opposite, as in \textit{oú phēmi} (\textit{nego} (Latin), I deny) {\dots} \textit{ouk axiô} (I demand not), \textit{ouk eô} (\textit{veto} (Latin), I forbid, I refuse, I advise against (also I do not permit))'  How we deal with the Latin here should probably affect how we deal with it in the next quotation.
%Brett: Agreed. Done
%PE: I have (unenthusiastically) changed "negated" to "negatived".
\end{quote}
\is{litotes}
\begin{quote}
Eine ähnliche Litotes liegt vor, wenn \textit{phēmí} die Negation an sich zieht, die logisch richtiger beim abhängigen Infinitive stehen würde: \textit{oú phēmi toûto kalôs ékhein} nego hoc bene se habere % restored "Eine ähnliche"

\emergencystretch=3em
`A similar litotes occurs when \textit{phēmí} attracts the negation to itself, which logically would more correctly stand with the dependent infinitive: \textit{oú phēmi toûto kalôs ékhein} (\textit{nego hoc bene se habere} in Latin, `I deny that this is in a good state')'\phantom{.}\hfill\citep[\href{https://archive.org/details/p2ausfhrlichegra02khuoft/page/180/mode/2up?q=litotes&view=theater}{180}]{kuhner1904ausfuhrliche}
\end{quote}

\noindent This is explained as change into the contrary:

\begin{quote}
\textit{\emph{ouk eô} prohibeo {\dots} \emph{ou stérgō} odi {\dots} \emph{ou sumbouleúō} dissuadeo}

\emergencystretch=3em
`I prohibit, I hate, I dissuade'\hfill \citep[\href{https://archive.org/details/p2ausfhrlichegra02khuoft/page/182/mode/2up?q=litotes&view=theater}{182}]{kuhner1904ausfuhrliche}
\end{quote}

\is{position of negative|(}
\is{raising of negation|(}
As an ``accusative with an infinitive'' can be considered as a kind of dependent clause, the mention of Latin \textit{nego Gaium venisse} (`I say that Gaius has not come') naturally leads us to the strong tendency found in many languages to attract to the main verb a negative which should logically belong to the dependent nexus. In many cases, \textit{I don't think he has come} and similar sentences really mean `I think he has not come'; though \textit{I hope} (\textit{expect})\textit{ he won't come} is more usual than the less logical \textit{I do not hope} (\textit{expect})\textit{ he will come}, which is usual in Danish and German, and also, according to \citet[\href{https://archive.org/details/englishaswespeak00joycuoft/page/19/mode/2up?view=theater&q=\%22So+in+our+modern+speech\%22}{20}]{joyce1910english} among the Irish, who will say, e.g. \textit{It is not my wish that you should go to America at all}, by which is meant the positive assertion: `It is my wish that you should not go', --- as well as \textit{I didn't pretend to understand what he said} for `I pretended not to understand'.
\is{attraction!of negative to verb}

A few Scandinavian examples may be given of this tendency to insert the negative in the main sentence: \refp{ex:05-99}.

\ea \label{ex:05-99}
\ea
\gll saa vil jeg \textit{aldrig} ønske, at du maa blive gift\\
 then will I never wish that you may become married\\
\glt `{\dots} then I wish that you will never get married'\\\hfill(\href{https://tekster.kb.dk/text/adl-texts-hostrup01-shoot-workid54989#idm140699400034704}{Hostrup, \textit{Gjenboerne} 3.6}) % &&
%Brett: What's this? % Peter: I've no idea. Just a slip of the fingers, perhaps.
\ex
\gll Jeg tror \textit{ikke}, at mange har læst Brand og at færre har forstaaet den\\
 I believe not that many have read Brand and that fewer have understood it\\
\glt `I believe that not many have read Brand and that fewer have understood it'\hfill(Schandorff, news 1897; note here the continuation,\\\hfill which shows that what is meant is: \textit{tror at ikke mange {\dots}}) % It's not even clear to me (PE) which Schandorff this is.
\is{quantifiers!negatived}

\ex
\gll Men det lot {'o [= hun]} ikke, som 'o hørte\\
 but that pretended she not {as if} she heard\\
\glt `But she acted as if she hadn't heard that'\hfill(\href{https://www.nb.no/items/URN:NBN:no-nb_digibok_2008072810004?page=31&searchText=%22Men%20det%20lot%22}{Bjørnson, \textit{Guds} 21})
\ex
\gll Han trodde \textit{icke} presterna voro annat än examinerade studenter \textit{och} att deras besvärjelseord bara var mytologi\\
 he believed not priests.\DEF{} were other than examined students and that their incantations just were mythology\\
\glt `He believed that priests were nothing but graduated students, and that their incantations were mere mythology'\\\hfill(\href{https://litteraturbanken.se/f%C3%B6rfattare/StrindbergA/titlar/GiftasII1886/sida/134/faksimil}{Strindberg, \textit{Giftas} 2.134}; note also here the positive continuation) 
\z
\z

Compare from French \refp{ex:05-103}.

\ea \label{ex:05-103}
\gll il ne faut pas que tu meures\\
 it not need not that you die\\
\glt `you must not die'\hfill(\href{https://archive.org/details/vermischtebeitr04toblgoog/page/n197/mode/2up?view=theater&q=%22faut+pas+que+tu+meures%22}{Tobler, \textit{Beiträge} 1.164})
\z
\is{raising of negation|)}

\il{English!not@\textit{not}|(}
In English, we must note the distinction between \textit{I don't suppose} (\textit{I am not afraid}), where the main nexus is negatived, and \textit{I suppose not} (\textit{I am afraid not}) where the nexus is positive, but the object (a whole sentence understood) is negative; how old is this use of \textit{not} for a whole sentence? Examples: \refp{ex:05-104}.

\ea \label{ex:05-104}
\ea I'm afraid not\hfill(\href{https://archive.org/details/in.ernet.dli.2015.219151/page/n171/mode/2up?q=%22afraid+not%22&view=theater}{Congreve, \textit{Love} 121})
\ex `` {\dots} whether it ever came to my knowledge until this moment?'' --- ``I believe not directly'' {\dots}. --- ``Why, you know not''\hfill(\href{https://archive.org/details/personalhistory05dickgoog/page/n47/mode/2up?q=%22whether+it+ever%22&view=theater}{Dickens, \textit{David} 93}) % "until this moment" restored; CD's "Why" instead of OJ's "Well". (Incidentally, the original three paragraphs have other grammatical interest too.)
\ex ``I am afraid you can't learn it, my poor fellow.'' --- ``I am afraid not''\\\hfill(\href{https://archive.org/details/lifeadventuresofdickrich/page/330/mode/2up?q=%22you+can%27t+learn+it%22&view=theater}{Dickens, \textit{Nicholas} 311}) % "my poor fellow" restored
\ex ``can you bear the thought of that?'' --- {\dots} ``I should imagine not, indeed!''\hfill(\href{https://archive.org/details/lifeadventuresofdickrich/page/616/mode/2up?q=%22bear+the+thought%22&view=theater}{ibid 590}) % OJ mangles three paragraphs into one; I (PE) have replaced the second with dots and separated the other two.
\ex ``I should not mind'' {\dots}. --- ``I daresay not, because you have nothing particular to say.'' {\dots} --- ``But I have something particular to say.'' --- ``I hope not.'' --- ``Why should you hope not?''\hfill(\href{https://archive.org/details/dukeschildrennov00troluoft/page/194/mode/2up?q=%22I+should+not+mind%22&view=theater}{Trollope, \textit{Children} 2.81}) % repunctuated
\ex ``I'll tell the boys and they'll draw you like a badger.'' --- ``Please not, old man.''\hfill(\href{https://archive.org/details/lightthatfailed0000rudy_q6g8/page/228/mode/2up?q=%22please+not%2C+old+man%22&view=theater}{Kipling, \textit{Light} 217}) % "and they'll draw you like a badger" restored
\ex I believe I asked him to hold his tongue about them---he says not.\\\hfill(\href{https://babel.hathitrust.org/cgi/pt?id=hvd.hnpei8&seq=9&q1=hold+his+tongue}{Conway, \textit{Called} 1}) % "about them" restored. (This is a single sentence, spoken by one person.) 
\z
\z
\is{position of negative|)}

\is{adverbs!negative}
\is{proforms|(}
\is{subordinator, negative|(}
\label{not_that}Inversely, we have a negative adverb standing for a whole main sentence, \il{English!not that@\textit{not that}|(}\textit{not that} meaning `I do not say that' or `the reason is not that' as in \refp{ex:05-111}. We shall see in \chapref{ch:12} the use of \il{English!not but@\textit{not but}}\textit{not but} (\textit{that}) and \textit{not but what} in the same sense.

\ea \label{ex:05-111}
\ea Not that I lou'd Cæsar lesse, but that I lou'd Rome more\\\hfill(\href{https://internetshakespeare.uvic.ca/doc/JC_F1/scene/3.2/index.html#tln-1550}{Shakespeare, \textit{Cæs} 3.2.22}) % The uvic.ca page has not "Cæsar" but instead "Caesar". Is it simplifying, or is OJ indulging another minor eccentricity?
%Brett: the first folio has "Cæsar". We can link to it, but it's hard to search and hard to read https://archive.org/details/mrvvilliamshakes00shak/page/120/mode/2up?q=%22that+Friend+demand%2C+why+Brutus+role+againft+Ca%7E+far%2C+this+is+my+anfwer+%3A%22&view=theater % Peter: ⟨æ⟩ is a pain, but as it's spattered across the Danish that's so prominent in this book, and as changing those examples back to ⟨aa⟩ would be most anachronistic, let's tolerate it in English as well.
\ex Not that the heart can be good without knowledge\\\hfill(\href{https://archive.org/details/bunyanspilgrims00moffgoog/page/108/mode/2up?q=%22heart+can+be+good%22&view=theater}{Bunyan, \textit{Progress} 113})
\ex Not that I agree with everything I have said in this essay\\\hfill(\href{https://archive.org/details/intentions01wild/page/256/mode/2up?q=%22agree%22&view=theater}{Wilde, \textit{Intentions} 212})
\ex Not that he had forgotten them\hfill(\href{https://archive.org/details/wonderfulyear00lockuoft/page/326/mode/2up?q=%22not+that+he+had+forgotten%22&view=theater}{Locke, \textit{Year} 309})
\z
\z\il{English!not that@\textit{not that}|)}\il{English!not@\textit{not}|)}

In other languages correspondingly: \refp{ex:05-115}.

\ea \label{ex:05-115}
\ea
\gll Ikke at han havde (or: skulde ha) glemt dem\\
 not that he had {} should have forgotten them\\
\glt `Not that he had forgotten them'
\ex
\gll nicht dass er sie vergessen hätte\\
 not that he them forgotten would have\\
\glt `not that he would have forgotten them'
\ex
\gll Non pas qu'il parlât à personne\\
 not not {that he} spoke to anyone\\
\glt `Not that he spoke to anyone'\hfill(\href{https://www.gutenberg.org/cache/epub/61876/pg61876-images.html}{Rolland, \textit{Foire} 306})
\z
\z
\is{subordinator, negative|)}

\il{English!not@\textit{not}|(}
When we say (``He'll come back'') \textit{Not he!} % PE: I think what OJ is saying is "When we say (in response to 'He'll come back') 'Not he!', it is not really...." And so he deliberately puts "He'll come back" in quotation marks, which I've restored.
it is not really \textit{he} that is negatived, but the nexus, although the predicative part of it is unexpressed; the exclamation is a complete equivalent of \textit{He won't!} (with stress on \textit{won't}): \refp{ex:05-118}.

\ea \label{ex:05-118}
\ea Who, I rob? I a theefe? Not I.\hfill(\href{https://internetshakespeare.uvic.ca/doc/1H4_F1/scene/1.2/index.html#tln-240}{Shakespeare, \textit{H4A} 1.2.153}) 
\ex Please not, old man.\hfill(\href{https://archive.org/details/lightthatfailed0000rudy_q6g8/page/228/mode/2up?q=%22please+not%2C+old+man%22&view=theater}{Kipling, \textit{Light} 217})
\ex Were I a Steam-engine, wouldst thou take the trouble to tell lies of me? Not thou!\hfill(\href{https://archive.org/details/sartorresartus02unkngoog/page/222/mode/2up?view=theater&q=%22were+I+a+steam-engine%22}{T. Carlyle, \textit{Sartor} 169})
\ex Meg don't know what he likes. Not she!\hfill(\href{https://archive.org/details/chimes00dick/page/58/mode/2up?q=%22meg+don%27t+know+what%22&view=theater}{Dickens, \textit{Chimes} 30})
\ex They wouldn't have touched \textit{us} {\dots} Not they!\\\hfill(\href{https://archive.org/details/freelands00galsrich/page/222/mode/2up?q=%22not+they%21%22&view=theater}{Galsworthy, \textit{Freelands} 255}) % OJ has "wouldn't touch", but JG writes "wouldn't have touched"
\ex ``it'll perhaps rain cats and dogs to-morrow {\dots}'' {\dots} --- ``Not \textit{it}''\\\hfill(\href{https://archive.org/details/silasmarnerbygeo00elio/page/30/mode/2up?q=%22cats+and+dogs%22&view=theater}{Eliot, \textit{Silas} 44})
\ex ``Do you think it will last long?'' --- ``Not it!''\hfill(\href{https://archive.org/details/cu31924013586940/page/238/mode/2up?q=%22will+last+long%22&view=theater}{Bennett, \textit{Wives} 1.263})
\ex ``Bit late now, isn't it?'' --- ``Not it.''\hfill(\href{https://archive.org/details/cardstoryofadven00bennuoft/page/244/mode/2up?view=theater&q=%22bit+late+now%22}{Bennett, \textit{Card} 244}) % OJ merely says "Bennett Cd. 244": no example, no indication of what "Cd." is, etc.
\ex All sorts of things accumulate, sir.{\dots} Not \textit{you}, of course, in particular.\\\hfill(\href{https://archive.org/details/in.ernet.dli.2015.475865/page/n129/mode/2up?q=%22Not+you%2C+of+course%2C+in+particular.%22&view=theater}{Wells, \textit{Stories} 49}) % (i) Let's try to find a better copy than this one. (ii) OJ doesn't specify the example, only the page number.
%Brett: Is this a better copy? % Peter: It's not as bad a copy. I've resuscitated the emphasis on "you".
\z
\z
\is{proforms|)}

The following examples \refp{ex:05-127} show the accusative used as a modern (vulgar or half-vulgar) ``disjointed'' nominative.

\ea \label{ex:05-127}
\ea
We sha'n't hang up on any misunderstanding. Not us.\\\hfill(\href{https://archive.org/details/annveronicamoder0000hgwe/page/360/mode/2up?q=%22any+misunderstanding%22&view=theater}{Wells, \textit{Veronica} 338}) % OJ has "shan't hang upon", but that isn't what the Harper & Brothers 1909 edition says.
\ex ``you were all in the same room together, were not you?''\\\il{English!no@\textit{no}}``No, indeed, not us.''\hfill(\href{https://archive.org/details/sensesensibility00austrich/page/242/mode/2up?q=%22all+in+the+same+room%22&view=theater}{Austen, \textit{Sense} 269}) % PE: Adjusted JA's punctuation to accord with the edition linked to.
\z
\z\il{English!not@\textit{not}|)}

In Old English we have the corresponding \il{English!Old English!nic@\textit{nic}}\textit{nic} \refp{ex:05-129}. \textit{Nic} is spelt \textit{nîc} and \textit{nyc} (\href{https://archive.org/details/holygospelsinan01skeagoog/page/n75/mode/2up?view=theater&q=nyc}{\textit{John}, ed. Skeat, 1.21}), spelt \textit{nicc} and \textit{nicht} (\href{https://archive.org/details/holygospelsinan01skeagoog/page/n365/mode/2up?view=theater&q=nicht}{ibid 18.17}). This (with the positive counterpart \textit{I}, which is probably the origin of \textit{ay} (`yes'), and \textit{ye we} in \refp{ex:05-130}) closely resembles the French \textit{naje} `not I' (in the third person \textit{nenil}) and the positive \textit{oje} `hoc ego' (in the third person \textit{oïl, oui}), see Tobler (\citeyear[\href{https://www.jstor.org/stable/pdf/40845615.pdf}{423}]{tobler1877franzosische}; \citeyear[\href{https://archive.org/details/vermischtebeitr04toblgoog/page/n17/mode/2up?view=theater}{1}]{tobler1886vermischte}); \citet[\href{https://www.jstor.org/stable/45042774?seq=3}{465}]{paris1878periodiques}.

\ea \label{ex:05-129}
\gll Wilt þu fon sumne hwæl? --- Nic\\
 want you catch some whale {} not I\\
\glt `Do you want to catch a whale? --- No.'\hfill(\href{https://archive.org/details/anglosaxonoldeng01wriguoft/page/51/mode/2up?view=theater&q=%22wilt%22}{Wright \& Wülcker 1.94})
\z\il{English!Old English!nic@\textit{nic}|)}
%% SG: More idiomatically, the reply just translates to "no". It's a dedicated first-person negation
%BR: done

\ea \label{ex:05-130}
wille ye doo this {\dots} --- {\dots} ye we, lorde\hfill(\href{https://archive.org/details/TheHistoryOfReynardTheFoxArber/page/n87/mode/2up?q=%22wille+ye+doo+this%22&view=theater}{Caxton, \textit{Reynard} 58})
\z
\is{nexal negation!and special negation|)}
\is{scope of negation|)}
\is{special negation!and nexal negation|)}

% CHAPTER 6 ENCLITICS
\chapter{Enclitics}\label{ch:enclitics}
This chapter covers the enclitic\index[sub]{enclitic} suffixes of Southern Yauyos Quechua. In \SYQ, as in other Quechuan languages, enclitics attach to both nouns and verbs as well as to adverbs and negators. Enclitics always follow all inflectional suffixes, verbal and nominal; and, with the exception of restrictive \phono{-lla}, all follow all case suffixes, as well. \SYQ{} counts sixteen enclitics. \phono{-Yá} (emphatic) indicates emphasis. Consistently translated in Spanish by \textit{pues}.\footnote{An anonymous reviewer points out that \it{pues} is used in Andean Spanish “to negotiate common ground, shared knowledge. As such, it is possible that \phono{-ya} is also an interactional or stance marker,” a way a participant in a conversation may negotiate what other participants know or should know.} \phono{-chu} (interrogation, negation, disjunction) indicates absolute and disjunctive questions, negation, and disjunction. \phono{-lla} (restrictive) generally indicates exclusivity or limitation in number; it is generally translated as ‘just’ or ‘only’. \phono{-lla} may express an affective or familiar attitude. \phono{-ña} (discontinuitive) indicates transition, change of state or quality. In affirmative statements, it is generally translated as ‘already’; in negative statements, as ‘no more’ or ‘no longer’; in questions, as ‘yet’. \phono{-pis} (inclusion) indicates the inclusion of an item or event into a series of similar items or events; it is generally translated as ‘too’ or ‘also’ or, when negated, ‘neither’. \phono{-puni} (certainty, precision); it is generally translated ‘necessarily’, ‘definitely’, ‘precisely’. This last is attested only in the \QII{} dialects, where it is infrequently employed. \phono{-qa} (topic marker) indicates the topic of the clause; it is generally left untranslated.\footnote{\phono{-qa} may nevertheless be indicated in Spanish translations by intonation, gesture, and various circumlocutions of speech, as an anonymous reviewer points out.}\\
\phono{-raq} (continuative) indicates continuity of action, state or quality. Translated ‘still’ or, negated, ‘yet’. \phono{-taq} (sequential) indicates the sequence of events. In this capacity, translated ‘then’ or ‘so’. \phono{-taq} also marks content questions. \phono{-mI} (evidential~--~direct experience) indicates that the speaker has personal-experience evidence for the proposition under the scope of the evidential. Usually left untranslated.\\
\phono{-shI} (evidential~--~reportative/quotative) indicates that the speaker has non-perso\-nal-experience evidence for the proposition under the scope of the evidential. \phono{-shI} appears systematically in stories. Often translated as ‘they say.’ \phono{-trI} (evidential~--~conjectural) indicates that the speaker is making a conjecture to the proposition under the scope of the evidential from a set of propositions for which she has either direct or not-direct evidence. Generally translated in Spanish as \spanish{seguro} ‘for sure’, indicating possibility or probability. \phono{-ari} (assertive force) indicates conviction on the part of the speaker. Translated as ‘certainly’ or ‘of course’.\footnote{An anonymous reviewer writes that in other varieties of Quechuan, “\phono{-ari} is interpersonal. It expresses solidarity, affirming what someone else says, thinks or believes to be true.”} \phono{-ik} and \phono{-iki} (evidential modifiers) indicate increasing evidence strength (and increased assertive force or conjectural certainty, in the case of the direct and conjectural modifiers, \phono{-mI} and \phono{-trI}, respectively). Generally translated in Spanish as \spanish{pues} and \spanish{seguro}, respectively. Examples in Table~\ref{Tab30} are fully glossed in the corresponding sections.

% TABLE 30
% \newcommand{\tabexefour}[4]{\Qyell{\phono{#1}}&#2&\Qyell{\textit{#3}}&#4\\}%
\begin{table}[!ht]
\renewcommand*\arraystretch{1.3}
\small\centering
\caption{Enclitic suffixes, with examples}\label{Tab30}
\begin{tabularx}{\textwidth}{p{6ex}@{~}p{13ex}@{~}L@{~}L}
\lsptoprule
\tabexefour{-Yá}{emphasis}{¡Mana-\pb{yá} rupa-chi-nchik-chu! ¡Ari-yá!}{‘We do \pb{\emph{not}} set on fire!’ \mbox{‘Yes, indeed!’}}
\tabexefour{-chu\tss{1}}{interrogation}{¿Iskwila-man trura-shu-rqa-nki-\pb{chu} mama-yki?}{‘\pb{Did} your mother put you in school?’}
\tabexefour{-chu\tss{2}}{negation}{Chay-tri \pb{mana} suya-wa-rqa-\pb{chu}.}{‘That must be why she would\pb{n’t} have waited for me.’}
\tabexefour{-chu\tss{3}}{disjunction}{¿Qari-\pb{chu} ka-nki warmi-\pb{chu} ka-nki?}{‘Are you a man \pb{or} a woman?’}
\tabexefour{-lla}{restriction}{Uma-\pb{lla}-ña traki-\pb{lla}-ña ka-ya-sa.}{‘There was \pb{only} the head \pb{only} the hand.’}
\tabexefour{-ña}{discontuity}{Chay-shi ni-n kundinadaw-\pb{ña}-m wak-qa ka-ya-n.}{‘That one, they say, is \pb{already} condemned.’}
\tabexefour{-pis}{inclusion}{Tukuy tuta tusha-n qaynintin-ta-\pb{pis}.}{‘They dance all night and the next day, \pb{too}.’}
\tabexefour{-puni}{certainty}{Mana-\pb{puni}-m.}{‘By no means’, ‘Not on your life’}
\tabexefour{-qa}{topic}{Mana yatra-q-ni-n-\pb{qa}.}{‘Those of them who didn’t know’}
\tabexefour{-raq}{continuity}{Kama-n-pi puñu-ku-ya-pti-n-\pb{raq} tari-ru-n.}{‘He found him \pb{still} sleeping in his bed.’}
\tabexefour{-taq}{sequence}{hinaptin-ña-\pb{taq}-shi}{‘\pb{then}’ ‘so’}
\tabexefour{-mI}{evidential-direct}{Yanga-ña-\pb{m} qipi-ku-sa puri-ni.}{‘In vain, I walk around carrying it.’}
\tabexefour{-shI}{evidential-reportative}{Qari-n-ta-\pb{sh} wañu-ra-chi-n.}{‘She killed her husband, \pb{they say}.’}
\tabexefour{-trI}{evidential-conjecture}{Awa-ya-n-\pb{tr-iki} kama-ta.}{‘He \pb{must} be weaving a blanket.’}
\tabexefour{-ari}{assertive force}{Chay-\pb{sh-ari} kanan avansa-ru-nqa.}{‘That one \pb{definitely} will advance now, \pb{they say}.’}
\tabexefour{-ikI}{evidential \-modification}{Kay-na-lla-\pb{m-iki} kay urqu-pa-qa yatra-nchik.}{‘Just like this we live on this mountain.’}
\lspbottomrule
\end{tabularx}
\end{table}

\section{Sequence}
Combinations of individual enclitics\index[sub]{enclitic!sequence} generally occur in the order indicated in the table below. In complementary distribution are: \phono{-raq} with \phono{-ña}; the evidentials with each other as well as with \phono{-qa}; \phono{-ari} with \phono{-ikI;} and \phono{-Yá} with \phono{-ikI}.

\begin{center}
\small
\begin{tabular}{*{9}{c}}
\lsptoprule
	&	&	&				&	&	& \phono{-qa}	&	&				\\
	&	&	&				&	&	& \phono{-mI}	&	&				\\
	&	&	& \phono{-Raq}	&	&	& \phono{-shI}	&	& \phono{-ikI}	\\
\phono{-lla} & \phono{-puni} & \phono{-pis} & \phono{-ña} & \phono{-taq} & \phono{-chu} & \phono{-trI} & \phono{-Yá} & \phono{-aRi}\\
\lspbottomrule
\end{tabular}
\end{center}

\section{Individual enclitics}\label{sec:indenc}
In \SYQ, as in other Quechuan languages, the enclitics can be divided into two classes: (a)~those which position the utterance with regard to others salient in the discourse (restrictive/limitative \phono{-lla}, discontinuative \phono{-ña}, additive \phono{-pis}, topic marking \phono{-qa}, continuative \phono{-Raq}, sequential \phono{-taq}, and interrogative/negative/disjunctive \phono{-chu}); and (b)~those that position the speaker with regard to the utterance (emphatic \phono{-YÁ}, certainty marker \phono{-puni}, and the evidentials \phono{-mi}, \phono{-shi}, and \phono{-tri} along with their modifiers \phono{-ik}, \phono{-iki}, and \phono{-aRi}.). §~\ref{ssec:emphatic}--\ref{ssec:emotive} cover all enclitics except the evidentials and their modifiers, in alphabetical order. The evidentials and their modifiers are the subject of §~\ref{ssec:evidence}.

\subsection{Emphatic \phono{-Yá}}\label{ssec:emphatic}\index[sub]{emphatic}
Realized as \phono{-yá} in all environments~(\ref{Glo6:Ari}--\ref{Glo6:Sirbisatatr}) except following an evidential, in which case both the \phono{I} of the evidential and the \phono{Y} of the emphatic are elided and \phono{Yá} is realized as \phono{á}~(\ref{Glo6:Balikushatr}--\ref{Glo6:Unayqa}).\\ 

% 1
\gloexe{Glo6:Ari}{}{amv}%
{¡Ari\pb{yá}!}%amv que first line
{\morglo{ari-yá}{yes-\lsc{emph}}}%morpheme+gloss
\glotran{Yes \pb{indeed}.}{}%eng+spa trans
{}{}%rec - time

% 2
\gloexe{Glo6:Mana}{}{amv}%
{¡Mana-\pb{yá} rupa-chi-nchik-chu!}%amv que first line
{\morglo{mana-yá}{no-\lsc{emph}}\morglo{rupa-chi-nchik-chu}{burn-\lsc{caus}-\lsc{1pl}-\lsc{neg}}}%morpheme+gloss
\glotran{We do \emph{\pb{not}} set on fire!}{}%eng+spa trans
{}{}%rec - time

% 3
\gloexe{Glo6:Pantyunpa}{}{amv}%
{Pantyunpa\pb{yá}. ¡Ima wasiypitr pampamushaq!}%amv que first line
{\morglo{pantyun-pa-\pb{yá}}{cemetery-\lsc{loc}-\lsc{emph}}\morglo{ima}{what}\morglo{wasi-y-pi-tr}{house-\lsc{1}-\lsc{loc}-\lsc{evc}}\morglo{pampa-mu-shaq}{bury-\lsc{cisl}-\lsc{1.fut}}}%morpheme+gloss
\glotran{In the cemetery\pb{!} I doubt I’m going to bury someone in my house.}{}%eng+spa trans
{}{}%rec - time

% 4
\gloexe{Glo6:Imayna}{}{amv}%
{¿Imayna\pb{yá} piru paykuna yatran warmi u qari?}%amv que first line
{\morglo{imayna-yá}{how-\lsc{emph}}\morglo{piru}{but}\morglo{pay-kuna}{they-\lsc{pl}}\morglo{yatra-n}{know-\lsc{3}}\morglo{warmi}{woman}\morglo{u}{or}\morglo{qari}{man}}%morpheme+gloss
\glotran{How \pb{ever} can they know if it will be a woman or a man?}{}%eng+spa trans
{}{}%rec - time

% 5
\gloexe{Glo6:Sirbisatatr}{}{amv}%
{Sirbisatatr mas mastaqa rantikurun. Sirbisatayá.}%amv que first line
{\morglo{sirbisa-ta-tr}{beer-\lsc{acc}-\lsc{evc}}\morglo{mas}{more}\morglo{mas-ta-qa}{more-\lsc{acc}-\lsc{top}}\morglo{ranti-ku-ru-n}{buy-\lsc{refl}-\lsc{urgt}-\lsc{3}}\morglo{sirbisa-ta-yá}{beer-\lsc{acc}-\lsc{emph}}}%morpheme+gloss
\glotran{\spkr~1: “They must have sold a lot more beer.” \spkr~2: “Beer, \pb{all right}!”}{}%eng+spa trans
{}{}%rec - time

% 6
\gloexe{Glo6:Balikushatr}{}{lt}%
{Balikushatr kara. Payta\pb{má} rikarani.}%lt que first line
{\morglo{baliku-sha-tr}{request.a.service-\lsc{prf}-\lsc{evc}}\morglo{ka-ra}{be-\lsc{pst}}\morglo{pay-ta-m-á}{he-\lsc{acc}-\lsc{evd}-\lsc{emph}}\morglo{rika-ra-ni}{see-\lsc{pst}-\lsc{1}}}%morpheme+gloss
\glotran{He must have been requested. I saw him.}{}%eng+spa trans
{}{}%rec - time

% 7
\gloexe{Glo6:Trabahayta}{}{ch}%
{Trabahayta kanan kumunalta trulala\pb{má}.}%ch que first line
{\morglo{trabaha-y-ta}{work-\lsc{inf}-\lsc{acc}}\morglo{kanan}{now}\morglo{kumunal-ta}{community-\lsc{acc}}\morglo{trula-la-m-á}{put-\lsc{pst}-\lsc{evd}-\lsc{emph}}}%morpheme+gloss
\glotran{Now he’s put the community to work.}{}%eng+spa trans
{}{}%rec - time

% 8
\gloexe{Glo6:Unayqa}{}{sp}%
{Unayqa Awkichanka inkantakura\pb{shá} wak altupa yantaman riptiki.}%sp que first line
{\morglo{unay-qa}{before-\lsc{top}}\morglo{Awkichanka}{Awkichanka}\morglo{inkanta-ku-ra-sh-á}{enchant-\lsc{refl}-\lsc{pst}-\lsc{evr}-\lsc{emph}}\morglo{wak}{\lsc{dem.d}}\morglo{altu-pa}{high-\lsc{loc}}\morglo{yanta-man}{firewood-\lsc{all}}\morglo{ri-pti-ki}{go-\lsc{subds}-\lsc{2}}}%morpheme+gloss
\glotran{In olden times, Awkichanka, too, bewitched, \pb{they say}, up hill if you went for firewood.}{}%eng+spa trans
{}{}%rec - time

\subsection{Interrogation, negation, disjunction \phono{-chu}}\label{ssec:innedi}
\phono{-chu} indicates absolute~(\ref{Glo6:Iskwilaman}) and disjunctive questions~(\ref{Glo6:Qari}), (\ref{Glo6:Don}), negation~(\ref{Glo6:Chaytri}), and disjunction~(\ref{Glo6:Kandilaryapa}).\footnote{An anonymous reviewer points out that in Huaylas Q, negative \phono{-tsu} is distinguished from polar question \phono{-ku}. Huaylas is not unique among Quechuan languages in making this distinction.}\\

% 1
\gloexe{Glo6:Iskwilaman}{}{amv}%
{¿Iskwilaman trurashurqanki\pb{chu} mamayki?}%amv que first line
{\morglo{iskwila-man}{school-\lsc{all}}\morglo{trura-shu-rqa-nki-chu}{put-\lsc{2.obj}-\lsc{pst}-\lsc{2}-\lsc{q}}\morglo{mama-yki}{mother-\lsc{3}}}%morpheme+gloss
\glotran{\pb{Did} your mother put you in school?}{}%eng+spa trans
{}{}%rec - time

% 2
\gloexe{Glo6:Qari}{}{amv}%
{¿Qari\pb{chu} kanki warmi\pb{chu} kanki?}%amv que first line
{\morglo{¿qari-chu}{man-\lsc{q}}\morglo{ka-nki}{be-\lsc{2}}\morglo{warmi-chu}{woman-\lsc{q}}\morglo{ka-nki}{be-\lsc{2}}}%morpheme+gloss
\glotran{Are you a man \pb{or} a woman?}{}%eng+spa trans
{}{}%rec - time

% 3
\gloexe{Glo6:Don}{}{amv}%
{¿Don Juan\pb{chu} icha alman\pb{chu} hamuyan?}%amv que first line
{\morglo{Don}{Don}\morglo{Juan-chu}{Juan-\lsc{q}}\morglo{icha}{or}\morglo{alma-n-chu}{soul-\lsc{3}-\lsc{q}}\morglo{hamu-ya-n}{come-\lsc{prog}-\lsc{3}}}%morpheme+gloss
\glotran{Is it Don Juan, \pb{or} is his spirit coming?}{}%eng+spa trans
{}{}%rec - time

% 4
\gloexe{Glo6:Chaytri}{}{amv}%
{Chaytri \pb{mana} suyawarqa\pb{chu}.}%amv que first line
{\morglo{chay-tri}{\lsc{dem.d}-\lsc{evc}}\morglo{mana}{no}\morglo{suya-wa-rqa-chu}{wait-\lsc{1.obj}-\lsc{pst}-\lsc{neg}}}%morpheme+gloss
\glotran{That’s why she would\pb{n’t} have waited for me.}{}%eng+spa trans
{}{}%rec - time

% 5
\gloexe{Glo6:Kandilaryapa}{}{amv}%
{Kandilaryapa\pb{chu} bintisinkupa\pb{chu}.}%amv que first line
{\morglo{kandilarya-pa-chu}{Candelaria-\lsc{loc}-\lsc{disj}}\morglo{binti-sinku-pa-chu}{twenty-five-\lsc{loc}-\lsc{disj}}}%morpheme+gloss
\glotran{\pb{Either} on Candelaria \pb{or} on the twenty-fifth.}{}%eng+spa trans
{}{}%rec - time

\noindent
Where it functions to indicate interrogation\index[sub]{interrogation!\phono{-chu}} or negation\index[sub]{negation!\phono{-chu}}, \phono{-chu} attaches to the sentence fragment that is the focus of the interrogation or negation~(\ref{Glo6:Chaypachu}).\\

% 6
\gloexe{Glo6:Chaypachu}{}{amv}%
{¿Chaypa\pb{chu} tumarqanki?}%amv que first line
{\morglo{chay-pa-chu}{\lsc{dem.d}-\lsc{loc}-\lsc{q}}\morglo{tuma-rqa-nki}{take-\lsc{pst}-\lsc{2}}}%morpheme+gloss
\glotran{Did you take [pictures] \pb{there}?}{}%eng+spa trans
{}{}%rec - time

\noindent
Where it functions to indicate disjunction\index[sub]{disjunction} --~in either disjunctive questions or disjunctive statements~-- \phono{-chu} generally attaches to each of the disjuncts~(\ref{Glo6:Mario}).\\

% 7
\gloexe{Glo6:Mario}{}{amv}%
{Mario\pb{chu} karqa Julián\pb{chu} karqa.}%amv que first line
{\morglo{Mario-chu}{Mario-\lsc{disj}}\morglo{ka-rqa}{be-\lsc{pst}}\morglo{Julián-chu}{Julián-\lsc{disj}}\morglo{ka-rqa}{be-\lsc{pst}}}%morpheme+gloss
\glotran{It was \pb{either} Mario \pb{or} Julián.}{}%eng+spa trans
{}{}%rec - time

\noindent
Questions that anticipate a negative answer are indicated by \phono{mana-chu}~(\ref{Glo6:Manachu}).\\

% 8
\gloexe{Glo6:Manachu}{}{ch}%
{¿\pb{Manachu} kuska linman?}%ch que first line
{\morglo{mana-chu}{no-\lsc{q}}\morglo{kuska}{together}\morglo{li-n-man}{go-\lsc{3}-\lsc{cond}}}%morpheme+gloss
\glotran{\pb{Couldn’t} they go together?}{}%eng+spa trans
{}{}%rec - time

\noindent
\phono{mana-chu} may also “soften” questions~(\ref{Glo6:Paysanu}).\\

% 9
\gloexe{Glo6:Paysanu}{}{amv}%
{Paysanu, ¿\pb{manachu} vakata rantiyta munanki?}%amv que first line
{\morglo{paysanu}{countryman}\morglo{mana-chu}{no-\lsc{q}}\morglo{vaka-ta}{cow-\lsc{acc}}\morglo{ranti-y-ta}{buy-\lsc{inf}-\lsc{acc}}\morglo{muna-nki}{want-\lsc{2}}}%morpheme+gloss
\glotran{My countryman, \pb{do you not} want to buy a cow?}{}%eng+spa trans
{}{}%rec - time

\noindent
It may also be used, like \phono{aw} ‘yes’, in the formation of tag questions~(\ref{Glo6:Lliw}).\\

% 10
\gloexe{Glo6:Lliw}{}{ach}%
{Lliw lliwtriki wañukushun, puchukashun entonces, ¿\pb{manachu}?}%ach que first line
{\morglo{lliw}{all}\morglo{lliw-tr-iki}{all-\lsc{evc}-\lsc{iki}}\morglo{wañu-ku-shun}{die-\lsc{refl}-\lsc{1pl.fut}}\morglo{puchuka-shun}{finish.off-\lsc{1pl.fut}}\morglo{intunsis}{therefore}\morglo{mana-chu}{no-\lsc{q}}}%morpheme+gloss
\glotran{We’ll all have to die, to finish off then, \pb{isn’t that so}?}{}%eng+spa trans
{}{}%rec - time

\noindent
In negative sentences, \phono{-chu} generally co-occurs with \phono{mana} ‘not’~(\ref{Glo6:mana}); \phono{-chu} is also licensed by additive enclitic \phono{-pis}~(\ref{Glo6:Kaspin}), (\ref{Glo6:Manchakushpa}) and \phono{ni} ‘nor’~(\ref{Glo6:Apuraw}), (\ref{Glo6:wayta}).\\

% 11
\gloexe{Glo6:mana}{}{lt}%
{Aa, \pb{mana}yá kan\pb{chu}. \pb{Mana}yá bula kan\pb{chu}.}%lt que first line
{\morglo{aa}{ah}\morglo{mana-yá}{no-\lsc{emph}}\morglo{ka-n-chu}{be-\lsc{3}-\lsc{neg}}\morglo{mana-yá}{no-\lsc{emph}}\morglo{bula}{ball}\morglo{ka-n-chu}{be-\lsc{3}-\lsc{neg}}}%morpheme+gloss
\glotran{Ah, there are\pb{n’t} any. There are\pb{n’t} any balls.}{}%eng+spa trans
{}{}%rec - time

% 12
\gloexe{Glo6:Kaspin}{}{amv}%
{Kaspin\pb{pis} kan\pb{chu}.}%amv que first line
{\morglo{kaspi-n-pis}{stick-\lsc{3}-\lsc{add}}\morglo{ka-n-chu}{be-\lsc{3}-\lsc{neg}}}%morpheme+gloss
\glotran{She does\pb{n’t} have a stick.}{}%eng+spa trans
{}{}%rec - time

% 13
\gloexe{Glo6:Manchakushpa}{}{ach}%
{Manchakushpa tuta\pb{s} puñu:\pb{chu}.}%ach que first line
{\morglo{mancha-ku-shpa}{scare-\lsc{refl}-\lsc{subis}}\morglo{tuta-s}{night-\lsc{add}}\morglo{puñu-:-chu}{sleep-\lsc{1}-\lsc{neg}}}%morpheme+gloss
\glotran{Being scared, I \pb{don’t} sleep at night.}{}%eng+spa trans
{}{}%rec - time

% 14
\gloexe{Glo6:Apuraw}{}{amv}%
{Apuraw wañururqariki. \pb{Ni} apanña\pb{chu}.}%amv que first line
{\morglo{apuraw}{quick}\morglo{wañu-ru-rqa-r-iki}{die-\lsc{urgt}-\lsc{pst}-\lsc{r}-\lsc{iki}}\morglo{ni}{nor}\morglo{apa-n-ña-chu}{bring-\lsc{3}-\lsc{disc}-\lsc{neg}}}%morpheme+gloss
\glotran{He died quickly. They \pb{didn’t even} bring him [to the hospital].}{}%eng+spa trans
{}{}%rec - time

% 15
\gloexe{Glo6:wayta}{}{amv}%
{\pb{Manam} wayta\pb{chu} \pb{ni} pishqu\pb{chu}.}%amv que first line
{\morglo{mana-m}{no-\lsc{evd}}\morglo{wayta-chu}{flower-\lsc{neg}}\morglo{ni}{nor}\morglo{pishqu-chu}{bird-\lsc{neg}}}%morpheme+gloss
\glotran{\pb{Neither} a flower \pb{nor} a bird.}{}%eng+spa trans
{}{}%rec - time

\noindent
In prohibitions, \phono{-chu} co-occurs with \phono{ama} ‘don’t’~(\ref{Glo6:wawqi}).\\

% 16
\gloexe{Glo6:wawqi}{}{ach}%
{“¡\pb{Ama} wawqi:taqa wañuchiy\pb{chu}!” niptinshi wañurachin paywantapis.}%ach que first line
{\morglo{ama}{\lsc{proh}}\morglo{wawqi-:-ta-qa}{brother-\lsc{1}-\lsc{acc}-\lsc{top}}\morglo{wañu-chi-y-chu}{die-\lsc{caus}-\lsc{imp}-\lsc{neg}}\morglo{ni-pti-n-shi}{say-\lsc{subds}-\lsc{3}-\lsc{evr}}\morglo{wañu-ra-chi-n}{die-\lsc{urgt}-\lsc{caus}-\lsc{3}}\morglo{pay-wan-ta-pis}{he-\lsc{instr}-\lsc{acc}-\lsc{add}}}%morpheme+gloss
\glotran{When he said, “\pb{Don’t} kill my brother!” they killed him with him, too.}{}%eng+spa trans
{}{}%rec - time

\noindent
\phono{-chu} does not appear in subordinate clauses, where negation is indicated with a negative particle alone~(\ref{Glo6:qali}), (\ref{Glo6:qatrachakunanpaq}).\footnote{An anonymous reviewer points out that elsewhere in Quechua, the correlates of negative \phono{-chu} typically can appear in subordinate clauses. There are no naturally-occurring examples of this in the Yauyos corpus.}\\

% 17
\gloexe{Glo6:qali}{}{ch}%
{\pb{Mana} qali kaptinqa ñuqanchikpis taqllakta hapishpa qaluwanchik.}%ch que first line
{\morglo{mana}{no}\morglo{qali}{man}\morglo{ka-pti-n-qa}{be-\lsc{subds}-\lsc{3}-\lsc{top}}\morglo{ñuqanchik-pis}{we-\lsc{add}}\morglo{taqlla-kta}{plow-\lsc{acc}}\morglo{hapi-shpa}{grab-\lsc{subis}}\morglo{qaluwa-nchik}{turn.earth-\lsc{1pl}}}%morpheme+gloss
\glotran{When there are \pb{no} \pb{men}, we grab the plow and turn the earth.}{}%eng+spa trans
{}{}%rec - time

% 18
\gloexe{Glo6:qatrachakunanpaq}{}{amv}%
{\pb{Mana} qatrachakunanpaq mandilchanta watachakun.}%amv que first line
{\morglo{mana}{no}\morglo{qatra-cha-ku-na-n-paq}{dirty-\lsc{fact}-\lsc{refl}-\lsc{nmlz}-\lsc{3}-\lsc{purp}}\morglo{mandil-cha-n-ta}{apron-\lsc{dim}-\lsc{3}-\lsc{acc}}\morglo{wata-cha-ku-n}{tie-\lsc{dim}-\lsc{refl}-\lsc{3}}}%morpheme+gloss
\glotran{She’s tying on an apron \pb{so} she \pb{doesn’t} get dirty.}{}%eng+spa trans
{}{}%rec - time

% 19
\gloexe{Glo6:lluqsiptiyki}{}{amv}%
{Manam lluqsiptiyki(qa *\pb{chu}), waqashaqmi.}%amv que first line
{\morglo{mana-m}{no-\lsc{evd}}\morglo{lluqsi-pti-yki-qa}{go.out-\lsc{subds}-\lsc{2}-\lsc{top}}\morglo{chu}{neg}\morglo{waqa-shaq-mi}{cry-\lsc{1.fut}-\lsc{evd}}}%morpheme+gloss
\glotran{\pb{If} you \pb{don’t} go, I’ll cry.}{}%eng+spa trans
{}{}%rec - time

\noindent
In negative sentences, \phono{-chu} never occurs on the same segment as does an evidential enclitic~(\ref{Glo6:lluqsirqanki}).\\

% 20
\gloexe{Glo6:lluqsirqanki}{}{amv}%
{Mana lluqsirqanki(*mi)\pb{chu}.}% que first line
{\morglo{mana}{no}\morglo{lluqsi-rqa-nki-mi-chu}{go.out-\lsc{pst}-\lsc{2}-\lsc{evd}-\lsc{neg}}}%morpheme+gloss
\glotran{You \pb{didn’t} leave.}{}%eng+spa trans
{}{}%rec - time

\noindent
Interrogative \phono{-chu} does not appear in questions using interrogative pronouns~(\ref{Glo6:hamurqa}).\footnote{\phono{¿*Pi-taq} \phono{hamu-n-chu?} \phono{¿*Pi-taq-chu} \phono{hamu-n?} ‘Who is coming?’}\\

% 21
\gloexe{Glo6:hamurqa}{}{amv}%
{*¿Pi hamurqa\pb{chu}?}% que first line
{\morglo{pi}{who}\morglo{hamu-rqa-chu}{come-\lsc{pst}-\lsc{neg}}}%morpheme+gloss
\glotran{\pb{Who} came?}{}%eng+spa trans
{}{}%rec - time

\subsection{Restrictive, limitative \phono{-lla}}
\phono{-lla} indicates exclusivity or limitation in number\index[sub]{restrictive}: the individual~(\ref{Glo6:Iskwilapam}--\ref{Glo6:Kichwa}) or event/event type~(\ref{Glo6:Fwirti}), (\ref{Glo6:lliwtam}) remains limited to itself and is accompanied by no other.\\

% 1
\gloexe{Glo6:Iskwilapam}{}{sp}%
{Iskwilapam niytu:kunaqa wawa:kunaqa rinmi ñuqa\pb{lla}m ka: analfabitu.}%sp que first line
{\morglo{iskwila-pa-m}{school-\lsc{loc}-\lsc{evd}}\morglo{niytu-:-kuna-qa}{nephew-\lsc{1}-\lsc{pl}-\lsc{top}}\morglo{wawa-:-kuna-qa}{baby-\lsc{1}-\lsc{pl}-\lsc{top}}\morglo{ri-n-mi}{go-\lsc{3}-\lsc{evd}}\morglo{ñuqa-lla-m}{I-\lsc{rstr}-\lsc{evd}}\morglo{ka-:}{be-\lsc{1}}\morglo{analfabitu}{illiterate}}%morpheme+gloss
\glotran{My grandchildren are in school. My children went. I’m the \pb{only} illiterate one.}{}%eng+spa trans
{}{}%rec - time

% 2
\gloexe{Glo6:Runapi}{}{amv}%
{Runapi uma\pb{lla}ña traki\pb{lla}ña kayasa.}%amv que first line
{\morglo{runa-pi}{person-\lsc{gen}}\morglo{uma-lla-ña}{head-\lsc{rstr}-\lsc{disc}}\morglo{traki-lla-ña}{foot-\lsc{rstr}-\lsc{disc}}\morglo{ka-ya-sa}{be-\lsc{prog}-\lsc{npst}}}%morpheme+gloss
\glotran{\pb{Just} the head and the hand remained of the person.}{}%eng+spa trans
{}{}%rec - time

% 3
\gloexe{Glo6:Kichwa}{}{ch}%
{Kichwa\pb{lla}ktam limakuya: kaytrawlaq manam kastillanukta lima:chu.}%ch que first line
{\morglo{kichwa-lla-kta-m}{Quechua-\lsc{rstr}-\lsc{acc}-\lsc{evd}}\morglo{lima-ku-ya-:}{speak-\lsc{refl}-\lsc{prog}-\lsc{1}}\morglo{kay-traw-laq}{\lsc{dem.p}-\lsc{loc}-\lsc{cont}}\morglo{mana-m}{no-\lsc{evd}}\morglo{kastillanu-kta}{Spanish-\lsc{acc}}\morglo{lima-:-chu}{speak-\lsc{1}-\lsc{neg}}}%morpheme+gloss
\glotran{I’m talking \pb{just} Quechua. Here, still, we don’t speak Spanish.}{}%eng+spa trans
{}{}%rec - time

% 4
\gloexe{Glo6:Fwirti}{}{ch}%
{Fwirti kashpa\pb{lla}má linchik pustaman.}%ch que first line
{\morglo{fwirti}{strong}\morglo{ka-shpa-lla-m-á}{be-\lsc{subis}-\lsc{rstr}-\lsc{evd}-\lsc{emph}}\morglo{li-nchik}{go-\lsc{1pl}}\morglo{pusta-man}{clinic-\lsc{all}}}%morpheme+gloss
\glotran{\pb{Only} if it’s bad will we go to the health clinic.}{}%eng+spa trans
{}{}%rec - time

% 5
\gloexe{Glo6:lliwtam}{}{ach}%
{Lliw lliwtam rantishpa\pb{lla}ñam kanan kamatapis chay polarkunatapis.}%ach que first line
{\morglo{lliw}{all}\morglo{lliw-ta-m}{all-\lsc{acc}-\lsc{evd}}\morglo{ranti-shpa-lla-ña-m}{buy-\lsc{subis}-\lsc{rstr}-\lsc{disc}-\lsc{evd}}\morglo{kanan}{now}\morglo{kama-ta-pis}{blanket-\lsc{acc}-\lsc{add}}\morglo{chay}{\lsc{dem.d}}\morglo{polar-kuna-ta-pis}{fleece-\lsc{pl}-\lsc{acc}-\lsc{add}}}%morpheme+gloss
\glotran{Now \pb{just} buying everything -- blankets, [polyester] fleece.}{}%eng+spa trans
{}{}%rec - time

\noindent
\phono{-lla} can generally be translated as ‘just’~(\ref{Glo6:Chayna}), (\ref{Glo6:Sirka}) or ‘only’~(\ref{Glo6:Chay}); it sometimes has an ‘exactly’ interpretation~(\ref{Glo6:Iskinanpi}).\\

% 6
\gloexe{Glo6:Chayna}{}{amv}%
{Chayna\pb{lla}m mikuchin~\dots{} pachachin.}%amv que first line
{\morglo{chayna-\pb{lla}-m}{thus-\lsc{rstr}-\lsc{evd}}\morglo{miku-chi-n}{eat-\lsc{caus}-\lsc{3}}\morglo{pacha-chi-n}{dress-\lsc{caus}-\lsc{3}}}%morpheme+gloss
\glotran{\pb{Just} like that, she feeds him, she clothes him.}{}%eng+spa trans
{}{}%rec - time

% 7
\gloexe{Glo6:Sirka}{}{sp}%
{Sirka\pb{lla}tam riya: manam karutachu.}%sp que first line
{\morglo{sirka-lla-ta-m}{close-\lsc{rstr}-\lsc{acc}-\lsc{evd}}\morglo{ri-ya-:}{go-\lsc{prog}-\lsc{1}}\morglo{mana-m}{no-\lsc{evd}}\morglo{karu-ta-chu}{far-\lsc{acc}-\lsc{neg}}}%morpheme+gloss
\glotran{I \pb{just} go close; I don’t go far.}{}%eng+spa trans
{}{}%rec - time

% 8
\gloexe{Glo6:Chay}{}{amv}%
{Chay\pb{lla}tam yatrani. Masta yatranichu.}%amv que first line
{\morglo{chay-lla-ta-m}{\lsc{dem.d}-\lsc{lim}-\lsc{acc}-\lsc{evd}}\morglo{yatra-ni}{know-\lsc{1}}\morglo{mas-ta}{more-\lsc{acc}}\morglo{yatra-ni-chu}{know-\lsc{1}-\lsc{neg}}}%morpheme+gloss
\glotran{I \pb{only} know that. I don’t know more.}{}%eng+spa trans
{}{}%rec - time

% 9
\gloexe{Glo6:Iskinanpi}{}{lt}%
{Iskinanpi sikya tuna\pb{lla}npi wallpay watrakunraq.}%lt que first line
{\morglo{iskina-n-pi}{corner-\lsc{3}-\lsc{loc}}\morglo{sikya}{aqueduct}\morglo{tuna-lla-n-pi}{corner-\lsc{rstr}-\lsc{3}-\lsc{loc}}\morglo{wallpa-y}{chicken-\lsc{1}}\morglo{watra-ku-n-raq}{give.birth-\lsc{refl}-\lsc{3}-\lsc{cont}}}%morpheme+gloss
\glotran{My hen lays eggs in the corner, \pb{right} in the corner of the canal.}{}%eng+spa trans
{}{}%rec - time

\noindent
It is very, very widely employed~(\ref{Glo6:abaskuna}--\ref{Glo6:Chaytam}).\\

% 10
\gloexe{Glo6:abaskuna}{}{amv}%
{Lliwta abaskuna albirhakuna ayvis\pb{lla} rantikuni apani llaqtatam.}%amv que first line
{\morglo{lliw-ta}{all-\lsc{acc}}\morglo{abas-kuna}{broad.beans-\lsc{pl}}\morglo{albirha-kuna}{peas-\lsc{pl}}\morglo{ayvis-lla}{sometimes-\lsc{rstr}}\morglo{ranti-ku-ni}{buy-\lsc{refl}-\lsc{1}}\morglo{apa-ni}{bring-\lsc{1}}\morglo{llaqta-ta-m}{town-\lsc{acc}-\lsc{evd}}}%morpheme+gloss
\glotran{Everything -- broad beans, peas -- \pb{once in while} I sell stuff -- I bring it into town.}{}%eng+spa trans
{}{}%rec - time

% 11
\gloexe{Glo6:kwintuqa}{}{sp}%
{Chayna\pb{lla}m. Chay\pb{lla}m kwintuqa. Mas kanchu manam.}%sp que first line
{\morglo{chayna-lla-m}{thus-\lsc{rstr}-\lsc{evd}}\morglo{chay-lla-m}{\lsc{dem.d}-\lsc{rstr}-\lsc{evd}}\morglo{kwintu-qa}{story-\lsc{top}}\morglo{mas}{more}\morglo{ka-n-chu}{be-\lsc{3}-\lsc{neg}}\morglo{mana-m}{no-\lsc{evd}}}%morpheme+gloss
\glotran{That’s the way it goes. That’s \pb{all} there is to the story. There’s no more.}{}%eng+spa trans
{}{}%rec - time

% 12
\gloexe{Glo6:Chaytam}{}{amv}%
{Chaytam aysashpa\pb{lla} pasachiwaq.}%amv que first line
{\morglo{chay-ta-m}{\lsc{dem.d}-\lsc{acc}-\lsc{evd}}\morglo{aysa-shpa-lla}{pull-\lsc{subis}-\lsc{rstr}}\morglo{pasa-chi-wa-q}{pass-\lsc{caus}-\lsc{1.obj}-\lsc{ag}}}%morpheme+gloss
\glotran{They had me cross the river pulling [me by the hand].}{}%eng+spa trans
{}{}%rec - time

\subsection{Discontinuative \phono{-ña}}
Discontinuitive. \phono{-ña}\index[sub]{discontinuitive} indicates transition --~change of state or quality. In affirmative statements, it can generally be translated as ‘already’~(\ref{Glo6:Kundinadaw}--\ref{Glo6:Paqwayanchik}); in negative statements, as ‘no more’ or ‘no longer’~(\ref{Glo6:Unaytrik}), (\ref{Glo6:Manana}); and in questions, as ‘yet’~(\ref{Glo6:Pasarun}), (\ref{Glo6:Rimaya}).\\

% 1
\gloexe{Glo6:Kundinadaw}{}{amv}%
{Kundinadaw\pb{ña}m wakqa kayan.}%amv que first line
{\morglo{kundinadaw-ña-m}{condemned-\lsc{disc}-\lsc{evd}}\morglo{wak-qa}{\lsc{dem.d}-\lsc{top}}\morglo{ka-ya-n}{be-\lsc{prog}-\lsc{3}}}%morpheme+gloss
\glotran{That one is \pb{already} condemned.}{}%eng+spa trans
{}{}%rec - time

% 2
\gloexe{Glo6:kukaywan}{}{amv}%
{Ñuqaqa kukaywan\pb{ña}m qawaruni.}%amv que first line
{\morglo{ñuqa-qa}{I-\lsc{top}}\morglo{kuka-y-wan-ña-m}{coca-\lsc{1}-\lsc{instr}-\lsc{disc}-\lsc{evd}}\morglo{qawa-ru-ni}{see-\lsc{urgt}-\lsc{1}}}%morpheme+gloss
\glotran{I saw it with my coca \pb{already}.}{}%eng+spa trans
{}{}%rec - time

% 3
\gloexe{Glo6:Paqwayanchik}{}{ch}%
{Paqwayanchik\pb{ña}m talpuyta, ¿aw? Papaktapis talpulalu:\pb{ña}m, kanan halakta, ¿aw?}%ch que first line
{\morglo{paqwa-ya-nchik-ña-m}{finish-\lsc{prog}-\lsc{1pl}-\lsc{disc}-\lsc{evd}}\morglo{talpu-y-ta}{plant-\lsc{inf}-\lsc{acc}}\morglo{aw}{yes}\morglo{papa-kta-pis}{potato-\lsc{acc}-\lsc{add}}\morglo{talpu-la-lu-:-ña-m}{plant-\lsc{unint}-\lsc{urgt}-\lsc{1}-\lsc{disc}-\lsc{evd}}\morglo{kanan}{now}\morglo{hala-kta}{corn-\lsc{acc}}\morglo{aw}{yes}}%morpheme+gloss
\glotran{We’re finishing the planting \pb{already}, no? We’ve \pb{already} planted the potatoes, now the corn, no?}{}%eng+spa trans
{}{}%rec - time

% 4
\gloexe{Glo6:Unaytrik}{}{sp}%
{Unaytrik. Kananqa kan\pb{ña}chu imapis.}%sp que first line
{\morglo{unay-tri-k}{before-\lsc{evc}-\lsc{ik}}\morglo{kanan-qa}{now-\lsc{top}}\morglo{ka-n-ña-chu}{be-\lsc{3}-\lsc{disc}-\lsc{neg}}\morglo{ima-pis}{what-\lsc{add}}}%morpheme+gloss
\glotran{That would be a long time ago. Now there isn’t anything \pb{any more}.}{}%eng+spa trans
{}{}%rec - time

% 5
\gloexe{Glo6:Manana}{}{amv}%
{\pb{Manaña} ni santu ni imapis.}%amv que first line
{\morglo{mana-ña}{no-\lsc{disc}}\morglo{ni}{nor}\morglo{santu}{saint}\morglo{ni}{nor}\morglo{ima-pis}{what-\lsc{add}}}%morpheme+gloss
\glotran{There are \pb{no longer} saints or anything.}{}%eng+spa trans
{}{}%rec - time

% 6
\gloexe{Glo6:Pasarun}{}{amv}%
{¿Pasarun\pb{ñachu}? Tapushun.}%amv que first line
{\morglo{pasa-ru-n-ña-chu}{pass-\lsc{urgt}-\lsc{3}-\lsc{disc}-\lsc{q}}\morglo{tapu-shun}{ask-\lsc{1pl.fut}}}%morpheme+gloss
\glotran{Did she go by \pb{yet}? Let’s ask.}{}%eng+spa trans
{}{}%rec - time

% 7
\gloexe{Glo6:Rimaya}{}{lt}%
{¿Rimaya\pb{nña}\pb{chu} kanan wakpi?}%lt que first line
{\morglo{rima-ya-n-ña-chu}{talk-\lsc{prog}-\lsc{3}-\lsc{disc}-\lsc{q}}\morglo{kanan}{now}\morglo{wak-pi}{\lsc{dem.d}-\lsc{loc}}}%morpheme+gloss
\glotran{Are they talking \pb{yet} there now?}{}%eng+spa trans
{}{}%rec - time

\noindent
It can appear freely but never unaccompanied, redundantly, by \phono{ña}~(\ref{Glo6:tukuchkani}), (\ref{Glo6:riqsiyan}).\\

% 8
\gloexe{Glo6:tukuchkani}{}{amv}%
{“¡\pb{Ñam} tukuchkani\pb{ña}!” ¡Puk! ¡Puk! ¡Puk! sikisapa sapu.}%amv que first line
{\morglo{ña-m}{\lsc{disc}-\lsc{evd}}\morglo{tuku-chka-ni-ña}{finish-\lsc{dur}-\lsc{1}-\lsc{disc}}\morglo{puk}{puk}\morglo{puk}{puk}\morglo{puk}{puk}\morglo{siki-sapa}{behind-\lsc{mult.poss}}\morglo{sapu}{frog}}%morpheme+gloss
\glotran{“I’m \pb{already} finishing up!” Puk! Puk! Puk! said the frog with the behind bigger than usual.}{}%eng+spa trans
{}{}%rec - time

% 9
\gloexe{Glo6:riqsiyan}{}{lt}%
{\pb{Ñam} riqsiyan\pb{ña} hukya yaykun.}%lt que first line
{\morglo{ña-m}{\lsc{disc}-\lsc{evd}}\morglo{riqsi-ya-n-ña}{know-\lsc{prog}-\lsc{3}-\lsc{disc}}\morglo{huk-ya}{one-\lsc{emph}}\morglo{yayku-n}{enter-\lsc{3}}}%morpheme+gloss
\glotran{They’re getting to know it \pb{already} and another comes in.}{}%eng+spa trans
{}{}%rec - time

\subsection{Inclusion \phono{-pis}}
\phono{-pis}\index[sub]{inclusion} indicates the inclusion of an item or event into a series of similar items or events. Translated as ‘and’, ‘too’, ‘also’, and ‘even’~(\ref{Glo6:Turnuchawan}--\ref{Glo6:Mamanwa}) or, when negated, ‘neither’ or ‘not even’~(\ref{Glo6:Imapaqtaq}--\ref{Glo6:Pata}).\\

% 1
\gloexe{Glo6:Turnuchawan}{}{ch}%
{Turnuchawan ñuqakunaqa trabaha: walmi\pb{pis} qali\pb{pis}.}%ch que first line
{\morglo{turnu-cha-wan}{turn-\lsc{dim}-\lsc{instr}}\morglo{ñuqa-kuna-qa}{I-\lsc{pl}-\lsc{top}}\morglo{trabaha-:}{work-\lsc{1}}\morglo{walmi-pis}{woman-\lsc{add}}\morglo{qali-pis}{man-\lsc{add}}}%morpheme+gloss
\glotran{We work in turns, the women \pb{and} the men.}{}%eng+spa trans
{}{}%rec - time

% 2
\gloexe{Glo6:Tukuy}{}{amv}%
{Tukuy tuta tushun qaynintinta\pb{pis}.}%amv que first line
{\morglo{tukuy}{all}\morglo{tuta}{night}\morglo{tushu-n}{dance-\lsc{3}}\morglo{qaynintin-ta-pis}{next.day-\lsc{acc}-\lsc{add}}}%morpheme+gloss
\glotran{They dance all night and the next day, \pb{too}.}{}%eng+spa trans
{}{}%rec - time

% 3
\gloexe{Glo6:subrinu}{}{amv}%
{Pay\pb{pis} chay subrinu wañukuptinñamik payqa tumarun.}%amv que first line
{\morglo{pay-pis}{he-\lsc{add}}\morglo{chay}{\lsc{dem.d}}\morglo{subrinu}{nephew}\morglo{wañu-ku-pti-n-ña-mi-k}{die-\lsc{refl}-\lsc{subds}-\lsc{3}-\lsc{disc}-\lsc{evd-\lsc{ik}}}\morglo{pay-qa}{he-\lsc{top}}\morglo{tuma-ru-n}{take-\lsc{urgt}-\lsc{3}}}%morpheme+gloss
\glotran{He, \pb{too}, when his nephew died, took it [poison].}{}%eng+spa trans
{}{}%rec - time

% 4
\gloexe{Glo6:Salchipullu}{}{amv}%
{Salchipullu rantikuqta\pb{pis} tumarun.}%amv que first line
{\morglo{salchipullu}{fried.chicken}\morglo{ranti-ku-q-ta-pis}{buy-\lsc{refl}-\lsc{ag}-\lsc{acc}-\lsc{add}}\morglo{tuma-ru-n}{take-\lsc{urgt}-\lsc{3}}}%morpheme+gloss
\glotran{She took [pictures] of the people selling fried chicken \pb{also}.}{}%eng+spa trans
{}{}%rec - time

% 5
\gloexe{Glo6:Mamanwa}{}{amv}%
{Maman wañukuptin\pb{pis} manam waqanchu.}%amv que first line
{\morglo{mama-n}{mother-\lsc{3}}\morglo{wañu-ku-pti-n-pis}{die-\lsc{refl}-\lsc{subds}-\lsc{3}-\lsc{add}}\morglo{mana-m}{no-\lsc{evd}}\morglo{waqa-n-chu}{cry-\lsc{3}-\lsc{neg}}}%morpheme+gloss
\glotran{\pb{Even} when his mother died, he didn’t cry.}{}%eng+spa trans
{}{}%rec - time

% 6
\gloexe{Glo6:Imapaqtaq}{}{amv}%
{“¿Imapaqtaq ñuqa waqashaq?” nin. “Warmiypaq\pb{pis} waqarqani\pb{chu}.”}%amv que first line
{\morglo{ima-paq-taq}{what-\lsc{purp}-\lsc{seq}}\morglo{ñuqa}{I}\morglo{waqa-shaq}{cry-\lsc{1.fut}}\morglo{nin}{say-\lsc{3}}\morglo{warmi-y-paq-pis}{woman-\lsc{1}-\lsc{ben}-\lsc{add}}\morglo{waqa-rqa-ni-chu}{cry-\lsc{pst}-\lsc{1}-\lsc{neg}}}%morpheme+gloss
\glotran{“Why am I going to cry?” he said. “I did\pb{n’t} cry for my wife, \pb{either}.”}{}%eng+spa trans
{}{}%rec - time

% 7
\gloexe{Glo6:Paykunaqa}{}{amv}%
{Paykunaqa \pb{manam} qawarqa\pb{pischu}.}%amv que first line
{\morglo{pay-kuna-qa}{he-\lsc{pl}-\lsc{top}}\morglo{mana-m}{no-\lsc{evd}}\morglo{qawa-rqa-pis-chu}{see-\lsc{pst}-\lsc{add}-\lsc{neg}}}%morpheme+gloss
\glotran{\pb{Neither} did they see us.}{}%eng+spa trans
{}{}%rec - time

% 8
\gloexe{Glo6:Pata}{}{amv}%
{Pata saqayta\pb{pis} atipan\pb{chu}.}%amv que first line
{\morglo{pata}{terrace}\morglo{saqa-y-ta-pis}{go.up-\lsc{inf}-\lsc{acc}-\lsc{add}}\morglo{atipa-n-chu}{be.able-\lsc{3}-\lsc{neg}}}%morpheme+gloss
\glotran{They ca\pb{n’t} \pb{even} go up one terrace.}{}%eng+spa trans
{}{}%rec - time

\noindent
\phono{-pis} may --~or, even, may generally~-- imply contrast with some preceding element. Where it scopes over subordinate clauses, it can often be translated ‘although’ or ‘even’~(\ref{Glo6:Uratam}), (\ref{Glo6:Hinaptin}).\\

% 9
\gloexe{Glo6:Uratam}{}{amv}%
{Uratam muna\pb{shpapis}.}%amv que first line
{\morglo{ura-ta-m}{hour-\lsc{acc}-\lsc{evd}}\morglo{muna-shpa-pis}{want-\lsc{subis}-\lsc{add}}}%morpheme+gloss
\glotran{\pb{Although} I want to know the time.}{}%eng+spa trans
{}{}%rec - time

% 10
\gloexe{Glo6:Hinaptin}{}{sp}%
{Hinaptin wasipiña rumiwan takaptin\pb{pis} uyan\pb{chu}.}%sp que first line
{\morglo{hinaptin}{then}\morglo{wasi-pi-ña}{house-\lsc{loc}-\lsc{disc}}\morglo{rumi-wan}{stone-\lsc{instr}}\morglo{taka-pti-n-pis}{hit-\lsc{subds}-\lsc{3}-\lsc{add}}\morglo{uya-n-chu}{be.able-\lsc{3}-\lsc{neg}}}%morpheme+gloss
\glotran{Later, at home, \pb{even when} they hit it with a rock, it couldn’t.}{}%eng+spa trans
{}{}%rec - time

\noindent
Attaching to interrogative-indefinite stems, it forms indefinites and, with \phono{mana}, negative indefinites~(\ref{Glo6:Chaynam}--\ref{Glo6:chambyakuqpaq}).\\

% 11
\gloexe{Glo6:Chaynam}{}{amv}%
{Chaynam \pb{imallatapis} wasiman apamun.}%amv que first line
{\morglo{chayna-m}{thus-\lsc{evd}}\morglo{ima-lla-ta-pis}{what-\lsc{rstr}-\lsc{acc}-\lsc{add}}\morglo{wasi-man}{house-\lsc{all}}\morglo{apa-mu-n}{bring-\lsc{cisl}-\lsc{3}}}%morpheme+gloss
\glotran{That way he brings a little \pb{something} to his house.}{}%eng+spa trans
{}{}%rec - time

% 12
\gloexe{Glo6:tiyndaman}{}{ach}%
{Llapa tiyndaman yaykushpaqa lliw lliwshi \pb{imantapis} apakun.}%ach que first line
{\morglo{llapa}{all}\morglo{tiynda-man}{store-\lsc{all}}\morglo{yayku-shpa-qa}{enter-\lsc{subis}-\lsc{top}}\morglo{lliw}{all}\morglo{lliw-shi}{all-\lsc{evr}}\morglo{ima-n-ta-pis}{what-\lsc{3}-\lsc{acc}-\lsc{add}}\morglo{apa-ku-n}{bring-\lsc{refl}-\lsc{3}}}%morpheme+gloss
\glotran{They entered all the stores and took everything and \pb{anything} they had.}{}%eng+spa trans
{}{}%rec - time

% 13
\gloexe{Glo6:chambyakuqpaq}{}{amv}%
{Alli chambyakuqpaq \pb{manam imapis} faltanmanchu.}%amv que first line
{\morglo{alli}{good}\morglo{chambya-ku-q-paq}{work-\lsc{refl}-\lsc{ag}-\lsc{ben}}\morglo{mana}{no}\morglo{ima-pis}{what-\lsc{add}}\morglo{falta-n-man-chu}{be.missing-\lsc{3}-\lsc{cond}-\lsc{neg}}}%morpheme+gloss
\glotran{\pb{Nothing} can be lacking for a good worker.}{}%eng+spa trans
{}{}%rec - time

\noindent
It is in free variation with \phono{-pas}, and, after a vowel, with \phono{-s}~(\ref{Glo6:Diskansakamuy}--\ref{Glo6:harquruwara}), the latter particularly common in the \ACH{} dialect.\\

% 14
\gloexe{Glo6:Diskansakamuy}{}{lt}%
{“¡Diskansakamuy wasikipa!” niwan kikin\pb{pas} diskansuman ripun.}%lt que first line
{\morglo{diskansa-ka-mu-y}{rest-\lsc{refl}-\lsc{cisl}-\lsc{imp}}\morglo{wasi-ki-pa}{house-\lsc{2}-\lsc{loc}}\morglo{ni-wa-n}{say-\lsc{1.obj}-\lsc{3}}\morglo{kiki-n-pas}{self-\lsc{3}-\lsc{add}}\morglo{diskansu-man}{rest-\lsc{all}}\morglo{ripu-n}{go-\lsc{3}}}%morpheme+gloss
\glotran{“Go rest in your house,” he said to me and he, himself, \pb{too}, went to rest.}{}%eng+spa trans
{}{}%rec - time

% 15
\gloexe{Glo6:Hinaptinqa}{}{sp}%
{Hinaptinqa yutu pawaptinqa chay, “¡Aaaapship ship ship!” Yutu\pb{pas} “¡Wwaaaayyy!”}%sp que first line
{\morglo{hinaptin-qa}{then-\lsc{top}}\morglo{yutu}{partridge}\morglo{pawa-pti-n-qa}{fly-\lsc{subds}-\lsc{3}-\lsc{top}}\morglo{chay}{\lsc{dem.d}}\morglo{aaaapship}{aaaapship}\morglo{ship}{ship}\morglo{ship}{ship}\morglo{yutu-pas}{partridge-\lsc{add}}\morglo{wwaaaayyy}{wwaaaayyy}}%morpheme+gloss
\glotran{Then, when the partridge jumped, he [cried], “Aaaap-ship-ship-ship!” The partridge, \pb{too}, [cried] “Wwaaaayyy!”}{}%eng+spa trans
{}{}%rec - time

% 16
\gloexe{Glo6:harquruwara}{}{lt}%
{Ñuqata\pb{s} harquruwara Kashapataman riranim.}%lt que first line
{\morglo{ñuqa-ta-s}{I-\lsc{acc}-\lsc{add}}\morglo{harqu-ru-wa-ra}{toss.out-\lsc{urgt}-\lsc{1.obj}-\lsc{pst}}\morglo{Kashapata-man}{Kashapata-\lsc{all}}\morglo{ri-ra-ni-m}{go-\lsc{pst}-\lsc{1}-\lsc{evd}}}%morpheme+gloss
\glotran{They threw me out, \pb{too}, and I went to Kashapata.}{}%eng+spa trans
{}{}%rec - time

\subsection{Precision, certainty \phono{-puni}}
\phono{-puni} indicates certainty\index[sub]{certainty} or precision\index[sub]{precision}. It can be translated as ‘necessarily’, ‘definitely’, ‘precisely’. It is attested only in the \AMV{} dialect, where, still, it is not widely employed.\\

% 1
\gloexe{Glo6:Paqarin}{}{amv}%
{Paqarin\pb{puni}m rishaq.~\updag}%amv que first line
{\morglo{paqarin-puni-m}{tomorrow-\lsc{cert}-\lsc{evd}}\morglo{ri-shaq}{go-\lsc{1.fut}}}%morpheme+gloss
\glotran{I’m going to go \pb{precisely} tomorrow.}{}%eng+spa trans
{}{}%rec - time

% 2
\gloexe{Glo6:puni}{}{amv}%
{Mana\pb{puni}m.~\updag}%amv que first line
{\morglo{mana-puni-m}{no-\lsc{cert}-\lsc{evd}}}%morpheme+gloss
\glotran{By no means.}{}%eng+spa trans
{}{}%rec - time

% 3
\gloexe{Glo6:wiqawninchikman}{}{amv}%
{Chay wiqawninchikman\pb{puni} chiri yakuta truranchik.}%amv que first line
{\morglo{chay}{\lsc{dem.d}}\morglo{wiqaw-ni-nchik-man-puni}{waist-\lsc{euph}-\lsc{1pl}-\lsc{all}-\lsc{cert}}\morglo{chiri}{cold}\morglo{yaku-ta}{water-\lsc{acc}}\morglo{trura-nchik}{put-\lsc{1pl}}}%morpheme+gloss
\glotran{We put cold water \pb{right} on our lower backs.}{}%eng+spa trans
{}{}%rec - time

\subsection{Topic-marking \phono{-qa}}\label{ssec:topic}
\phono{-qa}\index[sub]{topic marker} indicates the topic of a clause~(\ref{Glo6:sultiram}--\ref{Glo6:Difindiwanchik}), including in those cases where it attaches to subordinate clauses~(\ref{Glo6:pasiyuman}), (\ref{Glo6:Qipiruptinqa}).\\

% 1
\gloexe{Glo6:sultiram}{}{ch}%
{Madri sultiram kaya: ñuqalla\pb{qa}.}%ch que first line
{\morglo{madri}{mother}\morglo{sultira-m}{alone-\lsc{evd}}\morglo{ka-ya-:}{be-\lsc{prog}-\lsc{1}}\morglo{ñuqa-lla-qa}{I-\lsc{rstr}-\lsc{top}}}%morpheme+gloss
\glotran{\pb{I}’m a single mother.}{}%eng+spa trans
{}{}%rec - time

% 2
\gloexe{Glo6:Ganawniyki}{}{lt}%
{Ganawniyki\pb{qa} achkam miranqa.}%lt que first line
{\morglo{ganaw-ni-yki-qa}{cattle-\lsc{euph}-\lsc{2}-\lsc{top}}\morglo{achka-m}{a.lot-\lsc{evd}}\morglo{mira-nqa}{increase-\lsc{3.fut}}}%morpheme+gloss
\glotran{Your \pb{cattle} are going to multiply a lot.}{}%eng+spa trans
{}{}%rec - time

% 3
\gloexe{Glo6:waqakunki}{}{sp}%
{Qam\pb{qa} waqakunki sumaqllatam. Ñuqa\pb{qa} quyu quyuta waqayani.}%sp que first line
{\morglo{qam-qa}{you-\lsc{top}}\morglo{waqa-ku-nki}{cry-\lsc{refl}-\lsc{2}}\morglo{sumaq-lla-ta-m}{pretty-\lsc{rstr}-\lsc{acc}-\lsc{evd}}\morglo{ñuqa-\pb{qa}}{I-\lsc{top}}\morglo{quyu}{ugly}\morglo{quyu-ta}{ugly-\lsc{acc}}\morglo{waqa-ya-ni}{cry-\lsc{prog}-\lsc{1}}}%morpheme+gloss
\glotran{\pb{You} sing nicely. \pb{I}’m singing awfully.}{}%eng+spa trans
{}{}%rec - time

% 4
\gloexe{Glo6:Yatraqnin}{}{amv}%
{Yatraqnin\pb{qa}; mana yatraqnin\pb{qa} manayá.}%amv que first line
{\morglo{yatra-q-ni-n-qa}{know-\lsc{ag}-\lsc{euph}-\lsc{3}-\lsc{top}}\morglo{mana}{no}\morglo{yatra-q-ni-n-qa}{know	-\lsc{ag}-\lsc{euph}-\lsc{top}}\morglo{mana-yá}{no-\lsc{emph}}}%morpheme+gloss
\glotran{\pb{Those} of them who knew; not \pb{those} of them who didn’t know.}{}%eng+spa trans
{}{}%rec - time

% 5
\gloexe{Glo6:mikunchik}{}{amv}%
{Kanan\pb{qa} mikunchik munasanchik[ta] qullqi kaptin\pb{qa}.}%amv que first line
{\morglo{kanan-qa}{now-\lsc{top}}\morglo{miku-nchik}{eat-\lsc{1pl}}\morglo{muna-sa-nchik[-ta]}{want-\lsc{prf}-\lsc{1}-\lsc{acc}}\morglo{qullqi}{money}\morglo{ka-pti-n-qa}{be-\lsc{subds}-\lsc{3}-\lsc{top}}}%morpheme+gloss
\glotran{\pb{Now} we eat whatever we want when there’s money.}{}%eng+spa trans
{}{}%rec - time

% 6
\gloexe{Glo6:Llaqtaykipa}{}{amv}%
{Llaqtaykipa\pb{qa} ¿tarpunkichu sibadata?}%amv que first line
{\morglo{llaqta-yki-pa-qa}{town-\lsc{2}-\lsc{loc}-\lsc{top}}\morglo{tarpu-nki-chu}{plant-\lsc{2}-\lsc{q}}\morglo{sibada-ta}{barley-\lsc{acc}}}%morpheme+gloss
\glotran{In \pb{your town}, do you plant barley?}{}%eng+spa trans
{}{}%rec - time

% 7
\gloexe{Glo6:puriq}{}{amv}%
{Uray\pb{qa} puriq kani trakillawan trakinchikpis nananankama.}%amv que first line
{\morglo{uray-qa}{down.hill-\lsc{top}}\morglo{puri-q}{walk-\lsc{ag}}\morglo{ka-ni}{be-\lsc{1}}\morglo{traki-lla-wan}{foot-\lsc{rstr}-\lsc{instr}}\morglo{traki-nchik-pis}{foot-\lsc{1pl}-\lsc{add}}\morglo{nana-na-n-kama}{hurt-\lsc{nmlz-}\lsc{3}-\lsc{lim}}}%morpheme+gloss
\glotran{I would walk \pb{down hill} just on foot until our feet hurt.}{}%eng+spa trans
{}{}%rec - time

% 8
\gloexe{Glo6:Difindiwanchik}{}{amv}%
{Difindiwanchik malichukunapaq\pb{qa}.}%amv que first line
{\morglo{difindi-wa-nchik}{defend-\lsc{1.obj}-\lsc{1pl}}\morglo{malichu-kuna-paq-qa}{curse-\lsc{pl}-\lsc{abl}-\lsc{top}}}%morpheme+gloss
\glotran{It protects us against \pb{curses}.}{}%eng+spa trans
{}{}%rec - time

% 9
\gloexe{Glo6:pasiyuman}{}{ch}%
{Lluqsila pasiyuman yaykushpa\pb{qa} manaña puydila\uo{}chu piru.}%ch que first line
{\morglo{lluqsi-la}{go.out-\lsc{pst}}\morglo{pasiyu-man}{walk-\lsc{all}}\morglo{yayku-shpa-qa}{enter-\lsc{subis}-\lsc{top}}\morglo{mana-ña}{no-\lsc{disc}}\morglo{puydi-la-chu}{be.able-\lsc{pst}-\lsc{neg}}\morglo{piru}{but}}%morpheme+gloss
\glotran{They went out for a walk but \pb{when they went in}, they couldn’t.}{}%eng+spa trans
{}{}%rec - time

% 10
\gloexe{Glo6:Qipiruptinqa}{}{sp}%
{Qipiruptinqa~\dots{} chay kundur\pb{qa} qipiptin huk turuta pagaykun.}%sp que first line
{\morglo{qipi-ru-pti-n-qa}{carry-\lsc{urgt}-\lsc{subds}-\lsc{3}-\lsc{top}}\morglo{chay}{\lsc{dem.d}}\morglo{kundur-qa}{condor-\lsc{top}}\morglo{qipi-pti-n}{carry-\lsc{subds}-\lsc{3}}\morglo{huk}{one}\morglo{turu-ta}{bull-\lsc{acc}}\morglo{paga-yku-n}{pay-\lsc{excep}-\lsc{3}}}%morpheme+gloss
\glotran{\pb{When he carried her}, after the condor carried her, she payed him a bull.}{}%eng+spa trans
{}{}%rec - time

\subsection{Continuative \phono{-Raq}}
\phono{-Raq}\index[sub]{continuitive} --~realized in \CH{} as \phono{-laq}~(\ref{Glo6:Kichwallaktam}) and in all other dialects as \phono{-raq}~-- indicates continuity of action, state or quality.\\

% 1
\gloexe{Glo6:Kichwallaktam}{}{ch}%
{Kichwallaktam limakuya: kaytraw\pb{laq} manam kastillanukta lima:chu.}%ch que first line
{\morglo{kichwa-lla-kta-m}{Quechua-\lsc{rstr}-\lsc{acc}-\lsc{evd}}\morglo{lima-ku-ya-:}{talk-\lsc{refl}-\lsc{prog}-\lsc{1}}\morglo{kay-traw-laq}{\lsc{dem.p}-\lsc{loc}-\lsc{cont}}\morglo{mana-m}{no-\lsc{evd}}\morglo{kastillanu-kta}{Spanish-\lsc{acc}}\morglo{lima-:-chu}{talk-\lsc{1}-\lsc{neg}}}%morpheme+gloss
\glotran{I’m just talking Quechua. Here, \pb{still}, we don’t speak Spanish.}{}%eng+spa trans
{}{}%rec - time

\noindent
It can generally be translated ‘still’~(\ref{Glo6:Qamqa}--\ref{Glo6:Kamanpi}) or, negated, ‘yet’~(\ref{Glo6:Runtuwanmi}), (\ref{Glo6:mayqinniypis}).\\

% 2
\gloexe{Glo6:Qamqa}{}{ach}%
{Qamqa flaku\pb{raq}mi. Hawlapam qamtaqa wirayachisayki.}%ach que first line
{\morglo{qam-qa}{you-\lsc{top}}\morglo{flaku-raq-mi}{skinny-\lsc{cont}-\lsc{evd}}\morglo{hawla-pa-m}{cage-\lsc{loc}-\lsc{evd}}\morglo{qam-ta-qa}{you-\lsc{acc}-\lsc{top}}\morglo{wira-ya-chi-sayki}{fat-\lsc{inch}-\lsc{caus}-\lsc{1>2.fut}}}%morpheme+gloss
\glotran{You’re \pb{still} skinny. I’m going to fatten you up in a cage.}{}%eng+spa trans
{}{}%rec - time

% 3
\gloexe{Glo6:Taqsana}{}{amv}%
{Taqsana\pb{raq}tri. Millwata taqsashun.}%amv que first line
{\morglo{taqsa-na-raq-tri}{wash-\lsc{nmlz}-\lsc{cont}-\lsc{evc}}\morglo{millwa-ta}{wool-\lsc{acc}}\morglo{taqsa-shun}{wash-\lsc{1pl.fut}}}%morpheme+gloss
\glotran{It has to be cleaned \pb{still}. We have to clean the wool.}{}%eng+spa trans
{}{}%rec - time

% 4
\gloexe{Glo6:Kamanpi}{}{lt}%
{Kamanpi puñukuyaptin\pb{raq} tarirun.}%lt que first line
{\morglo{kama-n-pi}{bed-\lsc{3}-\lsc{loc}}\morglo{puñu-ku-ya-pti-n-raq}{sleep-\lsc{refl}-\lsc{prog}-\lsc{subds}-\lsc{3}-\lsc{cont}}\morglo{tari-ru-n}{find-\lsc{urgt}-\lsc{3}}}%morpheme+gloss
\glotran{He found him when he was sleeping \pb{still} in his bed.}{}%eng+spa trans
{}{}%rec - time

% 5
\gloexe{Glo6:Runtuwanmi}{}{amv}%
{Runtuwanmi qaquyanmi chaypa \pb{mana}\pb{raq}mi shakashwan.}%amv que first line
{\morglo{runtu-wan-mi}{egg-\lsc{instr}-\lsc{evd}}\morglo{qaqu-ya-n-mi}{massage-\lsc{prog}-\lsc{3}-\lsc{evd}}\morglo{chay-pa}{\lsc{dem.d}-\lsc{loc}}\morglo{mana-raq-mi}{no-\lsc{cont}-\lsc{evd}}\morglo{shakash-wan}{guinea.pig-\lsc{instr}}}%morpheme+gloss
\glotran{He’s massaging with an egg -- \pb{not yet} with the guinea pig.}{}%eng+spa trans
{}{}%rec - time

% 6
\gloexe{Glo6:mayqinniypis}{}{amv}%
{\pb{Mana}m mayqinniypis wañuni\pb{raq}chu.}%amv que first line
{\morglo{mana-m}{no-\lsc{evd}}\morglo{mayqin-ni-y-pis}{which-\lsc{euph}-\lsc{1}-\lsc{add}}\morglo{wañu-ni-raq-chu}{die-\lsc{1}-\lsc{cont}-\lsc{neg}}}%morpheme+gloss
\glotran{\pb{None} of us has died yet.}{}%eng+spa trans
{}{}%rec - time

\noindent
Marking rhetorical questions, it can indicate a kind of despair~(\ref{Glo6:Yawarnintachu}), (\ref{Glo6:gringukunaqa}).\\

% 7
\gloexe{Glo6:Yawarnintachu}{}{ach}%
{¿Yawarnintachu? ¿Imata\pb{raq} hurqura chay dimunyukuna?}%ach que first line
{\morglo{yawar-ni-n-ta-chu}{blood-\lsc{euph}-\lsc{3}-\lsc{acc}-\lsc{q}}\morglo{ima-ta-raq}{what-\lsc{acc}-\lsc{cont}}\morglo{hurqu-ra}{take.out-\lsc{pst}}\morglo{chay}{\lsc{dem.d}}\morglo{dimunyu-kuna}{Devil-\lsc{pl}}}%morpheme+gloss
\glotran{His blood? \pb{What in the world} did the devil suck out of him?}{}%eng+spa trans
{}{}%rec - time

% 8
\gloexe{Glo6:gringukunaqa}{}{ach}%
{Chay gringukunaqa altukunatash rin. ¿Imayna\pb{raq} chay runata wañuchin?}%ach que first line
{\morglo{chay}{\lsc{dem.d}}\morglo{gringu-kuna-qa}{gringo-\lsc{pl}-\lsc{top}}\morglo{altu-kuna-ta-sh}{high-\lsc{pl}-\lsc{acc}-\lsc{evr}}\morglo{ri-n}{go-\lsc{3}}\morglo{imayna-raq}{how-\lsc{cont}}\morglo{chay}{\lsc{dem.d}}\morglo{runa-ta}{\lsc{person}-\lsc{acc}}\morglo{wañu-chi-n}{die-\lsc{caus}-\lsc{3}}}%morpheme+gloss
\glotran{The gringos go to the heights, they say. \pb{How on earth} could they kill those people?}{}%eng+spa trans
{}{}%rec - time

\noindent
With subordinate clauses, it may indicate a prerequisite or a necessary condition for the event to take place, translating in English as ‘first’ or ‘not until’~(\ref{Glo6:ruwashpa}).\\

% 9
\gloexe{Glo6:ruwashpa}{}{amv}%
{Kisuta ruwashpa\pb{raq} trayamuyan.}%amv que first line
{\morglo{kisu-ta}{cheese-\lsc{acc}}\morglo{ruwa-shpa-raq}{make-\lsc{subis}-\lsc{cont}}\morglo{traya-mu-ya-n}{arrive-\lsc{cisl}-\lsc{prog}-\lsc{3}}}%morpheme+gloss
\glotran{\pb{Once} she makes the cheese, she’s coming.}{}%eng+spa trans
{}{}%rec - time

\noindent
\phono{Chay-raq} indicates an imminent future, translating in Andean Spanish \spanish{recién}~(\ref{Glo6:tapayan}). Employed as a coordinator, it implies a contrast between the coordinated elements (see~§~\ref{sec:coord}).\\

% 10
\gloexe{Glo6:tapayan}{}{amv}%
{Chay\pb{raq}mi tapayan. Qallaykuyani chay\pb{raq}.}%amv que first line
{\morglo{chay-raq-mi}{\lsc{dem.d}-\lsc{cont}-\lsc{evd}}\morglo{tapa-ya-n}{cover-\lsc{prog}-\lsc{3}}\morglo{qalla-yku-ya-ni}{begin-\lsc{excep}-\lsc{prog}-\lsc{1}}\morglo{chay-raq}{\lsc{dem.d}-\lsc{cont}}}%morpheme+gloss
\glotran{He’s \pb{just now going to} cap it. I’m \pb{just now} going to start.}{}%eng+spa trans
{}{}%rec - time

\subsection{Sequential \phono{-taq}}
\phono{-taq}\index[sub]{sequential} indicates the sequence of events~(\ref{Glo6:Tardiqa}).\\

% 1
\gloexe{Glo6:Tardiqa}{}{amv}%
{Tardiqa yapa listu suyan; yapa\pb{taq}shi trayarun.}%amv que first line
{\morglo{tardi-qa}{afternoon-\lsc{top}}\morglo{yapa}{again}\morglo{listu}{ready}\morglo{suya-n}{wait-\lsc{3}}\morglo{yapa-taq-shi}{again-\lsc{seq}-\lsc{evr}}\morglo{traya-ru-n}{arrive-\lsc{urgt}-\lsc{3}}}%morpheme+gloss
\glotran{In the afternoon, \pb{again}, ready, he waits. \pb{Then, again}, [the zombie] arrived.}{}%eng+spa trans
{}{}%rec - time

\noindent
Adelaar~(p.c.)\index[aut]{Adelaar, Willem F. H.} points out that in Ayacucho Quechua \phono{-ña-taq} is a fixed combination. It appears that may be the case here too~(\ref{Glo6:pikarushpa}--\ref{Glo6:makiywan}). In these examples \phono{-taq} seems to continue to indicate a sequence of events.\\

% 2
\gloexe{Glo6:pikarushpa}{}{amv}%
{Lliwta pikarushpa, kaymanñataq quturini trurani wakmanña\pb{taq}.}%amv que first line
{\morglo{lliw-ta}{all-\lsc{acc}}\morglo{pika-ru-shpa}{pick-\lsc{urgt}-\lsc{subds}}\morglo{kay-man-ña-taq}{\lsc{dem.d}-\lsc{all}-\lsc{disc}-\lsc{seq}}\morglo{qutu-ri-ni}{gather-\lsc{incep}-\lsc{1}}\morglo{trura-ni}{put-\lsc{1}}\morglo{wak-man-ña-taq}{\lsc{dem.p}-\lsc{all}-\lsc{disc}-\lsc{seq}}}%morpheme+gloss
\glotran{When I have all these sorted, \pb{then} I gather everything here and \pb{then} store it there.}{}%eng+spa trans
{}{}%rec - time

% 3
\gloexe{Glo6:takllawanmi}{}{ch}%
{Qaliqa takllawanmi halun. Qipantaña\pb{taq} kulpakta maqanchik pikuwan.}%ch que first line
{\morglo{qali-qa}{man-\lsc{top}}\morglo{taklla-wan-mi}{plow-\lsc{instr}-\lsc{evd}}\morglo{halu-n}{turn.earth-\lsc{3}}\morglo{qipa-n-ta-ña-taq}{behind-\lsc{3}-\lsc{acc}-\lsc{disc}-\lsc{seq}}\morglo{kulpa-kta}{clod-\lsc{acc}}\morglo{maqa-nchik}{hit-\lsc{1pl}}\morglo{piku-wan}{pick-\lsc{instr}}}%morpheme+gloss
\glotran{Men turn over the earth with a foot plow. Behind them, \pb{then}, we break up the clods with a pick.}{}%eng+spa trans
{}{}%rec - time

% 4
\gloexe{Glo6:makiywan}{}{amv}%
{Ñuqapa makiywan aytrichiyanmi. Kanan trakillaña\pb{taq}. Huknin makiwanña\pb{taq} kananmi.}%amv que first line
{\morglo{ñuqa-pa}{I-\lsc{gen}}\morglo{maki-y-wan}{hand-\lsc{1}-\lsc{instr}}\morglo{aytri-chi-ya-n-mi}{stir-\lsc{caus}-\lsc{prog}-\lsc{3}-\lsc{evd}}\morglo{kanan}{now}\morglo{traki-lla-ña-taq}{foot-\lsc{rstr}-\lsc{disc}-\lsc{seq}}\morglo{huk-ni-n}{one-\lsc{euph}-\lsc{3}}\morglo{maki-wan-ña-taq}{hand-\lsc{instr}-\lsc{disc}-\lsc{seq}}\morglo{kanan-mi}{now-\lsc{evd}}}%morpheme+gloss
\glotran{He’s stirring it with my hand. Now, the foot. Now with the other hand.}{}%eng+spa trans
{}{}%rec - time

\noindent
In a question introduced by an interrogative (\phono{pi-}, \phono{ima-}~\dots) \phono{-taq} attaches to the interrogative in case it is the only word in the phrase or, in case the phrase includes two or more words, to the final word in the phrase (\ref{Glo6:Ishpaykuruwan}--\ref{Glo6:Imanashaq}).\\

% 5
\gloexe{Glo6:Ishpaykuruwan}{}{amv}%
{¡Ishpaykuruwan! ¿Imapaq\pb{taq} ishpan?}%amv que first line
{\morglo{ishpa-yku-ru-wa-n}{urinate-\lsc{excep}-\lsc{urgt}-\lsc{1.obj}-\lsc{3}}\morglo{ima-paq-taq}{what-\lsc{purp}-\lsc{seq}}\morglo{ishpa-n}{urinate-\lsc{3}}}%morpheme+gloss
\glotran{It urinated on me! \pb{Why} does it urinate?}{}%eng+spa trans
{}{}%rec - time

% 6
\gloexe{Glo6:rikuq}{}{amv}%
{¿Ima rikuq\pb{taq} karqa sapatillayki?}%amv que first line
{\morglo{ima}{what}\morglo{rikuq-taq}{color-\lsc{seq}}\morglo{ka-rqa}{be-\lsc{pst}}\morglo{sapatilla-yki}{shoe-\lsc{2}}}%morpheme+gloss
\glotran{\pb{What color} were your shoes?}{}%eng+spa trans
{}{}%rec - time

% 7
\gloexe{Glo6:Imanashaq}{}{lt}%
{¿Imanashaq\pb{taq}? Diosllatañatriki.}%lt que first line
{\morglo{ima-na-shaq-taq}{what-\lsc{vrbz}-\lsc{1.fut}-\lsc{seq}}\morglo{Dios-lla-ta-ña-tr-iki}{God-\lsc{rstr}-\lsc{acc}-\lsc{disc}-\lsc{evc}-\lsc{iki}}}%morpheme+gloss
\glotran{\pb{What am I going to do}? It’s for God already.}{}%eng+spa trans
{}{}%rec - time

\noindent
In this capacity, \phono{-taq} may be the most transparent of the enclitics attaching to \phono{q}-phrases. In a clause with a conditional or in a subordinate clause, \phono{-taq} can indicate a warning~(\ref{Glo6:Kurasunniyman}).\\

% 8
\gloexe{Glo6:Kurasunniyman}{}{amv}%
{Kurasunniyman shakashta trurayan. Ñuqa niyani “¡Kaniruwaptinña\pb{taq}!”}%amv que first line
{\morglo{kurasun-ni-y-man}{heart-\lsc{euph}-\lsc{1}-\lsc{all}}\morglo{shakash-ta}{guinea.pig-\lsc{acc}}\morglo{trura-ya-n}{put-\lsc{prog}-\lsc{3}}\morglo{ñuqa}{I}\morglo{ni-ya-ni}{say-\lsc{prog}-\lsc{1}}\morglo{kani-ru-wa-pti-n-ña-taq}{bite-\lsc{urgt}-\lsc{1.obj}-\lsc{subds}-\lsc{3}-\lsc{disc}-\lsc{seq}}}%morpheme+gloss
\glotran{He’s putting the guinea pig over my heart. I’m saying, “\pb{Be careful} it doesn’t bite me!”}{}%eng+spa trans
{}{}%rec - time

\noindent
\phono{-taq} also functions as a conjunction~(\ref{Glo6:puchkawan}) (see~§~\ref{sec:coord}).\\

% 9
\gloexe{Glo6:puchkawan}{}{amv}%
{Warmiña\pb{taq} puchkawan qariña\pb{taq} tihiduwan.}%amv que first line
{\morglo{warmi-ña-taq}{women-\lsc{disc}-\lsc{seq}}\morglo{puchka-wan}{spinning-\lsc{instr}}\morglo{qari-ña-taq}{man-\lsc{disc}-\lsc{seq}}\morglo{tihidu-wan}{weaving-\lsc{instr}}}%morpheme+gloss
\glotran{Women with spinning \pb{and} men with weaving.}{}%eng+spa trans
{}{}%rec - time

\subsection{Emotive \phono{-ya}}\label{ssec:emotive}
\phono{-ya} indicates regret or resignation\index[sub]{emotive}. It can be translated ‘alas’ or ‘regretfully’ or with a sigh. Not very widely employed.\\

% 1
\gloexe{Glo6:Hinashpaqa}{}{amv}%
{Hinashpaqa\pb{ya}, “Wañurachishaqña wakchachaytaqa dimasllam sufriyan.”}%amv que first line
{\morglo{hinashpa-qa-ya}{then-\lsc{top}-\lsc{emo}}\morglo{wañu-ra-chi-shaq-ña}{die-\lsc{urgt}-\lsc{caus}-\lsc{1.fut}-\lsc{disc}}\morglo{wakcha-cha-y-ta-qa}{lamb-\lsc{dim}-\lsc{1}-\lsc{acc}-\lsc{top}}\morglo{dimas-lla-m}{too.much-\lsc{rstr}-\lsc{evd}}\morglo{sufri-ya-n}{suffer-\lsc{prog}-\lsc{3}}}%morpheme+gloss
\glotran{Then, \pb{alas}, “I’m going to kill my little lamb already -- he’s suffering too much,” [I said].}{}%eng+spa trans
{}{}%rec - time

% 2
\gloexe{Glo6:runakunaqa}{}{amv}%
{Unay runakunaqa yatrayan masta, masta\pb{ya}, lliwta~\dots{} aaaa.}%amv que first line
{\morglo{unay}{before}\morglo{runa-kuna-qa}{person-\lsc{pl}-\lsc{top}}\morglo{yatra-ya-n}{know-\lsc{prog}-\lsc{3}}\morglo{mas-ta}{more-\lsc{acc}}\morglo{mas-ta-ya}{more-\lsc{acc}-\lsc{emo}}\morglo{lliw-ta}{all-\lsc{acc}}\morglo{aaaa}{ahhh}}%morpheme+gloss
\glotran{In the old days, people knew more, more, everything, \pb{ahhh}.}{}%eng+spa trans
{}{}%rec - time

\subsection{Evidence}\label{ssec:evidence}
Evidentials\index[sub]{evidentials} indicate the type of the speaker’s source of information. \SYQ, like most\footnote{Note, though, that Huallaga Q counts four evidentials, (\phono{-mi}, \phono{-shi}, \phono{-chi}, snd \phono{-chaq}) (Weber 1989:76). South Conchucos Q counts six, (\phono{-mi}, \phono{-shi}, \phono{-chi}, \phono{-cha:}, and \phono{-cher}); Sihuas, too, counts six (Hintz and Hintz 2014).} other Quechuan languages, counts three evidential suffixes: direct \phono{-mi}~(\ref{Glo6:Taytacha}--\ref{Glo6:puntraw}), reportative \phono{-shi}~(\ref{Glo6:Radyukunapa}--\ref{Glo6:Qarinta}), and conjectural \phono{-tri}~(\ref{Glo6:trayarachiptiki}--\ref{Glo6:Wasiy}) (\ie~the speaker has her own evidence for P (generally visual); the speaker learned P from someone else; or the speaker infers P based on some other evidence). Following a short vowel, these are realized as \phono{-m}, \phono{sh}, and \phono{-tr}, respectively~(\ref{Glo6:puntraw}), (\ref{Glo6:Qarinta}), (\ref{Glo6:Wasiy}).\\

% 1
\gloexe{Glo6:Taytacha}{}{amv}%
{Taytacha José irransakurqa chaypa\pb{m}.}%amv que first line
{\morglo{tayta-cha}{father-\lsc{dim}}\morglo{José}{José}\morglo{irransa-ku-rqa}{herranza-\lsc{refl}-\lsc{pst}}\morglo{chay-pa-m}{\lsc{dem.d}-\lsc{loc}-\lsc{evd}}}%morpheme+gloss
\glotran{My grandfather José held herranzas \pb{there}.}{}%eng+spa trans
{}{}%rec - time

% 2
\gloexe{Glo6:Trurawarqaya}{}{amv}%
{Trurawarqaya huk ratu. Manayá puchukachiwarqachu. Trurawarqa\pb{m}.}%amv que first line
{\morglo{trura-wa-rqa-yá}{put-\lsc{1.obj}-\lsc{pst}-\lsc{emph}}\morglo{huk}{one}\morglo{ratu}{moment}\morglo{mana-yá}{no-\lsc{emph}}\morglo{puchuka-chi-wa-rqa-chu}{finish-\lsc{caus}-\lsc{1.obj}-\lsc{pst}-\lsc{neg}}\morglo{trura-wa-rqa-m}{put-\lsc{1.obj}-\lsc{pst}-\lsc{evd}}}%morpheme+gloss
\glotran{They put me in [school] a short while. They didn’t have me finish, but they did \pb{put me in}.}{}%eng+spa trans
{}{}%rec - time

% 3
\gloexe{Glo6:puntraw}{}{ach}%
{Qayna puntraw qanin puntrawlla\pb{m} trayamura:.}%ach que first line
{\morglo{qayna}{previous}\morglo{puntraw}{day}\morglo{qanin}{day.before.yesterday}\morglo{puntraw-lla-m}{day-\lsc{rstr}-\lsc{evd}}\morglo{traya-mu-ra-:}{arrive-\lsc{cisl}-\lsc{pst}-\lsc{1}}}%morpheme+gloss
\glotran{I arrived yesterday, \pb{just the day} before yesterday.}{}%eng+spa trans
{}{}%rec - time

% 4
\gloexe{Glo6:Radyukunapa}{}{sp}%
{Radyukunapa rimayta rimayan. Lluqsiyamun\pb{shi} tirrurista. Tirrurista rikariyamun\pb{shi}.}%sp que first line
{\morglo{radyu-kuna-pa}{radio-\lsc{pl}-\lsc{loc}}\morglo{rima-y-ta}{talk-\lsc{inf}-\lsc{acc}}\morglo{rima-ya-n}{talk-\lsc{prog}-\lsc{3}}\morglo{lluqsi-ya-mu-n-shi}{go.out-\lsc{prog}-\lsc{cisl}-\lsc{3}-\lsc{evr}}\morglo{tirrurista}{terrorist}\morglo{tirrurista}{terrorist}\morglo{rikari-ya-mu-n-shi}{appear-\lsc{prog}-\lsc{cisl}-\lsc{3}-\lsc{evr}}}%morpheme+gloss
\glotran{On the radio they talk for the sake of talking. Terrorists \pb{are coming out, they say}. Terrorists \pb{are appearing, they say}.}{}%eng+spa trans
{}{}%rec - time

% 5
\gloexe{Glo6:uchukllapa}{}{amv}%
{Chay uchukllapa pashñataq uywakuptinñataq\pb{shi} maqtaqa aparqa mikunanta.}%amv que first line
{\morglo{chay}{\lsc{dem.d}}\morglo{uchuk-lla-pa}{small-\lsc{rstr}-\lsc{loc}}\morglo{pashña-taq}{girl-\lsc{acc}}\morglo{uywa-ku-pti-n-ña-taq-shi}{raise-\lsc{refl}-\lsc{subds}-\lsc{3}-\lsc{disc}-\lsc{seq}-\lsc{evr}}\morglo{maqta-qa}{young.man-\lsc{top}}\morglo{apa-rqa}{bring-\lsc{pst}}\morglo{miku-na-n-ta}{eat-\lsc{nmlz}-\lsc{3}-\lsc{acc}}}%morpheme+gloss
\glotran{When \pb{he raised} the girl in that cave, the man brought her his food, \pb{they say}.}{}%eng+spa trans
{}{}%rec - time

% 6
\gloexe{Glo6:Qarinta}{}{amv}%
{Qarinta\pb{sh} wañurachin mashanta\pb{sh} wañurachin.}%amv que first line
{\morglo{qari-n-ta-sh}{man-\lsc{3}-\lsc{acc}-\lsc{evr}}\morglo{wañu-ra-chi-n}{die-\lsc{urgt}-\lsc{caus}-\lsc{3}}\morglo{masha-n-ta-sh}{son.in.law-\lsc{3}-\lsc{acc}-\lsc{evr}}\morglo{wañu-ra-chi-n}{die-\lsc{urgt}-\lsc{caus}-\lsc{3}}}%morpheme+gloss
\glotran{She killed her \pb{husband, they say}; she killed her \pb{son-in-law, they say}.}{}%eng+spa trans
{}{}%rec - time

% 7
\gloexe{Glo6:trayarachiptiki}{}{amv}%
{Qiñwalman trayarachiptiki wañukunman\pb{tri}.}%amv que first line
{\morglo{qiñwal-man}{quingual.grove-\lsc{all}}\morglo{traya-ra-chi-pti-ki}{arrive-\lsc{urgt}-\lsc{caus}-\lsc{subds}-\lsc{2}}\morglo{wañu-ku-n-man-tri}{die-\lsc{refl}-\lsc{3}-\lsc{cond}-\lsc{evc}}}%morpheme+gloss
\glotran{If you make her go all the way to the quingual grove, she might die.}{}%eng+spa trans
{}{}%rec - time

% 8
\gloexe{Glo6:Suwawan}{}{lt}%
{Suwawan\pb{tri}. Durasnuy kara mansanay kara qanin puntraw.}%lt que first line
{\morglo{suwa-wa-n-tri}{rob-\lsc{1.obj}-\lsc{3}-\lsc{evr}}\morglo{durasnu-y}{peach-\lsc{1}}\morglo{ka-ra}{be-\lsc{pst}}\morglo{mansana-y}{apple-\lsc{1}}\morglo{ka-ra}{be-\lsc{pst}}\morglo{qanin}{previous}\morglo{puntraw}{day}}%morpheme+gloss
\glotran{They \pb{may have robbed} me. The day before yesterday I had peaches and apples.}{}%eng+spa trans
{}{}%rec - time

% 9
\gloexe{Glo6:Wasiy}{}{amv}%
{Wasiy rahasa kayan. Saqaykurunqa\pb{tr}.}%amv que first line
{\morglo{wasi-y}{house-\lsc{1}}\morglo{raha-sa}{crack-\lsc{prf}}\morglo{ka-ya-n}{be-\lsc{prog}-\lsc{3}}\morglo{saqa-yku-ru-nqa-tr}{go.down-\lsc{excep}-\lsc{urgt}-\lsc{3.fut}-\lsc{evc}}}%morpheme+gloss
\glotran{My house is cracked. \pb{It’s going to fall down}.}{}%eng+spa trans
{}{}%rec - time

The evidential system of \SYQ{} is unusual among Quechuan languages, however, in that it overlays the three-way distinction standard to Quechua with a second three-way distinction. The set of evidentials in \SYQ{} thus counts nine members: \phono{-mI}, \phono{-m-ik}, and \phono{-m-iki}; \phono{-shI}, \phono{-sh-ik}, and \phono{-sh-iki}; and \phono{-trI}, \phono{-tr-ik}, and \phono{-tr-iki}. The \phono{-I}, \phono{-ik}, and \phono{-iki} forms are not allomorphs: they receive different interpretations, generally indicating increasing degrees of evidence strength or, in the case of modalized verbs, increasing modal force. §~\ref{ssec:evidence} describes this system in some detail. For further formal analysis, see \citet{Shimelman12}.\index[aut]{Shimelman, Aviva}

In addition to indicating the speaker’s information type, evidentials also function to indicate focus or comment and to complete copular predicates (for further discussion and examples, see §~\ref{sec:emphasis} and~\ref{sec:equative} on emphasis and equatives).

Evidentials are subject to the following distributional restrictions. They never attach to the topic or subject; these are, rather, marked with \phono{-qa}. In content questions, the evidential attaches to the question word or to the last word of the questioned phrase~(\ref{Glo6:chay}) (see~§~\ref{sec:interr} on interrogation).\\

% 10
\gloexe{Glo6:chay}{}{amv}%
{¿May\pb{mi} chay warmi?}%amv que first line
{\morglo{may-mi}{where-\lsc{evd}}\morglo{chay}{\lsc{dem.d}}\morglo{warmi}{woman}}%morpheme+gloss
\glotran{\pb{Where} is that woman?}{}%eng+spa trans
{}{}%rec - time

\noindent
Evidentials do not appear in commands or injunctions~(\ref{Glo6:Ruwaruchun}); finally, only one evidential may occur per clause~(\ref{Glo6:Vakay}).\\

% 11
\gloexe{Glo6:Ruwaruchun}{}{amv}%
{¡Ruwaruchun*mi/shi/tri!}% que first line
{\morglo{ruwa-ru-chun-*mi/shi/tri}{make-\lsc{urgt}-\lsc{injunc}-\lsc{evd}-\lsc{evr}-\lsc{evc}}}%morpheme+gloss
\glotran{\pb{Let} him do it!}{}%eng+spa trans
{}{}%rec - time

% 12
\gloexe{Glo6:Vakay}{}{amv}%
{¡Vakay wira wira\pb{m}, matraypi puñushpa, allin pastuta mikushpa\pb{m}.}% que first line
{\morglo{vaka-y}{cow-\lsc{1}}\morglo{wira}{fat}\morglo{wira-m}{fat-\lsc{evd}}\morglo{matray-pi}{cave-\lsc{loc}}\morglo{puñu-shpa}{sleep-\lsc{subis}}\morglo{allin}{good}\morglo{pastu-ta}{pasture.grass-\lsc{acc}}\morglo{miku-shpa-m}{eat-\lsc{refl}-\lsc{evd}}}%morpheme+gloss
\glotran{My cow is really fat, sleeping in a cave and eating good pasture grass.}{}%eng+spa trans
{}{}%rec - time

All three evidentials are interpreted as assertions. The first, \phono{-mI}, is generally left untranslated in Spanish; the second, \phono{-shI}, is often rendered \phono{dice} ‘they say’; the third is reflected in a change in verb tense or mode (see~§~\ref{ssec:conjectural}). The difference between the three is a matter, first, of whether or not evidence is from personal experience, and, second, whether that evidence supports the proposition, \phono{p}, immediately under the scope of the evidential or another set of propositions, \phono{P’}, that are evidence for \phono{p}, as represented in Table \ref{Tab31}.

% Table 31
\begin{table}
\small\centering
\caption{Evidential schema: “evidence from” by “evidence for”}\label{Tab31}
\begin{tabular}{lll}
\lsptoprule
	& Supports scope 			& Supports \phono{P’}		\\
	& proposition \phono{p} 	& evidence for \phono{p}		\\
\midrule
Direct 	&\lsc{direct} &\lsc{conjectural}\\
(personal experience) evidence 			& \phono{-mI} & \phono{-trI}		\\[1ex]
Reportative 	&\lsc{reportative}&\lsc{conjectural}\\
(non-personal experience) evidence 	& \phono{-shI} & \phono{-trI}	\\
\lspbottomrule
\end{tabular}
\end{table}

So, employing \phono{-mI}(\phono{p}), the speaker asserts predicate \phono{p} and represents that she has personal-experience evidence for \phono{p}; employing \phono{-shI}(\phono{p}), the speaker asserts \phono{p} and refers the hearer to another source for evidence for \phono{p}; and employing \phono{-trI}(\phono{p}), the speaker asserts \phono{p} and represents that \phono{p} is a conjecture from \phono{P’}, propositions for which she has either \phono{-mI}-type or \phono{-shI}-type evidence or both. That is, although \SYQ{} counts three evidential suffixes, it counts only two evidence types, direct and reportative; these two are jointly exhaustive. §~\ref{ssec:direct}--\ref{ssec:conjectural} cover \phono{-mI}, \phono{-shI}, and \phono{trI}, in turn. §~\ref{ssec:evidmodifi} covers the evidential modifiers, \phono{-ari} and \phono{-ik/iki}.

\subsubsection{Direct \phono{-mI}}\label{ssec:direct}
\phono{-mI}\index[sub]{evidentials!direct} indicates that the speaker speaks from direct experience. Unlike \phono{-shI} and \phono{-trI}, it is generally left untranslated. Note that in the examples below, with the exception of~(\ref{Glo6:Vakaqa}), the speaker’s knowledge is \emph{not} the product of visual experience.\\

% 1 (3)
\gloexe{Glo6:Vakaqa}{}{amv}%
{Vakaqa kaypa waqrayuq\pb{mi}ki kayan.}%amv que first line
{\morglo{vaka-qa}{cow-\lsc{top}}\morglo{kay-pa}{\lsc{dem.p}-\lsc{loc}}\morglo{waqra-yuq-m-iki}{horn-\lsc{poss}-\lsc{evd}-\lsc{iki}}\morglo{ka-ya-n}{be-\lsc{prog}-\lsc{3}}}%morpheme+gloss
\glotran{The cows here \pb{have horns}.}{}%eng+spa trans
{}{}%rec - time

% 2 (1)
\gloexe{Glo6:pakarayan}{}{amv}%
{Piñiy\pb{mi} pakarayan wasiypa wak ichuypa ukunpa.}%amv que first line
{\morglo{piñi-y-mi}{necklace-\lsc{1}-\lsc{evd}}\morglo{paka-ra-ya-n}{hide-\lsc{unint}-\lsc{intens}-\lsc{3}}\morglo{wasi-y-pa}{house-\lsc{1}-\lsc{loc}}\morglo{wak}{\lsc{dem.d}}\morglo{ichuy-pa}{straw-\lsc{gen}}\morglo{uku-n-pa}{inside-\lsc{3}-\lsc{loc}}}%morpheme+gloss
\glotran{\pb{My necklace} is hidden in my house under the straw.}{}%eng+spa trans
{}{}%rec - time

% 3 (2)
\gloexe{Glo6:Chaywan}{}{amv}%
{Chaywan\pb{mi} pwirtata ruwayani. Mana\pb{m} achkataq ruwanichu.}%amv que first line
{\morglo{chay-wan-mi}{\lsc{dem.d}-\lsc{instr}-\lsc{evd}}\morglo{pwirta-ta}{door-\lsc{acc}}\morglo{ruwa-ya-ni}{make-\lsc{prog}-\lsc{1}}\morglo{mana-m}{no-\lsc{evd}}\morglo{achka-taq}{a.lot-\lsc{acc}}\morglo{ruwa-ni-chu}{make.\lsc{1}-\lsc{neg}}}%morpheme+gloss
\glotran{I make doors with this. I don’t make a lot.}{}%eng+spa trans
{}{}%rec - time

% 4
\gloexe{Glo6:Karrupis}{}{ach}%
{Karrupis ashnakuyan\pb{mi}.}%ach que first line
{\morglo{karru-pis}{car-\lsc{add}}\morglo{ashna-ku-ya-n-mi}{smell-\lsc{refl}-\lsc{prog}-\lsc{3}-\lsc{evd}}}%morpheme+gloss
\glotran{The buses, too, \pb{stink}.}{}%eng+spa trans
{}{}%rec - time

% 5
\gloexe{Glo6:Qunirirachishunki}{}{ach}%
{Qunirirachishunki. Kaliyntamanchik\pb{mi}.}%ach que first line
{\morglo{quni-ri-ra-chi-shu-nki}{warm-\lsc{incep}-\lsc{caus}-\lsc{2.obj}-\lsc{2}}\morglo{kaliynta-ma-nchik-mi}{warm-\lsc{1.obj}-\lsc{1pl}-\lsc{evd}}}%morpheme+gloss
\glotran{It warms you up. \pb{It warms us up}.}{}%eng+spa trans
{}{}%rec - time

\subsubsection{Reportative \phono{-shI}}
\phono{-shI}\index[sub]{evidentials!reportative} indicates that the speaker’s evidence does not come from personal experience~(\ref{Glo6:Awkichanka}--\ref{Glo6:Lata}).\\

% 1
\gloexe{Glo6:Awkichanka}{}{amv}%
{Awkichanka urqupaqa inkantu\pb{sh} -- karru\pb{sh} chinkarurqa qutrapa.}%amv que first line
{\morglo{Awkichanka}{Awkichanka}\morglo{urqu-pa-qa}{hill-\lsc{loc}-\lsc{top}}\morglo{inkantu-sh}{spirit-\lsc{evr}}\morglo{karru-sh}{car-\lsc{evr}}\morglo{chinka-ru-rqa}{lose-\lsc{urgt}-\lsc{pst}}\morglo{qutra-pa}{lake-\lsc{loc}}}%morpheme+gloss
\glotran{In the hill Okichanka, there is \pb{a spirit, they say} -- a car was lost in a reservoir.}{}%eng+spa trans
{}{}%rec - time

% 2
\gloexe{Glo6:Mashwaqa}{}{ch}%
{Mashwaqa prustatapaq\pb{shi} allin.}%ch que first line
{\morglo{mashwa-qa}{mashua-\lsc{top}}\morglo{prustata-paq-shi}{prostate-\lsc{ben}-\lsc{evr}}\morglo{allin}{good}}%morpheme+gloss
\glotran{Mashua is good for the \pb{prostate}, \pb{they say}.}{}%eng+spa trans
{}{}%rec - time

% 3
\gloexe{Glo6:Chaypash}{}{amv}%
{Chaypa\pb{sh} runtuta mikuchishunki.}%amv que first line
{\morglo{chay-pa-sh}{\lsc{dem.d}-\lsc{loc}-\lsc{evr}}\morglo{runtu-ta}{egg-\lsc{acc}}\morglo{miku-chi-shu-nki}{eat-\lsc{caus}-\lsc{2.obj}-\lsc{2}}}%morpheme+gloss
\glotran{They’ll feed you eggs \pb{there}, \pb{they say}.}{}%eng+spa trans
{}{}%rec - time

% 4
\gloexe{Glo6:Lata}{}{ach}%
{Lata-wan yanu-shpa-taq-\pb{shi} runa-ta-pis miku-ru-ra.}%ach que first line
{\morglo{lata-wan}{can-\lsc{instr}}\morglo{yanu-shpa-taq-shi}{cook-\lsc{subis}-\lsc{seq}-\lsc{evr}}\morglo{runa-ta-pis}{person-\lsc{acc}-\lsc{add}}\morglo{miku-ru-ra}{eat-\lsc{urgt}-\lsc{pst}}}%morpheme+gloss
\glotran{They [the Shining Path] even \pb{cooked} people in metal pots and ate them, \pb{they say}.}{}%eng+spa trans
{}{}%rec - time

\noindent
It is used systematically in stories~(\ref{Glo6:Unay}), (\ref{Glo6:Chaypaqshi}).\\

% 5
\gloexe{Glo6:Unay}{}{sp}%
{Unay\pb{shi} kara huk asnu.}%sp que first line
{\morglo{unay-shi}{before-\lsc{evr}}\morglo{ka-ra}{be-\lsc{pst}}\morglo{huk}{one}\morglo{asnu}{donkey}}%morpheme+gloss
\glotran{\pb{Once upon a time, they say} there was a mule.}{}%eng+spa trans
{}{}%rec - time

% 6
\gloexe{Glo6:Chaypaqshi}{}{lt}%
{Chaypaq\pb{shi} kutirun maman kaqta papanin kaqta.}%lt que first line
{\morglo{chay-paq-shi}{\lsc{dem.d}-\lsc{abl}-\lsc{evr}}\morglo{kuti-ru-n}{return-\lsc{urgt}-\lsc{3}}\morglo{mama-n}{mother-\lsc{3}}\morglo{ka-q-ta}{be-\lsc{ag}-\lsc{acc}}\morglo{papa-ni-n}{father-\lsc{euph}-\lsc{3}}\morglo{ka-q-ta}{be-\lsc{ag}-\lsc{acc}}}%morpheme+gloss
\glotran{He returned \pb{from there, they say}, to his mother’s place, to his father’s place.}{}%eng+spa trans
{}{}%rec - time

\subsubsection{Conjectural \phono{-trI}}\label{ssec:conjectural}
\phono{-trI}\index[sub]{evidentials!conjectural} indicates that the speaker does not have evidence for the proposition directly under the scope of the evidential, but is, rather, conjecturing to that proposition from others for which she does have evidence~(\ref{Glo6:Awayan}--\ref{Glo6:hamuyan}).\\

% 1
\gloexe{Glo6:Awayan}{}{amv}%
{Awayan\pb{tr}iki kamata.}%amv que first line
{\morglo{awa-ya-n-tr-iki}{weave-\lsc{prog}-\lsc{evr}-\lsc{iki}}\morglo{kama-ta}{blanket-\lsc{acc}}}%morpheme+gloss
\glotran{\pb{He must be weaving} a blanket.}{}%eng+spa trans
{}{}%rec - time

% 2
\gloexe{Glo6:kayachuwan}{}{amv}%
{Wañuypaqpis kayachuwan\pb{tr}iki.}%amv que first line
{\morglo{wañu-y-paq-pis}{die-\lsc{inf}-\lsc{abl}-\lsc{add}}\morglo{ka-ya-chuwan-tr-iki}{be-\lsc{prog}-\lsc{1pl.cond}-\lsc{evc}-\lsc{iki}}}%morpheme+gloss
\glotran{\pb{We could be} also about to die.}{}%eng+spa trans
{}{}%rec - time

% 3
\gloexe{Glo6:Kukachankunata}{}{amv}%
{Kukachankunata aparuptiyqa tiyaparuwanqa\pb{tr}ik.}%amv que first line
{\morglo{kuka-cha-n-kuna-ta}{coca-\lsc{dim}-\lsc{3}-\lsc{pl}-\lsc{acc}}\morglo{apa-ru-pti-y-qa}{bring-\lsc{urgt}-\lsc{subds}-\lsc{1}-\lsc{top}}\morglo{tiya-pa-ru-wa-nqa-tr-ik}{sit-\lsc{ben}-\lsc{urgt}-\lsc{1.obj}-\lsc{evc}-\lsc{ik}}}%morpheme+gloss
\glotran{If I bring them their coca, \pb{they’ll accompany me sitting}.}{}%eng+spa trans
{}{}%rec - time

% 4
\gloexe{Glo6:Chayman}{}{ach}%
{Chayman\pb{tr}ik ayarikura.}%ach que first line
{\morglo{chay-man-tr-ik}{\lsc{dem.d}-\lsc{all}-\lsc{evc}-\lsc{ik}}\morglo{aya-ri-ku-ra}{cadaver-\lsc{incep}-\lsc{refl}-\lsc{pst}}}%morpheme+gloss
\glotran{She \pb{must} have become a cadaver.}{}%eng+spa trans
{}{}%rec - time

% 5
\gloexe{Glo6:Upyachinman}{}{ch}%
{Upyachinman\pb{tri}.}%ch que first line
{\morglo{upya-chi-ma-n-tri}{drink-\lsc{caus}-\lsc{1.obj}-\lsc{3}-\lsc{evc}}}%morpheme+gloss
\glotran{She \pb{might} make me drink.}{}%eng+spa trans
{}{}%rec - time

% 6
\gloexe{Glo6:rikuyan}{}{ach}%
{Yakuña\pb{tr} rikuyan pampantaqa.}%ach que first line
{\morglo{yaku-ña-tr}{water-\lsc{disc}-\lsc{evc}}\morglo{ri-ku-ya-n}{go-\lsc{refl}-\lsc{prog}-\lsc{3}}\morglo{pampa-n-ta-qa}{ground-\lsc{3}-\lsc{acc}-\lsc{top}}}%morpheme+gloss
\glotran{\pb{Water should} already be running along the ground.}{}%eng+spa trans
{}{}%rec - time

% 7
\gloexe{Glo6:Allintaqa}{}{sp}%
{Allintaqa. Kapas\pb{tr}iki palabrata kichwapa apakunqa kananpis.}%sp que first line
{\morglo{allin-ta-qa}{good-\lsc{acc}-\lsc{top}}\morglo{kapas-tr-iki}{possible-\lsc{evc}-\lsc{iki}}\morglo{palabra-ta}{word-\lsc{acc}}\morglo{kichwa-pa}{Quechua-\lsc{gen}}\morglo{apa-ku-nqa}{\lsc{bring}-\lsc{refl}-\lsc{3.fut}}\morglo{kanan-pis}{now-\lsc{add}}}%morpheme+gloss
\glotran{Good. \pb{Maybe} they’ll bring Quechua now, too.}{}%eng+spa trans
{}{}%rec - time

% 8
\gloexe{Glo6:hamuyan}{}{amv}%
{Ayvis kumpañaw hamuyan -- wañuypaqpis kayachuwantriki.}%
{\morglo{ayvis}{sometimes}\morglo{kumpañaw}{accompanied}\morglo{hamu-ya-n}{come-\lsc{prog}-\lsc{3}}\morglo{wañu-y-paq-pis}{die-\lsc{1}-\lsc{purp}-\lsc{add}}\morglo{ka-ya-chuwan-tr-iki}{be-\lsc{prog}-\lsc{1pl}.\lsc{cond}-\lsc{evc}-\lsc{iki}}}%morpheme+gloss
\glotran{Sometimes someone comes accompanied -- we might be also about to die.}{}%eng+spa trans
{}{}%rec - time

\subsubsection{Evidential modification}\label{ssec:evidmodifi}
\SYQ{} counts four evidential modifiers\index[sub]{evidentials!modification}, \phono{-ari} and the set \uo, \phono{-ik} and \phono{-iki}. §~\ref{par:assertive} and~\ref{par:evistre} cover \phono{-ari} and \phono{-\uo/-ik/iki}, respectively. The latter largely repeats~\citet{Shimelman12}.

\paragraph{Assertive force \phono{-aRi}}\label{par:assertive}
\phono{-aRi}\index[sub]{evidentials!assertive force} --~realized \phono{-ali} in \CH{}~(\ref{Glo6:Wayrakuyan}) and \phono{-ari} in all other dialects~-- indicates conviction on the part of the speaker.\footnote{The Quechuas of (at least) Ancash-Huailas \citet[151]{Parker76gram},\index[aut]{Parker, Gary J.} Cajamarca-Canaris \citet[158]{Quesada76}\index[aut]{Quesada Castillo, Félix} and Junin-Huanca \citet[238--9]{CerroP76a}\index[aut]{Cerrón-Palomino, Rodolfo M.} have suffixes \phono{-rI}, \phono{-rI} and \phono{-ari}, respectively, which, like the \SYQ{} \phono{-k} succeed evidentials and are most often translated \spanish{pues} ‘then’. It seems unlikely that the \lsc{ahq}, \lsc{ccq} and \lsc{jhq} forms correspond to the \phono{-k} or \phono{-ki} of \SYQ. First, unlike \phono{-ik} or \phono{-iki}, \phono{-rI} and \phono{-ari} may appear independent of any evidential and they may function as general emphatics. Second, \SYQ, too, has a suffix \phono{-ari} which, like \phono{-rI} and \phono{-ari}, functions as a general emphatic, also translating as \spanish{pues}. Third, the \SYQ{} \phono{-ari} is in complementary distribution with \phono{-k} and \phono{-ki}. Finally, unlike the \lsc{ahq}, \lsc{ccq} and \lsc{jhq} forms, the \SYQ{} \phono{-ari} cannot appear independently of the evidentials \phono{-mI} or \phono{-shI} or else of \phono{-y}, and, further, always forms an independent word with these.} \\

% 1
\gloexe{Glo6:Wayrakuyan}{}{amv}%
{Wayrakuyan\pb{mari}.}%amv que first line
{\morglo{wayra-ku-ya-n-m-ari}{wind-\lsc{refl}-\lsc{prog}-\lsc{3}-\lsc{evd}-\lsc{ari}}}%morpheme+gloss
\glotran{\pb{It’s windy}.}{}%eng+spa trans
{}{}%rec - time

\noindent
It can often be translated as ‘surely’ or ‘certainly’ or ‘of course’. \phono{-aRi} generally occurs only in combination with \phono{-mI}~(\ref{Glo6:llapa}), (\ref{Glo6:firmachiwan}), \phono{-shI}~(\ref{Glo6:shali}), (\ref{Glo6:shari}) and \phono{-Yá}~(\ref{Glo6:Qillakuyanki}--\ref{Glo6:Yatraqninqa}).\\

% 2
\gloexe{Glo6:llapa}{}{amv}%
{Mana\pb{mari} llapa ruwayaqhina kayani.}%amv que first line
{\morglo{mana-m-ari}{no-\lsc{evd}-\lsc{ari}}\morglo{llapa}{all}\morglo{ruwa-ya-q-hina}{make-\lsc{prog}-\lsc{ag}-\lsc{comp}}\morglo{ka-ya-ni}{be-\lsc{prog}-\lsc{1}}}%morpheme+gloss
\glotran{\pb{No, of course}, it seems like I’m making it all up.}{}%eng+spa trans
{}{}%rec - time

% 3
\gloexe{Glo6:firmachiwan}{}{lt}%
{Ñuqa[ta]s firmachiwan\pb{mari}. Piru mana\pb{shari} chay wawi warmiytapis firmachinraqchu.}%lt que first line
{\morglo{ñuqa[-ta]-s}{I-\lsc{acc}-\lsc{add}}\morglo{firma-chi-wa-n-m-ari}{sign-\lsc{caus}-\lsc{1.obj}-\lsc{3}-\lsc{evd}-\lsc{ari}}\morglo{piru}{but}\morglo{mana-sh-ari}{no-\lsc{evr}-\lsc{ari}}\morglo{chay}{\lsc{dem.d}}\morglo{wawi}{baby}\morglo{warmi-y-ta-pis}{woman-\lsc{1}-\lsc{acc}-\lsc{add}}\morglo{firma-chi-n-raq-chu}{sign-\lsc{caus}-\lsc{3}-\lsc{cont}-\lsc{neg}}}%morpheme+gloss
\glotran{\pb{They made me sign}, too. But they \pb{didn’t} make my daughter sign yet, \pb{they say}.}{}%eng+spa trans
{}{}%rec - time

% 4
\gloexe{Glo6:shali}{}{ch}%
{Viñacpaq\pb{shali}.}%ch que first line
{\morglo{Viñac-paq-\pb{sh-ali}}{Viñac-\lsc{abl}-\lsc{evr}-\lsc{ari}}}%morpheme+gloss
\glotran{\pb{From Viñac, she says, then}.}{}%eng+spa trans
{}{}%rec - time

% 5
\gloexe{Glo6:shari}{}{amv}%
{Ripun\pb{shari} umaqa kunkanman.}%amv que first line
{\morglo{ripu-n-sh-ari}{go-\lsc{3}-\lsc{evr}-\lsc{ari}}\morglo{uma-qa}{head-\lsc{top}}\morglo{kunka-n-man}{neck-\lsc{3}-\lsc{all}}}%morpheme+gloss
\glotran{The head \pb{went} [flying back] towards his neck, \pb{they say}.}{}%eng+spa trans
{}{}%rec - time

% 6
\gloexe{Glo6:Qillakuyanki}{}{lt}%
{¡Kurriy! Qillakuyanki\pb{trari}.}%lt que first line
{\morglo{kurri-y}{run-\lsc{imp}}\morglo{qilla-ku-ya-nki-tr-ari}{lazy-\lsc{refl}-\lsc{prog}-\lsc{2}-\lsc{evc}-\lsc{ari}}}%morpheme+gloss
\glotran{Run!~\dots{} \pb{You must be being lazy}.}{}%eng+spa trans
{}{}%rec - time

% 7
\gloexe{Glo6:Kidakushun}{}{ach}%
{Kidakushun kaypa\pb{yari}.}%ach que first line
{\morglo{kida-ku-shun}{stay-\lsc{refl}-\lsc{1pl.fut}}\morglo{kay-pa-y-ari}{\lsc{dem.p}-\lsc{loc}-\lsc{emph}-\lsc{ari}}}%morpheme+gloss
\glotran{We’re going to stay \pb{here}.}{}%eng+spa trans
{}{}%rec - time

% 8 (10)
\gloexe{Glo6:Yatraqninqa}{}{amv}%
{Yatraqninqa mana yatraqninqa mana\pb{yari}.}%amv que first line
{\morglo{yatra-q-ni-n-qa}{know-\lsc{ag}-\lsc{euph}-\lsc{3}-\lsc{top}}\morglo{mana}{no}\morglo{yatra-q-ni-n-qa}{know-\lsc{ag}-\lsc{euph}-\lsc{3}-\lsc{top}}\morglo{mana-y-ari}{no-\lsc{emph}-\lsc{ari}}}%morpheme+gloss
\glotran{The ones who knew how. The ones who didn’t know how, \pb{no, of course}.}{}%eng+spa trans
{}{}%rec - time

\noindent
It is far less often employed than \phono{-ik} and \phono{-iki.} It is, however, prevalent in the LT dialect\phono, which supplied the single instance of \phono{tr-ari} in the corpus~(\ref{Glo6:itana}).\\

% 9 (11)
\gloexe{Glo6:itana}{}{amv}%
{Chay wayra itana piru rimidyum Hilda. ¡Piru wachikun\pb{yari}!}%amv que first line
{\morglo{chay}{\lsc{dem.d}}\morglo{wayra}{wind}\morglo{itana}{thorn}\morglo{piru}{but}\morglo{rimidyu-m}{remedy-\lsc{evd}}\morglo{Hilda}{Hilda}\morglo{piru}{but}\morglo{wachi-ku-n-y-ari}{sting-\lsc{refl}-\lsc{3}-\lsc{emph}-\lsc{ari}}}%morpheme+gloss
\glotran{The wind thorns are medicinal, Hilda. But \pb{do they ever sting}!}{}%eng+spa trans
{}{}%rec - time

\paragraph{Evidence strength \phono{-ik} and \phono{-iki}}\label{par:evistre}\index[sub]{evidentials!evidence strength}
\SYQ{} is unusual\footnote{Ayacucho Q also makes use of \phono{-ki}.} in that each of its three evidentials counts three variants, formed by the suffixation of \phono{-\uo}, \phono{-ik} or \phono{-iki}. The resulting nine forms are direct \phono{-mI-\uo}, \phono{-m-ik} and\phono{-m-iki}~(\ref{Glo6:trayamunchu}--\ref{Glo6:Wanuchinakun}); reportative \phono{-shI-\uo}, \phono{-sh-ik} and \phono{-sh-iki}~(\ref{Glo6:susyukuna}--\ref{Glo6:nisha}); and conjectural \phono{-trI-\uo}, \phono{-tr-ik} and\phono{-tr-iki}~(\ref{Glo6:Imapaqraq}--\ref{Glo6:Alkansachin}).\footnote{In Lincha, \phono{-iki} may modify both \phono{-mI} and \phono{-shI} but not \phono{-trI}; in Tana, \phono{-iki} may modify all three evidentials.}\\

% 1
\gloexe{Glo6:trayamunchu}{}{ach}%
{Manam trayamunchu mana\pb{mik} rikarinchu.}%ach que first line
{\morglo{mana-m}{no-\lsc{evd}}\morglo{traya-mu-n-chu}{arrive-\lsc{cisl}-\lsc{3}-\lsc{neg}}\morglo{mana-m-ik}{no-\lsc{evd}-\lsc{ik}}\morglo{rikari-n-chu}{appear-\lsc{3}-\lsc{neg}}}%morpheme+gloss
\glotran{He \pb{hasn’t} arrived. He \pb{hasn’t} showed up.}{}%eng+spa trans
{}{}%rec - time

% 2
\gloexe{Glo6:rishaq}{}{lt}%
{Limatam rishaq. Limapaqa buskaq kan\pb{miki}. Sutintapis rimayan\pb{miki}. ¿Ichu manachu?}%lt que first line
{\morglo{Lima-ta-m}{Lima-\lsc{acc}-\lsc{evd}}\morglo{ri-shaq}{go-\lsc{1.fut}}\morglo{Lima-pa-qa}{Lima-\lsc{loc}-\lsc{top}}\morglo{buska-q}{look.for-\lsc{ag}}\morglo{ka-n-m-iki}{be-\lsc{3}-\lsc{evd}-\lsc{iki}}\morglo{suti-n-ta-pis}{name-\lsc{3}-\lsc{acc}-\lsc{add}}\morglo{rima-ya-n-m-iki}{talk-\lsc{prog}-\lsc{3}-\lsc{evd}-\lsc{iki}}\morglo{ichu}{or}\morglo{mana-chu}{no-\lsc{q}}}%morpheme+gloss
\glotran{I’m going to go to Lima. In Lima, \pb{there are} people who read cards, \pb{then}. They’re \pb{saying} his name, \pb{then}, yes or no?}{}%eng+spa trans
{}{}%rec - time

% 3
\gloexe{Glo6:Wanuchinakun}{}{sp}%
{Wañuchinakun ima\pb{miki} chaytaqa muna:chu.}%sp que first line
{\morglo{wañu-chi-naku-n}{die-\lsc{caus}-\lsc{recip}-\lsc{3}}\morglo{ima-m-iki}{what-\lsc{evd}-\lsc{iki}}\morglo{chay-ta-qa}{\lsc{dem.d}-\lsc{acc}-\lsc{top}}\morglo{muna-:-chu}{want-\lsc{1}-\lsc{neg}}}%morpheme+gloss
\glotran{They kill each other and \pb{what-not, then}. I don’t want that.}{}%eng+spa trans
{}{}%rec - time

% 4
\gloexe{Glo6:susyukuna}{}{amv}%
{Chay\pb{shik} chay susyukuna ruwapakurqa chay nichuchanta wañushpa chayman pampakunanpaq.}%amv que first line
{\morglo{chay-sh-ik}{\lsc{dem.d}-\lsc{evr}-\lsc{ik}}\morglo{chay}{\lsc{dem.d}}\morglo{susyu-kuna}{associates-\lsc{pl}}\morglo{ruwa-paku-rqa}{make-\lsc{jtacc}-\lsc{pst}}\morglo{chay}{\lsc{dem.d}}\morglo{nichu-cha-n-ta}{crypt-\lsc{dim}-\lsc{3}-\lsc{acc}}\morglo{wañu-shpa}{die-\lsc{subis}}\morglo{chay-man}{\lsc{dem.d}-\lsc{all}}\morglo{pampa-ku-na-n-paq}{bury-\lsc{refl}-\lsc{nmlz}-\lsc{3}-\lsc{purp}}}%morpheme+gloss
\glotran{\pb{That’s why, they say}, before, the members made each other the small crypts, to bury them when they died.}{}%eng+spa trans
{}{}%rec - time

% 5
\gloexe{Glo6:Llutanshiki}{}{lt}%
{Llutanshiki. Llutan runa\pb{shik} kan.}%lt que first line
{\morglo{llutan-sh-iki}{ugly-\lsc{evr}-\lsc{iki}}\morglo{llutan}{ugly}\morglo{runa-sh-ik}{person-\lsc{evr}-\lsc{ik}}\morglo{ka-n}{be-\lsc{3}}}%morpheme+gloss
\glotran{\pb{They’re messed up, they say}. There are messed up \pb{people, they say}.}{}%eng+spa trans
{}{}%rec - time

% 6
\gloexe{Glo6:nisha}{}{ch}%
{“¡Mátalo!” nisha\pb{shiki}.}%ch que first line
{\morglo{mátalo}{{}[Spanish]}\morglo{ni-sha-sh-iki}{say-\lsc{npst}-\lsc{evr}-\lsc{iki}}}%morpheme+gloss
\glotran{“Kill him!” \pb{she’s said, they say}.}{}%eng+spa trans
{}{}%rec - time

% 7
\gloexe{Glo6:Imapaqraq}{}{ach}%
{¿Imapaqraq chayta ruwara paytaqa? Yanqaña\pb{trik} chayta wañuchira.}%ach que first line
{\morglo{ima-paq-raq}{what-\lsc{purp}-\lsc{cont}}\morglo{chay-ta}{\lsc{dem.d}-\lsc{acc}}\morglo{ruwa-ra}{make-\lsc{pst}}\morglo{pay-ta-qa}{he-\lsc{acc}-\lsc{top}}\morglo{yanqa-ña-tr-ik}{lie-\lsc{disc}-\lsc{evc}-\lsc{ik}}\morglo{chay-ta}{\lsc{dem.d}-\lsc{acc}}\morglo{wañu-chi-ra}{die-\lsc{caus}-\lsc{pst}}}%morpheme+gloss
\glotran{What did they do that to him for? They \pb{must have} killed him \pb{just for the sake of it}.}{}%eng+spa trans
{}{}%rec - time

% 8
\gloexe{Glo6:Ablan}{}{sp}%
{Ablan\pb{shiki}. “Tragu, vino”, nishpa\pb{triki} ablayamun.}%sp que first line
{\morglo{abla-n-sh-iki}{talk-\lsc{3}-\lsc{evr}-\lsc{iki}}\morglo{tragu}{drink}\morglo{vino}{wine}\morglo{ni-shpa-tr-iki}{say-\lsc{subis}-\lsc{evc}-\lsc{iki}}\morglo{abla-ya-mu-n}{talk-\lsc{prog}-\lsc{cisl}-\lsc{3}}}%morpheme+gloss
\glotran{\pb{They talk, they say, for sure}. “Pay me liquor, wine,” \pb{they must be saying}, talking.}{}%eng+spa trans
{}{}%rec - time

% 9
\gloexe{Glo6:Alkansachin}{}{amv}%
{Alkansachin warkawan\pb{tri}. Kabrapis kasusam, piru. Riqsiyan\pb{triki} runantaqa.}%amv que first line
{\morglo{alkansa-chi-n}{reach-\lsc{caus}-\lsc{3}}\morglo{warka-wan-tri}{sling-\lsc{instr}-\lsc{evc}}\morglo{kabra-pis}{goat-\lsc{add}}\morglo{kasu-sa-m}{attention-\lsc{prf}-\lsc{evd}}\morglo{piru}{but}\morglo{riqsi-ya-n-tr-iki}{know-\lsc{prog}-\lsc{3}-\lsc{evc}-\lsc{iki}}\morglo{runa-n-ta-qa}{person-\lsc{3}-\lsc{acc}-\lsc{top}}}%morpheme+gloss
\glotran{She \pb{must make [the stones] reach} with the sling, \pb{for sure}. The goats obey her. They \pb{must know} their master, \pb{for sure}.}{}%eng+spa trans
{}{}%rec - time

Evidentials obligatorily take evidentional modifier (hereafter “\lsc{em}”) arguments; \lsc{em}s are enclitics and attach exclusively to evidentials. So, for example, \phono{*mishi-m} [cat-\lsc{evd}] and \phono{*mishi-ki} (cat-\lsc{iki}) are both ungrammatical. The corresponding grammatical forms would be \phono{mishi-m-\pb{\uo}} [cat-\lsc{evd}-\uo] and \phono{*mishi\pb{-mi}-ki} (cat-\lsc{evd}-\lsc{iki}), respectively. With all three sets of evidentials, the \phono{-ik} form is associated with some variety of increase over the \phono{-\uo} form; the \phono{-iki} form, with greater increase still. With all three evidentials, \phono{-ik} and \phono{-iki} --~except in those cases in which they take scope over universal-deontic-modal or future-tense verbs~-- indicate an increase in strength of evidence. With the direct \phono{-mI}, \phono{-ik} and \phono{-iki} generally also affect the interpretation of strength of assertion; with the conjectural \phono{-trI}, the interpretation of certainty of conjecture. In the case of universal-deontic modal and future-tense verbs, with both \phono{-mI} and \phono{trI}, \phono{-ik} and \phono{-iki} indicate increasingly strong obligation and increasingly imminent/certain futures, respectively.

\subsubsection{Evidentials in questions}
In questions, the evidentials\index[sub]{evidentials!questions} generally indicate that the speaker expects a response with the same evidential (\ie,~an answer based on direct evidence, reportative evidence or conjecture, in the cases of \phono{-mI}, \phono{-shI}, and \phono{-trI}, respectively)~(\ref{Glo6:Amador}--\ref{Glo6:Kutiramunman}).\\

% 1
\gloexe{Glo6:Amador}{}{ach}%
{¿Amador Garaychu? ¿\pb{Imam} sutin kara?}%ach que first line
{\morglo{Amador}{Amador}\morglo{Garay-chu}{Garay-\lsc{q}}\morglo{ima-m}{what-\lsc{evd}}\morglo{suti-n}{name-\lsc{3}}\morglo{ka-ra}{be-\lsc{pst}}}%morpheme+gloss
\glotran{Amador Garay? \pb{What} was his name?}{}%eng+spa trans
{}{}%rec - time

% 2
\gloexe{Glo6:Maypish}{}{ch}%
{¿\pb{Maypish} wasinta lulayan?}%ch que first line
{\morglo{may-pi-sh}{where-\lsc{loc}-\lsc{evr}}\morglo{wasi-n-ta}{house-\lsc{3}-\lsc{acc}}\morglo{lula-ya-n}{make-\lsc{prog}-\lsc{3}}}%morpheme+gloss
\glotran{\pb{Where did she} say she’s making her house?}{}%eng+spa trans
{}{}%rec - time

% 3
\gloexe{Glo6:Kutiramunman}{}{ach}%
{¿Kutiramunman\pb{chutr}? ¿\pb{Imatrik} pasan?}%ach que first line
{\morglo{kuti-ra-mu-n-man-chu-tr}{return-\lsc{urgt}-\lsc{cisl}-\lsc{q}-\lsc{evc}}\morglo{ima-tr-ik}{what-\lsc{evc}-\lsc{ik}}\morglo{pasan}{pass-\lsc{3}}}%morpheme+gloss
\glotran{\pb{Could} he come back? \pb{What would have} happened?}{}%eng+spa trans
{}{}%rec - time

\noindent
The use of \phono{-trI} in a question may, additionally, indicate that the speaker doesn’t actually expect any response at all~(\ref{Glo6:Kawsan}).\\

% 4
\gloexe{Glo6:Kawsan}{}{ach}%
{¿Kawsan\pb{chutr} mana\pb{chutr}? No se sabe.}%ach que first line
{\morglo{kawsa-n-chu-tr}{live-\lsc{3}-\lsc{q}-\lsc{evc}}\morglo{mana-chu-tr?}{no-\lsc{q}-\lsc{evc}}\morglo{No se sabe.}{{}[Spanish]}}%morpheme+gloss
\glotran{\pb{Would} he be alive or dead? We don’t know.}{}%eng+spa trans
{}{}%rec - time

\noindent
And the use of \phono{-shI} may indicate not that the speaker is expecting an answer based on reported evidence, but that the speaker is reporting the question~(\ref{Glo6:5}).

\chapter{Post-editing risks and data security -- which pitfalls can arise?}\label{sec:7}



    \objectives{
        You will learn...
        \begin{itemize}
            \item what risks can arise in translation and post-editing,
            \item what to keep in mind concerning data security when using MT systems.
        \end{itemize}
        }

\vspace{\baselineskip}


When we talk about PE, we also have to think about possible risks and security concerns. In this chapter, we want to outline the most important considerations, so you know what you have to keep in mind when you start working as a professional post-editor.

\section{Post-editing risks assessment}\label{sec:7:1}

Translating texts generates risks for all actors involved in the translation process (\citealt{canfora2015risikomanagement} or \citealt{canfora2018ostriches}). Although translation contains specific creative and cognitive aspects that alone can be the research focus of many scientific studies \citep{pym2018risk}, decisions made during the entire translation process are underpinned by the same principles as the decisions on any other business level. Therefore, these decisions should be made in an economic framework. One instrument to develop decision criteria in economics is risk management. Generally, business decision criteria can be differentiated between strategic (long-term), tactical (medium-term), and operative (short-term) decisions \citep{hofmann2012prozessgestutztes}. When considering risk management for a PE situation, the following business decisions are of special importance: 

\begin{itemize}
    \item strategic, e.g. if the organisation wants to use MT at all 
    \item operative, e.g. what PE guidelines – full vs. light – are necessary for the specific text or the respective text type in regard to the organisation’s general strategic decisions
\end{itemize}

The international standard ISO \citet{iso2009international} “Risk management – Principles and guidelines” can be used for the translation process in all contexts, because it is a horizontal standard. Risk management is considered an integral part of all processes in an organisation including translation processes (either in-house or as part of a supply chain risk management). In addition to the risks that emerge from translation in general, the use of MT and PE generates risk factors in particular, such as:

\begin{itemize}
    \item data breach: Confidential information are fed into a web-based MT system and end up on the web (as in the case of Statoil, cf. \href{https://www.csoonline.com/article/3236348/data-breached-in-translation.html}{CSO Online}, last accessed 20 August 2021).
    \item loss of control of processes: The clients cannot control whether the translator uses MT or the functionality of the MT system is not transparent to the user of the MT system at all, especially with neural MT.
    \item uncertain liability modalities: In cases of translation errors or problems, the responsibilities concerning liability are not clearly defined. This especially affects the use of MT and PE for high risk texts. In cases of claims for compensation where translation mistakes cause danger to life and limb, the client might partially be blamed.
    \item attitude towards MT: Clients might have difficulties in finding qualified translators and post-editors, because professional translators might still have prejudices against MT and PE (e.g. \citealt{cadwell2018resistance} ; \citealt{guerberof2013professional}; \citealt{laubli2017google}).
    \item quality issues: The quality of the post-edited text might not be sufficient for the purposes of the client or target group.
\end{itemize}
	
Basically, the client should consider whether the benefits outweigh the risks before using MT and PE. Or, in other words, the client has to decide whether the risks are tolerable in a given situation. This includes general considerations arising from the client’s own “risk management policy” (cf. \href{https://www.theirm.org/media/4709/arms_2002_irm.pdf}{IRM 2002}\footnote{last accessed 20 April 2021}). In the context of ISO 31000, the term risk management policy describes a “statement of the overall intentions and direction of an organization related to risk management” (\href{https://www.iso.org/standard/44651.html}{ISO Guide 73:2009}\footnote{last accessed 20 April 2021}). The general risk attitude of the organisation is an important factor in creating the risk management policy, which describes the “organization’s approach to assess and eventually pursue, retain, take or turn away from risk” (\href{https://www.iso.org/standard/44651.html}{ISO Guide 73}, term 3.7.1.1). Accordingly, an organisation can be more or less willing to take risks, and this so called “risk appetite” influences strategic decisions. An organisation with a higher appetite for risks is more willing to take the risks mentioned above than a risk adverse organisation. These decisions are usually made on a long-term basis and therefore usually concern the strategic part of business management (cf. \href{https://www.theirm.org/media/4709/arms_2002_irm.pdf}{IRM 2002}).

On the operative level, risk management can provide decision criteria for or against the use of MT and PE for certain text types. Therefore, the approach to risk management for translations can also be used for decision processes in MT and PE \citep{canfora2018ostriches}. This means that the potential risks must be identified before the actual translation process to foresee problems that might affect different actors involved in the translation process such as the translator, the TSP (Translation Service Provider), the client, the end user or any other agent. This initial analysis should consider the negative consequences of failures in the translation, such as impaired communication, loss of reputation, property damage, lawsuits or other legal consequences, injuries, which could even amount to danger to life and limb, etc. Afterwards, the likelihood that these risks could occur in each case and the priorities of the client regarding the translation risks need to be analysed in compliance with the strategic risk management policy. This means that the client or the project manager has to decide which translation risks must be avoided and which can be tolerated. Therefore, it is sensible to create different categories (e.g. very high-risk, high-risk, and low risk documents) and to categorise the source text documents according to the risk analysis and risk evaluation \citep{canfora2018ostriches}. In line with these categories, different processes can be shaped for the use of MT and PE. Hence, low-risk texts, for example, could be machine-translated with subsequent light PE or even without PE. High-risk texts require full PE so that a balance is created between risk considerations and the advantages of MT and PE. For very high-risk texts, the client has to evaluate whether a combination of MT, full PE, and additional quality control measures like revision ensure the necessary quality. If this is not the case, MT might have to be entirely disregarded for those text types because the risks are too high. Furthermore, it has to be assessed whether it is still more efficient to combine MT, full PE and additional quality measures. Maybe a translation workflow with only human translation would provide more security and higher productivity and reduce costs in the end.

For more details on risks in PE and sustainable workflows for NMT read \citet{canfora2020risks}, who isolate three main risks for NMT: possible damages to clients and customers, liability issues, and cyber risks.

\section{Post-editing and data security}\label{sec:7:2}

Data security is a very important issue when using MT, because not all systems protect your texts and data. If an in-house MT system is used, security considerations are less problematic because the texts that are fed into the system are safely stored on an internal system or server. Still, it might be reasonable to assess who can access the server and the MT system. Further, users of the MT system should be informed about confidentiality issues, especially when working with externals and freelancers. The same holds true for cloud-based systems, which typically use secure encoding. However, if an external system without secure encoding and/or a free online system is chosen, the source text is often saved on the provider’s server and hence might become accessible to third parties. This can be unproblematic, e.g. if we are dealing with the translation of a website and this text will be publicly available anyway. However, if the data are sensitive, MT systems that do not provide a safe environment must be avoided (cf. e.g. \citealt{kamocki2015all}).

The German company DeepL, which provides MT systems, clearly differentiates between the free and the paid versions of their service. Regarding free use, they state
\begin{quote}
    When you use our translation service, only enter texts that you are willing to transfer to our server. Transferring the texts is necessary to offer our service and conduct the translation. We process your texts and their translations for a limited amount of time to train and improve our neural networks and translation algorithms.
    
    If you edit our translation suggestions, these edits will also be transferred to our servers to check the edit for correctness and possibly update the translated text according to your corrections. We also save your edits for a limited amount of time to train and improve our translation algorithm.
    
    Please note that you must not use our translation services for texts containing any kind of personal data.
    \footnote{\url{https://www.deepl.com/privacy}, last accessed 20 April 2021

    Original text:``Wenn Sie unseren Übersetzungsservice nutzen, geben Sie nur Texte ein, die Sie auf unsere Server übertragen wollen. Die Übermittlung dieser Texte ist notwendig, damit wir die Übersetzung durchführen und Ihnen unseren Service anbieten können. Wir verarbeiten Ihre Texte und die Übersetzung für einen begrenzten Zeitraum, um unsere neuronalen Netze und Übersetzungsalgorithmen zu trainieren und zu verbessern.

    Wenn Sie Korrekturen an unseren Übersetzungsvorschlägen vornehmen, werden diese Korrekturen auch an unseren Server weitergeleitet, um die Korrekturen auf Richtigkeit zu überprüfen und gegebenenfalls den übersetzten Text entsprechend Ihren Änderungen zu aktualisieren. Wir speichern auch Ihre Korrekturen für einen begrenzten Zeitraum, um unseren Übersetzungsalgorithmus zu trainieren und zu verbessern.

    Bitte beachten Sie, dass Sie unseren Übersetzungsservice nicht für Texte mit personenbezogenen Daten jeglicher Art nutzen dürfen."}
\end{quote}

When it comes to the paid cloud version, DeepL has a much securer policy
\begin{quote}
    When you use DeepL Pro, your submitted texts or documents will not be stored permanently, but only as long as it takes to create and transmit the translation. After transmitting the translation to you, both the texts or documents you submitted as well as their translations will be deleted. When you use DeepL Pro, we do not use your texts to improve the quality of our service. [...]
    
    Please note that you can use DeepL Pro only for texts containing any kind of personal data if you have a job processing agreement with us [...].\footnote{\url{https://www.deepl.com/privacy}, last accessed 20 April 2021

    Original text: ``Bei der Verwendung von DeepL Pro werden die von Ihnen eingereichten Texte oder Dokumente nicht dauerhaft gespeichert und nur vorübergehend vorgehalten, soweit dies für die Erstellung und Übertragung der Übersetzung notwendig ist. Nach der Übertragung der Übersetzung an Sie werden sowohl die eingereichten Texte oder Dokumente als auch deren Übersetzungen gelöscht. Bei der Verwendung von DeepL Pro verwenden wir Ihre Texte nicht, um die Qualität unserer Dienstleistungen zu verbessern. [...]

    Bitte beachten Sie, dass Sie DeepL Pro grundsätzlich nur für Texte nutzen dürfen, die personenbezogene Daten jeglicher Art enthalten, wenn Sie mit uns eine Auftragsverarbeitungsvereinbarung abgeschlossen haben [...]."}
\end{quote}

This also means that you should never machine translate the text you get from your clients without their permission, especially if you want to use an online MT system as the data will probably be stored by the MT system. Or as \citet[15]{kamocki2015all} summarise for general use
\begin{quote}
    Private users should consider translating only those bits of texts that do not contain any information relating to third parties (which in practice may limit them to translating text into, and not from, a different language to their own). Businesses in particular may find such a limitation rather constricting and to protect their own data and the data of their clients, may prefer to opt for a payable offline MT tool instead of a ‘free’ online service.
\end{quote}

As is the case in many other AI and computational linguistic features, using MT has become so common, especially as it is often implemented on webpages that we might become a little careless. So, we can only advise you to think carefully before you decide to use MT systems, especially in a professional context. 

\newpage

\section*{Crossword puzzle -- chapter 7}

\begin{Puzzle}{10}{12}
|{}	|{}	|{}	|{}	|{}	|{}	|{}	|{}	|[3]C	|.
|{}	|{}	|{}	|{}	|{}	|{}	|{}	|{}	|O	|.
|{}	|{}	|{}	|{}	|{}	|{}	|[4]L	|{}	|N	|.
|[5]R	|I	|S	|K	|[2]S	|{}	|I	|{}	|F	|.
|{}	|{}	|{}	|{}	|T	|{}	|A	|{}	|I	|.
|{}	|{}	|{}	|{}	|R	|{}	|B	|{}	|D	|.
|[1]O	|P	|E	|R	|A	|T	|I	|V	|E	|.
|{}	|{}	|{}	|{}	|T	|{}	|L	|{}	|N	|.
|{}	|{}	|{}	|{}	|E	|{}	|I	|{}	|T	|.
|{}	|{}	|{}	|{}	|G	|{}	|T	|{}	|I	|.
|{}	|{}	|{}	|{}	|I	|{}	|Y	|{}	|A	|.
|{}	|{}	|{}	|{}	|C	|{}	|{}	|{}	|L	|.
\end{Puzzle}

\begin{PuzzleClues}{\textbf{Across}}
\Clue{1}{OPERATIVE}{On what business decision level must the kind of PE guidelines be decided?}
\Clue{5}{RISKS}{Before a document is post-edited, the potential ... have to be analysed and evaluated.}
\end{PuzzleClues}

\begin{PuzzleClues}{\textbf{Down}}
\Clue{2}{STRATEGIC}{On what business decision level must organisations decide whether or not to use MT?}
\Clue{3}{CONFIDENTIAL}{What kind of information should not be fed into a web-based MT system?}
\Clue{4}{LIABILITY}{The responsibilities concerning... are not clearly defined for PE, yet.}
\end{PuzzleClues}

\chapter{Attributive constructions in the Jewish dialect of Sanandaj} \label{ch:Sanandaj}

\renewcommand{\defaultDialect}{\JSan}

\section{Introduction}

The Iranian city of Sanandaj is located at the eastern extremity of the \ili{NENA} speaking zone. Compared to the three dialects surveyed so far, the grammar of the Jewish dialect of Sanandaj is the most divergent. This is certainly true for the AC system, which will be surveyed below, but can also be said about other domains of grammar, such as the verbal system. While the latter is outside the scope of this work, it is worthwhile noting two innovative features of the verbal system, which are of relevance to the current survey: First, the language exhibits an OV order (in contrast to the typical VO order found in most \ili{NENA} dialects); and second, the language makes extensive use of complex predication, i.e.\ predicates consisting of a combination of a light verb and a noun (termed here CP noun).\footnote{For an elaborate syntactic and semantic analysis of complex predication in \Per, see \citet{SamvelianComplex}.} 
 These and other features are in all probability related to an extensive \isi{language contact} with \Sor and \Per \citep[11f.]{KhanSanandaj}.\footnote{As \citet[11]{KhanSanandaj} notes, the Kurdish dialect of Sanandaj is not systematically described. Instead, I refer to standard \Sor for the sake of comparison. It should also be noted that \Hawr, a \ili{Gorani} language closely related to Kurdish, is spoken in the vicinity of Sanandaj.}
  While one may speculate that the divergence of \JSan is related to its peripheral location, it is worthwhile noting that the Christian dialect spoken in the same city presents a much more conservative grammar, but unfortunately it has not yet received  a detailed grammatical description.\footnote{See however \citet{PanoussiSenaya,HeinrichsSenaya} and the list of publications given in \citet{McPhersonCaldani}.}

The data for \JSan is based mainly on the grammatical description of \citet{KhanSanandaj}.\footnote{Khan's examples are cited according to the page in the grammar in which they are treated. Additionally, a reference to the textual corpus, if available, is given in square brackets according to Khan's system: a letter indicating the informant (A--E) and a sentence number. I have also consulted the grammatical description of \citet{SchallerSanandaj}, but as this description is mostly devoted to the verbal system, no examples are drawn from there.} Additional examples are drawn from an elicitation session I have conducted in Jerusalem with an elderly native speaker of the dialect, Ḥabib Nurani.\footnote{In Khan's description, he is marked as informant A.} \JSan is in some respects similar to \JSul, of which I give some comparative examples drawn from \citet{KhanSulemaniyya}. I present also some sporadic comparisons with \NMand, another Neo-Aramaic language spoken in Iran. 

The structure of the chapter is as follows:

First, I treat the usage of the possessive pronominal suffixes. In contrast to most other \ili{NENA} dialects, these are phrasal suffixes, as discussed in \sref{ss:JSan_X-y.poss}. 

A major difference in \JSan in comparison  to the dialects discussed so far is that the main AC in \JSan is not the CSC, but rather the zero-marked \isi{juxtaposition} construction, which is discussed in \sref{ss:JSan_juxt}. This construction has two further variants: \isi{juxtaposition} with agreement of the \secn with the \prim (see \sref{ss:JSan_juxt_agr}), and {inverse juxtaposition}\isi{inverse juxtaposition construction} with the \secn preceding the \prim (see \sref{ss:JSan_inverse}). 

The use of borrowed \ili{Iranic} \rel*s with clausal \secns is discussed in \sref{ss:JSan_rel}.

While \JSan does not make use of the Neo-CSC found in other dialects, it has a structural parallel formed by marking the \prim with the \ili{Iranic} \ez* suffix. This construction, as well as the idiomatic retention of the historical CSC and the possible emergence of a new CSC related to stress retraction, is discussed in \sref{ss:JSan_cst}.

From the above it is clear that \JSan has hardly retained any reflex of the \il{Aramaic!Classical}Classical Aramaic \d \lnk*. Indeed, \JSan has  only one reflex of this \lnk*, namely the \gen* marking of vowel-initial demonstratives. This is discussed in \sref{ss:JSan_gen}. On the other hand, \JSan has retained to a small extent the usage of the \isi{dative preposition} \transc{əl-} for marking \secns, as discussed in \sref{ss:JSan_dat}.

Conclusions and a general discussion of the various constructions are presented in \sref{ss:JSan_conclusions}.

\section{Possessive pronominal suffixes (X-y.\poss)} \label{ss:JSan_X-y.poss}



As in other \ili{NENA} dialects, a pronominal \secn may be expressed by a \isi{possessive suffix}. The \isi{possessive suffix} replaces the inflectional suffix (\transc{-a} or \transc{-e}) of the nominal \prim it attaches to:

\acex{Noun}{Pronoun}{14}
{bel-ef}
{house-\poss.3\masc}
{his house}
{KhanSanandaj}{61}

A particularity of \JSan in comparison with most other \ili{NENA} dialects is that the \isi{possessive pronoun} is suffixed NP-finally, rather than directly on the \prim noun, whenever the NP consists of a Noun+Adj. combination \citep[251]{KhanSanandaj}.
 
 \acex{Noun Phrase}{Pronoun}{1}
 {[xa ʾăxóna xet]-àfˈ}
 {\indef{} brother other-\poss.3\fem}
 {another brother of hers}
 {KhanSanandaj}{53 {[A:6]}}
 
 A similar pattern is found in \JSul:
 
 \acex[\JSul]
 {Noun Phrase}{Pronoun}{1104}
 {ʾaxón-a ruww-í}
 {brother-\free{} big-\poss.1\sg}
 {my elder brother}
 {KhanSulemaniyya}{262 {[R:94]}}
 
 
 In the current framework, this distribution makes the possessive suffixes of \JSan and \JSul phrasal suffixes rather then word-level suffixes (see \sref{ss:clitics_affixes}). The usage of the possessive suffixes NP-finally may very well be due to \isi{pattern replication} from \Sor (see \example{836}).

When attached to a verbal noun (such as an infinitive or a CP noun of a transitive verb), the pronoun denotes the object: 

\acex{Infinitive}{Pronoun}{92}
{(ʾila di-li ba\cb{}) găroš-ef\,\footnotemark}
{hand placed-1\sg{} in\cb{} pull.\inf-\poss.3\masc}
{(I began) to pull him.}
{KhanSanandaj}{331}

\footnotetext{One may want to analyse the combination \transc{ba}+infinitive as forming a gerund, as in \JZax (see \vref{ft:JZax_gerund}). As I am unaware of a gerund category in \JSan, I prefer to analyse the preposition \transc{ba} here, as well as in \example{91}, as forming part of the verbal complex.}

\acex{CP Noun}{Pronoun}{51}
{daʿwăt-ì (k-ol-í)ˈ}
{invitation-\poss.1\sg{} \ind-do-3\pl}
{They will invite me.}
{KhanSanandaj}{482 {[D:8]}}

  
When attached to a preposition, it denotes its complement. Note, however, that not all prepositions allow for a suffixed pronoun.

\acex{Preposition}{Pronoun}{62}
{reš-ef}
{on-\poss.3\masc}
{on it}
{KhanSanandaj}{224}

Interestingly, a pronoun attached to a true adverb can convert it to a noun:
\acex{Adverb}{Pronoun}{15}
{(gbé-wa xa\cb{}párča zayrá dă-en  ba\cb{}) lăxa-u}
{\ind.want.3\masc-\pst{} \indef\cb{}fabric yellow place-3\pl{} on\cb{} here-\poss.3\pl}
{(They had to put a yellow patch on) their (body place) here.\footnotemark}
{KhanSanandaj}{579 {[A:78]}}\vspace*{-4mm}
\footnotetext{From the context it seems that the informant pointed on a spot on his body (\transl{here}) referring to the same spot on the body of the referents.}

 \section{Simple juxtaposition (X Y)} \label{ss:JSan_juxt}
 \largerpage[2]
 The paradigmatically richest and most common construction in \JSan is the \isi{juxtaposition} construction, devoid of any special marking. In cases where a noun is modified by another noun, the \isi{juxtaposition} construction is the  functional parallel in \JSan of the CSC or the ALC in the previously surveyed  \ili{NENA} dialects. 
 
\acex{Noun}{Noun}{10}
{lišana bšəlmane}
{language Muslims}
{the language of the Muslims}
{KhanSanandaj}{199}

  
In this \JSan is very similar to \JSul, which also makes extensive use of the \isi{juxtaposition} construction:\footnote{Yet, in contrast to \JSan, \JSul also makes use of the Neo-\cst\ suffix \ed, as well as the \lnk* \d:

\acexfn[\JSul]
{Noun}{Noun}{1068}
{xə́zm-əd kaldà}
{family-\cst{} bride}
{the family of the bride}
{KhanSulemaniyya}{192 {[A:8]}}
\vspace*{-2mm}

\acexfnii[\JSul]
{Noun}{Noun}{1063}
{məšxa d\cb{} zetùne}
{oil \lnk\cb{} olives}
{olive oil}
{KhanSulemaniyya}{192 {[R:98]}}
\vspace*{-1mm}
}

\acex[\JSul]
{Noun}{Noun}{1059bis}
{šəmma brona}
{name son}
{the name of the boy}
{KhanSulemaniyya}{192}
   

While the above usage of the \isi{juxtaposition} construction in \JSan and \JSul for nominal modification
marks these dialects as special in comparison to the majority of \ili{NENA} dialects, there are also more {trivial} cases of \isi{juxtaposition},  such as its usage with \isi{adverbial} \secns:


\acex{Noun}{\PP}{13}
{ʾo gorá ga\cb{} lăxa (bărux-i \cb{}ye)}
{\textsc{dem.m} man in\cb{} here friend-\poss.1\sg{} \cb{}\cop.3\masc}
{the man here (is my friend)}
{KhanSanandaj}{252}

Another use of the \isi{juxtaposition} construction, which is  cross-dialectally common and present also in \JSan, is in quantification expressions (see also \example{1939}):

\acex
{Q. Noun Phrase}{Noun}{1933}
{[xa lewan] rəzza}
{one cup rice}
{one cup of rice}
{}{(own fieldwork)}

Adjectival and ordinal \secns normally appear in the \isi{juxtaposition-cum-agreement} construction, discussed in the next section. Yet, when the lexical items in question are invariable, such as loan-adjectives or the loan-ordinal \foreign{ʾăwal}{first}, they necessarily appear in the simple \isi{juxtaposition} construction:

\acex{Noun}{Adjective}{278}
{mal-ăwae qarwa}
{village-\pl{} near(\invar)}
{nearby villages}
{KhanSanandaj}{207}

\acex{Noun}{Ordinal}{36}
{gora ʾăwal}
{man first(\invar)} 
{the first man}
{KhanSanandaj}{213}

 

Clausal \secns are also found in this construction, yielding \concept{asyndetic relative clauses}.

\acex{Noun}{Clause}{43}
{(măt-í-wa-le ga\cb{}) xá \cb{}tʷka [qărirà hăwé].ˈ}
{put-\agent.3\pl-\pst-\patient.3\masc{} in\cb{} \indef{} \cb{}place cool \subj.be}
{They put it in a place that was cool.}
{KhanSanandaj}{381 {[A:83]}}


Asyndetic clausal \secns can also follow pronominal \prims, such as the indefinite pronoun \foreign{xa}{one}. In the following example there are two asyndetic relative clauses, one embedded in the other.\footnote{Note that the embedded \isi{relative clause} is separated from its \prim \transc{sawzì} by the \isi{copula} and a prosodic break. Alternatively, it could be analysed as an asyndetically conjoined \isi{relative clause} governed as well  by the \prim \transc{xa}.}


\acex{Pronoun}{Clause}{44}
{(bár kŭ̀leˈ kyà-waˈ) xa\cb{} [sawzì \cb{}ye,ˈ šaplultà kəmr-í-wa baq-éf].ˈ}
{after all \ind.come.3\fem-\pst{} one\cb{} vegetable \cb{}\cop.3\masc{} š. call-\agent.3\pl-\pst{} for-\poss.3\masc{}}
{(After everything else there came) something that is a vegetable, which is called \textit{šaplulta}.}
{KhanSanandaj}{382 {[B:68]}}

Examples of clausal \secns following demonstrative pronouns acting as \prims are given in \sref{ss:gen_pron_clause}. 

\largerpage
Occasionally, the \isi{juxtaposition} construction is used with an infinitival \prim followed by a nominal \secn, corresponding to the direct object of a transitive verbal lexeme.\footnote{Such cases can also be analysed as exponents of the \isi{completive relation} rather that the attributive one (see \sref{ss:threeRel}). Yet I prefer to to analyse them as attributive constructions, as discussed in \sref{ss:JSan_inverse}.}

\acex{Infinitive}{Conjoined Nouns (objects)}{91}
{(šerúʾ wí-lu ba\cb{}) yălopé hulaulà \cb{}uˈ yălopé făransà \cb{}uˈ ʿəbrì,ˈ fàrsi.ˈ} 
{start do-3\pl{} in learn.\inf{} Judaism \cb{}and learn.inf{} \ili{French} \cb{}and \ili{Hebrew} Persian}
{(They started) to learn Judaism, and to learn \ili{French}, \ili{Hebrew} and \ili{Persian}.}
{KhanSanandaj}{330 {[B:12]}}

 
Similarly, prepositions or conjunctions are complemented  by nouns or clauses without any special marking:

\acex{Preposition}{Noun}{54}
{reša mez}
{on table}
{on the table}
{KhanSanandaj}{224}

\acex{Conjunction}{Clause}{77}
{mangól [ga\cb{} lăxa k-olí]ˈ}
{as in\cb{} here \ind-do.3\pl}
{as they do here\footnotemark}
{KhanSanandaj}{393 {[B:67]}}

\footnotetext{An alternative analysis of this example is to see only the \isi{prepositional phrase} \transc{ga lăxa} as the complement of \transc{mangól}, the  verb \transc{kolí} being the main verb. This would correspond to the translation \transl{They do as (it is) here}.}

An adjective can also serve as the \prim of the \isi{juxtaposition} construction, whenever it is further specified by a nominal \secn. While the \secn in such cases is an \isi{adverbial} specification of the adjectival \prim, formally it uses the same \isi{juxtaposition} construction as the above examples (compare to \example{98}, where the AC is explicitly marked by the \ez* suffix): 


\acex{Adjective}{Noun}{99}
{(tamā́m-e yomá) ḥărík ḥaštà (xirá \cb{}y)ˈ} 
{entire-\ez{} day busy work become.\resl{} \cb{}\cop.3\masc} 
{(All day he has been) busy with work}
{KhanSanandaj}{570} 

\section{Juxtaposition-cum-agreement (X Y.\agr)} \label{ss:JSan_juxt_agr}

Similarly to other \ili{NENA} dialects, inflecting adjectives (which are typically but not exclusively of Aramaic origin) formally use  the \isi{juxtaposition} construction, and at the same time show agreement features with the \prim noun. 

\acex{Noun}{Adjective}{8}
{bela rŭwa}
{house(\masc) big.\masc}
{a big house}
{KhanSanandaj}{251}

Similarly, \isi{ordinals} above one, juxtaposed to their \prim, can optionally agree with it, similarly to adjectives:\footnote{Note that \isi{ordinals} are derived from the corresponding cardinals by means of the suffix \transc{-min}, being of \Per or \Sor origin (see \sref{ss:kurd_ez_ord}).}

\acex{Noun}{Ordinal}{279}
{baxta tre-min-\opt{ta}}
{woman(\fem) two-\ord-\opt\fem}
{the second woman}
{KhanSanandaj}{213}


\section{Inverse juxtaposition (Y X)} \label{ss:JSan_inverse}

\isi{inverse juxtaposition construction}Of special interest are constructions in which the order of the \secn and the \prim is reversed, so that the \secn precedes the \prim. There are two distinct kinds of these constructions, one which involves a verbal noun acting as a \prim, and the second which involves an adjective or an ordinal as the \secn.\footnote{Recall that the titles of the examples always reflect  the order \textbf{Primary--Secondary}, irrespective of the order of these constituents in the example.}

\subsection {Verbal nouns as \prims} \label{ss:JSan_inv_juxt_verbal}

The category of verbal nouns includes the infinitive and the active participle.\footnote{The resultative participle, on the other hand, does not participate in ACs, as its distribution is restricted to some compound tenses \citep[90--96]{KhanSanandaj}. Some resultative \isi{participles} have acquired an adjectival meaning, but in this case they do not function differently from other inflecting adjectives \citep[204]{KhanSanandaj}.} These nouns have the particularity that they can be complemented by a \secn which acts semantically as the direct object of the verbal lexeme. Moreover, I include in the category of verbal nouns also nouns participating in complex predicate formation (CP nouns), as their \secns are semantically the direct object of the entire complex predicate. Note that the extensive usage of complex predication in \JSan originates in the replication of an \ili{Iranic}, probably \Per, pattern \citep[of which see][]{SamvelianComplex}.

One may doubt whether constructions involving verbal nouns together with their complements should be regarded as ACs, rather than expressing simply a completive  relation (see \sref{ss:threeRel}). However, since verbal nouns behave categorically as nouns  (they share the privilege of occurrence of nouns), and complementation of  nouns yields by definition an AC, it seems justifiable to regard these constructions as ACs, albeit of a special kind. Two observations strengthen this claim: First, nouns and their complements participate sporadically in explicitly marked ACs (see \examples{49}{47} for verbal nouns modified by a \isi{possessive suffix}). Secondly, whenever their complement is an independent pronoun it is explicitly marked as genitive (see \sref{ss:JSan_gen_verbal}).

Notwithstanding the above analysis, verbal nouns expanded by a complement exhibit a key property of the verbal phrase of \JSan, namely the pre-verbal position of the complement. In fact, the OV order of \JSan is very probably a contact feature originating in \ili{Iranic} languages, as most \ili{NENA} dialects have a VO order. Thus, these ACs are of the {inverse juxtaposition}\isi{inverse juxtaposition construction} type, in which the \secn precedes the \prim (and see also \example{97}):

\acex{Participle}{Noun}{87}
{xola garš-ana}
{rope pull-\prtc}
{rope puller}
{KhanSanandaj}{252}

\acex{Infinitive}{Noun}{90}
{(ʾila hiw-li ba\cb{}) xola garoše}
{hand gave-1\sg{} in\cb{} rope pull.\inf}
{(I began) to pull the rope.}
{KhanSanandaj}{330}




\subsection{Adjectival and ordinal \secns} \label{ss:JSan_juxt_inverse_adj}

Normally an adjectival \secn follows the \prim noun (see \example{8}). However, according to \citet[251]{KhanSanandaj}, \textquote{[i]n some isolated cases the adjective is placed before the head [=the primary]. This is found where the adjective is evaluative, i.e.\ expressing the subjective evaluation by the speaker rather an objective description of the referent.} The following example is given:

\acex{Noun}{Adjective}{84}
{ʿáyza kā́sbi (hùl ta\cb{} nóš-ox).ˈ}
{good.\masc\footnotemark{} gain(\fem) take.\imp{} for\cb{} \refl-\poss.2\masc}
{Take the good earnings for yourself.}
{KhanSanandaj}{251 {[A:103]}}
\vspace*{-2mm}

\footnotetext{This example is peculiar in that the adjective disagrees in gender with the \isi{head noun}. It may be that some speakers treat \transc{ʿáyza} as an invariable adjective, being probably of foreign origin.}

Ordinal \secns can similarly appear before the \prim, in this case without any {evaluative} semantics. In the case of the ordinal \foreign{ʾăwaḷ}{one}, borrowed ultimately from \Arab, this yields the typical \ili{Arabic} order, but in \JSan this is only one possibility (contrast with  examples \vref{ex:36} and \vref{ex:35}):

\acex{Noun}{Ordinal}{276}
{ʾăwaḷ gora}
{first man}
{the first man}
{KhanSanandaj}{213}

Ordinals above one can optionally agree with the \prim noun, also when they precede it (compare \example{279}):

\acex{Noun}{Ordinal}{282}
{tre-min-\opt{ta} baxta}
{two-\ord-\opt\fem{} woman(\fem)}
{the second woman}
{KhanSanandaj}{213}

\section{Usage of relativizer (X \textsc{rel} Y)} \label{ss:JSan_rel}

Clausal \secns can be marked as such by the use of a \rel*.  Two distinct relativizers are available in \JSan: \transc{ya} and \transc{ke}, both borrowed from \ili{Iranic} languages. In particular, one finds \transc{ke} as a \rel* in \ili{Persian} \citep[136]{BalayEsmaili}. 

The \isi{relativizer} \transc{ya} is used mostly with definite \prims, while the \isi{relativizer} \transc{ke} has no such restriction. The exact distribution of these relativizers is outside the scope of this work. Prosodically, both relativizers are part of the clausal \secn, as they often cliticize to its first word.

\acex{Noun}{Clause}{39}
{ʾo\cb{} našé ya\cb{} [daʿwàt k-ol-í-wa-lu]ˈ}
{\definite\cb{} people \rel\cb{} invitation \ind-do-\agent\pl-\pst-\patient3\pl}
{the people whom they invited}
{KhanSanandaj}{378 {[A:42]}}

\acex{Pronoun}{Clause}{41}
{ʾonyé yá [ṭăbăqá ʾăwaḷ \cb{}ye-lù]ˈ}
{3\pl{} \rel{} class first \cb{}\cop.\pst-3\pl{}}
{Those who were the first class}
{KhanSanandaj}{379 {[B:5]}}

\acex{Noun}{Clause}{275}
{xá\cb{}ʿəda našé ke\cb{} [ga\cb{}xá meydā́n smix \cb{}èn]ˈ}
{\indef\cb{}few people \rel\cb{} in\cb{}\indef{} square stood.\resl{} \cb{}\cop.3\pl}
{a group of people who were standing in a square}
{KhanSanandaj}{380 {[A:109]}}

In \JSul, one finds conversely the cognate \rel* \transc{ga}\~\transc{ka} mostly with definite \prims, in restrictive relative clauses \citep[414]{KhanSulemaniyya}:\footnote{A similar restriction is found with the \isi{relativizer} \transc{ke} borrowed in \NMand \citep[165]{HaberlMandaic}.}

\acex[\JSul]
{Noun}{Noun}{1189}
{yóma ga\cb{} gezí ta\cb{} Merònˈ mìl-a.ˈ}
{day \rel\cb{} go-3\pl{} to\cb{} M. died-3\fem}
{She died on the day that they went to Mount Meron.}
{KhanSulemaniyya}{415 {[R:185]}}

In \JSan, the \rel* \transc{ke} is found following certain adverbials, notably \foreign{qắme}{before}: 

\acex{Conjunction}{Clause}{76}
{qắme ké hètˈ}
{before \rel{} \subj.come.2\masc}
{before you came}
{KhanSanandaj}{391}

In this usage, it can also combine with the \ez*  marking; see  \example{17}.

The \rel* \transc{ya} occurs once in the corpus of \citet{KhanSanandaj} complementing a temporal adverb. In this case, the entire construction gets a temporal meaning:

\acex{Adverb}{Clause}{42}
{ʾăta ya\cb{} [daʿwăt-í wilà \cb{}y]ˈ}
{now \rel\cb{} invitation-\poss.1\sg{} done.\resl{} =\cop.\masc}
{now that  they have invited me}
{KhanSanandaj}{379 {[D:15]}}
\antipar

  
\section{The construct state construction (X.\textsc{cst} Y)} \label{ss:JSan_cst}

\JSan has 3 different morphological means which can be classified under the broad category of \isi{construct state} as defined in \sref{ss:state}:

\subsection{The historical construct state marking}

The historical \il{Aramaic!Classical}Classical Aramaic \isi{construct state} marking, formed by \isi{apocope} of the \prim noun, is not productive any more in \JSan, yet a reflex of it is retained  in some collocations and idioms. For example, in the following example, the noun \foreign{belá}{house} appears as a reduced form \transc{be} with the meaning \transl{family of} (compare \Qar  \example{504}):

\acex{Noun}{Noun}{7}
{be\cb{} kalda}
{house.\cst\cb{} bride}
{family of the bride}
{KhanSanandaj}{201}

Similarly, two prepositions of nominal origin have retained an \isi{apocopated form} alongside their full form. These are the prepositions \foreign{reša}{on} (derived from the noun \foreign{reša}{head} by \isi{pattern replication} of Kurdish; see \vref{ft:reš}), which also has the \isi{apocopated form} \transc{reš}, and the preposition \foreign{txela}{under}, which also has the \isi{apocopated form} \transc{txel}. While both forms require a complement, I consider only the apocopated one to be positively marked as \cst*.

\largerpage
\acex{Preposition}{Noun}{59}
{reša/reš mez}
{on/on.\cst{} table}
{on the table}
{KhanSanandaj}{224}\antipar

\acex
{Preposition}{Noun}{59other}
{txela/txel mez}
{under/under.\cst{} table}
{under the table}
{KhanSanandaj}{225}\antipar
\newpage

\subsection{The Ezafe construction} \label{ss:JSan_Ez}

The closest structural parallel of \JSan to the Neo-CSC present in other dialects is the borrowed \ez* construction, in which an \ez* suffix \transc{-e}\~\transc{-y} marks the \prim as such.\footnote{The  variant form \transc{-y} is found in my  fieldwork data.} The form of the \ez* in \JSan seems to indicate a \Per origin, an assumption which is corroborated by its frequent usage with \Per words (see \example{280}).\footnote{The \Per \ez* is \transc{-e}, while in \Sor, it is normally \transc{-i} (but see \sref{ss:Kurdish_cst} for possible variation). Note that in the nearby \Hawr dialect the plural \ez* is realized as \transc{-e} \citep[133]{HolmbergOdden}.}

Indeed, the usage of the \ez* is most frequent \enquote{when the noun is an unadapted  loanword that ends in a consonant rather than in a nominal inflectional vowel} \citep[199]{KhanSanandaj}. These \isi{loanwords} are not necessarily of \ili{Iranic} origin. For instance, in the following example the \prim is the \MishHeb loan-noun \texthebrew{שַׁמָּשׁ} \transc{šămmaš}.  

\acex{Noun}{Noun}{2}
{šămáš-e kništà}
{beadle-\ez{} synagogue}
{the beadle of the synagogue}
{KhanSanandaj}{199 {[A:43]}}

Note that a similar restriction appears in \JSul, where the \Sor borrowed \ez* suffix \transc{-i} is most frequently used with \isi{loanwords} \citep[192f.]{KhanSulemaniyya}:

\acex[\JSul]
{Noun}{Noun}{1220}
{maktáb-i hulayè}
{school-\ez{} Jews}
{school of Jews}
{KhanSulemaniyya}{514 {[R:141]}}

\citet[199]{KhanSanandaj} also gives  examples (possibly elicited) of native Aramaic \prims marked by the \ez*. In these cases, the final number suffix (\sg: \transc{-a} or \pl:   \transc{-e}) is normally retained, but can also be elided in \enquote{fast speech} (in which case the stress falls on the \ez* suffix).

\acex
{Noun}{Noun}{1930}
{bel-\opt{á}-e bărux-i}
{house-\opt\sg-\ez{} friend-\poss.1\sg}
{the house of my friend}
{KhanSanandaj}{199f.}\antipar
\newpage

\acex
{Noun}{Noun}{1930plural}
{bat-\opt{é}-e bărux-i}
{houses-\opt\pl-\ez{} friend-\poss.1\sg}
{the houses of my friend}
{KhanSanandaj}{199f.}\antipar


The \pl* of \foreign{bela}{house} can be the irregular form \transc{baté} or the regular \transc{belé}. Therefore, one could argue that in \exampleabove{1930} the \prim's number distinction is lost. Yet, in my own elicitation I observed a slight phonetic difference between \foreign{belé}{houses} and \foreign{bel-é}{house.\sg-\ez}: in the latter form the \ez* is produced as \phonetic[æ] (and not as the expected \phonetic[e]), which is understandable if this vowel is analysed as a coalescence of a \sg* suffix \phonemic{-a} and an \ez* suffix \phonemic{-e}. On the other hand, the coalescence of the \ez* suffix with the \pl* suffix yields a construction identical to the \isi{juxtaposition} construction, as \citet[200]{KhanSanandaj} notes.


Similarly to the Neo-CSC of other \ili{NENA} dialects, as well as the  CSC of classical \ili{Semitic} languages (such as \BHeb or \Akk), the \ez* construction can be used not only with nominal \secns, but also with   infinitival or clausal \secns:

\acex{Noun}{Infinitival phrase}{97}
{(ʾaná) ḥawṣălá-e [ʾắra tărošè] (lít-i \cb{}u)ˈ}
{1\sg{} patience-\ez{} land build.\inf{} \neg.\exist-1\sg{} \cb{}and}
{(I don't have) the patience to build on the land.}
{KhanSanandaj}{571 {[C:6]}}

\acex{Noun}{Clause}{16}
{ʾo\cb{} baxtá-e [ləxm-ăkè k-ol-a-wa-le \cb{}ó]ˈ}
{\definite\cb{} woman-\ez{} bread-\definite{} \ind-do-\agent3\fem-\pst-\patient3\masc{} \cb{}open}
{the woman who made (lit.\ opened) the bread}
{KhanSanandaj}{381 {[B:22]}}

In contrast to the classical \ili{Semitic} CSC, however, the \ez* construction is also used with adjectival or ordinal \secns. In some cases, where both the \prim and the \secn are \Per words, the entire expression can be seen as a code-switch to \Per, as in the following example, where the AC corresponds to \Per \foreign{\textarabic{لباسِ خراب} ləbā́s-e xărā́b}{ragged clothes}: 

\acex{Noun}{Adjective}{280}
{ləbā́s-e xărā́b (lòš-wa)ˈ}
{clothing(\masc)-\ez{} bad(\invar) wear.3\masc-\pst}
{(He wore) ragged clothes.}
{KhanSanandaj}{251 {[A:108]}}\antipar
\newpage 

 
When used with native Aramaic adjectives as \secns, these inflect as expected:\footnote{Optional inflection of adjectives following the \ez* is attested sporadically also in \Hawr: 

\acexfn[\Hawr]
{Noun}{Adjective}{Hawr1}
{žæn-i zɪl-\opt{æ}}
{woman-\ez{} big-\opt\fem}
{big woman}
{HolmbergOdden}{130, fn.\ 2}\antipar
}

\acex{Noun}{Adjective}{11}
{bel-\opt{á}-e rŭwa}
{house(\textsc{m})-\opt{\sg}-\ez{} big.\masc}
{a big house}
{KhanSanandaj}{251}

\acex
{Noun}{Adjective}{1934}
{pəstan-e ʿista}
{gown(\textsc{f})-\ez{} beautiful.\fem}
{a beautiful gown}
{}{(own fieldwork)}

Ordinals behave similarly to adjectives. The loan-ordinal \foreign{ʾăwaḷ}{first} is invariable, while higher \isi{ordinals} show optional agreement:

\acex{Noun}{Ordinal}{35}
{gorá-e ʾăwaḷ}
{man-\ez{} first(\invar)}
{the first man}
{KhanSanandaj}{213}



\acex{Noun}{Ordinal}{281}
{baxtá-e tre-min-\opt{ta}}
{woman(\fem)-\ez{} two-\ord-\opt\fem}
{the second woman}
{KhanSanandaj}{213}


Note that  adjectives can also serve as the \prim of the \ez* construction (compare \example{99}):

\acex{Adjective}{Infinitive}{98}
{(ʾo\cb{} tré) ḥarik-é šyakà (\cb{}ye-lu).ˈ}
{\definite\cb{} two busy-\ez{} wrestle.\inf{} \cb{}\cop-3\pl}
{The two of them were busy wrestling.}
{KhanSanandaj}{331 (2)}
\antipar
\newpage 

The \ez* construction can easily be embedded. In the following examples, the \prims are NPs consisting themselves of the \ez* construction:

\acex
{Noun Phrase}{Adjective}{1935}
{[pəstan-e kald]-e zărif}
{gown-\ez{} bride-\ez{} beautiful(\invar)}
{a beautiful bridal gown}
{}{(own fieldwork)}

\acex
{Noun Phrase}{Noun}{1936}
{[bel-e smoq]-e tat-i}
{house-\ez{} red(\textsc{m})-\ez{} father-\poss.1\sg}
{the red house of my father}
{}{(own fieldwork)}

In such cases the \ez* behaves very similarly to its \Per model, and can similarly be analysed as a \isi{phrasal suffix} (see \sref{ss:ezafe_dispute}). There are also cases where the \secn consists of the \ez* construction:

\acex
{Noun}{Noun Phrase}{1937}
{bel-e [brat-e ʾamm-i]}
{house-\ez{} daughter-\ez{} aunt-\poss.1\sg}
{the house of my cousin (daughter of my aunt)\footnotemark}
{}{(own fieldwork)}

\footnotetext{Surprisingly, the speaker translated \transl{aunt} as \transc{ʾamma} and not as the expected \transc{ʾamta} \citep[cf.][538]{KhanSanandaj}.}

\acex
{Quantifier}{Noun Phrase}{1938}
{tamam-e [bat-e tat-i]}
{all-\ez{} houses-\ez{} father-\poss.1\sg}
{all the houses of my father}
{}{(own fieldwork)}

Conspicuously missing, in contrast to the \Per model, are cases with \isi{adverbial} \secns (see \example{1917}).  In Khan's description there is only one such case, consisting of the fixed \isi{prepositional phrase} \foreign{ʿăla ḥăda}{aside}, borrowed through \Per from \Arab \textarabic{على حد} \transc{ʿalā ḥadd} \citep[569]{KhanSanandaj}.

\acex
{Noun}{\PP}{12}
{tănurá-e ʿăla-ḥădá}
{oven-\ez{} on-edge}
{a separate oven}
{KhanSanandaj}{252 {[B:18]}} 

 
Productive prepositional phrases are lacking from Khan's description, and did not show up in my elicitation. On the other hand, prepositions and nouns serving as adverbials  can appear as \prims of the \ez* construction. When complemented by clauses  the \isi{relativizer} \transc{ke} is sometimes used as well:

\acex{Preposition}{Noun}{69}
{dawr-e mez}
{around-\ez{} table}
{around the table}
{KhanSanandaj}{220}

\acex{Adverbial noun}{Clause}{78}
{wáxt-e [híye bel-àn]ˈ}
{time-\ez{} came.3\masc{} house-1\pl}
{when he came to our house}
{KhanSanandaj}{394 (4)}

\acex{Adverbial noun}{Clause}{17}
{ta\cb{} zămān-e ke\cb{} [ʾanà xlulá wilí]}
{until\cb{} time-\ez{} \rel\cb{} 1\sg{} wedding did.1\sg}
{Until the time that I married}
{KhanSanandaj}{381 {[A:4]}}

Another case where the \ez* construction is not found is whenever the \prim is a noun serving to quantify the \secn. In such cases the \isi{juxtaposition} construction is used. Consider the following example, where the \secn itself is an \ez* construction (compare \example{1933}).

\acex
{Q. Noun Phrase}{Noun Phrase}{1939}
{[xa lewan] [reza-y yarixa]}
{one cup rice(\textsc{m})-\ez{} long(\masc)}
{one cup of long rice}
{}{(own fieldwork)}\antipar



\subsection {Stress retraction as emerging \isi{construct state} marking} \label{ss:JSan_cst_stress}

In \JSan, stress is commonly word-final. However, in non-pausal contexts, the stress of nouns and pronouns may be retracted \citep[53]{KhanSanandaj}. While this phenomenon occurs more widely than just in ACs, it may be seen as an \textit{emerging} construct-state marking.\footnote{Recall that the historical \ili{Semitic} \cst* began also as a prosodic phenomenon of stress-shift; see \sref{ss:cst_Semitic}.} Consider the following example, with attention to the stress position on the head:

\acex{Noun}{Noun}{9}
{bróna Jăhā̀n}
{son J.}
{the son of Jahan}
{KhanSanandaj}{53 {[A:17]}}

The same phenomenon of stress retraction occurs on the noun \foreign{ʾăxóna}{brother} appearing before an adjective in \example{1}.



\section{Genitive marking of \secns} \label{ss:JSan_gen}

A reflex of the \il{Aramaic!Classical}Classical Aramaic \lnk* \d is retained in \JSan only in one  environment, namely optionally preceding  vowel-initial demonstrative pronouns. As such, it has the same distribution as the \gen* prefix \d found in other dialects, and indeed, in \JSan too it can be analysed as a \gen* prefix, as it has no pronominal force typical of the \lnk* \d.\footnote{Cf.\ however \citet[200]{KhanSanandaj}, who assimilates it to the \lnk*, or \enquote{genitive particle} in his terminology: \textquote{The Aramaic genitive particle \textit{d} is used only when the dependent component of an \isi{annexation} construction contains a \isi{demonstrative pronoun}.} Khan uses the notion \enquote{particle} but in fact it is a bound morpheme.} The detailed argumentation for this analysis is given in \sref{ss:d_gen}, and see in particular \sref{ss:gen_dialectal} regarding \JSan.

The situation in \JSan can be contrasted with the situation in the closely related dialects of Kerend and Qarah Hasan, which have lost all trace of the \d \lnk* and always use the unmarked independent pronouns in the \secn position (\cite[11]{KhanSanandaj}, and see \example{Ker1}). 
 
 In the following, I discuss separately the occurrence of \gen* marked demonstratives as \isi{determiners} and conversely as independent genitive pronouns. A third subsection is devoted to \gen* marked demonstratives preceding clausal \secns.  

\subsection {Genitive determiners} 

Demonstrative pronouns used as \isi{determiners} of \secns  take an optional \gen* marking both after nominal and \isi{adverbial} \prims. Since the marking is optional, in contrast to \JZax, the unmarked forms cannot be analysed as non-genitive, but must rather be seen as unspecified forms (±\gen):

\acex{Noun}{Pronoun}{3}
{bela \opt{d}-o naša}
{house \opt{\gen}-\dem.\far{} man}
{the house of that man}
{KhanSanandaj}{200}

\acex{Preposition}{Pronoun}{58}
{reša/reš \opt{d}-o mez}
{on/on.\cst{} \opt{\gen}-\dem.\far{} table}
{on that table}
{KhanSanandaj}{224}

Note that in the last example, the preposition \transc{reša} can appear either in its full form or in its apocopate \cst* form \transc{reš}. Similarly, the \gen* prefix also follows  \prim nouns marked as \cst* by means of the \ez* (=\example{4bis}):

\acex{Noun}{Noun}{4}
{fešár-e d-o màe}
{pressure-\ez{} \gen-\dem.\far{} water}
{the pressure of the water}
{KhanSanandaj}{200 {[A:59]}}

\subsection {Independent genitive pronouns}  
The demonstrative pronouns may also appear as independent pronouns in the \secn position. In this case  they are obligatorily marked by the \gen* prefix. 


\acex{Noun}{Pronoun}{5}
{bela d-o}
{house \gen-3\sg}
{his house}
{KhanSanandaj}{200}

This situation can be contrasted with the closely related dialects of Kerend and Qarah Ḥasan, where such marking is always absent:

\acex[\Ker]
{Noun}{Pronoun}{Ker1}
{bela o}
{house 3\sg}
{his house}
{KhanSanandaj}{11}

As with the \gen* \isi{determiners}, the \gen* marking also appears  after prepositions, including those marked by apocopate \cst* (compare to \example{58} and contrast with \example{62}). Note that the prepositions often pro-cliticize to their complement, obscuring the fact that the \d is part of the \secn:

\acex{Preposition}{Pronoun}{63}
{ba\cb{} d-o}
{in\cb{} \gen-3\sg}
{in it}
{KhanSanandaj}{218}

\acex{Preposition}{Pronoun}{64}
{reša/reš d-o}
{on/on.\cst{} \gen-3\sg}
{on it}
{KhanSanandaj}{224}





Here too, the \gen* marking also occurs after the \ez*, irrespective of the category of the \prim (and see also \example{49}):


\acex{Noun}{Pronoun}{6}
{belá-e d-o}
{house-\ez{} \gen-3\sg}
{his house}
{KhanSanandaj}{200}



\acex{Preposition}{Pronoun}{73}
{ba-dawr-e d-o}
{in-around-\ez{} \gen-3\sg}
{around it}
{KhanSanandaj}{220}

\acex{Adjective}{Pronoun}{100}
{(ʾā́t hămešá) ḥărík-e d-èaˈ}
{2\sg{} always busy-\ez{} \gen-\dem.\sg}
{You are always busy with this.}
{KhanSanandaj}{570 {[A:102]}}


\subsection {Genitive pronouns preceding clausal \secns} \label{ss:gen_pron_clause}

Certain prepositions can be complemented by a clausal \secn, with the help of an intervening \isi{demonstrative pronoun}, itself marked by the \gen* prefix. 

\acex{Preposition}{Clause}{82}
{bar\cb{} d-èa [ʾay ḥášta wil-à-lu]}
{after\cb{} \gen-\dem.\near.\sg{} \dem.\near{} work(\fem) did-\patient.3\fem-\agent.3\pl}
{after they had done this work}
{KhanSanandaj}{392 {[B:17]}}

\acex
{Preposition}{Clause}{82other}
{qắme d-óa [ʾána b\cb{} ʿolā́m henàˈ]}
{before \gen-\dem.\far.\sg{} 1\sg{} in\cb{} world come.\subj-1\masc}
{before I was born}
{KhanSanandaj}{392 {[A:50]}}

The \gen* marking shows that the \isi{demonstrative pronoun} in each example acts as the \secn, i.e.\ the direct complement of the preposition. As it is followed by a clause, one cannot analyse the \dem* in this position as an NP determiner.\footnote{In \JZax, there are rare cases where a determiner is followed directly by a clause, as in \example{434}, yet in such cases the determiner/\dem* has referential power, quite distinct from the cases discussed here.} Indeed, \citet[392]{KhanSanandaj} writes that \textquote{[the] \isi{demonstrative pronoun} [...] is bound anaphorically to the following content clause}. Yet  what is the exact syntactic relation between the demonstrative and the clause? The role of the \dem* is to provide a nominal head acting as the complement of the preposition. As a nominal head, it governs the   clause and embeds it within an NP. It
 follows that the \isi{demonstrative pronoun} and  the clause stand in an \isi{attributive relation} with each other. Yet  this relationship is not marked, as it is instantiated by the \isi{juxtaposition} construction, discussed in \sref{ss:JSan_juxt}. Only the \isi{attributive relation} between the preposition and the \dem* is positively  marked  by means of a \gen* case prefix. The two attributive relations are schematized in \ref{tb:JSan_prep_clause}: 

\begin{table}[h!]
\centering
\begin{tabular}{ccccc}
\toprule
Prep. & $\mapsto_1$ 		& [\dem\ & $\mapsto_2$ & Clause]\textsubscript{NP.\gen} \\
	  & Genitive-marked		&		 & Zero-marked &								\\
\bottomrule
\end{tabular}
\caption{Clausal complement of a preposition mediated by a demonstrative} \label{tb:JSan_prep_clause}
\end{table}

In other dialects, also the second \isi{attributive relationship} is  marked by means of a \cst* marking of the pronominal \prim. Such is the case in \JUrm, as is shown in \examples{264}{265}. In fact, also in \JSan there are examples in which this relation is marked by a \rel* which follows the \isi{demonstrative pronoun}. 

\acex{Preposition}{Clause}{83}
{bár\cb{} d-ea ke\cb{} [xostá xlulá wil-wa-lù]}
{after\cb{} \gen-\dem.\near.\sg{} \rel\cb{} request marriage did-\pst-3\pl}
{after they made the request of the wedding}
{KhanSanandaj}{392 {[A:34]}}

\subsection{Genitivally marked complements of verbal nouns and verbs} \label{ss:JSan_gen_verbal}

The \isi{genitive case} is also used to mark complements of verbal nouns, be they infinitives, \isi{participles}, or complex-predicate nouns. When this marking appears alongside another AC marking, such as the \ez* in the following example, it is quite clear that it too marks the \isi{attributive relation}.

\acex{CP Noun}{Pronoun}{49}
{daʿwắt-e did-ăxun (wilì)ˈ}
{invitation-\ez{} \gen-2\pl{} (did.1\sg)}
{I invited you.}
{KhanSanandaj}{482 {[D:8]}}

Yet, in other cases, one finds the \gen* marking as the sole exponent of the \isi{attributive relation}, as in the following example which instantiates the \isi{inverse \isi{juxtaposition} construction} (see \sref{ss:JSan_inv_juxt_verbal}):

\acex{CP Noun}{Pronoun}{47}
{(kŭ́le ʾaṣər) did-án daʿwàt (k-ol-í)}
{every evening \gen-1\pl{} invitation \ind-do-3\pl}
{They will invite us every evening.}
{KhanSanandaj}{480 {[D:6]}}

Such cases pose an analytic difficulty as \JSan makes use of the \gen* pronouns  also to mark complements of finite verbs. 

\acex
{Verb}{Pronoun}{1944noPrep}
{did-ox grəš-li}
{\obl-2\masc{} pulled-1\sg}
{I pulled you.}
{KhanSanandaj}{159}

\acex
{Verb}{Pronoun}{JSan1bis}
{d-o grəš-le}
{\obl-3\masc{} pulled-\agent.3\masc}
{He pulled him.}
{KhanSanandaj}{159}


In light of such examples, one may re-interpret the \D\~\transc{did-} morphemes not as \gen* case markers but rather as \obl* case markers, fusing together \acc* and \gen* marking. This may be regarded as a development due to \isi{language contact}, as \obl* case is known in \ili{Iranic} languages, notably in \Kur (see \ref{ss:Kurd_obl}), which is however not in direct contact with \JSan, but also in \Hawr \citep[13]{MacKenzieHawrami}, spoken in closer proximity to Sanandaj. \citet[158]{KhanSanandaj} proposes an alternative cause, explaining \example{1944noPrep} as being a derivation of example \ref{ex:1944mod} below, in which the \acc* preposition \transc{həl} was dropped. The \gen* marking is thus justified, as the pronoun is a complement of a preposition:

\acex
{Verb}{Pronoun}{1944mod}
{həl\cb{} did-ox grəš-li\footnotemark}
{\acc\_Prep\cb{} \gen-2\masc{} pulled-1\sg}
{I pulled you.}
{KhanSanandaj}{158 {[modified]}}

\footnotetext{I took the liberty of changing the agent from \third to \first person, in order to provide a clear parallel to the previous example.}

Be that as it may,  the development of \D\~\transc{did-} into an oblique \isi{case marker} permits us to analyse its occurrence in example \ref{ex:47} as marking the object of the entire verbal complex \foreign{daʿwàt k-ol-í}{they invited} rather than marking an attributive \secn of \transc{daʿwàt} alone. Yet, taking into consideration clear cases such as example \ref{ex:49}, I prefer to analyse \transc{did-} first and foremost as a \gen* \isi{case marker}, being an exponent of the \isi{attributive relation}, whenever this is possible, and see any other grammatical functions as being secondary.

In this vein, I consider the following example to be showing a conjoined NP in the \secn position of the \isi{inverse \isi{juxtaposition} construction}, although only the pronominal complement is marked by \gen* case. 


\acex{CP Noun}{Pronoun+Noun}{48}
{(ʾaxtú tămà) [did-í \cb{}u daăk-í] daʿwát (lá kol-étun)ˈ}
{2\pl{} why \gen-1\sg{} \cb{}and mother-\poss.\sg{} invitation \neg{} \ind.make-2\pl}
{Why do you not invite me and my mother?}
{KhanSanandaj}{482 {[D:8]}}

Similarly, I consider pronominal complements of infinitives as being in \gen* case, as in the following example, exhibiting again the inverse order of constituents typical of verbal constructions in \JSan:

\acex{Infinitive}{Pronoun}{93}
{(ʾila di-le ba\cb{}) did-i găroše}
{hand placed-3\masc{} in\cb{} \gen-1\sg{} pull.\inf}
{He began to pull me.}
{KhanSanandaj}{331}


\section{Dative marking of \secns} \label{ss:JSan_dat}

The elicitation session revealed two examples of an elaborate construction in which the \prim is marked by an \ez* suffix and the \secn is marked by the \isi{dative preposition} \transc{əl-}.\footnote{I did not find any mention of this usage in \citet{KhanSanandaj}.} In both cases, a short pause or hesitation is marked after the \prim, which may explain the speaker's need to re-mark the \secn as such by means of the preposition. Note that the usage of the \isi{dative preposition} to mark \secns is not an innovation but rather a retention, as it is attested also in Syriac (see \sref{ss:dat_lnk}). A similar usage of this preposition is  attested in \NMand \citep[152]{HaberlMandaic}.

\acex
{Noun Phrase}{Noun}{1943}
{[xa\cb{} dana bela]-e ... əl\cb{} [brata amm-i]	}
{one\cb{} unit house-\ez{} {} \dat\cb{} daughter aunt-\poss.1\sg}
{one house of my cousin}
{}{(own fieldwork)}


\acex
{Noun Phrase}{Noun}{1942}
{[bel-e raba ʿayza]-y ... əl\cb{} tat-i}
{house-\ez{} much beautiful(\masc) {} \dat\cb{} father-\poss.1\sg}
{the very beautiful house of my father }
{}{(own fieldwork)}

Note that a similar usage of the \isi{dative preposition} is found in \JSul, but without the \ez* marking:

\acex[\JSul]
{Noun Phrase}{Pronoun}{1106}
{ʾaxonàˈ biš\cb{} zor-ăke ʾəl did-ànˈ}
{brother more\cb{} small-\definite{} \dat{} \gen-1\pl}
{the younger brother of ours}
{KhanSulemaniyya}{262 {[R:104]}}\antipar
\newpage 
 

Similarly, in predicative position one finds \secns marked by the \isi{dative preposition}.  The semantic \prim in such cases is the subject of the clause, but it does not form a syntactic constituent with the \secn. Therefore, I treat this construction as lacking a \prim.

\acex
{\zero}{Noun}{1940}
{ay bela [\zero{} əl\cb{} tat-i] \cb{}y}
{\dem.\near{} house \hphantom{[}\zero{} \dat\cb{} father-\poss.1\sg{} \cb{}\cop}
{This house is my father's.}
{}{{(own fieldwork)}}

Again, there is a very similar construction in \JSul:

\acex[\JSul]
{\zero}{Noun}{1098}
{ˈay\cb{} belá [\zero{} ʾəl\cb{} barux-ì] \cb{}ye.ˈ}
{\dem\cb{} house \hphantom{[}\zero{} \dat\cb{} friend-\poss.1\sg{} \cb{}\cop}
{This house belongs to my friend.}
{KhanSulemaniyya}{262 (9)}

Other dialects use the \lnk* \d in the predicative position (see for instance \examples{548}{549} on \Qar). As \JSan has lost the \lnk* \d, it uses instead the \isi{dative preposition} \transc{əl-}. 



\section{Conclusions} \label{ss:JSan_conclusions}

The AC system of \JSan is highly divergent in comparison to most other \ili{NENA} dialects, and in particular the dialects surveyed in the previous chapters. This divergence is at least partly related to extensive \isi{language contact} with \Sor and \Per. 

The most important innovation of \JSan (and related dialects such as the dialect of \Ker) is the loss of the \il{Aramaic!Classical}Classical Aramaic \lnk* \d. Not only is the \lnk* as such lost, but also its head-marking reflex, the \ed \cst* suffix found in other dialects, is absent in \JSan. The loss of these markers is clearly correlated with the rise of the usage of the zero-marked \isi{juxtaposition} construction in the dialect, discussed in \sref{ss:JSan_juxt}. Yet   the usage of the \isi{juxtaposition} construction is not necessarily a direct consequence of the loss of the D-markers: The D-marked constructions can coexist with the \isi{juxtaposition} construction, as is the case in \JSul (see \examples{1068}{1063}). Rather, it is probable that the \isi{juxtaposition} construction itself is contact induced, as will be discussed in \sref{ss:Juxt_general_usage}.

\newpage  
The only remnant of the \d \lnk* in \JSan is  the \gen* prefix \d used before vowel-initial demonstrative pronouns (see \sref{ss:JSan_gen}). Indeed, this very retention is one of my arguments in favour analysing the \phonetic[d] segment in this position as a \gen* prefix, since it follows an independent development path as compared to the \lnk* \d (see \sref{ss:gen_dialectal}). Possibly through contact with Kurdish or \Hawr, the \d prefix in \JSan shows moreover some progress towards becoming an \obl* \isi{case marker} (see \sref{ss:JSan_gen_verbal}). In the closely related dialects of Kerend and Qarah Ḥasan, on the other hand, even the \gen* prefix  is lost. 

Another interesting retention shared by \JSan and \JSul is the sporadic usage of the \isi{dative preposition} \transc{əl-} to mark \secns, found also in Syriac (see \sref{ss:JSan_dat} and compare to \sref{ss:dat_lnk}). Again, it seems that this retention is correlated with the demise of the usage of the \lnk* \d.


Alongside the extensive usage of the \isi{juxtaposition} construction, there is another construction replacing structurally the \ili{Semitic} CSC, namely the \ili{Iranic} \ez* construction, present both in  \JSul and \JSan. The fact that this construction is still largely confined to \isi{loanwords} may indicate that its introduction to these dialects is a relatively late process, not directly related to the loss of the D-markers or the usage of the \isi{juxtaposition} construction. On the other hand, it may be an indication of the cyclic nature of \isi{language change}: The loss of old grammatical markers (the D-markers) is subsequently compensated by adoption of new grammatical markers (the \ez*). This reasoning has also led us to postulate the possible emergence of a new \cst* marking due to stress shift (see \sref{ss:JSan_cst_stress}).

The grammatical developments discussed above have caused an important structural change in \JSan: In contrast to the situation in \il{Aramaic!Classical}Classical Aramaic, conserved in most \ili{NENA} dialects, the distributional distinction between nominal \secns (occurring typically after \cst* nouns or the \lnk*) and adjectival \secns (occurring typically after \free* nouns) has been levelled, as both the \ez* construction and the \isi{juxtaposition} construction treat these two types of \secns alike, the only difference being that native adjectival \secns agree in gender and number with the \prim. On the other hand, clausal \secns are sometimes signalled as such, as they are optionally preceded by borrowed \rel*s in both these constructions (see discussion in \sref{ss:JSan_rel}).






Finally, another important effect of \isi{language contact} is the emergence of the \isi{inverse \isi{juxtaposition} construction}, in which the \secn precedes the \prim. When this construction occurs with verbal nouns as \prims (see \sref{ss:JSan_inv_juxt_verbal}), this can be explained as a consequence of the general shift of the language to an OV order in the verbal domain. The usage of the inverse construction with ordinal \secns (see \sref{ss:JSan_juxt_inverse_adj}) is most probably a converging borrowing from \ili{Arabic} and Sorani (see \vref{tb:ordinals}). The rare usage of the inverse construction with adjectival \secns may also be related to contact (possibly with \Azr), but this requires further investigation.





\chapter{Pragmatic inference after Grice}\label{sec:9}

\section{Introduction}\label{sec:9.1}

Grice’s work on implicatures triggered an explosion of interest in pragmatics. In the subsequent decades, a wide variety of applications, extensions, and modifications of Grice’s theory have been proposed.



One focus of the theoretical discussion has been the apparent redundancy in the set of maxims and sub-maxims proposed by Grice. Many pragmaticists have argued that the same work can be done with fewer maxims.\footnote{See \citet[ch. 3]{Birner20122013} for a good summary of the competing positions on this issue.} In the extreme case, proponents of Relevance Theory have argued that only the Principle of Relevance is needed.



Rather than focusing on such theoretical issues directly, in this chapter we will discuss some of the analytical questions that have been of central importance in the development of pragmatics after Grice. In \sectref{sec:9.2} we return to the question raised in \chapref{sec:4} concerning the degree to which the English words \textit{and}, \textit{or}, and \textit{if} have the same meanings as the corresponding logical operators. Grice himself suggested that some apparently distinct “senses” of these words could be analyzed as generalized conversational implicatures. \sectref{sec:9.3} discusses a type of pragmatic “enrichment” that seems to be required in order to determine the truth-conditional meaning of a sentence. \sectref{sec:9.4} discusses how the relatively clean and simple distinction between semantics vs. pragmatics which we have been assuming up to now is challenged by recent work on implicatures.


\section{Meanings of English words vs. logical operators}\label{sec:9.2}

As we hinted in \chapref{sec:4}, the logical operators $\wedge$ ‘and’, $\vee$ ‘or’, and → ‘if…then’ seem to have a different and often narrower range of meaning than the corresponding English words. A number of authors have claimed that the English words are ambiguous, with the logical operators corresponding to just one of the possible senses. Grice argued that each of the English words actually has only a single sense, which is more or less the same as the meaning of the corresponding logical operator, and that the different interpretations arise through pragmatic inferences. Before we examine these claims in more detail, we will first illustrate the variable interpretations of the English words, in order to show why such questions arise in the first place.



Let us begin with \textit{and}.\footnote{We focus here on the use of \textit{and} to conjoin two clauses (or VPs), since this is closest to the function of logical $\wedge$. We will not be concerned with coordination of other categories in this chapter.} The truth table in \chapref{sec:4} makes it clear that logical $\wedge$ is commutative; that is, \textit{p$\wedge$}\textit{q} is equivalent to \textit{q$\wedge$}\textit{p}. This is also true for some uses of English \textit{and}, such as \REF{ex:9.1}. In other cases, however, such as (\ref{ex:9.2}--\ref{ex:9.4}), reversing the order of the clauses produces a very different interpretation.


\ea \label{ex:9.1}
\ea The {Chinese} invented the folding umbrella and the Egyptians invented the sailboat.\\
\ex The Egyptians invented the sailboat and the {Chinese} invented the folding umbrella.
                       \z
\z

\ea \label{ex:9.2}
\ea She gave him the key and he opened the door.\\
\ex He opened the door and she gave him the key.
                       \z
\z

\ea \label{ex:9.3}
\ea The Lone Ranger jumped onto his horse and rode into the sunset.\footnote{\citet[56]{Kempson1975}, cited in \citet{Gazdar1979}.}\\
\ex ?The Lone Ranger rode into the sunset and jumped onto his horse.
                       \z
\z

\ea \label{ex:9.4}
\ea The janitor left the door open and the prisoner escaped.\\
\ex ?The prisoner escaped and the janitor left the door open.
                       \z
\z


It has often been noted that when \textit{and} conjoins clauses which describe specific events, as (\ref{ex:9.2}--\ref{ex:9.3}), there is a very strong tendency to interpret it as meaning ‘and then’, i.e., to assume a sequential interpretation. When the second event seems to depend on or follow from the first, as in (\ref{ex:9.4}a), there is a tendency to assume a causal interpretation, ‘and therefore’. The question to be addressed is, do such examples prove that English \textit{and} is ambiguous, having two or three (or more) distinct senses?



We stated in \chapref{sec:4} that the $\vee$ of standard logic is the “inclusive or”, corresponding to the English \textit{and/or}. We also noted that the English word \textit{or} is often used in the “exclusive” sense (XOR), meaning ‘either … or … but not both’. Actually either interpretation is possible, depending on the context, as illustrated in \REF{ex:9.5}. (The reader should determine which of these examples contains an \textit{or} that would most naturally be interpreted with the exclusive reading, and which with the inclusive reading.) Does this variable interpretation mean that English \textit{or} is ambiguous? 


\ea \label{ex:9.5}
\ea Every year the Foundation awards a scholarship to a student of {Swedish} or {Norwegian} ancestry.\\
\ex You can take the bus or the train and still arrive by 5 o’clock.\\
\ex If the site is in a particularly sensitive area, or there are safety considerations, we can refuse planning permission.\footnote{\citet[113]{Saeed2009}.}\\
\ex Stop or I’ll shoot!\footnote{\citet[113]{Saeed2009}.}
                       \z
\z


Finally let us briefly consider the meaning of material implication (→) compared with English \textit{if}. If these two meant the same thing, then according to the truth table for material implication in \chapref{sec:4}, all but one of the sentences in \REF{ex:9.6} should be true. (The reader can refer to the truth table to determine which of these sentences is predicted to be false.) However, most English speakers find all of these sentences very odd; many speakers are unwilling to call any of them true.


\ea \label{ex:9.6}
\ea If Socrates was a woman then $1+1=3$.\footnotemark{} \\\addtocounter{footnote}{-1}
\ex If 2 is odd then 2 is even.\footnote{\url{http://en.wikipedia.org/wiki/Material_conditional}}\\
\ex If a triangle has three sides then the moon is made of green cheese.\\
\ex If the {Chinese} invented gunpowder then Martin Luther was {German}.
                       \z
\z


Similarly, analyzing English \textit{if} as material implication in \REF{ex:9.7} would predict some unlikely inferences, based on the rule of \textit{modus tollens}.


\ea \label{ex:9.7}
\ea If you’re hungry, there’s some pizza in the fridge.\\
  (predicted inference: \#If there’s no pizza in the fridge, then you’re not hungry.)\\
\ex If you really want to know, I think that dress is incredibly ugly.\\
  (predicted inference: \#If I don’t think that dress is ugly, then you don’t really want to know.)
                       \z
\z


Part of the oddness of the “true” sentences in \REF{ex:9.6} relates to the fact that material implication is defined strictly in terms of truth values; there does not have to be any connection between the meanings of the two propositions. English \textit{if}, on the other hand, is normally used only where the two propositions do have some sensible connection. Whether this preference can be explained purely in pragmatic terms is an interesting issue, as is the question of how many senses we need to recognize for English \textit{if} and whether any of these senses are equivalent to →. We will return to these questions in \chapref{sec:19}. In the present chapter we focus on the meanings of \textit{and} and \textit{or}.


\subsection{On the ambiguity of \textit{and}}\label{sec:9.2.1} 

In \chapref{sec:8} we mentioned that the sequential (‘and then’) use of English \textit{and} can be analyzed as a generalized conversational implicature motivated by the maxim of manner, under the assumption that its semantic content is simply logical \textit{and} ($\wedge$). An alternative analysis, as mentioned above, involves the claim that English \textit{and} is polysemous, with logical \textit{and} ($\wedge$) and sequential ‘and then’ as two distinct senses. Clearly both uses of \textit{and} are possible, given the appropriate context; example (\ref{ex:9.8}a) (like (\ref{ex:9.1}a) above) is an instance of the logical \textit{and} use, while (\ref{ex:9.8}b) (like (\ref{ex:9.1}b-c) above) is most naturally interpreted as involving the sequential ‘and then’ use. The question is whether we are dealing with semantic ambiguity (two distinct senses) or pragmatic inference (one sense plus a potential conversational implicature). How can we decide between these two analyses?


\ea \label{ex:9.8}
\ea Hitler was Austrian and Stalin was {Georgian}.\\
\ex They got married and had a baby.
                       \z
\z


\citet{Horn2004} mentions several arguments against the lexical ambiguity analysis for \textit{and}:


\begin{enumerate}[label=\roman*.]
\item The same two uses of \textit{and} are found in most if not all languages. Under the semantic ambiguity analysis, the corresponding conjunction in (almost?) every language would just happen to be ambiguous in the same way as in English.
\item No natural language contains a conjunction \textit{shmand} that would be ambiguous between “and also” and “and earlier” readings so that \textit{They had a baby shmand they got married} would be interpreted either atemporally (logical \textit{and}) or as “They had a baby and, before that, they got married.”
\item Not only temporal but causal asymmetry (‘and therefore’, illustrated in (\ref{ex:9.1}d)) would need to be treated as a distinct sense. And a variety of other uses (involving “stronger” or more specific uses of the conjunction) arise in different contexts of utterance. How many senses are we prepared to recognize?
\item The same “ambiguity” exhibited by \textit{and} arises when two clauses describing related events are simply juxtaposed (\textit{They had a baby. They got married.}). This suggests that the sequential interpretation is not in fact contributed by the conjunction \textit{and}.
\item The sequential ‘and then’ interpretation is defeasible, as illustrated in \REF{ex:9.9}. This strongly suggests that we are dealing with conversational implicature rather than semantic ambiguity.
\end{enumerate}

\ea \label{ex:9.9}
They got married and had a baby, but not necessarily in that order.
\z


Taken together, these arguments seem quite persuasive. They demonstrate that English \textit{and} is not polysemous; its semantic content is logical \textit{and} ($\wedge$). The sequential ‘and then’ use can be analyzed as a generalized conversational implicature.


\subsection{On the ambiguity of \textit{or}}\label{sec:9.2.2}

As noted in \chapref{sec:4}, similar questions arise with respect to the meaning(s) of \textit{or}. The English word \textit{or} can be used in either the inclusive sense ($\vee$) or the exclusive sense (XOR). The inclusive reading is most likely in (\ref{ex:9.10}a--b), while the exclusive reading is most likely in (\ref{ex:9.10}c--d).


\ea \label{ex:9.10}
\ea Mary has a son or daughter.\footnote{Barbara Partee, 2004 lecture notes. \url{http://people.umass.edu/partee/RGGU_2004/RGGU047.pdf}} \\
\ex We would like to hire a sales manager who speaks  {Chinese} or  {Korean}.\\
\ex I can’t decide whether to order fried noodles or pizza.\\
\ex Stop or I’ll shoot!\footnote{\citet[113]{Saeed2009}.}
                       \z
\z


Barbara Partee points out that examples like \REF{ex:9.11} are sometimes cited as sentences where only the exclusive reading of \textit{or} is possible; but in fact, such examples do not distinguish the two senses. These are cases where our knowledge of the world makes it clear that both alternatives cannot possibly be true. She says that such cases involve “intrinsically mutually exclusive alternatives”. Because we know that \textit{p$\wedge$q} cannot be true in such examples, \textit{p$\vee$q} and \textit{pXORq} are indistinguishable; if one is true, the other must be true as well.


\ea \label{ex:9.11}
\ea Mary is in Prague or she is in Stuttgart.\footnote{Barbara Partee, 2004 lecture notes. \url{http://people.umass.edu/partee/RGGU_2004/RGGU047.pdf}} \\
\ex Christmas falls on a Friday or Saturday this year.
                       \z
\z


\citet{Grice1978} argues that English \textit{or}, like \textit{and}, is not polysemous. Rather, its semantic content is inclusive \textit{or} ($\vee$), and the exclusive reading arises through a conversational implicature motivated by the maxim of quantity.



In fact, using \textit{or} can trigger more than one implicature. If a speaker says \textit{p or q} but actually knows that \textit{p} is true, or that \textit{q} is true, he is not being as informative as required or expected. So the statement \textit{p or q} triggers the implicature that the speaker does not know \textit{p} to be true or \textit{q} to be true. By the same reasoning, it triggers the implicature that the speaker does not know either \textit{p} or \textit{q} individually to be false. Now if \textit{p} and \textit{q} are both true, and the speaker knows it, it would be more informative (and thus expected) for the speaker to say \textit{p and q}. If he instead says \textit{p or q}, he is violating the maxim of quantity. Thus the statement \textit{p or q} also triggers the implicature that the speaker is not in a position to assert \textit{p and q}.



So in contexts where the speaker might reasonably be expected to know if \textit{p and q} were true, the statement \textit{p or q} will trigger the implicature that \textit{p and q} is not true, which produces the exclusive reading. When nothing can be assumed about the speaker’s knowledge, it is harder to see how to derive the exclusive reading from Gricean principles; several different explanations have been proposed. But another reason for thinking that the exclusive reading arises through a conversational implicature is that it is defeasible, e.g. \textit{I will order either fried noodles or pizza; in fact I might get both}.



\citet[81--82]{Gazdar1979} presents another argument against analyzing English \textit{or} as being polysemous. If \textit{or} is ambiguous between an inclusive and an exclusive sense, then when sentences containing \textit{or} are negated, the result should also be ambiguous, with senses corresponding to \textit{¬(p$\vee$q)} vs. \textit{¬(pXORq)}. The crucial difference is that \textit{¬(pXORq)} will be true and \textit{¬(p$\vee$q)} false if \textit{p$\wedge$q} is true. (The reader should consult the truth tables in \chapref{sec:4} to see why this is the case.) For example, if \textit{or} were ambiguous, sentence (\ref{ex:9.12}a) should allow a reading which is true if Mary has both a son and a daughter, and (\ref{ex:9.12}b) should allow a reading under which I would allow my daughter to marry a man who both smokes and drinks. However, for most English speakers these readings of (\ref{ex:9.12}a--b) are not possible, 
at least when the sentences are read with normal intonation
 
\largerpage 
\ea \label{ex:9.12}
\ea Mary doesn’t have a son or daughter.\footnote{Barbara Partee, 2004 lecture notes. \url{http://people.umass.edu/partee/RGGU_2004/RGGU047.pdf}} \\
\ex The man who marries my daughter must not smoke or drink.
                       \z
\z


\citet[47]{Grice1978}, in the context of discussing the meaning of \textit{or}, proposed a principle which he called \textbf{Modified Occam’s Razor}: “Senses are not to be multiplied beyond necessity.” This principle would lead us to favor an analysis of words like \textit{and} and \textit{or} as having only a single sense, with additional uses being derived by pragmatic inference, unless there is clear evidence in favor of polysemy.


\section{Explicatures: bridging the gap between what is said vs. what is implicated}\label{sec:9.3}

Grice’s model seems to assume that the speaker meaning (total meaning that the speaker intends to communicate) is the sum of the sentence meaning (“what is said”, i.e., the meaning linguistically encoded by the words themselves) plus implicatures. Moreover, implicatures were assumed not to affect the truth value of the proposition expressed by the sentence; truth values were assumed to depend only on sentence meaning.\footnote{Of course, the implicatures themselves also have propositional content, which may be true or false/misleading even if the literal sentence meaning is true.}



In many cases, however, the meaning linguistically encoded by the words themselves does not amount to a complete proposition, and so cannot be evaluated as being either true or false. Grice recognized that the proposition expressed by a sentence like (\ref{ex:9.13}a) is not complete, and its truth value cannot be determined, until the referents of pronouns and deictic elements are specified. Most authors also assume that any potential ambiguities in the linguistic form (like the syntactic and lexical ambiguities in \ref{ex:9.13}b) must be resolved before the propositional content and truth conditions of the sentence can be determined.


\ea \label{ex:9.13}
\ea She visited me here yesterday.\\
\ex Old men and women gathered at the bank.
                       \z
\z


Determining reference and disambiguation both depend on context, and so involve a limited kind of pragmatic reasoning. However, it turns out that there are many cases in which more significant pragmatic inferences are required in order to determine the propositional content of the sentence. Kent Bach (\citeyear{Bach1994}) identifies two sorts of cases where this is needed: “Filling in is needed if the sentence is semantically \textsc{under-determinate}, and fleshing out will be needed if the speaker cannot plausibly be supposed to mean just what the sentence means.”



The first type, which Bach refers to as \textsc{semantic under-determination}, involves sentences which fail to express a complete proposition (something capable of being true or false), even after the referents of pronouns and deictic elements have been determined and ambiguities resolved; some examples are presented in \REF{ex:9.14}.\footnote{Examples (\ref{ex:9.14}–\ref{ex:9.19}) are adapted from \citet{Bach1994}.}


\ea \label{ex:9.14}
\ea Steel isn’t strong enough.\\
\ex Strom is too old.\\
\ex The princess is late.\\
\ex Tipper is ready.
                       \z
\z


In these cases a process of \textsc{completion} (or “filling in” the missing information) is required to produce a complete proposition. This involves adding information to the propositional meaning which is unexpressed but implicit in the original sentence, as indicated in \REF{ex:9.15}. The hearer must be able to provide this information from context and/or knowledge of the world. The truth values of these sentences can only be determined after the implicit constituent is added to the overtly expressed meaning.


\ea \label{ex:9.15}
\ea Steel isn’t strong enough [to stop this kind of anti-tank missile].\\
\ex Strom is too old [to be an effective senator].\\
\ex The princess is late [for the party].\\
\ex Tipper is ready [to dance].
                       \z
\z


The under-determination of the sentences in \REF{ex:9.14} is not due to syntactic deletion or ellipsis; they are semantically incomplete, but not syntactically incomplete. The examples in (\ref{ex:9.16}--\ref{ex:9.17}) show that the potential for occurring in such constructions may be lexically specific, and that close synonyms may differ in this respect.


\ea \label{ex:9.16}
\ea[]{The king has arrived [at the palace].}
\ex[*]{The king has reached.}
                       \z
\z

\ea \label{ex:9.17}
\ea[]{Al has finished [speaking].}
\ex[*]{Al has completed.}
                       \z
\z


The second type of sentence that Bach discusses involves those in which “there is already a complete proposition, something capable of being true or false (assuming linguistically unspecified references have been assigned and any ambiguities have been resolved), albeit not the one that is being communicated by the speaker.” For example, imagine that a mother says (\ref{ex:9.18}a) to her young son who is crying loudly because he cut his finger.


\ea \label{ex:9.18}
\ea You’re not going to die.\\
\ex You’re not going to die [from this cut].
                       \z
\z


Clearly she does not intend to promise immortality, although that is what the literal meaning of her words seems to say. In order to determine the intended propositional content of the sentence, the meaning has to be \textsc{expanded} (or “fleshed out”) as shown in (\ref{ex:9.18}b). Once again, the hearer must be able to provide this additional information from context and/or knowledge of the world. A more complex kind of pragmatic reasoning is required here than would be involved in assigning referents to deictic elements or resolving lexical ambiguities. Further examples are provided in \REF{ex:9.19}, illustrating how identical sentence structures can be expanded differently on the basis of knowledge about the world.


\ea \label{ex:9.19}
\ea I have eaten breakfast [today].\\
\ex I have eaten caviar [before].\\
\ex I have nothing to wear [nothing appropriate for a specific event].\\
\ex I have nothing to repair [nothing at all].
                       \z
\z


Bach uses the term \textsc{impliciture} to refer to the kinds of inference illustrated in this section. The choice of this label is not ideal, because the words \textit{impliciture} and \textit{implicature} look so much alike. A very similar concept is discussed within Relevance Theory under the label \textsc{explicature},\footnote{\citet{SperberWilson1986}; \citet{Carston1988}.} expressing the idea that the overtly expressed content of the sentence needs to be explicated in order to arrive at the full sentence meaning intended by the speaker. In the discussion that follows we will adopt the term \textsc{explicature}.\footnote{We are ignoring for now the relatively minor differences between Bach’s notion of impliciture and the Relevance Theory notion of explicature; see \citet{Bach2010} for discussion.}



\citet[11]{Bach1994} describes the difference between “impliciture” (=explicature) and implicature as follows:


\begin{quote}
Although both impliciture and implicature go beyond what is explicit in the utterance, they do so in different ways. An implicatum is completely separate from what is said and is inferred from it (more precisely, from the saying of it). What is said is one proposition and what is communicated in addition to that is a conceptually independent proposition, a proposition with perhaps no constituents in common with what is said... 
\end{quote}

\begin{quote}
In contrast, implicitures are built up from the explicit content of the utterance by conceptual strengthening … which yields what would have been made fully explicit if the appropriate lexical material had been included in the utterance. Implicitures are, as the name suggests, implicit in what is said, whereas implicatures are implied by (the saying of) what is said. 
\end{quote}


In other words, implicatures are distinct from sentence meaning. They are communicated in addition to the sentence meaning and have independent truth values. A true statement could trigger a false implicature, or vice versa. Explicatures are quite different. The truth value of the sentence cannot be determined until the explicatures are added to the literal meanings of the words.



Since explicatures involve pragmatic reasoning, we must recognize the fact that pragmatic inferences can affect truth-conditional content. Further evidence that supports this same conclusion is discussed in the following section.


\section{Implicatures and the semantics/pragmatics boundary}\label{sec:9.4}

In \chapref{sec:1} we defined the semantic content of an expression as the meaning that is associated with the words themselves, independent of context. We defined pragmatic meaning as the meaning which arises from the context of the utterance. We have implicitly assumed that the truth conditions of a sentence depend only on the “semantic content” or sentence meaning, and not on pragmatic meaning. Many authors have made the same assumption, using the term “truth conditional meaning” as a synonym for “sentence meaning”. However, our discussion of explicatures has demonstrated that this view is too simplistic. Additional challenges to this simplistic view arise from research on implicatures.



As already discussed in \chapref{sec:8}, the conventional implicatures associated with words like \textit{but} or \textit{therefore} are part of the conventional meaning of these words, and not context-dependent; they would be part of the relevant dictionary definitions and must be learned on a word-by-word basis. Nevertheless, both Frege and Grice argued that these conventional implicatures do not contribute to the truth conditions of a sentence. So conventional meaning is not always truth-conditional. We will discuss this issue in more detail in \chapref{sec:11}.



The opposite situation has been argued to hold in the case of generalized conversational implicatures. In \sectref{sec:9.2} above we presented compelling evidence which shows that the sequential ‘and then’ use of \textit{and} is not due to lexical ambiguity (polysemy), but must be a pragmatic inference. It is often cited as a paradigm example of generalized conversational implicature. However, as noted by \citet{Levinson1995,Levinson2000} among others, this inference does affect the truth conditions of the sentence in examples like (\ref{ex:9.20}--\ref{ex:9.21}). Sentence (\ref{ex:9.20}a) could be judged to be true in the same context where (\ref{ex:9.20}b) is judged to be false. This difference can only be due to the sequential interpretation of \textit{and}; if \textit{and} means only $\wedge$, then the two sentences are logically equivalent. Similarly, if \textit{and} means only $\wedge$, then \REF{ex:9.21} should be a contradiction; the fact that it is not can only be due to the sequential interpretation of \textit{and}.


\ea \label{ex:9.20}
\ea  If the old king has died of a heart attack and a republic has been declared, then Tom will be quite content.\footnote{\citet[58]{Cohen1971}.}
\ex  If a republic has been declared and the old king has died of a heart attack, then Tom will be quite content.\footnote{\citet[69]{Gazdar1979}.}
\z \z

\ea \label{ex:9.21}
If he had three beers and drove home, he broke the law; but if he drove home and had three beers, he did not break the law.
\z


Such examples have been extensively debated, and a variety of analyses have been proposed. For example, proponents of Relevance Theory argue that the sequential ‘and then’ use of \textit{and} is an explicature: a pragmatic inference that contributes to truth conditions.\footnote{\citet{Carston1988,Carston2004}.} A similar analysis is proposed for most if not all of the inferences that Grice and the “neo-Griceans” have identified as generalized conversational implicatures: within Relevance Theory they are generally treated as explicatures.



This controversy is too complex to address in any detail here, but we might make one observation in passing. At the beginning of \chapref{sec:8} we provided an example (the story of the captain and his mate) of how we can use a true statement to implicate something false. That example involved a particularized conversational implicature, but it is possible to do the same thing with generalized conversational implicatures as well. The following example involves a scalar implicature. It is taken from a news story about how Picasso’s famous mural “Guernica” was returned to Spain after Franco’s death. The phrase \textit{Not all of them} in this context implicates \textit{not none} (that is, ‘I have some of them’) by the maxim of Quantity, because \textit{none} is a stronger (more informative) term than \textit{not all}.


\ea \label{ex:9.22}
  To demonstrate that the  {Spanish} Government had in fact paid Picasso to paint the mural in 1937 for the Paris International Exhibition, Mr. Fernandez Quintanilla had to secure documents in the archives of the late Luis Araquistain, Spain’s Ambassador to France at the time. But Araquistain’s son, poor and opportunistic, demanded \$2 million for the archives, which Mr. Fernandez Quintanilla rejected as outrageous. He managed, however, to obtain from the son photocopies of the pertinent documents, which in 1979 he presented to Roland Dumas [Picasso’s lawyer]… \\
  “This changes everything,” a startled Mr. Dumas told the  {Spanish} envoy when he showed him the photocopies of the Araquistain documents. “You of course have the originals?” the lawyer asked casually. “\textbf{\textit{Not all of them}},” replied Mr. Fernandez Quintanilla, not lying but not telling the truth, either.\\
   {}[\textit{The New York Times}, November 2, 1981; cited in \citet{Horn1992}]
\z


Mr. Fernandez Quintanilla was not lying, because the literal sentence meaning of his statement was true. But he was not exactly telling the truth either, because his statement triggered (and was clearly intended to trigger) an implicature that was false; in fact he had none of the originals.



Such examples show that generalized conversational implicatures can be used to communicate false information, even when the literal meaning of the sentence is true. It would be hard to account for this fact if these generalized conversational implicatures are considered to be explicatures, because explicatures do not have a truth value that is independent of the truth value of the literal sentence meaning. Rather, explicatures represent inferences that are needed in order to determine the truth value of the sentence.


\subsection{Why numeral words are special}\label{sec:9.4.1}

\largerpage[-1]
Scalar implicatures have received an enormous amount of attention in the recent pragmatics literature. Many early discussions of scalar implicatures relied heavily on examples involving cardinal numbers, which seem to form a natural scale (1, 2, 3, …). However, various authors have pointed out that numbers behave differently from other scalar terms.



\citet{Horn2004} uses examples (\ref{ex:9.23}--\ref{ex:9.25}) to bring out this difference. On the scale <\textit{none, some, many, all}>,  \textit{all} is a stronger (more informative) term than \textit{many}. Therefore, by the maxim of quantity, A’s use of \textit{many} in \REF{ex:9.23} entails ‘(at least) many’ and implicates ‘not all’.\footnote{\textit{Many} is used here in its proportional sense; see \chapref{sec:14} for discussion.} B’s reply states that the implicature does not in fact hold in the current situation; but this does not render the propositional content of the sentence false. That is why it would be unnatural for B to begin the reply with \textit{No}, as in B1. The acceptability of reply B2 follows from the fact that implicatures are defeasible.


\ea \label{ex:9.23}
A: Did many of the guests leave?\\
B1: ?No, all of them.\\
B2: Yes, (in fact) all of them.
\z


If numerals behaved in the same way as other scalars, we would expect A’s use of \textit{two} in \REF{ex:9.24} to entail ‘at least two’ and implicate ‘not more than two’. However, if B actually does have more than two children, it seems to be more natural here for B to reply with \textit{No} rather than \textit{Yes}. This indicates that B is rejecting the literal propositional content of the question, not an implicature.


\ea \label{ex:9.24}
A: Do you have two children?\\
B1: No, three.\\
B2: ?Yes, (in fact) three.
\z


Such examples suggest that numerals like \textit{two} allow two distinct readings: an ‘at least 2’ reading vs. an ‘exactly 2’ reading, and that neither of these is derived as an implicature from the other. A’s question in \REF{ex:9.24} is most naturally interpreted as involving the ‘exactly’ reading. However, there are certain contexts (such as discussing a government subsidy that is available for families with two or more children) in which the ‘at least’ reading would be preferred, and in such contexts reply B2 would be more natural.

 
Example (\ref{ex:9.25}a) is acceptable under the ‘exactly 3’ reading of the numeral, under which \textit{not three} is judged to be true whether the actual number is more than three or less than three. The fact that (\ref{ex:9.25}b) is unacceptable shows that the word \textit{like} does not have an ‘exactly (or merely) like’ reading. Based on the scale <\textit{hate, dislike, neutral, like, love/adore}>, using the word \textit{like} entails ‘at least like (=have positive feelings)’ and implicates ‘not more than like (not love/adore)’. Sentence (\ref{ex:9.25}b) attempts to negate the both the entailment and the implicature at the same time, and the result is unacceptable.\footnote{Of course, as pointed out at the end of \chapref{sec:8}, given the right context and using a special marked intonation it is sometimes possible to negate the implicature alone, as in: “She didn’t \textsc{líke} the movie — she \textsc{adóred} it.”}


\ea \label{ex:9.25}
\ea[]{Neither of us has three kids — she has two and I have four.}
\ex[\#]{Neither of us liked the movie — she adored it and I hated it.}
                       \z
\z


\citet{Horn1992} notes several other properties which set numerals apart from other scalar terms, and which demonstrate the two distinct readings for numerals:


\begin{enumerate}
\item Mathematical statements do not allow “at least” readings (\ref{ex:9.26}a). Also, round numbers are more likely to allow “at least” readings than very precise numbers (\ref{ex:9.26}b--c).

\ea \label{ex:9.26}
\ea[*]{$2+2=3$ (should be true under “at least 3” reading)}
\ex[]{I have \$200 in my bank account, if not more.}
\ex[]{I have \$201.37 in my bank account, \#if not more.}
                       \z
\z
 
\item Numerical scales are potentially reversible depending on the context (\ref{ex:9.27}--\ref{ex:9.28}); this kind of reversal is not possible with other scalar terms \REF{ex:9.29}.
\ea \label{ex:9.27}
\ea That bowler is capable of breaking 100 (he might even score 150).\\
\ex That golfer is capable of breaking 100 (he might even score 90).
                       \z
\z

\ea \label{ex:9.28}
\ea You can survive on 2000 calories per day (or more).\\
\ex You can lose weight on 2000 calories per day (or less).
                       \z
\z

\ea \label{ex:9.29}
\ea He ate some of your mangoes, if not all/*none of them.\\
\ex This classroom is always warm, if not hot/*cool.
                       \z
\z

\item The “at least” interpretation is only possible with the distributive reading of numerals, not the collective reading \REF{ex:9.30}; this is not the case with other scalar quantifiers \REF{ex:9.31}.

\largerpage
\ea \label{ex:9.30}
\ea Four salesmen have called me today, if not more.\\
\ex Four students carried this sofa upstairs for me, \#if not more.
                       \z
\z

\ea \label{ex:9.31}
\ea Most of the students have long hair, perhaps all of them.\\
\ex Most of the students surrounded the stadium, perhaps all of them.
                       \z
\z

\item The “at least” interpretation is disfavored when a numeral is the focus of a question \REF{ex:9.32}, but this is not the case with other scalar quantifiers \REF{ex:9.33}:

\ea \label{ex:9.32}
Q: Do you have two children?\\
A1: No, three.\\
A2: ?Yes, in fact three.
\z

\ea \label{ex:9.33}
Q: Are many of your friends linguists?\\
A1: ??No, all of them.\\
A2: Yes, in fact all of them.
\z
\end{enumerate}


It is important to bear in mind that sentences like \REF{ex:9.34} can have different truth values depending on which reading of the numeral is chosen:


\ea \label{ex:9.34}
If Mrs. Smith has three children, there will be enough seatbelts for the whole family to ride together.
\z


One possible analysis might be to treat the alternation between the ‘at least \textit{n}’ vs. ‘exactly \textit{n}’ readings as a kind of systematic polysemy. However, it seems that most pragmaticists prefer to treat numeral words as being underspecified or indeterminate between the two, with the intended reading in a given context being supplied by explicature.\footnote{See for example \citet{Horn1992} and \citet{Carston1998}.}

\largerpage
\section{Conclusion}\label{sec:9.5}

The large body of work exploring the implications of Grice’s theory has forced us to recognize that Grice’s relatively simple view of the boundary between semantics and pragmatics is not tenable. Early work in pragmatics often assumed that pragmatic inferences did not affect the truth-conditional content of an utterance, apart from the limited amount of contextual information needed for disambiguation of ambiguous forms, assignment of referents to pronouns, etc. Under this view, truth-conditional content is almost the same thing as conventional meaning.

 
In this chapter we have discussed various ways in which pragmatic inferences do contribute to truth-conditional content. We have seen that some (at least) generalized conversational implicatures affect truth-conditions, and we have seen that other types of pragmatic inferences, which we refer to as explicatures, are needed in order to determine the truth value of a sentence. In \chapref{sec:11} we discuss the opposite kind of challenge, namely cases where conventional meaning (semantic content) does not contribute to the truth-conditional meaning of a sentence. But first, in \chapref{sec:10}, we discuss a special type of conversational implicature known as an \textsc{indirect speech act}.



\furtherreading{



\citet[ch. 3]{Birner20122013} presents a good overview of the issues discussed here, including a very helpful comparison of Relevance Theory with the “neo-Gricean” approaches of Levinson and Horn. \citet{Horn2004} and \citet{Carston2004} provide helpful surveys of recent work on implicature, Horn from a neo-Gricean perspective and Carston from a Relevance Theory perspective. \citet{Bach2010} discusses the differences between his notion of “impliciture” and the Relevance Theory notion of explicature. \citet{Geurts2011} provides a good introduction to, and a detailed analysis of, scalar and quantity implicatures.

} 
\discussionexercises{
\paragraph*{A. Explicature.}

Identify the explicatures which would be necessary in order to evaluate the truth value for each of the following examples:\footnote{Examples (3-5) are taken from \citet{CarstonHall2012}.}
\bigskip

\begin{enumerate}
\item  {He arrived at the bank too early}.
\item  {All students must pass phonetics}.
\item  {No-one goes there anymore}.
\item  {To buy a house in London you need money}.
\item {}[Max: How was the party? Did it go well?]\\
  Amy:  {There wasn’t enough drink and everyone left early}.
\end{enumerate}
\paragraph*{B. Pragmatics in the lexicon.}

\citet{Horn1972} observes that many languages have lexical items which express positive universal quantification (\textit{all, every, everyone, everything, always, both}, etc.) and the corresponding negative concepts (\textit{no, none, nothing, no one, never, neither}, etc.). In each case, the positive term can be paraphrased in terms of the corresponding negative, and vice versa. For example, \textit{Everything is negotiable} can be paraphrased as \textit{Nothing} \textit{is non-negotiable}. However, most languages seem to lack negative counterparts to the existential quantifiers (\textit{some, someone, sometimes}, etc.). In order to paraphrase an existential statement like \textit{Something is negotiable}, we have to use a quantifying phrase, rather than a single word, as in \textit{Not everything is non-negotiable}.

Try to formulate a pragmatic explanation for this lexical asymmetry, i.e., the fact that few if any languages have lexical items that mean \textit{not everything, not everyone, not always, not both,} etc. (\textbf{Hint}: think about the kinds of implicatures that might be triggered by the various classes of quantifying words.)
}

\chapter{Defective verbs, copulae and movement grams}\label{DefectiveVerbsCopulae}
\section{Introduction}
In this chapter, a number of verbs and verbal constructions will be discussed, beginning with a description of Nyakyusa's two copula verbs (\sectref{Copulae}).\is{copula} The description will include the syntactic and semantic\is{syntax} conditions that govern copula use, copula-based existential constructions and the expression of predicative possession. This is followed by a description of the versatile defective verb \textit{tɪ} \lq say' (\sectref{defectiveti}). Lastly, two verbs of motion that have grammaticalized\is{grammaticalization} into auxiliaries\is{auxiliary} of (figurative) movement\is{motion} will be examined (\sectref{MovementGrams}).

\section{The copulae}
\subsection{Copula verbs}
\label{Copulae}\is{copula|(}
Nyakyusa has two copula verbs: defective \textit{lɪ} \lq be' and \textit{ja} \lq be(come)'. The former must be considered defective because it does not take the default final vowel and only occurs in three paradigms: a zero-marked present,\is{tense!present} the affirmative past (formed with the prefix \textit{a}-) and the negative\is{negative} past\is{tense!past} (formed with the prefixes \textit{ka}-\textit{a}-).

\begin{exe}
\ex
\begin{xlist}
\ex \textit{tʊ}-\textit{lɪ bakafu}\phantom{\textit{ka-a-}} `We are healthy.'
\ex \textit{tw}-\textit{a}-\textit{lɪ bakafu}\phantom{\textit{ka-}} `We were healthy.'
\ex \textit{tʊ}-\textit{ka}-\textit{a}-\textit{lɪ} \textit{bakafu} `We were not healthy.'
\end{xlist}
\end{exe}

The two copulae are in near complementary distribution: in all contexts other than the three mentioned above, \textit{ja} is used.\footnote{Interestingly, the distribution of the two copulae in principle corresponds to the distribution of the reflexes of \ili{Proto-Bantu} \textit{*bá} \lq dwell, be, become' and \textit{*dɪ̀} \lq be' in other languages of the Corridor, among them Ndali \citep[104]{BotneR2008}. \textit{ja} most likely stems from a verb of motion \textit{*gɪ̀} \lq go'; note the contextually triggered loss of the consonantal segmental for both the copula and the motion verb (\sectref{jaAspectualizer}).} This includes the infinitive\is{infinitive} and the \isi{negative} counterpart to the present\is{tense!present} (non-past) copula, which is formed with the \isi{negative} prefix \textit{ka}- and the final vowel -\textit{a}. In this context the consonantal segment often drops out, yielding \textit{kaa} with a long final vowel.\is{vowels!length} Note that \isi{stress} remains on \textit{kaa} and is not shifted to the new penultimate syllable.

\begin{exe}
\ex \textit{tʊkaja bakafu} $\sim$ \textit{tʊkaa bakafu} `We are not healthy.'
\end{exe}

Further, \textit{ja} is used in the present\is{tense!present} and past\is{tense!past} for generic\is{aspect!generic} statements:
\begin{exe}
\ex \gll ɪ-n-gambɪlɪ ɪ-si, boo=bʊ-no \textbf{si}-\textbf{kʊ}-\textbf{j}-\textbf{a}\\
\textsc{aug}-10-monkey \textsc{aug}-\textsc{prox}.10 \textsc{ref.14}=14-\textsc{dem} 10-\textsc{prs}-be(come)-\textsc{fv}\\
\glt `These monkeys, this is how they are!' [Thieving monkeys]
\ex \gll ba-a-kitɪk-aga ɪ-m-banda ɪɪ-nunu ɪ-n-golofu pa-katɪ paa-nyumba, ɪ-j-aa kw-ɪm-a=po ʊ-n-talɪko. lɪnga ɪɪ-nyumba nywamu \textbf{j}-\textbf{aa}-\textbf{j}-\textbf{aga} n=ɪ-m-banda i-bɪlɪ pamo i-tatʊ\\
2-\textsc{pst}-stick\_in\_ground-\textsc{ipfv} \textsc{aug}-9-\textsc{post} \textsc{aug}-good(9) \textsc{aug}-9-straight 16-middle 16-house(9), \textsc{aug}-9-\textsc{assoc} 15-stand/stop-\textsc{fv}=16 \textsc{aug}-3-beam if/when \textsc{aug}-house(9) big(9) 9-\textsc{pst}-be(come)-\textsc{ipfv} \textsc{com}=\textsc{aug}-10-post 10-two or 10-three\\
\glt `They would erect a good straight post in the middle of the house, on which lay the ridge pole.  ‎‎If the house was big, it would have two or three posts.' [Nyakyusa houses of long ago]
\end{exe}

Note that the use of \textit{ja} is obligatory for future\is{tense!future} time reference; that is, the unmarked present of \textit{lɪ} cannot normally be used as a futurate:\is{futurate}

\begin{exe}
\ex[*]{\gll kɪ-laabo a-lɪ kʊ-Tʊkʊjʊ\\
7-tomorrow 1-\textsc{cop}  17-T.\\
\glt (intended: \lq Tomorrow he will be at Tukuyu.')}
\end{exe}

There is one exception, however: copula \textit{lɪ} is licensed with reference to the future if a temporal anchor is introduced by a stressed form of the augmentless class 14 referential demonstrative \textit{bo} (\ref{exCOPliFutBo}); see \sectref{PredicativePosession} for the expression of predicative possession through the use of the copula plus the comitative \textit{na}. Likewise, a zero copula (see \sectref{CopulaUse}) is attested in this environment with reference to a future/hypothetical state-of-affairs (\ref{ex0CopFutBo}). See p.\nobreakspace\pageref{exPFVboFuture} in \sectref{PresentPerfectiveIntroduction} for a comparable case with the present perfective.\is{tense!present}\is{aspect!perfective}
\begin{exe}
\ex \label{exCOPliFutBo}
\gll lɪnga fi-kɪnd-ile ɪ-fy-ɪnja a-ma-longo ma-bɪlɪ, ɪɪ-nyumba sy-osa n-ka-aja a-ka \textbf{bo} \textbf{si}-\textbf{lɪ} n=ʊ-bʊ-meme\\
if/when 8-pass-\textsc{pfv} \textsc{aug}-8-year \textsc{aug}-6-ten 6-two \textsc{aug}-house(10) 10-all 18-12-homestead \textsc{aug}-\textsc{prox}.12 as 10-\textsc{cop} \textsc{com}=\textsc{aug}-14-electricity(<SWA)\\
\glt \lq In twenty years, all houses in this village will have electricity.' [ET]
\ex \label{ex0CopFutBo}
\gll lɪnga mu-sob-iisye, \textbf{bo} lw-ɪnʊ\\
if/when \textsc{2pl}-be\_lost-\textsc{caus.pfv} as 11-\textsc{poss.2pl}\\
\glt \lq If you lose it, grief will be yours.' [Chickens and Crow]
\end{exe}

\subsection{Copula use}\label{CopulaUse}
As described in the previous section, the choice between the two copula verbs \textit{lɪ} and \textit{ja} depends mainly on temporal reference and polarity. In the affirmative present\is{tense!present} (non-past), certain environments further license a zero copula or copulative use of the augmentless substitutives (\sectref{Demonstratives}); see \citet{StassenL2005} for a discussion of the term \textit{zero copula}.

With third person (noun class)\is{noun classes} subjects, nominal predication without any overt linking element is the common case (\ref{exZeroCupulaSG}, \ref{exZeroCupulaPL}). The predicate never carries an augment. An augmentless substitutive may be added, which seems to be related to focus (\ref{exSubstitutiveCopulative}). Note that copulative use of the substitutive also features in cleft sentences; see e.g. (\ref{exRelativeClauseParticipants2}; p.\nobreakspace\pageref{exRelativeClauseParticipants2}) and (\ref{exukutimaelijimo}; p.\nobreakspace\pageref{exukutimaelijimo}). 

\begin{exe}
\ex \begin{xlist}
\ex \label{exZeroCupulaSG}\gll ʊ-m-piki n-nywamu\\
\textsc{aug}-3-tree 3-big\\
\glt `The tree is big.' [ET]
\ex \label{exZeroCupulaPL}\gll ɪ-mi-piki mi-nywamu\\
\textsc{aug}-4-tree 4-big\\
\glt `The trees are big.' [ET]
\ex \label{exSubstitutiveCopulative} \gll ɪ-mi-piki ɪ-gɪ gyo mi-nywamu\\
\textsc{aug}-4-tree \textsc{aug}-\textsc{prox.4} \textsc{ref.4} 4-big\\
\glt \lq These trees, they are big.' [ET]
\end{xlist}
\end{exe}
Associatives and possessives are also normally used without an overt copula:
\largerpage[2]
\begin{exe}
\ex \gll gw-a ba-palamaani\\
3-\textsc{assoc} 2-neighbour\\
\glt \lq It (the tree) is the neighbours'.' [ET]
\ex \gll gw-angʊ\\
3-\textsc{poss.1sg}\\
\glt \lq It (the tree) is mine.' [ET]
\end{exe}

With certain types of predicates, however, the use of a copula verb is obligatory even with noun class\is{noun classes} subjects in the affirmative present.\is{tense!present} In some of these, an augmentless substitutive may replace the copula. First, numerals require either the copula or the augmentless substitutive:
\begin{exe}
\ex
\begin{xlist}
\ex \gll a-ba-ana ba-bɪlɪ\\
\textsc{aug}-2-child 2-two\\
\glt \lq two children' not: \lq The children are two.'
\ex \gll a-ba-ana ba-lɪ ba-bɪlɪ\\
\textsc{aug}-2-child 2-\textsc{cop} 2-two\\
\glt \lq The children are two.'
\ex \gll a-ba-ana bo ba-bɪlɪ\\
\textsc{aug}-2-child \textsc{ref.2} 2-two\\
\glt \lq The children are two.'
\end{xlist}
\end{exe}

Note that this does not hold for the quantifiers \textit{nandɪ} \lq little, few' and \textit{ingi} \lq much, many', which are treated as nominals:
\begin{exe}
\ex \gll ʊ-lw-ɪsi ʊ-lʊ lʊ-sisya. a-m-ɪɪsi ma-tiitʊ kangɪ ɪ-n-gwina \textbf{ny}-\textbf{ingi}\\
\textsc{aug}-11-river \textsc{aug}-\textsc{prox.1} 11-frightening \textsc{aug}-6-water 6-black again \textsc{aug}-10-crocodile 10-many\\
\glt \lq This river is frightening. The water is dark and the crocodiles are many.' [ET]
\ex \gll ʊ-n-tondolo mw-ingi, leelo a-ba-tondol-i \textbf{ba}-\textbf{nandɪ}\\
\textsc{aug}-3-harvest 3-many now/but \textsc{aug}-2-harvest-\textsc{agnr} 2-little\\
\glt \lq The harvest truly is great, but the labourers are few.' (Luke 10: 2)
\end{exe}

When adverbials (\ref{exCOPADV1}, \ref{exCOPADV2}) or \isi{ideophones} (\ref{exCOPIDPH1}, \ref{exCOPIDPH2}) are used predicatively, use of a copula verb is compulsory.

\begin{exe}
\ex \label{exCOPADV1}

\begin{xlist}
\ex[]{\gll ʊ-mw-ana a-lɪ nnoono\\
\textsc{aug}-1-child 1-\textsc{cop} so\_much\\
\glt `The child is too much.'}
\ex[*]{ʊmwana jo nnoono}
\ex[*]{ʊmwana nnoono}
\end{xlist}
\pagebreak

\ex \label{exCOPADV2} \begin{xlist}
\ex[]{\gll ɪ-m-bwa jɪ-lɪ kanunu\\
\textsc{aug}-9-dog 9-\textsc{cop} well\\
\glt `The dog is fine.'}

\ex[*]{ɪmbwa jo kanunu\footnotemark}
\ex[*]{ɪmbwa kanunu}
\end{xlist}

\ex \label{exCOPIDPH1} \begin{xlist}
\ex[]{\gll n-nyumba mu-lɪ kée\\
18-house(9) 18-\textsc{cop} vast\\
\glt `The house is empty.'}

\ex[*]{nnyumba mo kée}
\ex[*]{nnyumba kée}
\end{xlist}


\ex \label{exCOPIDPH2} \begin{xlist}
\ex[]{\gll ɪ-my-enda gɪ-lɪ swée\\
\textsc{aug}-4-cloth 4-\textsc{cop} intense\_white\\
\glt `The clothes are white.'}
\ex[*]{ɪmyenda gyo swée}
\ex[*]{ɪmyenda swée}
\end{xlist}
\end{exe}
\protect\footnotetext{This sentence would be acceptable with the meaning \lq This dog's name is Kanunu.'}

Locative\is{locative} predicates also require a copula verb (\ref{exCOPLocative}). This includes locative question words (\ref{exCopulaKuugu}). (\ref{exCOPLocativeSemantics}) illustrates that the \isi{locative} semantics are responsible for this rather than belonging to one of the locative noun classes.\is{noun classes}

\begin{exe}
\ex \label{exCOPLocative}
\begin{xlist}
\ex[]{\gll ʊ-mw-ana a-lɪ mu-m-piki\\
\textsc{aug}-1-child 1-\textsc{cop} 18-3-tree\\
\glt `The child is in a/the tree.'}
\ex[*]{ʊmwana jo mumpiki}
\ex[*]{ʊmwana mumpiki}
\end{xlist}
\ex \label{exCopulaKuugu}
\begin{xlist}
\ex[]{\gll ʊ-mw-ana a-lɪ kʊʊgʊ \textup{/} pooki \textup{/} mooki?\\
\textsc{aug}-1-child 1-\textsc{cop} where(17) {} where(16) {} where(18)\\
\glt `Where / Wherein is the child?}
\ex[*]{ʊmwana jo kʊʊgʊ \textup{/} pooki \textup{/} mooki?}
\ex[*]{ʊmwana kʊʊgʊ \hphantom{jo }\textup{/} pooki \textup{/} mooki?}
\end{xlist}

\ex\label{exCOPLocativeSemantics}
\begin{xlist}
\ex[]{\gll ʊ-mw-ana a-lɪ kɪfuki n=ʊ-m-piki\\
\textsc{aug}-1-child 1-\textsc{cop} near \textsc{com}=\textsc{aug}-3-tree\\
\glt `The child is near the tree.'}
\ex[*]{ʊmwana jo kɪfuki nʊmpiki}
\ex[*]{ʊmwana kɪfuki nʊmpiki}
\end{xlist}
\end{exe}

With first and second person subjects, the use of either the copula or a substitutive is obligatory for nominal predicates in the affirmative present.\is{tense!present} (\ref{exCopula2sg}--\ref{exCopulaSGNegativeExample}) illustrate this for the second person singular. With other types of predicates, the same regularities as for noun class\is{noun classes} subjects hold.

\begin{exe}
\ex[]{\label{exCopula2sg}\gll (ʊ-gwe) ʊ-lɪ n-nandɪ\\
\hphantom{(}\textsc{aug}-\textsc{2sg} \textsc{2sg}-\textsc{cop} 1-small\\
\glt \lq You are small.'}
\ex[]{\gll (ʊ-gwe) gwe n-nandɪ\\
\hphantom{(}\textsc{aug}-\textsc{2sg} \textsc{2sg} 1-small\\
\glt \lq You are small.'}
\ex[*]{\label{exCopulaSGNegativeExample}\gll ʊ-gwe n-nandɪ\\
\textsc{aug}-\textsc{2sg} 1-small\\}
\end{exe}

Lastly, the copula also forms a compulsory part of existential constructions and expressions of predicative possession, which are the topics of the following sections.

\subsection{Existential construction}\label{Existentials}
The presence or existence of an entity is expressed by a copula plus a \isi{locative} enclitic.\is{enclitic} With the copula \textit{lɪ}, the vowel segment is raised to /i/. Noun class 16 \textit{po} expresses proximity to the deictic centre or more definite locations, class 17 \textit{ko} distance from the deictic centre or general existence and class 18 \textit{mo} inside locations.
\begin{exe}
\ex \label{exExistentialaffirmiert1}
\gll \textbf{ga}-\textbf{a}-\textbf{li}=\textbf{po} a-ma-syabala, \textbf{sy}-\textbf{a}-\textbf{li}=\textbf{po} ɪ-n-jʊgʊ. \textbf{ba}-\textbf{a}-\textbf{li}=\textbf{po} baa-mwembe, \textbf{ga}-\textbf{a}-\textbf{li}=\textbf{po} a-m-ungu. fy-osa \textbf{fy}-\textbf{a}-\textbf{li}=\textbf{po} pa-ka-aja pa-n-gambɪlɪ\\
\textsc{6}-\textsc{pst}-\textsc{cop}=16 \textsc{aug}-6-groundnut 10-\textsc{pst}-\textsc{cop}=16 \textsc{aug}-10-jugo\_bean 2-\textsc{pst}-\textsc{cop}=16 2-mango 6-\textsc{pst}-\textsc{cop}=16 \textsc{aug}-6-pumpkin 8-all 8-\textsc{pst}-\textsc{cop}=16 16-12-home 16-9-monkey\\
\glt \lq There were groundnuts, there were jugo beans. There were mangoes, there were pumpkins. There was all sorts of food at Monkey's' [Monkey and Tortoise] 
\ex \label{exExistentialaffirmiert2}
\gll jɪ-kʊ-tɪ bo m-fw-ile ʊ-ne lɪnga ga-kɪnd-ile a-ma-sikʊ ma-nandɪ.\\
9-\textsc{prs}-\textsc{say} as \textsc{1sg}-die-\textsc{pfv} \textsc{aug}-\textsc{1sg} if/when 6-pass-\textsc{pfv} \textsc{aug}-6-day 6-little\\
\glt \lq It says I'm dead when few days have passed.'
\sn \gll po lɪlɪno n-gʊ-bʊʊk-aga kʊkʊtɪ na=a-ma-jolo kʊ-no ɪ-m-bʊlʊkʊtʊ jɪ-lambaleele kʊ-kʊ-jɪ-bʊʊl-a ʊkʊtɪ \lq\lq ʊ-ne n-gaalɪ \textbf{n}-\textbf{di}=\textbf{ko}, n-dɪ n=ʊ-bʊ-ʊmi''\\
then now/today \textsc{1sg}-\textsc{mod.fut}-go-\textsc{mod.fut} every \textsc{com}=\textsc{aug}-6-evening 17-\textsc{prox} \textsc{aug}-9-ear 9-lie\_down-\textsc{pfv} 17-15-9-tell-\textsc{fv} \textsc{comp} \phantom{\lq\lq}\textsc{aug}-\textsc{1sg} \textsc{1sg}-\textsc{pers} \textsc{1sg}-\textsc{cop}=17 \textsc{1sg}-\textsc{cop} \textsc{com}=\textsc{aug}-14-live\\
\glt `So now I shall go every evening when Ear has laid down, to tell it ``I'm still around, I'm alive.''{}' [Mosquito and Ear]
\end{exe}

When the copula is negated,\is{negative} non-existence or absence is expressed:
\begin{exe}
\ex \gll i-kʊ-suluk-a paa-si mu-m-piki n=ʊ-kʊ-keet-a ʊkʊtɪ kɪ-kapʊ kɪ-mo \textbf{kɪ}-\textbf{ka}-\textbf{j}-\textbf{a}=\textbf{po}\\
1-\textsc{prs}-descend-\textsc{fv} 16-below 18-3-tree \textsc{com}=\textsc{aug}-15-watch-\textsc{fv} \textsc{cmpl} 7-basket 7-one 7-\textsc{neg}-be(come)-\textsc{fv}=16\\
\glt `He climbs down the tree and sees that one basket is missing.' [Elisha Pear Story]
\ex \label{exExistentialNegated2}\gll ka-a-li=ko a-ka-aja ka-mo a-ka a-m-ɪɪsi \textbf{ga}-\textbf{ka}-\textbf{a}-\textbf{li}=\textbf{mo}\\ 
12-\textsc{pst}-\textsc{cop}=17 \textsc{aug}-12-village 12-one \textsc{aug}-\textsc{prox.12} \textsc{aug}-6-water 6-\textsc{neg}-\textsc{pst}-\textsc{cop}=18\\
\glt `There was a village in which there was no water.' [Water and toads]
\end{exe}

When \isi{locative} marking on the copula co-occurs with an overt locative noun phrase, both often reference the same locative noun class (\ref{exExistentialSameNounClasses}),\is{noun classes} but mixing of two locative classes is also found (\ref{exExistentialMixedNounClasses}).

\begin{exe}
\ex \label{exExistentialSameNounClasses} \gll a-ma-keeke \textbf{ga}-\textbf{a}-\textbf{li}=\textbf{mo} m-ingi \textbf{mw}-\textbf{ene} \textbf{mw}-\textbf{i}-\textbf{tengele} ɪ-ly-a n-ky-amba Rungwe\\
\textsc{aug}-6-type\_of\_grass 6-\textsc{pst}-\textsc{cop}=18 6-many 18-only 18-5-bush \textsc{aug}-5-\textsc{assoc} 18-7-mountain R.\\
\glt `There was a lot of a certain type of grass only in the bush on mount Rungwe.' [Nyakyusa houses of long ago]
\ex \label{exExistentialMixedNounClasses}\gll \textbf{n}-\textbf{k}-\textbf{iisʊ} kɪ-mo, \textbf{a}-\textbf{a}-\textbf{li}=\textbf{ko} ʊ-malafyale jʊ-mo. ɪ-n-gamu j-aake a-a-lɪ jo Kapyungu\\
18-7-land 7-one 1-\textsc{pst}-\textsc{cop}=17 \textsc{aug}-chief(1) 1-one \textsc{aug}-9-name 9-\textsc{poss.sg} 1-\textsc{pst}-\textsc{cop} \textsc{ref.1} K.\\
\glt `In some land there was a chief. His name was Kapyungu.' [Chief Kapyungu]
\end{exe}

Note that in examples (\ref{exExistentialaffirmiert1}--\ref{exExistentialMixedNounClasses}) the grammatical subject follows the existential copula. This is a common presentational construction. In fact, (\ref{exExistentialNegated2}, \ref{exExistentialMixedNounClasses}) are typical of the orientation sections of Nyakyusa narratives.\is{narrative}

\subsection{Expression of predicative possession}\label{PredicativePosession}
Ownership is expressed by using the copula together with an enclitic form of the comitative \textit{na} on the possessee.

\begin{exe}
\ex \gll ʊ-malafyale ʊ-jʊ \textbf{a}-\textbf{a}-\textbf{lɪ} \textbf{n}=ɪ-fi-panga fi-tatʊ\\
\textsc{aug}-chief(1) \textsc{aug}-\textsc{prox.1} 1-\textsc{pst}-\textsc{cop} \textsc{com}=\textsc{aug}-8-village 8-three\\
\glt `This chief had three villages.' [Chief Kapyungu]
\ex \label{exPredicativePosession2}\gll lɪnga \textbf{ʊ}-\textbf{ka}-\textbf{j}-\textbf{a} \textbf{n}=ʊ-bʊ-jo bw-a kʊ-gon-a=mo kʊ-gon-a muu-nyumba ɪ-sy-a kʊ-pang-a\\
if/when \textsc{2sg}-\textsc{neg}-be(come)-\textsc{fv} \textsc{com}=\textsc{aug}-14-place 14-\textsc{assoc} 15-sleep-\textsc{fv}=18 \textsc{2sg.prs}-sleep-\textsc{fv} 18-house(9) \textsc{aug}-10-\textsc{assoc} 15-rent-\textsc{fv}\\
\glt `If you do not have a place to sleep in, you sleep in a rented house.' [How to build modern houses]
\end{exe}

Typically, the noun expressing the possessee carries the augment, though use without the augment was also encountered:
\begin{exe}
\ex \gll n-ga-a na=m-bombo, n-ga-a na=heela\\
\textsc{1sg}-\textsc{neg}.be(come)-\textsc{fv} \textsc{com}=9-work \textsc{1sg}-\textsc{neg}.be(come)-\textsc{fv} \textsc{com}=money(10)\\
\glt `I don't have a job, I don't have money.' [overheard]
\end{exe}

The possessee can be referred to by a referential demonstrative without the augment. This is the case with anaphoric reference (\ref{exPossessionReferential}). The referential demonstrative can also be used cataphorically together with the overt noun phase it indexes (\ref{exPossessionBoth}).
\begin{exe}
\ex \label{exPossessionReferential}\gll kangɪ mpaka ʊ-si-keet-e taasi \textbf{ɪ}-\textbf{n}-\textbf{dalama} ɪ-si ʊ-lɪ \textbf{na}=\textbf{syo} muu-ny-ambɪ, pamopeene n=ʊ-tʊ-ndʊ ʊ-tʊ tʊ-bagiile ʊ-kʊ-kʊ-tʊʊl-a kʊ-m-bombo ɪ-jo\\
again no\_matter\_what \textsc{2sg}-10-watch-\textsc{subj} yet \textsc{aug}-10-money \textsc{aug}-\textsc{prox.10} \textsc{2sg}-\textsc{cop} \textsc{com}=\textsc{ref.10} 18-pocket(9) together \textsc{com}=\textsc{aug}-13-thing \textsc{aug}-\textsc{prox.13} 13-be\_able.\textsc{pfv} \textsc{aug}-15-\textsc{2sg}-help-\textsc{fv} 17-9-work \textsc{aug}-\textsc{ref.9}\\
\glt `Again, you should first look at the money which you have in your pocket, together with other things which can help you with this work.' [How to build modern houses]
\ex\label{exPossessionBoth} \gll lɪlɪno tʊ-ka-a \textbf{na}=\textbf{fyo} na=fi-mo \textbf{ɪ}-\textbf{fi}-\textbf{ndʊ} ɪ-fy-a kʊ-ly-a n-nyumba\\
now/today \textsc{1pl}-\textsc{neg}.be(come)-\textsc{fv} \textsc{com}=\textsc{ref.8} \textsc{com}=8-one \textsc{aug}-8-food \textsc{aug}-8-\textsc{assoc} 15-eat-\textsc{fv} 18-house(9)\\
\glt `Today we don't have anything to eat at home.' [Monkey and Tortoise]
\end{exe} 
\is{copula|)}
\section{\textit{tɪ} `say; think; do like'}\label{defectiveti} 
The verb \textit{tɪ} \lq say' must be considered defective for a number of reasons. First, its stem does not carry the final vowel -\textit{a}. Consequently, it does not change its shape in the subjunctive.\is{mood!subjunctive} In other respects, its vocalic segment, however, behaves much like the final vowel of regular verbs: in the imperfective\is{aspect!imperfective} \textit{tɪ} takes the shape \textit{tɪgɪ}, resembling the -VCV shape of the regular imperfective suffix (see \sectref{AlternationsIPFVaga}). Second, the vocalic segment is dropped when perfective\is{aspect!perfective} -\textit{ile} is suffixed, yielding \textit{tile} (not *\textit{tiile}).\is{vowels!length} Last, \textit{tɪ} does not accept any derivational suffixes.

For reasons of space and convenience, \textit{tɪ} is glossed as `say' throughout this study. However, as the following discussion will show, this versatile verb shows uses and functions that go far beyond that of a simple verb of speech. \citet{GueldemannT2000} convincingly argues that the use of \textit{tɪ} as a verb of speech across Bantu has arisen out of a more abstract cataphoric function.

Its use as a verb of speech is illustrated in (\ref{extiVerbOfSpeech}). To render speech or sound with verbs other than \textit{tɪ} itself, a form of \textit{tɪ}, either the infinitive\is{infinitive} (\ref{extiNARRINF}) or an inflected verb in a chaining construction (\ref{extiNARRNARR}), is also required.
 
\begin{exe}
\ex \label{extiVerbOfSpeech}
\gll po \textbf{ly}-\textbf{a}-\textbf{t}-\textbf{ile} \textup{\lq\lq}n-gʊ-lond-a ɪɪ-sindaano j-angʊ\textup{''}\\
then 5-\textsc{pst}-say-\textsc{pfv} \phantom{\lq\lq}\textsc{1sg}-\textsc{prs}-want-\textsc{fv} \textsc{aug}-needle(<SWA)(9) 9-\textsc{poss.1sg}\\
\glt \lq It [Crow] said \lq\lq I want my needle.''{}' [Chickens and Crow]
\ex \label{extiNARRINF} \gll a-lɪnkʊ-ba-bʊʊl-a a-ba-ndʊ \textbf{ʊ}-\textbf{kʊ}-\textbf{tɪ}\\
1-\textsc{narr}-2-tell-\textsc{fv} \textsc{aug}-2-person \textsc{aug}-15-say\\
\glt `He told the people:' [Chief Kapyungu]
\ex\label{extiNARRNARR} \gll kalʊlʊ a-lɪnkw-amul-a kʊ-m-manyaani gw-ake \textbf{a}-\textbf{lɪnkʊ}-\textbf{tɪ}\\
hare(1) 1-\textsc{narr}-answer-\textsc{fv} 17-1-friend 1-\textsc{poss.sg} 1-\textsc{narr}-say\\
\glt `Hare answered to his friend:' [Hare and Chameleon]
\end{exe}

The only attested examples in the corpus of illocutionary verbs of speech without a form of \textit{tɪ} are sections of narratives\is{narrative} that move to drama; see p.\nobreakspace\pageref{Drama} in \sectref{Drama}. Apart from speech in the strict sense, \textit{tɪ} is also used for rendering inner speech or thought:

\begin{exe}
\ex \gll ngɪmba mu-n-dumbula \textbf{i}-\textbf{kʊ}-\textbf{tɪ} \textup{\lq\lq}n-gʊ-tosiisye\textup{''}\\
behold 18-9-heart 1-\textsc{prs}-say \phantom{\lq\lq}\textsc{1sg}-\textsc{2sg}-pay\_back.\textsc{pfv}\\
\glt `But in his heart he [Skunk] is thinking ``I've paid you back.''{}' [Hare and Skunk]
\ex \gll ɪ-m-bwa jɪ-lɪnkʊ-j-a n-galɪ fiijo, \textbf{j}-\textbf{aa}-\textbf{t}-\textbf{ɪgɪ} pamo ɪ-li-paatama li-kʊ-j-a pa-kʊ-pok-a ɪɪ-nyama j-aake\\
\textsc{aug}-9-dog 9-\textsc{narr}-be(come)-\textsc{fv} 9-strict \textsc{intens} 9-\textsc{pst}-say-\textsc{ipfv} perhaps \textsc{aug}-5-cheetah 5-\textsc{prs}-be(come)-\textsc{fv} 16-15-plunder-\textsc{fv} \textsc{aug}-meat(9) 9-\textsc{poss.sg}\\
\glt `The dog became very angry, it was thinking that maybe the cheetah would seize his meat.' [Dogs laughed at each other]
\end{exe}

The verb \textit{tɪ} also serves to introduce ideophones.\is{ideophones} All cases in the data feature onomatopoeia. It is unclear if this is a restriction on the use of \textit{tɪ} or an artefact of the data at hand.

\begin{exe}
\ex\label{extiPFVPFVideoph1}
\gll ʊ-gw-a kɪ-bɪlɪ a-a-lʊ-kol-ile ʊ-lw-igi m-ma-ka ma-tupu. looli j-oope kw-a-lɪl-ile \textbf{kw}-\textbf{a}-\textbf{t}-\textbf{ile} \textup{\lq\lq}káa\textup{''}\\
\textsc{aug}-2-\textsc{assoc} 7-two 1-\textsc{pst}-11-grasp/hold-\textsc{pfv} \textsc{aug}-11-door 18-6-strength 6-sudden but 1-also 17-\textsc{pst}-cry-\textsc{pfv} 17-\textsc{pst}-say-\textsc{pfv} \phantom{\lq\lq}of\_sickle\_swinging\\
\glt `The second one grabbed the door with all his strength. But also with him, there was the sound ``káa!'' [of a sickle swinging]' [Wage of the thieves]
\end{exe}

\label{SubjunctiveTiBule}The subjunctive (\sectref{Subjunctive})\is{mood!subjunctive} of \textit{tɪ} is formed by prefixing the subject prefix,\is{subject marker} without any change in the final vowel (\ref{extiSubjunctive}). When the interrogative \textit{bʊle} `how' follows \textit{tɪ}, they optionally merge into one word, with the vocalic segment of the verb assimilating (\ref{extutubule}). The imperfective\is{aspect!imperfective} suffix in this case is attached to the right of the compound \isi{stem} and accordingly takes the shape -\textit{ege} (\ref{extutubulege}).

\begin{exe}
\ex \label{extiSubjunctive}\gll gw-itɪk-e ʊ-tɪ \textup{\lq\lq}ee, n-di=po\textup{''}\\
\textsc{2sg}-agree-\textsc{subj} \textsc{2sg}-say.\textsc{subj} \phantom{\lq\lq}yes \textsc{1sg}-\textsc{cop}=16\\
\glt \lq You shall answer ``Yes, I'm here.''{}' [Hare and Tugutu] %schoenes BSP, nicht so schlimm, dass teils fuer NARR verwendet
\ex \label{extutubule}\textit{tʊtɪ bʊle?} $\sim$ \textit{tʊtʊbʊle?}
\\ \lq What should we say/do?'
\ex \label{extutubulege}\textit{tʊtɪgɪ bʊle?} $\sim$ \textit{tʊtʊbʊlege?}
\\ \lq What should we be saying/doing?'
\end{exe}

As can be gathered from (\ref{extutubule}, \ref{extutubulege}), apart from introducing (inner) speech and sound, \textit{tɪ} can be understood to have a broader meaning of acting in a certain manner. Thus its subjunctive\is{mood!subjunctive} is also used as a prompt to imitate a certain action (\ref{exTiImitate}) (also cf. \citealt[97]{FelbergK1996}). Also note the related uses in (\ref{exTiWhat}, \ref{exukutimaelijimo}).
\begin{exe}
\ex \label{exTiImitate}
\gll \textbf{ʊ}-\textbf{tɪ} bʊ-no\\
\textsc{2sg}-say.\textsc{subj} 14-\textsc{dem}\\
\glt `Do like this!' [ET]
\ex \label{exTiWhat} \gll po jʊ-la ʊ-gw-a pa-lʊ-ʊlʊ \textbf{a}-\textbf{lɪnkʊ}-\textbf{tɪ} fi-ki, a-lɪnkʊ-kaan-a\\
then 1-\textsc{dist} \textsc{aug}-1-\textsc{assoc} 16-11-north 1-\textsc{narr}-say 8-what 1-\textsc{narr}-refuse-\textsc{fv}\\
\glt \lq The one from the north did what? He refused.' [Lake Kyungululu]
\ex \label{exukutimaelijimo}
\gll mwa=n-dugutu a-ka-a-bop-ile=po. a-a-ba-paal-ile a-ba-nine. bo a-ba a-a-ba-bɪɪk-ile \textbf{ʊ}-\textbf{kʊ}-\textbf{tɪ} maelɪ jɪ-mo, maelɪ jɪ-mo, maelɪ jɪ-mo\\
matronym=9-type\_of\_bird 1-\textsc{neg}-\textsc{pst}-run-\textsc{pfv}=\textsc{part} 1-\textsc{pst}-2-invite-\textsc{pfv} \textsc{aug}-2-companion \textsc{ref.2} \textsc{aug}-\textsc{prox.2} 1-\textsc{pst}-2-put-\textsc{pfv} \textsc{aug}-15-say mile(9)(<EN) 9-one mile(9) 9-one mile(9) 9-one\\
\glt `Mr. Tugutu did not run at all. He had gathered companions. Those are the ones he placed, like one mile, one mile, one mile.' [Hare and Tugutu]
\end{exe}

The infinitive\is{infinitive} \textit{ʊkʊtɪ} is further grammaticalized as a complementizer (\ref{exUkutiComplementizer}). It also serves to introduce clauses of purpose\is{subordinate clauses!purpose clause} and result;\is{subordinate clauses!result clause} see \sectref{SubjunctiveSubordinate}.
\begin{exe}
\ex\label{exUkutiComplementizer}
\gll ʊ-meenye ʊkʊtɪ Asia a-ka-kʊ-gan-a?\\
\textsc{2sg}-know.\textsc{pfv} \textsc{comp} A. 1-\textsc{neg}-\textsc{2sg}-love-\textsc{fv}\\
\glt `Do you know that Asia doesn't love you?' [Juma, Asia and Sambuka]
\end{exe}

Similarly, the \isi{infinitive} of \textit{tɪ} as the dependent element of the associative construction serves to introduce a clause as a nominal complement.\footnote{Following the associative particle, the augment on nouns is banned.} This is frequently used in the collocation \textit{kʊʊnongwa} (\textit{ɪ})\textit{jaa kʊtɪ} `for the reason that, because'.

\begin{exe}
\ex \label{exkunongwajaa} \gll kʊʊ-nongwa j-\textbf{aa} \textbf{kʊ}-\textbf{tɪ} Juma a-a-meenye ʊkʊtɪ Sambʊka m-manyaani gw-a Asia, po a-lɪnkw-itɪk-a ʊ-kʊ-bʊʊk-a kʊ-kʊ-many-a=ko kʊ-my-ake\\
17-issue(9) 9-\textsc{assoc} 15-say J. 1-\textsc{pst}-know.\textsc{pfv} \textsc{comp} S. 1-friend 1-\textsc{assoc} A. then 1-\textsc{narr}-agree-\textsc{fv} \textsc{aug}-15-go-\textsc{fv} 17-15-know-\textsc{fv}=17 17-4-\textsc{poss.sg}\\
\glt `Because Juma knew that Sambuka was a friend of Asia, he agreed to go and get to know her [Sambuka's] home.' [Juma, Asia and Sambuka]

\ex \gll Mfyage a-a-bʊʊk-ile kʊ-n̩-ganga ʊ-gw-a kɪ-tiitʊ, kʊ-kʊ-lond-a ʊ-n-kota ʊ-gw-\textbf{a} \textbf{kʊ}-\textbf{tɪ} ba-n̩-gan-ege fiijo a-ba-nyambala\\
M. 1-\textsc{pst}-go-\textsc{pfv} 17-1-healer \textsc{aug}-1-\textsc{assoc} 7-black 17-15-search-\textsc{fv} \textsc{aug}-3-medicine \textsc{aug}-3-\textsc{assoc} 15-say 2-1-love-\textsc{ipfv.subj} \textsc{intens} \textsc{aug}-2-man\\
\glt `Mfyage went to a witch doctor to find a medicine that would make men love her very much.' [Mfyage turns into a lion] 
\end{exe}

The verb \textit{tɪ} also serves as an auxiliary,\is{auxiliary} taking a subjunctive\is{mood!subjunctive} complement in a number of conventionalized constructions; see \sectref{ComplexConructionsWithSubjunctive}. It also forms part of the invariable \isi{evidential} of report \textit{baatɪ}.\footnote{This can doubtless be analyzed as \textit{ba-a-tɪ}. Given that this form is homophonous with the subsecutive with a noun class 2 subject (i.e. 3rd person plural used as impersonal), this might be an indication that the subsecutive configuration has developed diachronically out of a former perfective\is{aspect!perfective} or anterior,\is{aspect!anterior} thus \lq They (have) said'. Cf. also \ili{Ndali} \textit{báti}, which apparently fulfils the same function \citep[107]{BotneR2008}.} This particle serves to indicate that the source of information is hearsay. It can be used to distance oneself from what is reported, ascribing responsibility to the original source (\ref{exBaatiHare}) and is also commonly used to echo what has just been said (\ref{exBaatiDuka}, \ref{exBaatiMonkeyAndTortoise}).

\begin{exe}
\ex\label{exBaatiHare}
\gll Saliki a-lɪnkʊ-tɪ \textup{\lq\lq}ʊ-m̩-buut-ile kalʊlʊ ʊ-jʊ n-d-ile ʊ-buut-ege?\textup{''} ʊ-n-kasi gw-a Saliki a-lɪnkʊ-tɪ \textup{\lq\lq}keet-a ʊ-t-ile \textbf{baatɪ} n-heesya gw-ɪtu!? n-um̩-buut-iile ɪ-n-gʊkʊ. a-li=mo n-nyumba, a-lɪ pa-kʊ-ly-a=mo\textup{''}\\
S. 1-\textsc{narr}-say \phantom{\lq\lq}\textsc{2sg}-1-slaughter-\textsc{pfv} hare(1) \textsc{aug}-\textsc{prox.1} \textsc{1sg}-say-\textsc{pfv} \textsc{2sg}-slaughter-\textsc{ipfv.subj} \textsc{aug}-1-wife 1-\textsc{assoc} S. 1-\textsc{narr}-say \phantom{\lq\lq}look-\textsc{fv} \textsc{2sg}-say-\textsc{pfv} hearsay 1-guest 1-\textsc{poss.1pl} \textsc{1sg}-1-slaughter-\textsc{appl.pfv} \textsc{aug}-9-chicken 1-\textsc{cop}=18 18-house(9) 1-\textsc{cop} 16-15-eat-\textsc{fv}=some\\
\glt `Saliki said ``Have you slaughtered Hare, whom I told you to slaugher?'' Saliki's wife said ``Look, you said he is our guest!? I slaughtered a chicken for him. He's in the house, he's eating.''{}' [Saliki and Hare]
\ex \label{exBaatiDuka}
Context: The researcher has asked for a soda at a small shop. The friend of the shop owner is surprised by his language skills and repeats his words:\\
\gll \textbf{baatɪ} \textup{\lq\lq}n-gʊ-sʊʊm-a ɪɪ-kook\textup{''}\\
hearsay \phantom{\lq\lq}\textsc{1sg}-\textsc{prs}-beg-\textsc{fv} \textsc{aug}-C.(9)\\
\glt \lq [Quoting:] \lq\lq I'd like a Coke.''{}' [overheard]
\ex\label{exBaatiMonkeyAndTortoise}
Context: Tortoise's child has just told Monkey that Tortoise senior is sad.\\
\gll \textup{\lq\lq}fi-ki?\textup{''} \textup{\lq\lq}a-fw-ile ɗaaɗa gw-ake\textup{''} \textup{\lq\lq}n-koolel-e!\textup{''} a-lɪnkʊ-sook-a kajamba, i-kʊ-lɪl-a \textup{\lq\lq}hɪhɪhɪhɪɪ, hɪhɪhɪhɪɪ, a-fw-ile, a-fw-ile\textup{''}\\
\phantom{\lq\lq}8-what \phantom{\lq\lq}1-die-\textsc{pfv} sister(<SWA) 1-\textsc{poss.sg} \phantom{\lq\lq}1-call-\textsc{imp} 1-\textsc{narr}-leave-\textsc{fv} tortoise(1) 1-\textsc{prs}-cry-\textsc{fv} \phantom{\lq\lq}of\_crying of\_crying 1-die-\textsc{pfv} 1-die-\textsc{pfv}\\
\glt \lq [Monkey:] \lq\lq Why?'' [Tortoise's child:] \lq\lq His sister died'' [Monkey:] \lq\lq Call him!'' Tortoise came out, he is crying \lq\lq ''hihihihiii, hihihihii, she died, she died''.'
\sn \gll po mwa=n-gambɪlɪ \textup{\lq\lq}he? \textbf{baatɪ} a-fw-ile ɗaada, ee? po ndaga\textup{''}\\
then matronym=9-monkey \phantom{\lq\lq}\textsc{interj} hearsay 1-die-\textsc{pfv} sister yes then thanks\\
\glt \lq  Monkey: \lq\lq So your sister died, yes? My sympathy.''{}' [Monkey and Tortoise]
\end{exe}

Note that even in the wider discourse context of the preceding examples there is no referent of noun class 2 which the \textit{ba}- portion could cross-reference. Also note that \textit{tɪ} cannot normally be followed by another instance of itself:
\begin{exe}
\ex[*]{\gll keeta ʊ-t-ile ʊ-kʊ-tɪ n-heesya gw-ɪtʊ\\
look \textsc{2sg}-say-\textsc{pfv} \textsc{aug}-15-say 1-guest 1-\textsc{poss.1pl}\\}  
\label{exNoTwoTi}
\end{exe}
 
A homophonous form \textit{baatɪ} is also used as a call for attention (\ref{exBaatiAttention}).\footnote{Note that this parallels \ili{Swahili} \textit{ati}$\sim$\textit{eti}, which is similarly used as an evidential of report and as an interjection (\citealt[17]{MadanA1903}; \citealt[19]{MawJ2013}). It is unclear if this use of Nyakyusa \textit{baatɪ} is a result of a parallel development or if its usage has been influenced by Swahili.\il{Swahili}.}

\begin{exe}
\ex \label{exBaatiAttention}\textit{baatɪ} \lq Listen!'
\end{exe}

Another use of \textit{tɪ} is that of naming or calling people or entities. Note the \isi{object marker} in (\ref{extiOM}), which is otherwise not licensed with this verb.
\begin{exe}
\ex \gll ijolo n̩-dw-iho lw-a ba-Nyakyʊsa ba-a-lɪ na=a-ka-jɪɪlo k-a n-kiikʊlʊ ʊ-kʊ-n-tiil-a ʊ-gwise gw-a n̩-dʊme, ʊ-jʊ \textbf{tʊ}-\textbf{kʊ}-\textbf{tɪ} n-kamwana\\
old\_times 18-11-custom 11-\textsc{assoc} 2-Ny. 2-\textsc{pst}-\textsc{cop} \textsc{com}=\textsc{aug}-12-custom 12-\textsc{assoc} 1-woman \textsc{aug}-15-1-fear-\textsc{fv} \textsc{aug}-his\_father(1) 1-\textsc{assoc} 1-husband \textsc{aug}-\textsc{prox.1} \textsc{1pl}-\textsc{prs}-say 1-in\_law\\
\glt `Long ago in the tradition of the Nyakyusa people they had a custom of the woman fearing the father of her husband, whom we call Nkamwana.' [Should she save a life\ldots]

\ex \gll ky-a-li=po n=ɪ-kɪ-piki, \textbf{tʊ}-\textbf{kʊ}-\textbf{tɪ} ɪ-m-bale\\
7-\textsc{pst}-\textsc{cop}=16 \textsc{com}=\textsc{aug}-7-stump \textsc{1pl}-\textsc{prs}-say \textsc{aug}-7-type\_of\_wood\\
\glt \lq There was also a wood. We call it Mbale.' [Clothing long ago]

\ex \label{extiOM} \gll \textbf{bi}-\textbf{kʊ}-\textbf{n}-\textbf{tɪ} (jo) Mama Tuma\\ 2-\textsc{prs}-1-say \textsc{ref.1} M. T.\\\glt `They call her Mama Tuma' [ET]
\end{exe}

Lastly, \textit{tɪ} features in the conjunction \textit{kookʊtɪ} `that is to say, that means' (\ref{exkookuti}), in the universal quantifier \textit{kʊkʊtɪ} `every' (\ref{exkukuti}) and in \textit{ngatɪ} `as, like' (\ref{exngati}). 

\begin{exe}
\ex \label{exkookuti}
\gll bo ba-fik-ile pa-la ba-lɪnkʊ-sy-ag-a ɪ-n-gambɪlɪ si-tengeene m-mi-gʊnda gy-abo, si-kʊ-ly-a ɪ-fi-lombe kangɪ si-kw-ɪmb-a si-kʊ-tɪ \textup{\lq\lq}{ho! ho! ho!}\textup{''} \textbf{kookʊtɪ} \textup{\lq\lq}ee fi-nunu! ee fi-nunu! ee fi-nunu!\textup{''}\\
as 2-arrive-\textsc{pfv} 16-{dist} 2-\textsc{narr}-10-find-\textsc{fv} \textsc{aug}-10-monkey 10-live\_in\_peace.\textsc{pfv} 18-4-field 4-\textsc{poss.pl} 10-\textsc{prs}-eat-\textsc{fv} \textsc{aug}-8-maize again 10-\textsc{prs}-sing-\textsc{fv} 10-\textsc{prs}-say \phantom{\lq\lq}\textsc{interj} that\_is\_to\_say \phantom{\lq\lq}yes 8-good yes 8-good yes 8-good\\
\glt `When they arrived there, they found the monkeys looking at home in their fields, eating maize, singing and saying ``Ho! Ho! Ho!'' That is to say ``Yes, it's good! Yes, it's good! Yes, it's good!' (Thieving Monkeys)
\ex \label{exkukuti} \gll ʊ-gwe ʊ-lɪ na=a-ma-lʊndɪ lwele, \textbf{kʊkʊtɪ} kɪ-lʊndɪ k-oog-a a-m-ɪɪsi ɪ-n-dobo jɪ-mo\\
\textsc{aug}-\textsc{2sg} \textsc{2sg}-\textsc{cop} \textsc{com}=\textsc{aug}-6-leg eight every 7-leg \textsc{2sg.prs}-bath-\textsc{fv} \textsc{aug}-6-water \textsc{aug}-9-bucket 9-one\\
\glt `You have eight legs, every leg you bathe in one bucket of water.' [Hare and Spider]
\ex \label{exngati}
\gll po jʊ-la i-kw-and-a ʊ-kʊ-bin-a fiijo n=ʊ-kʊ-ʊbʊk-a ʊ-m̩-bɪlɪ \textbf{ngatɪ} lw-ifi\\
then 1-\textsc{dist} 1-\textsc{prs}-begin-\textsc{fv} \textsc{aug}-15-fall\_sick-\textsc{fv} \textsc{intens} \textsc{com}=\textsc{aug}-15-peel\_off-\textsc{fv} \textsc{aug}-3-body like 11-chameleon\\
\glt \lq Then that person begins to get very ill and his body peels like a chameleon.' [Killer woman]
\end{exe}
\section{Movement grams}\label{MovementGrams}
\is{motion|(}
In this section, two auxiliary verbs will be discussed that provide a sense of (figurative) movement:\footnote{The term \textit{movement gram} has been adopted from \citet{NicolleS2002}.} (\textit{j})\textit{a} `go' and \textit{isa} `come'. Both verbs are not only related in meaning, but also pattern together in syntactic terms.\is{syntax} As their complement, they both take an augmentless infinitive.\is{infinitive} Further, the \isi{simple present} of both verbs has undergone further \isi{grammaticalization} to a futurate,\is{futurate} a use in which the infinitive complement can take the imperfective\is{aspect!imperfective} suffix -\textit{aga}. 

\subsection{(\textit{\textit{j}})\textit{\textit{a}} `go'}\label{jaAspectualizer}
The movement verb (\textit{j})\textit{a}, which is glossed as `go' throughout this study, is  attested only as a movement gram, not as a main verb. Following the infinitive\is{infinitive} or \isi{simple present} prefix \textit{kʊ}-, only the vocalic segment is realized, yielding \textit{kwa}. This loss of the consonantal segment is shared with the copula of the same shape (\sectref{Copulae}), albeit in a different environment. Use of (\textit{j})\textit{a} construes the state-of-affairs encoded in the lexical verb against a preceding motion event (cf. \citealt[251]{WilkinsD1991}). One possible reading is that of two sequential sub-eventualities, hence \lq go (and) verb':

\begin{exe}
\ex \gll po lɪnga a-ba-bʊʊl-ile a-ba-paapi ba-ake pamo ʊ-gwise gw-a n̩-dʊmyana jʊ-la, \textbf{a}-\textbf{a}-\textbf{j}-\textbf{aga} \textbf{kʊ}-n-sʊʊm-ɪl-a kʊ-gwise gw-a n-kiikʊlʊ ʊkʊtɪ \textup{\lq\lq}ʊ-mw-anaako, n-gʊ-lond-a ʊkʊtɪ eeg-igw-ege n=ʊ-mw-anangʊ\textup{''}\\
then if/when 1-2-tell-\textsc{pfv} \textsc{aug}-2-parent 2-\textsc{poss.sg} or \textsc{aug}-his\_father(1) 1-\textsc{assoc} 1-boy 1-\textsc{dist} 1-\textsc{pst}-go-\textsc{ipfv} 15-1-beg-\textsc{appl}-\textsc{fv} 17-his\_father(1) 1-\textsc{assoc} 1-woman \textsc{comp} \phantom{\lq\lq}\textsc{aug}-1-your\_child \textsc{1sg}-\textsc{prs}-want-\textsc{fv} \textsc{comp} 1.marry-\textsc{pass}-\textsc{ipfv.subj} \textsc{com}=\textsc{aug}-1-my\_child\\
\glt \lq When he had told his parents or his father, he [father] would go to the woman's father and ask \lq\lq Your child, I want her to be married to my child.''{}' [Life and marriage long ago]

\ex Context: The researcher is on his way home in the afternoon.\\
\gll \textbf{ʊ}-\textbf{j}-\textbf{ile} \textbf{kʊ}-bomb-a?\\
\textsc{2sg}-go-\textsc{pfv} 15-work-\textsc{fv}\\
\glt \lq Did you go and work?' [overheard]
\end{exe}

In other cases, \textit{(j)a} does not introduce a change of location. This becomes clearest when it follows a form of the lexical verb \textit{bʊʊka} \lq go', as in (\ref{exJaFollowingBuuka1}, \ref{exJaPython}). Instead of introducing a second motion event, (\textit{j})\textit{a} recapitulates the goal-oriented motion expressed by preceding \textit{bʊʊka}. In (\ref{exJaFollowingBuuka1}), this involves going to the explicitly mentioned field, and in (\ref{exJaPython}) moving to the house, which is understood from the context.
\begin{exe}
\ex \label{exJaFollowingBuuka1}\gll bo ka-kɪnd-ile=po a-ka-balɪlo ka-nandɪ Pakyɪndɪ \textbf{a}-\textbf{lɪnkʊ}-\textbf{bʊʊk}-\textbf{a} kʊ-n̩-gʊnda. \textbf{a}-\textbf{lɪnkw}-\textbf{a} \textbf{kʊ}-mmw-ag-a ʊ-n-kasi n=ʊ-n-nyambala ʊ-jʊ-ngɪ mu-n̩-gʊnda mu-la\\
as 12-pass-\textsc{pfv}=\textsc{cmpr} \textsc{aug}-12-time 12-little P. 1-\textsc{narr}-go-\textsc{fv} 17-3-field 1-\textsc{narr}-go.\textsc{fv} 15-1-find-\textsc{fv} \textsc{aug}-1-wife \textsc{com}=\textsc{aug}-1-man \textsc{aug}-1-other 18-3-field 18-\textsc{dist}\\
\glt `When a short time had passed, Pakyindi went to the field. He (went and) found his wife with another man in that field.' [Sokoni and Pakyindi]
\ex \label{exJaPython}
Context: Python is hiding in a banana tree outside a house.\\
\gll po j-aa-tɪ \textup{\lq\lq}niine n-gʊ-bʊʊk-a bo a-ka-j-a=po maama jʊ-la ʊ-n-kiikʊlʊ.\textup{''} po \textbf{bo} \textbf{jɪ}-\textbf{bʊʊk}-\textbf{ile} ɪɪ-sota j-oope \textbf{j}-\textbf{aa}-\textbf{j}-\textbf{ile} \textbf{kw}-ɪmb-a\\
then 9-\textsc{subsec}-say \phantom{\lq\lq}\textsc{com.1sg} \textsc{1sg}-\textsc{prs}-go-\textsc{fv} as 1-\textsc{neg}-be(come)-\textsc{fv}=16 mother(<SWA) 1-\textsc{dist} \textsc{aug}-1-woman then as 9-go-\textsc{pfv} \textsc{aug}-python(9) 9-also 9-\textsc{pst}-go-\textsc{pfv} 15-sing-\textsc{fv}\\
\glt \lq Then it [Python] said \lq\lq Me too, I'm going when that woman isn't there.''  ‎‎When the python had gone [to the house], it sang.' [Python and woman]
\end{exe}
 
The preceding example illustrates another important point about (\textit{j})\textit{a}: this construal of a lexical state-of-affairs against the ground of a motion event is often employed in narratives\is{narrative} to trace the participants and their actions as they move through space. (\ref{exJaFollowingBuuka2}--\ref{exjafika}) are further examples of this.

\begin{exe}
\ex \label{exJaFollowingBuuka2}
Context: Hare and Skunk are staying together.\\
\gll po nsysyɪ j-oope \textbf{a}-\textbf{a}-\textbf{bʊʊk}-\textbf{ile}, \textbf{a}-\textbf{a}-\textbf{j}-\textbf{ile} \textbf{kʊ}-lond-a a-ma-ani. a-al-iis-ile na=a-ma-ani ga-la bo a-gon-ile ʊ-tʊ-lo kalʊlʊ\\
then skunk(1) 1-also 1-\textsc{pst}-go-\textsc{pfv} 1-\textsc{pst}-go-\textsc{pfv} 15-search-\textsc{fv} \textsc{aug}-6-leaf 1-\textsc{pst}-come-\textsc{pfv} \textsc{com}=\textsc{aug}-6-leaf 6-\textsc{dist} as 1-rest-\textsc{pfv} \textsc{aug}-13-sleep hare(1)\\
\glt `Skunk also went, he (went and) searched for leaves. He came with those leaves, while Hare was asleep.' [Hare and Skunk]

\ex \label{exJaNyeela}
Context: Hare is trapped in a pit.\\
\gll a-a-fum-ile na=a-ma-ka n-k-iina mu-la. a-a-nyeel-ile \textbf{a}-\textbf{a}-\textbf{j}-\textbf{ile} \textbf{kʊ}-tɪ \textup{\lq\lq}tuu!\textup{''} p-ii-sɪɪlya\\ 
1-\textsc{pst}-come\_from-\textsc{pfv} \textsc{com}=\textsc{aug}-6-force 18-7-pit 18-\textsc{dist} 1-\textsc{pst}-jump-\textsc{pfv} 1-\textsc{pst}-go-\textsc{pfv} 15-say \phantom{\lq\lq}of\_thunk 16-5-other\_side\\
\glt `He [Hare] came out of that pit with force. He jumped and made ``tuu!'' on the other side.' [Saliki and Hare]

\newpage
\ex\label{exjafika}
Context: A woman has just passed a branch-off.\\
\gll po a-lɪnkʊ-golok-a, a-lɪnkʊ-golok-a. \textbf{a}-\textbf{lɪnkw}-\textbf{a} \textbf{kʊ}-fik-a kʊ-jeng-iigwe kʊ-nunu fiijo. po \textbf{a}-\textbf{lɪnkw}-\textbf{a} \textbf{kʊ}-ba-ag-a ba-lɪndɪlɪli ba-a ka-aja ka-la. ba-lɪnkʊ-n̩-daalʊʊsy-a ba-lɪnkʊ-tɪ \textup{\lq\lq}kʊ-lond-a fi-ki?\textup{''}\\
then 1-\textsc{narr}-go\_straight-\textsc{fv} 1-\textsc{narr}-go\_straight-\textsc{fv}. 1-\textsc{narr}-go.\textsc{fv} 15-arrive-\textsc{fv} 17-build-\textsc{pass.pfv} 17-well \textsc{intens} then 1-\textsc{narr}-go.\textsc{fv} 15-2-find-\textsc{fv} 2-guard 2-\textsc{assoc} 12-village 12-\textsc{dist} 2-\textsc{narr}-1-ask-\textsc{fv} 2-\textsc{narr}-say \phantom{\lq\lq}\textsc{2sg.prs}-want-\textsc{fv} 8-what\\
\glt \lq She went straight, she went straight. She (went and) arrived at a place well built. She (went and) met the guards of that village. They asked \lq\lq What do you want?''{}' [Throw away the child]
\end{exe}

\largerpage
The simple present of (\textit{j})\textit{a} is further grammaticalized as a marker of prospective aspect,\is{aspect!prospective} which retains a possible spatial or motion reading. This is discussed in \sectref{Prospectivekwa}. Note that the movement gram (\textit{j})\textit{a} cannot express motion with purpose. For this, \textit{bʊʊka} \lq go' plus an infinitive\is{infinitive} marked for \isi{locative} class\is{noun classes} 17 or 18 has to be used; see \sectref{VerbalNounsArguments} for a discussion. Lastly, unlike its counterpart \textit{isa} (\sectref{isaAspectualizer}), the \isi{simple present} of (\textit{j})\textit{a} does not have a habitual\is{aspect!habitual} or generic\is{aspect!generic} reading:

\begin{exe}
\ex[*]{\gll kʊkʊtɪ ky-ɪnja n-gw-a kʊ-gy-ag-a ɪ-mi-kambɪlɪ kʊ-mi-gʊnda gy-ɪtʊ\\
every 7-year \textsc{1sg}-\textsc{prs}-go.\textsc{fv} 15-4-find-\textsc{fv} \textsc{aug}-6-monkey 17-4-field 4-\textsc{poss.1pl}\\
\glt (intended: \lq Every year I go and find damn monkeys in our fields.')}
\end{exe}

\subsection{\textit{isa} `come'}\label{isaAspectualizer}
The verb \textit{isa} \lq come', when used as an auxiliary,\is{auxiliary} has a figurative meaning of reaching, achieving or being led to a particular condition.

\begin{exe}

\ex \gll bo a-lɪ n=ʊ-lw-anda \textbf{iis}-\textbf{aga} \textbf{kʊ}-pon-a nalooli ʊ-mw-ana ʊ-n-kiikʊlʊ\\
as 1-\textsc{cop} \textsc{com}=\textsc{aug}-11-stomach 1.\textsc{pst}.come-\textsc{ipfv} 15-give\_birth-\textsc{fv} really \textsc{aug}-1-child \textsc{aug}-1-woman\\
\glt \lq When she was pregnant, she would eventually give birth to a girl.' [Life and marriage long ago]

\ex \label{exIsaAuxNegPRS}\gll mw-ilaamwisye ɪ-n-dagɪlo sy-angʊ. ʊ-mw-ana a-ka-bagɪl-a ʊ-kw-end-a kangɪ, \textbf{a}-\textbf{ti}-\textbf{kw}-\textbf{is}-\textbf{a} \textbf{kʊ}-job-a sikʊ kangɪ\\
\textsc{2pl}-disregard.\textsc{pfv} \textsc{aug}-10-rule 10-\textsc{poss.1sg} \textsc{aug}-1-child 1-\textsc{neg}-be\_able-\textsc{fv} \textsc{aug}-15-walk/travel-\textsc{fv} again, 1-\textsc{neg}-\textsc{prs}-come-\textsc{fv} 15-speak-\textsc{fv} ever again\\
\glt `You have disregarded my rules. The child can't walk, it'll never get to talk.' [Pregnant women]

\ex \gll po kanunu ʊ-kʊ-j-a m̩-bombi gw-abo kʊ-ka-balɪlo a-ka-a kʊ-lond-a ʊ-kʊ-kab-a ɪ-n-dalama ɪ-sy-a k-ʊʊl-ɪl-a ɪ-fi-bombelo, ɪ-fy-a \textbf{kw}-\textbf{is}-\textbf{a} \textbf{kʊ}-bomb-el-a kɪsita kʊ-lʊmbʊʊs-igw-a\\
then well \textsc{aug}-15-be(come)-\textsc{fv} 1-worker 1-\textsc{poss.pl} 17-12-time \textsc{aug}-12-\textsc{assoc} 15-want-\textsc{fv} \textsc{aug}-15-get-\textsc{fv} \textsc{aug}-10-money \textsc{aug}-10-\textsc{assoc} 15-buy-\textsc{appl}-\textsc{fv} \textsc{aug}-8-tool \textsc{aug}-8-\textsc{assoc} 15-come-\textsc{fv} 15-work-\textsc{appl}-\textsc{fv} without 15-humiliate-\textsc{pass}-\textsc{fv}\\%xxx check vowel length akaa
\glt \lq ‎‎And so it is good to be their worker for a time in which you want to get money to buy tools with, for later working with without being disparaged [lit. \ldots tools of coming to work with \ldots].' [Types of tools in the home]

\ex \gll a-ka-pango a-ka ki-kʊ-tʊ-many-isy-a ʊkʊtɪ tʊ-ng-iib-aga, \textbf{tʊ}-\textbf{ng}-\textbf{iis}-\textbf{a} \textbf{kʊ}-fw-a bo lʊʊ$\sim$lo sy-a-fw-ile ɪ-n-gambɪlɪ si-la\\
\textsc{aug}-12-story \textsc{aug}-\textsc{prox.12} 12-\textsc{prs}-\textsc{1pl}-know-\textsc{caus}-\textsc{fv} \textsc{comp} \textsc{1pl}-\textsc{neg.subj}-steal-\textsc{ipfv} \textsc{1pl}-\textsc{neg.subj}-come-\textsc{fv} 15-die-\textsc{fv} as \textsc{redupl}$\sim$\textsc{ref.11} 10-\textsc{pst}-die-\textsc{pfv} \textsc{aug}-10-monkey 10-\textsc{dist}\\
\glt `This story teaches us that we should not steal, otherwise we will die [lit. we should not come to die] just like those monkeys died.' [Thieving monkeys]
\end{exe}

In the affirmative subjunctive,\is{mood!subjunctive} a variant construction is attested in which the lexical verb is not expressed as an augment-less infinitive,\is{infinitive} but also figures in the subjunctive paradigm:

\begin{exe}
\ex \gll kangɪ ʊ-swɪl-enge=po n=ɪ-n-gʊlʊbe pa-ka-aja ʊkʊtɪ bo g-ʊʊl-iisye ɪ-n-dalama ɪ-syo \textbf{s}-\textbf{iis}-\textbf{e} \textbf{si}-\textbf{kʊ}-\textbf{tʊʊl}-\textbf{ege} ʊ-kʊ-ba-homb-a a-ba-fundi\\
again \textsc{2sg}-rear-\textsc{ipfv.subj}=16 \textsc{com}=\textsc{aug}-10-pig 16-12-homestead \textsc{comp} as \textsc{2sg}-buy-\textsc{caus.pfv} \textsc{aug}-10-money \textsc{aug}-\textsc{ref.10} 10-come-\textsc{subj} 10-\textsc{2sg}-help-\textsc{ipfv.subj} \textsc{aug}-15-2-pay-\textsc{fv} \textsc{aug}-2-workman(<SWA)\\
\glt \lq And you should be raising pigs at home so that when you have sold them, the money can be helping you to pay the workmen [lit. \ldots so that the money comes to help you \ldots].' [How to build modern houses]
\end{exe}

Note that the movement gram \textit{isa}, like its counterpart (\textit{j})\textit{a} (see \sectref{Prospectivekwa}) cannot express motion with purpose. For this, an infinitive\is{infinitive} marked for \isi{locative} class\is{noun classes} 16 or 18 has to be used; see \sectref{VerbalNounsArguments} for a discussion.

As (\ref{exIsaAuxNegPRS}) above indicates, the \isi{simple present} of \textit{isa} has a \isi{futurate} reading. Another example of this is given in (\ref{exIsaAuxFuturateMovement}). This is also the only use of \textit{isa} discussed by \citet{SchumannK1899} and \citet{EndemannC1914}. As (\ref{exIsaAuxFuturateHabGen}) illustrates, the \isi{simple present} of \textit{isa}, however, also allows for a habitual/generic reading.\is{aspect!habitual}\is{aspect!generic}

\begin{exe}
\ex \label{exIsaAuxFuturateMovement}
\gll lɪlɪno \textbf{tʊ}-\textbf{kw}-\textbf{is}-\textbf{a} \textbf{kʊ}-kin-a ʊ-m-pɪla\\
now/today \textsc{1pl}-\textsc{prs}-come-\textsc{fv} 15-play-\textsc{fv} \textsc{aug}-3-ball\\
\glt \lq Today we'll come to play football [ET]'
\ex \label{exIsaAuxFuturateHabGen} \gll kʊkʊtɪ ky-ɪnja \textbf{n}-\textbf{gw}-\textbf{is}-\textbf{a} \textbf{kʊ}-gy-ag-a ɪ-mi-gambɪlɪ m-mi-gunda gy-ɪtʊ\\
every 7-year \textsc{1sg}-\textsc{prs}-come-\textsc{fv} 15-4-find-\textsc{fv} \textsc{aug}-4-monkey 18-4-farm 4-\textsc{poss.1pl}\\
\glt \lq Every year I come to find damn monkeys in our fields.' [ET]
\end{exe}%auch nötig, weil a INF kein PRS.HAB mehr erlaubt

In the \isi{futurate} use of \textit{isa}, the infinitive complement can take the imperfective\is{aspect!imperfective} suffix \mbox{-\textit{aga}}, which yields a continuous/progressive\is{aspect!progressive} reading and can add an epistemic\is{modality} flavour (\ref{exIsaAuxImperfective1}). Imperfective\is{aspect!imperfective} \mbox{-\textit{aga}} is also used with a habitual/generic reading\is{aspect!habitual}\is{aspect!generic} (\ref{exIsaAuxImperfective2}). Lastly, this \isi{futurate} use of \textit{isa} in the \isi{simple present} has undergone further grammaticalization,\is{grammaticalization} yielding the indefinite future\is{future!indefinite future} construction (\sectref{isaFut}).

\begin{exe}
\ex \label{exIsaAuxImperfective1}
\gll i-kw-is-a kʊ-jeng-aga kʊ-la\\
1-\textsc{prs}-come-\textsc{fv} 15-build-\textsc{ipfv} 17-\textsc{dist}\\
\glt 1. \lq He will come to be building there (continuously).'\\
2. \lq He will come to build there (presumably).' [ET]
\ex \label{exIsaAuxImperfective2}
\gll bi-kw-is-a kʊ-kin-aga ʊ-m-pɪla kʊkʊtɪ ii-sikʊ\\
2-\textsc{prs}-come-\textsc{fv} 15-play-\textsc{ipfv} \textsc{aug}-3-ball every 5-day\\
\glt \lq They will come to play football every day.' [ET]
\end{exe}
\is{motion|)}

\chapter{Verbal nouns (infinitives)}
\label{ChapterInfinitives}\is{infinitive|(}
\section{Introduction}
In this chapter, verbal nouns (infinitives) will be discussed. After a description of their morphological structure and syntactic characteristics\is{syntax} (\sectref{VerbalNounsStructureCharacteristics}), the negation\is{negative} of verbal nouns and negative construction containing them will be described (\sectref{VerbalNounsNegation}). This is followed by an discussion of some of their functions (\sectref{VerbalNounsArguments}, \ref{VerbalNounsConverbsAndRelated}).
\section{Structure and characteristics of verbal nouns}\label{VerbalNounsStructureCharacteristics}
Verbal nouns (infinitives) share characteristics of both nouns and verbs. Formally, they are class 15 nouns\is{noun classes} and can hence be marked for one of the three \isi{locative} classes or carry the augment (\sectref{NominalMorphology}). Like any other noun phrase, infinitives can fulfil the syntactic\is{syntax} functions of subjects (\ref{exINFasSubject}), objects (\ref{exINFasObject}), the head of possessives (\ref{exINFasObject}), and of the dependent noun of the associative (\ref{exInfAssociative}).

\begin{exe}
\ex \label{exINFasSubject}
\gll kɪsita kʊ-bomb-a bo ʊ-lo, \textbf{kʊ}-ka-a-li=ko \textbf{ʊ}-\textbf{kʊ}-\textbf{keet}-\textbf{an}-\textbf{a} kʊ-maa-so n=ʊ-kʊ-ponani-a\\
without 15(\textsc{inf})-do-\textsc{fv} as \textsc{aug}-\textsc{ref}.11 15(\textsc{inf})-\textsc{neg}-\textsc{pst}-\textsc{cop}=17 \textsc{aug}-15(\textsc{inf})-see-\textsc{recp}-\textsc{fv} 17-6-eye \textsc{com}=\textsc{aug}-15(\textsc{inf})-greet.\textsc{recp}-\textsc{fv}\\
\glt `Without doing so, there was no looking each other in the eyes or greeting each other.' [Should she save a life\ldots]
\ex \label{exINFasObject} \gll n-\textbf{kʊ}-meenye \textbf{ʊ}-\textbf{kʊ}-\textbf{pɪɪj}-\textbf{a} \textbf{kw}-ake\\
\textsc{1sg}-15(\textsc{inf})-know.\textsc{pfv} \textsc{aug}-15(\textsc{inf})-cook-\textsc{fv} 15(\textsc{inf})-\textsc{poss.sg}\\
\glt \lq I know her cooking.' [ET]
\ex\label{exInfAssociative}
 \gll ba-a-tendekesy-aga ngatɪ kɪ-kombe ky-\textbf{a} \textbf{kʊ}-\textbf{nw}-\textbf{el}-\textbf{a} a-m-ɪɪsi\\
2-\textsc{pst}-prepare-\textsc{ipfv} like 7-cup 7-\textsc{assoc} 15(\textsc{inf})-drink-\textsc{appl}-\textsc{fv} \textsc{aug}-6-water\\
\glt \lq They prepared them [calabashes] just like a cup for drinking water.' [Lake Kyungululu]
\end{exe}

With respect to their verbal characteristics, verbal nouns can be modified by adverbials (\ref{exInfinitiveAdverbial}). They can take the complements licensed by the verb \isi{stem} and accordingly may carry an object marker;\is{object marker} see (\ref{exInfiniveImperfective}, \ref{exInfinitiveMotion1}) below. Further, infinitives can take post-final clitics;\is{enclitic} see (\ref{exNukuPreparationCulmination}) below.

\begin{exe}
\ex \label{exInfinitiveAdverbial}\gll ʊ-kʊ-jeng-a panandɪ$\sim$panandɪ jɪ-ka-j-a m-bombo n-gafu\\
\textsc{aug}-15-build-\textsc{fv} \textsc{redupl}$\sim$a\_little 9-\textsc{neg}-be(come)-\textsc{fv} 9-work 9-difficult\\ 
\glt `Building little by little is not difficult work.' [How to build modern houses]
\end{exe}

The \isi{stem} of verbal nouns consists of the base and the default final vowel \mbox{-\textit{a}}. With the movement\is{motion} grams (\textit{j})\textit{a} and \textit{isa} (\sectref{MovementGrams}), the infinitive may take the imperfective\is{aspect!imperfective} final suffix \mbox{-\textit{aga}}. The only other token of an infinitive carrying the imperfective\is{aspect!imperfective} suffix is the following example, where -\textit{aga} seems to indicate the generic aspect\is{aspect!generic} of the comitative infinitive vis-à-vis its perfective\is{aspect!perfective} superordinate verb.
\begin{exe}
\ex \label{exInfiniveImperfective} \gll a-ba-ndʊ ba-a-jeng-ile \textbf{n}=\textbf{ʊ}-\textbf{kʊ}-\textbf{tʊʊgasy}-\textbf{aga} n-ka-aja a-ko looli ba-a-taami-gw-aga ʊ-kʊ-ga-ag-a a-m-ɪɪsi a-ga-a kʊ-nw-a n-ʊ-kʊ-nw-esy-a ɪ-mi-tiimo gy-abo\\ 
\textsc{aug}-2-person 2-\textsc{pst}-build-\textsc{pfv} \textsc{com}=\textsc{aug}-15-settle-\textsc{ipfv} 18-12-homestead \textsc{aug}-\textsc{ref.12} but 2-\textsc{pst}-trouble-\textsc{pass}-\textsc{ipfv} \textsc{aug}-15-6-find-\textsc{fv} \textsc{aug}-6-water \textsc{aug}-6-\textsc{assoc} 15-drink-\textsc{fv} \textsc{com}=\textsc{aug}-15-drink-\textsc{caus}-\textsc{fv} \textsc{aug}-4-herd 4-\textsc{poss.pl}\\
\glt 'People (had) built in that village but they had trouble finding water for drinking and watering their cattle.' [Water and toads]
\end{exe}
\section{Verbal nouns and negation}\label{VerbalNounsNegation}\is{negative|(}
Verbal nouns in Nyakyusa cannot be negated morphologically. To express the negation of an infinitive, periphrastic constructions are used. The most common one is (\textit{ʊ})\textit{kʊsita}, (\textit{ɪ})\textit{kɪsita} `without' followed by an augmentless infinitive (\ref{exKisita}). The former also figures in the negative counterpart to the \isi{narrative tense} (\sectref{NarrativeTense}).

\begin{exe}
\ex \label{exKisita} \gll lɪnga a-lɪ na=fyo a-bagiile ʊ-kʊ-bomb-a ɪ-m-bombo jo$\sim$j-oosa ɪ-jɪ i-kʊ-lond-a ʊ-kʊ-bomb-a \textbf{kɪsita} \textbf{kʊ}-\textbf{taami}-\textbf{gw}-\textbf{a}\\
if/when 1-\textsc{cop} \textsc{com}=\textsc{ref.8} 1-be\_able.\textsc{pfv} \textsc{aug}-15-work-\textsc{fv} \textsc{aug}-9-work \textsc{redupl}$\sim$9-all \textsc{aug}-\textsc{prox.9} 1-\textsc{prs}-want-\textsc{fv} \textsc{aug}-15-work-\textsc{fv} without 15-trouble-\textsc{pass}-\textsc{fv}\\
\glt \lq If he has them [tools], he can do any kind of work which he wants to do, without being bothered.' [Types of tools in the home]
\end{exe}

A construction for constituent negation consists of the substitutive as a \isi{proclitic} to the general negator \textit{mma}, followed by the infinitive carrying the augment.\footnote{Cf. also \citet[69]{SchumannK1899} and \citet[84f]{EndemannC1914}.}
\begin{exe}
\ex \label{exRefMma1}\gll kʊkʊtɪ ii-sikʊ i-kʊ-kʊ-tʊk-a, kangɪ i-kʊ-tɪ \textup{\lq\lq}ʊ-ne \textbf{ne}=\textbf{mma} \textbf{ʊ}-\textbf{kw}-\textbf{eg}-\textbf{igw}-\textbf{a} na Juma, n-ga-n̩-gan-a\textup{''}\\
every 5-day 1-\textsc{prs}-\textsc{2sg}-insult-\textsc{fv} again 1-\textsc{prs}-say \phantom{\lq\lq}\textsc{aug}-\textsc{1sg} \textsc{1sg}=no \textsc{aug}-15-marry-\textsc{pass}-\textsc{fv} \textsc{com} J. \textsc{1sg}-\textsc{neg}-1-love-\textsc{fv}\\
\glt `Every day she speaks badly about you and she says ``Me, I'm not getting married to Juma, I don't love him.''{}' [Juma, Asia and Sambuka]
\end{exe}

This construction, with the class 15 substitutive \textit{ko}, thus \textit{komma}, also serves to form negative commands (\ref{exKomma}). A free variant \textit{somma} is also found (\ref{exSomma1}, \ref{exSomma2}).\footnote{The source of the initial fricative is unclear.} These prohibitives can be adressed to a single person (\ref{exKomma}, \ref{exSomma1}) as well as to the second person plural (\ref{exSomma2}).
\begin{exe}
\ex \label{exKomma}\gll \textbf{komma} \textbf{ʊ}-\textbf{kʊ}-\textbf{nyonyw}-\textbf{a} ɪ-fi a-p-eeliigwe ʊ-n-nino\\
\textsc{proh} \textsc{aug}-15(\textsc{inf})-desire-\textsc{fv} \textsc{aug}-\textsc{prox}.8 1-give-\textsc{pass}.\textsc{pfv} \textsc{aug}-1-your\_companion\\
\glt `Do not desire what your neighbour has been given.' [Chief Kapyungu]
\ex \label{exSomma1}
\gll \textbf{somma} \textbf{ʊ}-\textbf{kʊ}-\textbf{paasy}-\textbf{a}! lee po keet-a, ʊ-ka-a-job-aga bo ʊ-kaalɪ ʊ-kʊʊ-ny-eeg-a?\\
\textsc{proh} \textsc{aug}-15-worry-\textsc{fv} now/but then watch-\textsc{fv} \textsc{2sg}-\textsc{neg}-\textsc{pst}-speak-\textsc{ipfv} as \textsc{2sg}-\textsc{pers} \textsc{aug}-15-\textsc{1sg}-take-\textsc{fv}?\\
\glt \lq Don't worry! Now look, why didn't you speak before picking me up?' [Crocodile and Monkey]
\ex \label{exSomma2}
\gll lɪnga m-b-iigal-iile \textbf{somma} \textbf{ʊ}-\textbf{kʊ}-\textbf{sook}-\textbf{a} pa-nja\\
if/when \textsc{1sg}-\textsc{2pl}-close-\textsc{appl.pfv} \textsc{proh} \textsc{aug}-15-leave-\textsc{fv} 16-outside\\
\glt \lq When I've locked you (pl.) in, don't go outside.' [Python and woman]
\end{exe}
\is{negative|)}
\section{Functions of verbal nouns}\label{VerbalNounsFunctions}

\subsection{Arguments of auxiliaries, modal and motion verbs}\label{VerbalNounsArguments}
Verbal nouns serve as complements of phasal verbs,\is{phasal verbs} also called \textit{aspectualizers}, such as \textit{anda} `begin, start', \textit{mala} `finish, stop', \textit{leka} `seize' or  \textit{endelela} `continue'. These take either the infinitive with the augment or the infinitive marked for \isi{locative} class 16 as their complement. The latter is illustrated in (\ref{exInfinitive16Aspectualizer}). For numerous examples of the first see Chapter \ref{AspectualClassification}. It is unclear how far the two differ in meaning and use. Speaker preferences seem to play a role: the younger language assistants used the class 16 form more frequently than the older assistants.

\begin{exe}
\ex \label{exInfinitive16Aspectualizer}
\gll i-kʊ-kwel-a kangɪ mu-m-piki n=ʊ-kw-\textbf{endelel}-a \textbf{pa}-\textbf{kw}-\textbf{ap}-\textbf{a} a-ma-peasi\\
1-\textsc{prs}-climb-\textsc{fv} again 18-3-tree \textsc{com}=\textsc{aug}-15-continue-\textsc{fv} 16-15-pick-\textsc{fv} \textsc{aug}-6-pear(<SWA)\\
\glt \lq He climbs up the tree again and continues to pick pears.' [Elisha pear story]
\end{exe}

Infinitives, either with the augment or marked for \isi{locative} class 16, also figure as arguments of \isi{modality} and manipulation verbs, where they alternate with the subjunctive\is{mood!subjunctive} (\sectref{SubjunctiveSubordinate}). The alternation between infinitives and the subjunctive mood\is{mood!subjunctive} is also found following predicative expressions of (dis-)approval or preference, including the invariants \textit{kyajɪpo} \lq (it is) better' and \textit{paakipo} \lq (it is) preferable'.

\largerpage[2]
\begin{exe}
\ex \label{exInfinitiveApproval} \gll n̩-dʊ-baatɪko lw-a twe ba-Nyakyusa ʊ-n-nyambala \textbf{ʊ}-\textbf{kʊ}-\textbf{pɪɪj}-\textbf{a}, pamo \textbf{ʊ}-\textbf{kʊ}-\textbf{suk}-\textbf{a} ɪ-my-enda, pamo \textbf{ʊ}-\textbf{kʊ}-\textbf{neg}-\textbf{a} a-m-ɪɪsi bo ba-li=po a-ba-kiikʊlʊ \textbf{mw}-\textbf{iko}\\
18-11-procedure 11-\textsc{assoc} \textsc{1pl} 2-Ny. \textsc{aug}-1-man \textsc{aug}-15-cook-\textsc{fv} or \textsc{aug}-15-wash-\textsc{fv} \textsc{aug}-4-clothe or \textsc{aug}-15-draw\_liquid-\textsc{fv} \textsc{aug}-6-water as 2-\textsc{cop}=16 \textsc{aug}-2-woman 3-taboo\\
\glt \lq  In the custom of us, the Nyakyusa people, it is taboo for men to cook or wash clothes or draw water when women are around.' [Division of labour]
\end{exe}

Infinitives further function as oblique arguments of modal\is{modality} readings of verbs such as \textit{tola} \lq  defeat' and its passive \textit{toligwa} \lq fail', \textit{kɪnda} \lq surpass' or \textit{taamigwa} \lq be troubled' (\ref{exToligwa}). See also (\ref{exInfiniveImperfective}) on p.\nobreakspace\pageref{exInfiniveImperfective} above and (\ref{exOpeningHareTugutuSentence2}, \ref{exOpeningHareTugutuSentence3}) on p.\nobreakspace\pageref{exOpeningHareTugutuSentence2}.

\begin{exe}
\ex \label{exToligwa}
\gll tʊ-tol-iigwe ʊ-kʊ-lɪ-kol-a ii-bole\\
\textsc{1pl}-defeat-\textsc{pass.pfv} \textsc{aug}-15-5-grasp/hold-\textsc{fv} 5-leopard\\
\glt `We've failed to catch the leopard.' [Chief Kapyungu]
\end{exe}

Infinitives additionally marked for \isi{locative} classes\is{noun classes} 16 or 18 also constitute the lexical verb of periphrastic TMA constructions, namely the periphrastic progressive\is{aspect!progressive} (\sectref{Progressive}), the prospective/inceptive\is{aspect!prospective} (\sectref{ProspectiveKujapa}) and the \isi{narrative tense} (\sectref{NarrativeTense}). An infinitive with the augment or marked for \isi{locative} classes 16 or 18\is{noun classes} may further serve as the complement of the persistive\is{aspect!persistive} aspect \isi{auxiliary} (\sectref{Persistive}) and augmentless infinitives may serve as the semantic main verb of the movement grams (\sectref{MovementGrams}).\is{motion}

Lastly, verbs of \isi{motion} and related verbs such as \textit{ɪma} `stand, stop' or \textit{tʊma} \lq send' often take an infinitive complement additionally marked for one of the three \isi{locative} classes.\is{noun classes} Class 16 here indicates that the motion is in relation to a specific place where the eventuality of the verbal noun takes place (\ref{exLocInfMovementVerb16}). With class 17, this typically denotes motion with a purpose (\ref{exLocInfMovementVerb17purpose}). In a related fashion, a class 17 infinitive can specifically serve as a purpose clause in this context (\ref{exLocInfMovementVerb17both}). However, a pure motion reading \lq to / from' the eventuality is also possible (\ref{exLocInfMovementVerb17locational}). Infinitives marked for class 18 also predominantly give a purposive reading (\ref{exLocInfMovementVerb18purpose}), although a locational one is also attested (\ref{exLocInfMovementVerb18locational}).%\vspace{\baselineskip}
\begin{exe}
\ex \label{exLocInfMovementVerb16} \gll ɪ-m-bwa sy-ɪm-aga \textbf{pa}-\textbf{kʊ}-\textbf{ly}-\textbf{a} ɪ-fi-fupa\\
\textsc{aug}-10-dog 10-\textsc{pst}.stand/stop-\textsc{ipfv} 16-15-eat-\textsc{fv} \textsc{aug}-8-bone\\
\glt \lq The dogs would stop and eat the bones.' [Saliki and Hare]

\ex \label{exLocInfMovementVerb17purpose}
\gll a-ka-balɪlo ka-mo a-ba-hɪɪji ba-na ba-a-bʊʊk-ile \textbf{kʊ}-\textbf{kʊ}-\textbf{hɪɪj}-\textbf{a} ɪɪ-ng'ombe pa-kɪ-lo\\
\textsc{aug}-12-time 12-one \textsc{aug}-2-thief 2-four 2-\textsc{pst}-go-\textsc{pfv} 17-15-steal-\textsc{fv} \textsc{aug}-cow(10) 16-7-night\\
\glt \lq One time, four thieves went to steal cows at night.' [Wage of the thieves]
\ex \label{exLocInfMovementVerb17both}
\gll Kalʊlʊ a-lɪnkʊ-bʊʊk-a kʊ-lʊ-bʊbi \textbf{kʊ}-\textbf{kʊ}-\textbf{laalʊʊsy}-\textbf{a} lɪnga lʊ-mmw-ag-ile ʊ-n-kiikʊlʊ\\
Hare 1-\textsc{narr}-go-\textsc{fv} 17-11-spider 17-15-ask-\textsc{fv} if/when 11-1-find-\textsc{pfv} \textsc{aug}-1-woman\\
\glt \lq Hare went to Spider to ask if it had found a woman.' [Hare and Spider]
\ex \label{exLocInfMovementVerb17locational}
\gll bo lʊ-fum-ile \textbf{kʊ}-\textbf{kʊ}-\textbf{hah}-\textbf{a} \ldots\\
as 11-come\_from-\textsc{pfv} 17-15-seduce-\textsc{fv} {}\\
\glt \lq When it [Spider] returned from seducing \ldots' [Hare and Spider]
\ex \label{exLocInfMovementVerb18purpose}
\gll a-ba-ndʊ ba-a-bʊʊk-ile \textbf{n}-\textbf{kʊ}-\textbf{n}-\textbf{keet}-\textbf{a}\\
\textsc{aug}-2-person 2-\textsc{pst}-go-\textsc{pfv} 18-15-1-watch-\textsc{fv}\\
\glt \lq People went to see her.' [Mfyage turns into a lion]
\ex \label{exLocInfMovementVerb18locational}
\gll Saliki a-lɪnkw-is-a ʊ-kʊ-fum-a \textbf{n}-\textbf{kʊ}-\textbf{jaat}-\textbf{a}\\
S. 1-\textsc{narr}-come-\textsc{fv} \textsc{aug}-15-come\_from-\textsc{fv} 18-15-walk-\textsc{fv}\\
\glt \lq Saliki came from taking a walk.' [Saliki and Hare]
\end{exe}

\subsection{Uses as converbs and related functions}\label{VerbalNounsConverbsAndRelated}
Infinitives can be used in a fashion similar to converbs of simultaneity. The term converb is here understood in \citeauthor{HaspelmathM1995}'s (\citeyear[3]{HaspelmathM1995}) definition as \lq\lq a non-finite verb form whose main function is to mark adverbial subordination. Another way of putting it is that converbs are verbal adverbs''. Examples (\ref{exInfinitiveConverbSimultaneity1}--\ref{exInfinitiveMotion1}) illustrate this. The use of infinitives in a converb-like manner is especially common with verbs of motion, where each verb provides different components of a single motion event (\ref{exInfinitiveMotion1}).
\begin{exe}
\ex \label{exInfinitiveConverbSimultaneity1}\gll ba-lɪnkʊ-fimbɪlɪsy-a fiijo \textbf{ʊ}-\textbf{kʊ}-\textbf{n̩}-\textbf{daalʊʊsy}-\textbf{a} mpaka a-a-job-ile a-a-t-ile\\
2-\textsc{narr}-persuade-\textsc{fv} \textsc{intens} \textsc{aug}-15-1-ask-\textsc{fv} until 1-\textsc{pst}-speak-\textsc{pfv} 1-\textsc{pst}-say-\textsc{pfv}\\
\glt `They interrogated her much until she spoke.' [Killer woman]
\ex \label{exInfinitiveConverbSimultaneity2}\gll \textbf{ʊ}-\textbf{kʊ}-\textbf{keet}-\textbf{a} kʊ-mwanya ki-kʊ-bon-a ʊ-mu-ndʊ\\
\textsc{aug}-15-watch-\textsc{fv} 17-high 12-\textsc{prs}-see-\textsc{fv} \textsc{aug}-1-person\\
\glt \lq Looking up he sees a person.' [Nicholaus Pear Story]
\ex \label{exInfinitiveMotion1}\gll bo i-kw-and-a itolo ʊ-kʊ-kam-a, ʊ-n̩-dʊme a-lɪnkʊ-sook-a \textbf{ʊ}-\textbf{kʊ}-\textbf{fum}-\textbf{a} kʊʊ-sofu n=ʊ-kʊ-n-kol-a ʊ-n-kiikʊlʊ jʊ-la\\
as 1-\textsc{prs}-begin-\textsc{fv} just \textsc{aug}-15-milk-\textsc{fv} \textsc{aug}-1-husband 1-\textsc{narr}-leave-\textsc{fv} \textsc{aug}-15-come\_from-\textsc{fv} 17-room(9) \textsc{com}=\textsc{aug}-15-1-grasp/hold-\textsc{fv} \textsc{aug}-1-woman 1-\textsc{dist}\\
\glt `When she was starting to milk, the husband came out of [lit. left coming from] the bedroom and caught that woman.' [Killer woman]
\end{exe}

\label{ComitativeInfinitive}An infinitive together with a \isi{proclitic} form of the comitative \textit{na} can be used following another verb to create a tight link between the states-of-affairs of the two, which often occur in sequence. Most commonly the first verb is fully inflected. The relationship between these verbs can be one of cause and consequence (\ref{exNukuCauseConsequence}), preparation and culmination (\ref{exNukuPreparationCulmination}), or eventualities based on each other in a more general sense (\ref{exNukuBasedOnEachOther}). It is also attested with verbs expressing similar or conceptually related meanings (\ref{exNukuSimilarMeaning1}) and with the last verb in iconic repetitions (\ref{exNukuRepetition}). (\ref{exNukuINF}) illustrates coordination with a preceding infinitive complement.
\begin{exe}
\ex \label{exNukuCauseConsequence} \gll po jɪ-lɪnkʊ-jɪ-lʊm-a ɪɪ-nine \textbf{n}=\textbf{ʊ}-\textbf{kʊ}-\textbf{jɪ}-\textbf{gog}-\textbf{a}\\
then 9-\textsc{narr}-9-bite-\textsc{fv} \textsc{aug}-companion.9 \textsc{com}=\textsc{aug}-15-9-kill-\textsc{fv}\\
\glt `It [dog] bit the other one and killed it.' [Dogs laughed about each other]
\ex \label{exNukuPreparationCulmination} \gll i-kʊ-pɪmb-a ɪ-kɪ-kapʊ ky-osa n-ky-eni mu-n-jɪnga \textbf{n}=\textbf{ʊ}-\textbf{kʊ}-\textbf{sook}-\textbf{a}=\textbf{po}\\
1-\textsc{prs}-lift-\textsc{fv} \textsc{aug}-7-basket 7-all 18-7-forehead 18-9-bicycle \textsc{com}=\textsc{aug}-15-leave-\textsc{fv}=16\\
\glt `He loads a whole basket onto the front of his bicycle and rides away.' [Elisha Pear Story]
\ex \label{exNukuBasedOnEachOther}
\gll ʊ-n-kasi gw-a lʊ-bʊbi a-a-b-ambɪliile \textbf{n}=\textbf{ʊ}-\textbf{kʊ}-\textbf{ba}-\textbf{pɪɪj}-\textbf{ɪl}-\textbf{a} ɪ-fi-ndʊ ɪ-f-ingi fiijo\\
\textsc{aug}-1-wife 1-\textsc{assoc} 11-spider 1-\textsc{pst}-2-receive.\textsc{pfv} \textsc{com}=\textsc{aug}-15-2-cook-\textsc{appl}-\textsc{fv} \textsc{aug}-8-food \textsc{aug}-8-many \textsc{intens}\\
\glt `Spider's wife received them and cooked a lot of food for them.' [Hare and Spider]
\ex \label{exNukuSimilarMeaning1}\gll n̩goosi a-lɪnkʊ-kʊl-a \textbf{n}=\textbf{ʊ}-\textbf{kʊ}-\textbf{kiikʊlʊp}-\textbf{a}\\
N. 1-\textsc{narr}-grow-\textsc{fv} \textsc{com}=\textsc{aug}-15-become\_woman-\textsc{fv}\\
\glt `Ngoosi grew up and became a woman.' [Man and his in-law]
\ex\label{exNukuRepetition} \gll boo=bʊno$\sim$bʊ-no ba-lɪnkw-end-a, ba-lɪnkw-end-a, ba-lɪnkw-end-a \textbf{n}=\textbf{ʊ}-\textbf{kw}-\textbf{end}-\textbf{a}\\
\textsc{ref.14}=\textsc{redupl}$\sim$14-\textsc{dem} 2-\textsc{narr}-walk/travel-\textsc{fv} 2-\textsc{narr}-walk/travel-\textsc{fv} 2-\textsc{narr}-walk/travel-\textsc{fv} \textsc{com}=\textsc{aug}-15-walk/travel-\textsc{fv}\\
\glt \lq Thus they travelled, travelled, travelled and travelled.' [Pregnant women]
% bsp leider doppelt verwendet
\ex\label{exNukuINF} \gll ɪ-n-gwina j-iis-aga n-kʊ-j-eeg-a ɪ-n-gambɪlɪ \textbf{n}=\textbf{ʊ}-\textbf{kʊ}-\textbf{bʊʊk}-\textbf{a} na=jo pa-lʊ-sʊngo pa-kw-angal-a\\
\textsc{aug}-9-crocodile 9-\textsc{pst}.come-\textsc{ipfv} 18-15-9-take-\textsc{fv} \textsc{aug}-9-monkey \textsc{com}=\textsc{aug}-15-go-\textsc{fv} \textsc{com}=\textsc{ref.9} 16-11-island 16-15-be\_well-\textsc{fv}\\
\glt \lq Crocodile used to come to pick up monkey and go with him to an island to spend time together.' [Crocodile and Monkey] %beispiel schon bei NARR
\end{exe}

This structure is conventionalized with the verb \textit{enda} \lq walk/travel' as the first verb and serves as a marker of sequential events:

\begin{exe}
\ex \gll ʊ-n-hɪɪj-i ʊ-jʊ a-a-longwile n-ky-eni a-lɪnkw-\textbf{end}-a \textbf{n}=\textbf{ʊ}-\textbf{kʊ}-\textbf{kol}-\textbf{a} ʊ-lw-igi\\
\textsc{aug}-1-thief \textsc{aug}-\textsc{prox.1} 1-\textsc{pst}-lead.\textsc{pfv} 18-7-forehead 1-\textsc{narr}-walk/travel-\textsc{fv} \textsc{com}=\textsc{aug}-15-grasp/hold-\textsc{fv} \textsc{aug}-11-door\\
\glt \lq The thief who was going ahead then grabbed the door.' [Wage of the thieves]
\end{exe}

Most commonly, only one verb in a sequence is expressed by the comitative infinitive. In a few cases, however, up to three verbs (\ref{exNuku3verbs}) marked in this manner can be found.
\begin{exe}
\ex \label{exNuku3verbs}\gll a-a-gomok-a ʊ-mw-anike jʊ-la \textbf{n}=\textbf{ʊ}-\textbf{kʊ}-\textbf{fik}-\textbf{a} pa-ka-aja pa-la \textbf{n}=\textbf{ʊ}-\textbf{kʊ}-\textbf{m̩}-\textbf{bʊʊl}-\textbf{a} \textbf{n}=\textbf{ʊ}-\textbf{kʊ}-\textbf{n}-\textbf{nangɪsy}-\textbf{a} ɪ-si ʊ-n̩-dʊme\\
1-\textsc{subsec}-return-\textsc{fv} \textsc{aug}-1-young\_person 1-\textsc{dist} \textsc{com}=\textsc{aug}-15-arrive-\textsc{fv} 16-12-homestead 16-\textsc{dist} \textsc{com}=\textsc{aug}-15-1-tell-\textsc{fv} \textsc{com}=\textsc{aug}-15-1-show-\textsc{fv} \textsc{aug}-\textsc{prox.10} \textsc{aug}-1-husband\\
\glt `That young woman returned and arrived at home and told and showed her husband these things.' [Man and his in-law]
\end{exe}

Other infinitives serve a variety of functions which likely go back to their converb-like use. The infinitive of \textit{tɪ} \lq say' among other things serves as a complementizer; see \sectref{defectiveti}. The reciprocal/associative\is{reciprocal} of \textit{konga} `follow' is used as an infinitive in a preposition-like manner, together with a comitative phrase expressing reason (\ref{exINFkongana}). Similarly the infinitive of \textit{fika} \lq arrive', \textit{fuma} \lq come from', and \textit{anda} \lq start', as well as its \isi{applicative} \textit{andɪla}, are used in a preposition-like manner. Note that in the case of the first two, the original spatial meaning has been extended to a temporal one. (\ref{exINFfuma}, \ref{exINFandila}) illustrate this use for \textit{fuma} and \textit{andɪla}.
\begin{exe}
\ex \label{exINFkongana}\gll nalooli \textbf{ʊ}-\textbf{kʊ}-\textbf{kong}-\textbf{an}-\textbf{a} \textbf{n}=ʊ-lʊ-gano ʊ-lʊ a-a-lɪ na=lo n̩goosi a-lɪnkʊ-jong-a n=ʊ-n-nyambala jʊ-mo\\
really \textsc{aug}-15-follow-\textsc{recp}-\textsc{fv} \textsc{com}=\textsc{aug}-11-love \textsc{aug}-\textsc{prox.11} 1-\textsc{pst}-\textsc{cop} \textsc{com}=\textsc{ref.11} N. 1-\textsc{narr}-run\_away-\textsc{fv} \textsc{com}=\textsc{aug}-1-man 1-one\\
\glt `Because of the love that Ngoosi had, she eloped with a man.' [Man and his in-law]
\ex \label{exINFfuma}
\gll ʊ-mu-ndʊ ʊ-jʊ a-fumwike \textbf{ʊ}-\textbf{kʊ}-\textbf{fum}-\textbf{a} pa-tali\\
\textsc{aug}-1-person \textsc{aug}-\textsc{prox.1} 1-be(come)\_famous.\textsc{pfv} \textsc{aug}-15-come\_from-\textsc{fv} 16-long\\
\glt \lq This person has been famous since long ago.' [ET]
\ex \label{exINFandila}
\gll tw-al-iiswisye ɪ-mi-fuko \textbf{ʊ}-\textbf{kw}-\textbf{and}-\textbf{ɪl}-\textbf{a} n=ʊ-lʊ-bʊnjo mpaka pa-muu-si\\
\textsc{1pl}-\textsc{pst}-be(come)\_full.\textsc{caus.pfv} \textsc{aug}-4-sack \textsc{aug}-15-begin-\textsc{appl}-\textsc{fv} \textsc{com}=\textsc{aug}-11-morning until 16-3-daytime\\
\glt \lq We filled sacks from morning till afternoon.' [ET]
\end{exe}%

Lastly, the infinitive of \textit{anda} `begin, start' is further used as the dependent noun in the associative construction as the ordinal number `first':

\begin{exe}
\ex \gll a-lɪnkw-is-a ʊ-mu-ndʊ ʊ-gw-a kw-and-a\\
1-\textsc{narr}-come-\textsc{fv} \textsc{aug}-1-person \textsc{aug}-1-\textsc{assoc} 15-begin-\textsc{fv}\\
\glt `The first person came.' [Chief Kapyungu]
\end{exe}
\is{infinitive|)}
\chapter{Negation}
\label{bkm:Ref100140375}\label{bkm:Ref98752806}\label{bkm:Ref97905708}\hypertarget{Toc75352706}{}
Negation in Fwe is marked through verbal affixes, auxiliaries, and combinations thereof, depending on the TAM construction. The pre-initial prefix \textit{ka-} (Namibian Fwe) \textit{/ta-} (Zambian Fwe) is used to negate indicative verbs. Fwe also has two post-initial negative suffixes, \textit{ásha-}, used with subjunctive verb forms, and \textit{shá-}, used with infinitive verb forms. A negative final vowel suffix \textit{-i} is seen in certain constructions, but it is never the only marker of negation. Tone also plays a role in negation: the present and stative constructions have different tonal patterns for affirmative and negative forms. \tabref{tab:12:1} gives an overview of the different negative strategies used in Fwe.

\begin{table}
\label{bkm:Ref489278666}\caption{\label{tab:12:1}Negation}

\begin{tabularx}{\textwidth}{llQ}
\lsptoprule
Position & Form & Inflections in which it is used\\
\midrule
pre-initial & \textit{ka-} (Namibian Fwe) & present, near past \\
            & \textit{ta-} (Zambian Fwe) & perfective, stative\\
post-initial & \textit{(á)sha- / (á)sa-} & subjunctive/imperative\\
             & \textit{shá- / sá-}  & infinitive\\
final vowel suffix & {\itshape -i} & present, subjunctive\\
auxiliary & {\itshape aazyá} & stative, fronted-infinitive construction\\
auxiliary & {\itshape ka-/ta-ri} & remote past, future, past progressive, past imperfective, nominal predicates\\
\lspbottomrule
\end{tabularx}
\end{table}
\section{Negation of indicative verb forms}
\label{bkm:Ref490843739}\hypertarget{Toc75352707}{}
Indicative verb forms are negated with a pre-initial prefix \textit{ka-} or \textit{ta-}, and the final vowel suffix \textit{-i}. This is illustrated with the present indicative in (\ref{bkm:Ref99103972}--\ref{bkm:Ref99103973}).

\ea
\label{bkm:Ref99103972}
\glll ndìúrà\\
ndi-ur-á̲\\
\textsc{sm}\textsubscript{1SG}-buy-\textsc{fv}\\
\glt ‘I buy.’
\z

\ea
\glll kàndìúrì\\
ka-ndi-ur-í̲\\
\textsc{neg}-\textsc{sm}\textsubscript{1SG}-buy-\textsc{neg}\\
\glt ‘I don’t buy.’ (NF\_Elic15)
\z

\ea
\label{bkm:Ref99103973}
\glll tàndìúrì\\
ta-ndi-ur-í̲\\
\textsc{neg}-\textsc{sm}\textsubscript{1SG}-buy-\textsc{neg}\\
\glt ‘I don’t buy.’ (ZF\_Elic14)
\z

Present tense verbs also change their tone pattern when negated. Affirmative present verbs take MT 1 and 4 (see \sectref{bkm:Ref71539267}), but negated present verbs take only MT 3. The tonal difference between the affirmative and negative present is illustrated in (\ref{bkm:Ref74909280}).

\ea
\label{bkm:Ref74909280}
kàndìzíbârì (cf. ndìzìbárà ‘I forget’)\\
\gll ka-ndi-zibá̲r-i\\
\textsc{neg}-\textsc{sm}\textsubscript{1SG}-forget-\textsc{neg}\\
\glt ‘I don’t forget.’ (NF\_Elic15)
\z

The negative suffix \textit{-i} cannot be directly preceded by a passive suffix \textit{-(i)w}. When a passive verb is negated, the negative suffix \textit{-i} is not used, but rather the default final vowel suffix \textit{-a}, as in (\ref{bkm:Ref505773692}). However, when the passive suffix -\textit{(i)w} is separated from the final vowel by the occurrence of the habitual suffix \textit{-ang}, the negative suffix \textit{-i} is used, as in (\ref{bkm:Ref505773694}). Incompatibility with the passive suffix is also observed for the near past perfective suffix \textit{\--i} (see \sectref{bkm:Ref488767671}).\footnote{There are also other cases of overlap between the near past perfective and the negative present tense form. Both forms use a suffix \textit{\-–i}, neither of which ever causes spirantization (as opposed to certain other suffixes with /i/, where spirantization is attested in lexicalized cases). Both forms use melodic tone 3, which is assigned to the second stem syllable. In spite of these formal similarities, however, there is little semantic overlap between the negative and near past perfective meanings.}

\ea
\label{bkm:Ref505773692}
\glll kàcìhîkwà\\
ka-ci-hík-w-a\\
\textsc{neg}-\textsc{sm}\textsubscript{7}-cook-\textsc{pass}-\textsc{fv}\\
\glt ‘It cannot be cooked.’ (NF\_Elic15)
\z

\ea
\label{bkm:Ref505773694}
\glll báshàshéshíwàngì\\
ba-ásha-shesh-í̲w-ang-i\\
\textsc{sm}\textsubscript{2}-\textsc{neg}-marry-\textsc{pass}-\textsc{hab}-\textsc{neg}\\
\glt ‘They should not be married.’ (ZF\_Conv13)
\z

Of the two forms of the negative prefix, \textit{ka-} is mainly used in Namibian Fwe, and \textit{ta-} in Zambian Fwe. This areal distribution is also seen in several other Bantu languages of the region, including those of the Bantu Botatwe subgroup, such as Totela and Subiya, but also Yeyi, not part of Bantu Botatwe. Totela, which, like Fwe, has a Zambian and a Namibian variety, exhibits the same distribution as Fwe; \textit{ta-} is used in the Zambian variety \citep[82]{Crane2011}, and \textit{ka-} in the Namibian variety. Subiya and Yeyi, only spoken in Namibia, both only use \textit{ka-} (\citealt{Jacottet1896}: 57-58; \citealt{Seidel2008}: 405-408). The distribution of the \textit{ka-} and \textit{ta-} forms of the negative prefix thus more or less follows the national border between Zambia and Namibia.

The negative prefix \textit{ta}-/\textit{ka}- is placed directly before the subject marker of the verb. When the subject marker consists of a vowel only, vowel hiatus resolution takes place between the vowel of the negative prefix and the vowel of the subject marker. Aside from subject markers affected by predictable rules of vowel hiatus resolution, there are no special forms of subject markers used exclusively with negative verbs, as opposed to a tendency often observed in Bantu languages for subject markers of the first person singular to have a special negated form: the negated form of the first person singular is a morphologically regular combination of the negative prefix with the first person singular subject marker \textit{ndi-}, as in (\ref{bkm:Ref74909320}).

\ea
\label{bkm:Ref74909320}
tàndìbútùkì (cf. ndìbùtúkà ‘I run’)\\
\gll ta-ndi-bú̲tuk-i\\
\textsc{neg}-\textsc{sm}\textsubscript{1SG}-run-\textsc{neg}\\
\glt ‘I don’t run.’ (ZF\_Elic14)
\z

The prefix \textit{ka-/ta-} is also used to negate the near past perfective. This tense uses a past suffix \textit{-i} which is homophonous with the negative suffix \textit{-i}. Negated verbs in the near past perfective have the same tonal pattern as their affirmative counterparts, as illustrated in (\ref{bkm:Ref99104398}).

\ea
\label{bkm:Ref99104398}
kàndàzíbònì (cf. ndàzíbònì ‘I’ve seen them’)\\
\gll ka-ndi-a-zí-bon-i\\
\textsc{neg}-\textsc{sm}\textsubscript{1SG}-\textsc{pst}-\textsc{om}\textsubscript{10}-see-\textsc{npst}.\textsc{pfv}\\
\glt ‘I haven’t seen them.’ (NF\_Elic15)
\z

Verbs in the stative construction are also negated with the prefix \textit{ka-/ta-}, combined with lengthening of the last vowel of the verb, which is not seen in the affirmative stative. This can be seen as influence from the negative suffix \textit{-i}, which contributes an extra mora to the last vowel of the verb, but its vowel quality merges with the last vowel of the verb (/e/ or /i/, depending on the allomorph of the stative suffix, see \sectref{bkm:Ref431984198}). The length difference in the last vowel of affirmative and negative stative verbs is illustrated in (\ref{bkm:Ref75180218}--\ref{bkm:Ref75180219}).

\ea
\label{bkm:Ref75180218}
kàìbòrètêː (cf. ìbórêtè ‘it is rotten’)\\
\gll ka-i-bor-ete-í̲\\
\textsc{neg}-\textsc{sm}\textsubscript{1SG}-rot-\textsc{stat}-\textsc{neg}\\
\glt ‘It is not rotten.’ (ZF\_Elic14)
\z

\ea
\label{bkm:Ref75180219}
kàndìyìzyîː (cf. ndìyìzyì ‘I know it’)\\
\gll ka-ndi-i\textsubscript{H}-zyi-í̲\\
\textsc{neg}-\textsc{sm}\textsubscript{1SG}-\textsc{om}\textsubscript{9}-know.\textsc{stat}-\textsc{neg}\\
\glt ‘I don’t know it.’ (NF\_Elic15)
\z

The negation of stative verbs also involves a change in tone pattern. Affirmative stative verbs take a high tone on the second stem syllable (MT 3, see \sectref{bkm:Ref71540417}). Negated stative verbs take a high tone on the last mora of the verb (MT 1, see \sectref{bkm:Ref71540433}). The deletion of the lexical tone of the root, as seen in the affirmative stative, also affects the negated stative. Optional high tone spread, i.e. the copying of high tones up to the first syllable of the verb stem, is never seen in negated stative verbs, though it is very common in affirmative stative verbs. The different tone patterns of affirmative and negated stative verbs are illustrated in (\ref{bkm:Ref98833526}--\ref{bkm:Ref98833527}).

\ea
\label{bkm:Ref98833526}
tàndìshèshètêː (cf. ndìshéshêtè ‘I am married’)\\
\gll ta-ndi-she\textsubscript{H}sh-ete-í̲\\
\textsc{neg}-\textsc{sm}\textsubscript{1SG}-marry-\textsc{stat}-\textsc{neg}\\
\glt ‘I am not married.’
\z

\ea
\label{bkm:Ref98833527}
tàtùkàtìtêː (cf. tùkátîtè ‘we are thin’)\\
\gll ta-tu-kat-ite-í̲\\
\textsc{neg}-\textsc{sm}\textsubscript{1PL\-}-become\_thin-\textsc{stat}-\textsc{neg}\\
\glt ‘We are not thin.’ (ZF\_Elic14)
\z
\section{Negation of imperative and subjunctive verb forms}
\label{bkm:Ref499029715}\hypertarget{Toc75352708}{}
Imperative and subjunctive verb forms are negated with a post-initial prefix \mbox{\textit{ásha-},} combined with the negative suffix \textit{-i}, as in (\ref{bkm:Ref99104532}--\ref{bkm:Ref99104535}). In Namibian Fwe, the prefix has a free variant \textit{ása-}, as in (\ref{bkm:Ref99104547}) (see \sectref{bkm:Ref70695065} on the free variation between /s/ and /sh/ in grammatical prefixes).

\ea
\label{bkm:Ref99104532}
wáshàyáshàmì òkìmúmé bùryò\\
\gll o-ásha-yásham-i        o-ki\textsubscript{H}-mum-é̲    bu-ryo\\
\textsc{sm}\textsubscript{2SG}-\textsc{neg}.\textsc{sbjv}-open\_mouth-\textsc{neg}  \textsc{sm}\textsubscript{2SG}-\textsc{refl}-close-\textsc{pfv}.\textsc{sbjv}  \textsc{np}\textsubscript{14}-only\\
\glt ‘Don’t open your mouth, just close it like that.’ (ZF\_Narr13)
\z

\ea
mwáshàbútùkì câhà\\
\gll mu-ásha-bútuk-i    cáha\\
\textsc{sm}\textsubscript{2PL}-\textsc{neg}.\textsc{sbjv}-run-\textsc{neg}  very\\
\glt ‘Don’t go too fast.’ (NF\_Elic17)
\z

\ea
\label{bkm:Ref99104535}
ndìryá bùryó kànînì òkùtêyè ndáshànúnì\\
\gll ndi-ry-á  bu-ryó  ka-níni okutéye  ndi-ásha-nun-í̲ \\
\textsc{sm}\textsubscript{1SG}-eat-\textsc{fv}  \textsc{np}\textsubscript{14}-only  \textsc{adv}-little
that    \textsc{sm}\textsubscript{1SG}-\textsc{neg}.\textsc{sbjv}-become\_fat-\textsc{neg}\\
\glt ‘I only eat a little, so that I do not become fat.’ (NF\_Elic17)
\z

\ea
\label{bkm:Ref99104547}
kònó náàryá òkùtêyè ásàrémùhì\\
\gll konó  ná̲-a-a-ry-á      okutéye  á-sa-rémuh-i\\
but  \textsc{rem}-\textsc{sm}\textsubscript{1}-\textsc{pst}-eat-\textsc{fv}  that    \textsc{sm}\textsubscript{1}-\textsc{neg}.\textsc{sbjv}-find\_out-\textsc{neg}\\
\glt ‘But she ate, so that he wouldn’t find out.’ (NF\_Narr17)
\z

The negative subjunctive/imperative prefix may be realized as \textit{ásha-/ása-} or \textit{sha}-/\textit{sa-}. When the first vowel /a/ is dropped, the high tone of the suffix is realized on the subject marker, as in (\ref{bkm:Ref99104565}).

\ea
\label{bkm:Ref99104565}
\glll músàndìtáfùnì\\
mú-sa-ndi-táfun-i\\
\textsc{sm}\textsubscript{2PL}-\textsc{neg}.\textsc{sbjv}-\textsc{om}\textsubscript{1SG}-chew-\textsc{neg}\\
\glt ‘Don’t eat me!’ (NF\_Narr17)
\z
\section{Negation of infinitive verb forms}
\hypertarget{Toc75352709}{}
Infinitive verb forms are negated with a post-initial prefix \textit{shá-}, as in (\ref{bkm:Ref99104688}--\ref{bkm:Ref99104690}). In Namibian Fwe, the prefix \textit{shá-} has a free variant \textit{sá-}, as in (\ref{bkm:Ref99104701}) (/s/ and /sh/ are interchangeable in grammatical prefixes; see \sectref{bkm:Ref70695065}).

\ea
\label{bkm:Ref99104688}
kùshátèèzà mbùkáꜝbábù\\
\gll ku-shá-teez-a    N-bu-kábabú\\
\textsc{inf}-\textsc{neg}.\textsc{inf}-listen-\textsc{fv}  \textsc{cop}-\textsc{np}\textsubscript{14}-problem\\
\glt ‘Not listening is a problem.’ (NF\_Elic17)
\z

\ea
\label{bkm:Ref99104690}
nàngá mwínàkò yóbùkòbà mbàngíː bànàdàmwá kókùsházyìbà òkùbàrà ècìpùrá nècìŋòrétwà ìyé cámàkúwà èwé mpàhó àkèːzyà kúkàrà nòrì mùntù ókùsìhà\\
\gll nangá  mú-e-N-nako  i-ó=bu-koba      N-ba-ngíː ba-na-dam-w-á̲    \textbf{kó-ku-shá-zyib-a}      o-ku-bar-a  e-ci-purá    ne-ci-ŋo\textsubscript{H}r-é̲twa    iyé  ci-á-ma-kuwá ewe    N-pa-hó    a-keːzy-a    kú-kar-a na=o-ri    mu-ntu  u-ó=ku-sih-a  \\
even  \textsc{np}\textsubscript{18}-\textsc{aug}-\textsc{np}\textsubscript{9}-time  \textsc{pp}\textsubscript{9}-\textsc{con}=\textsc{np}\textsubscript{14}-apartheid  \textsc{cop}-\textsc{pp}\textsubscript{2}-many
\textsc{sm}\textsubscript{2}-\textsc{pst}-beat-\textsc{pass}-\textsc{fv}  \textsc{adv}-\textsc{inf}-\textsc{neg}.\textsc{inf}-know-\textsc{fv}  \textsc{aug}-\textsc{inf}-read-\textsc{fv}
\textsc{aug}-\textsc{np}\textsubscript{7}-chair  \textsc{rem}-\textsc{sm}\textsubscript{7}-write-\textsc{stat}-\textsc{pass}  that  \textsc{pp}\textsubscript{7}-\textsc{con}=\textsc{np}\textsubscript{6}-white
\textsc{pers}\textsubscript{2SG}  \textsc{cop}-\textsc{np}\textsubscript{16}-\textsc{dem}.\textsc{iii}\textsubscript{16}  \textsc{sm}\textsubscript{1}-come-\textsc{fv}    \textsc{inf}-sit-\textsc{fv}
\textsc{com}=\textsc{sm}\textsubscript{2SG}-be  \textsc{np}\textsubscript{1}-person  \textsc{pp}\textsubscript{1}-\textsc{con}=\textsc{inf}-be\_black-\textsc{fv}\\
\glt ‘Even in the time of apartheid, many were beaten because of \textbf{not} \textbf{knowing} how to read. On a bench, it is written, whites only. You, that is where you sit, when you are a black person.’ (NF\_Song17)
\z

\ea
\label{bkm:Ref99104701}
\glll kùshábònà {\textasciitilde} kùsábònà\\
ku-shá-bon-a\\
\textsc{inf}-\textsc{neg}.\textsc{inf}-see-\textsc{fv}\\
\glt ‘to not see’
\z
\section{Negation with auxiliaries}
\label{bkm:Ref494466580}\hypertarget{Toc75352710}{}
All other verbal constructions are negated with the use of an auxiliary \textit{ri} ‘be’ or \textit{aazyá} ‘be not’, or a lexical verb \textit{síy} ‘stop, leave’. Negation with \textit{ri} ‘be’ involves the negative prefix \textit{ka}-/\textit{ta-} marked on the auxiliary, followed by the inflected lexical verb, which takes a high tone on the subject marker, showing that it is a relative verb (see \sectref{bkm:Ref490828195} on the formal properties of relative clause verbs). This construction is used to negate the remote past perfective, as in (\ref{bkm:Ref99104743}), the remote past imperfective, as in (\ref{bkm:Ref99104763}), and the near past imperfective, as in (\ref{bkm:Ref489284413}).

\newpage
\ea
\label{bkm:Ref99104743}
kàrì ndáyìbònà\\
\gll ka-ri    ndi-á̲-i-bon-a\\
\textsc{neg}-be  \textsc{sm}\textsubscript{1SG}-\textsc{pst}-\textsc{om}\textsubscript{9}-see-\textsc{fv}\\
\glt ‘I did not see it.’ (NF\_Elic15)
\z

\ea
\label{bkm:Ref99104763}
kàrì kátòmbwèr’ éꜝsózù\\
\gll ka-ri    ka-á̲-tombwer-á̲    e-∅-sozú\\
\textsc{neg}-be  \textsc{pst}.\textsc{ipfv}-\textsc{sm}\textsubscript{1}-weed-\textsc{fv}  \textsc{aug}-\textsc{np}\textsubscript{5}-grass\\
\glt ‘He was not weeding grass.’ (NF\_Narr15)
\z

\ea
\label{bkm:Ref489284413}
kàrì ndákùhîkà\\
\gll ka-ri    ndí̲-aku-hík-a\\
\textsc{neg}-be  \textsc{sm}\textsubscript{1SG}-\textsc{npst}.\textsc{ipfv}-cook-\textsc{fv}\\
\glt ‘I was not cooking.’ (NF\_Elic17)
\z

The auxiliary \textit{ri} ‘be’ with a negative prefix is also used to negate nominal predicates. Affirmative nominal predicates are marked by a copulative prefix only (see \sectref{bkm:Ref489963307}). When negated with the auxiliary \textit{ri}, the copulative prefix is maintained, as in (\ref{bkm:Ref99104807}--\ref{bkm:Ref99104808}).

\ea
\label{bkm:Ref99104807}
mbùrôtù kònó \textbf{kàrí} \textbf{mbùrótù} nênjà\\
\gll N-bu-rótu    konó  ka-rí    N-bu-rótu    nénja\\
\textsc{cop}-\textsc{np}\textsubscript{14}-good  but  \textsc{neg}-be  \textsc{cop}-\textsc{np}\textsubscript{14}-good  well\\
\glt ‘It is good, but it is not very good.’ (ZF\_Conv13)
\z

\ea
\label{bkm:Ref99104808}
òwú \textbf{kàrí} \textbf{ꜝ}\textbf{ngómùnzí} ꜝwángù\\
\gll o-ú    ka-rí    ngó-mu-nzí      u-angú\\
\textsc{aug}-\textsc{dem}.\textsc{i}\textsubscript{3}  \textsc{neg}-be  \textsc{cop}.\textsc{def}\textsubscript{3}-\textsc{np}\textsubscript{3}-village  \textsc{pp}\textsubscript{3}-\textsc{poss}\textsubscript{1SG}\\
\glt ‘This is not my village.’ (ZF\_Elic13)
\z

To express a negative future, the auxiliary \textit{ri} ‘be’ is used, marked with the negative prefix \textit{ka-/ta-}, followed by a subjunctive verb. To indicate a more remote future, the subjunctive verb takes a remoteness prefix \textit{na-/ne-}, as used in (\ref{bkm:Ref471225116}--\ref{bkm:Ref471225117}). To express a near future, the remoteness prefix is omitted, as in (\ref{bkm:Ref471225384}--\ref{bkm:Ref489284739}).

\ea
\label{bkm:Ref471225116}
rímwì zyûbà kàrì nèmúbûːꜝké nwè\\
\gll rí-mwi  ∅-zyúba ka-ri    ne-mú̲-bú̲ːk-e      enwé \\
\textsc{pp}\textsubscript{5}-other  \textsc{np}\textsubscript{5}-day
\textsc{neg}-be  \textsc{rem}-\textsc{sm}\textsubscript{2PL}-wake.\textsc{intr}-\textsc{pfv}.\textsc{sbjv}  \textsc{pers}\textsubscript{2PL}\\
\glt ‘One day, you are not going to wake up.’ (NF\_Narr15)
\z

\ea
\label{bkm:Ref471225117}
kàrì nándìsépè\\
\gll ka-ri    na-á̲-ndi-sep-é̲\\
\textsc{neg}-be  \textsc{rem}-\textsc{sm}\textsubscript{1}-\textsc{om}\textsubscript{1SG}-trust-\textsc{pfv}.\textsc{sbjv}\\
\glt ‘He will not trust me.’ (ZF\_Conv13)
\z

\ea
\label{bkm:Ref471225384}
kàrì ndífìyérè\\
\gll ka-ri    ndí̲-fi\textsubscript{H}yer-é̲\\
\textsc{neg}-be  \textsc{sm}\textsubscript{1SG}-sweep-\textsc{pfv}.\textsc{sbjv}\\
\glt ‘I will not sweep.’ (ZF\_Elic13)
\z

\ea
\label{bkm:Ref489284739}
kàrì ndícìpángè shûnù\\
\gll ka-ri    ndí̲-ci\textsubscript{H}-pá̲ng-e    shúnu\\
\textsc{neg}-be  \textsc{sm}\textsubscript{1SG}-\textsc{om}\textsubscript{7}-do-\textsc{pfv}.\textsc{sbjv}  today\\
\glt ‘I will not do it today.’ (NF\_Elic17)
\z

\begin{sloppypar}
The auxiliary \textit{aazyá} ‘be/have not’ is also used to negate the verb \textit{iná} ‘be at/have’, as in (\ref{bkm:Ref489285426}--\ref{bkm:Ref99104890}).
\end{sloppypar}

\ea
\label{bkm:Ref489285426}
kwìn’ écò ndíbwènè\\
\gll ku-iná    e-co    ndí̲-bwe\textsubscript{H}ne\\
\textsc{sm}\textsubscript{17}-be\_at  \textsc{aug}-\textsc{dem}.\textsc{iii}\textsubscript{7}  \textsc{sm}\textsubscript{1SG}.\textsc{rel}-see.\textsc{stat}\\
\glt ‘There is something that I see. / I see something.’
\z

\ea
\label{bkm:Ref99104890}
kùààzy’ écò ndíbwènè\\
\gll ku-aazyá  e-co    ndí̲-bwe\textsubscript{H}ne\\
\textsc{sm}\textsubscript{17}-be\_not  \textsc{aug}-\textsc{dem}.\textsc{iii}\textsubscript{7}  \textsc{sm}\textsubscript{1SG}.\textsc{rel}-see.\textsc{stat}\\
\glt ‘There is not something that I see. / I see nothing.’ (NF\_Elic15)
\z

Where the auxiliary \textit{iná} with a locative subject marker is used to express ‘something’, ‘someone’, or ‘somewhere’, its negated counterpart \textit{aazyá} is used to express ‘nothing’, ‘no one’, or ‘nowhere’. Subject markers of all three locative classes can be used with the verb \textit{aazyá}, e.g. class 16, as in (\ref{bkm:Ref494561456}), class 17, as in (\ref{bkm:Ref494561459}--\ref{bkm:Ref471307062}), and class 18, as in (\ref{bkm:Ref494561466}).

\ea
\label{bkm:Ref494561456}
ákèːzyà kùwànà ìyé hààzyá bàntù\\
\gll á̲-keːzy-a    ku-wan-a  iyé  ha-aazyá  ba-ntu\\
\textsc{sm}\textsubscript{1}.\textsc{rel}-come-\textsc{fv}  \textsc{inf}-find-\textsc{fv}  that  \textsc{sm}\textsubscript{16}-be\_not  \textsc{np}\textsubscript{2}-person\\
\glt ‘When he came to find that there were no people there…’ (NF\_Narr15)
\z

\ea
\label{bkm:Ref494561459}
kwààzyá mùntù\\
\gll ku-aazyá  mu-ntu\\
\textsc{sm}\textsubscript{17}-be\_not  \textsc{np}\textsubscript{1}-person\\
\glt ‘There is no one.’ (ZF\_Elic13)
\z

\ea
\label{bkm:Ref471307062}
kwàázyó kò nìbáwânè ménò\\
\gll ku-aazyá  o-kó      ni-bá̲-wá̲n-e      ma-inó\\
\textsc{sm}\textsubscript{17}-be\_not  \textsc{aug}-\textsc{dem}.\textsc{iii}\textsubscript{17}  \textsc{rem}-\textsc{sm}\textsubscript{2}-find-\textsc{pfv}.\textsc{sbjv}  \textsc{np}\textsubscript{6}-tooth\\
\glt ‘There’s nowhere where he can get the teeth.’ (NF\_Narr15)
\z

\ea
\label{bkm:Ref494561466}
òbú bùsùnsò mwáázyé zwàyì\\
\gll o-bú    bu-sunso  mu-aazyá  e-zwai\\
\textsc{aug}-\textsc{dem}.\textsc{i}\textsubscript{14}  \textsc{np}\textsubscript{14}-relish  \textsc{sm}\textsubscript{18}-be\_not  \textsc{aug}-salt\\
\glt ‘This relish, there is no salt in it.’ (ZF\_Elic14)
\z

The auxiliary \textit{aazyá} can also be used to negate a fronted infinitive construction. The fronted infinitive construction, which consists of an inflected verb preceded by an infinitive copy of the same verb stem (see \sectref{bkm:Ref431917326}), is illustrated in (\ref{bkm:Ref99105077}). It cannot be negated through the prefix \textit{ta}-/\textit{ka}- and the suffix -\textit{i}, as shown by the ungrammaticality of (\ref{bkm:Ref99105098}). Instead a construction is used with the negative \textit{aazyá} inflected for subject agreement, followed by the lexical verb in the infinitive, as in (\ref{bkm:Ref99105118}).

\ea
\label{bkm:Ref99105077}
kùhòndà ndíꜝhóndà\\
\gll ku-hond-a  ndí̲-hó̲nd-a\\
\textsc{inf}-cook-\textsc{fv}  \textsc{sm}\textsubscript{1SG}.\textsc{rel}-cook-\textsc{fv}\\
\glt ‘I am cooking.’
\z

\ea
\label{bkm:Ref99105098}
*kùhòndà tàndíꜝhóndì (ZF\_Elic14)
\z

\ea
\label{bkm:Ref99105118}
ndààzyá kùhòndà\\
\gll ndi-aazyá  ku-hond-a\\
\textsc{sm}\textsubscript{1SG}-be\_not  \textsc{inf}-cook-\textsc{fv}\\
\glt ‘I am not cooking.’
\z

\textit{aazyá} is also occasionally used to negate verbs that may also be negated with a prefix \textit{ka-/ta-} or an auxiliary \textit{ri} ‘be’. This is the case for verbs with a reduplicated stem, as in (\ref{bkm:Ref99105150}), which may be negated with a prefix \textit{ka-/ta-} and a suffix \textit{-i} in the present tense, as in (\ref{bkm:Ref99105160}), but most speakers prefer to use the auxiliary \textit{aazyá} followed by the reduplicated verb in the infinitive form, as in (\ref{bkm:Ref99105170}).

\ea
\label{bkm:Ref99105150}
\glll ndìtóːrátôːrà\\
ndi-toːra-tó̲ːr-a\\
\textsc{sm}\textsubscript{1SG}-\textsc{pl}2-pick-\textsc{fv}\\
\glt ‘I pick.’
\z

\ea
\label{bkm:Ref99105160}
\glll kàndìtóːrìtòːrì\\
ka-ndi-tó̲ːri-toːr-i\\
\textsc{neg}-\textsc{sm}\textsubscript{1SG}-\textsc{pl}2-pick-\textsc{neg}\\
\glt ‘I don’t pick.’
\z

\ea
\label{bkm:Ref99105170}
ndààzy’ ókùtóːràtòːrà\\
\gll ndi-aazyá  o-ku-tóːra-toːr-a\\
\textsc{sm}\textsubscript{1SG\-}-be\_not  \textsc{aug}-\textsc{inf}-\textsc{pl}2-pick-\textsc{fv}\\
\glt ‘I don’t pick.’ (NF\_Elic15)
\z

\textit{aazyá} is also used to negate verbs expressing states, either verbs in the stative construction, as in (\ref{bkm:Ref99105195}--\ref{bkm:Ref99105196}), or true stative verbs, as in (\ref{bkm:Ref99105197}). As shown in \sectref{bkm:Ref490843739}, stative verbs can also be negated with affixes on the verb. A meaning difference between periphrastic and morphological negation of stative verbs has not been observed.

\ea
\label{bkm:Ref99105195}
ècìyângò cààzyá kùbórêtè\\
\gll e-ci-ángo    ci-aazyá  ku-bor-é̲te\\
\textsc{aug}-\textsc{np}\textsubscript{7}-fruit  \textsc{sm}\textsubscript{7}-be\_not  \textsc{inf}-rot-\textsc{stat}\\
\glt ‘The fruit is not rotten.’ (ZF\_Elic14)
\z

\ea
\label{bkm:Ref99105196}
cààzy’ ókùhárîtwà\\
\gll ci-aazyá  o-ku-ar-í̲t-w-a\\
\textsc{sm}\textsubscript{7}-be\_not  \textsc{aug}-\textsc{inf}-close-\textsc{stat}-\textsc{pass}-\textsc{fv}\\
\glt ‘It is not closed.’ (NF\_Elic15)
\z

\ea
\label{bkm:Ref99105197}
ndàázyá kùshàkà kùrìhà òmùrándù\\
\gll ndi-aazyá  ku-shak-a  ku-rih-a  o-mu-randú\\
\textsc{sm}\textsubscript{1SG}-be\_not  \textsc{inf}-want-\textsc{fv}  \textsc{inf}-pay-\textsc{fv}  \textsc{aug}-\textsc{np}\textsubscript{3}-fine\\
\glt ‘I don’t want to pay a fine.’ (NF\_Elic15)
\z

The lexical verb \textit{síy} ‘leave, let go, stop’, is used in the imperative form and followed by an infinitive to express a prohibitive, as in (\ref{bkm:Ref471982528}--\ref{bkm:Ref492060835}).

\ea
\label{bkm:Ref471982528}
sìy’ ókùndìkwâtà\\
\gll si\textsubscript{H}-é̲      o-ku-ndi-kwát-a\\
stop-\textsc{pfv}.\textsc{sbjv}  \textsc{aug}-\textsc{inf}-\textsc{om}\textsubscript{1SG}-grab-\textsc{fv}\\
\glt ‘Don’t touch me.’ (NF\_Elic15)
\z

\ea
\label{bkm:Ref492060835}
òsìyé kùyángà kwìnà\\
\gll o-si\textsubscript{H}-é̲      ku-yá-ang-a    kwina\\
\textsc{sm}\textsubscript{2SG}-leave-\textsc{pfv}.\textsc{sbjv}  \textsc{inf}-go-\textsc{hab}-\textsc{fv}  \textsc{dem}.\textsc{iv}\textsubscript{17}\\
\glt ‘Never go there.’ (NF\_Elic17)
\z


\documentclass[output=paper,colorlinks,citecolor=brown,draftmode]{langscibook}
\ChapterDOI{10.5281/zenodo.7525116}
\author{Carolina González\affiliation{Florida State University} and Lara Reglero\affiliation{Florida State University}}
\title{Prosodic correlates of mirative and new information focus in Spanish wh-in-situ questions}
\abstract{This paper examines the prosodic correlates of focus in two types of wh-in-situ questions in Spanish: information-seeking (INF), and echo-surprise (SUR). We hypothesize that they will have different intonational properties since the former are associated with new-information focus, while the latter are compatible with mirative focus since they express unexpectedness and surprise \citep{BadanCrocco2019}. A total of 280 sentences from a contextualized elicitation task were analyzed in Praat following SpToBI conventions. Results show that INF and SUR have similar melodic contours, involving a rise through the first pre-nuclear accent, declination, and a steep final rise on the wh-phrase. However, SUR questions have a higher nuclear peak and larger focal tonal range than INF questions. Our results show clear scaling differences in the nuclear configuration consistent with a difference between new-information and mirative focus, which can be phonologically analyzed as nuclear upstep in SUR (L+¡H*), unlike in INF (L+H*).}

% %move the following commands to the "local..." files of the master project when integrating this chapter
% \usepackage{tabularx}
% \usepackage{langsci-basic}
% \usepackage{langsci-optional}
% \usepackage{langsci-gb4e}
% \usepackage{multirow}
% \bibliography{localbibliography}
% %\newcommand{\orcid}[1]{}
% \pagenumbering{arabic}
% \setcounter{page}{277}

% \usepackage{etoolbox}
% \makeatletter
% \patchcmd{\@footnotetext}{\setcounter{fnx}{0}}{\renewcommand{\thexnumi}{\roman{xnumi}}}{}{}
% \apptocmd{\@footnotetext}{
%     \@noftnotetrue
%     \renewcommand{\thexnumi}{\arabic{xnumi}}%
% }{}{}

\IfFileExists{../localcommands.tex}{
  \addbibresource{../localbibliography.bib}
  \usepackage{langsci-optional}
\usepackage{langsci-gb4e}
\usepackage{langsci-lgr}

\usepackage{listings}
\lstset{basicstyle=\ttfamily,tabsize=2,breaklines=true}

%added by author
% \usepackage{tipa}
\usepackage{multirow}
\graphicspath{{figures/}}
\usepackage{langsci-branding}

  
\newcommand{\sent}{\enumsentence}
\newcommand{\sents}{\eenumsentence}
\let\citeasnoun\citet

\renewcommand{\lsCoverTitleFont}[1]{\sffamily\addfontfeatures{Scale=MatchUppercase}\fontsize{44pt}{16mm}\selectfont #1}
  
  %% hyphenation points for line breaks
%% Normally, automatic hyphenation in LaTeX is very good
%% If a word is mis-hyphenated, add it to this file
%%
%% add information to TeX file before \begin{document} with:
%% %% hyphenation points for line breaks
%% Normally, automatic hyphenation in LaTeX is very good
%% If a word is mis-hyphenated, add it to this file
%%
%% add information to TeX file before \begin{document} with:
%% \include{localhyphenation}
\hyphenation{
affri-ca-te
affri-ca-tes
an-no-tated
com-ple-ments
com-po-si-tio-na-li-ty
non-com-po-si-tio-na-li-ty
Gon-zá-lez
out-side
Ri-chárd
se-man-tics
STREU-SLE
Tie-de-mann
}
\hyphenation{
affri-ca-te
affri-ca-tes
an-no-tated
com-ple-ments
com-po-si-tio-na-li-ty
non-com-po-si-tio-na-li-ty
Gon-zá-lez
out-side
Ri-chárd
se-man-tics
STREU-SLE
Tie-de-mann
}
  % \togglepaper[3]%%chapternumber
}{}


\shorttitlerunninghead{Prosodic correlates of mirative and new information focus in wh-in-situ}
\begin{document}
\shorttitlerunninghead{Prosodic correlates of mirative and new information focus in wh-in-situ}
\maketitle

\section{Introduction}\label{sec:13:1}
This study compares the prosodic correlates of focus in two types of Spanish wh-in-situ questions: Information-seeking (INF), and echo-surprise (SUR). While the main strategy to formulate a wh-question in Spanish involves wh-fronting (\ref{13:ex:1a}), wh-in-situ questions are also possible in some dialects, such as in North-Central Peninsular Spanish (\ref{13:ex:1b}) (\citealp{Jiménez1997,Uribe-Etxebarria2002,EtxepareUribe-Etxebarria2005,Reglero2007,RegleroTicio2013}, among others).

\ea \label{13:ex:1}
\ea \gll \label{13:ex:1a}¿\textbf{Qué}  llevó               Rosalía?\\
what  wear.\textsc{pst}.3\textsc{sg} Rosalía\\
\glt ‘What did Rosalía wear?’\\
\ex \label{13:ex:1b}
¿Rosalía llevó \textbf{qué}?
\z
\z

The pragmatic meanings of wh-in-situ questions in Spanish are varied. A sentence such as (\ref{13:ex:1b}) can be interpreted as an information-seeking (INF) question eliciting information in a neutral way \citep{Reglero2007, RegleroTicio2013}. Alternatively, (\ref{13:ex:1b}) can be interpreted as an echo question, i.e., a question requesting repetition of information (echo-repetition, henceforth REP) or conveying surprise (echo-surprise, henceforth SUR) \citep{Chernova2013, Chernova2017, RegleroTicio2013}. Regardless of the pragmatic reading, the in-situ wh-element carries the main focus of the question \citep{Horvath1986, Rochemont1986, Tuller1992, Zubizarreta1998,escandellvidal1999}.

In this study, we follow \citet{Reglero2007} and \citet{RegleroTicio2013} in considering INF questions as having new information focus; and we argue, based on \citet{BadanCrocco2019}, that SUR questions in Spanish have mirative focus, which conveys counter-expectational value. Spanish INF and SUR questions display some syntactic differences, including differences in word order. In addition, impressionistic reports and a previous small-scale study suggest some intonational differences as well \citep{gonzalez2018dime}. In the present study, we investigate the prosodic characteristics of INF and SUR within a larger set of speakers, and connect these differences to focus, taking into consideration relevant studies from other Romance languages.

\largerpage
Our study is framed within the Auto-Segmental (AM) model of intonation \citep{pierrehumbert1980,PierrehumbertBeckman1988,Ladd2008}, which views intonation as the anchoring of High (H) and Low (L) tones to metrically strong syllables and edges of phonological domains. We follow the conventions of the Spanish ToBI prosodic annotation system \citep{BeckmanMorgan2002,Estebas-VilaplanaPrieto,PrietoRoseano2010,hualde2015}. Stressed syllables bear pitch accents, indicated with *. The pitch accent on the last main stress of an utterance is the nuclear pitch; other stressed syllables bear prenuclear accents (unless deaccented). Edges of phonological domains bear boundary tones. In Spanish, boundary tones occur at the end of full intonational phrases (IPs) and intermediate phrases (ips); these are indicated with \% and -, respectively \citep{AguilarPrieto2009}. Figure 1 below provides an example of prosodic annotation for a statement with narrow focus on the direct object. The final IP boundary is low (L\%); the intermediate ip shows a steep rise (HH-). All pitch accents are rising; but while the nuclear peak is aligned with the stressed syllable (L+H*), prenuclear peaks are delayed, i.e., aligned with the post-tonic syllable (L+>H*).


\begin{figure}
    \includegraphics[scale=.15]{figures/GR_FIG1.png}
    \caption{Example of Spanish ToBI annotation. Participant 15. \textit{El niño mira a su abuelo}  ‘The child looks at his grandfather’ (narrow focus)}
    \label{13:Fig1}
\end{figure}

The rest of this paper is organized as follows. \sectref{sec:13:2} contextualizes the study in connection to focus and reviews its main syntactic and prosodic characteristics. \sectref{sec:13:3} introduces the methodology of the study. \sectref{sec:13:4} presents the results, and \sectref{sec:13:5} is the discussion. Concluding remarks are provided in \sectref{sec:13:6}.
\section{Properties of focus}\label{sec:13:2}
\subsection{Focus types}
Focus, or the information center of a sentence \citep{chomsky1971,Chomsky1976}, is expressed cross-linguistically in one or more of three ways:  prosodically, as in English; morphologically, as in Japanese; and syntactically, as in Russian (\citealp{Gutiérrez-Bravo2008} and references therein). In Spanish, focus can be expressed prosodically and syntactically (\citealp{Zubizarreta1998,Face2006,Chung2012}, among others).

Focus can be defined according to its size as broad or narrow, and according to its meaning as new information (or presentational), contrastive, or mirative \citep{DeLancey1997, Ladd2008,gussenhoven2008}. Under broad focus, the entire sentence is focused; this occurs when the whole sentence provides non-presupposed, new information, as shown in (\ref{13:ex:2}). On the other hand, under narrow focus only one sentential element is focused (\ref{13:ex:3}). The question in (\ref{13:ex:3a}) expresses the presupposition that Adriana bought something (this is the old, given information, or the sentence topic) but the value of the wh-word is unknown. The direct object in (\ref{13:ex:3b}) has narrow focus, and supplies the value for the variable bound by the wh-word.

\ea \label{13:ex:2}
\ea\label{13:ex:2a}
\gll   ¿Qué  pasó?\\
what happen.\textsc{pst}.3\textsc{sg}\\
\glt ‘What happened?’\\
\ex   \label{13:ex:2b}
\gll $[$\textsubscript{\textsc{focus}} Adriana compró un libro.$]$\\
{} Adriana  buy.\textsc{pst}.3\textsc{sg} a   book\\
\glt ‘Adriana bought a book.’  \\
\z
\z

\ea \label{13:ex:3}%Ex 3
\ea \label{13:ex:3a}
\gll  ¿Qué compró Adriana?\\
what buy.\textsc{pst}.3\textsc{sg} Adriana\\
\glt ‘What did Adriana buy?’\\
\ex  \label{13:ex:3b}
\gll Adriana compró         $[$\textsubscript{\textsc{focus}} un libro$]$.\\
Adriana  buy.\textsc{pst}.3\textsc{sg} {} a   book\\
\glt ‘Adriana bought a book.’  \\
\z
\z
Regarding meaning, new information focus corresponds to the non-presupposed part of the sentence \citep{Zubizarreta1998, chomsky1971, Chomsky1976, Jackendoff1972}, while contrastive focus negates the value assigned to a specific variable and provides a different value for it \citep{Zubizarreta1998}. On the other hand, mirative focus conveys surprise from unexpected information, has counter-expectational value, and transmits expressive and emotive attitude \citep{machuca_ayusoyRios2017,DeLancey1997,DeLancey2001,DeLancey2012,Dickinson2000,Cruschina2012, gili2015intonational, Jiménez-Fernández2015a, Jiménez-Fernández2015b,BianchiCruschina2016,BadanCrocco2019,belletti2017syntax}. The syntactic and prosodic characteristics of these focus types are reviewed next.
\subsection{Syntactic properties}
In Spanish, new information focus needs to appear as the rightmost element in the linear string to receive nuclear stress, i.e., to be assigned the main sentence prominence \citep{Zubizarreta1998,Gutiérrez-Bravo2008,López2009}. Using the question-answer test, (\ref{13:ex:4b}) is ungrammatical as an answer to (\ref{13:ex:4a}) because the new information focus \textit{un libro} ‘a book’ does not appear sentence-finally. In contrast, (\ref{13:ex:4c},\ref{13:ex:4d}) constitute valid answers since the focus appears in the rightmost position (note that in (\ref{13:ex:4d}), the pause -- indicated with \# -- effectively makes \textit{un libro} ‘a book’ rightmost in the linear string). \footnote{We follow \citegen{Zubizarreta1998} original intuitions here, but see \citet{ortegasantos2016} for a review of current experimental work that shows dialectal variation in the judgments (for example, in Argentinian Spanish, Mexican Spanish or Southern Iberian Spanish). As discussed by \citet{Jiménez-Fernández2015a}, Southern Peninsular Spanish has a specific position in the left periphery for new information focus in contrast to Standard Spanish (this includes speakers from Northern Spain and Madrid).}

\ea \label{13:ex:4}%Ex 4
\ea   \label{13:ex:4a}
\gll ¿Qué compró Adriana en la librería?\\
what buy.\textsc{pst}.3\textsc{sg}  Adriana in the bookstore\\
\glt ‘What did Adriana buy at the bookstore?’\\

\ex \gll * Adriana compró $[$\textsubscript{\textsc{focus}} un libro$]$ en la librería. \\ \label{13:ex:4b}
{}Adriana buy.\textsc{pst}.3\textsc{sg} {} a   book in the bookstore\\
\glt ‘Adriana bought a book at the bookstore.’  \\

\ex   \label{13:ex:4c}
\gll Adriana compró en la librería   $[$\textsubscript{\textsc{focus}} un libro$]$.\\
Adriana  buy.\textsc{pst}.3\textsc{sg} in the bookstore {} a book\\
\glt ‘Adriana bought a book at the bookstore.’  \\

\ex  \label{13:ex:4d}
\gll Adriana compró         $[$\textsubscript{\textsc{focus}} un libro$]$ \# en la librería.\\
Adriana  buy.\textsc{pst}.3\textsc{sg} {} a  book {} in the bookstore\\
\glt ‘Adriana bought a book at the bookstore.’  \\
\z
\z

Contrastive focus differs from new information focus in regards to word order; any element in the sentence can be contrastively focused, regardless of sentence position \citep{Zubizarreta1998}. One contextualized example is given in (\ref{13:ex:5}).\footnote{Here and throughout, capitalization is used to indicate elements with contrastive or mirative focus.} \\

\ea Contrastive statement \label{13:ex:5}%Ex 5
\ea  \label{13:ex:5a}
\gll ¿Qué compró Adriana?\\
what buy.\textsc{pst}.3\textsc{sg} Adriana\\
\glt ‘What did Adriana buy?’\\
\ex  \label{13:ex:5b}
\gll Adriana compró         $[$\textsubscript{\textsc{focus}} un LIBRO$]$ en la  librería      (no una revista).\\
Adriana buy.\textsc{pst}.3\textsc{sg} {} a book at the bookstore    not a magazine\\
\glt ‘Adriana bought a BOOK at the bookstore (not a magazine).’  \\
\z
\z

\citet{Jiménez-Fernández2015b} points to syntactic differences between contrastive and mirative focus (in the context of focus fronting). While contrastive focus can occur in an embedded sentence as a complement of a verb of saying (\ref{13:ex:6}), mirative focus is disallowed in this context (\ref{13:ex:7}) (this property was originally discussed by \citet{Cruschina2012} for Italian):\footnote{For a discussion on verb adjacency and its interaction with contrastive and mirative focus, see  \citet{Jiménez-Fernández2015b}.} \\

\ea Contrastive statement \label{13:ex:6}%Ex 6
\ea
\gll Juan va diciendo que  ha vendido la moto.\\
John go-\textsc{pres}.3\textsc{sg} say.\textsc{ger}    that have-\textsc{pres}.3\textsc{sg} sell.\textsc{ptcp}      the motorbike\\ \jambox{\citep[53]{Jiménez-Fernández2015b}}
\glt ‘John goes saying that he has sold the motorbike.’\\
\ex
\gll No, no. María dice              que el coche ha   vendido, no la    moto.\\
no   no  Mary say.\textsc{pres}.3\textsc{sg} that the car    have.\textsc{pres}.3\textsc{sg} sell.\textsc{ptcp}      not the motorbike\\ \jambox{\citep[53]{Jiménez-Fernández2015b}}
\glt ‘No, no. Mary says that he has sold the car, not the mortorbike.’  \\
\z
\z



\ea  \label{13:ex:7}%Ex 7
[??]{
\gll ¡¡No me lo puedo           creer!! ¡¡Va diciendo por ahí  que DOS BOTELLAS DE VODKA nos habíamos       bebido en la fiesta!!\\
not \textsc{cl}   it can.\textsc{pres}.1\textsc{sg} believe.\textsc{inft} go.\textsc{pres}.3\textsc{sg} say.\textsc{ger}   by there that  two bottles         of   vodka       \textsc{cl}  have.\textsc{pst}.1\textsc{pl} drink.\textsc{ptcp}   in the party\\ \jambox{\citep[53]{Jiménez-Fernández2015b}}
\glt ‘I can’t believe it! He goes saying everywhere that we had drunk TWO BOTTLES OF VODKA at the party!!’\\
}
\z


As mentioned in \sectref{sec:13:1}, in-situ wh-elements carry the main focus of a question \citep{Horvath1986,Rochemont1986,Tuller1992,Zubizarreta1998,escandellvidal1999}. \citet{Reglero2007} and \citet{RegleroTicio2013} argue that wh-phrases in INF questions have new information focus\footnote{See \citet{Uribe-Etxebarria2002} for a proposal in which in situ wh-questions in Spanish have contrastive focus. This is primarily based on a more restricted interpretation of wh-in-situ in Spanish (at least according to \citegen{Jiménez1997} intuition). Uribe-Etxebarria provides additional examples and a syntactic analysis that relates the interpretative properties of wh-in-situ in Spanish to their syntactic derivation.}  since they elicit non-presupposed information (i.e., the value of the wh-word is unknown; see (\ref{13:ex:3}), (\ref{13:ex:4})), and can also appear in out-of-the-blue contexts. One example is given in (\ref{13:ex:8}), where the question is introduced by \textit{dime una cosa} ‘tell me something’, a phrase eliciting new information.\footnote{This test is attributed to Ignacio Bosque (p. c.) \citep{RegleroTicio2013}. See also \citet{gonzalez2018dime,gonzalez_reglero2020}.}  In addition, the in situ wh-phrase needs to appear finally (\ref{13:ex:8b} -- \ref{13:ex:8d}) (see (\ref{13:ex:4}) above).\\

\ea \label{13:ex:8}%Ex 8
\ea  \label{13:ex:8a}  \gll Dime una cosa: ¿Rosalía llevó  qué?\\
tell.\textsc{imp}-\textsc{cl}.\textsc{dat}.1\textsc{sg} one thing   Rosalía wear.\textsc{pst}.3\textsc{sg}  what\\
\glt ‘Tell me something: What did Rosalía wear?’\\

\ex  \label{13:ex:8b}
\gll ¿Tú le diste el libro a quién? \\
you \textsc{cl}.\textsc{dat}.3\textsc{sg} give.\textsc{pst}.2\textsc{sg} the book to who\\
\glt ‘Who did you give the book to?’\\

\ex[??]{¿Tú le diste a quién el libro?\\} \label{13:ex:8c}


\ex  ¿Tú le diste a quién \# el libro? \label{13:ex:8d}
\z
\z

For SUR questions, \citet{RegleroTicio2013}  have argued that the wh-phrase has contrastive focus\footnote{Their claim applies to REP echo questions as well.}  since the echo wh-phrase does not need to appear finally (\ref{13:ex:9}) (see (\ref{13:ex:5}) above). In addition, SUR requires heavy contextualization, unlike INF (\ref{13:ex:10}).\\

\ea \label{13:ex:9}%Ex 9
 \gll ¿Rosalía llevó              QUÉ ayer?\\
Rosalía  wear.\textsc{pst}.3\textsc{sg} what yesterday\\
\glt ‘Rosalía wore WHAT yesterday?’\\
\z


\ea \label{13:ex:10} %Ex10
{Speaker 1:} \\
\gll Adela fue a visitar a Aristóteles.\\
Adela go.\textsc{pst}.3\textsc{sg}  to visit.\textsc{inft} \textsc{dom}  Aristotle\\
\glt ‘Adela went to visit Aristotle’\\

\sn {Speaker 2:}\\
\gll ¡No  me lo puedo creer!: ¿Adela    fue a visitar a QUIÉN?\\
\textsc{neg} \textsc{cl}.1\textsc{sg}  \textsc{cl}.\textsc{acc}.3\textsc{sg}  can.1\textsc{sg}    believe.\textsc{inft}  Adela    go.\textsc{pst}.3\textsc{sg}  to visit.\textsc{inft} \textsc{dom}  who\\
\glt ‘I can’t believe it! Adela went to visit WHO?’\\
\z

However, recent work on Italian argues that SUR in this language is associated with Mirative Focus (MirF) \citep{crocco2016,BadanFiori2017,BadanCrocco2019}. MirF is a type of focalization involving surprise and unexpectedness. For Italian in-situ questions, MirF and INF have different syntactic properties: the most obvious one is that INF needs to be fronted, unlike SUR (\ref{13:ex:5a},\ref{13:ex:5b}).\footnote{See \citet{BadanCrocco2019} for additional differences in embedded contexts (related to question availability and scope). They propose overt movement of the wh-phrase to a low focus position (MirF) in echo questions.} Unlike INF, the wh-phrase in Italian SUR is D-linked to a previous discourse. Both types of questions also show prosodic differences, as discussed in the following section.\\

\ea \label{13:ex:11}%Ex11
\ea \gll Dove vendono le mandorle?\\
where sell.\textsc{prs}.3\textsc{pl}  the almonds\\ \jambox{\citep[47]{BadanCrocco2019}}
\glt ‘Where do they sell the almonds?’\\
\ex \gll  	Le vendono DOVE le mandorle?\\
\textsc{cl}.\textsc{obj}.3\textsc{pl} sell.\textsc{prs}.3\textsc{pl}  where  the almonds\\ \jambox{\citep[47]{BadanCrocco2019}}
\glt ‘They sell (them) where the almonds?’\\
\z
\z



\subsection{Prosodic properties}
Prosodically, focused constituents tend to stand out over topics. As in many languages, in Spanish, focused elements can constitute separate intonational units \citep{Gutiérrez-Bravo2008}. A high intermediate boundary tone (H- or HH-) can occur between the old (topic) and new information (focus) (\citealp[268--270]{Hualde2014}; \citealp[369]{hualde2015}). In addition, non-focal elements tend to have reduced pitch range \citep[see for example][]{delamota1997, Face2002a}.

The realization of both prenuclear (non-final) and nuclear accents tends to differ in broad and narrow focus statements. In Madrid Spanish, pre-nuclear accents tend to have a higher pitch under narrow focus and/or be aligned with the stressed syllable, unlike under broad focus, where the peak tends to be displaced to the post-tonic \citep[][]{Face2001}. Stressed syllables are also longer under narrow focus in this dialect \citep{Face2000}. In Castilian Spanish, nuclear accents tend to have a low pitch accent (L*) under broad focus, and rising (L+H*) under narrow focus \citep{Estebas-VilaplanaPrieto}. However, in Spanish contact varieties, including in contact with Basque, prenuclear accents tend to have earlier peaks under broad focus, as well \citep{Elordieta2003,orourke2012}.

There are also prosodic differences between contrastive and new information focus. The former is characterized by higher pitch, expanded pitch range and/or earlier pitch alignment compared to the latter, at least in statements. In addition, an intermediate high or low boundary (H-, L-) can follow the contrastively focused constituent \citep{delamota1997,Face2002a,Face2002b}. Contrastive focus shows longer duration than new information focus sententially, in the focal constituent, and in its stressed syllable \citep{Chung2012}. However, sentence-finally elements in narrow focus appear to have similar pitch height and show early peak alignment, unlike in statements with broad focus, where late alignment is more frequent \citep{Domínguez2004}.

As mentioned in the previous section, wh-in-situ elements in Spanish are focused and are assigned nuclear stress since they are located at the end of the intonational phrase. The rest of the sentence is the topic since the information is presupposed. Impressionistic reports on the prosody of INF questions mention falling intonation and extra or ``marked'' stress (\citealp[63]{escandellvidal1999}; \citealp[]{Uribe-Etxebarria2002,RegleroTicio2013}). On the other hand, in situ-echo questions, particularly those conveying surprise, reportedly display (falling)-rising or sharp/strong intonation and have marked stress on the wh-phrase \citep{contreras1999, Pope1976, escandellvidal1999, Sobin2010, Chernova2013, Chernova2017}.

A preliminary investigation of wh-in-situ questions in four participants of North-Central Peninsular Spanish shows that INF have final rising intonation more often than SUR. The latter show an expanded sentential tonal range, and a substantially higher final High pitch compared to INF. On the other hand, the duration ratio of the wh-element (i.e, its duration relative to the sentence duration) is larger in INF than in SUR \citep{gonzalez2018dime}. These preliminary findings contradict the falling/falling-rising distinction previously reported for INF and SUR, but suggest that marked stress in INF is a perceptual result of increased duration of the wh-element, while sharp/strong intonation in SUR is related to expanded scaling and an elevated final pitch accent/boundary tone \citep[for stress correlates in Spanish, see][]{ortega2007, Ortega-LlebariaPrieto2011,Hualde2014}.

These preliminary results are also in line with other studies investigating intonational differences in pragmatic meaning for Spanish questions. For example, fronted wh-questions with a counter-expectational value have expanded pitch ranges compared to neutral questions. This difference usually goes hand in hand with a difference in boundary tone \citep[Argentinian Spanish:][]{gabriel2010} or nuclear configuration (\citealp[Peninsular Spanish:][]{Estebas-VilaplanaPrieto}; \citealp[][374]{hualde2015}; \citealp[Mexican Spanish:][]{delamota2010}; \citealp[Venezuelan Spanish:][]{astruc2010}).\footnote{In Ecuadorian Spanish, pitch range exclusively distinguishes between the two \citep{HuttenlauchFeldhausen2016}.} \footnote{A similar prosodic combination is also reported in Catalan and Italian \citep{gili2015intonational, prieto2015}.}  In addition, although Castilian Spanish echo-questions tend to be realized with upstepped rising nuclear accents (L+¡H*), those with a counter-expectational value tend to have a sharp final rise (HH\%) instead of a low boundary tone (L*) \citep[][]{Estebas-VilaplanaPrieto}.\footnote{In Brazilian Portuguese, neutral INF questions have falling intonation, while echo ones are rising \citep{kato2019}.}

\largerpage
Although earlier work considers that surprise echo questions have contrastive focus in Spanish, recent work on Italian suggests that mirative focus is involved since SUR questions have counter-expectational value \citep{BadanCrocco2019}. In addition to showing clear syntactic differences, SUR wh-in-situ questions in Italian are different prosodically from INF questions in several respects. First, the wh-phrase carries the main prominence of the sentence in SUR but not in INF contexts, where the main prominence falls on the verb. Second, the wh-phrase in SUR shows expanded scaling and has an upstepped rising pitch accent (L+¡H*); in comparison, INF questions have falling pitch accents, which are closely aligned with the verb. Finally, SUR questions have a high boundary tone after the wh-element and a clearly perceived disjuncture with the rest of the question. In contrast, in INF, the verb is followed by a low boundary tone, and a clear disjuncture is not typically perceived.
Assuming that INF have new information focus and SUR mirative focus, we explore the intonational properties of both question types to elucidate the prosodic characteristics of both types of focus. We examine data from 14 speakers of North-Central Peninsular Spanish, where non-fronted wh-in-situ questions can have a new information reading, in addition to echo readings. Two specific hypotheses are investigated: First, if Spanish INF and SUR have different foci, they will have distinct prosodic properties. Second, if SUR have MirF, they will differ from INF in one or more of the following: (i) intonational contour, (ii) pitch range, and/or (iii) F0 (\citealp{crocco2016,HuttenlauchFeldhausen2016,BadanFiori2017,machuca_ayusoyRios2017,BadanCrocco2019}, among others).

\section{Methodology}\label{sec:13:3}
\subsection{Participants and data collection}
Our participants are Spanish speakers from the Basque Country in northern Spain. Although bilingualism in Spanish and Basque is prevalent, and language contact with Basque influences some prosodic characteristics of Spanish in this area \citep{elordieta2020}, the impact of language contact is considered to be minimal or non-existing for this study  since Basque does not allow in-situ information or surprise echo questions \citep{EtxepareOrtiz2003, reglero2003non}.\footnote{Echo wh-questions in Basque are usually preverbal \citet{EtxepareOrtiz2003}, as shown in the example below:
\ea
\ea
\gll Zugandik  atera  dira kontu  zikin guzti horiek.\\
you.from  come  \textsc{aux} stories dirty  all     those\\
\glt‘All those dirty stories have come from you’\\

\ex
\gll Nigandik ZER    atera    dela?\\
me.from  what  come    \textsc{aux}.that\\
\glt‘(That) what has come from me?’ \\
\z
\z
\citet[][463]{EtxepareOrtiz2003} indicate that echo wh-questions with corrective/contrastive focus can appear finally with a preceding prosodic break; these are quite marked. \citet{DuguineIrurtzun2014} indicate that young Laubordin Basque speakers use an innovative strategy involving wh-in situ. None of the participants in our study come from this dialectal area.}

Data was collected in Summer 2015 in Bilbao, Spain. Participants completed two tasks: a reading task, and a controlled elicitation task. Both were facilitated via a powerpoint that included visual and auditory stimuli to provide contextual information to engage participants in the task and prompt the relevant pragmatic reading. Both tasks were designed to control the context and therefore the pragmatic reading of the stimuli. The reading task is most similar to the methodology employed in other intonational studies of Spanish, including \citet{PrietoRoseano2010} and \citet{Rao2013} and can be conceived of as involving ``scripted speech''. The controlled elicitation task, which we focus on in this paper, did not include a written script for participants to read from, and was designed to provide a more naturalistic realization of the stimuli.

The completed experiment took approximately one hour per participant. A total of 22 Spanish participants took part in the experiment; all were paid for their participation. Participants had varied degrees of Spanish-Basque bilingualism. Before the tasks, all participants completed a consent form and the Bilingual Linguistic Profile \citep[BLP;][]{birdsong2012bilingual} to obtain information on the language history, use proficiency, and attitudes towards Spanish and Basque. For this study, we report data from the elicitation data from 14 participants; all were 21--24 years old females from the province of Bizkaia.

\tabref{13:table1} provides additional participant information. Positive BLP dominance scores indicate Spanish dominance; scores close to zero indicate balanced bilingualism. Negative dominance scores indicate that participants are Basque dominant. Only three participants have negative dominance scores (P3, P7, P14); two of them are close to zero (P14, P7).\\

\begin{table}
\caption{Participant information: Procedence and BLP scores}
\label{13:table1}
 \begin{tabular}{lrrrl}
  \lsptoprule
ID & BLP Score & Spanish Score & Basque Score & Town\\
  \midrule
P15  & 168       & 199           & 31           & Santurtxi    \\
P11  & 155       & 190           & 35           & Leioa        \\
P9   & 123       & 209           & 86           & Trapagaran   \\
P22  & 85        & 161           & 76           & Galdakao     \\
P8   & 80        & 177           & 97           & Bilbao       \\
P1   & 76        & 201           & 125          & Leioa        \\
P5   & 51        & 182           & 131          & Sopelana     \\
P21  & 49        & 178           & 129          & Barakaldo    \\
P4   & 38        & 201           & 163          & Sopelana     \\
P20  & 26        & 180           & 154          & Galdakao     \\
P13  & 14        & 176           & 162          & Sopelana     \\
P14  & $-$2        & 170           & 172          & Arrankudiaga \\
P7   & $-$5        & 188           & 193          & Durango      \\
P3   & $-$40       & 159           & 199          & Gorliz       \\
  \lspbottomrule
 \end{tabular}
\end{table}

The target sentences for the elicitation task involved fronted and in-situ wh-questions, statements, and yes-no questions. Here we focus on in-situ SUR and INF questions. Contextualized examples are provided below; note that all participants completed a short practice before the tasks, and that the context and prompt were presented aurally (not in written form).

\ea %Ex12
\ea Context/Prompt: \\
\textit{Maite, Cristina, y Elena se han puesto a jugar al escondite con una amiga. Maite se ha escondido detrás de un árbol. Cristina detrás de un arbusto. Para preguntar por Elena una posibilidad sería decir: ¿y dónde se ha escondido Elena? ¿Cuál sería la otra manera de decirlo?}\\
‘Maite, Cristina and Elena are playing hide-and-seek with a friend. Maite hid behind a tree. Cristina hid behind a bush. To ask about Elena, one possibility would be to say: And where did Elena hide? What would be another way to ask this question?’\\
\ex
Expected target question:\\
\gll ¿(Y) Elena se ha escondido dónde?\\
and Elena \textsc{cl}.\textsc{refl} have.\textsc{prs}.3\textsc{sg} hide.\textsc{ptcp}  where \\
\glt ‘(And) where did Elena hide?’\\
\z
\z

\ea %Ex13
\ea SUR question: \\
\textit{Estás en la casa de una amiga y te enseña sus mascotas. Te dice: “El gato se llama Macacocogito.” Te sorprende muchísimo el extraño nombre de su gato. Hazle una pregunta para comprobar cómo se llama.}\\
‘You are at your friends’ house, and she shows you her pets. She says: “My cat’s name is Macacocogito”. You are completely surprised by the cat’s unusual name. Ask your friend a question to double-check the cat’s name.’\\
\ex
Expected target question:\\
\gll ¿El gato se        llama               C\'OMO?\\
the cat \textsc{cl}.\textsc{refl}  name.\textsc{prs}.3\textsc{sg} how \\
\glt ‘The cat’s name is WHAT?’\\
\z
\z

\subsection{Recording and coding}

Recording was conducted via a Tascam DR-05 digital recorder with built-in omni-directional microphones. Audio was recorded in 44,000 Hz in mono. 10 INF questions and 10 SUR questions were examined per participant for a total of 280 target sentences. Eight INF and six SUR questions had to be discarded because of waveform distortion and/or wh-fronting, leaving 266 sentences for the acoustic analysis.

Data was coded in Praat \citep{praat} according to Spanish ToBI conventions (\citealp{AguilarPrieto2009,FacePrieto2007} inter al.). Both authors were involved in the acoustic analysis. Disagreements, which occurred in approximately 5\% of the tokens, were resolved by consensus. The analysis focused on the following characteristics: (i) the overall melodic shape of the question, (ii) its nuclear configuration, (iii) the nuclear peak (in Hz.), and (iv) the focal tonal range (FTR), i.e., the difference between the lowest point at the beginning of the wh-phrase and its highest pitch. Pitch is reported in Hz and semitones (ST); the latter helps normalize the data and is more closely related to pitch perception. Specifically, a difference of 1.5 ST meets the perceptual threshold, i.e., it is considered to be perceivable by all speakers \citep{T’hart1981,Toledo2000,Pamies2002}. Paired two-tailed t-tests were conducted to establish whether these results are statistically significant.

\figref{13:Fig2}--\ref{13:Fig5} below provide examples of melodic contours for INF and SUR. \figref{13:Fig2} exemplifies the most frequent INF contour; it begins with an initial fall followed by a rise up to the first post-tonic syllable, which diphthongizes with the auxiliary verb to its right. Declination follows up to the beginning of the wh-question, realized with a steep final rise (L+H* HH\%). The FTR is 183 Hz, equivalent to 10.7 ST.

\begin{figure}
    \includegraphics[scale=.15]{figures/GR_FIG2.png}
    \caption{INF question. P21\_12  ‘And when has the third one gone out?’}
    \label{13:Fig2}
\end{figure}

\figref{13:Fig3} exemplifies an additional melodic pattern for INF, which starts with a slight initial fall up to the post-tonic syllable, followed by a slight rise on the verb \textit{fue}. Declination ensues, and the wh-question shows a final rise-fall (L+H* L\%). The FTR is 158 Hz, equivalent to 9.8 ST.

\begin{figure}
    \includegraphics[scale=.15]{figures/GR_FIG3.png}
    \caption{INF question. P15\_4  ‘And where did Marian go?’}
    \label{13:Fig3}
\end{figure}

Figure \ref{13:Fig4} shows a third melodic pattern for INF in our data, involving a rise up to the wh-word, followed by a final fall-rise (H+L* LH\%). The FTR is 89 Hz, equivalent to 7.4 ST.

\begin{figure}
    \includegraphics[scale=.15]{figures/GR_FIG4.png}
    \caption{INF question. P18\_13  ‘And how does Alejandra go up?’}
    \label{13:Fig4}
\end{figure}

SUR questions were realized similarly across participants. They involved an initial rise up to the first post-tonic syllable, declination up to wh-question, and a steep final rise (Figure~\ref{13:Fig5}). The nuclear configuration can be characterized as L+H* HH\%, as in Figure~\ref{13:Fig2}. The FTR is 191 Hz (11.6 ST).

\begin{figure}
    \includegraphics[scale=.15]{figures/GR_FIG5.png}
    \caption{SUR question. P3\_3  ‘The cat’s name is WHAT?’}
    \label{13:Fig5}
\end{figure}

\section{Results}\label{sec:13:4}
\subsection{Overall melodic contour}
All SUR questions in our dataset show three intonational movements: (i) a rise through the first post-tonic syllable; (ii) declination (i.e., pitch lowering) up to the wh-phrase, and (iii) a steep final rise (\figref{13:Fig5}). For INF questions, a similar pattern occurs in 85\% of cases, although an additional fall is usually present at the beginning (\figref{13:Fig2}). This fall occurs in cases where INF began with \textit{y} ‘and’, a pragmatic strategy available in INF questions to establish a transition between the previous discourse and the wh-in-situ question \citep{Jiménez1997}. Two additional melodic contours are attested for INF: one characterized by a final rise-fall (7.5\%) (\figref{13:Fig2}), and another with an overall rise up to the beginning of the wh-phrase followed by a nuclear fall-rise (7.5\%) (\figref{13:Fig4}). Most of these less frequent patterns are found in speakers 15 and 8, respectively.

\subsection{Nuclear configuration}
All SUR questions and most INF questions end in a high (HH\%) boundary tone. The main exceptions are participant 15, showing a low boundary tone (L\%) in 60\% of INF, and participant 8, with a rising (LH\%) boundary tone in 50\% of INF questions. Low or rising boundary tones are also found sporadically in participants 3, 7 and 13.

The realization of the nuclear accent is more variable. \tabref{13:table2} provides more information about the dominant nuclear configuration and its frequency per participant and type of question investigated. It can be observed that 10 of the participants analyzed (71\%) show similar nuclear pitch accents in both INF and SUR: five of them have a rising nuclear pitch accent (L+H*), and five show a low nuclear pitch accent (L*).

The four remaining participants have different nuclear pitch accents in INF and SUR. Three of the participants (P7, 8, 13) have a low or falling pitch accent (H+L*) in INF questions, and a rising pitch accent in SUR questions. Participant 15 shows a preference for a rising pitch accent in INF (L+H*), and a low pitch accent (L*) in SUR. As stated above, this participant tends to realize low or rising boundary tones in INF questions.

\begin{table}
\caption{Nuclear configurations}
\label{13:table2}
 \begin{tabular}{lrlrlr}
  \lsptoprule
ID & BLP Score & INF       & \%    & SUR       & \%    \\
  \midrule
P15         & 168       & L+H* L\%  & 60\%  & L* HH\%   & 90\%  \\
P11         & 155       & L* HH\%   & 80\%  & L* HH\%   & 60\%  \\
P9          & 123       & L* HH\%   & 60\%  & L* HH\%   & 70\%  \\
P22         & 85        & L+H* HH\% & 80\%  & L+H* HH\% & 89\%  \\
P8          & 80        & H+L* HH\% & 50\%  & L+H* HH\% & 70\%  \\
            &           & H+L* LH\% & 50\%  &           &       \\
P1          & 76        & L+H* HH\% & 100\% & L+H* HH\% & 80\%  \\
P5          & 51        & L+H* HH\% & 90\%  & L+H* HH\% & 100\% \\
P21         & 49        & L+H* HH\% & 100\% & L+H* HH\% & 60\%  \\
P4          & 49        & L* HH\%   & 100\% & L* HH\%   & 100\% \\
P20         & 38        & L* HH\%   & 55\%  & L* HH\%   & 80\%  \\
P13         & 14        & L* HH\%   & 89\%  & L+H* HH\% & 100\% \\
P14         & $-$2        & L+H* HH\% & 70\%  & L+H* HH\% & 100\% \\
P7          & $-$5        & L* HH\%   & 67\%  & L+H* HH\% & 100\% \\
P3          & $-$40       & L* HH\%   & 88\%  & L* HH\%   & 70\%\\
  \lspbottomrule
 \end{tabular}
\end{table}

There is no apparent correlation with bilingualism; the patterns showed by Basque dominant speakers P3, P7 and P14 are variable and comparable to those attested in Spanish dominant participants.

\subsection{Nuclear high}
\figref{13:Fig6} shows the values of the nuclear High for all participants in INF and SUR. Eleven participants (79\%) have a more elevated H in SUR. On average, the value of H in SUR contexts is +2.1 ST higher than in INF questions. This difference is above the perceptual threshold, suggesting that it is perceptually significant. Results from a paired two-tailed t-test indicate that this difference is statistically significant (\textit{p} = 0.0038). The examination of individual differences shows that the perceptual threshold is reached or surpassed in 8 of the participants. The remaining three participants do not follow this trend. Specifically, participants P9 and P11 have a more elevated nuclear High in INF contexts, while P5 shows a similar nuclear High in both pragmatic readings (Appendix~\ref{13:app:1:NHigh}).\\

\begin{figure}
    \includegraphics[width=\textwidth]{figures/GR_Fig6.png}
    \caption{Nuclear High in INF and SUR questions}
    \label{13:Fig6}
\end{figure}

\subsection{Focal tonal range}
\figref{13:Fig7} shows a box plot for the focal tonal range of INF and SUR questions for all participants pooled. It can be observed that the medians of INF and SUR are very different. On average, the FTR for SUR is +2.9 ST higher than for INF, well above the perceptual threshold. In addition, results from a paired two-tailed t-test indicate that this difference is statistically significant (\textit{p} < 0.001). The examination of individual differences shows that this perceptual difference holds for 11 participants. For participant P9, this difference approaches the perceptual threshold (1.4 ST.). Two participants do not follow this trend: P11, which has a higher FTR in INF, and P13, which has a similar FTR in both INF and SUR (Appendix~\ref{13:app:2:FTR}).

\begin{figure}
    \includegraphics[width=\textwidth]{figures/GR_Fig7.png}
    \caption{Focal Tonal Range in INF and SUR questions}
    \label{13:Fig7}
\end{figure}

\section{Discussion}\label{sec:13:5}
The present study set out to investigate the prosodic characteristics of two types of pragmatically different wh-in-situ questions in Spanish: those requesting new information (INF), and those expressing surprise (SUR). Both share some syntactic similarities, since the wh-in-situ phrase is sentence-final. Our analysis reveals some prosodic similarities as well: the general melodic contour tends to be similar for both in most speakers, generally comprising an initial rise, medial declination, and a steep final rise on the wh-question. In addition, a high (HH*) final boundary tone tends to be present in both question types.

Syntactically and pragmatically, INF and SUR also show some differences. INF are neutral and restricted to the rightmost position in the linear string, while SUR are counter-expectational and have a less restricted distribution. Prosodically, we find some differences as well: the nuclear High is significantly more elevated in SUR, and the focal tonal range is significantly expanded. A difference in FTR occurs in most participants, suggesting that this is the main prosodic cue distinguishing SUR from INF in this Spanish variety. We don’t observe differences according to degree of Basque/Spanish bilingualism. This is expected since, although language contact impacts the realization of some prosodic features in both languages \citep[see for example][]{Elordieta2003}, the wh-in-situ questions investigated here for Spanish are not grammatical in Basque \citep{EtxepareOrtiz2003, reglero2003non}.

The intonational properties identified in this study for Spanish SUR are comparable to those reported for Italian SUR questions \citep{BadanCrocco2019}. At first blush, unlike for Italian, the nuclear configurations of the wh-in-situ phrase in Spanish INF and SUR are similar, as in German, where the tonal contours of INF and SUR are reportedly the same \citep{ReppRosin2015}. However, we argue that Spanish INF and SUR have distinct nuclear contours: INF is most frequently realized with a rising nuclear accent (L+H*), while SUR involves upstepping (L+¡H*). The difference between these two tonal configurations is reportedly one of pitch range, as shown schematically in \figref{13:Fig8}. Upstepped rising nuclear accents are attested in Italian SUR \citep{BadanCrocco2019} and in Spanish counter-expectational questions (\citealp{AguilarPrieto2009,Estebas-VilaplanaPrieto};\citealp[][374]{hualde2015}).

\begin{figure}
    \includegraphics[scale=.95]{figures/GR_Fig8.png}
    \caption{Rising vs. upstepped rising pitch accents  \citep{AguilarPrieto2009}}
    \label{13:Fig8}
\end{figure}

The participants in our dataset have different degrees of Basque/Spanish bilingualism. We have not observed prosodic differences consistent with Spanish vs. Basque language dominance. Three participants (P5, P9, P13) show individual variation, with either an elevated nuclear peak or a higher FTR in SUR questions, but not both. Only P11 appears to be exceptional since she shows higher F0 and expanded FTR in INF than in SUR, unlike the rest of the participants. We leave open the possibility that low-statistical power and/or individual variation explains this different pattern.
\section{Conclusion}\label{sec:13:6}
This study has focused on the intonation of INF and SUR questions in Spanish. Results from an elicitation task in 14 female speakers from North-Central Peninsular Spanish show similarities in overall melodic contours and final boundary tones, but also differences in the height of the nuclear accent, the focal tonal range, and the nuclear pitch accent. We argue, following \citet{BadanCrocco2019} for Italian, that these differences are consistent with a difference between new information and mirative focus.

\hspace*{-.2pt}The analysis of intonation from the five male speakers remaining in our dataset and from the reading task will be relevant to further ascertain the patterns reported here and to inquire into possible gender differences in the intonation of wh-in-situ questions in Spanish. Future studies should investigate additional correlates of focus, including the presence of intermediate boundaries before the wh-element, wh-phrase duration and intensity, and the realization of pre-nuclear peaks (\citealp{Chung2012,Face2001,Face2002b, gryllia2016} inter alia).

We also would like to note that the investigation of SUR questions in French would be of great interest to further elucidate the prosodic properties of MirF in Romance. \citet{Glasbergen-PlasDoetjes2020} show that INF and repetition (REP) in-situ questions have similar tonal contours in French; however, REP wh-questions have extended pitch scaling and longer duration (cf. \citealp{DéprezKawahara2013,ChengRooryck2000, gryllia2016}. Based on our current understanding of in-situ questions in Italian and Spanish, we consider it extremely likely that French SUR in French will have even wider scaling than REP, and/or might involve a different tonal contour compared to REP and INF.

\largerpage[2]
\section*{Abbreviations}
\begin{tabularx}{.45\textwidth}{lQ}
1 & First person\\
2 & Second person \\
3 & Third person \\
\textsc{acc} & Accusative \\
\textsc{am} & Auto-Segmental (model) \\
\textsc{blp} & Bilingual Linguistics Profile \\
\textsc{cl} & Clitic \\
\textsc{dat} & Dative \\
\textsc{dom} & Differential object marking \\
\textsc{ftr} & Focal tonal range \\
\textsc{ger} & Gerund \\
\textsc{h} & High \\

\end{tabularx}
\begin{tabularx}{.45\textwidth}{lQ}

\textsc{imp} & Imperative \\
\textsc{inf} & Information-seeking \\
\textsc{inft} & Infinitive \\
\textsc{l} & Low\\
\textsc{MirF} & Mirative Focus \\
\textsc{neg} & Negation \\
\textsc{pl} & Plural \\
\textsc{prs} & Present \\
\textsc{pst} & Past \\
\textsc{ptcp} & Participle \\
\textsc{rep} & Echo-repetition \\
\textsc{sg} & Singular \\
\textsc{st} & Semitone \\
\textsc{sur} & Echo-surprise \\
\end{tabularx}

\section*{Acknowledgements}
We thank all participants for their time, and Eric Mart\'inez for helping design the materials for this study. Many thanks to Jessica Craft for her enthusiasm and professionalism collecting the data and preparing it for acoustic analysis, and to Erin Christopher, Tyler King and Gus O’Neil for their assistance with coding. We are also grateful to Alex Iribar at the Phonetics Lab at the University of Deusto for kindly allowing us to use their facilities; Jon Franco, who passed away in 2021, for his help recruiting participants and his support and encouragement; and Mark Amengual for assistance with the BLP. We also gratefully acknowledge the suggestions from three anonymous reviewers, and the assistance of the volume editors and Luis Avil\'es Gonz\'alez in the preparation of the final version of this manuscript. All errors are of course ours. This research was funded by an FSU COFRS grant awarded to the second author in 2014--2015.


%\newpage
\appendixsection{}\label{13:app:1:NHigh}

\begin{table}[H]
\caption{Nuclear High}
\label{13:table3}
 \begin{tabular}{l rrrrr}
  \lsptoprule
  ID & BLP&INF&SUR&\multicolumn{2}{c}{Difference} \\
  \cmidrule(lr){5-6}
  &score&&&(Hz.)&(ST)\\

  \midrule
P15&168&350&444&94&4.1\\
P11&155&383&343&$-$40&$-$1.9\\
P9&123&300&293&$-$7&$-$0.4\\
P22&85&310&349&39&2.1\\
P8&80&255&377&122&6.8\\
P1&76&355&383&28&1.3\\
P5&51&354&355&1&0.05\\
P21&49&595&424&29&1.2\\
P4&49&356&437&81&3.6\\
P20&38&304&313&9&0.5\\
P13&14&302&341&39&2.1\\
P14&$-$2&306&344&38&2\\
P7&$-$5&384&504&120&4.7\\
P3&$-$40&323&411&88&4.2\\
\hline
Average&&334 Hz&378 Hz&44 Hz&2.1 ST\\
  \lspbottomrule
 \end{tabular}
\end{table}

%\newpage
\appendixsection{}
\label{13:app:2:FTR}

\begin{table}[H]
\caption{FTR}
\label{13:table4}
 \begin{tabular}{lrrrrrr}
 %{l @{\hskip 0.75in} c| @{\hskip 0.25in} c @{\hskip 0.25in} c| @{\hskip 0.25in} c @{\hskip 0.25in} c| @{\hskip 0.25in} c@{\hskip 0.25in} c}
  \lsptoprule

 ID&BLP&
  \multicolumn{2}{c}{INF}&
  \multicolumn{2}{c}{SUR}&
  Difference\\
  \cmidrule(lr){3-4}\cmidrule(lr){5-6}
  &score&(Hz.)&(ST)&(Hz.)&(ST)&(ST)\\

  \midrule
P15&168&156&10.2&259&15.4&5.2\\
P11&155&212&14&173&12.2&$-$1.7\\
P9&123&112&8.1&124&9.5&1.4\\
P22&85&126&9&175&12.1&3.1\\
P8&80&88&7.3&180&11.2&3.9\\
P1&76&172&11.5&234&16.3&4.8\\
P5&51&152&9.7&173&11.5&1.8\\
P21&49&189&11.2&234&13.9&2.7\\
P4&49&166&10.9&254&15&4.1\\
P20&38&150&11.8&182&15&3.2\\
P13&14&159&12.9&178&12.8&$-$0.1\\
P14&$-$2&110&7.7&147&9.7&2\\
P7&$-$5&153&8.8&272&13.4&4.6\\
P3&$-$40&127&8.7&232&14.4&5.7\\
\hline
Average&&148 Hz&10.1 ST&200 Hz&13 ST&2.9ST\\
  \lspbottomrule
 \end{tabular}
\end{table}


\sloppy
\printbibliography[heading=subbibliography,notkeyword=this]


\end{document}

\documentclass[output=paper]{langsci/langscibook}
\ChapterDOI{10.5281/zenodo.3972852}

\author{Enoch O. Aboh\affiliation{University of Amsterdam}}

\title[Apparent violations of the final-over-final constraint]
      {Apparent violations of the final-over-final constraint:\newlineCover The case of Gbe languages}
      [Apparent violations of the final-over-final constraint: The case of Gbe languages]


\abstract{In a series of recent talks and articles, Theresa Biberauer, Anders
    Holmberg, Ian Roberts, and Michelle Sheehan argue that the final-over-final
    condition (FOFC) is an absolute universal regulating structure building.
    Yet, many languages deviate from FOFC thus suggesting that this condition
    is not ``surface-true''. The question therefore arises what factors make
    languages violate FOFC on the surface. In order to answer this question, we
    need a typology of FOFC-violating languages, as well as a detailed
    description of such violations. In this short essay, I describe FOFC violations
    in Gbe and some creoles, while relating the observed phenomena to some
    theoretical questions they raise.}

\maketitle

\begin{document}\glsresetall

\section{Introduction}

In a series of recent talks and articles, Theresa Biberauer, Anders Holmberg,
Ian Roberts, and Michelle Sheehan, analyse a very strong tendency across human
languages which appears to be indicative of an absolute universal regulating
structure building: The \gls{FOFC}\is{final-over-final condition} defined as in \REF{ex:aboh:14.1}, and further
discussed in \textcite{SheeBibRobHol2017}, henceforth
SBRH.\is{FOFC|see{final-over-final condition}}

\ea\label{ex:aboh:14.1} \emph{The final-over-final condition (FOFC)}
    \ea A head-final phrase αP cannot immediately dominate a head-initial
    phrase βP if α and β are members of the same extended projection.
    \ex *[\textsubscript{αP} [\textsubscript{βP} β γ] α], where β and  γ are
    sisters and α and β are members of the same extended projection.
    \z
\z

FOFC is not bidirectional since the reverse does not hold: “a head-initial
phrase αP may dominate a phrase βP which is either head-initial or head-final,
where α and β are heads in the same extended projection”
\parencite[cf.][171]{BibHolRob2014}.

Accordingly, \gls{FOFC}\is{final-over-final condition} makes strict predictions both in terms of surface typological
variation and possible outcomes of \isi{language change}
\parencite[cf.][]{BibNewShee2009}. For instance, \gls{FOFC}\is{final-over-final condition} predicts the structures
in (\ref{ex:aboh:14.2}a--c) to exist with the exclusion of the pattern in (\ref{ex:aboh:14.2}d)
\parencite[cf.][171]{BibHolRob2014}.

\begin{exe}
\ex\label{ex:aboh:14.2} Harmonic structures\\
\begin{minipage}[t]{.5\linewidth}
    \exi{}  a. Consistent head-final\smallskip\\
            \begin{tikzpicture}[baseline=(root.base)]

                \Tree 	[.\node(root){β$'$};
                            [.αP
                                γP
                                \textbf{\textit{α}}
                            ]
                            \textbf{\textit{β}}
                        ]

            \end{tikzpicture}
\end{minipage}%
\begin{minipage}[t]{.5\linewidth}
    \exi{b.} Consistent head-initial\smallskip\\
            \begin{tikzpicture}[baseline=(root.base)]

                \Tree 	[.\node(root){β$'$};
                            \textbf{\textit{β}}
                            [.αP
                                \textbf{\textit{α}}
                                γP
                            ]
                        ]

            \end{tikzpicture}
\end{minipage}\smallskip\\
\sn Disharmonic structures\\
\begin{minipage}[t]{.5\linewidth}
    \exi{} c. Initial-over-final\smallskip\\
            \begin{tikzpicture}[baseline=(root.base)]

                \Tree 	[.\node(root){β$'$};
                            \textbf{\textit{β}}
                            [.αP
                                γP
                                \textbf{\textit{α}}
                            ]
                        ]

            \end{tikzpicture}
\end{minipage}%
\begin{minipage}[t]{.5\linewidth}
    \exi{d.} Final-over-initial\smallskip\\
            \begin{tikzpicture}[baseline=(root.base)]

                \Tree 	[.\node(root){\llap{*}β$'$};
                            [.αP
                                \textbf{\textit{α}}
                                γP
                            ]
                            \textbf{\textit{β}}
                        ]

            \end{tikzpicture}
\end{minipage}

\end{exe}

In its strong version, the generalisation in \REF{ex:aboh:14.2} could suggest that
the human mind \enquote{prefers} harmonic structures (\ref{ex:aboh:14.2}a,b), tolerates one type of
disharmonic structure in (\ref{ex:aboh:14.2}c), and totally excludes the disharmonic structure
in (\ref{ex:aboh:14.2}d). This view is obviously misleading since, looking at surface
form only, disharmonic structures abound in languages. This is, for instance,
the case in Kwa (see the discussion below), and in Sinitic (cf.
\citealt{HsiehSybesma2007}, \citealt{SybesmaLi2007}, \citealt{Chan2013} and
references therein). On the basis of his database, \citet{Dryer1992}
concludes that completely harmonic languages actually represent a minority.
Instead, the common cross-linguistic pattern seems to be that languages are
rigidly consistent in some domains, but less so in other domains. FOFC
therefore seems to strictly constrain certain core structures only. Given its
surface flexibility, one could consider the \gls{FOFC}\is{final-over-final condition} effect to derive from
processing constraints facilitating parsing. If one were to adopt
\citeapos{hawkins83} \textit{cross-category harmony}, defined in terms of head
dependent order preferences, or his 1990 \textit{early immediate constituent} principle
suggesting fast recognition of the immediate constituents of a mother node, its
seems intuitive that the parser would prefer orders in which heads and
dependents can be easily identified. In this regard, learning biases seem to
favour certain orders over others. Under this view, \gls{FOFC}\is{final-over-final condition} would be essentially a
third factor phenomenon, required by “principles of efficient computation” in
terms of \citet{Chomsky2005} (cf. \citealt{Walkden2009} for discussion).

SBRH (2017) argue for a different view. \gls{FOFC}\is{final-over-final condition} is a property of structure
building. At this point, the question arises how the notion of \enquote{harmony}
relates to structure building and computation. If \isi{Merge} applies to (categorial)
features only, and embeds no spell-out specification, how can we decide that
(\ref{ex:aboh:14.2}d) is computationally disharmonic compared to (\ref{ex:aboh:14.2}a)? If on the other hand, one
assumes \citegen{grimshaw91} extended projection and some version of
\citegen{Kayne1994} \gls{LCA},\is{linear correspondence axiom} as SBRH (2017) do, then disharmonic structures can be
understood as involving featural mismatches within a functional sequence. Under
this latter view, the bulk of apparent counterexamples to \gls{FOFC}\is{final-over-final condition} would derive
from movement: structures obey \gls{FOFC}\is{final-over-final condition} underlyingly, even though movement
operations may lead to apparent surface violations.\is{LCA|see{linear correspondence axiom}}

It seems to me that two fundamental questions arise here that merit further
investigation: The first question deals with the relation between the \gls{LCA}\is{linear correspondence axiom} and
FOFC, and why the language faculty (in the narrow sense,
cf.~\citealt{hauserchomskyfitch}) would involve such apparently competing
linearization mechanisms. The issue is not trivial as it relates to the
question of the place of linearization within the human faculty of language
(cf.~\citealt{ChoGalOtt2019} and \citealt{Kayne2018} for discussion). I will not
address this question any further in this essay. The second question I will be
concerned with instead is of a typological nature.  Why do some languages seem
to violate \gls{FOFC}\is{final-over-final condition} massively on the surface form? If \citet{Dryer1992} is right,
such violations would be the norm, while \gls{FOFC}\is{final-over-final condition} compliant languages would be the
exception. Why would this be if \gls{FOFC}\is{final-over-final condition} holds on structure building? Why would
languages systematically diverge from core principles imposed by the
computational system? For example, there does not seem to be such a massive
violation of the extended projection principle, a potential universal of
natural languages constraining structure building. In order to understand FOFC
apparent violations therefore, we need to take a closer look at the empirical
facts.

As I will show in the following paragraphs, the Gbe languages (and for that
matter many Niger-Congo languages) involve apparent violations of FOFC. I have
discussed many of these patterns in previous work and proposed an analysis in
terms of the \gls{LCA}.\is{linear correspondence axiom} Since its formulation in the early 2000s, the tenants
of \gls{FOFC}\is{final-over-final condition} have also reported similar patterns cross-linguistically and have
suggested various analyses to account for them (see SBRH 2017 and references
therein).  For instance, final negative markers, such as instantiated in the
Fongbe example in \REF{ex:aboh:14.3a}, can be analysed as not being merged
within the functional sequence of TP \parencite[cf.][]{BibHolRob2014}.  That
such a view is indeed adequate can be shown by the fact that the \ili{Fongbe} yes-no
question in \REF{ex:aboh:14.3b} displays a similar sentence-final particle,
which \textcite{Aboh2010a,Aboh2010b} shows interacts with final negation in
Gbe, as indicated by example \REF{ex:aboh:14.3c}. In this example, the negative
particle precedes a focus marker which in turn precedes the question particle.

\ea\label{ex:aboh:14.3} \ili{Fongbe}
    \ea\label{ex:aboh:14.3a}
        \gll    K\`ɔkú ná x\`ɔ às\'ɔn \'ɔ \v{a} \\
                Koku \Fut{} buy crab \Det{} \Neg{} \\
        \glt    \enquote*{Koku will not buy the crab.}
    \ex\label{ex:aboh:14.3b}
        \gll    Kòfí ɖù às\'ɔn \'ɔ à? \\
                Kofi eat crab \Det{} \glossQ{} \\
        \glt    \enquote*{Did Kofi eat the crab?}
    \ex\label{ex:aboh:14.3c}
        \gll    Kòfí ɖù às\'ɔn \'ɔ \v{a} w\`ɛ à? \\
                Kofi eat crab \Det{} \Neg{} \Foc{} \glossQ{}\\
        \glt    \enquote*{\textsc{Did Kofi not eat the crab?}}
    \z
\z

Facts like these led \citet{Aboh2010a} to propose that the sentence-final
negative particle belongs to the C-domain in Gbe. These data from the Gbe
languages, already show that \gls{FOFC}\is{final-over-final condition} as formulated in \REF{ex:aboh:14.1} is certainly
not “surface-true”. Can we, however, claim that \gls{FOFC}\is{final-over-final condition} constrain the underlying
structure? Given that SBRH (2017) adopt \citegen{grimshaw91} notion of extended
projection, we can answer this question only if we are able to characterize
precisely the featural bundle of the different heads within the functional
sequence of the left periphery in the Gbe languages. Though there is now a
significant body of literature on the complementizer\is{complementizers} system of the Gbe (and
other Kwa) languages, it is reasonable to say that we still do not have a
fine-grained map of the featural specifications of C-type heads in these
languages, and we do not know how learners acquire these features.\largerpage[-1]

This last question becomes even more critical when considering acquisition\is{language acquisition} in
contact situations. Indeed, if \gls{FOFC}\is{final-over-final condition} is an inviolable condition, as suggested by
SBRH (2017), one could imagine that the \gls{PLD} that learners are
exposed to would not generally contain systematic cues for them to derive
FOFC-violating grammars. Put differently, learners must have a way of deducing
underlying FOFC-compliant structures from massively FOFC-violating surface
forms. One would therefore expect superficial FOFC-violating orders (e.g.,
VO-Aux, VO-question particle, VO-Neg) to be unstable and eventually lost in
contact situations. This expectation, however, is not met in the case of
certain creole languages. Indeed, creole languages which emerged in colonial
settings involving enslaved Niger-Congo learners (i.e., speakers of Kwa and
Kikongo) inherited typical Niger-Congo disharmonic structural properties and
therefore display comparable \gls{FOFC}\is{final-over-final condition} surface violations.

Since the original formulations of FOFC, I have discussed some of these surface
FOFC violations with Ian Roberts and Theresa Biberauer. I was therefore only
partially surprised on June 3, 2016 at 3:45pm, when I received a mail from Ian,
which read as follows:\footnote{I am always excited by mails from Ian who also
    happens to be one of my favourite teachers and now very good colleague and
    friend. Ian introduced me to diachronic syntax at a time I had no idea such
    a thing existed. Actually, he has in various ways inspired my recent work
    on language contact and change. In addition, as his student, I liked his
    \ili{French} accent at a time when as a \emph{Béninois} trying to make sense of
    \emph{Français Genevois}, I wondered what \ili{French} and African politicians
    meant by \enquote{la francophonie}. What’s the point if I have hard times
    understanding both \emph{Genevois} and my \ili{French} L2 speaker teacher of
    diachronic syntax? How can we account for such a variation in a principled
    manner? These questions obviously led me to my current work on \emph{hybrid
    grammars}, a concept that is actually not very far from work that Ian has
    done in collaboration with Robin Clark in the early 90s. But let us return
to our current topic of discussion.}

\blockquote{I'm looking at languages with N-A-Num-Dem U20 order in the DP to
    see what (if any) clausal word orders they correlate with. Am I right in
    thinking that \ili{Gungbe} has head-initial order in the clause? According to
    WALS, it has head-final question particles though. Is that correct? In that
    case it looks like an apparent FOFC-violator.}\largerpage[-1]

As suggested in Ian’s message, the discussion on sentences under example
\REF{ex:aboh:14.3} already indicated that the Gbe languages involve clause-final
particles that encode negation \REF{ex:aboh:14.3a}, interrogation
\REF{ex:aboh:14.3b} or a combination thereof \REF{ex:aboh:14.3c}. The following
sentence further shows that these languages display
noun-adjective-numeral-demonstrative order as illustrated in \REF{ex:aboh:14.4}.
Further note that within the DP, the determiner and the plural marker occur to
the right edge (see \citealt{Aboh2004a,Aboh2004b} and references therein for
discussion):

\ea\label{ex:aboh:14.4} \ili{Gungbe}\\
    \gll    [ Òxwé kp\`ɛví àwè éhè l\'ɔ l\'ɛ ] jró mì. \\
            {} house small two \Dem{} \Det{} \Pl{} {} please \Fsg-\Acc{} \\
    \glt    \enquote*{I like these two houses.}, lit.\ \enquote*{These
            two houses please me.}
\z

With regard to Ian’s message therefore these examples indicate that Gbe
languages may constitute counter-examples to FOFC. \citet{Sheehan2013} claims
that the number of such FOFC-violating languages is rather restricted. Since
the Gbe languages exhibit right edge (or final) functional elements both in the
nominal and clausal domain, it is important to look at the facts closely in
order to determine whether these languages represent genuine \gls{FOFC}\is{final-over-final condition} violations or
not. Given the importance of \gls{FOFC}\is{final-over-final condition} in the literature, we need to better
understand such cases of apparent violations in order to find out whether the
principle holds of structure building or whether it relates to surface
phenomena deriving from processing (cf.
\citealt{hawkins83,Hawkins1990,Walkden2009}). In order to make this first step,
the following sections are meant to present more data from Gbe and some creoles
which appear to be \gls{FOFC}\is{final-over-final condition} violations.

Recall from the formulation of \gls{FOFC}\is{final-over-final condition} in \REF{ex:aboh:14.1} that it excludes
structure (\ref{ex:aboh:14.2}d): no language should exist in which a consistent head-initial
structure is dominated by a head-final structure. Under \gls{FOFC}\is{final-over-final condition} therefore a
structure like the one in \REF{ex:aboh:14.3b} cannot have the underlying
representation \REF{ex:aboh:14.5a}, but must be analysed as in
\REF{ex:aboh:14.5b} in which the complement of the Interrogative functional
projection InterP raises to its specifier position. In these representations,
the sentence-final floating low tone expresses a question particle that takes
the clause as complement. It is worth noting, however, that \citet{Aboh2004a},
\citet{AbohPfau2011} propose the same analysis under the \gls{LCA},\is{linear correspondence axiom} hence the
necessity to tease FOFC-related and LCA-related effects apart.

\ea\label{ex:aboh:14.5}
    {\setlength\multicolsep{0pt}
    \begin{multicols}{2}
    \ea\label{ex:aboh:14.5a}
        \begin{tikzpicture}[baseline=(root.base)]

            \Tree 	[.\node(root){InterP};
                        Spec
                        [.Inter$'$
                            [.FinP
                                \edge[roof]; {Kòfí ɖù às\'ɔn \'ɔ}
                            ]
                            [.Inter
                                à
                            ]
                        ]
                    ]

        \end{tikzpicture}
    \ex\label{ex:aboh:14.5b}
        \begin{tikzpicture}[baseline=(root.base)]

            \Tree 	[.\node(root){InterP};
                        [.\node(finp){FinP};
                            \edge[roof]; {Kòfí ɖù às\'ɔn \'ɔ}
                        ]
                        [.Inter$'$
                            [.Inter
                                à
                            ]
                            \node (t) {\sout{FinP}};
                        ]
                    ]

            %\draw [->] (t.south) -- +(0,-1.0) -- +(-2.875,-1.0) -- +(-2.875,0);
            \node (fin) [below=.5cm of finp] {};
            \draw [arrow] (t.south)..controls +(south:2.0)
                and +(south east:1.75)..(fin.south);

        \end{tikzpicture}
    \z
    \end{multicols}}
\z

It appears from the examples in \REF{ex:aboh:14.3} and \REF{ex:aboh:14.4} that
the Gbe languages, like many Niger-Congo, display disharmonic structures, as
represented in (\ref{ex:aboh:14.2}c) and (\ref{ex:aboh:14.2}d), in various components of their grammar (e.g., TP,
CP, PP). Likewise, studies on creole languages have shown that some creole
languages, which emerged from the contact between Gbe languages and \ili{French}
(e.g., \ili{Haitian Creole}), or Gbe languages and \ili{English} (e.g., Sranan,
Saramaccan), exhibit similar disharmonic structures in areas of their grammar.
Together these facts suggest that such apparent violations of \gls{FOFC}\is{final-over-final condition} are
not isolated phenomena, and therefore require some explanation. Such an
explanation can only be based on a precise description of the facts. In what
follows, I take this first step and illustrate the main contexts in which
Gungbe apparently violates FOFC, and provide comparable examples in
\ili{Haitian Creole} and Suriname creoles (e.g., Sranan and Saramaccan). These
creoles emerged in the 17th century colonial plantations in Suriname and Haiti
where thousands of enslaved African speakers of Niger-Congo languages were
deported to the Americas and came into contact with the languages of European
their colonists, namely \ili{French} in Haiti and \ili{English} and \ili{Dutch} in Suriname.

\section{Initial-over-final in Gbe}

\citet{Aboh2010c} reports that \ili{Gungbe} involves two types of adpositions
labelled P1 and P2. Elements of the type P1 generally derive from posture or
locative verbs, while items of the type P2 derive from nouns expressing
landmarks or body-parts. P1 projects a head-initial structure as indicated in
\REF{ex:aboh:14.6a}. P2 on the other hand projects an apparent head-final
structure as in \REF{ex:aboh:14.6b}. When P1 and P2 co-occur, P1 must precede
the phrase headed by P2, as indicated by example \REF{ex:aboh:14.6c} further
described in \REF{ex:aboh:14.6d}.

\ea\label{ex:aboh:14.6} \ili{Gungbe}
    \ea\label{ex:aboh:14.6a}
        \gll    Súrù zé kw\'ɛ [ xlán mì ]. \\
                Suru take money {} P1 \Fsg{} {} \\
        \glt    \enquote*{Suru sent me some money.}
    \ex\label{ex:aboh:14.6b}
        \gll    Súrù x\'ɛ [ só l\'ɔ jí ]. \\
                Suru climb {} hill \Det{} P2 {} \\
        \glt    \enquote*{Suru climbed on top of the hill.}
    \ex\label{ex:aboh:14.6c}
        \gll    Súrù nyìn àgán [ xlán [ só l\'ɔ jí ]]. \\
                Suru throw stone {} P1 {} hill \Det{} P2 {} \\
        \glt    \enquote*{Suru threw a stone on top of the hill.}
    \ex\label{ex:aboh:14.6d}
        \begin{tikzpicture}[baseline=(root.base)]

            \Tree 	[.\node(root){P1P};
                        [.P1 xlán ]
                        [.P2P
                            [.DP
                                \edge[roof]; {só l\'ɔ}
                            ]
                            [.P2 jí ]
                        ]
                    ]

        \end{tikzpicture}
    \z
\z

Note that in this example, both the DP inside P2P and P2P itself display a
head-final structure embedded under the head-initial P1P. \citet{Biberauer2016}
discusses these examples and concludes that the determining factors allowing
these apparent \gls{FOFC}\is{final-over-final condition} violations could be the lower structural position of P2
compared to P1 as represented in \REF{ex:aboh:14.6d}. Furthermore, P1 and P2 are
categorially distinct: the former developed from verbs, while the latter
developed from landmark nouns (cf. \citealt{Aboh2010c}). While this view is
plausible, one would need to find out how it squares with \citegen{Aboh2010c}
subsequent suggestions that elements of the type P2 should be analysed as
heading a predicate within a possessive phrase (which according to him is
typical of such locative expressions).  The idea being that a sequence like
\emph{só l\'ɔ jí} in \REF{ex:aboh:14.6b} should be analogised to \emph{the mountain
top} in \ili{English}, in which \emph{jí}, expressing P2, heads a possessive
predicate. If this view is correct and if we maintain the notion of extended
projection as argued for in SBRH (2017), then both P1 and P2 belong to the same
extended projection, and we would have to demonstrate how they are categorially
distinct.

\section{Final-over-initial in Gbe}\largerpage

The discussion above about the yes--no question particle already showed that Gbe
languages involve instances of final-over-initial disharmonic orders within the
clausal left periphery (cf. \citealt{Aboh2016a} for further discussion). In
what follows, I show that similar disharmonic orders are found within the TP
too. In \ili{Fongbe}, for instance, the so-called completive aspect can be expressed
by complex structures in which two apparent verbs circumvent an object
(cf.~\citealt{DaCruz1995,Aboh2009,VandenBergAboh2013}).

\ea\label{ex:aboh:14.7} \ili{Fongbe} \parencite[363]{DaCruz1995}
    \multicolsep=.25\baselineskip
    \begin{multicols}{2}
    \ea\label{ex:aboh:14.7a}
        \gll    K\`ɔkú wà àz\v{ɔ} \'ɔ fó \\
                Koku do work \Det{} finish\\
        \glt    \enquote*{Koku finished doing the work.}
    \ex\label{ex:aboh:14.7b}
        \gll    K\`ɔkú ɖù m\`ɔlìnkún \'ɔ v\`ɔ \\
                Koku eat rise \Det{} finish\\
        \glt    \enquote*{Koku finished eating the rice.}
    \z
    \end{multicols}
\z

Under the assumption that the final verb is comparable to an auxiliary\is{auxiliaries} or
aspect marker of some sort, these sequences would be akin to [VO]-Aux order
which is banned in \ili{Germanic} \parencite[cf.][173]{BibHolRob2014}.
\Citet{DaCruz1995} analysed these constructions as instances of serial verb
constructions arguing that, in these constructions, the final V is a
lexical verb with the same thematic properties as in the examples in
\REF{ex:aboh:14.8} in which these verbs select for an internal argument.

\ea\label{ex:aboh:14.8} \ili{Fongbe} \parencite[363]{DaCruz1995}
    \multicolsep=.25\baselineskip
    \begin{multicols}{2}
    \ea\label{ex:aboh:14.8a}
        \gll    K\`ɔkú fó àz\v{ɔ} \'ɔ \\
                Koku finish work \Det{} \\
        \glt    \enquote*{Koku finished the work.}
    \ex\label{ex:aboh:14.8b}
        \gll    K\`ɔkú v\`ɔ m\'ɔlìnkún \'ɔ \\
                Koku finish rice \Det{} \\
        \glt    \enquote*{Koku finished the rice.}
    \z
    \end{multicols}
\z

In recent work, however, \citet{VandenBergAboh2013} argue that these
constructions should be analysed similarly to equivalent constructions in
Gungbe which do not involve two apparent verbs and in which the final position
is realised by the quantifier meaning \textit{kpó} ‘all’.

\ea\label{ex:aboh:14.9} \ili{Gungbe}
    \multicolsep=.25\baselineskip
    \begin{multicols}{2}
    \ea\label{ex:aboh:14.9a}
        \gll    Dónà wà àz\'ɔn kpó \\
                Dona do work all \\
        \glt    \enquote*{Dona did the work completely.}, \enquote*{Dona did
                all the work.}
    \ex\label{ex:aboh:14.9b}
        \gll    Dónà ɖù l\'ɛsì l\'ɔ kpó \\
                Dona eat rice \Det{} all \\
        \glt    \enquote*{Dona ate the rice completely.}, \enquote*{Dona ate
                all the rice.}
    \z
    \end{multicols}
\z

In terms of this proposal, the Gbe languages involve a TP-internal functional
projection that expresses event quantification and may be spelled out by a verb
root or a quantifier root that merges in its head. Under this view therefore,
the \ili{Fongbe} and \ili{Gungbe} sentences in \REF{ex:aboh:14.7a} and \REF{ex:aboh:14.9a},
respectively, can be described as in \REF{ex:aboh:14.10} in which the event
quantifier merges under F and takes a head-initial VP.

\ea\label{ex:aboh:14.10}
    \begin{tikzpicture}[baseline=(root.base)]
        \Tree 	[.\node(root){FP};
                    {}
                    [.F$'$
                        [.VP
                            {}
                            [.V$'$
                                [.V wà ]
                                [.DP {àz\'ɔn / àz\v{ɔ}} ]
                            ]
                        ]
                        [.F {fó / kpó} ]
                    ]
                ]

    \end{tikzpicture}
\z

If representation \REF{ex:aboh:14.10} corresponded to the underlying structure
then this and similar examples would be genuine violations of FOFC.
Alternatively, however, one can argue along the lines of
\citet{VandenBergAboh2013} that the functional element heading event
quantification is head-initial, but its complement must move leftward,
presumably to its specifier position, as in \REF{ex:aboh:14.11}. In terms of
\textcite{Aboh2004a,Aboh2004b,Aboh2010a}, this event quantifier head belongs to
the class of markers in Gbe whose complements must raise to their specifier
position.

\ea\label{ex:aboh:14.11}
    \begin{tikzpicture}[baseline=(root.base)]
        \Tree 	[.\node(root){FP};
                    [.VP
                            {}
                            [.V$'$
                                [.V wà ]
                                [.DP {àz\'ɔn / àz\v{ɔ}} ]
                            ]
                        ]
                    [.F$'$
                        [.F {fó / kpó} ]
                        \sout{VP}
                    ]
                ]

    \end{tikzpicture}
\z

Under this view and assuming that Gbe languages are underlyingly head-ini\-tial
no issue arises, but this conclusion is not immediately obvious if we assume
FOFC and if linearization is not part of core syntax.

\section{FOFC in language contact and change}\largerpage

The examples discussed thus far indicate that Gbe languages involve the
disharmonic orders in (\ref{ex:aboh:14.2}c) and (\ref{ex:aboh:14.2}d). These languages therefore seem to violate
FOFC, on the surface. As suggested in previous paragraphs, one could
hypothesise that such apparent violations of \gls{FOFC}\is{final-over-final condition} are unstable in contact
situation because \gls{FOFC}\is{final-over-final condition} constrains structure building.  Alternatively, one could
also imagine that the process being so robust in Gbe (and other Kwa), prevails
in contact situations involving Gbe or similar Niger-Congo languages and
European languages such as \ili{French} or \ili{English}. It is the latter scenario that
characterizes certain Atlantic creoles. These new languages display disharmonic
orders in areas of their grammar in a way comparable to Gbe. This is the case
in Haitian Creole spoken in Haiti, Sranan and Saramaccan spoken in Suriname.
These languages developed in the Caribbean in the late 17th and early 18th
century during European colonial expansion (cf. \citealt{Aboh2015} and
references cited there).
%
We now face the crucial question of why, during acquisition\is{language acquisition} in such
multilingual contexts, disharmonic structures win over harmonic ones even
though the computational system favours the latter.

\subsection{Initial-over-final within PP: Sranan}

Just as Gbe languages exhibit P1 and P2 categories with apparent different
headness properties, one finds equivalent adpositions in Early Sranan
\REF{ex:aboh:14.11b}, as well as in other Suriname creoles (cf. \citealt{Bruyn2003}
and references cited there).

\ea\label{ex:aboh:14.11b} Sranan \parencite[32]{Bruyn2003}
    \sn\gll Sinsi a komm \textit{na} hosso \textit{inni} \dots{} \\
            since \Tsg{} come P1 house P2 {} \\
    \glt    \enquote*{Since she entered the house \dots{}}
\z

The surface string in \REF{ex:aboh:14.11b} indicates that like in Gbe, Sranan P1 is
head-initial and takes a complement which is head-final.
\citeauthor{Aboh2010c} (\citeyear{Aboh2010c}, \citeyear{Aboh2015},
\citeyear{Aboh2016b}, \citeyear{Aboh2017}) discusses these patterns as
well as other varying word orders found within the PP in these creoles and
shows how they derive from a recombination of syntactic features selected from
Gbe-languages and from \ili{English}.

\subsection{Final-over-initial within the DP: Haitian Creole}

Similar recombination is found within the DP in \ili{Haitian Creole}
\parencite{AbohDeGraff2014,Aboh2015}. This language exhibits both prenominal
and postnominal adjectives. The definite/specificity marker must follow the
noun phrase, while the indefinite marker \emph{yon} must precede:

\ea\label{ex:aboh:14.12} Haitian Creole \parencite[117--118]{DeGraff2007}
    \ea
        \gll    Nana vann gwo wòb la \\
                Nana sell big dress \Det{} \\
        \glt    \enquote*{Nana sold the big dress.}
    \ex
        \gll    Nana vann wòb jòn la \\
                Nana sell dress yellow \Det{} \\
        \glt    \enquote*{Nana sold the yellow dress.}
    \ex\label{ex:aboh:14.12c}
        \gll    Mwen te wè yon moun \\
                \Fsg{} \Ant{} see \Det{} person \\
        \glt    \enquote*{I saw someone.}
    \z
\z

Clearly, the distribution of adjectives in \ili{Haitian Creole} is similar to\largerpage
that of \ili{French} adjectives. Under \citet{Cinque2010}, \ili{French} and other \ili{Romance}
languages which exhibit similar distributive properties involve head-initial
structures and the relative position of adjectives (i.e., pre- vs post-nominal
adjective) is derived by N(P)-movement. Taking this as our starting point, it
must be the case that the post-nominal determiner-like element in Haitian
Creole dominates a head-initial structure. This view is further corroborated by
the fact that unlike adjectives, possessive pronouns, demonstratives as well as
the number marker follow the Gbe head-final order as illustrated by example
\REF{ex:aboh:14.13}.

\ea\label{ex:aboh:14.13}
    \ea Haitian \parencite[78]{Lefebvre1998}\\
    \sn
        \gll    krab mwen sa a yo\\
                crab \First{}.\Poss{} \Dem{} \Det{} \Pl{} \\
        \glt    \enquote*{these crabs of mine}
    \ex \ili{Gungbe} \parencite{Aboh2004a,Aboh2004b}
    \sn
        \gll    àgásá cè éhè l\'ɔ l\'ɛ \\
                crab \First{}.\Poss{} \Dem{} \Det{} \Pl{} \\
        \glt    \enquote*{these crabs of mine}
    \z
\z

Yet, example \REF{ex:aboh:14.12c} clearly shows that the indefinite determiner must
precede the noun, suggesting a head-initial pattern similar to \ili{French}
\textit{une} \textit{personne} ‘a person’. Again, what we see here is a
recombination of the Gbe disharmonic order with \ili{French} harmonic order with
mixed headness properties, leading to apparent FOFC-violations.

\subsection{Final-over-initial within TP: Sranan}

In the preceding paragraphs, I showed that \ili{Gungbe}, and Gbe languages in
general, involve event quantifiers which, on the surface, seem to exhibit a
head-final structure, even though they select a head-initial VP complement.
Similar constructions are found in the Suriname creoles as well. An example
from early Sranan is given in \REF{ex:aboh:14.14} in which the so-called completive
marker, \textit{keba}, follows the verb.

\ea\label{ex:aboh:14.14}Sranan
    \sn\gll yu syi tok, nownowdei mi leri \textit{keba} taki a \enquote*{oe} musu de ini wan lo geval wan \enquote*{u}. \\
            \Tsg{} see yet now.\Red{}-day \Fsg{} learn already that the \enquote*{oe} must be every one \textsc{lo} case a \enquote*{u} \\
    \glt    \textquoteleft{}You see, right, nowadays I have learned (I know)
            that the \enquote*{oe} must be (written) as \enquote*{u} in any
            case.\textquoteright{}
\z

These constructions are discussed in \citet{VandenBergAboh2013} who propose
an \gls{LCA}\is{linear correspondence axiom} account in the lines of representation \REF{ex:aboh:14.11} above. In terms
of this analysis, \textit{keba} (also realised sometimes as \textit{kba},
\textit{kaba}) is equivalent to the Gbe event quantifiers, in that it heads a
functional projection within TP that takes the VP preceding it as complement.
The latter must raise to [spec FP] to be licensed as described in
\REF{ex:aboh:14.11}.

The preceding paragraphs show that the Gbe languages and some creoles involve a
significant body of syntactic patterns which systematically violate \gls{FOFC}\is{final-over-final condition} on the
surface. These patterns are found within the determiner phrases, adpositional
phrases, tense or aspect phrases as well as within the complementizer\is{complementizers} system.
With regard to aspect phrases, for instance, the discussion on event
quantifiers suggests that these languages involve some event quantifier that
can project above the VP and surface as head-final structure even though the
embedded VP is head-initial. Assuming that these event quantifiers are
aspectual in nature (as commonly accepted in the literature), they are
comparable to aspect markers which, in many languages, are expressed by various
auxiliaries. Accordingly, we reach the description that these languages appear
to exhibit the order [VO]--Aux/Asp in which a head-initial VP precedes an
aspect marker or auxiliary\is{auxiliaries} which appears to be head-final. Since it is the
absence of the [VO]-Aux order in \ili{Germanic} which led to the postulation of
\gls{FOFC} (cf. SBRH 2017), one wonders why these languages display a sequence
in surface form that is banned in Germanic? If the ban in \ili{Germanic} holds on
surface form, why does it not apply to Gbe and similar languages as well? Given
such sharp discrepancies between Gbe languages (Niger-Congo), some creoles, and
Germanic, the question arises what fundamental aspect of Human Language
Capacity explains FOFC, and the observed cross-linguistic variation. Theresa
Biberauer’s chapter in SBRH (2017) addresses some of these questions, but I
hope that the data provided here will allow further research in this domain.

\printchapterglossary{}

%\section*{Acknowledgements}

{\sloppy
\printbibliography[heading=subbibliography,notkeyword=this]
}

\end{document}




% % copy the lines above and adapt as necessary

%%%%%%%%%%%%%%%%%%%%%%%%%%%%%%%%%%%%%%%%%%%%%%%%%%%%
%%%                                              %%%
%%%             Backmatter                       %%%
%%%                                              %%%
%%%%%%%%%%%%%%%%%%%%%%%%%%%%%%%%%%%%%%%%%%%%%%%%%%%%

% There is normally no need to change the backmatter section
\backmatter 
\phantomsection 
\addcontentsline{toc}{chapter}{\lsIndexTitle} 
\addcontentsline{toc}{section}{\lsNameIndexTitle}
\ohead{\lsNameIndexTitle} 
\printindex 
\cleardoublepage
  
\phantomsection 
\addcontentsline{toc}{section}{\lsLanguageIndexTitle}
\ohead{\lsLanguageIndexTitle} 
\printindex[lan] 
\cleardoublepage
  
\phantomsection 
\addcontentsline{toc}{section}{\lsSubjectIndexTitle}
\ohead{\lsSubjectIndexTitle} 
\printindex[sbj]
\ohead{} 

\end{document} 

% you can create your book by running
% xelatex main.tex
%
% you can also try a simple 
% make
% on the commandline
