\chapter{Conclusion}
\label{chapter15}
\setcounter{enums}{0}

\section{Summary}
\label{15:sec:summary}


The present study began with key motivations laid out
Chapter~\ref{chapter1} for the creation of a computational model of
information structure. Chapter~\ref{2:notes} offered preliminary notes
for understanding the current work.

The first part (Chapters \ref{chapter3} to \ref{chapter5}) scrutinized
meanings and markings of information structure from a cross-linguistic
standpoint. Information structure is composed of four components:
\isi{focus}, \isi{topic}, \isi{contrast}, and background. Focus
identifies that which is important and/or new in an utterance, which
cannot be removed from the sentence. Topic can be understood as what
the speaker is speaking about, and does not necessarily appear in a
sentence (unlike focus). Contrast applies to a set of
alternatives,\is{alternative set} which can be realized as either
focus or topic. Lastly, \isi{background} is defined as that which is
neither focus nor topic. There are three means of expressing
information structure: prosody,\is{prosody} lexical
markers,\is{lexical markers} and syntactic positioning.\is{syntactic
  positioning} Among them, the current work has been largely concerned
with the last two means, leaving room for improvement in modeling the
interaction between prosody and information structure as further work.
There are three lexical types responsible for marking information
structure: affixes, adpositions, and modifiers
(e.g.\ clitics).\is{adposition} Canonical positions of focus include
\isi{clause-initial}, \isi{clause-final}, \isi{preverbal}, and
\isi{postverbal}. Building upon these fundamental notions,
Chapter~\ref{chapter5} looked into several cases in which
discrepancies in form-meaning mapping of information structure happen.


The second part (Chapters \ref{chapter8} to \ref{chapter10-4})
proposed using \isi{ICONS} (Individual CONStraints)\is{Individual
  CONStraints} for representing information structure in MRS
(\citealt{copestake:etal:05}).\is{MRS} This was motivated by three
factors.  First, information structure markings should be
distinguished from information structure meanings in order to solve
the apparent mismatches between them. Second, the representation of
information structure should be
underspecifiable,\is{underspecification} because there are many
sentences whose information structure cannot be conclusively
identified in the context of sentence-level, text-based
processing. Third, information structure should be represented as a
\isi{binary relation} between an individual and a clause. In other
words, information structure roles should be filled out as being in a
relationship with the clause a constituent belongs to, rather than as
a property of a constituent itself. In order to meet these
requirements, three type hierarchies were suggested; \tdl{mkg},
\tdl{sform},\is{sentential forms} and most importantly
\tdl{info-str}.\is{\textit{info-str}}\is{\textit{mkg}}\is{\textit{sform}}
In addition to them, two types of flag features, such as L/R-PERIPH
and LIGHT,\is{lightness} were suggested for configuring \isi{focus}
and \isi{topic}.\is{L-PERIPH}\is{R-PERIPH} Using hierarchies and
features, the remaining chapters addressed multiclausal utterances and
specific forms of expressing information structure. Furthermore,
Chapter~\ref{chapter10-4} calculated focus projection via
ICONS.\is{focus projection}


The third part (Chapters \ref{chapter11} to \ref{chapter12}) created a
customization system for implementing information structure within the
\lingo \isi{Grammar Matrix} (\citealt{bender:etal:10}) and examined
how information structure improved transfer-based multilingual machine
translation.\is{transfer-based} Building on cross-linguistic and
corpus-based findings, a large part of HPSG/MRS-based\is{HPSG}\is{MRS}
constraints presented thus far was implemented in \isi{TDL}. A
web-based questionnaire was designed in order to allow users to
implement information structure constraints within the
\texttt{choices} file. Common constraints across languages were added
into the Matrix core (\texttt{matrix.tdl}), and language-specific
constraints were processed by Python scripts and stored into the
customized grammar. Evaluations of this library using regression tests
and \isi{Language CoLLAGE} \citep{bender:14} showed that this library
worked well with various types of languages.\is{regression test}
Finally, an experiment of multilingual machine translation verified
that using information structure reduced the number of infelicitous
translations dramatically.





\section{Contributions}
\label{15:sec:contributions}

The present study holds particular significance for general theoretic
studies of the grammar of information structure. Quite a few languages
are surveyed to capture cross-linguistic generalizations about
information structure meanings and markings, which can serve as an
important milestone for typological research on information structure.


The present study also makes a contribution to HPSG/MRS-based
studies\is{HPSG}\is{MRS} by enumerating strategies for representing
meanings and markings of information structure within the formalism in
a comprehensive and fine-grained way.  Notably, the present study
establishes a single formalism for representation and applies this
formalism to various types of forms in a straightforward and cohesive
manner. Moreover, the current model addresses how information
structure can be articulated within the HPSG/MRS framework and
implemented within a computational system in the context of
\isi{grammar engineering}.



The present study also shows that information structure can be used to
produce better performance in natural language processing systems. My
firm opinion is that information structure contributes to multilingual
processing; languages differ from each other not merely in the words
and phrases employed but in the structuring of information. It is my
expectation that this study will inspire future studies in
computational linguistics to pay more attention to information
structure.


Last but most importantly, the present model makes a contribution to
the \lingo \isi{Grammar Matrix} library. The actual library makes it
easy for other developers to adopt and build on my analyses of
information structure.  Moreover, the methodology of creating
libraries I employ in this study can be used for other libraries in
the system.  In order to construct the model in a fine-grained way, I
collected cross-linguistic findings about information structure
markings and exploited a multilingual parallel text in four
languages. These two methods are essential in further advancements in
the \lingo framework.





\section{Future Work}
\label{15:sec:future}



First, it is necessary to examine other types of particles responsible
for marking information structure.  Not all \isi{focus} sensitive
items are entirely implemented in \isi{TDL} in the current model even
for \ili{English}.\is{lexical markers} \ili{Japanese} and \ili{Korean}
employ a variety of lexical markers for expressing focus and
\isi{topic}, which are presented in \citet{hasegawa:11} and
\citet{lee:04}. A few focus markers in some languages have positional
restrictions. For example, as shown in Section \ref{4:sec:lexical},
the clitic \textit{tvv} in Cherokee signals focus and the focused
constituent with \textit{tvv} should be followed by other constituents
in the sentence. That is, two means of marking information structure
operate at the same time.  It would be interesting to investigate
these kinds of additional constraints in the future.\is{focus
  sensitive item}


Second, a few more types of constructions related to information
structure will be studied in future work.  The constructions include
echo questions, \textit{Yes}/\textit{No}-questions \citep{king:95},
coordinated clauses \citep{heycock:07}, double nominative
constructions \citep{kim:sells:07,choi:12}, floating quantifiers
\citep{yoshimoto:etal:06,kim:11b}, pseudo clefts
\citep{kim:07},\is{clefting} and \textit{it}-clefts in other languages
in the \isi{DELPH-IN} grammars.


Third,\is{focus projection} the method for computing \isi{focus}
projection in the present study also needs to be more thoroughly
examined. There are various constraints on how focus can be spread to
larger constituents. These are not addressed in the present study,
which looks at the focus projection of only simple sentences in
\ili{English}. The method the present study employs for handling focus
projection could be much reinforced in further studies.


Fourth, it would be interesting for future work to delve into how
scopal interpretation can be dealt with within the framework that the
present study proposes.  Topic has an influence on scopal
interpretation in that topic has the widest scope in a sentence
\citep{buring:97,portner:yabushita:98,erteschik:07}.  MRS employs
HCONS (Handle CONStraints) in order to resolve scope
ambiguity.\is{MRS} Further work can confirm whether HCONS+ICONS is
able to handle the relationship between \isi{topic} and scope resolution.


Finally, the evaluation of multilingual machine translation will be
extended with a large number of test suites. More grammatical
fragments related to \isi{ICONS} will be incorporated into the
\isi{DELPH-IN} resource grammars, such as \isi{ERG} (English Resource
Grammar, \citealt{flickinger:00}), \isi{Jacy}
(\citealt{siegel:etal:16}), \isi{KRG} (\ili{Korean} Resource
Grammar, \citealt{kim:etal:11}), ZHONG  (for the
\ili{Chinese} languages, \citealt{fan:15a,fan:15b}),\is{ZHONG}
\isi{INDRA} (for \ili{Indonesian}, \citealt{moeljadi:15}), and so
forth.

