\addchap{Acknowledgments}

First and foremost, I would like to express my deep gratitude to my
wonderful PhD adviser, Emily M. Bender. She has introduced me to the
study of information structure and provided me with tremendous help in
modeling information structure from a cross-linguistic perspective.


I have received such wonderful support from all of the faculty members
of the Dept.\ of Linguistics, University of Washington. I am deeply
grateful to Toshiyuki Ogihara, Fei Xia, Gina-Anne Levow, Sharon
Hargus, Richard Wright, Ellen Kaisse, Alicia Beckford Wassink, and
Julia Herschensohn. I have received great assistance from Mike Furr
and Joyce Parvi. I am also full of appreciation to fellow students:
Joshua Crowgey, Woodley Packard, Lisa Tittle Caballero, Varya
Gracheva, Marina Oganyan, Glenn Slayden, Maria Burgess, Zina Pozen, Ka
Yee Lun, Naoko Komoto, Sanae Sato, Prescott Klassen, T.J. Trimble,
Olga Zamaraeva, David Inman, and, deeply, Laurie Poulson.


After having completed my PhD, I worked as a research fellow at
Nanyang Technological University in Singapore. The experience at NTU
provided me with the opportunity to improve my understanding of
grammar engineering as well as grammatical theory across languages. I
would like to express special thanks to my great supervisor, Francis
Bond. I have also received such kind assistance from my colleagues at
NTU: Michael Wayne Goodman, Luis Morgado da Costa, Franti\v{s}ek
Kratochvíl, Joanna Sio Ut Seong, Giulia Bonansinga, Chen Bo, Zhenzhen
Fan, David Moeljadi, Tu{\'{\^{a}}}n Anh L\^{e}, Wenjie Wang, and
Takayaki Kuribayashi.


I participated in many helpful discussions with the DELPH-IN
developers. Ann Copestake and Dan Flickinger suggested using
Individual Constraints to represent information structure.  While
discussing this matter with Dan Flickinger, I was able to improve my
analysis on how to model information structure from the perspective of
grammar engineering. Stephan Oepen helped me improve functionality of
the information structure library.  I also had productive discussions
with Antske Fokkens, Tim Baldwin, Berthold Crysmann, Ned Letcher,
Rebecca Dridan, Lars Hellan, and Petya Osenova.



I have also received important aid from many linguists. Stefan
M{\"u}ller provided me such meaningful assistance with my study of
information structure. I had a great opportunity to discuss with 
Nancy Hedberg, which helped me understand better the interaction between 
prosody and information structure in English.  
Yo Sato let me know his previous comparative
study of information structure marking in Japanese and Korean. Bojan
Beli{\'c} helped me understand information structure properties in
Bosnian Croatian Serbian. 
Of course, they do not necessarily agree with my analysis.



I cannot miss my deep appreciation to the editors of this book series
``Topics at the Grammar-Discourse Interface''. Philippa Cook (Chief
Editor) kindly helped me develop my manuscript.  Felix Bildhauer and
the other anonymous reviewer gave me a number of great comments, which
gave me one more chance to improve my idea, though it should be noted
that I could not fully accommodate them in this book.  Thanks to
Sebastian Nordhoff's kind help and other proofreaders' assistance, I
could finish this book.  Additionally, I would like to express thanks
to Anke Holler, Cathrine Fabricius-Hansen. Needless to say, all
remaining errors and infelicities are my own responsibility.



Since becoming interested in linguistic studies, I have been receiving
invaluable guidance from many Korean linguists. Most of all, I would
like to express my respect to my MA adviser, Jae-Woong Choe. If it had
not been for his tutelage, I would not have made such progress in
linguistic studies. Jong-Bok Kim provided me with the opportunity to
participate in the KRG project. This led me to the study of
HPSG/MRS-based language processing. Seok-Hoon You introduced me to the
study of linguistics when I was an undergraduate and opened to door to
this academic field I enjoy today. Eunjeong Oh was my wonderful mentor
when I started my graduate courses. She helped raise me higher than I
could have done on my own. Suk-Jin Chang, Kiyong Lee, and Byung-Soo
Park formed the basis of HPSG-based studies in Korean, on which I
could build up my own understanding of HPSG-based linguistic models. I
would like to thank other Korean linguists who helped me so much:
Ho-Min Sohn, Chungmin Lee, Beom-mo Kang, Chung-hye Han, Hee-Rahk Chae,
Tosang Chung, Sang-Geun Lee, Myung-Kwan Park, Jin-ho Park, Hae-Kyung
Wee, Young Chul Jun, Hye-Won Choi, Eun-Jung Yoo, Minhaeng Lee,
Byong-Rae Ryu, Sae Youn Cho, Jongsup Jun, Incheol Choi, Kyeong-min
Kim, and many others.



After I joined Incheon National University, I have been supported by
the faculty members of the Dept. of English Language and Literature. I
give my thanks to Hyebae Yoo, Hwasoon Kim, Jung-Tae Kim, Yonghwa Lee,
Seenhwa Jeon, Soyeon Yoon, and Hwanhee Park. I also want to express
thanks to Kory Lauzon.




Lastly and most importantly, I would like to say that I love Ran Lee
and Aaron Song so much.


This material is based upon work supported by the National Science
Foundation under Grant No. 0644097. Any opinions, findings, and
conclusions or recommendations expressed in this material are those of
the author and do not necessarily reflect the views of the National
Science Foundation.

 