\chapter{Les conjoints fragmentaires : le gapping} \label{ch2}
\section{Introduction}
\largerpage[2]
Ce chapitre examine en détail les propriétés des phrases fragmentaires dans la coordination, à travers la construction à gapping\footnote{
 \citet{AbeilleEtAl2010} proposent le terme français \textit{construction trouée}. Je garde l’étiquette anglaise, afin de faciliter la lecture.}, terme qui apparaît pour la première fois dans les travaux de \citet{Ross1967,Ross1970}.

Je regroupe sous le nom de gapping les constructions dans lesquelles une séquence de syntagmes apparemment sans tête verbale, mais ayant néanmoins le conte\-nu d’une phrase, se combine avec une phrase complète qui détermine sa forme et son interprétation. Ainsi, en \REF{ch2:ex1} on coordonne une phrase ordinaire (\textit{John eats apples}) et une séquence elliptique (\textit{and Mary pears}) à laquelle manque la tête verbale \textit{eats}. Dans la description de ce phénomène, j’utilise les étiquettes suivantes : la phrase elliptique qui présente un \textit{trou} (angl. \textit{gap}) verbal est une \textit{phrase trouée~}; la phrase complète qui contribue à la reconstruction du contenu dans la phrase trouée est la \textit{phrase source~}; le \textit{trou} ou le \textit{matériel manquant} désigne l’élément qui manque dans la phrase trouée\textit{~}; l’\textit{antécédent} est le matériel présent dans la phrase source, qui contribue à l’interprétation de la phrase trouée ; les \textit{éléments résiduels} (angl. \textit{remnants}) font référence aux constituants qui composent la séquence elliptique ; les \textit{éléments corrélats} sont les constituants parallèles aux éléments résiduels dans la phrase complète. 

\ea
John \uline{eats} apples, and Mary pears. \label{ch2:ex1} 
\z

Depuis les travaux fondateurs de \citet{Ross1967,Ross1970}, le gapping a constitué l’objet de nombreux travaux de recherche\footnote{
 Parmi les plus importants figurent les travaux de \citet{Koutsoudas1971,Jackendoff1971,Maling1972,Hankamer1973,Hudson1976,Hudson1989,Kuno1976,Neijt1979,Siegel1984,Oehrle1987,Jayaseelan1990,Steedman1990,Steedman2000,Gardent1991,Kehler1994,Johnson1996/2004,Johnson2009,Kim1997,Kim1998,Kim2006,Hartmann2000,ZoernerEtAl2000,Coppock2001,Carlson2001,Carlson2002,Lin2002,LopezEtAl2003,CarlsonEtAl2005,Chaves2005,Winkler2005,Reich2006,Hernandez2007,Hoyt2008,Vicente2010,Boone2014}, etc.}. On observe une certaine disproportion quant aux langues étudiées : l’anglais est de loin la langue la plus étudiée, suivi par l’\ili{allemand} \citep{Hartmann2000,Winkler2005,Osborne2006,Reich2006,Repp2009}, le \ili{japonais} et le \ili{coréen} \citep{Kim1997,Kim1998,AbeEtAl1999,Lee2005,Yosuke2009}. D’autres langues représentées sont : le \ili{hollandais} \citep{Neijt1979,Aelbrecht2009}, le \ili{chinois} \citep{Paul1999,Tang2001,RuixiRessy2008}, le \ili{grec} classique \citep{GaetaEtAl2001}, le \ili{latin} \citep{Panhuis1979}, le \ili{russe} \citep{Kazenin2001,Agafonova2014}, le \ili{turc} \citep{Ince2009}, le \ili{persan} \citep{Farudi2013}, et aussi des langues comme le \ili{quechua} \citep{Pulte1971}, le \ili{zapotec} \citep{Rosenbaum1977}, le \ili{dargwa} et le \ili{chuvash} \citep{Kazenin2001}, ou encore les \ili{langues bantoues} \citep{ManusEtAl2011}. En revanche, il y a très peu d’études consacrées au gapping dans les \ili{langues romanes} : \citet{Tran2010} et \citet{Centeno2011} pour l’\ili{espagnol}, et \citet{Zribi-Hertz1986} et \citet{AbeilleEtAl2010} pour le français.

Ce chapitre est une monographie du gapping en roumain, faite dans une approche comparative avec le français (et parfois avec l’anglais). L’avantage d’étu\-dier le roumain est double : d’une part, cela permet de combler une lacune dans la description empirique des phénomènes elliptiques en roumain, et d’autre part, cela permet de vérifier la pertinence des contraintes et des analyses postulées pour le gapping dans d’autres langues, afin d’avoir une perspective générale sur le fonctionnement de ce type d’ellipse dans la grammaire. En particulier, le roumain nous permet de confronter les \isi{contraintes de parallélisme} (tellement discutées pour le gapping) à certaines particularités typologiques (p.ex. \is{ordre de mots}ordre libre des mots, \isi{pro-drop}, \isi{marquage casuel}, conjonctions spéciales, etc.). Pour cela, je m’appuie sur les travaux de \citet{Bilbiie2009}, \citet{AbeilleEtAl2010} et \citet{AbeilleEtAl2014}. 

Les données utilisées dans cette étude sont de natures diverses. Les distributions simples du gapping ne posent aucun problème d’acceptabilité pour les locuteurs. De façon générale, les données utilisées sont ainsi des exemples construits, soumis au jugement des locuteurs natifs. Pour certains aspects cependant, je me sers d'exemples attestés (sur internet ou dans les textes de presse), dont l’acceptabilité a été vérifiée auprès de plusieurs locuteurs natifs. 

\largerpage[-2]
Le chapitre est organisé de la manière suivante : Dans la section~\ref{ch2:sect2.2}, je pré\-sente les critères définitoires pour l’identification correcte du gapping par rapport à d’autres constructions réputées elliptiques (en particulier la coordination de séquences, connue sous le nom de \is{Conjunction Reduction}\textit{Conjunction Reduction} ou encore \is{Argument Cluster Coordination (ACC)}\textit{Argument Cluster Coordination}) dans une perspective typologique. Pour les langues à \is{ordre de mots}ordre des mots relativement libre, comme le roumain, certaines configurations sont syntaxiquement ambiguës, se prêtant a priori à une double analyse selon que la coordination se place au niveau phrastique ou bien à un niveau sous-phrastique. Dans les trois sections qui suivent (section~\ref{ch2:sect2.3}, section~\ref{ch2:sect2.4} et section~\ref{ch2:sect2.5}), je me concentre sur les configurations non ambiguës de gapping en roumain ({\cad} celles dans lesquelles le verbe antécédent dans la phrase source est en position médiane ou finale), en y ajoutant les données du français. Dans la section~\ref{ch2:sect2.3}, j’établis les propriétés du gapping, en discutant dans un premier temps les contraintes générales qui s’appliquent au matériel manquant et aux éléments résiduels, et ensuite les \isi{contraintes de parallélisme} au niveau syntaxique, sémantique et discursif. Je synthétise ensuite dans la section~\ref{ch2:sect2.4} les analyses proposées dans différents cadres théoriques, en insistant sur les problèmes que la plupart de ces analyses n’arrivent pas à résoudre. Ensuite, dans la section~\ref{ch2:sect2.5}, je donne une \is{approche constructionnelle}analyse constructionnelle du gapping, en m’inspirant des \is{approche non structurale}analyses non structurales, qui ne postulent pas de \isi{reconstruction syntaxique} ou \isi{mouvement}. Enfin, après avoir décrit en détail les distributions non ambiguës de gapping, je reviens, dans la section~\ref{ch2:sect2.6}, aux configurations ambiguës en roumain relevées dans la section~\ref{ch2:sect2.2} ({\cad} celles dans lesquelles le verbe se trouve en position initiale), qui se prêtent a priori à deux analyses : gapping ou coordination de séquences. Après avoir invalidé l’hypothèse d’une reconstruction syntaxique, je montre qu’il faut distinguer en roumain les configurations avec la conjonction \textit{iar} ‘et’ des configurations avec la conjonction \textit{şi} ‘et’. Je donne des arguments pour aligner les structures en \textit{iar} ‘et’ sur les cas standard de gapping discutés dans les sections précédentes, on a donc dans ces cas une coordination d’une phrase complète avec un \isi{fragment}. En revanche, les configurations avec la conjonction \textit{şi} ‘et’ restent ambiguës, n’étant pas incompatibles avec une analyse en termes de coordination de séquences dans la dominance syntaxique d’un prédicat verbal. Une analyse formelle est donnée en fin de chapitre pour rendre compte de cette deuxième possibilité.  

\section{Le gapping et l’ordre des mots à travers les langues} \label{ch2:sect2.2}
\largerpage[-2]
De manière générale, les critères minimaux pris en compte pour l’identification des phrases trouées concernent le nombre des éléments résiduels et la catégorie du trou : il s’agit d’une phrase elliptique qui compte au moins deux éléments résiduels et où manque au moins le verbe principal. Le premier critère distingue ainsi le gapping du \is{Stripping}stripping\footnote{\label{ch2:fn3} Dans la littérature, on trouve le terme \is{Stripping}\textit{stripping} (voir aussi la section~\ref{ch1:sect1.4.1.1}), mais distributionnellement il réfère à des constructions hétérogènes, c’est pour cela que \citet{Abeille2006,AbeilleToAppear} utilise des termes différents. Elle distingue le \is{Stripping}stripping \REF{ch2:foot3:exi} des constructions différées \REF{ch2:foot3:exii} et des ellipses polaires \REF{ch2:foot3:exiii}. Dans le premier cas, il n’y a pas d’ellipse, car il y a toujours un adverbe propositionnel qui constitue la tête de la phrase (cet adverbe propositionnel est mis en gras dans les exemples \REF{ch2:foot3:exi}). Dans les deux derniers cas, le deuxième conjoint est elliptique, dans le sens où on doit récupérer une partie du contenu du premier conjoint, pour qu’on puisse interpréter le deuxième conjoint.

\ea \label{ch2:foot3:exi}
\ea Paul viendra, [mais Marie \textbf{non}].
\ex Paul viendra, [mais Marie \textbf{certainement pas}].
\ex  Paul ne viendra pas, [mais Marie \textbf{peut-être}].
\z
\z

\ea \label{ch2:foot3:exii}
\ea Paul viendra, [ou bien Marie].
\ex Paul viendra, [et même Marie].
\ex Paul est venu hier, [mais pas Marie].
\z
\z

\ea \label{ch2:foot3:exiii}
\ea Paul viendra [et Marie aussi].
\ex Paul ne viendra pas [et Marie non plus].
\z
\z
} (constructions différées \REF{ch2:ex2a} ou ellipses polaires \REF{ch2:ex2b}), où la phrase elliptique est réduite à un seul élément résiduel, éventuellement accompagné d’un adverbe. Le deuxième critère sépare le gapping de la construction \is{Pseudogapping}pseudogapping, où il y a deux éléments résiduels, mais accompagnés d’un verbe auxiliaire ou modal \REF{ch2:ex2c}. 

\ea 
\ea  John can play the guitar, but not Mary. \label{ch2:ex2a}  
\ex  John can play the guitar, and Mary too. \label{ch2:ex2b}
\ex  John can play the guitar and Mary can the violin. \label{ch2:ex2c}  
\z
\z

En revanche, ces deux critères ne permettent pas a priori de faire la différence entre le gapping et la coordination de séquences (angl. \is{Conjunction Reduction}\textit{Conjunction Reduction} ou \is{Argument Cluster Coordination (ACC)}\textit{Argument Cluster Coordination}, dorénavant ACC) en \REF{ch2:ex3}, surtout si on les regarde dans une perspective typologique. Par conséquent, dans cette section je présente deux aspects dont on doit tenir compte quand on analyse le gapping. Le premier concerne la directionnalité du gapping à travers les langues et le deuxième la position de la tête verbale.


\ea \label{ch2:ex3}
\ea  We play [poker] [at our house], and [bridge] [at Betsy’s house]. 
\ex {} [At our house] we play [poker], and [at Betsy’s house], [bridge].
\z
\z

\subsection{Directionnalité du gapping : analepse vs. catalepse} \label{ch2:sect2.2.1}

\citet{Ross1970} est le premier à établir une corrélation entre la directionnalité du gapping et \is{ordre de mots}l’ordre des mots dans une langue, en particulier la position de la tête verbale dans la phrase. Ainsi, les langues à tête non finale strictement SVO, comme l’anglais \REF{ch2:ex4} ou le français \REF{ch2:ex5}, ou strictement VSO, comme le \ili{gaélique irlandais} \REF{ch2:ex6}, se caractérisent par la présence de l’\is{analepse (forward ellipsis)}analepse ou ellipse progressive (angl. \textit{forward gapping}), ce qui implique que la phrase source précède toujours la séquence elliptique (SVO+SO, *SO+SVO), alors qu’une langue à tête finale strictement SOV, comme le \ili{japonais} \REF{ch2:ex7} ou le \ili{coréen}, présente la \is{catalepse (backward ellipsis)}catalepse ou l'ellipse régressive (angl. \textit{backward gapping}), où la phrase source suit la séquence elliptique (SO+SOV, *SOV+SO). 

\ea \label{ch2:ex4}
\ea John likes apples and Mary pears.
\ex *John apples and Mary likes pears.
\z
\z

\ea \label{ch2:ex5}
\ea Jean aime les pommes et Marie les poires.
\ex *Jean les pommes et Marie aime les poires.
\z
\z

\ea \label{ch2:ex6}
\il{gaélique irlandais}\langinfo{Gaélique irlandais}{}{\citealt[177]{Steedman2000}}\\
\ea 
\gll Chonaic Eoghan Siobhán agus Eoghnaí Ciarán.\\
voir.\textsc{pst} Eoghan Siobhán et Eoghnai Ciarán\\
\glt ‘Eoghan a vu Siobhán et Eoghnaí Ciarán.’                     

\ex
\gll *Eoghan Siobhán agus chonaic Eoghnaí Ciarán.\\
Eoghan Siobhán et voir.\textsc{pst} Eoghnai Ciarán\\
\glt ‘Eoghan a vu Siobhán et Eoghnaí Ciarán.’ 
\z
\z

\ea \label{ch2:ex7}
\il{japonais}\langinfo{Japonais}{}{\citealt[251]{Ross1970}}\\
\ea 
\gll Watakusi wa sakana o, Biru wa gohan o tabeta.\\ 
\textsc{1sg} \textsc{top} poisson \textsc{acc} Biru \textsc{top} riz \textsc{acc}   manger.\textsc{pst}\\
\glt ‘J’ai mangé du poisson et Biru du riz.’              

\ex 
\gll *Watakusi wa sakana o tabeta, Biru wa gohan o.\\
\textsc{1sg} \textsc{top} poisson \textsc{acc} manger.\textsc{pst} Biru \textsc{top} riz \textsc{acc}\\
\glt ‘J’ai mangé du poisson et Biru du riz.’   
\z
\z

Si la généralisation de \ia{Ross, John Robert}Ross est pertinente pour les langues à tête strictement non finale (voir aussi \citealt{Jackendoff1971}, \citealt{Lobeck1995}, etc.), elle est loin de capter tous les faits empiriques qu’on observe à travers les langues, en particulier le comportement inattendu de certaines langues à tête finale ou encore les différences qu’on peut trouver parmi les langues à \is{ordre de mots}ordre de mots libre. Les langues à tête finale ne rentrent pas toutes dans la généralisation de \ia{Ross, John Robert}Ross. Un premier cas de figure est relevé par \citet{Hernandez2007} pour le \ili{persan} standard, qui est une langue à tête finale, mais qui, contrairement à des langues comme le \ili{japonais} ou le \ili{coréen}, permet uniquement le gapping \is{analepse (forward ellipsis)}progressif, {\cad} SOV+SO, comme on voit en \REF{ch2:ex8}.

\ea \label{ch2:ex8}
\il{persan}\langinfo{Persan}{}{\citealt[2123]{Hernandez2007}}\\
\ea 
\gll Æli sib xord væ Mærzi hulu.\\
Ali pomme a-mangé et Marzo pêche\\
\glt ‘Ali a mangé des pommes et Marzo des pêches.’        

\ex 
\gll *Æli sib væ Mærzi hulu xord.\\
Ali pomme et Marzo pêche a-mangé\\
\glt ‘Ali mange des pommes et Marzo des pêches.’
\z
\z

Un deuxième cas de figure est représenté par l’existence de langues à tête finale qui présentent à première vue les deux directions de gapping~(SO+SOV et aussi SOV+SO) : \ili{basque}, \ili{chuvash}, \ili{hindi}, \ili{punjabi}, \ili{turc}, les subordonnées en \ili{allemand} (cf. \citealt{Maling1972,MallinsonEtAl1981,Kazenin2001,Hernandez2007,Haspelmath2007}, etc.). Je reprends en \REF{ch2:ex9} l’exemple du \ili{basque}, cité par \citet{Haspelmath2007}.

\ea \label{ch2:ex9}
\il{basque}\langinfo{Basque}{}{\citealt[42--43]{Haspelmath2007}}\\
\ea
\gll Linda-k ardau eta Ander-ek esnea edaten dabez. \label{ch2:ex9a}\\
Linda-\textsc{erg} vin.\textsc{abs} et Ander-\textsc{erg} lait.\textsc{abs} boire \textsc{3pl.3sg}\\
\glt ‘Linda va boire du vin et Ander du lait.’

\ex 
\gll Linda-k ardau edaten du, eta Ander-ek esnea. \label{ch2:ex9b}\\
Linda-\textsc{erg} vin.\textsc{abs} boire \textsc{3sg.fut} et Ander-\textsc{erg} lait.\textsc{abs}\\
\glt ‘Linda va boire du vin et Ander du lait.’  
\z
\z

En \ili{allemand}, on observe que dans les phrases racines il s’agit plutôt d’un ordre SVO et donc on a uniquement \is{analepse (forward ellipsis)}l’analepse \REF{ch2:ex10}, alors que dans les subordonnées, il présente à première vue les deux directions de l’ellipse \REF{ch2:ex11}.

\ea \label{ch2:ex10}
\il{allemand}\langinfo{Allemand}{}{\citealt[192]{Crysmann2006}}\\
\ea 
\gll Peter trank Wein, und Marie Bier.\\
Peter a-bu vin et Marie bière\\
\glt ‘Peter a bu du vin et Marie de la bière.’      

\ex 
\gll *Peter Wein, und Marie trank Bier.\\
Peter vin et Marie a-bu bière\\
\glt ‘Peter a bu du vin et Marie de la bière.’
\z
\z

\ea \label{ch2:ex11}
\il{allemand}\langinfo{Allemand}{}{\citealt[192]{Crysmann2006}}\\
\ea 
\gll Ich vermute daß Peter Wein, und Marie Bier getrunken hat.\\
je suppose que Peter vin et Marie bière bu a\\
\glt ‘Je suppose que Peter a bu du vin et Marie de la bière.’      

\ex 
\gll Ich vermute daß Peter Wein getrunken hat, und Marie Bier.\\
je suppose que Peter vin bu a et Marie bière\\
\glt ‘Je suppose que Peter a bu du vin et Marie de la bière.’ 
\z
\z

On a observé le comportement des langues SVO et des langues à tête finale. Mais que se passe-t-il dans les langues à \isi{ordre de mots} libre ? Là aussi, on ne peut pas parler d’homogénéité des faits. Il y a des langues qui peuvent avoir les deux directions du gapping, avec différents ordres possibles, dont le nombre varie d’une langue à l’autre. C’est le cas du \ili{russe}, du \ili{latin} ou du \ili{grec} classique (qui ont au moins les ordres typiques SVO+SO, SO+SOV, SOV+SO), auxquels on peut ajouter le \ili{zapotec} qui est encore plus libre (SVO+SO, SO+SVO, SOV+SO, SO+SVO\footnote{
 \citet{Ross1970} considère qu’il n’y a pas de langue ayant cet ordre dans le gapping. Cette hypothèse s’avère fausse pour le \ili{quechua} (cf. \citealt{Pulte1971,Pulte1973}) et le \ili{zapotec} (cf. \citealt{Rosenbaum1977}).}, OVS+OS, OS+OVS, VSO+SO, *SO+VSO, cf. \citealt{Rosenbaum1977}) ou encore le \ili{tojolabal} qui peut avoir toutes les combinaisons possibles dans les deux sens (cf. \citealt{Furbee1974}). Je me limite à un exemple du \ili{russe} en \REF{ch2:ex12}\footnote{
 Selon Tatiana Philippova (c.p.), on a une préférence nette pour l’emploi de la conjonction \textit{a} ‘et’ au lieu de la conjonction \textit{i} ‘et’ dans tous les exemples en \REF{ch2:ex12}. Cette préférence est attendue dans les constructions à gapping, car le \ili{russe}, comme le roumain, dispose d’une conjonction spécialisée pour le \is{contraste sémantique}contraste multiple, cf. \citet{BilbiieEtAl2011}. Pour plus de détails, voir la section~\ref{ch2:sect2.3.4.4}.}, qui vient de \citet{Ross1970}.

\ea \label{ch2:ex12}
\il{russe}\langinfo{Russe}{}{\citealt[251--252]{Ross1970}}\\
\ea 
\gll Ja pil vodu, i Anna vodku.\\
je bois eau et Anna vodka\\
\glt ‘Je bois de l’eau et Anna de la vodka.’

\ex 
\gll Ja vodu, i Anna vodku pila.\\
je eau et Anna vodka boit.\textsc{fem}\\
\glt ‘Je bois de l’eau, et Anna de la vodka.’

\ex 
\gll Ja vodu pil, i Anna vodku.\\
je eau bois et Anna vodka\\
\glt ‘Je bois de l’eau, et Anna de la vodka.’  
\z
\z

Enfin, on observe qu’il y a des langues qui peuvent avoir (plus ou moins) tous les ordres possibles, mais qui permettent uniquement l’ellipse \is{analepse (forward ellipsis)}progressive (SVO+SO, SVO+OS, SOV+SO, SOV+OS, VSO+SO, VSO+OS, OVS+OS, OVS+SO). C’est le cas du \ili{quechua} bolivien, du \ili{cherokee} (cf. \citealt{Pulte1971,Pulte1973}), etc. J’ajoute ici le roumain qui ne permet pas le gapping \is{catalepse (backward ellipsis)}régressif \REF{ch2:ex13}, mais qui est une langue à \isi{ordre de mots} relativement libre, donc on s’attend à avoir plusieurs ordres qui soient possibles dans les deux conjoints, ce qui s’avère être le cas en \REF{ch2:ex14}: ordre SVO+SO en \REF{ch2:ex14a}, ordre OVS+OS en \REF{ch2:ex14b}, ordre VSO+SO en \REF{ch2:ex14c}, ordre VOS+OS en \REF{ch2:ex14d}, ordre SOV+SO en \REF{ch2:ex14e}\footnote{Les syllabes en majuscules dans tous les exemples de cet ouvrage marquent la \isi{saillance prosodique}, indiquant le \isi{focus informationnel}.} ou bien ordre OSV+OS en \REF{ch2:ex14f}.

\ea \label{ch2:ex13}
\ea
\gll Ion mănâncă mere, iar Maria pere.\\
Ion mange pommes et Maria poires\\
\glt ‘Ion mange des pommes et Maria des poires.’

\ex 
\gll *Ion mere, iar Maria pere mănâncă.\\
Ion pommes et Maria poires mange\\
\glt ‘Ion mange des pommes et Maria des poires.’
\z
\z

\ea \label{ch2:ex14}
\ea 
\gll Ion spală vasele, iar Maria rufele. \label{ch2:ex14a}\\
Ion lave vaisselle.\textsc{def.pl} et Maria linge.\textsc{def.pl}\\
\glt ‘Ion fait la vaisselle et Maria la lessive.’

\ex 
\gll Vasele le spală Ion, iar rufele Maria. \label{ch2:ex14b} \\
vaisselle.\textsc{def.pl} \textsc{acc.3pl.f} lave Ion et linge.\textsc{def.pl} Maria\\
\glt ‘Ion fait la vaisselle, et Maria la lessive.’

\ex 
\gll Mâine va spăla Ion vasele, iar Maria rufele. \label{ch2:ex14c} \\
demain va laver Ion vaisselle.\textsc{def.pl} et Maria linge.\textsc{def.pl}\\
\glt ‘Demain Ion va faire la vaisselle, et Maria la lessive.’

\ex 
\gll Mâine va spăla vasele Ion, iar rufele Maria. \label{ch2:ex14d} \\
demain va laver vaisselle.\textsc{def.pl} Ion et linge.\textsc{def.pl} Maria\\
\glt ‘Demain Ion va faire la vaisselle, et Maria la lessive.’

\ex  
\gll Ion VAsele le spală, iar Maria RUfele. \label{ch2:ex14e} \\
Ion vaisselle.\textsc{def.pl} \textsc{acc.3pl.f} lave et Maria linge.\textsc{def.pl}\\
\glt ‘C’est la vaisselle que Ion fait, et Maria, c’est la lessive qu’elle fait.’

\ex 
\gll Vasele ION le spală, iar rufele MaRIa. \label{ch2:ex14f} \\
vaisselle.\textsc{def.pl} Ion \textsc{acc.3pl.f} lave et linge.\textsc{def.pl} Maria\\
\glt ‘C’est Ion qui fait la vaisselle, et c’est Maria qui fait la lessive.’
\z
\z

En conclusion, on ne peut pas postuler un principe universel quant à la directionnalité du gapping. On observe une tendance à avoir le gapping \is{analepse (forward ellipsis)}progressif plutôt que \is{catalepse (backward ellipsis)}régressif, mais, en l’absence d’une étude empirique adéquate, il est difficile de fournir une explication convaincante. A priori, ce fait est susceptible de deux explications : (i) soit il s’agit d’une contrainte interne à la grammaire, qui dérive du paramètre de la dépendance en syntaxe, (ii) soit la contrainte mise en jeu est plutôt externe à la grammaire et doit être mise en relation avec les faits de \isi{processing}, comme le proposent \citet{MallinsonEtAl1981} ou \citet{GaetaEtAl2001}. Invoquer les facteurs de \isi{processing} expliquerait aussi pourquoi on a tendance à utiliser plutôt l’\isi{anaphore} que la cataphore. Une telle explication est invoquée par \citet[90]{Ramat1987} : «~An ellipsis which refers to a constituent not previously introduced, places a heavy burden on short-term memory. [...] It is thus only natural that gapping of what is contextually known should be preferred.~»  


\subsection{Identification du gapping : ellipse médiane vs. ellipse périphérique} \label{ch2:sect2.2.2} 

Dans ce paragraphe, je présente les critères définitoires du gapping, nous permettant de distinguer cette construction de la coordination de séquences (abrégée \is{Argument Cluster Coordination (ACC)}ACC). 

Comme la plupart des travaux sur l’ellipse ont été consacrés à l’anglais (donc, implicitement aux langues de type SVO), on définit le gapping par rapport à la position du matériel manquant. Par conséquent, on considère souvent que le phénomène du gapping implique un trou en position médiane, {\cad} on a au moins deux éléments résiduels qui ‘encadrent’ le matériel manquant, ce qui distingue le gapping d’autres types d’ellipse \citep{Jackendoff1971,Lobeck1995}\footnote{Contrairement à \citet{Jackendoff1971} et \citet{Lobeck1995}, je considère qu’on ne peut utiliser ce critère pour distinguer le gapping et le \is{Sluicing}sluicing, car on peut trouver des exemples avec \is{Sluicing}sluicing où il y a deux éléments résiduels \REF{ch2:foot7:exi}.
 
\ea \label{ch2:foot7:exi}
\gll Cineva  a  lovit  pe  cineva,  dar  nu  ştiu  cine  pe  cine.\\
quelqu’un  a  frappé \textsc{dom} quelqu’un  mais  \textsc{neg}  savoir.\textsc{prs.1sg}  qui  \textsc{dom}  qui\\
\glt ‘Quelqu’un a frappé quelqu’un d’autre, mais je ne sais pas qui a frappé qui.’
\z
}. \citet{Jackendoff1971}, par exemple, prend ce critère au sérieux et distingue le gapping des structures à \is{Conjunction Reduction}«~réduction de conjoints~» (qui, selon lui, regroupent \is{Argument Cluster Coordination (ACC)}ACC et \is{Right-Node Raising (RNR)}RNR), ces dernières n’ayant pas le matériel manquant en position médiane (dans les \is{Argument Cluster Coordination (ACC)}ACC, il est en position initiale ; dans les \is{Right-Node Raising (RNR)}RNR, il est en position finale). Cependant, pour les langues n’ayant pas un ordre SVO ou pour celles qui ont un \isi{ordre de mots} relativement libre, il n’est pas toujours évident que l’étiquette \textit{gapping} soit le terme approprié pour décrire les faits observés (voir aussi \citealt{Haspelmath2007}). Dans beaucoup de ces langues (pour lesquelles on peut considérer que le sujet et l’objet sont au même niveau dans la structure syntaxique), une séquence d’au moins deux éléments résiduels sans tête verbale se prête a priori à deux analyses possibles, au moins dans certaines de leurs configurations. 

Ce qui nous permet de faire la distinction entre le gapping et une éventuelle coordination de séquences est, à première vue, l’adjacence des éléments (résiduels et corrélats) par rapport à la tête de la phrase complète. Si tous ces éléments suivent ou précèdent la tête, on devrait trouver des arguments empiriques pour décider si c’est du gapping ou bien si c’est une coordination de séquences. Les deux distributions majeures qui se prêtent a priori à l’une ou l’autre des deux constructions sont :

\begin{itemize}
\item[(i)] l’ordre SO+SOV, pour les langues qui permettent la \is{catalepse (backward ellipsis)}catalepse, comme c’est le cas de \il{allemand}l’allemand dans les phrases subordonnées, du \ili{japonais} ou du \ili{coréen}. Voir dans ce sens les discussions de \citet{Lee2005} pour le \ili{coréen}, \citet{Osborne2006} pour \il{allemand}l’allemand, \citet{Yosuke2009} pour le \ili{japonais}. Voir aussi \citet{Kazenin2001} pour des discussions sur le \ili{dargwa} et le \ili{chuvash}.

\item[(ii)] l’ordre VSO+SO, pour les langues qui permettent \is{analepse (forward ellipsis)}l’analepse, voir le \ili{russe} \citep{Kazenin2001}, le roumain \citep{Bilbiie2010}, etc.  
\end{itemize}

En revanche, le terme de gapping semble approprié (sans confusion possible avec la coordination de séquences) pour trois distributions à travers les langues : 

\begin{itemize}
 \item[(i)] position médiane de la tête avec les deux directions du gapping (p.ex. SVO+ SO, SO+SVO, etc.) ; 
\item[(ii)] position finale de la tête, corrélée avec \is{analepse (forward ellipsis)}l’analepse (p.ex. SOV+SO) ;
\item[(iii)] position initiale de la tête, corrélée avec la \is{catalepse (backward ellipsis)}catalepse (p.ex. SO+VSO).
\end{itemize}

 \newpage
Je vais illustrer les problèmes liés à l’identification du gapping en utilisant l’exemple de \il{allemand}l’allemand. \citet{Haspelmath2007} donne trois exemples en \ili{allemand}, avec à chaque fois un ordre différent. Selon les distributions inventoriées ci-dessus, on va considérer \REF{ch2:ex16b} comme un cas incontestable de gapping, mais quel type envisager pour l’exemple \REF{ch2:ex15} ou bien \REF{ch2:ex16a} : gapping ou \is{Argument Cluster Coordination (ACC)}ACC\footnote{
Pour les subordonnées en \ili{allemand}, \citet{Maling1972} donne deux arguments montrant qu’on a affaire à deux constructions différentes en \REF{ch2:ex16a} et \REF{ch2:ex16b} : (i) on ne peut pas élider uniquement \is{auxiliaire}l’auxiliaire dans la distribution SOV+SO, alors que cela est possible pour la distribution SO+SOV ; (ii) l’ordre SO+SOV exige des contraintes de \isi{linéarisation} plus fortes que l’ordre SOV+SO. Voir aussi \citet{Ince2009} qui considère que \is{analepse (forward ellipsis)}l’analepse en \ili{turc} doit être analysée différemment de la \is{catalepse (backward ellipsis)}catalepse (un de ses arguments est le fait que la \is{catalepse (backward ellipsis)}catalepse ne permet qu’une identité parfaite entre l’antécédent et le matériel manquant).} ?

\ea \label{ch2:ex15}
\il{allemand}\langinfo{Allemand}{}{\citealt[44]{Haspelmath2007}}\\
\gll Liebt Julia Romeo und Kleopatra Cäsar ?\\
aime Julia Romeo et Kleopatra Cäsar\\
\glt ‘Est-ce que Juliette aime Roméo et Cléopâtre César ?’   
\z 

\ea \label{ch2:ex16}
\il{allemand}\langinfo{Allemand}{}{\citealt[43]{Haspelmath2007}}\\
\ea 
\gll ... dass Georg Wein und Barbara Bier trinkt. \label{ch2:ex16a}\\
... que Georg vin et Barbara bière boit\\
\glt ‘... que Georg boit du vin et Barbara de la bière.’            

\ex 
\gll ... dass Georg Wein trinkt und Barbara Bier. \label{ch2:ex16b}\\
... que Georg vin boit et Barbara bière\\
\glt ‘... que Georg boit du vin et Barbara de la bière.’                  
\z
\z

\is{accord}L’accord permet de définir un test. Ainsi,~\is{accord}l’accord au pluriel peut être un argument pour considérer la structure en question plutôt comme une coordination de séquences, comme c’est le cas en \ili{dargwa}\footnote{
 Voir aussi l’exemple \REF{ch2:ex9a} ci-dessus en \ili{basque}, qui se prête a priori à la même analyse.}, une langue ayant l’ordre SO+SOV en \REF{ch2:ex17}.

\ea \label{ch2:ex17}
\il{dargwa}\langinfo{Dargwa}{}{\citealt{Kazenin2001}}\\
\gll dul mutal, dil rasul malHal[Qalalij \{\textbf{b-}atalRibda {\textbar} \textbf{*w}-atalRibda\}. \\
\textsc{1erg} Mutal.\textsc{abs} \textsc{2erg} Rasul.\textsc{abs} à-Makhachkala 
\{\textsc{1pl}-envoyer.\textsc{pst} {\textbar} \textsc{1sg}-envoyer.\textsc{pst}\}\\
\glt ‘J’ai envoyé Mutal à Makhachkala, et toi, Rasul.’            
\z

\newpage 
Mon hypothèse semble être confirmée par les données du \ili{russe} et du \ili{chuvash}, reprises de \citet{Kazenin2001}. En \ili{russe}, avec l’ordre VSO+SO, on a \is{accord}l’accord au pluriel \REF{ch2:ex18b}, alors que l’ordre SVO+SO présente plutôt un \isi{accord} au singulier \REF{ch2:ex18a}. Le \ili{chuvash}, langue SOV qui permet et \is{analepse (forward ellipsis)}l’analepse et la \is{catalepse (backward ellipsis)}catalepse, présente une variation \is{accord}d’accord (singulier ou pluriel) avec l’ordre SO+SOV \REF{ch2:ex19b}, mais uniquement \is{accord}l’accord au singulier pour l’ordre SOV+SO \REF{ch2:ex19a}\footnote{
 Pour d’autres asymétries entre \is{analepse (forward ellipsis)}l’analepse et la \is{catalepse (backward ellipsis)}catalepse en \ili{chuvash}, voir \citet{Kazenin2001}.}. Par conséquent, \is{accord}l’accord au pluriel est possible uniquement si le verbe précède (ou suit) les deux conjoints.

\ea
\il{russe}\langinfo{Russe}{}{\citealt{Kazenin2001}}\\
\ea 
\gll Kolja  \{poedet {\textbar} *poedut\}  zavtra  v  Moskvu,  a  Vasja  v  Peterburg. \label{ch2:ex18a}\\
Kolja  \{aller.\textsc{fut.}\textbf{\textsc{sg}} {\textbar} aller.\textsc{fut.}\textbf{\textsc{pl}}\}  demain  à  Moscou  et  Vasja  à  Petersburg\\
\glt ‘Kolja ira demain à Moscou, et Vasja à Saint Petersburg.’

\ex 
\gll Zavtra  \{poedut {\textbar} *poedet\}: Kolja  v  Moskvu,  a  Vasja  v  Peterburg. \label{ch2:ex18b}\\
demain \{aller.\textsc{fut.}\textbf{\textsc{pl}} {\textbar} aller.\textsc{fut.}\textbf{\textsc{sg}}\}  Kolja  à  Moscou  et  Vasja  à  Petersburg\\
\glt ‘Demain Kolja ira à Moscou et Vasja à Saint Petersburg.’  
\z
\z

\ea
\il{russe}\langinfo{Russe}{}{\citealt{Kazenin2001}}\\
\ea 
\gll Vasja  KanaS-a  \{kaja-T {\textbar} *kaja-C-C-e\},  Petja  SupaSkar-a. \label{ch2:ex19a}\\
Vasja  Kanash-\textsc{dat} \{aller-\textsc{prs.3}\textbf{\textsc{sg}} {\textbar} aller-\textsc{prs-}\textbf{\textsc{pl}}-3\} Petja  Cheboksary-\textsc{dat}\\
\glt ‘Vasja va à Kanash, et Petja à Cheboksary.’

\ex
\gll Vasja  KanaS-a,  Petja  SupaSkar-a  \{kaja-T {\textbar} kaja-C-C-e\}. \label{ch2:ex19b}\\
Vasja  Kanash-\textsc{dat}  Petja  Cheboksary-\textsc{dat} \{aller-\textsc{prs.3}\textbf{\textsc{sg}} {\textbar} aller-\textsc{prs-}\textbf{\textsc{pl}}-3\}\\
\glt ‘Vasja va à Kanash, et Petja à Cheboksary.’ 
\z
\z

Est-ce que le gapping pose un problème d’identification en français et en rou\-main, les deux étant compatibles uniquement avec \is{analepse (forward ellipsis)}l’analepse ?

\newpage 
Comme le français est une langue à \is{ordre de mots}ordre des mots fixe, {\cad} généralement SVO, le trou se trouve dans la plupart des cas en position médiane. Cependant, il y a les cas \is{inversion du sujet}d’inversion du sujet \REF{ch2:ex20} ou encore les séquences à deux compléments \REF{ch2:ex21}, qui peuvent se prêter a priori à une double analyse. Voir dans ce sens l’analyse de \citet{Mouret2007,Mouret2008} et la discussion dans la section~\ref{ch2:sect2.6}.

\ea
Alors surgit d’un champ un renard et d’un buisson une biche. \label{ch2:ex20}
\z

\ea \label{ch2:ex21}
\ea Paul apportera un disque à Marie et un livre à Jean. 
\ex A Marie Paul apportera un disque et à Jean un livre. 
\z
\z

Quant au roumain, on peut considérer comme cas de gapping tous les exemples avec tête médiane (SVO+SO ou OVS+OS) et aussi les exemples avec tête finale (SOV+SO ou OSV+OS). Si le verbe est en position initiale (VSO+SO ou bien VOS+OS), il reste à vérifier si c’est du gapping ou une coordination de séquences \is{Argument Cluster Coordination (ACC)}(ACC). C’est pour cela que la suite de ce chapitre se concentrera tout d’abord sur les cas non ambigus de gapping, en montrant leurs propriétés, leurs contraintes, les analyses proposées dans la littérature, ainsi que l’analyse la plus adéquate pour rendre compte de ces contraintes, et ensuite finira par une présentation détaillée des cas ambigus où la tête est en première position, afin de décider sur le type de construction envisagé : gapping ou ACC. 

Pour conclure, on a observé dans cette section que la perspective typologique rend encore plus difficile l’étude de l’ellipse. Les faits discutés ci-dessus nous montrent qu’il faudrait se méfier de l’adéquation des étiquettes proposées pour certaines constructions elliptiques dans certaines langues et qu’une étude empirique des données devrait être faite avant de postuler l’existence d’un certain type elliptique dans une langue. Ce problème explique le flou terminologique qui existe dans la littérature sur les types d’ellipse et en partie la pléthore d’analyses proposées pour expliquer un même phénomène. 

Deux points ont été abordés dans une perspective typologique : d’une part, la direction de l’ellipse, et d’autre part, le contraste potentiel entre le gapping et \is{Argument Cluster Coordination (ACC)}ACC. On a observé que le gapping ne se comporte pas comme un phénomène uniforme quant à sa distribution à travers les langues. De plus, dans certaines distributions et dans certaines langues, il doit être distingué sur une base empirique des occurrences d’ACC. Donc, les critères doivent être relativisés aux contraintes \is{ordre de mots}d’ordre des mots des différentes langues.

\newpage
\section{Propriétés du gapping} \label{ch2:sect2.3}

Comme le gapping n’a pas été décrit pour le roumain, j’insiste surtout sur les données de cette langue ; les données du français seront présentées quand il y a des différences entre le fonctionnement du gapping dans les deux langues. Pour une analyse détaillée du gapping en français, voir \citet{AbeilleEtAl2010} et \citet{AbeilleEtAl2014}. En ce qui concerne le roumain, comme je l’ai mentionné plus haut, je commence l’étude du gapping en regardant essentiellement les cas non ambigus, {\cad} les structures dans lesquelles la phrase source a le verbe tête en position médiane ou en position finale.


\subsection{Contextes phrastiques du gapping} \label{ch2:sect2.3.1}

\subsubsection{Gapping et types de phrase}

Une phrase trouée est composée d’au moins deux syntagmes interprétés comme valents ou ajouts (d’un verbe antécédent de la phrase source), ayant un contenu propositionnel qui est récupéré à partir de la phrase source. Ce contenu propositionnel apparaît a priori avec tous les \is{type de phrase}types de phrase, bien que les types interrogatif et exclamatif soient plus contraints, en particulier en français. Dans les deux langues, les types déclaratif et désidératif ne posent aucun problème en termes d’acceptabilité (voir les déclaratives \REF{ch2:ex22a} et \REF{ch2:ex23a}, ainsi que les désidératives \REF{ch2:ex22b} et \REF{ch2:ex23b}). 

\ea
\ea Ion \uline{a cumpărat} o carte pentru Dana, iar Petre un stilou pentru Maria. \label{ch2:ex22a}
\glt  ‘Ion a acheté un livre pour Dana, et Petre un stylo pour Maria.’   

\ex  Mâine \uline{fă-mi}, te rog, o pizza, iar poimâine o friptură de viţel ! \label{ch2:ex22b}
\glt  ‘Demain fais-moi, s’il te plaît, une pizza, et après-demain un rôti de veau.’ 
\z
\z


\ea
\ea  Jean \uline{a acheté} un livre à Marie et Paul un stylo à Anne. \label{ch2:ex23a}
\ex Demain \uline{va} à la piscine et après-demain au stade ! \label{ch2:ex23b}
\z
\z

Quant aux \is{type de phrase}types interrogatif et exclamatif, les contextes qui sont parfaitement acceptables sont ceux dans lesquels le syntagme interrogatif ou exclamatif est mis en facteur et prend donc \isi{portée large} sur la coordination dans son ensemble. D’ailleurs, l’ellipse du verbe est même requise dans ces contextes (la répétition du verbe dans le deuxième conjoint dégrade significativement l’acceptabilité de la phrase). Un petit point comparatif à noter est le fait qu’en roumain les interrogatives partielles avec un syntagme \textit{qu-} extrait demandent généralement un sujet postverbal\footnote{
 Le syntagme interrogatif \textit{de ce} ‘pourquoi’ permet le sujet préverbal sous certaines conditions :

\ea
\ea {} [De ce] (Ion) merge (Ion) cu maşina, iar Maria pe jos ?
\glt ‘Pourquoi Ion y va en voiture, et Maria à pied ?’

\ex De ce (Ion) vine (Ion) azi (Ion) ?
\glt ‘Pourquoi Ion vient-il aujourd’hui ?’

\ex De ce (*Ion) vine (Ion) ? 
\glt ‘Pourquoi Ion vient-il ?’
\z
\z
} (comparer (\ref{ch2:ex24a}--\ref{ch2:ex24b}), ainsi que (\ref{ch2:ex24c}--\ref{ch2:ex24d}) pour les interrogatives et (\ref{ch2:ex25a}--\ref{ch2:ex25b}) et (\ref{ch2:ex25c}--\ref{ch2:ex25d}) pour les exclamatives), alors qu’en français on n’observe pas cette contrainte: les deux placements du sujet, préverbal et postverbal, sont possibles, cf. \REF{ch2:ex26} et \REF{ch2:ex27}. Par conséquent, en roumain, les interrogatives ou exclamatives avec la mise en facteur du syntagme \textit{qu-} sont à priori ambiguës quant à l’identification du type de construction : une construction à gapping (donc, une coordination de phrases) ou bien une construction \is{Argument Cluster Coordination (ACC)}ACC (donc, une coordination de séquences), car le verbe dans ces contextes précède à la fois les éléments corrélats et les éléments résiduels (voir les sections~\ref{ch2:sect2.2.2} et~\ref{ch2:sect2.6}). 

\ea
\ea {}
\gll [Pe  cine]  \uline{sună}  Ion  dimineaţa  şi  Maria  (*sună)  {seara ?} \label{ch2:ex24a}\\
\textsc{dom}  qui  appelle  Ion  matin\textsc{.def.f}  et  Maria  appelle   soir.\textsc{def.f}\\ 
\glt ‘Qui est-ce que Ion appelle le matin et Maria le soir ?’    

\ex  
\gll *[Pe  cine]  Ion \uline{sună}  dimineaţa  şi  Maria  {seara ?} \label{ch2:ex24b}\\
\textsc{dom}  qui  Ion  appelle  matin\textsc{.def.f}  et  Maria  soir\textsc{.def.f}\\
\glt  ‘Qui est-ce que Ion appelle le matin et Maria le soir ?’

\ex {} [Ce ziar] \uline{citeşte} Ion dimineaţa şi Maria (*citeşte) {seara ?} \label{ch2:ex24c}
\glt  ‘Quel journal lit Ion le matin et Maria le soir ?’   

\ex  
\gll *[Ce  ziar]  Ion  \uline{citeşte}  dimineaţa  şi  Maria  {seara ?} \label{ch2:ex24d}\\
quel  journal  Ion  lit  matin.\textsc{def.f}  et  Maria  soir.\textsc{def.f}\\
\glt  ‘Quel journal lit Ion le matin et Maria le soir ?’         
\z
\z


\ea
\ea {} [Ce răbdare] \uline{are} Maria cu copiii ei şi Ion (*are) cu studenţii lui ! \label{ch2:ex25a}
\glt ‘Quelle patience a Maria avec ses enfants et Ion avec ses étudiants !’    

\ex *[Ce răbdare] Maria \uline{are} cu copiii ei şi Ion cu studenţii lui ! \label{ch2:ex25b}
\glt  ‘Quelle patience Maria a avec ses enfants et Ion avec ses étudiants !’    

\ex {} [Ce oameni săraci] \uline{a întâlnit} Ion în Dolj şi Maria (*a întâlnit) în {Vaslui !} \label{ch2:ex25c}
\glt  ‘Quels gens pauvres a rencontré Ion en Dolj et Maria en Vaslui !’

\ex  *[Ce oameni săraci] Ion \uline{a întâlnit} în Dolj şi Maria în Vaslui ! \label{ch2:ex25d}
\glt  ‘Quels gens pauvres Ion a rencontré en Dolj et Maria en Vaslui !’
\z
\z


\ea \label{ch2:ex26}
\ea {} [Quels livres] Paul \uline{a-t-il lus} hier et Marie (*a-t-elle lus) aujourd’hui ? 
\ex {} [Quels livres] \uline{a lus} Paul hier et (??a lus) Marie aujourd’hui ?
\z
\z


\ea \label{ch2:ex27}
\ea {} [Quelle patience] Paul \uline{a montrée} avec ses enfants et Marie (*a montrée) avec ses étudiants ! 
\ex {} [Quelle patience] \uline{a montrée} Paul avec ses enfants et (??a montrée) Marie avec ses étudiants ! 
\z
\z

En revanche, si le syntagme interrogatif ou exclamatif n’est pas mis en facteur (donc, on a un syntagme interrogatif ou exclamatif dans chaque conjoint), les jugements sont très difficiles à faire, en particulier en français\footnote{
Certaines exclamatives sont parfaitement acceptables, mais dans ces cas, la séquence trouée peut apparaître toute seule comme \isi{phrase averbale}. 

\ea
\ea  (C’est incroyable) Quelle chance \uline{on a eue} à Londres et quelle poisse à Berlin !
\ex  Quelle poisse à Berlin !
\z
\z

\ea  
\ea Quelle tristesse (\uline{on voyait}) parmi les soldats, mais quelle joie parmi les officiers !
\ex  Quelle joie parmi les officiers !
\z
\zlast
}. La seule observation qui est claire en français concerne l’acceptabilité des phrases avec des syntagmes interrogatifs sujet (\ref{ch2:ex30a}--\ref{ch2:ex30b}). Il reste à préciser les facteurs qui jouent sur les degrés d’acceptabilité qu’on observe\footnote{
 Les éléments qu’on pourrait examiner sont : (i) le marquage d’un des éléments résiduels, (ii) la présence d’un sujet dans la séquence trouée et (iii) s’il y a un sujet dans la phrase trouée, vérifier si le placement du sujet dans la phrase source joue un rôle (en particulier, si les phrases avec sujet préverbal sont préférées aux phrases avec sujet postverbal).}.


\ea
\ea \textbf{Cine} \uline{vine} azi şi \textbf{cine} mâine ? 
\glt  ‘Qui vient aujourd’hui et qui demain ?’   

\ex  (Mă întreb) \textbf{Ce} carte \uline{să-i ofer} lui Ion şi \textbf{ce} disc Mariei.
\glt  ‘(Je me demande) Quel livre offrir à Ion et quel disque à Maria.’    

\ex  \textbf{Câte} minute \uline{ai vorbit} tu şi \textbf{câte} eu ? 
\glt  ‘Combien de minutes as-tu parlé, toi, et combien moi ?’       
\z
\z

\ea
\ea \%\textbf{Ce} rochie frumoasă \uline{are} Ioana şi \textbf{ce} pantaloni demodaţi soţul ei !
\glt  ‘Quelle jolie robe a Ioana et quel pantalon démodé a son mari !’    

\ex  
\gll \%E uimitor  \textbf{cât} \textbf{de}  frig  \uline{e}  înăuntru  şi  \textbf{cât} \textbf{de} cald {afară !}\\
est  étonnant  combien  de  froid  est  dedans  et  combien  de  chaud  dehors\\
\glt  ‘C’est étonnant à quel point il fait froid à l’intérieur et à quel point il fait chaud dehors !’
\z
\z


\ea
\ea  \textbf{Qui} \uline{va} à Rome et \textbf{qui} à Florence ? \label{ch2:ex30a}
\ex  Je me demande \textbf{lesquelles} \uline{vont aller} à la piscine, \textbf{lesquelles} au musée. \label{ch2:ex30b}
\ex  \%Parmi les livres que ton frère a reçus, \textbf{lesquels} \uline{a-t-il offerts} à Paul et \textbf{lesquels} à Marie ?
\ex  \%Dans cette bibliothèque, il y a plein de livres de mon enfance. Je me souviens \textbf{lesquels} \uline{je lisais} à 6 ans et \textbf{lesquels} à 8 ans. 
\z
\z


\ea
\ea  ??\textbf{Quel} bonheur Paul \uline{a connu} à Paris et \textbf{quelle} tristesse Marie à Londres ! 
\ex ??C’est incroyable \textbf{quelle} chance \uline{a eue} Paul et \textbf{quelle} malchance son frère ! 
\z
\z


\subsubsection{Gapping et les phrases coordonnées}

On considère souvent que le gapping est compatible uniquement avec la coordination, son emploi étant exclu de la subordination (voir, p.ex., \citealt{Johnson2009}). Le gapping apparaît toujours dans des constructions «~parallèles~» du point de vue sémantique et discursif (la notion de parallélisme étant développée par la suite), ce qui explique l’occurrence massive de ce type d’ellipse dans des phrases liées par la coordination ou la \isi{juxtaposition}. Tout type de coordination y est présent : coordination de phrases racines \REF{ch2:ex32a} ou coordination de phrases subordonnées\footnote{
 Le gapping n’est pas un phénomène réservé aux phrases racines (\textit{contra} \citealt{Hankamer1971,Hankamer1979,Ince2009}).} \REF{ch2:ex32b}, coordination simple (\ref{ch2:ex32a}--\ref{ch2:ex32b}) ou \is{coordination omnisyndétique (ou corrélative)}coordination omnisyndétique ou «~corrélative~» \REF{ch2:ex32c}\footnote{
 Pour plus de détails sur les \is{coordination omnisyndétique (ou corrélative)}coordinations omnisyndétiques («~corrélatives~»), voir \citet{Bilbiie2008} pour le roumain et \citet{Mouret2007} pour le français.}. 

\ea
\ea Dan \uline{vine} azi, (\textbf{iar}) Maria mâine. \label{ch2:ex32a}
\glt  ‘Dan vient aujourd’hui, (et) Maria demain.’   

\ex  Mi s-a spus că Dan \uline{vine} azi, \textbf{iar} Maria mâine. \label{ch2:ex32b}
\glt  ‘On m’a dit que Dan viendrait aujourd’hui, et Maria demain.’ 

\ex  \textbf{Fie} Dan \uline{va cânta} la vioară, \textbf{fie} Maria la pian. \label{ch2:ex32c}
\glt  ‘Soit Dan va jouer du violon, soit Maria du piano.’       
\z
\z

Si l’on a une coordination de subordonnées, la phrase trouée ne peut pas être introduite par un complémenteur \REF{ch2:ex33}. Cela s’explique si l’on suppose que le complémenteur exige la présence d’une tête verbale finie (cf. \citealt{Godard1989}). Or, la contrainte générale qui pèse sur le matériel manquant est d’omettre au moins la tête verbale (y compris le complémenteur). 

\ea \label{ch2:ex33}
\ea Mi s-a spus \textbf{că} Dan \uline{vine} azi şi (*\textbf{că}) Maria mâine.
\glt  ‘On m’a dit que Dan viendrait aujourd’hui et Maria demain.’   

\ex  Vreau \textbf{ca} Dan \uline{să vină} azi şi (*\textbf{ca}) Maria mâine.
\glt  ‘Je veux que Dan vienne aujourd’hui et Maria demain.’ 
\z
\z

Beaucoup d’auteurs (\citealt{Jackendoff1971,Koutsoudas1971,Hankamer1979,Wilder1994,Johnson2009}, etc.) considèrent qu’il est impossible d’avoir une phrase trouée enchâssée sous une phrase source racine \REF{ch2:ex34}. \citet{Sag1976} notamment observe que le gapping opère uniquement au nœud phrastique le plus haut et jamais à un nœud enchâssé. On observe des contraintes similaires en roumain \REF{ch2:ex35} et en français \REF{ch2:ex36} : généralement, la phrase trouée ne peut pas être enchâssée dans une phrase source racine.

\ea \label{ch2:ex34}
\ea  *Sam \uline{played} tuba \textbf{whenever} Max sax. \citep{Jackendoff1971}       
\ex *McTavish \uline{plays} bagpipe \textbf{despite the fact that} McCawley the contrafagotto d’amore.                  
\z
\z

\ea \label{ch2:ex35}
\ea *Maria \uline{cântă} la vioară, \{\textbf{pentru că {\textbar} deşi}\} Ion la pian.
\glt  ‘Maria joue du violon, \{parce que {\textbar} quoique\} Ion joue du piano.’   

\ex  *Maria \uline{mănâncă} o pară, \{\textbf{înainte ca {\textbar} după ce}\} Ion un măr.
\glt  ‘Maria mange une poire, \{avant que {\textbar} après que\} Ion mange une pomme.’ 
\z
\z

\ea \label{ch2:ex36}
\ea  *Marie \uline{va souvent} à la piscine, \textbf{parce que} son mari au stade.
\ex *Marie \uline{va} toujours \uline{aux concerts}, \textbf{bien que} son mari jamais.
\z
\z

Cependant, \citet{Izutsu2008} note quelques exemples de gapping avec des complémenteurs comme \textit{whereas} ou\textit{ while}, qui semblent être acceptables en anglais \REF{ch2:ex37}\footnote{
 Il reste à expliquer pourquoi cela ne marche pas dans les cas les plus simples (verbe simple sans auxiliaire, sujet ou objet direct sans marquage), comme celui illustré par \citet{CulicoverEtAl2005} en \REF{ch2:foot16exi} :
 
 \ea \label{ch2:foot16exi}
 John \uline{drank} whisky, \{*whereas {\textbar} *while\} Mary beer. 
 \zlast
}. Dans les deux langues étudiées dans ce livre, on observe que les subordonnants oppositifs \textit{în timp ce} ‘alors que’ en roumain \REF{ch2:ex38} et\textit{ alors que} ou \textit{tandis que} en français \REF{ch2:ex39}\footnote{
 Selon Anne Dagnac (c.p.) à qui on doit ces exemples, ces subordonnants permettent le gapping à condition qu’un des éléments résiduels soit un élément négatif.} permettent le gapping dans la séquence qu’ils introduisent. L’acceptabilité de ces introducteurs dans ces contextes est certainement due à leur emploi discursif compatible avec le gapping. Contrairement aux subordonnants conditionnels, concessifs, etc., ces introducteurs oppositifs entretiennent, du point de vue discursif, une \is{relation discursive}relation symétrique (de contraste et/ou parallélisme) ; or, comme on le verra dans la section~\ref{ch2:sect2.3.4.3}, ce type de relation est le prototype des \is{relation discursive}relations discursives dans les constructions à gapping. 

\ea \label{ch2:ex37}
\ea  Men \uline{are valued} for their economic status, \textbf{whereas} women for their appearance. \citep[654]{Izutsu2008} 
\ex Boys \uline{are encouraged} to go out for work, \textbf{while} girls to stay at home. 
\z
\z

\ea \label{ch2:ex38}
\ea Unele pisici \uline{vomită} tot timpul, \textbf{în timp ce} altele foarte rar.
\glt  ‘Certains chats vomissent tout le temps, alors que d’autres très rarement.’   

\ex  Unii dintre noi \uline{pierd} milioane pe lună, \textbf{în timp ce} alții niciun leu.
\glt  ‘Certains d’entre nous perdent des millions par mois, alors que d’autres même pas un sou.’ 
\z
\z

\ea \label{ch2:ex39}
\ea  Paul \uline{va} souvent \uline{à la piscine}, \textbf{alors que} moi jamais.
\ex Paul \uline{a vu} plusieurs hérons, \textbf{tandis que} moi aucun.    
\z
\z

Cependant, la question se pose de savoir si ces éléments comme le roumain \textit{în timp ce} ‘alors que’, qui figurent normalement sur la liste des complémenteurs (ou conjonctions de subordination, selon la terminologie traditionnelle), se comportent effectivement comme des subordonnants dans les constructions à gapping (et dans les coordinations en général). Il semble y avoir au moins deux arguments empiriques pour distinguer entre un \textit{în timp ce} subordonnant et un \textit{în timp ce} coordonnant. D’une part, une subordonnée temporelle introduite par \textit{în timp ce} peut précéder et suivre la phrase racine \REF{ch2:ex40}, alors qu’une coordonnée introduite par \textit{în timp ce} ne peut pas être antéposée \REF{ch2:ex41}. D’autre part, le subordonnant temporel \textit{în timp ce} impose des contraintes sur le temps du verbe (il exige généralement un verbe à l’indicatif imparfait ; comparer \REF{ch2:ex40a}, où le verbe est à l’imparfait, et \REF{ch2:ex42a}, où le verbe est au passé composé), alors que le coordonnant \textit{în timp ce} n’impose pas de contrainte particulière \REF{ch2:ex42b}). Enfin, on constate que le subordonnant \textit{în timp ce} a une interprétation temporelle ({\cad} il marque la simultanéité des événements dans les phrases liées), tandis que le coordonnant \textit{în timp ce} a plutôt un sens abstrait (il marque une relation de parallélisme et \is{contraste sémantique}contraste entre les phrases, sans qu’il s’agisse de simultanéité temporelle).

\ea \label{ch2:ex40}
\ea Cineva ne-a spart casa, \textbf{în timp ce} noi eram în vacanţă. \label{ch2:ex40a}
\glt  ‘Quelqu’un est entré par effraction dans la maison, pendant qu’on était en vacances.’ 

\ex  \textbf{In timp ce} noi eram în vacanţă, cineva ne-a spart casa.
\glt  ‘Pendant qu’on était en vacances, quelqu’un est entré par effraction dans la maison.’
\z
\z


\ea \label{ch2:ex41}
\ea Eu \uline{am făcut} zeci de proiecte, \textbf{în timp ce} ea nici măcar unul.
\glt  ‘J’ai fait des dizaines de projets, alors qu’elle même pas un seul.’   

\ex  
\gll *\textbf{In} \textbf{timp} \textbf{ce}  ea  nici  măcar  unul,  eu  am  făcut  zeci  de  proiecte.\\
en  temps  que  elle  ni  même  un.\textsc{def}  je  ai  fait  dizaines  de  projets\\
\glt  ‘J’ai fait des dizaines de projets, alors qu’elle même pas un seul.’
\z
\z


\ea
\ea ??Cineva ne-a spart casa, \textbf{în timp ce} noi am fost în vacanţă. \label{ch2:ex42a}
\glt  ‘Quelqu’un est entré par effraction dans la maison, pendant qu’on a été en vacances.’

\ex  Am constatat că unele pisici au reacţionat destul de violent la administrarea medicamentului, \textbf{în timp ce} altele n-au reacţionat în niciun fel. \label{ch2:ex42b}
\glt  ‘On a constaté que certains chats ont réagi assez violemment à l’administration du médicament, alors que d’autres n’ont pas réagi du tout.’
\z
\z

Un autre cas intermédiaire est celui des structures \isi{comparatives}. Comme l’an\-glais \REF{ch2:ex43}, les deux langues étudiées (roumain \REF{ch2:ex44} et français \REF{ch2:ex45}) permettent le gapping dans les \isi{comparatives}. Comme discuté à la fin de cette section, les contraintes du gapping dans les \isi{comparatives} sont beaucoup moins strictes que celles observées dans les coordinations. Il reste à établir s’il s’agit du même phénomène ; ce qui est clair est que, contrairement aux subordonnées ordinaires, les structures \isi{comparatives} se rapprochent beaucoup plus des structures coordonnées (\citealt{Moltmann1992,Osborne2009}, etc.), cf. discussion dans la section~\ref{ch1:sect1.4.2}.


\ea \label{ch2:ex43}
\ea  Robin \uline{speaks} French \textbf{better than} Leslie German. \citep{CulicoverEtAl2005} 
\ex Bill \uline{ate} \textbf{more} peaches than Harry grapes. \citep{Jackendoff1971} 
\z
\z


\ea \label{ch2:ex44}
\ea O mustrare \uline{pătrunde} mai mult pe omul priceput \textbf{decât} o sută de lovituri pe cel nebun.
\glt ‘Une réprimande fait plus d’impression sur l’homme intelligent \textbf{que} cent coups sur l’insensé.’  

\ex  
\gll Ana  \ulg{îl}{22}  iubește  pe  Ion  mai  mult  [decât  eu  pe  tine].\\ 
Ana \textsc{acc.3sg.m} aime \textsc{dom}  Ion  plus  beaucoup  que  \textsc{nom.1sg} \textsc{dom}  \textsc{acc.2sg}\\  
\glt ‘Ana aime Ion plus que moi je t’aime.’
\z
\z


\ea \label{ch2:ex45}
\ea  Jean \uline{est doué} en bricolage \textbf{comme} Marie en décoration. \citep{AmsiliEtAl2008}  
\ex Jean \uline{gagne} plus de coupes \textbf{que} Pierre de médailles.     
\z
\z

Concernant la relation entre le gapping et la subordination, l’(in)acceptabilité de ce type d’ellipse sous l’enchâssement a été récemment discutée dans \citet{Farudi2013} et \citet{Johnson2014}. A partir des exemples comme \REF{ch2:ex46} en anglais, \citet{Johnson2014} considère que le non-enchâssement est une contrainte forte (et en même temps un diagnostic) pour les constructions à gapping \is{No Embedding Constraint}(angl. \textit{No Embedding Constraint}). Cependant, comme le note \citet{Farudi2013}, cette contrainte ne s’applique pas dans toutes les langues (par exemple. le \ili{persan}). Revenant aux deux langues qui nous intéressent, on observe que le français se comporte comme l’anglais, ne permettant pas de gapping enchâssé \REF{ch2:ex47}\footnote{
 Il faudrait toutefois voir de plus près les données, car, selon \citet{Green1976}, l’anglais permettrait dans les subordonnées certains phénomènes spécifiques à la phrase racine dans les mêmes contextes que ceux mentionnés ici pour le roumain.}. En revanche, en roumain l’enchâssement est possible dans certains contextes avec «~amalgamation syntaxique~» \citep{Lakoff1974}, où un verbe épistémique comme \textit{a crede} ‘croire’ \REF{ch2:ex48a}, \textit{a vedea} ‘voir’ \REF{ch2:ex48b}, \textit{a ști} ‘savoir’ \REF{ch2:ex48c} (à une personne déictique, en particulier à la première personne) ou impersonnel comme \textit{a părea} ‘paraître’ \REF{ch2:ex48d} exprime une \is{prédicat d'attitude propositionnelle}attitude propositionnelle par rapport au contenu de la phrase trouée. Leur emploi dans ces contextes spécifiques semble être très différent et assez marginal par rapport à l’emploi ordinaire de ces verbes, ce qui impose une analyse syntaxique différente (p.ex. verbes «~faibles~», cf. \citealt{Blanche-BenvenisteEtAl2007}, angl. \textit{grafts}, cf. \citealt{vanRiemsdijk2006}, ou encore angl. \textit{hedges}, cf. \citealt{Lakoff1973}). En dehors de ces exemples, la phrase trouée est toujours au même niveau syntaxique que la phrase contenant l’antécédent \citep{Lobeck1995}.

\ea \label{ch2:ex46}
*Alfonse \uline{stole} the emeralds, and \textbf{I think that} Mugsy the pearls. \citep{Hankamer1979}
\z

\ea \label{ch2:ex47}
*Jean \uline{aime} Marie et \textbf{je crois qu}’elle aussi Jean.
\z

\ea
\ea 
\gll Nu  eu  \ulg{îl}{22.8}  urăsc  pe  el,  ci  \textbf{cred} \textbf{că}  EL  pe  mine. \label{ch2:ex48a}\\
\textsc{neg} je  \textsc{acc.3sg.m} hais  \textsc{dom} \textsc{3sg.m}  mais  crois.\textsc{1sg}  que  il \textsc{dom} \textsc{acc.1sg}\\ 
\glt ‘Ce n’est pas moi qui le hais, mais je crois que c’est lui qui me hait.’    

\ex  
\gll Ion  \ulg{o}{24.9}  iubește  pe  Ana  şi  \textbf{văd} \textbf{că}  și  ea  pe  el. \label{ch2:ex48b}\\
Ion  \textsc{acc.3sg.f}  aime  \textsc{dom} Ana  et  vois.\textsc{1sg}  que  aussi  elle  \textsc{dom} lui\\
\glt  ‘Ion aime Ana et je vois qu'elle aussi, elle l’aime.’   

\ex 
\gll Ion  \ulg{este}{20}  îndrăgostit  de  Ana,  \textbf{nu} \textbf{știu} \textbf{însă} \textbf{dacă}  și  ea  de  el. \label{ch2:ex48c} \\
Ion  est  amoureux  de  Ana  \textsc{neg}  sais.\textsc{1sg}  cependant  si  aussi  elle  de  lui\\ 
\glt  ‘Ion est amoureux de Ana, mais je ne sais pas si elle est amoureuse de lui.’ 

\ex  
\gll Ion  \ulg{este}{20}  îndrăgostit  de  Ana  şi  \textbf{pare-se} \textbf{că}  și  ea  de  el. \label{ch2:ex48d}\\
Ion  est  amoureux  de  Ana  et  parraît-\textsc{refl}  que  aussi  elle  de  lui\\
\glt  ‘Ion est amoureux de Ana et, paraît-il, elle de lui aussi.’ 
\z
\z

Revenons aux emplois typiques de gapping dans les phrases coordonnées. On observe que le gapping peut apparaître dans une coordination multiple, avec plusieurs phrases coordonnées. Dans ce cas, plusieurs options se présentent : soit il y a une seule phrase trouée (et en général c’est le dernier conjoint) \REF{ch2:ex49a}, soit il y en a plusieurs \REF{ch2:ex49b}, coordonnées à une phrase source\footnote{
 A priori, l’anglais serait différent, selon \citet{McCawley1988}, qui note que, dans une coordination multiple, tous les conjoints sont troués, sauf le premier.}. Les exemples avec une phrase trouée encadrée par deux phrases complètes sont marginaux \REF{ch2:ex49c}, mais ils s’améliorent si l’on a plus de trois conjoints (\ref{ch2:ex50a}--\ref{ch2:ex50a}). 

\ea
\ea La petrecere, Dan \uline{a băut} bere, Maria \uline{a băut} vin, iar Ioana suc. \label{ch2:ex49a}
\glt  ‘A la fête, Dan a bu de la bière, Maria a bu du vin, et Ioana du jus.’   

\ex  La petrecere, Dan \uline{a băut} bere, Maria vin, iar Ioana suc. \label{ch2:ex49b}
\glt  ‘A la fête, Dan a bu de la bière, Maria du vin, et Ioana du jus.’  

\ex  ?La petrecere, Dan \uline{a băut} bere, Maria vin, iar Ioana \uline{a băut} suc. \label{ch2:ex49c}
\glt  ‘A la fête, Dan a bu de la bière, Maria du vin, et Ioana a bu du jus.’
\z
\z

\ea
\ea Mama \uline{vrea} o casă, tata o maşină, Ion \uline{vrea} un câine, iar Maria o pisică. \label{ch2:ex50a}
\glt ‘Ma mère veut une maison, mon père une voiture, Ion veut un chien, et Maria un chat.’

\ex O specialitate \uline{o făceam} cu domnul profesor, alta cu doamna domnului, o alta cu fiul lor, iar alte specialităţi \uline{le-am făcut} cu fraţii, finii şi nora marilor profesori. \label{ch2:ex50b}\\
%(phrase librement adaptée de : \url{http://www.ziare.com/scoala/educatie/seism-in-mijlocul-anului-universitar-1077918})
\glt ‘Une spécialité on la faisait avec M. le professeur, une autre avec la dame du monsieur, une autre avec leur fils, et d’autres spécialités, on les a faites avec les frères, les filleuls et la belle-fille de ces grands professeurs.’   
\z
\z

Des exemples plus complexes de coordination multiple apparaissent dans les coordinations récursives (avec des éléments corrélatifs), qui permettent la coordination de phrases trouées entre elles \REF{ch2:ex51}.

\ea \label{ch2:ex51}
\ea Dan \uline{va cânta} la vioară, iar apoi [\textbf{fie} Maria la pian, \textbf{fie} Ion la trompetă].\\
\glt ‘Dan va jouer du violon, et ensuite soit Maria du piano, soit Ion de la trompette.’ 

\ex \textbf{Fie} [Ion \uline{va merge} cu trenul şi Dan cu maşina], \textbf{fie} [Dan cu trenul şi Ion cu maşina].
\glt ‘Soit Ion va prendre le train et Dan la voiture, soit Dan le train et Ion la voiture.’ \z
\z


\subsubsection{Inventaire des coordonnants}

Le choix des conjonctions qui peuvent être utilisées dans les constructions à gapping est conditionné par les contraintes sémantiques et discursives qu’on étudiera dans la section~\ref{ch2:sect2.3.4}. Le gapping permet l’emploi de toute conjonction qui est compatible avec une \isi{relation discursive} symétrique, ce qui exclut donc les conjonctions \textit{or} et \textit{car} en français. Toutes les autres conjonctions sont possibles : fr. \textit{et}, \textit{ou}, \textit{mais}, \textit{ni}, \textit{soit...} \textit{soit...}; roum. \textit{şi} ‘et’, \textit{sau} ‘ou’, \textit{iar} ‘et’, \textit{dar} ‘mais’, \textit{ci} ‘mais’\footnote{
 Le roumain, comme \il{espagnol}l’espagnol et \il{allemand}l’allemand, distingue entre une conjonction adversative (\textit{dar} ‘mais’) et une conjonction corrective (\textit{ci} ‘mais’).}, \textit{fie...} \textit{fie...} ‘soit... soit...’, ainsi que des éléments comme fr. \textit{ainsi que}, \textit{comme}, \textit{que} comparatif\footnote{
 Voir \citet{Mouret2007} et \citet{MouretEtAl2008} pour des discussions sur le statut conjonctif de ces éléments en français. A noter la différence entre le complémenteur \textit{que}, qui n’autorise pas le gapping, et le comparatif \textit{que}, qui apparaît dans une phrase trouée.} ; roum. \textit{ca (şi)} et \textit{precum (şi)} ‘ainsi que’, ou encore \textit{la fel ca} ‘comme’. 

Les premiers travaux considéraient que le gapping est beaucoup moins acceptable avec \textit{but} en anglais ou \textit{mais} en français. A l’instar de \citet{Repp2009}, on observe que \textit{but} ne pose aucun problème s’il apparaît dans un contexte approprié, certes plus contraint que les autres conjonctions typiques. Il est légitimé surtout par la présence de certains opérateurs sémantiques dans le deuxième conjoint (p.ex. \textit{only, even}) ou bien par une différence de \isi{polarité} entre la phrase source et la phrase trouée. Ainsi, en roumain, l’adversatif \textit{dar} ‘mais’ est permis dans la phrase trouée, s’il est accompagné d’un \isi{adverbe associatif} comme l’additif \textit{şi} ‘aussi’ en \REF{ch2:ex52a} ou le restrictif \textit{numai} ‘seulement’ en \REF{ch2:ex52b}, ou bien si la phrase trouée contient un mot négatif qui renverse la \isi{polarité} par rapport à la phrase source \REF{ch2:ex53}. Tous ces éléments (\textit{şi} ‘aussi’, \textit{numai} ‘seulement’, \textit{nici măcar} ‘même pas’, \textit{nimic} ‘rien’) renforcent le \is{contraste sémantique}contraste et contribuent au mouvement argumentatif de la conjonction adversative \textit{dar} ‘mais’. 

\ea
\ea Ne e greu, pretenţiile \uline{sunt} mari, \textbf{dar} \textbf{şi} răsplata pe măsură. \label{ch2:ex52a}
\glt ‘C’est difficile pour nous, les exigences sont grandes, mais la récompense à la hauteur aussi.’  

\ex Ion \uline{schiază} pe orice fel de pistă, \textbf{dar} Maria \textbf{numai} pe cele mai uşoare. \label{ch2:ex52b}
\glt ‘Ion skie toutes les pistes, mais Maria seulement les plus faciles.’
\z
\z

\ea \label{ch2:ex53}
\ea Ioanei \uline{îi plac} toate dulciurile, \textbf{dar} Mariei \textbf{nici măcar} tortul făcut în casă.
\glt ‘Ioana aime toutes les sucreries, mais Maria même pas le gâteau fait maison.’  

\ex Băiatul \uline{a mâncat} ceva, \textbf{dar} fata \textbf{nimic}.
\glt ‘Le garçon a mangé quelque chose, mais la fille rien du tout.’ 
\z
\z

Le correctif \textit{ci} ‘mais’ est autorisé lui aussi dans les constructions à gapping, à condition que la phrase source contienne la \isi{négation} (de constituant) \textit{nu} qui a \is{portée étroite}portée uniquement dans la phrase source \REF{ch2:ex54a} (voir \citealt{Toosarvandani2011} pour les données de l’anglais). Toujours pour la correction, on peut utiliser la conjonction \textit{şi} ‘et’ avec la \isi{négation} de constituant \textit{nu} dans la phrase trouée cette fois-ci, la phrase source ayant une \isi{polarité} positive \REF{ch2:ex54b}.

\ea
\ea 
\gll \textbf{NU}  Ana\textsubscript{i}  \ulg{îl}{25.3}  iubeşte  pe  Dan\textsubscript{j},  \textbf{ci}  EL\textsubscript{j}  pe  ea\textsubscript{i}. \label{ch2:ex54a}\\
\textsc{neg}  Ana  \textsc{acc.3sg.m}  aime  \textsc{dom}  Dan  mais  il  \textsc{dom}  elle\\
\glt ‘Ce n’est pas Ana qui aime Dan, mais c’est Dan qui l’aime.’  

\ex 
\gll DAN\textsubscript{j}  \ulg{o}{24.5}  iubeşte  pe  Ana\textsubscript{i},  \textbf{şi}  \textbf{NU}  ea\textsubscript{i}  pe  el\textsubscript{j}. \label{ch2:ex54b}\\
Dan  \textsc{acc.3sg.f}  aime  \textsc{dom}  Ana  et  \textsc{neg}  elle  \textsc{dom}  lui\\
\glt ‘C’est Dan qui aime Ana, et pas l’inverse.’
\z
\z

Néanmoins, les conjonctions qui apparaissent le plus souvent avec le gapping en roumain sont \textit{şi} et \textit{iar} ‘et’, avec une fréquence extrêmement élevée de la conjonction \textit{iar}. Si la conjonction \textit{şi} est assez sous-spécifiée, pouvant être utilisée dans tous les contextes et à tous les niveaux en dehors du gapping, la conjonction \textit{iar} impose plusieurs contraintes \citep{BilbiieEtAl2011}. La préférence pour cette conjonction dans les constructions à gapping s’explique essentiellement par la contrainte sémantico-discursive qu’elle impose aux conjoints : \textit{iar} lie (uniquement) des phrases qui présentent au moins deux \is{paire contrastive}paires contrastives ({\cad} chaque paire réunit des éléments qui ont un \textit{intégrateur commun}, cf. \citealt{Lang1984}, et qui marquent une opposition sémantique), dont une est constituée par des \is{topique}topiques (ce qui justifie l’intitulé de \is{contraste sémantique}\textit{contraste thématique} attribué habituellement à cette conjonction, cf. \citealt{Zafiu2005}). Or, on verra dans la section~\ref{ch2:sect2.3.4.2} que la contrainte la plus importante dans le gapping est exactement celle-ci, {\cad} le double \is{contraste sémantique}contraste. On observe ainsi qu’il y a une superposition entre le gapping et la conjonction \textit{iar~}: les deux imposent la même contrainte sémantico-discursive. L’exemple typique avec \textit{iar} est \REF{ch2:ex55a}, où l’on a (au moins) deux \is{paire contrastive}paires contrastives obligatoires : une paire d’individus (\textit{Ioana, Maria}) et une paire de fruits (\textit{un măr, o pară}). En dehors du gapping, on observe que la conjonction \textit{iar} est autorisée en \REF{ch2:ex55b}, où l’on coordonne des phrases, mais pas en \REF{ch2:ex55c}, où l’on coordonne des syntagmes nominaux, bien qu’il y ait un contraste. Cette conjonction spécialisée pour le double \is{contraste sémantique}contraste en roumain n’existe pas dans les autres \ili{langues romanes}, mais elle a un correspondant dans les \ili{langues slaves} (voir le fonctionnement de la conjonction \textit{a} en \ili{russe}, cf. \citealt{JasinskajaEtAl2009,Kazenin2001,Agafonova2014}, etc.).  

\ea
\ea Ioana a mâncat un măr, \textbf{iar} *(Maria) (a mâncat) o pară. \label{ch2:ex55a}
\glt ‘Ioana a mangé une pomme, et Maria (a mangé) une poire.’   

\ex La Bucureşti plouă, \textbf{iar} *(la Braşov) ninge. \label{ch2:ex55b}
\glt ‘A Bucarest il pleut, et à Braşov il neige.’  

\ex Ioana mănâncă mere verzi \{\textbf{şi} {\textbar} *\textbf{iar}\} pere galbene. \label{ch2:ex55c}
\glt ‘Ioana mange des pommes vertes et des poires jaunes.’  
\z
\z

Comme mentionné ci-dessus, la phrase trouée peut ne pas être introduite par une conjonction. Dans certains cas de \isi{juxtaposition}, on remarque la présence des connecteurs adverbiaux (p.ex. \textit{însă} ‘cependant’ en \REF{ch2:ex56a}, \textit{dimpotrivă} ‘au contraire’ en \REF{ch2:ex56b}) qui rendent explicite la \isi{relation discursive}~de contraste requise par les constructions à gapping.


\ea \label{ch2:ex56}
\ea Eu \uline{apreciez} mai mult valorile spirituale, ea \textbf{însă}, mai mult~pe cele materiale. \label{ch2:ex56a} 
\glt ‘Moi, j’apprécie surtout les valeurs spirituelles, elle en revanche, surtout les valeurs matérielles.’   

\ex Studenţii \uline{erau} încântaţi, profesorii, \textbf{dimpotrivă}, extrem de abătuţi. \label{ch2:ex56b}
\glt ‘Les étudiants étaient enchantés, les professeurs, au contraire, extrêmement abattus.’
\z
\z


\subsubsection{Antécédent linguistique} 

Les travaux traditionnels considèrent que le matériel manquant dans le gapping ne peut pas avoir un antécédent pragmatique, extra-linguistique \citep{HankamerEtAl1976,Chao1988}, le gapping ne permettant donc pas \is{emploi exophorique}l'emploi exophorique. C’est pour cette raison que \citet{HankamerEtAl1976} considèrent le gapping (tout comme \is{Verb Phrase Ellipsis (VPE)}VPE, le \is{Sluicing}sluicing, le \is{Stripping}stripping ou les \is{anaphore}anaphores en \textit{so} en anglais) comme un cas \is{anaphore de surface}d’anaphore «~de surface~» (angl. \textit{surface anaphora}), qu’ils distinguent des \is{anaphore profonde}anaphores «~profondes~» (angl. \textit{deep anaphora}, p.ex. les expressions pronominales ordinaires, les \is{anaphore}anaphores de complément nul ou les \is{anaphore}anaphores en \textit{do it} en anglais) qui peuvent avoir un antécédent situationnel. Leur exemple est donné en \REF{ch2:ex57}. Cependant, on pourrait envisager un contexte comme celui décrit en \REF{ch2:ex58} pour le roumain, qui pourrait permettre un antécédent extra-linguistique dans les phrases avec deux éléments résiduels. En l’absence d’exemples attestés, je me limite dans ce travail uniquement aux occurrences de gapping avec antécédents linguistiques et je laisse la question ouverte en ce qui concerne l’antécédent non linguistique. Il faudrait regarder aussi les titres de journaux avec ellipse multiple, où il n’y a pas d’antécédent explicite \REF{ch2:ex59}\footnote{
 On doit quand même faire attention à ce type d’exemples, pour bien distinguer les phrases elliptiques des phrases non elliptiques \is{phrase averbale}averbales (en l’occurrence, \REF{ch2:ex59a} semble être un énoncé elliptique, alors que le deuxième exemple \REF{ch2:ex59b} ressemble plutôt à une \isi{phrase averbale} existentielle).}.

\ea \label{ch2:ex57}
[\textit{Hankamer produces an orange, proceeds to peel it, and just as Sag produces an apple, says :}] \#And Ivan, an apple. \citep{HankamerEtAl1976} 
\z

\ea \label{ch2:ex58}
[\textit{C’est la fête de Noël. La mère de Marie et Jean entre dans la pièce avec deux grands paquets joliment décorés, et s’adresse d’abord à Marie et ensuite à Jean :}] Tu pachetul roşu. [\textit{et quelques secondes plus tard :}] Iar tu pe cel albastru.
\glt ‘Toi, le paquet rouge. Et toi, celui qui est bleu.’     
\z


\ea \label{ch2:ex59}
\ea Ieri pe tron, azi în închisoare. \label{ch2:ex59a}
\glt  ‘Hier sur le trône, aujourd’hui en prison.’   

\ex Ploi în vestul ţării, caniculă în sud. \label{ch2:ex59b}\\%(\url{www.adevarul.ro/actualitate/eveniment/Ploi\_in\_vestul\_tarii-canicula\_in\_sud\_0\_317368576.html})\\
\glt ‘Pluies dans l’ouest du pays, canicule dans le sud.’
\z
\z


\subsubsection{Gapping et les autres constructions elliptiques}

Le gapping est co-occurrent avec d’autres types d’ellipse, ce qui a suscité un grand débat (surtout dans les années 1970-1980) concernant l’extension ou non de la règle de gapping à d’autres séquences elliptiques. A priori, on n’a pas d’arguments pour distinguer le gapping à l’intérieur d’un énoncé du gapping qui apparaît dans le \isi{dialogue}, dans les \is{réponse courte}réponses courtes (angl. \textit{short answers}) ou dans les \is{question courte}questions courtes (angl. \textit{short questions}). 

Contrairement à ce que soutiennent, entre autres, \citet{HankamerEtAl1976}, \citet{Williams1977}\footnote{
 A partir de cette observation (à laquelle s’ajoute la contrainte sur les syntagmes nominaux complexes), \citet{Williams1977} distingue les ellipses bornées (p.ex. le gapping) des ellipses non bornées (p.ex. \is{Verb Phrase Ellipsis (VPE)}VPE) et il propose une règle spécifique à chaque type : les ellipses bornées relèvent de la grammaire de la phrase, alors que les autres doivent être décrites dans une grammaire du discours (\textit{Sentence Grammar} vs. \textit{Discourse Grammar}).} ou \citet{Lobeck1995}, on constate que le gapping peut apparaître aussi dans le \isi{dialogue} \REF{ch2:ex60}, donc il peut intervenir au-delà des limites d’un énoncé, si les locuteurs des énoncés en question communiquent de façon coopérative (voir aussi \citealt{Merchant2001}, qui reprend \citealt[160]{SagEtAl1985}).

\ea \label{ch2:ex60}
\ea A : - Eu \uline{vreau} să merg la mare. B : - Iar eu la munte.
\glt A : ‘- Je veux aller à la mer.’ B : ‘- Et moi à la montagne.’

\ex A : - Trebuie să lucrez, deşi \uline{nu prea am spor} după o masă copioasă.\newline B : - Nici eu după o întâlnire cu Ion.
\glt A : ‘- Je dois travailler, bien que je ne sois pas très productif après un repas copieux.’\newline B : ‘- Moi non plus après une rencontre avec Ion.’ 
\z
\z

Un type d’ellipse proche du gapping est le \is{Stripping}stripping (appelé aussi en anglais \is{Bare Argument Ellipsis (BAE)}\textit{Bare Argument Ellipsis} (BAE)). Bien que le terme \is{Stripping}\textit{stripping} regroupe des constructions assez hétérogènes (voir \citealt{Abeille2006} et note \ref{ch2:fn3} ci-dessus), les constructions qui m’intéressent ici sont surtout les ellipses polaires, qui présentent habituellement un élément résiduel accompagné d’un adverbe polaire (comme les adverbiaux \textit{şi}\footnote{
 La conjonction \textit{şi} ‘et’ est distincte de l’adverbe additif homonyme \textit{şi} ‘aussi’. Voir plus de détails dans \citet{Bilbiie2008}.} ‘aussi’ \REF{ch2:ex61a} et \textit{nici} ‘non plus’ \REF{ch2:ex61b} en roumain, ou \textit{aussi} \REF{ch2:ex62a} et \textit{non plus} \REF{ch2:ex62b} en français), auxquelles j’ajoute les exemples avec \textit{şi/dar nu} ‘et/mais pas’ \REF{ch2:ex61c} en roumain, et \textit{mais pas} \REF{ch2:ex62c} en français. 

\ea
\ea 
\gll Binele  va  ieşi  biruitor,  [şi,  odată  cu  el,  \textbf{şi}  Dreptatea]. \label{ch2:ex61a}\\
bien.\textsc{def}  va  sortir  vainqueur  et  une\_fois  avec  lui  aussi  justice.\textsc{def}\\ 
\glt ‘Le Bien triomphera et, avec lui, la Justice aussi.’   

\ex  
\gll Ion  nu  vrea  să  aibă  copii,  [şi  \textbf{nici}  nevastă-sa]. \label{ch2:ex61b}\\
Ion  \textsc{neg}  veut  \textsc{sbjv}  avoir.\textsc{sbjv.3}  enfants  et  non\_plus  femme-\textsc{poss.3sg}\\
\glt ‘Ion ne veut pas avoir d’enfants, et sa femme non plus.’      

\ex 
\gll Libertate,  egalitate,  fraternitate –  [dar  \textbf{nu} \textbf{și}  pentru  romi]. \label{ch2:ex61c}\\
liberté  égalité  fraternité –  mais  pas  aussi  pour  Roms\\
\glt ‘Liberté, égalité, fraternité – mais pas pour les Roms.’
\z
\z


\ea
\ea  Jean viendra à la fête [et Marie \textbf{aussi}]. \label{ch2:ex62a}     
\ex  Jean n’est pas venu à la fête [et Marie \textbf{non plus}]. \label{ch2:ex62b}
\ex  Jean est venu hier [mais \textbf{pas} Marie]. \label{ch2:ex62c}
\z
\z

Ce genre d’exemples est analysé comme un sous-type de gapping par \citet{HankamerEtAl1976}, \citet{Williams1977}, \citet{Chao1988}, \citet{Gardent1991}, \citet{Lobeck1995}, \citet{Hendriks1995}, \citet{Hartmann2000} ou encore \citet{Toosarvandani2011}. En l’absence d’une description adéquate de ces constructions, je ne peux pas me prononcer sur leur lien avec le gapping. En revanche, les cas intermédiaires \REF{ch2:ex63}, avec deux éléments résiduels, ne semblent pas poser de problème particulier pour une analyse à gapping. Je ferai référence à ces exemples à plusieurs reprises dans ce chapitre.


\ea \label{ch2:ex63}
\ea Ne e greu, pretenţiile sunt mari, [dar \textbf{şi} răsplata pe măsură].
\glt ‘C’est difficile pour nous, les exigences sont grandes, mais la récompense aussi (sera) à la hauteur.’       

\ex  
\gll Ion  nu  merge  la  film  [şi  \textbf{nici}  Maria  la  teatru].\\
Ion  \textsc{neg}  va  à  film  et  non\_plus  Maria  à  théâtre\\
\glt ‘Ion n’ira pas au cinéma, ni Maria au théâtre.’

\ex  
\gll MaRIa\textsubscript{i}  îl  loveşte  pe  Ion\textsubscript{j},  [şi  \textbf{nu}  el\textsubscript{j}  pe  ea\textsubscript{i}].\\
Maria  \textsc{acc.3sg.m}  frappe  \textsc{dom}  Ion  et  \textsc{neg}  lui  \textsc{dom}  elle\\ 
\glt ‘C’est Maria qui frappe Ion, et pas l’inverse.’      
\z
\z

Les structures \isi{comparatives} constituent un autre type d’ellipse qui permet des séquences qui ressemblent au gapping (cf. \citealt{Zribi-Hertz1986,CulicoverEtAl2005,AmsiliEtAl2008}). Assimiler les séquences \isi{comparatives} comportant deux éléments résiduels aux occurrences typiques de gapping est un choix controversé, car on considère souvent que le gapping est compatible uniquement avec les coordonnants ; or, les marqueurs comparatifs ne sont pas habituellement inclus dans la liste des conjonctions. Une solution serait d’élargir l’inventaire des marqueurs de coordination, afin d’inclure les marqueurs comparatifs \citep{MatosEtAl2008}. Cependant, cette solution est problématique, car, contrairement aux conjonctions ordinaires, qui ne peuvent pas se combiner entre elles, les marqueurs comparatifs sont compatibles avec une conjonction (p.ex. \textit{plus que Paul} \textbf{\textit{et plus}}\textit{ que Jean}). De plus, même si l’on inclut les marqueurs comparatifs dans la classe des conjonctions, il reste néanmoins des différences notables entre le gapping dans la coordination et le gapping dans les \isi{comparatives}. \citet{Jackendoff1971} est le premier qui l’affirme, en se basant sur les différences observées entre \REF{ch2:ex64a} et \REF{ch2:ex64b}, montrant que les constructions \isi{comparatives} permettent beaucoup plus de types d’ellipse que les constructions coordonnées.

\ea
\ea Bill ate more peaches \textbf{than} \{Harry {\textbar} Harry did {\textbar} Harry did grapes {\textbar} Harry grapes {\textbar} Harry will grapes\}. \label{ch2:ex64a}      
\ex Bill ate the peaches \textbf{and} \{*Harry {\textbar} *Harry did {\textbar} *Harry did the grapes {\textbar} Harry the grapes {\textbar} *Harry will the grapes\}. \label{ch2:ex64b}
\z
\z

Il y a d’autres éléments qui suggèrent la souplesse des contraintes sur les constructions \isi{comparatives}, par rapport à celles agissant dans une structure coordonnée. Si la coordination ne permet que \is{analepse (forward ellipsis)}l’analepse \REF{ch2:ex65}, l’ellipse peut présenter les deux directions (\is{analepse (forward ellipsis)}analepse et \is{catalepse (backward ellipsis)}catalepse) dans les \isi{comparatives} \REF{ch2:ex66}.

\ea \label{ch2:ex65}
\ea Ion mănâncă un măr, \textbf{iar} Maria o pară.
\glt ‘Ion mange une pomme, et Maria une poire.’  
\ex 
\gll *Ion  un  măr,  \textbf{iar}  Maria  o  pară  mănâncă.\\
Ion  une  pomme  et  Maria  une  poire  mange\\
\glt ‘Ion mange une pomme, et Maria une poire.’
\z
\z


\ea \label{ch2:ex66}
\ea Ion s-a bagat şi el în discuţie \textbf{ca} musca în lapte.
\glt  ‘Ion s’est mêlé à la conversation comme une mouche dans le lait.’  

\ex  Exact \textbf{ca} o muscă în lapte, Ion s-a băgat în discuţie pe nepusă masă.
\glt  ‘Tout comme une mouche dans le lait, Ion s’est mêlé à la conversation d’une manière intempestive.’
\z
\z

Contrairement aux structures coordonnées \REF{ch2:ex67a}, dans une construction \is{comparatives}comparative (\ref{ch2:ex67b}--\ref{ch2:ex67c}) on peut avoir une discordance en ce qui concerne le temps, l’aspect ou le mode \citep{McShane2005} :

\ea
\ea *Ion pleacă azi, \textbf{iar} Maria ieri. \label{ch2:ex67a}
\glt ‘Ion part aujourd’hui et Maria est partie hier.’

\ex Ion se comportă cu mine acum \textbf{ca} Maria ieri. \label{ch2:ex67b}
\glt ‘Ion se comporte avec moi maintenant comme Maria (s’est comportée) hier.’  

\ex Maria se uita la mine \textbf{precum} câinele la stăpân. \label{ch2:ex67c}
\glt ‘Maria me regardait comme le chien (regarde) son maître.’
\z
\z

De plus, la contrainte de parallélisme sémantique et \is{contraste sémantique}contraste est moins stricte dans les \isi{comparatives}. Les éléments d’une \isi{paire contrastive} peuvent appartenir à des domaines assez éloignés, p.ex. la paire contrastive <\textit{Maria, câinele}> en \REF{ch2:ex68b}, ou encore ils peuvent avoir un statut syntaxique différent, p.ex. \is{affixe/clitique pronominal}affixe pronominal \textit{mă} vs. syntagme nominal en \REF{ch2:ex69b}.

\ea 
\ea \#Maria ascultă de profesor, \textbf{iar} câinele de stăpân.
\glt ‘Maria écoute son professeur, et le chien son maître.’  

\ex Maria ascultă de profesor, \textbf{precum} câinele de stăpân. \label{ch2:ex68b}
\glt  ‘Maria écoute son professeur, comme le chien son maître.’  
\z
\z


\ea
\ea 
\gll *Paul  mă  iubeşte  \textbf{şi}  Dan  pe  Ioana.\\
Paul  \textsc{acc.1sg} aime  et  Dan  \textsc{dom} Ioana\\
\glt ‘Paul m’aime, moi, et Dan aime Ioana.’  

\ex 
\gll Paul  mă  educă  \textbf{precum}  dascălul  pe  elevii  lui. \label{ch2:ex69b}\\
Paul  \textsc{acc.1sg} éduque  comme  maître.\textsc{def} \textsc{dom} élèves.\textsc{def}  \textsc{poss.3sg.m}\\
\glt ‘Paul m’éduque comme un maître ses disciples.’  
\z
\z

Enfin, on observe qu’une structure \is{comparatives}comparative elliptique peut être contenue dans son antécédent (\is{Antecedent Contained Ellipsis}\textit{Antecedent Contained Ellipsis}) en \REF{ch2:ex70} :

\ea \label{ch2:ex70}
Pentru omenire, zâmbetele sunt [\textbf{precum} soarele pentru flori].
\glt ‘Pour les humains, les sourires sont comme le soleil pour les fleurs.’
\z

Certes, une étude détaillée reste à faire pour voir si l’on peut envisager une analyse uniforme pour ces deux structures. Dans ce livre, je ne me prononce pas sur l’une ou l’autre des approches. Cependant, je considère que la reconstruction syntaxique ne marche pour aucune des deux constructions (voir \citealt{AmsiliEtAl2008} pour une approche similaire des comparatives en français). 

D’autres constructions permettant des séquences à deux éléments résiduels en dehors de la coordination sont les ellipses appelées \is{Sluicing}sluicing \REF{ch2:ex71a} et certaines subordonnées ayant la fonction ajout : ajouts circonstanciels \REF{ch2:ex71b}, ajouts additifs \REF{ch2:ex71c}, ajouts exceptifs \REF{ch2:ex71d} et ajouts relatifs sans verbe \REF{ch2:ex71e}. Les propriétés de ces constructions sont assez différentes des propriétés du gapping (voir chapitre~\ref{ch3}), c’est pour cela qu’on ne les discutera pas ici. 


\ea
\ea 
\gll Cineva  a  sărutat  pe  cineva,  dar  nu  ştiu  [cine  pe  cine]. \label{ch2:ex71a}\\
quelqu’un  a  embrassé  \textsc{dom} quelqu’un  mais  \textsc{neg}  sais.\textsc{1sg}  qui  \textsc{dom}  qui\\
\glt ‘Quelqu’un a embrassé quelqu’un, mais je ne sais pas qui a embrassé qui.’

\ex {} [Deşi pentru prima oară în străinătate], nu-i era deloc dor de ţară. \label{ch2:ex71b}
\glt ‘Quoique pour la première fois à l’étranger, son pays ne lui manquait pas du tout.’

\ex Toţi copiii au adus câte ceva, [inclusiv Maria o prăjitură]. \label{ch2:ex71c}
\glt ‘Tous les enfants ont apporté quelque chose, y compris Maria un gâteau.’

\ex Niciun elev nu-şi făcuse temele, [mai puţin Ion tema la engleză]. \label{ch2:ex71d}
\glt ‘Aucun élève n’avait fait ses devoirs, mis à part Ion le devoir d’anglais.’  

\ex Mai mulţi prieteni au plecat în străinătate, [dintre care 2 la Roma]. \label{ch2:ex71e}
\glt ‘Plusieurs amis sont partis à l’étranger, dont 2 à Rome.’
\z
\z

Pour conclure, on doit dire que les \is{fragment}phrases fragmentaires avec au moins deux éléments résiduels ne sont pas restreintes à la coordination. Cependant, leur occurrence dans la coordination (et \isi{juxtaposition}) est conditionnée par des con\-traintes particulières, qui ne s’appliquent pas en dehors de la coordination standard. En particulier, toutes les constructions discutées dans cette sous-section peuvent ne comporter qu’un seul élément résiduel, ce qui n’est pas le cas du gapping dans la coordination. Par conséquent, dans ce chapitre je me limite à la description de ce type d’ellipse en prenant en compte les structures coordonnées. Je reviendrai néanmoins sur un des types d’ajouts mentionnés ci-dessus (à savoir les relatives sans verbe) dans le chapitre~\ref{ch3}. 


\subsection{Contraintes générales sur le matériel manquant} \label{ch2:sect2.3.2}


Minimalement, le matériel manquant doit obligatoirement inclure le verbe tête de la phrase, qu’il s’agisse d’un \isi{auxiliaire} ou non \REF{ch2:ex72}. Ce qui inclut le gapping dans le groupe des ellipses sans tête, selon la typologie de \citet{Chao1988}. On explique dès lors l’impossibilité de garder le complémenteur dans une phrase trouée (cf. l’exemple \REF{ch2:ex33} ci-dessus), si l’on admet que les complémenteurs (au moins roum. \textit{că} ‘que’, fr. \textit{que} complétif) exigent la présence d’une tête verbale finie.

\ea \label{ch2:ex72}
\ea Maria \uline{mănâncă} un măr, iar Ion o pară.
\glt ‘Maria mange une pomme, et Ion une poire.’  

\ex Maria \uline{a mâncat} un măr, iar Ion o pară.
\glt ‘Maria a mangé une pomme, et Ion une poire.’
\z
\z

En dehors des formes verbales finies standard, on peut avoir comme matériel manquant un verbe au participe présent\footnote{
 Le participe présent est considéré comme ayant le comportement d’une forme finie en français (voir le placement de la \isi{négation} ou encore la possibilité d’être hôte des \is{affixe/clitique pronominal}clitiques pronominaux).} (roumain \REF{ch2:ex73} et français \REF{ch2:ex74}). On considère souvent que le gapping est limité aux phrases à verbe fini, mais l’exemple \REF{ch2:ex75} en français, où le matériel manquant correspond à un infinitif, nous oblige à alléger cette contrainte (au moins pour le français)\footnote{
 En roumain, on utilise le subjonctif dans ce type de contextes :
 
 \ea
 \gll {De ce}  \ulg{să}{10}  merg  în  China  și  {de ce}  în  {India ?}\\
 pourquoi  \textsc{sbjv} aller.\textsc{sbjv.1sg}  en  Chine  et  pourquoi  en  Inde\\
 \glt ‘Pourquoi aller en Chine et pourquoi en Inde ?’
 \z
 
}.                             

\ea \label{ch2:ex73}
Monitorizarea face parte din contractul de consultanţă [...], care prevede trei astfel de analize lingvistice, două dintre ele \uline{fiind efectuate} în martie şi mai, iar următoarea în septembrie, informează NewsIn. %(\url{http://www.cna.ro/Revista-Presei-CNA-20-iunie-2008.html})
\glt  ‘Le monitorage fait partie du contrat de consultance [...], qui prévoit trois analyses linguistiques de ce type, deux étant effectuées en mars et mai, et la suivante en septembre, informe NewsIn.’    
\z

\ea \label{ch2:ex74}
Paul \uline{étant pris} le matin et Marie l’après-midi, la réunion est reportée.
\z

\ea \label{ch2:ex75}
Pourquoi \uline{aller} en Chine et pourquoi en Inde ?
\z

Si le verbe est à un temps composé et contient un \isi{auxiliaire}, le gapping opère nécessairement sur les deux éléments en roumain \REF{ch2:ex76}, ce qui l’oppose à des langues comme le français \REF{ch2:ex77a} ou l’anglais \REF{ch2:ex77b}, dans lesquelles le trou peut correspondre seulement à un \isi{auxiliaire}. Cette différence peut être due au comportement clitique des \is{auxiliaire}auxiliaires en roumain, \is{auxiliaire}l’auxiliaire et le verbe auxilié formant un sous-constituant en syntaxe (structure à \isi{complexe verbal}, cf. \citealt{Barbu1999}, \citealt{Bilbiie2011}, etc.).

\ea \label{ch2:ex76}
\ea *Maria \uline{va} citi o poveste, iar Ion recita o poezie.
\glt ‘Maria va lire une histoire, et Ion réciter un poème.’  

\ex *Dan \uline{a} mâncat un sandviş, iar Maria băut o bere.
\glt ‘Dan a mangé un sandwich, et Maria bu une bière.’  
\z
\z

\ea
\ea Paul \uline{a} écrit un roman et Marie fini sa thèse. \label{ch2:ex77a}         
\ex Kim \uline{will} lead the party and Pat bring up the rear. \label{ch2:ex77b}         
\z
\z

En roumain \REF{ch2:ex78}, comme en français \REF{ch2:ex79a}, les formes verbales composées demandent l’élision de \is{auxiliaire}l’auxiliaire, si le participe passé ou l’infinitif est élidé. La situation est différente en anglais \REF{ch2:ex79b}, ce qui explique la possibilité du \is{Pseudogapping}pseudogap\-ping en anglais, mais pas dans les deux langues romanes. 

\ea \label{ch2:ex78}
\ea 
\gll Ion  a  \uline{mâncat}  mere,  iar  Maria  (*a)  banane.\\
Ion  a  mangé  pommes  et  Maria  a  bananes\\
\glt ‘Ion a mangé des pommes et Maria des bananes.’  

\ex 
\gll Ion  va  \uline{mânca}  mere,  iar  Maria  (*va)  banane.\\
Ion  va  manger  pommes  et  Maria  va  bananes\\
\glt ‘Ion va manger des pommes, et Maria des bananes.’
\z
\z


\ea
\ea Jean a \uline{mangé} des pommes et Marie (*a) des bananes. \label{ch2:ex79a}         
\ex John will \uline{vote} for Bush and Mary (will) for Nader. \label{ch2:ex79b}
\z
\z

Le trou peut correspondre à une \isi{expression idiomatique}\footnote{
 Si le matériel manquant contient toute l’\isi{expression idiomatique}, le gapping est parfaitement acceptable. En revanche, si le matériel manquant ne contient qu’une partie de l’expression idiomatique, le gapping est moins acceptable, voire agrammatical, cf. les exemples \REF{ch2:foot27i} en français (\textit{donner lieu} vs. \textit{donner naissance}) signalés par Olivier Bonami.

\ea \label{ch2:foot27i}
\ea *La mise en cause de Paul \uline{donne} \textbf{lieu} à une polémique et ses commentaires \textbf{naissance} à un scandale.
\ex La mise en cause de Paul \uline{donne lieu} à une polémique et ses commentaires à un scandale.
\z
\z
} : \textit{a da foc} litt. ‘donner feu’ pour ‘mettre le feu’ \REF{ch2:ex80a}, \textit{a-şi bate joc} litt. ‘se battre jeu’ pour ‘se moquer’ \REF{ch2:ex80b}.

\ea
\ea Ca să se amuze în lipsa părinţilor, băiatul \uline{a dat foc} grajdului, iar Maria grămezii de coceni. \label{ch2:ex80a}
\glt ‘Pour s’amuser en l’absence de leurs parents, le garçon a mis le feu à l’écurie, et Maria aux tas d’épis.’   

\ex Guvernul \uline{îşi bate joc} de munca Senatului, iar Senatul de munca Guvernului. \label{ch2:ex80b}
\glt ‘Le Gouvernement se moque du travail du Sénat, et le Sénat du travail du Gouvernement.’
\z
\z

A part la tête verbale, on peut omettre d’autres éléments (sujets \REF{ch2:ex81a}, compléments \REF{ch2:ex81b}, ajouts \REF{ch2:ex81c}) qui peuvent être de même niveau \REF{ch2:ex82a} ou enchâssés \REF{ch2:ex82b}. Certains ajouts (ou compléments optionnels) présents uniquement dans la phrase source peuvent s’interpréter dans les deux phrases (et dans ce cas ils font partie du matériel manquant) ou bien uniquement dans la phrase source \REF{ch2:ex83}. 

\ea
\ea La Valea Leurzii, \uline{întemeietorul de şcoală a fost} C. Ionescu, iar la Buciumeni, Ion Apostolescu, fiu al satului. \label{ch2:ex81a}
\glt ‘A Valea Leurzii, le fondateur de l’école a été C. Ionescu, et à Buciumeni, Ion Apostolescu, fils du village.’

\ex Ion \uline{îşi face temele} cu mama, iar Maria cu sora ei mai mare. \label{ch2:ex81b}
\glt ‘Ion fait ses devoirs avec sa mère, et Maria avec sa sœur aînée.’   

\ex Ion \uline{aleargă în parc} dimineaţa, iar Maria seara. \label{ch2:ex81c}
\glt ‘Ion court dans le parc le matin, et Maria le soir.’
\z
\z


\ea
\ea Mama \uline{i-a făcut un cadou Mariei} de Paşte, iar tata de Crăciun. \label{ch2:ex82a}
\glt ‘La mère a fait un cadeau à Maria à Pâques, et le père à Noël.’

\ex Dan \uline{şi-a dorit să înceapă să scrie} o nuvelă, iar Maria o piesă de teatru. \label{ch2:ex82b}
\glt ‘Dan a voulu commencer à écrire une nouvelle, et Maria une pièce de théâtre.’
\z
\z

\ea \label{ch2:ex83}
\ea Ion \uline{merge \textbf{cu familia}} la munte, iar Dan la mare.
\glt ‘Ion va avec sa famille à la montagne, et Dan à la mer.’

\ex Deși ne iubim foarte mult, ajungem la divergențe cu privire la lucruri pe care eu \uline{le-am învățat} într-un fel \uline{\textbf{în familia mea}}, iar el în alt fel.
\glt ‘Bien qu’on s’aime beaucoup, on arrive à des divergences sur des trucs que j’ai appris d’une certaine façon dans ma famille, et lui d’une autre façon.’

\ex Ion \uline{\textbf{întotdeauna} merge} la film, iar Maria la teatru.
\glt ‘Ion va toujours au cinéma, et Maria au théâtre.’   

\ex Ion \uline{\textbf{tocmai} a sosit} acum 5 minute, iar Maria azi-dimineaţă.
\glt ‘Ion vient d’arriver il y a 5 minutes, et Maria (est arrivée) ce matin.’
\z
\z

Le trou ne correspond pas nécessairement à un constituant. Les éléments manquants peuvent être discontinus \REF{ch2:ex84} ou en position finale \REF{ch2:ex85}. Contrairement au roumain, le français ne permet pas facilement le positionnement final du matériel manquant, ayant une préférence pour une construction \is{pseudoclivée}(pseudo)clivée \REF{ch2:ex86}. L’acceptabilité de ces exemples en roumain s’explique par trois propriétés de cette langue : (i) \is{ordre de mots}l’ordre des mots assez libre, (ii) le manque d’isomorphisme entre la fonction syntaxique et la position dans l’arbre, et (iii) le marquage prosodique des éléments focalisés.

\ea \label{ch2:ex84}
\ea Ion \uline{merge} SÂMbăta \uline{la piaţă}, iar Maria duMInica.
\glt ‘Ion va le samedi au marché, et Maria le dimanche.’

\ex Ion \uline{crede că} FRANţa \uline{va câştiga}, iar Maria ArgenTIna.
\glt ‘Ion croit que la France va gagner, et Maria l’Argentine.’
\z
\z


\ea \label{ch2:ex85}
\ea In sectorul 4, PoPEScu \uline{are şanse să câştige}, iar în sectorul 1, PăuNEScu.
\glt  ‘Dans le 4\textsuperscript{ème} arrondissement, Popescu a une chance de gagner (les élections), et dans le 1\textsuperscript{er}, Păunescu.’

\ex La noi în casă, păRINţii \uline{iau deciziile}, dar la voi, coPIii.
\glt ‘Chez nous, les parents prennent les décisions, mais chez vous, les enfants.’
\z
\z


\ea \label{ch2:ex86}
\ea ??Chez nous, les parents \uline{décident} et chez vous les enfants.
\ex Chez nous, les parents \uline{décident} et chez vous, ce sont les enfants.
\ex Chez nous, \uline{ce sont} les parents \uline{qui décident} et chez vous, les enfants. \z
\z

On a longtemps considéré que la règle du gapping était bloquée lorsque la phrase source comportait une \isi{négation} \citep{Ross1967}. La conclusion des premiers travaux (\citealt{Ross1967,Jackendoff1971,Zribi-Hertz1986}, etc.)~est que la \isi{négation} ne peut pas être élidée, à moins que la phrase trouée ne contienne la conjonction \textit{nor} ou \textit{or} \REF{ch2:ex87}. 

\ea \label{ch2:ex87}
\ea *I didn’t eat fish and Bill ice-cream. \citep{Sag1976}          
\ex I didn’t eat fish nor/or Bill ice-cream.           
\z
\z

\citet{Repp2009} fait une analyse exhaustive du comportement de la \isi{négation} dans les constructions à gapping, en montrant non seulement que la \isi{négation} est tout à fait acceptable dans ces contextes, mais surtout qu’il y a plusieurs interprétations possibles si on a une \isi{négation} dans la phrase source. Ainsi, on obtient trois lectures : (i) \isi{négation} distribuée sur les deux conjoints, (ii) \isi{négation} avec \isi{portée étroite}, et (iii) \isi{négation} avec \isi{portée large}. Les facteurs qui jouent sur l’interprétation de la \isi{négation} dans le gapping seraient, selon elle, l’intonation, le type de conjonction utilisée, le type de \isi{négation} (propositionnelle pour les deux premières lectures, ou illocutoire pour la \isi{portée large}), la présence de certains opérateurs sémantiques dans la phrase trouée ou bien la forme du trou verbal (verbe fini, \isi{auxiliaire} avec ou sans verbe fini, modal, item à \isi{polarité} négative). Je montre brièvement la présence de ces trois interprétations de la \isi{négation} dans le gapping en roumain.

Dans le premier cas, la \isi{négation} est distribuée sur les deux conjoints : (¬A) $\land$ (¬B). La phrase trouée, tout comme la phrase source, est négative. Selon \citet{Repp2009}, c’est la lecture par défaut dans les constructions à gapping. Au niveau prosodique, chaque conjoint constitue une unité prosodique autonome, et le verbe antécédent ne reçoit pas d’accent prosodique particulier. 

\ea
\ea La nunta Anei, lui Ion \uline{\textbf{nu} i-a plăcut} muzica, iar Mariei mâncarea.
\glt ‘Au mariage d’Ana, Ion n’a pas aimé la musique, et Maria la nourriture.’

\ex (¬A) $\land$ (¬B) = [Ce n’est pas le cas que Ion ait aimé la musique] et [ce n’est pas le cas que Maria ait aimé la nourriture].
\z
\z

La deuxième interprétation se résume à une \isi{portée étroite} de la \isi{négation} : (¬A) $\land$ (B). La \isi{négation} s’interprète uniquement dans la phrase source, tandis que la phrase trouée est positive. Chaque conjoint est une unité prosodique autonome (cf. \citealt{Oehrle1987}). La \isi{négation} ainsi que les éléments contrastés sont marqués prosodiquement. Cette interprétation est disponible en roumain au moins dans trois contextes : (i) de manière générale, dans tous les emplois de la conjonction corrective \textit{ci} ‘mais’ \REF{ch2:ex89}, (ii) si la phrase trouée contient un \isi{adverbe associatif}, comme le restrictif \textit{doar} ‘seulement’ \REF{ch2:ex90a}, et (iii) si la phrase trouée est introduite par un connecteur adversatif/argumentatif, p.ex. \textit{însă, dar} ‘mais’ \REF{ch2:ex90b}.

\ea \label{ch2:ex89}
\ea \textbf{Nu} Ion\textsubscript{i} \uline{o loveşte} pe Maria, \textbf{ci} Maria pe el\textsubscript{i}.
\glt  ‘Ce n’est pas Ion qui frappe Maria, mais Maria Ion.’

\ex (¬A) $\land$ (B) = [Ce n’est pas le cas que Ion frappe Maria], mais [c’est le cas que Maria frappe Ion].
\z
\z

\ea \label{ch2:ex90}
\ea A : - Ce au cumpărat Ion şi Maria de la târg ? B : - Ion \textbf{n}-\uline{a cumpărat} mai nimic, iar Maria \textbf{doar} o pereche de papuci. \label{ch2:ex90a}
\glt A : ‘- Qu’est-ce que Ion et Maria ont acheté au foire ?’ B : ‘- Ion n’a pas acheté grand-chose, et Maria seulement une paire de chaussons.’  

\ex  A : - La câte întrebări au răspuns Ion şi Maria ? B : - Ion \textbf{n}-\uline{a răspuns} la aproape nicio întrebare, \textbf{însă} Maria la toate, şi încă fără greşeală. \label{ch2:ex90b}
\glt A : ‘- A combien de questions ont répondu Ion et Maria ?’ B : ‘- Ion n’a répondu à presque aucune question, mais Maria à toutes, et sans faute.’
\z
\z

Enfin, il y a des constructions à gapping avec une «~montée~» sémantique de la \isi{négation}. Dans ces contextes, la \isi{négation} (souvent avec un modal) dans la phrase source prend \isi{portée large} sur la coordination dans son ensemble (une seule \isi{négation} qui porte sur les deux conjoints) : ¬(A $\land$ B). \citet{Repp2009} considère que la \isi{négation} dans ce contexte n’est pas une \isi{négation} propositionnelle (comme dans les deux premiers cas), mais plutôt une \isi{négation} au niveau illocutoire. De plus, contrairement aux deux autres lectures, la \isi{portée large} de la \isi{négation} est corrélée au niveau prosodique avec le fait que les deux conjoints forment une seule unité prosodique (pas de pause possible entre les conjoints), cf. \citet{Oehrle1987}. La \isi{négation} (plus le modal, ou l’\isi{auxiliaire} en anglais) est prosodiquement proéminente, cf. \citet{Winkler2005}. Cette interprétation est plus difficile à obtenir hors contexte, c’est pour cela que je reprends le modèle de \citet{Repp2009} en \REF{ch2:ex91} et je fabrique des contextes pour forcer cette interprétation en roumain \REF{ch2:ex92}.

\ea \label{ch2:ex91}
Kim DIDn’t play bingo and Sandy sit at home all night. I am sure Sandy went to a club herself. That’s what she always does when Kim plays bingo. \citep[171]{Repp2009}         
\z

\ea {}
[\textit{Contexte : Ion et Maria sont frère et sœur. Ion est devenu très riche, mais sa sœur est restée très pauvre. Les gens commentent~le fait que Ion n’aide pas sa sœur.}] \label{ch2:ex92}
\ea Ion \textbf{nu poate} locui într-un palat şi Maria într-o cocioabă. Trebuie să facă ceva să-şi ajute sora !
\glt ‘Ion ne peut pas habiter dans un palais et Maria dans une baraque. Il doit faire quelque chose pour aider sa sœur.’  

\ex ¬(A $\land$ B) = Ce n’est pas le cas que [Ion habite dans un palais et Maria dans une baraque].
\z
\z


\subsubsection{Degré d’identité entre le trou et son antécédent}

Avant de passer en revue les contraintes générales sur le matériel manquant, on doit préciser le degré d’identité qui s’établit entre le matériel antécédent et le matériel manquant, en relevant d’abord les ressemblances et ensuite les différences. 

\paragraph{Ressemblances}

En ce qui concerne les ressemblances, deux aspects sont très importants. Premièrement, le verbe manquant doit appartenir au même \isi{paradigme de flexion} que le verbe antécédent et avoir le même sens. Généralement il s’agit du même lexème, sauf dans certains exemples avec des formes homonymes ou avec des \is{zeugme}zeugmes sémantiques (ou attelages), présents dans certaines citations littéraires, mais rejetés dans l’usage ordinaire, qui ont une lecture ironique incitée par le jeu de mot (voir les exemples français en \REF{ch2:ex93}). Dans ces occurrences inattendues, le matériel manquant et le verbe antécédent appartiennent au même \is{paradigme de flexion}paradigme flexionnel, mais avec deux acceptions différentes d’un même terme (ils n’ont pas le même sens et constituent donc deux lexèmes différents). Leur emploi reste cependant très marginal\footnote{
 Pour plus de détails sur la description et l’analyse de ces exemples, voir \citet{Clement2010}.}. 

\ea \label{ch2:ex93}
\ea Son corps \uline{nageait} dans l’eau verte, et son esprit dans l’opulence. (Troyat, cité par \citealt[233]{Clement2010})  
\ex La pie \uline{vole} des bijoux et l’oiseau vers son nid.
\z
\z

Cette contrainte sur l’identité lexématique nous permet de rendre compte de l’inacceptabilité de l’exemple en \REF{ch2:ex94a}, où le verbe antécédent et le matériel manquant n’appartiennent pas au même \isi{paradigme de flexion} (voir la flexion différente \textit{acord} vs. \textit{acordez} au présent en \REF{ch2:ex94a} et \REF{ch2:ex94c}), n’ont pas la même valence et n’ont pas le même sens (\textit{a acorda}\textsubscript{1} : octroyer une aide à quelqu’un \textit{vs.} \textit{a acorda}\textsubscript{2} : régler un instrument musical) ; il s’agit clairement de deux lexèmes distincts.


\ea
\ea \#Pe perioada concediului, eu \uline{am acordat} ajutoare săracilor, iar soţul meu piane. \label{ch2:ex94a}
\glt ‘Pendant nos congés d’été, j’ai accordé des aides aux pauvres, et mon mari des pianos.’ 

\ex Eu \textbf{acord} ajutoare săracilor. \label{ch2:ex94b}
\glt ‘J’accorde des aides aux pauvres.’ 

\ex Eu \textbf{acordez} piane. \label{ch2:ex94c}
\glt ‘J’accorde des pianos.’
\z
\z

Les mêmes contraintes s’appliquent aux autres éléments faisant partie du matériel manquant. Ainsi, en \REF{ch2:ex95}, où le matériel manquant contient une copule suivie d’un nom prédicatif, il faut que le nom prédicatif du matériel manquant soit du même lexème que le nom prédicatif antécédent\footnote{
 Jason Merchant (c.p.) me signale que le même effet est observé en \ili{espagnol} ou en \ili{grec} où la base morphologique est la même ; selon lui, c’est plutôt la différence d’interprétation visée qui cause l’inacceptabilité, et un peu moins la différence de flexion.}.

\ea \label{ch2:ex95}
\ea 
\gll *Filip  \ulg{e}{11}  frate  cu  directorul,  iar  Maria  cu  secretara.\\
Filip  est  frère  avec  directeur.\textsc{def}  et  Maria  avec  secrétaire.\textsc{def}\\
\glt ‘Filip est le frère du directeur et Maria est la sœur de la secrétaire.’

\ex 
\gll *Maria  \ulg{e}{10.2}  soră  cu  secretara,  iar  Filip  cu  directorul.\\
Maria  est  sœur  avec  secrétaire.\textsc{def}  et  Filip  avec  directeur.\textsc{def}\\
\glt ‘Maria est la sœur de la secrétaire, et Filip est le frère du directeur.’ 

\ex 
\gll Filip  \ulg{e}{11}  frate  cu  directorul,  iar  Florin  (e  frate)  cu  secretara.\\
Filip  est  frère  avec  directeur.\textsc{def}  et  Florin  (est  frère)  avec  secrétaire.\textsc{def}\\
\glt ‘Filip est le frère du directeur et Florin (est le frère) de la secrétaire.’

\ex 
\gll Maria  \ulg{e}{10.2}  soră  cu  secretara,  iar  Ioana  (e  soră)  cu  directorul.\\
Maria  est  sœur  avec  secrétaire\textsc{.def}  et  Ioana  (est  sœur)  avec  directeur.\textsc{def}\\
\glt ‘Maria est la sœur de la secrétaire, et Ioana (est la sœur) du directeur.’ 
\z
\z

Deuxièmement, ils doivent partager les mêmes propriétés de temps\footnote{
 On parle ici du gapping typique de la coordination. On a vu en (\ref{ch2:ex67b}--\ref{ch2:ex67c}) ci-dessus que les \isi{comparatives} autorisaient une différence de temps entre les événements des deux conjoints.}, mode, voix et aspect. Ainsi, en \REF{ch2:ex96}, on ne peut pas avoir de temps ou de modes différents. De même, les exemples en \REF{ch2:ex97} montrent qu’une \is{asymétrie syntaxique}discordance de voix (passive-active ou bien active-passive) est possible s’il n’y a pas ellipse ; en revanche, l’emploi du gapping impose une identité de voix entre les deux conjoints.


\ea \label{ch2:ex96}
\ea 
\gll *Ion  \ulg{a}{10}  sosit  ieri,  iar  Maria  mâine.\\
Ion  a  arrivé  hier  et  Maria  demain\\
\glt ‘Ion est arrivé hier, et Maria arrivera demain.’

\ex   
\gll *Ion  \ulg{ar}{34.1}  merge  la  \ule{film}  azi,  iar  Maria  ieri.\\ 
Ion  \textsc{aux.cond.3}  aller  à  film  aujourd’hui  et  Maria  hier\\
\glt ‘Ion irait au cinéma aujourd’hui, et Maria y serait allée hier.’
\z
\z

\ea \label{ch2:ex97}
\ea 
\gll Ion  \ulg{a}{21}  fost  muşcat  de  un  câine,  iar  pe  Ana  *(a  muşcat-o)  o  şopârlă.\\
Ion  a  été  mordu  de  un  chien  et  \textsc{dom} Ana  a  mordu-\textsc{acc.3sg.f} un  lézard\\
\glt ‘Ion a été mordu par un chien, et Ana par un lézard.’

\ex  
\gll Pe  Ana  \ulg{a}{17}  muşcat-o  o  şopârlă,  iar  Ion  *(a  fost  muşcat)  de  un  câine.\\
\textsc{dom} Ana  a  mordu-\textsc{acc.3sg.f} un  lézard  et  Ion  a  été  mordu  de  un  chien\\ 
\glt  ‘Ana a été mordue par un lézard, et Ion par un chien.’
\z
\z

Pour les langues qui présentent des marques aspectuelles, comme le \ili{russe}, on observe qu’on doit avoir le même aspect dans une construction à gapping : en \ili{russe} \REF{ch2:ex98}, le gapping n’accepte pas de discordance imperfectif vs. perfectif dans les deux conjoints.

\ea \label{ch2:ex98}
\il{russe}\langinfo{Russe}{}{\citealt[9]{Repp2009}}\\
\gll *Wtchera  ja  pisala  pismo  dwa  tchasa,  a  ty  \st{napisala} \st{pismo}  za  dri tchasa.\\ 
hier  je  écrire.\textsc{pst.dur} lettre  2  heures  et  tu  écrire.\textsc{pst.perf} lettre  en 3 heures\\ 
\glt ‘Hier j’ai écrit une lettre pendant 2 heures et toi en 3 heures.’             
\z

Selon \citet{Repp2009}, les propriétés TAM (temps-aspect-mode) ont une fonction d’ancrage référentiel dans le monde factuel. La phrase trouée est une phrase non ancrée, qui a besoin d’emprunter son ancrage à la phrase source. Ce rôle d’ancrage empêche ainsi les TAM d’avoir des valeurs différentes dans la phrase trouée par rapport à la phrase source.

\paragraph{Différences} 

Du côté des différences, on observe que le matériel manquant ne possède pas nécessairement les mêmes propriétés de personne \REF{ch2:ex99a} ou de nombre \REF{ch2:ex99b}.

\ea
\ea 
\gll Eu  \uline{vreau}  un  ceai,  iar  Ioana  (vrea)  o  cafea. \label{ch2:ex99a}\\
je  veux  un  thé  et  Ioana  (veut)  un  café\\
\glt ‘Je veux un thé, et Ioana un café.’

\ex 
\gll Noi  \uline{citim}  o  carte,  iar  tu  (citeşti)  un  ziar. \label{ch2:ex99b}\\
nous  lisons  un  livre  et  tu  (lis)  un  journal\\
\glt ‘Nous lisons un livre, et toi un journal.’
\z
\z

Quant au genre, on observe une certaine \is{asymétrie syntaxique}asymétrie en fonction de la catégorie qui apparaît dans la composition du matériel manquant, et cela dans les deux langues (roumain et français)\footnote{
 \is{asymétrie syntaxique}L’asymétrie en genre est par ailleurs possible dans d’autres constructions elliptiques. Voir \citet{Merchant2011} pour une discussion sur ce type d’asymétrie avec \is{ellipse nominale}l’ellipse nominale en \ili{grec}.}. S’il s’agit d’un verbe au participe (\REF{ch2:ex100a} et \REF{ch2:ex101a}) ou d’un adjectif prédicatif (\REF{ch2:ex100b} et \REF{ch2:ex101b}), on n’a pas nécessairement le même genre. En revanche, si le matériel manquant contient un nom prédicatif, il doit généralement avoir le même genre que le nom prédicatif dans la phrase source. Les \is{asymétrie syntaxique}asymétries de genre qui sont permises concernent uniquement les formes syncrétiques \REF{ch2:ex102a} ou les formes homophones (\ref{ch2:ex102b}--\ref{ch2:ex102c}) en français, ou bien les noms de métier qui permettent l’emploi du masculin pour les deux genres (\REF{ch2:ex103a} et \REF{ch2:ex104a}). Si le matériel antécédent contient un nom prédicatif sous sa forme au féminin, le matériel manquant ne peut pas correspondre à un nom prédicatif au masculin (\REF{ch2:ex103b} et \REF{ch2:ex104b}). 

\ea
\ea
\gll Fata  \ulg{e}{12.3}  iubită  de  toţi,  dar  băiatul  (nu  e  iubit)  de nimeni. \label{ch2:ex100a}\\
fille.\textsc{def}  est  aimée  de  tous  mais  garçon.\textsc{def}  (\textsc{neg} est  aimé)  de  personne\\
\glt ‘La fille est aimée par tous, mais le garçon par personne.’

\ex Maria \uline{e încântată} de noua ei rochie, iar Ion (e încântat) de noua lui maşină. \label{ch2:ex100b}
\glt ‘Maria est contente de sa nouvelle robe, et Ion de sa nouvelle voiture.’
\z
\z


\ea
\ea La lettre \uline{a été écrite} par la secrétaire et le mail (a été écrit) par le directeur. \label{ch2:ex101a} 
\ex Jean \uline{est content} de son travail et Marie (est contente) de ses vacances. \label{ch2:ex101b}
\z
\z

\ea
\ea Jean \uline{est secrétaire} dans un garage BMW à Paris et Maria (est secrétaire) dans un collège à Lyon. \label{ch2:ex102a} 
\ex Jean \uline{est ami} avec Paul et Marie (est amie) avec Sophie. \label{ch2:ex102b}
\ex Marie \uline{est amie} avec Sophie et Jean (est ami) avec Paul. \label{ch2:ex102c}
\z
\z


\ea
\ea Ion \uline{e profesor} la un liceu din Cluj, iar Ana la o şcoală generală din Iaşi. \label{ch2:ex103a}
\glt ‘Ion est professeur dans un lycée à Cluj, et Ana dans une école élémentaire à Iaşi.’

\ex  ??Ana \uline{e profesoară} la o şcoală generală din Iaşi, iar Ion la un liceu din Cluj. \label{ch2:ex103b}
\glt ‘Ana est professeure dans une école élémentaire à Iaşi, et Ion dans un lycée à Cluj.’ 

\ex  
\gll Ion  e  \{profesor  {\textbar}  *profesoară\}.\\
Ion  est  \{professeur.\textsc{m}  {\textbar}  professeur.\textsc{f\}}\\
\glt  ‘Ion est professeur.’ 

\ex 
\gll Ana  e  \{profesor  {\textbar}  profesoară\}.\\
Ana  est  \{professeur.\textsc{m}  {\textbar}  professeur.\textsc{f\}}\\
\glt ‘Ana est professeur.’ 
\z
\z


\ea
\ea Patrick \uline{est directeur} de l’UFR et Marie du laboratoire. \label{ch2:ex104a} 
\ex *Marie \uline{est directrice} du laboratoire et Patrick de l’UFR. \label{ch2:ex104b}
\ex Patrick est \{directeur {\textbar} *directrice\}.
\ex Marie est \{directrice {\textbar} directeur\}.
\z
\z

Ce comportement rappelle la distinction entre \isi{flexion inhérente} et \isi{flexion contextuelle}, due à \citet{Booij1994,Booij1996,Booij2007}\footnote{
 Merci à Olivier Bonami pour ce commentaire.}. Le premier terme fait référence à la flexion d’un mot qui n’est pas demandée par le contexte syntaxique (p.ex. le genre sur les noms, le pluriel des noms, les marques de temps sur un verbe), alors que le deuxième terme caractérise toute flexion qui est dictée par le contexte syntaxique dans lequel un mot apparaît (p.ex. les marques \is{accord}d’accord en genre sur un adjectif, \is{accord}l’accord entre un verbe et un sujet, le \isi{marquage casuel}). Pour résumer les différences liées aux marques de personne, nombre et genre dans les constructions à gapping, on peut donc dire que le gapping maintient la \isi{flexion inhérente}, mais pas nécessairement la \isi{flexion contextuelle}.

Deuxièmement, le matériel manquant ne prend pas nécessairement les mêmes \is{affixe/clitique pronominal}affixes pronominaux ou \is{affixe/clitique adverbial}adverbiaux\footnote{
 Pour plus de détails sur le statut affixal des \is{affixe/clitique pronominal}pronoms atones et de certains \is{affixe/clitique adverbial}adverbes (la \isi{négation} \textit{nu} ‘non’ et cinq ‘semi-adverbes’) en roumain, voir \citet{Barbu1999,Barbu2003}, \citet{Monachesi2005} et \citet{Bilbiie2011}.}. On a deux cas de figure : soit les affixes en question, s’ils sont reconstruits avec un verbe dans la phrase trouée, n’ont pas la même forme que les affixes dans la phrase source (voir l’opposition masculin/féminin en \REF{ch2:ex105a}\footnote{
 Le \isi{redoublement clitique} des objets est une propriété générale du roumain. Pour l’objet direct [+spécifique] et [+animé], le \isi{redoublement clitique} est co-occurrent avec le \is{marquage différentiel de l'objet}marquage différentiel de l'objet par la forme \textit{pe}.} et pluriel/singulier en \REF{ch2:ex105b}), soit ces affixes ne sont présents que dans un des conjoints (en \REF{ch2:ex106a} \is{affixe/clitique pronominal}l’affixe pronominal \textit{le} ‘les’ ne peut apparaître que dans la phrase source, alors qu’en \REF{ch2:ex106b} \is{affixe/clitique adverbial}l’affixe adverbial \textit{nu} ‘non’ est interprété uniquement dans la phrase trouée). Les affixes ont le même comportement en français : on n’a pas les mêmes affixes pronominaux en \REF{ch2:ex107} ; \is{affixe/clitique pronominal}l’affixe pronominal \textit{en} en \REF{ch2:ex108a} et \is{affixe/clitique adverbial}l’affixe adverbial \textit{ne} en \REF{ch2:ex108b} s’appliquent uniquement à la phrase trouée.

\ea
\ea 
\gll Ion \textbf{l}-\ulg{a}{23.9}  văzut  pe  Dan,  iar  Ana  (a  văzut-\textbf{o})  pe  Maria. \label{ch2:ex105a}\\
Ion \textsc{acc.3sg.m-}a  vu  \textsc{dom} Dan  et  Ana  (a  vu-\textsc{acc.3sg.f}) \textsc{dom}  Maria\\
\glt ‘Ion a vu Dan, et Ana (a vu) Maria.’

\ex  
\gll Eu  \textbf{i}-\ulg{am}{18}  \ule{văzut}  pe  [Ion  şi  Maria],  iar  Ana  (\textbf{l}-a  văzut)  pe  Paul. \label{ch2:ex105b}\\ 
je  \textsc{acc.3pl.m}-ai  vu  \textsc{dom} Ion  et  Maria  et  Ana  (\textsc{acc.3sg.m}-a  vu)  \textsc{dom}  Paul\\
\glt ‘J’ai vu Ion et Maria, et Ana (a vu) Paul.’
\z
\z

\ea
\ea  
\gll Ion  \textbf{le}-\ulg{a}{16}  \ule{citit}  pe  toate,  dar  Ana  ((*\textbf{le}-)a  citit)  doar  câteva. \label{ch2:ex106a}\\ 
Ion  \textsc{acc.3pl.f}-a  lu  \textsc{dom} toutes  mais  Ana  (\textsc{acc.3pl.f}-a  lu)  seulement  quelques\\
\glt ‘Ion les a tous lus, mais Ana seulement quelques-uns.’

\ex  
\gll Ion  \ulg{a}{9}  citit  câteva  dintre  ele,  dar  Ana  (*(\textbf{nu})  a  citit)  absolut  niciuna. \label{ch2:ex106b}\\
Ion  a  lu  quelques  parmi  elles  mais  Ana  (\textsc{neg} a  lu)  absolument  aucune\\
\glt ‘Ion en a lu quelques-uns, mais Ana absolument aucun.’ 
\z
\z

\ea \label{ch2:ex107}
\ea Paul \textbf{en} \uline{a lu} seulement certains, mais Marie (\textbf{les} a) presque tous (lus).   \ex Paul \textbf{les} \uline{a lus}, vos livres, et Marie (\textbf{en} a lu) seulement certains.   
\z
\z

\ea
\ea Paul \uline{a lu} tous vos livres et Marie (\textbf{en} a lu) quelques-uns. \label{ch2:ex108a}      
\ex Paul \uline{en a lu} certains, et Marie (*(\textbf{n}’)en a lu) absolument aucun. \label{ch2:ex108b}
\z
\z

Troisièmement, la \isi{polarité} dans les deux conjoints n’est pas toujours la même. D’une part, on a des cas où la phrase source est positive, alors que la phrase trouée est négative \REF{ch2:ex109}\footnote{
 Le roumain est une langue à \isi{concordance négative} stricte, les mots négatifs étant co-occurrents avec la \isi{négation} \textit{nu} ‘non’. \citet{Falaus2008} et \citet{Iordachioaia2010} analysent les mots négatifs du roumain comme des \is{quantifieur négatif}quantifieurs négatifs (cf. ils peuvent apparaître sans un autre élément négatif, introduisant à eux seuls une \isi{négation} sémantique).}. D’autre part, il y a des cas où la \isi{négation} est présente uniquement dans la phrase source et a \isi{portée étroite}, ne s’appliquant donc pas à la phrase trouée ; pour ce deuxième cas de figure, voir les exemples mentionnés ci-dessus en \REF{ch2:ex90}.


\ea \label{ch2:ex109}
\ea De ce unii \uline{au} totul, iar eu (\textbf{nu} am) \textbf{nimic} ?
\glt ‘Pourquoi certains ont tout, et moi (je n’ai) rien ?’

\ex  Eram la un metru depărtare de un vampir care \uline{ştia} prea multe despre mine, iar eu (\textbf{nu} ştiam) \textbf{nimic} despre el.
\glt ‘J’étais à un mètre d’un vampire qui savait trop de choses sur moi, et moi (je ne savais) rien sur lui.’ 

\ex Tu \uline{ai primit} mereu cadouri de ziua ta, dar eu (\textbf{nu} am primit) \textbf{nimic}.
\glt ‘Tu as toujours reçu des cadeaux pour ton anniversaire, mais moi (je n’ai) rien (reçu).’

\ex Mulți tipi \uline{au fantezii} cu femei cu forme, dar \textbf{niciunul} (\textbf{nu} are fantezii) cu femei doar piele și os.
\glt ‘Beaucoup d’hommes ont des fantasmes avec des femmes rondes, mais aucun (n’a de fantasme) avec des femmes trop maigres.’ 

\ex Eu \uline{îmi cer scuze} întotdeauna, dar el (\textbf{nu}-și cere scuze) \textbf{niciodată}.
\glt ‘Je demande toujours des excuses, mais lui jamais.’

\ex Mariei \uline{îi plac} doar merele şi perele, însă Ioanei (\textbf{nu}-i plac) \textbf{nici măcar} astea.
\glt ‘Maria aime seulement les pommes et les poires, mais Ioana (n’aime) même pas celles-ci.’ 
\z
\z


\subsection{Contraintes générales sur les éléments résiduels} \label{ch2:sect2.3.3}


Contrairement à d’autres types d’ellipse, la séquence trouée dans les coordinations à gapping doit comporter au moins deux éléments résiduels\footnote{
A ne pas confondre l’exemple \REF{ch2:ex110} avec l’exemple \REF{ch2:foot36i} qui contient un seul élément résiduel accompagné obligatoirement d’un adverbial \textit{şi} ‘aussi’ (et de la conjonction homonyme \textit{şi} ‘et’). A priori, les propriétés (distribution, intonation, etc.) ne semblent pas être les mêmes dans les deux cas, d’où l’hypothèse selon laquelle dans l’exemple \REF{ch2:foot36i} on a une ellipse polaire plutôt que du gapping.
 
\ea 
\gll Ioana  mănâncă  un  măr,  şi  şi  Maria. \label{ch2:foot36i}\\
 Ioana  mange  une  pomme  et  aussi  Maria\\
\glt ‘Ioana mange une pomme, et Maria aussi.’
\z

}, mis en correspondance avec des éléments parallèles dans la phrase source \REF{ch2:ex110}.                      

\ea {}
[Ioana] \uline{mănâncă} [un măr], iar [Maria] [*(o pară)].\label{ch2:ex110}
\glt ‘Ioana mange une pomme, et Maria une poire.’    
\z

La séquence trouée peut contenir plus de deux éléments résiduels \REF{ch2:ex111}. Certains travaux sur le gapping en anglais (\citealt{Jackendoff1971,Kuno1976,Haspelmath2007}, etc.) considèrent que le gapping permet strictement deux éléments résiduels. Cependant, on observe que, même en anglais, la présence des éléments résiduels multiples n’est pas bloquée par une contrainte grammaticale (cf. \citealt{Kuno1976,SagEtAl1985,Zribi-Hertz1986,Steedman1990}).

\ea \label{ch2:ex111}
\ea De Paşte, părinţii \uline{au mers} la mare cu bunicii, iar copiii la munte cu prietenii.
\glt ‘A Pâques, les parents sont allés à la mer avec les grands-parents, et les enfants à la montagne avec leurs amis.’

\ex  Seara, Ion \uline{vorbeşte} cu prietena lui pe Skype, iar Maria cu amantul pe Messenger.
\glt  ‘Le soir, Ion parle avec sa copine sur Skype, et Maria avec son amant sur Messenger.’ 
\z
\z

Les jugements d’acceptabilité sont sensibles à des contraintes psycholinguistiques de \isi{processing} (ou traitement de l’information). En particulier, selon \citet{Zribi-Hertz1986}, la restriction sur le nombre de constituants relève de la performance. De manière générale, les exemples qui posent un problème d’acceptabilité ont des séquences elliptiques composées uniquement de syntagmes nominaux\footnote{
 Voir, dans ce sens, \citet[157]{SagEtAl1985} : «~processing difficulty associated with sequences of NPs found in ellipsis contexts~».}. Leur traitement est difficile sous deux aspects. D’abord, une séquence elliptique composée uniquement de syntagmes nominaux est beaucoup plus difficile à traiter qu’une séquence où les éléments résiduels reçoivent un marquage morpho-syntaxique (\isi{marquage casuel}, préposition, etc.). C’est ce qui expliquerait les différences dans l’acceptabilité des exemples \REF{ch2:ex112} en anglais : \REF{ch2:ex112b} est meilleur que \REF{ch2:ex112a}, car les trois éléments résiduels ont chacun un marquage différent (aucun marquage pour le premier, la préposition \textit{with} pour le deuxième, la préposition \textit{about} pour le troisième). Comme le roumain est une langue à \isi{marquage casuel} et \is{marquage prépositionnel}prépositionnel, les séquences à trois éléments résiduels ne posent pas de problème particulier \REF{ch2:ex113}. 

\ea \label{ch2:ex112}
\ea  *Millie \uline{will send} the President an obscene telegram, and [Paul] [the Queen] [a pregnant duck]. \citep[25]{Jackendoff1971} \label{ch2:ex112a} 
\ex  Some \uline{talked} with you about politics and [others] [\textbf{with} me] [\textbf{about} music]. \citep[193]{Winkler2005} \label{ch2:ex112b} 
\z
\z


\ea \label{ch2:ex113}
\ea 
\gll Ion  \ulg{i-a}{16.8} dat  Mari\textbf{ei} o  carte,  iar  [Dan]  [Ioan\textbf{ei}]  [niște  flori].\\
Ion  \textsc{dat.3sg}{}-a  donné  Maria.\textsc{dat} un  livre  et  Dan  Ioana.\textsc{dat}  des  fleurs\\ 
\glt  ‘Ion a offert un livre à Maria, et Dan des fleurs à Ioana.’

\ex  Eu \uline{am vorbit} cu Ion despre Maria, iar [tu] [\textbf{cu} Dan] [\textbf{despre} Ana].
\glt  ‘J’ai parlé avec Ion au sujet de Maria, et toi avec Dan au sujet de Ana.’  

\ex  Steaua \uline{l-a ales} pe M. Stoica director sportiv, iar [Dinamo] [\textbf{pe} Rednic] [antrenor].
\glt  ‘Steaua a choisi M. Stoica comme directeur sportif, et Dinamo (a choisi) Rednic comme entraîneur.’
\z
\z

Il semble aussi que les séquences avec des éléments résiduels de même type sémantique (p.ex. individus) peuvent poser plus de problèmes d’acceptabilité que les séquences avec des éléments résiduels de type différent. Ainsi, l’acceptabilité de l’exemple \REF{ch2:ex114a} est dégradée à cause d’un nombre important de noms d’individus. Cependant, \citet{Johnson2014} considère que les phrases à trois éléments résiduels (tous des syntagmes nominaux) s’améliorent s’il s’agit d’une réponse à une \isi{question multiple} \REF{ch2:ex114b}. Tous ces aspects que je viens de mentionner montrent l’importance des facteurs non syntaxiques dans l’acceptabilité des constructions à gapping avec plus de deux éléments résiduels. Ces facteurs sont responsables aussi d’autres «~violations~» qu’on observe dans certains contextes de gapping (en particulier, les \isi{contraintes de localité} discutées plus loin), ce qui nous oblige à réfuter une approche purement syntaxique de ce type d’ellipse.  

\ea
\ea  *Arizona \uline{elected} Goldwater Senator, and Massachussets McCormack Congressman. \citep[25]{Jackendoff1971} \label{ch2:ex114a}
\ex  A : - Who will send who what ?

B : - Sally \uline{will send} Ron pickles, and Martha Hermione kumquats. \citep{Johnson2014} \label{ch2:ex114b}

\z
\z

La phrase trouée prototypique contient généralement un élément résiduel correspondant à un sujet dans la phrase source, donc la séquence typique est sujet-complément ou sujet-ajout. Mais, étant donné qu’on peut omettre d’autres éléments, en dehors de la tête verbale ({\cad} sujets, compléments, ajouts) et que le roumain a un \isi{ordre de mots} assez libre, les éléments résiduels peuvent avoir des fonctions~différentes par rapport au verbe antécédent. Ainsi, la phrase trouée peut être une séquence complément-sujet \REF{ch2:ex115a}, ajout-sujet \REF{ch2:ex115b}, complément-complément \REF{ch2:ex115c}, ajout-complément \REF{ch2:ex115d}, complément-ajout \REF{ch2:ex115e}\footnote{
 Je considère que la position préverbale dans les exemples \REF{ch2:ex115} ne correspond pas à une fonction syntaxique spécifique (p.ex. \textit{antéposé/extrait}, voir l’exemple \REF{ch2:foot38exi} en français), mais plutôt à une fonction discursive ({\cad} en l’absence d’un marquage prosodique particulier, le premier élément est interprété généralement comme un \isi{topique} du point de vue discursif). 
 
 \ea \label{ch2:foot38exi}
 A Pierre, \uline{Paul a promis d’apporter} un disque et à Marie un livre.
 \z

}. 

\ea \label{ch2:ex115}
\ea 
\gll Mariei  \ulg{îi}{17.1}  plac  fructele,  iar  Ioanei  prăjiturile. \label{ch2:ex115a}\\
Maria.\textsc{dat} \textsc{dat.3sg}  plaisent  fruits.\textsc{def}  et  Ioana.\textsc{dat}  gâteaux.\textsc{def}\\
\glt  ‘Maria aime les fruits, et Ioana les gâteaux.’

\ex  
\gll Ieri  \ulg{a}{11.2}  venit  Ion,  iar  azi  Maria. \label{ch2:ex115b}\\
hier  est  venu  Ion  et  aujourd’hui  Maria\\
\glt  ‘Hier, c’est Ion qui est venu, et aujourd’hui, c’est Maria.’   

\ex  
\gll Mariei  \ulg{i-am}{15}  dat  o  carte,  iar  Ioanei  un  stilou. \label{ch2:ex115c}\\
Maria.\textsc{dat} \textsc{dat.3sg}{}-ai  donné  un  livre  et  Ioana.\textsc{dat}  un  stylo\\
\glt  ‘A Marie j’ai donné un livre, et à Ioana un stylo.’

\ex  
\gll Dimineaţa  \uline{mănânc}  cereale,  iar  seara  fructe. \label{ch2:ex115d}\\
matin.\textsc{def.f}  mange.\textsc{1sg}  céréales  et  soir.\textsc{def.f}  fruits\\
\glt  ‘Le matin je mange des céréales, et le soir des fruits.’   

\ex  
\gll Engleza  \ulg{o}{21.4}  învăţ  la  şcoală,  iar  franceza  acasă. \label{ch2:ex115e}\\
anglais.\textsc{def}  \textsc{acc.3sg.f}  apprends.\textsc{1sg}  à  école  et  français\textsc{.def}  à\_maison\\
\glt  ‘L’anglais, je l’apprends à l’école, et le français à la maison.’
\z
\z

En dehors des deux éléments résiduels, la séquence trouée peut contenir un adverbe de phrase, qui modifie la séquence en entier et qui sémantiquement prend comme argument une proposition ou un \isi{acte illocutoire}, dans le cas des énonciatifs et de certains connecteurs \citep{BonamiEtAl2005}. Ainsi, en roumain, la séquence trouée peut comporter un connecteur \REF{ch2:ex116a}, un adverbe modal \REF{ch2:ex116b}, évaluatif \REF{ch2:ex116c} ou énonciatif \REF{ch2:ex116d}. Cela est un argument pour considérer la séquence trouée comme ayant un contenu propositionnel. 

\ea
\ea Eu \uline{am zis} o vorbă, \textbf{apoi} el alta, și uite așa am ajuns la conflict. \label{ch2:ex116a}
\glt  ‘J’ai dit un mot, ensuite lui un autre, et c’est ainsi qu’on est arrivé au conflit.’

\ex  Ion \uline{vine} azi, iar Maria \textbf{probabil} mâine. \label{ch2:ex116b}
\glt  ‘Ion vient aujourd’hui, et Maria probablement demain.’

\ex  
\gll La  examen,  Ion  \ulg{a}{9}  luat  nota  10,  iar  Maria,  \textbf{din  nefericire},  nota  2. \label{ch2:ex116c}\\
à  examen  Ion  a  pris  note.\textsc{def}  10  et  Maria  par  malheur  note.\textsc{def}  2\\
\glt  ‘Ion a eu 10/10 à l’examen et Maria malheureusement 2/10.’  

\ex  
\gll Cu  el  \ulg{am}{20}  rezolvat  eu  ceva,  dar  cu  ea,  \textbf{sincer},  nimic. \label{ch2:ex116d}\\
avec  lui  ai  résolu  je  quelque\_chose  mais  avec  elle  franchement  rien\\
\glt  ‘Avec lui, j’ai résolu quelque chose, mais avec elle, franchement, rien.’
\z
\z

\textbf{Les éléments résiduels sont des \is{constituant majeur}constituants majeurs}. Minimalement les éléments résiduels sont des projections maximales (\citealt{Sag1976,Hartmann2000}, etc.), ce qui explique l’agrammaticalité des exemples suivants, où le deuxième élément résiduel ne peut pas former de syntagme à lui tout seul (un déterminant sans nom \REF{ch2:ex117a} ou une préposition transitive sans son complément \REF{ch2:ex117b}). En revanche, si l’on substitue aux éléments en question un pronom \REF{ch2:ex118a} ou une préposition utilisée intransitivement \REF{ch2:ex118b} (qui chacun forme un syntagme unaire), les exemples ne posent aucun problème de grammaticalité\footnote{
 \citet{Hartmann2000} et \citet{Repp2009} considèrent que les prépositions ne peuvent pas être des \is{constituant majeur}constituants majeurs en \ili{allemand}. Mais il faudrait faire la différence entre les prépositions toujours transitives et les prépositions qui peuvent avoir un emploi intransitif.}.  

\ea
\ea 
\gll *Lui  Ion  \ulg{îi}{19}  place  acest  costum,  iar  Mariei,  acel. \label{ch2:ex117a}\\
\textsc{dat} Ion \textsc{dat.3sg}  plaît  ce  costume  et  Maria.\textsc{dat} \textsc{det.dem.sg.m}\\
\glt  ‘Ion aime ce costume, et Maria celui-là.’       

\ex  
\gll *Maria  \ulg{îşi}{29.6}  pune  geanta  sub  masă,  iar  Ion  pe. \label{ch2:ex117b}\\ 
Maria  \textsc{refl.3sg} pose  serviette.\textsc{def}  sous  table  et  Ion  sur\\
\glt ‘Maria pose sa serviette sous la table, et Ion au-dessus.’
\z
\z


\ea
\ea 
\gll Lui  Ion  \ulg{îi}{19}  place  acest  costum,  iar  Mariei,  acela. \label{ch2:ex118a}\\
\textsc{dat}  Ion \textsc{dat.3sg}  plaît  ce  costume  et  Maria.\textsc{dat} \textsc{dem.sg.m}\\
\glt  ‘Ion aime ce costume, et Maria, celui-là.’       

\ex  
\gll Maria  \ulg{îşi}{29.6}  pune  geanta  sub  masă,  iar  Ion,  deasupra. \label{ch2:ex118b}\\
Maria  \textsc{refl.3sg} pose  serviette.\textsc{def}  sous  table  et  Ion  au-dessus\\
\glt  ‘Maria pose sa serviette sous la table, et Ion, au-dessus.’
\z
\z

Essentiellement, ces éléments résiduels doivent être des \is{constituant majeur}\textbf{constituants majeurs}, {\cad} des valents ou ajouts d’une tête verbale (ou prédicative) dans la phrase source. Par conséquent, on ne peut pas avoir comme élément résiduel une sous-partie d’un \isi{constituant majeur} \citep{Hankamer1971,Hankamer1973,Neijt1979,Gardent1991,Hartmann2000}, dépendant d’une tête non verbale et non prédicative, p.ex. un nom sans son déterminant \REF{ch2:ex119a}, un syntagme nominal sans sa tête prépositionnelle\footnote{
 Apparemment, certains locuteurs acceptent les exemples sans préposition en anglais (voir \citealt{CulicoverEtAl2005}), alors que la plupart les considèrent agrammaticaux (voir \citealt{Gardent1991}). \citet{Chaves2005} considère qu’il y a une gradience dans l’acceptabilité de ces exemples. 
\ea
\ea  *Jim reads a book to Fred, and Mary, Peter. (\citealt{Chaves2005} : 11) 
\ex  ?John is going to Japan, and his sister, Australia.
\ex  Jim reads to his brother, and Mary, our kids.                       
\z
\z
} \REF{ch2:ex119b}, un syntagme nominal (ou un syntagme prépositionnel) complément d’un autre syntagme nominal \REF{ch2:ex119c}, un syntagme adjectival ajout à un syntagme nominal \REF{ch2:ex119d}, etc. 

\ea
\ea Paul \uline{a mâncat} \textbf{o} portocală, iar Maria *(\textbf{o}) banană. \label{ch2:ex119a}
\glt  ‘Paul a mangé une orange, et Maria une banane.’   

\ex  Maria \uline{vorbeşte} \textbf{cu} un avocat, iar Ion *(\textbf{cu}) o actriţă. \label{ch2:ex119b}
\glt  ‘Maria parle avec un avocat, et Ion avec une actrice.’       

\ex  Ion \uline{citeşte} \textbf{introducerea} unui roman, iar Ana *(\textbf{introducerea}) unui eseu. \label{ch2:ex119c}
\glt  ‘Ion lit l’introduction d’un roman, et Ana l’introduction d’un essai.’     

\ex 
\gll Ion  \ulg{şi-a}{22.4}  vândut  \textbf{maşina}  albastră,  iar  Maria  *(\textbf{maşina})  roşie. \label{ch2:ex119d}\\
Ion  \textsc{refl.3sg}-a  vendu  voiture.\textsc{def}  bleue  et  Maria  (voiture.\textsc{def})  rouge\\
\glt  ‘Ion a vendu sa voiture bleue, et Maria sa voiture rouge.’ 
\z
\z

Les exemples qui semblent violer cette contrainte sont en fait de faux contre-exemples, car les éléments résiduels qui à première vue semblent être des dépendants d’une tête nominale et non prédicative peuvent être en fait réanalysés comme des compléments du verbe tête \REF{ch2:ex120}. Ainsi, en \REF{ch2:ex120a}, le syntagme adjectival \textit{dulce} ‘douce’ n’est pas un ajout à la tête nominale \textit{ciorba} ‘la soupe’ dans la phrase source, mais un attribut de l’objet.

\ea \label{ch2:ex120}
\ea 
\gll Mariei \ulg{îi}{29.4} place ciorba acră, iar lui Ion, dulce. \label{ch2:ex120a}\\
Maria.\textsc{dat} \textsc{dat.3sg} plaît soupe.\textsc{def} aigre et \textsc{dat} Ion douce\\
\glt ‘Maria aime la soupe aigre, et Ion douce.’ 

\ex 
\gll Mie  \ulg{îmi}{30.4}  place  ciocolata  cu  mentă,  iar  lui,  cu  stafide.\\
moi\textsc{.dat}  \textsc{dat.1sg} plaît  chocolat.\textsc{def}  avec  menthe  et  lui.\textsc{dat}  avec  raisins\_secs\\
\glt  ‘Moi, j’aime le chocolat à la menthe, et lui aux raisins secs.’
\z
\z

Dans la plupart des cas, tous les éléments résiduels dépendent de la même tête qui est à la fois verbale et prédicative, {\cad} le verbe racine. Cependant, il y a des cas où l’un des éléments résiduels dépend (i) d’une tête non verbale mais prédicative, ou bien (ii) d’une tête verbale mais enchâssée. 

(i) Le premier cas est illustré par les constructions à \isi{prédicat complexe}. Le deuxième élément résiduel peut dépendre d’un complément du verbe racine, si celui-ci est un verbe attributif \REF{ch2:ex121} ou un verbe support \REF{ch2:ex122}. Ces exemples ne posent pas de problème pour la condition de \isi{constituant majeur}, car l’élément résiduel est (ré)analysé dans ces cas comme un complément du verbe tête, via l’héritage ou la «~composition d’arguments~» (cf. \citealt{AbeilleEtAl2003a}). La condition générale qui doit être remplie est que les éléments résiduels doivent être légitimés par \textbf{une} des \is{tête prédicative}têtes prédicatives de la phrase source, tout en respectant l’ordre licite de mots dans la grammaire.

\ea \label{ch2:ex121}
\ea Paul \uline{e foarte mândru} de fiul lui, iar Maria, (foarte mândră) de fiica ei.\\
\glt ‘Paul est très fier de son fils, et Marie (très fière) de sa fille.’   

\ex Cel din stânga mea \uline{e premierul} pentru criză, iar cel din dreapta, (premierul) pentru haos. (\textit{Dilema veche} VII : 341)
\glt ‘Celui à ma gauche est le ministre pour la crise, et celui à ma droite (le ministre) pour le chaos.’

\newpage 
\ex Unii \uline{devin dependenţi} de exerciţiile fizice, alţii, (dependenţi) de curele de slăbire.
\glt ‘Certains deviennent dépendants aux exercices physiques, d’autres (dépendants) aux cures d’amincissement.’
\z
\z


\ea \label{ch2:ex122}
\ea Hasan \uline{a făcut o călătorie} la Mecca, iar Elena (o călătorie) la muntele Athos.
\glt  ‘Hasan a fait un voyage à la Mecque, et Elena, (un voyage) au Mont Athos.’   

\ex  Criminalii \uline{au teamă} de polițiști, iar polițiștii (teamă) de criminali.
\glt  ‘Les criminels ont peur des policiers, et les policiers (peur) des criminels.’

\ex  Decebal \uline{a dus lupte aprige} cu romanii, iar Ştefan cel Mare (lupte aprige) cu turcii.
\glt  ‘Decebal a mené de lourds combats contre les Romains, et Stefan le Grand contre les Turcs.’

\ex  Băiatului \uline{i-a pus numele} Flavius, iar fetei (numele) Dorina.
\glt  ‘Au garçon, on lui a donné le nom de Flavius, et à la fille (le nom de) Dorina.’
\z
\z

(ii) Un deuxième cas est illustré par les exemples dans lesquels un des éléments résiduels ne dépend pas directement du verbe racine, mais d’un verbe enchâssé. L’exemple classique est celui de \citet{Ross1970} en \REF{ch2:ex123}, avec plusieurs infinitifs enchâssés en anglais. En roumain, l’exemple typique de verbe enchâssé est le subjonctif en \REF{ch2:ex124}, qui, dans l’état actuel de la langue, prend de plus en plus la place de l’infinitif \citep[392]{GALR2005}. Le gapping opère donc facilement à travers une complétive non marquée par un complémenteur, comme c’est le cas des subordonnées au subjonctif introduites par \textit{să} en roumain\footnote{
 La forme \textit{să} précédant les formes verbales au subjonctif n’est pas un complémenteur, mais une marque flexionnelle. Voir les détails dans \citet{Barbu1999} et \citet[chapitre 1]{Bilbiie2011}.}.

\ea \label{ch2:ex123}
I \uline{want to try to begin to write} a novel and Mary a play. \citep[250]{Ross1970}
\z

\ea \label{ch2:ex124}
\ea
\gll Ion  \ulg{încearcă}{18}  să  intre  la  drept,  iar  Maria  la  medicină.\\
Ion  essaie  \textsc{sbjv}  entrer.\textsc{sbjv.3} à  droit  et  Maria  à  médecine\\
\glt  ‘Ion essaie de passer l’examen d’entrée à la faculté de droit, et Maria à la faculté de médecine.’  

\ex  
\gll Ion  \ulg{pare}{16.3}  să  fie  bolnav,  iar  Maria  obosită.\\
Ion  semble  \textsc{sbjv}  être.\textsc{sbjv.3} malade  et  Maria  fatiguée\\
\glt  ‘Ion semble être malade, et Maria fatiguée.’

\ex  
\gll Dan  \ulg{şi-a}{68.5}  dorit  să  înceapă  a  scrie  o  nuvelă,  iar  Ana un poem.\\
Dan  \textsc{refl.3}{}-a  désiré  \textsc{sbjv}  commencer.\textsc{sbjv.3}  \textsc{inf}  écrire  une  nouvelle  et  Ana un  poème\\
\glt  ‘Dan a voulu commencer à écrire une nouvelle, et Ana un poème.’
\z
\z

Un cas plus complexe toujours en lien avec les verbes enchâssés concerne les subordonnées introduites par un vrai complémenteur. Si l’on accepte comme élément résiduel un dépendant d’un verbe enchâssé à l’infinitif (ou au subjonctif en roumain), les choses ne sont pas claires avec les autres types d’enchâssement, qui sont plus complexes. Ainsi, \citet{Koutsoudas1971}, \citet{Hankamer1979}, \citet{Wilder1994}, \citet{Johnson1996/2004}, \citet{Williams1997}, etc. considèrent qu’on ne peut pas avoir d’élément résiduel enchâssé dans une complétive en \textit{that} (\textit{que} en français, \textit{că} ‘que’ en roumain). \citet{Gardent1991} donne des exemples acceptables en anglais \REF{ch2:ex125}, mais elle considère que la grammaticalité est liée au statut syntaxique de la subordonnée par rapport au verbe racine : les complétives en \textit{that} vs. les circonstancielles ajouts (comparer (\ref{ch2:ex125a}--\ref{ch2:ex125b}) et \REF{ch2:ex125c}). Cependant, un regard attentif des données laisse entrevoir des exemples qui sont acceptables en roumain (et en français).    

\ea \label{ch2:ex125}
\ea  This doctor \uline{said that I should eat} salmon and that doctor tuna. \label{ch2:ex125a}
\ex  The child \uline{insisted that she wanted} chips and the mother salad. \label{ch2:ex125b}
\ex  *John \uline{left without telling} his boss and Bill his colleagues. \citep{Gardent1991} \label{ch2:ex125c}
\z
\z

Je me limite ici aux discussions sur les complétives en angl. \textit{that}, fr. \textit{que} et roum. \textit{că} ‘que’. Les autres exemples d’enchâssement seront mentionnés dans la section~\ref{ch2:sect2.4.3.1}, où je discute les \isi{contraintes de localité}. On trouve un exemple classique dans l’œuvre de \ia{Diderot, Denis}Diderot en \REF{ch2:ex126}, avec un gapping multiple, où le deuxième élément résiduel est toujours un dépendant d’un verbe enchâssé dans une complétive en \textit{que}. 

\ea \label{ch2:ex126}
Et les voilà embarqués dans une querelle interminable sur les femmes ; l’un \uline{prétendant qu’elles étaient} bonnes, l’autre méchantes : et ils avaient tous deux raison ; l’un sottes, l’autre pleines d’esprit : et ils avaient tous deux raison ; l’un fausses, l’autre vraies : et ils avaient tous deux raison ; [...] l’un folles, l’autre sensées, l’un grandes, l’autre petites : et ils avaient tous deux raison. (Diderot, \textit{Jacques le fataliste et son maître})
\z

Il est vrai que les jugements d’acceptabilité ne sont pas clairs. Mais ce qui a été observé et en anglais et en français laisse penser que ce n’est pas une contrainte syntaxique qui rend compte de ce genre d’exemples, mais des facteurs sémantiques et psycholinguistiques. En français, le gapping à travers les complétives en \textit{que} est tout à fait acceptable si le sujet enchâssé est un clitique explétif impersonnel \REF{ch2:ex127a}, ou un clitique référentiel \REF{ch2:ex127b}, et assez dégradé si le sujet enchâssé est un syntagme nominal \REF{ch2:ex127c}. De même, \citet[113]{Merchant2001}, \citet{Lasnik2006} et \citet{Repp2009} observent qu’en anglais l’élément résiduel peut être dépendant d’un verbe enchâssé dans une complétive en \textit{that} si le sujet racine et le sujet enchâssé sont \is{coréférence}coréférents (voir les données en \REF{ch2:ex128} et \REF{ch2:ex129}).

\ea
\ea Paul \uline{\{dit / pense\} qu’\textbf{il} faut aller} à Rome et Marie à Florence. \label{ch2:ex127a}
\ex Paul \uline{dit qu’\textbf{il} est allé} à Rome et Marie à Florence. \label{ch2:ex127b}
\ex ??Paul \uline{dit que \textbf{l’orage} a détruit} la culture de seigle et Marie la culture de blé.\label{ch2:ex127c}
\z
\z

\ea \label{ch2:ex128}
\ea  Jim \uline{said that he called} his mum and John his dad.
\ex *Jim \uline{claimed that Alan went} to the ballgame and John to the movies. \citep[12]{Repp2009}     
\z
\z

\ea \label{ch2:ex129}
\ea  John\textsubscript{i} \uline{thinks that \textbf{he}\textbf{\textsubscript{i}} will see} Susan and Harry Mary.
\ex *John \uline{said \textbf{you} kissed} Mary, and Bill Mary. \citep{Lasnik2006}
\z
\z

Quant au roumain, on peut avoir des enchâssées dont le sujet n’est pas \is{coréférence}co\-référent au sujet racine, mais dans la plupart de ces exemples, il s’agit d’un sujet «~inclus~» dans la flexion verbale ({\cad} \isi{pro-drop}). Le \isi{pro-drop} du sujet joue probablement un rôle dans leur acceptabilité. Le seul problème est qu’en roumain ces exemples, en l’absence d’un contexte spécifique, posent un problème d’ambiguïté concernant le niveau auquel opère le gapping : au niveau de la phrase racine ou bien au niveau des phrases subordonnées. Hors contexte particulier, on a une préférence pour un gapping plus «~bas~», dans la subordonnée uniquement, mais il reste à vérifier quel type de facteur joue sur cette préférence (en particulier, s’il s’agit d’une contrainte grammaticale ou si c’est lié à la facilité du \isi{processing}). Si on arrive à avoir un bon contexte, avec une intonation particulière, corrélée parfois avec la \isi{juxtaposition}, on arrive à établir le parallélisme avec la phrase racine, comme en \REF{ch2:ex130}.

\ea \label{ch2:ex130}
\ea 
\gll Care  e  cauza  reală  a  {varicelor ?}  Unii  \ulg{spun}{16}  că  sunt  genele, alții  –  tocurile  înalte,  alții  –  excesul  de  greutate...\\ 
quelle  est  cause.\textsc{def}  réelle  \textsc{gen}  varices.\textsc{gen}  certains  disent  que  sont  gènes.\textsc{def} d’autres  –  talons.\textsc{def}  hauts  d’autres  –  excès.\textsc{def}  de  poids\\
\glt ‘Quelle est la cause réelle des varices ? Certains disent que c’est les gènes, d’autres – les hauts talons, d’autres – le surpoids...’

\ex 
\gll Cardiologul  \ulg{mi-a}{51.1}  spus  că  ar  trebui  să  \uline{mănânc} mai multe  lipide,  iar  nutriţionistul,  mai  multe  glucide.\\
cardiologue.\textsc{def}  m-a  dit  que  \textsc{aux.cond.3} devoir  \textsc{sbjv} manger.\textsc{sbjv.1sg}  plus beaucoup  lipides  et  nutritionniste.\textsc{def}  plus  beaucoup  glucides\\
\glt ‘Le cardiologue m’a dit que je devrais manger plus de lipides, et le nutritionniste, plus de glucides.’ 
\z
\z

Les mêmes contraintes non syntaxiques qu’on avait invoquées pour expliquer les différences d’acceptabilité pour le gapping avec plus de deux éléments résiduels, peuvent rendre compte de l’acceptabilité des exemples en \REF{ch2:ex131} : l’exemple \REF{ch2:ex131a} est meilleur que l’exemple \REF{ch2:ex131b}, car les éléments résiduels ne sont sémantiquement pas de même nature (humain \textit{vs}. pays) et, de plus, il met en jeu les connaissances encyclopédiques du locuteur, ce qui facilite l’organisation des éléments résiduels et corrélats en \is{paire contrastive}paires contrastives.  

\ea \label{ch2:ex131}
\ea Ion \uline{crede că} FRANţa \uline{va câştiga}, iar Maria ArgenTIna. \label{ch2:ex131a}
\glt  ‘Ion pense que la France va gagner, et Maria l’Argentine.’

\ex  ??Ion \uline{crede că} Ana \uline{va câştiga}, iar Maria Ioana. \label{ch2:ex131b}
\glt  ‘Ion pense que Ana va gagner, et Maria Ioana.’ 
\z
\z

Pour conclure, les éléments résiduels doivent pouvoir être mis en correspondance avec des \is{constituant majeur}constituants majeurs dans la phrase source, et en particulier avec les dépendants d’un verbe racine ou enchâssé. A priori, tous les problèmes qu’on rencontre avec l’enchâssement ne sont pas d’ordre syntaxique, mais plutôt sémantique et/ou psycholinguistique, ce qu’on observera aussi avec les \isi{contraintes de localité}.

\subsection{Contraintes de parallélisme} \label{ch2:sect2.3.4}

\subsubsection{Syntaxe} \label{ch2:sect2.3.4.1}


Un des arguments majeurs qu’on mentionne habituellement en faveur d’une \isi{reconstruction syntaxique} dans les constructions elliptiques est la présence des \is{effets de connectivité}effets de~«~connectivité~» discutés dans le chapitre~\ref{ch1}, section~\ref{ch1:sect1.5.1.1}, {\cad} un parallélisme structural entre la phrase trouée et la phrase source, en ce qui concerne les propriétés morpho-syntaxiques des éléments résiduels (\isi{marquage casuel}, \isi{marquage prépositionnel}, catégorie et fonction syntaxique, nombre des éléments, \is{ordre de mots}ordre des mots, etc.), cf. \citet{Hartmann2000,CulicoverEtAl2005,Culicover2009}.

Dans les constructions à gapping, on observe qu’un élément résiduel doit avoir la même fonction syntaxique et généralement la même marque casuelle. L’identité fonctionnelle est une conséquence de la contrainte d’identité sémantique qu’on discutera dans la section~\ref{ch2:sect2.3.4.2} : ainsi, en \REF{ch2:ex132a}, on ne peut avoir une paire contrastive <\textit{cartea, noaptea}>, où \textit{cartea} ‘le livre’ est un complément, alors que \textit{noaptea} ‘la nuit’ est un ajout par rapport au verbe antécédent \textit{a citi} ‘lire’. De même, une paire contrastive comporte généralement la même marque casuelle, à l’exception de quelques idiosyncrasies casuelles en roumain, comme c’est l’exemple du datif en \REF{ch2:ex132b}, où le syntagme contenant un élément quantitatif (p.ex. \textit{la} \textbf{\textit{trei} }\textit{dintre copii} ‘à trois parmi les enfants’) reçoit la marque prépositionnelle \textit{la} (demandant une forme d’accusatif), alors que son correspondant résiduel dans la séquence trouée comporte la marque synthétique (affixée) habituelle (p.ex. \textit{tutu}\textbf{\textit{ror}} \textit{copii}\textbf{\textit{lor}} ‘à tous les enfants)\footnote{
 Philip Miller (c.p.) signale un exemple similaire en anglais où il ne s’agit pas de la même valence dans les deux phrases :
 
 \ea
 John \uline{sent} [Mary] roses and Bob a nice book [to Paul].
 \z
}.


\ea
\ea \#Ion \uline{citeşte} cartea, iar Maria noaptea. \label{ch2:ex132a}
\glt  ‘Ioana lit le livre, et Maria lit la nuit.’ 

\ex  
\gll Ion  \ulg{oferă}{11}  mere  [\textbf{la}  trei  dintre  copii],  iar  Maria  [tutu\textbf{ror}  copii\textbf{lor}]. \label{ch2:ex132b}\\
Ion  offre  pommes  à  trois  parmi  enfants  et  Maria  tous.\textsc{dat} enfants.\textsc{dat}\\ 
\glt ‘Ion offre des pommes à trois des enfants, et Maria à tous les enfants.’ 
\z
\z

En revanche, on observe que le parallélisme structural n’est pas strict en ce qui concerne la catégorie grammaticale, le nombre de dépendants réalisés, ainsi que l’ordre dans lequel apparaissent les éléments résiduels par rapport à leurs corrélats dans la phrase source. Ainsi, comme \citet{SagEtAl1985} l’observent pour l’anglais, un élément résiduel et son corrélat n’ont pas nécessairement la même catégorie syntaxique (p.ex. syntagme nominal \textit{vs.} syntagme prépositionnel en \REF{ch2:ex133a} ou bien syntagme nominal \textit{vs.} phrase en \REF{ch2:ex133b}), à condition que chacune de ces catégories constitue un dépendant possible du prédicat antécédent. 

\ea
\ea Ioana \uline{citeşte} [ziua]\textsubscript{NP}, iar Maria [pe-ntuneric]\textsubscript{PP}. \label{ch2:ex133a}
\glt  ‘Ioana lit pendant la journée, et Maria dans l’obscurité.’ 

\ex  
\gll Mie  \ulg{îmi}{15.7}  place  [muzica]\textsubscript{NP},  iar  prietenului  meu  [să  facă  sport]\textsubscript{S}. \label{ch2:ex133b}\\
moi.\textsc{dat} \textsc{dat.1sg}  plaît  musique.\textsc{def}  et  ami.\textsc{dat}  \textsc{poss.1sg}  \textsc{sbjv}  faire.\textsc{sbjv.3} sport\\ 
\glt ‘Moi j’aime la musique, et mon ami (aime) faire du sport.’ 
\z
\z

De plus, la phrase trouée peut comporter un nombre de dépendants différent du nombre de dépendants dans la phrase source. On a deux cas de figure : (i) un élément résiduel des deux \is{paire contrastive}paires contrastives n’a pas de corrélat lexical dans la phrase source (le corrélat est ainsi implicite)\footnote{\label{ch2:fn43}
 On pourrait rajouter ici les exemples (\ref{ch2:foot43i}--\ref{ch2:foot43ii}), dans lesquels la séquence qui suit la conjonction \textit{iar} est analysée comme une phrase trouée (voir la section~\ref{ch2:sect2.6}). Les éléments mis en gras n’ont pas de correspondant explicite dans le premier conjoint, mais leur présence est obligatoire pour la grammaticalité des exemples (cf. la contrainte du double \is{contraste sémantique}contraste avec la conjonction \textit{iar}).
 
 \ea
 \ea Nu am nicio legătură cu biserica, \uline{sunt} [un simplu credincios], iar *(\textbf{de meserie}) [şofer]. \label{ch2:foot43i}
 \glt  ‘Je n’ai aucun lien avec l’église, je suis un simple croyant, et quant à mon métier, chauffeur.’

\ex Ioana \uline{mănâncă} [un măr], iar *(\textbf{apoi}) [o pară]. \label{ch2:foot43ii}
\glt ‘Ioana mange une pomme, et ensuite une poire.’
\z
\z
}, ou bien (ii) en dehors des deux paires contrastives, la phrase trouée contient un élément supplémentaire. Le premier cas est illustré par le phénomène du \isi{pro-drop} : on observe ainsi qu’en roumain l’élément résiduel sujet peut ne pas avoir de corrélat lexical dans la phrase source (\is{subject drop}\textit{subject drop}), et cela avec toutes les personnes, comme illustré en \REF{ch2:ex134} ; de même, on peut avoir un résiduel objet sans corrélat explicite dans la phrase source (\is{object drop}\textit{object drop}), comme illustré par les exemples \REF{ch2:ex135} en français, ou bien des cas dans lesquels un des corrélats dans la phrase source correspond à un élément «~faible~», p.ex. un \is{affixe/clitique adverbial}clitique adverbial \REF{ch2:ex136a} ou \is{affixe/clitique pronominal}pronominal \REF{ch2:ex136b}.

\ea \label{ch2:ex134}
\ea 
\gll Lunea  \uline{merg}  la  film,  iar  \textbf{sora} \textbf{mea}  la  muzeu.\\
lundi.\textsc{def}  aller.\textsc{prs.1sg} à  film  et  sœur.\textsc{def} \textsc{poss.1sg}  à  musée\\ 
\glt  ‘Le lundi, je vais au cinéma, et ma sœur au musée.’  

\ex  
\gll \ulg{Nu}{60.6}  l-am  înțeles  niciodată  pe  Ion  și  nici  \textbf{el}  pe  mine.\\
\textsc{neg} \textsc{acc.3sg.m-aux.1sg}  compris  jamais  \textsc{dom}  Ion  et  ni  lui  \textsc{dom}  moi\\ 
\glt ‘Je n’ai jamais compris Ion, et il ne m’a jamais compris non plus.’

\ex  
\gll O  vorbă  n-ai  scos,  {de parcă}  \ulg{ai}{15}  fi  văduv,  iar  \textbf{eu}  menajera  ta.\\
un  mot  \textsc{neg-}as  sorti  {comme si}  \textsc{cond.2sg}  être veuf  et  moi  domestique.\textsc{def} \textsc{poss.2sg}\\
\glt  ‘Tu n’as pas dit un mot, comme si tu étais veuf, et moi, ta domestique.’ %(\url{http://www.scribd.com/doc/19489125/Familie-casnicie1})  

\ex  
\gll \uline{Are}  vreo  35  de  ani,  iar  \textbf{soțul} \textbf{ei}  cu  vreo  10  ani  mai mult\\ 
avoir.\textsc{prs.3sg} environ 35  de  ans  et  mari.\textsc{def}  \textsc{poss.3sg.f}  avec  environ  10  ans  plus beaucoup\\ 
\glt  ‘Elle a environ 35 ans, et son mari 10 ans de plus qu’elle.’ 
\z
\z


\ea \label{ch2:ex135}
\ea Marie \uline{nage} bien, mais Paul \textbf{seulement la brasse}.
\ex  Paul \uline{a mangé ce matin}, mais Marie \textbf{seulement un fruit}.
\ex  Paul \uline{boit} trop, mais son frère \textbf{que de l’eau}.
\z
\z

\ea
\ea 
\gll Dan  \textbf{tot}  \textbf{mai}  \uline{citeşte},  dar  prietena  lui  \textbf{absolut}  \textbf{nimic}. \label{ch2:ex136a}\\
Dan  \textsc{adv} \textsc{adv}  lit  mais  copine.\textsc{def} \textsc{poss.3sg.m}  absolument  rien\\ 
\glt  ‘Dan lit un peu, mais sa copine absolument rien.’  

\ex  
\gll Ion  \textbf{mi}-\uline{e}  prieten,  iar  \textbf{ţie}  duşman. \label{ch2:ex136b}\\
Ion  \textsc{dat.1sg}-est  ami  et  toi.\textsc{dat} ennemi\\ 
\glt  ‘Ion est mon ami, et pour toi, un ennemi.’ 
\z
\z

Le deuxième cas est illustré par les exemples en \REF{ch2:ex137}, où la séquence trouée contient un élément de plus par rapport aux deux \is{paire contrastive}paires contrastives obligatoires, n’ayant pas de corrélat dans la phrase source. Cet élément «~solitaire~» peut être réinterprété\footnote{
 Cette observation reprend la note 4 dans l’article de \citet[186]{AbeilleEtAl2010}.}, en considérant que la deuxième \isi{paire contrastive} (p.ex. <\textit{un ziar}, \textit{o jucărie (pentru fetiţa ei)}> en \REF{ch2:ex137a}) met en jeu deux propriétés (et non deux entités), la deuxième propriété étant obtenue par montée de type du syntagme nominal résiduel \textit{o jucărie} ‘un jouet’ et composition fonctionnelle avec le terme «~solitaire~» \textit{pentru fetiţa ei} ‘pour sa fille’, en suivant l’analyse proposée en \is{grammaire catégorielle}Grammaire Catégorielle par \citet{Steedman2000}. 

\ea \label{ch2:ex137}
\ea {} [Ion] \uline{a cumpărat} [un ziar], iar [Maria] [o jucărie] \textbf{pentru fetiţa ei}. \label{ch2:ex137a}
\glt  ‘Ion a acheté un journal, et Maria un jouet pour sa fille.’  

\ex {} [Dan] \uline{merge} [la munte], iar [Maria] \textbf{probabil} [la mare]. \label{ch2:ex137b} 
\glt  ‘Dan va à la montagne, et Maria probablement à la mer.’ 
\z
\z

Enfin, comme le note \citet{SagEtAl1985} pour l’anglais \REF{ch2:ex138a}, on observe que \is{ordre de mots}l’ordre des éléments résiduels n’est pas nécessairement le même que l’ordre des corrélats dans la phrase source, à condition que l’ordre en question soit possible par ailleurs dans la grammaire (\textit{contra} \citealt{Johnson2014}, qui postule un ordre rigide dans les constructions à gapping en anglais \REF{ch2:ex138b}). Ces \is{asymétrie syntaxique}asymétries d’ordre sont beaucoup plus visibles en roumain \REF{ch2:ex139}. Cela s’explique par les faits suivants : (i) \is{ordre de mots}l’ordre des mots relativement libre en roumain, (ii) la \isi{saillance prosodique} ({{\cad}} \isi{focus prosodique}, marqué dans les exemples suivants par des majuscules), et (iii) la présence d’une conjonction, notamment la conjonction \textit{iar} ‘et’, qui impose des contraintes discursives facilitant l’organisation des \is{paire contrastive}paires contrastives. Le manque de parallélisme en ce qui concerne l’ordre des éléments résiduels et corrélats semble être moins acceptable avec la \isi{juxtaposition} \REF{ch2:ex140b}.  

\ea
\ea A policeman \uline{walked in} at 11, and at 12, a fireman. \citep{SagEtAl1985} \label{ch2:ex138a} 
\ex ??\uline{I bought} a car on Monday, and on Tuesday, a motorcycle. \citep[25]{Johnson2014} \label{ch2:ex138b}
\z
\z

\ea \label{ch2:ex139}
\ea 
\gll Dimineaţa  (EU)  \uline{spăl}  (EU)  \uline{vesela}  (EU),  iar  seara  IOAna.\\
matin.\textsc{def}  (je)  lave.\textsc{1sg}  (je)  vaisselle.\textsc{def}  (je)  et soir.\textsc{def}  Ioana\\ 
\glt ‘Le matin c’est moi qui fais la vaisselle, et le soir c’est Ioana.’  

\ex 
\gll EU  \ulg{spăl}{16.3}  vesela  dimineaţa,  iar  seara  IOAna.\\
je  lave.\textsc{1sg}  vaisselle.\textsc{def}  matin.\textsc{def}  et  soir.\textsc{def} Ioana\\
\glt ‘C’est moi qui fais la vaisselle le matin, et le soir c’est Ioana.’

\ex  
\gll Eu  \ulg{spăl}{16.3}  vesela  dimiNEAţa,  iar  Ioana  SEAra.\\
je  lave.\textsc{1sg}  vaisselle.\textsc{def}  matin.\textsc{def}  et  Ioana  soir.\textsc{def}\\
\glt ‘Je fais la vaisselle le matin, et Ioana le soir.’ 

\ex 
\gll DimiNEAţa  \uline{spăl}  eu  \uline{vesela},  iar  Ioana  SEAra.\\
matin.\textsc{def}  lave.\textsc{1sg}  je  vaisselle.\textsc{def}  et  Ioana  soir.\textsc{def}\\
\glt ‘C’est le matin que je fais la vaisselle, et quant à Ioana, elle fait la vaisselle le soir.’ 
\z
\z

\ea \label{ch2:ex140}
\ea 
\gll Fiul  lor  \uline{studiază}  dreptul,  fiica  mea  medicina. \label{ch2:ex140a}\\
fils.\textsc{def} \textsc{poss.3pl} étudie  droit.\textsc{def}  fille.\textsc{def}  \textsc{poss.1sg}  médecine.\textsc{def}\\ 
\glt ‘Leur fils étudie le droit, ma fille la médecine.’  

\ex  
\gll Fiul  lor  \uline{studiază}  dreptul,  *(\textbf{iar})  medicina,  fiica  mea. \label{ch2:ex140b}\\
fils.\textsc{def} \textsc{poss.3pl} étudie  droit\textsc{.def}  et  médecine\textsc{.def}  fille.\textsc{def} \textsc{poss.1sg}\\
\glt ‘Leur fils étudie le droit, et la médecine, c’est ma fille.’ 
\z
\z

Sur la base de ces observations, on doit conclure que le parallélisme syntaxique n’opère pas au niveau de la catégorie ou encore au niveau de l’ordre des mots (\textit{contra} \citealt{Hartmann2000,CulicoverEtAl2005,Culicover2009}), mais plutôt au niveau de la structure argumentale du prédicat antécédent. Ce parallélisme syntaxique «~relâché~» exige simplement que les éléments résiduels remplissent les conditions de sélection du prédicat antécédent dans la phrase source. On observe ainsi que les éléments résiduels et leurs corrélats dans les constructions à gapping obéissent à la même contrainte syntaxique que les conjoints dans les structures coordonnées ordinaires, à savoir la \is{généralisation de Wasow}généralisation dite «~de \ia{Wasow, Thomas}Wasow~» \citep{GazdarEtAl1985,PullumEtAl1986} : une construction coordonnée est syntaxiquement bien formée dans un contexte phrastique si et seulement si chacun des termes coordonnés peut apparaître seul dans ce contexte sans en altérer les propriétés (voir le contraste en \REF{ch2:ex141} et \REF{ch2:ex142} en anglais). Cette généralisation commune aux constructions à gapping et aux structures coordonnées s’applique en égale mesure aux deux langues étudiées dans cet ouvrage. Ainsi, en roumain, dans la coordination à gapping en \REF{ch2:ex143}, le verbe \textit{a cere} ‘demander’ est compatible à la fois avec un syntagme nominal et une phrase au subjonctif, mais moins avec un syntagme verbal à l’infinitif ; il aura le même comportement dans une coordination ordinaire \REF{ch2:ex144}. De même en français, le verbe \textit{réclamer} impose les mêmes propriétés de sélection dans les constructions à gapping \REF{ch2:ex145} et les coordinations ordinaires \REF{ch2:ex146}. 

\ea \label{ch2:ex141}
\ea  Pat \uline{has become} [crazy]\textsubscript{AP} and Chris [an incredible bore]\textsubscript{NP}.
\ex *Pat \uline{has become} [crazy]\textsubscript{AP} but Chris [in good spirit]\textsubscript{NP}.
\ex  He became \{crazy {\textbar} an incredible bore {\textbar} *in good spirit\}. \citep[156--158]{SagEtAl1985}
\z
\z

\ea \label{ch2:ex142}
\ea He has become [crazy]\textsubscript{AP} and [an incredible bore]\textsubscript{NP}.
\ex *He has become [crazy]\textsubscript{AP} but [in good spirit]\textsubscript{PP}.
\z
\z

\ea \label{ch2:ex143}
\ea La meeting-ul de azi, unii \uline{cereau} [demisia Preşedintelui]\textsubscript{NP}, alţii [să li se mărească salariile]\textsubscript{S}.\textsubscript{} 
\glt  ‘Au meeting d’aujourd’hui, certains demandaient la démission du Président, d’autres (demandaient) qu’on leur augmente les salaires.’

\ex ??La meeting-ul de azi, unii \uline{cereau} [demisia Preşedintelui]\textsubscript{NP}, alţii [a avea salarii mai mari]\textsubscript{VPinf}.
\glt ‘Au meeting d’aujourd’hui, certains demandaient la démission du Président, d’autres (demandaient) d’avoir de meilleurs salaires.’

\ex Manifestanții cer \{demisia Preşedintelui {\textbar} să li se mărească salariile {\textbar} ?a avea salarii mai mari\}.
\glt ‘Les manifestants demandent \{la démission du Président {\textbar} qu’on leur augmente les salaires {\textbar} d’avoir de meilleurs salaires\}.’
\z
\z


\ea \label{ch2:ex144}
\ea Manifestanții cer [demisia Preşedintelui]\textsubscript{NP} şi [să li se mărească salariile]\textsubscript{S}.
\glt ‘Les manifestants demandent la démission du Président et qu’on leur augmente les salaires.’

\ex ??Manifestanții cer [demisia Preşedintelui]\textsubscript{NP} şi [a avea salarii mai mari]\textsubscript{VPinf}.
\glt ‘Les manifestants demandent la démission du Président et d’avoir de meilleurs salaires.’ 
\z
\z


\ea \label{ch2:ex145}
\ea Certains \uline{réclament} [des augmentations]\textsubscript{NP}, d’autres [qu’on leur garantisse la sécurité]\textsubscript{S}.
\ex *Certains \uline{réclament} [des augmentations]\textsubscript{NP}, d’autres [être mieux protégés]\textsubscript{VPinf}.
\ex Ils réclament \{des augmentations {\textbar} qu’on leur garantisse la sécurité {\textbar} *être mieux protégés\}.
\z
\z

\ea \label{ch2:ex146}
\ea Ils réclament [des augmentations]\textsubscript{NP} et [qu’on leur garantisse la sécurité]\textsubscript{S}.
\ex *Ils réclament [des augmentations]\textsubscript{NP} et [être mieux protégés]\textsubscript{VPinf}.
\z
\z


\subsubsection{Sémantique} \label{ch2:sect2.3.4.2}


Au niveau sémantique, chaque élément résiduel doit être mis en correspondance avec un corrélat dans la phrase source. Pour le type d’ellipse qui nous intéresse ici, cette correspondance se traduit par une relation de \is{contraste sémantique}contraste\footnote{La notion de \is{contraste sémantique}\textit{contraste} doit être prise en compte pour d’autres types d’ellipse aussi, p.ex. \is{Bare Argument Ellipsis (BAE)}\textit{Bare Argument Ellipsis} (BAE). Voir les détails dans \citet{KonietzkoEtAl2010}.}. On peut donc dire qu’une coordination à gapping doit contenir au moins deux \is{paire contrastive}paires contrastives (\citealt{Kuno1976,Sag1976,Hartmann2000,FeryEtAl2005,Winkler2005,Repp2009}, etc.)\footnote{
 On utilise parfois le terme de \textit{contrastive foci}, mais on évite le terme de \is{focus}\textit{focus} ici. On l’utilisera uniquement dans son sens strict, d’information nouvelle (voir la section~\ref{ch2:sect2.3.4.4}).}. Ainsi, en \REF{ch2:ex147}, les deux paires mises en jeu sont <\textit{Ioana, Maria}> et <\textit{un măr, o pară}>.

\ea Ioana \uline{a mâncat} un măr şi Maria o pară. \label{ch2:ex147}
\glt ‘Ioana a mangé une pomme et Maria une poire.’   
\z 

Le roumain se distingue des autres \ili{langues romanes} (et se rapproche ainsi des \ili{langues slaves}) par le fait qu’il comporte une conjonction spécialisée pour le \is{contraste sémantique}contraste dans la coordination \citep{BilbiieEtAl2011}. Il s’agit de la conjonction \textit{iar} ‘et’\footnote{
 Voir une analyse détaillée dans \citet[chapitre 2]{Bilbiie2011}.}, qui est la conjonction la plus fréquente dans les coordinations à gapping ; la contrainte majeure imposée par cette conjonction est que les conjoints doivent contenir au moins deux \is{paire contrastive}paires contrastives (\ref{ch2:ex148a}--\ref{ch2:ex148b}), la conjonction \textit{iar} étant donc exclue dans les contextes à une seule \is{paire contrastive}paire contrastive \REF{ch2:ex148c}. Dans la plupart des cas, les corrélats dans la phrase source sont lexicalisés. Cependant, il y a certains cas où le corrélat dans une \isi{paire contrastive} est implicite, p.ex. le \isi{pro-drop} en \REF{ch2:ex134} ou bien les exemples mentionnés en-dessous de la note \ref{ch2:fn43}. 

\ea
\ea Ioana \uline{a mâncat} un măr, iar *(Maria) o pară. \label{ch2:ex148a}
\glt ‘Ioana a mangé une pomme, et Maria une poire.’  

\ex Ion \uline{i-a dat} o carte Mariei, iar Dan un stilou Ioanei. \label{ch2:ex148b} 
\glt ‘Ion a offert un livre à Maria, et Dan un stylo à Ioana.’

\ex Ioana a mâncat un măr \{şi {\textbar} *iar\} o pară. \label{ch2:ex148c} 
\glt ‘Ioana a mangé une pomme et une poire.’   
\z
\z

Une \isi{paire contrastive} se définit essentiellement par deux aspects concomitants : (i) l’appartenance à un même \isi{ensemble d'alternatives}, et (ii) la présence d’une opposition sémantique entre ses éléments. Ainsi, le \is{contraste sémantique}contraste suppose à la fois une relation de ressemblance et dissemblance \citep{Sag1976,Rooth1992,VallduviEtAl1998}. La même idée apparaît chez \citet{Umbach2005}, qui définit le parallélisme sémantique en utilisant la terminologie de \citet{Lang1984} : la ressemblance des éléments d’une paire s’établit grâce à la présence implicite d’un intégrateur commun, {\cad} un concept qui subsume les deux conjoints ; tandis que la dissemblance est garantie par la condition d’indépendance sémantique ou la propriété d’être distincts, {\cad} les éléments en question ne doivent pas se subsumer (voir aussi \citealt{Zeevat2004}). C’est ce qui explique les différences d’acceptabilité pour les exemples en \REF{ch2:ex149}. Selon \citet{Umbach2005}, dans l'exemple \REF{ch2:ex149a} on viole le principe de l’indépendance sémantique, car \textit{a drink} subsume \textit{a martini}. En revanche, dans l'exemple \REF{ch2:ex149b}, la condition d’avoir un intégrateur commun détermine le choix d’un certain sens (compatible avec \textit{the beer}) pour le syntagme \textit{the port}.  

\ea \label{ch2:ex149}
\ea \#John \uline{had} a drink, and Mary a martini. \label{ch2:ex149a}
\ex John \uline{bought} the beer, and Mary the port. (exemple adapté d'après \citealt{Umbach2005}) \label{ch2:ex149b}
\z
\z

Dans cet ouvrage, j’utiliserai plutôt les notions d’\isi{ensemble d'alternatives} et opposition sémantique. Dans une coordination à gapping, on a donc pour chaque paire un ensemble restreint d’alternatives identifiables et explicites. Les alternatives de chaque ensemble doivent avoir un type approprié. Ainsi, appartiennent au même \isi{ensemble d'alternatives} des éléments dénotant différents agents, différentes indications temporelles, différentes indications spatiales, différents objets, etc. L’énoncé devient inacceptable si la paire contient des éléments qui ne font pas partie du même \isi{ensemble d'alternatives} ou qui n’ont pas le même type sémantique \REF{ch2:ex150} : on ne peut pas avoir comme \isi{paire contrastive} un instrument et un thème \REF{ch2:ex150a}, un thème et une indication temporelle \REF{ch2:ex150b}, un agent animé et une cause non animée \REF{ch2:ex150c} ou encore un co-agent et un instrument \REF{ch2:ex150d}\footnote{
Un exemple comme \REF{ch2:ex150d} peut être amélioré si le syntagme prépositionnel \textit{cu Maria} ‘avec Maria’ désigne un moyen de transport possédé par Maria (la voiture de Maria) : 

\ea 
Ion \uline{merge la lucru} cu Maria, iar Dan cu trenul.
\glt ‘Ion va au travail avec Maria, et Dan (y va) en train.’
\z
}.


\ea \label{ch2:ex150}
\ea \#Maria \uline{cântă} \textbf{la pian}, iar Ioana \textbf{arii de Chopin}. \label{ch2:ex150a}
\glt ‘Maria joue du piano, et Ioana (joue) des aires de Chopin

\ex \#Ioana \uline{mănâncă} \textbf{mere}, iar Maria \textbf{la miezul nopţii}. \label{ch2:ex150b} 
\glt ‘Ioana mange des pommes, et Maria (mange) à minuit.’

\ex \#\uline{Sunt păzit} \textbf{de Cel-de-Sus}, iar tu \textbf{de orice rău}. \label{ch2:ex150c} 
\glt ‘Je suis protégé par le bon Dieu, et toi (tu es protégé) de tout mal.’

\ex \#Ion \uline{merge la film} \textbf{cu Maria}, iar Dan \textbf{cu maşina}. \label{ch2:ex150d} 
\glt ‘Ion va au cinéma avec Maria, et Dan (va au cinéma) en voiture.’
\z
\z

Selon la première contrainte pesant sur le \is{contraste sémantique}contraste ({\cad} la ressemblance des éléments mis en contraste, cf. \citealt{Umbach2005}), les éléments doivent appartenir au même \isi{ensemble d'alternatives} (ayant en commun un type sémantique et une archi-propriété), mais ils ne doivent pas se subsumer. Ce qui explique l’inacceptabilité de l’exemple \REF{ch2:ex151a}, où l’on a comme deuxième \isi{paire contrastive} <\textit{un măr, un fruct}> définie comme <hyponyme, hyperonyme>. Mais comment expliquer l’acceptabilité de l’exemple \REF{ch2:ex151b}, qui semblent contredire la contrainte qu’on vient de préciser ? Selon \citet{BilbiieEtAl2011}, dans la paire <\textit{toate, câteva}> la subsomption est fausse si l’on établit la liste exhaustive des membres associés à \textit{câteva} ‘quelques-unes’, {\cad} si on interprète \textit{câteva} ‘quelques-unes’ comme \textit{quelques-unes, mais pas toutes}. Or, cette lecture exhaustive est fortement préférée par les locuteurs dans cet exemple. En revanche, on ne peut pas exhaustifier un hyperonyme, p.ex. \textit{fruit} = \textit{fruit, mais pas pomme}. Ainsi, il est impossible de dériver une inférence scalaire à partir de l’assertion d’un hyperonyme (\textit{Maria a mangé un fruit} n’implique pas \textit{Maria n’a pas mangé une pomme}), ce qui n’est pas le cas avec les implicatures quantitatives en \REF{ch2:ex151b} (\textit{Maria a répondu à quelques questions} implique \textit{Maria n’a pas répondu à toutes les questions}).

\ea
\ea \#Ioana \uline{a mâncat} un \textbf{măr}, iar Maria un \textbf{fruct}. \label{ch2:ex151a}
\glt ‘Ioana a mangé une pomme, et Maria un fruit.’  

\ex Ioana \uline{a răspuns} la \textbf{toate} întrebările, iar Maria la \textbf{câteva}. \label{ch2:ex151b} 
\glt ‘Ioana a répondu à toutes les questions, et Maria à quelques-unes.’
\z
\z

Selon la deuxième contrainte pesant sur le \is{contraste sémantique}contraste ({\cad} la dissemblance, cf. \citealt{Umbach2005}), il faut qu’il y ait une distinction entre les éléments mis en contraste. Tournons-nous à présent vers d’autres cas complexes, qui nécessitent des discussions sur l’identité lexématique (appelée «~identité de surface~» dans \citealt{HinterwimmerEtAl2008}), ainsi que sur l’éventuelle co-indiciation des éléments qui forment une \isi{paire contrastive}. La deuxième contrainte d’une \isi{paire contrastive} exige que les éléments soient distincts, d’où résulterait le fait trivial selon lequel une \isi{paire contrastive} ne peut pas contenir des éléments qui présentent une identité lexématique \REF{ch2:ex152}. 


\ea \label{ch2:ex152}
\ea \#\textbf{Ioana} \uline{plăteşte} chiria şi \textbf{Ioana} impozitele.
\glt ‘Ioana paie le loyer et Ioana les impôts.’  

\ex \#Ion \uline{vorbeşte} \textbf{cu Maria} şi Dan \textbf{cu Maria}. 
\glt ‘Ion parle avec Maria et Dan avec Maria.’

\ex \#Ion \uline{a cumpărat} (nişte) \textbf{flori} şi Maria (nişte) \textbf{flori}. 
\glt ‘Ion a acheté des fleurs et Maria des fleurs.’   
\z
\z

Cependant, il y a au moins trois types de contextes qui contredisent cette contrainte liée à la non-identité lexématique : les pronoms à interprétation déictique \REF{ch2:ex153}, les éléments interrogatifs \REF{ch2:ex154} ou encore les numéraux\footnote{Voir \citet{HinterwimmerEtAl2008} et \citet{Repp2009} pour plus de détails sur ce point en anglais. Ils observent que seuls certains types de syntagmes quantifiés ({\cad} ce qu’ils appellent les \textit{indéfinis} \textit{spécifiques} ou \textit{indéfinis} \textit{topiques}) permettent l’identité «~de surface~» en anglais (voir le contraste entre \REF{ch2:foot49ia} et \REF{ch2:foot49ib} ci-dessous) et cela, uniquement en position préverbale (voir le contraste entre \REF{ch2:foot49ia} et \REF{ch2:foot49ic}). De plus, ils observent des différences entre les numéraux simples comme \textit{three} en \REF{ch2:foot49iia} et les numéraux dans des quantifieurs plus complexes comme \textit{less than three}, les derniers demandant a priori une identité «~de surface~» en anglais (comparer \REF{ch2:foot49iib} et \REF{ch2:foot49iic}).
 
\ea
\ea One student \uline{called} the director and one student the dean. \label{ch2:foot49ia} 
\ex *A student \uline{called} the director and a student the dean. \label{ch2:foot49ib}
\ex *The director \uline{called} one student and the dean one student. \citep[9]{Repp2009} \label{ch2:foot49ic} 
\z\z

\ea
\ea Three children \uline{chose} the book and three (children) the CD. \label{ch2:foot49iia} 
\ex *Less than three children \uline{chose} the book and less than three (children) the CD. \label{ch2:foot49iib}
\ex Less than three children \uline{chose} the book and less than four (children) the CD. \citep[244]{HinterwimmerEtAl2008} \label{ch2:foot49iic} 
\z\z
} \REF{ch2:ex155}. Fondamentalement, on observe que les éléments «~identiques~» dans ces paires ne renvoient pas au même référent.


\ea \label{ch2:ex153}
\ea Uite cele două rochii pe care le-am cumpărat ieri : \textbf{pe-asta} \uline{am cumpărat-o} pentru cununia civilă, iar \textbf{pe-asta} pentru cununia religioasă.~ 
\glt ‘Voici les deux robes que j’ai achetées hier : celle-ci je~l’ai achetée pour la cérémonie civile, et celle-là pour la cérémonie religieuse.’   

\ex Priveşte-i pe cei doi colegi ai mei pe scenă : \textbf{el} [\textit{arătând cu degetul spre dreapta}] \uline{a făcut} medicina, iar \textbf{el} [\textit{arătând cu degetul spre stânga}] dreptul.
\glt ‘Regarde mes deux collègues sur l’estrade : lui [\textit{en pointant du doigt vers la droite}] a fait des études de médecine, et lui [\textit{en pointant du doigt vers la gauche}] des études de droit.’ 
\z
\z

\ea \label{ch2:ex154}
\ea \textbf{Cine} \uline{vine} azi şi \textbf{cine} mâine ?
\glt ‘Qui vient aujourd’hui et qui demain ?’  

\ex \textbf{De când} \uline{te-ai sculat} tu şi \textbf{de când} eu ?  [\textit{question de reproche}]
\glt ‘A quelle heure tu t’es réveillé et à quelle heure moi ?’
\z
\z

\ea \label{ch2:ex155}
\ea Dintre cei şase copii selecţionaţi, \textbf{trei} \uline{vin} azi şi \textbf{trei} mâine.
\glt ‘Parmi les six enfants sélectionnés, trois viennent aujourd’hui et trois demain.’ 

\ex Dintre cele patru mere rămase, Ioana \uline{a luat} \textbf{două} şi Maria \textbf{două}.
\glt ‘Parmi les quatre pommes qui restaient, Ioana en a pris deux, et Maria deux.’
\z
\z

Ainsi, \citet{Hartmann2000} et \citet{Repp2009} vont plus loin et postulent la condition d’un contraste référentiel entre un élément résiduel et son corrélat. Par conséquent, une \isi{paire contrastive} ne peut comporter des éléments qui renvoient au même référent. On explique ainsi l’inacceptabilité des exemples en \REF{ch2:ex156}, dans lesquels la première paire contrastive contient des éléments qui sont co-indicés\footnote{
 On note certains exemples marginaux comme \REF{ch2:foot50i} dans lesquels une \isi{paire contrastive} contient des syntagmes nominaux renvoyant au même référent, mais avec des interprétations différentes (intensionnelle vs. extensionnelle). 
 \ea {} [Le président de la République]\textsubscript{i} est agnostique, mais [l’homme Sarkozy]\textsubscript{i} catholique. \label{ch2:foot50i}
 \z
}.

\ea \label{ch2:ex156}
\ea \#Maria\textsubscript{i} \uline{participă} la concursul de fotografie şi Maria\textsubscript{i} la festivalul de muzică.
\glt ‘Maria participe au concours de photographie et Maria au festival de musique.’  

\ex \#Maria\textsubscript{i} \uline{participă} la concursul de fotografie şi [proasta asta]\textsubscript{i} la festivalul de muzică. 
\glt ‘Maria participe au concours de photographie et cette cruche au festival de musique.’
\z
\z

Néanmoins, les exemples inacceptables avec identité lexématique et co-indicia\-tion en \REF{ch2:ex152} peuvent être améliorés (au moins pour la \isi{paire contrastive} ne contenant pas le résiduel en première position dans la phrase trouée)\footnote{
Il reste à expliquer pourquoi \REF{ch2:ex157a} et \REF{ch2:ex158b}, {\cad} les exemples dans lesquels la \isi{paire contrastive} établie par identité lexématique et co-indiciation contient des éléments apparaissant en première position dans la phrase (ici, des sujets), ne peuvent pas être améliorés par la présence d’un \isi{adverbe associatif} comme \textit{tot} ‘aussi’ en roumain ou \textit{aussi} en français. Une explication possible serait liée aux différences d’association qu’engendre la position de l’adverbe. Si l’adverbe \textit{aussi} est en position préverbale, on a nécessairement une association étroite de l’adverbe, qui met en parallèle uniquement l’associé de \textit{aussi} dans la séquence trouée et son corrélat dans la phrase source. En revanche, en position finale, l’adverbe additif peut avoir une association large, donc il peut avoir comme associé toute la séquence trouée dans les constructions à gapping. Dans ce dernier cas, on obtient un \is{contraste sémantique}contraste plus large entre les événements pris dans leur totalité, ce qui fournit (en plus du parallélisme) l’opposition sémantique dont on a besoin dans une construction à gapping. Une autre explication serait liée au statut informationnel de l’associé de \textit{aussi}. Il a été remarqué que l’associé de \textit{aussi} est prosodiquement distingué, {\cad} il est un \isi{focus prosodique}, et donc, au niveau discursif, un \isi{focus informationnel} \citep{Jackendoff1972}. Or, le prototype d’une séquence trouée dans une construction à gapping est une séquence contenant un \isi{topique contrastif} (en première position) et un \isi{focus contrastif} (en deuxième position), cf. \citet{Winkler2005}.} si l’on emploie des adverbes additifs comme \textit{tot} ‘aussi’ en roumain \REF{ch2:ex157} ou \textit{aussi} en français \REF{ch2:ex158}. \citet{KonietzkoEtAl2010} remarquent eux aussi que, dans certains cas, le dispositif permettant le contraste dans les ellipses «~contrastives~» est l’emploi de ce qu’ils appellent une particule discursive (p.ex. \textit{aussi}).

\largerpage
\ea \label{ch2:ex157}
\ea ?\textbf{Ioana} \uline{plăteşte} chiria şi \textbf{tot Ioana} impozitele ; prietenul ei nu plăteşte niciodată nimic. \label{ch2:ex157a}
\glt ‘Ioana paie le loyer et toujours Ioana les impôts ; son ami ne paie jamais rien.’  

\ex Ion \uline{vorbeşte} \textbf{cu Maria}, iar Dan \textbf{tot} \textbf{cu Maria}. \label{ch2:ex157b} 
\glt ‘Ion parle avec Maria et Dan avec Maria aussi.’

\ex Ion \uline{a cumpărat} (nişte) \textbf{flori} şi Maria \textbf{tot} (nişte) \textbf{flori}. \label{ch2:ex157c}
\glt ‘Ion a acheté des fleurs et Maria des fleurs aussi.’  
\z
\z

\ea \label{ch2:ex158}
\ea \#\textbf{Marie} \uline{paie} le loyer et \textbf{Marie} les impôts. \label{ch2:ex158a}
\ex ??\textbf{Marie} \uline{paie} le loyer et puis \textbf{Marie aussi} les impôts ; son ami ne paie jamais rien. \label{ch2:ex158b}
\ex \#Jean \uline{est arrivé} \textbf{aujourd’hui} et Marie \textbf{aujourd’hui}.
\ex Jean \uline{est arrivé} \textbf{aujourd’hui} et Marie \textbf{aujourd’hui} \textbf{aussi}.
\ex \#Jean \uline{parle} \textbf{avec Marie} et Pierre \textbf{avec Marie}.
\ex Jean \uline{parle} \textbf{avec Marie} et Pierre \textbf{avec Marie} \textbf{aussi}.
\z
\z

Dans ces cas, le lien entre la relation de \is{contraste sémantique}contraste (exigée dans une \isi{paire contrastive}) et la relation de parallélisme (caractérisant l’emploi de \textit{aussi}) est compliqué à établir. D’une part, l’absence du marqueur de parallélisme rend la phrase dégradée, car \textit{aussi} est obligatoire (cf. \citealt{Saebo2004,AmsiliEtAl2010}, etc.). Le caractère obligatoire de \textit{aussi} est demandé par la relation de similarité qui doit exister entre l’associé de \textit{aussi} et un élément appartenant à \is{ensemble d'alternatives}l’ensemble d’alternatives de l’associé. D’autre part, pour obtenir le double \is{contraste sémantique}contraste dont on a besoin dans une construction à gapping, on est obligé de postuler que, en dehors du contraste local qui s’établit entre deux éléments formant une \isi{paire contrastive}, on a parfois un contraste plus large entre les événements pris dans leur totalité, grâce à la présence d’un adverbial comme \textit{aussi} dont l’associé est la séquence trouée dans son ensemble. Un argument qui pourrait être donné en faveur de cette hypothèse est la préférence pour le positionnement de l’adverbe additif sur le deuxième élément résiduel : en français, \textit{aussi} en position postverbale peut avoir une association large sur toute la phrase (pour plus de détails sur le fonctionnement de \textit{aussi} en français, voir \citealt{Winterstein2010}).

Parfois, l’opposition sémantique à l’intérieur d’une \isi{paire contrastive} est lexicalisée : l’élément résiduel peut être une expression désignant par elle-même l’idée de \is{contraste sémantique}contraste, comme le syntagme nominal \textit{contrariul} ‘le contraire’ en \REF{ch2:ex159}. Ainsi, la deuxième paire contrastive dans ces exemples est construite du syntagme nominal \textit{contrariul} ‘le contraire’ et d’un autre syntagme, qui peut être un syntagme nominal comme \textit{ceva} ‘quelque chose’ \REF{ch2:ex159a} ou \textit{singurătatea} ‘la solitude’ \REF{ch2:ex159b} ou bien une phrase comme \textit{că pilula scade riscul de cancer} ‘que la pilule réduit le risque de cancer’ \REF{ch2:ex159c}.

\ea \label{ch2:ex159}
\ea Eşti bombardat zilnic cu tot felul de informaţii, unii \uline{susţin} \textbf{ceva}, alţii \textbf{contrariul}, chiar nu mai ştii ce să mai crezi. \label{ch2:ex159a}
\glt ‘On est bombardé chaque jour avec toute sorte d’informations, certains soutiennent quelque chose, d’autres le contraire, on ne sait plus quoi croire.’  

\ex Sunt oameni \uline{care preferă} \textbf{singurătatea}, iar alţii, \textbf{contrariul}.\label{ch2:ex159b} 
\glt ‘Il y a des gens qui préfèrent la solitude, et d’autres, le contraire.’

\ex Unii \uline{spun} \textbf{că pilula scade riscul de cancer}, iar alţii \textbf{contrariul}. \label{ch2:ex159c}
\glt ‘Certains disent que la pilule réduit le risque de cancer, et d’autres le contraire.’  
\z
\z

 
Les deux facettes du \is{contraste sémantique}contraste dans les paires rendent aussi compte de l’inacceptabilité des exemples en \REF{ch2:ex160}\footnote{
 Ces exemples sont acceptés uniquement s’ils sont interprétés comme des occurrences de \is{zeugme}zeugmes sémantiques, ayant une lecture ironique.}, où la deuxième paire est construite des morceaux \is{expression idiomatique}d’expressions idiomatiques qui ne peuvent pas réaliser un contraste approprié, bien que le matériel manquant ait la même forme que le matériel antécédent.

\ea \label{ch2:ex160}
\ea 
\gll \#Mie  \ulg{îmi}{14.1}  arde  \textbf{sufletul}  de  durere,  iar  ţie  \textbf{călcâiele}  să mergi  la  discotecă.\\  
moi.\textsc{dat} \textsc{dat.1sg} brûle  âme.\textsc{def}  de  douleur  et  toi.\textsc{dat}  talons.\textsc{def} \textsc{sbjv} aller.\textsc{sbjv.2sg}  à  discothèque\\
\glt ‘Mon âme brûle de douleur, et tes talons d’impatience pour aller à la discothèque.’

\ex 
\gll \#De  când  stă  beat  prin  şanţuri,  nevastă-sa \uline{îşi} \uline{duce} \textbf{crucea} fără  să  crâcnească,  iar  el \textbf{zilele} de  azi  pe  mâine.\\ 
depuis  quand  reste.\textsc{3sg}  ivre  dans  fossés  femme-\textsc{poss.3sg} \textsc{refl.3sg}  porte croix.\textsc{def}  sans  \textsc{sbjv}  broncher.\textsc{sbjv.3}  et  lui  jours.\textsc{def}  de  aujourd’hui  à  demain\\
\glt ‘Depuis qu’il est toujours ivre au bord de la route, sa femme porte sa croix sans broncher, et lui vit au jour le jour.’ 

\ex \#După întrevederea de ieri cu Ion, eu \uline{am ajuns cu el} \textbf{la o înţelegere}, iar Maria \textbf{la cuţite}. 
\glt ‘Après la rencontre d’hier avec Ion, je suis arrivé avec lui à un accord, et Maria à couteaux tirés.’

\ex \#Faţă de incidentul produs în firmă, cei mai mulţi \uline{păstrează} \textbf{tăcerea}, iar alţii \textbf{amintiri de neuitat}. 
\glt ‘Face à l’incident survenu dans l’entreprise, la plupart garde le silence, et d’autres des souvenirs inoubliables.’

\ex 
\gll \#Eu \uline{pun} \textbf{o} \textbf{vorbă} \textbf{bună} pentru  el,  iar  el  (în  schimb) \textbf{paie} \textbf{pe} \textbf{foc}.\\ 
je  mets  un  mot  bon  pour  lui  et  lui  (en  revanche)  paille  sur  feu\\
\glt ‘J’interviens en sa faveur, et lui (en revanche) attise la querelle.’   
\z
\z

Pour conclure, les coordinations à gapping mettent en jeu un parallélisme sémantique~fort, {\cad} il doit y avoir au moins deux \is{contraste sémantique}contrastes sémantiques entre les éléments résiduels et les corrélats. Les \is{paire contrastive}paires contrastives exploitent chacune un \isi{ensemble d'alternatives} qui fournit les éléments qui vont être mis en contraste. 


\subsubsection{Relations discursives} \label{ch2:sect2.3.4.3}


On considère généralement que les \is{relation discursive}relations discursives qui s’établissent entre les phrases liées par coordination appartiennent à l’un des deux grands types suivants : relations symétriques vs. relations asymétriques\footnote{
 \citet{Asher1993} et \citet{AsherEtAl2003} utilisent plutôt la distinction \textit{relations de coordination} vs. \textit{relations de subordination}. Ces termes peuvent créer une confusion avec les termes utilisés habituellement pour les deux types de phrases liées. Par conséquent, j’éviterai ces termes dans cet ouvrage.} \citep{Asher1993,AsherEtAl2003,Kehler1996,Kehler2000,Kehler2002}. Dans les phrases entretenant une \isi{relation discursive} symétrique, les événements coordonnés sont indépendants l’un par rapport à l’autre, ce qui explique la possibilité d’inverser l’ordre des conjoints sans changer les conditions de vérité de la phrase. En revanche, une \isi{relation discursive} asymétrique place les conjoints dans une relation hiérarchique en quelque sorte, dans le sens où le deuxième événement dépend du premier, p.ex. dans une relation de type cause-effet. Par conséquent, tout changement dans l’ordre des conjoints entraîne des différences d’interprétation.  

\citet{LevinEtAl1986} sont les premiers à observer une différence discursive entre les coordinations standard et les coordinations à gapping. Si une coordination simple comme celle en \REF{ch2:ex161a} est compatible avec les deux types de relations, le gapping en \REF{ch2:ex161b} impose une lecture symétrique. On ne peut donc avoir l’interprétation selon laquelle Nan devient furieux à cause du fait que Sue était fâchée. 

\ea
\ea Sue became upset and Nan became downright angry. \label{ch2:ex161a}
\ex Sue \uline{became} upset and Nan downright angry. \citep[83]{Kehler2002} \label{ch2:ex161b}  
\z
\z

\citet{Kehler2002} reprend l’observation de \citet{LevinEtAl1986} et l’applique non seulement à la conjonction \textit{and}, mais aussi aux conjonctions \textit{or} et \textit{but}. A l’instar de \citet{Kehler2002}, on peut dire que c’est le facteur discursif qui explique l’impossibilité du gapping avec un marqueur de subordination comme \textit{because, even though, despite the fact that, although}, etc. (voir les exemples mentionnés auparavant en \REF{ch2:ex34} pour l’anglais, \REF{ch2:ex35} pour le roumain et \REF{ch2:ex36} pour le français) ou encore avec une conjonction comme fr. \textit{or} ou \textit{car}, et roum. \textit{or}, car tous ces connecteurs indiquent une \isi{relation discursive} asymétrique de type cause-effet. Par conséquent, le gapping sera compatible uniquement avec les marqueurs qui entretiennent des \is{relation discursive}relations discursives symétriques entre les phrases, en particulier avec tout élément lexical qui n’est pas contradictoire avec la notion de \is{contraste sémantique}contraste : ce sont le plus souvent les conjonctions (de coordination), mais aussi certains marqueurs «~hybrides~» comme \textit{în timp ce} ‘alors que’ en roumain ou \textit{alors que / tandis que} en français (cf. les données en (\ref{ch2:ex38}--\ref{ch2:ex39}) discutées dans la section~\ref{ch2:sect2.3.1}), certains connecteurs adverbiaux \REF{ch2:ex56} ou encore certains adverbes additifs (comme \textit{aussi, de même} en français). On explique aussi pourquoi les premiers travaux excluaient le connecteur adversatif \textit{but} de la liste des conjonctions possibles avec le gapping, car ce type de connecteur (en particulier, dans son usage argumentatif) demande un conjoint droit argumentativement plus fort que celui de gauche (voir \citealt{Winterstein2010} pour une analyse du connecteur \textit{mais} en français), par conséquent les conjoints n’ont pas le même type de contribution et entretiennent en quelque sorte une asymétrie discursive. Mais, comme le note \citet{Winterstein2010}, l’adversatif \textit{mais} en français (comme d’ailleurs la conjonction \textit{but} en anglais) a plusieurs emplois, dont un usage contrastif (opposition sémantique, cf. \citealt{Lakoff1971}) en \REF{ch2:ex162a}. Cet usage se distingue des autres emplois de \textit{mais} par le fait que la coordination met en jeu deux \is{paire contrastive}paires contrastives, avec un élément provenant de chacun des conjoints dans chacune des paires (p.ex. <\textit{Lemmy, Ritchie}> et <\textit{basse, guitare}>). De plus, l’interprétation dans ce type d’emploi contrastif est symétrique, ce qui nous permet d’inverser l’ordre des conjoints en \REF{ch2:ex162b} sans modification de sens, contrairement à l’usage argumentatif (déni d’attente, cf. \citealt{Lakoff1971}) en \REF{ch2:ex163a}, dont l’interprétation n’est pas symétrique (\REF{ch2:ex163a} et \REF{ch2:ex163b} ne sont donc pas équivalents). Enfin, dans son emploi contrastif, la conjonction \textit{mais} peut être facilement remplacée par la conjonction \textit{et} en \REF{ch2:ex162c}, avec un changement de sens à peine perceptible ; en revanche, la substitution de \textit{mais} par \textit{et} dans l’usage argumentatif change le sens de l’énoncé global (\REF{ch2:ex163a} et \REF{ch2:ex163c} ont ainsi des interprétations différentes). On observe donc que les contraintes liées à l’usage contrastif de \textit{mais} sont proches de celles du gapping, ce qui explique l’occurrence de la conjonction \textit{mais} dans ce type de constructions elliptiques. 

\largerpage
\ea
\ea Lemmy \uline{joue} de la basse, \textbf{mais} Ritchie de la guitare. \label{ch2:ex162a}
\ex Ritchie \uline{joue} de la guitare, \textbf{mais} Lemmy de la basse. \label{ch2:ex162b} 
\ex Lemmy \uline{joue} de la basse, \textbf{et} Ritchie de la guitare. \citep[42]{Winterstein2010} \label{ch2:ex162c} 
\z
\z

\ea
\ea Lemmy fume, mais il est en bonne santé. \label{ch2:ex163a}
\ex Lemmy est en bonne santé, mais il fume. \label{ch2:ex163b} 
\ex Lemmy fume, et il est en bonne santé. \citep[43]{Winterstein2010} \label{ch2:ex163c} 
\z
\z

Selon \citet{Kehler2000,Kehler2002}, les \is{relation discursive}relations discursives s’organisent en trois types majeurs. D’une part, on a les relations de ressemblance (dont le prototype est la relation de parallélisme, paraphrasée par \textit{and similarly}), qui caractérise toute connexion de deux ou plusieurs séquences dans laquelle on met l’accent sur les similarités ou les contrastes qui s’établissent entre les entités ou les événements en question (p.ex. le parallélisme, le contraste, la généralisation, l’exemplification, l’exception, l’élaboration). D’autre part, on a les relations de type cause-effet (dont la relation canonique est de type résultat, paraphrasée par \textit{and therefore}), dans lesquelles on doit repérer une sorte d’implication entre les propositions dénotées par les énoncés (p.ex. le résultat, l’explication, le déni d’attente, la concession). Enfin, on a les relations de contiguïté (dont le prototype est la relation de narration, paraphrasée par \textit{and then}), qui impliquent le plus souvent une séquence d’événements. 

Parmi les trois types de relations mentionnés ci-dessus, le gapping est très naturel avec les \is{relation discursive}relations discursives de ressemblance (où les événements sont interprétés comme étant indépendants l’un par rapport à l’autre), et en particulier avec les relations de parallélisme (\ref{ch2:ex164a}--\ref{ch2:ex165a}) et contraste (\ref{ch2:ex164b}--\ref{ch2:ex165b}). Pourquoi ces deux relations ? Parce qu’elles explicitent les deux conditions du \isi{contraste sémantique} discutées dans la section~\ref{ch2:sect2.3.4.2} : d’une part, la relation de \is{contraste sémantique}contraste est rendue explicite par la constitution même des \is{paire contrastive}paires contrastives ; d’autre part, le parallélisme est licite avec le gapping, car les éléments de chaque \isi{paire contrastive} doivent être parallèles, {\cad} appartenir au même \isi{ensemble d'alternatives}. On explique ainsi la fréquence massive de la conjonction \textit{iar} dans les constructions à gapping, car cette conjonction est, par définition, compatible uniquement avec les relations de parallélisme et contraste. 

\ea
\ea 
\gll Ion  \ulg{o}{24.6}  iubește  pe  Maria  și  \textbf{și}  Maria  pe  Ion. \label{ch2:ex164a}\\
Ion  \textsc{acc.3sg.f} aime \textsc{dom} Maria  et  aussi  Maria \textsc{dom} Ion\\
\glt ‘Ion aime Maria et Maria aussi Ion.’  

\ex Amândoi soții au fost azi la vot. Ion \uline{a votat} cu Băsescu, \textbf{însă} Maria cu Antonescu. \label{ch2:ex164b} 
\glt ‘Les deux époux ont été aujourd’hui au vote. Ion a voté pour Băsescu, mais par contre Maria pour Antonescu.’   
\z
\z


\ea
\ea Paul \uline{aime} Marie et \textbf{réciproquement} Marie Paul. \label{ch2:ex165a}  
\ex Mes amis ont voté aujourd’hui. Jean \uline{a voté} pour Sarkozy, mais \textbf{par contre} Michel pour Royal. \label{ch2:ex165b}
\z
\z

\largerpage 
Cette contrainte liée au type de \isi{relation discursive} explique aussi pourquoi les coordinations à gapping obéissent toujours à la \isi{Contrainte sur les Structures Coordonnées} (CSC, \citealt{Ross1967}), selon laquelle dans une structure coordonnée \is{extraction}l’extraction ou la cliticisation d’un constituant hors d’un terme conjoint est interdite à moins d’opérer simultanément hors de chacun des conjoints. Si cette contrainte \is{extraction parallèle}d’extraction parallèle a été longtemps considérée comme un test syntaxique pour distinguer la coordination des autres constructions, on sait aujourd’hui que cette contrainte est dépendante du type de \isi{relation discursive} qui s’établit entre les conjoints \citep{Lakoff1986,Kehler2002,KubotaEtAl2008} : les \is{relation discursive}relations discursives symétriques ne permettent qu’une \isi{extraction} parallèle (obéissant ainsi la CSC), tandis que les \is{relation discursive}relations discursives asymétriques sont compatibles avec une \isi{extraction} asymétrique (violant ainsi la CSC). Comme les constructions à gapping se caractérisent par des \is{relation discursive}relations discursives symétriques, \is{extraction}l’extraction d’un constituant doit opérer simultanément hors de chacun des conjoints \REF{ch2:ex166}, une \isi{extraction} asymétrique étant impossible\footnote{
 Contrairement aux constructions à gapping, la construction \is{Bare Argument Ellipsis (BAE)}\textit{Bare Argument Ellipsis} permet \is{extraction}l’extraction asymétrique seulement d’un des conjoints, cf. l’exemple \REF{ch2:foot54i}, ce qui est en argument pour l’analyser comme ajout.
 
\ea \label{ch2:foot54i}
\gll Iată  o  carte  al  cărei  editor  este  necunoscut,  nu  însă  şi  autorul  ei.\\
voici  \textsc{indf.f}  livre.\textsc{f}  \textsc{gen.sg.m} quel.\textsc{gen.sg.f} éditeur  est  inconnu  \textsc{neg}  cependant  aussi  auteur.\textsc{def} \textsc{poss.3sg.f}\\
\glt ‘Voici un livre dont l’éditeur est inconnu, mais pas son auteur.’
\z
}. 

\ea \label{ch2:ex166}
\ea 
\gll Acesta  e  primul  \textbf{film}  în  care  Ion  \uline{joacă}  rolul  principal,  iar Maria  un  rol  secundar  (*în  film).\\ 
celui-ci  est  premier.\textsc{def}  film  dans  lequel  Ion  joue  rôle.\textsc{def}  principal et Maria  un  rôle  secondaire  (dans  film)\\
\glt ‘C’est le premier film où Ion joue le rôle principal, et Maria un rôle secondaire.’  

\ex 
\gll Acesta  e  \textbf{filmul}  pe  care  Ion  \uline{vrea}  să-l  vadă,  iar Maria  să-l  cumpere  (*filmul).\\ 
celui-ci  est  film.\textsc{def}  \textsc{dom}  lequel  Ion  veut  \textsc{sbjv-acc.3sg.m}  voir.\textsc{sbjv.3}  et Maria  \textsc{sbjv-acc.3sg.m}  acheter.\textsc{sbjv.3}  (film.\textsc{def})\\
\glt ‘C’est le film que Ion veut voir, et Maria acheter.’   
\z
\z

A part les relations de parallélisme et contraste, le gapping n’est pas très naturel avec les autres relations de ressemblance, comme l’exemplification, la généralisation, l’exception ou encore l’élaboration, bien qu’elles soient symétriques. Ces relations ne réalisent pas un \is{contraste sémantique}contraste approprié entre les éléments parallèles dans une \isi{paire contrastive} \citep{Kehler2000,Kehler2002}. Ainsi, dans le cas de l’exemplification en \REF{ch2:ex167a} et de la généralisation \REF{ch2:ex167b}, on a des paires dans lesquelles un élément subsume un autre : dans les paires <\textit{plantele medicinale, sunătoarea}>, <\textit{anumite boli, durerile de stomac}>, le deuxième élément de chaque paire est l’hyponyme du premier ; dans les paires <\textit{Cristea, oamenii politici>}, \textit{<pe ţărani, pe cei neinstruiţi}>, le premier élément est une instance de la classe dénotée par le deuxième élément ; or, cela contredit la première condition observée pour le \isi{contraste sémantique}. Les mêmes observations s’appliquent aux exemples français en \REF{ch2:ex168}. 

\ea
\ea ??Plantele medicinale \uline{sunt indicate} în anumite boli şi\textbf{ spre exemplu} sunătoarea în durerile de stomac. \label{ch2:ex167a}
\glt ‘Les plantes médicinales sont recommandées pour certaines maladies et ainsi la verveine est recommandée pour les maux d’estomac.’  

\ex ??Ponta \uline{îi manipulează} pe ţărani şi, \textbf{în general}, oamenii politici pe cei neinstruiţi. \label{ch2:ex167b} 
\glt ‘Cristea manipule les paysans et, en général, les hommes politiques manipulent les gens sans instruction.’   
\z
\z

\ea \label{ch2:ex168}
\ea ??Un président \uline{flatte} son électorat et \textbf{ainsi} Chirac les électeurs de droite.  
\ex ??Chirac \uline{flatte} les électeurs de droite et \textbf{généralement} les hommes politiques leur électorat.            
\z
\z

Bien que ce ne soit pas la relation privilégiée, le gapping peut apparaître avec une relation de contiguïté, {\cad} narration qui présente une séquence d’événements dans une progression temporelle \citep{Hendriks2004}, comme illustré pour le roumain en \REF{ch2:ex169} et pour le français en \REF{ch2:ex170}. 

\ea Eu \uline{am zis} o vorbă, \textbf{apoi} el alta, și uite așa am ajuns la conflict. \label{ch2:ex169}
\glt ‘J’ai dit un mot, ensuite lui un autre, et c’est ainsi qu’on est arrivé au conflit.’
\z

\ea Marie \uline{a composé} le numéro de Paul et \textbf{ensuite} Jean le numéro d’Anne. \label{ch2:ex170}
\z

En revanche, on observe que dans les deux langues étudiées dans cet ouvrage le gapping n’est pas préféré avec les relations de type cause-effet : résultat (\ref{ch2:ex171a}--\ref{ch2:ex172a}), concession (\ref{ch2:ex171b}--\ref{ch2:ex172b}), condition (\ref{ch2:ex171c}--\ref{ch2:ex172c}). Ce type de contrainte discursive sur les constructions à gapping a l’air d’être indépendante de la langue ; elle opère non seulement en roumain et en français, mais aussi en anglais, comme le montre l’exemple \REF{ch2:ex173} extrait de \citet{CulicoverEtAl2005}. Selon le principe du contraste symétrique/équilibré (angl. \textit{principle of balanced contrast}) de \citet{Repp2009}, les conjoints dans une construction à gapping doivent avoir le même type de contribution par rapport à un \isi{topique} discursif. Or, comme le remarque \citet{Hendriks2004}, les relations de type cause-effet, contrairement aux relations de parallélisme et contraste, construisent un \isi{topique} non contrastif. Les conjoints dans une relation cause-effet ne peuvent pas être tous des réponses adéquates à une \isi{question multiple} implicite, comme c’est le cas des relations symétriques (cf. section~\ref{ch2:sect2.3.4.4}).

\ea
\ea Copilul \uline{era} grav bolnav şi (\#\textbf{deci}) părinţii lui extrem de obosiți. \label{ch2:ex171a}
\glt ‘L’enfant était gravement malade, et donc ses parents étaient extrêmement malheureux.’

\ex Alex \uline{a venit} cu mașina şi (\#\textbf{totuşi}) soția lui pe jos. \label{ch2:ex171b} 
\glt ‘Alex est venu en voiture et pourtant sa femme est venue à pied.’ 

\ex Ion \uline{va pleca} la Paris sau (\#\textbf{în caz contrar}) Maria la Roma. \label{ch2:ex171c} 
\glt ‘Ion va partir à Paris ou sinon Maria va partir à Rome.’  
\z
\z


\ea 
\ea Leur fils \uline{était} malade depuis plus de deux mois et (\#\textbf{donc}) ses parents extrêmement fatigués. \label{ch2:ex172a} 
\ex D’habitude, Jean agit de la même façon que sa femme Marie, mais pas aujourd’hui. Il \uline{a voté} pour Sarkozy, mais (\#\textbf{étonnamment}) Marie pour Royal. \label{ch2:ex172b}  
\ex Jean \uline{ira} à Paris ou (\#\textbf{sinon}) Marie à Rome. \label{ch2:ex172c}               
\z
\z


\ea \label{ch2:ex173}
*Robin \uline{saw} Leslie and \{\textbf{so {\textbar} thus {\textbar} consequently {\textbar} therefore}\} Leslie, Robin.
\z

Cette contrainte discursive à l’œuvre dans les constructions à gapping explique pourquoi ce type d’ellipse n’est pas compatible avec les constructions corrélatives comparatives en (\ref{ch2:ex174a}--\ref{ch2:ex175a}), qui ont une interprétation conditionnelle (de type \textit{if...then...}, cf. \citealt{Beck1997}), ni avec les coordinations causales en (\ref{ch2:ex174b}--\ref{ch2:ex175b}), ou encore les contextes de subordination en (\ref{ch2:ex174c}--\ref{ch2:ex175c}), car dans tous ces trois types de structures, les \is{relation discursive}relations discursives sont asymétriques. 

\ea
\ea 
\gll \textbf{Cu} \textbf{cât}  \uline{e}  Ion  mai  liniștit,  \textbf{cu} \textbf{atât}  *(e)  Ana  mai  fericită. \label{ch2:ex174a}\\ 
avec  combien  est  Ion  plus  calme  avec  autant  est  Ana  plus  heureuse\\
\glt ‘Plus Ion est calme, plus Maria est heureuse.’  

\ex Ion \uline{are} costum albastru, \textbf{căci} soția lui *(are) rochia roșie. \label{ch2:ex174b} 
\glt ‘Ion est habillé en costume bleu, car sa femme est habillée en robe rouge.’ 

\ex Maria \uline{cântă} la vioară, \textbf{pentru că} soțul ei *(cântă) la pian. \label{ch2:ex174c} 
\glt  ‘Maria joue du violon, parce que son mari joue du piano.’  
\z
\z


\ea
\ea \textbf{Plus} Marie \uline{lira} des romans et \textbf{plus} Jean *(lira) de BD. \label{ch2:ex175a}
\ex Jean \uline{a mis} un costume, \textbf{car} Marie *(a mis) une jolie robe. \label{ch2:ex175b}  
\ex Jean \uline{voit} une autre fille, \textbf{parce que} sa copine *(voit) son meilleur ami. \label{ch2:ex175c}
\z
\z

Par conséquent, on doit maintenir l’idée d’un parallélisme discursif fort dans les constructions à gapping, qui privilégie les relations symétriques de parallélisme et contraste. 


\subsubsection{Structure informationnelle} \label{ch2:sect2.3.4.4}


On a vu dans la section~\ref{ch2:sect2.3.4.1} que le gapping n’exigeait pas un parallélisme syntaxique strict en ce qui concerne l’ordre des éléments résiduels et corrélats. Mais comment expliquer l’inacceptabilité de l’exemple en \REF{ch2:ex176} ? Le but de cette section est de montrer l’importance de la \isi{structure informationnelle} pour la légitimation d’une coordination à gapping, en particulier le gapping avec la conjonction \textit{iar} ‘et’ en roumain, qui est de loin la conjonction la plus fréquente avec ce type d’ellipse. 

\ea \label{ch2:ex176}
\gll \#Ioana  \uline{mănâncă}  un  măr,  iar  o  pară  Maria.\\
Ioana  mange  une  pomme  et  une  poire  Maria\\
\glt ‘Ioana mange une pomme, et Maria une poire.’    
\z

Le parallélisme sémantique fort, qui exige des \is{paire contrastive}paires contrastives, est corrélé à un parallélisme informationnel \citep{Winkler2005}. Deux aspects sont à discuter ici : (i) le rapprochement avec les \is{question multiple}questions multiples, et (ii) les notions de \is{topique contrastif}topique (contrastif) et \is{focus informationnel}focus (informationnel).

On a proposé de rapprocher les constructions à gapping des couples question-réponse, en particulier de la notion de congruence des réponses \citep{Kuno1976,Kuno1982,Steedman2000,Reich2006,Winkler2005,Hoyt2008,Repp2009,Johnson2014}. La motivation pour une telle approche vient du fait que l’acceptabilité des phrases trouées semble être fortement dépendante du contexte discursif, comme l’avaient suggéré \citet{Kuno1976}, \citet{Prince1986}, \citet{Steedman2000}, etc. Ainsi, les constructions à gapping sont acceptables lorsqu’on présuppose une proposition ouverte (\textit{open proposition}, cf. \citealt{Prince1986}), qui a la forme d’une \isi{question multiple}. Je reprends l’affirmation de \citet[212]{Prince1986} : «~gappings are felicitous just in case they can be taken to instantiate an OP [=open proposition] corresponding to the full conjunct, where the leftmost constituents bear the same sort of anaphoric (set) relation to something in the prior context found in Topicalization and where the rightmost constituents instantiate the variable in the OP.~»

Dans cette perspective, le prototype discursif dans le gapping est une réponse en liste de paires à une \isi{question multiple} implicite \REF{ch2:ex177}. Les conjoints (source et celui troué) dans une construction à gapping mettent en valeur une même \is{Question Under Discussion (QUD)}question en discussion (QUD, {\cad} \textit{Question Under Discussion}), cf. \citet{Reich2006}. Voir dans ce sens la remarque de \citet[248]{Steedman1990} : «~even the most basic gapped sentence, like \textit{Fred ate bread, and Harry, bananas}, is only really felicitous in contexts which support (or can accommodate) the presupposition that the topic under discussion is \textit{Who ate what}.~» \citet{Johnson2014} remarque que certains des exemples qui sont considérés agrammaticaux en anglais (p.ex. les phrases à trois éléments résiduels en \REF{ch2:ex114a}) s’améliorent s’il s’agit d’une réponse à une \isi{question multiple} \REF{ch2:ex114b}.

\ea \label{ch2:ex177}
A : - Who ate what ?\\
B : - Fred \uline{ate} bread, and Harry bananas.
\z

\largerpage[-1]
Cette hypothèse se justifie empiriquement dans les constructions à gapping avec \textit{iar} ‘et’ en roumain \REF{ch2:ex178b} (et avec la conjonction \textit{a} en \ili{russe} \REF{ch2:ex178c}, cf. \citealt{Kazenin2001} et \citealt{JasinskajaEtAl2009}). Cette conjonction spécialisée pour le double \is{contraste sémantique}contraste au niveau de la phrase est la plus fréquente avec le gapping~et relie des phrases qui répondent implicitement à une \isi{question multiple}, cf. \citet{BilbiieEtAl2011} (pour le \ili{russe} \textit{a}, voir \citealt{JasinskajaEtAl2010}). La description synthétique des deux conjonctions figure en \tabref{ch2:tab1}. La spécificité de la conjonction \textit{iar} au sein des conjonctions de coordination en roumain est marquée par la négation des traits ci-dessous. La conjonction \textit{iar} s’oppose ainsi à la conjonction \textit{și} ‘et’ par la négation du trait SINGLE (${\lnot}$SINGLE) qui indique que chaque conjoint doit être une réponse à une \isi{question multiple}, avec au moins deux éléments \textit{qu-}. La conjonction \textit{iar} s’oppose aussi à la conjonction corrective \textit{ci} ‘mais’ (par le trait ${\lnot}$CORRECTION) et à la conjonction argumentative \textit{dar} ‘mais’ (par le trait ${\lnot}$(WHETHER,2\textsuperscript{nd}))\footnote{
 Pour plus de détails sur la formalisation sémantique de ces conjonctions, voir \citet{BilbiieEtAl2011}.}. 

\ea \textbf{Qui} aime \textbf{quoi} ? 
\ea  
\gll Lui  Ion  \ulg{îi}{18.7}  place  fotbalul,  iar  Mariei  baschetul. \label{ch2:ex178b}\\
\textsc{dat}  Ion \textsc{dat.3sg}  plaît  football.\textsc{def}  et  Maria.\textsc{dat}  basketball.\textsc{def}\\
\glt ‘Ion aime le football, et Maria le basketball.’

\ex Oleg \uline{ljubit} futbol \textbf{a} Roma basketbol. \label{ch2:ex178c} 
\glt ‘Oleg aime le football, et Roma le basketball.’ 
\z
\z


\begin{table}
\caption{Conjonctions «~contrastives~» en roumain et russe}
\label{ch2:tab1}

\begin{tabularx}{.8\textwidth}{lX}
\lsptoprule
roumain \textit{iar} & ${\lnot}$SINGLE,${\lnot}$CORRECTION,${\lnot}$(WHETHER,2\textsuperscript{nd})\\
russe \textit{a} & ${\lnot}$SINGLE, ${\lnot}$(WHETHER,2\textsuperscript{nd},WHY)\\
\lspbottomrule
\end{tabularx}

\end{table} 

Si les \is{paire contrastive}paires contrastives fournissent les réponses à une \isi{question multiple} (implicite), est-ce que leur contribution est identique du point de vue informationnel ? En particulier, quel est le statut informationnel des éléments résiduels ? Dans la littérature, on trouve deux analyses possibles pour le statut informationnel des éléments résiduels : (i) tous les éléments résiduels sont des \isi{focus} \citep{Kuno1976,Hartmann2000,Johnson2014}, ou (ii) un des éléments résiduels est un \isi{topique} \citep{Winkler2005,Repp2009,KonietzkoEtAl2010}. Dans toutes ces analyses, le \isi{topique} et/ou le \isi{focus} en question sont contrastifs, {\cad} ils sont associés à des alternatives.

La première analyse ne peut pas tenir pour le roumain (cf. \citealt{BilbiieEtAl2011}) et le \ili{russe} (cf. \citealt{JasinskajaEtAl2009}), au moins pour le gapping avec \textit{iar} (et respectivement \textit{a} en \ili{russe}). Je discute par la suite les éléments qui justifient l’analyse de la phrase trouée introduite par \textit{iar} en roumain comme une séquence topique-focus.

De manière générale, on observe qu’en roumain \is{ordre de mots}l’ordre naturel des éléments suit la structuration du discours. Ainsi, la différence majeure entre \REF{ch2:ex179} et \REF{ch2:ex180} consiste dans l’ordre relatif des personnes (\textit{Ioana, Maria}) et des activités (\textit{cinéma, théâtre}), en fonction du type de question posée. Le placement naturel de l’élément qui résout la question est à la fin du conjoint, alors que l’élément distingué pour répondre à une question ({\cad} celui qui donne une indication sur la manière de résoudre la question, cf. la notion de \textit{sorting key} de \citealt{Kuno1982}) apparaît en position initiale.  

\ea \label{ch2:ex179}
A : Cu cine ieşi la film şi cu cine la teatru ?
\glt A : ‘Avec qui tu sors au cinéma et avec qui au théâtre ?’

\ea B\textsubscript{1} : La film, \uline{ies} cu Ioana, iar la teatru cu Maria. 
\glt ‘Au cinéma je sors avec Ioana, et au théâtre avec Maria.’

\ex B\textsubscript{2} : \#Cu Ioana \uline{ies} la film, iar cu Maria la teatru. 
\glt ‘Avec Ioana je sors au cinéma, et avec Maria au théâtre.’      
\z
\z

\ea \label{ch2:ex180}
A : Unde ieşi cu fetele weekendul ăsta ?
\glt A : ‘Où est-ce que tu sors avec tes filles ce weekend ?’

\ea B\textsubscript{1} : Cu Ioana \uline{ies} la film, iar cu Maria la teatru. 
\glt ‘Avec Ioana je sors au cinéma, et avec Maria au théâtre.’   

\ex B\textsubscript{2} : \#La film \uline{ies} cu Ioana, iar la teatru cu Maria. 
\glt ‘Au cinéma je sors avec Ioana, et au théâtre avec Maria.’   
\z
\z

On peut expliquer ces différences si on fait appel aux notions de \is{topique contrastif}topique con\-trastif et \isi{focus informationnel}, telles que définies par \citet{Buring2003}. Je dois préciser que la notion de \isi{topique contrastif} de \citet{Buring2003} correspond à la notion de «~clé de tri~» (angl. \textit{sorting key}) de \citet{Kuno1982} ; la seule différence entre les deux notions concerne le domaine d’application : \textit{sorting key} est un terme utilisé par \ia{Kuno, Susumu}Kuno exclusivement dans les couples question-réponse, alors que le \isi{topique contrastif} est un terme utilisé de manière plus générale dans le modèle de \ia{Büring, Daniel}Büring. Les phrases présentent deux valeurs sémantiques associées à un \is{topique contrastif}topique (contrastif) et respectivement à un \is{focus informationnel}focus (informationnel). Le \isi{topique contrastif} est inclus dans la question à laquelle répond l’énoncé et, d’une manière générale, est l’élément qui est saillant dans le discours (dans beaucoup de cas, il reprend un élément déjà mentionné dans le discours). En revanche, le \isi{focus informationnel} est l’élément qui répond à la question, indiquant l’information nouvelle. Généralement, celui-ci est marqué par un contour prosodique spécifique. 

A travers la littérature, la plupart des exemples avec gapping ont comme première \isi{paire contrastive} un ensemble de noms propres. Si l’on veut observer le comportement informationnel des éléments résiduels, il s’avère difficile de le faire en se limitant à ce type d’expressions, car ils sont discursivement neutres, {\cad} ils peuvent être utilisés en toutes circonstances, l’utilisation d’un nom propre demandant simplement que les interlocuteurs sachent de qui ils parlent. Par conséquent, j’utilise trois moyens spécifiques, afin de tester le statut informationnel des éléments résiduels introduits par \textit{iar~}: (i) la réalisation emphatique d’un accent lexical ({\cad} \isi{saillance prosodique}), (ii) la variation entre les syntagmes nominaux spécifiés par un déterminant indéfini vs. déterminant défini, et (iii) le comportement des \is{adverbe associatif}adverbes associatifs sensibles au focus. 

L’identification du \isi{focus informationnel} peut être forcée par la présence d’une \isi{saillance prosodique} sur l’élément en question. J’utilise le gras dans les exemples \REF{ch2:ex181} et \REF{ch2:ex182} pour indiquer quelle est la conjonction la plus naturelle et préférée par les locuteurs ; je marque la \isi{saillance prosodique} en utilisant simultanément les majuscules et le gras. Si l’on regarde le premier conjoint, on observe que le \isi{focus informationnel} n’a pas un ordre contraint (il peut être en position finale – et c’est l’ordre habituel –, ou bien en position initiale, et cela uniquement s’il reçoit une \isi{saillance prosodique}). En revanche, si l’on regarde le conjoint introduit par la conjonction \textit{iar}, on observe que, indépendamment de \is{ordre de mots}l’ordre des éléments dans la phrase source, les locuteurs n’aiment pas avoir en position initiale un élément résiduel distingué prosodiquement (cf. les réponses B\textsubscript{2} en \REF{ch2:ex181b} et \REF{ch2:ex182b}). Par conséquent, dans les réponses B\textsubscript{3} en \REF{ch2:ex181c} et \REF{ch2:ex182c}, le parallélisme syntaxique est violé, afin d’éviter un \isi{focus informationnel} en première position dans la phrase trouée. On voit donc que l’ordre dans lequel ils apparaissent ne correspond pas nécessairement à l’ordre de leurs corrélats dans la phrase source (\textit{contra} \citealt{KonietzkoEtAl2010}).  

\ea \label{ch2:ex181}
A : Cu cine ieşi la film şi cu cine la teatru ?
\glt A : ‘Avec qui tu sors au cinéma et avec qui au théâtre ?’

\ea B\textsubscript{1} : La film, \uline{ies} cu I\textbf{OA}na, \{\textbf{iar} {\textbar} şi\} la teatru cu Ma\textbf{RI}a. 
\glt ‘Au cinéma je sors avec Ioana, et au théâtre avec Maria.’

\ex B\textsubscript{2} : Cu I\textbf{OA}na \uline{ies} la film, \{\#iar {\textbar} ??şi\} cu Ma\textbf{RI}a la teatru. \label{ch2:ex181b}
\glt ‘C’est avec Ioana que je sors au cinéma, et c’est avec Maria que je sors au théâtre.’

\ex B\textsubscript{3} : Cu I\textbf{OA}na \uline{ies} la film, \{\textbf{iar} {\textbar} şi\} la teatru cu Ma\textbf{RI}a. \label{ch2:ex181c} 
\glt ‘C’est avec Ioana que je sors au cinéma, et au théâtre, c’est avec Maria.’      
\z
\z

\ea \label{ch2:ex182}
A : Unde ieşi cu fetele weekendul ăsta ?
\glt A : ‘Où est-ce que tu sors avec tes filles ce weekend ?’

\ea B\textsubscript{1} : Cu Ioana \uline{ies} la \textbf{FILM}, \{\textbf{iar} {\textbar} şi\} cu Maria la \textbf{TEA}tru. 
\glt ‘Avec Ioana je sors au cinéma, et avec Maria au théâtre.’

\ex B\textsubscript{2} : La \textbf{FILM} \uline{ies} cu Ioana, \{\#iar {\textbar} ??şi\} la \textbf{TEA}tru cu Maria. \label{ch2:ex182b} 
\glt ‘Au cinéma je sors avec Ioana, et au théâtre avec Maria.’

\ex B\textsubscript{3} : La \textbf{FILM} \uline{ies} cu Ioana, \{\textbf{iar} {\textbar} şi\} cu Maria la \textbf{TEA}tru. \label{ch2:ex182c} 
\glt ‘Au cinéma je sors avec Ioana, et avec Maria au théâtre.’      
\z
\z

Un argument supplémentaire justifiant la partition \is{topique}topique--\is{focus}focus dans la séquence trouée introduite par \textit{iar} vient des différences qu’on observe avec les syntagmes nominaux accompagnés d’un déterminant. Ainsi, on préfère avoir comme premier élément résiduel un syntagme nominal spécifié par un déterminant défini (\textit{stiloul} ‘le stylo’, cf. \REF{ch2:ex183b}) plutôt qu’un syntagme nominal avec un déterminant indéfini (\textit{un stilou} ‘un stylo’, cf. \REF{ch2:ex183a}). Or, on suppose habituellement qu’une \isi{expression référentielle} définie introduit un référent connu et identifiable dans le discours, alors qu’une \isi{expression référentielle} indéfinie doit désigner un référent non préalablement identifié, {\cad} une information nouvelle. Un exemple d’indéfini qui est acceptable comme premier élément résiduel est celui des indéfinis génériques \REF{ch2:ex183c} ; cela n’a rien d’étonnant : \citet{Kuno1972} et \citet{Kuroda1972} ont observé que le syntagme recevant le marqueur \isi{topique} en \ili{japonais} ou \ili{coréen} avait toujours une interprétation définie ou générique ; \citet{Gundel1988} et \citet{GundelEtAl2004} considèrent que les indéfinis sont généralement exclus comme \is{topique}topiques sauf s’ils ont une interprétation générique, le référent étant dans ces cas familier ou au moins identifiable de manière unique, car l’interlocuteur est censé avoir une représentation de la classe ou de l’espèce s’il connaît le sens des mots. 

\ea \label{ch2:ex183}
\ea \#Mariei \uline{i-am oferit} o carte, iar \textbf{un stilou} Ioanei. \label{ch2:ex183a}
\glt ‘A Maria j’ai offert un livre, et à Ioana un stylo.’  

\ex MaRIei \uline{i-am oferit} cartea, iar \textbf{stiloul} IOAnei. \label{ch2:ex183b}
\glt ‘A Maria j’ai offert le livre, et à Ioana le stylo.’   

\ex  O casă \uline{costă} 200.000 de euro, iar \textbf{o maşină} 20.000. \label{ch2:ex183c}
\glt ‘Une maison coûte 200.000 euros, et une voiture 20.000.’      
\z
\z

Enfin, le fait que le premier constituant suivant \textit{iar} doit être un \isi{topique contrastif} explique aussi pourquoi cet élément résiduel ne peut être modifié par des \is{adverbe associatif}adverbes associatifs comme l’adverbe \textit{și} ‘aussi’ en \REF{ch2:ex184a} ou \textit{nici} ‘non plus’ en \REF{ch2:ex184b}, qui s’associent à des éléments portant le \isi{focus informationnel}. Dans ces contextes, on utilise la conjonction \textit{și} ‘et’ au lieu de la conjonction \textit{iar}. 


\ea
\ea 
\gll Anei  \ulg{îi}{17.1}  plac  merele,  \{\textbf{şi} {\textbar} *\textbf{iar}\}  [şi Marei]\textsubscript{F}  perele. \label{ch2:ex184a}\\
Ana.\textsc{dat} \textsc{dat.3sg}  plaisent  pommes.\textsc{def}  \{et {\textbar} et\}  aussi  Mara.\textsc{dat}  poires.\textsc{def}\\
\glt ‘Ana aime les pommes et Mara aussi les poires.’

\ex 
\gll Anei  \ulg{nu-i}{20.1}  plac  merele,  \{\textbf{şi} {\textbar} *\textbf{iar}\}  [nici  Marei]\textsubscript{F}  perele. \label{ch2:ex184b}\\
Ana.\textsc{dat} \textsc{neg-dat.3sg}  plaisent  pommes.\textsc{def} \{et {\textbar} et\}  ni  Mara.\textsc{dat} poires.\textsc{def}\\
\glt ‘Ana n’aime pas les pommes et Mara non plus les poires.’ 
\z
\z

L’analyse des constructions avec \textit{iar} montre ainsi que le premier élément résiduel de la phrase trouée est un \isi{topique contrastif}. Selon \citet{Winkler2005}, un \isi{topique contrastif} présente trois propriétés : (i) il a une intonation montante (mais cette propriété varie à travers les langues), (ii) il occupe une position initiale dans la phrase, et (iii) cf. \citet{Molnar1998}, il exige la présence dans le même conjoint d’un \isi{focus contrastif}. Un \isi{focus contrastif} est différent d’un focus non contrastif (cf. \citealt{Repp2010}) : il appartient à un \isi{ensemble d'alternatives} fermé/restreint, dont les alternatives sont identifiables dans le discours ; ce qu’on dit sur le \is{focus contrastif}focus con\-trastif ne peut pas s’appliquer à un autre élément du même ensemble, p.ex. à son corrélat.

On arrive ainsi à distinguer entre le gapping \REF{ch2:ex185a}, dont la phrase trouée est une séquence \isi{topique contrastif} -- \isi{focus contrastif}, et les \is{Bare Argument Ellipsis (BAE)}BAE (\ref{ch2:ex185b}--\ref{ch2:ex185c}), dont la séquence elliptique contient simplement un \isi{focus contrastif}\footnote{
 A priori, les cas de \is{Stripping}stripping avec un adverbe propositionnel seraient différents des \is{Bare Argument Ellipsis (BAE)}BAE. Voir les données de \citet{KonietzkoEtAl2010}, où le premier élément précédant l’adverbe propositionnel est interprété comme un \isi{topique contrastif}. A priori, cette hypothèse semble être correcte pour le roumain, vu la possibilité d’employer la conjonction \textit{iar} dans ces contextes \REF{ch2:foot56i}, alors que cela n’est pas possible pour les \is{Bare Argument Ellipsis (BAE)}BAE. 
 \ea \label{ch2:foot56i}
 \ea Ioana joacă [volei]\textsubscript{CT}, iar [tenis]\textsubscript{CT} de asemenea.
 \glt ‘Ioana joue au volleyball, et au tennis aussi.’
 \ex Ioana joacă [volei]\textsubscript{CT}, dar [tenis]\textsubscript{CT} nu.
 \glt ‘Ioana joue au volleyball, mais au tennis non.’
\z\z}.   


\ea 
\ea{} [Ioana]\textsubscript{CT} \uline{joacă} [volei]\textsubscript{CF}, iar [Maria]\textsubscript{CT} [tenis]\textsubscript{CF}. \label{ch2:ex185a}
\glt ‘Ioana joue au volleyball, et Maria au tennis.


\ex Ioana joacă [volei]\textsubscript{CF}, şi [nu tenis]\textsubscript{CF}. \label{ch2:ex185b} 
\glt ‘Ioana joue au volleyball, et non pas au tennis.’    

\ex Ioana joacă [volei]\textsubscript{CF}, dar [şi tenis]\textsubscript{CF}. \label{ch2:ex185c} 
\glt ‘Ioana joue au volleyball, mais aussi au tennis.’
\z
\z

\citet{Schwabe2000} va plus loin et considère que la \isi{structure informationnelle} joue un rôle très important dans l’identification d’une structure syntaxique ou sémantique pour les constructions elliptiques ; en particulier, c’est le deuxième conjoint qui détermine la \isi{structure informationnelle} de l’antécédent (cf. \citealt{Rooth1992,Rooth1996}). Cela se rapproche de l’hypothèse de \citet[141]{Kuno1982} qui considère le premier élément dans une construction à gapping comme une «~clé de tri~» (\textit{sorting key}) : «~in a multiple \textit{wh-}word question, the fronted \textit{wh-}word represents the key for sorting relevant pieces of information in the answer. » Par conséquent, en l’absence d’un verbe, on arrive à avoir une bonne interprétation des éléments résiduels grâce aussi à la \isi{structure informationnelle} : on identifie le \isi{topique contrastif} dans la phrase trouée, on lui trouve le corrélat dans la phrase source ; l’autre élément résiduel sera un \isi{focus contrastif}, corrélé à un autre \isi{focus contrastif} dans la phrase source. 

Je finis cette section en précisant que le matériel manquant doit être donné dans le discours. A part le verbe antécédent, tout autre matériel qui fait partie du fond (dans la phrase source) ne peut être répété dans la phrase trouée (\ref{ch2:ex186a}--\ref{ch2:ex186b}), sauf s’il s’agit d’un résiduel qui n’est pas un dépendant direct du verbe antécédent \REF{ch2:ex186c} (voir aussi les exemples (\ref{ch2:ex120}--\ref{ch2:ex122}) ci-dessus). 

\ea
\ea Ioana \uline{vine mâine} cu trenul, iar Maria (\#mâine) cu autobuzul. \label{ch2:ex186a}
\glt ‘Ioana vient demain en train, et Maria (demain) en bus. 

\ex Maria \uline{a cumpărat flori} pentru mama ei, iar Ion (\#flori) pentru bunica lui. \label{ch2:ex186b} 
\glt ‘Maria a acheté des fleurs pour sa mère, et Ion (des fleurs) pour sa grand-mère.’

\ex Maria \uline{a luat trenul care merge} la Briançon, iar Ion ??(trenul care merge) la Saint-Gervais. \label{ch2:ex186c} 
\glt ‘Maria a pris le train qui va à Briançon, et Ion le train qui va à Saint-Gervais.’
\z
\z

Pour conclure, on observe qu’on doit postuler aussi un parallélisme informationnel pour les constructions à gapping~présentant la conjonction \textit{iar} en roumain : le résiduel doit avoir le même statut informationnel que son corrélat. En particulier, une construction à gapping contient au moins une \isi{paire contrastive} avec des \is{topique}topiques et une \isi{paire contrastive} avec des \isi{focus}. Une étude détaillée reste à faire afin de vérifier si cette généralisation s’applique aussi en dehors des coordinations avec \textit{iar}.   


\subsubsection{Prosodie} \label{ch2:sect2.3.4.5}


En ce qui concerne la prosodie, on s’est posé la question de savoir si le gapping est associé à une prosodie spéciale. Les principales hypothèses ont été faites sur l’anglais et \il{allemand}l’allemand \citep{Hartmann2000,Schwarz2000,Carlson2001,Carlson2002,FeryEtAl2005,Winkler2005}. Il y a un travail en cours (basé sur des expériences de production) pour le français \citep{AbeilleEtAlInPrep}. Quant au roumain, cela n’a pas été étudié et le sujet ne pourra pas être abordé en détail dans cet ouvrage. Dans cette section, je veux simplement énumérer les points qui semblent importants à étudier quand on s’intéresse au marquage prosodique des constructions à gapping.

Les recherches faites dans ce sens tournent autour de cinq questions majeures : (i) Est-ce que le verbe antécédent est marqué prosodiquement ? (ii) Est-ce qu’à la place du matériel manquant on a un marquage prosodique particulier ? (iii) Est-ce que les éléments contrastifs reçoivent un accent (particulier) ? ou uniquement les éléments résiduels ? (iv) Est-ce qu’il y a une pause entre les conjoints ? (v) Est-ce qu’on observe une compression de registre sur le deuxième conjoint ?

Je me limite ici à simplement résumer les principales hypothèses qui dérivent des travaux faits sur la prosodie de ces constructions.  

(i) On considère que le verbe antécédent est typiquement désaccentué, ce qui permet l’ellipse dans le conjoint troué \citep{Hartmann2000,Schwarz2000}. Cependant, l’étude en cours faite sur le français, qui compare les séquences elliptiques avec leurs contreparties complètes, observe que la \isi{désaccentuation} du matériel antécédent dans la phrase source ne caractérise pas seulement les constructions elliptiques, mais elle peut apparaître aussi dans une coordination non elliptique.  

(ii) Dans une approche syntaxique de l’ellipse, on s’attend à ce qu’il y ait toujours une rupture à l’endroit attendu du matériel manquant. Mais il n’y a pas toujours de pause à l’endroit où se trouve le matérial manquant. En français, \citet{AbeilleEtAlInPrep} remarquent un enchaînement prosodique entre les éléments résiduels de la séquence trouée. 

(iii) On considère qu’au moins les éléments résiduels reçoivent un accent con\-trastif, qui est plus fort qu’un accent syntagmatique habituel, ce qui explique l’impossibilité d’avoir des pronoms inaccentués (ou faibles) dans la phrase trouée en \REF{ch2:ex187} (cf. \citealt{SagEtAl1985}) :
 
\ea \label{ch2:ex187}
*You talked to John’s mother, and I him.  [\textit{him} sans accentuation prosodique] \citep[161]{SagEtAl1985} 
\z

\noindent \citet{Hartmann2000} et \citet{FeryEtAl2005} considèrent que la prosodie souligne le \isi{contraste sémantique} : les différents accents sur les éléments résiduels, ainsi que (parfois) sur leurs corrélats facilitent la construction des \is{paire contrastive}paires contrastives. Selon \citet{FeryEtAl2005}, les éléments résiduels et corrélats non finaux reçoivent un pitch accent montant (L*H), alors que les éléments résiduels et corrélats finaux reçoivent un accent descendant (H*L). 

(iv) De manière générale, chaque conjoint constitue une unité prosodique autonome, {\cad} il y a une frontière marquée par une pause intonative entre les conjoints. Dans les langues citées plus haut, on observerait ainsi un ton de frontière haut (optionnel) à la fin de la phrase source et un ton de frontière bas dans la phrase trouée. Néanmoins, la pause intonative reste optionnelle. Voir, dans ce sens, les coordinations à gapping avec \isi{portée large} de la \isi{négation}, où les conjoints forment une seule unité prosodique (cf. \citealt{Oehrle1987,Winkler2005}).

(v) \citet{FeryEtAl2005} observent une compression de registre dans la séquence trouée. Cependant, l’étude préliminaire faite sur le français montre que le registre du deuxième conjoint est rarement compressé. 

Les travaux faits sur l’anglais et \il{allemand}l’allemand attribuent la spécificité prosodique du gapping au phénomène d’ellipse ; on tirerait donc de l’analyse prosodique des arguments en faveur d’une analyse par ellipse. Cependant, en l’absence d’une description des mêmes contextes sans ellipse, on ne peut pas attribuer une spécificité prosodique aux constructions à gapping.

Une étude détaillée reste à faire pour le roumain, afin de confirmer ou infirmer ces hypothèses avancées sur l’anglais et \il{allemand}l’allemand. Pour le français, une partie de ces hypothèses semble s’infirmer, cf. l’étude en cours de \citet{AbeilleEtAlInPrep}.

L’hypothèse qu’on pourrait faire est que l’intonation dans le gapping est plutôt sensible aux aspects sémantiques (parallélisme entre les \is{paire contrastive}paires contrastives) et pragmatiques (paire de \is{topique}topiques et paire de \is{focus}focus), et moins aux aspects syntaxiques (\textit{contra} \citealt{FeryEtAl2005}), mais cela reste à être vérifié.


\section{Les analyses proposées et leurs limites} \label{ch2:sect2.4}


Les analyses proposées dans la littérature pour traiter les constructions à gapping sont nombreuses et variées. La plupart de ces travaux ont comme cadre théorique le courant dominant de la \isi{grammaire générative} sous ses différentes formes (Théorie Standard étendue, Principes et Paramètres, Programme Minimaliste). Le point commun de toutes ces approches, par-delà leur hétérogénéité, réside dans le rôle très important attribué à la syntaxe pour obtenir l’interprétation de la phrase trouée, la plupart faisant appel à un mécanisme de \isi{reconstruction syntaxique}. L’idée générale est que le matériel manquant a une certaine structure syntaxique à un certain niveau de la représentation, ce qui justifie l’appellation \is{approche structurale}d'\textit{approches structurales}, selon la terminologie proposée par \citet{Merchant2009} et discutée dans la section~\ref{ch1:sect1.5.1}. Leur but est de trouver une solution qui aligne la séquence trouée sur un constituant ordinaire. Les approches syntaxiques peuvent être ainsi synthétisées en suivant deux critères : (i) la taille postulée pour la séquence trouée, en fonction du niveau auquel opère la coordination, et (ii) la nature du matériel manquant.  

Cette section a quatre parties. Les deux premières parties développent les deux critères mentionnés ci-dessus ({\cad} la taille de la séquence trouée et la nature du matériel manquant), afin d’avoir une synthèse des propositions faites au sein des \is{approche structurale}approches structurales. La troisième partie montre les problèmes qu’on rencontre avec les deux analyses dominantes, \is{effacement}l’effacement et le \isi{mouvement} du verbe. Enfin, dans la quatrième partie, je présente brièvement les analyses alternatives proposées dans une perspective \is{approche non structurale}non structurale, qui me permettent de développer ensuite une \is{approche constructionnelle}analyse constructionnelle en HPSG.


\subsection{Taille de la séquence trouée} \label{ch2:sect2.4.1}

Le premier critère qu’on peut utiliser pour résumer~les analyses proposées est le niveau auquel opère la coordination. On arrive ainsi à deux types majeurs : les approches à grand conjoint (\textit{large conjunct approach}) et les approches à petit conjoint (\textit{small conjunct approach}). Dans les deux cas, on part du principe que la coordination opère uniquement entre des constituants de même niveau. Les approches à grand conjoint considèrent que la coordination se passe au niveau supérieur de la phrase (dans les cadres qui postulent des catégories fonctionnelles, cela correspond au CP cf. angl. \textit{Complementizer Phrase}, TP cf. angl. \textit{Tense Phrase}, ou encore IP cf. angl. \textit{Inflection Phrase}, en fonction des notations choisies). En revanche, le deuxième type d’approche voit la coordination comme ayant lieu à un niveau inférieur, {\cad} en-dessous de TP, au niveau VP (\textit{Verb Phrase}) ou \textit{v}P (\textit{Voice Phrase}).  

\subsubsection{Coordination au niveau de la phrase} \label{ch2:sect2.4.1.1}

Les travaux inscrits dans cette perspective débutent avec \citet{Ross1967,Ross1970} et continuent avec \citet{Jackendoff1971}, \citet{Hankamer1973,Hankamer1979}, \citet{Stillings1975}, \citet{Sag1976}, \citet{Neijt1979}, \citet{vanOirsouw1987}, \citet{Wilder1994,Wilder1997}, \citet{AbeEtAl1997,AbeEtAl1999}, \citet{Kim1997}, \citet{Hartmann2000}, etc. Selon eux, dans les constructions à gapping, on coordonne deux phrases. Le conjoint troué a une structure complexe similaire à celle de la phrase source. 

Pour dériver le trou, on fait appel à un certain mécanisme syntaxique de réduction, par lequel le verbe et éventuellement d’autres éléments sont effacés \is{PF-deletion}au niveau PF (\textit{Phonological Form}) ou copiés \is{LF-copy}au niveau LF (\textit{Logical Form}), toujours en correspondance avec la phrase source. 

Dans les cadres postulant un homomorphisme entre les relations de constituance syntaxique et les relations de portée sémantique, cette approche rend compte tout de suite des exemples où la \isi{négation} présente dans la phrase source se distribue sur chaque conjoint, cf. les données de \citet{Repp2009} en \REF{ch2:ex188}.

\ea \label{ch2:ex188}
\ea  Max \uline{didn’t read} the book and Martha the magazine.      
\ex  = [It is not the case that Max read the book] and [it is not the case that Martha read the magazine]. 
\z
\z

Elle est justifiée aussi par les coordinations de CPs (dans la terminologie de \citealt{Repp2009}), dans lesquelles les éléments résiduels sont des syntagmes \textit{qu-} \REF{ch2:ex189a} ou d’autres syntagmes antéposés \REF{ch2:ex189b}, généralement considérés comme étant plus haut que le syntagme verbal dans la structure syntaxique. 

\ea
\ea  When \uline{did} John \uline{arrive} and when Mary ? \citep[34]{Repp2009} \label{ch2:ex189a}
\ex  On this table, \uline{they put} a lamp, and on that table, a radio. \citep[158]{SagEtAl1985} \label{ch2:ex189b}
\z
\z

Un argument de plus pour considérer une coordination à un niveau supérieur est fourni par l’impossibilité d’avoir une \is{asymétrie syntaxique}asymétrie de voix dans le gapping. \citet{Merchant2008a,Merchant2008b} corrèle la «~taille~» de l’ellipse avec la possibilité ou non d’avoir des asymétries de voix entre le syntagme elliptique et la phrase source. On distingue ainsi les \is{high ellipsis}ellipses «~hautes~» (élidant plus que le simple syntagme verbal, comme le \is{Sluicing}sluicing – l’ellipse du TP) des \is{low ellipsis}ellipses «~basses~» (p.ex. \is{Verb Phrase Ellipsis (VPE)}VPE). Seul le dernier type d’ellipse permet les \is{asymétrie syntaxique}asymétries de voix. Comme les asymétries ne sont pas permises dans les constructions à gapping, on pourrait considérer, selon le raisonnement de \ia{Merchant, Jason}Merchant, que le gapping est une opération qui a lieu à un niveau supérieur, {\cad} celui de la phrase.

Les problèmes de ce type d’approche ont été discutés à plusieurs reprises par \citet{Johnson1996/2004,Johnson2000,Johnson2009} et seront mentionnés dans la section suivante. 


\subsubsection{Coordination sous-phrastique} \label{ch2:sect2.4.1.2}

Le deuxième type d’approche apparaît dans les travaux de \citet{Siegel1984,Siegel1987}, \citet{Johnson1996/2004,Johnson2000,Johnson2008,Johnson2009}, \citet{Coppock2001}, \citet{Lin2002}, \citet{LopezEtAl2003}, \citet{Winkler2005}, \citet{Hulsey2008}, \citet{Agafonova2014}, etc. Dans cette perspective, le conjoint troué est plus petit qu’une phrase, {\cad} il est un VP/\textit{v}P (le syntagme verbal dans lequel le sujet est généré). 

Je mentionne par la suite les arguments inventoriés par les adeptes de cette approche contre l’analyse à grand conjoint, tels qu’ils apparaissent dans les discussions de \citet{Johnson1996/2004,Johnson2000,Johnson2009}\footnote{
 Les exemples qui n’ont aucune indication concernant l’auteur viennent de \citet{Johnson2000,Johnson2009}.}. 

Si l’on admet l’homomorphisme syntaxe-sémantique (comme c’est le cas dans les \is{grammaire dérivationnelle}grammaires dérivationnelles), l’approche à grand conjoint prédit le fait qu’aucun élément de la phrase source ne peut lier ou avoir portée sur un élément dans la phrase trouée. Or, cette prédiction s’avère être fausse. Dans les constructions à gapping, il y a des cas de \isi{coréférence} croisée, ainsi que des situations où un élément prend \isi{portée large} sur toute la coordination, comme le notent \citet{Oehrle1987} et \citet{McCawley1993} pour le premier cas, et \citet{Siegel1984,Siegel1987} et \citet{Oehrle1987} pour le deuxième. 

D’abord, certaines constructions à gapping permettent la \isi{coréférence} entre le sujet de la phrase source et une expression \is{anaphore}anaphorique dans le deuxième sujet, à condition que le verbe ne soit pas répété dans le deuxième conjoint ({\cad} à condition qu’il y ait du gapping), cf. les jugements différents d’acceptabilité dans les exemples repris de \citet{Johnson2000,Johnson2009} en \REF{ch2:ex190a} et \REF{ch2:ex190b}. Johnson explique cette possibilité par le fait que le quantifieur présent dans la phrase source est en dehors du domaine de la coordination et c-commande le pronom en question, ce qui prédit correctement la \isi{coréférence}. 

\ea \label{ch2:ex190}
\ea No woman\textsubscript{i} \uline{can join} the army and her\textsubscript{i} girlfriend the navy. \label{ch2:ex190a}     
\ex *No woman\textsubscript{i} can join the army and her\textsubscript{i} girlfriend can join the navy. \label{ch2:ex190b} 
\z
\z

Ensuite, il y a des exemples comme \REF{ch2:ex191} avec un modal nié qui a \isi{portée large} sur la coordination \REF{ch2:ex191a}, n’ayant pas une lecture distributive dans chaque con\-joint \REF{ch2:ex191b}. Même observation pour un quantifieur présent dans le syntagme sujet de la phrase source \REF{ch2:ex192a}, ou encore un adverbe quantificationnel \REF{ch2:ex192b}. L’approche à petit conjoint prédit correctement ces données, car tous ces éléments se situent en dehors du domaine de la coordination. 

 
\ea \label{ch2:ex191}
Mrs. Smith \uline{\textbf{can’t}} dance or Mr. Smith sing. 
\ea = Mrs. Smith can’t dance and Mr. Smith can’t sing. \label{ch2:ex191a}
\ex ≠ Mrs. Smith can’t dance or Mr. Smith can’t sing. \label{ch2:ex191b}
\z
\z

\ea \label{ch2:ex192}
\ea \textbf{Not every} girl \uline{ate} a GREEN banana and her mother a RIPE one. \label{ch2:ex192a}      
\ex A German stepherd \uline{is \textbf{rarely}} named Kelly or an Irish setter Fritz. \label{ch2:ex192b}
\z
\z

Un argument de plus, invoqué par \citet{Siegel1987}, est constitué par la \is{asymétrie syntaxique}discordance casuelle qu’on observe en anglais avec les pronoms sujets de la phrase source et respectivement de la séquence trouée. Tandis que le premier sujet est toujours au nominatif, on constate une préférence des locuteurs pour un pronom sujet à l’accusatif dans la séquence trouée \REF{ch2:ex193a}, et apparemment c’est ce qui se passe en dehors de l’ellipse aussi \REF{ch2:ex193b}. L’analyse à petit conjoint prédit ces cas, si on considère (dans les \is{grammaire dérivationnelle}grammaires dérivationnelles) que le sujet de la séquence trouée ne monte pas vers IP pour vérifier ou recevoir le cas nominatif\footnote{
 Le fait d’avoir le sujet à l’accusatif dans la séquence trouée n’est en rien un argument pour motiver le \isi{mouvement} du verbe. On pourrait considérer l’accusatif dans ce cas comme la forme par défaut (comme dans les \is{réponse courte}réponses courtes en \REF{ch2:foot58i}) ou, sinon, comme une idiosyncrasie morpho-syntaxique. D’ailleurs \citet{Kim2006} donne plusieurs exemples \REF{ch2:foot58ii} avec des coordinations de pronoms marqués différemment :
 
 \ea \label{ch2:foot58i}
 A : - Who’s entering ? B : - \{Me {\textbar} *I\}.
 \z

\ea \label{ch2:foot58ii}
\ea Them and us are going to the game together.
\ex She and him will drive to the movies.
\ex All debts are cleared between you and I. \citep[603]{Kim2006}  
\z
\z
}.


\ea 
\ea I \uline{cooked} fish, and \{him {\textbar} ?he\} rice. \citep{ZoernerEtAl2000} \label{ch2:ex193a} 
\ex We can’t eat caviar and \{him {\textbar} ??he\} can’t eat beans. \citep[184]{Winkler2005} \label{ch2:ex193b}
\z
\z

Un autre type de données difficile à concilier avec une analyse postulant une coordination phrastique dans les constructions à gapping est lié aux items à \isi{polarité} négative qui peuvent apparaître dans la séquence trouée \REF{ch2:ex194}. Ce type d’approche considère que l’item à \isi{polarité} négative dans le deuxième conjoint se trouve dans une position où la \isi{négation} ne peut le c-commander, donc son légitimeur devrait être en dehors du conjoint qui le contient.  

\ea \label{ch2:ex194}
During dinner \uline{he didn’t address} his colleagues from Stuttgart or at any time his boss, for that matter. \citep[186]{Winkler2005} 
\z

Cependant, cette analyse rencontre des difficultés. Un premier problème est lié à la portée de la \isi{négation} \citep{Repp2009,Toosarvandani2011}. Cette approche prédit toujours la \isi{portée large} de tous les éléments, mais elle ne prédit jamais la \isi{portée étroite} de la \isi{négation}, où on a une \isi{polarité} différente dans les deux conjoints, cf. l’exemple de \citet[2]{Repp2009} en \REF{ch2:ex195}.

\ea \label{ch2:ex195}
\ea Pete \uline{was}n’t \uline{called} by Vanessa, but John by Jessie.      
\ex = [It is not the case that Pete was called by Vanessa] but [it is the case that John was called by Vanessa]. 
\z
\z

Un autre problème, relevé par \citet{Ince2009}, est la \isi{coréférence} qui peut s’établir entre un objet pronominal dans le deuxième conjoint et le sujet de la phrase source. Si on applique l’approche de Johnson à l’exemple \REF{ch2:ex196}, \textit{John} c-commande \textit{him} et en même temps il appartient au domaine de liage de \textit{him}. Ce qui a comme résultat une \is{principes du liage}violation du principe B, selon lequel un pronom doit être libre dans son domaine de liage.

\ea \label{ch2:ex196}
John\textsubscript{i} \uline{will hug} Mary and Mary him\textsubscript{i}. \citep[205]{Ince2009} 
\z

\citet{Ince2009} observe une autre difficulté de cette analyse en \ili{turc}, où \is{ordre de mots}l’ordre des éléments résiduels et corrélats n’est pas le même \REF{ch2:ex197}. Si on a l’ordre SOV-OS comme en \REF{ch2:ex197b}, l’objet résiduel est analysé ici comme un syntagme antéposé (TopP), et par conséquent il est nécessairement plus haut que le \textit{v}P. Donc, on ne voit pas comment obtenir une coordination de \textit{v}P, si au moins le deuxième conjoint n’est pas un \textit{v}P.

\ea \label{ch2:ex197}
\il{turc}\langinfo{Turc}{}{\citealt[199]{Ince2009}}\\
\ea 
\gll Adam  kitab-ı  \uline{okudu},  çocuk da dergi-yi. \label{ch2:ex197a}\\
homme  livre-\textsc{acc}  lire.\textsc{pst.3sg}  enfant \textsc{conj} revue-\textsc{acc}\\
\glt ‘L’homme a lu le livre et l’enfant la revue.’

\ex 
\gll Adam  kitab-ı  \uline{okudu},  dergi-yi de çocuk. \label{ch2:ex197b}\\
homme  livre-\textsc{acc}  lire.\textsc{pst.3sg}  revue-\textsc{acc} \textsc{conj} enfant\\
\glt ‘L’homme a lu le livre et l’enfant la revue.’ 
\z
\z

\newpage 
Enfin, une coordination de \textit{v}P dans les constructions à gapping ne rend pas compte de l’occurence des adverbes de phrase \REF{ch2:ex198} dans la séquence trouée ou dans les deux conjoints (cf. \citealt{Gardent1991,Ince2009})\footnote{
 Ce dernier aspect ne constitue pas un problème insurmontable, si l’on considère que les adverbes de phrase peuvent s’attacher à un verbe ou un syntagme verbal et prendre portée sur toute la phrase dans un exemple comme \REF{ch2:foot59i}.
 
\ea \label{ch2:foot59i}
Paul \textbf{necessarily} will go to see Mary.
\z
}. 



\ea \label{ch2:ex198}
John \uline{will} \textbf{probably} \uline{go to see} Mary and \textbf{necessarily} Paul, Sarah. \citep[49]{Gardent1991} 
\z


\subsection{Nature du matériel manquant} \label{ch2:sect2.4.2}

Les approches syntaxiques \is{approche structurale}structurales peuvent encore être classées en fonction d’un autre critère, celui de la nature du matériel manquant. Les différentes solutions proposées pour la \isi{reconstruction syntaxique} tombent dans une des trois classes majeures : (i) le trou est le résultat d’une ellipse par \isi{effacement} \is{PF-deletion}au niveau PF (\textit{Phonological Form}) ; (ii) le trou est une ellipse générée dès le départ sous la forme d’un \isi{élément vide} qui est reconstruit au niveau LF (\textit{Logical Form}) ; (iii) le trou n’est pas obtenu par l’ellipse, mais il est la trace d’un \isi{mouvement}. Dans ce dernier cas, le \isi{mouvement} se prête à deux analyses : soit il opère simultanément dans les deux conjoints \is{Across-The-Board (ATB) Movement}(\textit{Across-the-Board movement}), soit il opère latéralement \is{sideward movement}(\textit{sideward movement}). Je présente brièvement les trois grandes analyses, mais j’insisterai plus sur les deux analyses dominantes : \isi{effacement} du verbe (donc, une ellipse à base de \isi{reconstruction syntaxique}) vs. \isi{mouvement} du verbe (donc, pas d’ellipse).

\subsubsection{Effacement au niveau PF} \label{ch2:sect2.4.2.1}
Dans cette approche, la phrase trouée a une structure syntaxique ordinaire, similaire à la phrase source, avec un verbe (et éventuellement d’autres éléments) identique à son antécédent. A un moment donné de la dérivation, lorsque la structure syntaxique est traitée par le module phonologique, le contenu phonologique du verbe / VP / TP est effacé (ou il n’y a pas d’insertion lexicale tardive), ce qui a comme résultat une structure «~silencieuse~», non prononcée, en surface. Le matériel manquant est par conséquent présent et articulé au niveau syntaxique, mais non réalisé au niveau phonologique. L’interprétation de la phrase trouée est obtenue avant \is{effacement}l’effacement.

En fonction du type \is{effacement}d’effacement qui entre en jeu, on peut avoir deux types d’analyses :

\noindent (i) Les analyses qui postulent simplement \is{effacement}l’effacement du verbe (avec éventuellement d’autres éléments ‘non-contrastifs’), sans invoquer une opération supplémentaire (\citealt{Ross1967,Ross1970,Jackendoff1971,Hankamer1973,Hankamer1979,Stillings1975,Kuno1976,Neijt1979,vanOirsouw1987,Wilder1994,Wilder1997,Hartmann2000,FeryEtAl2005}, etc.). Si le verbe est accompagné d’autres éléments, la règle du gapping opère un effacement de non-constituants.

\ea
John \uline{likes} caviar and [\textsubscript{IP} Mary \st{likes} beans].       
\z

\noindent (ii) Les analyses qui postulent \is{effacement}l’effacement d’un constituant syntagmatique incluant le trou, {\cad} un VP en \REF{ch2:ex200a} (dans les approches à petit conjoint) ou bien un TP/IP en \REF{ch2:ex200b} (dans les approches à grand conjoint)\footnote{
 Généralement, les adeptes de cette approche considèrent que la coordination dans les constructions à gapping a lieu au niveau de la phrase, mais pas tous (voir, par exemple, \citealt{Coppock2001}, \citealt{Lin2002}).}. L’effacement dans ces cas a lieu après avoir déplacé les éléments résiduels contrastifs\footnote{
 On assume généralement que les éléments résiduels se déplacent à gauche. Mais il y a des auteurs qui considèrent que le matériel lourd qui est interne au VP se déplace à droite, via l’opération de \is{Heavy NP Shift}\textit{Heavy NP Shift} \citep{Jayaseelan1990,Kim1997,Kim2006}. Un argument contre ce type d’opération vient du fait que les éléments résiduels peuvent ne pas être des NP lourds, p.ex. les pronoms ou certains adverbes.}  (\citealt{Sag1976,Jayaseelan1990,Kim1997,Kim1998,Kim2006,Coppock2001,Lin2002,KonietzkoEtAl2010,MolnarEtAl2010}, etc.).

\ea
\ea John likes caviar and [\textsubscript{VP} Mary\textsubscript{1} [\textsubscript{VP} beans\textsubscript{2} [\textsubscript{VP} \st{\textit{t}}\textsubscript{1} \st{likes} \st{\textit{t}}\textsubscript{2}]]]. \citep{Coppock2001} \label{ch2:ex200a} %the command \st should include the \textsubscript too

\ex John likes caviar and [\textsubscript{TopP} Mary\textsubscript{1} [\textsubscript{FocP} beans\textsubscript{2} [\textsubscript{TP} \st{\textit{t}}\textsubscript{1} \st{likes} \st{\textit{t}}\textsubscript{2}]]]. \label{ch2:ex200b} %the command \st should include the \textsubscript too
\z
\z

Contrairement à l’approche en termes de \isi{proforme nulle} (cf. section~\ref{ch2:sect2.4.2.2}), le trou effacé est syntaxiquement structuré, donc des opérations syntaxiques intervenant avant \is{effacement}l’effacement phonologique (le \isi{mouvement}, le \is{principes du liage}liage, etc.) peuvent l’affecter au même titre que dans une phrase ordinaire. Par conséquent, le test utilisé par ces approches est la possibilité ou non d’extraire des éléments enchâssés dans le syntagme contenant le verbe effacé.

Une des analyses les plus discutées dans cette perspective est le travail de \citet{Coppock2001}, qui essaie de justifier \is{effacement}l’effacement en observant les similarités qui existent entre le gapping et les types d’ellipse pour lesquels on assume habituellement une analyse par \isi{effacement} (en particulier, \is{Verb Phrase Ellipsis (VPE)}VPE). Bien que son analyse soit une approche à petit conjoint (la coordination dans le gapping opérant à un niveau sous-phrastique) comme les analyses à base de \isi{mouvement} du verbe (cf. section~\ref{ch2:sect2.4.2.3}), Coppock mentionne trois aspects qui soutiennent \is{effacement}l’effacement et qui ne sont pas pris en charge, selon elle, par les analyses en termes de \isi{mouvement} du verbe. En résumé, ce sont : (i) la désambiguïsation de la portée et de \is{anaphore}l’anaphore, (ii) les avantages empiriques de la condition \isi{e-GIVENness}, et (iii) la sélectivité dans l’application des \isi{contraintes d'îles}. 

En ce qui concerne le premier aspect, \citet{Coppock2001} observe que le gapping, tout comme \is{Verb Phrase Ellipsis (VPE)}VPE, désambiguïse la portée des quantifieurs et les interprétations de \is{anaphore}l’anaphore\footnote{
 \citet{Johnson2009} considère qu’en réalité ces deux aspects sémantiques ne posent aucun problème pour une analyse en termes de \is{Across-The-Board (ATB) Movement}mouvement ATB. Les deux approches concurrentes en rendent compte.}. En conformité avec la condition de parallélisme de portée, formulée par \citet{Fox2000}, le gapping réduit l’ambiguïté liée à la portée des quantifieurs : tandis qu’une phrase simple ordinaire contenant un \isi{quantifieur existentiel} et un \isi{quantifieur universel} présente une ambiguïté de portée en \REF{ch2:ex201a} (($\forall$ > $\exists$), ($\exists$ > $\forall$)), une phrase plus complexe avec une séquence elliptique en \REF{ch2:ex201b} ne présente plus d’ambiguïté, le syntagme nominal résiduel du deuxième conjoint forçant une seule interprétation dans la phrase source aussi, {\cad} $\exists$ > $\forall$.   

\ea
\ea A student accompanied every visitor. ($\forall$ > $\exists$), ($\exists$ > $\forall$) \label{ch2:ex201a}  
\ex A student \uline{accompanied every visitor} yesterday, and Mr. Johnson, today. ($\exists$ > $\forall$), (*$\forall$ > $\exists$) \label{ch2:ex201b}  
\z
\z

Comme \is{Verb Phrase Ellipsis (VPE)}VPE, le gapping élimine aussi l’ambiguïté liée à l’interprétation d’une \isi{anaphore} dans un contexte avec plusieurs pronoms (angl. \textit{the Many-Pronouns Puzzle}, cf. \citealt{FiengoEtAl1994}) : l’exemple non elliptique en \REF{ch2:ex202a} a plus de possibilités de lecture que l’exemple avec ellipse en \REF{ch2:ex202b}, qui interdit l’interprétation stricte (angl. \textit{strict reading}) du premier des pronoms si le deuxième pronom a une interprétation relâchée (angl. \textit{sloppy reading}). 

\ea
\ea Max said he gave his mother a bracelet, and Oscar said he gave his mother a watch. (stricte-stricte, relâché-relâché, stricte-relâché, relâché-stricte) \label{ch2:ex202a}
\ex Max said he gave his mother a bracelet, and Oscar a watch. (stricte-stricte, relâché-relâché, *stricte-relâché, relâché-stricte) \label{ch2:ex202b}
\z
\z

\citet{Coppock2001} applique aux constructions à gapping la condition de légitimation de l’ellipse, formulée par \citet{Merchant2001}~pour les cas de \is{Sluicing}sluicing, {\cad} la condition de \isi{e-GIVENness}. La phrase source α a deux constituants marqués F (\isi{focus}) qui peuvent être remplacés par des variables liées comme en \REF{ch2:ex203a}. La phrase trouée contient elle aussi deux constituants marqués F qui peuvent être remplacés par des variables liées comme en \REF{ch2:ex203b}. La clotûre existentielle étant la même dans les deux phrases, elles s’impliquent mutuellement, ce qui fait que la séquence elliptique est discursivement donnée (e-GIVEN) et, par conséquent, \is{effacement}l’effacement est légitimé. 

\ea
\ea {} [α John\textsubscript{F} likes caviar\textsubscript{F}] and [γ Mary\textsubscript{F} beans\textsubscript{F}]. \label{ch2:ex203a}  
\ex F-clo (α) = $\exists$\textit{x}.$\exists$\textit{y}(\textit{x} likes \textit{y}) \label{ch2:ex203b}
\ex F-clo (γ) = $\exists$\textit{x}.$\exists$\textit{y}(\textit{x} likes \textit{y})
\z
\z

Un premier avantage relevé par \citet{Coppock2001} réside dans le fait que cette condition rendrait immédiatement compte des effets prosodiques, comme le mon\-trent les exemples suivants\footnote{
 L’analyse de Johnson (en termes de \isi{mouvement} du verbe, cf. section~\ref{ch2:sect2.4.2.3}) n’inclut pas l’exemple avec «~trou à distance~» \REF{ch2:ex204a}, considéré agrammatical, ni l’exemple \REF{ch2:ex205a}, considéré ambigu.} de \citet{Sag1980} en \REF{ch2:ex204}, et respectivement \citet{Hankamer1973} en \REF{ch2:ex205}.

\ea \label{ch2:ex204}
\ea John\textsubscript{F} said he wants caviar\textsubscript{F} for dinner, and Mary\textsubscript{F} beans\textsubscript{F}. \label{ch2:ex204a}
\ex John said he\textsubscript{F} wants caviar\textsubscript{F} for dinner, and
Mary\textsubscript{F} beans\textsubscript{F}. \citep{Sag1980} \label{ch2:ex204b}
\z
\z


\ea \label{ch2:ex205}
\ea Massachussets\textsubscript{F} elected McCormack\textsubscript{F} Congressman, and Pennsylvania\textsubscript{F} Schweicker\textsubscript{F}. \label{ch2:ex205a}      
\ex ≠ Massachussets elected [Pennsylvania Schweicker]\textsubscript{F}. \citep{Hankamer1973} \label{ch2:ex205b}
\z
\z

Un deuxième avantage en lien avec la condition de \citet{Merchant2001} concerne les antécédents éparpillés \is{split antecedent}(\textit{split antecedents}). Cette contrainte permet le manque d’isomorphisme structural entre la phrase source et la phrase trouée, ce qui rend possibles les cas de gapping \REF{ch2:ex206} où le trou a un antécédent éparpillé (\textit{contra} \citealt{HankamerEtAl1976})\footnote{
 \citet[304]{Johnson2009} considère que les exemples en \REF{ch2:ex206} sont agrammaticaux.}. Une analyse à base de \isi{mouvement} du verbe (à la Johnson, cf. section~\ref{ch2:sect2.4.2.3}) échoue, car il n’y a pas de destination unique pour les verbes déplacés ATB, et de plus le trou n’est isomorphe à aucun des antécédents.

\ea \label{ch2:ex206}
\ea Wendy wants to sail around the world because she loves travel, and Bruce wants to climb Kilimanjaro in order to prove to himself that he can, but neither in order to show off for anyone.  
\ex Fred bought Suzy flowers in order to thank her, and Bob took her out to eat because they both like sushi, but neither because they want to date her.
\ex John calls home on Sundays, and Jill balances her checkbook every other week, but neither very consistently. \citep{Coppock2001}
\z
\z

Enfin, \citet{Coppock2001} considère que \is{effacement}l’effacement est préférable au \isi{mouvement} du verbe, car il nous permet de rendre compte des différences observées dans les constructions à gapping par rapport aux \isi{contraintes d'îles}. Contrairement aux premiers travaux (\citealt{Neijt1979}, etc.), \citet{Coppock2001} considère que le gapping est sensible aux \isi{contraintes d'îles} de manière sélective, ce qui justifierait la dichotomie proposée par \citet{Merchant2001} qui distingue les \isi{îles PF} et les \isi{îles propositionnelles}. Les îles ne se comportent pas de la même façon par rapport aux ellipses, elles ne sont donc pas homogènes. Si certaines ellipses violent les \isi{contraintes d'îles}, il s’agirait toujours des \isi{îles PF}. Elle observe ainsi que le gapping permettrait \is{extraction}l’extraction hors de certaines \isi{îles PF} (p.ex. Contrainte sur la Branche Gauche en \REF{ch2:ex207}), mais il obéirait à toutes les contraintes \is{îles propositionnelles}d’îles propositionnelles (contraintes sur les relatives \REF{ch2:ex208a}, les sujets phrastiques \REF{ch2:ex208b}, les ajouts \REF{ch2:ex208c}, les interrogatives \textit{qu-} \REF{ch2:ex208d})\footnote{
 Dans l’approche de \citet{Coppock2001}, cela est un argument pour le \isi{mouvement} A’ des éléments résiduels.}. Si le gapping est un \isi{effacement}, on prédirait ces effets, qui sont d’ailleurs observés avec d’autres types d’ellipse aussi. Cependant, cet argument ne peut être tenu, comme on le verra dans la section~\ref{ch2:sect2.4.3.1}.

\ea \label{ch2:ex207}
\ea I \uline{make} too strong \uline{an espresso}, and Fred (*makes) too weak.      
\ex Mary \uline{wrote} too long \uline{a paper}, and Suzy (*wrote) too short. \citep{Coppock2001}
\z
\z

\ea \label{ch2:ex208}
\ea *Suzy \uline{doesn’t like men who play} instruments, and Mary, sports. \label{ch2:ex208a}      
\ex *\uline{That} John \uline{hangs out with Mary is bothersome} to Suzy, and Suzy, to Laura. \label{ch2:ex208b}
\ex *John \uline{must be a fool to have married} Jane, and Bill, Martha. \label{ch2:ex208c}
\ex *John \uline{wondered what to cook} today, and Peter tomorrow. \citep{Coppock2001} \label{ch2:ex208d}
\z
\z


\subsubsection{Reconstruction au niveau LF} \label{ch2:sect2.4.2.2}

Dans ce deuxième type \is{approche structurale}d’approche structurale, le matériel manquant est un \isi{élément vide} (généré dès le départ). Comme il n’y a pas de matériel lexical, la structure syntaxique n’est pas prononcée. Deux sous-types d’approches peuvent être distingués :

(i) Le trou est une expression anaphorique qui est nulle phonologiquement. Cette \isi{proforme nulle} est interprétée comme un pronom ordinaire par des moyens purement sémantiques, au niveau LF (\citealt{Wasow1972,Williams1977,Williams1997,Zribi-Hertz1986}, etc.). Le gapping est conçu ainsi comme un mécanisme interprétatif établissant une relation anaphorique entre un \isi{élément vide} et un antécédent identifié en forme logique \REF{ch2:ex209}. Le trou n’est pas structuré syntaxiquement, {\cad} il n’a pas une structure interne, par conséquent, aucune opération syntaxique n’est susceptible de l’affecter. 

\ea \label{ch2:ex209}
Marlon \uline{boit}\textsubscript{i} du rhum et Raquel Ø\textsubscript{i} du whisky. \citep[398]{Zribi-Hertz1986}  
\z

(ii) Le trou est une copie de son antécédent \is{LF-copy}au niveau LF \REF{ch2:ex210}. Une fois que l’épel du premier conjoint a eu lieu, l’antécédent dans la phrase source est copié dans le site de l’ellipse de la phrase trouée au niveau LF, ce qui apporte aux \is{élément vide}éléments vides la bonne interprétation \citep{AbeEtAl1997,AbeEtAl1999,Repp2009}. Les travaux de ce type assument généralement le déplacement des éléments résiduels\footnote{
 Comme pour les approches à \isi{effacement}, le déplacement des éléments résiduels (en anglais) a lieu soit dans une seule direction, à gauche \citep{Repp2009}, soit dans les deux sens, {\cad} un élément résiduel à gauche et l’autre à droite \citep{AbeEtAl1999}.}. Selon \citet{Repp2009}, on copie uniquement le matériel nécessaire pour construire une dérivation convergente à partir de la numération appauvrie de la phrase trouée. Cela veut dire qu’on ne copie pas les ajouts.

\ea {} [\textsubscript{IP} John [\textsubscript{I’} [\textsubscript{I’} \uline{talked} \textit{t}\textsubscript{1}] about Bill\textsubscript{1}]] and [\textsubscript{IP} Mary [\textsubscript{I’} [\textsubscript{I’} \textit{e} ] about Susan]]. \citep{AbeEtAl1999} \label{ch2:ex210} 
\z

La principale motivation pour postuler une approche en termes \is{élément vide}d’élément vide~est fondée sur l’observation que le site de l’ellipse se comporte sous certains aspects comme un pronom ordinaire. Ainsi, dans les constructions à gapping le matériel manquant dans la phrase trouée pourrait avoir un antécédent éparpillé \is{split antecedent}(\textit{split antecedent}), comme un pronom ordinaire. Dans ce sens, voir les données de \citet{Coppock2001} en \REF{ch2:ex208} ci-dessus. Une deuxième ressemblance avec les pronoms ordinaires consisterait dans le fait que le matériel manquant pourrait correspondre à un antécédent non linguistique, ayant donc un \isi{emploi exophorique}, comme c’est le cas des pronoms déictiques. Si cela semble évident pour \is{Verb Phrase Ellipsis (VPE)}VPE, cf. \citet{Lobeck1995}, pour le gapping c’est discutable \citep{HankamerEtAl1976}.

Néanmoins, cette approche est problématique pour au moins deux raisons. Il faut préciser que le matériel manquant dans le gapping a des propriétés qui le distinguent clairement d’une expression pronominale \citep{Kehler2002}. Il ne peut pas avoir un emploi cataphorique sous enchâssement \REF{ch2:ex211}. En plus, il ne peut pas avoir accès à l’antécédent d’une phrase source qui ne précède pas immédiatement la phrase trouée \REF{ch2:ex212}.

\ea \label{ch2:ex211}
*If George the newspaper reporters, Al \uline{will make a statement blasting} the press. \citep[91]{Kehler2002} 
\z

\ea \label{ch2:ex212}
A : George \uline{made a statement blasting} the press. He’s going to pay a big price for that.\\
B : \#And Al the newspaper reporters. In his case the fallout will be minimal, however. \citep[91]{Kehler2002} 
\z


\subsubsection{Mouvement (parallèle ou latéral) du verbe} \label{ch2:sect2.4.2.3}

A côté de l’effacement \is{PF-deletion}au niveau PF et de la reconstruction \is{LF-copy}au niveau LF, on trouve une troisième approche syntaxique, fondée sur deux points essentiels : d’abord, on considère que la coordination se place non au niveau de la phrase, mais au niveau du \textit{v}P (voir les arguments donnés dans la section~\ref{ch2:sect2.4.1.2}) ; ensuite, le trou est analysé comme une trace du \isi{mouvement}.

Ces approches se divisent en deux sous-types, en fonction du type de mouvement envisagé pour le verbe dans le gapping :

(i) Mouvement parallèle \is{Across-The-Board (ATB) Movement}(\textit{Across-The-Board Movement}) : le matériel «~partagé~» par les deux conjoints est extrait simultanément des deux conjoints \REF{ch2:ex213}. Ainsi, le verbe du chaque conjoint se déplace dans le Spec,PredP/TP/IP (après avoir déplacé les éléments résiduels contrastifs), cf. \citet{Johnson1996/2004,Johnson2000,Johnson2009} and \citet{ZoernerEtAl2000}. Les étapes envisagées sont : d’abord, \is{extraction}l’extraction des éléments résiduels et corrélats vers une position A’, à la périphérie gauche du \textit{v}P en question ; ensuite, \is{extraction parallèle}l’extraction parallèle des deux \textit{v}Ps coordonnés vers la position Spec,PredP, et enfin, le déplacement du sujet du premier conjoint vers Spec,TP. On observe ainsi que ce type d’analyse ne voit pas un phénomène d’ellipse dans les constructions à gapping (voir, dans ce sens, les différences que Johnson établit entre le gapping, d’une part, et \is{Verb Phrase Ellipsis (VPE)}VPE et le \is{Pseudogapping}pseudogapping, d’autre part).

\ea \label{ch2:ex213} 
{} [\textsubscript{TP} Randy\textsubscript{j} [\textsubscript{T’} \uline{drank}\textsubscript{i} [\textsubscript{vP} \textit{t}\textsubscript{j}\textit{ t}\textsubscript{i} scotch] and [\textsubscript{vP} Amy \textit{t}\textsubscript{i} rum]]]. \citep{Johnson2009}  
\z

(ii) Mouvement latéral \is{sideward movement}(\textit{sideward movement}) du \textit{v}P libéré de ses éléments résiduels, qui ont été extraits en préalable à la périphérie gauche du \textit{v}P \citep{LopezEtAl2003,AgbayaniEtAl2004,Winkler2005}\footnote{\citet{LopezEtAl2003} et \citet{Winkler2005} font plutôt appel à une analyse mixte : \isi{mouvement} latéral du verbe, ensuite \isi{effacement} du VP libéré, dans une \is{grammaire dérivationnelle}approche dérivationnelle par phases. Il s’agit d’un phénomène de type minimaliste copier-et-fusionner (angl. \is{Copy-and-Merge}\textit{Copy-and-Merge}), qui copie des constituants et les fusionne avec des objets syntaxiques qui ne sont pas reliés et qui sont indépendamment fusionnés.}. Les étapes envisagées sont les suivantes. Les éléments résiduels se déplacent vers une position A’, à la périphérie gauche du \textit{v}P ; le \textit{v}P qui contient les traces des éléments résiduels est copié, déplacé latéralement vers l’autre \textit{v}P avec lequel il est coordonné, et il fusionne avec les éléments corrélats dans le premier conjoint ; le sujet du premier conjoint se déplace vers la position Spec,TP (pour ainsi satisfaire le \isi{Principe de la Projection Etendue}) ; le \textit{v}P du premier conjoint est antéposé pour des raisons liées à l’ordre des mots ; les deux conjoints fusionnent, et enfin les copies \textit{v}Ps qui sont restées en bas sont effacées. 

Quand je discute l’analyse du \isi{mouvement} du verbe pour le gapping, je prends en compte essentiellement l’analyse de \ia{Johnson, Kyle}Johnson. Pour plus d’informations sur les différences entre les deux types de \isi{mouvement}, voir \citet{Winkler2005}.

En dehors des aspects qu’on a vus dans la section~\ref{ch2:sect2.4.1.2}, militant pour une coordination de \textit{v}P, on peut ajouter d’autres éléments qui justifient le \isi{mouvement} cette fois-ci. Un des avantages majeurs de l’analyse de Johnson, selon \citet{Vicente2010}, réside dans l’explication élégante qu’elle donne pour deux restrictions qu’on observe avec le gapping (mais pas avec \is{Verb Phrase Ellipsis (VPE)}VPE) : le fait que le gapping apparaît uniquement avec la coordination et qu’il ne peut être enchâssé.  

Selon \citet{ZoernerEtAl2000}, l'analyse par \isi{mouvement} du verbe rend mieux compte des discordances qu’on pourrait trouver entre un trou et son antécédent, en ce qui concerne \is{accord}l’accord, comme illustré en \REF{ch2:ex214}. Le verbe «~monté~» s’accorde avec le sujet de la phrase source.  

\ea \label{ch2:ex214}
The president \uline{approves} the education bill, and the senators approve the health bill.
\z

\citet{Repp2009}, dans la synthèse qu’elle fait sur les approches du gapping, présente quelques aspects qui semblent être prédits par le \isi{mouvement} du verbe. D’abord, ce type d’analyse fait une prédiction sur les langues qui ont ou qui n’ont pas de gapping : les langues qui ne présentent pas le \isi{mouvement} du verbe, ne peuvent pas avoir de gapping (p.ex. le \ili{chinois})\footnote{
 Néanmoins, il faut faire attention à ce type de généralisations, car il semble que le gapping ne soit pas complètement impossible en \ili{chinois} (cf. \citealt{Paul1999,RuixiRessy2008}).}. Une autre prédiction concerne le fait que le gapping semble être plus contraint dans les langues qui permettent le \isi{scrambling} que dans les langues qui ne l’ont pas (comparer \il{allemand}l’allemand et l’anglais). Selon \citet{LopezEtAl2003}, le \isi{mouvement} du \isi{complexe verbal} vers une position flexionnelle rend compte aussi de la directionnalité du gapping dans les langues à tête finale (\is{catalepse (backward ellipsis)}catalepse en \ili{japonais} et en \ili{coréen}).  


\subsection{Limites des analyses structurales} \label{ch2:sect2.4.3}

Le but de cette section est de montrer qu’aucune des \is{approche structurale}analyses structurales majeures, p.ex. effacement \is{PF-deletion}au niveau PF (à la \citealt{Coppock2001}), ou bien \isi{mouvement} du matériel manquant (à la \citealt{Johnson1996/2004,Johnson2009}), n’offre une solution adéquate pour les constructions à gapping.


\subsubsection{Problèmes de l’extraction des éléments résiduels} \label{ch2:sect2.4.3.1}

Toutes les propositions récentes dans les deux types d’approches majeures (effacement \is{PF-deletion}au niveau PF / reconstruction \is{LF-copy}au niveau LF ou bien \isi{mouvement} du matériel manquant) font appel à une opération de déplacement des éléments résiduels (et éventuellement des éléments corrélats dans la phrase source) vers des projections fonctionnelles périphériques. Les motivations d’une telle opération sont plutôt internes aux cadres théoriques respectifs. D’abord, on doit dire que les premiers travaux proposant la \isi{reconstruction syntaxique} rencontrent des difficultés majeures quant à la notion de constituant : dans les constructions à gapping, on efface un élément ou une série d’éléments qui ne forme pas un syntagme. Pour éviter ce problème de la non-constituance, les \is{grammaire dérivationnelle}grammaires dérivationnelles proposent l’opération de \isi{mouvement} : on déplace des éléments, afin d’obtenir dans le site de l’ellipse un syntagme. Deuxièmement, dans un cadre théorique basé sur l’idée d’un homomorphisme entre la syntaxe et les autres niveaux linguistiques, postuler une opération de déplacement faciliterait la modélisation de la \isi{structure informationnelle} : si les éléments résiduels sont des \is{topique contrastif}topiques ou des \is{focus contrastif}focus contrastifs, ils doivent avoir accès à une position syntaxique spécifique, qui permet ensuite \is{effacement}l’effacement (ou la légitimation d’une structure vide) de tout matériel qui est donné (\textit{given}) dans le discours. Enfin, cette opération permet aux cadres théoriques en question d’obtenir un mécanisme uniforme pour d’autres types d’ellipse aussi (p.ex. \is{Verb Phrase Ellipsis (VPE)}VPE, \is{Sluicing}sluicing, etc.).

Si l’on suppose un déplacement des éléments résiduels hors du site de l’ellipse, on s’attend à ce que cette opération obéisse aux contraintes qui pèsent sur toute \isi{extraction}, en particulier aux \isi{contraintes de localité}, connues aussi sous le nom d’effets \is{contraintes d'îles}d’îles\footnote{
 Pour une discussion de ces îlots en lien avec l’ellipse, voir, entre autres, \citet{Ross1967} and \citet{Merchant2001,Merchant2004}.}. A première vue, la prédiction semble être valable : on affirme souvent que le gapping est une opération locale, qui est contrainte par les «~barrières syntaxiques~» \citep{Hankamer1973,Neijt1979,Chao1988,Johnson1996/2004,Coppock2001,Winkler2005}. Ceux qui ont une perspective holistique des îles considèrent que le gapping obéit à toutes les \isi{contraintes d'îles} observées (cf. \citealt{Ross1967,Neijt1979}), tandis que d’autres (plus récemment) observent que le gapping obéit au moins à certaines \isi{contraintes d'îles}, et en particulier aux \isi{îles propositionnelles}, {\cad} îles qui contiennent un domaine propositionnel enchâssé (cf. \citealt{Hartmann2000,Coppock2001}). D’après la typologie de \citet{Merchant2001}, les \isi{îles propositionnelles} qui nous intéressent ici sont : l’îlot sujet, l’îlot relatif et l’îlot circonstanciel. Si l’on envisage le déplacement des éléments résiduels à l’extérieur du site elliptique, on s’attend donc à ce que le gapping obéisse au moins à ces \isi{îles propositionnelles}, ce qui implique qu’on ne peut avoir comme élément résiduel un constituant enchâssé dans une de ces îles. Selon \citet{Hartmann2000}, \citet{Coppock2001} et en partie \citet{Repp2009}\footnote{
 Selon \citet{Repp2009}, le gapping ne respecte pas les contraintes de l’îlot relatif et de la Branche Gauche, en revanche il respecte les contraintes de l’îlot sujet phrastique et de l’îlot ajout. Elle finit par dire qu’il faudrait une analyse détaillée du comportement du gapping par rapport aux différents \is{contraintes d'îles}types d’îles.}, cette prédiction est valable dans les constructions à gapping : on ne peut pas extraire un élément résiduel hors d’une phrase sujet \REF{ch2:ex215a}, hors d’une phrase relative \REF{ch2:ex215b} ou hors d’un ajout circonstanciel \REF{ch2:ex215c}.

\ea \label{ch2:ex215}
\ea *\uline{That John hangs out with} Mary \uline{annoys} Suzy, and Suzy Laura. \citep{Coppock2001} \label{ch2:ex215a}
\ex *Some \uline{wanted to hire the woman who worked on} Greek, and others Albanian. \citep{Merchant2009} \label{ch2:ex215b}
\ex *John \uline{must be a fool to have married} Jane, and Bill, Martha. \citep{Coppock2001} \label{ch2:ex215c}
\z
\z

Cependant, on observe qu’un élément résiduel peut apparaître dans ce qui devrait être un îlot d’extraction en anglais, cf. \citet{CulicoverEtAl2005}. Ainsi, en \REF{ch2:ex216}, on a une violation de la contrainte de l’îlot ajout, alors qu’en \REF{ch2:ex217}, on viole la contrainte de l’îlot relatif.

\ea \label{ch2:ex216}
\ea Robin \uline{knows a lot of reasons why} dogs \uline{are good pets}, and Leslie, cats. \citep[273]{CulicoverEtAl2005}     
\ex Robin \uline{believes that everyone pays attention to you when you speak} French, and Leslie, German. \citep[273]{CulicoverEtAl2005}
\z
\z

\ea \label{ch2:ex217}
\ea In the past, \uline{it has been} the husband \uline{who has been} dominant and the wife passive. (brwn-21990, \citealt{Bilbiie2013a})
\ex Bo \uline{decided who is working} tomorrow, and Mia, the next day. \citep{Chaves2005}
\ex That’s not how I remember it at all. And yet he \uline{cannot be the one who’s} correct, and everyone else – millennia of people – wrong. (Hanya Yanagihara, \textit{A Little Life}, p. 321-322)\footnote{
 Je dois à Philip Miller cet exemple. La première phrase est le monologue intérieur du personnage, alors que la deuxième phrase est la voix du narrateur.}

\z
\z

L’élément résiduel peut violer les \isi{contraintes d'îles} en roumain et en français aussi. Je mentionne quelques exemples roumains dans lesquels un des éléments résiduels peut être le dépendant d’un verbe enchâssé dans un îlot sujet (au subjonctif ou à l’infinitif) en \REF{ch2:ex218}, dans un îlot relatif en \REF{ch2:ex219} ou dans un îlot circonstanciel en \REF{ch2:ex220}.

\ea \label{ch2:ex218}
\ea 
\gll \ulg{Să}{13}  înveţi  la  pian  \uline{e}  greu,  iar  la  vioară  şi  mai  greu.\\
\textsc{sbjv}  apprendre.\textsc{sbjv.2sg} à  piano  est  lourd  et  à  violon  aussi plus  lourd\\
\glt ‘Apprendre le piano est difficile, et le violon encore plus difficile.’

\ex 
\gll Mariei  \ulg{îi}{39.2}  place  să  meargă  la  mare,  iar  lui  Ion  la  munte.\\
Maria.\textsc{dat} \textsc{dat.3sg}  plaît  \textsc{sbjv} aller.\textsc{sbjv.3} à  mer  et  \textsc{dat}  Ion  à  montagne\\
\glt ‘Marie aime aller à la mer, et Ion à la montagne.’

\ex 
\gll \%\ulg{A}{12.4} merge  la  teatru  \uline{e}  pasiunea  Mariei,  iar  la  film  pasiunea  lui  Ion.\\
\textsc{inf}  aller  à  théâtre  est  passion.\textsc{def}  Maria.\textsc{dat}  et  à  film  pasion.\textsc{def}  \textsc{dat}  Ion\\
\glt ‘Aller au théâtre est la passion de Maria, et au cinéma la passion de Ion.’
\z
\z


\ea \label{ch2:ex219}
\ea
\gll Sunt  oameni \ulg{care}{14}  preferă singurătatea,  iar alţii, contrariul.\\
sont  gens  qui  préfèrent  solitude.\textsc{def}  et  d’autres  contraire.\textsc{def}\hspace*{-2mm}\\
\glt ‘Il y a des gens qui préfèrent la solitude, et d’autres, le contraire.’

\ex 
\gll In  viața  de  zi  cu  zi,  sunt  unii  \ulg{care}{8.3}  văd  partea  plină a  paharului, iar  alții  doar  jumătatea  goală.\\ 
dans vie.\textsc{def} de jour avec jour sont certains qui voient  partie.\textsc{def} pleine \textsc{gen} verre.\textsc{gen} et d’autres seulement  moitié.\textsc{def} vide\\
\glt ‘Dans la vie de tous les jours, il y a certains qui voient le verre à moitié plein, et d’autres le verre à moitié vide.’

\ex 
\gll \ulg{Chestia}{65.2}  care  îi  enervează  cel  mai  tare  pe  bărbați  \uline{este} critica,  iar  pe  femei  lipsa  de  atenție.\\ 
question.\textsc{def}  qui  \textsc{acc.3pl}  énerve  le  plus  fort  \textsc{dom} hommes  est critique.\textsc{def} et \textsc{dom} femmes manque.\textsc{def} de attention\\
\glt ‘L’aspect qui énerve le plus les hommes est la critique, et les femmes le manque d’attention.’

\ex 
\gll \ulg{Ceea}{61}  ce  o  motivează  cel  mai  bine  pe  o  femeie \uline{este} dragostea,  iar  pe  un  bărbat  respectul.\\
ce  qui  \textsc{acc.3sg.f}  motive  le  plus  bien  \textsc{dom}  une  femme  est amour.\textsc{def}  et  \textsc{dom}  un  homme  respect.\textsc{def}\\
\glt ‘Ce qui motive le plus une femme est l’amour, et un homme le respect.’
\z
\z


\ea \label{ch2:ex220}
\ea  
\gll \ulg{Dacă}{11.5}  veniți  cu  rulota,  \uline{plătiți}  25  de  lei  \ule{pe}  \ule{noapte},  iar cu  cortul,  15  lei.\\  
si  venir.\textsc{prs.2pl}  avec  caravane.\textsc{def}  payer.\textsc{prs.2pl}  25  de  lei  par  nuit  et avec  tente.\textsc{def}  15  lei\\
\glt ‘Si vous venez en caravane, vous payez 25 lei par nuit, et avec la tente, 15 lei.’  

\ex 
\gll Ion  \ulg{mănâncă}{19.3}  uitându-se  la  documentare,  iar  Ana  la  telenovele.\\
Ion  mange  regarder.\textsc{ptcp.prs-refl.3}  à  documentaires  et  Ana  à  feuilletons\\
\glt ‘Ion mange en regardant des documentaires, et Ana des feuilletons.’

\ex 
\gll Ion  \ulg{se}{50}  chinuie  încercând  să  înveţe  chineza,  iar  Ana  coreeana.\\
Ion  se  tracasse  essayant  \textsc{sbjv}  apprendre.\textsc{sbjv.3} chinois.\textsc{def}  et  Ana  coréen.\textsc{def}\\
\glt ‘Ion se donne du mal en essayant d’apprendre le chinois, et Ana le coréen.’

\ex 
\gll \ulg{Iți}{20.4}  trebuie  50~000  de  lei  \ulg{ca}{31}  să-ți  iei  o garsonieră,  iar  o  Dacie  1300,  cam  tot  atât.\\
\textsc{dat.2sg} faut  50~000  de  lei \textsc{comp} \textsc{sbjv-dat.2sg} prendre.\textsc{sbjv.2sg}  un studio  et  une  Dacia  1300  presque  même  tant\\
\glt ‘Il te faut 50~000 lei pour t’acheter un studio, et (pour) une Dacia 1300 presque pareil.’
\z
\z


Les mêmes observations s’appliquent au français, comme on peut voir dans les exemples en \REF{ch2:ex221} pour l’îlot sujet, \REF{ch2:ex222} pour l’îlot relatif et \REF{ch2:ex223} pour l’îlot circonstanciel.

\ea \label{ch2:ex221}
\ea \uline{Comprendre} le texte traduit \uline{c’est} laborieux et le texte original encore plus laborieux.
\ex \uline{Qu’on n’aime pas} le latin, \uline{je comprends} dans une certaine mesure, mais l’italien, pas du tout.
\z
\z

\ea \label{ch2:ex222}
\ea \uline{C’est} Paul \uline{qui fait} la vaisselle et Marie la lessive.
\ex Nous avons deux maîtresses, Roxane et Mathilde. Il y a aussi Katia \uline{qui nous fait} l’anglais et Philippe la musique. (planche, Musée de Vaujany)
\z
\z

\ea \label{ch2:ex223}
\ea \uline{Quand tu parles} chinois, tout le monde \uline{t’admire}, mais anglais personne.
\ex \uline{Si vous venez} en train, \uline{vous mettrez} 4 heures et en voiture, seulement 2 heures.
\z
\z

Toutes ces données montrent qu’il y a des îles qui sont violées dans le gapping, et cela indépendamment de la classification proposée par \citet{Merchant2001}. Ainsi, l’extraction des éléments résiduels à la périphérie gauche du conjoint ne se justifie pas empiriquement. De plus, \citet[463]{Culicover2009} ajoute le fait que cette \isi{extraction} «~multiple~» n’est pas motivée en dehors des constructions elliptiques, car l’anglais ne permet pas la topicalisation multiple. Par conséquent, l’\isi{effacement} ou toute autre opération supposée affecte un élément ou une suite d’éléments qui ne forment pas un constituant.

Cette discussion sur les \is{contraintes d'îles}effets d’îles dans les structures elliptiques nécessiterait une étude approfondie concernant la nature exacte de ces contraintes, ce qui dépasse largement mon objet d’étude. La motivation pour une telle étude vient des différences qu’on observe quant à l’acceptabilité des violations de ces contraintes : \is{extraction}l’extraction hors d’une même île est considérée possible dans certains exemples, mais inacceptable dans d’autres occurrences. Les \is{grammaire dérivationnelle}grammaires dérivationnelles (dans lesquelles s’inscrivent les analyses dominantes proposées pour le gapping) proposent une approche syntaxique des îles, les effets observés étant dus à des contraintes de compétence (cf. \citealt{Ross1967}). Mais ce type d’approche ne peut pas expliquer les différences d’acceptabilité qu’on observe avec une même île. 

La conclusion de plusieurs travaux (\citealt{Hankamer1973,Kuno1976,Sag1976,SagEtAl1985,Gardent1991}, etc.) est que bien des contraintes qui jouent sur l’interprétation des exemples à gapping sont de nature non syntaxique. Ainsi, le fait que certaines îles soient respectées pourrait ne pas être une question de grammaticalité dans ce type spécifique de constructions, mais plutôt une question d’accessibilité ({\cad} \is{extraction}l’extraction hors de ces îles n’est pas agrammaticale, mais simplement elle n’est pas préférée pour d’autres raisons) ; la sensibilité aux îles devrait donc être expliquée en termes de facteurs psycholinguistiques. Notons que, indépendamment de l’étude de l’ellipse, des travaux récents ont montré que, parmi les facteurs non syntaxiques qui gèrent l’acceptabilité des exemples avec îles, un rôle très important revient aux facteurs psycholinguistiques (voir, par exemple, \citealt{FanselowEtAl2006} : «~\is{processing}Processing difficulty can make grammatical sentences unacceptable.~»), mais aussi à d’autres types de facteurs (discursifs, prosodiques, etc.). Pour une approche non syntaxique des \isi{contraintes d'îles}, voir \citet{Kluender1998}, \citet{FanselowEtAl2006}, \citet{AmbridgeEtAl2008}, \citet{HofmeisterEtAl2010}, etc.      


\subsubsection{Problèmes de l’extraction du matériel manquant} \label{ch2:sect2.4.3.2}

On a vu dans les sections~\ref{ch2:sect2.4.1.2} et~\ref{ch2:sect2.4.2.3} que la principale motivation pour une approche à la \citet{Johnson1996/2004,Johnson2000,Johnson2009} est d’ordre sémantique. Une analyse qui place la coordination à un niveau sous-phrastique d’où on a extrait le matériel manquant permettrait de rendre compte du fait que certains opérateurs ({\cad} la \isi{négation}, les modaux, certains quantifieurs) dans la phrase source peuvent avoir \isi{portée large} sur toute la coordination. 

Cependant, ce type d’analyse présente plusieurs difficultés, que j’ai regroupées en deux parties : des problèmes plutôt internes à ce type d’approche, et ensuite des problèmes empiriques, qui sont indépendants de tout cadre théorique. 

Selon \citet{Vicente2010}, le \isi{mouvement} du verbe est problématique dans le cas des trous complexes, où on devrait déplacer non seulement la tête verbale, mais aussi d’autres constituants. Johnson propose une solution à ce problème ({{\cad}} \textit{remnant predicate movement}), mais elle n’est pas empiriquement adéquate, car on ne peut pas l’appliquer à d’autres types de déplacement enregistrés en anglais en \REF{ch2:ex224} et \REF{ch2:ex225}. Ainsi, les trous complexes, qui subissent un déplacement à la Johnson, ne peuvent pas être antéposés (\ref{ch2:ex224b}--\ref{ch2:ex225b}) :

\ea \label{ch2:ex224}
\ea Phil \uline{read things} quickly, and Mike thoroughly.
\ex *Read things, Mike (did) quickly. \citep[510]{Vicente2010} \label{ch2:ex224b} 
\z
\z

\ea \label{ch2:ex225}
\ea Randy \uline{wants to write} a novel, and Amy a play.
\ex  *Want to write, Randy (did) a novel. \citep[510]{Vicente2010} \label{ch2:ex225b}
\z
\z

Comme \citet{Johnson2009} le précise lui-même, le \isi{mouvement} des trous complexes nécessite plusieurs déplacements. Ainsi, pour obtenir l’exemple \REF{ch2:ex226a} avec gapping, on extrait la séquence \textit{give me} de la phrase trouée à travers son sujet, ce qui veut dire qu’on doit permettre au sujet d’intervenir entre les deux compléments d’objet du verbe \textit{give}. Si ce type de déplacement est permis dans les constructions à gapping en anglais, pourquoi ne trouve-t-on pas de matériel intervenant entre les deux compléments du verbe \textit{give} en anglais \REF{ch2:ex226b} ? On observe donc que \is{extraction}l’extraction du matériel manquant intéragit avec la \isi{linéarisation} des éléments d’une manière inattendue : ce type d’opération permet des \is{ordre de mots}ordres de mots qui ne sont pas attestés par ailleurs dans~la langue\footnote{
 \citet{Johnson2009} considère que c’est un problème aussi pour les approches à la \citet{Coppock2001}, utilisant \is{effacement}l’effacement.}.

\ea \label{ch2:ex226}
\ea Ice cream \uline{gives me} brain-freeze if I eat it too fast and beans \st{give me} indigestion if I eat them too slow. \label{ch2:ex226a}
\ex *Ice cream gives me in the morning brain-freeze. \citep[314]{Johnson2009} \label{ch2:ex226b}
\z
\z

De plus, \is{extraction}l’extraction verbale exige que les deux \textit{v}P (de la phrase source et de la phrase trouée) soient identiques. Cela pose un problème pour l’une des trois possibilités d’interprétation de la \isi{négation} dans les constructions à gapping (voir \citealt{Repp2009}). Si cette \isi{approche structurale} peut rendre compte de la \isi{portée large} de la \isi{négation} ({\cad} la \isi{négation} est interprétée à un niveau supérieur par rapport à la coordination) et aussi des cas où la \isi{négation} est interprétée à l’intérieur de chaque conjoint (étant distribuée sur les deux conjoints), on n’arrive toujours pas à obtenir la \isi{portée étroite} de la \isi{négation} ({\cad} les cas où la \isi{négation} s’interprète uniquement dans la phrase source, cf. les exemples (\ref{ch2:ex89}--\ref{ch2:ex90}) ci-dessus). 

On y ajoute deux faits empiriques montrant que ce type d’approche n’est pas adéquate en roumain et en français. Un premier point faible de cette analyse est le fait qu’elle prédit de manière incorrecte la distribution des items corrélatifs. En roumain et en français, si une coordination de phrases présente des items corrélatifs (\is{coordination omnisyndétique (ou corrélative)}conjonctions corrélatives, p.ex. \textit{fie...fie...} ‘soit...soit...’ en roumain, \textit{ou bien...ou bien...} ou \textit{ni...ni...} en français ; adverbes corrélatifs, p.ex. \textit{nici...nici...} ‘ni... ni...’ en roumain), chaque conjoint doit être introduit par un corrélatif \citep{Bilbiie2008,Mouret2007}, cf. \REF{ch2:ex227a} et \REF{ch2:ex228a} en roumain et \REF{ch2:ex229a} et \REF{ch2:ex230a} en français. Dans l’approche de \ia{Johnson, Kyle}Johnson, le premier élément corrélat dans la phrase source (habituellement, le sujet) est extrait de manière asymétrique hors du premier conjoint. Cela devrait permettre l’occurrence d’un item corrélatif après la tête verbale dans la phrase source, ce qui s’avère être agrammatical en \REF{ch2:ex227b} et \REF{ch2:ex228b} en roumain et \REF{ch2:ex229b} et \REF{ch2:ex230b} en français.  

\ea
\ea \textbf{Fie} Dan \uline{va cânta} la vioară, \textbf{fie} Maria la pian. \label{ch2:ex227a}
\glt ‘Soit Dan va jouer du violon, soit Maria du piano.’   

\ex 
\gll *Dan  \ulg{va}{11.2}  cânta  \textbf{fie} la  vioară,  \textbf{fie}  Maria  la  pian. \label{ch2:ex227b}\\
Dan  va  chanter  soit  à  violon  soit  Maria  à  piano\\
\glt ‘Soit Dan va jouer du violon, soit Maria du piano.’
\z
\z


\ea
\ea 
\gll \textbf{Nici}  directorul  \ulg{nu}{22}  are  obligaţii  faţă  de  mine,  şi  \textbf{nici}  eu  faţă  de  el. \label{ch2:ex228a}\\
ni  directeur.\textsc{def} \textsc{neg}  a  obligations  face  de  moi  et  ni  moi  face  de  lui\\
\glt ‘Ni le directeur n’a d’obligations envers moi, ni moi envers lui.’  

\ex  
\gll *Directorul  \ulg{nu}{22}  are  obligaţii  \textbf{nici}  faţă  de  mine,  şi  \textbf{nici}  eu  faţă  de  el. \label{ch2:ex228b}\\
directeur.\textsc{def} \textsc{neg}  a  obligations  ni  face  de  moi  et  ni  moi  face  de  lui\\
\glt ‘Ni le directeur n’a d’obligations envers moi, ni moi envers lui.’
\z
\z


\ea
\ea \textbf{Ou bien} Paul \uline{dormira} chez Marie \textbf{ou bien} Marie chez Paul. \label{ch2:ex229a} 
\ex *Paul \uline{dormira} \textbf{ou bien} chez Marie \textbf{ou bien} Marie chez Paul.   
\z \label{ch2:ex229b}
\z

\ea
\ea \textbf{Ni} le compromis \uline{ne me paraît} justifié, \textbf{ni} l’acceptation pure et simple nécessaire. (exemple cité par \citealt{GrevisseEtAl1991}) \label{ch2:ex230a}     
\ex *Le compromis \uline{ne me paraît} \textbf{ni} justifié, \textbf{ni} l’acceptation pure et simple nécessaire. \label{ch2:ex230b}                   
\z
\z

De plus, cette analyse prédit incorrectement que la conjonction \textit{iar} (utilisée massivement dans les constructions à gapping en roumain) relie des éléments sous-phrastiques, alors qu’il est communément admis que cette conjonction coordonne uniquement des contenus propositionnels ({\cad} des phrases, cf. \citealt{BilbiieEtAl2011}). 

Par conséquent, ce type d’approche ne peut pas s’appliquer au roumain et au français. Quant à la motivation d’une coordination «~basse~» (notamment la \isi{portée large} des opérateurs sémantiques), je précise que les problèmes relevés par Johnson sont de nature sémantique et non syntaxique et peuvent trouver une solution convenable dans un cadre théorique qui ne pose pas d’homomorphisme syntaxe-sémantique\footnote{
 La solution se baserait sur l’asymétrie qu’on observe entre la coordination ordinaire (sans ellipse) et la coordination à gapping. La coordination ordinaire ne permet pas à un élément issu d’un des conjoints d’avoir \isi{portée large} sur toute la coordination, les contraintes de portée étant donc très strictes. En revanche, dans les coordinations à gapping, on peut relâcher ces contraintes et autoriser donc un élément du premier conjoint à prendre \isi{portée large}, à condition que cela fasse sens sémantiquement : p.ex. un quantifieur qui lie une variable dans le deuxième conjoint ou bien un ajout adverbial comme la \isi{négation} ou d’autres adverbes. Cela est possible dans un cadre comme HPSG, qui peut utiliser le langage \is{Minimal Recursion Semantics (MRS)}\textit{Minimal Recursion Semantics} pour la sous-spécification syntaxique de la portée des quantifieurs.}. 


\subsubsection{Problèmes de l’effacement} \label{ch2:sect2.4.3.3}

Dans les analyses qui postulent une \isi{reconstruction syntaxique} du verbe à l’endroit même du trou, \is{effacement}l’effacement ou autre opération envisagée a lieu sous une condition d’identité entre le matériel manquant et~le matériel antécédent. Par conséquent, toute discordance qui apparaît entre les deux entraîne des difficultés supplémentaires qui obligent la théorie en question à faire appel à des stipulations parfois coûteuses.

Dans les constructions à gapping, on observe que le matériel manquant qui doit être reconstruit dans la phrase trouée ne correspond pas toujours à une copie du matériel antécédent dans la phrase source. La condition d’identité doit donc être remaniée afin de prendre en compte les différentes asymétries qu’on observe.

Un fait bien connu et discuté pour l’anglais aussi est l’absence d’identité stricte entre les deux verbes par rapport aux marques \is{accord}d’accord (différence en nombre ou/et en personne, cf. \REF{ch2:ex231} en roumain et \REF{ch2:ex232} en français). Ces données peuvent trouver néanmoins une solution dans les versions récentes de \is{effacement}l’effacement (voir \citealt{BeaversEtAl2004} and \citealt{ChavesEtAl2008}). 

\ea \label{ch2:ex231}
\ea 
\gll Noi  \uline{citim}  o  carte,  iar  tu  (\{citeşti {\textbar} *citim\})  un  ziar.\\
nous  lisons  un  livre  et  tu  (\{lis {\textbar} lisons\})  un  journal\\
\glt ‘Nous lisons un livre, et toi un journal.’

\ex 
\gll Eu  \uline{iubesc}  animalele,  iar  Ana  (\{iubeşte {\textbar} *iubesc\})  florile.\\
\textsc{nom.1sg}  aime.\textsc{1sg}  animaux.\textsc{def}  et  Ana  (\{aime.\textsc{3sg} {\textbar} aime.\textsc{1sg\}})  fleurs.\textsc{def}\\
\glt ‘J’aime les animaux et Ana les fleurs.’
\z
\z


\ea \label{ch2:ex232}
\ea Paul \uline{va} à Paris et ses enfants (\{vont {\textbar} *va\}) à Rome.      
\ex Nos enfants \uline{partent} demain et nous (\{partons {\textbar} *part\}) la semaine prochaine.                    
\z
\z

Pour le roumain et le français, le manque d’identité est encore plus aigu quand on prend en compte le comportement des clitiques. Les \is{affixe/clitique pronominal}clitiques pronominaux affixés au verbe antécédent ne sont pas nécessairement les mêmes que ceux qui seraient affixés au verbe reconstruit dans la phrase trouée, cf. \REF{ch2:ex233} en roumain et \REF{ch2:ex234} en français. 

\ea \label{ch2:ex233}
\ea 
\gll Ion \textbf{l}{}-\ulg{a}{19} împins pe Dan, iar Dan (\{a  împins-\textbf{o} {\textbar} \textbf{*l}{}-a  împins\}) pe Maria.\\
Ion \textsc{3sg.m}-a  poussé \textsc{dom} Dan et Dan (\{a  poussé-\textsc{3sg.f} {\textbar} \textsc{3sg.m}-a  poussé\}) \textsc{dom} Maria\\
\glt ‘Ion a poussé Dan, et Dan (a poussé) Maria.’

\ex 
\gll Eu \textbf{i}{}-\ulg{am}{15} văzut pe [Ion şi Maria], iar Ana (\{\textbf{l}{}-a  văzut {\textbar} \textbf{*i}{}-a văzut\}) pe Paul.\\
je \textsc{3pl.m}-ai vu \textsc{dom} Ion et Maria et Ana (\{\textsc{3sg.m}{}-a vu {\textbar} \textsc{3pl.m}{}-a vu\}) \textsc{dom} Paul\\
\glt ‘J’ai vu Ion et Maria, et Ana (a vu) Paul.’ 
\z
\z


\ea \label{ch2:ex234}
\ea Paul \textbf{en} \uline{a lu} seulement certains, mais Marie (\{\textbf{les} {\textbar} \textbf{*en}\} a lu) presque tous.   
\ex Paul \textbf{les} \uline{a lus}, vos livres, et Marie (\{\textbf{en} {\textbar} \textbf{*les}\} a lu) seulement certains.  
\z
\z

De plus, certains clitiques (\is{affixe/clitique pronominal}pronominaux ou \is{affixe/clitique adverbial}adverbiaux) sont interdits dans l’un des conjoints, mais obligatoires dans l’autre. Ainsi, en \REF{ch2:ex235a} le clitique pronominal \textit{le} en roumain est obligatoire dans la source, mais interdit dans la phrase trouée reconstruite ; de même, dans l’exemple français \REF{ch2:ex236a}, le clitique pronominal \textit{en} n’apparaît pas dans la source, mais il apparaît dans la phrase trouée reconstruite. Dans les deux langues, la présence d’un mot négatif comme élément résiduel dans la phrase trouée (\textit{niciuna} ‘aucune’ en \REF{ch2:ex235b} et \textit{aucun} en \REF{ch2:ex236b}) entraîne l’emploi du clitique adverbial de \isi{négation} sur le verbe reconstruit, alors qu’il est absent dans la phrase source. Enfin, dans l’exemple \REF{ch2:ex236c} en français, on observe que l’élément résiduel \textit{moi} ne peut fonctionner comme sujet d’un verbe reconstruit ; si l’on reconstruit un verbe, le pronom fort \textit{moi} doit être redoublé par le clitique pronominal \textit{je}.

\ea
\ea 
\gll Ion  \textbf{le}{}-\ulg{a}{11.8}  citit  pe  toate,  dar  Ana  ((*\textbf{le}{}-)a  citit)  doar  câteva. \label{ch2:ex235a}\\
Ion \textsc{3pl.f}-a lu \textsc{dom} toutes  mais  Ana  (\textsc{3pl.f}-a  lu)  seulement  quelques-unes\\
\glt ‘Ion les a tous lus, mais Ana seulement quelques-uns.’

\ex  
\gll Ion \ulg{a}{8.8} citit  câteva  dintre  ele,  dar  Maria  (*(\textbf{nu})  a  citit)  absolut niciuna. \label{ch2:ex235b}\\
Ion  a  lu  quelques-unes  parmi  elles  mais  Maria  (\textsc{neg} a  lu)  absolument aucune\\
\glt ‘Ion en a lu quelques-uns, mais Maria absolument aucun.’ 
\z
\z


\ea
\ea Paul \uline{a lu} tous vos livres et Marie (\textbf{en} a lu) quelques-uns. \label{ch2:ex236a}      
\ex Paul \uline{en a lu} peu, et Marie (*(\textbf{n}’)en a lu) absolument aucun. \label{ch2:ex236b}
\ex Marie \uline{aime} les pommes et moi (*(\textbf{j}’)aime) les oranges. \label{ch2:ex236c}
\z
\z

On observe le même problème avec les adverbes restrictifs \textit{decât} ‘que’ et \textit{doar} ‘seulement’ en \REF{ch2:ex237}. Pour marquer la restriction, le roumain utilise l’adverbe \textit{decât} ‘que’ dans les contextes négatifs et \textit{doar} ‘seulement’ dans les contextes positifs. L’adverbe \textit{decât} est licite uniquement s’il suit un verbe nié. Si l’on assume une théorie à base \is{effacement}d’effacement, on devrait pouvoir utiliser \textit{decât} dans la phrase trouée aussi \REF{ch2:ex237a}, car son légitimeur ({\cad} le verbe nié) serait présent dans la structure syntaxique. Or, les locuteurs ont une préférence nette pour l’emploi de \textit{doar} (qui ne demande pas de \isi{négation} et pas de verbe non plus) dans la phrase avec gapping \REF{ch2:ex237b}. 

\ea \label{ch2:ex237}
\ea ??Ion \uline{nu ştie} \textbf{decât} engleza, iar Maria \textbf{decât} germana. \label{ch2:ex237a}
\glt ‘Ion ne parle que l’anglais, et Maria que l’allemand.’

\ex Ion \uline{nu ştie} \textbf{decât} engleza, iar Maria \textbf{doar} germana. \label{ch2:ex237b}
\glt ‘Ion ne parle que l’anglais, et Maria seulement l’allemand.' 
\z
\z

Si ces problèmes de discordance peuvent trouver une solution adéquate (bien que très coûteuse), il y a un autre fait qui, à ma connaissance, ne peut être pris en compte par une approche syntaxique à base d’effacement. Selon le \isi{principe de récupérabilité de l’ellipse} \citep{Chomsky1964}, une forme linguistique non réalisée phonologiquement doit pouvoir être insérée in situ. Or, dans les constructions à gapping, la reconstruction du matériel supposé effacé ne donne pas toujours lieu à une phrase grammaticale. Comme l’ont observé \citet{CulicoverEtAl2005} en \REF{ch2:ex238}, il y a des éléments (comme les connecteurs \textit{as well as} and \textit{and/but not} en anglais) qui apparaissent dans une phrase trouée, mais qui ne se combinent pas avec une phrase finie, ce qui est attendu si le verbe manquant est effectivement absent de la structure. Ce type de connecteurs a le même comportement en roumain \REF{ch2:ex239} et en français \REF{ch2:ex240} : on ne peut pas avoir un verbe fini dans la phrase trouée si celle-ci est introduite par la conjonction lexicalisée \textit{precum şi} ‘ainsi que’ \REF{ch2:ex239a} en roumain ou \textit{ainsi que} \REF{ch2:ex240a} en français, ou si la conjonction habituelle est immédiatement suivie par la \isi{négation} de constituant \textit{nu}\footnote{
 On doit distinguer entre trois \textit{nu} différents en roumain : \REF{ch2:foot74i} l’adverbe \isi{négation} de constituant, \REF{ch2:foot74ii} le \is{affixe/clitique adverbial}clitique adverbial \isi{négation} de phrase, qui est affixé au verbe, et \REF{ch2:foot74iii} l’adverbe pro-phrase. Pour une discussion sur leurs propriétés différentes, voir \citet{Barbu2004}, \citet{Ionescu2003}, etc.
\ea Lupul îşi schimbă părul, dar \textbf{nu} năravul. \label{ch2:foot74i}
\glt ‘Le loup change son pelage, mais pas son instinct.’
\z

\ea Lupul îşi schimbă părul, dar \textbf{nu}{}-şi schimbă năravul. \label{ch2:foot74ii}
\glt ‘Le loup change son pelage, mais ne change pas son instinct.’
\z

\ea Lupul îşi schimbă părul, dar năravul, \textbf{nu}. \label{ch2:foot74iii}
\glt ‘Le loup change son pelage, mais son instinct, non.’
\z
} \REF{ch2:ex239b} en roumain ou \textit{non pas} \REF{ch2:ex240b} en français. De même, pour les connecteurs \is{comparatives}comparatifs \textit{ca şi}, \textit{la fel ca} ‘comme’ \REF{ch2:ex239c} en roumain, qui n’introduisent jamais une phrase finie. 


\ea \label{ch2:ex238}
\ea Robin \uline{speaks} French \textbf{as well as} Leslie (*speaks) German.   
\ex Robin \uline{speaks} French \textbf{and not} Leslie (*speaks) German.  
\z
\z

\ea \label{ch2:ex239}
\ea Istoria veche a egiptenilor \uline{mă pasionează} dintotdeauna, \textbf{precum şi} cea aztecă (*mă pasionează) de ceva vreme. \label{ch2:ex239a}
\glt ‘L’histoire ancienne des Egyptiens me passionne depuis toujours, ainsi que celle des Aztèques depuis un bon moment.’  

\ex DAN \uline{va dormi} la Maria \textbf{şi nu} ea (*va dormi) la el. \label{ch2:ex239b}
\glt ‘Dan va dormir chez Maria, et non pas elle chez lui.’

\ex Ion \uline{se comportă} cu Maria \{\textbf{ca şi {\textbar} la fel ca}\} fratele lui (*se comportă) cu Ana. \label{ch2:ex239c}
\glt ‘Ion se comporte avec Maria comme son frère avec Ana.’
\z
\z


\ea \label{ch2:ex240}
\ea Paul \uline{a cueilli} des framboises \textbf{ainsi que} Marie (*a cueilli) des fraises. \label{ch2:ex240a}  
\ex Paul \uline{dormira} chez Marie et \textbf{non pas} Marie (*dormira) chez Paul. \label{ch2:ex240b}  
\z
\z

Un problème interne aux théories présupposant l’homomorphisme syntaxe-sémantique concerne la portée de certains opérateurs sémantiques. Comme le note Johnson, la \isi{négation}, les modaux ou encore certains quantifieurs peuvent avoir une \isi{portée large} sur toute la coordination (voir les données ci-dessus en (\ref{ch2:ex190}--\ref{ch2:ex192})). Or, dans une théorie à base \is{effacement}d’effacement, ces opérateurs apparaissent dans chaque conjoint, donc la \isi{portée large} ne peut pas avoir lieu. \citet{Coppock2001} propose une solution à ce problème : la coordination a lieu non au niveau de la phrase, mais au niveau du \textit{v}P. Cependant, les arguments empiriques mentionnés dans la section précédente invalident cette possibilité. Par conséquent, la \is{portée large}portée~large de ces opérateurs reste un problème.

Un problème plus général, qui concerne et \is{effacement}l’effacement et le \isi{mouvement} du verbe, est lié à la portée de la \isi{négation} dans le gapping. Les données sur la \isi{négation} ont tourné d’une analyse à l’autre, chacune choisissant celles qui étaient en sa faveur : ainsi, \is{effacement}l’effacement est adapté pour rendre compte de la portée distribuée de la \isi{négation}, alors que le \is{Across-The-Board (ATB) Movement}mouvement ATB est adapté pour la \isi{portée large}. Cependant, et l’effacement et le mouvement ATB sont loin d’avoir un traitement complet des trois interprétations de la \isi{négation} à l’intérieur d’une même théorie, en particulier elles ne captent pas la \isi{portée étroite} de la \isi{négation}, quand la \isi{polarité} est différente dans les deux conjoints : négative vs. positive ou bien positive vs. négative \citep{Repp2009}.

Vu les problèmes empiriques mentionnés dans cette section, il s’avère très difficile de maintenir une approche en termes \is{effacement}d’effacement ou de \isi{mouvement} du verbe, combiné éventuellement avec une \isi{extraction} des éléments résiduels. Je propose donc d’abandonner ce type d’approches pour les constructions à gapping et d’adopter une solution qui ne postule pas de structure syntaxique pour le matériel manquant. 


\subsection{Analyses alternatives : une {approche non structurale}} \label{ch2:sect2.4.4}

Les problèmes discutés dans la section précédente peuvent trouver une solution dans une \is{approche constructionnelle}approche à base de constructions, qui postule la récupération du matériel antécédent, sans qu’il soit présent (sous une forme ou autre) dans la structure syntaxique de la phrase trouée. Dans cette perspective, le trou n’a pas de représentation syntaxique (pas \is{effacement}d’effacement, pas \is{élément vide}d’élément vide, pas de \isi{mouvement}). Par conséquent, la séquence trouée sera une suite de deux ou plusieurs syntagmes sans tête verbale, avec un contenu propositionnel similaire à celui de la phrase source. On appelera cette séquence une \is{fragment}\textit{phrase fragmentaire}.

Parmi les analyses proposées dans cette perspective, on peut citer les travaux de \citet{SagEtAl1985} en \is{Generalized Phrase Structure Grammar (GPSG)}GPSG (\textit{Generalized Phrase Structure Grammar}), \citet{Steedman1990,Steedman2000}, \citet{Gardent1991} et \citet{Hoyt2008} en \is{Combinatory Categorial Grammar (CCG)}CCG (\textit{Combinatory Categorial Grammar}), \citet{CulicoverEtAl2005} et \citet{Culicover2009} dans leur projet d’une syntaxe plus simple (\textit{Simpler Syntax})\footnote{
 A la liste des analyses proposées pour le gapping, je devrais ajouter aussi les analyses à base de «~partage~» (\textit{sharing} \textit{approach}), cf. \citet{Goodall1987}, \citet{Moltmann1992}, etc. L’idée générale est que la phrase source et la phrase trouée sont projetées dans le même arbre (un seul nœud S/IP). Dans cet arbre, le matériel qui apparaît uniquement dans la phrase source est partagé littéralement par les deux conjoints. En revanche, les \is{paire contrastive}paires contrastives ne sont pas partagées. On distingue trois versions : (i) Les \is{paire contrastive}paires contrastives apparaissent sous le même nœud comme une liste ordonnée ({\cad} théorie de la factorisation). (ii) Les éléments résiduels apparaissent dans l’arbre comme s’ils créaient une structure bidimensionnelle. (iii) Le matériel partagé est présent simultanément dans les deux conjoints, grâce à la possibilité d’avoir un nœud dominé par plusieurs nœuds-mère ({\cad} théorie de la dominance multiple).}. 

Par la suite, je présente brièvement les analyses de \citet{Steedman1990,Steedman2000}, \citet{SagEtAl1985}, \citet{Gardent1991} et \citet{CulicoverEtAl2005}, vu le fait que l’analyse qu’on retiendra pour les constructions à gapping en roumain et en français (dans la section~\ref{ch2:sect2.5}) a beaucoup de points en commun avec ces approches. 

Les travaux sur l’ellipse faits en \isi{grammaire catégorielle} \citep{Dowty1988,Steedman1990,Steedman2000} ont l’avantage d’offrir une analyse assez aisée des coordinations de non-constituants, car ce type de grammaire permet une extension de la notion de constituance grâce à des règles combinatoires flexibles (p.ex. on peut combiner le verbe soit avec l’objet, soit avec le sujet). Le résultat est que, contrairement aux grammaires syntagmatiques, dans une \isi{grammaire catégorielle} les règles combinatoires permettent à toute séquence de non-constituants de fonctionner comme un constituant ordinaire, dans la portée syntaxique d’un prédicat extérieur à la structure. Si cela s’applique directement à des constructions comme \is{Argument Cluster Coordination (ACC)}ACC ou \is{Right-Node Raising (RNR)}RNR, pour le gapping cela nécessite des modifications de la grammaire. 

Les deux processus majeurs envisagés en \isi{grammaire catégorielle} sont la montée de type et la composition fonctionnelle. Pour toute séquence de trois constituants, ces règles permettent de déterminer le troisième, à partir des deux premiers. Ce qui est particulier au gapping, par rapport à d’autres coordinations de non-constituants, est l’introduction d’une règle spéciale pour l’ellipse, qui permet la décomposition de la phrase source, afin de pouvoir isoler le prédicat et obtenir une catégorie fonctionnelle similaire à celle obtenue pour la phrase trouée. J’illustre la représentation syntaxique du gapping en \REF{ch2:ex241} et \REF{ch2:ex242}. On commence par la composition fonctionnelle de la séquence trouée \REF{ch2:ex241}. On obtient une catégorie complexe qu’on ne peut pas combiner telle quelle avec la catégorie phrastique (S) de la phrase source. Ce qui oblige à postuler une règle de décomposition dans la phrase source \REF{ch2:ex242} ; c’est une régle «~révélatrice~» qui s’applique au verbe antécédent (ayant le statut discursif de «~donné~»), afin de récupérer un constituant «~révélateur~» qui va contribuer à déterminer l’interprétation de la séquence trouée. Après avoir mis de côté le verbe «~topique~», on obtient ainsi un constituant ayant la même catégorie fonctionnelle que celle obtenue par la composition de la séquence trouée. Ayant deux constituants de même catégorie, on peut maintenant les coordonner. Le résultat obtenu est finalement appliqué au verbe «~topique~», qu’on avait séparé lorsqu’on avait appliqué la première règle.

\ea \label{ch2:ex241}
Composition de la phrase trouée\\
\begin{tabbing}
Harry eats beans \= conj~~~ \= S/(S/NP) \kill
Harry eats beans \> and \> Barry~~~~~~~~~~~~ potatoes\\
\> ------ \> -------->T~~ ---------------------------<T\\
\> conj \> S/(S{\textbackslash}NP) ~ (S{\textbackslash}NP){\textbackslash}((S{\textbackslash}NP)/NP)\\            
\> ----------------------->\&\\
\>   [S/(S{\textbackslash}NP)]\&\\
\>  ------------------------------------------------------>Bx\\
\> \> [S{\textbackslash}((S{\textbackslash}NP)/NP)]\& 
\end{tabbing}

\z 

\ea \label{ch2:ex242}
Décomposition de la phrase source\\ 
\begin{tabbing}
Harry eats  \= beans \= <decompose  \= and \= Barry potatoes \kill
Harry eats  \> beans \>             \>and Barry potatoes\\
-----------------------------\>\>   \>------------------------------\\
            \> S      \>            \>  [S{\textbackslash}((S{\textbackslash}NP)/NP)]\&\\ 
-----------------------------<decompose\\
(S{\textbackslash}NP)/NP S{\textbackslash}((S{\textbackslash}NP)/NP)\\
            \>    ---------------------------------------------------------------<\&\\ 
            \> \hspace*{15mm}S{\textbackslash}((S{\textbackslash}NP)/NP)\\
-------------------------------------------------------<\\
            \> \> S
\end{tabbing}
\z 

\citet{Gardent1991} considère que la proposition de \citet{Steedman1990} pose quel\-ques problèmes : (i) linguistiquement, elle ne permet pas de prédire quelles combinaisons sont acceptées ou refusées par la grammaire ; (ii) computationnellement, elle permet des structures qui sont distinctes du point de vue de leur dérivation, mais qui sont équivalentes au niveau sémantique ; (iii) empiriquement, la catégorie fonctionnelle obtenue est inadéquate, car elle ne rend pas compte des cas où ce qui manque est, à part le verbe, aussi un syntagme nominal ou un syntagme prépositionnel. 

Par conséquent, \citet{Gardent1991} propose une solution, en se basant sur le travail de \citet{SagEtAl1985}. Ces derniers analysent la catégorie de la séquence trouée comme une variable sur une ou plusieurs catégories de type X\textsuperscript{2*}. Cette proposition est adoptée par \citet{Gardent1991}, qui y ajoute la notion de \textit{product category}, {\cad} une séquence de catégories qui fonctionne comme une catégorie complexe. La légitimation du \isi{fragment} se fait à l’interface syntaxe-sémantique. Fondamentalement, la récupération de l’information met en jeu, dans les deux approches, un mécanisme de \is{substitution}\textit{substitution}, définie sur les arbres syntaxiques, cf. \figref{ch2:fig1} et \figref{ch2:fig2}. Grossièrement, une phrase trouée est légitimée dans la grammaire si et seulement si la séquence de catégories formant une catégorie complexe peut être \textit{substituée} dans l’arbre de dérivation de la phrase source. La reconstruction de l’arbre est donc un mécanisme essentiel pour l’analyse du gapping, car elle assure l’interprétabilité d’une phrase et elle vérifie aussi la grammaticalité de la phrase trouée.

\begin{figure} 
\raisebox{13mm}{\footnotesize substitute}
\small
$\left(
\parbox{.35\textwidth}{\hrule height 0pt width 0pt
\vspace{0pt}
\begin{forest} 
[{S,~[NP(Paul),~NP(Sarah)]},
  [NP\\Jon]
    [S/NP
      [(S/NP)/NP\\likes] [NP\\Mary]
    ]
]
\end{forest}
}
\right)
$
\raisebox{13mm}{\LARGE=}
\parbox{.3\textwidth}{\hrule height 0pt width 0pt
\small
\vspace{0pt}
\begin{forest} 
[S 
  [NP\\Paul]
    [S/NP
      [(S/NP)/NP\\likes] [NP\\Sarah]
    ]
]
\end{forest}
}
\caption{Représentation d’une construction à gapping dans \citet{Gardent1991}}
\label{ch2:fig1}
\end{figure}


\begin{figure} 

\begin{forest} baseline
[S, s sep=15mm [S, s sep=10mm [NP, name=TerryNP] [VP [V\textsuperscript{0}, l=20mm [likes] ] [NP,name=StacyNP, l=20mm [Stacy,name=Stacy] ] ] ] [S, s sep=10mm [and] [NP,name=TracyNP [Tracy,name=Tracy] ] [NP,name=LeeNP [Lee,name=Lee] ] ] ]
\node [draw, circle, inner sep=-4pt, fit=(TracyNP) (Tracy)] (Tracygroup) {};
\node [draw, circle, inner sep=-3pt, fit=(LeeNP) (Lee)] (Leegroup) {};
\node [draw, inner sep=0pt,fit=(StacyNP) (Stacy)] (Stacygroup) {};
\node [baseline,below=\baselineskip of TerryNP] (Terry) {Terry};
\node [draw, inner sep=0pt,fit=(TerryNP) (Terry)] (Terrygroup) {};
\draw[{double equal sign distance, -Implies}] (Tracygroup) -- (Terrygroup);
\draw[{double equal sign distance, -Implies}] (Leegroup.240) -- (Stacygroup.315);
\end{forest}

\caption{Représentation d’une construction à gapping dans \citet{SagEtAl1985}}
\label{ch2:fig2}
\end{figure}

Ce mécanisme de \isi{substitution} permet de prendre en compte les informations syntaxiques fournies par la phrase source, qui s’appliquent aussi à la phrase trouée (p.ex. les contraintes de sous-catégorisation en \REF{ch2:ex243}). 

\ea \label{ch2:ex243}
\ea Pat has become crazy, and Chris depressed.      
\ex Pat has become crazy, and Chris an incredible bore.
\ex *Pat has become crazy, and Chris in good spirits. \citep[160]{SagEtAl1985}
\z
\z

De ce point de vue, \citet{SagEtAl1985} considèrent qu’une approche purement sémantique qui substitue simplement les interprétations (e.g. \citealt{Stump1978}) n’est pas adéquate. Un argument supplémentaire montrant la supériorité des approches à l’interface syntaxe-sémantique vient du fait que ce type de \isi{substitution} permet des discordances interprétatives entre les éléments formant une \isi{paire contrastive}, p.ex. l’interprétation \textit{de re} vs. \textit{de dicto}. Ainsi, en \REF{ch2:ex244}, l’élément résiduel \textit{a pencil} peut avoir une interprétation \textit{de re}, alors que son corrélat \textit{a piece of paper} a une interprétation \textit{de dicto}, et vice-versa. En revanche, dans une approche purement sémantique, l’élément résiduel et son corrélat doivent partager l’interprétation.

\ea \label{ch2:ex244}
Pat is looking for a piece of paper, and Chris, a pencil. \citep[162]{SagEtAl1985} 
\z

\citet{SagEtAl1985} n’ajoutent pas d’autres contraintes syntaxiques, leur hypothèse étant que la plupart des contraintes jouant sur l’acceptabilité des constructions à gapping sont de nature extra-syntaxique, cf. \citet{Hankamer1973}, \citet{Kuno1976} et \citet{Sag1976}. 

En revanche, \citet{Gardent1991} incorpore la contrainte de \is{constituant majeur}constituance majeure ({\cad} l’élément résiduel doit être l’argument ou l’ajout du verbe antécédent ou bien l’argument d’un verbe contenu dans l’argument phrastique du verbe antécédent), afin d’éviter les problèmes de surgénération qui dérivent de l’analyse de \citet{SagEtAl1985}, {\cad} des phrases qui devraient être agrammaticales sont acceptées par la grammaire, et des phrases qui ne devraient avoir qu’une seule lecture en reçoivent plusieurs.  

 
Une règle de légitimation de fragments à l’interface syntaxe-sémantique est proposée aussi par \citet{CulicoverEtAl2005}. Le même mécanisme est en place : La phrase trouée est analysée comme une \is{fragment}phrase fragmentaire sans tête verbale, dont la bonne formation est déterminée par un principe de \isi{substitution} (comme pour la coordination en général, cf. la \is{généralisation de Wasow}généralisation de \ia{Wasow, Thomas}Wasow). On doit pouvoir remplacer les éléments corrélats dans la phrase source par les éléments résiduels de la phrase trouée et obtenir une structure qui soit syntaxiquement et sémantiquement bien formée. 

Selon \citet{CulicoverEtAl2005}, les constructions à gapping relèvent d’un mécanisme de \is{légitimation indirecte}«~légitimation indirecte~» (angl. \textit{Indirect Licensing}) d’au moins deux constituants «~orphelins~». La \isi{légitimation indirecte} inclut : (i) l’intégration sémantique des constituants orphelins dans une structure propositionnelle P (faisant ainsi intervenir une \isi{reconstruction sémantique}), P étant pragmatiquement liée à la phrase source, et (ii) l’intégration syntaxique par substitution en parallèle (\textit{matching}) des \is{paire contrastive}paires contrastives. Pour que cela se fasse, les traits morphosyntaxiques des éléments résiduels doivent être compatibles avec ceux imposés aux corrélats par le prédicat antécédent. Les règles du gapping sont données en \REF{ch2:ex245}. Une représentation simplifiée de la syntaxe et de la sémantique d’une construction à gapping est donnée en \figref{ch2:fig3} et respectivement \REF{ch2:ex246}.

\ea \label{ch2:ex245}
Règles du gapping dans \citet{CulicoverEtAl2005} 

\noindent
Syntaxe:
\begin{avm}\relax
[XP\textsubscript{i}\textsuperscript{ORPHAN1} YP\textsubscript{j}\textsuperscript{ORPHAN2}]\textsuperscript{IL}
\end{avm}\\
Structure conceptuelle (CS):	
\begin{avm}\relax
$\left[\mathcal{F}\left(\left[\begin{array}{l}
                                \text{X\textsubscript{i}}\\\text{\scshape c-focus}
                               \end{array}
\right], \left[
	  \begin{array}{l}
	  \text{Y\textsubscript{j}}\\\text{\scshape c-focus}
	  \end{array}\right]\right)\right]$
\end{avm}
\z

   
   \begin{figure} 
   
   \includegraphics{figures/Ch2Fig3.pdf}
   
   \caption{La syntaxe d’une construction à gapping dans \citet{CulicoverEtAl2005}}
   \label{ch2:fig3}
   \end{figure}


\ea \label{ch2:ex246}
La sémantique d’une construction à gapping dans \citet{CulicoverEtAl2005}

\begin{avm}
$\left[\left[\text{\scshape speak} \left(\left[\begin{array}{l}\text{\scshape robin}\\\text{\scshape c-focus}\end{array}\right], \left[\begin{array}{l} \text{\scshape french}\\\text{\scshape c-focus}\end{array}\right]\right)\right]   \text{ and }\left[\mathcal{F}\left(\left[\begin{array}{l}\text{\scshape leslie}\\\text{\scshape c-focus}\end{array}\right], \left[\begin{array}{l}\text{\scshape german}\\\text{\scshape c-focus}\end{array}\right]\right)\right]\right]$
\end{avm}
\z

La différence majeure entre la \isi{substitution} postulée par \citet{CulicoverEtAl2005} et la \isi{substitution} proposée par \citet{SagEtAl1985} ou \citet{Gardent1991} est que la première est définie sur la structure argumentale du prédicat, alors que dans le deuxième cas, elle se définit sur les arbres syntaxiques. 

Mis à part les arguments convaincants qu’ils donnent contre une \isi{reconstruction syntaxique} du verbe dans la phrase trouée, le reproche qu’on peut faire à l’approche de \citet{CulicoverEtAl2005} est que les \isi{contraintes de parallélisme} syntaxique sont très strictes ; on a montré dans la section~\ref{ch2:sect2.3.4.1} que le roumain (comme le français) permettait une certaine souplesse quant à la catégorie syntaxique, le nombre d’éléments contrastés et \is{ordre de mots}l’ordre des mots. Un autre reproche concerne le statut syntaxique qu’ils donnent à la phrase trouée : dans leur vision, la phrase trouée est un ajout à la phrase source. Néanmoins, les propriétés des coordinations canoniques en général et celles à gapping en particulier semblent être différentes des \is{conjoint incident}conjoints incidents (cf. \citealt{Abeille2005}) : premièrement, une construction à gapping peut présenter une conjonction «~corrélative~» (et on obtient ainsi une \is{coordination omnisyndétique (ou corrélative)}coordination omnisyndétique), alors qu’un \isi{conjoint incident} ne peut jamais être introduit par une conjonction corrélative ; deuxièmement, si la contrainte sur \is{extraction}l’extraction parallèle des deux conjoints peut s’appliquer aux constructions à gapping, elle ne s’applique pas aux \is{conjoint incident}conjoints incidents ; troisièmement, le dernier conjoint dans une coordination à gapping manque de mobilité, alors que le \isi{conjoint incident} est mobile dans la phrase. Pour l’illustration de ces différences en roumain, voir \citet{Bilbiie2008}. 

A l’instar des analyses discutées ici, je présente dans la section suivante une possibilité d’analyse dans le cadre HPSG, tout en restant dans l’esprit de ces \is{approche non structurale}approches non structurales.


\section{Une analyse constructionnelle en HPSG} \label{ch2:sect2.5}

Cette section repose sur l’analyse présentée dans \citet{AbeilleEtAl2014}. Le modèle qu’on retient ici est une \is{approche constructionnelle}version constructionnelle de HPSG (cf. \citealt{Sag1997,GinzburgEtAl2000,SagEtAl2003,Sag2012}), qui opère avec des hiérarchies de constructions à héritage, permettant de représenter non seulement leurs propriétés communes, mais aussi leurs propriétés spécifiques. 

Je présente brièvement l’architecture générale en HPSG (section~\ref{ch2:sect2.5.1}) et l’analyse formelle des constructions coordonnées (section~\ref{ch2:sect2.5.2}), pour ensuite proposer une analyse formelle des constructions à gapping à l’interface syntaxe-sémantique-discours (section~\ref{ch2:sect2.5.3}). J’insiste sur l’analyse des coordinations à gapping en termes de \is{fragment}\textit{fragments}, comme cela a été proposé par \citet{GinzburgEtAl2000} pour les \is{question courte}questions et les \is{réponse courte}réponses courtes en anglais, et par \citet{CulicoverEtAl2005} pour plusieurs phénomènes elliptiques.  


\subsection{Architecture générale en HPSG} \label{ch2:sect2.5.1}

En HPSG, l’unité fondamentale de la langue est le \textit{signe} tel que défini par \ia{Saussure, Ferdinand de}Saussure ({\cad} une relation entre une forme et un contenu). Les signes peuvent être des mots (angl. \textit{words}) ou des syntagmes (angl. \textit{phrases}). Chaque signe linguistique peut être représenté sous la forme d’une structure de traits, utilisée comme cadre unique pour représenter des informations linguistiques hétérogènes (phonologiques, syntaxiques, sémantiques, discursives). La liste de types de signes est organisée selon une hiérarchie. Une hiérarchie simplifiée des signes est donnée en \figref{ch2:fig4}. Et les mots et les syntagmes ont un contenu phonologique (représenté sous l’attribut PHON) et une variété de propriétés syntaxiques et sémantiques (regroupées sous l’attribut SYNSEM)\footnote{
 Les traits PHON et SYNSEM décrivant le signe en HPSG rappellent la dichotomie \ia{Saussure, Ferdinand de}saussurienne \is{signifiant}\textit{signifiant} vs. \is{signifié}\textit{signifié}.}. 

\begin{figure} 

   \includegraphics{figures/Ch2Fig4.pdf}

\caption{Hiérarchie simplifiée des signes}
\label{ch2:fig4}
\end{figure}

En plus de ces traits qui s’appliquent simultanément aux mots et aux syntagmes, il y a des traits qui sont spécifiques à certains sous-types. Ainsi, les mots, contrairement aux syntagmes, comportent une structure argumentale (cf. le trait ARG-ST) qui regroupe dans une seule liste d’objets \textit{synsem} tous les éléments qu’ils sous-catégorisent. Les syntagmes, et non les mots, ont un trait DAUGHTERS (abrégé DTRS), dont la valeur est une liste de \textit{signes}, qui enregistre les constituants immédiats. 

Les synsems qui apparaissent sur la structure argumentale d’un mot peuvent être canoniques (\textit{canonical}) ou non canoniques (\textit{non-canonical}). Les synsems canoniques figurent non seulement dans la structure argumentale d’un mot, mais aussi dans ses traits de valence, contrairement aux synsems non canoniques, qui n’apparaissent que dans la structure argumentale, comme l’indique le \isi{Principe de conservation des arguments} en \REF{ch2:ex247}. Selon ce principe, les arguments sous-catégorisés apparaissent à l’identique sur les traits de valence du prédicat (cf. la coïndiciation des variables), à l’exception des arguments typés comme non canoniques. 


\ea \label{ch2:ex247}
Principe de conservation des arguments\\
\includegraphics{figures/Ch2247.pdf}
\z

La hiérarchie d’objets \textit{synsem} est donnée en \figref{ch2:fig5}. En dehors des arguments canoniques (réalisés localement), on identifie quatre sous-types d’arguments non canoniques dans les \ili{langues romanes} (voir aussi \citealt{MillerEtAl1997}, \citealt{Monachesi1999} et \citealt{GinzburgEtAl2000}) : (i) le sous-type \textit{gap} concerne les éléments extraits dans les \is{dépendance à distance}dépendances à distance ; (ii) le sous-type \textit{pro} (ou pronoms ‘nuls’) peut être utilisé pour le phénomène de \isi{pro-drop} du sujet en roumain ou \isi{pro-drop} de l’objet, ainsi que pour le sujet des impératifs et de certains infinitifs (donc, pour des éléments n’ayant pas de réalisation phonologique) ; (iii) le sous-type \textit{pron-affix} concerne la réalisation des \is{affixe/clitique pronominal}clitiques pronominaux au datif et à l’accusatif, qui, bien qu’ils aient une réalisation phonologique, doivent être analysés comme des affixes verbaux (cf. \citealt{MillerEtAl1997} et \citealt{Monachesi1999}), et (iv) \textit{adv-affix} concerne la réalisation des \is{affixe/clitique adverbial}adverbiaux apparaissant à l’intérieur du \isi{complexe verbal} (p.ex. les adverbes roumains \textit{cam} ‘un peu’, \textit{mai} ‘encore / plus’, \textit{și} ‘aussi’, \textit{tot} ‘encore’, \textit{prea} ‘très’ ou encore la \isi{négation} de phrase \textit{nu}), qui sont analysés eux aussi comme des affixes verbaux \citep{Barbu1999,Barbu2003,Monachesi2000}. Les éléments ainsi typés n’apparaissent pas dans les traits de valence, ils apparaissent uniquement sur la structure argumentale de la tête qui les sous-catégorise. Ils n’ont donc pas de réalisation en syntaxe. 

\begin{figure} 

   \includegraphics{figures/Ch2Fig5.pdf}

\caption{Hiérarchie des valeurs \textit{synsem}}
\label{ch2:fig5}
\end{figure}

Revenons aux syntagmes, qui, comme on l’a déjà précisé, ont un trait DAUGHTERS qui enregistre les constituants immédiats. A la suite de \citet{Sag1997} et \citet{GinzburgEtAl2000}, on représente les différents types de syntagmes dans une hiérarchie à deux dimensions, avec un type phrastique (CLAUSALITY) et un type combinatoire (HEADEDNESS), comme en \figref{ch2:fig6}. Un syntagme hérite ainsi non seulement d’une construction phrastique ({\cad} un sous-type de \is{type de phrase}\textit{clause}, comme le type déclaratif, interrogatif, désidératif ou exclamatif) ou non phras\-tique ({\cad} un sous-type de \textit{non-clause}), mais aussi d’une \is{syntagme endocentrique}construction endocentrique ({\cad} un sous-type de \textit{headed-ph}) ou \is{syntagme exocentrique}exocentrique ({\cad} un sous-type de \textit{non-headed-ph}).  

\begin{figure} 

   \includegraphics[width=\textwidth]{figures/Ch2Fig6.pdf}

\caption{Classification des syntagmes}
\label{ch2:fig6}
\end{figure}

Les syntagmes sans tête comportent un attribut NON-HEAD-DTRS où sont enregistrés les constituants immédiats non-têtes \REF{ch2:ex248a}. En revanche, les syntagmes avec tête présentent un attribut HEAD-DTRS où figure leur constituant immédiat tête \REF{ch2:ex248b}. Tout \isi{syntagme endocentrique} obéit au \isi{Principe des traits de tête généralisé} (\textit{Generalized Head Feature Principle}, cf. \citealt{GinzburgEtAl2000}), qui dit que tous les traits syntaxiques et sémantiques ({\cad} la valeur de l’attribut SYNSEM) sont partagés par défaut (cf. le symbole /) entre un syntagme et sa tête, ce que note la coïndiciation de la variable [1] en \figref{ch2:fig6}.

\ea
\ea \textit{phrase} $\Rightarrow$ [NON-HEAD-DTRS \textit{list(sign)}] \label{ch2:ex248a}    
\ex  \textit{headed-phrase} $\Rightarrow$ [HEAD-DTR \textit{sign}] \label{ch2:ex248b}
\z
\z

Maintenant qu’on a défini les syntagmes avec tête, on peut définir la notion de phrase. La phrase est un signe syntagmatique à tête saturée (ses traits de valence ont pour valeur la liste vide notée < >). De plus, selon \citet{GinzburgEtAl2000}, le contenu d’une phrase doit être un sous-type de \textit{message}. On arrive ainsi à la représentation donnée en \REF{ch2:ex249} :

\ea \label{ch2:ex249}
Représentation simplifiée de la phrase\\
\includegraphics{figures/Ch2249.pdf}

% \noindent
% \textit{clause} $\Rightarrow$ 
% \begin{avm}
% [cat & [val & [subj & <>\\
%                comps & <>\\
%                spr <>]]\\
% cont & message]
% \end{avm}

\z


\subsection{Formalisation des constructions coordonnées} \label{ch2:sect2.5.2}

Je m’intéresse dans cette section à l’analyse formelle des constructions coordonnées, suivant en particulier les travaux de \citet{Sag2003,Sag2005}, \citet{Abeille2003,Abeille2005}, \citet{Mouret2006,Mouret2007} et \citet{Bilbiie2008,Bilbiie2011}. 

Une coordination est une structure hiérarchique à deux étages, comme illustrée en \figref{ch2:fig7} : le premier niveau est celui du syntagme conjoint ({\cad} le constituant dont fait partie la conjonction et la séquence qui la suit), tandis que le deuxième niveau correspond à la structure coordonnée dans son ensemble.

\begin{figure} 

   \includegraphics{figures/Ch2Fig7.pdf}

\caption{Structure hiérarchique de la coordination \textit{Ion și Maria} ‘Ion et Maria’}
\label{ch2:fig7}
\end{figure}

Je m’intéresse d’abord à la structure du syntagme conjoint. Celui-ci est généralement analysé comme un syntagme de type tête-complément, donc il a une \is{syntagme endocentrique}structure endocentrique. Sur la base de certains faits observés à travers les langues (corrélation entre la position de la conjonction et la position de la tête à travers les langues, des contraintes de sous-catégorisation ou des restrictions sur le type sémantique imposées souvent par la conjonction, blocage de l’assignation du cas), on a conclu que la conjonction régit le constituant avec lequel elle se combine, donc la fonction tête est associée à la conjonction (\citealt{Paritong1992,Munn1993,Kayne1994,Johannessen1998}, etc.). En même temps, on a observé que la conjonction ne détermine que partiellement la syntaxe externe du constituant qu’elle introduit \citep{Paritong1992,Johannessen1998,Abeille2003,Abeille2005,Abeille2006}, ayant ainsi un comportement différent par rapport aux têtes ordinaires (p.ex. un verbe ou un adjectif). De ce point de vue, le syntagme conjoint, tout comme d’autres syntagmes contenant une catégorie «~mineure~» (p.ex. déterminants, complémenteurs ou prépositions «~incolores~»), présente un mixte des propriétés de leurs constituants immédiats (voir discussion et illustrations dans \citealt[71--73]{Mouret2007}). 

Je considère que l’analyse des conjonctions en termes de \is{tête faible}têtes «~faibles~», telle qu’elle a été proposée par \citet{Abeille2003,Abeille2005}, rend compte parfaitement de ce comportement spécial des conjonctions (\citealt{Sag1997} et \citealt{Tseng2002} proposent une analyse similaire pour d’autres catégories «~mineures~»). Dans cette perspective, une catégorie «~mineure~» hérite du constituant qu’elle sélectionne la plupart des propriétés morpho-syntaxiques qu’elle transmet au syntagme. 

La conjonction sous-catégorise donc le constituant avec lequel elle se combine (ce qui justifie le terme de \textit{tête}), mais hérite une partie des propriétés morpho-syntaxiques de celui-ci (cf. le partage de variables en \REF{ch2:ex250}), sauf le trait CONJ. Par conséquent, syntaxiquement, ce sont les conjoints qui déterminent, entre autres, la catégorie de la structure coordonnée dans son ensemble, alors que, sémantiquement, c’est la conjonction qui gère l‘interprétation de l’ensemble à partir de la contribution sémantique de chaque conjoint.

\ea \label{ch2:ex250}
Entrée lexicale d’une conjonction\\
\includegraphics{figures/Ch2250.pdf}

% \noindent
% \textit{conj-word} $\Rightarrow$
% \begin{avm}
% [category & [head&@{1}\\
% 						marking&@{2}\\
% 						valence & [subj&@{3}\\
% 											 spr&@{4}\\
% 											 comps & <[head&@{1}\\
% 											 					marking&@{2}\\
% 											 					subj&@{3}\\
% 											 					spr&@{4}\\
% 											 					comps&@{5}\\
% 											 					conj&nil]> & $\oplus$ & @{5}]\\
% 						conj&$\neg$nil]]
% 						\end{avm}
                        
\z

Les traits de tête (HEAD) et de marque (MARKING) de la conjonction sont les mêmes que ceux de son complément. La conjonction hérite des traits SUBJ, SPR et COMPS de son complément, permettant ainsi au syntagme conjoint de rendre accessible les contraintes de sélection de ce complément. La conjonction a un trait CONJ à valeur non nulle, tandis que son complément possède un trait CONJ à valeur nulle. On obtient ainsi une \is{syntagme endocentrique}structure endocentrique des séquences [Conj X], comme illustrée en \figref{ch2:fig8} pour le conjoint \textit{sau pe masă} ‘ou sur (la) table’. 

\begin{figure} 

   \includegraphics{figures/Ch2Fig8.pdf}

\caption{Syntaxe simplifiée du syntagme conjoint}
\label{ch2:fig8}
\end{figure}

\largerpage[-1]
Revenant maintenant à la structure coordonnée dans son ensemble, la plupart des travaux alignent la structure coordonnée sur la \is{syntagme endocentrique}structure endocentrique du syntagme conjoint. Ainsi, \citet{Munn1993}, \citet{Kayne1994}, \citet{Johannessen1998}, \citet{Camacho2003}, \citet{Rebuschi2005}, etc. considèrent les structures coordonnées comme des constructions avec tête. En particulier, les coordinations sont, selon eux, soit des structures de type spécifieur-tête-compléments\footnote{
 Dans ce cas, la structure coordonnée est un syntagme ConjP, dans laquelle la conjonction est la tête, le premier conjoint est le spécifieur et le deuxième conjoint est le complément.} \citep{Kayne1994,Johannessen1998}, soit des structures de type tête-ajout\footnote{
 Selon cette analyse, le syntagme introduit par la conjonction est adjoint au premier terme coordonné qui est la tête de la coordination dans son ensemble. Cette analyse est reprise par \citet{Abeille2005} pour rendre compte des \is{conjoint incident}conjoints incidents en français.} \citep{Munn1993}. Cependant, par rapport aux syntagmes ordinaires avec tête, les propriétés de la structure coordonnée dans son ensemble ne sont pas projetées par une seule tête syntaxique, mais elles sont déterminées par tous les conjoints (voir discussion et arguments convaincants dans \citealt{Borsley1994} et \citealt{Borsley2005})\footnote{
 Pour une critique détaillée du syntagme ConjP, voir \citet{Chaves2007} et \citet{Mouret2007}.}. Puisqu’il n’y a pas de dépendance syntaxique entre les conjoints, on ne peut donc attribuer la fonction tête à aucun conjoint. Par conséquent, les propriétés d’une coordination dépendent des propriétés de \textit{chacun} des termes qui la composent et non pas des propriétés de la conjonction.

Pour rendre compte du comportement syntaxique des structures coordonnées, je fais appel à la distinction mentionnée ci-dessus dans la section~\ref{ch2:sect2.5.1} entre les syntagmes avec tête (\textit{headed-ph}) et les syntagmes sans tête (\textit{non-headed-ph}). Dans un syntagme avec tête, on a une relation de sélection entre les constituants : une branche dominante gère la catégorie et la distribution syntaxique de l’ensemble. Dans un syntagme sans tête, il n’y a aucune relation de sélection au niveau syntaxique. La structure coordonnée se prête mieux à une analyse en termes de construction sans tête, {\cad} un syntagme de type \textit{non-headed-ph} (cf. \citealt{Abeille2003,Abeille2005,Mouret2006,Mouret2007}, etc.). La syntaxe simplifiée d’une coordination de phrases est illustrée en \figref{ch2:fig9} pour l’exemple roumain \textit{Ion doarme și Maria citește} (‘Ion dort et Marie lit.’). 

\begin{figure} 

   \includegraphics{figures/Ch2Fig9.pdf}

\caption{Syntaxe simplifiée d’une coordination de phrases}
\label{ch2:fig9}
\end{figure}


Une coordination comporte au moins deux termes, qui peuvent ou non être introduits par une conjonction. La règle générale de la coordination est donnée en \REF{ch2:ex251}.

\ea \label{ch2:ex251}
Règle générale de la coordination\\
\includegraphics{figures/Ch2251.pdf}

% \noindent
% \textit{coord-phrase} $\Rightarrow$ \textit{non-headed-ph} \&
% \begin{avm}
% [dtrs & <$sign$, $sign$> \, $\oplus$ list($sign$)]
% \end{avm}
% \& \\
% \begin{avm}
% [synsem & [conj & nil]\\
%  dtrs & list([conj & nil]) \, $\oplus$  <[conj @1 & $\neg$nil],..., [conj & @1]   >]
%  \end{avm}

\z

En fonction de la distribution des conjonctions, on peut distinguer entre trois types de constructions coordonnées (cf. \citealt{Mouret2006}, \citealt{Mouret2007} pour le français, \citealt{Bilbiie2008}, \citealt{Bilbiie2011} pour le roumain, et de manière plus générale dans les \ili{langues romanes}) : (i) des coordinations simples, avec au moins une conjonction sur le dernier conjoint \REF{ch2:ex252} ; (ii) des coordinations \is{coordination omnisyndétique (ou corrélative)}omnisyndétiques ou corrélatives, avec un élément corrélatif répété sur chaque conjoint (y compris le conjoint initial), cf. \REF{ch2:ex253}, et (iii) des coordinations asyndétiques ou \is{juxtaposition}juxtaposées (sans aucune conjonction réalisée), cf. \REF{ch2:ex254}. 

\ea \label{ch2:ex252}
\ea Syntagme de type \textit{simplex-cord-ph}\\
\includegraphics{figures/Ch2252.pdf}

% [DTRS <[CONJ \textit{nil}]> $\oplus$ \textit{list(sign)} $\oplus$ <[CONJ ¬\textit{nil}]>]

\ex Ion, (\textbf{şi}) Maria \textbf{şi} Gheorghe
\glt ‘Ion, (et) Maria et Gheorghe’
\z
\z


\ea \label{ch2:ex253}
\ea Syntagme de type \textit{omnisyndetic-coord-ph}\\
\includegraphics{figures/Ch2253.pdf}

% [DTRS <[CONJ ¬\textit{nil}]> $\oplus$ \textit{list(sign)}]

\ex \textbf{fie} Ion, \textbf{fie} Maria, \textbf{fie} Gheorghe
\glt ‘soit Ion, soit Maria, soit Gheorghe’
\z
\z


\ea \label{ch2:ex254}
% \ea \textit{asyndetic-coord-ph} => \textit{coord-ph \&}

% [DTRS \textit{list(sign)} $\oplus$ <[CONJ \textit{nil}]>]  
\ea Syntagme de type \textit{asyndetic-coord-ph}\\
\includegraphics{figures/Ch2254.pdf}

\ex Ion, Maria, Gheorghe
\glt ‘Ion, Maria, Gheorghe’
\z
\z

Une hiérarchie de syntagmes qui contient les trois types de coordinations est donnée en \figref{ch2:fig10}. 

\begin{figure} 

   \includegraphics[width=\textwidth]{figures/Ch2Fig10.pdf}

\caption{La structure coordonnée dans une hiérarchie de syntagmes}
\label{ch2:fig10}
\end{figure}

Les termes d’une coordination ont souvent les mêmes propriétés syntaxiques, mais ils peuvent aussi différer sur certains aspects (catégorie syntaxique, cas, personne, mode verbal, temps, forme de la préposition, etc.). Le problème des \is{coordination de termes dissemblables}coordinations de termes dissemblables, comme en \REF{ch2:ex255a}, peut être résolu si on assume, cf. \citet{Sag2003,Sag2005}, que les valeurs de certains traits peuvent rester sous-spécifiées, même lorsque la structure de traits est complète du point de vue de la grammaire. Ainsi, on peut supposer que les entrées lexicales ne fixent pas une valeur précise pour la valeur de HEAD, mais imposent une borne supérieure, comme illustré en \REF{ch2:ex256}, où ≤ signifie ‘égal ou supérieur à’. Ainsi, \REF{ch2:ex256a} dit que la catégorie syntaxique de \textit{naiv} ‘naïf’ est moins spécifiée que, ou égale à la catégorie \textit{adjectif} (\textit{adj}), alors que \REF{ch2:ex256b} dit que la catégorie de \textit{imbecil} ‘imbécile’ est une valeur moins spécifiée que, ou égale à la catégorie \textit{nom} (\textit{noun}). 

\ea \label{ch2:ex255}
\ea Ion este [fie naiv]\textsubscript{AdjP}, [fie un imbecil]\textsubscript{NP}. \label{ch2:ex255a}
\glt ‘Ion est soit naïf, soit un imbécile.’

\ex Ion este \{naiv {\textbar} un imbecil\}.
\glt ‘Ion est \{naïf {\textbar} un imbécile\}.’  
\z
\z


\ea \label{ch2:ex256}
\ea naiv (‘naïf’) : [HEAD \fbox{1} {\textbar} \fbox{1} ≤ \textit{adj}] \label{ch2:ex256a}
\ex imbecil (‘imbécile’) : [HEAD \fbox{2} {\textbar} \fbox{2} ≤ \textit{noun}] \label{ch2:ex256b}
\z
\z

Par la suite, les traits de tête de la construction coordonnée sont obtenus par unification des traits de tête des branches (DTRS), qui peuvent, quant à elles, rester sous-spécifiées. On peut donc coordonner le syntagme adjectival et le syntagme nominal, car la coordination reçoit par unification une catégorie sous-spécifiée \textit{nominal}, qui est un super-type des noms et des adjectifs. Cette catégorie sous-spécifiée peut donc s’unifier sans problème avec les deux compléments prédicatifs sélectionnés par le verbe copule \textit{a fi} ‘être’ (car le verbe \textit{a fi} accepte comme complément et un syntagme adjectival et un syntagme nominal, cf. \REF{ch2:ex256b}), comme on voit en \figref{ch2:fig11}. En revanche, cette catégorie sous-spécifiée ne peut pas s’unifier avec le complément sélectionné par un verbe comme \textit{a întâlni} ‘rencontrer’ \REF{ch2:ex257a}, car ce verbe ne peut pas avoir comme complément un syntagme adjectival \REF{ch2:ex257b}.  

\ea
\ea 
\gll *Ion  a  întâlnit  fie  naiv,  fie  un  imbecil. \label{ch2:ex257a}\\
Ion  a  rencontré  soit  naïf  soit  un  imbécile\\
\glt ‘Ion a rencontré soit un naïf, soit un imbécile.’

\ex Ion a întâlnit \{*naiv {\textbar} un imbecil\}. \label{ch2:ex257b}
\glt ‘Ion a rencontré \{un naïf {\textbar} un imbécile\}.’
\z
\z


\begin{figure} 

   \includegraphics[width=\textwidth]{figures/Ch2Fig11.pdf}

\caption{Coordination de termes dissemblables, cf. \REF{ch2:ex255a}}
\label{ch2:fig11}
\end{figure}

Les \isi{contraintes de parallélisme} à l’œuvre dans les constructions coordonnées sont données en \REF{ch2:ex258}. Ces contraintes imposent qu’il y ait non seulement un partage des traits SLASH et VALENCE entre le syntagme coordonné dans son ensemble et les membres conjoints, mais aussi un partage, par défaut, du trait HEAD. L’identité de valeur pour le trait SLASH (trait qui enregistre les constituants manquants en cas d’extraction) autorise uniquement \is{extraction parallèle}l’extraction parallèle hors de chaque conjoint (cf. la \isi{Contrainte sur les Structures Coordonnées} de \citealt{Ross1967}). L’identité de valeur pour le trait VALENCE n’autorise que la coordination de prédicats avec les mêmes contraintes de sous-catégorisation. 

\ea \label{ch2:ex258}
Contraintes de parallélisme dans les constructions coordonnées\\
\includegraphics{figures/Ch2258.pdf}

% \noindent
% \textit{coord-phrase} $\Rightarrow$
% \begin{avm}
% [synsem & [head / @H\\
% 						 valence @V\\
% 						 slash @S]\\
%  dtrs & <[head / @H\\
% 					valence @V\\
% 					slash @S],...,
% 					[head / @H\\
% 					valence @V\\
% 					slash @S]>]
% \end{avm}	

\z  


\subsection{Formalisation des constructions à gapping} \label{ch2:sect2.5.3}

D’abord, je présente la formalisation des séquences de constituants appelées \is{cluster}\textit{clusters}, qui nous permet de générer toute séquence qu’on peut avoir dans une phrase trouée (section~\ref{ch2:sect2.5.3.1}). Ensuite, je montre comment la notion de \is{fragment}\textit{fragment} nous permet d’attribuer un contenu propositionnel à la séquence trouée et, en particulier, comment on récupère le contenu de l’antécédent afin d’obtenir la bonne interprétation dans la phrase trouée (section~\ref{ch2:sect2.5.3.2}). Enfin, je montre le fonctionnement spécifique des constructions à gapping, en postulant un sous-type de syntagme coordonné, appelé \textit{gapping-ph} (section~\ref{ch2:sect2.5.3.3}).

\subsubsection{Une théorie des clusters} \label{ch2:sect2.5.3.1}

La séquence trouée est composée d’au moins deux éléments résiduels. On a montré dans la section~\ref{ch2:sect2.4.3} qu’elle n’a pas toujours la même distribution que la phrase source. En particulier, on observe qu’en français, par exemple, la conjonction \textit{ainsi} \textit{que} peut être suivie d’une séquence de syntagmes ayant un contenu propositionnel, mais elle ne peut pas être suivie d’une phrase \REF{ch2:ex259a}. \citet{Mouret2006,Mouret2007}, à la suite de \citet{AbeilleEtAl1996}, observe la même contrainte pour la coordination de séquences (ou \is{Argument Cluster Coordination (ACC)}\textit{Argument Cluster Coordination}, abrégé ACC) en \REF{ch2:ex259b}, ce qui l’amène à analyser ces séquences comme \is{cluster}\textit{clusters}.

\ea \label{ch2:ex259}
\ea Paul a cueilli des framboises, ainsi que Marie (*a cueilli) des fraises. \label{ch2:ex259a} 
\ex Paul a offert un livre à Marie, ainsi qu’(*il a offert) un CD à Anne. \label{ch2:ex259b}  
\z
\z

\citet{Mouret2006,Mouret2007} propose que la notion de \is{cluster}\textit{cluster} soit définie indépendamment de la coordination dans la grammaire, car elle est pertinente aussi pour les séquences elliptiques dans le domaine de la subordination \REF{ch2:ex260} ou dans le \isi{dialogue} \REF{ch2:ex261}, dans lesquelles les constituants immédiats n’entretiennent pas de relations fonctionnelles.  

\ea \label{ch2:ex260}
\ea Tout comme <Marie son thé>, Paul a apprécié son café. 
\ex Plusieurs personnes sont parties à l’étranger, dont <deux à Rome>.  
\z
\z

\ea \label{ch2:ex261}
A : Je me demande ce que Paul peut bien vendre comme livres et à qui, dans sa librairie miteuse.

B : <Des livres d’occasion à quelques collectionneurs aventureux>, j’imagine.
\z

Par conséquent, on reprend ici la notion de \is{cluster}\textit{cluster} afin de rendre compte de la constituance des séquences trouées dans les constructions à gapping. On propose que la séquence de constituants non standard, sans tête verbale, dans le gapping soit modélisée comme un sous-type de syntagme sans tête ({\cad} \textit{cluster-ph}). Une hiérarchie de syntagmes, incluant le syntagme de type \isi{cluster}, apparaît en \figref{ch2:fig12}.  

\begin{figure} 

   \includegraphics[width=\textwidth]{figures/Ch2Fig12.pdf}

\caption{Le cluster dans une hiérarchie de syntagmes}
\label{ch2:fig12}
\end{figure}

Comme règle syntaxique, on a juste besoin d’une règle générant les séquences de syntagmes, similaire à la composition de catégories en \isi{grammaire catégorielle} \citep{Steedman1990}. La description formelle d’un syntagme \isi{cluster} est donnée en \REF{ch2:ex262}. Les constituants immédiats du cluster (p.ex. les éléments résiduels dans une séquence trouée) sont enregistrés dans un trait de tête CLUSTER, qui prend comme valeur la liste (non-vide) des descriptions \textit{synsem} de ses branches. Les propriétés syntaxiques et sémantiques des éléments résiduels sont ainsi accessibles au niveau de la construction. En plus, le \isi{cluster} est un syntagme saturé pour ses traits de valence (cf. la valeur vide des attributs SUBJ, SPR et COMPS). Il amalgame les valeurs SLASH de ses constituants (ce qui nous permet de rendre compte de la contrainte \is{extraction parallèle}d’extraction parallèle hors d’une coordination de séquences). Les autres propriétés (y compris la catégorie) sont sous-spécifiées, ce qui lui permet la combinaison avec des formes comme \textit{ainsi que} en français, qui n’est jamais compatible avec une phrase finie. 

\ea \label{ch2:ex262}
Syntagme de type cluster (cf. \citealt{Mouret2006,Mouret2007})\\
\includegraphics{figures/Ch2262.pdf}

% \noindent
% \textit{cluster-ph} $\Rightarrow$ \textit{non-headed-ph} \& \\
% \begin{avm}
% [head & [\tp{head}\\
% 				 cluster & nelist\(synsem\) <@{1}, ..., @{n}>]\\
% subj & < >\\
% spr & < > \\
% comps & < >\\
% slash & $\Sigma$$_1$ $\cup$ ... $\cup$ $\Sigma$$_n$\\
% n-hd-dtrs & <[synsem @{1} [slash & $\Sigma$$_1$]], ..., [synsem @{n} [slash & $\Sigma$$_n$]]>]
%  \end{avm}

\z

Postuler un trait spécifique CLUSTER peut générer a priori toute séquence de syntagmes. Cette sous-spécification massive peut être évitée si on restreint le potentiel combinatoire des clusters. Pour cela, \citet{Mouret2007} propose la contrainte en \REF{ch2:ex263}, qui assure que les clusters n’apparaissent jamais dans la structure argumentale d’un prédicat.  

\ea \label{ch2:ex263}
\textit{word} $\Rightarrow$ [ARG-ST \textit{list}([CLUSTER < >])]
\z

La contrainte qui décrit les clusters en \REF{ch2:ex262} nous permet maintenant de dériver une séquence de deux éléments résiduels dans le gapping \REF{ch2:ex264}, comme le montre la représentation simplifiée en \figref{ch2:fig13}.

\ea \label{ch2:ex264}
\gll Anei  i-am  dat  un  măr,  iar  [[Mariei]  [o  banană]].\\ 
Ana.\textsc{dat} \textsc{dat.3sg}{}-ai  donné  une  pomme  et  Maria.\textsc{dat}  une  banane\\
\glt ‘A Ana j’ai donné une pomme et à Maria une banane.’
\z


\begin{figure} 

   \includegraphics{figures/Ch2Fig13.pdf}

\caption{Représentation simplifiée de \REF{ch2:ex264}}
\label{ch2:fig13}
\end{figure}


\subsubsection{Une théorie des fragments} \label{ch2:sect2.5.3.2}

Grâce à la règle syntaxique associée au syntagme \isi{cluster} \REF{ch2:ex262}, la grammaire peut maintenant générer toute séquence de deux éléments résiduels qu’on rencontre dans les constructions à gapping. Il faut montrer toutefois comment on arrive à attribuer un contenu propositionnel à la séquence trouée et, en particulier, comment on récupère le contenu de l’antécédent afin d’obtenir la bonne interprétation dans la phrase trouée. 

Dans la section~\ref{ch2:sect2.4.3}, on a observé qu’une séquence trouée n’a pas nécessairement le même comportement syntaxique qu’une phrase ordinaire, bien qu’elle ait le même type de contenu sémantique ({\cad} un sous-type de \textit{message}). Cela nous amène à considérer que la reconstruction de l’ellipse se passe plutôt en sémantique qu’en syntaxe. La même idée apparaît dans \citet{GinzburgEtAl2000} où on propose la notion de \is{fragment}\textit{fragment} pour rendre compte de la structure des \is{question courte}questions et des \is{réponse courte}réponses courtes dans le \isi{dialogue}. Ainsi, dans la réponse/question du locuteur B en \REF{ch2:ex265} et respectivement \REF{ch2:ex266}, on a un syntagme nominal exhaustivement dominé par une phrase, ayant l’interprétation d’une phrase déclarative ({{\cad}} \textit{John left}) et respectivement interrogative ({{\cad}} \textit{Who called}). On se donne en syntaxe la notion de \is{fragment}\textit{fragment} conçue comme une construction à laquelle sont associées des conditions de bonne formation syntaxiques et interprétatives.

\ea \label{ch2:ex265}
A : Who left ? 

B : [[John]\textsubscript{NP}]\textsubscript{S}. 
\z

\ea \label{ch2:ex266}
A : Someone called. 

B : [[Who]\textsubscript{NP}]\textsubscript{S}?  
\z

La structure syntaxique des \is{fragment}fragments dans les constructions à gapping con\-tient uniquement les éléments résiduels (qui, quant à leur constituance, forment un \isi{cluster}). Par conséquent, les fragments sont des expressions dont la contribution sémantique n’est donnée que partiellement par leur forme ; ce sont des unités syntaxiques dont l’interprétation nécessite une connaissance de la relation sémantique principale de l’énoncé. Leur contribution sémantique est une fonction de trois éléments : (i) le type de \isi{fragment}, (ii) l’information contextuelle, et (iii) le contenu littéral du \isi{fragment}. Par exemple, la contribution sémantique du \isi{fragment} \textit{who} en \REF{ch2:ex266} est une fonction de son type (ici, une \isi{question courte}, ayant le même contenu qu’une phrase interrogative, {\cad} une abstraction propositionnelle), l’information contextuelle (la phrase source \textit{Someone called} fournit l’antécédent nécessaire à la résolution du \isi{fragment}), et le contenu littéral \textit{who} (qui fournit le paramètre pour l’abstraction propositionnelle). Ainsi, le \isi{fragment} \textit{who} a un contenu similaire avec la phrase interrogative complète \textit{Who came}.

Ce sont des \is{fragment}fragments phrastiques, car (i) leur interprétation est univoque dans le contexte, et (ii) ils ont le même type sémantique que les phrases complètes, {{\cad}} proposition, question, visée (cf. \citealt{GinzburgEtAl2000}).

Les fragments phrastiques ressemblent aux expressions \is{anaphore}anaphoriques, dans le sens où leur contenu est relié à un antécédent qui est déterminé contextuellement. En particulier, les \is{fragment}fragments se rapprochent des \is{anaphore descriptive}anaphores descriptives, qui, contrairement aux \is{anaphore d'instance}anaphores d’instance, ne désignent pas la même instance que l’antécédent ; ce type d’expressions \is{anaphore}anaphoriques introduit une nouvelle entité sémantique qui partage une partie de sa description avec l’antécédent, mais l’entité elle-même n’est pas partagée, ce qui explique la différence d’indices en \REF{ch2:ex267b}\footnote{
 Ce n’est pas le cas des \is{anaphore d'instance}anaphores d’instance, qui partagent le même indice avec leur antécédent.}.

\ea \label{ch2:ex267}
\ea{} [Paul has lost his keys again]\textsubscript{i}. It\textsubscript{i} happened yesterday. \label{ch2:ex267a}
\ex{} [Paul has lost his keys again]\textsubscript{i}. It\textsubscript{j} has never happened to me. \label{ch2:ex267b} 
\z
\z

Le \isi{fragment}, tel qu’il est défini par \citet{GinzburgEtAl2000}, est la branche unaire d’un syntagme tête. Il a l’ensemble des propriétés d’une phrase, y compris la catégorie syntaxique VERBAL. La contrainte générale qu’ils donnent pour un \isi{fragment} phrastique figure en \REF{ch2:ex268}.

\ea \label{ch2:ex268}
Syntagme de type fragment dans \citet{GinzburgEtAl2000}\\
\includegraphics{figures/Ch2268.pdf}

% \noindent
% \begin{avm}
% [\tp{head-fragment-ph} \\
% category | head & [\tp{verbal} \\ vform & finite] \\
% content & message \\
% context | sal-utt & \{[category @1 \\ content | index & @2]\}] $\longrightarrow$ [category @1 [head & nominal] \\ content | index @2]
% \end{avm}

\z

Elle assure que la catégorie de la branche tête – restreinte à un sous-type de \textit{nominal} ({\cad} un nom ou une préposition) – est identique à la catégorie de l’élément parallèle (ou corrélat) dans la phrase source, figurant dans le trait contextuel SAL-UTT (abrégé de \is{Salient Utterance (SAL-UTT)}SALIENT-UTTERANCE). En revanche, la catégorie du syntagme supérieur ({\cad} du fragment) a la même catégorie qu’un verbe fini : le \isi{fragment} peut ainsi fonctionner comme une phrase indépendante ou bien comme le complément d’un verbe qui sélectionne une phrase finie (et non un syntagme nominal). Enfin, cette contrainte assure que la branche tête partage la même valeur avec l’élément corrélat contenu dans SAL-UTT. Cela permet d’intégrer le contenu de la branche tête dans un contenu obtenu contextuellement. 

Je donne en \figref{ch2:fig14} l’analyse que \citet{GinzburgEtAl2000} proposent pour une \isi{réponse courte}, p.ex. \textit{John}, à une question, p.ex. \textit{Who left}? 

\begin{figure} 

   \includegraphics{figures/Ch2Fig14.pdf}

\caption{Analyse de la réponse courte \textit{John.} à la question \textit{Who left?}}
\label{ch2:fig14}
\end{figure}

La catégorie de la branche tête est un syntagme nominal, comme le requiert la contrainte donnée en \REF{ch2:ex268}, alors que la catégorie du syntagme supérieur est de type \textit{verbal} (= phrase finie, de type déclaratif). Le \isi{fragment} est une phrase indépendante (cf. [IC +]). Le contenu de la phrase est une proposition. Si dans la plupart des syntagmes avec tête le contenu est fourni en grande partie par la branche tête, dans un syntagme de type fragment avec tête, le contenu est construit essentiellement à partir de la question saillante dans le contexte ({\cad} MAX-QUD, abrégé de \is{Maximal Question Under Discussion (MAX-QUD)}MAXIMAL-QUESTION-UNDER-DISCUSSION). Le trait MAX-QUD nous donne l’accès au contenu de la phrase source (ici, une question). L’élément parallèle (ou corrélat) se trouvant dans la phrase source est identifié grâce au trait SAL-UTT. La branche tête du fragment et son corrélat dans la phrase source partagent le même indice\footnote{Pour les autres détails concernant cette description, voir \citet[chapitre 8, section 8.1.4]{GinzburgEtAl2000}.}.

On observe donc que, dans l’approche de \citet{GinzburgEtAl2000}, le \isi{fragment} est un syntagme qui a le contenu d’une phrase et qui domine exhaustivement une tête de même catégorie et de même indice qu’un élément parallèle saillant dans le contexte. En tant que telle, l’analyse est inadéquate pour les constructions à gapping, qui ont au moins deux éléments résiduels et parfois sans identité catégorielle avec leurs corrélats dans la phrase source (voir les exemples \REF{ch2:ex143} et \REF{ch2:ex145} de la section~\ref{ch2:sect2.3.4.1}).   

Par conséquent, on étend l’analyse de \citet{GinzburgEtAl2000}, afin d’inclure d’autres constructions elliptiques. Cette notion de \is{fragment}\textit{fragment} peut a priori s’appliquer non seulement aux phrases elliptiques indépendantes (dans le \isi{dialogue}), mais aussi aux constructions elliptiques coordonnées ou subordonnées, contenant (parfois) des séquences avec plus d’un élément résiduel : le gapping \REF{ch2:ex269a} et \REF{ch2:ex270a}, l’ellipse dans les \isi{comparatives} \REF{ch2:ex269b} et \REF{ch2:ex270b}, le \is{Stripping}stripping \REF{ch2:ex269c} et \REF{ch2:ex270c}, les ajouts exceptifs \REF{ch2:ex269d} et \REF{ch2:ex270d}, les ajouts relatifs partitifs \REF{ch2:ex269e} et \REF{ch2:ex270e} ou encore les ajouts concessifs \REF{ch2:ex269f} et \REF{ch2:ex270f}. 

\largerpage

\ea \label{ch2:ex269}
\ea Tudor a cumpărat o carte, [iar Maria *(o păpuşă)]. \label{ch2:ex269a}
\glt ‘Tudor a acheté un livre, et Maria une poupée.’

\ex Ioana a mâncat mai multe mere [decât Ion (pere)]. \label{ch2:ex269b}
\glt ‘Ioana a mangé plus de pommes que Ion (des poires).’  

\ex 
\gll Toată  lumea  îl  apreciază  pe  Ion,  [chiar  şi  duşmanii  lui]. \label{ch2:ex269c}\\
tout  monde.\textsc{def} \textsc{acc.3sg.m} apprécie  \textsc{dom}  Ion  même  aussi  ennemis.\textsc{def} \textsc{poss.3sg.m}\\
\glt ‘Tout le monde apprécie Ion, même ses ennemis.’  

\ex Niciun elev nu-şi făcuse temele, [mai puţin Ion (tema la engleză)]. \label{ch2:ex269d}
\glt ‘Aucun élève n’avait fait ses devoirs, sauf Ion (le devoir d’anglais).’  

\ex In România, trăiesc aproximativ 8~000 de evrei, [dintre care jumătate în Bucureşti]. \label{ch2:ex269e}
\glt ‘En Roumanie vivent environ 8~000 juifs, dont la moitié à Bucarest.’  

\ex Acoperişurile care sunt ude, [chiar dacă foarte puţin], pot fi foarte alunecoase. \label{ch2:ex269f} 
\glt ‘Les toitures qui sont mouillées, quoique très peu, peuvent être très glissantes.’  
\z
\z


\ea \label{ch2:ex270}
\ea Paul aime les pommes [et Marie les oranges]. \label{ch2:ex270a}
\ex Paul aime autant les pommes [que Marie (les oranges)]. \label{ch2:ex270b}
\ex Paul viendra, [ou (peut-être) Anne]. \label{ch2:ex270c}
\ex Tout le monde réussit sa vie, [sauf Marie]. \label{ch2:ex270d}
\ex Plusieurs personnes sont venues me voir, [dont Marie hier]. \label{ch2:ex270e}
\ex Ses enfants l’appellent régulièrement, [quoique (Marie) assez peu]. \label{ch2:ex270f}  
\z
\z

On reprend de \citet{GinzburgEtAl2000} la hiérarchie de syntagmes, en particulier le sous-type \textit{head-fragment-ph} comme sous-type de \textit{head-only-ph}. Une hiérarchie des syntagmes contenant le syntagme qui nous intéresse ici est donnée en \figref{ch2:fig15}.

\begin{figure} 

\includegraphics[width=\textwidth]{figures/Ch2Fig15.pdf}% \input{figures/ch2_trees.tex}

\caption{Le fragment dans une hiérarchie de syntagmes}
\label{ch2:fig15}
\end{figure}

Le syntagme \textit{head-fragment-ph} a une seule branche tête, qui correspond à un cluster tel que défini dans la section~\ref{ch2:sect2.5.3.1}. La représentation arborescente du \isi{fragment} et de sa branche \isi{cluster} est donnée en \figref{ch2:fig16}.

\begin{figure} 

\includegraphics[width=\textwidth]{figures/Ch2Fig16.pdf}% \input{figures/ch2_trees.tex}

\caption{Représentation arborescente du fragment et de sa branche cluster}
\label{ch2:fig16}
\end{figure}

Le fragment hérite de sa branche tête (\textit{cluster-ph}) sa catégorie sous-spécifiée (comme le montre le partage de variables correspondant au trait CAT), ce qui lui permet de se combiner avec des foncteurs sélectionnant des catégories non finies, comme c’est le cas de la conjonction \textit{ainsi que} en français, discutée dans la section~\ref{ch2:sect2.5.3.1} (voir exemple \REF{ch2:ex259}). Une représentation simplifiée de la séquence \textit{ainsi que Marie des fraises} en français est donnée en \figref{ch2:fig17}. 

\begin{figure} 

\includegraphics{figures/Ch2Fig17.pdf}% \input{figures/ch2_trees.tex}

\caption{Arbre simplifié pour la séquence \textit{ainsi que Marie des fraises} en français}
\label{ch2:fig17}
\end{figure}

Le \isi{fragment} dans le gapping obéit à la contrainte syntaxique décrite en \REF{ch2:ex271} : les éléments résiduels (figurant sur la liste du trait CLUSTER) doivent unifier leurs traits de tête avec les traits de tête de leurs corrélats dans la phrase source. Pour cela, on utilise le trait contextuel \is{Salient Utterance (SAL-UTT)}SAL-UTT de \citet{GinzburgEtAl2000}\footnote{
 \citet{Ginzburg2012} présente une approche similaire, en utilisant le trait \is{Focus Establishing Constituents (FEC)}FEC (\textit{Focus Establishing Constituents}).}, qui enregistre les éléments corrélats dans la phrase source. De plus, on introduit un trait MAJOR dans la description syntaxique des éléments corrélats dans la phrase source, afin de rendre compte de la contrainte sur les \is{constituant majeur}constituants majeurs de \citet{Hankamer1971}, discutée dans la section~\ref{ch2:sect2.3.3} : chaque corrélat doit dépendre d’un prédicat verbal dans la phrase source (donc, chaque corrélat doit être [MAJOR +]). On postule ici que les corrélats apparaissent sur la structure argumentale d’un prédicat verbal dans la phrase source, sans qu’ils soient nécessairement réalisés syntaxiquement. S’ils ne sont pas réalisés, ils correspondent à des synsems non canoniques. Comme je l’ai précisé dans la section~\ref{ch2:sect2.5.1}, les synsems non canoniques apparaissent sur la structure argumentale d’un prédicat, mais ils ne figurent pas dans sa valence (voir en particulier \REF{ch2:ex247} et \figref{ch2:fig5}). Avec ces contraintes, on peut maintenant rendre compte des exemples de gapping (discutés dans la section~\ref{ch2:sect2.3.4.1}) dans lesquels un des corrélats correspond à un pronom nul (cf. le phénomène de \isi{pro-drop} en roumain \REF{ch2:ex272a}) ou bien est un affixe verbal (\is{affixe/clitique pronominal}pronominal \REF{ch2:ex272b} ou \is{affixe/clitique adverbial}adverbial \REF{ch2:ex272c}).   

\ea \label{ch2:ex271}
Contrainte syntaxique du \textit{head-fragment-ph}\\
\includegraphics{figures/Ch2271.pdf}

% \noindent
% \textit{head-fragment-ph} $\Rightarrow$
% \begin{avm}
% [context | sal-utt \{[head @{H$_1$} \\ major + ],..., [head @{H$_n$} \\ major + ]\} \\
% category | head | cluster <[head & @{H$_1$}],..., [head & @{H$_n$}]>]
% \end{avm}  

\z


\ea \label{ch2:ex272}
\ea 
\gll Lunea  \uline{merg}  la  film,  iar  \textbf{sora} \textbf{mea}  la  muzeu. \label{ch2:ex272a}\\
lundi.\textsc{def} aller.\textsc{prs.1sg} à  film  et  sœur.\textsc{def} \textsc{poss.1sg}  à  musée\\ 
\glt ‘Le lundi, je vais au cinéma, et ma sœur au musée.’

\ex 
\gll Ion  \textbf{mi}{}-\uline{e} prieten,  iar  \textbf{ţie}  duşman. \label{ch2:ex272b}\\ 
Ion  \textsc{dat.1sg}-est  ami  et  toi.\textsc{dat} ennemi\\ 
\glt ‘Ion est mon ami, et pour toi, un ennemi.’

\ex 
\gll Dan  \textbf{tot}  \textbf{mai}  \uline{citeşte},  dar  prietena  lui  \textbf{absolut} \textbf{nimic}. \label{ch2:ex272c}\\
Dan \textsc{adv} \textsc{adv} lit  mais  copine.\textsc{def} \textsc{poss.3sg.m}  absolument  rien\\ 
\glt ‘Dan lit un peu, mais sa copine absolument rien.’
\z
\z

\newpage 
Dans la section~\ref{ch2:sect2.5.2}, on a vu que les entrées lexicales et les syntagmes qu’elles projettent peuvent rester sous-spécifiées quant à leurs traits de tête. Il s’ensuit que les éléments résiduels et leurs corrélats dans les constructions à gapping n’ont pas nécessairement la même catégorie syntaxique, pourvu que le résultat de l’unification de leurs traits de tête soit en accord avec les contraintes de sous-catégorisation du verbe antécédent. Pour illustrer cela, je reprends les deux exemples de \isi{coordination de termes dissemblables} donnés dans la section mentionnée ci-dessus, en les mettant cette fois-ci dans une construction à gapping \REF{ch2:ex273}. Une représentation simplifiée des deux phrases est donnée en \figref{ch2:fig18} et respectivement \figref{ch2:fig19}.  

\ea \label{ch2:ex273}
\ea Ion \uline{este} naiv, iar Gheorghe un imbecil. \label{ch2:ex273a}
\glt ‘Ion est naïf, et Gheorghe un imbécile.’   

\ex 
\gll *Ion  \ulg{a}{14}  întâlnit  un  imbecil,  iar  Gheorghe  naiv. \label{ch2:ex273b}\\
Ion  a  rencontré  un  imbécile  et  Gheorghe  naïf\\ 
\glt ‘Ion a rencontré un imbécile, et Gheorghe un naïf.’ 
\z
\z


\begin{figure} 

\includegraphics[width=\textwidth]{figures/Ch2Fig18.pdf}% \input{figures/ch2_trees.tex}

\caption{Représentation simplifiée de la phrase \REF{ch2:ex273a}}
\label{ch2:fig18}
\end{figure}

Ainsi, en \figref{ch2:fig18} le verbe \textit{a fi} ‘être’ sous-catégorise un complément dont la catégorie est sous-spécifiée ({\cad} le super-type \textit{nominal}, qui regroupe les sous-types \textit{adj} et \textit{noun}). Parallèlement, l’unification des traits de tête de la deuxième paire contrastive ({\cad} HEAD \fbox{3} correspondant au syntagme adjectival \textit{naiv} et HEAD \fbox{4} correspondant au syntagme nominal \textit{un imbecil}) réussit, car il existe un super-type \textit{nominal} commun aux deux éléments (\fbox{4} est donc résolu comme \textit{nominal}). Par conséquent, la \is{coordination de termes dissemblables}coordination des termes dissemblables est possible, car il n’y a aucun désaccord entre les contraintes de sous-catégorisation du verbe attributif et la réalisation effective de ses compléments.

En revanche, l’unification de \fbox{2} et \fbox{4} échoue en \figref{ch2:fig19}, car le verbe \textit{a întâlni} ‘rencontrer’ n’accepte pas un complément sous-spécifié (en particulier, il ne peut pas sélectionner comme complément un syntagme adjectival).

\begin{figure} 

\includegraphics[width=\textwidth]{figures/Ch2Fig19.pdf}% \input{figures/ch2_trees.tex}

\caption{Représentation simplifiée de la phrase \REF{ch2:ex273b}}
\label{ch2:fig19}
\end{figure}

En ce qui concerne la \isi{reconstruction sémantique}, il y a plusieurs possibilités. Une possibilité, entre autres, est celle proposée par \citet{DalrympleEtAl1991} et \citet{Dalrymple2005}, en termes purement sémantiques, pour l’ellipse du verbe dans \is{Verb Phrase Ellipsis (VPE)}VPE. Ils proposent de définir le contenu du \isi{fragment} par l’application au contenu des éléments résiduels d’une fonction F qui résulte de \is{unification d'ordre supérieur}l’unification d’ordre supérieur (U) de deux lambda termes : (i) la représentation sémantique de la phrase source ; (ii) la représentation sémantique résultant de l’application d’une propriété P au contenu des éléments corrélats dans la phrase source. L’illustration de cette approche est donnée en \REF{ch2:ex274}\footnote{
 Voir les critiques de \citet{Ginzburg2012} par rapport à la couverture empirique de cette approche.}. 

\ea \label{ch2:ex274}
\ea John \uline{invited} Sue and Bill Jane.
\ex John invited Sue = invited’(john’, sue’)
\ex {} [F] = U(invited’(john’, sue’), P(john’, sue’)) = λx. λy. invited’(x,y)
\ex Bill Jane = [F][(bill’, jane’)] = λx. λy. [invited’(x,y)](bill’, jane’) 

= invited’(bill’, jane’) 
\z
\z

Une autre possibilité serait d’utiliser le langage \is{Minimal Recursion Semantics (MRS)}\textit{Minimal Recursion Semantics} (abrégé MRS), qu’on utilisera pour les phrases relatives sans verbe dans le chapitre~\ref{ch3}. Je ne me prononce pas sur l’une ou l’autre des pistes. On ajoute tout de même une contrainte sémantique \REF{ch2:ex275} à la définition du \isi{fragment} : le contenu du fragment doit être construit à partir du contenu de la phrase source, des éléments résiduels et des corrélats via une relation \textit{R}\textit{\textsubscript{sem}}. 

\ea \label{ch2:ex275}
Contrainte sémantique du \textit{head-fragment-ph}\\
\includegraphics{figures/Ch2275.pdf}

% \noindent 
% \textit{head-fragment-ph} $\Rightarrow$
% \begin{avm}[context [source & message @{M} \\ sal-utt & \{[content & @{C$_1$}],..., [content & @{C$_n$}]\}]\\
% category | head [\tp{head}\\ cluster & <[content & @{C$_1$'}],..., [content & @{C$_n$'}]>]\\
% content \textit{R}$_{sem}$(@{M}, <@{C$_1$}, @{C$_1$'}>,..., <@{C$_n$}, @{C$_n$'}>)]
% \end{avm} 

\z


\subsubsection{La construction à gapping} \label{ch2:sect2.5.3.3}



Les deux contraintes, syntaxique \REF{ch2:ex271} et sémantique \REF{ch2:ex275}, qu’on a postulées dans la section précédente peuvent être utilisées pour diverses constructions elliptiques (p.ex. pour les types d’ellipse exemplifiés plus haut en \REF{ch2:ex269} pour le roumain et \REF{ch2:ex270} pour le français). Pour rendre compte des propriétés spécifiques des constructions à gapping par rapport à d’autres types d’ellipse, on a besoin d’ajouter une contrainte particulière. Parmi les propriétés spécifiques du gapping dans la coordination, on note les aspects suivants : contrairement aux séquences qu’on peut rencontrer, par exemple, dans les ellipses \isi{comparatives}, la séquence trouée doit suivre la phrase source en roumain et en français ; au niveau discursif, la relation qui s’établit entre les conjoints est toujours une \is{relation discursive}relation symétrique. On postule d’abord un sous-type de syntagme coordonné, appelé \textit{gapping-ph} (cf. la hiérarchie donnée en \figref{ch2:fig20}). 

\begin{figure}
\includegraphics[width=\textwidth]{figures/Ch2Fig20.pdf}% \input{figures/ch2_trees.tex}
\caption{La construction à gapping dans une hiérarchie de syntagmes}
\label{ch2:fig20}
\end{figure}


Le \textit{gapping-ph} combine une liste non vide de phrases verbales (dont la dernière est analysée comme la source) avec une liste non vide de \is{fragment}fragments, chaque fragment contenant au moins deux éléments résiduels. La description de ce nouveau type de syntagme figure en \REF{ch2:ex276}. On s’assure ainsi qu’il y a toujours une phrase source complète, qui détermine l’interprétation de la séquence trouée. Cette contrainte a l’avantage de permettre des constructions à gapping avec plusieurs phrases sources et/ou plusieurs fragments. 

\newpage  
\ea \label{ch2:ex276}
La construction à gapping\\
\includegraphics{figures/Ch2276.pdf}

% \noindent
% \textit{gapping-ph} $\Rightarrow$ \textit{coord-ph} \& \\
% \begin{avm} 
% [head @{H} \textit{verbal}\\
% context | background \{..., \textit{sym-discourse-rel}(@{M$_1$},..., @{M$_j$}, @{M$_{j+1}$},..., @{M$_n$}), ...\}\\
% dtrs <[head @{H} [\tp{verbal}\\
% 									  cluster & elist]\\
% content @{M$_1$}],..., [head @{H} [\tp{verbal}\\
% 									  cluster & elist]\\
% content @{M$_j$}]> \, $\oplus$ \\

% <[head [cluster & <@1,..., @n >]\\ source @{M$_j$}\\ content @{M$_{j+1}$}],..., [head [cluster & <@{1'},..., @{n'} >]\\ source @{M$_j$}\\ content @{M$_{n}$}]>]  
% \end{avm}

\z

Un deuxième aspect important qui découle de cette contrainte concerne la \isi{relation discursive} qui s’établit entre les conjoints d’une construction à gapping. Dans le trait contextuel BACKGROUND, on spécifie qu’il doit exister entre le contenu du \isi{fragment} et le contenu de la phrase source une \isi{relation discursive} symétrique. Cette \isi{relation discursive}, à laquelle on ajoute le contenu de chaque élément résiduel, nous aide à déterminer le contenu de chaque \isi{fragment}. 

Troisièmement, cette contrainte assure que la catégorie de la coordination dans son ensemble est donnée par ses branches non elliptiques ({\cad} les phrases complètes) et non par ses branches fragmentaires. Ainsi, en \REF{ch2:ex276} on observe que le syntagme coordonné et la première liste de branches (qui enregistre les phrases complètes) partagent la même valeur de tête ({\cad} [HEAD \fbox{H}]). On transgresse ainsi la règle générale donnée ci-dessus en \REF{ch2:ex258} s’appliquant par défaut aux coordinations ordinaires, afin d’éviter la sous-spécification de la construction à gapping dans son ensemble, vu le fait que, contrairement aux fragments, le syntagme coordonné en entier a clairement la distribution d’une phrase verbale. A cette analyse, il reste à ajouter une contrainte qui spécifie de manière précise comment on a accès aux éléments corrélats dans la phrase source. 

Les règles postulées jusqu’ici nous permettent de générer les phrases en \REF{ch2:ex277}, dans lesquelles on a trois types \is{asymétrie syntaxique}d’asymétries syntaxiques : les éléments résiduels n’ont pas la même catégorie, pas le même ordre et pas le même nombre. Les arbres associés sont illustrés en \figref{ch2:fig21}, \figref{ch2:fig22} et \figref{ch2:fig23}. 

\ea \label{ch2:ex277}
\ea 
\gll Lunea  \uline{merg}  la  film,  iar  \textbf{sora} \textbf{mea}  la  muzeu. \label{ch2:ex277a}\\
lundi.\textsc{def} aller.\textsc{prs.1sg} à  film  et  sœur.\textsc{def} \textsc{poss.1sg}  à  musée\\ 
\glt ‘Le lundi, je vais au cinéma, et ma sœur au musée.’

\ex 
\gll Ion  \textbf{mi}{}-\uline{e} prieten,  iar  \textbf{ţie}  duşman. \label{ch2:ex277b}\\ 
Ion  \textsc{dat.1sg}-est  ami  et  toi.\textsc{dat} ennemi\\ 
\glt ‘Ion est mon ami, et pour toi, un ennemi.’

\ex 
\gll \textbf{Mai}  \uline{merg} acasă,  (dar)  la  socri  \textbf{niciodată}. \label{ch2:ex277c}\\
\textsc{adv}  aller.\textsc{prs.1sg}  maison.\textsc{adv}  (mais)  chez  beaux-parents  jamais\\ 
\glt ‘Je vais de temps en temps à la maison, mais chez mes beaux-parents jamais.’ 
\z
\z


\begin{sidewaysfigure} 
\includegraphics{figures/Ch2Fig21.pdf}% \input{figures/ch2_trees.tex}
\caption{Arbre simplifié de la phrase \REF{ch2:ex277a}}
\label{ch2:fig21}
\end{sidewaysfigure}


\begin{sidewaysfigure} 

\includegraphics[width=\textwidth]{figures/Ch2Fig22.pdf}% \input{figures/ch2_trees.tex}

\caption{Arbre simplifié de la phrase \REF{ch2:ex277b}}
\label{ch2:fig22}
\end{sidewaysfigure}


\begin{sidewaysfigure} 
\includegraphics{figures/Ch2Fig23.pdf}% \input{figures/ch2_trees.tex}
\caption{Arbre simplifié de la phrase \REF{ch2:ex277c}}
\label{ch2:fig23}
\end{sidewaysfigure}


\section{L’ellipse périphérique gauche : gapping ou coordination de séquences ?} \label{ch2:sect2.6}

Le but de ce chapitre était de décrire les propriétés du gapping en roumain et en français, de démontrer qu’une analyse à base de \isi{reconstruction syntaxique} n’était pas adéquate et de proposer une solution à l’interface syntaxe-sémantique en termes \is{approche constructionnelle}constructionnels, sans postuler \is{effacement}d’effacement, \is{élément vide}d’élément vide ou de \isi{mouvement} ({\cad} une structure syntaxique surfaciste, cf. angl. «~what you see is what you get~»). 

Dans tous les exemples de gapping observés jusqu’ici en roumain, le verbe antécédent dans la phrase source se trouvait en position médiane \REF{ch2:ex278a} et parfois en position finale \REF{ch2:ex278b}. C’est un choix que j’ai fait, car, comme je l’ai déjà précisé tout au début de ce chapitre dans la section~\ref{ch2:sect2.2.2}, ce sont des distributions non ambiguës quant au type de structure envisagée, {\cad} les deux distributions mettent en jeu une coordination de phrases, dont une est complète et l’autre fragmentaire. Par la suite, je ferai référence à ces contextes comme étant du gapping classique, pour le différencier d’une autre occurrence du gapping dans les configurations ayant le verbe antécédent en position initiale.

\ea \label{ch2:ex278}
\ea {}
\gll [Ion  \uline{vine}  azi],  [iar  Maria  mâine]. \label{ch2:ex278a}\\
Ion  vient  aujourd’hui  et  Maria  demain\\
\glt ‘Ion vient aujourd’hui, et Maria demain.’

\ex {}
\gll [Ion AZI \uline{vine}],  [iar  Maria MÂIne]. \label{ch2:ex278b}\\
Ion  aujourd’hui  vient  et  Maria  demain\\
\glt ‘Ion c’est aujourd’hui qu’il vient, et Maria demain.’
\z
\z

Je m’intéresse dans cette section à la distribution restante, {\cad} les structures dans lesquelles le verbe antécédent est en position initiale. Toujours dans la section~\ref{ch2:sect2.2.2}, j’avais mentionné que, en dehors d’une étude empirique des données, les contextes elliptiques avec un verbe en position initiale, comme en \REF{ch2:ex279}, se prêtaient \textit{a priori} à deux analyses. Selon la première possibilité d’analyse \REF{ch2:ex279a}, on coordonne deux phrases : une phrase source qui contient le verbe en position initiale, et une phrase trouée ({\cad} une coordination phrastique). Selon la deuxième analyse \REF{ch2:ex279b}, on coordonne deux séquences de syntagmes dans la portée syntaxique du prédicat verbal, donc il n’y a aucune ellipse dans la structure ({\cad} une coordination sous-phrastique). Ainsi, on peut étiqueter les deux structures possibles comme du gapping dans le premier cas \REF{ch2:ex279a}, ou bien comme une coordination de séquences\footnote{Pour la description des coordinations de «~non-constituants~» se trouvant à droite du verbe tête, on trouve dans la littérature les termes suivants : \is{Conjunction Reduction}\textit{Conjunction Reduction}, \is{Left Peripheral Ellipsis}\textit{Left Peripheral Ellipsis} ou \is{Argument Cluster Coordination (ACC)}\textit{Argument Cluster Coordination}. Je reprends ici le terme utilisé par \citet{Mouret2007,Mouret2008} : coordination de séquences.} dans le deuxième cas \REF{ch2:ex279b}.

\ea \label{ch2:ex279}
\ea {}
\gll [\uline{Vine}  Ion  azi]  [şi  Maria  mâine]. \label{ch2:ex279a}\\
vient  Ion  aujourd’hui  et  Maria  demain\\
\glt ‘Ion vient aujourd’hui, et Maria demain.’

\ex 
\gll \uline{Vine} [Ion azi] [şi Maria mâine]. \label{ch2:ex279b}\\
vient  Ion  aujourd’hui  et  Maria  demain\\
\glt ‘Ion vient aujourd’hui et Maria demain.’
\z
\z

Par ailleurs, la distinction gapping vs. coordination de séquences se fait habituellement en termes de position dans l’arbre syntaxique ; ainsi, en anglais ou en français, on peut dire que les éléments contrastifs dans une séquence à gapping ne sont pas nécessairement au même niveau \REF{ch2:ex280a} et \REF{ch2:ex281a}, tandis que dans une coordination de séquences, les éléments contrastifs sont nécessairement des constituants sœurs \REF{ch2:ex280b} et \REF{ch2:ex281b}. Quant au roumain, on n’a pas d’arguments empiriques pour postuler une structure hiérarchique, et en particulier il n’y a pas de syntagme verbal fini\footnote{
 Voir \citet{Bilbiie2011} pour plus de détails. Parmi les arguments empiriques pour une \isi{structure plate} en roumain (avec le verbe et les dépendants au même niveau), on note ici \is{ordre de mots}l’ordre libre des constituants (les fonctions peuvent être définies indépendamment de la position dans l’arbre syntaxique, grâce à un système assez riche de marques morphosyntaxiques).}, ce qui implique que tous les dépendants du verbe se trouvent au même niveau (y compris le sujet). Je considère donc que la distinction gapping classique vs. coordination de séquences est plutôt une question d’adjacence, {\cad} la \isi{linéarisation} des éléments contrastifs dans la phrase source, par rapport au verbe tête.

\ea \label{ch2:ex280}
\ea We play poker at our house, and bridge at Betsy’s house. \label{ch2:ex280a} 
\ex At our house \uline{we play} poker, and at Betsy’s house, bridge. \label{ch2:ex280b} 
\z
\z

\ea \label{ch2:ex281}
\ea Paul apportera un disque à Marie et un livre à Jean. \label{ch2:ex281a} 
\ex A Marie \uline{Paul apportera} un disque et à Jean un livre. \label{ch2:ex281b} 
\z
\z

Les mêmes configurations avec le placement médian ou initial du verbe antécédent se retrouvent dans les subordonnées : en \REF{ch2:ex282a}, on a un exemple typique de gapping, alors qu’en (\ref{ch2:ex282b}--\ref{ch2:ex282c}), le verbe est suivi de deux séquences de syntagmes pour lesquelles on peut supposer deux possibilités d’analyse.

\ea \label{ch2:ex282}
\ea 
\gll Vreau  ca  [Ion  \ulg{să}{11}  vină  azi]  [şi  Maria  mâine]. \label{ch2:ex282a}\\
vouloir.\textsc{prs.1sg}  que  Ion  \textsc{sbjv}  venir.\textsc{sbjv.3}  aujourd’hui  et  Maria  demain\\
\glt ‘Je veux que Ion vienne aujourd’hui, et Maria demain.’

\ex 
\gll Vreau  [\ulg{să}{10}  vină  Ion  azi]  [şi  Maria  mâine]. \label{ch2:ex282b}\\
vouloir.\textsc{prs.1sg}  \textsc{sbjv}  venir.\textsc{sbjv.3}  Ion  aujourd’hui  et  Maria  demain\\
\glt ‘Je veux que Ion vienne aujourd’hui, et Maria demain.’

\ex 
\gll Vreau să vină [Ion azi] [şi Maria mâine]. \label{ch2:ex282c}\\
vouloir.\textsc{prs.1sg}  \textsc{sbjv}  venir.\textsc{sbjv.3}  Ion  aujourd’hui  et  Maria  demain\\
\glt ‘Je veux que Ion vienne aujourd’hui, et Maria demain.’
\z
\z

L’hypothèse selon laquelle il y a deux possibilités d’analyse pour les coordinations de séquences se trouvant à droite du verbe est justifiée, entre autres, par le placement des conjonctions corrélatives dans les \is{coordination omnisyndétique (ou corrélative)}coordinations omnisyndétiques. Ainsi, dans un contexte typique de gapping, où le verbe se trouve en position médiane, la conjonction initiale doit obligatoirement précéder la phrase source, comme en \REF{ch2:ex283}, alors que, dans le cas où le verbe précède les séquences, la conjonction initiale peut apparaître soit à l’initiale du verbe tête \REF{ch2:ex284a}, soit à l’initiale de la coordination de séquences \REF{ch2:ex284b}.

\ea \label{ch2:ex283}
\ea \textbf{Fie} Ion \uline{vine} azi, \textbf{fie} Maria mâine. \label{ch2:ex283a}
\glt  ‘Soit Ion vient aujourd’hui, soit Maria demain.’

\ex
\gll *Ion \uline{vine}  \textbf{fie}  azi,  \textbf{fie}  Maria  mâine. \label{ch2:ex283b}\\
Ion  vient  soit  aujourd’hui  soit  Maria  demain\\
\glt ‘Soit Ion vient aujourd’hui, soit Maria demain.’
\z
\z


\ea \label{ch2:ex284}
\ea 
\gll \textbf{Fie} \uline{vine}  Ion  azi,  \textbf{fie}  Maria  mâine. \label{ch2:ex284a}\\
soit  vient  Ion  aujourd’hui  soit  Maria  demain\\
\glt ‘Soit Ion vient aujourd’hui, soit Maria demain.’

\ex 
\gll \uline{Vine} \textbf{fie} Ion  azi,  \textbf{fie}  Maria  mâine. \label{ch2:ex284b}\\
vient  soit  Ion  aujourd’hui  soit  Maria  demain\\
\glt ‘Soit Ion vient aujourd’hui, soit Maria demain.’
\z
\z

Les coordinations de séquences (à tête initiale) m’intéressant ici concernent non seulement les séquences avec sujet postverbal, mais aussi les séquences typiques de compléments/ajouts. Comme on vient de le voir pour l’anglais en \REF{ch2:ex280} et le français en \REF{ch2:ex281}, le roumain permet deux \is{linéarisation}linéarisations : soit les deux séquences sont à droite du verbe tête (p.ex. \REF{ch2:ex285a} et \REF{ch2:ex286a}), soit un des compléments/ajouts est préverbal (p.ex. \REF{ch2:ex285b} et \REF{ch2:ex286b}).  

\ea \label{ch2:ex285}
\ea 
\gll \ulg{I-am}{14.8}  dat  Ioanei  o  carte,  iar  Mariei  un  stilou. \label{ch2:ex285a}\\
\textsc{dat.3sg}{}-ai  donné  Ioana.\textsc{dat} un  livre  et Maria.\textsc{dat}  un  stylo\\
\glt ‘J’ai donné à Ioana un livre, et à Maria un stylo.’

\ex 
\gll Ioanei  \ulg{i-am}{14.8}  dat  o  carte,  iar  Mariei  un  stilou. \label{ch2:ex285b}\\
Ioana.\textsc{dat} \textsc{dat.3sg}{}-ai  donné  un  livre  et  Maria.\textsc{dat}  un  stylo\\
\glt ‘A Ioana j’ai donné un livre, et à Maria un stylo.’
\z
\z


\ea \label{ch2:ex286}
\ea 
\gll \ulg{Am}{8.5} fost  în  2004  la  Roma,  iar  în  2005  la  Londra. \label{ch2:ex286a}\\
ai  été  en  2004  à  Rome  et  en  2005  à  Londres\\
\glt ‘J’ai été en 2004 à Rome, et en 2005 à Londres.’

\ex 
\gll In 2004, \ulg{am}{8.5} fost  la  Roma,  iar  în  2005  la  Londra. \label{ch2:ex286b}\\
en  2004,  ai  été  à  Rome  et  en  2005  à  Londres\\
\glt ‘En 2004, j’ai été à Rome, et en 2005 à Londres.’ 
\z
\z

Dans ce qui suit, je m’intéresse donc aux coordinations de séquences (sujet, complément et ajout tout confondu) se trouvant à droite du verbe tête. Le but est de vérifier si ce type de structures se comporte comme une coordination de phrases (dont une fragmentaire) comme en \figref{ch2:fig24a} ou bien comme une coordination (sous-phrastique) de séquences dans la portée syntaxique du verbe tête, comme en \figref{ch2:fig24b}. 

\begin{figure} 

\includegraphics{figures/Ch2Fig24a.pdf}% \input{figures/ch2_trees.tex}

\caption{Coordination de phrases}
\label{ch2:fig24a}
\end{figure}


\begin{figure} 

\includegraphics{figures/Ch2Fig24b.pdf}% \input{figures/ch2_trees.tex}

\caption{Coordination sous-phrastique de  clusters}
\label{ch2:fig24b}
\end{figure}

Je commence par mentionner brièvement les arguments empiriques à l’encontre d’une \isi{reconstruction syntaxique} dans ce type de structures (section~\ref{ch2:sect2.6.1}). Par conséquent, les deux autres possibilités restantes sont exactement celles que je viens de postuler ci-dessus : (i) une analyse similaire à celle proposée pour les constructions à gapping, {\cad} une structure fragmentaire sans tête verbale, dont la bonne formation réside dans un principe de \isi{substitution}, ou bien (ii) une analyse sans ellipse, {\cad} une coordination de séquences (ou \is{cluster}clusters), qui satisfait les exigences de sous-catégorisation d’un prédicat comme une suite de constituants ordinaires. La première solution envisageable a été déjà discutée pour le gapping en roumain et en français dans la section~\ref{ch2:sect2.5}. C’est pour cela que, dans la deuxième partie de cette section, je discuterai plutôt l’analyse sans ellipse, proposée par \citet{Mouret2006,Mouret2007,Mouret2008} pour la coordination de séquences en français (section~\ref{ch2:sect2.6.2}). Ensuite, dans une troisième partie (section~\ref{ch2:sect2.6.3}), je motiverai sur une base strictement empirique le besoin de postuler les deux analyses ({\cad} gapping vs. coordination de séquences) pour rendre compte des différences qui existent entre les coordinations avec \textit{iar} ‘et’ et les coordinations avec la conjonction \textit{şi} ‘et’. Dans la dernière section, je présente la formalisation de ces données dans le cadre HPSG (section~\ref{ch2:sect2.6.4}).    


\subsection{Pas de reconstruction syntaxique} \label{ch2:sect2.6.1}


Une approche syntaxique de l’ellipse (cf. \citealt{vanOirsouw1987,Wilder1997,Crysmann2003,BeaversEtAl2004,ChavesEtAl2008}, etc.), qui postule la présence (sous une forme ou autre) d’un verbe dans le second conjoint, est inadéquate, tout comme pour les constructions à gapping observées dans les sections précédentes. Contrairement à ce qui est prédit par le \isi{principe de récupérabilité de l’ellipse} (cf. \citealt{Chomsky1964}), on n’arrive pas toujours à restituer une tête verbale dans la coordination de séquences à droite d’un verbe. Les mêmes arguments qu’on a discutés pour le gapping avec le verbe en position non initiale valent ici. 

La reconstruction du matériel supposé présent dans le second conjoint ne donne pas toujours lieu à une phrase grammaticale. Ainsi, certains connecteurs lexicalisés comme \textit{ca (şi)} ‘comme’ en \REF{ch2:ex287a}, \textit{precum şi} ‘ainsi que’ en \REF{ch2:ex287b} ne peuvent jamais se combiner avec un verbe fini.

\ea \label{ch2:ex287}
\ea 
\gll \ulg{S-a}{25.4}  băgat  şi  Ion  în  discuţie  \textbf{ca}  (*s-a  băgat) musca  (*s-a  băgat)  în  lapte. \label{ch2:ex287a}\\ 
\textsc{refl.3sg-aux} introduit  aussi  Ion  en  discussion  comme  (\textsc{refl.3sg-aux} introduit) mouche.\textsc{def} (\textsc{refl.3sg-aux} introduit) en  lait\\
\glt ‘Ion s’est mêlé à la conversation sans être concerné (comme une mouche dans le lait).’

\ex  
\gll \ulg{I-am}{14.7}  dat  lui  Ion  un  măr,  \textbf{precum  şi}  (*i-am  dat) Mariei  (*i-am  dat)  o  pară. \label{ch2:ex287b}\\ 
\textsc{dat.3sg-}ai  donné  \textsc{dat}  Ion  une  pomme  {ainsi  que}  (\textsc{dat.3sg-}ai  donné) Maria.\textsc{dat}  (\textsc{dat.3sg-}ai  donné)  une  poire\\
\glt ‘J’ai donné à Ion une pomme, ainsi qu’à Maria une poire.’
\z
\z

Le même contraste est observé avec la \isi{négation} de constituant \textit{nu} en \REF{ch2:ex288a} qui se comporte différemment de la \isi{négation} de phrase \textit{nu} (cf. \citealt{Barbu2004}), bien qu’elles soient homonymes en roumain. Contrairement à la \isi{négation} de phrase dans le contexte verbal, la \isi{négation} de constituant reçoit toujours une \isi{saillance prosodique}, ce qui explique les différences observées en \REF{ch2:ex288}. En français et anglais, ce contraste est plus évident : des adverbes comme fr. \textit{non pas} en \REF{ch2:ex289} ou angl. \textit{not} en \REF{ch2:ex290} peuvent introduire une séquence, alors qu’ils sont exclus avec une forme verbale finie (cf. \citealt{Mouret2006}, \citealt{Mouret2007}, \citealt{Mouret2008} pour le français, et \citealt{CulicoverEtAl2005} pour l’anglais).

\ea \label{ch2:ex288}
\ea 
\gll \ulg{I-am}{14.6}  dat  MaRIei  o  pară  şi  \{\textbf{NU} {\textbar} \textbf{nicideCUM}\}  (*i-am dat)  lui  Ion  un  măr. \label{ch2:ex288a}\\  
\textsc{dat.3sg}{}-ai  donné  Maria.\textsc{dat}  une  poire  et  \{non-pas {\textbar} pas-du-tout\}  (\textsc{dat.3sg}{}-ai donné) \textsc{dat}  Ion  une  pomme\\
\glt ‘J’ai donné à Maria une poire, mais pas à Ion une pomme.’

\ex 
\gll ??I-am  dat  Mariei  o  pară  şi  \textbf{nu}  i-am  dat  lui  Ion  un măr. \label{ch2:ex288b}\\
\textsc{dat.3sg}{}-ai donné  Maria.\textsc{dat}  une  poire  et \textsc{neg} \textsc{dat.3sg}{}-ai donné \textsc{dat}  Ion  une pomme\\
\glt ‘J’ai donné à Maria une poire, mais je n’ai pas donné à Ion une pomme.’
\z
\z


\ea \label{ch2:ex289}
Paul \uline{offrira} un disque à Marie et \textbf{non pas} (*offrira) un livre à Jean. 
\z

\ea \label{ch2:ex290}
Paul \uline{gave} a record to Mary and \textbf{not} (*gave) a book to Bill.
\z

La \isi{reconstruction syntaxique} d’un verbe est problématique aussi dans le listage de paires avec \is{accord}«~accord cumulatif~» \REF{ch2:ex291}, lorsque le verbe initial reçoit un \isi{accord} au pluriel, et ce quelle que soit la valeur de nombre de chacun des sujets postverbaux (ici, le premier sujet postverbal est au singulier). 

\ea \label{ch2:ex291}
\gll In  prima  zi  \ulg{se}{19} {iau :}  o  picătură dimineaţa,  două picături  la  prânz,  trei  picături  seara.\\ 
en  premier.\textsc{def}  jour  \textsc{refl.acc.3} prendre.\textsc{prs.3pl}  une  goutte  matin.\textsc{def}  deux gouttes  à  midi  trois  gouttes  soir.\textsc{def}\\
\glt ‘Le premier jour, on doit prendre une goutte le matin, deux gouttes à midi, trois gouttes le soir.
\z

Si on assume une \isi{reconstruction syntaxique} «~parallèle~», {\cad} on reconstruit un verbe dans le second conjont au même endroit que dans le premier conjoint, on n’arrive pas facilement à rendre compte de la distribution idiosyncrasique imposée par la conjonction \textit{iar} ‘et’ en \REF{ch2:ex292}, qui ne permet pas une forme verbale finie prédicative dans la position initiale du deuxième conjoint.

\ea \label{ch2:ex292}
\ea 
\gll Vreau  \ulg{să}{11}  vină  Ion  azi,  \textbf{iar}  (*să  vină)  Maria (să  vină)  mâine.\\
vouloir.\textsc{prs.1sg}  \textsc{sbjv}  venir.\textsc{sbjv.3}  Ion  aujourd’hui  et  (\textsc{sbjv}  venir.\textsc{sbjv.3})  Maria (\textsc{sbjv}  venir.\textsc{sbjv.3)} demain\\
\glt ‘Je veux que Ion vienne aujourd’hui, et Maria demain.’

\ex 
\gll \ulg{I-am}{14.7}  dat  Mariei  o  carte,  \textbf{iar}  (*i-am  dat)  lui  Ion (i-am  dat)  un  CD.\\
\textsc{dat.3sg}{}-ai  donné  Maria.\textsc{dat}  un  livre  et  (\textsc{dat.3sg}{}-ai  donné)  \textsc{dat}  Ion (\textsc{dat.3sg}{}-ai  donné)  un  CD\\
\glt ‘J’ai donné à Maria un livre, et à Ion un CD.’
\z
\z

Un exemple plus délicat est \REF{ch2:ex293} où, en dehors du fait que le matériel manquant ne peut être reconstruit qu’en position médiane (à cause des contraintes imposées par la conjonction \textit{iar} ‘et’ sur le placement du verbe prédicatif dans une phrase), on observe une \is{asymétrie syntaxique}asymétrie liée à la catégorie syntaxique, au cas et à la fonction syntaxique de la paire <\textit{pisicii}, \textit{la pui}> <‘au chat’, ‘aux chatons’> : syntagme nominal vs. syntagme prépositionnel, génitif vs. accusatif avec préposition, complément du nom déverbal vs. ajout phrastique.

\ea \label{ch2:ex293}
\gll \ulg{Se}{66}  recomandă deparazitarea \textbf{pisicii} de  patru  ori pe  an,  iar \textbf{la} \textbf{pui} (se  recomandă  deparazitarea)  o  dată pe  lună.\\
\textsc{refl.acc.3} recommander.\textsc{prs.3} vermifugation.\textsc{def} chat.\textsc{gen} de  quatre  fois sur  an  et  chez  chatons (\textsc{refl.acc.3} recommander.\textsc{prs.3} vermifugation.\textsc{def})  une  fois sur  mois.\\ 
\glt ‘Il est recommandé de vermifuger le chat quatre fois par an, et les chatons une fois par mois.’
\z

Généralement, les approches syntaxiques de l’ellipse postulent un homomorphisme syntaxe-sémantique, {\cad} les relations de portée sémantiques dérivent directement de la structure syntaxique. Ce genre d’analyse rencontre des difficultés quant à l’interprétation de certains éléments ou la portée de certains opérateurs sémantiques sur la coordination de séquences. Ainsi, si on postule la reconstruction du matériel manquant dans le deuxième conjoint, on n’arrive pas à dériver la lecture interne («~référentiellement dépendante~», cf. \citealt{LacaEtAl2001}) des adjectifs relationnels \textit{acelaşi} ‘même’ et \textit{diferit} ‘différent’\footnote{
 Ces adjectifs présentent aussi une lecture anaphorique ou externe, dans laquelle le second terme de la relation d’identité ou de la non-identité est à récupérer dans le contexte gauche, {\cad} il a été déjà mentionné dans le discours. Cette lecture ne nous intéresse pas ici.}, dans laquelle les arguments de la relation d’identité ou de la non-identité se trouvent dans la phrase même (\ref{ch2:ex294}--\ref{ch2:ex295}). Cette lecture interne va de pair avec une interprétation à la fois distributive et réciproque (cf. \citealt{VanPeteghem2002}), les entités en question fonctionnant comme les arguments d’un même prédicat (dyadique) à sens réciproque. Afin d’obtenir cette lecture interne, les syntagmes nominaux modifiés par ces \is{adjectif relationnel}adjectifs relationnels (p.ex. \textit{acelaşi articol} ‘le même article’ en \REF{ch2:ex294} ou \textit{un cadou diferit} ‘un cadeau différent’ en \REF{ch2:ex295}) doivent pouvoir s’appliquer à une \isi{éventualité plurielle} (cf. \citealt{Carlson1987}). S’ils se combinent avec une éventualité singulière, cette lecture interne devient impossible, ce qui explique l’inacceptabilité des exemples \REF{ch2:ex294b} et \REF{ch2:ex295b}, où on reconstruit le verbe avec son complément manquant dans le deuxième conjoint.

\ea \label{ch2:ex294}
\ea \uline{Am prezentat} \textbf{acelaşi} articol ieri la curs şi azi la seminar. \label{ch2:ex294a}
\glt ‘J’ai présenté le même article hier en cours et aujourd’hui en séminaire.’

\ex 
\gll \#Am  prezentat  \textbf{acelaşi}  articol  ieri  la  curs  şi  am  prezentat  \textbf{acelaşi}  articol azi  la  seminar. \label{ch2:ex294b}\\
ai  présenté  même  article  hier  à  cours  et  ai  présenté  même  article aujourd’hui  à  séminaire\\
\glt ‘J’ai présenté le même article hier en cours et aujourd’hui en séminaire.’
\z
\z

\largerpage
\ea \label{ch2:ex295}
\ea 
\gll \ulg{Voi}{44.4}  cumpăra  un  cadou  \textbf{diferit}  pentru  Ion  de  Crăciun  şi  pentru  Ana  de  Paşte. \label{ch2:ex295a}\\
vais  acheter  un  cadeau  différent  pour  Ion  de  Noël  et  pour  Ana  de  Pâques\\
\glt ‘Je vais acheter un cadeau différent à Ion pour Noël et à Maria pour Pâques.’

\ex  
\gll \#Voi  cumpăra  un  cadou  \textbf{diferit}  pentru  Ion  de  Crăciun  şi  voi  cumpăra  un cadou  \textbf{diferit}  pentru  Ana  de  Paşte. \label{ch2:ex295b}\\
vais  acheter  un  cadeau  différent  pour  Ion  de  Noël  et  vais  acheter  un cadeau  différent  pour  Ana  de  Pâques\\
\glt ‘Je vais acheter un cadeau différent à Ion pour Noël et à Ana pour Pâques.’
\z
\z

Toujours au niveau sémantique, on remarque des non-équivalences entre les exemples «~elliptiques~» et les exemples avec \isi{reconstruction syntaxique} concernant la portée de la conjonction \textit{şi} ‘and’ en \REF{ch2:ex296} ou de la disjonction \textit{sau} ‘ou’ en \REF{ch2:ex297} par rapport à la \isi{négation} accompagnant le verbe initial (voir \citealt{HuddlestonEtAl2002} pour des données similaires en anglais). Ainsi dans les contextes «~elliptiques~» en \REF{ch2:ex296a} et \REF{ch2:ex297a}, la conjonction \textit{şi} ‘et’ et la disjonction \textit{sau} ‘ou’ ont une \isi{portée étroite} par rapport à la \isi{négation} verbale, qui a \isi{portée large}, alors que dans les versions reconstruites en \REF{ch2:ex296b} et \REF{ch2:ex297b}, la conjonction et la disjonction ont une \isi{portée large}.   

\ea \label{ch2:ex296}
\ea \uline{Nu pot să-i dau} lui Ion o bicicletă \textbf{şi} Mariei doar un stilou. \label{ch2:ex296a}
\glt ‘Je ne peux pas donner à Ion une bicyclette et à Maria seulement un stylo.’

\ex ≠ Nu pot să-i dau lui Ion o bicicletă \textbf{şi} nu pot să-i dau Mariei doar un stilou. \label{ch2:ex296b}
\glt ‘Je ne peux pas donner à Ion une bicyclette et je ne peux pas donner à Maria seulement un stylo.’ 
\z
\z


\ea \label{ch2:ex297}
\ea \uline{Nu i-am dezvăluit nimic} lui Paul despre Maria \textbf{sau} Mariei despre Paul. \label{ch2:ex297a}
\glt ‘Je n’ai rien dévoilé à Paul concernant Maria ou à Maria concernant Paul.’

\ex ≠ Nu i-am dezvăluit nimic lui Paul despre Maria \textbf{sau} nu i-am dezvăluit nimic Mariei despre Paul. \label{ch2:ex297b}
\glt ‘Je n’ai rien dévoilé à Paul concernant Maria ou je n’ai rien dévoilé à Maria concernant Paul.’ 
\z
\z

Sur la base de ces arguments, on doit conclure que la \isi{reconstruction syntaxique} n’opère pas non plus dans la coordination de séquences à droite d’un verbe tête, indépendamment de la conjonction utilisée dans ces coordinations.

Il reste maintenant à vérifier si l’on a des arguments empiriques pour désambiguïser ces coordinations avec le verbe en position initiale. En particulier, je veux voir si, dans ce type de configurations, il s’agit d’une coordination d’unités avec contenu propositionnel (donc, une coordination de phrases, dont une fragmentaire, comme dans les constructions à gapping) ou bien s’il s’agit d’une coordination sous-phrastique de séquences dans la portée syntaxique d’un prédicat. Avant d’étudier les données du roumain, je présente d’abord l’analyse proposée par \citet{Mouret2006,Mouret2007,Mouret2008} pour la coordination de séquences en français.


\subsection{La coordination de séquences en français} \label{ch2:sect2.6.2}


Dans cette section, je présente brièvement les arguments mentionnés par \citet{Mouret2006,Mouret2007,Mouret2008} pour une analyse sans ellipse de la coordination de séquences en français, qui pour lui est une coordination sous-phrastique. 

L’argument majeur contre une coordination phrastique (et donc contre une structure fragmentaire du deuxième conjoint) en français est la distribution des conjonctions corrélatives (ou doubles) dans les \is{coordination omnisyndétique (ou corrélative)}coordinations omnisyndétiques. Si la coordination de séquences présente des conjonctions doubles comme \textit{et...} \textit{et...} ou \textit{ou bien...} \textit{ou bien...}, la conjonction initiale se place obligatoirement après le prédicat partagé (p.ex. \REF{ch2:ex298a} et \REF{ch2:ex299a}) et non devant celui-ci (p.ex. \REF{ch2:ex298b} et \REF{ch2:ex299b}), ce qui s’explique si on admet que le prédicat verbal est extérieur à la structure coordonnée. 

\ea \label{ch2:ex298}
\ea Paul compte \uline{apporter} \textbf{et} un disque à Marie \textbf{et} un livre à Jean. \label{ch2:ex298a}
\ex *Paul compte \textbf{et} \uline{apporter} un disque à Marie \textbf{et} un livre à Jean. \label{ch2:ex298b} 
\z
\z

\ea \label{ch2:ex299}
\ea \uline{Paul apportera} \textbf{ou bien} un disque à Marie \textbf{ou bie}n un livre à Jean. \label{ch2:ex299a} 
\ex *\textbf{Ou bien} \uline{Paul apportera} un disque à Marie \textbf{ou bien} un livre à Jean. \label{ch2:ex299b} 
\z
\z

Un argument supplémentaire qui confirme cette hypothèse est la distribution et l’interprétation des \is{adverbe associatif}adverbes restrictifs et additifs. Ces adverbes peuvent introduire une coordination de séquences et s’y associer sémantiquement. On observe ainsi que les adverbes \textit{seulement} en \REF{ch2:ex300a} et \textit{aussi} en \REF{ch2:ex300b} s’associent à toute la coordination de séquences et non seulement au premier conjoint. Ces phénomènes d’association sont problématiques pour les approches postulant une structure fragmentaire pour le deuxième conjoint, car on n’arrive pas à expliquer comment un adverbe peut prendre la coordination dans son ensemble comme associé sémantique s’il est enchâssé dans le premier conjoint.

\ea \label{ch2:ex300} 
\ea Paul \uline{offrira} \textbf{seulement} <un disque à Pierre et un livre à Marie>, alors qu’il aurait pu offrir aussi une bouteille de vin à Jean. \label{ch2:ex300a} 
\ex Paul \uline{offrira} \textbf{aussi} <un disque à Pierre et un livre à Marie>, alors qu’il aurait pu offrir seulement une bouteille de vin à Jean. \label{ch2:ex300b}  
\z
\z

Un dernier argument pour une analyse sans ellipse concerne certains phénomènes \is{accord}d’accord observés avec les sujets postverbaux en français. Ainsi, dans les tours narratifs, deux stratégies d’accord sont possibles lorsqu’on coordonne des séquences comportant chacune un sujet postverbal. Si la coordination est interprétée comme la conjonction de deux événements successifs (cf. la présence de l’adverbial \textit{quelques secondes plus tard} en \REF{ch2:ex301a}), le verbe apparaît au singulier, s’accordant indépendamment avec chacun des sujets postverbaux. En revanche, si la coordination met en jeu une relation symétrique (cf. la présence de l’adverbe \textit{simultanément} en \REF{ch2:ex301b}), le verbe apparaît obligatoirement au pluriel, peu importe la valeur de nombre de chacun des sujets postverbaux. Cette deuxième stratégie est problématique pour toute structure à ellipse, car on n’arrive pas à expliquer comment un verbe peut apparaître au pluriel si celui-ci appartient uniquement au premier conjoint. En revanche, cela s’explique facilement si on postule une analyse sans ellipse, avec un prédicat verbal à l’extérieur de la séquence coordonnée. 

\ea \label{ch2:ex301}
\ea Alors \{\uline{surgit} {\textbar} *surgirent\} d’un champ un renard et \textbf{quelques secondes plus tard} d’un buisson une biche. \label{ch2:ex301a} 
\ex Alors \{*surgit {\textbar} \uline{surgirent}\} \textbf{simultanément} d’un champ un renard et d’un buisson une biche. \label{ch2:ex301b} 
\z
\z

Sur la base de ces propriétés, \citet{Mouret2006,Mouret2007,Mouret2008} conclut que la coordination de séquences en français ne met en jeu aucune ellipse, son analyse se rapprochant de celles proposées en \is{grammaire catégorielle}grammaires catégorielles \citep{Dowty1988,Steedman1990,Steedman2000} ou dans d’autres cadres surfacistes \citep{Hudson1988,MaxwellEtAl1996}. On aura ainsi une coordination sous-phrastique de séquences sans tête qui seront autorisées dans la portée syntaxique d’un verbe tête (plus de détails dans la section~\ref{ch2:sect2.6.4}).


\subsection{Une double analyse en roumain} \label{ch2:sect2.6.3}


On a vu que la \isi{reconstruction syntaxique} posait des problèmes pour les coordinations de séquences en roumain. En revanche, une analyse sans ellipse se justifie facilement pour les coordinations de séquences en français, où l’on considère que la coordination opère à un niveau sous-phrastique dans ces cas. Qu’en est-il pour le roumain ?

Fondamentalement, on observe que le roumain permet l’emploi de la conjonction \textit{iar} ‘et’ dans ces contextes, qu’il s’agisse d’une séquence contenant un sujet postverbal \REF{ch2:ex302a} ou non \REF{ch2:ex302b}. Or, la conjonction \textit{iar} ‘et’ ne lie que des syntagmes avec contenu propositionnel, {\cad} des phrases. Cela semble suggérer que l’analyse proposée pour le français ne peut pas s’appliquer a priori aux coordinations avec \textit{iar}, pour lesquelles une approche similaire à celle proposée pour les constructions typiques de gapping serait plus adéquate.  

\ea \label{ch2:ex302}
\ea 
\gll Vreau  \ulg{să}{11}  vină  Ion  azi,  \textbf{iar}  Maria  mâine. \label{ch2:ex302a}\\
vouloir.\textsc{prs.1sg}  \textsc{sbjv}  venir.\textsc{sbjv.3}  Ion  aujourd’hui  et  Maria  demain\\
\glt ‘Je veux que Ion vienne aujourd’hui, et Maria demain.’

\ex  
\gll \ulg{I-am}{15} dat  Mariei  o  carte,  \textbf{iar}  Anei  un  CD. \label{ch2:ex302b}\\
\textsc{dat.3sg}{}-ai  donné  Maria.\textsc{dat}  un  livre  et  Ana.\textsc{dat}  un  CD\\
\glt ‘J’ai donné à Maria un livre, et à Ana un CD.’
\z
\z

En même temps, les coordinations de séquences en roumain peuvent être liées par la conjonction \textit{şi} ‘et’, qui n’est pas contrainte quant au type de catégorie coordonnée. On s’attend donc à ce que les propriétés mentionnées par \citet{Mouret2006,Mouret2007,Mouret2008} pour le français s’appliquent aux coordinations de séquences liées par la conjonction \textit{şi}, mais pas à celles coordonnées par \textit{iar}.

Cette hypothèse est confirmée par les différences de comportement qu’on observe dans les coordinations de séquences liées par \textit{iar}, par rapport à celles liées par \textit{şi}. Par la suite, j’applique une série de tests (y compris ceux proposés par Mouret pour le français) pour montrer qu’une double analyse doit être envisagée pour les coordinations de séquences en roumain.

Bien que ce ne soit pas un test décisif comme en français, je commence par discuter le placement des items corrélatifs\footnote{
 Je les appelle \textit{items corrélatifs} et non \textit{conjonctions doubles}, car en roumain on a, à côté des conjonctions doubles (\textit{fie... fie...} ‘soit... soit...’, \textit{sau... sau...} ‘ou... ou...’, \textit{ori... ori...} ‘ou... ou...’), des adverbes corrélatifs (\textit{şi... şi...} ‘et... et...’ et \textit{nici... nici...} ‘ni... ni...’). Pour une analyse détaillée de ces constructions, voir \citet{Bilbiie2008,Bilbiie2011}.}. On observe une différence nette entre les coordinations de séquences avec des adverbes corrélatifs (p.ex. \textit{şi... şi...} ‘et... et...’) et les coordinations avec des conjonctions doubles (p.ex. \textit{fie... fie...} ‘soit... soit...’) : l’adverbe corrélatif \textit{şi}\footnote{
 A ne pas confondre la conjonction \textit{şi} et l’adverbe corrélatif \textit{şi}, qui, malgré l’homonymie de forme, ne partagent pas les mêmes propriétés distributionnelles \citep{Bilbiie2008}.} se place obligatoirement après le prédicat \REF{ch2:ex303}, alors que la conjonction double \textit{fie} peut apparaître aussi devant le prédicat \REF{ch2:ex304}. On ne peut pas tester le placement des items corrélatifs avec la conjonction \textit{iar}, car il n’y a pas de structure corrélative disponible. Mais on peut déjà constater qu’il y a un double comportement des \is{coordination omnisyndétique (ou corrélative)}coordinations omnisyndétiques, ce qui va dans le sens de mon hypothèse.

\ea \label{ch2:ex303}
\ea 
\gll \ulg{I-am}{14.6}  dat  \textbf{şi}  Mariei  o  pară  (şi)  \textbf{şi}  lui  Ion  un  măr. \label{ch2:ex303a}\\
\textsc{dat.3sg}{}-ai  donné  \textsc{adv}  Maria.\textsc{dat}  une  poire,  (et)  \textsc{adv}  \textsc{dat}  Ion  une  pomme\\
\glt ‘J’ai donné et à Maria une poire, et à Ion une pomme.’

\ex 
\gll *\textbf{Şi}  \ulg{i-am}{14.8}  dat  Mariei  o  pară,  (şi)  \textbf{şi}  lui  Ion  un  măr. \label{ch2:ex303b}\\
\textsc{adv} \textsc{dat.3sg}{}-ai  donné  Maria.\textsc{dat}  une  poire  (et)  \textsc{adv}  \textsc{dat}  Ion  une  pomme\\
\glt ‘J’ai donné et à Maria une poire, et à Ion une pomme.’ 
\z
\z


\ea \label{ch2:ex304}
\ea 
\gll \uline{Vine}  \textbf{fie}  Ion  azi,  \textbf{fie}  Maria  mâine. \label{ch2:ex304a}\\
 vient  soit  Ion  aujourd’hui  soit  Maria  demain\\
\glt ‘Soit Ion vient aujourd’hui, soit Maria demain.’

\ex 
\gll \textbf{Fie} \uline{vine}  Ion  azi,  \textbf{fie}  Maria  mâine. \label{ch2:ex304b}\\
soit  vient  Ion  aujourd’hui  soit  Maria  demain\\
\glt ‘Soit Ion vient aujourd’hui, soit Maria demain.’
\z
\z

En revanche, le deuxième argument invoqué par \citet{Mouret2006,Mouret2007,Mouret2008} est un test qui montre bien la différence entre les structures avec la conjonction \textit{şi} ‘et’ et les structures avec la conjonction \textit{iar} ‘et’. Ainsi, un \isi{adverbe associatif} comme le restrictif \textit{doar} ‘seulement’ \REF{ch2:ex305a}, \textit{numai} ‘seulement’ \REF{ch2:ex305b} ou \textit{decât} ‘seulement’ \REF{ch2:ex305c} peut avoir \isi{portée large} sur la coordination (donc, il prend facilement la coordination dans son ensemble comme associé sémantique) si on a la conjonction \textit{şi}, alors que cela est inacceptable avec la conjonction \textit{iar}. Cette \isi{portée large} des adverbes restrictifs avec la conjonction \textit{şi} permet dans ces cas un \is{affixe/clitique pronominal}pronom clitique pluriel sur le verbe partagé, comme le clitique pluriel \textit{le} ‘leur’ en \REF{ch2:ex305a}. 

\ea \label{ch2:ex305}
\ea 
\gll Ion  \{îi  {\textbar}  le\}  \uline{dă} \textbf{doar}  [Anei  un  stilou  \{\textbf{şi} {\textbar} \textbf{\#iar}\}  Inei o  carte],  deşi  ar  putea  să-i  dea  şi Danei  un  CD. \label{ch2:ex305a}\\
Ion  \{\textsc{dat.3sg} {\textbar} \textsc{dat.3pl}\} donne  seulement  Ana.\textsc{dat}  un  stylo  \{et {\textbar} et\}  Ina.\textsc{dat} un  livre  bien\_que  \textsc{aux.cond.3}  pouvoir  \textsc{sbjv-dat.3sg}  donner.\textsc{sbjv.3}  aussi Dana.\textsc{dat}  un  CD\\
\glt ‘Ion offrira seulement à Ana un stylo et à Ina un livre, bien qu’il puisse offrir à Dana aussi un CD.’

\ex 
\gll \uline{Merg} nu \textbf{numai}  [azi  la  film  \{\textbf{şi} {\textbar} \textbf{\#iar}\}  mâine  la  teatru], ci  şi  vineri  la  operă. \label{ch2:ex305b}\\ 
aller.\textsc{prs.1sg}  non  seulement  aujourd’hui  à  film  \{et {\textbar} et\}  demain  à  théâtre mais  aussi  vendredi  à  opéra\\
\glt ‘Je vais non seulement aujourd’hui au cinéma et demain au théâtre, mais aussi vendredi à l’opéra.’ 

\ex
\gll Nu \ulg{i-am}{14.8} dat \textbf{decât} [Anei  o  carte  \{\textbf{şi} {\textbar} \textbf{\#iar}\}  lui  Ion  un  CD], deşi  aş  fi  putut  să-i  dau  şi  Inei  un caiet. \label{ch2:ex305c}\\ 
\textsc{neg} \textsc{dat.3sg}{}-ai  donné  seulement  Ana.\textsc{dat}  un  livre  \{et {\textbar} et\}  \textsc{dat} Ion  un  CD bien\_que  \textsc{aux.cond.1sg} \textsc{pst} pu  \textsc{sbjv-dat.3sg}  donner.\textsc{sbjv.1sg}  aussi  Ina.\textsc{dat}  un cahier\\
\glt ‘Je n’ai offert qu’à Ana un livre et à Ion un stylo, bien que j’aie pu offrir à Ina aussi un cahier.’  
\z
\z

On note encore une différence entre les deux constructions avec la conjonction \textit{şi} et respectivement \textit{iar} en ce qui concerne \is{accord}l’accord. On discute d’abord l’accord lié au \isi{redoublement clitique} et ensuite \is{accord}l’accord avec les sujets postverbaux. Ainsi, on observe que la coordination de séquences avec la conjonction \textit{şi} permet le \isi{redoublement clitique} soit sous sa forme au singulier, soit sous une forme au pluriel \REF{ch2:ex306a} ; ainsi, l’emploi du \is{affixe/clitique pronominal}pronom clitique au datif pluriel \textit{le} ‘leur’ redoublant deux compléments nominaux au singulier (\textit{Anei} ‘à Ana’ et \textit{Elenei} ‘à Elena’) indique que les compléments redoublés dans les séquences sont sous la portée syntaxique du prédicat verbal qui contient le clitique pluriel. En revanche, la coordination de séquences avec \textit{iar} en \REF{ch2:ex306b} ne permet que l’emploi d’un clitique singulier (\textit{le} ‘lui’) dans ces contextes. 

\ea \label{ch2:ex306}
\ea 
\gll Ion  \{\textbf{îi} {\textbar} \textbf{le}\}  \ulg{va}{6.5}  da  Anei  un  stilou  \textbf{şi}  Elenei  o  carte. \label{ch2:ex306a}\\
Ion  \{\textsc{dat.3sg} {\textbar} \textsc{dat.3pl}\} va  donner  Ana.\textsc{dat}  un  stylo  et  Elena.\textsc{dat}  un  livre\\
\glt ‘Ion va donner à Ana un stylo et à Elena un livre.’

\ex 
\gll Ion \{\textbf{îi} {\textbar} \textbf{*le}\}  \ulg{va}{6.7}  da  Anei  un  stilou,  \textbf{iar}  Elenei  o  carte. \label{ch2:ex306b}\\
Ion  \{\textsc{dat.3sg} {\textbar} \textsc{dat.3pl}\}  va  donner  Anei.\textsc{dat}  un  stylo  et  Elena.\textsc{dat}  un  livre\\
\glt ‘Ion va donner à Ana un stylo et à Elena un livre.’ 
\z
\z

Une autre différence liée à \is{accord}l’accord concerne le placement postverbal des sujets. On observe deux stratégies \is{accord}d’accord : (i) soit le verbe s’accorde de manière indépendante avec le sujet de chaque séquence \REF{ch2:ex307}, l’interprétation étant celle d’une conjonction de deux événements indépendants (explicitée par l’emploi d’une expression adverbiale comme \textit{câteva secunde mai târziu} ‘quelques secondes plus tard’ en \REF{ch2:ex307a} ou bien \textit{mai întâi} ‘d’abord’... \textit{apoi} ‘ensuite’ en \REF{ch2:ex307b}), et dans ce cas, l’emploi des deux conjonctions \textit{şi} et \textit{iar} est possible ; (ii) soit le verbe reçoit \is{accord}l’accord au pluriel \REF{ch2:ex308}, l’interprétation étant plutôt celle d’un événement complexe (explicitée par l’emploi d’une expression adverbiale comme \textit{simultan} ‘simultanément’), et dans ce cas, les locuteurs préfèrent la conjonction \textit{şi} à la place de \textit{iar}. 

\ea \label{ch2:ex307}
\ea 
\gll La  auzul  împuşcăturii,  \ulg{\textbf{a}}{20.3}  ţâşnit  dintr-un  tufiş  un  iepure, \{\textbf{şi} {\textbar} \textbf{iar}\}  (\textbf{câteva} \textbf{secunde} \textbf{mai} \textbf{târziu},)  dintr-o  scorbură  o  veveriţă. \label{ch2:ex307a}\\
à  bruit.\textsc{def}  coup-de-feu.\textsc{gen} \textsc{aux.3sg} surgi  d’un  buisson  un  lapin \{et {\textbar} et\}  (quelques  secondes  plus  tard)  d’un  terrier  un  écureuil\\
\glt ‘Au bruit du coup de feu, surgit d’un buisson un lapin et (quelques secondes plus tard) d’un terrier un écureuil.’

\ex
\gll \ulg{\textbf{Ai}}{18.4} intrat  (\textbf{mai} \textbf{întâi}) tu  pe  banda  întâi,  \{\textbf{şi} {\textbar} \textbf{iar}\}  (\textbf{apoi}) el  pe  banda  a  doua. \label{ch2:ex307b}\\
\textsc{aux.2sg} entré  (tout d’abord)  toi  sur  voie.\textsc{def}  premier  \{et {\textbar} et\}  (ensuite) lui  sur  voie.\textsc{def}  \textsc{art.f}  deuxième\\
\glt ‘(D’abord) toi, tu t’es engagé sur la voie de droite, (ensuite) lui sur la voie de gauche.’ 
\z
\z


\ea \label{ch2:ex308}
\ea 
\gll La  auzul  împuşcăturii,  \textbf{au}  ţâşnit  (\textbf{simultan})  dintr-un  tufiş un  iepure  \{\textbf{şi} {\textbar} \textbf{\#iar}\}  dintr-o  scorbură  o  veveriţă.\\
à  bruit.\textsc{def}  coup-de-feu.\textsc{gen}  \textsc{aux.3pl}  surgi  (simultanément)  d’un  buisson un  lapin  \{et {\textbar} et\}  d’un  terrier  un  écureuil\\
\glt ‘Au bruit du coup de feu, surgirent (simultanément) d’un buisson un lapin et d’un terrier un écureuil.’

\ex 
\gll \textbf{Aţi}  intrat  (\textbf{simultan})  tu  pe  banda  întâi  \{\textbf{şi} {\textbar} \textbf{\#iar}\} el pe banda  a  doua.\\
\textsc{aux.2pl}  entré  (simultanément)  toi  sur  voie.\textsc{def}  premier  \{et {\textbar} et\}  lui  sur voie.\textsc{def}  \textsc{art.f}  deuxième\\
\glt ‘Vous vous y êtes engagés en même temps, toi sur la voie de droite, lui sur la voie de gauche.’ 
\z
\z

L’hypothèse selon laquelle les coordinations de séquences avec la conjonction \textit{şi} sont différentes syntaxiquement des coordinations de séquences avec \textit{iar} est justifiée par un autre fait empirique, mentionné par ailleurs plus haut dans la section~\ref{ch2:sect2.6.1}, à savoir la portée des \is{adjectif relationnel}adjectifs relationnels. Les adjectifs \textit{acelaşi} ‘même’ et \textit{diferit} ‘différent’ admettent une lecture interne avec effet de distributivité et de réciprocité, lorsqu’ils s’appliquent à une \isi{éventualité plurielle} \citep{VanPeteghem2002,Carlson1987}, ces adjectifs ayant donc portée sur les séquences coordonnées. Cela explique pourquoi on ne peut pas substituer la conjonction \textit{iar} à la conjonction \textit{şi} en \REF{ch2:ex309}, cette lecture interne étant déclenchée si et seulement si l’adjectif en question s’applique à une pluralité à interprétation distributive.

\ea \label{ch2:ex309}
\ea \uline{Am prezentat \textbf{acelaşi} articol} ieri la curs \{\textbf{şi} {\textbar} \textbf{\#iar}\} azi la seminar. 
\glt ‘J’ai présenté le même article hier en cours et aujourd’hui en séminaire.’ 

\ex \uline{Voi cumpăra un cadou \textbf{diferit}} pentru Ion de Crăciun \{\textbf{şi} {\textbar} \textbf{\#iar}\} pentru Maria de Paşte. 
\glt ‘Je vais acheter un cadeau différent à Ion pour Noël et à Maria pour Pâques.’
\z
\z

Un fait qui rapproche beaucoup les coordinations de séquences des coordinations à gapping est la contrainte de \is{constituant majeur}constituance majeure (discutée dans la section~\ref{ch2:sect2.3.3}), {\cad} seuls des dépendants d’une tête verbale (racine ou enchâssée) sont légitimés dans la séquence à droite de la conjonction. Ainsi, les exemples en \REF{ch2:ex310} sont inacceptables, car un des constituants dans chaque séquence dépend d’une tête non-verbale (en \REF{ch2:ex310}, le premier élément de chaque séquence est le dépendant d’une tête nominale non prédicative). En revanche, la coordination de deux séquences composées chacune d’un complément du verbe enchâssé et d’un complément du verbe matrice est acceptable en roumain, au moins avec la conjonction \textit{iar}, comme on l’observe en \REF{ch2:ex311a} (contrairement à ce que \citealt{Mouret2007} ou \citealt{Mouret2008} constate pour le français) ; de même, pour les coordinations de séquences dépendant d’une tête non verbale, mais \is{tête prédicative}prédicative dans les structures à \isi{prédicat complexe} \REF{ch2:ex311b}. En même temps, pour toutes les occurrences de coordinations de séquences en dehors du domaine verbal, on a une préférence pour la conjonction \textit{şi} plutôt que pour la conjonction \textit{iar} \REF{ch2:ex312a}. Cela se vérifie dans l’exemple attesté \REF{ch2:ex312b}, où on a deux coordinations de séquences sous la portée syntaxique du verbe \textit{având} ‘ayant’ et respectivement sous la portée syntaxique de la préposition \textit{cu} ‘avec’ : on observe que dans le premier cas, le locuteur a utilisé la conjonction \textit{iar}, tandis que dans le deuxième cas, le locuteur a utilisé la conjonction \textit{şi}.
\largerpage

\ea \label{ch2:ex310}
\ea 
\gll *\ulg{Am}{26.3}  cumpărat  mere  roşii  azi  \{şi {\textbar} iar\}  verzi  ieri.\\
ai  acheté  pommes  rouges  aujourd’hui  \{et {\textbar} et\}  vertes  hier\\
\glt ‘J’ai acheté des pommes rouges aujourd’hui et des pommes vertes hier.’

\ex 
\gll *Ina  \ulg{i-a}{36}  dat  lucrurile  de  fetiţă  Danei  \{şi {\textbar} iar\}  de  băieţel lui  Dan.\\
Ina  \textsc{dat.3sg}{}-a  donné  affaires.\textsc{def}  de  fillette  Dana.\textsc{dat}  \{et {\textbar} et\}  de  petit-garçon \textsc{dat}  Dan\\
\glt ‘Ina a offert les vêtements de fille à Dana et les vêtements de garçon à Dan.’

\ex 
\gll *Paul  \ulg{dezaprobă}{23}  propunerea  Mariei  de  a  ieşi  în  parc  şi  Ioanei de  a  merge  la  film.\\
Paul  désapprouve  proposition.\textsc{def}  Maria.\textsc{gen}  de  \textsc{inf}  sortir  en  parc  et  Ioana.\textsc{gen} de  \textsc{inf}  aller  à  film\\
\glt ‘Paul désapprouve la proposition de Maria de sortir dans le parc et la proposition de Ioana d’aller au cinéma.’
\z
\z


\ea \label{ch2:ex311}
\ea 
\gll Paul  \ulg{i-a}{30.3}  recomandat  Mariei  \ulg{să}{16}  meargă  la  mare,  iar  Danei la  munte. \label{ch2:ex311a}\\
Paul  \textsc{dat.3sg}{}-a  recommandé  Maria.\textsc{dat}  \textsc{sbjv}  aller.\textsc{sbjv.3} à  mer  et  Dana.\textsc{dat} à  montagne\\
\glt ‘Paul a recommandé à Maria d’aller à la mer, et à Dana à la montagne.’ 

\ex 
\gll \ulg{Este}{15}  un  tipar  sintactic  frecvent  realizat,  iar  semantic,  destul  de eterogen. \label{ch2:ex311b}\\
est  un  modèle  syntaxique  fréquemment  réalisé  et  sémantique  assez  de hétérogène\\
\glt ‘C’est un patron syntaxique fréquemment réalisé, et sémantiquement, il est assez hétérogène.’ 
\z
\z


\ea \label{ch2:ex312}
\ea 
\gll \uline{Cu}  Ion  director  \{şi {\textbar} ??iar\}  Maria  secretară,  firma  nu  va  merge niciodată  bine. \label{ch2:ex312a}\\
avec  Ion  directeur  \{et {\textbar} et\}  Maria  secrétaire  société.\textsc{def} \textsc{neg}  va  aller jamais  bien\\
\glt ‘Avec Ion comme directeur et Maria comme secrétaire, la boîte n’ira jamais mieux.’

\ex Există limbi «~head first~», \uline{având} deci capul de grup pe prima poziţie, \textbf{iar} determinanţii postpuşi, şi limbi «~head last~», \uline{cu} regentul pe ultima poziţie \textbf{şi} determinanţii antepuşi. \label{ch2:ex312b}
\glt ‘Il y a des langues «~head first~», ayant donc la tête de groupe en première position, et les déterminants postposés, et des langues «~head last~», avec le régent en position finale et les déterminants antéposés.’ 
\z
\z

Les coordinations de séquences avec \textit{şi} et avec \textit{iar} se distinguent aussi prosodiquement. Les conjoints coordonnés par la conjonction \textit{şi}, en dehors d’une intonation particulière, présentent habituellement plutôt une \isi{prosodie intégrée} des conjoints, {\cad} les conjoints forment une seule unité prosodique \REF{ch2:ex313a}. En revanche, les conjoints coordonnés par \textit{iar} présentent une \isi{prosodie incidente} (marquée par une pause à l’oral et obligatoirement par une virgule à l’écrit), {{\cad}} chaque conjoint constitue une unité prosodique autonome \REF{ch2:ex313b}. 

\ea \label{ch2:ex313}
\ea \uline{Am fost} luni la film ({\textbar}) \textbf{şi} marți la teatru. \label{ch2:ex313a}
\glt ‘J’ai été lundi au cinéma et mardi au théâtre.’

\ex \uline{Am fost} luni la film, {\textbar} \textbf{iar} marți la teatru. \label{ch2:ex313b}
\glt ‘J’ai été lundi au cinéma, et mardi au théâtre.’
\z
\z

Une autre différence entre les deux conjonctions concerne le nombre d’éléments présents dans le deuxième conjoint. Pour les coordinations à gapping, on avait postulé le double \is{contraste sémantique}contraste comme une contrainte sémantique majeure (section~\ref{ch2:sect2.3.4.2}) : le conjoint fragmentaire doit ainsi contenir au moins deux constituants qui seront mis en contraste avec des éléments dans le premier conjoint (avec l’observation qu’un élément dans le premier conjoint peut être implicite). On observe la même contrainte avec les coordinations de séquences liées par \textit{iar}, mais pas nécessairement dans les contextes avec la conjonction \textit{şi}. Ainsi, la séquence introduite par la conjonction \textit{şi} peut contenir un seul constituant immédiat \REF{ch2:ex314a}, alors que la séquence introduite par \textit{iar} doit contenir au moins deux constituants immédiats (p.ex. \REF{ch2:ex314b} et \REF{ch2:ex315}). 

\ea \label{ch2:ex314}
\ea \uline{Am cumpărat} o jucărie pentru fata mea \{\textbf{şi} {\textbar} \textbf{*iar}\} un ziar. \label{ch2:ex314a}
\glt ‘J’ai acheté un jouet pour ma fille et un journal.’

\ex \uline{Am cumpărat} un ziar (pentru băiatul meu), \{şi {\textbar} \textbf{iar}\} pentru fata mea o jucărie. \label{ch2:ex314b}
\glt ‘J’ai acheté un journal (pour mon garçon), et pour ma fille un jouet.’
\z
\z


\ea \label{ch2:ex315}
\ea 
\gll Nu  am  nicio  legătură  cu  biserica,  \uline{sunt} [un  simplu  credincios],  iar *(\textbf{de} \textbf{meserie})  [şofer]. \label{ch2:ex315a}\\
\textsc{neg}  ai  aucun  lien  avec  église.\textsc{def} suis  un  simple  croyant et de  métier  chauffeur\\
\glt ‘Je n’ai aucun lien avec l’église, je suis un simple croyant, et quant à mon métier, je suis chauffeur.’

\ex Ioana \uline{mănâncă} [un măr], iar *(\textbf{apoi}) [o pară]. \label{ch2:ex315b}
\glt ‘Ioana mange une pomme, et ensuite une poire.’
\z
\z

Cela s’explique par la contrainte discursive imposée par la conjonction \textit{iar}, qui ne peut être immédiatement suivie que par un \isi{topique contrastif} et non par un \isi{focus informationnel} (voir la section~\ref{ch2:sect2.3.4.4}). Ce \isi{topique contrastif} est distingué prosodiquement si \textit{iar} est présent, {\cad} il forme un syntagme intonatif à lui tout seul, alors que ce n’est pas le cas avec la conjonction \textit{şi}. Cela explique pourquoi la conjonction \textit{iar} est préférée quand \is{ordre de mots}l’ordre des éléments dans les séquences n’est pas le même \REF{ch2:ex314b} : la distinction discursive et prosodique du premier élément suivant \textit{iar} permet la mise en parallèle nécessaire pour établir le \is{contraste sémantique}contraste.

Enfin, on peut ajouter ici le fait que, dans certains contextes ambigus, l’emploi de la conjonction permet de réduire l’ambiguïté : ainsi, si l’emploi de la conjonction \textit{şi} en \REF{ch2:ex316a} peut être compatible avec une interprétation du groupe nominal \textit{câinele} ‘le chat’ comme complément du verbe ({\cad} le lapin a vu et l’ours et le chien), l’emploi de la conjonction \textit{iar} en \REF{ch2:ex316b} amène plutôt à une interprétation de ce groupe nominal comme sujet ({\cad} le lapin et le chien ont vu chacun l’ours). Si dans le premier cas une coordination sous-phrastique est envisageable, dans le deuxième cas c’est plutôt une coordination phrastique qui ressort.

\ea \label{ch2:ex316}
\ea Iepurele a văzut [ursul ieri \textbf{şi} câinele azi]. \label{ch2:ex316a} 
\glt ‘Le lapin a vu l’ours hier et le chien aujourd’hui.’

\ex {} [Iepurele \uline{a văzut} ursul ieri], \textbf{iar} [câinele azi]. \label{ch2:ex316b} 
\glt ‘Le lapin a vu l’ours hier, et le chien aujourd’hui.’
\z
\z

Sur la base de toutes ces différences empiriques, on doit distinguer entre les coordinations de séquences liées par la conjonction \textit{iar} et les coordinations de séquences liées par la conjonction \textit{şi}\footnote{
 A priori, les séquences \is{juxtaposition}juxtaposées ont un comportement similaire à celles coordonnées par la conjonction \textit{şi}. Il reste à voir le comportement de la conjonction adversative \textit{dar} ‘mais’. A priori, cette conjonction se rapproche plutôt de la conjonction \textit{iar}.}. Les coordinations avec \textit{iar} mettent en jeu plutôt une coordination de deux phrases, dont une \is{fragment}fragmentaire, alors que celles liées par \textit{şi} restent ambiguës, étant a priori compatibles avec les deux analyses : soit une coordination de phrases dont une fragmentaire, soit une coordination sous-phrastique de séquences sous la portée syntaxique d’un prédicat. Par conséquent, l’analyse proposée par \citet{Mouret2006,Mouret2007,Mouret2008} pour le français ne peut pas être étendue aux coordinations avec \textit{iar}, mais peut être une des solutions possibles pour les coordinations avec \textit{şi}. Les coordinations de séquences avec \textit{iar} reçoivent ainsi une analyse commune avec les constructions à gapping discutées dans la section~\ref{ch2:sect2.5}. 

Pour conclure, les coordinations «~elliptiques~» avec \textit{iar} reçoivent une seule analyse, indépendamment de la position du verbe. En particulier, la séquence introduite par \textit{iar}, qui contient toujours au moins deux constituants immédiats, est un fragment ayant un contenu propositionnel. En revanche, les coordinations «~elliptiques~» avec la conjonction \textit{şi} restent ambiguës entre une structure à ellipse fragmentaire (coordination phrastique) et une structure sans ellipse (coordination sous-phrastique) dans les configurations à verbe initial en roumain.

D’ailleurs, le français aussi présente des cas ambigüs, qui se prêtent à deux analyses\footnote{
 Je remercie François Mouret pour ce commentaire.}. Ainsi, on observe en \REF{ch2:ex317} une différence entre les séquences \textit{un livre à Jean} vs. \textit{à Jean un livre}. Si la séquence \textit{un livre à Jean} semble privilégier une coordination de séquences en \REF{ch2:ex317b}, la séquence \textit{à Jean un livre} est ambiguë en \REF{ch2:ex317c}.

\ea \label{ch2:ex317}
\ea Paul apportera un disque ou bien Jean un livre. (gapping) \label{ch2:ex317a}
\ex Paul apportera un disque à Marie ou un livre à Jean. (coordination de séquences) \label{ch2:ex317b}
\ex Paul apportera à Marie un disque ou bien à Jean un livre. (analyse ambiguë) \label{ch2:ex317c}
\z
\z

Cette différence entre les deux séquences est justifiée par le comportement différent qu’elles ont par rapport au placement des conjonctions doubles. Ainsi, la séquence \textit{un livre à Jean} n’est pas compatible avec le placement initial, en début de phrase, de la \is{coordination omnisyndétique (ou corrélative)}conjonction corrélative \textit{ou bien} (\ref{ch2:ex318a}--\ref{ch2:ex318b}), contrairement à ce qu’on observe avec la séquence \textit{à Jean un livre} (\ref{ch2:ex319a}--\ref{ch2:ex319b}). Cette différence peut être mise en relation avec la possibilité d’antéposer ou non le premier constituant de la séquence à droite de la conjonction : le syntagme nominal \textit{un livre} ne peut jamais être antéposé en début de phrase \REF{ch2:ex318c}, alors que le syntagme prépositionnel \textit{à Jean} le permet \REF{ch2:ex319c}.  
\largerpage

\ea \label{ch2:ex318}
\ea Paul apportera \textbf{ou bien} un disque à Marie \textbf{ou bien} un livre à Jean. \label{ch2:ex318a}
\ex *\textbf{Ou bien} Paul apportera un disque à Marie \textbf{ou bien} un livre à Jean. \label{ch2:ex318b}
\ex *Un livre, Paul apportera à Jean. \label{ch2:ex318c}
\z
\z

\ea \label{ch2:ex319}
\ea Paul apportera \textbf{ou bien} à Marie un disque \textbf{ou bien} à Jean un livre. \label{ch2:ex319a} 
\ex \textbf{Ou bien} Paul apportera à Marie un disque \textbf{ou bien} à Jean un livre. \label{ch2:ex319b}
\ex A Jean, Paul apportera un livre. \label{ch2:ex319c}
\z
\z

Il y a donc des propriétés syntaxiques qui séparent nettement la coordination phrastique de la coordination sous-phrastique dans les contextes «~elliptiques~» ambigus, p.ex. contraintes sur les constituants de la séquence, contraintes sur la forme du verbe, distribution des conjonctions doubles, distribution des \is{adverbe associatif}adverbes associatifs, etc. Une étude empirique s’avère nécessaire avant de postuler l’existence d’un certain type d’ellipse dans une langue.


\subsection{Théorie des coordinations de clusters en HPSG} \label{ch2:sect2.6.4}

\begin{figure}[b]
\includegraphics[width=\textwidth]{figures/Ch2Fig25.pdf}% \input{figures/ch2_trees.tex}
\caption{Syntaxe simplifiée des coordinations de séquences avec la conjonction \textit{iar}}
\label{ch2:fig25}
\end{figure}

 
Dans la section~\ref{ch2:sect2.5}, on a proposé une \is{approche constructionnelle}analyse constructionnelle pour les coordinations à gapping dans lesquelles le verbe antécédent se trouvait en position médiane. Dans la section~\ref{ch2:sect2.6.3}, j’ai montré qu’en roumain les coordinations de séquences avec la conjonction \textit{iar} sont incontestablement des coordinations de phrases, indépendamment de la position du verbe et indépendamment de la fonction syntaxique des éléments coordonnés. Par conséquent, le problème de l’ambiguïté (discuté au début de ce chapitre, dans la section~\ref{ch2:sect2.2.2}), qui a priori se pose pour les coordinations de séquences ayant le verbe en position initiale, ne se pose pas pour les coordinations avec \textit{iar} en roumain. Ainsi, toutes les configurations décrites en \REF{ch2:ex320} se prêtent à une seule analyse. La première configuration \REF{ch2:ex320a}, qui constitue d’ailleurs le prototype syntaxique du gapping, a été analysée en détails dans la section~\ref{ch2:sect2.5}. J’étends donc la même analyse aux autres configurations en (\ref{ch2:ex320b}--\ref{ch2:ex320c}--\ref{ch2:ex320d}). Une syntaxe simplifiée de la configuration \REF{ch2:ex320d}, exemplifiée en \REF{ch2:ex321}, est donnée en \figref{ch2:fig25}.   


 
\ea \label{ch2:ex320}
\ea sujet \textsc{verbe} complément \textit{iar} sujet complément \label{ch2:ex320a} 
\ex \textsc{verbe} sujet complément \textit{iar} sujet complément \label{ch2:ex320b}
\ex \textsc{verbe} complément sujet \textit{iar} complément sujet \label{ch2:ex320c}
\ex \textsc{verbe} complément complément \textit{iar} complément complément \label{ch2:ex320d}
\z
\z


\ea \label{ch2:ex321}
\gll Ii  dau  Mariei  o  carte,  \textbf{iar}  Ioanei  un  stilou.\\
\textsc{dat.3sg} donner.\textsc{prs.1sg}  Maria.\textsc{dat}  un  livre  et  Ioana.\textsc{dat}  un  stylo\\
\glt ‘Je donne à Maria un livre, et à Ioana un stylo.’
\z



En revanche, dans les configurations à verbe initial, le problème de l’ambiguïté se pose pour les coordinations «~elliptiques~» avec la conjonction \textit{şi~}: bien qu’on puisse les analyser comme des coordinations de phrases, elles présentent (au moins dans certains contextes) des propriétés qui nous laissent aussi la possibilité de les analyser comme des coordinations sous-phrastiques dans la portée syntaxique d’un prédicat verbal ({\cad} coordinations de \is{cluster}clusters, sans ellipse). Une représentation simplifiée de la syntaxe de ces coordinations de clusters avec la conjonction \textit{şi} est donnée en \figref{ch2:fig26}. 

\begin{figure} 

\includegraphics[width=\textwidth]{figures/Ch2Fig26.pdf}% \input{figures/ch2_trees.tex}

\caption{Syntaxe simplifiée des coordinations de clusters avec la conjonction \textit{şi}}
\label{ch2:fig26}
\end{figure}

\largerpage
La première possibilité d’analyse a été largement discutée dans la section~\ref{ch2:sect2.5}. Je me concentre maintenant plutôt sur la deuxième analyse, proposée par \citet{Mouret2006,Mouret2007} pour les coordinations de séquences en français.

Le point commun entre l’analyse postulée pour les constructions à gapping et l’analyse que je présente ici pour les coordinations de \is{cluster}clusters est l’emploi d’un syntagme sans tête, qu’on a appelé \textit{cluster-ph} dans la section~\ref{ch2:sect2.5}. Par conséquent, tout ce qu’on avait discuté dans la section dédiée à la théorie des clusters ({{\cad}} section~\ref{ch2:sect2.5.3.1}) s’applique aussi à l’analyse des coordinations de clusters. Je ne reprends donc ici que la description du syntagme cluster définie en \REF{ch2:ex262} et reprise en \REF{ch2:ex322} ; pour les autres détails, revoir la section~\ref{ch2:sect2.5.3.1}. 

\ea \label{ch2:ex322}
Syntagme de type cluster (cf. \citealt{Mouret2006,Mouret2007})\\
\includegraphics{figures/Ch2322.pdf}

% \noindent
% \textit{cluster-ph} $\Rightarrow$ \textit{non-headed-ph} \& \\
% \begin{avm}
% [head & [\tp{head}\\
% 				 cluster & nelist\(synsem\) <@{1}, ..., @{n}>]\\
% subj & < >\\
% spr & < > \\
% comps & < >\\
% slash & $\Sigma$$_1$ $\cup$ ... $\cup$ $\Sigma$$_n$\\
% n-hd-dtrs & <[synsem @{1} [slash & $\Sigma$$_1$]], ..., [synsem @{n} [slash & $\Sigma$$_n$]]>]
%  \end{avm}

\z

\largerpage
Pour pouvoir utiliser le \isi{cluster} en dehors des structures coordonnées, on se donne dans la grammaire la possibilité qu’un \isi{cluster} ait un seul constituant immédiat ({\cad} un cluster unaire), ce qui est mis en évidence par le fait que la valeur de liste correspondant aux branches non-têtes est \textit{nelist} (\textit{non-empty-list}), {\cad} une liste composée d’au moins un élément. 

Il nous reste à montrer comment on légitime la coordination de \is{cluster}clusters dans la portée syntaxique d’un prédicat (en l’occurrence, un verbe). Pour cela, je reprends la proposition faite par \citet{Mouret2006,Mouret2007} pour le français. Il propose une \isi{règle lexicale} post-flexionnelle ({\cad} qui relie une entrée de type \textit{word} à une nouvelle entrée de type \textit{word}), qui permet à un prédicat donné d’être partiellement saturé par une coordination de clusters, plutôt que par une suite ordinaire de constituants, c’est ce que Mouret appelle une complémentation alternative des prédicats. Les règles lexicales, qui sont utilisées en HPSG pour les changements de valence, sont représentées sous forme de structures de traits (avec deux attributs  INPUT et OUTPUT, cf. \citealt{BriscoeEtAl1999}). La règle que \citet{Mouret2006,Mouret2007} propose pour la légitimation des coordinations de clusters dans la dépendance d’un prédicat en français figure en \REF{ch2:ex323}. 

\ea \label{ch2:ex323}
Règle lexicale pour la complémentation alternative des prédicats\\
\includegraphics{figures/Ch2323.pdf}

% \noindent
% \begin{avm}
% [\tp{cluster-coord-lexical-rule}\\
% input & [\tp{word}\\
% 				comps @{L$_1$} + & @{L$_2$} nelist <[cat & @{1}], ..., [cat & @{n}]>]\\
% output & [\tp{word}\\
% 				comps @{L$_1$} + & <[coord + \\
% 													cluster <[cat & @{1}], ..., [cat & @{n}]>]>]]\end{avm}\\
% \begin{avm}												
% \& @{L$_2$} $\neq$ <[coord +\\ cluster & nelist\(synsem\)]>		
% \end{avm}

\z

La règle remplace une sous-liste non vide de compléments \fbox{L\textsubscript{2}} dans la liste des compléments attendus par l’entrée lexicale INPUT par une coordination de \is{cluster}clusters (décrite comme COORD + et ayant une valeur de liste non vide pour le trait CLUSTER) dans la liste des compléments attendus par l’entrée lexicale OUTPUT. On exclut la récursion à l’infini par la contrainte conjointe à la règle suivant laquelle la liste \fbox{L\textsubscript{2}} remplacée par la coordination de clusters ne peut pas elle-même correspondre à une coordination de clusters.

L’introduction de la description des coordinations de \is{cluster}clusters dans la liste COMPS et non dans la liste ARG-ST du prédicat rend compte, d’une part, de l’absence de cliticisation ou \is{extraction}d’extraction des coordinations de clusters et, d’autre part, de l’absence de coordinations de clusters de niveaux différents (cf. les données en \REF{ch2:ex310}). En HPSG, la liste COMPS ne contient que des synsems canoniques. Donc, elle ne contient pas de gaps (correspondant aux éléments extraits), de pronoms «~nuls~» ou affixes (\is{affixe/clitique pronominal}pronominaux ou \is{affixe/clitique adverbial}adverbiaux). De plus, cette liste enregistre les compléments intrinsèques d’un prédicat ou ceux qu’il hérite de ses arguments, mais non les compléments plus enchâssés. 

La règle proposée en \REF{ch2:ex323}, par son trait COMPS, prend en compte toute coordination de \is{cluster}clusters en français : clusters contenant un sujet postverbal \REF{ch2:ex324a}, les clusters contenant des compléments \REF{ch2:ex324b} ou encore les clusters contenant un mélange de modifieurs et d’arguments \REF{ch2:ex324c}. Car en français, le sujet postverbal des \is{construction inaccusative}constructions inaccusatives \citep{Marandin1999}, ainsi que les modifieurs postverbaux \citep{Mouret2007} sont analysés comme des compléments syntaxiques. 

\ea \label{ch2:ex324}
\ea Alors surgit d’un champ un renard et quelques secondes plus tard d’un buisson une biche. \label{ch2:ex324a} 
\ex Paul offrira un disque à Marie et un livre à Jean. \label{ch2:ex324b}
\ex Paul joue au tennis le lundi et au football le mardi. \label{ch2:ex324c}
\z
\z

Les contraintes syntaxiques imposées par le prédicat de départ à ses compléments sont préservées dans la liste CLUSTER (cf. le partage de valeur des traits CAT), ce qui prédit correctement la possibilité de coordinations de \is{cluster}clusters dissemblables en ce qui concerne leur catégorie \REF{ch2:ex325a} ou bien leur nombre \REF{ch2:ex325b}. 

\ea \label{ch2:ex325}
\ea Les enseignants attendent des élèves [qu’ils respectent les règles de l’établissement]\textsubscript{S} et de leur proviseur [un soutien sans faille]\textsubscript{NP}. \label{ch2:ex325a}
\ex Paul écrira [un petit poème] et [[une lettre] [à sa mère]]. \citep[337]{Mouret2007} \label{ch2:ex325b} 
\z
\z

L’analyse proposée pour le français s’applique aussi aux coordinations de \is{cluster}clus\-ters avec la conjonction \textit{şi} en roumain. J’illustre les conséquences de cette \isi{règle lexicale} sur un exemple de coordinations de clusters contenant des compléments \REF{ch2:ex326}. Le verbe \textit{a da}\textsubscript{1} ‘donner’ contient dans sa liste de compléments un syntagme nominal à l’accusatif et un syntagme nominal au datif. La \isi{règle lexicale} donnée en \REF{ch2:ex323} autorise une entrée alternative \textit{a da}\textsubscript{2}, qui permet la combinaison de ce prédicat et d’une coordination de clusters. Les deux entrées lexicales correspondant à ce verbe sont données en \REF{ch2:ex327}. Le résultat de l’interaction des propriétés de ce prédicat et des contraintes qui définissent les structures coordonnées est représenté de manière simplifiée en \figref{ch2:fig27}.    

\ea \label{ch2:ex326}
\gll Ii  dau  o  carte  Mariei  şi  un  stilou  Ioanei.\\
\textsc{dat.3sg} donner.\textsc{prs.1sg}  un  livre  Maria.\textsc{dat}  et  un  stylo  Ioana.\textsc{dat}\\
\glt ‘Je donne un livre à Maria et un stylo à Ioana.’
\z

\ea \label{ch2:ex327}
Entrées lexicales du verbe \textit{a da} ‘donner’\\
\includegraphics{figures/Ch2327.pdf}

% \noindent
% \textit{a da}$_1$: 
% \begin{avm}
% [comps <NP$_{acc}$> &  $\oplus$ <NP$_{dat}$>]
% \end{avm}\\
% \textit{a da}$_2$: 
% \begin{avm}
% [comps <[coord +\\ cluster <NP$_{acc}$> &  $\oplus$ <NP$_{dat}$>]>]
% \end{avm}

\z


\begin{figure} 

\includegraphics[width=\textwidth]{figures/Ch2Fig27.pdf}% \input{figures/ch2_trees.tex}

\caption{Représentation simplifiée de la phrase \REF{ch2:ex326}}
\label{ch2:fig27}
\end{figure}

Comme montré par \citet{Mouret2006,Mouret2007}, cette approche nous permet d’analyser aussi les coordinations de \is{cluster}clusters de longueurs différentes. Ainsi, un verbe comme \textit{a scrie}\textsubscript{1} ‘écrire’ en \REF{ch2:ex328} contient dans sa liste de compléments un syntagme nominal à l’accusatif et optionnellement un syntagme nominal au datif (le fait d’être optionnel est indiqué par l’emploi des parenthèses dans la règle donnée en \REF{ch2:ex329}). La même \isi{règle lexicale} utilisée plus haut pour le verbe \textit{a da} ‘donner’ autorise l’entrée alternative \textit{a scrie}\textsubscript{2}, qui légitime la coordination de clusters dans la dominance du prédicat. Les deux entrées lexicales correspondant à ce verbe sont données en \REF{ch2:ex329}. Une représentation simplifiée de la phrase en \REF{ch2:ex328} est illustrée en \figref{ch2:fig28}.

\ea \label{ch2:ex328}
\gll Voi scrie  tema  şi  o  scrisoare  mamei.\\
\textsc{aux.fut.1sg} écrire  devoir.\textsc{def}  et  une  lettre  mère.\textsc{dat}\\
\glt ‘Je vais écrire le devoir et une lettre à ma mère.’
\z


\ea \label{ch2:ex329}
Entrées lexicales du verbe \textit{a scrie} ‘écrire’\\
\includegraphics{figures/Ch2329.pdf}

% \noindent
% \textit{a scrie}$_1$: 
% \begin{avm}
% [comps <NP$_{acc}$> &  $\oplus$ <(NP$_{dat}$)> ]
% \end{avm}\\
% \textit{a scrie}$_2$: 
% \begin{avm}
% [comps <[coord +\\ cluster <NP$_{acc}$> &  $\oplus$ <(NP$_{dat}$)>]>]
% \end{avm}

\z


\begin{figure} 

\includegraphics[width=\textwidth]{figures/Ch2Fig28.pdf}% \input{figures/ch2_trees.tex}
\caption{Représentation simplifiée de la phrase \REF{ch2:ex328}}
\label{ch2:fig28}
\end{figure}

Je ne me prononce pas sur le contenu de ses \is{cluster}clusters. Le contenu de l’ensemble peut être calculé au moyen d’une fonction qui compose le contenu des parties (cf. \citealt{Steedman2000}).


\section{Conclusion}\label{ch2:sect2.7}
\setlength{\parindent}{\normalparindent}
Dans ce chapitre, j’ai étudié les constructions à gapping en roumain et en français, dans lesquelles une séquence de syntagmes sans tête verbale, ayant néanmoins le contenu d’une phrase, se combine avec une phrase complète qui détermine sa forme et son interprétation. Comme critères minimaux de définition, je retiens l’absence de la tête verbale (plus éventuellement le sujet ou d’autres dépendants verbaux), ainsi que la présence d’au moins deux éléments résiduels dans la séquence trouée. Contrairement à ce qui est souvent postulé dans la littérature, la position médiane du trou, ainsi que la présence obligatoire d’un élément résiduel sujet ne constituent pas dans ce livre de vrais critères de définition. Bien que le gapping soit possible en dehors des structures coordonnées, il semble que les propriétés sont différentes d’une construction à l’autre, c’est pour cela que dans cet ouvrage je me limite au gapping dans la coordination, en laissant de côté d’autres constructions, comme les structures \isi{comparatives}.

Les deux vrais critères de définition mentionnés ci-dessus permettent de distinguer le gapping du \is{Pseudogapping}pseudogapping ou du \is{Stripping}stripping, mais ils ne permettent pas toujours de distinguer le gapping d’une coordination de séquences (angl. \is{Argument Cluster Coordination (ACC)}\textit{Argument Cluster Coordination}, abrégé ACC). En roumain, comme dans plusieurs langues, on a des configurations dans lesquelles on peut avoir descriptivement un flou de constructions. Les configurations qui ne sont pas ambiguës en roumain sont celles dans lesquelles le verbe antécédent se trouve en position médiane (\ref{ch2:ex330a}--\ref{ch2:ex330b}) ou en position finale (\ref{ch2:ex330c}--\ref{ch2:ex330d}) dans la phrase source. En revanche, les coordinations qui sont ambiguës entre une construction à gapping et une coordination de séquences sont celles dans lesquelles le verbe se trouve en position initiale \REF{ch2:ex331}\footnote{Je donne ici uniquement les distributions comportant un sujet et un complément. Mais une liste exhaustive des configurations possibles devrait inclure aussi les combinaisons complément-complément, compl\-ément-ajout, ajout-complément, sujet-ajout et ajout-sujet.}.

\ea \label{ch2:ex330}
Configurations non ambiguës de gapping en roumain 
\ea sujet \textsc{verbe} complément \textsc{conj} sujet complément \label{ch2:ex330a}
\ex complément \textsc{verbe} sujet \textsc{conj} complément sujet \label{ch2:ex330b}
\ex sujet complément \textsc{verbe} \textsc{conj} sujet complément \label{ch2:ex330c}
\ex complément sujet \textsc{verbe} \textsc{conj} complément sujet \label{ch2:ex330d}
\z
\z

\ea \label{ch2:ex331}
Configurations ambiguës en roumain 
\ea \textsc{verbe} sujet complément \textsc{conj} sujet complément \label{ch2:ex331a}
\ex \textsc{verbe} complément sujet \textsc{conj} complément sujet \label{ch2:ex331b}
\z
\z

Afin d’observer les propriétés spécifiques des coordinations à gapping, je me suis concentrée d’abord sur la description des configurations non ambiguës, en étudiant les contraintes générales pesant sur le matériel manquant et sur les éléments résiduels, ainsi que les \isi{contraintes de parallélisme}.

Parmi les contraintes générales s’appliquant au matériel manquant, on observe que le trou contient nécessairement le verbe tête de la phrase (y compris \is{auxiliaire}l’auxiliaire) et optionnellement d’autres éléments (sujets, compléments, ajouts) ; il peut correspondre à une \isi{expression idiomatique} (mais pas à une portion d’expression idiomatique) ; il ne correspond pas nécessairement à un constituant ; il peut comporter une \isi{négation} (qui se prête à plusieurs interprétations). En ce qui concerne le degré d’identité qui s’établit entre le matériel antécédent et le matériel manquant, on note qu’ils doivent appartenir au même \isi{paradigme de flexion} et avoir le même sens (même lexème), partager les mêmes propriétés de temps, mode, voix et aspect, et, de manière générale, partager les marques de \isi{flexion inhérente}. En revanche, ils peuvent différer par les marques de \isi{flexion contextuelle}, les affixes qu’une forme verbale peut prendre ou encore la \isi{polarité}. Parmi les contraintes générales s’appliquant aux éléments résiduels, on observe que la séquence trouée doit comporter au moins deux éléments résiduels, mis en correspondance avec des éléments parallèles dans la phrase source. Les éléments résiduels doivent être des \is{constituant majeur}constituants majeurs ({\cad} des arguments ou ajouts d’une tête verbale – racine ou enchâssée – dans la phrase source). 

J’ai étudié les \isi{contraintes de parallélisme} au niveau syntaxique, sémantique et discursif. On a observé que le parallélisme le plus strict opère au niveau sémanti\-co-discursif, mais en ne négligeant pas complètement la syntaxe. Au niveau syntaxique, bien que les éléments résiduels puissent différer de leurs corrélats en ce qui concerne la \is{asymétrie syntaxique}catégorie syntaxique, la position ou encore leur réalisation «~de surface~», ils doivent chacun correspondre à des arguments ou ajouts possibles du prédicat manquant. Cette généralisation est identique à celle qui caractérise les \is{coordination de termes dissemblables}coordinations de termes dissemblables (cf. la \is{généralisation de Wasow}généralisation de \ia{Wasow, Thomas}Wasow). Au niveau sémantique, il doit y avoir au moins deux \is{paire contrastive}paires contrastives, avec un élément provenant de chacun des conjoints dans chacune des paires. Au niveau discursif, entre les phrases reliées on doit avoir une relation \is{relation discursive}symétrique (les relations privilégiées étant le parallélisme et le contraste). Le prototype discursif dans le gapping est une réponse en liste de paires à une \isi{question multiple} implicite. Du point de vue de la \isi{structure informationnelle}, il semble qu’au moins les coordinations à gapping avec la conjonction \textit{iar} en roumain doivent contenir minimalement une \isi{paire contrastive} avec des \is{topique}topiques et une \isi{paire contrastive} avec des \isi{focus}. 

Ensuite, j’ai présenté les analyses qui sont proposées dans la littérature pour ren\-dre compte des constructions à gapping. Elles se regroupent en trois appro\-ches majeures : (i) ellipse syntaxique, avec reconstruction \textit{in situ} du matériel manquant, cf. \figref{ch2:fig29a} ; (ii) ellipse sémantique, avec \isi{légitimation indirecte}, cf. \figref{ch2:fig29b}, et (iii) analyse sans ellipse, avec \isi{mouvement} du matériel «~manquant~», cf. \figref{ch2:fig29c}.

\begin{figure}[p]
\includegraphics{figures/Ch2Fig29a.pdf}% \input{figures/ch2_trees.tex}
\caption{Ellipse syntaxique}
\label{ch2:fig29a}
\end{figure}

\begin{figure}[p]
\includegraphics{figures/Ch2Fig29b.pdf}% \input{figures/ch2_trees.tex}
\caption{Ellipse sémantique}
\label{ch2:fig29b}
\end{figure}

\begin{figure}[p]
\includegraphics{figures/Ch2Fig29c.pdf}% \input{figures/ch2_trees.tex}
\caption{Analyse sans ellipse}
\label{ch2:fig29c}
\end{figure}

J’ai donné des arguments empiriques en faveur d’une \isi{approche constructionnelle} des coordinations à gapping (avec une \isi{reconstruction sémantique} de l’ellipse, cf. \figref{ch2:fig29b}) et contre les approches alternatives en termes d’ellipse syntaxique ou \isi{mouvement}. Ainsi, on a vu que les deux processus syntaxiques majeurs envisagés par les approches postulant la \isi{reconstruction syntaxique} et/ou le \isi{mouvement}, à savoir \is{extraction}l’extraction des éléments résiduels et \is{extraction}l’extraction du matériel manquant, ne sont pas justifiés empiriquement (cf. la violation des \is{contraintes de localité}contrain\-tes de localité, la portée de certains opérateurs sémantiques, la distribution des items corrélatifs dans les coordinations \is{coordination omnisyndétique (ou corrélative)}omnisyndétiques, l’absence d’identité stricte entre le matériel antécédent et le matériel manquant, la présence de certains items incompatibles avec une phrase finie, etc.). Par conséquent, une ellipse sémantique (et non syntaxique) doit être envisagée pour les constructions à gapping. Dans cette perspective, la construction à gapping dans son ensemble est une coordination entre une (ou plusieurs) phrase verbale non elliptique, complète, et une (ou plusieurs) phrase \is{fragment}fragmentaire sans verbe. On a vu comment cette analyse peut être formalisée dans un cadre \is{approche constructionnelle}constructionnel, comme HPSG dans ses versions plus récentes. La phrase fragmentaire hérite à la fois d’un type de \is{fragment}\textit{fragment} (utilisé aussi pour les \is{question courte}questions et les \is{réponse courte}réponses courtes) pour ce qui est de son interprétation sémantique et de ses contraintes contextuelles, et d’un type de \is{cluster}\textit{cluster} (utilisé aussi pour la coordination de séquences dans les constructions \is{Argument Cluster Coordination (ACC)}ACC) pour ce qui est de sa constituance ({\cad} sa structure interne de syntagmes non reliés en termes de fonctions syntaxiques).

Après avoir décrit et analysé les configurations non ambiguës de gapping, j’ai consacré une section à l’étude des configurations ambiguës données en \REF{ch2:ex331}, dans lesquelles le verbe se trouve en position initiale, configurations se prêtant a priori à deux analyses en fonction du niveau auquel opère la coordination : coordination phrastique ou bien coordination sous-phrastique dans la portée syntaxique d’un prédicat verbal. Après avoir invalidé l’hypothèse d’une \isi{reconstruction syntaxique}, j’ai donné des arguments empiriques pour distinguer en roumain les coordinations de séquences avec la conjonction \textit{iar} des coordinations de séquences avec la conjonction \textit{şi}. Les configurations a priori ambiguës se désambiguïsent si les séquences sont coordonnées par la conjonction \textit{iar}. Dans ce cas, il s’agit toujours d’une coordination de phrases, dont une \is{fragment}fragmentaire, ce qui nous permet d’aligner ces coordinations avec \textit{iar} sur les cas standard de gapping discutés dans les sections précédentes. En revanche, les configurations avec la conjonction \textit{şi} restent ambiguës~entre les deux structures, à savoir une coordination phrastique (comme pour les constructions à gapping) ou bien une coordination sous-phrastique (comme pour les coordinations de \is{cluster}clusters en français). L’étude des configurations ambiguës en roumain nous montre qu’on peut avoir deux structures différentes en fonction de la conjonction (\textit{iar} vs. \textit{şi}) et que, de plus, les deux structures sont parfois nécessaires pour analyser un même exemple (en particulier, pour les coordinations avec la conjonction \textit{şi}). Une étude empirique s’avère donc nécessaire avant de postuler l’existence d’un certain type d’ellipse dans une langue.

 