\documentclass[output=paper,colorlinks,citecolor=brown]{langscibook}
\ChapterDOI{10.5281/zenodo.14266339}
\author{Agnes Korn\orcid{0000-0003-0302-6751}\affiliation{CNRS – UMR 8041 Centre de recherche sur le Monde iranien (CeRMI)}}
\title{Notes on word order in Bashkardi}
\abstract{This contribution discusses some features of word order in Bashkardi, a group of varieties of the Iranian branch of Indo-European spoken inland of the Strait of Hormuz in Southern Iran. The data are from recordings made by Ilya Gershevitch in 1956, when Persian influence was less strong than today. The findings include an average of 30\% non-subject elements being in postverbal position. Goals of motion and of caused-motion show a strong preference for this position, except for Goals of `put'-expressions, which are close to the overall average. More than 20\% of nominal direct objects are postverbal, while pronouns are very rare in this position. }

%move the following commands to the ``local...'' files of the master project when integrating this chapter
% \usepackage{tabularx}
% \usepackage{langsci-optional}
% \usepackage{langsci-gb4e}
% \usepackage{enumitem}
% \bibliography{localbibliography}
% \newcommand{\orcid}[1]{}
% \let\eachwordone=\itshape

\IfFileExists{../localcommands.tex}{
 \addbibresource{../collection_tmp.bib}
 \addbibresource{../localbibliography.bib}
 % add all extra packages you need to load to this file

\usepackage{tabularx,multicol}
\usepackage{url}
\urlstyle{same}

\usepackage{listings}
\lstset{basicstyle=\ttfamily,tabsize=2,breaklines=true}

\usepackage{langsci-basic}
\usepackage{langsci-optional}
\usepackage{langsci-lgr}
\usepackage{langsci-osl}
% \usepackage{./langsci/styles/langsci-lgr}
% \usepackage{./langsci/styles/langsci-osl}
% \usepackage{langsci-gb4e}

\usepackage{tikz}
\usetikzlibrary{patterns,calc}
\pgfdeclarepatternformonly{south east lines}{\pgfqpoint{-0pt}{-0pt}}{\pgfqpoint{3pt}{3pt}}{\pgfqpoint{3pt}{3pt}}{
    \pgfsetlinewidth{0.6pt}
    \pgfpathmoveto{\pgfqpoint{0pt}{3pt}}
    \pgfpathlineto{\pgfqpoint{3pt}{0pt}}
    \pgfpathmoveto{\pgfqpoint{.2pt}{-.2pt}}
    \pgfpathlineto{\pgfqpoint{-.2pt}{.2pt}}
    \pgfpathmoveto{\pgfqpoint{3.2pt}{2.8pt}}
    \pgfpathlineto{\pgfqpoint{2.8pt}{3.2pt}}
    \pgfusepath{stroke}}
    
\usepackage{stmaryrd}
\usepackage{wasysym}
\usepackage{multirow}
\usepackage{caption}
\usepackage{subcaption}
\usepackage{mathrsfs}
\usepackage{qtree}

\usepackage{linguex}


 %pminos do not split footnotes
% \interfootnotelinepenalty=10000 %Footnote in Laporte chapters has to be split SN


%\DeclareIndexNameFormat{default}{%
%\nameparts{#1}%
%\usebibmacro{index:name}%
%{\index[names]}%
%{\namepartfamily}%
%{\namepartgiveni}%
% {}% L1
% {}% L2
%{\namepartprefix}% generates spurious space L3
%{\namepartsuffix}% generates spurious space L4
%}

%  {\DeclareIndexNameFormat{default}{%
%     \usebibmacro{index:name}{\index[names]}{#1}{#3}{#5}{#7}}}

%\DeclareIndexNameFormat{default}{%
%  \usebibmacro{index:name}{\sindex[nom]}{#1}{#3}{#5}{#7}}

%\DeclareIndexNameFormat{default}{%
%  \usebibmacro{index:name}{\sindex[person]}{#1}{#3}{#5}{#7}}
%\DeclareIndexNameFormat{default}{%
%\nameparts{#1} \usebibmacro{index:name}{\sindex[person]]}{\namepartfamily}{‌​\namepartgiven}{\nam‌​epartprefix}{\namepa‌​rtsuffix}}

%\newcommand{\smiley}{:)}

%\renewbibmacro*{index:name}[5]{%
%\usebibmacro{index:entry}{#1}%
%{\iffieldundef{usera}{}{\thefield{usera}\actualoperator}\mkbibindexname{#2}{#3}{#4}{#5}}}

% \newcommand{\noop}[1]{}

%remove for final
%\overfullrule=1mm

\newcommand{\tobi}[2]}}
\renewcommand{\S}[1]{\tobi{#1}{\textsc{*}}}

% this volume references
% puts: [this volume]
% already defined: \citetv
%\newcommand{\citepv}[1]{(\citeauthor{#1} \citeyear*{#1} [this volume])}
\newcommand{\citealtv}[1]{\citeauthor{#1} \citeyear*{#1} [this volume]}

%parentheses around example number
\newcommand{\pref}[1]{(\ref{#1})}

% in-text examples

\newcommand{\lnex}[1]{\textit{#1}} %target lang word
\newcommand{\lnlit}[1]{(lit.: `#1')} %literal reading
\newcommand{\lnlat}[1]{(#1)} % latinization
\newcommand{\lntrans}[1]{`#1'} %translation
\newcommand{\lnexl}[2]%
{\lnex{#1}{} \lnlat{#2}} % ex with latinization
\newcommand{\lnexlat}[3]{\lnex{#1}{} \lnlat{#2}{} \lntrans{#3}} % ex with latinization and tranl.

%ch01
\newcommand{\co}[1]{\mbox{\textbf{#1}}}

%ch09

\newcommand{\cyrbulg}[1]{\begin{otherlanguage*}{bulgarian}#1\end{otherlanguage*}}


%ch10
\newcommand{\nlp}{{\small NLP}}
\newcommand{\mwe}{{\small MWE}}
\newcommand{\rae}{{\small RAE}}
\newcommand{\lvc}{{\small LVC}}
\newcommand{\pos}{{\small P}o{\small S}}
%\newcommand{\todo}[1]{ \textcolor{red}{#1} }

%\renewcommand{\labelenumi}{\theenumi}
%\ainamefmt{{vv}{ll}{, ff}{, jj}} % fullname

\newcommand{\biberror}[1]{{\color{red}#1}}

\newcommand{\osenovaitem}{--~}
 %% hyphenation points for line breaks
%% Normally, automatic hyphenation in LaTeX is very good
%% If a word is mis-hyphenated, add it to this file
%%
%% add information to TeX file before \begin{document} with:
%% %% hyphenation points for line breaks
%% Normally, automatic hyphenation in LaTeX is very good
%% If a word is mis-hyphenated, add it to this file
%%
%% add information to TeX file before \begin{document} with:
%% %% hyphenation points for line breaks
%% Normally, automatic hyphenation in LaTeX is very good
%% If a word is mis-hyphenated, add it to this file
%%
%% add information to TeX file before \begin{document} with:
%% \include{localhyphenation}
\hyphenation{
    Beck-man
    Ngu-yen
    back-chan-nel
    back-chan-nels
    mo-not-o-nous
    ste-reo-typ-i-cal
}

\hyphenation{
    Beck-man
    Ngu-yen
    back-chan-nel
    back-chan-nels
    mo-not-o-nous
    ste-reo-typ-i-cal
}

\hyphenation{
    Beck-man
    Ngu-yen
    back-chan-nel
    back-chan-nels
    mo-not-o-nous
    ste-reo-typ-i-cal
}

%  \boolfalse{bookcompile}
%  \togglepaper[5]%%chapternumber
}{}

\begin{document}
\maketitle\label{WOWA:ch:5}

\section{Introduction}\label{Bashkardi:ss:1}

\subsection{Affiliation and location}\label{Bashkardi:ss:1.1}

For the purposes of this contribution, the term ``Bashkardi'' (ISO 639-3: bsg) refers to the varieties spoken inland of the Strait of Hormuz in Southern Iran in a region called Bašākerd (see the  {map} in \figref{Bashkardi:fig:1}).\footnote{
See \citet[79–80]{korn_notes_2017} for more details. 
} By its being situated far away from the core of the Western Asian Transition Zone (see \citetv{chapters/1_Haigetal_Intro}), Bashkardi offers a convenient point of comparison with Iranian languages within that zone.

Bashkardi has been said to belong to the South-Western sub-branch of Iranian (e.g. \citealt{skjaervo1989baskardi}: 846). However, the differences among the Bashkardi varieties – particularly between North (NBsh.) and South Bashkardi\il{Bashkardi!South} (SBsh.) – and the features they share with Balochi\il{Balochi} (which is classified as North-Western Iranian) suggest that ``Bashkardi'' could be the result of a linguistic area where Iranian varieties of different genetic affiliations have converged and developed shared features (cf. \citealt{korn_greeting_2021}: 302f.). In this sense, ``Bashkardi'' is a collective term ``merely on the basis of territorial coverage'' (\citealt{voskanian2007lexical}: 122).

% \setlength{\footheight}{49.46017pt}
\largerpage

\subsection{Data and previous works}\label{Bashkardi:ss:1.2}

The data for this article are from recordings made by Ilya Gershevitch during his visit to Iran in 1956. He did not publish any text, but did use his data for his articles on historical linguistics.\footnote{
See \citet{gershevitch_travels_1959} for an account of his journey, \citet{korn_voices_2015} for a description of Gershevitch’s materials and \citeauthor{korn_notes_2017} (\citeyear[81]{korn_notes_2017}, \citeyear[301–302]{korn_greeting_2021}) for work on the data. 
} The data he cites there are summarised in \citet{skjaervo1989baskardi,skjaervo1989languages}. 

Two of the texts (H and G in what follows) are published in \citet{korn_bashkardi_2021,korn_greeting_2021}, and the corpus used for the present study is available at \citet{korn_bashkardi_N_2022,korn_bashkardi_S_2022}.\footnote{\label{fn:korn:3}All examples in what follows are from \citet{korn_bashkardi_N_2022,korn_bashkardi_S_2022}, specifying North / South Bashkardi\il{Bashkardi!South}\il{Bashkardi!North} as well as the text and sentence number in \url{https://multicast.aspra.uni-bamberg.de/resources/wowa/\#corpora}. Further examples can be found in the works mentioned in this section.}

Elements of Bashkardi\il{Bashkardi} grammar are summarised in \citeauthor{korn_notes_2017} (\citeyear{korn_notes_2017}, description of the nominal system with historical interpretation), and a brief sketch of North Bashkardi\il{Bashkardi!North} grammar is presented in \citet{korn_greeting_2021}. 

Some data are available from varieties spoken in the region today: \citet{seddiqinejad2010barresi}, describing the South Bashkardi\il{Bashkardi!South Dahwast} of Dahwast; \citet{barbera_cannibal_2023,barbera2024Fool} on the South Bashkardi\il{Bashkardi!South Garu} of Garu; \citet{barbera_lingua_2005,barbera_minabi_2020} on the variety of the town of Minab; and \citet{pelevin2010materials} on the speech of Bandar-Abbas. A comparison with Gershevitch's recordings shows that the morphosyntax of Bashkardi has since become much more similar to that of Persian\il{Persian} (see \citealt{korn_notes_2017}: 93–95), which highlights the importance of the data of 1956. 

\begin{figure}
 \includegraphics[width=\textwidth]{figures/Bashkardi_fig_1.png}
 \caption{Location of Bashkardi varieties with locations mentioned in Gershevitch's materials}
 \label{Bashkardi:fig:1}
\end{figure}

\subsection{Some elements of Bashkardi grammar}\label{Bashkardi:ss:1.3}

For the discussion to follow, some grammatical features need to be kept in mind (see also \sectref{Bashkardi:ss:2.4.0}). In Bashkardi as defined in Sections \ref{Bashkardi:ss:1.1}-\ref{Bashkardi:ss:2}, there is no case distinction. The only elements which are (historically) case-marked are the pronominal clitics (enclitic pronouns, \textsc{pc}). They are used for the possessor and the indirect object\is{object!indirect} as well as for the \isi{direct object} in the present domain (i.e. in clauses whose verb form is based on the present stem), and for the transitive subject in the past domain (clauses with verb forms based on the past stem; this includes the perfect forms). In the latter domain, verbal agreement (if any) is with the \isi{direct object}, and in some cases with the indirect object\is{object!indirect} (\isi{recipient}).

In both domains, nominal and pronominal indirect objects (see \sectref{Bashkardi:ss:2.4.2}) and definite direct objects (see \sectref{Bashkardi:ss:2.4.1}) can take a directional \isi{preposition} (`to'), but are mostly unmarked, i.e. all elements marking syntactic relations mentioned in \sectref{Bashkardi:ss:2.1} are employed only sporadically.

Bashkardi is a heavily pro-drop language, which reduces potential examples of pronominal arguments as well as of other arguments that speakers can retrieve from the context. 


In addition to the pronominal clitics, Bashkardi shows a number of other items which are consistently realized as enclitic to the preceding word, viz. the directional \isi{clitic} (\sectref{Bashkardi:ss:2.1.3.2}), the marker of \isi{specificity} (\sectref{Bashkardi:ss:2.1.2}), the possessor \isi{clitic} (\sectref{Bashkardi:ss:2.1.1.2}), the forms of the \isi{copula}, the subordinator \textit{ke} (see \sectref{Bashkardi:ss:2.3}) and the coordinating connective \textit{o} `and'.\footnote{\textit{ke} and \textit{o} are noted as independent words and without `=' in the examples below; the same applies to the forms of the \isi{copula} unless they are part of a verb form.}

There are important differences between North and South Bashkardi\il{Bashkardi!North}\il{Bashkardi!South} (cf. \sectref{Bashkardi:ss:1.1}), which include the use of TAM markers as well as the forms of the pronominal clitics and the verbal endings. Furthermore, there is a certain amount of variation also within each dialect group, and among speakers individually.

\section{Word order profile}\label{Bashkardi:ss:2}


\subsection{Noun phrases and adpositional phrases}\label{Bashkardi:ss:2.1}

\begin{sloppypar}
Bashkardi noun phrases and adpositional phrases are \isi{head-initial} (right-branching) as they are in Persian\il{Persian}, but there are also left-branching patterns (which is the dominant pattern in Balochi\il{Balochi}). Demonstratives and numerals precede the noun.\footnote{For the Bsh. noun phrase, see \citet{korn_notes_2017}, specifically pp. 91--92 for adjectives, pp. 88--90 for \isi{possessive} NPs, p. 85 for the marking of \isi{definiteness} and p. 83--86 for number, and 86--87 for adpositional elements.}
\end{sloppypar}

\subsubsection{Adjectives and possessors}\label{Bashkardi:ss:2.1.1}

In general, adjectives and possessors follow the head noun and are linked to it by means of a \isi{clitic} called ``ezâfe'' (\textit{=i }and variants, \textsc{ez}), which is also used for arguments of complex predicates (see \sectref{Bashkardi:ss:2.2}).

\paragraph{Adjectives}\label{Bashkardi:ss:2.1.1.1}\mbox{}\\

{\noindent}Examples for the general pattern of an adjective being linked to its head noun by the ezâfe include (\ref{Bashkardi:ex:4}) and (\ref{Bashkardi:ex:13a}); an example without ezâfe is (\ref{Bashkardi:ex:11}).

Some adjectives are preposed, cf. \textit{aǰab} in (\ref{Bashkardi:ex:1}) and \textit{heil=o} in (\ref{Bashkardi:ex:3a}) and (\ref{Bashkardi:ex:3b}) – surely copied from Persian\il{Persian}, where \textit{aǰab} and \textit{xeili} are preposed as well.

\ea\label{Bashkardi:ex:1}
North Bashkardi \il{Bashkardi!North}(I:59) \\
\gll aǰab nöuk ei mõ hast=ī \\
amazing grandson to I exists=\textsc{pc.3sg} \\
\glt `What an amazing grandson I have!' 
\z

\paragraph{Possessors}\label{Bashkardi:ss:2.1.1.2}\mbox{}\\

{\noindent}The general pattern of a possessor being linked to the possessum by ezâfe is shown in \textit{mahala=i} \textsc{pn} `X's place' in (\ref{Bashkardi:ex:16d}) and \textit{čürak-e šēr} in (\ref{Bashkardi:ex:14}). No ezâfe is required when the possessum is a body part, or when the possessor is a \isi{pronoun} of the 1st or 2nd person (\ref{Bashkardi:ex:4}, \ref{Bashkardi:ex:8b}) or the reflexive \isi{pronoun}, although the ezâfe may be used here as well (\ref{Bashkardi:ex:13a}, \ref{Bashkardi:ex:17b}).

There is also a (rare) \isi{head-final} pattern (looking like a Balochi\il{Balochi} noun phrase), with or without the possessor \isi{clitic} (\textit{=ī}, \textsc{poss}), (\ref{Bashkardi:ex:2}). 

\ea\label{Bashkardi:ex:2}
North Bashkardi \il{Bashkardi!North}(H:88, 103, 118, 134) \\
\gll tūlag-a=ī sīr=å \\ 
jackal\textsc{-pl}\textsc{=poss} wedding\textsc{=dir} \\ 
\glt `[I will go] to the jackals' wedding.'
\z

Pronominal clitics are frequently used for the possessor (\ref{Bashkardi:ex:6a}, \ref{Bashkardi:ex:7b}, \ref{Bashkardi:ex:10c}, \ref{Bashkardi:ex:19}). 

They are also used for the possessor in the \textit{mihi est} construction (the Bashkardi pattern expressing `have', see \sectref{Bashkardi:ss:2.4.3}), also in addition to a noun phrase (\ref{Bashkardi:ex:10d}) or prepositional phrase (\ref{Bashkardi:ex:1}) expressing the possessor. In most cases, `exists' occupies the clause-final position (\ref{Bashkardi:ex:3a}, \ref{Bashkardi:ex:10b}, \ref{Bashkardi:ex:12a}), but it can also be fronted (\ref{Bashkardi:ex:3b}, \ref{Bashkardi:ex:10b}).


\ea\label{Bashkardi:ex:3}
South Bashkardi \il{Bashkardi!South}(C:31f.) \\
\ea\label{Bashkardi:ex:3a}
\gll \textup{[Kadxodā:]} heil=o būv=et heš \\
{} many\textsc{=spc} date\_palm\textsc{=pc\textsc{.2sg}} exists \\
\glt `Do you have many palm trees?' --
\ex\label{Bashkardi:ex:3b}
\gll heš=om haštåd ben bū \\
exists\textsc{=pc.1sg} eighty trunk\textup{\footnotemark} date\_palm \\
\glt `I have eighty palm trees.'
\z
\z\footnotetext{
\textit{ben} can quite well be considered a numeral classifier here; ‘trunk’ is meant to render the literal meaning.}

\subsubsection{Demonstratives and numerals}\label{Bashkardi:ss:2.1.2}

{Demonstratives} precede the head noun without any linker. This is quite frequent, so that demonstratives approach the functions of a definite article (\ref{Bashkardi:ex:4}, \ref{Bashkardi:ex:5}, \ref{Bashkardi:ex:6a}, \ref{Bashkardi:ex:10c}, \ref{Bashkardi:ex:10d}, \ref{Bashkardi:ex:11}, \ref{Bashkardi:ex:14}, \ref{Bashkardi:ex:19}).

There also is a (rare) suffix \textit{-ak} that seems to express \isi{definiteness} (\ref{Bashkardi:ex:4}). 

\ea\label{Bashkardi:ex:4}
North Bashkardi \il{Bashkardi!North}(H:18) \\
\gll tūla=i xwara=ī hamī måst-ak-ūn mon a-xwar-ed\\
jackal\textsc{=ez} voracious\textsc{=spc} \textsc{dem1} yoghurt\textsc{-def}\textsc{-pl}{\footnotemark} I \textsc{ipfv-}eat\textsc{.prs}\textsc{-3sg} \\
\glt `A voracious jackal keeps eating this yoghurt of mine.'
\z\footnotetext{Certain liquids, dairy products etc. are treated as plural in Bashkardi.}

{Numerals} are followed by the head noun without (\ref{Bashkardi:ex:16b}) or with an intervening numeral classifier (\textit{ben} in (\ref{Bashkardi:ex:3b}), \textit{tå} in (\ref{Bashkardi:ex:10d})). In addition to the numeral `one' (\ref{Bashkardi:ex:10d}), singularity may be expressed by the \isi{specificity} \isi{clitic} (\textsc{spc}, etymologically `one')\footnote{
This term follows \textsc{Heine}, see \citet[85]{korn_notes_2017}.
} \textit{=ī, =ē}, SBsh. also \textit{=ō}, which is placed at the end of the NP (\ref{Bashkardi:ex:4}). When introducing a new entity, the NP frequently shows both (\ref{Bashkardi:ex:9}, \ref{Bashkardi:ex:10a}, \ref{Bashkardi:ex:10b}, \ref{Bashkardi:ex:14}) so that `one' is ``circumposed''; (\ref{Bashkardi:ex:5}) even shows two instances of the \isi{specificity} marker. 

\ea\label{Bashkardi:ex:5} 
North Bashkardi \il{Bashkardi!North}(K:83) \\
\gll yak=ē gozer=ē hast=e hamē mahal \\ 
one\textsc{=spc} important\textsc{=spc} exists=\textsc{cop.3sg}{\footnotemark} \textsc{dem1} place \\
\glt `There is an important man in this place.'
\z

\footnotetext{Cf. the parallel constructions with \textsc{cop.pst} in (\ref{Bashkardi:ex:10a}, \ref{Bashkardi:ex:10b}, \ref{Bashkardi:ex:10d}). For \isi{copula} forms added to a finite verb in Balochi\il{Balochi}, see \citet[651–652]{korn_notes_2019}.}

\subsubsection{Adpositions}\label{Bashkardi:ss:2.1.3}

\paragraph{Prepositions}\label{Bashkardi:ss:2.1.3.1}\mbox{}\\

{\noindent}Syntactic relations can optionally be clarified by prepositions, some of which are clearly borrowed from Persian\il{Persian} (\textit{barå\_i} `for' in (\ref{Bashkardi:ex:16c}), \textit{γeir=e} `except for' in (\ref{Bashkardi:ex:17b}), perhaps also \textit{bå} `with' in (\ref{Bashkardi:ex:11}, \ref{Bashkardi:ex:13a}, \ref{Bashkardi:ex:19})). Note the (sporadic) marking of objects with prepositions (see also \sectref{Bashkardi:ss:1.3}).

For more precise meanings, nouns may be used as prepositions (relational nouns), e.g. \textit{sar} `head; on, above' or \textit{dah} `entrance' in (\ref{Bashkardi:ex:16e}). Prepositions may also be combined with each other (\ref{Bashkardi:ex:6b}) or with a relational noun. 

\ea\label{Bashkardi:ex:6}
South Bashkardi \il{Bashkardi!South}(E:10f.) \\
\ea\label{Bashkardi:ex:6a}
\gll ī dā=yom kāšt=om kūš \\
\textsc{dem1} hand\textsc{=pc.1sg}{\footnotemark} draw\textsc{.pst}\textsc{=pc.1sg} knife \\
\glt `[With] this (= the other) my hand I drew the knife.
\ex\label{Bashkardi:ex:6b}
\gll zat=om be-rū der=e \\
hit\textsc{.pst}\textsc{=pc.1sg} to-on heart\textsc{=pc.3sg} \\
\glt I struck [it] into his [the leopard's] heart.'
\z
\z\footnotetext{\textit{y} is the hiatus-bridging element attaching the pronominal \isi{clitic} \textit{=om}.}

\paragraph{Postposed elements}\label{Bashkardi:ss:2.1.3.2}\mbox{}\\

{\noindent}As for postposed elements, the (rare) directional \isi{clitic}\footnote{
This \isi{clitic} should be distinguished from the ``directional \isi{clitic}'' that attaches to verb forms in Gorani\il{Gorani} and Kurdish\il{Kurdish}, see \citetv{chapters/1_Haigetal_Intro,chapters/9_Mohammadirad_Gorani}.
} \textit{=å / =a} is appended to the noun, or to the NP (note its position following the pronominal \isi{clitic} in (\ref{Bashkardi:ex:7b})). It may occur in combination with a \isi{preposition} (\ref{Bashkardi:ex:16e}). 


% \pagebreak
\ea\label{Bashkardi:ex:7}
North Bashkardi \il{Bashkardi!North}(K:126) \\
\ea\label{Bashkardi:ex:7a}
\gll xo, hålå på ba\textup{-\O} tå be-rr-īn \\ 
well now foot become\textsc{.prs}\textsc{\textsc{-imp}\textsc{.2sg}} so\_that \textsc{mod-}go\textsc{.prs}\textsc{-1pl} \\ 
\glt `Now get up; let's go,
\ex\label{Bashkardi:ex:7b}
\gll tå dehngōn\textup{\footnotemark}=et=a seråk be-dah-om. \\ 
so{\_}that landlord\textsc{=pc\textsc{.2sg}}\textsc{=dir} showing \textsc{mod-}give\textsc{.prs}\textsc{-1sg} \\
\glt [and] I will show [you] your landlord \\ (and you will see how miserable he is now).'
\z
\z

\footnotetext{
\textit{dehngōn} (cf. Persian\il{Persian} \textit{dehqān, dehgān}) shows an unetymological \textit{-n-} throughout this text. 
}

Note also the \isi{possessive} \isi{clitic} mentioned in \sectref{Bashkardi:ss:2.1.1.2}.

\subsection{Verbal expressions}\label{Bashkardi:ss:2.2}

Auxiliaries are rare in Bashkardi; \emph{TAM categories} are expressed by the presence or absence of verbal prefixes, as seen throughout the examples above and below. The perfect system uses the perfect participle followed by the inflected \isi{copula}, as does the progressive, which is built from the past stem or infinitive.\footnote{
See \citet{korn2022areal} and \citet{korn2017contact} for the functions of the prefixes, the progressive and the verbal system in general. 
}

The uninflected element \textit{bååt} `it is necessary' (certainly copied from Persian\il{Persian} \textit{bāyad}, but not a transparent verb form in Bashkardi) is occasionally found; it is placed towards the beginning of the clause; the main verb is in the subjunctive (\ref{Bashkardi:ex:8b}). 

\pagebreak
\ea\label{Bashkardi:ex:8} 
North Bashkardi \il{Bashkardi!North}(J:46f) \\
\ea\label{Bashkardi:ex:8a} 
\gll ei eståd=ī go gorg ke \\
to master\textsc{=pc.3sg} say\textsc{.pst} wolf \textsc{sub} \\
\glt `To the blacksmith said the wolf:
\ex\label{Bashkardi:ex:8b}
\gll to bååt-ē gap=e pošbår mo be-zan-ī \\
you\textsc{.sg} it\_is\_necessary\textsc{-2sg} talk\textsc{=ez} support I \textsc{mod-}hit\textsc{.prs}\textsc{-2sg} \\
\glt ``You have to speak up for me, (...).''{}'
\z
\z

\emph{Complex predicates} are frequent, e.g. \textit{på b-} `get up' (\ref{Bashkardi:ex:7a}) and \textit{seråk da-} `show' (\ref{Bashkardi:ex:7b}). The nominal part can take an \isi{argument} constructed like a possessor (\ref{Bashkardi:ex:8b}), (\ref{Bashkardi:ex:9}), so that it is positioned within the complex predicate.

There is a continuum of combinations of a transitive verb with an indefinite (or generic) \isi{direct object} to conventionalised complex predicates, which can be analysed as single predicative items. For instance, \textit{šekåyat kan-} in (\ref{Bashkardi:ex:13b}) could be analysed as a complex predicate `to complain' or as `lodge a complaint', thus \textit{šekåyat} a \isi{direct object}. (\ref{Bashkardi:ex:8b}) could show \textit{gap-e pošbår mo} `my defense (talk of my support)' as \isi{direct object} of \textit{zan-}; alternatively, \textit{pošbår mo} `my support' could be the \isi{direct object} of the complex predicate \textit{gap zan-} (`talk-hit' is used for `speak' also in Balochi\il{Balochi}, vs. Persian\il{Persian} \textit{ḥarf zan-}); or one could consider \textit{gap-e pošbår zan-} as a complex predicate `speak up', in which case \textit{mo} would be the beneficiary.

There is thus a certain amount of subjectivity in the interpretation, which potentially affects the statistics of several categories. 

\newcounter{mpFootnoteValueSaver}

\ea\label{Bashkardi:ex:9}\setcounter{mpFootnoteValueSaver}{\value{footnote}}
South Bashkardi \il{Bashkardi!South}(A:41) \\
\gll hålå to yeu gap=o seråk=i yamah adeh\textup{-\O} \\
now you\textsc{.sg} one talk\textsc{=spc} showing\textsc{=ez} we{\footnotemark} give\textsc{.prs}\footnotemark\textsc{-imp}\textsc{.2sg} \\
\glt `Now show us some talk [from the tape recorder]!'
\z

\stepcounter{mpFootnoteValueSaver}%
\footnotetext[\value{mpFootnoteValueSaver}]{As in Persian\il{Persian}, the non-verbal element of a complex predicate can take an \isi{argument}, which is attached by the ezâfe  (for which see \sectref{Bashkardi:ss:2.1.1}).}%
\stepcounter{mpFootnoteValueSaver}%
\footnotetext[\value{mpFootnoteValueSaver}]{The \textsc{prs} stem of `give' is \textit{adeh-} in Gershevitch’s SBsh. data; prefixing the imperfective \textit{a-} yields \textit{ādeh-} \citep[848]{skjaervo1989baskardi}.}


There are also \emph{directional preverbs} such as \textit{or-} in (\ref{Bashkardi:ex:16e}).

In the case of \emph{clause-initial verbs}, it is often unclear whether we are looking at an element postposed to the verb, or at an instance of fronting of the verb. The latter may occur at the beginning of a tale, introducing the tale's characters (\ref{Bashkardi:ex:10a}, \ref{Bashkardi:ex:10b})\footnote{
See \citet{korn_once_2020} for features of folk tales in Bashkardi.
} or at the beginning of a new episode, but also in pragmatic contexts that still need to be established (potential examples include (\ref{Bashkardi:ex:6b}); see also the \textit{mihi est} pattern in \sectref{Bashkardi:ss:2.1.1.2}).

\ea\label{Bashkardi:ex:10}
South Bashkardi \il{Bashkardi!South}(K:1-5) \\
\ea\label{Bashkardi:ex:10a}
\gll hast=a ya måldår=ē \\
exists=\textsc{cop}\textsc{.pst}.3sg one rich\_man\textsc{=spc} \\
\glt `There was a rich man. \\ {\hspace{-1cm}} (someone in the background:) Hm.
\ex\label{Bashkardi:ex:10b} 
\gll ya måldår=ē hast=a {\hspace{3cm}} hast=ar=ī ya sålål \\
one rich\_man\textsc{=spc} exists=\textsc{cop.pst}\textsc{.3sg} {} 
exists\textsc{=cop.pst.3sg}\textsc{=pc.3sg} one shepherd \\
\glt There was a rich man, [and] he had a shepherd.
\ex\label{Bashkardi:ex:10c} 
\gll ī sålål=ī fakīr a \\
this shepherd\textsc{=pc.3sg} poor \textsc{cop.pst.3sg} \\
\glt This shepherd of his was poor.
\ex\label{Bashkardi:ex:10d} 
\gll ī fakīr do tå čuk=ī hast=ī=a o ya zã \\
\textsc{dem1} poor\_man two piece child\textsc{=pc.3sg} exists\textsc{=pc.3sg}\textsc{=cop.pst}\textsc{.3sg} and one woman \\
\glt This poor man had two children, and one wife.'
\z
\z

\subsection{Complex sentences}\label{Bashkardi:ss:2.3}

The {subordinator} \textit{ke} introduces any kind of subordinate clause – i.e. relative (\ref{Bashkardi:ex:11}), \isi{complement} (\ref{Bashkardi:ex:13a}), adverbial – as well as quoted speech (\ref{Bashkardi:ex:8a}). The subordinate generally follows the matrix clause. Owing to its \isi{clitic} nature (see \sectref{Bashkardi:ss:1.3}), \textit{ke} is attached to the first stressed element or to the first multi-word constituent if the subordinate precedes the matrix clause, as in (\ref{Bashkardi:ex:12a}), where it even interrupts the noun phrase `as many guns'.

\ea\label{Bashkardi:ex:11}
North Bashkardi \il{Bashkardi!North}(F:24) \\
\gll hamå best-ōn sorx ke bå åteš sorx en \\
\textsc{dem2} pebble\textsc{-pl} red \textsc{sub} with fire red \textsc{cop.3pl} \\
\glt `those red-hot pebbles, which are red-hot from the fire'
\z

\ea\label{Bashkardi:ex:12}
South Bashkardi \il{Bashkardi!South}(E:33f.) \\
\ea\label{Bashkardi:ex:12a}
\gll har-kader ke tofak=an hat \\
any-extent \textsc{sub} gun\textsc{=pc.1pl} \textsc{cop.pst.3sg} \\
\glt `Insofar as we had guns (any amount of guns that we had),
\ex\label{Bashkardi:ex:12b}
\gll doulat=ī a dā-y a-bert-om \\
State\textsc{=pc.3sg} from hand\textsc{-hi} \textsc{ipfv-}carry\textsc{.pst}\textsc{-1pl} \\
\glt the State kept taking [them] from our hand.'
\z
\z

\ea\label{Bashkardi:ex:13}
South Bashkardi \il{Bashkardi!South}(D:13f.) \\
\ea\label{Bashkardi:ex:13a}
\gll hål a-xåh-om {\hspace{7cm}} ke ra-m bå ǰamīat=e xailī=e xo \\
now \textsc{ipfv-}want\textsc{.prs}\textsc{-1pl} {} \textsc{sub} go\textsc{.prs}\textsc{-1pl} with population\textsc{=ez} very\textsc{=ez} \textsc{refl} \\
\glt `Now we want (that) we go with a large group of our [people]
\ex\label{Bashkardi:ex:13b}
\gll dar bandir-abbås šekåyat=ī kan-om \\
in \textsc{pn} complaint\textsc{=spc} do\textsc{.prs}\textsc{-1pl} \\
\glt [and] lodge a complaint in Bandar Abbas.'
\z
\z

When \textit{ke} is combined with a nominal (e.g. `[the] time \textsc{sub}' = `when'), this neo-{conjunction} (borrowed or calqued on Persian\il{Persian} models) is accentuated and introduces the subordinate clause. However, this is rare in the data.

Indeed, {chains of main clauses} are often preferred to overt marking of subordination, as is demonstrated by (\ref{Bashkardi:ex:10}) and (\ref{Bashkardi:ex:16}). These passages also show that Tail-Head-Linkage is frequent. Repeated material often appears in a different \isi{word order}, e.g. with old information placed in front of the verb (\ref{Bashkardi:ex:10b}, \ref{Bashkardi:ex:10c}, \ref{Bashkardi:ex:16b}).

\subsection{Word order in main clauses}\label{Bashkardi:ss:2.4}

\subsubsection{Generalities}\label{Bashkardi:ss:2.4.0}

Like other Ir. languages, Bashkardi main clauses with NPs as constituents are for their majority SOV. However, other word orders are not rare, and this in spite of the absence of marking of noun phrases (see \sectref{Bashkardi:ss:1.3}). In the data used for this project, Bashkardi shows 30\% of non-subject elements in postverbal position. This count includes nouns and pronouns (following the WOWA project's concept, pronominal clitics are not taken into account in the discussion and percentages to follow), with or without adpositional elements, but excludes clauses (e.g. \isi{complement} clauses, which could likewise be considered as direct objects).

Information structure seems to play an important \isi{role} for \isi{word order}, and the status of an element as new or old information may override the SOV pattern and also the preferences for the placement of a particular element mentioned below (see also Sections \ref{Bashkardi:ss:2.2} and \ref{Bashkardi:ss:2.3}). 

Many of the phenomena discussed below are also found in other Iranian languages, but references are limited to other chapters in this volume.\footnote{
Percentages in the text to follow are rounded, and refer to North and South Bashkardi\il{Bashkardi!North}\il{Bashkardi!South} taken together.
}

\subsubsection{Direct objects}\label{Bashkardi:ss:2.4.1}

Overall, 21\% of all direct objects are in postverbal position (\tabref{Bashkardi:tab:1}). Compared to other Iranian languages in the WOWA sample (cf. \citetv{chapters/1_Haigetal_Intro}), Bashkardi has the highest frequency of nominal postverbal direct objects.

\begin{table}
 \centering
 \begin{tabular}{lrrrrrr}
 \lsptoprule
& \multicolumn{2}{l}{Preverbal} & \multicolumn{2}{l}{Postverbal} & \multicolumn{2}{l}{All positions} \\
\midrule
Nominal & 198 & (77\%) & 58 & (23\%) & 256 & (100\%) \\
Pronominal & 30 & (94\%) & 2 & (6\%) & 32 & (100\%) \\
\midrule
All direct objects & 228 & (79\%) & 60 & (21\%) & 288 & (100\%) \\
\lspbottomrule
 \end{tabular}
 \caption{Position of direct objects (n / \%)}
 \label{Bashkardi:tab:1}
\end{table}

\paragraph{Nominal direct objects}\label{Bashkardi:ss:2.4.2.1}\mbox{} \\

\noindent{}{Nominal direct objects} (see also \sectref{Bashkardi:ss:2.2}) are predominantly preverbal (\ref{Bashkardi:ex:4}, \ref{Bashkardi:ex:7b}, \ref{Bashkardi:ex:9}, \ref{Bashkardi:ex:16b}, \ref{Bashkardi:ex:16e}). However, postverbal direct objects such as in (\ref{Bashkardi:ex:6a}) and (\ref{Bashkardi:ex:14}) make up 23\%, which is an important percentage, even if significantly less than the overall average of 30\% mentioned for postverbal non-subject elements in \sectref{Bashkardi:ss:2.4.0}.

\ea\label{Bashkardi:ex:14}
North Bashkardi \il{Bashkardi!North}(K:46) \\
\gll hamē čürak=e šēr zar=ī ya gart=ī \\
\textsc{dem1} kid\textsc{=ez} lion hit\textsc{.pst}\textsc{=pc.3sg} one roar\textsc{=spc} \\
\glt `The lion cub gave a roar.'
\z

\paragraph{Pronominal direct objects}\label{Bashkardi:ss:2.4.2.2}\mbox{} \\

\noindent{}{Pronominal direct objects} only occur under specific pragmatical conditions, since being expressed by a \isi{pronoun} would imply that the referents are known, in which situation pro-drop applies (see \sectref{Bashkardi:ss:1.3}); this is the case for the knife in (\ref{Bashkardi:ex:6b}), the guns in (\ref{Bashkardi:ex:12b}), the goat kid in (\ref{Bashkardi:ex:15a}) and the talk in (\ref{Bashkardi:ex:17c}). When they do occur, they are nearly always in preverbal position, and indefinite pronoun\is{pronoun!indefinite}s always so (\ref{Bashkardi:ex:15a}, \ref{Bashkardi:ex:17a}).

\ea
North Bashkardi \il{Bashkardi!North}(G:29) \\
\ea\label{Bashkardi:ex:15a}
\gll har-čī mon a-g-om: ma-koš\textup{-\O}, \\
any-thing I \textsc{ipfv-}say\textsc{.prs}\textsc{-1sg} \textsc{proh-}kill\textsc{.prs}\textsc{-imp}\textsc{.2sg} \\
\glt `However much I say: ``Don't kill [it] (the goat kid)!'',
\ex\label{Bashkardi:ex:15b}
\gll å a-koš-i=e \\
\textsc{dem2} \textsc{ipfv-}kill\textsc{.prs}\textsc{-3sg}\textsc{=pc.3sg} \\
\glt  he will kill it.'
\z
\z

\subsubsection{Targets}\label{Bashkardi:ss:2.4.2}

Targets, a cover-term for elements indicating the end point of an action or event (see \tabref{Bashkardi:tab:2} and \citealt{Korn2022Targets}: 90), occur postverbally in more than 50\% of all instances, which is significantly more than the 30\% found across all non-subject elements. Pronouns are rarely used in these functions; they do not share the postverbal tendency.\footnote{The non-pronominal instances include nouns and adverbial expressions (e.g. `here, to this place'), which are common for Targets.}

\begin{table}
\setcounter{mpFootnoteValueSaver}{\value{footnote}}
    \caption[Postverbalness of Targets]{Postverbalness of Targets{\protect\footnotemark} }
    \begin{tabular}{lrrrrrr}
\lsptoprule
& total n & \multicolumn{2}{r}{of which} & \multicolumn{3}{l}{pronouns/} \\
& & \multicolumn{2}{r}{postverbal} & \multicolumn{3}{r}{of which postverbal} \\
\midrule
Goals of motion & 102 & 74 & (73\%) & 2 & 1 & (50\%) \\
Goals of caused motion proper & 25 & 18 & (72\%) & 3 & 0 & (0\%) \\
`put'-expressions & 22 & 8 &(36\%) & 1 & 0 & (0\%) \\
\midrule
Recipients & 35 & 21 & (60\%) & 16 & 5 & (31\%) \\
Beneficiaries and rec-ben\footnotemark & 16 & 6 & (38\%) & 9 & 1 & (11\%) \\
\midrule
Addressees & 7 & 2 & (29\%) & 2 & 0 & (0\%) \\
\midrule
Final states & 39 & 6 & (15\%) & 1 & 0 & (0\%) \\
\midrule
Sum & 246 & 135 & (55\%) & 34 & 7 & (21\%) \\
\lspbottomrule
    \end{tabular}
    \label{Bashkardi:tab:2}
\end{table}

\stepcounter{mpFootnoteValueSaver}%
\footnotetext[\value{mpFootnoteValueSaver}]{The n here refers to the total number of the category, e.g.: there are 102 goals of motion in the data, of which 74 (73\%) are postverbal; 2 of the 102 instances are pronouns, of which one is postverbal.}%
\stepcounter{mpFootnoteValueSaver}%
\footnotetext[\value{mpFootnoteValueSaver}]{``rec-ben'' is a category for items where it is difficult to decide whether they are recipients or beneficiaries.}

\begin{sloppypar}
As for all languages discussed in Chapter 1 (cf. \citetv{chapters/1_Haigetal_Intro}), \emph{Goals of verbs of motion} show the highest postverbal percentage. Preverbal instances are often part of a Tail-Head-Linkage chain (see \sectref{Bashkardi:ss:2.3}), as in (\ref{Bashkardi:ex:16b}).
\end{sloppypar}

\ea\label{Bashkardi:ex:16}
South Bashkardi \il{Bashkardi!South}(A:2--8) \\
\ea\label{Bashkardi:ex:16a}
\gll sabåh a-rra-īn gaverx \\
morning \textsc{ipfv-}go\textsc{.prs}\textsc{-1sg} \textsc{pn} \\
\glt `In the morning I go to Gaverx. 
\ex\label{Bashkardi:ex:16b}
\gll gaverx a-rra-īn čūr xom būr a-kan-īn \\
\textsc{pn} \textsc{ipfv-}go\textsc{.prs}\textsc{-1sg} four date load \textsc{ipfv-}do\textsc{.prs}\textsc{-1sg} \\
\glt I go to Gaverx (When I have arrived in Gaverx...), I load four [loads of] dates.
\ex\label{Bashkardi:ex:16c}
\gll a-p-īn ba mahala barå\_i amīrī \\
\textsc{ipfv-}come\textsc{.prs}\textsc{-1sg} to home for \textsc{pn} \\
\glt I come [and bring them] home for Amiri.
\ex\label{Bashkardi:ex:16d}
\gll az\_bād\_e a-rra-īn mahale=i ahmad=i madī \\
after\_that \textsc{ipfv-}go\textsc{.prs}\textsc{-1sg} home\textsc{=ez} \textsc{pn}\textsc{=ez} \textsc{pn} \\
\glt Then I go the house of Ahmad Mahdi.
\ex\label{Bashkardi:ex:16e}
\gll ahmad=i mādī or-gir-īn a-p-īn dah gare=a \\
\textsc{pn}\textsc{=ez} \textsc{pn} up-take\textsc{.prs}\textsc{-1sg} \textsc{ipfv-}go\textsc{.prs}\textsc{-1sg} into \textsc{pn}\textsc{=dir}\footnotemark \\
\glt I take Ahmad Mahdi [and] I come to Gaverx.'
\z
\z

\footnotetext{It is not clear whether \textit{gare-a} refers to Gaverx or to another place.}

Targets of expressions meaning `put' (\ref{Bashkardi:ex:18}) pattern like the overall average. They thus differ from \emph{Goals of caused motion} such as `bring', `send', which pattern like Goals of motion (\ref{Bashkardi:ex:17c}).

\ea\label{Bashkardi:ex:17}
South Bashkardi \il{Bashkardi!South}(A:47f.) \\
\ea\label{Bashkardi:ex:17a}
\gll to heč a-n-k-en \\
you\textsc{.sg} nothing \textsc{ipfv-neg-}do\textsc{.prs}\textsc{-2sg}  \\
\glt  `You don't do anything
\ex\label{Bashkardi:ex:17b}
\gll γeir=e gap-an=e yamah a-čīn-e \\
besides\textsc{=ez} talk\textsc{-pl}\textsc{=ez} we \textsc{ipfv-}collect\textsc{.prs}\textsc{-2sg} \\
\glt but collect our talk,
\ex\label{Bashkardi:ex:17c}
\gll o or-gir-e {\hspace{7cm}} o a-rr-e a-bar-e ba tehrūn \\
and up-take\textsc{.prs}\textsc{-2sg} {} and \textsc{ipfv-}go\textsc{.prs}\textsc{-2sg} \textsc{ipfv-}carry\textsc{.prs}\textsc{-2sg} to \textsc{pn} \\
\glt take [it] and go [and] bring [it] to Tehran.'
\z
\z

\ea\label{Bashkardi:ex:18} 
North Bashkardi \il{Bashkardi!North}(F:32) \\
\gll å wurå dega a-hr-end=eh \\
\textsc{dem} there again \textsc{ipfv-}put\textsc{.prs-3pl}\textsc{=pc.3sg} \\
\glt `Then they put it (the bread) there (aside).'
\z

\emph{Recipients} are somewhat Goal-like in terms of placement, while \emph{Beneficiaries} pattern like Goals of `put'.

\emph{Addressees} are rare, as they are usually retrievable from the context (see \sectref{Bashkardi:ss:1.3}). Where they are expressed (\ref{Bashkardi:ex:8a}), they pattern like the overall average.

A postverbal example of postverbal \emph{final state} (`turns into a stone / is turned into a stone') is `snakes' in (\ref{Bashkardi:ex:19}), which is even marked by a \isi{preposition}, recalling a similar pattern in Kurdish\il{Kurdish} (see \citealt{Haig2022PostPredicateCon}). However, this position is rare in the data.

\ea\label{Bashkardi:ex:19}
North Bashkardi \il{Bashkardi!North}(K:129) \\
\gll hamå mål=ī kolliya büd-e vå mår \\
\textsc{dem2} cattle\textsc{=pc.3sg} entirely become\textsc{.pst-prf} with snake \\
\glt `... [and] that cattle of his had all become snakes.'
\z

\subsubsection{Other obliques}\label{Bashkardi:ss:2.4.3}

Non-subject elements other than direct objects and Targets occur mostly preverbally, but all categories do show postverbal instances (\tabref{Bashkardi:tab:3}).

\begin{table}
    \centering
    \begin{tabular}{lrrrrrr}
\lsptoprule
& total n & \multicolumn{2}{r}{of which} & \multicolumn{3}{l}{pronouns/} \\
& & \multicolumn{2}{r}{postverbal} & \multicolumn{3}{r}{of which postverbal} \\
\midrule
Instrument and Comitative & 12 & 8 & (67\%) & 2 & 2 & (100\%) \\
\midrule
Location & 16 & 6 & (38\%) & --- & & \\
Predicative location & 13 & 2 & (20\%) & --- & & \\
\midrule
Predicative & 50 & 5 & (10\%) & 7 & 0 & (0\%) \\
Possessum & 47 & 6  & (13\%) & 5 & 0 & (0\%) \\
Ablative, Cause and Stimulus & 32 & 3 & (9\%) & 2 & 0 & (0\%) \\
\midrule
Time & 62 & 3 & (5\%) & --- & & \\
Other & 49 & 12 & (24\%) & 2 & 0 & (0\%) \\
\midrule
Sum & 281 & 45 & (16\%) & 18 & 2 & (11\%) \\
\lspbottomrule
    \end{tabular}
    \caption{Postverbalness of other obliques{\protect\footnotemark[19]}}
    \label{Bashkardi:tab:3}
\end{table}

Instrumentals and \isi{comitative} expressions are predominantly found following the verb (\ref{Bashkardi:ex:6a}), while the preverbal position also occurs (\ref{Bashkardi:ex:13a}). 

Locational expressions are found both preceding (\ref{Bashkardi:ex:13b}) and following the verb (\ref{Bashkardi:ex:5}). Even more strongly preverbal are predicative expressions of location (`X is in the house / on the table') and other predicatives (`X is green / Y is my uncle'). 

The same applies to ablative-like expressions (\ref{Bashkardi:ex:12b}). 

Patterning similarly are possessed items, which in the WOWA project refers to the X in the \textit{mihi est} pattern `to me is/exists X = I have X' (see \sectref{Bashkardi:ss:2.1.1.2}).\footnote{
The possessor in the \textit{mihi est} construction could alternatively be interpreted as \isi{recipient} or beneficiary, which would affect the statistics. 
}

Temporal adverbials are found in clause-initial position in the vast majority of cases (\ref{Bashkardi:ex:9}, \ref{Bashkardi:ex:7a}, \ref{Bashkardi:ex:13a}, \ref{Bashkardi:ex:16a}, \ref{Bashkardi:ex:16d}); their placement thus is not so much a question of pre- vs. postverbal. 

\section{Areal features}\label{Bashkardi:ss:3}

As mentioned in \sectref{Bashkardi:ss:2.4.0}, Bashkardi shares the basic SOV order and other features with other Iranian languages. Its basically \isi{head-initial} structure of nominal and adpositional phrases is shared with Persian\il{Persian}, which obviously has influenced all other languages spoken in the country (and beyond). In other respects, however, Bashkardi is closer to Balochi\il{Balochi}, by which it is geographically surrounded (and with which it must surely have been in contact for centuries), for instance in the preservation of (some type) of (post-)ergativity. Bashkardi differs from both Persian\il{Persian} and Balochi\il{Balochi} (but somewhat agrees with late Middle Persian\il{Persian (Middle)}) in the absence of marking of arguments, since the prepositions and postposed elements mentioned in \sectref{Bashkardi:ss:2.1} are used only sporadically to mark the direct or indirect object\is{object!indirect} or the \isi{Goal} of movement.

Some of the features just mentioned are discussed in \citet{korn2022areal}, which also suggests the possibility that only (some of) North Bashkardi\il{Bashkardi!North} is of the Persian\il{Persian} sub-branch (South West Iranian) to which Bashkardi has been held to belong, while (some of) South Bashkardi\il{Bashkardi!South} could be a Balochi\il{Balochi} dialect historically (see \sectref{Bashkardi:ss:1.1}). In this case, both \isi{head-final} and \isi{head-initial} noun phrase structures could be inherited in Bashkardi, contributing to the variation found in the data.

Seeing that \isi{word order} is rather free in many Iranian languages (including colloquial Persian\il{Persian (colloquial)}, but excluding more strictly verb-final standard Persian\il{Persian}), I assume the same freedom for their Middle Iranian predecessors. I also think that a tendency towards the postverbal position of Goals of verbs of movement (and maybe some other types of Targets) could be inherited from Middle Iranian.\footnote{
Cf. \citet[122]{Korn2022Targets}, and see \citet{jugel_word_2022} for a study of \isi{word order} in Middle Iranian. See also \citetv[\sectref{Balochi:ss:6}]{chapters/4_NourzaeiHaig_Balochi} for discussion.}

There is a rather far-reaching agreement of Bashkardi \isi{word order} with the South Balochi\il{Balochi} data investigated in \citet{Korn2022Targets}, while there are also some points of difference. A systematic comparison of the two languages will be the \isi{topic} of a separate article, which will also discuss the influence of \isi{weight}, \isi{flagging}, etc. to the placement of non-subject elements. 

\section*{Abbreviations}
\begin{tabularx}{.45\textwidth}{lQ}
\textsc{cop} & {copula} \\
\textsc{def} & marker of {definiteness} (see \ref{Bashkardi:ss:2.1.2})  \\
\textsc{dem} & demonstrative {pronoun} (1: proximal, 2: distal)  \\
\textsc{dir} & directional {clitic} (see \ref{Bashkardi:ss:2.1.3.2}) \\
\textsc{ez} & ezâfe (see \ref{Bashkardi:ss:2.1.1})  \\
\textsc{hi} & hiatus-bridging consonant \\
\textsc{imp} & imperative \\
\textsc{ipfv} & imperfective prefix  \\
\textsc{mod} & modal prefix (for subjunctive and imperative)  \\
n & number of instances  \\
NBsh. & North Bashkardi  \\
\textsc{neg} & negation \\
\end{tabularx}
\begin{tabularx}{.45\textwidth}{lQ}
\textsc{NP} & noun phrase \\
\textsc{pc} & pronominal {clitic} (see \ref{Bashkardi:ss:1.3})  \\
\textsc{pl} & plural  \\
\textsc{pn} & name  \\
\textsc{poss} & possessor {clitic} (see \ref{Bashkardi:ss:2.1.1.2})  \\
\textsc{prf} & perfect (participle)  \\
\textsc{proh} & prohibitive prefix  \\
\textsc{prs} & present stem  \\
\textsc{pst} & past stem  \\
\textsc{refl} & reflexive {pronoun}  \\
SBsh. & South Bashkardi  \\
\textsc{sg} & singular  \\
\textsc{spc} & marker of {specificity} (see \ref{Bashkardi:ss:2.1.2})  \\
\textsc{sub} & subordinator (see \ref{Bashkardi:ss:2.3})  \\
\textsc{tam} & tense-aspect-mood  \\
\end{tabularx}\medskip

\section*{Acknowledgements}

I am grateful to Gerardo Barbera, Thomas Jügel and Geoffrey Haig for very useful comments. All percentages mentioned above were kindly calculated by Christian Rammer, who also made the map.




{\sloppy\printbibliography[heading=subbibliography,notkeyword=this]}

\end{document}
