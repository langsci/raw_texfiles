\documentclass[output=paper,colorlinks,citecolor=brown,draftmode]{langscibook}
\ChapterDOI{10.5281/zenodo.14266343}
\author{Mohammad Rasekh-Mahand\orcid{}\affiliation{Bu-Ali Sina University, Hamadan} and Elham Izadi\affiliation{Bu-Ali Sina University, Hamadan} and Mehdi Parizadeh\affiliation{Bu-Ali Sina University, Hamadan} and Geoffrey Haig\orcid{0000-0002-5410-3692}\affiliation{Bamberg University} and Nils Schiborr\affiliation{Bamberg University}}
\title{Post-predicate elements in modern colloquial Persian: A multifactorial analysis}
\shorttitlerunninghead{Post-predicate elements in modern colloquial Persian}
\abstract{We investigate post-verbal elements in contemporary spoken Persian, based on the HamBam corpus \citep{HaigRasekhMahand2022HamBam}, and compare the results with \citet{frommer_post-verbal_1981}. We apply two multi-variate analyses to the HamBam data (logistic regression, gradient boosting), which suggest semantic/syntactic role (e.g., Goals, direct objects) is the primary predictor of post-verbal placement; other factors, such as weight, are marginal. Our findings confirm those of \citet{frommer_post-verbal_1981} for the least formal spoken registers of Persian (>80\% rates of post-verbal Goal\is{Goal!post-verbal}s). However, we detect a shift in register distribution in today's spoken language compared to the late 1970s.}

%move the following commands to the ``local...'' files of the master project when integrating this chapter
% \usepackage{tabularx}
% \usepackage{langsci-optional}
% \usepackage{langsci-gb4e}
% \usepackage{enumitem}
% \bibliography{localbibliography}
% \newcommand{\orcid}[1]{}
% \let\eachwordone=\itshape

\IfFileExists{../localcommands.tex}{
 \addbibresource{../collection_tmp.bib}
 \addbibresource{../localbibliography.bib}
 % add all extra packages you need to load to this file

\usepackage{tabularx,multicol}
\usepackage{url}
\urlstyle{same}

\usepackage{listings}
\lstset{basicstyle=\ttfamily,tabsize=2,breaklines=true}

\usepackage{langsci-basic}
\usepackage{langsci-optional}
\usepackage{langsci-lgr}
\usepackage{langsci-osl}
% \usepackage{./langsci/styles/langsci-lgr}
% \usepackage{./langsci/styles/langsci-osl}
% \usepackage{langsci-gb4e}

\usepackage{tikz}
\usetikzlibrary{patterns,calc}
\pgfdeclarepatternformonly{south east lines}{\pgfqpoint{-0pt}{-0pt}}{\pgfqpoint{3pt}{3pt}}{\pgfqpoint{3pt}{3pt}}{
    \pgfsetlinewidth{0.6pt}
    \pgfpathmoveto{\pgfqpoint{0pt}{3pt}}
    \pgfpathlineto{\pgfqpoint{3pt}{0pt}}
    \pgfpathmoveto{\pgfqpoint{.2pt}{-.2pt}}
    \pgfpathlineto{\pgfqpoint{-.2pt}{.2pt}}
    \pgfpathmoveto{\pgfqpoint{3.2pt}{2.8pt}}
    \pgfpathlineto{\pgfqpoint{2.8pt}{3.2pt}}
    \pgfusepath{stroke}}
    
\usepackage{stmaryrd}
\usepackage{wasysym}
\usepackage{multirow}
\usepackage{caption}
\usepackage{subcaption}
\usepackage{mathrsfs}
\usepackage{qtree}

\usepackage{linguex}


 %pminos do not split footnotes
% \interfootnotelinepenalty=10000 %Footnote in Laporte chapters has to be split SN


%\DeclareIndexNameFormat{default}{%
%\nameparts{#1}%
%\usebibmacro{index:name}%
%{\index[names]}%
%{\namepartfamily}%
%{\namepartgiveni}%
% {}% L1
% {}% L2
%{\namepartprefix}% generates spurious space L3
%{\namepartsuffix}% generates spurious space L4
%}

%  {\DeclareIndexNameFormat{default}{%
%     \usebibmacro{index:name}{\index[names]}{#1}{#3}{#5}{#7}}}

%\DeclareIndexNameFormat{default}{%
%  \usebibmacro{index:name}{\sindex[nom]}{#1}{#3}{#5}{#7}}

%\DeclareIndexNameFormat{default}{%
%  \usebibmacro{index:name}{\sindex[person]}{#1}{#3}{#5}{#7}}
%\DeclareIndexNameFormat{default}{%
%\nameparts{#1} \usebibmacro{index:name}{\sindex[person]]}{\namepartfamily}{‌​\namepartgiven}{\nam‌​epartprefix}{\namepa‌​rtsuffix}}

%\newcommand{\smiley}{:)}

%\renewbibmacro*{index:name}[5]{%
%\usebibmacro{index:entry}{#1}%
%{\iffieldundef{usera}{}{\thefield{usera}\actualoperator}\mkbibindexname{#2}{#3}{#4}{#5}}}

% \newcommand{\noop}[1]{}

%remove for final
%\overfullrule=1mm

\newcommand{\tobi}[2]}}
\renewcommand{\S}[1]{\tobi{#1}{\textsc{*}}}

% this volume references
% puts: [this volume]
% already defined: \citetv
%\newcommand{\citepv}[1]{(\citeauthor{#1} \citeyear*{#1} [this volume])}
\newcommand{\citealtv}[1]{\citeauthor{#1} \citeyear*{#1} [this volume]}

%parentheses around example number
\newcommand{\pref}[1]{(\ref{#1})}

% in-text examples

\newcommand{\lnex}[1]{\textit{#1}} %target lang word
\newcommand{\lnlit}[1]{(lit.: `#1')} %literal reading
\newcommand{\lnlat}[1]{(#1)} % latinization
\newcommand{\lntrans}[1]{`#1'} %translation
\newcommand{\lnexl}[2]%
{\lnex{#1}{} \lnlat{#2}} % ex with latinization
\newcommand{\lnexlat}[3]{\lnex{#1}{} \lnlat{#2}{} \lntrans{#3}} % ex with latinization and tranl.

%ch01
\newcommand{\co}[1]{\mbox{\textbf{#1}}}

%ch09

\newcommand{\cyrbulg}[1]{\begin{otherlanguage*}{bulgarian}#1\end{otherlanguage*}}


%ch10
\newcommand{\nlp}{{\small NLP}}
\newcommand{\mwe}{{\small MWE}}
\newcommand{\rae}{{\small RAE}}
\newcommand{\lvc}{{\small LVC}}
\newcommand{\pos}{{\small P}o{\small S}}
%\newcommand{\todo}[1]{ \textcolor{red}{#1} }

%\renewcommand{\labelenumi}{\theenumi}
%\ainamefmt{{vv}{ll}{, ff}{, jj}} % fullname

\newcommand{\biberror}[1]{{\color{red}#1}}

\newcommand{\osenovaitem}{--~}
 %% hyphenation points for line breaks
%% Normally, automatic hyphenation in LaTeX is very good
%% If a word is mis-hyphenated, add it to this file
%%
%% add information to TeX file before \begin{document} with:
%% %% hyphenation points for line breaks
%% Normally, automatic hyphenation in LaTeX is very good
%% If a word is mis-hyphenated, add it to this file
%%
%% add information to TeX file before \begin{document} with:
%% %% hyphenation points for line breaks
%% Normally, automatic hyphenation in LaTeX is very good
%% If a word is mis-hyphenated, add it to this file
%%
%% add information to TeX file before \begin{document} with:
%% \include{localhyphenation}
\hyphenation{
    Beck-man
    Ngu-yen
    back-chan-nel
    back-chan-nels
    mo-not-o-nous
    ste-reo-typ-i-cal
}

\hyphenation{
    Beck-man
    Ngu-yen
    back-chan-nel
    back-chan-nels
    mo-not-o-nous
    ste-reo-typ-i-cal
}

\hyphenation{
    Beck-man
    Ngu-yen
    back-chan-nel
    back-chan-nels
    mo-not-o-nous
    ste-reo-typ-i-cal
}

%  \boolfalse{bookcompile}
%  \togglepaper[5]%%chapternumber
}{}

\begin{document}
\maketitle\label{WOWA:ch:7}
\lehead{M. Rasekh-Mahand, E. Izadi, M. Parizadeh, G. Haig \& N. Schiborr}
\section{Introduction}

Persian\il{Persian (New)} has some odd features regarding its \isi{word order} typology. It has prepositions (though the object marker\is{object!marker} is an enclitic \textit{=rā}), and post-nominal adjectives, genitives and relative clauses. These are features generally associated with the \isi{head-initial} languages (\citealt{Dryer1992Greeburg}). Yet the verb occurs in the final position of the clause, especially in written and formal registers (\citealt{FaghiriSamvelian2014Accesibility}: 220, \citealt{HaigRasekhMahand2019}, \citealt{FaghiriSamvelian2020SOV}). In this respect, Persian\il{Persian (New)} \isi{word order} is disharmonic (\citealt{hawkins_asymmetry_2008}), showing a mixture of \isi{head-initial} and \isi{head-final} features. A second aspect of disharmonic \isi{word order} is that while direct objects are fairly consistently pre-verbal (\isi{OV}), certain other kinds of constituents may follow the verb. The first and systematic study of post-verbal phenomena in Persian\il{Persian (New)} is \citegen{frommer_post-verbal_1981} dissertation. This pioneering study focused exclusively on the syntax of less formal Persian\il{Persian (New)} (`Informal Persian', IP), was based on a corpus graded according to levels of formality within IP, and employed statistical analyses rigorous for its time. We summarize Frommer's main findings in \sectref{Persian:2}.

In this paper, we take another look at post-predicate phenomena in modern colloquial Persian\il{Persian (colloquial)}, using data made available through the HamBam corpus (\citealt{HaigRasekhMahand2022HamBam}) which contains annotated recordings of contemporary spoken Persian\il{Persian (colloquial)}.\footnote{\url{https://multicast.aspra.uni-bamberg.de/resources/hambam/}} In order to facilitate comparison with the other data-sets from WOWA (\citetv{chapters/1_Haigetal_Intro}, \sectref{Intro:ss:3}) we have selected a sub-set of texts from HamBam and created a data-base conforming with the WOWA format, online available as \citet{Izadi2022Persian}. On the basis of this data, we are able to compare colloquial spoken Persian\il{Persian (colloquial)} of today with \citegen{frommer_post-verbal_1981}, data compiled in the late 1970's, allowing us to address the question of whether \citegen{frommer_post-verbal_1981} findings still hold after 40 years. In \sectref{Persian:2}, a summary of \citegen{frommer_post-verbal_1981} study and his findings is provided. In Sections \ref{Persian:3} and \ref{Persian:4}, we present our findings based on data from HamBam, pursuing both a qualitative and quantitative approach, the latter testing the effects of predictor variables, including \isi{weight}, \isi{role}, \isi{flagging}, \isi{animacy}, and \isi{register}. \sectref{Persian:5} compares our findings with \citegen{frommer_post-verbal_1981}, identifying a hitherto undetected shift in \isi{register} differentiation in spoken Persian\il{Persian (colloquial)} of the late 1970's, and today's language. \sectref{Persian:6} summarizes the main findings.

\section{\citet{frommer_post-verbal_1981}}\label{Persian:2}

Post-predicate elements in Persian\il{Persian (New)} have generally received only passing attention, particularly as they are generally considered to be a feature of informal spoken language, thus apparently lacking systematicity. However, \citet[201--205]{Lazard1957Persian} had already observed that in the colloquial spoken language, spatial Goals were normally post-posed, but the phenomenon did not attract more systematic investigation until \citegen{frommer_post-verbal_1981} dissertation. In his study, Frommer focussed entirely on post-verbal elements, conducting a systematic investigation across several registers of what he refers to as ``Informal Persian'' (IP). The first subcorpora of his corpus consists of informal conversations in a home setting recorded by an in-group member (2595 clauses). The second part is spoken, but more formal, based on broadcasts from Radio \textit{Payām} (1068 clauses). The third part is from the dialogue parts of two plays written by a famous Persian\il{Persian (New)} writer, Sādeq Čubak (1670 clauses), and the last part is a children's story \textit{Kuti o Muti}, adapted for radio broadcasting (451 clauses). It is important to note that \citegen{frommer_post-verbal_1981} research does not consider formal written Persian\il{Persian (New)} (e.g. academic, or conservative journalistic texts), but only different registers within Informal Persian (IP), as opposed to the highly formalized written language. His guiding assumption is that while verb-finality is quite strictly maintained in ``Informal Persian''\il{Persian (New)}, IP differs because it frequently permits constituents to occur post-verbally. The aim of \citegen{frommer_post-verbal_1981} work is thus to elucidate the nature and function of the post-verbal elements in IP (\citealt{frommer_post-verbal_1981}: 58--59).

\citet{frommer_post-verbal_1981} noted that among the post-predicate elements, the ``Goals'' of verbs of motion and caused motion were among the most frequently post-posed elements. Frommer used the term `destination', which includes physical places, pro-forms (\textit{jā} `place'; \textit{injā} `here'; \textit{unjā} `there'; \textit{kojā} `where'), and abstract, or what he calls quasi-destinations (e.g. \textit{raft xarid} `went shopping'). This usage is close enough to the WOWA term `Goal', which we will adopt here throughout. The following examples illustrate Goals, with the assumed canonical pre-predicate placement in (\ref{Persian:ex:1}) contrasting with the post-predicate placement in (\ref{Persian:ex:2}) (post-predicate elements are in bold through the paper):


\ea\label{Persian:ex:1}
Colloquial New Persian \il{Persian (colloquial)}(constructed) \\
\gll mi-xā-d be madrese be-r-e \\
\textsc{ind-}want\textsc{.prs-3sg} to school \textsc{sbjv-}go\textsc{.prs-3sg} \\
\glt `He wants to go to school.'
\z

\ea\label{Persian:ex:2}
Colloquial New Persian \il{Persian (colloquial)}(constructed) \\
\gll mi-xā-d be-r-e \textbf{be} \textbf{madrese} \\
\textsc{ind-}want\textsc{.prs-3sg} \textsc{sbjv-}go\textsc{.prs-3sg} to school \\
\glt `He wants to go to school.'
\z

Turning to \citegen{frommer_post-verbal_1981} actual data, Table 1, adapted from \citet[127]{frommer_post-verbal_1981}, shows overall rates of clauses with post-predicate elements (V-X), and the respective proportions of Goals and non-Goals among the post-predicate elements. We have merged the data from the two play scripts because they do not differ significantly from each other.

\begin{table}
\fittable{
 \begin{tabular}{l r rr rr rr}
\lsptoprule
& & \multicolumn{2}{c}{V-Goal} & \multicolumn{2}{c}{V-non-Goal} & \multicolumn{2}{c}{V-X} \\
\cmidrule(lr){3-4}\cmidrule(lr){5-6}\cmidrule(lr){7-8}
Genre & Clause & N & \% & N & \% & N & \% \\
\midrule
Casual spoken & 2595 & 168 & 6.5 & 262 & 10.1 & 430 & 16.6 \\
Radio \textit{Payām} (spoken) & 1068 & 15 & 1.4 & 123 & 11.5 & 138 & 12.9 \\
Two plays (dialogues, written) & 1670 & 78 & 4.7 & 18 & 1.1 & 96 & 5.7 \\
Children's story (written) & 451 & 14 & 3.1 & 4 & 0.9 & 18 & 4.0 \\
\lspbottomrule
 \end{tabular}
 }
 \caption{Overall frequency of post-predicate elements and non-destination elements \cite[127]{frommer_post-verbal_1981}}
 \label{Persian:tab:1}
\end{table}

As the table (5) shows, post-predicate elements (V-X) are overall more frequent in spoken form compared to written, with finer distinctions obtaining within the two written and two spoken sources. An important difference is that while the majority of post-predicate elements in written form are Goals, in spoken genres, a much wider range of post-predicate elements is attested, and Goals only make up less than half (\citealt{frommer_post-verbal_1981}: 128). Thus, a major distinction between the two spoken and the written registers is that the former tolerates a much wider range of post-predicate elements. 

While Table 1 indicates the number of clauses with post-predicate elements, \tabref{Persian:tab:2} indicates the percentage of Goals which are post-predicate in the different registers:


\begin{table}
 \begin{tabularx}{\textwidth}{lrYY}
\lsptoprule
Genre & Total Goals & V-Goals & Percentage \\
\midrule
Casual spoken & 203 & 168 & 82.8\% \\
Radio \textit{Payām} (spoken) & 38 & 15 & 39.5\% \\
Two plays (dialogues, written) & 149 & 78 & 52.3\% \\
Children's story (written) & 26 & 14 & 53.8\% \\
\lspbottomrule
 \end{tabularx}
 \caption{Overall frequency of post-predicate Goals\is{Goal!post-verbal} \citep[131]{frommer_post-verbal_1981}}
 \label{Persian:tab:2}
\end{table}

The score for spoken casual data is very high compared with other three forms, so it is undeniably the case that casual speech favours higher rates of post-verbal Goal\is{Goal!post-verbal}s (\citealt{frommer_post-verbal_1981} :131). However, even among the other three registers, around 50\% of all Goals are post-verbal, indicating that the phenomenon of post-verbal Goal\is{Goal!post-verbal}s cannot be explained solely through reference to sloppy speech in informal conversational registers. Rather, it must be considered a genuine feature of vernacular Persian\il{Persian (colloquial)}, evident even in (less formal) written language. Frommer also identifies a relationship between \isi{word order} and \isi{flagging}: Goals in post-predicate position are more likely to lack the normal prepositional \isi{flagging}: ``prepositionless destinations are more post-posable'' (\citealt{frommer_post-verbal_1981}: 132). He explains that since prepositionless Goals are associated with casual style, and VX Goals are too, these two casual features are linked together making the post-predicate Goals\is{Goal!post-verbal} more prepositionless (\citealt{frommer_post-verbal_1981}: 183). 

The predicates occurring with post-predicate Goals\is{Goal!post-verbal} reveal a sensitivity to individual lexical verbs. post-predicate Goals\is{Goal!post-verbal} occur mainly with two motion verbs, \textit{raftan}, `to go' and \textit{āmadan}, `to come' and two caused-motion verbs, \textit{gozāštan}, `to put' and \textit{bordan}, `to carry'. \tabref{Persian:tab:3} shows the frequency:

\begin{table}
 \begin{tabularx}{\textwidth}{lrYYr}
\lsptoprule
Predicate & Total Goals & goal-V & V-\isi{Goal} & Percent of V-\isi{Goal} \\
\midrule
\textit{raftan} (to go) & 93 & 12 & 81 & 87.1\% \\
\textit{āmadan} (to come) & 35 & 3 & 32 & 91.4\% \\
\textit{gozāštan} (to put) & 17 & 0 & 17 & 100\% \\
\textit{bordan} (to carry) & 6 & 2 & 8 & 75\% \\
\lspbottomrule
 \end{tabularx}
 \caption{Overall frequency of post-predicate Goals\is{Goal!post-verbal} with specific predicates \citep[133]{frommer_post-verbal_1981}}
 \label{Persian:tab:3}
\end{table}

The tokens for other predicates in \citegen{frommer_post-verbal_1981} data are too few to infer plausible conclusions. But for the verbs in \tabref{Persian:tab:3}, the tokens are sufficient to illustrate the strength of the post-predicate tendency, which is close to categorical. \citet[172]{frommer_post-verbal_1981} summarizes his findings for the casual genre in the form of the following hierarchy for post-posability:

\ea 
The hierarchy of post-posability \citep[172]{frommer_post-verbal_1981}: \\
Goal (without \isi{preposition}) > Goal (with \isi{preposition}) > PP (non-Goal, including IO) > DO (with \textit{rā}) and ADV (without \isi{preposition}) > SU > DO (without \textit{rā})
\z

\tabref{Persian:tab:4} shows the frequencies of post-predicate arguments in the casual spoken \isi{register} in Frommer's data, which underly the hierarchy of post-posability.

\begin{table}
 \begin{tabularx}{\textwidth}{lrYY}
\lsptoprule
Constituent type & Total & VX & \% VX \\
\midrule
Goals with prepositions & 134 & 117 & 87.3\% \\
Goals without prepositions & 69 & 51 & 73.9\% \\
Prepositional arguments (not Goals) & 526 & 95 & 18.1\% \\
Objects with \textit{rā} & 224 & 21 & 9.4\% \\
Adverbs without prepositions & 1270 & 96 & 7.6\% \\
Subjects & 1083 & 52 & 4.8\% \\
Objects without \textit{rā} & 422 & 6 & 1.4\% \\
Total & 3728 & 438 & 12\% \\
\lspbottomrule
 \end{tabularx}
 \caption{Post-predicate elements hierarchy (\citealt[172]{frommer_post-verbal_1981}, casual spoken genre only)}
 \label{Persian:tab:4}
\end{table}

\citet[135]{frommer_post-verbal_1981} also analyzed the effects of \isi{information structure} on post-posing. He distinguished between \isi{focus} (the constituent that conveys new information or asks for information as a wh-element and normally is the \isi{intonation} center of the clause) and non-focused, old, background information as two main parts of \isi{information structure} of the clause. He analyzed only the casual spoken data for this feature. \tabref{Persian:tab:5} shows the statistics of non-focused post-predicate elements.

\begin{table}
 \begin{tabularx}{.8\textwidth}{Xr}
\lsptoprule
Constituent type & \% non-focused \\
\midrule
Goals without prepositions & 8.5\% \\
Goals with prepositions & 15.7\% \\
\midrule
Prepositional arguments (not Goals) & 77\% \\
Objects with \textit{rā} & 83.3\% \\
Adverbs without prepositions & 89\% \\
Subjects & 90.4\% \\
Objects without \textit{rā} & 100\% \\
\lspbottomrule
 \end{tabularx}
 \caption{The frequency of non-focused elements in post-predicate position \citep[137]{frommer_post-verbal_1981}}
 \label{Persian:tab:5}
\end{table}

\begin{sloppypar}
As the table shows, post-predicate constituents are generally non-focused (given) information - but this does not hold for Goals. Thus, while post-predicate position strongly disfavors new information, for Goals, this constraint is neutralized, with the vast majority of post-predicate Goals\is{Goal!post-verbal} being in \isi{focus}. It can be provisionally concluded that \isi{focus} versus non-\isi{focus} is not relevant for the placement of Goals, though it is clearly relevant for other constituents. \citet{frommer_post-verbal_1981} explored the effect of other factors, e.g., clause type (main or subordinate), verb type (simple or complex) and \isi{heaviness}, but he found no significant effects, at least in spoken language genres. 
\end{sloppypar}

\citet[179--181]{frommer_post-verbal_1981} summarizes his main findings as follows: 
\begin{enumerate}[label=(\alph*)]
 \item {P}ost-predicate placement is markedly prevalent in informal spoken Persian\il{Persian (colloquial)} and less frequent in formal written Persian\il{Persian (New)}. 
 \item Goals are the most frequent elements in post-predicate position, and more than 80 percent of them are post-posed in casual speech. 
 \item Goals are mainly new information in post-predicate position, contrary to other post-posed elements. 
 \item Grammatical \isi{weight} has no significant effect on post-posing elements. 
\end{enumerate}

Following \citet[532]{Haiman1980iconicity}, \citet[182]{frommer_post-verbal_1981} postulates that putting Goals in post-predicate position is related to iconicity of sequence, asserting that order of elements in language mirrors order of appearance in experience. Hence, Goals are the \isi{endpoint} of a motion and appearing in final position reflects their nature (see also \citealt{Haig2022PostPredicateCon},  for related claims). Finally, \citet[183]{frommer_post-verbal_1981} asks if post-predicate phenomenon represents an ongoing change: Is Persian\il{Persian (New)} fully grammaticalizing this position? Could it be a sign of changing from SOV to \isi{SVO}? Or is the VX variability a stable situation? As \citet{frommer_post-verbal_1981} recognized, his data could not resolve these questions, but forty years later we are in a better position to address them. 

\section{Post-predicate elements in the HamBam corpus}\label{Persian:3}

The data of this section come from the HamBam corpus (\citealt{HaigRasekhMahand2022HamBam}), a collection of annotated recordings of contemporary spoken Persian\il{Persian (colloquial)}. All figures cited here stem from a data set extracted from HamBam, and analysed in the WOWA framework (\citealt{Izadi2022Persian}). The texts gathered in this corpus are predominantly monological in nature, and represent colloquial, unscripted spoken Persian\il{Persian (colloquial)}. They have been broadly categorized into informal (recordings made in private homes, between kin and friends, concerned with oral history and various anecdotes), and more formal speech (e.g. radio interviews and podcasts), designed for public broadcasting. This broad two-way distinction does not readily map onto \citegen{frommer_post-verbal_1981} four-way distinction; we discuss the issue of \isi{register} in \sectref{Persian:5}. The speakers are of both genders, various ages, different educational levels and occupations. \tabref{Persian:tab:6} shows the total number of analyzed tokens and the rate of post-predicate elements.

\begin{table}
 \fittable{\begin{tabular}{l@{}rr}
\lsptoprule
Total number of clause units identified & 3220 & 100\% \\
Number of analyzed tokens & 1625 & 50.5\% \\
Number of clauses lacking a classifiable token & 1595 & 49.5\% \\
Rate of post-predicate elements (all roles) among analyzed tokens & 413 & 25.4\% \\
\lspbottomrule
 \end{tabular}}
 \caption{Frequency of post-predicate elements in HamBam corpus (figures based on \citealt{Izadi2022Persian})}
 \label{Persian:tab:6}
\end{table}

It is important to note that in keeping with the WOWA data-base structure (see \citetv{chapters/1_Haigetal_Intro}, \sectref{Intro:ss:3}), we have only considered non-subject constituents, hence the number of non-classified tokens is high, since there are many sentences which contain just a subject (see below for a discussion on subjects' status). In addition, in some clauses more than one token is analyzed. This means that ``number of tokens'' means the number of analyzed constituents, and should not be confused with number of clauses (which is the unit used in several other studies). As \tabref{Persian:tab:6} shows, one out of four tokens analyzed occurred after the verb.

WOWA employs a finer-grained, and slightly different classification of constituent types than that used in \citet{frommer_post-verbal_1981}, and for the comparison we adapt the WOWA system. \tabref{Persian:tab:7} shows the overall frequency of post-predicate elements by \isi{role}, including nominal and pronominal tokens.


\begin{table}
 \begin{tabularx}{\textwidth}{lrYY}
\lsptoprule
Constituent type & Total & VX & \% VX \\
\midrule
Caused \isi{Goal} & 60 & 55 & 91.7\% \\
Goal & 206 & 167 & 81.1\% \\
Direct \isi{object} (\textsc{def+indef}) & 437 & 18 & 4.1\% \\
Locative & 146 & 29 & 19.9\% \\
Ablative (source) & 50 & 5 & 10\% \\
Other (non-classifiable) & 315 & 98 & 31.1\% \\
Comitative & 46 & 12 & 26.1\% \\
Instrument & 28 & 4 & 14.3\% \\
`become' \isi{complement} & 24 & 6 & 25\% \\
Addressee & 69 & 8 & 11.6\% \\
Benefactive & 11 & 3 & 27.3\% \\
Recipient+\isi{benefactive} & 13 & 2 & 15.4\% \\
Copula \isi{complement} & 205 & 5 & 2.4\% \\
Stimulus & 4 & 1 & 25\% \\
Recipient & 11 & 0 & 0\% \\
\lspbottomrule
 \end{tabularx}
 \caption{Post-predicate elements of different roles based on HamBam corpus}
 \label{Persian:tab:7}
\end{table}

In the following sections, we first discuss non-direct objects in \sectref{Persian:3.1}, direct objects in \sectref{Persian:3.2}, and briefly touch on subjects in \sectref{Persian:3.3}. For some of the roles in \tabref{Persian:tab:7}, the number of tokens or the number of post-predicate tokens is too small to gain a reliable conclusion, so they are not considered further here.

\subsection{Non-direct objects}\label{Persian:3.1}

\subsubsection{Goals and caused Goals}

It is clear from \tabref{Persian:tab:7} that Goals of verbs of caused motion and motion behave fundamentally differently from all other roles. The frequency of post-predicate Goals\is{Goal!post-verbal} in our data (collapsing caused-motion and simple Goals) is around 83\%, which is more than three times higher than any other single \isi{role}, ignoring the `unclassified' category for a moment. This confirms the special \isi{role} of Goals already identified for Persian\il{Persian (New)} by \citet{frommer_post-verbal_1981}, and since confirmed in other studies on post-predicate elements in Iranian and neighboring languages (\citetv{chapters/1_Haigetal_Intro}, \citealt{Jahani2018Post-verbal}, \citealt{stilo_preverbal_2018}, \citealt{Korn2022Targets}).

Furthermore, the figure of around 80\% matches the figure for post-predicate Goals\is{Goal!post-verbal} in \citegen{frommer_post-verbal_1981} casual spoken data, provided in \tabref{Persian:tab:4} above. It is also replicated in another corpus of spontaneous spoken Persian\il{Persian (colloquial)}, (\citealt{Haig2017Keynote}). This suggests that the approximately 80\% level for post-predicate Goals\is{Goal!post-verbal} is a fairly stable linguistic variable for spoken Persian\il{Persian (colloquial)}, which has not varied significantly over the last 40 years; we turn to this in \sectref{Persian:5} below; in the meantime, we provide illustrative examples of simple and caused Goals from our data.

Goals are the arguments of motion verbs (e.g., \textit{go, come}) and caused Goals are the arguments of caused motion verbs (e.g., \textit{put, bring, send, carry}). The data in \tabref{Persian:tab:7} suggest that Goals of caused motion are more likely to be post-predicate than simple motion Goals, but a Fisher Exact test yields a p-value of 0.0504, which is only borderline significant. Examples (\ref{Persian:ex:4}) and (\ref{Persian:ex:5}) are sentences with caused motion verbs and post-predicate Goals\is{Goal!post-verbal}, while (\ref{Persian:ex:6}-\ref{Persian:ex:7}) illustrate the much rarer pattern with pre-verbal Goal\is{Goal!pre-verbal}s:


\ea\label{Persian:ex:4}
Colloquial New Persian \il{Persian (colloquial)}(\citealt[J, 1226]{Izadi2022Persian})\\
\gll rad kard-e bud-am \textbf{tuy=e} \textbf{čub} \\
send do\textsc{.pst-ptcpl} be\textsc{.pst-1sg} in\textsc{=ez} wood \\
\glt `I had stuck it into wood.'
\z


\newpage
\ea\label{Persian:ex:5}
Colloquial New Persian \il{Persian (colloquial)}(\citealt[J, 1299]{Izadi2022Persian})\\
\gll in rā be-gozār \textbf{ruy=e} \textbf{sar=at} \\
this \textsc{ra} \textsc{imp-}put\textsc{.2sg} on\textsc{=ez} head\textsc{=2sg} \\
\glt `Put this on your head.' 
\z

\ea\label{Persian:ex:6}
Colloquial New Persian \il{Persian (colloquial)}(\citealt[J, 1288]{Izadi2022Persian}) \\
\gll tuy=aš āb rixt-e bud-im \\
inside\textsc{=3sg} water pour\textsc{-ptcpl} be\textsc{.pst-1pl} \\
\glt `We poured water into it.' 
\z

\ea\label{Persian:ex:7}
Colloquial New Persian \il{Persian (colloquial)}(\citealt[F, 0836]{Izadi2022Persian}) \\
\gll in pāy=aš rā kuče bo-gzār-ad \\
this foot\textsc{=3sg} \textsc{ra} alley \textsc{sbjv-}put\textsc{.prs-3sg} \\
\glt `({I}f) he puts his foot in the alley (i.e. {I}f he goes out.).' 
\z

\citet[132]{frommer_post-verbal_1981} suggests verb-specific effects here: \textit{rixtan} `to pour' behaves differently, for example compared with \textit{gozāštan} `to put', where for the former the Goals are not post-posed, but for the latter, all of the Goals appear in post-verbal position. \tabref{Persian:tab:8} shows the most frequent verbs of caused motion in our corpus. The Goals of \textit{āvordan} `bring' categorically appear after the verb and for two other frequent verbs, just one token appears pre-verbally.

\begin{table}
 \begin{tabularx}{\textwidth}{lYYr}
\lsptoprule
Caused \isi{motion predicate} & Total & VX & Percent of total VX \\
\midrule
\textit{āvordan} (to bring) & 11 & 11 & 100\% \\
\textit{gozāštan} (to put) & 12 & 11 & 92\% \\
\textit{bordan} (to carry) & 11 & 10 & 91\% \\
\lspbottomrule
 \end{tabularx}
 \caption{Overall frequency of post-predicate caused Goals with specific predicates}
 \label{Persian:tab:8}
\end{table}

Examples (\ref{Persian:ex:8}-\ref{Persian:ex:11}) illustrate Goals of simple motion, the first two post-verbal and the second two examples pre-verbal: 

\ea\label{Persian:ex:8}
Colloquial New Persian \il{Persian (colloquial)}(\citealt[C, 0256]{Izadi2022Persian}) \\
\gll raft-am \textbf{doktor} \\
go\textsc{.pst-1sg} \textbf{doctor} \\
\glt `I went to (the) doctor.' 
\z


\newpage
\ea\label{Persian:ex:9}
Colloquial New Persian \il{Persian (colloquial)}(\citealt[P, 1849]{Izadi2022Persian}) \\
\gll parid-and \textbf{ruy=e} \textbf{miz} \\
jump\textsc{.pst-3pl} over\textsc{=ez} table \\
\glt `They jumped onto the table.'
\z

\ea\label{Persian:ex:10}
Colloquial New Persian \il{Persian (colloquial)}(\citealt[C, 0237]{Izadi2022Persian}) \\
\gll doktor raft-am \\
doctor go\textsc{.pst-1sg} \\
\glt `I went to (the) doctor.' 
\z

\ea\label{Persian:ex:11}
Colloquial New Persian \il{Persian (colloquial)}(\citealt[ZB, 3016]{Izadi2022Persian}) \\
\gll tu harf-hā=yaš, sohbat-hā=yaš be injā resid \\
in speech\textsc{-pl=3sg} talk\textsc{-pl=3sg} to this reach\textsc{.pst.3sg} \\
\glt `In his speech, his talk reached to this point.'
\z

The overall frequency of post-verbal Goal\is{Goal!post-verbal}s is closely matched by the frequencies of post-verbal Goal\is{Goal!post-verbal}s associated with the two most frequent motion verbs; see \tabref{Persian:tab:9}.

\begin{table}
 \begin{tabularx}{\textwidth}{lYYr}
\lsptoprule
Motion Predicate & Total & VX & Percent of total VX \\
\midrule
\textit{raftan} (to go) & 84 & 69 & 82\% \\
\textit{āmadan} (to come) & 32 & 26 & 81\% \\
\lspbottomrule
 \end{tabularx}
 \caption{Overall frequency of post-predicate Goals\is{Goal!post-verbal} with specific predicates}
 \label{Persian:tab:9}
\end{table}

We conclude that in spoken contemporary Persian\il{Persian (colloquial)}, for Goals of motion and caused motion verbs the default position is post-verbal. 

\subsubsection{Local roles, excluding Goals: Locative and Source}

Apart from Goals, some other roles referring to location such as Locatives and Source, are also relatively frequently postposed. About 20\% of Locatives appear in post-predicate position, illustrated in (\ref{Persian:ex:12}) and (\ref{Persian:ex:13}), while Source is much less frequently postposed (about 10\%), see (\ref{Persian:ex:14})

\ea\label{Persian:ex:12}
Colloquial New Persian \il{Persian (colloquial)}(\citealt[W, 2447]{Izadi2022Persian}) \\
\gll tavaqqof dāšt-e ast \textbf{tu} \textbf{Andimešk} \\
stop have\textsc{.pst-ptcpl} be\textsc{.prs.3sg} in {A}ndimeshk \\
\glt `He stopped in Andimeshk.'
\z

\ea\label{Persian:ex:13}
Colloquial New Persian \il{Persian (colloquial)}(\citealt[W, 2487]{Izadi2022Persian}) \\
\gll šahid šod \textbf{tu} \textbf{jebhe} \\
martyr become\textsc{.pst.3sg} in war \\
\glt `He died as a martyr in war.'
\z

\ea\label{Persian:ex:14}
Colloquial New Persian \il{Persian (colloquial)}(\citealt[Q, 1912]{Izadi2022Persian}) \\
\gll dast=aš rā greft \textbf{az} \textbf{man} \\
hand\textsc{=3sg} \textsc{ra} take\textsc{.pst.3sg} from me \\
\glt `He took his hand from me.'
\z

\subsubsection{Non-local obliques: Instrument, comitative, stimulus}

Among the general obliques, Comitatives are more frequent than Instrument and Stimulus roles in post-predicate position. Out of 46 tokens of Comitatives (\ref{Persian:ex:15},  \ref{Persian:ex:16} and \ref{Persian:ex:17}), 12 tokens (26\%) are post-posed. 
Out of 28 tokens of Instruments, 4 are postposed (\ref{Persian:ex:18}), and there are only two postposed Stimulus tokens (\ref{Persian:ex:19}). The following examples illustrate these roles.

\ea\label{Persian:ex:15}
Colloquial New Persian \il{Persian (colloquial)}(\citealt[N, 1694]{Izadi2022Persian}) \\
\gll ke yeki be-š-im \textbf{bā} \textbf{ham-digar} \\
that united \textsc{sbjv-}become\textsc{.prs-1pl} with each-other \\
\glt `That we become united with each other.'
\z

\ea\label{Persian:ex:16}
Colloquial New Persian \il{Persian (colloquial)}(\citealt[N, 1699]{Izadi2022Persian}) \\
\gll bāz zendegi mi-kon-am \textbf{bā=hāšun} \\
again life \textsc{ind-}do\textsc{.prs-1sg} with\textsc{=3pl} \\
\glt `I live with them again.'
\z

\ea\label{Persian:ex:17}
Colloquial New Persian \il{Persian (colloquial)}(\citealt[P, 1803]{Izadi2022Persian}) \\
\gll hatta bā doxtar-hā rābet=aš xeyli jāleb bud \\
even with girl\textsc{-pl} relation\textsc{=3sg} very good be\textsc{.pst.3sg} \\
\glt `Even his relationship with the girls was good.'
\z

\ea\label{Persian:ex:18}
Colloquial New Persian \il{Persian (colloquial)}(\citealt[K, 1365]{Izadi2022Persian}) \\
\gll ba'd ešāre kard \textbf{bā} \textbf{dast} \\
then refer do\textsc{.pst.3sg} with hand \\
\glt `Then he indicated with his hand.'
\z

\ea\label{Persian:ex:19}
Colloquial New Persian \il{Persian (colloquial)}(\citealt[P, 1819]{Izadi2022Persian}) \\
\gll hasudi na-kon-and \textbf{be} \textbf{ham-digar} \\
envy \textsc{neg-}do\textsc{.prs-3pl} to each-other \\
\glt `They are not jealous of each other.'
\z

\subsubsection{Other roles}

This group consists of tokens which are not classifiable in other groups. Mostly they are adverbs of time and manner, or various unclassified constituent types. Overall, preverbal position is preferred for this heterogenous group, but post-verbal position is also possible (\ref{Persian:ex:20}).

\ea\label{Persian:ex:20}
Colloquial New Persian \il{Persian (colloquial)}(\citealt[P, 0005]{Izadi2022Persian}) \\
\gll xābid-am \textbf{tā} \textbf{sā'at=e} \textbf{do} \\
sleep\textsc{.pst-1sg} till hour\textsc{=ez} two \\
\glt `I slept till 2 o'clock.'
\z

\ea\label{Persian:ex:21}
Colloquial New Persian \il{Persian (colloquial)}(\citealt[P, 1813]{Izadi2022Persian}) \\
\gll hatta {vasat=e} {kelas} {har} {nim} {saat} jāy=aš rā avaz mi-kard \\
Even middle\textsc{=ez} class every half hour place\textsc{=3sg} \textsc{ra} change \textsc{ipfv-}do\textsc{.pst.3sg} \\
\glt `Even in class, he changed his seat every half hour.'
\z

\ea\label{Persian:ex:22}
Colloquial New Persian \il{Persian (colloquial)}(\citealt[O, 1763]{Izadi2022Persian}) \\
\gll guš-hā=yaš bā māsk kār mi-kard \\
ear\textsc{-pl=3sg} with mask work \textsc{ind-}do\textsc{.pst.3sg} \\
\glt `His ears worked (despite being) with the mask.'
\z

\subsubsection{Addressees}

Addressees of speech verbs appear mostly in pre-predicate position (\ref{Persian:ex:23}), with around 11\% post verbal (\ref{Persian:ex:24}):

\ea\label{Persian:ex:23}
Colloquial New Persian \il{Persian (colloquial)}(\citealt[O, 1747]{Izadi2022Persian}) \\
\gll ba'd be āqāh=e goft-am \\
then to man\textsc{=def} tell\textsc{.pst-1sg} \\
\glt `Then I said to the man.'
\z

\ea\label{Persian:ex:24}
Colloquial New Persian \il{Persian (colloquial)}(\citealt[F, 0630]{Izadi2022Persian}) \\
\gll vali na-gu \textbf{be} \textbf{rezā} \\
but \textsc{neg-}tell\textsc{.2sg} to Reza \\
\glt `But, don't tell Reza.'
\z

\subsubsection{Become-complements}


Complements of \textit{become} have been identified as candidates for post-verbal position in Iranian languages (see \citealt{Korn2022Targets} for a discussion). Our data contain 24 tokens, six of which (25\%) occur post-verbally (\ref{Persian:ex:25}), while the majority is pre-verbal (\ref{Persian:ex:26}): 

\ea\label{Persian:ex:25}
Colloquial New Persian \il{Persian (colloquial)}(\citealt[ZC, 3104]{Izadi2022Persian}) \\
\gll dah ruz šod \textbf{davāzdah} \textbf{ruz} \\
ten day became\textsc{.pst.3sg} twelve day \\
\glt `(The promised) ten days become twelve days.'
\z

\ea\label{Persian:ex:26}
Colloquial New Persian \il{Persian (colloquial)}(\citealt[T, 2227]{Izadi2022Persian}) \\
\gll xalāban šod \\
pilot become\textsc{.pst.3sg} \\
\glt `He became a pilot.'
\z

\subsubsection{Benefactive}

Both Recipients and Benefactives have been claimed to pattern similarly to Goals in some languages (\citetv{chapters/1_Haigetal_Intro}). In our data, all Recipients are preverbal, and the majority of Benefactives likewise, though the absolute number of tokens is low (we include under `Benefactives' tokens that are ambiguous between a Recipient and Benefactive reading, coded as ``rec-ben'' in WOWA). Of the 24 tokens of Benefactives, five were post-verbal (\ref{Persian:ex:27}).

\ea\label{Persian:ex:27}
Colloquial New Persian \il{Persian (colloquial)}(\citealt[M, 1537]{Izadi2022Persian}) \\
\gll in yek pitzā āvar-d \textbf{barāy-e} \textbf{mādar-e} \textbf{man} \\
This one pizza bring\textsc{-pst.3sg} for\textsc{-ez} mother\textsc{-ez} I \\
\glt `He brought a pizza for my mother.'
\z

\subsubsection{Summary: Non-direct objects}

With regard to the non-direct objects position in our data, the first and expected observation is that Goals and Goals of caused motion verbs are distinct from all other roles, and appear in post-predicate position near categorically. However, it is also important to note that the second most likely post-posed \isi{argument} after Goals are actually locations (around 20\%). This suggests a general principle of constituents indicating spatial location (either static (loc) or dynamic (Goals) are more likely to be post-predicate than any others. It may also be linked to the feature of +/- humanness; this possibility is explored in \sectref{Persian:4} below. These findings question the validity of lumping Addressees and Recipients with Goals into a meta-\isi{role} ``Target'' (\citealt{asadpour_word_2022}). The Persian\il{Persian (colloquial)} data suggest that Addressees are actually less likely to be post-predicate than, for example, locations, while Recipients are categorically pre-verbal. Thus spoken Persian\il{Persian (colloquial)} provides little support for the assumption of a meta-\isi{role} that would encompass Goals, Recipients, and Addressees. Rather, they reinforce the special status of Goals, in opposition to all other constituent types.

\subsection{Direct objects}\label{Persian:3.2}

The first point about post-predicate direct objects is that they are overall very infrequent. As \tabref{Persian:tab:10} shows only 18 tokens (about 4\%) of direct objects of different kinds appear post-verbally, demonstrating that spoken Persian\il{Persian (colloquial)} is fairly consistently \isi{OV}. The frequency of different kinds of direct objects is provided in \tabref{Persian:tab:10}, further distinguishing \isi{animacy}, \isi{definiteness}, and noun vs. \isi{pronoun}.

\begin{table}
 \begin{tabularx}{\textwidth}{lrYY}
\lsptoprule
Direct objects & Total & VX & Percent \\
\midrule
Nominal, all & 372 & 17 & 4.6\% \\
{\quad}Nominal, human & 63 & 5 & 7.9\% \\
{\quad}Nominal, animate & 5 & 0 & 0\% \\
{\quad}Nominal, inanimate & 285 & 11 & 3.9\% \\
Nominal, indefinite & 204 & 7 & 3.4\% \\
Nominal, definite & 168 & 10 & 6\% \\
Pronominal (1, 2, 3) & 44 & 1 & 2.3\% \\
DO with RA & 240 & 12 & 5\% \\
DO without RA & 197 & 6 & 3\% \\
\lspbottomrule
 \end{tabularx}
 \caption{Post-predicate Direct Objects in HamBam corpus}
 \label{Persian:tab:10}
\end{table}

Although the absolute number of direct objects in post-predicate position is low, the findings suggest that most of the post-predicated direct objects are human, and definite ones appear more freely in post-verbal position compared to indefinite ones. When pronominal, they appear rarely in post-predicate position, and direct objects with \textit{=rā} move more freely to post-verbal position compared to those without \textit{=rā}. The following are examples of direct objects in post-predicate position:

\ea\label{Persian:ex:28}
Colloquial New Persian \il{Persian (colloquial)}(\citealt[F, 0797]{Izadi2022Persian}) \\
\gll in ke māšin zad \textbf{Mehrdad} \textbf{rā} \\
this that car hit\textsc{.pst.3sg} mehrdad \textsc{ra} \\
\glt `When the car hit Mehrdad.'
\z

\ea\label{Persian:ex:29}
Colloquial New Persian \il{Persian (colloquial)}(\citealt[Q, 1919]{Izadi2022Persian}) \\
\gll faqat did-am \textbf{yek} \textbf{daste} \textbf{mu} \\
Just see\textsc{.pst-1sg} one bunch hair \\
\glt `I just saw a bunch of hair.'
\z

\ea\label{Persian:ex:30}
Colloquial New Persian \il{Persian (colloquial)}(\citealt[ZB, 3034]{Izadi2022Persian}) \\
\gll tu=ye sohbat-hā=yāš bargašt az=am porsid \textbf{esm-ā} \textbf{rā} \\
in\textsc{=ez} talk\textsc{-pl=3sg} return\textsc{.pst.3sg} from\textsc{=1sg} ask\textsc{.pst.3sg} name\textsc{-pl} \textsc{ra} \\
\glt `During his talk, he asked the names from me.'
\z

\ea\label{Persian:ex:31}
Colloquial New Persian \il{Persian (colloquial)}(\citealt[ZA, 2939]{Izadi2022Persian}) \\
\gll va motasefāne jav gereft \textbf{man} \textbf{rā} \\
and unfortunately excitement take\textsc{.pst.3sg} I \textsc{ra} \\
\glt `And, unfortunately I was excited.'
\z

\subsection{Subjects}\label{Persian:3.3}

Up to now we have analyzed non-subject roles (see \tabref{Persian:tab:6}), because these are coded in the WOWA data set (\citealt{Izadi2022Persian}). In order to consider subjects, we turned to the full HamBam corpus. We extracted 843 tokens of nominal and pronominal subjects, from which 27 were post-posed (3\%). Some examples of post-predicate subjects are as follows:

\ea\label{Persian:ex:32}
Colloquial New Persian \il{Persian (colloquial)}(\citealt[F, 0657]{Izadi2022Persian}) \\
\gll bord=eš \textbf{āqā=he} \\
take\textsc{.pst=3sg} man\textsc{=def} \\
\glt `The man took it.'
\z

\ea\label{Persian:ex:33}
Colloquial New Persian \il{Persian (colloquial)}(\citealt[F, 0754]{Izadi2022Persian}) \\
\gll hiči um-ad \textbf{doktor=e} \\
anyway come\textsc{.prs-3sg} doctor\textsc{=def} \\
\glt `Anyway, the doctor came.'
\z

The post-predicate subjects were all definite, and can reasonably be classified as afterthoughts: the speaker has already established the reference, which is thus presumably active in the listener's mind, and the \isi{afterthought} simply re-confirms the given status of the referent. Post-posing of subjects is therefore overall very seldom in our data (see \citetv{chapters/3_Skopeteas_Prosody}, on the information status and \isi{prosody} of post-verbal elements in Persian\il{Persian (New)}).

\section{A multivariate analysis of post-verbal syntax in contemporary Persian}\label{Persian:4}

Having introduced and illustrated individual factors identified in \citet{frommer_post-verbal_1981} and the more recent data from WOWA/HamBam (\citealt{Izadi2022Persian}; \citealt{HaigRasekhMahand2022HamBam}), in this section we apply two different methodologies that control for the interactions of individual factors in order to assess their respective impact in driving post-verbal placement in spoken Persian\il{Persian (colloquial)}. For these purposes, we analyse the full data set in \citet{Izadi2022Persian}; \citegen{frommer_post-verbal_1981} actual corpus data are unfortunately not available. 

For the first analysis (\sectref{Persian:4.1}), we run a series of generalized logistic regression models; in a second step (\sectref{Persian:4.2}), we implement methods from the machine learning toolbox, namely a gradient boosting machine (GBM, \citealt{Friedman2001Greedy}) and, for the purposes of illustration, a classification tree. In both approaches, the response variable is positioning (\textit{pre-verbal} vs. \textit{post-verbal}).


\subsection{Logistic regression analysis}\label{Persian:4.1}

We run four generalized logistic regression models predicting post-verbal placement, one for each of the following roles or groupings of roles: 
\begin{enumerate}[label=(\roman*)]
 \item direct objects, 
 \item Goals, 
 \item locations and sources, and 
 \item various other obliques (incl. addressees, recipients, beneficiaries, and instrumentals).
\end{enumerate} 
The preceding sections have already confirmed that \isi{role} is the primary factor in determining post-verbal placement, but also that the relationship differs substantially between roles. It is for this reason that we deem running separate logistic regression models for each \isi{role} (or \isi{role} combination) prudent, as doing so enables us to identify any pertinent associations within individual roles more clearly. The following five predictors are implemented for each \isi{role}: \isi{register} (\textit{public} vs. private), form (\textit{nominal} vs. pronominal), the presence of \isi{flagging} (\textit{none} vs. case marking for objects and prepositions for other roles), humanness (\textit{non-human} vs. human), and phrase \isi{weight} (measured in characters, roughly equivalent to phonological \isi{weight}; for details on quantifying \isi{weight} in the WOWA data, see \citetv{chapters/1_Haigetal_Intro}, \sectref{Intro:ss:4}).

In the model summaries in Tables \ref{Persian:tab:11}-\ref{Persian:tab:14}, the values of $e^\beta$ (the log odds) for each of the predictors assess of how much each of the predictors in the model affects the likelihood of the response variable yielding one or the other outcome. Log odds above 1 indicate higher odds of a post-verbal outcome, while values below 1 do the same for pre-verbal outcomes, both under conditions that all other predictors are held at their respective reference levels (i.e. the ones in italics above, and a theoretical value of 0 for phrase \isi{weight}).\footnote{Note that these values are on a logarithmic scale, i.e. log odds from 1 to 0 for negative outcomes map onto log odds from 1 to infinity for positive outcomes.} For instance, in \tabref{Persian:tab:11}, which summarizes the model outcomes for direct objects, the log odds for the presence of \isi{flagging} (i.e. case-marking) are $e^\beta$ = 2.52, meaning a case-marked \isi{direct object} has 2.52 times higher odds of being post-verbal compared to a \isi{direct object} without case-marking, relative to the base odds of a post-verbal outcome overall (which can be found in the row labelled ``intercept'', here $e^\beta$ = 0.05). However, the model deems this prediction to likely be a matter of chance with a probability of \textit{p} = 0.098, and it should therefore not be taken as evidence for a causal correlation. For the purposes of this analysis, we set the threshold for significance at \textit{p} < 0.05.

\tabref{Persian:tab:11} shows the model results for direct objects, \tabref{Persian:tab:12} for Goals, \tabref{Persian:tab:13} for locations and sources, and \tabref{Persian:tab:14} for other obliques. With the exception of \isi{register} for locations/sources ($e^\beta$ = 8.12 times higher odds of post-verbal positioning for private \isi{register}, with p < 0.05) and \isi{flagging} for other obliques ($e^\beta$ = 0.02 times lower odds for PPs, with \textit{p} < 0.01), none of the predictors in any of the four models pass the threshold for statistical significance. As such, what little variation in positioning there is for direct objects and Goals cannot be adequately explained by \isi{register}, humanness, form, the presence of \isi{flagging}, or phrase \isi{weight}.


\begin{table}
 \begin{tabularx}{\textwidth}{llrrYYr}
\lsptoprule
\textbf{model coefficients} && $e^β$ & $β$ & SE & $z$-val. & $p$-val. \\
\midrule
(intercept) && 0.05 & -3.06 & 0.88 & -3.49 & < 0.001 \\
\isi{register} & = private & 1.00 & 0.00 & 0.66 & 0.01 & 0.996 \\
form & = pronominal & 0.16 & -1.83 & 1.15 & -1.60 & 0.110 \\
\isi{flagging} & = case-marked & 2.52 & 0.92 & 0.56 & 1.65 & 0.098 \\
humanness & = human & 1.56 & 0.45 & 0.70 & 0.64 & 0.521 \\
\isi{weight} & per character & 0.94 & -0.06 & 0.06 & -1.13 & 0.259 \\
\tablevspace
\multicolumn{7}{l}{\textbf{deviance residuals}} \\
\midrule
min. & lower & \multicolumn{2}{l}{median} & \multicolumn{2}{l}{upper} & max. \\
-0.49 & -0.33 & \multicolumn{2}{l}{-0.25} & \multicolumn{2}{l}{-0.21} & 2.87 \\
\tablevspace
\multicolumn{7}{l}{\textbf{model evaluation}} \\
\midrule
\multicolumn{1}{l}{observations} & \multicolumn{3}{l}{434} & \multicolumn{3}{l}{(17 post-verbal)} \\
\multicolumn{1}{l}{null deviance} & \multicolumn{3}{l}{143.48} & \multicolumn{3}{l}{on 433 degrees of freedom} \\
\multicolumn{1}{l}{resid. deviance} & \multicolumn{3}{l}{137.77} & \multicolumn{3}{l}{on 428 degrees of freedom} \\
\lspbottomrule
 \end{tabularx}

 \caption{Logistic regression model for direct objects}
 \label{Persian:tab:11}
\end{table}

\begin{table}
 \begin{tabularx}{\textwidth}{llrrYYr}
\lsptoprule
\textbf{model coefficients} && $e^β$ & $β$ & SE & $z$-val. & $p$-val. \\
\midrule
(intercept) && 7.78 & 2.05 & 0.64 & 3.21 & < 0.01 \\
\isi{register} & = private & 0.84 & -0.170 & 0.59 & -0.29 & 0.772 \\
form & = pronominal & 0.42 & -0.87 & 0.92 & -0.95 & 0.342 \\
\isi{flagging} & = \isi{preposition} & 0.63 & -0.47 & 0.36 & -1.31 & 0.192 \\
humanness & = human & 0.66 & -0.42 & 0.80 & -0.52 & 0.601 \\
\isi{weight} & per character & 1.00 & 0.00 & 0.04 & 0.01 & 0.992 \\
\tablevspace
\multicolumn{7}{l}{\textbf{deviance residuals}} \\
\midrule
min. & lower & \multicolumn{2}{l}{median} & \multicolumn{2}{l}{upper} & max. \\
-2.09 & 0.53 & \multicolumn{2}{l}{0.53} & \multicolumn{2}{l}{0.66} & 1.13 \\
\tablevspace
\multicolumn{7}{l}{\textbf{model evaluation}} \\
\midrule
\multicolumn{1}{l}{observations} & \multicolumn{3}{l}{266} & \multicolumn{3}{l}{(222 post-verbal)} \\
\multicolumn{1}{l}{null deviance} & \multicolumn{3}{l}{238.62} & \multicolumn{3}{l}{on 265 degrees of freedom} \\
\multicolumn{1}{l}{resid. deviance} & \multicolumn{3}{l}{233.10} & \multicolumn{3}{l}{on 260 degrees of freedom} \\
\lspbottomrule
 \end{tabularx}
 \caption{Logistic regression model for Goals}
 \label{Persian:tab:12}
\end{table}

\begin{table}
 \begin{tabularx}{\textwidth}{llrrYYr}
\lsptoprule
\textbf{model coefficients} && $e^β$ & $β$ & SE & $z$-val. & $p$-val. \\
\midrule
(intercept) & & 0.033 & -3.42 & 1.11 & -3.07 & < 0.01 \\
\rowcolor{lightgray}
\isi{register} & = private & 8.12 & 2.09 & 1.05 & 1.99 & < 0.05 \\
form & = pronominal & 1.56 & 0.44 & 1.03 & 0.43 & 0.665 \\
\isi{flagging} & = \isi{preposition} & 1.06 & 0.06 & 0.43 & 0.13 & 0.897 \\
humanness & = human & 1.31 & 0.27 & 0.89 & 0.30 & 0.764 \\
\isi{weight} & per character & 0.98 & -0.02 & 0.04 & -0.47 & 0.640 \\
\tablevspace
\multicolumn{7}{l}{\textbf{deviance residuals}} \\
\midrule
min. & lower & \multicolumn{2}{l}{median} & \multicolumn{2}{l}{upper} & max. \\
-0.93 & -0.66 & \multicolumn{2}{l}{-0.64} & \multicolumn{2}{l}{-0.24} & 2.37 \\
\tablevspace
\multicolumn{7}{l}{\textbf{model evaluation}} \\
\midrule
\multicolumn{1}{l}{observations} & \multicolumn{3}{l}{196} & \multicolumn{3}{l}{(34 post-verbal)} \\
\multicolumn{1}{l}{null deviance} & \multicolumn{3}{l}{180.85} & \multicolumn{3}{l}{on 195 degrees of freedom} \\
\multicolumn{1}{l}{resid. deviance} & \multicolumn{3}{l}{172.77} & \multicolumn{3}{l}{on 190 degrees of freedom} \\
\lspbottomrule
 \end{tabularx}
 \caption{Logistic regression model for locations and sources}
 \label{Persian:tab:13}
\end{table}

\begin{table}
 \begin{tabularx}{\textwidth}{llrrYYr}
\lsptoprule
\textbf{model coefficients} && $e^β$ & $β$ & SE & $z$-val. & $p$-val. \\
\midrule
(intercept) & & 1.53 & 0.43 & 1.74 & 0.24 & 0.807 \\
\isi{register} & = private & 4.08 & 1.41 & 1.10 & 1.28 & 0.199 \\
form & = pronominal & 0.38 & -0.98 & 0.86 & -1.14 & 0.256 \\
\rowcolor{lightgray}
\isi{flagging} & = \isi{preposition} & 0.02 & -3.73 & 1.22 & -3.04 & < 0.01 \\
humanness & = human & 1.19 & 0.18 & 0.91 & 0.19 & 0.848 \\
\isi{weight} & per character & 1.01 & 0.01 & 0.06 & 0.25 & 0.806 \\
\tablevspace
\multicolumn{7}{l}{\textbf{deviance residuals}} \\
\midrule
min. & lower & \multicolumn{2}{l}{median} & \multicolumn{2}{l}{upper} & max. \\
-2.13 & -0.38 & \multicolumn{2}{l}{-0.37} & \multicolumn{2}{l}{-0.19} & 2.52 \\
\tablevspace
\multicolumn{7}{l}{\textbf{model evaluation}} \\
\midrule
\multicolumn{1}{l}{observations} & \multicolumn{3}{l}{131} & \multicolumn{3}{l}{(17 post-verbal)} \\
\multicolumn{1}{l}{null deviance} & \multicolumn{3}{l}{101.12} & \multicolumn{3}{l}{on 130 degrees of freedom} \\
\multicolumn{1}{l}{resid. deviance} & \multicolumn{3}{l}{70.69} & \multicolumn{3}{l}{on 125 degrees of freedom} \\
\lspbottomrule
 \end{tabularx}
 \caption{Generalized linear regression model for other obliques}
 \label{Persian:tab:14}
\end{table}

\subsubsection{Discussion}

These results largely confirm observations of the preceding sections, but reveal additional subtleties. For two of the roles tested here, direct objects and Goals, position relative to the verb is essentially predictable from the nature of the \isi{role} itself. Factors such as phrase \isi{weight}, whose importance has been stressed repeatedly in the literature on \isi{word order} variation, appear to have no consistent effect on position relative to the verb for these two roles in our spoken New Persian\il{Persian (colloquial)} data. Furthermore, there is no effect of public versus private registers of spoken language. 

For other roles, we find slight effects of \isi{flagging}, such that absence of a \isi{preposition} favours post-verbal placement for other obliques, in partial confirmation of one of \citegen{frommer_post-verbal_1981} observations mentioned in \sectref{Persian:2}. However, somewhat surprisingly, this effect is absent with non-\isi{Goal} spatial relations (locations and sources). With locations and sources, we find an effect of \isi{register} (private \isi{register} favours post-verbal placement). The effects of \isi{flagging}, and \isi{register}, have been noted in the literature, but this is the first time that we are able to disentangle the role-specific effects.

One way of looking at our results is to consider the syntax of spoken Persian\il{Persian (colloquial)} as defined in terms of two opposing role-specific rules (direct objects are pre-verbal, Goals are post-verbal), with other roles being pulled in opposing directions, subject to a range of distinct contextual and register-related factors which are only partially captured in the current model.


\subsection{Gradient boosting models}\label{Persian:4.2}

While in the previous section we have examined each \isi{role} independently for its association between positioning and various factors, in this section we answer the more general question of which factors most strongly influence post-verbal placement overall, when all the factors identified here are included in the model. To do this we utilize an iterative classification algorithm, specifically a gradient boosting machine (GBM, \citealt{Friedman2001Greedy}) – a cousin of the random forest algorithm that tends to yield comparatively better results for small and skewed data sets – and, for the purposes of illustration, a single classification tree.

There are a few differences in how we arrange the predictors for these analyses compared to the regression models in the previous section. First, there is no need to maintain separate models for each of the roles, as the model will automatically select \isi{role} as a classifying factor in whatever it deems most appropriate, alongside the other factors of \isi{register}, form, humanness, the presence of \isi{flagging}, and phrase \isi{weight}. Second, since there are an unequal number of data points for each \isi{role}, we calculate case weights for each \isi{role} to prevent more common roles from dominating the model results. In effect this means that data points for the less common roles (Goals, locations/sources, and other obliques) are given proportionally greater \isi{weight} compared to the most common \isi{role} (direct objects).\footnote{For example, since there are 434 data points for direct objects but only 266 for Goals, each \isi{Goal} is treated as if it were 434 / 266 = 1.63 data points instead.} The respective case weights for each \isi{role} are listed in \tabref{Persian:tab:15} further below. Third, we refactor our measure of phrase \isi{weight} from a scalar variable into a categorical one with four levels (≤ 5 characters, 6–8 characters, 9–12 characters, and ≥ 13 characters long) in order to simplify interpretation of model results.\footnote{In a gradient boosting model, the estimation of the relative influence of the predictors in the model is based on how many major splits it produces over the many thousand iterations of tree-building; as a consequence, there is an inherent bias for predictors with many levels compared to, for instance, binary predictors, as the latter can only ever be selected once for a split in each iteration of the tree. Even with this change, since the other predictors in the model are all binary, we still need to account for a bias towards overvaluing phrase \isi{weight} when assessing the results.}

\figref{Persian:fig:1} shows a \isi{binary classification tree},\footnote{Hyperparameters for the classification tree: maximum tree depth = 6; learning rate = 0.001; minimum number of observations in nodes = 10; cross-validation folds = 10. See \tabref{Persian:tab:13} for the case weights applied to different roles.} a visual representation of the output of a recursive partitioning algorithm. Starting from the top, each ``node'' in the tree produces a ``split'' in the data along the values of a predictor. Which values of which predictor are selected by the algorithm at which split in the tree is determined by how cleanly they divide the data (i.e. by reducing the rate of misclassification). For instance, the first split is between roles, differentiating Goals (which are overall 83\% post-verbal) from all other roles (direct objects, locations/sources). Thus the observation from other WOWA data sets of Iranian languages (e.g. \citetv{chapters/4_NourzaeiHaig_Balochi}), that spatial Goals are indeed a special case that need to be distinguished from all other roles, is confirmed in our analysis. The left branch then splits again along roles, and so on. Overall, we can identify three groups of leaf nodes: The two leftmost nodes are almost exclusively pre-verbal, containing, respectively, all direct objects, and locations/sources and other obliques in public speech. The two leaf nodes in the center contain data points that are around 20\% post-verbal; they include locations/sources and other obliques in private speech that are either flagged with a \isi{preposition}, or not flagged but non-human. Those that meet the latter set of criteria but are human instead are conversely 80\% post-verbal; however, there are only 10 such cases in the corpus. The rightmost leaf node, as already noted above, contains all Goals and is 83\% post-verbal. It is important to note that while single tree models are nicely illustrative and (largely) intuitive to interpret, their predictions are not robust. Small changes to the data or the hyperparameters of the model can effect substantial differences in the structure of the resulting tree.

\begin{figure}
 \includegraphics[width=\textwidth]{figures/RMFigure1Persian.png}
 \caption{Binary classification tree.}
 \label{Persian:fig:1}
\end{figure}

These shortcomings are addressed by so-called ensemble models. Unlike classification tree algorithms, gradient boosting machines (and other methods) do not fit a single tree to the data once but rather perform a self-improving fitting process that learns as it goes, usually over thousands of iterations of trees. This greatly improves accuracy and allows each predictor to appear in a variety of contexts, thereby more thoroughly unraveling the often highly complex effects of the predictors on the response \citep[336]{Strobl2009recursive}. The downside is that this makes the model results more difficult to interpret in their entirety, as there is no single ``final'' tree generated by the model. 

That said, there are nevertheless many ways of summarizing their output that offer critical insight into the relationships between the model parameters. One such way is by looking at the relative importance of each predictor, which is determined by how often a particular predictor was selected for a ``split'' across the many thousand iterations of trees generated by the model. The results of this analysis can be found in \figref{Persian:fig:2}, which provides an answer to the question of the relative importance of different factors in determining whether a constituent is placed pre- or post-verbally. Unsurprisingly, the semantic \isi{role} of the constituent is given predominant importance, a consequence of the practically diametrically opposed profiles of Goals (chiefly post-verbal) and all other roles (largely pre-verbal, albeit to different degrees). Likewise unsurprising is the relative lack of importance of the other predictors in the model, a reflection of the results of the regression models in the previous section. In the grand scheme of things, the effect of \isi{register} for placement of locations/sources that we had noted in the previous section fails to materialize as particularly influential, and the association of \isi{flagging} for other \isi{oblique} roles is only marginally more so. How much of the importance of phrase \isi{weight} is due to biases in the structure of the predictors (see footnote above) is difficult to say, but given the lack of any sort of association in the regression models, it is unlikely to play much of a \isi{role}.

\begin{figure}
 \includegraphics[width=\textwidth]{figures/RMFigure2Persian.png}
 \caption{Relative importance of the predictors in the gradient boosting model.}
 \label{Persian:fig:2}
\end{figure}


\begin{table}
 \fittable{\begin{tabular}{lll}
\lsptoprule
\textbf{response} & position & (pre-verbal, post-verbal) \\
\textbf{predictors} & \isi{role} & (\isi{direct object}, \isi{Goal}, location/sources, other \isi{oblique}) \\
 & \isi{register} & (public, private) \\
 & form & (nominal, pronominal) \\
 & \isi{flagging} & (none, flagged) \\
 & humanness & (non-human, human) \\
 & \isi{weight} & (≤ 5, 6–8, 9–12, ≥ 13 characters) \\
\midrule
\multicolumn{2}{l}{\textbf{error distribution function}} & Bernoulli (for binary response variables) \\
\multicolumn{2}{l}{\textbf{observations}} & 1027 (290 post-verbal) \\
\midrule
\multicolumn{2}{l}{\textbf{model hyperparameters}} \\
number of trees & 10000 \\
learning rate & 0.001 \\
interaction depth & 7 \\
min. obs. in nodes & 25 \\
cross-validation folds & 10 \\
\midrule
\multicolumn{3}{l}{\textbf{case weights} (balancing out differences in the number of observations across roles)} \\
direct objects & × 1.00 & (434 obs.) \\
Goals & × 1.63 & (266 obs.) \\
locations/sources & × 2.21 & (196 obs.) \\
other obliques & × 3.31 & (131 obs.) \\
\lspbottomrule
 \end{tabular}}
 \caption{Parameters for the gradient boosting model}
 \label{Persian:tab:15}
\end{table}

In sum, both models unsurprisingly confirm the overwhelming impact of \isi{role} as the single most important predictive factor. Specifically, in contemporary spoken Persian\il{Persian (colloquial)}, noun phrases bearing the semantic \isi{role} of Goals of motion or caused motion are placed post-verbally with a very high probability, regardless of other factors. The other factors examined here play only a marginal \isi{role}, with next most important predictor in this model being \isi{weight}. The remaining predictors including humanness, form, and \isi{register} turn out to have little impact overall, or their influence on \isi{word order} are at best confined to specific contexts.

\section{Comparing \citet{frommer_post-verbal_1981} and the HamBam data: The role of register}\label{Persian:5}

Having presented \citegen{frommer_post-verbal_1981} data and our data from the HamBam corpus (\citealt{HaigRasekhMahand2022HamBam}), we now undertake a more detailed comparison between the two, and address the question of \isi{register} differentiation. First of all, we need to specify the nature of the \isi{register} levels in the two data sets, and address the issue of comparability. As noted above in \sectref{Persian:2}, \citegen{frommer_post-verbal_1981} data includes a mix of spoken and written corpora, and differing grades of formality within each. For \citegen{frommer_post-verbal_1981} spoken data, two levels of formality were included: casual conversational data in a domestic setting, and spoken language from the radio broadcasts of Radio Payām. We refer to these two registers as ``private'' and ``public'' respectively. The radio broadcasts included a mix of music, news ``read in formal Persian\il{Persian (New)}'', pre-recorded commercials, but also spontaneous ``banter between co-hosts, and often live telephone conversations between the hosts and the listeners'' (\citealt{frommer_post-verbal_1981}: 74). Frommer included only those sections of the recordings which he considered ``to be the most relaxed and spontaneous, and bore the phonological and morphological hallmarks of colloquial style''. The main difference between private and public spoken registers, as defined here, is that the former is exclusively between familiar persons in a private setting, while the latter involves a mix of familiar and unfamiliar interlocuters, produced with the knowledge that the language is publicly broadcast. Both, however, involve spoken, and largely spontaneous language.

We apply a similar distinction between private and public to the HamBam corpus. Recordings are characterized as ``private'' when they stem from interactions in private settings, between familiar interlocutors (kin or close friends). Recordings characterized as ``public'' are from publicly available sources such as radio and podcasts, often involving more academic and abstract subject matter, while still remaining quite spontaneous. Unfortunately, the amount of ``public'' \isi{register} in HamBam texts is not very high, so this aspect of the comparison is tentative. \tabref{Persian:tab:16} provides an overview of the HamBam data that feed into the comparison.


\begin{table}
 \fittable{\begin{tabular}{l@{}rrrrr}
\lsptoprule
& total & public & \% Po & private & \% Po \\
\midrule
Total tokens & 3219 & 574 & 17.8\% & 2645 & 82.2\% \\
Number of analyzed tokens & 1624 & 273 & 16.8\% & 1351 & 83.2\% \\
Number of non-classified tokens & 1595 & 301 & 18.9\% & 1294 & 81.1\% \\
Rate of post-predicate elements (all roles) & 413 & 51 & 18.7\% & 362 & 26.8\% \\
\lspbottomrule
 \end{tabular}}
 \caption{Frequency of post-predicate elements in private and public genres of HamBam corpus}
 \label{Persian:tab:16}
\end{table}

Note that the basic unit used for quantitative analysis in \citet{frommer_post-verbal_1981} was the clause, whereas in HamBam, the basic unit is a referential, non-subject constituent. This makes a global comparison of rates of post-verbality difficult: Frommer calculates the proportion of clauses containing any post-verbal material among the totality of clauses in the corpus; our measure would be the proportion of all relevant constituents in the corpus that occur post-verbally. In fact, our metric is likely to make the overall value higher than \citegen{frommer_post-verbal_1981} (because we do not count, for example, clauses lacking a relevant non-subject constituent). 

\largerpage[-3]
A role-specific comparison is more reliable, at least for those roles which are defined in a comparable way in both studies. \tabref{Persian:tab:17} summarizes the findings for direct objects and Goals, in both corpora, distinguishing the two spoken registers public vs. private.\footnote{Data sources in \citet{frommer_post-verbal_1981}: Direct objects all: p.143, Table 11; with and without =rā: p. 143, Table 12; Goals (all): p. 130--131, Table 5; other PP (not Goals): p.160, Table 21. For ``other PP (not Goals)'' in HamBam we included all tokens flagged with <prep> in \citet{Izadi2022Persian}, excluding Goals and direct objects. This may not be fully identical with \citegen{frommer_post-verbal_1981} category of other PP's, so at this point, the comparison between the two data sets should be treated with caution.} Post-verbal frequencies are grey-shaded.

\begin{table}
 \begin{tabularx}{\textwidth}{X rrrr@{\qquad}rrrr}
\lsptoprule
& \multicolumn{4}{c}{\citet{frommer_post-verbal_1981}} & \multicolumn{4}{c}{HamBam (2022)} \\
\cmidrule(lr){2-5}\cmidrule(lr){6-9}
& \multicolumn{2}{c}{public} & \multicolumn{2}{c}{private} & \multicolumn{2}{r}{public~~~~} & \multicolumn{2}{c}{private} \\
\cmidrule(lr){2-3}\cmidrule(lr){4-5}\cmidrule(lr){6-7}\cmidrule(lr){8-9}
& N & \%Po & N & \%Po & N & \%Po  & N & \%Po  \\
\midrule
Direct objects (all) & 337 & 0.9\cellcolor{lightgray} & 646 & 4.2\cellcolor{lightgray} & 80 & 3.8 \cellcolor{lightgray}& 356 & 4.9\cellcolor{lightgray} \\
Direct objects (+\textit{rā}) & 178 & 1.1\cellcolor{lightgray} & 224 & 9.4\cellcolor{lightgray} & 47 & 4\cellcolor{lightgray} & 193 & 5.1\cellcolor{lightgray} \\
Direct objects (bare) & 159 & 0.6\cellcolor{lightgray} & 422 & 1.4\cellcolor{lightgray} & 33 & 3\cellcolor{lightgray} & 163 & 2.5\cellcolor{lightgray} \\
Goals (all) & 38 & 39.5\cellcolor{lightgray} & 203 & 82.8\cellcolor{lightgray} & 27 & 85\cellcolor{lightgray} & 239 & 83\cellcolor{lightgray} \\
Goals (PP) & 24 & 16.7\cellcolor{lightgray} & 51 & 73.9\cellcolor{lightgray} & 9 & 88.9\cellcolor{lightgray} & 94 & 78.7\cellcolor{lightgray} \\
Goals (bare) & 14 & 78.6\cellcolor{lightgray} & 134 & 87.3\cellcolor{lightgray} & 18 & 83.3\cellcolor{lightgray} & 143 & 86.7\cellcolor{lightgray} \\
Other PP (not Goals) & 622 & 13\cellcolor{lightgray} & 519 & 17.9\cellcolor{lightgray} & 98 & 14.3\cellcolor{lightgray} & 369 & 20\cellcolor{lightgray} \\
\lspbottomrule
 \end{tabularx}
 \caption{Comparing post-verbal frequencies for selected roles across two genres and two time periods}
 \label{Persian:tab:17}
\end{table}

Considering first the private \isi{register}, it is evident that little has changed between the late 1970's and 2020's: The frequency of post-predicate elements in the selected roles has remained more or less the same. There is nevertheless one important difference between the late 1970's and the 2020's: In \citegen{frommer_post-verbal_1981} data, private and public spoken language differ, for all categories, and in the same direction (frequencies of post-verbal placement increase between public and private), most notably for Goals. In the recent HamBam data, on the other hand, the differences between public and private are negligible, and even go in the unexpected direction for some roles (for example in public speech, Goals are overall slightly more frequently postposed than in private speech). In other words, in today's Persian\il{Persian (New)} there is almost no difference between private and public speech regarding the analyzed parameters. \figref{Persian:fig:3} visualizes the difference between the two time periods with regard to \isi{register}.

\begin{figure}
 \includegraphics[width=\textwidth]{figures/RMFigure3Persian.png}
 \caption{Register differences per role in spoken Persian 1981 and the 2020's respectively}
 \label{Persian:fig:3}
\end{figure}

One way of interpreting these findings is in terms of ``levelling up'' over the the last 40 years. Apparently, 40 years ago the speech of the private domain was significantly different from the public domain with regard to post-verbal syntax. But today, the difference has largely disappeared, and public speech is now essentially identical to that of the old private domain. While levelling up is generally discussed in the context of dialects (regional variants, or variation based on socio-econominc status, \citealt[200]{Dillard1972BlackE}, \citealt[98--99]{Trudgill1986Dialects}), the relevant concept here would be ``levelling up across registers'', where ``\isi{register}'' refers to ``conventionalized and recurrent'' \isi{intra-speaker variation} in the way speakers adapt their utterances according the context, and in particular on the perceived degree of formality (\citealt{pescumaetal2023situating}: 2). With respect to Persian\il{Persian (New)}, our data suggest that at the time of \citegen{frommer_post-verbal_1981} research in the late 1970's, speakers of Persian\il{Persian (New)} adapted their speech along the parameter of post-verbal placement of Goals, distinguishing between public and private domains. In the 2020's, however, it appears that the norms that were previously operative for the private domain have since spread to encompass spoken language in the public domain, i.e. that speakers no longer feel the necessity to adapt their speech in this regard (though other features of speech, such as lexical choice, \isi{phonology} etc. continue to demarcate public and private speech situations).

These findings, though still tentative, open up a range of novel perspectives for understanding the dynamics of language change, particularly syntactic change. Thus while our data suggest that over the last 40 years, nothing has changed in the extreme values defined by the least formal \isi{register}, by investigating different registers we are able to demonstrate that the distribution of this speech variant across different contexts has changed, namely in the form of levelling from below. This aligns with \citeauthor{labov2001principles}'s (\citeyear{labov2001principles}: 437) observations that change in features below the level of consciousness (which we believe holds for the syntactic phenomena under investigation here) initially ``develop in spontaneous speech at the most informal level.'' 

Returning to the question at the outset of this section then, we can tentatively conclude that private, informal speech has not changed noticeably in the last 40 years. For this \isi{register}, our findings confirm \citegen{frommer_post-verbal_1981} findings, suggesting a rather stable linguistic variable. Where we have identified a difference is that 40 years ago, there existed a more formal kind of ``public'' spoken language, distinguished from private speech by lower levels of post-verbal Goal\is{Goal!post-verbal}s. In our more recent data, this \isi{register} appears to have merged with that of private speech. 


\section{Summary}\label{Persian:6}

Building on the pioneering work of \citet{frommer_post-verbal_1981}, this chapter is the most comprehensive and accountable analysis of post-predicate elements in spoken Persian\il{Persian (colloquial)} currently available. We base our findings on a purpose-built and fully-accessible digital corpus of spoken colloquial Persian\il{Persian (colloquial)}, the Hamedan-Bamberg Corpus of Contemporary Spoken Persian\il{Persian (colloquial)} (HamBam, \citealt{HaigRasekhMahand2022HamBam}), which we have adapted to the WOWA coding conventions (\citealt{Izadi2022Persian}). While the corpus size is modest in comparison to the written language corpora that underpin most contemporary corpus-based research on Persian\il{Persian (colloquial)} (\citealt{Faghirietal2018Canonical,FaghiriSamvelian2020SOV}, among others), our data identify systematic differences between spoken and written Persian\il{Persian (New)} syntax; we conclude that generalizations regarding Persian\il{Persian (New)} syntax \textit{per se} need to be tested for both modes of language production. 

In the context of the present volume, we note that spoken Persian\il{Persian (colloquial)} exhibits traits that are shared with the spoken Iranian languages of Western Asia (e.g. Balochi\il{Balochi}, see \citetv{chapters/4_NourzaeiHaig_Balochi}; Gorani and Kurdish, see \citetv{chapters/9_Mohammadirad_Gorani}), most notably the strong tendency to place Goals post-verbally. In this sense, the spoken Persian\il{Persian (colloquial)} investigated here is more typical for Western Iranian languages than the standard written variety of Persian\il{Persian (New)}, which is quite strictly verb-final (for early classical Persian\il{Persian (Early New)}, see \citetv{chapters/8_Parizadeh_ENP}). We also consider whether spoken Persian\il{Persian (colloquial)} has undergone any changes over the last 40--50 years, through a comparison of our findings with those of \citet{frommer_post-verbal_1981}. As mentioned, there are certain difficulties with comparing the metrics used in both studies; furthermore, \citegen{frommer_post-verbal_1981} original data are not available to us for verification. But for those measures which can be reliably compared, we find no difference in frequencies of post-verbal Goal\is{Goal!post-verbal}s or direct objects, in the least formal sections of the samples at least, which leads us to conclude that the >80\% levels of post-verbal Goal\is{Goal!post-verbal}s is a fairly stable variable in spoken Persian\il{Persian (colloquial)} (it is also the value identified in another corpus of spoken Persian\il{Persian (colloquial)}, \citealt{Adibfar2019Persian}). We do, however, find a difference in the way that post-verbal placement of Goals is mediated according to \isi{register}. In the older sample, the public \isi{register} exhibits lower levels of post-verbal Goal\is{Goal!post-verbal}s than the private \isi{register}. This finding is consistent with the general consensus that standard written Persian\il{Persian (New)}, the pole of maximal formality, is a verb-final language, i.e. with negligible rates of post-verbal elements. From this perspective, the more formal registers of spoken Persian\il{Persian (colloquial)} would be expected to be nearer to the extreme level of formality found in formal written Persian\il{Persian (New)}, in keeping with \citegen{frommer_post-verbal_1981} conclusion regarding formality effects on \isi{word order}. In our contemporary spoken data, however, we found no significant effects of \isi{register}. This suggests that today's spoken language has extended what was the informal, private \isi{register}, to more public settings. We assume, however, that the written language remains overwhelmingly verb final, though we have not investigated this systematically here. Spoken language may thus be prone to relatively rapid ``change'', but it is not the structures themselves that change; rather, it is the social indexing attached to the already available structures.

In \sectref{Persian:4} we conducted two different kinds of multi-variate analysis on the HamBam data in order to provide a statistically more rigorous answer to the question of what drives post-verbal placement in spoken Persian\il{Persian (colloquial)}. The results confirm the effect of \isi{role}, most specifically Goals versus the rest, which swamps most other factors. The logistic regression model identified effects of \isi{register} and \isi{flagging}, but only for specific roles; in the boosted decision trees, these effects turn out as marginal. Thus overall we find little evidence for a consistent effect of \isi{weight}, humanness, or form. However, we note that our analysis is relatively coarse-grained, and a more detailed examination of individual contexts may well uncover additional predictors that were missed in our model.




\section*{Abbreviations}
\begin{tabularx}{.45\textwidth}{@{}lQ@{}}
1 & first person \\
2 & second person \\
3 & third person \\%15
\textsc{abl} & {ablative} \\
\textsc{addr} & {addressee} \\
\textsc{ben} & beneficiary \\
\end{tabularx}
\begin{tabularx}{.45\textwidth}{@{}lQ@{}}
\textsc{cop} & {copula} \\
\textsc{def} & definite \\
\textsc{deic} & deictic \\
\textsc{dem} & demonstrative \\
\textsc{drct} & directional \\
\textsc{exist} & existential \\
\end{tabularx}

\begin{tabularx}{.45\textwidth}{@{}lQ@{}}
\textsc{ez} & ezafe \\
\textsc{fut} & future \\
\textsc{ind} & indicative \\
\textsc{indef} & indefinite \\
\textsc{instr} & instrumental \\
\textsc{loc} & {locative} \\
N & total number of tokens \\
\textsc{neg} & negator \\
\textsc{pl} & plural \\
\textsc{po} & post-posed \\
\textsc{pp} & prepositional phrase \\
\end{tabularx}%16
\begin{tabularx}{.45\textwidth}{@{}lQ@{}}
\textsc{prs} & present \\
\textsc{pst} & past \\
\textsc{ptcpl} & participle \\
\textsc{ra} & object-marking {clitic} \textit{=rā}\\
\textsc{rel} & relative \\
\textsc{sbjv} & subjunctive \\
\textsc{sg} & singular \\
\textsc{V} & {verb} \\
\textsc{VX} & {verb} > any constituant \\
WOWA & = \citet{Haig.Stilo.Dogan.Schiborr2022} \\
\\
\end{tabularx}

\sloppy
\printbibliography[heading=subbibliography,notkeyword=this]
\cleardoublepage
\end{document}
