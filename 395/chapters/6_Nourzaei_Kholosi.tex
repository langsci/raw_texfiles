\documentclass[output=paper,colorlinks,citecolor=brown]{langscibook}
\ChapterDOI{10.5281/zenodo.14266341}
\author{Maryam Nourzaei\affiliation{Uppsala University}}
\title{Kholosi}
\abstract{This chapter studies the word order configuration of Kholosi\il{Indic!Kholosi}, an Indo-Aryan outlier spoken in the Southwest of Iran. Kholosi\il{Indic!Kholosi} shows considerable influence of Persian\il{Persian} and other western Iranian languages, including phonology and morphology, but also in syntax: It exhibits regular OV, but predominantly post-posed goals in a manner that matches that of the neighbouring Iranian languages. Kholosi\il{Indic!Kholosi} appears to have converged more closely with Iranian than other Indo-Aryan outliers in Iran have (e.g. Jadgali\il{Indic!Jadgali}). The Kholosi\il{Indic!Kholosi} case illustrates how languages that have shifted comparatively recently into the Western Asian Transition Zone can adapt their syntax to match the profile of neighbouring languages. }

%move the following commands to the ``local...'' files of the master project when integrating this chapter
% \usepackage{tabularx}
% \usepackage{langsci-optional}
% \usepackage{langsci-gb4e}
% \usepackage{enumitem}
% \addbibresource{localbibliography.bib}
% % \bibliography{localbibliography}
% \newcommand{\orcid}[1]{}
% \let\eachwordone=\itshape

\IfFileExists{../localcommands.tex}{
 \addbibresource{../collection_tmp.bib}
 \addbibresource{../localbibliography.bib}
 % add all extra packages you need to load to this file

\usepackage{tabularx,multicol}
\usepackage{url}
\urlstyle{same}

\usepackage{listings}
\lstset{basicstyle=\ttfamily,tabsize=2,breaklines=true}

\usepackage{langsci-basic}
\usepackage{langsci-optional}
\usepackage{langsci-lgr}
\usepackage{langsci-osl}
% \usepackage{./langsci/styles/langsci-lgr}
% \usepackage{./langsci/styles/langsci-osl}
% \usepackage{langsci-gb4e}

\usepackage{tikz}
\usetikzlibrary{patterns,calc}
\pgfdeclarepatternformonly{south east lines}{\pgfqpoint{-0pt}{-0pt}}{\pgfqpoint{3pt}{3pt}}{\pgfqpoint{3pt}{3pt}}{
    \pgfsetlinewidth{0.6pt}
    \pgfpathmoveto{\pgfqpoint{0pt}{3pt}}
    \pgfpathlineto{\pgfqpoint{3pt}{0pt}}
    \pgfpathmoveto{\pgfqpoint{.2pt}{-.2pt}}
    \pgfpathlineto{\pgfqpoint{-.2pt}{.2pt}}
    \pgfpathmoveto{\pgfqpoint{3.2pt}{2.8pt}}
    \pgfpathlineto{\pgfqpoint{2.8pt}{3.2pt}}
    \pgfusepath{stroke}}
    
\usepackage{stmaryrd}
\usepackage{wasysym}
\usepackage{multirow}
\usepackage{caption}
\usepackage{subcaption}
\usepackage{mathrsfs}
\usepackage{qtree}

\usepackage{linguex}


 %pminos do not split footnotes
% \interfootnotelinepenalty=10000 %Footnote in Laporte chapters has to be split SN


%\DeclareIndexNameFormat{default}{%
%\nameparts{#1}%
%\usebibmacro{index:name}%
%{\index[names]}%
%{\namepartfamily}%
%{\namepartgiveni}%
% {}% L1
% {}% L2
%{\namepartprefix}% generates spurious space L3
%{\namepartsuffix}% generates spurious space L4
%}

%  {\DeclareIndexNameFormat{default}{%
%     \usebibmacro{index:name}{\index[names]}{#1}{#3}{#5}{#7}}}

%\DeclareIndexNameFormat{default}{%
%  \usebibmacro{index:name}{\sindex[nom]}{#1}{#3}{#5}{#7}}

%\DeclareIndexNameFormat{default}{%
%  \usebibmacro{index:name}{\sindex[person]}{#1}{#3}{#5}{#7}}
%\DeclareIndexNameFormat{default}{%
%\nameparts{#1} \usebibmacro{index:name}{\sindex[person]]}{\namepartfamily}{‌​\namepartgiven}{\nam‌​epartprefix}{\namepa‌​rtsuffix}}

%\newcommand{\smiley}{:)}

%\renewbibmacro*{index:name}[5]{%
%\usebibmacro{index:entry}{#1}%
%{\iffieldundef{usera}{}{\thefield{usera}\actualoperator}\mkbibindexname{#2}{#3}{#4}{#5}}}

% \newcommand{\noop}[1]{}

%remove for final
%\overfullrule=1mm

\newcommand{\tobi}[2]}}
\renewcommand{\S}[1]{\tobi{#1}{\textsc{*}}}

% this volume references
% puts: [this volume]
% already defined: \citetv
%\newcommand{\citepv}[1]{(\citeauthor{#1} \citeyear*{#1} [this volume])}
\newcommand{\citealtv}[1]{\citeauthor{#1} \citeyear*{#1} [this volume]}

%parentheses around example number
\newcommand{\pref}[1]{(\ref{#1})}

% in-text examples

\newcommand{\lnex}[1]{\textit{#1}} %target lang word
\newcommand{\lnlit}[1]{(lit.: `#1')} %literal reading
\newcommand{\lnlat}[1]{(#1)} % latinization
\newcommand{\lntrans}[1]{`#1'} %translation
\newcommand{\lnexl}[2]%
{\lnex{#1}{} \lnlat{#2}} % ex with latinization
\newcommand{\lnexlat}[3]{\lnex{#1}{} \lnlat{#2}{} \lntrans{#3}} % ex with latinization and tranl.

%ch01
\newcommand{\co}[1]{\mbox{\textbf{#1}}}

%ch09

\newcommand{\cyrbulg}[1]{\begin{otherlanguage*}{bulgarian}#1\end{otherlanguage*}}


%ch10
\newcommand{\nlp}{{\small NLP}}
\newcommand{\mwe}{{\small MWE}}
\newcommand{\rae}{{\small RAE}}
\newcommand{\lvc}{{\small LVC}}
\newcommand{\pos}{{\small P}o{\small S}}
%\newcommand{\todo}[1]{ \textcolor{red}{#1} }

%\renewcommand{\labelenumi}{\theenumi}
%\ainamefmt{{vv}{ll}{, ff}{, jj}} % fullname

\newcommand{\biberror}[1]{{\color{red}#1}}

\newcommand{\osenovaitem}{--~}
 %% hyphenation points for line breaks
%% Normally, automatic hyphenation in LaTeX is very good
%% If a word is mis-hyphenated, add it to this file
%%
%% add information to TeX file before \begin{document} with:
%% %% hyphenation points for line breaks
%% Normally, automatic hyphenation in LaTeX is very good
%% If a word is mis-hyphenated, add it to this file
%%
%% add information to TeX file before \begin{document} with:
%% %% hyphenation points for line breaks
%% Normally, automatic hyphenation in LaTeX is very good
%% If a word is mis-hyphenated, add it to this file
%%
%% add information to TeX file before \begin{document} with:
%% \include{localhyphenation}
\hyphenation{
    Beck-man
    Ngu-yen
    back-chan-nel
    back-chan-nels
    mo-not-o-nous
    ste-reo-typ-i-cal
}

\hyphenation{
    Beck-man
    Ngu-yen
    back-chan-nel
    back-chan-nels
    mo-not-o-nous
    ste-reo-typ-i-cal
}

\hyphenation{
    Beck-man
    Ngu-yen
    back-chan-nel
    back-chan-nels
    mo-not-o-nous
    ste-reo-typ-i-cal
}

%  \boolfalse{bookcompile}
%  \togglepaper[5]%%chapternumber
}{}

\begin{document}
\maketitle\label{WOWA:ch:6}

\section{Introduction}\label{Kholosi:ss:1}

Although Kholosi\il{Indic!Kholosi} is spoken in Iran, it belongs to the Indo-Aryan branch of Indo-European, and is thus separated from the core location of its closest relatives by hundreds of kilometres. This chapter provides the first available analysis of \isi{word order} in Kholosi\il{Indic!Kholosi}, and explores the respective roles of inheritance and contact in shaping its syntax. Kholosi\il{Indic!Kholosi} is predominantly verb final, but exhibits mixed adpositional typology, with age-graded variability: younger speakers mix Iranian prepositions and Indo-Aryan postpositions, while the older speakers produce only postpositions. Unlike its Indo-Aryan relatives \citep{dahl_ergativity_2016}, most of which exhibit some form of split ergativity, Kholosi\il{Indic!Kholosi} shows \isi{accusative} alignment throughout, possibly through the influence of Persian\il{Persian} and other contact languages lacking ergativity. The data for the present study are extracted from two tales which are available in the WOWA corpus (see full details in \citealt{nourzaei_documentation_2022}), and were supplemented by texts from my on-going analysis.

Kholosi\il{Indic!Kholosi} is spoken mainly in two villages, Kholos and Gotav, in Hormozgan province, Iran (see \figref{Kholosi:fig:1}). Additional small speech communities are found in other regions such as Bastak, Jenah, Bandar Abbas, Bandar Lenge and Bandar Khamir. In addition, Kholosi\il{Indic!Kholosi} is also spoken by Kholosi\il{Indic!Kholosi} people living outside of Iran for example in Bahrein, Doha, Dubai, and Abu Dhabi. The total population of the Kholosi\il{Indic!Kholosi} speakers is uncertain. However, based on recent field studies by myself, the number of speakers is estimated as at least 2,300.

\begin{figure}
 \centering
 \includegraphics{figures/Nourzaei_Kholosi_fig1.png}
 \caption{Location of Kholosi Speakers (from \citealt{nourzaei_orality_2023})}
 \label{Kholosi:fig:1}
\end{figure}


There is another Indo-Aryan language spoken in Iran, Jadgali\il{Indic!Jadgali}, located mainly in Chabahar, Dashtiyari, and Polan regions in Sistan and Balochistan province \citep{barjasteh_delforooz_sociolinguistic_2008}. The distance between the Kholosi\il{Indic!Kholosi} community to the Jadgal communities is more than 600 km. The Kholosi\il{Indic!Kholosi} and Jadgali speakers have not traditionally had any contact and only became acquainted with each other recently (see \citealt{nourzaei_orality_2023}). The location of the Jadgali\il{Indic!Jadgali} speech communities is indicated in \figref{Kholosi:fig:2}.

The question of how the Kholosi\il{Indic!Kholosi} reached their current location, and what their historical connection to the Jadgali\il{Indic!Jadgali} speech community may have been, remains controversial. There are four different accounts: (a) they directly moved to their present homeland from India (see \citealt{anonby_shipwrecked_2016}); (b) they moved here via Dashtiyari in Balochistan; (c) they migrated to their present homeland from Sindh during the Safavid dynasty; or (d) they migrated from the Makoran coast as a separate group, distinct from the Jadgal. 

\begin{figure}
 \includegraphics[width=\textwidth]{figures/Nourzaei_Kholosi_fig2.png}
 \caption{Location of Kholosi and Jadgali speakers (taken from \citealt{nourzaei_orality_2023})}
 \label{Kholosi:fig:2}
\end{figure}

The area where Kholosi\il{Indic!Kholosi} is spoken is linguistically highly diverse, and Kholosi\il{Indic!Kholosi} speakers are multilingual. Contact languages include two different language families: Indo-European (Iranian) and Semitic. Kholosi\il{Indic!Kholosi} speakers in Iran are in direct contact with Bastaki\il{Bastaki}, Lārestāni\il{Lārestāni} (both West Iranian) as local vernaculars, and Persian\il{Persian} as the official language of Iran via TV, Radio, and education, and Arabic\il{Arabic} as a liturgical language via the Koran and Islamic literature. Likewise, Kholosi\il{Indic!Kholosi} speakers outside of Iran are in contact with Arabic\il{Arabic} as an official language (see also \citealt{anonby_shipwrecked_2016}: 2). In addition, Kholosi\il{Indic!Kholosi} would have been in contact with other languages during the migration of the Kholosi\il{Indic!Kholosi} speakers to their current location, but the identity of these languages has not been established with any certainty. 

Kholosi\il{Indic!Kholosi} is used as the first language in the Kholosi\il{Indic!Kholosi} community. Parents speak Kholosi\il{Indic!Kholosi} with their children to a large extent. In Gotav village, however, there is a tendency for the Kholosi\il{Indic!Kholosi} residents to use Bastaki\il{Bastaki} among themselves and with their children. In Kholos, many parents speak with their children in Persian\il{Persian} to prepare them for school. Outside these two villages, e.g., Buchir, it is observed that Kholosi\il{Indic!Kholosi} speakers have lost their language and have switched to Bastaki\il{Bastaki}/Lārestāni\il{Lārestāni} (known as Achomi\il{Achomi, see Bastaki/Lārestāni}, Khodemuni\il{Khodemuni, see Bastaki/Lārestāni}), which are the vernacular languages in these regions. In Kholosi\il{Indic!Kholosi} families with an exogamic marriage pattern, the Kholosi\il{Indic!Kholosi} parents do not use Kholosi\il{Indic!Kholosi} as their first language. Instead, the common language between parents and children at home is Bastaki\il{Bastaki}, Lārestāni\il{Lārestāni}, or even Persian\il{Persian}. Among Kholosi\il{Indic!Kholosi} speakers living abroad, a tendency to use Arabic\il{Arabic} at home is reported. Note that some of the Kholosi\il{Indic!Kholosi} speakers switched to Arabic\il{Arabic}.

Kholosi\il{Indic!Kholosi} lacks  an established writing system, and is generally restricted to oral domains. For example, speakers use standard Persian\il{Persian} when texting each other via cell phone or writing letters. There are also no TV or radio programmes in Kholosi\il{Indic!Kholosi}. The language of teaching is Persian\il{Persian}; however, if the teacher is a Kholosi\il{Indic!Kholosi} speaker, Kholosi\il{Indic!Kholosi} can be used in the classroom. For religious instruction, Arabic\il{Arabic} and Persian\il{Persian} are used. In the past, Kholosi\il{Indic!Kholosi} served as a language of religious instruction in Islamic schools; however today, only Persian\il{Persian} is used. Their religious leaders (Mullah) use Persian\il{Persian} for their sermons after Friday and Ids (festival) prayers.

\subsection{Previous studies}\label{Kholosi:ss:1.1}

Kholosi\il{Indic!Kholosi} is an under-documented language (see \citealt{nourzaei_documentation_2022}). The first paper (a questionnaire-based investigation) on Kholosi\il{Indic!Kholosi} was published by \citet{anonby_shipwrecked_2016}, which confirmed that Kholosi\il{Indic!Kholosi} is an Indo-Aryan language. In addition, \citet{Anonby2019Kholosi} published a short paper in Encyclopædia Iranica. The first Kholosi\il{Indic!Kholosi} text to be transcribed, glossed and translated into English in FLEX has been published by myself, and includes a sociolinguistic study of Kholosi\il{Indic!Kholosi} \citet{nourzaei_orality_2023}. I have also written a grammar sketch \citep{nourzaei_forthcoming_kholosi}, and an outline of the Kholosi\il{Indic!Kholosi} nominal and pronominal system is under review. \citet{arora_kholosi_2021,nourzaei_forthcoming_morphosyntax} published online a list of Kholosi\il{Indic!Kholosi} lexical items, based on English words presented to one speaker via the medium of Persian\il{Persian} translation and through an online messenger service. The present study is the first contribution to the \isi{word order} in Kholosi\il{Indic!Kholosi}, with special \isi{focus} on post-predicate elements. 

\subsection{Data and methodology}\label{Kholosi:ss:1.2}

The background analysis for this paper is based on the ongoing documentation described above, while the quantitative data stems from two tales entitled ``Untidy fox,'' recorded from a 32-year-old female speaker from Kholos in 2020, and ``Prophet Musa,'' recorded from a male speaker from Kholos in 2017 and available in the WOWA database \citep{nourzaei_kholosi_2022}. I have also included data from a growing corpus of Kholosi\il{Indic!Kholosi} narrative speech including folktales, biographical tales, and procedural texts spoken by one male and two female Kholosi\il{Indic!Kholosi} speakers from Kholos and Gotav. The speakers have different social backgrounds and are between 32 and 82 years of age. All speakers are fluent in Persian\il{Persian} and vernacular languages such as Bastaki\il{Bastaki} and Lārestāni\il{Lārestāni}, and can read Arabic\il{Arabic} very well. One of the female speakers is an Arabic\il{Arabic} teacher at an Islamic school in Kholos, and the male speaker is fluent in Arabic\il{Arabic}. The texts were recorded in WAV and MP3 format. They were then imported into ELAN software,\footnote{ \url{https://tla.mpi.nl/tools/tla-tools/elan/}} transcribed phonemically, and double-checked by the linguistic consultants. After that, a morpheme-by-morpheme glossing was carried out using {FLEx} software.\footnote{\url{http://fieldwork.sil.org/}} Finally, a free clause-by-clause translation of the texts was produced.\footnote{Both the texts and sound files can be found online, see \citet{nourzaei_kholosi_2022}.} 

\subsection{Word order in Indo-Aryan}\label{Kholosi:ss:1.3}

Like the Iranian languages, the Indo-Aryan languages show an \isi{OV} order in the clause. Noun phrases across all of IA are left-branching (Modifier + N) \citet[370]{masica_indo-aryan_1991}. Examples from Sindhi, one of Kholosi's presumably closest Indo-Aryan relatives, illustrate this feature: \textit{vada vātō} `big mouth'; \textit{sāfu dīle} `pure heart'; \textit{čau darō} `four doors' and \textit{ghara ǰō dhaṇī} `house-of the master' i.e. the master of the house (\citealt[88, 119] {trumpp_grammar_1872}). \citet[218]{masica_indo-aryan_1991} states that across Indo-Aryan, primarily nouns, pronouns, adjectives, and in some languages, also certain numerals and adverbs may inflect for gender, number, case, and \isi{definiteness}. While case is found in all IA languages, gender is not universal, and a grammatical marker for \isi{definiteness} characterizes only certain languages. Case, number and gender are noted for Sindhi (\citealt[119, 131]{trumpp_grammar_1872}), but reliable information on \isi{definiteness} is not available.
\citet[373]{masica_indo-aryan_1991} mentions that for all VP arguments, e.g. the direct and indirect objects, goals and source of motion, all occur to the left of the main verb stem (except in aberrant Kashmiri). Examples (\ref{Kholosi:ex:1}-\ref{Kholosi:ex:2})  from Gujarati\il{Indic!Gujarati} and Sinhalese\il{Indic!Sinhalese} demonstrate preverbal \isi{Goal} and \isi{direct object} in turn, glosses added.

\ea\label{Kholosi:ex:1}
Gujarati \il{Indic!Gujarati}(\citealt[352]{masica_indo-aryan_1991})\\
\gll māre/mane ghɛr jav\=an\~u che \\
\textsc{1sg.obl} home go\textsc{.inf} \textsc{cop.3sg} \\
\glt `I have to go home.'
\z

\ea\label{Kholosi:ex:2}
Sinhalese \il{Indic!Sinhalese}(\citealt[333]{masica_indo-aryan_1991})\\
\gll mama bat kanavā \\
\textsc{1sg} rice eat\textsc{.prs} \\
\glt `I eat rice.'
\z

We are unaware of any corpus-based investigation of post-posed elements in Indo-Aryan that might assist in reconstructing the ancestor \isi{word order} of Kholosi\il{Indic!Kholosi}. Preliminary consideration of available material, such as \citet{liljegren_grammar_2016} on Palula\il{Indic!Palula} (Indo-Aryan, Dardic) indicate a dominant \isi{head-final} VP, and in the experimental investigation of Hindi \isi{word order} by \citet{patil_focus_2008}, only verb-final variants are considered. In the absence of evidence to the contrary, I will therefore assume that the Indo-Aryan relatives of Kholosi\il{Indic!Kholosi} are consistently verb-final, from which we can provisionally infer that the ancestor language of Kholosi\il{Indic!Kholosi} was likewise verb-final.

\section{Some elements of Kholosi\il{Indic!Kholosi} grammar}\label{Kholosi:ss:2}

\subsection{Alignment}\label{Kholosi:ss:2.1}

Some basics are needed as background information for the discussion which follows. Kholosi\il{Indic!Kholosi} exhibits \isi{accusative} alignment with all verbal categories, as in Persian\il{Persian}. The subject is canonically in the nominative case and the \isi{object} in the \isi{oblique} case. The verb agrees with the subject both in number and person. Note that in the past domain, transitive and intransitive verbs have different sets of endings, which presumably reflects an earlier \isi{ergative} alignment in Kholosi\il{Indic!Kholosi}. The following examples present \isi{accusative} alignment for transitive and intransitive verbs in the present (\ref{Kholosi:ex:3}-\ref{Kholosi:ex:4}) and past (\ref{Kholosi:ex:5}-\ref{Kholosi:ex:6}) domains.

\ea\label{Kholosi:ex:3}
Kholosi \il{Indic!Kholosi}(\citealt{nourzaei_kholosi_2022}, B,0234)\\
\gll sandūġ=e čūb-ī deres ker-aw \\
box\textsc{=indv} wood\textsc{-adjv} complete do\textsc{.prs-3sg} \\
\glt `He makes a wooden box.'
\z

\ea\label{Kholosi:ex:4}
Kholosi \il{Indic!Kholosi}(\citealt{nourzaei_kholosi_2022}, B, 0160)\\
\gll yak hāro=e denyā t=eǰ-aw \\
one boy\textsc{=indv} world to=come\textsc{.prs-3sg} \\
\glt `A boy will be born (lit. comes to the world).'
\z


\newpage
\ea\label{Kholosi:ex:5}
Kholosi \il{Indic!Kholosi}(\citealt{nourzaei_kholosi_2022}, A, 0019)\\
\gll hoko roz robā šamšer=es pale pord-ū \\
one day fox sword\textsc{=pc.3sg} \textsc{prev} search\textsc{.pst-3sg} \\
\glt `One day the fox, (he) searched for his sword.' 
\z

\ea\label{Kholosi:ex:6}
Kholosi \il{Indic!Kholosi}(\citealt{nourzaei_kholosi_2022}, A, 0005)\\
\gll robā nōk-ō dar he sere mā vađ-ō tʰo \\
fox\textsc{.m} small\textsc{-m} in textsc{prox} house in big\textsc{-m} become\textsc{.pst.3sg} \\
\glt `The small fox grew up in this house.'
\z

\subsection{Morphological case system}\label{Kholosi:ss:2.2}

Kholosi\il{Indic!Kholosi} shows three morphological cases: Direct (unmarked), \isi{oblique} \textit{=ke}, and genitive \textit{=ǰō/ǰī} cases.\footnote{ It is currently unclear whether these `case markers' should be considered to be some form of particle (i.e. postpositions), or phrasal affix, or \isi{clitic}. They are provisionally treated here as clitics.} The genitive case is partially replaced with Iranic Ezafe construction (see for more details \citealt{nourzaei_forthcoming_kholosi}, and \citealt{nourzaei_forthcoming_morphosyntax}). Kholosi\il{Indic!Kholosi} has Differential Object Marking, though the exact nature of the triggering factors has not yet been established. Indefinite and inanimate objects such as `wooden box' in (\ref{Kholosi:ex:3}) are unmarked for case, while other objects can be marked with \textit{=ke}, as in (\ref{Kholosi:ex:25}).

\subsection{Person marking clitics}\label{Kholosi:ss:2.3}

\begin{sloppypar}
Similar to Iranic spoken in this region, e.g. Koroshi\il{Balochi!Koroshi}, Kholosi\il{Indic!Kholosi} has a full set of person-marking clitics, which have different functions, e.g., possession, direct and indirect object\is{object!indirect} markings. The person-marking clitics in Kholosi\il{Indic!Kholosi} are presented in \tabref{Kholosi:tab:1}. The person-marking clitics attach to various hosts: nouns, \textit{kolāh=ās} `his hat,' adverbs, \textit{roz=e baad=os} `its next day,' postposition, \textit{ag=es} `to him,' relational nouns, \textit{ak=es} `his front,' verbs, \textit{ `čaī=ves} `says to him'. Note, some of these forms are copied from Iranic, such as 2nd person \textit{=ū} and 3rd person \textit{=es/eš}.
\end{sloppypar}

\begin{table}
 \centering
 \begin{tabular}{lll}
\lsptoprule
\textsc{sg} & 1st & \textit{=oe/=e/oy/=ī/yāe} \\
 & 2nd & \textit{=o/ū/āv/āo/} \\
 & 3rd & \textit{=ās/yās/os/es/eš} \\
\textsc{pl} & 1st & \textit{=om=ām=mām/em} \\
 & 2nd & \textit{=om=ām} \\
 & 3rd & \textit{=ān/on/en} \\
\lspbottomrule
 \end{tabular}
 \caption{Person-marking clitics in Kholosi}
 \label{Kholosi:tab:1}
\end{table}

\subsection{Morphological gender marker}\label{Kholosi:ss:2.4}

In other respects, Kholosi\il{Indic!Kholosi} differs from Iranic. This includes a grammatical gender distinction in some verb endings - in particular the 1st, 2nd and 3rd persons singular - numbers, and adjectives (preferably Indo-Aryan), comparative suffix \textit{-r}, imperfective suffix \textit{-d}, and genitive suffix \textit{-ǰ}, which distinguish masculine and feminine gender. Masculine adjectives are marked with (\textit{o/ō}) and feminine adjectives with (\textit{ī}). In (\ref{Kholosi:ex:7}), \textit{-ō} on the adjectives \textit{gahr} `red' agrees with the river Nile, which is masculine. Gender agreement can be within the NP, as in (\ref{Kholosi:ex:8}), or between subject and predicate, as in (\ref{Kholosi:ex:7}); the details remain to be elucidated.

\ea\label{Kholosi:ex:7}
Kholosi \il{Indic!Kholosi}(\citealt{nourzaei_kholosi_2022}, B, 0542)\\
\gll rūd=e nīl gahr-ō tʰ=ū \\
river\textsc{=ez} {N}ile\textsc{.m} red\textsc{-\textbf{m}} become\textsc{.pst=cop.prs.3sg} \\
\glt `The river Nile turned red.' 
\z

Similarly, in (\ref{Kholosi:ex:8}), \textit{-ī} on the \isi{adjective} \textit{kalt} `big' agrees with `dragon', which is feminine. 

\ea\label{Kholosi:ex:8}
Kholosi \il{Indic!Kholosi}(UP)\\
\gll aždehā=ye kalt-ī \\
dragon\textsc{.f=ez} big\textsc{\textbf{-f}} \\
\glt `a big dragon'
\z

Gender is also marked directly on some nouns, with phonologically comparable suffixes to those outlined above: \textit{nonō} `grandfather' and \textit{nonī} `grandmother'; \textit{čōrkō} `boy' and \textit{čōrkī} `girl'; \textit{hārrō} `boy'and \textit{herrī} `girl'.

\section{Word order in the NP and the clause}\label{Kholosi:ss:3}

\subsection{Adjective/noun}\label{Kholosi:ss:3.1}

In the noun phrase, Kholosi\il{Indic!Kholosi} has adopted some Persian\il{Persian} syntactic features. For example, some adjectives may follow the noun and be linked to it via the `ezafe,' as in (\ref{Kholosi:ex:9}), or occur without an `ezafe', as in (\ref{Kholosi:ex:10}).

\ea\label{Kholosi:ex:9}
Kholosi \il{Indic!Kholosi}(\citealt{nourzaei_kholosi_2022}, A, 0044)\\
\gll moškel=e vađ-o \\
problem\textsc{=ez} big\textsc{-m} \\
\glt `a big problem'
\z

\ea\label{Kholosi:ex:10}
Kholosi \il{Indic!Kholosi}(\citealt{nourzaei_kholosi_2022}, A, 0005)\\
\gll rōbā nōk-ō \\
fox small\textsc{-m} \\
\glt `the small fox'
\z

\subsection{Possessor/possessed}\label{Kholosi:ss:3.2}

Kholosi\il{Indic!Kholosi} also follows the general Persian\il{Persian} pattern in that the possessed precedes possessor, either with an intervening `ezafe' (\textsc{ez}), as in (\ref{Kholosi:ex:11}), or without, as in (\ref{Kholosi:ex:12}). Note the use of the genitive marker on the possessor in (\ref{Kholosi:ex:12}).

\ea\label{Kholosi:ex:11}
Kholosi \il{Indic!Kholosi}(UP)\\
\gll laškar=e ferawn \\
army\textsc{=ez} pharaoh \\
\glt `pharaoh's army'
\z

\ea\label{Kholosi:ex:12}
Kholosi \il{Indic!Kholosi}(\citealt{nourzaei_kholosi_2022}, A, 0148)\\
\gll māv rūbāh kočūlū=ǰo \\
mother fox small\textsc{=gen.m} \\
\glt `mother of small fox'
\z

However, the data demonstrate some variation, and the possessor can also precede the possessed, as (\ref{Kholosi:ex:13}), which also illustrates the expected Indo-Aryan genitive marking of the possessor. 

\ea\label{Kholosi:ex:13}
Kholosi \il{Indic!Kholosi}(\citealt{nourzaei_kholosi_2022}, B, 0266)\\
\gll hazrat mūsā=ǰī dođā \\
prophet {M}usa\textsc{=gen.f} sister \\
\glt `the sister of the prophet Musa'
\z

Possession and similar relations have also been attested with juxtaposition, as \textit{māre šahr} `people of the city' which have no marking of possession in (\ref{Kholosi:ex:14}).

\ea\label{Kholosi:ex:14}
Kholosi \il{Indic!Kholosi}(\citealt{nourzaei_kholosi_2022}, B, 0549)\\
\gll tamām=e sāher-ēn=ke \textbf{māre} \textbf{šahr} daʔvat dī-yaw \\
all\textsc{=ez} magician\textsc{-pl=obl} people city invite give\textsc{.prs-3sg} \\
\glt `He invites all the magicians [and] people of the city.'
\z

\subsection{Demonstrative/noun}\label{Kholosi:ss:3.3}

Demonstrative pronoun\is{pronoun!demonstrative}s precede the head nouns without any linker. 

\ea\label{Kholosi:ex:15}
\ea\label{Kholosi:ex:15a}
Kholosi \il{Indic!Kholosi}(\citealt{nourzaei_kholosi_2022}, A, 0005)\\
\gll \textbf{he} sere \\
\textsc{prox} house \\
\glt `this house'
\ex\label{Kholosi:ex:15b}
Kholosi \il{Indic!Kholosi}(UP)\\
\gll \textbf{ho} motor \\
\textsc{dist} car \\
\glt `that car'
\z
\z

\subsection{Numeral/noun}\label{Kholosi:ss:3.4}

In Kholosi\il{Indic!Kholosi}, numerals precede head nouns. The head nouns show number and gender agreement with only Indo-Aryan numerals, as in the following examples.

\ea\label{Kholosi:ex:16}
\ea\label{Kholosi:ex:16a}
Kholosi \il{Indic!Kholosi}(\citealt{nourzaei_kholosi_2022}, B, 0160)\\
\gll hēk-ō hār-ō \\
one\textsc{-m} boy\textsc{-m} \\
\glt `a boy' 
\ex\label{Kholosi:ex:16b}
Kholosi\il{Indic!Kholosi}\\
\gll bahr-ā hēr-ā \\
two\textsc{-m} boy\textsc{-m.pl} \\
\glt `two boys'
\z
\z

\ea\label{Kholosi:ex:17}
\ea\label{Kholosi:ex:17a}
Kholosi \il{Indic!Kholosi}(UP)\\
\gll hīk-ī čōrk-ī \\
one\textsc{-f} girl\textsc{-f} \\
\glt `a girl'
\ex\label{Kholosi:ex:17b}
Kholosi \il{Indic!Kholosi}(UP)\\
\gll bahr-ī čōrk-ī-ū \\
two\textsc{-f} girl\textsc{-f-pl} \\
\glt `two girls'
\z
\z

\subsection{Adpositions}\label{Kholosi:ss:3.5}

Kholosi\il{Indic!Kholosi} has a mixed typology with respect to adpositions, with prepositions, postpositions, and combinations of the two. The most widely attested prepositions are obvious borrowings from Iranic, while postpositions are Indo-Aryan. (\ref{Kholosi:ex:18}) shows a combination of an Iranic \isi{preposition} and an Indo-Aryan postposition and (\ref{Kholosi:ex:19}) just Indo-Aryan postposition.

\ea\label{Kholosi:ex:18}
Kholosi \il{Indic!Kholosi}(\citealt{nourzaei_kholosi_2022}, A, 0032)\\
\gll \textbf{az} mān=ās \textbf{tāv} soāl ka-ī \\
from mother\textsc{=pc.3sg} from question do\textsc{.pst-3sg} \\
\glt `he asked of his mother'
\z

\ea\label{Kholosi:ex:19}
Kholosi \il{Indic!Kholosi}(\citealt{nourzaei_kholosi_2022}, A, 0034)\\
\gll mov=ās gen=ās \textbf{mā} pēr-ī \\
mother\textsc{=pc.3sg} room\textsc{=pc.3sg} in look\textsc{.pst-3sg} \\
\glt`His mother looked inside his room.'
\z

There is a tendency to use the Iranic \isi{preposition} \textit{dāxel} as a postposition in Kholosi\il{Indic!Kholosi}, with the same meaning, as in (\ref{Kholosi:ex:20})

\ea\label{Kholosi:ex:20}
Kholosi \il{Indic!Kholosi}(UP)\\
\gll ham=e atte \textbf{dāxel} māyexamīr kaṛ-e \\
\textsc{emph=prox} dough inside yeast do\textsc{.prs-2sg} \\
\glt `You add yeast to (lit. in) this dough.'
\z

\subsection{Auxiliary/main verb}\label{Kholosi:ss:3.6}

Independent auxiliaries expressing tense, aspect, mood, or voice, are not attested in the present data; TAM categories are expressed by suffixes on the verb. There are no TAM prefixes that would parallel the widely-attested aspectual and modal prefixes of the Iranic system (Persian\il{Persian} \textit{mi-} and \textit{be-}, for example). However, modal verbs such as `want' precede the main verbs, as in Iranic (\ref{Kholosi:ex:32}), and this could well be considered a result of \isi{contact influence}, in line with the claim that clause (or predicate) combining is a favoured domain for \isi{contact influence} \citep{haig_linguistic_2001}.

\subsection{Complement clause/matrix verb}\label{Kholosi:ss:3.7}

The subordinators \textit{ǰō} and \textit{ke} `that' introduce various subordinate clauses, including relative, \isi{complement} and adverbial - as well as quoted speech. The particle \textit{ke} is an Iranic borrowing. Similarly, to Persian\il{Persian}, the complement clause\is{complement!clause} follows the matrix verb in Kholosi\il{Indic!Kholosi}, and complementizers generally occur in the initial position within the complement clause\is{complement!clause}. In these respects, Kholosi\il{Indic!Kholosi} entirely matches Persian\il{Persian} and other West Iranian languages.

\ea\label{Kholosi:ex:21}
Kholosi \il{Indic!Kholosi}(\citealt{nourzaei_kholosi_2022}, B, 0457)\\
\gll čī-ya ǰō hat=ī mā lat=e \\
say\textsc{.prs-3sg} \textsc{cmp} hand\textsc{=pc.1sg} in stick\textsc{=cop.prs.3sg} \\
\glt `He says, ``There is a stick in my hand''.'
\z

\ea\label{Kholosi:ex:22}
Kholosi \il{Indic!Kholosi}(\citealt{nourzaei_kholosi_2022}, B, 0247)\\
\gll pesa-vān ǰō bale yak sandūġ=e hat \\
see\textsc{.pst-3pl} \textsc{cmp} yes one box\textsc{=indv} \textsc{cop.prs.3sg} \\
\glt `They saw that, yes, there is a box [on the water].' 
\z

\subsection{Light verb complements}\label{Kholosi:ss:3.8}

Kholosi\il{Indic!Kholosi} follows the widespread Iranic  pattern of combining non-verbal elements with light verbs, in that order, to create complex predicates. The attested light verbs are `become,' `come,' `do,' and `give' as in examples (\ref{Kholosi:ex:23}) and (\ref{Kholosi:ex:24}).

\ea\label{Kholosi:ex:23}
Kholosi \il{Indic!Kholosi}(\citealt{nourzaei_kholosi_2022}, A, 0166)\\
\gll ferawn peǰal tāv bīdār tʰī-yaw \\
pharaoh sleep from awake become\textsc{.prs-3sg} \\
\glt `The Pharaoh gets up (lit. wakes up from sleep).' 
\z

\section{VP constituents in the Kholosi\il{Indic!Kholosi} WOWA data}\label{Kholosi:ss:4}

\begin{sloppypar}
In this section we exemplify the main combinations of non-subject constituents and verb, based on the quantitative analysis of the WOWA Kholosi\il{Indic!Kholosi} data set \citep{nourzaei_kholosi_2022}. An overview of the findings is presented in \tabref{Kholosi:tab:2} below. Kholosi\il{Indic!Kholosi} makes extensive use of pro-drop in different syntactic functions, so pronominal arguments are infrequent in the data. In line with general practice in the WOWA framework (see \citetv{chapters/1_Haigetal_Intro}), we will not consider the position of subjects.
\end{sloppypar}

\subsection{Direct objects}\label{Kholosi:ss:4.1}

In Kholosi\il{Indic!Kholosi}, generally nominal direct objects occur in pre-verbal position, as in the following example. They may be separated from the verb by other constituents, as in (\ref{Kholosi:ex:24}), but are only very rarely post-posed after the verb.


\newpage
\ea\label{Kholosi:ex:24}
Kholosi \il{Indic!Kholosi}(\citealt{nourzaei_kholosi_2022}, A, 0139)\\
\gll šamšīr=os o=te āvīzān ker-a-va \\
sword\textsc{=pc.3sg} there=to hang do\textsc{.prs-3sg-sbjv} \\
\glt `He should hang his sword in the other direction.' 
\z

In the Kholosi\il{Indic!Kholosi} data used for this project, only 2\% of all direct objects are in postverbal position, which makes it even lower than obtained for most Iranian languages, which range between 3--10\%. This may be a reflex of the Indo-Aryan origins, but the differences are certainly not great.

Free pronouns as direct objects are rare in Kholosi\il{Indic!Kholosi} (14 examples), of which all are pre-verbal, as in (\ref{Kholosi:ex:25}). Direct objects (and other objects) may also be expressed through a clitic pronoun\is{clitic!pronoun}\is{pronoun!clitic} on the verb, which then follows the verb, but as it is part of the same phonological word as the verb, these are not counted as `post-verbal' in line with standard practice for WOWA. An example of a \isi{clitic} \isi{object} is provided in (\ref{Kholosi:ex:26}).

\ea\label{Kholosi:ex:25}
Kholosi \il{Indic!Kholosi}(UP)\\
\gll meskīn=e golī pʰepī=yāe mo=ke vađ-o ka-ī \\
poor\textsc{=ez} goli aunt\textsc{=pc.1sg} \textsc{1sg=obl} big\textsc{-m} do\textsc{.pst-3sg} \\
\glt `My poor aunt, Goli, raised me.' 
\z

\ea\label{Kholosi:ex:26}
Kholosi \il{Indic!Kholosi}(\citealt{nourzaei_kholosi_2022}, B, 305)\\
\gll ferawn konǰān-do=sū čembī dīv=es-ī-ya \\
pharaoh want\textsc{.pst-ipfv=cop.pst.3sg} kiss give\textsc{.prs=pc.3sg-3sg-sbjv} \\
\glt `Like this pharaoh was about to kiss him.' 
\z

\subsection{Copula complements}\label{Kholosi:ss:4.2}

Copulas are generally clitics which cliticize to the copular \isi{complement}, and are thus generally clause-final, as in (\ref{Kholosi:ex:27}-\ref{Kholosi:ex:28}). This is in line with a near-universal structure for copular constructions across the Western Asian Transition Zone, though with exceptions at the western periphery (see \citealt{haig_introduction_2018}, \citetv{chapters/1_Haigetal_Intro}, \citetv{chapters/9_Mohammadirad_Gorani}).

\ea\label{Kholosi:ex:27}
Kholosi \il{Indic!Kholosi}(UP)\\
\gll pon=āe=ǰī mīṛī=ya \\
father\textsc{=pc.1sg=gen} wife\textsc{=cop.prs.3sg} \\
\glt `She is my father's wife.'
\z

\ea\label{Kholosi:ex:28}
Kholosi \il{Indic!Kholosi}(UP)\\
\gll čekada zaīfā=e čang-ī=ū \\
how{\_}much woman\textsc{=indv} kind\textsc{-f=cop.pst.3sg} \\
\glt `She was a very kind woman.' 
\z

Note that in the present data, one example of a postposed \isi{complement} of copular verb is attested (\ref{Kholosi:ex:29}):

\ea\label{Kholosi:ex:29}
Kholosi \il{Indic!Kholosi}(\citealt{nourzaei_kholosi_2022}, B, 0230)\\
\gll hat \textbf{ǰīnd-o} \\
\textsc{cop.prs.3sg} alive\textsc{-m} \\
\glt `He is alive.'
\z

\subsection{Goal/verb}\label{Kholosi:ss:4.3}

In Kholosi\il{Indic!Kholosi}, the largest number of post-verbal elements are goals of caused motion, such as `throw', `put,' and `hit' (see \tabref{Kholosi:tab:2}) as in (\ref{Kholosi:ex:30}-\ref{Kholosi:ex:31}).

\ea\label{Kholosi:ex:30}
Kholosi \il{Indic!Kholosi}(\citealt{nourzaei_kholosi_2022}, B, 0193)\\
\gll hazrat=e mūsā=ī māv petr=es=ke lār-aw tanūr mā \\
prophet\textsc{=ez} musa\textsc{=gen} mother boy\textsc{=pc.3sg=obl} throw\textsc{.prs-3sg} tanur in \\
\glt `The Prophet Musa's mother throws her son into the Tanur.' 
\z

\ea\label{Kholosi:ex:31}
Kholosi \il{Indic!Kholosi}(UP)\\
\gll kolan lī-yāv=es fer mā \\
you\_know put\textsc{.prs-1pl=pc.3sg} oven in \\
\glt `You know, we put it into the oven.' 
\z

Similarly, goals of simple motion verbs such as `go,' `come,' `fall,' are often post-verbal (see \tabref{Kholosi:tab:2}).

\ea\label{Kholosi:ex:32}
Kholosi \il{Indic!Kholosi}(\citealt{nourzaei_kholosi_2022}, A, 0024)\\
\gll ho konǰān-do=sū ven-a-va \textbf{rafīk-en=des} akaya\\
\textsc{dist} want\textsc{.pst-ipfv=cop.pst.3sg} go\textsc{.prs-3sg-sbjv} friend\textsc{-pl=pc.3sg} front\\
\glt `He wanted to go to his friends.' 
\z

\ea\label{Kholosi:ex:33}
Kholosi \il{Indic!Kholosi}(\citealt{nourzaei_kholosi_2022}, B, 0366)\\
\gll eč-aw \textbf{yek} \textbf{ǰā=e} \\
come\textsc{.prs-3sg} one place\textsc{=indv} \\
\glt `He comes to a certain place.'
\z

\subsection{Recipient/verb}\label{Kholosi:ss:4.4}

Similarly to goals, the majority of recipients also follow the verb, as in (\ref{Kholosi:ex:34}-\ref{Kholosi:ex:35}), though the absolute number of recipients in the data is quite low, cf. \tabref{Kholosi:tab:2}.

\ea\label{Kholosi:ex:34}
Kholosi \il{Indic!Kholosi}(\citealt{nourzaei_kholosi_2022}, B, 0394)\\
\gll tʰīo ponǰ=ās dī-aw \textbf{payambar} \textbf{mūsā=ke} \\
daughter \textsc{refl.gen=pc.3sg} give\textsc{.prs-3sg} prophet Musa\textsc{=obl} \\
\glt `He gives his daughter to the prophet Musa.' 
\z

\ea\label{Kholosi:ex:35}
Kholosi \il{Indic!Kholosi}(\citealt{nourzaei_kholosi_2022}, B, 0205)\\
\gll xebar dī-yen hazrat=e \textbf{mūsā=ī} \textbf{māo=ke} \\
news do\textsc{.prs-3pl} prophet\textsc{=ez} Musa\textsc{=gen} mother\textsc{=obl} \\
\glt `They informed the Prophet Musa's mother.'
\z

\subsection{Addressee/verb}\label{Kholosi:ss:4.5}

Nominal addressees in Kholosi\il{Indic!Kholosi} precede the verb as in (\ref{Kholosi:ex:36}), but addressees expressed by a person marking \isi{clitic} attach directly to the verb as in (\ref{Kholosi:ex:37}), or attach to a postposition as in (\ref{Kholosi:ex:38}).

\ea\label{Kholosi:ex:36}
Kholosi \il{Indic!Kholosi}(UP)\\
\gll am=te xodā=ke ča-yen \\
this=at God\textsc{=obl} say\textsc{.prs-3pl} \\
\glt `At this point they said to God...' 
\z

\ea\label{Kholosi:ex:37}
Kholosi \il{Indic!Kholosi}(\citealt{nourzaei_kholosi_2022}, B, 0520)\\
\gll payāmbar hazrat=e mūsā xayle kām-hā vet-ī=yās \\
prophet prophet\textsc{=ez} {M}usa very thing\textsc{-pl} say\textsc{.pst-3sg=pc.3sg} \\
\glt `The Prophet Musa said to him a lot of things.' 
\z

\ea\label{Kholosi:ex:38}
Kholosi \il{Indic!Kholosi}(\citealt{nourzaei_kholosi_2022}, B, 0172)\\
\gll tā yek nafar manī=os čī-yaw ag=es \\
till one person meaning\textsc{=pc.3sg} say\textsc{.prs-3sg} to\textsc{=pc.3sg} \\
\glt `Till a person says its meaning to him.' 
\z

\subsection{Complements of `become'}\label{Kholosi:ss:4.6}

In Kholosi\il{Indic!Kholosi} complements of `become' are preverbal, as in the following examples.



\ea\label{Kholosi:ex:39}
Kholosi \il{Indic!Kholosi}(\citealt{nourzaei_kholosi_2022}, B, 0334)\\
\gll ferawn motmaen tʰo ǰo... \\
pharaoh relaxed become\textsc{.pst.3sg} \textsc{cmp} \\
\glt `The Pharaoh became relaxed that...' 
\z

\ea\label{Kholosi:ex:40}
Kholosi \il{Indic!Kholosi}(\citealt{nourzaei_kholosi_2022}, B, 0483)\\
\gll heždehā=te tʰī-aw \\
dragon=to become\textsc{.prs-3sg} \\
\glt `It will turn into a dragon.' 
\z

\subsection{Place constituents and place constituents of a copular verb}\label{Kholosi:ss:4.7}

Locative constituents of a copular verb i.e., (`X is in the garden') (\ref{Kholosi:ex:41}) precede the predicate.

\ea\label{Kholosi:ex:41}
Kholosi \il{Indic!Kholosi}(daily conversation)\\
\gll Hasan dar sere mā=e \\
hasan in home in\textsc{=cop.prs.3sg} \\
\glt `Hasan is at home.' 
\z

\subsection{Other obliques}\label{Kholosi:ss:4.8}

Most other obliques such as instruments, ablatives, comitatives, and beneficiaries appear before the verb; see the figures for `other' in \tabref{Kholosi:tab:2} below. The following examples demonstrate various other semantic relations.

\ea\label{Kholosi:ex:42}
Kholosi \il{Indic!Kholosi}(\citealt{nourzaei_kholosi_2022}, A, 0003)\\
\gll hok-ō xānavāda hīnkī sānda zendegī kar-d=ayaū \\
one\textsc{-m} family each other life do\textsc{.prs-ipfv=cop.prs.3pl} \\
\glt `A family used to live with one another.' 
\z

\ea\label{Kholosi:ex:43}
Kholosi \il{Indic!Kholosi}(\citealt{nourzaei_kholosi_2022}, A, 0065)\\
\gll xers bā šamšīr=os be havā=te be zarba čīn-d=ū \\
bear with sword\textsc{=pc.3sg} with air=at with beat beat\textsc{.prs-ipfv=cop.pst.3sg} \\
\glt `The bear was striking with his sword into the air.' 
\z

\ea\label{Kholosi:ex:44}
Kholosi \il{Indic!Kholosi}(\citealt{nourzaei_kholosi_2022}, B, 0162)\\
\gll yak mangāl=e az ġawm=e banī esrāīl tāv boland tʰo \\
one fire\textsc{=indv} from tribe\textsc{=ez} sons\_of {I}srael from upright become\textsc{.pst.3sg} \\
\glt `A fire rose from the people of Israel.' 
\z


\section{Summary}\label{Kholosi:ss:5}

So far, I have discussed various \isi{word order} parameters, and the post-verbal placement of different constituents related to the verb. This historically \isi{head-final} language shows some shifts towards \isi{head-initial} syntax, leading to some degree of inconsistency in head-directionality (see \citealt{Dryer1992Greeburg}). We also note the occasional use of mixed \isi{head-final} and \isi{head-initial} constructions involving borrowed prepositions combined with inherited post-positions, see (\ref{Kholosi:ex:18}). The \isi{head-initial} configurations which have been illustrated in this chapter are given in \tabref{tab:6:configurations}.

\begin{table}
  \caption{\label{tab:6:configurations}Head-initial configurations}
\begin{tabular}{ll}
\lsptoprule
\textbf{Head} & \textbf{Complement} \\
\midrule
Noun & Adjective \\
Possessed & Possessor \\
Matrix clause & Complement clause \\
Complementizer & Complement clause \\
Verb & Goal \\
Verb & Recipient \\
\lspbottomrule
\end{tabular}
\end{table}

Within the VP, most kinds of verbal arguments remain consistently pre-verbal (90\% or more), with the sole exception of goals and recipients. \tabref{Kholosi:tab:2} shows the relevant figures for the most important roles, as identified in the WOWA framework (see \citetv{chapters/1_Haigetal_Intro}). Note that ``Goals'' subsumes both goals of verbs of motion and of caused motion (both of which are predominantly post-verbal in Kholosi\il{Indic!Kholosi}).

\begin{table}
    \fittable{\begin{tabular}{lrrrr}
\lsptoprule
& Total number of tokens & preverbal & post-verbal & \% post-verbal \\
\midrule
\textbf{Ablative} & 22 & 22 & 0 & 0\% \\
\textbf{Addressee} & 14 & 13 & 1 & 7\% \\
\textbf{Become} & 14 & 13 & 1 & 7\% \\
\textbf{Comitative} & 13 & 13 & 0 & 0\% \\
\textbf{Copular expression} & 41 & 39 & 2 & 5\% \\
\textbf{Direct object} & 67 & 65 & 2 & 3\% \\
\textbf{Definite direct object} & 88 & 86 & 2 & 2\% \\
\textbf{Goal} & 56 & 21 & 35 & 62\% \\
\textbf{Static location} & 43 & 42 & 1 & 2\% \\
\textbf{Other} & 128 & 126 & 2 & 2\% \\
\textbf{Possessed} & 12 & 12 & 0 & 0\% \\
\textbf{Recipient} & 8 & 2 & 6 & 75\% \\
\midrule
\textbf{Totals} & \textbf{506} & \textbf{454} & \textbf{52} &  \\
\lspbottomrule
    \end{tabular}}
    \caption{Percentages of post-predicate placement of different constituents in Kholosi}
    \label{Kholosi:tab:2}
\end{table}
					
Compared to its Indo-Aryan relative Jadgali\il{Indic!Jadgali}, Kholosi\il{Indic!Kholosi} exhibits greater influence of Iranic. Kholosi\il{Indic!Kholosi} has adopted some features of Persian\il{Persian} noun phrases, for example post-posed adjectives, and some prepositions, while Jadgali\il{Indic!Jadgali} retains the Indo-Aryan structures. Jadgali\il{Indic!Jadgali} also retains its split \isi{ergative} alignment, while Kholosi\il{Indic!Kholosi} shares the same alignment (\isi{accusative}) with Persian\il{Persian}, and makes more extensive use of person-marking clitics. Overall, it seems that Kholosi\il{Indic!Kholosi} has undergone a greater degree of syntactic \isi{convergence} with Iranian languages. Kholosi\il{Indic!Kholosi} nevertheless maintains grammatical features that distinguish it from (neighbouring) Iranic. These include a grammatical gender distinction on some numerals, verb endings, and adjectives (if the host item is Indo-Aryan), and also postpositions, which are a prominent feature of Indo-Aryan languages. Note that its morphological gender system is not strong as such (see Nourzaei \citealt{nourzaei_forthcoming_morphosyntax}). Unlike Iranian languages, Kholosi\il{Indic!Kholosi} lacks any prefixal TAM elements. 

At the level of the clause, Kholosi\il{Indic!Kholosi} has adapted to the profile of the Iranian languages of Western Asia: Modal verbs precede main verbs, subordinate clauses follow matrix clauses, and complementizers occur clause-initially. One of the most salient features of these languages is the combination of near-categorical \isi{OV} with a high frequency of post-verbal Goal\is{Goal!post-verbal}s. Kholosi\il{Indic!Kholosi} likewise exhibits this combination (>90\% \isi{OV}, and >60\% VG, cf. \tabref{Kholosi:tab:2}). Note, however, that rates of post-verbal Goal\is{Goal!post-verbal}s do not reach the levels found further westward in the Iranian languages of Mesopotamia. But Kholosi does share with several Iranian languages the spread of post-verbal placement to include recipients, entirely in line with the sequence predicted in \textcitetv{chapters/1_Haigetal_Intro}. Other types of constituents (such as locatives, copula complement\is{copula!complement}s, adverbs, addressees) show significantly lower rates of pos-verbal position.

Although Kholosi\il{Indic!Kholosi} retains abundant evidence of its Indo-Aryan origins in the lexicon and in morphology, it exhibits syntactic \isi{convergence} with areally contiguous languages, which has rendered it significantly different from its Indo-Aryan relatives located outside the Western Asian Transition Zone. The Kholosi\il{Indic!Kholosi} case thus has considerable implications for assessing the \isi{role} of contact versus inheritance in predicting \isi{word order}. 


\section*{Abbreviations}
\begin{tabularx}{.45\textwidth}{lQ}
\textsc{adjv} & adjectivizer \\
\textsc{cmp} & complementizer \\
\textsc{cop} & {copula} \\
\textsc{dist} & distal \\
\textsc{emph} & {emphasis} \\
\textsc{ez} & ezafe particle \\
\textsc{f} & feminine \\
\textsc{gen} & genitive \\
IA & Indo-Aryan \\
\textsc{ipfv} & imperfective \\
\textsc{indv} & individuation {clitic} \\
\textsc{m} & masculine \\
\textsc{obl} & {oblique} case \\
\end{tabularx}
\begin{tabularx}{.48\textwidth}{lQ}
\textsc{pc} & person-marking enclitic \\
\textsc{pl} & plural \\
\textsc{prev} & preverb\\
\textsc{prox} & proximal deixis \\
\textsc{prs} & present\\
\textsc{pst} & past \\
\textsc{refl} & reflexive {pronoun} \\
\textsc{sbjv} & subjunctive \\
UP & Unpublished text \\
WATZ & Western Asian Transition Zone\\
WOWA & Word Order in {\hspace{.5cm}} Western Asia \\
\end{tabularx}

\section*{Acknowledgements}

I would like to thank the Kholosi\il{Indic!Kholosi} speakers, in particular Hawa Sabui, Mosaeeb, Said and Yusef Sabui for double checking the data, two reviewers for comments on earlier versions, and Geoffrey Haig for discussion and feedback throughout. The responsibility for all remaining errors is of course my own. 

\sloppy
\printbibliography[heading=subbibliography,notkeyword=this]

\end{document}
