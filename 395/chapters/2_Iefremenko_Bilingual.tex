\documentclass[output=paper,colorlinks,citecolor=brown]{langscibook}
\ChapterDOI{10.5281/zenodo.14266333}
\author{Kateryna Iefremenko\orcid{0000-0003-3711-0935}\affiliation{Leibniz-Centre General Linguistics; University of Potsdam}}
\title{Word order in the speech of Kurmanji-Turkish bilinguals}
\abstract{The paper investigates word order, particularly the domain of post-predicate position, in Turkish and Kurmanji as two languages located in the Western Asian Transition Zone that are in an intense and long-term contact with each other. Both languages are OV; however, each of them allows placement of constituents in post-predicate position. The results of the analysis show that there is an effect of flagging and semantic role in Kurmanji, which is in line with previous research on word order in Kurmanji, and an effect of weight in Turkish on the employment of post-predicate elements. At the same time, the qualitative analysis showed that there are instances of noncanonical placement of case-flagged goals in Kurmanji that occur in one particular construction \textit{erdê ketin} `to fall on the ground.'}

\IfFileExists{../localcommands.tex}{
   \addbibresource{../localbibliography.bib}
   \addbibresource{../collection_tmp.bib}
   \bibliography{../localbibliography}
   \usepackage{langsci-optional}
\usepackage{langsci-gb4e}
\usepackage{langsci-lgr}

\usepackage{listings}
\lstset{basicstyle=\ttfamily,tabsize=2,breaklines=true}

%added by author
% \usepackage{tipa}
\usepackage{multirow}
\graphicspath{{figures/}}
\usepackage{langsci-branding}

   
\newcommand{\sent}{\enumsentence}
\newcommand{\sents}{\eenumsentence}
\let\citeasnoun\citet

\renewcommand{\lsCoverTitleFont}[1]{\sffamily\addfontfeatures{Scale=MatchUppercase}\fontsize{44pt}{16mm}\selectfont #1}
  
   %% hyphenation points for line breaks
%% Normally, automatic hyphenation in LaTeX is very good
%% If a word is mis-hyphenated, add it to this file
%%
%% add information to TeX file before \begin{document} with:
%% %% hyphenation points for line breaks
%% Normally, automatic hyphenation in LaTeX is very good
%% If a word is mis-hyphenated, add it to this file
%%
%% add information to TeX file before \begin{document} with:
%% %% hyphenation points for line breaks
%% Normally, automatic hyphenation in LaTeX is very good
%% If a word is mis-hyphenated, add it to this file
%%
%% add information to TeX file before \begin{document} with:
%% \include{localhyphenation}
\hyphenation{
affri-ca-te
affri-ca-tes
an-no-tated
com-ple-ments
com-po-si-tio-na-li-ty
non-com-po-si-tio-na-li-ty
Gon-zá-lez
out-side
Ri-chárd
se-man-tics
STREU-SLE
Tie-de-mann
}
\hyphenation{
affri-ca-te
affri-ca-tes
an-no-tated
com-ple-ments
com-po-si-tio-na-li-ty
non-com-po-si-tio-na-li-ty
Gon-zá-lez
out-side
Ri-chárd
se-man-tics
STREU-SLE
Tie-de-mann
}
\hyphenation{
affri-ca-te
affri-ca-tes
an-no-tated
com-ple-ments
com-po-si-tio-na-li-ty
non-com-po-si-tio-na-li-ty
Gon-zá-lez
out-side
Ri-chárd
se-man-tics
STREU-SLE
Tie-de-mann
}
%    \boolfalse{bookcompile}
%    \togglepaper[1]%%chapternumber
}{}
\begin{document}
\maketitle\label{WOWA:ch:2}


\section{Introduction}\label{Bilingual:ss:1}

In this paper, I investigate a possible structural \isi{convergence} in the \isi{word order}, namely the post-predicate domain, of Turkish and Kurmanji\il{Kurdish (Northern)} as two languages located in the Western Asian Transition Zone (for a definition, see \citetv{chapters/1_Haigetal_Intro}). Both languages are \isi{OV} languages, but each of them employs the post-predicate position to some extent. In Turkish, \isi{word order} is determined by information structural requirements, and thus the post-predicate position is reserved for background information. In Kurmanji\il{Kurdish (Northern)}, the post-predicate position is driven by verb semantics, i.e., \isi{Goal} arguments are placed in the post-predicate position.

\subsection{Factors of language change}\label{Bilingual:ss:1.1}

\largerpage

\begin{sloppypar}
In situations of \isi{language contact}, it is typical for languages to influence each other. According to \citet{thomason2001language}, linguistic factors of contact-induced language change determine what will change once social factors have created the situation where something will change. However, linguistic factors can be overridden by social factors, and therefore \citet{thomason1988language} define the intensity of contact\is{language contact!intensity of contact} as the most important social factor for the prediction of a contact-induced language change. \citet{thomason2001language} provides a four-stage borrowing scale depending on the intensity of contact\is{language contact!intensity of contact}. In casual contact, when borrowers are not fluent in the source language, only content words are borrowed. In a slightly more intense contact setting when borrowers are fluent bilinguals but form a minority in the community, function words as well as content words can be borrowed at the lexical level, and some minor structural borrowing is possible. In a moderately intense contact setting when there are more borrowers than in the previous stage and social factors favor borrowing, more change can be expected on the structural level; \isi{word order} features, clause-combining strategies, or inflectional affixes might be borrowed. And finally, in an intense contact setting when there is extensive \isi{bilingualism} among the speakers and social factors strongly favor borrowing, anything can be borrowed, resulting even in major typological change of the borrowing language. 
\end{sloppypar}

The intensity of contact\is{language contact!intensity of contact} is determined by the duration of the contact and the number of speakers in the community. The contact between Turkish\il{Turkic!Turkish} and Kurmanji\il{Kurdish (Northern)} has begun in the Ottoman Empire and intensified with the establishment of the Turkish Republic where Turkish\il{Turkic!Turkish} became the country's sole official language (\citealt{yagmur2001languages}). Nowadays, Turkish\il{Turkic!Turkish} is dominant and even the only language in official spheres as education, government, business, while Kurmanji\il{Kurdish (Northern)} is used mostly in the families. As for the number of speakers in the community, the size of the Kurmanji\il{Kurdish (Northern)}-speaking communities varies depending on the region: in the eastern and south-eastern regions of Turkey, Kurmanji\il{Kurdish (Northern)} is spoken by the majority of the population, while in the western regions, Turkish prevails even in informal contexts. Thus, on the borrowability scale, the contact between Kurmanji\il{Kurdish (Northern)} and Turkish in Turkey can be referred to as stage 3 (a setting of moderately intense contact) where structural changes, including \isi{word order} alternations, can be expected. 

Another social factor, which is not provided in \citet{thomason1988language} as one of the most important factors for predicting the outcome of the contact, but which I find relevant for my particular research is the \isi{societal status} of a language: minority and majority language. It is believed that typically a more prestigious donor language influences a less prestigious \isi{recipient} language (\citealt{johanson2002structural}). As has been noted above, Turkish in Turkey is the dominant language in such spheres of life as education, business, media, and in the western regions of Turkey it prevails also in the informal settings. Thus, Turkish is clearly the majority language of the society. As for Kurmanji\il{Kurdish (Northern)} in Turkey, its \isi{societal status} heavily depends on the region: while in the eastern and south-eastern regions Kurmanji\il{Kurdish (Northern)} is the dominant language of the majority of speakers (though only in informal settings), in the western regions, it is the minority language. For my study, the data from Kurmanji-Turkish bilinguals in Turkey were collected in Ankara, the capital of Turkey, where Turkish is the dominant language, and Kurmanji\il{Kurdish (Northern)} is a minority language in this context. 

\subsection{Word order in language contact situations}\label{Bilingual:ss:1.2}

A number of studies (\citealt{thomason1988language}; \citealt{thomason2001language}; \citealt{Heine2008contact}) pointed that \isi{word order} is prone to change in \isi{language contact} scenarios. Following this, there are studies that exemplify this claim for Turkic languages. For example, the \isi{word order} in Karaim changed from \isi{OV} to \isi{VO} due to the contact with Slavic and Baltic languages (\citealt{csato2002karaim}). Similar to Karaim, Gagauz, which has stayed in a long term contact with Slavic languages, underwent change that resulted in a \isi{VO} order becoming dominant (\citealt{menz1999gagausische}). Another study by \citet{keskin2023directionality} investigated \isi{word order} across numerous Turkic varieties in the Balkans where Turkish has been in contact mostly with Indo-European languages for centuries. Analysis of the post-predicate domain in these Turkic varieties showed that the further the Turkic variety is located from the borders of Turkey, the higher is the frequency of the VX order in this variety. On the other hand, studies that investigated comparatively recent contact of Turkish with Indo-European languages, i.e., Turkish as a heritage language in the Netherlands, Germany and the U.S., did not find a shift from \isi{OV} to \isi{VO} order in these varieties (\citealt{dogruoz2007postverbal}; \citealt{schroeder2023postverbal}). Finally, as for research on \isi{word order} in Turkish in contact with another language in a context where Turkish is the majority language of the society, there are no studies, at least to my knowledge, apart from those done by our research group based on the same data set as the one presented in this paper (\citealt{iefremenko2023postpredicate}; \citealt{iefremenko2023wordorder}). 

With respect to \isi{word order} in Kurmanji\il{Kurdish (Northern)} in Turkey, there are studies that investigate the post-predicate position across different dialects of Kurmanji\il{Kurdish (Northern)} (\citealt{haig_verb-goal_2015}; \citealt{haig2018northern}; \citealt{gundogdu2019asymmetries}), but the \isi{focus} is more on \isi{word order} typology and its diachronic change due to contact with other languages. Besides, \citet{asadpour_typologizing_2022} investigated the placement of goals in Mukri Kurdish\il{Kurdish (Central)!Mukri} spoken in Iran and in the contact languages (such as Armenian\il{Armenian}, Azeri\il{Turkic!Azeri} Turkic and Northeastern Neo-Aramaic\il{Neo-Aramaic (NENA)}) and found that in fact the combination of several factors such as \isi{information structure}, semantics and morphosyntax explains the placement of constituents in the post-predicate position in the analyzed languages. 

Thus, this study will try to fill a gap first by investigating possible changes in the \isi{word order} in Turkish in contact with another language in a context where Turkish is the majority language of the society; second, by investigating \isi{word order}, namely the post-predicate position, in Kurmanji\il{Kurdish (Northern)}, that is in intense contact with Turkish\il{Turkic!Turkish} and is the minority language of the society. The analysis is based on the data that come from 30 Kurmanji-Turkish adult bilingual\is{bilingualism} speakers residing in the Turkish-dominant region in Turkey \citep{iefremenko2021KurdishAnkara,iefremenko2021oghuz}.

\section{Word order in Kurmanji\il{Kurdish (Northern)}}\label{bilingual:ss:2}

Kurdish is a macro-language that consists of a continuum of closely related languages spoken by Kurds over a large geographic area spanned across several countries, such as Turkey, Iran, Iraq, Syria, among others (\citealt{sheyholislami2015language}). Northern Kurmanji\il{Kurdish (Northern)} is one of the Kurdish languages, predominantly spoken in southeast Turkey, northwest and northeast Iran, northern Iraq, and northern Syria. It is classified as a member of the northwest Iranian branch of the west Iranian languages, within the Iranian branch of Indo-European language family (\citealt{haigMatras2002kurdish}). 

Like other West Iranian languages, Kurmanji\il{Kurdish (Northern)} is an \isi{OV} language, although it is not always verb-final (\citealt{haig_verb-goal_2015}). Kurmanji\il{Kurdish (Northern)} indeed systematically places certain elements after the verb. According to \citet{haig_post-predicate_2014}, the post-predicate position is reserved for ``goals”, where it is a cover term for:

\begin{enumerate}[label=\alph*)]
\item locational goals of verbs of motion (e.g., \textit{go, run, fall}) and caused motion (e.g., \textit{put, place, take})

\ea\label{Bilingual:ex:1}
Northern Kurdish Yavuzeli \il{Kurdish (Northern)!Yavuzeli}\citep[K002]{matras2016dialects} \\
\gll ez çû-m-e mal-ê \\
\textsc{1sg} go\textsc{.pst-1sg-drct} house\textsc{-obl.f} \\
\glt `I went home.'
\z

\ea\label{Bilingual:ex:2} 
Karakoçan \il{Kurdish (Northern)!Karakoçan}\citep[K075]{matras2016dialects} \\
\gll jinik qutîk-ek anî mal-ê \\
woman box\textsc{-indef} bring\textsc{.pst.3sg} house\textsc{-obl.f} \\
\glt `The woman brought a box into the house.' 
\z

\item recipients of verbs of \isi{transfer} (e.g., \textit{give})
\ea\label{Bilingual:ex:3}
Northern Kurdish Elbistan \il{Kurdish (Northern)!Elbistan}\citep[K022]{matras2016dialects} \\
\gll we ew ne-dê mi \\
\textsc{2pl} this \textsc{neg}-give\textsc{.pst.3sg} \textsc{1sg.obl} \\
\glt `You didn't give it to me.'
\z

\item addressees of verbs of speech (e.g., \textit{say, speak, promise})
\ea\label{Bilingual:ex:4} 
Northern Kurdish Siirt \il{Kurdish (Northern)!Siirt}\citep[K008]{matras2016dialects} \\
\gll min got-e wî \\
\textsc{1sg.obl} say\textsc{.pst.3sg-drct} \textsc{3sg.obl} \\
\glt `I said it to him.'
\z

\end{enumerate}

In other words, the \isi{word order} of Kurmanji\il{Kurdish (Northern)} is not a pure \isi{OV}, but rather OVX. Goals are systematically yet not consistently placed in the post-predicate position in Kurmanji\il{Kurdish (Northern)}. \citet{Haig2022PostPredicateCon} states that the position of post-predicate elements in Kurmanji\il{Kurdish (Northern)} is syntactically fixed and is not the result of pragmatically driven scrambling or stylistic variation, i.e., factors that account for example for the post-predicate position in Turkish\il{Turkic!Turkish}. Factors that influence the position of goals are \isi{flagging} and regional variation (\citealt{haig_verb-goal_2015}; \citealt{gundogdu2019asymmetries}). On the other hand, a recent study by \citet{asadpour_typologizing_2022} on Mukri Kurdish\il{Kurdish (Central)!Mukri} spoken in Iran showed that \isi{information structure} also plays a \isi{role} in the placement of \isi{Goal} arguments. Namely, accessible inferable information occurs in the post-predicate position, while topicalized goals are placed in the preverbal position. Nonetheless, considering that Kurdish varieties are spoken across several countries and as a result come into contact with different languages, the variations observed in the studies could potentially stem from distinctions between these varieties.

I will now elaborate on the relation between the \isi{word order} in Kurmanji\il{Kurdish (Northern)} and the type of \isi{flagging} of \isi{Goal} arguments. Goals in Kurmanji\il{Kurdish (Northern)} can be flagged by case (as demonstrated in \ref{Bilingual:ex:5}) or by an \isi{adposition} (see examples \ref{Bilingual:ex:6}–\ref{Bilingual:ex:8}). 

\ea\label{Bilingual:ex:5}
Northern Kurdish Tatvan \il{Kurdish (Northern)!Tatvan}\citep[K024]{matras2016dialects} \\
\gll min xarin anî od-ê \\
\textsc{1sg.obl} food bring\textsc{.pst.3sg} room\textsc{-obl.f} \\
\glt `I brought the food to the room.' 
\z

Case-flagged \isi{Goal} arguments are always placed right after the predicate and cannot be separated by an \isi{adverb} or any other \isi{argument}. Importantly, the \isi{Goal} \isi{argument} cannot be placed immediately before the verb. As \citet[110]{gundogdu2019asymmetries} emphasizes, ``in a Kurmanji\il{Kurdish (Northern)} clause, at most two case-flagged NPs (subject and \isi{direct object}) are licensed in the preverbal position”. Thus, example \ref{Bilingual:ex:5a} below where the \isi{argument} \textit{odê} (to the room) is placed in the immediate pre-predicate position would be considered noncanonical. 

\ea\label{Bilingual:ex:5a}
Northern Kurdish Tatvan \il{Kurdish (Northern)!Tatvan}(constructed example) \\
\gll min xarin od-ê anî \\
\textsc{1sg.obl} food room\textsc{-obl} bring\textsc{.pst.3sg} \\
\glt `I brought the food to the room.'
\z

The second means of marking goals is by the help of an \isi{adposition}. In Kurmanji\il{Kurdish (Northern)}, there are several types of adpositions, namely basic prepositions, locational nouns which can be used together with a \isi{preposition} (see example \ref{Bilingual:ex:6}), postpositions, and circumpositions. 

\ea\label{Bilingual:ex:6}
Northern Kurdish Pertek \il{Kurdish (Northern)!Pertek}\citep[K028]{matras2016dialects} \\
\gll lawik-ê qiçik di-bez-e ber\_bi dî-ya xwe \\
boy\textsc{-ez.m} little \textsc{prs-}run\textsc{.3sg-drct} towards mother\textsc{-ez.f} own \\
\glt `The little boy is running to his mother.'
\z

In general, the position of \isi{Goal} arguments flagged by adpositions is more flexible compared to those flagged by case. The only exception are locational nouns that are not preceded by a \isi{preposition}: they are always placed in post-predicate position. The reason for such position is that this type of adpositions historically evolved from nouns. Hence, similar to case-flagged goals, locational nouns are placed in the post-predicate position (\citealt{haig_verb-goal_2015}; \citealt{haig_post-predicate_2014}). As for the position of the other types of adpositions, it is largely dependent on dialect. As it is noted in \citet{haig_verb-goal_2015} and \citet{haig_post-predicate_2014}, initially the OVX \isi{word order} of Kurmanji\il{Kurdish (Northern)} emerged due to the contact with early Aramaic/Iranian languages. As a result, there are different preferences in placing goals across the modern dialects of Kurmanji\il{Kurdish (Northern)}: namely, goals are predominantly post-predicative in the southernmost dialects where the language stayed in a long-lasting contact with Neo-Aramaic, while in the northern and western dialects goals are overwhelmingly pre-predicative due to their extensive contact with Armenian\il{Armenian} and Turkish\il{Turkic!Turkish}. Besides, in the south, there is a tendency for post-predicative goals to be accompanied by a \isi{preposition} or a circumposition. At the same time, in the dialects of Central Anatolia, the combination of a post-predicative \isi{Goal} accompanied by a \isi{preposition} is very restricted. 

\section{Word order in Turkish\il{Turkic!Turkish}}\label{Bilingual:ss:3}

Turkish\il{Turkic!Turkish} is considered to have a relatively free \isi{word order}, with basic \isi{word order} being (S)\isi{OV}, which means that even though \isi{word order} variation is possible, in some instances \isi{word order} must stay fixed. In Turkish\il{Turkic!Turkish}, variation in \isi{word order} serves pragmatic purposes such as signaling topics and distinguishing between old and new information (\citealt{erguvanli1984function}). Hence, \isi{word order} in Turkish\il{Turkic!Turkish} is strongly motivated by \isi{information structure}: a link to the previous context or topicalized information appears sentence-initially, new information occurs immediately before the verb, and backgrounded information can be placed post-predicatively (\citealt{erguvanli1984function}; \citealt{kornfilt1997turkish}). Thus, (\ref{Bilingual:ex:7a}) has pragmatically neutral unmarked order, whereas (\ref{Bilingual:ex:7b}–\ref{Bilingual:ex:7f}) are pragmatically marked. 

\ea
\ea\label{Bilingual:ex:7a} 
Turkish \il{Turkic!Turkish}(self-constructed examples) \\
S\isi{OV} \\
\gll Murat araba-yı sat-tı. \\ 
Murat car\textsc{-acc} sell\textsc{-pst.3sg} \\
\glt `Murat sold the car.'
\ex\label{Bilingual:ex:7b}
OSV \\
\gll Araba-yı Murat sat-tı. \\
car\textsc{-acc} Murat sell\textsc{-pst.3sg} \\
\glt `It is Murat who sold the car.'
\ex\label{Bilingual:ex:7c} 
\isi{SVO} \\
\gll Murat sat-tı araba-yı. \\
Murat sell\textsc{-pst.3sg} car\textsc{-acc} \\
\glt `Murat sold the car.' (\isi{emphasis} on the verb)
\ex\label{Bilingual:ex:7d} 
OVS \\
\gll Araba-yı sat-tı Murat. \\
car\textsc{-acc} sell\textsc{-pst.3sg} Murat \\
\glt `It is the car that Murat sold.'
\ex\label{Bilingual:ex:7e} 
\isi{VSO} \\
\gll Sat-tı Murat araba-yı. \\
sell\textsc{-pst.3sg} Murat car\textsc{-acc} \\
\glt `It (the car) was sold by Murat.'
\ex\label{Bilingual:ex:7f} 
VOS\\
\gll Sat-tı araba-yı Murat. \\
sell\textsc{-pst.3sg} car\textsc{-acc} Murat \\
\glt `The car was sold (by Murat).'
\z
\z


Furthermore, it is important to note that there are differences in the employment of post-predicate structures in spoken and written modes as well as formal and informal language. Particularly, there seem to be restrictions on the use of post-predicate structures in formal written situations. For example, in legal documents as well as news items (both written in newspapers and journals as well as read on radio or TV), post-predicate structures are extremely rare (\citealt{erguvanli1984function}: 67). Exceptions are columns in papers when authors deliberately choose more informal style, or literature pieces where an author uses their own style and may employ backgrounding techniques for specific pragmatic reasons (\citealt{erguvanli1984function}: 67). On the other hand, elements placed in post-predicate position frequently occur in informal spoken mode, which is usually characterized as spontaneous and unplanned (\citealt{schroeder1995postpredicate}). Thus, spoken language is full of (self-)corrections and afterthoughts, while the same process in written language can be employed with the help of editing and corrections. Another reason of a higher number of post-predicate structures in spoken language compared to written one is that utterances are limited in size due to speaker's awareness of the listener's capacity limitations (\citealt{chafe1985linguistic}). Thus, as \citet[206]{schroeder1995postpredicate} emphasizes, the employment of the post-predicate position helps the listener to keep track of the topical development and the deictic framework in which the predication holds. 

\section{Research questions}\label{bilingual:ss:4}
\largerpage
To summarize the preceding discussion, both languages in contact that are under research here have \isi{OV} \isi{word order}, but both employ the post-predicate position in a different way. While Kurmanji\il{Kurdish (Northern)} systematically places \isi{Goal} arguments --- particularly those flagged with case --- in the post-predicate position, \isi{word order} in Turkish\il{Turkic!Turkish} is determined by information structural requirements, and the post-predicate position is reserved for background information, regardless of the semantic \isi{role} of the elements. 

As discussed in \sectref{Bilingual:ss:1}, the contact between the two languages has lasted for centuries and is fairly intense, and according to \citegen{thomason1988language} borrowability scale, we can expect structural changes, including the ones concerning \isi{word order}. In the paper, I will investigate potential convergences in \isi{word order}, in particular in the post-predicate domain, in Turkish\il{Turkic!Turkish} and Kurmanji\il{Kurdish (Northern)} in Turkey, basing the analysis on the variables encoded in WOWA (see \citetv{chapters/1_Haigetal_Intro}). For the reason that WOWA categories do not incorporate \isi{information structure} as one of the variables, I do not explore its effect on the \isi{word order} changes in the investigated languages. This is a limitation of the current paper.

\section{Methodology}\label{bilingual:ss:5}

\subsection{Participants}\label{bilingual:ss:5.1}

The data for this study come from 30 Kurmanji-Turkish bilinguals (9 females and 21 males). The participants were exposed to Kurmanji\il{Kurdish (Northern)} from birth or an early age in their family and started acquiring Turkish\il{Turkic!Turkish} mainly when they entered school (though some participants were already exposed to Turkish\il{Turkic!Turkish} in their families). The place of birth of the participants varies: the majority of the speakers were born and raised in the east and the south-east of Turkey, but some were born and grew up in the western cities of Turkey. At the time of data collection, all participants were living in Ankara, where the dominant language of the society is Turkish\il{Turkic!Turkish}. In an urban city, such as Ankara, speakers of different dialects interact between each other on a daily basis; hence, such contact may lead to dialect levelling. Therefore, my data were not controlled for the dialect of Kurmanji\il{Kurdish (Northern)}. The age of the participants varies between 23 and 37 years, with the mean age being 28.1; thus, all the participants are young adults. All 30 participants had a high school degree, and most of them completed bachelor's or master's degree. So, all the speakers are highly proficient in Turkish\il{Turkic!Turkish} since it is the language of education in Turkey. As for the education in Kurmanji\il{Kurdish (Northern)}, 14 participants stated that they had taken at least a several-month language course in Kurmanji\il{Kurdish (Northern)}, one participant had earned a master's degree from a Kurmanji\il{Kurdish (Northern)} language and literature department, and several participants indicated that they actively used Kurmanji\il{Kurdish (Northern)} for business purposes. However, the results of the section on the self-rated proficiency in the questionnaire showed that on average the speakers rate their proficiency in Kurmanji\il{Kurdish (Northern)} lower (x̄ = 17.53 out of 20 possible) compared to their proficiency in Turkish\il{Turkic!Turkish} (x̄ = 19.32 out of 20 possible). 

\subsection{Data collection}\label{bilingual:ss:5.2}

\begin{sloppypar}
The data used for this study were collected with the help of the ``Language Situations'' method (\citealt{wiese2020language}). This method combines controlled elicitation with spontaneous data, and thus is suitable for systematic comparisons across contact-linguistic constellations as well as different languages. It captures quasi-naturalistic productions across different communicative situations, including informal versus formal, and written versus spoken communicative situations. The elicitation comprised two sessions (one in Kurmanji\il{Kurdish (Northern)} and one in Turkish\il{Turkic!Turkish}), with at least three days between the sessions. Participants were shown a short video of a car accident and were asked to imagine themselves having witnessed it. After that they were asked to recount the incident in four different imagined situations: to a friend via a WhatsApp voice message (informal spoken), to a friend via a WhatsApp text message (informal written), to the police via a voice mail (formal spoken), to the police in a form of a written witness report (formal written). To exclude a possible effect of priming, the order of the communicative situations and the order of the languages were balanced. 
\end{sloppypar}

At the end of the second session, each participant was asked to fill out an extensive questionnaire. The questionnaire comprised nine sections: participants' general information, educational and professional background of the participants, family information, information about the languages, self-assessment of their language skills (in Kurmanji\il{Kurdish (Northern)}, Turkish\il{Turkic!Turkish}, as well as in foreign languages, on a five-point scale), participants' language use with family members and peers, a section concerning media use and free time (texting WhatsApp messages or writing emails in Kurmanji\il{Kurdish (Northern)}, Turkish\il{Turkic!Turkish}, as well as in foreign languages, three scores of frequency), questions concerning personal character traits, and feedback on the participation in the study. The questionnaire was available in Turkish\il{Turkic!Turkish} as it is the language for official documents in Turkey. It was always filled out by the participants themselves in the presence of an elicitor.

\subsection{Annotation and statistical analysis}\label{bilingual:ss:5.3}

For the current study on the post-predicate elements in Iranian and neighboring languages, only the informal spoken productions of the speakers were analyzed.\footnote{(In-)formality and mode as factors that might influence the \isi{word order} in Turkish\il{Turkic!Turkish} and Kurmanji\il{Kurdish (Northern)} are analyzed and discussed in another study by \citet{iefremenko2023wordorder}.} The Kurmanji\il{Kurdish (Northern)} data comprised 745 communication units\footnote{A communication unit is an independent clause with its modifiers and dependent (subordinate) clauses.} , out of which 507 were analyzed because they contained pre- or post-predicate arguments (\citealt{iefremenko2021KurdishAnkara}). The Turkish\il{Turkic!Turkish} data consisted of 799 communication units, out of which 587 were analyzed (\citealt{iefremenko2021oghuz}). Data coding was done in line with the WOWA coding method (\citetv{chapters/1_Haigetal_Intro}) and annotated for \isi{animacy}, \isi{weight}, semantic \isi{role}, \isi{flagging}, and position. What is important for the current analysis is that afterthoughts and self-repairings, which usually occur in the post-predicate position in \isi{OV} languages, were not excluded from the analysis. Besides, there are several aspects of coding that are important for the analysis of Turkish\il{Turkic!Turkish} and Kurmanji\il{Kurdish (Northern)} data. First, adpositions are not counted as part of the \isi{weight} because they are considered to be part of the \isi{flagging}. Thus, for instance, in the example (\ref{Bilingual:ex:8}) the token \textit{rê da} (on the street) is considered to have \isi{weight} 1 because the postposition \textit{da} is not counted.

\ea\label{Bilingual:ex:8}
Northern Kurdish Ankara \il{Kurdish (Northern)!Ankara}\citep[A, 3]{iefremenko2021KurdishAnkara} \\
\gll iii ez rê da di-çû-m \\
hmm \textsc{1sg} street on \textsc{prog-}go\textsc{.pst-1sg} \\
\glt `I was going on the street.' 
\z

Second, subjects were excluded from the analysis across all the analyzed languages in WOWA. This has to be taken into account when interpreting the results for Turkish\il{Turkic!Turkish} because as indicated in \sectref{Bilingual:ss:3}, unlike Kurmanji\il{Kurdish (Northern)}, the \isi{word order} in Turkish\il{Turkic!Turkish} is determined by \isi{information structure} and as a result, subjects can also be placed in the post-predicate position as long as they are not new information.

The statistical analyses were run in R (\citealt{R_core_team}). In addition to the base package of R, I used tidyverse for data manipulation and visualization (\citealt{wickham2019tidyverse}). I ran binomial generalized linear regression models using the lme4 package (\citealt{bates2015lme4}).

\section{Results}\label{Bilingual:ss:6}

\subsection{Kurmanji\il{Kurdish (Northern)}}\label{Bilingual:ss:6.1}

First, I will start with the analysis of post-predicate constituents in Kurmanji\il{Kurdish (Northern)}.

\figref{Bilingual:fig:1} shows the percentage of tokens in the pre- and post-predicate position in the Kurmanji\il{Kurdish (Northern)} data. The line inside each box indicates the median, while a dot represents the percentage of tokens used in pre- or post-predicate position by one speaker, calculated in relation to the total number of communication units for each speaker. \figref{Bilingual:fig:1} demonstrates that the majority of the communication units have pre-predicate arguments, with around 20\% taking post-predicate arguments. Also, we can see that individual variability in the group is high: some participants place tokens post-predicatively in up to 70\% of their utterances, while in some speakers all utterances are pre-predicate.

\begin{figure}
 \centering
 \includegraphics[width=\linewidth]{figures/Figure 1.png}
 \caption{Percentage of tokens in pre- or post-predicate position in Kurmanji }
 \label{Bilingual:fig:1}
\end{figure}

As stated in \sectref{bilingual:ss:2}, \isi{word order} in Kurmanji\il{Kurdish (Northern)} is determined by verb semantics where goals are placed in post-predicate position. Therefore, high individual variability in the group might be caused by some speakers' frequent use of goals. In order to establish whether there is an effect of one of the variables from the WOWA coding scheme, namely \isi{animacy}, \isi{weight}, \isi{role} or \isi{flagging} (as described in \sectref{bilingual:ss:5.3}), I ran four binomial generalized linear regression models with the dependent variable Position (preverbal, coded as 0 vs. postverbal, coded as 1) and the independent variables Animacy, Weight, Flagging, and Role.

\subsubsection{Animacy} Animacy is a categorical independent variable with seven levels in the Kurmanji\il{Kurdish (Northern)} data. The results of the regression given in \tabref{Bilingual:tab:1} show that there is no correlation between \isi{animacy} and the position of the tokens in the Kurmanji\il{Kurdish (Northern)} data.

\begin{table}
 \centering
 \begin{tabularx}{\textwidth}{XrYYl}
\lsptoprule
\textbf{fixed effect} & \textbf{β} &  \textbf{σ} & \multicolumn{2}{r}{\textbf{p-value}}  \\
\midrule
\textbf{(intercept)} & -1.65e+01&  5.656e+02 & 0.97 &  (ns) \\
\textbf{Anim-adv} & 1.28e+01    &  5.656e+02 & 0.98 &  (ns) \\
\textbf{Anim-anim} & 1.36e+01   &  5.656e+02 & 0.98 &  (ns) \\
\textbf{Anim-bp} & 1.35e+01     &  5.656e+02 & 0.98 &  (ns) \\
\textbf{Anim-hum} & 1.45e+01    &  5.656e+02 & 0.97 &  (ns) \\
\textbf{Anim-inan} & 1.56e+01   &  5.656e+02 & 0.97 &  (ns) \\
\textbf{Anim-other} & -9.53e-09 &  2.465e+03 & 1.00 &  (ns) \\
\lspbottomrule
 \end{tabularx}
 \caption{Regression table for binomial GLM with the dependent variable Position and the independent variable Animacy in Kurmanji.}
 \label{Bilingual:tab:1}
\end{table}

\subsubsection{Weight} Weight is a discrete numeric variable, based on number of phonological words. Four values are distinguished: one to three phonological words, and more than three (which is coded as ``four''). The results of the regression provided in \tabref{Bilingual:tab:2} show that there is no correlation between the \isi{weight} of the constituents and the position of the tokens in relation to the verb.

\begin{table}
 \begin{tabularx}{\textwidth}{XrYYl}
\lsptoprule
\textbf{fixed effect} & \textbf{β} &  \textbf{σ} & \multicolumn{2}{r}{\textbf{p-value}}  \\
\midrule
\textbf{(intercept)} & -1.26 &  0.30 & < .0001 \\
\textbf{Weight} & -0.15 &  0.19 & 0.43 &  (ns) \\
\lspbottomrule
 \end{tabularx}
 \caption{Regression table for binomial GLM with the dependent variable Position and the independent variable Weight in Kurmanji}
 \label{Bilingual:tab:2}
\end{table}

\subsubsection{Flagging} Flagging is a categorical independent variable with eight levels in the Kurmanji\il{Kurdish (Northern)} data. The results of the regression analysis presented in \tabref{Bilingual:tab:3} show that there is an effect of Flagging on the placement of arguments in the pre- or post-predicate position in Kurmanji\il{Kurdish (Northern)}.

\begin{table}
 \begin{tabularx}{\textwidth}{lrYYl}
\lsptoprule
\textbf{fixed effect} & \textbf{β} &  \textbf{σ} & \multicolumn{2}{r}{\textbf{p-value}}  \\
\midrule
\textbf{(intercept)} & -2.67 &  0.36 & < .0001 \\
\textbf{Flag-case} & 2.23 &  0.43 & < .0001 & *** \\
\textbf{Flag-circ} & -0.09 & 0.69 & 0.88 &  (ns) \\
\textbf{Flag-lvc-poss} & -11.89 &  882.74 & 0.98 &  (ns) \\
\textbf{Flag-postp} & 0.84 &  0.57 & 0.11 &  (ns) \\
\textbf{Flag-prep} & -0.06 &  0.51 & 0.89 &  (ns) \\
\textbf{Flag-prep-relnoun} & 1.91 &  0.58 & 0.001 &  ** \\
\textbf{Flag-relnoun} & 3.84 &  0.52 & < .0001 & *** \\
\lspbottomrule
 \end{tabularx}
 \caption{Regression table for binomial GLM with the dependent variable Position and the independent variable Flagging in Kurmanji}
 \label{Bilingual:tab:3}
\end{table}

The model showed that arguments flagged with case (see example \ref{Bilingual:ex:9}), locational nouns (see example \ref{Bilingual:ex:10}), or a \isi{preposition} together with a locational noun (see example \ref{Bilingual:ex:11}) are more likely to be placed in the post-predicate position than in the pre-predicate one.

\ea\label{Bilingual:ex:9}
Northern Kurdish Ankara \il{Kurdish (Northern)!Ankara}\citep[Y, 629]{iefremenko2021KurdishAnkara} \\
\gll kûçik xe avêt-e top-e \\
 dog oneself throw\textsc{.pst.3sg-drct} ball\textsc{-obl.f} \\
\glt `The dog threw itself towards the ball.' 
\z

\newpage
\ea\label{Bilingual:ex:10} 
Northern Kurdish Ankara \il{Kurdish (Northern)!Ankara}\citep[Y, 626]{iefremenko2021KurdishAnkara} \\
\gll çû iii ser rê \\
go\textsc{.pst.3sg} hm to street \\
\glt `He went to the street.' 
\z

\ea\label{Bilingual:ex:11} 
Northern Kurdish Ankara \il{Kurdish (Northern)!Ankara}\citep[C, 57]{iefremenko2021KurdishAnkara} \\
\gll iii di rê da iii jin-ekî kûçik-ê wî ew ê derbas bû-n-a li\_hemberê rê \\
% hm in street \textsc{circ} hm woman\textsc{-indef-ez} dog\textsc{-ez} her it hm passing become\textsc{.pst-3pl-drct} across street \\
hm in street \textsc{circ} hm woman\textsc{-indef} dog\textsc{-ez.m} her they \textsc{fut} passing become\textsc{.pst-3pl-drct} across street \\
% \glt `Hm, on the street a woman, her dog, it crossed the street to the opposite side.' 
\glt `Hm, on the street, (there was) a woman (and) her dog, they were about to cross the street to the opposite side.' 
\z

\subsubsection{Role} Role is a categorical independent variable with 15 levels in the Kurmanji\il{Kurdish (Northern)} data. The results of the regression provided in \tabref{Bilingual:tab:4} show that there is an effect of \isi{role} on the placement of arguments in the post-predicate position.

\begin{table}
 \begin{tabularx}{\textwidth}{XrYYl}
\lsptoprule
\textbf{fixed effect} & \textbf{β} &  \textbf{σ} & \multicolumn{2}{r}{\textbf{p-value}}  \\
\midrule
\textbf{(intercept)} & -2.85 & 0.59 & < .0001 \\
\textbf{Role-addr } & 0.28 &  1.19 & 0.8 &  (ns) \\
\textbf{Role-ben} & 0.9 &    1.22 & 0.45 &  (ns) \\
\textbf{Role-com} & 1.06 &  0.96 & 0.27 &  (ns) \\
\textbf{Role-cop} & -14.71 & 907.61 & 0.98 &  (ns) \\
\textbf{Role-cop-loc} & -14.71 &  3956.1 & 0.99 &  (ns) \\
\textbf{Role-do} & -1.61 &  1.16 & 0.16 &  (ns) \\
\textbf{Role-do-def} & -14.71 &  1978.09 & 0.99 &  (ns) \\
\textbf{Role-goal} & 3.65 &  0.63 & < .0001 & *** \\
\textbf{Role-goal-c} & 2.39 &  0.66 & < .0001 & *** \\
\textbf{Role-instr} & -14.71 &  1318.72 & 0.99 &  (ns) \\
\textbf{Role-loc} & -0.15 &  0.72 & 0.83 &  (ns) \\
\textbf{Role-other} & -0.51 &  1.17 & 0.66 &  (ns) \\
\textbf{Role-rec} & 20.41 &  3956.10 & 0.99 &  (ns) \\
\textbf{Role-stim} & 2.85 &  1.53 & 0.06 &  (ns) \\
\lspbottomrule
 \end{tabularx}
 \caption{Regression table for binomial GLM with the dependent variable Position and the independent variable Role in Kurmanji}
 \label{Bilingual:tab:4}
\end{table}

\begin{figure}
 \centering
 \includegraphics[width=\linewidth]{figures/Figure 2.png}
 \caption{Percentage of tokens with a particular role in the pre- and post-predicate position in Kurmanji.}
 \label{Bilingual:fig:2}
\end{figure}

\figref{Bilingual:fig:2} shows the percentage of tokens with a particular \isi{role} in the pre- and post-predicate position in the Kurmanji\il{Kurdish (Northern)} data, which was calculated in relation to the total number of tokens in the data. The line inside each box indicates the median, while dots represent the percentage of tokens used in pre- or post-predicate position by one speaker.

\largerpage
The model as well as \figref{Bilingual:fig:2} show that goals of motion  (see example \ref{Bilingual:ex:12}) and caused motion  (see example \ref{Bilingual:ex:13}) are more likely to be placed in the post-predicate position than in the pre-predicate one.

\ea\label{Bilingual:ex:12}
Northern Kurdish Ankara \il{Kurdish (Northern)!Ankara}\citep[C, 69]{iefremenko2021KurdishAnkara} \\
\gll çû ber\_bi ereba \\
go\textsc{.pst.3sg} towards car \\
\glt `He went towards the car.' 
\z



\ea\label{Bilingual:ex:13}
Northern Kurdish Ankara \il{Kurdish (Northern)!Ankara}\citep[E, 108]{iefremenko2021KurdishAnkara} \\
\gll û kûçik jî xe awêt-e top-ê \\
and dog also oneself throw\textsc{.pst.3sg-drct} ball\textsc{-obl.f} \\
\glt `And the dog threw itself to the ball.' 
\z

Thus, the analysis demonstrated that in Kurmanji\il{Kurdish (Northern)} goals of motion and caused motion, which are flagged with case, locational noun, or sometimes a \isi{preposition} and a locational noun, are likely to be placed in post-predicate position. This is in fact what is known from previous literature on the post-predicate position in Kurmanji\il{Kurdish (Northern)} (\citealt{haig_verb-goal_2015}; \citealt{haig2018northern}; \citealt{gundogdu2019asymmetries}). 

Regarding the other types of goals, as shown in \figref{Bilingual:fig:2}, there are no examples of addressees placed in post-predicate position in my data, and all examples containing an \isi{addressee} \isi{argument} are placed in pre-predicate position. The reason for this might be \isi{flagging}: all pre-predicate examples are flagged with a circumposition (example \ref{Bilingual:ex:14}) or a postposition (example \ref{Bilingual:ex:15}).



\ea\label{Bilingual:ex:14} 
Northern Kurdish Ankara \il{Kurdish (Northern)!Ankara}\citep[F, 134]{iefremenko2021KurdishAnkara} \\
\gll mi go e ji te ra bi-bêj-im \\
\textsc{1sg.obl} say\textsc{.pst} hm to \textsc{2sg.obl} \textsc{circ} \textsc{subj}-say\textsc{.prs-1sg} \\
\glt `I said, I am telling (this) to you.' 
\z

\ea\label{Bilingual:ex:15}
Northern Kurdish Ankara \il{Kurdish (Northern)!Ankara}\citep[F, 114]{iefremenko2021KurdishAnkara} \\
\gll mi go ez te re bêj-im \\
\textsc{1sg.obl} say\textsc{.pst} \textsc{1sg} \textsc{2sg.obl} \textsc{postp} \textsc{subj}.say\textsc{.prs-1sg} \\
\glt `I said, I am telling (this) to you.' 
\z

With regard to recipients, there are only two examples in the Kurmanji\il{Kurdish (Northern)} data, and they are placed post-predicatively (example \ref{Bilingual:ex:16}). Presumably, the reason for a small number of recipients in the data is the method: the video shown to the participants did not trigger the use of verbs of \isi{transfer}. 

\ea\label{Bilingual:ex:16} 
Northern Kurdish Ankara \il{Kurdish (Northern)!Ankara}\citep[T, 498]{iefremenko2021KurdishAnkara} \\
\gll xeber-ê bi-di-m te \\
news\textsc{-obl} \textsc{subj-}give\textsc{.prs-1sg} \textsc{2sg.obl} \\
\glt `Let me tell (lit. give) you the news.' 
\z

\begin{sloppypar}
Apart from goals of (caused) motion, addressees, and recipients, \figref{Bilingual:fig:2} shows that locations (abbreviated in \figref{Bilingual:fig:2} as ``loc'' – see example \ref{Bilingual:ex:17}) and sources of motion (abbreviated in \figref{Bilingual:fig:2} as ``abl'' – see example \ref{Bilingual:ex:18}) can also be placed in post-predicate position in Kurmanji\il{Kurdish (Northern)}.
\end{sloppypar}

\ea\label{Bilingual:ex:17} 
Northern Kurdish Ankara \il{Kurdish (Northern)!Ankara}\citep[F, 126]{iefremenko2021KurdishAnkara} \\
\gll du erebe piştî hevdu di-çû him iii li\_ser rê da \\
two car after each.other \textsc{prog-}go\textsc{.pst.3sg} both hmm on street \textsc{circ} \\
\glt `Two cars were coming one after another on the street.' 
\z

\ea\label{Bilingual:ex:18} 
Northern Kurdish Ankara \il{Kurdish (Northern)!Ankara}\citep[A, 28]{iefremenko2021KurdishAnkara} \\
\gll e du erebe di-hat-in ji wî alî \\
hm two car \textsc{prog-}come\textsc{.pst-3pl} from his side \\
\glt `Hm, two cars were coming from his side.' 
\z

Another interesting observation is placement of case-flagged goals of motion. As the previous literature (\citealt{haig_verb-goal_2015}; \citealt{haig2018northern}; \citealt{gundogdu2019asymmetries}) shows, goals of motion flagged with case are always placed in immediate post-predicate position. Yet there are examples in my data where case-flagged goals of motion are placed in pre-predicate position (see example \ref{Bilingual:ex:19} and \ref{Bilingual:ex:20}).

\ea\label{Bilingual:ex:19} 
Northern Kurdish Ankara \il{Kurdish (Northern)!Ankara}\citep[N, 373]{iefremenko2021KurdishAnkara} \\
\gll nan û av erd-î ket \\
bread and water ground\textsc{-obl.m} fall\textsc{.pst.3sg} \\
\glt `Bread and water fell on the ground.' 
\z

\ea\label{Bilingual:ex:20} 
Northern Kurdish Ankara \il{Kurdish (Northern)!Ankara}\citep[O, 386]{iefremenko2021KurdishAnkara} \\
\gll pişte zilam iii gok-ê xwe imm (--) ji dest-ê xwa imm (--) erdêk/\_erd-ê ket \\
after man hmm ball\textsc{-ez.f} own hmm ~ from hand\textsc{-ez.m} own hmm ~ ground\textsc{-obl.m} fall\textsc{.pst.3sg} \\
\glt `Afterwards, the man, hm, his ball, hm, fell from his hands on the ground.' 
\z

In total in the data there are four examples with a case-flagged \isi{Goal} of motion placed in pre-predicate position, and all four examples are actually the same construction\footnote{ For a definition of a construction see \citet[257--262]{croft2004cognitive}.} \textit{erdê} \textit{ketin} `to fall on the ground'. In the contact language Turkish\il{Turkic!Turkish}, \textit{erdê ketin} is rendered as \textit{yere düşmek}. In Turkish\il{Turkic!Turkish}, \textit{yere düşmek} is a fixed construction. This is also supported by the Turkish\il{Turkic!Turkish} data from Kurmanji-Turkish bilinguals in Ankara: there are 10 instances of \textit{yere düşmek}, and in all of them the \isi{argument} \textit{yere} `on the ground' is placed in the immediate pre-predicate position. Besides, I have also checked other data collected with the help of the same method, namely Turkish\il{Turkic!Turkish} monolingual speakers from Izmir and Eskişehir (\citealt{wieseetal2021rueg}) as well as Kurmanji-Turkish\il{Turkic!Turkish} heritage speakers in Berlin, and found that in all the instances of \textit{yere düşmek}, the \isi{argument} \textit{yere} was placed in the immediate pre-predicate position. Thus, I argue that the stable position of the \isi{argument} \textit{yere} in Turkish\il{Turkic!Turkish} leads to \isi{transfer} of the whole construction into Kurmanji\il{Kurdish (Northern)} and as a result, the \isi{argument} \textit{erdê} is placed pre-predicatively. Besides, in Kurmanji\il{Kurdish (Northern)} there are other constructions with the verb \textit{ketin}, which are set phrases and where the \isi{Goal} is always placed in pre-predicate position, e.g., \textit{bi rê ketin} `to set off (on a journey)'. This might explain the fact why I do not find examples of \isi{transfer} of constructions with other verbs. 

Thus, the analysis of the post-predicate position in Kurmanji\il{Kurdish (Northern)} of Kurmanji-Turkish bilingual\is{bilingualism} speakers in Ankara showed that Kurmanji\il{Kurdish (Northern)} is an \isi{OV} language where goals of motion and caused motion flagged with case, a locational noun, or a \isi{preposition} used together with a locational noun, are systematically placed in post-predicate position.\footnote{As for addressees and recipients, there is not enough data in the data set to make conclusions about their position in a sentence in relation to the verb.} However, the analysis showed that other elements, such as location and source of motion, can also be placed in post-predicate position in Kurmanji\il{Kurdish (Northern)}. Apart from this, the data demonstrated that some speakers employ the pre-predicate position for case-flagged goals of motion, which is a non-canonical position for such arguments. 

\subsection{Turkish\il{Turkic!Turkish}}\label{Bilingual:ss:6.2}

In this section, I will present the results of the analysis of the majority language of the society the speakers live in – which is Turkish\il{Turkic!Turkish}. 

\figref{Bilingual:fig:3} shows frequency distribution of the coded tokens used in pre- or post-predicate position. The percentage was calculated in relation to the total number of utterances that contain tokens placed pre- or post-predicatively. Each dot represents the mean use of tokens by a speaker. In \figref{Bilingual:fig:3}, we can see that in general the pre-predicate position of arguments is preferred: only 10\% of utterances contain post-predicate elements. However, the figure also shows that there is individual variability: some speakers use up to 25\% of post-predicate structures in their narrations.

\begin{figure}
    \includegraphics[width=\linewidth]{figures/Figure 3.png}
    \caption{Percentage of tokens in pre- or post-predicate position in Turkish\il{Turkic!Turkish}.}
    \label{Bilingual:fig:3}
\end{figure}


Similar to Kurmanji\il{Kurdish (Northern)}, in Turkish\il{Turkic!Turkish}, each token was annotated for \isi{animacy}, \isi{weight}, \isi{role} and \isi{flagging} (see \citetv{chapters/1_Haigetal_Intro}). To find out whether there is an effect of one of these variables, I ran four binomial generalized linear regression models with the dependent variable Position (preverbal, coded as 0 vs. postverbal, coded as 1) and the independent variables Animacy, Weight, Flagging, and Role.

\subsubsection{Animacy} Animacy is a categorical independent variable with six levels in the Turkish\il{Turkic!Turkish} data. The results of the regression given in \tabref{Bilingual:tab:5} show that there is no correlation between Animacy and the Position of the tokens in the Turkish\il{Turkic!Turkish} data.

\begin{table}
 \begin{tabularx}{\textwidth}{XrYYl}
\lsptoprule
\textbf{fixed effect} & \textbf{β} &  \textbf{σ} & \multicolumn{2}{r}{\textbf{p-value}}  \\
\midrule
\textbf{(intercept)} & -1.757e+01 & 2.284e+03 & 0.99 &  (ns) \\
\textbf{Anim-adv} & 1.512e+01 &  2.284e+03 & 0.99 &  (ns) \\
\textbf{Anim-anim} & -5.966e-08 &  2.412e+03 & 1.00 &  (ns) \\
\textbf{Anim-bp} & 1.498e+01 &  2.284e+03 & 0.99 &  (ns) \\
\textbf{Anim-hum} & 1.612e+01 &  2.284e+03 & 0.99 &  (ns) \\
\textbf{Anim-inan} & 1.446e+01 &  2.284e+03 & 0.99 &  (ns) \\
\lspbottomrule
 \end{tabularx}
 \caption{Regression table for binomial GLM with the dependent variable Position and the independent variable Animacy in Turkish}
 \label{Bilingual:tab:5}
\end{table}

\subsubsection{Weight} Weight is a discrete numeric variable, ranging from one to four and more phonological words. The results of the regression provided in \tabref{Bilingual:tab:6} show that there is a positive effect of \isi{weight} on the placement of the elements in post-predicate position, meaning that heavier elements (those that consist of three or four phonological words) are more likely to be placed in post-predicate position than those with the \isi{weight} of one or two phonological words.

\begin{table}
 \begin{tabularx}{\textwidth}{XrYYl}
\lsptoprule
\textbf{fixed effect} & \textbf{β} &  \textbf{σ} & \multicolumn{2}{r}{\textbf{p-value}}  \\
\midrule
\textbf{(intercept)} & -3.32 &  0.35 & < .0001 \\
\textbf{Weight} & \textbf{0.5} &  \textbf{0.19} & \textbf{0.008} & \textbf{**} \\
\lspbottomrule
 \end{tabularx}
 \caption{Regression table for binomial GLM with the dependent variable Position and the independent variable Weight in Turkish}
 \label{Bilingual:tab:6}
\end{table}

Thus, for instance, in example (\ref{Bilingual:ex:21}) the structure \textit{yolun diğer tarafından} `from the other side of the street' consists of three phonological words and is placed in post-predicate position. 

\ea\label{Bilingual:ex:21} 
Turkish \il{Turkic!Turkish}\citep[U, 622] {iefremenko2021oghuz} \\
\gll bi tane araba gel-iyor-du yol-un diğer taraf-ın-dan \\
one piece car come\textsc{-prog-pst.3sg} road\textsc{-gen} other side\textsc{-poss-abl} \\
\glt `There was a car coming from the other side of the street.' 
\z

\subsubsection{Flagging} Flagging is a categorical independent variable with four levels in the Turkish\il{Turkic!Turkish} data. The result of the regression analysis presented in \tabref{Bilingual:tab:7} shows that there is no effect of \isi{flagging} on the placement of arguments in pre- or post-predicate position in Turkish\il{Turkic!Turkish}.

\begin{table}
 \begin{tabularx}{\textwidth}{lrYYl}
\lsptoprule
\textbf{fixed effect} & \textbf{β} &  \textbf{σ} & \multicolumn{2}{r}{\textbf{p-value}}  \\
\midrule
\textbf{(intercept)} & -2.25 & 0.37 & < .0001 \\
\textbf{Flag-case} & -0.38 &  0.41 & 0.35 &  (ns) \\
\textbf{Flag-postp} & 0.05 &  0.71 & 0.94 &  (ns) \\
\textbf{Flag-postp-relnoun} & -0.5 &  0.81 & 0.47 & (ns) \\
\lspbottomrule
 \end{tabularx}
 \caption{Regression table for binomial GLM with the dependent variable Position and the independent variable Flagging in Turkish}
 \label{Bilingual:tab:7}
\end{table}

\subsubsection{Role} Role is a categorical independent variable with 16 levels in the Turkish\il{Turkic!Turkish} data. The results of the regression provided in \tabref{Bilingual:tab:8} show that there is no correlation between the Role of a token and its Position in relation to the verb.

\begin{table}
 \begin{tabularx}{\textwidth}{XrYYl}
\lsptoprule
\textbf{fixed effect} & \textbf{β} &  \textbf{σ} & \multicolumn{2}{r}{\textbf{p-value}}  \\
\midrule
\textbf{(intercept)} & -2.68 & 0.46 & < .0001 \\
\textbf{Role-addr } & 1.38 &  0.79 & 0.08 &  (ns) \\
\textbf{Role-ben} & 0.93 &  0.71 & 0.19 &  (ns) \\
\textbf{Role-com} & 1.29 &  0.79 & 0.10 &  (ns) \\
\textbf{Role-cop} & -15.88 &  3261.3 & 0.99 &  (ns) \\
\textbf{Role-cop-loc} & -15.88 &  4612.2 & 0.99 &  (ns) \\
\textbf{Role-do} & -0.67 &  0.74 & 0.36 &  (ns) \\
\textbf{Role-do-def} & -15.88 &  1171.5 & 0.98 &  (ns) \\
\textbf{Role-goal} & 0.16 &  0.56 & 0.77 &  (ns) \\
\textbf{Role-goal-c} & 0.19 &  1.13 & 0.86 &  (ns) \\
\textbf{Role-instr} & -15.88 &  1966.64 & 0.99 &  (ns) \\
\textbf{Role-loc} & -0.3 &  0.65 & 0.63 &  (ns) \\
\textbf{Role-other} & 0.65 &  0.61 & 0.28 &  (ns) \\
\textbf{Role-poss} & -15.88 &  6522.6 & 0.99 &  (ns) \\
\textbf{Role-rec} & -15.88 &  4621.2 & 0.99 &  (ns) \\
\textbf{Role-stim} & 21.24 &  6522.6 & 0.99 &  (ns) \\
\lspbottomrule
 \end{tabularx}
 \caption{Regression table for binomial GLM with the dependent variable Position and the independent variable Role in Turkish}
 \label{Bilingual:tab:8}
\end{table}

Thus, the analysis of \isi{word order} in Turkish\il{Turkic!Turkish} of Kurmanji-Turkish bilingual\is{bilingualism} speakers showed that the only significant predictor is the \isi{weight} of the elements: those constituents consisting of three and four phonological words are more likely to be placed in post-predicate position. Unlike Kurmanji\il{Kurdish (Northern)}, in Turkish\il{Turkic!Turkish} such variables as Role or Flagging do not have an effect on the placement of arguments in post-predicate position in my data.

Before proceeding to the discussion of results, I would like to briefly discuss the findings from the Turkish\il{Turkic!Turkish} data from Erzurum (and Erzurum Province) (\citealt{dogan_oghuz_2021}) that were analysed using the same scheme of the WOWA project. From the first look at the normalized numbers of different roles of the tokens, it is evident that the data sets from Ankara and Erzurum differ in the distribution of post-predicate arguments. For this reason, I have conducted similar analyses for the Erzurum data as I did for the data from Ankara (see Appendix for the results of the regression analyses). 

First of all, the Turkish\il{Turkic!Turkish} data from Erzurum is substantially pre-predicate (with around 12\% of the utterances containing post-predicate positions), that is very similar to the Turkish\il{Turkic!Turkish} data collected in Ankara. However, the regression analyses run on the Turkish\il{Turkic!Turkish} data from Erzurum showed that there is no effect of \isi{weight}, meaning that regardless of the number of phonological words that tokens consist of, they can be placed in the pre- or post-predicate position with no preference. But the two variables that have a positive effect on the employment of the post-predicate position are Role and Flagging. Namely, the model with Role as an independent variable showed that goals of motion are often placed in post-predicate position in Turkish\il{Turkic!Turkish}. At the same time, the regression model with Flagging as an independent variable demonstrated that tokens flagged with case or a \isi{postpositional} relational noun are likely to be placed in post-predicate position. 

Looking back at the results of the Kurmanji\il{Kurdish (Northern)} data from Ankara, we see that these two variables, Role and Flagging, were also found to be significant. Namely, the analysis of the Kurmanji\il{Kurdish (Northern)} data showed that goals of motion, and tokens flagged with case or a locational noun (sometimes in a combination with a \isi{preposition}) are likely to be placed in post-predicate position in Kurmanji\il{Kurdish (Northern)}. Hence, similar factors influence the employment of the post-predicate position in the data sets from Kurmanji\il{Kurdish (Northern)} in Ankara and Turkish\il{Turkic!Turkish Erzurum} in Erzurum, while there are no common factors in the Turkish\il{Turkic!Turkish Ankara} data sets collected in Ankara and Erzurum.

Nevertheless, it must be acknowledged that the two Turkish\il{Turkic!Turkish Ankara} data sets (from Ankara and Erzurum) are not exactly comparable. First, the Turkish\il{Turkic!Turkish Erzurum} data from Erzurum come from three speakers, whereby one speaker is a young adult and two speakers are in their sixties, and their educational status is not known, unlike the speakers in Ankara who are all educated young adults. Second, the data were collected in the nineties, nearly 30 years prior to the data collection in Ankara. But what is most important is that it is not known whether the speakers from Erzurum are in fact bilingual\is{bilingualism} in Turkish\il{Turkic!Turkish} and Kurmanji\il{Kurdish (Northern)}, or if they are Turkish\il{Turkic!Turkish} monolingual speakers. But what definitely differentiates the speakers in Erzurum from the speakers in Ankara is the number of Kurmanji\il{Kurdish (Northern)} speakers in the community in general and the \isi{societal status} of Turkish\il{Turkic!Turkish}. Even though Turkish\il{Turkic!Turkish} certainly remains the language of formal contexts (such as education, business, etc.), Kurmanji\il{Kurdish (Northern)} is used more extensively in informal contexts, compared to Ankara (though we cannot compute the index of language use for the Erzurum data due to the absence of metadata for the speakers). 

\section{Discussion}\label{Bilingual:ss:7}

In the study, I investigated \isi{word order}, namely the post-predicate domain, in Turkish\il{Turkic!Turkish} and Kurmanji\il{Kurdish (Northern)} that have been in a long-lasting contact with each other. Both languages are \isi{OV}, but each of them employs the post-predicate position in a different way. While in Turkish\il{Turkic!Turkish} the \isi{word order} is determined by information structural requirements and the post-predicate position is reserved for background information, in Kurmanji\il{Kurdish (Northern)} the post-predicate position is the position for \isi{Goal} arguments, particularly those flagged with case. Taking into account the different conditions encoded in WOWA, I investigated whether there is a structural \isi{convergence} in \isi{word order} in Turkish\il{Turkic!Turkish} and in Kurmanji\il{Kurdish (Northern)} in Turkey.

First of all, the quantitative analysis showed that both languages predominantly place arguments in the pre-predicate position: with around 20\% of the utterances in Kurmanji\il{Kurdish (Northern)} and 10\% of the utterances in Turkish\il{Turkic!Turkish} being post-predicate. Thus, both languages retain \isi{OV} \isi{word order}. 

For Kurmanji\il{Kurdish (Northern)}, the regression analyses showed that the employment of the post-predicate position depends on the semantic \isi{role} of the elements in the respective clause and their \isi{flagging}. Namely, goals of motion and caused motion are likely to be placed in post-predicate position, and this is in line with what has been described in previous research on Kurmanji\il{Kurdish (Northern)} \isi{word order} (\citealt{haig_verb-goal_2015}; \citealt{haig2018northern}; \citealt{gundogdu2019asymmetries}). As for addressees and recipients, the regression analysis did not show that there is a tendency to place them post-predicatively in my data. The main reason for this is different types of \isi{flagging}: while the post-predicatively placed goals are flagged with case, all the pre-predicate examples of addressees and recipients are flagged with a circumposition or a postposition. Besides, in general I have not found many examples of addressees and recipients in my data possibly because the video shown to the participants did not trigger the use of verbs of speech and verbs of \isi{transfer}. In fact, another variable that showed an effect on the placement of tokens in post-predicate position in Kurmanji\il{Kurdish (Northern)} is \isi{flagging}. That is, tokens that are flagged with case, locational noun, or a \isi{preposition} together with a locational noun are likely to be placed in post-predicate position. Thus, so far the analysis showed that the Kurmanji\il{Kurdish (Northern)} data is in line with most of the previous studies on \isi{word order} in Kurmanji\il{Kurdish (Northern)}: \isi{Goal} arguments flagged with case or a locational noun are placed post-predicatively. 

At the same time, the qualitative analysis of the Kurmanji\il{Kurdish (Northern)} data showed that, albeit infrequently, such elements as location and source of motion are placed in the post-predicate position. Nevertheless, it is not clear whether the use of these elements in post-predicate position is language contact-induced, for two reasons: First, I did not find enough examples of this kind, and second, the analysed data is spoken and therefore is full with afterthoughts and self-repairings, and it is not fully clear whether certain post-predicative elements are in fact afterthoughts or self-corrections.

Another interesting observation from the Kurmanji\il{Kurdish (Northern)} data is that there are examples where case-flagged goals of motion are placed in pre-predicate position. Remember that in \sectref{bilingual:ss:2}, I emphasized that case-flagged goals are licensed in post-predicate position. However, my data demonstrate that such constructions in Kurmanji\il{Kurdish (Northern)} can also be pre-predicate, but this concerns only one construction \textit{erdê ketin} `to fall on the ground', whereby \textit{erdê} `on the ground' as a case-flagged \isi{Goal} \isi{argument} is placed before the verb \textit{ketin} `to fall'. Interestingly, the same construction in the contact language Turkish\il{Turkic!Turkish} is rendered as \textit{yere düşmek}, whereby the \isi{Goal} \isi{argument} \textit{yere} `on the ground' is always placed in pre-predicate position. I argue that the pre-predicate position of the \isi{argument} \textit{erdê} occurs due to \isi{transfer} of the whole construction from the dominant language Turkish\il{Turkic!Turkish}. Besides, the reason why \isi{transfer} occurs only with this particular construction might lie in the verb itself: in Kurmanji\il{Kurdish (Northern)} there are other set phrases with the verb \textit{ketin} where the \isi{Goal} is always placed in pre-predicate position, e.g., \textit{bi rê ketin} `to set off (on a journey)'.

As for the post-predicate position in Turkish\il{Turkic!Turkish}, the analysis showed that the only significant predictor of the employment of the post-predicate position, among those encoded in WOWA, is the \isi{weight} of tokens, that is, tokens that consist of three and four phonological words are likely to be placed in post-predicate position. The placement of heavier constituents in post-predicate position is generally a phenomenon typical of spoken language. As \citet{schroeder1995postpredicate} emphasizes, in spoken discourse information is conveyed in smaller chunks to make it more accessible to the hearer and the post-predicate position allows the hearer to keep track of the topical development and the deictic framework in which a predication takes place. On the other hand, the analysis of the Turkish\il{Turkic!Turkish Erzurum} data from Erzurum, where the contact with Kurmanji\il{Kurdish (Northern)} is more intense and Kurmanji\il{Kurdish (Northern)} is present in more spheres of life compared to Ankara, showed that Turkish\il{Turkic!Turkish} speakers in Erzurum tend to place goals of motion, particularly those flagged with case and relational noun, in post-predicate position. At the same time, \isi{weight} of constituents did not have an effect on the placement of constituents in the data from Erzurum. 

In sum, the results of the analysis based on the WOWA encoding scheme show a minimal degree of \isi{convergence} between the two languages of Kurmanji-Turkish speakers in Ankara. In Kurmanji\il{Kurdish (Northern)} semantic \isi{role} of constituents and their \isi{flagging} are determining factors in the placement of the constituents in relation to the verb, which is in line with the previous research (\citealt{haig_verb-goal_2015}; \citealt{haig2018northern}; \citealt{gundogdu2019asymmetries}). In Turkish\il{Turkic!Turkish} \isi{weight} was proven to be a significant factor: longer constituents are more likely to be placed in the post-predicate position. The result that points to possible signs of an ongoing contact-induced language change is the changes in the \isi{word order} of particular constructions. However, such changes are observed only in the minority language Kurmanji\il{Kurdish (Northern)} and not the majority language Turkish\il{Turkic!Turkish}. Thus, such results point to the effect of \isi{societal status} of the languages on the direction of the language change: a more prestigious language Turkish\il{Turkic!Turkish} influences a less prestigious Kurmanji\il{Kurdish (Northern)!Ankara} in Ankara.

At the same time, the analysis of the Erzurum Turkish\il{Turkic!Turkish Erzurum} data suggests that another social factor – the intensity of contact\is{language contact!intensity of contact} – has an impact on the occurrence of changes in \isi{word order} of the languages in contact. Unlike Turkish\il{Turkic!Turkish Ankara} in Ankara, in Erzurum Turkish\il{Turkic!Turkish Erzurum} semantic \isi{role} and \isi{flagging} have an effect on the employment of the post-predicate position. Hence, when Kurmanji\il{Kurdish (Northern)} is present in more spheres of life and the community size is bigger, changes also happen in the majority language Turkish\il{Turkic!Turkish}. Even though it is not clear whether the speakers in the Erzurum data set are in fact bilinguals, the predictors for the placement of tokens post-predicatively are the same as in Kurmanji\il{Kurdish (Northern)}.

Finally, it is important to emphasize that \isi{information structure} was not encoded in WOWA and, consequently, has not been analyzed in this chapter. Therefore, it remains unclear whether information structural constraints are loosened in the Turkish\il{Turkic!Turkish} of bilinguals and whether \isi{information structure} plays a \isi{role} in the \isi{word order} of Kurmanji\il{Kurdish (Northern)} as a result of contact with Turkish\il{Turkic!Turkish}. This presents a limitation of the current study; however, the \isi{role} of \isi{information structure} on \isi{word order} in Turkish\il{Turkic!Turkish} and Kurmanji\il{Kurdish (Northern)} of the same speakers is discussed in another study by \citet{iefremenko2023wordorder}.

\section*{Abbreviations}
\begin{tabularx}{.45\textwidth}{lQ}
\textsc{abl} & {ablative} \\
\textsc{acc} & {accusative} \\
\textsc{circ} & circumposition \\
\textsc{drct} & directional \\
\textsc{ez} & ezafe \\
\textsc{f} & feminine \\
\textsc{fut} & future \\
\textsc{gen} & genitive \\
\textsc{indef} & indefinite \\
\textsc{m} & masculine \\
\end{tabularx}
\begin{tabularx}{.45\textwidth}{lQ}
\textsc{neg} & negation \\
\textsc{obl} & {oblique} \\
\textsc{pl} & plural \\
\textsc{poss} & {possessive} \\
\textsc{postp} & postposition \\
\textsc{prog} & progressive \\
\textsc{prs} & present \\
\textsc{pst} & past \\
\textsc{sg} & singular \\
\textsc{subj} & subjunctive \\
\end{tabularx}

\section*{Acknowledgements}

I would like to thank all members of the project ``Post-predicate Elements in Iranian: Inheritance, Contact, and Information Structure" and the participants of the project's workshops for their invaluable feedback and comments. I also extend my gratitude to Prof. Dr. Christoph Schroeder, the anonymous reviewer and the editors of this volume for their helpful and highly constructive suggestions. Special thanks to Dr. Shuan Karim for the assistance with LaTeX. The research was supported through funding by the Deutsche Forschungsgemeinschaft (DFG, German Research Foundation) for the Research Unit “Emerging Grammars in Language Contact Situations”, project P9 (313607803) and the Alexander-von-Humboldt-Stiftung.

{\sloppy\printbibliography[heading=subbibliography,notkeyword=this]}

\newpage
\begin{paperappendix}
\section{Erzurum Turkish (Turkic, \citealt{dogan_oghuz_2021})}

\begin{table}
 \begin{tabularx}{\textwidth}{XrYYl}
\lsptoprule
\textbf{fixed effect} & \textbf{β} &  \textbf{σ} & \multicolumn{2}{r}{\textbf{p-value}}  \\
\midrule
\textbf{(intercept)} & -1.557e+01 &  1.029e+03 & 0.98 &  (ns) \\
\textbf{Anim-adv} & 1.439e+01 &  1.029e+03 & 0.98 &  (ns) \\
\textbf{Anim-anim} & 1.279e+01 &  1.029e+03 & 0.99 &  (ns) \\
\textbf{Anim-bp} & 1.339e+01 &  1.029e+03 & 0.99 &  (ns) \\
\textbf{Anim-hum} & 1.331e+01 &  1.029e+03 & 0.99 &  (ns) \\
\textbf{Anim-inan} & 1.383e+01 &  1.029e+03 & 0.98 &  (ns) \\
\textbf{Anim-other} & 2.205e-08 &  1.188e+03 & 1.00 &  (ns) \\
\lspbottomrule
 \end{tabularx}
 \caption{\textbf{Animacy:} Regression table for binomial GLM with the dependent variable Position and the independent variable Animacy in Turkish spoken in Erzurum}
 \label{Bilingual:tab:9}
\end{table}

\begin{table}
 \begin{tabularx}{\textwidth}{XrYYl}
\lsptoprule
\textbf{fixed effect} & \textbf{β} &  \textbf{σ} & \multicolumn{2}{r}{\textbf{p-value}}  \\
\midrule
\textbf{(intercept)} & -2.10 &  0.29 & < .0001 \\
\textbf{Weight} & 0.16 &  0.19 & 0.39 &  (ns) \\
\lspbottomrule
 \end{tabularx}
 \caption{\textbf{Weight:} Regression table for binomial GLM with the dependent variable Position and the independent variable Weight in Turkish spoken in Erzurum}
 \label{Bilingual:tab:10}
\end{table}

\begin{table}
 \begin{tabularx}{\textwidth}{lrYYl}
\lsptoprule
\textbf{fixed effect} & \textbf{β} &  \textbf{σ} & \multicolumn{2}{r}{\textbf{p-value}}  \\
\midrule
\textbf{(intercept)} & -3.20 &  0.45 & < .0001 & *** \\
\textbf{Flag-case} & 1.46 &  0.47 & 0.02 & ** \\
\textbf{Flag-postp-relnoun} & 2.25 &  0.63 & 0.0003 & *** \\
\lspbottomrule
 \end{tabularx}
 \caption{\textbf{Flagging:} Regression table for binomial GLM with the dependent variable Position and the independent variable Flagging in Turkish spoken in Erzurum}
 \label{Bilingual:tab:11}
\end{table}

% \makeatletter
% \setlength{\@fptop}{0pt}
% \makeatother

\newpage
\begin{table}[t]
 \begin{tabularx}{\textwidth}{XrYYl}
\lsptoprule
\textbf{fixed effect} & \textbf{β} &  \textbf{σ} & \multicolumn{2}{r}{\textbf{p-value}}  \\
\midrule
\textbf{(intercept)} & -2.30 &  6.05 & < .0001 &  *** \\
\textbf{Role-addr } & 5.129e-02 &  9.588e-01 & 0.95 &  (ns) \\
\textbf{Role-becm} & -9.531e-02 &  1.207e+00 & 0.93 \\
\textbf{Role-ben} & 1.386e+00 &  8.466e-01 & 0.1 &  (ns) \\
\textbf{Role-cop} & -1.526e+01 &  9.890e+02 & 0.98 &  (ns) \\
\textbf{Role-do} & -7.793e-01 &  7.588e-01 & 0.3 &  (ns) \\
\textbf{Role-do-def} & -2.632e-02 &  6.632e-01 & 0.96 &  (ns) \\
\textbf{Role-goal} & 2.042e+00 &  6.438e-01 & 0.001 & ** \\
\textbf{Role-goal-c} & 1.022e+00 &  7.032e-01 & 0.14 &  (ns) \\
\textbf{Role-instr} & -1.526e+01 &  1.251e+03 & 0.99 &  (ns) \\
\textbf{Role-loc} & -6.592e-01 &  1.251e+03 & 0.43 &  (ns) \\
\textbf{Role-other} & 1.372e-14 &  9.574e-01 & 1.00 &  (ns) \\
\textbf{Role-poss} & -1.526e+01 &  2.797e+03 & 0.99 &  (ns) \\
\textbf{Role-rec} & -3.365e-01 &  9.499e-01 & 0.72 &  (ns) \\
\textbf{Role-stim} & -1.526e+01 &  3.956e+03 & 0.99 &  (ns) \\
\lspbottomrule
 \end{tabularx}
 \caption{\textbf{Role:} Regression table for binomial GLM with the dependent variable Position and the independent variable Role in Turkish spoken in Erzurum}
 \label{Bilingual:tab:12}
\end{table}

\vfill
~
\end{paperappendix}

\end{document}
