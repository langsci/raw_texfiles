\documentclass[output=paper,colorlinks,citecolor=brown,draftmode]{langscibook}
\ChapterDOI{10.5281/zenodo.14266337}
\author{Maryam Nourzaei\affiliation{Uppsala University} and Geoffrey Haig\orcid{0000-0002-5410-3692}\affiliation{University of Bamberg}}
\title{Balochi: A cross-dialect investigation of post-verbal elements}
\abstract{\begin{sloppypar}This chapter investigates word order in three Balochi varieties: Coastal Balochi (Coastal), Koroshi Balochi (Koroshi) and Turkmenistan Balochi (Turkmen). Although all three are closely related, they are areally widely dispersed, making Balochi an interesting test case for investigating the effects of areality on closely related varieties. All three varieties are predominantly OV. However, pronominal direct objects show a stronger tendency to post-verbal placement, especially in Coastal, echoing similar findings for other Iranian languages of the region. All three varieties exhibit predominantly post-verbal Goal\is{Goal!post-verbal}s (VG), with the highest values found in Koroshi, confirming the expected correlation between higher frequency of VG and geographic proximity to the southwestern Mesopotamian region of the Western Asian Transition Zone.\end{sloppypar}}


\IfFileExists{../localcommands.tex}{
   \addbibresource{../localbibliography.bib}
   \addbibresource{../collection_tmp.bib}
   % add all extra packages you need to load to this file

\usepackage{tabularx,multicol}
\usepackage{url}
\urlstyle{same}

\usepackage{listings}
\lstset{basicstyle=\ttfamily,tabsize=2,breaklines=true}

\usepackage{langsci-basic}
\usepackage{langsci-optional}
\usepackage{langsci-lgr}
\usepackage{langsci-osl}
% \usepackage{./langsci/styles/langsci-lgr}
% \usepackage{./langsci/styles/langsci-osl}
% \usepackage{langsci-gb4e}

\usepackage{tikz}
\usetikzlibrary{patterns,calc}
\pgfdeclarepatternformonly{south east lines}{\pgfqpoint{-0pt}{-0pt}}{\pgfqpoint{3pt}{3pt}}{\pgfqpoint{3pt}{3pt}}{
    \pgfsetlinewidth{0.6pt}
    \pgfpathmoveto{\pgfqpoint{0pt}{3pt}}
    \pgfpathlineto{\pgfqpoint{3pt}{0pt}}
    \pgfpathmoveto{\pgfqpoint{.2pt}{-.2pt}}
    \pgfpathlineto{\pgfqpoint{-.2pt}{.2pt}}
    \pgfpathmoveto{\pgfqpoint{3.2pt}{2.8pt}}
    \pgfpathlineto{\pgfqpoint{2.8pt}{3.2pt}}
    \pgfusepath{stroke}}
    
\usepackage{stmaryrd}
\usepackage{wasysym}
\usepackage{multirow}
\usepackage{caption}
\usepackage{subcaption}
\usepackage{mathrsfs}
\usepackage{qtree}

\usepackage{linguex}


   %pminos do not split footnotes
% \interfootnotelinepenalty=10000 %Footnote in Laporte chapters has to be split SN


%\DeclareIndexNameFormat{default}{%
%\nameparts{#1}%
%\usebibmacro{index:name}%
%{\index[names]}%
%{\namepartfamily}%
%{\namepartgiveni}%
% {}% L1
% {}% L2
%{\namepartprefix}% generates spurious space L3
%{\namepartsuffix}% generates spurious space L4
%}

%  {\DeclareIndexNameFormat{default}{%
%     \usebibmacro{index:name}{\index[names]}{#1}{#3}{#5}{#7}}}

%\DeclareIndexNameFormat{default}{%
%  \usebibmacro{index:name}{\sindex[nom]}{#1}{#3}{#5}{#7}}

%\DeclareIndexNameFormat{default}{%
%  \usebibmacro{index:name}{\sindex[person]}{#1}{#3}{#5}{#7}}
%\DeclareIndexNameFormat{default}{%
%\nameparts{#1} \usebibmacro{index:name}{\sindex[person]]}{\namepartfamily}{‌​\namepartgiven}{\nam‌​epartprefix}{\namepa‌​rtsuffix}}

%\newcommand{\smiley}{:)}

%\renewbibmacro*{index:name}[5]{%
%\usebibmacro{index:entry}{#1}%
%{\iffieldundef{usera}{}{\thefield{usera}\actualoperator}\mkbibindexname{#2}{#3}{#4}{#5}}}

% \newcommand{\noop}[1]{}

%remove for final
%\overfullrule=1mm

\newcommand{\tobi}[2]}}
\renewcommand{\S}[1]{\tobi{#1}{\textsc{*}}}

% this volume references
% puts: [this volume]
% already defined: \citetv
%\newcommand{\citepv}[1]{(\citeauthor{#1} \citeyear*{#1} [this volume])}
\newcommand{\citealtv}[1]{\citeauthor{#1} \citeyear*{#1} [this volume]}

%parentheses around example number
\newcommand{\pref}[1]{(\ref{#1})}

% in-text examples

\newcommand{\lnex}[1]{\textit{#1}} %target lang word
\newcommand{\lnlit}[1]{(lit.: `#1')} %literal reading
\newcommand{\lnlat}[1]{(#1)} % latinization
\newcommand{\lntrans}[1]{`#1'} %translation
\newcommand{\lnexl}[2]%
{\lnex{#1}{} \lnlat{#2}} % ex with latinization
\newcommand{\lnexlat}[3]{\lnex{#1}{} \lnlat{#2}{} \lntrans{#3}} % ex with latinization and tranl.

%ch01
\newcommand{\co}[1]{\mbox{\textbf{#1}}}

%ch09

\newcommand{\cyrbulg}[1]{\begin{otherlanguage*}{bulgarian}#1\end{otherlanguage*}}


%ch10
\newcommand{\nlp}{{\small NLP}}
\newcommand{\mwe}{{\small MWE}}
\newcommand{\rae}{{\small RAE}}
\newcommand{\lvc}{{\small LVC}}
\newcommand{\pos}{{\small P}o{\small S}}
%\newcommand{\todo}[1]{ \textcolor{red}{#1} }

%\renewcommand{\labelenumi}{\theenumi}
%\ainamefmt{{vv}{ll}{, ff}{, jj}} % fullname

\newcommand{\biberror}[1]{{\color{red}#1}}

\newcommand{\osenovaitem}{--~}
   %% hyphenation points for line breaks
%% Normally, automatic hyphenation in LaTeX is very good
%% If a word is mis-hyphenated, add it to this file
%%
%% add information to TeX file before \begin{document} with:
%% %% hyphenation points for line breaks
%% Normally, automatic hyphenation in LaTeX is very good
%% If a word is mis-hyphenated, add it to this file
%%
%% add information to TeX file before \begin{document} with:
%% %% hyphenation points for line breaks
%% Normally, automatic hyphenation in LaTeX is very good
%% If a word is mis-hyphenated, add it to this file
%%
%% add information to TeX file before \begin{document} with:
%% \include{localhyphenation}
\hyphenation{
    Beck-man
    Ngu-yen
    back-chan-nel
    back-chan-nels
    mo-not-o-nous
    ste-reo-typ-i-cal
}

\hyphenation{
    Beck-man
    Ngu-yen
    back-chan-nel
    back-chan-nels
    mo-not-o-nous
    ste-reo-typ-i-cal
}

\hyphenation{
    Beck-man
    Ngu-yen
    back-chan-nel
    back-chan-nels
    mo-not-o-nous
    ste-reo-typ-i-cal
}

%    \boolfalse{bookcompile}
%    \togglepaper[1]%%chapternumber
}{}

\begin{document}
\maketitle\label{WOWA:ch:4}


\section{Introduction: Language background and data sources}\label{Balochi:ss:1}

Balochi is a northwestern Iranian language, which belongs to the Indo-Iranian branch of Indo-European. Syntactically, Balochi is \isi{OV}, but shows mixed adpositional typology (see \sectref{Balochi:ss:3.2}) as well as dialectally differentiated alignment systems (see below). Word order in the NP is generally \isi{head-final}: adjectives precede nouns, and take an attributive linker \textit{-ēn/\textbf{ễ}}. Possessors also precede the possessed, and the possessor takes a so-called genitive case, though this may vary in the westernmost dialects; see \sectref{Balochi:ss:3} for details.

Balochi provides an excellent window on the interplay of areal and genetic influence in shaping \isi{word order}. The unity of `Balochi' as the descendants of a historically reconstructable sub-group of Iranian is justified in \citet[21]{korn_towards_2005}, and there is no doubt that the three doculects under consideration here can be assigned to Balochi. Today, however, due to successive population movements, varieties of Balochi are spoken across a vast area, including Southeastern and Southwestern Iran, Southwestern Pakistan, and also in Afghanistan, India, Africa, Turkmenistan, Oman and the UAE. Almost all Balochi speakers are multilingual, with contact languages belonging to four different language families, and different genera within them: Indo-European (Iranian, Indo-Aryan and Slavic), Dravidian, Turkic and Semitic. Disentangling what is inherited Proto-Balochi from the multiple contact effects is methodologically challenging, for syntax just as it is for \isi{phonology} and morphology. We address the implications of our study for broader questions of Iranian diachronic syntax in \sectref{Balochi:ss:5}.

% \setlength{\footheight}{59.06021pt}
\largerpage

Research on Balochi recognizes three main dialects: southern, eastern and western Balochi. Each of these dialects demonstrates its own sub-divisions \citep[see][636--637]{jahani_balochi_2009}. In addition, a group of dialects to the southwest is distinguished, collectively referred to as Koroshi\il{Balochi!Koroshi}. The total number of Balochi speakers is uncertain, though, \citet{jahani_balochi_2013} reports an estimate of at least 10 million speakers. For the comparisons undertaken in this chapter, we have selected data from three geographically dispersed locations, each of which lies within some larger dialect region of Balochi. We refer to the data from these locations as doculects (the variety documented in a specific data set), rather than dialects, because we cannot assume that each doculect is necessarily representative of its larger dialect region. The three doculects are labelled Turkmen\il{Balochi!Turkmen} (Turkmenistan Balochi), Coastal Balochi\il{Balochi!Coastal} (Coastal Balochi), and Koroshi\il{Balochi!Koroshi} (Koroshi Balochi) respectively. The data for Coastal Balochi\il{Balochi!Coastal} come from two villages, Korsar and Sedighzahi, Dashtiyari, Iran. For Koroshi\il{Balochi!Koroshi}, data comes from from Deh Piyaleh, Shiraz, and Marvdasht, Iran, while for Turkmen\il{Balochi!Turkmen} the data come from the Mari region, Turkmenistan. The location of the three doculects, and the main dialect divisions of Balochi, are provided in Fig. 1.

\begin{figure}
 \centering
 \includegraphics[width=\textwidth]{figures/Balochi_fig1.png}
 \caption{Location of the main Balochi dialects and doculects analysed}
 \label{Balochi:fig:1}
\end{figure}

Turkmen\il{Balochi!Turkmen} belongs to the larger Western Balochi dialect group, Coastal Balochi\il{Balochi!Coastal} dialect belongs to the Southern dialect group, and Koroshi\il{Balochi!Koroshi} is part of Koroshi\il{Balochi!Koroshi}, but would probably also be considered part of Southern Balochi in a broad dialect division (see also \citealt{nourzaei_koroshi_2015}: 22). 

It is important to bear in mind that our doculects are not necessarily representative for the entire dialect group to which they belong, and in fact our results suggest that there is considerable internal variation (at least on the \isi{word order} parameters that we have investigated) within the larger dialect groups that have traditionally been recognized. For each doculect, a WOWA data set was compiled as the basis for quantitative comparison. Text types are traditional narrative texts for Coastal Balochi\il{Balochi!Coastal} and Koroshi\il{Balochi!Koroshi}, while the texts for Turkmen\il{Balochi!Turkmen} include traditional narratives, a procedural text, and a life story; see \citet{nourzaei_balochi_coastal_2021,nourzaei_balochi_koroshi_2021} and \citet{haig_balochi_2022} for details and access to the data. In citing examples we follow the original transcriptions for Coastal Balochi\il{Balochi!Coastal} and Koroshi\il{Balochi!Koroshi} but adapt the original transcriptions of Turkmen\il{Balochi!Turkmen} to bring it closer to the other two.

\begin{sloppypar}
The doculects vary according to their alignment systems: Turkmen\il{Balochi!Turkmen} and Koroshi\il{Balochi!Koroshi} display the same alignment system in both past and present domain, while Coastal Balochi\il{Balochi!Coastal} demonstrates \isi{ergative} only for 3rd person in transitive clauses based on past-tense verb forms. Koroshi\il{Balochi!Koroshi}, due to reduction of its morphological case system, uses more person clitics than Turkmen\il{Balochi!Turkmen} and Coastal Balochi\il{Balochi!Coastal} to index verb arguments. The most recent studies on Coastal Balochi\il{Balochi!Coastal} are \citet{nourzaei_participant_2017,nourzaei_documenting_forthcoming} and \citet{korn_notes_2019}. Earlier works on Koroshi\il{Balochi!Koroshi} include a grammatical sketch in Persian\il{Persian} \citep{emadi_guyes-e_2005}, \citet{nourzaei_koroshi_2015}, \citet{nourzaei_participant_2017,nourzaei_documenting_forthcoming}. It is assumed that speakers of Koroshi\il{Balochi!Koroshi} originally migrated from Balochistan to the Southwest of Iran. The most significant study on Turkmen\il{Balochi!Turkmen} dialect is \citet{axenov_balochi_2006}. Turkmen\il{Balochi!Turkmen} Baloch is the result of migration from either Iran or Afghanistan and north-eastern parts of Iran. For the history of the Baloch migration to Turkmenistan, see \citet[71]{axenov_balochi_2000}, among others. In addition to the quantitative analysis based on the WOWA data sets, we also draw on additional published corpora and analysis \citep{barjasteh_delforooz_discourse_2010,nourzaei_koroshi_2015,nourzaei_participant_2017,nourzaei_documenting_forthcoming,axenov_balochi_2006}. 
\end{sloppypar}

This chapter is organized as follows. \sectref{Balochi:ss:2} summarizes previous literature, while \sectref{Balochi:ss:3} provides a sketch of some relevant aspects of Balochi grammar, including NP-internal \isi{word order}. In \sectref{Balochi:ss:4}, we present the core of the quantitative analysis, based on the major clausal constituent types recognized in the WOWA framework, while \sectref{Balochi:ss:5} discusses the implications of the Balochi findings from a comparative (West) Iranian perspective.

\section{Previous studies on word order in Balochi}\label{Balochi:ss:2}

With regard to clause-level constituents, empirical comparative research only commenced quite recently, in particular \citet{Jahani2018Post-verbal} and \citet{Korn2022Targets}. \citet{Jahani2018Post-verbal} covers four dialects, viz., Koroshi\il{Balochi!Koroshi}, Sistani\il{Balochi!Sistani}, Turkmen\il{Balochi!Turkmen} and Southern Balochi (Iran and Pakistan), and considers four \isi{argument} types: goals, recipients, addressees, and final states of change-of-state predicates. She notes that across all varieties, goals of motion tend to be postverbal, while with regard to the other roles, there are differences: While Koroshi generally has all four \isi{argument} types post-verbally, Sistani\il{Balochi!Sistani} and Turkmenistan Balochi has goals, recipients, and addressees, but not final-states in post-verbal position. For Southern Balochi, also \citet{Jahani2018Post-verbal} includes a more detailed study of spoken versus written genres, observing a very significant tendency towards strict pre-verbal placement of all \isi{argument} types in written Southern Balochi, while spoken Southern Balochi places 90\% of goals after the verb.\footnote{
The data from \citet{Jahani2018Post-verbal} are taken from the slides available from the conference talk. For Southern Balochi, the data for recipients on the overview slide do not match the findings reported from the oral vs. written case study, so we make no claim regarding recipients in Southern Balochi.  
} The differences between oral and written language are very much in line with the observations for Persian\il{Persian}, where colloquial spoken Persian\il{Persian (colloquial)} has a high frequency of post-verbal Goal\is{Goal!post-verbal}s (around 80\%, see \citetv{chapters/7_RasekhMahandetal_Persian}), while the frequency is much lower for formal written Persian\il{Persian}. These pioneering observations suggest that the overall situation in Balochi shows some similarities with the situation that has been identified for languages further to the west (e.g. Kurdish\il{Kurdish}, \citealt{haig_kurdish_2022}), in particular the special status of goals (see \citetv{chapters/1_Haigetal_Intro} for summary), but there are also significant differences. We turn to explanations in \sectref{Balochi:ss:5} below.

\begin{sloppypar}
\citet[114]{Korn2022Targets} draws on data from Afro-Baloch varieties, taken from Nourzaei and Korn's unpublished Afro-Baloch data. Specifically, the data come from three locations: Shirgowaz (close to Bahukalat), Konark and Karewan. As such, they are only indirectly comparable to our Coastal Balochi\il{Balochi!Coastal} doculect, although both Korn's data and our own Coastal Balochi\il{Balochi!Coastal} doculect can be included under the broad umbrella term of Southern Balochi. Korn's data confirm the strong tendency to place goals post-verbally, though she suggests that goals of caused motion are overall less likely to be post-verbal, a finding which is confirmed in our own Coastal Balochi\il{Balochi!Coastal} data in Section 4 below. Korn suggests that many of the caused goals in her data are human, making them overall more similar to recipients \citep[104]{Korn2022Targets}. Recipients are commonly post-verbal, especially when the \isi{recipient} or \isi{benefactive} is ``mentioned for the first time ''\citep[105]{Korn2022Targets}, but pre-verbal position is also well-attested, so that no clear conclusion can be reached with regard to a canonical or unmarked position; rather, the interaction of various factors such as \isi{animacy}, verb semantics, overall presence of co-arguments, and \isi{information structure} co-determine the placement. For addressees, only few overt examples were present in the data, but all occur preverbally \citep[107]{Korn2022Targets}. This finding confirms \citegen{Jahani2018Post-verbal} observation that addressees are overall more likely to be preverbal than either goals or recipients. Again, this confirms observations from other languages in WATZ, according to which addressees are less likely to be post-verbal than recipients or goals (see \citetv{chapters/1_Haigetal_Intro}, \sectref{Intro:ss:4}).
\end{sloppypar}

Korn also investigates direct objects, noting that post-verbal placement ``is surprisingly frequent'' \citep[113]{Korn2022Targets}, though no figures are provided; we return to this below. Korn also discusses the possible impact of \isi{flagging}, and \isi{animacy}, noting a tendency for [+human] arguments to be pre-verbal. She concludes with a suggestion for the pathway towards ``generalization of the post-verbal position'' in Balochi \citep[118]{Korn2022Targets}, according to which non-human goals would have been the first kinds of constituent in this position, extending then to include metaphorical uses of direction (purpose) and then other kinds of adverbial, while a second line of extension would proceed via [+human] goals to recipients and beneficiaries. Ultimately, an extension to direct objects is considered as a final possibility. \citet{Korn2022Targets} also discusses the impact of \isi{flagging}, verb serialization, and interactions with subordination. In sum, \citet{Jahani2018Post-verbal} and \citet{Korn2022Targets} provide a very informative overview of the relevant characteristics of Balochi, which already identifies some of the areas of cross-dialect variation. The present study builds on these observations but extends the the range of \isi{argument} types considered, and is based on more accessible data.

\section{Basic features of Balochi: Morphosyntax and NP-internal word order}\label{Balochi:ss:3}

\subsection{Alignment, person marking, and nominal inflection}\label{Balochi:ss:3.1}

\begin{sloppypar}
Among the three doculects considered here, only Coastal Balochi\il{Balochi!Coastal} displays stem-sensitive ergativity, i.e. verbal forms with the present stem pattern accusatively (A in direct case and agreeing with the verb, P in \isi{oblique} or \isi{object} case) while verbal forms with the past stem pattern ergatively (P in direct case and agreeing with the verb (only for third person), A in \isi{oblique} case) (see for details \citealt{nourzaei_participant_2017}). Both Koroshi\il{Balochi!Koroshi} and Turkmen\il{Balochi!Turkmen} Balochi exhibit only \isi{accusative} alignment.
\end{sloppypar}

Doculects differ regarding the usage of person-marking clitics. Both Coastal Balochi\il{Balochi!Coastal} and Turkmen\il{Balochi!Turkmen} only use the person marking clitics for the 3rd person, while Koroshi\il{Balochi!Koroshi} uses the person-marking clitics for all persons. The existence of person-marking clitics in Balochi has a strong correlation with the case system. In varieties such as Coastal Balochi\il{Balochi!Coastal} and Turkmen\il{Balochi!Turkmen} (see above) which have a rich morphological case system, they are less commonly used, while in varieties such as Koroshi\il{Balochi!Koroshi} and Sarawani\il{Balochi!Sarawani} (see \citealt{baranzehi_sarawani_2003}: 86), which display a reduced case system, they are more common. 

Balochi has a morphological case system containing at least a direct case, an \isi{oblique} case, and a genitive case, which is the system found in Koroshi\il{Balochi!Koroshi}. In Sistani\il{Balochi!Sistani}, Afghan and Turkmen Balochi, a \isi{locative} case has developed from the genitive (cf. \citealt{buddruss_aus_1988}: 48, \citealt{axenov_balochi_2006}: 80--82, \citealt{korn_new_2008}, and the data in \citealt{barjasteh_delforooz_discourse_2010}). In Coastal Balochi\il{Balochi!Coastal}, it is sporadically attested as well, typically with human names (\citealt{nourzaei_participant_2017}: 61, \citealt{korn_notes_2019}). In addition, another form appears in Sistani\il{Balochi!Sistani} Balochi that is derived from the \isi{oblique} case and is only used to mark direct objects, whence its name, ``\isi{object} case'' (cf. \citealt{korn_new_2008}: 61--63, \citealt{nourzaei_participant_2017}: 62). In Coastal Balochi\il{Balochi!Coastal} and Koroshi\il{Balochi!Koroshi} doculects, \isi{object} case is only available for pronouns (cf. \citealt{nourzaei_participant_2017}: 44).

With respect to plural marking, Koroshi\il{Balochi!Koroshi} differs from other varieties in that it has an innovated plural marker \textit{-obār} in all cases (direct, genitive and \isi{oblique}, \citealt{nourzaei_koroshi_2015}: 29) while in Coastal Balochi\il{Balochi!Coastal} and Sistani\il{Balochi!Sistani}, the marker \textit{-ān} does not mark the plural on nouns in the direct case, which are thus not directly marked for number. Instead, plural number can be indicated by plural agreement markers on the verb. Nouns in the \isi{oblique} case, on the other hand, show a number opposition. No variety of Balochi has retained grammatical gender.

\subsection{NP-internal word order}\label{Balochi:ss:3.2}

\subsubsection{Adjective and noun}\label{Balochi:ss:3.2.1}

The most common pattern regarding adjective-noun ordering across the dialects is that adjectives precede nouns, and that attributive adjectives take a suffix \textit{-ēn/\textbf{ễ}} (\isi{adjective} attribute suffix, \textsc{ATTR}). 

\ea\label{Balochi:ex:1}
\ea\label{Balochi:ex:1a}
Coastal Balochi \il{Balochi!Coastal}(\citealt{nourzaei_balochi_coastal_2021}, C, 1171)\\
\gll \textbf{zar'd-ễ} negē'na \\
yellow-\textsc{attr} stone \\
\glt `the yellow stone'
\ex\label{Balochi:ex:1b}
Turkmen Balochi \il{Balochi!Turkmen}(\citealt{haig_balochi_2022}, A, 0049)\\
\gll \textbf{ǰwān-ēn} zāg=ē \\
good-\textsc{attr} son=\textsc{indv} \\
\glt `a good son' 
\ex\label{Balochi:ex:1c}
Koroshi Balochi \il{Balochi!Koroshi}(\citealt{nourzaei_koroshi_2015}:42)\\
\gll \textbf{bōr-ēn} pašm-ā \\
beige-\textsc{attr} wool-\textsc{obl} \\
\glt `the beige wool'
\z
\z

However, Koroshi\il{Balochi!Koroshi} exhibits borrowed ezafe constructions (N-ADJ order) from Persian\il{Persian} shown in (\ref{Balochi:ex:2}).

\ea\label{Balochi:ex:2}
\ea\label{Balochi:ex:2a}
Koroshi Balochi \il{Balochi!Koroshi}(\citealt{nourzaei_balochi_koroshi_2021}, A, 0006)\\
\gll ādam=e \textbf{xūb}=ī \\
person=\textsc{ez} good=\textsc{indv} \\
\glt `a good person'
\ex\label{Balochi:ex:2b}
Koroshi Balochi \il{Balochi!Koroshi}(UP)\\
\gll ǰāhel=e nūrānī=ye \\
boy=\textsc{ez} handsome=\textsc{indv} \\
\glt `a handsome boy'
\z
\z

\subsubsection{Possessor and noun}\label{Balochi:ss:3.2.2}

Across the dialects, possessors normally precede the possessed noun ({POSS}-N order), and the possessor takes a so-called genitive case (for details regarding different forms of genitive case (see \citealt{nourzaei_participant_2017}, \citealt{nourzaei_koroshi_2015}, \citealt{korn_notes_2019}).

\ea\label{Balochi:ex:3}
\ea\label{Balochi:ex:3a}
Coastal Balochi \il{Balochi!Coastal}(\citealt{nourzaei_balochi_coastal_2021}, A, 0315)\\
\gll \textbf{sī'morg-e} dap-ā \\
fabulous\_bird-\textsc{gen} mouth-\textsc{obl} \\
\glt `(into) the fabulous bird's mouth'
\ex\label{Balochi:ex:3b}
Turkmen Balochi \il{Balochi!Turkmen}(\citealt{haig_balochi_2022}, A, 0083)\\
\gll \textbf{xānbādorr-ī} ǰenēn-ā \\
Khanbadur-\textsc{gen} wife-\textsc{obl} \\
\glt `the wife of Khanbadur'
\ex\label{Balochi:ex:3c}
Koroshi Balochi \il{Balochi!Koroshi}(\citealt{nourzaei_balochi_koroshi_2021} A, 0007)\\
\gll \textbf{šāh-ay} awal-īn bač \\
king-\textsc{gen} first-\textsc{attr} son \\
\glt `the king's first son'
\z
\z


In Koroshi\il{Balochi!Koroshi}, possessors may follow nouns (N-{POSS}) in a kind of ezafe construction, most likely borrowed from Persian\il{Persian}. Most examples in \citet{nourzaei_koroshi_2015} suggest not so much morphological borrowing as wholesale borrowing of phrases from Persian, as in (\ref{Balochi:ex:4}). The actual productivity of N-\textsc{ez} {POSS} constructions in Koroshi thus remains to be established. 

\ea\label{Balochi:ex:4}
Koroshi Balochi \il{Balochi!Koroshi}(\citealt{nourzaei_koroshi_2015}:212)\\
\gll ya banne=ye xodā=ī \\
one servant=\textsc{ez} god=\textsc{indv} \\
\glt `a fellow (lit. a servant) of God'
\z

The ordering of pronominal possessors may differ from nominal possessors. In Coastal \il{Balochi!Coastal} Balochi, pronominal possessors generally follow the possessed: 

\ea\label{Balochi:ex:5}
\ea\label{Balochi:ex:5a}
Coastal Balochi \il{Balochi!Coastal}(\citealt{nourzaei_balochi_coastal_2021}, B, 505)\\
\gll gohār-ā \textbf{otīg-a} \\
sister-\textsc{obl} \textsc{refl}-\textsc{obl} \\
\glt `your sister'
\ex\label{Balochi:ex:5b}
Coastal Balochi \il{Balochi!Coastal}(UP)\\
\gll māt manīg \\
mother mine \\
\glt `my mother'
\z
\z

\subsubsection{Demonstrative and noun}\label{Balochi:ss:3.2.3}

Across all dialects, demonstratives precede nouns, with some marginal exceptions which are ignored here (\citealt{korn_notes_2019}).

\ea\label{Balochi:ex:6}
\ea\label{Balochi:ex:6a}
Coastal Balochi \il{Balochi!Coastal}(\citealt{nourzaei_balochi_coastal_2021}, B, 0711)\\
\gll \textbf{ē} pet \\
\textsc{prox} father \\
\glt `this father'
\ex\label{Balochi:ex:6b}
Turkmen Balochi \il{Balochi!Turkmen}(\citealt{haig_balochi_2022}, A, 0030)\\
\gll \textbf{ē} \textbf{bādešā} \\
\textsc{prox} king \\
\glt `this king'
\ex\label{Balochi:ex:6c}
Koroshi Balochi \il{Balochi!Koroshi}(\citealt{nourzaei_balochi_koroshi_2021}, B, 0570)\\
\gll \textbf{ē} ǰo'ġlā \\
\textsc{prox} boy.\textsc{obl} \\
\glt `this boy'
\z
\z

\subsubsection{Numeral and noun}\label{Balochi:ss:3.2.4}

Plural marking retains to a large extent the archaic pattern, also found in Kurdish\il{Kurdish (Northern)}, whereby only nouns in \isi{oblique} cases are overtly plural marked with a suffix, but plural subject nouns are not overtly marked. All dialects share the commonality of ordering the numerals before head nouns. 

\subsubsection{Adpositions}\label{Balochi:ss:3.2.5}

The dialects have prepositions, postpositions, and circumpositions. Postpositions are generally relational nouns in the \isi{oblique} case, with the NP \isi{complement} in the genitive case. Similar to nouns in the \isi{oblique} case, the adpositions can also be used alone as adverbs. Prepositions usually trigger the \isi{oblique} case of the noun. Note that there is a tendency for losing the \isi{oblique} case after prepositions in Koroshi\il{Balochi!Koroshi}. The respective frequencies of prepositional and \isi{postpositional} use shows interesting cross-dialectal variation, which we sum up in \tabref{Balochi:tab:1} below. In general, Koroshi\il{Balochi!Koroshi} has the strongest tendency to use prepositions (see for more details \citealt{nourzaei_koroshi_2015}: 43--46). Examples of prepositions and postpositions follow:

\ea\label{Balochi:ex:7}
\ea\label{Balochi:ex:7a}
Coastal Balochi \il{Balochi!Coastal}(\citealt{nourzaei_balochi_coastal_2021}, B, 0678)\\
\gll čāt-e \textbf{tōkā} \\
well-\textsc{gen} inside \\
\glt `inside the well'
\ex\label{Balochi:ex:7b}
Turkmen Balochi \il{Balochi!Turkmen}(\citealt{haig_balochi_2022}, D, 0576)\\
\gll gis-ay \textbf{tā} \\
house-\textsc{gen} inside \\
\glt `inside the house'
\ex\label{Balochi:ex:7c}
Koroshi Balochi \il{Balochi!Koroshi}(\citealt{nourzaei_koroshi_2015}:36)\\
\gll \textbf{dawr}=e ī mēdag-ā \\
around=\textsc{ez} \textsc{prox} encampment-\textsc{obl} \\
\glt `around this encampment'
\ex\label{Balochi:ex:7d}
Koroshi Balochi \il{Balochi!Koroshi}(\citealt{nourzaei_balochi_koroshi_2021}, B, 0553)\\
\gll čāh-ay \textbf{tōxā} por=e šamšīr=o nayza a=kan-t \\
well-\textsc{gen} in full=\textsc{ez} sword=and spear \textsc{vcl}=do.\textsc{prs}-\textsc{3sg} \\
\glt `She fills the well with swords and spears.'
\z
\z

In \isi{contrast} to Koroshi\il{Balochi!Koroshi}, Coastal Balochi\il{Balochi!Coastal} and Turkmen\il{Balochi!Turkmen} possess circumpositions:

\begin{sloppypar}
\ea\label{Balochi:ex:8}
Turkmen Balochi \il{Balochi!Turkmen}(\citealt{axenov_balochi_2006}:150, glosses follow the source)\\
\gll \textbf{bi} diga gis-ē {tā} \\
to other house-\textsc{ind} inside \\
\glt `to another house'
\z
\end{sloppypar}

\begin{sloppypar}
For our quantitative analysis of adpositions in actual usage, based on the WOWA data, we will consider the respective frequencies of prepositional and \isi{postpositional} \isi{flagging}, across those functions that we would generally expect to favour adpositional over other types of \isi{flagging} (e.g. case marking, or lack of any overt \isi{flagging}).\footnote{
The procedure was as follows: Taking the three Balochi WOWA data sets, we selected the total number of tokens in the following functions: ABL(ative); ADDR(essee); BEN(efactive); COM(itative); GOAL; GOAL-C(aused); INSTR(umental); LOC(ative); REC(ipient); REC-BEN. We then extracted those that were flagged with \isi{preposition} or pre-nominal relational nouns (lumped together as ``prepositional"), and those that were flagged with postpositions, or post-nominal relational nouns (lumped together as ``\isi{postpositional}"). Together this yielded 343 tokens.
} \tabref{Balochi:tab:1} shows the respective frequencies of \isi{postpositional} and prepositional \isi{flagging} across the three dialects for NPs in these functions.
\end{sloppypar}

\begin{table}
 \begin{tabularx}{\textwidth}{l Yr Yr Yr Y}
 \lsptoprule
& \multicolumn{2}{r}{Coastal} & \multicolumn{2}{r}{Koroshi} & \multicolumn{2}{r}{Turkmen} \\
\cmidrule(lr){2-3}\cmidrule(lr){4-5}\cmidrule(lr){6-7}
Adpositional type & N & \% & N & \% & N & \% & Totals \\
\midrule
Postpositional & 83 & 86 & 2 & 2 & 12 & 10 & 97 \\
Prepositional & 13 & 14 & 129 & 98 & 104 & 90 & 246 \\
\midrule
Totals & 96 & & 131 & & 116 & & 343 \\
\lspbottomrule
 \end{tabularx}
 \caption{Balochi prepositional and postpositional flagging  frequency}
 % \caption{Frequency of prepositional and postpositional flagging in Balochi}
 \label{Balochi:tab:1}
\end{table}


\tabref{Balochi:tab:1} suggests a two-way split across Balochi, between the predominantly prepositional Koroshi\il{Balochi!Koroshi} and Turkmen\il{Balochi!Turkmen} on the one hand, versus predominantly \isi{postpositional} Coastal Balochi\il{Balochi!Coastal} on the other. The high frequency of \isi{postpositional} \isi{flagging} in Coastal Balochi\il{Balochi!Coastal} is intriguing. At this point we have no convincing explanation. It may be a retention of earlier Balochi structures, which has been lost in other varieties through greater contact with other west Iranian languages, notably Persian\il{Persian}. It may be connected to \isi{multilingualism} with Urdu and other \isi{postpositional} Indo-Aryan languages spoken in the region, but this is speculative. Interestingly, in our data Turkmen\il{Balochi!Turkmen} is dominant prepositional, although we might have expected higher rates of postpositions due to contact with Turkic. It is possible that Turkic has had less influence because the migrations of Turkmen\il{Balochi!Turkmen} speakers to Turkmenistan was relatively recent, and they retain contact with Sistani\il{Balochi!Sistani} Balochi speakers in Iran. This would be in line with the findings of \citet{haig_which_2023}, according to which adpositional type is a relatively conservative \isi{word order} parameter that only shifts under intense and long contact.

\subsection{Auxiliary and main verb, complement clause and matrix clause}\label{Balochi:ss:3.3}

TAM categories are expressed by the presence or absence of verbal prefixes (\citealt{jahani_balochi_2009}, \citealt{axenov_balochi_2006}, \citealt{nourzaei_koroshi_2015} among others). The perfect system uses the participle followed by the inflected \isi{copula}, which cliticizes to the verb, while the progressive (e.g., Koroshi\il{Balochi!Koroshi} and Coastal Balochi\il{Balochi!Coastal}) is built from infinitive plus \isi{clitic} \isi{copula}. Historically, then, we can assume at least some examples of V-Aux order, which have since univerbated through cliticization of the original \isi{auxiliary}. In contemporary Balochi, however, prosodically independent (non-\isi{clitic}) \isi{auxiliary} verbs are infrequent, so establishing a regular order of \isi{auxiliary} and verb is not straightforward. Examples with cliticized auxiliaries are the following:

\ea\label{Balochi:ex:9}
\ea\label{Balochi:ex:9a}
Turkmen Balochi \il{Balochi!Turkmen}(\citealt{haig_balochi_2022}, D, 0548)\\
\gll ammā pa wat-ī māl-ān yakk yakk=ī nām išt=at-an \\
\textsc{1pl} for \textsc{refl}-\textsc{gen} animal-\textsc{pl} one one=\textsc{indv} name put.\textsc{pst}=\textsc{cop}.\textsc{pst}-\textsc{1pl} \\
\glt `We had given names to everyone of our sheep.'
\ex\label{Balochi:ex:9b}
Koroshi Balochi \il{Balochi!Koroshi}(\citealt{nourzaei_balochi_koroshi_2021}, A, 0296)\\
\gll hasan kačal faġat \textbf{nay-āk-ag=en} \\
Hasan bald only \textsc{neg}-come.\textsc{pst}-\textsc{pp}=\textsc{cop}.\textsc{3sg} \\
\glt `Only Hasan the Bald has not come along.'
\z
\z


The Sistani\il{Balochi!Sistani} variety of Western Balochi has an \isi{auxiliary} verb \textit{dāšten} `have', a recent borrowed element from Persian\il{Persian}, to build progressive construction in past and present domain (see for details \citealt{nourzaei_progressive_nodate}). As in Persian\il{Persian}, this \isi{auxiliary} precedes the main verb as in (\ref{Balochi:ex:10}): 

\ea\label{Balochi:ex:10}
\ea\label{Balochi:ex:10a}
Sistani Balochi \il{Balochi!Sistani}\citep{nourzaei_progressive_nodate}\\
\gll \textbf{dār-īn} wān-īn sāket be \\
have.\textsc{prs}-\textsc{1sg} read.\textsc{prs}-\textsc{1sg} quiet \textsc{imp}.become.\textsc{prs}.\textsc{2sg} \\
\glt `{B}e quiet, I am studying!'
\ex\label{Balochi:ex:10b}
Sistani Balochi \il{Balochi!Sistani}\citep{nourzaei_progressive_nodate}\\
\gll \textbf{dāšt-on} šot-on ke čākar āt \\
have.\textsc{pst}-\textsc{1sg} go.\textsc{pst}-\textsc{1sg} \textsc{clm} Chakar come.\textsc{pst}.\textsc{3sg} \\
\glt `I was going when Chakar came.'
\z
\z

The subordinator \textit{ke} may introduce various kinds of subordinate clause, i.e, relative, \isi{complement} and adverbial, as well as quoted speech. Across the dialects, the \isi{complement} clauses normally follow the main clause and are either linked to it by asyndetic subordination (juxtaposition) without any overt marker of subordination other than rising \isi{intonation}, or with an overt complementizer. A number of compound conjunctions, composed of nouns or other elements plus ke, such as \textit{mawġeī ke/waġteke} `when', and \textit{be šartī ke} `on the condition that' are also used. Additional subordinating conjunctions include \textit{tā/ta} `until, so that' and \textit{aga/aya} `if'. In all dialects subordination closely follows the basic pattern of Persian\il{Persian} and copies its compound conjunctions (see also \citealt{jahani_balochi_2009}: 678). Examples of \isi{complement} clauses with verbs of speech and perception, with and without complementizers follow:

\ea\label{Balochi:ex:11}
\ea\label{Balochi:ex:11a}
Coastal Balochi \il{Balochi!Coastal}(\citealt{nourzaei_balochi_coastal_2021}, C, 1061)\\
\gll \textbf{pet-ā} \textbf{go} \textbf{ke} man-ī čo nī 'mã 'ta-rā 'sīr da'y-ẫ \\
father-\textsc{obl} say.\textsc{pst} \textsc{clm} \textsc{1sg}-\textsc{gen} child now \textsc{1sg} \textsc{2sg}-\textsc{obj} wedding give.\textsc{prs}-\textsc{1sg} \\
\glt `{T}he father said /that/, ``My son, now I will marry you off.'' '
\ex\label{Balochi:ex:11b}
Turkmen Balochi \il{Balochi!Turkmen}(\citealt{axenov_balochi_2006})\\
\gll \textbf{gis-ay} \textbf{wāond} \textbf{gušt=ī} \textbf{ke} mnī piss iškārī=e at-ī \\
house-\textsc{gen} owner say.\textsc{pst}=\textsc{pc}.\textsc{3sg} \textsc{clm} \textsc{1sg}.\textsc{gen} father hunter=\textsc{indv} \textsc{cop}.\textsc{pst}-\textsc{3sg} \\
\glt `{T}he owner of the house said that my father was a hunter.'
\ex\label{Balochi:ex:11c}
Koroshi Balochi \il{Balochi!Koroshi}(\citealt{nourzaei_koroshi_2015}:143)\\
\gll \textbf{a=genn-an} \textbf{bale} aždahā=am pīk-ay dawr=e šāh-ay ǰanek-ay garden-ā \\
\textsc{vcl}=see.\textsc{prs}-\textsc{3pl} yes dragon=\textsc{add} twist.\textsc{pst.pp}-\textsc{cop}.\textsc{prs}.\textsc{3sg} around=\textsc{ez} king-\textsc{gen} daughter-\textsc{gen} neck-\textsc{obl} \\
\glt `{T}hey see [that] indeed the dragon was wrapped around the neck of the king's daughter.' 
\z
\z

\section{Order of clause-level constituents: A quantitative analysis}\label{Balochi:ss:4}

In this section we present the results of the quantitative analysis, drawing on the set of constituent types defined in the WOWA framework; not all categories are considered, as some have too few tokens for meaningful quantitative analysis.

\subsection{Direct object and verb}\label{Balochi:ss:4.1}

Across the dialects, nominal direct objects are overwhelmingly in the preverbal position (>90\% \isi{OV} in all three WOWA Balochi doculects). This confirms previous research on Balochi (\citealt{jahani_balochi_2009} and \citealt{Korn2022Targets}), and also reflects the overall tendency for Iranian languages in WOWA to be consistently \isi{OV} in discourse, with the exception of Kumzari (see \citealt{haig_kumzari_2022}, \citetv{chapters/1_Haigetal_Intro}). Our analysis does, however, identify some cross-dialectal differences. \tabref{Balochi:tab:2} provides the figures for direct objects, distinguishing between pronominal direct objects (\ref{Balochi:ex:12a}) and nominal direct objects (\ref{Balochi:ex:12b}). WH-words, \isi{clitic} pronouns, and adverbials in \isi{object} function have been excluded from the \isi{pronoun} category, but we include reflexives. In \tabref{Balochi:tab:2}, and in the following Tables, N refers to the absolute number of tokens included in each category, ``Po'' refers to the number of those tokens that were post-verbal, and ``\%'' provides the percentage of post-verbal tokens in each category.

\begin{table}
    \begin{tabularx}{\textwidth}{l rYr rYY rYY r}
    \lsptoprule
& \multicolumn{3}{c}{Coastal} & \multicolumn{3}{c}{Koroshi} & \multicolumn{3}{c}{Turkmen} & \\
\cmidrule(lr){2-4}\cmidrule(lr){5-7}\cmidrule(lr){8-10}
& N & Po & \% & N & Po & \% & N & Po & \% & Totals \\
\midrule
Nominal & 339 & 23 & 6.8 & 182 & 4 & 2.2 & 193 & 3 & 1.6 & 714 \\
Pronominal & 98 & 27 & 27.6 & 20 & 0 & 0 & 55 & 2 & 3.6 & 173 \\
\midrule
Totals & {437} & &  & {202} & &  &  {248} & & &  887 \\
\lspbottomrule
    \end{tabularx}
    \caption{Nominal vs. pronominal post-verbal direct object frequencies}
    \label{Balochi:tab:2}
\end{table}

The absolute number of pronominal objects in the data is quite low, particularly in Koroshi\il{Balochi!Koroshi}, and those that are present are overwhelmingly human (82\%, 142 out of 173); we turn to the interplay of humanness and pronominality in \tabref{Balochi:tab:5} below. Examples of pronominal and nominal direct objects respectively are the following:

\ea\label{Balochi:ex:12}
\ea\label{Balochi:ex:12a}
Turkmen Balochi \il{Balochi!Turkmen}(\citealt{haig_balochi_2022}, A, 0007)\\
\gll annūn b-raw-an ke \textbf{šmā-rā} gis=a da-īn \\
now \textsc{sbjv}-go.\textsc{prs}-\textsc{1pl} \textsc{clm} \textsc{2pl}-\textsc{obj} house=\textsc{vcl} give.\textsc{prs}-\textsc{1sg} \\
\glt `{N}ow let us go, I will marry you off.'
\ex\label{Balochi:ex:12b}
Koroshi Balochi \il{Balochi!Koroshi}(\citealt{nourzaei_balochi_koroshi_2021}, A, 0022)\\
\gll man 'wad=om \textbf{as'p-ok-ā} 'gott a=kan-ān \\
\textsc{1sg} \textsc{refl}=\textsc{pc}.\textsc{1sg} horse-\textsc{def}-\textsc{obl} raise \textsc{vcl}=do.\textsc{prs}-\textsc{1sg} \\
\glt `I myself will raise the horse.'
\z
\z

With regard to nominal objects, the three dialects exhibit <10\% levels of post-verbal placement. However, in Coastal Balochi\il{Balochi!Coastal}, rates of post-verbal nominal objects are more than double the other two dialects. A pair-wise Fisher's Exact Test indicates that the difference between Coastal Balochi\il{Balochi!Coastal} and the other two reach significance (p<0.05). For pronominal objects, the differences are much more pronounced, and again, it is Coastal Balochi\il{Balochi!Coastal} that diverges from the other two.

\citet[112]{Korn2022Targets} had already noted that postverbal direct objects ``are surprisingly frequent'' in the geographically close variety of Southern Balochi that she investigates. Our data suggest that the spoken varieties of Southern Balochi, such as our ``Coastal \il{Balochi!Coastal} Balochi,'' may indeed differ from other varieties of Balochi, in particular with regard to pronominal objects. Other research (\citealt{stilo_preverbal_2018}, \citealt{haig_which_2023}) has suggested that pronominal objects are the most mobile, in the sense that they are more likely to shift across the predicate from the canonical \isi{object} position, and our findings provide further support for this assumption. 

As \citet[113]{Korn2022Targets} notes, post-posing direct objects is probably related to \isi{information structure}, but the exact nature of the triggering factors is ``not always obvious.'' The WOWA data base does provide a rough classification of direct objects into definite and indefinite, which we have analysed in \tabref{Balochi:tab:3}.

\begin{table}
    \begin{tabularx}{\textwidth}{l rYY rYY rYY r}
    \lsptoprule
& \multicolumn{3}{c}{Coastal} & \multicolumn{3}{c}{Koroshi} & \multicolumn{3}{c}{Turkmen} & \\
\cmidrule(lr){2-4}\cmidrule(lr){5-7}\cmidrule(lr){8-10}
& N & Po & \% & N & Po & \% & N & Po & \% & Totals \\
\midrule
Definite & 110 & 9 & 8 & 126 & 3 & 2 & 89 & 2 & 2 & 325 \\
Indefinite & 229 & 14 & 6 & 56 & 1 & 2 & 104 & 1 & 1 & 389 \\
\midrule
Totals & 339 &&& 182 &&& 193 &&& 714 \\
\lspbottomrule
    \end{tabularx}
    \caption{Definite vs. indefinite post-verbal nominal direct object frequencies}
    \label{Balochi:tab:3}
\end{table}

Based on the admittedly blunt \isi{instrument} of the \isi{definiteness} classification in WOWA, the distinction between definite and indefinite does not contribute much to the explanation. Either the absolute figures are too low (Koroshi\il{Balochi!Koroshi} and Turkmen\il{Balochi!Turkmen}), or do not reach significance (Coastal Balochi\il{Balochi!Coastal}). 

A second factor that is often claimed to be relevant in placement of direct objects is \isi{weight}. The WOWA data set distinguishes four levels of syntactic \isi{weight}, based on orthographic words excluding clitics and adpositions: 1, 2, 3, and >3.\footnote{An
    additional measure of \isi{weight} (number of characters) is also available, but was not applied here; see \citetv{chapters/1_Haigetal_Intro}.
} An example of a heavy \isi{direct object} (three words) is \ref{Balochi:ex:13}, a light \isi{direct object} is (\ref{Balochi:ex:12b}) above.

\ea\label{Balochi:ex:13}
Coastal Balochi \il{Balochi!Coastal}(\citealt{nourzaei_balochi_coastal_2021}, A, 0028)\\
\gll 'mã \textbf{ō'tī} \textbf{'mačč-e} \textbf{'hōš} de'gar-ā 'na-dāt-ag=ã \\
\textsc{1sg} \textsc{refl}.\textsc{gen} date\_palm-\textsc{gen} cluster other-\textsc{obl} \textsc{neg}-give.\textsc{pst}-\textsc{pp}=\textsc{cop}.\textsc{prs}.\textsc{3pl} \\
\glt `I didn't give \textbf{my date palm clusters} to anyone.'
\z

The figures comparing postverbal placement of the lightest (1 word) with the heaviest (>2 words) nominal direct objects are shown in \tabref{Balochi:tab:4} (all dialects combined).

\begin{table}
    \begin{tabularx}{.8\textwidth}{lYYY}
\lsptoprule
 & \multicolumn{3}{c}{All dialects} \\
  \cmidrule(lr){2-4}
 & N & Po & \% \\
\midrule
Light (1 word) & 329 & 16 & 4.9 \\
Heavy (>2 words) & 86 & 0 & 0 \\
\midrule
Totals & 415 \\
\lspbottomrule
    \end{tabularx}
    \caption{Frequencies of post-verbal nominal direct objects, light vs. heavy}
    \label{Balochi:tab:4}
\end{table}

\tabref{Balochi:tab:4} echoes findings from the WOWA spoken language corpora, which suggest that \isi{weight} is not a significant factor in triggering \isi{object} postposing (see \citetv{chapters/1_Haigetal_Intro}). In fact, the opposite tendency is suggested by our data.

Turning now to the factor of \isi{animacy}, here reduced to human versus non-human, there is an interaction between humanness, pronominality, and post-verbal placement. \tabref{Balochi:tab:5} provides the relevant figures:

\begin{table}
    \begin{tabularx}{\textwidth}{l YrY YrY Y}
\lsptoprule
 & & {+Hum}  & & & {-Hum} & \\
 \cmidrule(lr){2-4} \cmidrule(lr){5-7}
 & N & Po & \%Po & N & Po & \%Po & Totals \\
\midrule
Nominal & 171 & 10 & 5.8 & 543 & 20 & 3.7 & 714 \\
Pronominal & 142 & 27 & 19 & 31 & 2 & 6.5 & 173 \\
\midrule
Totals & 313 & & & 574 & & & 887 \\
\lspbottomrule
    \end{tabularx}
    \caption{Post-verbal direct objects (all doculects), according to humanness and pronominality}
    \label{Balochi:tab:5}
\end{table}

First, these figures suggest that there is a strong correlation between humanness and pronominality: more than 80\% of all pronominal objects are human (142 out of 173). Thus non-human pronominal objects are a rarity in Balochi, confirming a cross-linguistic tendency to avoid non-human object pronouns\is{pronoun!object} in discourse \citep{Haig.Stilo.Dogan.Schiborr2022}. Second, pronominality generally increases the likelihood of post-verbal placement, irrespective of humanness. Third, humanness alone is only marginally relevant: a human, nominal \isi{direct object} is not significantly more likely to be postverbal than a non-human, nominal \isi{object}. The difference between human and non-human only becomes relevant when the \isi{direct object} is pronominal. 

Closer inspection of the data reveal that the general tendency to avoid non-human object pronouns\is{pronoun!object} is most pronounced in Coastal Balochi\il{Balochi!Coastal}, where almost 90\% of object pronouns\is{pronoun!object} are human. Furthermore, in Coastal Balochi\il{Balochi!Coastal} the tendency to place these [+human] object pronouns\is{pronoun!object} after the verb reaches around 30\%, and it is the Coastal Balochi\il{Balochi!Coastal} data which actually account for most of the effects shown in \tabref{Balochi:tab:5}. In fact, Coastal Balochi\il{Balochi!Coastal} exhibits a nascent tendency towards the system found in Kumzari (\citealt{haig_kumzari_2022}, based on \citealt{anonby_grammar_2015}), where the majority of pronominal objects are post-verbal, and nominal objects are still predominantly pre-verbal.

\begin{sloppypar}
In sum, we have seen that all doculects exhibit dominant \isi{OV} order, but closer inspection reveals a difference between Coastal Balochi\il{Balochi!Coastal} and the other two: although all three dialects are dominant \isi{OV} overall, Coastal Balochi\il{Balochi!Coastal} shows a significantly higher number of postverbal objects than the other two, with the strongest effect found for human pronominal objects. This finding is difficult to reconcile with our earlier observation that Coastal Balochi\il{Balochi!Coastal} is more \isi{postpositional} than either of the other doculects (\sectref{Balochi:ss:3}). If it is more strictly \isi{postpositional}, we might have expected it to be more rigidly \isi{OV}, but this is not the case. We see no obvious explanation for the relatively high degree of \isi{object} mobility in Coastal Balochi\il{Balochi!Coastal}. But we have seen elsewhere that dominant use of postpositions does not necessarily entail strict \isi{OV}; both Georgian and Armenian are post-positional, yet \isi{object} positioning is actually more flexible than it is for the mostly prepositional Iranian languages (\citetv{chapters/13_Hodgsonetal_Armenian}). Conversely, strict \isi{OV} does not necessarily entail postpositions; witness some of the Iranian languages closer to the Mesopotamian core of WOWA, which are strictly \isi{OV} but dominant prepositional, or the varieties of Neo-Aramaic which have shifted to \isi{OV}, but remain prepositional (see \citetv{chapters/15_Noorlander_NAINEI}). While we have no explanation for the divergent characteristics of Coastal Balochi\il{Balochi!Coastal}, they further underscore the potential disconnect between adpositional order and verb/\isi{object} ordering in discourse.
\end{sloppypar}

Our data further suggest that \isi{weight} is not a significant factor in predicting post-verbal placement of objects. We have identified the difference between nominal and pronominal form as relevant, though only in conjunction with humanness. Basically, we find that neither humanness, nor pronominal form alone predicts post-verbal placement. However, with pronouns, a strong interaction with humanness becomes apparent; this tendency is strongest for Coastal Balochi\il{Balochi!Coastal}, but we can only speculate on the reasons for this at present.

\subsubsection{Copula complements}\label{Balochi:ss:4.2}

A similar picture emerges here as with direct objects: all dialects overwhelmingly prefer preverbal placement, as in the following examples:



\ea\label{Balochi:ex:14}
\ea\label{Balochi:ex:14a}
Coastal Balochi \il{Balochi!Coastal}(\citealt{nourzaei_balochi_coastal_2021}, A, 0299)\\
\gll \textbf{te'yōk-ā} bī \\
alone-\textsc{obl} \textsc{cop}.\textsc{pst}.\textsc{3sg} \\
\glt `{H}e was alone.'
\ex\label{Balochi:ex:14b}
Turkmen Balochi \il{Balochi!Turkmen}(\citealt{haig_balochi_2022}, A, 0154)\\
\gll \textbf{xānbādorr}=o man=on \\
Khanbadur=\textsc{foc} \textsc{1sg}=\textsc{cop}.\textsc{prs}.\textsc{1sg} \\
\glt `I am Khanbadur.'
\ex\label{Balochi:ex:14c}
Koroshi Balochi \il{Balochi!Koroshi}(\citealt{nourzaei_balochi_koroshi_2021}, A, 0006)\\
\gll ē 'šāh 'ǰan=e da'wom=ī \textbf{'xeylī} \textbf{ā'dam=e} \textbf{'xūb=ī} 'na-bod-ag=en \\
\textsc{prox} king wife=\textsc{ez} second=\textsc{pc}.\textsc{3sg} very person=\textsc{ez} good=\textsc{indv} \textsc{neg}-become.\textsc{pst}-\textsc{pp}=\textsc{cop}.\textsc{prs}.\textsc{3sg} \\
\glt `{T}his king's second wife was not a very good person.'
\z
\z

The relevant figures are provided in \tabref{Balochi:tab:6}.

\begin{table}
    \begin{tabularx}{\textwidth}{l rrr rrY rrY r}
    \lsptoprule
& \multicolumn{3}{c}{Coastal} & \multicolumn{3}{c}{Koroshi} & \multicolumn{3}{c}{Turkmen} & \\
\cmidrule(lr){2-4}\cmidrule(lr){5-7}\cmidrule(lr){8-10}
& N & Po & \% & N & Po & \% & N & Po & \% & Totals \\
\midrule
Copula complements & 310 & 13 & 4.2 & 53 & 0 & 0 & 66 & 2 & 3 & 429 \\
\lspbottomrule
    \end{tabularx}
    \caption{Frequencies of post-verbal copula complements}
    \label{Balochi:tab:6}
\end{table}

Examples of a post-verbal \isi{copula} from Coastal Balochi\il{Balochi!Coastal} and Turkmen\il{Balochi!Turkmen} are the following:

\ea\label{Balochi:ex:15}
Coastal Balochi \il{Balochi!Coastal}(\citealt{nourzaei_balochi_coastal_2021}, C, 0997)\\
\gll ke ā-'ī 'nām=a rahīm'baxš \\
\textsc{clm} \textsc{dist}-\textsc{gen} name=\textsc{cop}.\textsc{pst}.\textsc{3sg} Rahimbakhsh \\
\glt `who is called Rahimbakhsh'
\z


\ea\label{Balochi:ex:16}
Turkmen Balochi \il{Balochi!Turkmen}(\citealt{haig_balochi_2022}, A, 0022)\\
\gll man=om ast-om \textbf{pādišā=īē} \\
\textsc{1sg}=\textsc{add} exist.\textsc{prs}-\textsc{1sg} king=\textsc{indv} \\
\glt `I am a king.'
\z

\begin{sloppypar}
These findings confirm a broader tendency observable across the entire WOWA data base, namely that in dominant \isi{OV} languages, the position of copula complement\is{copula!complement}s is generally conservative in the sense that they are even less prone to post-posing than direct objects are. In our sample, we find near-categorical pre-verbal placement of copula complement\is{copula!complement}s.
\end{sloppypar}

\subsection{Verb and goal}\label{Balochi:ss:4.3}

\citet{Korn2022Targets} and \citet{Jahani2018Post-verbal} suggest that among \isi{endpoint} constituents, goals of verbs of movement (e.g., `go', `come') and goals of verbs of caused motion (e.g., `put', `throw', `bring') have the highest rates of post-predicate position among all \isi{argument} types, and across all data. Our data confirm this result. Examples of goals and goals of caused motion are the following:

\ea\label{Balochi:ex:17}
\ea\label{Balochi:ex:17a}
Goal \\
Coastal Balochi \il{Balochi!Coastal}(\citealt{nourzaei_balochi_coastal_2021}, A, 0043) \\
\gll 'šo \textbf{mē'tag-ā} \\
go.\textsc{pst}.\textsc{3sg} home-\textsc{obl} \\
\glt `{H}e went home.'
\ex\label{Balochi:ex:17b}
Goal \\
Koroshi Balochi \il{Balochi!Koroshi}(\citealt{nourzaei_balochi_koroshi_2021}, A, 0047)\\
\gll ma'rō 'raft-ay \textbf{lō'g-ā} \\
today go.\textsc{pst}-\textsc{2sg} home-\textsc{obl} \\
\glt `{T}oday (when) you go home.'
\ex\label{Balochi:ex:17c}
Goal \\
Turkmen Balochi \il{Balochi!Turkmen}(\citealt{haig_balochi_2022}, A, 0160)\\
\gll šot bi \textbf{yak} \textbf{ǰā=ē} \\
go.\textsc{pst}.\textsc{3sg} to one place=\textsc{indv} \\
\glt `{H}e went to a certain place.'
\ex\label{Balochi:ex:17d}
Caused Goal \\
Coastal Balochi \il{Balochi!Coastal}(\citealt{nourzaei_balochi_coastal_2021}, A, 0066)\\
\gll ǰat=e 'zahm=e \textbf{'dast=e}  \\
beat.\textsc{pst}=\textsc{pc}.\textsc{3sg} sword=\textsc{indv} hand=\textsc{pc}.\textsc{3sg} \\
\glt `{H}e struck the sword at its hand.'
\ex\label{Balochi:ex:17e}
Caused Goal \\
Koroshi Balochi \il{Balochi!Koroshi}(\citealt{nourzaei_balochi_koroshi_2021}, B, 0510)\\
\gll ma-ba'r-ā bod-a ma-prē'n-ā bod-a mā \textbf{dar'yā-hā} \\
\textsc{ipfv}-take.\textsc{prs}-\textsc{backg}.\textsc{3sg} become.\textsc{pst}-\textsc{pp} \textsc{ipfv}-throw.\textsc{prs}-\textsc{backg.3sg} become.\textsc{pst}-\textsc{pp} into sea-\textsc{obl} \\
\glt `{S}he used to take it [and] throw it into the sea.'
\ex\label{Balochi:ex:17f}
Caused Goal \\
Turkmen Balochi \il{Balochi!Turkmen}(\citealt{haig_balochi_2022}, D, 0609)\\
\gll mn-ā ǰat \textbf{be} \textbf{ḍigār-ā} \\
\textsc{1sg}-\textsc{obl} beat.\textsc{pst}.\textsc{3sg} on ground-\textsc{obl} \\
\glt `({T}he donkey) threw me onto the ground.'
\z
\z

However, the dialects in our sample differ in the degree to which goals are postposed. \tabref{Balochi:tab:7} compares nominal goals and caused-motion goals, excluding adverbs:.


\begin{table}
    \begin{tabularx}{\textwidth}{l rYY rYY rYY r}
    \lsptoprule
& \multicolumn{3}{c}{Coastal} & \multicolumn{3}{c}{Koroshi} & \multicolumn{3}{c}{Turkmen} & \\
\cmidrule(lr){2-4}\cmidrule(lr){5-7}\cmidrule(lr){8-10}
& N & Po & \% & N & Po & \% & N & Po & \% & Totals \\
 \midrule
Goal & 85 & 60 & 70.6 & 68 & 63 & 92.6 & 23 & 8 & 34.8 & 176 \\
Caused Goal & 21 & 6 & 28.6 & 17 & 13 & 76.5 & 13 & 7 & 53.8 & 51 \\
\midrule
Totals & 106  & & & 85  & & & 36 & & & 227 \\
\lspbottomrule
    \end{tabularx}
    \caption{Frequencies of post-verbal nominal goals}
    \label{Balochi:tab:7}
\end{table}

\begin{sloppypar}
For Koroshi, the results are typical for an Iranian language close to the Mesopotamian core of the Western Asian Transition Zone, and also for colloquial spoken Persian\il{Persian (colloquial)} (\citetv{chapters/1_Haigetal_Intro}): more than 80\% of all goals are post-verbal. For the other two dialects, the results are somewhat puzzling. For Turkmen\il{Balochi!Turkmen}, significantly lower rates of post-verbal Goal\is{Goal!post-verbal}s can plausibly be related to effects of Central Asian varieties of Turkic, in line with the predictions of \citet{haig_which_2023} and \citetv{chapters/1_Haigetal_Intro}, which suggest that the placement of Goals is highly sensitive to \isi{language contact}. However, the absolute number of goals in the Turkmen\il{Balochi!Turkmen} data set is quite low, so this is a provisional conclusion. For Coastal Balochi\il{Balochi!Coastal}, the most puzzling aspect is the overall low levels of postverbal caused goals, an observation that echoes \citegen{Korn2022Targets} finding for another Southern Balochi variety. For most data sets in the WOWA sample, the difference between caused and simple goals is not significant, and the two roles can be unified for most analyses. But for Coastal Balochi\il{Balochi!Coastal}, the difference is striking; we have no explanation for this; it definitely merits further investigation (verb-specific effects, for example).
\end{sloppypar}

\subsection{Recipients and addressees}\label{balochi:ss:4.4}

A number of previous studies have shown that goals, recipients, and addressees do not necessarily pattern alike in the \isi{OV} languages of Western Asia (\citetv{chapters/1_Haigetal_Intro}, \citealt{Korn2022Targets}, \citealt{stilo_preverbal_2018}). Particularly addressees may exhibit quite different \isi{word order} properties, even between closely related varieties (\citealt{Jahani2018Post-verbal} for Balochi, \citealt{haig_kurdish_2022} for Kurdish\il{Kurdish}). Small corpora of naturalistic spoken language are problematic for testing \isi{word order} of addressees and recipients, because relevant tokens are not particularly frequent, and often pronominal. In varieties with \isi{clitic} pronouns, the position of the \isi{pronoun} is determined by language-specific clitic-placement principles, which may be quite distinct from the principles governing \isi{word order} of prosodically independent constituents. Our findings here are correspondingly tentative. Examples of recipients and addressees are provided below, while \tabref{Balochi:tab:8} gives the frequencies per doculect.

\ea\label{Balochi:ex:18}
\ea\label{Balochi:ex:18a}
Recipient \\
Coastal Balochi \il{Balochi!Coastal}(\citealt{nourzaei_balochi_coastal_2021}, A, 0 527) \\
\gll k-ā'r-ã de'-yã \textbf{'pet-a} \\
\textsc{k.ipfv}-bring.\textsc{prs}-\textsc{3pl} give.\textsc{prs}-\textsc{3pl} father-\textsc{obl} \\
\glt `{T}hey bring [and] give [them] to [their] father.' 
\ex\label{Balochi:ex:18b}
Addressee \\
Coastal Balochi \il{Balochi!Coastal}(\citealt{nourzaei_balochi_coastal_2021}, A, 0023)\\
\gll o'š-ī \textbf{čo'k-ān} \\
say.\textsc{prs}-\textsc{3sg} child-\textsc{obl}.\textsc{pl} \\
\glt `{H}e says to [his] children.'
\ex\label{Balochi:ex:18c}
Recipient \\
Koroshi Balochi \il{Balochi!Koroshi}(\citealt{nourzaei_balochi_koroshi_2021}, B, 580)\\
\gll pu'l-ā a='dā \textbf{āle'm-ok-ā} \\
money-\textsc{obl} \textsc{vcl}=give.\textsc{prs}.\textsc{3sg} wise\_man-\textsc{def}-\textsc{obl} \\
\glt `{S}he gives the money to the doctor (lit. wise man).'
\ex\label{Balochi:ex:18d}
Addressee \\
Koroshi Balochi \il{Balochi!Koroshi}(\citealt{nourzaei_balochi_koroshi_2021}, A, 0320)\\
\gll 'ya 'rō 'šāh \textbf{ba} \textbf{dūmād-o'bār=ay} a='š-īt \\
one day king to son\_in\_law-\textsc{pl}=\textsc{pc}.\textsc{3sg} \textsc{vcl}=say.\textsc{prs}-\textsc{3sg} \\
\glt `{O}ne day, the king says to his sons-in-law.' 
% \ex\label{Balochi:ex:18e}
\z
\z


\begin{table}
    \begin{tabularx}{\textwidth}{l rYY rYY rYY r}
    \lsptoprule
& \multicolumn{3}{c}{Coastal} & \multicolumn{3}{c}{Koroshi} & \multicolumn{3}{c}{Turkmen} & \\
\cmidrule(lr){2-4}\cmidrule(lr){5-7}\cmidrule(lr){8-10}
& N & Po & \% & N & Po & \% & N & Po & \% & Totals \\
\midrule
Addressees & 40 & 16 & 40 & 12 & 4 & 33 & 5 & 0 & 0 & 57 \\
Recipients & 19 & 4 & 20 & 3 & 2 & 66 & 5 & 1 & 20 & 27 \\
\midrule
Totals & 59 & & & 15 & & & 10 & & & 84 \\
\lspbottomrule
    \end{tabularx}
    \caption{Frequencies of post-verbal nominal addressees and recipients (includes recipient/benefactives)}
    \label{Balochi:tab:8}
\end{table}

The findings do not entirely match those of \citet{Jahani2018Post-verbal} and \citet[109]{Korn2022Targets}, who suggest regular pre-verbal addressees in ``Southern Balochi'', to which our Coastal Balochi\il{Balochi!Coastal} would belong. The findings for Coastal Balochi\il{Balochi!Coastal} are also puzzling in view of a general tendency noted in \citetv{chapters/1_Haigetal_Intro}, according to which addressees are generally less likely to be post-verbal, which would align with the findings of \citet{Jahani2018Post-verbal} and \citet{Korn2022Targets}. However, in our Coastal Balochi\il{Balochi!Coastal} sample, this is not the case; we have no explanation for this mismatch; this requires further research.

Turning to pronominal addressees and recipients, \tabref{Balochi:tab:9} provides the relevant figures.


\begin{table}
    \begin{tabularx}{\textwidth}{l rYY rYY rYY r}
    \lsptoprule
& \multicolumn{3}{c}{Coastal} & \multicolumn{3}{c}{Koroshi} & \multicolumn{3}{c}{Turkmen} & \\
\cmidrule(lr){2-4}\cmidrule(lr){5-7}\cmidrule(lr){8-10}
& N & Po & \% & N & Po & \% & N & Po & \% & Totals \\
\midrule
Addressees & 13 & 2 & 15 & 8 & 1 & 1 & 6 & 0 & 0 & 27 \\
Recipients & 40 & 9 & 25 & 15 & 8 & 50 & 6 & 2 & 33 & 48 \\
\midrule
Totals & 43 & & & 23 & & & 12 & & & 75 \\
\lspbottomrule
    \end{tabularx}
    \caption{Frequencies of post-verbal pronominal addressees and recipients (includes recipient/benefactives)}
    \label{Balochi:tab:9}
\end{table}

With pronominal arguments, the expected trend for higher frequency of post-posed recipients is confirmed in all doculects. Examples of nominal addressees are provided in (\ref{Balochi:ex:19a}-\ref{Balochi:ex:19c}) and a pronominal example is shown in (\ref{Balochi:ex:19d}). 

\ea\label{Balochi:ex:19}
\ea\label{Balochi:ex:19a}
Coastal Balochi \il{Balochi!Coastal}(\citealt{nourzaei_balochi_coastal_2021}, B, 0504) \\
\gll 'nī lōṭā-'ēn-ī \textbf{brā't-ẫ} \\
now call-\textsc{caus}.\textsc{prs}-\textsc{3sg} brother-\textsc{obl}.\textsc{pl} \\
\glt `{T}hen he called the brothers.' 
\newpage
\ex\label{Balochi:ex:19b}
Coastal Balochi \il{Balochi!Coastal}(\citealt{nourzaei_balochi_coastal_2021}, A, 0188)\\
\gll 'nī ǰa'nek-ā go'š-ī \textbf{'to} 'dar ā \\
now girl-\textsc{obl} say.\textsc{prs}-\textsc{3sg} \textsc{2sg} \textsc{prev} \textsc{sbjv}.come.\textsc{prs}.\textsc{2sg} \\
\glt `{T}hen he [the boy in the well] said to the girl, ``You get out.'' 
\ex\label{Balochi:ex:19c}
Koroshi Balochi \il{Balochi!Koroshi}(\citealt{nourzaei_balochi_koroshi_2021}, A, 0320)\\
\gll 'ya 'rō 'šāh \textbf{ba} \textbf{dūmād-o'bār=ay} a='š-īt \\
one day king to son\_in\_law-\textsc{pl}=\textsc{pc}.\textsc{3sg} \textsc{vcl}=say.\textsc{prs}-\textsc{3sg} \\
\glt `{O}ne day, the king says to his sons-in-law.' 
\ex\label{Balochi:ex:19d}
Turkmen Balochi \il{Balochi!Turkmen}(\citealt{haig_balochi_2022}, D, 0540)\\
\gll mnī piss \textbf{pamman} sendbad-ī nakl-ā kort \\
\textsc{1sg}.\textsc{gen} father for.\textsc{1sg} Sindbad-\textsc{gen} story-\textsc{obl} do.\textsc{pst}.\textsc{3sg} \\
\glt `{M}y father told me the story of Sindbad.'
\z
\z

In summary, the findings for addressees and recipients largely confirm a trend observed across other Iranian languages of the region, according to which recipients are generally more likely to be post-verbal than addressees. However, in Coastal Balochi\il{Balochi!Coastal} for nominal arguments (\tabref{Balochi:tab:8}) only, the trend is reversed. We currently lack an explanation for this, which definitely requires more research.

\subsection{Complements of `become', Place, Source, Instrument, Benefactive, Comitative}\label{Balochi:ss:4.5}
\largerpage
The absolute numbers of tokens for the remaining roles are quite low in several doculects, so quantitative analysis is not always meaningful. We have combined the results in an overview \tabref{Balochi:tab:10}. Due to low absolute figures, we include all possible POS types, including pronouns, adverbs etc. \figref{Balochi:fig:2} provides an overview visualization of all roles considered.


\begin{table}
    \begin{tabularx}{\textwidth}{l rYr rYr rYY r}
    \lsptoprule
& \multicolumn{3}{c}{Coastal} & \multicolumn{3}{c}{Koroshi} & \multicolumn{3}{c}{Turkmen} & \\
\cmidrule(lr){2-4}\cmidrule(lr){5-7}\cmidrule(lr){8-10}
& N & Po & \% & N & Po & \% & N & Po & \% & Totals \\
\midrule
`become'-compl. & 14 & 1 & 7 & 12 & 4 & 33 & 10 & 0 & 0 & 36 \\
Place & 61 & 12 & 19.7 & 21 & 7 & 33.3 & 25 & 0 & 0 & 107 \\
Source & 12 & 1 & 8.3 & 7 & 3 & 43.9 & 21 & 2 & 9.5 & 40 \\
Instrument & 18 & 5 & 28 & 5 & 0 & 0 & 12 & 1 & 8 & 35 \\
Comitative & 18 & 3 & 17 & 13 & 4 & 30 & 18 & 2 & 11 & 49 \\
Benefactive & 13 & 2 & 15 & 12 & 1 & 8 & 2 & 0 & 0 & 27 \\
\midrule
\lspbottomrule
    \end{tabularx}
    \caption{Other roles: frequencies of post-verbal placement}
    \label{Balochi:tab:10}
\end{table}

\begin{sloppypar}
With regard to complements of change-of-state verbs (`become'), Coastal Balochi\il{Balochi!Coastal} and Turkmen\il{Balochi!Turkmen} are overwhelmingly pre-verbal, while Koroshi\il{Balochi!Koroshi} shows around one third post-verbal placement. This would be expected given that the attested Iranian languages which have dominant post-verbal `become'-complements are all from the southwestern Mesopotamian periphery of WATZ (\citealt{Haig.Stilo.Dogan.Schiborr2022}, \citetv{chapters/1_Haigetal_Intro}). An example of post-verbal `become'-\isi{complement} from Koroshi\il{Balochi!Koroshi} is provided in (\ref{Balochi:ex:20}).
\end{sloppypar}

\ea\label{Balochi:ex:20}
Koroshi Balochi \il{Balochi!Koroshi}(\citealt{nourzaei_balochi_koroshi_2021}, A, 0491) \\
\gll ba ha'm=ī kasa-ō-'ēn ga'hār=eš a='b-ant \textbf{ka'nīz=o} \textbf{naw'kar} \\
to \textsc{emph}=\textsc{prox} small-\textsc{dim}-\textsc{attr} sister=\textsc{pc}.\textsc{3pl} \textsc{vcl}=become.\textsc{prs}-\textsc{3pl} maidservant=and male\_servant \\
\glt `[You know, these six sons-in-law and their wives came and] became servants to this  their youngest sister.'
\z

Turning to local roles source and place, it has been suggested that \isi{oblique} arguments generally tend to prefer post-predicate position (\citealt{jing_word_2021}, based predominantly on written-language corpora). Above, we have shown that this certainly applies to goals, but the data for source and place provide only weak support for assuming a general tendency applying to all obliques. Nevertheless, it is noteworthy that the other obliques in \tabref{Balochi:tab:10} do show notably higher rates of post-verbal placement than direct objects. Examples for place and source are the following:

\ea\label{Balochi:ex:21}
\ea\label{Balochi:ex:21a}
Coastal Balochi \il{Balochi!Coastal}(\citealt{nourzaei_balochi_coastal_2021}, A, 0327) \\
\gll ha'm=ē 'gōšt pa'dā kašt=ī \textbf{da'p-ā} \textbf{'če} \\
\textsc{emph}=\textsc{prox} meat again pull.\textsc{pst}.\textsc{3sg}=\textsc{pc}.\textsc{3sg} mouth-\textsc{obl} from \\
\glt `{I}t took out this meat from its mouth again.' 
\ex\label{Balochi:ex:21b}
Coastal Balochi \il{Balochi!Coastal}(\citealt{nourzaei_balochi_coastal_2021}, A, 0063)\\
\gll 'hanga 'nešt ha'm=ē \textbf{'mačč-e} \textbf{'čērā} \\
still sit.\textsc{pst}.\textsc{3sg} \textsc{emph}=\textsc{prox} date\_palm-\textsc{gen} under \\
\glt `{S}till he [the boy] was sitting under this date-palm.'
\ex\label{Balochi:ex:21c}
Koroshi Balochi \il{Balochi!Koroshi}(\citealt{nourzaei_balochi_koroshi_2021}, A, 0217)\\
\gll 'kār a=kan-ān \textbf{'mā} \textbf{'ī} \textbf{bā'ġ-ā} \\
work \textsc{vcl}=do.\textsc{prs}-\textsc{1sg} in \textsc{prox} garden-\textsc{obl} \\
\glt `I will work in this garden.'
\ex\label{Balochi:ex:21d}
Koroshi Balochi \il{Balochi!Koroshi}(\citealt{nourzaei_balochi_koroshi_2021}, B, 0612)\\
\gll \textbf{ba} \textbf{mad'rasā=om} ġada'ġan=eš kod-a \\
to school.\textsc{obl}=\textsc{add} forbidden=\textsc{pc}.\textsc{3pl} do.\textsc{pst}-\textsc{pp} \\
\glt `{A}t school they have actually forbidden (your) coming.'
\z
\z

The data for instruments, comitatives, and benefactives are quite thin, but suggest a tendency for all three roles to be predominantly pre-verbal. This is particularly noteworthy for benefactives, which are >80\% pre-verbal in all doculects. Benefactives are sometimes included under the umbrella term of ``target'' (\citealt{asadpour_typologizing_2022,asadpour_word_2022}, \citealt{Korn2022Targets}) together with goals, but the Balochi data suggest that benefactives are far less likely to be postverbal than goals, and in fact no more likely to be postverbal than sources and locations. However, it should be noted that the majority of benefactives (about 90\%) in our data are pronominal, so the bias towards preverbal position may be linked to the pronominal status, but given the paucity of nominal benefactives in the data, this remains to be clarified.

\section{Summary: Post-verbal constituents in Balochi}\label{Balochi:ss:5}

\figref{Balochi:fig:2} summarizes the quantitative data from post-verbal positioning of various non-subject constituents considered in the preceding section. The values for the roles Recipient, Addressee, Benefactive, and Comitative are combined to ``Oblique'' in \figref{Balochi:fig:2}; see the preceding section for individual roles.

\begin{figure}
    % \includegraphics[width=\textwidth]{figures/Balochi_fig2.png}
    \pgfplotstableread{data/ch4-fig2.csv}\ChapterFourFigureTwoData
    \begin{tikzpicture}
	\small
	\begin{axis}
		[
            axis lines*=left,
			bar width=3ex,
			font=\small,
			height=6cm,
			legend cell align=left,
			legend pos=north west,
			nodes near coords,
			width=\textwidth,
			xtick=data,
			xticklabels from table={\ChapterFourFigureTwoData}{Data},
			x tick label style={text width=1.25cm, align=center, font=\sloppy\small},
			y tick label style={font=\small},
			point meta=explicit,
            ybar,
			ymin=0,
			ymax=100,
			ylabel=\%,
			ylabel near ticks
		]
		\foreach \i in {Coastal,Koroshi,Turkmen}
		  {
		  	\addplot table [x expr=\coordindex, x=Data, y=\i, meta=\i N] {\ChapterFourFigureTwoData};
		    \edef\temp{\noexpand\addlegendentry{\i}}
		    \temp
 		  }
	\end{axis}
    \end{tikzpicture}
    \caption{Frequency of post-verbal arguments in three Balochi doculects. \emph{Note}: Ticks above bars indicate absolute count.}
    \label{Balochi:fig:2}
\end{figure}

As expected, the highest rates of post-verbal placement for all doculects are found for goals, confirming the by-now familiar finding across the entirety of WOWA. Note that it is only in Koroshi\il{Balochi!Koroshi} that both recipients and goals are dominantly (>60\%) post-verbal, a pattern that has parallels with varieties of Kurdish\il{Kurdish} (\citealt{haig_kurdish_2022}). In our data, addressees do not pattern with goals in any doculect, and overall do not differ from other obliques, showing less than 50\% post-verbal placement. 
With regard to direct objects, all doculects are overwhelmingly \isi{OV}, with post-verbal objects accounting for less than 10\% of all objects. However, Coastal Balochi\il{Balochi!Coastal} shows a significantly higher number of post-verbal objects than either of the other two doculects. In our sample, \isi{weight} and \isi{definiteness} do not seem to be relevant in accounting for post-verbal direct objects, in any doculect. We do, however, find an effect of humanness, particularly in combination with pronominal direct objects (\tabref{Balochi:tab:5} above). Overall, the most likely direct objects to be postposed are human, pronominal objects, with the difference most obvious for Coastal Balochi\il{Balochi!Coastal}.

\section{Conclusions and implications for comparative Iranian syntax}\label{Balochi:ss:6}

\tabref{Balochi:tab:11} summarizes the overall degree of \isi{head-initial} syntax in various phrase types in the three doculects, based on the discussion in \sectref{Balochi:ss:3}.

\begin{table}
    \begin{tabular}{r|c|c|c|}
    \toprule
    \hline
\strut Parameter & Koroshi & Turkmen & Coastal \\
\hline
% \cline{2-4}
\strut N/Adjective & N-Adj & \multicolumn{2}{c|}{Adj-N} \\ \cline{2-4}
\strut N/Possessor & N-Poss & \multicolumn{2}{c|}{Poss-N} \\ \cline{2-4}
\strut Adp/N & Prep-N & Prep-N & N-Postp \\ \cline{2-4}
\strut Main clause/Compl. clause & \multicolumn{3}{c|}{Main-Compl. Clause} \\ \cline{2-4}
\strut Aux(Modal)/Main verb & \multicolumn{3}{c|}{Aux(Modal)-Main verb} \\ \cline{2-4}
\strut Complementizer/Clause & \multicolumn{3}{c|}{Compl-Clause} \\
\hline
\bottomrule
    \end{tabular}
    \caption{Distribution of head-initial syntax in three Balochi doculects}
    \label{Balochi:tab:11}
\end{table}

Across these typological parameters, Koroshi\il{Balochi!Koroshi} is the only variety that is consistently \isi{head-initial}, presumably linked to a greater degree of contact with Persian\il{Persian} and other western Iranian languages exhibiting \isi{head-initial} NPs. All doculects linearize main and \isi{complement} clauses, complementizers, and auxiliaries alike, sharing these values with probably all other Iranian languages of WOWA. The most surprising aspect of \tabref{Balochi:tab:11} is the dominance of postpositions in Coastal Balochi\il{Balochi!Coastal}, for which we have no ready explanation; this is certainly an area for future research.

Finally, we consider the relevance of the Balochi data for addressing the larger question of how post-verbal syntax in Iranian may have emerged: What kind of historical scenario is compatible with the areal distribution that is now becoming evident? First, let us consider the Balochi data in the light of a feature that characterizes all other contemporary spoken western Iranian language in the WOWA sample: in connected discourse, nominal goals (including caused-motion goals) are 60--100\% post-verbal. The sole exception turns out to be Turkmen\il{Balochi!Turkmen} Balochi (Turkmen), where the figure drops to around 40\%. Turkmen\il{Balochi!Turkmen} Balochi is thus different from its relatives in this respect, and from any of the Iranian languages spoken in Mesopotamia, where post-verbal Goal\is{Goal!post-verbal}s are >85\%. The most obvious explanation for the outlier status of Turkmenistan Balochi is its outlier geographic position, beyond the northeastern periphery of WATZ, and farthest from the Mesopotamian core that has been identified as harbouring the highest levels of post-verbal Goal\is{Goal!post-verbal}s. We take this as tentative support for the hypothesis that the placement of goals is particularly sensitive to \isi{contact influence}.

Nevertheless, 40\% post-verbal Goal\is{Goal!post-verbal}s is higher than many other \isi{OV} languages in our sample (vernacular standard Turkish of Ankara, \citetv{chapters/2_Iefremenko_Bilingual} for example), so geography cannot be the whole story. Generally, it appears that Iranian languages are just more prone to post-verbal Goal\is{Goal!post-verbal}s than Turkic languages, and this is a matter of shared inheritance, rather than geography alone. What we might therefore assume is that the shared ancestor of all western Iranian languages had some initial baseline level of post-verbal spatial goals, which may already have exceeded 50\% in some of the (unattested) languages of that period. This feature was thus present prior to the dispersion of the various branches of what is traditionally termed `West Iranian' (see also \citealt{Korn2022Targets}: 122). The result would be traces of post-verbal Goal\is{Goal!post-verbal}s (in the narrow sense) in all West Iranian languages, with actual levels dependent on their respective contact biographies over the last 2000 years. In those Iranian languages that shared territory with with Semitic languages (in particular Aramaic) for at least a millennium in the Mesopotamian and western Zagros regions, levels of post-verbal Goal\is{Goal!post-verbal}s converge to the near 100\% that characterize the Semitic contact languages. Where Semitic influence is lacking, and contact with other \isi{OV} languages (e.g. Indo-Aryan) was present, levels would have dropped, as in Balochi of Turkmenistan, or at least not risen further. Alternatively, the unrelated \isi{OV} contact languages might converge to Iranian in this respect, as is the case for Qashqai Turkic, which exhibits 70\% post-verbal nominal goals \citep{schreiber_oghuz_2021}, a figure close to its Iranian neighbours. The actual outcome would thus depend on the local patterns of \isi{multilingualism} and power relations among the speech communities, but a broad geographical tendency is nonetheless discernible.

This is essentially a refinement of the account outlined in \citet{haig_verb-goal_2015}, who noted that the wide distribution of post-verbal ``Goals'' (see below on terminology) across the Iranian languages of western Asia is suggestive of ``an old trait of West Iranian that was inherited by the daughter languages,'' rather than independent parallel development \citep[421]{haig_verb-goal_2015}. post-verbal Goal\is{Goal!post-verbal}s are attested in the Old Iranian period, as in the following Old Persian\il{Persian (Old)} examples, cited from \citet[421]{haig_verb-goal_2015}; see also \citet{Jahani2018Post-verbal} for further examples: 

\ea\label{Balochi:ex:22}
\ea\label{Balochi:ex:22a}
Old Persian \il{Persian (Old)}(DB V, 9–10, \citealt{kent_old_1953}: 133) \\
\gll pasāva Gaubaruva hadā kārā ašiyava \textbf{ūvjam} \\
then Gobryas with army marched to\_Elam \\
\glt `Then Gobryas marched to Elam with an army.'
\ex\label{Balochi:ex:22b}
Old Persian \il{Persian (Old)}(DB V, 21, \citealt{kent_old_1953}: 133) \\
\gll pasāva hadā kārā ašiyavam abiy \textbf{sakām} \\
then with army went.\textsc{1sg} to Scythia \\
\glt `Then I went with an army to Scythia.'
\z
\z

However, the formulation in \citet{haig_verb-goal_2015} was still framed in terms of an assumed macro-\isi{role} sense of ``Goal'', which included addressees, and recipients, among other things. More recent research (\citetv{chapters/2_Iefremenko_Bilingual}), including this chapter, demonstrates that this approach is not tenable. Instead, a finer-grained approach is required that consistently distinguishes \isi{Goal} in the narrower sense of motion and caused motion from other roles. As shown above in \figref{Balochi:fig:2}, outside of spatial goals, no other \isi{role} surpasses 50\% rates of post-verbal placement in Balochi, with the exception of \isi{recipient}, and that only in Koroshi\il{Balochi!Koroshi}, and this finding is replicated for several other Iranian languages.

Our revised suggestion would be that in proto-West Iranian, post-verbal Goal\is{Goal!post-verbal}s (in the narrow sense) were already a frequent option (perhaps majority), while postposing other arguments was possible, but determined by pragmatic and semantic principles, with no clear role-driven preferences; the multi-variate analysis of spoken Persian\il{Persian (colloquial)} in \citetv{chapters/7_RasekhMahandetal_Persian}, demonstrates such a system empirically. In some Iranian languages, most notably Kurdish\il{Kurdish} of the Mesopotamian region, post-verbal placement spread from goals to other \isi{argument} types, ultimately encompassing a bundle of semantically-related roles. The initial \isi{focus} on these languages led to the (misleading) assumption of a macro-\isi{role} (``Goal'' in \citealt{haig_post-predicate_2014} and \citealt{haig_verb-goal_2015}; ``Target'' in \citealt{asadpour_word_2022} and \citealt{jugel_word_2022}; see \citetv{chapters/1_Haigetal_Intro}, for discussion. The role-specific analysis adhered to here suggests an areally-mediated shift in the frequency of post-verbal Goal\is{Goal!post-verbal}s in Balochi (lowest in Turkmen\il{Balochi!Turkmen}), and in the range of arguments that behave similarly to goals (the Koroshi\il{Balochi!Koroshi} clustering of \isi{recipient} and \isi{Goal}).

The developments just sketched are speculative to the extent that they reconstruct possible diachronic developments, based on synchronic feature distributions. The little historical data that is available is unfortunately difficult to interpret. There is a general problem with projecting from written texts to spoken language; an investigation of contemporary formal written Persian\il{Persian}, for example, would fail to \isi{register} the fact that spoken contemporary Persian\il{Persian} has a fairly stable rate of around 80\% post-verbal Goal\is{Goal!post-verbal}s (\citetv{chapters/1_Haigetal_Intro}). Early Classical Persian\il{Persian (Early New)} texts (11--13 centuries AD) have virtually no trace of post-verbal Goal\is{Goal!post-verbal}s (\citetv{chapters/8_Parizadeh_ENP}), but we doubt that these figures reflect the spoken language of the time, given the comparative evidence from other Iranian languages. Similarly, \citet{Jahani2018Post-verbal} compares oral and written Southern Balochi, noting that in written Balochi, goals are never post-posed, while in spoken Southern Balochi the relevant figure is 90\%. These differences raise a methodological problem with regard to reconstructing the history of post-verbal syntax; differences between yesterday's written and today's spoken language may just reflect differences between written and spoken registers that have always existed, rather than being evidence of any change across time.

This chapter provides a tentative cross-dialect survey of \isi{word order} in Balochi, based on three doculects representing geographically dispersed varieties, with differing contact profiles. The picture that emerges confirms some of the previous observations in \citet{Jahani2018Post-verbal} and \citet{Korn2022Targets}, but extends the purview of these studies. We investigated a larger number of roles, and considered additional factors such as \isi{weight} and humanness. On the other hand, the pan-dialectal approach means that we are unable to zoom in on finer semantic, pragmatic, or stylistic factors, such as the effects of individual verb classes, \isi{register}, or \isi{information structure} that figure in \citet{Korn2022Targets}. 
Generally, our findings confirm the overall prediction that both overall frequencies of post-verbal Goal\is{Goal!post-verbal}s, and the number (and nature) of other \isi{role} types that are prone to post-verbal placement, decrease with increasing distance from the Mesopotamian core of WATZ. We have also identified Coastal Balochi\il{Balochi!Coastal} as exhibiting a significantly higher frequency of postpostions in discourse, and interestingly, a higher number of post-verbal direct objects, with a strong interaction with pronominal form and humanness. These findings raise interesting questions regarding the possible \isi{role} of contact with Indo-Aryan languages in the region, which have yet to be explored.

\section*{Abbreviations}
\begin{tabularx}{.45\textwidth}{lQ}
\textsc{add} & additive particle \\
\textsc{attr} & attributive \\
\textsc{backg} & backgrounding \\
\textsc{caus} & causative \\
\textsc{clm} & complementizer \\
Coastal & Coastal Balochi doculect \\
\textsc{cop} & {copula} \\
\textsc{def} & definite \\
\textsc{dist} & distal \\
\textsc{emph} & {emphasis} \\
\textsc{ez} & ezafe particle \\
\textsc{foc} & {focus} particle \\
\textsc{gen} & genitive  \\
\textsc{ind} & indefinite \\
\textsc{indv} & individuation {clitic} \\
\textsc{ipfv} & imperfective \\
\textsc{k.ipfv} & \textit{k}-variant of imperfective \\
Koroshi & Koroshi Balochi doculect \\
\end{tabularx}
\begin{tabularx}{.45\textwidth}{lQ}
\textsc{neg} & negation \\
\textsc{obj} & {object} case \\
\textsc{obl} & {oblique} case \\
\textsc{pc} & person-marking enclitic \\
\textsc{pl} & plural \\
\textsc{pp} & past participle \\
\textsc{proh} & prohibitive prefix \\
\textsc{prox} & proximal  \\
\textsc{prs} & present \\
\textsc{pst} & past stem \\
\textsc{refl} & reflexive {pronoun} \\
\textsc{sbjv} & subjunctive \\
\textsc{sg} & singular \\
Turkmen & Turkmenistan Balochi doculect \\
UP & Unpublished text \\
\textsc{vcl} & verbal {clitic} \\
WATZ & Western Asian Transition Zone\\
\\
\end{tabularx}

\sloppy
\printbibliography[heading=subbibliography,notkeyword=this]

\end{document}
