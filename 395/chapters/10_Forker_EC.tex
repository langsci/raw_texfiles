\documentclass[output=paper,colorlinks,citecolor=brown,draftmode]{langscibook}
\ChapterDOI{10.5281/zenodo.14266349}
\author{Diana Forker\orcid{0000-0003-4247-9163}\affiliation{Friedrich Schiller University Jena}}
\title{Post-predicate elements in Kartvelian and East Caucasian}
\abstract{Kartvelian (or South Caucasian) and East Caucasian (or Nakh-Daghestanian) languages are usually described as ``flexible SOV'' languages which allow all logically possible word order permutations in main clauses. In this paper, I explore post-predicate elements in both language families and show that, in general, post-predicate elements are common in natural texts and influenced to various degrees by features such as genre/style, semantic role, information structure, heaviness and also language contact.}

%move the following commands to the ``local...'' files of the master project when integrating this chapter
% \usepackage{tabularx}
% \usepackage{langsci-optional}
% \usepackage{langsci-gb4e}
% \usepackage{enumitem}
% \bibliography{localbibliography}
% \newcommand{\orcid}[1]{}
% \let\eachwordone=\itshape
\IfFileExists{../localcommands.tex}{
 \addbibresource{../collection_tmp.bib}
 \addbibresource{../localbibliography.bib}
 % add all extra packages you need to load to this file

\usepackage{tabularx,multicol}
\usepackage{url}
\urlstyle{same}

\usepackage{listings}
\lstset{basicstyle=\ttfamily,tabsize=2,breaklines=true}

\usepackage{langsci-basic}
\usepackage{langsci-optional}
\usepackage{langsci-lgr}
\usepackage{langsci-osl}
% \usepackage{./langsci/styles/langsci-lgr}
% \usepackage{./langsci/styles/langsci-osl}
% \usepackage{langsci-gb4e}

\usepackage{tikz}
\usetikzlibrary{patterns,calc}
\pgfdeclarepatternformonly{south east lines}{\pgfqpoint{-0pt}{-0pt}}{\pgfqpoint{3pt}{3pt}}{\pgfqpoint{3pt}{3pt}}{
    \pgfsetlinewidth{0.6pt}
    \pgfpathmoveto{\pgfqpoint{0pt}{3pt}}
    \pgfpathlineto{\pgfqpoint{3pt}{0pt}}
    \pgfpathmoveto{\pgfqpoint{.2pt}{-.2pt}}
    \pgfpathlineto{\pgfqpoint{-.2pt}{.2pt}}
    \pgfpathmoveto{\pgfqpoint{3.2pt}{2.8pt}}
    \pgfpathlineto{\pgfqpoint{2.8pt}{3.2pt}}
    \pgfusepath{stroke}}
    
\usepackage{stmaryrd}
\usepackage{wasysym}
\usepackage{multirow}
\usepackage{caption}
\usepackage{subcaption}
\usepackage{mathrsfs}
\usepackage{qtree}

\usepackage{linguex}


 %pminos do not split footnotes
% \interfootnotelinepenalty=10000 %Footnote in Laporte chapters has to be split SN


%\DeclareIndexNameFormat{default}{%
%\nameparts{#1}%
%\usebibmacro{index:name}%
%{\index[names]}%
%{\namepartfamily}%
%{\namepartgiveni}%
% {}% L1
% {}% L2
%{\namepartprefix}% generates spurious space L3
%{\namepartsuffix}% generates spurious space L4
%}

%  {\DeclareIndexNameFormat{default}{%
%     \usebibmacro{index:name}{\index[names]}{#1}{#3}{#5}{#7}}}

%\DeclareIndexNameFormat{default}{%
%  \usebibmacro{index:name}{\sindex[nom]}{#1}{#3}{#5}{#7}}

%\DeclareIndexNameFormat{default}{%
%  \usebibmacro{index:name}{\sindex[person]}{#1}{#3}{#5}{#7}}
%\DeclareIndexNameFormat{default}{%
%\nameparts{#1} \usebibmacro{index:name}{\sindex[person]]}{\namepartfamily}{‌​\namepartgiven}{\nam‌​epartprefix}{\namepa‌​rtsuffix}}

%\newcommand{\smiley}{:)}

%\renewbibmacro*{index:name}[5]{%
%\usebibmacro{index:entry}{#1}%
%{\iffieldundef{usera}{}{\thefield{usera}\actualoperator}\mkbibindexname{#2}{#3}{#4}{#5}}}

% \newcommand{\noop}[1]{}

%remove for final
%\overfullrule=1mm

\newcommand{\tobi}[2]}}
\renewcommand{\S}[1]{\tobi{#1}{\textsc{*}}}

% this volume references
% puts: [this volume]
% already defined: \citetv
%\newcommand{\citepv}[1]{(\citeauthor{#1} \citeyear*{#1} [this volume])}
\newcommand{\citealtv}[1]{\citeauthor{#1} \citeyear*{#1} [this volume]}

%parentheses around example number
\newcommand{\pref}[1]{(\ref{#1})}

% in-text examples

\newcommand{\lnex}[1]{\textit{#1}} %target lang word
\newcommand{\lnlit}[1]{(lit.: `#1')} %literal reading
\newcommand{\lnlat}[1]{(#1)} % latinization
\newcommand{\lntrans}[1]{`#1'} %translation
\newcommand{\lnexl}[2]%
{\lnex{#1}{} \lnlat{#2}} % ex with latinization
\newcommand{\lnexlat}[3]{\lnex{#1}{} \lnlat{#2}{} \lntrans{#3}} % ex with latinization and tranl.

%ch01
\newcommand{\co}[1]{\mbox{\textbf{#1}}}

%ch09

\newcommand{\cyrbulg}[1]{\begin{otherlanguage*}{bulgarian}#1\end{otherlanguage*}}


%ch10
\newcommand{\nlp}{{\small NLP}}
\newcommand{\mwe}{{\small MWE}}
\newcommand{\rae}{{\small RAE}}
\newcommand{\lvc}{{\small LVC}}
\newcommand{\pos}{{\small P}o{\small S}}
%\newcommand{\todo}[1]{ \textcolor{red}{#1} }

%\renewcommand{\labelenumi}{\theenumi}
%\ainamefmt{{vv}{ll}{, ff}{, jj}} % fullname

\newcommand{\biberror}[1]{{\color{red}#1}}

\newcommand{\osenovaitem}{--~}
 %% hyphenation points for line breaks
%% Normally, automatic hyphenation in LaTeX is very good
%% If a word is mis-hyphenated, add it to this file
%%
%% add information to TeX file before \begin{document} with:
%% %% hyphenation points for line breaks
%% Normally, automatic hyphenation in LaTeX is very good
%% If a word is mis-hyphenated, add it to this file
%%
%% add information to TeX file before \begin{document} with:
%% %% hyphenation points for line breaks
%% Normally, automatic hyphenation in LaTeX is very good
%% If a word is mis-hyphenated, add it to this file
%%
%% add information to TeX file before \begin{document} with:
%% \include{localhyphenation}
\hyphenation{
    Beck-man
    Ngu-yen
    back-chan-nel
    back-chan-nels
    mo-not-o-nous
    ste-reo-typ-i-cal
}

\hyphenation{
    Beck-man
    Ngu-yen
    back-chan-nel
    back-chan-nels
    mo-not-o-nous
    ste-reo-typ-i-cal
}

\hyphenation{
    Beck-man
    Ngu-yen
    back-chan-nel
    back-chan-nels
    mo-not-o-nous
    ste-reo-typ-i-cal
}

%  \boolfalse{bookcompile}
%  \togglepaper[5]%%chapternumber
}{}


\colorlet{lightgray}{gray!40}
% \definecolor{lightgray}{rgb}{0.83, 0.83, 0.83}
\begin{document}
\maketitle\label{WOWA:ch:10}

\section{Introduction}\label{EC:ss:1}

Kartvelian (or South Caucasian) and East Caucasian (or Nakh-Daghestanian) languages are two of the three indigenous language families of the Caucasus. Kartvelian is the largest indigenous family in the Caucasus in terms of speakers, mainly due to Georgian\il{Kartvelian!Georgian}, which is the national language of the Republic of Georgia. East Caucasian is the largest Caucasian family in terms of numbers of languages. In both language families \isi{word order} has been studied, and the general consensus is that the languages have free \isi{word order} at the clausal level, with SOV being in some way classified as basic. In this paper, I want to discuss post-predicate items in both families based on the available literature and corpus data. I will examine a number of features that influence the availability of elements after the verb:

\begin{itemize}
\item morphosyntactic properties (grammatical function, word class)
\item pragmatic properties (\isi{heaviness}, \isi{information structure})
\item extralinguistic features (\isi{language contact})
 
\end{itemize}

I will concentrate on Georgian\il{Kartvelian!Georgian} as representative of Kartvelian, but also include data from Laz\il{Kartvelian!Laz} and two other Kartvelian languages. With respect to East Caucasian I will rely on corpus data from Sanzhi Dargwa\il{Caucasian (East)!Sanzhi Dargwa}, Chirag Dargwa\il{Caucasian (East)!Chirag Dargwa}, Tabasaran\il{Caucasian (East)!Tabasaran} and Hinuq\il{Caucasian (East)!Hinuq}. However, the corpus data originate from different sources and have been annotated in different ways, so the results are not always directly comparable. This paper pursues the gradient approach to \isi{word order} as advocated by \citet{levshina_why_2023}, and generally applied in this volume. 


\section{Post-predicate items in Kartvelian}\label{EC:ss:2}

\subsection{Word order profile of Kartvelian languages}\label{EC:ss:2.1}

Kartvelian languages have \isi{head-final} noun phrases with e.g., demonstratives and numerals preceding the noun, but admit some exceptions. In all languages, it is \isi{possessive} pronouns that are most commonly positioned after the noun.

Megrelian\il{Kartvelian!Megrelian} (or Mingrelian) and to a lesser extent also Georgian\il{Kartvelian!Georgian} can have postnomial modifiers such as adjectives and relative clauses and partially also modifying genitives (\citealt{aronson_modern_1991}, \citealt{harris_mingrelian_1991}, \citealt{pourtskhvanidze_fokuspartikeln_2015}: 169--170), though postposed adjectives and genitives in Georgian\il{Kartvelian!Georgian} are described as archaic and following Old Georgian\il{Kartvelian!Georgian (Old)} patterns (\citealt{testelets_wordKartvelian_1998}). In the development of Modern Georgian\il{Kartvelian!Georgian} from Old Georgian\il{Kartvelian!Georgian (Old)}, a clear shift from \isi{head-initial} to \isi{head-final} order in the noun phrase has been observed (see the references in \citealt{testelets_wordKartvelian_1998}). In Laz\il{Kartvelian!Laz}, the tendency to postpose \isi{possessive} pronouns is so strong that occasionally, possessives pronouns may even follow a postposition (\citealt{holisky_laz_1991}). In Svan\il{Kartvelian!Svan}, postposed modifiers, including \isi{possessive} pronouns, are very rare and archaic and seem to be restricted to poetry and lyrics (\citealt{schmidt_svan_1991}: 537; \citealt{tuite_short_1998}; see \citealt{testelets_wordKartvelian_1998} for examples). 

Kartvelian languages have postpositions (with a few exceptions, for Georgian\il{Kartvelian!Georgian} see \citealt{harris_word_2000}). Auxiliaries follow the lexical verb, which is usually in a non-finite form, but the reverse order is not unusual (\citealt{harris_word_2000}). The position of relative clauses depends on the formal type of relative clause. Relative clauses built with participles precede the head while those formed with a relative \isi{pronoun} follow the head. Relative clauses with a gap may follow or precede the head. (\citealt{harris_word_2000}). All Kartvelian languages have subordinating conjunctions in clause-initial, clause-second, or preverbal position (\citealt{boeder_modern_2021}), and Megrelian\il{Kartvelian!Megrelian} also has a clause-final subordinator (\citealt{boeder_south_2005}: 70; \citealt{testelets_kartvelian_2021}: 522--523).
As mentioned, Kartvelian languages are usually characterized as having free \isi{word order} at the clausal level, with verb final order as the unmarked pattern (\citealt{boeder_south_2005}: 64). In the following section I will examine \isi{word order} in Georgian\il{Kartvelian!Georgian} because it is by far the most studied language among the Kartvelian languages. 

\subsection{Georgian\il{Kartvelian!Georgian}: Previous studies on word order at the clausal level}\label{EC:ss:2.2}

After a series of elicitation tests \citet{skopeteas_georgian_2021} comes to the conclusion that certain asymmetries between V-medial and V-final orders suggest that the basic \isi{word order} is verb-final. One of his arguments is the position of prepositional complements such as comitatives and themes which are preferably placed as expected for verb-final languages (comitatives before themes) and have rigid scope in the basic order.

However, apart from a few exceptional cases, both V-O and O-V orders occur in free variation and V-O is not triggered by any special pragmatic or semantic configurations (\citealt{testelets_wordKartvelian_1998}, \citealt{asatiani_information_2012}, \citealt{skopeteas_word_2009}, \citealt{boeder_modern_2021}). All other orders are also attested in elicitation and natural texts (see, e.g. \citealt{pourtskhvanidze_fokuspartikeln_2015}: 161--162 for examples). With respect to frequency, based on data from the internet \citet{skopeteas_georgian_2021} found that in a total of 925 non-idiomatic VPs 64.1\% have O-V order while 35.9\% gave V-O order. Other authors present different numbers. \citet{vogt_grammaire_1971} counted 50 randomly chosen pages in the influential novella, \textit{jaq'os xiznebi} `Jaqo's dispossessed' by Mikheil Javakhishvili, which was first published in 1924. S-O-V order is attested in 75\% of the sentences; S-V-O in one sixth of them (\citealt{vogt_grammaire_1971}: 222). He also counted traditional folk tales (published in 1958) and got slightly different results: the subject occupies the initial position in two third of the examples; the \isi{direct object} precedes or follows the verb with roughly the same frequency, which means that O-V and V-O are equally frequent. \citet[421]{stilo_areal_2014} counted indefinite and definite direct objects in a small corpus of colloquial (spoken) Tbilisi Georgian\il{Kartvelian!Georgian Tbilisi} that consisted of 500--600 clauses. He found that 60.2\% of the definite objects occurred before the verb (O-V) while 39.8\% occurred after the verb (V-O). For indefinite direct objects the differences increase a bit: O-V is found in 64.9\% of the clauses whereas V-O in only 35.81\%. These numbers are comparable to Skopeteas counts of internet texts. In another unpublished data set of conversational informal Georgian\il{Kartvelian!Georgian} the total number of nominal direct objects (excluding pronouns) that are V-O is 160 out of 364 (around 44\%), (\citealt{stilo_preverbal_2018}). For definite objects, the V-O figure is 96 of 206 (around 47\%) and for indefinite objects V-O is found for 64 out of 158 (about 40\%).

\citet{skopeteas_word_2009} also suggest that stylistic factors might have an effect, which is, in fact, also suggested by Vogt's data, since he found a difference between a classical written novella and folk tales that belong to traditional oral literature. Finally, \citet{skopeteas_georgian_2021} cites a study by \citet{apridonidze_sitqvatganlageba_1986} that shows that the more constituents a clause has the more likely it is that the verb does not occur in final position but that the clause contains post-predicate elements.

In sum, Georgian\il{Kartvelian!Georgian} seems to have S-O-V as basic \isi{word order} in terms of a rather formal theory of grammar and a certain degree of variation in terms of actually attested patterns in natural data. For the latter, style/genre and number of constituents in the clause play a \isi{role}, but possibly also other factors such as grammatical function, \isi{definiteness}, pragmatics and \isi{heaviness} of constituents. 

\subsection{Georgian\il{Kartvelian!Georgian} word order in a small corpus study}\label{EC:ss:2.3}

For this paper, I recorded nine oral texts and conducted a small corpus study. The texts have been elicited from nine speakers of Georgian\il{Kartvelian!Georgian} by means of the Pear Story Movie \citep{chafe_pear_1980}.\footnote{The film is available at \url{https://www.youtube.com/watch?v=bRNSTxTpG7U} and \url{https://drive.google.com/file/d/1jF6vtUdOlBN9LFzJ_NujWfP9Ef6ObK9K/view}.} The corpus contains 2,901 words, and 644 clauses, of which 556 contain a verb and at least one \isi{argument} or other constituent with a relevant grammatical function for this study. The other 88 clauses have been left out because they only consist of the verb, or verb with particle or \isi{adverb} outside the purview of this study. 

I classified arguments and adjuncts into the following grammatical functions, based on semantics rather than formal marking:

\begin{itemize}
\item S: subject of monovalent verb
\item A: subject of bivalent verb
\item P: \isi{object} of bivalent verb
\item \isi{Goal}: spatial \isi{Goal}, objects such as \isi{addressee}, \isi{recipient}
\item location: spatial location
\item source: spatial origin, source
\item \isi{instrument}: \isi{instrument} or tool of any sort
\item beneficiary
\end{itemize}

I counted only nominal referents fulfilling these grammatical functions (including \isi{postpositional} phrases) and excluded temporal adverbials and some other types of adjuncts as well as adverbs and particles. I included various types of subordinate clauses, in particular relative clauses, \isi{complement} clauses and purpose clauses and also non-declarative utterances (of which there were only very few).

As a starting point, I compared my counts with \citegen{apridonidze_sitqvatganlageba_1986} findings, according to which the probability of a constituent occurring after the verb increases when there are more constituents in the clause. In fact, this is not surprising, since the more constituents a clause has, the more possibilities there are to put at least one after the verb. However, my own data are not as neat as his data and also differ quantitatively. \tabref{EC:tab:1} shows my data for clauses with at least two and up to five constituents (including the verb). I counted all nominal constituents, but did not include pronouns, particles or adverbs. There is no monotonic decrease in the probability for verb-final order, but a clear tendency. Of clauses with only two constituents (e.g. a verb and S or a verb and P), around 40\% have X-V order and 60\% V-X order. For clauses with three constituents (e.g. verb, A and P or verb S and \isi{Goal}) the percentage of V-X order increases to 73\%.

\begin{table}
 \begin{tabularx}{\textwidth}{lYYYYYYYY}
\lsptoprule
\# constituents & 2 & \% & 3 & \% & 4 & \% & 5 & \% \\
\midrule
X-V & 122 & 40.67 & 48 & 26.82 & 7 & 11.11 & 2 & 15.38 \\
V-X & 178 & 59.33 & 131 & 73.18 & 56 & 88.89 & 11 & 84.61 \\
\midrule
total & 300 & & 179 & & 63 & & 13 & \\
\lspbottomrule
 \end{tabularx}
 \caption{Word order at the clausal level in relation to the number of constituents}
 \label{EC:tab:1}
\end{table}

\tabref{EC:tab:2} summarizes first the constituent order patterns of all clauses and, second, the position of S, A and P (including relative pronouns and demonstrative pronoun\is{pronoun!demonstrative}s). In my data, around two third of the clauses are not verb-final but contain at least one nominal or pronominal constituent after the verb. Compared to the other studies cited above this is a considerably higher amount of post-predicate items. Subjects of monovalent verbs (S) and to an even larger extent nominal subjects of bivalent verbs (A) overwhelmingly precede the verb, whereas objects of bivalent verbs (P) tend to follow the verb. These numbers are astonishing in the light of the data cited above. The only hypothesis that comes to my mind is that this might reflect a difference in written vs. spoken language (see \citetv{chapters/1_Haigetal_Intro} and \citetv{chapters/7_RasekhMahandetal_Persian} for similar findings), but Stilo's data also represent spoken language, albeit not elicited monologues.

\begin{table}
\begin{tabularx}{\textwidth}{lrYYYYYYY}
\lsptoprule
 & clauses & \% & S & \% & A & \% & P & \% \\
 \midrule
X-V & 186 & 33.45 & 121 & \cellcolor{gray!50}69.12 & 93 & \cellcolor{gray!50}93.00 & 100 & 38.76 \\
V-X & 370 & \cellcolor{gray!50}66.55 & 54 & 30.85 & 7 & 7.00 & 158 & \cellcolor{gray!50}61.24 \\
\midrule total & 556 & & 175 & & 100 & & 258 & \\
\lspbottomrule
 \end{tabularx}
 \caption{The position of S, A and P}
 \label{EC:tab:2}
\end{table}

In order to check whether the part of speech had an influence on the position of arguments and adjuncts in the surveyed functions, first of all, I excluded all relative pronouns from the counts. Relative pronouns in my corpus are placed before the verb, which is their normal position (\citealt{harris_word_2000}), and the majority of relative pronouns occur in the function of S or A. In a second step, I omitted all demonstrative pronoun\is{pronoun!demonstrative}s used in \isi{anaphoric}\is{anaphoric!pronoun}\is{pronoun!!anaphoric}
function from the counts. There were no personal pronoun\is{pronoun!personal}s for first and second person due to the type of \isi{stimulus} used for the narratives. I thus cannot make any statements on the position of those pronouns. \tabref{EC:tab:2a} summarizes the counts for nominal S, A and P arguments. As a comparison with \tabref{EC:tab:2} shows, there are only small differences in the percentages. Furthermore, there are not many demonstrative pronoun\is{pronoun!demonstrative}s in my small corpus and from \tabref{EC:tab:2b} it is clear that those pronouns do not differ much in their position/function from the nouns (and the very few indefinite pronoun\is{pronoun!indefinite}s).

\begin{table}
 \begin{tabularx}{\textwidth}{lYYYYYY}
\lsptoprule
 & S & \% & A & \% & P & \% \\
\midrule
X-V & 100 & \cellcolor{gray!50}66.23 & 66 & \cellcolor{gray!50}91.67 & 93 & 37.96 \\
V-X & 51 & 33.77 & 6 & 8.33 & 152 & \cellcolor{gray!50}62.04 \\
\midrule total & 151 & & 72 & & 245 & \\
\lspbottomrule
 \end{tabularx}
 \caption{The position of S, A and P (excluding relative pronouns and demonstrative pronoun\is{pronoun!demonstrative}s in anaphoric\is{anaphoric!pronoun}\is{pronoun!!anaphoric}
 function)}
 \label{EC:tab:2a}
\end{table}

\begin{table}
 \begin{tabularx}{\textwidth}{lYYYrrrrr}
\lsptoprule
 & S & A & P & \isit{Goal} & location & source & \isit{instrument} & beneficiary \\
\midrule
X-V & \cellcolor{gray!50}9 & \cellcolor{gray!50}13 & 3 & 1 & 0 & 1 & 0 & 4 \\
V-X & 3 & 1 & \cellcolor{gray!50}6 & \cellcolor{gray!50}3 & 0 & 0 & \cellcolor{gray!50}3 & 2 \\
\midrule total & 12 & 14 & 9 & 4 & 0 & 1 & 3 & 6 \\
\lspbottomrule
 \end{tabularx}
 \caption{The position of arguments and adjunctions expressed as demonstrative pronoun\is{pronoun!demonstrative}s}
 \label{EC:tab:2b}
\end{table}

Goals are most commonly placed after the verb, with an overall higher frequency of post-verbal placement than any other \isi{argument} type investigated here. This is a feature that my Georgian\il{Kartvelian!Georgian} corpus shares with all other spoken-language corpora in the WOWA data base (see \citetv{chapters/1_Haigetal_Intro}). Instruments are also more frequently in post-verbal than in preverbal position. By \isi{contrast}, referents expressing locations are usually found before the verb; cf. \tabref{EC:tab:3} and \tabref{EC:tab:3a} for relevant figures. In (\ref{EC:ex:1}) \isi{instrument}, \isi{Goal} and (metaphorical) location appear after the verb. For source and beneficiary, the distribution is roughly half-half and no clear tendency could be detected. The corpus also contains 38 \isi{complement} clauses of which 35 are in a position after the matrix predicate.

\begin{table}
\begin{tabularx}{\textwidth}{X rr rr rr rr rr}
\lsptoprule
 & \isit{Goal} & \% & loc & \% & src & \% & instr & \% & ben & \% \\
\midrule
X-V & 22 & 25.88 & 52 & \cellcolor{gray!50}61.18 & 21 & 46.67 & 18 & 35.29 & 8 & 53.33 \\
V-X & 63 & \cellcolor{gray!50}74.12 & 33 & 38.82 & 24 & 53.33 & 33 & \cellcolor{gray!50}64.71 & 7 & 46.67 \\
\midrule
total  & 85 & & 85 & & 45 & & 51 & & 15 & \\
\lspbottomrule
 \end{tabularx}
 \caption{The position of other types of arguments and adjuncts}
 \label{EC:tab:3}
\end{table}

\begin{table}
\begin{tabularx}{\textwidth}{X rrrrrrrrrr}
\lsptoprule
 & \isit{Goal} & \% & loc & \% & src & \% & instr & \% & ben & \% \\
 \midrule
X-V & 20 & 25.00 & 44 & \cellcolor{gray!50}57.14 & 16 & 40.00 & 18 & 37.5 & 4 & 44.44 \\
V-X & 60 & \cellcolor{gray!50}75.00 & 33 & 42.86 & 24 & 60.00 & 30 & \cellcolor{gray!50}62.5 & 5 & 55.56 \\
\midrule
total  & 80 & & 77 & & 40 & & 48 & & 9 & \\
\lspbottomrule
 \end{tabularx}
 \caption{The position of other types of arguments and adjuncts (excluding relative pronouns and demonstrative pronoun\is{pronoun!demonstrative}s in anaphoric\is{anaphoric!pronoun}\is{pronoun!!anaphoric}
 function)}
 \label{EC:tab:3a}
\end{table}

\newpage
\ea\label{EC:ex:1}
V-INST-GOAL-LOC\footnote{In this study of spoken Georgian, the concepts of GOAL and INST are used slightly differently from how they are used in other papers of this volume. I used case marking as a major indicator, i.e. \isi{dative} marking for GOAL and instrumental case marking for INST.} \\
Georgian \il{Kartvelian!Georgian}(Georgian Pear Story Corpus) \\
\gll mi-e-q'rdn-ob-a zurg-it k'ibe-s albat im-is pikr=ši\\
\textsc{pv-appl-}lean\_on\textsc{-tm-S.3sg} back\textsc{-inst} stairs\textsc{-dat} probably \textsc{dem.dist-gen} thought\textsc{=loc} \\
\glt `(He) leans with the back to the stairs, probably in his thoughts.'
\z

Then I looked into the position of new referents, more specifically of new human referents which are introduced into the Pear Stories one by one. Most speakers mentioned five human referents, and there was a very clear tendency to express them either as S or as P and put them in a position after the verb (\tabref{EC:tab:4}). In example (\ref{EC:ex:2}), the main protagonist of the film, a young boy, is introduced into the narrative in the function of S occurring as the last item in the clause.

\ea\label{EC:ex:2}
TIME-V-INST-S \\
Georgian \il{Kartvelian!Georgian}(Georgian Pear Story Corpus) \\
\gll cot'a xan=ši ga-mo-čn-d-eb-a velosip'ed-it bič'i \\
a\_little period\textsc{=loc} \textsc{pv-pv-}appear\textsc{-intr-tm-S.3sg} bike\textsc{-inst} boy \\
\glt `After a little while a boy with a bike will appear.'
\z

Then I counted the position of light versus heavy noun phrases with one, two, three, four or more words. In \tabref{EC:tab:4} relative clauses are excluded, i.e. all noun phrases that head relative clauses have been omitted from the counts and only noun phrases with demonstratives, adjectives and the like have been included. However, there are basically no differences between noun phrases that consist of one, two or three words. Only noun phrases containing four or more words show an increased tendency for a position after the verb.

\begin{table}
 \begin{tabularx}{\textwidth}{Xrrrrrrrrrr}
\lsptoprule
 & \multicolumn{2}{l}{new} & \multicolumn{2}{l}{4 (+)} words & \multicolumn{2}{l}{3 words} & \multicolumn{2}{l}{2 words} & \multicolumn{2}{l}{1 word} \\
\midrule
X-V & 11 & 25\% & 12 & 42.86\% & 30 & 53.57\% & 150 & 55.76\% & 233 & 54.57\% \\
V-X & 33 & \cellcolor{gray!50}75\% & 16 & 57.14\% & 26 & 46.43\% & 119 & 44.24\% & 194 & 45.43\% \\
\midrule
total  & 44 & & 28 & & 56 & & 269 & & 427 & \\
 \lspbottomrule
 \end{tabularx}
 \caption{Newness (human referents) and heaviness of constituents (without relative clauses)}
 \label{EC:tab:4}
\end{table}

Finally, \tabref{EC:tab:5} presents first of all the position of relative clauses and their heads. 29 out of a total of 34 nominal heads of relative clauses occur in a position after the verb. Second, I added the noun phrases with relative clauses to the counts in \tabref{EC:tab:4}. To illustrate that with an example, we can look at (\ref{EC:ex:3}): the NP `that lower part of a tree' consists of two constituents before the head noun, namely a demonstrative and a genitive, and a relative clause following it. For the manner of counting displayed in \tabref{EC:tab:5} this NP has four constituents (demonstrative, genitive, head noun and relative clause). These numbers suggest that the strongest effect on the position is the presence of a relative clause in the NP, which leads to >80\% post-verbal placement - and this seems to be irrespective of how many other constituents there are in the NP. Similarly, the corpus contains 38 \isi{complement} clauses of which three occur before the verb and 35 after the verb. This can be generalized: if an NP contains a clausal constituent, it is nearly categorically likely to be post-verbal. The effect of number of constituents, on the other hand, is quite small by comparison, and is only really significant for +4 constituents vs. 1 constituent.

\begin{table}
\fittable{
\begin{tabular}{lrrrrrrrrrr}
\lsptoprule
 & \multicolumn{2}{c}{relative clause} & \multicolumn{2}{c}{4(+) constituents} & \multicolumn{2}{c}{3 constituents} & \multicolumn{2}{c}{2 constituents} & \multicolumn{2}{c}{1 constituent} \\
\midrule
X-V & 5 & 14.71\% & 15 & 37.50\% & 30 & 45.45\% & 151 & 54.12\% & 232 & 54.08\% \\
V-X & 29 & 85.29\% & 25 & 62.50\% & 36 & 54.55\% & 128 & 45.88\% & 197 & 45.92\% \\
\midrule
total   & 34 & & 40 & & 66 & & 279 & & 429 & \\
\lspbottomrule
 \end{tabular}
 }
 \caption{Position of head noun of relative clause and heaviness (including relative clauses; all semantic roles)}
 \label{EC:tab:5}
\end{table}


\ea\label{EC:ex:3}
V-GOAL-REL \\
Georgian \il{Kartvelian!Georgian}(Georgian Pear Story Corpus) \\
\gll ga-i-vl-ian am x-is ʒira-s [roml=idana=c uk've ʒirs ar-is ča-mo-sul-i mama-k'ac-i] \\
\textsc{pv-refl-}go\textsc{-S.3pl} \textsc{dem.prox} tree\textsc{-gen} lower\_bottom\textsc{-dat} which\textsc{=abl=add} already down be\textsc{-S.3sg} \textsc{pv-pv-}go\textsc{.ptcp-nom} father-man\textsc{-nom} \\
\glt `(The boys) pass by the foot of this tree, from which already has come down the old man.'
\z

Summarizing we can state that all kinds of arguments and adjuncts can occur after the verb, but direct objects (\ref{EC:ex:4}), goals (\ref{EC:ex:1}), (\ref{EC:ex:2}) including indirect objects, and instruments (\ref{EC:ex:1}), (\ref{EC:ex:2}) are particularly prone to be placed after the verb, which means that the grammatical function has an impact on the position of the respective item. Furthermore, pragmatics plays a \isi{role}: newly introduced (human) referents mostly follow the verb (\ref{EC:ex:2}) (non-human referents have not been counted). Very heavy noun phrases and nouns heading a relative clause also tend to be positioned after the verb (\ref{EC:ex:3}), but demonstrative pronoun\is{pronoun!demonstrative}s in \isi{anaphoric}\is{anaphoric!pronoun}\is{pronoun!!anaphoric}
function do not differ in their preferences from nominals. 

\ea\label{EC:ex:4}
V-A-P\\
Georgian \il{Kartvelian!Georgian}(Georgian Pear Story Corpus) \\
\gll šemdeg da-i-berṭq'-d-a am bič'-ma šarval-i \\
afterwards \textsc{pv-refl-}shake\_out\textsc{-impf-S.3sg} \textsc{dem.prox} boy\textsc{-erg} trousers\textsc{-nom} \\
\glt `Then this guy shook out his trousers.'
\z

Due to the limits of my corpus, further research is needed that targets also first- and second-person pronouns, examines the impact of (in)\isi{definiteness} and the positional properties of subordinate clauses with non-finite verbs such as participles and masdars.


\subsection{Megrelian\il{Kartvelian!Megrelian}, Svan\il{Kartvelian!Svan} and Laz\il{Kartvelian!Laz}}\label{EC:ss:2.4}

What concerns Megrelian\il{Kartvelian!Megrelian} and Svan\il{Kartvelian!Svan}, post-predicate elements do not seem to be rare. We find subjects, direct objects, indirect objects, obliques such as instruments and others, temporal and spatial adverbials both in nominal as well as pronominal form in all grammatical descriptions surveyed (\citealt{harris_mingrelian_1991}, \citealt{holisky_laz_1991}, \citealt{rostovtsev-popiel_megrelian_2021}, \citealt{tuite_short_1998}, \citealt{schmidt_svan_1991}).

Laz\il{Kartvelian!Laz} is the least flexible Kartvelian language with respect to \isi{word order}, even though the claim by \citet[518]{testelets_kartvelian_2021,testelets_kartvelian_2021} that Laz\il{Kartvelian!Laz} is relatively strict verb final with only very few constructions allowing for a restricted range of post-predicate elements has to be rejected. In \isi{contrast} to the other three Kartvelian languages, Laz\il{Kartvelian!Laz} is mainly spoken in Turkey and thus under heavy Turkish\il{Turkic!Turkish} influence. \citet[737]{lacroix_description_2009} makes some generalizations about post-predicate elements in Laz\il{Kartvelian!Laz}. They are mostly (i) known / topical, or (ii) part of an idiomatic expression, or (iii) new referents in introductory sentences, or (iv) specify a referent that has already been mentioned in the sentence. Laz\il{Kartvelian!Laz} has a much lower frequency of post-predicate elements when compared to Georgian\il{Kartvelian!Georgian}. This is certainly true for the texts in \citet{kutscher_ardesen_1998} and in \citet{stilo_laz_2021}. However, otherwise it seems that roughly the same range of elements are allowed as in the other Kartvelian languages, both in elicitation and in natural texts (e.g. \citealt{kutscher_ardesen_1998}, \citealt{lacroix_description_2009}): subjects, objects, obliques, adverbials (goals, locations). In the 11 Arhavi Laz\il{Kartvelian!Laz} texts collected by Lacroix, published in his grammar \citep{lacroix_description_2009} and coded for WOWA by Don Stilo \citep{stilo_laz_2021}, 400 items (noun phrases and adverbials) have been categorized with respect to their position: 391 occur before the verb (around 98\%) and only 9 after the verb (around 2\%). Items occurring after the verb serve as direct objects, addressees, locations, goals, and one is a possessed referent in a \isi{possessive} construction. With these numbers, Laz\il{Kartvelian!Laz} is among the most consistently verb-final languages in the entire WOWA data set.

As shown for Georgian\il{Kartvelian!Georgian} and just mentioned, in introductory sentences or, more generally, in contexts in which new referents are introduced into a narration, the new referents often follow the verb. These new referents are usually either subjects or direct objects as in (\ref{EC:ex:5}) from Svan\il{Kartvelian!Svan}.

\ea\label{EC:ex:5}
Svan \il{Kartvelian!Svan}(\citealt{schmidt_svan_1991}: 539) \\
\gll ašxwin ləcte otzəzax bepšw \\
once water.to they.apparently.sent child\textsc{.nom} \\
\glt `Once (they) sent a child to the water.'
\z

But postverbal items can be topical, too. Example (\ref{EC:ex:6}) from Megrelian\il{Kartvelian!Megrelian} and example (\ref{EC:ex:7}) from Laz\il{Kartvelian!Laz} illustrate postverbal subjects that encode established referents.

\ea\label{EC:ex:6}
Megrelian \il{Kartvelian!Megrelian}(\citealt{rostovtsev-popiel_megrelian_2021}: 557) \\
\gll k'in=i mida-rt-es o-nadir-u-ša boš-ep-k \\
back\textsc{=ev} \textsc{pv-}go\textsc{-3sg.pst} \textsc{supine-}hunt\textsc{-supine}-all boy\textsc{-pl-erg} \\
\glt `The boys left for hunting again.'
\z

\ea\label{EC:ex:7}
Laz \il{Kartvelian!Laz}(\citealt{holisky_laz_1991}: 469) \\
\gll i.bgar-u do xolo meyoč-u oxorǯa-k \\
cry\textsc{-3sg} and again curse\textsc{-3sg} wife\textsc{-erg} \\
\glt `The wife cried and cursed again.'
\z

Heaviness might play a \isi{role}. In (\ref{EC:ex:8}), from Svan\il{Kartvelian!Svan}, the first main clause contains a postverbal focal \isi{object} whereas the focal \isi{object} in the second main clause is in preverbal position. The postverbal \isi{object} of the first clause does not even directly follow the verb, but is separated from it by an inserted subordinate conditional clause. It is heavy, consisting of a participial relative clause and an \isi{adjective}, which might be a reason for its postverbal position.

\ea\label{EC:ex:8}
Svan \il{Kartvelian!Svan}(\citealt{tuite_short_1998}: 19) \\
\gll eče-ži a-d-isg-x, [xoxra bepšw-ild-ær axa æt-[i]-dagr-i-w-x], \textbf{eǯær-e} \textbf{le-pane} \textbf{xoxra} \textbf{dir-ild-ær-s} i let'wra a-t'wr-e-x ečeču \\
there-at \textsc{ver-}put\textsc{-sm-pl} little child\textsc{-dim-pl.nom} if \textsc{pv-ver-}die\textsc{-sm-imp-pl} \textsc{3pl-gen} \textsc{ptcp-}consecrate little bread\textsc{-dim-pl-dat} and candle\textsc{.dat} \textsc{ver-}light\textsc{-sm-pl} there \\
\glt `If small children from the household have died they set there little loaves of bread consecrated to them, and light a candle.'
\z

I cannot say whether grammatical functions have an impact on the likelihood or frequency of being placed after the verb. The following two examples show an inanimate \isi{Goal} (\ref{EC:ex:9}) as well as an animate \isi{direct object}, an animate indirect object\is{object!indirect} plus animate adverbial (\ref{EC:ex:10}).

\ea\label{EC:ex:9}
Laz \il{Kartvelian!Laz}(\citealt{holisky_laz_1991}: 409) \\
\gll igzal-es bee-pe diška-ša \\
go\textsc{-3pl} child\textsc{-pl} firewood-all \\
\glt `The children went for firewood.' 
\z

\ea\label{EC:ex:10}
Megrelian \il{Kartvelian!Megrelian}(\citealt{harris_mingrelian_1991}: 374) \\
\gll mapa-k kimeč tina mec'amale-s čil-o \\
king\textsc{-erg} gave \textsc{3sg} doctor\textsc{-dat} wife\textsc{-adv} \\
\glt `The king gave her to the doctor as [his] wife.'
\z

In Laz\il{Kartvelian!Laz}, indefinite postpredicate items are also a feature of some idiomatic expression such as `set the table', `make someone's wedding' and `drink tea' (\ref{EC:ex:11}).

\ea\label{EC:ex:11}
Laz \il{Kartvelian!Laz}(\citealt{lacroix_description_2009}: 741) \\
\gll hek do-v-es didi duğuni \\
there \textsc{pv-}make\textsc{-aor.i.3pl} big wedding \\
\glt `There they made big weddings.'
\z

In sum, in Kartvelian SOV is a common and possibly the basic \isi{word order}, but other orders are also possible and attested in texts. There are no hard constraints concerning the grammatical function or \isi{role} of postverbal arguments or adjuncts or their parts of speech. Laz\il{Kartvelian!Laz} differs from all other Kartvelian languages in terms of actual frequency of postpredicate items in natural texts, which is likely due to a substantial impact of Turkish\il{Turkic!Turkish}. Note for example that the Laz\il{Kartvelian!Laz} corpus in \citet{stilo_laz_2021} exhibits less than 5\% post-verbal Goal\is{Goal!post-verbal}s (cf. the figure of 75\% from spoken Georgian\il{Kartvelian!Georgian} (Tables \ref{EC:tab:3} and \ref{EC:tab:3a} above), and comparable figures across the WOWA sample). It is possible that the texts in the Laz\il{Kartvelian!Laz} corpus of \citet{stilo_laz_2021} have been edited in some manner; this remains to be clarified.


\section{Post-predicate items in East Caucasian}\label{EC:ss:3}

\subsection{Word order profile of East Caucasian}\label{EC:ss:3.1}

Noun phrases are normally \isi{head-final} \citep{ganenkov_nakh-dagestanian_2021}. However, various types of modifiers (except for demonstratives) can occur after the head noun and there is some indication that in many cases the postponed modifier does not form one NP with the preceding nominal, but rather makes up its own NP, e.g. because it needs to be case marked, nominalized or bear other types of special marking (e.g. Dargwa languages, Akhvakh). \citet[274]{testelets_wordDaghestanian_1998} characterizes postposed modifiers as focused, contrasted, or restrictive. It seems that in natural texts genitives, in particular \isi{possessive} pronouns, are postposed more commonly than any other type of modifier (see examples below).

East Caucasian languages have postpositions. Auxiliaries follow the lexical verbs. Major complementation strategies are non-finite verb forms (infinitive, masdar, participles, converbs), quotative particles, which are usually placed to the right of the clause, or enclitics and zero marking. Complementizers, which are often loans, play only a marginal \isi{role}. Complement clauses may precede or follow the matrix verb.

As the other two indigenous families of the Caucasus, East Caucasian languages are predominantly \isi{head-final} (SOV), but allow for all logically possible orders. Thus, we find postverbal arguments and adjuncts of all kinds in the literature (\citealt{testelets_wordDaghestanian_1998}, \citealt{van_den_berg_east_2005}) and in natural texts they are common. Word order in subordinate clauses is more restricted. For instance, in Sanzhi Dargwa\il{Caucasian (East)!Sanzhi Dargwa} relative clauses are verb-final with very few exceptions; \isi{complement} clauses and adverbial clauses show a stronger tendency for verb-final order than main clauses, but far less than relative clauses \citep{forker_grammar_2020}. A similar distribution is found in Hinuq\il{Caucasian (East)!Hinuq}: relative clauses are strictly \isi{head-final} whereas \isi{complement} and adverbial clauses occasionally contain post-predicate elements \citep{forker_grammar_2013}. 

\subsection{Post-predicate elements}\label{EC:ss:3.2}

Based on the literature and on counts from the Multicast corpora for Chirag Dargwa\il{Caucasian (East)!Chirag Dargwa}, Sanzhi Dargwa\il{Caucasian (East)!Sanzhi Dargwa} \citep{forker_multi-cast_2019} and Tabasaran\il{Caucasian (East)!Tabasaran} \citep{bogomolova_multi-cast_2021} a few generalizations concerning the conditions for post-predicate elements are possible. In general, they are far more frequent than in Adyghe\il{Circassian!Adyghe} (see \citetv{chapters/1_Haigetal_Intro}, \tabref{EC:tab:6}). Grammatical functions play a \isi{role} in all three languages. Chirag Dargwa\il{Caucasian (East)!Chirag Dargwa} and Sanzhi Dargwa\il{Caucasian (East)!Sanzhi Dargwa} have far more post-predicate subjects than any other kinds of elements whereas for Tabasaran\il{Caucasian (East)!Tabasaran} the difference between subjects and objects is relatively small. Goals, which include addressees, are more often found in postverbal position than obliques (= indirect objects, beneficiaries, instruments, sources, etc.) and locations (\tabref{EC:tab:6}). For instance, in Sanzhi 57.9\% of the 57 goals in main clauses occur after the verb (33 items).

\begin{table}
 \begin{tabularx}{\textwidth}{lrrrrrrrrr}
\lsptoprule
 & \multicolumn{2}{r}{Chirag Dargwa} & \multicolumn{2}{r}{Sanzhi Dargwa} & \multicolumn{2}{r}{Tabasaran} \\
\midrule
texts        && 11   && 8    && 5 \\
words        && 5347 && 3857 && 5450 \\
main clauses && 1183 && 945  && 1210 \\
all clauses  && 1377 && 1066 && 1383 \\
\midrule
\multicolumn{8}{l}{post-predicate elements in main clauses (nouns, pronouns, other items)} \\
\midrule
subject & 65& (23.81\%) & 62& (26.05\%) & 141& (28.54\%) \\
\isi{object} & 31& (28.44\% & 38& (41.76\%) & 77& (30.68\%) \\
\isi{Goal} + \isi{addressee} & 20& (32.79\%) & 33& (57.89\%) & 64& (54.7\%) \\
\isi{oblique} & 21& (34.42\%) & 18& (35.29\%) & 21& (31.34\%) \\
location & 12& (26.53\%) & 18& (35.29\%) & 28& (45.9\%) \\
\midrule
total per clause (all roles) & 150& (12.68\%) & 169& (17.88\%) & 331& (27.36\%) \\
\lspbottomrule
 \end{tabularx}
 \caption{Post-predicate elements in Chirag, Sanzhi and Tabasaran}
 \label{EC:tab:6}
\end{table}

Chechen\il{Caucasian (East)!Chechen} and Ingush\il{Caucasian (East)!Ingush} can be added to the East Caucasian languages for which we know that postverbal items are common in natural texts. For Ingush\il{Caucasian (East)!Ingush}, \citet[678]{nichols_ingush_2011} states that in ``main clauses, other than episode-initial and other all-new ones, verb-second order is most common.'' According to \citet{nichols_chechen_1994}, in Chechen\il{Caucasian (East)!Chechen} OVS is not uncommon in elicited sentences (see also \citealt[32]{komen_focus_2007} for a similar assessment).

When comparing preverbal objects (\isi{OV}) to postverbal objects (\isi{VO}) in Chirag\il{Caucasian (East)!Chirag Dargwa}, Sanzhi\il{Caucasian (East)!Sanzhi Dargwa}, Tabasaran\il{Caucasian (East)!Tabasaran} and Hinuq\il{Caucasian (East)!Hinuq} in main (\tabref{EC:tab:7}) it turns out that almost between 30 and 40\% of the objects occurs after the predicate. This is less than in Georgian\il{Kartvelian!Georgian} (Tables \ref{EC:tab:2} and \ref{EC:tab:2a}), but still much more than in the Iranian and Turkic verb-final languages in the WOWA sample, which generally exhibit >80\% \isi{OV} order (see \citetv{chapters/1_Haigetal_Intro}). This is suggestive of a distinct kind of \isi{OV} for these languages (and Georgian\il{Kartvelian!Georgian} as well). Furthermore, in Sanzhi Dargwa\il{Caucasian (East)!Sanzhi Dargwa} and Chirag Dargwa\il{Caucasian (East)!Chirag Dargwa}, and to a small degree also in Tabasaran\il{Caucasian (East)!Tabasaran}, pronominal objects have a greater tendency to be placed after the verb than nominal objects; for obliques, goals and locations no such tendencies can be observed (\tabref{EC:tab:7}).

\begin{table}[t]
% \begin{tabularx}{\textwidth}{lrrrrrrrrr}
% \lsptoprule
% Language & O (all Os)\footnotemark & \isit{VO} & \% & NP & \isit{VO} for NPs & \% & pro & \isit{VO} for pro & \% \\
% \midrule
% Chirag Dargwa & 109 & 31 & 28.44\% & 79 & 19 & \textbf{24.05\%} & 20 & 9 & \textbf{45.00\%} \\
% Sanzhi Dargwa & 91 & 38 & 41.76\% & 71 & 29 & \textbf{40.85\%} & 13 & 9 & \textbf{69.23\%} \\
% Tabasaran & 251 & 77 & 30.68\% & 204 & 68 & \textbf{33.33\%} & 22 & 8 & \textbf{36.36\%} \\
% \lspbottomrule
%  \end{tabularx}
 \begin{tabularx}{\textwidth}{lrYY}
\lsptoprule
 & Chirag Dargwa & Sanzhi Dargwa & Tabasaran \\
\midrule
O (all Os)\footnote{Note that all Os consist of lexical NPs, pronouns and other items. In the table, only lexical NPs and pronouns are listed separately. } & 109 & 91 & 251 \\
\isit{VO} & 31 & 38 & 77 \\
\% & 28.44\% & 41.76\% & 30.68\% \\
\tablevspace
NP & 79 & 71 & 204 \\
\isit{VO} for NPs & 19 & 29 & 68 \\
\% & \textbf{24.05\%} & \textbf{40.85\%} & \textbf{33.33\%} \\
\tablevspace
pro & 20 & 13 & 22 \\
\isit{VO} for pro & 9 & 9 & 8 \\
\% & \textbf{45.00\%} & \textbf{69.23\%} & \textbf{36.36\%} \\
\lspbottomrule
\end{tabularx}

 \caption{Postverbal objects (VO) in Chirag\il{Caucasian (East)!Chirag Dargwa}, Sanzhi\il{Caucasian (East)!Sanzhi Dargwa}, and Tabasaran\il{Caucasian (East)!Tabasaran} in main clauses }
 \label{EC:tab:7}
\end{table}

\begin{sloppypar}
I was not able to systematically check for \isi{heaviness} and the position of headed relative clauses. Instead, I will examine the literature on \isi{information structure} and the placement of arguments and adjuncts in post-verbal position. \citet[129]{ganenkov_nakh-dagestanian_2021} state ``The postverbal field is reserved for background information—that is, those arguments that are recoverable from the context but still mentioned for the sake of clarity.'' This generalization can be made for Hinuq\il{Caucasian (East)!Hinuq}, Dargwa languages, Archi, Avar, Lak\il{Caucasian (East)!Lak}, Ingush\il{Caucasian (East)!Ingush} and probably more East Caucasian languages (\citealt{forker_word_2016}; \citealt{komen_post-verbal_2017}, \citealt{testelets_wordDaghestanian_1998}: 260--261). In particular when the verb is focused topical arguments can follow it (\ref{EC:ex:12}). 
\end{sloppypar}

\ea\label{EC:ex:12}
[Then the wife of a student hears about the news.] \\
Lak \il{Caucasian (East)!Lak}(\citealt{khalilova_lakskie_1976}: 204--205) \\
\gll [mu=gu]\textsuperscript{TOP} maħattal x̂-unu d-ur [wa iš-ira-j]\textsuperscript{TOP} \\
\textsc{dem.prox=add} amazed become\textsc{-pst.ger} \textsc{ii-cop} this issue\textsc{-obl-spr} \\
\glt `She also got amazed because of this issue.'
\z

Verb fronting is a typical way of marking predicate \isi{focus} and leads to post-predicate elements that are either topical or can also be focal. In (\ref{EC:ex:13}), the verb is located in the clause-initial position while the \isi{argument} NPs retain their unmarked SO order.

\ea\label{EC:ex:13}
[Husband and wife fought and a scandal happened and] \\
Sanzhi Dargwa \il{Caucasian (East)!Sanzhi Dargwa}(\citealt{forker_grammar_2020}: 523) \\
\gll [b-aˁq-ib ca-b]\textsuperscript{FOC} sub-li xːunul-li-j \\
\textsc{n-}hit\textsc{.pfv-pret} \textsc{cop-n} husband\textsc{-erg} woman\textsc{-obl-dat} \\
\glt `The husband hit the wife.'
\z

Focus, in particular wide \isi{focus} and contrastive \isi{focus} can also occur after the verb. In some languages, it is especially common with goals, including spatial goals, addressees, recipients, etc. It is possible to have simultaneously pre- and postverbal wide \isi{focus}. In (\ref{EC:ex:14}) from Budukh\il{Caucasian (East)!Budukh}, we have contrasted focal elements in clause-initial position as well as in clause-final position.

\ea\label{EC:ex:14}
Budukh \il{Caucasian (East)!Budukh}(\citealt{talibov_buduxskij_2007}: 273) \\
\gll [q'aǯir-a]\textsuperscript{FOC} suˤre-rber č-aʁ-ar [qːiˤšːlaχ-ǯ-e]\textsuperscript{FOC}, [jaz-ǯ-e]\textsuperscript{FOC}  ʕoˤšχ-ar-i [daʁ-ǯ-a]\textsuperscript{FOC} \\
winter\textsc{-loc} herd\textsc{-pl} \textsc{sub-}go\textsc{-msd} qishlaq\textsc{-obl-loc} autumn\textsc{-obl-loc} return\textsc{-msd-prs} mountain\textsc{-obl-loc} \\
\glt `In winter the herds go to the qishlaqs, in autumn they return to the mountains.'
\z

In Hinuq\il{Caucasian (East)!Hinuq} (\ref{EC:ex:15}), postverbal topics tend to precede postverbal foci (i.e. V-TOP-FOC) rather than the other way around (V-FOC-TOP), which is also the usual order for preverbal topics and foci and corresponds to what has been observed for many languages: known information precedes new information. However, apart from those sentences in which one of the NPs is a \isi{Goal}, two NPs following the verb are not frequently found.

\ea\label{EC:ex:15}
Hinuq \il{Caucasian (East)!Hinuq}(\citealt{forker_grammar_2013}: 759) \\
\gll Ø-ežinnu uži-ž r-aš-a goɬ [hayɬoz]\textsuperscript{TOP} [nasibaw žo]\textsuperscript{FOC} \\
\textsc{i-}old son\textsc{-dat} \textsc{v-}find\textsc{-inf} be he\textsc{.dat} predestined thing(\textsc{v}) \\
\glt `The oldest son will find the thing predestined for him.'
\z

A special context that leads to the occurrence of post-predicate elements are floating modifiers of nouns that are separated from the head noun by other constituents. The head nouns (possibly in combination with other modifiers) are often focal and occur in preverbal position while the floating modifier is displaced postverbally. Especially common are floating genitives in the form of topical personal pronoun\is{pronoun!personal}s and demonstratives as in example (\ref{EC:ex:16}) (see also \citealt{forker_grammar_2020}: 410, 512--518, \citealt{creissels_floating_2013}, \citealt{komen_post-verbal_2017} for more examples). \citet{creissels_floating_2013} analyzes such constructions in Akhvakh. In \isi{contrast} to genitives occurring in their canonical prenominal position floating genitives agree with the head noun in gender and fulfill ``a \isi{possessive} framing function, in the sense that the floating genitive identifies the personal sphere of its referent as the frame within which the predication expressed by the clause holds'' (\citealt{creissels_floating_2013}: 333). 

\ea\label{EC:ex:16}
Icari Dargwa \il{Caucasian (East)!Icari Dargwa}{(\citealt{sumbatova2003grammar}: 160)} \\
\gll č'ug qːatːa-d ħaˁjwan-ti d-ir-iri nišːa-la \\
down canyon\textsc{-n.pl.iness} cattle\textsc{-pl} \textsc{n.pl-}become\textsc{-hab.pst} \textsc{1pl-gen} \\
\glt `Down in the canyon there was our cattle.'
\z

When comparing postverbal subjects with postverbal objects, it seems that the former are more influenced by \isi{information structure} than the latter. \citet{komen_post-verbal_2017}, based on counts in a corpus of Chechen\il{Caucasian (East)!Chechen}, found that one third of Chechen\il{Caucasian (East)!Chechen} subjects occur after the finite verb in main clauses, and of those post-verbal subjects one third are pronominal. Postverbal subjects occur in utterances with presentational \isi{focus} to introduce new referents by means of NPs, in existential clauses, to express paragraph-internal cohesion, i.e. with topical and pronominal subjects, and in reported speech constructions. This is a phenomenon also found in other East Caucasian languages as well as in Northwest Caucasian (\citetv{chapters/11_Forker_Adyghe} and Kartvelian (\sectref{EC:ss:2}). Intransitive \isi{thetic} sentences show very clear \isi{word order} preferences. Presentational sentences that introduce new referents (usually human, but sometimes also non-human, e.g. in fairy tales) frequently place the new referent in post-predicate position (\ref{EC:ex:17}).
 
\ea\label{EC:ex:17}
Ingush \il{Caucasian (East)!Ingush}(\citealt{komen_post-verbal_2017}) \\
\gll Qoalagh='a [qeachaav cwalxa cwa bearii]\textsuperscript{FOC}. \\
third=and arrived alone one horseman \\
\glt `A third lone rider arrived.'
\z

Direct speech constructions where the verb of speech follows the quote often have postverbal subjects (\ref{EC:ex:18}), and this type of construction is also common in the Northwest Caucasian language Adyghe\il{Circassian!Adyghe} (\citetv{chapters/1_Haigetal_Intro}).

\newpage
\ea\label{EC:ex:18}
Chechen \il{Caucasian (East)!Chechen}({\citealt{komen_post-verbal_2017}}) \\
\gll ``t'aaqqa ishkoliehw diesha a aatta xir du,'' oolura txan neenavashas \\
then school\textsc{.loc} learn\textsc{.inf} add easy will be say\textsc{.impf} \textsc{1pl.gen} uncle\textsc{.erg} \\
\glt ` ``Then he would learn more easily at school'' our uncle said.' 
\z

A last factor influencing the likelihood at least for postverbal objects is \isi{language contact}. \tabref{EC:tab:8} summarizes counts in four different text collections in Hinuq\il{Caucasian (East)!Hinuq} \citep{forker_word_2016,forker_impact_2019}. There is almost no difference concerning the position of the \isi{direct object} between the older published texts and my own texts recorded 60 years later (17\%). The pear stories collected with the same \isi{stimulus} as the Georgian\il{Kartvelian!Georgian} texts discussed in \sectref{EC:ss:2.3} have more postverbal objects (25\%). The frog stories produced by speakers under 30 years living in the ethnolinguistically mixed village Monastirski and Shamkhal in the lowlands show an even larger amount of postverbal objects (43\%). This can possibly be attributed to the greater influence of Russian and ongoing \isi{language shift} among young speakers in the lowlands and resembles what has been said about Laz\il{Kartvelian!Laz} in Turkey when compared to the other Kartvelian\il{Kartvelian} languages in Georgia (\sectref{EC:ss:2.3}).

\begin{table}
% \begin{tabularx}{\textwidth}{QQQQQ}
% \lsptoprule
% & old published texts & new traditional texts & pear stories & frog stories \\
% \midrule
% \# words & 1,507 & 2,503 & 1,583 & 2,033 \\
% age of speakers & 14--29 & 12--62 & 13--30 & 19--29 \\
% place of recording & Chechnya & Hinuq & Hinuq & Monastirski, Shamkhal \\
% year & 1950 & 2006--2009 & 2006--2007 & 2013 \\
% \isi{OV} & 139 (82.74\%) & 137 (82.53\%) & 125 (74.40\%) & 72 (57.14\%) \\
% \isi{VO} & 29 (17.26\%) & 29 (17.47\%) & 43 (25.60\%) & \textbf{54 (42.86\%)} \\
% \midrule total & 168 & 166 & 168 & 126 \\
% \lspbottomrule
%  \end{tabularx}
\small
 \begin{tabularx}{\textwidth}{l Y Q r r@{~}r r@{~}r r}
\lsptoprule
  \# words & \mbox{age of} speakers & \mbox{place of} recording & year & \multicolumn{2}{c}{\isi{OV}} & \multicolumn{2}{c}{\isi{VO}} & total \\
\midrule
\multicolumn{6}{l}{\textbf{old published texts}}   \\
\midrule
1,507 & 14--29 & Chechnya              & 1950       & 139& (82.74\%) & 29 & (17.26\%)        & 168 \\
\tablevspace
\multicolumn{6}{l}{\textbf{new traditional texts}} \\
\midrule
2,503 & 12--62 & Hinuq                 & 2006--2009 & 137& (82.53\%) & 29 & (17.47\%)        & 166 \\
\tablevspace
\multicolumn{6}{l}{\textbf{pear stories}}          \\
\midrule
1,583 & 13--30 & Hinuq                 & 2006--2007 & 125& (74.40\%) & 43 & (25.60\%)        & 168 \\
\tablevspace
\multicolumn{6}{l}{\textbf{frog stories}}          \\
\midrule
2,033 & 19--29 & Monastirski, Shamkhal & 2013       & 72 &(57.14\%)  & \textbf{54} & \textbf{(42.86\%)} & 126 \\
\lspbottomrule
\end{tabularx}

 \caption{O-V vs. V-O in Hinuq texts \citep{forker_word_2016,forker_impact_2019}}
 \label{EC:tab:8}
\end{table}

Russian is usually assumed to have free \isi{word order}, but with an underlying \isi{SVO} structure (\citealt{tomlin_basic_1986}, though see the debate in \textsc{Theoretical Linguistics} 48(1--2) 2022, in particular \citet{haider_slavic_2022}). Corpus studies come to different results, but it seems that V-O is more frequent than O-V. For instance, \citet{bazhukov_order_2021} count the order of DO, IO and V for ditransitive verbs in the SynTagRus corpus and get 1420 O-V clauses vs. 4978 V-O clauses. \citet{billings2015corpus} analyzed 500 clauses in the Russian National Corpus (RNC). The most numerous patterns were \isi{SVO} (448) and SOV (22 clauses). However, \citet[856--857]{levshina_why_2023} compared 100 sentences of spoken Russia to 100 sentences of written Russian (Fiction and News) and found remarkable differences between the modalities: the conversations contained 61 examples of \isi{OV}, and only 39 examples of \isi{VO}, whereas both the fiction and news contained 17 examples of \isi{OV} and 83 examples of \isi{VO} each. Hinuq\il{Caucasian (East)!Hinuq} speakers are exposed to written standard Russian through the educational system, through the media, etc., but also to other forms of Russian such as oral (colloquial and standard) Russian through the media and non-standard Russian as spoken in the Caucasus. Russian impact on \isi{word order} patterns in a similar vein as it is possibly found in Hinuq\il{Caucasian (East)!Hinuq} (SOV > \isi{SVO}) has been document for Sakha\il{Turkic!Sakha} (Turkic) \citep{grenoble_evidence_2019} and Udmurt (Uralic) \citep{asztalos_head-final_2021}.

In sum, postverbal items in East Caucasian occur relatively frequently. They fulfil various grammatical functions. In particular goals are prone to occur after the verb. There are also indications that in some languages (Sanzhi Dargwa\il{Caucasian (East)!Sanzhi Dargwa}, Chirag Dargwa\il{Caucasian (East)!Chirag Dargwa}, Chechen\il{Caucasian (East)!Chechen}) part of speech plays a \isi{role} in the sense that nominals and pronouns do not behave alike when it comes to their position with respect to the verb; see \citetv{chapters/11_Forker_Adyghe} for additional parallels. There are two constructions in which postverbal items are frequently found in many East Caucasian languages, namely \isi{thetic} introductory sentences and reported speech constructions, and similar constructions have been identified in Kartvelian\il{Kartvelian} and Adyghe\il{Circassian!Adyghe}. Information structure affects \isi{word order} at the clausal level, but it is not possible to identify any strict rules. This means that topical as well as focal items can appear after the verb. Furthermore, East Caucasian languages have a special construction in which modifiers split from their head and appear in a postverbal position. Finally, data for Hinuq\il{Caucasian (East)!Hinuq} suggest that Russian has an impact on the frequency of postpredicate items, in particular with younger speakers living in ethnically mixed places in the Dagestanian lowlands.


\section{Discussion}\label{EC:ss:4}

All indigenous Caucasian language families (Kartvelian, East Caucasian, but also Northwest Caucasian) are more rigid with respect to \isi{word order} in noun phrases and subordinate clauses and declarative main clauses enjoy the most flexibility.

Post-predicate items in Georgian\il{Kartvelian!Georgian} (Kartvelian) and East Caucasian are relatively common (when compared to Northwest Caucasian) and can be triggered by 

\begin{itemize}
\item certain constructions such as \isi{thetic} utterances and general \isi{information structure}
\item certain semantic roles (e.g. in Georgian\il{Kartvelian!Georgian}, goals, and in Sanzhi Dargwa\il{Caucasian (East)!Sanzhi Dargwa} and Tabasaran\il{Caucasian (East)!Tabasaran}, goals and addressees) show a greater preference than other semantic roles
\item \isi{heaviness} and the presence of relative clauses in Georgian\il{Kartvelian!Georgian} (no data for East Caucasian available)
\end{itemize}

As the data from Laz\il{Kartvelian!Laz} (Kartvelian) show, \isi{language contact} has a strong impact on the flexibility of constituent order at the clausal level and on the presence vs. absence of postpredicate elements. Laz\il{Kartvelian!Laz} is the only one of the Kartvelian languages mainly spoken in Turkey and resembles Turkic with respect to \isi{word order} patterns.

When comparing the three indigenous language families of the Caucasus (and excluding Laz\il{Kartvelian!Laz}), it turns out that Northwest Caucasian languages are the least flexible languages. One is tempted to hypothesize that this is due to their head-marking profile. They have little to no case marking but richer verbal indexing than the other two families. Studies have found a robust negative correlation between rigid \isi{word order} and case marking (\citealt{sinnemaki2014complexity}, \citealt{levshina_cross-linguistic_2021}). 

The label ``flexible SOV'' for Kartvelian and East Caucasian is very coarse-grained and corpus data from different sources show a spectrum of different \isi{word order} patterns and varying degrees of frequencies. \citet{levshina_why_2023} show that \isi{word order} patterns are subject to influence by many factors, some of them competing with each other, so that \isi{word order} flexibility is a common outcome. In their study they mention one East Caucasian language, Avar, as an ``SOV flexible'' languages with a higher degree of flexibility than other languages in the same study (Malayalam, Hindi, Spanish, Korean, and English.), which fits to the data from Chirag\il{Caucasian (East)!Chirag Dargwa}, Sanzhi\il{Caucasian (East)!Sanzhi Dargwa} and Tabasaran\il{Caucasian (East)!Tabasaran} in this paper. 






% \section*{Abbreviations}
% \begin{tabularx}{.45\textwidth}{lQ}
\section*{Abbreviations}
\begin{tabularx}{.45\textwidth}{@{}lQ@{}}
1 & first person \\
3 & third person \\
A & agent \\
\textsc{abl} & {ablative} \\
\textsc{add} & additive \\
\textsc{adv} & adverbial \\
\textsc{aor} & aorist \\
\textsc{appl} & applicative \\
\textssc{cop} & {copula} \\
\textsc{dat} & {dative} \\
\textsc{dem} & demonstrative \\
\textsc{dim} & diminutive \\
\textsc{dist} & distal \\
DO & Direct {object} \\
\textsc{erg} & {ergative} \\
\textsc{ev} & euphonic vowel \\
\textsc{gen} & genitive \\
\textsc{ger} & gerund \\
\textsc{i} & masculine gender \\
\textsc{ii} & feminine gender \\
\textsc{imp} & imperative \\
\textsc{impf} & imperfective \\
\textsc{iness} & inessive \\
\textsc{inf} & infinitive \\
\textsc{inst} & instrumental \\
\textsc{intr} & intransitive \\
\end{tabularx}%
\begin{tabularx}{.45\textwidth}{@{}lQ@{}}
IO & indirect object\is{object!indirect} \\
\textsc{loc} & {locative}\\
\textsc{msd} & masdar \\
\textsc{n} & neuter \\
\textsc{nom} & nominative \\
\textsc{obl} & {oblique} \\
\textsc{pfv} & perfective \\
\textsc{pl} & plural \\
\textsc{pret} & preterite \\
\textsc{prox} & proximate \\
\textsc{prs} & present \\
\textsc{pst} & past \\
\textsc{ptcp} & participle \\
\textsc{pv} & preverb \\
\textsc{refl} & reflexive \\
\textsc{S} & subject (single {argument} of an intransitive {verb}) \\
\textsc{sg} & singular \\
\textsc{sm} & series marker \\
\textsc{spr} & superessive \\
\textsc{sub} & subjunctive \\
\textsc{supine} & supine \\
\textsc{tm} & thematic marker (present stem formant) \\
\textsc{v} & gender V \\
\textsc{ver} & version \\
\end{tabularx}

% \end{tabularx}
% \begin{tabularx}{.45\textwidth}{lQ}

% \end{tabularx}

\sloppy
\printbibliography[heading=subbibliography,notkeyword=this]

\end{document}
