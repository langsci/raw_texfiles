\documentclass[output=paper,colorlinks,citecolor=brown,draftmode]{langscibook}
\ChapterDOI{10.5281/zenodo.14266355}
\author{Katherine Hodgson\orcid{0000-0002-1767-9262}\affiliation{University of Cambridge} and Victoria Khurshudyan\affiliation{Institut National des Langues et Civilisations Orientales (INALCO)} and Pollet Samvelian\affiliation{Université Paris III - Sorbonne Nouvelle}}
\title{Post-predicate arguments in Modern Eastern Armenian}
\abstract{This study uses a corpus of oral narratives to investigate the positioning of DOs, as well as other arguments, including goals, in Modern Eastern Armenian (MEA). We find a preference for OV, very strong for indefinite DO (96.8\% OV), weaker for definite DO (67\% OV). Animacy and weight appear to have a slight effect favouring VO, but the numbers are not statistically significant. Definiteness, animacy, and weight also appear to have a slight effect favouring post-predicate position for other roles. Other roles articulated as pronouns occur less frequently in post-predicate position than lexical NPs. There is no clear evidence that any of the other factors investigated (givenness, topicality, crowding effect, verb type, clause type) affects the position of arguments. The (generally informal spoken) MEA data from this study exhibit  similarities to the ``OVX'' pattern that characterizes comparable registers in languages of the Mesopotamia region, with goals showing a preference for post-predicate position (68.9\%), in contrast to other arguments.}

%move the following commands to the "local..." files of the master project when integrating this chapter
% \usepackage{tabularx}
% \usepackage{langsci-optional}
% \usepackage{langsci-gb4e}
% \usepackage{enumitem}
% \bibliography{localbibliography}
% \newcommand{\orcid}[1]{}
% \let\eachwordone=\itshape


\IfFileExists{../localcommands.tex}{
 \addbibresource{../collection_tmp.bib}
 \addbibresource{../localbibliography.bib}
 \usepackage{langsci-optional}
\usepackage{langsci-gb4e}
\usepackage{langsci-lgr}

\usepackage{listings}
\lstset{basicstyle=\ttfamily,tabsize=2,breaklines=true}

%added by author
% \usepackage{tipa}
\usepackage{multirow}
\graphicspath{{figures/}}
\usepackage{langsci-branding}

 
\newcommand{\sent}{\enumsentence}
\newcommand{\sents}{\eenumsentence}
\let\citeasnoun\citet

\renewcommand{\lsCoverTitleFont}[1]{\sffamily\addfontfeatures{Scale=MatchUppercase}\fontsize{44pt}{16mm}\selectfont #1}
  
 %% hyphenation points for line breaks
%% Normally, automatic hyphenation in LaTeX is very good
%% If a word is mis-hyphenated, add it to this file
%%
%% add information to TeX file before \begin{document} with:
%% %% hyphenation points for line breaks
%% Normally, automatic hyphenation in LaTeX is very good
%% If a word is mis-hyphenated, add it to this file
%%
%% add information to TeX file before \begin{document} with:
%% %% hyphenation points for line breaks
%% Normally, automatic hyphenation in LaTeX is very good
%% If a word is mis-hyphenated, add it to this file
%%
%% add information to TeX file before \begin{document} with:
%% \include{localhyphenation}
\hyphenation{
affri-ca-te
affri-ca-tes
an-no-tated
com-ple-ments
com-po-si-tio-na-li-ty
non-com-po-si-tio-na-li-ty
Gon-zá-lez
out-side
Ri-chárd
se-man-tics
STREU-SLE
Tie-de-mann
}
\hyphenation{
affri-ca-te
affri-ca-tes
an-no-tated
com-ple-ments
com-po-si-tio-na-li-ty
non-com-po-si-tio-na-li-ty
Gon-zá-lez
out-side
Ri-chárd
se-man-tics
STREU-SLE
Tie-de-mann
}
\hyphenation{
affri-ca-te
affri-ca-tes
an-no-tated
com-ple-ments
com-po-si-tio-na-li-ty
non-com-po-si-tio-na-li-ty
Gon-zá-lez
out-side
Ri-chárd
se-man-tics
STREU-SLE
Tie-de-mann
}
%  \boolfalse{bookcompile}
%  \togglepaper[5]%%chapternumber
}{}


\begin{document}
\maketitle\label{WOWA:ch:13}

\section{Introduction}\label{Armenian:ss:1}
\subsection{Overview}
\begin{sloppypar}
Modern Eastern Armenian\il{Armenian (Eastern)} (MEA) is generally grouped with SOV languages based on the branching direction of its various constituents (quite consistently left-branching), but also some syntactic properties of its VP (e.g. the preverbal position of the \isi{focus} and bare/indefinite objects), which are assumed to be characteristic of \isi{OV} languages. The quantitative study of \citet{stilo_preverbal_2018} also shows a clear preference for \isi{OV} order. However, according to some studies (\citealt{dum-tragut_armenian_2009,badikyan_zamanakakic_1976}, among others), definite DOs display a clear preference for the postverbal position, and grammars of literary MEA generally consider \isi{VO} to be the unmarked \isi{word order}. In a recent study, \citet{samvelian_persistence_2023} conducted a quantitative investigation of \isi{word order} in MEA. Based on a small task-oriented corpus and two experiments, they conclude that (S)\isi{VO} order cannot be qualified as a marked option in MEA, since definite DOs display a strong preference for the \isi{VO} order in “out-of-the-blue” sentences. Their results confirm, on the other hand, the preference of non-definite DOs for the \isi{OV} order, indicating furthermore that indefinite (with the indefinite article) and bare DOs are equally likely to favor the \isi{OV} order. Apart from \isi{definiteness}, length is also mentioned as a relevant factor, since long indefinite and bare DOs are more likely to behave like definite DOs by appearing in postverbal position.
\end{sloppypar}

Apart from DOs, other arguments, especially goals of verbs of motion and caused motion, may also occur in post-predicate position in MEA. This phenomenon has been shown to have an areal dimension, being characteristic of many languages of Western Asia; see contributions to this volume. The purpose of the present study is to investigate factors affecting the position (pre- vs. post-predicate) of DOs, as well as certain other types of arguments that may occur variably in pre- or post-predicate position. Specifically, we aim to check whether the same \isi{word order} preferences for the placement of DOs reported in \citet{samvelian_persistence_2023} also hold in a corpus of unscripted spoken language,  as well as to investigate the \isi{word order} preferences of goals and other types of arguments that show variable position. Besides \isi{definiteness} and length, we also investigate the relevance of some other factors in the placement of DOs and other arguments:

\begin{itemize}
    \item Topicality: It has been claimed that \isi{word order} in MEA is closely linked to \isi{information structure} \citep{hodgson_relative_2019}. Given that our corpus is annotated for \isi{topic} persistence and \isi{referential distance}, as well as other factors connected to \isi{topicality}, such as \isi{givenness} and \isi{animacy}, we can check whether there is a correlation between \isi{topicality} and postverbal placement.
    \item Type of referential element (lexical NP vs. \isi{pronoun}).
    \item Crowding effect: \citet{hayrapetyan_ui_1981} mentions the presence (or not) of the subject as a relevant feature in the pre- or postverbal placement of the DO in Classical Armenian\il{Armenian (Classical)}. This factor may also be relevant in MEA.
    \item We also investigate the possibility that clause type (main vs. subordinate) and verb type (simple vs. periphrastic) could have an effect on the position of objects.
\end{itemize}

\subsection{Typological preliminaries and background}\label{Armenian:ss:1.1}
\largerpage
The position of direct objects in Eastern Armenian\il{Armenian (Eastern)} is somewhat controversial. Although Armenian is often assumed to be an \isi{OV} language, Eastern Armenian\il{Armenian (Eastern)} in particular shows considerable flexibility in the position of objects, especially definite objects, and in WALS is described as having no dominant order \citep{dryer_order_2013_OV}. The placement of DOs has been the \isi{focus} of a few studies, including corpus-based and experimental ones. \citet{samvelian_persistence_2023} provide a thorough survey of the literature on the issue. They note that despite the fact that typological and theoretical syntactic studies have generally grouped Armenian with SOV languages (\citealt[227--228]{der-houssikian_semantic_1978}, \citealt[286, 310]{dryer_aspects_1998}, \citealt{dum-tragut_word_2002}, \citealt[190]{giorgi_word_2016}, \citealt[625]{hawkins_implicational_1979}, \citeyear[286]{hawkins_word_1983}, \citealt[6]{hodgson_discourse_2013}, \citealt[81]{kahnemuyipour_second_2011,kahnemuyipour_positional_2017}, \citealt[8]{kozintseva_modern_1995}, \citealt[263]{minassian_grammaire_1980}, \citealt[101]{tamrazian_focus_1991}, \citeyear[7]{tamrazian_syntax_1994}, among others), Armenian grammars and descriptive studies, on the other hand, generally refer to the postverbal position as the “natural” position of the \isi{object}. This is either overtly claimed based on small scale quantitative studies \citep{badikyan_zamanakakic_1976}, or is induced by the fact that in most examples illustrating transitive sentences and the DO, the latter is placed in the postverbal position.

Given the fact that OV-\isi{VO} variation is not trivial in MEA, \citet{samvelian_persistence_2023} build on quantitative methods in order to provide a reliable picture of \isi{word order} distribution in MEA, as well as different factors that may be involved in the choice of \isi{OV} versus \isi{VO} order. The main factor investigated by \citet{samvelian_persistence_2023} is \isi{definiteness}. Indeed, several studies have mentioned the relevance of this factor in the placement of the DO (\citealt{badikyan_zamanakakic_1976}, \citealt{dum-tragut_armenian_2009}, \citealt{stilo_preverbal_2018}), with definite DOs occurring more frequently in postverbal position. In a preliminary corpus investigation based on a sample of 570 transitive sentences from the EANC (Eastern Armenian National Corpus, see below), \citet{FaghiriSamvelian2020SOV} report that definite DOs are overwhelmingly postverbal (79.1\% \isi{VO} vs. 20.9\% \isi{OV}). 

The results of \citet{samvelian_persistence_2023} are based on a small-scale corpus study and 2 experiments. Their task-oriented spoken corpus includes recordings of picture-based storytelling from 10 native speakers of MEA (see \citealt{khurshudyan2006sredstva} and \citealt{samvelian_persistence_2023} for a detailed description of the corpus and its annotation). Their corpus contains 231 finite declarative transitive sentences with SOV or \isi{SVO} orders. The results confirm that definite DOs tend to appear in the postverbal position. \citet[476]{samvelian_persistence_2023} also show the relevance of length in the placement of the DO: “long and/or heavy bare and indefinite DOs appear frequently in the \isi{SVO} order, while simple (that is short or minimal) bare and indefinite DOs are more likely to appear in the SOV order.”

\citet{samvelian_persistence_2023} also provide two sentence production experiments. The first experiment, which compares definite and indefinite DOs, uses a cued sentence recall protocol, and the second one, which compares indefinite and bare DOs, uses constrained sentence production protocol. Furthermore, \citet{samvelian_persistence_2023} study the effect of \isi{animacy} and length. 

In the first experiment, 82.7\% of definite DOs occur in the postverbal position, while the percentage of postverbal indefinite DOs is 38.3\%. As for \isi{animacy}, although its main effect is not significant, it nevertheless has an effect on the order with definite DOs: animate definite DOs favor \isi{SVO} more than inanimate definite DOs do. The second experiment shows that both indefinite and bare objects are overwhelmingly preverbal: 74.4\% vs. 75.7\% respectively. The experiment also confirms the effect of length: non-simple (or long) DOs are more likely to appear in the \isi{SVO} order.

To sum up, both \citet{FaghiriSamvelian2020SOV} and \citet{samvelian_persistence_2023} report a clear-cut divide between definite DOs on the one hand and indefinite and bare DOs on the other hand. In both corpus-based and experimental studies, definite DOs are overwhelmingly postverbal. Therefore, not only does \isi{definiteness} seem to be the best predictor for the pre- or postverbal placement of the DO, but also postverbal position seems to be the default placement for definite DOs.

The results of \citet{samvelian_persistence_2023} are in sharp \isi{contrast} with the findings of \citet{stilo_preverbal_2018}. In his data from Colloquial Yerevan Armenian\il{Armenian (Eastern)!Yerevan}, \citet{stilo_preverbal_2018} finds that both definite and indefinite DOs show a strong preference for preverbal position: indefinite DOs are 95\% preverbal, while definite DOs show a slightly weaker, but still strong preference for preverbal position (86.1\%). The other dialects of Armenian included in the study of \citet{stilo_preverbal_2018} also show a general strong preference for preverbal position in the case of both indefinite and definite DOs: indefinite DOs are preverbal in 100\% of cases in the dialects of Erzurum and Stepanakert, and 81\% preverbal in Lori, while definite DOs are 98\% preverbal in Stepanakert, 91.9\% preverbal in Erzurum, and 83.2\% preverbal in Lori.

The \isi{contrast} between the results of \citet{stilo_preverbal_2018} and those of \citet{samvelian_persistence_2023} can be accounted for by several facts: dialectal variation, the language \isi{register}, and the “genre” of the corpus (discourse, oral, written, etc.), as well as the fact that the data in the experimental studies of \citet{samvelian_persistence_2023} include only “out-of-the-blue” declarative non-embedded sentences, which may be considered unmarked, while the present study includes all types of sentences. The data from colloquial Yerevan used by \citet{stilo_preverbal_2018}, which have not been available to us, might also reveal other reasons for the discrepancies. A similar pattern has been observed in Romeyka\il{Hellenic!Romeyka}, where those studies based on elicitation of out-of-the-blue sentences, with overt subjects and objects, yield predominantly \isi{VO} structures, while the data from connected spontaneous spoken discourse show a higher rate of \isi{OV} (\citetv{chapters/12_Schreiber_Romeyka}). As \citet{samvelian_persistence_2023} note in their conclusion, it is very likely that the rate of SOV is higher in spontaneous oral discourse. It is also likely that geographical variation, perhaps associated with different \isi{language contact} patterns, is relevant, as \citet{stilo_preverbal_2018} presents data from various dialects spoken in different parts of the Armenian-speaking world. However, even his data from Yerevan show a far lower percentage of post-predicate arguments than those of \citet{samvelian_persistence_2023}.

In this context, it is important to note that formal literary MEA, as used in writing and formal speech, shows significant differences in morphology, \isi{phonology}, and syntax\footnote{Morphological differences between formal literary EA and colloquial EA as spoken in Yerevan include the form of the 3sg. present \isi{auxiliary} (formal \textit{e} vs. colloquial \textit{a}) and certain forms of the ``emphatic'' \isi{pronoun} (formal nom.pl. \textit{irenkʿ} vs colloquial \textit{irankʿ}, formal gen.sg. \textit{ir}, dat.sg. \textit{iren} vs. colloquial gen.sg. \textit{ira}, dat.sg. \textit{iran}) among many others.  Phonological differences include the change of the diphthong /ay/ > [e], as in the demonstratives (distal), which have the forms \textit{es}, \textit{ed}, \textit{en} in the colloquial language; see also the formal/colloquial correspondences \textit{hayr/her} `father,' \textit{mayr/mer} `mother,' \textit{dzayn/dzen} `voice,' \textit{layn/len} `wide,' etc. Syntactic differences include the use of different cases with certain adpositions, for example the use of the \isi{dative} of 1st and 2nd person pronouns with postpositions such as \textit{het} `with,' \textit{mot} `close to, by,' \textit{hamar} `for' etc. in the formal literary language, as opposed to the genitive in the colloquial language: formal \textit{indz het} `with me,' \textit{indz mot} `close to me,' \textit{indz hamar} `for me' vs. colloquial \textit{im mot, im het, im hamar}, as well as differences in relativization strategies (notably use of indeclinable complementizer instead of declined relative \isi{pronoun}) described in \citet{hodgson_relative_2019}.} from any form of colloquial spoken Armenian, including the colloquial MEA spoken in Yerevan from which Stilo's data are taken. As stated in the introduction, \isi{VO} order is promoted as the unmarked \isi{word order} in grammars of the formal literary language.\footnote{Donabédian (p.c.) even reports hearing a teacher of formal literary Western Armenian state that \isi{OV} order is ``Turkish,'' and therefore incorrect.} It is likely that the difference between the results of \citet{stilo_preverbal_2018}, even for speakers from Yerevan, and those of \citet{samvelian_persistence_2023} reflects the fact that the former deals with colloquial language, and the latter includes data with characteristics of formal literary language,\footnote{For example, examples (18--21) in \citet{samvelian_persistence_2023}, taken from their corpus, show features characteristic of formal literary language (3sg. aux. \textit{e},  genitive of ``emphatic'' \isi{pronoun} \textit{ir} etc.), and while the words used in the experiments could be described as register-neutral, the instructions that appear on the screen as shown in \citet{samvelian_persistence_2023} Fig. 1 are in formal literary EA, which is likely to have prompted the use of this form of language in the responses.} indicating another potential syntactic difference between literary and colloquial MEA. The present study provides further evidence in support of the proposal that \isi{register} is a key factor in the difference between the results of \citet{stilo_preverbal_2018} and \citet{samvelian_persistence_2023}, since the ``outlier'' speaker with the highest percentage of \isi{VO} (speaker 3, with 44\% \isi{VO}, as opposed to an average of 17.4\% \isi{VO} for all the other speakers) is also the only one who uses certain word forms associated with the formal literary language (see \tabref{Armenian:tab:3} and associated discussion). These findings echo similar findings from Persian\il{Persian}, where different grades of formality (and the difference between spoken and written modalities) have a significant impact on certain aspects of \isi{word order}, see \textcitetv{chapters/7_RasekhMahandetal_Persian}) for details.

The discussion of the placement of DOs in \citet{stilo_preverbal_2018} takes place in context of the wider issue of post-predicate constituents in many \isi{OV} languages of Western Asia. These languages differ from ``rigid'' \isi{OV} languages in that they regularly allow at least certain constituents to appear in post-predicate position, giving (O)VX \isi{word order}. \citet{stilo_preverbal_2018} and subsequent studies show that the type of constituent that is most common and widespread in post-predicate position is \isi{Goal} of verbs of motion or caused motion (see \citetv{chapters/1_Haigetal_Intro}, for an overview). In some languages, similar behaviour is shown by other constituents which, like goals, can be considered to have the semantic property of ``\isi{endpoint},'' namely \isi{recipient}, \isi{benefactive}, adddressee, and \isi{object} of change of state predicates such as `become'. Arguments without this semantic property, such as \isi{ablative}, instrumental, \isi{locative}, and \isi{comitative}, do not typically appear in post-predicate position in these languages (see \citealt{haig_introduction_2018}, \citealt{stilo_preverbal_2018}). The data in \citet{stilo_preverbal_2018}, which include four different dialects of Armenian (Erzurum\il{Armenian (Western)!Erzurum}, Lori\il{Armenian (Eastern)!Lori}, Stepanakert\il{Armenian (Eastern)!Stepanakert}, and colloquial Yerevan\il{Armenian (Eastern)!Yerevan}), indicate that in all of these, goals show a preference for post-predicate position. In some dialects, \isi{recipient} and \isi{benefactive} also show a tendency to appear in post-predicate position, though not as frequently as goals. Addressee arguments do not seem to be affected by this tendency in Armenian, showing 90--100\% pre-predicate position in all the dialects investigated. In fact, most of the historically \isi{OV} languages in this volume exhibit the same split (see \citetv{chapters/1_Haigetal_Intro},  \sectref{Intro:ss:4}), distinguishing addressees from spatial goals. Instrumental, \isi{ablative}, \isi{comitative}, and \isi{locative} are predominantly pre-predicate across the whole area.

This pattern is associated with a particular geographic location, with the epicentre in the Mesopotamian region (modern Northern Iraq, Western Iran, and southeastern Turkey) (\citealt{haig_introduction_2018}, \citetv{chapters/1_Haigetal_Intro}). As for the typological profile of languages showing OVX \isi{word order}, these have been characterised as \isi{OV} languages showing some properties typical of \isi{VO} languages, such as prepositions and/or initial complementizers. The phenomenon reflects the area's status as a ``\isi{transition zone}'' from Turkic-type \isi{head-final}, through Iranian mixed typologies, to Semitic \isi{head-initial} (\citealt{stilo2006circumpositions}, \citealt{haig_introduction_2018}, \citetv{chapters/1_Haigetal_Intro}). Armenian, too, could be described as a language with mixed typology, having undergone a change from mainly \isi{head-initial} Classical Armenian\il{Armenian (Classical)} (prepositions, probable \isi{VO} preference) to mainly \isi{head-final} modern Armenian, which does, however, retain some properties associated with \isi{head-initial} languages, such as initial complementizers. Various proposals have been made regarding the causes of the OVX phenomenon, including iconicity (arguments with \isi{endpoint semantics} appearing in final position), and contact with languages such as Aramaic (or Russian) that typically have post-predicate arguments (\citealt{haig_introduction_2018}, \citetv{chapters/1_Haigetal_Intro}). However, in the case of Armenian, it is possible that post-predicate arguments represent a conservative feature retained from Classical Armenian\il{Armenian (Classical)}. This could apply to the presence of \isi{VO} orders as well as OVX, as suggested by both \citet{samvelian_persistence_2023} and \citet{stilo_preverbal_2018}, although the fact that post-predicate Goals\is{Goal!post-verbal}, but not objects, seem to appear less frequently in subordinate clauses, which have been claimed to show a general tendency for more conservative syntax, perhaps suggests the possibility that OVX may be a more recent phenomenon.

\section{The Methodology and the corpus}\label{Armenian:ss:2}
\subsection{Overview}
The corpus is composed of 7 oral narratives by 7 participants (6 women and a man), who narrate their favorite movies. It was compiled within the framework of the Eastern Armenian National Corpus\footnote{Eastern Armenian National Corpus is a comprehensive corpus of Modern Eastern Armenian\il{Armenian (Eastern)} comprising approximately 110 million tokens. It encompasses written and oral data starting from the mid 19th century to the present. The corpus is provided with full morphological annotation, offering robust search functionalities, and is openly accessible at \url{www.eanc.net}. For more details on Eastern Armenian National Corpus, see \citet{khurshudyan-etal-2022-eastern}.} (henceforth EANC ArmFilmNarr). The recording was done in Yerevan in 2007--2008, and the total recording time is 1.35h. The corpus contains 11,832 tokens, divided into 2,241 clauses. The corpus was originally created for the study of \isi{topic} accessibility and continuity of subjects and DO in MEA (in a distinct research project) based on a framework inspired by \citet{givon_focus_1975}, \citet{givon_topic_1983} and \citet{du_bois_discourse_1987} (for more information on this study, see \citealt{hodgson_word_nodate}).  The purpose of the present study is to investigate factors affecting the position (pre- vs. post-predicate) of DOs, as well as certain other types of arguments that may occur variably in pre- or post-predicate position. Thus, we note the position (pre-or post-predicate) of monotransitive DO, \isi{Goal}, \isi{addressee}, \isi{recipient}, \isi{benefactive}, \isi{endpoint} of `become,' \isi{ablative}, instrumental, \isi{locative}, and \isi{comitative} arguments. For each of these, we note other factors that could potentially affect the position of the \isi{argument}: \isi{weight} (number of words/\isi{intonation} words), type of \isi{anaphoric} element (e.g \isi{pronoun}\is{anaphoric!pronoun}\is{pronoun!!anaphoric}
vs. definite NP vs. indefinite NP), \isi{animacy}, and \isi{givenness}. The presence and position of an overt subject \isi{argument} is also noted, in order to test for crowding effects, i.e. the tendency to avoid having more than one overt \isi{argument} on the same side of the predicate. The type of verb form is also noted, as the distinction between periphrastic and simple verbs could potentially affect the position of arguments, as could clause type (main vs. subordinate clause). For DO, figures are also given for \isi{topic} persistence (number of mentions of the referent in the following 10 clauses) and \isi{referential distance} (distance in clauses to previous mention of the referent, noted as 1, 2, or 3, with 3 indicating 3 or more). \figref{Armenian:fig:1} provides an illustrative example.

\setlength{\tabcolsep}{3pt}
\begin{figure}
    \fittable{\begin{tabular}{|c|c|c|c|c|c|c|c|c|c|c|c|c|c|c|c|c|c|c|c|c|c|c|c|c|c|c|c|c|c|c|c|c|c|c|c|c|c|c|c|c|c|c|c|c|c|c|c|c|c|c|c|c|c|c|c|c|c|c|c|c|c|c|c|c|c|c|c|c|c|c|c|c|}
\toprule
\rotatebox{90}\# & \rotatebox{90}{text} & \rotatebox{90}{Predicate type} & \rotatebox{90}{Voice} & \rotatebox{90}{Order}  & \rotatebox{90}{\cellcolor{gray!30}Arg.O} & \rotatebox{90}{\cellcolor{green!30}Arg.A} & \rotatebox{90}{\cellcolor{gray!30}Pers.O} & \rotatebox{90}{\cellcolor{green!30}Pers.A} & \rotatebox{90}{\cellcolor{gray!30}H.O} & \rotatebox{90}{\cellcolor{green!30}H.A} & \rotatebox{90}{\cellcolor{gray!30}NG.O} & \rotatebox{90}{\cellcolor{green!30}NG.A} & \rotatebox{90}{\cellcolor{gray!30}Pre/PostV O} & \rotatebox{90}{\cellcolor{green!30}Pre/PostV A} & \rotatebox{90}{\cellcolor{gray!30}Weight O} & \rotatebox{90}{\cellcolor{green!30}Weight A} & \rotatebox{90}{Other role} & \rotatebox{90}{Weight} & \rotatebox{90}{Type} & \rotatebox{90}{Animacy} & \rotatebox{90}{Given/New} & \rotatebox{90}{Pre/post} & \rotatebox{90}{Others} & \rotatebox{90}{Verb form} & \rotatebox{90}{Clause type} & \rotatebox{90}{Force, Negation} & \rotatebox{90}{\cellcolor{gray!30}RD.O} & \rotatebox{90}{\cellcolor{green!30}RD.A} & \rotatebox{90}{\cellcolor{gray!30}TP.O} & \rotatebox{90}{\cellcolor{green!30}TP.A} \\
\midrule
59 & 2\_F & \textsc{tr} &  & \textsc{locOVC} & NP\textsc{.def} & V.\textsc{agr} & 3 & 3 & NH & H & N & G & Pre &  & 1 & & Loc & 1 & \textsc{pro.loc} & I & N & Pre &  & IPT-C &  &  & 3 & 1 & 0 & 9\\
\bottomrule
% \multicolumn{42}{c}{/ ընդեղ / դուռը բացու՛մ ա,}\\
\multicolumn{31}{c}{\large\textit{/ əndeγ / duŕə bacʿum a}}\\
\multicolumn{31}{c}{\large`there she opens the door'}\\
    \end{tabular}}
    \caption{Example of annotation of a sentence}
    \label{Armenian:fig:1}
\end{figure} \setlength{\tabcolsep}{6pt}


% \begin{figure}
%     \centering
%     \includegraphics[width=\linewidth]{figures/Armenian_fig1.png}
%     \caption{Example of annotation of a sentence}
%     \label{Armenian:fig:1}
% \end{figure}

In the first column (clause-arm), we find the clause (in \figref{Armenian:fig:1}, the clause and its translation are shown below the columns). The second column shows the type of predicate (here ``tr,'' i.e. monotransitive). There is a column for voice, which is left blank when active. The \isi{word order} of the clause as a whole is shown (LocOVC, i.e. Locative, Object, (lexical) Verb, Copula/\isi{auxiliary}). Then, the properties of O are noted in the grey columns, and A in the green columns. These include type of \isi{argument} (O is NP.DEF, i.e. definite NP, and A is V.agr, i.e. verb agreement), person (3sg for both), \isi{animacy} (H = human, NH = non-human), \isi{givenness} (G = given, N = new), the position of overt arguments (Pre, i.e. pre-predicate, for O, A is left blank, because it is expressed by verb agreement alone), and \isi{weight} of overt arguments (blank for A, 1, i.e. one word, for O). The next column shows that the clause also contains a \isi{locative} (Loc). In the following columns, its values for \isi{weight} (1), type of \isi{argument}, i.e. type of \isi{anaphoric} element (PRO, i.e. proform), \isi{animacy}, \isi{givenness}, and position are given. Other columns show the verb type (IPT-C, i.e. imperfective participle + \isi{copula}/\isi{auxiliary}), which is of interest particularly in terms of simple (monolectic) vs. complex (participle + \isi{auxiliary}) verb forms, and the clause type (left blank for main clauses). Referential distance (RD) and \isi{topic} persistence (TP) are noted in separate columns for subjects and objects of main clauses in which both are 3rd person. Examples of the main categories of \isi{anaphoric} elements (examples \ref{Armenian:ex:1}-\ref{Armenian:ex:10}) and verb types (examples \ref{Armenian:ex:11}-\ref{Armenian:ex:17}) are given below:


\subsection{Anaphoric elements}

\ea\label{Armenian:ex:1}
\textbf{Verb agreement} (for subject only) \\
Eastern Armenian \il{Armenian (Eastern)} \\
\gll čʿ-git-\textbf{em} \\
\textsc{neg-}know\textsc{-1sg.prs} \\
\glt `\textbf{I} don't know.'
\z

\ea\label{Armenian:ex:2}
\textbf{Zero anaphora} (especially for DO, but also some other object-like elements: example (\ref{Armenian:ex:2}) shows zero anaphora for both DO and \isi{recipient})\\
Eastern Armenian \il{Armenian (Eastern)} \\
\gll dra hamar a nvir-um \\
\textsc{dem2.gen} for be\textsc{.3sg.prs} give\textsc{-ipfv} \\
\glt `He gives \textbf{it to her} because of that.'
\z
 
\ea\label{Armenian:ex:3}
\textbf{Agreement marker} (effectively genitive \isi{clitic}, can be used for objects of some adpositions) \\
Eastern Armenian \il{Armenian (Eastern)} \\
\gll aγjik-ə het-\textbf{ə} xos-um a \\
girl\textsc{-def} with\textsc{-agr3} speak\textsc{-ipfv} be\textsc{.3sg.prs} \\
\glt `The girl talks with \textbf{him}.'
\z

\ea\label{Armenian:ex:4}
\textbf{Personal pronoun} \\
Eastern Armenian \il{Armenian (Eastern)} \\
\gll \textbf{kʿez} šat em sir-um \\
\textsc{2sg.dat} much be\textsc{.1sg.prs} love\textsc{-ipfv} \\
\glt `I love \textbf{you} very much.'
\z

\ea\label{Armenian:ex:5}
\textbf{Demonstrative} \\
Eastern Armenian \il{Armenian (Eastern)} \\
\gll \textbf{ed} bacʿ-um en \\
\textsc{dem2} open\textsc{-ipfv} be\textsc{.3pl.prs} \\
\glt `They open \textbf{that}.'
\z

\ea\label{Armenian:ex:6}
\textbf{``Emphatic''\footnote{For a discussion of this element and its behaviour in this corpus, see \citet{hodgson_word_nodate}. See also \citet{donabedian-demopoulos_recherche_2007}, \citet{sigler_logophoric_2001}}}. \\
Eastern Armenian \il{Armenian (Eastern)} \\
\gll menkʿ \textbf{iran} k-gtn-enkʿ \\
\textsc{1pl.nom} \textsc{emp.dat} \textsc{fut-}find\textsc{-1pl.prs} \\
\glt `We will find \textbf{him}.'
\z

\ea\label{Armenian:ex:7}
\textbf{Pro-adverb} \\
Eastern Armenian \il{Armenian (Eastern)} \\
\gll \textbf{əndeγ} duŕ-ə bacʿ-um a \\
there door\textsc{-def} open\textsc{-ipfv} be\textsc{.3sg.prs} \\
\glt `\textbf{There} she opens the door.'
\z

\ea\label{Armenian:ex:8}
\textbf{Bare NP} \\
Eastern Armenian \il{Armenian (Eastern)} \\
\gll \textbf{takʿsi} a gal-is \\
taxi be\textsc{.3sg.prs} come\textsc{-ipfv} \\
\glt `\textbf{A taxi} comes.'
\z

\ea\label{Armenian:ex:9}
\textbf{NP with indefinite article} \\
Eastern Armenian \il{Armenian (Eastern)} \\
\gll tencʿ \textbf{mi} \textbf{takʿsi} gal-is a \\
thus \textsc{ia} taxi come\textsc{-ipfv} be\textsc{.3sg.prs} \\
\glt `So \textbf{a taxi} comes.'
\z

\ea\label{Armenian:ex:10}
\textbf{Definite NP} \\
Eastern Armenian \il{Armenian (Eastern)} \\
\gll \textbf{takʿsi-n} kangn-acʿn-um en  \\
taxi\textsc{-def} stand\textsc{-caus-ipfv} be\textsc{.pl.prs} \\
\glt `They stop \textbf{the taxi}.'
\z


\subsection{Verb types}
\subsubsection{Simple}

\ea\label{Armenian:ex:11}
\textbf{Monolectic present of some verbs} \\
Eastern Armenian \il{Armenian (Eastern)} \\
\gll \textbf{un-i} ir ŕistaran-ə  \\
have\textsc{-3sg.prs} \textsc{emp.gen} restaurant\textsc{-def} \\
\glt `He has his restaurant.'
\z

\ea\label{Armenian:ex:12}
\textbf{Aorist} \\
Eastern Armenian \il{Armenian (Eastern)} \\
\gll dukʿ mard \textbf{span-ecʿikʿ}  \\
\textsc{2pl.nom} person kill\textsc{-2pl.aor} \\
\glt `You killed a person.' 
\z

\ea\label{Armenian:ex:13}
\textbf{Subjunctive} \\
Eastern Armenian \il{Armenian (Eastern)} \\
\gll vor iran dur \textbf{ga}  \\
\textsc{comp} \textsc{emp.dat} like come\textsc{.3sg.sub} \\
\glt `so that she likes it' 
\z

\ea\label{Armenian:ex:14}
\textbf{Future/conditional forms with the prefix k-}  are also classed as ``simple'' here \\
Eastern Armenian \il{Armenian (Eastern)} \\
\gll \textbf{k-ogn-enkʿ}  \\
 \textsc{fut-}help\textsc{-1pl} \\
\glt `We will help.' 
\z


\subsubsection{Complex (participle + \isi{auxiliary})}


\ea\label{Armenian:ex:15}
\textbf{Present} \\
Eastern Armenian \il{Armenian (Eastern)} \\
\gll heto taŕ-er-ə \textbf{gr-um} \textbf{a}  \\
after letter\textsc{-pl-def} write\textsc{-ipfv} be\textsc{.3sg.prs} \\
\glt `After (that) it writes the letters.'
\z

\ea\label{Armenian:ex:16}
\textbf{Perfect} \\
Eastern Armenian \il{Armenian (Eastern)} \\
\gll heto arden bolševik-ner-ə \textbf{mt-el} \textbf{en} Hayastan  \\
after already bolshevik\textsc{-pl-def} enter\textsc{-pfv} be\textsc{.3pl.prs} Armenia \\
\glt `Then the Bolsheviks have already entered Armenia.'
\z

\ea\label{Armenian:ex:17}
\textbf{Future} \\
Eastern Armenian \il{Armenian (Eastern)} \\
\gll Yes \textbf{gn-alu em}  \\
\textsc{1sg.nom} go\textsc{-fpt} be\textsc{.1sg.prs} \\
\glt `I will go.'
\z


\section{Results and analyses}\label{Armenian:ss:3}

\subsection{Word order variation}\label{Armenian:ss:3.1}

Our corpus of oral narratives of favorite films comprising 2,241 clauses shows, overall, twelve possible \isi{word order} combinations (see \tabref{Armenian:tab:1}).


\begin{table}
    \begin{tabularx}{.8\textwidth}{XlYY}
\lsptoprule
\# & WO & \# & \% \\
\midrule
1. & SV & 415 & 41.8\% \\
2. & \isit{OV} & 275 & 27.7\% \\
3. & AOV & 94 & 9.5\% \\
4. & \isit{VO} & 69 & 6.9\% \\
5. & VS & 52 & 5.2\% \\
6. & AVO & 32 & 3.2\% \\
7. & AV & 30 & 3.0\% \\
8. & OVA & 10 & 1.0\% \\
9. & OAV & 10 & 1.0\% \\
10. & VA & 3 & 0.3\% \\
11. & VAO & 2 & 0.2\% \\
12. & VOA & 1 & 0.1\% \\ 
\midrule
  & Total & 993\footnote{The difference between the total in this table (993) and the total number of clauses in the corpus (2,241) is due to the fact that the table only counts clauses that contain overt subjects and/or objects, and a verb. Since Armenian makes frequent use of zero anaphora for both subjects and objects, many clauses have no overt subject or \isi{object}. There are also some incomplete or syntactically anomalous clauses that have not been counted here.}\hspace*{-.5em}  & 100\% \\
\lspbottomrule
    \end{tabularx}
    \caption{The distribution of word order configurations in EANC ArmFilmNarr corpus }
    \label{Armenian:tab:1}
\end{table}

According to the corpus results, the most frequent \isi{word order} is SV, with around 42\% (415) of all occurrences, which could be accounted for by the abundant number of intransitive verb constructions typical of the narrative genre.

\subsection{Direct object}\label{Armenian:ss:3.2}
\largerpage
\subsubsection{General overview}\label{Armenian:ss:3.2.1}

The second most frequent \isi{word order} is \isi{OV}. The absence of the agent (A) can be accounted for by MEA's pro-drop character, which is particularly frequent in oral discourse. \isi{OV} \isi{word order} also indicates the prevalence of preverbal position for objects in general. Since almost all clausal objects in MEA are postverbal, it is more informative to limit ourselves to non-clausal objects. Therefore, we concentrate on non-clausal objects, and every mention of objects will be understood as referring to monotransitive, non-clausal objects, unless otherwise stated. The overall distribution of non-clausal objects is 78.7\% preverbal vs. 21.3\% postverbal (see \tabref{Armenian:tab:2}). The data do not show a significant difference between the behaviour of lexical NP and pronominal objects, with the former showing 21.9\% \isi{VO}, the latter 18.9\% \isi{VO} (see \tabref{Armenian:tab:2}). Armenian does not possess \isi{clitic} pronouns, so the pronouns in question are ``strong,'' independent pronouns, with the equivalent of ``weak'' unstressed pronouns being zero anaphora.

\begin{table}
    \begin{tabularx}{.8\textwidth}{XlYY}
\lsptoprule
DO type & Total & \isit{VO} & \% \isit{VO} \\
\midrule
Lexical NP & 374 & 82 & 21.9\% \\
Pronominal & 106 & 20 & 18.9\% \\
\midrule
Total & 480  & 102 & 21.3\% \\
\lspbottomrule
    \end{tabularx}
    \caption{The distribution of all overt monotransitive non-clausal DOs in EANC ArmFilmNarr corpus }
    \label{Armenian:tab:2}
\end{table}


The net preference for preverbal position (21.3\% \isi{VO}, i.e. 78.7\% \isi{OV}) shown in these data is contrary to the study by \citet{samvelian_persistence_2023}, but roughly consistent with the data in \citet{stilo_preverbal_2018}. Potential factors that could account for this difference are discussed above in Section \ref{Armenian:ss:2}: discourse mode (oral vs. written), genre (oral narratives vs. out-of-the-blue sentences), and \isi{register} (colloquial vs. formal). The current oral corpus of favorite film narratives potentially includes all sentence modalities which can be part of a structured narrative, and not simply declarative, out-of-the-blue sentences, as was the case for the data used in the study by \citet{samvelian_persistence_2023}. 

\begin{sloppypar}
Another important difference is \isi{register}, since \citet{samvelian_persistence_2023} include language with characteristics of formal \isi{register}, whereas the current corpus covers generally colloquial \isi{register}, although with some semi-formal elements (the context of the recordings was generally informal, though the speakers were aware that they were being recorded, which may have prompted them to use some elements of formal/literary EA). In \citet{samvelian_persistence_2023}, it has already been proposed that EA \isi{word order} variation could also be correlated with \isi{register} variation, as also discussed for Persian\il{Persian} in \citetv{chapters/7_RasekhMahandetal_Persian}; the significant morphological, phonological, and syntactic differences between colloquial and formal/literary EA are discussed in Section \ref{Armenian:ss:2}.\footnotemark[1]

Evidence that \isi{register} is indeed a significant factor in this context is provided by the fact that Speaker 3\footnote{Speaker 3 is the ``outlier'' with a significantly higher percentage of postverbal DOs than the other speakers (44\%, compared to an average of 17.4\% for all  other speakers, and 21.3\% in the corpus as a whole, see \tabref{Armenian:tab:3})} (see \tabref{Armenian:tab:3}) is the only speaker to use formal/literary forms of demonstratives. For example, her speech contains two examples of medial demonstrative \textit{ayd} as opposed to colloquial \textit{ed/et}) and of the 3rd person ``emphatic'' \isi{pronoun} nominative plural \textit{irenkʿ} as opposed to colloquial \textit{irankʿ} (gen.sg. \textit{ir} is also used once by Speaker 1 as well as by Speaker 3).
\end{sloppypar}

\begin{table}
    \begin{tabularx}{.8\textwidth}{lYYY}
\lsptoprule
Speaker & Total O & \isit{VO} & \% \isit{VO} \\
\midrule
1 & 23 & 5 & 21.7\% \\
2 & 79 & 13 & 16.5\% \\
3 & 45 & 20 & 44.4\% \\
4 & 51 & 12 & 23.5\% \\
5 & 34 & 6 & 17.6\% \\
6 & 43 & 7 & 16.3\% \\
7 & 63 & 8 & 12.7\% \\
\lspbottomrule
    \end{tabularx}
    \caption{The distribution of postverbal DOs according to speaker in EANC ArmFilmNarr corpus}
    \label{Armenian:tab:3}
\end{table}

\subsubsection{Impact factors}\label{Armenian:ss:3.2.2}

The following set of impact factors that could potentially be relevant for EA \isi{word order} variation have been included in our analysis:

\begin{enumerate}%[label=\alph*.]
    \item \isi{definiteness}
    \item \isi{givenness}
    \item \isi{animacy}
    \item \isi{topic} persistence 
    \item \isi{referential distance}
    \item \isi{weight} (heavy NP shift)
    \item lexical vs. pronominal Os
    \item crowding / null subject (pro-dropping) effects /overt A
    \item main vs. subordinate clauses
    \item simple vs. complex verb forms
\end{enumerate}


\subsubsubsection{%a)
Definiteness}

``Definiteness'' here refers to the presence of the definite article (which is enclitic on the noun). According to \citet{samvelian_persistence_2023}, EA \isi{word order} variation is directly correlated with \isi{definiteness}, with definite DOs being mainly postverbal, and indefinite/bare DOs preverbal.  We checked the distribution of preverbal and postverbal bare, indefinite and definite DOs in the current study, and the results confirm the correlation between the \isi{definiteness} of DOs and postverbal position (see \tabref{Armenian:tab:4}).

 \begin{table}
    \begin{tabularx}{.8\textwidth}{lYYY}
\lsptoprule
Object type & Total & \isit{VO} & \% \isi{VO} \\
\midrule
Definite NP & 232 & 76 & 33\% \\
Indefinite NP (with indefinite article) & 21 & 2 & 10\% \\
Bare NP (indefinite without article) & 103 & 2 & 2\% \\
\midrule
Total & 356\footnotemark & 275 & mean: 15\% \\
\lspbottomrule
    \end{tabularx}
    \caption{The distribution of preverbal and postverbal bare, indefinite (with article) and definite DOs in EANC ArmFilmNarr corpus}
    \label{Armenian:tab:4}
\end{table}\footnotetext{This total is smaller that of \tabref{Armenian:tab:4}, because it only includes lexical NPs, while \tabref{Armenian:tab:4} also includes pronominal objects.}

As \tabref{Armenian:tab:4} shows, the distribution of definite DOs is 67\% preverbal vs. 33\% postverbal, whereas that of bare DOs is 98\% preverbal vs. 2\% postverbal. The indefinite DOs have an intermediate position with 90\% preverbal vs. 10\% postverbal. Therefore, there is a certain hierarchy of postverbal \isi{word order} possibility depending on \isi{definiteness}, in which definite DOs are the most postverbal, followed by indefinite (with article) DOs, and bare DOs are the least postverbal of all. This hierarchy corresponds to the grammatical semantics of definite {\Rightarrow} specific {\Rightarrow} non-specific. The peculiarity of EA indefiniteness is that it is bipartite with indefinite (specific) and bare (non-specific) semantics, and the ``classical'' indefinite noun would tend to be bare, i.e., non-specific, in EA, rather than with an indefinite article, i.e., specific. As the indefinite article is mainly used for marking (specific) indefiniteness, its usage is fairly rare.
Hence, the results of the current study show general tendencies that are consistent with \citet{samvelian_persistence_2023} in that EA postverbal \isi{word order} is associated with \isi{definiteness} and that the more definite/specific the DO is, the more likely it is to be postverbal.


\subsubsubsection{%b)
Givenness} ``Given,'' as opposed to ``new,'' is used here to indicate referents that have been previously mentioned in the discourse. Given Os are considerably more likely to appear in post-predicate position than Os which represent new information. However, this effect is weaker than the effect of grammatical \isi{definiteness} (definite NP O = 32.9\% post-predicate, indefinite (including bare) NP O = 3.3\% post-predicate), although the net preference of post-predicate Os for \isi{givenness} is still considerable (over 80\% of post-predicate Os are given) (see \tabref{Armenian:tab:5}). Thus, it is likely that the \isi{givenness} effect is merely a reflection of the effect of \isi{definiteness} (note that in Armenian, certain categories of NPs may be grammatically definite even if not given, e.g. possessives, partitives, and some nominalized non-nominal constituents):

\begin{table}
    \begin{tabularx}{.8\textwidth}{lYYY}
\lsptoprule
 & Total & \isit{VO} & \% \isi{VO} \\
\midrule
O given & 289 & 86 & 29.8\% \\
O new & 190 & 17 & 8.9\% \\
\lspbottomrule
    \end{tabularx}
    \caption{The distribution of given and new DOs in EANC ArmFilmNarr corpus}
    \label{Armenian:tab:5}
\end{table}


\subsubsubsection{%c)
Animacy} We observe that animate Os are more likely to appear in post-predicate position than inanimate ones. Note, however, that animate referents are considerably more likely to be definite than inanimate ones (animate DO = 79.7\% definite, inanimate DO = 59.5\% definite), so in order to understand whether the effect of \isi{animacy} is significant, we need to investigate it in combination with \isi{definiteness}. The results are shown in \tabref{Armenian:tab:6}.

\begin{table}
    \begin{tabularx}{.8\textwidth}{lYYY}
\lsptoprule
 & Total & \isit{VO} & \% \isi{VO} \\
\midrule
Animate O definite NP & 63 & 27 & 42.8\% \\
Inanimate O definite NP & 170 & 51 & 30\% \\
Animate O indefinite NP & 16 & 0 & 0\% \\
Inanimate O indefinite NP & 113 & 4 & 3.5\% \\
\lspbottomrule
    \end{tabularx}
    \caption{The distribution of animate and inanimate DOs in EANC ArmFilmNarr corpus }
    \label{Armenian:tab:6}
\end{table}

We find that among definite Os, animate referents appear slightly more frequently in post-predicate position than inanimate ones, although a Chi-square 2x2 contingency table shows that the difference between definite animate and definite inanimate DOs is not statistically significant. This is similar to the results of \citet[481]{samvelian_persistence_2023}, who find that the main effect of \isi{animacy} is not significant, but “there is a marginally significant interaction … indicating that \isi{animacy} has an effect on the order of definite DOs”. The number of indefinite DOs in post-predicate position is, as we have seen, very small, both for animate and inanimate referents.

\subsubsubsection{%d)
Topic persistence} Based on the methodology of \citet{givon_topic_1983}, \isi{topic} persistence was measured as the number of occurrences of that referent in the following 10 clauses (for more details on the methodology and on MEA data see \citealt{hodgson_word_nodate}). The average \isi{topic} persistence of preverbal Os is equal to 1.6, whereas that of postverbal ones is 1.9. Therefore, postverbal Os have higher \isi{topic} persistence than preverbal Os. Taking into account the previous impact factors, it could be argued that prototypically postverbal Os are mostly definite, given, and animate, hence have higher \isi{topic} persistence. The average \isi{topic} persistence of postverbal Os is higher than that of preverbal definite Os, which show an average \isi{topic} persistence of 1.5 (the same as that of definite Os as a whole). Human Os have on average a considerably higher \isi{topic} persistence (3.7) than non-human Os (1), so the fact that postverbal position is possibly associated with \isi{animacy}/humanity could be a factor in the higher average \isi{topic} persistence of postverbal as opposed to preverbal Os. Indeed, if we look at objects that are both definite and human, there is no significant difference in average \isi{topic} persistence between preverbal and postverbal examples (preverbal: 3.08, postverbal: 3.13).

\begin{table}
    \begin{tabularx}{.8\textwidth}{lY}
\lsptoprule
 & Average \isi{topic} persistence \\
\midrule
Pre-predicate O & 1.6 \\
Post-predicate O & 1.9 \\
\lspbottomrule
    \end{tabularx}
    \caption{Average topic persistence of pre- and post-predicate DOs in EANC ArmFilmNarr corpus}
    \label{Armenian:tab:7}
\end{table}

\subsubsubsection{%e)
Referential distance} There is only a very small difference between the average \isi{referential distance} (distance in clauses to previous mention of the referent) of pre-predicate and post-predicate Os, with post-predicate Os showing a slightly smaller average \isi{referential distance} (2.3) than pre-predicate Os (2.5) (see \tabref{Armenian:tab:8}).

\begin{table}
    \begin{tabularx}{.8\textwidth}{lY}
\lsptoprule
 & Average \isi{referential distance} \\
\midrule
Pre-predicate O & 2.5 \\
Post-predicate O & 2.3 \\
\lspbottomrule
    \end{tabularx}
    \caption{Average referential distance of pre- and post-predicate DOs in EANC ArmFilmNarr corpus}
    \label{Armenian:tab:8}
\end{table}

\begin{sloppypar}
\subsubsubsection{%f)
Weight (heavy NP shift)} In order to check the correlation between the \isi{weight} of DOs and their \isi{word order} variation, the length of DOs in words was measured, more particularly for 1-word, 2-word and 3+ word Os (see \tabref{Armenian:tab:9}). The results showed a higher frequency of postverbal position for 2-word Os (35\% postverbal) vs. 1-word Os (15.4\% postverbal). The fact that non-specific indefinites, lacking the indefinite article, which, as we have seen, overwhelmingly appear in preverbal position, are often one-word phrases is a probable factor in the higher frequency of preverbal position for 1-word Os. When O is composed of more than three words, the frequency of postverbal position is actually lower than that for 2-word DOs (23\% postverbal), though still higher than that for 1-word DOs.
\end{sloppypar}

Subsequently, the measurements were refined in order to explain the distribution of 3+word Os as well as the difference in behaviour between 2-word and 3+word Os. To do this, supplementary factors of \isi{definiteness} and the pronominal character of Os were added to the analysis of 2-word and 3+ word Os (see \tabref{Armenian:tab:9}).

\begin{table}
\begin{tabularx}{\textwidth}{X@{}rrrrrrrrr}
\lsptoprule
 & \multicolumn{3}{c}{1-word O} & \multicolumn{3}{c}{2-word O} & \multicolumn{3}{c}{3+-word O} \\
 \midrule
&Total&\isi{VO}&\%\isi{VO}&Total&\isi{VO}&\%\isi{VO}&Total&\isi{VO}&\%\isi{VO} \\
\midrule
Bare&73&0&0\%&22&2&9.1\%&5&0&0\% \\
Indefinite&0&0&0\%&5&1&20\%&14&1&7.1\% \\
Definite&105&27&25.7\%&83&36&43.4\%&38&12&31.6\% \\
Pronominal&88&14&16.7\%&10&3&30\%&4&1&25\% \\
Total&266&41&15.4\%&120&42&35\%&61&14&23\% \\
\lspbottomrule
    \end{tabularx}
    \caption{The distribution of 1-word, 2-word, and 3+ word Os correlated with O type in EANC ArmFilmNarr corpus}
    \label{Armenian:tab:9}
\end{table}

According to the results, 2-word definite Os are 43.4\% postverbal (cf. 21.3\% postverbal for all Os\footnote{This figure excludes those Os that appear in both pre- and post-predicate position.} ) and 3+ word definite Os are 31.6\% postverbal. We then zoomed in on indefinite 3+word Os, distinguishing those with the simple indefinite article \textit{mi} `a' and the indefinite expression composed of the indefinite article \textit{mi} `a' and the quantifier \textit{hat} `unit' which is often in reality the equivalent of the indefinite article in colloquial \isi{register} (see \tabref{Armenian:tab:11}). 21\% of all 3+ word Os include the simple indefinite article \textit{mi} `a,' whereas 79\% include the indefinite article with a quantifier (see \tabref{Armenian:tab:10}). The indefinite article in EA being principally unstressed, the expression of the indefinite article with a quantifier \textit{mi hat} `a unit' could be considered as one \isi{prosodic word}, hence it could be included in our measurements as one word rather than two due to its ``real \isi{weight}.'' However, even when \textit{mi hat} is counted as one word rather than two, 2-word objects still show a higher percentage of \isi{OV} order (36.2\%) than 3+-word objects (26.9\%), something which is unexpected from the point of view of heavy NP shift. The generally inconclusive evidence for an effect of \isi{weight} are consistent with the findings from other corpora of spontaneous spoken language investigated in this volume, which report only marginal effects of \isi{weight} on \isi{object} placement (see \citetv{chapters/1_Haigetal_Intro}, \sectref{Intro:ss:5}).

\begin{table}
\begin{tabularx}{.8\textwidth}{lYYY}
\lsptoprule
3+-word indefinite O & Total & \isit{VO} & \% \isi{VO} \\
\midrule
\textit{mi}&3&1&33.3\% \\
\textit{mi hat}&11&0&0\% \\
\lspbottomrule
    \end{tabularx}
    \caption{The distribution of 3+ word indefinite Os in EANC ArmFilmNarr corpus}
    \label{Armenian:tab:10}
\end{table}


\subsubsubsection{%g)
Lexical vs. pronominal O} The linguistic character of DOs showed little impact on the \isi{word order} distribution, with lexical DOs being 21.9\% postverbal, and pronominal DOs being 18.9\% postverbal (see \tabref{Armenian:tab:11}). Overall, preverbal DOs are largely dominant in both cases and the distribution proportion in line with that of all DOs in the corpus (21.3\% postverbal, \tabref{Armenian:tab:2}).

\begin{table}
    \begin{tabularx}{\textwidth}{lYYY}
\lsptoprule
Type of O & Total & \isit{VO} & \% \isi{VO} \\
\midrule
Lexical NP&374&82&21.9\% \\
Pronoun&106&20&18.9\% \\
\lspbottomrule
    \end{tabularx}
    \caption{The distribution of preverbal and postverbal lexical and pronominal DOs in EANC ArmFilmNarr corpus}
    \label{Armenian:tab:11}
\end{table}

\subsubsubsection{%h)
Crowding / Null Subject Effect} One of the hypotheses concerning the impact factors was that \isi{argument} crowding or its opposite, null subject, could affect \isi{word order} variation so that the presence of an overt subject could induce postverbal DOs to avoid crowding, or the presence of a preverbal DOs could be correlated to null subject effect.

\begin{table}
    \begin{tabularx}{\textwidth}{lrYY}
\lsptoprule
& Total overt DO & \isit{VO} & \% \isi{VO} \\
\midrule
Overt S&148&35&23.6\% \\
No overt S&339&69&20.4\% \\
No overt S or other \isi{role}\footnote{``Other \isi{role}'' refers to those elements discussed in section \ref{Armenian:ss:3.3}.} &267&51&19.1\% \\
Overt preverbal S or other \isi{role}&175&39&22.3\% \\
\lspbottomrule
    \end{tabularx}
    \caption{The distribution of DOs with and without other overt arguments in EANC ArmFilmNarr corpus}
    \label{Armenian:tab:12}
\end{table}

To check this, we first observed the distribution of overt/no overt subjects (S) which, however, manifested insignificant impact on the position of DO (see \tabref{Armenian:tab:12}). Overall, the distribution is almost identical, therefore, no correlation is observed between the crowding/null subject effect and the \isi{word order} variation. The same is true if we take into account the presence of elements with other roles as well as subject; the presence of a preverbal \isi{argument} does not seem to be associated with any increase in postverbal position for DO.

\subsubsubsection{%i)
Main vs. subordinate clauses} In some languages, \isi{object} position differs depending on whether the \isi{object} is in a main clause or a subordinate clause. Our MEA data do not show any significant difference in the position of objects between main clauses and subordinate clauses:

\begin{table}
    \begin{tabularx}{.8\textwidth}{lYYY}
\lsptoprule
Clause type & Total O & \isit{VO} & \% \isi{VO} \\
\midrule
Main clause&391&80&20.5\% \\
Subordinate clause&87&20&22.9\% \\
\lspbottomrule
    \end{tabularx}
    \caption{The distribution of DOs in main and subordinate clauses in EANC ArmFilmNarr corpus}
    \label{Armenian:tab:13}
\end{table}

\subsubsubsection{%j)
Simple vs. complex verb forms} In some languages, different types of verb forms are associated with different positions of arguments. For example, some verb forms of nominal origin may show \isi{argument} positions analogous to those of noun modifiers, which in modern Armenian almost invariably precede the element they modify. MEA has complex (periphrastic) verb forms involving participles, which have some nominal characteristics, as well as simple verb forms inherited from Classical Armenian\il{Armenian (Classical)}.

Thus, we might expect a stronger preference for pre-predicate arguments with periphrastic verb forms. However, direct objects show no evidence for such a pattern, with little difference between different types of verb forms (see \tabref{Armenian:tab:14}). In fact, periphrastic verb forms show a slightly higher percentage of post-predicate Os than simple verb forms (21.3\% vs. 18.3\%).

\begin{table}
    \begin{tabularx}{.8\textwidth}{lYYY}
\lsptoprule
Verb type & Total O\footnotemark & \isit{VO} & \% \isi{VO} \\
\midrule
Complex verb&314&67&21.3\% \\
Simple verb&115&21&18.3\% \\
\lspbottomrule
    \end{tabularx}
    \caption{The distribution of DOs with complex and simple verbs in EANC ArmFilmNarr corpus}
    \label{Armenian:tab:14}
\end{table}\footnotetext{Here, only clauses with a single participle + \isi{auxiliary} or a single simple verb are counted.}

\subsection{Other post-predicate arguments}\label{Armenian:ss:3.3}

\subsubsection{General overview}\label{Armenian:ss:3.3.1}

In addition to objects, other types of arguments also show variable position in Armenian and neighbouring languages of Western Asia. As discussed in \citet{haig_introduction_2018}, previous studies find that across the area, there is a tendency for \isi{Goal} arguments of verbs of motion and caused motion to appear in post-predicate position. In some languages, this tendency is extended to other elements that could be considered to share ``\isi{endpoint}'' semantics, namely \isi{recipient}, \isi{benefactive}, \isi{endpoint} of change of state verbs such as `become,' and \isi{addressee}; see \textcitetv[\sectref{Intro:ss:4}]{chapters/1_Haigetal_Intro} for an updated overview.

\citet{stilo_preverbal_2018} finds that in his Armenian data, there is indeed a tendency for goals to appear in post-predicate position. He finds a weaker tendency for \isi{benefactive} and \isi{recipient}, but no such tendency for \isi{addressee}, which strongly prefers pre-predicate position. He finds that instrumental, \isi{ablative}, \isi{locative}, and \isi{comitative} arguments show a preference for pre-predicate position across the area, being generally unaffected by the tendency for post-predicate position associated with goals and goal-like elements. We investigate all these types of arguments, and find that goals do indeed show a preference for post-predicate position (approximately 70\% post-predicate). This tendency also seems to apply to \isi{benefactive}, though the numbers involved are small. The numbers for \isi{recipient} of `give,' and of verbs with similar meanings, are also small, but unlike \citegen{stilo_preverbal_2018} data, show few examples in post-predicate position. Like \citet{stilo_preverbal_2018}, we find that \isi{addressee} arguments, which are rarely expressed overtly, show no tendency to appear in post-predicate position in our Armenian data. In \isi{contrast}, the name in sentences such as `they call him/her/it X,' which are also infrequent, shows a marked preference for post-predicate position (see \tabref{Armenian:tab:15}).

\begin{table}
\begin{tabularx}{\textwidth}{Qrrr}
\lsptoprule
Other roles & Total & Post-predicate & \% Post-predicate \\
\midrule
Name (`they call him/her/it X')    &5&4&80\% \\
Benefactive &11&8&72.7\% \\
Goal&122&84&68.9\% \\
Ablative                                    &50&20&40\% \\
Comitative                                 &44&13&29.5\% \\
Location &116&30&25.8\% \\
Become                                      &9&2&22.2\% \\
Instrumental&71&15&21.1\% \\         
Recipient (\isi{transfer} of possession in general)  &11&2           &18\% \\
Recipient (of verb \textit{tal} `give')&6&1 &16.7\% \\
Addressee                                   &6&0          &0\% \\
\midrule
Total other roles\footnote{This figure represents the total of 11 roles, minus Recipient (of verb \textit{tal} `give'), as this is already included in Recipient (\isi{transfer} of possession in general).}                         &445&178        &40\% \\
\lspbottomrule
\end{tabularx}
\caption{Frequency of post-predicate placement, other roles}
\label{Armenian:tab:15}
\end{table}

If we consider goals separately from the other roles in \tabref{Armenian:tab:15}, we see that they have a much higher instance of post-predicate position (for the other categories with very high post-predicate figures, i.e. name and \isi{benefactive}, the numbers are too small to draw firm conclusions):

\ea
Goal: 68.9\% post-predicate\\
Other roles: 29.1\% post-predicate
\z
\begin{sloppypar}
The relatively high percentage of post-predicate position (40\%) for \isi{ablative} is unexpected in the light of proposals that interpret the tendency for post-predicate position as an iconic expression of ``\isi{endpoint}'' semantics. It is possible that a different type of analogy is at work here, with \isi{ablative} equated with \isi{Goal} as both are typical arguments of verbs of motion (\isi{ablative} as starting point, \isi{Goal} as \isi{endpoint}). However, it is more likely that the relatively high post-predicate percentage of \isi{ablative} in these texts simply reflects the fact that it is more likely than average to be given (76\% vs. 65\% total other roles) and animate (32\% vs. 27\% total other roles), factors which seem to favour post-predicate position (grammatical \isi{definiteness} is not relevant here, as the \isi{ablative} case ending in MEA cannot co-occur with the definite article).
\end{sloppypar}

\subsubsection{Impact factors}\label{Armenian:ss:3.3.2}

\subsubsubsection{%a)
Definiteness} As in the case of DOs, ``\isi{definiteness}'' is used here to indicate the presence of the definite article. Note that many of the roles investigated here (those that take \isi{ablative}, instrumental, \isi{locative}, or genitive case, and most nominative goals and locations) grammatically exclude the definite article. The fact that goals, which show a marked preference for postverbal position, are often grammatically indefinite for reasons of morphology rather than semantics gives a large number of postverbal indefinites, masking the effect of \isi{definiteness} in itself. Therefore, goals are shown separately from other roles in \tabref{Armenian:tab:16} below.  However, even when goals are excluded, we still find that definites are more likely than indefinites to appear in post-predicate position (38.1\% vs. 26.9\%). Indeed, goals themselves also show a higher percentage of post-predicate position when grammatically definite (75\% vs. 64.9\% for indefinite). Nonetheless, the effect of \isi{definiteness} on roles other than DO is not statistically significant.

\begin{table}
    \begin{tabularx}{\textwidth}{lYYr}
\lsptoprule
& Total & Post-predicate & \% Post-predicate \\
\midrule
Indefinite NP \isi{Goal}&57&37&64.9\% \\
Definite NP \isi{Goal}&52&39&75\% \\
Indefinite NP other&119&32&26.9\% \\
Definite NP other&105&40&38.1\% \\
\lspbottomrule
    \end{tabularx}
    \caption{The distribution of definite and indefinite NP other roles in EANC ArmFilmNarr corpus}
    \label{Armenian:tab:16}
\end{table}

\subsubsubsection{%b)
Givenness} In the light of the fact that many other roles cannot take the definite article for grammatical reasons, it might be expected that \isi{givenness} could show a stronger correlation than \isi{definiteness} with post-predicate position. However, for roles other than \isi{Goal}, we find no apparent effect of \isi{givenness} at all. Note that the correlation between \isi{givenness} and post-predicate position is much weaker for other roles than for \isi{direct object}, with new other roles showing 30.1\% post-predicate position, compared to 8.9\% for new direct objects (see \tabref{Armenian:tab:17}).

\begin{table}
    \begin{tabularx}{\textwidth}{lYYr}
\lsptoprule
& Total & Post-predicate & \% Post-predicate \\
\midrule
Given \isi{Goal}&77&56&72.7\% \\
New \isi{Goal}&39&24&61.5\% \\
Given other&221&64&29.0\% \\
New other&103&31&30.1\% \\
\lspbottomrule
    \end{tabularx}
    \caption{The distribution of given and new other roles in EANC ArmFilmNarr corpus}
    \label{Armenian:tab:17}
\end{table}

\begin{sloppypar}
\subsubsubsection{%c)
Animacy} As with direct objects, we see that animate referents appear more frequently in post-predicate position than inanimate ones, although, once again, the effect does not reach statistical significance. However, the difference is smaller here than that found with direct objects, where animates are 28\% post-predicate and inanimates 18\%.
A possible reason for this is that many of the inanimate other \isi{role} referents are goals, which differ from all other roles in showing a preference for post-predicate position (goals are almost exclusively inanimate). Thus, in \tabref{Armenian:tab:18}, we present the data for other roles excluding goals. It can be seen that if we discount goals, the effect of \isi{animacy} becomes more apparent.
\end{sloppypar}

\begin{table}
    \begin{tabularx}{\textwidth}{lYrr}
\lsptoprule
& Total & Post-predicate & \% Post-predicate \\
\midrule
Total other roles animate&122&45&36.9\% \\
Total other roles inanimate&331&135&40.7\% \\
Other roles animate -\isi{Goal}&113&38&33.6\% \\
Other roles inanimate -\isi{Goal}&218&58&26.6\% \\
\lspbottomrule
    \end{tabularx}
    \caption{The distribution of animate and inanimate other roles in EANC ArmFilmNarr corpus }
    \label{Armenian:tab:18}
\end{table}

As we have seen, animate referents are more likely to be definite than inanimate ones, so in order to accurately gauge the effect of \isi{animacy}, \isi{definiteness} must also be taken into account, as shown in \tabref{Armenian:tab:19}. As with DOs, we find that for definite NPs animate referents do show a higher proportion of post-predicate position than inanimates, although again, this does not reach statistical significance, while for indefinite NPs, which show lower frequency of post-predicate position, there does not seem to be any effect of \isi{animacy}. For the reasons discussed above, these figures do not include goals.

\begin{table}
    \begin{tabularx}{\textwidth}{rrrrrr Yrrrrr}
\lsptoprule
\multicolumn{6}{c}{Definite NP} &  \multicolumn{6}{c}{Indefinite NP} \\
\cmidrule(lr){1-6}\cmidrule(lr){7-12}
 \multicolumn{3}{c}{Animate} & \multicolumn{3}{c}{Inanimate} & \multicolumn{3}{c}{Animate} & \multicolumn{3}{c}{Inanimate} \\
\cmidrule(lr){1-3}\cmidrule(lr){4-6}\cmidrule(lr){7-9}\cmidrule(lr){10-12}
N&VX&\%VX&N&VX&\%VX&N&VX&\%VX&N&VX&\%VX \\

\cmidrule(lr){1-6}\cmidrule(lr){7-12}
41&19&46.3\%&64&21&32.8\%&25&7&28\%&94&25&26.6\% \\

\cmidrule(lr){1-6}\cmidrule(lr){7-12}
\multicolumn{6}{c}{Total definite NP: 105} & \multicolumn{6}{c}{Total indefinite NP: 119} \\
\multicolumn{6}{c}{Total definite VX: 40} & \multicolumn{6}{c}{Total indefinite VX: 32} \\
\multicolumn{6}{c}{\% definite VX: 38.1\%} & \multicolumn{6}{c}{\% indefinite VX: 26.9\%} \\
\lspbottomrule
    \end{tabularx}
    \caption{The distribution of animate and inanimate other roles in EANC ArmFilmNarr corpus divided by definiteness }
    \label{Armenian:tab:19}
\end{table}

\begin{sloppypar}
\subsubsubsection{%d)
Weight} As in the case of DOs, we find that other roles with \isi{weight} 1 (composed of one word) show a lower frequency of post-predicate position than those which are longer, indicating that \isi{weight} may be a factor promoting post-predicate position. This is the case even if we exclude pronouns, which are typically composed of one word and show a stronger preference for pre-predicate position than lexical NPs (see following Section e) ``Lexical vs. pronominal''). Very heavy elements (\isi{weight} 4+) show the highest percentage of post-predicate position, indicating the possibility of heavy NP shift. The effect of \isi{weight} is not particularly strong, but other roles present somewhat stronger evidence than DOs for its relevance as a factor favouring post-predicate position (see \tabref{Armenian:tab:20}).
\end{sloppypar}

\begin{table}
    \begin{tabularx}{\textwidth}{lYYY}
\lsptoprule
Weight & Total & Post-predicate & \% Post-predicate \\
\midrule
1 & 239 & 87 & 36.4\% \\
1 (NP only) & 19 & 7 & 36.8\% \\
2 & 139 & 59 & 42.4\% \\
3 & 51 & 21 & 41.4\% \\
3+ & 74 & 32 & 43.2\% \\
4+ & 23 & 11 & 47.8\%
\\
\lspbottomrule
    \end{tabularx}
    \caption{The distribution of other roles according to weight in EANC ArmFilmNarr corpus }
    \label{Armenian:tab:20}
\end{table}

\begin{sloppypar}
\subsubsubsection{%e)
Lexical vs. pronominal} A comparison of lexical NPs and pronouns shows that the former appear more frequently in post-predicate position in the EANC ArmFilmNarr corpus.
However, note that certain types of pronouns, such as interrogative and relative pronouns, show particular syntactic behaviour that places them in pre-predicate position, so it will be more informative to investigate those types of pronouns which can appear in either pre- or post-predicate position. For this reason, we look at demonstrative pronoun\is{pronoun!demonstrative}s, personal pronoun\is{pronoun!personal}s, and the ``emphatic'' \isi{pronoun} \textit{inkʿə} (for a discussion of this element and its behaviour in this corpus, see \citealt{hodgson_word_nodate}; see also \citealt{donabedian-demopoulos_recherche_2007}). Since all these elements are inherently definite, it is also informative to compare them with definite NPs. We still find that lexical NPs show a clearly higher percentage of post-predicate occurrences than any of these pronouns, and if we compare these pronouns with definite NPs, the difference is greater still. It is interesting that other roles seem to show a clearer difference between the behaviour of lexical and pronominal elements than DOs, which show only a small difference in the frequency of post-predicate position (see \tabref{Armenian:tab:21}).
\end{sloppypar}

\begin{table}
    \begin{tabularx}{\textwidth}{lYYr}
\lsptoprule
Type of element & Total & Post-predicate & \% Post-predicate \\
\midrule
Total pronouns & 116 & 31 & 26.7\% \\
Total lexical NP & 337 & 149 & 44.2\% \\
Demonstratives & 12 & 4 & 33.3\% \\
``Emphatic'' \isi{pronoun} & 10 & 2 & 20\% \\
Personal pronouns & 17 & 6 & 35.3\% \\
Definite NP & 157 & 79 & 50.3\% \\
\lspbottomrule
    \end{tabularx}
    \caption{The distribution of lexical and pronominal other roles in EANC ArmFilmNarr corpus }
    \label{Armenian:tab:21}
\end{table}

\subsubsubsection{%f)
Crowding effect} As discussed in the corresponding section on DOs, it has been proposed that the presence of another pre-predicate \isi{argument} could favour post-predicate position, in order to avoid ``crowding'' of more than one \isi{argument} on the same side of the predicate. However, our data do not provide evidence for this, as in fact the percentage of post-predicate arguments is higher (51.3\%) when there is no other overt \isi{argument} than when there is an overt pre-predicate subject and/or \isi{object} (36.8\%). When both subject and \isi{object} are pre-predicate, other arguments are less frequent still in post-predicate position (33.3\%). In the presence of a post-predicate subject, other roles appear more infrequently in post-predicate position (20\%), implying that there may be a tendency to avoid more than one post-predicate \isi{argument}. However, when there is a postverbal \isi{object}, the figures are very close to the average for other roles as a whole (37.9\%, as compared to ≈ 40\% for other roles in general). Thus, these data do not provide conclusive evidence for any type of \isi{crowding effect} (see \tabref{Armenian:tab:22}).

\begin{table}
    \begin{tabularx}{\textwidth}{lYrr}
\lsptoprule
& Total & Post-predicate & \% Post-predicate \\
\midrule
No overt S, A, or O & 146 & 75 & 51.3\% \\
Overt preverbal S & 149 & 51 & 34.2\% \\
Overt preverbal A & 7 & 3 & 42.9\% \\
Overt preverbal O & 46 & 21 & 45.6\% \\
Overt preverbal A and O & 21 & 7 & 33.3\% \\
Total overt preverbal only & 223 & 82 & 36.8\% \\
Overt preverbal A, postverbal O & 8 & 2 & 25\% \\
Overt preverbal O, postverbal A & 0 & 0 & 0\% \\
Overt postverbal S & 25 & 5 & 20\% \\
Overt postverbal A & 0 & 0 & 0\% \\
Overt postverbal O & 29 & 11 & 37.9\% \\
Overt postverbal A and O & 0 & 0 & 0\% \\
Total overt postverbal only & 54 & 16 & 29.6\% \\
\lspbottomrule
    \end{tabularx}
    \caption{The distribution of other roles according to the presence of other overt arguments in EANC ArmFilmNarr corpus }
    \label{Armenian:tab:22}
\end{table}

\subsubsubsection{%g)
Main vs. subordinate clause} In the section on direct objects, we saw that in these data, there is no significant difference between the position of direct objects in main vs. subordinate clauses. However, other roles, and especially goals, show a higher frequency of post-predicate position in main clauses as compared to subordinate clauses. Subordinate clauses have been observed to show more conservative \isi{word order} patterns than main clauses, for example the persistence of \isi{OV} in subordinate clauses in Germanic languages such as German\il{German}. Thus a possible explanation for these findings is that the tendency for post-predicate Goals\is{Goal!post-verbal} is a relatively recent phenomenon that has spread by contact from other languages of the area, such as Iranian languages and Neo-Aramaic\il{Neo-Aramaic}, and has not spread fully to subordinate clauses. In this context, it is interesting that direct objects do not show such a difference, suggesting that their variable position could indeed be a conservative feature inherited from Classical Armenian\il{Armenian (Classical)}, rather than a more recent contact-induced phenomenon. However, note that the percentage of postverbal goals (48\%, see \tabref{Armenian:tab:23}) is still much higher than that of postverbal DOs (22.9\%, see \tabref{Armenian:tab:13}) in subordinate clauses, so we can say that the tendency for post-predicate Goals\is{Goal!post-verbal} is still present. In addition, we only have a small number of subordinate clauses with \isi{Goal} arguments, so it is not possible to draw firm conclusions on this issue.
\largerpage
\begin{table}
    \begin{tabularx}{\textwidth}{lYYr}
\lsptoprule
& Total & Post-predicate & \% Post-predicate \\
\midrule
MC total other roles & 341 & 144 & 42.2\% \\
SC total other roles & 84 & 27 & 32.1\% \\
MC goals only & 94 & 70 & 74.5\% \\
SC goals only & 25 & 12 & 48\% \\
\lspbottomrule
    \end{tabularx}
    \caption{The distribution of other roles in main and subordinate clauses in EANC ArmFilmNarr corpus }
    \label{Armenian:tab:23}
\end{table}

\subsubsubsection{%h)
Simple vs. complex verb form} We observe that complex (participle + \isi{auxiliary}) verb forms show a lower percentage of post-verbal arguments, including goals, than simple verb forms. A possible explanation for this phenomenon is that arguments of participial (nominalized) verb forms tend to show positional characteristics of noun modifiers, i.e. preceding the element they modify. It is also possible that the phenomenon is linked to properties of \isi{focus} marking in Eastern Armenian\il{Armenian (Eastern)}, as the \isi{auxiliary} in complex verb forms can mark \isi{focus} when the focused element precedes the lexical verb, but not when it follows. However, neither of these explanations is particularly convincing given the fact that direct objects do not seem to show this pattern (as seen in Section \ref{Armenian:ss:3.2.2}, in this corpus, complex verb forms in fact show a slightly higher percentage of post-predicate DOs than simple verb forms). Further research is clearly needed to clarify the interaction between verb type, \isi{word order}, and \isi{information structure} in Armenian.

\begin{table}
    \begin{tabularx}{\textwidth}{lYrr}
\lsptoprule
& Total & Post-predicate & \% Post-predicate \\
\midrule
Complex verb total other roles & 288 & 110 & 38.2\% \\
Complex verb goals only & 77 & 53 & 68.8\% \\
Simple verb total other roles & 53 & 27 & 50.9\% \\
Simple verb goals only & 18 & 14 & 77.8\% \\
\lspbottomrule
    \end{tabularx}
    \caption{The distribution of other roles with complex and simple verbs in EANC ArmFilmNarr corpus }
    \label{Armenian:tab:24}
\end{table}

\subsubsection{Summary of other roles}\label{Armenian:ss:3.3.3}

The data from our MEA corpus confirms that in MEA, as in other languages of the wider area, goals of verbs of motion and caused motion show a preference for post-predicate position (approximately 70\%). However, in \isi{contrast} to some languages of Western Asia (see \citetv{chapters/1_Haigetal_Intro}), there is no such tendency observed for other constituents sharing the semantics of ``\isi{endpoint},'' such as \isi{recipient} or \isi{addressee}. An apparent exception is \isi{benefactive}, which shows an even stronger preference for post-predicate position (72.7\%), although the small number of examples makes this less reliable. It is also worth noting that the number of overt \isi{recipient} and \isi{addressee} referents in this corpus is very low. Apart from \isi{Goal}, virtually all the other roles investigated which have more than 20 examples (\isi{comitative}, location, instrumental) show similar figures, of 20--30\% post-predicate position, similar to those for direct objects overall. Ablative has a slightly higher figure (40\% post-predicate), but this may be because the \isi{ablative} referents in this corpus show higher than average figures for \isi{givenness} and \isi{animacy}, which may have some effect favouring post-predicate position.

Definiteness, too, is shown to favour post-predicate position, but indefinite other roles do not show the extreme preference for pre-predicate position that is characteristic of indefinite direct objects. Even bare indefinite other roles show 38.7\% post-predicate position, while indefinite other roles in general (including those with the indefinite article or an indefinite proform) show 39.2\%, virtually identical to the average of other roles as a whole (40\%). Weight appears to have an effect on position, with heavier elements appearing somewhat more frequently in post-predicate position. Other roles show more evidence of an effect of \isi{weight} than direct objects, although this is still not particularly strong. Other roles also show more clearly than direct objects the differences in behaviour between pronouns and lexical NPs, with the latter being more likely to appear in post-predicate position (44.2\%, vs. 26.7\% for pronouns). As with direct objects, there is no evidence for a \isi{crowding effect}, whereby the presence of other pre-predicate arguments could promote post-predicate position in order to avoid ``crowding'' of arguments on one side of the predicate. Unlike direct objects, other roles present possible evidence that post-predicate constituents may be more common in main than subordinate clauses, and with simple rather than complex verb forms. A possible explanation for the former could be that the post-predicate \isi{Goal} phenomenon is a recent contact-induced development that has spread more slowly to subordinate clauses, although the numbers are too small to draw any firm conclusions. The link between \isi{argument} position and verb form is a \isi{topic} for further research.

\section{Conclusions}\label{Armenian:ss:4}

As regards direct objects, \isi{definiteness} proved to be a key impact factor for the postverbal position: 33\% (def) vs. 10\% (indef) vs. 2\% (bare). This is consistent with the previous study by \citet{samvelian_persistence_2023}, with the difference that in the present study, the percentage of post-predicate definite Os is considerably lower than those reported by \citet{samvelian_persistence_2023}, who find 82.7\% of definite Os in post-predicate position in their first experiment. The percentage of post-predicate definite DOs in the present study is intermediate between the very high figures found by \citet{samvelian_persistence_2023}, and the very low figures (around 14\% for colloquial Yerevan\il{Armenian (Eastern)!Yerevan}) reported by \citet{stilo_preverbal_2018}. One probable factor behind the difference is that the data in the experimental studies of \citet{samvelian_persistence_2023} include only out-of-the-blue sentences; note that a similar pattern has been observed in Romeyka\il{Hellenic!Romeyka}, where those studies based on elicitation of out-of-the-blue sentences yield predominantly \isi{VO} structures, while the data from connected spontaneous spoken discourse show a much higher rate of \isi{OV} (\citetv{chapters/12_Schreiber_Romeyka}). As \citet{samvelian_persistence_2023} note in their conclusion, it is very likely that the rate of SOV is higher in spontaneous oral discourse.

Another relevant factor, also proposed by \citet{samvelian_persistence_2023}, is \isi{register}. \citet{samvelian_persistence_2023} includes data with characteristics of formal literary language, which show significant morphological, phonological, and syntactic differences from colloquial Yerevan\il{Armenian (Eastern)!Yerevan} EA, such that the two should be considered different forms of language.  The association of post-predicate Os with formal registers is supported by the fact that the speaker who uses by far the highest percentage of postverbal Os in the present study (44\%, as opposed to an average of 17.4\% for all the other speakers, and 12.7\% for the speaker with the lowest percentage) is also the only one who uses certain word forms associated with the formal literary language. Thus, if we discount this one speaker, who uses a more formal \isi{register}, the percentage of postverbal Os in this study is not so different from that recorded by \citet{stilo_preverbal_2018} for colloquial Yerevan\il{Armenian (Eastern)!Yerevan}. In addition, a similar figure (approx. 90\% \isi{OV}) is obtained for nominal direct objects in the Agulis corpus of spoken vernacular Armenian \citep{hodgson_armenian_nodate}.

\begin{sloppypar}
The effect of grammatical \isi{definiteness} is stronger than that of the pragmatic property of \isi{givenness} (29.8\% of given Os appear in post-predicate position, vs. 8.9\% of new information Os). Among definite Os, animate referents show a slightly higher percentage of post-predicate position (42.8\% for animates vs. 30\% for inanimates), although this does not reach statistical significance. This, too, is consistent with the findings of \citet{samvelian_persistence_2023}. Given that \isi{definiteness} and \isi{animacy} are characteristics associated with topical referents, it is unsurprising that \isi{topic} persistence is higher for postverbal Os (1.9 (postverbal) vs. 1.6 (preverbal)). Post-predicate Os are also associated with slightly lower average \isi{referential distance} than pre-predicate ones (2.3 vs. 2.5), which is also to be expected given the association of post-predicate position with \isi{topicality} in general. Some effect of heavy NP shift effect was observed, with longer NPs being more frequent in post-predicate position. (13\% (1-word Os) vs. 31\% (2-word Os) vs. 19\% (3+word Os)). This is also broadly consistent with the findings of \citet{samvelian_persistence_2023}. Neither the \isi{object} type (lexical or pronominal) nor the crowding / null subject effect had any evident impact on the position of DOs. Overall, the present corpus study of oral narratives showed that \isi{OV} \isi{word order} is more frequent in MEA than \isi{VO} (79\% preverbal vs. 21\% postverbal). A prototypical postverbal O in this corpus is definite, given (91\%), human (41\% vs. 27\% of preverbal objects), with higher \isi{topic} persistence.
\end{sloppypar}

As regards other post-predicate constituents, the same factors that have been found to be associated with post-predicate position for DOs  (\isi{definiteness}, \isi{animacy}, and \isi{weight}) seem to have a slight effect in the case of other roles too, although the numbers involved cannot be considered statistically significant. The effect of \isi{definiteness} is less pronounced than for DOs, with 50.3\% of definite NP other roles appearing in post-predicate position, vs. 39.2\% of indefinites. As we can see, indefinite other roles do not show such a strong tendency to avoid post-predicate position as indefinite DOs. As with DOs, animate definite other roles (excluding goals) are more likely to appear in post-predicate position than inanimate definite ones (46.3\% vs. 32.8\%). As with DOs, \isi{animacy} does not seem to affect the position of indefinites. Other roles present slightly stronger evidence of heavy NP shift to post-predicate position than DOs, with those comprised of 4+ words showing the highest percentage of post-predicate position (\isi{weight} 1 = 36.4\% post-predicate, \isi{weight} 2 = 42.4\%, \isi{weight} 3 = 41.4\%, \isi{weight} 4+ = 47.8\%), although the observed effect is still fairly weak. Other roles present evidence that pronominal arguments are less likely to appear in post-predicate position than lexical NPs (26.7\% post-predicate, vs. 44.2\% for lexical NPs), while for DOs there is no apparent difference. As for direct objects, no evidence is found of a \isi{crowding effect}.

The preference for goals to appear in post-predicate position is a separate issue, that has been shown to have an areal dimension. This preference is confirmed by the data in this study, where 68.9\% of goals appear in post-predicate position. This study presents no evidence that the preference for post-predicate position is extended to other constituents with ``\isi{endpoint}'' semantics, such as recipients or addressees, with the possible exception of \isi{benefactive}. However, the numbers of overt examples of all these types of \isi{argument} (\isi{recipient}, \isi{addressee}, \isi{benefactive}) are very small, so we cannot draw a firm conclusion here. The fact that the tendency for goals to appear in post-predicate position is more pronounced in main than subordinate clauses is possible evidence that it is a relatively recent, contact-induced phenomenon. The position of direct objects shows no significant difference between main and subordinate clauses, and it is possible that the existence of postverbal objects, and perhaps other arguments, too, is a conservative characteristic inherited from Classical Armenian\il{Armenian (Classical)} (see \citealt{samvelian_persistence_2023}, \citealt{stilo_preverbal_2018}), which may perhaps also explain its apparent association with formal \isi{register}. This is a \isi{topic} for future research. In any case, Modern Eastern Armenian\il{Armenian (Eastern)} can be said to fit the typological profile of an ``OVX'' language, in that despite showing mainly \isi{head-final} characteristics, it also has some characteristics associated with typically \isi{head-initial} languages, such as initial complementizers.

\section*{Abbreviations}
\begin{tabularx}{.55\textwidth}{@{}lQ@{}}
1 & first person \\
2 & second person \\
3 & third person \\
\textsc{agr3} & third-person agreement marker \\
\textsc{aor} & aorist \\
\textsc{caus} & causative \\
\textsc{dat} & dative \\
\textsc{dem2} & medial demonstrative \\
\textsc{def} & definite \\
\textsc{emp} & emphatic \\
\textsc{fpt} & future participle \\
\end{tabularx}%
\begin{tabularx}{.4\textwidth}{@{}lQ@{}}
\textsc{fut} & future \\
\textsc{gen} & genitive \\
\textsc{ia} & indefinite article \\
\textsc{ipfv} & imperfective \\
\textsc{neg} & negative \\
\textsc{nom} & nominative \\
\textsc{pfv} & perfective \\
\textsc{prs} & present \\
\textsc{pl} & plural \\
\textsc{sg} & singular \\
\textsc{sub} & subjunctive \\
\end{tabularx}




\sloppy
\printbibliography[heading=subbibliography,notkeyword=this]

\end{document}
