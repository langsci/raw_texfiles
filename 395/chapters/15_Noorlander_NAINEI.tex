\documentclass[output=paper,colorlinks,citecolor=brown,draftmode]{langscibook}
\ChapterDOI{10.5281/zenodo.14266359}
\author{Paul M. Noorlander\orcid{0000-0002-9407-1453}\affiliation{University of Cambridge}}
\title{Neo-Aramaic in Iran and northeastern Iraq}
\abstract{This chapter offers a brief overview of the word order typology of Neo-Aramaic dialects spoken by Jewish and Christian minorities of Iran and northeastern Iraq. A characteristic of the dialects in this region is the contact-induced shift from VO to OV word order under the influence of neighbouring Iranian and Turkic\il{Turkic} languages. In Iranian Azerbaijan, convergence with Azeri\il{Turkic!Azeri} has resulted in an additional increase in Adjective-Noun order, and a different treatment of Addressees from Goals. In many respects, however, the constituent order remains consistent with that of so-called VO languages, such as prepositional marking and Noun-Genitive order.}

%move the following commands to the "local..." files of the master project when integrating this chapter
% \usepackage{tabularx}
% \usepackage{langsci-optional}
% \usepackage{langsci-gb4e}
% \usepackage{enumitem}
% \bibliography{localbibliography}
% \newcommand{\orcid}[1]{}
% \let\eachwordone=\itshape
% \usepackage{tipa}

\IfFileExists{../localcommands.tex}{
 \addbibresource{../collection_tmp.bib}
 \addbibresource{../localbibliography.bib}
 \usepackage{langsci-optional}
\usepackage{langsci-gb4e}
\usepackage{langsci-lgr}

\usepackage{listings}
\lstset{basicstyle=\ttfamily,tabsize=2,breaklines=true}

%added by author
% \usepackage{tipa}
\usepackage{multirow}
\graphicspath{{figures/}}
\usepackage{langsci-branding}

 
\newcommand{\sent}{\enumsentence}
\newcommand{\sents}{\eenumsentence}
\let\citeasnoun\citet

\renewcommand{\lsCoverTitleFont}[1]{\sffamily\addfontfeatures{Scale=MatchUppercase}\fontsize{44pt}{16mm}\selectfont #1}
  
 %% hyphenation points for line breaks
%% Normally, automatic hyphenation in LaTeX is very good
%% If a word is mis-hyphenated, add it to this file
%%
%% add information to TeX file before \begin{document} with:
%% %% hyphenation points for line breaks
%% Normally, automatic hyphenation in LaTeX is very good
%% If a word is mis-hyphenated, add it to this file
%%
%% add information to TeX file before \begin{document} with:
%% %% hyphenation points for line breaks
%% Normally, automatic hyphenation in LaTeX is very good
%% If a word is mis-hyphenated, add it to this file
%%
%% add information to TeX file before \begin{document} with:
%% \include{localhyphenation}
\hyphenation{
affri-ca-te
affri-ca-tes
an-no-tated
com-ple-ments
com-po-si-tio-na-li-ty
non-com-po-si-tio-na-li-ty
Gon-zá-lez
out-side
Ri-chárd
se-man-tics
STREU-SLE
Tie-de-mann
}
\hyphenation{
affri-ca-te
affri-ca-tes
an-no-tated
com-ple-ments
com-po-si-tio-na-li-ty
non-com-po-si-tio-na-li-ty
Gon-zá-lez
out-side
Ri-chárd
se-man-tics
STREU-SLE
Tie-de-mann
}
\hyphenation{
affri-ca-te
affri-ca-tes
an-no-tated
com-ple-ments
com-po-si-tio-na-li-ty
non-com-po-si-tio-na-li-ty
Gon-zá-lez
out-side
Ri-chárd
se-man-tics
STREU-SLE
Tie-de-mann
}
%  \boolfalse{bookcompile}
%  \togglepaper[5]%%chapternumber
}{}


\begin{document}
\maketitle\label{WOWA:ch:15}

\section{Introduction}
Aramaic\footnote{The orthography has been adapted slightly to normalize transcription across dialects. \citegen{Khan2016CUrmi} <c>, <ɟ>, <k̭> correspond to <k>, <g> and <q> here. \citegen{Panoussi1990Senaya} <e> and \citegen{Khan2004SuleyHalab} <ĭ> both correspond to <ə> here. \citet{Garbell1065a} and \citegen{Khan2008Barwar} <o> and <u> for /ø/ and /y/ in Jewish Urmi\il{Neo-Aramaic (NENA)!J. Urmi} respectively correspond to <ö> and <ü> here. Superscript ⁺ indicates the following word or syllable is pronounced with additional velarization or pharyngealization.} is a Semitic language that has been attested in writing since the first millennium BC and used to be spoken more widely in West Asia. The modern Aramaic vernaculars in Iran and northeastern Iraq mainly belong to the North Eastern Neo-Aramaic (NENA) subgroup.\footnote{For the closely related Central Neo-Aramaic group in Anatolia, see \textcitetv{chapters/16_Noorlander_Anatolia}.}  Another relevant Neo-Aramaic subgroup, known as Mandaic\il{Mandaic}, spoken by the Mandaeans of southwestern Iran and southern Iraq \parencite[e.g.][]{Haberl2011NeoMandaic} lies beyond the scope of this chapter.

NENA comprises a continuum of highly diverse and severely endangered dialects of Jewish (J.) and Christian (C.) communities that used to span an area from western Iran to southeastern Turkey. Most of the Jewish dialects are extinct or border extinction, and only a rapidly diminishing number of elderly speakers–generally known as \textit{kurdim}–reside in Israel today. Apart from their regional identification, e.g. \textit{sənaye} `people from Səna, i.e. Sanandaj,' \textit{ʾurməžnaye} `people from Urmi, i.e. Urmia,' the Christian speakers self-identify as \textit{suraye} `Syrian Christian' and refer to their language as \textit{surət} `Syriac'. The Christians belong to various denominations, primarily the Chaldean Catholic Church and the Assyrian Church of the East, which may or may not coincide also with their linguistic and ethnic identification, respectively. Since more recent times, however, the self-identification among  native speakers in both the homeland and diaspora as the Assyrian people, i.e. \textit{ʾaturaye}, has extended beyond tribal, religious and geographic affiliations, and the same holds true for Chaldeans, albeit to a lesser extent. The havoc wreaked by the tumultuous 20th century and the recent atrocities in the name of Islamic State in the spread of the Syrian Civil War into Iraq has resulted in the massive displacement of Aramaic-speaking Christians and the destruction of entire villages. Consequently, the vast majority of speakers from Iran no longer reside in their original environment but as diaspora communities in Northern America (San Diego, California \& Detroit, Michigan) and Australia (Sydney). Today, the largest Assyrian communities in Iran reside in Urmia and Tehran. The majority of Aramaic speakers in the Middle East, however, is found in Iraqi Kurdistan where the use of the literary \textit{koine} based on the variety of the Urmia county (West Azerbaijan, Iran) has become increasingly widespread and accepted in education, media and sermons. 

\figref{NAINEI:fig:1} displays a selection of originally Neo-Aramaic speaking towns in Iran and northeastern Iraq. The Greater Zab River serves as an isogloss for both Jewish and Christian dialects, dividing the Jewish dialects into two major groups: \textit{Lishana Deni}, e.g. Zakho and Duhok, in the west vs. the Trans-Zab Jewish subgroup in the east \parencite{Mutzafi2008TransZab}, such as Arbel (Erbil, Hewlêr), Urmi (Urmia, Orumiyeh) and Sanandaj (Sine). The dialects around the settlement Barzan represent a \isi{transition zone}. The Trans-Zab cluster has been heavily influenced by contiguous Iranian languages (e.g. \citealt{Kapeliuk2004IrTuArAr}; \citealt{Noorlander2014Diversity}; \citealt{Khan2020ContactChange}), in particular those dialects in the southeastern periphery, in Iranian Kurdistan and Kermanshah. The Trans-Zab Jewish dialects in the north also had outposts into southeastern Turkey, namely Başkale and Gawar (Yüksekova; \citealt{Garbell1065a}). While the Christian dialects form a continuum from Turkey through Iraq to Iran, clusters can also be recognized in Iranian Azerbaijan and the Iraqi provinces of Erbil and Sulaymaniyah, with, however, only one easternmost outpost Sena (Sanandaj) in Iranian Kurdistan. The dialects near the Iraq-Iran-Turkey borders such as Diyana (Soran) constitute a \isi{transition zone}.

\largerpage[2]
In what follows, I shall \isi{focus} on the NENA dialects in the eastern periphery whose statistically dominant ordering of subject, \isi{object} and verb can be characterized as SOV, and where the post-predicate slot is reserved for Goals (for a definition, see §\ref{NAINEI:sec:2.2.2}), i.e. SVG. The \isi{object} placement in these dialects is distinct from the typology of (Central) Semitic \isi{word order}, as well as the majority of Neo-Aramaic dialects in modern-day Turkey and Iraq (see \citetv{chapters/16_Noorlander_Anatolia}). This chapter will show, however, that the same basic, i.e. statistically dominant, \isi{word order} does not hold to the same degree in every dialect for each \isi{argument} type, in line with the general rationale of the WOWA project.

\begin{figure}[t]
    \includegraphics[width=\textwidth]{figures/NoorNAFig1.png}
    \caption{Location of the main Northeastern Neo-Aramaic dialects discussed in this chapter}
    \label{NAINEI:fig:1}
\end{figure}

\tabref{NAINEI:tab:1} shows a list of the datasets from the WOWA corpus with their sources\footnote{Numbered texts and numbered segments are separated by colons, e.g. 25:§2 means Text 25, Paragraph 2, and page numbers and segments by periods, e.g. 101.§2, Page 101, Paragraph 2, page numbers and lines by dots, e.g. 101.1, Page 101, Line 1.}  and partial metadata used for the analysis of non-subject arguments and their respective position before or after the predicate in accordance with the framework and coding guidelines of the WOWA databank.\footnote{See \href{https://multicast.aspra.uni-bamberg.de/resources/wowa/data/_docs/guidelines/wowa_coding-guidelines.pdf}{https://multicast.aspra.uni-bamberg.de/resources/wowa/data/\_docs/guidelines/wowa\_coding-guidelines.pdf}.} A handful of additional data were taken from \citet[120--128]{Panoussi1990Senaya} for Christian Sanandaj, which is not part of the WOWA corpus. Concerning the J. Urmi\il{Neo-Aramaic (NENA)!J. Urmi} doculect based on \citet{Khan2008Barwar} and C. Urmi\il{Neo-Aramaic (NENA)!C. Urmi} doculect based on \citet{Khan2016CUrmi}, it is likely that these dialects cannot be taken as representatives of NENA of Iranian Azerbaijan as a whole. Even a cursory glance at the material collected by \citet{Garbell1065a} and \citet{Hopkins1989Naghada} suggests that there seems to be more variation, and the same holds true for the Jewish and Christian dialects of Salmas documented by \citet{Duval2009NASalamas} and \citet{Tsereteli1976}. The final results from other texts could, therefore, be different and approximate more closely the typology of the NENA varieties elsewhere.

\citet{Hopkins1999NAIran}, \citet{Khan2012NENA,Khan2019Anatolia,Khan2019WIran}, and \citet[100--206]{Noorlander2021Alignment} provide general overviews of the Trans-Zab Jewish NENA dialects, especially in Iran. \citet{Gutman2018AttNENA}\footnote{See especially p. 143 for J. Zakho\il{Neo-Aramaic (NENA)!J. Zakho}, pp. 220–230 for J. Urmi\il{Neo-Aramaic (NENA)!J. Urmi}, pp. 232–234 for J. Sanandaj\il{Neo-Aramaic (NENA)!J. Sanandaj}, p. 291 for Kurdish\il{Kurdish}. Compare also \citet{Cohen2012NENAZaxo} on Jewish Zakho.}  provides a comparative overview of Noun-Genitive orders, \citet{NoorlanderMolin2022WordOrder} an overview of Verb-Object and Verb-Oblique, \citet[398--401]{Khan2020ContactChange} that of Auxiliary-Verb and Verb-Object. Most grammatical descriptions do not discuss \isi{word order} in detail, except for Khan's voluminous grammars \parencite{Khan2008Barwar,Khan2016CUrmi} and \citet{Coghill2018ISNA}, though, apart from \citet{Molin2021Dohok} and \citet{NoorlanderMolin2022WordOrder}, no statistics are provided. Nevertheless, for virtually all NENA dialects considered here, which pattern constitutes the basic \isi{word order} is largely unquestioned, except for Christian Urmi and Sardarid (see §\ref{NAINEI:sec:2.2.1}). 

The following sections provide a general overview of \isi{word order} for which a synopsis is offered in \tabref{NAINEI:tab:2}, where plus (+) corresponds to placement after the head, and minus (–) corresponds to placement before the head, respectively. Word order configurations in NENA, however, are sensitive to pragmatic effects not coded in the WOWA corpus, e.g. any \isi{argument} can undergo focalization to the immediately pre-verbal position or \isi{topicalization} to clause-initial position (e.g. \citealt{NoorlanderMolin2022WordOrder}: 243–245; the difference between definite and indefinite arguments is coded, however, see §\ref{NAINEI:sec:2.2.1}.). For our purpose, \isi{word order} patterns of the clause will be identified on the basis of frequency, as discussed in \textcitetv{chapters/1_Haigetal_Intro}, in line with \parencite[73--78]{dryer_word_2007}. Finally, we use the basic surface-syntax-based parameters coded for the WOWA corpus rather than pragmatic or formal criteria of movement.

\begin{table}
    \begin{tabularx}{\textwidth}{lrR{1.5cm}R{1.8cm}Q}
\lsptoprule
\textbf{Doculect} & \textbf{Speakers} & \textbf{Total tokens} & \textbf{Analysed tokens} & \textbf{Source} \\
\midrule
J. Sanandaj & 4 & 2837 & 1184 & \citealt{Noorlander2022WOWAJSana} based on \citealt{Khan2009JSana} \\
\tablevspace
J. Urmi & 1 & 923 & 502 & \citealt{NoorlanderStilo2022JUrmi}  based on \citealt{Khan2008Barwar}: 398–439 \\
\tablevspace
C. Urmi & 2 & 865 & 724 & \citealt{Noorlander2022WOWACUrmi} based on \citealt{Khan2016CUrmi}: Texts A2, A39  \\
\tablevspace
C. Shaqlawa & 3 & 524 & 444 & \citealt{Noorlander2022WOWACShaqlawa}  based on \citealt{Khanetal2022FolkloreII}: Texts 4, 23 and 35 \\
\lspbottomrule
    \end{tabularx}
    \caption{NENA datasets from the WOWA corpus discussed in this chapter}
    \label{NAINEI:tab:1}
\end{table}


\begin{table}
    \begin{tabularx}{\textwidth}{Xccccccc}
\lsptoprule
\textbf{Doculect} & \textbf{VO} & \textbf{VAddr} & \textbf{VGoal} & \textbf{NGen} & \textbf{NAdj} & \textbf{AdjSt} & \textbf{CopPred} \\
\midrule
C. Barwar & + & + & + & + & + & + & +/– \\
C. Shaqlawa & – & + & + & + & + & +/– & +/– \\
C. Urmi & – & – & + & + & + & +/– & +/– \\
C. Sanandaj & – & + & + & + & + & (?) & – \\
J. Sanandaj & – & + & + & + & + & +/– & – \\
J. Urmi & – & – & + & + & +/– & – & – \\
\lspbottomrule
    \end{tabularx}
    \caption{Overview of dominant configurations }
    \label{NAINEI:tab:2}
\end{table}

\section{Word order profile}
\subsection{Noun phrases}
\subsubsection{Determiner/noun}
Across all doculects of NENA considered here, Demonstrative-Numeral-Noun order predominates, e.g.

\ea\label{NAINEI:ex:1}
Demonstrative-Numeral-Noun\\
C. Urmi \il{Neo-Aramaic (NENA)!C. Urmi}\citep[A55:§7]{Khan2016CUrmi} \\
\gll ʾannə tré ʾojaxə \\
     \textsc{dem.pl} two clan\textsc{.pl} \\
\glt `these two clans'
\z

Noun-Demonstrative-Adjective order also occurs, particularly in the expressions of `the elder' or `the eldest': 

\ea\label{NAINEI:ex:2}
Noun-Adjective-Demonstrative-Adjective\\
C. Urmi \il{Neo-Aramaic (NENA)!C. Urmi}\citep[A1:§29]{Khan2016CUrmi}\\
\gll kačala ʾasli \textbf{ʾo} ⁺gur-a \\
     bald.person\textsc{.m.sg} original \textsc{dem.m.sg} big\textsc{-m.sg} \\
\glt `\textbf{the} elder, original, baldhead' 
\z

\subsubsection{Noun/attribute}
Attributes, such as \isi{adjective} phrases,  follow the head noun they modify, e.g.

\ea\label{NAINEI:ex:3}
Noun-Adjective\\
J. Sanandaj \il{Neo-Aramaic (NENA)!J. Sanandaj}\citep[B:§58]{Khan2009JSana} \\
\gll knəšta rab-ta \\
     synagogue\textsc{.sg.f} big\textsc{-sg.f} \\
\glt `a big synagogue'
\z

\ea\label{NAINEI:ex:4}
Noun-Adjective\\
C. Sanandaj \il{Neo-Aramaic (NENA)!C. Sanandaj}\citep[125.§6]{Panoussi1990Senaya}\\
\gll ṣoma rab-a \\
     fast\textsc{.m.sg} big\textsc{-m.sg} \\
\glt `the great fast'
\z

\ea\label{NAINEI:ex:5}
Noun-Modifier-Adjective\\
C. Urmi \il{Neo-Aramaic (NENA)!C. Urmi}\citep[A42: §34]{Khan2016CUrmi} \\
\gll brata \textbf{ʾuxča} šap̂ə́r-ta  \\
     girl\textsc{.sg.f} such beautiful\textsc{-sg.f} \\
\glt `\textbf{such a} beautiful girl' 
\z

\begin{sloppypar}
Adjective-Noun order, e.g. (\ref{NAINEI:ex:6}), if tolerated, is pragmatically restricted–generally increasing the attribute's emotional significance–and its higher rate of occurrence is area-specific, namely specific to Iranian Azerbaijan, see §\ref{NAINEI:sec:3.2.5} for discussion. 
\end{sloppypar}

\ea\label{NAINEI:ex:6}
Modifier-Adjective-Noun\\
C. Urmi \il{Neo-Aramaic (NENA)!C. Urmi}\citep[A3:§81]{Khan2016CUrmi}\\
\gll ʾuxča \textbf{šapir-a} qal-a  \\
     such beautiful\textsc{-m.sg} voice\textsc{-m.sg} \\
\glt `such a \textbf{beautiful} voice' 
\z

The primary adjectives denoting relative size, i.e. `small' and `big,' tend to remain closer to the head noun (cf. \citealt{Khan2016CUrmi}II: 44):

\ea\label{NAINEI:ex:7}
J. Koy Sanjaq \il{Neo-Aramaic (NENA)!J. Koy Sanjaq}\citep[202.§26]{Mutzafi2004Koya} \\
\gll xa belá \textbf{ruww-á} jwan \\
     a house\textsc{.m.sg} big\textsc{-m.sg} beautiful \\
\glt `a beautiful, \textbf{large} house' 
\z

\ea\label{NAINEI:ex:8}
J. Sulemaniyya \il{Neo-Aramaic (NENA)!J. Sulemaniyya}\citep[R:§144]{Khan2004SuleyHalab}\\
\gll bela \textbf{ruww-á} hulaʾ-á \\
     house\textsc{.m.sg} big\textsc{-m.sg} Jewish\textsc{-m.sg} \\
\glt `a \textbf{big} Jewish house'
\z

Genitive constructions show Noun-Genitive order (see \citealt{Gutman2018AttNENA}, especially Chapter 4, for an overview and recent analysis of NENA genitive constructions). Prepositions similarly also serve as heads, e.g. J. Koy Sanjaq\il{Neo-Aramaic (NENA)!J. Koy Sanjaq} \textit{qam-əd ʔod=belá} lit. front-of of=house `in front of the house' \parencite[175]{Mutzafi2004Koya}. Juxtaposition can also suffice, e.g. C. Diyana\il{Neo-Aramaic (NENA)!C. Diyana-Zariwaw}-Zariwaw \textit{šəmma sawun-i} lit. name grandfather-my `the name of my grandfather' \parencite[315]{Napiorkowska2015DiyanaZ}. 

\ea\label{NAINEI:ex:9}
Noun-Genitive\\
C. Sardarid \il{Neo-Aramaic (NENA)!C. Sardarid}\citep[13:§5]{Younansardaroud2001Sardarid} \\
\gll šəmm-əd d-o naša \\
     name\textsc{-cstr} \textsc{gen-dem.m.sg} man\textsc{.m.sg} \\
\glt `the name of that man' 
\z

\ea\label{NAINEI:ex:10}
Noun-Genitive\\
J. Koy Sanjaq \il{Neo-Aramaic (NENA)!J. Koy Sanjaq}\citep[1B:§18]{Mutzafi2004Koya}\\
\gll šuḷṭan-əd ʾod=ḥaywan-é \\
     king\textsc{-cstr} \textsc{link=}animal\textsc{-pl} \\
\glt `the king of the animals' 
\z

Genitive-Noun order is restricted, also known as the emotive genitive,\footnote{See \citet{Hopkins2009emotiveGen}, \citet[100--102]{Cohen2012NENAZaxo}, \citet[143, 182,315]{Gutman2018AttNENA}.} intensifying the speaker's emotional attitude, e.g.

\ea\label{NAINEI:ex:11}
Genitive-Noun\\
C. Diyana \il{Neo-Aramaic (NENA)!C. Diyana}\citep[18.1:§35]{Napiorkowska2015DiyanaZ} \\
\gll ala munix-əd xəmyan-i \\
     God\textsc{.m.sg} give.rest\textsc{.ptcp.m.sg-cstr} uncle\textsc{.m.sg-}my \\
\glt `my late uncle' 
\z

Other constituents can intervene between head and genitive, as shown in (\ref{NAINEI:ex:12a}). Pronominal possessors are suffixed directly to the head or expressed by a following independent genitive, e.g. (\ref{NAINEI:ex:12b}). 

\ea
\ea\label{NAINEI:ex:12a}
Coordination, Noun-Genitive\\
C. Diyana \il{Neo-Aramaic (NENA)!C. Diyana}\citep[18.7:§16]{Napiorkowska2015DiyanaZ}\\
\gll šop-əd ʾaqle ʾu ʾəd=xzür-u \\
     print\textsc{.pl-cstr} foot\textsc{.pl} and \textsc{link=}pig\textsc{.pl-}their \\
\glt `their (lit. the) footprints and also [those] of their piglets' 
\ex\label{NAINEI:ex:12b}
free \isi{pronoun}, Noun-Genitive\\
C. Diyana \il{Neo-Aramaic (NENA)!C. Diyana}\citep[18.6:§7]{Napiorkowska2015DiyanaZ}  \\
\gll dost-əd \textbf{did-i} \\
     friend\textsc{.m.sg-cstr} \textsc{gen-}my  \\
\glt `a friend of \textbf{mine}' 
\z
\z

\subsection{Verbal complements}
\subsubsection{Object/verb}\label{NAINEI:sec:2.2.1}
While Verb-Object predominates in NENA dialects, the dialects in Iran and northeastern Iraq generally show Object-Verb order. Earlier treatments of the Christian dialects in Iran identified no primary \isi{word order} on the basis of frequency (\citealt{Younansardaroud2001Sardarid}: 209; \citealt{Khan2020ContactChange}: 398–401). As shown in \tabref{NAINEI:tab:3}, when lumping all types of direct objects together, the statistically dominant order overall in the NENA doculects considered here is \isi{OV}. \tabref{NAINEI:tab:3} gives the general numerical data for \isi{direct object} placement in the NENA doculects in Iran as well as C. Shaqlawa\il{Neo-Aramaic (NENA)!C. Shaqlawa} (NE Iraq), excluding \textit{wh}-elements.

\begin{table}[t]
        \begin{tabularx}{.5\textwidth}{lYY}
\lsptoprule
\textbf{Doculect} & \multicolumn{2}{c}{\textbf{Object}} \\
\cmidrule{2-3}
& \textbf{\textit{n}} & \textbf{PP} \\
\midrule
C. Urmi & 258 & 16\% \\
C. Shaqlawa & 108 & 12\% \\
J. Sanandaj & 386 & 5\% \\
C. Sanandaj & 50 & 4\% \\
J. Urmi & 172 & 1\% \\
\lspbottomrule
    \end{tabularx}
    \caption{Rate of post-predicate (PP) objects }
    \label{NAINEI:tab:3}
\end{table}

\begin{table}[b]
        \begin{tabular}{lrrrrrrrr}
\lsptoprule
& \multicolumn{2}{c}{\textbf{Indefinite}} & & & \multicolumn{2}{c}{\textbf{Definite}} & &  \\
\textbf{Doculect} & \multicolumn{2}{c}{\textbf{nominal}} & \multicolumn{2}{c}{\textbf{Other}} & \multicolumn{2}{c}{\textbf{nominal}} & \multicolumn{2}{c}{\textbf{Pronoun}}  \\
 \cmidrule(lr){2-3}\cmidrule(lr){4-5}\cmidrule(lr){6-7}\cmidrule(lr){8-9}
 & \textbf{\textit{n}} & \textbf{PP} & \textbf{\textit{n}} & \textbf{PP} & \textbf{\textit{n}} & \textbf{PP} & \textit{\textbf{n}} & \textbf{PP} \\
\midrule
C. Urmi & 49 & 49\% & 19 & 16\% & 153 & 8\% & 37 & 8\% \\
C. Shaqlawa & 45 & 13\% & 7 & 29\% & 49 & 8\% & 7 & 29\% \\
J. Urmi & 65 & 0\% & 13b & 8\% & 87 & 1\% & 13 & 0\% \\
C. Sanandaj & 32 & 3\% & – & – & 18 & 6\% & (1 & 0\%) \\
J. Sanandaj & 244 & 3\% & 20\parbox{0mm}{\footnote{This also includes arguments bound as a possessor to the nominal element of light verb constructions.}} & 5\% & 82\parbox{0mm}{\footnote{The four tokens with the idiomatic phrase `May God give X rest' with \isi{VO} order in J. Sanandaj\il{Neo-Aramaic (NENA)!J. Sanandaj} have been excluded here \parencite[see][]{NoorlanderMolin2022WordOrder}.}} & 7\% & 18 & 0\% \\
\lspbottomrule
    \end{tabular}
    \caption{Rate of post-predicate (PP) objects divided by definiteness and argument type}
    \label{NAINEI:tab:4}
\end{table}


Different types of objects, however, should be considered in their own right, drawing on the distinctions made in the WOWA data (also possessums, see §\ref{NAINEI:sec:3.2.1}). \tabref{NAINEI:tab:4} gives the statistics for direct objects divided in accordance with the additional variables of \isi{definiteness} and pronominal categories coded in the WOWA corpus, which are illustrated in (\ref{NAINEI:ex:13}--\ref{NAINEI:ex:14}). ``Pronoun,'' here, includes personal and demonstratives, both bare and prepositional, such as (\ref{NAINEI:ex:13b}) and (\ref{NAINEI:ex:14b}), but excludes indefinite and reflexive pronouns, which are subsumed under ``Other,'' such as (\ref{NAINEI:ex:13c}) and (\ref{NAINEI:ex:14c}). The number of pronominal tokens is, however, relatively low, especially in the case of C. Shaqlawa\il{Neo-Aramaic (NENA)!C. Shaqlawa} and C. Sanandaj\il{Neo-Aramaic (NENA)!C. Sanandaj}. It is thus not possible to draw any conclusions about these two dialects without more tokens. Moreover, bound pronominal objects are more common than their independent counterparts in NENA \parencite[see][]{NoorlanderMolin2022WordOrder}.

\ea\label{NAINEI:ex:13}
\ea\label{NAINEI:ex:13a}
Nominal definite (flagged), Object-Verb\\
J. Sanandaj \il{Neo-Aramaic (NENA)!J. Sanandaj}\citep[A:\S 18]{Khan2009JSana} \\
\gll ʾay broná \textbf{həl-d-ay} \textbf{bratá} gbe-Ø \\
     \textsc{dem.sg} boy\textsc{.m.sg} \textsc{dom-gen-dem.sg} girl\textsc{.sg.f} \textsc{ind.}want-\textsc{a.3sg.m} \\
\glt `The boy loves \textbf{the girl}.'
\ex\label{NAINEI:ex:13b}
Pronominal, Object-Verb\\
J. Sanandaj \il{Neo-Aramaic (NENA)!J. Sanandaj}\citep[C:\S 3]{Khan2009JSana} \\
\gll ʾaná \textbf{ʾea} šmi-li mən Bahrám \\
     I this heard\textsc{.pfv-a.1sg} from Bahram \\
\glt `I heard \textbf{this} from Bahram.' 
\ex\label{NAINEI:ex:13c}
Other, Object-Verb\\
J. Sanandaj \il{Neo-Aramaic (NENA)!J. Sanandaj}\citep[A:\S 48]{Khan2009JSana} \\
\gll \textbf{kŭl-e} kalw-ā́-wa-le \\
     all-of.it\textsc{.sg.m} write\textsc{-A.3pl-pst-O.3sg.m} \\
\glt `They would write \textbf{everything} down.'
\z
\z

\ea\label{NAINEI:ex:14}
\ea\label{NAINEI:ex:14a}
Nominal indefinite, Verb-Object\\
C. Urmi \il{Neo-Aramaic (NENA)!C. Urmi}\citep[A39:\S 42]{Khan2016CUrmi} \\
\gll Ø-mayy-ət ⁺raba goz-ə \\
     \textsc{sbjv-}bring\textsc{-A.2sg.m} many walnut\textsc{-pl} \\
\glt `You should bring many walnuts.'
\ex\label{NAINEI:ex:14b}
Pronominal (flagged), Object-Verb\\
C. Urmi \il{Neo-Aramaic (NENA)!C. Urmi}\citep[A2:\S 25]{Khan2016CUrmi} \\
\gll ʾatən \textbf{qa-diyyi} bət-⁺qa\d{t}l-ət.ˈ \\
     you\textsc{.sg.m} \textsc{dom-gen.1sg} \textsc{fut-}kill\textsc{-A.2sg.m} \\
\glt `You shall kill \textbf{me}.'
\ex\label{NAINEI:ex:14c}
Other, Object-Verb\\
C. Urmi \il{Neo-Aramaic (NENA)!C. Urmi}\citep[A2:\S 35]{Khan2016CUrmi} \\
\gll \textbf{gan-o} ⁺rupp-a-la ⁺ʾal-sepa \\
     \textsc{refl-3sg.f} threw\textsc{.pfv-O.3sg.f-A.3sg.f} on-sword\textsc{.sg.m} \\
\glt `She threw \textbf{herself} onto the sword.'
\z
\z


All else being equal, \tabref{NAINEI:tab:4} demonstrates that \isi{OV} order has grammaticalized completely in the Jewish doculects as well as Christian Sanandaj. J. Urmi\il{Neo-Aramaic (NENA)!J. Urmi}, as represented in \citet{Khan2008Barwar}, seems to have the most rigid kind of \isi{OV}. It is possible, however, that Jewish NENA doculects of Iran collected by \citet{Garbell1065a} and \citet{Hopkins1989Naghada}, although predominately \isi{OV}, contain a  higher rate of post-predicate Os than \citet{Khan2008Barwar}. The higher rate of independent pronouns in Iranian Azerbaijan, namely J. and C. Urmi\il{Neo-Aramaic (NENA)!C. Urmi}, may well be due to contact with Azeri\il{Turkic!Azeri}. If \isi{OV} order was completely grammaticalized in Christian Urmi, we would expect a rate similar to that in Jewish Urmi. Definiteness, however, is a major factor in \isi{object} placement in C. Urmi\il{Neo-Aramaic (NENA)!C. Urmi}. An overall decrease in the rate of post-predicate objects can be observed: the indefinite nominals and other pronouns are more likely to occur in pre-verbal position than, respectively, the definite nominals and personal and demonstrative pronouns.  This also seems to hold true for C. Shaqlawa\il{Neo-Aramaic (NENA)!C. Shaqlawa}, but to a lesser extent still, i.e. only 13\% of the indefinite objects are post-predicate.

\subsubsection{Verb/goal}\label{NAINEI:sec:2.2.2}

The \isi{endpoint} of motion verbs and caused motion verbs, such as `to come' and `to bring' respectively, are subsumed under \textit{Goal} (abbreviated G) here, while  \textit{Recipient} (R) refers to the human \isi{endpoint} of a \isi{transfer} like `to give' and \textit{Addressee} (Addr) to that of verbs of speech, e.g. `to say,' `to ask,' `to talk'. \textit{Beneficiaries} (Ben), i.e. indirect participants who are advantaged or disadvantaged by the action, have also been added here for completeness' sake. These \isi{argument} classes are illustrated in (\ref{NAINEI:ex:15}--\ref{NAINEI:ex:18}) for C. Sanandaj\il{Neo-Aramaic (NENA)!C. Sanandaj}. \tabref{NAINEI:tab:5} displays the statistics resulting from the relevant datasets, which comprises all pronouns and full nominals. Here, the tokens from \citet{Younansardaroud2001Sardarid} for the dialect of Sardarid have also been added.

\ea\label{NAINEI:ex:15}
\ea\label{NAINEI:ex:15a}
Goal, motion verb\\
C. Sanandaj \il{Neo-Aramaic (NENA)!C. Sanandaj}\citep[1:\S 4]{Panoussi1990Senaya} \\
\gll say \textbf{arxe} \\
     \textsc{imp.}go\textsc{.sg.m} mill\textsc{.pl} \\
\glt `Go \textbf{to the mill}!'
\ex\label{NAINEI:ex:15b}
Goal, caused motion verb\\
C. Sanandaj \il{Neo-Aramaic (NENA)!C. Sanandaj}\citep[2:\S 14]{Panoussi1990Senaya} \\
\gll tam-dāre-Ø-le \textbf{gaw} \textbf{ṣanoq-aw} \\
     \textsc{pst.pfv-}put\textsc{-A.3sg.m-O.3sg.m} in chest-his \\
\glt `He placed him \textbf{in the wooden chest}.'
\z
\z

\ea\label{NAINEI:ex:16}
Addressee\\
C. Sanandaj \il{Neo-Aramaic (NENA)!C. Sanandaj}\citep[4:\S 13]{Panoussi1990Senaya} \\
\gll mere \textbf{tlas-a} gor-əd baxta \\
     said\textsc{.pfv.pst.3sg.m} to\textsc{-3sg.f} husband\textsc{.sg.m-cstr} woman\textsc{.sg.f} \\
\glt `The woman's husband said \textbf{to her}...'
\z

\ea\label{NAINEI:ex:17}
Recipient\\
C. Sanandaj \il{Neo-Aramaic (NENA)!C. Sanandaj}\citep[3:\S 16]{Panoussi1990Senaya} \\
\gll tm-ēw-ə́n-wa-lu \textbf{tlas-ox}  \\
     \textsc{pst.pfv-}give\textsc{-A.1sg.m-pst-O.3pl} to\textsc{-2sg.m} \\
\glt `I had given them \textbf{to you}.'
\z

\ea\label{NAINEI:ex:18}
Beneficiary\\
C. Sanandaj \il{Neo-Aramaic (NENA)!C. Sanandaj}\citep[2:\S 2]{Panoussi1990Senaya} \\
\gll ayət \textbf{ta} \textbf{kalba} hādax gī-wəd-lox \\
     you\textsc{.sg.m} for dog\textsc{.sg.m} such \textsc{ant-}did\textsc{-A.2sg.m}  \\
\glt `You have done such a thing \textbf{for a dog}.'
\z

\begin{table}[t]
        \begin{tabular}{lrrrrrrrr}
\lsptoprule
\textbf{Doculect} & \multicolumn{2}{c}{\textbf{G}} & \multicolumn{2}{c}{\textbf{R}} & \multicolumn{2}{c}{\textbf{Addr}} & \multicolumn{2}{c}{\textbf{Ben}}  \\
  \cmidrule(lr){2-3}\cmidrule(lr){4-5}\cmidrule(lr){6-7}
& \textbf{\textit{n}} & \textbf{PP} & \textbf{\textit{n}} & \textbf{PP} & \textbf{\textit{n}} & \textbf{PP} & \textit{\textbf{n}} & \textbf{PP} \\
\midrule
J. Urmi & 59 & 86\% & 19 & 11\% & 55 & 33\% & 19 & 32\% \\
C. Urmi & 129 & 92\% & 11 & 73\% & 37 & 24\% & 7 & 43\% \\
C. Sardarid & --- & --- & 11\parbox{0mm}{\footnote{\citealt{Younansardaroud2001Sardarid}: 11:§5, 13:§1, 13:§2, 15:§3, 17:§1.}} & 100\% & 5\parbox{0mm}{\footnote{\citealt{Younansardaroud2001Sardarid}: 9:§3, 15:§4, 16:§2, 2x 16:§3, 2x 16:§5, 2x 16:§6, 17:§8, 17:§10.}} & 0\% & --- & --- \\
C. Shaqlawa & 44 & 91\% & 28 & 96\% & 31 & 97\% & 18 & 100\% \\
C. Sanandaj & 44 & 84\% & 4 & 75\% & 5 & 80\% & 11 & 55\% \\
J. Sanandaj & 207 & 91\% & 38 & 87\% & 32 & 72\% & 16 & 81\% \\
\lspbottomrule
    \end{tabular}
    \caption{Rate of post-predicate (PP) Goals, Recipients and Addressees (nominal and pronominal)}
    \label{NAINEI:tab:5}
\end{table}
These data are consistent with the findings in \citet{NoorlanderMolin2022WordOrder}. The post-verbal position is preferred for Goals across all dialects, and the same holds true for Recipients and Addressees in Iranian Kurdistan, here represented by the Jewish and Christian dialects of Sanandaj. The handful of tokens in C. Sanandaj\il{Neo-Aramaic (NENA)!C. Sanandaj} are relatively low, but suggest a typology similar to that of its Jewish counterpart, except in the case of beneficiaries, which in general do not seem to betray a clear tendency. It is far more common for Addressees than for Recipients to be placed before the verb in the Christian NENA dialects of Urmi and Sardarid (see §\ref{NAINEI:sec:3.2.3} for the areal significance of this Addressee/Recipient split), even though Recipients and Addressees are   generally marked by the same \isi{preposition} \textit{qa}-, e.g. 

\ea
\ea\label{NAINEI:ex:19a}
Verb-Recipient \\
C. Sardarid \il{Neo-Aramaic (NENA)!C. Sardarid}\citep[17:\S 10]{Younansardaroud2001Sardarid} \\
\gll ʾaxnan xa ton čapač jarāy Ø-yav-ax \textbf{qa} \textbf{dar⁺bar} \\
     we a ton sawdust must \textsc{sbjv-}give\textsc{-1pl} to court \\
\glt `We must give a ton of sawdust \textbf{to the court}.'
\ex\label{NAINEI:ex:19b}
Addressee-Verb \\
C. Sardarid \il{Neo-Aramaic (NENA)!C. Sardarid}\citep[15:\S 3]{Younansardaroud2001Sardarid} \\
\gll ⁺Šāh ⁺ʾAbbās \textbf{qa} \textbf{vazir} mār=ələ \\
     Shah Abbas to vizier \textsc{grd.}say\textsc{=cop.3sg.m} \\
\glt `Shah Abbas says \textbf{to the vizier}...'
\z
\z

In comparison to objects, the placement of the aforementioned \isi{endpoint} roles turns out to be more flexible overall. In the rare occasion that a ditransitive clause contains two full nominal objects, each \isi{argument} class typically occurs at either side of the verb: the Theme, like the O, before the verb, but the Recipient, like Goals, after it, and thus OVR as illustrated in (\ref{NAINEI:ex:19a}). The same order, i.e. OVR, is also common in most Kurdish\il{Kurdish} varieties \parencite{Haig2022PostPredicateCon}. This constituent order is cross-linguistically rare, as most languages reflect a preference to place both arguments at either side (\citealt{Haspelmath2015Ditransitive}). 

In the Jewish Urmi doculect \parencite{Khan2008JUrmi}, the pre-verbal position of O and R as well as Addressees is apparently the norm. The relative position of the Theme (O) and the Recipient (R) in a ditransitive clause is not entirely fixed, e.g. 

\ea
\ea\label{NAINEI:ex:20a}
Recipient-Theme-Verb\\
J. Urmi \il{Neo-Aramaic (NENA)!J. Urmi}\citep[\S 122]{Khan2008JUrmi} \\
\gll ba-⁺yal-i ⁺ruzi fərya höl-Ø \\
     to-children-my provision abundant \textsc{imp.}give\textsc{-sg} \\
\ex\label{NAINEI:ex:20b}
Theme-Recipient-Verb\\
J. Urmi \il{Neo-Aramaic (NENA)!J. Urmi}\citep[\S 113]{Khan2008JUrmi} \\
\gll ⁺ruzi fərya ba-⁺yal-i höl-Ø \\
     provision abundant to-children-my \textsc{imp.}give\textsc{-sg} \\
\glt `Give abundant provision to my children.'
\z
\z

\begin{sloppypar}
The most common order for the J. Urmi\il{Neo-Aramaic (NENA)!J. Urmi} dialect described in \citet{Khan2008JUrmi}, however, is ROV, especially for pronominal Recipients. A contrastive or topical O may precede the R, and immediately pre-verbal placement may add narrow \isi{focus} to the Recipient \parencite[see also][244--246]{NoorlanderMolin2022WordOrder}. 
\end{sloppypar}

This notwithstanding, the dominant order in the majority of Trans-Zab Jewish dialects is OVR. Statistics based on others doculects of Jewish NENA in Iran approximate more closely the typology of that of Jewish varieties in Iranian Kurdistan. Texts in \citet{Garbell1065a} and \citet{Hopkins1989Naghada} contain far more cases of post-verbal Recipients and Addresseees than our J. Urmi\il{Neo-Aramaic (NENA)!J. Urmi} doculect here \parencite{Khan2008JUrmi},\footnote{\citegen{Khan2008JUrmi} texts are based on one male speaker, as the number of speakers available was much smaller than at the time of \citet{Garbell1065a}, who was able to consult more speakers.}  as illustrated in (\ref{NAINEI:ex:21}) below, which suggests VR and VAddr are the more frequent position among Trans-Zab Jewish NENA dialects as a whole.

\ea\label{NAINEI:ex:21}
\ea\label{NAINEI:ex:21a}
Theme-Verb-Recipient\\
J. Urmi \il{Neo-Aramaic (NENA)!J. Urmi}\citep[149.18]{Garbell1065a} \\
\gll əsrá dehwé hwəl-le \textbf{ba} \textbf{d-ö} \textbf{mar} \textbf{xmará} \\
     ten gold\textsc{.pl} gave\textsc{.pfv-A.3sg.m} to \textsc{gen-dem.sg} owner\textsc{.cstr} donkey\textsc{.sg.m} \\
\glt `He gave ten pieces of gold \textbf{to the donkey owner}.'
\ex\label{NAINEI:ex:21b}
Verb-Addressee\\
J. Urmi \il{Neo-Aramaic (NENA)!J. Urmi}\citep[149.20]{Garbell1065a} \\
\gll mər-a \textbf{ba} \textbf{d-ö} \textbf{görá} \\
     said\textsc{.pfv-A.3sfg} to \textsc{gen-dem.sg} man\textsc{.sg.m} \\
\glt `He said \textbf{to that man}...'
\z
\z

\subsubsection{Become/complement}\label{NAINEI:sec:2.2.3}
In contradistinction to direct objects but similarly to Goals, the final state of change-of-state verbs, such as `to become,' `to turn into,' typically follows the predicate \parencite[e.g.][323]{Khan2008JUrmi}, as shown in (22). Under this class one may also subsume the complements of `to name' and `to fill,' although, here, the \isi{object} \isi{complement} does not represent the final outcome of the primary \isi{object}, but rather specifies the content of the verb. Neverthless, the pre-verbal position seems to be more frequent due to \isi{language contact} (see §\ref{NAINEI:sec:3.1.1}). The \isi{complement} can also be treated as a Recipient in J. Urmi\il{Neo-Aramaic (NENA)!J. Urmi} and J. Sanandaj\il{Neo-Aramaic (NENA)!J. Sanandaj} and flagged as such \parencite[see][251--252]{NoorlanderMolin2022WordOrder}.

\ea
\ea\label{NAINEI:ex:22a}
Become-Complement\\
J. Solduz \il{Neo-Aramaic (NENA)!J. Solduz}\citep[209]{Garbell1065a} \\
\gll pra xdər-e \textbf{dehwé} \\
     earth\textsc{.sg.m} became\textsc{.pfv-S.3sg.m} gold\textsc{.pl} \\
\glt `The earth turned \textbf{into pieces of gold}.'
\ex\label{NAINEI:ex:22b}
Object-Verb-Complement\\
J. Solduz \il{Neo-Aramaic (NENA)!J. Solduz}\citep[231]{Garbell1065a} \\
\gll tunnú xurjine ⁺məly-i-la \textbf{dehwé} \\
     both saddle.bag\textsc{.pl} filled\textsc{.pfv-O.3pl-A.3sg.f} gold\textsc{.pl} \\
\glt `She filled both saddle bags \textbf{with pieces of gold}.'
\z
\z

With the verbs `to say' and `to make,' the double \isi{object} construction shifts the semantics to that of `to name X Y' and `to make X into Y'. Thus, with the ambivalent verb (C.) \textit{(h)wd} or (J.) \textit{(h)wl} `to make,' the post-verbal placement of the \isi{object} correlates with its two-\isi{argument} valence and the semantics of the resulting condition, i.e. `to turn into,' rather than the effect of a single \isi{argument} verb, i.e. `to make' (\citealt{NoorlanderMolin2022WordOrder}: 252–253); \isi{contrast} \textit{kăbā́b} (\ref{NAINEI:ex:23a}) with (\ref{NAINEI:ex:23b}) below.

\ea
\ea\label{NAINEI:ex:23a}
Object-Verb\\
J. Sanandaj \il{Neo-Aramaic (NENA)!J. Sanandaj}\citep[B:\S 35]{Khan2009JSana} \\
\gll \textbf{kăbā́b} kol-i-wa  \\
     kebab\textsc{.sg.m} make\textsc{-A.3pl-pst} \\
\glt `They made \textbf{kebab}.'
\ex\label{NAINEI:ex:23b}
Verb-Complement\\
J. Sanandaj \il{Neo-Aramaic (NENA)!J. Sanandaj}\citep[B:\S 35]{Khan2009JSana} \\
\gll kol-í-wa-le \textbf{kăbā́b}   \\
     make\textsc{-A.3pl-pst-O.3sg.m} kebab\textsc{.sg.m} \\
\glt `They made it into \textbf{kebab}.'
\z
\z

\subsubsection{Other obliques}
Here, obliques are confined to constituents related to \textit{Place}, such as the \isi{locative} \isi{complement} of position verbs like `to sit,' e.g. (\ref{NAINEI:ex:24}), and the \textit{Source} of motion, e.g. (\ref{NAINEI:ex:25}). They are more likely post-predicate than objects, as given in \tabref{NAINEI:tab:6} for the same datasets, which comprises both lexical and pronominal arguments. In \tabref{NAINEI:tab:6}, we observe that the rate of post-predicate locatives is higher in the Christian varieties overall and in Christian Urmi especially, whereas the Jewish varieties show a stronger verb-final preference. Thus, even in C. Sanandaj\il{Neo-Aramaic (NENA)!C. Sanandaj}, the rate of post-predicate Oblique is high, while the dialect otherwise patterns almost exactly like its Jewish counterpart. The relatively high rates of post-predicate Obliques suggests a general tendency for local case relations (Source, Place, Goal) to occur after the predicate, at least more frequently than objects. This is matched by similar findings from Balochi, see \textcitetv{chapters/4_NourzaeiHaig_Balochi}.


\ea\label{NAINEI:ex:24}
Place/Locative\\
C. Sanandaj \il{Neo-Aramaic (NENA)!C. Sanandaj}\citep[4:\S 11]{Panoussi1990Senaya} \\
\gll ay xōr-e \textbf{gāw} \textbf{mezgəd} ītīw-a =le \\
     \textsc{dem.sg} friend\textsc{.sg.m-}his in mosque seated\textsc{.ptcp-sg.m} \textsc{=cop.3sg.m} \\
\glt `That friend of his is sitting \textbf{in the mosque}.'
\z

\ea\label{NAINEI:ex:25}
Source/Ablative\\
C. Sanandaj \il{Neo-Aramaic (NENA)!C. Sanandaj}\citep[3:\S 11]{Panoussi1990Senaya} \\
\gll kod yōma gaz m-šāqə́l-Ø-wa-le \textbf{mən} \textbf{šūqa} \\
     each day honey\textsc{.sg.m} \textsc{fut-}take\textsc{-A.3sg.m-pst-O.3sg.m} from market  \\
\glt `Every day he bought Turkish honey \textbf{from the market}.'
\z

\begin{table}
        \begin{tabular}{lcccc}
\lsptoprule
\textbf{Doculect} & \multicolumn{2}{c}{\textbf{Place}} & \multicolumn{2}{c}{\textbf{Source}} \\
\midrule
 & \textbf{\textit{n}} & \textbf{PP} & \textbf{\textit{n}} & \textbf{PP} \\
C. Urmi & 64 & 81\% & 39 & 44\% \\
C. Sanandaj & 11 & 64\% & 11 & 100\% \\
C. Shaqlawa & 12 & 58\% & 7 & 57\% \\
J. Urmi  & 40 & 40\% & 19 & 21\% \\
J. Sanandaj & 64 & 39\% & 28 & 14\% \\
\lspbottomrule
    \end{tabular}
    \caption{Rate of post-predicate (PP) place and source constituents (both nominal and pronominal)}
    \label{NAINEI:tab:6}
\end{table}

\subsection{Other predicate types}
\subsubsection{Copulas}\label{NAINEI:sec:2.3.1}
Post-predicate \isi{copula} placement correlates with verb-final syntax (e.g. \citealt{dryer_word_2007}: 91; see §2.4.2, for the use of the \isi{copula} in verbal clauses), and constitutes a typological trait of the languages in the area (e.g. \citealt{Matras2009LC}: 270; \citealt{haig_western_2017}: 404–405). The syntax of the \isi{copula} in main clauses in the Trans-Zab Jewish dialects differs from that in the Christian dialects in the same region \parencite{Khan2012Copula}. Post-predicate placement is almost categorical in these Jewish dialects, with the only exception being certain modal contexts, consistent with an overall higher rate of \isi{OV} in these Jewish dialects. The post-predicate position is favoured but less fixed in the Christian dialects.

NENA dialects generally distinguish between two \isi{copula} bases:

\begin{enumerate}[label=(\alph*)]
    \item pronominal copulas, e.g. C. Urmi\il{Neo-Aramaic (NENA)!C. Urmi} \textit{ʾina} `they are,' J. Urmi\il{Neo-Aramaic (NENA)!J. Urmi} \textit{ʾilu} `they are';
    \item verbal copulas, i.e. \textit{(h)wy} or \textit{(h)vy} `to be,' e.g. C. Urmi\il{Neo-Aramaic (NENA)!C. Urmi} \textit{ʾavi} `they may be,' J. Urmi\il{Neo-Aramaic (NENA)!J. Urmi} \textit{haweni} `they are'.
\end{enumerate}

Other particles can be added to either base  to express negation, past tense, subordination, and deixis, depending on the dialect. Importantly, the latter is excluded from this discussion, since the deictic (or presentative) \isi{copula} has a fixed pre-predicate position throughout \parencite[e.g.][227--247]{Molin2021Dohok}.  

When the \isi{copula} is placed after a constituent, there is a strong tendency for the \isi{copula} to undergo cliticization and attachment to the immediately preceding element, for which reason I shall distinguish between bound, i.e. enclitic, and unbound copulas.

Post-predicate, thus often bound, copulas are used in present and/or past tense affirmative clauses, e.g.

\ea
\ea\label{NAINEI:ex:26a}
Predicate-Copula, present (bound)\\
J. Koy Sanjaq \il{Neo-Aramaic (NENA)!J. Koy Sanjaq}\citep[1A:\S 1]{Mutzafi2004Koya} \\
\gll ʾoni=š be`eṛəx tremma nafar-e \textbf{=lu} \\
     they\textsc{=add} approximately two.hundred person\textsc{-pl} \textsc{=cop.3pl}  \\
\glt `They \textbf{are} about two hundred people.'
\ex\label{NAINEI:ex:26b}
Predicate-Copula, past\\
J. Koy Sanjaq \il{Neo-Aramaic (NENA)!J. Koy Sanjaq}\citep[1A:\S 1]{Mutzafi2004Koya} \\
\gll kullú ʾoni xet mšəlmān-é \textbf{we-lū} \\
     all.of.them those others Muslim\textsc{-pl} \textsc{cop.pst-3pl}  \\
\glt `All those others \textbf{were} Muslims.'
\z
\z

\ea
\ea\label{NAINEI:ex:27a}
Predicate-Copula, present (bound)\\
C. Koy Sanjaq \il{Neo-Aramaic (NENA)!C. Koy Sanjaq}\citep[215.§10]{Askar2021NAArmota} \\
\gll šm-ew šúmʿŭn \textbf{=ile} ba ăṣəl \\
     name\textsc{.sg.m-}his Simon \textsc{=cop.3sg.m}  in origin \\
\glt `His name \textbf{is} actually Simon.'
\ex\label{NAINEI:ex:27b}
Predicate-Copula, past\\
C. Koy Sanjaq \il{Neo-Aramaic (NENA)!C. Koy Sanjaq}\citep[220.§25]{Askar2021NAArmota} \\
\gll ana dkan-əd osta akram \textbf{yən-wa} \\
    I shop-of artisan Akram \textsc{cop.1sg.m-pst} \\
\glt `I \textbf{was} in the artisan shop Akram.' 
\z
\z
In deontic contexts, especially in idiomatic wishes, Copula-Predicate order is used in all dialects, e.g.
\ea\label{NAINEI:ex:28}
Copula-Predicate\\
J. Shino \il{Neo-Aramaic (NENA)!J. Shino}\citep[231.7]{Garbell1065a} \\
\gll ⁺šultaná \textbf{Ø-hawe-Ø} basim-a \\
     king\textsc{.sg.m} \textsc{sbjv-}be\textsc{-3sg.m} healthy\textsc{-sg.m}  \\
\glt `May the king \textbf{be} blessed!'
\z

\ea\label{NAINEI:ex:29}
Copula-Predicate\\
C. Urmi \il{Neo-Aramaic (NENA)!C. Urmi}\citep[A2:\S 4]{Khan2016CUrmi} \\
\gll malka \textbf{Ø-ʾav-ət} basim-a \\
     king\textsc{.sg.m} \textsc{sbjv-}be\textsc{-2sg.m} healthy\textsc{-sg.m} \\
\glt `May the king \textbf{be} well!'
\z
All \isi{copula} forms, whether bound or unbound, follow the predicate in all Jewish dialects of Iran and most Jewish dialects of northeastern Iraq, except in these deontic contexts. 

The \isi{copula} placement is more free in the Christian varieties, and the Jewish dialects in the Erbil region. Moreover, the Trans-Zab Jewish dialects in Iraq differ in negative \isi{copula} placement. It can either follow or precede the predicate in J. Arbel\il{Neo-Aramaic (NENA)!J. Arbel} \parencite[320]{Khan1999JArbel}, while the negative \isi{copula} follows the nominal predicate in J. Koy Sanjaq\il{Neo-Aramaic (NENA)!J. Koy Sanjaq} \parencite[107]{Mutzafi2004Koya} and J. Sulemaniyya\il{Neo-Aramaic (NENA)!J. Sulemaniyya} \parencite[254]{Khan2004SuleyHalab}, as it does in the Jewish dialects of Iran. Copula syntax is summarized in \tabref{NAINEI:tab:7} by contrasting Christian and Jewish Urmi.

\begin{table}
        \begin{tabularx}{\textwidth}{XXl}
\lsptoprule
 & \textbf{C. Urmi} & \textbf{J. Urmi} \\
\midrule
`My son is hungry.' & \textit{bruni kpā́nə\textbf{=lə}} & \textit{brönā́ kpiná\textbf{=ile}} \\
`He is hungry.' & \textit{kpā́nə\textbf{=lə}}  & \textit{kpiná\textbf{=ile}} \\
`I am a king.' & \textit{ʾana xa malk\textbf{=ən}} & \textit{ʾaná xa ⁺šültane\textbf{=len}} \\
`Who is your friend?'  & \textit{xorux máni\textbf{=lə}} & \textit{⁺barüxö́x mắni\textbf{=le}} \\
`My son is not hungry.' & \textit{bruni \textbf{lelə} kpina} & \textit{brönā́ kpiná \textbf{lewe}} \\
`He is/IS hungry.' & \textit{\textbf{ʾilə} kpina} & --- \\
`I am/AM a king.' & \textit{ʾana \textbf{ʾivən} xa malka} & --- \\
\lspbottomrule
    \end{tabularx}
    \caption{Copula placement in Jewish and Christian Urmi}
    \label{NAINEI:tab:7}
\end{table}

The unbound \isi{copula} freely occurs before the predicate in the dialects of Iranian Azerbaijan, and similarly also other dialects in Iraq such as C. Diyana\il{Neo-Aramaic (NENA)!C. Diyana} \parencite{Napiorkowska2015DiyanaZ}, as illustrated in (\ref{NAINEI:ex:30a}), instead of being cliticized to the predicate, e.g. (\ref{NAINEI:ex:30b}). The cliticization of the \isi{copula} to the \textit{subject} constituent, such as \textit{garda} `net' in ⁺\textit{ham} \textit{gárdə=la} ⁺\textit{allo} `also a net is on her' \parencite[II: 289]{Khan2016CUrmi}, is rare. The \isi{complement} occurs after the \isi{copula} in 16/96 (17\%) cases in the C. Urmi\il{Neo-Aramaic (NENA)!C. Urmi} doculect and 11/86 (13\%) cases in the C. Shaqlawa\il{Neo-Aramaic (NENA)!C. Shaqlawa} doculect, which, as expected, occurs more frequently than in the J. Sanandaj\il{Neo-Aramaic (NENA)!J. Sanandaj} doculect, which only has 5/215 (2\%) cases. Copula-Predicate order with a lexical subject, as shown in (\ref{NAINEI:ex:30a}), can be analysed as a cleft sentence, i.e. `I am (the one who is) the vizir of your father,' which is used in contexts of identification and specification and the expression of properties that are permanent or contra-presuppositional \parencite[II: 158--162]{Khan2016CUrmi}. 

\ea
\ea\label{NAINEI:ex:30a}
Copula-Predicate\\
C. Urmi \il{Neo-Aramaic (NENA)!C. Urmi}\citep[A2:\S 25]{Khan2016CUrmi} \\
\gll ʾana \textbf{ʾīn-va} vazzir-ət bab-ət diyy-ux \\
     I \textsc{cop.1sg-pst} vizier\textsc{-cstr} father\textsc{-cstr} \textsc{gen-2sg.m} \\
\glt `I \textbf{was} the vizier of your father.'
\ex\label{NAINEI:ex:30b}
Predicate-Copula\\
C. Urmi \il{Neo-Aramaic (NENA)!C. Urmi}\citep[A2:\S 25]{Khan2016CUrmi} \\
\gll ʾana ʾatxa naš\textbf{=ən-va} \\
    I such man\textsc{.sg.m=cop.1sg-pst} \\
\glt `I \textbf{was} such a \textbf{man}.'
\z
\z

\begin{sloppypar}
As compared in (\ref{NAINEI:ex:31}) and (\ref{NAINEI:ex:32}), while the affirmative \isi{copula} tends to be post-predicate, the negative \isi{copula} is generally pre-predicate  in Christian dialects, \isi{contrast} \textit{=la} `she is' and \textit{lewa} `she is not' in (\ref{NAINEI:ex:31}) and \textit{=ən} `I am' and \textit{lēn} `I am not' in (\ref{NAINEI:ex:32}).
\end{sloppypar}

\ea\label{NAINEI:ex:31}
\ea\label{NAINEI:ex:31a}
Copula-Predicate\\
C. Shaqlawa \il{Neo-Aramaic (NENA)!C. Shaqlawa}\citep[Text 23:§28]{Khanetal2022FolkloreII}\\
\gll har máre-wən bráte \textbf{=la}  \\
     every.time say\textsc{.inf-1sg.m} girl\textsc{.sg.f} \textsc{=cop.3sg} \\
\glt `I keep saying \textbf{it is} a girl.' 
\ex\label{NAINEI:ex:31b}
Predicate-Copula\\
C. Shaqlawa \il{Neo-Aramaic (NENA)!C. Shaqlawa}\citep[Text 23:§28]{Khanetal2022FolkloreII}\\
\gll har máre-wat \textbf{lewa} brata \\
    every.time say\textsc{.inf-2sg.f} \textsc{neg.cop.3sg.f} girl\textsc{.sg.f} \\
\glt `You keep saying \textbf{it is not} a girl.'
\z
\z

\ea\label{NAINEI:ex:32}
\ea\label{NAINEI:ex:32a}
Copula-Predicate\\
C. Urmi \il{Neo-Aramaic (NENA)!C. Urmi}\citep[A43:\S 15]{Khan2016CUrmi} \\
\gll ʾána=da \textbf{lēn} tliqa yala \\
     I\textsc{=add} \textsc{neg.cop.1sg} lost child \\
\glt `I \textbf{am not} a lost child.'
\ex\label{NAINEI:ex:32b}
Predicate-Copula\\
C. Urmi \il{Neo-Aramaic (NENA)!C. Urmi}\citep[A43:\S 15]{Khan2016CUrmi} \\
\gll ána=da brūn-malk \textbf{=ən} \\
    I\textsc{=add} son-king \textsc{=cop.1sg}  \\
\glt `I \textbf{am} the son of a king.' 
\z
\z

Finally, the Christian dialects may have a special \isi{copula} that fuses the relativizer \textit{d} with the \isi{copula}, e.g. \textit{d }`that/which/who' + \textit{ʾile} `is' > \textit{ṱ-ile} `that/which/who is,' e.g.

\ea\label{NAINEI:ex:33}
Copula-Predicate\\
C. Shaqlawa \il{Neo-Aramaic (NENA)!C. Shaqlawa}\citep[Text 4:§42]{Khanetal2022FolkloreII} \\
\gll ṣăṭáne =le \textbf{ṱ-ile} xor-a \\
     devil\textsc{.sg.m} \textsc{=cop.3sg} \textsc{rel-cop.3sg.m} friend\textsc{.sg.m-}her \\
\glt `It is the devil \textbf{who is} her friend.'
\z

\subsubsection{Auxiliaries}\label{NAINEI:sec:2.3.2}
The combination of Verb-Auxiliary and Object-Verb ordering is considered to be a form of harmonic \isi{word order} \parencite[e.g.][100]{Dryer1992Greeburg}, since both are considered \isi{head-final} in the standard assumption that the \isi{auxiliary} constitutes the head of \isi{auxiliary} in content verb pairings, since it bears verbal inflectional properties pertinent to the clause. The \isi{copula} can serve as an \isi{auxiliary} in NENA dialects, expressing several TAM properties as well as negation together with the non-finite verb. The syntax of the \isi{copula} as an \isi{auxiliary} in a verbal predicate can sometimes differ from its syntax in a non-verbal predicate.

First of all, as expected, the Object-Verb-Auxiliary pattern of the \isi{copula} is what we find in the Jewish Trans-Zab dialects, as illustrated in (\ref{NAINEI:ex:34a}). Future and modal auxiliaries, such as ⁺\textit{msy} `be able' in (\ref{NAINEI:ex:34b}), show Auxiliary-Object-Verb order in–at least historically–same subject modal complements.

\ea
\ea\label{NAINEI:ex:34a}
Object-Verb-Auxiliary\\
J. Urmi \il{Neo-Aramaic (NENA)!J. Urmi}\citep[194]{Garbell1065a} \\
\gll balki xa danká mən-nu əl-d-o araqčə́n \textbf{xəzy-á} \textbf{Ø-hawe-la} \\
     maybe a \textsc{clf} from\textsc{-3pl} \textsc{dom-gen-dem.sg} cap\textsc{.sg.m} seen\textsc{.ptcp-O.sg.m} \textsc{sbjv-}be\textsc{-A.3sg.f} \\
\glt `Maybe one of them \textbf{would have seen} that cap.'
\ex\label{NAINEI:ex:34b}
Auxiliary-Object-Verb\\
J. Urmi \il{Neo-Aramaic (NENA)!J. Urmi}\citep[194]{Garbell1065a} \\
\gll \textbf{⁺məssé} ⁺maé m-mešá \textbf{⁺palə́t-Ø} \\
    can\textsc{.3sg.m} water\textsc{.pl} in-forest\textsc{.sg.f} take.out\textsc{.sbjv-A.3sg.m} \\
\glt `He \textbf{would be able to find} water in the forest.'
\z
\z

In northeastern Iraq, however, the negative \isi{copula}, which functions as a negative \isi{auxiliary} with certain verbal forms, always precedes the content verb in J. Koy\il{Neo-Aramaic (NENA)!J. Koy Sanjaq} Sanjaq, such as \textit{lewan} in (\ref{NAINEI:ex:35}), and contrasts with its post-predicate placement as a \isi{copula} \parencite[108]{Mutzafi2004Koya} in §\ref{NAINEI:sec:2.3.1}. Other Trans-Zab Jewish dialects do not make use of this, but place a negator (\textit{la}) before the verb phrase, such as Neg-V-Aux in (\ref{NAINEI:ex:36}). 

\ea\label{NAINEI:ex:35}
Object-Negative Auxiliary-Verb\\
J. Koy Sanjaq \il{Neo-Aramaic (NENA)!J. Koy Sanjaq}\citep[2A:\S 4]{Mutzafi2004Koya} \\
\gll ʾixalá \textbf{le-wan} xəl-tá \\
     food\textsc{.sg.m} \textsc{neg-cop.A.1sg.f} eaten\textsc{.ptcp-A.sg.f} \\
\glt `\textbf{I have not} eaten any food.`
\z

\ea\label{NAINEI:ex:36}
Object-Negator-Verb-Auxiliary\\
J. Sulemaniyya \il{Neo-Aramaic (NENA)!J. Sulemaniyya}(hypothetical example based on \citealt{Khan2004SuleyHalab})\\
\gll xalá \textbf{la-}xəlte \textbf{=yan} \\
     food\textsc{.sg.m} \textsc{neg-}eaten\textsc{.ptcp-A.sg.f} \textsc{=cop.A.1sg.f} \\
\glt `\textbf{I have not} eaten any food.'
\z

The Christian NENA varieties, in turn, favour Auxiliary-Verb order throughout, as illustrated in (\ref{NAINEI:ex:37}), without affecting the Object-Verb order.  C. Urmi\il{Neo-Aramaic (NENA)!C. Urmi} could be characterised as `\isi{VO}' in light of the pre-verbal \isi{auxiliary} placement in \isi{contrast} to the Verb-Auxiliary order in others dialects where \isi{OV} predominates throughout \parencite[399]{Khan2020ContactChange}. There is no a priori reason, however, to consider the position of the \isi{auxiliary} more `basic' than that of the \isi{object} in the manifestation of Aux-\isi{OV} order, cf. also Dutch and German, which also exhibit Aux-\isi{OV} order in main clauses and are generally considered V-final in formal approaches to \isi{word order}. See also \citet[345]{Stilo2015AIAtlas} on Aux-\isi{OV} in same subject `want' complements in Colloquial Armenian\il{Armenian} and Azeri\il{Turkic!Azeri}; Aux-\isi{OV} is also the norm for Kurdish\il{Kurdish} (see \citetv{chapters/9_Mohammadirad_Gorani}).

\ea\label{NAINEI:ex:37}
\ea\label{NAINEI:ex:37a}
Auxiliary-Object-Verb\\
C. Urmi \il{Neo-Aramaic (NENA)!C. Urmi}\citep[B17L\S 4]{Khan2016CUrmi} \\
\gll ʾaxnan \textbf{k-av-ax-va} makkə praxa \\
     we \textsc{ind-}be\textsc{-1pl-pst} maize\textsc{.pl} hull\textsc{.inf} \\
\glt `while we \textbf{were hulling} the maize'
\ex\label{NAINEI:ex:37b}
Auxiliary-Object-Verb\\
C. Urmi \il{Neo-Aramaic (NENA)!C. Urmi}\citep[A16:\S 3]{Khan2016CUrmi} \\
\gll \textbf{lá-⁺ʾams-ən} ʾid-i \textbf{yavv-ən-na} \\
    \textsc{neg-}can\textsc{-S.1sg.m} hand\textsc{.sg.f-}my give-\textsc{A.1sg.m-O.3sg.f} \\
\glt `I \textbf{cannot give} my hand.'
\z
\z

\subsubsection{Complement of non-finite verbs}\label{NAINEI:sec:2.3.3}

Finally, the \isi{complement} of non-finite verb follows the syntax of the \isi{complement} of finite verbs. Thus, phasal verbs like `to begin' can combine with a nominal or non-finite form of the verb, such as an infinitive or gerund, respectively. If the nominal/non-finite form of the verb had followed the common pattern of Noun-Attribute, the order would have been Verb-Complement, as would be expected for the majority of NENA dialects. However, in the Trans-Zab Jewish dialects, if the \isi{complement} corresponds with an \isi{object}, it also betrays the same syntax, and is thus placed before the non-finite verb, as shown in (\ref{NAINEI:ex:105a}). The placement of the prefixal \isi{preposition} \textit{b-} before the \isi{complement} \textit{ʔərbe} `sheep' of the infinitive \textit{zwana} `to sell' suggests a type of compounding or noun incorporation strategy, i.e. `He began sheep-selling'. Interestingly, the same order occurs in C. Urmi, as shown in (\ref{NAINEI:ex:105b}), although here the \isi{preposition} \textit{b-} always attaches to the verbal form. How common this is among the other NENA dialects in the region requires further investigation.

\newpage
\ea\label{NAINEI:ex:105}
\ea\label{NAINEI:ex:105a}
Object-Verb(Non-finite)\\
J. Arbel \il{Neo-Aramaic (NENA)!J. Arbel}\citep[S:§72]{Khan1999JArbel} \\ 
\gll bde-le bə-ʔərbe zwana \\
begin\textsc{.pfv}-\textsc{A.3msg} at-sheep\textsc{.pl} selling\textsc{.inf} \\
\glt `He began selling sheep.'
\ex\label{NAINEI:ex:105b}
Object-Verb(Non-finite)\\
C. Urmi \il{Neo-Aramaic (NENA)!C. Urmi} \\ 
\gll ⁺šuri-lun ʔərbə bə-zvana \\
begin\textsc{.pfv}-\textsc{A.3pl} sheep\textsc{.pl} at-selling\textsc{.inf} \\
\glt `They began selling sheep.'
\z
\z

\section{Areal issues}
In the NENA-speaking area, predicate-final order becomes more prevalent in the east, where, today, varieties of Central Kurdish\il{Kurdish (Central)} are dominant.\footnote{This corresponds to \citegen{Stilo2005IranianBuffer} Buffer Zone 1, bordering a low \isi{OV} and high \isi{OV} zone.} The \isi{convergence} towards \isi{OV} in the Christian NENA varieties of Iranian Azerbaijan, however, is incomplete, as it still maintains a greater degree of flexibility for \isi{object} placement and betrays features of predicate-initial typology (§\ref{NAINEI:sec:3.2}) in its ordering of indefinite objects and possessums (§\ref{NAINEI:sec:3.2.1}), and that of light verb complements (§\ref{NAINEI:sec:3.2.2}, see also §\ref{NAINEI:sec:2.3.2} on auxiliaries). Thus, although Christian Urmi is by sheer frequency characterizable as predicate-final, it is, on closer examination, typologically more mixed than the Tran-Zab Jewish group.

The areal \isi{convergence} in \isi{word order} can be merely incidental, such as in the case of Noun-Genitive, Demonstrative-Noun-Adjective, and the prevalence of Verb-Goal and Become-Complement, which are common to Semitic and Iranian. The higher rates of \isi{OV} in the main dialects discussed in this chapter, however, are doubtless due to influence from neighbouring \isi{OV} languages, and yet the outcome differs per region and community. For example, the \isi{word order} profile of the Jewish and Christian dialects of Sanandaj as representative of the dialects of Iranian Kurdistan matches in many ways that of the local Iranian varieties, i.e. Central Kurdish\il{Kurdish (Central)} and Gorani\il{Gorani} (see \citetv{chapters/9_Mohammadirad_Gorani}). The fact that they show a similar general tendency towards Object-Verb and Predicate-Copula order is most likely due to contact with Iranian. However, many configurations are still liable to language-internal developments, such as the syntax of objects, which is sensitive to \isi{definiteness} in dialects like C. Urmi. Moreover, the same syntactic structure is attested where contact with neighbouring \isi{OV} languages presumably played no direct \isi{role}, such as Aux-\isi{OV} order and the pre-verbal placement of the \isi{object} before non-finite verbal forms (see \ref{NAINEI:sec:2.3.3}). The syntax of copulas and auxiliaries (see \ref{NAINEI:sec:2.3.2} and \ref{NAINEI:sec:2.3.2}) also exhibits language-internal peculiarities. Convergence in \isi{word order} can thus be partly understood as contact-induced reinforcement of pre-existing parallel patterns (e.g. \citealt{Silva-Corvalan1994ContactChange}, \citeyear{Silva-Corvalan2008Limits}\footnote{I am indebted to G. Haig for directing my attention to this reference.}; \citealt{Heine2008contact}: 42–43, 48–49), which presupposes that syntactic \isi{OV} presumably goes back to an original situation of more fluid order driven by pragmatic configurations.

Areal features are present in all Trans-Zab Jewish varieties, but the effect of contact on Christian varieties seems to be more varied. Doubtless, speakers' attitudes play a crucial \isi{role} in susceptibility to contact-induced change in NENA dialects \parencite{Noorlander2014Diversity} and thus \isi{word order} shifts. One may think that Christian NENA varieties show considerably higher rates of \isi{VO} than their Jewish peers. This seems to hold true for the Christian dialects of Iranian Azerbaijan, but it does not seem to apply to the southeasterly located dialects such as Christian Shaqlawa, Koy Sanjaq, Sulemaniyya and Sanandaj, where \isi{OV} seems to be as rigid as it can be in Trans-Zab Jewish dialects. 

Regional effects, in turn, may also cut across communal differences. Both Jewish and Christian NENA dialects of Iranian Azerbaijan, for instance, show a higher rate of Addressee-Verb order for verbs of speech, which parallels the northernmost dialects of Kurdish\il{Kurdish} \parencite{Haig2022PostPredicateCon}, and Adjective-Noun order for evaluative adjectives, which parallels Azeri\il{Turkic!Azeri}. On the other hand, other features are completely insensitive to \isi{word order} shifts: \isi{flagging}, for instance, remains prepositional in all NENA dialects, and none of the Neo-Aramaic varieties discussed here have developed a system of postpositions, despite having undergone a shift from \isi{VO} to \isi{OV}. 

Further research is required on the embedding of the shift from \isi{VO} to \isi{OV} in other internally and externally caused developments, as well as on the social-historical circumstances. Moreover, it is possible that the shift documented for NENA, especially in the east, is ultimately rooted in the spoken varieties of Mesopotamia during the Achaemenid period.\footnote{See \citet{haig_which_2023} for a discussion of this from a wider areal and typological perspective.}

\subsection{Iraqi and Iranian Kurdistan}
\subsubsection{Complement/become}\label{NAINEI:sec:3.1.1}
Similarly to Kurdish\il{Kurdish},  a general post-predicate proclivity becomes apparent in \isi{object} complements (see §\ref{NAINEI:sec:2.2.3}) of verbs of naming and turning something into something else in the Jewish varieties of northeastern Iraq and Iranian Kurdistan. The Iranian \isi{preposition} \textit{ba-} (see \citetv{chapters/9_Mohammadirad_Gorani}) has been transferred into these Jewish varieties along with the pattern, e.g.

\ea
\ea\label{NAINEI:ex:38a}
 Verb-Complement\\
J. Sanandaj \il{Neo-Aramaic (NENA)!J. Sanandaj}\citep[D:\S 1]{Khan2009JSana} \\
\gll xir-Ø \textbf{ba-}xá broná \\
     became\textsc{.pfv-S.3sg.m} to\textsc{-indef} boy\textsc{.sg.m} \\
\glt `He became a boy.'
\ex\label{NAINEI:ex:38b}
Verb-Complement\\
J. Sanandaj \il{Neo-Aramaic (NENA)!J. Sanandaj}\citep[B:\S 41]{Khan2009JSana} \\
\gll kol-i-wa-le \textbf{ba-}lešá \\
    make\textsc{-A.3pl-pst-O.3ms} to-dough\textsc{.ms} \\
\glt `They made it into dough.'
\z
\z

This tendency for post-predicate placement is, however, not supported by the statistics, as the pre-verbal position, for example in (\ref{NAINEI:ex:39}) below, occurs far more often than in the NENA dialects outside of the relevant area. The rate of post-predicate final states is 22\% (4/18) in C. Shaqlawa\il{Neo-Aramaic (NENA)!C. Shaqlawa}, 27\% (3/11) in J. Urmi\il{Neo-Aramaic (NENA)!J. Urmi}, 30\% (12/41) in J. Sanandaj\il{Neo-Aramaic (NENA)!J. Sanandaj}, and 62\% (8/13) in C. Urmi\il{Neo-Aramaic (NENA)!C. Urmi}, which can be contrasted with 90\% (18/20) in C. Barwar\il{Neo-Aramaic (NENA)!C. Barwar}. 

\ea\label{NAINEI:ex:39}
\ea\label{NAINEI:ex:39a}
Complement-Verb\\
C. Shaqlawa \il{Neo-Aramaic (NENA)!C. Shaqlawa}\citep[Text 35:§33]{Khanetal2022FolkloreII} \\
\gll ʾen-aw \textbf{kor} pəš-lu \\
     eye\textsc{.pl}-his blind became\textsc{.pfv-S.3sg.m} \\
\glt `His eyes became \textbf{blind}.'
\ex\label{NAINEI:ex:39b}
Complement-Verb\\
C. Shaqlawa \il{Neo-Aramaic (NENA)!C. Shaqlawa}\citep[Text 28:§19]{Khanetal2022FolkloreII} \\
\gll ʾāt bet-i \textbf{nura} qam-ʾawd-ət-e \\
    you\textsc{.sg} house\textsc{.sg.m}-my fire\textsc{.sg.f} \textsc{pst.pfv-}make\textsc{-A.2sg.m-O.3sg.m} \\
\glt `You have turned my home \textbf{into a hell}.'
\z
\z

\subsubsection{Light-verb complements}
Light verb constructions are also consistent with their usage in neighbouring languages, the non-referential nominal element preceding the light verb. The \isi{object} NP, as expected, occurs before the complex verb, e.g. (\ref{NAINEI:ex:40}). In Christian dialects of northeastern Iraq, recent Arabic\il{Arabic} loans are also incorporated into this strategy \parencite[474--475]{Hakeem2021ArabicNAAnkawa}, e.g. (\ref{NAINEI:ex:41}).



\ea\label{NAINEI:ex:40}
Noun-Light Verb\\
C. Sanandaj \il{Neo-Aramaic (NENA)!C. Sanandaj}\citep[126:\S 16]{Panoussi1990Senaya} \\
\gll xay-u ləbas-i \textbf{ḥazər} \textbf{k-od-i-lu}  \\
     one-of.them clothes-my ready \textsc{ind}-make\textsc{-A.3pl-O.3pl} \\
\glt `One of them \textbf{prepares} my clothes.'
\z

\ea\label{NAINEI:ex:41}
Noun-Light Verb\\
C. Shaqlawa \il{Neo-Aramaic (NENA)!C. Shaqlawa}\citep[Text 12:§28]{Khanetal2022FolkloreII} \\
\gll dăbi ʾana \textbf{ʾistəraḥat} \textbf{Ø-ʾawd-ən} \\
     must I rest \textsc{sbjv}-do\textsc{-1sg.m} \\
\glt `I must \textbf{rest}.'
\z

\subsubsection{Attribute/noun}

The NENA dialects in this area have, in general, not changed the placement of attributes, contrary to the dialects in Iranian Azerbaijan (§\ref{NAINEI:sec:3.2.5}). Attribute-Noun order is infrequent in the majority of NENA dialects, drawing attention to the listener with additional emotional \isi{weight} from the speaker's perspective. This position thus rarely occurs with more objective statements, except when two or more attributes are contrasted.\footnote{See \citet[143, 224, 232--233, 246--247, 250, 255]{Gutman2018AttNENA}.}  In several cases, however, the lexical item or morpheme is transferred along with the pattern. In (\ref{NAINEI:ex:42a}) below, for instance, the Persian\il{Persian} loan-\isi{adjective} \textit{behtarīn} `best' precedes the noun, and thus follows the expected Iranian order for superlatives. Interestingly, in (\ref{NAINEI:ex:42b}) below, the originally Kurdish \isi{adjective} \textit{zirej} precedes the head, presumably for pragmatic reasons, even though Noun-Adjective order is also the convention in Kurdish.

\ea\label{NAINEI:ex:42}
\ea\label{NAINEI:ex:42a}
Adjective-Noun\\
C. Sanandaj \il{Neo-Aramaic (NENA)!C. Sanandaj}\citep[122:\S 9]{Panoussi1990Senaya} \\
\gll behtarī́n ixale \\
     best food\textsc{.sg.m} \\
\glt `the best kinds of food'
\ex\label{NAINEI:ex:42b}
Adjective-Noun\\
C. Diyana \il{Neo-Aramaic (NENA)!C. Diyana}\citep[301]{Napiorkowska2015DiyanaZ} \\
\gll rɒba zirej naš-e \\
     much clever\textsc{.invar} person-\textsc{pl} \\
\glt `very clever people'
\z
\z

The \isi{adjective} \textit{xoš} is another case in point: it is a Persian\il{Persian} \textit{Wanderwort} in the NENA speaking area -- also found in Turkish and Iraqi Arabic -- and seems to be equally compatible with Adjective-Noun order in virtually all NENA dialects, in particular in combination with the noun \textit{naša} `man':

\ea\label{NAINEI:ex:43}
Adjective-Noun\\
C. Shaqlawa \il{Neo-Aramaic (NENA)!C. Shaqlawa}\citep[Text 17:§12]{Khanetal2022FolkloreII} \\
\gll xoš naša \\
     good man\textsc{.sg.m} \\
\glt `a good man'
\z

\ea\label{NAINEI:ex:44}
Adjective-Noun\\
J. Arbel \il{Neo-Aramaic (NENA)!J. Arbel}\citep[462.§326]{Khan1999JArbel} \\
\gll xoš naše=le \\
     good man\textsc{=cop.3sg.m} \\
\glt `He is a good man.'
\z

Furthermore, ordinal numbers in the Jewish NENA dialects of northeastern Iraq and Iranian Kurdistan all add the Iranian suffix \textit{-(a)min} to the native Aramaic numeral, e.g. \textit{tmanyamā́n} `eighth' from \textit{tmanya} `eight' + \textit{min}. Ordinals follow their heads as attributes as they do in the majority of NENA dialects, and this order incidentally convergences with Kurdish\il{Kurdish}. In J. Sanandaj\il{Neo-Aramaic (NENA)!J. Sanandaj}, however, the ordinal can also precede the noun as it does in Gorani\il{Gorani} \parencite[24]{MacKenzie1966}, e.g.

\ea\label{NAINEI:ex:45}
Ordinal-Noun\\
J. Sanandaj \il{Neo-Aramaic (NENA)!J. Sanandaj}\citep[213]{Khan2009JSana} \\
\gll arba-mā́n gorá \\
     four\textsc{-ord} man\textsc{.sg.f} \\
\glt `fourth man'
\z

\subsection{Iranian Azerbaijan}\label{NAINEI:sec:3.2}
\subsubsection{Object/verb and possessum/existential}\label{NAINEI:sec:3.2.1}
There is a direct correlation between \isi{object} and possesum placement in NENA dialects. In locative-exisential possessor constructions, exemplified in (\ref{NAINEI:ex:46}--\ref{NAINEI:ex:47}), the possessum can be considered the object-like \isi{argument}, and hence, not surprisingly, it occupies the same position as \isi{object} NPs in verbal clauses in the majority of NENA dialects. \figref{NAINEI:fig:2} shows the post-predicate possessum placement and the corresponding figures for definite objects. The \isi{VO} dialects of NENA, represented here by Barwar (northwestern Iraq) in the top, and the \isi{OV} dialects of NENA, represented by J. Urmi\il{Neo-Aramaic (NENA)!J. Urmi} in the bottom, are largely consistent throughout, although \isi{definiteness} is a factor of preverbal \isi{object} placement in C. Barwar\il{Neo-Aramaic (NENA)!C. Barwar}. The possessum (25 out of 38), however, occurs post-verbally far more often than definite objects in C. Urmi\il{Neo-Aramaic (NENA)!C. Urmi}, and is comparable to \textit{in}definite objects (24 out of 49, see §\ref{NAINEI:sec:2.2.1}). It is conceivable that discourse organisation also plays a \isi{role} in possessum-possessor constructions. Since possessums are generally indefinite, we would expect to observe the same distribution. The post-predicate possessums occur more frequently still, although not in a statistically significant way. This higher rate of post-predicate possessums might be due to formulaic language, e.g. opening formulas introducing discourse-new arguments, as illustrated in (\ref{NAINEI:ex:46}), which would favour post-predicate placement.

\begin{figure}[t]
        \includegraphics{figures/NoorNAFIg2.png}
    \caption{Rate of post-predicate (PP) possessums and definite objects}
    \label{NAINEI:fig:2}
\end{figure}

\ea\label{NAINEI:ex:46}
Existential-Possessum\\
C. Urmi \il{Neo-Aramaic (NENA)!C. Urmi}\citep[A39:\S 1]{Khan2016CUrmi} \\
\gll ʾət-va \textbf{xa-malka}. ʾaha malka ʾə́t-va-le \textbf{⁺\d{t}la} \textbf{bnunə} \\
     \textsc{exist-pst} one-king\textsc{.sg.m} \textsc{dem.sg.m} king\textsc{.sg.m} \textsc{exist-pst-3sg.m} three son\textsc{.pl} \\
\glt `There was \textbf{a king}. This king had \textbf{three sons}.'
\z
\ea\label{NAINEI:ex:47}
Possessum-Existential\\
C. Sanandaj \il{Neo-Aramaic (NENA)!C. Sanandaj}\citep[3:\S 1--2]{Panoussi1990Senaya} \\
\gll \textbf{xa} \textbf{gora} k-awe… ay gora məšulman-a \textbf{tre} \textbf{išənyase} k-āwe-le \\
     one man\textsc{.sg.m} \textsc{ind-}be \textsc{dem.sg} man\textsc{.sg.m} muslim\textsc{-sg.m} two woman\textsc{.pl} \textsc{ind-}be\textsc{-poss.3sg.m} \\
\glt `There was (lit. is) \textbf{a man}… This Muslim man had (lit. has) \textbf{two wives}.'
\z

\subsubsection{Light-verb complements}\label{NAINEI:sec:3.2.2}
Moreover, similarly to indefinite objects and to possessums, the nominal element of light verb constructions from Kurdish\il{Kurdish}, Persian\il{Persian} and Azeri\il{Turkic!Azeri} can occur after the light verb in C. Urmi\il{Neo-Aramaic (NENA)!C. Urmi}, e.g. \textit{kullə} \textit{ʾodilun} ⁺\textit{\textbf{hazər}} `they should prepare everything' \parencite[A3:\S 70]{Khan2016CUrmi}. By \isi{contrast}, such non-referential nominal elements precede the light verb in J. Urmi\il{Neo-Aramaic (NENA)!J. Urmi} as do all types of objects and possessums \parencite[173]{Garbell1965b}, e.g. \textit{ixalé} ⁺\textit{\textbf{hazər}} \textit{wudun} `prepare food!' \parencite[J. Urmi,][156]{Garbell1065a}. 

\subsubsection{Addressee/verb \& verb/goal}\label{NAINEI:sec:3.2.3}
Addressees, in turn, seem to have shifted in the same predicate-final direction as objects in Iranian Azerbaijan (§\ref{NAINEI:sec:2.2.2}), which is more prevalent in the northeast. Here, the syntactic organisation is role-specific and area-specific. The pre-verbal Addresseee but post-verbal Goal and Recipient split in C. Urmi\il{Neo-Aramaic (NENA)!C. Urmi} and Sardarid Azerbaijan matches the findings for some varieties of Central Kurdish\il{Kurdish (Central)} and northernmost dialects of Northern Kurdish\il{Kurdish (Northern)} (\citealt{Haig2022PostPredicateCon}: 358–359; \citetv{chapters/9_Mohammadirad_Gorani}). Influence from especially Azeri\il{Turkic!Azeri} and Persian\il{Persian} but also Armenian\il{Armenian}, however, cannot be ruled out either. A higher rate of Verb-Goal order, for example, also occurs in local Azeri\il{Turkic!Azeri} dialects due to Colloquial Persian\il{Persian (colloquial)} influence \parencite[75--77]{Kiral2001Persian}.

\subsubsection{Complement/become}
The Complement, i.e. a resulting condition, is generally placed before the verb in the NENA dialects of Iranian Azerbaijan studied here, which may even be fronted before the \isi{object}, as can be observed in (\ref{NAINEI:ex:48}). This pre-predicate placement matches the \isi{word order} in Turkic\il{Turkic} and Persian\il{Persian}, rather than what is expected in Kurdish\il{Kurdish}. The same holds true for Recipient-Theme-Verb order in the Jewish Urmi material in \citet{Khan2008Barwar}, as exemplified by (20) in §\ref{NAINEI:sec:2.2.2}.

\newpage
\ea\label{NAINEI:ex:48}
Complement-Object-Verb\\
C. Sardarid \il{Neo-Aramaic (NENA)!C. Sardarid}\citep[15:\S 1]{Younansardaroud2001Sardarid} \\
\gll trə ʾaxunvatə ⁺tla d-ani vəd-li \\
     two brother\textsc{.pl} three \textsc{gen-dem.pl} made\textsc{.pfv-1sg} \\
\glt `Into two brothers I turned three of them.'
\z

\subsubsection{Attribute/noun}\label{NAINEI:sec:3.2.5}
Adjective-Noun and Genitive-Noun, alongside Object-Verb, are the dominant orders in Turkic\il{Turkic} languages that NENA dialects in Iranian Azerbaijan have been in contact with,\footnote{See also \citet[171--172]{Garbell1065a}, \citet[39]{Khan2016CUrmi}, \citet[220--224, 332--334]{Gutman2018AttNENA}.} which could be considered parallel to the shift from \isi{VO} to \isi{OV} \citep[222, 233, 250--251]{Gutman2018AttNENA}. The \isi{complement} of nominal forms of the verb, such as agent nominalisations or participles ending in \textit{-ana}, precedes the verb in compound-like attributes across Trans-Zab Jewish NENA dialects, similarly to their \isi{OV} typology, \isi{contrast} (\ref{NAINEI:ex:49a}) with (\ref{NAINEI:ex:49b}) and (\ref{NAINEI:ex:49c}). In C. Urmi\il{Neo-Aramaic (NENA)!C. Urmi}, the attribute generally follows the head, as given in (\ref{NAINEI:ex:50a}), but sporadically the opposite order does occur, as illustrated in (\ref{NAINEI:ex:50b}). Also, when the \isi{adjective} constitutes the head of a compound, J. Urmi\il{Neo-Aramaic (NENA)!J. Urmi} patterns in a \isi{head-final} fashion but C. Urmi\il{Neo-Aramaic (NENA)!C. Urmi} \isi{head-initial}, cf. J. Urmi\il{Neo-Aramaic (NENA)!J. Urmi} \textit{dəqná-xwará} lit. `beard-white' and C. Urmi\il{Neo-Aramaic (NENA)!C. Urmi} lit. \textit{xvār-dəqna} `white-beard' for `grey-bearded'.

\ea
\ea\label{NAINEI:ex:49a}
Noun-Genitive\\
J. Urmi \il{Neo-Aramaic (NENA)!J. Urmi}\citep{Garbell1965b}\\
\gll gör-ət \textbf{xalünt-xun} \\
     husband\textsc{.sg.m-cstr} sister\textsc{.sg.f}-your\textsc{.2pl} \\
\glt `the husband of \textbf{your sister}'
\ex\label{NAINEI:ex:49b}
Genitive-Noun\\
J. Urmi \il{Neo-Aramaic (NENA)!J. Urmi}\citep[212]{Garbell1065a} \\
\gll \textbf{xalünt-xun} gör-an-a \\
     sister\textsc{.sg.f}-your\textsc{.2pl} marry\textsc{-agn-sg.m} \\
\glt `\textbf{your sister's} husband'
\ex\label{NAINEI:ex:49c}
Genitive-Noun (compound-like)\\
J. Urmi \il{Neo-Aramaic (NENA)!J. Urmi}\citep[86]{Garbell1065a} \\
\gll \textbf{masy-e} döq-an-a \\
     fish\textsc{-pl} catch\textsc{-agn-sg.m} \\
\glt `fisherman, lit. \textbf{fish} catcher' 
\z
\z
\ea
\ea\label{NAINEI:ex:50a}
Noun-Genitive\\
C. Urmi \il{Neo-Aramaic (NENA)!C. Urmi}\citep[III:44]{Khan2016CUrmi} \\
\gll doq-an-əd \textbf{nuyn-e} \\
     catch\textsc{-agn-cstr} fish\textsc{-pl} \\
\glt `fisherman, lit. catcher of \textbf{fish}'
\ex\label{NAINEI:ex:50b}
Genitive-Noun\\
C. Urmi \il{Neo-Aramaic (NENA)!C. Urmi}\citep[A48:\S 32]{Khan2016CUrmi} \\
\gll ʾo \textbf{prəzla} \textbf{⁺taptəpp-án-a} damurči \\
    \textsc{dem.sg} iron\textsc{.sg.m} bash\textsc{-agn-sg.m} blacksmith\textsc{.sg.m} \\
\glt `the \textbf{iron hammerer} blacksmith'
\z
\z

Similarly, ordinal numbers formed out of the fusion of the Iranian suffix \textit{-amin} and Turkic\il{Turkic} suffix \textit{-inji} (-IncI)  also precede the noun in J. Urmi\il{Neo-Aramaic (NENA)!J. Urmi} as they do in the source language of this morphological \isi{transfer}, e.g. \textit{xa tre-minji baxtá} `a second woman' \parencite[206]{Garbell1065a}.

Finally, more frequent Adjective-Noun order due to Azeri\il{Turkic!Azeri} occurs among NENA speakers in Iranian Azerbaijan. Contrast the following two near-identical  descriptions in J. Sanandaj\il{Neo-Aramaic (NENA)!J. Sanandaj} (W Iran) and J. Urmi\il{Neo-Aramaic (NENA)!J. Urmi} (NW Iran):

\ea\label{NAINEI:ex:51}
Noun-Adjective\\
J. Sanandaj \il{Neo-Aramaic (NENA)!J. Sanandaj}\citep[B:\S 158]{Khan2009JSana} \\
\gll kništa \textbf{rab-ta}=w kništa \textbf{zor-ta} \\
     synagogue\textsc{.sg.f} big\textsc{-sg.f}=and synagogue\textsc{.sg.f} small\textsc{-sg.f} \\
\glt `a \textbf{big} synagogue and a \textbf{small} synagogue'
\z

\ea\label{NAINEI:ex:52}
Adjective-Noun\\
J. Urmi \il{Neo-Aramaic (NENA)!J. Urmi}\citep[§156]{Khan2008Barwar} \\
\gll xa \textbf{rab-ta} knəšta xa \textbf{zör-ta} knəšta \\
     a big\textsc{-sg.f} synagogue\textsc{.sg.f} a small\textsc{-sg.f} synagogue\textsc{.sg.f} \\
\glt `a \textbf{big} synagogue and a \textbf{small} synagogue'
\z

\tabref{NAINEI:tab:8} contrasts corpus-based frequencies of Adjective-Noun order for the adjectives `small, young' and `red' in Jewish and Christian Urmi. This is consistent with a dominant Noun-Adjective order in C. Urmi\il{Neo-Aramaic (NENA)!C. Urmi} \parencite[II:39]{Khan2016CUrmi}. There is a greater degree of flexibility in J. Urmi\il{Neo-Aramaic (NENA)!J. Urmi},\footnote{See \citet[82--84]{Garbell1065a}, \citet[216--219]{Khan2008JUrmi}; \citet{Gutman2018AttNENA}.} however, and a dominant order cannot be identified for more evaluative adjectives like \textit{zora} `small, young' in this dialect.

\begin{table}
        \begin{tabularx}{\textwidth}{lrrYY}
\lsptoprule
& \multicolumn{2}{c}{\textbf{J. Urmi}} & \multicolumn{2}{c}{\textbf{C. Urmi}} \\
& \multicolumn{2}{c}{\textbf{(\citealt{Garbell1065a}\footnote{including Jewish Shino.};} \textbf{\citealt{Khan2008Barwar})} }  & \multicolumn{2}{c}{\textbf{(\citealt{Khan2016CUrmi})}} \\
\cmidrule(lr){2-3}\cmidrule(lr){4-5}
& \textbf{\textit{n}} & \textbf{ADJ-N} & \textbf{\textit{n}} & \textbf{ADJ-N} \\
\midrule
small, young & 33 & 49\% & 81\parbox{0mm}{\footnote{excluding the high frequency idiom \textit{yala sura} `baby.'}} & 6\% \\
red & 20 & 0\% & 18 & 0 \\
\lspbottomrule
    \end{tabularx}
    \caption{Rate of AdjN in NENA dialects of Iranian Azerbaijan}
    \label{NAINEI:tab:8}
\end{table}

When two adjectives modify the noun, they can also be placed at either side, cf. (\ref{NAINEI:ex:53c}). When the \isi{adjective} modifies a noun that is part of a nominal annexation construction, however, it tends to follow the noun phrase, as illustrated in (\ref{NAINEI:ex:53d}).

\ea
\ea\label{NAINEI:ex:53a}
Adjective-Noun\\
J. Urmi \il{Neo-Aramaic (NENA)!J. Urmi}\citep[83]{Garbell1065a} \\
\gll \textbf{zür-ta} xalünt-u \\
     small\textsc{-sg.f} sister\textsc{.sg.f-}their \\
\glt `their \textbf{youngest} sister'
\ex\label{NAINEI:ex:53b}
Noun-Adjective\\
J. Urmi \il{Neo-Aramaic (NENA)!J. Urmi}\citep[172]{Garbell1065a} \\
\gll brata \textbf{zür-ta} \\
    girl\textsc{.sg.f} small\textsc{-sg.f} \\
\glt `the \textbf{youngest} daughter'
\ex\label{NAINEI:ex:53c}
Adjective-Noun-Adjective\\
J. Urmi \il{Neo-Aramaic (NENA)!J. Urmi}\citep[192]{Garbell1065a} \\
\gll xa \textbf{zür-ta} tkana \textbf{šüšaband} \\
    a small\textsc{-sg.f} shop\textsc{.sg.f} glass.covered \\
\glt `a \textbf{small glass-covered} shop'
\ex\label{NAINEI:ex:53d}
Noun-Genitive-Adjective\\
J. Urmi \il{Neo-Aramaic (NENA)!J. Urmi}\citep[86]{Garbell1065a} \\
\gll brat-ət ⁺šültana \textbf{zür-ta} \\
    girl\textsc{.sg.f} king\textsc{.sg.m} small\textsc{-sg.f} \\
\glt `the king's \textbf{youngest} daughter'
\z
\z

Adjective/Noun configurations are said not to correlate with other configurations \parencite[e.g.][95--96]{Dryer1992Greeburg}, but the above finding for Jewish Urmi warrants further research into this understudied area. For example, is the syntax of adjectives more likely to be affected by contact than other nominal attributes or not?

Furthermore, sporadically, the attribute can consist of a gerundial verb phrase preceding the head noun, reminiscent of the Relative-Noun order in Turkic\il{Turkic} \parencite[173]{Garbell1965b}, e.g. (\ref{NAINEI:ex:54}--\ref{NAINEI:ex:55}). 

\ea\label{NAINEI:ex:54}
Numeral-Adjective-Gerund-Noun\\
C. Urmi \il{Neo-Aramaic (NENA)!C. Urmi}\citep[A56:\S 1]{Khan2016CUrmi} \\
\gll xa šap̂ər-ta max ⁺šrá bəllaya brata \\
     a beautiful\textsc{-sg.f} like lantern shine\textsc{.grnd} girl\textsc{.sg.f} \\
\glt `a beautiful girl shining like a lantern'
\z

\ea\label{NAINEI:ex:55}
Gerund-Noun\\
J. Shino/Solduz \il{Neo-Aramaic (NENA)!J. Solduz}\il{Neo-Aramaic (NENA)!J. Shino}\citep[84]{Garbell1065a} \\
\gll ba-šatoe xriw-e ⁺mae \\
     for-drink\textsc{.grnd} bad\textsc{-pl} water\textsc{.pl} \\
\glt `water bad for drinking'
\z

\subsubsection{Standard/adjective}
Finally, in the Jewish dialects of Iranian Azerbaijan, all instances of the standard of comparison seem to occur consistently before the \isi{adjective} rather than after it, compare (\ref{NAINEI:ex:56}) with (\ref{NAINEI:ex:57}). A stronger preference for the position before the predicate converges with the typology of local \isi{OV} languages.

\ea\label{NAINEI:ex:56}
Standard-Adjective\\
J. Solduz \il{Neo-Aramaic (NENA)!J. Solduz}\citep[211.19]{Garbell1065a} \\
\gll \textbf{mənn-áw} ⁺raba ⁺raba bis-sqil-é ita g-d-ay ⁺ahrá \\
     from-her very very more-beautiful\textsc{-pl} \textsc{exist} in\textsc{-gen-dem.sg.f} city\textsc{.sg.f} \\
\glt `There are far, far more beautiful women \textbf{than she} in this city.'
\z

\ea\label{NAINEI:ex:57}
Adjective-Standard\\
J. Koy Sanjaq \il{Neo-Aramaic (NENA)!J. Koy Sanjaq}\citep[1B:\S 24]{Mutzafi2004Koya} \\
\gll ʾo bā́š-faqir Ø-hawé-Ø \textbf{mənn-éw} \\
     he more-poor \textsc{sbjv-}be\textsc{-3sg.m} from\textsc{-3sg.m} \\
\glt `Even if he is poorer \textbf{than him}.'
\z

\section*{Abbreviations}
\begin{tabularx}{.5\textwidth}{@{}lQ@{}}
1 & 1st person \\
2 & 2nd person \\
3 & 3rd person \\
A & agent \\
\textsc{add} & additive \\
Addr & addresseee \\
\textsc{Adj} & {adjective} \\
\textsc{agn} & agent nominalization \\
\textsc{ant} & anterior \\
\textsc{aux} & {auxiliary} \\
\textsc{ben} & beneficiary \\
C. & Christian (linguistic variety) \\
\textsc{cop} & {copula} \\
\textsc{cstr} & construct \\
\textsc{dem} & demonstrative \\
{DOM} & Diffrential Object Marking \\
\textsc{exist} & existential \\
\textsc{f} & feminine \\
G & {Goal} \\
\textsc{gen} & genitive \\
\textsc{grd} & gerund \\
\end{tabularx}%
\begin{tabularx}{.5\textwidth}{@{}lQ@{}}
\textsc{imp} & Imperative \\
J. & Jewish (linguistic variety) \\
\textsc{link} & linker \\
\textsc{m} & masculine \\
n & total number of tokens \\
NENA & Northeastern Neo-Aramaic \\
NP & Noun phrase \\
O & {object} \\
\textsc{pl} & plural \\
PP & post-predicate \\
\textsc{pred} & predicate \\
\textsc{pst} & past \\
\textsc{ptcp} & participle \\
R & {recipient} \\
\textsc{refl} & Reflexive \\
S & subject (intransitive) \\
\textsc{sg} & singular \\
St & standard of comparison \\
V & {verb} \\
WOWA & = \citet{Haig.Stilo.Dogan.Schiborr2022} \\
& \\
\end{tabularx}

\section*{Acknowledgements}
I am indebted to the anonymous reviewers as well as to Dorota Molin and especially to Geoffrey Haig for their helpful comments on an earlier draft of this chapter. I hereby acknowledge the generous support of the European Research Council (ALHOME, 101021183). Views and opinions expressed are however those of the author only and do not necessarily reflect those of the European Union or the European Research Council Executive Agency. Neither the European Union nor the granting authority can be held responsible for them.

\nocite{HaigKhan2019LLWA}

\sloppy
\printbibliography[heading=subbibliography,notkeyword=this]

\end{document}
