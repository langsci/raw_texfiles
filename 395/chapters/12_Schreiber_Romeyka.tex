\documentclass[output=paper,colorlinks,citecolor=brown]{langscibook}
\ChapterDOI{10.5281/zenodo.14266353}
\author{Laurentia Schreiber\orcid{0000-0002-0051-1164}\affiliation{University of Bamberg} and Mark Janse\orcid{0000-0002-2824-3894}\affiliation{University of Cambridge}}
\title{Word order variation in Romeyka}
\abstract{The present chapter describes word order variation in Romeyka based on the multilingual spoken language dataset of the Word Order in Western Asia (WOWA) corpus. Descending from VO ancestry, Romeyka shows under contact from Turkish increasingly head-final syntax. While cross-linguistically relevant factors such as semantic role, flagging and phonological weight do not offer much explanation for the flexibility between pre- and post-verbal constituents in Romeyka, information structure and phrase type do seem to be relevant. In addition, inter-speaker variation has been found significant to account for word order variation in Romeyka, suggesting that in a setting of language shift, individual forms of bilingualism affect word order.}

%move the following commands to the ``local...'' files of the master project when integrating this chapter
% \usepackage{tabularx}
% \usepackage{langsci-optional}
% \usepackage{langsci-gb4e}
% \usepackage{enumitem}
% \bibliography{localbibliography}
% \newcommand{\orcid}[1]{}
% \let\eachwordone=\itshape

\IfFileExists{../localcommands.tex}{
 \addbibresource{../collection_tmp.bib}
 \addbibresource{../localbibliography.bib}
 \usepackage{langsci-optional}
\usepackage{langsci-gb4e}
\usepackage{langsci-lgr}

\usepackage{listings}
\lstset{basicstyle=\ttfamily,tabsize=2,breaklines=true}

%added by author
% \usepackage{tipa}
\usepackage{multirow}
\graphicspath{{figures/}}
\usepackage{langsci-branding}

 
\newcommand{\sent}{\enumsentence}
\newcommand{\sents}{\eenumsentence}
\let\citeasnoun\citet

\renewcommand{\lsCoverTitleFont}[1]{\sffamily\addfontfeatures{Scale=MatchUppercase}\fontsize{44pt}{16mm}\selectfont #1}
  
 %% hyphenation points for line breaks
%% Normally, automatic hyphenation in LaTeX is very good
%% If a word is mis-hyphenated, add it to this file
%%
%% add information to TeX file before \begin{document} with:
%% %% hyphenation points for line breaks
%% Normally, automatic hyphenation in LaTeX is very good
%% If a word is mis-hyphenated, add it to this file
%%
%% add information to TeX file before \begin{document} with:
%% %% hyphenation points for line breaks
%% Normally, automatic hyphenation in LaTeX is very good
%% If a word is mis-hyphenated, add it to this file
%%
%% add information to TeX file before \begin{document} with:
%% \include{localhyphenation}
\hyphenation{
affri-ca-te
affri-ca-tes
an-no-tated
com-ple-ments
com-po-si-tio-na-li-ty
non-com-po-si-tio-na-li-ty
Gon-zá-lez
out-side
Ri-chárd
se-man-tics
STREU-SLE
Tie-de-mann
}
\hyphenation{
affri-ca-te
affri-ca-tes
an-no-tated
com-ple-ments
com-po-si-tio-na-li-ty
non-com-po-si-tio-na-li-ty
Gon-zá-lez
out-side
Ri-chárd
se-man-tics
STREU-SLE
Tie-de-mann
}
\hyphenation{
affri-ca-te
affri-ca-tes
an-no-tated
com-ple-ments
com-po-si-tio-na-li-ty
non-com-po-si-tio-na-li-ty
Gon-zá-lez
out-side
Ri-chárd
se-man-tics
STREU-SLE
Tie-de-mann
}
%  \boolfalse{bookcompile}
%  \togglepaper[5]%%chapternumber
}{}


\begin{document}
\maketitle\label{WOWA:ch:12}

\section{Introduction}\label{Romeyka:ss:1}

Romeyka\il{Hellenic!Romeyka} is a variety of Pontic\il{Hellenic!Pontic Greek} Greek (henceforth PG) that is at present still spoken by Muslims in Trabzon Province in northeastern Turkey, although its status can be characterized as endangered (\citealt{schreiber2016assessing}; \citealt{schreiberSitaridou2017assessing}). Romeyka\il{Hellenic!Romeyka} belongs to the Hellenic branch of Indo-European. Like most modern Greek dialects, it is descended from postclassical (Koine) Greek\il{Hellenic!Koine}, with an undisputable Ionic \isi{substrate} dating back to the Milesian colonization of the Black Sea coast in the 7th--6th c. BCE. The closest relatives of Pontic\il{Hellenic!Pontic Greek} are Pharasiot\il{Hellenic!Pharasiot}  and Cappadocian\il{Hellenic!Cappadocian}, collectively known as (East) Asia Minor Greek (henceforth AMG).\footnote{The designation ``East Asia Minor Greek'' is due to \citeauthor{janseClitic2008} (e.g., \citeyear{janseClitic2008}: 191--192, \citeyear{janseBack2020}2020: 202--203) and implied in \citeauthor{dawkins1916}' ``Greek of eastern Asia Minor'' (\citeyear{dawkins1916}: 213). The qualification ``East'' is necessary in light of \citegen{ralli2020selected} broader acceptation of the geographical designation ``Asia Minor Greek'' (\citealt{ralli2020selected}). For a different view on the internal relationship between Pontic\il{Hellenic!Pontic Greek}, Pharasiot\il{Hellenic!Pharasiot}  and Cappadocian\il{Hellenic!Cappadocian} see \citet[40--55]{karatsareas2016asia}.} 

The location of Romeyka\il{Hellenic!Romeyka}, which is nestled in the remote, mountainous area of the Pontic Alps, has probably contributed to the preservation of the language and of some archaic features and facilitated the development of three main dialect areas around the townships of Of/Çaykara (abbreviated ROf), Sürmene (abbreviated RSür), and Tonya (see \citealt{schreibergrammar2022}). Romeyka\il{Hellenic!Romeyka} has been in contact with Turkish\il{Turkic!Turkish} varieties at least since the 17th century, although early contacts may date back till the 11th c. (\citealt{drettasPontic1997}: 5--6) and it is difficult to assess the intensity of contact\is{language contact!intensity of contact} with Turkish\il{Turkic!Turkish} throughout the Middle Ages and the Modern period. At least since intensified labour migration to larger cities in Turkey (as well as abroad) since the 1960s, the influence of Turkish\il{Turkic!Turkish} has significantly increased, causing cultural assimilation and \isi{language shift} at least in the urban speech communities (\citealt{schreiberSitaridou2017assessing}). Today, the majority of Romeyka\il{Hellenic!Romeyka} speakers are recessive bilinguals, with Turkish\il{Turkic!Turkish} as dominant language, which affects the linguistic structure of Romeyka\il{Hellenic!Romeyka} and facilitates contact-induced changes (\citealt{schreibergrammar2022}).

The density of documentation and grammatical description of Romeyka\il{Hellenic!Romeyka} is moderate, with increased interest in the field in the last years. After \citegen{mackridgeGreekSpeaking1987} work on the Muslim Pontic\il{Hellenic!Pontic Greek} Greek variety still spoken in Turkey, \citet{sitaridouINF2014,sitaridouGSEnclaves2013,sitaridouModality2014,sitaridouReframing2016,sitaridouVulnerabke2021} has contributed research mainly on the syntactic domain of Romeyka\il{Hellenic!Romeyka}, followed by Neocleous' 2017 doctoral thesis on \isi{word order} and \isi{information structure} in Romeyka\il{Hellenic!Romeyka} (\citealt{neocleous_word_2020}, cf. \citealt {neocleousEvolution2022}). Recently, \citet{schreibergrammar2022} has presented the first comprehensive grammatical description of Romeyka\il{Hellenic!Romeyka} based on a naturalistic spoken language corpus. A considerably larger body of literature is available on Pontic\il{Hellenic!Pontic Greek} Greek as spoken by Christian speakers in Turkey before the Greek-Turkish population exchange in 1923 (\citealt{deffner1878}, \citealt{parcharidis1880,parcharidis1888}) and in Greece after 1923 (\citealt{dawkins1931,dawkins1937}, \citealt{papadopoulos1933,papadopoulos1955,papadopoulos1958-1961}, \citealt{tombaidis1992,tombaidis1996}, \citealt{drettasPontic1997,drettas1999}, \citealt{revithiadouSpyropoulos2009}) as well as on its closest relatives Cappadocian\il{Hellenic!Cappadocian} (\citealt{janseCappadocian2023} and references therein) and Pharasiot\il{Hellenic!Pharasiot}  (\citealt{bagriacik_pharasiot_2018} and references therein).

The aim of the present chapter is to analyze the WOWA dataset of Romeyka\il{Hellenic!Romeyka} (\citealt{schreiber2021pontic}) with regard to the grammatical entities that occur in the post-verbal domain and to \isi{word order} in Romeyka\il{Hellenic!Romeyka} in general. Since Romeyka\il{Hellenic!Romeyka} is a shifting language under strong contact with Turkish\il{Turkic!Turkish} (see \citealt{schreiber2016assessing} and \citealt{schreiberSitaridou2017assessing} for a sociolinguistic assessment of language vitality), which displays fundamentally different word orders, the data reveal a lot of variation which needs to be accounted for. Thereby, theoretically relevant topics from Greek linguistics such as \isi{word order} properties of strong and weak pronouns and the question about a potential shift in Romeyka\il{Hellenic!Romeyka} \isi{word order} directionality will be touched upon. Most importantly, it will be argued that basic \isi{word order} properties (see Section \ref{Romeyka:ss:2} for a discussion of terminology) in Romeyka\il{Hellenic!Romeyka} as a shifting language varies significantly between speakers and seems to be crucially determined by the individual multilingual profiles of the speakers.

\begin{sloppypar}
The WOWA dataset of Romeyka\il{Hellenic!Romeyka} has been compiled and coded according to the methodology of the WOWA corpus outlined in \citetv{chapters/1_Haigetal_Intro}; basic descriptive statistics have been carried out in R (logistic regression models). The data consist of five coherent texts produced by three speakers, extracted from the larger naturalistic spoken language corpus of Romeyka\il{Hellenic!Romeyka} compiled by \citet{schreiber_inprep}.\footnote{For an overview of speakers and data in the WOWA dataset of Romeyka\il{Hellenic!Romeyka}, see \tabref{Romeyka:tab:1} in Section \ref{Romeyka:ss:6}. For a description of data collection and metadata see \url{multicast.aspra.uni-bamberg.de/resources/wowa/data/hellenic/ponticgreek_romeyka/wowa_hell_ponticgreek_romeyka__metadata.pdf.}} The texts were recorded during fieldwork in Turkey in June/July 2019. Romeyka\il{Hellenic!Romeyka} examples in the present chapter are referenced as follows: examples that stem from the WOWA dataset are referenced as \citet[text ID, token ID]{schreiber2021pontic}, and examples from the Romeyka\il{Hellenic!Romeyka} corpus by \citet{schreiber_inprep} are referenced by the respective code in the corpus (as explained in \citealt{schreibergrammar2022}).
\end{sloppypar}

In the following, Section \ref{Romeyka:ss:2} summarizes the present stage of research on \isi{word order} in Romeyka\il{Hellenic!Romeyka}. Section \ref{Romeyka:ss:3} sketches the general impact of \isi{information structure}. In Section \ref{Romeyka:ss:4}, the \isi{word order} profile of Romeyka\il{Hellenic!Romeyka} is characterized based on the WOWA data by focusing on different clause types (the NP in Section \ref{Romeyka:ss:4.1}, the PP in Section \ref{Romeyka:ss:4.2}), semantic and grammatical roles (Sections \ref{Romeyka:ss:4.3}-\ref{Romeyka:ss:4.5}), auxiliaries in Section \ref{Romeyka:ss:4.6}, and complex clauses in Section \ref{Romeyka:ss:4.7}. Section \ref{Romeyka:ss:5} on areal aspects of \isi{language contact} and Section \ref{Romeyka:ss:6} on the impact of recessive \isi{bilingualism} discuss further important factors in the variability of Romeyka\il{Hellenic!Romeyka} \isi{word order} patterns.

\section{Romeyka\il{Hellenic!Romeyka} word order: Background and previous analyses}\label{Romeyka:ss:2}

It is not straightforward to determine the present \isi{word order} profile of Romeyka\il{Hellenic!Romeyka} 
(a) due to a high pragmatically conditioned variability (see Section \ref{Romeyka:ss:3}), and
(b) due to ongoing \isi{language shift} to Turkish\il{Turkic!Turkish} and consequently high variation. 
It is generally agreed that Greek \isi{word order} is determined by \isi{information structure}. This was the case in Ancient Greek (\citealt{vanemdeboas2019}: 702--721), in Medieval Greek\footnotemark (\citealt{holtonetal2019}: 2022--2024) as well as in Standard Modern Greek (\citealt{holtonetal2012}: 518--520). As far as PG is concerned, it has been asserted that there is no basic worder (\citealt{drettasPontic1997}: 277--280). On the other hand, \citet{neocleousEvolution2022,neocleous_word_2020}, following \citet{sitaridouReframing2016}, investigated the diachronic development of Romeyka\il{Hellenic!Romeyka} \isi{word order} within the Minimalist framework and concluded that Romeyka\il{Hellenic!Romeyka} is a ``mixed directionality language'' \citep{neocleousNever2022} with inherited \isi{VO} order in pragmatically unmarked main clauses and \isi{OV} order in subordinate clauses reinforced by contact with Turkish\il{Turkic!Turkish}. 
\footnotetext{{See also \citet[45]{horrocksClitics1990} and \citet[11]{rafiyenko2Serzant020} for Postclassical Greek. It should be noted that in later publications, \citet{horrocksHistory2007} assumes \isi{VSO} as the informationally most neutral \isi{word order} in both Postclassical \citep[623]{horrocksHistory2007} and Late Medieval and Early Modern Greek \citep[2022--2023]{horrocksCambridge2019}.}}

\citet{neocleousEvolution2022,neocleous_word_2020} claims that Romeyka\il{Hellenic!Romeyka} has in main clauses an underlying \isi{VO} \isi{word order}, which is evident when all possible information-structural complications are set aside. Deviations from \isi{VO} are explained through information-structural conditions holding in certain contexts (for a similar approach for Pharasiot\il{Hellenic!Pharasiot}, see \citealt{bagriacik_pharasiot_2018}). In subordinate declarative clauses, unmarked \isi{word order} is, according to Neocleous, \isi{OV} with finite verbs and \isi{VO} with infinitives. Although Neocleous seems not to distinguish between \isi{complement} clauses preceding and following the matrix clause, most of his examples of subordinate clauses follow the matrix clause.

As has been addressed by \textcitetv{chapters/1_Haigetal_Intro}, it is by no means straightforward to determine unmarked or ``basic'' \isi{word order} in a language and there are different accounts on how to establish this (see \citealt{dryerDetermining2013} for a ``rule of thumb'' in determining dominant \isi{word order} based on relative frequency, cf. also \citealt{dryerFrequency1995}). \citet{neocleousEvolution2022,neocleous_word_2020} defines a basic \isi{word order} for both matrix and subordinate clauses in Romeyka\il{Hellenic!Romeyka} based on the pragmatically unmarked \isi{word order}, which he defines as an ```all-\isi{focus} sentence', aka `a presentational \isi{focus} sentence', containing neither old information nor any presuppositions'' (\citealt{neocleous_word_2020}: 143). He elicits such clauses in response to the question `What happened?', departing from the assumption that in response to this question all information is new and thus of equal discourse-pragmatic status.\footnote{Certainly, \citet{neocleous_word_2020} is not the first to use this approach; for a description of how neutral \isi{word order} has been defined in the previous literature, see \citet[146--151]{bagriacik_pharasiot_2018}.} However, departing from a naturalistic spoken language corpus of multilingual speakers as in \citet{schreibergrammar2022}, frequencies of \isi{word order} patterns show a different picture, although certain limitations apply as well, such as the size of the corpus. While many observations of Neocleous are confirmed by the present data (e.g., the impact of \isi{information structure}), the analysis cannot be adopted wholesale (see especially Section \ref{Romeyka:ss:6} on \isi{inter-speaker variation}). Furthermore, the present analysis follows largely the WOWA approach (see \citetv{chapters/1_Haigetal_Intro}) which is role-specific; the existing literature on \isi{word order} in Romeyka\il{Hellenic!Romeyka} does, to our knowledge, not include claims specific to Goals, Locations, etc. So, our prominent question should not be whether Greek had at any stage (Postclassical, Medieval, Modern) an unmarked \isi{VO} \isi{word order}, but rather whether \isi{OV} has become more prominent in Romeyka\il{Hellenic!Romeyka} --- regardless of \isi{information structure} --- under \isi{contact influence} from Turkish\il{Turkic!Turkish}. Therefore, our analysis shall be based on frequencies of \isi{head-final} orders. Note that this is still not straightforward to deal with, as Turkish\il{Turkic!Turkish} allows for \isi{word order} variation as well (\citealt{goksel2005turkish}: 343--349), at least in informal spoken language --- and certainly in the Trabzon Turkish\il{Turkic!Turkish Trabzon} dialect (see \citealt{schreiberMuslim}).

\section{The role of information structure}\label{Romeyka:ss:3}

This section summarizes the \isi{role} of \isi{information structure} in Romeyka\il{Hellenic!Romeyka} \isi{word order} as proposed by \citet{neocleousEvolution2022,neocleous_word_2020}, although the naturalistic corpus data do not always conform to these predictions and, for example, inter-speaker differences will need to be kept in mind (see also \citealt{Janse_Schreiber_InPrep}). Word order in Romeyka\il{Hellenic!Romeyka} is largely determined by \isi{information structure}, defined in terms of the concepts of \isi{topic} and \isi{focus}. A \isi{topic} of a clause is defined here broadly as old/given information, that is, ``an entity that has usually already been introduced into the discourse and is taken up again'' (\citealt{bagriacik_pharasiot_2018}: 114) and which is, if not already familiar to the hearer (or at least to the speaker), ``agreed on by the speakers'' (\citealt{solticLate2015}: 48, following \citealt{gundelfretheim2004}). A constituent is in \isi{focus} if it contains emphasized information which is generally assumed to be in this context new to the hearer (cf. ``information \isi{focus}'' in \citealt{bagriacik_pharasiot_2018}: 115). Both \isi{topic} and \isi{focus} can also yield contrastive information, thus called contrastive \isi{topic} and contrastive \isi{focus}. A contrastive \isi{topic} is ``an element that induces alternatives which have no impact on the \isi{focus} value and creates oppositional pairs with respect to other topics'' (\citealt{frascarellihinterholzl2007}: 88; cf. \citealt{bagriacik_pharasiot_2018}: 267).

\citet[105]{neocleous_word_2020} argues that there is a single subject position in Romeyka\il{Hellenic!Romeyka} main and subordinate clauses and that all subjects (no differentiation regarding \isi{specificity}) in pragmatically unmarked orders in Romeyka\il{Hellenic!Romeyka} are left-dislocated topics,{\footnotemark} appearing in the left-most clause position (\ref{Romeyka:ex:1}).

\ea\label{Romeyka:ex:1}
Romeyka \il{Hellenic!Romeyka}(\citealt{schreiber2021pontic}: B, 0238) \\
\gll [ena peðas]\textsubscript{TOP} ebidže havus \\
a boy make\textsc{.aor.3sg} pool \\
\glt `A boy made a pool.'  \\
\z

\footnotetext{We are aware that the term left-dislocation is associated with specific mechanisms in the generative literature (\citealt{kaltsaTopicalization2010}, \citealt{neocleous_word_2020,neocleousEvolution2022}), but since we are not working within a generative framework, we prefer to use ``\isi{topicalization}'' for preverbal topics which are still within the clause and ``\isi{topic} left-dislocation'' for preverbal topics which are prosodically detached from the nuclear clause and constitute a separate \isi{intonation} unit (\citealt{janseClitic2008}), ``backgrounding'' for postverbal topics within the clause and ``\isi{topic} right-dislocation'' for postverbal topics which constitute a separate \isi{intonation} unit (in accordance with \citealt[167--168]{janseClitic2008}).}

Both the pre-verbal and the post-verbal domain can host topics, but contrastive topics are only possible in pre-verbal position (example (\ref{Romeyka:ex:2}); \citealt{neocleous_word_2020}: 186). Definite \isi{object} topics, both pre-verbal (\citealt{neocleous_word_2020}: 128) and post-verbal (\citealt{schreiber2021pontic}), optionally trigger clitic doubling\is{clitic!doubling}, that is, the coindexation of a (pro)nominal \isi{object} \isi{topic} by a clitic pronoun\is{clitic!pronoun}\is{pronoun!clitic}. Clitic doubling of definite \isi{object} topics is an inherited Greek feature, which is obligatory in PG and other East Asia Minor Greek varieties such as Cappadocian\il{Hellenic!Cappadocian} and Pharasiot\il{Hellenic!Pharasiot} (\citealt{janseClitic2008}), but not systematically attested in the Romeyka\il{Hellenic!Romeyka} corpus. In PG (\citealt{drettasPontic1997}: 276--280), clitic doubling\is{clitic!doubling} occurs for nominal \isi{object} topics in left-dislocated position (but not when the topicalizer \textit{=ba(l)} occurs) by means of a referential resumptive \isi{pronoun} to distinguish a \isi{topic} from pre-verbal \isi{focus} since focalization never triggers clitic doubling\is{clitic!doubling}.\footnote{Compare \citeauthor{horrocksCambridge2019}' \citeyear{horrocksCambridge2019} rule [92] for the obligatory co-occurrence of clitic-doubling with dislocated and non-dislocated preverbal topics versus rule [93] for the absence of clitic-doubling with preverbal foci in Late Medieval and Early Modern Greek \citep[2024--2025]{horrocksCambridge2019}.} 

Romeyka\il{Hellenic!Romeyka} displays, like PG, an (arguably)\footnote{On the enclitic status of \textit{pa(l)} in PG, see \citeauthor{papadopoulos1955} (\citeyear{papadopoulos1955}: 119, \citeyear{papadopoulos1958-1961}: volume 2: 138), \citet{setatos1994}, \citet[46, 434]{drettasPontic1997}, \citet[225--226]{Janse2002Aspects}, \citet[131--132]{ralli2006syntactic}, \citet[263]{kaltsaTopicalization2010}. Compare the use of \textit{πάλιν} as an enclitic \isi{topic} marker in Late Medieval Greek (\citealt{solticLate2013}).} \isi{clitic} \isi{topicalization} particle \textit{=ba(l)} (3a/b), which according to \citet[120]{neocleous_word_2020} assigns contrastive \isi{topic} to the marked constituent, although the manifold functions of this particle outlined in \citet[141--143]{schreibergrammar2022} require more detailed investigation. The \isi{focus} position in Romeyka\il{Hellenic!Romeyka} (for both information and contrastive \isi{focus}) is immediately to the left of the verb (\ref{Romeyka:ex:4}) and can be filled by several constituents (\citealt{neocleous_word_2020}: 129); multiple \isi{focus} is possible resulting in movement of all focused constituents to pre-verbal position (\citealt{neocleous_word_2020}: 181).

In sum, according to \citet{neocleous_word_2020,neocleousEvolution2022}, the pragmatically neutral basic \isi{word order} in Romeyka\il{Hellenic!Romeyka} main clauses is (S)\isi{VO}, as illustrated in (\ref{Romeyka:ex:1}) above; if \isi{OV} order occurs in main clauses, this is argued to be due to either focalization or \isi{topicalization}. 


\newpage
\ea\label{Romeyka:ex:2}
Romeyka \il{Hellenic!Romeyka}(\citealt{schreibergrammar2022}: 247, ex. 75a, constructed example) \\
\gll [avudo to saχan]\textsubscript{TOPi} epero to\textsubscript{i} \\
this the plate take\textsc{.prs.1sg} \textsc{opn.cl.3sg} \\
\glt `I take this plate.'  \\
\z

\ea\label{Romeyka:ex:3}
\ea\label{Romeyka:ex:3a}
Romeyka \il{Hellenic!Romeyka}(\citealt{schreiber2021pontic}: D, 0390) \\
\gll ulin efteme \\
all make\textsc{.prs.1pl} \\
\glt `We make everything.'  \\
\ex\label{Romeyka:ex:3b}
Romeyka \il{Hellenic!Romeyka}(\citealt{schreiber2021pontic}: D, 0391) \\
\gll pikniki ba efteme \\
picnic \textsc{top} make\textsc{.prs.1pl} \\
\glt `We make a picnic.'
\z
\z

\ea\label{Romeyka:ex:4}
Romeyka \il{Hellenic!Romeyka}(\citealt{schreiber2021pontic}: B, 0243) \\
\gll [ta is]\textsubscript{foc} ebidže \\
the footprints make\textsc{.aor.3sg} \\
\glt `He left footprints.'  \\
\z

\section{Word order profile}\label{Romeyka:ss:4}

\subsection{Word order in the nominal phrase (NP)}\label{Romeyka:ss:4.1}

Word order in the NP is \isi{head-final} in Romeyka\il{Hellenic!Romeyka} and nominal modification is pre-nominal with the exception of enclitic genitive pronouns. Attributive adjectives precede the head noun (\ref{Romeyka:ex:5}),\footnote{While we apply in this chapter generally a very simplified glossing system that ignores some morphological information (and we also abstained from indicating word accent in the examples), we indicate case information only in Section \ref{Romeyka:ss:4.1} on NP \isi{word order}, but ignore nominal number and gender. It has to be noted that case \isi{flagging} vs. bare marking, as it is differentiated in the WOWA coding strategy, cannot be considered a reliable factor in Romeyka\il{Hellenic!Romeyka}, since nominative and \isi{accusative}(/\isi{oblique}) case endings are often reduced and it can be partly only inferred from the syntactic context which case is expected in a certain example. In the coding of the WOWA dataset, only those tokens are tagged as `case' which show a clear case ending, like \textsc{masc.sg.nom}. \textit{-os} and \textsc{masc.sg.acc}. \textit{-on}.} as do demonstratives (\ref{Romeyka:ex:6}) and numerals (\ref{Romeyka:ex:7}). Definite NPs trigger determiner spreading\is{determiner spreading} on attributive adjectives and in principle also on numerals,\footnote{Although the Romeyka\il{Hellenic!Romeyka} corpus shows several deviations (see \citealt{schreibergrammar2022}).} that is, the definite article occurs before each modifying element as well as before the head noun (e.g., examples \ref{Romeyka:ex:6}, \ref{Romeyka:ex:8}, \ref{Romeyka:ex:10}; for details on determiner spreading\is{determiner spreading} and nominal agreement in Romeyka\il{Hellenic!Romeyka} see \citealt{schreibergrammar2022}).

\ea\label{Romeyka:ex:5}
Romeyka \il{Hellenic!Romeyka}(\citealt{schreiber2021pontic}: A, 0042) \\
\gll geniše ðromo utš en \\
broad road\textsc{.nom} \textsc{neg} be\textsc{.prs.3sg} \\
\glt `There is no broad road.'
\z

\ea\label{Romeyka:ex:6}
Romeyka \il{Hellenic!Romeyka}(\citealt{schreiber_inprep}: 08\_04072019M\_3; 161) \\
\gll hatšino d omorfo don dobo \\
this\textsc{.acc} the nice\textsc{.acc} the\textsc{.acc} place\textsc{.acc} \\
\glt `this nice place'  \\
\z

\ea\label{Romeyka:ex:7}
Romeyka \il{Hellenic!Romeyka}(\citealt{schreiber2021pontic}: C, 0285) \\
\gll ðio nomade  \\
two persons\textsc{.nom} \\
\glt `two people'  \\
\z

\ea\label{Romeyka:ex:8}
Romeyka \il{Hellenic!Romeyka}(\citealt{schreiber_inprep}: 04\_01072019F\_2; 087) \\
\gll mo ta dœrt tane ta za \\
with the\textsc{.acc} four piece the\textsc{.acc} cows\textsc{.acc} \\
\glt `with the four cows'  \\
\z

Romeyka\il{Hellenic!Romeyka} has pre-nominal nominal genitives (both nouns and NPs), i.e., the possessor precedes the possessed (\ref{Romeyka:ex:9}, \ref{Romeyka:ex:10}).

\ea\label{Romeyka:ex:9}
Romeyka \il{Hellenic!Romeyka}(\citealt{schreiber2021pontic}: C, 0303) \\
\gll tu spid i arθob \\
the\textsc{.gen} house\textsc{.gen} the\textsc{.nom} people\textsc{.nom} \\
\glt `the people of the house'  \\
\z

\ea\label{Romeyka:ex:10}
Romeyka \il{Hellenic!Romeyka}(\citealt{schreiber_inprep}: 08\_04072019M\_2; 090--092) \\
\gll du dünja olon da tehlikelija da dobe  \\
the\textsc{.gen} world\textsc{.gen} all\textsc{.gen} the\textsc{.nom} dangerous\textsc{.nom} the\textsc{.nom} places\textsc{.nom} \\
\glt `the world's most dangerous places'  \\
\z

\begin{sloppypar}
As for pronominal possession, Romeyka\il{Hellenic!Romeyka} has pre-nominal full \isi{possessive} pronouns (\ref{Romeyka:ex:11}) but also post-nominal weak/enclitic \isi{possessive}(/genitive) pronouns (\ref{Romeyka:ex:12}).{\footnotemark} As with other pre-nominal modifiers in definite NPs, the head noun keeps its definite article when combined with a full (pre-nominal) \isi{possessive} \isi{pronoun} (\ref{Romeyka:ex:11}), which historically includes an incorporated definite article as well.
\end{sloppypar}


\ea\label{Romeyka:ex:11}
Romeyka \il{Hellenic!Romeyka}(\citealt{schreiber2021pontic}: A, 0129) \\
\gll temetero do barχari \\
\textsc{poss.1pl} the pasture \\
\glt `our pasture'
\z

\footnotetext{Throughout this chapter, we indicate \isi{clitic} status only for weak \isi{possessive} and object pronouns\is{pronoun!object}, and refrain from doing so for all other parts of speech that are traditionally considered clitics in Greek linguistics, such as definite articles, prepositions, relativizer, auxiliaries, negation and modal particles, and the coordinating conjunction \textit{tše} `and'.}

\ea\label{Romeyka:ex:12}
Romeyka \il{Hellenic!Romeyka}(\citealt{schreiber2021pontic}: E, 0564) \\
\gll andras ades \\
husband\textsc{.nom} \textsc{poss.cl.3sg}\\
\glt `her husband'  \\
\z

Relative clauses in Romeyka\il{Hellenic!Romeyka} are in principle pre-nominal (\ref{Romeyka:ex:13}), although post-nominal relative clauses exist as well (\ref{Romeyka:ex:14}). It is argued that pre-nominal relative clauses in Romeyka\il{Hellenic!Romeyka} have evolved under \isi{contact influence} from Turkish\il{Turkic!Turkish}, while post-nominal relative clauses are a Hellenic relic (\citealt{neocleous_word_2020}; \citealt{schreibergrammar2022}).

\ea\label{Romeyka:ex:13}
Romeyka \il{Hellenic!Romeyka}(\citealt{gandon2016}: 222, ex. 517, glosses modified) \\
\gll opse iða [alis p epiren] ineka \\
yesterday see\textsc{.aor.1sg} Alis\textsc{.nom} \textsc{rel} take\textsc{.aor.3sg} woman\textsc{.acc} \\
\glt `Yesterday I saw the woman who Ali married.'  \\
\z

\ea\label{Romeyka:ex:14}
Romeyka \il{Hellenic!Romeyka}(\citealt{neocleous_word_2020}: 71, ex. 87, presentation/glosses modified) \\
\gll o peðas [op erθen aso cicenin] temon t anepsin en \\
the child \textsc{rel} come\textsc{.aor.3sg} from.the grocery my the nephew be\textsc{.prs.3sg} \\
\glt `The child who came from the grocery's is my nephew.'  \\
\z

\subsection{Adpositional phrases}\label{Romeyka:ss:4.2}

Romeyka\il{Hellenic!Romeyka} is a prepositional language. Prepositional phrases express in Romeyka\il{Hellenic!Romeyka} semantic roles of location, \isi{Goal}, source/origin, \isi{instrument} or \isi{benefactive}, as well as some temporal (and other) adjuncts. However, these semantic categories are not exclusively flagged with prepositions. The overall rate of prepositional marking in the WOWA dataset for Romeyka\il{Hellenic!Romeyka} is 61\% (percentage calculated based on the occurrence of prepositions in the functions \isi{ablative}, \isi{addressee}, \isi{benefactive}, \isi{comitative}, (caused) \isi{Goal}, instrumental, \isi{locative}, \isi{recipient} and beneficiary). For differences in the frequency with which adpositional phrases occur after the verb, see the sections on the respective semantic roles below.

Finally, it must be noted that Romeyka\il{Hellenic!Romeyka} has a complex system of spatial orientation (see \citealt{schreibergrammar2022}) and some spatial adverbs could be potentially considered circumpositions (see also \citeauthor{karatsareas2016asia} \citeyear{karatsareas2016asia} work on circumpositions in Cappadocian\il{Hellenic!Cappadocian}), if they co-occur with a \isi{preposition}, although their status as bound elements is not clear (examples \ref{Romeyka:ex:15}, \ref{Romeyka:ex:16} vs. \ref{Romeyka:ex:17}).

\ea\label{Romeyka:ex:15}
Romeyka \il{Hellenic!Romeyka}(\citealt{schreiber_inprep}: 05\_03072019M\_3; 29) \\
\gll s ena sergi eban \\
on a blanket above \\
\glt `on (top of) a blanket'
\z

\ea\label{Romeyka:ex:16}
Romeyka \il{Hellenic!Romeyka}(\citealt{schreiber_inprep}: 01\_28062019F\_3; 42--43) \\
\gll s oros apes-merea \\
at.the forest inside-somewhere \\
\glt `inside the forest'  \\
\z

\ea\label{Romeyka:ex:17}
Romeyka \il{Hellenic!Romeyka}(\citealt{schreiber_inprep}: 01\_04022016F\_1; 052) \\
\gll eb-ebuka asi ɣorɣoran \\
from-under from.the Gorgoras\textsc{.acc} \\
\glt `from lower Gorgoras' [as differentiated from upper Gorgoras]  \\
\z

\subsection{Ordering of spatial expressions relative to the verb}\label{Romeyka:ss:4.3}

\subsubsection{Locations}\label{Romeyka:ss:4.3.1}

The \isi{preposition} indicating location (as well as \isi{Goal}/direction) is \textit{s} `to, at', which can merge with determiners and object pronouns\is{pronoun!object}, e.g., \textit{s }`to, at' + \textit{to}\textsc{.det.acc.sg} > \textit{so} `to the' (\ref{Romeyka:ex:18}), \textit{s }`to, at' + \textit{emasuna}\textsc{.opn.1pl} > \textit{semasuna} `to us'. The overall frequency of post-verbal locations (including PPs and spatial adverbials) is 42\%, as in (\ref{Romeyka:ex:19}), the majority of locations is pre-verbal. There is no significant statistical correlation found between the position and the independent variables \isi{animacy}, \isi{weight} or \isi{flagging} (i.e., any overt phonological marking of case). However, there is some \isi{inter-speaker variation} to be observed (see \tabref{Romeyka:tab:3} in Section \ref{Romeyka:ss:6} below). The position of spatial adjuncts is sensitive to discourse. The immediate pre-verbal position is argued to be the (information) \isi{focus} position (\citealt{neocleous_word_2020}: 132, 148). In this vein, as exemplified in (\ref{Romeyka:ex:19}), multiple dislocation of (place) constituents is possible in Romeyka\il{Hellenic!Romeyka}, as is the case for SMG (\citealt{alexiadou1997Adverb}: 58).

\ea\label{Romeyka:ex:18} 	
Romeyka \il{Hellenic!Romeyka}(\citealt{schreiber2021pontic}: A, 0187) \\
\gll eskiden gœl\textsubscript{FOC} ebiname so bodami\textsubscript{TOP} \\
formerly lake make\textsc{.ipf.1pl} at.the valley \\
\glt `In earlier times, we made a lake at the valley.'
\z

\ea\label{Romeyka:ex:19}
Romeyka \il{Hellenic!Romeyka}(\citealt{schreiber2021pontic}: A, 0001) \\
\gll emistine\textsubscript{TOP} [aða sin otšena]\textsubscript{FOC} jašaevume \\
we here at.the Ogene live\textsc{.prs.1pl} \\
\glt 	`We live here at Ogene.'  \\
\z

\subsubsection{Goals}\label{Romeyka:ss:4.3.2}

For the discussion of goals in Romeyka\il{Hellenic!Romeyka}, only goals of motion verbs and verbs of caused motion are considered. Goals in the WOWA dataset are predominantly expressed by prepositional phrases headed by the prepositions \textit{s} `to' and \textit{os} `until', but also adverbially. 78\% of goals (both prepositional and adverbial) are post-verbal (\ref{Romeyka:ex:20}), which shows that goals are much more likely to be post-verbal than locations. While \isi{information structure} affects \isi{word order} variation in spatial adjuncts in Romeyka\il{Hellenic!Romeyka}, this is not likely to be the only factor determining \isi{word order} in light of the significantly higher number of post-verbal Goal\is{Goal!post-verbal}s; see \citetv{chapters/7_RasekhMahandetal_Persian} on spoken Persian for a similar observation. However, in light of an assumed \isi{VO}/VX order in declarative clauses in Romeyka\il{Hellenic!Romeyka} (\citealt{neocleous_word_2020}) --- as in other dialects of modern Greek --- it is rather the high number of pre-verbal locations and sources (see Sections \ref{Romeyka:ss:4.3.1} and \ref{Romeyka:ss:4.3.3}) that is striking and requires explanation. Comparing (\ref{Romeyka:ex:20}) with (\ref{Romeyka:ex:21}), \isi{information structure} accounts for the pre-verbal position of the PP in (\ref{Romeyka:ex:21}). Otherwise, no statistically significant correlation is found between position and the dependent variables \isi{animacy}, \isi{weight}, and \isi{flagging}, although adverbial goals have a higher likelihood to be post-verbal than PPs. Finally, it has to be noted that there is high \isi{inter-speaker variation} (see \tabref{Romeyka:tab:4} in Section \ref{Romeyka:ss:6} below).

\ea\label{Romeyka:ex:20}
Romeyka \il{Hellenic!Romeyka}(\citealt{schreiber2021pontic}: A, 0175) \\
\gll pao so raši \\
go\textsc{.prs.1sg} to.the mountain \\
\glt `I go to the mountain.'  \\
\z

\ea\label{Romeyka:ex:21}
Romeyka \il{Hellenic!Romeyka}(\citealt{schreiber2021pontic}: A, 0136) \\
\gll [son barχari muna]\textsubscript{TOP} [direk araba]\textsubscript{FOC} bai \\
to.the pasture \textsc{poss.cl.1pl} direct car go\textsc{.prs.3sg} \\
\glt 	`A bus goes directly to our pasture.'  \\
\z

As for caused goals in the WOWA dataset, 100\% are post-verbal (\ref{Romeyka:ex:22}), but note that due to very low token numbers (N=4) in the generally small WOWA dataset for Romeyka\il{Hellenic!Romeyka} this information is only tentative and needs further investigation. 

\ea\label{Romeyka:ex:22}
Romeyka \il{Hellenic!Romeyka}(\citealt{schreiber2021pontic}: D, 0488) \\
\gll eferenam son barχari ksila \\
bring\textsc{.ipf.1pl} to.the pasture wood \\
\glt `We brought wood to the pasture.'  \\
\z

\subsubsection{Sources}\label{Romeyka:ss:4.3.3}

The majority of sources in the WOWA dataset, i.e., 80\%, are pre-verbal, including PPs with the \isi{preposition} \textit{as} `from' and adverbials. This result needs to be taken with some caution, though, since nearly all examples stem from a single speaker. Information structure is likely to account for the position of the source, cf. ex. (\ref{Romeyka:ex:23a}) in assumed historically unmarked VX position vs. (\ref{Romeyka:ex:23b}) in pre-verbal contrastive \isi{focus} position where the PP contrasts with the PP in (\ref{Romeyka:ex:23a}), i.e., the previous sentence in the same recording.

\ea\label{Romeyka:ex:23}
\ea\label{Romeyka:ex:23a}
Romeyka \il{Hellenic!Romeyka}(\citealt{schreiber2021pontic}: A, 0024--0025) \\
\gll bazen para berename edroɣame asa bakale \\
sometimes money take\textsc{.ipf.1pl} eat\textsc{.ipf.1pl} from.the shops \\
\glt `Sometimes we took money, we ate [food] from the shops.'  \\
\ex\label{Romeyka:ex:23b}
Romeyka \il{Hellenic!Romeyka}(\citealt{schreiber2021pontic}: A, 0026) \\
\gll bazen aso spidi eberenam tš ebejname \\
sometimes from.the house take\textsc{.ipf.1pl} and go\textsc{.ipf.1pl} \\
\glt `Sometimes we took [food] from the house.' 
\z
\z


\subsection{Ordering of direct objects relative to the verb}\label{Romeyka:ss:4.4}

\subsubsection{Nominal direct object}\label{Romeyka:ss:4.4.1}

\begin{sloppypar}
This section focuses on direct objects (DOs) of transitive verbs. In the WOWA dataset, 66\% of nominal direct objects are post-verbal (arguments of `have'/existentials, which are predominantly pre-verbal, not included in this count). The percentage appears to be the same for definite and indefinite nominal DOs, which suggests that \isi{definiteness} does not play a \isi{role}. Note, however, that in Cappadocian\il{Hellenic!Cappadocian}, \isi{word order} is sensitive to \isi{definiteness}, that is, indefinite \isi{object} NPs tend to occur in post-verbal position (\citealt{janseObject2006}). The difference in \isi{word order} between examples (\ref{Romeyka:ex:24}) and (\ref{Romeyka:ex:25}) is probably rather than in \isi{definiteness} to be found in \isi{information structure}, with the \isi{object} NP in (\ref{Romeyka:ex:24}) in \isi{focus} position and the \isi{object} NP in (\ref{Romeyka:ex:25}) in \isi{topic} position. However, there may be also an interference from the frequent occurrence of post-verbal nominal \isi{object} NPs in Trabzon Turkish\il{Turkic!Turkish Trabzon} (\citealt{schreiberMuslim}), a phenomenon described for informal spoken Standard Turkish\il{Turkic!Turkish} as ``backgrounding'', and which applies usually to definite NPs, although a non-definite NP can be placed in spoken Turkish\il{Turkic!Turkish} in the post-verbal position ``if it refers to an entity or category that has been mentioned (or implied) in the immediately preceding discourse'' (\citealt{goksel2005turkish}: 346).
\end{sloppypar}

Numerically, more complex NPs/PPs in the WOWA dataset, i.e., those with more than two words usually involving some kind of nominal modification or genitives, tend to be pre-verbal (\ref{Romeyka:ex:26}). Although this is contrary to the traditional assumption that very complex NPs/PPs preferably appear at the end of an utterance (see, e.g., \citealt{behaghel_beziehungen_1909} departing from German), it does align with similar findings from the spoken-language corpora investigated in this volume (see, e.g., \citetv{chapters/7_RasekhMahandetal_Persian} and \citetv{chapters/14_Leitner_Khuzistani}). However, it is likely that \isi{information structure} is the more decisive factor, which might just happen to overlap with the factor of ``\isi{weight}'' (see, e.g., \ref{Romeyka:ex:26}).

A significant correlation was found between the dependent variable position and the independent variable \isi{animacy}: human and animate nominal DOs are slightly more likely to be pre-verbal than inanimate nominal DOs. Interestingly, a reverse significant effect was found for pronominal human and animate DOs, which tend to be more likely post-verbal than inanimate pronominal DOs (see Section \ref{Romeyka:ss:4.4.2}). As is the case for all semantic categories, the order of nominal DOs with regard to the verb is sensitive to \isi{information structure} (see Section \ref{Romeyka:ss:3}) and \isi{inter-speaker variation} is relevant here as well (see \tabref{Romeyka:tab:2} in Section \ref{Romeyka:ss:6}). Finally, it also needs to be stated that the linear order of nominal objects with regard to the verb seems to be at times within the same speaker in free variation, even if other variables remain stable (e.g., \ref{Romeyka:ex:27} vs. \ref{Romeyka:ex:28}), i.e., there is \isi{intra-speaker variation}. This needs to be most likely attributed to ongoing \isi{language shift} under strong influence from Turkish\il{Turkic!Turkish} \isi{OV} orders, as \isi{information structure} cannot always be convincingly invoked as explanation for all \isi{OV} orders.

\ea\label{Romeyka:ex:24}
Romeyka \il{Hellenic!Romeyka}(\citealt{schreiber2021pontic}: E, 0564) \\
\gll andras ades ba ksila gofdi \\
husband \textsc{poss.cl.3sg} \textsc{top} wood cut\textsc{.prs.3sg} \\
\glt 	`Her husband cuts wood.'  \\
\z

\newpage

\ea\label{Romeyka:ex:25}
Romeyka \il{Hellenic!Romeyka}(\citealt{schreiber2021pontic}: E, 0572) \\
\gll efidže so ðormo ban ena ksilo \\
leave\textsc{.aor.3sg} at.the road above a wood \\
\glt `He lost a log on the street.'  \\
\z

\ea\label{Romeyka:ex:26}	
Romeyka \il{Hellenic!Romeyka}(\citealt{schreiber2021pontic}: A, 0144--0145) \\
\gll ifedi eɣo ebiɣa išde muskaræ utš dane muskar ebiɣa \\
last\_year I buy\textsc{.aor.1sg} \textsc{dp} calves three piece calve buy\textsc{.aor.1sg} \\
\glt `Last year I bought calves. I bought three calves.' 
\z

\ea\label{Romeyka:ex:27}
Romeyka \il{Hellenic!Romeyka}(\citealt{schreiber2021pontic}: A, 0073) \\
\gll efteme fasulijas \\
do\textsc{.prs.1pl} beans \\
\glt `We do beans.' [i.e., `We grow beans.']  \\
\z

\ea\label{Romeyka:ex:28} 	
Romeyka \il{Hellenic!Romeyka}(\citealt{schreiber2021pontic}: A, 0099) \\
\gll opsar ebsame \\
fish catch\textsc{.aor.1pl} \\
\glt `We fished.'
\z

\subsubsection{Pronominal direct objects}\label{Romeyka:ss:4.4.2}

The percentage of post-verbal pronominal direct objects in the WOWA dataset is with 58\% slightly lower than that of nominal DOs. Importantly, (en)\isi{clitic} object pronouns\is{pronoun!object} are not coded in the corpus since they are bound. Romeyka\il{Hellenic!Romeyka} has both \isi{clitic} object pronouns\is{pronoun!object} following the predicate and free object pronouns\is{pronoun!object} preceding the predicate. The two are mainly differentiated based on the criterion of stress, whereby enclitic object pronouns\is{pronoun!object} have a reduced phonological form and do not impact the stress pattern of the verb. However, in the Romeyka\il{Hellenic!Romeyka} corpus (\citealt{schreiber_inprep}) there appear to be also post-verbal full object pronouns\is{pronoun!object}, which are sometimes difficult to differentiate from weak ones and which are probably reinforced by contact with Turkish\il{Turkic!Turkish} (see below; see also \citealt{schreibergrammar2022}: 103). In the WOWA dataset, the percentage of 58\% post-predicate pronominal DOs only refers to full pronominal forms, the vast majority being third person pronouns. While none of the coded categories like \isi{weight} or \isi{flagging} is significant for the position of pronominal DOs, there is a significant tendency for pronominal DOs denoting human or animate entities to appear post-verbally as opposed to inanimate pronominal DOs (\ref{Romeyka:ex:29} vs. \ref{Romeyka:ex:30}). However, it is obvious from the examples, that \isi{information structure} plays a significant \isi{role} here as well (see \ref{Romeyka:ex:29}). Still, it is unclear whether 42\% of pre-verbal pronominal DOs are largely due to \isi{information structure} since pronouns are usually given information which can occur in pre- and post-verbal position (see Section \ref{Romeyka:ss:3}).

\ea\label{Romeyka:ex:29}
Romeyka \il{Hellenic!Romeyka}(\citealt{schreiber2021pontic}: D, 0408--0409) \\
\gll laɣo dune ado utš eksero \\
how be\textsc{.ipf.3sg} this \textsc{neg} know\textsc{.prs.1sg} \\
\glt 	`How it was? I don't know this.'  \\
\z

\ea\label{Romeyka:ex:30}
Romeyka \il{Hellenic!Romeyka}(\citealt{schreiber2021pontic}: B, 0246) \\
\gll utš iðan adona \\
\textsc{neg} see\textsc{.aor.3pl} \textsc{opn.3sg} \\
\glt `They did not see him [=the bear].'  \\
\z

An important issue is the post-verbal placement of full pronominal objects (often third person pronouns) versus weak enclitic object pronouns\is{pronoun!object}. In the WOWA dataset, 60\% of pronominal DOs are weak enclitic pronouns (thus not coded for position), 40\% are full pronominal forms out of which 58\% are post-verbal. In comparison, in the larger Romeyka\il{Hellenic!Romeyka} corpus, 63\% of all pronominal DOs are \isi{clitic} pronouns (\citealt{schreibergrammar2022}: 100, Table 12). In (\ref{Romeyka:ex:30}), a weak \isi{object} \isi{pronoun} would be likely to occur since `the bear' is a \isi{topic} which has been mentioned several times in the preceding context. Instead, the full pronominal form is placed in post-predicate (enclitic) position. As another example, in (\ref{Romeyka:ex:31}), the full pronominal form in post-verbal position is preferred over the weak pronominal form \textit{=(a)ta}. It is possible that a contact explanation can account for the preference of post-posed full object pronouns\is{pronoun!object} in Romeyka\il{Hellenic!Romeyka}, as (Trabzon) Turkish\il{Turkic!Turkish Trabzon} has no \isi{clitic} object pronouns\is{pronoun!object} and object pronouns\is{pronoun!object} can appear post-verbally in informal spoken Turkish (also cf. only 13\% post-verbal pronominal DOs in \citealt{hodgson_pontic_nodate}). According to \citet[30]{brendemoen2005}, post-verbal pronominal DOs in Trabzon Turkish\il{Turkic!Turkish Trabzon} have arisen due to contact with PG. The potential mutual influence suggests a convergence-type of change between Romeyka\il{Hellenic!Romeyka} and Trabzon Turkish\il{Turkic!Turkish Trabzon}.

\ea\label{Romeyka:ex:31}
Romeyka \il{Hellenic!Romeyka}(\citealt{schreiber2021pontic}: B, 0234) \\
\gll ama utš eboresane dosin adonusine \\
but \textsc{neg} can\textsc{.aor.3pl} give\textsc{.inf} \textsc{opn.3pl} \\
\glt `But they could not hit them.'  \\
\z

In sum, \isi{VO} is assumed to be the unmarked order of pronominal DO and verb in Romeyka\il{Hellenic!Romeyka}, whereby \isi{OV} orders are motivated by \isi{information structure}. The remaining question is what motivates the post-verbal position of full object pronouns\is{pronoun!object}. In terms of \isi{information structure}, pronominal \isi{VO} with full object pronouns\is{pronoun!object} can only be explained, if their information structural value is comparable to that of weak \isi{clitic} pronouns, that is ``familiar topics''. If the post-verbal position of full pronominal DOs in the Romeyka\il{Hellenic!Romeyka} corpus is split up according to person and number, it is evident that third person singular full object pronouns\is{pronoun!object} are most likely to be post-verbal, although strikingly third person plural full object pronouns\is{pronoun!object} have the least chance to be post-predicate (unlike ex. \ref{Romeyka:ex:31}).\footnote{Distribution of bound (weak) pronominal object pronouns\is{pronoun!object} in the Romeyka\il{Hellenic!Romeyka} corpus (\citealt{schreiber_inprep}) per person and number: \textsc{1sg}: 41 bound out of 67, i.e., 61\% bound; \textsc{2sg}: 18 out of 28, i.e., 64\% bound; \textsc{3sg}: 72 out of 133; i.e., 54\% bound; \textsc{1pl}: 17 out of 26, i.e., 65\% bound; \textsc{2pl}: 6 out of 10, i.e., 60\% bound; \textsc{3pl}: 63 out of 79, i.e., 80\% bound.} So probably, gender could play a \isi{role} in the sense that gender cannot be differentiated in weak third person singular object pronouns\is{pronoun!object} but does reflect in strong third person object pronouns\is{pronoun!object}. Furthermore, it is not clear whether phonological \isi{weight} could also play a \isi{role}, although \textsc{3sg} strong pronominal forms are not significantly shorter (i.e., more ``clitic-like'') than \textsc{3pl} object pronouns\is{pronoun!object}.

\subsection{Ordering of other obliques relative to the verb}\label{Romeyka:ss:4.5}

Other obliques refer to the semantic roles of recipients, addressees, comitatives, instruments, benefactives, and others. In the WOWA dataset, these categories do not figure prominently, so any quantitative analysis is pointless. Instead, the present section discusses some examples for each semantic category.

The WOWA dataset contains only two tokens for recipients, both of which are pronominal. Both pronominal recipients occur pre-verbally (\ref{Romeyka:ex:32}; but cf. \ref{Romeyka:ex:35}). In general, the unmarked \isi{word order} for nominal recipients is \isi{VO} (\ref{Romeyka:ex:33}) with \isi{OV} orders triggered by \isi{information structure}, namely \isi{focus} (\ref{Romeyka:ex:34}; see \citealt{schreibergrammar2022}: 249--250). In ditransitive constructions, the unmarked \isi{word order} is V--IO--DO (\citealt{schreibergrammar2022}: 249). This applies also to pronominal recipients (see (\ref{Romeyka:ex:35}) and unlike (\ref{Romeyka:ex:32}) with a topicalized \isi{object} \isi{pronoun}). 

\ea\label{Romeyka:ex:32}
Romeyka \il{Hellenic!Romeyka}(\citealt{schreiber2021pontic}: A, 0182) \\
\gll emenan ndona na ðiɣune \\
\textsc{opn.1sg} what \textsc{prt} give\textsc{.prs.3pl} \\
\glt `What do they give to me?'  \\
\z

\ea\label{Romeyka:ex:33}
Romeyka \il{Hellenic!Romeyka}(\citealt{schreibergrammar2022}: 250, ex. 93, questionnaire data) \\
\gll ta mila ðokan ti mana tuna \\
the apples give\textsc{.aor.3pl} the mother \textsc{poss.cl.3pl} \\
\glt `They gave the apples to their mother.'  \\
\z

\ea\label{Romeyka:ex:34}
Romeyka \il{Hellenic!Romeyka}(\citealt{schreibergrammar2022}: 250, ex. 96, questionnaire data) \\
\gll din batsi eðotše ena ido ..sturatši \\
the girl give\textsc{.aor.3sg} a \textsc{dem} ..stick \\
\glt `He gave a stick to the girl.'  \\
\z

\ea\label{Romeyka:ex:35}
Romeyka \il{Hellenic!Romeyka}(\citealt{schreibergrammar2022}: 251, ex. 102, questionnaire data) \\
\gll etšine bal ðotš emena milo \\
he \textsc{top} give\textsc{.aor.3sg} \textsc{opn.1sg} apple \\
\glt `He gave me an apple.'  \\
\z

In the WOWA dataset, all three tokens of addressees are pronominal and appear post-verbally (\ref{Romeyka:ex:36}, the stress pattern \textit{ípen ádona} reveals that the \isi{pronoun} must be indeed the post-posed strong form and not a \isi{clitic}). In general, like with other obliques, nominal addressees are expected to follow the verb in unmarked \isi{word order}, although they can move to pre-verbal \isi{focus} position (note that the Romeyka corpus does not feature an example of a nominal \isi{addressee}, but cf. ex. (\ref{Romeyka:ex:37}) with \textit{ta patsiðes} `the girls' as contrastive \isi{topic}). Pronominal addressees, seem to occur nearly in all cases (of the Romeyka\il{Hellenic!Romeyka} corpus) post-verbally.

\ea\label{Romeyka:ex:36}
Romeyka \il{Hellenic!Romeyka}(\citealt{schreiber2021pontic}: E, 0582) \\
\gll iben adona \\
say\textsc{.aor.3sg} \textsc{opn.3sg} \\
\glt `He told her […].'
\z

\ea\label{Romeyka:ex:37}
Romeyka \il{Hellenic!Romeyka}(\citealt{schreiber_inprep}: 02\_02022015F\_1; 073--074) \\
\gll eleɣane ištera ta peðia kopela ta patsiðes eleɣane ɣospiðes \\
say\textsc{.ipf.3pl} later the boys girl the girls say\textsc{.ipf.3pl} prostitutes \\
\glt `Then they said to the boys girl, to the girls they said prostitutes.'  \\
\z

Comitatives, i.e., referents denoting accompanying persons (or at least animate entities), occur in the WOWA dataset predominantly in post-verbal position (\ref{Romeyka:ex:38a}), although pre-verbal placement is possible as well (see the PP as pre-posed given \isi{topic} in (\ref{Romeyka:ex:38b}), although stylistic variation seems to play a \isi{role} here as well).

\ea\label{Romeyka:ex:38}
\ea\label{Romeyka:ex:38a}
Romeyka \il{Hellenic!Romeyka}(\citealt{schreiber2021pontic}: A, 0060--0061) \\
\gll jaja ebejnane me ta za \\
on\_foot go\textsc{.ipf.3pl} with the cows \\
\ex\label{Romeyka:ex:38b}
\gll me ta za jaja ebejnane \\
with the cows on\_foot go\textsc{.ipf.3pl} \\
\glt `They went on foot with the cows.'  \\
\z
\z

Instruments appear in the WOWA dataset in the vast majority pre-verbally, which may be explained by focalization of new information (\ref{Romeyka:ex:39}) or contrastive \isi{focus}. However, based on the data of the Romeyka\il{Hellenic!Romeyka} corpus, it seems that instruments do not differ from other obliques in terms of unmarked \isi{VO} order (\ref{Romeyka:ex:40}). Ex. (\ref{Romeyka:ex:41}) shows that for information structural reasons both nominal obliques and objects can occur pre-verbally, i.e., Romeyka\il{Hellenic!Romeyka} allows for multiple \isi{focus}, which is argued to be ``order-preserving'' (\citealt{neocleous_word_2020}: 181--182). However, then it is not clear why the \isi{oblique} precedes the DO in (\ref{Romeyka:ex:29}). 

\ea\label{Romeyka:ex:39}
Romeyka \il{Hellenic!Romeyka}(\citealt{schreiber2021pontic}: E, 0570) \\
\gll mo d aksinari ešgise da ksila \\
with the axe split\textsc{.aor.3sg} the woods \\
\glt `He split the wood with the axe.'  \\
\z

\ea\label{Romeyka:ex:40}
Romeyka \il{Hellenic!Romeyka}(\citealt{schreiber_inprep}: 09\_04072019\_7; 11) \\
\gll ekoftame me ti kerenti \\
cut\textsc{.ipf.1pl} with the scythe \\
\glt `We cut with the scythe.'  \\
\z

\ea\label{Romeyka:ex:41}
Romeyka \il{Hellenic!Romeyka}(\citealt{schreiber2021pontic}: A, 0036) \\
\gll ula me ta rašes ta ɣomare ekovalename  \\
all with the pannier the loads carry\textsc{.ipf.1pl} \\
\glt `We always carried the loads with the panniers.'  \\
\z

Benefactives refer to situations where X does something in the interest of Y, generally implying that Y is a sentient being. Although no benefactives appear in the WOWA dataset, they are realized in Romeyka\il{Hellenic!Romeyka} by means of the \isi{preposition} \textit{ja(t)} `for' (\ref{Romeyka:ex:42}). Their position seems not to differ from that of comitatives, which are prepositional phrases as well, and which appear often in the pre-verbal domain for information structural reasons, although their unmarked \isi{word order} is post-verbal.

\ea\label{Romeyka:ex:42}
Romeyka \il{Hellenic!Romeyka}(\citealt{schreibergrammar2022}: 147, ex. 565, questionnaire data) \\
\gll sade jad emena faji utš eθelisa pseθinimo \\
only for \textsc{opn.1sg} food \textsc{neg} want\textsc{.aor.1sg} cook\textsc{.nmz} \\
\glt `I did not want to cook just for myself.'  \\
\z

Among other obliques in the WOWA dataset, i.e., those which are none of the semantic roles above, 57\% are in post-verbal position, which equals roughly the overall percentage of post-verbal placement in the WOWA dataset which is 55\%. Since nearly all of these other obliques are prepositional phrases, they follow the constraints outlined for other PPs above.

\subsection{Auxiliaries}\label{Romeyka:ss:4.6}

\citet{neocleousNever2022} discuss the following auxiliaries in Romeyka\il{Hellenic!Romeyka}: 
(i) \textit{ime} (+ particle), 
(ii) \textit{iχa} (+ infinitive), 
(iii) \textit{eš(i)} (+ finite verb in present tense or imperfective past). 
They always precede the main verb, irrespective of whether it is finite, infinitive or particle. Additionally, there is a periphrastic progressive construction with \textit{steko/stekome} `stand' or \textit{kahome} `sit.'

The invariable form \textit{eš(i)} plus finite verb is used as a periphrastic progressive denoting processes that are close to completion and goes most likely back to the \textsc{3sg} present tense form of the verb \textit{eχo} > \textit{eš(i)} (ex. \ref{Romeyka:ex:43}; cf. \citealt{drettasPontic1997}: 334 on a related progressive form in PG).\footnote{According to M. Bagriacik (p.c.), the form \textit{eš} goes probably back to a homonymous existential \isi{auxiliary} already existing in older stages of Greek, rather than to the `have'-\isi{auxiliary}, as the former presents a more likely grammaticalization pathway for a progressive form.} The inflected imperfective forms of the verb \textit{eχo} `have' are used as an \isi{auxiliary} in the formation of counterfactual conditional clauses (\ref{Romeyka:ex:44}), which are formed by one of the modal particles \textit{na/an/as} + inflected imperfective of \textit{eχo} + non-finite verb/inflected infinitive (\citealt{schreibergrammar2022}: 298, Table 37; see also \citealt{sitaridouModality2014}).

\ea\label{Romeyka:ex:43}
Romeyka \il{Hellenic!Romeyka}(\citealt{schreiber2021pontic}: B, 0241) \\
\gll eš erde argo \\
\textsc{aux} come\textsc{.aor.3sg} bear \\
\glt 	`The bear is/was? coming.'  \\
\z

\ea\label{Romeyka:ex:44}
Romeyka \il{Hellenic!Romeyka}(\citealt{schreiber_inprep}: 04\_01072019F\_13; 53) \\
\gll eɣo na m iχa škisen da\textup{\footnotemark} da ksila […] \\
I \textsc{prt} \textsc{neg} \textsc{aux} split\textsc{.inf?} \textsc{opn.cl.3pl} the woods \\
\glt `If I had not chopped the wood, […].'  \\
\z

\footnotetext{The analysis of the verb form \textit{škis-en=da} is not clear, we analyse \textit{=ta} here as a weak \isi{object} \isi{pronoun} but it also resembles the suffix \textit{-ta} marking gerunds like \textit{jelaχ-ta} `laughing'. The form \textit{škisen-} could be potentially a reduced infinitive, cf. \textit{škisini}\textsc{.inf}.}

With the limited use of the \isi{auxiliary} \textit{iχa} (morphologically homonymous with imperfective \textsc{1sg} form of \textit{eχo} suggesting two functions of \textit{eχo} as lexical verb and \isi{auxiliary}), Romeyka\il{Hellenic!Romeyka} stands out from other varieties of modern Greek like SMG. Its closest relative Cappadocian\il{Hellenic!Cappadocian} has a pluperfect with impersonal \textit{iton/itan} (homonymous with the 3rd person aorist form of \textit{ime}) preceded by a finite main verb. Diachronically AMG also uses the \isi{auxiliary} \textit{ime}, morphologically homonymous with the \isi{copula} \textit{ime} `be' (i.e., \textit{ime} can thus function both as existential and \isi{auxiliary}). For the \isi{word order} properties of \isi{copula} and `become' complements in Romeyka\il{Hellenic!Romeyka}, see Section \ref{Romeyka:ss:4.7.2}.

In both cases where forms of \textit{eχo} are used as an \isi{auxiliary} in Romeyka\il{Hellenic!Romeyka}, the \isi{auxiliary} precedes the main verb, although importantly, in the counterfactual conditionals the verb following the \isi{auxiliary} is not (i.e., in the case of the infinitive) or only partly (i.e., in the case of the ``inflected infinitive'', see \citealt{schreibergrammar2022}: 229--233 going back to \citealt{sitaridouModality2014}) inflected. \citet[269]{neocleous_word_2020} confirms AuxV order in Romeyka\il{Hellenic!Romeyka} main and subordinate clauses. The construction Vfin + \textit{iton} in pluperfects which exists in Cappadocian\il{Hellenic!Cappadocian} is not attested for Romeyka\il{Hellenic!Romeyka}.

\subsection{Complementation}\label{Romeyka:ss:4.7}

\subsubsection{Complement and adjunct clauses}\label{Romeyka:ss:4.7.1}

In Romeyka\il{Hellenic!Romeyka} both finite and non-finite complementation exist. Depending on the type of predicate, a complementizer is used to introduce the complement clause\is{complement!clause} (CC), while other clause types do not require a complementizer. In \citet[ 278]{schreibergrammar2022}, it has been argued that more than one complementation strategy is available for some clause types due to \isi{contact influence} from Turkish\il{Turkic!Turkish}, resulting in an increase of non-finite complementation strategies. While the non-finite strategy of using infinitives in some CCs is an archaic trait of Romeyka\il{Hellenic!Romeyka}, non-finite deverbal nouns as a complementation strategy have increased under contact with Turkish\il{Turkic!Turkish}. Within the finite complementation strategies, complementation by means of the complementizer \textit{na} is more restricted in Romeyka\il{Hellenic!Romeyka} compared to SMG, while juxtaposition with paratactic syntax and without complementizer is widespread (\ref{Romeyka:ex:45}), especially with verbs of saying and in (in)direct speech. 

\ea\label{Romeyka:ex:45}
Romeyka \il{Hellenic!Romeyka}(\citealt{schreiber_inprep}: 02\_2906019F\_1; 02) \\
\gll egusame [o jaja evren arkon] \\
hear\textsc{.aor.1pl} the Yahya find\textsc{.aor.3sg} bear \\
\glt `We heard that Yahya has found a bear.'  \\
\z

In Romeyka\il{Hellenic!Romeyka} CCs, the complement clause\is{complement!clause} predominantly follows the matrix verb (\ref{Romeyka:ex:45}). For some predicate types, among which are verbs of saying, the reverse order is possible as well, see for example the preverbal headless relative clause in (\ref{Romeyka:ex:46}); even circum-positions exist (\ref{Romeyka:ex:47}).

\ea\label{Romeyka:ex:46}
Romeyka \il{Hellenic!Romeyka}(\citealt{schreiber_inprep}: 01\_28062019F\_3; 24) \\
\gll [to leɣo] utš eɣrigo \\
what say\textsc{.prs.1sg} \textsc{neg} understand\textsc{.prs.1sg} \\
\glt `I don't understand what I say.'  \\
\z

\ea\label{Romeyka:ex:47}
Romeyka \il{Hellenic!Romeyka}(\citealt{schreiber_inprep}: 08\_04072019M\_2; 066) \\
\gll [t aleɣo] kseris [dohna e] \\
the horse know\textsc{.prs.2sg} what be\textsc{.prs.3sg} \\
\glt `Do you know what ``aleɣo'' is?'
\z

\textit{Na}-clauses and infinitives obligatorily follow the main verb (\ref{Romeyka:ex:48}), (\ref{Romeyka:ex:49}). However, deverbal nouns calquing Turkish\il{Turkic!Turkish} \isi{complement} clauses appear before the main verb but they are strictly speaking NPs and thus no actual CCs (\ref{Romeyka:ex:50}). Nominalizations selected by some aspectual verbs like \textit{bašlaevo} `start' which requires a PP follow the main verb (\ref{Romeyka:ex:51}). If a complementizer is used, it appears at the beginning of the CC.

\ea\label{Romeyka:ex:48} 	
Romeyka \il{Hellenic!Romeyka}(\citealt{schreiber_inprep}: 02\_02022015F\_1; 014) \\
\gll utš eθelena [n andriza] \\
\textsc{neg} want\textsc{.ipf.1sg} \textsc{prt} marry\textsc{.ipf.1sg} \\
\glt `I didn't want to marry.'
\z

\ea\label{Romeyka:ex:49} 
Romeyka \il{Hellenic!Romeyka}(\citealt{schreibergrammar2022}: 285, ex. 376, questionnaire data) \\
\gll utš eboresa [tšimeθina] \\
\textsc{neg} could\textsc{.aor.1sg} sleep\textsc{.inf.1sg} \\
\glt `I could not sleep.'
\z

\ea\label{Romeyka:ex:50}
Romeyka \il{Hellenic!Romeyka}(\citealt{schreiber2021pontic}: A, 0193) \\
\gll [hab-aðadžega to panimo] eɣo utš eɣabo \\
from-here the go\textsc{.nmz} I \textsc{neg} like\textsc{.prs.1sg} \\
\glt `I don't want to go from here.'  \\
\z

\ea\label{Romeyka:ex:51}
Romeyka \il{Hellenic!Romeyka}(\citealt{schreiber_inprep}: 04\_01072019F\_13; 30) \\
\gll ebašlaepse [so borbatima] \\
start\textsc{.aor.3sg} at.the walking \\
\glt  `She started to walk.'
\z

Subordination of \isi{adjunct} and relative clauses exhibits pre-dominantly \isi{head-final} syntax, although there is a lot of variation only partly dependent on the clause type. Romeyka\il{Hellenic!Romeyka} \isi{adjunct} and relative clauses are predominantly finite, although some non-finite strategies exist. The dependent clause is predominantly pre-posed and generally introduced by pre-verbal adverbial subordinators or relative markers (ex. \ref{Romeyka:ex:52}; for the syntax of relative clauses, see also Section \ref{Romeyka:ss:4.1}. above).

\ea\label{Romeyka:ex:52}
Romeyka \il{Hellenic!Romeyka}(\citealt{schreiber_inprep}: 04\_01072019F\_13; 45) \\
\gll [omon d eruise s ormi] ejendune natsurula \\
when \textsc{rel} fall\textsc{.aor.3sg} to.the river become\textsc{.ipf.3sg} wet \\
\glt `When she fell into the river, she got wet.'  \\
\z

\begin{sloppypar}
With regard to head-order directionality within the complement clause\is{complement!clause}, according to \citet[118]{neocleous_word_2020}, the unmarked \isi{word order} in subordinate/\isi{complement} clauses is \isi{head-final} (\ref{Romeyka:ex:53}) as a consequence of contact with Turkish\il{Turkic!Turkish} \isi{head-final} syntax. However, the example in (\ref{Romeyka:ex:45}) above seems to form a counter-example to this generalization, since the CC in (\ref{Romeyka:ex:45}) is a pragmatically neutral, information-structurally unmarked kind of statement (the \isi{object} is new, not topical nor contrastive) and following \citet{neocleous_word_2020}, we would expect \isi{OV} order here. Thus, both \isi{OV} and \isi{VO} orders seem to be possible in unmarked CCs; potential restrictions with regard to predicate types need further research.
\end{sloppypar}

\ea\label{Romeyka:ex:53}
Romeyka \il{Hellenic!Romeyka}(\citealt{schreiber2021pontic}: C, 0332) \\
\gll eterezen [o argo erθen] \\
look\textsc{.aor.3sg} the bear come\textsc{.aor.3sg} \\
\glt `He saw that the bear came.'  \\
\z

\subsubsection{Copula and `become'-complements}\label{Romeyka:ss:4.7.2}

In the WOWA dataset, only 8\% of copula complement\is{copula!complement}s appear after the \isi{copula}, which means that the \isi{copula} \textit{ime} `be' appears predominantly in clause-final position. \citet[117]{neocleous_word_2020} also confirms that copula clause\is{copula!clause}s with the 3rd person form \textit{en/ine} are always \isi{head-final}. However, when it comes to an explanation of the very low number of copula complement\is{copula!complement}s that appear post-verbally in the WOWA dataset (N=2), it is not straightforward to determine any decisive factor. In any case, predicate nominals (\ref{Romeyka:ex:54}) and predicate adjectives (\ref{Romeyka:ex:55}) seem to behave alike. Information structure can also account for some of the pre-predicate copula complement\is{copula!complement}s (\ref{Romeyka:ex:56}), while interrogative copula clause\is{copula!clause}s are \isi{head-final} due to the \isi{focus} position of the \textit{wh}-element (\ref{Romeyka:ex:57}). Finally, since the overwhelming majority of clause-final copulas occur in the text of a single speaker (see \tabref{Romeyka:tab:5} in Section \ref{Romeyka:ss:6} below), the speaker variable may have an effect as well. Still, it is not clear what determines post-predicate copula complement\is{copula!complement}s as in (\ref{Romeyka:ex:58}), although \isi{information structure} may serve as an explanation here as well.

\ea\label{Romeyka:ex:54}
Romeyka \il{Hellenic!Romeyka}(\citealt{schreiber2021pontic}: A, 0128) \\
\gll ta mandria muna boš en \\
the stables \textsc{poss.cl.1pl} empty be\textsc{.prs.3sg} \\
\glt `Our stables are empty.'  \\
\z

\ea\label{Romeyka:ex:55}
Romeyka \il{Hellenic!Romeyka}(\citealt{schreiber2021pontic}: A, 0020) \\
\gll džumartesi tatili en \\
saturday holiday be\textsc{.prs.3sg} \\
\glt `Saturday is a free day.'  \\
\z

\ea\label{Romeyka:ex:56}
\ea\label{Romeyka:ex:56a}
Romeyka \il{Hellenic!Romeyka}(\citealt{schreiber2021pontic}: D, 0504--0506) \\
\gll ama do vutero poli \\
but the butter much \\
\glt `But the butter [was] plentiful.'
\ex\label{Romeyka:ex:56b}
\gll do vutero bolin adone \\
the butter much be\textsc{.ipf.3sg} \\
\glt `The butter was plentiful.'
\ex\label{Romeyka:ex:56c}
\gll eliɣo utš en do diri \\
little \textsc{neg} be\textsc{.prs.3sg} the cheese \\
\glt `The cheese was not scarce.' 
\z
\z

\ea\label{Romeyka:ex:57}
Romeyka \il{Hellenic!Romeyka}(\citealt{schreiber2021pontic}: D, 0408) \\
\gll laɣo dune \\
how be\textsc{.ipf.3sg} \\
\glt `How was it?'  \\
\z

\ea\label{Romeyka:ex:58}
Romeyka \il{Hellenic!Romeyka}(\citealt{schreiber2021pontic}: A, 0049) \\
\gll en ja ta ðorma muna patikas \\
be\textsc{.prs.3sg} \textsc{dp} the roads \textsc{poss.cl.1pl} unpaved \\
\glt `Our roads are unpaved (as you know).'
\z

Finally, it has to be noted that the \isi{copula} --- and especially its \textsc{3sg} present tense form \textit{en} --- is often omitted (\ref{Romeyka:ex:59}, also \ref{Romeyka:ex:56a} above; see also the last column in \tabref{Romeyka:tab:5} in Section \ref{Romeyka:ss:6}). Due to clause-final null-copulas in Turkish\il{Turkic!Turkish}, it is compelling to assume a \isi{contact influence} here, although an internal explanation may also play a \isi{role} due to the particular nature of the verb \textit{ime}. In Cappadocian\il{Hellenic!Cappadocian}, the full verb \textit{ime} `be' is used in existentials, where it is never dropped; besides, a \isi{copula} \textit{ime} `be' exists which behaves as a \isi{clitic} (\textit{=me}) since MedGr times and thus results always in verb-final orders, although it is never left out in third person singular present tense.

\ea\label{Romeyka:ex:59}
Romeyka \il{Hellenic!Romeyka}(\citealt{schreiber_inprep}: 02\_9062019F\_1; 24) \\
\gll etšinos χaremenos \\
he happy \\
\glt `He is happy.'  \\
\z

\begin{sloppypar}
Interestingly, `become'-complements in the WOWA dataset are 69\% post-verbal, suggesting a different behaviour compared to copula clause\is{copula!clause}s. However, it must be noted that nearly all `become'-complements are produced by the same speaker (i.e., Speaker 2, see \tabref{Romeyka:tab:1} in Section \ref{Romeyka:ss:6}), so this result is to be treated with caution. Although the number of coded tokens is too small for any statistical analysis, there is in the dataset a tendency for adjectives as `become'-complements to be more likely post-predicate than nominal complements (but cf. \textit{ejendune gedže} `it became night' (\citealt{schreiber2021pontic}: B, 0262; E, 0529), also \textit{inete akšemis} `it became evening' (\citealt{schreiber2021pontic}: D, 0437) vs. \textit{gedže ejendune} (\citealt{schreiber2021pontic}: B, 0242)). Again, \isi{information structure} has an influence here; see (\ref{Romeyka:ex:60}) vs. (\ref{Romeyka:ex:61}) in \isi{topic} position.
\end{sloppypar}

\ea\label{Romeyka:ex:60}
Romeyka \il{Hellenic!Romeyka}(\citealt{schreiber2021pontic}: D, 0445) \\
\gll ula inumunesten annera \\
all become\textsc{.ipf.1pl} soakingly\_wet \\
\glt `We became allover soakingly wet.'  \\
\z

\ea\label{Romeyka:ex:61}
Romeyka \il{Hellenic!Romeyka}(\citealt{schreiber2021pontic}: A, 0210) \\
\gll emeklis ba na inese na stetšis aðatšeka \\
retired \textsc{top} \textsc{prt} become\textsc{.prs.2sg} \textsc{prt} stay\textsc{.prs.2sg} here \\
\glt 	`When you become retired, you will stay here.'  \\
\z

\section{Areal issues \& language contact}\label{Romeyka:ss:5}

\begin{sloppypar}
Similarly to the well-documented influences of Anatolian Turkish\il{Turkic!Turkish Anatolian} on AMG described by \citet{dawkins1916}, who inspired much of the literature on \isi{language contact}, long-standing \isi{language contact} with Turkish\il{Turkic!Turkish Trabzon} in the Trabzon area has evidently led to contact-induced changes in Romeyka\il{Hellenic!Romeyka} in several domains of the language including the lexicon and grammar (for a tentative overview see Chapter 6 in \citealt{schreibergrammar2022}; see also \citealt{brendemoen1998,brendemoen1999,brendemoen2002,brendemoen2005,brendemoen2006,brendemoen2010,brendemoen2019} for Greek influences on Turkish\il{Turkic!Turkish}). Importantly, similarly strong influences of Turkish\il{Turkic!Turkish} are reported from Laz\il{Kartvelian!Laz} (a Kartvelian language, see \citealt{kutscher2008}; \citealt{lacroix_laz_2018,lacroix_description_2009}, \citealt{ozturkpochtrager2011}) and Homshetsma\il{Armenian (Western)!Homshetsma}/Hemshinli\il{Armenian (Western)!Hemshinli} (Western Armenian; \citealt{vaux2007}) which are both minority languages in the neighbouring provinces of Rize and Artvin in northeastern Turkey and which share a sociolinguistically and historically similar contact setting with Turkish\il{Turkic!Turkish} as dominant language of the area. Although a comparison of contact-induced influences from Turkish\il{Turkic!Turkish} on the morphosyntax of Laz\il{Kartvelian!Laz} and Hemshinli is beyond the scope of this chapter, \citet[282--284]{neocleous_word_2020} reports a clear restructuring of Laz\il{Kartvelian!Laz} \isi{word order} patterns based on Turkish\il{Turkic!Turkish} to pragmatically unmarked \isi{word order} \isi{OV} (see also the extremely low percentage of post-predicate elements in the WOWA dataset of Arhavi Laz\il{Kartvelian!Laz Arhavi}, \citealt{stilo_laz_2021}) and unlike its Kartvelian relative Georgian (\citealt{neocleous_word_2020}: 243--247). As far as what has been deducible for Hemshinli, it also shares similarities with Turkish\il{Turkic!Turkish}, for example in non-finite subordination which is otherwise less common in related varieties outside the specific contact setting (\citealt{gandon2016}: 210--212). Finally, comparing the \isi{word order} properties of PG as spoken in Armenia (\citealt{hodgson_pontic_nodate}) with that of Romeyka, Armenian PG has only an overall score of 32\% post-posed elements, as opposed to 55\% in Romeyka.
\end{sloppypar}

Another aspect of \isi{language contact} in the area is the \isi{contact influence} PG exerted on the Turkish\il{Turkic!Turkish} Eastern Black Sea dialect in several domains including \isi{word order} \citep{brendemoen1998,brendemoen1999,brendemoen2005,brendemoen2006,brendemoen2019}. For example, post-verbal pronominal DOs in Trabzon Turkish\il{Turkic!Turkish Trabzon} (\ref{Romeyka:ex:62}) are argued by \citet[30]{brendemoen2005} to arise due to contact with PG (cf. the percentage of post-verbal pronominal DOs in colloquial spoken Turkish\il{Turkic!Turkish Ankara} of Ankara (\citealt{iefremenko2021oghuz}) at 7\%). When considering the areal picture of contact-induced changes and \isi{convergence}, the influence of indigenous minority languages on the majority language should not be neglected, although this seems to apply to a lesser degree to the influence of Armenian (and Laz\il{Kartvelian!Laz}?) on regional Trabzon Turkish\il{Turkic!Turkish Trabzon} (\citealt{brendemoen2005}: 29). In any case, potential mutual contact influences highlight the point that for the purpose of inter-language comparison, not primarily the respective standard variety (such as Istanbul Turkish\il{Turkic!Turkish Istanbul}) should be considered, but rather regional varieties (as is done by \citealt{neocleous_word_2020}).

\ea\label{Romeyka:ex:62}
Trabzon Turkish \il{Turkic!Turkish Trabzon}(\citealt{brendemoen2005}: 30, ex. 2, presentation adapted) \\
\gll  yedi oni \\
eat\textsc{.pst.3sg} \textsc{opn.3sg} \\
\glt `He ate it.'
\z

\section{The role of inter-speaker variation}\label{Romeyka:ss:6}

In order to explain (some of) the variability in \isi{word order} patterns in Romeyka\il{Hellenic!Romeyka} that appear in the quantitative data and can only partly be explained by linguistic factors, it is crucial to consider the nature of Romeyka\il{Hellenic!Romeyka} which is currently spoken in Turkey as shifting variety, and the composition of the present sample. As \citet{schreibergrammar2022} has shown, the character of Romeyka\il{Hellenic!Romeyka} as shifting language is not (yet) characterized by language attrition as defined by \citet[12]{thomason2001language} as including structural simplification and loss without compensation but rather with a high inter- (and intra-)speaker variation. Indeed, it is the idiolectal variation based on the individual multilingual profiles of the speakers which explains some of the variability in \isi{word order} patterns. In other words, the overall figure of 55\% post-verbal placement (including 66\% post-verbal placement of nominal DOs) in the WOWA dataset of Romeyka\il{Hellenic!Romeyka} does not reflect a stable norm in ``the speech community'', but a mean value aggregating over very different individuals. As \citet{craevschi_historical_2022} has shown, the Romeyka\il{Hellenic!Romeyka} data set exhibits the greatest degree of \isi{inter-speaker variation} among the twenty-four WOWA data sets analysed by him. In fact, the influence of the independent variable ``Speaker'' outweighs significantly the influence of all other variables which were controlled for in the analysis, even when the imbalanced contribution of the three speakers to the overall data set is taken into consideration (Speaker 3 only contributes around 10\% of the total tokens). This is shown in \tabref{Romeyka:tab:1}.

\begin{table}
\fittable{
\begin{tabular}{rllr}
\lsptoprule
Speaker & Speaker characteristics & Texts provided & Total tokens \\
\midrule
1 & male, middle-aged, ROf, Karaçam & A & 198 \\
2 & female, middle-aged, ROf, Karaçam & B, D, E & 251 \\
3 & male, middle-aged, RSür, Beşköy & C & 52 \\
\lspbottomrule
\end{tabular}
}

    \caption{Overview of speakers and their corresponding texts in the WOWA dataset}
    \label{Romeyka:tab:1}
\end{table}


\begin{table}[b]
\fittable{
\begin{tabular}{lrrr}
\lsptoprule
 & Total nominal DOs (`do'+`do-def') & Total \isi{VO} & \% \isi{VO} \\
\midrule
Speaker 1 (=text A) & 51 & 18 & 35\% \\
Speaker 2 (=text B, D, E) & 103 & 86 & 83\% \\
Speaker 3 (=text C) & 21 & 12 & 57\% \\
\lspbottomrule
\end{tabular}
}
    \caption{Percentage of post-verbal nominal direct objects per speaker}
    \label{Romeyka:tab:2}
\end{table}

The influence of the speaker variable is not the same for all constituent types, and due to the low absolute numbers of tokens for some constituent types, cannot be readily statistically validated for all roles. It does turn out to be significant for predicting the placement of nominal DOs (see also \citealt{craevschi_historical_2022}). As shown in \tabref{Romeyka:tab:2}, Speaker 1 has with nominal DOs predominantly \isi{OV}, while {Speaker 2} has dominantly \isi{OV} order. Speaker 3 appears to be largely balanced but note the smaller absolute number of tokens in the data of Speaker 3. A similar inter-speaker difference is visible for free pronominal DOs, although the percentages for post-verbal placement are there lower in general (see also Section \ref{Romeyka:ss:4.4.2}).


\begin{sloppypar}
A similar picture of \isi{inter-speaker variation} arises for the semantic roles of locations (\tabref{Romeyka:tab:3}) and goals (\tabref{Romeyka:tab:4}). While Speaker 1 shows predominantly pre-verbal locations (and more or less balanced pre- and post-verbal Goal\is{Goal!post-verbal}s), Speaker 2 uses predominantly post-verbal locations and especially goals. Speaker 3 shows clearly preverbal locations --- other than with nominal DOs --- although the amount of data provided by Speaker 3 is too little to get a clear picture.
\end{sloppypar}

\begin{table}
    \begin{tabularx}{\textwidth}{lYYY}
\lsptoprule
 & Total locations & Total VX & \%VX \\
\midrule
Speaker 1 (=text A) & 32 & 11 & 34\% \\
Speaker 2 (=text B, D, E) & 35 & 23 & 66\% \\
Speaker 3 (=text C) & 16 & 1 & 6\% \\
\lspbottomrule
    \end{tabularx}
    \caption{Percentage of post-verbal locations per speaker}
    \label{Romeyka:tab:3}
\end{table}

\begin{table}
\begin{tabularx}{\textwidth}{lrYY}
\lsptoprule
 & Total goals (no pronouns) & Total VX & \%VX \\
\midrule
Speaker 1 (=text A) & 32 & 17 & 53\% \\
Speaker 2 (=text B, D, E) & 44 & 43 & 98\% \\
Speaker 3 (=text C) & 1 & 1 & nc \\
\lspbottomrule
\end{tabularx}
\caption{Percentage of post-verbal Goal\is{Goal!post-verbal}s per speaker}
\label{Romeyka:tab:4}
\end{table}

Ignoring for a moment Speaker 3 (due to low absolute token numbers), it is evident that Speakers 1 and 2 differ rather consistently: Speaker 2 postposes these three \isi{argument} types approximately twice as often as Speaker 1. With regard to nominal direct objects, locations, and goals, then, an account of \isi{word order} purely in terms of language-internal features such as \isi{information structure}, \isi{animacy}, or \isi{weight}, is clearly missing a very important dimension of variation.

To account for the \isi{inter-speaker variation} just described, one would need to consider the individual multilingual profiles and language competences of the speakers. Speaker 1 and 2 stem from the same municipality, which is located remotely and is reported to have high language vitality (\citealt{schreiber2016assessing}); both speakers are roughly of the same age group. Still, Speaker 1 shows more \isi{head-final} syntax than Speaker 2. This can be explained by interference from Turkish\il{Turkic!Turkish} \isi{head-final} word orders due to a multilingual profile where Turkish\il{Turkic!Turkish} is (at least at the time of data collection) the dominant language. Gender is likely to account for the dominance of Turkish\il{Turkic!Turkish} in the multilingual repertoire of (male) Speaker 1 as opposed to (female) Speaker 2. Usually, men are more exposed to Turkish\il{Turkic!Turkish} than woman due to higher mobility for various reasons (like labour and military service, but also differences in education, see \citealt{schreiber2016assessing}). However, it should be noted that individual language biographies can easily override this gender bias. Since unfortunately no detailed biographical information is available for Speaker 1 and 2, gender is initially assumed here as the decisive variable causing the different \isi{word order} patterns.

As for the individual multilingual profile of (male) Speaker 3 who shows a more balanced \isi{word order} with regard to pre-predicate and post-predicate elements, it has to be mentioned that apart from Turkish\il{Turkic!Turkish} and Romeyka\il{Hellenic!Romeyka} he is also competent in Modern Greek, which is likely to have in turn an influence on his data.

In sum, inter-speaker differences in the individual multilingual repertoires of the speakers are able to account for at least some of the variation in \isi{word order} patterns found in Romeyka\il{Hellenic!Romeyka}.

However, when it comes to copula complement\is{copula!complement}s (\tabref{Romeyka:tab:5}), the picture of \isi{inter-speaker variation} is different with overall low numbers of post-predicate complements irrespective of the percentage of post-predicate other semantic roles. This suggests that there is indeed a change in word-order patterns in copula complement\is{copula!complement}s, which is reflected in the data by all three speakers. Moreover, all three speakers tend to omit some copulas; the explanation of this phenomenon, however, requires further research and could be potentially affected by the method of data elicitation for the WOWA dataset (see below).

\begin{table}
\begin{tabularx}{\textwidth}{lYrrY}
\lsptoprule
 & Total copula complement\is{copula!complement}s & Total VX & \% VX & Omitted copulas \\
\midrule
Speaker 1 (=text A) & 19 & 2 & 10\% & 2 \\
Speaker 2 (=text B, D, E) & 4 & 0 & 0\% & 6 \\
Speaker 3 (=text C) & 3 & 1 & 33\% & 4 \\
\lspbottomrule
\end{tabularx}
    \caption{Copula complements in the WOWA dataset per speaker}
    \label{Romeyka:tab:5}
\end{table}


In order to be able to integrate the findings on \isi{word order} variation in the present WOWA dataset correctly, some critical notes on the dataset and methodology of data collection are in order. Firstly, with 500 tokens, the present dataset is very small and on the lower edge of what can meaningfully be analysed by quantitative means. It has also to be noted that the WOWA Romeyka\il{Hellenic!Romeyka} dataset has a relatively high number of tokens (N=98) that could not be coded due to non-canonical constructions such as mixing with Turkish\il{Turkic!Turkish} or elliptical constructions which also add to the picture of Romeyka\il{Hellenic!Romeyka} as a shifting variety. Secondly, part of the five texts in the WOWA dataset were elicited by means of a storytelling task prompted by picture cards which could have had an influence on the data as well, especially with regard to the omission of copulas which was striking in the narratives elicited with the help of the picture cards. Finally, the fact that the five texts stem from three different speakers which are not directly comparable in terms of gender, age, speech community and their multilingual competences further complicates the analysis. It has to be mentioned, though, that with regard to diatopic variation, the three main dialect areas (\citealt{schreiberRomeyka2018}) are not expected to show significant differences in the domain of \isi{word order}, except for the potentially different amount of exposure to and use of Turkish\il{Turkic!Turkish}, which is considered to be higher in the villages closer to the sea and lower in the more remote mountain villages (see \citealt{schreiberRomeyka2018}).

\section{Conclusion}\label{Romeyka:ss:7}

The aim of the present chapter was to analyze the WOWA dataset of Romeyka\il{Hellenic!Romeyka} (\citealt{schreiber2021pontic}) with regard to \isi{word order} in Romeyka\il{Hellenic!Romeyka} in general and the grammatical entities that occur in the post-verbal domain. Romeyka\il{Hellenic!Romeyka} has been described as having inherited \isi{VO} \isi{word order}, which has developed mixed directionality under contact with Turkish\il{Turkic!Turkish}, which is visible especially in unmarked \isi{OV} order in subordinate clauses (\citealt{neocleous_word_2020,neocleousEvolution2022}) but also in ongoing change in certain domains such as copula complement\is{copula!complement}s, although the diachronic picture is complex here. Furthermore, the analysis of the present WOWA dataset has revealed that \isi{inter-speaker variation} mirrors ongoing \isi{language shift} to Turkish\il{Turkic!Turkish}, which complicates the attempt to define default \isi{word order} patterns for the language and requires a nuanced methodology in assessing \isi{word order} as if to allow for any meaningful conclusion. In general, \isi{information structure} accounts for much of the variability in \isi{word order} patterns that has been described above. Furthermore, PPs seem to behave differently than other obliques/objects, as well in locations and goals as in benefactives and comitatives.

To conclude, the picture of \isi{word order} in Romeyka\il{Hellenic!Romeyka} is by no means clearly deducible from quantitative data, as is indicated by the number of 55\% overall frequency for post-verbal placement. There are several aspects accounting for the significant variation found in the present WOWA dataset: (i) the \isi{role} of pragmatics, namely \isi{information structure}, is highly relevant in Romeyka\il{Hellenic!Romeyka} with at the same time hardly a significant correlation of other factors such as semantic \isi{role}, \isi{flagging} or \isi{weight}, although goals show a tendency to be post-predicate which seems to qualify as an areal (or otherwise for certain reasons universal?) pattern in other languages of the area as well; (ii) the high amount of \isi{inter-speaker variation} which can be traced back to different levels of Turkish\il{Turkic!Turkish} influence in the individual multilingual profiles of the speakers and reflects the status of Romeyka\il{Hellenic!Romeyka} as a shifting variety; but also (iii) considerable \isi{intra-speaker variation} as a characteristic of \isi{language shift}; and (iv) ongoing language change as in the case of copula complement\is{copula!complement}s, be it internally caused and reinforced by \isi{language contact}, which requires a very fine-grained and domain-specific investigation not only on the synchronic feature but involving diachronic developments as well as potential contact influences.


% Abbreviations

% References




\section*{Abbreviations}
\begin{tabularx}{.45\textwidth}{lQ}
\textsc{acc} & accusative \\
\textsc{aor} & aorist \\
\textsc{aux} & auxiliary \\
\textsc{cl} & clitic \\
\textsc{dem} & demonstrative \\
\textsc{dp} & discourse particle \\
\textsc{gen} & genitive \\
\textsc{inf} & infinitive \\
\textsc{ipf} & imperfective \\
\textsc{neg} & negation \\
\textsc{nmz} & nominalization \\
\end{tabularx}
\begin{tabularx}{.45\textwidth}{lQ}
\textsc{nom} & nominative \\
\textsc{opn} & object pronoun \\
\textsc{pl} & plural \\
\textsc{poss} & possessive pronoun \\
\textsc{prs} & present tense \\
\textsc{prt} & aspectual particle \\
\textsc{pst} & past tense \\
\textsc{rel} & relativizer \\
\textsc{sg} & singular \\
\textsc{top} & topicalizer \\
1, 2, 3 & 1st, 2nd, 3rd person \\
\end{tabularx}

\section*{Acknowledgements}

\begin{sloppypar}
The authors thank the Romeyka\il{Hellenic!Romeyka} speakers and acknowledge financial support from the Alexander-von-Humboldt Foundation within the framework of the project on ``Post-predicate Elements in Iranian: Inheritance, Contact, and Information Structure'' (PI's: Geoffrey Haig, Mohammad Rasekh-Mahand) for field research in Trabzon/Çaykara in June/July 2019. Furthermore, our gratitude is due to an anonymous reviewer for their very helpful, detailed remarks.
\end{sloppypar}

{\sloppy\printbibliography[heading=subbibliography,notkeyword=this]}

\end{document}
