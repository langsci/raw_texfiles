\documentclass[output=paper,colorlinks,citecolor=brown,draftmode]{langscibook}
\ChapterDOI{10.5281/zenodo.14266357}
\author{Bettina Leitner\orcid{0000-0001-6712-302X}\affiliation{Universität Wien}}
\title{Khuzestani Arabic}
\abstract{This chapter describes the basic word order profile of Khuzestani Arabic and discusses possible reasons for deviations from the default word order VX (X representing non-subject arguments). This discussion includes an analysis of where the change from VX to XV may be triggered by language contact or by language internal reasons related to information structure. The description is mainly based on data from the WOWA-corpus (\url{https://multicast.aspra.uni-bamberg.de/resources/wowa/\#semitic}) and supplemented by the author's own corpus data.}

%move the following commands to the ``local...'' files of the master project when integrating this chapter
% \usepackage{tabularx}
% \usepackage{langsci-optional}
% \usepackage{langsci-gb4e}
% \usepackage{enumitem}
% \bibliography{localbibliography}
% \newcommand{\orcid}[1]{}
% \let\eachwordone=\itshape

\IfFileExists{../localcommands.tex}{
 \addbibresource{../collection_tmp.bib}
 \addbibresource{../localbibliography.bib}
 % add all extra packages you need to load to this file

\usepackage{tabularx,multicol}
\usepackage{url}
\urlstyle{same}

\usepackage{listings}
\lstset{basicstyle=\ttfamily,tabsize=2,breaklines=true}

\usepackage{langsci-basic}
\usepackage{langsci-optional}
\usepackage{langsci-lgr}
\usepackage{langsci-osl}
% \usepackage{./langsci/styles/langsci-lgr}
% \usepackage{./langsci/styles/langsci-osl}
% \usepackage{langsci-gb4e}

\usepackage{tikz}
\usetikzlibrary{patterns,calc}
\pgfdeclarepatternformonly{south east lines}{\pgfqpoint{-0pt}{-0pt}}{\pgfqpoint{3pt}{3pt}}{\pgfqpoint{3pt}{3pt}}{
    \pgfsetlinewidth{0.6pt}
    \pgfpathmoveto{\pgfqpoint{0pt}{3pt}}
    \pgfpathlineto{\pgfqpoint{3pt}{0pt}}
    \pgfpathmoveto{\pgfqpoint{.2pt}{-.2pt}}
    \pgfpathlineto{\pgfqpoint{-.2pt}{.2pt}}
    \pgfpathmoveto{\pgfqpoint{3.2pt}{2.8pt}}
    \pgfpathlineto{\pgfqpoint{2.8pt}{3.2pt}}
    \pgfusepath{stroke}}
    
\usepackage{stmaryrd}
\usepackage{wasysym}
\usepackage{multirow}
\usepackage{caption}
\usepackage{subcaption}
\usepackage{mathrsfs}
\usepackage{qtree}

\usepackage{linguex}


 %pminos do not split footnotes
% \interfootnotelinepenalty=10000 %Footnote in Laporte chapters has to be split SN


%\DeclareIndexNameFormat{default}{%
%\nameparts{#1}%
%\usebibmacro{index:name}%
%{\index[names]}%
%{\namepartfamily}%
%{\namepartgiveni}%
% {}% L1
% {}% L2
%{\namepartprefix}% generates spurious space L3
%{\namepartsuffix}% generates spurious space L4
%}

%  {\DeclareIndexNameFormat{default}{%
%     \usebibmacro{index:name}{\index[names]}{#1}{#3}{#5}{#7}}}

%\DeclareIndexNameFormat{default}{%
%  \usebibmacro{index:name}{\sindex[nom]}{#1}{#3}{#5}{#7}}

%\DeclareIndexNameFormat{default}{%
%  \usebibmacro{index:name}{\sindex[person]}{#1}{#3}{#5}{#7}}
%\DeclareIndexNameFormat{default}{%
%\nameparts{#1} \usebibmacro{index:name}{\sindex[person]]}{\namepartfamily}{‌​\namepartgiven}{\nam‌​epartprefix}{\namepa‌​rtsuffix}}

%\newcommand{\smiley}{:)}

%\renewbibmacro*{index:name}[5]{%
%\usebibmacro{index:entry}{#1}%
%{\iffieldundef{usera}{}{\thefield{usera}\actualoperator}\mkbibindexname{#2}{#3}{#4}{#5}}}

% \newcommand{\noop}[1]{}

%remove for final
%\overfullrule=1mm

\newcommand{\tobi}[2]}}
\renewcommand{\S}[1]{\tobi{#1}{\textsc{*}}}

% this volume references
% puts: [this volume]
% already defined: \citetv
%\newcommand{\citepv}[1]{(\citeauthor{#1} \citeyear*{#1} [this volume])}
\newcommand{\citealtv}[1]{\citeauthor{#1} \citeyear*{#1} [this volume]}

%parentheses around example number
\newcommand{\pref}[1]{(\ref{#1})}

% in-text examples

\newcommand{\lnex}[1]{\textit{#1}} %target lang word
\newcommand{\lnlit}[1]{(lit.: `#1')} %literal reading
\newcommand{\lnlat}[1]{(#1)} % latinization
\newcommand{\lntrans}[1]{`#1'} %translation
\newcommand{\lnexl}[2]%
{\lnex{#1}{} \lnlat{#2}} % ex with latinization
\newcommand{\lnexlat}[3]{\lnex{#1}{} \lnlat{#2}{} \lntrans{#3}} % ex with latinization and tranl.

%ch01
\newcommand{\co}[1]{\mbox{\textbf{#1}}}

%ch09

\newcommand{\cyrbulg}[1]{\begin{otherlanguage*}{bulgarian}#1\end{otherlanguage*}}


%ch10
\newcommand{\nlp}{{\small NLP}}
\newcommand{\mwe}{{\small MWE}}
\newcommand{\rae}{{\small RAE}}
\newcommand{\lvc}{{\small LVC}}
\newcommand{\pos}{{\small P}o{\small S}}
%\newcommand{\todo}[1]{ \textcolor{red}{#1} }

%\renewcommand{\labelenumi}{\theenumi}
%\ainamefmt{{vv}{ll}{, ff}{, jj}} % fullname

\newcommand{\biberror}[1]{{\color{red}#1}}

\newcommand{\osenovaitem}{--~}
 %% hyphenation points for line breaks
%% Normally, automatic hyphenation in LaTeX is very good
%% If a word is mis-hyphenated, add it to this file
%%
%% add information to TeX file before \begin{document} with:
%% %% hyphenation points for line breaks
%% Normally, automatic hyphenation in LaTeX is very good
%% If a word is mis-hyphenated, add it to this file
%%
%% add information to TeX file before \begin{document} with:
%% %% hyphenation points for line breaks
%% Normally, automatic hyphenation in LaTeX is very good
%% If a word is mis-hyphenated, add it to this file
%%
%% add information to TeX file before \begin{document} with:
%% \include{localhyphenation}
\hyphenation{
    Beck-man
    Ngu-yen
    back-chan-nel
    back-chan-nels
    mo-not-o-nous
    ste-reo-typ-i-cal
}

\hyphenation{
    Beck-man
    Ngu-yen
    back-chan-nel
    back-chan-nels
    mo-not-o-nous
    ste-reo-typ-i-cal
}

\hyphenation{
    Beck-man
    Ngu-yen
    back-chan-nel
    back-chan-nels
    mo-not-o-nous
    ste-reo-typ-i-cal
}

%  \boolfalse{bookcompile}
%  \togglepaper[5]%%chapternumber
}{}


\begin{document}
\maketitle\label{WOWA:ch:14}



\section{Introduction and data}\label{Khuzestani:ss:1}

The Arabic variety spoken in the southwestern Iranian region of Khuzestan belongs to the Mesopotamian dialectal area and the subgroup of \textit{gələt} dialects\footnote{The term \textit{gələt}, first used by \citet{Blanc1964CDBaghdad} to classify the Iraqi Arabic\il{Arabic (Gələt)!Iraqi} dialects, is based on the \textsc{1sg pfv} verb for `to say': \textit{gələt} versus \textit{qəltu}—the latter being the other group of dialects spoken in Iraq and southern Anatolia. \textit{gələt}-dialects are associated with Bedouin and rural Arabic, even though nowadays of course almost no speakers live as nomads any longer and many have moved to cities, cf. \citet{leitnerNew2021}.}. The grammar of Khuzestani Arabic\il{Arabic (Gələt)!Khuzestani} (KhA) has been described mainly by Bruce Ingham and the author of this chapter (among the most important contributions are \citealt{ingham1973,ingham1976,ingham2008}, and \citealt{leitnerGrammar2022}). This chapter presents the first analysis of \isi{word order} structures in KhA since \citet{ingham1991} and has the advantage of being based on a comparatively large text corpus. 

Arab settlement in southern and western Iran (i.e. Khuzestan and Fars) is already documented for Sasanian times (226–651 AD), and thus precedes the arrival of the Arab Muslim armies (\citealt{zarrinkub1975}: 27). However, the real Arab dispersal into Iran began after the initial Islamic victories when many tribes from the vicinities of Kufa and Basra entered Iranian soil following the conquest (cf. \citealt{leitnerGrammar2022}: 6–7 and the references mentioned there). Many of the Arab tribes who immigrated into Khuzestan had originated in Arabia (cf. \citealt{savory1986}: 81; cf. \citealt{nadjmabadi2009}: 132, Fn. 28–29; \citealt{field1939}: 604) and first settled in southern Iraq. Their subsequent immigration to Khuzestan led to an extensive Arabization of the province, parts of which were officially called Arabistan from the 16th/17th century until 1923 (\citealt{oppenheim1967}: 3, 10; cf. \citealt{ingham1997}: ix). Most Arab tribes, such as the Kaʕab, adopted Shiism after their settlement in Iran, but some remained Sunnis, e.g. the Muntafiq, who migrated to Hoveyzeh in 1812 (\citealt{savory1986}: 81). 

\begin{sloppypar}
The Iran-Iraq War (1980–-1988) forced many families to flee their hometowns and thus led to considerable demographic changes. Both the city of Khorramshahr/Muḥammara (and its port) and the city of Abadan (and its refineries) were completely destroyed in the course of the Iran-Iraq War by Iraqi artillery and aerial bombardments.
\end{sloppypar}

The region's capital city Ahvaz in turn has witnessed an immense growth in the past decades (according to \citealt{nejatian2015} the number of inhabitants in Ahvaz grew from 334,399 in 1976 to 724,653 in 1991, and to 1,112,021 in 2011). 

Persian\il{Persian (New)}, being the majority language as well as the only official language and language of education and administration in Iran, plays a crucial \isi{role} for the people in Khuzestan. The majority of the Khuzestani Arabs are bilingual\is{bilingualism}, but there are still monolingual Arabic speakers, especially among the older generation. KhA is insulated from influence by MSA\il{Arabic!Modern Standard}, but sharing a long geographically open border with Iraq, Khuzestan is not totally isolated from the Arabic-speaking world. The linguistic influence Persian\il{Persian (New)} has had on KhA is strongest in lexicon, but it is also evident in some aspects of its \isi{phonology} and syntax. This paper will \isi{focus} on possible \isi{language contact} influences in the domain of syntax with a \isi{focus} on \isi{word order} and sentence structure. For this purpose, we will characterize the \isi{word order} profile of KhA relying primarily on the WOWA corpus data\footnote{\url{https://multicast.aspra.uni-bamberg.de/resources/wowa/\#semitic}.}. For a more general evaluation of \isi{language contact} phenomena in Khuzestani Arabic\il{Arabic (Gələt)!Khuzestani}, cf. \citet{gazsi2011}, \citet{matras2007}; \citet{leitnerContact-induced2020}. 

The data for this contribution was gathered in field studies in Khuzestan in 2016 and comprises 6 texts (about 9,600 transcribed words) from 8 different speakers (5 female, 3 male, aged between 30 and 65). \tabref{Khuzestani:tab:1} provides an overview of this data, which contains three narrative interviews, one conversation, one traditional tale, and one procedural description (recipe). Examples taken from the author's data other than that of the WOWA-corpus data will be labelled as ``(own data)''.

\begin{table}
\begin{tabularx}{\textwidth}{lQlll}
\lsptoprule
& \textbf{Text name} & \textbf{Genre} & \textbf{Gender } & \textbf{Age} \\
\midrule
A & Mōze - Past times & Narrative interview & F & 65 \\
B & Shepherd & Narrative interview & M & 30 \\
C & Hamidiye women & Conversation & F, F & 60, 60 \\
D & Ghazawiyya Palm farmer & Narrative interview & M, M & 35, 30 \\
E & Umm Saʕad - Ḥamda & Traditional tale & F & 45 \\
F & Amine - recipes & Procedural text & F & 30 \\
\lspbottomrule
 \end{tabularx}
 \caption{Metadata WOWA corpus}
 \label{Khuzestani:tab:1}
\end{table}


The main body of this chapter contains general information on KhA sentence structure and \isi{word order} (Section \ref{Khuzestani:ss:2}). This is followed by a discussion on the factors that may trigger pre-predicate position of constituents in KhA and the likelihood of an explanation of these changes via Persian\il{Persian (New)} influence or, otherwise, due to language-internal reasons related to \isi{information structure} and focus-fronting (Section \ref{Khuzestani:ss:3}).

\section{Word order profile}\label{Khuzestani:ss:2}
\largerpage
Out of the dataset's 546 analyzable tokens\footnote{Tokens are defined as non-subject-constituents with one of the following roles: \isi{direct object}; \isi{Goal}; \isi{recipient}; \isi{addressee}; location; \isi{instrument}; \isi{comitative}; \isi{copula} \isi{complement} noun; possessed NP in a \isi{possessive} expression; and \isi{complement} of a change-of-state verb.}, 479 were found in post-predicate position and only 67 in pre-predicate position, which confirms that Khuzestani Arabic\il{Arabic (Gələt)!Khuzestani}, like most Arabic varieties, is a \isi{VO} language. This suggests that the influence of the contact language Persian\il{Persian (New)}, an \isi{OV} language, on KhA \isi{word order} is not very strong. In the following, a brief \isi{word order} profile of KhA will be given. 

\subsection{Adjective/noun}\label{Khuzestani:ss:2.1}

Adjectives generally follow nouns in KhA as in most other Arabic varieties (\ref{Khuzestani:ex:1}). 

\ea\label{Khuzestani:ex:1}
Khuzestani Arabic \il{Arabic (Gələt)!Khuzestani}(own data) \\
\gll baḷḷa xall nšərrb-a māy fāyər yiġsil ṣadr-a \\
\textsc{dm} \textsc{hort} make\_drink\textsc{.ipfv.1pl-3sg.m} water boiling wash\textsc{.ipfv.3sg.m} breast\textsc{-3sg.m} \\
\glt `Let us make him drink hot [lit. boiling] water that makes him feel good [lit. cleans his breast].' 
\z

\subsection{Possessor/possessed}\label{Khuzestani:ss:2.2}

Possession is expressed either via synthetic nominal attribution or via the analytic genitive.
The basic syntagm for synthetic nominal attribution constructions (Arabic \textit{ʔiḍāfa}) is NOUN (in construct state) + NOUN/DEF-NOUN. The second noun is usually a (definite or indefinite) substantive, as in (\ref{Khuzestani:ex:2}) and (\ref{Khuzestani:ex:3}).

\ea\label{Khuzestani:ex:2}
Khuzestani Arabic \il{Arabic (Gələt)!Khuzestani}(\citealt{leitnerArabic2021}: D, 0591) \\
\gll ḥalīb əl-hōš \\
milk \textsc{def-}cow\textsc{.coll} \\
\glt `cow's milk' 
\z

\ea\label{Khuzestani:ex:3}
Khuzestani Arabic \il{Arabic (Gələt)!Khuzestani}(\citealt{leitnerArabic2021}: A, 0166) \\
\gll ṭāsa-t rōba \\
bowl\textsc{-cs} yoghurt \\
\glt `a bowl [full] of yoghurt' 
\z

\begin{sloppypar}
The two default types of the analytic genitive syntagm are: NOUN (POSSESSED) + MĀL + NOUN (POSSESSOR) (\ref{Khuzestani:ex:4}) and NOUN + MĀL-PRONOMINAL SUFFIX (\ref{Khuzestani:ex:5}). There are also examples in which the element after \textit{māl} is an \isi{adverb}. This linker for nominal attribution is usually labeled in Arabic dialectology as a ``genitive exponent'' or ``genitive marker'' (see \citealt{leitnerGrammar2022}: 176--189 and the references mentioned there for a more detailed elaboration of such constructions in KhA). 
\end{sloppypar}

\ea\label{Khuzestani:ex:4}
Khuzestani Arabic \il{Arabic (Gələt)!Khuzestani}(\citealt{leitnerArabic2021}: E, 0652) \\
\gll ġaṣәr māl malək \\
castle \textsc{gl.sg.m} king \\
\glt `a castle of a king' 
\z

\ea\label{Khuzestani:ex:5}
Khuzestani Arabic \il{Arabic (Gələt)!Khuzestani}(\citealt{leitnerArabic2021}: D, 0537) \\
\gll əl-xūṣ māl-a \\
\textsc{def-}palm\_fronds \textsc{gl.sg.m-3sg.m} \\
\glt `its [the palm's] fronds'
\z

\ea\label{Khuzestani:ex:6}
Khuzestani Arabic \il{Arabic (Gələt)!Khuzestani}(\citealt{leitnerArabic2021}: A, 0217) \\
\gll maṯal ʕad-na xamsīn ḥōliyye ʕad-na sittīn ḥōliyye \\
for\_example at\textsc{-1pl} fifty young\_female\_buffalo at\textsc{-1pl} sixty young\_female\_buffalo \\
\glt `We had like fifty [young female] buffaloes or sixty buffaloes.' 
\z

For \isi{possessive} constructions with the \isi{preposition} \textit{ʕad}, lit. `at', the default syntagm is \textit{ʕad}-PRO (POSSESSOR) + POSSESSED (\ref{Khuzestani:ex:6}). 

While \textit{māl} can be used both predicatively and attributively, \textit{ʕad} can only be used predicatively. In the WOWA corpus, there are 4 out of 32 instances of the \isi{possessive} construction with \textit{ʕad} that have the POSSESSED in pre-predicate position, i.e. preceding \textit{ʕad}-PRO, as in (\ref{Khuzestani:ex:7}): \textit{ʔamān ma ʕad-na} `we don't have safety'. This marked pre-predicative \isi{word order} is often used in combination with negation to stress non-possession of a certain item and often co-occurs with a repetition of an already mentioned noun (here: `safety'), as illustrated in the following example. 

\ea\label{Khuzestani:ex:7}
Khuzestani Arabic \il{Arabic (Gələt)!Khuzestani}(\citealt{leitnerArabic2021}: B, 0224) \\
\gll əs-surūḥ w ʕalaf w taʕb w nṭāra w b-əl-ʔaham ham ha-l-ayyām masʔala masʔalt əl-ʔamān ʔamān ma ʕad-na əl-ḥalāl b-īd-ak w yəmbāg \\
\textsc{def-}grazing and fodder and exhaustion and watching and in\textsc{-def-}most\_important also \textsc{dem-def-}days question question \textsc{def-}safety safety \textsc{neg} at\textsc{-1pl} \textsc{def-}cattle in-hand\textsc{-2sg.m} and be\_stolen\textsc{.ipfv.3sg.m} \\
\glt `The grazing and the fodder, the exhaustion, and the guarding, and the important thing these days is the question of safety, we don't have safety, the cattle that is in your hand might be stolen [any minute].' 
\z

\textit{kəllšāy} `everything' often precedes a negative possessive-construction \textit{mā ʕad}-PRO, as in the following example (\ref{Khuzestani:ex:8}), to express `to really have nothing', i.e. for emphasizing the fact that one really does not own anything (and with that implicitly contrasting her with others who have more). This structure might be a calque on Persian\il{Persian (New)} \textit{hičči na-dāšt} `she had nothing', but is also found in Iraqi Arabic\il{Arabic (Gələt)!Iraqi} (see e.g. \citealt[172]{leitner2021lehrbuch} ``\textit{kull wakit ma ʕindi}. `Ich habe gar keine Zeit.'{}'' and ``\textit{kull šī māku}. `Es gibt gar nichts.'{}''; cf. Section \ref{Khuzestani:ss:3}). 

\ea\label{Khuzestani:ex:8}
Khuzestani Arabic \il{Arabic (Gələt)!Khuzestani}(\citealt{leitnerArabic2021}: E, 0614) \\
\gll w hāy l-əbnayya bass әhəya təsraḥ b-əl-ġanam kəllšāy mā ʕad-ha \\
and \textsc{dem} \textsc{def-}girl but \textsc{3sg.f} graze\textsc{.ipfv.3sg.f} with\textsc{-def-}sheep everything \textsc{neg} at\textsc{-3sg.f} \\
\glt `And this girl was always just grazing the sheep, she had nothing [else].' 
\z

\subsection{Demonstrative/noun }\label{Khuzestani:ss:2.3}

Demonstratives usually come before the noun (\ref{Khuzestani:ex:9}). 

\ea\label{Khuzestani:ex:9}
Khuzestani Arabic \il{Arabic (Gələt)!Khuzestani}(\citealt{leitnerArabic2021}: C, 0502) \\
\gll hāḏann lə-hədūm \\
\textsc{dem.pl.f} \textsc{def-}clothes \\
\glt `these clothes' 
\z

Though their position before the noun prevails, demonstratives can also follow the head. In such constructions, however, the noun is often emphasized (\ref{Khuzestani:ex:10}).

\ea\label{Khuzestani:ex:10}
Khuzestani Arabic \il{Arabic (Gələt)!Khuzestani}(own data) \\
\gll əl-walad hāḏa rabbō-(h) \\
\textsc{def-}boy \textsc{dem.sg.m} raise\textsc{.pfv.3pl.m-3sg.m} \\
\glt `They raised this boy.' 
\z

Also, the noun can be both preceded and followed by a demonstrative (\ref{Khuzestani:ex:11}; such constructions are usually limited to the \textsc{sg} proximal demonstratives \textsc{m} \textit{hāḏ}, \textsc{f} \textit{hāḏi, hāy}). 

\ea\label{Khuzestani:ex:11}
Khuzestani Arabic \il{Arabic (Gələt)!Khuzestani}(own data) \\
\gll əḥna hāy əl-ʕarab hāy \\
\textsc{1pl} \textsc{dem} \textsc{def-}arabs \textsc{dem} \\
\glt `we Arabs [COLL]' 
\z

\subsection{Numeral/noun }\label{Khuzestani:ss:2.4}

Numerals generally precede nouns (\ref{Khuzestani:ex:12}, \ref{Khuzestani:ex:13}).
\ea\label{Khuzestani:ex:12}
Khuzestani Arabic \il{Arabic (Gələt)!Khuzestani}(\citealt{leitnerArabic2021}: A, 0217) \\
\gll sittīn ḥōliyye \\
sixty young\_female\_water\_buffalo \\
\glt `sixty [young female] water buffaloes' 
\z

\ea\label{Khuzestani:ex:13}
Khuzestani Arabic \il{Arabic (Gələt)!Khuzestani}(\citealt{leitnerArabic2021}: B, 0226) \\
\gll sitt əšhur \\
six month\textsc{.pl} \\
\glt `six months' 
\z

\subsection{Adpositions }\label{Khuzestani:ss:2.5}

Prepositional phrases usually follow the verbal predicate, as in (\ref{Khuzestani:ex:14}) and (\ref{Khuzestani:ex:15}). A counterexample is e.g. provided by (\ref{Khuzestani:ex:18}), where \textit{mən zuġur} `from childhood (on)' precedes the predicate. 

\ea\label{Khuzestani:ex:14}
Khuzestani Arabic \il{Arabic (Gələt)!Khuzestani}(\citealt{leitnerArabic2021}: E, 0654) \\
\gll gālat ʔāna ʔaḏ̣əllan bә-hāḏa ġaṣәr māl əl-malək \\
say\textsc{.pfv.3sg.f} \textsc{1sg} stay\textsc{.ipfv.1sg} in\textsc{-dem.sg.m} castle \textsc{gl.sg.m} \textsc{def-}king \\
\glt `She said: ``I will stay in this, the king's castle.''' 
\z

\ea\label{Khuzestani:ex:15}
Khuzestani Arabic \il{Arabic (Gələt)!Khuzestani}(\citealt{leitnerArabic2021}: F, 0751) \\
\gll w ənnōb tāli nrawwi əb-ṣīniyye \\
and then next form\_balls\textsc{.ipfv.1pl} on-tablet \\
\glt `…and then we form balls [of dough for baking bread] on the tablet.' 
\z

\subsection{Auxiliary/main verb }\label{Khuzestani:ss:2.6}

The default position for \isi{auxiliary} verbs is before the main verb (\ref{Khuzestani:ex:16}, \ref{Khuzestani:ex:17}).

\ea\label{Khuzestani:ex:16}
Khuzestani Arabic \il{Arabic (Gələt)!Khuzestani}(\citealt{leitnerArabic2021}: F, 0785) \\
\gll gabul ma čānu ystəfādūn ləbləbi \\
formerly \textsc{neg} \textsc{aux.3pl.m} use\textsc{.ipfv.3pl.m} chick\_peas \\
\glt `In former times they didn't use chick peas [for cooking].' 
\z

\ea\label{Khuzestani:ex:17}
Khuzestani Arabic \il{Arabic (Gələt)!Khuzestani}(\citealt{leitnerArabic2021}: A, 0187) \\
\gll ġarafna l-əl-hōr gəmna ənḥušš \\
row\textsc{.pfv.1pl} to\textsc{-def-}marshland \textsc{aux.1pl} cut\_grass\textsc{.ipfv.1pl} \\
\glt `We rowed to the marshland [\textit{hōr}], we started to cut grass.' 
\z

However, as has been suggested in \citet{leitnerContact-induced2020,leitnerClause-final2022}, there seems to be an ongoing change probably triggered by contact with Persian\il{Persian (New)} that yields clause-final position of the \isi{auxiliary}, cf. also the following example (\ref{Khuzestani:ex:18}). This development is paralleled by the tendency towards postpredicate position of copulas (cf. Section \ref{Khuzestani:ss:3} below; \citealt{leitnerClause-final2022}). Of course, it may never be entirely ruled out that it is rather pragmatic reasons that cause some of the postpositions of the \isi{auxiliary} (e.g. as a time frame setter, cf. \citealt{brustad2000}), but the comparative numbers presented in \citet{leitnerClause-final2022} speak rather for an explanation as a contact feature.

\ea\label{Khuzestani:ex:18}
\ea\label{Khuzestani:ex:18a}
Khuzestani Arabic \il{Arabic (Gələt)!Khuzestani}(own data) \\
\gll hāde ham mən zuġur yəštəġəl čān \\
\textsc{dem.sg.m} also from childhood work\textsc{.ipfv.3sg.m} \textsc{aux} \\
\ex\label{Khuzestani:ex:18b}
New Persian \il{Persian (New)}(own data) \\
\gll in ham az kudeki kār mi-kard \\
\textsc{dem.sg} also from childhood work \textsc{prog-}do\textsc{.pst.3sg} \\
\glt `This one has also been working from childhood on.' 
\z
\z


\subsection{Complement clause/matrix verb }\label{Khuzestani:ss:2.7}

Complement clauses follow the matrix verb and a complementizer \textit{əlli} (\ref{Khuzestani:ex:19}) or \textit{ənnu} `that' (\ref{Khuzestani:ex:20}). In general, however, the complementizers are often omitted and asyndetic constructions are preferred as in (\ref{Khuzestani:ex:21}). 

\ea\label{Khuzestani:ex:19}
Khuzestani Arabic \il{Arabic (Gələt)!Khuzestani}(own data) \\
\gll ətgūl әlli lyōm mā ətrūḥ l-əš-šəġəl \\
say\textsc{.ipfv.3sg.f} that today \textsc{neg} go\textsc{.ipfv.3sg.f } to\textsc{-def-}work \\
\glt `She says that today she won't go to work.' 
\z

\ea\label{Khuzestani:ex:20}
Khuzestani Arabic \il{Arabic (Gələt)!Khuzestani}(own data) \\
\gll w maʕrūf ənnu məṯəl \\
and known that for\_example \\
\glt `And [it is] known that for example…' 
\z

\ea\label{Khuzestani:ex:21}
Khuzestani Arabic \il{Arabic (Gələt)!Khuzestani}(own data) \\
\gll gāl əlyōm māku ṭalʕa \\
say\textsc{.pfv.3sg.m} today \textsc{exist.neg} going\_out \\
\glt `He said that today there is no going out.' 
\z

\subsection{Nominal direct object/verb }\label{Khuzestani:ss:2.8}

The default or unmarked \isi{word order} is \isi{VO} as in (\ref{Khuzestani:ex:22}). This order appears 274 times out of a total of 317 direct objects in the WOWA-KhA-corpus. 43 tokens are found in pre-predicate position (\isi{OV}). Of these 43 OV-constructions, 27 show a resumptive \isi{pronoun} (co-referential with the \isi{object}) after the verb as in (\ref{Khuzestani:ex:23}) and only 16 had no resumptive \isi{pronoun} after the verb, e.g. (\ref{Khuzestani:ex:24}). The 27 examples of \isi{OV} + resumptive \isi{pronoun} thus are not plain \isi{OV} constructions, but instead cases of \isi{topicalization} (as further discussed in Section \ref{Khuzestani:ss:3}). 

\ea\label{Khuzestani:ex:22}
Khuzestani Arabic \il{Arabic (Gələt)!Khuzestani}(\citealt{leitnerArabic2021}: C, 0434) \\
\gll šəfət-l-i maʕǧiza \\
see\textsc{.pfv.1sg}-for-\textsc{1sg} miracle \\
\glt `I saw a miracle.' 
\z

\ea\label{Khuzestani:ex:23}
Khuzestani Arabic \il{Arabic (Gələt)!Khuzestani}(\citealt{leitnerArabic2021}: C, 0392) \\
\gll əl-ḥaywāna nəḥlib-ha \\
\textsc{def-}animal\textsc{.sg} milk\textsc{.ipfv.1pl-3sg.f} \\
\glt `We milk the cattle.' 
\z

\ea\label{Khuzestani:ex:24}
Khuzestani Arabic \il{Arabic (Gələt)!Khuzestani}(\citealt{leitnerArabic2021}: A, 0100) \\
\gll ləbasne əxwīəṣāt-ne ləbasne əḥžīəlāt-ne yaʕni šīəla-t balbūl ləbasne yaʕni əṭ-ṭōg u-māṣxa ləbasne \\
wear\textsc{.pfv.1pl} ring\textsc{.dim.pl-1pl} wear\textsc{.pfv.1pl} bracelet\textsc{.dim.pl-1pl} \textsc{dm} shawl\textsc{.dim-cs} balbūl wear\textsc{.pfv.1pl} \textsc{dm} \textsc{def-}necklace and-māsxa wear\textsc{.pfv.1pl} \\
\glt `… we put on our rings, we put our bracelets, … we wore the \textit{balbūl} shawl{\footnotemark}, [and] the necklace and \textit{māsxa} [kind of jewelry].' 
\z

\footnotetext{A thin shawl, lit. ``made of (the material) \textit{balbūl}'', cf. \citet[179]{steingass2001} on the Persian\il{Persian (New)} term \textit{bulbul čašm} `a sort of silk'.}

\subsection{Pronominal direct object/verb }\label{Khuzestani:ss:2.9}

Pronominal direct objects are generally suffixed to the verb and thus inherently postverbal as in (\ref{Khuzestani:ex:25}) and (\ref{Khuzestani:ex:26}). Only in cases where the speaker wants to express additional \isi{emphasis} and/or mark it as the \isi{topic} of an utterance (cf. \citealt{brustad2000}: 331, 333 on comparable examples of independent pronouns that are sentence-initial and the \isi{topic} but not subject of a sentence), an independent \isi{pronoun} may additionally be mentioned preceding the verb with the suffixed \isi{pronoun}, see example (\ref{Khuzestani:ex:27}) (and the discussion of such examples in Section \ref{Khuzestani:ss:3}). 

\ea\label{Khuzestani:ex:25}
Khuzestani Arabic \il{Arabic (Gələt)!Khuzestani}(\citealt{leitnerArabic2021}: C, 0356) \\
\gll təsmaʕ-ni \\
hear\textsc{.ipfv.2sg.m-1sg} \\
\glt `You hear me.' 
\z

\ea\label{Khuzestani:ex:26}
Khuzestani Arabic \il{Arabic (Gələt)!Khuzestani}(own data) \\
\gll hāy ətfəhm-əč səʔli-ha suʔāl \\
\textsc{dem} understand\textsc{.ipfv.3sg.f-2sg.f} ask\textsc{.imp.sg.f-3sg.f} question \\
\glt `She understands you. Ask her a question!' 
\z

\ea\label{Khuzestani:ex:27}
Khuzestani Arabic \il{Arabic (Gələt)!Khuzestani}(\citealt{leitnerArabic2021}: C, 0412) \\
\gll waḷḷa āna iyā-ni hād əl-bīəhdāš{\footnotemark} māl salf-i \\
by\_god \textsc{1sg} come\textsc{.pfv.3sg.m-1sg} \textsc{dem.sg.m} \textsc{def-}healthcare\_center \textsc{gl.sg.m} district\textsc{-1sg} \\
\glt `And I – He came to me, [from] this healthcare center of my district...' 
\z

\footnotetext{< Pers. \textit{behdāšt} `hygiene, healthcare' (\citealt{junker2002}: 108).}

\subsection{Goal/verb}\label{Khuzestani:ss:2.10}

The default position of goals is post-predicate (\ref{Khuzestani:ex:28}).

\ea\label{Khuzestani:ex:28}
Khuzestani Arabic \il{Arabic (Gələt)!Khuzestani}(\citealt{leitnerArabic2021}: C, 0464) \\
\gll ham rəḥna ən-naxal rəḥna l-əš-šilib \\
also go\textsc{.pfv.1pl} \textsc{def-}palm\_groves go\textsc{.pfv.1pl} to\textsc{-def-}rice\_fields \\
\glt `We also went to the palm groves, we went to the rice fields…' 
\z

From the 83 goals in the WOWA-corpus only five were in pre-predicate position: three times the \isi{adverb} \textit{hnā} `here' as in (\ref{Khuzestani:ex:29}), once \textit{əb-baṭn-a} `in its belly' (\ref{Khuzestani:ex:30}) and once the \textsc{1sg} \isi{pronoun} \textit{āna}, which is however indicated as well by a pronominal affix on the verb and added sentence-initially for \isi{emphasis} and marking it as a sentence \isi{topic} (see example (\ref{Khuzestani:ex:38}) below and the discussion on whether it should be considered a pre-predicate token in Section \ref{Khuzestani:ss:3}). 

\ea\label{Khuzestani:ex:29}
Khuzestani Arabic \il{Arabic (Gələt)!Khuzestani}(\citealt{leitnerArabic2021}: A, 0091) \\
\gll wa hnā yō ḥaṭṭan warde yō əzmām \\
and here or put\textsc{.pfv.3pl.f} nose\_ring or nose\_ring \\
\glt `And here they put a \textit{warde} or a \textit{zmām} [two types of nose rings].' 
\z

\ea\label{Khuzestani:ex:30}
Khuzestani Arabic \il{Arabic (Gələt)!Khuzestani}(\citealt{leitnerArabic2021}: F, 0739) \\
\gll nṭēḥ bə-diyāy əb-baṭn-a yḥuṭṭūn šwayyūn təmən w kəšməš w fəlfəl ʔaswad w l-ḥawār \\
fall\textsc{.ipfv.1pl} with-chicken in-belly\textsc{-3sg.m} put\textsc{.ipfv.3pl.m} some rice and raisins and pepper black and \textsc{def}-spices \\
\glt `We take the chicken, in its belly we put some rice and raisins and black pepper and spices.' 
\z

\subsection{Other obliques/verb}\label{Khuzestani:ss:2.11}
In the WOWA-corpus, of all 25 obliques labelled as ``other'' (which are mostly adverbs) 6 were found in pre-predicate position as in the following example (\ref{Khuzestani:ex:31}): 

\ea\label{Khuzestani:ex:31}
Khuzestani Arabic \il{Arabic (Gələt)!Khuzestani}(\citealt{leitnerArabic2021}: A, 0095) \\
\gll waḷḷa dāyman hēč mā nilbas \\
by\_god always like\_this \textsc{neg} dress\textsc{.ipfv.1pl} \\
\glt `It's not always that we dress like this.' 
\z

\subsection{Copular and become-constructions}\label{Khuzestani:ss:2.12}

In general, there is no present tense \isi{copula} in Arabic. Several dialects have, as a consequence of contact with languages that do have obligatory present tense copulas, developed an obligatory \isi{copula} for the imperfective (cf. e.g. \citealt{prochazka2019} on the dialects of Eastern Anatolia), but KhA has not despite its long term contact with Persian\il{Persian (New)}. 

\begin{sloppypar}
The WOWA-corpus features one imperfective copular-\isi{complement} in pre-predicate position (out of 17 copular constructions\footnote{Of these 17 copular constructions, 15 copulas were imperfective forms and two were perfective forms; 14 copulas were forms of \textit{ṣār} lit. `to become' and only three were based on the lexeme \textit{kān} `to be'.}), (\ref{Khuzestani:ex:32}), and two pre-predicate become-complements (out of 28), one of them is cited here in (\ref{Khuzestani:ex:33}) (\textit{ha-l-gadd-āt-ha tṣīr} `it becomes about this size'). The latter structure again might be due to \isi{contact influence} or due to \isi{information packaging} and pragmatic reasons, as it seems to be an echo or a recall of the previous \textit{yṣīr čibīr-e} `it becomes big'.
\end{sloppypar}

\begin{sloppypar}
Even though in the WOWA-corpus we find only one attestation of pre-predicate copulas, we know from previous studies that they are more often found in sentence-final position than other verbs in KhA (see \citealt{leitnerClause-final2022}). 
\end{sloppypar}

\ea\label{Khuzestani:ex:32}
Khuzestani Arabic \il{Arabic (Gələt)!Khuzestani}(\citealt{leitnerArabic2021}: B, 0278) \\
\gll hāḏanni ṭəlyān ṭəli yṣīr \\
\textsc{dem.pl.f} lamb\textsc{.pl} lamb \textsc{cop} \\
\glt `These are \textit{ṭəlyān} [``lambs''], \textit{ṭəli} [``lamb''] we call it.'
\z

\ea\label{Khuzestani:ex:33}
Khuzestani Arabic \il{Arabic (Gələt)!Khuzestani}(\citealt{leitnerArabic2021}: D, 0508) \\
\gll hāy əl-faḥla yṣīr čibīr-e w hāy ən-naxla la zəġīr-e eh zəġīr-e taġrīban ha-l-gadd-āt-ha tṣīr nəfšəg gadd čaff īd hāy waḥda-t ən-naṯye \\
\textsc{dem} \textsc{def-}male\_palm become\textsc{.ipfv.3sg.m} big\textsc{-f} and \textsc{dem} \textsc{def-}palm no small\textsc{-f} yes small\textsc{-f} about \textsc{dem-def-}size\textsc{-pl.f-3sg.f} become\textsc{.ipfv.3sg.f} split\textsc{.ipfv.1pl} size palm hand \textsc{dem} thing\textsc{-cs} \textsc{def}-female\_palm \\
\glt `This male palm grows big, but this palm not, small, yes small, it becomes about this size, we split – as [big as] a hand – this thing of the female palm.' 
\z

\section{Areal issues and information structure}\label{Khuzestani:ss:3}

As stated in the introduction, KhA is not entirely isolated from the rest of the Arabic-speaking world due to its border with Iraq, although it is spoken in a region in which the sociolinguistically dominant language is not Arabic but Persian\il{Persian (New)}. The use of KhA is mostly restricted to conversations within the family and among friends. 

To the best of my knowledge, we do not have a detailed survey on \isi{word order} in (Muslim/\textit{gələt}) Iraqi Arabic\il{Arabic (Gələt)!Iraqi}, with which KhA is closely related. But since the overall picture of KhA \isi{word order} shows that most \isi{word order} features are most likely inherited, as the post-predicate position of objects and other complements is the default position, we can assume that KhA in this regard does not deviate much from its neighbor Iraqi Arabic\il{Arabic (Gələt)!Iraqi}. Against this background, this Section will \isi{focus} on those cases of non-default \isi{word order} trying to propose possible explanations. The following thus is a brief outline of some of the factors that may trigger pre-predicate position in KhA and discusses whether changes in \isi{word order} and use of marked \isi{word order} are likely to result from contact with Persian\il{Persian (New)} or are rather to be explained as information structural strategies. The latter builds on the theoretical approaches of \citet[315--362]{brustad2000} and her analysis of \isi{information packaging} and its influences on \isi{word order} in spoken Arabic and \citet{ingham1991} and his analysis of KhA sentence structure. Ingham proposes a basic division of sentence types into i) \isi{uninodal}, in which new information comes first and which are usually verb-initial unless there is \isi{focus} fronting (e.g. \ref{Khuzestani:ex:36}), and ii) \isi{binodal}, in which the \isi{topic} precedes new information/the comment. Example (\ref{Khuzestani:ex:34}) illustrates the \isi{binodal} sentence type (\citealt{ingham1991}: 722), where the \isi{object} is fronted (fronted objects are usually given information and definite nouns\footnote{Cf.
    \citet[339]{brustad2000} and the examples provided there on the fact that \isi{topic} in Arabic also includes temporal verbs.
}) is topical and indexed by a resumptive \isi{pronoun} marking the original post-verbal position of the \isi{object}. This example clearly shows how putting an \isi{object} in pre-predicate position can be used as an information structural tool to indicate the \isi{topic} of a sentence (cf. \citealt{brustad2000}: 348--349). The phrase \textit{hāy əd-dār} `this house' already appears earlier on in the sentence, but in each instance the speaker clearly refers to another room and the one following the conjunction \textit{bass} `but' is the one made the \isi{topic} followed by the new information, vic. that it should not be opened. After the preceding rhythmic enumeration (``You may open this house, and you may open this room, and you may open this room''), the part after \textit{bass} is also accompanied by a differing \isi{intonation} contour (pitch goes up with \textit{dār}), and a short pause before she goes on saying \textit{lā thəddin-ha} `(this one) don't you open it'. Within the latter phrase, we find the feminine singular \isi{object} \isi{pronoun} \textit{-ha} referring back to the \isi{object} \textit{hāy əd-dār} `this house', thus the construction resembles that of a normal \isi{topicalization} structure as found in all varieties of Arabic.

\ea\label{Khuzestani:ex:34}
Khuzestani Arabic \il{Arabic (Gələt)!Khuzestani}(\citealt{leitnerArabic2021}: E, 0635) \\
\gll gāl-ha thəddīn hāḏ əl-bīət w thəddīn hāy əd-dār w thəddīn hāy əd-dār bass hāy əd-dār lā thəddīn-ha \\
tell\textsc{.pfv.3sg.m-3sg.f} open\textsc{.ipfv.2sg.f} \textsc{dem} \textsc{def-}house and open\textsc{.ipfv.2sg.f} \textsc{dem} \textsc{def-}room and open\textsc{.ipfv.2sg.f} \textsc{dem} \textsc{def-}room but \textsc{dem} \textsc{def-}room \textsc{neg} open\textsc{.ipfv.2sg.f-3sg.f} \\
\glt `He said: ``You [may] open this house, and you [may] open this room, and you [may] open this room, but don't you open this room.''' 
\z

Out of the 43 \isi{OV} instances in the KhA data from the WOWA-corpus, 27 featured a resumptive \isi{pronoun} on the verb as in the example above or in example (\ref{Khuzestani:ex:23}). As stated above, this structure is not foreign to Arabic and must not be attributed to \isi{contact influence}. Whether or not such topicalized sentences appear more commonly in Arabic dialects that start shifting towards \isi{OV} due to contact with an \isi{OV} language has yet to be determined.

The WOWA-corpus of KhA contains 16 \isi{OV} phrases that do not feature a resumptive \isi{pronoun} on the verb such as (\ref{Khuzestani:ex:24}) above cited here again (\ref{Khuzestani:ex:35}) for the sake of the discussion:

\ea\label{Khuzestani:ex:35}
Khuzestani Arabic \il{Arabic (Gələt)!Khuzestani}(\citealt{leitnerArabic2021}: A, 0100) \\
\gll ləbasne əxwīəṣāt-ne ləbasne əḥžīəlāt-ne yaʕni šīəla-t balbūl ləbasne yaʕni əṭ-ṭōg u-māṣxa ləbasne \\
wear\textsc{.pfv.1pl} ring\textsc{.dim.pl-1pl} wear\textsc{.pfv.1pl} bracelet\textsc{.dim.pl-1pl} \textsc{dm} shawl\textsc{.dim-cs} balbūl wear\textsc{.pfv.1pl} \textsc{dm} \textsc{def-}necklace and-māsxa wear\textsc{.pfv.1pl} \\
\glt `… we put on our rings, we put our bracelets, … we wore the \textit{balbūl} shawl, [and] the necklace and the \textit{māsxa} [kind of jewelry].' 
\z

In this example, in which the speaker (rhythmically) lists various items that women used to wear in the past for weddings, she switches from the default \isi{VO} structure to the marked \isi{OV} in the middle of the sentence (note that this turn is introduced by the discourse marker \textit{yaʕni}). The shift in \isi{word order} is paralleled by a shifted stress distribution in the second (\isi{OV}) part of the sentence where in both cases the speaker puts the main stress on the verb \textit{ləbasne} `we wore', whereas in the first part the main stress lies on the objects. Information packaging seems to be the most likely cause for such a change towards marked \isi{word order}. According to \citet[343]{brustad2000} ``objects that are contrastive may occupy pre-verbal position (OVS)'' and later on adds that ``objects without resumptive pronouns are highly contrastive'' \citep[348]{brustad2000}. In this example, such a \isi{contrast} may lie in the shift from the event-oriented first part (with a \isi{focus} on the habitual event, vic. what they used to wear) towards a topic-oriented second part (with a \isi{focus} on items that are part of a set of things they used to wear). In \citegen{ingham1991} terminology, the latter part of this sentence seems to be of the type ``\isi{uninodal} with \isi{focus} fronting'' (\citealt{ingham1991}: 721–722) and is thus also not new or undocumented for Arabic dialects. However, against the definition of this sentence type, the nuclear stress in this very example does not fall on the fronted item but on the following verb. It remains unclear, why the speaker once marks a pre-predicate \isi{object} with the definite article (\textit{əṭ-ṭōg}, albeit the only pre-predicate \isi{direct object} with definite article in the WOWA corpus) and once without (\textit{māṣxa}), as both seem to have a generic character representing a set of items. In general this example does not seem to support \citegen{ingham1991} assumption that possibly a ``definite `true' \isi{object}'' cannot appear in KhA in preverbal position without a resumptive \isi{pronoun} (\citealt{ingham1991}: 722, Fn. 5).

Another instance of a pre-predicate \isi{object} (\textit{kəllšī} `everything') without resumptive \isi{pronoun} is the following (\ref{Khuzestani:ex:36}): 

\ea\label{Khuzestani:ex:36}
Khuzestani Arabic \il{Arabic (Gələt)!Khuzestani}(\citealt{leitnerArabic2021}: C, 0363) \\
\gll ḥəṣadna lammēna təbən lammēna ǧanēna ḥaywān kəllšī sawwēna ya ʕazīz galb-i \\
harvest\textsc{.pfv.1pl} gather\textsc{.pfv.1pl} straw gather\textsc{.pfv.1pl} breed\textsc{.pfv.1pl} cattle everything do\textsc{.pfv.1pl} \textsc{voc} dear heart\textsc{-1sg} \\
\glt `We harvested, gathered straw, we gathered – we bred cattle, we made everything, my dear.' 
\z

This again is a \isi{uninodal} sentence with \isi{focus} fronting, in this case also fulfilling the requirement of heavy stress on the fronted \isi{object} (\textit{kəllšī} `everything'). The indefinite pronoun\is{pronoun!indefinite} \textit{kəllšī} appears three times in the WOWA-corpus and always in pre-predicate position. It may also precede negated verbs as in the following example (\ref{Khuzestani:ex:37}), in which we find additionally a fronted and topicalized \isi{object} \textit{əl-ʕarūs} (indexed by the resumptive \isi{pronoun} \textit{-hən} on the verb):

\ea\label{Khuzestani:ex:37}
Khuzestani Arabic \il{Arabic (Gələt)!Khuzestani}(\citealt{leitnerArabic2021}: A, 0072) \\
\gll əl-ʕarūs kəllšī mā nsawwī-l-hən \\
\textsc{def-}bride everything \textsc{neg} make\textsc{.ipfv.1pl-dat-3pl.f} \\
\glt `The bride[s], we didn't do anything with them [like putting on henna, etc.].' 
\z

As stated above for example (\ref{Khuzestani:ex:8}), where \textit{kəllšāy} `everything' precedes a negative possessive-construction, these structures may be calques on (Spoken) Persian\il{Persian (colloquial)} structures such as \textit{hičči na-kardam} `we did nothing' and \textit{har kari kardim} `we did everything'. Their existence in Iraqi Arabic\il{Arabic (Gələt)!Iraqi} (see Section \ref{Khuzestani:ss:2.2} above) might speak against this, as influence of Persian\il{Persian (New)} on Iraqi Arabic\il{Arabic (Gələt)!Iraqi} is mostly restricted to the lexical domain. Taking all this under consideration, it seems most likely to be an inherited structure in KhA, \isi{focus} fronting of \textit{kəllšāy} $\sim$ \textit{kəllšī} that indicates \isi{contrast}\footnote{ In example (\ref{Khuzestani:ex:37}) the speaker was clearly contrasting wedding traditions of the past and the present.}, or, as in (\ref{Khuzestani:ex:36}), some kind of closure of an enumeration (a category or context not mentioned by Brustad or Ingham). Of course, Persian\il{Persian (New)} \isi{word order} might have reinforced the use of this structure and increased the frequency of fronting of \textit{kəllšī}. 

The following sentence (\ref{Khuzestani:ex:38}) (already cited above as (\ref{Khuzestani:ex:27}) and repeated here for the sake of convenience) is most likely also of the \isi{binodal} type with the \textsc{1sg} \isi{pronoun} \textit{āna} presenting the sentence \isi{topic} and being taken up by a referential \isi{pronoun} on the verb.

\ea\label{Khuzestani:ex:38}
Khuzestani Arabic \il{Arabic (Gələt)!Khuzestani}(\citealt{leitnerArabic2021}: C, 0412) \\
\gll waḷḷa āna iyā-ni hād əl-bīəhdāš{\footnotemark} māl salf-i \\
by\_god \textsc{1sg} come\textsc{.pfv.3sg.m-1sg} \textsc{dem.sg.m} \textsc{def-}healthcare\_center \textsc{gl.sg.m} district\textsc{-1sg} \\
\glt `And I – He came to me, [from] this healthcare center of my district ...' 
\z

\footnotetext{< Pers. \textit{behdāšt} `hygiene, healthcare' (\citealt{junker2002}: 108).}

A feature that can more likely by attributed to \isi{contact influence} is the sentence-final position of auxiliaries (cf. Section \ref{Khuzestani:ss:2.6} for examples), copulas and the verb `to become' (cf. Section \ref{Khuzestani:ss:2.12} for examples). This fits very well into the stages of shift towards XV structures as described by \citet{ElZarkaZiagos2020WOCA} for Southern Iranian Arabic\il{Arabic (Gələt)!Southern Iranian}. There, this shift seems to be more advanced than in KhA, but in both varieties it appears to have started with elements such as copulas and auxiliaries (see \citealt{leitnerClause-final2022}). 

We can thus conclude that, overall, the data and its analysis clearly show that the inherited default \isi{word order} (VX) is retained in Khuzestani Arabic\il{Arabic (Gələt)!Khuzestani}. However, pragmatic factors related primarily to \isi{information structure} and to a much lesser degree contact with Persian\il{Persian (New)} may cause that elements are moved to pre-predicate position, thus yielding XV \isi{word order}. For future studies on KhA \isi{word order} and \isi{information packaging}, it would be interesting to further include a comparison of \isi{information structure} in Persian\il{Persian (New)} and KhA.


\section*{Abbreviations}
\begin{tabularx}{.45\textwidth}{lQ}
\textsc{aux} & {auxiliary} \\
\textsc{coll} & collective (noun) \\
\textsc{cop} & {copula} (noun) \\
\textsc{cs} & construct state \\
\textsc{dat} & {dative} \\
\textsc{def} & definite (article) \\
\textsc{dem} & demonstrative \\
\textsc{dim} & diminutive \\
\textsc{dm} & discourse marker \\
\textsc{exist} & existential particle \\
\textsc{f} & feminine \\
\textsc{gl} & genitive linker \\
\textsc{hort} & hortative particle \\
\textsc{imp} & imperative \\
\end{tabularx}
\begin{tabularx}{.45\textwidth}{lQ}
\textsc{ipfv} & imperfective \\
KhA & Khuzestani Arabic \\
\textsc{m} & masculine \\
MSA & Modern Standard Arabic \\
\textsc{neg} & negation \\
Pers. & Persian \\
\textsc{pfv} & perfective \\
\textsc{pl} & plural \\
\textsc{pro} & {pronoun} \\
\textsc{prog} & progressive marker \\
\textsc{pst} & past tense \\
\textsc{sg} & singular \\
\textsc{voc} & vocative particle  \\
\\
\end{tabularx}

% \section*{Acknowledgements}





{\sloppy\printbibliography[heading=subbibliography,notkeyword=thi]}

\end{document}
