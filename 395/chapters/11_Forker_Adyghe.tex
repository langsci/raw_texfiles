\documentclass[output=paper,colorlinks,citecolor=brown, draft]{langscibook}
\ChapterDOI{10.5281/zenodo.14266351}
\author{Diana Forker\orcid{0000-0003-4247-9163}\affiliation{Friedrich Schiller University Jena}}
\title{Post-predicate elements in Adyghe}
\abstract{In this chapter, I study post-predicate elements in the Northwest Caucasian language Adyghe\il{Circassian!Adyghe}. In the literature, Adyghe\il{Circassian!Adyghe} is characterized as having SOV as its basic pattern, but as being in principle a ``free'' word order language. There are no corpus-based studies on word order in Adyghe\il{Circassian!Adyghe} (or any other Northwest Caucasian language) up to now, so this study is a step towards filling this research gap.

I first examine examples of post-predicate elements in the literature on Adyghe\il{Circassian!Adyghe}, which confirm the expectations and exemplify various types of arguments and adjuncts as well as subordinate clauses that can appear after the verb. In a second step, I identify and count post-predicate elements in 20 Adyghe\il{Circassian!Adyghe} texts collected between 1969 and 2017 among the Adyghe\il{Circassian!Adyghe} people in the Caucasian homeland and the Turkish\il{Turkic!Turkish} diaspora by various researchers. Only around 10\% of the main clauses contain post-predicate elements of which the majority are pragmatic particles, but post-predicate subject, objects and adjuncts are also attested. Most post-predicate referents are topical, but focal referents can also be found. Differences in genre play a relatively big role: personal accounts contain around as twice as many instances of post-predicate elements than traditional narratives. Furthermore, the texts from Adygea show a greater frequency of post-predicate elements than those from Turkey which might be due to the influence of two typologically and genealogically different contact languages (Russian\il{Russian} and Turkish\il{Turkic!Turkish}).
}

%move the following commands to the ``local...'' files of the master project when integrating this chapter
% \usepackage{tabularx}
% \usepackage{langsci-optional}
% \usepackage{langsci-gb4e}
% \usepackage{enumitem}
% \bibliography{localbibliography}
% \newcommand{\orcid}[1]{}
% \let\eachwordone=\itshape
\IfFileExists{../localcommands.tex}{
 \addbibresource{../collection_tmp.bib}
 \addbibresource{../localbibliography.bib}
 \usepackage{langsci-optional}
\usepackage{langsci-gb4e}
\usepackage{langsci-lgr}

\usepackage{listings}
\lstset{basicstyle=\ttfamily,tabsize=2,breaklines=true}

%added by author
% \usepackage{tipa}
\usepackage{multirow}
\graphicspath{{figures/}}
\usepackage{langsci-branding}

 
\newcommand{\sent}{\enumsentence}
\newcommand{\sents}{\eenumsentence}
\let\citeasnoun\citet

\renewcommand{\lsCoverTitleFont}[1]{\sffamily\addfontfeatures{Scale=MatchUppercase}\fontsize{44pt}{16mm}\selectfont #1}
  
 %% hyphenation points for line breaks
%% Normally, automatic hyphenation in LaTeX is very good
%% If a word is mis-hyphenated, add it to this file
%%
%% add information to TeX file before \begin{document} with:
%% %% hyphenation points for line breaks
%% Normally, automatic hyphenation in LaTeX is very good
%% If a word is mis-hyphenated, add it to this file
%%
%% add information to TeX file before \begin{document} with:
%% %% hyphenation points for line breaks
%% Normally, automatic hyphenation in LaTeX is very good
%% If a word is mis-hyphenated, add it to this file
%%
%% add information to TeX file before \begin{document} with:
%% \include{localhyphenation}
\hyphenation{
affri-ca-te
affri-ca-tes
an-no-tated
com-ple-ments
com-po-si-tio-na-li-ty
non-com-po-si-tio-na-li-ty
Gon-zá-lez
out-side
Ri-chárd
se-man-tics
STREU-SLE
Tie-de-mann
}
\hyphenation{
affri-ca-te
affri-ca-tes
an-no-tated
com-ple-ments
com-po-si-tio-na-li-ty
non-com-po-si-tio-na-li-ty
Gon-zá-lez
out-side
Ri-chárd
se-man-tics
STREU-SLE
Tie-de-mann
}
\hyphenation{
affri-ca-te
affri-ca-tes
an-no-tated
com-ple-ments
com-po-si-tio-na-li-ty
non-com-po-si-tio-na-li-ty
Gon-zá-lez
out-side
Ri-chárd
se-man-tics
STREU-SLE
Tie-de-mann
}
%  \boolfalse{bookcompile}
%  \togglepaper[5]%%chapternumber
}{}

\usepackage{xcolor}
\definecolor{C1}{RGB}{0,0,0}
\definecolor{C2}{RGB}{46,37,133}
\definecolor{C3}{RGB}{51,117,56}
\definecolor{C4}{RGB}{93,168,153}
\definecolor{C5}{RGB}{148,203,236}
\definecolor{C6}{RGB}{148,203,0}
\begin{document}
\maketitle\label{WOWA:ch:11}

\section{Introduction}\label{Adyghe:ss:1}

In this contribution, I outline the main facts of \isi{word order} in Adyghe\il{Circassian!Adyghe} and then analyze in more detail post-predicate elements.

Adyghe\il{Circassian!Adyghe} belongs to the Circassian branch of the Northwest Caucasian language family. The homeland of the Adyghe\il{Circassian!Adyghe} people is the northwestern Caucasus region (Russian\il{Russian} Federation). Since the conquest of the Caucasus by Russia in the second half of the 19th century there are diaspora communities in Turkey, Jordan, Israel, Syria and other countries. 

% \setlength{\footheight}{49.46017pt}
\largerpage

The main goal of my contribution is to identify first of all grammatically acceptable post-predicate elements and examine them with respect to their syntactic functions, grammatical roles and their information structural properties. My second goal is to study post-predicate elements in natural texts by comparing data from the Adyghe\il{Circassian!Adyghe} homeland with data from the diaspora community in Turkey. Adyghe\il{Circassian!Adyghe} diaspora communities in Turkey (and other places) have been relatively isolated from the original speech community for more than 150 years. The two communities are under the influence of two distinct languages of wider communication, Turkish\il{Turkic!Turkish} and Russian\il{Russian} respectively, and the data suggest that these differing contact scenarios have led to divergence.

Like all Northwest Caucasian languages, Adyghe\il{Circassian!Adyghe} is polysynthetic with highly complex verbal morphology. Verb forms contain pronominal prefixes indexing all syntactic arguments of the predicate, i.e. intransitive and transitive subjects, \isi{direct object}, indirect object\is{object!indirect}, etc. Parts of speech are not always clearly differentiated in terms of \isi{inflection}. A wide range of grammatical markers for person, tense, number, modality, and negation can be added to any content word. Adyghe\il{Circassian!Adyghe} has \isi{ergative} alignment that shows up in case marking and agreement. The suffix \textit{-r} (``\isi{absolutive}'') marks intransitive subjects and direct objects. The suffix \textit{-m} (``\isi{oblique}'') marks transitive subjects (agents), as well as indirect objects, certain adverbials (temporal, spatial), and adnominal possessors. Nonspecificity and indefiniteness are indicated by the omission of case suffixes. Proper nouns and first and second pronouns do not distinguish \isi{absolutive} and \isi{oblique}. Clauses contain at least a predicate, which can be verbs, nouns, pronouns, adjectives or even postpositions. Predicates take pronominal prefixes and tense morphology. Copula clauses consist of a \isi{copula} \isi{complement} and the \isi{copula} verb. Overt \isi{argument} NPs are optional as it is expected for polysynthetic languages (\citealt{testelets2017adyghe}). This property makes the study of \isi{word order} patterns at the clausal level based on natural texts somewhat difficult because arguments are recurrently only expressed through pronominal prefixes.

Major grammatical descriptions of Adyghe\il{Circassian!Adyghe} are \citet{jakovlev1941grammatika}, \citet{rogava1966grammatika} and \citet{arkadiev2009aspekty}. \citet{kumakhov2009circassian} analyze Circassian clause structure, including \isi{word order}. There are no corpus-based studies on \isi{word order} in Adyghe\il{Circassian!Adyghe}. But there is a corpus of Standard Adyghe\il{Circassian!Adyghe} (\citealt{arkhangelskiy2018west}), available on the Internet (\url{http://adyghe.web-corpora.net/}).

\newpage
\begin{sloppypar}
This paper is mainly based on 20 Adyghe\il{Circassian!Adyghe} texts (6,146 words) recorded in Adygea (Caucasus) and Turkey.\footnote{
There is no WOWA-data set for Adyghe\il{Circassian!Adyghe}, or any other Northwest Caucasian language to date.
} All examples are marked by [H] for `homeland' and [D] for `diaspora.' Some of these texts have been published in \citet{hohlig1997kontaktbedingter}, \citet{paris1974princesse} and \citet{feer2019grammar}, and the texts from \citet{paris1974princesse} can also be found in the online Pangloss Collection (see references). Other texts collected by Monika Höhlig in the 1990s and by Feer between 2016 and 2017 were kindly provided to me by both researchers. The texts are monologues that can be roughly divided into two types, namely (i) traditional narratives such as legends, fairy tales and anecdotes and (ii) personal accounts/autobiographies. Following \citet{hohlig1997kontaktbedingter}, I also classified the texts according to the age of the speakers into old, middle and young generation. The texts have been chosen such as to roughly equally represent Adyghe\il{Circassian!Adyghe} from the homeland in contact with Russian\il{Russian} and Adyghe\il{Circassian!Adyghe} from the diaspora in contact with Turkish\il{Turkic!Turkish}. A further criterion was genre. 
\end{sloppypar}

A full list with sources can be found in the appendix.

\section{Word order profile of Adyghe\il{Circassian!Adyghe}}\label{Adyghe:ss:2}

\subsection{Word order patterns in noun phrases and other constituents}\label{Adyghe:ss:2.1}

In this section and the following section, I present an overview of \isi{word order} patterns in Adyghe\il{Circassian!Adyghe} noun phrases and clauses including a few examples with post-predicate elements. A more detailed discussion of post-predicate elements in texts will be given in Section \ref{Adyghe:ss:3}. 

Constituent order within the NP is mixed. Adjectives, simple cardinal numerals except for the numeral `one' and resultative verbs follow the noun. The numerals are suffixed by means of a linking morpheme as in the following example (\ref{Adyghe:ex:1}).

\ea\label{Adyghe:ex:1}
noun + adjective-numeral + \isi{adjective} \\
Adyghe \il{Circassian!Adyghe}(courtesy of Y. Lander) [H] \\
\gll ha ʁʷež'-jə-ṭʷ gʷere \\
dog yellow\textsc{-lnk}-two certain \\
\glt `two certain yellow dogs' 
\z

Demonstratives (\ref{Adyghe:ex:2}), non-referential modifying nouns and appositive names (\ref{Adyghe:ex:3}), relative clauses (\ref{Adyghe:ex:4}), possessors including \isi{possessive} prefixes (\ref{Adyghe:ex:2}), (\ref{Adyghe:ex:3}), the cardinal numeral `one' (\ref{Adyghe:ex:4}) and ordinal numerals precede the nominal head. \citet{lander2017nominal} labels the prenominal modifiers ``non-adjectival'' and notes that they are ungradable. Postnominal modifiers can but need not be gradable. Very commonly modifiers enter into a close connection with the modified noun and are pronounced and written together such as the noun and the \isi{adjective} in the second noun phrase `his grandmother' in example (\ref{Adyghe:ex:2}). These units are called ``nominal complex'' in \citet{lander2017nominal}, who argues that they form a single word, based on their morphosyntactic properties. For instance, case marking and plural marking occur only once per unit (\ref{Adyghe:ex:13}).

\ea\label{Adyghe:ex:2}
demonstrative + noun + \isi{adjective} and possessive-noun-\isi{adjective}, postposition \\
Adyghe \il{Circassian!Adyghe}(courtesy of M. Höhlig) [H] \\
\gll [mwe č̣'ele c̣əč̣'ə-r] [ja-ne-ẑ] djə qə-zə-ḳʷe-č̣'e ... \\
that boy little\textsc{-abs} \textsc{poss-}mother-old to \textsc{dir-}\textsc{rel.temp}-go\textsc{-inst} \\
\glt `when that little boy went to his grandmother ...'
\z

\ea\label{Adyghe:ex:3}
\isi{possessive} construction, \isi{postpositional} phrase \\
Adyghe \il{Circassian!Adyghe}\citep{paris1974princesse} [D] \\
\glll [[qahraman gʷaše-m] jə-šə-šxa-p̣ʷe] jə-dež'-g'e \\
Kahraman princess\textsc{-obl} \textsc{poss-}horse-eat-place \textsc{poss-}to\textsc{-inst} \\
possessor {} possessed postposition \\
\glt `to the manger of the horses of princess Kahraman'
\z

\ea\label{Adyghe:ex:4}
relative clause + numeral-noun \\
Adyghe \il{Circassian!Adyghe}\citep{paris1974princesse} [D] \\
\gll dečːʼəɣəməqːʷe pšəpəjə-r [...] [xeʁegʷə-m jə-sə-xe-me ʔape-g'e qː-ja-ʁa-λaʁʷe-w] zə-c̣əfə-ʁ \\
Detcheghemeqo Pshepeye\textsc{-abs} {} country\textsc{-obl} \textsc{loc-}live\textsc{-pl-obl.pl} finger\textsc{-inst} \textsc{dir-}\textsc{3pl.A-}\textsc{caus-}see\textsc{-adv} one-human.being\textsc{-pst} \\
\glt `Detcheghemeqo Pshepeye [...] was a person whom the inhabitants (lit. `the ones living there') of the country respected.' (lit. `one human being that they pointed at with their fingers') 
\z

Adyghe\il{Circassian!Adyghe} has exclusively postpositions, which have mostly been grammaticalized from nouns (\citealt{arkadiev2018grammaticalization}). The \isi{postpositional} \isi{complement} is often additionally expressed via a \isi{possessive} prefix (\ref{Adyghe:ex:3}).

In complex verb forms, auxiliaries follow the lexical verb and some have already grammaticalized into suffixes (\citealt{kimmelman2011auxiliaries}, \citealt{arkadiev2018grammaticalization}). They mostly express aspectual meanings and epistemic modality (probability and necessity) and can also occur in conditional clauses. In the following example, the \isi{auxiliary} verb \textit{χʷə-} `be, happen' bears the conditional suffix \textit{-me} and forms a complex verb together with the lexical verb \textit{ʔʷе-} `say, speak, tell, talk'.

\ea\label{Adyghe:ex:5}
Adyghe \il{Circassian!Adyghe}(courtesy of M. Höhlig) [H] \\
\gll gʷəxeλ-ew jə-ʔe-r qə-mə-ʔʷa-xe χʷə-me, ... \\
intent\textsc{-adv} \textsc{loc-}be\textsc{-abs} \textsc{dir-neg-}speak\textsc{-trm} happen\textsc{-cond} \\
\glt `if (the guest) did not tell the intentions that he had ...'
\z

\subsection{Word order at the clausal level}\label{Adyghe:ss:2.2}

\begin{sloppypar}
At the level of the main clause, Adyghe\il{Circassian!Adyghe} is, as other Northwest languages, at the same time left-branching / verb-final, with SOV being considered as a kind of default pattern, though tolerating a fair degree of flexibility (\citealt{jakovlev1941grammatika}, \citealt{rogava1966grammatika}, \citealt{kumakhov2009circassian}, \citealt{lander2014relativizacija}, \citealt{testelets2017adyghe}). All logically possible orders are available. As typical for SOV languages, focal items occur in the preverbal position and contrastive items are said to occur sentence-initially (\citealt{arkadiev2021northwest}).
\end{sloppypar}

\citet[91]{jakovlev1941grammatika} list the following patterns for transitive verbs with overt subject, \isi{direct object}, and an \isi{adjunct} noun expressing location (\ref{Adyghe:ex:6}-\ref{Adyghe:ex:11a}, \ref{Adyghe:ex:11b}). Verb-final order with the subject preceding the \isi{direct object} (SOV) is analyzed as basic and neutral with respect to \isi{emphasis} and \isi{information structure} (\ref{Adyghe:ex:6}). The reversal of subject and \isi{direct object} (OSV) illustrated in (\ref{Adyghe:ex:7}) is characterized as also possible but less used. All other patterns are called ``inverse'' (\ref{Adyghe:ex:8}-\ref{Adyghe:ex:11a}, \ref{Adyghe:ex:11b}). They write that in verb-second patterns the final element is most highlighted and the penultimate element that immediately follows the verb is also highlighted, albeit to a lesser extent.

\ea
\ea\label{Adyghe:ex:6}
Adyghe \il{Circassian!Adyghe}\citep[91]{jakovlev1941grammatika} [H] \\
basic pattern: \textcolor{C1}{S}-\textcolor{C2}{O}-\textcolor{C3}{LOC}-\textcolor{C4}{V} [SOV] \\
\gll \textcolor{C1}{č̣'ale-me} \textcolor{C2}{baǯ'e-xe-r} \textcolor{C3}{mezə-m} \textcolor{C4}{š'a-λeʁʷə-ʁe-x} \\ 
boy\textsc{-obl.pl} fox\textsc{-pl}\textsc{-abs} forest\textsc{-obl} \textsc{loc-}see\textsc{-pst}\textsc{-pl} \\
\ex\label{Adyghe:ex:7}
\textcolor{C2}{O}-\textcolor{C1}{S}-\textcolor{C3}{\textcolor{C3}{LOC}}-\textcolor{C4}{\textcolor{C4}{V}} [OSV] \\
\gll \textcolor{C2}{baǯ'e-xe-r} \textcolor{C1}{č̣'ale-me} \textcolor{C3}{mezə-m} \textcolor{C4}{š'a-λeʁʷə-ʁe-x} \\
fox\textsc{-pl-abs} boy\textsc{-obl.pl} forest\textsc{-obl} \textsc{loc-}see\textsc{-pst}\textsc{-pl} \\
\ex\label{Adyghe:ex:8}
\textcolor{C1}{S}-\textcolor{C4}{V}-\textcolor{C2}{O}-\textcolor{C3}{LOC} [\isi{SVO}] \\
\gll \textcolor{C1}{č̣'ale-me} \textcolor{C4}{š'a-λeʁʷə-ʁe-x} \textcolor{C2}{baǯ'e-xe-r} \textcolor{C3}{mezə-m} \\
boy\textsc{-obl.pl} \textsc{loc-}see\textsc{-pst}\textsc{-pl} fox\textsc{-pl}\textsc{-abs}  forest\textsc{-obl}\\
\ex\label{Adyghe:ex:9}
\textcolor{C3}{LOC}-\textcolor{C4}{V}-\textcolor{C2}{O}-\textcolor{C1}{S} [VOS] \\
\gll \textcolor{C3}{mezə-m} \textcolor{C4}{š'a-λeʁʷə-ʁe-x} \textcolor{C2}{baǯ'e-xe-r} \textcolor{C1}{č̣'ale-me} \\
forest\textsc{-obl} \textsc{loc-}see\textsc{-pst}\textsc{-pl} fox\textsc{-pl}\textsc{-abs} boy\textsc{-obl.pl} \\
\ex\label{Adyghe:ex:10}
\textcolor{C2}{O}-\textcolor{C4}{V}-\textcolor{C3}{LOC}-\textcolor{C1}{S} [OVS] \\
\gll \textcolor{C2}{baǯ'e-xe-r} \textcolor{C4}{š'a-λeʁʷə-ʁe-x} \textcolor{C3}{mezə-m} \textcolor{C1}{č̣'ale-me} \\
fox\textsc{-pl-abs} \textsc{loc-}see\textsc{-pst}\textsc{-pl} forest\textsc{-obl} boy\textsc{-obl.pl} \\
\ex\label{Adyghe:ex:11a}
\textcolor{C4}{V}-\textcolor{C1}{S}-\textcolor{C2}{O}-\textcolor{C3}{LOC}[\isi{VSO}] \\
\gll \textcolor{C4}{š'a-λeʁʷə-ʁe-x} \textcolor{C1}{č̣'ale-me} \textcolor{C2}{baǯ'e-xe-r} \textcolor{C3}{mezə-m} \\ 
\textsc{loc-}see\textsc{-pst}\textsc{-pl} boy\textsc{-obl.pl} fox\textsc{-pl-abs} forest\textsc{-obl} \\
\ex\label{Adyghe:ex:11b}
\textcolor{C4}{V}-\textcolor{C1}{S}-\textcolor{C3}{LOC}-\textcolor{C2}{O} [\isi{VSO}] \\
\gll \textcolor{C4}{š'a-λeʁʷə-ʁe-x} \textcolor{C1}{č̣'ale-me} \textcolor{C3}{mezə-m} \textcolor{C2}{baǯ'e-xe-r} \\ 
\textsc{loc-}see\textsc{-pst}\textsc{-pl} boy\textsc{-obl.pl} forest\textsc{-obl} fox\textsc{-pl}\textsc{-abs} \\
\glt `The boys saw the foxes in the forest.'
\z
\z

In verb-initial patterns, again the final element is emphasized and the verb to a lesser degree (\ref{Adyghe:ex:11a}, \ref{Adyghe:ex:11b}).

\citet[117]{kumakhov2009circassian} also illustrate the six available constituent order patterns for subject, \isi{direct object} and verb (albeit with a pronominal subject). \citet[112]{kumakhov2009circassian} further state that constituent order varies with \isi{information structure}. For example, in answers to questions that target the subject \isi{SVO} is more common than SOV.

The position of the indirect object\is{object!indirect} in pragmatically neutral clauses with nominal arguments is between the subject and the \isi{direct object} (S-DO-IO-V) according to \citet[114--115]{kumakhov2009circassian} (12--13). In this pattern, the two arguments that bear identical case markers (\textit{-m}) (S, IO) are separated by the \isi{direct object} in the \isi{absolutive} case (\textit{-r}). 

\ea\label{Adyghe:ex:12}
Adyghe \il{Circassian!Adyghe}\citep[114]{kumakhov2009circassian} [H] \\
\textcolor{C1}{S}-\textcolor{C2}{DO}-\textcolor{C5}{IO}-\textcolor{C4}{V} \\
\gll \textcolor{C1}{č̣'ale-m} \textcolor{C2}{txeλə-r} \textcolor{C5}{pŝaŝe-m} \textcolor{C4}{r-jə-tə-ʁ} \\
boy\textsc{-obl} book\textsc{-abs} girl\textsc{-obl} \textsc{obl-3sg.A-}give\textsc{-pst} \\
\glt `The boy gave the book to the girl.'
\z

\ea\label{Adyghe:ex:13} 
Adyghe \il{Circassian!Adyghe}(courtesy of M. Höhlig) [H] \\
\textcolor{C1}{S}-\textcolor{C2}{DO}-\textcolor{C5}{IO}-\textcolor{C4}{V} \\
\gll \textcolor{C1}{jež'} \textcolor{C1}{ja-te=ja-ne-xe-m-jə} \textcolor{C1}{ə-š-xe-m-jə} \textcolor{C2}{pŝeŝeẑəje-r} [\textcolor{C5}{zə-fe-mə-je-w} \textcolor{C5}{məλkʷə} \textcolor{C5}{z-jə-ʔe}] \textcolor{C5}{wənaʁʷe} \textcolor{C5}{gʷere-m} \textcolor{C4}{r-a-tə-ʁ} \\
self \textsc{poss-}father\textsc{=poss-}mother\textsc{-pl-obl-add} \textsc{3sg.poss-}brother\textsc{-pl-obl-add} girl\textsc{-abs} \textsc{rel.IO-}\textsc{ben}\textsc{-neg-}want\textsc{-adv} property \textsc{rel.IO-loc-}be family certain\textsc{-obl} \textsc{obl-3pl.A-}give\textsc{-pst} \\
\glt `Her parents and brothers gave the girl to a certain family with property who she did not want.'
\z

However, S-IO-DO-V is also attested (\ref{Adyghe:ex:14}), in particular when either the subject or the indirect object\is{object!indirect} is a personal pronoun\is{pronoun!personal}. 

\ea\label{Adyghe:ex:14} 
Adyghe \il{Circassian!Adyghe}\citep[65]{rogava1966grammatika} [H] \\
\textcolor{C1}{S}-\textcolor{C5}{IO}-\textcolor{C2}{DO}-\textcolor{C4}{V} \\
\gll \textcolor{C1}{hač̣'e-m} \textcolor{C5}{č̣'ale-m} \textcolor{C2}{šə-r} \textcolor{C4}{r-jə-tə-ʁ} \\
guest\textsc{-obl} boy\textsc{-obl} horse\textsc{-abs} \textsc{obl-3sg.A-}give\textsc{-pst} \\ 
\glt `The guest gave the horse to the boy.'
\z

The \isi{addressee} in (\ref{Adyghe:ex:15}) exemplifies another indirect object\is{object!indirect} in preverbal position. Adjuncts such as instruments (\ref{Adyghe:ex:16}) or beneficiaries expressed by means of \isi{postpositional} phrases (\ref{Adyghe:ex:17}) usually occur between the subject and the verb. Yet positions before the subject (\ref{Adyghe:ex:16}) and after the predicate are also allowed (Section \ref{Adyghe:ss:3}).

\ea\label{Adyghe:ex:15}
Adyghe \il{Circassian!Adyghe}(courtesy of M. Höhlig) [H] \\
\gll šə-m apəλə-m r-a-ʔʷa-ʁ mə-rə mə-rə  ə-ʔʷe-re-r \\
horse\textsc{-obl} stableman?\textsc{-obl} \textsc{obl-3pl.A-}speak\textsc{-pst} this\textsc{-pred} this\textsc{-pred} \textsc{3sg.A-}speak\textsc{-dyn-abs} \\
\glt `They said to the stableman: ``It is like this and like that.''' 
\z

\ea\label{Adyghe:ex:16}
Adyghe \il{Circassian!Adyghe}(courtesy of M. Höhlig) [H] \\
\gll əpč̣'e adəɣe-me ʔa-č̣'e jaṭe-č̣'e qə-r-a-jə-č̣'ə-š'tə-ʁe-x \\
earlier Adyghe\textsc{-obl.pl} hand\textsc{-inst} clay\textsc{-inst} \textsc{dir-dat-3pl.A-}smear\textsc{-el-aux-pst-pl} \\
\glt [Now houses are built with plaster.] `Earlier the Adyghe people smeared (the houses) with clay with the hands.' 
\z

\ea\label{Adyghe:ex:17}
Adyghe \il{Circassian!Adyghe}(courtesy of R. Feer) [H] \\
\gll c̣əf-me a-paje we p-ṣ̂ə-ʁe \\
human.being\textsc{-obl.pl} \textsc{3pl-}for \textsc{2sg} \textsc{2sg.A-}do\textsc{-pst} \\
\glt `You built it for the people.' 
\z

Temporal and spatial adverbials including locations and goals frequently occur in clause-initial positions before the subject if there is any overtly expressed subject (\ref{Adyghe:ex:18}, \ref{Adyghe:ex:19}) or otherwise directly before the verb (\ref{Adyghe:ex:19}, \ref{Adyghe:ex:20}), but also occasionally after the verb (Section \ref{Adyghe:ss:3}).

\ea\label{Adyghe:ex:18} 
Adyghe \il{Circassian!Adyghe}(courtesy of M. Höhlig) [H] \\
TIME-LOC-S-COP-PRED \\
\gll a zeman-m hakʷənhable-m č'əle-m thamate ja-ʔa-ʁ hakʷərəne təʁʷəẑ a-ʔʷe-w \\
that time\textsc{-obl} Hakunhable\textsc{-obl} aul\textsc{-obl} elder \textsc{loc-}be\textsc{-pst} Hakuren Teguz \textsc{3pl.A-}speak\textsc{-adv} \\
\glt `At that time in Hakunhable, the village-elder was called Hakuren Teguz.'
\z

\ea\label{Adyghe:ex:19}
Adyghe \il{Circassian!Adyghe}(courtesy of R. Feer) [H] \\
TIME-S-LOC-COP-PURP[DO-V] \\
\gll aj ə-pe-re mafe-xe-m χʷaǯ'e-r beʒerə-m š'ə-ʔa-ʁ [č'em qə-š'efə-n-ew] \\
that\textsc{.obl} \textsc{3sg.}\textsc{poss-}earlier\textsc{-adj} day\textsc{-pl}\textsc{-obl} hodja\textsc{-abs} bazar\textsc{-obl} \textsc{loc-}be\textsc{-pst} cow \textsc{dir-}buy\textsc{-mod}\textsc{-adv} \\
\glt `The days before, the Hodja was at the market to buy a cow.' 
\z
 
\ea\label{Adyghe:ex:20}
Adyghe \il{Circassian!Adyghe}(courtesy of M. Höhlig) [H] \\
TIME-GOAL-CV \\
\gll wtoroj klass nes škʷelə-m sə-ḳʷa-ʁ \\
second[R] class[R] to school[R]\textsc{-obl} \textsc{1sg.}\textsc{abs-}go\textsc{-pst} \\
\glt `Until the second class I went to school.' 
\z

Complement clauses are marked by a variety of strategies among which the most frequent ones are the bare verbal stem without any tense or other markers, the modal/potentialis form with or without additional case suffixes, a specialized factive form with the prefix \textit{zere-}, case markers (adverbial case as in (\ref{Adyghe:ex:4}), (\ref{Adyghe:ex:19}), (\ref{Adyghe:ex:22}) and instrumental case as in (\ref{Adyghe:ex:2})) and the conditional suffix in combination with the additive (\ref{Adyghe:ex:23}) (see \citealt{serdobolskaya2016semantics} for a detailed analysis). There are no complementizers. Complement clauses usually precede the matrix clause, (\ref{Adyghe:ex:22}), (\ref{Adyghe:ex:23}) but they can also follow or be embedded.

Reported speech can be marked with a quotative particle that has been grammaticalized from a non-finite form of the verb of speech \textit{ʔʷe-} `say, speak, tell, talk' (which, however, retains its person prefixes). The quotative particle follows the quote (\ref{Adyghe:ex:15}), (\ref{Adyghe:ex:23}). The clause expressing the quote follows the matrix clause with the verb of speech (\ref{Adyghe:ex:15}) or precedes it as in the next two examples (\ref{Adyghe:ex:22}, \ref{Adyghe:ex:23}). 

\ea\label{Adyghe:ex:22}
Adyghe \il{Circassian!Adyghe}\citep{feer2019grammar} [H] \\
\gll [təʁʷəẑəqʷe qəzbeč' s-λeʁʷə-n-ew] sə-feja-ʁ, [sə-de-gʷəš'əʔe-n-ew] sə-faj ə-ʔʷa-ʁ \\
Tuguzhuko Kyzbech \textsc{1sg.A-}see\textsc{-mod}\textsc{-adv} \textsc{1sg.}\textsc{abs-}want\textsc{-pst} \textsc{1sg.}\textsc{abs-}\textsc{com-}talk\textsc{-mod}\textsc{-adv} \textsc{1sg.}\textsc{abs-}must \textsc{3sg.A-}speak\textsc{-pst} \\
\glt `{}``I would like to see Tuguzhuko Kyzbech, I want/need to speak with him,'' he said.'
\z

\ea\label{Adyghe:ex:23}
Adyghe \il{Circassian!Adyghe}(courtesy of R. Feer) [H] \\
\gll [a-fede š'ə-ʔe-m-jə] s-ṣ̂a-xe-re-p nəʔa  ə-ʔʷa-ʁ je-ʔʷe pŝaŝe-m \\
\textsc{dat-}similar \textsc{loc-}be\textsc{-cond}\textsc{-add} \textsc{1sg.}\textsc{io-}know\textsc{-trm}\textsc{-dyn}\textsc{-neg} only \textsc{3sg.A-}speak\textsc{-pst} \textsc{dat-}speak girl\textsc{-obl} \\
\glt `The girl said, ``I don't know at all if there is somebody similar.'''
\z

Adverbial subordination is expressed through specialized and general converbs, relativization, and the additive suffix, but not by means of subordinating particles (Forker In Press). Adverbial clauses, in particular chaining clauses, precede the main clause, but a position after the main clause is also possible (Section \ref{Adyghe:ss:3}).

Here is a summary of the preferred ordering patterns:
\begin{itemize}
\item fixed \isi{word order}, but not consistent across different NP types
\item only postpositions
\item auxiliaries after lexical verb
\item flexible \isi{word order} at clausal level with preference for verb-final patterns, in particular for SOV
\item \isi{complement} and adverbial clauses precede main clauses
\item relative clauses precede head nouns
\end{itemize}

From this follows that Adyghe\il{Circassian!Adyghe} has a certain preference for \isi{head-final} patterns within the clause and in clause combining. 

\section{Examining post-predicate elements}\label{Adyghe:ss:3}

In this section, I will explore post-predicate elements in more detail and not discuss the relative positions of pre-predicate items with respect to each other. In particular, I will explore the relative frequency of post-predicate elements in homeland Adyghe\il{Circassian!Adyghe} vs. diaspora Adyghe\il{Circassian!Adyghe} and the impact of genre/style.

\subsection{Post-predicate elements in elicitation and texts}\label{Adyghe:ss:3.1}

As explained in Section \ref{Adyghe:ss:2.2}, Adyghe\il{Circassian!Adyghe} has a tendency for \isi{head-final} \isi{word order} at the clausal level. But at the same time the \isi{word order} is described as ``free'' and in the published works on Adyghe\il{Circassian!Adyghe}, one finds many instances of post-predicate elements in examples that have probably been elicited. Examples (\ref{Adyghe:ex:8}-\ref{Adyghe:ex:11a}, \ref{Adyghe:ex:11b}) show post-verbal subjects and direct objects. In sentences (\ref{Adyghe:ex:24}) and (\ref{Adyghe:ex:25}), we find post-verbal indirect objects functioning as recipients and causees respectively. In (\ref{Adyghe:ex:25}) also the \isi{direct object} appears after the verb. 

\ea\label{Adyghe:ex:24}
Adyghe \il{Circassian!Adyghe}\citep[220]{paris1974princesse} [D] \\
in\isi{direct object} (\isi{recipient}) \\
\gll λ̣ə-m qʷəẑ r-j-e-tə \textbf{ŝʷəzə-m} \\
man\textsc{-obl} pear \textsc{dat-}\textsc{3sg.A}\textsc{-dyn}-give woman\textsc{-obl} \\
\glt `The man gives / is giving a pear to the woman.'
\z

\ea\label{Adyghe:ex:25}
Adyghe \il{Circassian!Adyghe}\citep[388]{letuchiy2009affiksy} [H] \\
in\isi{direct object} (causee) \\
\gll t-ṣ̂ə-n t-ṣ̂e-re-p-ŝə, qe-ẑʷə-ʁa-ʔʷə-ba ade \textbf{wered} \textbf{a-š'} \\
\textsc{1pl.A}-do\textsc{-mod} \textsc{1pl.A}-do\textsc{-dyn}\textsc{-neg}\textsc{-cs} \textsc{dir-}\textsc{2pl.A-}\textsc{caus-}say\textsc{-prt} well song that\textsc{-obl} \\
\glt `As for doing we will not do (anything), so he may sing a song!' (lit. `You let him / you cause him to sing a song.') 
\z

The post-verbal elements can be definite or indefinite (which for nouns correlates with case marking, i.e. the omission of the case marking indicates indefiniteness), e.g. in (\ref{Adyghe:ex:8}) and (\ref{Adyghe:ex:9}) post-verbal direct objects bear the \isi{absolutive} suffix \textit{-r} and are definite, whereas in (\ref{Adyghe:ex:25}) the \isi{direct object} is not marked for case and thus indefinite.

Not only arguments but also adjuncts can occur after the verb. In (\ref{Adyghe:ex:8}) and (\ref{Adyghe:ex:10}), the case-marked noun denoting a location is placed in post-verbal position. In the literature, one also finds examples of \isi{complement} clauses that follow the matrix clause with a variety of complement-taking predicates, e.g. `want', `need, must', `fear', or `know' (\ref{Adyghe:ex:27}) and some examples of post-verbal adverbial clauses (\ref{Adyghe:ex:28}). 

\ea\label{Adyghe:ex:27}
Adyghe \il{Circassian!Adyghe}\citep[533]{serdobolskaya2009semantika} [H] \\
\gll s-j-e-negʷəje [ṭʷə qe-s-hə-n-ew] \\
\textsc{1sg.}\textsc{io-}\textsc{dat-}\textsc{dyn-}suppose two \textsc{dir-}\textsc{1sg.}\textsc{io-}carry.away\textsc{-mod}\textsc{-adv} \\
\glt `I suppose / fear that I get a two (= bad mark).' 
\z

\ea\label{Adyghe:ex:28}
Adyghe \il{Circassian!Adyghe}\citep[691]{testelets2009nevyražennye} [H] \\
\gll pŝaŝe-r qe-ʁə-ʁ [sə-z-de-gʷəš'əʔe-m] \\
girl\textsc{-abs} \textsc{dir-}cry\textsc{-pst} \textsc{1sg.}\textsc{abs-}\textsc{rel.}\textsc{temp-}\textsc{com-}talk\textsc{-obl} \\
\glt `The girl cried when I talked to her (= the girl or another female person).' 
\z

\begin{sloppypar}
In short, Adyghe\il{Circassian!Adyghe} allows for post-verbal arguments and adjuncts with various syntactic functions and grammatical roles as well as for post-verbal \isi{complement} and adverbial clauses. Based on the literature we cannot say whether post-verbal elements are a marginal phenomenon, and if some syntactic functions or grammatical roles are more frequently found there than others, due the lack of previous corpus studies. 
\end{sloppypar}

Therefore, I examined post-predicate elements in declarative main clauses of 20 texts containing a total of 6,146 words distributed over 1,154 main clauses. The texts have been glossed by the researchers who recorded them (Paris, Höhlig and Feer) and by myself. In a second step, I manually annotated them for the presence or absence of post-predicate elements (see Section \ref{Adyghe:ss:3.2} below and Appendix for more information on the texts). I did not count all overt and covert arguments and all overt adjuncts and their positions with respect to the verb but only the post-predicate ones (and I am fully aware of the fact that this makes comparison with data from WOWA corpora impossible\footnote{
Furthermore, in WOWA only non-subject referential expressions are considered, rather than all kinds of post-verbal items (e.g. modal particles, clausal constituents). 
} and evaluation of frequency of post-predicate elements rather speculative). 

The results of these counts are presented in \tabref{Adyghe:tab:1}. The first thing to notice is that for a language that is described as having ``flexible \isi{word order}'', in natural texts the position after the verb is not frequently occupied despite being easily filled in elicitation. In around one out of ten main clauses, we find post-predicate elements of which the largest group are \isi{focus} / modal particles.

\begin{table}
\begin{tabularx}{\textwidth}{Qrr}
\lsptoprule
\textbf{Arguments} & \textbf{33} \\
\midrule
{Subjects (independent of transitivity and semantic \isi{role})} & 20 \\
{Direct objects (of transitive and ditransitive verbs)} & 10 \\
{Indirect objects (2 addressees and 1 beneficiary; no recipients attested)} & 3 \\
\tablevspace
\textbf{Adjuncts} & \textbf{46} \\
\midrule
{Possessors} & 4 \\
{Instruments} & 3 \\
{Temporal adverbials} & 12 \\
{Spatial adverbials (13 locations, 6 goals, 2 sources)} & 21 \\
{Manner adverbials} & 5 \\
\tablevspace
\textbf{Particles} & \textbf{29} \\
\midrule
\tablevspace
\textbf{Clauses} & \textbf{19} \\
\midrule
{Adverbial clauses} & 16 \\
{Complement clauses} & 2 \\
{Relative clauses} & 1 \\
\midrule
 \textbf{Total} & \textbf{126} \\
\lspbottomrule
 \end{tabularx}
 \caption{Grammatical and semanto-pragmatic functions of post-predicate elements in Adyghe texts (both diaspora and homeland)}
 \label{Adyghe:tab:1}
\end{table}

The most common particle is \textit{nah} `more' (\ref{Adyghe:ex:29}), (\ref{Adyghe:ex:36}), others being \textit{nəʔa} `only' (\ref{Adyghe:ex:23}), \textit{armərme} `otherwise, if not this', \textit{mewš'tew} `like this' and also Russian\il{Russian} loans such as \textit{uže} `already' and \textit{daže} `even' and the Turkish\il{Turkic!Turkish} indirect evidential particle \textit{ (je)məš} and the particle \textit{yani} `that is, namely'. Pragmatic particles such as the ones listed are expected to have a great deal of freedom and thus do not really support the claim that the \isi{word order} of Adyghe\il{Circassian!Adyghe} is flexible.

\ea\label{Adyghe:ex:29}
Adyghe \il{Circassian!Adyghe}\citep[218]{hohlig1997kontaktbedingter} [H] \\
\gll jež' djela-ʁe \textbf{nah} \\
self fool\textsc{-pst} more \\
\glt `He was a fool.' 
\z

Due to the way in which I annotated and counted the data, I cannot make any statements concerning the relative frequency of certain types of postverbal arguments and adjuncts. I will instead present the attested post-predicate elements and discuss their information-structural properties, whenever possible taking into account audio recordings of some of the examples that were provided to me by Monika Höhlig. 

First of all, there is one construction that regularly leads to post-predicate elements according to the literature and the examined texts, namely direct reported speech (\citealt{rogava1966grammatika}: 395--402). When the matrix clause with the verb of speech interrupts the quote or follows it, the usual \isi{word order} is reversed and the subject (\ref{Adyghe:ex:23}) or the \isi{addressee}, if there is no overt subject (\ref{Adyghe:ex:30}) follows the verb of speech.

\ea\label{Adyghe:ex:30}
Adyghe \il{Circassian!Adyghe}\citep[396]{rogava1966grammatika} [H] \\
\gll səd-a aməd wə-z-ʁegʷəmeč̣'ə-re-r? j-e-wəpč̣ə-ʁ \textbf{bzəλfəʁe-m} \\
what\textsc{-q} Amid \textsc{2sg.}\textsc{pr-}\textsc{rel.}\textsc{temp-}disturb\textsc{-dyn}\textsc{-abs} \textsc{3sg.A}\textsc{-dyn}-ask\textsc{-pst} woman\textsc{-obl} \\
\glt `{}``What is it that worries you, Amid?'' (s/he) asked the woman.' 
\z

Occasionally one encounters \isi{thetic} sentences in which the subject appears in clause-final position. A \isi{thetic} utterance is fully focused with no topical constituent. Introductory clauses in traditional fairy tales or narratives about well-known personalities may follow this pattern, as in the following example from a story about a famous singer and composer. The V-S pattern for introductory \isi{thetic} sentences is also attested in other verb-final languages from the Caucasus such as Kartvelian and East Caucasian (\citealt{forker_information_2021}, \citetv{chapters/10_Forker_EC}). In example (\ref{Adyghe:ex:31}), the speaker makes a short break before uttering the subject encoded as personal name (and there is no falling \isi{intonation} in \isi{contrast} to the examples discussed below).

\ea\label{Adyghe:ex:31}
Adyghe \il{Circassian!Adyghe}(courtesy of M. Höhlig) [H] \\
\gll mə č'əle ẑə-m de-sə-ʁ [weredəʔʷe ʔaze-w] [qebar-xe-r-jə zeč̣'e qə-ʔʷate-w] [txədeẑ-xe-r-jə qə-ʔʷate-w] \textbf{ḳʷaj} \textbf{zefes} \\
this aul old\textsc{-obl} \textsc{com-}sit\textsc{-pst} singer art.master\textsc{-adv} story\textsc{-pl}\textsc{-abs}\textsc{-add} all \textsc{dir-}tell\textsc{-adv} legend\textsc{-pl}\textsc{-abs}\textsc{-add} \textsc{dir-}tell\textsc{-adv} Kway Zefes \\
\glt `In this old village lived the master singer, story-teller and legend-teller, Kway Zefes.'
\z

In introductory \isi{thetic} sentences, also other constituents besides the subject can follow the verb, like the \isi{instrument} in (\ref{Adyghe:ex:32}), which has a falling \isi{intonation} towards the end of the sentence and no intonational break before the \isi{post-predicate element}.

\ea\label{Adyghe:ex:32}
Adyghe \il{Circassian!Adyghe}(courtesy of M. Höhlig) [H] \\
\gll neməc-xe-r jewjə ǯ'ambeč'əje qə-da-ha-ʁ \textbf{mašine-č̣'e} \\
German[R]\textsc{-pl}\textsc{-abs} \textsc{prt} Dzhambechi \textsc{dir-}\textsc{loc-}enter\textsc{-pst} car[R]\textsc{-inst} \\
\glt [beginning of narrative] `The Germans came to Dzhambechi by car.' 
\z

The next example illustrates left dislocation of the subject combined with a post-predicate presumptive \isi{pronoun}. In left dislocation, a referential constituent both precedes and is dislocated from a core clause with which it is associated. Within the core clause, there is an \isi{anaphoric} \is{anaphoric!pronoun}\is{pronoun!!anaphoric}
co-referential resumptive \isi{pronoun} (\citealt{westbury2016left}). A typical function of left dislocation is to introduce referents that are not purely brand-new, but merely inactive. This means that the referent is assumed to be identifiable, but only minimally accessible, having been in one way or another evoked in the prior discourse or in the extra-linguistic context. This is what we find in (\ref{Adyghe:ex:33}): The story is about a group of boys, one of which is singled out by means of the left-dislocated element given in curly brackets and then resumed through the \isi{pronoun} in subject function following the verb.

\ea\label{Adyghe:ex:33}
Adyghe \il{Circassian!Adyghe}\citep[234]{hohlig1997kontaktbedingter} [H] \\
\gll \{japλ̣enere, z-jə-hatəq z-jə-haləʁʷə qə-z-ṣ̂ʷe-t-təʁʷə-ʁe-m\} qe-ḳʷe-ž'ə-ʁ \textbf{a-r-jə} \\
fourth \textsc{rel.}\textsc{io-}\textsc{poss-}flat.bread \textsc{rel.}\textsc{io-}\textsc{poss-}bread \textsc{dir-}\textsc{rel.}\textsc{temp-mal-}\textsc{1pl.A-}steal\textsc{-pst}\textsc{-obl} \textsc{dir-}go\textsc{-re-pst} that\textsc{-abs}\textsc{-add} \\
\glt [All went back home from school. The three of us are standing there.] `The fourth one whose flat bread, whose bread we had stolen, he also went home.'
\z

In the following, I use the term ``\isi{topic}'' in the sense of ``aboutness \isi{topic}'' (e.g. \citealt{krifka2007basic}). The topical item is identified through the utterance and then some piece of information about it is provided in the comment. Post-predicate elements that function as aboutness topics in Adyghe\il{Circassian!Adyghe} are not emphasized by means of \isi{intonation}. The pitch accent is usually somewhere at the beginning, and  towards the end of the utterance the \isi{intonation} it falls and becomes flat; the voice sometimes gets lower and quieter such that in some examples, the last syllable of the final \isi{post-predicate element} is barely audible. For instance, in (\ref{Adyghe:ex:34}) the pitch accent of the second clause falls on the first verb (\textit{šxe-n}).

\ea\label{Adyghe:ex:34}
Adyghe \il{Circassian!Adyghe}(courtesy of M. Höhlig) [H] \\
\gll a-xe-r-jə a šxe-š't; šxe-n faje-ba \textbf{c̣əfə-r} \\
that\textsc{-pl}\textsc{-abs}\textsc{-add} that eat\textsc{-fut} eat\textsc{-mod} must\textsc{-prt} human.being\textsc{-abs} \\
\glt [Talking about the behavior of German and Soviet soldiers during WWII. The speaker is finishing her narration and after this she switches to a different \isi{topic}.] `They (=Germans) as well will eat it; the human being must eat.' 
\z
 
Topical elements often convey given information and can, e.g., be expressed by means of pronouns. In all following examples, the post-predicate elements represent given information. For those examples that have personally been provided to me by Monika Höhlig (\ref{Adyghe:ex:36}, \ref{Adyghe:ex:38}-\ref{Adyghe:ex:40}) I could clarify the \isi{intonation}. In all examples, the post-predicate elements are deaccented by means of a falling \isi{intonation}. 

\ea\label{Adyghe:ex:35}
Adyghe \il{Circassian!Adyghe}\citep[236]{hohlig1997kontaktbedingter} [H] \\
subject \\
\gll tawərəχ-ew š'ə-t-ep \textbf{a-r} \\
legend\textsc{-adv} \textsc{loc-}stand\textsc{-neg} that\textsc{-abs} \\
\glt [Here is what I want to tell you.] `It is not an (old) legend.' 
\z

\ea\label{Adyghe:ex:36}
Adyghe \il{Circassian!Adyghe}(courtesy of M. Höhlig) [H] \\
discourse particle + subject \\
\gll ǯ'ə zewže t-ṣ̂ʷeḳʷedə-ʁe-xe \textbf{nah} \textbf{a-xe-r} \\
now all \textsc{1pl.}\textsc{io-}get.lost\textsc{-pst}\textsc{-pl} more.than that\textsc{-pl}\textsc{-abs} \\
\glt [talking about various Adyghe traditions] `Now we lost all those (i.e. traditions).'
\z

\ea\label{Adyghe:ex:37}
Adyghe \il{Circassian!Adyghe}\citep[219]{hohlig1997kontaktbedingter} [H] \\
\isi{direct object} \\
\gll te q-jə-p-hə-ʁ \textbf{a} \textbf{š'aλe-r} s-ʔʷ-əj sə-kʷəwa-ʁ \\
from.where \textsc{dir-}\textsc{loc-}\textsc{2sg.A}-carry.away\textsc{-pst} that bucket\textsc{-abs} \textsc{1sg.A}-speak\textsc{-add} \textsc{1sg.}\textsc{abs-}shout\textsc{-pst} \\
\glt ` ``From where did you take that bucket?'' I said shouting.'
\z

\ea\label{Adyghe:ex:38}
Adyghe \il{Circassian!Adyghe}(courtesy of M. Höhlig) [H] \\
\isi{direct object} \\
\gll ... neməč̣' ha-xe-m a-šxə-ʁ \textbf{ṭʷə-jə} ... \\
{} other dog\textsc{-pl}\textsc{-obl} \textsc{3pl.A-}eat\textsc{-pst} two\textsc{-add} \\
\glt [Talking about the mysterious disappearance of turkeys and the fault of the dogs] `... the other dogs ate (our) two (turkeys) ...' 
\z
 
\newpage
\ea\label{Adyghe:ex:39}
Adyghe \il{Circassian!Adyghe}(courtesy of M. Höhlig) [H] \\
possessor \\
\gll jəλes ṭʷeč̣'ə-re blə-re ə-nəbž'ə-ʁ \textbf{č'etəwə-m} \\
year twenty\textsc{-coord} seven\textsc{-coord} \textsc{3sg.}\textsc{poss-}age\textsc{-pst} cat\textsc{-obl} \\
\glt [The dogs were also afraid of it, it was a very good cat] `The cat was 27 years old.'
\z

\ea\label{Adyghe:ex:40}
Adyghe \il{Circassian!Adyghe}(courtesy of M. Höhlig) [H] \\
manner adverbial (\isi{adjunct}) \\
\gll woot, nepeməč̣' tradicie-w səd ǯ'ərjə t-xe-λə-n \textbf{aj} \textbf{fede-w} \\
well[R] other traditions[R]\textsc{-adv} what again \textsc{1pl.}\textsc{io-}\textsc{loc-}lie\textsc{-mod} that\textsc{.obl} be.similar\textsc{-adv}  \\
\glt [talking about various Adyghe traditions] `Well, which other traditions do we have here like that.' 
\z

The postpredicate elements in examples (\ref{Adyghe:ex:35}-\ref{Adyghe:ex:40}) can hardly be said to be highlighted or emphasized, in \isi{contrast} to what \citet[91]{jakovlev1941grammatika} have stated about the post-predicate elements in examples (\ref{Adyghe:ex:8}-\ref{Adyghe:ex:11a}, \ref{Adyghe:ex:11b}). 

However, post-predicate elements are not always topical and do not always convey given information. The postverbal adjuncts in (\ref{Adyghe:ex:41}) and (\ref{Adyghe:ex:42}) convey new information and are part of the ``presentational'' or ``information'' \isi{focus} of the sentence in which they occur. Information \isi{focus} expresses the most important or new information in the utterance (\citealt{krifka2007basic}). Because I lack recordings of (\ref{Adyghe:ex:41}) and (\ref{Adyghe:ex:42}) I cannot say anything about the intonational patterns. Yet I hypothesize that the elicited examples (\ref{Adyghe:ex:8}-\ref{Adyghe:ex:11a}, \ref{Adyghe:ex:11b}) from \citet[91]{jakovlev1941grammatika} should be interpreted in a similar manner, i.e. the post-predicate elements as being part of the information \isi{focus}.

\ea\label{Adyghe:ex:41}
Adyghe \il{Circassian!Adyghe}\citep[255]{hohlig1997kontaktbedingter} [D] \\
spatial adverbial (\isi{Goal}) \\
\gll qe-ze-ḳʷe-ž'ə-m, se sə-q-jə-č̣'ə-ʁ \textbf{mutfaqə-m} \\
\textsc{dir-}\textsc{rel.temp-}go\textsc{-re-obl} \textsc{1sg} \textsc{1sg.}\textsc{abs-}\textsc{dir-}\textsc{loc-}go.out\textsc{-pst} kitchen[T]\textsc{-obl} \\
\glt [In the evening my husband came home.] `When he came I went out to the kitchen.'
\z

\newpage
\ea\label{Adyghe:ex:42}
Adyghe \il{Circassian!Adyghe}\citep[274]{hohlig1997kontaktbedingter} [D] \\
temporal and spatial adverbials (time span and location) \\
\gll adəɣa bze te-gʷəš'aʔ \textbf{zepət} \textbf{wəne-m-jə} \textbf{kʷеž'e-m-jə} \\
Adyghe language \textsc{1pl.A}-speak all.time house\textsc{-obl}\textsc{-add} village[T]\textsc{-obl}\textsc{-add} \\
\glt [talking about language knowledge and language use] `We always speak Adyghe at home and in the village.' 
\z

\begin{sloppypar}
Adverbial clauses seem to be more variable concerning their position than \isi{complement} clauses (or relative clauses), whereby we can notice that post-predicate adverbial clauses repeatedly express cause (\ref{Adyghe:ex:43}) or purpose (\ref{Adyghe:ex:19}). For purpose clauses, we can assume an explanation based on iconicity. The purpose of an action resembles a spatial \isi{Goal} and thus the linear order of the clauses reflects the temporal or spatial order of the events. Causal or concessive adverbial clauses refer to situations that do not necessarily occur prior to the situation expressed in the main clause but rather provide the reason or cause for it (\ref{Adyghe:ex:43}) or signal \isi{contrast} or concession in relation to it (\ref{Adyghe:ex:44}). In both examples (\ref{Adyghe:ex:43}) and (\ref{Adyghe:ex:44}), the adverbial clauses constitute separate intonational units, i.e. there is a short break between the preceding main clauses and the sentence-final adverbial clauses. 
\end{sloppypar}

\ea\label{Adyghe:ex:43}
Adyghe \il{Circassian!Adyghe}(courtesy of M. Höhlig) [H] \\
\isi{adverbial clause} expressing a cause or reason  \\
\gll abʒexa bze-č̣'e t-jə-wənaʁʷe parjə gʷəš'əʔa-re-p [\textbf{č̣'emgʷe-me} \textbf{t-a-xe-s-ŝə}] \\
Abdzakh language\textsc{-inst} \textsc{1pl-poss-}family nobody talk\textsc{-dyn-neg} Temirgoi\textsc{-obl.pl} \textsc{1pl.abs-3pl.IO-loc-}sit\textsc{-cvb} \\
\glt `In our family nobody speaks Abdzakh because we live among the Temirgoi.' 
\z
 
\ea\label{Adyghe:ex:44}
Adyghe \il{Circassian!Adyghe}(courtesy of M. Höhlig) [H] \\
concessive conditional clause \\
\gll xet š'əš' qə-tje-wa-ʁe-m-jə jə-pče ʔʷə-jə-xə-š't. [\textbf{wəč̣'aḳʷe} \textbf{qə-ʔʷə-ha-ʁe-m-jə} \textbf{a-š'} \textbf{fed}] \\
who part \textsc{dir-}\textsc{loc-}beat\textsc{-pst}\textsc{-cond}\textsc{-add} \textsc{poss-}door \textsc{loc-}\textsc{3sg.A-}open\textsc{-fut} killer \textsc{dir-}speak-lat\textsc{-pst}\textsc{-cond}\textsc{-add} that\textsc{-obl} similar \\
\glt `[You know the Adyghe traditions], whoever might knock, you will open your door, even if it were a killer.' 
\z

By \isi{contrast}, in chaining constructions that also represent adverbial subordination, the dependent clauses do not follow the main clause because the chained clauses do not or only to a limited extent convey temporal reference but their temporal interpretation depends on the inflected predicate of the main clause. The temporal order of events in chaining constructions is therefore iconically reflected in the order of the chained clauses and the main clause (\citealt{forkerclausechain}).

The sentence in (\ref{Adyghe:ex:45}) illustrates a complement clause\is{complement!clause} that follows the matrix clause. This order is not particularly common in the surveyed texts which contain only two instances (\tabref{Adyghe:tab:3}). As said above, I did not count all \isi{complement} clauses in my data in order to assess the frequency of preposed vs. postposed ones and compare them with adverbial clauses. However, the paper by \citet{serdobolskaya2016semantics} about \isi{complement} clauses in Adyghe\il{Circassian!Adyghe} contains many examples of sentence-final complements which at least shows that in elicitation they are easily available (and probably triggered by Russian\il{Russian} \isi{word order} patterns).

\ea\label{Adyghe:ex:45}
Adyghe \il{Circassian!Adyghe}\citep[251]{hohlig1997kontaktbedingter} [H] \\
complement clause\is{complement!clause} \\
\gll s-ṣ̂e-re-p [\textbf{qə-ze-re-s-ʔʷete-r}] \\
\textsc{1sg.A}-know\textsc{-dyn}\textsc{-neg} \textsc{dir-}\textsc{rel.}\textsc{io-mnr-}\textsc{1sg.A}-tell\textsc{-abs} \\
\glt `I do not know how I should tell it.' 
\z


\subsection{Summary of the quantitative analysis}\label{Adyghe:ss:3.2}

\tabref{Adyghe:tab:2} gives a summary of the texts ordered by place of recording. The texts from Adygea (Northern Caucasus, homeland [H]) are almost double in terms of number of words and clauses but also with respect to the number of post-predicate elements per clause when compared with the texts from Turkey (diaspora [D]). The texts from Turkey have been recorded between 1969 and 1990 and are thus older than the texts from Adygea, which have been recorded between 1990 and 2017. All texts represent three different Adyghe\il{Circassian!Adyghe} dialects, namely Shapsug\il{Circassian!Adyghe Shapsug} (3 texts), Abzakh\il{Circassian!Adyghe Abzakh} (15 texts) and Temirgoy\il{Circassian!Adyghe Temirgoy} (2 texts). The texts can be divided into two categories, namely into (A) traditional folklore narratives and folkloristic anecdotes and (B) personal, autobiographical narrations. The two genres also exhibit differences in style. In traditional narratives a formal style prevails (e.g. fewer loan words, long and structurally complicated sentences) whereas in personal accounts an informal style is found (e.g. many Russian\il{Russian} loan words, shorter and structurally simpler sentences). Detailed information about the sources of the texts, the time and the place of the recording can be found in the Appendix.

\begin{table}
 \begin{tabularx}{\textwidth}{lYYY}
\lsptoprule
 & Adygea & Turkey & Total \\
\midrule
\# texts & 11 & 9 & 20 \\
\# words & 4,083 & 2,063 & 6,146 \\
\# main clauses & 770 & 384 & 1,154 \\
\# post-predicate elements & 100 & 26 & 126 \\
\% post-predicate elements per clause & \textbf{12.99\%} & \textbf{6.77\%} & \textbf{10.91\%} \\
\lspbottomrule
 \end{tabularx}
 \caption{Comparing post-predicate elements in texts from Adygea and from Turkey}
 \label{Adyghe:tab:2}
\end{table}

If we compare the two genres --- personal accounts and traditional narratives / anecdotes --- with each other we find that personal accounts have more than twice as many post-predicate items than traditional narratives. This difference is more pronounced in Turkey than in Adygea, but it is observable in both places (\tabref{Adyghe:tab:3}).

\begin{table}
 \begin{tabularx}{\textwidth}{lYYYY}
\lsptoprule
 & \multicolumn{2}{c}{\textbf{Personal accounts}} & \multicolumn{2}{c}{\textbf{Traditional narratives}} \\
 \cmidrule(lr){2-3}\cmidrule(lr){4-5}
 & \textbf{Adygea} & \textbf{Turkey} & \textbf{Adygea} & \textbf{Turkey} \\
\midrule
\# main clauses & 554 & 158 & 216 & 226 \\
\# post-pred & 79 & 19 & 21 & 7 \\
\% post-pred per clause & \textbf{14.26\%} & \textbf{12.02\%} & \textbf{9.7\%} & \textbf{3.1\%} \\
\% post-pred per clause & \multicolumn{2}{c}{\qquad\textbf{13.76\%}} & \multicolumn{2}{c}{\qquad~~\textbf{6.3\%}} \\
\lspbottomrule
 \end{tabularx}
 \caption{Comparing post-predicate elements according to genre}
 \label{Adyghe:tab:3}
\end{table}

Thus, it seems that place as well as genre correlates with the number of post-predicate elements. Traditional narratives from Turkey contain the lowest number of them (7 items), which are almost exclusively focal / modal and evidential particles, in particular the particle \textit{nah}. The traditional narratives from Adygea also have post-predicate subjects in reported speech constructions (\ref{Adyghe:ex:23}) and some more post-predicate elements in various functions and of different formal types. On the other side of the spectrum, we find the personal accounts from Adygea that have as many as 79 post-predicate elements. 

The difference seems to be due to genre in combination with style, but it might have been enlarged by \isi{language contact} with two typologically and genealogically different contact languages, Turkish\il{Turkic!Turkish} and Russian\il{Russian}. Turkish\il{Turkic!Turkish} is a \isi{head-final} language; the unmarked \isi{word order} is SOV (e.g. \citealt{erguvanli1984function}: 43; \citealt{goksel2005turkish}: 338). Post-predicate elements are restricted to informal style and mostly found in spoken language, but also in more informal writing. They are never stressed and generally backgrounded, i.e. they convey information that is shared by speaker and \isi{addressee}; WH-words are not allowed to be placed after the predicate (\citealt{erguvanli1984function}: 43--63; \citealt{goksel2005turkish}: 345--346). Not only S, DO and IO may occur in post-verbal position, but also various adverbials and subordinate clauses (\isi{complement} clauses, relative clauses and adverbial clauses; \citealt{erguvanli1984function}: 63--66). Although these authors confirm that speakers accept and produce post-posed elements of various kinds (under particular information-structural conditions), the corpus data from colloquial standard Turkish\il{Turkic!Turkish} (\citealt{Iefremenko2021}) in WOWA indicate that such structures are relatively infrequent in actual usage: about 94\% of the coded non-subject and non-pronominal constituents in this data set are pre-verbal.

By \isi{contrast}, Russian\il{Russian} has been characterized as having free or pragmatically governed \isi{word order} with many features of \isi{SVO} languages (\citealt{dryer2022slavic}). A small corpus study by \citet{billings2015corpus} (500 transitive clauses from the Russian\il{Russian} National Corpus) has shown that \isi{SVO} dominates (89.6\%), SOV is the second most frequent order (4.4\%) and all other orders are also attested but rather infrequently. It is possible but not necessary that the differences between the texts from Adygea and the texts from Turkey are due to contact with Russian\il{Russian} and Turkish\il{Turkic!Turkish} respectively. The detailed study by \citet{hohlig1997kontaktbedingter} that compares Adyghe\il{Circassian!Adyghe} in contact with Russian\il{Russian} in the homeland and Turkish\il{Turkic!Turkish} in the diaspora shows many examples of how both contact languages influence Adyghe\il{Circassian!Adyghe} in various parts of the grammar although it does not explicitly discuss \isi{word order}. In order to test the contact hypothesis in the future, it is necessary to study the \isi{word order} in the oldest written examples of Adyghe\il{Circassian!Adyghe} which have been produced by speakers who probably have not been exposed to Russian\il{Russian} or Turkish\il{Turkic!Turkish} to the same degree as today's speakers, which goes beyond the scope of this study. And when testing \isi{contact influence}, we have to keep in mind that \isi{language contact} can not only lead to a change towards another pattern, but also to an increased rigidity in the inherited pattern (e.g. \citealt{namboodiripad2019english}). Thus, the preference for verb-final utterances in Adyghe\il{Circassian!Adyghe} in Turkey could in principle also be due to such an effect and not to a preference for `copying' the Turkish\il{Turkic!Turkish} pattern.


\section{Post-predicate elements in other Northwest Caucasian languages: Kabardian\il{Circassian!Kabardian}, Ubykh\il{Circassian!Ubykh}, Abkhaz\il{Circassian!Abkhaz} and Abaza\il{Circassian!Abaza}}\label{Adyghe:ss:4}

\begin{sloppypar}
In general, all Northwest Caucasian languages are described as \isi{head-final} (\citealt{arkadiev2021northwest}), but with flexible \isi{word order} that permits all logically possible permutations. For Ubykh\il{Circassian!Ubykh}, the only Northwest Caucasian language that is no longer spoken, the latest grammar states that postverbal constituents are extremely rare in Ubykh\il{Circassian!Ubykh} texts (\citealt{fenwick2011grammar}: 151--153), which must have been collected in Turkey after the forced exodus of the Northwest Caucasian people to the Ottoman Empire in the second half of the 19th century. In addition to SOV, only OSV is a relatively common alternative, but apparently marked order in Ubykh\il{Circassian!Ubykh} that ``appears to provide a certain degree of \isi{emphasis} to the fronted \isi{absolutive} \isi{object}'' \citep[151]{fenwick2011grammar}. Judging from the publication of traditional narratives from all Northwest Caucasian languages in \citet{colarusso1999north}, SOV is clearly dominant, which is in accordance with my Adyghe\il{Circassian!Adyghe} data presented in Section \ref{Adyghe:ss:3}.
\end{sloppypar}

In elicitation, which for Ubykh\il{Circassian!Ubykh} cannot be done anymore, Kabardian\il{Circassian!Kabardian}, Abkhaz\il{Circassian!Abkhaz} and Abaza\il{Circassian!Abaza} are similar to Adyghe\il{Circassian!Adyghe}. \citet[112--130]{kumakhov2009circassian} provide elicited Kabardian\il{Circassian!Kabardian} examples of postverbal subjects and objects in nominal and pronominal form. They add that \isi{heaviness} influences the position of arguments, i.e. heavy arguments are preferably placed at the beginning of sentences but can also occur at the end. Furthermore, they show that because proper nouns only facultatively take case markers for \isi{absolutive} and \isi{oblique} they require a strict S-DO-IO order but can still be placed in post-predicate position. 

\ea
\ea\label{Adyghe:ex:46a}
Kabardian \il{Circassian!Kabardian}\citep[129]{kumakhov2009circassian} \\
S-DO-V-IO \\
\gll Murat Nazir χʷ-i-še-nu-s' \textbf{Asłen}. \\
Murat Nazir \textsc{ver-3sg.A}-lead\textsc{-fut-assert} Aslan \\
\ex\label{Adyghe:ex:46b}
S-V-DO-IO\\
\gll Murat χʷišenus' \textbf{Nazir} \textbf{Asłen}. \\
Murat \textsc{ver-3sg.A}-lead\textsc{-fut-assert} Nazir Aslan \\
\ex\label{Adyghe:ex:46c}
V-S-DO-IO\\
\gll χʷišenus' \textbf{Murat} \textbf{Nazir} \textbf{Asłen}. \\
\textsc{ver-3sg.A}-lead\textsc{-fut}-assert Murat Nazir Aslan \\
\glt `Murat will lead Nazir to Aslan.'
\z
\z

Similarly, Abkhaz\il{Circassian!Abkhaz} does not employ case marking for core argument\is{argument!core}s, such that in case of ambiguity the order of subject and \isi{object} is fixed as S-O. Postverbal placement of arguments is allowed if this order is kept (\citealt{chirikba2003abkhaz}: 60)

Abkhaz\il{Circassian!Abkhaz} allows for \isi{thetic} utterances that introduce new referents (often in subject position) to occur after the verb, (\ref{Adyghe:ex:47}) as is also the case for Adyghe\il{Circassian!Adyghe}.

\ea\label{Adyghe:ex:47}
Abkhaz \il{Circassian!Abkhaz}\citep[259]{chirikba2003evidential} \\
VS \\
\gll jə-qʼa-n aǯər-jə-pa-cʷa hʷa jʷə-ǯja [a]-aj.šj-cʷa \\
\textsc{3pl-}be\textsc{-pst.fin} Adzhyr-3-son\textsc{-pl} \textsc{quot} two\textsc{-hum} \textsc{art-}brother\textsc{-pl} \\
\glt `There lived two brothers (reportedly known as) Sons of Adzhyr.'
\z

In the following example from Abaza\il{Circassian!Abaza}, the \isi{direct object} appears after the verb:

\ea\label{Adyghe:ex:48}
Abaza \il{Circassian!Abaza}(courtesy of P. Arkadiev) \\
IO-V-DO \\
\gll mhamatg'arə́j j-ʕa-jə-r-t-ṭ á-dg'əl\\
muhamat.girey \textsc{3sg.n.abs-cisl-3sg.m.IO-3pl.erg-}give\textsc{(aor)-dcl} \textsc{def-}land\\
\glt `They gave land to Muhamat-Girey.'
\z

In answers to WH-questions in Kabardian\il{Circassian!Kabardian}, the narrowly focused element can occur after the verb, but this order is marked compared to the neutral SOV order \citet[142]{kumakhov2009circassian}. 

In direct reported speech constructions with quotes preceding the matrix verb of speech subjects, or if the subject is omitted, other remaining constituents regularly follow the verb as shown in examples from Kabardian\il{Circassian!Kabardian} (e.g. \citealt{colarusso1999north}: sentence 70 from the Kabardian\il{Circassian!Kabardian} texts), Abkhaz\il{Circassian!Abkhaz} (\citealt{chirikba2003abkhaz}, sentence 15) and Abaza\il{Circassian!Abaza} (\ref{Adyghe:ex:49}). This pattern was also noted for Adyghe\il{Circassian!Adyghe} (Section \ref{Adyghe:ss:3.1}).

\ea\label{Adyghe:ex:49}
Abaza \il{Circassian!Abaza}(courtesy of P. Arkadiev) \\
QUOTE-V-S \\
\gll wə-z-ʕa-j aĉ̣ə-ja? j-hʷa-ṭ a-ʁəč', zaḳ-g'əj a-jə-m-rə-χ'-ʒa-ḳʷa\\
\textsc{2m.abs-}\textsc{rel.rsn-cisl-}come what\textsc{-qn} \textsc{3m.erg-}say\textsc{-dcl} \textsc{def-}thief one\textsc{.cln-add} \textsc{3n.IO-3m.erg-neg-caus-}cool.down\textsc{-intf-cvb.neg}\\
\glt `{}``Why did you come?'' said the thief, not calming down at all.' 
\z

Northwest Caucasian languages also allow certain types of adverbial clauses and \isi{complement} clauses in post-predicate position (e.g. examples in \citealt{kumakhov2009circassian}: 192; \citealt{arkadiev2020abaza}). For instance, according to \citealt{chirikba2003abkhaz}: 64), adverbial clauses with the present converb can either precede or follow the main clause.


\section{Discussion}\label{Adyghe:ss:5}

Summarizing, we can state that post-predicate elements are relatively rare in Adyghe\il{Circassian!Adyghe} texts, despite the possibility of eliciting them. There are no restrictions concerning grammatical functions and parts of speech of post-predicate elements in Adyghe\il{Circassian!Adyghe} --- they can be arguments and adjuncts of various kinds, from short pronouns to more elaborate noun phrases, particles, etc. When comparing grammatical functions, I found more post-predicate subjects than objects and goals. However, I do not have data about the relative frequency. Thus, it might be the case that for functions such as subjects, locations or goals the probability of occurring in a position after the verb is higher than for objects, but this needs to be tested in future research. Subjects are probably overall more frequently represented in the texts than objects (or goals), since the vast majority of verbs have subject arguments, but only a sub-set of them allow for or even require objects or goals. Direct reported speech constructions in which the quote precedes the verb of speech regularly put the subject (or indirect object\is{object!indirect} in the function of \isi{addressee} if there is no subject) after the verb.

The majority of (non-clausal) post-predicate elements are topical, such that referents encode given information, but occasionally one also finds focal elements. In text-initial utterances, the newly introduced referent sometimes follows the verb. There is a correlation with genre and geographical origin of speakers. More traditional genres such as legends, fairy tales and anecdotes show a smaller amount of post-predicate elements than personal accounts and autobiographies; and texts recorded in Adygea contain more post-predicate elements than those recorded in Turkey, which might be due to \isi{language contact} (mainly between Russian\il{Russian} and Adyghe\il{Circassian!Adyghe} speakers in Adygea).

The other Northwest Caucasian languages seem to behave similarly to Adyghe. For Ubykh\il{Circassian!Ubykh}, which is no longer spoken, texts gathered in Turkey point to a very strong tendency for verb-final \isi{word order}. This fits well to the observed difference between the Adyghe\il{Circassian!Adyghe} texts from Turkey and those from Adygea. It also matches with the observations on Laz\il{Kartvelian!Laz}, a Kartvelian outlier in Turkey (\citetv{chapters/10_Forker_EC}).

Finally, when comparing Northwest Caucasian languages to the other two indigenous language families in the Caucasus (see \citetv{chapters/10_Forker_EC}) we can safely state that Northwest Caucasian languages shows the biggest preference for verb-final order and thus the lowest number of post-predicate items.

\section*{Abbreviations}
\begin{tabularx}{.45\textwidth}{@{}lQ@{}}
\textsc{1} & first person \\
\textsc{2} & second person \\
\textsc{3} & third person \\
A & agent \\
\textsc{abs} & {absolutive} \\
\textsc{add} & additive \\
\textsc{adj} & {adjective} \\
\textsc{adv} & adverbial \\
\textsc{assert} & assertive \\
\textsc{aux} & {auxiliary} \\
\textsc{ben} & {benefactive} \\
\textsc{caus} & causative \\
\textsc{cisl} & cislocative \\
\textsc{cln} & non-human numeral classifier \\
\textsc{com} & {comitative} \\
\textsc{cond} & conditional \\
\textsc{coord} & coordination \\
\textsc{cs} & causal \\
\textsc{cvb} & converb \\
\textsc{dat} & {dative} \\
\textsc{dcl} & declarative \\
\textsc{def} & definite \\
\textsc{dir} & directional \\
\textsc{dyn} & dynamic \\
\textsc{el} & elative \\
\textsc{erg} & {ergative} \\
\textsc{fin} & finite \\
\textsc{fut} & future \\
\textsc{inst} & instrumental \\
\end{tabularx}%
\begin{tabularx}{.45\textwidth}{@{}lQ@{}}
\textsc{intf} & intensifier \\
IO & indirect object\is{object!indirect} \\
\textsc{lat} & lative \\
\textsc{lnk} & linking element \\
\textsc{loc} & {locative} \\
\textsc{m} & male \\
\textsc{mal} & malefactive \\
\textsc{mod} & modal \\
\textsc{n} & neuter \\
\textsc{neg} & negation \\
\textsc{obl} & {oblique} \\
\textsc{pl} & plural \\
\textsc{pos}s & possession \\
\textsc{pr }& possessor series of personal prefixes \\
\textsc{prt} & particle \\
\textsc{pst} & past tense \\
\textsc{q} & question marker \\
\textsc{qn} & non-human question \\
R & Russian loan \\
\textsc{re} & refactive \\
\textsc{rec} & reciprocal \\
\textsc{rel} & relativizer \\
\textsc{rsn} & reason \\
\textsc{sg} & singular \\
T & Turkic loan \\
\textsc{temp} & temporal \\
\textsc{trm} & terminative \\
\textsc{ver} & version \\
\\
\end{tabularx}


\section*{Digital corpora}
\subsection*{Adyghe}

\begin{itemize}
  \item \fullcite{paris1974princesse}
\item \fullcite{ThePanglossCollectionDetchiyimko}
\end{itemize}


\subsection*{Ankara Turkish\il{Turkic!Turkish Ankara}}
\begin{itemize}
  \item \fullcite{Iefremenko2021}
\end{itemize}






% \section*{Abbreviations}
% \begin{tabularx}{.45\textwidth}{lQ}

% \end{tabularx}
% \begin{tabularx}{.45\textwidth}{lQ}

% \end{tabularx}

\sloppy
\printbibliography[heading=subbibliography,notkeyword=this]

\end{document}
