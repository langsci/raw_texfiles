\documentclass[output=paper,colorlinks,citecolor=brown,draftmode]{langscibook}
\ChapterDOI{10.5281/zenodo.14266345}
\author{Mehdi Parizadeh\affiliation{Bu-Ali Sina University, Hamedan} and Mohammad Rasekh-Mahand\affiliation{Bu-Ali Sina University, Hamedan}}
\title{Post-predicate elements in Early New Persian (10--13th Century CE)}
\abstract{In this chapter, we investigate post-predicate elements in written Early New Persian texts from four different sources, covering the 10th to 13th centuries. The analysis shows that post-predicate elements are overall far less frequent than in contemporary spoken Persian corpora, and that the effect of semantic role is negligible. In the written texts investigated here, post-verbal elements do not form a semantically homogenous group, and goals are not more prone to postposing than other roles. However, we do find an overall effect of weight, with heavier constituents more likely to be postposed, as well as an effect of register.}

%move the following commands to the ``local...'' files of the master project when integrating this chapter
% \usepackage{tabularx}
% \usepackage{langsci-optional}
% \usepackage{langsci-gb4e}
% \usepackage{enumitem}
% \bibliography{localbibliography}
% \newcommand{\orcid}[1]{}
% \let\eachwordone=\itshape


\IfFileExists{../localcommands.tex}{
 \addbibresource{../collection_tmp.bib}
 \addbibresource{../localbibliography.bib}
 % add all extra packages you need to load to this file

\usepackage{tabularx,multicol}
\usepackage{url}
\urlstyle{same}

\usepackage{listings}
\lstset{basicstyle=\ttfamily,tabsize=2,breaklines=true}

\usepackage{langsci-basic}
\usepackage{langsci-optional}
\usepackage{langsci-lgr}
\usepackage{langsci-osl}
% \usepackage{./langsci/styles/langsci-lgr}
% \usepackage{./langsci/styles/langsci-osl}
% \usepackage{langsci-gb4e}

\usepackage{tikz}
\usetikzlibrary{patterns,calc}
\pgfdeclarepatternformonly{south east lines}{\pgfqpoint{-0pt}{-0pt}}{\pgfqpoint{3pt}{3pt}}{\pgfqpoint{3pt}{3pt}}{
    \pgfsetlinewidth{0.6pt}
    \pgfpathmoveto{\pgfqpoint{0pt}{3pt}}
    \pgfpathlineto{\pgfqpoint{3pt}{0pt}}
    \pgfpathmoveto{\pgfqpoint{.2pt}{-.2pt}}
    \pgfpathlineto{\pgfqpoint{-.2pt}{.2pt}}
    \pgfpathmoveto{\pgfqpoint{3.2pt}{2.8pt}}
    \pgfpathlineto{\pgfqpoint{2.8pt}{3.2pt}}
    \pgfusepath{stroke}}
    
\usepackage{stmaryrd}
\usepackage{wasysym}
\usepackage{multirow}
\usepackage{caption}
\usepackage{subcaption}
\usepackage{mathrsfs}
\usepackage{qtree}

\usepackage{linguex}


 %pminos do not split footnotes
% \interfootnotelinepenalty=10000 %Footnote in Laporte chapters has to be split SN


%\DeclareIndexNameFormat{default}{%
%\nameparts{#1}%
%\usebibmacro{index:name}%
%{\index[names]}%
%{\namepartfamily}%
%{\namepartgiveni}%
% {}% L1
% {}% L2
%{\namepartprefix}% generates spurious space L3
%{\namepartsuffix}% generates spurious space L4
%}

%  {\DeclareIndexNameFormat{default}{%
%     \usebibmacro{index:name}{\index[names]}{#1}{#3}{#5}{#7}}}

%\DeclareIndexNameFormat{default}{%
%  \usebibmacro{index:name}{\sindex[nom]}{#1}{#3}{#5}{#7}}

%\DeclareIndexNameFormat{default}{%
%  \usebibmacro{index:name}{\sindex[person]}{#1}{#3}{#5}{#7}}
%\DeclareIndexNameFormat{default}{%
%\nameparts{#1} \usebibmacro{index:name}{\sindex[person]]}{\namepartfamily}{‌​\namepartgiven}{\nam‌​epartprefix}{\namepa‌​rtsuffix}}

%\newcommand{\smiley}{:)}

%\renewbibmacro*{index:name}[5]{%
%\usebibmacro{index:entry}{#1}%
%{\iffieldundef{usera}{}{\thefield{usera}\actualoperator}\mkbibindexname{#2}{#3}{#4}{#5}}}

% \newcommand{\noop}[1]{}

%remove for final
%\overfullrule=1mm

\newcommand{\tobi}[2]}}
\renewcommand{\S}[1]{\tobi{#1}{\textsc{*}}}

% this volume references
% puts: [this volume]
% already defined: \citetv
%\newcommand{\citepv}[1]{(\citeauthor{#1} \citeyear*{#1} [this volume])}
\newcommand{\citealtv}[1]{\citeauthor{#1} \citeyear*{#1} [this volume]}

%parentheses around example number
\newcommand{\pref}[1]{(\ref{#1})}

% in-text examples

\newcommand{\lnex}[1]{\textit{#1}} %target lang word
\newcommand{\lnlit}[1]{(lit.: `#1')} %literal reading
\newcommand{\lnlat}[1]{(#1)} % latinization
\newcommand{\lntrans}[1]{`#1'} %translation
\newcommand{\lnexl}[2]%
{\lnex{#1}{} \lnlat{#2}} % ex with latinization
\newcommand{\lnexlat}[3]{\lnex{#1}{} \lnlat{#2}{} \lntrans{#3}} % ex with latinization and tranl.

%ch01
\newcommand{\co}[1]{\mbox{\textbf{#1}}}

%ch09

\newcommand{\cyrbulg}[1]{\begin{otherlanguage*}{bulgarian}#1\end{otherlanguage*}}


%ch10
\newcommand{\nlp}{{\small NLP}}
\newcommand{\mwe}{{\small MWE}}
\newcommand{\rae}{{\small RAE}}
\newcommand{\lvc}{{\small LVC}}
\newcommand{\pos}{{\small P}o{\small S}}
%\newcommand{\todo}[1]{ \textcolor{red}{#1} }

%\renewcommand{\labelenumi}{\theenumi}
%\ainamefmt{{vv}{ll}{, ff}{, jj}} % fullname

\newcommand{\biberror}[1]{{\color{red}#1}}

\newcommand{\osenovaitem}{--~}
 %% hyphenation points for line breaks
%% Normally, automatic hyphenation in LaTeX is very good
%% If a word is mis-hyphenated, add it to this file
%%
%% add information to TeX file before \begin{document} with:
%% %% hyphenation points for line breaks
%% Normally, automatic hyphenation in LaTeX is very good
%% If a word is mis-hyphenated, add it to this file
%%
%% add information to TeX file before \begin{document} with:
%% %% hyphenation points for line breaks
%% Normally, automatic hyphenation in LaTeX is very good
%% If a word is mis-hyphenated, add it to this file
%%
%% add information to TeX file before \begin{document} with:
%% \include{localhyphenation}
\hyphenation{
    Beck-man
    Ngu-yen
    back-chan-nel
    back-chan-nels
    mo-not-o-nous
    ste-reo-typ-i-cal
}

\hyphenation{
    Beck-man
    Ngu-yen
    back-chan-nel
    back-chan-nels
    mo-not-o-nous
    ste-reo-typ-i-cal
}

\hyphenation{
    Beck-man
    Ngu-yen
    back-chan-nel
    back-chan-nels
    mo-not-o-nous
    ste-reo-typ-i-cal
}

%  \boolfalse{bookcompile}
%  \togglepaper[5]%%chapternumber
}{}

\begin{document}
\maketitle\label{WOWA:ch:8}

\section{Introduction}\label{ENP:ss:1}

The history of Persian language is normally divided into three main eras: Old Persian\il{Persian (Old)} (OP) (6th to 4th century B.C.E), Middle Persian\il{Persian (Middle)} (MP) (3rd to 7th century C.E.), and New Persian\il{Persian (New)} (NP) (8th to present time). The New Persian\il{Persian (New)} era is subdivided differently by different scholars (see \citealt{paul_persian_2018}: 572--576), but normally the first four centuries are referred to as Early New Persian\il{Persian (Early New)}, which has some peculiar morpho-syntactic features. \citet[576]{paul_persian_2018} gives the following subdivisions: Early New Persian (8th-12th centuries), Standard New Persian\il{Persian (New)} (13th -19th centuries) and Modern (High New) Persian\il{Persian (New)} (19th century-present). In this study, we analyze the post-predicate elements from a selection of texts written in Early New Persian\il{Persian (Early New)}. Since there are no authentic written materials from the first two centuries, our corpus covers materials from 10th to 13th centuries. Post-predicate elements are reportedly frequent in contemporary spoken Persian\il{Persian (colloquial)} (\citealt{frommer_post-verbal_1981}, \citetv{chapters/7_RasekhMahandetal_Persian}). \citet{lazard_langue_1963} is among the first studies which dealt with constituent order in ENP. This study focuses on written materials from the first records of New Persian\il{Persian (New)}, in search of similarities and differences between them. A further important source for ENP are Early Judaeo-Persian\il{Persian (Early Judeo)} texts, written mainly from the 10th--12th centuries CE. The Early Judaeo-Persian\il{Persian (Judeo)} corpus contains autographs of private letters that represent an authentic colloquial style that is hardly found in ENP literary texts, their analysis could broaden our understanding of \isi{word order} in ENP \citep{paul_grammar_2013}, but lie beyond the scope of the current study. Persian boasts a rich legacy of written texts, affording scholars the opportunity to scrutinize the attributes of this language over the course of centuries. Although various aspects of the language have been examined from a historical standpoint, no prior research has addressed the post-predicate phenomena. By shedding light on \isi{word order} in Early New Persian\il{Persian (Early New)}, this study brings a much-needed historical perspective into the discussion and a point of departure for quantitative studies on historical Persian (and Iranian) syntax. It also contrasts with the majority of studies in this volume, which \isi{focus} on spoken language

% \setlength{\footheight}{59.06021pt} 
\largerpage

\section{The corpus}\label{ENP:ss:2}

Four books from the Early New Persian\il{Persian (Early New)} were selected: \textit{Tārikh Tabari} (10th cent.), \textit{Qābusnameh} (11th cent.), \textit{Tazkerat al-Oliyā} (12th cent.) and \textit{Fihe mā Fih} (13th cent.). Regarding the sampling methodology, approximately 2000 words (five pages) were selected from each book and included in the sample. It is noteworthy that all the selected pages were the first pages of their respective books, with the exception of \textit{Tārikh Tabari} (10th cent.) which was limited to pages 414--419. These samples yielded in total 2228 tokens of non-subject constituents for the analysis. Further details about these texts are briefly introduced in the subsequent sections. 

\textit{Tārikh Tabari} (10th cent.) is written by Mohammad Ibn Jarir al-Tabari, who was a Persian historian and Islamic scholar from Tabarestan, north of Iran. This book is one of the most reliable and famous references for understanding the history of religions and prophets in Iran. The book begins with the creation time and recounts stories of the prophets and kings until the time of the prophet of Islam. In the first part, it narrates the history before Islam, and in the second part, the history after Islam. This book is among the oldest texts available in Early New Persian\il{Persian (Early New)}. From the 11th century, we have used \textit{Qābusnameh}, written by Onsor al-Ma'ali KeyKavus Ibn Iskandar-e Ziyāri, the ruler of parts of Tabarestan, north of Iran. It is arranged in 44 chapters besides an introduction. It is dedicated to his son, Gilānšāh. In this book, he gives advice to his son as a guidance for his governance. He discusses issues such as military practices and social customs. 

\textit{Tazkerat al-Oliyā} (12th cent.) is a hagiographic collection of ninety-six Sufi stories by the Persian poet and mystic, Farīd al-Dīn Aṭṭār. Aṭṭār's only surviving prose work has 72 chapters, beginning with the life of the Sixth Shi'a Imam, and ending with the Sufi Martyr, Mansur Hallāj. The lives of the Sufis in this book are set in a more or less uniform format. Each biography starts with a set of embellished phrases, rhyming with one another and mentioning the subject's name, and alluding to his or her attributes before expounding on them through stories about their lives, and then by quotations from their sayings. We have excluded these sections from our analysis. 

\textit{Fihe Mā Fih} (13th cent.) is in prose composed by Molānā Jalāl al-Din Mohammad Balkhi (Rumi). The subject of \textit{Fihe Mā Fih} is mystical criticism and Rumi's interpretations of sacred texts. It includes notes written by his students, compiled over the course of 30 years. The text is simple, and it contains thoughts on mysticism, religion and morality.

\begin{sloppypar}
\tabref{ENP:tab:1} shows the general information and the overall frequency of post-predicate non-subject elements in these selected texts. (TT stands for \textit{Tārikh Tabari} (10th cent.), QA stands for \textit{Qābusnāmeh} (11th cent.), TO stands for \textit{Tazkerat al-Oliyā} (12th cent.), FF stands for \textit{Fihe Mā Fih} (13th cent).
\end{sloppypar}

\begin{table}
 \centering
\fittable{\begin{tabular}{lrrrr}
\lsptoprule
 & TT (10 c.) & QA (11 c.) & TO (12 c.) & FF (13 c.) \\
\midrule
Sample length (words) & 2945 & 2322 & 2447 & 2015 \\
Total number of tokens & 694 & 521 & 541 & 505 \\
Number of non-classified tokens & 166 & 81 & 109 & 100 \\
\% post-pred tokens & 0\% & 9.7\% & 4.4\% & 3.7\% \\
\lspbottomrule
 \end{tabular}}
 \caption{Overview of the Early New Persian text corpus}
 \label{ENP:tab:1}
\end{table}

\section{Post-predicate elements in different roles}\label{ENP:ss:3}

The analysis reveals that the rate of post-predicate elements in a written corpus derived from Early New Persian\il{Persian (Early New)} exhibits certain peculiarities in comparison to the spoken-language corpora in WOWA. The primary observation is that the overall frequency of post-predicate elements is markedly low. The range of post-verbal elements varies between zero in \textit{Tārikh Tabari} to approximately 10 percent in \textit{Qābusnāmeh}, while \textit{Tazkerat al-Oliyā} and \textit{Fihe Mā Fih} have an average of approximately 4 percent. Out of the total of 1819 tokens analyzed, only 89 were found to be in post-predicate position, which amounts to an average of 4.8 percent. 

The second point to be noted is that the occurrence of post-predicate elements in written discourse is significantly influenced by the \isi{register}, the content of the text, and the personal style of the writer. For instance, in \textit{Tārikh Tabari} (10th century), which is a historical account of important events and figures, the writer has adopted a highly formal writing style, and consequently, no elements in post-predicate position are found. Conversely, in \textit{Qābusnāmeh} (11th century), which is a father's advice to his son, the writer employs a more informal style due to the subject matter, and as a result, the text displays the highest frequency of post-posed elements. Thus, the variation in the frequency of post-predicate elements may be attributed to the \isi{register}, as demonstrated by \citet{frommer_post-verbal_1981} for different varieties of spoken and written informal Persian (\citetv{chapters/7_RasekhMahandetal_Persian}). The number of analyzed tokens and frequency of post-predicate elements for the whole corpus is presented in \tabref{ENP:tab:2}:

\begin{table}
 \begin{tabularx}{\textwidth}{lYY}
 \lsptoprule
Total length & 2261 & 100\% \\
Number of analyzed tokens & 1807 & 79.9\% \\
Number of non-classified tokens & 454 & 20.1\% \\
Rate of post-predicate elements (all roles) & 77 & 4.3\% \\
\lspbottomrule
 \end{tabularx}
 \caption{Frequency of post-predicate elements in Early New Persian}
 \label{ENP:tab:2}
\end{table}

The non-classified tokens are mainly those which do not have a verb and could not be analyzed. According to the \tabref{ENP:tab:2}, the overall frequency of post-posed elements in Early New Persian\il{Persian (Early New)} is determined to be about 4.8\%. As no comparable study on written texts in Persian is available, a comparison with such texts could not be made. However, when compared with other studies on Persian post-predicate elements, the rate is found to be significantly lower than those observed in spoken New Persian\il{Persian (colloquial)} (\citealt{frommer_post-verbal_1981}, \citetv{chapters/7_RasekhMahandetal_Persian}). \citet{frommer_post-verbal_1981} reports 16.6\% and 12.6\% of post-predicate elements for informal and formal spoken Persian, respectively. \textcitetv{chapters/7_RasekhMahandetal_Persian} report 18.7\% for public \isi{register} and 26.8\% for private \isi{register}. Contemporary research on \isi{word order} in formal written Persian\il{Persian (New)} does not actually consider post-predicate elements, focusing solely on the ordering of elements (e.g. direct and indirect objects) before the verb (\citealt{FaghiriSamvelian2014Accesibility}, \citealt{FaghiriSamvelian2020SOV}). Thus, the general observation is that post-predicate phenomena are less frequent in writing.

In the remaining part of this section, we examine the different roles and frequency of elements in post-predicate position, and provide examples from various roles. It is worth noting that most of the roles contain a limited number of tokens, and post-verbal placement is characterized by exceptionally low frequencies. In Early New Persian\il{Persian (Early New)}, the \isi{benefactive} \isi{role} exhibits the highest proportion of post-predicate elements, with 10 out of 57 occurrences appearing in post-predicate position. The following are some examples: 

\ea\label{ENP:ex:1}
Benefactive: \\
Early New Persian \il{Persian (Early New)}(\citealt{parizadeh_persian_2022}: C, 259) \\
\gll tazkere-i sāxt-am \textbf{oliyā} \textbf{rā} \\
biography\textsc{-indf} build\textsc{.pst-1sg} clergies \textsc{ra} \\
\glt `I created a biography for the clergies.'
\z

\begin{sloppypar}
In example (\ref{ENP:ex:1}), the \isi{benefactive} is marked with \textit{rā}. According to \citet{rasekh-mahand2024different}, it is not uncommon for the \isi{benefactive} \isi{role} to be marked by \textit{rā} in addition to other roles in Early New Persian\il{Persian (Early New)} texts. Out of the ten tokens of benefactives in post-predicate position, four are marked by \textit{rā}. 
\end{sloppypar}

Tokens categorized as ``Other'' represent the most frequent type of \isi{role} found in post-predicate position, with a total of 50 out of 508 occurrences. The following examples demonstrate their appearance in post-verbal position:

\ea\label{ENP:ex:2}
Other: \\
Early New Persian \il{Persian (Early New)}(\citealt{parizadeh_persian_2022}: D, 221) \\
\gll va vey rā vasiyat na-kard-i \textbf{be} \textbf{tafsil} \\
and \textsc{3sg} \textsc{ra} will \textsc{neg-}do\textsc{.pst-2sg} to details \\
\glt `And you did not bequeath him in detail.'
\z

\ea\label{ENP:ex:3}
Other: \\
Early New Persian \il{Persian (Early New)}(\citealt{parizadeh_persian_2022}: B, 436) \\
\gll darviš-i miy-ām-ad \textbf{pāy} \textbf{berahne} \\
poor\textsc{-indf} \textsc{ind-}come\textsc{.pst-3sg} feet bare \\
\glt `A poor man was coming on foot.' 
\z

\ea\label{ENP:ex:4}
Other: \\
Early New Persian \il{Persian (Early New)}(\citealt{parizadeh_persian_2022}: C, 52) \\
\gll yek ketāb-e digar mi-bāyest jodāgāne \\
one book\textsc{-ez} other \textsc{ind-}must\textsc{.3sg} separately \\
\glt `There must be another book separately.' 
\z

Out of the 57 tokens classified as ablatives, only three appear in post-verbal position. Similarly, among the 42 tokens of \isi{stimulus}, only two tokens are found in post-verbal position. Out of 28 instances of instruments, only one has appeared in post-predicate position:

\ea\label{ENP:ex:5}
Ablative: \\
Early New Persian \il{Persian (Early New)}(\citealt{parizadeh_persian_2022}: B, 346) \\
\gll valikan qaraz dar ruze mehr-i ast \textbf{az} \textbf{xodāvand-e} \textbf{molk} bar molk-e xiš \\
but aim in fasting love\textsc{-indf} be\textsc{.prs.3sg} from God\textsc{-ez} world to world\textsc{-ez} self \\
\glt `The purpose of fasting is the love that God has for his created world.'
\z

\ea\label{ENP:ex:6}
Stimulus: \\
Early New Persian \il{Persian (Early New)}(\citealt{parizadeh_persian_2022}: C, 3) \\
\gll va jamāʔati az dust-ān rā raqbati tamām mi-did-am \textbf{be} \textbf{soxan-e} \textbf{in} \textbf{qom} \\
and people from friends\textsc{-pl} \textsc{ra} interested much \textsc{ind-}see\textsc{.pst-1sg} to word\textsc{-ez} this group \\
\glt `And I saw some friends were very interested in the words of this group.'
\z

\ea\label{ENP:ex:7}
Stimulus: \\
Early New Persian \il{Persian (Early New)}(\citealt{parizadeh_persian_2022}: C, 4) \\
\gll va ma-rā niz meyl-i azim bud \textbf{be} \textbf{motāleʔe-ye} \textbf{ahvāl} \textbf{va} \textbf{soxan-e} \textbf{išān} \\
and \textsc{1sg-ra} too desire\textsc{-indf} great be\textsc{.pst.3sg} to study\textsc{-ez} vita and utterance\textsc{-ez} \textsc{3pl} \\
\glt `I was very interested in studying their lives and sayings.' 
\z

\ea\label{ENP:ex:8}
Instrument: \\
Early New Persian \il{Persian (Early New)}(\citealt{parizadeh_persian_2022}: D, 110) \\
\gll ke mi-bin-am \textbf{be} \textbf{češm-e} \textbf{sar} \\
that \textsc{ind-}see\textsc{.prs-1sg} with eye\textsc{-ez} head \\
\glt `That I see through the eyes in my head.'
\z

The number of locatives in out corpus is 93, from which 2 tokens appear post-verbally:


\newpage
\ea\label{ENP:ex:9}
Locative: \\
Early New Persian \il{Persian (Early New)}(\citealt{parizadeh_persian_2022}: D, 324) \\
\gll surat-hā-ye xub namāy-ad \textbf{dar} \textbf{šekam-e} \textbf{ān} \textbf{surat-hā-ye} \textbf{bad} \\
form\textsc{-pl-ez} good show\textsc{.prs-3sg} in abdomen\textsc{-ez} that form\textsc{-pl-ez} bad \\
\glt `He shows good forms inside bad forms.'
\z

\ea\label{ENP:ex:10}
Locative: \\
Early New Persian \il{Persian (Early New)}(\citealt{parizadeh_persian_2022}: C, 297) \\
\gll ke in če dard bud-e ast \textbf{dar} \textbf{jān-hā-ye} \textbf{išān} \\
that this what pain be\textsc{.pst-ptcpL} is in heart\textsc{-pl-ez} their \\
\glt `That which was a pain in their hearts.' 
\z

In all other Iranian languages examined in the WOWA data set, the highest rate of post-verbal elements is associated with goals of motion and caused motion verbs. However, in Early New Persian\il{Persian (Early New)} texts, only one instance out of 59 was discovered in post-verbal position, and there is no indication of a tendency toward goal-last placement as observed in other languages.

\ea\label{ENP:ex:11}
Goal: \\
Early New Persian \il{Persian (Early New)}(\citealt{parizadeh_persian_2022}: B, 223) \\
\gll va hame-ye peyqāmbar-ān rā be rāstgui dān-ad az ādam tā \textbf{peyqāmbar-e} \textbf{mā} \textbf{Mohammad} \\
and all\textsc{-ez} prophet\textsc{-pl} RA to truthfulness know\textsc{.prs-3sg} from Adam till prophet\textsc{-ez} \textsc{1pl} mohammad \\
\glt `And he considers all the prophets from Adam to our Prophet Mohammad to be truthful.' 
\z

Out of the 67 tokens of complements of `become', only one is found in post-predicate position. Complements of copular verbs typically precede the verb, but among the 304 instances, three have been observed in post-verbal position.

\ea\label{ENP:ex:12}
Complements of `become': \\
Early New Persian \il{Persian (Early New)}(\citealt{parizadeh_persian_2022}: B, 69) \\
\gll yā be tarkib az do bov-ad \textbf{čon} \textbf{jesm} \\
or in combination from two become\textsc{.prs-3sg} like body \\
\glt `And in combination it becomes one from two things like body.'
\z


\newpage
\ea\label{ENP:ex:13}
Copula complements: \\
Early New Persian \il{Persian (Early New)}(\citealt{parizadeh_persian_2022}: B, 2) \\
\gll ke hiččiz ni-st \textbf{az} \textbf{budani} \textbf{va} \textbf{nābudani} \\
so nothing \textsc{neg-}be\textsc{.prs.3sg} from being and non-being \\
\glt `There is nothing either being or non-being.'
\z

\ea\label{ENP:ex:14}
Copula complements: \\
Early New Persian \il{Persian (Early New)}(\citealt{parizadeh_persian_2022}: B, 15) \\
\gll va mesāl-e šenāxtan čon manquš ast \textbf{va} \textbf{šenāsande} \textbf{čon} \textbf{naqāš} \\
and example\textsc{-ez} recognition like paint is and recognizer like painter \\
\glt `And the example of recognition to recognizer is like paint to painter.' 
\z

In Early New Persian\il{Persian (Early New)}, direct objects typically appear before the verb. In our dataset, only four tokens out of 464 were observed in post-verbal position:

\ea\label{ENP:ex:15}
Direct \isi{object}: \\
Early New Persian \il{Persian (Early New)}(\citealt{parizadeh_persian_2022}: C, 115) \\
\gll tā be-dān-i \textbf{fazl-e} \textbf{išān} \textbf{va} \textbf{eflās-e} \textbf{xod} \\
until \textsc{sbjv-}know\textsc{.prs-2sg} privilege\textsc{-ez} \textsc{3pl} and misery\textsc{-ez} self \\
\glt `So that you understand their privilege and your own misery.' 
\z

Thus far, we have examined the placement of various constituents in relation to the verb in Early New Persian\il{Persian (Early New)}. In terms of preverbal arguments, benefactives and other roles exhibit a relatively high frequency in post-predicate position, accounting for 60 out of 77 post-verbal tokens. On the other hand, some roles such as \isi{addressee}, \isi{comitative}, \isi{recipient}, and \isi{possessive} are not attested in post-predicate position. Other roles, comprising only 16 tokens, are rarely observed in post-predicate position. \tabref{ENP:tab:3} shows the total number of clauses in relation to post and preverbal positions for each verb \isi{argument}.

\begin{table}
\fittable{\begin{tabular}{l@{}rrrrr}
\lsptoprule
& \textbf{Total number} \\
 & \textbf{of clauses} & \textbf{preverbal} & \textbf{post-verbal} & \textbf{preverbal} & \textbf{post-verbal} \\
\midrule
\textbf{Benefactive} & 57 & 47 & 10 & 82.5\% & 17.5\% \\
\textbf{Other} & 508 & 458 & 50 & 91.2\% & 9.8\% \\
\textbf{Ablative} & 57 & 54 & 3 & 94.7\% & 5.3\% \\
\textbf{Stimulus} & 42 & 40 & 2 & 95.2\% & 4.8\% \\
\textbf{Instrumental} & 28 & 27 & 1 & 96.4\% & 3.6\% \\
\textbf{Locative} & 93 & 91 & 2 & 97.9\% & 2.1\% \\
\textbf{Goal} & 59 & 58 & 1 & 98.3\% & 1.7\% \\
\textbf{`become' complement} & 67 & 66 & 1 & 98.5\% & 1.5\% \\
\textbf{Copular complement} & 304 & 301 & 3 & 99\% & 1\% \\
\textbf{Direct object} & 464 & 460 & 4 & 99.2\% & 0.8\% \\
\textbf{Addressee} & 46 & 46 & 0 & 100\% & 0\% \\
\textbf{Comitative} & 17 & 17 & 0 & 100\% & 0\% \\
\textbf{Goal (caused motion)} & 14 & 14 & 0 & 100\% & 0\% \\
\textbf{Recipient} & 18 & 18 & 0 & 100\% & 0\% \\
\textbf{Recipient + Benefactive} & 16 & 16 & 0 & 100\% & 0\% \\
\textbf{Possessive} & 17 & 17 & 0 & 100\% & 0\% \\
\textbf{Total} & 1807 & 1730 & 77 & 95.7\% & 4.3\% \\
\lspbottomrule
 \end{tabular}}
 \caption{Early New Persian }
 \label{ENP:tab:3}
\end{table}

\figref{ENP:fig:1} illustrates the proportions of post-predicate placement for different constituents in Early New Persian\il{Persian (Early New)}.

\begin{figure}
 \includegraphics[width=.9\linewidth]{figures/ENP_fig1.png}
 \caption{Post-predicate placement of different constituents in Early New Persian}
 \label{ENP:fig:1}
\end{figure}

\tabref{ENP:tab:4} presents the frequency of post-predicate elements in each century, with the exception of \textit{Tārikh Tabari} (10th century), which did not contain any post-predicate elements and is therefore not included in this table.

\begin{table}
\begin{tabularx}{\textwidth}{lYYr}
\lsptoprule
&  {QA. 11th cent.} &  {TO. 12th cent.} &  {FF. 13th cent.} \\
 \midrule
  {benefactive} &  5 & 4 & 1 \\
 other&31 & 10 & 9 \\
  {ablative} &2 & 1 & 0 \\
  {stimulus} &0 & 2 & 0 \\
  {instrument} &1 & 0 & 0 \\
  {locative} &1 & 1 & 0 \\
  {Goal} &1 & 0 & 0 \\
  `become' complements &1 & 0 & 0 \\
   {copula} complements &3 & 0 & 0 \\
 direct  {object}& 0 & 1 & 3 \\
 \midrule
 Total & 45 & 19 & 13 \\
\lspbottomrule
 \end{tabularx}
 \caption{Post-predicate elements in three centuries}
 \label{ENP:tab:4}
\end{table}

\tabref{ENP:tab:4} reveals that the majority of post-predicate elements occur in \textit{Qābusnāmeh} (11th century), which is known for its informal \isi{register} in comparison to other texts.

\section{Heaviness}\label{ENP:ss:4}

The impact of \isi{heaviness} or \isi{weight} on \isi{word order} has been extensively discussed in the field of linguistics, with scholars such as \citet{behaghel_beziehungen_1909}, \citet{quirk_grammar_1972}, \citet{Hawkins1994performance}, \citet{Wasow1997weight}, and \citet{ArnoldLosongco2000} among others exploring this phenomenon. Some argue that in \isi{VO} languages, short constituents tend to precede heavy ones (\citealt{Wasow1997weight}, \citealt{Stallingsetal1998Phrasal}, \citealt{Hawkins1990universals,Hawkins1994performance}), while others maintain that the reverse order holds \citep{YamashitaChang2001headfinal}. Several studies have examined the effect of \isi{weight} on \isi{word order} in Persian\il{Persian (Early New)} (e.g., \citealt{RasekhMahand2016Relative}, \citealt{FaghiriSamvelian2014Order,FaghiriSamvelian2020SOV}, \citealt{FaghiriSamvelian2014Accesibility,Faghirietal2018Canonical}), providing various analyses and occasionally conflicting results (for a detailed review, see \citetv{chapters/7_RasekhMahandetal_Persian}). We investigated the impact of \isi{weight} on post-predicate placement in our dataset, adopting the basic classification of constituent \isi{weight} applied in WOWA, which recognizes four classes: w1, consisting of one phonological word; w2, consisting of two phonological words; w3, consisting of three phonological words; and w4, consisting of four or more words. \tabref{ENP:tab:5} demonstrates that as the \isi{weight} of the token increases, the likelihood of appearing in post-predicate position also increases.

\begin{table}
 \begin{tabularx}{.8\textwidth}{lrYYY}
\lsptoprule
Weight & W1 & W2 & W3 & W ≥ 4 \\
\midrule
Post-predicate & 27 & 28 & 13 & 9 \\
Total & 1100 & 471 & 143 & 93 \\
Percent & 2.5\% & 5.9\% & 9\% & 9.7\% \\
\lspbottomrule
 \end{tabularx}
 \caption{Percentages of post-verbal placement according to weight (across all constituent types)}
 \label{ENP:tab:5}
\end{table}

The transition between groups w1 and w2, as well as between w2 and w4, exhibits a noticeable jump in post-predicate placement probability, as illustrated in \figref{ENP:fig:2}. While the difference between groups w3 and w4 is not as substantial, it is still discernible.

\begin{figure}
 \includegraphics[width=.8\textwidth]{figures/ENP_fig2.png}
 \caption{Weight effects on post-predicate elements in Early New Persian}
 \label{ENP:fig:2}
\end{figure}
 
Our findings indicate that \isi{weight} is a factor for predicting post-predicate placement in written Early New Persian\il{Persian (Early New)}. This is consistent with much of the corpus-based literature (\citealt{Hawkins1994performance}, \citealt{Stallingsetal1998Phrasal}, \citealt{ArnoldLosongco2000}), and is particularly intriguing in the context of this volume, where \isi{weight} has not yielded consistent or significant results in some of the other languages studied (see \citetv{chapters/7_RasekhMahandetal_Persian}). The most plausible explanation for this discrepancy is the fact that the Early New Persian\il{Persian (Early New)} corpus is the only written corpus included in the WOWA collection; see \citet{SchnellSchiborr2022Cross} for empirical evidence for the differences between spoken and written corpora in this respect). However, it may also be connected to the content of the texts; this remains to be investigated.

\section{Summary}\label{ENP:ss:5}

In this chapter, we examined post-predicate elements in written Early New Persian\il{Persian (Early New)} texts. Our analysis revealed that there are relatively few post-predicate elements in these texts compared to related spoken materials, with an average of 4.3\%. However, the post-predicate elements that do occur exhibit significant syntactic and semantic diversity, encompassing a range of distinct functions. The most common roles are benefactives and other items that are not easily classified using the WOWA tagging set (coded as ``other''). Notably, the ``goals last'' effect commonly observed in spoken-language corpora studied in this volume is not present in written Early New Persian\il{Persian (Early New)}. Nevertheless, our findings suggest that \isi{weight} does play a \isi{role} in post-verbal phenomena in these written texts, with longer constituents being more likely to appear in post-predicate position.

The general paucity of post-verbal elements when compared to contemporary spoken Persian\il{Persian (colloquial)} (\citetv{chapters/7_RasekhMahandetal_Persian}), could be ascribed to at least three different causes: a difference in medium (spoken versus written), a difference in chronological stage of the language, or a difference in \isi{register}, or some combination thereof. For obvious reasons, we have no reliable record of the spoken language in the ENP stage, so it is impossible to say whether our written texts faithfully reflect the language as it was spoken at the time. However, we do have both spoken and written texts for contemporary Persian, and initial findings suggest that there is a considerable difference between them with regard to post-verbal elements. Overall, it can be noted that less formal registers favour greater frequency of post-verbal elements, and spoken language is overall much more likely to exhibit high frequencies of post-verbal elements (\citetv{chapters/7_RasekhMahandetal_Persian}). Our data also exhibit a slight effect of \isi{register} (the least formal text has the highest overall rate of post-verbal elements), so we are inclined to consider the \isi{register} and medium effects as persistent characteristics of the Persian culture of literacy over the last 1000 years. In other words, we assume that the spoken language of the ENP period was probably significantly different with respect to post-verbal elements, though the magnitude of the difference is impossible to estimate. We therefore urge caution in interpreting our results as baselines for ``the'' Persian language; rather we assume that our results reflect quite specific characteristics of written language, which do not necessarily faithfully reflect the spoken language of the period.

\section*{Abbreviations}
\begin{tabularx}{.45\textwidth}{@{}lQ@{}}
1 & first person \\
2 & second person \\
3 & third person \\
\textsc{ez} & ezafe \\
\end{tabularx}
\begin{tabularx}{.45\textwidth}{@{}lQ@{}}
\textsc{ind} & indicative \\
\textsc{neg} & negator \\
\textsc{pl} & plural \\
\textsc{prs} & present \\
\end{tabularx}

\begin{tabularx}{.45\textwidth}{@{}lQ@{}}
\textsc{pst} & past \\
\textsc{ptcpl} & participle \\
\textsc{ra} & object-marking {clitic} \textit{=rā}\\
\textsc{sbjv} & subjunctive \\
\end{tabularx}
\begin{tabularx}{.45\textwidth}{@{}lQ@{}}
\textsc{sg} & singular \\
V & {verb} \\
WOWA & = \citet{Haig.Stilo.Dogan.Schiborr2022} \\
\\
\end{tabularx}

\sloppy
\printbibliography[heading=subbibliography,notkeyword=this]

\end{document}
