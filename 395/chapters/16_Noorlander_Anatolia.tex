\documentclass[output=paper,colorlinks,citecolor=brown,draftmode]{langscibook}
\ChapterDOI{10.5281/zenodo.14266361}
\author{Paul M. Noorlander\orcid{0000-0002-9407-1453}\affiliation{University of Cambridge}}
\title{Arabic and Aramaic in eastern Anatolia}
\abstract{This chapter offers a brief overview of the word order typology of \textit{Qəltu}-Arabic and Neo-Aramaic dialects spoken by minorities in southeastern Anatolia. Constituent ordering is generally consistent with the typology of VO languages, and representative of the majority of Central Semitic languages. Convergence with local languages, however, has resulted in the development of often post-predicate copulas and a higher rate of OV order, and in some doculects even a complete shift to OV.} 

%move the following commands to the "local..." files of the master project when integrating this chapter
% \usepackage{tabularx}
% \usepackage{langsci-optional}
% \usepackage{langsci-gb4e}
% \usepackage{enumitem}
% \bibliography{localbibliography}
% \newcommand{\orcid}[1]{}
% \let\eachwordone=\itshape

\IfFileExists{../localcommands.tex}{
 \addbibresource{../collection_tmp.bib}
 \addbibresource{../localbibliography.bib}
 % add all extra packages you need to load to this file

\usepackage{tabularx,multicol}
\usepackage{url}
\urlstyle{same}

\usepackage{listings}
\lstset{basicstyle=\ttfamily,tabsize=2,breaklines=true}

\usepackage{langsci-basic}
\usepackage{langsci-optional}
\usepackage{langsci-lgr}
\usepackage{langsci-osl}
% \usepackage{./langsci/styles/langsci-lgr}
% \usepackage{./langsci/styles/langsci-osl}
% \usepackage{langsci-gb4e}

\usepackage{tikz}
\usetikzlibrary{patterns,calc}
\pgfdeclarepatternformonly{south east lines}{\pgfqpoint{-0pt}{-0pt}}{\pgfqpoint{3pt}{3pt}}{\pgfqpoint{3pt}{3pt}}{
    \pgfsetlinewidth{0.6pt}
    \pgfpathmoveto{\pgfqpoint{0pt}{3pt}}
    \pgfpathlineto{\pgfqpoint{3pt}{0pt}}
    \pgfpathmoveto{\pgfqpoint{.2pt}{-.2pt}}
    \pgfpathlineto{\pgfqpoint{-.2pt}{.2pt}}
    \pgfpathmoveto{\pgfqpoint{3.2pt}{2.8pt}}
    \pgfpathlineto{\pgfqpoint{2.8pt}{3.2pt}}
    \pgfusepath{stroke}}
    
\usepackage{stmaryrd}
\usepackage{wasysym}
\usepackage{multirow}
\usepackage{caption}
\usepackage{subcaption}
\usepackage{mathrsfs}
\usepackage{qtree}

\usepackage{linguex}


 %pminos do not split footnotes
% \interfootnotelinepenalty=10000 %Footnote in Laporte chapters has to be split SN


%\DeclareIndexNameFormat{default}{%
%\nameparts{#1}%
%\usebibmacro{index:name}%
%{\index[names]}%
%{\namepartfamily}%
%{\namepartgiveni}%
% {}% L1
% {}% L2
%{\namepartprefix}% generates spurious space L3
%{\namepartsuffix}% generates spurious space L4
%}

%  {\DeclareIndexNameFormat{default}{%
%     \usebibmacro{index:name}{\index[names]}{#1}{#3}{#5}{#7}}}

%\DeclareIndexNameFormat{default}{%
%  \usebibmacro{index:name}{\sindex[nom]}{#1}{#3}{#5}{#7}}

%\DeclareIndexNameFormat{default}{%
%  \usebibmacro{index:name}{\sindex[person]}{#1}{#3}{#5}{#7}}
%\DeclareIndexNameFormat{default}{%
%\nameparts{#1} \usebibmacro{index:name}{\sindex[person]]}{\namepartfamily}{‌​\namepartgiven}{\nam‌​epartprefix}{\namepa‌​rtsuffix}}

%\newcommand{\smiley}{:)}

%\renewbibmacro*{index:name}[5]{%
%\usebibmacro{index:entry}{#1}%
%{\iffieldundef{usera}{}{\thefield{usera}\actualoperator}\mkbibindexname{#2}{#3}{#4}{#5}}}

% \newcommand{\noop}[1]{}

%remove for final
%\overfullrule=1mm

\newcommand{\tobi}[2]}}
\renewcommand{\S}[1]{\tobi{#1}{\textsc{*}}}

% this volume references
% puts: [this volume]
% already defined: \citetv
%\newcommand{\citepv}[1]{(\citeauthor{#1} \citeyear*{#1} [this volume])}
\newcommand{\citealtv}[1]{\citeauthor{#1} \citeyear*{#1} [this volume]}

%parentheses around example number
\newcommand{\pref}[1]{(\ref{#1})}

% in-text examples

\newcommand{\lnex}[1]{\textit{#1}} %target lang word
\newcommand{\lnlit}[1]{(lit.: `#1')} %literal reading
\newcommand{\lnlat}[1]{(#1)} % latinization
\newcommand{\lntrans}[1]{`#1'} %translation
\newcommand{\lnexl}[2]%
{\lnex{#1}{} \lnlat{#2}} % ex with latinization
\newcommand{\lnexlat}[3]{\lnex{#1}{} \lnlat{#2}{} \lntrans{#3}} % ex with latinization and tranl.

%ch01
\newcommand{\co}[1]{\mbox{\textbf{#1}}}

%ch09

\newcommand{\cyrbulg}[1]{\begin{otherlanguage*}{bulgarian}#1\end{otherlanguage*}}


%ch10
\newcommand{\nlp}{{\small NLP}}
\newcommand{\mwe}{{\small MWE}}
\newcommand{\rae}{{\small RAE}}
\newcommand{\lvc}{{\small LVC}}
\newcommand{\pos}{{\small P}o{\small S}}
%\newcommand{\todo}[1]{ \textcolor{red}{#1} }

%\renewcommand{\labelenumi}{\theenumi}
%\ainamefmt{{vv}{ll}{, ff}{, jj}} % fullname

\newcommand{\biberror}[1]{{\color{red}#1}}

\newcommand{\osenovaitem}{--~}
 %% hyphenation points for line breaks
%% Normally, automatic hyphenation in LaTeX is very good
%% If a word is mis-hyphenated, add it to this file
%%
%% add information to TeX file before \begin{document} with:
%% %% hyphenation points for line breaks
%% Normally, automatic hyphenation in LaTeX is very good
%% If a word is mis-hyphenated, add it to this file
%%
%% add information to TeX file before \begin{document} with:
%% %% hyphenation points for line breaks
%% Normally, automatic hyphenation in LaTeX is very good
%% If a word is mis-hyphenated, add it to this file
%%
%% add information to TeX file before \begin{document} with:
%% \include{localhyphenation}
\hyphenation{
    Beck-man
    Ngu-yen
    back-chan-nel
    back-chan-nels
    mo-not-o-nous
    ste-reo-typ-i-cal
}

\hyphenation{
    Beck-man
    Ngu-yen
    back-chan-nel
    back-chan-nels
    mo-not-o-nous
    ste-reo-typ-i-cal
}

\hyphenation{
    Beck-man
    Ngu-yen
    back-chan-nel
    back-chan-nels
    mo-not-o-nous
    ste-reo-typ-i-cal
}

%  \boolfalse{bookcompile}
%  \togglepaper[5]%%chapternumber
}{}

\begin{document}
\maketitle\label{WOWA:ch:16}

\section{Introduction}

For centuries, Jews, Christians and Muslims have co-existed in the historical region of eastern Anatolia. Jewish (J.) and Christian (C.) communities used to speak their own Aramaic and/or Arabic variety, predating the arrival of Turkish\il{Turkic!Turkish} and Kurdish\il{Kurdish}. Aramaic was one of the principal languages in Syria, Anatolia and Mesopotamia before the Islamic period, and following the Arab conquests, most of the Jews and Christians gradually shifted to Arabic\il{Arabic}, leading to a diversity of regional and communal dialects. Nowadays, Anatolia is characterized by increasing nationalization and homogeneity within a predominantly Turkish-speaking Muslim society. The ethnic cleansing during the First World War alongside continuous persecution and systematic marginalization of minorities led to a massive displacement of these minorities. Virtually all Jews left the region under duress to Israel after 1948.

Linguistically, eastern Anatolia constitutes part of a continuum of Arabic\il{Arabic} and Aramaic dialects that once extended from Syria-Palestine to modern-day Iraq and Iran. \figref{Arabic:fig:1} displays a map of the original locations and distribution of several Arabic\il{Arabic} and Aramaic dialects in Anatolia.

\begin{figure}
    {%
    \setlength{\fboxsep}{0pt}%
    \setlength{\fboxrule}{1pt}%
    \fbox{\includegraphics[width=.95\textwidth]{figures/Figure1Arabic.jpg}}
    }%
    \caption{\textit{Qəltu}-Arabic and Neo-Aramaic dialects in southeastern Anatolia}
    \label{Arabic:fig:1}
\end{figure}

\begin{sloppypar}
The present outline takes a corpus-based approach to \isi{word order} following the design of the WOWA corpus (\citealt{Haig.Stilo.Dogan.Schiborr2022} and \citetv{chapters/1_Haigetal_Intro}) and relies on datasets from within and without WOWA. Those approaches (e.g. \citealt{Dahlgren1998WOA}; \citealt{ElZarkaZiagos2020WOCA}) that exclude pragmatically-driven fronting or \isi{topicalization} to clause-initial position from \isi{word order} statistics may reach different conclusions than the current chapter. \tabref{Arabic:tab:1} lists the WOWA datasets in \textit{Qəltu}-Arabic\il{Arabic (Qəltu)}, as well as Central and Northeastern Neo-Aramaic, used in this chapter with their respective sources.\footnote{Numbered texts and numbered segments are separated by colons, e.g. 25:§2 means Text 25, Paragraph 2, and page numbers and segments by periods, e.g. 101.§2, Page 101, Paragraph 2.} These datasets were designed for the corpus-based analysis of non-subject arguments (see \citealt{Dahlgren1998WOA} for a corpus-based study of subjects in Arabic, and \citealt{Molin2021Dohok} for that in NENA) and their respective position before or after the predicate in accordance with the framework and coding guidelines of the WOWA databank.\footnote{See \url{https://multicast.aspra.uni-bamberg.de/resources/wowa/data/_docs/guidelines/wowa_coding-guidelines.pdf}} The sets for Barwar\il{Neo-Aramaic (NENA)!C. Barwar} and J. Dohok\il{Neo-Aramaic (NENA)!J. Dohok}, though dialects originally spoken in northwestern Iraq, are, for all practical purposes, considered also representative of the majority of NENA dialects in eastern Anatolia (see \citetv{chapters/15_Noorlander_NAINEI} for a discussion of NENA in Iran and northeastern Iraq). A handful of \isi{object} tokens were also counted for the NENA dialect of Bohtan \citep[116--137]{Fox2009NABohtan} and Hertevin \citep{Jastrow1988NAHertevin}, which are not part of the WOWA corpus. Moreover, Arabic data outside of the WOWA corpus were taken from \citet{Noorlander2024OVArabi} for Kinderib and Daragözü \textit{Qəltu}-Arabic\il{Arabic (Qəltu)} and Cilician Levantine Arabic\il{Arabic!Cilician}.
\end{sloppypar}

\begin{table}
    \begin{tabularx}{\textwidth}{p{2.2cm}rR{1.5cm}R{1.8cm}Q}
\lsptoprule
\textbf{Doculect} & \textbf{Speakers} & \textbf{Total tokens} & \textbf{Analysed tokens} & \textbf{Source} \\
\midrule
Ṭuroyo, Midyat, CNA & 2 & 1014 & 778 & \citet{Noorlander2022Turoyo} based on the digitalization by with \citet{LyavdanskyEtAl2020Turoyo} of \citegen{Ritter1967Turoyo} texts 1--2, 24, 27 \\
\tablevspace
Mlaḥso,\newline CNA & 2 & 824 & 703 & \citet{Noorlander2022Mlahso} based on \citet[74--129]{Jastrow1994Mlahso} \\
\tablevspace
J. Dohok,\newline NENA & 4 & 916 & 517 & \citet{Molin2022NEDohok} \\
\tablevspace
Barwar, NENA & 2 & 963 & 963 & \citet{Stilo2022WOWACBarwar} based on \citet{Khan2008JUrmi} \\
\tablevspace
Kaʿbiye, \textit{Qəltu}-Arabic & 3 & 788 & 643 & \citet{Noorlander2022Arabic} based on \citet[Texts II, IX, XII, XIV]{Jastrow2022CADiyarbakir} \\
\tablevspace
J. Baghdad,\newline  \textit{Qəltu}-Arabic & 2 & 1339 & 490 & \citet{BarMosheCraevschi2022Arabic} \\
\lspbottomrule
    \end{tabularx}
    \caption{Datasets from the WOWA corpus discussed in this chapter}
    \label{Arabic:tab:1}
\end{table}

\subsection{Arabic}

The Arabic dialects of modern-day Turkey belong to four major groups \citep{Jastrow2006ATurkey}:

\begin{enumerate}
    \item \textit{Anatolian Arabic}, i.e. sedentary Mesopotamian, divided into at least four subgroups
    \begin{enumerate}
        \item Mardin (Muslim, Christian, Jewish);
        \item Siirt (Muslim, Christian);
        \item Diyarbakir (Christian, Jewish);
        \item Kozluk-Sason-Muş (Muslim).
    \end{enumerate}
    \item \textit{Khawetna Arabic} (Khatuniya), i.e. Bedouin Mesopotamian, stretching from northern Syria into Turkish Mardin \citep{Talay1999AKhawetnaG}
    \item \textit{Cilician-Antiochian Arabic} (\citealt{Arnold1998AAntioch}; \citealt{Prochazka2002ACukurova}), a type of Levantine Arabic spoken by various ethnoreligious groups, of which Alevis constitute the majority, on the coastal region along the Mediterranean (Turkish\il{Turkic!Turkish Mersin}\il{Turkic!Turkish Hatay}\il{Turkic!Turkish Adana} Mersin, Adana and Hatay);
    \item \textit{Shawi Arabic}, originally Bedouin, in modern-day Urfa \citep{Prochazka2003Bed};
\end{enumerate}

\begin{sloppypar}
The Anatolian dialects belong to the so-called \textit{qəltu}-subgroup of \textit{Mesopotamian Arabic} that preserves the voiceless uvular stop /q/ and first singular suffix \textit{-tu}, as in \textit{qəltu} `I said,' against the later arrivals in Mesopotamia of ultimately Bedouin origin which innovated a corresponding velar stop /g/ and lost the final \textit{-u}, thus \textit{gələt} `I said' (\citealt{Blanc1964CDBaghdad}; \citealt{Jastrow1978MAqetlu1}, \citeyear{Jastrow2006ATurkey}; \citealt{Talay2012AMesopotamia}). Today, only a few Muslim speakers of \textit{Qəltu}-Arabic\il{Arabic (Qəltu)!Siirt}\il{Arabic (Qəltu)!Mardin}\il{Arabic (Qəltu)} remain in Siirt, Mardin and the mountains between Kozluk-Muş. Outside of Turkey, \textit{Qəltu}-Arabic\il{Arabic (Qəltu)} is also represented by: 
\end{sloppypar}

\begin{itemize}
    \item dialects spoken in Syria \citep{Behnstedt1992};
    \item Jewish dialects of Iraqi Kurdistan \citep{Jastrow1990JAAqraArbil}; 
    \item several varieties along the Tigris, including the various ethnoreligious communities of the Mosul Plain \citep{Jastrow1979AMosul} and the Jewish \citep{BarMoshe2019JAB} and Christian \citep{AbuHaidar1991CABaghdad} communities of Baghdad; 
    \item and finally Muslim and Jewish communities along the Euphrates, spanning from Hīt in Iraq to Khatuniya in Syria and Salāx in Turkey \citep{Talay1999AKhawetnaG}.
\end{itemize}

Anatolian \textit{Qəltu}-Arabic\il{Arabic (Qəltu)} dialects form a continuum with Levantine Arabic \citep{Talay2014Levantine}, and, as minorities outside of the core Arabic-speaking regions, they share similarities with other peripheral varieties of Arabic such as Cypriotic Arabic\il{Arabic!Cypriotic} and Central Asian Arabic\il{Arabic!Central Asian} (\citealt{Akkus2017PArabic}); though, the former is of Levantine origin and the latter presumably of Iraqi origin.

\begin{sloppypar}
The various other Bedouin dialects spoken in Iraq subsumed under Mesopotamian Arabic have more in common with the varieties of Arabia, which includes the Muslim \textit{gələt}-dialects of Baghdad in contradistinction to the Jewish and Christian \textit{qəltu}-dialects of the same city \citep{Blanc1964CDBaghdad} as well as the Arabic of Khuzestan\il{Arabic (Gələt)!Khuzestani} (\citetv{chapters/14_Leitner_Khuzistani}).
\end{sloppypar}

Overviews of Mesopotamian Arabic are offered by \citet{Jastrow1978MAqetlu1} and \citep{Talay2012AMesopotamia}, and that of Arabic varieties in Turkey by \citet{Jastrow2006ATurkey}, \citet{Arnold2015CTA} and \citet{prochazka2019}. Word order has been an understudied area, although \citet[131–141]{Jastrow1978MAqetlu1} and \citet[204--218]{Birnstiel2022CopulaKA} provide comparative studies of \isi{copula} syntax, for example, and \citet{Dahlgren1998WOA} offers large-scale, corpus-based studies of Arabic \isi{word order}, especially concerning the position of subjects. A comparative, corpus-based study of \isi{object} placement in several Anatolian Arabic dialects can be found in \citet{Noorlander2024OVArabi}.

\subsection{Aramaic}

Aramaic is represented in eastern Anatolia by 

\begin{itemize}
    \item the diverse group of \textit{Northeastern Neo-Aramaic} (NENA) spoken by Jews and Christians of Iranian Kurdistan, Iranian Azerbaijan, Iraqi Kurdistan, and southeastern Anatolia; 
    \item \textit{Central Neo-Aramaic} (CNA) consisting of the Neo-Aramaic dialects known as Ṭuroyo\il{Neo-Aramaic (CNA)!Ṭuroyo} spoken by the Christians of Ṭur ʿAbdin in modern-day Mardin and Şırnak, and the extinct dialect of Mlaḥso\il{Neo-Aramaic (CNA)!Mlaḥso} (Turkish\il{Turkic!Turkish Lice}: Lice) in Diyarbakir \citep{Jastrow1994Mlahso}.
\end{itemize}

This chapter focuses on the dialects in southeastern Anatolia, i.e. the dialects of Ṭuroyo\il{Neo-Aramaic (CNA)!Ṭuroyo} and Mlaḥso\il{Neo-Aramaic (CNA)!Mlaḥso}, as well as the NENA dialects in the western periphery. (The eastern periphery is treated in \citetv{chapters/15_Noorlander_NAINEI}).

The Neo-Aramaic dialects of rural Ṭur ʿAbdin and that of the city of Midyat exhibit slight variation, while Ṭuroyo\il{Neo-Aramaic (CNA)!Ṭuroyo} in general is distinct from Mlaḥso\il{Neo-Aramaic (CNA)!Mlaḥso}. \citet{Waltisberg2016STuroyo}, relying primarily on \citet{Ritter1967Turoyo}, provides a corpus-based overview of Ṭuroyo\il{Neo-Aramaic (CNA)!Ṭuroyo} syntax.

NENA used to be spoken in Christian villages in Şırnak and Siirt, south of the Bohtan river, notably Hertevin (Aramaic name: Artun; \citealt{Jastrow1988NAHertevin}), Borb-Ruma (i.e. Bohtan, \citealt{Fox2009NABohtan}) and Mount Judi \citep{Sinha2000NABespen}. The Jewish NENA dialects of Cizre \citep{Nakano1973Gzira} in Şırnak and of Challa \citep{Fassberg2010} in Hakkari belong to the so-called \textit{Lishana Deni} cluster whose core region is in northwestern Iraq. The Christian NENA dialects of the tribes in Hakkari used to form a densely populated area, e.g. Tkhuma and Upper/Lower Ṭyare, extending into Turkish Van \citep{Tsereteli1963}, Iranian Azerbaijan, and Iraqi Kurdistan, such as Lower Barwar\il{Neo-Aramaic (NENA)!C. Barwar} \citep{Khan2008JUrmi}. Several communities originating in Anatolia re-settled in Iraq or along the Khabur river northwest of Al-Hasaka in Syria \citep{Talay2008NAKhaburAssyrer,Talay2009NAKhaburAssyrer}.

\citet{Talay2008NAKhaburAssyrer} and \citet{Khan2019Anatolia} provide overviews of the NENA varieties of eastern Anatolia. Individual grammars, for instance, of Barwar\il{Neo-Aramaic (NENA)!C. Barwar} \citep[823--950]{Khan2008JUrmi}, of Tel Kepe (Mosul Plain; \citealt{Coghill2018ISNA}), of Jewish Zakho \citep{Cohen2012NENAZaxo} and of Jewish Dohok (\citealt{Molin2024JDuhok}) offer detailed studies of \isi{information structure} and \isi{word order}. \citet{NoorlanderMolin2022WordOrder} offer corpus-based \isi{word order} comparisons in Jewish NENA.

\section{Word order profile}

\subsection{Noun phrases}\label{Arabic:2.1}

Noun phrases display Numeral-Noun-Adjective order, e.g.

\ea\label{Arabic:ex:1}
\textit{Qəltu}-Arabic Kinderib \il{Arabic (Qəltu)!Kinderib}\citep[2.5:§19]{Jastrow2003AKinderib}\\
\gll fə-θəθ aṛbaʕ xams šəberi zġār \\
     in-three\textsc{.f} four\textsc{.f} five\textsc{.f} straw\_basket.\textsc{pl} little\textsc{.pl} \\
\glt `in three, four, five little straw baskets'
\z

\ea\label{Arabic:ex:2}
NENA Txuma \il{Neo-Aramaic (NENA)!C. Txuma}\citep[166.§16]{Talay2009NAKhaburAssyrer} \\
\gll ʔarpa xamšá plaš-e xelan-e \\
     four five battle\textsc{.m-pl} severe\textsc{-pl} \\
\glt `four, five severe battles' 
\z

\ea\label{Arabic:ex:3}
CNA Mlaḥso \il{Neo-Aramaic (CNA)!Mlaḥso}\citep[130.§137, §139]{Jastrow1994Mlahso}\\
\gll tre aḥé… ə=aḥó rab-ó  \\
     two brother\textsc{.m.pl} \textsc{def.sg=}brother\textsc{.msg} big\textsc{-msg} \\
\glt `the two brothers… the elder brother'
\z

Noun-Numeral occurs with the numeral `one' in \textit{Qəltu}-Arabic\il{Arabic (Qəltu)} and Ṭuroyo\il{Neo-Aramaic (CNA)!Ṭuroyo}, as exemplified in (\ref{Arabic:ex:4}--\ref{Arabic:ex:5}). Adjective-Noun order also sporadically occurs (see §3.1.7.).

\newpage
\ea\label{Arabic:ex:4}
\textit{Qəltu}-Arabic Kaʿbiye \il{Arabic (Qəltu)!Kaʿbiye}\citep[99]{Jastrow2022CADiyarbakir}\\
\gll kaṛm wēḥəd \\
     vineyard\textsc{.msg} one\textsc{.msg} \\
\glt `a vineyard' 
\z

\ea\label{Arabic:ex:5}
CNA Ṭuroyo, Midən \il{Neo-Aramaic (CNA)!Ṭuroyo Midən}\citep[275.§9]{Jastrow1985Laut} \\
\gll barθo ḥðo \\
     girl\textsc{.fsg} one\textsc{.fsg} \\
\glt `a girl' 
\z

\begin{sloppypar}
Demonstratives precede the noun everywhere except in CNA where the demonstrative is a suffix added to the determined noun and/or to the \isi{adjective} that immediately follows it \citep[46--47]{Waltisberg2016STuroyo}, as shown in (\ref{Arabic:ex:7c}), which originated in the patterns Noun-Demonstrative and Noun-Adjective-Demonstrative.  Adjectives agree in \isi{definiteness} with their head nominal in \textit{Qəltu}-Arabic and Ṭuroyo, cf. (\ref{Arabic:ex:6}) and (\ref{Arabic:ex:7}). Attributive demonstratives in the majority of NENA correspond to prefixal definite articles in CNA, e.g. \textit{u=} < \textit{*hū} in (\ref{Arabic:ex:7a}) and \textit{a=} < \textit{*han} in (\ref{Arabic:ex:7b}). The near deixis proclitic attributive demonstratives can also be augmented with a deictic suffix \textit{-ha} in the NENA dialect of Hertevin \citep[33–34]{Jastrow1988NAHertevin}, e.g. \textit{ʾád=ʾoda-ha} `this room over here,' which parallels the situation in Kurdish\il{Kurdish} (see \citetv{chapters/9_Mohammadirad_Gorani}).
\end{sloppypar}

\ea\label{Arabic:ex:6}
\textit{Qəltu}-Arabic Kaʿbiye \il{Arabic (Qəltu)!Kaʿbiye}\citep[XXI:§7]{Jastrow2022CADiyarbakir}\\
\gll \textbf{ād} əl-ḥāfez əl-məskin-∅ \\
     \textsc{dem.sg} \textsc{def-}blind\textsc{.msg} \textsc{def-}poor\textsc{-msg} \\
\glt `\textbf{this} poor blind beggar'
\z

\ea\label{Arabic:ex:7}
\ea\label{Arabic:ex:7a}
CNA Ṭuroyo, Kfarze \il{Neo-Aramaic (CNA)!Ṭuroyo Kfarze}\citep[67:§92]{Ritter1967Turoyo}\\
\gll ú=səsy-ayði ú=kom-o \\
     \textsc{def.msg=}horse\textsc{.msg}-my \textsc{def.msg}=black\textsc{-msg} \\
\glt `my black horse' 
\ex\label{Arabic:ex:7b}
CNA Ṭuroyo, Anḥəl \il{Neo-Aramaic (CNA)!Ṭuroyo Anḥəl}\citep[58:§119]{Ritter1967Turoyo}\\
\gll ám=medon\textbf{-ani} áḥ=ḥren-e \\
     \textsc{def.pl=}thing\textsc{.mpl}-\textsc{dem.pl} \textsc{def.pl=}other\textsc{-pl} \\
\glt `\textbf{those} other things' 
\ex\label{Arabic:ex:7c}
CNA Ṭuroyo, Midən \il{Neo-Aramaic (CNA)!Ṭuroyo Midən}\citep[266.§9]{Jastrow1985Laut}\\
\gll á=\textup{[}tre kŭrfe kom\textup{]}\textbf{-anək} \\
     \textsc{def.pl=}two\textsc{.m} snake\textsc{.mpl} black\textsc{-dem.pl} \\
\glt `\textbf{those} two black snakes' 
\z
\z

\ea\label{Arabic:ex:8}
NENA Hertevin \il{Neo-Aramaic (NENA)!C. Hertevin}\citep{Jastrow1988NAHertevin}\\
\gll \textbf{ʔád}=naša ṭaw-a \\
     \textsc{dem.sg=}man\textsc{.msg} good\textsc{-msg} \\
\glt `\textbf{this} good man' 
\z

\ea\label{Arabic:ex:9}
NENA Bohtan \il{Neo-Aramaic (NENA)!C. Bohtan}\citep[122.§87]{Fox2009NABohtan}\\
\gll \textbf{at} abra xen-a \\
     \textsc{dem.sg} man\textsc{.msg} other\textsc{-msg} \\
\glt `\textbf{this} other boy'
\z


While the morphology of attributive constructions varies considerably, Noun-Genitive order predominates in both Arabic and Aramaic for both genitive nouns and pronouns, e.g. 

\ea\label{Arabic:ex:10}
\textit{Qəltu}-Arabic Āzəx \il{Arabic (Qəltu)!Āzəx}\citep[VI6:§40]{Jastrow1981MAqetlu2} \\
\gll bayt əl-ḥakəm \\
     house\textsc{.msg} \textsc{def-}judge\textsc{.msg} \\
\glt `the house of the judge'
\z

\ea\label{Arabic:ex:11}
\textit{Qəltu}-Arabic Kaʿbiye \il{Arabic (Qəltu)!Kaʿbiye}\citep[IV:§16]{Jastrow2022CADiyarbakir}\\
\gll bayt abu-y \\
     house\textsc{.msg} father\textsc{.msg-}my \\
\glt `the house of my father' 
\z

\ea\label{Arabic:ex:12}
CNA Ṭuroyo, Midyat \il{Neo-Aramaic (CNA)!Ṭuroyo Midyat}\citep[24:§55]{Ritter1967Turoyo} \\
\gll ú=bayto d-ú=taǧər u í=zangan-ayðe kul-a \\
     \textsc{def.msg=}house\textsc{.msg} \textsc{gen-def.sg=}merchant\textsc{.msg} and \textsc{def.fsg=}wealth\textsc{.fsg-}his all\textsc{-fsg} \\
\glt `the house of the merchant and all his wealth' 
\z

\ea\label{Arabic:ex:13}
CNA Mlaḥso \il{Neo-Aramaic (CNA)!Mlaḥso}\citep[126.§126]{Jastrow1994Mlahso}\\
\gll beytó d-ə=malkó  \\
     house\textsc{.msg} \textsc{gen-def.sg=}king\textsc{.msg} \\
\glt `the house of the king'
\z

\ea\label{Arabic:ex:14}
NENA Hertevin \il{Neo-Aramaic (NENA)!C. Hertevin}\citep[156.§583]{Jastrow1988NAHertevin} \\
\gll l-ʔarʔ-əd bab-ew Yaqo \\
     to-land\textsc{-cstr} father\textsc{.msg-}his Jakob \\
\glt `to the land of his father Jakob'
\z

\subsection{Verbal complements}
\subsubsection{Verb/object}

Verb-Object order predominates in all relevant languages,\footnote{See \citet{Dahlgren1998WOA} on Arabic, \citet[289--290]{Waltisberg2016STuroyo} on Ṭuroyo\il{Neo-Aramaic (CNA)!Ṭuroyo}, and \citet{NoorlanderMolin2022WordOrder} on NENA.} with the exception of Mlaḥso\il{Neo-Aramaic (CNA)!Mlaḥso} and the NENA dialect of Bohtan (Borb-Ruma), where Object-Verb predominates.  Drawing on distinctions made in the WOWA corpus, however, \isi{argument} type and \isi{definiteness} are major factors in several \isi{VO} doculects, as borne out by the statistics in \figref{Arabic:fig:2} categorized according to referentiality and identifiability, i.e. definite as opposed to indefinite NPs, and \isi{argument} type, i.e. pronouns as opposed to nouns.  ``Pronoun,'' here, comprises independent personal and demonstrative pronoun\is{pronoun!demonstrative}s, as illustrated in (\ref{Arabic:ex:15c}) for Arabic and (\ref{Arabic:ex:16a}) for Aramaic, while all remaining independent pronouns, such as reflexive and indefinite pronoun\is{pronoun!indefinite}s, are subsumed under ``Other,'' exemplified in (\ref{Arabic:ex:15d}) and (\ref{Arabic:ex:16d}). In general, \figref{Arabic:fig:2} shows there is an increasing likelihood for post-verbal position across all doculects if the \isi{object} is indefinite. Even in the case of Mlaḥso\il{Neo-Aramaic (CNA)!Mlaḥso} where \isi{OV} order is largely grammaticalized, indefinite objects are slightly more likely post-verbal than their definite counterpart (see §\ref{Arabic:3.1.1} on variation among speakers and the \isi{role} of contact in Mlaḥso\il{Neo-Aramaic (CNA)!Mlaḥso}). This, however, does not mean that \isi{definiteness} plays a significant \isi{role} in every dialect. An additional, preliminary study of 96 \isi{direct object} NPs and their placement in NENA Hertevin\il{Neo-Aramaic (NENA)!C. Hertevin} (\cite{Jastrow1988NAHertevin}: §§86--100, §§166--172, §§307--323, §§419--466) reveals no significant difference between definite and indefinite \isi{object} NPs: 25 out of 35 (0.71) indefinites are post-verbal and 47 out of 61 (0.77) definites are post-verbal.

\begin{figure}
% % %     \includegraphics{figures/Figure2Arabic.png}\\
    \pgfplotsset{
	/pgfplots/bar cycle list/.style={/pgfplots/cycle list={
	{lsYellow,fill=lsYellow!50!white,mark=none},
	{lsSoftGreen,fill=lsSoftGreen!50!white,mark=none},
	{lsMidGreen,fill=lsMidGreen!50!white,mark=none},
	{lsMidDarkBlue,fill=lsMidDarkBlue!90!white,mark=none}
		},
	},
}
    \footnotesize
    \pgfplotstableread{data/ch16-fig2.csv}\ChapterSixTFigureTwoData
    \begin{tikzpicture}
	\begin{axis}
		[
            axis lines*=left,
			bar width=2ex,
			font=\footnotesize,
			height=6cm,
			enlarge y limits={value=0.2,upper},
			legend cell align=left,
			legend pos=north east,
			nodes near coords,
			nodes near coords style={/pgf/number format/fixed,font=\footnotesize},
			width=\textwidth,
			xtick=data,
			xticklabels from table={\ChapterSixTFigureTwoData}{Data},
			x tick label style={align=center, text width=1.25cm},
			y tick label style={font=\footnotesize},
            ybar=2pt,
			ymin=0,
			ymax=100,
			ylabel=\%,
			ylabel near ticks
		]
		\foreach \i in {{Indefinite NP},{Definite NP},{Other NP},{Pronominal NP}}
		  {
		  	\addplot table [x expr=\coordindex, x=Data, y=\i] {\ChapterSixTFigureTwoData};
		    \edef\temp{\noexpand\addlegendentry{\i}}
		    \temp
 		  }
	\end{axis}
    \end{tikzpicture}
    \caption{Rate of post-predicate (PP) nominal and pronominal objects}
    \label{Arabic:fig:2}
\end{figure}

\ea
\ea\label{Arabic:ex:15a}
Definite, \isi{OV}\\
\textit{Qəltu}-Arabic Kaʿbiye \il{Arabic (Qəltu)!Kaʿbiye}\citep[II:§19]{Jastrow2022CADiyarbakir} \\
\gll mō na-ʕṛef \textbf{madṛa} əšṭōr ya-ḥṣəd-ū-\textbf{nu} \\
     \textsc{neg} \textsc{A.1pl-}know millet\textsc{.msg} how \textsc{A.3-}harvest\textsc{-A.pl-O.3msg} \\
\glt `We do not know how to harvest \textbf{the millet}.'

\ex\label{Arabic:ex:15b}
Indefinite, \isi{VO}\\
\textit{Qəltu}-Arabic Kaʿbiye \il{Arabic (Qəltu)!Kaʿbiye}\citep[II:§5]{Jastrow2022CADiyarbakir} \\
\gll tə-t-ṛūḥ-o tá-ḥṣəd-o \textbf{madṛa} \\
     \textsc{fut-S.2-}go\textsc{-S.pl} \textsc{A.2-}harvest\textsc{-A.pl} millet\textsc{.msg} \\
\glt `You shall go to harvest \textbf{millet}.'
\newpage
\ex\label{Arabic:ex:15c}
Pronominal, \isi{OV}\\
\textit{Qəltu}-Arabic Kaʿbiye \il{Arabic (Qəltu)!Kaʿbiye}\citep[XIV:§2]{Jastrow2022CADiyarbakir} \\
\gll \textbf{ənti} ana t-āxəd-ki \\
     \textsc{2fsg} \textsc{1sg} \textsc{fut-A.1sg.}take\textsc{-O.2fsg} \\
\glt `I will take \textbf{you}.'
\ex\label{Arabic:ex:15d}
Other, \isi{OV}\\
\textit{Qəltu}-Arabic Kaʿbiye \il{Arabic (Qəltu)!Kaʿbiye}\citep[II:§7]{Jastrow2022CADiyarbakir} \\
\gll \textbf{kəll-en} qatl-ōn \\
     all-of.them killed\textsc{-A.3pl} \\
\glt `They killed \textbf{them all}.'
\z
\z

\ea
\ea\label{Arabic:ex:16a}
Definite, \isi{OV}\\
CNA Ṭuroyo, Midyat \il{Neo-Aramaic (CNA)!Ṭuroyo Midyat}\citep[27:§25]{Ritter1967Turoyo} \\
\gll \textbf{í=hadiy-ayo} gə-mšayáʕ-no-le=yo \\
     \textsc{def.fsg=}gift\textsc{.fsg-dem.fsg} \textsc{fut-}send\textsc{-A.1msg}\textsc{-R.3msg}\textsc{=T.3msg} \\
\glt `I will send him \textbf{the present}.'
\ex\label{Arabic:ex:16b}
Indefinite, \isi{VO}\\
CNA Ṭuroyo, Midyat \il{Neo-Aramaic (CNA)!Ṭuroyo Midyat}\citep[27:§25]{Ritter1967Turoyo} \\
\gll ṭlə́b-le-le \textbf{hadiye} men-i \\
     asked\textsc{.pfv-A.3msg}\textsc{-R.3msg} gift\textsc{.fsg} from\textsc{-1sg}  \\
\glt `He asked \textbf{a present} for himself from me.'

\newpage
\ex\label{Arabic:ex:16c}
Pronominal, \isi{OV}\\
CNA Ṭuroyo, Midyat \il{Neo-Aramaic (CNA)!Ṭuroyo Midyat}\citep[83:§50]{Ritter1967Turoyo} \\
\gll \textbf{haθe} ono ló=ḥəzy-o-li \\
     \textsc{dem.f.sg} \textsc{1sg} \textsc{neg=}saw\textsc{.pfv}\textsc{-O.3f.sg}\textsc{-A.1sg} \\
\glt `I did not see \textbf{this one}.'
\ex\label{Arabic:ex:16d}
Other, \isi{VO}\\
CNA Ṭuroyo, Midyat \il{Neo-Aramaic (CNA)!Ṭuroyo Midyat}\citep[27:§54]{Ritter1967Turoyo} \\
\gll mayta-lle \textbf{kul-le} \\
     brought\textsc{.pfv-A.3pl} all-of.them \\
\glt `He brought \textbf{them all}.'
\z
\z

The tokens of object pronouns\is{pronoun!object} are low, since they are, by default, expressed with \isi{object} suffixes on the verb; their independent expression thus attracts attention in the discourse. This notwithstanding, the data indicate their \isi{word order} preference should not be characterised in the same manner as nouns, suggesting \isi{OV} order instead in the NENA doculects and Kaʿbiye \textit{Qəltu}-Arabic\il{Arabic (Qəltu)} doculect. 

A fronted definite \isi{object} generally triggers \isi{object} indexing, i.e. clitic doubling\is{clitic!doubling} or cross-referencing verbal \isi{object} suffixes, as illustrated in (\ref{Arabic:ex:15a}) for Arabic and (\ref{Arabic:ex:16a}) for Aramaic. Topicalization with pronominal resumption is well-known in Arabic linguistics and also occurs in Anatolian Arabic, e.g. see \citet[164--165]{Wittrich2001AAzex} for examples of \isi{topicalization} in the \textit{Qəltu}-Arabic\il{Arabic (Qəltu)!Āzəx} dialect of Āzəx. Moreover, pre-verbal placement is conditioned by \isi{specificity} in at least the Kozluk-Sason-Muş dialects \citep{Akkus2017PArabic}. The \isi{object} indexing originally used to be dedicated to \isi{topicalization} and subsequently developed into a differential \isi{object} coding strategy. Other studies of Arabic \isi{word order} (\citealt{Dahlgren1998WOA}; \citealt{ElZarkaZiagos2020WOCA}), however, have treated such definite objects -- often in clause-initial position -- with additional \isi{object} indexing on the verb as clause-external arguments, and not instances of \isi{OV}.  How frequently this pre-verbal placement occurs, be it due to \isi{topicalization} or \isi{specificity}, has not been studied in detail. Preposed objects of any kind -- be they topicalized to clause-initial position or simply fronted before the verb -- were subsumed under \isi{OV} in the current approach, and the importance of treating them as such is borne out by the statistics in \figref{Arabic:fig:2}. The additional factor of \isi{object} indexing requires further investigation, but see \citet{NoorlanderMolin2022WordOrder} on Jewish NENA and \citet{Noorlander2024OVArabi} on Anatolian Arabic.

\subsubsection{Verb/goal}

Lexicalized arguments -- in \isi{contrast} to bound verbal person markers -- denoting \isi{endpoint} roles (see \citetv{chapters/1_Haigetal_Intro}) like the \isi{Goal} of verbs of (caused) motion, such as `go' and `bring,' are generally post-verbal throughout, as given in the first column of \tabref{Arabic:tab:2}. The second column of \tabref{Arabic:tab:2} combines the tokens of \textit{Recipients} of ditransitive verbs like `give' and \textit{Beneficiaries}, and the last column shows the tokens of \textit{Addressees} of verbs of speech like `say'. These respective roles are illustrated for Kaʿbiye Arabic and Mlaḥso\il{Neo-Aramaic (CNA)!Mlaḥso} Aramaic in (\ref{Arabic:ex:17}--\ref{Arabic:ex:18}). Thus, endpoints are generally treated the same across doculects, except for Recipient/Beneficiaries and especially Addressees in Mlaḥsó, which are pre-verbal more often than in the other doculects (this is an areal phenomenon; see §\ref{Arabic:3.1.4}).

\begin{table}
    \begin{tabularx}{.8\textwidth}{l rr Yr Yr}
\lsptoprule
\textbf{Doculect} & \multicolumn{2}{c}{\textbf{G}} & \multicolumn{2}{c}{\textbf{R/Ben}} & \multicolumn{2}{c}{\textbf{Addr}} \\
 \cmidrule(lr){2-3}\cmidrule(lr){4-5}\cmidrule(lr){6-7}
 & \textbf{\textit{n}} & \textbf{PP} & \textbf{\textit{n}} & \textbf{PP} & \textbf{\textit{n}} & \textbf{PP} \\
\midrule
Midyat, Ṭuroyo, CNA & 140 & 97\% & 24 & 96\% & 25 & 100\% \\
J. Dohok, NENA & 112 & 99\% & 34 & 95\% & 31 & 100\% \\
Kinderib, \textit{Qəltu}-Arabic & 76 & 100\% & 21 & 95\% & (5 & 60\%) \\
Kaʿbiye, \textit{Qəltu}-Arabic & 87 & 92\% & 29 & 79\% & 12 & 75\% \\
Mlaḥso, CNA & 152 & 97\% & 42 & 69\% & 34 & 50\% \\
\lspbottomrule
    \end{tabularx}
    \caption{Rate of post-predicate (PP) goal-like arguments}
    \label{Arabic:tab:2}
\end{table}



\ea\label{Arabic:ex:17}
\ea\label{Arabic:ex:17a}
Verb-Goal\\
\textit{Qəltu}-Arabic Kaʿbiye \il{Arabic (Qəltu)!Kaʿbiye}\citep[XIV:§4]{Jastrow2022CADiyarbakir} \\
\gll tə-ḥreb tə-ǧi \textbf{bayt} \\
     \textsc{S.2fsg}-flee \textsc{S.2fsg}-come house\textsc{.msg} \\
\glt `She flees \textbf{home}.'
\ex\label{Arabic:ex:17b}
Verb-Recipient\\
\textit{Qəltu}-Arabic Kaʿbiye \il{Arabic (Qəltu)!Kaʿbiye}\citep[XIII:§1]{Jastrow2022CADiyarbakir} \\
\gll ana t-a-ḥṭi bənt-i \textbf{šān} \textbf{əben} \textbf{axū-y} \\
     I \textsc{fut-A.1sg-}give daughter\textsc{.fsg-}my to son.\textsc{msg.cstr} brother-my  \\
\glt `I shall give my daughter \textbf{to the son of my brother}.' 
\newpage
\ex\label{Arabic:ex:17c}
Verb-Addressee\\
\textit{Qəltu}-Arabic Kaʿbiye \il{Arabic (Qəltu)!Kaʿbiye}\citep[XIV:§24]{Jastrow2022CADiyarbakir} \\
\gll əl-xōǧa y-qūl-∅ \textbf{šā-ll-ūlād} \\
     \textsc{def-}Khoja \textsc{A.3m-}say-\textsc{A.sg} to\textsc{-def-}boy\textsc{.pl}  \\
\glt `The Khoja says \textbf{to the lads}.'
\z
\z

\ea\label{Arabic:ex:18}
\ea\label{Arabic:ex:18a}
Verb-Goal\\
CNA Mlaḥso \il{Neo-Aramaic (CNA)!Mlaḥso}\citep[108.§25]{Jastrow1994Mlahso} \\
\gll ase-lan \textbf{ʕayni} \textbf{beytó} \\
     went\textsc{.pfv-S.1pl} same house\textsc{.msg} \\
\glt `We went \textbf{to the same house}.'
\ex\label{Arabic:ex:18b}
Verb-Recipient\\
CNA Mlaḥso \il{Neo-Aramaic (CNA)!Mlaḥso}\citep[163.§117]{Jastrow1994Mlahso} \\
\gll ə=brat-ezav hiv-le \textbf{lə=nošk-ano} \\
     \textsc{def.sg=}daughter\textsc{.fsg-}his gave\textsc{.pfv-A.3msg} \textsc{def.sg=}person\textsc{-dem.msg} \\
\glt `He gave \textbf{that person} his daughter.'
\ex\label{Arabic:ex:18c}
Addressee-Verb\\
CNA Mlaḥso \il{Neo-Aramaic (CNA)!Mlaḥso}\citep[90.§115]{Jastrow1994Mlahso} \\
\gll malkó \textbf{el-áv} emir-le \\
     king\textsc{.msg} to\textsc{-3msg} said\textsc{.pfv-A.3msg} \\
\glt `The king said \textbf{to him}...'
\z
\z

Relative \isi{object} ordering in three-\isi{argument} clauses depends on construction type. In the \textit{prepositional \isi{dative} construction}, the prepositional Recipient comes second. In a \textit{double \isi{object} construction}, both arguments are treated like O, and the Recipient comes first. This alternation is shown in (\ref{Arabic:ex:19}) for NENA.\footnote{For more examples, see  \citep{Coghill2014} and \citet[144--153]{Noorlander2018Alignment} for NENA, \citet{Waltisberg2016STuroyo} for Ṭuroyo\il{Neo-Aramaic (CNA)!Ṭuroyo}, \citet{CamilleriElSadekSadler2014ADat} and \citet{Birnstiel2022CopulaKA} for Arabic.}

\ea\label{Arabic:ex:19}
\ea\label{Arabic:ex:19a}
Theme(O)-Recipient(Prepositional)\\
NENA Hertevin \il{Neo-Aramaic (NENA)!C. Hertevin}\citep[§437]{Jastrow1988NAHertevin} \\
\gll daḥ b-yăw-aḥ-la ḥakimut-an \textbf{l-ohá} \\
     how \textsc{fut}-give-\textsc{A.1pl-O.3fsg} government\textsc{.fsg-}our to-\textsc{dem.msg} \\
\glt `How are we to grant lordship over us \textbf{to that one}?'
\ex\label{Arabic:ex:19b}
Recipient(O)-Theme(O)\\
NENA Hertevin \il{Neo-Aramaic (NENA)!C. Hertevin}\citep[§437]{Jastrow1988NAHertevin} \\
\gll yăw-á-lehən \textbf{ʔan} \textbf{čičoke} beʔe \\
     give\textsc{-A.3fsg-O.3pl} \textsc{dem.pl} chick\textsc{.pl} egg\textsc{.pl} \\
\glt `She would give \textbf{these chicks} eggs.'
\z
\z

The default order of the prepositional \isi{dative} construction in both Ṭuroyo\il{Neo-Aramaic (CNA)!Ṭuroyo} \citep[298]{Waltisberg2016STuroyo} and NENA (e.g. \citealt[251]{NoorlanderMolin2022WordOrder}) as well as Anatolian \textit{Qəltu}-Arabic\il{Arabic (Qəltu)} is \textit{Verb-Theme-Recipient}:\footnote{\citet[218--230]{Birnstiel2022CopulaKA} points to the default order of Verb-Recipient-Theme, but this does not seem to hold for all of \textit{Qəltu}-Arabic\il{Arabic (Qəltu)}.}

\ea\label{Arabic:ex:20}
CNA Ṭuroyo, Midyat \il{Neo-Aramaic (CNA)!Ṭuroyo Midyat}\citep[27:§30]{Ritter1967Turoyo} \\
\gll húle-le=ste kallat ġalabe \textbf{l-ú-ḥaloq-ano} \\
     gave\textsc{.pfv-A.3msg=add} money\textsc{.pl} much to\textsc{-def.msg}-barber\textsc{-dem.msg} \\
\glt `He gave \textbf{the barber} much money.'
\z

\ea
\ea\label{Arabic:ex:21a}
\textit{Qəltu}-Arabic Kinderib \il{Arabic (Qəltu)!Kinderib}\citep[6.1:§24]{Jastrow2003AKinderib} \\
\gll tə-ʕṭi-∅ rġīf \textbf{lə-ṣāḥdət} \textbf{ət-tannōṛ} \\
     \textsc{3f-}give\textsc{-sg} flatbread\textsc{.msg} to-owner\textsc{.fsg.cstr} \textsc{def-}oven \\
\glt `She gives \textbf{the owner of the oven} a flatbread.' 
\ex\label{Arabic:ex:21b}
\textit{Qəltu}-Arabic Hasankeyf \il{Arabic (Qəltu)!Hasankeyf}\citep[6.4.2:§5]{Fink2020AHasankeyf} \\
\gll īlzam yə-ʕṭi-∅ paṛāṭ \textbf{l-əmm} \textbf{w-abū} \\
     must \textsc{A.3m-}give\textsc{-A.sg} money\textsc{.pl} to-mother\textsc{.fsg} and-father\textsc{.msg} \\
\glt `He must give \textbf{his mother and father} money.' 
\ex\label{Arabic:ex:21c}
\textit{Qəltu}-Arabic Hasköy \il{Arabic (Qəltu)!Hasköy}\citep[I.2.2.:§9]{Talay2002AHaskoyT} \\
\gll qən-na n-āxəz pēlāv \textbf{šā} \textbf{ixṭ-aṭ-na} \\
     wanted\textsc{.pfv-A.1pl} \textsc{A.1pl-}buy shoe\textsc{.sg} for sister\textsc{-fsg.cstr-}our \\
\glt `We wanted to buy shoes \textbf{for our sister}.'
\z
\z

\subsubsection{Become/complement}\label{Arabic:2.2.3}

Final sates of change-of-state\is{inchoative} verbs, such as `become,' `make,' 'turn into (tr./intr.),' tend to follow the predicate:

\ea
\ea\label{Arabic:ex:22a}
\textit{Qəltu}-Arabic Kaʿbiye \il{Arabic (Qəltu)!Kaʿbiye}\citep[XIII:§2]{Jastrow2022CADiyarbakir} \\
\gll ṣār-∅ kačal \\
     became\textsc{.pfv-S.3msg} bald \\
\glt `He turned bald.'
\ex\label{Arabic:ex:22b}
\textit{Qəltu}-Arabic Kaʿbiye \il{Arabic (Qəltu)!Kaʿbiye}\citep[IX:§11]{Jastrow2022CADiyarbakir} \\
\gll n-say-en kəde \textbf{pūšiy-āt} \\
     \textsc{A.1pl-}make\textsc{-O.3pl} such headscarf\textsc{-pl} \\
\glt `We make them \textbf{into headscarves}.' 
\z
\z

\ea
\ea\label{Arabic:ex:23a}
CNA Mlaḥso \il{Neo-Aramaic (CNA)!Mlaḥso}\citep[112.§50]{Jastrow1994Mlahso} \\
\gll kul-én ve-len nayar-ezan \\
     all-of.them became\textsc{.pfv-S.3pl} enemy\textsc{.pl}-our \\
\glt `They all became our enemies.'
\ex\label{Arabic:ex:23b}
CNA Mlaḥso \il{Neo-Aramaic (CNA)!Mlaḥso}\citep[ 116.§75]{Jastrow1994Mlahso} \\
\gll băṭrăk Elyás sim-le el-áv \textbf{šammás} \\
     patriarch\textsc{.msg} Elyas made\textsc{.pfv-A.3msg} \textsc{dom-3msg} deacon\textsc{.msg} \\
\glt `Patriarch Elyas made him \textbf{deacon}.' 
\z
\z

In dialects with a higher rate of \isi{OV}, such as the NENA dialect of Bohtan, the relative placement of the \isi{argument} of the verb `make,' distinguishes between the preverbal \isi{direct object} and the postverbal \isi{endpoint} denoting the final state, as reflected in the following example (see \citetv{chapters/15_Noorlander_NAINEI} for parallels in Iran):

\ea
\ea\label{Arabic:ex:24a}
Object-Verb\\
NENA Bohtan \il{Neo-Aramaic (NENA)!C. Bohtan}\citep[55]{Fox2009NABohtan} \\
\gll \textbf{xa} \textbf{kaboba} iwad-le \\
     one kebab\textsc{.msg} make\textsc{.impv-O.3msg} \\
\glt `Make \textbf{a kebab}!' 
\ex\label{Arabic:ex:24b}
Verb-Complement\\
NENA Bohtan \il{Neo-Aramaic (NENA)!C. Bohtan}\citep[56]{Fox2009NABohtan} \\
\gll wəd-lo-le \textbf{kaboba} \\
     made\textsc{.pfv-A.3fsg-O.3msg} kebab\textsc{.msg} \\
\glt `She made it \textbf{into a kebab}.' 
\z
\z

\subsubsection{Other obliques}\label{Arabic:2.2.4}
\largerpage
Post-verbal position is the most common for obliques. The first column of \tabref{Arabic:tab:3} gives the statistics of place and source constituents, illustrated in (\ref{Arabic:ex:25a}) for Arabic and (\ref{Arabic:ex:26a}) for Aramaic. Pre-verbal placement, by \isi{contrast}, is common in Kaʿbiye Arabic, e.g. (\ref{Arabic:ex:25b}), and predominates in Mlaḥso\il{Neo-Aramaic (CNA)!Mlaḥso}, e.g. (\ref{Arabic:ex:26b}). Moreover, while all obliques generally behave similarly across doculects, instruments, given in the second column of \tabref{Arabic:tab:3}, are more likely pre-verbal than other obliques in Ṭuroyo\il{Neo-Aramaic (CNA)!Ṭuroyo} Aramaic.

\begin{table}
    \begin{tabularx}{.8\textwidth}{lYYYY}
\lsptoprule
\textbf{Doculect} & \multicolumn{2}{c}{\textbf{Place/Source}} & \multicolumn{2}{c}{\textbf{Instrument}} \\
\midrule
& \textbf{\textit{n}} & \textbf{PP} & \textbf{\textit{n}} & \textbf{PP} \\
\midrule
Kinderib, \textit{Qəltu}-Arabic & 43 & 98\% & 13 & 92\% \\
Barwar, NENA & 177 & 74\% & 21 & 71\% \\
Midyat, Ṭuroyo, CNA & 121 & 88\% & 22 & 55\% \\
Kaʿbiye, \textit{Qəltu}-Arabic & 82 & 61\% & 19 & 47\% \\
Mlaḥso, CNA & 56 & 34\% & 13 & 15\% \\
\lspbottomrule
    \end{tabularx}
    \caption{Rate of post-predicate (PP) oblique arguments (pronominal and nominal)}
    \label{Arabic:tab:3}
\end{table}

\ea
\ea\label{Arabic:ex:25a}
Verb-Oblique\\
\textit{Qəltu}-Arabic Kinderib \il{Arabic (Qəltu)!Kinderib}\citep[7.6:§3]{Jastrow2003AKinderib} \\
\gll t-haššəs-a \textbf{fə-l-ṃayy} \\
     \textsc{A.3fsg-}make.swell\textsc{-O.3f} in-\textsc{def}-water \\
\glt `She swells them \textbf{in the water}.'
\ex\label{Arabic:ex:25b}
Oblique-Verb\\
\textit{Qəltu}-Arabic Kaʿbiye \il{Arabic (Qəltu)!Kaʿbiye}\citep[ II:§16]{Jastrow2022CADiyarbakir} \\
\gll \textbf{fə-l-ṃayye} faṭṭs-uw-a \\
     in\textsc{-def-}water drowned\textsc{.pfv-A.3pl-O.3fsg} \\
\glt `They drowned her \textbf{in the water}.'
\z
\z

\ea
\ea\label{Arabic:ex:26a}
Verb-Oblique\\
CNA Ṭuroyo \il{Neo-Aramaic (CNA)!Ṭuroyo}\citep[1:§1]{Ritter1967Turoyo} \\
\gll yətáw-no-wo \textbf{b-Băbak} \\
     be.settled\textsc{-S.1msg-pst} in-Bebek \\
\glt `I used to live \textbf{in Bebek}.' 
\ex\label{Arabic:ex:26b}
Oblique-Verb\\
CNA Mlaḥso \il{Neo-Aramaic (CNA)!Mlaḥso}\citep[80.§50]{Jastrow1994Mlahso} \\
\gll \textbf{b-ə́-hawše} \textbf{d-ə́-deyro} yativ-ina \\
     in\textsc{-def.sg-}court\textsc{.msg} \textsc{gen-def.sg}-church\textsc{.msg} sat\textsc{.ant-S.1pl}  \\
\glt `We sat down \textbf{in the churchyard}.' 
\z
\z

\subsection{Other predicates}

\subsubsection{Copulas}\label{Arabic:2.3.1}

Semitic languages generally lack a verbal \isi{copula} in present realis clauses. The juxtaposition of a (pro)noun and a nominal predicate are sufficient, and this occurs in all relevant major subgroups of Arabic and Aramaic in eastern Anatolia. The use of pronominal copulas, however, is a hallmark of \textit{Qəltu}-Arabic\il{Arabic (Qəltu)} as well as Central and Northeastern Neo-Aramaic (and a well-documented areal phenomenon, see §\ref{Arabic:3.2}). Their relative position not only varies across dialects, but also depends on clause type. Overall, pronominal copulas exist for all clause types, except past and irrealis clauses, where a copular 'be' verb is preferred, e.g. Arabic \textit{kwn}, Aramaic \textit{hwy}. The paradigmatic organisation of pronouns and copulas is closely intertwined in Central Neo-Aramaic and \textit{Qəltu-}Arabic, as compared for the peripheral dialects in \tabref{Arabic:tab:4}. It is not unlikely that the pronominal \isi{inflection} in Mlaḥso was based on an Arabic model. An overview with contrastive illustrations from a sample of the relevant languages is given in \tabref{Arabic:tab:5}.\footnote{On \isi{copula} syntax in \textit{Qəltu}-Arabic, see \citet[131--141]{Jastrow1978MAqetlu1} and \citet[204--218]{Birnstiel2022CopulaKA}, on that in NENA, see e.g. Khan (\citeyear{Khan2002Qaraqosh}: 322–330, \citeyear{Khan2008JUrmi} 823–842), \citet[30--65]{Cohen2012NENAZaxo} and \citet[140--174]{Molin2024JDuhok}, on that in Ṭuroyo\il{Neo-Aramaic (CNA)!Ṭuroyo}, see \citet[112–122, 208–210, 238–240]{Waltisberg2016STuroyo}.}

\begin{table}
    \begin{tabularx}{.8\textwidth}{XXXXl}
\lsptoprule
& \multicolumn{2}{l}{\textbf{Sason Arabic}} & \multicolumn{2}{l}{\textbf{Mlaḥso, CNA}} \\
& \multicolumn{2}{l}{\citep[14]{Akkus2017PArabic}} & \multicolumn{2}{l}{\citep{Jastrow1988NAHertevin}} \\
\midrule
\textsc{3msg} & \textit{iyu} & \textit{=ye} & \textit{hiye} & \textit{=yo} \\
\textsc{3fsg} & \textit{iya} & \textit{=ye} & \textit{hiya} & \textit{=yo} \\
\textsc{3pl} & \textit{iyen} & \textit{=nen} & \textit{hiyen} & \textit{=ne} \\
\lspbottomrule
    \end{tabularx}
    \caption{Comparison of personal pronoun\is{pronoun!personal}s and copula in Sason Arabic and Mlaḥso}
    \label{Arabic:tab:4}
\end{table}


 \begin{table}[b!]
%  \small
     \begin{tabularx}{\textwidth}{QQQQQ}
\lsptoprule
\multicolumn{4}{c}{\textbf{\textit{Qəltu}-Arabic}} & \\
\cmidrule{1-4}
\textbf{Siirt}\footnote{\citet[137]{Jastrow1978MAqetlu1}} & \textbf{Kinderib}\footnote{\citet[137]{Jastrow1978MAqetlu1}, \citet[73--76, 172--175]{Lahdo2009ATillo}} & \textbf{Kaʿbiye}\footnote{\citet[47]{Jastrow2022CADiyarbakir}} & \textbf{Hasankeyf}\footnote{\citet[76--77, 152--153]{Fink2020AHasankeyf}} & \\
\midrule
\textit{mayye fəlbayt} & \textit{mayye fəlbayt} & \textit{fəlbayt maye} & \textit{mo fəlbayt=e} & `She is not at home.' \\
\tablevspace
\textit{əlbənt iyy fəlbayt} & \textit{əlbənt fəlbayt=ye} & \textit{əlbənt fəlbayt=ye} & \textit{əlbənt fəlbayt=e} & `The girl is at home.' \\
\tablevspace
\textit{əlbənt ayysap iyye?} & \textit{əlbent aynī=ye?} & \textit{əlbənt əndaḥ=ḥe?} & \textit{əlbənt angəs=e?} & `Where is the girl?' \\

\tablevspace
\multicolumn{2}{c}{\textbf{NENA}} & \multicolumn{2}{c}{\textbf{CNA}} & \\
\cmidrule{1-2}\cmidrule{3-4}
\textbf{J. Challa}\footnote{\citet[100--101]{Fassberg2010}} & \textbf{Bohtan}\footnote{\citet{Fox2009NABohtan}} & \textbf{Ṭuroyo} & \textbf{Mlaḥso}\footnote{\citet[§35, §62, §142]{Jastrow1994Mlahso}} & \\
\midrule
\textit{lewa go besa} & \textit{lewa bata} & \textit{latyo bú=bayto} & \textit{b-beytó letyo} & `She is not at home.' \\
\tablevspace
\textit{brata (ʔila) go besa(=la)} & \textit{brata bata=la} & \textit{í=barθo bú=bayto=yo} & \textit{ə=brató b-beyto=yo} & `The girl is at home.' \\
\tablevspace
\textit{ʔaya ma=yle?} & \textit{awa m=ile?} & \textit{hawo mən=yo?} & \textit{awo mən=yo?} & `What is that?' \\
\tablevspace
\lspbottomrule
\multicolumn{5}{l}{These are hypothetical examples to illustrate the patterns.} \\
    \end{tabularx}
    \caption{Comparison of copula placement in \textit{Qəltu}-Arabic and Central Neo-Aramaic}
    \label{Arabic:tab:5}
\end{table}

Generally, the negative and relative \isi{copula} precede the predicate, as does the so-called deictic \isi{copula}–or ``demonstrative'' or ``presentative'' \isi{copula}–, which asserts an actual state of affairs in the immediately observable present, often with speaker or listener deixis. The latter is thus frequently used after verbs of perception, generally incompatible with questions, and ultimately derived from a deictic particle combined with a pronominal element. These three \isi{copula} types are illustrated in (\ref{Arabic:ex:27}) and (\ref{Arabic:ex:28}).

\ea\label{Arabic:ex:27}
\ea\label{Arabic:ex:27a}
Negative \isi{copula}\\
\textit{Qəltu}-Arabic Qartmin \il{Arabic (Qəltu)!Qartmin}\citep[137]{Jastrow1978MAqetlu1} \\
\gll \textbf{ma-nne} fə-l-bayt \\
\textsc{neg-cop.3msg} in-\textsc{def}-house\textsc{.msg} \\
\glt `\textbf{They are not} at home.'
\ex\label{Arabic:ex:27b}
Relative \isi{copula}\\
\textit{Qəltu}-Arabic Qartmin \il{Arabic (Qəltu)!Qartmin}\citep[138]{Jastrow1978MAqetlu1} \\
\gll \textbf{la-nne} fə-s-suri \\
\textsc{rel-cop.3pl} in-\textsc{def}-Syria \\
\glt `(Yazidis) \textbf{who are} in Syria'
\ex\label{Arabic:ex:27c}
Deictic \isi{copula}\\
\textit{Qəltu}-Arabic Qartmin \il{Arabic (Qəltu)!Qartmin}\citep[140]{Jastrow1978MAqetlu1} \\
\gll \textbf{kənā́} rəfqət-u ǧaw \\
\textsc{deic.cop.3pl} companion\textsc{.mpl-}his came\textsc{.pfv.3pl} \\
\glt `\textbf{Look! There they are}, his companions are coming.'
\z
\z

\ea\label{Arabic:ex:28}
\ea\label{Arabic:ex:28a}
Negative \isi{copula}\\
CNA Midən, Ṭuroyo \il{Neo-Aramaic (CNA)!Ṭuroyo Midən}\citep[115:§206]{Ritter1967Turoyo} \\
\gll ám=may-ani dəθxu \textbf{lan-ne} basim-e \\
\textsc{def.pl=}water\textsc{.pl-dem.pl} \textsc{gen.2pl} \textsc{neg-cop.3pl} nice\textsc{-pl} \\
\glt `That water of yours \textbf{is not} tasty.'

\ex\label{Arabic:ex:28b}
Relative \isi{copula}\\
CNA Midən, Ṭuroyo \il{Neo-Aramaic (CNA)!Ṭuroyo Midən}\citep[115:§24]{Ritter1967Turoyo} \\
\gll \textbf{d-kət-ne} yatiw-e \\
\textsc{rel-exist-cop.3pl} seated\textsc{-pl} \\
\glt `(those) \textbf{who are} seated' 
\ex\label{Arabic:ex:28c}
Deictic \isi{copula}\\
CNA Midən, Ṭuroyo \il{Neo-Aramaic (CNA)!Ṭuroyo Midən}\citep[116:§48]{Ritter1967Turoyo} \\
\gll \textbf{ka-lən=ne} tamo \\
\textsc{deic-3pl=cop.3pl} there \\
\glt `\textbf{Look, they are} there!'
\z
\z

The default position of the \isi{copula} in affirmative present realis clauses is after the nominal predicate in the majority of the Anatolian \textit{Qəltu}-Arabic\il{Arabic (Qəltu)} varieties as well as in that of Neo-Aramaic. The \isi{copula}, however, is not necessarily clause-final, as shown in (\ref{Arabic:ex:29b}), nor is it obligatory, as shown in (\ref{Arabic:ex:29c}). The predicate is expected to be downgraded to the background when following the \isi{copula} in a focalisation strategy like (\ref{Arabic:ex:29b}) where the subject \isi{pronoun} is in narrow \isi{focus} for the purpose of identification or specification. The \isi{copula} can be completely lacking in clauses like (\ref{Arabic:ex:29c}), a structure that is akin to the original Semitic non-verbal clause, reflecting \textit{Topic-Comment} order. The frequency of this structure is unknown, and this has not been coded in the WOWA corpus.

\ea
\ea\label{Arabic:ex:29a}
Default\\
CNA Midən, Ṭuroyo \il{Neo-Aramaic (CNA)!Ṭuroyo Midən}\citep[79:§12]{Ritter1967Turoyo} \\
\gll ono Ḥóre\textbf{=no}! \\
I Hore\textsc{=cop.1sg} \\
\glt `I \textbf{am} Hore!'
\ex\label{Arabic:ex:29b}
Focalisation\\
CNA Midən, Ṭuroyo \il{Neo-Aramaic (CNA)!Ṭuroyo Midən}\citep[82:§40]{Ritter1967Turoyo} \\
\gll óno\textbf{=no} Šēx Dhām \\
I\textsc{=cop.1sg} Sheikh Dham \\
\glt `\textbf{It is} I, Sheikh Dham.' 
\ex\label{Arabic:ex:29c}
Absent\\
CNA Midən, Ṭuroyo \il{Neo-Aramaic (CNA)!Ṭuroyo Midən}\citep[116:§1]{Ritter1967Turoyo} \\
\gll ono Slemā́n Ḥanna Maskobi \\
I Sleman Hanna Maskobi \\
\glt `I am Sleman Hanna Maskobi.'
\z
\z

\textit{Pre}-posed copulas, in turn, are a typical trait of the Arabic dialects of Siirt \citep{Lahdo2009ATillo}, comparable to the situation in Levantine Arabic, except for interrogative clauses (\citealt[132]{Jastrow1978MAqetlu1}, and §\ref{Arabic:3.2}). Across NENA dialects in the region, pre-predicate placement of the affirmative \isi{copula} is possible when the predicate serves the purpose of identification or when it expresses a transitory state, in which case the predicate will be most often adverbial, e.g. (\ref{Arabic:ex:30a}) (see, for example, \citealt[219--262]{Molin2022NEDohok} regarding J. Dohok\il{Neo-Aramaic (NENA)!J. Dohok}, NENA). The latter parallels the structure of the deictic \isi{copula}, as shown in (\ref{Arabic:ex:30b}).

\ea
\ea\label{Arabic:ex:30a}
Present affirmative \isi{copula}\\
NENA J. Challa \il{Neo-Aramaic (NENA)!J. Challa}\citep[5.6:§17]{Fassberg2010} \\
\gll ʔā́ya \textbf{=le} l-tama \\
he =\textsc{cop.3msg} on-there \\
\glt `He \textbf{is} there.' 
\ex\label{Arabic:ex:30b}
Present deictic \isi{copula}\\
NENA J. Challa \il{Neo-Aramaic (NENA)!J. Challa}\citep[102]{Fassberg2010} \\
\gll ʕaqida dexun, \textbf{wəl-le} l-axxa \\
military.leader \textsc{2pl.gen} \textsc{deic-cop.3msg} on-here\\
\glt `Your leader, \textbf{he is} right here.' 
\z
\z


\subsubsection{Auxiliaries}

Tense-Aspect-Mood is generally expressed by preverbal uninflected particles, which may have originated in auxiliaries. Complements of modal and phasal verbs immediately follow, as given in (\ref{Arabic:ex:31}). Deictic copulas are also pre-verbal, conveying imminent or ongoing events in the immediately observable present (see §\ref{Arabic:3.2} for examples).

\ea\label{Arabic:ex:31}
Auxiliary-Verb-Object\\
\ea\label{Arabic:ex:31a}
\textit{Qəltu}-Arabic Kaʿbiye \il{Arabic (Qəltu)!Kaʿbiye}\citep[II:§2]{Jastrow2022CADiyarbakir} \\
\gll \textbf{mo} \textbf{y-ṭiq}-∅ yə-šreb-∅ ṃayy əl-fəšqi \\
\textsc{neg} \textsc{S.3-}can\textsc{-msg} \textsc{A.3-}drink\textsc{-msg} water \textsc{def-}dung \\
\glt `\textbf{He cannot} drink the sewage water.' 
\ex\label{Arabic:ex:31b}
CNA Midyat, Ṭuroyo \il{Neo-Aramaic (CNA)!Ṭuroyo Midyat}\citep[24:§12]{Ritter1967Turoyo} \\
\gll hat \textbf{lú-k-qudr-ət} ∅-mbaṭl-ət úw- amro d-aloho \\
you\textsc{.sg} \textsc{neg-ind-}can-\textsc{S.2msg} \textsc{sbjv-}stop\textsc{-A.2msg} \textsc{def.msg-} command\textsc{.msg} \textsc{gen-}god\textsc{.msg} \\
\glt `You \textbf{cannot} thwart God's command.'
\z
\z

\subsubsection{Complements of non-finite verbs}

The bare \isi{object} of infinitives regularly precedes the verb in the NENA dialects of Hertevin and Bohtan, illustrated in (\ref{Arabic:ex:ad1}-\ref{Arabic:ex:ad2}) below; cf. by \isi{contrast}, NENA Barwar \citep[B5:§175]{Khan2008Barwar} \textit{lewa mšurya xala \textbf{gəlla}} 'It had not started eating \textbf{grass}'; see also §\ref{NAINEI:sec:2.3.3} in \textcitetv{chapters/15_Noorlander_NAINEI}. Interestingly, however, in Bohtan the general rate of post-verbal \isi{object} NPs is 0.33 (out of 93), while this is 0.75 (out of 96) in Hertevin. In Herevin, therefore, only objects of infinitives have shifted to \isi{OV}.

\ea\label{Arabic:ex:ad1}
NENA Hertevin \il{Neo-Aramaic (NENA)!C. Hertevin}\citep[§318-§319]{Jastrow1988NAHertevin} \\
\ea\label{Arabic:ex:ad1a}
Object-Verb(Infinitive)\\
\gll b-dar-aḥ ʾida  \textbf{gəlla} člaʾa\\
\textsc{fut-}throw-\textsc{A.1pl} hand\textsc{.fsg} grass\textsc{.msg} mow\textsc{.inf}\\
\glt `We started mowing \textbf{grass}.' 
\ex\label{Arabic:ex:ad1b}
Verb(Finte)-Object\\
\gll čléʾ-laḥ-le \textbf{gəll-an}\\
mowed\textsc{.pfv}-\textsc{A.1pl}-\textsc{O.3msg} grass\textsc{.msg}-our\\
\glt `We mowed our grass.'
\z
\z

\ea\label{Arabic:ex:ad2}
NENA Bohtan \il{Neo-Aramaic (NENA)!C. Bohtan}\citep[§118]{Jastrow1988NAHertevin} \\
\ea\label{Arabic:ex:ad2a}
Object-Verb(Infinitive)\\
\gll ani darəš-i  \textbf{qanyon-e} xlowa\\
they begin-\textsc{A.3pl} sheep-\textsc{pl} milk\textsc{.inf}\\
\glt `They started milking \textbf{sheep}.' 
\ex\label{Arabic:ex:ad2b}
Verb(Finite)-Object\\
\gll oyün=se xolü-∅ \textbf{qanyon-e}\\
he=\textsc{add} milk-\textsc{A.3msg} sheep-\textsc{pl}\\
\glt `He too was milking sheep.'
\z
\z


\section{Areal issues}

Turkish \il{Turkic!Turkish} and Northern Kurdish \il{Kurdish (Northern)} have been the dominant languages in eastern Anatolia since recent times, and Aramaic-Armenian \isi{bilingualism} presumably also occurred \citep[3]{Jastrow1994Mlahso}. Aramaic and Arabic have been in contact with each other from the beginning and both at various stages with Persian – and Greek  –in Antiquity. Moreover, Aramaic and \textit{Qəltu}-Arabic varieties of Anatolia and the Mediterranean Region were an integral part of dialect continua stretching from the Levant through Anatolia to Mesopotamia \citep{Jastrow2007RomanoArabic}. Intra-Semitic contact and Semitic-Iranian contact therefore has a long and complex history.\footnote{For areal perspectives on Anatolia, see inter alia \citet{haig_linguistic_2001,Haig2014EALingArea,haig_western_2017}; \citet[270]{Matras2009LC}; \citet{HaigKhan2019LLWA}; \citet{Khan2019Anatolia}; \citet{DonabedianSitaridou2021LCAnatolia}; on Aramaic and Kurdish\il{Kurdish} specifically, see \citet{Noorlander2014Diversity}, and on Arabic in Anatolia, e.g. \citet{Talay2007TKinfluenceA}, \citet{Prochazka2020AIST} and \citet{Akkus2020AA}.}

\begin{sloppypar}
Finding themselves between mainstream Arabic, a prototypical example of a \isi{head-initial} language, and Turkish, that of a \isi{head-final} language, the relevant Arabic and Aramaic varieties exhibit numerous hallmarks that can be characterized against the backdrop of these two spheres of influences. At the same time, these contact-induced typological traits cannot be disentangled from language-internal developments. Iranian languages like Kurdish, in turn, have a more mixed \isi{word order} typology, being in several respects rather similar to Semitic, except for the more rigid preverbal \isi{object} position, the fixed clause-final \isi{copula} placement, and the higher number of postpostions, which exhibit distinctly \isi{head-final} syntax. While the influence of such neighbouring V-final languages -- be they Iranian or Turkic -- is undeniable, intensive exposure to contact with such languages may not always radically change \isi{word order} typology. For example, the Arabic and Aramaic varieties -- originally -- spoken in Iranian Khuzestan are still characterized as \isi{VO} (\citealt[735]{Haberl2011NeoMandaic}; \citealt{ElZarkaZiagos2020WOCA}), suggesting that, if this characterization is correct -- which is still a matter of debate --, \isi{word order} can be remarkably stable. This notwithstanding, local \isi{OV} languages doubtless affected the peripheral Anatolian Arabic dialects belonging to the Diyarbakir or Kozluk-Sason-Muş cluster and the Neo-Aramaic dialect of Mlaḥso\il{Neo-Aramaic (CNA)!Mlaḥso} in Diyarbakir and that of Bohtan in Siirt. Furthermore, a higher rate of \isi{OV} can be indicative of greater \isi{word order} flexibility for pragmatically driven configurations, which could yet need not result from contact, and this is presumably the situation in the majority of NENA dialects. In fact, it is plausible that continuous interaction with the wider Arabic-speaking world reinforced more rigid \isi{VO} syntax, thereby serving as an anticatalyst against a shift to \isi{OV} or more flexibile \isi{word order}, although much more comparative data is needed to establish how frequently, for instance, \isi{object} \isi{topicalization} occurs across Arabic dialects. Thus, the rather inflexible \isi{VO} syntax in Ṭuroyo may well be the result of Arabic influence, and not necessarily an archaic feature. The greatest extent of Arabic influence on NENA, in turn, is observed on the Mosul Plain in northern Iraq (\citealt{Khan2002Qaraqosh}; \citealt{Coghill2020NA}).
\end{sloppypar}

\subsection{Kurdish\il{Kurdish} and Turkish\il{Turkic!Turkish} influence}

\subsubsection{Object/verb}\label{Arabic:3.1.1}


\begin{figure}[b]
% % %     \includegraphics{figures/Figure3Arabic.png}
	\footnotesize
    \pgfplotstableread{data/ch16-fig3.csv}\ChapterSixTFigureThreeData
    \newlength\ChapterSixTeenFigureThreeXTickL
    \settowidth\ChapterSixTeenFigureThreeXTickL{Mlaḥso, Diyarbakir, CNA }
    \begin{tikzpicture}
	\begin{axis}
		[
            axis lines*=left,
			bar width=3ex,
			font=\footnotesize,
			height=6cm,
			legend cell align=left,
			legend pos=north east,
			nodes near coords,
			nodes near coords style={/pgf/number format/fixed,font=\footnotesize},
			width=\textwidth,
			xtick=data,
			xticklabels from table={\ChapterSixTFigureThreeData}{Data},
			x tick label style={rotate=60, anchor=east, font=\sloppy\footnotesize, text width=\the\ChapterSixTeenFigureThreeXTickL, align=right},
			y tick label style={font=\footnotesize},
            ybar=3pt,
			ymin=0,
			ymax=100,
			ylabel=\%,
			ylabel near ticks
		]
		\foreach \i in {Definite O,Indefinite O}
		  {
		  	\addplot table [x expr=\coordindex, x=Data, y=\i] {\ChapterSixTFigureThreeData};
		    \edef\temp{\noexpand\addlegendentry{\i}}
		    \temp
 		  }
	\end{axis}
    \end{tikzpicture}
    \caption{The rate of post-predicate (PP) definite and indefinite objects in Arabic and Aramaic}
    \label{Arabic:fig:3}
\end{figure}


\begin{sloppypar}
The data suggest that a combination of both the area-specific contact situation and the language-specific syntax of definite objects reinforced \isi{OV} dominance. \figref{Arabic:fig:3} shows the relative frequencies in percentages of post-predicate objects in several Arabic and Aramaic dialects with the variables of \isi{definiteness} and indefiniteness. The more rigid \isi{VO} varieties are found on the left of the figure, with higher rates of \isi{VO} overall in Jewish Baghdadi \textit{Qəltu}-Arabic\il{Arabic (Qəltu)!J. Baghdadi} \citep{BarMosheCraevschi2022Arabic}, and the more rigid \isi{OV} varieties on the right, with the lowest rates of \isi{VO} overall in Mlaḥso Neo-Aramaic (Diyarbakir). The difference between definite and indefinite objects, as indicated by the first and second bars, respectively, is smallest in those dialects closer to the ends of the spectrum.

Starting from right (\isi{OV}) to left (\isi{VO}), significant variation is observed among Mlaḥso\il{Neo-Aramaic (CNA)!Mlaḥso} speakers. Speaker 2 recorded in Diyarbakir, has conventionalized \isi{OV} order consistently, showing extensive imposition of Turkic and/or Iranian, much like the Jewish NENA dialect of Urmi (\citetv{chapters/15_Noorlander_NAINEI}). This is different from the other idiolect of Mlaḥso\il{Neo-Aramaic (CNA)!Mlaḥso}, Speaker 1, recorded in Qamishli, where the slightly higher rate of \isi{VO} order is likely due to interference with Arabic and/or Ṭuroyo\il{Neo-Aramaic (CNA)!Ṭuroyo}. In the NENA dialect of Bohtan (Borb-Ruma), the high rate reflects a conventionalization of \isi{OV} \isi{word order}, but \isi{definiteness} remains a factor, and indefinite objects are lagging behind in the VO-to-\isi{OV} shift. Definiteness is even a stronger factor in the \textit{Qəltu}-Arabic\il{Arabic (Qəltu)!Daragözü} dialect of Daragözü, which is part of Kozluk-Sason-Muş group of Anatolian Arabic, where the \isi{OV}/\isi{VO} split depending on \isi{definiteness} is largely grammaticalized: \isi{OV} order for definite objects against \isi{VO} order for indefinite objects. While the same tendency is also reflected in especially the Jewish NENA dialect of Dohok \citep{Molin2022NEDohok} and to some extent also in the \textit{Qəltu}-Arabic\il{Arabic (Qəltu)!Kaʿbiye} variety of Kaʿbiye, these two varieties waver more strongly towards \isi{VO}. Finally, the statistics are also given for Cilician Levantine Arabic\footnote{Contact with Turkish\il{Turkic!Turkish} could still play a \isi{role} here, see \citet{Noorlander2024OVArabi}.}, showing, like Jewish Baghdadi, no significant difference, presumably due to contact with mainstream Arabic.
\end{sloppypar}

Historically, the complete syntacticization of \isi{OV} as in Central Asian Arabic\il{Arabic!Central Asian} (e.g. \citealt{Seeger2002AChorasan}; \citealt{Jastrow2004UzbekA}) and in Jewish NENA of Iran (e.g. \citealt{Khan2020ContactChange}; \citealt{NoorlanderMolin2022WordOrder}; \citetv{chapters/15_Noorlander_NAINEI}) is only sporadically observed in Anatolia, and \isi{definiteness} remains a major factor in regulating \isi{object} placement. While \isi{VO} is certainly the more archaic order, alternative pragmatically driven configurations were presumably part of Central Semitic syntax as a whole, but perhaps not with equal frequency across the entire subgroup. The shift to \isi{OV} in Central Asian Arabic\il{Arabic!Central Asian} is generally explained in terms of the conventionalization and thus increase in frequency of a former \isi{topicalization} strategy, e.g. \textit{That man -- I saw him yesterday}, due to the exposure to intensive contact  with Uzbek and Tajik (e.g. \citealt[452]{Versteegh1984UzbekA}; \citealt{Ratcliffe2004BukharaA}; \citealt[56]{Souag2017CDoubling}); the same holds for \isi{OV} in NENA under Kurdish\il{Kurdish} influence \citep[410–412]{haig_verb-goal_2015}. These explanations are consistent with the data as well as with the view of contact-induced \isi{word order} \isi{convergence} as a result of the extension of a pre-existing parallel construction (e.g. \citealt{Silva-Corvalan1994ContactChange,Silva-Corvalan2008Limits}\footnote{I am indebted to G. Haig for directing my attention to this reference.}; \citealt{Heine2008contact}).

\subsubsection{Possessum/existential}

\begin{sloppypar}
Possessum placement in locative-existential predicative possession correlates with \isi{object} placement, as illustrated in (\ref{Arabic:ex:32}--\ref{Arabic:ex:33}). Lower rates of post-predicate possessums correlate with lower rates of post-predicate objects. The dialects are contrasted in \figref{Arabic:fig:4} where the line indicates the frequency of post-predicate possessums across dialects decreasing significantly from core into periphery, and the bars indicate the frequency of post-predicate definite objects.
\end{sloppypar}

\begin{figure}
% % %     \includegraphics{figures/Figure4Arabic.png}
    \begin{tikzpicture}
     \small
     \begin{axis}
	  [
        axis lines*=left,
		bar width=4ex,
		font=\small,
		height=5cm,
% 		enlarge x limits=0.50,
		legend cell align=left,
		legend pos = north east,
		width=\textwidth,
		xtick=data,
		x tick label style={text width=1.5cm, align=center},
		symbolic x coords={{Midyat, Ṭuroyo, CNA},{Barwar, NENA},{Cilicia, Lev. Arabic},{Kaʿbiye, qəltu-Arabic},{Mlaḥso, CNA},{Bohtan, NENA}},
        ybar,
		ylabel=\%,
		ylabel near ticks,
		ymin=0,
		ymax=100,
		ymajorgrids=true
	  ]
	\addplot [style={black!80,fill=black!50}] 
	  coordinates{
          ({Midyat, Ṭuroyo, CNA},88)
          ({Barwar, NENA},71)       
          ({Cilicia, Lev. Arabic},85)      
          ({Kaʿbiye, qəltu-Arabic},70)
          ({Mlaḥso, CNA},16)               
          ({Bohtan, NENA},20)
	  };
	\addlegendentry{Definite O}
	\addplot [smooth, mark=*, black, thick] 
	  coordinates{
          ({Midyat, Ṭuroyo, CNA},94)     
          ({Barwar, NENA},98)            
          ({Cilicia, Lev. Arabic},90)      
          ({Kaʿbiye, qəltu-Arabic},55)   
          ({Mlaḥso, CNA},22)               
          ({Bohtan, NENA},9)            
	  };
     \addlegendentry{Possessum}
     \end{axis}
     \end{tikzpicture}
    \caption{The rate of post-predicate definite and indefinite objects in Arabic and Aramaic}
    \label{Arabic:fig:4}
\end{figure}


\ea\label{Arabic:ex:32}
\ea\label{Arabic:ex:32a}
Existential-Possessum\\
Cilician Arabic Çukurova \il{Arabic!Cilician}\citep[2.4:§1]{Prochazka2002ACukurova} \\
\gll kān il-a \textbf{ibin} \\
\textsc{pst.cop.3msg} to\textsc{-3fsg} son\textsc{.msg} \\
\glt `She had \textbf{a son}.' 
\ex\label{Arabic:ex:32b}
Possessum-Existential\\
\textit{Qəltu}-Arabic Kaʿbiye \il{Arabic (Qəltu)!Kaʿbiye}\citep[XIV:§12]{Jastrow2022CADiyarbakir} \\
\gll ənti mādām \textbf{bayt} \textbf{abu} lə-ki \\
you\textsc{.fsg} because house\textsc{.msg} father\textsc{.msg} to\textsc{-2fsg} \\
\glt `because you have \textbf{a father house}'
\z
\z

\ea\label{Arabic:ex:33}
\ea\label{Arabic:ex:33a}
Existential-Possessum\\
CNA Ṭuroyo \il{Neo-Aramaic (CNA)!Ṭuroyo}\citep[27:§1]{Ritter1967Turoyo} \\
\gll kət-wo šulṭono kət-wo-le \textbf{bəstono} \\
\textsc{exist-pst} sultan\textsc{.msg} \textsc{exist-pst-A.3msg} garden\textsc{.msg} \\
\glt `There was once a sultan who had \textbf{a garden}.' 
\ex\label{Arabic:ex:33b}
Possessum-Existential\\
CNA Mlaḥso \il{Neo-Aramaic (CNA)!Mlaḥso}\citep[§71]{Jastrow1994Mlahso} \\
\gll \textbf{karme} hito el-əna \\
vineyards\textsc{.pl} \textsc{exist} to\textsc{-1pl} \\
\glt `We have \textbf{vineyards}.'
\z
\z

Sporadically, existential possessors -- otherwise predicative -- can even be expressed adnominally as they are in Kurdish\il{Kurdish} (cf. \citealt{Fox2009NABohtan}: 116 fn. 103) and Turkish\il{Turkic!Turkish}, cf. (\ref{Arabic:ex:34}) and (\ref{Arabic:ex:35}) with (\ref{Arabic:ex:36}) and (\ref{Arabic:ex:37}).

\ea\label{Arabic:ex:34}
Northern Kurdish\il{Kurdish (Northern)}\\
\gll bira-yek=î \textbf{min} he-ye \\
brother\textsc{-indef=ez.msg} my \textsc{exist-cop.3sg} \\
\glt `\textbf{I} \textbf{have} a brother.'
\z

\ea\label{Arabic:ex:35}
Turkish\il{Turkic!Turkish}\\
\gll oğlu\textbf{-m} yok \\
son-my \textsc{neg.exist} \\
\glt `\textbf{I} \textbf{have} no son.'
\z

\ea\label{Arabic:ex:36}
CNA Mlaḥso \il{Neo-Aramaic (CNA)!Mlaḥso}\citep[106.§12]{Jastrow1994Mlahso} \\
\gll aḥ-\textbf{i} hito \\
brother\textsc{.msg}-my \textsc{exist} \\
\glt `\textbf{I} \textbf{have} a brother.'
\z

\ea\label{Arabic:ex:37}
NENA Bohtan \il{Neo-Aramaic (NENA)!C. Bohtan}\citep[4.1:§1]{Fox2009NABohtan} \\
\gll oyün iwa baxt-\textbf{ǝw}=u abr-\textbf{ǝw} \\
he \textsc{pst.cop.3sg} wife\textsc{.fsg}-his=and son\textsc{.msg}-his \\
\glt `He \textbf{had} a wife and a son.'
\z

\subsubsection{Light-verb complements}

Similarly, the non-referential \isi{complement} of light verb constructions follows the verb where \isi{VO} predominates, as given in (\ref{Arabic:ex:38}--\ref{Arabic:ex:40}).\footnote{For more examples, see e.g. \citet[184]{Talay2007TKinfluenceA}; \citet[150]{Akkus2020AA}; \citet[97]{Prochazka2020AIST}.}

\ea\label{Arabic:ex:38}
T. \textit{banyo etmek}\\
\textit{Qəltu}-Arabic Mardin \il{Arabic (Qəltu)!Mardin}\citep[I.1:§60]{Jastrow1981MAqetlu2} \\
\gll t-a-ǧi a-sawiy-u ḅāṇyo \\
\textsc{fut-S.1sg-}come \textsc{A.1sg-}do\textsc{-O.3msg} bathroom \\
\glt `I shall come to wash him.'
\z

\ea\label{Arabic:ex:39}
T. \textit{telefon etmek}, K. \textit{telefon kirin}\\
CNA Midyat, Ṭuroyo \il{Neo-Aramaic (CNA)!Ṭuroyo Midyat}\citep[7:§13]{Ritter1967Turoyo} \\
\gll səm-li talafṓn l-ú= ḥakimo d-áḥ= ḥəyewən \\
did\textsc{.pfv-A.1sg} telephone to\textsc{-def.msg}= doctor\textsc{.msg} \textsc{gen-def.pl}= animal\textsc{.pl} \\
\glt `I called the vet.'
\z

\ea\label{Arabic:ex:40}
T. \textit{idare etmek} \\
NENA Hertevin \il{Neo-Aramaic (NENA)!C. Hertevin}\citep[156.§505]{Jastrow1988NAHertevin} \\
\gll ∅-ʔod-aḥ-be ʔidara \\
\textsc{sbvj-}do\textsc{-A.3msg-}by.it management \\
\glt `so that we come through it'
\z

The \isi{complement} regularly precedes the light verb in dialects with a higher \isi{OV} rate:

\ea\label{Arabic:ex:41}
T. \textit{keyf etmek}, K. \textit{kêf kirin}\\
\textit{Qəltu}-Arabic Kaʿbiye \il{Arabic (Qəltu)!Kaʿbiye}\citep[XIII:§4]{Jastrow2022CADiyarbakir} \\
\gll kēf saw-ōn \\
joy did\textsc{.pfv-A.3pl} \\
\glt `They celebrated.'  
\z

\ea\label{Arabic:ex:42}
K. \textit{kar kirin}\\
CNA Mlaḥso \il{Neo-Aramaic (CNA)!Mlaḥso}\citep[106.§17]{Jastrow1994Mlahso} \\
\gll l-á=ṭay-e kar sim-no \\
for\textsc{-def=}muslim\textsc{-pl} labour do\textsc{-1msg} \\
\glt `I shall work for the Muslims.'
\z

\ea\label{Arabic:ex:43}
Russ. \textit{sobraniye}\\
NENA Bohtan \il{Neo-Aramaic (NENA)!C. Bohtan}\citep[4.2:§2]{Fox2009NABohtan} \\
\gll sabroni yawd-i \\
meeting do\textsc{-A.3pl} \\
\glt `They held a meeting.'
\z


\subsubsection{Addressee/verb \& verb/goal}\label{Arabic:3.1.4}
\largerpage

The Addressee placement in Mlaḥso\il{Neo-Aramaic (CNA)!Mlaḥso} also converges with Kurdish\il{Kurdish} \isi{word order} typology. In Mlaḥso\il{Neo-Aramaic (CNA)!Mlaḥso} the post-predicate rate of addressees is only 56\% (18/32), whereas this rate is a 100\% in Ṭuroyo\il{Neo-Aramaic (CNA)!Ṭuroyo Midyat} (Midyat; 26/26). Mlaḥso\il{Neo-Aramaic (CNA)!Mlaḥso} also shows a higher rate of Beneficiary-Verb order (11/23) as opposed to Verb-Recipient order (16/17). Addressee-Verb and Beneficiary-Verb order as opposed to Verb-Goal order corresponds to the Northern Kurdish\il{Kurdish (Northern)} \isi{word order} pattern in the same region \citep{Haig2022PostPredicateCon}. Imposition from Iranian – rather than Turkish\il{Turkic!Turkish} – is thus the most likely explanation for this syntactic split in Mlaḥso\il{Neo-Aramaic (CNA)!Mlaḥso}. Oblique-Verb order in Mlaḥso\il{Neo-Aramaic (CNA)!Mlaḥso} (§\ref{Arabic:2.2.4}) could be due to contact with either Iranian or Turkish.

\subsubsection{Wh-in-situ}

Wh-elements, or interrogatives, regularly remain in-situ in Kurdish\il{Kurdish}, as given in (\ref{Arabic:ex:44a}) for direct objects and (\ref{Arabic:ex:44b})  for goals, although the latter can also undergo fronting as in (\ref{Arabic:ex:44c}) \citep[339]{Haig2022PostPredicateCon}. The same order would be obtained for Turkish\il{Turkic!Turkish}, although, here, most arguments, including goals, regularly precede the predicate contrary to Kurdish\il{Kurdish} (see \citetv{chapters/1_Haigetal_Intro}).

%

\ea\label{Arabic:ex:44}
Northern Kurdish\il{Kurdish (Northern)}\\
\ea\label{Arabic:ex:44a}
Object \textit{wh}-in-situ\\
\gll tu di-zan-î min \textbf{çi} kir \\
\textsc{A.dir.2sg} \textsc{ind-}know\textsc{-A.2sg} \textsc{A.obl.1sg} what did\textsc{.pst} \\
\glt `You know \textbf{what} I did.'

\ex\label{Arabic:ex:44b}
Goal \textit{wh}-in-situ\\
\gll tu di-č-î \textbf{kîve} \\
\textsc{A.dir.2sg} \textsc{ind-}go\textsc{-S.2sg} where \\
\glt `\textbf{Where} are you going?'
\ex\label{Arabic:ex:44c}
Goal \textit{wh}-fronted\\
\gll tu kîve di-č-î \\
\textsc{A.dir.2sg} \textbf{where} \textsc{ind}-go-\textsc{S.2sg} \\
\glt `\textbf{Where} are you going?'
\z
\z

\begin{sloppypar}
Generally, wh-fronting occurs in the relevant Arabic and Aramaic dialects, as expected for \isi{VO} typology. Occasionally, the wh-element stays in situ, as illustrated in (\ref{Arabic:ex:45}). In the NENA dialect of Bohtan -- and the CNA dialect of Mlaḥso -- the \isi{object} interrogative, stays in situ in line with their \isi{OV} typology, as shown in (\ref{Arabic:ex:46}).  When wh-fronting occurs, the clause-initial slot remains open for a topical element, which usually is the subject. The resulting order of Subject-Interrogative-Predicate converges with that found in Northern Kurdish\il{Kurdish (Northern)} and Turkish\il{Turkic!Turkish}, as illustrated in (\ref{Arabic:ex:46}--\ref{Arabic:ex:47}) below, and in §\ref{Arabic:3.2}.
\end{sloppypar}


\ea\label{Arabic:ex:45}
CNA Midən \il{Neo-Aramaic (CNA)!Ṭuroyo Midən}\citep[115:§7]{Ritter1967Turoyo} \\
\gll adlalyo g-əzz-ano \textbf{l-ayko}  \\
tonight \textsc{fut-}go\textsc{-S.1pl} to-where\\
\glt `\textbf{Where} shall we go tonight?' 
\z


\ea\label{Arabic:ex:46}
NENA Bohtan \il{Neo-Aramaic (NENA)!C. Bohtan}\citep[4.3:§38]{Fox2009NABohtan}\\
\gll yad-ǝt ona \textbf{moy} wid-ena \\
know\textsc{-A.2msg} I what do\textsc{.perf-A.1msg} \\
\glt `You know \textbf{what} I have done.'
\z

\ea\label{Arabic:ex:ad3}
CNA Midən \il{Neo-Aramaic (CNA)!Ṭuroyo Midən}\citep[73:§273]{Ritter1967Turoyo} \\
\gll hat \textbf{mə} ko-saym-ət  \\
you\textsc{.sg} what \textsc{ind-}do\textsc{-A.2msg}\\
\glt `\textbf{What} are you doing?' 
\z

\ea\label{Arabic:ex:47}
\textit{Qəltu}-Arabic Hasköy \il{Arabic (Qəltu)!Hasköy}\citep[I.2.4]{Talay2002AHaskoyT} \\
\gll ina \textbf{šəna} āsi \\
I what \textsc{A.1sg.}do \\
\glt `\textbf{What} should I do?' 
\z

\subsubsection{Postpositions?}
The relevant Aramaic and Arabic dialects in contact with \isi{OV} languages have maintained prepositional marking. In one case, however, the NENA dialect of Borb-Ruma (Bohtan) developed a postposition \textit{=ləl} out of the \isi{preposition} \textit{lal-} \citep[101–102]{Fox2009NABohtan}, e.g. (\ref{Arabic:ex:48a}). There is, however, no direct correspondence to a postposition in any of the neighbouring languages, which, in fact, would not generally use a posposition with inanimate goals, cf. (\ref{Arabic:ex:49b}). When it attaches to the predicate, this is presumably an instance of \isi{convergence} with the Northern Kurdish\il{Kurdish (Northern)} directional particle \textit{=e} \citep[101–102]{Fox2009NABohtan}, cf. (\ref{Arabic:ex:48a}) with (\ref{Arabic:ex:49a}).

\ea
\ea\label{Arabic:ex:48a}
Postposition \\
NENA Bohtan \il{Neo-Aramaic (NENA)!C. Bohtan}\citep[126:§139]{Fox2009NABohtan} \\
\gll üzü-∅-wa matwota\textbf{=lǝl} \\
go\textsc{.ant-S.3msg-pst} \textbf{village}\textsc{.pl=drct} \\
\glt `He had gone \textbf{to the villages}.'
\ex\label{Arabic:ex:48b}
Directional particle \\
NENA Bohtan \il{Neo-Aramaic (NENA)!C. Bohtan}\citep[118:§35]{Fox2009NABohtan} \\
\gll duwa yar-o\textbf{=lal} \textbf{gawr-aw} \\
woman\textsc{.fsg} say\textsc{-A.3fsg=drct} man\textsc{.msg-}her \\
\glt `His mother says \textbf{to her husband}...' 
\z
\z

\newpage
\ea
\ea\label{Arabic:ex:49a}
Directional particle and \isi{oblique}\\
Bahdini, Northern Kurdish\il{Kurdish (Northern)}\\
\gll ez di-çû-m=\textbf{e} \textbf{mal-ê} \\
\textsc{1sg.dir} \textsc{ind-}go\textsc{-S.1sg=drct} house\textsc{-obl.f} \\
\glt `I'm going \textbf{home}.'
\ex\label{Arabic:ex:49b}
Directional particle and \isi{oblique}\\
Bahdini, Northern Kurdish\il{Kurdish (Northern)}\\
\gll min got=\textbf{e} \textbf{Mesûd-î} \\
\textsc{1sg.obl} say\textsc{.pst=drct} Masoud\textsc{-obl.m} \\
\glt `I said \textbf{to Masoud}.'
\z
\z

\subsubsection{Other OV correlates}

Several other phenomena are related to the \isi{head-final} typology of especially Turkish\il{Turkic!Turkish}, which are again imposed more strongly in the northern and western periphery. A case in point is \isi{copula} placement: in Mlaḥso\il{Neo-Aramaic (CNA)!Mlaḥso} Neo-Aramaic and the \textit{Qəltu}-Arabic\il{Arabic (Qəltu)!Diyarbakir} dialects of Diyarbakir and Kozluk-Sason-Muş areas, the post-predicate position has been conventionalised for also the negated \isi{copula}. In the dialect of Hasankeyf, in turn, the exact parallel to Kurmanji occurs: the negator itself is pre-predicate but the \isi{copula} post-predicate (see \ref{Arabic:tab:5} in §\ref{Arabic:2.3.1}). This tendency is reflected in the statistics of the doculects: the rate of post-predicate copula complement\is{copula!complement}s is only 3\% (2/65) in Mlaḥso\il{Neo-Aramaic (CNA)!Mlaḥso} (CNA) and 5\% (4/79) in Kaʿbiye (Qəlṭu-Arabic) against 20\% (14/71) in Ṭuroyo\il{Neo-Aramaic (CNA)!Ṭuroyo} (CNA) and 35\% (54/156) in C. Barwar\il{Neo-Aramaic (NENA)!C. Barwar} (NENA).

While it is difficult to determine the language which ultimately provided the model for the development of post-predicate copulas in these Semitic languages, it is reasonable to assume an interplay of language-internal changes as well as contact-induced reinforcement and areal diffusion. The placement of a \isi{pronoun} serving as the subject of a non-verbal clause, for instance, was not entirely fixed and its post-predicate position was part of the repertoire of Central Semitic. The starting point could have been interrogative clauses, as in the \textit{Qəltu}-Arabic\il{Arabic (Qəltu)!Siirt} dialect of Siirt: the \isi{copula} otherwise precedes the predicate but it is placed after the interrogative, which incidentally converges with Northern Kurdish\il{Kurdish (Northern)} syntax, as compared in (\ref{Arabic:ex:50}--\ref{Arabic:ex:52}). \textit{Subject-Interrogative-Copula} is the common order in the majority of other varieties of Anatolian Arabic besides Neo-Aramaic. One can \isi{contrast} this with the dialects of Arabic that did not develop an analytical \isi{copula}, such as (\ref{Arabic:ex:51}).

\newpage
\ea\label{Arabic:ex:50}
Northern Kurdish\il{Kurdish (Northern)} (p.c. with Ergin Öpengin)\\
\gll ev çi-qas=e \\
\textsc{dem} what-value\textsc{=cop.3sg} \\
\glt `How much is this?'
\z

\ea\label{Arabic:ex:ad4}
CNA Ṭuroyo \il{Neo-Aramaic (CNA)!Ṭuroyo}\\
\gll hano məq-qa=yo \\
\textsc{dem.msg} what-value\textsc{=cop.3msg} \\
\glt `How much is this?'
\z

\ea\label{Arabic:ex:51}
\textit{Qəltu}-Arabic Āzəx \il{Arabic (Qəltu)!Āzəx}\citep[135]{Jastrow1978MAqetlu1} \\
\gll hāza b-áš-qad=u \\
\textsc{dem.msg} at-what-value\textsc{=cop.3msg} \\
\glt `How much is this?'
\z

\ea\label{Arabic:ex:52}
\textit{Gələt}-Arabic Khawetna \il{Arabic (Gələt)!Khawetna}\citep[54]{Talay1999AKhawetnaG} \\
\gll b-əš-qadd hāða \\
at-what-value \textsc{dem.msg} \\
\glt `How much is this?'
\z

The further extension of this post-predicate position to other contexts and its increasing obligatorisation was presumably not only due to contact with clause-final \isi{copula} languages such as Iranian \textit{par excellence}, but also embedded in a cluster of changes in the verbal system, and this applies especially to Aramaic. See \citet{NoorlanderandStilo2015} for further parallel developments in the verbal system of Eastern Neo-Aramaic, which is completely derived from original verbal adjectives and enclitic pronouns that used to be mobile clitics, for example in Syriac (e.g. \citealt[15]{Noorlander2018Alignment}), and that of other languages in the area, such as Northern Kurdish\il{Kurdish (Northern)}, where present tense endings, e.g. \textit{-im} as in \textit{diçim} `I go,' are identical to the original verb `to be,' e.g. \textit{=im} as in \textit{li vir=im} `I am here'.

\begin{sloppypar}
Moreover, Adjective-Noun order would be highly marked in all relevant Semitic languages but typical of Turkic. Incidentally, the Turkish\il{Turkic!Turkish} loan \isi{adjective} \textit{dēri} `another, last, next,' i.e. Turkish\il{Turkic!Turkish} \textit{diğeri} `the other,' has been transferred with its corresponding Adjective-Noun order, e.g. 
\end{sloppypar}

\ea\label{Arabic:ex:53}
\textit{Qəltu}-Arabic Āzəx \il{Arabic (Qəltu)!Āzəx}\citep[121]{Wittrich2001AAzex} \\
\gll dēri yawm \\
next day\textsc{.msg} \\
\glt `the next day'
\z

\ea\label{Arabic:ex:54}
CNA Ṭuroyo\il{Neo-Aramaic (CNA)!Ṭuroyo}\\
\gll deri yawmo \\
next day\textsc{.msg}\\
\glt `the next day'
\z

In Kaʿbiye \textit{Qəltu}-Arabic, Adjective-Noun order sporadically occurs under the influence of Turkish\il{Turkic!Turkish} \citep[7–8]{Jastrow2022CADiyarbakir}, cf. (\ref{Arabic:ex:55}) and a tentative Turkish\il{Turkic!Turkish} rendering in (\ref{Arabic:ex:56}). Further research is necessary to investigate its frequency and distribution.

\ea\label{Arabic:ex:55}
\textit{Qəltu}-Arabic Kaʿbiye \il{Arabic (Qəltu)!Kaʿbiye}\citep[VI:§45]{Jastrow2022CADiyarbakir} \\
\gll ktir kwayyəs faqad ṣūf y-ṣīr-∅ \\
very nice \textsc{indef} wool\textsc{.msg} \textsc{S.3m-}become\textsc{-S.sg} \\
\glt `It will become some very nice wool.'
\z

\ea\label{Arabic:ex:56}
Turkish\il{Turkic!Turkish}\\
\gll çok güzel bir yün ol-acak-∅ \\
very nice \textsc{indef} wool become\textsc{-fut-S.3sg} \\
\glt `It will become some very nice wool.'
\z

Similarly, there are numerous cases of pre-predicate final states of change-of-state verbs (§\ref{Arabic:2.2.3}) as shown in example (\ref{Arabic:ex:55}) in peripheral dialects like Kaʿbiye Arabic due to Turkish\il{Turkic!Turkish} influence. Thus, the rate of post-predicate final states is 60\% (15/25) in Kaʿbiye (\textit{qəltu-}Arabic) and 67\% (24/36) in Mlaḥso\il{Neo-Aramaic (CNA)!Mlaḥso} (CNA) against 90\% (18/20) in C. Barwar\il{Neo-Aramaic (NENA)!C. Barwar} (NENA) and 100\% (20/20) in Ṭuroyo\il{Neo-Aramaic (CNA)!Ṭuroyo} (CNA).

Finally, in both Aramaic and Arabic, the standard of comparison, introduced with the source \isi{preposition} \textit{mən-} `from,' seems to \textit{precede} the \isi{adjective} in Diyarbakir, as it does in local Turkish\il{Turkic!Turkish} and Northern Kurdish\il{Kurdish (Northern)} varieties, which is consistent with the higher rate of \isi{OV} order. The same also holds for Sason Arabic \citep[144--145]{Akkus2020AA}. 

\ea\label{Arabic:ex:57}
\textit{Qəltu}-Arabic Kaʿbiye \il{Arabic (Qəltu)!Kaʿbiye}\citep[IX:§19]{Jastrow2022CADiyarbakir} \\
\gll \textbf{mən} \textbf{sáyn-na}=ste ʔáxrab=we \\
from language-our\textsc{=add} worse\textsc{=cop.3msg}  \\
\glt `It is even worse \textbf{than our language}.'
\z

\ea\label{Arabic:ex:58}
CNA Mlaḥso \il{Neo-Aramaic (CNA)!Mlaḥso}\citep[112.§48]{Jastrow1994Mlahso} \\
\gll hay-ó ṭaw \textbf{m-á=ṭay-e}=zi tə ḥarb-ó \\
become\textsc{.perf-S.3sfg} Muslim\textsc{.msg} from\textsc{-def.pl.}Muslim\textsc{-pl=add} more bad\textsc{-msg} \\
\glt `She became Muslim, worse \textbf{than the Muslims} themselves.'
\z

\begin{sloppypar}
The opposite Adjective-Standard order predominates elsewhere, cf. (\ref{Arabic:ex:59}--\ref{Arabic:ex:60}) (and see \citealt[50--51, 117--118]{Waltisberg2016STuroyo} for more examples), even in the NENA dialect of Bohtan (Borb-Ruma) where \isi{OV} order is the most common, as shown in (\ref{Arabic:ex:61}).
\end{sloppypar}

\ea\label{Arabic:ex:59}
\textit{Qəltu}-Arabic Mardin \il{Arabic (Qəltu)!Mardin}\citep[I2:§32]{Jastrow1981MAqetlu2} \\
\gll ʕamm-i agbaṛ \textbf{mən} \textbf{abū-y} kān \\
uncle\textsc{.msg-}my bigger from father\textsc{.msg-}my \textsc{cop.pst.3msg} \\
\glt `My uncle was older \textbf{than my father}.'
\z

\ea\label{Arabic:ex:60}
CNA Midən, Ṭuroyo \il{Neo-Aramaic (CNA)!Ṭuroyo Midən}\citep[83:§39]{Ritter1967Turoyo} \\
\gll ono rab-∅ \textbf{mín-ux}=no \\
I big\textsc{-cmpr} from\textsc{-2msg=cop.1sg} \\
\glt `I am older \textbf{than you}.'
\z

\ea\label{Arabic:ex:61}
NENA C. Bohtan \il{Neo-Aramaic (NENA)!C. Bohtan}\citep[95]{Fox2009NABohtan} \\
\gll ay brota ṭo qaryan-ita=la \textbf{mənn-ət} \textbf{d-aw} \textbf{abra} \\ 
\textsc{dem.fsg} girl\textsc{.sg} more short\textsc{-fsg=cop.3fsg} from\textsc{-cstr} \textsc{gen-dem.msg} boy\textsc{.msg} \\
\glt `This girl is shorter \textbf{than that boy}.'
\z

\subsection{Levant-Anatolia continuum}\label{Arabic:3.2}

Several typological features indicate diffusion into eastern Anatolia from the Levant and Mesopotamia, resulting in many parallels between Aramaic and Arabic (see e.g. \citealt{Weninger2012AAContact} and \citealt{Prochazka2020AIST}), and to some extent also Iranian, to name a few: First of all, differential \isi{object} indexing (also known as clitic doubling\is{clitic!doubling}) possibly combined with the \isi{preposition} \textit{l-} is a feature shared by Aramaic (e.g. \citealt{Coghill2014}; \citealt[290–294, 307–308, 350–370]{Noorlander2021Alignment}) and Arabic (e.g. \citealt{Souag2017CDoubling}); this, however, correlates with pre-posed objects especially in Anatolian \textit{qəltu-}Arabic and Ṭuroyo\il{Neo-Aramaic (CNA)!Ṭuroyo}. The correlation between differential \isi{object} indexing and \isi{word order} requires further investigation, but see also §\ref{Arabic:3.1.1} on \isi{definiteness}, of which indexing may be an epiphenomenon.

Verb-Goal and Become-Complement order is shared by Semitic and Kurdish\il{Kurdish} more widely \citep{haig_verb-goal_2015,Haig2022PostPredicateCon}, more specifically Verb-Addressee placement converges with Bahdini, i.e. southeastern varieties of Northern Kurdish\il{Kurdish (Northern)} \citep[354–359]{Haig2022PostPredicateCon}.

\begin{sloppypar}
In nominal syntax, Arabic adjectives such as \textit{ʔawwəl} `first,' \textit{θēni} `next, another,' \textit{ġēr} `other,' and \textit{flān} `so-and-so' are borrowed with their respective Adjective-Noun order in Neo-Aramaic (see also \citealt[40--41]{Waltisberg2016STuroyo}), e.g. NENA Hertevin (\citealt{Jastrow1988NAHertevin}) \textit{plan dokta} `such-and-such a place', NENA Bohtan (\citealt{Fox2009NABohtan}) \textit{fəllan mota} `such-and-such a village', and oftentimes also in Kurdish, e.g. \textit{filan kes} `such-and-such a person'. Noun-Numeral order for the numeral `one', as illustrated in (\ref{Arabic:ex:4}--\ref{Arabic:ex:5}), as well as the development of a prefixal definite article have presumably been reinforced in Ṭuroyo through contact with Arabic. In Mlaḥso, when only the genitive noun is marked for \isi{definiteness}, this is presumably based on an Arabic model, cf. (\ref{Arabic:ex:11}) and (\ref{Arabic:ex:13}) in §\ref{Arabic:2.1}.
\end{sloppypar}

\begin{sloppypar}
Semitic and Iranian converge in Noun-Attribute order. Here, the attachment of proclitic determiners to the following \isi{adjective} in Aramaic (see \citealt{Waltisberg2016STuroyo} for more examples), as shown in (\ref{Arabic:ex:64}) and (\ref{Arabic:ex:65}), converges not only with the \textit{ezafe} in Northern Kurdish\il{Kurdish (Northern)}, e.g. (\ref{Arabic:ex:62}), but also with the definite article in Arabic dialects, e.g. (\ref{Arabic:ex:63}).
\end{sloppypar}

\ea\label{Arabic:ex:62}
Northern Kurdish\il{Kurdish (Northern)}\\
\gll biray=ê min\textbf{=ê} mezin \\
brother\textsc{=ez.msg} my\textsc{=ez.msg} big  \\
\glt `my elder brother'
\z

\ea\label{Arabic:ex:63}
\textit{Qəltu}-Arabic Kaʿbiye \il{Arabic (Qəltu)!Kaʿbiye}\citep[III:§3]{Jastrow2022CADiyarbakir} \\
\gll axū-y \textbf{lə-}gbīr-∅=ste \\
brother.of\textsc{.msg-}my \textsc{def-}big\textsc{-msg=add} \\
\glt `my elder brother'
\z

\ea\label{Arabic:ex:64}
CNA Midyat \il{Neo-Aramaic (CNA)!Ṭuroyo Midyat}\citep[11:§36]{Ritter1967Turoyo} \\
\gll ʔaḥun-i \textbf{ú꞊}rab-o \\
brother\textsc{.MSG-}my \textsc{DEF.MSG-}big\textsc{-MSG} \\
\glt `my eldest brother'
\z

\ea\label{Arabic:ex:65}
NENA Upper Barwar \il{Neo-Aramaic (NENA)!C. Barwar}\citep[516.§2]{Talay2009NAKhaburAssyrer} \\
\gll xon-i \textbf{ʔó}꞊goṛ-a \\
brother\textsc{.msg-}my \textsc{dem.msg-}big\textsc{-msg} \\
\glt `my eldest brother'
\z


The effects of \isi{convergence} with both Arabic and Iranian can be further illustrated by the divergent usage patterns of post-predicate person markers. This pronominal series, often cliticized to the immediately preceding predicate, occurs across Arabic and Aramaic dialects to indicate both the present affirmative \isi{copula} \textit{and} the pronominal theme-\isi{object} in ditransitive constructions (e.g. \citealt{Retso1987CopulaObject}, \citealt{Birnstiel2022CopulaKA}). The first is partly modelled on the clause-final \isi{copula} in the neighbouring Iranian languages. The second, however, suggests close interaction with the Arabic-speaking Levant and Arabia. While the post-predicate \isi{copula} is a feature common to all languages of the West Asian \isi{transition zone} (e.g. \citealt[404–405]{haig_western_2017}), there are notable differences, such as the pre-predicate placement of negated copulas and relative copulas (see §\ref{Arabic:2.3.1}) in the majority of both Aramaic and Arabic dialects. In Syriac, however, the enclitic \isi{copula} and the bound pronominal objects of the third person plural were also identical in form, cf. \textit{šappirin=ennun} `they are beautiful' and \textit{qṭal=ennun} `he killed them', derived from \textit{hennun} `they'. This copula-\isi{object} syncretism, compared in \tabref{Arabic:tab:6}, applies especially to the \textit{Qəltu}-Arabic\il{Arabic (Qəltu)} and Neo-Aramaic dialects of Mardin (\citet{Grigore2007AMArdin}) and Siirt provinces in Turkey and that of the Mosul Plain (\citealt{Jastrow1979AMosul}; \citealt{Khan2002Qaraqosh}) in Iraq, but is also common in the Levant and Ḥejaz \citep{Retso1987CopulaObject}. The Mardini dialects in Syria have enclitic pronouns only for the third person in both the \isi{copula} and the \isi{object} marking function \citep{IsakssonLahdo2002ThreeBT}. They are confined to the \isi{object} marking function in Mosul \textit{Qəltu}-Arabic \citep{Jastrow1979AMosul}. Other \textit{Qəltu}-Arabic\il{Arabic (Qəltu)} dialects do not have such enclitic copulas, notably in Iraq and Syria, although they may be restricted to the negative \isi{copula}, e.g. \textit{Gələt-}Arabic Khawetna \textit{ma-hi} `she is not' \citep[54–55]{Talay1999AKhawetnaG}.
    
\begin{table}[t]
    \begin{tabularx}{\textwidth}{XlXl}
\lsptoprule
\textbf{Syrian Arabic} & \multicolumn{2}{c}{\textbf{\textit{Qəltu}-Arabic}} & \\
\textbf{Cilician} & \textbf{Mosul} & \textbf{Mardin} \\
\midrule
\textit{ʕaṭa-ni hinni} & \textit{ʕaṭā́-nī=yəm} & \textit{ʕaṭā́-nī=nne} & `he gave them to me' \\
\textit{hinni fəl-bayt} & \textit{hīyəm fəl-bēt} & \textit{fəl-báyt=ənne} & `they are at home' \\
\midrule
& \textbf{NENA} & \textbf{CNA} & \\
& \textbf{Qaraqosh} & \textbf{Ṭuroyo} & \\
\midrule
& \textit{kewí-li=na} & \textit{kobí-li=ne} & `they give them to me' \\
& \textit{ṭawe=na} & \textit{ṭáwwe=ne} & `they are good' \\
\lspbottomrule
    \end{tabularx}
    \caption{Comparison of copula placement in \textit{Qəltu}-Arabic and Central Neo-Aramaic}
    \label{Arabic:tab:6}
\end{table}

\begin{sloppypar}
Pre-predicate deictic copulas are a shared feature of Arabic, Aramaic and southeastern Northern Kurdish\il{Kurdish (Northern)!Bahdini} (Bahdini), compared in \tabref{Arabic:tab:7}. Deictic copulas characterized by an initial deictic morpheme \textit{k}- occur across \textit{Qəltu}-Arabic\il{Arabic (Qəltu)} dialects, Ṭuroyo\il{Neo-Aramaic (CNA)!Ṭuroyo} and the NENA dialects of the Mosul Plain, denoting a situation in the immediately observable present or the imminent future. The Bahdini future particle \textit{꞊ê}, derived from \textit{dê} and \textit{wê} -- presumably eroded forms of the 3sg. present form of `want' -- may also have been influenced by preverbal TAM markers in the same region. At the same time, the morpheme attaches to the subject \isi{pronoun} and effectively inflects it for TAM similarly to the pre-predicate \textit{ezafe}-based \isi{copula} and similarly to the pre-predicate deictic copulas in Arabic and Aramaic. For a discussion of the situation in Bahdini Kurdish\il{Kurdish (Northern)!Bahdini}, see \citet[247–249]{Chyet1995NAK} and \citet[405–407]{haig_western_2017}. Since Neo-Aramaic dialects in general have a plethora of deictic copulas, it is conceivable that these deictic copulas spread from Aramaic into \textit{Qəltu}-Arabic -- unless they are a parallel development -- and possibly from Semitic even also into Bahdini varieties of Northern Kurdish\il{Kurdish (Northern)!Bahdini}.
\end{sloppypar}

\begin{table}[t]
\small
\begin{tabularx}{\textwidth}{lllll}
\lsptoprule
& \textbf{Bəḥzani} & \textbf{Qaraqosh} & \textbf{Midyat} & \textbf{Bahdini} \\
& \textbf{\textit{Qəltu}-Arabic} & \textbf{NENA} & \textbf{CNA} & \textbf{Kurdish} \\
& \citep[139]{Jastrow1978MAqetlu1} & \citep{Khan2002Qaraqosh} & & \\
\midrule
Copula-Predicate & \textit{kū fəl-bayt} & \textit{kile b-beθa} & \textit{kəlé bú꞊bayto} & \textit{ew꞊ê li malê} \\ 
\multicolumn{5}{l}{`He is at home.'} \\  
\midrule
Auxiliary-Verb & \textit{kū ∅-yakəl} & \textit{kile k-axəl}\footnote{The verbal forms in Neo-Aramaic are in the realis/indicative, lit. `Here he is, he eats', as opposed to the irrealis/subjunctive in Arabic and Kurdish.} &
\stepcounter{footnote}{-1}
\textit{kəlé k-oxəl}\footnote{The verbal forms in Neo-Aramaic are in the realis/indicative, lit. `Here he is, he eats', as opposed to the irrealis/subjunctive in Arabic and Kurdish.} & \textit{ew꞊ê bi-xwe}\footnote{The future particle \textit{=ê} in Bahdini Kurdish does not inflect for number/gender in contradistinction to the \textit{ezafe} \isi{copula} \textit{=ê} that does inflect for number/gender.}\\
\multicolumn{5}{l}{`He is eating/he is about to eat.' (Arabic/Aramaic).' `He will eat.' (Kurdish)} \\   
\lspbottomrule
\end{tabularx}
\caption{Comparison of deictic copula and Auxiliary-Verb order}
\label{Arabic:tab:7}
\end{table}

\section*{Abbreviations}
\begin{tabularx}{.45\textwidth}{@{}lQ@{}}
1 & first person \\
2 & second person \\
3 & third person \\
A & agent \\
\textsc{add} & additive \\
\textsc{ant} & anterior tense \\
\textsc{cmpr} & comparative \\
\textsc{cop} & {copula} \\
\textsc{cstr} & construct state \\
\textsc{def} & definite article \\
\end{tabularx}%
\begin{tabularx}{.45\textwidth}{@{}lQ@{}}
\textsc{deic} & deictic \\
\textsc{dem} & demonstrative \\
\textsc{dir} & direct case \\
{DOM} & Differential Object Marking \\
\textsc{drct} & directional \\
\textsc{exist} & existential \\
\textsc{ez} & ezafe \\
\textsc{f} & feminine \\
\textsc{fut} & future \\
\end{tabularx}%

\begin{tabularx}{.45\textwidth}{@{}lQ@{}}
\textsc{gen} & genitive \\
\textsc{ind} & indicative \\
\textsc{m} & masculine \\
n & total number of tokens \\
\textsc{neg} & negator \\
O & {object} \\
\textsc{obl} & {oblique} case \\
\textsc{perf} & perfect \\
\textsc{pfv} & perfective \\
\textsc{pl} & plural \\
PP & post-predicate \\
\textsc{pst} & past \\
R & {recipient} \\
\textsc{rel} & relative \\
S & subject (intransitive) \\
\textsc{sbjv} & subjunctive \\
\end{tabularx}%
\begin{tabularx}{.45\textwidth}{@{}lQ@{}}
\textsc{sg} & singular \\
T & theme \\
CNA & Central Neo-Aramaic \\
NENA & Northeastern Neo-Aramaic \\
V & {verb} \\
\textsc{loc} & {locative} \\
Lev. & Levantine \\
\textsc{abl} & {ablative} \\
\textsc{ben} & beneficiary \\
\textsc{addr} & addressee \\
\textsc{instr} & Instrumental \\
WOWA & = \citet{Haig.Stilo.Dogan.Schiborr2022} \\
ZAL & Zeitschrift für Arabische Linguistik \\
\\
\end{tabularx}

\section*{Acknowledgements}

This research was made possible by the generous contribution of the European Research Council. I am grateful to Nikita Kuzin for giving me access to several texts from the Ṭuroyo\il{Neo-Aramaic (CNA)!Ṭuroyo} corpus and to Masoud Mohammadirad and especially Geoffrey Haig for their helpful comments.



\sloppy
\printbibliography[heading=subbibliography,notkeyword=this]

\end{document}
