\documentclass[output=paper,colorlinks,citecolor=brown,collectionchapter]{langscibook}
\ChapterDOI{10.5281/zenodo.14266331}
\author{Geoffrey Haig\orcid{0000-0002-5410-3692}\affiliation{University of Bamberg} and Mohammad Rasekh-Mahand\affiliation{Bu-Ali Sina University Hamedan} and Donald Stilo\affiliation{Max Planck Institute, Leipzig (retired)} and Laurentia Schreiber\orcid{0000-0002-0051-1164}\affiliation{University of Bamberg} and Nils Schiborr\affiliation{University of Bamberg}}
\title[Post-predicate elements in the Western Asian Transition Zone]{Post-predicate elements in the Western Asian Transition Zone: Data, theory, and methods}
\abstract{This chapter spells out the conceptual and methodological foundations for the volume, summarizes previous research, illustrates the methodology and analysis, and presents the results of two case studies. We provide evidence in support of a semantically finer-grained approach to word order that distinguishes between various non-subject constituents, and illustrate how this can be leveraged to detect areal effects in syntax. We implement this approach on a sample of language corpora from 35 languages, including Turkic, Iranian, Semitic, Hellenic, Kartvelian, and Armenian from what we term the Western Asian Transition Zone (WATZ). In a first case study, we demonstrate the existence of a robust Goals Last effect across the entire database and formulate a revised hierarchy for postverbal placement. Our approach identifies the specific properties of spatial goals that distinguish them from metaphorically related roles such as recipient, addressee, and benefactive, which previous studies had conflated. In a second case study, we investigate weight effects on post-verbal placement, concluding that overall, the impact of weight is minimal, a finding reflected in several chapters of the volume. The final section summarizes the contributions to the volume, and the Appendices provide raw data summaries across the entire WOWA data set, and information on sources.}


\IfFileExists{../localcommands.tex}{
   \addbibresource{../localbibliography.bib}
   \addbibresource{../collection_tmp.bib}
   \bibliography{../localbibliography}
   % add all extra packages you need to load to this file

\usepackage{tabularx,multicol}
\usepackage{url}
\urlstyle{same}

\usepackage{listings}
\lstset{basicstyle=\ttfamily,tabsize=2,breaklines=true}

\usepackage{langsci-basic}
\usepackage{langsci-optional}
\usepackage{langsci-lgr}
\usepackage{langsci-osl}
% \usepackage{./langsci/styles/langsci-lgr}
% \usepackage{./langsci/styles/langsci-osl}
% \usepackage{langsci-gb4e}

\usepackage{tikz}
\usetikzlibrary{patterns,calc}
\pgfdeclarepatternformonly{south east lines}{\pgfqpoint{-0pt}{-0pt}}{\pgfqpoint{3pt}{3pt}}{\pgfqpoint{3pt}{3pt}}{
    \pgfsetlinewidth{0.6pt}
    \pgfpathmoveto{\pgfqpoint{0pt}{3pt}}
    \pgfpathlineto{\pgfqpoint{3pt}{0pt}}
    \pgfpathmoveto{\pgfqpoint{.2pt}{-.2pt}}
    \pgfpathlineto{\pgfqpoint{-.2pt}{.2pt}}
    \pgfpathmoveto{\pgfqpoint{3.2pt}{2.8pt}}
    \pgfpathlineto{\pgfqpoint{2.8pt}{3.2pt}}
    \pgfusepath{stroke}}
    
\usepackage{stmaryrd}
\usepackage{wasysym}
\usepackage{multirow}
\usepackage{caption}
\usepackage{subcaption}
\usepackage{mathrsfs}
\usepackage{qtree}

\usepackage{linguex}


   %pminos do not split footnotes
% \interfootnotelinepenalty=10000 %Footnote in Laporte chapters has to be split SN


%\DeclareIndexNameFormat{default}{%
%\nameparts{#1}%
%\usebibmacro{index:name}%
%{\index[names]}%
%{\namepartfamily}%
%{\namepartgiveni}%
% {}% L1
% {}% L2
%{\namepartprefix}% generates spurious space L3
%{\namepartsuffix}% generates spurious space L4
%}

%  {\DeclareIndexNameFormat{default}{%
%     \usebibmacro{index:name}{\index[names]}{#1}{#3}{#5}{#7}}}

%\DeclareIndexNameFormat{default}{%
%  \usebibmacro{index:name}{\sindex[nom]}{#1}{#3}{#5}{#7}}

%\DeclareIndexNameFormat{default}{%
%  \usebibmacro{index:name}{\sindex[person]}{#1}{#3}{#5}{#7}}
%\DeclareIndexNameFormat{default}{%
%\nameparts{#1} \usebibmacro{index:name}{\sindex[person]]}{\namepartfamily}{‌​\namepartgiven}{\nam‌​epartprefix}{\namepa‌​rtsuffix}}

%\newcommand{\smiley}{:)}

%\renewbibmacro*{index:name}[5]{%
%\usebibmacro{index:entry}{#1}%
%{\iffieldundef{usera}{}{\thefield{usera}\actualoperator}\mkbibindexname{#2}{#3}{#4}{#5}}}

% \newcommand{\noop}[1]{}

%remove for final
%\overfullrule=1mm

\newcommand{\tobi}[2]}}
\renewcommand{\S}[1]{\tobi{#1}{\textsc{*}}}

% this volume references
% puts: [this volume]
% already defined: \citetv
%\newcommand{\citepv}[1]{(\citeauthor{#1} \citeyear*{#1} [this volume])}
\newcommand{\citealtv}[1]{\citeauthor{#1} \citeyear*{#1} [this volume]}

%parentheses around example number
\newcommand{\pref}[1]{(\ref{#1})}

% in-text examples

\newcommand{\lnex}[1]{\textit{#1}} %target lang word
\newcommand{\lnlit}[1]{(lit.: `#1')} %literal reading
\newcommand{\lnlat}[1]{(#1)} % latinization
\newcommand{\lntrans}[1]{`#1'} %translation
\newcommand{\lnexl}[2]%
{\lnex{#1}{} \lnlat{#2}} % ex with latinization
\newcommand{\lnexlat}[3]{\lnex{#1}{} \lnlat{#2}{} \lntrans{#3}} % ex with latinization and tranl.

%ch01
\newcommand{\co}[1]{\mbox{\textbf{#1}}}

%ch09

\newcommand{\cyrbulg}[1]{\begin{otherlanguage*}{bulgarian}#1\end{otherlanguage*}}


%ch10
\newcommand{\nlp}{{\small NLP}}
\newcommand{\mwe}{{\small MWE}}
\newcommand{\rae}{{\small RAE}}
\newcommand{\lvc}{{\small LVC}}
\newcommand{\pos}{{\small P}o{\small S}}
%\newcommand{\todo}[1]{ \textcolor{red}{#1} }

%\renewcommand{\labelenumi}{\theenumi}
%\ainamefmt{{vv}{ll}{, ff}{, jj}} % fullname

\newcommand{\biberror}[1]{{\color{red}#1}}

\newcommand{\osenovaitem}{--~}
   %% hyphenation points for line breaks
%% Normally, automatic hyphenation in LaTeX is very good
%% If a word is mis-hyphenated, add it to this file
%%
%% add information to TeX file before \begin{document} with:
%% %% hyphenation points for line breaks
%% Normally, automatic hyphenation in LaTeX is very good
%% If a word is mis-hyphenated, add it to this file
%%
%% add information to TeX file before \begin{document} with:
%% %% hyphenation points for line breaks
%% Normally, automatic hyphenation in LaTeX is very good
%% If a word is mis-hyphenated, add it to this file
%%
%% add information to TeX file before \begin{document} with:
%% \include{localhyphenation}
\hyphenation{
    Beck-man
    Ngu-yen
    back-chan-nel
    back-chan-nels
    mo-not-o-nous
    ste-reo-typ-i-cal
}

\hyphenation{
    Beck-man
    Ngu-yen
    back-chan-nel
    back-chan-nels
    mo-not-o-nous
    ste-reo-typ-i-cal
}

\hyphenation{
    Beck-man
    Ngu-yen
    back-chan-nel
    back-chan-nels
    mo-not-o-nous
    ste-reo-typ-i-cal
}

%    \boolfalse{bookcompile}
%    \togglepaper[1]%%chapternumber
}{}

\begin{document}
\maketitle\label{WOWA:ch:1}
\lehead{G. Haig, M. Rasekh-Mahand, D. Stilo, L. Schreiber \& N. Schiborr}
\section{Theoretical preliminaries}\label{Intro:ss:1}

\subsection{General background}\label{Intro:ss:1.1}

This volume represents the collaborative outcome of a team of researchers, all of whom contributed expertise and data on languages of what we loosely refer to as the Western Asian Transition Zone (WATZ, cf. \sectref{Intro:ss:1.2}). The companion enterprise to this volume is a portfolio of online accessible, multiply-reusable digital resources, comprising of two data-sets: WOWA  (\textit{Word Order in Western Asia}), a multi-lingual corpus containing approximately 40 data sets from a sample of languages across WATZ (\citealt{Haig.Stilo.Dogan.Schiborr2022}), and HamBam (\textit{The Hamedan-Bamberg Corpus of Contemporary spoken Persian\il{Persian (colloquial)}}, \citealt{HaigRasekhMahand2022HamBam}). HamBam is a richly annotated corpus of a single language, colloquial spoken Persian\il{Persian (colloquial)}, based on the Multi-CAST architecture (\citealt{haig_verb-goal_2015,Schnell2023are}).\footnote{For WOWA see \url{https://multicast.aspra.uni-bamberg.de/resources/wowa/}; for HamBam see \url{https://multicast.aspra.uni-bamberg.de/resources/hambam/}.} It is designed for finer-grained investigations of \isi{word order}, \isi{prosody}, and \isi{register} in a single language (Persian), while WOWA is a multi-lingual database designed to investigate the transition-zone phenomena outlined in the following paragraphs. Most of the research reported here is based on WOWA.

% \setlength{\footheight}{59.06021pt}
\largerpage

Our research is intended to satisfy the requirements of ``reproducible research,'' in the spirit of \citet{berez-kroeker_reproducible_2018}, and the \isi{emphasis} on accessibility and accountability of primary data, and maximal transparency of analysis procedures have been guiding principles throughout: research should be conducted in a manner ``which allows readers to confirm claims about language structure through direct access to the original observational data'' \citep[6]{berez-kroeker_reproducible_2018}. In this overview chapter, we introduce and exemplify the data sources, theoretical concepts and research questions, and illustrate the main findings with two case-studies. \figref{Intro:fig:1:doculacts} shows the location of the doculects in WOWA at the time of writing; an overview of all data sources is available in the Appendix to this chapter.

\begin{figure}
    \includegraphics[width=\textwidth]{figures/HaigetalIntroFig1.png}
    \caption{Locations of doculects in WOWA (November 2023)}
    \label{Intro:fig:1:doculacts}
\end{figure}

\subsection{The Western Asian Transition Zone (WATZ)}\label{Intro:ss:1.2}

The concept of ``\isi{transition zone}'' has been discussed in various guises (e.g. ``buffer zone'' \citealt{Stilo2005IranianBuffer}, ``intersection zone'' \citealt{stilo_circumpositions_2009}, or ``typological sandwich'' \citealt{szeto_sinitic_2021}, see \citealt{haig_which_2023}). Here, we continue the terminology introduced in \citet{haig_introduction_2018}; we define a \isi{transition zone} as a geographic area lying at the intersection of two contiguous regions characterized by diametrically opposing values for some linguistic feature. The choice of feature is essentially unrestrained, and determined primarily by the research questions of the investigators, though considerations of general theoretical pertinence, availability of appropriate data, and operationalizability of the feature values concerned, will also play a \isi{role}. It is clear that a \isi{transition zone}, as just defined, can be identified at very different degrees of granularity: even a dialect isogloss, separating related dialects characterized by, for example, the presence versus absence of nasalized vowels, could be considered a \isi{transition zone}, albeit at the micro-level of structural differentiation. 

The Western Asian Transition Zone is at the other extreme of granularity. It is defined by the overlap of two areas of continental scale, which differ with regard to the following features (and a number of ancillary features, which are taken up at various points below): \isi{OV} versus \isi{VO} \isi{word order}, and adpositional type (prepositions vs. postpositions, with some additional complications). An approximation of the global distribution of these two features can be found by considering the two maps in WALS (\citealt{dryer_order_2013_APNP,dryer_order_2013_OV}), Feature 83A (\isi{OV}/\isi{VO}) and Feature 85A (\isi{adposition} order). On both maps, two adjacent macro-areas can be identified within Eurasia and the Indian Sub-Continent, each dominated by languages with opposing feature values: The first is the central and eastern Asian land block, dominated by \citeauthor{robbeets_transeurasian_2017}' (\citeyear{robbeets_transeurasian_2017}) ``Transeurasian languages'' (Turkic, Mongolic, Tungusic, Japonic, Koreanic), which are uniformly characterized by \isi{OV} and postpositions. These features flow seamlessly into the Indian Sub-Continent, dominated by Eastern Iranian, Dravidian\il{Dravidian}, and Indo-Aryan. The second macro-area is to the southwest, where we find the \isi{VO} and prepositional languages of North Africa, Western Europe and the Circum-Mediterranean region (Afro-Asiatic, Romance\il{Romance}, Hellenic\il{Hellenic}). The area of overlap between these two macro-areas is what we refer to as the Western Asian Transition Zone (WATZ). The core of WATZ is the lower catchment areas of the Euphrates and Tigris valleys, springing from the elevated plateau of today's Eastern Turkey and descending to the alluvial plains of northern Iraq and Syria. Today, this region is divided between four nation states, Turkey, Syria, Iraq and Iran. 

Like most of the units that have been suggested in areal linguistics,\footnote{\citet[75]{friedman_reassessing_2017} discuss the issue of defining the boundaries of Sprachbünde, noting ``the boundaries are as elastic as the micro-zones of \isi{convergence} that add up to the larger \isi{convergence} area.'' The boundaries, and even the set of languages and varieties involved, remain disputed even for intensely-researched linguistic areas such as the Balkan Sprachbund.} the geographic boundaries of WATZ cannot be precisely defined. Partly, this is due to the nature of a \isi{transition zone}, which often implies fade-out phenomena \citep{stilo_intersection_2012}, rather than abrupt transitions from one feature value to another. An example of such a fade-out phenomenon is the feature ``order of spatial Goals relative to their governing predicate,'' as in a clause such as \textit{the girl went to the market}. We can distinguish two options: the \isi{Goal} precedes the verb (GV), or it follows the verb (VG). Mapping the frequencies of these two orders across the languages of WATZ reveals an increase in the frequency of Verb-Goal (VG) ordering as one progresses westwards. For example, Iranian languages with long-standing Semitic influence, from the southern and westernmost peripheries of WATZ, show almost 100\% post-verbal Goal\is{Goal!post-verbal}s (in the sense of ``Goals of movement''; see below on terminology), matching the figure that characterizes most of Semitic, and of the westernmost branches of Indo-European. Similarly, Turkic\il{Turkic} languages such as Qashqai\il{Turkic!Qashqai}, which are under heavy influence from western Iranian languages, also show high rates (>60\%) of post-verbal Goal\is{Goal!post-verbal}s (\citealt{haig_which_2023}). Note that both Iranian and Turkic\il{Turkic} are generally considered to be \isi{OV} (sometimes erroneously equated with ``verb final''), so the high frequency of post-verbal Goal\is{Goal!post-verbal}s is not expected for these languages. For \isi{OV} languages situated further northward and eastward, the figures drop, for example in the Balochi\il{Balochi!Turkmen} variety of Turkmenistan (\citetv{chapters/4_NourzaeiHaig_Balochi}), or Mazandarani\il{Mazandarani} of the Caspian region \citep{stilo_mazandarani_2022}. Furthermore, initial counts from related \isi{OV} languages further eastward beyond WATZ suggest they drop still further; in Indo-Iranian (Dardic) Kalasha\il{Indic!Kalasha} (Northern Pakistan, bordering Afghanistan), provisional counts of the texts in \citet{petersen_kalasha_2015} suggest rates of post-verbal Goal\is{Goal!post-verbal}s below 30\%, while in the texts from Dukhan, a Turkic\il{Turkic!Dukhan} variety from Northern Mongolia, the figure approaches zero. If the basic hypothesis behind WATZ is correct, then we might predict similar low values of VG for unrelated \isi{OV} languages toward the eastern fringe of Asia, such as spoken Japanese\il{Japanese} and Korean\il{Korean}, but this remains to be tested.

Although our current data coverage is low-density and uneven, we tentatively hypothesize that in the \isi{OV} languages of Asia, increasing rates of post-verbal Goal\is{Goal!post-verbal}s roughly correlate with increasing proximity to the Mesopotamian core of WATZ.\footnote{It is worth noting that areality alone does not fully account for the findings; the phylogenies of the languages concerned are also relevant, with the Iranian languages apparently the most prone to areally-induced word-order variation; see \citet{haig_which_2023}, \citet[42]{bickel_areas_2017} on the interplay of areality, inheritance, universal principles in co-determining language structure, and \citet{haig_goals_2023}, for arguments in favour of a universal Goals Last principle.} But we can identify no precise geographical isogloss that constitutes a categorical border separating VG from GV. Rather, we are dealing with a continuum of values, which we assume extends beyond the region defined by the sample in \figref{Intro:fig:1:doculacts}.

Returning to the broader theoretical interest of transition zones, it has been suggested that regions of intense contact (a hallmark of transition zones) are havens for typologically rare constructions. \citet[137]{harris_historical_1995} note that certain word-order constellations only emerge in contact situations, and our data lend some credence to this view.  Furthermore, transition zones are overall smaller than the macro-areas that engender them, and are therefore likely to contain fewer languages. The probability that any random language sample includes languages from transition zones is thus lower than the probability of selecting languages from within established linguistic areas. To this extent, the contributions to this volume are thus intended to counterbalance an existing bias in language sampling. To conclude, transition zones are not definable in terms of a precisely circumscribed geographic region. Rather, they should be seen as hypotheses which demarcate a potentially fruitful set of languages located at a region of conflicting feature values, which can serve as an experimental setting for investigating the broader question: what happens when languages with opposing feature values collide? 

\subsection{Word order}\label{Intro:ss:1.3}

The term ``\isi{word order}'' is often taken as synonymous for the traditional Greenberg'ian six-way typology (S/V/O). In our work, however, we follow \citet{dryer_six-way_1997,dryer_against_2013}, who proposes decomposing the traditional six-way typology into two binary sub-features, S/V and V/O. Here we \isi{focus} almost exclusively on the relative ordering of \isi{direct object} and verb (V/O), while setting aside the position of subjects (S); see below for empirical justification, and \citet{dryer_six-way_1997,dryer_against_2013} for the arguments against the six-way typology. In line with recent work in corpus-based typology (see \sectref{Intro:ss:1.4}), we extend the typology to include the position of other verbal arguments, such as Recipients, Locatives, copula complement\is{copula!complement}s, Goals, Addressees, relative to the verb. Our motivation for this is entirely data-driven: a large body of research (see \sectref{Intro:ss:2}) on the languages of Western Asia suggests that beyond direct objects, other less prominent and often overlooked constituent types provide sensitive indicators for \isi{contact influence} (\citealt{haig_which_2023}). Furthermore, the default assumption that the position of direct objects (i.e. \isi{OV} vs. \isi{VO}) can be generalized across other verbal arguments, is falsified in many languages of the region, which exhibit consistent \isi{OV} order, but simultaneously have post-verbal placement of certain non-direct objects; see \sectref{Intro:ss:4} below for an overview of the relevant findings from WOWA. Word order typology has tended to either ignore non-direct objects, or to subsume them under umbrella terms (e.g. `obliques,' \citealt{hawkins_asymmetry_2008}, \citealt{levshina_token-based_2019,jing_word_2021}; or `PP' in \citealt{frommer_post-verbal_1981}). Our research indicates that a finer-grained semantic approach to non-direct-objects is more appropriate, which we spell out in \sectref{Intro:ss:4} below.

A perennial debate in \isi{word order} typology concerns how, or indeed whether, one can identify some kind of ``basic \isi{word order}'' for a language. There are a number of issues at stake, which we will address here. First, although scholars such as \citet{mithun1992basic} have argued against the universality of basic \isi{word order}, it is important to note that most researchers engaged in \isi{word order} typology (see e.g. \citealt{dryer_word_2007} and \citealt{song2018linguistic} for summaries) have never claimed that every language has a ``basic \isi{word order}.'' Thus, the global overview of the \isi{VO} vs. \isi{OV} word-order parameter in \citep{dryer_order_2013_OV} assigns all languages in the sample to one of three types: \isi{OV}, \isi{VO}, and ``no dominant order,'' with the latter comprising some 7\% of the 1,518 languages in \citegen{dryer_order_2013_OV} sample. It has always been acknowledged that it is not possible to identify a single order as ``basic'' for every single language. But that does not invalidate the enterprise of \isi{word order} typology as a whole, any more than the fact that some languages do not have lexical tone invalidates a cross-linguistic approach to tone systems. 

At least three different approaches to basic \isi{word order} can be identified. First, the frequency of usage. Contrary to what is often claimed, a frequency approach does not necessarily imply a simple majority decision. \citet[11]{dryer_word_2007} proposes that the basic (or dominant) order is that variant which is at least twice as frequent as the next most frequent order. In the case of a binary feature such as \isi{OV} vs. \isi{VO}, that would mean one order would account for at least 66\% of the relevant cases in order to count as the basic order. The frequency approach raises a host of methodological issues related to corpus size and representativity and ignores many finer nuances, some of which we take up below.

A second approach seeks to define a particular constructional sub-type which is taken as prototypical for the construction under consideration. For example, \citet{siewierska1988word} provides a widely-cited rule-of-thumb for identifying the basic S/V/O order cross-linguistically. According to her, the basic order is that which:

\begin{quote}
    [...] occurs in stylistically neutral, independent, indicative clauses with full noun phrase (NP) participants, where the subject is definite, agentive and human, the \isi{object} is a definite semantic patient, and the verb represents an action, not a state or an event \citep[8]{siewierska1988word}.
\end{quote}


\begin{sloppypar}
This would rule out, for example, interrogatives, subordinate clauses, or clauses in which either S or O is pronominal, and it would also exclude transitive clauses with verbs like `see' or `know,' which express states rather than actions and do not involve a `semantic patient.' In principle, this is a reasonable approach and has gained some currency. However, we see no obvious justification for ruling out transitive clauses with indefinite direct objects, particularly as overall, direct objects are among the most likely \isi{argument} types to host indefinite referents (\citealt{haig_universals_2021}: 164, \citealt{Schnell2023are}).
\end{sloppypar}

Another variant of the prototype approach is implemented by \citet{bagriacik_pharasiot_2018} and \citet{neocleous_word_2020}, both investigating varieties of Asia Minor Greek.\footnote{Although both authors work within a Minimalist framework, thus approach ``\isi{word order}'' from a rather different perspective to the typologically-oriented approach of \citet{siewierska1988word}, they nevertheless attempt to define a ``basic'' clause type on the basis of certain pragmatic and surface morphosyntactic properties, from which other orders are derived.} These authors \isi{focus} on what is variously termed ``pragmatically neutral'' or ``unmarked'' transitive clauses, in which neither \isi{argument} ``is associated with either a \isi{topic}, or a \isi{focus} reading'' \citep[150]{bagriacik_pharasiot_2018}. Such clauses only occur under very specific discourse conditions, three of which are identified by Bağrıaçık as follows: (i) answer to an all \isi{focus} question (\textit{What happened?}); (ii) introductory clauses of narratives, where both subject and \isi{object} are newly introduced into the world of discourse; (iii) generic statements (\textit{the Earth orbits around the Sun, Sam knows Tibetan}), \citep[151--154]{bagriacik_pharasiot_2018}. \citet{neocleous_word_2020} adopts a similar approach, though with a different formalization of the concept of ``pragmatically unmarked.'' Note that \citet{bagriacik_pharasiot_2018} concept of pragmatically neutral transitive clause is different from \citegen{siewierska1988word}. Siewierska stipulates that in a basic transitive clause, both subject and \isi{object} are required to be definite, while Bağrıaçık's ``introductory clauses,'' for example, imply that neither would be definite.

Along with prototype, and frequency-based approaches, there are two other connotations of ``basic \isi{word order}.'' The first is the concept of inherited, or historical \isi{word order}. By this, we mean the \isi{word order} that is reconstructible for the assumed proto-language of the languages under investigation. For example, there is little reason to doubt that both Turkic\il{Turkic} and Iranian languages had \isi{OV} \isi{word order} at the oldest period of attestation, or that proto-Semitic was \isi{VO}, so these languages can be reasonably classified as historically \isi{OV} and \isi{VO} respectively. However, claiming that Semitic is historically \isi{VO} does not equate to a claim that this is the ``basic order'' for all modern Semitic languages; basic order can change.\footnote{It may seem superfluous to labour this point, but it is nevertheless misunderstood in \citet[42]{asadpour_typologizing_2022}, who interprets references in the literature to ``historical'' and ``inherited'' \isi{word order} in Neo-Aramaic as claims regarding ``basic \isi{word order}'' in contemporary Neo-Aramaic. These are clearly separate claims.}

Finally, a basic \isi{word order} may be motivated by theory-internal considerations. This is the case for German\il{German}, for which the basic \isi{word order} is often claimed to be \isi{OV}, with \isi{VO} considered as secondarily derived via (some version of) verb movement. If we strictly applied the criteria of \citet{bagriacik_pharasiot_2018}, which relies on the concept of pragmatically neutral clause, or \citet[116]{neocleous_word_2020}, which invokes (among other things) \isi{word order} in main declarative clauses, we would obtain a different result for German\il{German}, because \isi{OV} order in German is actually found in subordinate clauses. The fact that different criteria yield different results is reflected in the classification of German\il{German} as ``no dominant order'' in \citet{dryer_order_2013_OV}.

Having briefly considered various interpretations of ``basic \isi{word order},'' we turn to a methodological issue in connection with \isi{word order} and small corpora. Both \citet{siewierska1988word} and \citet{bagriacik_pharasiot_2018} require a transitive clause to have two nominal (as opposed to pronominal) arguments. However, for research based on small corpora of spoken language, often without recourse to native speakers' judgements, this approach runs into an immediate problem. Cross-linguistically, in natural discourse, very few transitive clauses contain two overt lexical (as opposed to pronominal) arguments. \citet[62--63]{du_bois_argument_2003} provides data from five spoken language corpora indicating that the overall frequency of clauses with two lexical core argument\is{argument!core}s lies between two and seven percent, and similar findings are reported in the literature (see contributions in \citealt{du_bois_argument_2003}). Additional restrictions, such as requiring both S and O to be “pragmatically neutral” \citep{bagriacik_pharasiot_2018}, or both to be “definite” \citep{siewierska1988word}, would further reduce the pool of valid tokens. It is no surprise that \citet{bagriacik_pharasiot_2018} investigation of \isi{word order} in Pharasiot Greek\il{Hellenic!Pharasiot} is largely informed by the elicitation of grammaticality judgements, rather than a quantitative analysis of naturalistic data.

Cross-linguistic research of spontaneous spoken discourse demonstrates that transitive subjects overwhelmingly express given information (>90\%, \citealt[165]{haig_universals_2021}), and are consequently predominantly either zero, or pronominal in form, rather than lexical NPs (cf. \citeauthor{du_bois_discourse_1987}' \citeyear{du_bois_discourse_1987} `Avoid Lexical A'-constraint; see discussion in \citealt{haig_discourse_2016}). If we were obliged to exclude all clauses with pronominal or zero subjects, we would vastly decrease the number of potentially analysable tokens in the sample. For objects, however, the likelihood of lexical NP expression is very much higher, and the population of analysable tokens correspondingly larger. This is a further motivation to eschew the six-way SVO-typology, and to \isi{focus} on the relative ordering of verb and \isi{direct object} only. In the future, we may expand our investigation to include the position of subjects in the WOWA data, which are fully accessible and amenable for additional coding of subjects (see also \citealt{Molin2022NEDohok}, \citetv{chapters/7_RasekhMahandetal_Persian}, \citetv{chapters/11_Forker_Adyghe,chapters/10_Forker_EC}). Currently, however, we continue to work with the binary feature \isi{OV}/\isi{VO}.

\subsection{Corpus-based approaches to word order}\label{Intro:ss:1.4}

Methodologically, we apply a corpus-based typological approach (\citealt{walchli_data_2009,levshina_token-based_2019,futrell_dependency_2020,gerdes_typometrics_2021,haig_universals_2021,SchnellSchiborr2022Cross}, among many others). Within such an approach, the \isi{emphasis} shifts away from assigning a `basic \isi{word order}' to a particular ``language.'' Rather, in corpus-based approaches, statements on \isi{word order} refer to frequency distributions derived from actual corpora, and are probabilistic in nature. Strictly speaking, corpus-based approaches to \isi{word order} yield a characterization of a specific corpus (a ``doculect''), rather than ``a language,'' though we continue to use the over-simplified terminology here. 

Corpus-based typological approaches to word-order typology are dominated by research on large, written corpora of languages with a pre-existing orthographic norm, for which copious quantities of pre-digitalized text are available (see in particular Universal Dependencies (UD)-consortium, \citealt{nivre_universal_2020}).\footnote{\url{https://universaldependencies.org/}} Consequently, there is a bias towards standardized (and mostly Eurasian) languages, and more importantly, towards written language. Cross-linguistic research based on spoken language corpora, on the other hand, is still in its infancy \citep[see among others][]{Schnelletal2021documentation,mettouchi2021prosodic,SchnellSchiborr2022Cross,seifart2022doreco,levshina_why_2023}. \citegen{frommer_post-verbal_1981} research had already demonstrated significant differences between formal written and spoken Persian\il{Persian (colloquial)}, indicating that written language data cannot be assumed to reliably reflect the structures of spoken language. Any research agenda that purports to investigate the impacts of processing and production constraints on language structure would be well advised to \isi{focus} on the mode of language production where these constraints are operative in real time --- and that would not be written language (see \citealt{Schnelletal2021documentation} for summary arguments). In \sectref{Intro:ss:5}, we present an investigation into the \isi{role} of \isi{weight} as a predictor for \isi{word order}, based on our spoken language corpora, which illustrates the importance of controlling for modality.

As mentioned above, we \isi{focus} on various non-subject constituents, and their position relative to the governing predicate, for example \isi{direct object} and verb, or Addressee and verb, and so on. Our database is thus designed to answer the following questions: 

\ea
\ea\label{Intro:ex:1a}
What is the probability that non-subject \isi{argument} A, in doculect X, belonging to language family Y, spoken at location Z, occurs after its governing predicate?
\ex\label{Intro:ex:1b}
Which variables influence this probability?
\z
\z

From the answers to \ref{Intro:ex:1a} we could infer word-order ``types,'' by setting some pre-defined quantitative boundaries. For example, \citet[559]{levshina_token-based_2019} provisionally classifies a language (=corpus) with greater than 80\% \isi{VO} as ``\isi{VO},'' with less than 20\% \isi{VO} as ``\isi{OV},'' and 20--80\% as ``mixed.'' However, this is a matter of heuristic interpretation of the raw data, rather than principled classification of ``Type.'' It should be obvious that two doculects with values of 79\% and 81\% \isi{VO} respectively are not necessarily exemplars of fundamentally distinct types (see \citealt{walchli_data_2009} on ``data reduction typology''). The variables that were tested for question \ref{Intro:ex:1b} are presented in \sectref{Intro:ss:3.2}. Questions \ref{Intro:ex:1a}-\ref{Intro:ex:1b}; they can be investigated both at the level of individual doculects, or by applying appropriate statistical methods to the entire sample or some sub-set thereof. In Sections \ref{Intro:ss:4} and  \ref{Intro:ss:5} below we present case studies for the impact of semantic \isi{role} (\sectref{Intro:ss:4}), and of \isi{weight} (Section  \ref{Intro:ss:5}). Having outlined the theoretical background and the research questions, we illustrate the structure of the data sets and the methodology in the following sections.

\section{Previous research on Western Asia, and terminological issues}\label{Intro:ss:2}

\begin{sloppypar}
The assumptions and aims of the WOWA project were inspired by insights gained over many years of previous research, and it is appropriate to briefly outline the main currents of that research. Typologists have long been aware of the ``mixed typology'' of Iranian languages, e.g. \citet[19]{comrie_language_1989} on Persian\il{Persian}, a language with \isi{OV} in the VP, but \isi{head-initial} NPs, prepositions, and clause-initial complementizers (see \citealt{dabir-moghaddam_typological_2018} for a recent summary). Don Stilo developed the idea that the mixed typology of Persian\il{Persian} was shared to differing degrees by other West Iranian languages, and that the degree and nature of West Iranian mixed typologies followed an approximate areal distribution. Stilo's claim was that West Iranian was sandwiched between the opposing typologies of Semitic (consistently \isi{head-initial}) and Turkic\il{Turkic} (\isi{head-final}), with different West Iranian languages synchronizing with the profile of their respective geographic neighbours. These ideas were fleshed out with a survey of adpositional types in \citet{Stilo2005IranianBuffer,stilo2006circumpositions,stilo_circumpositions_2009}, and developed in a number of other publications (\citeyear{stilo_intersection_2012,stilo_preverbal_2018,stilo_investigating_2018}).
\end{sloppypar}


\citet{frommer_post-verbal_1981} noted a further non-harmonic aspect of West Iranian syntax: the post-verbal positioning of certain kinds of non-direct-\isi{object} arguments. \citet{frommer_post-verbal_1981} focussed on the syntax of ``informal Persian\il{Persian}'' (IP), including both spoken and written samples from different registers. This was, in fact, the crucial breakthrough: formal written Persian\il{Persian}, the more usual \isi{object} of study, is rather consistently ``verb final,'' hence post-verbal elements are a fringe phenomenon that had not been systematically investigated.\footnote{\citet{Lazard1957Persian} had already noted the predominance of post-verbal Goal\is{Goal!post-verbal}s in informal spoken Persian\il{Persian (colloquial)}, but did not systematically investigate the \isi{topic}.} \citet{frommer_post-verbal_1981} was the first systematic analysis of post-predicate elements in different registers of informal Persian\il{Persian};\footnote{Frommer did not actually investigate formal written Persian, and we still lack a systematic study. Parizadeh's (\citetv{chapters/8_Parizadeh_ENP}) study of Early Classical New Persian\il{Persian (Early New)} (11--14th Century CE) demonstrates near 100\% verb finality in these written texts, which matches the native speaker intuition of one of our authors regarding contemporary formal written (e.g. academic prose) Persian\il{Persian}. More recent corpus-based approaches to written Persian\il{Persian} (e.g. \citealt{Faghirietal2018Canonical}) investigate the relative ordering of pre-verbal constituents, while post-verbal constituents lie outside the purview of this research. Formal written Persian\il{Persian} is thus essentially considered to be a ``verb final'' language.} his findings can be summed up as follows: (i) across the different registers of informal Persian\il{Persian}, there is a cline of formality such that a lower degree of formality correlates with an increase in post-predicate elements; in-group domestic conversational Persian\il{Persian (colloquial)} exhibited the highest levels; (ii) semantic \isi{role} is crucial, with goals of motion (``destinations'' in Frommer's terminology) as the leaders in post-verbal placement, across all registers; (iii) information status is relevant for post-verbal placement (focal versus non-focal) of direct objects, but appears to be irrelevant for goals of motion; (iv) there is a stronger tendency for post-verbal constituents to lack overt \isi{flagging}. By and large, these findings have been confirmed on more recent corpora of spoken Persian\il{Persian (colloquial)} (\citetv{chapters/7_RasekhMahandetal_Persian}).

From a comparative Iranian perspective, the remarkable aspect of Frommer's findings is that they closely align with findings from the lesser-researched and generally non-standardized West Iranian languages documented in this volume and elsewhere. What this suggests is that the phenomena which Frommer identified were not merely irregularities specific to informal spoken Persian\il{Persian (colloquial)}, but in fact reflected \isi{word order} traits of considerable antiquity, which characterize most (perhaps all) West Iranian languages (see \citetv{chapters/4_NourzaeiHaig_Balochi,chapters/5_Korn_Bashkardi,chapters/6_Nourzaei_Kholosi,chapters/7_RasekhMahandetal_Persian,chapters/8_Parizadeh_ENP,chapters/9_Mohammadirad_Gorani}). From this perspective, it is the strictly verb-final, formal written Persian\il{Persian} that is the exception when it comes to West Iranian \isi{word order}. This has considerable implications for the diachronic study of \isi{word order}, which is largely reliant on written language sources.

Research on post-predicate elements in other Iranian languages began with Kurdish\il{Kurdish} \citep{haig_vg-word_2011,haig_post-predicate_2014,Haig2022PostPredicateCon}, and has since expanded to neighbouring languages (\citealt{haig_verb-goal_2015,haig_western_2017,stilo_investigating_2018,Jahani2018Post-verbal,asadpour_typologizing_2022}, and contributions to this volume). Most of this research is based on corpora of spoken narrative texts (see below), though increasingly enhanced with experimental data (see \citetv{chapters/3_Skopeteas_Prosody}). It has emerged that while all West Iranian languages investigated to date (with the exception of Kumzari, \citealt{anonby_grammar_2015}) are consistently \isi{OV} (see below), like spoken Persian\il{Persian (colloquial)}, they are not ``verb final'' because a significant number of non-\isi{direct object} arguments regularly follow the verb. A similar pattern can also be observed in Turkic\il{Turkic} languages in contact with Iranian \citep{schreiber_contact-induced_2021,stilo_oghuz_2021}. A second point that quickly emerged from the earlier studies is that the nature, and systematicity, of post-verbal arguments follows an approximate areal distribution, along the lines of \citegen{stilo_circumpositions_2009} suggestions. Among Iranian \isi{OV} languages, the highest frequencies, and greatest variety of post-verbal \isi{argument} types, are attested among varieties of northern Kurdish\il{Kurdish (Northern)} spoken in Iraqi Kurdistan and adjacent regions of southeastern Turkey, Syria and Iran, a region we provisionally refer to as Mesopotamia. Mesopotamia is of course also home to a number of historically \isi{VO} Semitic languages, (Neo-)Aramaic and Arabic, which have co-existed with Kurdish\il{Kurdish} and other \isi{OV} languages for centuries, and indeed for millennia in the case of Aramaic. In these \isi{VO} languages, it is universally the case that other non-\isi{direct object} arguments also follow the verb, and it seems plausible to assume that the syntax of these languages had some impact on the Iranian languages with which they shared territory for at least 2000 years, and ultimately also on Turkic\il{Turkic} (perhaps via Iranian in many cases).

In a pilot study, \citet{haig_post-predicate_2014} compared \isi{word order} in naturalistic texts from a sample of languages mostly from Mesopotamia.\footnote{The varieties were Northern Kurdish\il{Kurdish (Northern)} from Iraqi Kurdistan, Northern Kurdish\il{Kurdish (Northern)!Midyat} from Midyat, Southeastern Turkey, Northern Kurdish\il{Kurdish (Northern)!Muš}\il{Kurdish (Northern)!Erzurum} from Muš and Erzurum; Northeastern Neo-Aramaic (Jewish) from Urmi\il{Neo-Aramaic (NENA)!J. Urmi}, West Iran, and from Koy Sanjaq\il{Neo-Aramaic (NENA)!J. Koy Sanjaq}, Iraqi Kurdistan, and Turkish\il{Turkic!Turkish Erzincan} from Erzincan, Turkey. For comparison, they also included corpus data from a dominant \isi{VO} language, Cypriot Greek. It was already apparent from this small data set that outside of Mesopotamia, Recipients, Addressees, and Goals of motion do not necessarily pattern alike.} The authors identified four types of arguments that are predominantly post-verbal in these languages, cited in the original formulation as follows:

\begin{itemize}
    \item Recipients of verbs of transferred possession (e.g. GIVE)
    \item Destination or direction of verbs of movement (e.g. GO, RUN, FALL)
    \item Destination or direction of verbs of caused motion \\ (e.g. PUT, PLACE, TAKE)
    \item Addressees of verbs of speech (e.g. SAY, SPEAK, PROMISE)
\end{itemize}


Examples illustrating these four types, from Badini Kurdish\il{Kurdish (Northern)!Gullī}\il{Kurdish (Northern)!Akre} (from the Gullī and Akre dialects of Iraqi Kurdistan, from \citealt{haig_post-predicate_2014}, citing \citealt{mackenzie_kurdish_1962}) are provided in (\ref{Intro:ex:2}-\ref{Intro:ex:5}):

\ea\label{Intro:ex:2} 
Recipient\\
Northern Kurdish Akre \il{Kurdish (Northern)!Akre}\citep{mackenzie_kurdish_1962} \\
\gll min kič-ā xo dā \textbf{ta} \\
\textsc{1sg.obl} daughter\textsc{-ez.f} \textsc{refl} give\textsc{.pst.3sg} \textsc{2sg.obl} \\
\glt `I have given my daughter \textbf{to you}.'
\z

\ea\label{Intro:ex:3} 
Addressee\\
Northern Kurdish Akre \il{Kurdish (Northern)!Akre}\citep{mackenzie_kurdish_1962} \\
\gll sultān-ī got-a \textbf{ahmad} \textbf{halwāčī} \\
Sultan\textsc{-obl.m} say\textsc{.pst.3sg-drct} Ahmad Halwachî \\
\glt `the Sultan said \textbf{to Ahmad Halwachî}.'
\z

\ea\label{Intro:ex:4} 
Goal of simple motion \\
Northern Kurdish Akre \il{Kurdish (Northern)!Akre}\citep{mackenzie_kurdish_1962} \\
\gll harduk rābon, hāt-in-a \textbf{bāžar-ī}  \\
both get\_up\textsc{.pst.3pl} come\textsc{.pst-3pl-drct} town\textsc{-m.obl} \\
\glt `Both of them got up and came \textbf{to the town}.'
\z

\ea\label{Intro:ex:5} 
Goal of caused motion  \\
Northern Kurdish Gullī \il{Kurdish (Northern)!Gullī}\citep{mackenzie_kurdish_1962} \\
\gll kir \textbf{t=sēnīk-ā} \textbf{dayk-ā} \textbf{xwa} \textbf{dā} \\
do\textsc{.pst.3sg} \textsc{adp=}tray\textsc{-f.ez} mother\textsc{-f.ez} \textsc{refl} \textsc{adp} \\
\glt `(He) put (it) \textbf{on his mother's tray}.'
\z

In the Mesopotamian languages investigated in \citet{haig_post-predicate_2014}, all four semantic types exhibited broadly similar rates of post-verbal placement, which motivated the authors to define a macro-\isi{role}, labeled ``Goal,'' that would encompass all four types (and some further types such as final state of a change-of-state verb, see below). In retrospect, this terminological decision proved injudicious, for two reasons. First, it introduced ambiguity to the term ``Goal,'' which could either be understood in the narrower sense of ``Goal of verb of motion,'' or in the broader sense that would include Recipient, Addressee, etc. Second, it has become increasingly evident that many languages of WATZ do not lump Addressees, Recipients, and Goals of verbs of motion together (\sectref{Intro:ss:4} below for data), thus casting doubt on the validity of a macro-category altogether. A broadly similar macro-category was subsequently adopted by \citet{asadpour_typologizing_2022,asadpour_word_2022}, who relabels it as ``Target,'' and this terminology has been used in the contributions to \citet{asadpour_Jugel_2022}. While the re-labeling alleviates the ambiguity problem, it does not resolve the empirical problem that outside of some varieties of Kurdish\il{Kurdish} with deep historical ties to Semitic languages, Addressees, Recipients, and spatial Goals do not pattern alike among the \isi{OV} languages of WATZ, so the motivation for assuming a priori a meta-category is questionable.\footnote{There is also a lack of consensus about the nature and number of categories that are included under ``Target''; some researchers include variously Benefactives, and Final States of change-of-state predicates, rendering comparison across different publications difficult.} In an effort to restore clarity, we therefore eschew the macro-category sense of ``Goal'' in this volume, reserving the term ``Goal'' strictly in the sense of ``Goal or \isi{endpoint} of a predicate of motion or caused motion.'' See end of \sectref{Intro:ss:4.1} for discussion of ``Recipient'' vs. ``Goal,'' and \REF{Intro:extab:3} for an overview of roles distinguished in WOWA.

One of the observations in the earlier literature concerned the syntax of `final state' constituents, defined here as expressions indicating the final state of a change-of-state (`become') predicate. \citet{haig_western_2017,Haig2022PostPredicateCon} noted that in much of Kurdish\il{Kurdish}, the final states are significantly more likely to be post-verbal than the complements of copular expressions that do not imply a change of state, even when the lexical verb is the same;\footnote{Interestingly, post-verbal placement of a change-of-state \isi{complement} is much more likely when the \isi{complement} is nominal (e.g. `she became (a) teacher'), rather than adjectival (e.g. `she became rich').} compare (6) (change-of-state) and (7) (static state).

\ea\label{Intro:ex:6} 
Central Kurdish Sanandaj \il{Kurdish (Central)!Sanandaj}\citep[I, 1016]{mohammadirad_Sanandaj_Kurdish_2022} \\
\gll bū-m=a \textbf{wirdafirūš} \\
be\textsc{.subj-1sg=drct} peddler \\
\glt `(I will) become (a) \textbf{peddler}.'
\z

\ea\label{Intro:ex:7} 
Southern Kurdish Bijar \il{Kurdish (Southern)!Bijar}\citep[D, 0282]{mohammadirad_Bijar_Kurdish_2022} \\
\gll aware \textbf{kur} bī ... \\
if boy be\textsc{.subj.3sg} \\
\glt `If it were \textbf{a boy} ...'
\z

Similar phenomena have been noted for unrelated \isi{OV} languages in close contact with Kurdish\il{Kurdish}. For example, in the Northeastern Neo-Aramaic (NENA) dialect of the Jewish speech community from Urmi\il{Neo-Aramaic (NENA)!J. Urmi} (West Iran), ``the \isi{complement} of the verb \textit{qlb} `turn into' is invariably placed after it'' \citep[323]{Khan2008JUrmi}. Although post-posing of complements of change-of-state predicates is widespread in the region, it is not grammaticalized to the same extent in all languages. Having outlined some of the main currents in earlier research and clarified terminology, in the remaining sections, we describe the design of the database and present two case studies illustrating cross-corpus results.

\section{Design of the WOWA (Word Order in Western Asia) database}\label{Intro:ss:3}

The WOWA sample includes data sets from 35 languages and varieties, based on monological, unscripted, spoken texts (see \figref{Intro:fig:1:doculacts} and the Appendices for details). As most of the project was conducted during the 2020--2022 pandemic, it was not possible to systematically select locations and languages in which to conduct dedicated fieldwork; rather, we have been obliged to rely on pre-existing resources. The result is that we were unable to compile a geographically or phylogenetically balanced sample of varieties. Nevertheless, the present sample represents the largest and most systematic data source currently available for investigating \isi{word order} across the region.

The data sets stem from a range of distinct research contexts. Some are based on texts extracted from the published output of scholars working within individual philologies (e.g. the Neo-Aramaic texts of Barwar\il{Neo-Aramaic (NENA)!C. Barwar}, Northern Iraq, originally published in \citealt{Khan2008Barwar}, a sub-set of which is analysed for a WOWA data set, \citealt{stilo_oghuz_2021}). Other data-sets are taken from published sources of national language academies in the framework of dialect surveys (e.g. the Erzurum dialect of Turkish\il{Turkic!Turkish Erzurum}, which feed into \citealt{dogan_oghuz_2021}), while others stem from contemporary language documentation projects, such as the Hazarrudi Tat\il{Tat!Hazarrudi} texts used in \citet{izadifar_tati_2022} and the Qashqai\il{Turkic!Qashqai} texts in \citet{schreiber_oghuz_2021}.

For most data sets, the most widely represented genre is traditional narrative, but some data sets also include stimulus-based narratives (e.g. Pear story \citep{chafe_pear_1980} retellings). The texts have been transcribed according to the academic tradition of the original researcher (we have not attempted to impose a common transcription scheme), and translated into English (in one case, into German\il{German}). Generally, each data set includes more than one text, in most cases from different speakers; the composition of each data set is described in the accompanying metadata, and the source of each token (i.e. the individual text, and speaker) is recoverable. The main criteria for inclusion of a dataset in WOWA are a minimum yield of 500 codable tokens, reliable and authentic spoken data, and no restrictions on data accessibility.

\subsection{Content of each data set}\label{Intro:ss:3.1}

The list below provides the downloadable resource types included in WOWA. The first three are available for all data sets, while the other three are accessible to varying degrees, depending on the nature of the source data:

\begin{enumerate}
\item All files: Complete data set in a single ZIP-directory.
\item Coded values: The actual coded data (see below), in Excel and TSV format.
\item Metadata: A text document containing information on sources, references, speaker metadata, links, and other relevant information.
\newpage
\item Source texts: Contains an orthographic rendering of the entire text with a translation, often from a published source, or provided by the contributor. In some cases, the source texts include additional morphological glossing or other information.
\item Sound files: The original sound files (where available), in .WAV and .MP3 formats.
\end{enumerate}

\subsection{Segmentation and token coding}\label{Intro:ss:3.2}

The basic units of the database are tokens of prosodically independent (rather than bound), referential, non-subject constituents. Creating the database thus involves identifying the relevant tokens, and coding them for a series of features (see below). Texts selected for inclusion into the corpus are first segmented into strings that correspond approximately to meaningful utterances (in many cases this corresponds to a clause), termed utterance units. Each utterance unit is accompanied by a translation into English (column ``utterance\_translation'' in \tabref{Intro:tab:1} below). Utterance units are consecutively numbered and entered as single rows in the database, initially implemented in an Excel spreadsheet.

In a second step, all tokens of referential, non-subject constituents are identified and entered into a distinct cell (token) aligned with its source utterance unit. Note that clausal constituents (\isi{complement} clauses, etc.) are not included as tokens. If an utterance unit contains more than one relevant token, that row of the data is repeated. If an utterance unit contains no relevant token (for example, a simple intransitive clause often does not contain any overt non-subject constituent, see 0006 in \tabref{Intro:tab:1} below), then the token column remains empty. Note that these non-coded utterances remain in the data set, which thus preserves the overall unity of the original text, and maintains its re-usability for future research.

\begin{table}
\small
    \begin{tabularx}{\textwidth}{rQQll}
\lsptoprule
token {ID} & utterance unit & utterance translation & token & token\_translation \\
\midrule
0001 & \textit{bale čemā rustā de i nefar ve} & yes, there was a person in our village & \textit{čemā rustā de} & in our village \\
\tablevspace
0002 & \textit{šekārči ve} & he was a hunter & \textit{šekārči} & hunter \\
\tablevspace
0003 & \textit{ševi šekār} & he had gone for hunting & \textit{šekār} & hunting \\
\tablevspace
0004 & \textit{ševi šekār čemā kua de} & he had gone hunting in our mountains & \textit{šekār} & hunting \\
\tablevspace
0005 & \textit{ševi šekār čemā kua de} & he had gone hunting in our mountains & \textit{čemā kua de} & in our mountains \\
\tablevspace
0006 & \textit{i jangali ve} & there was a forest & (no token) & \\
\tablevspace
0007 & \textit{berā de i šekāri bezzeše} & he killed (hit) some prey there & \textit{i šekāri} & some prey \\
\tablevspace
0008 & \textit{berā de i šekāri bezzeše} & he killed (hit) some prey there & \textit{berā de} & there \\
\lspbottomrule
\end{tabularx}

    \caption{Fragment of Hazarrudi Tat data set \citep{izadifar_tati_2022}}
    \label{Intro:tab:1}
\end{table}

The basic structure is illustrated in \tabref{Intro:tab:1}, from the Hazarrudi Tat\il{Tat!Hazarrudi} data set \citep{izadifar_tati_2022}. The utterance in 0006 does not contain a relevant token, thus the token columns are empty. The utterance in 0007, on the other hand, contains two relevant tokens (`some prey,' and `there'). Each receives its own ID (0007 and 0008), and the utterance unit is repeated, enabling tokens to be systematically associated with their contexts across all analysis steps. WOWA currently contains approximately 20,000 analyzed tokens in context.

\newpage
Once identified, each token is coded for a number of features, which fall into the following three types:

\begin{sloppypar}
\begin{itemize}
    \item[] \textbf{Doculect-related features:}
    \begin{itemize}
        \item Genetic affiliation (e.g. Iranian, southwestern)
        \item Doculect location (latitude, longitude)
    \end{itemize}
    \item[] \textbf{Context-related features}
    \begin{itemize}
        \item Text and speaker identification (unique identifiers are assigned, which are described in the accompanying metadata document)
    \end{itemize}
    \newpage
    \item[] \textbf{Linguistic features of the token and immediate context}
    \begin{itemize}
        \item Classifiable versus non-classifiable (if an utterance unit contains either no relevant token or none that can be unambiguously classified). Non-classifiable tokens are not included in statistical analyses
        \item Pronominal versus nominal form
        \item Animacy
        \item Definiteness (only applied to direct objects)
        \item Weight
        \item Role (see \REF{Intro:extab:3})
        \item Flag (\isi{adposition}, or case-marking)
        \item Position relative to the governing predicate (the dependent variable): before (0) vs. after (1)
        \item Comments (free text entry)
    \end{itemize}
\end{itemize}
\end{sloppypar}

Obviously, the set of linguistic features could easily be extended to include, for example, main versus subordinate clause, finer-grained metrics of \isi{topicality}, and so on. The final decision on which features to include was a compromise determined by the partially conflicting demands of theoretical relevance, and practical concerns such as economy of time and resources, simplicity of implementation across multiple languages with multiple coders, and replicability and transparency of coding-decisions.  Previous research has pointed to the importance of pronominal versus nominal (e.g. \citealt{gerdes_typometrics_2021}), \isi{animacy}, \isi{weight} (references in \sectref{Intro:ss:5}), informativity \citep{FaghiriSamvelian2020SOV}, \isi{flagging} and \isi{role} (see \sectref{Intro:ss:2} above), and these are also features that best satisfy the practical constraints just mentioned. Note that the raw data are available for coding additional features in the future. For each linguistic feature, coders select from a pre-defined set of options, which are explained in the Coding Guidelines.\footnote{\url{https://multicast.aspra.uni-bamberg.de/resources/wowa/data/_docs/guidelines/wowa_coding-guidelines.pdf }}

Coders work with the project coordinators, and problematic issues are resolved collaboratively to maximize cross-coder consistency. The coding scheme was presented and discussed collectively at two workshops (2019, 2020), and continued to evolve over the course of the project, before a final version was adopted in 2020. It should be evident that in a project of this nature, with multiple contributors working on multiple languages, compromise is inevitable. We have strived to maintain the fine line between maximal simplicity and generality (limiting the number of coding options), while maintaining sufficient flexibility for capturing the range of cross-\isi{language variation} contained in the data. Nevertheless, some degree of coding indeterminacy is inevitable, and for this reason, we include the coding option ``other'' in all linguistic categories to capture those instances where the analyst cannot decide among the available options. The full list of coding options is available in the Coding Guidelines; by way of illustration, we demonstrate in \tabref{Intro:tab:2} the linguistic coding of the eight items from \tabref{Intro:tab:1} above.

\begin{table}
\footnotesize
    \begin{tabularx}{\textwidth}{QQllrrllr}
\lsptoprule
token & token translation & pro & anim & \isi{weight} & weight2 & \isi{role} & flag & position \\
\midrule
\textit{čemā rustā de} & in our village &  & inan & 2 & 11 & loc & postp & 0 \\
\tablevspace
\textit{šekārči} & hunter &  & hum & 1 & 7 & cop & bare & 0 \\
\tablevspace
\textit{šekār} & hunting &  & inan & 1 & 5 & \isi{Goal} & bare & 1 \\
\tablevspace
\textit{šekār} & hunting &  & inan & 1 & 5 &  \isi{Goal} & bare & 1 \\
\tablevspace
\textit{čemā kua de} & in our mountains &  & inan & 2 & 9 & loc & postp & 1 \\
\tablevspace
(no token) &  &  &  &  &  &  &  & \\
\tablevspace
\textit{i šekāri} & a prey &  & inan & 2 & 7 & do & bare & 0 \\
\tablevspace
\textit{berā de} & there &  & adv & 1 & 6 & loc & postp & 0 \\
\lspbottomrule
    \end{tabularx}
    \caption{Coding the linguistic values for the tokens in \tabref{Intro:tab:1}}
    \label{Intro:tab:2}
\end{table}

The ``pro'' column is empty in \tabref{Intro:tab:2}, because there are no pronominal tokens in this stretch of discourse. The ``\isi{weight}'' column records orthographic words, except function words solely employed as \isi{flagging} devices (e.g. simple adpositions). The ``weight2'' column is a finer-grained \isi{weight} metric that is automatically generated, based on the number of characters contained in the transcription of the token (thus the first token consists of 11 characters); it provides a rough proxy for the number of phonological segments in each token. The column ``position'' is the dependent variable, and offers a binary option of <0> (pre-verbal) versus <1> (post-verbal).

\largerpage
The proprietary spreadsheet format used for data entry was dictated by practical considerations; most contributors use MS Excel (or equivalent) and were able to enter their data into the template that we provided. For the actual analysis, data are exported to R, a powerful and flexible programming language and platform for statistical computing.

With regard to the \isi{pronoun} category, we have included only prosodically independent pronouns as tokens. For languages that make extensive use of \isi{clitic} object pronouns\is{pronoun!object}, this means that the number of classifiable \isi{object} tokens in these languages may be very low, which has a detrimental impact on the statistical analysis (authors have the option of noting the presence of \isi{clitic} pronouns in the comments column (e.g. \citealt{schreiber2021pontic}), so the information is available for future analyses.) There are sound empirical reasons for distinguishing free and \isi{clitic} pronouns, illustrated below from spoken Persian\il{Persian (colloquial)} (\citetv{chapters/7_RasekhMahandetal_Persian}): around 95\% of nominal direct objects precede the verb (\isi{OV}), as in (\ref{Intro:ex:8}). Clitic object pronouns\is{pronoun!object}, on the other hand, frequently right-attach to the verb, and indeed must do so if the verb is the sole available host, as in (\ref{Intro:ex:9}) and (\ref{Intro:ex:10}):

\ea\label{Intro:ex:8} 
Colloquial New Persian \il{Persian (colloquial)}\citep[C, 0263]{Izadi2022Persian} \\
\gll \textbf{doz=ra}̂ bord bâlâ \\
dosage\textsc{=acc} carry\textsc{.pst.3sg} upwards \\
\glt `(He) increased \textbf{the dosage}.'
\z

\ea\label{Intro:ex:9} 
Colloquial New Persian \il{Persian (colloquial)}\citep[V, 2375]{Izadi2022Persian} \\
\gll mi-šenâs-im\textbf{=ešân} \\
\textsc{indic-}know\textsc{.prs-1pl=3pl} \\
\glt `We know \textbf{them}.'
\z

\ea\label{Intro:ex:10} 
Colloquial New Persian \il{Persian (colloquial)}\citep[oh\_f\_accident\_0166]{HaigRasekhMahand2022HamBam}  \\
\gll be-bar-id\textbf{=aš} \\
\textsc{imper-}take\textsc{.prs-2pl=3sg} \\
\glt `Take \textbf{him}!'
\z

It would make little sense to count constructions such as (\ref{Intro:ex:9}) and (\ref{Intro:ex:10}) as `\isi{VO},' apparently in \isi{contrast} to the \isi{OV} of (\ref{Intro:ex:8}). Examples (\ref{Intro:ex:9}) and (\ref{Intro:ex:10}) illustrate a language-specific rule of cliticization, which permits no variability of \isi{object} placement in these examples. Clitic placement is a fascinating issue in its own right but of limited relevance for the principles operating in the linearization of independent phrases in syntax. Consequently, \isi{clitic} pronouns hosted by the predicate are not included in calculations of pre- versus post-verbal \isi{argument} placement.  Clitic pronouns hosted by an item distinct from the predicate, on the other hand, are coded as ``bound,'' and the normal coding procedures applied. Depending on the analysis, pronominal tokens may be filtered out of a given sample.

\section{The impact of semantic role: the ``Goals Last'' effect}\label{Intro:ss:4}

\subsection{Background}\label{Intro:ss:4.1}

For the majority of languages in the sample, the variable ``Role'' turned out to be the most influential factor in determining pre- versus post-verbal position. The category ``Role'' in WOWA distinguishes the 19 categories shown in \REF{Intro:extab:3}.


\ea {Role categories recognized in WOWA (see Coding Guidelines, \sectref{Intro:ss:3.2})} \label{Intro:extab:3}
\begin{description}
\item[\bfseries\upshape\textsc{abl}] source of motion (`she came out of the house')
\item[\bfseries\upshape\textsc{addr}] {addressee} of a verb of speech (`they spoke to him/asked her/begged the King')
\item[\bfseries\upshape\textsc{becm}] `become,' i.e. the final state of a change-of-state ({inchoative}), predicate, such as `become X,' `turn into X'
\item[\bfseries\upshape\textsc{becm-c}] final state of a caused change-of-state predicate (`they made him King,' `she turned him to stone')
\item[\bfseries\upshape\textsc{ben}] {benefactive}; a person who benefits, or is disadvantaged, by an event without being directly impinged on by the action
\item[\bfseries\upshape\textsc{com}] {comitative}; a person who accompanies another participant in some action, or state (`I went to the market with my father')
\item[\bfseries\upshape\textsc{cop}] {complement} of a copular expression (`they were farmers')
\item[\bfseries\upshape\textsc{cop-loc}] locational {complement} of a copular expression (`she was in the car')
\item[\bfseries\upshape\textsc{do}] {direct object}, which needs to be identified on language-specific criteria such as typical case marking properties
\item[\bfseries\upshape\textsc{do-def}] definite direct object\is{direct object!definite} (which will include most pronouns), i.e. an item whose identity is recoverable from the context through previous mention or assumed deictic reference (`she took that cup')
\item[\bfseries\upshape\textsc{goal}] {endpoint} or destination of a verb of motion (`it fell on the table')
\item[\bfseries\upshape\textsc{goal-c}] {endpoint} or destination of a verb of caused motion (`he put it on the table')
\item[\bfseries\upshape\textsc{instr}] {instrument} for carrying out an action
\item[\bfseries\upshape\textsc{loc}] static location (with no implication of movement) of a participant or event
\item[\bfseries\upshape\textsc{other}] none of the available categories
\item[\bfseries\upshape\textsc{poss}] possessed in a clause expressing possession `she had two brothers,' unless the language has a HAVE verb and expresses the possessed in the same way as a {direct object} (do)
\item[\bfseries\upshape\textsc{rec}] {recipient} of a theme in an event of {transfer}, typically GIVE
\item[\bfseries\upshape\textsc{rec-ben}] recipient-{benefactive}. This is included for contexts in which it is unclear whether a particular token is the {recipient}, or a {benefactive} of an action (`he bought the apples for us' --- {recipient} or {benefactive}?)
\item[\bfseries\upshape\textsc{stim}] {stimulus}, typically of verbs of emotion, perception, desire --- if they are not coded as direct objects (English `she was afraid of the snake' (stim), but not `she hates snakes' (coded as <do>))
\end{description}
\z

    % \begin{table}
    % \small
    % \begin{tabularx}{\textwidth}{lQ}
    %     \lsptoprule
    % Abbreviation & Explanation \\
    % \midrule
    % \textsc{abl} & source of motion (`she came out of the house') \\
    % \textsc{addr} & {addressee} of a verb of speech (`they spoke to him/asked her/begged the King') \\
    % \textsc{becm} & `become,' i.e. the final state of a change-of-state ({inchoative}), predicate, such as `become X,' `turn into X' \\
    % \textsc{becm-c} & final state of a caused change-of-state predicate (`they made him King,' `she turned him to stone') \\
    % \textsc{ben} & {benefactive}; a person who benefits, or is disadvantaged, by an event without being directly impinged on by the action \\
    % \textsc{com} & {comitative}; a person who accompanies another participant in some action, or state (`I went to the market with my father') \\
    % \textsc{cop} & {complement} of a copular expression (`they were farmers') \\
    % \textsc{cop-loc} & locational {complement} of a copular expression (`she was in the car') \\
    % \textsc{do} & {direct object}, which needs to be identified on language-specific criteria such as typical case marking properties \\
    % \textsc{do-def} & definite direct object\is{direct object!definite} (which will include most pronouns), i.e. an item whose identity is recoverable from the context through previous mention or assumed deictic reference (`she took that cup') \\
    % \textsc{goal} & {endpoint} or destination of a verb of motion (`it fell on the table') \\
    % \textsc{goal-c} & {endpoint} or destination of a verb of caused motion (`he put it on the table') \\
    % \textsc{instr} & {instrument} for carrying out an action \\
    % \textsc{loc} & static location (with no implication of movement) of a participant or event \\
    % \textsc{other} & none of the available categories \\
    % \textsc{poss} & possessed in a clause expressing possession `she had two brothers,' unless the language has a HAVE verb and expresses the possessed in the same way as a {direct object} (do) \\
    % \textsc{rec} & {recipient} of a theme in an event of {transfer}, typically GIVE \\
    % \textsc{rec-ben} & recipient-{benefactive}. This is included for contexts in which it is unclear whether a particular token is the {recipient}, or a {benefactive} of an action (`he bought the apples for us' --- {recipient} or {benefactive}?) \\
    % \textsc{stim} & {stimulus}, typically of verbs of emotion, perception, desire --- if they are not coded as direct objects (English `she was afraid of the snake' (stim), but not `she hates snakes' (coded as <do>)) \\
    % \lspbottomrule
    %     \end{tabularx}
    %     \caption{Role categories recognized in WOWA (see Coding Guidelines, \sectref{Intro:ss:3.2})}
    %     \label{Intro:tab:3}
    % \end{table}

The data reveal very divergent token frequencies of different roles. In fact, some are so infrequent that they offer little leverage for statistical purposes. \figref{Intro:fig:2} below provides the respective proportions of different roles, whereby we have lumped together those \isi{role} types that occur only marginally.

\begin{figure}
    \includegraphics[width=\textwidth]{figures/HaigetalIntroFig2.png}
    \caption{The respective frequency of non-subject roles across the WOWA sample}
    \label{Intro:fig:2}
\end{figure}

The most frequent \isi{role} type in our data are direct objects, which account for around one third of all tokens. The next most frequent are goals of motion, including both simple motion and caused motion (distinguished in \figref{Intro:fig:2} by different shades of blue). Recipients and addressees are relatively infrequent, and are therefore lumped together here (see below). Similarly, roles such as \isi{instrument}, \isi{comitative}, \isi{stimulus}, and \isi{benefactive} do not occur in sufficient numbers to permit meaningful quantitative analysis, so have been combined under ``other obliques.'' The ``other \isi{role}'' category includes tokens that were not classifiable under any of the available role-categories.

Turning now to the respective frequencies of post-verbal placement, \tabref{Intro:tab:4} visualizes the general trend discernible across the data set. Four categories are distinguished: Goals (including caused goals), Recipient+Addressee, other obliques, and direct objects. For each of those four categories, we have colour-coded any frequency values for post-verbal placement which exceed 66\%.

\definecolor{introblue}{RGB}{42, 127, 255}
\definecolor{introred}{RGB}{255, 85, 85}
\definecolor{introyellow}{RGB}{255, 204, 0}
\definecolor{introgreen}{RGB}{113, 200, 55}

\begin{table}
    \resizebox{\linewidth}{!}{\begin{tabular}{llllll}
\lsptoprule
&& \multicolumn{4}{l}{\% post-verbal} \\
&  &  & Recipients + & Other \hspace{1cm} & Direct \hspace{.8cm} \\
& Doculect & Goals \hspace{1cm} & Addressees & Obliques & Objects \\
\midrule
{\purplepentagon} & \textit{Laz (Arhavi)}\il{Kartvelian!Laz Arhavi} & 4 & 6 & 1 & 3 \\
{\bluediamond} & \textit{Persian (New, Early Classical)}\il{Persian (Early New)} & 5 & 0 & 9 & 2 \\
{\redcircle} & \textit{Oghuz (Ankara)}\il{Turkic!Turkish Ankara} & 7 & 19 & 7 & 2 \\
{\redcircle} & \textit{Oghuz (Erzurum)}\il{Turkic!Turkish Erzurum} & 38 & 8 & 9 & 7 \\
{\bluediamond} & \textit{Balochi (Turkmen)}\il{Balochi!Turkmen} & 48 & 14 & 6 & 2 \\
{\bluediamond} & \textit{Kurdish (Northern, Ankara)}\il{Kurdish (Northern)!Ankara} & 59 & 13 & 6 & 0 \\
{\greenstar} & \textit{Kholosi (Kholos)}\il{Indic!Kholosi} & 62 & 39 & 1 & 1 \\
{\bluediamond} & \textit{Mazandarani (Kordxeyl)}\il{Mazandarani!Kordxeyl} & 63 & 12 & 8 & 3 \\
{\bluediamond} & \textit{Balochi (Coastal)}\il{Balochi!Coastal} & 63 & 28 & 18 & 7 \\
{\bluediamond} & \textit{Bashkardi (Northern)}\il{Bashkardi!North} & 63 & 56 & 31 & 27 \\
{\redcircle} & \textit{Oghuz (Bayat)}\il{Turkic!Turkish Bayat} & 64 & 43 & 13 & 4 \\
{\bluediamond} & \textit{Zazaki (Cewlig)}\il{Zazaki!Cewlig} & \cellcolor{gray!35} 91 & 15 & 0 & 5 \\
{\redcircle} & \textit{Oghuz (Tabriz)}\il{Turkic!Turkish Tabriz} & \cellcolor{gray!35} 75 & 21 & 13 & 1 \\
{\bluediamond} & \textit{Persian (New)} \il{Persian (New)} & \cellcolor{gray!35} 84 & 25 & 19 & 5 \\
{\bluediamond} & \textit{Kurdish (Southern, Bijar)}\il{Kurdish (Southern)!Bijar} & \cellcolor{gray!35} 97 & 27 & 11 & 2 \\
{\bluediamond} & \textit{Tat (Daykuscu)}\il{Tat!Daykuscu} & \cellcolor{gray!35} 78 & 34 & 30 & 15 \\
{\bluediamond} & \textit{Kurdish (Northern, Muš)}\il{Kurdish (Northern)!Muš} & \cellcolor{gray!35} 89 & 36 & 5 & 3 \\
{\bluediamond} & \textit{Bashkardi (Southern)}\il{Bashkardi!South} & \cellcolor{gray!35} 80 & 38 & 36 & 11 \\
{\bluediamond} & \textit{Vafsi (Gurchani)}\il{Vafsi!Gurchani} & \cellcolor{gray!35} 88 & 38 & 11 & 2 \\
{\bluediamond} & \textit{Kurdish (Northern, Lachin)}\il{Kurdish (Northern)!Lachin} & \cellcolor{gray!35} 81 & 40 & 4 & 2 \\
{\bluediamond} & \textit{Balochi (Koroshi)}\il{Balochi!Koroshi} & \cellcolor{gray!35} 90 & 42 & 27 & 2 \\
{\bluediamond} & \textit{Talyshi (Lerik)}\il{Talyshi!Lerik} & \cellcolor{gray!35} 72 & 45 & 26 & 2 \\
{\redcircle} & \textit{Oghuz (Gagauz)}\il{Turkic!Turkish Gagauz} & \cellcolor{gray!35} 73 & 47 & 38 & 51 \\
{\bluediamond} & \textit{Zazaki (Siwereg)}\il{Zazaki!Siwereg} & \cellcolor{gray!35} 100 & 52 & 8 & 5 \\
{\bluediamond} & \textit{Tati (Hazarrudi)}\il{Tat!Hazarrudi} & \cellcolor{gray!35} 92 & 53 & 16 & 3 \\
{\redcircle} & \textit{Oghuz (Qashqai)}\il{Turkic!Qashqai} & \cellcolor{gray!35} 71 & \cellcolor{gray!35} 88 & 16 & 8 \\
{\bluediamond} & \textit{Kurdish (Central, Sanandaj)}\il{Kurdish (Central)!Sanandaj} & \cellcolor{gray!35} 94 & \cellcolor{gray!35} 95 & 22 & 1 \\
{\bluediamond} & \textit{Gorani (Gawraju)}\il{Gorani!Gawraju} & \cellcolor{gray!35} 96 & \cellcolor{gray!35} 82 & 36 & 5 \\
{\yellowtriangle} & \textit{NENA (Jewish, Sanandaj)}\il{Neo-Aramaic (NENA)!J. Sanandaj} & \cellcolor{gray!35} 92 & \cellcolor{gray!35} 81 & 40 & 5 \\
{\bluediamond} & \textit{Kumzari (Musandam)}\il{Kumzari!Musandam} & \cellcolor{gray!35} 100 & \cellcolor{gray!35} 97 & \cellcolor{gray!35} 87 & 7 \\
{\yellowtriangle} & \textit{NENA (Christian, Barwar)}\il{Neo-Aramaic (NENA)!C. Barwar} & \cellcolor{gray!35} 96 & \cellcolor{gray!35} 100 & \cellcolor{gray!35} 74 & \cellcolor{gray!35} 83 \\
{\yellowtriangle} & \textit{NENA (Jewish, Dohok)}\il{Neo-Aramaic (NENA)!J. Dohok} & \cellcolor{gray!35} 99 & \cellcolor{gray!35} 98 & \cellcolor{gray!35} 93 & \cellcolor{gray!35} 90 \\
{\yellowtriangle} & \textit{Arabic (Jewish, Baghdad)}\il{Arabic (Gələt)!Baghdad (Jewish)} & \cellcolor{gray!35} 100 & \cellcolor{gray!35} 100 & \cellcolor{gray!35} 85 & \cellcolor{gray!35} 97 \\
\lspbottomrule
    \end{tabular}}
    \caption{Frequency of post-verbal placement for different roles; shaded values indicate frequencies above 66\%. Goals and direct objects include only nominal expressions; Recipients, Addressees, and other oblique roles (Locations, Sources, Instrumentals, Benefactives, and Comitatives) also include pronominal expressions. Two doculects with less than 8 observations in each role category are excluded. See Appendix B for raw data. (NENA = Northeastern Neo-Aramaic).}
    \label{Intro:tab:4}
\end{table}

The data in \tabref{Intro:tab:4} permit a number of generalizations, which to our knowledge have hitherto not been recognized. First of all, the data underscore the exceptional status of spatial Goals when compared to any other \isi{role}. More than two thirds of the sample languages have dominant post-verbal placement of Goals (>66\%), and for the majority of these languages, dominant post-verbal placement is restricted only to Goals. In several of these languages, the frequency of post-verbal Goal\is{Goal!post-verbal}s is twice as high as the frequency of post-posing any other constituent. In other words, Goals are different. We can formulate this as a potential universal in (\ref{Intro:ex:11}):

\ea\label{Intro:ex:11} 
If a language postposes any \isi{role} with a greater than two-thirds frequency, it will postpose Goals.
\z

Note that \tabref{Intro:tab:4} combines the two roles Recipient and Addressee, due to the low absolute numbers of tokens in these categories. However, this obscures the fact that Recipients and Addressees do not actually pattern consistently across all data sets. In fact, three distinct patterns can be identified in the sample, bearing in mind that the relevant absolute numbers are small:
\begin{enumerate}[label=(\roman*)]
    \item both Addressee and Recipient occur before the verb (e.g. Oghuz Erzurum, Zazaki Cewlig\il{Zazaki!Cewlig}, modern spoken Persian\il{Persian (colloquial)}); 
    \item both occur after the verb (e.g. NENA Jewish Duhok\il{Neo-Aramaic (NENA)!J. Duhok}, Central Kurdish\il{Kurdish (Central)!Sanandaj} Sanandaj, Kumzari\il{Kumzari}); 
    \item Recipients occur after the verb, but Addressees before the verb (e.g. Vafsi Gurchani\il{Vafsi!Gurchani}, Northern Kurdish\il{Kurdish (Northern)!Muš} Muš, or Armenian of Agulis\il{Armenian (Eastern)!Agulis}). 
\end{enumerate}
Crucially, no language is attested in the sample that has post-verbal Addressees, but pre-verbal Recipients. Bearing this in mind, we can formulate the following implicational hierarchy indicating the conditions under which a particular \isi{role} type may occur as dominant (at least 66\%) postverbal.\footnote{The hierarchy in (\ref{Intro:ex:12}) is similar to others that have been formulated in the literature (e.g. \citealt{frommer_post-verbal_1981,haig_introduction_2018,stilo_preverbal_2018}), some of which add additional roles, or employ somewhat different terminology. One \isi{role} that has often been included is Benefactive; however, our data for this \isi{role} are sparse, and were unfortunately not always consistently coded,  rendering interpretation of the results difficult. Currently they are included under `other'; this requires more research.}

\ea\label{Intro:ex:12} 
Goals > Recipients > Addressees > Other > Direct \isi{object}
\z

The hierarchy in (\ref{Intro:ex:12}) is to be understood as an implicational universal, which can be formulated as follows: based on token frequency in corpora of spontaneous spoken language, and frequencies of nominal as opposed to pronominal constituents,  if a language postposes any of the roles in (12) with greater than two-thirds frequency (dominant), then it will also dominantly postpose all higher roles on the hierarchy. Thus for the languages in the WATZ sample --- regardless of genetic affiliation --- there is no language that has, for example, dominant post-verbal Addressees, but not dominant post-verbal Goal\is{Goal!post-verbal}s. There are, however, a considerable number that only have dominant post-verbal Goal\is{Goal!post-verbal}s, and no other roles. Finally, if any language has dominant post-verbal direct objects, then all other roles are likewise dominant post-verbal. Thus there are no languages that, for example, combine dominant post-verbal objects with pre-verbal Goal\is{Goal!pre-verbal}s.

Currently it is impossible to say with certainty whether the regularities illustrated in \tabref{Intro:tab:4} and expressed in (\ref{Intro:ex:11}) and (\ref{Intro:ex:12}) represent a peculiarity of the languages in and around WATZ, or whether they reflect a deeper trait of connected spoken language, which should surface in spoken-language corpora of any language. We suggest there are grounds for assuming that (\ref{Intro:ex:12}) does reflect, at least in part, universal tendencies. \citet{haig_goals_2023} compare the two ends of the hierarchy (direct objects and Goals) across a more extensive sample of spoken language corpora and report that in no corpus do frequencies of postverbal objects exceed frequencies of postverbal Goals, regardless of language type or genetic affiliation.

\begin{sloppypar}
Even if only parts of (\ref{Intro:ex:12}) should turn out to be valid outside of WATZ, this would have considerable implications for understanding, for example, diachronic change in \isi{word order}. Essentially, (\ref{Intro:ex:12}) predicts an ordered sequence in a shift from \isi{OV} to \isi{VO} and vice versa. If a language is, for example, consistently verb-final (i.e. no \isi{role} is dominant post-verbal), then (\ref{Intro:ex:12}) predicts that any change towards less verb-finality would occur first with Goals, and proceed down the hierarchy, with direct objects the last to shift. For the other direction, i.e. a language that is consistently verb-initial in the VP (like English), the prediction is that if any \isi{argument} type shifts across the verb, it will be direct objects first, and Goals last. There is thus an asymmetry in the way \isi{VO} and \isi{OV} languages can be expected to move closer to one another. Preliminary observation of word-order change in WATZ suggests that this holds, regardless of whether the shift is considered internally motivated, or contact induced.
\end{sloppypar}

At this point, the relationship between \isi{recipient} and \isi{Goal} roles merits discussion. In earlier work \citep{Haig2022PostPredicateCon}, it was argued, on the basis of Kurdish\il{Kurdish} data, that these roles share a common semantic component, defined as ``event \isi{endpoint},'' which motivated the shared post-verbal syntax in Kurdish\il{Kurdish}. However, as we have seen, for the majority of other languages in our sample, recipients and goals do not pattern alike. On the assumption that both share \isi{endpoint semantics}, the question arises as to what inhibits post-verbal placement of recipients? To understand this, it is important to recall that \isi{word order} is the product of competing motivations, of which iconicity is but one. These include verb-\isi{object} adjacency (the tendency for direct objects and verbs not to be separated by other constituents), \isi{weight}, \isi{animacy}, and agency considerations. Thus \isi{word order} in any given context is the product of multiple factors, including \isi{information structure}, semantics, and configurational constraints. Recipients differ strikingly from goals in several dimensions relevant here: they are overwhelmingly human, with high frequency pronominal and first or second person, and are treated syntactically as direct objects in many languages \citep{Haspelmath2015Ditransitive}. Thus we suggest that \isi{endpoint semantics} are simply overridden by other factors in the ordering of recipients. The distinction between goals of verbs of motion, and recipients is clearly maintained in \citegen{Haspelmath2015Ditransitive} concept of ``ditransitive construction,'' which presupposes an element of \isi{transfer} of possession, while goals of verbs of caused-motion, such as `put,' lack such an entailment and are thus outside the purview of the typology of ditransitive constructions. We might add that conversely, \isi{transfer} of possession does not necessarily entail movement: it is possible to give someone a house, or a piece of land, which involves no actual change of location of the ``theme.'' Our revised conclusion is thus that although shared \isi{endpoint semantics} mean goals and recipients may pattern alike in some languages, the overall \isi{weight} of evidence suggests that a distinction should be maintained (see \citealt{haig_goals_2023} for a more detailed discussion).

\section{The impact of weight on post-posing}\label{Intro:ss:5}

\subsection{Background}\label{Intro:ss:5.1}

\begin{sloppypar}
It is fair to say that in both experimental and corpus-based approaches to word-order typology, considerations of \isi{weight} (however formalized) have attracted more attention than any other single factor (e.g. \citealt{FaghiriSamvelian2020SOV,SchnellSchiborr2022Cross}, \citealt[5--10]{wasow_factors_2022}). However, as \citet{yao_np_2018} points out, most of the relevant research considers \isi{weight} as a factor in determining the relative order of \textbf{constituents occurring on the same side of the predicate}, for example the relative ordering of the two PP's in (\ref{Intro:ex:13}) \citep[from][6]{wasow_factors_2022}.
\end{sloppypar}

% \begin{sloppypar}
\ea\label{Intro:ex:13} 
\textit{The gamekeeper looked [through his binoculars] [into the blue but slightly overcast sky].}
\z

% \end{sloppypar}

\begin{sloppypar}
For languages such as English\il{English}, which regularly place objects and other non-subject verbal dependents after the verb (the ``post-verbal domain,'' \citealt{yao_np_2018}), it seems that short constituents tend to precede longer constituents (``short before long''), as illustrated in (\ref{Intro:ex:13}). But this trend is apparently reversed for the pre-verbal domain in \isi{head-final} languages like Japanese\il{Japanese}, where long constituents reportedly preferably precede short \citep{YamashitaChang2001headfinal}. However, it is less clear which prediction would hold in languages which permit constituents to occur on either side of the predicate (``cross-domain NP shift,'' \citealt{yao_np_2018}). \citet{yao_np_2018} investigates cross-domain NP shift for \isi{object} placement in Mandarin\il{Mandarin}, which varies between a post-verbal (\isi{VO}) and a pre-verbal (\textit{bă}-\isi{OV}) option. Interestingly, this detailed study reveals no linear correlation between \isi{VO} versus \isi{OV}, and NP-length. \citet[560]{levshina_token-based_2019} notes a significant effect of length only for \isi{VO} languages, and most notably for clausal rather than nominal constituents. Research on diachronic syntax does consider cross-domain shift for direct objects, for example pragmatically driven \isi{object} fronting in \isi{VO} languages, and ``heavy NP shift'' in \isi{OV} languages \citep[205]{faarlund_word_2010}. However, for other kinds of constituent there is a general lack of research that would guide the expectations of a length effect for constituent order relative to the verb.
\end{sloppypar}

\subsection{Method and results}

\begin{sloppypar}
In the absence of a clear hypothesis from the literature, here we present an initial exploration of length effects on pre- versus post-verbal placement of different constituents in the WOWA sample. It is important to note that overall, the leverage of the \isi{weight} factor is significantly reduced in our spoken-language WOWA data, when compared to the written-language or experimental corpora that form the basis of most previous research. \citet[178]{SchnellSchiborr2022Cross} observe that in the spoken language corpora from the Multi-CAST collection \citep{haig_multi-cast_2023}, almost 90\% of all NPs in the data consist of maximally three words, with the majority being two words or less. In written Universal Dependency corpora on the other hand,  36\% of NPs contain four or more words \citep[178--179]{SchnellSchiborr2022Cross}. Note furthermore that for WOWA, we have not included clausal constituents in our analyses, which rules out the kinds of very long tokens that figure in written language corpora (\isi{complement} clauses, NPs with embedded relative clauses, etc.). 
\end{sloppypar}

\begin{sloppypar}
For the WOWA project, we operationalized ``\isi{weight}'' with two measures: (i) length of the \isi{object} phrase in words; and (ii) a finer-grained metric using characters, which provides a proxy for phonological \isi{weight}. The data shows a strong power law-like distribution of weights, with 64\% of analyzed tokens in certain roles (direct objects, Goals, Recipients and Addressees, as well as Locations, Sources, Instrumentals, Benefactives, and Comitatives) consisting of a single word and 91\% of two or fewer words. Similarly, for \isi{weight} in characters, 67\% of these tokens contain 8 or fewer characters, and 91\% contain 13 or fewer.
\end{sloppypar}

Two analyses were conducted both based on the finer-grained character-based \isi{weight} metric.  The first includes 33 data sets in WOWA, split by \isi{role}; we exclude those data sets that have fewer than 8 observations in a \isi{role} or display no variation in positioning. \tabref{Intro:tab:5} shows the mean correlation (Pearson) between pre- and post-verbal positioning (0, 1) and \isi{weight}.


\begin{table}
    \begin{tabular}{lrrr}
\lsptoprule
Roles & \textit{R} -value & SD & Observations \\
\midrule
direct objects & +0.007 & 0.117 & 8364 \\
goals & +0.022 & 0.161 & 4172 \\
recipients+addressees & $-$0.046 & 0.251 & 1340 \\
other obliques & +0.005 & 0.103 & 4444 \\
\lspbottomrule
    \end{tabular}
    \caption{Pearson correlation of weight with position }
    \label{Intro:tab:5}
\end{table}

All correlation coefficients hover around zero, with no substantial variation between data sets. Only a small handful of data sets have a coefficient exceeding a value of ±0.4 in any of the roles, three of which are for Recipients/Addressees, which due to their comparative rarity unfortunately offer the least robust results in general.

The second analysis takes into consideration the basic \isi{word order} of each doculect, because as \citet[11]{wasow_factors_2022} notes, different predictions hold for head final versus other languages. Consequently, we follow \citet{levshina_token-based_2019}  and divide the sample doculects into three groups, based on the frequency of nominal post-verbal direct objects in the corpora (the sample is not balanced across these three groups, due to the dominance of \isi{OV} Iranian languages): ``\isi{OV}'' (<25\% \isi{VO}), N=13; ``mixed'' (25--75\% \isi{VO}), N=16; and ``\isi{VO}'' (>75\% \isi{VO}), N=4.  \figref{Intro:fig:3} shows the mean \isi{weight} in characters for four \isi{role} types, distinguishing pre- and post-verbal placement. For each \isi{role} type, we present the findings split according to the three word-order types mentioned above (<25\% \isi{VO}; 25--75\% \isi{VO}; >75\% \isi{VO}).

\begin{figure}
    \includegraphics[width=\linewidth]{figures/intro_fig_3.png}
    \caption{Mean weight in characters of pre- and post-verbal constituents for four role types, split according to word-order type of the doculect}
    \label{Intro:fig:3}
\end{figure}

\begin{sloppypar}
\figref{Intro:fig:3} suggests a weak correlation between post-verbal placement and higher \isi{weight}, for direct objects (bottom right panel) across all \isi{word order} types, but none of the individual differences reach significance. Any claim for a correlation can only therefore draw on the fact that the minimal differences are in the same direction for each language type. For other roles, no consistent pattern can be identified. Turning to word-order type, it is only the >75\% \isi{VO} languages that exhibit a weak but consistent association of \isi{weight} and likelihood of post-verbal placement, with the strongest effect occurring for Goals in dominantly \isi{VO} languages; this would merit closer investigation, but note the low absolute figures (n=15) for pre-verbal Goal\is{Goal!pre-verbal}s in the four \isi{VO} doculects.
\end{sloppypar}

In sum, our investigation of the effect of \isi{weight} reveals only weak effects for only some roles, and some language types. This is partly attributable to the aforementioned narrow envelope of variation for \isi{weight} in spoken language, where the overwhelming majority of tokens consist of only one to two words. As noted, clausal constituents were not considered in our data. Equally, we emphasize that our investigation considers \isi{weight} as a factor in cross-domain shift, i.e. shift across the predicate, as opposed to relative position of constituents on the same side of the predicate, which has been the \isi{focus} of most existing research \citep[see][]{wasow_factors_2022}. As \citet{yao_np_2018} notes, research on cross-domain \isi{weight} effects is scarcely available, and their own results, like ours, reveal no clear \isi{weight} effects of \isi{weight}. We provisionally conclude that \isi{weight} effects noted in the literature do not carry over to cross-domain \isi{word order} variation in spoken language. This is definitely an area that would merit further research; see \citetv{chapters/3_Skopeteas_Prosody}, on the contrasting prosodic properties that hold in the pre- and post-verbal domain respectively.

\section{Summary and prospects, residual issues}\label{Intro:ss:6}

In sections one to three above, we have outlined the rationale, research context, and methodologies implemented in the WOWA project. In Sections \ref{Intro:ss:4} and \ref{Intro:ss:5}, we illustrate two use cases for exploring the entire database. The results for semantic \isi{role} provide abundant evidence for the special \isi{role} of spatial Goals in \isi{word order} variability, in particular of the dominantly \isi{OV} languages of Western Asia. The findings for \isi{weight}, however, do not yield a simple picture, suggesting that cross-domain word-order variation \citep{yao_np_2018} requires a distinct set of explanations to those that have been proposed for same-domain \isi{word order}, which focusses on the respective ordering of constituents occurring on the same side of the verb \citep{wasow_factors_2022}. 

We hope that our research will stimulate further research in this direction, and that in the future, the hitherto neglected effects of semantic \isi{role} are afforded due consideration. In \sectref{Intro:ss:5} we demonstrate that \isi{role} provides the best overall predictor of post-verbal placement, with Goals outstripping any other roles by a considerable margin. We formulated our findings in the form of an implicational universal (\ref{Intro:ex:12}), which embodies a number of testable hypotheses for future work on spoken-language corpora, and also has considerable implications for understanding \isi{word order} change.

Our findings also lend broad support for the concept of Transition Zone, indicating a gradual shift towards higher frequencies of verb-final constituents in the westward regions of WATZ. However, we require a more balanced and denser sample of doculects to develop a more robust framework for mapping structural variation to geospatial features, and to control for phylogenetic distance. Other issues that remain to be considered are measures of corpus-internal variation (see \citealt{craevschi_historical_2022} for provisional findings), co-\isi{argument} effects, and the \isi{role} of additional morphosyntactic features such as agreement, clause type (modality, negation, subordination, etc.), and more nuanced controlling for \isi{information structure} (see \citetv{chapters/13_Hodgsonetal_Armenian}, \citetv{chapters/16_Noorlander_Anatolia,chapters/15_Noorlander_NAINEI}).

Finally, our data point to the potential impact of \isi{register} and modality (spoken versus written language) on \isi{word order}. While the overwhelming majority of data in WOWA represents informal spoken language, in those data sets where data from more formal registers are available, they indicate some striking differences in \isi{word order} (\citealt{chapters/4_NourzaeiHaig_Balochi}\il{Balochi}, \citetv{chapters/7_RasekhMahandetal_Persian} and \citetv{chapters/8_Parizadeh_ENP} on Persian\il{Persian}, \citealt{chapters/13_Hodgsonetal_Armenian} on East Armenian). These differences invite further research, but in the meantime we urge caution when comparing cross-linguistic data, and emphasize the necessity for controlling for modality and \isi{register}.

\section{The organization of the volume}\label{Intro:ss:7}

The volume consists of 16 chapters, divided into four sections, each of which is introduced below: 

\begin{enumerate}[label=\Roman*]
    \item Theoretical and methodological issues (Chapters 1--3);
    \item Case studies from Iranian and Indo-Aryan languages (Chapters 4--9); 
    \item Case studies from the Caucasus and Black Sea (Chapters 10--13);
    \item Case studies from Semitic languages (Chapters 14--16).
\end{enumerate}

\subsection{Section I: Theoretical and methodological issues}\label{Intro:ss:7.1}

Section I includes the current introductory chapter, and two further chapters. Chapter \ref{WOWA:ch:2} by Kateryna Iefremenko investigates elements in the post-verbal domain of young adult bilingual\is{bilingualism} speakers of Kurmanji and Turkish\il{Turkic!Turkish Ankara} in Ankara in comparison with Turkish\il{Turkic!Turkish Erzurum} in Erzurum and under consideration of the sociolinguistic dichotomy between Turkish\il{Turkic!Turkish} as dominant national language and Kurmanji as regional language. Although the findings on elements in the post-verbal domain in the two languages are generally in line with previous research, the results show that the Turkish\il{Turkic!Turkish Erzurum} dialect of Erzurum tends to have more frequent post-verbal Goal\is{Goal!post-verbal}s than other varieties of Turkish\il{Turkic!Turkish}, which only apply post-verbal positions based on \isi{information structure} and \isi{weight} considerations. The higher rates of post-verbal Goal\is{Goal!post-verbal}s in the Erzurum dialect may plausibly reflect \isi{contact influence} from neighbouring Kurmanji Kurdish dialects, which exhibit typical Iranian post-verbal Goal\is{Goal!post-verbal}s of motion and caused motion. In one particular construction  \textit{erdê ketin} `to fall on the ground' the \isi{Goal} nevertheless appears predominantly pre-verbal among bilingual\is{bilingualism} Kurmanji Kurdish speakers; it may be potentially modelled on Tr. \textit{yere düşmek} `to fall on the ground.' The methodology of this study is unique in the context of the volume, in that explicitly bilingual\is{bilingualism} data were analysed that were elicited from the speakers by means of video prompts.

In Chapter \ref{WOWA:ch:3}, Stavros Skopeteas investigates prosodic structure in the pre-verbal and post-verbal domain in a sample of (primarily \isi{OV}) languages that includes Turkish\il{Turkic!Turkish}, Georgian, Caucasian Urum, Eastern Armenian, and Persian\il{Persian}. Skopeteas identifies three main types of \isi{OV} languages, distinguished according to the nature of constraints that determine whether objects may occur in the post-verbal domain. In some languages, post-verbal objects are very restricted, and are only permitted if they are outside the \isi{focus} domain of the clause, i.e. express given information, or afterthoughts (e.g. Standard Turkish\il{Turkic!Turkish}). In other languages, post-verbal objects are permitted as part of broad sentence \isi{focus} (e.g. Persian\il{Persian}), while in others, even objects with narrow \isi{focus} are also permitted (e.g. Georgian). In a sense, then, these three types represent increasing levels of tolerance for the integration of focal material into the post-verbal domain. The author reviews extant research on these languages and reports experimental results that illustrate the typology, and explores the interaction of prosodic and syntactic phrasing. This line of research complements the corpus-based approach of most contributions, which capture frequency patterns of linear ordering in naturalistic discourse, but leaves little space for systematic investigation of prosodic structure. 

\subsection{Section II: Case studies from Iranian and Indo-Aryan languages}\label{Intro:ss:7.2}

This is the largest section in the book and contains six chapters, each of which deal with one (or a group of) Iranian or Indo-Aryan languages. In Chapter \ref{WOWA:ch:4}, Maryam Nourzaei and Geoffrey Haig present an overview of \isi{word order} across three varieties of Balochi\il{Balochi}, each from areally diverse locations. The results provide further confirmation of the overall trend identified in this volume, that proximity to Mesopotamia correlates with an overall increase in post-verbal constituents; the westernmost variety of Balochi\il{Balochi!Koroshi} (Koroshi) exhibits both overall higher frequencies of post-verbal constituents, but also a greater range of \isi{role} types permitted in this position, when compared to the two more easterly varieties. In Chapter \ref{WOWA:ch:5}, Agnes Korn presents data from two varieties of Bashkardi in southern Iran. The data stem from legacy materials recorded in the 1950's, providing a rare opportunity to explore the possibility of recent changes in the language, but also for considering micro-variation across the two varieties. 

Chapter \ref{WOWA:ch:6} (Maryam Nourzaei) illustrates the only Indo-Aryan language in the sample, Kholosi\il{Indic!Kholosi}, a language island in southern Iran that has preserved aspects of Indo-Aryan morphosyntax, but has adapted in \isi{word order} to conform with the post-verbal placement of spatial Goals that characterizes all of its currently neighbouring languages. In Chapter \ref{WOWA:ch:7}, Mohammad Rasekh-Mahand and co-authors provide an in-depth study of spoken Persian\il{Persian (colloquial)}, comparing the recent HamBam data with the results of \citet{frommer_post-verbal_1981}. For the least formal registers of Persian\il{Persian}, they report stable values over the 40-year time span with regards to most aspects of post-verbal syntax, but note a shift in \isi{register} distribution in the modern data when compared to the older sample. Chapter \ref{WOWA:ch:8} is the sole chapter based on written data, and investigates a sample of Early New Persian\il{Persian (Early New)} texts (10--13th Century CE). The texts reveal some internal variation, but an overall remarkably consistent degree of verb-finality, with little evidence for the post-verbal syntax that characterizes all spoken western Iranian languages investigated so far. These findings raise questions regarding the diachronic development of post-verbal syntax in West Iranian, but also regarding the relationship between the spoken and written languages; it is possible (and we believe plausible) that the Early New Persian\il{Persian (Early New)} texts are not representative of the spoken language of the time, any more than today's formal written Persian\il{Persian} texts are representative of contemporary spoken Persian\il{Persian (colloquial)}. In Chapter \ref{WOWA:ch:9}, Masoud Mohammadirad takes a comparative look at three varieties from the Zagros region (Gorani Gawraǰu; Central
Kurdish Sanandaj; Southern Kurdish Bijar\il{Kurdish}). The findings are suggestive of Gorani \isi{substrate} effects in southernmost dialects of Central Kurdish\il{Kurdish (Central)}. 

\subsection{Section III: Case studies from the Caucasus and Black Sea}

In Chapters \ref{WOWA:ch:10} and \ref{WOWA:ch:11}, Diana Forker investigates \isi{word order} in Kartvelian and East Caucasian and Adyghe\il{Circassian!Adyghe} respectively. The data come from several sources, mostly outside the WOWA framework, but can be interpreted within the same framework. Role effects (Goals) are noticeable, though considerably less prevalent than in the Iranian languages and other languages of Mesopotamia. In Chapter \ref{WOWA:ch:12}, Laurentia Schreiber and Mark Janse investigate \isi{word order} patterns in Romeyka\il{Hellenic!Romeyka} in bilingual\is{bilingualism} speakers under \isi{language shift} to Turkish\il{Turkic!Turkish}. While \isi{information structure} and phrase type are the most relevant factors determining the dominant word orders in Romeyka\il{Hellenic!Romeyka}, significant \isi{inter-speaker variation} indicates the ongoing process of \isi{language shift}. Chapter \ref{WOWA:ch:13} presents original spoken-language data from East Armenian (Katherine Hodgson, Victoria Khurshudyan and Pollet Samvelian). This research adds a new perspective to the growing literature on \isi{word order} in East Armenian, complementing existing research based on experimental and written-language data. The authors confirm a \isi{definiteness} effect on \isi{direct object} ordering, with definite direct objects showing greater word-order flexibility with respect to the verb (higher frequency of \isi{VO} ordering), while indefinite objects remain overwhelmingly \isi{OV}. They also identify \isi{inter-speaker variation} and the effect of \isi{register}. The Goals Last effect documented for most of the language of WATZ (\sectref{Intro:ss:4} above) is also confirmed in these data, though in somewhat weaker magnitude than in the Iranian languages of Mesopotamia.

\subsection{Section IV: Case studies from Semitic languages}\label{Intro:ss:7.4}

This section includes three contributions on Semitic languages. In Chapter \ref{WOWA:ch:14}, Bettina Leitner describes the basic \isi{word order} profile of Khuzestani Arabic\il{Arabic (Gələt)!Khuzestani}, a linguistic island of Arabic in Iran, and discusses reasons for deviations from the default \isi{word order} VX, such as \isi{language contact} and internal change. In Chapter \ref{WOWA:ch:15}, Paul Noorlander discusses Neo-Aramaic\il{Neo-Aramaic (NENA)} dialects in Iran and northeastern Iraq, which include at least one dialect that has undergone a complete shift from \isi{VO} to \isi{OV} (Jewish Urmi\il{Neo-Aramaic (NENA)!J. Urmi}). While the impetus for the shift is almost certainly \isi{language contact}, Noorlander illustrates how internal factors, in particular \isi{information structure}, shape the way these changes have played out. Paul Noorlander also contributes Chapter \ref{WOWA:ch:16} on Arabic and Neo-Aramaic in Eastern Anatolia, a region of high linguistic diversity. Noteworthy findings include the variability in copula construction\is{copula!construction}
s, which contrasts with the otherwise fairly regular presence of clause-final \isi{copula} elements in most of WATZ. 

\sloppy
\printbibliography[heading=subbibliography,notkeyword=this]

\lehead{G. Haig, M. Rasekh-Mahand, D. Stilo, L. Schreiber \& N. Schiborr}
\newpage
\begin{paperappendix}
\lehead{G. Haig, M. Rasekh-Mahand, D. Stilo, L. Schreiber \& N. Schiborr}
\section{Data sources and raw figures}
\largerpage[1]
% \vspace*{-\baselineskip}
\begin{table}[h!]
\resizebox{.8\textwidth}{!}{
\begin{tabular}{lll}
    \lsptoprule
doculect & affiliation & source \\
\midrule
\textit{Oghuz (Ankara)}\il{Turkic!Turkish Ankara} &  {\redcircle} Turkic & \citealt{iefremenko2021oghuz} \\
\textit{Oghuz (Bayat)}\il{Turkic!Turkish Bayat} & {\redcircle} Turkic & Unpubl. \\
\textit{Oghuz (Erzurum)}\il{Turkic!Turkish Erzurum} & {\redcircle} Turkic & \citealt{dogan_oghuz_2021} \\
\textit{Oghuz (Gagauz)}\il{Turkic!Turkish Gagauz} & {\redcircle} Turkic & \citealt{Dogan_Gagauz_2021} \\
\textit{Oghuz (Qashqai)}\il{Turkic!Qashqai} & {\redcircle} Turkic & \citealt{schreiber_oghuz_2021} \\
\textit{Oghuz (Tabriz)}\il{Turkic!Turkish Tabriz} & {\redcircle} Turkic & \citealt{stilo_oghuz_2021} \\
\textit{Balochi (Coastal)}\il{Balochi!Coastal} & {\bluediamond} Iranian, western & \citealt{nourzaei_balochi_coastal_2021} \\
\textit{Balochi (Koroshi)}\il{Balochi!Koroshi} & {\bluediamond} Iranian, western & \citealt{nourzaei_balochi_koroshi_2021} \\
\textit{Balochi (Turkmen)}\il{Balochi!Turkmen} & {\bluediamond} Iranian, western & \citealt{haig_balochi_2022} \\
\textit{Bashkardi (Northern)}\il{Bashkardi!North} & {\bluediamond} Iranian, western & \citealt{korn_bashkardi_N_2022} \\
\textit{Bashkardi (Southern)}\il{Bashkardi!South} & {\bluediamond} Iranian, western & \citealt{korn_bashkardi_S_2022} \\
\textit{Gorani (Gawraǰū)} & {\bluediamond} Iranian, western & \citealt{mohammadirad_gorani_2022} \\
\textit{Kumzari (Musandam)}\il{Kumzari!Musandam} & {\bluediamond} Iranian, western & \citealt{haig_kumzari_2022} \\
\textit{Kurdish (Central, Sanandaj)}\il{Kurdish (Central)!Sanandaj} & {\bluediamond} Iranian, western & \citealt{mohammadirad_Sanandaj_Kurdish_2022} \\
\textit{Kurdish (Northern, Ankara)}\il{Kurdish (Northern)!Ankara} & {\bluediamond} Iranian, western & \citealt{iefremenko2021KurdishAnkara} \\
\textit{Kurdish (Northern, Lachin)}\il{Kurdish (Northern)!Lachin} & {\bluediamond} Iranian, western & \citealt{Stilo_NK_Lachin_2022} \\
\textit{Kurdish (Northern, Muš)}\il{Kurdish (Northern)!Muš} & {\bluediamond} Iranian, western & \citealt{haig_kurdish_2022} \\
\textit{Kurdish (Southern, Bijar)}\il{Kurdish (Southern)!Bijar} & {\bluediamond} Iranian, western & \citealt{mohammadirad_Bijar_Kurdish_2022} \\
\textit{Mazandarani (Kordxeyl)}\il{Mazandarani!Kordxeyl} & {\bluediamond} Iranian, western & \citealt{stilo_mazandarani_2022} \\
\textit{Persian (New)} & {\bluediamond} Iranian, western & \citealt{Izadi2022Persian} \\
\textit{Persian (New, Early Classical)}\il{Persian (Early New)} & {\bluediamond} Iranian, western & \citealt{parizadeh_persian_2022} \\
\textit{Talyshi (Lerik)}\il{Talyshi!Lerik} & {\bluediamond} Iranian, western & \citealt{Stilo_Talyshi_Lerik_2021} \\
\textit{Tat (Daγkušču)} & {\bluediamond} Iranian, western & Unpubl.  \\
\textit{Tati (Hazārrudi)} & {\bluediamond} Iranian, western & \citealt{izadifar_tati_2022} \\
\textit{Vafsi (Gurchani)}\il{Vafsi!Gurchani} & {\bluediamond} Iranian, western & \citealt{Dogan_Vafsi_2022} \\
\textit{Zazakî (Çewlîg)} & {\bluediamond} Iranian, western & \citealt{Demir.Dogan2021_Cewlig} \\
\textit{Zazakî (Siwêreg)} & {\bluediamond} Iranian, western & \citealt{Demir.Dogan2021_Siwereg} \\
\textit{NENA (Christian, Barwar)}\il{Neo-Aramaic (NENA)!C. Barwar} & {\yellowtriangle} West Semitic & \citealt{Stilo2022WOWACBarwar} \\
\textit{NENA (Jewish, Dohok)}\il{Neo-Aramaic (NENA)!J. Dohok} & {\yellowtriangle} West Semitic & \citealt{Molin2022NEDohok} \\
\textit{NENA (Jewish, Sanandaj)}\il{Neo-Aramaic (NENA)!J. Sanandaj} & {\yellowtriangle} West Semitic & \citealt{Noorlander2022WOWAJSana} \\
\textit{Arabic (Jewish, Baghdad)}\il{Arabic (Gələt)!Baghdad (Jewish)} & {\yellowtriangle} West Semitic & \citealt{BarMosheCraevschi2022Arabic} \\
\textit{Arabic (Khuzestan)} & {\yellowtriangle} West Semitic & \citealt{leitnerArabic2021} \\
\textit{Kholosi (Kholos)}\il{Indic!Kholosi} & {\greenstar} Indo-Aryan & \citealt{nourzaei_kholosi_2022} \\
\textit{Laz (Arhavi)}\il{Kartvelian!Laz Arhavi} & {\purplepentagon} Kartvelian & \citealt{stilo_laz_2021} \\
\textit{Pontic Greek (Romeyka)} & {\violetcross} Hellenic & \citealt{schreiber2021pontic} \\
\lspbottomrule
    \end{tabular}
    }
    \caption{Data sources: 35 doculects in WOWA (April 2024). Legend for abbreviations: OV = object-verb word order; VO = verb-object word order; NENA = North Eastern Neo-Aramaic. ``Unpubl.'' indicates data-sets which are fully annotated, but due to accessibility issues cannot be published online.}
    \label{Intro:tab:a1}
\end{table}
\clearpage
\lehead{G. Haig, M. Rasekh-Mahand, D. Stilo, L. Schreiber \& N. Schiborr}

\begin{table}
    \fittable{\begin{tabular}{lrrrrrrr}
\lsptoprule
&&&& \multicolumn{2}{l}{token weight} & \multicolumn{2}{l}{token weight} \\
&&& valid & \multicolumn{2}{l}{in words} & \multicolumn{2}{l}{in characters} \\
\midrule
doculect & texts & words & tokens & mean & SD & mean & SD \\
\midrule
\textit{Oghuz (Ankara)}\il{Turkic!Turkish Ankara} & 28 & 4145 & 587 & 1.42 & 0.68 & 8.80 & 5.18 \\
\textit{Oghuz (Bayat)}\il{Turkic!Turkish Bayat} & 1 & 3037 & 835 & 1.46 & 0.72 & 7.62 & 4.17 \\
\textit{Oghuz (Erzurum)} & 3 & 3860 & 636 & 1.35 & 0.58 & 7.63 & 3.77 \\
\textit{Oghuz (Gagauz)} & 2 & 5220 & 594 & 1.39 & 0.65 & 7.63 & 4.12 \\
\textit{Oghuz (Qashqai)} & 5 & 2915 & 557 & 1.52 & 0.82 & 7.84 & 4.56 \\
\textit{Oghuz (Tabriz)} & 13 & 3468 & 851 & 1.47 & 0.71 & 8.37 & 4.88 \\
\textit{Balochi (Coastal)} & 3 & 6768 & 1535 & 1.42 & 0.60 & 8.11 & 4.79 \\
\textit{Balochi (Koroshi)} & 2 & 3083 & 573 & 1.53 & 0.73 & 9.11 & 4.63 \\
\textit{Balochi (Turkmen)} & 4 & 4323 & 580 & 1.60 & 0.84 & 8.25 & 4.94 \\
\textit{Bashkardi (Southern)} & 5 & 947 & 234 & 1.35 & 0.63 & 6.85 & 3.37 \\
\textit{Bashkardi (Northern)} & 6 & 2744 & 596 &  &  &  &  \\
\textit{Gorani (Gawraju)} & 7 & 8782 & 1015 & 1.35 & 0.64 & 7.21 & 4.12 \\
\textit{Kumzari (Musandam)} & 2 & 4496 & 592 & 1.25 & 0.58 & 5.49 & 3.04 \\
\textit{Kurdish (Central, Sanandaj)} & 11 & 8502 & 1180 & 1.37 & 0.62 & 7.48 & 3.91 \\
\textit{Kurdish (Northern, Ankara)} & 30 & 4728 & 507 & 1.45 & 0.60 & 8.03 & 4.31 \\
\textit{Kurdish (Northern, Lachin)} & 28 & 3714 & 773 & 1.84 & 0.76 & 7.37 & 4.25 \\
\textit{Kurdish (Northern, Mus)} & 2 & 2711 & 693 & 1.47 & 0.64 & 4.84 & 1.99 \\
\textit{Kurdish (Southern, Bijar)} & 8 & 7251 & 1150 & 1.45 & 0.72 & 7.53 & 4.61 \\
\textit{Mazandarani (Kordxeyl)} & 7 & 3193 & 676 & 1.56 & 0.70 & 7.30 & 4.16 \\
\textit{Persian (New)} & 30 & 12564 & 1628 & 1.65 & 0.84 & 9.71 & 6.02 \\
\textit{Persian (New, Early Classical)} & 3 & 6751 & 1278 & 1.59 & 0.86 & 9.40 & 6.67 \\
\textit{Talyshi (Lerik)} & 3 & 2872 & 650 & 1.76 & 0.73 & 7.16 & 3.80 \\
\textit{Tat (Daykuscu)} & 1 & 1316 & 320 & 1.38 & 0.54 & 7.38 & 3.52 \\
\textit{Tati (Hazarrudi)} & 8 & 4068 & 665 & 1.37 & 0.68 & 7.00 & 4.08 \\
\textit{Vafsi (Gurchani)} & 10 & 4751 & 733 & 1.56 & 0.76 & 7.89 & 3.75 \\
\textit{Zazaki (Cewlig)} & 1 & 2444 & 410 & 1.43 & 0.66 & 5.89 & 3.28 \\
\textit{Zazaki (Siwereg)} & 1 & 1972 & 352 & 1.39 & 0.59 & 5.99 & 3.19 \\
\textit{NENA (Christian, Barwar)} & 5 & 3517 & 963 & 1.38 & 0.66 & 7.83 & 4.68 \\
\textit{NENA (Jewish, Dohok)} & 11 & 3295 & 514 & 1.26 & 0.54 & 6.85 & 3.56 \\
\textit{NENA (Jewish, Sanandaj)} & 4 & 7166 & 1184 & 1.25 & 0.52 & 6.74 & 3.05 \\
\textit{Arabic (Jewish, Baghdad)} & 4 & 3057 & 490 & 1.39 & 0.68 & 9.01 & 5.14 \\
\textit{Arabic (Khuzestan)} & 6 & 6391 & 546 & 1.33 & 0.65 & 7.90 & 3.94 \\
\textit{Kholosi (Kholos)} & 2 & 3171 & 516 & 1.54 & 0.77 & 8.72 & 4.89 \\
\textit{Laz (Arhavi)} & 11 & 1389 & 400 & 1.22 & 0.51 & 7.72 & 4.43 \\
\textit{Pontic Greek (Romeyka)} & 5 & 2946 & 501 & 1.66 & 0.72 & 8.37 & 3.76 \\
\lspbottomrule
    \end{tabular}}
    \caption{Raw figures for the WOWA data sets, corpus size and mean token weights in words and characters}
    \label{Intro:tab:B1}
\end{table}
\clearpage
\lehead{G. Haig, M. Rasekh-Mahand, D. Stilo, L. Schreiber \& N. Schiborr}

\begin{table}
    \fittable{\begin{tabular}{lrrrrrr}
    \lsptoprule
& \multicolumn{3}{l}{nominal direct objects} & \multicolumn{3}{l}{nominal goals} \\
\midrule
doculect & n(post) & n(all) & \%(post) & n(post) & n(all) & \%(post) \\
\midrule
\textit{Oghuz (Ankara)}\il{Turkic!Turkish Ankara} & 2 & 88 & 2 & 9 & 123 & 7 \\
\textit{Oghuz (Bayat)}\il{Turkic!Turkish Bayat} & 10 & 283 & 4 & 75 & 117 & 64 \\
\textit{Oghuz (Erzurum)}\il{Turkic!Turkish Erzurum} & 16 & 229 & 7 & 46 & 120 & 38 \\
\textit{Oghuz (Gagauz)}\il{Turkic!Turkish Gagauz} & 78 & 154 & 51 & 64 & 88 & 73 \\
\textit{Oghuz (Qashqai)}\il{Turkic!Qashqai} & 12 & 147 & 8 & 58 & 82 & 71 \\
\textit{Oghuz (Tabriz)}\il{Turkic!Turkish Tabriz} & 2 & 219 & 1 & 88 & 117 & 75 \\
\textit{Balochi (Coastal)}\il{Balochi!Coastal} & 23 & 338 & 7 & 71 & 112 & 63 \\
\textit{Balochi (Koroshi)}\il{Balochi!Koroshi} & 4 & 182 & 2 & 77 & 86 & 90 \\
\textit{Balochi (Turkmen)}\il{Balochi!Turkmen} & 3 & 192 & 2 & 20 & 42 & 48 \\
\textit{Bashkardi (Southern)}\il{Bashkardi!South} & 8 & 73 & 11 & 41 & 51 & 80 \\
\textit{Bashkardi (Northern)}\il{Bashkardi!North} & 50 & 182 & 27 & 58 & 92 & 63 \\
\textit{Gorani (Gawraju)}\il{Gorani!Gawraju} & 13 & 275 & 5 & 233 & 243 & 96 \\
\textit{Kumzari (Musandam)}\il{Kumzari!Musandam} & 8 & 115 & 7 & 83 & 83 & 100 \\
\textit{Kurdish (Central, Sanandaj)}\il{Kurdish (Central)!Sanandaj} & 3 & 295 & 1 & 267 & 283 & 94 \\
\textit{Kurdish (Northern, Ankara)}\il{Kurdish (Northern)!Ankara} & 0 & 81 & 0 & 70 & 119 & 59 \\
\textit{Kurdish (Northern, Lachin)}\il{Kurdish (Northern)!Lachin} & 3 & 197 & 2 & 90 & 111 & 81 \\
\textit{Kurdish (Northern, Muš)}\il{Kurdish (Northern)!Muš} & 6 & 217 & 3 & 107 & 120 & 89 \\
\textit{Kurdish (Southern, Bijar)}\il{Kurdish (Southern)!Bijar} & 7 & 298 & 2 & 272 & 281 & 97 \\
\textit{Mazandarani (Kordxeyl)}\il{Mazandarani!Kordxeyl} & 8 & 319 & 3 & 68 & 108 & 63 \\
\textit{Persian (New)}\il{Persian (New)} & 19 & 377 & 5 & 218 & 258 & 84 \\
\textit{Persian (New, Early Classical)}\il{Persian (Early New)} & 4 & 257 & 2 & 1 & 21 & 5 \\
\textit{Talyshi (Lerik)}\il{Talyshi!Lerik} & 4 & 164 & 2 & 73 & 102 & 72 \\
\textit{Tat (Daykuscu)}\il{Tat!Daykuscu} & 15 & 100 & 15 & 35 & 45 & 78 \\
\textit{Tati (Hazarrudi)}\il{Tat!Hazarrudi} & 4 & 153 & 3 & 111 & 121 & 92 \\
\textit{Vafsi (Gurchani)}\il{Vafsi!Gurchani} & 4 & 257 & 2 & 146 & 166 & 88 \\
\textit{Zazaki (Cewlig)}\il{Zazaki!Cewlig} & 4 & 85 & 5 & 90 & 99 & 91 \\
\textit{Zazaki (Siwereg)}\il{Zazaki!Siwereg} & 4 & 86 & 5 & 46 & 46 & 100 \\
\textit{NENA (Christian, Barwar)}\il{Neo-Aramaic (NENA)!C. Barwar} & 262 & 315 & 83 & 105 & 109 & 96 \\
\textit{NENA (Jewish, Dohok)}\il{Neo-Aramaic (NENA)!J. Dohok} & 188 & 210 & 90 & 105 & 106 & 99 \\
\textit{NENA (Jewish, Sanandaj)}\il{Neo-Aramaic (NENA)!J. Sanandaj} & 18 & 331 & 5 & 171 & 185 & 92 \\
\textit{Arabic (Jewish, Baghdad)}\il{Arabic (Gələt)!Baghdad (Jewish)} & 159 & 164 & 97 & 77 & 77 & 100 \\
\textit{Arabic (Khuzestan)} & 267 & 308 & 87 & 77 & 81 & 95 \\
\textit{Kholosi (Kholos)}\il{Indic!Kholosi} & 2 & 138 & 1 & 34 & 55 & 62 \\
\textit{Laz (Arhavi)}\il{Kartvelian!Laz Arhavi} & 4 & 128 & 3 & 2 & 54 & 4 \\
\textit{Pontic Greek (Romeyka)} & 116 & 175 & 66 & 62 & 78 & 7 \\
\lspbottomrule
    \end{tabular}}
    \caption{Raw figures of the WOWA data sets, rates of post-verbal placement of nominal direct objects and goals}
    \label{Intro:tab:B2}
\end{table}
\clearpage
\lehead{G. Haig, M. Rasekh-Mahand, D. Stilo, L. Schreiber \& N. Schiborr}

\begin{table}
    \fittable{\begin{tabular}{lrrrrrr}
    \lsptoprule
& \multicolumn{3}{l}{pronominal direct objects} & \multicolumn{3}{l}{pronominal goals} \\
\midrule
doculect & n(post) & n(all) & \%(post) & n(post) & n(all) & \%(post) \\
\midrule
\textit{Oghuz (Ankara)}\il{Turkic!Turkish Ankara} & 1 & 14 & 7 & 2 & 19 & 11 \\
\textit{Oghuz (Bayat)}\il{Turkic!Turkish Bayat} & 6 & 70 & 9 & 4 & 8 & 50 \\
\textit{Oghuz (Erzurum)}\il{Turkic!Turkish Erzurum} & 4 & 54 & 7 & 1 & 11 & 9 \\
\textit{Oghuz (Gagauz)}\il{Turkic!Turkish Gagauz} & 26 & 64 & 41 & 7 & 11 & 64 \\
\textit{Oghuz (Qashqai)}\il{Turkic!Qashqai} & 1 & 28 & 4 & 6 & 11 & 55 \\
\textit{Oghuz (Tabriz)}\il{Turkic!Turkish Tabriz} & 6 & 59 & 10 & 11 & 16 & 69 \\
\textit{Balochi (Coastal)}\il{Balochi!Coastal} & 27 & 99 & 27 & 1 & 2 & 50 \\
\textit{Balochi (Koroshi)}\il{Balochi!Koroshi} & 0 & 18 & 0 & 0 & 1 & 0 \\
\textit{Balochi (Turkmen)}\il{Balochi!Turkmen} & 2 & 55 & 4 & 0 & 5 & 0 \\
\textit{Bashkardi (Southern)}\il{Bashkardi!South} & 0 & 9 & 0 & 0 & 1 & 0 \\
\textit{Bashkardi (Northern)}\il{Bashkardi!North} & 2 & 23 & 9 & 1 & 5 & 20 \\
\textit{Gorani (Gawraju)}\il{Gorani!Gawraju} & 0 & 32 & 0 & 3 & 4 & 75 \\
\textit{Kumzari (Musandam)}\il{Kumzari!Musandam} & 52 & 81 & 64 & 40 & 40 & 100 \\
\textit{Kurdish (Central, Sanandaj)}\il{Kurdish (Central)!Sanandaj} & 0 & 24 & 0 & 11 & 13 & 85 \\
\textit{Kurdish (Northern, Ankara)}\il{Kurdish (Northern)!Ankara} & 1 & 11 & 9 & 5 & 9 & 56 \\
\textit{Kurdish (Northern, Lachin)}\il{Kurdish (Northern)!Lachin} & 0 & 34 & 0 & 3 & 5 & 60 \\
\textit{Kurdish (Northern, Muš)}\il{Kurdish (Northern)!Muš} & 2 & 41 & 5 & 4 & 12 & 33 \\
\textit{Kurdish (Southern, Bijar)}\il{Kurdish (Southern)!Bijar} & 0 & 45 & 0 & 2 & 2 & 100 \\
\textit{Mazandarani (Kordxeyl)}\il{Mazandarani!Kordxeyl} & 6 & 62 & 10 & 8 & 13 & 62 \\
\textit{Persian (New)} & 1 & 63 & 2 & 1 & 3 & 33 \\
\textit{Persian (New, Early Classical)}\il{Persian (Early New)} & 0 & 63 & 0 & 0 & 4 & 0 \\
\textit{Talyshi (Lerik)}\il{Talyshi!Lerik} & 2 & 73 & 3 & 8 & 14 & 57 \\
\textit{Tat (Daykuscu)}\il{Tat!Daykuscu} & 7 & 17 & 41 & 4 & 4 & 100 \\
\textit{Tati (Hazarrudi)}\il{Tat!Hazarrudi} & 4 & 60 & 7 & 13 & 17 & 76 \\
\textit{Vafsi (Gurchani)}\il{Vafsi!Gurchani} & 3 & 46 & 7 & 2 & 2 & 100 \\
\textit{Zazaki (Cewlig)}\il{Zazaki!Cewlig} & 4 & 41 & 10 & 11 & 13 & 85 \\
\textit{Zazaki (Siwereg)}\il{Zazaki!Siwereg} & 3 & 30 & 10 & 3 & 7 & 43 \\
\textit{NENA (Christian, Barwar)}\il{Neo-Aramaic (NENA)!C. Barwar} & 15 & 44 & 34 & 14 & 17 & 82 \\
\textit{NENA (Jewish, Dohok)}\il{Neo-Aramaic (NENA)!J. Dohok} & 21 & 31 & 68 & 6 & 6 & 100 \\
\textit{NENA (Jewish, Sanandaj)}\il{Neo-Aramaic (NENA)!J. Sanandaj} & 1 & 48 & 2 & 17 & 21 & 81 \\
\textit{Arabic (Jewish, Baghdad)}\il{Arabic (Gələt)!Baghdad (Jewish)} & 13 & 18 & 72 & 0 & 2 & 0 \\
\textit{Arabic (Khuzestan)} & 5 & 10 & 50 & 0 & 1 & 0 \\
\textit{Kholosi (Kholos)}\il{Indic!Kholosi} & 2 & 16 & 12 & 1 & 1 & 100 \\
\textit{Laz (Arhavi)}\il{Kartvelian!Laz Arhavi} & 0 & 35 & 0 & 0 & 5 & 0 \\

\lspbottomrule
    \end{tabular}}
    \caption{Raw figures of the WOWA data sets, rates of post-verbal placement of pronominal direct objects and goals}
    \label{Intro:tab:B3}
\end{table}
\clearpage
\lehead{G. Haig, M. Rasekh-Mahand, D. Stilo, L. Schreiber \& N. Schiborr}

\begin{table}
    \resizebox{.9\linewidth}{!}{\begin{tabular}{lrrrrrr}
\lsptoprule
& \multicolumn{3}{l}{recipients/addressees} & \multicolumn{3}{l}{other obliques} \\
\midrule
doculect & n(post) & n(all) & \%(post) & n(post) & n(all) & \%(post) \\
\midrule
\textit{Oghuz (Ankara)}\il{Turkic!Turkish Ankara} & 3 & 16 & 19 & 13 & 199 & 7 \\
\textit{Oghuz (Bayat)}\il{Turkic!Turkish Bayat} & 21 & 49 & 43 & 20 & 152 & 13 \\
\textit{Oghuz (Erzurum)}\il{Turkic!Turkish Erzurum} & 4 & 52 & 8 & 10 & 116 & 9 \\
\textit{Oghuz (Gagauz)}\il{Turkic!Turkish Gagauz} & 7 & 15 & 47 & 43 & 114 & 38 \\
\textit{Oghuz (Qashqai)}\il{Turkic!Qashqai} & 7 & 8 & 88 & 13 & 81 & 16 \\
\textit{Oghuz (Tabriz)}\il{Turkic!Turkish Tabriz} & 11 & 53 & 21 & 24 & 190 & 13 \\
\textit{Balochi (Coastal)}\il{Balochi!Coastal} & 31 & 112 & 28 & 25 & 138 & 18 \\
\textit{Balochi (Koroshi)}\il{Balochi!Koroshi} & 10 & 24 & 42 & 17 & 64 & 27 \\
\textit{Balochi (Turkmen)}\il{Balochi!Turkmen} & 3 & 21 & 14 & 7 & 110 & 6 \\
\textit{Bashkardi (Southern)}\il{Bashkardi!South} & 3 & 8 & 38 & 5 & 14 & 36 \\
\textit{Bashkardi (Northern)}\il{Bashkardi!North} & 23 & 41 & 56 & 15 & 49 & 31 \\
\textit{Gorani (Gawraju)}\il{Gorani!Gawraju} & 27 & 33 & 82 & 37 & 102 & 36 \\
\textit{Kumzari (Musandam)}\il{Kumzari!Musandam} & 93 & 96 & 97 & 61 & 70 & 87 \\
\textit{Kurdish (Central, Sanandaj)}\il{Kurdish (Central)!Sanandaj} & 21 & 22 & 95 & 44 & 200 & 22 \\
\textit{Kurdish (Northern, Ankara)}\il{Kurdish (Northern)!Ankara} & 2 & 15 & 13 & 12 & 209 & 6 \\
\textit{Kurdish (Northern, Lachin)}\il{Kurdish (Northern)!Lachin} & 22 & 55 & 40 & 9 & 244 & 4 \\
\textit{Kurdish (Northern, Muš)}\il{Kurdish (Northern)!Muš} & 16 & 45 & 36 & 7 & 139 & 5 \\
\textit{Kurdish (Southern, Bijar)}\il{Kurdish (Southern)!Bijar} & 11 & 41 & 27 & 17 & 148 & 11 \\
\textit{Mazandarani (Kordxeyl)}\il{Mazandarani!Kordxeyl} & 6 & 50 & 12 & 9 & 119 & 8 \\
\textit{Persian (New)} & 17 & 67 & 25 & 51 & 262 & 19 \\
\textit{Persian (New, Early Classical)}\il{Persian (Early New)} & 0 & 41 & 0 & 16 & 176 & 9 \\
\textit{Talyshi (Lerik)}\il{Talyshi!Lerik} & 23 & 51 & 45 & 37 & 142 & 26 \\
\textit{Tat (Daykuscu)}\il{Tat!Daykuscu} & 12 & 35 & 34 & 17 & 57 & 30 \\
\textit{Tati (Hazarrudi)}\il{Tat!Hazarrudi} & 9 & 17 & 53 & 33 & 212 & 16 \\
\textit{Vafsi (Gurchani)}\il{Vafsi!Gurchani} & 14 & 37 & 38 & 8 & 72 & 11 \\
\textit{Zazaki (Cewlig)}\il{Zazaki!Cewlig} & 4 & 27 & 15 & 0 & 49 & 0 \\
\textit{Zazaki (Siwereg)}\il{Zazaki!Siwereg} & 15 & 29 & 52 & 4 & 49 & 8 \\
\textit{NENA (Christian, Barwar)}\il{Neo-Aramaic (NENA)!C. Barwar} & 13 & 13 & 100 & 148 & 200 & 74 \\
\textit{NENA (Jewish, Dohok)}\il{Neo-Aramaic (NENA)!J. Dohok} & 41 & 42 & 98 & 50 & 54 & 93 \\
\textit{NENA (Jewish, Sanandaj)}\il{Neo-Aramaic (NENA)!J. Sanandaj} & 58 & 72 & 81 & 57 & 141 & 40 \\
\textit{Arabic (Jewish, Baghdad)}\il{Arabic (Gələt)!Baghdad (Jewish)} & 11 & 11 & 100 & 70 & 82 & 85 \\
\textit{Arabic (Khuzestan)} & 5 & 5 & 100 & 32 & 34 & 94 \\
\textit{Kholosi (Kholos)}\il{Indic!Kholosi} & 7 & 18 & 39 & 1 & 84 & 1 \\
\textit{Laz (Arhavi)}\il{Kartvelian!Laz Arhavi} & 2 & 36 & 6 & 1 & 87 & 1 \\
\textit{Pontic Greek (Romeyka)} & 3 & 5 & 60 & 43 & 114 & 38 \\
\lspbottomrule
    \end{tabular}}
    \caption{Raw figures for the WOWA data sets, rates of post-verbal placement of nominal and pronominal addressees/recipients and various other obliques (locations, sources, instruments, benefactives, comitatives)}
    \label{Intro:tab:B4}
\end{table}
\end{paperappendix}
\cleardoublepage
\end{document}
