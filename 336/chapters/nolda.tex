%% -*- coding:utf-8 -*-
\documentclass[output=paper
  ,nobabel
  ,draftmode
  ,colorlinks, citecolor=brown
]{langscibook}

\ChapterDOI{10.5281/zenodo.7142720}

\IfFileExists{../localcommands.tex}{%hack to check whether this is being compiled as part of a collection or standalone
   % add all extra packages you need to load to this file

\usepackage{tabularx,multicol}
\usepackage{url}
\urlstyle{same}

\usepackage{listings}
\lstset{basicstyle=\ttfamily,tabsize=2,breaklines=true}

\usepackage{langsci-basic}
\usepackage{langsci-optional}
\usepackage{langsci-lgr}
\usepackage{langsci-osl}
% \usepackage{./langsci/styles/langsci-lgr}
% \usepackage{./langsci/styles/langsci-osl}
% \usepackage{langsci-gb4e}

\usepackage{tikz}
\usetikzlibrary{patterns,calc}
\pgfdeclarepatternformonly{south east lines}{\pgfqpoint{-0pt}{-0pt}}{\pgfqpoint{3pt}{3pt}}{\pgfqpoint{3pt}{3pt}}{
    \pgfsetlinewidth{0.6pt}
    \pgfpathmoveto{\pgfqpoint{0pt}{3pt}}
    \pgfpathlineto{\pgfqpoint{3pt}{0pt}}
    \pgfpathmoveto{\pgfqpoint{.2pt}{-.2pt}}
    \pgfpathlineto{\pgfqpoint{-.2pt}{.2pt}}
    \pgfpathmoveto{\pgfqpoint{3.2pt}{2.8pt}}
    \pgfpathlineto{\pgfqpoint{2.8pt}{3.2pt}}
    \pgfusepath{stroke}}
    
\usepackage{stmaryrd}
\usepackage{wasysym}
\usepackage{multirow}
\usepackage{caption}
\usepackage{subcaption}
\usepackage{mathrsfs}
\usepackage{qtree}

\usepackage{linguex}


   %pminos do not split footnotes
% \interfootnotelinepenalty=10000 %Footnote in Laporte chapters has to be split SN


%\DeclareIndexNameFormat{default}{%
%\nameparts{#1}%
%\usebibmacro{index:name}%
%{\index[names]}%
%{\namepartfamily}%
%{\namepartgiveni}%
% {}% L1
% {}% L2
%{\namepartprefix}% generates spurious space L3
%{\namepartsuffix}% generates spurious space L4
%}

%  {\DeclareIndexNameFormat{default}{%
%     \usebibmacro{index:name}{\index[names]}{#1}{#3}{#5}{#7}}}

%\DeclareIndexNameFormat{default}{%
%  \usebibmacro{index:name}{\sindex[nom]}{#1}{#3}{#5}{#7}}

%\DeclareIndexNameFormat{default}{%
%  \usebibmacro{index:name}{\sindex[person]}{#1}{#3}{#5}{#7}}
%\DeclareIndexNameFormat{default}{%
%\nameparts{#1} \usebibmacro{index:name}{\sindex[person]]}{\namepartfamily}{‌​\namepartgiven}{\nam‌​epartprefix}{\namepa‌​rtsuffix}}

%\newcommand{\smiley}{:)}

%\renewbibmacro*{index:name}[5]{%
%\usebibmacro{index:entry}{#1}%
%{\iffieldundef{usera}{}{\thefield{usera}\actualoperator}\mkbibindexname{#2}{#3}{#4}{#5}}}

% \newcommand{\noop}[1]{}

%remove for final
%\overfullrule=1mm

\newcommand{\tobi}[2]}}
\renewcommand{\S}[1]{\tobi{#1}{\textsc{*}}}

% this volume references
% puts: [this volume]
% already defined: \citetv
%\newcommand{\citepv}[1]{(\citeauthor{#1} \citeyear*{#1} [this volume])}
\newcommand{\citealtv}[1]{\citeauthor{#1} \citeyear*{#1} [this volume]}

%parentheses around example number
\newcommand{\pref}[1]{(\ref{#1})}

% in-text examples

\newcommand{\lnex}[1]{\textit{#1}} %target lang word
\newcommand{\lnlit}[1]{(lit.: `#1')} %literal reading
\newcommand{\lnlat}[1]{(#1)} % latinization
\newcommand{\lntrans}[1]{`#1'} %translation
\newcommand{\lnexl}[2]%
{\lnex{#1}{} \lnlat{#2}} % ex with latinization
\newcommand{\lnexlat}[3]{\lnex{#1}{} \lnlat{#2}{} \lntrans{#3}} % ex with latinization and tranl.

%ch01
\newcommand{\co}[1]{\mbox{\textbf{#1}}}

%ch09

\newcommand{\cyrbulg}[1]{\begin{otherlanguage*}{bulgarian}#1\end{otherlanguage*}}


%ch10
\newcommand{\nlp}{{\small NLP}}
\newcommand{\mwe}{{\small MWE}}
\newcommand{\rae}{{\small RAE}}
\newcommand{\lvc}{{\small LVC}}
\newcommand{\pos}{{\small P}o{\small S}}
%\newcommand{\todo}[1]{ \textcolor{red}{#1} }

%\renewcommand{\labelenumi}{\theenumi}
%\ainamefmt{{vv}{ll}{, ff}{, jj}} % fullname

\newcommand{\biberror}[1]{{\color{red}#1}}

\newcommand{\osenovaitem}{--~}
   %% hyphenation points for line breaks
%% Normally, automatic hyphenation in LaTeX is very good
%% If a word is mis-hyphenated, add it to this file
%%
%% add information to TeX file before \begin{document} with:
%% %% hyphenation points for line breaks
%% Normally, automatic hyphenation in LaTeX is very good
%% If a word is mis-hyphenated, add it to this file
%%
%% add information to TeX file before \begin{document} with:
%% %% hyphenation points for line breaks
%% Normally, automatic hyphenation in LaTeX is very good
%% If a word is mis-hyphenated, add it to this file
%%
%% add information to TeX file before \begin{document} with:
%% \include{localhyphenation}
\hyphenation{
    Beck-man
    Ngu-yen
    back-chan-nel
    back-chan-nels
    mo-not-o-nous
    ste-reo-typ-i-cal
}

\hyphenation{
    Beck-man
    Ngu-yen
    back-chan-nel
    back-chan-nels
    mo-not-o-nous
    ste-reo-typ-i-cal
}

\hyphenation{
    Beck-man
    Ngu-yen
    back-chan-nel
    back-chan-nels
    mo-not-o-nous
    ste-reo-typ-i-cal
}

    \bibliography{../localbibliography}
    \togglepaper[23]
}{}

\title[Headedness as an epiphenomenon]{Headedness as an epiphenomenon: Case studies on compounding
  and blending in German}
\author{Andreas Nolda\orcid{0000-0003-4532-8256}\affiliation{Berlin-Brandenburg Academy of Sciences and Humanities}}
\abstract{This paper demonstrates how statements like ``compounds
are right-headed in German'' can be interpreted in a paradigmatic
approach to morphology in terms of word-formation relations between lexical
units, without presupposing word structures with ``head
constituents''. Using the theoretical framework of the
Pattern-and-Restriction Theory (\citealt{nolda:2012:konversion:deutschen,nolda:2018:explaining:linguistic}), it
is shown in four case studies that right-headedness applies in German not only
to compounds, but in principle also to blends – a domain where ``head
constituents'' are notoriously difficult to ascertain. Headedness
properties such as being a word-formation product which is categorially and/or
semantically determined by its last basis are identified solely on the basis of
word-formation relations and the involved formation patterns. In a paradigmatic
approach of this kind, headedness emerges as an epiphenomenon of the
word-formation relations between lexical units in a linguistic system.}

\begin{document}
\maketitle

\section{Overview}
In a paper discussing the question ``\foreignlanguagedummy{british}{Do words have
heads?}'', \citet[5–8]{becker:1990:word:heads}
distinguishes two kinds of morphological description: \emph{syntagmatic
morphology} and \emph{paradigmatic morphology}. Syntagmatic
morphology in the sense of \citet{becker:1990:word:heads}
describes morphological regularities in terms of relations between constituents
in word structures. Paradigmatic morphology, in turn, describes morphological
regularities in terms of relations between words (or stems, one might add). Key
notions of syntagmatic approaches include ``head'' and
``non-head'', whereas descriptions in paradigmatic approaches
make explicit or implicit reference to ``products'',
``bases'', ``morphological processes'',
``word-formation rules'', and ``morphological
restrictions'' (as pioneered in the work of \citealt{aronoff:1976:word:formation}). The contrast between syntagmatic and
paradigmatic morphology thus roughly coincides with \citeauthor{hockett:1954:two:models}’s \citeyearpar{hockett:1954:two:models} distinction between ``Item and
Arrangement'' and ``Item and Process''.

According to \citet[6]{becker:1990:word:heads},
paradigmatic approaches can cope for any kind of morphological phenomena,
whereas syntagmatic approaches are designed for concatenative morphology:
\begin{quotation}
Clearly for every syntagmatic analysis there is a
corresponding paradigmatic analysis, however the reverse is not valid: There are
structures that can be analysed paradigmatically but not syntagmatically, since
a syntagmatic analysis is only possible for additive rules but not for
subtractive or substitutional processes.
\end{quotation} Compounds, for example, are
readily analysed in both approaches, since they are basically formed by means of
concatenation. For blends, however, there is no straightforward syntagmatic
analysis, because their formation can involve various kinds of shortening
operations.

Although paradigmatic approaches do not encode head relations in word
structures, relations such as the ``categorial dependency'' of
a compound on one of its bases can still be accounted for by appropriate
formation rules. To put it in the words of \citet[2]{Zwicky:1985:heads}: ``category determination resides not in
constituents but in rules performing morphological operations''. A similar
point can be made for the ``semantic dependency'' which is
typical for endocentric subordinative compounds. Thus head properties like
``categorial dependency'' or ``semantic
dependency'' can in principle be determined in a paradigmatic approach
without presupposing any structural heads.\footnote{As an anonymous reviewer points
out, a related approach is taken by Construction Morphology which describes the
headedness of words in terms of constructional schemas (\citealt[Section 1.4 and~3.1]{booij:2010:construction:morphology}, \citealt{arcodia:2012:constructions:headedness}, and
others). Still, Construction Morphology is, in my view, more akin to syntagmatic
approaches to word formation: schemas directly specify properties of products,
whereas properties of bases are only indirectly specified via properties of
constituents of the product.}

In the present paper, it will be demonstrated how a statement of the sort
``compounds are right-headed in \ili{German}'' can be
interpreted in a paradigmatic approach like the \emph{\foreignlanguagedummy{british}{Pattern-and-Restriction Theory}} (\emph{PR}). PR is a
general theory of word formation which was developed and axiomatically
formalised by \citet{nolda:2012:konversion:deutschen,nolda:2018:explaining:linguistic}. PR’s major theoretical tools are
\emph{formation patterns} and associated \emph{formation
restrictions}, which are used to describe \emph{word-formation
relations} between lexical units in a spoken or written linguistic
system.\footnote{There is a sample computer implementation of PR called ``PPR'' (``System for Processing Formation Patterns and
Restrictions'', available at \url{http://andreas.nolda.org/software.html\#ppr}), which can be used for
testing the soundness of theoretical and empirical hypotheses in the PR
framework. It currently provides a very limited lexicon and selected formation
patterns for spoken and written Modern \ili{German} systems, including some of the
patterns discussed in this paper.} Instead of syntagmatically encoding
them in word structures, PR states word-formation relations paradigmatically
between \emph{lexical units}, the latter being understood in the sense of
\emph{Integrational Linguistics} (\emph{IL}) as pairings of a
paradigm and a lexical meaning (\citealt{lieb:1983:integrational:linguistics:1,lieb:1992:paradigma:klassifikation,lieb:2005:notions:paradigm}).\footnote{For a closely related word-formation theory
in the IL framework cf.\ the \emph{Process Model of Word Formation} by \citet{lieb:2013:general:theory}.}

Using this theoretical framework, it will be shown that right-headedness not
only applies to compounding products in \ili{German}, but also to certain blending
products. This claim will be based on case studies on selected compounds and
blends in spoken Modern \ili{German}. Given a word-formation relation between a
word-formation product and at least two word-formation bases, the following
subkinds of headedness will be distinguished: \begin{itemize}
\item{} the property of the product of being \emph{categorially determined}
by one of the bases and
\item{} the property of the product of being \emph{semantically determined}
by one of the bases.
\end{itemize} These properties are based on properties of the formation patterns by
means of which word-formation products are formed from word-formation bases.
``Right-headedness'' can then be reconstructed as a descriptive
term for the property of being a product that is categorially and/or
semantically determined by the last basis. In this paradigmatic approach, thus,
headedness emerges as an epiphenomenon of the relations between word-formation
products and word-formation bases.

This paper is structured as follows. Section \ref{section.units} introduces the notions of \emph{lexical
word} and \emph{lexical stem} presupposed from IL. Sections \ref{section.comp} and \ref{section.blend} discuss selected formation patterns and their
associated formation restrictions in four case studies on compounds and blends
in a system of spoken Modern \ili{German}. On the basis of these case studies,
right-headedness will be reconstructed as a purely descriptive term in PR in
Section \ref{section.conclusion}. Theoretical notions
of PR are introduced in a mostly informal way as we go along.

\section{Lexical units}
\label{section.units}
According to the IL conception, a \emph{lexical word} consists of a
word paradigm and a lexical meaning. For lexical words in spoken linguistic
systems, the following informal notation will be used in this paper:\footnote{\label{note.phonological-notation}In the phonological notation used here, the IPA
symbol ``\foreignlanguagedummy{nil}{ˈ}'' represents a
\emph{primary lexical accent}, understood as the potential of a syllable
for bearing a non-contrastive syntactic accent (\citealt{lieb:1999:was:wortakzent}); \emph{deaccented lexical accents}
(``secondary lexical accents'') are represented by the IPA
symbol ``\foreignlanguagedummy{nil}{ˌ}''. In syllables with
primary or deaccented lexical accents, the IPA symbol ``\foreignlanguagedummy{nil}{ː}'' marks vowels which are phonetically realised as
long, smoothly cut, tense vowels, while unmarked vowels in such syllables are
phonetically realised as short, abruptly cut, lax ones. Contrasts in vowel
quality between tense and lax vowels are ignored. ``\foreignlanguagedummy{nil}{{[}ə{]}}'' represents phonologically unspecified schwa, which
may, or may not, be inserted epenthetically in phonetic units; ``\foreignlanguagedummy{nil}{{[}ə{]}r}'' is the phonological representation
for syllabic vocalic {[}\foreignlanguagedummy{nil}{ɐ}{]}.
Capital ``\foreignlanguagedummy{nil}{D}'' stands for an
archiphonemic sound which is unspecified for voice (or tenseness) and is
realised as {[}\foreignlanguagedummy{nil}{d}{]} unless it
undergoes final devoicing (tensing) (cf.\ \citealt[374–375]{lieb:1999:morph:wort}).} \begin{labeledlist}{${\text{/\foreignlanguagedummy{nil}{ˈnorD{[}ə{]}n}/\/}}_{\text{‘northern region’\/}}^{\operatorname{W}}$:}
\item[${\text{/\foreignlanguagedummy{nil}{ˈnorD{[}ə{]}n}/\/}}_{\text{‘north’\/}}^{\operatorname{W}}$:] lexical word consisting of a word paradigm with the citation form
/\foreignlanguagedummy{nil}{ˈnorD{[}ə{]}n}/ and a lexical
meaning paraphrased here as ‘north’.
\item[${\text{/\foreignlanguagedummy{nil}{ˈnorD{[}ə{]}n}/\/}}_{\text{‘northern region’\/}}^{\operatorname{W}}$:] lexical word consisting of a word paradigm with the citation form
/\foreignlanguagedummy{nil}{ˈnorD{[}ə{]}n}/ and a lexical
meaning paraphrased here as ‘north region’.
\end{labeledlist} Lexical words are grammatically characterised by means of \emph{lexical
categorisations} such as: \begin{exe}
\ex \label{displayed.lexical-categorisation}\raggedright
noun in the masculine
\end{exe} Categorisations like (\ref{displayed.lexical-categorisation}) are modelled as sets of \emph{word
categories}: \begin{exe}
\ex \raggedright $\displaystyle \left\{\smash{\operatorname{Noun},\operatorname{Masc-N}}\right\}$
\end{exe} 
(``$\operatorname{Masc-N}$'' stands for ‘nominal word in the masculine’,
i.e. masculine noun or pronoun.\footnote{In this paper, a distinction is made
between \emph{nouns} and \emph{nominal words}, the latter comprising
nouns, adjectives, pronouns, and articles.} For a complete list of the
symbols cf.\ the appendix.)

A \emph{word paradigm} is a relation between \emph{word forms}
and the \emph{paradigmatic categorisations} they realise. For instance,
the word form /\foreignlanguagedummy{nil}{ˈnorD{[}ə{]}n}/ in
the word paradigm of the lexical words ${\text{/\foreignlanguagedummy{nil}{ˈnorD{[}ə{]}n}/\/}}_{\text{‘north’\/}}^{\operatorname{W}}$ and ${\text{/\foreignlanguagedummy{nil}{ˈnorD{[}ə{]}n}/\/}}_{\text{‘northern region’\/}}^{\operatorname{W}}$ realises the following paradigmatic categorisations:
\begin{exe}
\ex \label{displayed.paradigmatic-categorisations}\begin{xlist}
\ex \raggedright
nominal word form in the nominative singular
\ex \raggedright
nominal word form in the accusative singular
\ex \raggedright
nominal word form in the dative singular
\end{xlist}
\end{exe} 
Categorisations like those in (\ref{displayed.paradigmatic-categorisations}) are modelled in IL as sets of
\emph{word-form categories}: \begin{exe}
\ex \begin{xlist}
\ex \raggedright $\displaystyle \left\{\smash{\operatorname{Nom-Nf},\operatorname{Sing-Nf}}\right\}$
\ex \raggedright $\displaystyle \left\{\smash{\operatorname{Acc-Nf},\operatorname{Sing-Nf}}\right\}$
\ex \raggedright $\displaystyle \left\{\smash{\operatorname{Dat-Nf},\operatorname{Sing-Nf}}\right\}$
\end{xlist}
\end{exe} 
These sets can be thought of as specifications of corresponding
``paradigm cells''.\footnote{A closely related paradigm notion can
be found in \citeauthor{stump:2001:inflectional:morphology}’s \citeyearpar[43]{stump:2001:inflectional:morphology} Paradigm Function
Morphology (for discussion cf.\ \citealt{lieb:2005:notions:paradigm}).}

Lexical meanings like ‘north’ or ‘northern region’
are understood in IL as \emph{concepts} of a certain kind (for details cf.\
\citealt{lieb:1985:conceptual:meaning}). As a rule,
concepts are uniquely determined by their intension. The intension of
‘north’, for example, may be identified with the property of being
a direction oriented towards the North Pole.\footnote{\label{note.vector}From a
mathematical point of view, directions may be modelled as families of parallel
vectors with arbitrary length.} As problems of lexical semantics are
beyond the scope of the present paper, I won’t explicitly define lexical
meanings here; instead, an intuitive understanding of the paraphrases in single
quotation marks will be taken for granted.

In analogy to lexical words, IL conceives \emph{lexical stems} as
consisting of a stem paradigm and a lexical meaning, the latter being identical
to the lexical meaning of the corresponding lexical word (if any).\footnote{Lexical
stems without corresponding lexical words may be assumed for ``combining
forms'', insofar as the latter are best analysed not as affixes, but as
stems which are ``bound'', or ``trapped'' in
the sense of \citet{lieb:2013:general:theory}. Conversely,
there may be lexical words without corresponding lexical stems; this arguably
is the case for ``nominalised adjectives'' in Modern
\ili{German}, which are usually seen as being directly formed from adjectival words
(not stems) through syntactic conversion (for discussion cf.\ \citealt[Section 3.2.2]{nolda:2012:konversion:deutschen}).} Lexical
stems will be notated as follows in this paper: \begin{labeledlist}{${\text{/\foreignlanguagedummy{nil}{ˈnorD}/\/}}_{\text{‘northern region’\/}}^{\operatorname{St}}$:}
\item[${\text{/\foreignlanguagedummy{nil}{ˈnorD}/\/}}_{\text{‘north’\/}}^{\operatorname{St}}$:] lexical stem consisting of a stem paradigm with the citation form
/\foreignlanguagedummy{nil}{ˈnorD}/ and the lexical
meaning ‘north’.
\item[${\text{/\foreignlanguagedummy{nil}{ˈnorD}/\/}}_{\text{‘northern region’\/}}^{\operatorname{St}}$:] lexical stem consisting of a stem paradigm with the citation form
/\foreignlanguagedummy{nil}{ˈnorD}/ and the lexical
meaning ‘northern region’.
\end{labeledlist} These lexical stems can be grammatically characterised by means of the
following lexical categorisation: \begin{exe}
\ex \raggedright
noun stem in the masculine
\end{exe} 
This categorisation is modelled as a set of \emph{stem
categories}: \begin{exe}
\ex \raggedright $\displaystyle \left\{\smash{\operatorname{NounSt},\operatorname{Masc-NSt}}\right\}$
\end{exe}

\noindent
A \emph{stem paradigm} relates \emph{stem forms} to paradigmatic
categorisations consisting of \emph{stem-form categories}. According to
the view taken here, the form /\foreignlanguagedummy{nil}{ˈnorD}/ of ${\text{/\foreignlanguagedummy{nil}{ˈnorD}/\/}}_{\text{‘north’\/}}^{\operatorname{St}}$ realises the following categorisations: \begin{exe}
\ex \begin{xlist}
\ex \raggedright
nominal basic stem form
\ex \raggedright
nominal compounding stem form
\end{xlist}
\end{exe} 
Or, in set-theoretic terms: \begin{exe}
\ex \begin{xlist}
\ex \raggedright $\displaystyle \left\{\smash{\operatorname{Basic-NStf}}\right\}$
\ex \raggedright $\displaystyle \left\{\smash{\operatorname{Comp-NStf}}\right\}$
\end{xlist}
\end{exe}
/\foreignlanguagedummy{nil}{ˈnorD}/ is
a compounding stem form because it can be used as a first base form in the
formation of compounds like ${\text{/\foreignlanguagedummy{nil}{ˈnorD}/ /\foreignlanguagedummy{nil}{ˌtoːr}/\/}}_{\text{‘north gate’\/}}^{\operatorname{St}}$.\footnote{\label{note.linking-element}There are also compounding stem forms
like /\foreignlanguagedummy{nil}{ˈjaːr}/ /\foreignlanguagedummy{nil}{{[}ə{]}s}/ with a linking element.
(Actually the lexical stem ${\text{/\foreignlanguagedummy{nil}{ˈjaːr}/\/}}_{\text{‘year’\/}}^{\operatorname{St}}$ also has a compounding stem form without linking element; for
discussion, cf.\ Section \ref{section.subord-comp}.)} It is a basic stem form because from it
all stem forms in the stem paradigm can be formed, including the singular stem
form /\foreignlanguagedummy{nil}{ˈnorD}/ /\foreignlanguagedummy{nil}{{[}ə{]}n}/ and the derivation stem
form /\foreignlanguagedummy{nil}{ˈnørD}/, the latter
being used as a base form in the derivation of ${\text{/\foreignlanguagedummy{nil}{ˈnørD}/ /\foreignlanguagedummy{nil}{lix}/\/}}_{\text{‘northern’\/}}^{\operatorname{St}}$.\footnote{``\foreignlanguagedummy{nil}{x}'' denotes
the phoneme underlying both {[}\foreignlanguagedummy{nil}{ç}{]} and {[}\foreignlanguagedummy{nil}{χ}{]} in systems of spoken Modern \ili{German}.} In
contrast, the stem paradigm of ${\text{/\foreignlanguagedummy{nil}{ˈnorD}/\/}}_{\text{‘northern region’\/}}^{\operatorname{St}}$ contains the derivation stem form /\foreignlanguagedummy{nil}{ˈnorD}/, which is used as a base form in the formation
of derivates like ${\text{/\foreignlanguagedummy{nil}{ˈnorD}/ /\foreignlanguagedummy{nil}{iʃ}/\/}}_{\text{‘Nordic’\/}}^{\operatorname{St}}$. (For this conception of stem paradigms – with basic stem forms,
inflection stem forms, as well as word-formation stem forms – cf.\ \citealt[Chapter 2]{fuhrhop:1998:grenzfaelle:morphologischer}.)

As a matter of fact, then, the stem paradigms of the lexical stems ${\text{/\foreignlanguagedummy{nil}{ˈnorD}/\/}}_{\text{‘north’\/}}^{\operatorname{St}}$ and ${\text{/\foreignlanguagedummy{nil}{ˈnorD}/\/}}_{\text{‘northern region’\/}}^{\operatorname{St}}$ overlap: they share at least some form–categorisation pairs. In
addition, their lexical meanings are related through a semantic relation (viz.
metonomy). These lexical stems therefore are variants of the same
\emph{lexicological stem}, to be called ``${\text{/\foreignlanguagedummy{nil}{ˈnorD}/\/}}^{\operatorname{LSt}}$''. Similarly, the lexical words ${\text{/\foreignlanguagedummy{nil}{ˈnorD{[}ə{]}n}/\/}}_{\text{‘north’\/}}^{\operatorname{W}}$ and ${\text{/\foreignlanguagedummy{nil}{ˈnorD{[}ə{]}n}/\/}}_{\text{‘northern region’\/}}^{\operatorname{W}}$ are variants of the same \emph{lexicological word}, called
``${\text{/\foreignlanguagedummy{nil}{ˈnorD{[}ə{]}n}/\/}}^{\operatorname{LW}}$'' here. In general, lexicological units are sets of lexical
units of the same part of speech with identical or overlapping paradigms and
related lexical meanings (cf.\ \citealt{nolda:2016:formation:prepositional,nolda:2018:explaining:linguistic}).\footnote{This conception of lexical units and
lexicological units roughly corresponds to the distinction made by \citet[Chapter 3]{cruse:1986:lexical:semantics} between
``lexical units'' and ``lexemes''.} In
informal contexts, I shall denote lexical and lexicological units – be they
spoken or written – by their orthographic citation forms in italics.

Conventionalised, ``existing'' lexical units like \emph{\foreignlanguagedummy{german}{Norden}} are part of the
\emph{vocabulary} of the linguistic system; the same holds for
conventionalised word-formation products like \emph{\foreignlanguagedummy{german}{Nordtor}} and \emph{\foreignlanguagedummy{german}{nördlich}}. The vocabulary is a subset of the
\emph{lexicon} of the linguistic system, which also includes
non-conventionalised and ``non-existent'', but still
``possible'' lexical units such as \emph{\foreignlanguagedummy{german}{Nordpfeil}}: \begin{exe}
\ex \raggedright \begin{taggedline}[0.99]{(\citesource{anonymous:2012:gps-schatzsuche:koblenz})}
\begin{pairingline}
\pairing{\foreignlanguagedummy{german}{Der}}{the}
\pairing{\foreignlanguagedummy{german}{Nordpfeil}}{north.arrow}
\pairing{\foreignlanguagedummy{german}{bewegt}}{move.\textsc{\addfontfeatures{UprightFeatures={Letters=UppercaseSmallCaps}}3SG}}
\pairing{\foreignlanguagedummy{german}{sich}}{itself}
\pairing{\foreignlanguagedummy{german}{Richtung}}{direction}
\pairing{\foreignlanguagedummy{german}{Norden},}{north}
\pairing{\foreignlanguagedummy{german}{wie}}{as}
\pairing{\foreignlanguagedummy{german}{er}}{he}
\pairing{\foreignlanguagedummy{german}{soll}.}{shall.\textsc{\addfontfeatures{UprightFeatures={Letters=UppercaseSmallCaps}}3SG}}
\end{pairingline}
\end{taggedline}
\bottomline{‘The compass needle turns north as it should.’}
\end{exe} 
(``\citesource{anonymous:2012:gps-schatzsuche:koblenz}'' refers to an entry
in the list of sources.)

In the PR view, the investigation of word formation is concerned with the
formation of lexical units in the lexicon of a given linguistic system. This
heuristic principle is twofold. First, it states that monosemous lexical units,
not potentially polysemous lexicological units, are the objects of
word-formation description. This assumption is motivated by the observation that
some, but not necessarily all, variants of a lexicological unit may count as
word-formation products while others may be derived by different processes, such
as metonomy or metaphor. Second, word-formation description is not restricted to
conventionalised lexical units in the vocabulary subset of the lexicon, because
questions of conventionalisation (commonly discussed under the label of
``lexicalisation'') are orthogonal to the investigation of word
formation.

Word forms and stem forms are conceived as \emph{sequences} of
syntactic or morphological \emph{atoms}. The singular stem form
/\foreignlanguagedummy{nil}{ˈnorD}/ /\foreignlanguagedummy{nil}{{[}ə{]}n}/, for instance, is a
sequence consisting of two morphological atoms: /\foreignlanguagedummy{nil}{ˈnorD}/ and /\foreignlanguagedummy{nil}{{[}ə{]}n}/, which are phonological units in a spoken
linguistic system.\footnote{As a theory of word formation is, PR is neutral with
respect to questions of phonological representation. For the sake of this paper,
I make the minimal assumption that the phonological representation of atoms in
spoken linguistic systems not only specifies segmental phonological properties
but also suprasegmental ones, in particular syllable structures and lexical
accents (cf.\ also Note \ref{note.phonological-notation}).} Sequences with
$n$ members are modelled as total functions from \emph{position
indices} $\left\{\smash{1,…,n}\right\}$ to atoms: \begin{exe}
\ex \raggedright $\displaystyle \{\begin{array}[t]{@{\hspace{0em}}l@{\hspace{0em}}}
\left\langle \smash{1,\text{/\foreignlanguagedummy{nil}{ˈnorD}/\/}}\right\rangle ,\\
\left\langle \smash{2,\text{/\foreignlanguagedummy{nil}{{[}ə{]}n}/\/}}\right\rangle \}
\end{array}$
\end{exe} 
An alternative, non-set-theoretic, notation is given in (\ref{displayed.nord-en-sequence}): \begin{exe}
\ex
\label{displayed.nord-en-sequence}\attop{\raisebox{\dimexpr-\height+\ht\strutbox}{\psset{linewidth=0.1ex,nodesep=0.4ex,treesep=1em,levelsep=8ex,treefit=loose,tnsep=0.8ex}
\begin{psTree}[thislevelsep=0ex]{\Tn}\TR[name=d0e1479,edge={\ncangles[arm=0,angleA=-90,angleB=90]}]{\strut $1$}~*{\strut /\foreignlanguagedummy{nil}{ˈnorD}/}\TR[name=d0e1488,edge={\ncangles[arm=0,angleA=-90,angleB=90]}]{\strut $2$}~*{\strut /\foreignlanguagedummy{nil}{{[}ə{]}n}/}
\end{psTree}}}
\end{exe} 
The basic stem form /\foreignlanguagedummy{nil}{ˈnorD}/ and the pseudo-suffix\footnote{Pseudo-suffixes of
this sort are termed ``\foreignlanguagedummy{german}{morphologischer
Rest}'' by \citet[209]{eisenberg:2013:grundriss:deutschen:1}, as pointed out by an
anonymous reviewer.} /\foreignlanguagedummy{nil}{{[}ə{]}n}/, in contrast, are unit sequences, involving a
single member each: \begin{exe}
\ex \attop{\raisebox{\dimexpr-\height+\ht\strutbox}{\TR[name=d0e1514,edge=none]{\strut $1$}~*{\strut /\foreignlanguagedummy{nil}{ˈnorD}/}}}
\end{exe}
\begin{exe}
\ex \attop{\raisebox{\dimexpr-\height+\ht\strutbox}{\TR[name=d0e1525,edge=none]{\strut $1$}~*{\strut /\foreignlanguagedummy{nil}{{[}ə{]}n}/}}}
\end{exe} The same holds for the word form /\foreignlanguagedummy{nil}{ˈnorD{[}ə{]}n}/: \begin{exe}
\ex \attop{\raisebox{\dimexpr-\height+\ht\strutbox}{\TR[name=d0e1539,edge=none]{\strut $1$}~*{\strut /\foreignlanguagedummy{nil}{ˈnorD{[}ə{]}n}/}}}
\end{exe} 
In the following, I shall stick to notations like ``/\foreignlanguagedummy{nil}{ˈnorD{[}ə{]}n}/'' and ``/\foreignlanguagedummy{nil}{ˈnorD}/ /\foreignlanguagedummy{nil}{{[}ə{]}n}/'' for word and stem forms.

\largerpage
Forms can be combined in two ways. By \emph{concatenation} $⁀$, two forms are combined by adapting the position indices in the
second form without changing the overall number of atoms. For example, the
concatenation of the forms /\foreignlanguagedummy{nil}{ˈnorD}/ and /\foreignlanguagedummy{nil}{{[}ə{]}n}/ results in the form /\foreignlanguagedummy{nil}{ˈnorD}/ /\foreignlanguagedummy{nil}{{[}ə{]}n}/: \begin{exe}
\ex \raggedright
/\foreignlanguagedummy{nil}{ˈnorD}/ $⁀$ /\foreignlanguagedummy{nil}{{[}ə{]}n}/ $=$ /\foreignlanguagedummy{nil}{ˈnorD}/ /\foreignlanguagedummy{nil}{{[}ə{]}n}/
\end{exe} 
By \emph{fusion} $⁐$, the rightmost atom of the first form and the leftmost atom of the
second form are merged into one,\footnote{Merging atoms is discussed at length in
\citet{lieb:1999:morph:wort} under the label of
``phonological connection'' (``\foreignlanguagedummy{german}{phonologische
Verbindung}'').} thereby reducing the number of atoms accordingly:
\begin{exe}
\ex \raggedright
/\foreignlanguagedummy{nil}{ˈnorD}/ $⁐$ /\foreignlanguagedummy{nil}{{[}ə{]}n}/ $=$ /\foreignlanguagedummy{nil}{ˈnorD{[}ə{]}n}/
\end{exe}

\section{Compounding}
\label{section.comp}

In the Pattern-and-Restriction Theory, word formation is not described
syntagmatically in terms of relations between constituents in word structures,
such as ``heads'' and ``non-heads''; rather,
it is described paradigmatically in terms of relations between lexical units
functioning as products and bases.\footnote{In comparison to \citeauthor{anderson:1992:a-morphous:morphology}’s \citeyearpar{anderson:1992:a-morphous:morphology} \emph{A-Morphous
Morphology}, PR is both less radical and more uniform. It is less radical
because it does not away with morphological segmentation of morphological forms
into morphological atoms (``morphs''). As a matter of fact,
morphological segmentation is used in this paper as a major criterion for
distinguishing between compounding and blending (cf.\ Section \ref{section.blend} below). PR is more uniform because it does not rely on
word structures for the description of any kind of word formation, while in
A-Morphous Morphology, word structures are still assumed for
compounding.} The main difference between compounding and
blending on the one hand and other word-formation processes like derivation,
conversion, or clipping on the other hand is the number of bases involved in the
formation of a product: derivation, conversion, and clipping products are formed
through \emph{one-place} word-formation processes, involving one basis at
a time, whereas compounding and blending products are formed from two or more
bases through word-formation processes which are at least
\emph{two-place}. The same distinction holds for the formation patterns
used in PR to describe those word-formation relations: derivation, conversion
and clipping patterns are one-place, while compounding and blending patterns are
at least two-place.\footnote{The close relationship of compounding and blending is
also stated by \citet[Chapter 4]{donalies:2002:wortbildung:deutschen}, who classifies
blending even as a subtype of compounding.}

\largerpage
The treatment of compounding in PR will be illustrated in Section \ref{section.subord-comp} and \ref{section.coord-comp} below in two case studies on selected compounds,
each involving a productive compounding pattern in some system of spoken Modern
\ili{German}. Typical blending patterns are discussed in Section \ref{section.subord-blend} and \ref{section.coord-blend} in two case studies on selected blends. Of
course, these formation patterns are only a proper subset of the totality of
compounding and blending patterns in spoken Modern \ili{German}; what is more, there
will be no substantial reference to compounding or blending in written \ili{German}
(for two recent studies on that subject matter cf.\ \citealt{scherer:2013:schreibung:fenster} and \citealt{borgwaldt:2013:fugenelemente:bindestriche}). These case studies will
serve as a basis for the reconstruction of right-headedness as a descriptive
term in PR in Section \ref{section.conclusion}.

\subsection{Case study I: \emph{\foreignlanguagedummy{german}{Nordtor}}}
\label{section.subord-comp}
In the PR view, the major task of word-formation description is to explain
or predict statements of \emph{word-formation relations} between
conventionalised or non-conventionalised lexical units in a linguistic system.
Consider, for example, the word-formation relation stated in (\ref{displayed.nordtor-traditional-relation.words}), which is usually symbolised as
in (\ref{displayed.nordtor-traditional-relation.symbols}) in traditional accounts of
word formation like \citet{fleischer:et:al:2012:wortbildung:deutschen} for Modern \ili{German}:
\begin{exe}
\ex \label{displayed.nordtor-traditional-relation}\begin{xlist}
\ex \label{displayed.nordtor-traditional-relation.words}\raggedright
\emph{\foreignlanguagedummy{german}{Nordtor}} is formed from
\emph{\foreignlanguagedummy{german}{Norden}} and \emph{\foreignlanguagedummy{german}{Tor}}.
\ex \label{displayed.nordtor-traditional-relation.symbols}\raggedright $\displaystyle \text{\emph{\foreignlanguagedummy{german}{Nordtor}}\/}<\text{\emph{\foreignlanguagedummy{german}{Norden}}\/}+\text{\emph{\foreignlanguagedummy{german}{Tor}}\/}$
\end{xlist}
\end{exe} 
Using the notation for lexical words introduced in Section \ref{section.units} and the ambiguous constant ``$\mathbf{S}$'' for some specific, yet undetermined, system of spoken
Modern \ili{German}, we can reformulate these statements as follows: \begin{exe}
\ex \begin{xlist}
\ex \raggedright
${\text{/\foreignlanguagedummy{nil}{ˈnorDˌtoːr}/\/}}_{\text{‘north gate’\/}}^{\operatorname{W}}$ is formed from ${\text{/\foreignlanguagedummy{nil}{ˈnorD{[}ə{]}n}/\/}}_{\text{‘north’\/}}^{\operatorname{W}}$ and ${\text{/\foreignlanguagedummy{nil}{ˈtoːr}/\/}}_{\text{‘gate’\/}}^{\operatorname{W}}$ in $\mathbf{S}$.
\ex \raggedright $\displaystyle {\text{/\foreignlanguagedummy{nil}{ˈnorDˌtoːr}/\/}}_{\text{‘north gate’\/}}^{\operatorname{W}}<^{\mathbf{S}}{\text{/\foreignlanguagedummy{nil}{ˈnorD{[}ə{]}n}/\/}}_{\text{‘north’\/}}^{\operatorname{W}}+{\text{/\foreignlanguagedummy{nil}{ˈtoːr}/\/}}_{\text{‘gate’\/}}^{\operatorname{W}}$
\end{xlist}
\end{exe} 
An analogous word-formation relation holds between the
corresponding lexical stems: \begin{exe}
\ex \begin{xlist}
\ex \raggedright
${\text{/\foreignlanguagedummy{nil}{ˈnorD}/ /\foreignlanguagedummy{nil}{ˌtoːr}/\/}}_{\text{‘north gate’\/}}^{\operatorname{St}}$ is formed from ${\text{/\foreignlanguagedummy{nil}{ˈnorD}/\/}}_{\text{‘north’\/}}^{\operatorname{St}}$ and ${\text{/\foreignlanguagedummy{nil}{ˈtoːr}/\/}}_{\text{‘gate’\/}}^{\operatorname{St}}$ in $\mathbf{S}$.
\ex \raggedright $\displaystyle {\text{/\foreignlanguagedummy{nil}{ˈnorD}/ /\foreignlanguagedummy{nil}{ˌtoːr}/\/}}_{\text{‘north gate’\/}}^{\operatorname{St}}<^{\mathbf{S}}{\text{/\foreignlanguagedummy{nil}{ˈnorD}/\/}}_{\text{‘north’\/}}^{\operatorname{St}}+{\text{/\foreignlanguagedummy{nil}{ˈtoːr}/\/}}_{\text{‘gate’\/}}^{\operatorname{St}}$
\end{xlist}
\end{exe} 
Such word-formation relations implicitly involve a
\emph{word-formation process} and a certain \emph{formation
pattern}, which are made explicit in (\ref{displayed.nordtor-explicit-relation-between-words}) and (\ref{displayed.nordtor-explicit-relation-between-stems}):
\begin{exe}
\ex \label{displayed.nordtor-explicit-relation-between-words}\begin{xlist}
\ex \raggedright
${\text{/\foreignlanguagedummy{nil}{ˈnorDˌtoːr}/\/}}_{\text{‘north gate’\/}}^{\operatorname{W}}$ is formed from ${\text{/\foreignlanguagedummy{nil}{ˈnorD{[}ə{]}n}/\/}}_{\text{‘north’\/}}^{\operatorname{W}}$ and ${\text{/\foreignlanguagedummy{nil}{ˈtoːr}/\/}}_{\text{‘gate’\/}}^{\operatorname{W}}$ through compounding in $\mathbf{S}$ by means of Pattern \ref{pattern.subord-comp}.
\ex \raggedright $\displaystyle {\text{/\foreignlanguagedummy{nil}{ˈnorDˌtoːr}/\/}}_{\text{‘north gate’\/}}^{\operatorname{W}}<_{\operatorname{comp}\left(\smash{\text{Pattern \ref{pattern.subord-comp}\/}}\right)}^{\mathbf{S}}{\text{/\foreignlanguagedummy{nil}{ˈnorD{[}ə{]}n}/\/}}_{\text{‘north’\/}}^{\operatorname{W}}+{\text{/\foreignlanguagedummy{nil}{ˈtoːr}/\/}}_{\text{‘gate’\/}}^{\operatorname{W}}$
\end{xlist}
\end{exe}
\begin{exe}
\ex \label{displayed.nordtor-explicit-relation-between-stems}\begin{xlist}
\ex \raggedright
${\text{/\foreignlanguagedummy{nil}{ˈnorD}/ /\foreignlanguagedummy{nil}{ˌtoːr}/\/}}_{\text{‘north gate’\/}}^{\operatorname{St}}$ is formed from ${\text{/\foreignlanguagedummy{nil}{ˈnorD}/\/}}_{\text{‘north’\/}}^{\operatorname{St}}$ and ${\text{/\foreignlanguagedummy{nil}{ˈtoːr}/\/}}_{\text{‘gate’\/}}^{\operatorname{St}}$ through compounding in $\mathbf{S}$ by means of Pattern \ref{pattern.subord-comp}.
\ex \raggedright $\displaystyle {\text{/\foreignlanguagedummy{nil}{ˈnorD}/ /\foreignlanguagedummy{nil}{ˌtoːr}/\/}}_{\text{‘north gate’\/}}^{\operatorname{St}}<_{\operatorname{comp}\left(\smash{\text{Pattern \ref{pattern.subord-comp}\/}}\right)}^{\mathbf{S}}{\text{/\foreignlanguagedummy{nil}{ˈnorD}/\/}}_{\text{‘north’\/}}^{\operatorname{St}}+{\text{/\foreignlanguagedummy{nil}{ˈtoːr}/\/}}_{\text{‘gate’\/}}^{\operatorname{St}}$
\end{xlist}
\end{exe}

\noindent
According to PR, formation patterns combine for \emph{formation means}
– a \emph{formal means} (\emph{FM}), a \emph{paradigmatic
means} (\emph{PM}), a \emph{lexical means} (\emph{LM}),
and a \emph{semantic means} (\emph{SM}). Pattern \ref{pattern.subord-comp} consists of the following means: \begin{quotation}
\begin{pattern}
\label{pattern.subord-comp}\vspace{-1.25\baselineskip}
\begin{labeledlist}{PM:}
\item[FM:] \raggedright deaccentuation of the second base form and concatenation
\item[PM:] \raggedright identity with the categorisation of the second base form
\item[LM:] \raggedright identity with the categorisation of the second basis
\item[SM:] \raggedright formation of a concept according to the scheme ‘entity denoted by
the second basis in a classificatory relation to an entity denoted by the
first basis’
\end{labeledlist}
\end{pattern}
\end{quotation} Formation means are modelled as set-theoretic operations: formal
means operate on forms, paradigmatic means operate on paradigmatic
categorisations, lexical means operate on lexical categorisations, and semantic
means operate on concepts. In Pattern \ref{pattern.subord-comp}, the means are all two-place operations,
relating two arguments to one value each. Therefore, Pattern \ref{pattern.subord-comp} can be said to be two-place, too. Generally
speaking, an $n$-place formation pattern has an $n$-place formal means, an $n$-place paradigmatic means, an $n$-place lexical means, and an $n$-place semantic means.

I shall now illustrate the application of the means in Pattern \ref{pattern.subord-comp}. In order to do some, I shall use
the following semi-formal notation: \begin{exe}
\ex \raggedright
\begin{labeledlist}{$M$:}
\item[$M$:] \raggedright $x_{1}$ $+$ $x_{2}$ $↦$ $x$
\end{labeledlist}
\end{exe} 
This is to be interpreted as in (\ref{displayed.formal-application}) for arbitrary two-place
operations $M$ with $x_{1}$ and $x_{2}$ in the domain of $M$ and $x$ in the range of $M$: \begin{exe}
\ex \label{displayed.formal-application}\raggedright $\displaystyle M\left(\smash{x_{1},x_{2}}\right)=x$
\end{exe}

\noindent
FM in Pattern \ref{pattern.subord-comp}
assigns the form /\foreignlanguagedummy{nil}{ˈnorD}/ /\foreignlanguagedummy{nil}{ˌtoːr}/ with initial accent to /\foreignlanguagedummy{nil}{ˈnorD}/ and /\foreignlanguagedummy{nil}{ˈtoːr}/: \begin{exe}
\ex \raggedright
\begin{labeledlist}{FM:}
\item[FM:] \raggedright /\foreignlanguagedummy{nil}{ˈnorD}/ $+$ /\foreignlanguagedummy{nil}{ˈtoːr}/ $↦$\\*{}
/\foreignlanguagedummy{nil}{ˈnorD}/ $⁀$ /\foreignlanguagedummy{nil}{ˌtoːr}/ $=$ /\foreignlanguagedummy{nil}{ˈnorD}/ /\foreignlanguagedummy{nil}{ˌtoːr}/
\end{labeledlist}
\end{exe} 
This can be achieved in the following steps:\footnote{\label{note.extensionality}Note that this is one of many equivalent formulations
of FM in Pattern \ref{pattern.subord-comp} which all
give rise to the same set-theoretic operation, i.e. the same extensional
relation between arguments and values. What matters in PR is which arguments and
values are related by the means, not the way this is achieved. Thus, PR clearly
is a declarative theory of word formation, and not a derivational or
transformational one.} \begin{enumerate}
\item{} The second base form /\foreignlanguagedummy{nil}{ˈtoːr}/ is deaccented to /\foreignlanguagedummy{nil}{ˌtoːr}/.\footnote{\emph{Deaccentuation} replaces
primary lexical accents by deaccented lexical accents (``secondary
lexical accents'').}
\item{} The results /\foreignlanguagedummy{nil}{ˈnorD}/
and /\foreignlanguagedummy{nil}{ˌtoːr}/ are combined by
means of the concatenation operation, denoted by ``$⁀$''.
\end{enumerate} In the same way, the form /\foreignlanguagedummy{nil}{ˈnorD}/ /\foreignlanguagedummy{nil}{ˌtoːr}/ /\foreignlanguagedummy{nil}{ə}/ can be formed with FM in Pattern \ref{pattern.subord-comp}: \begin{exe}
\ex \raggedright
\begin{labeledlist}{FM:}
\item[FM:] \raggedright /\foreignlanguagedummy{nil}{ˈnorD}/ $+$ /\foreignlanguagedummy{nil}{ˈtoːr}/ /\foreignlanguagedummy{nil}{ə}/ $↦$\\*{}
/\foreignlanguagedummy{nil}{ˈnorD}/ $⁀$ /\foreignlanguagedummy{nil}{ˌtoːr}/ /\foreignlanguagedummy{nil}{ə}/ $=$ /\foreignlanguagedummy{nil}{ˈnorD}/ /\foreignlanguagedummy{nil}{ˌtoːr}/ /\foreignlanguagedummy{nil}{ə}/
\end{labeledlist}
\end{exe}

\noindent
A paradigmatic means determines the ``paradigm cells'' which
are occupied by the product forms. PM in Pattern \ref{pattern.subord-comp} does so by copying the paradigmatic
categorisation of the second base form to the product form: \begin{exe}
\ex \begin{xlist}
\ex \raggedright
\begin{labeledlist}{PM:}
\item[PM:] \raggedright $\left\{\smash{\operatorname{Comp-NStf}}\right\}$ $+$ $\left\{\smash{\operatorname{Basic-NStf}}\right\}$ $↦$ $\left\{\smash{\operatorname{Basic-NStf}}\right\}$
\end{labeledlist}
\ex \raggedright
\begin{labeledlist}{PM:}
\item[PM:] \raggedright $\left\{\smash{\operatorname{Comp-NStf}}\right\}$ $+$ $\left\{\smash{\operatorname{Sing-NStf}}\right\}$ $↦$ $\left\{\smash{\operatorname{Sing-NStf}}\right\}$
\end{labeledlist}
\ex \raggedright
\begin{labeledlist}{PM:}
\item[PM:] \raggedright $\left\{\smash{\operatorname{Comp-NStf}}\right\}$ $+$ $\left\{\smash{\operatorname{Plur-NStf}}\right\}$ $↦$ $\left\{\smash{\operatorname{Plur-NStf}}\right\}$
\end{labeledlist}
\end{xlist}
\end{exe} 
Thereby, each product form inherits its paradigmatic
categorisation from the second base form; effectively, the former also inherits
the inflection class of the latter as far as number marking is
concerned.\footnote{As a matter of fact, inheritance of paradigmatic categorisations
can also occur in derivation, as argued for in \citet[367–368]{nolda:2019:wortbildung:flexion} with reference to the
formation of prefix verbs in Modern \ili{German}.} PM in Pattern \ref{pattern.subord-comp} is an example for a
\emph{last-base-inheriting} operation, i.e. an $n$-place operation on categorisations (with $n≥2$) that copies its $n$-th argument to the value. Similarly, a first-base-inherting operation
copies its first argument to the value. First-base-inheriting operations,
last-base-inheriting operations, etc. are \emph{base-inheriting}.

LM in Pattern \ref{pattern.subord-comp} is a
last-base-inheriting operation, too, which copies the lexical categorisation of
the second basis to the product: \begin{exe}
\ex \raggedright
\begin{labeledlist}{LM:}
\item[LM:] \raggedright $\left\{\smash{\operatorname{NounSt},\operatorname{Masc-NSt}}\right\}$ $+$ $\left\{\smash{\operatorname{NounSt},\operatorname{Neut-NSt}}\right\}$ $↦$ $\left\{\smash{\operatorname{NounSt},\operatorname{Neut-NSt}}\right\}$
\end{labeledlist}
\end{exe} 
This accounts, in particular, for the fact that the lexical gender of
nominal compounds formed by means of this and other compounding patterns in
systems of Modern \ili{German} is identical to the lexical gender of the second
basis. In addition, it ensures that the part of speech of compounds is identical
to that of their second basis (which is trivially the case in noun–noun
compounds).\footnote{No attempt is made here to further classify base-inheritence
along the lines of \citet{scalise:et:al:2010:head:compounding} who distinguish between
``categorial heads'' (determining the part-of-speech category)
and ``morphological heads'' (determining other categories such
as lexical gender or inflection class).}

\largerpage
Finally, SM in Pattern \ref{pattern.subord-comp}
takes care of the \emph{word-formation meaning}, i.e. of those aspects of
the lexical meaning of the product that are word-formation-related. One
word-formation-related aspect of the meaning of ${\text{/\foreignlanguagedummy{nil}{ˈnorD}/ /\foreignlanguagedummy{nil}{ˌtoːr}/\/}}_{\text{‘north gate’\/}}^{\operatorname{St}}$ is the fact that any entity denoted by it is also denoted by the
second basis ${\text{/\foreignlanguagedummy{nil}{ˈtoːr}/\/}}_{\text{‘gate’\/}}^{\operatorname{St}}$; put differently, the second base meaning is implied by the product
meaning. SM in Pattern \ref{pattern.subord-comp}
therefore has to be a \emph{last-base-implying} operation, i.e. an
$n$-place operation on concepts (with $n≥2$) such that each of its value implies the $n$-th argument. In the case of a \emph{first-base-implying}
operation, each value implies the first argument. First-base-implying
operations, last-base-implying operations, etc. are \emph{base-implying}.
Word-formation products formed by means of a formation pattern with a
base-implying semantic means are traditionally called ``endocentric''; those formed by means of a pattern with a
semantic means that is not base-implying are called ``exocentric''.

A further word-formation-related aspect of the product meaning concerns the
relation between the base meanings in compounds like ${\text{/\foreignlanguagedummy{nil}{ˈnorD}/ /\foreignlanguagedummy{nil}{ˌtoːr}/\/}}_{\text{‘north gate’\/}}^{\operatorname{St}}$. This is a debated matter in the literature (a recent overview can be
found in \citealt{olsen:2012:semantics:compounds}).
Following \citet[316–319]{dowty:1979:word:meaning}, I
assume that ${\text{/\foreignlanguagedummy{nil}{ˈnorD}/ /\foreignlanguagedummy{nil}{ˌtoːr}/\/}}_{\text{‘north gate’\/}}^{\operatorname{St}}$ and other compounds formed by this pattern have a word-formation
meaning which involves an ``(appropriately) classificatory
relation'' between the denotata of the bases (for discussion cf.\ \citealt{downing:1977:creation:use}): \begin{exe}
\ex \raggedright
\begin{labeledlist}{SM:}
\item[SM:] \raggedright ‘north’ $+$ ‘gate’ $↦$ ‘gate in a classificatory relation to the north’
\end{labeledlist}
\end{exe}

\largerpage
\noindent
Note that the word-formation meaning ‘gate in a classificatory
relation to the north’ is underspecified with respect to the lexical
meaning of the product, which actually denotes gates on the north side of some
building. In general, PR does not require that the word-formation meaning be
identical to the lexical meaning of the product as long as the former is implied
by the latter (cf.\ \citealt{nolda:2018:explaining:linguistic}).

Taken together, the formal, paradigmatic, lexical, and semantic means in
Pattern \ref{pattern.subord-comp} specify a two-place
\emph{formation operation} on \emph{formation instances}. Formation
instances are quadruples like those in (\ref{displayed.nord-instance}), (\ref{displayed.tor-instances}), and (\ref{displayed.nordtor-instances}) combining arguments or values of the
means in Pattern \ref{pattern.subord-comp}: \begin{exe}
\ex \label{displayed.nord-instance}\raggedright $\displaystyle ⟨\begin{array}[t]{@{\hspace{0em}}l@{\hspace{0em}}}
\text{/\foreignlanguagedummy{nil}{ˈnorD}/\/},\\
\text{$\left\{\smash{\operatorname{Comp-NStf}}\right\}$\/},\\
\text{$\left\{\smash{\operatorname{NounSt},\operatorname{Masc-NSt}}\right\}$\/},\\
\text{‘north’\/}⟩
\end{array}$
\end{exe}
\begin{exe}
\ex \label{displayed.tor-instances}\begin{xlist}
\ex \label{displayed.tor-instances.basic-stem-form}\raggedright $\displaystyle ⟨\begin{array}[t]{@{\hspace{0em}}l@{\hspace{0em}}}
\text{/\foreignlanguagedummy{nil}{ˈtoːr}/\/},\\
\text{$\left\{\smash{\operatorname{Basic-NStf}}\right\}$\/},\\
\text{$\left\{\smash{\operatorname{NounSt},\operatorname{Neut-NSt}}\right\}$\/},\\
\text{‘gate’\/}⟩
\end{array}$
\ex \label{displayed.tor-instances.singular-stem-form}\raggedright $\displaystyle ⟨\begin{array}[t]{@{\hspace{0em}}l@{\hspace{0em}}}
\text{/\foreignlanguagedummy{nil}{ˈtoːr}/\/},\\
\text{$\left\{\smash{\operatorname{Sing-NStf}}\right\}$\/},\\
\text{$\left\{\smash{\operatorname{NounSt},\operatorname{Neut-NSt}}\right\}$\/},\\
\text{‘gate’\/}⟩
\end{array}$
\ex \label{displayed.tor-instances.plural-stem-form}\raggedright $\displaystyle ⟨\begin{array}[t]{@{\hspace{0em}}l@{\hspace{0em}}}
\text{/\foreignlanguagedummy{nil}{ˈtoːr}/ /\foreignlanguagedummy{nil}{ə}/\/},\\
\text{$\left\{\smash{\operatorname{Plur-NStf}}\right\}$\/},\\
\text{$\left\{\smash{\operatorname{NounSt},\operatorname{Neut-NSt}}\right\}$\/},\\
\text{‘gate’\/}⟩
\end{array}$
\end{xlist}
\end{exe}
\begin{exe}
\ex \label{displayed.nordtor-instances}\begin{xlist}
\ex \label{displayed.nordtor-instances.basic-stem-form}\raggedright $\displaystyle ⟨\begin{array}[t]{@{\hspace{0em}}l@{\hspace{0em}}}
\text{/\foreignlanguagedummy{nil}{ˈnorD}/ /\foreignlanguagedummy{nil}{ˌtoːr}/\/},\\
\text{$\left\{\smash{\operatorname{Basic-NStf}}\right\}$\/},\\
\text{$\left\{\smash{\operatorname{NounSt},\operatorname{Neut-NSt}}\right\}$\/},\\
\text{‘gate in a classificatory relation to the north’\/}⟩
\end{array}$
\ex \label{displayed.nordtor-instances.singular-stem-form}\raggedright $\displaystyle ⟨\begin{array}[t]{@{\hspace{0em}}l@{\hspace{0em}}}
\text{/\foreignlanguagedummy{nil}{ˈnorD}/ /\foreignlanguagedummy{nil}{ˌtoːr}/\/},\\
\text{$\left\{\smash{\operatorname{Sing-NStf}}\right\}$\/},\\
\text{$\left\{\smash{\operatorname{NounSt},\operatorname{Neut-NSt}}\right\}$\/},\\
\text{‘gate in a classificatory relation to the north’\/}⟩
\end{array}$
\ex \label{displayed.nordtor-instances.plural-stem-form}\raggedright $\displaystyle ⟨\begin{array}[t]{@{\hspace{0em}}l@{\hspace{0em}}}
\text{/\foreignlanguagedummy{nil}{ˈnorD}/ /\foreignlanguagedummy{nil}{ˌtoːr}/ /\foreignlanguagedummy{nil}{ə}/\/},\\
\text{$\left\{\smash{\operatorname{Plur-NStf}}\right\}$\/},\\
\text{$\left\{\smash{\operatorname{NounSt},\operatorname{Neut-NSt}}\right\}$\/},\\
\text{‘gate in a classificatory relation to the north’\/}⟩
\end{array}$
\end{xlist}
\end{exe} 
The formation instances in (\ref{displayed.nord-instance}), (\ref{displayed.tor-instances}), and (\ref{displayed.nordtor-instances}) \emph{instantiate} the bases
and products involved in the word-formation relation (\ref{displayed.nordtor-explicit-relation-between-stems}): the \emph{base
instance} (\ref{displayed.nord-instance})
instantiates the first basis ${\text{/\foreignlanguagedummy{nil}{ˈnorD}/\/}}_{\text{‘north’\/}}^{\operatorname{St}}$, the \emph{base instances} (\ref{displayed.tor-instances.basic-stem-form}), (\ref{displayed.tor-instances.singular-stem-form}), and (\ref{displayed.tor-instances.plural-stem-form}) each
instantiate the second basis ${\text{/\foreignlanguagedummy{nil}{ˈtoːr}/\/}}_{\text{‘gate’\/}}^{\operatorname{St}}$, and the \emph{product instances} (\ref{displayed.nordtor-instances.basic-stem-form}), (\ref{displayed.nordtor-instances.singular-stem-form}), and (\ref{displayed.nordtor-instances.plural-stem-form}) instantiate the product
${\text{/\foreignlanguagedummy{nil}{ˈnorD}/ /\foreignlanguagedummy{nil}{ˌtoːr}/\/}}_{\text{‘north gate’\/}}^{\operatorname{St}}$. The first and second components of those formation instances
represent formal and categorial properties of one of their forms, while the
third and fourth components represent categorial and semantic properties of the
lexical unit itself.

%\largerpage[2]
The formation operation specified by Pattern \ref{pattern.subord-comp} takes base-instance pairs like $\left\langle \smash{\text{(\ref{displayed.nord-instance})\/},\text{(\ref{displayed.tor-instances.basic-stem-form})\/}}\right\rangle $, $\left\langle \smash{\text{(\ref{displayed.nord-instance})\/},\text{(\ref{displayed.tor-instances.singular-stem-form})\/}}\right\rangle $, and $\left\langle \smash{\text{(\ref{displayed.nord-instance})\/},\text{(\ref{displayed.tor-instances.plural-stem-form})\/}}\right\rangle $ as arguments and assigns to them the product instances (\ref{displayed.nordtor-instances.basic-stem-form}), (\ref{displayed.nordtor-instances.singular-stem-form}), and (\ref{displayed.nordtor-instances.plural-stem-form}), respectively. From a logical
point of view, there is nothing that would exclude base-instance pairs in the
domain of this formation operation where the first base instance is, say, a
singular stem form like /\foreignlanguagedummy{nil}{ˈnorD}/ /\foreignlanguagedummy{nil}{{[}ə{]}n}/; this, however, is excluded on empirical
grounds. In addition, it must be taken care of co-occurrence restrictions on
base instances. For example, the compounding stem form /\foreignlanguagedummy{nil}{ˈjaːr}/, occurring in just a few conventionalised
compounds like ${\text{/\foreignlanguagedummy{nil}{ˈjaːr}/ /\foreignlanguagedummy{nil}{ˌbuːx}/\/}}_{\text{‘yearbook’\/}}^{\operatorname{St}}$, is compatible only with a handful of stem forms, whereas compounding
with the compounding stem form /\foreignlanguagedummy{nil}{ˈjaːr}/ /\foreignlanguagedummy{nil}{{[}ə{]}s}/ (already mentioned cf.\ Note \ref{note.linking-element} in Section \ref{section.units}) is fully productive. Last, but not least, the base
instances in the domain of our formation operation have to be restricted to
instances of noun stems.

Thus, only a proper subset of the domain of the formation operation specified
by Pattern \ref{pattern.subord-comp} is actually used
for word formation in $\mathbf{S}$. This subset is the \emph{formation restriction} which is
associated with Pattern \ref{pattern.subord-comp} in
$\mathbf{S}$. It restricts what bases are available for word formation in
$\mathbf{S}$ by means of Pattern \ref{pattern.subord-comp}; indirectly, it also restricts what products
which can be formed in $\mathbf{S}$ from those bases by means of the pattern.

The formation restriction associated with Pattern \ref{pattern.subord-comp} in $\mathbf{S}$ can partially or totally be identified in a word-formation
grammar of $\mathbf{S}$ in terms of declarative constraints like those in Restriction \ref{restriction.subord-comp}, which consists of a
\emph{formal constraint} (\emph{FC}), a \emph{paradigmatic
constraint} (\emph{PC}),\footnote{The paradigmatic constraint in
Restriction \ref{restriction.subord-comp} effectively
excludes $\operatorname{Comp-NStf}$ and $\operatorname{Der-NStf}$ from the paradigmatic categorisation of the second base form because,
at least in Modern \ili{German} systems, word-formation stem forms are not necessarily
inherited by the product. For instance, the only compounding stem form in the
paradigm of ${\text{/\foreignlanguagedummy{nil}{ˈbeːr}/\/}}_{\text{‘berry’\/}}^{\operatorname{St}}$ is /\foreignlanguagedummy{nil}{ˈbeːr}/ /\foreignlanguagedummy{nil}{{[}ə{]}n}/; in the paradigm of the compound
${\text{/\foreignlanguagedummy{nil}{ˈerD}/ /\foreignlanguagedummy{nil}{ˌbeːr}/\/}}_{\text{‘strawberry’\/}}^{\operatorname{St}}$, however, the compounding stem form is /\foreignlanguagedummy{nil}{ˈerD}/ /\foreignlanguagedummy{nil}{ˌbeːr}/, not /\foreignlanguagedummy{nil}{ˈerD}/ /\foreignlanguagedummy{nil}{ˌbeːr}/ /\foreignlanguagedummy{nil}{{[}ə{]}n}/. A similar point can be made for
${\text{/\foreignlanguagedummy{nil}{ˈfrau}/\/}}_{\text{‘woman’\/}}^{\operatorname{St}}$ and ${\text{/\foreignlanguagedummy{nil}{ˈjuŋ}/ /\foreignlanguagedummy{nil}{ˌfrau}/\/}}_{\text{‘virgin’\/}}^{\operatorname{St}}$: the former has both a derivation stem form with umlaut (as in
/\foreignlanguagedummy{nil}{ˈfroi}/ /\foreignlanguagedummy{nil}{ˌlain}/) and a derivation stem
form without umlaut (as in /\foreignlanguagedummy{nil}{ˈfrau}/ /\foreignlanguagedummy{nil}{lix}/), whereas the latter only has a derivation
stem form with umlaut (/\foreignlanguagedummy{nil}{ˈjuŋ}/ /\foreignlanguagedummy{nil}{ˌfroi}/ /\foreignlanguagedummy{nil}{lix}/).} and a \emph{lexical
constraint} (\emph{LC}): \begin{quotation}
\begin{restriction}
\label{restriction.subord-comp}\vspace{-1.25\baselineskip}
\begin{labeledlist}{PC:}
\item[FC:] \raggedright The base forms are compatible.
\item[PC:] \raggedright The paradigmatic categorisation of the first base form contains $\operatorname{Comp-NStf}$.\\*{}
The paradigmatic categorisation of the second base form contains $\operatorname{Basic-NStf}$, $\operatorname{Sing-NStf}$, or $\operatorname{Plur-NStf}$.
\item[LC:] \raggedright The lexical categorisations of the bases contain $\operatorname{NounSt}$.
\end{labeledlist}
\end{restriction}
\end{quotation}
%\largerpage[2]
In other cases, there may be reason to include a \emph{semantic
constraint} (\emph{SC}) or a \emph{general constraint}
(\emph{GC}).\footnote{An example for a general constraint would be the
requirement that the second basis of a compound must be a compound itself. Such
a constraint is needed for the formation restriction associated with the
compounding pattern by means of which a product like ${\text{/\foreignlanguagedummy{nil}{ˌlanD}/ /\foreignlanguagedummy{nil}{{[}ə{]}s}/ /\foreignlanguagedummy{nil}{ˈʃuːl}/ /\foreignlanguagedummy{nil}{ˌamt}/\/}}_{\text{‘federal education authority’\/}}^{\operatorname{St}}$ with non-initial accent is formed from the noun stem ${\text{/\foreignlanguagedummy{nil}{ˈlanD}/\/}}_{\text{‘federal state’\/}}^{\operatorname{St}}$ and the compound ${\text{/\foreignlanguagedummy{nil}{ˈʃuːl}/ /\foreignlanguagedummy{nil}{ˌamt}/\/}}_{\text{‘education authority’\/}}^{\operatorname{St}}$. Of course, apart from this general constraint, there are further, in
particular semantic, constraints (cf.\ the study of \citealt{benware:1987:accent:variation}).}
As a matter of fact, all of
the above constraints are input-related, applying to components of base
instances. In other cases there may also be output-related constraints on the
product instances which the formation operation specified by Pattern \ref{pattern.subord-comp} assigns to the base instances
(cf.\ Section \ref{section.blend}.)

Word-formation processes are conceived in PR as one-place functions from
$n$-place formation patterns to the $n$-place formation operations specified by the patterns; the
word-formation processes are said to be $n$-place themselves. The word-formation process of two-place
compounding ($\operatorname{comp}^{2}$), for example, is a function from two-place formation patterns like
Pattern \ref{pattern.subord-comp} to the two-place
formation operations specified by them. As a rule, two-place compounding is
involved in word-formation relations like (\ref{displayed.nordtor-explicit-relation-between-words}) and (\ref{displayed.nordtor-explicit-relation-between-stems})
between two bases and one product. When the arity is clear from the context, I
shall continue to speak of ``compounding''
(``$\operatorname{comp}$'') \emph{\foreignlanguagedummy{french}{tout court}}.

Given this conception, the word-formation relation stated in (\ref{displayed.nordtor-explicit-relation-between-stems})
can be logically derived in PR from the word-formation theory and a
word-formation grammar of $\mathbf{S}$. This derivation requires, in particular, that the following
conditions hold: \begin{enumerate}
\item{} There is a base-instance pair instantiating ${\text{/\foreignlanguagedummy{nil}{ˈnorD}/\/}}_{\text{‘north’\/}}^{\operatorname{St}}$ and ${\text{/\foreignlanguagedummy{nil}{ˈtoːr}/\/}}_{\text{‘gate’\/}}^{\operatorname{St}}$ in the formation restriction associated with Pattern \ref{pattern.subord-comp} in $\mathbf{S}$.
\item{} The formation process specified by Pattern \ref{pattern.subord-comp} assigns to those base instances a product
instance instantiating ${\text{/\foreignlanguagedummy{nil}{ˈnorD}/ /\foreignlanguagedummy{nil}{ˌtoːr}/\/}}_{\text{‘north gate’\/}}^{\operatorname{St}}$.
\item{} The word-formation process $\operatorname{comp}$ in $\mathbf{S}$ assigns this formation process to Pattern \ref{pattern.subord-comp}.
\end{enumerate} In the case at hand, there are three base-instance pairs which fulfil
these conditions together with one product instance each: \begin{itemize}
\item{} the base-instance pair $\left\langle \smash{\text{(\ref{displayed.nord-instance})\/},\text{(\ref{displayed.tor-instances.basic-stem-form})\/}}\right\rangle $ with the product instance (\ref{displayed.nordtor-instances.basic-stem-form});
\item{} the base-instance pair $\left\langle \smash{\text{(\ref{displayed.nord-instance})\/},\text{(\ref{displayed.tor-instances.singular-stem-form})\/}}\right\rangle $ with the product instance (\ref{displayed.nordtor-instances.singular-stem-form});
\item{} the base-instance pair $\left\langle \smash{\text{(\ref{displayed.nord-instance})\/},\text{(\ref{displayed.tor-instances.plural-stem-form})\/}}\right\rangle $ with the product instance (\ref{displayed.nordtor-instances.plural-stem-form}).
\end{itemize} Each of them can be used for explaining or predicting the word-formation
relation (\ref{displayed.nordtor-explicit-relation-between-stems}) in PR (for the
logic of explanation and prediction in PR cf.\ \citealt{nolda:2018:explaining:linguistic}).\footnote{Since PR does not presuppose
any word structures which represent the formation history, forms like
/\foreignlanguagedummy{nil}{ˈnorD}/ /\foreignlanguagedummy{nil}{ˌtoːr}/ /\foreignlanguagedummy{nil}{ə}/ and their categorisation as plural stem form
can be motivated by word-formation as well as by inflection. In the former case,
exemplified above, the stem form /\foreignlanguagedummy{nil}{ˈnorD}/ /\foreignlanguagedummy{nil}{ˌtoːr}/ /\foreignlanguagedummy{nil}{ə}/ is formed from the stem forms /\foreignlanguagedummy{nil}{ˈnorD}/ and /\foreignlanguagedummy{nil}{ˈtoːr}/ /\foreignlanguagedummy{nil}{ə}/ by means of FM in Pattern \ref{pattern.subord-comp}, and its paradigmatic
categorisation $\left\{\smash{\operatorname{Plur-NStf}}\right\}$ is copied by PM in Pattern \ref{pattern.subord-comp} from the paradigmatic
categorisation of /\foreignlanguagedummy{nil}{ˈtoːr}/ /\foreignlanguagedummy{nil}{ə}/. In the latter case, /\foreignlanguagedummy{nil}{ˈnorD}/ /\foreignlanguagedummy{nil}{ˌtoːr}/ /\foreignlanguagedummy{nil}{ə}/ is formed from the stem form /\foreignlanguagedummy{nil}{ˈnorD}/ /\foreignlanguagedummy{nil}{ˌtoːr}/ by the formal means in a certain
inflectional formation pattern (``plural formation by means of
suffixation with /\foreignlanguagedummy{nil}{ə}/''), and the paradigmatic means in that
pattern determines its categorisation as plural stem form. (A similar point is
made in \citet[369]{nolda:2019:wortbildung:flexion} for
the formation of past-tense stem forms of prefix verbs in Modern
\ili{German}.)}

Word-formation relations obtained in this way are \emph{direct}
word-formation relations. Such word-formation relations can be explicitly stated
in PR as follows: \begin{exe}
\ex \label{displayed.nordtor-direct-relation}\begin{xlist}
\ex \raggedright
${\text{/\foreignlanguagedummy{nil}{ˈnorD}/ /\foreignlanguagedummy{nil}{ˌtoːr}/\/}}_{\text{‘north gate’\/}}^{\operatorname{St}}$ is directly formed from ${\text{/\foreignlanguagedummy{nil}{ˈnorD}/\/}}_{\text{‘north’\/}}^{\operatorname{St}}$ and ${\text{/\foreignlanguagedummy{nil}{ˈtoːr}/\/}}_{\text{‘gate’\/}}^{\operatorname{St}}$ through compounding in $\mathbf{S}$ by means of Pattern \ref{pattern.subord-comp}.
\ex \raggedright $\displaystyle {\text{/\foreignlanguagedummy{nil}{ˈnorD}/ /\foreignlanguagedummy{nil}{ˌtoːr}/\/}}_{\text{‘north gate’\/}}^{\operatorname{St}}⪪_{\operatorname{comp}\left(\smash{\text{Pattern \ref{pattern.subord-comp}\/}}\right)}^{\mathbf{S}}{\text{/\foreignlanguagedummy{nil}{ˈnorD}/\/}}_{\text{‘north’\/}}^{\operatorname{St}}+{\text{/\foreignlanguagedummy{nil}{ˈtoːr}/\/}}_{\text{‘gate’\/}}^{\operatorname{St}}$
\end{xlist}
\end{exe} 
From this direct word-formation relation between the lexical
stems ${\text{/\foreignlanguagedummy{nil}{ˈnorD}/\/}}_{\text{‘north’\/}}^{\operatorname{St}}$, ${\text{/\foreignlanguagedummy{nil}{ˈtoːr}/\/}}_{\text{‘gate’\/}}^{\operatorname{St}}$, and ${\text{/\foreignlanguagedummy{nil}{ˈnorD}/ /\foreignlanguagedummy{nil}{ˌtoːr}/\/}}_{\text{‘north gate’\/}}^{\operatorname{St}}$ the \emph{indirect} word-formation relation (\ref{displayed.nordtor-indirect-relation}) between the
corresponding lexical words can likewise be logically derived in PR (for details
cf.\ again \citealt{nolda:2018:explaining:linguistic}):
\begin{exe}
\ex \label{displayed.nordtor-indirect-relation}\begin{xlist}
\ex \raggedright
${\text{/\foreignlanguagedummy{nil}{ˈnorDˌtoːr}/\/}}_{\text{‘north gate’\/}}^{\operatorname{W}}$ is indirectly formed from ${\text{/\foreignlanguagedummy{nil}{ˈnorD{[}ə{]}n}/\/}}_{\text{‘north’\/}}^{\operatorname{W}}$ and ${\text{/\foreignlanguagedummy{nil}{ˈtoːr}/\/}}_{\text{‘gate’\/}}^{\operatorname{W}}$ through compounding in $\mathbf{S}$ by means of Pattern \ref{pattern.subord-comp}.
\ex \raggedright $\displaystyle {\text{/\foreignlanguagedummy{nil}{ˈnorDˌtoːr}/\/}}_{\text{‘north gate’\/}}^{\operatorname{W}}⋖_{\operatorname{comp}\left(\smash{\text{Pattern \ref{pattern.subord-comp}\/}}\right)}^{\mathbf{S}}{\text{/\foreignlanguagedummy{nil}{ˈnorD{[}ə{]}n}/\/}}_{\text{‘north’\/}}^{\operatorname{W}}+{\text{/\foreignlanguagedummy{nil}{ˈtoːr}/\/}}_{\text{‘gate’\/}}^{\operatorname{W}}$
\end{xlist}
\end{exe}

\subsection{Case study II: \emph{\foreignlanguagedummy{german}{Nordosten}}}
\label{section.coord-comp}
The object of the next case study is the compound \emph{\foreignlanguagedummy{german}{Nordosten}}. The lexical word ${\text{/\foreignlanguagedummy{nil}{ˌnorDˈost{[}ə{]}n}/\/}}_{\text{‘north-east’\/}}^{\operatorname{W}}$ and its stem ${\text{/\foreignlanguagedummy{nil}{ˌnorD}/ /\foreignlanguagedummy{nil}{ˈost}/\/}}_{\text{‘north-east’\/}}^{\operatorname{St}}$ are formed as follows in the linguistic system $\mathbf{S}$ under discussion: \begin{exe}
\ex \raggedright $\displaystyle {\text{/\foreignlanguagedummy{nil}{ˌnorDˈost{[}ə{]}n}/\/}}_{\text{‘north-east’\/}}^{\operatorname{W}}⋖_{\operatorname{comp}\left(\smash{\text{Pattern \ref{pattern.coord-comp}\/}}\right)}^{\mathbf{S}}{\text{/\foreignlanguagedummy{nil}{ˈnorD{[}ə{]}n}/\/}}_{\text{‘north’\/}}^{\operatorname{W}}+{\text{/\foreignlanguagedummy{nil}{ˈost{[}ə{]}n}/\/}}_{\text{‘east’\/}}^{\operatorname{W}}$
\end{exe}
\begin{exe}
\ex \label{displayed.nordosten-direct-relation}\raggedright $\displaystyle {\text{/\foreignlanguagedummy{nil}{ˌnorD}/ /\foreignlanguagedummy{nil}{ˈost}/\/}}_{\text{‘north-east’\/}}^{\operatorname{St}}⪪_{\operatorname{comp}\left(\smash{\text{Pattern \ref{pattern.coord-comp}\/}}\right)}^{\mathbf{S}}{\text{/\foreignlanguagedummy{nil}{ˈnorD}/\/}}_{\text{‘north’\/}}^{\operatorname{St}}+{\text{/\foreignlanguagedummy{nil}{ˈost}/\/}}_{\text{‘east’\/}}^{\operatorname{St}}$
\end{exe} 
Pattern \ref{pattern.coord-comp} consists of
the following means: \begin{quotation}
\begin{pattern}
\label{pattern.coord-comp}\vspace{-1.25\baselineskip}
\begin{labeledlist}{PM:}
\item[FM:] \raggedright deaccentuation of the first base form and concatenation
\item[PM:] \raggedright identity with the categorisation of the second base form
\item[LM:] \raggedright identity with the categorisation of the second basis
\item[SM:] \raggedright formation of a concept according to the scheme ‘sum of the entities
denoted by the bases’
\end{labeledlist}
\end{pattern}
\end{quotation}
%\largerpage
The formation restriction associated with Pattern \ref{pattern.coord-comp} in $\mathbf{S}$ satisfies
the constraints in Restriction \ref{restriction.coord-comp}:
% on the next page.
\pagebreak[2]
%\begin{figure}
\begin{quotation}
\begin{restriction}
\label{restriction.coord-comp}\vspace{-1.25\baselineskip}
\begin{labeledlist}{PC:}
\item[FC:] \raggedright The base forms are compatible.
\item[PC:] \raggedright The paradigmatic categorisation of the first base form contains $\operatorname{Basic-NStf}$.\\*{}
The paradigmatic categorisation of the second base form contains $\operatorname{Basic-NStf}$, $\operatorname{Sing-NStf}$, or $\operatorname{Plur-NStf}$.
\item[LC:] \raggedright The lexical categorisations of the bases contain $\operatorname{NounSt}$.
\item[SC:] \raggedright The bases denote entities of the same sort for which a sum operation is
defined.
\end{labeledlist}
\end{restriction}
\end{quotation}
%\vspace{-1\baselineskip}
%\end{figure}
Note that PC in Restriction \ref{restriction.coord-comp} requires that the first base form is
categorised as a basic stem form. By this requirement it is predicted that there
are no specific compounding stem forms – and thus no linking elements –
occurring in compounds of this type.\footnote{A similar point is made by \citet[29]{becker:1992:compounding:german} with respect to
copulative compounds like \emph{\foreignlanguagedummy{german}{Fürstbischof}}, which have no linking elements
either.} SC ensures that the semantic types of the base concepts are
appropriate for SM in Pattern \ref{pattern.coord-comp}.

FM in Pattern \ref{pattern.coord-comp} differs from
the formal means in Pattern \ref{pattern.subord-comp}
only with respect to accentuation. In the present example, the product forms
/\foreignlanguagedummy{nil}{ˌnorD}/ /\foreignlanguagedummy{nil}{ˈost}/ and \mbox{/\foreignlanguagedummy{nil}{ˌnorD}/} /\foreignlanguagedummy{nil}{ˈost}/ /\foreignlanguagedummy{nil}{{[}ə{]}n}/ have final accent: \begin{exe}
\ex \begin{xlist}
\ex \raggedright
\begin{labeledlist}{FM:}
\item[FM:] \raggedright /\foreignlanguagedummy{nil}{ˈnorD}/ $+$ /\foreignlanguagedummy{nil}{ˈost}/ $↦$\\*{}
/\foreignlanguagedummy{nil}{ˌnorD}/ $⁀$ /\foreignlanguagedummy{nil}{ˈost}/ $=$ /\foreignlanguagedummy{nil}{ˌnorD}/ /\foreignlanguagedummy{nil}{ˈost}/
\end{labeledlist}
\ex \raggedright
\begin{labeledlist}{FM:}
\item[FM:] \raggedright /\foreignlanguagedummy{nil}{ˈnorD}/ $+$ /\foreignlanguagedummy{nil}{ˈost}/ /\foreignlanguagedummy{nil}{{[}ə{]}n}/ $↦$\\*{}
/\foreignlanguagedummy{nil}{ˌnorD}/ $⁀$ /\foreignlanguagedummy{nil}{ˈost}/ /\foreignlanguagedummy{nil}{{[}ə{]}n}/ $=$ /\foreignlanguagedummy{nil}{ˌnorD}/ /\foreignlanguagedummy{nil}{ˈost}/ /\foreignlanguagedummy{nil}{{[}ə{]}n}/
\end{labeledlist}
\end{xlist}
\end{exe}

%\largerpage[-1]
\noindent
PM and LM in Pattern \ref{pattern.coord-comp} are identical to the paradigmatic and lexical
means in Pattern \ref{pattern.subord-comp}. These
last-base-inheriting operations copy their second argument to the value:
\begin{exe}
\ex \begin{xlist}
\ex \raggedright
\begin{labeledlist}{PM:}
\item[PM:] \raggedright $\left\{\smash{\operatorname{Basic-NStf}}\right\}$ $+$ $\left\{\smash{\operatorname{Basic-NStf}}\right\}$ $↦$ $\left\{\smash{\operatorname{Basic-NStf}}\right\}$
\end{labeledlist}
\ex \raggedright
\begin{labeledlist}{PM:}
\item[PM:] \raggedright $\left\{\smash{\operatorname{Basic-NStf}}\right\}$ $+$ $\left\{\smash{\operatorname{Sing-NStf}}\right\}$ $↦$ $\left\{\smash{\operatorname{Sing-NStf}}\right\}$
\end{labeledlist}
\end{xlist}
\end{exe}
\begin{exe}
\ex \raggedright
\begin{labeledlist}{LM:}
\item[LM:] \raggedright $\left\{\smash{\operatorname{NounSt},\operatorname{Masc-NSt}}\right\}$ $+$ $\left\{\smash{\operatorname{NounSt},\operatorname{Masc-NSt}}\right\}$ $↦$ $\left\{\smash{\operatorname{NounSt},\operatorname{Masc-NSt}}\right\}$
\end{labeledlist}
\end{exe}
In the example at hand, LM in Pattern \ref{pattern.coord-comp} happens to apply trivially since both bases have
the same lexical gender.

The main difference between Pattern \ref{pattern.subord-comp} and \ref{pattern.coord-comp} arguably is the semantic means. SM in Pattern \ref{pattern.coord-comp} constructs concepts expressing the
sum of the entities denoted by the bases: \begin{exe}
\ex \raggedright
\begin{labeledlist}{SM:}
\item[SM:] \raggedright ‘north’ $+$ ‘east’ $↦$ ‘sum of north and east’
\end{labeledlist}
\end{exe} 
Here, ``sum'' is understood in a broad
sense covering arithmetic sum as in the formation of the ``complex
numeral'' \emph{\foreignlanguagedummy{german}{hunderteins}},
mereological sum as in the formation of the ``fusional
compound'' \emph{\foreignlanguagedummy{german}{Mecklenburg-Vorpommern}}, sum of directions as in the
formation of the ``intermediate-denoting compound'' \emph{\foreignlanguagedummy{german}{Nordosten}}, etc.\footnote{\hspace{1pt}``Complex numeral'', ``fusional
compound'', and ``intermediate-denoting
compound'' are \citeauthor{waelchli:2005:co-compounds:natural}’s \citeyearpar{waelchli:2005:co-compounds:natural} descriptive terms. In general,
such compounds are exocentric, while ``appositional compounds''
like \ili{English} \emph{singer-bassist} are endocentric
(for discussion, cf.\ \citealt{olsen:2014:coordinative:structures}).} As illustrated in
Figure \ref{figure.direction-sum}, the sum operation on
directions works in an analogous way to the sum operation on vectors, the only
difference being that vectors have a length and an orientation, whereas
directions have an orientation only (cf.\ Note \ref{note.vector} in Section \ref{section.units}
above). Obviously, SM in Pattern \ref{pattern.coord-comp} is not a base-implying operation: the direction
denoted by the product is denoted neither by the first basis nor by the second
basis. This semantic means has another characteristic property instead – it is a
\emph{commutative} operation, i.e. an $n$-place operation (with $n≥2$) whose values are independent of the order of its arguments: \begin{exe}
\ex \raggedright
\begin{labeledlist}{SM:}
\item[SM:] \raggedright ‘east’ $+$ ‘north’ $↦$
‘sum of east and north’ $=$ ‘sum of north and east’
\end{labeledlist}
\end{exe}
\begin{figure}[tp]\centering \psset{linewidth=0.1ex,nodesep=0.4ex}
\begin{psmatrix}[rowsep=9em,colsep=10.5em]
[name=figure0direction1sum0north] & [name=figure0direction1sum0north1east]\\
[name=figure0direction1sum0center] & [name=figure0direction1sum0east]
\end{psmatrix}
\ncdiag[arm=0,angleA=0,angleB=0,linestyle=dashed,dash=1ex 0.5ex,arrowscale=1.5]{->}{figure0direction1sum0center}{figure0direction1sum0north}\naput[labelsep=0.4ex,nrot=:U]{\strut north}
\ncdiag[arm=0,angleA=0,angleB=180,linestyle=dashed,dash=1ex 0.5ex,arrowscale=1.5]{->}{figure0direction1sum0center}{figure0direction1sum0east}\nbput[labelsep=0.4ex,nrot=:U]{\strut east}
\ncdiag[arm=0,angleA=0,angleB=180,linestyle=dashed,dash=1ex 0.5ex,arrowscale=1.5]{->}{figure0direction1sum0center}{figure0direction1sum0north1east}\naput[labelsep=0.4ex,nrot=:U,npos=1.53]{\strut sum of north and east $=$}\nbput[labelsep=0.4ex,nrot=:U]{\strut sum of east and north}
\caption{Sum of directions}
\label{figure.direction-sum}
\end{figure}

%\largerpage[-1]
\largerpage
\noindent
Commutativity of semantic means can be used to distinguish in arbitrary
linguistic systems $S$ between coordinative word formation (like the formation of ${\text{/\foreignlanguagedummy{nil}{ˌnorD}/ /\foreignlanguagedummy{nil}{ˈost}/\/}}_{\text{‘north-east’\/}}^{\operatorname{St}}$ through compounding in $S$ by means of Pattern \ref{pattern.coord-comp}) and subordinative word formation (like the
formation of ${\text{/\foreignlanguagedummy{nil}{ˈnorD}/ /\foreignlanguagedummy{nil}{ˌtoːr}/\/}}_{\text{‘north gate’\/}}^{\operatorname{St}}$ through compounding in $S$ by means of Pattern \ref{pattern.subord-comp}): \begin{quotation}
\begin{definition}
\label{definition.coordinative-wf}Let $n≥2$.

\noindent $n$-place \emph{coordinative word-formation} in $S$ is that $n$-place word-formation process in $S$ whose arguments are all $n$-place formation patterns in $S$ with a semantic means that is commutative.
\end{definition}
\end{quotation}
%\pagebreak
\begin{quotation}
\begin{definition}
Let $n≥2$.

\noindent $n$-place \emph{subordinative word-formation} in $S$ is that $n$-place word-formation process in $S$ whose arguments are all $n$-place formation patterns in $S$ with a semantic means that is not commutative.
\end{definition}
\end{quotation} By applying this terminology to the products themselves, we can
distinguish between coordinative (or ``copulative'') products
and subordinative (or ``determinative'') products in $S$ being formed through of coordinative or subordinative word formation
in $S$, respectively. The compound ${\text{/\foreignlanguagedummy{nil}{ˌnorD}/ /\foreignlanguagedummy{nil}{ˈost}/\/}}_{\text{‘north-east’\/}}^{\operatorname{St}}$, then, is a coordinative compound in $\mathbf{S}$ because it is formed through coordinative compounding in $\mathbf{S}$, while ${\text{/\foreignlanguagedummy{nil}{ˈnorD}/ /\foreignlanguagedummy{nil}{ˌtoːr}/\/}}_{\text{‘north gate’\/}}^{\operatorname{St}}$ is a subordinative compound in $\mathbf{S}$ formed through subordinative compounding in $\mathbf{S}$. Note that for ${\text{/\foreignlanguagedummy{nil}{ˌnorD}/ /\foreignlanguagedummy{nil}{ˈost}/\/}}_{\text{‘north-east’\/}}^{\operatorname{St}}$ being a coordinative compound in $\mathbf{S}$, it is both necessary and sufficient to be formed through
coordinative compounding in $\mathbf{S}$ – i.e. through compounding by means of a formation pattern with a
commutative semantic means; it is irrelevant, however, whether or not there is
in $\mathbf{S}$ a conventionalised synonymous lexical unit ${\text{/\foreignlanguagedummy{nil}{ˌost}/ /\foreignlanguagedummy{nil}{ˈnorD}/\/}}_{\text{‘east-north’ $=$ ‘north-east’\/}}^{\operatorname{St}}$ with ``reversed bases''.

\section{Blending}
\label{section.blend}
It is a debated matter in the literature whether blends result from word
formation or word creation. Proponents of the latter position cite as arguments:
deliberate formation, deviant patterns, unpredictable forms, and more or less
intransparent meanings (cf., in particular, \citealt{ronneberger-sibold:2006:lexical:blends,ronneberger-sibold:2015:word:creation}). Others argue that blending is a
word-formation process \emph{sui generis} with specific,
but systematic, formation patterns (an opinion hold, \emph{inter
alia}, by \citealt{mueller:et:al:2011:kontamination}). In the view taken here, there
exists a subset of blends in Modern \ili{German} systems that, although deliberately
created, are formed through a word-formation process by means of formation
patterns which are very similar to the compounding patterns discussed in
Section \ref{section.comp} above. With appropriate
restrictions, these patterns can be used to form conventionalised as well as
non-conventionalised blends (a point also made by \citealt{schulz:2004:jein:fortschrott}). Among these patterns, I shall discuss
two by means of which blends like \emph{\foreignlanguagedummy{german}{Naturlaub}} or \emph{\foreignlanguagedummy{german}{Kurlaub}} can be formed from bases with an overlapping
part.

\subsection{Case study III: \emph{\foreignlanguagedummy{german}{Naturlaub}}}
\label{section.subord-blend}
The first blend to be discussed is \emph{\foreignlanguagedummy{german}{Naturlaub}}. It appears to be more or less
conventionalised in certain varieties of Modern \ili{German} and occurs in two major
graphematic forms:\footnote{In the \ili{German} Reference Corpus (DeReKo-2018-I),
\emph{\foreignlanguagedummy{german}{Naturlaub}} and \emph{\foreignlanguagedummy{german}{NatUrlaub}} are the only graphematic forms
with more than one token (141 tokens and 102 tokens, respectively). The
relatively high number of \emph{\foreignlanguagedummy{german}{NatUrlaub}} tokens may be due to the fact that this form
is also a brand name (cf.\ \citealt[413]{friedrich:2008:kontamination:form}).} \begin{exe}
\ex \raggedright \begin{taggedline}[0.99]{(\citesource{anonymous:2018:naturlaub:norden})}
\begin{pairingline}
\pairing{\foreignlanguagedummy{german}{Naturlaub}}{nature.vacation}
\pairing{\foreignlanguagedummy{german}{im}}{in.the}
\pairing{\foreignlanguagedummy{german}{Norden}}{north}
\end{pairingline}
\end{taggedline}
\bottomline{‘nature vacation in the north’}
\end{exe}
\begin{exe}
\ex \raggedright \begin{taggedline}[0.99]{(\citesource{anonymous:0:naturbuecher})}
\begin{pairingline}
\pairing{\foreignlanguagedummy{german}{Grüße}}{greeting.\textsc{\addfontfeatures{UprightFeatures={Letters=UppercaseSmallCaps}}PL}}
\pairing{\foreignlanguagedummy{german}{aus}}{out.of}
\pairing{\foreignlanguagedummy{german}{dem}}{the}
\pairing{\foreignlanguagedummy{german}{NatUrlaub}}{nature.vacation}
\end{pairingline}
\end{taggedline}
\bottomline{‘greetings from nature vacation’}
\end{exe}

\noindent
In the linguistic system $\mathbf{S}$ under discussion (some specific system of spoken Modern \ili{German}), the
corresponding lexical word ${\text{/\foreignlanguagedummy{nil}{naˈtuːrlauB}/\/}}_{\text{‘nature vacation’\/}}^{\operatorname{W}}$ and its homophonous stem ${\text{/\foreignlanguagedummy{nil}{naˈtuːrlauB}/\/}}_{\text{‘nature vacation’\/}}^{\operatorname{St}}$ are formed as follows:\footnote{According to the analysis proposed here,
the blend ${\text{/\foreignlanguagedummy{nil}{naˈtuːrlauB}/\/}}_{\text{‘nature vacation’\/}}^{\operatorname{St}}$ has the base stem form /\foreignlanguagedummy{nil}{naˈtuːrlauB}/ with a single morphological atom. The
homonymous compound ${\text{/\foreignlanguagedummy{nil}{naˈtuːr}/ /\foreignlanguagedummy{nil}{ˌlauB}/\/}}_{\text{‘nature foliage’\/}}^{\operatorname{St}}$, however, has a bipartite base stem form /\foreignlanguagedummy{nil}{naˈtuːr}/ /\foreignlanguagedummy{nil}{ˌlauB}/ with two morphological atoms and an
additional, deaccented, lexical accent on the second atom.} \begin{exe}
\ex \raggedright $\displaystyle {\text{/\foreignlanguagedummy{nil}{naˈtuːrlauB}/\/}}_{\text{‘nature vacation’\/}}^{\operatorname{W}}⋖_{\operatorname{blend}\left(\smash{\text{Pattern \ref{pattern.subord-blend}\/}}\right)}^{\mathbf{S}}{\text{/\foreignlanguagedummy{nil}{naˈtuːr}/\/}}_{\text{‘nature’\/}}^{\operatorname{W}}+{\text{/\foreignlanguagedummy{nil}{ˈuːrlauB}/\/}}_{\text{‘vacation’\/}}^{\operatorname{W}}$
\end{exe}
\begin{exe}
\ex \label{displayed.naturlaub-direct-relation}\raggedright $\displaystyle {\text{/\foreignlanguagedummy{nil}{naˈtuːrlauB}/\/}}_{\text{‘nature vacation’\/}}^{\operatorname{St}}⪪_{\operatorname{blend}\left(\smash{\text{Pattern \ref{pattern.subord-blend}\/}}\right)}^{\mathbf{S}}{\text{/\foreignlanguagedummy{nil}{naˈtuːr}/\/}}_{\text{‘nature’\/}}^{\operatorname{St}}+{\text{/\foreignlanguagedummy{nil}{ˈuːrlauB}/\/}}_{\text{‘vacation’\/}}^{\operatorname{St}}$
\end{exe}
As to Pattern \ref{pattern.subord-blend}, I
propose to assume the following means for it: \begin{quotation}
\begin{pattern}
\label{pattern.subord-blend}\vspace{-1.25\baselineskip}
\begin{labeledlist}{PM:}
\item[FM:] \raggedright deaccentuation of the first base form and fusion before the overlapping
part\footnote{The \emph{overlapping part} of two forms $f_{1}$ and $f_{2}$ is the largest common part of the last non-affix atom in $f_{1}$ and the first non-affix atom in $f_{2}$ that contains a syllabic full vowel and spans up to a syllable
boundary. Different lexical accents are ignored.}
\item[PM:] \raggedright identity with the categorisation of the second base form
\item[LM:] \raggedright identity with the categorisation of the second basis
\item[SM:] \raggedright formation of a concept according to the scheme ‘entity denoted by
the second basis in a classificatory relation to an entity denoted by the first
basis’
\end{labeledlist}
\end{pattern}
\end{quotation} The formation restriction associated with Pattern \ref{pattern.subord-blend} in $\mathbf{S}$ should satisfy the following constraints: \begin{quotation}
\begin{restriction}
\label{restriction.subord-blend}\vspace{-1.25\baselineskip}
\begin{labeledlist}{PC:}
\item[FC:] \raggedright There is exactly one non-affix atom in the first base
form.\\*{}
There is an overlapping part of the base forms.\\*{}
The second base form has a primary lexical accent on or after the overlapping
part.\\*{}
The base forms are segmentally distinct from the product form.
\item[PC:] \raggedright The paradigmatic categorisation of the first base form contains $\operatorname{Basic-NStf}$.\\*{}
The paradigmatic categorisation of the second base form contains $\operatorname{Basic-NStf}$, $\operatorname{Sing-NStf}$, or $\operatorname{Plur-NStf}$.
\item[LC:] \raggedright The lexical categorisations of the bases contain $\operatorname{NounSt}$.
\end{labeledlist}
\end{restriction}
\end{quotation} As in Restriction \ref{restriction.coord-comp}, PC in Restriction \ref{restriction.subord-blend} requires that the first base form is
categorised as a basic stem form. There are no empirical reasons to assume
specific compounding or blending stem forms here; in particular, there are no linking elements
occurring in Modern \ili{German} blends (\citealt[78]{mueller:et:al:2011:kontamination}).

\largerpage[-1]
FM in Pattern \ref{pattern.subord-blend} assigns the
product form /\foreignlanguagedummy{nil}{naˈtuːrlauB}/
(a unit sequence) to the base forms /\foreignlanguagedummy{nil}{naˈtuːr}/ and /\foreignlanguagedummy{nil}{ˈuːrlauB}/:\footnote{For the sake of the argument, I
assume that in the linguistic system $\mathbf{S}$ under discussion (some specific system of spoken Modern \ili{German}), the
vowel in the first syllable in /\foreignlanguagedummy{nil}{ˈuːrlauB}/ is typically realised as a long, smoothly
cut, tense vowel – as in the phonetic transcription of \emph{\foreignlanguagedummy{german}{Urlaub}} by the \citet[872]{dudenredaktion:2015:duden:aussprachewoerterbuch}.}
\begin{exe}
\ex \label{displayed.fm-naturlaub}\raggedright
\begin{labeledlist}{FM:}
\item[FM:] \raggedright /\foreignlanguagedummy{nil}{naˈtuːr}/ $+$ /\foreignlanguagedummy{nil}{ˈuːrlauB}/
$↦$\\*{}
/\foreignlanguagedummy{nil}{nat{\phantom{uːr}}}/ $⁐$ /\foreignlanguagedummy{nil}{ˈuːrlauB}/
$=$ /\foreignlanguagedummy{nil}{naˈtuːrlauB}/
\end{labeledlist}
\end{exe}
\pagebreak
\noindent
This is achieved in the following general way:\footnote{Of course, this
is just one of many equivalent formulations of FM in Pattern \ref{pattern.subord-blend} (cf.\ Note \ref{note.extensionality} in Section \ref{section.subord-comp}).

There is another blending pattern in
systems of spoken Modern \ili{German} for blends where the first base form contains
more than one non-affix atom. As far as I can see, this pattern differs from
Pattern \ref{pattern.subord-blend} only with respect to
the formal means which works in the following way: \begin{enumerate}
\item{} The second base form is deaccented.
\item{} The overlapping part and any part before it are deleted in the second base
form.
\item{} Any part of the first base form after the overlapping part is
deleted.
\item{} The results are combined by means of the fusion operation.
\end{enumerate} Put in a nutshell, this formal means is ``deaccentuation of the
second base form and fusion after the overlapping part''. This pattern
can be used not only to form (non-conventionalised) blends like ${\text{/\foreignlanguagedummy{nil}{ˈzelB}/ /\foreignlanguagedummy{nil}{st}/ /\foreignlanguagedummy{nil}{ˌmorD}/ /\foreignlanguagedummy{nil}{ˌzeː}/\/}}_{\text{‘suicidal North Sea’\/}}^{\operatorname{St}}$ where the overlapping part is a proper part of an atom, but also in
borderline cases like ${\text{/\foreignlanguagedummy{nil}{ˈʃraiB}/ /\foreignlanguagedummy{nil}{ˌtiʃ}/ /\foreignlanguagedummy{nil}{ˌtenis}/\/}}_{\text{‘desktop-ping-pong’\/}}^{\operatorname{St}}$ (\citealt[300]{schulz:2004:jein:fortschrott}) where the overlapping part
spans a full atom.} \begin{enumerate}
\item{} \label{item.fm-subord-blend.deacc-1}The first base form /\foreignlanguagedummy{nil}{naˈtuːr}/ is deaccented to /\foreignlanguagedummy{nil}{naˌtuːr}/.
\item{} \label{item.fm-subord-blend.del-1}The overlapping part /\foreignlanguagedummy{nil}{uːr}/ and any part after it are deleted
in /\foreignlanguagedummy{nil}{naˌtuːr}/.
\item{} \label{item.fm-subord-blend.del-2}Any part of /\foreignlanguagedummy{nil}{ˈuːrlauB}/ before the overlapping part /\foreignlanguagedummy{nil}{uːr}/ is deleted.
\item{} The results /\foreignlanguagedummy{nil}{nat}/ and
/\foreignlanguagedummy{nil}{ˈuːrlauB}/ are combined by
means of the fusion operation, denoted by ``$⁐$''.
\end{enumerate} In (\ref{displayed.fm-naturlaub}), the
deaccented lexical accent introduced in step \ref{item.fm-subord-blend.deacc-1} is removed with the overlapping part
/\foreignlanguagedummy{nil}{uːr}/ in step \ref{item.fm-subord-blend.del-1}. This is not the case in the
formation of blends like \emph{\foreignlanguagedummy{german}{Triolade}} (cf.\ \citealt[479]{friedrich:2008:kontamination:form}): \begin{exe}
\ex \raggedright $\displaystyle {\text{/\foreignlanguagedummy{nil}{ˌtriːoˈlaːdə}/\/}}_{\text{‘bar with three types of chocolate’\/}}^{\operatorname{St}}⪪_{\operatorname{blend}\left(\smash{\text{Pattern \ref{pattern.subord-blend}\/}}\right)}^{\mathbf{S}}{\text{/\foreignlanguagedummy{nil}{ˈtriːo}/\/}}_{\text{‘trio’\/}}^{\operatorname{St}}+{\text{/\foreignlanguagedummy{nil}{ʃokoˈlaːdə}/\/}}_{\text{‘chocolate’\/}}^{\operatorname{St}}$
\end{exe} 
Here, the accented syllable in the first base form /\foreignlanguagedummy{nil}{ˈtriːo}/ is before the overlapping part
/\foreignlanguagedummy{nil}{o}/. The lexical accent is
thus not removed in step \ref{item.fm-subord-blend.del-1}: \begin{exe}
\ex \raggedright
\begin{labeledlist}{FM:}
\item[FM:] \raggedright /\foreignlanguagedummy{nil}{ˈtriːo}/ $+$ /\foreignlanguagedummy{nil}{ʃokoˈlaːdə}/ $↦$\\*{}
/\foreignlanguagedummy{nil}{ˌtriː{\phantom{o}}}/ $⁐$ /\foreignlanguagedummy{nil}{{\phantom{ʃok}}oˈlaːdə}/ $=$ /\foreignlanguagedummy{nil}{ˌtriːoˈlaːdə}/
\end{labeledlist}
\end{exe}
As to step \ref{item.fm-subord-blend.del-2}, it
does not delete anything in (\ref{displayed.fm-naturlaub}) since the overlapping part /\foreignlanguagedummy{nil}{uːr}/ is at the very beginning of the
second base form.\footnote{This is the case because in the view taken here, there is
no underlying initial glottal stop in the base form /\foreignlanguagedummy{nil}{ˈuːrlauB}/. If one would assume a base
form with such a consonant, then the latter would be deleted in step \ref{item.fm-subord-blend.del-2} anyway.} For a
non-trivial example consider the blend \emph{\foreignlanguagedummy{german}{Mordsee}}
(cf.\ \citealt[408]{friedrich:2008:kontamination:form}):
\begin{exe}
\ex \oneline{$\displaystyle {\text{/\foreignlanguagedummy{nil}{ˈmorD}/ /\foreignlanguagedummy{nil}{ˌzeː}/\/}}_{\text{‘murderous North Sea’\/}}^{\operatorname{St}}⪪^{\mathbf{S}}{\text{/\foreignlanguagedummy{nil}{ˈmorD}/\/}}_{\text{‘murder’\/}}^{\operatorname{St}}+{\text{/\foreignlanguagedummy{nil}{ˈnorD}/ /\foreignlanguagedummy{nil}{ˌzeː}/\/}}_{\text{‘North Sea’\/}}^{\operatorname{St}}$}
\end{exe}
In this case, step \ref{item.fm-subord-blend.del-2} deletes a non-empty part of the second
base form /\foreignlanguagedummy{nil}{ˈnorD}/ /\foreignlanguagedummy{nil}{ˌzeː}/ before the overlapping part /\foreignlanguagedummy{nil}{orD}/: \begin{exe}
\ex \raggedright
\begin{labeledlist}{FM:}
\item[FM:] \raggedright /\foreignlanguagedummy{nil}{ˈmorD}/ $+$ /\foreignlanguagedummy{nil}{ˈnorD}/ /\foreignlanguagedummy{nil}{ˌzeː}/ $↦$\\*{}
/\foreignlanguagedummy{nil}{m{\phantom{orD}}}/ $⁐$ /\foreignlanguagedummy{nil}{ˈ{\phantom{n}}orD}/ /\foreignlanguagedummy{nil}{ˌzeː}/ $=$ /\foreignlanguagedummy{nil}{ˈmorD}/ /\foreignlanguagedummy{nil}{ˌzeː}/
\end{labeledlist}
\end{exe}
The output-related constraint in Restriction \ref{restriction.subord-blend} according to which the base forms are
segmentally distinct from the product form is a necessary condition for the
recoverability of the bases.

As can be seen from (\ref{displayed.fm-naturlaub}),
FM in Pattern \ref{pattern.subord-blend} reduces the
number of atoms: the number of atoms in the product form is lower than the total
number of atoms in the base forms.\footnote{A similar point is made by \citet[198]{plank:1981:morphologische:irregularitaeten}, who
states: ``\foreignlanguagedummy{german}{das Resultat einer Kontamination soll den Eindruck
einer einfachen morphologischen Einheit ohne interne Konstruktionsfuge erwecken
{[}the result of blending shall give the impression of a simple
morphological unit without an internal construction boundary{]}}''. As
a consequence, the first base form can, in principle, be recovered not only by
reference to the phonological material up to the overlapping part (if any) but
also by reference to material after it. As pointed out by \citet[296]{schulz:2004:jein:fortschrott}, in \emph{\foreignlanguagedummy{german}{Tragikomik}}, which is formed by means of another
blending pattern from the bases \emph{\foreignlanguagedummy{german}{Tragik}} and \emph{\foreignlanguagedummy{german}{Komik}}, the final \emph{\foreignlanguagedummy{german}{ik}} helps to recover the first base form. Similarly,
there may be blends where the second base form can be recovered by reference to
material before the overlapping part. Such effects are excluded in compounds
because of the internal morphological boundary.} Such
\emph{fusioning} formal means can be used to define ``blending'' and ``compounding'' for arbitrary linguistic systems $S$.\footnote{By Definitions \ref{definition.blending} and \ref{definition.compounding}, two-place word-formation processes are
effectively partitioned into two-place compounding and two-place blending. A
further candidate for a two-place word-formation process is reduplication which,
however, is assumed here to be a one-place process, producing a total or partial
copy from a single basis.} \begin{quotation}
\begin{definition}
\label{definition.blending}Let $n≥2$.

\noindent $n$-place \emph{blending} ($\operatorname{blend}^{n}$) in $S$ is that $n$-place word-formation process in $S$ whose arguments are all $n$-place formation patterns in $S$ with a formal means that is fusioning.
\end{definition}
\end{quotation}
\begin{quotation}
\begin{definition}
\label{definition.compounding}Let $n≥2$.

\noindent $n$-place \emph{compounding} ($\operatorname{comp}^{n}$) in $S$ is that $n$-place word-formation process in $S$ whose arguments are all $n$-place formation patterns in $S$ with a formal means that is not fusioning.
\end{definition}
\end{quotation} (By convention, the arity specification ``$2$'' in ``$\operatorname{blend}^{2}$'' and ``$\operatorname{comp}^{2}$'' is dropped if $n=2$.)\footnote{Compounding processes with an arity greater than $2$ might be assumed in Modern \ili{German} for tripartite coordinative
compounds like \emph{\foreignlanguagedummy{german}{rot-grün-blau}},
arguably denoting the mereological sum of red, green, and blue parts. For
potential tripartite blends in Modern \ili{German} cf.\ \citet[Section 4.6]{friedrich:2008:kontamination:form}.} Those
definitions can be supplemented in word-formation theory by an empirical
hypothesis stating that the formal means in any blending pattern are not only
fusioning but also shortening.\footnote{In contrast to axioms, theorems, hypotheses,
etc., definitions are non-empirical since they can be neither true nor false.
This distinction between non-empirical definitions and empirical sentences is
blurred in much of the linguistic literature (for discussion cf.\ \citealt[Section 2.2]{budde:2012:woerter:saetze}). For
instance, shortening is used by \citet[78]{mueller:et:al:2011:kontamination} and others as a
\emph{defining} criterion for blending, by means of which blending is
distinguished from compounding. In my view, this is problematic because the
notion of compounding should not exclude by definition the existence of
compounding patterns with formal means that involve shortening operations such
as apocope.}

PM in Pattern \ref{pattern.subord-blend} is
identical to the paradigmatic means in Pattern \ref{pattern.subord-comp} and \ref{pattern.coord-comp}. Again, this last-base-inheriting operation copies
the paradigmatic categorisation of the second base form to the product form:
\begin{exe}
\ex \begin{xlist}
\ex \raggedright
\begin{labeledlist}{PM:}
\item[PM:] \raggedright $\left\{\smash{\operatorname{Basic-NStf}}\right\}$ $+$ $\left\{\smash{\operatorname{Basic-NStf}}\right\}$ $↦$ $\left\{\smash{\operatorname{Basic-NStf}}\right\}$
\end{labeledlist}
\ex \raggedright
\begin{labeledlist}{PM:}
\item[PM:] \raggedright $\left\{\smash{\operatorname{Basic-NStf}}\right\}$ $+$ $\left\{\smash{\operatorname{Sing-NStf}}\right\}$ $↦$ $\left\{\smash{\operatorname{Sing-NStf}}\right\}$
\end{labeledlist}
\end{xlist}
\end{exe}

\noindent
LM in Pattern \ref{pattern.subord-blend} – likewise
identical to the last-base-inheriting lexical means in Pattern \ref{pattern.subord-comp} and \ref{pattern.coord-comp} – copies the lexical categorisation of the second
base to the product: \begin{exe}
\ex \raggedright
\begin{labeledlist}{LM:}
\item[LM:] \raggedright $\left\{\smash{\operatorname{NounSt},\operatorname{Fem-NSt}}\right\}$ $+$ $\left\{\smash{\operatorname{NounSt},\operatorname{Masc-NSt}}\right\}$ $↦$ $\left\{\smash{\operatorname{NounSt},\operatorname{Masc-NSt}}\right\}$
\end{labeledlist}
\end{exe}
As a consequence, the product has the same lexical gender as the
second basis.

SM in Pattern \ref{pattern.subord-blend} is the same
as the last-base-implying semantic means in Pattern \ref{pattern.subord-comp}. Applied to the base meanings in (\ref{displayed.naturlaub-direct-relation}), it determines
the following underspecified word-formation meaning: \begin{exe}
\ex \raggedright
\begin{labeledlist}{SM:}
\item[SM:] \raggedright ‘nature’ $+$ ‘vacation’ $↦$ ‘vacation in a classificatory relation to nature’
\end{labeledlist}
\end{exe}
Since this semantic means is not commutative, ${\text{/\foreignlanguagedummy{nil}{naˈtuːrlauB}/\/}}_{\text{‘nature vacation’\/}}^{\operatorname{St}}$ is a subordinative blend (cf.\ \citealt[413]{friedrich:2008:kontamination:form}, who classifies this blend
as determinative and endocentric). Thus, as argued independently by \citet[Section 5]{mueller:et:al:2011:kontamination} and others,
the dichotomy between subordinative and coordinative products, introduced above
for compounds, carries over to blends.

\subsection{Case study IV: \emph{\foreignlanguagedummy{german}{Kurlaub}}}
\label{section.coord-blend}
The object of the last case study is the conventionalised blend \emph{\foreignlanguagedummy{german}{Kurlaub}}: \begin{exe}
\ex \raggedright \begin{taggedline}[0.99]{(\citesource{anonymous:1981:boeses:wort})}
\begin{pairingline}
\pairing{\foreignlanguagedummy{german}{Der}}{the}
\pairing{\foreignlanguagedummy{german}{Kurlaub}}{health.cure.vacation}
\pairing{\foreignlanguagedummy{german}{werde
eingeschränkt},}{restricted.\textsc{\addfontfeatures{UprightFeatures={Letters=UppercaseSmallCaps}}3SG}.\textsc{\addfontfeatures{UprightFeatures={Letters=UppercaseSmallCaps}}PASS}.\textsc{\addfontfeatures{UprightFeatures={Letters=UppercaseSmallCaps}}SBJV}}
\pairing{\foreignlanguagedummy{german}{nur}}{only}
\pairing{\foreignlanguagedummy{german}{für}}{for}
\pairing{„\foreignlanguagedummy{german}{notwendige}}{necessary}
\pairing{\foreignlanguagedummy{german}{Kuren}``}{cure.\textsc{\addfontfeatures{UprightFeatures={Letters=UppercaseSmallCaps}}PL}}
\pairing{\foreignlanguagedummy{german}{sollten}}{shall.\textsc{\addfontfeatures{UprightFeatures={Letters=UppercaseSmallCaps}}3PL}.\textsc{\addfontfeatures{UprightFeatures={Letters=UppercaseSmallCaps}}SBJV}}
\pairing{\foreignlanguagedummy{german}{Rentenversicherer}}{pension.insurance.\textsc{\addfontfeatures{UprightFeatures={Letters=UppercaseSmallCaps}}PL}}
\pairing{\foreignlanguagedummy{german}{und}}{and}
\pairing{\foreignlanguagedummy{german}{Krankenkassen}}{health.insurance.\textsc{\addfontfeatures{UprightFeatures={Letters=UppercaseSmallCaps}}PL}}
\pairing{\foreignlanguagedummy{german}{noch}}{still}
\pairing{\foreignlanguagedummy{german}{zahlen}.}{pay}
\end{pairingline}
\end{taggedline}
\bottomline{‘Combinations of health cure and vacation will be restricted, pension
insurances and health insurances shall only continue to pay for ``necessary
cures''.’}
\end{exe}
In the linguistic system $\mathbf{S}$, the lexical word ${\text{/\foreignlanguagedummy{nil}{ˈkuːrlauB}/\/}}_{\text{‘health cure plus vacation’\/}}^{\operatorname{W}}$ and its stem ${\text{/\foreignlanguagedummy{nil}{ˈkuːrlauB}/\/}}_{\text{‘health cure plus vacation’\/}}^{\operatorname{St}}$ are formed as follows: \begin{exe}
\ex \raggedright $\displaystyle {\text{/\foreignlanguagedummy{nil}{ˈkuːrlauB}/\/}}_{\text{‘health cure plus vacation’\/}}^{\operatorname{W}}⋖_{\operatorname{blend}\left(\smash{\text{Pattern \ref{pattern.coord-blend}\/}}\right)}^{\mathbf{S}}{\text{/\foreignlanguagedummy{nil}{ˈkuːr}/\/}}_{\text{‘health cure’\/}}^{\operatorname{W}}+{\text{/\foreignlanguagedummy{nil}{ˈuːrlauB}/\/}}_{\text{‘vacation’\/}}^{\operatorname{W}}$
\end{exe}
\begin{exe}
\ex \label{displayed.kurlaub-direct-relation}\raggedright $\displaystyle {\text{/\foreignlanguagedummy{nil}{ˈkuːrlauB}/\/}}_{\text{‘health cure plus vacation’\/}}^{\operatorname{St}}⪪_{\operatorname{blend}\left(\smash{\text{Pattern \ref{pattern.coord-blend}\/}}\right)}^{\mathbf{S}}{\text{/\foreignlanguagedummy{nil}{ˈkuːr}/\/}}_{\text{‘health cure’\/}}^{\operatorname{St}}+{\text{/\foreignlanguagedummy{nil}{ˈuːrlauB}/\/}}_{\text{‘vacation’\/}}^{\operatorname{St}}$
\end{exe}
Pattern \ref{pattern.coord-blend} combines
means from Pattern \ref{pattern.coord-comp} and \ref{pattern.subord-blend}: \begin{quotation}
\begin{pattern}
\label{pattern.coord-blend}\vspace{-1.25\baselineskip}
\begin{labeledlist}{PM:}
\item[FM:] \raggedright deaccentuation of the first base form and fusion before the overlapping
part
\item[PM:] \raggedright identity with the categorisation of the second base form
\item[LM:] \raggedright identity with the categorisation of the second basis
\item[SM:] \raggedright formation of a concept according to the scheme ‘sum of the entities
denoted by the bases’
\end{labeledlist}
\end{pattern}
\end{quotation} Restriction \ref{restriction.coord-blend}
contains the corresponding constraints from Restriction \ref{restriction.coord-comp} and \ref{restriction.subord-blend}: \begin{quotation}
\begin{restriction}
\label{restriction.coord-blend}\vspace{-1.25\baselineskip}
\begin{labeledlist}{PC:}
\item[FC:] \raggedright There is exactly one non-affix atom in the first base
form.\\*{}
There is an overlapping part of the base forms.\\*{}
The second base form has a primary lexical accent on or after the overlapping
part.\\*{}
The base forms are segmentally distinct from the product form.
\item[PC:] \raggedright The paradigmatic categorisation of the first base form contains $\operatorname{Basic-NStf}$.\\*{}
The paradigmatic categorisation of the second base form contains $\operatorname{Basic-NStf}$, $\operatorname{Sing-NStf}$, or $\operatorname{Plur-NStf}$.
\item[LC:] \raggedright The lexical categorisations of the bases contain $\operatorname{NounSt}$.
\item[SC:] \raggedright The bases denote entities of the same sort for which a sum operation is
defined.
\end{labeledlist}
\end{restriction}
\end{quotation}

FM in Pattern \ref{pattern.coord-blend} is identical
to the formal means in Pattern \ref{pattern.subord-blend} and assigns the fused product form /\foreignlanguagedummy{nil}{ˈkuːrlauB}/ to the base forms
/\foreignlanguagedummy{nil}{ˈkuːr}/ and /\foreignlanguagedummy{nil}{ˈuːrlauB}/: \begin{exe}
\ex \raggedright
\begin{labeledlist}{FM:}
\item[FM:] \raggedright /\foreignlanguagedummy{nil}{ˈkuːr}/ $+$ /\foreignlanguagedummy{nil}{ˈuːrlauB}/
$↦$\\*{}
/\foreignlanguagedummy{nil}{k{\phantom{uːr}}}/ $⁐$ /\foreignlanguagedummy{nil}{ˈuːrlauB}/
$=$ /\foreignlanguagedummy{nil}{ˈkuːrlauB}/
\end{labeledlist}
\end{exe}

\noindent
PM and LM in Pattern \ref{pattern.coord-blend} are
the same as the last-base-inheriting paradigmatic and lexical means in the
patterns discussed so far: \begin{exe}
\ex \raggedright
\begin{labeledlist}{PM:}
\item[PM:] \raggedright $\left\{\smash{\operatorname{Basic-NStf}}\right\}$ $+$ $\left\{\smash{\operatorname{Basic-NStf}}\right\}$ $↦$ $\left\{\smash{\operatorname{Basic-NStf}}\right\}$
\end{labeledlist}
\end{exe}
\begin{exe}
\ex \raggedright
\begin{labeledlist}{LM:}
\item[LM:] \raggedright $\left\{\smash{\operatorname{NounSt},\operatorname{Fem-NSt}}\right\}$ $+$ $\left\{\smash{\operatorname{NounSt},\operatorname{Masc-NSt}}\right\}$ $↦$ $\left\{\smash{\operatorname{NounSt},\operatorname{Masc-NSt}}\right\}$
\end{labeledlist}
\end{exe}
In particular, the lexical means ensures that the product inherits
its lexical gender from the second basis.

SM in Pattern \ref{pattern.coord-blend} is identical
to the semantic means in Pattern \ref{pattern.coord-comp}: \begin{exe}
\ex \raggedright
\begin{labeledlist}{SM:}
\item[SM:] \raggedright ‘health cure’ $+$ ‘vacation’ $↦$ ‘sum of health cure and vacation’
\end{labeledlist}
\end{exe}
\largerpage
The sum operation involved in this example combines events, e.g.
health-cure treatments in the morning and vacation activities during the rest
of the day. These combined events denoted by the product are denoted neither by
the first basis nor by the second basis (at least not as a whole); this is what
is to be expected from a semantic means that is not base-implying.\footnote{SM
in Pattern \ref{pattern.coord-blend} may involve sum operations of quite
different sorts. In the case of the blend ${\text{/\foreignlanguagedummy{nil}{demokraˈtuːr}/\/}}_{\text{‘democracy plus dictatorship’\/}}^{\operatorname{St}}$,
for example, SM assigns to the base concepts ‘democracy’ and ‘dictatorship’ a
concept that denotes the combination of two political systems which neither is a
proper democracy nor a full-fledged dictatorship.} Since the semantic means is
commutative, ${\text{/\foreignlanguagedummy{nil}{ˈkuːrlauB}/\/}}_{\text{‘health cure plus vacation’\/}}^{\operatorname{St}}$ is a coordinative blend (also classified as coordinative and
exocentric by \citealt[387]{friedrich:2008:kontamination:form}).

\section{Conclusion}
\label{section.conclusion}
In the case studies in Sections \ref{section.subord-comp}, \ref{section.coord-comp}, \ref{section.subord-blend}, and \ref{section.coord-blend}, I discussed the word-formation relations (\ref{displayed.nordtor-direct-relation}), (\ref{displayed.nordosten-direct-relation}), (\ref{displayed.naturlaub-direct-relation}), and (\ref{displayed.kurlaub-direct-relation}), repeated here for
convenience: \begin{exe}
\ex \raggedright $\displaystyle {\text{/\foreignlanguagedummy{nil}{ˈnorD}/ /\foreignlanguagedummy{nil}{ˌtoːr}/\/}}_{\text{‘north gate’\/}}^{\operatorname{St}}⪪_{\operatorname{comp}\left(\smash{\text{Pattern \ref{pattern.subord-comp}\/}}\right)}^{\mathbf{S}}{\text{/\foreignlanguagedummy{nil}{ˈnorD}/\/}}_{\text{‘north’\/}}^{\operatorname{St}}+{\text{/\foreignlanguagedummy{nil}{ˈtoːr}/\/}}_{\text{‘gate’\/}}^{\operatorname{St}}$
\end{exe}
\begin{exe}
\ex \raggedright $\displaystyle {\text{/\foreignlanguagedummy{nil}{ˌnorD}/ /\foreignlanguagedummy{nil}{ˈost}/\/}}_{\text{‘north-east’\/}}^{\operatorname{St}}⪪_{\operatorname{comp}\left(\smash{\text{Pattern \ref{pattern.coord-comp}\/}}\right)}^{\mathbf{S}}{\text{/\foreignlanguagedummy{nil}{ˈnorD}/\/}}_{\text{‘north’\/}}^{\operatorname{St}}+{\text{/\foreignlanguagedummy{nil}{ˈost}/\/}}_{\text{‘east’\/}}^{\operatorname{St}}$
\end{exe}
\begin{exe}
\ex \raggedright $\displaystyle {\text{/\foreignlanguagedummy{nil}{naˈtuːrlauB}/\/}}_{\text{‘nature vacation’\/}}^{\operatorname{St}}⪪_{\operatorname{blend}\left(\smash{\text{Pattern \ref{pattern.subord-blend}\/}}\right)}^{\mathbf{S}}{\text{/\foreignlanguagedummy{nil}{naˈtuːr}/\/}}_{\text{‘nature’\/}}^{\operatorname{St}}+{\text{/\foreignlanguagedummy{nil}{ˈuːrlauB}/\/}}_{\text{‘vacation’\/}}^{\operatorname{St}}$
\end{exe}
\begin{exe}
\ex \raggedright $\displaystyle {\text{/\foreignlanguagedummy{nil}{ˈkuːrlauB}/\/}}_{\text{‘health cure plus vacation’\/}}^{\operatorname{St}}⪪_{\operatorname{blend}\left(\smash{\text{Pattern \ref{pattern.coord-blend}\/}}\right)}^{\mathbf{S}}{\text{/\foreignlanguagedummy{nil}{ˈkuːr}/\/}}_{\text{‘health cure’\/}}^{\operatorname{St}}+{\text{/\foreignlanguagedummy{nil}{ˈuːrlauB}/\/}}_{\text{‘vacation’\/}}^{\operatorname{St}}$
\end{exe} ``$\mathbf{S}$'' stood for some specific, yet undetermined, system of
spoken Modern \ili{German}. The products in those word-formation relations – the
subordinative compound ${\text{/\foreignlanguagedummy{nil}{ˈnorD}/ /\foreignlanguagedummy{nil}{ˌtoːr}/\/}}_{\text{‘north gate’\/}}^{\operatorname{St}}$, the coordinative compound ${\text{/\foreignlanguagedummy{nil}{ˌnorD}/ /\foreignlanguagedummy{nil}{ˈost}/\/}}_{\text{‘north-east’\/}}^{\operatorname{St}}$, the subordinative blend ${\text{/\foreignlanguagedummy{nil}{naˈtuːrlauB}/\/}}_{\text{‘nature vacation’\/}}^{\operatorname{St}}$, as well as the coordinative blend ${\text{/\foreignlanguagedummy{nil}{ˈkuːrlauB}/\/}}_{\text{‘health cure plus vacation’\/}}^{\operatorname{St}}$ – are \emph{right-headed} in the following sense. All of them
are \emph{categorially determined} by the last basis, i.e. they are formed
by means of a formation pattern with last-base-inheriting paradigmatic and
lexical means. Some of them – viz. ${\text{/\foreignlanguagedummy{nil}{ˈnorD}/ /\foreignlanguagedummy{nil}{ˌtoːr}/\/}}_{\text{‘north gate’\/}}^{\operatorname{St}}$ and ${\text{/\foreignlanguagedummy{nil}{naˈtuːrlauB}/\/}}_{\text{‘nature vacation’\/}}^{\operatorname{St}}$ – are also \emph{semantically determined} by the last basis,
because they are formed by means of a formation pattern with a
last-base-implying semantic means.

Presupposing the Pattern-and-Restriction Theory, these headedness properties
could be established independently of any word structures; in particular, no
reference was made to ``heads'' or
``non-heads''. Rather, it was demonstrated that those
properties are based on properties of the formation patterns by means of which
products are formed from bases through certain word-formation processes. These
processes are not restricted to compounding, but apply in principle also to
blending. Put differently, headedness properties of compounds and blends can be
identified in the Pattern-and-Restriction Theory solely on the basis of
word-formation relations and the involved formation patterns – without the
assumption of ``head constituents'', which are notoriously
difficult to ascertain for blends. In the paradigmatic approach followed in this
paper, headedness thus emerges as an epiphenomenon of the word-formation
relations between lexical units in a linguistic system. This notion is readily
reconstructed as a descriptive term, but has no theoretical significance in such
an approach to word formation.

\appendix
\section*{\appendixname}
\subsection*{List of symbols}
Notational conventions: \begin{labeledlist}{``$\phantom{}^{\operatorname{W}}$'':}
\item[``$\phantom{}^{\operatorname{St}}$'':] lexical stem.
\item[``$\phantom{}^{\operatorname{W}}$'':] lexical word.
\end{labeledlist} Symbols for categories: \begin{description}
\item[``$\operatorname{Acc-Nf}$'':] nominal word form in the accusative.
\item[``$\operatorname{Basic-NStf}$'':] nominal basic stem form.
\item[``$\operatorname{Comp-NStf}$'':] nominal compounding stem form.
\item[``$\operatorname{Dat-Nf}$'':] nominal word form in the dative.
\item[``$\operatorname{Der-NStf}$'':] nominal derivation stem form.
\item[``$\operatorname{Fem-NSt}$'':] nominal stem in the feminine.
\item[``$\operatorname{Masc-N}$'':] nominal word in the masculine.
\item[``$\operatorname{Masc-NSt}$'':] nominal stem in the masculine.
\item[``$\operatorname{Neut-NSt}$'':] nominal stem in the neuter.
\item[``$\operatorname{Nom-Nf}$'':] nominal word form in the nominative.
\item[``$\operatorname{Noun}$'':] noun.
\item[``$\operatorname{NounSt}$'':] noun stem.
\item[``$\operatorname{Plur-NStf}$'':] nominal stem form in the plural.
\item[``$\operatorname{Sing-Nf}$'':] nominal word form in the singular.
\item[``$\operatorname{Sing-NStf}$'':] nominal stem form in the singular.
\end{description} Symbols for word-formation relations: \begin{labeledlist}{``$⋖$'':}
\item[``$<$'':] word-formation relation.
\item[``$⪪$'':] direct word-formation relation.
\item[``$⋖$'':] indirect word-formation relation.
\end{labeledlist} Symbols for word-formation processes: \begin{description}
\item[``$\operatorname{blend}$'':] (two-place) blending.
\item[``$\operatorname{comp}$'':] (two-place) compounding.
\end{description} Variables: \begin{labeledlist}{``$M$'':}
\item[``$f$'':] sequences of morphological or syntactic atoms.
\item[``$M$'':] two-place operations.
\item[``$n$'':] natural numbers $≥$ $1$.
\item[``$S$'':] linguistic systems.
\item[``$x$'':] arguments or values of two-place operations.
\end{labeledlist} Ambiguous constant: \begin{description}
\item[``$\mathbf{S}$'':] some specific system of spoken Modern \ili{German}.
\end{description}


{\sloppy
\printbibliography[heading=subbibliography,notkeyword=source]
\newrefcontext[labelprefix=S]
\printbibliography[heading=subbibliography,keyword=source,env=sources,title={List of sources}]
}

\end{document}


%\section*{\acknowledgmentsUS}
