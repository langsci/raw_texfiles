% -*- coding:utf-8 -*-
\documentclass[output=paper
 ,nobabel
 ,draftmode
 ,colorlinks, citecolor=brown
]{langscibook}


\IfFileExists{../localcommands.tex}{%hack to check whether this is being compiled as part of a collection or standalone
   % add all extra packages you need to load to this file

\usepackage{tabularx,multicol}
\usepackage{url}
\urlstyle{same}

\usepackage{listings}
\lstset{basicstyle=\ttfamily,tabsize=2,breaklines=true}

\usepackage{langsci-basic}
\usepackage{langsci-optional}
\usepackage{langsci-lgr}
\usepackage{langsci-osl}
% \usepackage{./langsci/styles/langsci-lgr}
% \usepackage{./langsci/styles/langsci-osl}
% \usepackage{langsci-gb4e}

\usepackage{tikz}
\usetikzlibrary{patterns,calc}
\pgfdeclarepatternformonly{south east lines}{\pgfqpoint{-0pt}{-0pt}}{\pgfqpoint{3pt}{3pt}}{\pgfqpoint{3pt}{3pt}}{
    \pgfsetlinewidth{0.6pt}
    \pgfpathmoveto{\pgfqpoint{0pt}{3pt}}
    \pgfpathlineto{\pgfqpoint{3pt}{0pt}}
    \pgfpathmoveto{\pgfqpoint{.2pt}{-.2pt}}
    \pgfpathlineto{\pgfqpoint{-.2pt}{.2pt}}
    \pgfpathmoveto{\pgfqpoint{3.2pt}{2.8pt}}
    \pgfpathlineto{\pgfqpoint{2.8pt}{3.2pt}}
    \pgfusepath{stroke}}
    
\usepackage{stmaryrd}
\usepackage{wasysym}
\usepackage{multirow}
\usepackage{caption}
\usepackage{subcaption}
\usepackage{mathrsfs}
\usepackage{qtree}

\usepackage{linguex}


   %pminos do not split footnotes
% \interfootnotelinepenalty=10000 %Footnote in Laporte chapters has to be split SN


%\DeclareIndexNameFormat{default}{%
%\nameparts{#1}%
%\usebibmacro{index:name}%
%{\index[names]}%
%{\namepartfamily}%
%{\namepartgiveni}%
% {}% L1
% {}% L2
%{\namepartprefix}% generates spurious space L3
%{\namepartsuffix}% generates spurious space L4
%}

%  {\DeclareIndexNameFormat{default}{%
%     \usebibmacro{index:name}{\index[names]}{#1}{#3}{#5}{#7}}}

%\DeclareIndexNameFormat{default}{%
%  \usebibmacro{index:name}{\sindex[nom]}{#1}{#3}{#5}{#7}}

%\DeclareIndexNameFormat{default}{%
%  \usebibmacro{index:name}{\sindex[person]}{#1}{#3}{#5}{#7}}
%\DeclareIndexNameFormat{default}{%
%\nameparts{#1} \usebibmacro{index:name}{\sindex[person]]}{\namepartfamily}{‌​\namepartgiven}{\nam‌​epartprefix}{\namepa‌​rtsuffix}}

%\newcommand{\smiley}{:)}

%\renewbibmacro*{index:name}[5]{%
%\usebibmacro{index:entry}{#1}%
%{\iffieldundef{usera}{}{\thefield{usera}\actualoperator}\mkbibindexname{#2}{#3}{#4}{#5}}}

% \newcommand{\noop}[1]{}

%remove for final
%\overfullrule=1mm

\newcommand{\tobi}[2]}}
\renewcommand{\S}[1]{\tobi{#1}{\textsc{*}}}

% this volume references
% puts: [this volume]
% already defined: \citetv
%\newcommand{\citepv}[1]{(\citeauthor{#1} \citeyear*{#1} [this volume])}
\newcommand{\citealtv}[1]{\citeauthor{#1} \citeyear*{#1} [this volume]}

%parentheses around example number
\newcommand{\pref}[1]{(\ref{#1})}

% in-text examples

\newcommand{\lnex}[1]{\textit{#1}} %target lang word
\newcommand{\lnlit}[1]{(lit.: `#1')} %literal reading
\newcommand{\lnlat}[1]{(#1)} % latinization
\newcommand{\lntrans}[1]{`#1'} %translation
\newcommand{\lnexl}[2]%
{\lnex{#1}{} \lnlat{#2}} % ex with latinization
\newcommand{\lnexlat}[3]{\lnex{#1}{} \lnlat{#2}{} \lntrans{#3}} % ex with latinization and tranl.

%ch01
\newcommand{\co}[1]{\mbox{\textbf{#1}}}

%ch09

\newcommand{\cyrbulg}[1]{\begin{otherlanguage*}{bulgarian}#1\end{otherlanguage*}}


%ch10
\newcommand{\nlp}{{\small NLP}}
\newcommand{\mwe}{{\small MWE}}
\newcommand{\rae}{{\small RAE}}
\newcommand{\lvc}{{\small LVC}}
\newcommand{\pos}{{\small P}o{\small S}}
%\newcommand{\todo}[1]{ \textcolor{red}{#1} }

%\renewcommand{\labelenumi}{\theenumi}
%\ainamefmt{{vv}{ll}{, ff}{, jj}} % fullname

\newcommand{\biberror}[1]{{\color{red}#1}}

\newcommand{\osenovaitem}{--~}
   %% hyphenation points for line breaks
%% Normally, automatic hyphenation in LaTeX is very good
%% If a word is mis-hyphenated, add it to this file
%%
%% add information to TeX file before \begin{document} with:
%% %% hyphenation points for line breaks
%% Normally, automatic hyphenation in LaTeX is very good
%% If a word is mis-hyphenated, add it to this file
%%
%% add information to TeX file before \begin{document} with:
%% %% hyphenation points for line breaks
%% Normally, automatic hyphenation in LaTeX is very good
%% If a word is mis-hyphenated, add it to this file
%%
%% add information to TeX file before \begin{document} with:
%% \include{localhyphenation}
\hyphenation{
    Beck-man
    Ngu-yen
    back-chan-nel
    back-chan-nels
    mo-not-o-nous
    ste-reo-typ-i-cal
}

\hyphenation{
    Beck-man
    Ngu-yen
    back-chan-nel
    back-chan-nels
    mo-not-o-nous
    ste-reo-typ-i-cal
}

\hyphenation{
    Beck-man
    Ngu-yen
    back-chan-nel
    back-chan-nels
    mo-not-o-nous
    ste-reo-typ-i-cal
}

    \togglepaper[8]
}{}


\ChapterDOI{10.5281/zenodo.7142714}

\title[Head alignment in German compounds]{Head alignment in German compounds: Implications for prosodic constituency and morphological parsing} 

\author{Renate Raffelsiefen\orcid{0000-0002-9513-9798}\affiliation{Leibniz-Institut für Deutsche Sprache, Mannheim}}
% \chapterDOI{} %will be filled in at production

% \epigram{}

\abstract{The notion of head alignment was introduced to account for the observation that in a word with multiple feet, one is more prominent than the others. In particular, this notion is meant to capture the characteristic edge-orientation of main stress by requiring the (left or right) word boundary and the respective (left or right) boundary of the head foot to coincide \citep{McCarthyPrince1993}. In the present paper the notion of head alignment will be applied to compounds, which are also characterized by the property that one of their members, located in a margin position, is most prominent. 

The adequacy of an analysis in terms of head alignment hinges on the question of whether observable
prominence peaks associate with the boundaries of independently motivated constituents. It will be
argued that such links exist for German compounds, indicating reference to at least three distinct
compound categories established on morphological grounds: copulative, phrasal, and a default class
of “regular” compounds. The evidence for the relevant distinctions sheds light on morphological
parsing, indicating that compound categories can be -- and often are -- determined by properties pertaining to their complete form, rather than by conditions affecting their (original) construction.}

\begin{document}

\maketitle
%
%
%\itdopt{EE: Bei einigen Bäumen geht sm edges nicht (TeX capacity exceeded, sorry [input stack size=5000].)
%
%Die ganzen captions der Bäume musste ich mir ausdenken. Bitte nochmal checken, ob sie zu wenig/zu kurz sind.}
%
%\itdopt{St: Ich würde Abbildungen a und b weglassen und dann im Text "`Abbildung 7 links"' schreiben.}
%


\section{Introduction}\label{sec-intro-raf}

\il{German|(}%
The motivation for the notion \emph{head} in linguistics rests on consistent criteria for singling out units within given constructions where those units associate with some sort of dominant role. Originally intended to account for semantic relations among words in sentences in a dependency framework,%
%
\footnote{The origin of the notion \emph{head} in grammar is often credited to \citet{Sweet1898}, who identified asymmetric relations among ``head-words'' vis-à-vis ``adjunct-words'' within given constructions on semantic grounds (adjunct-words modify the meaning of the head-word). Referring to the sentence \emph{He is not very strong} he proposes the role assignments in (\ref{ex-stonewall}.a), illustrating the possibility of different roles played by the same words in the same sentence. The relevant roles for the compounds \emph{stonewall} and \emph{bookseller} are shown in (\ref{ex-stonewall}.b) \citep[I:16, Sections~40 and~41]{Sweet1898}.

\ea\label{ex-stonewall}
%\begin{tabular}{@{}{l}{\hspace{1cm}}{l}@{}}
\begin{tabular}[t]{@{~}l@{~~}ll}
\multicolumn{2}{@{~}l}{head-word} & adjunct-word\\
a. & he & strong\\
   & strong & very\\
   & very & not\\
b. & wall & stone\\
   & seller & book
\end{tabular}
\z
%\begin{multicols}{2}
%head-word\jambox{adjunct-word}
%\ea\label{ex-verystrong}
%he\\
%strong\\
%very
%
%\ex\label{ex-stonewall}
%wall\\
%seller
%\z
%\columnbreak
%adjunct-word\\
%strong\\
%very\\
%not\\
%stone\\
%book
%\end{multicols}
%\z

\noindent
The concept \emph{head} is motivated by various independent correlates. For instance, head-words are claimed to determine agreement (\eg the dependence of the verb forms on the respective head word in \emph{she walks} versus \emph{they walk}) as well as govern case (\eg specific case forms of objects determined by verbs or prepositions \citep[I:32]{Sweet1898}.
}
%\itdopt{Fußnote 1 vervollständigen}
the head concept was subsequently extended to other areas of linguistics (derivational morphology,
phonology) as well as to constituency-based grammar models. These applications have met with great
skepticism \citep{Hudson1987,Croft1996} and the need to recognize the notion \emph{head} as an
independent concept has been questioned \citep{Nichols1993,Hawkins1993}.  \citet[41]{Bauer2017}
summarizes as follows: ``If some notion of head is to be retained [\ldots] it needs to be made clear
that the derivational head is not the same as the head in a compound is not the same as a syntactic
head.'' 

Bauer's assessment raises the issue of general properties defining the concept \emph{head} in natural language. Relevant statements assert two consistent properties: ``uniqueness'' (reference to the numeral \emph{one}) and ``dominance'': 

\begin{quote}
    The intuition to be captured with the notion HEAD is that in certain syntactic constructs one constituent in some sense `characterizes' or `dominates' the whole \citep[2]{Zwicky85a}.
\end{quote}

\begin{quote}
    Headedness is a pervasive phenomenon throughout different components of the grammar, which fundamentally encodes an asymmetry between two or more items, such that one is in some sense more important than the other(s) \citep{MoskalSmith2019}.
\end{quote}

\largerpage[-1]
\noindent
With reference to constituent structure, I then propose the following criteria for recognizing heads: 

\begin{quote}
The empirical motivation for the notion \emph{head} rests on independently defined constituents for which a unique daughter constituent can be identified in a consistent manner. That daughter's role is characterized by dominance vis-à-vis given sister constituents.
\end{quote}

%\itdopt{im manuskript eingerückt, so belassen? EE: Kenne Manuskript nicht, aber in Word nicht eingerückt}

\noindent
As for phonology, in particular prosodic phonology, there are two well-known references to the notion \emph{head}, the constraint \textsc{Headedness} \citep{Selkirk1995} and the notion of head alignment \citep{McCarthyPrince1993}.\footnote{The notion \emph{head} has also been invoked to account for segmental phenomena in the frameworks of Government Phonology, Dependency Phonology and various offshoots of these theories (see the contributions in \citealp{CarrEwen2005}).} Both notions will be discussed in Section~\ref{sec-wordprosody}, where I will argue that only head alignment conforms to the concept of headedness stated above. 

Head alignment was first proposed to capture main stress in a phonological word, by picking out one of the respective feet based on its position within the word \citep[98]{McCarthyPrince1993}. This notion comprises both ``uniqueness'' and ``dominance'', assuming that dominance could be expressed in form of prosodic prominence, manifest in increased pitch, loudness, along with various properties concerning (language-specific) phonetic implementation, including the lengthening and strengthening of articulatory gestures associated with segments in stressed syllables (cf.\ \citealp[125]{Lehiste1970},\footnote{\citet*[125]{Lehiste1970} observes that whereas increased loudness and vocal fold vibrations are both caused by an increase in respiratory effort, greater duration of stressed syllables is an independent factor characteristic of Western European 
languages.} \citealp[90]{Ladefoged2003}). Indeed, McCarthy and Prince explicitly mention the generality of their use of the  notion \emph{head} in this context:

\begin{quote}
    The head-alignment constraint has obvious cognates in both morphology and syntax, presumably to be expressed in the same way \citep[99]{McCarthyPrince1993}.
\end{quote}

\largerpage[-1]
\noindent
Still, reference to the notion \emph{head} in connection with  prosodic prominence differs from the
notion \emph{head} typically used in syntax or morphology in one major regard. In syntax or
morphology, the daughter functioning as the head is determined based on her inherent properties,
which percolate to the mother node. Even when percolation is linked to a margin position (cf.\
Williams' ``Righthand-Head rule'' \citealp[248]{Williams1981}) the occupant of the margin position
appears to be determined by inherent properties.\footnote{When forming a compound, the speaker will
  consider the intended referent, ensuring that the stem forming the semantic head will emerge in
  rightmost position, where its features percolate upward. For instance, if a speaker wishes to
  refer to a tree (which bears fruit) she will chose the order \emph{fruit tree}, \ili{German}
  \emph{Obstbaum}, where the features of the rightmost member percolate upward (for instance,
  masculine in \ili{German}). If she wishes to refer to fruit (which grows on trees) she will chose the
  order \emph{tree fruit}, \ili{German} \emph{Baumobst}, where again the features of the rightmost member
  percolate upward (in this case, neuter in \ili{German}).} In prosody, the daughter functioning as the
head is picked based on her (independently determined) position relative to her sisters, where
prominence is a consequence, rather than a condition, for the head status. This also concerns the
patterns of head alignment in \ili{German} compounds to be explored below. For instance, in the \ili{German}
compound \emph{Augapfel} `eyeball' the righthand member \emph{-apfel} clearly constitutes the
morphosyntactic head, passing on its features (singular, masculine) to the compound as a whole. Yet,
it is the lefthand member \emph{aug-}, which is most prominent, forming the prosodic head due its
representation as a separate phonological word at the left periphery of the (prosodic) compound
constituent. 


\largerpage
The generalization that prosodic head status is determined by the position of a constituent relative
to the respective mother constituent raises the question of what determines hierarchical prosodic
structure. As for compounds, such structure appears to result from an isomorphic mapping of
morphological to prosodic structure,\footnote{\label{fn-isomorphy}Isomorphy does not always persist,
  due to the effects of higher-ranking constraints. A well-known case concerns cohering affixes,
  which due to their shape and position integrate into the phonological word of the adjacent stem
  (see \citet{Raffelsiefen2022} for a review of relevant asymmetries). In compounds, non-isomorphy
  may result from prominence reversals, resulting in stress patterns no longer matching the
  morphological compound type (\eg \emph{North Séa}, a sort of compound associated with final main
  stress in \ili{English}, but \emph{Nòrth Sea óil}, where the constituent \emph{North} is more prominent than
  \emph{Sea}, to avoid a stress clash \citep[137]{Fudge1984}. See also \citet{Gussenhoven2011} for
  an overview of relevant phenomena in \ili{English}.} which shifts the burden of the analysis to the
question of what determines compound morphology. The present article argues that the prominence patterns seen in compounds
correlate systematically with various morphological compound types, themselves
determined by conditions pertaining to morphological parsing. As for \ili{German} compounds, relevant
parsing patterns motivate at least three distinct compound types (copulative, phrasal, and
``default''), each mapping to separate prosodic compound constituents, where main stress is
determined by compound-specific head alignment constraints.    

The article is organized as follows. Section~\ref{sec-wordprosody} discusses general issues pertaining to the notion \emph{head} in word prosody, focusing on the constraint \textsc{Headedness} and head alignment constraints. The latter are also compared to an alternative approach to capturing prominence relations in terms of a binary branching formalism.  Section~\ref{sec-regular} explores possible conditions on parsing words as ``regular'' compounds, as opposed to simple phonological words, in \ili{German}. Sections~\ref{sec-copcom} and~\ref{sec-phracom} investigate the notion of head alignment in regard to copulative and phrasal compounds, respectively. Section~\ref{sec-conclu-rf} gives a summary. 

\section{Heads in word prosody}
\label{sec-wordprosody}

A much-cited reference to heads in phonology concerns a constraint on dominance relations referring to the so-called \emph{Prosodic Hierarchy}. A version of part of this hierarchy is shown in Figure~\ref{ex-hierarchy}, where the term \emph{composite group} encompasses clitic groups, (certain) affixed words, and compounds \citep[xvii]{NesporVogel2007}. The dotted lines represent additional structure not specified here.


\begin{figure}
\begin{forest}
[utterance (U)
        [,edge=dotted]
	[composite group (CG), edge=dotted
		[,edge=dotted]
		[phonological word (ω)
			[,edge=dotted]
			[foot (Σ)
				[,edge=dotted]
				[syllable (σ)
					[,edge=dotted]
				]
				[,edge=dotted]
			]
			[,edge=dotted]
		]
		[,edge=dotted]
	]
        [,edge=dotted]
]
\end{forest}
\caption{Prosodic Hierarchy (excerpt)}\label{ex-hierarchy}
\end{figure}

\largerpage
The constraint \textsc{Headedness} proposed by \citet[443]{Selkirk1995}, which is claimed to be universally obeyed, is meant to ensure that every (non-terminal) unit dominates at least one unit belonging to the immediately lower layer of the Prosodic Hierarchy:\footnote{The constraint expresses ``Principle 1'' as stated by \citet[7]{NesporVogel2007}: ``A given nonterminal unit of the Prosodic Hierarchy, X\textsuperscript{p}, is composed of one or more units of the immediately lower category, X\textsuperscript{p-1}.'' The novelty in (\ref{ex-headedness}) concerns the conceptualization of this principle as an (inviolable) constraint involving reference to the notion \emph{head}.} 

\ea\label{ex-headedness}
\textsc{Headedness}: Any C\textsubscript{i} must dominate a C\textsubscript{i-1}
\z

\noindent
There is no reference here to a unique daughter, nor to a daughter's dominant role vis-à-vis her sister constituents.\footnote{Selkirk has meanwhile distanced herself from the notion of headedness as stated in the constraint in (\ref{ex-headedness}), characterizing that formulation as ``unfortunate'' (cf.\ \citealp*[1]{ElordietaSelkirk2018}, footnote 1).} 
%
In fact, constituents in the Prosodic Hierarchy are often associated with daughters belonging to the {\bf same} rank, a condition known as \emph{Strict Layer Hypothesis}. The constraint \textsc{Headedness} as stated in (\ref{ex-headedness}) simply requires the presence of some constituent of the relevant type (without expressing any asymmetry vis-à-vis potential sister constituents). This requirement, as it refers to a single-line hierarchy, is in fact problematic as it implies that for individual (non-top) units in the hierarchy there is a certain type of mother constituent which requires their existence. For compounds, clitic groups or affixed words there is no such mother, as there is no  prosodic constituent which necessarily dominates such daughter constituents. I conclude that the constraint \textsc{Headedness}, apart from the fact that it does not cover typical head properties, hinders a coherent integration of complex words such as compounds into the Prosodic Hierarchy. 

Before turning to the notion of head alignment, consider briefly the question of why \textsc{Headedness} performs so unevenly, expressing exceptionless generalizations for some parts of the hierarchy (\eg a foot must dominate a syllable) while wreaking havoc for other parts. This outcome is hardly unexpected, given the lack of homogeneity of the Prosodic Hierarchy, where some constituents (i.e.\ the phonological word and all higher-level prosodic domains) necessarily involve alignment with morphological or syntactic boundaries, whereas others (the lower constituents) are shaped by parsing the given phonemic material. 

As for the phonological word and higher-level prosodic constituents, restrictions concerning their
presence naturally pertain to the respective morphosyntactic structures from which they
originate. For instance, \ili{German} morphology allows for some recursiveness in compounding, resulting
in various compound categories nested within each other. In fact, even phrases occur regularly as
part of \ili{German} compounds for as long as they do not occupy the morphological head position (\eg
\emph{Mit-dem-Kopf-durch-die-Wand-Mentalität} `head through the wall mentality'). Pertinent
restrictions on compound formation or on morphological parsing will be reflected in the respective
prosodic form, due to the isomorphic mapping of relevant structures. The constraint
\textsc{Headedness} is then neither needed nor suited to express restrictions on relations among relevant
prosodic constituents. There is also no need to invoke \textsc{Headedness} to control dominance
relations among lower prosodic constituents (foot, syllable) as these are inherently defined by the
relation to the constituents they dominate.\footnote{A reviewer asks ``If \textsc{Headedness} [cf.\
  (\ref{ex-headedness}) R.R.] is dispensed with [\dots] then what excludes ill-formed structures
  such as a σ dominating a Σ, a Σ dominating a ω, and so on, from occurring in a language?'' 
An answer lies in lexical semantics. For instance, the definition of a foot as a unit ``consisting
of a group of two or more syllables in which one syllable bears the main stress''
\citep[147]{Trask1996} establishes the concept ``syllable'' as a meronym of the concept
``foot''. This status is manifest in acceptable sentences like \char"221A\emph{A foot has
  syllables}, as opposed to illicit *\emph{A syllable has feet} (See
\citealp[157--180]{cruse:1986:lexical:semantics} for tests relevant to establishing a meronymic relation among words). In the cases of prosodic constituents not due to morphology-prosody mapping it is then the lexical concepts associated with the relevant symbols (\eg Σ, σ), which determine their hierarchical organization and exclude ill-formed dominance relations.}
%\itdopt{footnote 7 vervollständigen EE: Sollte das nicht vllt in ein Bsp sein mit den ganzen Grammatikalitätsurteilen?}

\begin{sloppypar}
Consider next the notion of \emph{head alignment} introduced by \citet{McCarthyPrince1993} in connection with a sub-theory of Optimality Theory known as \emph{Generalized Alignment}. The sub-theory restricts the positions of various constituents relative to one another, expressed in the schema in (\ref{ex-genalign3}), where a designated edge of every prosodic (PCat) or morphological (GCat) constituent of a certain category Cat1 is required to coincide with a designated edge of some other constituent Cat2.
\end{sloppypar}

\ea\label{ex-genalign3}
Generalized Alignment \citep[80]{McCarthyPrince1993}\\
\smallskip
  Align (Cat1, Edge1, Cat2, Edge2)  =\textsubscript{def}\\
  \hphantom{Where} ${\forall}$ Cat1 ${\exists}$ Cat2 such that Edge1 of Cat1 and Edge2 of Cat2 coincide. \\
Where \\
 \hphantom{Where} Cat1, Cat2 ${\in}$ PCat ${\cup}$ GCat\\
 \hphantom{Where} Edge1, Edge2 ${\in}$ \{Right, Left\}
\z
%\itdopt{2004? \textsc{Align}?}
%\itdopt{orignial ex \url{https://scholarworks.umass.edu/cgi/viewcontent.cgi?article=1011&context=linguist_faculty_pubs}}

\noindent
Alignment constraints as defined in (\ref{ex-genalign3}) capture both characteristic properties
associated with heads, obligatoriness and asymmetry. Obligatoriness is captured through the use of
the universal and existential quantifiers, asymmetry is captured through the reference to one
edge, left or right. Edge-orientedness has indeed been recognized as a salient property of heads in syntax (cf.\
Williams' \emph{Righthand-Head rule} \citealp{Williams1981}, see also
\citealp{TrommelenZonneveld1986}) and the idea to use alignment to associate heads with prominence
is already present in the earliest work in Optimality Theory. Specifically, the constraint
\textsc{Align-Head} stated in (\ref{ex-align4}) has been posited to express prominence at the word
level \citep[98]{McCarthyPrince1993}.\footnote{That constraint replaces earlier \textsc{Edgemost}
  proposed by \citet{PS2004a-cite}, which was intended to express the same generalization (\eg
  \textsc{Edgemost}(Hd-F;left;Wd) \citealp[34-38]{PS2004a-cite}, which requires the head foot of the
  phonological word to occur in initial position.)}

\ea\label{ex-align4}
\textsc{Align}(ω, Edge\textsubscript{i}, Head(ω), Edge\textsubscript{i})
\z

\noindent
The constraint in (\ref{ex-align4}) requires a specific edge (left or right) of every phonological
word to coincide with the same edge of its head, thereby giving prosodic head status to the
respective daughter constituent. It thereby accounts for the characteristic edge orientation of
\emph{main} stress (\eg reference to the first syllable or the antepenult syllable). Deviations from regular patterns are captured in terms of constraint interaction in Optimality Theory, where head alignment constraints, too, can be violated under domination (see Section~\ref{sec-regular}).  

How does the analysis in terms of head alignment compare to other approaches to capturing relative prominence? An alternative formalism invokes the labels ``strong'' versus ``weak'' referring to sister constituents. The principle is stated as follows:

\begin{quote}
The relative prominence relation defined for sister nodes is such that one node is assigned the value strong (s) and all the other nodes are assigned the value weak (w) \citep[7]{NesporVogel2007}. 
\end{quote}
%\itdopt{S: Does this paper have roman numbers or arabic? Oben steht xvii und hier 7. EE: Römisch ist vom Vorwort}

\noindent
Strong/weak markings of relative prominence raise a question in cases where a given domain contains only a single or more than two daughter constituents. Although it has been proposed that ``the sole syllable of a monosyllabic foot is by convention strong'' \citep[15]{Selkirk1986}, others have insisted that strong/weak-marking necessarily presupposes binarity: 

\begin{quote}
It [the annotation of tree nodes with the symbols w (for ``weak'') and s (for ``strong''); RR] represents a local property of the tree structure, a relation defined on sister nodes, and the apparent `node labels' s and w cannot have any existence independent of the definition of such a relation. Therefore an isolated [s], an isolated [w], and the configurations [ss] and [ww] are meaningless. \citep[256]{LibermanPrince1977}
\end{quote}

%\largerpage[2]
\noindent
The motivation for a uniform representation for both the (sole) syllable in a monosyllabic word and a stressed syllable in a polysyllabic word as \emph{strong} concerns structural similarities. Consider the pattern that \ili{German} words may contain several schwas but must include at least one full vowel. Assuming the validity of universal constraints requiring phonological words to contain a foot, and syllables to contain a nucleus, this pattern follows from the independently motivated constraints in (\ref{ex-align17}). The head alignment constraint in (\ref{align17a}) assigns head status to the first syllable within the foot, regardless of how many syllables there are in total. The constraint in (\ref{align17b}) bans the presence of schwa in head syllables. 

\eal\label{ex-align17}
\ex\label{align17a}
\textsc{Align}(Σ,L, Head(Σ),L)

\ex\label{align17b}
*\textsc{Schwa}/\sub{$\sigma$Hd}
\zl

\noindent
Satisfaction of both constraints in (\ref{ex-align17}) is demonstrated by the \ili{German} words in Figure~\ref{w-head}. Words with only schwa as in Figures~\ref{ex-tee-align},~\ref{ex-tee-schwa} violate either head alignment or the ban on schwa in head syllables. Words containing no vowel other than schwa are thereby ruled out. 

\begin{figure}
\begin{subfigure}{.3\textwidth}
\centering
\begin{forest}%sm edges
[ω
	[Σ\sub{Hd}
		[$\sigma$\sub{Hd}\\
		$\Delta$\\
		{(e}
		]
		[$\sigma$\\
		$\Delta$\\
		{ə)\sub{ω}}
		]
	]
]
\end{forest}
\caption{\emph{Ehe} `marriage'}\label{w-ehe}
\end{subfigure}
%
\begin{subfigure}{.3\textwidth}
\centering
\begin{forest}%sm edges
[ω
	[Σ\sub{Hd}
		[$\sigma$\sub{Hd}\\
		$\Delta$\\
		{(te)\sub{ω}}
		]
	]
]
\end{forest}
\caption{\emph{Tee} `tea'}\label{w-tee}
\end{subfigure}
%
\begin{subfigure}{.3\textwidth}
\centering
\begin{forest}%sm edges
[ω
	[Σ\sub{Hd}
		[$\sigma$\sub{Hd}\\
		$\Delta$\\
		{(e}
		]
		[$\sigma$\\
		$\Delta$\\
		{bə}
		]
		[$\sigma$\\
		$\Delta$\\
		{nə)\sub{ω}}
		]
	]
]
\end{forest}
\caption{\emph{Ebene} `plain'}\label{w-ebene}
\end{subfigure}
\caption{Wellformedness due to satisfaction of both constraints (\ref{align17a}) and (\ref{align17b}) in German}\label{w-head}
\end{figure}
%
\begin{figure}
\begin{subfigure}{.49\textwidth}
\centering
\begin{forest}%sm edges
[*ω
	[Σ\sub{Hd}
		[$\sigma$\\
		$\Delta$\\
		{(tə)}
		]
	]
]
\end{forest}
\caption{*\textsc{Align}(Σ,L,Head(Σ),L)}\label{ex-tee-align}
\end{subfigure}
%
\begin{subfigure}{.49\textwidth}
\centering
\begin{forest}%sm edges
[*ω
	[Σ\sub{Hd}
		[$\sigma$\sub{Hd}\\
		$\Delta$\\
		{(tə)}
		]
	]
]
\end{forest}
\caption{*\textsc{Schwa}/\sub{σHd}}\label{ex-tee-schwa}
\end{subfigure}
\caption{Illformedness due to violation of either constraint (\ref{align17a}) or (\ref{align17b})}
\end{figure}

Consider now the account of the relevant restrictions on schwa in an alternative formalism in terms
of strong/weak relations among sister constituents. Both monosyllabic feet as in Figure~\ref{w-tee}
and ternary feet as in Figure~\ref{w-ebene} can be accommodated in a binary branching framework
also: the former by admitting empty syllables (see Figure~\ref{ex-tee-binary}), the latter by
allowing foot-internal nested branching (see Figure~\ref{ex-ebene-binary}), (cf.\
\citealp[57]{Giegerich1985}). Here the relevant restriction concerning the distribution of schwa in
\ili{German} could be expressed in terms of requirements on branching structures along with a ban on schwa
in positions dominated by a strong node. Potential empirical differences among the descriptions lie
in (possibly missing) independent motivation, in particular regarding the empty syllable in  Figure~\ref{ex-tee-binary} and the additional /s/ node resulting in an asymmetric representation of the two unstressed syllables in Figure~\ref{ex-ebene-binary}.\footnote{The assumption of empty nuclei in connection with word-final onsets differs is that it is motivated by correlating properties concerning syllable structure and stress (cf.\ \citealp{HarrisGussman2002}, \citet[88--90]{Raffelsiefen2021}. This sort of evidence is missing for the assumption of the empty syllable in Figure~\ref{ex-tee-binary}.}  

\begin{figure}
\centering
\subcaptionbox{\emph{Ehe} `marriage'\label{ex-ehe-binary}}[.3\textwidth]{%
\begin{forest}
%sm edges, 
for children={inner sep=0}
[
	[s
		[/e\hphantom{/}]
	]
	[w
		[ə\rlap{/}]
	]
]
\end{forest}}%
%
\subcaptionbox{\emph{Tee} `tea'\label{ex-tee-binary}}[.3\textwidth]{%
\begin{forest}
%sm edges, 
for children={inner sep=0}
[
	[s
		[/te\hphantom{/}]
	]
	[w
		[$\emptyset$\rlap{/}]
	]
]
\end{forest}}%
%
\subcaptionbox{\emph{Ebene} `plain'\label{ex-ebene-binary}}[.3\textwidth]{%
\begin{forest}
where n children=0{tier=word}{},
%sm edges, 
for children={inner sep=0}
[
	[s, for children={inner sep=0}
		[s
			[/e\hphantom{/}]
		]
		[w
			[bə]
		]
	]
	[w
		[nə\rlap{/}]
	]
]
\end{forest}}
\caption{Alternative representations for the words in Figure~\ref{w-head} in terms of binary branching.}
\end{figure}

%\largerpage[2]
The perhaps most important argument made in support of s/w branching structures concerns the generalization that for any given constituent, regardless of its complexity, there is always a single most prominent syllable. This property, referred to as the \emph{designated terminal element} \citep[259]{LibermanPrince1977}, is captured nicely in binary s/w branching formalisms in that any given constituent contains exactly one syllable dominated exclusively by strong nodes. However, the property in question is captured by the notion of head alignment as well as every prosodic constituent likewise contains exactly one terminal element which itself forms a head while also being dominated exclusively by head constituents (see Sections~\ref{sec-copcom} and~\ref{sec-phracom}). 

\section{``Regular'' compounds}
\label{sec-regular}


To illustrate the relevance of internal morphological structure for word prosody consider the relative prominence patterns in the four-syllable words in (\ref{ex-eggplant}) versus (\ref{ex-oberschiene}), where main and secondary accents are marked.

\eal\label{ex-aubergine}
\ex\label{ex-eggplant}

/ˌobɛʀˈʒinə/ <Aubergine>    `eggplant'\\
% \textipa{/obE\;R\textprimstress Zin@/}%Sieht komisch aus weil nicht Libertine

\ex\label{ex-oberschiene}
/ˈobəʀˌʃinə/ <Oberschiene> `top rail'
\zl

\largerpage
%\enlargethispage{3pt}
\noindent
The words differ regarding their morphological structure in that one is a simplex whereas the other is a compound consisting of two stems. Assuming the morphological structures in (\ref{ex-explant-oberschiene}a,c) and a mapping of morphological to prosodic structures, where every stem (STM) boundary aligns with a corresponding phonological word boundary (ω),%
%
\footnote{Cf.\ \citet{NesporVogel2007} and the arguments for \ili{English} in \citet{Raffelsiefen1999},
  for \ili{German} in \citet[cf.\ footnote 12]{Wiese2000}, \citet{Raffelsiefen2000}.} and where every
morphological compound category (COMP-M) aligns with a prosodic compound category (COMP-P), the
prosodic structures in (\ref{ex-explant-oberschiene}b,d) arise. Square brackets represent the
boundaries of morphological constituents, round brackets those of prosodic constituents. 

\ea
\label{ex-explant-oberschiene}
\begin{multicols}{2}
\ea\label{ex-eggplant-stm}
[[obɛʀʒinə]\sub{STM}]\sub{W}\\

\hphantom{[[obɛ}$\Downarrow$

\ex\label{ex-eggplant-comp}
(obɛʀʒinə)\sub{ω}

\columnbreak

\ex\label{ex-oberschiene-stm}
[[[obəʀ]\sub{STM}[ʃinə]\sub{STM}]\sub{COMP-M}\\

\hphantom{[[[obəʀ]\sub{STM}}$\Downarrow$

\ex\label{ex-oberschiene-comp}
((obəʀ)\sub{ω}(ʃinə)\sub{ω})\sub{COMP-P}
\z
	
\end{multicols}
\z
%\itdopt{EE: Die Abstände waren nicht einheitlich, musste c und d hinzufügen St: Frage: braucht man
%  da überhaupt a und b? Wäre Beispielnummer ohne a, b, c, d nicht ausreichend? Assuming the
%  structures at the top of (\mex{0}), [\ldots] the prosodic structures at the bottom of (\mex{0}) arise. RR: Buchstaben bitte lassen, da auch weiter unten wieder auf einzelne referiert wird}

\noindent
Prosodic words by definition constitute domains for the prosodic organization of phonemic material,
including syllabification and foot formation. Assuming that all syllables are parsed into feet this
would result in the single phonological word \emph{Aubergine} dominating two (trochaic) feet as in
Figure~\ref{fig-distinct-structures} left,
compared to two phonological words in \emph{Oberschiene} each dominating a single foot in
Figure~\ref{fig-distinct-structures} right. Significantly, there are clear judgments regarding the
prominence among the respective sister constituents: within the single phonological word the final
(branching) daughter constituent is strongest whereas within the compound, the initial daughter
constituent is strongest (for reasons of space, the symbol Δ is used to represent the complete
prosodic organization within the respective constituent).\footnote{It is true that speakers may
  declare their inability to assess the location of the main prominence in a given constituent. A
  diagnostic test to be applied here is to expose them to two exaggerated stress patterns, stressing
  the actual main stressed syllable to the extreme vis-à-vis some other syllable (\eg
  (obɛʀˈ\textbf{ʒi}nə)\textsubscript{ω} versus (ˈ\textbf{o}bɛʀʒinə)\textsubscript{ω}), to elicit a
  preference. The robust results obtained for such tests (in this case main stress on the syllable
  /ʒi/) reflect on grammatical knowledge, supporting the sort of head markings shown in the trees as
  in the figures in Figure~\ref{fig-distinct-structures}.} 
\newpage

%\ea Aubergine
%\begin{multicols}{2}
%\ea
\begin{figure}
\hfill
\begin{forest}sm edges
[ω
	[Σ\\
	Δ\\
	{(obɛʀ}%,name=ober
	]
	[$\Sigma$\\
	$\Delta$\\
	{ʒinə)}
	]
]
%\node at (ober.south east){Aubergine};
\end{forest}
\hfill
\begin{forest}sm edges
[COMP-P
	[ω
		[Σ\\
		Δ\\
		{(obəʀ)}%,name=ober
		]
	]
	[ω
		[Σ\\
		Δ\\
		{(ʃinə)}
		]
	]
]
%\node at (ober.south east){Oberschiene};
\end{forest}
\hfill\mbox{}
%
	\caption{\label{fig-distinct-structures}Distinct prosodic constituent structures resulting from alignment with distinct morphological structures}
\end{figure}

\begin{figure}
\centering
\hfill
\begin{forest}%sm edges
[ω
	[Σ\\
	Δ\\
	{(obɛʀ}
	]
	[\textbf{$\Sigma$\sub{Hd}}\\
	$\Delta$\\
	{ʒinə)}
	]
]
\end{forest}
\hfill
\begin{forest}sm edges
[COMP-P
	[\textbf{ω\sub{Hd}}
		[Σ\\
		Δ\\
		{(obəʀ)}%,name=ober
		]
		]
		[ω
		[Σ\\
		Δ\\
		{(ʃinə)}
		]
	]
]
\end{forest}
\hfill\mbox{}
\caption{\label{fig-head-alignment}Right-oriented head alignment in phonological words versus left-oriented head alignment in compounds}
\end{figure}



\largerpage
Given the prosodic structures in Figure~\ref{fig-distinct-structures} the head alignment
constraints in (\ref{ex-align8}) pick out the rightmost foot in \emph{Aubergine} (cf.\
Figure~\ref{fig-head-alignment} left) and the leftmost phonological word in \emph{Oberschiene} (cf.\
Figure~\ref{fig-head-alignment} right), accounting for the accent patterns (primary versus secondary
stress) stated in (\ref{ex-aubergine}). 
\eal\label{ex-align8}
\ex

\textsc{Align}(ω,R, Head(ω),R)
\ex\label{ex-align8b}
\textsc{Align}(COMP-P,L, Head(COMP-P),L)\\
\zl

\noindent
The importance of morphological structure for an adequate mapping to prosodic constituents draws attention to the question of what exactly motivates the concept of a morphological compound. The presence of meaningful morphemes determining the meaning of the whole in a compositional fashion is neither sufficient nor necessary. The insufficiency is demonstrated by so-called root compounds, which may entirely consist of meaningful morphemes, yet form single phonological words. Consider the morphologically complex words in (\ref{ex-sino-ozean}), which are among the 173 words ending in -(\emph{o})\emph{loge} characterized by consistent form-meaning recurrences (\eg X$+$\emph{loge} `expert in X') listed in \citet{Muthmann1989}. Some of those words even include initial morphemes corresponding to free-standing words (\emph{Ozean} `ocean'). All the same, their stress patterns reveal the application of the head alignment referring to the category ω, rather than COMP-P.

\eal\label{ex-sino-ozean}
\ex\label{ex-sinologe}
[zinoˈloɡə] %\hphantom{ɡə'} 
Sin+o+log+e `sinologist'

\ex\label{ex-ozeanologe}
[ot\textsuperscript{s}eɑnoˈloɡə] Ozean+o+log+e `oceanographer'
\zl

\noindent
The assumption that internal stem boundaries necessarily align with phonological word boundaries
along with the observation that the words in (\ref{ex-sino-ozean}) form single domains of prosodic
organization suggest that the complex words in question do not include separate stems. This in turn
suggests that stem-boundaries are not inherent features of morphemes but rather are assigned to
complex words due to conditions on morphological parsing. The string /ʃinə/ in the word
\emph{Oberschiene} qualifies as stem because it occurs in the rightmost position (see
(\ref{ex-oberschiene-stm})), and corresponds to an independent word /ʃinə/ \emph{Schiene} `rail',
which is a hypernym of \emph{Oberschiene}. The string /loɡə/ is not a stem because it does not
correspond to an independent word, the string /ot\textsuperscript{s}eɑn/ in (\ref{ex-ozeanologe}) is
not a stem because it does not occur in the rightmost position of the complex word. There may be
independent reasons to assume internal morphological constituents, perhaps \emph{roots} (\eg
[[ot\textsuperscript{s}eɑn]\sub{R}o[loɡ]\sub{R}ə]), to capture shared properties among the words in
question. What matters is that the constraints aligning phonological word boundaries with
morphological boundaries do not refer to them and the entire words are mapped into single
phonological words (\eg (ot\textsuperscript{s}eɑnoˈloɡə)\textsubscript{ω}). Root compounds do not
yield prosodic compounds.\footnote{Wiese's claim that each root forms a separate phonological word
  \citep[298]{Wiese2000} is refuted not only by evidence from stress but also syllable structure
  (\eg the possible syllabification of a root-initial consonant in the coda when an obstruent
  follows as in /tɛʀ.mɔ\textbf{s.t}ɑt/, \emph{Thermostat} `thermostat', /ho.ʀɔ\textbf{s.k}op/
  \emph{Horoskop} `horoscope', involving the possible roots [stɑt] and [skop], respectively.)} 

\largerpage
Focusing now on the conditions under which a word is morphologically parsed into multiple stems there is evidence for various factors, including those listed in (\ref{ex-factors}):\footnote{The criteria in (\ref{ex-factors}) are in part based on a study of four syllable words in \citew{BPG95a}. A more comprehensive study based on a much larger corpus will likely result in a considerable expansion of the list in (\ref{ex-factors}), along with more information regarding the relative weight of those factors in connection with morphological parsing.}   

\eal\label{ex-factors}

\ex\label{ex-rightmost} 
Position of relevant string: rightmost (morphological head position)\\
     
\ex\label{ex-hyponymy}
Status of relevant string: free (corresponding to a free-standing word)\\
  
\ex\label{ex-free-bound}  
Semantic relation between relevant string and compound: hyponymy\\

\ex\label{ex-stem-end}
Phonological form of potential stem forms: Ending in /ə+({ʀ,l,n})/ \\

\ex\label{ex-phon-viol}
Phonological form of compound: violation of phonotactics (juncture-related) \\
 
\ex\label{ex-phon-rep}
Phonological form of compound: form symmetry (reduplication) \\
    
\zl

\noindent
The factors in (\ref{ex-factors}a,b,c) have been
mentioned in the preceding paragraph in connection with the stem status of \emph{-schiene} in {\it
  Oberschiene}. Such cases where all three of those conditions are jointly satified, known as
``endocentric'' compounds, indicate regular internal stem structure manifest in robust left\hyp
oriented stress placement of main stress in \ili{German}. The sufficiency of
those conditions is illustrated by comparing the words in (\ref{ex-karwoche}) with
(\ref{ex-kartoffel}), none of which start with a string corresponding to a free-standing
word. Moreover the words in each line exhibit somewhat similar syllable and foot structures. The
systematic difference in the stress patterns, namely initial main stress in (\mex{1}a) compared to
main stress on the final trochee in (\mex{1}b), appears then to be entirely due to the status as
endocentric compounds of the words in (\mex{1}a).\footnote{The claim that the relation between
  \emph{Ménopàuse} and \emph{Pause} is a case of hyponymy can be questioned as the meaning of
  \emph{Pause} does not just involve the concept of cessation but typically also a return to some
  prior activity. The stress pattern of the compound can perhaps be seen as evidence that temporary
  cessation suffices here to establish hyponymy. By the same token, the string \emph{-phrase} in
  \emph{Periphráse} `periphrasis' fails to qualify as a hyponym as a \emph{Periphrase} `periphrasis'
  is simply not conceptualized as some kind of \emph{Phrase} `phrase'. Consequently, the word is
  parsed as a single stem, which forms a single phonological words, where main stress falls on the
  final trochee.} 


\eal
\ex\label{ex-karwoche}
\noemph{Kárwòche} `Holy Week' (\noemph{Woche} `week')\\
\noemph{Brómbèere} `blackberry'  (\noemph{Beere} `berry')\\
\noemph{Ménopàuse} `menopause'  (\noemph{Pause} `pause')\\
\noemph{Tsétseflìege} `tsetse fly'  (\noemph{Fliege} `fly') 
\ex\label{ex-kartoffel} 
\noemph{Kartóffel} `potato'  (∄ \noemph{toffel}) \\
\noemph{Trompéte} `trumpet'  (∄ \noemph{pete}) \\
\noemph{Menostáse} `menostasis'  (∄ \noemph{stase}, ∄ \noemph{tase}) \\
\noemph{Gelatíne} `gelatin'  (∄ \noemph{tine}) 
%\itdopt{St: Das sollten zwei Beispiele mit je a, b, c, d sein.}
\zl

%\largerpage[-1]
\noindent
Assuming that the parsing of words into morphological constituents must be exhaustive and provided that the
respective preceding material does not correspond to a recognizable prefix,\footnote{
The recognition of prefixes takes priority by the Elsewhere Condition.
} that material is
classified as a stem as well. Words decomposed into multiple stems are then classified as
(morphological) compounds. The resulting morphological structures are illustrated in
(\ref{ex-karwkar-m}):  

\eal\label{ex-karwkar-m}
\ex\label{ex-karwoche-m}
{[[kɑʀ]\sub{STM}[vɔçə]\sub{STM}]\sub{COMP-M}} \noemph{Karwoche} `Holy Week'\\

\ex\label{ex-kartoffel-m}
{[[kaʀˈtɔfəl]\sub{STM}]\sub{W}} \noemph{Kartoffel} `potato'\\

\ex\label{ex-menopause-m}
{[[meno]\sub{STM}[pauzə]\sub{STM}]\sub{COMP-M}} \noemph{Menopause} `menopause'\\

\ex\label{ex-menostase-m}
{[[menɔstɑzə]\sub{STM}]\sub{W}} \noemph{Menostase} `menostasis'\\

\zl

\noindent
The isomorphic mapping of the morphological structures in (\ref{ex-karwkar-m}) to prosodic structures yields the outputs in (\ref{ex-karwkar-p}). The position of main stress then follows from the relevant head alignment constraints: right-oriented for single phonological words but left-oriented for regular compounds (see (\ref{ex-align8})).

\eal\label{ex-karwkar-p}
\ex\label{ex-karwoche-p}
((ˈkɑʀ)\sub{ωHd}(ˈvɔçə)\sub{ω})\sub{COMP-P} \noemph{Karwoche} `Holy Week'\\

\ex\label{ex-kartoffel-p}
(kaʀˈtɔfəl)\sub{ω} \noemph{Kartoffel} `potato'\\

\ex\label{ex-menopause-p}
((ˈmeno)\sub{ωHd}(ˈpauzə)\sub{ω})\sub{COMP-P} \noemph{Menopause} `menopause'\\

\ex\label{ex-menostase-p}
(ˌmenɔsˈtɑzə)\sub{ω} \noemph{Menostase} `menostasis'\\

\zl

\noindent
The relevance of hyponymy, rather than just the presence of a string corresponding to a free-standing word, is demonstrated by the cases in (\ref{ex-Berga-meter}). Those in (\ref{ex-Bergamotte}) satisfy conditions (\ref{ex-rightmost}) and  (\ref{ex-hyponymy}), but clearly are not cases of hyponymy. The relevant assessment is more uncertain in (\ref{ex-kilometer}): the statement that a \emph{Kilometer} is kind of a \emph{Meter} is hardly acceptable, yet the relevant strings are semantically closely related and also have the same gender (masculine nouns). 


\eal\label{ex-Berga-meter}
\ex\label{ex-Bergamotte}
/ˌbɛʀɡɑˈmɔtə/ \noemph{Bergamotte} `bergamot' (∃ \noemph{Motte} `moth')  \\ 
/ˌt\textsuperscript{s}ɛntʀiˈfuɡə/ \noemph{Zentrifuge} `centrifuge' (∃ \noemph{Fuge} `joint')  \\ 

\ex\label{ex-kilometer}
/ˌkiloˈmetəʀ/ $\sim$ /ˈkiloˌmetəʀ/  \noemph{Kilometer} `kilometer' (∃ \noemph{Meter} `meter')  \\ 
/ˌt\textsuperscript{s}ɛntiˈmetəʀ/ $\sim$ /ˈt\textsuperscript{s}ɛntiˌmetəʀ/  \noemph{Zentimeter} `centimeter' (∃ \noemph{Meter} `meter')  \\ 


\zl


\largerpage[1]
\noindent
It is presumably the uncertainty concerning the question of what exactly qualifies as hyponymy which manifests in stress variation. Final main stress indicates a single phonological word, itself indicative of a single stem originating from the non-acceptance of \emph{Kilometer} as a hyponym of \emph{Meter}. Initial main stress originates in the acceptance of that hyponymy, which yields a morphological and prosodic compound structure.

\eal\label{ex-Berga-meter-P}
\ex\label{ex-Bergamotte-P}

(ˌbɛʀɡɑˈmɔtə)\sub{ω} \noemph{Bergamotte} `bergamot'   \\ 

\ex\label{ex-pumpernickel-P}
(ˌkiloˈmetəʀ)\sub{ω} $\sim$ ((kilo)\sub{ωHd}(metəʀ)\sub{ω})\sub{COMP-P} \noemph{Kilometer} `kilometer'   \\ 

\zl

\noindent
While (some sort of) hyponymy appears to be mostly sufficient for inducing left-oriented head alignment\footnote{There are rare cases like \emph{Pfefferminze} `peppermint', which is often pronounced with main stress on the final trochee despite being a hyponym of \emph{Minze} `mint'. Here initial main stress also occurs and is certainly acceptable. Cases like \emph{Karfréitag} `Good Friday', \emph{Rosenmóntag} `Rose Monday', where initial main stress is unacceptable, differ in that the compounds are proper nouns, which inherently resist participation in hyponymy relations.} it is not a necessary condition. Consider the words ending in the string \emph{-peter} in (\ref{ex-peter-w}), where hyponymy clearly fails as that string corresponds only to a derogatory term for males or to the male name \emph{Peter}. Yet the words in (\ref{ex-wackelpeter}) exhibit apparently stable initial main stress.\footnote{These words have been attested for over a century where the original role of the constituent \emph{Peter} is unclear according to \ili{German} etymological dictionaries.}


\eal\label{ex-peter-w}

\ex\label{ex-salpeter}
/ˌzalˈpetəʀ/ \noemph{Salpeter} `saltpeter'  \\ 

\ex\label{ex-wackelpeter}
/ˈvakəlˌpetəʀ/ \noemph{Wackelpeter} wobble.Peter `jello' \\
/ˈt\textsuperscript{s}iɡənˌpetəʀ/ \noemph{Ziegenpeter} goat.Peter `mumps' \\
/ˈhakəˌpetəʀ/ \noemph{Hackepeter} chop.Peter `ground meat' \\

\zl

\noindent
The relevant prosodic structures are shown in (\ref{ex-peter-P}), one associated with right-oriented
head alignment determining main stress in the phonological word \emph{Salpeter}, the other with
left-oriented head alignment determining the main stress in the compound \emph{Wackelpeter} (see (\ref{ex-align8})).  



\eal\label{ex-peter-P}
\ex\label{ex-salpeter-P}
(ˌzalˈpetəʀ)\sub{ω} \noemph{Salpeter} `saltpeter'   \\ 
\ex\label{ex-wackelpeter-P} 
((vakəl)\sub{ωHd}(petəʀ)\sub{ω})\sub{COMP-P} \noemph{Wackelpeter} `jello' \\ 

\zl

\noindent
The different prosodic structures in (\ref{ex-peter-P}) draw attention to properties relevant for
morphological parsing which pertain to the initial compound member. They indicate the possible
significance of final schwa syllables, presumably connected to their typical restriction to
stem-final position in \ili{German}. This holds in particular for closed schwa syllables, which, along
with other closed syllables containing tense vowels, diphthongs or coda clusters, are mostly
confined to phonological word-final position in \ili{German}. The words in (\ref{ex-hellnickel}), which clearly are
not endocentric compounds and consist of more or less meaningless trochees only, support the
significance of the presence of closed schwa syllables. The words in (\ref{ex-pumpernickel}), where
feet end in closed schwa syllables, appear to be marked by a stable compound prosody, manifest in robust
initial main stress, whereas the former compound in (\ref{ex-hellebarde}) has been reanalyzed as a
simple stem with main stress on the final trochee.  


\eal\label{ex-hellnickel}

\ex\label{ex-pumpernickel}
/ˈpʊmpəʀˌnɪkəl/ \noemph{Pumpernickel} `pumpernickel'  \\ 
/ˈknɪkəʀˌbɔkəʀ/ \noemph{Knickerbocker} `knickerbockers'  \\ 

\ex\label{ex-hellebarde}
/ˌhɛləˈbaʀdə/ \noemph{Hellebarde} `halberd'   \\ 

\zl

\noindent
The possible significance of phonotactic violations indicative of the presence of internal stem boundaries is supported by various additional cases exhibiting word-initial main stress. None of the words in (\ref{ex-feldwebel}) ends in a string which matches an independent word and some do not contain any independently recurring strings. Yet, all of these words exhibit seemingly stable prosodic compound structure, with main stress on the initial member: 

\ea\label{ex-feldwebel}

\begin{tabular}[t]{@{}l@{~~}ll@{}}
a.& Féldwèbel `sergeant'  & (∃ \noemph{Feld} `field', ∄ \noemph{webel}) \\
  &Táusendsàssa `jack-of-all-trades' &  (∃ \noemph{tausend }'thousand', ∄ \noemph{sassa}) \\
b.&Quácksàlber `charlatan'  &(∄ \noemph{quack}, ∄ \noemph{salber})  \\
  &Káulquàppe `tadpole'  & (∄ \noemph{kaul}, ∄ \noemph{quappe})
\end{tabular}
\z%


\noindent
The parsing of the words in (\ref{ex-feldwebel}) as compounds is likely motivated by certain deviations from regular phonotactics. Specifically, the intervocalic clusters shown in the lefthand column in (\ref{ex-feldwebel-v}) are such that they violate word-internal phonotactic constraints (\eg constraints against complex codas, against onsets with an insufficient sonority increase, against certain mismatches in voicing) but are consistent with the assumption of the internal phonological word boundaries shown in the middle column. 

\ea\label{ex-feldwebel-v}

\begin{tabular}[t]{@{}l@{~~}ll@{~~}ll@{}}
a. & ∄ (...ldv...)\sub{ω} & b. & ((fɛld)\sub{ωHd}(vebəl)\sub{ω})   & `sergeant' \\
   & ∄ (..ulkv...)\sub{ω} &    & ((kaul)\sub{ωHd}(kvapə)\sub{ω})   & `tadpole' \\
   & ∄ (...ndz...)\sub{ω} &    & ((tauzənd)\sub{ωHd}(zasɑ)\sub{ω}) & `jack-of-all-trades'  \\
   & ∄ (...kz...)\sub{ω}  &    & ((kvak)\sub{ωHd}(zalbəʀ)\sub{ω})  & `charlatan' 
\end{tabular}
\z


\noindent
The idea that certain violations of word-internal phonotactics might motivate a parsing of the word as a compound also accounts for the prosody of the root compounds in (\ref{ex-measure-dev}). All of those words are neuter nouns ending in the string \emph{-meter} associated with a measuring device, preceded by a string which corresponds to a unit of measurement (\eg \emph{Var} `var', \emph{Ohm} `ohm'). The stress on the final trochee in (\ref{ex-varmeter}) is expected, as \emph{-meter} does not correspond to an independent word and the compounds are therefore not endocentric. The unexpected placement of main stress on the initial syllable in (\ref{ex-voltmeter}) again correlates with phonotactic violations. Unlike the wellformed intervocalic cluster /ʀm/ the clusters /ltm/ or the geminate /mm/ violate \ili{German} phonotactics. The assumption that such violations motivate the decomposition of a word into a compound accounts then for the stress differences in (\ref{ex-measure-dev}), in particular, the placement of main stress on the initial syllable in (\ref{ex-voltmeter}):  

\eal\label{ex-measure-dev}

\ex\label{ex-varmeter}
/ˌvɑʀˈmetəʀ/ \noemph{Varmeter} `varmeter' (\noemph{Var} `var')  \\ 
/amˌpeʀˈmetəʀ/ \noemph{Amperemeter} `amperemeter' (\noemph{Ampere} `ampere')  \\ 
 
\ex\label{ex-voltmeter}
/ˈvɔltˌmetəʀ/ \noemph{Voltmeter} `voltmeter' (\noemph{Volt} `volt')  \\ 
/ˈomˌmetəʀ/ \noemph{Ohmmeter} `ohmmeter' (\noemph{Ohm} `ohm')  \\ 

\zl

\noindent
The relevant prosodic structures can be illustrated with (ˌvɑʀˈmetəʀ)\sub{ω} versus ((vɔlt)\sub{ωHd}(metəʀ)\sub{ω})\sub{COMP-P}, marked by left-oriented head alignment.

Yet another, more marginal, factor likely influencing morphological parsing concerns the presence of certain repetitions in the structure of words (cf.\ (\ref{ex-phon-rep})). This influence explains the presence of initial main stress in the words consisting of two similar trochees in (\ref{ex-pillepalle}). The observation that main stress falls on the final trochee in (\ref{ex-wischiwaschi}) indicates the relevance of the presence of final schwa syllables for morphological parsing (cf.\ (\ref{ex-stem-end})).


\eal\label{ex-Pille-Waschi}
\ex\label{ex-pillepalle}
/ˈtɪŋəlˌtaŋəl/ \noemph{Tingeltangel} `honky-tonk'  \\ 
/ˈpɪləˌpalə/ \noemph{Pillepalle} `trivia'  \\ 

\ex\label{ex-wischiwaschi}
/ˌvɪʃiˈvaʃi/ \noemph{Wischiwaschi} `wishy-washy'   \\ 

\zl

\noindent
The relevant prosodic structures are then ((tɪŋəl)\sub{ωHd}(taŋəl)\sub{ω})\sub{COMP-P}, marked by left-oriented head alignment referring to regular compounds, versus the single phonological word (ˌvɪʃiˈvaʃi)\sub{ω}, which is subject to right-oriented head alignment (see (\ref{ex-align8})).

\largerpage
A possible testing ground for exploring the relative weight of the various types of factors listed
in (\ref{ex-factors}) are data from loan word adaptation illustrated in (\ref{ex-attentat}). All of
these words entered the language in the form of segment strings associated with a prominent foot at
the word end. The adaptation of that prominence pattern in \ili{German} indicates that some of these words
were parsed as compounds, others as simplexes. 


\ea\label{ex-attentat}

\begin{tabular}[t]{@{}lll@{~}l@{}}
a.&Old French\il{French!Old}& attentát ⇒ & [ˈatənˌ{}tɑt]\sub{\textsc{n.neut}}\\
&&&(∄ \noemph{atten}, ∃ [tɑt]\sub{\textsc{n.fem}} \noemph{Tat} `deed')\\
&&&	\noemph{Attentat} `assassination attempt'\\
b.&Old French\il{French!Old}& aventúre ⇒ & [ˈɑbənˌ{}tɔiəʀ]\sub{\textsc{n.neut}}\\
&&& (∄ \noemph{aben}, ∃ [tɔiəʀ]\sub{\textsc{A}} \noemph{teuer} `expensive')\\
&&& \noemph{Abenteuer} `adventure'\\
c.& \ili{French}& pamplemóusse ⇒ &  [ˈpampəlˌmuzə]\sub{\textsc{n.fem}}\\
&&& (∄ \noemph{Pampel}, ∃ \noemph{muse})\\
&&& \noemph{Pampelmuse} `grapefruit'\\
\end{tabular}

\begin{tabular}[t]{@{}lll@{~}l@{}}
d.& \ili{Dutch}& appelsína ⇒ &  [ˌap\textsuperscript{f}əlˈzinə]\sub{\textsc{n.fem}}\\
&&& (∃ \noemph{Apfel} `apple', ∄ \noemph{sine})\\
&&& \noemph{Apfelsine} `orange'\\
e.& \ili{Latin}& petrosílium ⇒& [ˌpetəʀˈziliə]\sub{\textsc{n.fem}}\\
&&& (∃ \noemph{Peter}, ∄ \noemph{silie})\\
&&&	\noemph{Petersilie}	`parsley'\\
f.& \ili{Italian}& cavoli rápe ⇒& [ˌkolˈʀɑbi]\sub{\textsc{n.masc}}\\
&&& (∃ \noemph{Kohl} `cabbage', ∄ \noemph{rabi})\\
&&&	\noemph{Kohlrabi}	`cabbage turnip'\\
g.& \ili{Italian}& intermézzo ⇒& [ˌɪntəʀˈmɛt\textsuperscript{s}o]\sub{\textsc{n.neut}}\\
&&& (∃ \noemph{inter} `inter', ∄ \noemph{mezzo})\\
&&&	\noemph{Intermezzo}	`interlude'\\
\end{tabular}
\z


\largerpage
\noindent
Consider the stress shift to the initial syllable in \emph{Attentat}, which correlates with the
accidental correspondence of the final string with the common \ili{German} word \emph{Tat} `deed', which
may have been accepted as a hypernym.\footnote{This is not a clear case of endocentricity as the
  gender in the words \emph{Attentat} and \emph{Tat} does not match. Still, morphological parsing of
  \emph{Attentat} with reference to the noun \emph{Tat} `deed' is supported by the formation of the
  agentive \emph{Attentäter} `assassin', which matches the regular agentive based on \emph{Tat}
  (i.e.\ \emph{Täter} `culprit'). Further support comes from the historical plural form
  \emph{Attentaten}, which matches the plural of \emph{Tat} (\emph{Taten} `deeds'). (Eventually the
  plural was changed to \emph{Attentate}, conforming to the regular plural ending for polysyllabic
  neuter nouns.)} The presence of the preceding foot ending in a closed schwa syllable, presumably a
spelling pronunciation, may have further motivated the parsing of the word as a compound. The
presence of internal closed schwa syllables is also a likely important factor for parsing the words
\emph{Ábenteuer} and \emph{Pámpelmuse} as compounds. This parsing, indicated by the shift of the
main stress to the initial syllable, correlates with the correspondence of the respective word-final
strings to independent words (i.e.\ \emph{teuer} `expensive' and \emph{Muse} `muse'), which,
however, are clearly not hypernyms. The relevance of the status of the final string is demonstrated
by the remaining cases, none of which ends in a string corresponding to a free-standing
word. Consider the cases of \emph{Apfelsíne} and \emph{Petersílie}, where stress remained on the
penult, indicative of the parsing of the words as single stems. This is despite the presence of a
word-initial string which ends in a closed schwa syllable and also corresponds to an independent
word.\footnote{The existence of other words for citrus fruit ending in the stressed string {-íne}
  (\emph{Mandaríne} `tangerine', \emph{Klementíne} `clementine') may also have stabilized the stress
  in \emph{Apfelsíne}.} The likely importance of correspondence of the word-final string to some
free-standing word is further supported by the persistence of penult main stress in \emph{Kohlrábi}
and \emph{Intermézzo}. Here stress indicates the parsing of these words as a single stem, despite
the presence of various cues indicative of morphological complexity (\eg the presence of schwa or a
tense vowel in a closed syllable). The observation that similar violations appear to have been
sufficient to motivate the parsing as a morphological compound elsewhere, as in \emph{Káulquàppe},
may indicate the relevance of the final schwa syllable. The absence of a final
string matching an independent word, along with the presence of the final full vowel, may exclude the
parsing of  \emph{Kohlrábi} and \emph{Intermézzo} as a compound in \ili{German}.\footnote{The cases
  \emph{Apfelsíne} and \emph{Kohlrábi} also suggest that the restructuring of the initial string in
  analogy with semantically related existing words (\eg \emph{Apfel} `apple' and \emph{Kohl}
  `cabbage') does not necessarily indicate recognition of stem structure. As was noted earlier in
  connection with the word \emph{Ozeanologe} in (\ref{ex-ozeanologe}), the morphological parsing of
  a word is not determined by the inherent status of word-internal material.} 

Summarizing the review of potential factors motivating the morphological analysis of a word as a compound, rather than as a single stem, all of the factors listed in (\ref{ex-factors}) are likely to play some role. It appears that endocentricity is not a necessary prerequisite for the parsing of a word as a compound, as other factors, or perhaps certain combinations of factors, may also suffice to motivate internal stem structure. A finding of particular interest here is that the data support a top-down approach to morphological parsing. This is because the identification of relevant factors, including phonotactic violations, presupposes access to the complete word, not individual morphemes. 

Focusing now on the topic of head alignment constraints it appears that the simple correlation
established so far, left-oriented head alignment for compounds and right-oriented head alignment for
simplexes (see (\ref{ex-align8})), is riddled with exceptions. Consider first the stress patterns in the apparent simplexes in (\ref{frikadelle}), which indicate that the pattern of the final foot carrying main stress holds only for words consisting of (at least) two trochees (see (\ref{frikadelle}a)). If the final foot contains only one syllable as in the trisyllabic words in (\ref{frikadelle}b,c), the position of the head foot is determined lexically and is hence potentially contrastive.

\largerpage
\ea\label{frikadelle}
\begin{tabular}[t]{@{}l@{~~}ll@{~~}ll@{~~}l@{}}
a. & (ˌfʀɪkɑˈdɛlə)\sub{ω} &b. & (ˌkaʀʊˈsɛl)\sub{ω}  &c. & (ˈdet\textsuperscript{s}iˌbɛl)\sub{ω} \\
   & \noemph{Frikadelle}    &   & \noemph{Karussell}    &  & \noemph{Dezibel}  \\
   & `meatball'           &   &  `merry-go-round'   &  & `decibel' \\
   & (ˌobɛʀˈʒinə)\sub{ω}   &   & (ˌmaɡɑˈt\textsuperscript{s}in)\sub{ω} &  & (ˈtʀampoˌlin)\sub{ω} \\
   & \noemph{Aubergine}     &   & \noemph{Magazin}      &  &  \noemph{Trampolin}  \\
   & `eggplant'           &   & `magazine'          &  & `'trampoline'\\
   & (ˌɪntəʀˈmɛt\textsuperscript{s}o)\sub{ω} &  & (ˌɪntəʀˈval)\sub{ω} &  & (ˈɪntəʀˌviu)\sub{ω} \\
   & \noemph{Intermezzo}    &   & \noemph{Intervall}  &  &  \noemph{Interview} \\
   & `interlude'          &   & `interval'        &  & `interview'\\
\end{tabular}
\z

\noindent
Trisyllabic words ending in {/ə/, /əʀ/, or /ən/} also exhibit potential contrast regarding stress, where either the final trochee or the initial monosyllabic foot can form the head foot (see (\ref{pantomime}b,c)). Again, phonological words consisting of two trochees have stable main stress on the final syllable, regardless of their ending (see (\ref{pantomime}a)).

\ea\label{pantomime}
\begin{tabular}[t]{@{}l@{~~}ll@{~~}ll@{~~}l@{}}
a. & (ˌpantoˈmimə)\sub{ω}  & b. & (ʀeˈklɑmə)\sub{ω}  & c. & (ˈbʀoˌzɑmə)\sub{ω} \\
   & \noemph{Pantomime}      &    & \noemph{Reklame}     &    & \noemph{Brosame} \\
   & `pantomime'           &    & `advertising'      &    & `crumb'\\
   & (ˌkandeˈlɑbəʀ)\sub{ω} &    & (ʀɑˈbaʀbəʀ)\sub{ω} &    & (ˈbɛʀˌzɛʀkəʀ)\sub{ω} \\
   & \noemph{Kandelaber}     &    & \noemph{Rhabarber}  &    & \noemph{Berserker} \\
   & `candelabra'          &    & `rhubarb'          &    & `berserk' 
\end{tabular}
\z


\noindent
Dependencies of phonological regularities on particular contexts as in (\ref{frikadelle}) or
(\ref{pantomime}) actually support the assumption of head alignment constraints as in
(\ref{ex-align8}). This is because such dependencies lend themselves to modeling in terms of ranked
constraints in a framework such as Optimality Theory, where grammar consists of constraint
interaction.\footnote{The relevant analysis would invoke faithfulness constraints requiring preservation of head feet in input forms, which make their force felt in the respective three-syllable words in (\ref{frikadelle}/\ref{pantomime}b,c), but not in phonological words consisting of two trochees as in (\ref{frikadelle}/\ref{pantomime}a), where markedness constraints prevail.} 

The relevant stress patterns support not only the notion of head alignment, conceived of as a violable constraint, but also the assumption of the two distinct head alignment constraints in (\ref{ex-align8}). The two constraints differ not only with respect to the constituent targeted for alignment and in their orientation (right- versus left-oriented) but also in the way they interact with other constraints. For ``regular'' compounds consisting of two members it holds that main stress always falls on the initial member, regardless of its size or phonological structure. The relevant head alignment constraint is hence undominated by phonological markedness, unlike the head alignment constraint operating within phonological words.  
    
\largerpage
To capture the regularities pertaining to the position of main stress in words it is not only necessary
to distinguish compounds from simplexes (and consider the internal foot structure in simplexes, see
(\ref{frikadelle}), (\ref{pantomime})), but also to distinguish among
various types of compounds. In particular there are two special compound types, ``copulative'' and
``phrasal'',  which both associate with head alignment referring to their rightmost margin. The
classification of all three compound types originates from differences in morphological structure,
which are mapped to the respective prosodic constituents as shown in (\ref{ex-W}) (COPCOMP =
copulative compound, PHRASCOMP = phrasal compound): 

\ea\label{ex-W}
COPCOMP-M ⇒ COPCOMP-P\\
PHRASCOMP-M ⇒ PHRASCOMP-P\\
COMP-M ⇒ COMP-P\\
W ⇒ ω\\
 \z
 
\noindent
The segmental material of a given string may allow for various morphological parsings, which raises
the question of how to resolve potential conflicts. Here the Elsewhere Principle comes into play,
meaning that more specific conditions take precedence over less specific ones. \emph{Copulative
  compounds} are characterized by the most specific condition in that the relation between all
members is restricted to the effect that all must be equal in certain ways (see
Section~\ref{sec-copcom}). \emph{Phrasal compounds} are defined by a range of relations between its
members, where all of those relations are ``phrasal'' in some sense to be made explicit (see
Section~\ref{sec-phracom}). \emph{Regular compounds} are mostly endocentric but can lack semantic
restrictions altogether (cf.\ cases like \emph{Abenteuer, Kaulquappe} discussed above). The more
specific conditions motivating the parsing of a word into stems, as opposed to parsing it as a
single stem, have been demonstrated in some detail in this section. The order among the four
morphological categories in (\ref{ex-W}) then mirrors the level of restrictedness pertaining to the
conditions on morphological parsing, with the most specific type listed first (i.e.\ copulative
compounds) and the default listed last (simplexes).\footnote{\label{fn-any-evidence}Evidence in
  support of the distinctions among the categories in (\ref{ex-W}) argues against the alternative
  representations of the relevant complex words in terms of so-called \emph{recursive phonological
    words}. Such structures have been advocated by \citet[6]{McCarthyPrince1993} and
  \citet{ItoMester2009}, where the prosodic difference between for instance \emph{Aubergine} versus
  \emph{Oberschiene} would be represented as (obɛʀʒinə)\sub{ω} versus
  ((obəʀ)\sub{ω}(ʃinə)\sub{ω})\sub{ω} instead. Adequate reference to the sort of distinct prosodic
  domains motivated by the properties mentioned above could be achieved only by employing various
  diacritics, which would then amount to a notational variant of the prosodic categories in
  (\ref{ex-W}). For relevant discussion see \citep[xvii]{NesporVogel2007}),
  \citep[150--152]{Vogel2010}.}  


In ending this section, I will briefly draw attention to the occurrence of idiosyncratic stress properties in compounds, such as final main stress in compounds ending in \emph{pie} in \ili{English} (\eg \emph{apple-píe,  meat-píe} versus \emph{ápple cake, méatball}). These could be captured by marking relevant stems as head constituents in the lexicon, where this marking requires alignment with a margin position in the relevant compound. Hence, \emph{pie} associates with main stress only when occurring in the rightmost position of a compound, not as an inherent property of that stem. Interestingly, such cases motivating the diacritic marking of lexical items as compound heads, while somewhat common in \ili{English} \citep[144--149]{Fudge1984}, appear not to exist in \ili{German}.

\section{Copulative compounds}
\label{sec-copcom}

Copulative compounds are characterized by a structure which can be exhaustively parsed into two or more stems which are “on equal footing” \citep[83]{Bauer2017}.\footnote{Bauer writes: “...contains two elements which are on equal footing, not one which is subordinate to the other..." The wording “contains two” is inaccurate here in that it allows for additional (non-equal) material to be included in such a compound and also wrongly excludes the possibility of copulative compounds consisting of three or more members.}
The expression “on equal footing” captures the essence of copulative compounds, which are restricted to the effect that each member must exhibit the same relation to the respective mother constituent. This entails that members may not differ regarding their category (\eg no combinations of adjective and noun stems) and must belong to the same lexical field (\eg only color terms, only certain kinds of names (for instance, only river names)). The possibility that all members are meaningless also exist, where shared properties concern phonology (\eg only syllables starting with the same onset consonant as in \ili{English} \emph{tic-tac-toe}). Copulative compounds may include any number of members arranged in a strictly flat structure. Main stress associates systematically with the final member, perhaps a cross-linguistic property of such compounds. This restriction is captured by the right-oriented head alignment constraint in (\ref{ex-align21}), distinct from the left-oriented constraint associated with regular compounds (cf.\ (\ref{ex-align8})). 

\ea\label{ex-align21}
\textsc{Align}(COPCOMP-P, R, Head(COPCOMP-P), R)
\z

\noindent
The head alignment constraint as stated in (\ref{ex-align21}) draws attention to the fact that the terms “copulative” or “coordinative” compound are frequently used to refer to words with initial main stress such as \emph{Mántelkleid} `coat dress' or {Díchtermaler} `poet painter' in \ili{German} (cf.\ \citealp{OrtnerOrtner1984}). Assuming the validity of the head alignment constraint in (\ref{ex-align21}) this raises the question of whether relevant forestressed compounds differ independently  from those with stress on their final member. I will argue that such differences exist, pertaining to semantics but also to (segmental) phonology. This section aims then to identify necessary and sufficient conditions for motivating the classification of words as “copulative” in \ili{German}. The study focuses first on conditions pertaining to free-standing compounds, followed by cases where a (copulative) compound appears embedded as the first member in a complex compound.  

Correlations between stress and meaning differences relevant to the question of how to delineate copulative compounds in \ili{German} can be illustrated with words consisting of combinations of color adjectives. Compare the compound with final stress in (\ref{ex-blauweiß}) with that in  (\ref{ex-graugrün}), which exhibits initial stress:

\eal\label{ex-blaugrau}
\ex\label{ex-blauweiß}
Die griechische Fahne ist \emph{blauwéiß}. \\
`The Greek flag is blue and white.'

\ex\label{ex-graugrün}
Ihre Augen sind \emph{gráugrün}.\\
`Her eyes are gray-green.'
\zl

\noindent
The color term compounds in (\ref{ex-blaugrau}) have been noted to differ in that in (\ref{ex-blauweiß}) the individual members associate with distinct entities (\eg distinct stripes in the flag) while those in (\ref{ex-graugrün}) refer to a single entity\footnote{I.e.\ a pair of eyes, or rather the respective irises.}, yielding some sort of modifying reading:
`grayish green eyes' \citep[44]{Puempeletal1992}.%
%
\footnote{The authors do not mention the correlation between the relevant semantic differences and differences in stress. Those are attested for instance in the transcriptions in the online \emph{Duden} dictionary, where \emph{graugrün} is transcribed with only the initial syllable marked for stress, while the representation of \emph{blauweiß} includes stress marks for both syllables (\url{https://www.duden.de}). The marking of both members for stress is the general convention adopted there for representing final stress in two-member compounds.} 
These data suggest then that only the compound with final main stress, \emph{blauwéiß}, qualifies as
copulative. This is because the modifier versus head roles attributed to the two compound members in
the fore-stressed compound \emph{gráugrün} indicate an asymmetry which does not fit the notion of a
copulative compound.  

The question is then which property ultimately distinguishes the two compound types in (\ref{ex-blaugrau}) and determines the difference in stress. Taking the descriptions of the semantic differences stated above as a point of departure, the relevant difference could pertain to properties of referents, that is, their lending themselves to some sort of dissection into mutually exclusive parts, each to be associated with one of the compound members.\footnote{See the notion of ``heteroreferential'' versus ``homoreferential'' compounds discussed by \citet[608]{Renner2008}.} Such an approach would be odd in that  it implies a dependence of morphological classification on properties pertaining to the physical characteristics of entities in the extralinguistic world.
 
 The other difference mentioned, namely the asymmetry associated with the modifier-head roles, is a
 more plausible route to finding factors relevant to determining morphological categorization. The
 asymmetry in question appears to be connected to the relation between the compound and its
 rightmost member, which is hyponymic in the case of \emph{graugrün} `gray-green', but not in the
 case of \emph{blauweiß} `blue and white'. That is, \emph{graugrün} can be considered a kind of
 green but \emph{blauweiß} cannot be considered a kind of white. Assuming that there is a natural
 preference for interpreting a compound as a hyponym of its semantic head constituent (i.e.\ in
 \ili{German} the member in rightmost position), and assuming further that the recognition of such a
 relation tends to relegate the preceding member to a subordinate status, it follows that hyponymy
 hinders the classification of a compound as copulative. The classification as a regular compound
 ensues by default, resulting in initial main stress in \emph{gráugrün}, analogous to other endocentric compounds where the initial constituent functions as modifier such as \emph{héllgrün} `light green' or \emph{ápfelgrün} `apple green'.
 
The classification of a compound as copulative is then possible only if hyponymy fails and, moreover, the relevant daughter constituents are assessed to be ``on equal-footing'' (for instance, all are color adjectives). These conditions are met in the compound \emph{blauweiß} `blue-white', which consequently is subject to right-oriented head alignment manifest in final main stress (\emph{blauwéiß}). 

The two relevant claims are then that there is a preference for interpreting a compound as endocentric (as long as it can be construed as a hyponym of its semantic head constituent) and that \ili{German} grammar does not allow ``double-endocentric'' compounds.
As for the analysis of given words, this approach results in compounds with stress on the initial member functioning as a modifier, whenever the hyponymy in question can be established. For instance, the compound \emph{Kö́niginmutter} `queen mother' denotes a kind of \emph{Mutter} `mother' meeting the requirement that the compound is a hyponym of its semantic head constituent, which results in the classification as a regular compound manifest in stress on the initial member.\footnote{The compound \emph{Kö́niginmutter} is clearly not double-endocentric, in fact, the referent cannot be a monarch herself. The particular modification which has lexicalized here according to dictionaries is that \emph{Kö́niginmutter} designates the mother of a reigning monarch, who can be female or male.} When creating novel words ``double"=endocentricity'' is avoided in \ili{German} by resorting to syntax. To express the concept of a \emph{hunter-gatherer}, meant to convey the notion of someone being both a \emph{hunter} and a \emph{gatherer} in equal measure, the conjunction \emph{und} is employed \emph{Jäger und Sammler} or \emph{Sammler und Jäger}, consisting of the nouns (\emph{Jäger} `hunter' and \emph{Sammler} `gatherer'). \ili{German} differs then from \ili{English}, where \emph{queenmóther} has final stress and the compound \emph{hunter-gátherer} does apparently convey an equal semantic status between the compound members.\footnote{\ili{English} appears to resist the formation of adjective compounds characterized by ``equal-footing'', regardless of possible hyponymy (\eg \emph{gray-green}, where \emph{gray} modifies, but \emph{black and white}, \emph{sweet and sour}).}

The assumption of a categorical difference between the two compound types illustrated in
(\ref{ex-blaugrau}) is also manifest in the sequencing of the members. Endocentric compounds in
\ili{German} (or \ili{English}) are characterized by the presence of the semantic head in the rightmost position
whereas the order among the members of copulative compounds is determined by the respective inherent
properties of each member and relevant asymmetries. That is, the order between the members
\emph{grau} and \emph{grün} in word-formation hinges on the intended meaning, \emph{graugrün} is
chosen to express a shade of green while \emph{grüngrau} is chosen to express a shade of gray. By
contrast, the order between \emph{blau} und \emph{weiß} is determined by relevant inherent
properties of the individual members, including perhaps a principle that the darker color precedes
the lighter one (\emph{schwarzweiß} `black and white', but ?\emph{weißschwarz}). A potential
phonological asymmetry pertains to syllable count such that the member with fewer syllables comes
first (\eg \emph{weiß-rosa} `white-pink' may be preferred to \emph{rosa-weiß}, see below).

%\largerpage
The claim that endocentricity blocks the classification of a compound as copulative in \ili{German} will be further examined with respect to the words in (\ref{ex-Strumpfliebe}). They all are composed of some sort of cohyponyms, therefore meeting a prerequisite for the classification as a copulative compound.
Those in (\ref{ex-Strumpfhose}) can be considered hyponyms of the concept \emph{garment}, those in
(\ref{ex-Fürstbischof}) could belong to a hypernym defined by a high ranking position, those in
(\ref{ex-Malerdichter}) are hyponyms of professions associated with art. The compounds in
(\ref{ex-taubstumm}) consist of adjective combinations, each associating with a common hypernym,
those in (\ref{ex-Hassliebe}) consist of antonyms, again associated with a common
hypernym.\footnote{In \ili{English}, too, the corresponding compounds are cited as examples for copulative
  compounds, and indeed several of them are end-stressed (\eg \emph{deafmúte}, \emph{painter-póet})
  \citep[61]{Olsen2000}. This is in accordance with the above-mentioned possibility that \ili{English}
  does allow for double-endocentric compounds, classified as copulative and therefore receiving
  final stress.}

\eal\label{ex-Strumpfliebe}
\ex\label{ex-Strumpfhose}
\noemph{Strúmpfhose} `pantyhose' (\noemph{Strumpf} `stocking' + \noemph{Hose} `pants') \\ 
\noemph{Mántelkleid} `coatdress' (\noemph{Mantel} `coat', \noemph{Kleid} `dress')

\ex\label{ex-Fürstbischof}
\noemph{Fǘrstbischof} `Prince Bishop' (\noemph{Fürst} `prince', \noemph{Bischof} `bishop') \\
\noemph{Prínzgemahl} `husband of a governing monarch' (\noemph{Prinz} `prince' + \noemph{Gemahl} `husband')

\ex\label{ex-Malerdichter}
\noemph{Málerdichter} `painter-poet' (\noemph{Maler} `painter' + \noemph{Dichter} `poet') \\
\noemph{Díchterkomponist} `poet-composer' (\noemph{Dichter} `poet' + \noemph{Komponist} `composer')

\ex\label{ex-taubstumm}
\noemph{táubstumm} `deafmute' (\noemph{taub} `deaf' + \noemph{stumm} `mute') \\
\noemph{dúmmdreist} `impudent' (\noemph{dumm} `dumb' + \noemph{dreist} `brash') \\
\noemph{láuwarm} `lukewarm' (\noemph{lau} `mild' + \noemph{warm} `warm') \\
\noemph{násskalt} `dank' (\noemph{nass} `wet' + \noemph{kalt} `cold') \\

\ex\label{ex-Hassliebe}
\noemph{Hássliebe} `love-hate relationship' (\noemph{Hass} `hate' + \noemph{Liebe} `love') \\
\noemph{Fréundfeind} `frenemy' (\noemph{Freund} `friend' + \noemph{Feind} `enemy')

\zl

\noindent
A consultation of a standard dictionary (\url{https://www.duden.de/}) confirms both the presence of
a hyponymy relation and main stress on the initial member. In most cases, an asymmetry to the effect
that the initial member functions as a modifier is supported as well, such as \emph{Mantelkleid}
'(\emph{Kleid}) `dress' tailored like a coat' (not a coat tailored like a dress), \emph{Fǘrstbischof} '(\emph{Bischof}) `bishop' with the rank of a prince' (not a prince who also works as a bishop),
\emph{Prinzgemahl} '\emph{Gemahl} `husband' of a reigning monarch' (not some sort of prince), \emph{taubstumm} '(\emph{stumm}) `mute' as a result of being born deaf'.\footnote{For discussion of these compounds and arguments against their classification as copulative, see also \citew{Becker92a}.}

It is true that several of the definitions given in (\ref{ex-Malerdichter}) and (\ref{ex-Hassliebe}) do indicate an interpretation characterized by `equal-footing' among the members. For instance, \emph{Málerdichter} is defined as someone who is both a painter and a poet (\url{https://www.duden.de/rechtschreibung/Malerdichter}). One may wonder if this definition is influenced by those of corresponding compounds in \ili{English}, which have final stress and possibly are truely copulative. A somewhat random search of relevant data in the internet does indicate an asymmetry, where \emph{Malerdichter} `painter-poet' differs from \emph{Dichtermaler} `poet-painter', depending on which role is considered more central with regard to the referent in question.\footnote{Compare the use of the compound \emph{Dichtermaler} ending in \emph{Maler} in reference to \emph{Oskar Kokoschka}, a well-known painter who also produced some literary work (\url{https://oe1.orf.at/artikel/216337/Wenn-Dichtermaler-malerdichten}) to that of the compound \emph{Malerdichter} ending in \emph{Dichter} referring to \emph{Max Jacob}, who in the French Wikipedia article is described as a `modernist poet and novelist but also a painter'  (\url{https://fr.wikipedia.org/wiki/Max_Jacob}).} 

\largerpage
My skepticism concerning the adeqacy of the truely `equal-footing' readings provided by
(\url{https://www.duden.de/}) also pertains to the cases of antonymy in (\ref{ex-Hassliebe}),
including the definition of \emph{Hassliebe} as `a strong emotional bond that alternates between
hate and love'. In my view the proposition that \emph{Hassliebe} is a kind of \emph{Liebe} `love'
seems far more adequate than the proposition that \emph{Hassliebe} is a kind of \emph{Hass}
`hatred'. Here, too, the dictionary definition in \ili{German} may have been influenced by the
corresponding expression in \ili{English}, \emph{love-hate relationship}, where the compound
\emph{love-hate} is embedded and therefore automatically interpreted on equal footing (see the
discussion below (\ref{ex-muttersohn})).

Turning now to the conditions for classifying a compound as copulative in \ili{German} it was argued that the case of \emph{blau-wéiß}
indicates the relevance of two prerequisites: failed hyponymy in the relation between the compound and its rightmost daughter (i.e.\ exocentricity) and shared properties among the members (in this case their status as cohyponyms of the concept \emph{color term}). Consider the additional data in (\ref{ex-rotkalt}), all of which exhibit stress on their final member and meet the requirement of shared structure between the respective members. 


\eal\label{ex-rotkalt}

\ex\label{ex-südwest}
\noemph{südwést} `southwest' (\noemph{süd} `south' + \noemph{west} `west') \\

\ex\label{ex-rotgelb}
\noemph{rot-gélb} `red-yellow' (coalition of the party represented by the color \noemph{rot} (social-democrats) and that represented by the color \noemph{gelb} (liberals))\\

\ex\label{ex-feuchtkalt}
\noemph{feuchtkált} `cold and humid' (\noemph{feucht} `humid' + \noemph{kalt} `cold') \\ 
\noemph{feuchtwárm} `warm and humid' (\noemph{feucht} `humid' + \noemph{warm} `warm') \\ 

\ex\label{ex-manisch}
\noemph{manisch-depressív} `manic-depressive' \\
\noemph{passiv-aggressív} `passive-aggressive' \\ 

\ex\label{ex-Marxismus}
\noemph{Marxismus-Leninísmus} `Marxism-Leninism' \\

\zl

\noindent
The failure of hyponymy in \emph{südwést} is evident as \emph{southwest} can hardly be considered a kind of \emph{west}. This assessment relates to the fact that \emph{west} is not a kind of direction, rather, \emph{west} is a member of a set of items constituting the inventory of direction terms. Such expressions, being defined by their place in some sort of nomenclature, appear to have a name-like status which insulates them from participation in hyponymic relations. 
 
 It is then the exocentric status of the compound \emph{südwést}, along with the fact that all sisters are on a par (i.e.\ both \emph{süd} and \emph{west} belong to the inventory of direction terms in question), which motivates the classification of the compound as copulative. Right-oriented head alignment results, manifest in final main stress (\emph{südwést}).\footnote{Note also that the order among the members in the respective compounds has nothing to do with potential modifier-head roles but is determined by inherent properties and relevant asymmetries (in this particular case the (arbitrary) convention that expressions pertaining to latitude (\emph{north}, \emph{south}) precede those pertaining to longitude (\emph{east}, \emph{west})).}

\largerpage
The relevant effect is also seen in (\ref{ex-rotgelb}), where the two color terms each refer to a \ili{German} political party and the compound refers to a coalition among the two parties. Here the color terms serve as names, which resist functioning as hypernyms.\footnote{This effect may also be at play in the analysis of the compound \emph{blauweiß} in (\ref{ex-blaugrau}). That is, the classification as a copulative compound might be due to the conception of \emph{blau} and \emph{weiß} as basic color terms belonging to a (more or less) fixed inventory, rather than being conceived as parts of a spectrum where they are subject to modification. On that view, the different morphological analyses of the compounds in (\ref{ex-blaugrau}) reflect differences in the conceptions of the color terms, presumably influenced by the nature of the color combinations in question.} 

The observation that final stress is somewhat less robust in (\ref{ex-feuchtkalt}) correlates with the less clear status of the relevant compound members (\eg \emph{feucht}, \emph{kalt}), which only tenuously qualify as part of a nomenclature of adjectives referring to climate.\footnote {The compound is transcribed with final stress in \url{https://www.duden.de/rechtschreibung/feuchtkalt}, but with initial stress only in \url{https://de.wiktionary.org/wiki/feuchtkalt}.} Final main stress appears to be more natural in the phrase \emph{feuchtkáltes Klima} `humid-cold climate' than in the phrases \emph{féuchtkalte Hand} `moist-cold hand' or \emph{násskaltes Wetter} `wet-cold weather', which do not evoke the sense of the compound members belonging to a fixed nomenclature. The observation that final stress in phrases like \emph{feuchtkáltes Klima} is still far less robust than final stress in for instance \emph{südwést} can then again be linked to the semantic relations in question: \emph{feuchtkalt} is more easily conceptualized as a hyponym of \emph{kalt} than \emph{südwést} being viewed as a hyponym of \emph{west}. 

Stress variation also characterizes the compounds in (\ref{ex-manisch}),\footnote{Cf.\ the transcription with final stress in \url{https://www.duden.de/suchen/dudenonline/manisch-depressiv} versus that with initial stress only in (\url{https://de.wiktionary.org/wiki/passiv-aggressiv}.} whose members belong to an inventory of psychological terms used to describe pathological states of mind but could also be viewed as regular adjectives. Here an additional factor motivating the classification of a compound as copulative may come into play, pertaining to formal similarities between the respective members. The occurrence of (Latinate) suffixes may be relevant there, especially that of identical suffixes as in (\ref{ex-Marxismus}). Another type of similarity linked to final stress is seen in (\ref{ex-süßsauer}), namely, alliteration, as the respective members start with the same phoneme. 


\ea\label{ex-süßsauer}
\noemph{süßsáuer} `sweet and sour' (\noemph{süß} `sweet' + \noemph{sauer} `sour') \\ 
\noemph{hübsch-hä́sslich} (\noemph{hübsch} `pretty' + \noemph{hässlich} `ugly')\\  
\noemph{feuchtfrö́hlich} (\noemph{feucht} `moist (referring to the consumption of alcohol' + \noemph{fröhlich} `cheerful') \\ 

\z

\largerpage[2] 
\noindent
The potential relevance of alliteration or other shared structural properties (\eg identical affixes) as a factor motivating the classification of compounds as copulative makes sense as ``sameness'' (``equal footing'') among members is the central characteristic of such compounds. Yet another phonological factor which appears to correlate with final stress concerns the ordering among compound members in accordance with syllable count (fewer before more syllables). This pattern is also natural in that the placement of longer compound members in the final position results in a congruence between (main) stress and (maximal) size.\footnote{The impact of the phonological factors on the classification of compounds is likely strongest when various factors align, such as alliteration and sequencing in accordance with syllable count in the case of \emph{süßsáuer} `sweet and sour'. Here final stress is rather robust, despite the fact that the compound might be considered some sort of hyponym of its rightmost member. When none of these factors apply, comparable compounds are prone to be pronounced with initial stress as in \emph{bíttersüß} `bittersweet', cf.\ \url{https://en.wiktionary.org/wiki/bitters\%C3\%BC\%C3\%9F}, which indicates the parsing of that word as a regular compound. Possible pronunciations with initial stress are also expected in cases such as \emph{feucht-fröhlich}, where relevant phonological factors align, but the compound members fail to share semantic properties
\url{https://de.wiktionary.org/wiki/feuchtfr\%C3\%B6hlich}.}

The conditions pertaining to structural similarities between compound members discussed here likely play a role in motivating the classification of a compound as copulative but differ substantially from the condition named earlier, namely the unequivocal status of compound members as proper nouns. This is because proper nouns do not participate in hyponymies , thereby firmly establishing the status of a compound as exocentric. Coupled with shared properties among the compound members (\eg all forenames in (\ref{ex-Marie-Luise}), all names of territories in (\ref{ex-Schleswig}), and all last names referring to individuals in (\ref{ex-Daimler})) such compounds 
are characterized by robust stress on their final member, regardless of their respective phonological forms:\footnote{Here, too, the sequence among the members is typically determined by syllable count, such that the shorter member comes first. The few counter-examples to this generalization found in a website listing fifty double-names  (\url{https://www.familie.de/schwangerschaft/vornamen/die-50-schoensten-doppelnamen-fuer-jungen-und-maedchen/}) appear to be borrowed already as double-names from other languages (\eg \emph{Mary Lou}). The only exception to this rule I know of concerns the names of \ili{German} companies as in (\ref{ex-Daimler}), where the sequence among the members, all referring to male company founders, possibly favors a masculine cadence.} 

\eal\label{ex-Marie-Klöckner}
\ex\label{ex-Marie-Luise}
\noemph{Marie-Luíse} (two female first names, used as double-name) \\ 
\noemph{Ann-Kathrín} (two female first names, used as double-name) \\ 
\noemph{Klaus-Díeter} (two male first names, used as double-name) \\
\ex\label{ex-Schleswig}
\noemph{Schleswig-Hólstein} (two names of territories associated with a single German state) \\ 
\noemph{Baden-Wǘrttemberg} (two names of territories associated with a single German state)\\
\noemph{Nordrhein-Westfálen} (two names of territories associated with a single German state)
\ex\label{ex-Daimler}
\noemph{Daimler-Bénz} (two last names of company founders referring to a single corporation) \\
\noemph{Magirus-Déutz} (two last names of company founders referring to a single corporation)

\noemph{Klöckner-Humboldt-Déutz} (three last names of company founders referring to a single corporation)
\\

\zl

\largerpage
\enlargethispage{3pt}
\noindent
The two compounds in (\ref{ex-Daimler-Dichter}), one copulative, the other regular, illustrate the isomorphic mapping of the relevant distinct morphological structures to the corresponding prosodic compounds, which form the basis for applying the respective left- versus right-oriented head alignment constraints.


\eal\label{ex-Daimler-Dichter}
\ex {}[[daimləʀ]\textbf{\sub{STM}}[bɛnt\textsuperscript{s}]\textbf{\sub{STM}}]\textbf{\sub{COPCOMP-M}}

\ex {}[[mɑləʀ]]\textbf{\sub{STM}}[dɪçtəʀ]\textbf{\sub{STM}}]\textbf{\sub{COMP-M}}
\z

\ex\label{ex-Daimler-Benz-P} 
\ea ((daimləʀ)\sub{ω}(bɛnt\textsuperscript{s})\sub{ωHd})\sub{COPCOMP-P}

\ex ((mɑləʀ)\sub{ωHd}(dɪçtəʀ)\sub{ω})\sub{COMP-P}
\z
\z
%\itdopt{S: Brauchen wir fett?}


\noindent
The most common type of copulative compound in \ili{German} involves expressions known as letter names, each of which associate with individual letters of the alphabet (\eg [ve] `W', [ɡe] `G'). Consider the words in the left-hand column in (\ref{ex-kürzung}), which can be decomposed exhaustively into stems corresponding to a letter name each (see the right-hand column in (\ref{ex-kürzung})). The regular classification of such words as copulative compounds in \ili{German} is manifest in systematic main stress on the word-final syllable. 


\settowidth\jamwidth{[[kɑ]\sub{STM}[pe]\sub{STM}[de]\sub{STM}[ɛs]\sub{STM}[u]\sub{STM}]\sub{COPCOMPM}\,}
\eal\label{ex-kürzung}
\ex\label{ex-wg}
{}[veˈɡe] $\langle$WG$\rangle$ \jambox{[[ve]\sub{STM}[ɡe]\sub{STM}]\sub{COPCOMP-M}}

\ex {}[ɛsɛmˈfau] $\langle$SMV$\rangle$ \jambox{[[ɛs]\sub{STM}[ɛm]\sub{STM}[fau]\sub{STM}]\sub{COPCOMP-M}}

\ex {}[øpeɛnˈfau] $\langle$ÖPNV$\rangle$ [[ø]\sub{STM}[pe]\sub{STM}[ɛn]\sub{STM}[fau]\sub{STM}]\sub{COPCOMP-M}

\ex\label{ex-kpdsu}
{}[kɑpedeɛsˈu]\jambox{[[kɑ]\sub{STM}[pe]\sub{STM}[de]\sub{STM}[ɛs]\sub{STM}[u]\sub{STM}]\sub{COPCOMP-M}}

$\langle$KPdSU$\rangle$

\ex\label{ex-abc}
{}[ɑbeˈt\textsuperscript{s}e] $\langle$ABC$\rangle$ \jambox{[[ɑ]\sub{STM}[be]\sub{STM}[t\textsuperscript{s}e]\sub{STM}]\sub{COPCOMP-M} }
\zl

\largerpage
\noindent
The cases in (\ref{ex-kürzung}a--d)
differ from that in (\ref{ex-abc}) regarding the morphosyntactic properties of the compound. 
The letters in the acronyms in (\ref{ex-kürzung}a--d) are determined by correspondence with the initials given in the respective source expressions, whose morphosyntactic head determines the gender of both the complete expression (cf.\ the lefthand column in (\ref{ex-komposita})) and of the corresponding letter compound (cf.\ the righthand column in (\ref{ex-komposita})). By contrast, the compound in (\ref{ex-abc}) is due to independent conventions pertaining to the listing of letter names, whose inherent gender (always neuter) determines the category of the letter compound (see (\ref{ex-abc-compound})).

\settowidth\jamwidth{[[ɑ]\sub{\textsc{n.neut}}[be]\sub{\textsc{n.neut}}[tse]}
\eal\label{ex-komposita}
\ex\label{ex-wohngemeinschaft}
\textbf{W}ohn\textbf{g}emein[schaft]\sub{\textsc{n.fem}}
\jambox{WG]\sub{\textsc{n.fem}} }

`shared apartment'

\ex
\textbf{S}chüler\#\textbf{m}it\textbf{v}erwalt[ung]\sub{\textsc{n.fem}}
\jambox{SMV]\sub{\textsc{n.fem}} }

`student representation'

\ex
\textbf{Ö}ffentlicher \textbf{P}ersonen\#\textbf{n}ah[\textbf{v}erkehr]\sub{\textsc{n.masc}}
\jambox{ÖPNV]\sub{\textsc{n.masc}} }

`public transportation'


\ex\label{ex-kpd}
\textbf{K}ommunistische \textbf{P}art[ei]\sub{\textsc{n.fem}}
\jambox{KPdSU]\sub{\textsc{n.fem}} }

\textbf{d}er \textbf{S}owjet\textbf{u}nion\\
`Communist party of the Soviet Union'

\ex\label{ex-abc-compound}
{}[ɑ]\sub{\textsc{n.neut}} [be]\sub{\textsc{n.neut}} [t\textsuperscript{s}e]\sub{\textsc{n.neut}}
\jambox{ABC]\sub{\textsc{n.neut}} }

\zl

\noindent
Differences pertaining to the creation of letter compounds illustrated in (\ref{ex-komposita}) will not affect the classification of morphological compounds proposed here, which is determined by particular parsing strategies based on complete words.\footnote{Letter name compounds illustrate the quintessential failure of hyponymy (a \emph{WG} is not some kind of \emph{G}) along with the clear equivalence among all compound members (all correspond to letter names).} Those strategies focus on decomposing the word into stems, establishing the status of the compound as endo- versus exocentric based on hyponymy, and on assessing similarities between the respective members. All of the words in the left-hand column in (\ref{ex-kürzung}) will be parsed in a parallel fashion, resulting in identical types of morphological structures, namely the copulative compounds shown in the righthand column in (\ref{ex-kürzung}), which are mapped to the corresponding prosodic domains in (\ref{ex-komposita-head}). Alignment of internal stem boundaries with phonological word boundaries results in separate syllabification domains, manifest in the association of stem final consonants with the syllable coda, even when a vowel follows, as in /ɛ\textbf{s.ɛ}m.fau/\footnote{This syllable structure is supported by the potential glottalization affecting both instances of the vowel /ɛ/, due to their respective positions in the initial position of a phonological word (i.e.\ ((\textbf{ɛ}s)\sub{ω}(\textbf{ɛ}m)\sub{ω}(fau)\sub{ω})\sub{COPCOMP-P}))}. 
%\itdopt{anpassen}
Right-oriented head alignment associated with copulative compounds captures the fact that main prominence always falls on the very last member, regardless of the respective total number of members.\footnote{Initial stress as in \emph{LKW} (< \emph{Lastkraftwagen} `truck') is due to (historical) prosodic fusion of the former compound ((ɛl)\textsubscript{ω}(kɑ)\textsubscript{ω}(ve)\textsubscript{ωHd})\textsubscript{COPCOMP-P} into a single phonological word (ˈɛlkɑ\textsecstress ve)\textsubscript{ω}. Here initial main stress reflects the previous rhythmic secondary stress (see \citet[102]{Raffelsiefen2021})}


\eal\label{ex-komposita-head}
\ex\label{ex-wohngemeinschaft-head}
((ve)\sub{ω}\textbf{(ɡe)\sub{ωHd}})\sub{COPCOMP-P}   		\jambox{WG}

\ex ((ɛs)\sub{ω}(ɛm)\sub{ω}\textbf{(fau)\sub{ωHd}})\sub{COPCOMP-P}  		\jambox{SMV}

\ex ((ø)\sub{ω}(pe)\sub{ω}(ɛn)\sub{ω}\textbf{(fau)\sub{ωHd}})\sub{COPCOMP-P}   		\jambox{ÖPNV}

\ex ((kɑ)\sub{ω}(pe)\sub{ω}(de)\sub{ω}(ɛs)\sub{ω}\textbf{(u)\sub{ωHd}})\sub{COPCOMP-P}  		\jambox{KPdSU}

\ex\label{ex-abc-head}
((ɑ)\sub{ω}(be)\sub{ω}\textbf{(t\textsuperscript{s}e)\sub{ωHd}})\sub{COPCOMP-P}  \jambox{ABC}

\zl

\largerpage
\noindent
The claim that the exhaustive parsability of a given word into `equivalent' strings along with
failed hyponymy sufficiently motivate the classification of a compound as copulative is demonstrated
further in (\ref{ex-tam-tam}). Here one condition for the classification of compounds as copulative,
namely that all members must be on equal footing, is evidently satisfied by the exhaustive
decomposability of the word into homophonous meaningless strings.\footnote{Meaninglessness follows
  from the fact that the relevant strings do not recur in other contexts.} The exocentricity
condition is satisfied as well, as meaningless strings necessarily fail to engage in hyponymy. Main
stress on the final member is rather robust in such cases, deviating strikingly from the unmarked
organization of disyllabic words as trochees in \ili{German}.\footnote{\label{fn-words-associated}Words
  associated with children's speech are special in that stress tends to vary, particularly when the
  word consists of two open syllables. Initial stressed syllables can appear ambisyllabically closed
  here, resulting in a lax stressed vowel no longer homophonous with the final vowel (\eg /piˈpi/,
  /ˈpipi/, or /ˈpɪpi/ \emph{Pipi} `pee', /poˈpo/, /ˈpopo/ or /ˈpɔpo/ \emph{Popo} `popo'). Other
  exceptions are loanwords, where the stress is adopted from that given in the donor language (\eg
  /ˈɡaɡɑ/ \emph{gaga} from \ili{English} /ˈɡɑɡɑ/ \emph{gaga}). The persistence of stress in borrowings
  also follows from the Elsewhere Condition, as already specified foot structure takes precedence
  over respective structures created from scratch.}  

\eal\label{ex-tam-tam}

\ex \noemph{Tamtám}\\
    `ballyhoo'

\ex \noemph{Töfftö́ff}\\
    `motorized vehicle (children's speech)'

\ex \noemph{plemplém}\\
    `cuckoo'

\ex \noemph{ballabálla}\\
    `cuckoo'

\zl

\noindent
The claim that the analysis as a copulative compound is determined not only by stem homophony but also by failed hyponymy rests on the  cases of so called REAL-X reduplication in (\ref{ex-freund-freund}), a construction characterized by a simple repetition of content words.\footnote{The construction appears to have been borrowed into \ili{German}, where it has become somewhat productive (cf.\ \citealp{Freywald2015}).}



\eal\label{ex-freund-freund}

\ex \noemph{Fréund-freund} \\
`friend-friend' (a real friend, not a Facebook friend) \\

\ex \noemph{jétzt-jetzt} \\
`now-now' (really now, not earlier or later) \\

\ex \noemph{híer-hier} \\
`here-here' (really here, not just close to here) \\

\ex \noemph{Búch-buch} \\
`book-book' (a real paper book, not an e-book) \\


\zl

\largerpage
\noindent
Referring to similar data in \ili{English} and \ili{Spanish}, \citet[48]{Horn1993} notes that reduplication as in (\ref{ex-freund-freund}) induces a modifying reading, narrowing the set of potential referents to those representing the ``real'' cases. The relation is thus clearly hyponymic (\eg a \emph{Freundfreund} is in fact a prototypical kind of \emph{Freund}), which rules out the categorization of the compound as copulative. The classification as a regular compound by default results. The analysis of the relevant two cases is illustrated in (\ref{ex-freund-head}), where the isomorphic mapping along with the distinct head alignment constraints yield the distinct stress patterns:

\eal\label{ex-freund-head}

\ex\label{ex-tamtam-comp}
[[tam]\sub{STM}[tam]\sub{STM}]\textbf{\sub{COPCOMP-M}} $\rightarrow$ ((tam)\sub{ω}(tam)\sub{ωHd})\textbf{\sub{COPCOMP-P}} 

\ex\label{ex-freundfreund-comp}
[[fʀɔind]\sub{STM}[fʀɔind]\sub{STM}]\textbf{\sub{COMP-M}} $\rightarrow$
((fʀɔind)\textbf{\textsubscript{ωHd}}(fʀɔind)\textsubscript{ω})\sub{COMP-P}
\zl

\noindent
The assumption that the exhaustive decomposability of a given word into identical meaningless
strings is sufficient to motivate a status as a copulative compound in \ili{German} is supported by the
stress patterns of certain shortened forms. The examples in (\ref{ex-shortening}) illustrate a
productive rule of abbreviating morphologically complex expressions by retaining only the initial
string of given stems, up to and including the first syllable nucleus (so-called
\emph{Silbenkurzwörter}). In abbreviations consisting of three open syllables main stress falls
regularly on the penult as in (\ref{ex-hanuta}), under certain phonological conditions also on the
initial syllable as in (\ref{ex-haribo}). The single case where this type of shortening exhibits
final main stress is shown in (\ref{ex-rororo}):  

\eal\label{ex-shortening}
\ex\label{ex-hanuta}
\textbf{Ha}sel\#\textbf{nu}ss\#\textbf{ta}fel $\rightarrow$
[hɑˈnutɑ] $\langle$Hanuta$\rangle$\\
`hazelnut bar' (™ candy)

\ex\label{ex-haribo}
\textbf{Ha}ns \textbf{Ri}egel, \textbf{Bo}nn
$\rightarrow$ [ˈhɑʀiˌbo] $\langle$Haribo$\rangle$\\
`(name of individual), (city name)' (™ candy)

\ex\label{ex-rororo}
\textbf{Ro}wohlt \textbf{Ro}tations\#\textbf{ro}mane $\rightarrow$
[ʀoʀoˈʀo] $\langle$rororo$\rangle$\\
`(name of individual) rotation novels' (™ publishing company)
\zl

\largerpage[2]
\noindent
The stress patterns shown in (\mex{0}a,b) indicate that the entire (trisyllabic) shortening forms a single domain for syllabification and foot formation. The final two syllables are organized as a trochaic foot, unless the string in question violates certain markedness constraints, in which case the first two syllables form a foot (see \citet[91--92]{Raffelsiefen2021}).  Head alignment then picks out the rightmost binary foot. The only potential source for main stress on a final syllable is the analysis of the word as a copulative compound where all syllables, including the last, form separate phonological words as is shown in (\ref{ex-rororo-stm}):%
%
\footnote{In the regular vocabulary there are other sources for main stress on the last syllable, including borrowings (\eg \ili{French} [ʀokoˈko] \emph{Rokoko} `rococo') or stress-attracting suffixes. The absence of any such influences in the shortening data makes them so valuable for studying stress patterns (cf.\ \citealp{Raffelsiefen2021} for detailed discussion).}

\settowidth\jamwidth{((ʀo)\sub{ω}(ʀo)\sub{ω}(ʀo)\sub{\textbf{ωHd}})\sub{COPCOMP}}
\eal\label{ex-shortening-stm}

\ex\label{ex-hanuta-stm}
{[[hɑnutɑ]\sub{STM}]\sub{W}}
\jambox{(hɑ(nutɑ)\sub{ΣHd})\sub{ω}}

\ex\label{ex-haribo-stm}
{[[hɑʀibo]\sub{STM}]\sub{W}}
\jambox{((hɑʀi)\sub{ΣHd}(bo)\sub{Σ})\sub{ω}}

\ex\label{ex-rororo-stm}
{[[ʀo]\sub{STM}[ʀo]\sub{STM}[ʀo]\sub{STM}]\sub{COPCOMP-M}}
\jambox{((ʀo)\sub{ω}(ʀo)\sub{ω}(ʀo)\sub{ωHd})\sub{COPCOMP}}
\zl

\noindent
The patterns in (\ref{ex-shortening-stm}) strongly motivate a top-down parsing mechanism, where given words are decomposed into stems, based on their complete form. They moreover support the Elsewhere Principle, as the applicability of the special conditions motivating the classification as a copulative compound takes precedence over a simple parsing of the word as single stem. At the same time the principle accounts for an asymmetry concerning possible variation. A pronunciation of \emph{rororo} with non-final main stress is conceivable but final stress in words like \emph{Hanuta}, \emph{Haribo} is not. This is because the presence of non-final stress in \emph{rororo} would simply indicate that the (human) parser has failed to notice the sameness of the relevant substrings, resulting in an analysis of the word as a single stem and consequently a single phonological word. Final stress on the words in (\ref{ex-shortening}a,b) is ruled out as these words do not lend themselves to a morphological parsing as anything other than a single stem.

\largerpage[2]
The observation that the exhaustive parsability of a given word into a sequence of identical (meaningless) strings motivates the morphological classification as a copulative compound raises the question of what exact conditions qualify as ``sameness''. The answer can be found in correlations between stress regularities and patterns of partial phonological sameness. The data in (\ref{ex-remmidemmi})–(\ref{ex-tingeltangel}), consisting mostly of  obscure parts lacking correspondents among independent words, indicate
systematic stress differences depending on which aspects of phonological vary. Words that can be
exhaustively decomposed into rhyming constituents, characterized by variance of the respective
initial onsets only, have main stress on the final constituent (cf.\ the words consisting of
disyllabic rhyming constituents in (\ref{ex-remmidemmi}) and those with monosyllabic rhyming
constituents in (\ref{ex-klimbim})\footnote{The word \emph{Héckmeck} is an exception.}).  

\begin{multicols}{2}
\eal\label{ex-remmidemmi}
\ex {}[ˌʀ\textbf{ɛmi}ˈd\textbf{ɛmi}]\\
$\langle$Remmidemmi$\rangle$
`racket'

\ex {}[ˌk\textbf{ʊdəl}ˈm\textbf{ʊdəl}]\\
$\langle$Kuddelmuddel$\rangle$
`jumble'

\ex {}[ˌʃ\textbf{ɪki}ˈm\textbf{ɪki}]\\
$\langle$Schickimicki$\rangle$
`in-crowd'
%
\columnbreak
%
\ex {}[ˌl\textbf{ɑʀi}ˈf\textbf{ɑʀi}]\\
$\langle$Larifari$\rangle$
`airy-fairy'

\ex {}[ˌt\textbf{ɛçtəl}ˈm\textbf{ɛçtəl}]\\
$\langle$Techtelmechtel$\rangle$
`affair'

\ex {}[ˌʀ\textbf{ambɑ}ˈt\textsuperscript{s}\textbf{ambɑ}]\\
$\langle$Rambazamba$\rangle$
`uproar'
\zl
\end{multicols}
%

\ea\label{ex-klimbim}
\begin{tabular}[t]{@{~}l@{~~}l@{\hspace{2.4cm}}l@{~~}l@{}}
a. & {}[ˌkl\textbf{ɪm}ˈb\textbf{ɪm}] & d. &  {}[ˌʀ\textbf{ʊk}ˈt\textsuperscript{s}\textbf{ʊk}]\\
   & $\langle$Klimbim$\rangle$ `junk'     &    & $\langle$ruckzuck$\rangle$ `fast'\\[2pt]
%
b. & {}[ˌʀ\textbf{ʊms}ˈb\textbf{ʊms}] & e. & {}[ˌtʀ\textbf{ɑ}ˈʀ\textbf{ɑ}]\\
   & $\langle$rumsbums$\rangle$ `abruptly' &    & $\langle$Trara$\rangle$ `ballyhoo'\\[2pt]
%
c. & {}[ˌʀ\textbf{at\textsuperscript{s}}ˈf\textbf{at\textsuperscript{s}}] & f. & {}[ˌh\textbf{ʊʃ}ˈp\textsuperscript{f}\textbf{ʊʃ}]\\
   & $\langle$ratzfatz$\rangle$ `fast'                                         &    & $\langle$huschpfusch$\rangle$ `disorderly'\\
\end{tabular}
\z
% \begin{multicols}{2}
% \eal\label{ex-klimbim}
% \ex {}[ˌkl\textbf{ɪm}ˈb\textbf{ɪm}]\\
% $\langle$Klimbim$\rangle$
% `junk'

% \ex {}[ˌʀ\textbf{ʊms}ˈb\textbf{ʊms}]\\
% $\langle$rumsbums$\rangle$
% `abruptly'

% \ex {}[ˌʀ\textbf{at\textsuperscript{s}}ˈf\textbf{at\textsuperscript{s}}]\\
% $\langle$ratzfatz$\rangle$
% `fast'
% %
% \columnbreak
% %
% \ex {}[ˌʀ\textbf{ʊk}ˈt\textsuperscript{s}\textbf{ʊk}]\\
% $\langle$ruckzuck$\rangle$
% `fast'

% \ex {}[ˌtʀ\textbf{ɑ}ˈʀ\textbf{ɑ}]\\
% $\langle$Trara$\rangle$
% `ballyhoo'

% \ex {}[ˌh\textbf{ʊʃ}ˈp\textsuperscript{f}\textbf{ʊʃ}]\\
% $\langle$huschpfusch$\rangle$
% `disorderly'

% \zl
% \end{multicols}


\noindent
By contrast, words which can be decomposed into constituents that are identical except for the stressed vowel have main stress on the initial constituent (cf.\ (\ref{ex-tingeltangel})).

\ea\label{ex-tingeltangel}
\begin{tabular}[t]{@{}l@{~}ll@{~}l@{}}
a. & [ˈtɪ\textbf{ŋəl}ˌta\textbf{ŋəl}]            & d. & {}[ˈ\textbf{v}ɪ\textbf{ʀ}ˌ\textbf{v}a\textbf{ʀ}]\\
   & $\langle$Tingeltangel$\rangle$ `honky-tonk' &    & $\langle$Wirrwarr$\rangle$ `clutter'\\[3pt]
b. & [ˈ\textbf{kʀ}ɪ\textbf{kəl}ˌ\textbf{kʀ}ɑ\textbf{kəl}] & e. & [ˈ\textbf{ʃn}ɪ\textbf{k}ˌ\textbf{ʃn}a\textbf{k}]\\
   & $\langle$Krickelkrakel$\rangle$ `illegible writing'  &    & $\langle$Schnickschnack$\rangle$ `nicknack'\\[3pt]
c. & [ˈ\textbf{kʀ}ɪ\textbf{ms}ˌ\textbf{kʀ}a\textbf{ms}]   & f. & [ˈ\textbf{t\textsuperscript{s}s}ɪ\textbf{k}ˌ\textbf{t\textsuperscript{s}}a\textbf{k}]\\
   & $\langle$Krimskrams$\rangle$ `hodgepodge'            &    & $\langle$zickzack$\rangle$ `zigzag'\\
\end{tabular}
\z

\noindent
These generalizations indicate that the exhaustive decomposability of a word into rhyming constituents (i.e.\ constituents which are the same, except for the word-initial onset) satisfies the condition for the classification as a copulative compound (cf.\ (\ref{ex-remmidemmi-klimbim-comp}a,b)).
The stress on the final constituent then follows from the relevant head alignment constraint. By contrast, the exhaustive decomposability into constituents which are the same except for the nucleus fails to satisfy the requirements for copulative compounds. Separate stems are still recognized, which form a regular compound by default (cf.\ (\ref{ex-tingeltangel})). Such compounds are consequently leftheaded, manifest in main stress on the initial member:\footnote{While some of the positions of main stress in (\ref{ex-remmidemmi})–(\ref{ex-tingeltangel}) also conform to the stress patterns seen in single phonological words, the assumption of single prosodic domains would be inconsistent with most of the data in (\ref{ex-remmidemmi})–(\ref{ex-tingeltangel}). In (\ref{ex-tingeltangel}) a single domain would cause main stress to fall on the penult in the four-syllable words, not the initial syllable. Final main stress in (\ref{ex-klimbim}) would be unexpected as single phonological words consisting of two syllables regularly form trochees. Moreover, several words exhibit intervocalic clusters which are not found in single phonological words but rather indicate a compound structure (\eg \emph{Krimskrams}, \emph{Schnickschnack}).}

\eal\label{ex-remmidemmi-klimbim-comp}
\ex\label{ex-remmidemmi-comp}
{}[[ʀɛmi]\sub{STM}[dɛmi]\sub{STM}]\sub{COPCOMP-M} $\rightarrow$
((ʀɛmi)\sub{ω}(dɛmi)\sub{ωHd})\sub{COPCOMP-P}\\
$\langle$Remmidemi$\rangle$

\ex\label{ex-klimbim-comp}
{}[[klɪm]\sub{STM}[bɪm]\sub{STM}]\sub{COPCOMP-M} $\rightarrow$
((klɪm)\sub{ω}(bɪm)\sub{ωHd})\sub{COPCOMP-P}\\
$\langle$Klimbim$\rangle$

\ex\label{ex-tingeltangel-comp}
{}[[tɪŋəl]\sub{STM}[taŋəl]\sub{STM}]\sub{COMP-M} $\rightarrow$
((tɪŋəl)\sub{ωHd}(taŋəl)\sub{ω})\sub{COMP-P}\\
$\langle$Tingeltangel$\rangle$

\zl

\noindent
To summarize, stress patterns indicate rather narrow conditions defining the class of copulative
compounds in \ili{German}. The requirement that meaning relations between the compound and each of its
respective members must be the same, along with the disallowance of double-endocentric compounds,
results in the large ratio of cases characterized by the necessary failure of hyponymy, including
those where members are meaningless (\emph{Tamtám, Klimbím}) or consist of names
(\emph{Schleswig-Hólstein}) or name-like words (nomenclatures, \eg \emph{südwést}). All of these
compounds are exocentric. Words which consist of combinations of similar stems but also lend
themselves to an analysis as endocentric compound exhibit stress variance or indeed initial stress
only in \ili{German} (\eg \emph{Mántelkleid}, \emph{Málerdichter}, \emph{táubstumm}). The correlations
between semantics and stress motivate the recognition of separate compound classes here, where
copulative compounds in \ili{German} are necessarily exocentric. 
 
 In contrast to the rather confined conditions restricting copulative compounds considered so far,
 there is one context where robust main stress on the final member correlates with a wider range of
 cases, namely when a compound is embedded as an initial member in another compound. Here compound
 members typically share semantic similarities (\eg kinship terms, body parts, antonyms) but often
 correspond to regular content words. Examples are given in (\ref{ex-muttersohn}).  
 


\eal\label{ex-muttersohn}
\ex {[\textbf{[}[mʊtəʀ]\sub{STM}[zon]\sub{STM}\textbf{]\sub{COPCOMP-M}}[kɔnflɪkt]\sub{STM}]\sub{COMP-M}}\\
Mutter-Sohn-Konflikt\\
`mother-son-conflict'

\ex {[\textbf{[}[fʀɔind]\sub{STM}[faind]\sub{STM}\textbf{]\sub{COPCOMP-M}}[ʃema]\sub{STM}]\sub{COMP-M}}\\
Freund-Feind-Schema\\
`friend-foe scheme'

\ex {[\textbf{[}[hɛʀt\textsuperscript{s}]\sub{STM}[lʊŋən]\sub{STM}\textbf{]\sub{COPCOMP-M}}[mɑʃinə]\sub{STM}]\sub{COMP-M}}\\
Herz-Lungen-Maschine\\
`heart-lung-machine'

\ex {[\textbf{[}[kɔstən]\sub{STM}[nʊt\textsuperscript{s}ən]\sub{STM}\textbf{]\sub{COPCOMP-M}}[ɑnɑlyzə]\sub{STM}]\sub{COMP-M}}\\
Kosten-Nutzen-Analyse\\
`cost-benefit-analysis'

\zl

\begin{sloppypar}
\noindent
Why is it possible to have a copulative compound \emph{Mutter-Sohn} `mother-son' with both members
on equal footing and robust final stress embedded in \emph{Mutter-Sohn-Konflikt} in
(\ref{ex-muttersohn}), while the similar free-standing combination \emph{Múttersöhnchen}
`mother\#son+diminutive' `Momma's boy' can only be analyzed as a regular compound with initial main
stress? The latter case is easily explained by the endocentricity of the compound:
\emph{Múttersöhnchen} is a hyponym of \emph{Söhnchen}, where the preceding constituent functions as
a modifier (i.e.\ `a son who is spoiled by his mother'). This analysis is not available for the
embedded compound because the relevant mother constituent (i.e.\ \emph{Mutter-Sohn}) does not
associate with a separate concept and therefore does not allow for a hyponymy relation to be
established. The lack of a separate concept relates to the fact that the embedded compound does not
refer independently, only free-standing words do. Here, too, it is the exocentric status of the
(embedded) compound which is crucial to its classification as copulative, manifest in main stress on
the final member (i.e.\ \emph{Mutter-Sóhn} `mother-son').
\end{sloppypar}

The effect in question is also seen in the comparison between the free-standing compound \emph{Fréundfeind} and cases where \emph{Freund-Feind} is embedded as in \emph{Freund-Féind-Schema}. The free-standing word associates with a concept describing particular individuals (say, a foe who sometimes acts like a friend), arguably a regular endocentric compound with main stress on the initial member. By contrast, the embedded constituent
does not associate with a particular concept but only the entire compound does (it denotes a view of all people to fall into two classes, friend or foe). Again, the absence of a concept associated with the embedded compound motivates an exocentric analysis, manifest in main stress on the final member \emph{Freund-Féind}.\footnote{It is of course also possible to embed an existing endocentric compound such as \emph{Fréund-Feind}, in which case main stress will fall on the initial syllable in the complex compound (\eg \emph{Fréund-Feind-Konflikt}, meaning a conflict which one has with a frenemy.}

The claim that the robustness of stress on the second member in the compounds in
(\ref{ex-muttersohn}) is due to the absence of a separate concept associated with the respective
immediate mother constituent makes sense of the overall differences in productivity among embedded
and free-standing copulative compounds. Generally speaking, endocentricity appears to be a
prerequisite for the productive formation of new compounds based on content words in \ili{German} or
\ili{English}. This is presumably due to the role of endocentricity in the creation and learnability of
concepts. Given access to the vast inventory of content words there is no difficulty in finding
combinations associated with sensible concepts such as \emph{Staublunge} `dust+lung', meaning `lung
disease caused by the inhalation of dust'. The picture changes drastically when the relevant
inventory is confined to words associating with a particular hypernym, say \emph{organs}. What
concept, applicable to an entity in the world, is expressed by combining terms for organs such as
\emph{Herzlunge} `heart lung', \emph{Leberniere } `liver kidney'? This problem does not arise for
embedded compounds as they do not need associate with a separate concept. Complex words with
embedded compounds such as \emph{heart-lung-machine} are then easily formed and understood (i.e.\ `a
machine which involves the heart and the lung'). 

The few contexts where free-standing copulative compounds are reasonably productive are likewise
explained by conditions relating to the creation and understanding of concepts. The most productive
type in \ili{German}, namely letter compounds, are characterized by their ability to simply inherit the
concept associated with the full form (\eg the compound \emph{WG} associates with the same concept
as the full form \emph{Wohngemeinschaft} `shared apartment'). Other types are marked by conventions,
such as names of adjacent territories to designate the respective combined area
(\emph{Schleswig-Holstein}). The latter convention does not apply to names of rivers, which, even
when flowing close to each other, are not conceived of as single entities. Copulative compounds
consisting of river names therefore occur only as embedded compounds (\eg \emph{Kocher-Jágst-Radweg}
`Kocher-Jagst-bike path', \emph{Oder-Néiße-Grenze} `Oder-Neiße-border').

Turning now to the question of parsing and consequent morphological classification we find that the stress patterns support an initial scan of the complete word for conformity with restrictions on copulative compounds, in accordance with the Elsewhere Principle. Words not amenable to being decomposed exhaustively into stems which are ``on equal footing'' are subject to a subsequent scan. Here parsing aims at recognizing contiguous substrings conforming with restrictions on copulative compounds. Examples for the resulting morphological structures are given in (\ref{ex-ro-ro-all}).\footnote{The  constituent \emph{Ro-ro} of the compound \emph{Ro-ró-Schiff} is also a ``Silbenkurzwort'' accidentally consisting of homophonous syllables (cf.\ the case of \emph{Rororo} discussed in (\ref{ex-rororo-stm})). Initial main stress in the compound \emph{Gó-go-Girl} is due to the fact that this word has been borrowed with initial stress, cf.\ footnote~\ref{fn-words-associated}.}


\eal\label{ex-ro-ro-all}
\ex\label{ex-ro-ro}
 {[\textbf{[}[ʀo]\sub{STM}[ʀo]\sub{STM}\textbf{]\sub{COPCOMP-M}}[ʃɪf]\sub{STM}]\sub{COMP-M}}\\
Ro-ro-Schiff (\textbf{Ro}ll-on-\textbf{ro}ll-off-Schiff)\\
`roll-on-roll-off-ship'

\ex {[\textbf{[}[vɪn]\sub{STM}[vɪn]\sub{STM}\textbf{]\sub{COPCOMP-M}}[zituɑt\textsuperscript{s}ion]\sub{STM}]\sub{COMP-M}}\\
Win-win-Situation\\
`win-win-situation'

\ex {[\textbf{[}[petəʀ]\sub{STM}[paul]\sub{STM}\textbf{]\sub{COPCOMP-M}}[kɪʀçə]\sub{STM}]\sub{COMP-M}}\\
Peter-Paul-Kirche\\
`Peter-Paul-church'

\ex {[\textbf{[}[ɑ]\sub{STM}[be]\sub{STM}[t\textsuperscript{s}e]\textbf{\sub{STM}}\textbf{]}\sub{COPCOMP-M}[ʃʏt\textsuperscript{s}ə]\sub{STM}]\sub{COMP-M}}\\
ABC-Schütze\\
`abecedarian'

\ex\label{ex-hno} {[\textbf{[}[hɑ]\sub{STM}[ɛn]\sub{STM}[o]\sub{STM}\textbf{]\sub{COPCOMP-M}}[ɑʀt\textsuperscript{s}t]\sub{STM}]\sub{COMP-M}}\\
HNO-Arzt (\textbf{H}als-\textbf{N}asen-\textbf{O}hren-Arzt)\\
`ear, nose and throat doctor'

\zl

\noindent
Mapping the morphological structures in (\ref{ex-muttersohn}) and (\ref{ex-ro-ro-all}) to prosodic
structures will result in embedded copulative compounds as shown in Figure~\ref{ex-hnobäume}. 
\begin{figure}
\hfill\hfill
\begin{subfigure}{.49\textwidth}
\centering
\scalebox{.95}{%
\begin{forest}%sm edges
	for tree={s sep=4ex}
[COMP-M
	[COPCOMP-M
		[{[ʀo]\rlapsub{STM}}]
		[{[ʀo]\rlapsub{STM}},tier=word]
	]
	[,empty nodes
		[{[ʃɪf]\rlapsub{STM}},tier=word]
	]
]
\end{forest}
}
%\caption{}
\end{subfigure}%
\hfill
\begin{subfigure}{.49\textwidth}
\centering
\scalebox{.95}{%
\begin{forest}%sm edges
[COMP-P, s sep=4ex
	[COMP-P\sub{Hd}, s sep=2ex
		[(ʀo)\rlapsub{ω}]
		[\textbf{(ʀo)\rlapsub{ωHd}},tier=word]
	]
	[,empty nodes
		[(ʃɪf)\rlapsub{ω},tier=word]
	]
]
\end{forest}
}
%\caption{}
\end{subfigure}
\hfill\mbox{}

\vspace{10pt}
\hfill\hfill
\begin{subfigure}{.49\textwidth}
\scalebox{.95}{%
\begin{forest}%sm edges
	for tree={s sep=4ex}
[COMP-M
	[COPCOMP-M
		[{[hɑ]\rlapsub{STM}}]
		[{[ɛn]\rlapsub{STM}}]
		[{[o]\rlapsub{STM}},tier=word]
	]
	[,empty nodes
		[{[ɑʀt\textsuperscript{s}t]},tier=word]
	]
]
\end{forest}
}
%\caption{}
\end{subfigure}%
%
\hfill
\begin{subfigure}{.49\textwidth}
\centering
\scalebox{.95}{%
\begin{forest}%sm edges
[COMP-P, s sep=4ex
	[COMP-P\rlapsub{Hd},s sep=2ex
		[(hɑ)\rlapsub{ω}]
		[(ɛn)\rlapsub{ω}]
		[\textbf{(o)\rlapsub{ωHd}},tier=word]
	]
	[,empty nodes
		[(ɑʀt\textsuperscript{s}t)\rlapsub{ω},tier=word]
	]
]
\end{forest}
}
%\caption{}
\end{subfigure}
\hfill\mbox{}
\caption{Isomorphic mapping and prosodic trees for (\ref{ex-ro-ro}) and (\ref{ex-hno})}
\label{ex-hnobäume}
\end{figure}
The mapping of morphological compounds to prosodic constituents is strictly isomorphic. Prominence patterns are determined by left- or right-oriented head alignment. The terminal unit forming a head constituent itself and also dominated exclusively by head constituents, boldfaced in Figure~\ref{ex-hnobäume}, will emerge as most prominent in the entire construction (cf.\ the notion of \emph{designated terminal element} discussed in Section~\ref{sec-wordprosody} above).\footnote{While these structures are not affected by regular rhythmic reversal (cf.\ footnote~\ref{fn-isomorphy}), the main stress can appear ``shifted'' under certain complex conditions, including the avoidance of a stress clash (two adjacent head syllables) in combination with discourse-related properties (introduction of new information). For instance, the main stress on the constituent \emph{Rad} in \emph{Kocher-Jagst-Radweg} in the pronunciation observed in \url{https://www.youtube.com/watch?v=ZgnPua2aKu8}, at 2:42 in the video, is conditioned by both the presence of two head syllables next to each other (i.e.\ \emph{Jagst-Rad}) and by the fact that prior to the first mention of the compound, the two rivers were referred to repeatedly. In more neutral contexts, the main stress is on the final member of the copulative compound appearing as the initial constituent of the complex compound
(cf.\ also \emph{Oder-Sprée-Radweg} referring to the two rivers \emph{Oder} and \emph{Spree}).} 


\largerpage[-1]
All cases of complex compounds considered so far are regular compounds containing a copulative compound as the initial member. Since that initial member forms the prosodic head of the higher compound, main stress always falls on its last member (see the prosodic trees in Figure~\ref{ex-hnobäume}). The examples in (\ref{ex-kocher-jagst})--(\ref{ex-alt-jung}) illustrate additional complex compound types, where the morphological structure is determined by the conditions on the recognition of copulative compounds outlined above. Each of these cases contrasts the satisfaction of relevant conditions motivating the recognition of a copulative compound in (a) with non-satisfaction in (b), where the initial compound is classified as a regular compound instead. 


\eal\label{ex-kocher-jagst}
\ex {[\textbf{[}[kɔçəʀ]\sub{STM}[iakst]\sub{STM}\textbf{]\sub{COPCOMP-M}}[[ʀɑd]\sub{STM}[veɡ]\sub{STM}]]\sub{COMP-M}}\\
Kocher-Jagst-Radweg\\
`(river name + river name)-bicycle path'

\ex {[\textbf{[}[obəʀ]\sub{STM}[ʀain]\sub{STM}\textbf{]\sub{COMP-M}}[[ʀɑd]\sub{STM}[veɡ]\sub{STM}]\sub{COMP-M}]\sub{COMP-M}}\\
Oberrhein-Radweg\\
`(upper + river name)-bicycle path'

\zl

\eal\label{ex-arm-reich}
\ex {[\textbf{[}[aʀm]\sub{STM}[ʀaiç]\sub{STM}]\textbf{\sub{COPCOMP-M}}[[ɡə]\sub{PRFX}[fɛlə]\sub{STM}]\sub{STM}]\sub{COMP-M}}\\
Arm-reich-Gefälle\\
`gap between rich and poor'

\ex {[\textbf{[}[nɔi]\sub{STM}[ʀaiç]\sub{STM}]\textbf{\sub{COMP-M}}[[ɡə]\sub{PRFX}[tuə]\sub{STM}]\sub{STM}]\sub{COMP-M}}\\
Neureich-Getue\\
`nouveau riche posturing'

\zl

\eal\label{ex-alt-jung}
\ex {[\textbf{[}[alt]\sub{STM}[iʊŋ]\sub{STM}\textbf{]\sub{COPCOMP-M}}\textbf{[}[ve]\sub{STM}[ɡe]\sub{STM}\textbf{]\sub{COPCOMP-M}}]\sub{COMP-M}}\\
Alt-jung-WG\\
`old-young shared apartment'\footnotemark

\ex\label{ex-uralt} 
 {[\textbf{[}[uʀ]\sub{STM}[alt]\sub{STM}\textbf{]\sub{COMP-M}}[[ve]\sub{STM}[ɡe]\sub{STM}\textbf{]\sub{COPCOMP-M}}]\sub{COMP-M}}\\
Uralt-WG\\
`very old shared apartment'

\footnotetext{The expression \emph{Alt-jung-WG} refers to a shared apartment with old and young inhabitants. The word \emph{Uralt-WG} means a shared apartment where the inhabitants have lived together for a very long time.}

\zl
\noindent
The relevant morphological and prosodic structures associated with the
examples in (\mex{0}) are shown in Figure~\ref{fig-compbäume}.

\begin{figure}
\begin{subfigure}{.49\textwidth}
\centering
\scalebox{.95}{%
\begin{forest}%sm edges
	for tree={s sep=4ex}
[COMP-M
	[COPCOMP-M
		[{[alt]\rlapsub{STM}}]
		[{[iʊŋ]\rlapsub{STM}}]
	]
	[COPCOMP-M
		[{[ve]\rlapsub{STM}}]
		[{[ɡe]\rlapsub{STM}}]
	]
]
\end{forest}
}
%\caption{}
\end{subfigure}%
%
\begin{subfigure}{.49\textwidth}
\centering
\scalebox{.95}{%
\begin{forest}%sm edges
	for tree={s sep=2ex}
[COMP-P,s sep=3ex
	[COPCOMP-P\rlapsub{Hd}
		[(alt)\rlapsub{ω}]
		[\textbf{(iʊŋ)\rlapsub{ωHd}}]
	]
	[COPCOMP-P
		[(ve)\rlapsub{ω}]
		[(ɡe)\rlapsub{ωHd}]
	]
]
\end{forest}
}
%\caption{}
\end{subfigure}

\vspace{10pt}
\begin{subfigure}{.49\textwidth}
\centering
\scalebox{.95}{%
\begin{forest}%sm edges
	for tree={s sep=4ex}
[COMP-M
	[COMP-M
		[{[uʀ]\rlapsub{STM}}]
		[{[alt]\rlapsub{STM}}]
	]
	[COPCOMP-M
		[{[ve]\rlapsub{STM}}]
		[{[ɡe]\rlapsub{STM}}]
	]
]
\end{forest}
}
%\caption{}
\end{subfigure}%
%
\begin{subfigure}{.49\textwidth}
\centering
\scalebox{.95}{%
\begin{forest}%sm edges
%	for tree={s sep=3ex}
[COMP-P, s sep=2ex
	[COMP-P\sub{Hd}, s sep=4ex
		[\textbf{(uʀ)\rlapsub{ωHd}}]
		[(alt)\rlapsub{ω}]
	]
	[COPCOMP-P, s sep=2ex
		[(ve)\rlapsub{ω}]
		[(ɡe)\rlapsub{ωHd}]
	]
]
\end{forest}
}
%\caption{}
\end{subfigure}
\caption{Isomorphic mapping and prosodic trees for (\ref{ex-alt-jung})}
\label{fig-compbäume}
\end{figure}

\largerpage[-1]
The prosodic trees in Figure~\ref{ex-hnobäume} and Figure~\ref{fig-compbäume} indicate
insensitivity of head alignment in compounds to inherent properties of the respective daughters. The head daughter is picked solely on the basis of her presence in a specific margin position (left or right), which itself is determined by the isomorphic mapping of structures originating in morphological parsing mechanisms. As a result, the head of a compound may consist of a simple phonological word or various types of compounds. 



\section{Phrasal compounds}
\label{sec-phracom}

This section discusses evidence for a third compound category, here referred to as \emph{phrasal compound}, distinct from both regular and copulative compounds. Like copulative compounds, phrasal compounds associate with right-oriented head alignment but differ in that the relation between compound members is characterized by asymmetry. The reason for assuming a single compound category concerns the nature of that asymmetry, which indicates a functor-argument structure. 

This section presents three types of compounds, chosen to illustrate the striking disparities seen in \ili{German} compounds characterized by a functor-argument structure along with final stress. The focus is on a case marked by correspondence patterns involving syntactic phrases, which is also of interest to the parsing issue (bottom-up vs.\ top-down). The other two cases are a class of compounds exhibiting the same distribution as prepositional phrases (\eg \emph{bergáb} `mountain-down' meaning `downhill') and elative compounds (\eg \emph{steinréich} `stone rich', meaning `very rich').

The particular conditions characterizing the first type of phrasal compounds considered here can be illustrated with the words in (\ref{ex-meter}), which end in the same stem /metəʀ/, corresponding to the free-standing masculine noun \emph{Meter} `meter', but differ in stress. 

%\begin{multicols}{2}
\eal\label{ex-meter}
\ex\label{ex-qm} 
[`fɛstˌmetəʀ]\sub{\textsc{n.masc}}\\
$\langle$Festmeter$\rangle$\\
`solid meter' (i.e.\ `cubic meter')
%
%\columnbreak
%
\ex\label{ex-penalty}
[ˌɛlfˈmetəʀ]\sub{\textsc{n.masc}}\\
$\langle$Elfmeter$\rangle$\\
`eleven meters' (penalty kick in soccer)
\zl
%\end{multicols}

\noindent
The contrast seen in (\ref{ex-meter}) brings to mind well-known differences in compound versus phrasal stress which can be illustrated with the near-minimal pairs in (\ref{ex-very}). Similar examples from \ili{German} are listed in (\ref{ex-großvater}).

\settowidth\jamwidth{ \textbf{ein ( sehr ) kleiner Garten } }
\ea\label{ex-very}
\ea a (*very) \emph{wét suit} \jambox{a (very) \emph{wet súit}}

`diving equipment which may be dry' \jambox{`suit which is wet'}

\ex a (*very) \emph{blúeberry} \jambox{a (very) \emph{blue bérry}}

`type of berry	which may be  \jambox{`berry which is blue'}

unripe and green'
\z

\ex\label{ex-großvater}
\ea ein (*sehr) \emph{Gróßvater} \jambox{ein (sehr) \emph{großer Váter}}

`a (*very) grandfather' \jambox{`a (very) tall father'}

\ex ein (*sehr) \emph{Kléingarten} \jambox{ein (sehr) \emph{kleiner Gárten}}

`a (*very) garden plot, \jambox{`a (very) small garden'}

part of an allotment garden'
\z
\z

\noindent
The stress differences have been captured in terms of cyclic rules, where rules assigning stress to words (``Compound Rule'' in \citealp[17]{ChomskyHalle1968}) are ordered before those assigning stress in syntactic phrases (``Nuclear Stress Rule'' in \citealp[17]{ChomskyHalle1968}). The stress difference seen in (\ref{ex-very}), (\ref{ex-großvater}) is not captured by this type of classification as \emph{Elfmeter} `penalty kick' clearly patterns with words, not with phrases:

\eal\label{elfmeter}
\ex Sie hat einen (tollen) \emph{Elfmeter} geschossen. \\
`She took a (great) penalty kick.' \\

\ex Sie hat zwei (tolle) \emph{Elfmeter} geschossen. \\
	 `She took two (great) penalty kicks.' \\
\ex	Sie hat ein (tolles) \emph{Tor} geschossen. \\
	 `She scored a  (great) goal.' \\
\ex	Sie hat zwei (tolle) \emph{Tore} geschossen. \\
	 `She scored two (great) goals.'
\zl

\noindent
The property which distinguishes \emph{Elfmeter} from \emph{Festmeter} is the correspondence to the wellformed phrase \emph{elf Meter} `eleven meters', as in \emph{Es fehlen noch} [ɛlfmetəʀ]. `We still need eleven meters'. This correspondence is due to the fact that \emph{elf} `eleven' is a numeral, which is not inflected in \ili{German}, leading to homophonous forms in compounds and phrases. By contrast, adjectives such as \emph{fest} are inflected in attributive position in phrases, manifest in an ending containing schwa (\eg \emph{fest}\textbf{\emph{er}} \emph{Meter}), distinguishing them from compounds (cf.\ also the relevant differences illustrated in (\ref{ex-großvater})).

The perfect correspondence between the noun \emph{Elfmeter} and the phrase \emph{elf Meter} hinges on a second peculiarity, namely the absence of morphological plural marking in the noun \emph{Meter}. Note that phrases consisting of a numeral referring to the number 2 or higher require the following argument to be a plural form as shown in (\ref{ex-vierecken}). Such phrases differ then from compounds, where numerals combine with bare stems as in (\ref{ex-viereck}):

\begin{multicols}{2}
\ea\label{ex-vierecken}
\ea {[ˌfiʀˈɛkən]}\\
vier Ecken\\
`four angles'

\ex {[ˌdʀaiˈʀædəʀ]}\\
drei Räder\\
`three wheels'

\ex {[ˌfʏnfˈkɛmp\textsuperscript{f}ə]}\\
fünf Kämpfe\\
`five fights'

\ex {[ˌdʀaiˈzæt\textsuperscript{s}ə]}\\
drei Sätze\\
`three sentences'
\z
%
\columnbreak
%
\ex\label{ex-viereck}
\ea {[ˈfiʀˌɛk]}\\
Viereck\\
`quadrangle'

\ex {[ˈdʀaiˌʀɑd]}\\
Dreirad\\
`tricycle'

\ex {[ˈfʏnfˌkamp\textsuperscript{f}]}\\
Fünfkampf\\
`pentathlon' (sports)

\ex {[ˈdʀaiˌzat\textsuperscript{s}]}\\
Dreisatz\\
`rule of three' (mathematics)
\z
\z
\end{multicols}

\noindent
The formal discrepancies seen in (\ref{ex-vierecken}) versus (\ref{ex-viereck}) do not affect the
compound \emph{Elfmeter} as nouns denoting measuring units are typically not inflected for plural in
\ili{German}.\footnote{Measuring terms ending in schwa are a systematic exception here (\eg \{\emph{Tonne,
    Tonnen}\} `ton, tons', \{\emph{Meile, Meilen}\} `mile, miles'.} It is then the (accidental)
alignment of both properties, the exemption of numerals and of nouns denoting measuring units from
inflection, which yield the outcome of perfect correspondence in the relation between the compound
and the phrase. 

Assuming that perfect correspondence to a phrase motivates the classification of the noun \emph{Elfmeter} as a phrasal compound (PHRASCOMP-M), which maps into a prosodic phrasal compound (PHRASCOMP-P), the following representations result.

\eal
\ex\label{ex-festmeter}
{[[fɛst]\sub{STM}[metəʀ]\sub{STM}]\sub{COMP-M}}\\
\hphantom{[[fɛst]\sub{STM}} $\Downarrow$

((fɛst)\sub{ωHd}(metəʀ)\sub{ω})\sub{COMP-P}\\
$\langle$Festmeter$\rangle$

\ex\label{ex-viereck-comp}
[[fiʀ]\sub{STM}[ɛk]\sub{STM}]\sub{COMP-M}\\
\hphantom{[[fɛst]\sub{STM}} $\Downarrow$

((fiʀ)\sub{ωHd}(ɛk)\sub{ω})\sub{COMP-P}\\
$\langle$Viereck$\rangle$


\ex\label{ex-elfmeter}
[[ɛlf]\sub{STM}[metəʀ]\sub{STM}]\sub{PHRASCOMP-M}\\
\hphantom{[[fɛst]\sub{STM}} $\Downarrow$

((ɛlf)\sub{ω}(metəʀ)\sub{ωHd})\sub{PHRASCOMP-P}\\
$\langle$Elfmeter$\rangle$
\zl

\noindent
Left-oriented head alignment in the regular compounds accounts then for the prominence on the initial member in (\mex{0}a,b).
Stress on the final constituent in \emph{Elfmeter} as in (\ref{ex-elfmeter})
is captured by the right-oriented head alignment constraint in (\ref{ex-align43}):

\ea\label{ex-align43}
\textsc{Align}(PHRASCOMP-P, R, Head(PHRASCOMP-P), R)
\z

\noindent
As for the conditions motivating the classification of a compound as phrasal it is important that relevant strings are not just adjacent in syntax but form phrases. For instance, words such as \emph{Mö́chtegern} literally `would gladly', meaning `wanna-be' or \emph{Gérnegroß} literally `gladly big', meaning `braggart' do not match syntactic phrases and are consequently classified as regular compounds with initial stress. Compounds such as \emph{Síebenschläfer}, literally `seven sleeper(/s)', meaning `dormouse' (rumored to hibernate for seven months) or \emph{Zwö́lftonner} `twelve-tonner' (vehicle carrying a load of twelve tons), also cannot be classified as phrasal as the relevant constructions lack a functor-argument structure. These, too, are then classified as regular compounds and receive initial stress.

%\begin{sloppypar}
Additional data indicate the possible relevance of yet another condition, namely
exocentricity. Consider the compounds in (\ref{ex-mürbeteig}), whose initial member ends in schwa
preceded by a voiced obstruent, a context where schwa has tended to persist in \ili{German}. The presence
of stem-final schwa in the compounds leads to homophony with the respective phrases, where schwa is
(also) an inflectional marker.
%\end{sloppypar}


\eal\label{ex-mürbeteig}
\ex {Mǘrb[ə]teig} \jambox{cf.\ (der) mürb[ə] Teig}

brittle.dough   \jambox{\hphantom{cf.~}`(the) brittle dough'}
`shortcrust' 

\ex {Míes[ə]peter} \jambox{cf.\ (der) mies[ə] Peter}

wretched.guy     \jambox{\hphantom{cf.~}`(the) wretched Peter'}
`sourpuss' 

\ex {Leb[ə]wóhl} \jambox{cf.\ Leb[ə] wohl!}

Live.well      \jambox{\hphantom{cf.~}`Live well!'}
`farewell' 

\zl

\noindent
Like the compound \emph{Elfmeter}, those in (\ref{ex-mürbeteig}) are characterized by perfect
correspondence to a syntactic phrase with functor argument structure. The fact that \emph{Mürbeteig}
and \emph{Miesepeter}, two compounds likely originating from phrases historically, are (re)analyzed
as regular compounds with initial stress might be due to their endocentric status, in contrast to
\emph{Elfmeter} or \emph{Lebewohl}, which are clearly exocentric. 

A conclusive answer is not easily obtained, as the relevant conditions are so narrow that they are
rarely met and indeed none of the relevant cases considered so far is productive in \ili{German}. Numerals
cannot combine with nouns to form compounds even when corresponding perfectly to syntactic phrases:
*\emph{Viereimer} `four buckets', *\emph{Dreiesel} `three donkeys' are simply ungrammatical.  

Compounds like \emph{Elfmeter} merit attention only because of the striking robustness of final
stress: the typical shift to initial stress likely having affected \emph{Mürbeteig} or
\emph{Miesepeter} seems entirely unacceptable in \emph{Elfmeter} or \emph{Siebenmeter}, a penalty
kick in hockey defined by a distance of seven meters. Evidence that the sort of accidental
correspondence relations to syntactic phrases claimed to play a role in their analysis are in fact
highly significant comes from particular contexts where relevant conditions on phrasal compounds are
more easily met. The main context is again the embedding of a compound in non-final position, in
fact a likely source of \emph{Elfmeter}, which may be an elliptic form stemming from the complex
compound \emph{Elfmeterschuss} `eleven meters kick'.  

\largerpage
What seems special about the embedded context is again the fact that it need not associate with a
separate concept linked to an entity in the world.\footnote{
  See the discussion  below (\ref{ex-muttersohn}).}
As noted above, one cannot form a compound *\emph{Zweizimmer} `two-rooms' to denote two rooms in
\ili{German}, but there are countless combinations such as \emph{Zweizimmerwohnung} `two room
apartment'. Significantly, the conditions concerning the parsability of strings as phrasal compounds
are confirmed by such cases. Consider the stress contrast in the complex compounds in
(\ref{ex-doppel}), which is due to the fact that a combination of numeral plus noun (\eg \emph{zwei
  Flügel}) corresponds to a well-formed phrase in \ili{German}, whereas the combination
\emph{Doppelflügel} does not.  

\begin{multicols}{2}
\eal\label{ex-doppel}
\ex\label{ex-zweiflügeltür}
{[[Zwei\textbfremoved{flǘgel}]tür]}\\
`two wing door'

\ex\label{ex-zweimütter}
{[[Zwei\textbfremoved{mǘtter}]familie]}\\
`two mother family'
%
\columnbreak
%
\ex\label{ex-doppelflügeltür}
{[[\textbfremoved{Dóppel}flügel]tür]}\\
`double wing door'

\ex\label{ex-doppelverdiener}
{[[\textbfremoved{Dóppel}verdiener]familie]}\\
`two-income family' 
\zl
\end{multicols}
%\itdopt{ex 48 anpassen + akzente über ü}

\largerpage
\noindent
The relevant morphological structures, along with the strictly isomorphic mapping yielding the
prosodic structures, are shown in Figure~\ref{fig-doppelbaum}. Parsing is again subject to the
Elsewhere Condition, such that the conformity of a string with the conditions for phrasal compounds
takes priority, ensuring the classification of the relevant constituent as PHRASCOMP-M versus
COMP-M as shown in the lefthand side of Figure~\ref{fig-doppelbaum}.
%{fig-doppel-head}. 
The position of main stress is then due to isomorphic mapping along with the relevant head alignment
constraints. The most prominent member in the entire compound is the terminal unit, boldfaced
in the trees shown in the righthand side in Figure~\ref{fig-doppelbaum}, as that member both forms a head itself and is dominated exclusively by head constituents.  

\begin{figure}
\begin{subfigure}{.49\textwidth}
\centering
\scalebox{.95}{%
\begin{forest}%sm edges
[COMP-M
	[PHRASCOMP-M
		[{[t\textsuperscript{s}vai]}\sub{STM}]
		[{[flyɡəl]}\sub{STM}{}, tier=word]
	]
	[,empty nodes
		[{[tyʀ]\sub{STM}}, tier=word]
	]
]
\end{forest}
}
%\caption{Morphological structure of (\ref{ex-zweiflügeltür})}\label{fig-doppel-head}
\end{subfigure}
%
\begin{subfigure}{.49\textwidth}
\centering
\scalebox{.95}{%
\begin{forest}%sm edges
[COMP-P
	[PHRASCOMP-P\sub{Hd}
		[(t\textsuperscript{s}vai)\sub{ω}]
		[\textbf{(flyɡəl)\sub{ωHd}}, tier=word]
	]
	[,empty nodes
		[(tyʀ)\sub{ω}, tier=word]
	]
]
\end{forest}
}
%\caption{Prosodic structure of (\ref{ex-zweiflügeltür})}\label{ex-doppel-head-b}
\end{subfigure}
%\medskip
%\captionsetup{skip=23pt}

\vspace{10pt}
\begin{subfigure}{.49\textwidth}
\centering
\scalebox{.95}{%
\begin{forest}%sm edges
[COMP-M
	[COMP-M
		[{[dɔpəl]\sub{STM}}]
		[{[flyɡəl]\sub{STM}},tier=word]
	]
	[,empty nodes
		[{[tyʀ]\sub{STM}}, tier=word]
	]
]
\end{forest}
}
%\caption{Morphological structure of (\ref{ex-doppelflügeltür})}\label{fig-doppelflügel-head}
\end{subfigure}
%
\begin{subfigure}{.49\textwidth}
\centering
\scalebox{.95}{%
\begin{forest}%sm edges
[COMP-P
	[COMP-P\sub{Hd}
		[(\textbf{(dɔpəl)\sub{ω}}]
		[(flyɡəl)\sub{ωHd}, tier=word]
	]
	[,empty nodes
		[(tyʀ)\sub{ω}, tier=word]
	]
]
	\end{forest}
}
%\caption{Prosodic structure of (\ref{ex-doppelflügeltür})}\label{ex-doppelflügel-head-b}
\end{subfigure}
\caption{Morphological and prosodic structures for [[Zwei\textbfremoved{flǘgel}]tür]
`two wing door' %(\ref{ex-zweiflügeltür}) 
and [[\textbfremoved{Dóppel}flügel]tür] `double wing door'
%(\ref{ex-doppelflügeltür})
}\label{fig-doppelbaum}
\end{figure}
%\itdopt{ex 49 einfügen}


The relevance of the homophony between the singular and plural form of \emph{Flügel} for the
morphological parsing of the constituent \emph{Zweiflügel} as a phrasal compound can be demonstrated
further with the stress contrast in (\ref{ex-zwölf}). Stress consistently falls on the second member
in (\ref{ex-zwölf}a,b), where singular and plural forms are identical
(i.e.\ \emph{Zimmer}), compared to initial stress in (\ref{ex-zwölf}c,d), where the following noun
does not match the plural form.

\begin{multicols}{2}
\eal\label{ex-zwölf}
\ex\label{ex-dreizimmerwohnung}
{}[[Drei\textbfremoved{zímmer}]wohnung]\\
`three room apartment'

\ex\label{ex-finger}
{} [[Zwölf"|\textbfremoved{fínger}]darm]\\
`twelve finger gut' (duodenum)
%
\columnbreak
%
\ex\label{ex-dreiraumwohnung}
{} [[\textbfremoved{Dréi}raum]wohnung]\\
`three room apartment'

\ex\label{ex-zwölfton}
{} [[\textbfremoved{Zwö́lf}ton]musik]\\
`twelve-tone music'\\
~
\zl
\end{multicols}
%\todo{trema+akzent funtioniert nicht in bold/emph}

\noindent
The examples in (\ref{ex-drei}) illustrate a stress difference conditioned by the use of distinct word forms: the numeral is followed by a plural form in (\ref{ex-menü})–(\ref{ex-vierb}), thereby meeting the requirement for phrasal compounds, vis-à-vis the occurrence of the respective singular form in (\ref{ex-getriebe})–(\ref{ex-vierbett}), which results in the classification as a regular compound with initial stress.

\begin{multicols}{2}
\eal\label{ex-drei}
\ex\label{ex-menü}
Drei\textbfremoved{gä́nge}menü\\
`three course menu'

\ex\label{ex-dreiw}
Drei\textbfremoved{wége}hahn\\
`three-way valve'

\ex\label{ex-vierb}
Vier\textbfremoved{bétten}pension\\
`four bed pension'
%
\columnbreak
%
\ex\label{ex-getriebe}
\textbfremoved{Dréi}ganggetriebe\\
`three gear transmission'

\ex\label{ex-mehrweg}
\textbfremoved{Méhr}wegflasche\\
`returnable bottle'

\ex\label{ex-vierbett}
\textbfremoved{Víer}bettzimmer \\
`four bed room'
\zl
\end{multicols}
%\todo{trema+akzent funtioniert nicht in bold/emph}

\noindent
There is an alternative analysis of the stress differences in (\ref{ex-zwölf}) and (\ref{ex-drei}),
linked to the mono- versus disyllabicity of the second compound member (cf.\
\citealp[154]{Giegerich1985}, \citealp[301]{Wiese2000}). On Giegerich's account main stress falls on
the second member as in (\ref{ex-zwölf}a,b), unless that form is
monosyllabic. In that context, stress shifts to the initial member as in (\ref{ex-zwölf}c,d), to improve the rhythm. This analysis incorrectly predicts stress on the second member in cases like \emph{Zw\'{ö}lftonmusìk, Dréiganggetrìebe}, where the third member starts with an unstressed syllable.

The claim that the stress patterns of the embedded compounds in (\ref{ex-drei}) are determined by the question of whether or not they correspond to a well-formed phrase is consistent with the facts but 
raises the question of what determines the choice of the relevant plural versus singular forms in
the first place. As for the cases in (\ref{ex-drei}d--f), the relevant choice may be influenced by
the prevalence of corresponding compounds with the Numeral \emph{Ein}-, which is always followed by
a singular form (\eg \emph{Éinganggetriebe} `one gear transmission', \emph{Éinwegflasche} `one-way
(disposable) bottle', \emph{Éinbettzimmer} `one-bed' (single) room').% 
%
\footnote{This raises the question of why the relevant compounds with \emph{ein} have initial main stress, given that \emph{ein Gang}, \emph{ein Weg}, \emph{ein Bett} are perfectly well-formed phrases. The answer here may lie in the homophony between the numeral \emph{ein} `one' and the indefinite article \emph{ein} `a/an', causing stress on the numeral to mark the contrast to the article.}
%
A generalization likely affecting the choice of the singular forms in (\ref{ex-zwölf}c,d) concerns the marking of the relevant plural forms with umlaut. The relevant
correlations are far from perfect, due in part to various analogical influences as noted in
connection with (\ref{ex-drei}), but there is a tendency to avoid the phonologically marked umlaut
forms.\footnote{A regular exception concerns cases where plural is marked only by umlaut (\eg plural
  \emph{Mütter} - singular \emph{Mutter}). Here it is always the umlaut form which appears in the
  embedded compounds (\eg \emph{Zweimütterfamilie} `two mother family').} The data in
(\ref{ex-stern}) illustrate a general preference for plural forms in embedded compounds containing
numerals. If the plural form is marked with umlaut, as in (\ref{ex-raum}), the singular form is often chosen instead, with the result that the condition for the classification as a phrasal compound is no longer met. As a result, stress on the second member in the complex compounds in (\ref{ex-stern}) contrasts with initial stress in (\ref{ex-raum}), as the latter compounds are classified as regular by default.


%\eal\label{ex-dreisternehotel}
\ea\label{ex-stern}
\settowidth\jamwidth{`two river land' (Mesopotamia)}
\ea \{Stern, \textbf{Sterne}\} \jambox{Drei\textbf{stérne}hotel}

`star, stars' \jambox{`three-star hotel'}

\ex \{Tag, \textbf{Tage}\} \jambox{Drei\textbf{táge}bart}

`day, days' \jambox{`three-day beard'}

\ex \{Staat, \textbf{Staaten}\}	\jambox{Zwei\textbf{stáaten}lösung}

`state, states' \jambox{`two-state solution'}

\ex \{Burg, \textbf{Burgen}\} \jambox{Fünf\textbf{búrgen}tour}

`castle, castles' \jambox{`five castle tour'}

\ex \{Auge, \textbf{Augen}\} \jambox{Vier\textbf{áugen}gespräch}

`eye, eyes'	\jambox{`four-eyes talk'}

\ex \{Front, \textbf{Fronten}\} \jambox{Zwei\textbf{frónten}krieg}

`front, fronts'	\jambox{`war on two fronts'}

\ex \{Person, \textbf{Personen}\} \jambox{Drei\textbf{persónen}haushalt}

`person, people' \jambox{'three-person household'}

\ex \{Feld, \textbf{Felder}\} \jambox{Drei\textbf{félder}wirtschaft}

`field, fields'	\jambox{`three-field farming'}

\ex \{\textbf{Fuß} (measuring unit)\} \jambox{Zehn\textbf{fúß}container}

`foot' \jambox{`ten foot container'}

\ex \{\textbf{Karat}\} \jambox{Zehn\textbf{karát}ring}

`carat' \jambox{`ten carat ring'}
\z

%-----------------------------------------

\ex\label{ex-raum}
\ea \{\textbf{Raum}, Räume\}	\jambox{Dréi\textbf{raum}wohnung}

`room, rooms'	\jambox{`three room apartment'}

\ex \{\textbf{Ton}, Töne\} \jambox{Zwö́lf\textbf{ton}musik}

`tone, tones' \jambox{`twelve-tone music'}

\ex \{\textbf{Frucht}, Früchte\}	\jambox{Víer\textbf{frucht}gelee}

`fruit, fruits'	\jambox{`four fruit jam'}

\ex \{\textbf{Strom}, Ströme\} \jambox{Zwéi\textbf{strom}land}

`river, rivers' \jambox{`two river land' (Mesopotamia)}

\ex \{\textbf{Loch}, Löcher\} \jambox{Fǘnf\textbf{loch}felge}

`hole, holes' 	\jambox{`five lug rim'}

\ex \{\textbf{Wort}, Wörter\} \jambox{Zwéi\textbf{wort}gefüge}

`word, words' \jambox{`two word construction'}

\ex \{\textbf{Korn}, Körner\} \jambox{Fǘnf\textbf{korn}brot}

`grain, grains' \jambox{`five grain bread'}

\ex \{\textbf{Zug}, Züge\} \jambox{Zwéi\textbf{zug}samstag}

`train, trains'	\jambox{'two train Saturday'}

\ex \{\textbf{Fuß}, Füße\} \jambox{Zéhn\textbf{fuß}krebs}

`foot, feet' \jambox{`ten foot crab'}

\ex \{\textbf{Kanal}, Kanäle\} \jambox{Zwéi\textbf{kanal}ton}

`channel, channels' \jambox{`two channel sound'}
\z
\z
%\todo{Zwölfton, Fünfloch, Fünfkorn + bf geht nicht s.o.}
%\itdopt{ex 52 einfügen}

\noindent
The last examples are of particular interest as the nouns [kɑˈʀɑt] \emph{Karat} and [kɑˈnɑl]
\emph{Kanal} have very similar shapes and appear in the same metrical environment in the respective
compounds. The claim that correspondence to a well-formed phrase is crucial to the morphological
classification of the embedded compound explains the link between the lack of a distinct plural form
for \emph{Karat}, due to its status as a measuring unit, and the presence of stress on the second
member (i.e.\ \emph{Zehnkarátring}). Again, the homophony of the relevant noun forms allows for the
classification of the string \emph{Zehnkarat}{}- as a phrasal compound, which receives final stress. By contrast, the paradigm of the noun \emph{Kanal} contains a distinct plural form \emph{Kanäle}, which rules out the classification of \emph{Zweikanal-} as a phrasal compound. The classification of a regular compound ensues by default, resulting in stress on the initial member (i.e.\ \emph{Zwéikanalton}).
The same explanation pertains to the formations with \emph{Fuß} in the line above, used as a measuring term in one case (\emph{Zehnfúßmonitor}) and a regular noun associated with a distinct plural form in the other (\emph{Zéhnfußkrebs}). 

The expectation that all measuring terms attract main stress in the relevant three member compounds, regardless of their shape (disyllabic in (\ref{ex-fünfeuro}), monosyllabic in (\ref{ex-fünfuhr})) and of the metrical environment is born out.

%\eal
\ea\label{ex-fünfeuro}
\ea {}[[Fünf\textbfremoved{éuro}]job] \jambox{`five euro job'}

\ex {}[[Fünf\textbfremoved{méter}]turm] \jambox{`five meter tower'}

\ex {}[[Fünf\textbfremoved{prozént}]hürde] \jambox{`five percent hurdle'}

\ex {}[[Zehn\textbfremoved{pfénnig}]marke] \jambox{`ten penny stamp'}

\ex {}[[Drei\textbfremoved{líter}]auto] \jambox{`three liter car'}

\ex {}[[Drei\textbfremoved{gróschen}]heft] \jambox{`three penny booklet'}

\ex {}[[Zehn\textbfremoved{dóllar}]aktie] \jambox{`ten dollar stock'}

\ex {}[[Fünf\textbfremoved{héktar}]hof] \jambox{`five hectare farm'}

\ex {}[[Zehn\textbfremoved{fránken}]schein] \jambox{`ten franc bill'}

\ex {}[[Drei\textbfremoved{zéntner}]sack] \jambox{`three centner bag'}
\z

\ex\label{ex-fünfuhr}
\ea {}[[Fünf\textbfremoved{úhr}]zug] \jambox{`five o'clock train'}

\ex {}[[Zwei\textbfremoved{márk}]stück] \jambox{`two mark piece'}

\ex {}[[Zwei\textbfremoved{pfúnd}]brot] \jambox{`two pound bread'}

\ex {}[[Zwei\textbfremoved{cént}]stück] \jambox{`two cent piece'}

\ex {}[[Fünf\textbfremoved{grád}]winkel] \jambox{`five degree angle'}

\ex {}[[Zwölf\textbfremoved{zóll}]display] \jambox{`twelve inch display'}

\ex {}[[Fünf\textbfremoved{wátt}]verstärker] \jambox{`five watt amplifier'}

\ex {}[[Zehn\textbfremoved{Hértz}]Bereich] \jambox{`ten hertz range'}

\ex {}[[Fünf\textbfremoved{grámm}]beutel] \jambox{`five gram bag'}

\ex {}[[Zwei\textbfremoved{Mánn}]Band]\footnotemark \jambox{`two man band'}

\footnotetext{While the regular plural form of \emph{Mann} `man' is \emph{Männer} `men', the unmarked plural form Mann is grammatical after numerals (\eg \emph{mit zwei Mann} `with two men'). The internal compound in \emph{Zwei-Mánn-Band} `two-man-band' is accordingly analyzable as a phrasal compound, the possible stress on the second member follows from right head alignment.}
\z
\z

\enlargethispage{-5pt}
\noindent
The patterns demonstrated in (\ref{ex-doppel})--(\ref{ex-fünfuhr}) are relevant to the issue of morphological parsing in that they indicate reference to the surface forms of complex words when determining the classification of compounds, rather than to the properties of individual morphemes. In particular, the relevance of syncretism in paradigms for the conditions identified here makes sense only from an analytic, not from a synthetic perspective.

I will end the discussion of the particular phenomenon presented here, namely a condition on the
classification of compounds requiring correspondence with well-formed phrases, with a presentation
of cases straddling the boundary of compounding and derivational morphology.  The cases in question
concern adverbs ending in \emph{-weise}, a morpheme categorized as a suffix or suffixoid in \ili{German}
grammars. The data in (\ref{ex-erweise}a--c) illustrate a particular pattern associated with
\emph{-weise}, namely the derivation of adverbs from adjectives requiring the interfix \suffix{er}
(\eg \emph{dumm} `stupid' + \emph{-er-weise} `ly'). Those in (\ref{ex-erweise}d--f) illustrate other
adverbs, where the initial stem just happens to end in the phoneme sequence /əʀ/ \suffix{er}:


\begin{multicols}{2}
\eal\label{ex-erweise}
\ex\label{ex-dummerweise}
{dummerwéise}\\
`stupidly'

\ex\label{ex-netterweise}
{netterwéise}\\
`kindly'

\ex\label{ex-klugerweise}
{klugerwéise}\\
`wisely'


%
\columnbreak
%
\ex\label{ex-eimerweise}
{éimerweise}\\
`by the buckets'

\ex\label{ex-zentnerweise}
{zéntnerweise}\\
`by the hundredweight'

\ex\label{ex-kleckerweise}
{kléckerweise}\\
`in dribs and drabs'

\zl
\end{multicols}

\noindent
The remarkable pattern is seen in the lefthand column in (\ref{ex-erweise}), as consonant-initial
suffixes typically do not allow association with main stress in
\ili{German}.\footnote{\citet{Muthmann1989} lists 105 words ending in the string \emph{-erweise} where
  that string is preceded by an adjective. All of them carry main stress on \suffix{weise}. None of
  the remaining words ending in \emph{-weise} have final main stress.} This peculiarity is explained
by the correspondence of the words in the lefthand column to a noun phrase headed by the
free-standing noun \emph{Weise} `manner' illustrated in (\ref{ex-weise-synt}). In particular, it is
the embedding of the relevant noun phrase in a prepositional phrase containing the preposition
\emph{in} which is relevant here, as this preposition requires the adjective in the noun phrase to
end in inflectional \suffix{er}:  

\ea\label{ex-weise-synt}
\gll in dummer Weise \\
     in stupid manner\\
\glt `in a stupid manner' 
\z

\largerpage
\noindent
The assumption that the correspondence of derived adverbs such as \emph{dummerweise} to the syntactic phrase \emph{dummer Weise} shown in (\ref{ex-weise-synt}) motivates the classification of the adverb as a phrasal compound accounts for the highly unusual pattern of final main stress. Stress then again results from right-oriented head alignment. 

The stress patterns in the suffixations in (\ref{ex-erweise}) may seem odd in that main stress associates with the functor, rather than its argument.\footnote{The morpheme \suffix{weise} must be considered as functor in all of the cases in (\ref{ex-erweise}), where it functions as a suffix, but that is not the case for the noun \emph{Weise} in (\ref{ex-weise-synt}).} However, this correlation also pertains to one of the other cases of phrasal compounds to be presented here, illustrated in (\ref{ex-flussauf}):

\largerpage
\settowidth\jamwidth{((tsvaifəls)ω(onə)ωHd)\sub{PHRASCOMP-P}}
\ea\label{ex-flussauf}
\ea {}[[fluss][auf]\sub{Prep}]\sub{PP} \jambox{((flʊs)\sub{ω}(auf)\sub{ωHd})\sub{PHRASCOMP-P}}

river.up\\
`up the river'		

\ex {}[[berg][ab]\sub{Prep}]\sub{PP} \jambox{((bɛʀɡ)\sub{ω}(ap)\sub{ωHd})\sub{PHRASCOMP-P}}

mountain.down\\
`downhill'		

\ex {}[[kopf][über]\sub{Prep}]\sub{PP} \jambox{((kɔpf)\sub{ω}(ybəʀ)\sub{ωHd})\sub{PHRASCOMP-P}}

head.over\\
`head first'		

\ex {}[trepp][auf]\sub{Prep}]\sub{PP} \jambox{((tʀɛp)\sub{ω}(auf)\sub{ωHd})\sub{PHRASCOMP-P}}

stairs.up\\
`up the stairs'		

\ex {}[[fern][ab]\sub{Prep}]\sub{PP} \jambox{((fɛʀn)\sub{ω}(ap)\sub{ωHd})\sub{PHRASCOMP-P}}

far.from\\
`far away (from some point x)'		

\ex {}[[zweifels][ohne]\sub{Prep}]\sub{PP} \jambox{((t\textsuperscript{s}vaifəls)\sub{ω}(onə)\sub{ωHd})\sub{PHRASCOMP-P}}

doubt.without\\
`without any doubt'		

\ex {}[[kurz][um]\sub{Prep}]\sub{PP} \jambox{((kʊʀt\textsuperscript{s})\sub{ω}(ʊm)\sub{ωHd})\sub{PHRASCOMP-P}}
short.about\\
`in short'		

\ex {}[[rund][um]\sub{Prep}]\sub{PP} \jambox{((ʀʊnd)\sub{ω}(ʊm)\sub{ωHd})\sub{PHRASCOMP-P}}
round.about\\
`all around (some point x)'		

\ex {}[[gerade][aus]\sub{Prep}]\sub{PP} \jambox{((ɡəʀɑdə)\sub{ω}(aus)\sub{ωHd})\sub{PHRASCOMP-P}}

straight.out\\
`straight ahead'

\ex {}[[mit][unter]\sub{Prep}]\sub{PP} \jambox{((mɪt)\sub{ω}(ʊntəʀ)\sub{ωHd})\sub{PHRASCOMP-P}}

with.under\\
`from time to time'		

\ex {}[[neben][bei]\sub{Prep}]\sub{PP} \jambox{((nebən)\sub{ω}(bai)\sub{ωHd})\sub{PHRASCOMP-P}}

next(to).by\\
`by the way'		

\ex {}[[neben][an]\sub{Prep}]\sub{PP} \jambox{((nebən)\sub{ω}(an)\sub{ωHd})\sub{PHRASCOMP-P}}

next(to).at\\
`next door'		

\ex {}[[gegen][über]\sub{Prep}]\sub{PP} \jambox{((ɡeɡən)\sub{ω}(ybəʀ)\sub{ωHd})\sub{PHRASCOMP-P}}

against.over\\
`across from (vis-à-vis)'

\ex {}[[vor][ab]\sub{Prep}]\sub{PP} \jambox{((fɔʀ)\sub{ω}(ap)\sub{ωHd})\sub{PHRASCOMP-P}}

before.from\\
`in advance'		
\z
\z

\noindent
The compounds in (\ref{ex-flussauf}) consist of a stressed preposition preceded by some sort of argument, in some cases exhibiting non-compositional meanings. The claim that each member of these compounds nonetheless forms a separate phonological word is supported by the fact that stem-final consonants never syllabify as onsets before a vowel-initial preposition, as is indicated by the potential glottalization of the relevant vowel (\eg\ /flʊs[\textsuperscript{ʔ}a]uf/ \emph{flussauf}, /fɔʀ[\textsuperscript{ʔ}a]p/ \emph{vorab}).
 
A key property motivating the analysis of the compounds in (\ref{ex-flussauf}) as phrasal concerns the fact that the distribution of the complex expression matches that of a prepositional phrase and cannot be used to modify other words. This distinguishes the compounds in (\ref{ex-flussauf}) from similar-looking words which do function as modifiers and exhibit the characteristics of regular compounds, in particular initial stress. Compare the \emph{phrasal compound} \emph{rundum} `all around' with the regular compound \emph{rundum} `completely' in (\ref{ex-rundum56}):

\eal\label{ex-rundum56}
\ex
\gll In der Mitte brannte ein Feuer, \textbfremoved{rundúm} saßen die Kinder. \\
     in the middle burned a   fire   round.about            sat   the children\\
\glt `In the middle there was a fire, all around (it) the children sat.'

\ex\label{ex-rundum}
\gll Sie war \textbfremoved{rúndum} glücklich. \\
     she was round.about             happy\\
\glt `She was completely happy.'
\zl

\largerpage
\noindent
The classification of the example in (\ref{ex-rundum}), together with additional examples for regular compounds ending in prepositions, are given in (\ref{ex-roundabout}).%
%
\footnote{In some of these cases the compound can fuse into a single phonological word, forming a single domain for syllabification (\eg\ ((hɛl)\sub{ωHd}(auf)\sub{ω})\sub{COMP-P} \textasciitilde\ (ˈhɛlauf)\sub{ω}).}
%\itdopt{Fn 39 anpassen}

\settowidth\jamwidth{((tsvaifəls)\sub{ω}(onə)\sub{ωHd})\sub{PHRASCOMP-P}}
\ea\label{ex-roundabout}
\ea {}[[rund][um]\sub{Prep}]\sub{Adv} (glücklich) \jambox{((ʀʊnd)\sub{ωHd}(ʊm)\sub{ω})\sub{COMP-P}}

round.around\\
`completely (happy)'


\ex {}[[hell][auf]\sub{Prep}]\sub{Adv} (begeistert) \jambox{((hɛl)\sub{ωHd}(auf)\sub{ω})\sub{COMP-P}}

bright.up\\
`completely (enthusiastic)'


\ex {}[[voll][auf]\sub{Prep}]\sub{Adv} (zufrieden) \jambox{((fɔl)\sub{ωHd}(auf)\sub{ω})\sub{COMP-P}}

full.up\\
`completely (content)'


\ex {}[[weit][aus]\sub{Prep}]\sub{Adv} (besser) \jambox{((vait)\sub{ωHd}(aus)\sub{ω})\sub{COMP-P}}

far.out\\
`much (better)'


\ex {}[[über][aus]\sub{Prep}]\sub{Adv} (freundlich) \jambox{((ybəʀ)\sub{ωHd}(aus)\sub{ω})\sub{COMP-P}}

over.out
`most (friendly)'


\ex {}[[Schluck][auf]\sub{Prep}]\sub{\textsc{n.masc}} \jambox{((ʃlʊk)\sub{ωHd}(auf)\sub{ω})\sub{COMP-P}}

swallow.up\\
`hiccup'	
\z
\z

\largerpage
\noindent
There is nothing ``phrasal'' about these expressions and they are accordingly classified as regular compounds. As a result, they are subject to left-oriented head alignment manifest in main stress on the initial member. 

It is clear that the criteria for classifying compounds as phrasal differ substantially in the cases defined by correspondence with syntactic phrases discussed earlier and those defined by a distribution similar to syntactic phrases presented here. Still in both cases some sort of phrasal properties associate with right-oriented head alignment. This also concerns the last case of compounds associated with final main stress briefly presented in (\ref{hundemuede}). These are known as elative compounds, where the first member denotes a high degree of the property associated with the second member.

\begin{multicols}{2}
\ea\label{hundemuede} 
\ea hunde\textbfremoved{mǘde}\\
dog/s.tired\\
`very tired'

\ex schweine\textbfremoved{téuer}\\
pig/s.expensive\\
`very expensive'

\ex stein\textbfremoved{réich}\\
stone.rich\\
`very rich'
%
%\columnbreak
%
\ex stroh\textbfremoved{dúmm}\\
straw.stupid\\
`very stupid'

\ex kern\textbfremoved{gesúnd}\\
kernel/core.healthy\\
`very healthy'

\ex arsch\textbfremoved{kált}\\
arse.cold\\
`very cold'
\z
\z
\end{multicols}
%\todo{hundemüde geht nicht mit bf s.o.}


\noindent
Very similar compounds are seen in other languages, including \ili{Dutch}, where they have been explicitly
classified as phrasal \citep[157--158]{TrommelenZonneveld1986}. \citet{Hoeksema2012} also discusses
several properties setting elative compounds apart from regular compounds in \ili{Dutch}, including
(optional) emphatic lengthening of the vowel to indicate extra high degree
\citep[98]{Hoeksema2012}. He further notes the possibility of emphatic reduplicative conjunction,
which is also seen in regular free-standing adverbs of degree.  

I will not pursue this matter further but simply note that main stress on the final compound members in (\ref{hundemuede}) were captured by right-oriented head alignment, if elative compounds were recognized as phrasal in \ili{German} as well. 

\section{Summary}\label{sec-conclu-rf}

The present article explores the notion of head alignment, based mostly on stress patterns in \ili{German} compounds. Head alignment constraints, originally proposed by \citet{McCarthyPrince1993} to capture the most prominent foot in a phonological word, refer to either the left or the right boundary of a specific prosodic constituent, requiring that boundary to coincide with its head constituent. The basic generalization is that the position of main stress within any given prosodic domain always refers to one of the margins, the choice among which is determined by the category of the relevant domain. Reference to the term `head' in this alignment constraint is fitting as it encapsulates both central properties of heads in grammar: uniqueness (only one daughter is picked to function as head) and dominance (assuming that prominence associated with heads can be viewed as a form of dominance).
 
A central aim of this article is to draw attention to the heuristic value of the notion of head alignment for identifying and defining morphological categories. For instance, the (tentative) assumption of a right-oriented head alignment constraint referring to copulative compounds has motivated the assumption of exocentricity as one of the defining properties of such compounds in \ili{German}. Words with right-oriented main stress such as \emph{blau-wéiß, Schleswig-Hólstein, Klimbím} belong here whereas forestressed words often cited as examples for copulative compounds such as \emph{Mántelkleid, násskalt} or \emph{Hássliebe} do not meet this criterion. The latter words are indeed characterized by an asymmetry to the effect that the initial member is readily understood as a modifier. 

The possible confinement of copulative compounds in \ili{German} to those which are exocentric raises a
further issue pertaining to terminology. If \ili{English} does in fact allow truly `double-endocentric'
compounds such as \emph{hunter-gatherer}, meant to designate one who is equally a hunter and a
gatherer, whereas \ili{German} speakers must resort to syntax to express this sort of equality
(\emph{Sammler und Jäger}), the use of a single label (say, \emph{copulative compound}) in the
grammar of the two languages is bound to sow confusion. Here the single label is perhaps
best retained, in recognition of the fact that both languages have a class of compounds
characterized by a flat structure where all members are on a par and the final member carries main
stress. 
%This approach then allows for the possibility that this class may include double-endocentric
%compounds in some languages but only exocentric compounds in others.  
Two subclasses of copulative compounds must be distinguished then: those which allow
``double-endocentric'' compounds, characterized by equal hyponymy of the compound in relation to
each of its members, versus those restricted to exocentric compounds.

Another issue addressed throughout this article concerns the question of how the morphological
classification of individual words is established. The question centers on the concept of a  
`bottom-up' approach, where the structure of complex expressions is determined by the inherent
properties of the individual building blocks (morphemes) and the rules for combining them, versus a  
`top-down' approach, where reference to the complete form is essential to determining
categorization. Evidence for the latter model has been mentioned in connection with each compound
category, For instance, phonotactic violations resulting from the independently given segmental
structure of adjacent morphemes have been shown to motivate morphological decomposition in regular
compounds vis-à-vis simplexes, manifest in the location of main stress (\emph{Vóltmeter} versus
\emph{Varméter}). Similarly the independently conditioned presence or absence of inflectional
markers in compounds consisting of an attribute followed by a noun has been shown to motivate the
classification of one as a regular and the other as a phrasal compound, again manifest in the
location of main stress (\emph{Féstmeter} versus \emph{Elfméter}). 

Evidence for top-down parsing strategies is of interest in that it challenges the empirical adequacy of a pure bottom-up approach often taken for granted in formal linguistics. Here, too, the notion of head alignment constraints defined in terms of specific categories serves as a heuristic for guiding the search for relevant generalizations.%
\il{German|)}


\section*{\acknowledgmentsUS}

I thank the participants of the workshop on heads at the Freie Universität Berlin in May 2017 for
feedback to various ideas first presented there. I benefited from the comments by two anonymous
reviewers and in particular from discussions with Carlos Gussenhoven, Roger Schwarzschild, and Susan
Olsen, who I also thank for proofreading an earlier version of the article. Vanessa Dengel's and
Rebecca Karrer's assistance with preparing the manuscript is much appreciated and I am most grateful
to Elisabeth Eberle for the care she took with the typesetting. I especially wish to
thank Stefan Müller for his extraordinary generosity in helping with the typesetting and
bringing the chapter into its final form.

{\sloppy
\printbibliography[heading=subbibliography,notkeyword=this]
}

\end{document}


% en
%      <!-- Local IspellDict: en_US-w_accents -->
