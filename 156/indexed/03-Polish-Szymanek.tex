\documentclass[output=paper]{LSP/langsci} 
\author{Bogdan Szymanek\affiliation{John Paul II Catholic University
    of Lublin}
}
\title{Compounding in Polish and the absence of phrasal compounding} 
\abstract{In Polish, as in many other languages, phrasal compounds of the type found in English do not exist. Therefore, the following questions are worth considering: Why are phrasal compounds virtually unavailable in Polish? What sort of structures function in Polish as equivalents of phrasal compounds? Are there any other types of structures that (tentatively) could be regarded as “phrasal compounds”, depending on the definition of the concept in question? Discussion of these issues is preceded by an outline of nominal compounding in Polish. Another question addressed in the article is the following: How about phrasal compounds in other Slavic languages? A preliminary investigation that I have conducted reveals that, just like in Polish, phrasal compounds are not found in other Slavic languages. The only exception seems to be Bulgarian where a new word-formation pattern is on the rise, which ultimately derives from English phrasal compounds.}

\ChapterDOI{10.5281/zenodo.885119}
\maketitle

\begin{document}
  
\section{Introduction}\label{sec:szymanek:1}

In the \ili{Polish} language, there are no phrasal compounds comparable to \ili{English} forms like \textit{a scene-of-the-crime photograph} etc., with a non-head phrase-level constituent. Instead, phrases are used. For instance:


\ea%1
    \label{ex:szymanek:1} 
\ea 
a scene-of-the-crime photograph\\
\gll \textit{fotografia} (\textit{z}) \textit{miejsca} \textit{przestępstwa}\\
photograph (from) scene.\textsc{gen} crime.\textsc{gen} \\
\glt ‘photograph from/of the scene of the crime’ \\
\newpage
\ex 
 a “chicken and egg” situation (N+and+N)  \citep[44]{Trips2014}\\
 \ea   
 \gll ??\textit{sytuacja} “\textit{kury} \textit{i} \textit{jajka}”\\
situation chicken.\textsc{gen} and egg.\textsc{gen}\\
\glt ‘a situation of a chicken and an egg’

 \ex 
 \gll ?\textit{sytuacja} “\textit{kura} \textit{czy} \textit{jajko?”}\\
 situation chicken.\textsc{nom} or egg.\textsc{nom}\\
\glt    ‘a situation: “a chicken or an egg?”’

 \ex \textit{sytuacja} \textit{typu} – \textit{co} \textit{było} \textit{pierwsze:} \textit{kura} \textit{czy} \textit{jajko?}\\
situation type.\textsc{gen} what was first     chicken or egg\\
\glt    ‘a situation of the type – what was first: a chicken or an egg?’
  \z
  \ex 
  a “work or starve” philosophy   (conjoined verbs)  \citep[44]{Trips2014}\\
\gll  \textit{filozofia} “\textit{pracuj} \textit{lub} \textit{głoduj}”\\
philosophy work.\textsc{imp} or starve.\textsc{imp}\\
\z
\z 

It can be seen, on the basis of these relatively simple examples of \ili{English} phrasal compounds (PCs) that their \ili{Polish} equivalents appear in a variety of phrasal and clausal forms (including more or less elaborate periphrasis). Occasionally the translation will allow for alternative renderings, sensitive to subtle lexical and stylistic differences. From the viewpoint of translation into \ili{Polish}, the \ili{English} orthographic convention of enclosing pre-head elements within quotation marks somehow looks more palatable (familiar) than its alternative, i.e. hyphenation. But still, a word-by-word rendering of the \ili{English} PC \textit{a “chicken and egg” situation}, i.e. as \textit{*„kura i jajko” sytuacja} is utterly impossible. As regards (\ref{ex:szymanek:1}b) – the choice of the particular \ili{Polish} form is not only a question of (syntactic) grammaticality but rather of semantic equivalence and faithfulness (in translation) as well as of the degree of stylistic appropriateness. The problems are then comparable to those we encounter when translating idioms.

In their “Introduction” to the special issue of STUF, entitled \textit{Phrasal compounds from a typological and theoretical perspective}, \citet[236]{TripsKornfilt2015intro} point out that “there are no (comprehensive) studies [of phrasal compounds] available”,, for languages other than \ili{English}, \ili{German} or \ili{Turkish}, while there are only “some brief discussions of aspects of phrasal compounds” for a few other languages \citep[236]{TripsKornfilt2015intro}. Clearly, in order to understand the status and scope of phrasal compounding in a cross-linguistic perspective, we need to examine the structures of a greater number of (typologically diverse) languages.

\ili{Polish} is one such language for which there have been no reports in the literature concerning the category of phrasal compounds. That this is a non-issue in \ili{Polish} linguistics is further suggested by the fact that an established term like \textit{złożenie frazowe}, equivalent to \ili{English} ‘phrasal compound’, simply is not available in \ili{Polish}, in contradistinction to terms like \textit{derywaty odfrazowe} ‘(de)phrasal derivatives’ or \textit{derywaty od wyrażeń syntaktycznych} ‘derivatives from syntactic expressions’, which suggests that \ili{Polish} word-formation does operate on phrasal constituents, but only as long as they are inputs to affixal \isi{derivation}. Therefore, it is argued in this paper that phrasal compounds (of the type found in \ili{English}) do not exist in \ili{Polish}.\footnote{Cf. \citet[395]{Bisetto2015} for a similar claim concerning \ili{Italian} and \ili{Romance} languages in general.} Assuming the correctness of this prediction, the following questions are worth considering:

\begin{itemize}
\item Why are phrasal compounds virtually unavailable in \ili{Polish}?
\item What sort of structures function in \ili{Polish} as equivalents of \ili{English} phrasal compounds?
\item Are there any other types of structures in \ili{Polish}, that (tentatively) could be regarded as “phrasal compounds”, depending on the definition of the concept in question?
\item  How about phrasal compounds in other \ili{Slavic} languages?
\end{itemize}

\section{An outline of nominal compounding in Polish} \label{sec:szymanek:2}
Generally speaking, compounding in \ili{Polish} is much less productive than in a language like \ili{English}.\footnote{This section incorporates modified fragments from my article which originally appeared as \citet{Szymanek2009}.} The majority of the relevant data are compound nouns. Compound adjectives are also fairly common in contemporary \ili{Polish}, while the formation of compound verbs is completely unproductive.\footnote{For the sake of completeness, it should be noted that there are a few older (often obsolete and lexicalized) compound verbs in present-day use; e.g. \textit{lekceważyć} ‘snub, disregard’ < \textit{lekce} ‘lightly, little (obs.)’ + \textit{ważyć} ‘weigh’, \textit{zmartwychwstać} ‘rise from the dead’ < \textit{z} ‘from’ + \textit{martwych} ‘dead, gen. pl.’ + \textit{wstać} ‘rise’, etc.} Below I focus on the class of compound nouns, their structural diversity and certain formal properties. Such a delimitation of the scope of this article is dictated not only by the fact that compound nouns outnumber compounds of other types in \ili{Polish}, but also by our main topic, i.e. phrasal compounds, which are nouns.


Typically, a compound noun (or adjective) in \ili{Polish} must involve a so-called \isi{linking vowel} (interfix, intermorph, connective) which links, or separates, the two constituent stems. As a rule, the vowel in question is -{\textit{o}}{-, but there are other possibilities as well which surface in compound nouns incorporating some verbs or numerals in the first position. In the latter case, the intermorph is -}{\textit{i}}{-/-}{\textit{y}}{- or -}{\textit{u}}{-, respectively (see \citealt[458]{GrzegorczykowaPuzynina1999}). Consider the following straightforward examples where the linking element appears in bold type, hyphenated for ease of exposition:}\footnote{{Occasionally I will use hyphens to separate the elements of a compound, but it must be borne in mind that, according to the spelling convention, the majority of \ili{Polish} compounds are written as one word, with no hyphen. Exceptions involve some coordinate structures like} {\textit{Bośnia-Hercegowina}} {‘Bosnia-Herzegovina’ or} {\textit{czarno-biały}} {‘black and white’. Another boundary symbol, a raised dot, is used in some lists of examples to indicate the inflectional endings of words.}}

\ea\label{ex:szymanek:2} 
\begin{tabularx}{\linewidth}[t]{lllll}
Stem 1    &&    Stem 2  &&       Compound N\\
{gwiazd}·a ‘star’ & +    &  zbiór ‘collection’  &> &gwiazd-{\textbf{o}}-zbiór\\ &&&&‘constellation’\\
{siark}·a ‘sulphur’ & + & wodór ‘hydrogen’ & >& siark-{\textbf{o}}-wodór \\
&&&&‘hydrogen sulphide’\\
{star}·y ‘old’   & + & druk ‘print, n.’   & > &star-{\textbf{o}}-druk \\
&&&&‘antique book’\\
{żyw}·y ‘live’  &  + & płot ‘fence’   & >& żyw-{\textbf{o}}-płot \\
&&&&‘hedge’\\
{łam-a}·ć ‘break’  &  + & strajk ‘strike’   & > & łam-{\textbf{i}}-strajk \\
&&&&‘strike-breaker’\\
{mocz-y}·ć ‘soak, v.’ & + & mord·a ‘mug, kisser’ & >& mocz-{\textbf{y}}{-mord}·a\\
&&&& ‘heavy drinker’\\
{dw}·a ‘two’   & + & głos ‘voice’   & >& dw-{\textbf{u}}-głos\\ &&&&‘dialogue’\footnote{{The intermorph -}{\textit{u}}{- is heavily restricted in its distribution and it mainly appears after the numerals} {\textit{dwa}} {‘two’ (}{\textit{dwudźwięk}} {‘double note’) as well as} {\textit{sto}} {‘one hundred’ (}{\textit{stulecie}} {‘century’; exception:} {\textit{stonoga}} {‘centipede’).}}\\
{dw}·a ‘two’   & + & tygodnik ‘weekly’ & > &dw-{\textbf{u}}-tygodnik \\
&&&&‘biweekly’\\
\end{tabularx}
\z 

Prosodically, the compounds are distinguished from phrases by the fact that they receive a single stress on the penultimate syllable (in accordance with the regular pattern of word stress in \ili{Polish}). Thus, for instance, STA•ry•DRUK ‘old print’ (phrase) vs. sta•RO•druk ‘antique book’ (compound).


{Morphologically, the typical presence of the interfix (usually -}{\textit{o}}{-) does not exhaust the range of formal complications. In fact, there may be no interfix at all, in certain types of compounds. In some cases, the lack of an interfix seems to be lexically determined. For instance, most combinations involving the noun} {\textit{mistrz}} {‘master’ as their head have no \isi{linking vowel} (e.g.} {\textit{balet-mistrz}} {‘ballet master’,} {\textit{kapel-mistrz}} {‘bandmaster’,} {\textit{zegar-mistrz}} {‘clockmaker’; but} {\textit{tor-o-mistrz}} {‘railway specialist’,} {\textit{organ-o-mistrz}} {/} {\textit{organ-mistrz}} {‘organ specialist’). In other cases, the omission of the intermorph seems to be due to the \isi{phonological} characteristics of the input forms: if the final segment of the first constituent and/or the initial segment of the second constituent is a sonorant, the combination is likely to be realized without any intervening connective (e.g.} {\textit{pół-noc}} {‘midnight’,} {\textit{trój-kąt}} {‘triangle’,} {\textit{ćwierć-nuta} }{‘quarter note, crotchet’,} {\textit{noc-leg}} {‘lodging, accommodation’,} {\textit{hulaj-noga}} {‘scooter’ (see \citealt[68]{Kurzowa1976}).}


{Another feature that blurs the picture is the frequent occurrence of co-for\-ma\-tives, i.e. morphological elements which, side by side with the interfix itself, contribute to the structure of a given compound. Thus, for instance, fairly common are compound nouns of the following structure: STEM1+interfix+STEM2+\isi{suffix}, i.e. there is both an interfix and a \isi{suffix} which jointly function as exponents of the category (hence the \ili{Polish} traditional term:} {\textit{formacje interfiksalno-sufiksalne}}{). Consequently,} {\textit{nos-o-roż-ec}} {‘rhinoceros’ incorporates the input forms} {\textit{nos}} {‘nose’ and} {\textit{róg}} {‘horn’ (with stem-final palatalization), followed by the obligatory noun-forming \isi{suffix} -}{\textit{ec}} {(cf. *}{\textit{nos-o-róg}}{). The compound is then structurally analogous to its counterparts in \ili{Czech} and \ili{Slovak} (}{\textit{nosorožec}}{), while in \ili{Russian} the equivalent is simply} {\textit{nosorog}}{, with no \isi{suffix}. Consider a few more \ili{Polish} examples:}

\ea\label{ex:szymanek:3}
\begin{tabular}[t]{lll}
Stem 1    &    Stem 2  &       Compound N\\
{dług}·i ‘long’ & dystans ‘distance’  &dług-o-dystans-owiec \\
&&‘long-distance runner’\\
{obc}·y ‘foreign’ &  kraj ‘country’&  obc-o-kraj-owiec \\
&&‘foreigner’\\
{drug}·a ‘second’ &  klas·a ‘form’  &  drug-o-klas-ist·a \\
&&‘second-form pupil’\\
{prac}·a ‘job’&  daw-a·ć ‘give’  &prac-o-daw-c·a \\
&&‘employer’\\
{gryź}·ć ‘bite’  &piór·o ‘pen’  &  gryz-i-piór-ek \\
&&‘pen-pusher’\\
\end{tabular}
\z 

{It may be seen that each of the compounds on the list ends in a \isi{suffix}. The suffixes -}{\textit{ec}}{, -}{\textit{owiec}}{, -}{\textit{ist}}·{\textit{a}}{, -}{\textit{c}}·{\textit{a}}{, and -}{\textit{ek}} {are quite common in this function, so that they may be said to do some of the formative work, as far as compounding is concerned, together with the \isi{linking vowel}.}

\newpage
Various other \ili{Polish} compounds end in a \isi{suffix}, too, which has a fundamentally different status though, since it is inflectional. However, as we shall see, it may also have an important role to play, {from the point of view of word-formation. Incidentally, it will be noticed that the examples of compounds given so far are all masculine nouns, which typically have no overt inflectional ending in the \isi{nominative} sg. (thus e.g.} {\textit{gwiazdozbiór}}{\textit·}{ø,} {\textit{nosorożec}}·{ø). Here the gender of the whole combination is inherited from gender specification on the head (in case it is nominal). Thus} {\textit{gwiazdozbiór}} {is masculine because} {\textit{zbiór}} {is masculine, etc. Yet, in quite a few compounds there is a gender-class shift, for instance from feminine to neuter or masculine, as in the following examples:}

\ea\label{ex:szymanek:4} 
\begin{tabularx}{\linewidth}[t]{lll}
Stem 1    &  Stem 2     & Compound N\\
wod·a ‘water’  &  głow·a ‘head’ & wod-o-głowi·e\\ 
& [+feminine]   & ‘hydrocephalus’\\ &&[+neuter]\\
płask·a ‘flat’   & stop·a ‘foot’ & płask-o-stopi·e\\     
& [+feminine]   & ‘flat foot’ \\
&&  [+neuter]\\
czarn·a ‘black’  &  ziemi·a ‘earth’ & czarn-o-ziem·ø\\ 
& [+feminine]    &  ‘black earth’\\
&& [+masculine]\\
\end{tabularx}
\z 

Thus, the compound status of \textit{wodogłowie} (rather than *\textit{wodogłowa}) is signalled by two things: first, the presence of the usual connective -\textit{o}- and, secondly, the gender-class modification, which results in a distinct paradigm of declension (cf. a few forms in the singular: {\textit{głow}}{\textit·}{\textit{a}} \textsc{nom}, {\textit{głow}}{\textit·}{\textit{y}} \textsc{gen}, {\textit{głowi}}{\textit·}{\textit{e}} \textsc{dat} vs. {\textit{wodogłowi}}{\textit·}{\textit{e}} \textsc{nom}, {\textit{wodogłowi}}{\textit·}{\textit{a}} \textsc{gen}, {\textit{wodogłowi}}{\textit·}{\textit{u}} \textsc{dat}, etc.). Thirdly, in fact, one could mention the characteristic palatalization of the stem-final consonant in the [+neuter] compounds above (throughout the paradigm). Due to this effect, the paradigmatic shift may be looked upon as a significant co-formative which, together with the intermorph -{\textit{o}}{-, defines the structure of the compound in question (hence the \ili{Polish} term:} {\textit{formacje interfiksalno-paradygmatyczne}}{). In fact, the shift of paradigm need not result in gender modification; for instance, the \ili{Slovak} noun} {\textit{slov}}{\textit·}{\textit{o}} {‘word’ and the compound} {\textit{tvar-o-slovi}}{\textit·}{\textit{e}} {‘morphology’ are of the same gender, [+neuter], but their respective declensional paradigms are distinct. The same property is illustrated by the \ili{Polish} compound} {\textit{pust-o-słowi}}{\textit·}{\textit{e}} {‘verbosity’ [+neuter] <} {\textit{pust}}{\textit·}{\textit{y}} {‘empty’ +} {\textit{słow}}{\textit·}{\textit{o}} {‘word’ [+neuter].}

{On some accounts, this formal type is extended to cover also masculine compounds which have a verbal root as their second element, with a zero \isi{marker} of the nom. sg. For example: Polish} {\textit{ręk-o-pis}}{\textit·}{ø ‘manuscript’ <} {\textit{ręk}}{\textit·}{\textit{a}} {‘hand’ +} {\textit{pis}}{(-}{\textit{a}}{\textit·}{\textit{ć}}{) ‘write’; likewise Russian} {\textit{rukopis’}}{, \ili{Slovak} and Czech} {\textit{rukopis}}{. Further \ili{Polish} examples are given below:}

\ea \label{ex:szymanek:5}
\begin{tabularx}{\linewidth}[t]{QQQ}
Stem 1   &   Stem 2    &  Compound N\\
korek ‘cork’ &   ciąg(-ną·ć) ‘pull’ & kork-o-ciąg·ø ‘cork-screw’\\
śrub·a ‘screw, n.’ & kręc(-i·ć) ‘twist’ & śrub-o-kręt·ø ‘screwdriver’\\
paliw·o ‘fuel’  &  mierz(-y·ć) ‘measure’&  paliw-o-mierz·ø \\
&&‘fuel indicator’\\
piorun ‘lightning’ & chron(-i·ć) ‘protect’ & piorun-o-chron·ø ‘lightning conductor’\\
drog·a ‘road’  &  wskaz(-a·ć) ‘indicate’ & drog-o-wskaz·ø ‘signpost’\\
długo ‘long, adv.’  &pis(-a·ć) ‘write’  &  długo-pis·ø \\
&&‘ballpoint pen’\footnote{{Since adverbs do not inflect, the -}{\textit{o}} {vowel in} {\textit{długo-pis}}{, etc. may be interpreted not as an intermorph but rather as an integral element of the input form, at least in those cases where an adverb in -}{\textit{o}} {exists.}}\\
\end{tabularx}
\z 


Taking into account the syntactic category of the input forms which participate in the coining of compound nouns in \ili{Polish}, one needs to point out that, evidently, not all theoretically possible combinations are actually attested. To generalize, one can say for instance that only noun and verb stems may appear as second-position (final) constituents (see below). Alternatively, the verbal stems in question may be interpreted as (potential) nouns, too – products of verb-to-noun conversion. Incidentally, it is enough to distinguish between the first and second constituent, since nominal compounds in \ili{Polish} hardly ever contain more than two elements (in obvious contradistinction to, for example, \ili{English} compounds). In particular, \isi{recursion}, which is perhaps evidenced by certain types of compound adjectives in \ili{Polish}, is not really corroborated by the facts of N+N combination. To sum up, we list below the major syntactic types of compound nouns, with examples involving an intermorph only:

\ea\label{szymanek:6} 
\begin{tabularx}{\linewidth}[t]{llQ}
Stem 1 &   Stem 2 &   Example\\
  N   & N &   ocz-o-dół ‘eye socket’ \newline
            (< oko ‘eye’ + dół ‘pit’)\\
  V  &  N  &  łam-i-strajk ‘strike breaker’ \newline
            (< łamać ‘break’ + strajk ‘strike’)\\
  A &   N &   ostr-o-słup ‘pyramid’ \newline
 (< ostry ‘sharp’ + słup ‘pillar’)\\
  Num  &  N  &  dw-u-głos ‘dialogue’ \newline
            (< dwa ‘two’ + głos ‘voice’)\\
  N  &  V  &  wod-o-ciąg ‘waterworks’ \newline
 (< woda ‘water’ + ciagnąć ‘pull, draw’)\\
  Adv  &  V  &  szybk-o-war ‘pressure cooker’ \newline
 (< szybko ‘fast’ + warzyć ‘cook’)\\
  Pron &   V &   sam-o-lub ‘egoist’ \newline
            (< sam ‘oneself’ + lubić ‘to like’)\\
  Num  &  V  &  pierw-o-kup ‘pre-emption’ \newline
 (< pierwszy ‘first’ + kupić ‘buy’)\\
 \end{tabularx}
\z 


However, as has been pointed out, the intermorph (interfix) need not be the only exponent of the compounding operation. It may co-occur with a derivational \isi{suffix}, as a co-formative. Hence we get the following distributional pattern, illustrated below with compounds involving a noun in the head position (‘plus’ means presence and ‘minus’ means absence of an \isi{affix}):

\ea\label{ex:szymanek:7} 
\begin{tabularx}{\linewidth}[t]{llQ}
Interfix  &  Suffix   & Example\\
  +   & +   & nos-o-roż-ec ‘rhinoceros’\newline
 {(< nos ‘nose’ + róg ‘horn’)}\\
  +   & –  &  krwi-o-mocz ‘haematuria’\newline
            (< krew ‘blood’ + mocz ‘urine’)\\
  –  &  +  &  pół-głów-ek ‘halfwit’\newline
            (< pół ‘half’ + głowa ‘head’)\\
 –  &  –  &  balet-mistrz ‘ballet master’\newline
            (< balet ‘ballet’ + mistrz ‘master’)\\
            \end{tabularx}
\z 

{As may be seen, the full range of theoretically available options is actually attested (although with different degrees of productivity). A complete formal classification would have to superimpose yet another feature, namely the presence or absence of the paradigmatic \isi{marker}, often appearing in place of an overt \isi{suffix}. Thus, for instance,} {\textit{nos-o-roż-ec}} {contains the \isi{suffix} -}{\textit{ec}} {while, for example,} {\textit{głow-o-nóg}} {‘cephalopod’ has none; in the latter, the compounding operation is manifested by a paradigmatic (gender) shift: from [+feminine] (}{\textit{nog}}{\textit·}{\textit{a}} {‘leg’) to [+masculine].}

{When analysed from the functional perspective, the \ili{Polish} noun compounds present themselves as a highly diversified class. First, there are a number of examples of co-ordinate structures like:} {\textit{klubokawiarnia}} {‘a café that hosts cultural events’ (<} {\textit{klub}} {‘club’ +} {\textit{kawiarnia}} {‘café’),} {\textit{kursokonferencja}} {‘training conference’ (<} {\textit{kurs}} {‘course, training’ +} {\textit{konferencja}} {‘conference’),} {\textit{marszobieg}} {‘run/walk’ (<} {\textit{marsz}} {‘walk’ + bieg ‘run’),} {\textit{chłoporobotnik}} {‘a peasant farmer who works in a factory’ (<} {\textit{chłop}} {‘peasant’ +} {\textit{robotnik}} {‘manual worker’), etc. It may be argued that a combination of the type in question is semantically headed by both constituents and hence their order is potentially reversible (cf. ?}{\textit{kawiarnioklub}}{, ?}{\textit{biegomarsz}}{; see \citealt[59]{Kurzowa1976}). A formal variant within this class are juxtapositions like} {\textit{klub-kawiarnia}} {‘a café that hosts cultural events’ (cf.} {\textit{klobokawiarnia}} {above) or} {\textit{trawler-przetwórnia}} {‘factory trawler’. As may be seen, there is no intermorph here. Instead, both constituent nouns are hyphenated and they inflect.}\footnote{{A mixed pattern, formally speaking, is evidenced by co-ordinate structures like} {\textit{chłodziarko-zamrażarka}} {‘cooler-freezer’ where the first constituent is followed by the intermorph -}{\textit{o}}{- so it does not inflect; yet the hyphen is obligatory here.}} {The type is then formally similar to so-called copulative (dvandva) juxtapositions, evidenced by proper names like} {\textit{Bośnia-Hercegowina}} {‘Bosnia-Herzegovina’ or} {\textit{Alzacja-Lotaryngia}} {‘Alsace-Lorraine’. Here, again, both constituents may inflect (cf.} {\textit{Bośni-Hercegowiny}}{, gen.,} {\textit{Bośnią-Hercegowiną}}{, instr., etc.). Yet, in terms of headedness, the situation seems to be different here: neither constituent functions as the head.}


However, the majority of \ili{Polish} N+N or A+N compounds are hierarchically structured and subordinate, with the right-hand constituent functioning as the head. For example: \textit{światłowstręt} ‘photophobia’, \textit{gwiazdozbiór} ‘constellation’, \textit{czarnoziem} ‘black earth’, \textit{drobnoustrój} ‘micro-organism’. All the examples on this list are endocentric, i.e. the compound may be interpreted as a hyponym of its head (thus, for instance, \textit{światłowstręt} ‘photophobia’ means ‘kind of phobia’, etc.).\footnote{Left-headed N + N compounds are truly exceptional (\citealt[461]{GrzegorczykowaPuzynina1999}); cf., however, \textit{nartorolki} ‘grass skis’ when paraphrased as ‘skis with (small) rollers/wheels’. In order to be consistent with the \isi{right-headed} endocentric pattern, the form should rather be: (*) \textit{rolkonarty}.}  Exocentric combinations are also fairly common regardless of whether or not the compound incorporates an overt \isi{suffix}. For instance, \textit{nosorożec} ‘rhinoceros’ and \textit{stawonóg} ‘arthropod’ denote ‘kinds of animals’ although their second constituents make reference to horns or legs, respectively (cf. \textit{róg} ‘horn’, \textit{nog-a} ‘leg’). Other examples of the exocentric type: \textit{trójkąt} ‘triangle’ < \textit{trój}- ‘three’ + \textit{kąt} ‘angle’, \textit{równoległobok} ‘rhomboid’ < \textit{równoległy} ‘parallel’ + \textit{bok} ‘side’, \textit{obcokrajowiec} ‘foreigner’ < \textit{obcy} ‘foreign’ + \textit{kraj} ‘country’. Here the head of the compound is either unexpressed, as in \textit{trój-kąt} ‘(a flat figure with) three angles’ or is vaguely symbolized by the final \isi{suffix}, as in \textit{obc-o-kraj-owiec} ‘a person from a foreign country, foreigner’. According to an alternative interpretation, the latter example might be viewed as endocentric rather than exocentric, assuming that the meaning of ‘person’ is directly encoded by the \isi{suffix} -\textit{owiec}. Structures of the kind just illustrated are also \isi{right-headed} in themselves, since the first two constituents function as a complex, \isi{right-headed}, \isi{modifier} with respect to the implied head of the compound.



However, in exocentric compounds with a verbal element, this element mirrors the head of the corresponding verb phrase, regardless of whether it appears in the first or second position in the compound. This is illustrated with the following examples where the verb stem appears in bold face:

\ea\label{ex:szymanek:8} 
\begin{tabularx}{\linewidth}[t]{QQ}
V + N  &         N + V\\ 
\textbf{łam}-i-strajk ‘strike breaker’ &   list-o-\textbf{nosz} ‘postman’\\
  lit. ‘sb. who breaks a strike’    &   lit. ‘sb. who carries letters’\\
\tablevspace
\textbf{baw}-i-dam-ek ‘ladies’ man’ &   lin-o-\textbf{skocz}-ek ‘tightrope walker’\\
  lit. ‘sb. who amuses/entertains ladies’    & lit. ‘sb. who jumps (on) a tightrope’\\
  \end{tabularx}
\z 

{According to \citet{Nagórko2016}, \isi{left-headed} structures (V + N), “albeit with some exceptions, are considered dated or humorous, cf.} {\textit{gol-i-broda}} {‘barber; lit. shave-beard’[…],} {\textit{najm-i-morda}} {‘legal counsel; lit. hire-mug’. Therefore, the \ili{Polish} language is drifting, undoubtedly because of the foreign influence, towards the \isi{right-headed} type of compounding.”}

{The examples presented so far give the correct impression that the semantic structure of \ili{Polish} nominal compounds is quite diversified and, at times, fairly complex and/or ambiguous. However, due to space limitations, it is hardly possible to give a full-fledged semantic classification of the data under discussion (for details, see \citealt{Kurzowa1976} or \citealt{GrzegorczykowaPuzynina1999}). Suffice it to say that, by and large, the semantic categories that are discernible are reminiscent of those normally established in the context of ordinary (e.g. affixal) \isi{derivation} of \ili{Polish} nouns. Thus, one can identify, for instance, formations that are agentive (}{\textit{listonosz}} {‘postman’,} {\textit{dobroczyńca}} {‘benefactor’), instrumental (}{\textit{gazomierz}} {‘gas meter’), locative (}{\textit{jadłodajnia}} {‘eating place’), resultative (}{\textit{brudnopis}} {‘rough draft’), attributive (}{\textit{lekkoduch}} {‘good-for-nothing’), that denote activities (}{\textit{grzybobranie}} {‘mushroom picking’), states/conditions (}{\textit{płaskostopie}} {‘flat foot’) or inhabitants (}{\textit{Nowozelandczyk}} {‘New Zealander’), etc. For a detailed interpretation of the semantics of \ili{Polish} nominal compounds in terms of thematic relations, see \citet{Sambor1976}.}


{The examples of \ili{Polish} compound nouns given so far are dictionary-attested. Most of them have been in use for quite some time (including quite a few old or obsolete combinations), as they represent the native \ili{Polish} patterns of compound formation. Characteristically, there are a few lexical elements that have been abundantly exploited in native compounds. Consider the following list of attested nouns, each involving the verbal root} {\textit{pis}}{- ‘write’ as the right-hand constituent:} {\textit{brudnopis}} {‘rough draft’ (}{\textit{brudny}} {‘dirty’),} {\textit{czystopis}} {‘fair copy’ (}{\textit{czysty}} {‘clean’),} {\textit{dalekopis}} {‘teleprinter, telex’ (}{\textit{daleki}} {‘far’),} {\textit{cienkopis}} {‘fine felt-tip pen’ (}{\textit{cienki}} {‘thin, fine’),} {\textit{długopis}} {‘ballpoint pen’ (}{\textit{długi}} {‘long’),} {\textit{rękopis}} {‘manuscript’ (}{\textit{ręka}} {‘hand’), etc.}



{However, the past few decades have witnessed the extension of the traditional \ili{Polish} models of compound formation, mainly as a result of foreign influences and massive borrowing, especially from \ili{English}. Two specific patterns, illustrating such recent developments, are worth noting here. Firstly, these are compounds involving initial combining forms and clipped modifiers. For example:}\footnote{{Further examples may be found, for instance, in \citet[94]{Jadacka2001}, \citet{Waszakowa2015}.}}

\ea\label{ex:szymanek:9} 
\begin{tabularx}{\linewidth}[t]{ll}
eko-  &  ekoturystyka ‘eco-tourism’, ekorozwój ‘eco-development’\\
euro-  &  euroregion ‘Euroregion’, euroobligacja ‘Eurobond’\\
mikro-  &   mikromodel, ‘micromodel’, mikroksiążka ‘microbook’\\
pseudo- &   pseudoartysta ‘pseudo-artist’, pseudouczony ‘pseudo-scientist’\\
spec-  &  speckomisja, specustawa ‘special, i.e. extraordinary \\
& committee/law’\\
tele- &   telereportaż ‘TV report’\\
\end{tabularx}
\z 

Compositions of the type just illustrated do not contain the native \isi{linking vowel}. However, the use of such combining forms is facilitated when they happen to end with the vowel -\textit{o}, which is identical with the \ili{Polish} default connective, and hence the type now often gives rise to hybrid combinations (e.g. \textit{mikroksiążka} ‘microbook’).\footnote{It appears that at least some of the combining forms in question have actually acquired the status of prefixes.}


{Secondly, there are N+N compounds which are due to borrowing from \ili{English}; cf.} {\textit{seksbiznes}} {‘sex business’, etc. This has already led to a partial absorption and nativization of the \ili{English} pattern, as well as to its gradual spread (see next section for more examples of this type).}



Despite the new trends and foreign influences, the formation of compounds in \ili{Polish} still preserves much of its original character. The fact is that, generally speaking, compounding in \ili{Polish} is much less productive than in a language like \ili{English}. Besides, quite apart from the question of phrasal compounds, there are a number of structural patterns and peculiarities of \ili{English} compounds that simply do not exist in \ili{Polish} (or they are highly limited). To sum up this section, one can mention just a few such points of difference:

\begin{itemize}
\item No recursiveness (with minor exceptions); moreover – virtually no N+N compounds with more than two constituents; hence:
\item No structural ambiguity (cf. E. \textit{California history teacher})
\item No \isi{modifier} + head reversibility (cf. E. \textit{flower garden} {/} \textit{garden flower}, \textit{radio talk} {/} \textit{talk radio})
\item  No identical-constituent compounds (cf. E. (\textit{my}) \textit{friend friend})
\item No plural modifiers in compounds (cf. E.\textit{parks department} vs. the P. phrase \textit{wydział}\textsc{\textsubscript{nom}} \textit{parków}{\textsc{\textsubscript{gen pl}}}), including phrasal modifiers with co-ordination (cf. E. [[\textit{wines and spirits}] \textit{department}] vs. the P. phrase {\textit{dział}}{\textsc{\textsubscript{nom}}} {\textit{win}}{\textsc{\textsubscript{gen pl}}} {\textit{i spirytualiów}}{\textsc{\textsubscript{gen pl}}}).
\end{itemize}

\section{Why are phrasal compounds virtually unavailable in Polish?}\label{sec:szymanek:3}

As far as \ili{Polish} is concerned, it is hard to give any definitive reasons accounting for the lack of phrasal compounds of the type found in \ili{English}. It is more obvious though why the process of Noun+Noun compounding is less vigorous and productive in \ili{Polish} than in \ili{English}. However, since the phrasal compounds investigated in the \ili{Germanic} (and other) languages are nouns and have nominal heads, a closer examination of the peculiarities and structural restrictions governing the use of N+N composition in \ili{Polish} may explain, albeit indirectly, the unavailability of the special XP+N pattern.\footnote{{On the affinity between N+N compounds and PCs, see \citet{Pafel2015}.}}

\largerpage
The main reason why the class of N+N compounds in \ili{Polish} (and \ili{Slavic} in general) is not so numerous as in \ili{English} is the fact that \ili{Polish} grammar offers, and often imposes, alternative structural options for the combined expression of two nominal concepts. Where \ili{English} frequently has a N+N compound, \ili{Polish} may have (i) a noun phrase with an inflected noun \isi{modifier} (usually in the \isi{genitive}), (ii) a noun phrase incorporating a \isi{prepositional} phrase \isi{modifier}, or (iii) a noun phrase involving a denominal (relational) adjective as a \isi{modifier}, as is illustrated below:

\ea\label{ex:szymanek:10} 
\ea 
{\textit{telephone number}}

 {i.     numer telefon}·{u} 

 {ii.   *numer do telefon}·{u}

 {iii.   *numer telefon-icz-n}·{y}

 \ex 
 {\textit{computer paper}}

 {i.     *papier komputer}·{a}

 {ii.   papier do komputer}·{a}

 {iii.   papier komputer-ow}·{y}

\ex 
{\textit{toothpaste}}

 {i.     *past}·{a zęb}·{ów}


 {ii.   past}·{a do zęb}·{ów}

 {iii.   *past}·{a zęb-ow}·{a}
\z 
\z 

{Evidently, alternative structures are often available, cf.} {\textit{papier do komputera}} {vs.} {\textit{papier komputerowy} }{‘computer paper’. The kind of construction may depend on a variety of factors which need not concern us here. What is important is the fact that the \ili{Polish} expressions just cited are syntactic objects, and that they may involve both \isi{inflection} and \isi{derivation}, but not compounding.}\footnote{{According to some \ili{Polish} authors (see e.g. \citealt[120]{Jadacka2005}), fixed nominal phrases like} {\textit{pasta do zębów}} {‘toothpaste’,} {\textit{drukarka laserowa}} {‘laser printer’, etc. ought to be viewed as a special type of a generally conceived category of compounding: the so-called ‘juxtapositions’ (P.} {\textit{zestawienia}}). } {That is to say, there are no compounds like *}{\textit{komputeropapier}} {or *}{\textit{telefononumer}}{, to parallel the \ili{English} counterparts. On top of this, there may be a suffixal derivative based on the \isi{modifier}; see \citet{Ohnheiser2015} for further details and generalizations concerning these options in various \ili{Slavic} languages; see also \citet{tenHacken2013}.}


Consider additionally the following example where most of the structural options are actually attested, including a regular compound:

\ea\label{szymanek:11} 
{\textit{steamship}}  {(Polish} {\textit{para}} {‘steam’ +} {\textit{statek}} {‘ship’)}

 {i.     *statek par}·{y  (Genitive phrase)}


 {ii.   statek na par}·{ę  (N + PP)}

 {iii.   statek par-ow}·{y  (N + Relational Adjective)}

 {iv.   parowiec    (suffixal derivative; cf. E.} {\textit{steamer}})

 {v.   parostatek  (N-}{\textit{o}}{-N compound; E.} {\textit{steamship}})
\z

\largerpage[2]
The patterns illustrated above may partly explain why the number of dic\-tio\-nary-attested nominal compounds in \ili{Polish} is significantly lower than in \ili{English}. Quite simply, certain functions that are served by compounding in other languages tend to be realized by syntactic, inflectional and/or derivational means in \ili{Polish}. Analogical patterns, though in different proportions, are exploited by other \ili{Slavic} languages as well. 

{Another factor that seems to thwart the generation of phrasal compounds in \ili{Polish} is purely formal and quite general: as a rule, \ili{Polish} nominal compounds may involve only two lexical constituents. Thus, by virtue of this (fairly superficial) restriction alone, composites even remotely comparable to \ili{English} PCs are ruled out, since the number of lexical elements in the \isi{modifier} position of an \ili{English} phrasal compound is usually three or higher, not to mention the head itself. This constraint ties up, of course, with another remarkable characteristics of \ili{Polish} nominal compounding: there is no \isi{recursion}.}\footnote{{There are sporadic counterexamples suggesting that both constraints mentioned here, i.e. ‘no \isi{recursion}’ and ‘only two constituents’, are (rarely) violated, as in the following example often quoted in grammar books:} {\textit{Zwierzoczłekoupiór}} {[zwierz-o-człek-o-upiór] ‘animal-man-ghost’ (title of a novel by the \ili{Polish} writer Tadeusz Konwicki). In contrast to Noun+Noun compounds, limited recursiveness (iteration) is allowed in the case of certain types of compound adjectives in \ili{Polish}; cf.} ({\textit{słownik}}) {\textit{polsko-angielsko-niemiecko}}{(-...) -}{\textit{rosyjski}} {‘a \ili{Polish}-\ili{English}-\ili{German}(-...)-\ili{Russian} (dictionary)’.}} {By contrast, it is a well known feature of the \ili{English} pattern of Noun+Noun compounding that it is recursive. In connection with the particular contrast noted here, one can speculate that, perhaps, there is some linkage here between (the possibility of) \isi{recursion} and phrasal compounding, in a given language – in the sense that \isi{recursion} might be a precondition for phrasal compounding.}

{Yet another remarkable factor is the fact that \ili{Polish} does not offer any instances of literal borrowings of phrasal compounds, from languages like \ili{English} or \ili{German}, i.e. compositions which preserve the original lexical make up as well as the structural configuration of a PC in the source language. This seems to suggest that the characteristic structure of a PC is completely alien, from the viewpoint of \ili{Polish} grammar and, accordingly, any foreign instances of PCs that need to be nativized} {or translated into \ili{Polish} must be remodelled and encoded as prototypical phrasal constructions. This point may be illustrated with the following \ili{German} examples adapted from \citet[250]{Meibauer2007} and juxtaposed with corresponding expressions in \ili{Polish}:}\footnote{{The \ili{English} glosses attached to the original \ili{German} examples are not repeated after the \ili{Polish} near-equivalents since they apply, by and large. However, the present-tense (1}{\textsuperscript{st}} {person) form of the verb ‘to buy’, i.e. G.} {\textit{kaufe}} {has been replaced by the future perfective form} {\textit{kupię}} {in the \ili{Polish} version as it seems more plausible in the given context. Besides, the diminutive G. form} {\textit{Kärtchen}} {appears as P.} {\textit{karteczka}}{, i.e. (formally) a double diminutive.}}


\ea%12
    \label{ex:szymanek:12} 
         \ili{German} \citep[250]{Meibauer2007}
\ea 
Autokärtchen

 {car card.}{\textsc{dim}}


  P. autokarteczka


\ex   Kaufkärtchen

 {buy}{\textsubscript{V/N} }{card.}{\textsc{dim}}


  P. *kupkarteczka


\ex  Kaufe-Ihr-Auto-Kärtchen
{buy.}\textsc{1.ps.sg.}{-your-car card.}{\textsc{dim}}

  P. *kupię-Twoje-auto-karteczka


\ex  Kärtchen „Kaufe Ihr Auto”

{card.}{\textsc{dim}} {buy.}\textsc{1.ps.sg.} {your car”}

 {P. karteczka „kupię Twoje auto”}


\ex Kärtchen mit den Aufschrift „Kaufe Ihr Auto”

card.\textsc{dim} with the writing buy.\textsc{1.ps.sg.} your car”

 {P. karteczka z napisem „kupię Twoje auto”}


\ex  Kärtchen, auf denen „Kaufe Ihr Auto” steht

card.\textsc{dim} on which „buy.\textsc{1.ps.sg.} your car” is written

 {P. karteczka, na której jest napisane „kupię Twoje auto”}
\z
\z

{\citet[250]{Meibauer2007} presents the \ili{German} examples in this list as alternative modes of expression or “stylistic alternatives, some morphological, some syntactic”; cf., respectively, (12a–c) i.e. “complex words”, as opposed to (12d–f), i.e. “syntactic constructions”. The main focus is on case (12c), i.e. “an ad hoc phrasal compound with a CP as non-head” \citep[249]{Meibauer2007}. Now, from the viewpoint of \ili{Polish} morphology, this case (12c) is also significant, since it clearly demonstrates that a word-by-word rendering of the \ili{German} PC is ruled out (as a matter of principle); cf. *}{\textit{kupię-Twoje-auto-karteczka}}{. The compound structure evidenced in (12b), i.e. a composition involving a verbal/nominal root followed by a (diminutive) noun is also rather unlikely in \ili{Polish}, at least in this particular context and lexical configuration. As may be seen, what is freely available, both in \ili{German} and in \ili{Polish}, are various syntactic (periphrastic) modes of expression (cf. 12d–f). However, as far as \ili{Polish} is concerned, the syntactic options actually emerge as the only viable choice, given the fact that – according to \citet[250]{Meibauer2007} – a compound like G.} {\textit{Autokärtchen}} {(cf. P.} {\textit{autokarteczka}}{) is “underdetermined”, in comparison to} {\textit{Kaufe-Ihr-Auto-Kärtchen}}{. “The phrasal compound [}{\textit{Kaufe-Ihr-Auto-Kärtchen}}{] is as explicit as the syntactic construction [}{\textit{Kärtchen „Kaufe Ihr Auto”}}{], the main difference being that [the former] has a right-hand morphological head, whereas [the latter] shows a left-hand syntactic head.” \citep[250]{Meibauer2007}. Indeed, when we compare various \ili{German} or \ili{English} PCs and their renderings in \ili{Polish}, the superficially visible difference is the reversal of the} {linear order of the major constituents; cf. for instance E.} {\textit{a “work or starve” philosophy}} {vs. P.} {\textit{filozofia “pracuj lub głoduj”}}{. It must be emphasized, though, that – underlyingly – these locutions differ in \isi{grammatical} status: the \ili{English} expressions are compounds, i.e. lexical objects, while the \ili{Polish} ones are syntactic constructions.}


{As has been pointed out, literal borrowings of \ili{English} or \ili{German} PCs are hardly available in \ili{Polish}. By contrast, the \ili{English} type of ordinary Noun+Noun compounding (with a non-phrasal \isi{modifier}) has been partially assimilated in present-day \ili{Polish}, even though this type is not consistent with the default structure of a \ili{Polish} nominal composition, where the \isi{linking vowel} -}{\textit{o}}{- should appear between two lexical constituents.}\footnote{{However, as has been mentioned (cf. \sectref{sec:szymanek:2}), a precedent already exists, in \ili{Polish} morphology, for interfixless N+N compounds: there is the weak and now rather obsolete pattern of endocentric compositions, typically with the noun} {\textit{mistrz}} {‘master’ in head position, like in the following examples:} {\textit{baletmistrz}} {‘ballet master‘,} {\textit{kapelmistrz}} {‘bandmaster’ (cf. G.} {\textit{Kapellmeister}}), {\textit{chórmistrz}} {‘choirmaster’,} {\textit{zegarmistrz}} {‘clock maker, watch maker’, etc. However, this pattern is formally inconsistent: some other attested compounds with -}{\textit{mistrz}} {do show up the regular interfix -}{\textit{o}}{-; e.g.} {\textit{ogniomistrz}} {‘artillery sergeant’ (}{\textit{ogień}} {‘fire’),} {\textit{organomistrz}} {/} {\textit{organmistrz}} {‘organ master’ (cf. \citealt{Kurzowa1976} [2007]: 458).}} {Consider a few examples of recent neologisms and loan adaptations:}\footnote{{Such forms are more common in \ili{Russian}; cf. \citet{Ohnheiser2015}.}}



\ea%13
    \label{ex:szymanek:13} 
  biznesplan ‘business project/plan’

  seksbiznes ‘sex business’

  seksturystyka ‘sex tourism’

  dres kod ‘dress code’

  pomoc linia ‘help line’


{Duda pomoc ‘free-of-charge legal counselling offered, to ordinary people, by the presidential candidate Andrzej Duda and his staff’ (lit.} {\textit{Duda}} {<surname> +} {\textit{pomoc}} {‘help’)}
\z

{These examples clearly suggest that the \ili{English} pattern of N+N compounding is gaining ground in \ili{Polish}. According to \citet[93]{Jadacka2001}, \ili{Polish} neologistic compounds without an interfix (i.e. a linking element) have become increasingly common in the past few decades, even though a number of relevant examples are not yet dictionary-attested (cf.} {\textit{Duda pomoc}}{, dated 2015). Also, the occasional presence of native nouns in such combinations (cf.} {\textit{pomoc linia}}{) seems to suggest that this is now, indeed, a case of \isi{pattern borrowing}.}

{Significantly, the spread of the foreign interfixless Noun+Noun pattern of compounding in \ili{Polish} has not gone as far as in some other \ili{Slavic} languages, for instance in \ili{Russian} and \ili{Bulgarian}. According to \citet{Bagasheva2015}, \ili{Bulgarian} [N N] constructions “instantiate the grammaticalization of a new compound type in the language”. The \ili{Bulgarian} pattern in question extends to cover also cases where the prehead constituent of a compound is an initialism (just like in \ili{English}); cf.} {\textit{{ФБР агент}}} {‘FBI agent’,} {\textit{{ДНК}}} {\textit{{фактор}}} {‘DNA factor’. In \ili{Polish}, by contrast, such loan compounds are ruled out: the order of both constituents must be reversed so that the construction emerges as a phrase (here with an implicit (unmarked) \isi{genitive} case on the modifying initialism); cf. P. *}{\textit{FBI agent}} {vs} {\textit{agent FBI}}{, *}{\textit{DNA czynnik}} {vs.} {\textit{czynnik DNA}} {‘DNA factor’. More importantly though, only in \ili{Bulgarian} can we find examples of phrasal compounds modelled on the structure of \ili{English} PCs (see \sectref{sec:szymanek:5} for examples).}

{To sum up, as we have seen, phrasal compounds of the type found in \ili{English} or \ili{German} are impossible in \ili{Polish}, no matter whether they are actual borrowings or forced word-by-word translations, and this is regardless of whether a particular PC in the source language incorporates a} {phrasal or sentential prehead constituent, or just an initialism.}\footnote{{It does not matter as well whether a given PC is quotative or \isi{non-quotative} in character; cf. \citet{Pafel2015} on the contrast between quotative and \isi{non-quotative} PCs in \ili{German} and \ili{English}.}} {Incidentally, the behaviour of initialisms (and acronyms) in such constructions seems to offer a useful diagnostic here since – on the one hand – they are “lexical” because of their nounlike properties but – on the other hand – they are “phrasal” since they stand for fully fledged phrases (e.g.} {\textit{the FBI}} {=} {\textit{Federal Bureau of Investigation}}{, etc.). By using both phrasal compounds and initialisms/acronyms one can achieve greater text condensation. In \ili{English}, we can actually use a construction involving two abbreviations, in the \isi{modifier} and head positions; cf.} {\textit{the SNP MPs}} {‘the Scottish National Party Members of Parliament’.}\footnote{{Source of the example (spoken language): BBC Radio 4,} {\textit{Friday Night Comedy}}{, May 15th, 2015.}} {Again, no structure of this sort is possible in \ili{Polish}.}


Finally, it may be of interest to note that – even though phrasal prehead constituents are impossible in \ili{Polish} compounds – the occurrence of phrasal bases in affixal \isi{derivation} is completely unproblematic. In fact, according to the literature on \ili{Polish} morphology, there are several distinct patterns of de-phrasal derivatives (see next section).

\section{Generalizing the concept of “phrasal compound”: some relevant types of multi-word expressions in Polish}\label{sec:szymanek:4}

{As has been pointed out, phrasal compounds of the kind found in \ili{English} do not seem to exist in \ili{Polish}. In particular, clausal and sentential modifiers appear to be completely ruled out in \ili{Polish} compound nouns. But even phrases such as NP are rather unlikely in the prehead position. I have not been able to identify any convincing examples of the latter type of structure. Consider, however, the following recent example from the Internet,}\footnote{{Piotr Cywiński,} {Szczucie na Komorowskiego i wściekła sfora Dudy, czyli jak} {\textit{Gazeta Wyborcza}} {plewi chamstwo i pogardę w „szczujniach”,} \href{http://www.wPolityce.pl/}{{{www.wPolityce.pl}}} {, 4.03.2015. The phrase in question appeared in the following context:} {\textit{Bo z pewnością znajdą się złośliwcy, którzy spytają, a gdzie był rzecznik bon-tonu, savoir vivre’u, niestrudzony bojownik dobrych manier, gdy np. elektryk-eks-prezydent-noblista pokojowy Lech Wałęsa mówił o urzędującej wówczas głowie państwa Polskiego: „mamy durnia za prezydenta”?}} } {which, characteristically, involves a multiword complex \isi{modifier} with hyphenated constituents:}

\ea\label{ex:szymanek:14a}{\textit{elektryk-eks-prezydent-noblista pokojowy Lech Wałęsa}}
 {‘electrician-ex-president-Nobel-peace(-prize laureate) Lech Wałęsa’}
\z 

{Superficially, i.e. orthographically, this expression may look deceptively similar to the category of phrasal compounds that we are interested in; cf. the multiple use of hyphens, conjoining the lexical items in the prehead position (which is a characteristic feature of many \ili{English} phrasal compounds). However, the multiple use of hyphens certainly looks marked, odd, and eye-catching,} {from the viewpoint of the \ili{Polish} orthographic convention. Besides, multiple hyphens are neither necessary nor sufficient as a formal diagnostic for identifying PCs, even in \ili{English} (cf., for instance, \citealt[323]{Trips2012}). Probably, the motivation for the multiple use of hyphens, in the above example, was to achieve greater expressiveness.}\footnote{{On the expressive nature of phrasal compounds, see e.g. \citet{Meibauer2013}, \citet{Trips2014}.}} {But, more importantly, it is doubtful if the expression under discussion is a phrasal compound by strictly \isi{grammatical} criteria. It is not a \isi{determinative} compound because the first element does not determine the second element semantically (cf. \citealt[7]{TripsKornfilt2015typology}: “PCs are always \isi{determinative} compounds”, according to \citealt{Meibauer2003}). It is not a compound, to begin with. It rather appears that this is an instance of a non-restrictive appositional construction, using} {unconventional orthography (which makes a difference only in written language anyway); cf. the more usual spelling, with commas instead of hyphens:}

\ea\label{ex:szymanek:14b}{\textit{elektryk, eks(-)prezydent, noblista pokojowy Lech Wałęsa}} {‘id.’}
\z 

{If the notion of phrasal compounding is relaxed somewhat, so that the whole compound may correspond to a phrase, and not just its pre-head constituent, then certain examples in \ili{Polish} may appear relevant. Consider, first, the structure of the noun} {\textit{niezapominajka}} {‘forget-me-not’:}

\ea\label{ex:szymanek:15} 
\textit{niezapominajka} ‘forget-me-not’\\
\gll {nie zapominaj} -k  -a\\
{not forget.}\textsc{imp}.\textsc{ipfv} suff. suff.\textsc{infl}\\
 \z

{However, forms like} {\textit{niezapominajka}} {are lexicalized and extremely rare in \ili{Polish}.}

{It should be noted as well that the \ili{English} noun} {\textit{forget-me-not}} {is explicitly assigned to the category of \ili{English} ‘phrase compounds’ by \citet[206-207]{Bauer1983}. To be more precise, the noun in question is given as an example of “exocentric phrase compounds”, together with other plant names such as} {\textit{love-in-a-mist}}{, and} {\textit{love-lies-bleeding}}{. According to Bauer, apart from exocentric phrase compounds, there are also dvandva phrase compounds (e.g.} {\textit{whisky-and-soda}}{) and, finally, endocentric phrase compounds, including \isi{right-headed} structures with a phrase or sentence in the pre-head position (also \isi{left-headed} structures like} {\textit{son-in-law}}{). Evidently, the group of endocentric \isi{right-headed} expressions (=phrasal compounds proper) is treated by Bauer as a subclass within his broad category of ‘phrase compounds’.}

{If we apply this broad interpretation (in terms of ‘phrase compounds’) to the \ili{Polish} data, then it may be argued that there are, perhaps, some other relevant patterns and examples, apart from the aforementioned noun} {\textit{niezapominajka}}{. For instance, there is the unproductive pattern of so-called ‘solid compounds’ (P.} {\textit{zrosty}}{), which are directly motivated by a syntactic phrase so that they appear without an interfix. Instead, the first constituent ends with the inflectional ending required by the structure of the original phrase (see \citealt[195]{Nagórko1998}, \citeyear{Nagórko2016}; \citealt[471]{Szymanek2009}).}

\newpage\ea\label{ex:szymanek:16}
‘Solid compounds’ (P. \textit{zrosty}) motivated by syntactic phrases (no interfix)\\
\begin{tabularx}{.9\textwidth}{p{5cm}lp{5cm}}
{\textit{Phrase}} & > & {\textit{Compound}}\\
\parbox{5cm}{
\gll ok·a mgnieni·e   \\
{eye.}\textsc{gen} {blink}\\

(also: mgnienie oka – phrase)
}
&& okamgnieni·e \newline ‘a blink of an eye’\newline\\
\tablevspace
\parbox{5cm}{
\gll czc·i godn·y (Adj)\\
{esteem.}\textsc{gen} {worthy}\\

(also: godny czci – phrase)
}
&& czcigodn·y \newline‘esteemed, honourable’\newline\\
\end{tabularx}
\z

There is also a more numerous group of compound nouns made up of an adverb followed by a verbal root:

\ea\label{ex:szymanek:17}
‘Phrase compounds’ of the type [[Adverb + Verb]{\textsubscript{VP}} {(suff)]}{\textsubscript{N}}\\

\begin{tabular}{llll}
 Adverb & Verb &>& Compound N\\
 cienko &   pisać  & &   cienkopis ø  ‘fine felt-tip pen’\\
  ‘thinly’ &   ‘write’&&\\
\tablevspace
  cicho  &  dawać (w {\textasciitilde} j) &&   cichodajka  ‘woman on the game, hooker’\\
  ‘quietly’ & ‘give’    &&     (suff. -k)\\
  \end{tabular}
\z 

{The nouns given above may be regarded as ‘phrase compounds’ because they mirror a well-formed syntactic constituent, i.e. a type of VP (minus the thematic and inflectional characteristics on the verb). Importantly, the second element is not an attested deverbal noun (cf. *}{\textit{pis}}{, *}{\textit{dajka}}{), unlike in some other, similar forms (e.g.} {\textit{dalekowidz}} {‘long-sighted person’,} {\textit{jasnowidz}} {‘clairvoyant’, etc.).} 

{To take another example, there is a class of (mostly expressive, often obsolete) exocentric compounds, whose internal structure reflects that of a VP they appear to be based on, where the VP is of the type [Verb+Noun]:}\footnote{{The verb governs the accusative case on the object noun; hence the ending -}{\textit{ę}} {in the phrasal input, as opposed to the \isi{nominative} (-}{\textit{a}}{) in the compound. For more examples and discussion concerning this pattern, see Kurzowa (1976 [2007]: 440).}}\newpage

\ea\label{ex:szymanek:18} 
‘Phrase compounds’ of the type [[Verb + Noun]{\textsubscript{VP}}]{\textsubscript{N}}\\

\begin{tabular}{llll}
 Verb & Noun &> &Compound N\\
 czyścić  &   but·y   &&    czyścibut   ‘shoeshine (boy)’\\
 ‘clean’  &  ‘shoe.\textsc{acc-pl}’\\
 \tablevspace
 moczyć  &  mord·ę  &&    moczymord·a  ‘heavy drinker’\\
 ‘soak’   & ‘mug, kisser.\textsc{acc}’\\
 \tablevspace
 męczyć   & dusz·ę   &&   męczydusz·a  ‘bore, nudnik’\\
 ‘torment’&  ‘soul.\textsc{acc}’\\
 \end{tabular}
\z 

{However, it would be a risky move if we attempted to generalize, or extend any further, the notion of ‘phrase compounds’. Because then we might soon find ourselves in a point of no return, i.e. where, for instance, synthetic compounds would be treated as being fundamentally phrasal in nature, just because they correspond to a licit phrase type in \isi{syntax} (V NP); cf. P.} {\textit{kredytobiorca}} {‘borrower, lit. credit-taker’,} {\textit{kredytodawca}} {‘lender, lit. credit-giver’, etc. In other words, the generalization of the concept in question must have its limits.}

{It is a remarkable feature of the word-formation system in \ili{Polish} (and other \ili{Slavic} languages) that there are several other types of “multi-word expressions” which are based on (or which involve) phrasal constituents (see e.g. \citealt{Martincová2015,Ohnheiser2015}). Traditionally, the following phenomena have been interpreted, among others, as giving rise to de-phrasal lexical units:}\footnote{{See e.g. \citet[237]{Szymanek2010} for more examples and discussion of derivations based on phrases in \ili{Polish}.}}

\subsection*{Derived nouns and adjectives based on phrases} 

Consider, respectively, the examples in (20) and (21):

\ea\label{ex:szymanek:19} 
{\textit{Prepositional Phrase (P + Noun}}{\textit{\textsubscript{Infl}}}{\textit{)  >  De-phrasal Noun}}\\

\begin{tabular}{llll}
bez ‘without’ &  robot·y ‘work.{\textsc{gen}}’ &&   bezroboci·e ‘unemployment’    \\
do ‘to’     &    rzek·i ‘river.{\textsc{gen}}’ &&     dorzecz·e ‘river basin’     \\
na ‘on’  &  brzeg·u ‘rim, bank.{\textsc{loc}}’  &&  nabrzeż·e ‘embankment’        \\
pod ‘under’   &  dach·em ‘roof.{\textsc{instr}}’ &&   poddasz·e ‘attic’           \\
przed ‘before’& wiosn·ą ‘spring.{\textsc{instr}}’  &&  przedwiośni·e ‘early spring’\\
\end{tabular}
\z 


{The derivatives on the above list share a characteristic \isi{grammatical} property: they are all neuter gender nouns whose stem ends in a (functionally) palatalized consonant and hence they take the inflectional suffix} ·{\textit{e}} {in the \textsc{nom.sg.} (the input noun may be \textsc{masc.} (e.g.} {\textit{bok}} {‘side’) or \textsc{fem.} (e.g.} {\textit{rzek}}{\textit·}{\textit{a}} {‘river’)). This characteristic pattern of \isi{inflection} (together with the \isi{phonological} effect on the stem-final consonant) may be looked upon as a co-formative which, apart from the preposition, spells out the derivational process in question. Accordingly, the de-phrasal nouns on the list are one instance of so-called paradigmatic \isi{derivation} in \ili{Polish}. However, for quite a few masculine nouns derived from} {\isi{prepositional} phrases we do not observe any change in paradigm; for instance,} {\textit{podtekst}} {‘implied meaning, subtext’ and} {\textit{tekst}} {‘text’ are uniformly masculine (cf. *}{\textit{podtekście}}{, noun, neuter) and are declined according to the same paradigmatic pattern. Less commonly, the feminine paradigm is preserved; e.g.} {\textit{troska}} {‘worry, care’ –} {\textit{beztroska}} {‘carefreeness’. In still other formations, the preposition co-occurs with an overt nominalizing \isi{suffix} (most frequently -}{\textit{ek/-k}}{\textit·}{\textit{a}} {or -}{\textit{nik}}{): e.g.} {\textit{podnóżek}} {‘footrest, footstool’ vs.} {\textit{noga}} {‘foot’,} {\textit{narożnik}} {‘corner (of a building, room, etc.)’ vs.} {\textit{róg}} {‘corner’.}

{The status of the nouns analysed here is complicated by the fact that the majority of native \ili{Polish} prepositions have homophonous counterparts in various prefixes (the identity is not coincidental – it reflects a historical development: preposition > prefix). Therefore, some earlier studies of the data at hand stressed the prefixal character of the initial element, while others argued that the type is a specific instance of Preposition + Noun compounding. In more recent accounts (see \citealt[10]{Symoni-Sułkowska1987}), a compromise solution is opted for: nouns like} {\textit{podziemie}} {are viewed as a borderline phenomenon, between compounding and lexical \isi{derivation}. Still, it is stressed that they are based on \isi{prepositional} phrases; the prepositions (a syntactic category) that surface in the complex nouns acquire the secondary function of prefixes (a morphological category).}

\ea\label{ex:szymanek:20}
Prepositional Phrase (P + Noun\textsubscript{Infl})    >  De-phrasal Adjective\\
\begin{tabular}{lll}
 bez ‘without’  &  robot·y ‘work.{\textsc{gen}}’  &  bezrobotn·y      \\
&& ‘jobless’ \\
 między ‘between’  &wojn·ami ‘war.{\textsc{instr.pl}}’ &   międzywojenn·y \\
&&‘interwar’\\
 pod ‘under’  &  ziemi·ą ‘earth, ground.{\textsc{instr}}’ & podziemn·y\\ &&‘underground’\\
 przez ‘through’   & skór·ę ‘skin.{\textsc{acc}}’   &   przezskórn·y\\ &&‘transdermal’ \\
\end{tabular}
\z 

{Here, again, the status of such “de-phrasal” formations is controversial. In fact, the exact mode of their \isi{derivation} has received alternative accounts. The traditional view has it that the adjective} {\textit{podziemny}} {‘underground’ in, say,} {\textit{podziemny wybuch}} {‘underground explosion’ is derived from the \isi{prepositional} phrase (P+N)} {\textit{pod ziemią}} {‘under the ground’ (minus \isi{inflection} of the noun). Thus, the structure of the \isi{adjectival} stem may be represented as follows: [[[}{\textit{pod}}]{\textsubscript{P}} [{\textit{ziem}}{-]}{\textsubscript{N}}]{\textsubscript{PP} }{-}{\textit{n}}{-]}{\textsubscript{A}}{. This interpretation makes sense from the semantic viewpoint: the derived adjective and the corresponding phrase are functionally equivalent. One problem with this sort of analysis is that numerous \ili{Polish} prepositions are, by and large, phonetically indistinguishable from common native prefixes. This may encourage an alternative analysis of} {\textit{podziemny}}{: as a combination of a prefix (}{\textit{pod}}{-) and the denominal adjective} {\textit{ziemny}} {‘of earth, ground’. This analysis seems viable here since the adjective} {\textit{ziemny}} {happens to exist as an independent word. In fact, in the majority of comparable structures a denominal adjective may be extracted. However, there are also cases like} {\textit{pośmiertny}} {‘posthumous’ where only the \isi{derivation} from the \isi{prepositional} phrase} {\textit{po śmierci}} {‘after death’ is likely, in view of the fact that the denominal adjective *}{\textit{śmiertny}} {(<} {\textit{śmierć}} {‘death’) does not exist (see \citealt[499]{Kallas1999}). The third option, especially in cases like} {\textit{pośmiertny}}{, would be to argue that the adjective is a product of parasynthetic \isi{derivation}, with a simultaneous attachment of the prefix (}{\textit{po}}{-) and the \isi{suffix} (-}{\textit{n}}·{\textit{y}}).\footnote{{Parasynthetic \isi{derivation} seems a viable solution also in certain cases where a prefixless adjective is actually attested; e.g.} {\textit{mięsień}} {‘muscle’ >} {\textit{domięśniowy}} {‘intramuscular’,} {\textit{skóra}} {‘skin’ >} {\textit{przezskórny}} {‘transdermal’,} {\textit{ziemia}} {‘Earth’ >} {\textit{pozaziemski}} {‘extraterrestrial’.}} {Details aside, the dominant view today is that we are dealing here with \isi{derivation} from \isi{prepositional} phrases. This view is said to be supported by the syntactic and semantic equivalence of the phrasal input and the derivational output (for details, see \citealt[500]{Kallas1999}), i.e. by way of a purely formal, transpositional operation we get a lexical item corresponding to a syntactic phrase \citep[71]{Grzegorczykowa1979}. According to some accounts (e.g. \citealt{Wójcikowska1991}), \isi{derivation} of adjectives from \isi{prepositional} phrases is an instance of so-called ‘univerbation’ in \ili{Polish} morphology (see below).}

\subsection*{Univerbation}

\ea\label{ex:szymanek:21}
{{Noun Phrase (N + Adj)}} > {{Derived Noun}} {(id.)}

\begin{tabular}{llll}
 kuchenka mikrofalowa & ‘microwave oven’ & mikrofalówk·a &‘id.’\\
 szkoła zawodowa& ‘vocational school’  &  zawodówk·a&\\
 sklep warzywny &‘greengrocer’s shop’ &   warzywniak&\\
  statek kontenerowy &‘container ship’  &  kontenerowiec&\\
  \end{tabular}
\z

{From the semantic viewpoint, the derivatives listed in (22) above are based on the corresponding NPs, which have the status of set phrases (collocations). In a way, the head noun of the phrase is replaced by a nominal \isi{suffix}, like -}{\textit{k}}·{\textit{a}}{, -}{\textit{ak}}, {\textit{owiec}}{, (see \citealt[419]{GrzegorczykowaPuzynina1999}). Hence the process in question is described as univerbation or morphological condensation of a multi-word (phrasal) term (see \citealt[113ff]{Laskowski1981}).}

A somewhat different type of univerbation is evidenced by the following pairs:

\ea\label{ex:szymanek:22}
Noun Phrase (N + Adj) > Derived Noun (id.)\\

\begin{tabular}{llll}
 kuchenka mikrofalowa  & ‘microwave oven’  & mikrofal·a & ‘id.’\\
 piłka nożna & ‘football’      &   nog·a\\
 obraz olejny&  ‘an oil painting’  &     olej\\
 wódka żytnia & ‘rye vodka’   &    żyt·o\\
 telefon komórkowy & ‘cellphone’   &    komórk·a\\
 karta graficzna&  ‘video card’  &     grafik·a\\
 \end{tabular}
\z


{Again, on functional grounds, the derivative seems to be based on a NP, i.e. a noun modified by an attributive denominal adjective. However, in contradistinction to the previous group of examples, no nominal \isi{suffix} appears in the derived noun, but rather the bare stem of the adjective; compare} {\textit{kuchenka mikrofalowa}} {‘microwave oven’ >} {\textit{mikrofal}}·{\textit{a}} {‘id.’ vs.} {\textit{mikrofalówk}}·{\textit{a}} {‘id.’. Since most adjectives in the input phrases are denominal themselves, the product of the process is normally identical with the base-noun of the adjective (thus} {\textit{olej}}{\textsubscript{N}} {‘oil’ >} {\textit{olej-n}}·{\textit{y}}{\textsubscript{A}} {‘of oil’ /} {\textit{obraz olejny}} {‘oil painting’ >} {\textit{olej}}{\textsubscript{N}} {‘id.’). However, other examples demonstrate that the situation may be more complicated (see \citealt{Chludzińska-Świątecka1979}, \citealt[137]{Jadacka2001}); cf., for instance, the following derivations involving non-native adjectives:} {\textit{ogród zoologiczny}} {‘zoological garden’ >} {\textit{zoolog}} {‘id.’ or} {\textit{forma supletywna}} {‘suppletive form’ >} {\textit{supletyw}} {‘id.’. These colloquial creations demonstrate that, in formal terms, the mechanism that stands behind the derivatives under discussion is a sort of back-formation or desuffixation (neither} {\textit{zoolog}} {nor} {\textit{supletyw}} {exist as basic nouns).}

{Incidentally, it is worth pointing out that, in the \ili{Polish} literature, there is a suitable and widely used term to denote coinages of the kind just illustrated, which incorporate a phrasal constituent as their base:} {\textit{derywaty of wyrażeń syntaktycznych}}{, i.e. ‘derivatives from syntactic expressions’ or} {\textit{derywaty}} {\textit{odfrazowe}} {‘(de)phrasal derivatives’.}\footnote{{The latter term was used, for instance (many years ago), by \citet{Kreja1971}.}} {However – as far as I know – there is no similar \ili{Polish} term to denote the concept of “phrasal compounds” –} {\textit{złożenia frazowe}} {sounds acceptable only as a literal rendering of the \ili{English}, well-established term. The fact that, in the \ili{Polish} linguistic terminology, there is just no name for the phenomenon of phrasal compounding, seems to suggest that the concept is not considered worth naming, i.e. that phrasal compounds either do not exist or have not been identified as yet in the \ili{Polish} morphological system.}

\section{How about phrasal compounds in other Slavic languages?}\label{sec:szymanek:5}

{Antonietta Bisetto begins her contribution to the special issue of STUF on phrasal compounds with the following generalization: “\ili{Romance} languages seem to lack phrasal compounds of the kind present in some \ili{Germanic} languages” \citep[395]{Bisetto2015}. I have conducted some preliminary research on this issue}\footnote{{My thanks go to Pavol Štekauer for comments on \ili{Slovak} and \ili{Czech} as well as for soliciting relevant remarks from several other \ili{Slavic} experts.}}{, as regards the situation in the \ili{Slavic} languages, and – as far as I can see now – I think I can repeat Bisetto’s generalization, with minor reservations (see below): \ili{Slavic} languages – by and large –seem to lack phrasal compounds of the kind present in some \ili{Germanic} languages.}

My limited expertise and circumstantial evidence allows me merely to posit the above generalization as a working hypothesis. Further cross-linguistic research on this issue is necessary in order to verify this hypothesis so that it can be presented as a strong claim. A good example of the sort of research that is needed is the recent study by \citet{Körtvélyessy2016}, where the types and features of compounding (as well as \isi{affixation}) in 14 \ili{Slavic} languages are identified and compared. Crucially, “phrasal compounds” are not listed there among the major types of compounds in \ili{Slavic}. This omission seems to imply that, to say the least, the category in question is not relevant for the \ili{Slavic} languages at large (i.e. it may be inferred that either phrasal compounds do not exist in \ili{Slavic} languages or they are truly marginal).

{Indeed, one positive exception to this generalization may be \ili{Bulgarian}. According to \citet{Boyadzhieva2007}, a recent phenomenon in \ili{Bulgarian} “newspaper language” is the occasional use of structural equivalents of \ili{English} phrasal compounds.}\footnote{{Instead of the term ‘phrasal compounding’, Boyadzhieva uses the designation ‘syntactic compounding’.}} {They have originated as literal translations of the corresponding \ili{English} constructions, but then “they have gradually become quite frequent”. The analysis is based on a small sample of 23 structurally varied expressions, most of which have been gleaned from the \ili{Bulgarian} edition of the} {\textit{Cosmopolitan}} {magazine. It appears that at least some of the examples on the list closely imitate the structure of phrasal compounds in \ili{English} (unfortunately, \ili{English} glosses are not provided).} 

However, \citet{Boyadzhieva2007} points out as well that the “syntactic compounds” “are felt strange and untypical for the \ili{Bulgarian} language”. The recent occurrence of such structures is explained as a consequence of the fact that Modern \ili{Bulgarian} shows a strong tendency towards analyticity, in comparison to other \ili{Slavic} languages; however, \ili{Bulgarian} is said to be less analytic than \ili{English}. 

{Phrasal nominal compounds in Modern \ili{Bulgarian} are also briefly discussed and illustrated in a paper by \citet{Bagasheva2015}. The type in question, which is said to constitute a new development in the language, is considered against the broader background of innovative “[N N] constructions”, i.e. interfixless compounds like} {\textit{bingo zala}} {‘bingo hall’,} {\textit{biznes obyad}} {‘business lunch’, etc. The short list of “phrasal compounds” given by Bagasheva includes the following items:}

\ea%23
    \label{ex:szymanek:23} 

Phrasal compounds in \ili{Bulgarian} \citep{Bagasheva2015}\\

{\textit{{вземи-му-акъла-съвет}}} [vzemi mu akâla sâvet] 

‘take his mind away advice’

{\textit{{море}}}{-}{\textit{{слънце}}}{-}{\textit{{пясък}}} {\textit{{туризъм}}} [more-sluntse-pjasuk turizum] 
‘sea-sun-sand tourism’

{\textit{{семейство и приятели номера}}} [semejstvo i prijateli nomera] 

‘family and friends tricks’ 

{\textit{{завърти}}}{-}{\textit{{му}}}{-}{\textit{{ума}}}{-}{\textit{{посрещане}}} [zavârti mu uma posrešane] 

‘take his mind away welcoming’

{\textit{{промени-живота}}}{-}{\textit{{си}}}{-}{\textit{{предизвикателство}}} [promeni života si predizvikatelstvo] 

‘change   your life challenge’
\z

Except for the \ili{Bulgarian} data, I have not found any examples, from other \ili{Slavic} languages, that mirror the structure of phrasal compounds of the type found in \ili{English} (or \ili{German}). My informants mentioned only that rather different patterns of “syntactic” compounding may be involved, for instance, in certain surnames. For example:

\ea\label{ex:szymanek:24}  
\ili{Czech}\\
{\textit{Skočdopole}} {lit. ‘jump into field!’}


\gll      skoč     do   pole\\
 jump.\textsc{imp} to   field\\

{\textit{Nejezchleb}} {lit. ‘don’t eat bread!’}
 
\gll  ne  jez    chleb\\
 not eat.\textsc{imp} bread\\

 \ili{Ukrainian}\\
 {\textit{Nepiyvoda}} {lit. ‘don’t drink water!’}

 \gll ne   pij     voda\\
 not drink.\textsc{imp} water\\

 \ili{Polish}

 {\textit{Nieznaj}} {lit. ‘don’t (you) know!’}

\gll       nie  znaj\\
 not   know.\textsc{imp}\\

 {\textit{Niechwiej}} {lit. ‘don’t shake!’}

 \gll nie  chwiej\\
 not  shake.\textsc{imp}\\
\z

As can be seen, certain verb phrases in the imperative have been lexicalized to become proper nouns (surnames). 

\section{Conclusion}

To sum up, when we compare the patterns and principles of compounding in \ili{Polish} and \ili{English}, it is easy to notice that there are quite a few structural options that are attested in \ili{English} only (and vice versa). In this context, it should come as no surprise that phrasal compounding seems to be just another feature of this sort, i.e. it is not to be found in \ili{Polish}, just like in many other languages.

But let us repeat the vital question: Why aren’t there any compound nouns in \ili{Polish} of the type that is found in \ili{English}?

Here are some possible reasons that may conspire to produce the effect in question:

\begin{enumerate}
\item Compounding, as a general type of process in word-formation, is much less productive in \ili{Polish} than in \ili{English}. 
\item Instead of the characteristic \ili{English} N+N type of compounds, there are alternative and productive means in \ili{Polish} grammar (particularly ‘multi-word units’) often used for the expression of a combination of two (or more) nominal concepts.
\item In contrast to \ili{English}, the formation of compound nouns in \ili{Polish} is not characterized by \isi{recursion} or iteration. Moreover, there are virtually no compound nouns with more than two constituents (regardless of the category of the first element). By this limitation alone, it is hardly possible to have a complex, multi-word \isi{modifier}, in the form of a phrase.
\item While \ili{English} phrasal compounds are \isi{determinative} and right-headed, in Polish, some compounds are actually \isi{left-headed}, with a considerable proportion of exocentric structures.
\item Perhaps the unavailability of phrasal compounding in \ili{Polish} is also due to typological differences between \ili{English} and \ili{Polish}, i.e. the fact that \ili{Polish} morphology is predominantly synthetic while \ili{English} morphology is (more) analytic. It needs to be determined, on the basis of data from other languages, if a correlation of this sort exists and if it is significant; in other words, does the degree of synthesis in morphology correlate with the presence/absence of phrasal compounds, in various languages? Also, what is the role of language contact and borrowing in the spread of phrasal compounding?
\end{enumerate}

{\sloppy
\printbibliography[heading=subbibliography,notkeyword=this]
}

\end{document}