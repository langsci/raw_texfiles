\documentclass[output=paper]{LSP/langsci} 
\author{Alexandra Bagasheva\affiliation{Sofia University}
}
\title{On a subclass of nominal compounds in Bulgarian: The nature of
  phrasal compounds} \shorttitlerunninghead{On a subclass of nominal compounds in Bulgarian} 

\abstract{The paper focuses on the study of a rarity in the Bulgarian language – phrasal compounds (PCs). Although not recorded in the Bulgarian National Corpus (BulNC), such compounds have successfully infiltrated the language of lifestyle magazines and the jargon of tourism. Although it does not attempt to provide a quantitative study, this paper reviews the properties of PCs in Bulgarian against a checklist of cross-linguistically recognised properties of PCs, gleaned from the growing body of literature on this type of compound. An explanation for the appearance and nature of PCs in Bulgarian is sought in their being offshoots of the recent accommodation in the language of a novel subordinative, modifying [N\textsubscript{1}N\textsubscript{2}/N\textsubscript{2}N\textsubscript{1}]\textsubscript{N} compound type. From lexical or “matter” borrowing, root [N\textsubscript{1}N\textsubscript{2}/N\textsubscript{2}N\textsubscript{1}]\textsubscript{N} of a determinative type established themselves as a new strategy within compounding, recognisable as structural or “pattern” borrowing via upward strengthening, and paved the way for PCs.}

\ChapterDOI{10.5281/zenodo.885121}
\maketitle

\begin{document}

% \textbf{Key words:} \textit{nominal compounding, \ili{Bulgarian}, phrasal compounds, constructionalisation}

\section{Introduction}\label{sec:bagasheva:1}

The literature on morphology has recently abounded with discussions of linguistic phenomena that bear the label \textit{phrasal} (e.g. phrasal names - \citealt{Booij2009wordstructure}, phrasal lexemes - \citealt{Masini2009}, and phrasal compounds - \citealt{Lieber1992,Pafel2015,Trips2016,BagriacikRalli2015}, to name but a few).  Though not coterminous, the three items of interest share two properties that seriously challenge the \isi{Lexical Integrity Hypothesis}: they have a naming and not a descriptive function and they contain syntactic objects, phrases, in their structural makeup. They all seem to pose serious questions concerning the architecture of language, the nature of compounding, the nature of the lexicon (if this is assumed to be a separate component), the essence of the syntax-morphology interface (in a modular conception of language), etc. 

Without aiming at providing answers to such ambitious questions as the above, the current paper focuses on the nature and status of the non-homogenous group of compounds in which the non-head can host a constituent at phrase-level or above in a \ili{Slavonic} language – \ili{Bulgarian}. Acknowledging that studying the nature of an atypical linguistic element (for the specific language) will probably raise more questions than answer fundamental existing ones, the objectives of the paper are twofold: a) to analyse the properties of phrasal compounds at a stage in their development in a typologically specific language when they are considered a novelty and b) to provide a plausible scenario for the appearance of such constructions.

Phrasal compounds are recognised as characteristic of \ili{Germanic} languages (\citealt{Trips2012,Trips2016}) and opinions have been voiced that some can be found in \ili{Romance} languages (at least in \ili{Italian}, e.g. \citealt{Bisetto2015}). They have also been discussed as standard elements of the lexicon (as syntactic or morphological objects) in \ili{Turkish} and \ili{Greek} (\citealt{BagriacikRalli2015,Ralli2013locus,Ralli2013moderngreek}), but have been deemed virtually non-existent in \ili{Slavonic} languages \citep{Ohnheiser2015}.  In relation to these claims, the paper traces the appearance and features of phrasal compounds in \ili{Bulgarian} in an attempt to see whether they share any characteristics with well-studied and established phrasal compounds in \ili{English} and \ili{German}. An explanation is sought for their appearance and nature in the language and their restricted use (in terms of domains). 

In view of the set objectives the paper is structured as follows: part one reviews the findings of previous research on phrasal compounds in \ili{English} and \ili{German}; part two presents the adopted analytical framework; in part three the features of phrasal compounds in \ili{Bulgarian} are checked against a summative list of properties of phrasal compounds in other languages; in part four an analysis and tentative explanation of the data are provided; and part five concludes.

\section{Phrasal compounds – what we know so far}\label{sec:bagasheva:2}

In the literature dealing with phrasal compounds it has been unanimously recognised that they differ from root nominal compounds in that the non-head member can be a phrase, a clause or even a clause complex\footnote{The term \textit{clause complex} is taken from Halliday’s Systemic Functional Grammar (\citealt{Halliday1994},  \citealt{Halliday2014}) and is considered co-extensive with the standard maximal XP extension, i.e. a CP. The term is chosen to avoid any theoretical commitment.} (see \citealt{Trips2012}, \citeyear{Trips2016}). This property of phrasal compounds seems to have attracted the greatest attention since it challenges basic postulates of standard, generative or at least modular approaches to language with the debate focusing on the interface between morphology and \isi{syntax} (\citealt{BagriacikRalli2015}, \citealt{Botha1981}, \citealt{Lieber1988}, \citealt{Ralli2013locus,Ralli2013moderngreek}). Numerous researchers have tried to reconcile this fact with received postulates, such as the \isi{Lexical Integrity Hypothesis}  (e.g.  \citealt{Lieber2006}), or with incessant competition between the modules of morphology and \isi{syntax}  \citep{AckemaNeeleman2004}, others have postulated the existence of two types of PCs depending on their properties and locus of generation, i.e. they recognise morphological PCs and syntactic PCs (\citealt{BagriacikRalli2015}, \citealt{Botha1981}, \citealt{Lieber1988}) or have postulated two distinct mechanisms of clause\slash phrase to word conversion, which leads to the creation of two different types of PCs – pure quotative phrasal compounds and pseudo-phrasal compounds or \isi{non-quotative} PCs \citep{Pafel2015}. Attempts have also been made to explain away the nature of phrasal compounds by lexicalisation of the non-head constituent  \citep{BresnanMchombo1995}.   Others have analysed these compounds from alternative perspectives (e.g. Jackendoff’s parallel architecture – \citealt{Trips2012,Trips2016}; Construction Grammar - \citealt{Hein2015}; etc.) laying the emphasis on the semantics, pragmatics and usage patterns of phrasal compounds, besides the obvious structural features (e.g. \citealt{Meibauer2007,Meibauer2015}). 

There seems to be unanimous agreement, irrespective of the theoretical framework, that PCs have a single, unified meaning as a naming unit. This is achieved by reducing the structural complexity of PCs through downgrading, quoting or conversion to a word status. Alternatively, PCs are denied lexicalisability and are identified as metacommunicative (e.g. \citealt{Hohenhaus2007}). To be even more specific, \citet{Pafel2015} clearly states that the non-head constituent in genuine, quotative phrasal compounds is a noun. He argues that quotative phrasal compounds are morphologically and semantically regular NN compounds and they do not contain syntactic phrases. The scholar admits that \isi{non-quotative}, pseudo-phrasal compounds can have only nouns as heads, while in quotative compounds the head can also be \isi{adjectival}. If this criterion of the nature of the head is to be taken as diagnostic, then in \ili{Bulgarian} only quotative phrasal compounds exist. This possibility will be explored in part three. 

\largerpage
Among the general points of agreement seems to be the headedness of phrasal compounds. The prevalent opinion is that such constructions are \isi{right-headed}. 

In her paper \citet[322]{Trips2012} writes, 

\begin{quotation}
[w]hat makes these compounds so special is that the left-hand member is a complex, maximal phrase: as in the examples given above, it can be a whole sentence like an IP (or CP depending on the analysis), which clearly sets them apart from NNCs, the left-hand member of which is non-phrasal and thus an entity on the word level.
\end{quotation}

It logically follows that the head constituent is the rightmost member of the construction. However, the assumption that PCs are invariably \isi{right-headed} is somewhat premature as will be shown in part three. Suffice it here to say that just as there are both left- and \isi{right-headed} nominal compounds of the same type in \ili{Bulgarian}, e.g. \textit{{очи}-{череши} }[oči-čereši, ‘eyes-cherries’, \textit{large, beautiful eyes}] vs. \textit{{гайтан-вежди}} [gaytan-veždi, ‘woollen braid’, \textit{well-shaped eyebrows}], in the same manner PCs in \ili{Bulgarian} can easily have the nominal head on the left-hand side.

The last almost unanimously recognised feature of PCs that distinguishes them from \isi{subordinative} (\isi{determinative}, modifying) nominal compounds relates to \isi{recursion}. As \citet{Trips2012} and \citet[286]{Trips2016} maintain, “[i]t is well-known that the rule for building \isi{determinative} nominal compounds can be applied to the product of this rule infinitely […], while such \isi{recursion} is ill-formed in phrasal compounds”.  

To sum up, from the growing literature on phrasal compounds the following features of PCs undoubtedly characterise them crosslinguistically:

\begin{itemize}

\item[1)]  they behave like words both in terms of distribution (syntactic behaviour) and in terms of meaning;

\item[2)]  they have a non-disputed naming function;

\item[3)]  they have a stereotyping effect;

\item[4)]  while they can have a variable structural head, they are mostly nominal;

\item[5)] they do not tolerate \isi{recursion};

\item[6)]  they are (mostly) \isi{right-headed}. \end{itemize} 


Besides these recognised features of PCs, their status as lexicalised\slash lexicalis\-able or nonce formations has also attracted the attention of scholars, with opinions on the matter still divided (e.g.  \citealt{BresnanMchombo1995}; \citealt{Hohenhaus1998}; \citealt{Meibauer2007}; \citealt{Trips2012}, \citeyear{Trips2016}; etc.). The two diametrically opposed views are represented by \citet{BresnanMchombo1995}, who maintain that phrases which function as constituents of words are always lexicalised, and \citet{Hohenhaus1998}, who contends that PCs are non-lexicalisable context-dependent nonce formations. In between these extremes, \citet{Trips2012,Trips2016} and \citet{Meibauer2007} admit the existence of both lexicalised PCs and those produced on the fly. 

The overview presented here of findings relating to phrasal compounds cross-linguistically is far from exhaustive but it will suffice as a checklist to be used in describing PCs in \ili{Bulgarian}.

\section{The analytical framework}\label{sec:bagasheva:3}

As \citet[93]{Booij2010} maintains, if instead of recognising abstract rules, we subscribe to an anal\-ogy-based approach recognising \isi{constructional} schemas, we could pay due attention to semantic specialisations and apply the adequate degree of granularity of analysis (generalisation) to be able to describe a wide variety of word-formation data. 

Admittedly, a constructionist approach to language renders void the controversy over the distinction between compounds that are morphological formations as in Modern \ili{Greek} (\citet{Ralli2013moderngreek,Ralli2013locus}) and compounds that are syntactic formations such as those in \ili{Turkish} (\citealt{BagriacikRalli2015}; \citealt{Ralli2013locus}; etc.). However, the choice of this framework is not simply a matter of convenience. It is motivated by three separate considerations: first, constructions can have varying degrees of complexity and schematisation, without compromising the uniformity of pairing meaning and form characteristic of the constructicon\footnote{Within constructionist approaches to language, there is no dividing line between grammar and the lexicon. Language (or the constructicon) is conceived of as a lexicon-\isi{syntax} continuum or a complex network of constructions of varying degrees of complexity held together by inheritance relations (\citealt{Goldberg2003},  \citealt{Hoffmann2013}).}. Second, as demonstrated by various pieces of research, Booij’s Construction Morphology (\citealt{Booij2007}, \citealt{Booij2009wordstructure}) has sufficient explanatory power to analyze schemas of varying degrees of complexity and abstractness. Third, as \citet[2]{Fried2013} claims, 


\begin{quotation}
The \isi{constructional} approach is also proving itself fruitful in grappling with various broader analytic challenges, such as accounting for seemingly unmotivated syntactic patterns that do not easily fit in a synchronically attested \isi{grammatical} network for a given language, or that present a typologically odd and inexplicable pattern.                     
\end{quotation}                               


For the reasons stated above, the answers to the research questions are provided in the general framework of \isi{constructional} approaches to language and language change (\citealt{Booij2009a}, \citeyear{Booij2010}, \citealt{Croft2001}, \citealt{Goldberg2006}, \citealt{Hilpert2015}, \citealt{Traugott2013}, among others) and the onomasiological approach to word-formation \citep{Štekauer1998,Štekauer2005}, where the naming needs and active role of speakers are duly recognised.                                                                           Various offshoots of constructionalism with idiosyncratic views and analytical procedures have arisen in recent years (for relevant overviews see \citealt{Croft2007}, \citealt{Sag2012intro}), yet they all share a set of assumptions which allow for the non-differentiated adoption of a constructionalist analytical stance. The basic tenet of constructionalism adopted here holds that language is a constructicon, a set of taxonomic networks where each construction constitutes a node in the network that forms a continuum from the fully concrete to the highly schematic. The relations between constructions are ones of inheritance and motivation. A construction itself is a conventionalised pairing of meaning and form. The constructicon is acquired via language use and innovated via neoanalysis and analogisation \citep{Traugott2013}. Both processes are localized within constructions, or more precisely in actualized constructs. A construction is instantiated in actual language use by specified constructs that are fully phonetically specified and have contextually sensitive meaning, based on their conventionalised meaning or any appropriate extension thereof. A shift in any dimension of the construct might be strengthened via propagated use across the speech community into a modified or novel construction depending on the degree of dissimilarity from the initial one(s). The general model of a construction captures constructions that vary along at least three significant dimensions: type of concept, schematicity and complexity. Type of concept specifies the conventional meaning associated with the construction in terms of its contentfulness or procedural characteristics, i.e. whether it could be used referentially or whether it encodes intralinguistic relations. The dimension of schematicity is related to formal (\isi{phonological}) specificity and degree of abstraction of a token construct, and classifies constructions into substantive (fully specified), schematic (abstract), and intermediate or partial (at least one constituent is specified). The dimension of complexity captures the internal constituency of a construction and distinguishes between atomic, complex and intermediate. Within this constructicon, constructionalisation, defined as “the creation of form\textsubscript{new}-meaning\textsubscript{new} (combinations of) signs” that constitute “[…] new type nodes, which have new \isi{syntax} or morphology and new coded meaning, in the linguistic network of a population of speakers” \citep[22]{Traugott2013} is achieved incrementally via \isi{constructional} changes, defined as shifts along one of the dimensions of a construction \citep[26]{Traugott2013}.                                                                                                                                                        Constructionalisation is one of the ways in which language change is actualised. At the same time it has long been recognised that “languages can undergo structural change as a result of contact” \citep[48]{Heine2006}. In the case of \ili{Bulgarian}, the influx of lexical or ``matter'' (MAT) borrowing from \ili{English} characteristic of the last decades of the 20\textsuperscript{th} century (\citealt{Krumova-Cvetkova2013}, \citealt{Radeva2007}, etc.) brought about the establishment of a new \isi{subordinative}, nominal compound type, which proved fruitful ground for the accommodation of PCs. The establishment of this compound type involved the structural or ``pattern'' (PAT) borrowing of the abstract \isi{construction schema} [N\textsubscript{1}N\textsubscript{2}]\textsubscript{N}, where N\textsubscript{1} modifies N\textsubscript{2} or restricts its interpretation, filled out with native linguistic material exclusively. 


With the understanding of constructionalisation within the constructionalist approach and the notion of \isi{grammatical} replication we are able to analyse in a smooth and uniform manner the nativisation of a borrowing (more specifically lexical borrowing from \ili{English} such as \textit{{бинго маниак} }[bingo maniak, \textit{bingo maniac}] and the \isi{pattern borrowing} of the \isi{subordinative}, modifying nominal compound construction [N\textsubscript{1}N\textsubscript{2}]\textsubscript{N}) and their constructionalisation into a new node type or a newly boosted pattern. Simply put, this framework makes it possible to identify the stabilisation of the [N\textsubscript{1}N\textsubscript{2}]\textsubscript{N} pattern in \ili{Bulgarian} as the constructionalisation of a compound type, a new strategy within compounding, which paved the way for the advent of phrasal nominal compounds in the language. As  \citet{Arcodia2009} claim, phrasal constituents are only possible with \isi{subordinative} compounds, so it naturally follows that the establishment of the \isi{determinative} [N\textsubscript{1}N\textsubscript{2}/N\textsubscript{2}N\textsubscript{1}]\textsubscript{N} compound type in \ili{Bulgarian} serves as a prerequisite for the spread of phrasal compounds. Present-day [N\textsubscript{1}N\textsubscript{2}]\textsubscript{N} compounds in \ili{Bulgarian} (recognized as atypical of \ili{Slavic} languages, but characteristic of \ili{Germanic} languages) came into the language under foreign influence. In the \ili{Bulgarian} word-formation literature there is unanimous agreement that the new-found instigated productivity and the fixation of the pattern in terms of both form and meaning potential have been achieved under the influence of \ili{English} (\citealt{Krumova-Cvetkova2013}, \citealt{Murdarov1983}, \citealt{Radeva2007}, etc.), i.e. as the result of language contact. From an influx of lexical borrowing the pattern has grown into a structure accommodating exclusively native constituents (e.g. \textit{{чалга певец} }[čalga pevec, ‘pop folk singer’], \textit{{чалга изпълнител} }[čalga izpâlnitel, ‘performer of pop folk music’], \textit{{тото пункт} }[toto punkt, ‘lottery kiosk’]).

The onomasiological approach to language \citep{Štekauer1998}, and more specifically to word-formation, acknowledges the active agency of speakers in creating new lexical items. For onomasiologists the desire of members of a speech community to come up with the most appropriate (with appropriateness measured by the minimax effect, i.e. minimal cognitive effort, maximum communicative effect, operationalisable as degrees of explicitness) name for a conceptualized piece of extralinguistic reality is the driving force behind word-formation. When the conceptualisation is novel for the cultural context, borrowing is not a neglected resource. In other cases, all the resources of a language (constructicon) can be creatively employed for encoding the intended conceptualisation. It is in the minds and mouths of speakers that the establishment and use of a new name lie (with a host of factors playing a crucial role, purely conceptual, sociolinguistic, cognitive, etc., which will not be commented on for lack of space. For the interplay and different roles of the various relevant factors see \citealt{Štekauer2005}). 

Combining the two analytical perspectives (constructionalisation as both a mechanism and the result of language change, and MAT developing into PAT borrowing) leads to the following understanding of the appearance of phrasal compounds in \ili{Bulgarian}, with the \textsubscript{N} always being the categorial head (for \isi{inflection} purposes):


\ea\label{ex:bagasheva:bagasheva:1}
\ea\relax
[X \textsubscript{R} Y]\textsubscript{N} – \isi{construction schema} of nominal compounds

\ex 
\textsuperscript{} [N\textsubscript{1} N\textsubscript{2}/ N\textsubscript{2} N\textsubscript{1}]\textsubscript{N} – \isi{construction schema} of \isi{determinative}, root nominal compounds       

\ex\relax
[X\textsubscript{phrase\slash clause} N/ N X\textsubscript{phrase\slash clause}]\textsubscript{N} – \isi{construction schema} of phrasal compounds           
\z 
\z 


\textsubscript{R} – is the implicit intracompound relation between the constituents of the compound. As in \ili{English}, this relationship is multifarious. It captures both modifying and thematic relations. As far as intracompound relations are concerned, \ili{Bulgarian} [N\textsubscript{1}N\textsubscript{2}] \isi{determinative} compounds display the whole array of compound internal relations recognized in the literature (including psycholinguistic accounts, e.g. \citealt{Bauer2013}, \citealt{GagneShoben1997}, \citealt{Ryder1994}). 


The schematic representation of productive constructions above captures the specific portion of the compounding network in which the newly established \isi{subordinative}, modifying type of compound in \ili{Bulgarian} found its place. The three levels of abstraction (\ref{ex:bagasheva:bagasheva:1}a, \ref{ex:bagasheva:bagasheva:1}b, and \ref{ex:bagasheva:bagasheva:1}c) represent the hierarchy of the nominal compound network: (\ref{ex:bagasheva:bagasheva:1}a) represents the most abstract schema of all nominal compounds whose instantiations can vary significantly in terms of constituents (e.g. [NN\textsubscript{deverbal, suffixed}]: \textit{{родоотстъпник}} [rodootstâpnik, ‘clan-depart-er’, \textit{traitor}]), [VN]: \textit{{нехранимайко}} [nehranimajko, ‘not-feed-mother’, \textit{scoundrel}\slash\textit{good-for-nothing}], [NN\textsubscript{deverbal,suffixless}]: \textit{{изкуствовед}} [izkustvoved, ‘art-know-er’, \textit{art critic\slash expert}], etc. (\ref{ex:bagasheva:bagasheva:1}b) represents the schema of the newly established \isi{determinative}, root compound type (e.g. \textit{{чалга изпълнител}} [čalga izpâlnitel, ‘pop folk performer’]) whose accommodation in the language made possible the appearance of phrasal compounds, whose generalised \isi{construction schema} is represented by (\ref{ex:bagasheva:bagasheva:1}c).

Determinative and phrasal compounds differ along two parameters – the modifying constituent (noun vs. phrase\slash clause) and the nature of \textsubscript{R,} which in the case of [N\textsubscript{1} N\textsubscript{2}/ N\textsubscript{2} N\textsubscript{1}]\textsubscript{N} compounds can be quite specific, though invariably diverse, including thematic relations between the two constituents. In the case of phrasal compounds the simplest way to define \textsubscript{R} is to acknowledge that it is a severely underspecified relationship which accounts for the meaning ‘N is a type characterised by the stereotypical properties of X’. Even this difference, however, does not pose a problem for uniform treatment of the two subtypes of nominal compounds since \citet{Bauer2013} recognise a grossly semantically underspecified adnominal modification relationship in various types of constructions, even ones not restricted to different types of compounds. 

In the remainder of the paper it is assumed that all the analysed phrasal compounds are instantiations of the \isi{construction schema} hierarchy presented above.



\section{Phrasal compounds in Bulgarian}\label{sec:bagasheva:4}



\subsection{Bulgarian in comparison to Germanic languages}\label{sec:bagasheva:4.1}



Since phrasal compounds are acknowledged as characteristic of \ili{Germanic} languages (more specifically \ili{English} and \ili{German}), and their existence is denied for \ili{Bulgarian}, at the outset a rough typological sketch of the language under investigation is in order. In terms of analyticity the three languages might be ordered along a cline, with \ili{English} being the one with the greatest degree of analyticity, \ili{Bulgarian} occupying a middle position (with heavy inflectional paradigms for verbs, but virtually none for nouns) and \ili{German} taking the last position. All three languages are remotely genealogically related as they belong to different groups of the same family. All three languages can be identified as nominative-accusative. 

Applying the typology associated with syntactic harmony \citep{Hawkins1983}, the following summative facts can be presented (see \tabref{tab:bagasheva:1}).


\begin{table}
\caption{A typological sketch of Bulgarian, English and German}
\label{tab:bagasheva:1}

\begin{tabularx}{\textwidth}{QQQQ} 
\lsptoprule
& \ili{Bulgarian} & \ili{English} & \ili{German}\\
\midrule
Canonical word order & SVO & SVO & SVO, but also SOV in some types of embedded clauses\\
\tablevspace
Prepositions vs. postpositions & Prepositions & Prepositions & Prepositions
\newline
Postpositions (marginally)\\
\tablevspace
Modifier-head (including Numeral-Noun, Demonstrative-Noun, Possessive pronoun-Noun, Adjective-Noun) & Modifier-head & Modifier-head & Modifier-head\\
\tablevspace
Genitive-noun & Both orders (G-N, N-G) possible & Both orders (G-N, N-G) possible & Both orders (G-N, N-G) possible\\
\tablevspace
Head-\isi{relative clause} & H-R & H-R & Both orders (H-R, R-H) possible\\
\tablevspace
Case system (nouns) & No & No (barring the Genitive) & Yes\\
\tablevspace
Affixation & Highly productive & Highly productive & Highly productive\\
\tablevspace
Compounding & Less productive & Highly productive & Highly productive\\
\lspbottomrule
\end{tabularx}
\end{table}

\largerpage[-1]
If the typologically relevant features are anything to go by, then one would not predict any major differences between the three languages with regard to the behaviour of (phrasal) compounds in the three languages. However, \citet[1824]{Ohnheiser2015} claims that

\begin{quotation}
in \ili{Slavic} languages the formation of compounds including a verbal \isi{modifier} is impossible. 
\end{quotation}

This would suggest that phrasal compounds containing full \isi{predication} are ruled out for \ili{Bulgarian}. Admittedly, as a \ili{Slavonic} language, \ili{Bulgarian} is characterised by more productive \isi{affixation}, rather than compounding. As \citet[911]{Olsen2015} claims, 


\begin{quotation}
the \ili{Romance} and \ili{Slavic} languages are not highly compounding languages (especially if the default case of lexical combinations of basic stems without formatives or functional categories are the focus of attention). 
\end{quotation}

Yet, synthetic (e.g. \textit{{въжеиграч}} [vâžeigrač, ‘rope-play-er’, \textit{tight-rope walker}]; \textit{{факлоносец} }[faklonosec, ‘torch-bear-er’, \textit{torchbearer}]; \textit{{земевладелец} }[zemevladelec, ‘land-own-er’, \textit{landowner}]) and coordinative nominal (e.g. \textit{{вагон-ресто\-рант} }[vagon-restorant, ‘wagon-restaurant’, \textit{dining car}], \textit{{къща-музей}} [kâša mu\-zey,  \textit{house museum}], \textit{{страна-членка} }[strana-členka, \textit{member state}]) and \isi{adjectival} compounds (e.g. \textit{{патил препатил}} [patil prepatil, ‘having suffered having suffered too much’, \textit{experienced}], \textit{{кървавочервен}} [kârvavočerven,  \textit{blood red}], \textit{{тревнозелен}} [trevnozelen, \textit{grass green}]) are abundant in the language. What \ili{Bulgarian} is claimed to lack are verb compounds and root or primary nominal compounds of a \isi{subordinative}, modifying type. In the nominal field (where the input and the output of the process are nouns, albeit in the input often they are deverbal nouns), composition proper resulting in nominal compounds is mostly based on thematic relations between head and non-head and yields synthetic or parasynthetic compounds. 

Nonetheless, the emergence of many new composite substantives without a \isi{linking vowel} between the two components in \ili{Bulgarian} is a sign that the language is developing towards more pronounced analyticity  \citep[73]{Avramova2003} and this tendency concerns mainly the nominal system \citep[100]{Vačkova1998}. Not surprisingly, it is precisely new types of nominal compounds that have been emerging steadily in the language, namely \isi{subordinative}, modifying nominal compounds (with a subtype containing abbreviations as a modifying constituent) and phrasal compounds. Phrasal compounds of the type \textit{on-the-spot creations}, \textit{will-she-or-won’t-she-get-the-guy comedy,} etc. were in fact nonexistent in \ili{Bulgarian} before the 1990s and this model of hyphenated compound phrases has certainly been imported.



\subsection{Phrasal compounds in Bulgarian}\label{sec:bagasheva:4.2}
Phrasal compounds in \ili{Bulgarian} are special in several respects: a) they appear mostly in writing\footnote{For lack of a reliable corpus of spoken \ili{Bulgarian} the appearance of phrasal compounds in oral communication is not discussed in the current paper.}; b) they seem to be text type and genre specific – at present they appear systematically in the Psychology, Friends and Advice sections of the \ili{Bulgarian} edition of Cosmopolitan; c) they can be spelled with hyphens, in quotation marks or a combination of the two (besides, hyphenation sometimes includes the head, while enclosure in quotation marks never does) and d) they can be left- or \isi{right-headed}, with left-headedness being characteristic of non-lexicalised ones exclusively (without any other correlation observed between headedness and the type of PC involved, i.e. \isi{predication} vs. non-\isi{predication} or quotative vs. pseudo-phrasal compounds). Concerning the PCs marked by hyphens or quotation marks, none of the phrasal non-heads are lexicalised (i.e. they are not listed in any available lexicographic source and are not attested in the BulNC). Without reading too much into spelling conventions, we have to note that quoting is perceived as a stereotyping strategy, as can be gleaned from example (22) in the Appendix, in which we have a well-formed \isi{relative clause} which is quoted apart from the relativiser. The quoting achieves the effect of creating a special type or category of well-wishers (without describing well-wishers actually interested in someone’s thoughts).  

The very few lexicalised PCs in \ili{Bulgarian} are restricted to the jargon of tourism. They are either hybrid formations or sound like loan translations (even though the specific non-head phrases are not attested in the alleged source language, \ili{English}). They are recorded in specialised dictionaries in the field.



Comparing the list of \ili{Bulgarian} PCs with the set of properties defined above, we get the following picture:

\begin{itemize}

\item[1)] they behave like words both in terms of distribution (syntactic behaviour) and in terms of meaning – they can be premodified by adjectives, e.g. \textit{{направи-си-сам-проблеми-в-офиса ситуация}} [napravi-si-sam-problemi-v-ofisa situaciya, ‘create-your-own-problems-at-the-office situation’] {мно\-го трудна} \textit{{направи-си-сам-проблеми-в-офиса ситуация} }[mnogo trudna {\textasciitilde}, ‘very difficult {\textasciitilde}’]\textit{;} they can be marked for definiteness, e.g.   \textit{{направи-си-сам-проблеми-в-офиса ситуация}}{та} {от вчера [{\textasciitilde} situaciya}ta ot včera, ‘the {\textasciitilde} from yesterday’]; they can be marked for plurality, e.g. {множество} \textit{{направи-си-сам-проблеми-в-офиса ситуаци}}{и} [množestvo {\textasciitilde} situacii, ‘a lot of {\textasciitilde} situations’], etc.;

\item[2)] they have a non-disputed naming function – \textit{{вземи-му}-{акъла-съвет} }[vze\-mi-mu-akâla-sâvet, ‘blow-his-mind-away-advice’] names a specific\linebreak piece of advice which prescribes an easily identifiable course of action;

\item[3)] they have a stereotyping effect – \textit{{напрежението ето}-{че-моментът-най-сетне}-{настъпи} }[napreženieto eto-če-momentât-nay-setne-nastâpi, the-\linebreak tension there-the-time-has-finally-come’] names a very specific kind of tension which everyone is liable to a long-awaited moment has finally arrived;

\item[4)] while they can have a variable structural head, they are mostly nominal – all examples in \tabref{tab:bagasheva:2} are nouns;

\item[5)] they do not tolerate \isi{recursion} – It is impossible to recursively expand in either direction any of the compounds in \tabref{tab:bagasheva:2}; 


\item[6)] they are (mostly) \isi{right-headed} – this is not necessarily the case. As will become evident from \tabref{tab:bagasheva:2}, phrasal compounds in \ili{Bulgarian} have no marked headedness preference. 14 out of 29 compounds are \isi{right-headed}, while 15 have the categorial and semantic head on the left. All 29 are endocentric compounds and resemble \isi{determinative}, modifying nominal compounds in the language in all respects but two – the structural make-up of the non-head constituent and the relationship between the head and the non-head constituent. In the constructionist framework adopted here, these differences are of no relevance, since at the highest level of generalisation of the corresponding construction they are collapsed into a single, governing abstract schema.

\end{itemize}

Before presenting the most important features of PCs in \ili{Bulgarian} in table format (\tabref{tab:bagasheva:2}), we need to comment on two issues relating to the last two properties in this list.

As in other languages, PCs in \ili{Bulgarian} do not tolerate \isi{recursion} in the non-head position (for other languages see above). \citet[350]{Bauer2009} states that “[w]e do not have sufficient information to see whether \isi{recursion} or lack of \isi{recursion} in compounds is the default, or whether either of these correlates with any feature of compounding, though it would be something worth checking.”  Recursion is excluded for all types of compounds in \ili{Bulgarian} in general (e.g. *{груборазтуриколиба} [gruborazturikoliba, ‘brutally tear-down-hut’, \textit{brutal} \textit{adulterer}], *{лошовестоносец} [lošovestonosec, ‘bad-news-bring-er’, \textit{harbing\-er}], *{изящноизкуствовед [izya}štnoizkustvoved, ‘fine art leader’, \textit{fine art critic}] *{нова чалга певец} [nova čalga pevec, \textit{new} \textit{pop folk singer}]). What is allowed is independent coordination within the non-head constituent only (e.g. \textit{{авто и мототех\-ника}} [avto i mototehnika, ‘technical equipment for cars and motorcycles’]). But as \citet[84]{Plag2003} notes, this phenomenon is characteristic of both \isi{affixation} and compounding and is not indicative of any peculiar features of (nominal) compounds -- or as \citet[157]{Bell2011} concludes, coordination data has no bearing on the (morphosyntactic) nature of nominal compounds. It is reasonable to propose, with \citet[286]{Trips2016}, “that this restriction is subject to extra-\isi{grammatical} factors like limitations of processing”. However, the fact that \isi{recursion} is excluded as a possibility for PCs does not render them different from other compounds in \ili{Bulgarian} (none of which permits \isi{recursion}) and does not reveal that “phrasal compounds do not behave like normal compounds” as suggested by \citet{Trips2016}. Rather, at least in \ili{Bulgarian}, phrasal compounds behave very much like \isi{subordinative} nominal compounds. What is more,  \citet[11]{Arcodia2009} note that “having a phrasal constituent is possibly a unique property of subordinate compounds.”  To cut a long story short, phrasal compounds do not tolerate \isi{recursion} but this does not set them apart from other nominal compounds in the language.


Just as the non-admission of \isi{recursion} does not mark out phrasal compounds as unique in \ili{Bulgarian}, neither do their properties in relation to headedness. Extant [N\textsubscript{1}N\textsubscript{2}/N\textsubscript{2}N\textsubscript{1}]\textsubscript{N} compounds without a linking component are either considered appositive, as in \textit{{вагон-ресторант}} [vagon-restorant, ‘dining car’], \textit{{за\-мест\-ник-директор} }[zamestnik-direktor, ‘deputy director’], \textit{{кандидат-студент}} [kan\-di\-dat-student, ‘student applicant’] \citep[56-58]{Radeva2007}, or are interpreted as a group in their own right (with a variety of labels attached to them by different authors, see \citealt{Kirova2012}, \citealt{Murdarov1983}, \citealt{Radeva2007}, etc.) with a semantic operator of implicit comparison \citep[58]{Radeva2007} as in \textit{{очи-череши} }[oči-čerešhi, ‘eyes-cherries’, \textit{large, beautiful eyes}], \textit{{гайтан-вежди} }[gaytan-veždi, ‘woollen braid’, \textit{well-shaped eyebrows}], \textit{{снага-топола} }[snaga-topola, ‘body-poplar’, \textit{slender body}], etc. 

The problems in the classification and analysis of NN compounds stem from the fact that they occur with variable semantic heads (on the parametrised treatment of the concept of head in compounding see \citealt{GuevaraScalise2009};  \citealt{Scalise2006}). In \textit{{очи-череши} }[oči-čerešhi, ‘eyes-cherries’, \textit{large, beautiful eyes}] and \textit{{снага-топола} }[snaga-topola, ‘body-poplar’, \textit{slender body}] the element that is being described appears on the left and the meaning of the whole suggests that it is the semantic anchor: eyes like cherries and a body like a poplar. In the exocentric \textit{{гайтан-вежди} }[gaytan-veždi, ‘woollen braid’, \textit{well-shaped eyebrows}], it is the rightmost member that names the entity being described and the first constituent introduces the comparative attribute. In terms of categorial headedness the first two compounds have two categorial heads (as both constituents will be inflected), while in the third instance categorial and semantic head coincide. The way out of the analytical conundrum is to acknowledge that all such compounds (including the influx of endocentric \isi{subordinative} nominal compounds) are \isi{determinative}, modifying compounds with variable semantic and\slash or categorial head.

Likewise we need to allow for both left- and \isi{right-headed} phrasal compounds, in which the position of the head does not affect the set of remaining properties of the respective compounds. Thus, \textit{{да-се}-{почувстваш-добре ефект}} [da-se-počuvstvaš-dobre efekt, ‘to-start-feeling-well effect’] is \isi{right-headed}, while \textit{{напрежението ето}-{че-моментът-най-сетне}-{настъпи}} [napreženieto eto-če-\linebreak momentât-nay-setne-nastâpi, ‘the-tension there-the-time-has-finally-come’] is \isi{left-headed} but no ensuing predictions can be made as to the remaining properties of the compounds. In fact, the two compounds share all their properties – they contain fully-fledged declarative predications, they are both quotative and the head in both cases is of the same semantic type, namely Attitude.

And last but not least, \citegen{Trips2012} claim that the most frequent PCs (in her empirical study of PCs in \ili{English} based on data from the BNC) are quotative PCs holds true for \ili{Bulgarian} PCs (as evidenced in the table below). The table presents a summary of the outstanding properties of phrasal compounds in \ili{Bulgarian} and their classification in accordance with recognised classificatory criteria (\citegen{Pafel2015} quotative and pseudo PCs, \citegen{Trips2016} predicational and non-predicational PCs and their semantic types).


\begin{sidewaystable}
\caption{Phrasal compounds in Bulgarian} 
\label{tab:bagasheva:2}
\footnotesize
\begin{tabularx}{\textwidth}{QQQQp{1.5cm}l}
\lsptoprule

\textbf{Example in Bulgarian}\footnotemark[4]
\newline 
\textbf{(semantic type}\footnotemark[5] \textbf{of the head of the PC)} & \textbf{Transliteration} & \textbf{Meaning} & \textbf{Type in relation to the \isi{predication} status of the non-head (structural make-up)} & \textbf{Quotative or pseudo} & \textbf{Headedness}\\
\midrule 
\textit{{свали-го-съвет}}
\newline 
(attitude/utterance/medium conveying utterance) & svali-go-sâvet & hit-on-him-advice & full \isi{predication} (imperative) & quotative & right\\
\textit{{вземи-му-акъла-съвет}}
\newline 
(attitude/utterance/medium conveying utterance) & vzemi-mu-akâla-sâvet & blow-his-mind-away-advice & full \isi{predication}
\newline 
(imperative) & quotative & right\\
\textit{{завърти-му-ума-посрещане}}
\newline 
(action) & zavârti-mu- akâla-posrešane & make-his head-spin-welcoming & full \isi{predication}
\newline 
(imperative) & quotative & right\\
\textit{{промени-живота-си-предизвикателс-тво}}
\newline 
(action) & promeni-života-si-predizvikatels-tvo & change-your-life-challenge & full \isi{predication}
\newline 
(imperative) & quotative & right\\
\textit{{прочети-му-мислите съвет}}
\newline 
(attitude/utterance/medium conveying utterance) & pročeti-mu-mislite sâvet & read-his-thoughts advice & full \isi{predication}
\newline 
(imperative) & quotative & right\\
\textit{{предизвикай-го-да-говори-съвет}}
\newline 
(attitude/utterance/medium conveying utterance) & predizvikay-go-da-govori-sâvet & coerce-him-into-talking-advice & full \isi{predication}
\newline 
(imperative) & quotative & right\\
\textit{{типа кой-пръв-ще-успее-да-пъхне-лед-в-ризата-на-другия}}
\newline 
+(action) & +tipâ koy-prâv-še-uspee-da-pâhne-led-v-rizata-na-drugiya & the-type-who-will-first-succeed-in-putting-some-ice-down- the-shirt-of-the-other & full \isi{predication}
\newline 
(interrogative) & quotative & left\\
+(thing) & +\footnotemark[6]~tip-ne-e-za-izpuskane-nezawisimo-ot-cenata & type-it-is- not-to-be-missed-no-matter-the-price & full \isi{predication}
\newline 
(existential) & quotative & left\\
\lspbottomrule
\end{tabularx}
\end{sidewaystable}

\begin{sidewaystable}
\footnotesize
\begin{tabularx}{\textwidth}{QQQQp{1.5cm}l}
\lsptoprule
\textbf{Example in Bulgarian}\footnotemark[4]
\newline 
\textbf{(semantic type}\footnotemark[5] \textbf{of the head of the PC)} & \textbf{Transliteration} & \textbf{Meaning} & \textbf{Type in relation to the \isi{predication} status of the non-head (structural make-up)} & \textbf{Quotative or pseudo} & \textbf{Headedness}\\
\midrule
\textit{{тип не-е-за-изпускане-независимо-от-цената}}
\newline 
\textit{{факта Боже-не-може-да-бъде!}}
\newline 
(conceptual entity) & fakta Bože-ne-može-da-bâde! & the fact Oh-God-this-cannot-be- true! & full \isi{predication}
\newline 
(existential) & quotative & left\\
\textit{{разновидност ‘Боже-сетих-ли-се-да-изключа-котлона-днес’}}
\newline 
+(conceptual entity) & +raznovidonost ‘Bože-setih-li-se-da-izklyuča-kotlona-dnes’ & variety ‘Oh-God-did-I-remember-to-switch-off-the-cooker?’ & full \isi{predication}
\newline 
(interrogative) & quotative & left\\
\textit{{вариант ‘ей-тази-седмица-наистина-май-ще-ме-уволнят-или-зарежат-или-и-двете’}}
\newline 
+(conceptual entity) & +variant ‘ey-tazi-sedmica-naistina-may-še-me-uvolnyat-ili-zarežat-ili-i-dvete’ & option ‘wow-this-week-I-will-really-get-fired-or-ditched-or-both” & full \isi{predication}
\newline 
(coordinated clause complex) & quotative & left\\
\textit{{да-се-почувстваш-добре} {ефект}}
\newline 
(attitude) & da-se-počuvstvaš-dobre efekt & to-start-feeling-well effect & full \isi{predication}
\newline 
(\textit{da}-construction) & quotative & right\\
\textit{{напрежението ето-че-моментът-най-сетне-настъпи}}
\newline 
(attitude) & napreženieto eto-če-momentât-nay-setne-nastâpi & the-tension there-the-time-has-finally-come & full \isi{predication}
\newline 
(declarative) & quotative & left\\
\textit{{направи-си-сам-проблеми-в-офиса ситуация}}
\newline 
(action) & napravi-si-sam-problemi-v-ofisa situaciya & create-your-own-problems-at-the-office- situation & full \isi{predication}
\newline 
(imperative) & quotative & right\\

\lspbottomrule
\end{tabularx}
\end{sidewaystable}

\begin{sidewaystable}
\footnotesize
\begin{tabularx}{\textwidth}{QQQQp{1.5cm}l}
\lsptoprule
\textbf{Example in Bulgarian}\footnotemark[4]
\newline 
\textbf{(semantic type}\footnotemark[5] \textbf{of the head of the PC)} & \textbf{Transliteration} & \textbf{Meaning} & \textbf{Type in relation to the \isi{predication} status of the non-head (structural make-up)} & \textbf{Quotative or pseudo} & \textbf{Headedness}\\
\midrule

\textit{{развръзка за-вечни-времена}}
\newline 
(action) & razvrâzka-za-večni-vremena & denouement for-eternal-times & non-predicational (N-prep-Adj-N) & pseudo & left\\
\textit{{навика с-цигара-в-ръка}}
\newline 
(property) & navika-s-cigara-v-râka & the-habit-with-cigarette-in-hand & non-predicational (N-prep-N-prep-N) & pseudo & left\\
\textit{{предъвкването-между-другото}}
\newline 
(action) & predâvkvaneto-meždu-drugoto & the-chewing-over-casually-among-other-things & non-predicational(N-prep-N) & pseudo & left\\
\textit{{стърчащо-услужливо-дълго-нокътче}}
\newline 
(thing) & stârčašo-uslužlivo-dâlgo-nokâtče & the-sticking-out-conveniently-long-nail\textsubscript{dim} & *\footnotemark[7] non-predicational
\newline 
(Adj-Adv-Adj-N) & pseudo & right\\
\textit{{иначе-любимия-човек}}
\newline 
(individual) & inače-lyubimiya-čovek & the-otherwise-beloved-person & non-predicational
\newline 
(Adv-Adj-N) & pseudo & right\\
\textit{{намек “има нещо помежду им”}}
\newline 
(medium conveying an utterance) & namek “ima nešo pomeždu im” & hint “there is something going on between them” & full \isi{predication}
\newline 
(existential) & quotative & left\\
\textit{{колежките-кобри-по-душа}}
\newline 
(individual) & koležkite-kobri-po-duša & the-female-colleagues-real-cobras-in-their-hearts & non-predicational
\newline 
(N-N-prep-N) & pseudo & left\\
\textit{{просто-независеща-от-него-странност}}
\newline 
(property) & prosto-nezaviseša-ot-nego-strannost & a-simply-not-depending-on-him-foible & *non-predicational
\newline 
(N-Adj-prep-ProN-N) & pseudo & right\\
\textit{{обикаляне-без-купуване решение}}
\newline 
(action) & obikalayane-bez-kupuvane rešenie & window-shopping-without-buying decision & non-predicational
\newline 
(N-prep-N-N) & pseudo & right\\

\lspbottomrule
\end{tabularx}
\end{sidewaystable}

\begin{sidewaystable}
\footnotesize
\begin{tabularx}{\textwidth}{QQQQp{1.5cm}l}
\lsptoprule
\textbf{Example in Bulgarian}\footnotemark[4]
\newline 
\textbf{(semantic type}\footnotemark[5] \textbf{of the head of the PC)} & \textbf{Transliteration} & \textbf{Meaning} & \textbf{Type in relation to the \isi{predication} status of the non-head (structural make-up)} & \textbf{Quotative or pseudo} & \textbf{Headedness}\\
\midrule

\textit{{доброжелатели, които „само искат да знаят какво мислиш за бъдещето си“}}
\newline 
(individual) & dobroželateli, koito “samo iskat da znayat kakvo misliš za bâdešteto si” & well-wishers who “just want to know what you are thinking about your future” & \#\footnotemark[8] full \isi{predication}
\newline 
(declarative) & quotative & left\\
{ {{\textit{{функцията „На този ден“}}}}}
\newline 
(utterance\slash medium conveying utterance) & funkciyata “Na tozi den” & the function  “On that day” & non-predicational
\newline 
(N-prep-ProN\textsubscript{dem}-N) & quotative & left\\
\textit{{статус „Обичам мазни, потни чичковци“}}
\newline 
(utterance\slash medium conveying utterance) & status “Običam mazni, potni čičkovci” & status “I love sleazy, sweaty old guys” & full \isi{predication}
\newline 
(declarative) & quotative & left\\
\textit{reach-in \slash  roll-in \slash  walk-in {хладилници}}
\newline 
(thing) & reach-in hladilnici & reach-in refrigerators & *non-predicational
\newline 
(V-prep-N) & pseudo & right\\
\textit{roll-in {хладилници}}
\newline 
(thing) & roll-in
\newline 
hladilnici & roll-in refrigerators & *non-predicational
\newline 
(V-prep-N) & pseudo & right\\
 \textit{walk-in {хладилници}}
\newline 
(thing) & walk-in
\newline 
hladilnici & walk-in refrigerators & *non-predicational
\newline 
(V-prep-N) & pseudo & right\\
\textit{{море-слънце-пясък туризъм}}
\newline 
(thing) & more-slânce-pyasâk turizâm & sea-sun-sand tourism & non-predicational
\newline 
(N-N-N-N) & pseudo & right\\
\textit{{ски-слънце-сняг туризъм}}
\newline 
(thing) & ski-slânce-snyag turizâm & ski-sun-snow tourism & non-predicational
\newline 
(N-N-N-N) & pseudo & right\\
\lspbottomrule
\end{tabularx}
\end{sidewaystable}
\footnotetext[4]{Punctuation and capitalisation are given as they appear in the original sources.}
\footnotetext[5]{The semantic labels  used are those proposed and defined by \citet[161]{Trips2016}. It appears that they are applicable cross-linguistically and can be of value in future contrastive studies. In \ili{Bulgarian} there are not (at present) PCs with heads of the type Time.} 
\footnotetext[6]{ + The semantic labels associated with these phrasal compounds are derived from the noun which their heads elaborate, be it appositively or via a preposition. The actual semantic motivating nouns can be seen in the Appendix.}
\footnotetext[7]{ * These examples are based on deverbal (\textit{-ing}) or constituents used as adjectives with an initial verbal element in the \isi{derivation}. The syntactic status of the non-head constituent in the hybrid constructions is not unequivocal.} 
\footnotetext[8]{ \# This example stands apart from all the others in containing a fully-fledged \isi{relative clause}. The \isi{relative clause} is extremely strange – it is quoted and thus acquires a stereotyping\slash typifying effect. All deictic properties of the constituents of the \isi{relative clause} are cancelled or suspended and the well-wishers are not described as actually experiencing a desire at a particular time t\textsubscript{0} but are classified as members of a particular type or category. This example might be indicative of the way in which protosyntactic packaging or condensing leads to phrasal compounding.}
\addtocounter{footnote}{5}

\subsection{An interpretation of phrasal compounds in Bulgarian}\label{sec:bagasheva:4.3}


In view of the fact that the two editions of the Dictionary of New Words and Meanings in \ili{Bulgarian} (2001 and 2010) register nominal compounds of the \isi{determinative} type with nouns and abbreviations as non-head constituents and that the only available piece of research on PCs on \ili{Bulgarian} reports on a corpus harvested from printed and electronic media for the period 2004–2005, it is plausible to conclude that the development of PCs in \ili{Bulgarian} runs closely behind the establishment of the NN \isi{determinative} compound type.



The establishment of the [N\textsubscript{1}N\textsubscript{2}/N\textsubscript{2}N\textsubscript{1}]\textsubscript{N} root compound type with \isi{determinative}, modifying intracompound relations in \ili{Bulgarian} paved the way for the emergence of phrasal compounds. Once established, the [N\textsubscript{1} N\textsubscript{2}]\textsubscript{N} schema in the constructicon of \ili{Bulgarian} provided the grounds for tolerance of various kinds of  linguistic elements in the N\textsubscript{1} slot (phrases and abbreviations, e.g. \textit{{жп възел} }[žp vâzel, \textit{railway junction}], \textit{{ЕС лидер} }[ES lider] ‘EU leader’, \textit{{МВР център} }[MVR centâr, ‘centre of the Ministry of the Interior’]\textit{, {СДВР шеф} }[SDVR šef, ‘boss of the Sofia Directorate of the Interior’], etc.). In other words, the semantics of the pattern – N\textsubscript{2} of a type somehow related to N\textsubscript{1} – warrants the on-the-spot creation of non-lexicalised phrasal compounds. 

For the time being phrasal compounds are most frequent in lifestyle magazines and in the jargon of tourism. Their establishment in the language is far from complete and results from very specific sociocultural parameters in a weak contact situation. As argued here, what resulted from contact was the establishment of the [N\textsubscript{1}N\textsubscript{2}]\textsubscript{N} pattern whose successful constructionalisation led to the diversification of the construction into phrasal compounds (PCs) and abbreviation compounds (ACs), as we have just seen with the examples above. The users of the language who utilise PCs freely constitute special small speech communities in the sense of \citet{Lipka2002} and \citet{Hohenhaus2005} in relation to vocabulary knowledge and use.

None of the examples analysed here has been recorded in the BulNC to date. Although this is not a fact that can be capitalised on in a conclusive way, it appears that PCs in \ili{Bulgarian} do not have listeme status (to the exclusion of those found in the jargon of tourism). They appear mostly in writing, with the genres and text types restricted to the Psychology, Friends and Advice sections of popular lifestyle magazines. On the basis of these facts it could be argued that the appearance of such structures is a contact phenomenon in its infancy. 


The same used to apply to NN \isi{determinative} compounds that have been termed “an Anglo-Americanism in \ili{Slavic} morphosyntax”  \citep[277]{Vakareliyska2014}. The authors contend that 

\begin{quotation}
since 1990, most of the South and East \ili{Slavic} languages have independently adopted, to varying extents, \ili{English} loanblend [N[N]] constructions, in which an \ili{English} \isi{modifier} noun is followed by a head noun that previously existed in the language, for example, \ili{Bulgarian} ekšŭn geroi  ‘action heroes’ (ibid.). 
\end{quotation}

Yet, as argued by \citet{Bagasheva2016}  from lexical\footnote{A description of the process of lexical (MAT) borrowing going structural (PAT) and the establishment of a new \isi{constructional} type is provided in \sectref{sec:bagasheva:5.2} below.}  (or MAT) borrowing as in \textit{{поп идол}} [pop idol, \textit{pop idol}], via hybrid formations (or loanblends) such as \textit{{екшън герой}} [ekšân geroy, \textit{star from an action movie}], exclusively native root [N\textsubscript{1}N\textsubscript{2}]\textsubscript{N}s of a \isi{determinative} type such as \textit{{ужас}}-\textit{{тръпка}} [užas trâpka, \textit{horror vibe}] and \textit{{чалга певец} }[čalga pevec, \textit{pop folk singer}] established themselves as a new strategy within compounding (or PAT borrowing) via upward strengthening. The abstraction of a new pattern from lexical borrowings gave rise to \isi{constructional} changes in two networks, namely modification and compounding, leading to the constructionalisation of a new compounding strategy. The lexical replication of item-specific borrowings, e.g. \textit{{екшън филм} }[ekšân film, \textit{action movie}] grew into \isi{grammatical} replication as defined by \citet[49]{Heine2006}. This has led to the appearance of one novel type of nominal compound in \ili{Bulgarian} with three subtypes, marked by an asterisk in \figref{fig:bagasheva:1}. The remaining two types are the bahuvrihi [V N]\textsubscript{N} and the synthetic and parasynthetic ones [NV] \textsubscript{+/- suff} N characteristic of \ili{Slavonic} languages.

\begin{figure}
\caption{Types of nominal compound in Bulgarian}
\label{fig:bagasheva:1}
\begin{tikzpicture}[baseline]
\node at (0,0) [anchor=base] (XYN) {[XY] N\textsubscript{[-dyn; -rel]}};
\node [right=2cm of XYN,anchor=base] (N1N2N) {[N\textsubscript{1}+N\textsubscript{2}]\textsubscript{N}};
\node [below right=1em of N1N2N,anchor=base] (AbbrN) {*[Abbr N]\textsubscript{N}};
\node [below left= 2\baselineskip and .33cm of XYN] (VNN) {[VN]\textsubscript{N}};
\node [below = 2\baselineskip  of XYN] (suff) {[NV] \textsubscript{+/- suff} N};
\node [below right= 2\baselineskip and 0cm of XYN] (N2N1N) {*[N\textsubscript{1}N\textsubscript{2}/N\textsubscript{2}N\textsubscript{1}]\textsubscript{N}};
\node [below right= 2\baselineskip and 2.75cm of XYN] (Xphrase) {*[X\textsubscript{phrase/clause} N/NX\textsubscript{ phrase/clause}]\textsubscript{N}};
\draw[-{Triangle[]}] (XYN.south) -- (N1N2N);
\draw[-{Triangle[]}] (XYN.south) -- (AbbrN);
\draw[-{Triangle[]}] (XYN.south) -- (VNN);
\draw[-{Triangle[]}] (XYN.south) -- (suff);
\draw[-{Triangle[]}] (XYN.south) -- (N2N1N);
\draw[-{Triangle[]}] (XYN.south) -- (Xphrase);
\end{tikzpicture}
\end{figure}

The \isi{constructional} network of compounds is represented in \figref{fig:bagasheva:1}. at just two levels of abstraction and the highest node within nominal compounding can only be instantiated by one of the types it sanctions. The first node (from left to right) at the lower level of abstraction [VN]\textsubscript{N} denotes the type constituted by exocentric nominal compounds such as \textit{{развейпрах} }[razveyprah, ‘scatter-dust’, \textit{idler}], \textit{{разтуриколиба}} [razturikoliba, ‘tear-down-hut’, \textit{adulterer}]; \textit{{загоритен\-джере}} [zagoritendžere, ‘burn-pan’, \textit{a person with no sense of time}]. The second, [N V] \textsubscript{+/- suff} N, is actualised by two subtypes: suffixal and suffixless synthetic compounds (which may be semantically endo- or exocentric), e.g. suffixal \textit{{гласопо\-давател}} [glasopodavatel, ‘voice-giver’, \textit{voter}]; \textit{{гробокопач}} [grobokopač, ‘grave-dig-er’, \textit{gravedigger}]; {\textit{данъколпатец} [danâkoplatec, ‘tax-pay-er’,} \textit{taxpayer}]. The suffixless subtype comprises such compounds as \textit{{животновъд}} [životnovâd, ‘an\-i\-mal-breed’, animal breeder]; \textit{{езиковед}} [ezikoved, ‘language-know-er’, \textit{linguist}], etc. The third node, [N\textsubscript{1}N\textsubscript{2}/N\textsubscript{2}N\textsubscript{1}]\textsubscript{N}, branches into \isi{subordinative} (root) nominal compounds exemplified by the \isi{right-headed} \textit{{бинго зала} }[bingo zala, \textit{bingo hall}], \textit{{фитнес салон} }[fitnes salon, \textit{fitness centre}] and the \isi{left-headed} \textit{{гора закрилница} }[gora zakrilnica, ‘forest protector’, \textit{a forest to hide out in}]. The fifth node, [Abbr N]\textsubscript{N}, captures invariably \isi{right-headed} compounds in which the first constituent is an abbreviation as in \textit{{ФБР агент}} [FBR agent, \textit{FBI agent}], \textit{{МВР акция} }[MVR akciya=Ministry of the Interior action, \textit{police operation}], \textit{{ВиК части} }[ViK časti, \textit{plumbing parts}], \textit{{жп възел} }[žp vâzel, \textit{railway junction}], etc. The sixth node is a schematic representation of the construction actualising coordinative nominal compounds such as \textit{{плод-зеленчук} }[plod-zelenčuk, ‘fruit-vegetable’, greengrocer's] and \textit{{архитект-проектант} }[‘arhitekt-proektant’, architect-designer].

Within this nominal network, phrasal compounds in \ili{Bulgarian} bear all the hallmark characteristics that \citet[9]{Heringer1984} ascribes to \textit{episodic compounds}, in circulation mainly within small groups, such as the readership of the \ili{Bulgarian} Cosmopolitan. 



\section{Phrasal compounds in Bulgarian: some generalisations}\label{sec:bagasheva:5}



\subsection{The semantics of phrasal compounds in Bulgarian}\label{sec:bagasheva:5.1}



On the basis of their semantics phrasal compounds should be recognized as composites within the constructicon. As \citet[62-63]{Lampert2009} contends,


\begin{quotation}
two categories relevant for linguistic representations at all levels [...] must […] be kept apart: First, those that result from an ‘additive’ (or: computational) combination of semantically and/or formally simplex items, yielding [...] compositions of variable complexities in accordance with combinatorial rules; second, there are composites, which cannot readily be analyzed in terms of a ‘simple’ (additive) computation of their formal constituents and/or semantic components, but only as ‘wholes’ or Gestalts.
\end{quotation} 

This understanding of composites is fully in keeping with the constructionist understanding of schemas at a medium level of complexity and specificity, and captures two of the outstanding properties of phrasal compounds which warrant them a unique type status among compounds. First, they are not constituted by atomic elements and second, there is no uniform, straightforward computational mechanism which can represent the generation of their meaning (the relation between the constituents cannot be formulated in a linear, descriptive manner). The semantic generalisations concerning the meaning of PCs necessarily employ some marked mechanism, e.g. metonymy \citep{Trips2016}, heavy pragmatic inferencing \citep{Meibauer2015}, metacommunicatively executed stereotyping \citep{Hohenhaus2007}, fictive interaction, encyclopedic and episodic knowledge (...) \citep{Pascual2013}. Phrasal compounds in \ili{Bulgarian} display the same semantic complexities as PCs in other languages. 


For \citet{Pafel2015} quotative PCs behave exactly like [N\textsubscript{1}N\textsubscript{2}]\textsubscript{N} compounds, i.e. their semantics is computable on the basis of an intracompound modification relation as in \isi{determinative} nominal compounds. According to \citet[145]{HeineKuteva2009} there are crosslinguistically four basic types of compound structures with recognised internal relations: 

\begin{itemize}
\item[1)] modifying (tatpurusha) compounds where N\textsubscript{1} is a \isi{modifier} of N\textsubscript{2} (e.g. \textit{dog house}, \textit{{бинго зала} }[bingo zala, `bingo hall']; 
\item[2)] appositive (karmadharaya) compounds where the reference of N is the intersection of N\textsubscript{1} and N\textsubscript{2} (e.g. \textit{singer-songwriter}, \textit{{архитект-проектант} }[arhitekt-proektant, ‘architect-designer’]); 
\item[3)] additive (dvandva) compounds where N encompasses the meanings of both N\textsubscript{1} and N\textsubscript{2} with the latter being distinct referents of N (e.g. \textit{speaker-listener}, \textit{{плод-зеленчук} }[plod-zelenčuk, ‘fruit-vegetable’, \textit{greengrocer's}]); and 
\item[4)] alternative (bahuvrihi) compounds where the meaning of N lies outside the combination of N\textsubscript{1} and N\textsubscript{2} on the basis of metonymy (e.g. \textit{hunchback},  \textit{{загоритенджерa}} [zagoritendžere, ‘burn-pan’, \textit{a person with no sense of\linebreak time}]). 
\end{itemize}

Analysing the semantics of PCs in \ili{Bulgarian} indicates that they behave like modifying compounds. The lexicalised \textit{{море-слънце-пясък туризъм} }[more-slân\-ce-pyasâk turizâm, ‘sea-sun-sand tourism’] names a specific type of relaxation or tourism associated with summertime and specific activities relating to the availability of all three prerequisites, sea, sun, and sand. The same semantic interpretation applies to \textit{walk-in {хладилници}} [walk-in hladilnici, ‘walk-in refrigerators’] which denotes a specific type of refrigerating facility where one can walk in and pick out the required foodstuffs. The modifying relationship is fairly straightforward and these PCs look a lot like ordinary modifying nominal compounds except as regards the structure of the non-head constituent. Their semantics can be explained by what \citet[171]{Trips2016} suggests “R(x1,x2)”, i.e.  “[t]he only thing which is specified is that there is a relation between” the head and the non-head constituent. The only qualification to make is that unlike in \ili{English}, in \ili{Bulgarian} the type of relation involved is easily read off the compound. It is likely that the further spread and expected heightened productivity of PCs in \ili{Bulgarian} the nature of the relation (R) will become as diversified and underdetermined as in \ili{English}.

Non-lexicalised PCs, such as \textit{{предизвикай-го-да-говори-съвет} } [predizvikay-go-da-govori-sâvet, ‘coerce-him-into-talking-advice’] or \textit{{напрежението ето}-{че-моментът-най-сетне}-{настъпи}} [napreženieto eto-če-momentât-nay-setne-nas\-tâpi, ‘the-tension there-the-time-has-finally-come’] and \textit{{направи-си-сам-проб\-леми-в-офиса ситуация}} [napravi-si-sam-problemi-v-ofisa situaciya, ‘create-\linebreak your-own-problems-at-the-office-situation’], irrespective of the variation in headedness (left- or \isi{right-headed}) and the semantic type of the head element (Attitude and Action), display the same semantics, i.e. name a specified or stereotyped type of the head. The listener\slash reader will understand the tension in \textit{{напрежението ето}-{че-моментът-най-сетне}-{настъпи}} [napreženieto eto-če-momentât-nay-\linebreak setne-nastâpi, ‘the-tension there-the-time-has-finally-come’] as a specific type of tension that one experiences in the typified circumstances. The presumption that comprehension of  the PC depends on the “hearer’s knowledge of context, background, stereotypes, etc., adhering to pragmatic principles such as Gricean maxims” (Meibauer, quoted after \citealt[169]{Trips2016}) seems to be sufficient for explaining the functioning of \ili{Bulgarian} PCs in context. However, from the speaker’s perspective, this cannot explain how PCs are produced on the fly. Adopting \citegen[172-176]{Trips2016} account of the semantics of predicational PCs in \ili{English} provides a plausible explanation for the meaning computation of PCs in \ili{Bulgarian} from the speaker’s perspective. The “IS-A” relationship and metonymy-based typeshifts account in a satisfactory manner for the semantics of predicational PCs. As \citet[174]{Trips2016} specifies “the phrasal non-head is an utterance which undergoes one (or more) typeshift(s) either from UTTERANCE to THING […] or from UTTERANCE to EVENT(UALITY).” In this manner the necessary stereotyping of the situation described in the phrasal non-head is achieved.


A detailed semantic analysis of PCs in \ili{Bulgarian} was beyond the objectives and scope of the current paper, yet in the process of checking their properties against established compounds in other languages, an analysis of meaning generating principles was inevitable and this led to the conclusion that the elements in the \isi{modifier} slot function as situational proverbs. They name situations or event types but do not describe a discourse-anchored situation. The non-predicational ones denote entities of well-specified types.

\subsection{Can phrasal compounds in Bulgarian be the result of pattern borrowing?}\label{sec:bagasheva:5.2}


The similarities between \isi{subordinative}, modifying nominal compounds and PCs seem to go beyond their semantics. Like \isi{subordinative}, nominal compounds, phrasal compounds have arisen as the result of language contact. As \citet[1856]{Ohnheiser2015} has noted,

\begin{quotation}
to a significant extent the increase of new vocabulary in the modern \ili{Slavic} languages feeds on borrowings and loan translations or hybrid formations, a large part of which consist of \textit{compound patterns} and compound elements of foreign origin (\citealt[1856]{Ohnheiser2015}; emphasis added).
\end{quotation}


Both \isi{subordinative}, modifying [N\textsubscript{1}N\textsubscript{2}/N\textsubscript{2}N\textsubscript{1}]\textsubscript{N} compounds and phrasal nominal compounds in \ili{Bulgarian} probably result from this recognised influx of foreign compound patterns. 

In contact linguistics it has become common to make a distinction between direct borrowing of lexical and phonetic material and structural borrowing or \isi{grammatical} replication. As \citet[15]{Sakel2007}  claims, \begin{quotation}
we speak of MAT-borrowing when morphological material and its \isi{phonological} shape from one language is replicated in another language. 
\end{quotation}

Similarly \citet[49]{Heine2006} differentiate between lexical and \isi{grammatical} replication on the basis of borrowing or not of form or phonetic substance. Grammatical replication for them is recognised when a replica language “creates a new \isi{grammatical} structure (Rx) on the model of some structure (Mx) of another language” \citep[49]{Heine2006}. For \citet[15]{Sakel2007}, 
\begin{quotation}
PAT describes the case where only the patterns of the other language are replicated, i.e. the organisation, distribution and mapping of \isi{grammatical} or semantic meaning, while the form itself is not borrowed.
\end{quotation}

Whichever terminology we adopt, it appears that from the perspective of constructionalism the development of atypical constructions should be recognised as PAT borrowing or \isi{grammatical} replication. Crucially, it may be the case that in borrowing constructions of medium complexity the two types of borrowing are merely successive steps in the process of constructionalisation. 

When we combine the understanding of constructionalisation as a kind of language change with the typology of borrowing it transpires that the establishment of root, \isi{subordinative} nominal compounds in \ili{Bulgarian} is an example par excellence of constructionalisation in a contact situation. In keeping with \citegen{Traugott2013} postulations, the emergence of \isi{constructional} nodes with schematic slots is what is captured with the term “\isi{grammatical} constructionalisation”. In the development of [N\textsubscript{1}N\textsubscript{2}]\textsubscript{N} constructions in \ili{Bulgarian} exactly this process is observed: upward strengthening of a construction with schematized nodes. Importantly, \citet[22]{Traugott2013} emphasise that constructionalisation is associated with the emergence of a new node in a \isi{constructional} network, or more specifically “[…] when constructs begin to be attested which could not have been fully sanctioned by pre-existing \isi{constructional} types.” As the compounding network in \ili{Bulgarian} did not tolerate root, \isi{subordinative} [N\textsubscript{1}N\textsubscript{2}]\textsubscript{N}s before the influx of lexical borrowings from \ili{English} (\citealt{Brezinski2012}, \citealt{Krumova-Cvetkova2013}, \citealt{Murdarov1983}, \citealt{Radeva2007}, etc.), it is natural to conclude that the rapid spread of [N\textsubscript{1}N\textsubscript{2}]\textsubscript{N}s in the language and their development from lexical borrowing via loanblends into fully integrated constructions with native constituents is an illustration par excellence of MAT borrowing gone PAT \citep{Sakel2007}. The real question is whether phrasal compounds in the language follow the same scenario or whether they resulted from the further diversification of the already nativised compound pattern.

While [N\textsubscript{1}N\textsubscript{2}]\textsubscript{N}s have already firmly developed into PAT borrowing (hundreds of them being found with purely native constituents and duly recorded in lexicographic reference books), PCs are still a borderline phenomenon (in view of their restricted genre and text type appearance). However, the attested PCs in \ili{Bulgarian} cannot be interpreted as direct lexical borrowings or calques as no corresponding potential \ili{English} sources easily suggest themselves. This suggests that the paths of development of root, \isi{subordinative} nominal compounds and PCs in the language differ.


\citet[116]{Hilpert2015} contends that 
\begin{quotation}
whereas in grammaticalization, the experience of a linguistic unit leads to the progressive entrenchment of a more schematic construction, situated at a higher level in the \isi{constructional} network, \isi{constructional} change can manifest itself in the strengthening of several more specific sub-schemas, at lower levels of the \isi{constructional} network. This proposal will be called the upward strengthening hypothesis. 
\end{quotation}
  
  Following this hypothesis, the establishment of the new root, \isi{subordinative} compound type in \ili{Bulgarian} [N\textsubscript{1}N\textsubscript{2}]\textsubscript{N} should be interpreted as the result of upward strengthening, while that of PCs is the result of \isi{constructional} change. The newly established compound type acted as fertile soil in which phrasal compounds could be planted. The planting was aided by the peculiar formal and functional properties of phrasal compounds and the general meaning underspecification of nominal modification  \citep{Bauer2013}. 

To go a step further, we can specify that both \isi{subordinative}, modifying nominal compounds and PCs are the result of \isi{pattern borrowing}, not process borrowing. The distinction between the two types of borrowing, according to \citet{Renner2015}, is based on the degree to which the borrowing affects the receptor or replicator language’s extant constructions. \citeauthor{Renner2015} defines “contact-induced morphostructural change as all contact-induced morphological changes beyond the copying of a morpheme, i.e. the novel availability or increased profitability of a WF process or pattern caused by language contact” \citep{Renner2016}. The novel availability of a word-formation process he dubs process borrowing, while the increased profitability of a word-formation process or model is recognised as \isi{pattern borrowing}. The latter results in moderate structural change, with the core of the receptor language’s system remaining unaffected. The borrowing of \isi{subordinative} nominal compounds from \ili{English} into \ili{Bulgarian} did not lead to the introduction of a novel word-formation process, but enhanced the profitability of the existing but marginal [N\textsubscript{1}N\textsubscript{2}/N\textsubscript{2}N\textsubscript{1}]\textsubscript{N} compound pattern.


Admittedly, this scenario for the appearance of PCs is in keeping with the development of nominal compounds as suggested by  \citet{HeineKuteva2009}. It is possible that in the grammaticalisation chain of the combination of nouns another step may have to be added – after the fixation of the pattern into a compound, the modifying, non-head slot may tolerate other structural constituents which have been functionally downgraded (without any suggestions as to the nature of the downgrading mechanism) to acquire noun-like properties. This possibility does not violate or compromise the nature of compounding, since as \citet[35]{Gaeta2009} suggest compounds may be analyzed “by treating the properties of being a lexical unit and being the output of a morphological operation as independent”. Moreover, compounding has been recognised as a `pocket phenomenon' in language (\citealt{Bauer2001}; \citealt{Jackendoff2009}), where the rules of \isi{syntax} do not apply. 

Another plausible scenario, driven by  economy principles and the minimax effect in language functioning is the condensation of phrasal\slash clausal constituents to elements of word building. In this case, the ratio of explicitness\slash implicitness is manipulated, so that a large amount of descriptive information is left out and delegated to pragmatic inferencing to achieve a labelling effect (the result of stereotyping\slash typifying).  

Both scenarios are compatible with the understanding of language as a constructicon. After all, there seem not to be any restrictions as to the size or complexity of a construction (to the exclusion of psycholinguistic considerations of processing limitations), which allows for the existence of phrasal compounds which collapse features of different traditional structural elements. 


\section{Concluding remarks}\label{sec:bagasheva:6}

One of the basic functions of word-formation objects, i.e. words, is the categorising function \citep{Schmid2007}, tightly interwoven with the entrenchment of concepts. This is further supported by \citeauthor{Bolozky1999}’s belief that “lexical formation is first and foremost semantically based and concept driven” \citep[7]{Bolozky1999}. 

  Conceding that phrasal compounds have a unanimously acknowledged naming function leads to recognising their concept-creating or at least strongly typifying function, which is cognitively speaking tantamount to establishing a category. In the words of \citet[356]{Hohenhaus2005} ``Hypostatization is a side-effect of the naming function of word-formation, whereby the existence of a word seems to imply for speakers the existence in the real world of a single corresponding ‘thing’ or clearly delimited concept.'' 

In a nutshell, the greatest driving force behind the (still limited) advent of phrasal compounds in \ili{Bulgarian} is their type-creating power, be it metacommunicative, metonymy-driven or fictive. The pattern has been established and its instantiations share all properties of standard phrasal compounds in \ili{English} and \ili{German}, to the exclusion of headedness variability, which is characteristic of \ili{Bulgarian} phrasal compounds only. Their smooth accommodation in the compounding network of the language can easily be explained in terms of the ratio between explicitness and implicitness which they provide as subschema instantiations of the \isi{subordinative}, modifying \isi{constructional} node within the compounding network. 


In parallel to Ray Jackendoff and Eva Wittenberg’s interlinguistic hierarchy of grammars \citep{Jackendoff2012}, we propose that there is a similar intra-language hierarchy of meaning packaging options whose choice depends on at least the following variables: genre, immediate situational context, speaker’s preferences and linguistic background and the mode of interaction between interlocutors, which would determine the degree of explicitness necessitated in a given communicative exchange. Standard phrasal \isi{syntax} and compounds are seen as alternative modes of packaging following different internal logics. In keeping with \citegen{Jackendoff2009} contention that in compounds proto-syntactic combinatorial patterns prevail, we believe that the \isi{syntax} of a language has only an indirect influence on the shape and types of compounds in a given language mediated by the part-of-speech system with the concomitant inflectional morphology. Proto-\isi{syntax}, as the alternative name for “a simpler grammar”, is characterized according to \citet{Jackendoff2009} and  \citet[1]{Jackendoff2012} as an expression system which puts “more responsibility for understanding on pragmatics and understanding of context. As the grammar gets more complex, it provides more resources for making complex thoughts explicit.” Even though \citeauthor{Jackendoff2012} define the “hierarchy of grammars” as a continuum along which the \isi{grammatical} systems of languages with different degrees of complexity can be arranged, we assume that it is possible for the different resources of a single language to be arranged into a grammar hierarchy, where different patterns for packaging meaning display properties that can be arranged along the scales of complexity and explicitness. When a compound is used, the relation of what is explicitly expressed to possible interpretations is effected by pragmatics and general experiential knowledge.


Answers to the questions raised in the introduction can only be provided after a longitudinal or cross-sectional study of PCs in \ili{Bulgarian} is conducted within a decade and hopefully this is a promising continuation of the ongoing research reported here.


\section*{Appendix: Phrasal compounds in \ili{Bulgarian} in context}\label{sec:bagasheva:6.1}


The majority of examples (21 to be precise) have been taken from \citet{Boyadžieva2007} which presents a corpus study of Cosmopolitan BG for the period 2004-2005 is presented.



\ea%1
    \label{ex:bagasheva:1} 

         {{{Свали-го-съвет}}} [Cosmopolitan, BG, March 2005: 113]
    \z



\ea%2
    \label{ex:bagasheva:2} 
         {{{Вземи}}}{{-}}{{{му}}}{{-{акъла-съвет}}}{ }[Cosmopolitan, August 2005: 45]
    \z



\ea
{{{Завърти-}}}{{{му}}}{{-{ума-посрещане}}}{ }{[Cosmopolitan, BG, www]}
\z


\ea
{{Промени-живота-си-предизвикателство}}{ }{[Cosmopolitan, BG, www]}
\z


\ea%5
    \label{ex:bagasheva:5} 

         {{{Прочети-}}}{{{му}}}{{-{мислите съвет} }}[Cosmopolitan, BG, September 2004: 60]
    \z



\ea%6
    \label{ex:bagasheva:6} 

         {{{Предизвикай-го-да-говори-съвет}}}{ }[Cosmopolitan, BG, December 2004: 62]
\z


\ea 
{{{Често си спретват игри от} типа кой-пръв-ще-успее}-{да-пъхне-лед-в-ризата-на-другия}}{. }{[Cosmopolitan, BG, February 2005: 72]}
\z



\ea
{{Тя може да те спаси в кризисни моменти – от неотложна нужда за посещение на зъболекар до внезапната поява на блуза,} }{{{тип не-е-за-изпускане-независимо-от-цената}}}{. }{[Cosmopolitan, BG, www]}
\z


\ea
{{Това определено не се простира отвъд} }{{{факта Боже-не-може-да-бъде}}!, {харесвате едни и същи ястия и филми.}}{ [Cosmopolitan, BG, September 2004: 53]}
\z


\ea
{{Ако си една средностатистическа жена, няма как стресът да не е станал второто ти Аз – без значение дали става дума за леката му} }{{{разновидност ‘Боже-сетих-ли-}}}{{{се}}}{{-{да-изключа-котлона-днес’}}} {{или за} }{{{язво-формиращия вариант ‘ей-тази-седмица-}}}{{{наистина}}}{{-{май-ще-ме-уволнят-или-зарежат-или-и-двете}}}{. }{[Cosmopolitan, BG, www]}
\z


\ea
{{Този антидепресант повишава нивото на допамин в мозъка, като по този начин осигурява един от най-коварните ефекти на цигарите –} }{{{да-}}}{{{се}}}{{-{почувстваш-}}}{{{добре ефект}}}{. }{[Cosmopolitan, BG, July 2004: 129]}
\z


\ea{{{Така се елиминира} напрежението ето}-{че-моментът-най-сетне}-{настъпи}}{. }{[Cosmopolitan, BG, September 2005: 63]}


\z


\ea {{Изпускането на парата твърде скоро може да е равносилно на} }{{{направи-си-сам-проблеми-в-}}}{{{офиса}}}{ {{ситуация} }}{. }{[Cosmopolitan, BG, December 2004: 105]}


\z


\ea{{Как да доведеш нещата до щастлива} }{{{развръзка за-вечни-времена}}}{.}{ [Cosmopolitan, BG, www]}


\z


\ea{{Целта е да преодолееш тютюнджийската абстиненция, докато се откачиш от} }{{{навика с-цигара-в-ръка}}}{. }{[Cosmopolitan, BG, December 2004: 104]}


\z


\ea{{Едва ли си даваш сметка, до каква степен лошият режим на хранене и} }{{{предъвкването-}}}{{{между}}}{{-{другото}}} {{на разни дреболии пречат на диетата ти.} }{[Cosmopolitan, BG, September 2005: 119]}

\z


\ea{{Като начало,} {молим ви}, {не си отглеждайте} }{{{стърчащо-услужливо-дълго-нокътче} на кутрето. Поддържайте ноктите си добре подрязани и чисти.} }{[Grazia, BG, September 2004: 49]}


\z


\ea{{Забелязала ли си, че ти се иска да си купиш нещо почти веднага след скандал с} }{{{иначе-}}}{{{любимия}}}{{-{човек}}}{? }{[Cosmopolitan, BG, September 2005: 73]}


\z


\ea {{Даже в по}-{либералните фирми и най-дребният намек} }{{“{има нещо помежду им”}}}{ {кара} }{{{колежките-кобри-по-душа}}} {{да изпълзят от леговищата си.}}{[Cosmopolitan, BG, December 2004: 104]}


\z


\ea{{Просто-независеща-от-него}-{странност} }{[COSMO men, August 2005: 4]}


\z


\ea {{Ако шопингът все пак ти действа като мощна доза антидепресанти, има решение и то се нарича} }{{{обикаляне-без-купуване решение}}}{. }{[Cosmopolitan, BG, September 2005: 73]} 


\z


\ea{След всички мнения какво трябва и не трябва да правиш, когато си на 20, главата ти се замайва и ти иде да се скриеш далеч от всички тези съветници и} {{доброжелатели, които „само искат да знаят какво мислиш за бъдещето си“}}{ (}\url{http://www.cosmopolitan.bg/cosmo-zapovedi/11-neshta-za-koito-da-ne-se-obviniavash-na-20-18537.html}{, Cosmolitan BG 2016, last accessed 14 July 2016)}

\z


\ea{{Функцията „На този ден“} }{{е създадена, само за да те излага. (}}http://www.cosmopolitan.bg/svetut-okolo-teb/16-feisbuk-problema-koito-20-godishnite-sreshtat.html{, Cosmolitan BG 2016, last accessed 14 July 2016)}


\z


\ea {{За съжаление дори възрастните понякога си правят детински шегички и нищо чудно да осъмнеш със} }{{статус „Обичам мазни, потни чичковци“.}} (\url{http://www.cosmopolitan.bg/svetut-okolo-teb/16-feisbuk-problema-koito-20-godishnite-sreshtat.html}, Cosmolitan BG 2016, last accessed 14 July 2016) \z



\ea%25
    \label{ex:bagasheva:25} 

         {reach-in {хладилници} }(Horeva, Ph.D., manuscript, Sofia University, 2015)
    \z



\ea%26
    \label{ex:bagasheva:26} 

         {roll-in {хладилници} }(Horeva, Ph.D., manuscript, Sofia University, 2015)
    \z



\ea%27
    \label{ex:bagasheva:27} 

         {walk-in {хладилници} }(Horeva, Ph.D., manuscript, Sofia University, 2015)
    \z



\ea%28
    \label{ex:bagasheva:28} 

         {{море-слънце-пясък туризъм} }(Horeva, Ph.D., manuscript, Sofia University,    2015)
    \z


\ea{ {ски-слънце-сняг туризъм} }(Horeva, Ph.D., manuscript, Sofia University, 2015)
\z

 

{\sloppy
\printbibliography[heading=subbibliography,notkeyword=this]
}

\end{document}


