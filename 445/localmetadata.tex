\author{Jean Rohleder}
\title{A grammar of Vamale}
\subtitle{}
\renewcommand{\lsSeries}{cogl}
\renewcommand{\lsSeriesNumber}{9}
\renewcommand{\lsID}{445}
\renewcommand{\lsISBNdigital}{978-3-96110-479-6}
\renewcommand{\lsISBNhardcover}{978-3-98554-108-9}
\BookDOI{10.5281/zenodo.12606066}

\BackBody{Vamale is an endangered South Oceanic > Northern New Caledonian language, spoken by around 180 people on the northeastern coast of Grande Terre. This grammar was written as a PhD dissertation, on the basis of 11 months of fieldwork funded by ELDP. The data consists both of elicitation and relatively free interviews, as well as recordings of ceremonial speeches and casual conversations. ELAR contains open-access archive of all recordings and a dictionary, as well as a FLEx database in which many examples can be found in context. The appendix includes three texts, an oral history account of the 1917 colonial war, a traditional fable, and a longer modern retelling of a legend. The grammar intends to give a general overview of Vamale to a general linguistics audience. Its focus on syntax, and comparison with related languages should particularly interest Oceanists and areal typologists. With a dedicated chapter on the community's history and cultural information throughout the book, this account hopes to show the beauty and wealth of both the Vamale language and culture.}

\typesetter{Jean Rohleder}

\proofreader{Benjamin Brosig,
Carrie Dyck,
Elliott Pearl,
Ezekiel Bolaji,
Jeroen van de Weijer,
Katja Politt,
Lachlan Mackenzie,
Jean Nitzke,
Mary Ann Walter,
Philip Duncan,
Tabea Reiner,
Tim Ongenae,
Yvonne Treis
}
