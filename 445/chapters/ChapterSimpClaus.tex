% i want this section to say how i work and why, and what problems and particularities i encountered in the field. 

\chapter{Simple clauses} 
\label{ChapterSimpClaus} 
\is{Clauses!Simple clauses}
This chapter treats simple clauses and the elements they contain, and describes how a matrix clause is formed. As this is the first chapter to discuss matrix clauses and the elements that appear outside of verb and noun phrases, particles such as assertive \textit{tha} (\sectref{sec:ass}) and the repetitive marker \textit{mwa} (\sectref{sec:mwa}) are described here, as well as the discourse markers \textit{go} (\sectref{ssec:discourse go}) and \textit{ka} (\sectref{discourse ka}). Word order is also explicitly mentioned here for the first time (\sectref{sec:alignment}), as this is the first chapter to look at the domain of the clause. %will first introduce Vamale word order and describe the prevalent alignment patterns, before describing
Simple clauses have one main predicate, one subject, one TAM contour, and share a single polarity. A simple clause may contain several verbs, in the case of serial verb constructions (\sectref{sec:SVC}). The main clause types are declarative (\sectref{sec:decl_clauses}), interrogative (\sectref{sec:Interr_clauses}), imperative and prohibitive clauses (\sectref{sec:IMP_clauses}). A brief section addresses equative clauses, which in absence of a copula juxtapose two nouns (\sectref{ssec:Verbless_clauses}). %\sectref{sec:} on declarative xx
As the assertive \textit{tha} is discussed in this chapter, other modal particles and constructions are mentioned as well (\sectref{sec:Modality}). Most of them involve subordination, which is described separately again in \sectref{ssec:ma}. A typical sentence looks as follows:
\begin{center}
PN= TAM= root -\gl{obj}/-\textit{ke} (\gl{art}=) \gl{obj} (\textit{ka}= (\gl{art}=) NP\textsubscript{subject})
\end{center}
\noindent Noun phrases do not necessarily take articles, and argument expression through noun phrases is optional, as indexing on the verb already carries this information.

\section{Word order}
\label{sec:alignment}
\is{Word order}
\is{Clauses!Word order}
Word order in Vamale does not change across clause types. Vamale is head-first and usually VOS, though fronted, bi-clausal constructions are common: S, VO. The verb always precedes the object, and the traditionally non-marked order expects the subject after the noun. Oblique arguments tend to follow objects (\ref{ex:obj_then_obl}), but pragmatic choices can invert the unmarked order (\ref{ex:obl_then_obj}). In general, the earlier a constituent is introduced, the more marked it is. %The subject is often fronted, but may then still occur after the verb, indicating that a fronted subject is not a constituent of the clause.\footnote{Prosodically, too, the fronted subject has an own contour, which yields two intonation units: the fronted subject, and the clause.} Fronting can be used to focus on any constituent, a resumptive morpheme remains in the matrix clause.



\ea\label{ex:obj_then_obl}
\gll e=holeke {\ob}sikaa{\cb}\textsuperscript{obj} {\ob}nyasi i=apuli a= xhwata{\cb}\textsuperscript{obl}\\ 
 1\gl{sg}=thank.for cigarette \gl{ben} \gl{def}.\gl{sg}=man \gl{rel}= bald\\ 
\glt \qu{I thank the bald man for the cigarette.}
 \z
 
\ea\label{ex:obl_then_obj}
\gll e=holeke nyasi-m li=fati a= go=vi\\ 
 1\gl{sg}=thank.for \gl{ben}-2\gl{sg}.\gl{poss} \gl{def}.\gl{pl}=word \gl{rel}= 2\gl{sg}=say\\ 
\glt \qu{I thank you for the words.} {[B2:94]}
\z

%\a
%
%\gll  yo e=bwa ja fa-xaleke i ye nyako i apuli
% 1\gl{sg} 1\gl{sg} \gl{ipfv} ja \gl{caus}-see \gl{def}.\gl{sg}=tree \gl{obl} \gl{def} man
%\glt \qu{I have been making the man look at the tree.}
%


%\todo{I would add a gloss line just beneath the Vamale elements}
%%todo{test examples}
The basic word order VOS is generally more common with older speakers, while younger generations often front the subject. Stative verbs, especially with inanimate arguments such as in (\ref{ex:xhong}), seem to be exempt from the fronting mentioned above, regardless of the speaker demographic. With animate arguments, fronting is especially common, especially when the agent is stressed. Strategies to make the subject less salient to the listener include using the VOS order, as well as not mentioning the subject, and fronting the object, as in (\ref{ex:fronted_obj}).

\ea \label{ex:xhong}
\gll sinu i=xho-ng\\ 
 ill \gl{def}.\gl{sg}=leg-1\gl{sg}.\gl{poss}\\ 
\glt \qu{My leg hurts.}
\z


	
%	\ea\label{ex:itrs_sbj}
%	
%	\gll e=ta ka yo
%	 1\gl{sg} go.up \gl{sbj} 1\gl{sg}
%	\glt \qu{I go up}
%	
%	
%	\ea\label{ex:1sg_sbj}
%	
%	\gll yo, e=xale-ko (ka yo)
%	 1\gl{sg} 1\gl{sg} see-2\gl{sg}.\gl{obj} \gl{sbj} 1\gl{sg}
%	\glt \qu{Me, I see you}
%	
%	\ea
%	
%	\gll go ta
%	 2\gl{sg} go.up
%	\glt \qu{You go up}
%	
%	\ea\label{ex:2sg_sbj}
%	
%	\gll go, go xale-o
%	 2\gl{sg} 2\gl{sg} see-1\gl{sg}.\gl{obj}
%	\glt \qu{You, you see me}
%	
\ea \label{ex:alignment}	
\label{ex:fronted_obj}
	\gll go, e=xale-ko \\ 
	 2\gl{sg} 1\gl{sg}=see-2\gl{sg}\\ 
	\glt \qu{You, I see you.}
	\z
	
	\ea\label{ex:non-fronted_obj}
	\gll e=xale-ko \\ 
	 1\gl{sg}=see-2\gl{sg}.\gl{obj}\\ 
	\glt \qu{I see you.}
	\z 
	
%\footnotetext{Formerly preferred, see \Cref{text:tipije}}
%\footnotetext{Preferred now}

%Stative verbs use different morphemes to mark their subjects, and those suffixes are not identical (although very similar) to object-marking suffixes found on transitive verbs, as in (\ref{ex:1sg_sbj}). Inanimate participants are neither marked with object suffixes nor stative subject suffixes, but otherwise behave like animate ones, i.e. inanimate noun phrases are flagged as subjects and obliques. Both non-specific objects (irrespective of animacy) and inanimate non-specific stative subjects are marked with \textit{-n} \qu{\gl{nspec}} on the verb. This distributional similarity between inanimate objects and stative subjects, as well as the formal similarity of the -\gl{S\textsubscript{P}} and -\gl{p} suffixes raises questions concerning an animate/inanimate split. 
%This means that Vamale uses a split-S system in its person marking, marking transitive verbs like active intransitive ones, distinguishing stative verbs from the former set, and marking objects yet differently. A table summarizing the forms is given in \Cref{ChapterVerbs} (see \Cref{tab:markers2}), and is repeated here as \Cref{tab:markers3}.

%\begin{table}
%	\centering
%	\caption{Subject and object markers for active and stative verbs}
%	\label{tab:markers3}
%	\begin{tabular}{lllll}
%		&	Free form	& \gl{a}=/\gl{S\textsubscript{A}}= & -\gl{S\textsubscript{P}} & -\gl{p}\\
%		1\gl{sg} & \textit{io} & \textit{e} & \textit{-o(ng)} & \textit{-o} \\
%		1\gl{du}.\gl{incl}& \textit{gasu} & \textit{gasu} & \textit{-gasu} & \textit{-kaeu}\\
%		1\gl{pl}.\gl{incl} & \textit{gaa/gase} &\textit{ga(se)}&\textit{gaa}&\textit{-kaa}\\
%		1\gl{du}.\gl{excl} & \textit{abu} & \textit{abu} & \textit{-abu} & \textit{-(a)bu}\\
%		1\gl{pl}.\gl{excl} & \textit{abe}& \textit{abe} & \textit{-abe} & \textit{-(a)be}\\
%		2\gl{sg} & \textit{go} &\textit{go} & \textit{-go} & \textit{-ko}\\
%		2\gl{du} & \textit{gau} & \textit{gau} & \textit{-gau} & \textit{-kau}\\
%		2\gl{pl} &\textit{gavwe}& \textit{gavwe} & \textit{-gavwe} & \textit{-kavwe}\\
%		3\gl{sg} & \textit{ia} & \textit{a} & \textit{-(e)a} & \textit{-(e)a}\\
%		3\gl{du} & \textit{lu} &\textit{lu} & \textit{-lu} & \textit{-lu}\\
%		3\gl{pl} & \textit{le} & \textit{le} & \textit{-le} & \textit{-le}\\
%	\end{tabular}
%\end{table}
%

%\ea
%%\a
%%
%%\gll yo, a xale-o ka ya
%% 1\gl{sg} 2\gl{sg} see-1\gl{sg} \gl{agt} 3\gl{sg}
%%\glt \qu{Me, he saw me}
%% 
%
%\a
%
%\gll a xale-o ka i pupwaale
% 3\gl{sg} see-1\gl{sg} \gl{cnj} \gl{def} Caucasian
%\glt \qu{The white guy sees me}
%
%
%
%\a
%
%\gll a-xale-a i pupwaale ka i daahma
% 3\gl{sg} see-3\gl{sg}.\gl{obj} \gl{def} Caucasian \gl{agt} \gl{def} chief
%\glt \qu{The chief sees the White guy}
%
%
%\a
%
%\gll *a xale-a i pupwaale i daahma
% 3\gl{sg} see-3\gl{sg}.\gl{obj} \gl{def} Caucasian \gl{def} chief
%\glt \qu{(The chief sees the White guy)}
%
%
%\a
%
%\gll a majit i pupwaale
% 3\gl{sg} rest \gl{def} Caucasian
%\glt \qu{The white guy sleeps}
%
%\z 

%\ea 
%\ea
% 
%\gll a xale-o\textsuperscript{Sᵃ}
% 3\gl{sg} see-1\gl{sg}.\gl{obj}
%\glt \qu{He sees me.}
% 
%
%\ea
% 
%\gll sinu-ong\textsuperscript{Sᵖ}
% sick-1\gl{sg}.\gl{sbj}
%\glt \qu{I am sick/dying.}
% 
%
%\ea
% 
%\gll a majit
% 3\gl{sg} sleep
%\glt \qu{He sleeps.}
%
%
%\ea
% 
%\gll {*}a thathe
% 3\gl{sg} wound
%\glt \qu{(he is wounded)}
% 
%\z

%Nowadays, the word order is more often A, a=V P than a=V P ka A, so that there is no difference between A and S, and we have a nominative/accusative system for nouns as well.


\section{Clause types}
\label{sec:clause_types}
\is{Clause types}

Main clauses in this grammar are those who routinely appear alone and cannot be governed by another clause without subordinating morphology. This includes declarative clauses both with verbs and equative ones with nouns, as well as interrogative, imperative, and prohibitive clauses.
The morphosyntax of main (or matrix) clauses is the same as that of most subordinate ones, a topic discussed in further detail in \Cref{ChapterSub}. Exceptions to this are imperative clauses, which do not index the subject on the verb. 


\subsection{Declarative clauses}
\label{sec:decl_clauses}
\is{Clauses!Declarative clause}
Declarative clauses are the most common ones in Vamale. They usually contain a subject, either present as a proclitic subject marker, for active verbs and nominal predicates, or as a suffix, for stative verbs with an animate subject, and which can always be expressed as a free noun phrase after the verb phrase, flagged with \textit{ka} \qu{\gl{sbj}}, or fronted (in which case it is its own clause, prosodically distinct and syntactically not integrated). Exceptions which do not need a subject are impersonal verbs like \textit{vwasoon} \qu{impossible} (discussed in \sectref{sec:ZeroTrans}), and imperative clauses.

\subsubsection{Discourse marker \textit{ka}}
\label{discourse ka}
\is{Discourse markers!Discourse marker \textit{ka}}

Putting \textit{ka} at the end of a sentence is a question to the listener: do you follow? This \textit{ka} (\ref{ex:coorka1}) is probably related to \textit{ka} \qu{\gl{cnj}}.

	\ea \label{ex:coorka1}
	% \langinfo{}{}{} 
	\gll jaa nyu ka\\ 
	 many fish \gl{disc}\\ 
	\glt \qu{way too many fish, like} {[KL:249]}
	\z 
	
	\ea
	% \langinfo{}{}{} 
	\gll e=vii abe li=juu apuli ka\\ 
	 1\gl{sg}=say 1\gl{pl}.\gl{excl} \gl{def}.\gl{pl}=real human \gl{disc}\\ 
	\glt \qu{I mean us Kanaks, like.} {[KP:73]}
		\z


\subsubsection{Discourse marker \textit{go}}
\label{ssec:discourse go}
\is{Discourse markers!Discourse marker \textit{go}}

One of the most common morphemes found on a clausal level is \textit{go} \qu{well, now, and then}. It cannot be used to coordinate noun phrases nor verb phrases, and occurs before coordinators, assertive \textit{tha}, and other left-most boundaries of clauses (\ref{ex:go}). It is a frequent filler, stretched out to prolong the thinking time of the speaker (\textit{gooooo...} \qu{weeell}). Its meaning does not carry information about the relationship of the surrounding clauses: while in many cases, the clause following \textit{go} \qu{and then, furthermore} does semantically follow the previous ones (\ref{ex:go2}), \textit{go} \qu{now, on the other hand} can also mark a change of subject.


\ea\label{ex:go}
\gll go tha le=vwa goo i=\textit{coutume} a= tha le=vwa \\ 
 \gl{disc} \gl{ass} 3\gl{pl}=do enough \gl{def}.\gl{sg}=custom \gl{rel}= \gl{ass} 3\gl{pl}=do\\ 
\glt \qu{Well, they kept the ceremonies that they did, moderate.} {[KP:6]}
\z


\ea\label{ex:go2}
\gll go ka e=ilake hapi gavwe wago\\ 
 \gl{disc} \gl{cnj} 1\gl{sg}=ask \gl{comp} 2\gl{pl} brave \\ 
\glt \qu{Well, and I ask you to be brave.} {[HC19:20]}
\z

\subsubsection{Discourse marker \textit{bwa}}
\label{sec:bwa_disc}

Similarly to \textit{go}, \textit{bwa} is frequently used at the beginning of a clause. It contributes an attenuating meaning to the clause: \qu{just, first, quickly} (\ref{ex:bwa_disc}), but is also used like \textit{go} to keep the speaker role (\ref{ex:bwa_disc2}).


\ea\label{ex:bwa_disc}
\gll bwa the-balan=wan-ea ka i=yata-n Manu ka\\ 
 \gl{disc} \gl{thepunct}-\gl{real}=change-3\gl{sg}.\gl{obj} \gl{cnj} \gl{def}.\gl{sg}=name-3\gl{sg}.\gl{poss} M. \gl{disc}\\ 
\glt \qu{First thing that was done, he was quickly and suddenly replaced by the one called Manu, you know.} {[KG:146]}
\z

\ea\label{ex:bwa_disc2}
\gll ha-go bwaa, tha bwa \textit{devoilé} Wawa mwa ka Jon xaleke\\ 
 \gl{excm}-2\gl{sg} \gl{disc} \gl{ass} \gl{ipfv} unmasked W. \gl{deict} \gl{sbj} J. see\\ 
\glt \qu{Man, well, that's when Wawa was unmasked by John, see.} {[KG:256-257]}
\z

\subsection{Verbless clauses}
\label{ssec:Verbless_clauses}
\is{Clauses!Verbless clause}

Vamale has no copula, and thus juxtaposes two nouns do form an equative construction, where the first is the topic, and the second the comment. The example given in (\ref{ex:nyaiung}) is in Usa Vamale. 

\ea \label{ex:nyaiung}
\gll cahma yo ven papa-n ven apuli-ca {ko} {vin} {apuli} {cahni} {nyae}-{ung}\\ 
 \gl{top} 1\gl{sg} \gl{def}.\gl{sg} father-\gl{poss} \gl{def}.\gl{sg} person-\gl{prox} because \gl{def}.\gl{sg} person here child-1\gl{sg}.\gl{poss}\\ 
\glt \qu{[My sister married a man from Poindimié] and me, [I married] the father of the man here, for the man here is my son.} {[vamale-171129-consent-life:0:04:12-0:04:16]}
\z

The nominal predicate can also be marked with a third person proclitic \textit{a} (\ref{ex:xavee1}) and be followed by a subject NP flagged with \textit{ka} \qu{\gl{sbj}} (\ref{ex:xavee}).

\ea \label{ex:xavee}
\gll ehni i=a= xa-vee ka ya\\  \gl{dem} \gl{def}.\gl{sg}=\gl{rel}= \gl{agt}.\gl{nmlz}-fuck \gl{sbj} 3\gl{sg}\\ 
\glt \qu{It's him who is an unpleasant individual, him (not me).} {[KG:496]}
\z

\ea \label{ex:xavee1}
\gll jacob tha=a=juu xa-vee ma hmwaana\\ 
 J. \gl{ass}=3\gl{sg}=really \gl{agt}.\gl{nmlz}-fuck when like.this\\ 
\glt \qu{Jacob, he's there like a fool when it's like this (if there is no bed for him).} {[KL:121]}
\z

\subsection{Interrogative clauses}
\label{sec:Interr_clauses}
\is{Clauses!Interrogative clause}
Interrogative clauses are not marked by a special word order. Polar question clauses usually carry the assertive \textit{tha}, which content question ones do not. Pitch plays an important role in marking the clauses as interrogative: it rises towards the end, being highest on the stressed syllable of the predicate's main word.

\subsubsection{Polar questions}
\label{ssec:Polar_Q}
\is{Clauses!Polar questions}
\is{Questions!Polar questions}

Polar questions are mostly marked with pitch, either a rising pitch towards the end or with a high pitch on the word demanding confirmation (\textit{tha gavwe xaleke?} \qu{do you see?}).\footnote{A relevant description of pitch in a North New Caledonian language is \citegen{schooling_phonology_1992} \citetitle{schooling_phonology_1992}.} The assertive \textit{tha} is almost always present.

\ea
\gll  tha go=xa-xhwi pimwa?\\ 
 \gl{ass} 2\gl{sg}=\gl{agt}.\gl{nmlz}-eat chili\\ 
\glt \qu{Do you eat chili?} {[G11:2]}
\z

\subsubsection{Content questions}
\label{ssec:Content_Q}
\is{Questions!Question words}\is{Questions!Content questions}

Content questions replace the missing information with a question word, though examples where the missing information is at the end, and just left out, are also attested. The question words are listed in \Cref{tab:ques}. The pronoun \textit{kai} \qu{who}, maybe from \textit{ka i} \qu{and the}, does not have cognates in neighboring languages.\footnote{Voh-Koné \parencite[125]{rivierre_bwatoo_2006} as well as Cèmuhî \parencite[215]{rivierre_langue_1980} use \textit{de}.} The noun \textit{da} \qu{what}, used to be considered a impolite, threat-like form, because of its homophony with \textit{da} \qu{spear} (\ref{ex:da2}). \textit{hmwaeke} \qu{how?} is preferred, as in (\ref{ex:ko_hmwaeke}). The adverb \textit{hmwa-eke} \qu{how} is related to the verbs \textit{hmwa-ena/hmwa-ehni} \qu{thus}, the noun \textit{hmwa-goon} \qu{half} (probably composed of \textit{hmwa-} and \textit{goo-n} \qu{body, sum}), as well as the adverb \textit{hmwa-ka-n} \qu{like X}. \textit{nyeet} \qu{when} is an adverb (\ref{ex:nyeet}). 

\ea \label{ex:nyeet}
\gll go=ta-mwa-me nyeet?\\ 
 2\gl{sg}=go.up-\gl{rep}-\gl{dir.cp} when\\ 
\glt \qu{When did/will you come back?}
\z
 
The stative verbs \textit{heeve}\qu{where (mobile)} (\ref{ex:heeve}) and \textit{ve}\qu{where (immobile)} (\ref{ex:ve}) can be combined with \textit{nya} \qu{towards} and \textit{eca} \qu{\gl{indf}.\gl{pl}} to form adverbs.
\begin{table}
	\centering
	\caption{Question words}
\begin{tabular}{ll}
	\lsptoprule
	\textit{kai}& \qu{who}\\
	\textit{da}&\qu{what}\\
	\textit{hmwaeke} & \qu{how}\\
	\textit{gau ma} &\qu{with whom} (\sectref{ssec:comNP})\\
	\textit{heeve} & \qu{where (mobile)}\\
	\textit{ve} & \qu{where (immobile)}\\
	\lspbottomrule
\end{tabular}
\label{tab:ques}
\end{table}


\ea\label{ex:heeve}
\gll go=han heeve?\\ 
 2\gl{sg}=go where\\ 
\glt \qu{Where are you going?}
\z

\ea\label{ex:ve}
\gll go=ha-me moo ve?\\ 
 2\gl{sg}=go-\gl{dir.cp} stay where\\ 
\glt \qu{Where are you coming from?}
\z


\ea
\label{ex:ko_hmwaeke}
\gll sahnaang-eo ma le=vwa ko hmwaeke\\ 
 not.understand-1\gl{sg} \gl{subr} 3\gl{pl}=do because be.how\\ 
\glt \qu{I'm not sure why they do this (lit. I'm not sure that they do it because of what).} {[D6:11]}
\z


\ea\label{ex:da2}
\gll i=bol gase=vwa ko i=da?\\ 
 \gl{def}.\gl{sg}=ball 1\gl{pl}.\gl{incl}=do \gl{obl} \gl{def}.\gl{sg}=what\\ 
\glt \qu{The (cricket) ball, what did we make them with?} {[KL:126]}
\z

\subsection{Imperative and prohibitive clauses}
\label{sec:IMP_clauses}
\is{Clauses!Imperative clauses}

Orders take two forms: The simple form, used in intimate settings and with younger people, in situations of emotional affect etc, is the same as the quotation form of verbs, and consists in the stem (\ref{ex:imp}). The person meant by the imperative can be marked by adding a noun phrase \textit{ka} \qu{\gl{sbj}} + pronoun after the verb (subject indexing is still dropped) (\ref{ex:imp_PN}).



\ea\label{ex:imp}
\gll se!\\
 cry\\
\glt \qu{Cry!}
 \z

\ea\label{ex:imp_PN}
\gll xale-ke ka go!\\
 look-\gl{tr} \gl{sbj} 2\gl{sg}\\
\glt \qu{Look!}
\z


The other, more polite form is an (in)subordinated clause (\ref{ex:insub}), which may be joined by a main clause expressing the speaker's attitude towards the order: \textit{xahnang} \qu{good}, \textit{juu aman} \qu{important}, \textit{goon} \qu{possible, permitted}, etc. This is discussed in further detail in \sectref{ssec:Insub}. This construction is the most commonly used for stative verbs, which rarely occur in imperative settings.

\ea \label{ex:insub}
\gll (xahnang) ma go=soom...\\
 (good) \gl{subr} 2\gl{sg}=swim\\
\glt \qu{You could swim (that would be good)}
\z

\is{Clauses!Prohibitive clauses}
Prohibitive clauses, contrary to the imperative, admit person marking (\ref{ex:prohPN}), though this is rare; \textit{ka} + PN constructions are preferred. Every prohibitive predicate is preceded by a dedicated particle \textit{cipii} (\ref{ex:cipii}).\footnote{The prohibitive particle contains the negative prefix \textit{ci-} (PMP *(q)ati,  \citealt[88]{lynch_oceanic_2002}) also present in \textit{cika} \qu{\gl{neg}.\gl{exist}} (from POc *tikai, \citealt[88]{lynch_oceanic_2002}), in \textit{cia-} \qu{be absent}, and the neutral negator \textit{cipa}.} Bwatoo uses \textit{cipa} for both negating and prohibitive functions \parencite[58]{rivierre_bwatoo_2006}. %, but features \textit{cieden} \qu{be absent} and \textit{cau} \qu{\gl{neg}.\gl{exist}}. 
The Hienghène languages except Pije have similar negating and prohibitive particles as well \parencite[250]{haudricourt_dictionnaire_1982}, but closely-related Cèmuhî does not.\footnote{Cèmuhî has \textit{tíme} \qu{\gl{neg}} \parencite[184]{rivierre_langue_1980}, \textit{tíé} \qu{be absent} \parencite[111]{rivierre_langue_1980}, and \textit{tíc(í)é} \qu{\gl{neg}.\gl{exist}} \parencite[302]{rivierre_langue_1980}, but the prohibitive is \textit{nèmwó} \parencite[223]{rivierre_langue_1980}.} While \textit{cipa} \qu{\gl{neg}} precedes the subject marker, this is not the case for \textit{cipii} \qu{\gl{proh}}, suggesting a different status, possibly more akin to a preverb (\sectref{ssec:Preverbs}). The data on prohibitive clauses with person indexing is too sparse yet to draw a conclusion.

%%todo local context, oceanic context 
%Proto SE Admiralties pre-clausal *tapun \parencite[91]{lynch_oceanic_2002} may be relatedxx.

 \ea\label{ex:cipii}
	\gll cipii see\\
	 \gl{proh} cry\\
	\glt \qu{Don't cry!}
	 \z
	
	\ea\label{ex:prohPN}
	% \langinfo{}{}{} 
	\gll go=cipii weke\\
	 2\gl{sg}=\gl{proh} rage\\
	\glt \qu{Don't be angry.} {[B2:8]}
		\z 



\section{Modality}
\label{sec:Modality}
\is{Modality}
Modality is marked on the clause level, i.e. the particle \textit{tha} and the modal subordinating constructions do not depend on any part of the commented clause.
Modality is expressed in two ways: one employs modal words in a subordinating construction, to which the commented clause is a complement (see \sectref{ssec:ma}). The other uses dedicated particles and constructions, some transparently decategorialized from verbs and phrases still used with non-modal meaning. This section will first discuss \textit{tha} (\sectref{sec:ass}) and describe modal constructions in general, first epistemic, then deontic constructions.

\subsection{Assertive \textit{tha}}
\label{sec:ass}
\is{Assertive \textit{tha}}
The assertive \textit{tha} is a frequent particle on the far-left border of the predicate. Preceded, in subordinate clauses, by the subordinator (e.g. \textit{a} \qu{\gl{rel}}, \textit{cama} \qu{\gl{subr}}), and in equative constructions by the topic, \textit{tha} is analyzed as belonging to the predicate both syntactically and phonologically. The particle expresses that the speaker is invested in the content, i.e. that they believe the thing to be true, and is thus in complementary distribution with \textit{(b)o} \qu{\gl{irr}}. A notable exception, already discussed in \sectref{sec:bo}: if a formerly irrealis situation has been realized, the past situation can be marked as irrealis via \textit{bo} and still spoken of with confidence (\ref{ex:tha_bo}). %Assertive markers are attested in other Northern languages tooExists in Bwatoo, might be cognate with the \gl{cntr} in Nelemwa?

\ea \label{ex:tha_bo}
\gll e=caihnan hapi {tha} go={bo} vwa\\
 1\gl{sg}=know \gl{comp} \gl{ass} 2\gl{sg}=\gl{irr} do\\
\glt \qu{I knew that you would do it.} {[D6:10]}
\z

The vowel in \textit{tha=} [tʰa], as is the case for the negation \textit{cipa} and most subordinators, assimilates to \textit{e} ([e]) \qu{1\gl{sg}}, forming \textit{th=e} [tʰe], effectively forming a proclitic-host construction.

\subsection{Epistemic modality}
\is{Modality!Epistemic modality}
\begin{sloppypar}
Apart from \textit{tha}, Vamale uses a variety fixed expressions to express speaker certainty concerning the discussed information. Doubt is expressed by using \textit{cama} \qu{if, when (\gl{irr})} (\ref{ex:cama_mod}), \textit{bo} \qu{\gl{irr}}, or with verbs like \textit{cacahniing-} \qu{be unsure} and \textit{sahnaang-} \qu{be confused, not know}. Certainty, a part from \textit{tha}, is expressed by using realis TAM markers \textit{pa} \qu{\gl{prf}} and \textit{ja} \qu{\gl{prf}} (\ref{ex:ja_mod}). The particle for generally known truth \textit{ko} described for western varieties \parencite[55]{rivierre_bwatoo_2006} is not attested in Vamale; instead \textit{vwa hâwan nyakoo-n} \qu{there is a visible manifestation of it} is used (\ref{ex:vis_tru}).
\end{sloppypar}

	
	\ea \label{ex:cama_mod}
	%\langinfo{}{}{} xx
	\gll cama fine nya-koo-n\\
	 if count put-on-3\gl{sg}\\
	\glt \qu{It is doubtful.} (lit. \qu{Whether one counts on it})
	\z	
	
	\ea\label{ex:ja_mod}
%	\langinfo{}{}{} xx
	\gll th=e ja fine nya-koo-n\\
	 \gl{ass}=1\gl{sg} count \gl{subr} put-on-3\gl{sg}\\
	\glt \qu{I am sure of it/him/her.}
	\z 


\ea \label{ex:vis_tru}
\gll vwa hâwân nya-ko hapi a=welo\\
 \gl{exist} spirit put-on \gl{subr} 3\gl{sg}=crazy\\
\glt \qu{It is apparent that he is drunk [he is slow, slurred speech etc].} {[XL2:16]}
\z

\subsection{Deontic modality}
\is{Modality!Deontic modality}
Vamale expresses deontic modality through subordinating constructions with \textit{ma} \qu{\gl{subr}} (\ref{ex:modal_ma1}), further discussed in \sectref{ssec:ma}. The matrix clause is a single modal word and the subordinate one contains the Comment. Important constructions are \textit{goon ma...} \qu{enough \gl{subr}} \qu{it is possible/allowed to...}, \textit{siteke ma} \qu{taboo \gl{subr}} \qu{it is forbidden to...}, \textit{vwasoon ma} \qu{impossible \gl{subr}} \qu{it is difficult/impossible to...}, \textit{xahnang ma} \qu{good \gl{subr}} \qu{it is good to...}. The constructions may use \textit{cama} \qu{if, when (\gl{irr})} instead of \textit{ma} to express more hypothetical scenarios.

\ea \label{ex:modal_ma1}
\gll vwasoon {ma} le=fe-ta-ong ko-n \textit{salle} vukin da\\
 impossible \gl{subr} 3\gl{pl}=take-go.up-1\gl{sg}.\gl{obj} \gl{obl}-\gl{nspec} operation.room reason why\\
\glt \qu{They couldn't take me to the hospital because of what?} {[KG:191]}
\z

\section{Repetitive and deictic \textit{mwa}} 
\label{sec:mwa}
\is{Repetitive \textit{mwa}}

The particle \textit{mwa}, already mentioned in \sectref{sec:WCRepetitive}, versatile and docks onto verb and noun phrases alike, as well as adverbs. This chapter takes it into closer consideration now rather than in previous parts, because \textit{mwa} cannot be attribuated to one particular word class more than to another. It takes different, related meanings depending on the context (\ref{ex:daahma_mwa}). 

\ea \label{ex:daahma_mwa}
\gll i=daahma mwa\\ 
 \gl{def}.\gl{sg}=chief \gl{rep}\\
\glt \qu{the chief again/too/even}\\
\z

\subsection{Repetition}

The most common function is repetition, and its close cousin, restitutive \qu{back}. Other meanings conveyed by \textit{mwa} include \qu{also}, \qu{even}, \qu{on top of that}, but \textit{mwa} can also mark the preceding phrase as focused (see \sectref{sec:mwa} for a discussion). \textit{mwa} \qu{now}, appears to mostly anchor the listener's attention, similarly to \textit{mwa} \qu{even}, onto the noun phrase given (\ref{ex:mwa_now2}). 

\ea \label{ex:mwa_now2}
%\langinfo{}{}{}  
\gll hâ gaa mwa naen hmwa-ena\\
 yes 1\gl{pl}.\gl{incl} \gl{rep} now be.like-\gl{dist}\\
\glt \qu{Yes, we (however) are like that now.} {[KP:101]}
\z

Consider (\ref{ex:mwa_repb}), which was already shown in \sectref{sec:WCRepetitive} to illustrate that \textit{mwa} can dock onto verb and noun phrases alike. This example shall also serve to show how \textit{mwa}'s scope works: while \textit{nya si-m} \qu{put hand-2\gl{sg}.\gl{poss}} is the unmarked construction meaning \qu{to give to you}, \textit{[nya mwa] si-m} means that the object is handed down from a previous interaction (not with the present recipient): \qu{give again, to you}. However, \textit{nya si-m mwa} means that something is given, as other things were given before, to the same recipient: \qu{give to you this as well}. This \textit{mwa} can also mean that something is given here, now, even, etc. And finally, \textit{[[nya mwa] si-m mwa]} can have several meanings: either something is handed back and forth between Source and Recipient several times \qu{give back to you again}, or only once (and the last \textit{mwa} has a deictic function) \qu{give back to you now, give back to you as well}, or several Themes take the same (possibly reciprocal) Path: \qu{give this, too, to you} (22.07.2019, p.76). All of the above-mentioned constructions are attested, especially in customary exchange speeches.


\ea \label{ex:mwa_repb}
\gll e=vatipwe mwa nya mwa si-m mwa i=mwani mwa\\
 1\gl{sg}=drop \gl{rep} put \gl{rep} hand-2\gl{sg}.\gl{poss} \gl{rep} \gl{def}.\gl{sg}=money \gl{rep}\\
\glt \qu{I pass on to you too this money as well.} {[22.07.2019, p.76]}
\z

In (\ref{ex:movV-mwa}) and for all other movement verbs, as well as \textit{xhose} \qu{do again}, \textit{mwa} is analyzed as suffix, i.e. as having fused phonologically with its host. Apart from integrating the host's stress contour, which \textit{mwa} does with other words as well, \textit{mwa} assimilates to the root, which it does not do in other contexts.\footnote{\textit{xhosepwa} suggests a dropped \textit{-t} or \textit{-p}. The Pije and Fwâi cognates \textit{khô-peei} \qu{?-say} \parencite[155]{haudricourt_dictionnaire_1982} could be a diachronic hint at a morphologically complex, old Vamale form.} Compare \textit{hut-mwa} $\rightarrow$ /hupʷa/ \qu{go back down}, to \textit{hut=mwa} \qu{go down again}. 



\ea \label{ex:movV-mwa}
\gll go=ha-mwa-me\\
2\gl{sg}=go-\gl{rep}-\gl{dir.cp}\\
\glt \qu{You return to me.}\slash\qu{You come back.}
\z

\ea
\gll go=ha-me mwa\\
2\gl{sg}=go-\gl{dir.cp} \gl{rep}\\
\glt \qu{You come again.}
\z


Repetition and deixis are the two most frequent functions of \textit{mwa}. Examples (\ref{ex:mwa_rep1}) show different contexts in which \textit{mwa} is a repetitive particle.% Example (\ref{ex:mwa_deict}) shows \textit{mwa}referring to something recently mentioned, . 

\ea \label{ex:mwa_rep1}    
\gll e=tena mwa\textsuperscript{{\upshape REP}} i=hun-det\\ 1\gl{sg}=hear \gl{rep} \gl{def}.\gl{sg}=\gl{nmlz}-sound\\
\glt \qu{I hear the sound again.} {[JR:17]}
\z

\ea\label{ex:mwa_rep2}     
\gll e=thake i=vai nyako i=siibwi e=thawatap ka e=thake mwa i=e-thalo-ka-n\\
 1\gl{sg}=throw \gl{def}.\gl{sg}=stone \gl{obl} \gl{def}.\gl{sg}=rat 1\gl{sg}=miss \gl{cntr} 1\gl{sg}=throw \gl{rep} \gl{def}.\gl{sg}=\gl{ord}-two-\gl{clf}.\gl{poss}-\gl{nspec}\\
\glt \qu{I throw the stone at the rat, I miss and I throw again a second [time].} {[JR:26]}
    \z

\ea             
\gll	cipa=be=bo xale-ko mwa\textsuperscript{{\upshape REP}} hmakoo-n habu mwa\textsuperscript{{\upshape REP}}\\
	\gl{neg}=1\gl{pl}.\gl{excl}=\gl{irr} see-2\gl{sg} \gl{rep} find-\gl{nspec} long.ago \gl{rep}\\
\glt \qu{We will never see you again.} {[CD:14]}
\z 


\subsection{Deixis}
\is{Repetitive \textit{mwa}!Deixis}
Apart from repetition, \textit{mwa} is chiefly used for spatial and temporal deixis. %This takes on different forms: \textit{mwa} can mean \qu{then, at that time} (\ref{ex:thathea_mwa}), or refer to something recently mentioned (\ref{ex:mwa_deict}), as well refer to something close spatially, and known by the listener (\ref{ex:spat_mwa}).
As such, \textit{mwa} marks temporal and spatial immediacy to the speaker or the spoken-of event: in example (\ref{ex:temp_mwa}), \textit{mwa} is used to connect the events tightly together, expressing how quickly things followed each other. In (\ref{ex:mwa_deict}), the noise referred to is already known to the speaker, because it was recently mentioned. In (\ref{ex:thathea_mwa}), \textit{mwa} clarifies which event had happened by which moment in the narration: he had already died by the time they found him. Concerning spatial deixis in (\ref{ex:spat_mwa}), the area in question is visible from the speaker's chair. The semantic closeness to the deictic constructions discussed above is apparent: in both cases, \textit{mwa} designates something known to the speaker and to the hearer.

\ea\label{ex:temp_mwa}
\gll a cana ka th e=bwa vee mwa vwaseekan mwa ko {vukin-eong} mwa\\
 \gl{expl} vagina \gl{cnj} \gl{ass}=1\gl{sg} \gl{ipfv} fuck \gl{deict} sad \gl{deict} because cause-1\gl{sg}.\gl{poss} \gl{deict}\\
\glt \qu{Ah shit, I've just messed up right now, I sorry [for them] now because this right here is because of me.} {[KG:491]}
\z

\ea\label{ex:mwa_deict}
\gll xhose e=tena mwa\textsuperscript{{\upshape REP}} tha a bwa vwa det mwa\textsuperscript{\gl{deict}}\\
 again 1\gl{sg}=feel \gl{rep} \gl{ass}=3\gl{sg} \gl{ipfv} do sound \gl{deict}\\
\glt \qu{Again I heard him make said (\textit{mwa}) noise.} {[JR:18]}\\
\z

\ea\label{ex:thathea_mwa}
\gll go le=ja thathe-a {mwa}\textsuperscript{temp. deix} go le=nya siwa mwa\textsuperscript{\gl{rep}}\\
  then 3\gl{pl}=\gl{prf} kill-3\gl{pl} \gl{deict} then 3\gl{pl}=towards return \gl{rep}\\
 %\glt \qu{Et eux, il était déjà mort/après qu'il ait été tué, ils sont retournés à la maison}\\
\glt \qu{And they had finally killed him then, and then they went back.} {[HC1:14]}
 \z


\ea\label{ex:spat_mwa}
\gll na i=bee i=papa-n ena xahnuut pwanbaut {mwa}\textsuperscript{spat. deix}\\
  \gl{dem} \gl{def}.\gl{sg}=brother \gl{def}.\gl{sg}=father-\gl{poss} \gl{dist} downstream P. \gl{deict}\\
 \glt \qu{[The one who was killed] is the brother of the father of those down in P. there.} {[HC1:22]}
 \z

%\ea
%
%\gll go le fwade-a nya-mwa-la, ko ya mwa si-n i mwa-n hmat ko i vaa, a hnya-da-me ka thêama
%  then 3\gl{pl} seek.\gl{anim} towards-\gl{deict}-\gl{prox} because 3\gl{sg} \gl{deict} hand-3\gl{sg}\gl{poss} \gl{def}.\gl{sg}=container-\gl{poss} customary.money for \gl{sg}.\gl{def} war 3-\gl{sg} send-go.up-\gl{dir.cp} \gl{agt} chief
% \glt  \qu{And they were looking for him then. Because it was him, he had the shell money in his hand, sent by the High Chief [Bwarhat]}
% 


%\ea\label{ex:spat_mwa}
%
%\langinfo{}{}{} HC:
%\gll ma le ja hmaca-mwa Noumea le ja fee-le-mwa jevwan mwa acan ka li=xhaomu mwa le moo cahni 
%  \gl{irr} 3\gl{pl} \gl{prf} arrive-3\glwa} N. 3\gl{pl} \gl{prf} take-3\gl{pl}-3\glwa} l'ensemble 3\glwa} among \gl{agt} \gl{def}.\gl{pl}=elder 3\glwa} 3\gl{pl} stay here
% \glt \qu{Quand ils sont arrivés à Nouméa, ils ont pris l'ensemble, parmi eux des vieux qui restaient ici}
% 

In (\ref{ex:thathea_mwa}), \textit{mwa} could indicate that the death is already known to the speaker. In (\ref{ex:gaa_mwa}) and (\ref{ex:gaa_mwa2}), it may refer (back) to the speaker's group (i.e. his generation), and in (\ref{ex:mulip_mwa}) mark that the life he speaks about was mentioned before. Note that Mr.\ Fouan does not use \textit{-kaa} \qu{1\gl{pl}.\gl{incl}.\gl{obj}}, but \textit{-gaa}, which looks more like (stative) subject-indexing pronouns.


%\a
%
%\langinfo{}{}{} HC1:24
%\gll thathe-a nya xahan waneut ma le siwa-mwa\textsuperscript{REP} ka le ja fwadai \textbf{mwa}\textsuperscript{temp. deix} ka li=pupwaale, a ja han thaloot ka bwaaxat-ae vwa koo-n ka bwaaxat li=apuli-nea
%  kill-3\gl{sg} \gl{prox} over.there W.-K. when 3\gl{pl} return-\gl{rep} \gl{cntr} 3\gl{pl} \gl{prf} seek.\gl{inan} \gl{deict} \gl{sbj} \gl{def}.\gl{pl}=European 3\gl{sg} \gl{prf} go come.out \gl{sbj} B.-\gl{dem} do \gl{obl}-\gl{nspec} \gl{sbj} B. \gl{def}.\gl{pl}=person-3\gl{sg}.\gl{poss}
% \glt \qu{He was killed all the way over in Wan-Kuut when they came back. They had finally found out by then, had the white people, this Bwaarhat had ended up emerging, they did that, Bwaarhat, his men.}
% 

\ea\label{ex:gaa_mwa}
\gll i=ape-caihnan aman-le tha seen-le \textit{pas tout le monde}. go cama gaa {mwa} vwa li=bebe-n-le le=hnyaa-mwa la la	\\
 \gl{def}.\gl{sg}=\gl{loc}-know something-3\gl{pl}	\gl{ass} limit-3\gl{pl}.\gl{poss} not.everyone then \gl{top} 1\gl{pl}.\gl{incl} \gl{rep} \gl{exist} \gl{def}.\gl{pl}=baby-\gl{poss}-3\gl{pl} 3\gl{pl}=send-\gl{deict} \gl{prox} \gl{prox}\\
\glt \qu{Their knowledge is limited to them, not everyone. And, concerning us, there were their babies that they sent there [to school].} {[KM:61]}
      \z

\ea\label{ex:gaa_mwa2}
\gll hnya-mwa-ga can mwa-n-sohmun-ea le=ecaa-gaa {mwa} ko ca=aman a= saten\\
  send-\gl{deict}-1\gl{pl}.\gl{incl} in-\gl{nspec} house-\gl{poss}-study-3\gl{sg}.\gl{poss} 3\gl{pl}=learn-1\gl{pl}.\gl{incl} \gl{deict} \gl{obl} \gl{indf}.\gl{pl}=thing \gl{rel}= different\\
\glt \qu{They\textsubscript{1} sent us to his school, they\textsubscript{2} taught us other things.} {[KM:62]}
\z

%\ea             
%
%\langinfo{}{}{}	KM:63
%\gll cahma le, tha i hun-moo-le, le hnya-mwa-gaa xahan eca aman a xadaa saten
%  \gl{top} 3\gl{pl} \gl{ass} \gl{def}.\gl{sg}=\gl{nmlz}-be-3\gl{pl}.\gl{poss} 3\gl{pl} send-\gl{deict}-1\gl{pl}.\gl{incl} over.there \gl{indf}.\gl{sg}=thing \gl{rel} however different
%\glt \qu{Now them, it was how they were, they sent us there [to learn] something different}
%

\ea \label{ex:mulip_mwa}
\gll tha se mulip {mwa}, go a ga ca-n naen\\
 \gl{ass} other life \gl{deict} \gl{disc} \gl{cnj} 1\gl{pl}.\gl{incl} in-\gl{nspec} now\\
\glt \qu{It was another life then, and now here we are.} {[KM:64]}
\z

In local French, \textit{encore} \qu{again} is used as  \qu{even, on top of that}. This may be a calque from Kanak languages, as e.g. \textit{mwa} is used exactly like that in Vamale (\ref{ex:mwa_en-plus}). Another way of expressing \qu{on top of that} is \textit{xhopwe}, but the information in its scope is less surprising to the hearer.

\ea \label{ex:mwa_en-plus}
\gll ka lu e-bee-lu mwa!\\
 \gl{cnj} 3\gl{du} \gl{recp}-peer-3\gl{du} even\\
\glt \qu{And on top of that, they're related!}
\z
