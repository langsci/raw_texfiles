\chapter{Voice} 
\label{ChapterVoice} 
\is{Voice}

``Voice" in this grammar refers to various kinds of mapping semantic relations onto grammatical relations. Phenomena discussed in this chapter include reflexive, reciprocal, and middle constructions, all marked by the prefix \textit{(v)e-}, discussed in \sectref{sec:refl}, and the causative constructions prefixed by \textit{fa-} (\sectref{sec:caus}). A brief section on agent-omitting constructions is added for completion.

%There is something! look in nelemwa maybe
%something with ko? 

\section{Reflexive and reciprocal}
\is{Reflexive}
\is{Voice!Reflexive}
\label{sec:refl}

Vamale prefixes verbs and nouns with \textit{e-}, a morpheme that can have reflexive, reciprocal, and middle functions. It is a reflex of POc *pa(R)i- \parencite[151--152]{pawley_problems_1973}. The reconstructed meaning of *pa(R)i-...-i is that of ``mutual interaction between the entities denoted by the subject of the verb” and refers to ``unified or conjoined action by a plural subject, or repeated action by a singular subject, or unification of objects” \parencite[151--152]{pawley_problems_1973}. These reconstructed functions did not include reflexive. The reflexes of *pa(R)i- described today did not retain all of the meanings reconstructed for the Proto-Oceanic form. See \Cref{tab:paRi} for an overview of the reconstructed functions, which ones were retained in Vamale, and where in the chapter the retained functions are discussed. 

\subsection{Reflexive}
\is{Voice!Reflexive}
Reflexive constructions function in two ways. Either the subject marker and the object marker denote the same referent, as in example (\ref{ex:refl1}).\footnote{\textit{khêt} originally meant \qu{quartz}. Knives were also made from shells (mangrove oysters (\textit{thala}) gave their name to today's steel knives, though the word may have come from a Polynesian word for knife \textit{(h)ele}, \citealt{hollyman_polynesian_1959}), or reed and bamboo shards called \textit{xadan o} \qu{shard of bamboo}.} This is the most common way of forming a reflexive construction.

\ea
\label{ex:refl1}
\gll e=vwa-khêt-eo\\ 
 1\gl{sg}=do-blade-1\gl{sg}.\gl{obj}\\ 
\glt \qu{I shave myself.}
\z 

The other possibility is with the prefix \textit{e-} \qu{\gl{refl}}, formerly \textit{ve-} \parencite[585]{leenhardt_langues_1946},\footnote{Chief Luc Katelia Oué still uses \textit{veca} \qu{learn}, where most younger speakers use \textit{eca}. Usa Vamale, Temala Hmwaveke, and Fa Tiéta (still) tend to use \textit{vi} \qu{\gl{def}.\gl{sg}} where Coastal Vamale has \textit{i}.} see  (\ref{ex:refl2})% and (\ref{ex:refl_old})
. This \textit{e-} is distinct from the instrumental \textit{e-} (\sectref{ssec:ins.nmlz}), e.g. \textit{e-vwadi ya-n} \qu{thumb (lit. \gl{nmlz}.\gl{ins}-peel.with.fingers starchy.food-\gl{poss})}. Nêlêmwa cognates are \textit{ve-} \qu{\gl{nmlz}.\gl{ins}} and \textit{pe-} \qu{\gl{recp}} respectively \parencite[171]{bril_nelemwa_2002}. Note that Nêlêmwa does not use \textit{pe-} with a reflexive meaning \parencite[24]{bril_semantic_2005b}, nor is Proto-Oceanic *paRi- reconstructed to have had a reflexive function.%(Pawley \& Reid 1979: 110), though reflexive is not a function reconstructed for it.


\begin{table}
	\centering
	\caption{Functions of Proto-Oceanic retained in Vamale, after \cite[28]{bril_semantic_2005b}}
	\begin{tabularx}{\linewidth}{QQl}
		\lsptoprule
		Function and morph. makeup&Vamale example&Section\\
		\midrule
		Collective actors (\textit{paRi}, \textit{paRi}-...\textit{-i}) &\textit{e-moo} \qu{all stay}&\sectref{ssec:pluriactional}\\
		Collective entities, grouping (\textit{paRi}, \textit{paRi}+root reduplication) &\textit{e-tipwa} \qu{all fall (not strictly at the same time)}&\sectref{ssec:MID_grouping}\\
		Reciprocal actors (\textit{paRi}, \textit{paRi}-...\textit{-i}) & \textit{e-thaut} \qu{have a fight}&\sectref{ssec:recp}\\
		Comparison (\textit{paRi}) &\textit{e-hmwakan} \qu{be the same}&\sectref{ssec:MID_comparison}\\
			Iterative actions (\textit{paRi}, \textit{paRi}-...\textit{-i}) & \textit{e-hmain} \qu{more and more} & \sectref{ssec:MID_unbounded}\\
		\midrule
		\multicolumn{3}{l}{\emph{Not retained:}}\\
		\multicolumn{3}{l}{Intransitivizing, depatientive function (\textit{paRi}, \textit{paRi}-...\textit{-i})}\\
		\multicolumn{3}{l}{Reciprocal or associated OBJ (\textit{paRi})}\\
		\multicolumn{3}{l}{Reference to states and properties (\textit{paRi} and reduplication)}\\
		\multicolumn{3}{l}{Root reduplication in general was not retained}\\
		\lspbottomrule
	\end{tabularx}
\label{tab:paRi}
\end{table}

%\todo{ask Isabelle about iterative}

\ea
\label{ex:refl2}
\gll go=e-vwa-khêt\\ 
 2\gl{sg}=\gl{refl}-do-blade\\ 
\glt \qu{You shave yourself.}
\z

%\a
%\label{ex:refl_old}
%
%\langinfo{}{}{} KE:24
%\gll ya tha a juu ve-gal-eo
% 3\gl{sg} \gl{ass}=3\gl{sg} real \gl{refl}-lie-1\gl{sg}.\gl{obj}
%\glt \qu{He straight up lied to me} (\textit{gat} \qu{to lie}, lenition in the transitive form)
 

Self-directed actions in Bwatoo are only marked with \textit{e-} for a few verbs of grooming or cognition, and still use coreferential arguments, i.e. object suffixes \parencite[34]{bril_semantic_2005b}. This is not the case in Vamale, where cognition verbs do not need object suffixes, e.g. \textit{go e-hnyimake} \qu{2\gl{sg} \gl{refl}-think} \qu{you think to yourself} and \textit{e-fwajimwake} \qu{\gl{refl}-ask} \qu{to ask oneself}. Grooming verbs may take it or not. 
The reflexive is productive, but has become lexicalized in certain forms as well, e.g. \textit{wago} \qu{encourage} $\rightarrow$ \textit{e-wago} \qu{persevere}.

%e thôang "command" %todo{what to do with this? \textit{thoang} means habit;law}

\subsection{Reciprocal}
\label{ssec:recp}
\is{Voice!Reciprocal}
The reflexive \textit{e}- can also express reciprocality\is{Reciprocal} and middle functions. 
Interestingly, the reflexive function developed from the reciprocal marker \parencite[26]{bril_semantic_2005b}. Dixon describes the Fijian reflex \textit{vei-} as more collective than reciprocal \parencite[256]{dixon_grammar_1988}.
The prefix follows the subject marker, but \textit{e-} and the object markers are mutually exclusive in a reflexive context (a possible, poorly understood exception are grooming verbs). They can (and actually do) however cooccur in a reciprocal reading. The construction in (\ref{ex:refl3}) without the object suffix (\textit{lu=e-xaleke}) means \qu{they see themselves}. Reciprocal prefixes are also found on nouns (\ref{ex:refl3c}). 
%nor does the marker in REFL constructions -- how could the prefix "replace" the object suffix? If what you mean is that e- and the object markers are mutually exclusive in a REFL reader but can (and actually do) cooccur in a RECP reading, then say it like that :)

Examples of lexicalized forms are \textit{e-thuang} \qu{tease each other} which contrasts with \textit{thuang} \qu{joke}, and \textit{e-canim} \qu{play hide-and-seek}, from \textit{canim} \qu{hide oneself from someone}.


	
	\ea\label{ex:refl3}
	% \langinfo{}{}{}
	\gll lu=e-xale-lu\\ 
	 3\gl{du}=\gl{recp}-see-3\gl{du}.\gl{obj}\\ 
	\glt \qu{They see each other.} {[After 07.11.18 p.93]}
	\z
	
	\ea
	\gll lu=xale-lu\\ 
	 3\gl{du}=see-3\gl{du}.\gl{obj}\\ 
	\glt \qu{They see them.}
	\z

\ea\label{ex:refl3c}
% \langinfo{}{}{} 
\gll lu=e-\textit{copain}-\textit{copine}\\ 
 3\gl{du}=\gl{recp}-boyfriend-girlfriend\\ 
\glt \qu{They are boyfriend-girlfriend.} {[AG1:299]}
\z



%\subsubsection{Object reciprocals}
%\label{ssec:obj_recp}
%
%he joins the two ends of a rope \parencite[48]{bril_semantic_2005b} \textit{thaa} \qu{attach}  \textit{e-thaa} \qu{have a strong relationship, be close}?
\subsubsection{Comparison and symmetry}
\label{ssec:MID_comparison}
\is{Voice!Middle!Sameness}

The middle prefix, for certain words, expresses symmetry between two or more entities. 
This meaning of \qu{same} expressed with \textit{e-} \qu{\gl{mid}} rather than \textit{se} \qu{one, same} appears reliably on verbs and their derivates, such as \textit{e-hmwakan} \qu{\gl{mid}-same} \qu{be the same} and \textit{e-hmwa-goon} \qu{\gl{mid}-same-body}, \qu{be of same length} (\ref{ex:hmwagoon}), whereas nouns differ. Compare \textit{se fedala-lu} \qu{They have the same blood (same blood-3\gl{du}.\gl{poss})} and \textit{e-wada-lu} \qu{They have the same age} (\ref{ex:same_age}).

\ea \label{ex:hmwagoon}
% \langinfo{}{}{} 
\gll e-hmwaa-go-lu\\ 
 \gl{recp}-like-body-3\gl{pl}.\gl{poss}\\ 
\glt \qu{They are equally long.} {[example for the dictionary]}
\z
%

\ea \label{ex:same_age}
% \langinfo{}{}{} 
\gll e-wada-lu\\ 
 \gl{recp}-time-2\gl{du}.\gl{poss}\\ 
\glt \qu{They(2) have the same age.} (but:\textit{nievit joo-n?} \qu{how.many year-3\gl{sg}.\gl{poss}} \qu{how old is s/he?}) {[07.11.18 p.95]}\
\z

\subsubsection{Reciprocal relationships}
\label{ssec:rcp_relationship}
\is{Voice!Reciprocal kinship terms}


The prefix \textit{e-} also occurs on possessed nominals to express a reciprocal relationship outside of the family (\ref{ex:refl3c}) and (\ref{ex:rcp_rel}). Reciprocal relationship nouns for family members (``symmetrical kinship and dyadic relationship", \citealt[48]{bril_semantic_2005b}) are an own lexical field, prefixed with \textit{xa-}.\footnote{This is likely not related to the agentive nominalizer, as the cognates for both in other Northern languages are different. For example, in Pije, \textit{kee-} is the reciprocal prefix and the agentive one is \textit{ka-}.} A non-exhaustive list is found in \Cref{tab:recp_rel}. These reciprocal relations\is{Nouns!Reciprocal relations} do not extend beyond the lexical fields of kinship and marriage.

\begin{table}
	\caption{Reciprocal kinship terms}
		\begin{tabularx}{\textwidth}{llQ}
		\lsptoprule
		Complex form & Gloss of Morphemes & Meaning\\\midrule
		\textit{xa-bate}& \textit{xa}-tip & \qu{brothers}	
		\\ \textit{xa-betha} & &\qu{sisters}
		\\ \textit{xaa-vap-an} &\textit{xa}-hunt?-\gl{poss}& \qu{siblings}
		\\ \textit{xa-fa-thau-n} & \textit{xa}-\gl{caus}-wealth-\gl{poss} &\qu{brother-in-law and sister-in-law}
		\\ \textit{xa-nya-pwan-an}& \textit{xa}-put-on-\gl{poss} & \qu{paternal aunts}
		\\ \textit{xa-e-vwona-n}& \textit{xa}-\gl{recp}-maternal.uncle-\gl{poss}& \qu{maternal uncle and niece/nephew}
		\\ \textit{xa-vabu-n} &\textit{xa}-grandchild-\gl{poss}& \qu{grandfather and grandchildren}
	\\	\textit{xaa-maci}& & \qu{father and child}
	\\	\textit{xa-e-bee-n}\footnote{We also found a term \textit{xa-ve-bee-n} \qu{cross-cousins}, but it is doubtful that this is a true distinction given the allomorphy of \textit{e-} and \textit{ve-}.}&\textit{bee-n} \qu{brother, cousin, peer}& \qu{cousins}
	\\			\textit{xa-ive-n} &\textit{iven} \qu{sister-in-law}& \qu{sisters-in-law}
	\\			\textit{xa-xayaa-n}&\textit{xaya} \qu{stranger, guest}& \qu{husband and his brother-in-law} \\
	\textit{xa-mwaa-n thamo}&\textit{mwaa-n} \qu{daughter-in-law}& \qu{mother-in-law and her son-in-law}\footnote{Probably actually \qu{child-in-law of the opposite sex}. By analogy, \qu{father-in-law and his daughter-in-law} may be \textit{xa-mwaa-n xayu}.}\\
	\lspbottomrule
	\end{tabularx}
\label{tab:recp_rel}
\end{table}


Several terms are missing from my data: \qu{wife and her sister-in-law}, \qu{wife and her co-wife}, \qu{aunt and niece}, \qu{grand-mother and her grand-children}, among other things. Some terms may be gender-neutral, e.g. \qu{grand-mother and her grand-children} may also be \textit{xa-vabun}.


\ea \label{ex:rcp_rel}
% \langinfo{}{}{} 
\gll calibeen ma le=moo ma li=ehni e-bee-le\\ 
 sometimes if 3\gl{pl}=stay \gl{com} \gl{def}.\gl{pl}=\gl{dem}.\gl{prox} \gl{recp}-peer-3\gl{pl}.\gl{poss}\\ 
\glt \qu{Sometimes when they stay together with those, they are each other's cousins.} {[AG1:239]}
\z



These reciprocal relations\is{Nouns!Reciprocal relations} do not seem to extend beyond the lexical fields of kinship and marriage, possibly because the prefix \textit{xa-} \qu{\gl{agt}.\gl{nmlz}} is ambiguous in connection with other roots.

\section{Middle}
\label{ssec:mid}
\is{Voice!Middle}
The reflexive/reciprocal prefix \textit{e-} is used for other functions as well where %the action performed concerns the agent, and 
an intransitive action is somehow enhanced. While there are identifiable patterns, the meaning of \textit{e-} is lexically determined, yielding in some cases completely idiosyncratic meanings (\Cref{tab:mid}).

\begin{table}
	\caption{Base and middle forms}
	\begin{tabularx}{\linewidth}{QQ}
		\lsptoprule
		Base form and meaning & Middle form and meaning\\
		\midrule
		%	\multicolumn{3}{c}{Pairs where \textit{e-} changes the meaning} \\
		%	\midrule&\\
		%\textit{see} \qu{be alone} & \textit{e-see} \qu{be rare, one of a kind} \\
		%\textit{hmai-} \qu{be many} & \textit{e-hmai-} \qu{grow in number}\\ 
		%\textit{wago} \qu{encourage} & \textit{e-wago} \qu{persevere}\\ 
		%\textit{thalo} \qu{be two} & \textit{e-thalo} \qu{be in a pair, be two by two}\\
		\textit{wesidedo} \qu{mirror (noun)} & \textit{e-sidedo} \qu{look at oneself in a mirror}\\
		\textit{suni-xhiit} \qu{?-pain} & \textit{e-suni} \qu{to restrain oneself, to suffer}\\
		\midrule
		\textit{siiwa} \qu{go back} & \textit{e-siiwa} \qu{give a counter-gift}\\
		\textit{vadanke} \qu{forget} & \textit{e-vadanke} \qu{forgive someone}\\
		\textit{vwa nyako-} \qu{do; \gl{exist} for} & \textit{e-vwa nyako-} \qu{to seem}\\ 
		\textit{xhwatiin nyima-} \qu{small heart} & \textit{e-xhwatiin nyima-} \qu{be sad}\\
		\lspbottomrule
	\end{tabularx}
	\label{tab:mid}
\end{table}

Since the meanings associated with \textit{e-} are so diverse, and since the conventional analysis of it differs from verb to verb, additional words may be added to disambiguate: \textit{sisipo} \qu{together}, \textit{xhayu} \qu{randomly}, \textit{hato} \qu{alone} etc. 

% ``In ambiguous cases such as (48a), which may be interpreted as collective or intensive, the  collective meaning may be disambiguated by an adjunct meaning ‘together’:  \textit{wuung} in Nêlêmwa, \textit{thipo} in Bwatoo (49a)." \parencite[53]{bril_semantic_2005b}


%There is a reflexive possessive suffix, -\textit{gat}. Have only seen it with kinship so far. -\textit{gat} \qu{own} \textit{maahma-gat} \qu{own grand-father/older brother}. \todo{actually, probably derived from \textit{gat} \qu{false, lie}, and it's not really the older brother, but the great-grandfather}

\subsection{Lack of volitional agent}%: Spontaneous, unintentional actor, generic properties.}
\is{Voice!Middle!Lack of volitional agent}
A function of \textit{e-} to mark events initiated by forced of nature or other unintentional causes is well attested in Oceanic \parencite[33]{bril_semantic_2005b}. While this is described for Bwatoo \textit{ve-} \parencite[310]{rivierre_bwatoo_2006}, it could not be found in Vamale: a true lack of initiator. This is expressed with a manner verb \textit{hato} \qu{to do X alone}. The absence of this function otherwise present in New Caledonia may be due to attrition. There are only two possible candidates that this research project found for Vamale: 
\textit{e-vwa nyakoo-} \qu{\gl{mid}-\gl{exist} for-} \qu{take note of something, become aware} (\ref{ex:evwa}), and \textit{e-vaaya ko-} \qu{\gl{mid}-move on}\qu{shake, move on one's own} (\ref{ex:evaaya}). The first uses a non-volitional verb anyway, \textit{vwa} \qu{there is}, which casts doubt upon an analysis of \textit{e-} as marking a lack of initiator. %Note that both use \textit{ko}, though we are not sure which meaning \textit{ko} has here, nor whether \textit{-n} is an anaphoric or possessive morpheme.


\ea\label{ex:evwa}
% \langinfo{}{}{} 
\gll cahma In-Fwe cipa a=e-vwa nyakoo-n\\ 
 \gl{top} Bark-guettarda\_speciosa
 \gl{neg} 3\gl{sg}=\gl{mid}-\gl{exist} for-\gl{ana}\\ 
\glt \qu{But Figtree-Bark did not suspect anything.} {[GC:16]}
\z


\ea\label{ex:evaaya}
% \langinfo{}{}{} 
\gll e-vaaya ko-n ka i=that\\ 
 \gl{mid}-move on-\gl{nspec} \gl{sbj} \gl{def}.\gl{sg}=wind\\ 
\glt \qu{The wind shakes [the leaves].} {[X4:13]}
\z

\subsection{Lack of endpoint}%: Unbounded, non-completed, durative actions.}
\label{ssec:MID_unbounded}
\is{Voice!Middle!Lack of endpoint}
Verbs can acquire an unbounded meaning with \textit{e-}: while e.g. the active verb \textit{thana} \qu{to wander} becomes \textit{e-thana} \qu{to go for a stroll, to aimlessly amble}, stative verbs like \textit{gere} \qu{fat; be fat} and \textit{hmain} \qu{be numerous} change from describing mere states, to describing developments: \textit{e-gere} \qu{to grow fat}, \textit{e-hmain} \qu{more and more numerous}.

\subsection{Mode of grouping}
\label{ssec:MID_grouping}
\is{Voice!Middle!Grouping}
A rare function, which is described also for other Northern languages \parencite[46]{bril_semantic_2005b}, changes the meaning of a numeral verb to a grouping one: 
\textit{nya e-thaloo} \qu{put two by two}; 
\textit{e-thaloo moo koon} \qu{there are two kinds, they stay on this as two}. The word \textit{e-see} \qu{be rare} shown in (\ref{ex:e-see}) is derived from \textit{se} \qu{be alone, be one}. 

\ea \label{ex:e-see}
% \langinfo{}{}{} 
\gll xaleke, go, cahma wadat, tha e-see\\ 
 see well concerning gun \gl{ass} \gl{mid}-be.one\\ 
\glt \qu{You see, and well, guns were rare.} {[KL:140]}
\z


\subsection{Symmetrical point}% or space between some landmarks}
\is{Voice!Middle!Spatial Symmetry}
While \textit{can dawee-} \qu{inbetween} can be used to describe a member of a group, as in (\ref{ex:can_daweelu}), a derived form \textit{can e-dawee-} \qu{among} also exists, see (\ref{ex:can_e}). There is not enough data yet to precisely pinpoint the difference, though the indistinguishability of the group members may be a nuance introduced by \textit{e-}, given that this is a function of \textit{e-} attested in other contexts (see \sectref{ssec:pluriactional}). Other languages in the area use the middle prefix with a ``symmetrical point or space between some landmarks" \parencite[50]{bril_semantic_2005b}, but Vamale speakers were content with either form and only one unprompted occurrence of \textit{can e-dawee-} is attested.


\ea\label{ex:can_daweelu}
% \langinfo{}{}{} 
\gll lu=e-vi hapi na kai eca=a= nya wî- vwa wîî-n can dawee- can dawee-lu\\ 
 3\gl{du}=\gl{recp}-say \gl{comp} \gl{top} who \gl{indf}.\gl{sg}=\gl{rel}= give strength \gl{exist} strength-3\gl{sg}.\gl{poss} in between in between-3\gl{du}\\ 
\glt \qu{They discussed who was the stronger between the two of them.} {[JV:4]}
\z


\ea\label{ex:can_e}
% \langinfo{}{}{} 
\gll ma can e-dawee-le i=a= yata-n In-Thu\\ 
 \gl{cnj} in-\gl{nspec} \gl{mid}-between-3\gl{pl} \gl{def}.\gl{sg}=\gl{rel}= name-3\gl{sg}.\gl{poss} Bark-Banyan\\ 
\glt \qu{And among them was the one called Banyan Bark.} {[GC:6]}
\z

 \section{Causatives}
 \is{Causative}
 \is{Voice!Causative}
 \label{sec:caus}
 
The causative prefix \textit{fa-} is probably a reflex of Proto-Oceanic \textit{*pa-} \qu{\gl{caus}} \parencite[510]{ross_morphosyntactic_2004}. It can be applied to active (\ref{ex:caus}), stative (\ref{ex:stat_caus1}), and possessible verbs (\ref{ex:poss_caus}) alike. The Vamale causative is prototypical in \citeauthor{zuniga_grammatical_2019}'s sense, as it increases the base verb's valency by one, and introduces a new agent, which acts as the new subject of the clause, and is ``formally coded on the predicate complex" \parencite[15--16]{zuniga_grammatical_2019}.

\ea\label{ex:caus}
% \langinfo{}{}{} 
\gll ya a=fa-thuat {i}={apuli}\\ 
 3\gl{sg} 3\gl{sg}=\gl{caus}-go.out \gl{def}.\gl{sg}=person \\ 
\glt \qu{He had the man released.} {[Hc1:35]}
\z
\ea 
\label{ex:stat_caus1}
\gll a=fa-xhopwen\\ 
 3\gl{sg}=\gl{caus}-big\\ 
\glt \qu{S/he is conceited.}
\z 
\ea \label{ex:poss_caus}
	
		\gll {go}=fa-mu-nyima-n\\ 
	 2\gl{sg}=\gl{caus}-?-heart-3\gl{sg}.\gl{poss}\\ 
	\glt \qu{You scare him/her.}
	\z
%\ex
%
%
%\gll e=fai-mae
% 1\gl{sg}=light-fire
%\glt \qu{I light the fire.}
%
%\z
Possessible verbs which are subject to a causative derivation mark the demoted S exactly the same as it was for the base verb, adding the agentive argument in the beginning (\ref{ex:poss2_caus}). 


	\ea \label{ex:poss2_caus}
		\gll mu-nyima-n\\ 
	 \gl{def}.\gl{du}?-heart-3\gl{sg}.\gl{poss}\\ 
	\glt \qu{S/he is afraid.}
	\z 
	
	\ea
		\gll {go}=fa-mu-nyima-n\\ 
	 2\gl{sg}=\gl{caus}-?-heart-3\gl{sg}.\gl{poss}\\ 
	\glt \qu{You scare him/her.}
	\z
	%\e
	%
	%\gll e=fa-vwaseeka li=apuli
	%
	%
	%
	%\glt \qu{I make the people sad}
	%
	%




\gl{S\textsubscript{A}}-indexed verbs can be transitive or intransitive. Intransitive verbs, as expected of a prototypical causative construction, are derived to a transitive one, and the former S subject is demoted to the object, in bold in example (\ref{ex:2caus}). Base transitive verbs are more interesting: the partipant made to perform the action by the causer receives the Recipient or Experiencer marker \textit{nyako-} \qu{\gl{obl}; put on}\footnote{See \sectref{ssec:nyako}.} (\ref{ex:caus2}), since Vamale has no ditransitive verbs and cannot demote the causee to a core argument.
 

\ea\label{ex:2caus}
% \langinfo{}{}{} 
\gll ya a=fa-thuat {i}={apuli}\\ 
 3\gl{sg} 3\gl{sg}=\gl{caus}-go.out \gl{def}.\gl{sg}=person \\ 
\glt \qu{He had the man released.} {[Hc1:35]}
\z


\ea \label{ex:caus2}
% \langinfo{}{}{} 
\gll a=fa-tena nyakoo-n meeka-n li=aman\\ 
 3\gl{sg}=\gl{caus}-hear \gl{obl}-3\gl{sg} all-\gl{nspec} \gl{def}.\gl{pl}=thing\\ 
\glt \qu{He let him know everything (that was going on).} {[GC:129]}
\z

The subordinating construction \textit{vwa, ma \ldots} \qu{do, so that} is a more common way to express a cause than with \textit{fa-} \qu{\gl{caus}} in (\ref{ex:vwama}) and (\ref{ex:caus_vwa}), but does not change the verb's valency. Other constructions use a transitive main verb, and a purposive subordinate clause (\ref{ex:nyama}). Note that the verbs in (\ref{ex:caus}) and (\ref{ex:caus2}) do not have completely transparent meanings. Most verbs prefixed with \textit{fa-} that were recorded had idiosyncratic meanings, such as (\ref{ex:idio_caus}), which may be a hint that use of the prefix is in decline. 


\ea\label{ex:vwama}
% \langinfo{}{}{} 
\gll a=vwa ma vwa i=siya-n-ea\\ 
 3\gl{sg}=do \gl{subr} do \gl{def}.\gl{sg}=field-\gl{poss}-3\gl{sg}.\gl{poss}\\ 
\glt \qu{She makes him till his/her field.} {[07.11.2018 p.97]}
\z

\ea\label{ex:caus_vwa}
% \langinfo{}{}{} 
\gll Jean, a vwa ma a fe-an-de (i)= fatii-n (nya)si Vaina\\ 
 J. 3\gl{sg}\textsubscript{i} do \gl{subr} 3\gl{sg}\textsubscript{ii} take-go-\gl{dir.cf} \gl{def}.\gl{sg}= word-3\gl{sg}\textsubscript{i}.\gl{poss} \gl{ben} V.\\ 
\glt \qu{Jean made him carry his letter to Vaina.} {[07.11.2018 p.97]}
\z

\ea\label{ex:nyama}
% \langinfo{}{}{} 
\gll Jean a nya si Pierre i=fatii-n ma nya si Vaina\\ 
 J. 3\gl{sg}\textsubscript{i} give \gl{ben} P.  \gl{def}.\gl{sg}=word-3\gl{sg}\textsubscript{i}.\gl{poss} \gl{subr}.3\gl{sg}\textsubscript{ii} give \gl{ben} V.\\ 
\glt \qu{Jean made Pierre carry his letter to Vaina.} {[vamale-181107-jpnelemwa-06:00:19:33-00:19:37]}
\z

%\ea
\ea \label{ex:idio_caus}
\label{ex:stat_caus2}
\gll a=fa-xhopwen\\ 
 3\gl{sg}=\gl{caus}-big\\ 
\glt \qu{S/he is conceited.}
\z 


\section{``Passive"}
\is{Voice!Passive@``Passive''}
\is{ka!Subject-less \textit{ka}@\textit{ka}}
Vamale has no passive, i.e. no construction that adds a dedicated morpheme to the verb to alter the syntactic mapping of the participants, without changing their semantic nature (i.e. by demoting the agent and promoting the undergoer). Verbs whose agents are unidentified may take \textit{le} \qu{3\gl{pl}}, as in (\ref{ex:pass}), or may not be marked for subject at all, see (\ref{ex:thathe}). In subject-less scenarios, the undergoer may be additionally marked with \textit{ka} (\ref{ex:pass2}), which is reminiscent of absolutive \textit{ka} (\sectref{kan}). Since this is the only example with \textit{ka} attested so far, more research is needed in the future to reach a conclusion as to this \textit{ka}'s meaning: \qu{\gl{link}} or \qu{\gl{sbj}}, seen in other verb phrases. %While there is no morphological marking on the verb, i.e. no voice, the possible promotion of the undergoer suggests diathesis; but this \textit{ka} could be an extension of absolutive \textit{ka} otherwise only found in nominalizations. 


\ea\label{ex:pass}
% \langinfo{}{}{} 
\gll mwa (a=) le=vwa ko i=doop-ea\\ 
 house \gl{rel}= 3\gl{pl}=do \gl{obl} \gl{def}.\gl{sg}=soil-3\gl{sg}.\gl{poss}\\ 
\glt \qu{a house made of his earth} {[vamale-181127-jp\_nelemwa-1: 00:13:24-30]}
\z


\ea\label{ex:thathe}
% \langinfo{}{}{} 
\gll bwa xada thathee mwa maahma ca i=dingan xahan thêêdo\\ 
  \gl{ipfv} up.there kill \gl{rep} big.brother in \gl{def}.\gl{sg}=creek over.there T.\\ 
 \glt \qu{Up there, the chief was killed, in the creek over in Tendo.} {[HC1:9]}
\z 

\ea\label{ex:pass2}
% \langinfo{}{}{} 
\gll buuke ka i=aman nya-ko-n i=vin thi\\ 
 destroy \gl{link}? \gl{def}.\gl{sg}=thing put-on-\gl{ana} \gl{def}.\gl{sg}=shell clam\\ 
\glt \qu{Destroyed was the thing on which there is the [Shell Oil] clam shell.} {[KG:545]}
\z



%\section{Introducing non-argument NPs: \textit{nyako}-, \textit{nyasi}-}
%\is{Oblique!Beneficiary}
%The markers discussed in \Cref{sec:Casmark} mark semantic rather than syntactic roles, which suggests that there may not be cases as such.
%If there are no cases, there are no indirect arguments either, because this would imply two obligatory arguments marked differently. Vamale has no ditransitive verbs. Benefactive scenarios are expressed either with verb phrases or with the prepositions \textit{nyako} and \textit{nyasi}. In both cases, the added NP can be omitted. We thus argue that there are no indirect arguments in Vamale. Every NP marked with \textit{nyako / nyasi} \qu{obl} can be omitted without countering the verb's valency \todo{that's not how you say that, is it}. I will thus speak of oblique and (core) arguments. \textit{ko} and \textit{nyasi} / \textit{nyako} are all oblique markers, with different semantic connotations.
%\textit{Nya} \qu{put,give} combines with \textit{ko} \qu{on (adjacent, not on top)} and \textit{si}- \qu{hand, arm}, with a distinction in the patient's involvement in the process. \textit{Nyako}- denotes an asymmetrical relationship, with the patient playing a more passive part. \qu{I punish you}, \qu{I tell you}.
%\textit{Nyasi}-, if we follow the Bwatoo term, introduces a ``destinataire", a goal, who is more subjectively affected \parencite[63]{rivierre_bwatoo_2006}. 
% 
% \ex
% 
% \gll e=hole-ke nya-si-m
% 1\gl{sg} receive-\gl{tr} give-hand-2\gl{sg}.\gl{poss}
%\glt \qu{I thank you}
%
%\z
%
%Both constructions are used to introduce oblique arguments and alienable possessors.
%Before oblique arguments xx
%
%\ex
%
%\gll e=vi aman nya-koo-m
% 1\gl{sg} say something give-on-2\gl{sg}.\gl{poss}
%\glt \qu{I say something to you}
%
%\z

