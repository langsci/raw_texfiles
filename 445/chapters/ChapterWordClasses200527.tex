\chapter{Word classes} % Main chapter title
\label{ChapterWoCla} % Change X to a consecutive number; for referencing this chapter elsewhere, use \ref{ChapterX}

Vamale words can be divided into several groups according to their syntactic behavior: some big, open classes, some small, closed classes%(e.g. the assertive marker \textit{tha} is unique in its distribution), and a number of forms that exist in several classes %(e.g. \textit{balan} \qu{piece; \gl{cont} marker, just})
, and some with just one member. This chapter will present all of them based on distributional clues. Their names are partly derived from names given to cognate forms in other languages, and in all cases strive to reflect their syntactic function. Verbs, nouns, as well as aspectual markers (here called TAM for ``Tense, Aspect, Mood", presented in \sectref{sec:TAM}), have dedicated chapters later on. Vamale word classes present some aspects of typological interest. Location relative to nouns (e.g. ``on", ``next to") is expressed by derived nominal forms (see \sectref{ssec:WCPrepoNouns}, and not adpositions. There are no adjectives; rather, nominalized verb phrases, or stative verbs. This is discussed in \sectref{sec:adj}. Numerals are not an own word class either: they are formally verbs (see \sectref{ssec:Numerals}). Furthermore, as is common in Oceanic languages, verbs and nouns can be derived from each other with little to no morphology and their morphology overlaps in some cases (see \sectref{ssec:PossV} on verbs with nominal morphology). 
%Notes after FZ 14.02.2019:


\section{What is a word?}
\is{Wordhood}

Words in Vamale exist on several levels (phonological and grammatical), and in a way similar to the zero derivation which affects the verb-noun distinction, words can be reinterpreted and forged anew. As has already been sketched in \sectref{sec:Stress}, \textit{g-words}, i.e. words that act as grammatical units, and \textit{p-words}, i.e. units on a prosodic level with one main accent, do not coincide necessarily. One instance of such mismatches are g-words that attach to other g-words to form a phonological unit \parencite[1]{spencer_clitics_2012}, i.e clitics. Many function words are clitics, e.g. articles. Semantically general g-words which are always part of a phrase, but never its head, and which can be stressed (meaning they are also p-words), will be called ``particles" in this grammar. TAM markers are included in this broad category. The same grammatical element may be a p-word in some construction but part of a larger p-word in others, e.g. \textit{aman} \qu{thing}, and the influence of a word on the stress pattern of its environment can depend on pragmatic factors, such as focus, drawing out a pause to think of something further to add, etc. Furthermore, the boundary between often-used phrases and compounds, is also a fuzzy one (cf. \sectref{ssec:vwa} for an example). Other examples of fluidity between word status and phrase-status are verb phrases or even complete relative clauses which can function as head of a NP by the use of an article (see \sectref{sec:NmlzVP} and \sectref{sec:RelCl} respectively), stative verbs whose nominal morphology makes them useful for adjective-like functions (see \sectref{ssec:Verbs_n}), and serial verb constructions. 

\subsection{Concerning \textit{vwa}}
\label{ssec:vwa}
\is{Wordhood!\textit{vwa}}

The verb \textit{vwa} \qu{do} can combine with a number of morphemes to express an action or a state. It can combine with nouns (e.g. \textit{vwa mwa} \qu{build a house}) and verbs (\textit{vwa tau} \qu{to fish (lit. do hit (the water))}) to form compounds.
The compounds that these combinations create are mostly intransitive verbs with often idiosyncratic meanings: \textit{vwa vua} lit. \qu{do net}, \qu{throw a net}. \textit{Vwa uvu} \qu{do yam}, \qu{wind yam sprouts around a tutor}, \textit{vwa xhwaeo} \qu{do taro}, \qu{harvest taro}. Undergoers would be added as oblique arguments with \textit{ko} \qu{\gl{obl}}, see example (\ref{ex:vwa_ko}). Since \textit{vwa} is semantically vague and thus readily used as a filler verb, new compounds are often formed (e.g. \textit{vwa toki} \qu{do metal}, \qu{make a telephone call}), established compounds can broken up ad hoc into verb phrases with specific arguments (see example \ref{ex:vwa_ref}) and many remain transparent phrases. With few exceptions, this grammar will thus avoid marking \textit{vwa}-constructions clearly as compounds by connecting \textit{vwa} to its modifier or argument. \sectref{sec:ZeroTrans} takes a closer look at intransitive verbs.

\ea \label{ex:vwa_ko}
\gll go\textsubscript{\gl{agt}}=vwa-vua ko {\ob}i=vuman nyu{\cb}\textsubscript{\gl{obl}} nya-xahut\\
 2\gl{sg}=do-net \gl{obl} \gl{def}.\gl{sg}=group fish send-down.there\\
\glt \qu{You cast-net upon the fish school down there.}
\z

\ea \label{ex:vwa_ref}
\gll go\textsubscript{\gl{agt}}=vwa li=xhwaeo\textsubscript{\gl{p}} nya-an\\
 2\gl{sg}=do \gl{def}.\gl{pl}=taro towards-same.level\\
\glt \qu{You harvest the taros over there.}
\z

\subsection{Prosodic words and grammatical words}

Vamale's wealth in different classes of morphemes is mirrored on the phonological level: phonological units overlap or not with syntactic units, and the latter can be derived and re-analyzed relatively easily, compare (\ref{ex:rea1}) to (\ref{ex:rea2}). %For example, nouns that are alienably possessed are marked with -\textit{n} \qu{\gl{poss}}. When the possessor is indexed via a personal possessive suffix (1\gl{sg} etc.) rather than a noun, the possessed noun takes the suffixes \textit{-n-eong, -n-go, -n-gau},\ldots See \Cref{tab:morph_overview} for a complete list. 
Possessive constructions follow a head-first pattern: \gl{psm}-\gl{poss} \gl{psr}. The personal possessive suffixes mirror the construction found with nouns, and are cognate with ``indirect" possessive constructions in other languages, where the possessor is expressed as a pronoun, e.g. Pije \textit{thala-n nyan} \qu{knife-\gl{poss} 3\gl{sg}} \parencite[248]{haudricourt_dictionnaire_1982}. However, there are some arguments in favor of a suffix analysis: 
\begin{enumerate}
	\sloppy
	\item The 1\gl{sg} form \textit{-eong} and the 3\gl{sg} form \textit{-ea} are not identical (anymore) with the free pronouns \textit{io} and \textit{ia}.
	\item The syllabification now includes the formerly external possessor markers (i.e. \textit{puakan-ea} [pu.a.ˈka.nɛ̃.ã] \qu{pig.\gl{poss}-3\gl{sg}.\gl{poss}})
	\item The stress shift towards the last syllable before the possessive suffix: [ˈtʰa.la] \qu{knife} $\rightarrow$ [tʰa.ˈla.nɛ̃.ã] \qu{knife-3\gl{sg}.\gl{poss}}
\end{enumerate}


\ea\label{ex:rea1}
[ˌhɲ̊ĩmãke eˈtɕaːmãn]\\
\gll hnyimake (eca)=aman\\
 think.about \gl{indf}.\gl{sg}=thing\\
\glt \qu{think about something.}
\z


\ea\label{ex:rea2}
[i.ˌhɲ̊ĩ.mãˈke.a.mãn]\\
\gll i=hnyimake-aman\\
 \gl{def}.\gl{sg}=think-thing\\
\glt \qu{the thinking of something}
\z	
	
%They are pronounced like suffixes. Given the absence of infixes, I must try to squeeze something inbetween.

\subsection{Affixes and clitics}
\is{Wordhood!Affixes}
\is{Wordhood!Clitics}
\begin{sloppypar}
Vamale morphemes cover much of the spectrum of grammatical wordhood: forms that are units on both syntactic and phonological grounds, some that are either one or the other, and some flexibility depending on the context.  
This grammar will follow \textcite{spencer_clitics_2012} in defining clitics: A clitic is a form of a word which is phonologically attached to another word, its host. Clitics cannot be stressed, there cannot be a pause between them and their host, they have the same functions as g-words, and can attach to various kinds of syntactic units \parencite[1]{spencer_clitics_2012}.
\end{sloppypar}

\begin{enumerate}
	\item Verbs, nouns, and adverbs are g-words as well as p-words, because they can move relatively freely and can be stressed.
	\item Subject markers and articles are proclitics, tied to a predicate (verbal or nominal), and noun phrase respectively, they cannot be stressed.
	\item Aspectual markers are particles because they are phonologically autonomous, but since they are grammatically dependent on predicates, Zúñiga's term ``anti-clitic" may be more precise \parencite{zuniga_anti_2014}.

\end{enumerate}

%\begin{itemize}
%	\item	i=xhong
%	\item	i=juu xhong
%	\item	i=joakan sapwen
%	\item	i=juu joakan ma juu xhopwen juu sapwen-eong (the real thick and real big traditional garment of mine)
%	
%	
%	\item	e=xaleke
%	\item	e=juu xaleke
%	\item	e=bwa juu xaleke
%	
%	\item	e=xale-a
%	\item	*e=xale juu a
%	
%	\item	sinu-ong
%	\item	*sinu 
%\end{itemize}

\section{The distinction between verbs and nouns}
\is{Wordhood!Verbs vs nouns}
A common problem in Oceanic linguistics concerns the distinction between verbs and nouns, which can be considered blurry. A verb phrase can be nominalized just by putting an article in front of it, after which it functions as an argument. A noun phrase can be used as a predicate, and preceded by aspect and modality markers. As discussed by \textcite[3]{creissels_non-verbal_}, one can distinguish three types of non-verbal predicates: nominal, adjectival, and locative. Vamale has no adjectives, but nominal and locative predicates are well attested. example (\ref{ex:le_niehni}) is an example of a pronominal predicate, (\ref{ex:ca_i_bureau}) shows a predicative locative noun phrase, composed of a prepositional noun \textit{ca} \qu{in} and a common noun, and a nominal predicate is shown in (\ref{ex:predN}). 


\ea\label{ex:le_niehni}
%% \ili{}{}{} 
\gll li=bee-m, le=niehni\\
 \gl{def}.\gl{pl}=peer-2\gl{sg}.\gl{poss} 3\gl{pl}=all.those\\
\glt \qu{Your friends, they are those over there.} {[}X4:14]
\z



\ea\label{ex:ca_i_bureau}
%\ili{}{}{} KG:515
\gll cahma patron ya tha=a=juu ena a ca i=bureau hihihi\\
 \gl{top} boss 3\gl{sg} \gl{ass}=3\gl{sg}=very \gl{dist} 3\gl{sg} in \gl{def}.\gl{sg}=office \\
\glt \qu{The boss however, there he was in his office.} {[}KG:515]
\z


%here? e=daahma. "i am (a) chxxief." grammar works on the basis of predicates, not verbs. subject markers attach to predicates, not "verbs". 

%Example of \textit{la} for null-derivation:
%\ea
%\a
%
%\ili{}{}{} 2019-07-22 JP see me: 14
%\gll lu=moo ca la a=see la
%
% 3\gl{du}=stay in be.here \gl{rel}=same be.here
%
%\glt \qu{They stay in the same place (lit they stay in a place that is there )}
%
%
%\z

\ea\label{ex:predN}
\gll go=bo i=daahma\\
 2\gl{sg}=\gl{irr} \gl{def}.\gl{sg}=chief\\
\glt \qu{You would be chief.}
\z


The transitive suffix \textit{-ke} can be found in or at the end of nouns, but only nouns of verbal origin. 
\begin{itemize}
%	\item \textit{xhwi aman} \qu{eat something, food}
	\item \textit{hnyimake} \qu{think, pay attention}, \textit{hnyimake-aman} \qu{think something}, \textit{i ape-hnyimake-aman} \qu{\gl{def}.\gl{sg} \gl{nmlz}-think-thing}, \qu{the thought}
	\item \textit{vwa-siteke} \qu{Sunday, (lit. do-sacred, pray)}
	\item \textit{e-topweeke-aman} \qu{hook, (lit. \gl{nmlz}-hang-thing)}
%	\item \textit{ba} \qu{wall}, \textit{bake} \qu{stack stones}
%	\item fubun \qu{heap}, \textit{fubuuke} \qu{make a heap of something}
\end{itemize}

\subsection{Syntactic criteria}

While nouns and verbs share some syntactic slots, and some nouns and verbs may look the same (see \sectref{ssec:PossV}), and many of the words surrounding both noun and verb phrases have shared origins, like \textit{ma} \qu{\gl{subr}, \gl{com}}, \textit{ko} \qu{on, \gl{obl}}, or \textit{xhwat} \qu{a small piece}, \qu{a little bit}, there are differences still: nouns cannot take arguments, they have different modifiers, and if TAM markers are used, they do not mean the same thing.
Numerals, for instance, are verbs because they take arguments, e.g. \textit{se-a} \qu{the only one, he is one/alone}, \textit{thaloo mu mani} \qu{the birds are two}, \textit{thien li=ba} \qu{the sardines (are) three}. \textit{Nievit} \qu{how many} works the same way: \textit{Nievit sinan xat?} \qu{how.many sign sun}, \qu{what time is it?}.%, singular maybe because the answer can be \textit{se} "one", or because the referent of \textit{sinan} is a sign on the clock? \textit{Nievit-le}.

Within a noun phrase, modifiers are either nominalized verbs that precede the noun, as in (\ref{ex:WCadj1}), possessors, or relative clauses, as in (\ref{ex:WCadj2}). Verbal modifiers are either adverbs or adverbial clauses, or verbs that integrate the verb phrase. 


\ea\label{ex:WCadj1}
\gll i=bwaakala a=xhopwen\\
 \gl{def}.\gl{sg}=boat \gl{rel}=big\\
\glt \qu{the big boat}
\z


\ea\label{ex:WCadj2}
\gll i=joakan sapwen\\
 \gl{def}.\gl{sg}=thick dress\\
\glt \qu{the thick dress}
\z 

%Distributional not clear, but morphological and semantic criteria. Hier erwähnen: Adjektive werden wie ersetzt? Sind die anders als andere Verben?

\subsection{Morphological criteria}
\label{ssec:morph_crit}
\is{Morphology}
\begin{sloppypar}
Verbs and nouns share some morphology, most notably possessive suffixes. While these suffixes are examined in more detail in \sectref{ssec:PossV} and \sectref{ssec:active-n} for verbs, and in \sectref{sec:Poss} for nouns, an overview is provided in \Cref{tab:morph_overview}. Inalienable (I and Ib) and alienable (II) possessive suffixes can thus occur on verbs, either to mark the undergoer-like subject of some stative verbs or the undergoer argument of some active transitive verbs. 
\end{sloppypar}
 
\begin{table}
	\centering
	\caption{Possessive suffixes, \textsc{obj} and -\gl{S\textsubscript{P}}}
	
	\begin{tabular}{llccccc}
	\lsptoprule
	%	&&&\multirow{3}{}{Possession}&&&\\
		&& I& Ib& II & -\gl{S\textsubscript{P}} &\gl{obj}\\\midrule
\gl{sg} &	1&	\textit{-ng}&	\textit{-ong}&	\textit{-eong} &\textit{-ong} & \textit{-eo}\\
		&	2&	\textit{-m}&	\textit{-am}&	\textit{-go}&\textit{-go} & \textit{-ko}\\
		&	3&	\textit{-n}&	\textit{-an}&	\textit{-ea} & \textit{-(e)a} & \textit{-}a \\
		\midrule
\gl{du} &	1\gl{incl}&\textit{-ju}&	\textit{-aju}&\textit{-ju} &\textit{-gaeu/-gasu} &\textit{-kaeu}\\
		&	1\gl{excl}&	\textit{-bu}&	\textit{-abu}&	\textit{-bu}& \textit{-gabu}&\textit{-kabu}\\
		&	2&	\textit{-u}&	\textit{-au}&	\textit{-gau} &\textit{-gau} & \textit{-kau}\\
		&	3&	\textit{-lu}&	\textit{-alu}&	\textit{-lu }&\textit{-lu} & \textit{-lu} \\
		\midrule
\gl{pl} &	1\gl{incl}&	\textit{-j}e&\textit{-aje}&	\textit{-je} &\textit{-gaa} & \textit{-kaa}\\
		&	1\gl{excl}&	\textit{-be}&	\textit{-abe}&	\textit{-be} & \textit{-abe}&\textit{-kabe} \\
		&	2&	\textit{-vwe}&	\textit{-avwe}&	\textit{-vwe} &  \textit{-gavwe}& \textit{-kavwe}\\
		&	3&\textit{-le}&\textit{-ale}&	\textit{-le} & \textit{-le }&\textit{-le}\\
	\lspbottomrule
	\end{tabular}
\label{tab:morph_overview}
\end{table}
 
Reflexive/reciprocal \textit{e-}, mostly found on verbs, also occurs with nouns in predicate function, as shown in (\ref{ex:refl_nouns}–\ref{ex:refl_nouns2}). This is a relatively rare occurrence and was only attested with a reciprocal meaning.

\ea \label{ex:refl_nouns}
%\ili{}{}{} vamale-180809-1:00:15:20
\gll lu=e-\textit{copain}-\textit{copine}\\
 3\gl{du}=\gl{recp}-boyfriend-girlfriend\\
\glt \qu{The two are boyfriend and girlfriend.} {[}vamale-180809-1:00:15:20]
\z


\ea \label{ex:refl_nouns2}
\gll calibeen ma le=moo ma li=ehni, e-bee-le\\
 sometimes \gl{subr} 3\gl{pl}=stay \gl{com} \gl{def}.\gl{pl}=\gl{dem}.\gl{prox}  \gl{recp}-peer-3\gl{pl}.\gl{poss}\\
\glt \qu{Sometimes when they stay together with those, they are each other's cousins.} {[}AG1:239]
\z

Some particles and affixes, however, mark the resulting construction as definitely verbal. This includes causative \textit{fa-}, transitive markers \textit{-ke} and \textit{-i}, and manner prefixes (discussed in \sectref{sec:Manner}). For example, \textit{xhaavwa} in (\ref{ex:xhavwa}) has a different stem than \textit{xhavwaleke} in (\ref{ex:xhavwaleke}), both in vowel quantity and the last syllable \textit{-le}.


\ea\label{ex:xhavwa}
\gll e=xhaavwa\\
 1\gl{sg}=wait\\
\glt \qu{I'm waiting around.}
\z

\ea\label{ex:xhavwaleke}
\gll e=xhavwale-ke (aman)\\
 1\gl{sg}=wait.for-\gl{tr} something\\ 
\glt \qu{I'm waiting for something.}
%\a
%
%\gll xa-xhavwale-ke (aman)
% \gl{nmlz}-wait.for-\gl{appl} something 
%\glt \qu{Someone who waits for something}
%
\z

Object markers in general are sure signs of verbhood. §\ref{sec:WC_nouns}--\ref{sec:WC_sbjka} below will explore the noun phrase with its members: articles, nouns, demonstrative and personal pronouns (followed by the subject marker proclitics which are historically related), and some case markers.

TAM markers, which, like subject markers, are shared by both predicative nouns as well as verbs, will form the link in this chapter between the two big groups. Verbs will only be sketched here, as they form the most diverse and biggest word class of the language. After introducing adverbs, smaller classes, often with a wider scope, will be introduced.

%\subsubsection{\textit{-n}}
\is{Specificity!\textit{-n}}
The suffix \textit{-n} \qu{\gl{nspec}, \gl{ana}} warrants an early introduction, as it occurs across word classes, affecting dependent verbs, prepositional nouns and regular inalienable nouns alike. In essence, \textit{-n} has two functions: one is to mark the generic nature of the argument of its verbal host (\ref{ex:verbal-n}), or the generic possessor of its nominal host (\ref{ex:nominal-n}), while the other function is anaphoric or cataphoric. Generic nouns do not take an article, whereas specific nouns take definite or indefinite articles (see \sectref{sec:WCArticles}). 


\ea\label{ex:verbal-n}
\gll ka na naen cipa hmwaka-n habu\\
 \gl{cnj} \gl{dem} now \gl{neg} be.like-\gl{nspec} long.ago\\
\glt \qu{But this is today, not like [things were) yesterday.} {[KP:12]}
\z


\ea\label{ex:nominal-n}
\gll kon th=e ra-ta-meebam nyeca-n sohmun\\
 then \gl{ass}=1\gl{sg} \gl{nmlz}-be.sitting-sleep inside-\gl{nspec} school\\
\glt \qu{You must know that I kept sleeping on my chair at school.} {[PE1:187]}
\z 

%Two other suffixes \textit{-n} must be mentioned: in noun phrases, a suffix \textit{-n} indexes inalienable 3\gl{sg} possessors on the possessed noun, be it specific (\ref{ex:inalienable}) or generic (\ref{ex:nspec}). Another suffix \textit{-n} serves as an alienable possessum-marking construct suffix (no indication of person), as in (\ref{ex:alienable}) (see \sectref{sec:Poss} for a more detailed discussion). The two \textit{-n} are distinguished on the basis of their different distribution. Inalienable \textit{-n} \qu{3\gl{sg}.\gl{poss}} is part of a paradigm of possessive pronoun marking suffixes, excluding a possessor NP following, while the construct suffix has no allomorphs, grammatically conditioned or otherwise, and occurs between a nominal possessor and its possessum. As the (in)alienability of nouns is lexically conditioned, the two suffixes do not contrast on the same word.
%
%
%\ea\label{ex:inalienable}
%\gll i=bwa-n/-m\\
% \gl{def}.\gl{sg}=head-3\gl{sg}.\gl{poss}/2\gl{sg}.\gl{poss}\\
%\glt \qu{his/your head}
%\z
%
%
%\ea\label{ex:nspec}
%\gll i=bwa-n apuli\\
% \gl{def}.\gl{sg}=head-3\gl{sg}.\gl{poss} human\\
%\glt \qu{the human head}
%\z
%
%
%\ea\label{ex:alienable}
%\gll i=mwa-n i=apuli\\
% \gl{def}.\gl{sg}=house-\gl{poss} \gl{def}.\gl{sg}=person\\
%\glt \qu{the house of the person}
%\z

The suffix \textit{-n} \qu{\gl{nspec}, \gl{ana}} is analyzed as one single suffix despite its distribution both on nouns and verbs, as it always marks the head of the phrase and the generic nature of its dependents. \is{Specificity! Generic \textit{-n}}

%\todo{s that a possessive marker ?	You need to give a closer analysis of these –n, not just give them labels, to demonstrate their category and function}

\is{Anaphoric \textit{-n}}The other function mentioned above is anaphora in the wider sense: relating to something already mentioned (\ref{ex:n_ana}), mentioned in another clause soon after, or known in general (\ref{ex:hmwaana_koon}). The referent's specificity is not important in this case. %The syntactic roles avalailable to the participant indexed with \textit{-n} are lower ones: objects, the modifiers of prepositional phrases, and, in the case of nominalizations featuring \textit{ka-n}, intransitive or inanimate subjects. 


\ea\label{ex:n_ana}
\gll ehni xhwan da, abe=fate gavwe koo-\textbf{n} go abe=sate-\textbf{n}, cipa hmai-n, cipa-bu ju-vaa udu hmwaka-u\\
 \gl{prox} little what 1\gl{pl}.\gl{excl}=share 2\gl{pl} \gl{obl}-\gl{ana} then 1\gl{pl}.\gl{excl}=be.different-\gl{ana} \gl{neg} be.many-\gl{nspec} \gl{neg}-1\gl{du}.\gl{excl} real-too drink like-2\gl{du}\\
\glt \qu{This [wine and beer) is nothing much, we share with you of this, and [as for) us, it's different, not much, we don't drink as hard as you do.} {[L2:3]}
\z


\ea\label{ex:hmwaana_koon}
\gll hmwaa-na koo-n!\\
 like-\gl{dist} on-\gl{ana}\\
\glt \qu{Like that!/Now that's a proper way of doing it!}
\z

\subsection{Semantic criteria}

As in other Kanak languages \parencites[89]{bril_nelemwa_2002}[32]{rivierre_bwatoo_2006}, there are few roots exclusively dedicated to one category, and they are mostly nominal roots. Examples include kinship terms, body parts, topographical and meteorological terms, many animals and plants, as well as parts of the house, boats, and tools. 

\section{Articles}
\label{sec:WCArticles}
\is{Articles}
%members: \textit{i}/\textit{vi}, \textit{mu}, \textit{ni}/\textit{li}, \textit{muca}, \textit{eca}, \textit{ca}
%\\criteria: come before a noun, are part of the NP
%\\size: 6

%\begin{letter}{I}

\fmtLemma{iba}
\fmtCitation{(o)i\-ba}
\fmtPOS{n}%
\fmtNC{1a/2}
\fmtSense{husband}
\fmtExampleRL{Oi\-ba waa\-mwe mu\-sai\-za mu\-koo\-to.}
\fmtExampleEN{Her husband is a big man.}

\fmtLemma{ibaa\-gi\-ro}
\fmtCitation{(e)i\-baa\-gi\-ro}
\fmtPOS{n}%
\fmtNC{5/6}
\fmtSense{abattoir, slaughterhouse}
\fmtExampleRL{Enya\-ma ngi\-guu\-ra mu ibaa\-gi\-ro.}
\fmtExampleEN{I buy meat from the abattoir.}

\fmtLemma{ibaa\-ti}
\fmtCitation{(e)i\-baa\-ti}
\fmtPOS{n}%
\fmtNC{5/6}
\fmtSource{from Ar.\,via Sw.}
\fmtSense{iron sheet}
\fmtExampleRL{Eki\-ka kya ibaa\-ti ki\-ni ki\-wa\-ngaa\-li.}
\fmtExampleEN{This type of iron sheet is long lasting.}

\fmtLemma{ibai\-zi\-ro}
\fmtCitation{(e)i\-bai\-zi\-ro}
\fmtPOS{n}%
\fmtNC{5/6}
\fmtSense{carpentry workshop}
\fmtExampleRL{Eki\-ko\-nko\-ro nki\-guu\-re mu ibai\-zi\-ro.}
\fmtExampleEN{I bought the stool in the carpentry workshop.}

\fmtLemma{iba\-la}
\fmtCitation{(e)i\-ba\-la}
\fmtPOS{n}%
\fmtNC{5/6}
\fmtSense{stain, spot, smudge}
\fmtExampleRL{Olu\-go\-ye lwa\-mwe lu\-li\-ku ei\-ba\-la eri\-ra\-gu\-zu.}
\fmtExampleEN{His garment has a black stain.}

\fmtLemma{iba\-nza}
\fmtCitation{(e)i\-ba\-nza}
\fmtPOS{n}%
\fmtNC{5/6}
\fmtSense{debt}
\fmtExampleRL{Oki\-mai\-te\-ku nti nta n’ei\-ba\-nza lya\-mu?}
\fmtExampleEN{Do you know that I have your debt?}

\fmtLemma{iba\-ra}
\fmtCitation{(e)i\-ba\-ra}
\fmtPOS{n}%
\fmtNC{5/6}
\fmtSense{name}
\fmtExampleRL{Nko\-be\-ra ei\-ba\-ra lya\-mu erya ki\-ka.}
\fmtExampleEN{Tell me your clan name.}

\fmtLemma{iba\-ti\-ri\-zo}
\fmtCitation{(e)i\-ba\-ti\-ri\-zo}
\fmtPOS{n}%
\fmtNC{5/6}
\fmtSense{baptismal font (an article of church furniture used for baptism)}
\fmtExampleRL{Ei\-ba\-ti\-ri\-zo lya ba\-po\-le\-si\-ta\-nti li\-bba mu ka\-ni\-sa.}
\fmtExampleEN{The baptismal font is found in the church.}

\fmtLemma{ibba}
\fmtCitation{(okw)i\-bba}
\fmtPOS{v}%
\fmtTrans{tr}
\fmtPFV{ibbi\-re}
\fmtSense{to steal, to rob}
\fmtExampleRL{Ai\-bba omu\-wo\-go gwa mu\-lu\-mi ampi bu\-li ki\-ro.}
\fmtExampleEN{He steals the farmer's cassava almost every day.}

\fmtLemma{ibbaa\-le}
\fmtCitation{(e)i\-bbaa\-le}
\fmtPOS{n}%
\fmtNC{5/6}
\fmtSubentry(a)
\fmtSense{stone, rock}
\fmtExampleRL{Omwa\-na aku\-cu\-ma ama\-bbaa\-le oku nyu\-mba.}
\fmtExampleEN{The child is throwing stones at the house.}
\fmtSubentry(b)
\fmtSense{(water) ice}
\fmtExampleRL{Amai\-zi ga\-koo\-re ama\-bbaa\-le.}
\fmtExampleEN{The water has turned into ice.}
\fmtSubentry(c)
\fmtSense{Ibbaale is a place of cultural significance where a number of shrines are found, it also has caves believed to be the habitat of the ghost called Ibbaale}

\fmtLemma{ibba\-go}
\fmtCitation{(e)i\-bba\-go}
\fmtPOS{n}%
\fmtNC{5/6}
\fmtSubentry(a)
\fmtSense{draft, preliminary version}
\fmtExampleRL{So\-ma\-mu omu ibba\-go lya\-nge.}
\fmtExampleEN{Read through my draft please.}
\fmtSubentry(b)
\fmtSense{bill, draft law, proposed piece of legislation, proposal}
\fmtExampleRL{Ei\-bba\-go lya itee\-ka li\-twa\-lii\-bwe omu nga\-ngo.}
\fmtExampleEN{The bill has been taken to parliament.}

\fmtLemma{ibba\-nga}
\fmtCitation{(e)i\-bba\-nga}
\fmtPOS{n}%
\fmtNC{5}
\fmtSubentry(a)
\fmtSense{sky}
\fmtExampleRL{Enye\-nyee\-zi zi\-ri mu ibba\-nga.}
\fmtExampleEN{The stars are in the sky.}
\fmtSubentry(b)
\fmtSense{air}
\fmtExampleRL{Ende\-ge eta\-mbuu\-ra mu ibba\-nga.}
\fmtExampleEN{The aeroplane flies in the air.}
\fmtSubentry(c)
\fmtSense{space}
\fmtExampleRL{Wa\-ndii\-ka\-nga nga oku\-le\-ka\-wo ama\-bba\-nga aka\-ti wa bi\-ga\-mbo.}
\fmtExampleEN{Leaving spaces between words when writing.}
\fmtSubentry(d)
\fmtSense{time, period}
\fmtExampleRL{Aba\-ruu\-li ba\-maa\-re ei\-bba\-nga lya mya\-ka iku\-mi na isa\-tu nga bee\-re\-me\-re.}
\fmtExampleEN{The Baruuli have been independent for a period of thirteen years.}

\fmtLemma{ibba\-nge}
\fmtCitation{(e)i\-bba\-nge}
\fmtPOS{n}%
\fmtNC{5}
\fmtSense{freedom, liberty}
\fmtExampleRL{OKa\-nca ni\-ye ya\-tu\-wai\-re ei\-bba\-nge lya ku\-si\-nza.}
\fmtExampleEN{It is God who gave us the freedom to worship.}

\fmtLemma{ibba\-ngo}
\fmtCitation{(e)i\-bba\-ngo}
\fmtPOS{n}%
\fmtNC{5/6}
\fmtSubentry(a)
\fmtSense{hunchback}
\fmtExampleRL{Ei\-bba\-ngo lya mu\-ka\-li li\-ri\-ku ekyo\-ya.}
\fmtExampleEN{The woman's hunchback is infected with a wound.}
\fmtSubentry(b)
\fmtSense{(of an animal) hump}
\fmtExampleRL{Engu\-ndu zi\-bba n’a\-ma\-bba\-ngo aga\-koo\-to.}
\fmtExampleEN{Bulls usually have big humps.}

\fmtLemma{Ibba\-ngo}%
\fmtOne{1}
\fmtCitation{(o)Ibba\-ngo}
\fmtPOS{n}%
\fmtNC{1a}
\fmtSubentry(c)
\fmtSense{royal name for Bunyala kings}
\fmtExampleRL{Ibba\-ngo iba\-ra lya mu\-ka\-ma wa Bu\-nya\-la.}
\fmtExampleEN{Ibbango is a name given to the king of Bunyala.}

\fmtLemma{ibba\-ni}
\fmtCitation{(e)i\-bba\-ni}
\fmtPOS{n}%
\fmtNC{5}
\fmtSense{breast infection, mastitis}
\fmtExampleRL{Omu\-ka\-li alwai\-re ei\-bba\-ni.}
\fmtExampleEN{The woman has mastitis.}

\fmtLemma{ibba\-no}%
\fmtOne{1}
\fmtCitation{(e)i\-bba\-no}
\fmtPOS{n}%
\fmtNC{5}
\fmtSubentry(a)
\fmtSense{misfortune, disaster}
\fmtExampleRL{Ei\-bba\-no li\-gwi\-re oku kya\-lo.}
\fmtExampleEN{A misfortune has befallen the village.}
\fmtSubentry(b)
\fmtSense{tragedy (e.g. when speaking of death)}
\fmtExampleRL{Omu\-ntu okwe\-zi\-nda li\-bba ibba\-no.}
\fmtExampleEN{It is a tragedy for one to commit suicide.}

\fmtLemma{ibba\-no}%
\fmtOne{2}
\fmtPOS{interj}
\fmtSense{used to address mourners, my condolences}
\fmtExampleRL{Ibba\-no!}
\fmtExampleEN{What a pity!}

\fmtLemma{ibbi\-ka}
\fmtCitation{(okw)i\-bbi\-ka}
\fmtPOS{v}%
\fmtTrans{tr}
\fmtPFV{ibbi\-ki\-re}
\fmtSense{to dip, to immerse}
\fmtExampleRL{Ai\-bbi\-ka eki\-ba\-ju omu ma\-lwa.}
\fmtExampleEN{He dips the calabash in the local brew.}

\fmtLemma{ibbi\-ra}
\fmtCitation{(okw)i\-bbi\-ra}
\fmtPOS{v}%
\fmtTrans{intr}
\fmtPFV{ibbii\-re}
\fmtSense{to drown}
\fmtExampleRL{Ya\-yi\-bbii\-re omu nya\-nja.}
\fmtExampleEN{He drowned in the lake.}

\fmtLemma{ibbo}
\fmtCitation{(o)i\-bbo}
\fmtPOS{n}%
\fmtNC{1a}
\fmtSense{cowpea (\textit{Vigna ungui\-cu\-la\-ta}) powder (made from crushed dried cooked leaves of the cowpea plant)}
\fmtExampleRL{Oibbo wa maa\-ma awu\-mbi\-re.}
\fmtExampleEN{My mother's cowpea powder has been eaten by worms.}

\fmtLemma{ibbo\-mbo}
\fmtCitation{(e)i\-bbo\-mbo}
\fmtPOS{n}%
\fmtNC{5}
\fmtSense{\textit{Mo\-mo\-rdi\-ca foe\-ti\-da} (a perennial climbing vine)}
\fmtExampleRL{Ei\-bbo\-mbo ri\-ta\-mba eki\-koo\-ro.}
\fmtExampleEN{\textit{Mo\-mo\-rdi\-ca foetida} cures cough.}

\fmtLemma{ibbo\-ngo}
\fmtCitation{(e)i\-bbo\-ngo}
\fmtPOS{n}%
\fmtNC{5/6}
\fmtSense{tooth gap}
\fmtExampleRL{OKya\-ku\-wa ali\-na ama\-bbo\-ngo ya\-gwi\-re amai\-no ni ga\-zwa\-mu.}
\fmtExampleEN{Kyakuwa has tooth gaps: she lost her teeth when she fell.}

\fmtLemma{ibbu\-gu\-mu}
\fmtCitation{(e)i\-bbu\-gu\-mu}
\fmtPOS{n}%
\fmtNC{5}
\fmtSubentry(a)
\fmtSense{warmth, heat, hotness}
\fmtExampleRL{Ebi\-ro bi\-ni e Na\-ka\-so\-ngo\-lo ndoo\-yo bbu\-gu\-mu.}
\fmtExampleEN{These days there is no warmth in Nakasongola.}
\fmtSubentry(b)
\fmtSense{excitement, thrill, pleasure, delight, joy}
\fmtExampleRL{Abee\-gi ba\-fu\-ni\-re ei\-bbu\-gu\-mu.}
\fmtExampleEN{The students have gotten some excitement.}

\fmtLemma{ibbu\-ka}%
\fmtOne{1}
\fmtCitation{(okw)i\-bbu\-ka}
\fmtPOS{v}%
\fmtTrans{tr}
\fmtPFV{ibbu\-ki\-re}
\fmtSense{to soak}
\fmtExampleRL{Omwa\-na akwi\-bbu\-ka engo\-ye omu bbe\-nsa\-ni.}
\fmtExampleEN{The child is soaking the clothes in the washing basin.}

\fmtLemma{ibbu\-ka}%
\fmtOne{2}
\fmtPOS{interj}
\fmtSense{expression used in response to a child's sneeze, lit.\,open}
\fmtExampleRL{Omwa\-na ni\-ye ye\-syai\-mu\-la ba\-mwi\-ra\-mu nti, ibbu\-ka!}
\fmtExampleEN{When a child sneezes, people around say ``Ibbuka!".}

\fmtLemma{ibbu\-lu\-ku\-ka}
\fmtCitation{(okw)i\-bbu\-lu\-ku\-ka}
\fmtPOS{v}%
\fmtTrans{intr}
\fmtPFV{ibbu\-lu\-ku\-ki\-re}
\fmtSubentry(a)
\fmtSense{to emerge, to come into view}
\fmtExampleRL{Omu\-saa\-nyi akwi\-bbu\-lu\-ku\-ka omu mai\-zi.}
\fmtExampleEN{The swimmer is emerging from the water.}
\fmtSubentry(b)
\fmtSense{to recover financially}
\fmtExampleRL{Ya\-bbai\-re asee\-ge\-re bai\-tu ati\-bu\-ni ai\-bbu\-lu\-ku\-ki\-re.}
\fmtExampleEN{He incurred a financial loss, but he has now recovered.}

\fmtLemma{ibbu\-lu\-la}
\fmtCitation{(okw)i\-bbu\-lu\-la}
\fmtPOS{v}%
\fmtTrans{tr}
\fmtPFV{ibbu\-lwi\-re}
\fmtSense{to recover something from water}
\fmtExampleRL{Bai\-bbu\-lwi\-re obwa\-to omu mai\-zi.}
\fmtExampleEN{They recovered the boats from the water.}

\fmtLemma{ibbu\-mba}
\fmtCitation{(e)i\-bbu\-mba}
\fmtPOS{n}%
\fmtNC{5}
\fmtSense{clay}
\fmtExampleRL{Ei\-bbu\-mba tu\-li\-bu\-mbi\-sya nsu\-wa.}
\fmtExampleEN{We use the clay to mould pots.}

\fmtLemma{ibbu\-tyai}
\fmtCitation{(e)i\-bbu\-tyai}
\fmtPOS{n}%
\fmtNC{5}
\fmtSense{fermented millet flour}
\fmtExampleRL{Lee\-ta oi\-bbu\-tyai tu\-mu\-sii\-ke.}
\fmtExampleEN{Bring the fermented millet flour and let's roast it.}

\fmtLemma{ibee\-re}
\fmtCitation{(e)i\-bee\-re}
\fmtPOS{n}%
\fmtNC{5/6}
\fmtSense{breast}
\fmtExampleRL{Omwa\-la amee\-re ama\-bee\-re.}
\fmtExampleEN{The girl has grown breasts.}

\fmtLemma{ibe\-ga}
\fmtCitation{(e)i\-be\-ga}
\fmtPOS{n}%
\fmtNC{5/6}
\fmtSubentry(a)
\fmtSense{back}
\fmtExampleRL{Ta omwa\-na oku ibe\-ga.}
\fmtExampleEN{Put the child on your back.}
\fmtSubentry(b)
\fmtSense{shoulder}
\fmtExampleRL{Omu\-gu\-gu aku\-gu\-twa\-ra ku ibe\-ga.}
\fmtExampleEN{He carries the luggage on his shoulder.}

\fmtLemma{ibii\-ki\-ro}
\fmtCitation{(e)i\-bii\-ki\-ro}
\fmtPOS{n}%
\fmtNC{5/6}
\fmtSubentry(a)
\fmtSense{store, storeroom, storehouse}
\fmtExampleRL{Ebi\-yi\-mba bi\-bii\-ke omu ibii\-ki\-ro.}
\fmtExampleEN{Keep the beans in the storeroom.}
\fmtSubentry(b)
\fmtSense{treasury}
\fmtExampleRL{Empii\-ya mu\-zi\-te omu ibii\-ki\-ro.}
\fmtExampleEN{Put the money in the treasury.}
\fmtSubentry(c)
\fmtSense{archive}
\fmtExampleRL{E Nsa\-mbya wa\-li\-yo ei\-bii\-ki\-ro eri\-koo\-to.}
\fmtExampleEN{There is a big archive at Nsambya.}

\fmtLemma{ibi\-ra}
\fmtCitation{(e)i\-bi\-ra}
\fmtPOS{n}%
\fmtNC{5/6}
\fmtSense{hook, fishhook, barb}
\fmtExampleRL{Ei\-bi\-ra li\-kwa\-ta mpuu\-ta.}
\fmtExampleEN{The hook is used to catch the Nile perch.}

\fmtLemma{ibi\-ri}
\fmtCrossreference{used in isolation and with class 9 nouns, see (-)biri}
\fmtPOS{num}
\fmtSense{two}
\fmtExampleRL{Tu\-li ku na\-mba abi\-ri na ibi\-ri.}
\fmtExampleEN{We are number twenty two.}

\fmtLemma{ibu\-ba}
\fmtCitation{(e)i\-bu\-ba}
\fmtPOS{n}%
\fmtNC{5}
\fmtSubentry(a)
\fmtSense{benign tumour, painless tumour}
\fmtExampleRL{Mbai\-re n’ei\-bu\-ba li\-ni oku\-zwa ni na\-bya\-lii\-bwe.}
\fmtExampleEN{I have had this benign tumour since I was born.}
\fmtSubentry(b)
\fmtSense{state of being possessive}
\fmtExampleRL{Aba\-sai\-za ba\-ta n’ei\-bu\-ba oku ba\-ka\-li baa\-bwe.}
\fmtExampleEN{Men are very possessive of their wives.}

\fmtLemma{ibu\-ga}
\fmtCitation{(e)i\-bu\-ga}
\fmtPOS{n}%
\fmtNC{5/6}
\fmtSense{large gourd}
\fmtExampleRL{Ntee\-ra omwe\-nge gwa\-nge omu ibu\-ga.}
\fmtExampleEN{Put my alcohol into the large gourd.}

\fmtLemma{ibu\-la}
\fmtCrossreference{\textnormal{see} ibura}
\fmtDialect{Lunyala}

\fmtLemma{ibu\-mbi\-ro}
\fmtCitation{(e)i\-bu\-mbi\-ro}
\fmtPOS{n}%
\fmtNC{5/6}
\fmtSense{potter's workshop, pottery}
\fmtExampleRL{Omu\-bu\-mbi abu\-mbi\-ra mu ibu\-mbi\-ro.}
\fmtExampleEN{The potter works at the pottery workshop.}

\fmtLemma{ibu\-nda}
\fmtCitation{(e)i\-bu\-nda}
\fmtPOS{n}%
\fmtNC{5/6}
\fmtSense{side of the stomach or abdomen}
\fmtExampleRL{Oso\-bo\-la oku\-bo\-na ei\-bu\-nda lya mbu\-li.}
\fmtExampleEN{You can see the side of the goat's abdomen.}

\fmtLemma{ibu\-ra}%
\fmtOne{1}
\fmtDialect{Ruruuli}
\fmtCitation{(e)i\-bu\-ra}
\fmtPOS{n}%
\fmtNC{5}
\fmtSense{scarcity, shortage, undersupply, insufficiency, scarceness}
\fmtExampleRL{Ebi\-ro bi\-ni wa\-li\-wo ei\-bu\-ra lya me\-re.}
\fmtExampleEN{There is a shortage of food these days.}

\fmtLemma{ibu\-ra}%
\fmtOne{2}
\fmtDialect{Ruruuli}
\fmtCitation{(okw)i\-bu\-la}
\fmtPOS{v}%
\fmtTrans{ditr}
\fmtPFV{ibwi\-re}
\fmtSense{to serve (food), dish out}
\fmtExampleRL{Ko\-ba omaa\-wo atwi\-buu\-re eme\-re.}
\fmtExampleEN{Tell your mother to serve us food.}

\fmtLemma{ibuu\-ra}
\fmtCitation{(e)i\-buu\-ra}
\fmtPOS{n}%
\fmtNC{5/6}
\fmtSense{deserted house}
\fmtExampleRL{Abaa\-na ba ku nguu\-do bai\-ngi ba\-su\-la mu ibuu\-ra edi.}
\fmtExampleEN{Many street children sleep in that deserted house.}

\fmtLemma{ibu\-ya}
\fmtCitation{(e)i\-bu\-ya}
\fmtPOS{n}%
\fmtNC{5/6}
\fmtSense{occupation, profession, activity}
\fmtExampleRL{Oku\-lii\-sya ni\-kyo ibu\-ya lya Ba\-nya\-la eri\-ku\-lu.}
\fmtExampleEN{Cattle farming is the main activity among the Banyala.}

\fmtLemma{-ibw}
\fmtPOS{sf}
\fmtSense{passive suffix (extension) on verbs}

\fmtLemma{ibwa}
\fmtCitation{(e)i\-bwa}
\fmtPOS{n}%
\fmtNC{5/6}
\fmtSense{wound}
\fmtExampleRL{Ei\-bwa lya mwo\-jo oyo ka li\-rwi\-re oku\-wo\-na!}
\fmtExampleEN{The boy's wound has taken a long time to heal!}

\fmtLemma{ibyai\-re}
\fmtCitation{(e)i\-byai\-re}
\fmtPOS{n}%
\fmtNC{5}
\fmtSense{children, offspring, progeny}
\fmtExampleRL{Ata n’ei\-bya\-re ly’a\-baa\-na mu\-kaa\-ga.}
\fmtExampleEN{She has six offspring.}

\fmtLemma{ibya\-ro}
\fmtCitation{(e)i\-bya\-ro}
\fmtPOS{n}%
\fmtNC{5/6}
\fmtSense{traditional maternity centre, maternity ward}
\fmtExampleRL{Omu\-ka\-li ba\-mu\-twai\-re mu ibya\-ro ku\-bya\-la.}
\fmtExampleEN{The woman has been taken to the maternity centre to give birth.}

\fmtLemma{ica}%
\fmtOne{1}
\fmtCitation{(okw)i\-ca}
\fmtPOS{v}%
\fmtTrans{intr}
\fmtPFV{ici\-rye}
\fmtSense{to breathe}
\fmtExampleRL{Omwa\-na wa\-nge akwi\-ca ku\-bbi.}
\fmtExampleEN{My child is not breathing properly.}

\fmtLemma{ica}%
\fmtOne{2}
\fmtCitation{(okw)i\-ca}
\fmtPOS{v}%
\fmtTrans{tr}
\fmtPFV{ici\-rye}
\fmtSense{to accomplish, to complete, to finish, to finalise}
\fmtExampleRL{Omu\-ko\-li yai\-ci\-rye olu\-bbi\-mbi lwa\-mwe.}
\fmtExampleEN{The worker completed his assignment.}

\fmtLemma{icu\-mbi\-ro}
\fmtCitation{(e)i\-cu\-mbi\-ro}
\fmtPOS{n}%
\fmtNC{5/6}
\fmtSense{kitchen}
\fmtExampleRL{Eme\-re eri mu icu\-mbi\-ro.}
\fmtExampleEN{The food is in the kitchen.}

\fmtLemma{icu\-mu}
\fmtCitation{(e)i\-cu\-mu}
\fmtPOS{n}%
\fmtNC{5}
\fmtSense{an allergy that causes severe sneezing}
\fmtExampleRL{OGa\-va mu\-lwai\-re wa icu\-mu.}
\fmtExampleEN{Gava has an allergy which makes him sneeze.}

\fmtLemma{icwi}
\fmtPOS{interj}
\fmtSense{oh my goodness (used to express surprise, displeasure, shock, pity)}
\fmtExampleRL{Icwi! Awa\-ngwi\-re atyai aka\-lu\-lu?}
\fmtExampleEN{My goodness, how did he win the election?}

\fmtLemma{idaa\-la}
\fmtCitation{(e)i\-daa\-la}
\fmtPOS{n}%
\fmtNC{5/6}
\fmtSubentry(a)
\fmtSense{step, stage, phase}
\fmtExampleRL{Oku\-ku\-la ku\-bbaa\-mu ama\-daa\-la asa\-tu.}
\fmtExampleEN{Growth goes through three phases.}
\fmtSubentry(b)
\fmtSense{ladder}
\fmtExampleRL{Ndee\-te\-ra ei\-daa\-la nku\-ta\-ka ku\-nii\-na mu\-saa\-le gu\-ni.}
\fmtExampleEN{Bring the ladder, I would like to climb this tree.}
\fmtSubentry(c)
\fmtSense{level, rank}
\fmtExampleRL{Musobo\-la oku\-ku\-laa\-ku\-la\-na oku\-zwa oku idaa\-la li\-mwe okwi\-ra oku li\-ndi.}
\fmtExampleEN{You can progress from one rank to another.}

\fmtLemma{idaa\-li}
\fmtCitation{(e)i\-daa\-li}
\fmtPOS{n}%
\fmtNC{5/6}
\fmtSense{asparagus fern}
\fmtExampleRL{Ama\-daa\-li ga\-bbaa\-ku ama\-wa.}
\fmtExampleEN{The asparagus fern has thorns.}

\fmtLemma{ide\-mbe}
\fmtCitation{(e)i\-de\-mbe}
\fmtPOS{n}%
\fmtNC{5}
\fmtSense{freedom, rights, peace}
\fmtExampleRL{Oli\-na ei\-de\-mbe oku\-si\-moo\-la nka n’o\-ku\-ta\-ka.}
\fmtExampleEN{You have the freedom to speak the way you want.}

\fmtLemma{idi\-nda}
\fmtCitation{(e)i\-di\-nda}
\fmtPOS{n}%
\fmtNC{5/6}
\fmtSense{cuff (the end part of a sleeve)}
\fmtExampleRL{Engo\-ye zi\-bbaa\-ku ama\-di\-nda.}
\fmtExampleEN{Clothes have cuffs.}

\fmtLemma{iduu\-ka}
\fmtCitation{(e)i\-duu\-ka}
\fmtPOS{n}%
\fmtNC{5/6}
\fmtSource{from Sw.}
\fmtSense{shop, store}
\fmtExampleRL{Ya\-ba oku iduu\-ka ogu\-le su\-kaa\-li.}
\fmtExampleEN{Go to the shop and buy sugar.}

\fmtLemma{ifu\-lo}
\fmtCrossreference{\textnormal{see} ifuro}
\fmtDialect{Lunyala}

\fmtLemma{ifu\-mbe}
\fmtCitation{(o)i\-fu\-mbe}
\fmtPOS{n}%
\fmtNC{1a/10}
\fmtSense{African civet (\textit{Civettictis civetta}}
\fmtExampleRL{Ei\-fu\-mbe zi\-rya ama\-paa\-paa\-li.}
\fmtExampleEN{African civets eat pawpaws.}

\fmtLemma{ifu\-mbi\-ro}
\fmtCrossreference{\textnormal{see} isumbiro}

\fmtLemma{ifu\-ro}
\fmtDialect{Ruruuli}
\fmtCitation{(e)i\-fu\-lo}
\fmtPOS{n}%
\fmtNC{5/6}
\fmtSense{foam, froth}
\fmtExampleRL{Amai\-zi ga\-koo\-re ei\-fu\-ro.}
\fmtExampleEN{The water foams.}

\fmtLemma{ifu\-ru\-ngu}
\fmtCitation{(o)i\-fu\-ru\-ngu}
\fmtPOS{n}%
\fmtNC{1a/10}
\fmtSense{turaco}
\fmtExampleRL{Ei\-fu\-ru\-ngu zi\-bba za bbu\-lu omwe\-kwa\-ti era zi\-rya mu\-no ama\-pee\-ra.}
\fmtExampleEN{The turaco is dark blue and eats a lot of guavas.}

\fmtLemma{iga}
\fmtCitation{(e)i\-ga}
\fmtPOS{n}%
\fmtNC{5/6}
\fmtSense{cooking stone}
\fmtExampleRL{Ekyo\-to ki\-bba n’a\-mai\-ga asa\-tu.}
\fmtExampleEN{A fireplace has three cooking stones.}

\fmtLemma{igaa\-la}%
\fmtCitation{(okw)i\-gaa\-la}
\fmtPOS{v}%
\fmtTrans{intr}/\fmtTrans{tr}
\fmtPFV{iga\-lii\-re}
\fmtSubentry(a)
\fmtSense{to block (a passage)}
\fmtExampleRL{Epai\-pu na\-gi\-gaa\-lii\-re amai\-zi ga\-le\-ke ku\-se\-nse\-ka.}
\fmtExampleEN{I blocked the pipe so that the water is not wasted.}
\fmtSubentry(b)
\fmtSense{to patch up, to mend, to repair, to fix}
\fmtExampleRL{Igaa\-la eki\-tu\-li eki\-ri omu yu\-ni\-foo\-mu ya mwa\-na.}
\fmtExampleEN{Patch up the hole in the child's uniform.}
\fmtSubentry(c)
\fmtSense{to dress a wound}
\fmtExampleRL{Omu\-sa\-wu ai\-gaa\-la ebyo\-ya bu\-li ki\-ro.}
\fmtExampleEN{The doctor dresses the wound every day.}
\fmtSubentry(d)
\fmtSense{(of a wound) to heal}
\fmtExampleRL{Ekyo\-ya ki\-gaa\-lii\-re ku\-sai.}
\fmtExampleEN{The wound has healed properly.}

\fmtLemma{iga\-la}
\fmtCrossreference{\textnormal{see} igara}
\fmtDialect{Lunyala}

\fmtLemma{iga\-le}
\fmtCitation{\textnormal{(e.g.}\,kii\-ga\-le, bii\-ga\-le\textnormal{)}}
\fmtPOS{adj}
\fmtSense{covered, closed}
\fmtExampleRL{Ekii\-na kii\-ga\-le.}
\fmtExampleEN{The pit is covered.}

\fmtLemma{iga\-na}
\fmtCitation{(e)i\-ga\-na}
\fmtPOS{n}%
\fmtNC{5/6}
\fmtSense{herd}
\fmtExampleRL{OKi\-rya ata n’ei\-ga\-na lya nte.}
\fmtExampleEN{Kirya has a herd of cattle.}

\fmtLemma{iga\-ni}
\fmtCitation{(e)i\-ga\-ni}
\fmtPOS{n}%
\fmtNC{5}
\fmtSense{dedication, effort}
\fmtExampleRL{Omu\-ko\-li wa\-nge ako\-la emi\-ru\-mu n’ei\-ga\-ni.}
\fmtExampleEN{My worker does the job with dedication.}

\fmtLemma{iga\-nzo}
\fmtCitation{(e)i\-ga\-nzo}
\fmtPOS{n}%
\fmtNC{5/6}
\fmtSubentry(a)
\fmtSense{fishing ground}
\fmtExampleRL{Baa\-bi\-re ku iga\-nzo ku\-te\-ga ncwi.}
\fmtExampleEN{They have gone to the fishing ground to fish.}
\fmtSubentry(b)
\fmtSense{sudd (an area of floating vegetation in a stretch of the White Nile, thick enough to impede navigation)}
\fmtExampleRL{Aba\-te\-gi aba\-mwei ba\-bba ku iga\-nzo.}
\fmtExampleEN{Some fishermen live on the sudd.}

\fmtLemma{iga\-ra}
\fmtDialect{Ruruuli}
\fmtCitation{(okw)i\-ga\-ra}
\fmtPOS{v}%
\fmtTrans{tr}
\fmtPFV{igai\-re}
\fmtSense{to close, to shut}
\fmtExampleRL{Iga\-ra olwi\-ge lwa\-nge.}
\fmtExampleEN{Close my door.}

\fmtLemma{iga\-ra\-wo}
\fmtCitation{(okw)i\-ga\-ra\-wo}
\fmtPOS{v}%
\fmtTrans{tr}
\fmtPFV{igai\-re\-wo}
\fmtSense{to close, to bring to an end, to end, to conclude, to finish}
\fmtExampleRL{Iga\-ra\-wo olu\-kii\-ko obwi\-re bwa\-bi\-re.}
\fmtExampleEN{Close the meeting, it is late.}

\fmtLemma{igee\-ra}
\fmtCitation{(okw)i\-gee\-ra}
\fmtPOS{v}%
\fmtTrans{intr}
\fmtPFV{ige\-rei\-re}
\fmtSense{to come closer, to approach, to draw closer}
\fmtExampleRL{Ko\-ba aba\-jwa\-ka abo bai\-gee\-re ani.}
\fmtExampleEN{Ask those guards to come closer.}

\fmtLemma{ige\-go}
\fmtCitation{(e)i\-ge\-go}
\fmtPOS{n}%
\fmtNC{5/6}
\fmtSense{molar tooth}
\fmtExampleRL{Ama\-ge\-go ni\-go ge gee\-ra\-yo omu mu\-nwa.}
\fmtExampleEN{The molar teeth are the last ones in the mouth.}

\fmtLemma{ige\-mo}
\fmtCitation{(e)i\-ge\-mo}
\fmtPOS{n}%
\fmtNC{5/6}
\fmtSense{cuff (\eg of trousers)}

\fmtLemma{ige\-mu}
\fmtCrossreference{\textnormal{see} igemo}

\fmtLemma{ige\-ree\-ra}
\fmtCitation{(okw)i\-ge\-ree\-ra}
\fmtPOS{v}%
\fmtTrans{tr}
\fmtPFV{ige\-re\-rei\-re}
\fmtSense{to approach, to come toward, to come nearer}
\fmtExampleRL{Omu\-sai\-za oMu\-ruu\-li tai\-ge\-ree\-ra daa\-ni waa\-mwe.}
\fmtExampleEN{A Muruuli man doesn't come close to his mother-in-law.}

\fmtLemma{igi\-gi}
\fmtCitation{(e)i\-gi\-gi}
\fmtPOS{n}%
\fmtNC{5/6}
\fmtSense{curtain}
\fmtExampleRL{Omu\-lya\-ngo gwa\-mwe agu\-tai\-re\-mu ei\-gi\-gi.}
\fmtExampleEN{He has put a curtain in his doorway.}

\fmtLemma{igi\-ka}
\fmtCitation{(okw)i\-gi\-ka}
\fmtPOS{v}%
\fmtTrans{tr}
\fmtPFV{igi\-ki\-re}
\fmtSubentry(a)
\fmtSense{to support, to prop up}
\fmtExampleRL{Ba\-kwe\-ndya oku\-yi\-gi\-ka enko\-ndo oku nyu\-mba eni.}
\fmtExampleEN{They need to support the house with a pole.}
\fmtSubentry(b)
\fmtFixedexpression{igi\-ka (a)mai\-ga}
\fmtSense{to put in place cooking stones, to set up cooking stones}
\fmtExampleRL{Bai\-gi\-ki\-re amai\-ga omu jo\-koo\-ni.}
\fmtExampleEN{The have set up the cooking stones in the kitchen.}

\fmtLemma{igi\-ki\-rya}
\fmtCitation{(okw)i\-gi\-ki\-rya}
\fmtPOS{v}%
\fmtTrans{tr}
\fmtPFV{igi\-kii\-rye}
\fmtSense{to support (e.g.\,cooking stones) with a smaller stone}
\fmtExampleRL{Yi\-gi\-ki\-rya eki\-na\-ga omu\-ku\-bi gu\-ta\-su\-ku\-ka.}
\fmtExampleEN{Support the pan with a supporting stone so that the sauce does not spill.}

\fmtLemma{igi\-ki\-ryo}
\fmtCitation{(e)i\-gi\-ki\-ryo}
\fmtPOS{n}%
\fmtNC{5/6}
\fmtSense{supporting stone}
\fmtExampleRL{Epi\-pa gi\-yi\-gi\-ke ei\-gi\-ki\-ryo eta\-gwa.}
\fmtExampleEN{Put a supporting stone there so that the barrel does not fall down.}

\fmtLemma{igo\-be}
\fmtCitation{(e)i\-go\-be}
\fmtPOS{n}%
\fmtNC{5}
\fmtSense{(of dogs) mating season}
\fmtExampleRL{Embwe\-ne zi\-ri ku igo\-be.}
\fmtExampleEN{The dogs are in their mating season.}

\fmtLemma{igo\-lo}
\fmtCitation{(e)i\-go\-lo}
\fmtPOS{adv}
\fmtSubentry(a)
\fmtSense{in the afternoon}
\fmtExampleRL{Ewaa\-mu nai\-za\-yo igo\-lo.}
\fmtExampleEN{I will come to your place in the afternoon.}
\fmtSubentry(b)
\fmtSense{last evening}
\fmtExampleRL{Ya\-boi\-ne omwe\-be\-mbe\-zi wa iya\-nga ei\-go\-lo.}
\fmtExampleEN{He met the president last evening.}
\fmtSubentry(c)
\fmtSense{yesterday}
\fmtExampleRL{Ei\-go\-lo ya\-bbai\-re\-ku ani.}
\fmtExampleEN{He was here yesterday.}

\fmtLemma{igo\-loi\-go\-lo}
\fmtCitation{(e)i\-go\-loi\-go\-lo}
\fmtPOS{adv}
\fmtSense{early evening, late afternoon}
\fmtExampleRL{Ya\-yi\-ki\-re ani ei\-go\-loi\-go\-lo.}
\fmtExampleEN{He arrived here in the early evening.}

\fmtLemma{igo\-lo\-li\-mwei}
\fmtCitation{(e)i\-go\-lo\-li\-mwei}
\fmtPOS{adv}
\fmtSense{early evening}
\fmtExampleRL{Naa\-bi\-re igo\-lo\-li\-mwei.}
\fmtExampleEN{I left in the early evening.}

\fmtLemma{igo\-mbe}
\fmtCitation{(o)i\-go\-mbe}
\fmtPOS{n}%
\fmtNC{1a}
\fmtSubentry(a)
\fmtSense{western wind}
\fmtExampleRL{Oi\-go\-mbe ala\-nga ike\-ndi.}
\fmtExampleEN{The western wind announces rain.}
\fmtSubentry(b)
\fmtSense{wind that predicts rain}
\fmtExampleRL{Oi\-go\-mbe ala\-nga nti si\-zo\-ni ya\-ku\-si\-ga eri ampi okwi\-ka.}
\fmtExampleEN{The wind that predicts rain is an indicator that the sawing season is about to start.}

\fmtLemma{igo\-mbo\-lo\-la}
\fmtCitation{(e)i\-go\-mbo\-lo\-la}
\fmtPOS{n}%
\fmtNC{5/6}
\fmtSense{sub-county}
\fmtExampleRL{Aya\-bi\-re oku igo\-mbo\-lo\-la.}
\fmtExampleEN{He has gone to the sub-county.}

\fmtLemma{igo\-na}
\fmtCitation{(e)i\-go\-na}
\fmtPOS{n}%
\fmtNC{5/6}
\fmtSense{type of vegetation which grows on a lake shore}

\fmtLemma{igo\-se}
\fmtCitation{(e)i\-go\-se}
\fmtPOS{n}%
\fmtNC{5/6}
\fmtSense{testicle of an animal}
\fmtExampleRL{Ama\-go\-se ga ntaa\-ma ga\-so\-bo\-la oku\-lii\-bwa.}
\fmtExampleEN{The testicles of a sheep can be eaten.}

\fmtLemma{igu\-gu\-nyu}
\fmtCitation{(e)i\-gu\-gu\-nyu}
\fmtPOS{n}%
\fmtNC{9/10}
\fmtSense{hip}
\fmtExampleRL{Ei\-gu\-gu\-nyu li\-ku\-ndu\-ma bu\-ndi.}
\fmtExampleEN{My hip hurts.}

\fmtLemma{igu\-ka}
\fmtCitation{(oku)\-i\-gu\-ka}
\fmtPOS{v}%
\fmtTrans{intr}
\fmtPFV{igu\-ki\-re}
\fmtSubentry(a)
\fmtSense{to arrive}
\fmtExampleRL{Omu\-ge\-ni omu\-ku\-lu ai\-gu\-ki\-re.}
\fmtExampleEN{The guest of honour has arrived.}
\fmtSubentry(b)
\fmtSense{to open, to become open}
\fmtExampleRL{Ekya\-nko\-mee\-ro olwi\-ge lwa\-nge lu\-yi\-gu\-ki\-re.}
\fmtExampleEN{Finally, my door is open again.}
\fmtSubentry(c)
\fmtSense{(of a tree) to fall}
\fmtExampleRL{Omu\-saa\-le gu\-yi\-gu\-ki\-re.}
\fmtExampleEN{The tree has fallen down.}

\fmtLemma{igu\-ki\-ra}
\fmtCitation{(okw)i\-gu\-ki\-ra}
\fmtPOS{v}%
\fmtTrans{intr}
\fmtPFV{igu\-kii\-re}
\fmtSense{to come out}
\fmtExampleRL{N’o\-ya\-kaa\-ra e No\-ngo oi\-gu\-ki\-ra mu Kye\-ri\-ma.}
\fmtExampleEN{When you go via Nongo you come out of Kyerima.}

\fmtLemma{igu\-la}
\fmtCitation{(oku)\-i\-gu\-la}
\fmtPOS{v}%
\fmtTrans{intr}
\fmtPFV{igwi\-re}
\fmtSense{to open}
\fmtExampleRL{Ya\-yi\-gwi\-re\-wo olwi\-ge.}
\fmtExampleEN{She opened the door.}

\fmtLemma{igu\-mba}
\fmtCitation{(e)i\-gu\-mba}
\fmtPOS{n}%
\fmtNC{5/6}
\fmtSense{bone}
\fmtExampleRL{Omu\-sai\-za aye\-nde\-ke\-re ei\-gu\-mba.}
\fmtExampleEN{The man's bone is broken.}

\fmtLemma{igu\-ru}
\fmtCitation{(e)i\-gu\-ru}
\fmtPOS{n}%
\fmtNC{5}
\fmtSubentry(a)
\fmtSense{sky}
\fmtExampleRL{Oku igu\-ru ndoo\-ku bi\-cu.}
\fmtExampleEN{There are no clouds in the sky.}
\fmtSubentry(b)
\fmtSense{top, place above}
\fmtSubentry(c)
\fmtSense{heaven, paradise}
\fmtExampleRL{OYe\-su atu\-be\-nde\-rei\-re omu igu\-ru.}
\fmtExampleEN{Jesus is waiting for us in heaven.}
\fmtSubentry(d)
\fmtFixedexpression{igu\-ru we lye\-su\-ru\-kii\-re}
\fmtSense{horizon}
\fmtExampleRL{Ei\-gu\-ru we lye\-su\-ru\-kii\-re ti ni\-gyo nka\-ngo ya nsi.}
\fmtExampleEN{The horizon is not the end of the earth.}

\fmtLemma{igwa\-ra}
\fmtCitation{(e)i\-gwa\-ra}
\fmtPOS{n}%
\fmtNC{5/6}
\fmtSense{vuvuzela}
\fmtExampleRL{Aba\-pii\-na ba\-fuu\-wa ama\-gwa\-ra.}
\fmtExampleEN{The Bapiina clan plays the vuvuzela.}

\fmtLemma{ijaa\-bbu}
\fmtCitation{(e)i\-jaa\-bbu}
\fmtPOS{n}%
\fmtNC{9/10}
\fmtSource{from Ar. }
\fmtSense{hijab (a head covering worn in public by some Muslim women)}
\fmtExampleRL{Abbwe\-re ijaa\-bbu.}
\fmtExampleEN{She is wearing a hijab.}

\fmtLemma{ijwi}
\fmtPOS{interj}
\fmtSense{my goodness (used to express shock, surprise)}
\fmtExampleRL{Ijwi\-… Afii\-re ki\-? ata\-bbai\-re mu\-lwai\-re!}
\fmtExampleEN{My goodness, what has he died of? He was not sick.}

\fmtLemma{ika}
\fmtCitation{(okw)i\-ka}
\fmtPOS{v}%
\fmtTrans{intr}
\fmtPFV{iki\-re}
\fmtSubentry(a)
\fmtSense{to arrive at, to come to, to reach}
\fmtExampleRL{Ndi ampi okwi\-ka e wa\-nge.}
\fmtExampleEN{I am about to arrive home.}
\fmtSubentry(b)
\fmtSense{to extend, to stretch}
\fmtExampleRL{Eki\-ba\-nja kya\-nge kii\-ka edi.}
\fmtExampleEN{My land extends up to there.}
\fmtSubentry(c)
\fmtSense{(of taste) to be appropriate, to be of the right measure, to be enough}
\fmtExampleRL{Omuu\-nyu gwi\-ki\-re omu mu\-ku\-bi.}
\fmtExampleEN{There is enough salt in the sauce.}

\fmtLemma{ikaa\-ka}
\fmtCitation{(e)i\-kaa\-ka}
\fmtPOS{n}%
\fmtNC{5}
\fmtSense{obstinacy, inflexibility, stubbornness}
\fmtExampleRL{Omu\-sai\-za oyo ya\-gai\-ne oku\-sa\-su\-la aba\-ko\-li baa\-mwe wa ikaa\-ka.}
\fmtExampleEN{That man refused to pay his workers, he is so obstinate.}

\fmtLemma{ikaa\-nya}
\fmtCitation{(okw)i\-kaa\-nya}
\fmtPOS{v}%
\fmtTrans{intr}
\fmtPFV{ikai\-nye}
\fmtSense{to reach an agreement, to reach a consensus}
\fmtExampleRL{Mwi\-kaa\-nye mu\-zwe mu kwa\-raa\-ta\-na.}
\fmtExampleEN{Settle your differences and stop quarrelling.}

\fmtLemma{ikaa\-ra}
\fmtCitation{(okw)i\-kaa\-ra}
\fmtPOS{v}%
\fmtTrans{intr}/\fmtTrans{tr}
\fmtPFV{ikai\-re}
\fmtSubentry(a)
\fmtSense{to oppress, to abuse, to maltreat, to ill-treat, to tyrannise}
\fmtExampleRL{Lwa\-ki mu\-ka\-li we oi\-kaa\-ra\-nga abaa\-na ba\-mwi\-ra waa\-mu?}
\fmtExampleEN{Woman, why do you treat your friend's children badly?}
\fmtSubentry(b)
\fmtSense{to sit, to sit down}
\fmtExampleRL{Ikaa\-ra oku ki\-ko\-nko\-ro ekyo ndoo\-wo nte\-be gi\-ndi.}
\fmtExampleEN{Sit on that wooden stool, there are no more seats.}
\fmtSubentry(c)
\fmtSense{to settle, to calm down, to quiet down}
\fmtExampleRL{Ikaa\-ra ansi oso\-me ebi\-ta\-bo bya\-mu.}
\fmtExampleEN{Settle down and read your books.}
\fmtSubentry(d)
\fmtFixedexpression{ikaa\-ra ku lu\-bu\-ga}
\fmtSense{(of females) to live in one's parents' home without being married}
\fmtExampleRL{Omwa\-la akyai\-kai\-re ku lu\-bu\-ga, akya\-li ku\-fu\-na mu\-sai\-za aku\-mu\-twa\-la.}
\fmtExampleEN{The girl is still living at her parents' home, she has not yet got a man to marry her.}

\fmtLemma{ikaa\-ryo}
\fmtCitation{(e)i\-kaa\-ryo}
\fmtPOS{n}%
\fmtNC{5/6}
\fmtSense{drying lane}
\fmtExampleRL{Ebi\-tee\-re mu\-bi\-kaa\-rye mu ikaa\-ryo.}
\fmtExampleEN{Dry the potatoes on the drying lane.}

\fmtLemma{ika\-byo}
\fmtCitation{(e)i\-ka\-byo}
\fmtPOS{n}%
\fmtNC{5}
\fmtSense{rudeness, lack of manners, impudence, insolence}
\fmtExampleRL{Ei\-ka\-byo ti\-ri\-kwi\-ki\-ri\-zi\-sya kwe\-nde\-bwa ba\-ntu.}
\fmtExampleEN{Rudeness does not make you popular with people.}

\fmtLemma{ikai\-ka\-na}
\fmtCitation{(okw)i\-kai\-ka\-na}
\fmtPOS{v}%
\fmtTrans{intr}
\fmtPFV{ikai\-kai\-ne}
\fmtSubentry(a)
\fmtSense{to stabilise, to normalise, to calm}
\fmtExampleRL{Enya\-nja ei\-kai\-kai\-ne.}
\fmtExampleEN{The sea has calmed.}
\fmtSubentry(b)
\fmtSense{to calm down, to become tranquil}
\fmtExampleRL{Ba\-nza oi\-kai\-ka\-ne nga okya\-li ku\-ba\-za.}
\fmtExampleEN{Calm down before you speak.}

\fmtLemma{ika\-la}%
\fmtOne{1}
\fmtCitation{(okw)i\-ka\-la}
\fmtPOS{v}%
\fmtTrans{intr}/\fmtTrans{tr}
\fmtPFV{ikai\-re}
\fmtSubentry(a)
\fmtSense{to sit (down), to take a seat}
\fmtExampleRL{Ika\-la ansi olye eme\-re.}
\fmtExampleEN{Sit down and eat.}
\fmtSubentry(b)
\fmtSense{to live, to reside, to stay}
\fmtExampleRL{Nyi\-ka\-la Ka\-nju\-ki ku kya\-lo Bu\-yo\-be.}
\fmtExampleEN{I live in Kanjuki-Buyombe village.}
\fmtSubentry(c)
\fmtSense{(of \eg committee) to convene, to meet}
\fmtExampleRL{Olu\-ku\-mbaa\-no lwa Bu\-ruu\-li lwi\-kai\-re.}
\fmtExampleEN{The council of the Baruuli has convened.}
\fmtSubentry(d)
\fmtSense{to write (an exam), to take, to sit}
\fmtExampleRL{Nkwi\-za kwi\-ka\-la ebi\-buu\-lyo bya\-nge ema\-mbya.}
\fmtExampleEN{I will sit my exams tomorrow.}

\fmtLemma{ika\-la}%
\fmtOne{2}
\fmtCitation{(e)i\-ka\-la}
\fmtPOS{n}%
\fmtNC{5/6}
\fmtSubentry(a)
\fmtSense{charcoal}
\fmtExampleRL{Ama\-ka\-la ga\-bu\-sye\-re ebi\-ro bi\-ni mu\-ga\-kee\-ke\-ree\-rye.}
\fmtExampleEN{Charcoal is very expensive these days, use it sparingly.}
\fmtSubentry(b)
\fmtSense{battery}
\fmtExampleRL{Ti\-ndi\-na ma\-ka\-la ga la\-di\-yo ya\-nge.}
\fmtExampleEN{I don’t have batteries for my radio.}

\fmtLemma{ika\-lii\-ra}
\fmtCitation{(okw)i\-ka\-lii\-ra}
\fmtPOS{v}%
\fmtTrans{intr}
\fmtPFV{ika\-lii\-re}
\fmtSense{to sit and wait for a long time}
\fmtExampleRL{Nkwi\-za ku\-mwi\-ka\-lii\-ra okwi\-ca ni na\-mu\-bo\-na.}
\fmtExampleEN{I will sit and wait until I see him.}

\fmtLemma{ika\-li\-sya}
\fmtCitation{(okw)i\-ka\-li\-sya}
\fmtPOS{v}%
\fmtTrans{tr}
\fmtPFV{ika\-li\-si\-rye}
\fmtSense{to enthrone, to install}
\fmtExampleRL{Aba\-nya\-la bai\-ka\-li\-si\-rye Omu\-ka\-ma waa\-bwe.}
\fmtExampleEN{The Banyala have enthroned their king.}

\fmtLemma{ika\-nsu}
\fmtCitation{(e)i\-ka\-nsu}
\fmtPOS{n}%
\fmtNC{5}
\fmtSense{hot temper, tendency to become angry very quickly}
\fmtExampleRL{Mwa\-na we ei\-ka\-nsu lya\-mu li\-ri\-kwi\-kya a\-lai.}
\fmtExampleEN{My child, your hot temper will destroy you.}

\fmtLemma{ika\-ta}%
\fmtOne{1}
\fmtCitation{(okw)i\-ka\-ta}
\fmtPOS{v}%
\fmtTrans{tr}
\fmtPFV{ika\-ti\-re}
\fmtSubentry(a)
\fmtSense{to press (a boil, a pustule)}
\fmtExampleRL{Ika\-ta ei\-yu\-te otoo\-le\-mu amaa\-ra.}
\fmtExampleEN{Press the boil to extract the pus.}
\fmtSubentry(b)
\fmtSense{to massage}
\fmtExampleRL{Ika\-ta omu\-go\-ngo gwa\-mwe a\-bbee\-ku ka\-le\-ka\-le.}
\fmtExampleEN{Massage his back so that he gets some relief.}
\fmtSubentry(c)
\fmtSense{to massage the belly of a mother of a new born baby to tone the muscles}

\fmtLemma{ika\-ta}%
\fmtOne{2}
\fmtCitation{(okw)i\-ka\-ta}
\fmtPOS{v}%
\fmtTrans{tr}
\fmtPFV{ika\-ti\-re}
\fmtSense{to throw down with a heavy plop}
\fmtExampleRL{Ai\-ka\-ti\-re ensa\-wu ansi.}
\fmtExampleEN{He has thrown down the bag with a heavy plop.}

\fmtLemma{ika\-ti\-ra}
\fmtCitation{(okw)i\-ka\-ti\-ra}
\fmtPOS{v}%
\fmtTrans{tr}
\fmtPFV{ika\-tii\-re}
\fmtSense{to pack, to package}
\fmtExampleRL{Ika\-ti\-ra ensa\-wu ya pa\-mba ku\-sai.}
\fmtExampleEN{Pack the cotton into the sack.}

\fmtLemma{ika\-yu}
\fmtCitation{(e)i\-ka\-yu}
\fmtPOS{n}%
\fmtNC{5}
\fmtSense{rudeness, lack of manners, impudence, insolence}
\fmtExampleRL{Ei\-ka\-yu li\-ku\-no\-be\-sya aba\-ntu.}
\fmtExampleEN{Rudeness makes people hate you.}

\fmtLemma{ike\-ka}
\fmtCitation{(e)i\-ke\-ka}
\fmtPOS{n}%
\fmtNC{5}
\fmtSense{season of slicing sweet potatoes}
\fmtExampleRL{Twi\-ge\-rei\-re ebi\-ro bya ike\-ka.}
\fmtExampleEN{We are approaching the sweet potatoe slicing season.}

\fmtLemma{ike\-ndi}
\fmtCitation{(o)i\-ke\-ndi}
\fmtPOS{n}%
\fmtNC{1a}
\fmtSense{rain}
\fmtExampleRL{Oi\-ke\-ndi asi\-nga ku\-gwa mu kwe\-zi Kwai\-se\-nsa\-ncwi.}
\fmtExampleEN{Usually there is heavy rain in May.}

\fmtLemma{ike\-re\-nda}
\fmtCitation{(e)i\-ke\-re\-nda}
\fmtPOS{n}%
\fmtNC{5/6}
\fmtSense{tablet (of medicine), pill}
\fmtExampleRL{Mmii\-re ei\-ke\-re\-nda li\-mwei lyo\-nkai.}
\fmtExampleEN{I have taken only one tablet.}

\fmtLemma{ikii\-ra}%
\fmtOne{1}
\fmtCitation{(okw)i\-kii\-ra}
\fmtPOS{v}%
\fmtTrans{tr}
\fmtPFV{iki\-rii\-re}
\fmtSense{to approach, to go near to, to come close to}
\fmtExampleRL{Ya\-ba oi\-kii\-re omu\-ku\-lu omu\-ma\-nyi\-sye ebi\-zi\-bu bya\-mu.}
\fmtExampleEN{Go and approach the head and tell him about your problems.}

\fmtLemma{ikii\-ra}%
\fmtOne{2}
\fmtCitation{(okw)i\-kii\-ra}
\fmtPOS{v}%
\fmtTrans{intr}
\fmtPFV{iki\-rii\-re}
\fmtSense{to be completed, to be achieved, to come true}
\fmtExampleRL{Obu\-la\-ngi bwa Yo\-kaa\-na bwai\-ki\-rii\-re.}
\fmtExampleEN{John's prophecy was fulfilled.}

\fmtLemma{ikii\-ra}%
\fmtOne{3}
\fmtCitation{(okw)i\-kii\-ra}
\fmtPOS{v}%
\fmtTrans{intr}
\fmtPFV{iki\-rii\-re}
\fmtSense{to be righteous, to be true}

\fmtLemma{ikii\-ri}
\fmtCitation{\textnormal{(e.g.}\,gii\-kii\-ri, bii\-kii\-ri, mwi\-kii\-ri\textnormal{)}}
\fmtPOS{adj}
\fmtSubentry(a)
\fmtSense{(of food) clean, pure, undefiled}
\fmtExampleRL{Eme\-re tu\-maa\-re oku\-gi\-sa\-bi\-ra ati\-bu\-ni gi\-kii\-ri.}
\fmtExampleEN{We have prayed for the food, now it is clean.}
\fmtSubentry(b)
\fmtSense{righteous}
\fmtExampleRL{Ki\-ta\-ka n’o\-bba mwi\-kii\-ri omu mai\-so ga Ka\-nca.}
\fmtExampleEN{It is better to be righteous before God.}

\fmtLemma{ikii\-rya}%
\fmtOne{1}
\fmtCitation{(okw)i\-kii\-rya}
\fmtPOS{v}%
\fmtTrans{tr}
\fmtPFV{iki\-rii\-rye}
\fmtSubentry(a)
\fmtSense{to perfect, to polish, to hone, to refine}
\fmtExampleRL{Ya\-ki\-koo\-re emi\-ru\-ndi gi\-ngi okwi\-ca ni ya\-kii\-ki\-rii\-rye.}
\fmtExampleEN{He practiced it so much that he perfected it.}
\fmtSubentry(b)
\fmtSense{to fulfil}
\fmtExampleRL{Lwa\-ki osuu\-bi\-za\-nga n’o\-ta\-yi\-kii\-rya.}
\fmtExampleEN{Why do you make promises, which you never fulfil?}

\fmtLemma{ikii\-rya}%
\fmtOne{2}
\fmtCitation{(okw)i\-kii\-rya}
\fmtPOS{v}%
\fmtTrans{intr}
\fmtPFV{iki\-rii\-rye}
\fmtSense{(of crops) to reach maturity, to become ripe}
\fmtExampleRL{Ebi\-duu\-ma bi\-ta\-ndi\-ki\-re okwi\-kii\-rya.}
\fmtExampleEN{Maize is about to ripen.}

\fmtLemma{iki\-ngo}
\fmtCitation{(e)i\-ki\-ngo}
\fmtPOS{n}%
\fmtNC{5/6}
\fmtSense{air potato, air yam (\textit{Dio\-sco\-rea bulbi\-fe\-ra})}
\fmtExampleRL{Ama\-ki\-ngo ga\-woo\-ma.}
\fmtExampleEN{Air potatoes are tasty.}

\fmtLemma{iki\-ra}
\fmtCitation{(okw)i\-ki\-ra}
\fmtPOS{v}%
\fmtTrans{tr}
\fmtPFV{ikii\-re}
\fmtSense{to perform rituals for (a new garden)}
\fmtExampleRL{Baa\-bi\-re kwi\-ki\-ra mu\-si\-ri gwa bi\-yaa\-ta.}
\fmtExampleEN{They have gone to perform rituals for the new potato garden.}

\fmtLemma{iki\-ra\-na}
\fmtCitation{(okw)i\-ki\-ra\-na}
\fmtPOS{v}%
\fmtTrans{intr}
\fmtPFV{iki\-rai\-ne}
\fmtSense{to match perfectly, to be equal, (of a musical instrument) to be in tune}
\fmtExampleRL{Oyo omu\-ka\-li wa\-nge twai\-ki\-rai\-ne.}
\fmtExampleEN{My wife and I are a perfect match.}

\fmtLemma{iki\-rii\-ra}
\fmtCitation{(okw)i\-ki\-rii\-ra}
\fmtPOS{v}%
\fmtTrans{intr}
\fmtPFV{iki\-rii\-re}
\fmtSubentry(a)
\fmtSense{to be perfect, to be faultless}
\fmtExampleRL{To\-so\-bo\-la kwi\-ki\-rii\-ra mu bu\-li ki\-mwei.}
\fmtExampleEN{You cannot be perfect in everything.}
\fmtSubentry(b)
\fmtSense{to reach the destination on the same day}
\fmtExampleRL{Aya\-bi\-re lwa\-lee\-ro nai\-ki\-rii\-ra mi\-re\-mbe.}
\fmtExampleEN{He left and arrived on the same day safely.}

\fmtLemma{iki\-rii\-ri}
\fmtCitation{\textnormal{(e.g.}\,gii\-kii\-ri, bii\-kii\-ri, mwi\-kii\-ri\textnormal{)}}
\fmtPOS{adj}
\fmtSense{perfect, saint, holy}
\fmtExampleRL{OKa\-nca mwi\-kii\-ri\-ri. Eki\-sa kya\-mwe kya bwi\-ki\-rii\-ri.}
\fmtExampleEN{God is holy. His kindness is saintly.}

\fmtLemma{iki\-ri\-za}
\fmtCitation{(okw)i\-ki\-ri\-za}
\fmtPOS{v}%
\fmtTrans{tr}
\fmtPFV{iki\-rii\-ze}
\fmtSubentry(a)
\fmtSense{to allow, to let, to permit, to give permission}
\fmtExampleRL{Nyi\-ki\-ri\-za nswe\-re omwa\-ra waa\-mu.}
\fmtExampleEN{Allow me to marry your daughter.}
\fmtSubentry(b)
\fmtSense{to agree, to consent to, to say yes to}
\fmtExampleRL{Na\-mu\-ko\-be\-re ebi\-zi\-bu bya\-nge nai\-ki\-ri\-za oku\-mpa ese\-nte.}
\fmtExampleEN{I told him about my problem, and he agreed to give me money.}
\fmtSubentry(c)
\fmtSense{to accept, to trust}
\fmtExampleRL{We\-mbe iki\-ri\-za by'a\-ku\-ko\-ba le\-ke oya\-ka\-le\-mu.}
\fmtExampleEN{Just accept whatever he is saying, you will succeed.}
\fmtSubentry(d)
\fmtSense{to have faith, to practice (a religion)}
\fmtExampleRL{Nta n’o\-kwi\-ki\-ri\-za kwi\-ngi omu mwa\-na wa\-nge.}
\fmtExampleEN{I have a lot of faith in my child.}

\fmtLemma{iki\-ri\-za\-ga\-nya}
\fmtCitation{(okw)i\-ki\-ri\-za\-ga\-nya}
\fmtPOS{v}%
\fmtTrans{intr}
\fmtPFV{iki\-ri\-za\-gai\-nye}
\fmtSense{to reach a consensus}
\fmtExampleRL{Twi\-ki\-re oku nyi\-ki\-ri\-za\-ga\-nya.}
\fmtExampleEN{We have reached a consensus.}

\fmtLemma{iki\-rya}
\fmtCitation{(okw)i\-ki\-rya}
\fmtPOS{v}%
\fmtTrans{tr}
\fmtPFV{iki\-rii\-rye}
\fmtSense{to keep in a place temporarily}
\fmtExampleRL{Lee\-ta omwa\-na omwi\-ki\-rye ani.}
\fmtExampleEN{Bring the child and keep her here temporarily.}

\fmtLemma{iko\-da}
\fmtCitation{(e)i\-ko\-da}
\fmtPOS{n}%
\fmtNC{5}
\fmtSense{rudeness, impoliteness, naughtiness}
\fmtExampleRL{Abaa\-na ba bi\-ro bi\-ni ba\-ta n’ei\-ko\-da.}
\fmtExampleEN{Today's children are naughty.}

\fmtLemma{iko\-ke}
\fmtCitation{(e)i\-ko\-ke}
\fmtPOS{n}%
\fmtNC{5}
\fmtSense{ash}
\fmtExampleRL{Oba ekyo\-to ki\-woo\-re, too\-la\-mu ei\-ko\-ke.}
\fmtExampleEN{Remove the ash from the cooking place if it has cooled down.}

\fmtLemma{iko\-me\-ra}
\fmtCitation{(e)i\-ko\-me\-ra}
\fmtPOS{n}%
\fmtNC{5/6}
\fmtSense{prison}
\fmtExampleRL{Omu\-sai\-za ya\-fi\-rii\-re omu iko\-me\-ra.}
\fmtExampleEN{The man died in the prison.}

\fmtLemma{ikoo\-ko}
\fmtCitation{(e)i\-koo\-ko}
\fmtPOS{n}%
\fmtNC{5/6}
\fmtSense{cambium and sapwood parts of a tree}
\fmtExampleRL{Ebi\-ye\-nze bya\-bya\-lii\-re omu ikoo\-ko lya mu\-saa\-le.}
\fmtExampleEN{Cockroaches laid eggs on the bark of the tree.}

\fmtLemma{ikoo\-ro}
\fmtCitation{(e)i\-koo\-ro}
\fmtPOS{n}%
\fmtNC{5/6}
\fmtSense{factory, plant}
\fmtExampleRL{Ei\-koo\-ro lya bi\-kai\-zo li\-yii\-re.}
\fmtExampleEN{The sugar factory has burnt down.}

\fmtLemma{iko\-ti}
\fmtCitation{(e)i\-ko\-ti}
\fmtPOS{n}%
\fmtNC{5/6}
\fmtSense{neck}
\fmtExampleRL{Omu\-sai\-za ali\-na ei\-ko\-ti eri\-koo\-to nk’e\-rya mbe\-re\-ge.}
\fmtExampleEN{The man has a big neck like a pig's.}

\fmtLemma{iku\-bbi\-ro}
\fmtCitation{(e)i\-ku\-bbi\-ro}
\fmtPOS{n}%
\fmtNC{5}
\fmtSense{thorough beating, good thrashing}
\fmtExampleRL{Baa\-mu\-ku\-bbi\-re n’a\-si\-ga\-la omu iku\-bbi\-ro.}
\fmtExampleEN{He received a good thrashing.}

\fmtLemma{iku\-de}
\fmtCitation{(e)i\-ku\-de}
\fmtPOS{n}%
\fmtNC{5/6}
\fmtSense{hip}
\fmtExampleRL{Omu\-ka\-li ali\-na ama\-ku\-de aga\-koo\-to.}
\fmtExampleEN{The woman has big hips.}

\fmtLemma{iku\-ka}
\fmtCitation{(e)i\-ku\-ka}
\fmtPOS{n}%
\fmtNC{5/6}
\fmtSense{type of water potato, aquatic vegetation}
\fmtExampleRL{Ama\-ku\-ka gai\-ta encwi.}
\fmtExampleEN{Floating water plants kill fish.}

\fmtLemma{iku\-lu\-lu}
\fmtCitation{(o)i\-ku\-lu\-lu}
\fmtPOS{n}%
\fmtNC{1a/10}
\fmtSense{ostrich}
\fmtExampleRL{Iku\-lu\-lu ni\-kyo ki\-nyo\-nyi eki\-si\-nga obu\-koo\-to.}
\fmtExampleEN{The ostrich is the biggest bird.}

\fmtLemma{iku\-mbaa\-ni\-ro}
\fmtCitation{(e)i\-ku\-mbaa\-ni\-ro}
\fmtPOS{n}%
\fmtNC{5/6}
\fmtSubentry(a)
\fmtSense{social centre}
\fmtExampleRL{Oku bu\-li iru\-gu\-za ku\-bbaa\-ku ei\-ku\-mbaa\-ni\-ro.}
\fmtExampleEN{Every sub-county has a social centre.}
\fmtSubentry(b)
\fmtSense{temple, house of worship, synagogue}
\fmtExampleRL{Omu\-ka\-ma Su\-le\-maa\-ni ya\-yo\-mbo\-kei\-re oKa\-nca ei\-ku\-mbaa\-ni\-ro.}
\fmtExampleEN{King Solomon constructed a temple for God.}

\fmtLemma{iku\-mbe}
\fmtCitation{(e)i\-ku\-mbe}
\fmtPOS{n}%
\fmtNC{9/10}
\fmtSense{public, community}
\fmtExampleRL{Ki\-sai ebi\-fo bya iku\-mbe oku\-bba n’e\-byo\-loo\-ni.}
\fmtExampleEN{It is good for public places to have toilets.}

\fmtLemma{iku\-mi}
\fmtPOS{num}
\fmtSubentry(a)
\fmtSense{ten}
\fmtExampleRL{Omu\-cu\-ngwa gu\-li\-mu emi\-ru\-ndi iku\-mi.}
\fmtExampleEN{An orange comes in ten types.}
\fmtSubentry(b)
\fmtFixedexpression{mwe\-zi gwa iku\-mi}
\fmtSense{October}
\fmtSubentry(c)
\fmtFixedexpression{mwe\-zi gwa iku\-mi na gu\-mwei }
\fmtSense{November}
\fmtSubentry(d)
\fmtFixedexpression{mwe\-zi gwa iku\-mi n’eibi\-ri}
\fmtSense{December}
\fmtSubentry(e)
\fmtFixedexpression{iku\-mi na (-)mwei}
\fmtCitation{\textnormal{(e.g.}\,iku\-mi n'e\-mwei, iku\-mi n'o\-mwei\textnormal{)}}
\fmtSense{eleven}
\fmtSubentry(f)
\fmtFixedexpression{iku\-mi na ibi\-ri}
\fmtSense{twelve}
\fmtSubentry(g)
\fmtFixedexpression{iku\-mi na inai}
\fmtSense{fourteen}
\fmtSubentry(h)
\fmtFixedexpression{iku\-mi na isa\-tu}
\fmtSense{thirteen}
\fmtSubentry(i)
\fmtFixedexpression{iku\-mi na itaa\-nu}
\fmtSense{fifteen}
\fmtSubentry(j)
\fmtSense{sixteen}
\fmtFixedexpression{iku\-mi na mu\-ka\-nga}
\fmtSubentry(k)
\fmtFixedexpression{iku\-mi na mu\-na\-nai}
\fmtSense{eighteen}
\fmtSubentry(l)
\fmtFixedexpression{iku\-mi na mu\-sa\-nju}
\fmtSense{seventeen}
\fmtSubentry(m)
\fmtFixedexpression{iku\-mi na mwe\-ndai}
\fmtSense{nineteen}

\fmtLemma{iku\-ndi}
\fmtCrossreference{\textnormal{see also} mukundi}
\fmtCitation{(e)i\-ku\-ndi}
\fmtPOS{n}%
\fmtNC{5/6}
\fmtSense{navel, umbilicus}
\fmtExampleRL{Ei\-ku\-ndi lya mwa\-na li\-ku\-mu\-lu\-ma.}
\fmtExampleEN{The child's navel hurts.}

\fmtLemma{iku\-ra}
\fmtCitation{(e)i\-ku\-ra}
\fmtPOS{n}%
\fmtNC{5/6}
\fmtSubentry(a)
\fmtSense{pearl}
\fmtExampleRL{OYe\-su ni\-ye ibbaa\-le erya iku\-ra.}
\fmtExampleEN{Jesus is a precious pearl.}
\fmtSubentry(b)
\fmtSense{stone put inside the drum}
\fmtExampleRL{Ei\-bbaa\-le eri\-bba mu\-nda mu ngo\-ma ba\-lye\-ta ei\-ku\-ra.}
\fmtExampleEN{The stone put inside the drum is called 'ikura'.}

\fmtLemma{iku\-ta}
\fmtCitation{(okw)i\-ku\-ta}
\fmtPOS{v}%
\fmtTrans{intr}
\fmtPFV{iku\-ti\-re}
\fmtSense{to be satisfied, to be content}
\fmtExampleRL{Ka nsuu\-bi\-re nti ati oi\-ku\-ti\-re.}
\fmtExampleEN{Let me hope that you are now satisfied.}

\fmtLemma{ikuu\-ku}
\fmtCitation{(o)i\-kuu\-ku}
\fmtPOS{n}%
\fmtNC{1a/10}
\fmtSubentry(a)
\fmtSense{type of weed}
\fmtExampleRL{Oi\-kuu\-ku mu\-zi\-bu wa ku\-zi\-kya.}
\fmtExampleEN{Ikuuku is difficult to weed.}
\fmtSubentry(b)
\fmtSense{dangerous thing or person}
\fmtExampleRL{OKa\-ju\-ra afuu\-ki\-re ikuu\-ku ta\-kya\-wuu\-ra.}
\fmtExampleEN{Kajura has become very dangerous because he no longer listens.}

\fmtLemma{ikya}
\fmtCitation{(okw)i\-kya}
\fmtPOS{v}%
\fmtTrans{tr}
\fmtPFV{iki\-rye}
\fmtSubentry(a)
\fmtSense{to deliver, to bring, to take, to transport, to courier}
\fmtExampleRL{Ikya obu\-kwe\-nda bu\-ni eji je\-ge\-nge.}
\fmtExampleEN{Deliver this message to the king.}
\fmtSubentry(b)
\fmtSense{to have ready for harvest}
\fmtExampleRL{OMu\-ka\-la\-sa ai\-ki\-rye emi\-ye\-mbe.}
\fmtExampleEN{Mukalasa has his mangoes ready for harvest.}
\fmtSubentry(c)
\fmtSense{to do something for the sake of it}
\fmtExampleRL{Ako\-la kwi\-kya mu\-rwa.}
\fmtExampleEN{He is working for the sake of it.}
\fmtSubentry(d)
\fmtFixedexpression{ikya (o)mwo\-yo}
\fmtSense{to calm down, relax}
\fmtExampleRL{Mwa\-na wa\-nge ikya omwo\-yo oi\-za ku\-te\-ree\-ra.}
\fmtExampleEN{My dear, relax, everything will be okay.}
\fmtSubentry(e)
\fmtFixedexpression{ikya (o)mwo\-yo}
\fmtSense{to sigh, to breathe out, to exhale}
\fmtExampleRL{Ni ya\-mboi\-ne nai\-kya omwo\-yo.}
\fmtExampleEN{He sighed with relief when he saw me.}

\fmtLemma{ima}
\fmtCitation{(okw)i\-ma}
\fmtPOS{v}%
\fmtTrans{ditr}
\fmtPFV{imi\-re}
\fmtSense{to deny, to refuse, to turn down, to reject, to decline}
\fmtExampleRL{Baa\-nyi\-mi\-re omu\-ki\-sa gwa kwa\-ba omu iso\-me\-ro.}
\fmtExampleEN{I was denied the opportunity to go to school.}

\fmtLemma{imaa\-mu}
\fmtCitation{(o)i\-maa\-mu}
\fmtPOS{n}%
\fmtNC{1a/2}
\fmtSource{from Ar. }
\fmtSense{imam (an Islamic leader)}
\fmtExampleRL{Aba\-si\-raa\-mu ba Ga\-li\-raa\-ya ba\-fu\-ni\-re oi\-maa\-mu owo\-ndi.}
\fmtExampleEN{The Muslims at Galiraa\-ya have got a new imam.}

\fmtLemma{imi}
\fmtCitation{\textnormal{(e.g.}\,mwi\-mi, bee\-mi\textnormal{)}}
\fmtPOS{adj}
\fmtSense{mean, stingy, ungenerous}
\fmtExampleRL{Omu\-ntu omwi\-mi ai\-ma n’o\-waa\-mwe.}
\fmtExampleEN{A stingy person does not want to share even with his relative.}

\fmtLemma{imo\-ko\-le}
\fmtCitation{(e)i\-mo\-ko\-le}
\fmtPOS{n}%
\fmtNC{5}
\fmtSense{left side, the left}
\fmtExampleRL{Omu\-ko\-no gwa\-nge ogwa imo\-ko\-le gu\-ndu\-ma.}
\fmtExampleEN{I feel pain in my left arm.}

\fmtLemma{imu\-ka}
\fmtCitation{(okw)i\-mu\-ka}
\fmtPOS{v}%
\fmtTrans{intr}
\fmtPFV{imu\-ki\-re}
\fmtSubentry(a)
\fmtSense{to stand up, to rise, to get on one's feet}
\fmtExampleRL{Imu\-ka twa\-be.}
\fmtExampleEN{Stand up and let's go.}
\fmtSubentry(b)
\fmtSense{to wake up}
\fmtExampleRL{Imu\-ka olye eme\-re.}
\fmtExampleEN{Wake up and eat.}
\fmtSubentry(c)
\fmtSense{(of a war) to erupt}
\fmtExampleRL{Olu\-ta\-lo lwa\-yi\-mu\-kii\-re e Co\-ngo.}
\fmtExampleEN{A war erupted in Congo.}

\fmtLemma{imu\-ki}
\fmtCitation{\textnormal{(e.g.}\,kii\-mu\-ki, bii\-mu\-ki, kye\-mu\-ki\textnormal{)}}
\fmtPOS{adj}
\fmtSense{growing well}
\fmtExampleRL{Ebi\-duu\-ma bya\-nge bii\-mu\-ki.}
\fmtExampleEN{My maize is growing well.}

\fmtLemma{imu\-ki\-ra\-mu}
\fmtCitation{(okw)i\-mu\-ki\-ra\-mu}
\fmtPOS{v}%
\fmtTrans{intr}
\fmtPFV{imu\-kii\-re\-mu}
\fmtSense{to act immediately}
\fmtExampleRL{Na\-mwe\-te\-re nai\-mu\-ki\-ra\-mu.}
\fmtExampleEN{I called him and he acted immediately.}

\fmtLemma{imu\-la}
\fmtCitation{(okw)i\-mu\-la}
\fmtPOS{v}%
\fmtTrans{tr}
\fmtPFV{imwi\-re}
\fmtSubentry(a)
\fmtSense{to lift, to pick up, to raise, to hold up, to elevate}
\fmtExampleRL{Imu\-la ensa\-wu eyo ogye\-tii\-ke oku mu\-twe.}
\fmtExampleEN{Lift that bag and carry it on your head.}
\fmtSubentry(b)
\fmtSense{to wake up, to awaken, to cause to stop sleeping}
\fmtExampleRL{Ya\-ba oi\-mu\-le oi\-tee\-wo alye kyai\-go\-lo.}
\fmtExampleEN{Go and wake up your father for supper.}
\fmtSubentry(c)
\fmtSense{to raise one's hand}
\fmtExampleRL{Eyee\-te\-je\-rye ayi\-mu\-le omu\-ko\-no.}
\fmtExampleEN{Raise your hand if you have understood.}
\fmtSubentry(d)
\fmtSense{to invoke (a spirit), to summon}
\fmtExampleRL{Baa\-bi\-re kwi\-mu\-la mba\-ndwa.}
\fmtExampleEN{They have gone to invoke the spirits.}
\fmtSubentry(e)
\fmtSense{to introduce, to recognise the presence of people by asking them to stand up}
\fmtExampleRL{Yi\-mu\-la aba\-ge\-ni abo le\-ke aba\-ntu ba\-ba\-ma\-nye.}
\fmtExampleEN{Introduce the visitors so that people may know them.}
\fmtSubentry(f)
\fmtSense{to help, to assist, to aid}
\fmtExampleRL{Bu\-li ki\-ro ONya\-nge\-ndo ayi\-mu\-la omu\-lwai\-re.}
\fmtExampleEN{Nyangendo helps a patient every night.}

\fmtLemma{imuu\-ra}
\fmtCitation{(okw)i\-muu\-ra}
\fmtPOS{v}%
\fmtTrans{tr}
\fmtPFV{imu\-lii\-re}
\fmtSense{to break the hard crust of the potato mound in order to allow water and air to permeate through it}

\fmtLemma{inaa\-mii\-ra}
\fmtCitation{(okw)i\-naa\-mii\-ra}
\fmtPOS{v}%
\fmtTrans{intr}
\fmtPFV{ina\-mi\-rii\-re}
\fmtSense{to hunch, to bend (down), to incline one's body}
\fmtExampleRL{Lwa\-ki oyi\-naa\-mii\-ra otyo, omu\-go\-ngo gu\-kwi\-za ku\-lwa\-la.}
\fmtExampleEN{Why are you bending like that? Your back will get injured.}

\fmtLemma{inai}
\fmtCrossreference{used in isolation and with class 9 nouns, see also \textit{nai}}
\fmtPOS{num}
\fmtSense{four}
\fmtExampleRL{Oyi\-nza oku\-kwa\-ta ena\-mba ei\-nai eki\-tu\-ndu.}
\fmtExampleEN{You may get number four and a half.}

\fmtLemma{ina\-la}
\fmtCitation{(okw)i\-na\-la}
\fmtPOS{v}%
\fmtTrans{intr}
\fmtPFV{inai\-re}
\fmtSense{to become impoverished, to become poor}
\fmtExampleRL{Omu\-sai\-za oyo yaa\-li mu\-guu\-da bai\-tu ati ata\-ndi\-ki\-re okwi\-na\-la.}
\fmtExampleEN{That man was rich, but he became poor.}

\fmtLemma{ina\-ma}
\fmtCitation{(okw)i\-na\-ma}
\fmtPOS{v}%
\fmtTrans{intr}
\fmtPFV{ina\-mi\-re}
\fmtSubentry(a)
\fmtSense{to bend (down), to incline one's body}
\fmtExampleRL{Akwi\-na\-ma oku\-ko\-ma ebi\-sa\-si\-ro.}
\fmtExampleEN{He is bending down to pick up rubbish.}
\fmtSubentry(b)
\fmtSense{to bow (as a sign of respect or greeting)}
\fmtExampleRL{Aba\-ja\-paa\-ni bai\-na\-ma nga ba\-ku\-la\-mu\-kya.}
\fmtExampleEN{The Japanese bow while greeting.}

\fmtLemma{ina\-mi\-ra}
\fmtCitation{(okw)i\-na\-mi\-ra}
\fmtPOS{v}%
\fmtTrans{intr}
\fmtPFV{ina\-mii\-re}
\fmtSense{to chat, to pass time, to converse, to chatter}
\fmtExampleRL{Aboo\-jo mwi\-ze twi\-na\-mi\-re.}
\fmtExampleEN{Friends, come and let's chat.}

\fmtLemma{ina\-mi\-ri}
\fmtCitation{\textnormal{(e.g.}\,mwi\-na\-mi\-ri\textnormal{)}}
\fmtPOS{adj}
\fmtSense{to be sad}
\fmtExampleRL{Ya\-li mwi\-na\-mi\-ri n'a\-tu\-li\-ka ama\-li\-ga.}
\fmtExampleEN{She was so sad, tears flowed down her face.}

\fmtLemma{ina\-mya}
\fmtCitation{(okw)i\-na\-mya}
\fmtPOS{v}%
\fmtTrans{tr}
\fmtPFV{ina\-mi\-rye}
\fmtSense{to bend, to force into a curve or an angle}
\fmtExampleRL{Ina\-mya omu\-ye\-mbe tu\-no\-ko\-le emi\-ye\-mbe.}
\fmtExampleEN{Bend the mango tree so that we can pick the mangoes.}

\fmtLemma{ina\-yi\-na}
\fmtCitation{(okw)i\-na\-yi\-na}
\fmtPOS{v}%
\fmtTrans{intr}
\fmtPFV{ina\-yi\-ni\-re}
\fmtSense{to loiter, to stroll, to wander, to roam around}
\fmtExampleRL{Ebi\-so\-lo bya\-mu bi\-kwi\-na\-yi\-na kya\-njo.}
\fmtExampleEN{Your animals are roaming around.}

\fmtLemma{ina\-za}
\fmtCitation{(okw)i\-na\-za}
\fmtPOS{v}%
\fmtTrans{tr}
\fmtPFV{inai\-ze}
\fmtSubentry(a)
\fmtSense{to destroy, to ruin, to harm, to crash, to demolish}
\fmtExampleRL{Ei\-ba\-ra eri\-bbi lii\-na\-za ebi\-see\-ra bya mwa\-na ebya mu mai\-so.}
\fmtExampleEN{A bad name destroys a child's future.}
\fmtSubentry(b)
\fmtSense{to impoverish, to make poor}
\fmtExampleRL{Ega\-vu\-me\-nti ya Obo\-te ya\-yi\-nai\-ze Obu\-ga\-nda.}
\fmtExampleEN{Obote's government impoverished the Buganda Kingdom.}

\fmtLemma{indu\-ka}
\fmtCitation{(okw)i\-ndu\-ka}
\fmtPOS{v}%
\fmtTrans{intr}
\fmtPFV{indu\-ki\-re}
\fmtSubentry(a)
\fmtSense{to turn, to move around, to rotate}
\fmtExampleRL{Yi\-ndu\-ka olo\-le oku gu\-lyo.}
\fmtExampleEN{Turn and face right.}
\fmtSubentry(b)
\fmtSense{to change direction, to go different way}
\fmtExampleRL{Ya\-yi\-ndu\-ki\-re n’a\-kwa\-ta ogwa bu\-gwa.}
\fmtExampleEN{He changed direction and took the western route.}
\fmtSubentry(c)
\fmtSense{to break away, to separate from}
\fmtExampleRL{OMu\-se\-ve\-ni ya\-yi\-ndu\-ki\-re oku DP.}
\fmtExampleEN{Museveni broke away from the Democratic Party.}

\fmtLemma{indu\-la}
\fmtCitation{(okw)i\-ndu\-la}
\fmtPOS{v}%
\fmtTrans{tr}
\fmtPFV{indwi\-re}
\fmtSubentry(a)
\fmtSense{to change, to modify}
\fmtExampleRL{Omwo\-jo aku\-yi\-ndu\-la ama\-mi\-ro.}
\fmtExampleEN{The boy's voice is changing due to puberty.}
\fmtSubentry(b)
\fmtSense{to transfer}
\fmtExampleRL{ODa\-ma\-li a\-bbai\-re mu\-ntu mu\-sai, wa\-bu\-la ba\-mu\-yi\-ndwi\-re.}
\fmtExampleEN{Damali has been a good person, but she has been transferred.}
\fmtSubentry(c)
\fmtSense{to avert (disease, bad weather, calamity), to prevent, to ward off}
\fmtExampleRL{Eki\-yi\-ndu\-la ki\-yi\-ndu\-la emi\-sa\-ngo.}
\fmtExampleEN{The Abutilon indicum plant is believed to avert judgement of cases in courts of law.}
\fmtSubentry(d)
\fmtSense{to transfer labour pains to the husband (It is believed that a woman can transfer the labour pains to the father of the child.)}
\fmtExampleRL{Omu\-ka\-li anyi\-ndu\-lii\-re eki\-da.}
\fmtExampleEN{The woman has transferred her labour pains to me.}
\fmtSubentry(e)
\fmtSense{to open (a page of a book)}
\fmtExampleRL{Yi\-ndu\-la oyi\-re oku olu\-pa\-pu\-la lwa 104 olwa ki\-dee\-ro kya Lu\-ga\-nda.}
\fmtExampleEN{Open page 104 of the Luganda dictionary.}
\fmtSubentry(f)
\fmtSense{to interpret, to translate}
\fmtExampleRL{Ayi\-nduu\-ra aba\-ge\-nyi oru\-ba\-zo oku\-zwa omu\-li\-mi olu\-ndi.}
\fmtExampleEN{She interprets speech from a different language for visitors.}

\fmtLemma{indu\-lu\-ge\-nsi\-ya}
\fmtCitation{(e)i\-ndu\-lu\-ge\-nsi\-ya}
\fmtPOS{n}%
\fmtNC{9}
\fmtSource{from En.}
\fmtSense{indulgence}
\fmtExampleRL{N’oju\-na aboo\-ro ng’o\-li mu ki\-sii\-bo ofu\-na ei\-ndu\-lu\-ge\-nsi\-ya oku\-zwa ewa Ka\-nca.}
\fmtExampleEN{When you give alms to the poor, you receive indulgences from God.}

\fmtLemma{ingaa\-ni\-sya}
\fmtCitation{(okw)i\-ngaa\-ni\-sya}
\fmtPOS{v}%
\fmtTrans{tr}
\fmtPFV{inga\-ni\-si\-rye}
\fmtSense{to exchange, to trade, to swap}
\fmtExampleRL{Aba\-sa\-mbi ba mu\-pii\-ra ba\-te\-ra okwi\-ngaa\-ni\-sya emi\-joo\-zi.}
\fmtExampleEN{Football players usually exchange their jerseys.}

\fmtLemma{ingai}%
\fmtCitation{\textnormal{(e.g.}\,zi\-ngai\textnormal{)}}
\fmtPOS{interrog}
\fmtSubentry(a)
\fmtSense{how much, how many}
\fmtExampleRL{Omu\-wai\-re empii\-ya zi\-ngai?}
\fmtExampleEN{How much money have you given him?}
\fmtSubentry(b)
\fmtFixedexpression{ka ingai}
\fmtSense{how often}
\fmtExampleRL{Ka ingai nga nkwi\-za ani n’o\-ta\-bo\-ne\-ke\-ku?}
\fmtExampleEN{How often have I come here without seeing you?}

\fmtLemma{inga\-na}
\fmtCrossreference{\textnormal{see} enkana}
\fmtCitation{(okw)i\-nga\-na}
\fmtPOS{v}%
\fmtTrans{intr}
\fmtPFV{ingai\-ne}
\fmtSubentry(a)
\fmtSense{to be equal, to be like, to be the same
}
\fmtExampleRL{Omu\-sai\-za oyo n’o\-mu\-ka\-li waa\-mwe bai\-nga\-na.}
\fmtExampleEN{That man and his wife are equal.}
\fmtSubentry(b)
\fmtSense{to quarrel, to verbally assault, to argue, to fight}
\fmtExampleRL{Bai\-nga\-na mu\-no bai\-tu ba\-si\-ga\-la ba\-kwe\-ndya\-ga\-na.}
\fmtExampleEN{They quarrel a lot, but still love each other.}

\fmtLemma{ingi}
\fmtCitation{\textnormal{(e.g.}\,ki\-ngi, bi\-ngi\textnormal{)}}
\fmtPOS{adj}
\fmtSense{numerous, multiple, many}
\fmtExampleRL{Ebi\-duu\-ma bi\-ngi ebi\-ro bi\-ni.}
\fmtExampleEN{Maize is plentiful these days.}

\fmtLemma{ingi\-ni\-ya}
\fmtCitation{(o)i\-ngi\-ni\-ya}
\fmtPOS{n}%
\fmtNC{1a/2}
\fmtSource{from En.}
\fmtSense{engineer}
\fmtExampleRL{Omwa\-na wa\-nge aku\-so\-mee\-ra bwa ingi\-ni\-ya.}
\fmtExampleEN{My child is studying to become an engineer.}

\fmtLemma{ingi\-ra}
\fmtCitation{(okw)i\-ngi\-ra}
\fmtPOS{v}%
\fmtTrans{intr}
\fmtPFV{ingii\-re}
\fmtSubentry(a)
\fmtSense{to enter, to go inside}
\fmtExampleRL{Yi\-ngi\-ra oyi\-ka\-le ansi.}
\fmtExampleEN{Enter and sit down!}
\fmtSubentry(b)
\fmtSense{to intervene, to interfe\-re}
\fmtExampleRL{Iza oi\-ngi\-re omu nso\-nga za mu\-to waa\-mu.}
\fmtExampleEN{Come and intervene in the matters concerning your younger brother.}

\fmtLemma{ingi\-rii\-ra}
\fmtCitation{(okw)i\-ngi\-rii\-ra}
\fmtPOS{v}%
\fmtTrans{intr}/\fmtTrans{tr}
\fmtPFV{ingi\-rii\-re}
\fmtSubentry(a)
\fmtSense{to attack, to assault}
\fmtExampleRL{Bbaa\-mwi\-ngi\-rii\-re ne ba\-mwi\-bba ebi\-ntu byo\-na.}
\fmtExampleEN{They attacked her and stole everything from her.}
\fmtSubentry(b)
\fmtSense{to impede, to hinder, to inhibit, to disturb, to interfe\-re}
\fmtExampleRL{Lwa\-ki okwi\-ngi\-rii\-ra entee\-ka\-tee\-ka za mwi\-raa\-wo?}
\fmtExampleEN{Why do you interfere with your colleague’s affairs?}
\fmtSubentry(c)
\fmtSense{(of a man) to live at his wife's home}
\fmtExampleRL{Omu Ba\-ruu\-li omu\-sai\-za okwi\-ngi\-rii\-ra omu\-ka\-li.}
\fmtExampleEN{It is a taboo among the Baruuli for a man to live at his wife's home.}

\fmtLemma{ingi\-rya}%
\fmtOne{1}
\fmtCitation{(okw)i\-ngi\-rya}
\fmtPOS{v}%
\fmtTrans{tr}
\fmtPFV{ingii\-rye}
\fmtSubentry(a)
\fmtSense{to take into, to bring in, to take in}
\fmtExampleRL{OBi\-ntu\-bi\-zi\-bu akwi\-ngi\-rya obu\-sye omu nyu\-mba.}
\fmtExampleEN{Bintubizibu is taking millet flour into the house.}
\fmtSubentry(b)
\fmtSense{to adopt (a child)}
\fmtExampleRL{Omwa\-na oyo na\-mwi\-ngii\-rye omu lu\-bu\-ga lwa\-nge.}
\fmtExampleEN{I adopted that child into my family.}

\fmtLemma{ingi\-rya\-mu}
\fmtCitation{(okw)i\-ngi\-rya\-mu}
\fmtPOS{v}%
\fmtTrans{tr}
\fmtPFV{ingii\-rye\-mu}
\fmtSense{to include, to incorporate}
\fmtExampleRL{Baa\-bwi\-rye bi\-ga\-mbo ki ebya\-ku\-ta omu nso\-boo\-ro.}
\fmtExampleEN{They asked which words need to be included in the dictionary.}

\fmtLemma{ingi\-sya}
\fmtCitation{(okw)i\-ngi\-sya}
\fmtPOS{v}%
\fmtTrans{tr}
\fmtPFV{ingi\-si\-rye}
\fmtSense{to exchange, to swap, to barter}
\fmtExampleRL{Bai\-ngi\-si\-rye ena\-mba za si\-mu ni bai\-ti\-rai\-ne.}
\fmtExampleEN{They exchanged phone numbers when they met.}

\fmtLemma{ini}
\fmtCitation{(e)i\-ni}
\fmtPOS{n}%
\fmtNC{5/6}
\fmtSense{liver}
\fmtExampleRL{Ei\-ni lya\-mwe li\-rwai\-re.}
\fmtExampleEN{His liver is infected.}

\fmtLemma{ini\-ka}
\fmtCitation{(okw)i\-ni\-ka}
\fmtPOS{v}%
\fmtTrans{tr}
\fmtPFV{ini\-ki\-re}
\fmtSubentry(a)
\fmtSense{to immerse, to dip, to duck, to submerge}
\fmtExampleRL{Ka\-bo\-na ya\-yi\-ni\-ki\-re omwa\-na omu mai\-zi.}
\fmtExampleEN{The priest immersed the child in water.}
\fmtSubentry(b)
\fmtSense{to soak}
\fmtExampleRL{Ini\-ka obu\-sye tu\-ko\-le ama\-lwa.}
\fmtExampleEN{Soak the millet and we will make beer.}

\fmtLemma{ini\-ko}
\fmtCitation{(e)i\-ni\-ko}
\fmtPOS{n}%
\fmtNC{5/6}
\fmtSubentry(a)
\fmtSense{flank hip bone}
\fmtExampleRL{Ya\-ye\-nde\-kei\-re mu ini\-ko.}
\fmtExampleEN{He sustained a fracture of the flank hip bone.}
\fmtSubentry(b)
\fmtSense{pelvic girdle}
\fmtExampleRL{Aye\-nde\-ke\-re ama\-ni\-ko.}
\fmtExampleEN{He has broken his pelvic girdle.}

\fmtLemma{ino\-li}
\fmtCitation{(e)i\-no\-li}
\fmtPOS{n}%
\fmtNC{5/6}
\fmtSense{fat}
\fmtExampleRL{Ontaa\-ma a\-bba n’a\-ma\-no\-li gai\-ngi.}
\fmtExampleEN{The mutton has a lot of fat.}

\fmtLemma{iso\-ngo\-lya nku\-be\-be}
\fmtCitation{(e)i\-so\-ngo\-lya nku\-be\-be}
\fmtPOS{n}%
\fmtNC{5/6}
\fmtSense{sharp-pointed stump or log}
\fmtExampleRL{Ei\-so\-ngo\-lya nku\-be\-be li\-su\-mi\-ta abaa\-na nga ba\-ku\-zee\-nye\-ra omu ki\-si\-ko.}
\fmtExampleEN{Sharp-pointed logs often prick children while they play in the bush.}

\fmtLemma{inu\-ka}
\fmtCitation{(okw)i\-nu\-ka}
\fmtPOS{v}%
\fmtTrans{intr}
\fmtPFV{inu\-ki\-re}
\fmtSense{to end the day's work}
\fmtExampleRL{Obwi\-re bwa kwi\-nu\-ka ni\-bwo bu\-ni.}
\fmtExampleEN{It is time to end the day's work.}

\fmtLemma{inu\-la}
\fmtCitation{(okw)i\-nu\-la}
\fmtPOS{v}%
\fmtTrans{tr}
\fmtPFV{inwi\-re}
\fmtSense{to ask (someone) to end the day's work}
\fmtExampleRL{Ya\-ba omwi\-nu\-le.}
\fmtExampleEN{Go and ask her to stop working.}

\fmtLemma{inu\-lu\-la}
\fmtCitation{(okw)\-e\-nu\-lu\-la}
\fmtPOS{v}%
\fmtTrans{intr}
\fmtPFV{inu\-lwi\-re}
\fmtSubentry(a)
\fmtSense{to save oneself from trouble}
\fmtExampleRL{Omu\-li\-raa\-nwa ta\-li\-na se\-nte za kwe\-nu\-lu\-la ku\-zwa mu bu\-zi\-bu bw’a\-li\-mu.}
\fmtExampleEN{My neighbour has no money to save himself from the trouble he is in.}
\fmtSubentry(b)
\fmtSense{to attain independence}
\fmtExampleRL{Uga\-nda ye\-nu\-lwi\-re oku\-zwa omu mi\-ko\-no gya Ba\-ju\-ngu mu 1962.}
\fmtExampleEN{Uganda attained independence from the British in 1962.}

\fmtLemma{inu\-nu\-ka}
\fmtCitation{(okw)i\-nu\-nu\-ka}
\fmtPOS{v}%
\fmtTrans{intr}
\fmtPFV{inu\-nu\-ki\-re}
\fmtSense{to straighten one's back}
\fmtExampleRL{Ni ya\-boi\-ne omu\-ka\-li waa\-mwe n’ai\-nu\-nu\-ka.}
\fmtExampleEN{When he saw his wife, he straightened his back.}

\fmtLemma{ipaa\-paa\-li}
\fmtCitation{(e)i\-paa\-paa\-li}
\fmtPOS{n}%
\fmtNC{5/6}
\fmtSense{papaya}
\fmtExampleRL{Na\-guu\-re ama\-paa\-paa\-li ataa\-nu nga ga baa\-na ba\-nge.}
\fmtExampleEN{I bought five papayas for my children.}

\fmtLemma{ipee\-ra}
\fmtCitation{(e)i\-pee\-ra}
\fmtPOS{n}%
\fmtNC{9/10}
\fmtSource{from Port.\,via Sw.}
\fmtSense{guava}
\fmtExampleRL{Ba\-mu\-lii\-re\-ku ei\-pee\-ra.}
\fmtExampleEN{He has missed eating guava.}

\fmtLemma{ipee\-sa}
\fmtCitation{(e)i\-pee\-sa}
\fmtPOS{n}%
\fmtNC{5/6}
\fmtSense{button}
\fmtExampleRL{Esaa\-ti ya\-nge eri\-ku ama\-pee\-sa ataa\-nu.}
\fmtExampleEN{My shirt has five buttons.}

\fmtLemma{-ir}
\fmtPOS{sf}
\fmtSense{applicative suffix (extension) on verbs}

\fmtLemma{ira}%
\fmtOne{1}
\fmtCitation{(okw)i\-ra}
\fmtPOS{v}%
\fmtTrans{intr}
\fmtSubentry(a)
\fmtSense{to go, to move towards, to move in the direction of}
\fmtExampleRL{Ira edi.}
\fmtExampleEN{Go that way.}
\fmtSubentry(b)
\fmtSense{to come (back), to return}
\fmtExampleRL{Bai\-ra di e Ka\-yu\-nga?}
\fmtExampleEN{When are they returning to Kayunga?}
\fmtSubentry(c)
\fmtFixedexpression{ira (e)nyu\-ma}
\fmtSense{to move in reverse, to move back}
\fmtExampleRL{Emo\-to\-ka ya\-gii\-rye enyu\-ma ku\-bba ye\-ra\-bii\-re ensa\-wu yaa\-mu.}
\fmtExampleEN{He reversed the car because he had forgotten his bag.}
\fmtSubentry(d)
\fmtFixedexpression{ira (e)nyu\-ma}
\fmtSense{to regress, to deteriorate}
\fmtExampleRL{Bbi\-zi\-nee\-nsi yaa\-mwe ei\-re enyu\-ma.}
\fmtExampleEN{His business is not performing well.}

\fmtLemma{ira}%
\fmtOne{2}
\fmtCitation{(e)i\-ra}
\fmtPOS{n}%
\fmtNC{5/6}
\fmtSense{pearl}
\fmtExampleRL{Abaa\-na baa\-boi\-ne ei\-ra oku nya\-nja.}
\fmtExampleEN{The children found a pearl at the lake shore.}

\fmtLemma{ira\-ga\-zu}
\fmtCrossreference{\textnormal{see} iragazu}
\fmtCitation{\textnormal{(e.g.}\,gi\-ra\-gu\-zu, bye\-ra\-ga\-zu, lwi\-ra\-ga\-zu\textnormal{)}}
\fmtPOS{adj}
\fmtSense{black}
\fmtExampleRL{Olu\-go\-ye lwa\-mwe lwi\-ra\-ga\-zu.}
\fmtExampleEN{His cloth is black.}

\fmtLemma{ira\-gu\-lya}
\fmtCitation{(okw)i\-ra\-gu\-lya}
\fmtPOS{v}%
\fmtTrans{tr}
\fmtPFV{ira\-gwi\-rye}
\fmtSubentry(a)
\fmtSense{to soil, to smear, to stain,}
\fmtExampleRL{Omwa\-na ai\-ra\-gwi\-rye saa\-ti ya\-nge.}
\fmtExampleEN{A child has soiled my shirt.}
\fmtSubentry(b)
\fmtSense{to dye black, to blacken, to make black}
\fmtExampleRL{Ku mba\-ga ya mwa\-ra wa\-nge aba\-ka\-li bai\-ra\-gu\-rya ei\-so\-ke.}
\fmtExampleEN{For my daughter’s wedding, the women will dye their hair black.}

\fmtLemma{ira\-gu\-ra}
\fmtCitation{(okw)i\-ra\-gu\-ra}
\fmtPOS{v}%
\fmtTrans{intr}
\fmtPFV{ira\-gwi\-re}
\fmtSense{to be dirty, to be filthy}
\fmtExampleRL{Egi\-raa\-si ekwi\-ra\-gu\-ra.}
\fmtExampleEN{The glass is dirty.}

\fmtLemma{ira\-gu\-ya}
\fmtCitation{(okw)i\-ra\-gu\-rya}
\fmtPOS{v}%
\fmtTrans{intr}
\fmtPFV{ira\-gwi\-rye}
\fmtSense{(of a beard) to grow}
\fmtExampleRL{Omu\-taa\-ni wa\-nge ai\-ra\-gwi\-rye omu\-le\-zu.}
\fmtExampleEN{My son has started growing a beard.}

\fmtLemma{ira\-gu\-zu}
\fmtCrossreference{\textnormal{see} iragazu}

\fmtLemma{irai}
\fmtCitation{(e)i\-rai}
\fmtPOS{n}%
\fmtNC{9/10}
\fmtSubentry(a)
\fmtSense{ancient times, old age}
\fmtExampleRL{Mpee\-rya eki\-ta\-bo ekyo ekya irai.}
\fmtExampleEN{Give me that ancient book.}
\fmtSubentry(b)
\fmtSense{once, long ago, in the past, earlier, already}
\fmtExampleRL{Yaa\-bi\-re irai nga okya\-bba\-si\-rye.}
\fmtExampleEN{He went a long time ago while you were still asleep.}
\fmtSubentry(c)
\fmtFixedexpression{irai budi}
\fmtSense{long ago}
\fmtExampleRL{Ei\-rai bu\-di ng’o\-mu\-ka\-li ta\-lya ndu.}
\fmtExampleEN{In the past women used not to eat lungfish.}
\fmtSubentry(d)
\fmtFixedexpression{irai n’(e)i\-rai}
\fmtSense{since time immemorial, very long ago, in the distant past}
\fmtExampleRL{Ei\-rai n’ei\-rai ng’a\-ba\-ka\-li aBa\-ga\-nda ti\-ba\-vu\-ga gaa\-li.}
\fmtExampleEN{Long ago, Baganda women did not ride bicycles.}

\fmtLemma{ira\-ka}%
\fmtOne{1}
\fmtCitation{(e)i\-ra\-ka}
\fmtPOS{n}%
\fmtNC{5/6}
\fmtSubentry(a)
\fmtSense{sound, noise}
\fmtExampleRL{Ki\-ki ekyo eki\-koo\-re ei\-ra\-ka eri\-koo\-to?}
\fmtExampleEN{What made the loud sound?}
\fmtSubentry(b)
\fmtSense{voice}
\fmtExampleRL{Ka\-ngu\-la oku ira\-ka aba\-ntu ba\-ku\-wuu\-re.}
\fmtExampleEN{Speak loudly and clearly so that people hear you.}

\fmtLemma{ira\-ka}%
\fmtOne{2}
\fmtCitation{(e)i\-ra\-ka}
\fmtPOS{n}%
\fmtNC{5/6}
\fmtSense{\textit{Cy\-pe\-rus ro\-tu\-ndus} (coco-grass, Java grass, nut grass)}

\fmtLemma{ira\-lu}
\fmtCitation{(e)i\-ra\-lu}
\fmtPOS{n}%
\fmtNC{5}
\fmtSubentry(a)
\fmtSense{madness, insanity}
\fmtExampleRL{Omu Ka\-mpa\-la mu\-li\-mu aba\-gwi wa ira\-lu bai\-ngi.}
\fmtExampleEN{There are many madmen in Kampala.}
\fmtSubentry(b)
\fmtSense{stubbornness}
\fmtExampleRL{Le\-ke\-ra\-wo ei\-ra\-lu.}
\fmtExampleEN{Stop this stubbornness.}

\fmtLemma{ira\-mi\-ryo}
\fmtCitation{(e)i\-ra\-mi\-ryo}
\fmtPOS{n}%
\fmtNC{5/6}
\fmtSense{place of worship (\eg a shrine)}
\fmtExampleRL{Baa\-bi\-re mu ira\-mi\-ryo ku\-ra\-mya.}
\fmtExampleEN{They have gone to the shrine to worship.}

\fmtLemma{ira\-mu}
\fmtCitation{(okw)i\-ra\-mu}
\fmtPOS{v}%
\fmtTrans{tr}
\fmtPFV{ire\-mu}
\fmtSubentry(a)
\fmtSense{to do again, to repeat}
\fmtExampleRL{Nje okwi\-ra\-mu okwa\-ba ewa mu\-ka\-li oyo, ko\-yi!}
\fmtExampleEN{Me, to go back to that woman's place? No way!}
\fmtSubentry(b)
\fmtSense{to answer, to respond, to reply}
\fmtExampleRL{Odi na\-mwi\-ra\-mu, ``e\-se\-nte nkwi\-za ku\-ku\-wa," nti, ``oo\-h!" nti, ``ee\-h." aa\-h, n’a\-ya\-ba.}
\fmtExampleEN{And the other one responded, ``I will give the money". Then he said ``Ooh!", ``Eeh!", ``Aah", and he left.}
\fmtSubentry(c)
\fmtSense{to repeat (an exam)}
\fmtExampleRL{Ira\-mu omu ki\-bii\-na oso\-bo\-le okwa\-ka\-la oku\-sai.}
\fmtExampleEN{Repeat the class so that you get good grades.}

\fmtLemma{ira\-myo}%
\fmtOne{1}
\fmtCitation{(e)i\-ra\-myo}
\fmtPOS{n}%
\fmtNC{5/6}
\fmtSubentry(a)
\fmtSense{church}
\fmtExampleRL{Ei\-ra\-myo lya ba\-ka\-tu\-li\-ki ka li\-sai!}
\fmtExampleEN{The Catholic church is very good.}
\fmtSubentry(b)
\fmtSense{church service}
\fmtExampleRL{Ei\-ra\-myo lya lee\-ro li\-ku\-le\-mbei\-rwe Na\-ka\-zi\-bwe.}
\fmtExampleEN{Today's church service was led by Nakazibwe.}

\fmtLemma{ira\-myo}%
\fmtOne{2}
\fmtPOS{interj}
\fmtSense{the traditional way of thanking the potter}
\fmtExampleRL{Ira\-myo! - Binaga!}
\fmtExampleEN{Let them bake well! – The pots!}

\fmtLemma{ira\-nga}
\fmtCitation{(e)i\-ra\-nga}
\fmtPOS{n}%
\fmtNC{5/6}
\fmtSubentry(a)
\fmtSense{canna lily plant}
\fmtExampleRL{Eki\-me\-ra kya ei\-ra\-nga ki\-zwa\-ku ama\-la\-nga.}
\fmtExampleEN{The canna lily plant produces the canna lily grain.}
\fmtSubentry(b)
\fmtSense{canna lily seed}
\fmtExampleRL{Ama\-ra\-nga ni\-go ba\-tee\-ka omu nsa\-nsi.}
\fmtExampleEN{Canna lily seeds are used to make shakers.}
\fmtSubentry(c)
\fmtSense{steel ball}
\fmtExampleRL{Egaa\-li ebaa\-mu ama\-ra\-nga.}
\fmtExampleEN{The bicycle has steel balls.}

\fmtLemma{ira\-nti}
\fmtCitation{(e)i\-ra\-nti}
\fmtPOS{n}%
\fmtNC{5/6}
\fmtSense{pole supporting a hunting net}
\fmtExampleRL{Ebi\-ti\-mba mu\-bi\-te oku ma\-ra\-nti.}
\fmtExampleEN{Set the hunting nets on the poles.}

\fmtLemma{-ire}
\fmtPOS{sf}
\fmtSense{perfective aspect suffix on verbs}

\fmtLemma{ire\-mbo}%
\fmtCitation{(e)i\-re\-mbo}
\fmtPOS{n}%
\fmtNC{5/6}
\fmtSubentry(a)
\fmtSense{(of a house) front side}
\fmtExampleRL{Ya\-ka\-ngii\-re mu ire\-mbo n’a\-yi\-ra\-yo.}
\fmtExampleEN{He stopped in front of the house and went back.}
\fmtSubentry(b)
\fmtSense{home}
\fmtExampleRL{Ki\-ta\-ka bu\-li mu\-ntu n’a\-bba n’ei\-re\-mbo.}
\fmtExampleEN{Every person deserves to have a home.}

\fmtLemma{iri\-bwa}
\fmtCitation{(okw)i\-ri\-bwa}
\fmtPOS{v}%
\fmtTrans{intr}
\fmtPFV{irii\-bwe}
\fmtSense{to regain consciousness, to come to}
\fmtExampleRL{Abbai\-re aka\-bi\-re ai\-rii\-bwe.}
\fmtExampleEN{A person who had fainted regained consciousness.}

\fmtLemma{iri\-ga}
\fmtCitation{(e)i\-ri\-ga}
\fmtPOS{n}%
\fmtNC{5/6}
\fmtSense{tear drop, tear}
\fmtExampleRL{Taa\-li\-na go\-nze ei\-ri\-ga eri\-mwei omu mai\-so.}
\fmtExampleEN{She doesn’t have even a single teardrop.}

\fmtLemma{irii\-bwa}
\fmtCitation{(e)i\-rii\-bwa}
\fmtPOS{n}%
\fmtNC{5/6}
\fmtSense{cucumber}
\fmtExampleRL{Ama\-lii\-bwa ga\-woo\-ma nk’e\-byo\-ngu.}
\fmtExampleEN{Cucumbers taste like pumpkins.}

\fmtLemma{irii\-ka}
\fmtCitation{(okw)i\-rii\-ka}
\fmtPOS{v}%
\fmtTrans{intr}
\fmtPFV{irii\-ki\-re}
\fmtSense{to grumble, to complain, to grouse, to whine, to mutter, to protest, to make a fuss}
\fmtExampleRL{Le\-ke\-ra\-wo oku\-kwi\-rii\-ka.}
\fmtExampleEN{Stop grumbling.}

\fmtLemma{irii\-ra}%
\fmtOne{1}
\fmtCitation{(okw)i\-rii\-ra}
\fmtPOS{v}%
\fmtTrans{intr}
\fmtPFV{iri\-rii\-re}
\fmtSense{to go and return on the same day}
\fmtExampleRL{Ba\-yi\-ri\-rii\-re e Ka\-mpa\-la.}
\fmtExampleEN{They went to Kampala and returned on the same day.}

\fmtLemma{irii\-ra}%
\fmtOne{2}
\fmtCitation{(okw)i\-rii\-ra}
\fmtPOS{v}%
\fmtTrans{intr}
\fmtPFV{iri\-rii\-re}
\fmtSense{to be next}
\fmtExampleRL{Nii\-we okwi\-rii\-ra.}
\fmtExampleEN{You are next.}

\fmtLemma{irii\-ro}
\fmtCitation{(e)i\-rii\-ro}
\fmtPOS{n}%
\fmtNC{5/6}
\fmtSense{dining room, sitting room}
\fmtExampleRL{Eme\-re tu\-gi\-rii\-ra mu i\-rii\-ro.}
\fmtExampleEN{We eat food in the dining room.}

\fmtLemma{irii\-rya}
\fmtCitation{(okw)i\-rii\-rya}
\fmtPOS{v}%
\fmtTrans{tr}
\fmtPFV{irii\-rye}
\fmtSubentry(a)
\fmtSense{to return, to replace, to bring back, to give back}
\fmtExampleRL{Nyi\-rii\-rya esaa\-wa ya\-nge.}
\fmtExampleEN{Bring me back my watch.}
\fmtSubentry(b)
\fmtSense{to replace seeds that have failed to germinate}
\fmtExampleRL{Yi\-rii\-rya ebi\-duu\-ma ebyo.}
\fmtExampleEN{Replace the maize seeds that failed to germinate.}
\fmtSubentry(c)
\fmtSense{to give a tip, to tip}
\fmtExampleRL{Omaa\-ma amai\-te okwi\-rii\-rya aba\-ko\-li ba mu woo\-tee\-ri.}
\fmtExampleEN{Mom often gives a tip to hotel workers.}

\fmtLemma{irii\-si\-ryo}
\fmtCitation{(e)i\-rii\-si\-ryo}
\fmtPOS{n}%
\fmtNC{5/6}
\fmtSense{animal farm}
\fmtExampleRL{Ata n’ei\-rii\-si\-ryo li\-koo\-to.}
\fmtExampleEN{He has a big animal farm.}

\fmtLemma{iri\-ka}
\fmtCitation{(okw)i\-ri\-ka}
\fmtPOS{v}%
\fmtTrans{tr}
\fmtPFV{iri\-ki\-re}
\fmtSense{to keep together, to gather in one place}
\fmtExampleRL{Enyama, ebi\-yaa\-ta, su\-kaa\-li byo\-na\-byo\-na ba\-bi\-yi\-ri\-ki\-re omu ki\-sii\-ka ki\-mwei.}
\fmtExampleEN{They kept all the meat, potatoes, and sugar together in one room.}

\fmtLemma{iri\-ndu\-ka}
\fmtCitation{(okw)i\-ri\-ndu\-ka}
\fmtPOS{v}%
\fmtTrans{intr}
\fmtPFV{iri\-ndu\-ki\-re}
\fmtSense{to move in large numbers}
\fmtExampleRL{Olu\-go\-lo\-go\-mbo lwa mbu\-li lu\-yi\-ri\-ndu\-ki\-re mu bi\-duu\-ma bya\-nge.}
\fmtExampleEN{Many goats moved into my maize garden.}

\fmtLemma{iri\-ngi\-ta}
\fmtCitation{(okw)i\-ri\-ngi\-ta}
\fmtPOS{v}%
\fmtTrans{intr}
\fmtPFV{iri\-ngi\-ti\-re}
\fmtSense{to roll, to move by turning over and over}
\fmtExampleRL{Eki\-bbaa\-le ki\-kwi\-ri\-ngi\-ta.}
\fmtExampleEN{The stone is rolling.}

\fmtLemma{iri\-ngi\-tya}
\fmtCitation{(okw)i\-ri\-ngi\-tya}
\fmtPOS{v}%
\fmtTrans{tr}
\fmtPFV{iri\-ngi\-ti\-rye}
\fmtSense{to roll, to cause to move by turning over and over}
\fmtExampleRL{Yi\-ri\-ngi\-tya eki\-ko\-nge ekyo.}
\fmtExampleEN{Roll that tree stump.}

\fmtLemma{iro}
\fmtCrossreference{\textnormal{see also} irowo}
\fmtCitation{(e)i\-ro}
\fmtPOS{n}%
\fmtNC{5}
\fmtSense{craving for smoking, smoking urge}
\fmtExampleRL{Omu\-nywi wa taa\-ba a\-bba n’ei\-ro.}
\fmtExampleEN{A smoker has a craving for cigarettes.}

\fmtLemma{iro\-bo}
\fmtCitation{(e)i\-ro\-bo}
\fmtPOS{n}%
\fmtNC{5/6}
\fmtSense{hook, fishhook, barb}
\fmtExampleRL{Ei\-ro\-bo lya\-nge ndi\-te\-ge\-sya ma\-le.}
\fmtExampleEN{I use my hook to catch the mudfish.}

\fmtLemma{iro\-go}
\fmtCitation{(e)i\-ro\-go}
\fmtPOS{n}%
\fmtNC{5}
\fmtSense{witchcraft}
\fmtExampleRL{Aba\-ntu aba\-mwe ba\-ko\-la ei\-ro\-go.}
\fmtExampleEN{Some people practice witchcraft.}

\fmtLemma{iro\-mba}
\fmtCitation{(e)i\-ro\-mba}
\fmtPOS{n}%
\fmtNC{5/6}
\fmtSense{big navel}
\fmtExampleRL{Omu\-ka\-li oyo ata n’ei\-ro\-mba.}
\fmtExampleEN{The woman has a big navel.}

\fmtLemma{iro\-ngo}
\fmtCitation{(o)i\-ro\-ngo}
\fmtPOS{n}%
\fmtNC{1a}
\fmtSense{east wind}
\fmtExampleRL{Oi\-ro\-ngo n’a\-ku\-bba ala\-nga isa\-na.}
\fmtExampleEN{The east wind announces a drought.}

\fmtLemma{iro\-wo}
\fmtCrossreference{\textnormal{see also} iro}
\fmtCitation{(e)i\-ro\-wo}
\fmtPOS{n}%
\fmtNC{5}
\fmtSense{craving for smoking, smoking urge}
\fmtExampleRL{Ata n’ei\-ro\-wo lya ku\-nywa taa\-ba.}
\fmtExampleEN{She has a craving to smoke a cigarette.}

\fmtLemma{iru\-ba}
\fmtCitation{(e)i\-ru\-ba}
\fmtPOS{n}%
\fmtNC{5/6}
\fmtSense{well, borehole}
\fmtExampleRL{Baa\-bi\-re ku\-sa\-ka iru\-ba.}
\fmtExampleEN{They have gone to dig the well.}

\fmtLemma{iru\-bu\-za}
\fmtCitation{(okw)i\-ru\-bu\-za}
\fmtPOS{v}%
\fmtTrans{tr}
\fmtPFV{iru\-bi\-zi\-re, iru\-bwi\-ze}
\fmtSense{to move (animals) from one grazing place to the other}
\fmtExampleRL{Ya\-ba oi\-ru\-bu\-ze embu\-li, obwi\-re bwi\-ki\-re.}
\fmtExampleEN{Go, it is time to transfer the goats.}

\fmtLemma{iru\-gu\-za}
\fmtCitation{(e)i\-ru\-gu\-za}
\fmtPOS{n}%
\fmtNC{5/6}
\fmtSense{sub-county}
\fmtExampleRL{Na\-bya\-lii\-bwe mu iru\-gu\-za lya Ibbaa\-le.}
\fmtExampleEN{I was born in Bbaale sub-county.}

\fmtLemma{iru\-ka}
\fmtCitation{(okw)i\-ru\-ka}
\fmtPOS{v}%
\fmtTrans{intr}
\fmtPFV{iru\-ki\-re}
\fmtSense{to run, to run away, (of a vehicle) to move fast}
\fmtExampleRL{Omwa\-na ai\-ru\-ka mu\-no oku\-si\-nga aba\-ku\-lu.}
\fmtExampleEN{The child runs faster than adults.}

\fmtLemma{iru\-ke}
\fmtCitation{(e)i\-ru\-ke}
\fmtPOS{n}%
\fmtNC{5/6}
\fmtSense{gorilla}
\fmtExampleRL{Amai\-ru\-ke kya\-bu\-la\-mbu\-li omu Uga\-nda.}
\fmtExampleEN{Gorillas are a tourist attraction in Uganda.}

\fmtLemma{iru\-kii\-ra}
\fmtCitation{(okw)i\-ru\-kii\-ra}
\fmtPOS{v}%
\fmtTrans{tr}
\fmtPFV{iru\-ki\-rii\-re}
\fmtSense{to help, to rescue, to assist}
\fmtExampleRL{Mwi\-ru\-kii\-re ba\-di aba\-gwi\-re oku bu\-ta\-ndu.}
\fmtExampleEN{Rescue those who were involved in the accident.}

\fmtLemma{iru\-kya}
\fmtCitation{(okw)i\-ru\-kya}
\fmtPOS{v}%
\fmtTrans{tr}
\fmtPFV{iru\-ki\-rye}
\fmtSubentry(a)
\fmtSense{to chase away, to drive away}
\fmtExampleRL{Iru\-kya ebi\-nyo\-nyi omu ndu\-mi\-ro ya bwi\-ta.}
\fmtExampleEN{Chase away those birds from the millet garden.}
\fmtSubentry(b)
\fmtSense{to fire, to dismiss}
\fmtExampleRL{Ba\-mwi\-ru\-kyi\-rye oku mu\-li\-mu mu bu\-ta\-li bwe\-nka\-nya.}
\fmtExampleEN{He was dismissed from his job unfairly.}

\fmtLemma{iru\-ma}%
\fmtOne{1}
\fmtCitation{(o)i\-ru\-ma}
\fmtPOS{n}%
\fmtNC{1a/10}
\fmtSense{dove, pigeon}
\fmtExampleRL{Oi\-ru\-ma awoo\-ma mu\-no.}
\fmtExampleEN{Dove is very tasty.}

\fmtLemma{iru\-ma}%
\fmtOne{2}
\fmtCitation{(e)i\-ru\-ma}
\fmtPOS{n}%
\fmtNC{5}
\fmtSense{season for farming, sowing season}
\fmtExampleRL{Twi\-ge\-rei\-re ebi\-ro bya iru\-ma.}
\fmtExampleEN{We are approaching the sowing season.}

\fmtLemma{Iru\-ma}%
\fmtOne{3}
\fmtCitation{Iru\-ma}
\fmtPOS{n}%
\fmtNC{1a}
\fmtSense{Kabaka's landing site}
\fmtExampleRL{Ba\-ta\-nga\-rai\-ne oku mwa\-lo gwa Ka\-ba\-ka ogwe\-te\-bwa Iru\-ma.}
\fmtExampleEN{They met at Kabaka's landing site called Iruma.}

\fmtLemma{iru\-ngu}%
\fmtOne{1}
\fmtCitation{(e)i\-ru\-ngu}
\fmtPOS{n}%
\fmtNC{5/6}
\fmtSense{desert}
\fmtExampleRL{Ki\-zi\-bu oku\-lu\-ma ebi\-too\-ke omu iru\-ngu.}
\fmtExampleEN{It is hard to grow matoke in the desert.}

\fmtLemma{Iru\-ngu}%
\fmtOne{2}
\fmtCitation{(o)i\-ru\-ngu}
\fmtPOS{n}%
\fmtNC{1a}
\fmtSense{type of spirit responsible for success in hunting and riches}

\fmtLemma{irwa\-li\-ro}
\fmtCrossreference{\textnormal{see} irwaro}

\fmtLemma{irwa\-ro}
\fmtCitation{(e)i\-rwa\-ro}
\fmtPOS{n}%
\fmtNC{5/6}
\fmtSense{hospital, ward}
\fmtExampleRL{Ba\-mu\-twai\-re mu irwa\-ro lya Na\-ka\-so\-ngo\-la.}
\fmtExampleEN{He has been taken to Nakasongola hospital.}

\fmtLemma{irya}
\fmtCitation{(okw)i\-rya}
\fmtPOS{v}%
\fmtTrans{tr}
\fmtPFV{irye}
\fmtSubentry(a)
\fmtSense{to return, to put back, to take back}
\fmtExampleRL{Irya enkai\-to zaa\-mwe ma\-ngu\-ma\-ngu.}
\fmtExampleEN{Return his shoes immediately.}
\fmtSubentry(b)
\fmtFixedexpression{irya (o)bu\-ku\-ku}
\fmtSense{to go back to past grudges}
\fmtExampleRL{OBa\-li\-mwe\-za na\-mu\-ko\-be\-re\-ku, n’a\-nyi\-rii\-rya obu\-ku\-ku.}
\fmtExampleEN{I warned Balimweza, but he kept going back to his past grudges.}
\fmtSubentry(c)
\fmtFixedexpression{irya (o)bwe\-nku\-lu\-mu}
\fmtSense{to regurgitate}
\fmtExampleRL{Embu\-li zi\-mwe oku bi\-so\-lo ebii\-rya obwe\-nku\-lu\-mu.}
\fmtExampleEN{Goats are one of the animals which regurgitate.}
\fmtSubentry(d)
\fmtFixedexpression{irya (a)ma\-gi\-ta}
\fmtSense{to perform the ceremony where the bride goes home to officially visit her relatives after the wedding}
\fmtExampleRL{Omu\-ka\-li aya\-bi\-re kwi\-rya\-yo ma\-gi\-ta.}
\fmtExampleEN{The woman has gone back home to officially visit her relatives after marriage.}
\fmtSubentry(e)
\fmtFixedexpression{irya (o)mu\-sa\-ngo}
\fmtSense{to commit an offence, to commit a crime, to break the law}
\fmtExampleRL{Lwa\-ki okwi\-rya emi\-sa\-ngo bu\-li ka\-see\-ra?}
\fmtExampleEN{Why do you keep committing offences every now and then?}

\fmtLemma{irya\-ku}
\fmtCitation{(okw)i\-rya\-ku}
\fmtPOS{v}%
\fmtTrans{tr}
\fmtPFV{irye\-ku}
\fmtSubentry(a)
\fmtSense{to do next}
\fmtExampleRL{Ni na\-ma\-la e wa\-nge, nkwi\-rya\-ku waa\-mu.}
\fmtExampleEN{When I finish working at my home, I will work at yours next.}
\fmtSubentry(b)
\fmtSense{to have (a further child)}
\fmtExampleRL{Omwa\-na tu\-mwi\-rye\-ku.}
\fmtExampleEN{Let us have another child, please.}
\fmtSubentry(c)
\fmtSense{to remarry, to marry again, to marry (another woman)}
\fmtExampleRL{Omu\-ka\-li oku\-mwi\-rya\-ku di?}
\fmtExampleEN{When will you remarry?}

\fmtLemma{irya\-mu}
\fmtCitation{(okw)i\-rya\-mu}
\fmtPOS{v}%
\fmtTrans{tr}
\fmtPFV{irye\-mu}
\fmtSubentry(a)
\fmtSense{to refill, to replace}
\fmtExampleRL{Irya\-mu amai\-zi omu ki\-do\-mo\-la.}
\fmtExampleEN{Refill the jerry can with water.}
\fmtSubentry(b)
\fmtSense{to replant}
\fmtExampleRL{Ebi\-duu\-ma bya\-nge bya\-gai\-ne oku\-me\-ra, nkwi\-rya\-mu bi\-ndi.}
\fmtExampleEN{My maize did not germinate well, so I am replanting.}
\fmtSubentry(c)
\fmtFixedexpression{irya\-mu (a)maa\-ni}
\fmtSense{to encourage, to support}
\fmtExampleRL{Aba\-wa\-gi\-zi bai\-rya\-mu amaa\-ni etii\-mu yaa\-bwe.}
\fmtExampleEN{The fans encourage their team.}

\fmtLemma{irya\-wo}
\fmtCitation{(okw)i\-rya\-wo}
\fmtPOS{v}%
\fmtTrans{tr}
\fmtPFV{irye\-wo}
\fmtSubentry(a)
\fmtSense{to shut, to close}
\fmtExampleRL{Ai\-za eka nai\-rya\-wo olwi\-ge.}
\fmtExampleEN{He comes home and shuts the door.}
\fmtSubentry(b)
\fmtSense{to restore, to revive, to bring back}
\fmtExampleRL{Twi\-rye\-wo emi\-rwa, enge\-so n’e\-mi\-kii\-ro\-kii\-ro gyai\-swe.}
\fmtExampleEN{Let us revive our norms, values and traditions.}

\fmtLemma{irya\-yo}
\fmtCitation{(okw)i\-rya\-yo}
\fmtPOS{v}%
\fmtTrans{tr}
\fmtPFV{irye\-yo}
\fmtSense{to take back, to return, to put back, to pay back}
\fmtExampleRL{Irya\-yo omu\-kee\-ka ogwo omu nyu\-mba.}
\fmtExampleEN{Take that mat back to the house.}

\fmtLemma{isa}
\fmtCitation{(e)i\-sa}
\fmtPOS{n}%
\fmtNC{5}
\fmtSense{(of a person) category, kind, type}
\fmtExampleRL{Nku\-te mu isa ki?}
\fmtExampleEN{What kind of a person are you?}

\fmtLemma{isaa\-ba\-tuu\-bi}
\fmtCitation{(o)i\-saa\-ba\-tuu\-bi}
\fmtPOS{n}%
\fmtNC{1a/2}
\fmtSense{glutton}
\fmtExampleRL{Oyo omu\-sai\-za isaa\-ba\-tuu\-bi.}
\fmtExampleEN{That man is a glutton.}

\fmtLemma{isaa\-mi\-ro}
\fmtCitation{(e)i\-saa\-mi\-ro}
\fmtPOS{n}%
\fmtNC{5/6}
\fmtSense{bark cloth making workshop}
\fmtExampleRL{Ka\-twa\-be omu isaa\-mi\-ro nku\-saa\-mi\-re\-yo olu\-bu\-go.}
\fmtExampleEN{Let us go to the bark cloth making workshop and I will make bark cloth.}

\fmtLemma{isaa\-nii\-ko}
\fmtCitation{(e)i\-saa\-nii\-ko}
\fmtPOS{n}%
\fmtNC{5}
\fmtSense{used banana leaves}
\fmtExampleRL{Ei\-saa\-nii\-ko li\-ni li\-na\-ge omu lu\-su\-ku.}
\fmtExampleEN{Throw away the used banana leaves at the banana plantation.}

\fmtLemma{isaa\-si}
\fmtCitation{(e)i\-saa\-si}
\fmtPOS{n}%
\fmtNC{5}
\fmtSense{pity, mercy}
\fmtExampleRL{OYe\-su agi\-ra ei\-saa\-si oku ba\-ntu aba\-ku\-bo\-naa\-bo\-na.}
\fmtExampleEN{Jesus has mercy upon people who are suffering.}

\fmtLemma{isaa\-yi\-rii\-ryo}
\fmtCitation{(e)i\-sai\-ri\-rii\-ryo}
\fmtPOS{n}%
\fmtNC{5/6}
\fmtSense{processing plant}
\fmtExampleRL{Opa\-mba tu\-mu\-twa\-le omu isaa\-yi\-rii\-ryo ba\-mu\-ko\-le\-mu enko\-si.}
\fmtExampleEN{Let us bring the cotton to the processing plant and they will make it into fibre.}

\fmtLemma{isaa\-zi}
\fmtCitation{(e)i\-saa\-zi}
\fmtPOS{n}%
\fmtNC{5/6}
\fmtSense{milking parlour}
\fmtExampleRL{Ya\-ba omu isaa\-zi ondee\-te\-re\-yo ama\-tai.}
\fmtExampleEN{Go to the milking parlour and bring me some milk.}

\fmtLemma{isa\-ba\-guu\-da}
\fmtCitation{(o)i\-sa\-ba\-guu\-da}
\fmtPOS{n}%
\fmtNC{1a/2}
\fmtSense{richest person}
\fmtExampleRL{Oi\-sa\-ba\-guu\-da wa ku\-ni ni\-ye naa\-ni?}
\fmtExampleEN{Who is the richest person here?}

\fmtLemma{isa\-ba\-ka\-bo\-na}
\fmtCitation{(o)i\-sa\-ba\-ka\-bo\-na}
\fmtPOS{n}%
\fmtNC{1a/2}
\fmtSense{high priest, archpriest}
\fmtExampleRL{Oisa\-ba\-ka\-bo\-na akwi\-za olwa\-lee\-ro oku\-taa\-ku abaa\-na emi\-ko\-no.}
\fmtExampleEN{The high priest is coming today to carry out the children's confirmation.}

\fmtLemma{isa\-ba\-la\-bi\-ri\-zi}
\fmtCitation{(o)i\-sa\-ba\-la\-bi\-ri\-zi}
\fmtPOS{n}%
\fmtNC{1a/2}
\fmtSense{archbishop (of the Church of Uganda)}
\fmtExampleRL{Oi\-sa\-ba\-la\-bi\-ri\-zi ni\-ye akuu\-ra aba\-la\-bi\-ri\-zi.}
\fmtExampleEN{The Archbishop is the head of all the bishops.}

\fmtLemma{isa\-ba\-loo\-lee\-ri}
\fmtCitation{(o)i\-sa\-ba\-loo\-lee\-ri}
\fmtPOS{n}%
\fmtNC{1a/2}
\fmtSense{archbishop (of the Church of Uganda)}

\fmtLemma{Isa\-ba\-nya\-la}
\fmtCitation{(o)Isa\-Ba\-nya\-la}
\fmtPOS{n}%
\fmtNC{1a/2}
\fmtSense{Isabanyala (the traditional leader of Banyala)}
\fmtExampleRL{OIsa\-ba\-nya\-la li\-mwei oku ma\-ba\-ra ga mu\-ka\-ma wa Bu\-nya\-la.}
\fmtExampleEN{Isabanyala is one of the names given to the cultural leader of the Banyala.}

\fmtLemma{Isa\-ba\-ruu\-li}
\fmtCitation{(o)Isa\-ba\-ruu\-li}
\fmtPOS{n}%
\fmtNC{1a/2}
\fmtSense{Isabaruuli (the traditional leader of Baluuri)}
\fmtExampleRL{OIsa\-ba\-ruu\-li li\-mwei oku ma\-ba\-ra ga mu\-ka\-ma wa Bu\-ruu\-li.}
\fmtExampleEN{Isabaruuli is one of the names given to the cultural leader of the Baruuli.}

\fmtLemma{isa\-ba\-wa\-ndii\-ki}
\fmtCitation{(o)i\-sa\-ba\-wa\-ndii\-ki}
\fmtPOS{n}%
\fmtNC{1a/2}
\fmtSense{secretary general}
\fmtExampleRL{OBa\-la\-ba ni\-ye isa\-ba\-wa\-ndii\-ki wa Bu\-ka\-ma bwa Bu\-nya\-la.}
\fmtExampleEN{Balaba is the secretary general of the Bunyala kingdom.}

\fmtLemma{isa\-ga}
\fmtCitation{(e)i\-sa\-ga}
\fmtPOS{n}%
\fmtNC{5/6}
\fmtSense{twig}
\fmtExampleRL{Ku\-mbaa\-nia ama\-sa\-ga tu\-su\-mbe ocaa\-yi.}
\fmtExampleEN{Collect the twigs and we will make tea.}

\fmtLemma{isa\-ga\-ma}
\fmtCitation{(e)i\-sa\-ga\-ma}
\fmtPOS{n}%
\fmtNC{5}
\fmtSense{blood}
\fmtExampleRL{Aba\-ka\-ro\-mo\-ja ba\-lya ei\-sa\-ga\-ma lya nte.}
\fmtExampleEN{The Karamojong eat cow blood.}

\fmtLemma{isa\-ju}
\fmtCitation{(e)i\-sa\-ju}
\fmtPOS{n}%
\fmtNC{5/6}
\fmtSense{fat}
\fmtExampleRL{Abaa\-na ba\-ta\-ka mu\-no oku\-lya ama\-sa\-ju.}
\fmtExampleEN{Children like to eat fats very much.}

\fmtLemma{isa\-le}
\fmtCitation{(e)i\-sa\-le}
\fmtPOS{n}%
\fmtNC{5/6}
\fmtSense{sea shell}
\fmtExampleRL{Ama\-sa\-le gai\-ngi e Na\-ma\-sa\-le.}
\fmtExampleEN{There are many seashells at the Namasale landing site.}

\fmtLemma{isa\-ma\-ka}
\fmtCrossreference{\textnormal{see} isemaka}

\fmtLemma{isa\-mba}%
\fmtOne{1}
\fmtCitation{(o)i\-sa\-mba}
\fmtPOS{n}%
\fmtNC{1a/10}
\fmtSense{eagle}
\fmtExampleRL{Oi\-sa\-mba alii\-re akaa\-na ka nko\-ko.}
\fmtExampleEN{The eagle ate the chick.}

\fmtLemma{isa\-mba}%
\fmtOne{2}
\fmtCitation{(e)i\-sa\-mba}
\fmtPOS{n}%
\fmtNC{5/6}
\fmtSource{from Sw.}
\fmtSense{sha\-mba, plantation}
\fmtExampleRL{Nta n’ei\-sa\-mba lya njaa\-ye.}
\fmtExampleEN{I have a marijuana plantation.}

\fmtLemma{isa\-mbi\-ro}
\fmtCitation{(e)i\-sa\-mbi\-ro}
\fmtPOS{n}%
\fmtNC{5/6}
\fmtSense{playground, football pitch}
\fmtExampleRL{Ei\-sa\-mbi\-ro lya mu\-pii\-ra li\-ni li\-fu\-ndi.}
\fmtExampleEN{This football pitch is small.}

\fmtLemma{isa\-mbi\-ryo}
\fmtCitation{(e)i\-sa\-mbi\-ryo}
\fmtPOS{n}%
\fmtNC{5/6}
\fmtSense{spur (of a cock)}
\fmtExampleRL{Empa\-nga zi\-bba n’a\-ma\-sa\-mbi\-ryo oku ma\-gu\-lu.}
\fmtExampleEN{Cocks have spurs on their legs.}

\fmtLemma{Isa\-mbu\-la}
\fmtCitation{(e)Isa\-mbu\-la}
\fmtPOS{n}%
\fmtNC{9}
\fmtSense{July}
\fmtExampleRL{Ndii\-za omu kwe\-zi kwa Isa\-mbu\-la.}
\fmtExampleEN{I will come in July.}

\fmtLemma{isa\-mbya}
\fmtCitation{(e)i\-sa\-mbya}
\fmtPOS{n}%
\fmtNC{5}
\fmtSense{second rain season, second season for planting (which usually begins from around August)}
\fmtExampleRL{Nku\-bboo\-ka bi\-yaa\-ta bya isa\-mbya.}
\fmtExampleEN{I am planting potatoes for the second rainy season.}

\fmtLemma{isa\-mi}
\fmtCitation{(e)i\-sa\-mi}
\fmtPOS{n}%
\fmtNC{5}
\fmtSense{lake fly}
\fmtExampleRL{Ei\-sa\-mi li\-bba li\-ngi mu bi\-ro bya kya\-nda.}
\fmtExampleEN{There are too many lake flies during the dry season.}

\fmtLemma{isa\-na}%
\fmtOne{1}
\fmtCitation{(e)i\-sa\-na}
\fmtPOS{n}%
\fmtNC{5}
\fmtSubentry(a)
\fmtSense{sun, sunshine}
\fmtExampleRL{N’o\-lo\-la omu isa\-na ensaa\-le z’ai\-sa\-na zi\-kwi\-ta amai\-so.}
\fmtExampleEN{If you look directly into the sun, the rays will injure your eyes.}
\fmtSubentry(b)
\fmtSense{drought, dry spell}
\fmtExampleRL{Ei\-sa\-na ka lya\-ka\-li\-rii\-rye!}
\fmtExampleEN{The drought is excessive.}
\fmtSubentry(c)
\fmtFixedexpression{isa\-na li\-gwi\-re}
\fmtSense{at dusk}
\fmtExampleRL{Nai\-za nga ei\-sa\-na li\-gwi\-re.}
\fmtExampleEN{I will come at dusk.}
\fmtSubentry(d)
\fmtFixedexpression{isa\-na li\-mii\-re}
\fmtSense{at dusk}
\fmtExampleRL{Ya\-fii\-re nga ei\-sa\-na li\-mii\-re.}
\fmtExampleEN{He died at dusk.}
\fmtSubentry(e)
\fmtFixedexpression{isa\-na lya kya\-nda}
\fmtSense{dry season}
\fmtExampleRL{Omu ike\-ka tu\-bba n’ei\-sa\-na lya kya\-ndya.}
\fmtExampleEN{We are experiencing a dry season during the sweet potato slicing season.}
\fmtSubentry(f)
\fmtSense{hepatitis}
\fmtExampleRL{Oyo mu\-lwai\-re wa isa\-na.}
\fmtExampleEN{That person is suffering from hepatitis.}

\fmtLemma{isa\-na}%
\fmtOne{2}
\fmtCitation{(okw)i\-sa\-na}
\fmtPOS{v}%
\fmtTrans{tr}
\fmtPFV{isai\-ne}
\fmtSense{to look like, to be like, to resemble}
\fmtExampleRL{We oi\-sa\-na itee\-wo.}
\fmtExampleEN{You look like your dad.}

\fmtLemma{isa\-nga}
\fmtCitation{(e)i\-sa\-nga}
\fmtPOS{n}%
\fmtNC{5/6}
\fmtSense{tusk}
\fmtExampleRL{Onzo\-gi a\-bba n’a\-ma\-sa\-nga.}
\fmtExampleEN{The elephant has tusks.}

\fmtLemma{isa\-ni}
\fmtCitation{(e)i\-sa\-ni}
\fmtPOS{n}%
\fmtNC{5/6}
\fmtSense{statue, idol}
\fmtExampleRL{Ei\-sa\-ni lya Bi\-ki\-ra Ma\-ria li\-sai.}
\fmtExampleEN{The statue of Holy Mary is nice.}

\fmtLemma{isa\-nii\-sii\-rya}
\fmtCitation{(okw)i\-sa\-nii\-sii\-rya}
\fmtPOS{v}%
\fmtTrans{intr}
\fmtPFV{isa\-nii\-si\-rii\-rye}
\fmtSense{to look familiar, to look recognisable}
\fmtExampleRL{Eki\-fo ki\-ni nku\-ki\-sa\-nii\-sii\-rya.}
\fmtExampleEN{This place looks familiar to me.}

\fmtLemma{isa\-ni\-ka}
\fmtCitation{(okw)i\-sa\-ni\-ka}
\fmtPOS{v}%
\fmtTrans{tr}
\fmtPFV{isa\-ni\-ki\-re}
\fmtSense{to look like, to resemble}

\fmtLemma{isa\-nsa}
\fmtCitation{(e)i\-sa\-nsa}
\fmtPOS{n}%
\fmtNC{5/6}
\fmtSense{type of wild fruit which grows in a cluster and is red when ripe}
\fmtSense{type of wild cooking matoke}

\fmtLemma{isa\-nso}
\fmtCitation{(e)i\-sa\-nso}
\fmtPOS{n}%
\fmtNC{5/6}
\fmtSense{twig}
\fmtExampleRL{Ama\-sa\-nso ga mu\-saa\-le ge\-nde\-ke\-re.}
\fmtExampleEN{The tree's twigs are broken.}

\fmtLemma{isa\-nyu}
\fmtCitation{(e)i\-sa\-nyu}
\fmtPOS{n}%
\fmtNC{5}
\fmtSense{happiness, joy}
\fmtExampleRL{Ota\-sii\-sa isa\-nyu lyai\-swe.}
\fmtExampleEN{Don’t ruin our happiness.}

\fmtLemma{isa\-ra}
\fmtCitation{(e)i\-sa\-ra}
\fmtPOS{n}%
\fmtNC{5}
\fmtSense{stale local brew}
\fmtExampleRL{Ti\-nta\-ka ku\-nywa isa\-ra.}
\fmtExampleEN{I don’t like drinking stale local brew.}

\fmtLemma{isa\-sa}
\fmtCitation{(e)i\-sa\-sa}
\fmtPOS{n}%
\fmtNC{5/6}
\fmtSense{blacksmith's workshop}
\fmtExampleRL{Omu\-ka\-li ali omu kwe\-zi tai\-ki\-ri\-zi\-bwa kwi\-ka\-la omu isa\-sa.}
\fmtExampleEN{A woman who is menstruating is not allowed to sit in a blacksmith's workshop.}

\fmtLemma{isa\-si}
\fmtCitation{(e)i\-sa\-si}
\fmtPOS{n}%
\fmtNC{5/6}
\fmtSense{bullet}
\fmtExampleRL{Ba\-ku\-bbi\-re ongi\-ri ei\-sa\-si.}
\fmtExampleEN{They shot the warthog with a bullet.}

\fmtLemma{isa\-tu}
\fmtCrossreference{used in isolation and with class 9 nouns, see (-)satu}
\fmtCitation{(e)i\-sa\-tu}
\fmtPOS{num}
\fmtSense{three}
\fmtExampleRL{Na\-mba isa\-tu ku gu\-mo\-so.}
\fmtExampleEN{Number three is on the left hand.}

\fmtLemma{isa\-za}
\fmtCitation{(e)i\-sa\-za}
\fmtPOS{n}%
\fmtNC{5/6}
\fmtSense{county}
\fmtExampleRL{Na\-bya\-lii\-bwe omu isa\-za lya Bu\-ko\-ma\-nsi\-mbi.}
\fmtExampleEN{I was born in Bukomansimbi county.}

\fmtLemma{ise\-ba\-lo\-ngo}
\fmtCitation{(o)i\-se\-ba\-lo\-ngo}
\fmtPOS{n}%
\fmtNC{1a/2}
\fmtSense{father of twins}
\fmtExampleRL{Oi\-se\-ba\-lo\-ngo n’a\-fa, azi\-ki\-bwa igo\-loi\-go\-lo.}
\fmtExampleEN{When a father of twins dies, he is buried in the evening.}

\fmtLemma{ise\-bi\-ntu}
\fmtCitation{(o)i\-se\-bi\-ntu}
\fmtPOS{n}%
\fmtNC{1a/2}
\fmtSense{property owner, rich person}
\fmtExampleRL{OBu\-gu\-rwa ni\-ye ise\-bi\-ntu wa ku\-ni.}
\fmtExampleEN{Bugurwa is the property owner in this area.}

\fmtLemma{ise\-ma\-ka}
\fmtCitation{(o)i\-se\-ma\-ka}
\fmtPOS{n}%
\fmtNC{1a/2}
\fmtSense{family head}
\fmtExampleRL{Ise\-ma\-ka ye\-nkai ni\-ye ai\-ki\-ri\-zi\-bwa okwi\-kaa\-ra omu nte\-be.}
\fmtExampleEN{It is only the head of the family who is allowed to sit in the chair.}

\fmtLemma{ise\-ma\-kaa\-bi\-ri}
\fmtCitation{(o)i\-se\-ma\-kaa\-bi\-ri}
\fmtPOS{n}%
\fmtNC{1a/2}
\fmtSense{polygamous man}
\fmtExampleRL{Oi\-se\-ma\-kaa\-bi\-ri ta\-bba na mi\-re\-mbe.}
\fmtExampleEN{A polygamous man is never at peace.}

\fmtLemma{ise\-ma\-tee\-ka}
\fmtCitation{(o)i\-se\-ma\-tee\-ka}
\fmtPOS{n}%
\fmtNC{1a/2}
\fmtSense{constitution}
\fmtExampleRL{Oi\-se\-ma\-tee\-ka wai\-swe asai\-ne akuu\-mi\-bwe ku\-sai.}
\fmtExampleEN{Our constitution should be well protected.}

\fmtLemma{ise\-nde\-mwa\-ki}
\fmtCitation{(o)i\-se\-nde\-mwa\-ki}
\fmtPOS{n}%
\fmtNC{1a/2}
\fmtSense{go-getter, achiever, high achiever, man/woman of action}
\fmtExampleRL{Abai\-se\-nde\-mwa\-ki ba\-bba ba\-wa\-ngu\-zi.}
\fmtExampleEN{Go-getters are always successful.}

\fmtLemma{ise\-nga}
\fmtCitation{(o)i\-se\-nga}
\fmtPOS{n}%
\fmtNC{1a/2}
\fmtSense{paternal aunt}
\fmtExampleRL{Aya\-bi\-re n’oi\-se\-nga waa\-mwe.}
\fmtExampleEN{She has gone with her paternal aunt.}

\fmtLemma{ise\-nku\-lu}
\fmtCitation{(o)i\-se\-nku\-lu}
\fmtPOS{n}%
\fmtNC{1a/2}
\fmtSense{leader, head}
\fmtExampleRL{Ni\-ye ise\-nku\-lu wa ki\-to\-ngo\-le kya bya\-bwo\-mi.}
\fmtExampleEN{He is the head of the health department.}

\fmtLemma{ise\-nte\-be}
\fmtCitation{(o)i\-se\-nte\-be}
\fmtPOS{n}%
\fmtNC{1a/2}
\fmtSense{chairperson, chairman, chairwoman}
\fmtExampleRL{Ni\-ye ise\-nte\-be wa di\-si\-tu\-li\-ki\-ti ya Isi\-ngi\-ro.}
\fmtExampleEN{He is the chairman of Isingiro district.}

\fmtLemma{ishi}
\fmtPOS{interj}
\fmtSense{yuck, ew, ick (expression of strong distaste or disgust)}
\fmtExampleRL{Ishi, to\-so\-bo\-la ku\-cu\-lee\-ra!}
\fmtExampleEN{Yuck, can't you settle down!}

\fmtLemma{isi\-bya\-la}
\fmtCitation{(o)i\-si\-bya\-la}
\fmtPOS{n}%
\fmtNC{1a/2}
\fmtSense{father-in-law}
\fmtExampleRL{Oi\-si\-bya\-la wa\-nge mu\-lwai\-re mu\-no.}
\fmtExampleEN{My father-in-law is very sick.}

\fmtLemma{isi\-ga}
\fmtCitation{(e)i\-si\-mba}
\fmtPOS{n}%
\fmtNC{5/6}
\fmtSense{planting season}

\fmtLemma{isi\-mba}%
\fmtOne{1}
\fmtCitation{(o)i\-si\-mba}
\fmtPOS{n}%
\fmtNC{1a/10}
\fmtSense{genet}
\fmtExampleRL{Oi\-si\-mba alii\-re onko\-ko wa\-nge.}
\fmtExampleEN{The genet has eaten my hen.}

\fmtLemma{isi\-mba}%
\fmtOne{2}
\fmtCitation{(e)i\-si\-mba}
\fmtPOS{n}%
\fmtNC{5/6}
\fmtSense{planting season}
\fmtExampleRL{OMu\-se\-ve\-n ya\-bya\-lii\-bwe omu bi\-ro bya isi\-mba.}
\fmtExampleEN{Museven was born during the planting season.}

\fmtLemma{isi\-mbi}
\fmtCitation{(e)i\-si\-mbi}
\fmtPOS{n}%
\fmtNC{9/10}
\fmtSense{bedroom}
\fmtExampleRL{Ba\-kwi\-ra\-mu oku\-saa\-yii\-rya ei\-si\-mbi.}
\fmtExampleEN{They are redecorating the bedroom.}

\fmtLemma{Isi\-ra\-yi\-ri}
\fmtCitation{Isi\-ra\-yi\-ri}
\fmtPOS{n}%
\fmtNC{9}
\fmtSense{Israel}
\fmtExampleRL{OMu\-sa yaa\-kai\-rye abaa\-na ba Isi\-ra\-yi\-ri omu nya\-nja\-gi\-koo\-to eye\-te\-bwa Re\-d sea.}
\fmtExampleEN{Moses led the Israelites through the Red Sea.}

\fmtLemma{isi\-re}
\fmtPOS{adv}
\fmtSense{nonsensically}
\fmtExampleRL{Omu\-sai\-za oyo aba\-za isi\-re bu\-li ka\-see\-ra.}
\fmtExampleEN{That man always talks nonsensically.}

\fmtLemma{isi\-sye}
\fmtCitation{(o)i\-si\-sye}
\fmtPOS{n}%
\fmtNC{1a}
\fmtSense{type of water weed which grows underneath water and is breeding ground for small fish}

\fmtLemma{iso\-goo\-ro}
\fmtCitation{(e)i\-so\-goo\-ro}
\fmtPOS{n}%
\fmtNC{5/6}
\fmtSense{brewing container}
\fmtExampleRL{Ei\-so\-goo\-ro lya\-nge lii\-zwi\-re ensa\-nde.}
\fmtExampleEN{My brewing container is full of juice.}

\fmtLemma{isoi\-gi}
\fmtCitation{(e)i\-soi\-gi}
\fmtPOS{n}%
\fmtNC{5}
\fmtSense{dry banana leaf}
\fmtExampleRL{Aba\-sa\-gi\-ki ba NRM ba\-zwa\-la ei\-soi\-gi nka aba\-syo\-li.}
\fmtExampleEN{NRM supporters wear dry banana leaves like night dancers.}

\fmtLemma{iso\-ke}
\fmtCitation{(e)i\-so\-ke}
\fmtPOS{n}%
\fmtNC{5}
\fmtSubentry(a)
\fmtSense{hair}
\fmtExampleRL{Ge\-mba ei\-so\-ke oku mu\-twe ogwa mwa\-na.}
\fmtExampleEN{Shave off the child's hair.}
\fmtSubentry(b)
\fmtFixedexpression{iso\-ke lya bi\-ra\-ndi}
\fmtSense{artificially curled hair}
\fmtExampleRL{Omu\-ka\-li wa\-nge ata n’ei\-so\-ke lya bi\-ra\-ndi.}
\fmtExampleEN{My wife has curled hair.}

\fmtLemma{iso\-mba}
\fmtCitation{(o)i\-so\-mba}
\fmtPOS{n}%
\fmtNC{1a/10}
\fmtSubentry(a)
\fmtSense{giant pouched rat}
\fmtExampleRL{Ei\-so\-mba zi\-ma\-lii\-re ebi\-ko\-po.}
\fmtExampleEN{Giant pouched rats have carried away all my cups.}
\fmtSubentry(b)

\fmtLemma{iso\-me\-ro}
\fmtCitation{(e)i\-so\-me\-ro}
\fmtPOS{n}%
\fmtNC{5/6}
\fmtSense{school}
\fmtExampleRL{Ayo\-mbo\-ke\-re ei\-so\-me\-ro lya lu\-ka\-le.}
\fmtExampleEN{He built a community school.}

\fmtLemma{iso\-mo}
\fmtCitation{(e)i\-so\-mo}
\fmtPOS{n}%
\fmtNC{5/6}
\fmtSense{subject, lesson}
\fmtExampleRL{Mu\-kwi\-ra mu iso\-mo ki?}
\fmtExampleEN{Which lesson are you going to?}

\fmtLemma{iso\-nge\-ryo}
\fmtCitation{(e)i\-so\-nge\-ryo}
\fmtPOS{n}%
\fmtNC{5/6}
\fmtSense{canine tooth}
\fmtExampleRL{Ongi\-ri oni mu\-ku\-lu, ali\-na ei\-so\-nge\-ryo li\-na\-mu\-lai.}
\fmtExampleEN{This warthog is so old, it has a long canine tooth.}

\fmtLemma{iso\-ngi\-no}
\fmtCitation{(e)i\-so\-ngi\-no}
\fmtPOS{n}%
\fmtNC{5/6}
\fmtSense{hip}
\fmtExampleRL{ONa\-ncwe\-zi ata n’a\-ga\-so\-ngi\-no aga\-koo\-to.}
\fmtExampleEN{Nancwezi has big hips.}

\fmtLemma{iso\-ngo\-lyo}
\fmtCrossreference{\textnormal{see} isongeryo}

\fmtLemma{iso\-nko}
\fmtCitation{(e)i\-so\-nko}
\fmtPOS{n}%
\fmtNC{5/6}
\fmtSense{snail shell}
\fmtExampleRL{Eki\-ri\-ŋa\-li\-ŋa ki\-bba mu iso\-nko.}
\fmtExampleEN{A snail lives in a shell.}

\fmtLemma{iso\-roo\-ryo}
\fmtCitation{(e)i\-so\-roo\-ryo}
\fmtPOS{n}%
\fmtNC{5/6}
\fmtSense{collection centre (\eg coffee, pineapple)}
\fmtExampleRL{Ei\-so\-roo\-ryo lya mwa\-ni li\-ri Kye\-ri\-ma.}
\fmtExampleEN{The coffee collection centre is in Kyerima.}

\fmtLemma{isu\-bi}
\fmtCitation{(e)i\-su\-bi}
\fmtPOS{n}%
\fmtNC{5}
\fmtSense{grass}
\fmtExampleRL{Ente te\-zi\-kya\-li\-na isu\-bi olwa kya\-nda.}
\fmtExampleEN{The cows no longer have grass because of the drought.}

\fmtLemma{isu\-mbi\-ro}
\fmtCitation{(e)i\-su\-mbi\-ro}
\fmtPOS{n}%
\fmtNC{5/6}
\fmtSense{kitchen}
\fmtExampleRL{Ba\-li mu isu\-mbi\-ro ba\-ku\-lya me\-re.}
\fmtExampleEN{They are in the kitchen eating.}

\fmtLemma{isu\-mo}
\fmtCrossreference{\textnormal{see} isumu}

\fmtLemma{isu\-mu}
\fmtCitation{(e)i\-su\-mu}
\fmtPOS{n}%
\fmtNC{5/6}
\fmtSense{spear}
\fmtExampleRL{Mpee\-rya ei\-su\-mu eri\-ndi.}
\fmtExampleEN{Give me another spear.}

\fmtLemma{isu\-nsuu\-ro}
\fmtCitation{(e)i\-su\-nsuu\-ro}
\fmtPOS{n}%
\fmtNC{5/6}
\fmtSense{ginnery}
\fmtExampleRL{Opa\-mpa mu\-twa\-le oku isu\-nsuu\-ro nga obwi\-re bu\-kya\-li kwa\-ka\-la.}
\fmtExampleEN{Take the cotton to the ginnery before it is too late.}

\fmtLemma{isuu\-ka}%
\fmtOne{1}
\fmtCitation{(okw)i\-suu\-ka}
\fmtPOS{v}%
\fmtTrans{tr}
\fmtPFV{isuu\-ki\-re}
\fmtSense{to congratulate}
\fmtExampleRL{Omwa\-na oyo mwi\-suu\-kye emi\-so\-mo.}
\fmtExampleEN{Congratulate that child for completing her studies.}

\fmtLemma{isuu\-ka}%
\fmtOne{2}
\fmtCitation{(okw)i\-suu\-ka}
\fmtPOS{v}%
\fmtTrans{tr}
\fmtPFV{isuu\-ki\-re}
\fmtSense{to recover (from an illness), to get better}
\fmtExampleRL{Yai\-suu\-ki\-re obu\-lwai\-re.}
\fmtExampleEN{He recovered from the illness.}

\fmtLemma{isuu\-ka}%
\fmtOne{3}
\fmtCitation{(e)i\-suu\-ka}
\fmtPOS{n}%
\fmtNC{5/6}
\fmtSense{bed sheet}
\fmtExampleRL{Nguu\-re ama\-suu\-ka ga\-yaa\-ka.}
\fmtExampleEN{I bought new bed sheets.}

\fmtLemma{isuu\-ka\-yo}
\fmtPOS{interj}
\fmtSense{Welcome back!}
\fmtExampleRL{Isuu\-ka\-yo oku iso\-me\-ro.}
\fmtExampleEN{Welcome back from school!}

\fmtLemma{isuu\-kya}
\fmtCitation{(okw)\-ei\-suu\-kya}
\fmtPOS{v}%
\fmtTrans{tr}
\fmtPFV{eisuu\-ki\-rye}
\fmtSense{to congratulate}
\fmtExampleRL{Twa\-be twi\-suu\-kye omwa\-na oku\-fu\-na edi\-gu\-li.}
\fmtExampleEN{Let’s go and congratulate the graduate.}

\fmtLemma{isuu\-ya}
\fmtCitation{(e)i\-suu\-ya}
\fmtPOS{n}%
\fmtNC{5/6}
\fmtSense{prognathism (abnormally extended jaw)}
\fmtExampleRL{Omu\-sai\-za ali\-na ama\-suu\-ya aga\-koo\-to.}
\fmtExampleEN{The man has a bad case of prognathism.}

\fmtLemma{iswe}
\fmtPOS{pro}
\fmtSense{we, us (first person plural independent pronoun)}
\fmtExampleRL{Iswe tu\-ku\-ta\-ka kwe\-ga omu ki\-fo kya kwe\-cwa\-cwa\-na.}
\fmtExampleEN{As for us, we want to learn instead of going on strike.}

\fmtLemma{iswei\-gi}
\fmtCitation{(e)i\-swei\-gi}
\fmtPOS{n}%
\fmtNC{5/6}
\fmtSense{dry banana leaf}
\fmtExampleRL{Omu\-lwai\-re ya\-bba\-sya\-nga ansi oku iswei\-gi.}
\fmtExampleEN{The sick person could sleep on dried banana leaves.}

\fmtLemma{iswe\-na}
\fmtPOS{pro}
\fmtSubentry(a)
\fmtSense{all of us}
\fmtExampleRL{Iswe\-na tu\-kwi\-za.}
\fmtExampleEN{All of us are coming.}
\fmtSubentry(b)
\fmtSense{we too, we as well}
\fmtExampleRL{Iswe\-na tu\-lya ama\-too\-ke bu\-li ki\-ro.}
\fmtExampleEN{We too eat matoke every day.}

\fmtLemma{-isy}
\fmtPOS{sf}
\fmtSense{causative suffix (extension) on verbs}

\fmtLemma{isya}
\fmtCitation{(okw)i\-sya}
\fmtPOS{v}%
\fmtTrans{tr}
\fmtPFV{isi\-rye}
\fmtSense{to brew}
\fmtExampleRL{Ya\-yi\-si\-rye ama\-lwa aga\-sai aga ba\-ge\-ni.}
\fmtExampleEN{He brewed very nice finger millet brew for the visitors.}

\fmtLemma{isya\-du}
\fmtPOS{adv}
\fmtSense{scornfully, contemptuously, mockingly, in a sneering manner, in a superior manner}
\fmtExampleRL{Omwo\-jo oyo aba\-za isya\-du.}
\fmtExampleEN{That boy speaks scornfully.}

\fmtLemma{isye\-na}
\fmtPOS{pro}
\fmtSense{all of us}
\fmtExampleRL{Twa\-bbaa\-yo isye\-na.}
\fmtExampleEN{All of us will be there.}

\fmtLemma{isyo\-ma}
\fmtCitation{(e)i\-syo\-ma}
\fmtPOS{n}%
\fmtNC{5/6}
\fmtSense{creeping aerial yam}
\fmtExampleRL{Nei\-zi\-ra ma\-syo\-ma.}
\fmtExampleEN{I belong to the creeping aerial yam clan.}

\fmtLemma{ita}%
\fmtOne{1}
\fmtCitation{(okw)i\-ta}
\fmtPOS{v}%
\fmtTrans{tr}
\fmtPFV{iti\-re}
\fmtSubentry(a)
\fmtSense{to kill, to murder, to put to death}
\fmtExampleRL{Twi\-ta njo\-ka za ka\-suu\-me zo\-nkai.}
\fmtExampleEN{We only kill dangerous snakes.}
\fmtSubentry(b)
\fmtSense{to slaughter, to butcher, to kill, to slay}
\fmtExampleRL{Omu\-ge\-ni omu\-la\-ge oso\-bo\-la oku\-mwi\-ti\-ra ombu\-li.}
\fmtExampleEN{You can slaughter a goat for an invited guest.}
\fmtSubentry(c)
\fmtSense{to destroy}

\fmtLemma{ita}%
\fmtOne{2}
\fmtCitation{(okw)i\-ta}
\fmtPOS{v}%
\fmtTrans{tr}
\fmtPFV{iti\-re}
\fmtSense{to plant (sweet potato vines)}
\fmtExampleRL{Abaa\-la nywe\-na mwa\-be mu\-yi\-ti\-re ebi\-yaa\-ta.}
\fmtExampleEN{All of you girls, go and plant the potato vines.}

\fmtLemma{itaa\-gi}
\fmtCitation{(e)i\-taa\-gi}
\fmtPOS{n}%
\fmtNC{5/6}
\fmtSense{branch (\eg of a tree or organisation)}
\fmtExampleRL{Ei\-taa\-gi lya mu\-saa\-le lye\-nde\-ke\-re.}
\fmtExampleEN{The tree branch broke.}

\fmtLemma{itaa\-li}
\fmtCitation{(e)i\-taa\-li}
\fmtPOS{n}%
\fmtNC{5/6}
\fmtSense{maatali drums used primarily by Muslims at social and religious functions}
\fmtExampleRL{Aba\-si\-raa\-mu bu\-ku\-bba ama\-taa\-li oku mi\-ko\-lo.}
\fmtExampleEN{Moslems play mataali drums during functions.}

\fmtLemma{itaa\-nu}
\fmtCrossreference{used in isolation and with class 9 nouns, see (-)taanu}
\fmtCitation{(e)i\-taa\-nu}
\fmtPOS{num}
\fmtSense{five}

\fmtLemma{ita\-ba\-lo\-ngo}
\fmtCitation{(o)i\-ta\-ba\-lo\-ngo}
\fmtPOS{n}%
\fmtNC{1a/2}
\fmtSense{father of twins}
\fmtExampleRL{Oi\-ta\-ba\-lo\-ngo ya\-ka\-zwa ani aya\-bi\-re Ma\-sa\-ka.}
\fmtExampleEN{The father of the twins has just left, he has gone to Masaka.}

\fmtLemma{ita\-bi}
\fmtCitation{(e)i\-ta\-bi}
\fmtPOS{n}%
\fmtNC{5/6}
\fmtSense{branch, bough, subdivision or section (of a company or association)}
\fmtExampleRL{Omu\-saa\-le gu\-ni gu\-li\-ku ama\-ta\-bi gai\-ngi.}
\fmtExampleEN{This tree has many branches.}

\fmtLemma{ita\-faa\-li}
\fmtCitation{(e)i\-ta\-faa\-li}
\fmtPOS{n}%
\fmtNC{5/6}
\fmtSense{brick}
\fmtExampleRL{Lee\-ta ani ei\-ta\-faa\-li li\-mwei.}
\fmtExampleEN{Bring one brick here.}

\fmtLemma{ita\-ka\-li}
\fmtCitation{(e)i\-ta\-ka\-li}
\fmtPOS{n}%
\fmtNC{5}
\fmtSense{land, soil, ground}
\fmtExampleRL{Nguu\-re ei\-ta\-ka\-li obu\-kai\-re iku\-mi.}
\fmtExampleEN{I bought the land for ten million.}

\fmtLemma{ita\-ko}
\fmtCitation{(e)i\-ta\-ko}
\fmtPOS{n}%
\fmtNC{5/6}
\fmtSource{from Sw.}
\fmtSense{buttock, cheek}
\fmtExampleRL{Bo\-na omu\-sai\-za ali\-na ama\-ta\-ko aga\-koo\-to.}
\fmtExampleEN{Look at the man with big buttocks.}

\fmtLemma{ita\-ma}
\fmtCitation{(e)i\-ta\-ma}
\fmtPOS{n}%
\fmtNC{5/6}
\fmtSubentry(a)
\fmtSense{cheek}
\fmtExampleRL{Omwa\-la ata n’a\-ma\-ta\-ma ga\-koo\-to.}
\fmtExampleEN{The girl has big cheeks.}
\fmtSubentry(b)
\fmtSense{blade (of a spear)}
\fmtExampleRL{Ei\-su\-mu lya\-mwe li\-ri\-ku ama\-ta\-ma ga\-ga\-zi.}
\fmtExampleEN{His spear has big blades.}

\fmtLemma{ita\-mbi\-ro}
\fmtCitation{(e)i\-ta\-mbi\-ro}
\fmtPOS{n}%
\fmtNC{5/6}
\fmtSubentry(a)
\fmtSense{hospital, clinic}
\fmtExampleRL{Omu\-lwai\-re mu\-mu\-twa\-le omu ita\-mbi\-ro.}
\fmtExampleEN{Take the patient to hospital.}
\fmtSubentry(b)
\fmtSense{altar}
\fmtExampleRL{Omu\-buu\-li\-zi ya\-fee\-rei\-re ku ita\-mbi\-ro.}
\fmtExampleEN{The priest died at the altar.}

\fmtLemma{ita\-mii\-ro}
\fmtCitation{(e)i\-ta\-mii\-ro}
\fmtPOS{n}%
\fmtNC{5}
\fmtSense{drunkenness}
\fmtExampleRL{Ei\-ta\-mii\-ro lya\-mu\-su\-li\-si\-rye mu bbaa\-la.}
\fmtExampleEN{Drunkenness made him sleep in the bar.}

\fmtLemma{ita\-mu\-to}
\fmtCitation{(o)i\-ta\-mu\-to}
\fmtPOS{n}%
\fmtNC{1a/2}
\fmtSense{paternal uncle}
\fmtExampleRL{Na\-ku\-rii\-re wa ita mu\-to.}
\fmtExampleEN{I grew up in my paternal uncle's home.}

\fmtLemma{ita\-nga}
\fmtCitation{(e)i\-ta\-nga}
\fmtPOS{n}%
\fmtNC{5/6}
\fmtSense{tell-tale (a piece of fabric on a sailboat that shows the direction of the wind)}
\fmtExampleRL{Bbwa ei\-ta\-nga oku lya\-to tu\-so\-bo\-le oku\-ma\-nya ompu\-ngo gy’a\-ku\-li\-ga.}
\fmtExampleEN{Tie a tell-tale on the boat so that we know where the wind is blowing from.}

\fmtLemma{Ita\-nge}
\fmtCitation{(o)Ita\-nge}
\fmtPOS{n}%
\fmtNC{1a}
\fmtSense{Father, God}
\fmtExampleRL{Ka\-to\-nda otaa\-ta, Ka\-to\-nda mwa\-na ne Ka\-to\-nda omwo\-yo akwe\-ra, amii\-na.}
\fmtExampleEN{God the Father, the Son, and the Holy Spirit, amen.}

\fmtLemma{ita\-ngi}
\fmtCrossreference{\textnormal{see} itaagi}

\fmtLemma{ite}
\fmtCitation{(o)i\-te}
\fmtPOS{n}%
\fmtNC{1a/2}
\fmtSense{father}
\fmtExampleRL{Oi\-te wa\-nge ya\-li\-na ente bi\-nai.}
\fmtExampleEN{My father had four hundred cows.}

\fmtLemma{itee\-ka}
\fmtCitation{(e)i\-tee\-ka}
\fmtPOS{n}%
\fmtNC{5/6}
\fmtSense{law}
\fmtExampleRL{Ama\-tee\-ka ga Uga\-nda ba\-ga\-kyu\-sa\-kyu\-sa bu\-li bwi\-re.}
\fmtExampleEN{Uganda's laws are changed every now and then.}

\fmtLemma{itee\-wo}
\fmtCitation{(o)i\-tee\-wo}
\fmtPOS{n}%
\fmtNC{1}
\fmtSense{your father}
\fmtExampleRL{Ya\-ba oi\-mu\-le oi\-tee\-wo alye ekyai\-go\-lo.}
\fmtExampleEN{Go and wake up your father for supper.}

\fmtLemma{ite\-gu\-la}
\fmtCitation{(e)i\-te\-gu\-la}
\fmtPOS{n}%
\fmtNC{5/6}
\fmtSense{roofing tile}
\fmtExampleRL{Wa\-si\-gai\-re\-wo ei\-te\-gu\-la li\-mwei.}
\fmtExampleEN{There is one roofing tile left.}

\fmtLemma{ite\-mbo}
\fmtCitation{(e)i\-te\-mbo}
\fmtPOS{n}%
\fmtNC{5/6}
\fmtSense{young banana plantation}
\fmtExampleRL{Ngo\-di aku\-zu\-ba mu ite\-mbo lya\-mwe.}
\fmtExampleEN{He is there weeding his young banana plantation.}

\fmtLemma{ite\-nde\-ke\-ro}
\fmtCitation{(e)i\-te\-nde\-ke\-ro}
\fmtPOS{n}%
\fmtNC{5/6}
\fmtSense{training institute, college, university}
\fmtExampleRL{Omwa\-na aya\-bi\-re mu ite\-nde\-ke\-ro lya bu\-lu\-mi.}
\fmtExampleEN{The child has joined the Institute of Agriculture.}

\fmtLemma{ite\-nte}
\fmtCitation{(e)i\-te\-nte}
\fmtPOS{n}%
\fmtNC{5}
\fmtSense{stubbornness, obstinacy}
\fmtExampleRL{Omwa\-na wa ite\-nte ba\-mu\-ku\-bbi\-re.}
\fmtExampleEN{The stubborn child was beaten.}

\fmtLemma{ite\-ri}
\fmtCitation{(o)i\-te\-ri}
\fmtPOS{n}%
\fmtNC{1a}
\fmtSense{dayflower (\textit{Commelina})}
\fmtExampleRL{Oi\-te\-ri ya\-mee\-re ampi omu ki\-tu\-nga.}
\fmtExampleEN{A dayflower grew near the swamp.}

\fmtLemma{itii\-ra}
\fmtCitation{(okw)i\-tii\-ra}
\fmtPOS{v}%
\fmtTrans{tr}
\fmtPFV{iti\-rii\-re}
\fmtSense{to irrigate, to water}
\fmtExampleRL{Itii\-ra ebi\-ru\-me bya\-mu le\-ke obi\-fu\-ne\-mu.}
\fmtExampleEN{Irrigate your crops if you want to benefit from them.}

\fmtLemma{iti\-ki\-ra}
\fmtCitation{(okw)i\-ti\-ki\-ra}
\fmtPOS{v}%
\fmtTrans{intr}
\fmtPFV{iti\-kii\-re}
\fmtSense{to be thorough, to be efficient}
\fmtExampleRL{Ti\-ya\-li mwi\-ti\-ki\-re omu mu\-li\-mu gwa\-mwe.}
\fmtExampleEN{He was not thorough in his work.}

\fmtLemma{iti\-ki\-ri}%
\fmtOne{1}
\fmtCitation{\textnormal{(e.g.}\,kye\-ti\-ki\-ri, byee\-ti\-ki\-ri, lwe\-ti\-ki\-ri\textnormal{)}}
\fmtPOS{adj}
\fmtSense{real, true, original}
\fmtExampleRL{Ga\-ni ama\-tai ga\-ti\-ki\-ri ti bi\-cu\-pu\-li ebi\-ri oku ka\-ta\-le.}
\fmtExampleEN{It is real milk, not the fake one on the market.}

\fmtLemma{iti\-ki\-ri}%
\fmtOne{2}
\fmtCitation{\textnormal{(e.g.}\,mwi\-ti\-ki\-ri, bee\-ti\-ki\-ri\textnormal{)}}
\fmtPOS{adj}
\fmtSense{dark-skinned, swarthy}
\fmtExampleRL{Oi\-te wa\-nge mu\-sai\-za mwi\-ti\-ki\-ri.}
\fmtExampleEN{My father is a dark-skinned man.}

\fmtLemma{iti\-mba}
\fmtCitation{(o)i\-ti\-mba}
\fmtPOS{n}%
\fmtNC{1a/10}
\fmtSense{python}
\fmtExampleRL{Ya\-boi\-ne oi\-ti\-mba omu ki\-si\-ko ei\-go\-lo.}
\fmtExampleEN{He saw a python in the bush yesterday.}

\fmtLemma{iti\-nde}
\fmtCitation{(e)i\-ti\-nde}
\fmtPOS{n}%
\fmtNC{5/6}
\fmtSense{clod, lump of soil}
\fmtExampleRL{Omu\-lu\-mi aku\-te\-ma ma\-ti\-nde.}
\fmtExampleEN{The farmer is breaking clods of soil.}

\fmtLemma{iti\-ra}%
\fmtOne{1}
\fmtCitation{(okw)i\-ti\-ra}
\fmtPOS{v}%
\fmtTrans{intr}
\fmtPFV{itii\-re}
\fmtSense{to have a running stomach, to have an upset stomach}
\fmtExampleRL{Omwa\-na wa\-nge akwi\-ti\-ra.}
\fmtExampleEN{My child has a running stomach.}

\fmtLemma{iti\-ra}%
\fmtOne{2}
\fmtCitation{(okw)i\-ti\-ra}
\fmtPOS{v}%
\fmtTrans{intr}
\fmtPFV{itii\-re}
\fmtSense{to fumigate a container with grass}
\fmtExampleRL{Ekya\-nzi ba\-ki\-ti\-ra na ki\-cu\-nga.}
\fmtExampleEN{A milk gourd is cleaned with smoke from a small kiln.}

\fmtLemma{iti\-ra}%
\fmtOne{3}
\fmtCitation{(okw)i\-ti\-ra}
\fmtPOS{v}%
\fmtTrans{tr}
\fmtPFV{itii\-re}
\fmtSense{to cull, to thin out}
\fmtExampleRL{Obwi\-ta ni bu\-bba bwi\-ngi ba\-bwi\-ti\-ra.}
\fmtExampleEN{When there are too many millet plants in one place, they are thinned out.}

\fmtLemma{iti\-ra\-na}
\fmtCitation{(okw)i\-ti\-ra\-na}
\fmtPOS{v}%
\fmtTrans{intr}
\fmtPFV{iti\-rai\-ne}
\fmtSense{to meet face to face, to encounter}
\fmtExampleRL{Nyi\-ti\-rai\-ne n’o\-nyi\-bya\-la wa\-nge.}
\fmtExampleEN{I met my mother-in-law face-to-face.}

\fmtLemma{iti\-ri\-ga\-na}
\fmtCitation{(okw)i\-ti\-ri\-ga\-na}
\fmtPOS{v}%
\fmtTrans{intr}
\fmtPFV{iti\-ri\-gai\-ne}
\fmtSense{to lose control}
\fmtExampleRL{Owa boo\-da\-boo\-da yai\-ti\-ri\-gai\-ne epi\-ki\-pi\-ki.}
\fmtExampleEN{The bodaboda driver lost control.}

\fmtLemma{ito\-faa\-li}
\fmtCrossreference{\textnormal{see} itafaali}

\fmtLemma{itoi\-go}
\fmtCitation{(e)i\-toi\-go}
\fmtPOS{n}%
\fmtNC{5}
\fmtSense{rainy season which lasts from March to June, first planting season}
\fmtExampleRL{Ebi\-yaa\-ta tu\-bi\-bboo\-ka kwa itoi\-go.}
\fmtExampleEN{We grow sweet potatoes during the rainy season.}

\fmtLemma{ito\-ndo}
\fmtCitation{(e)i\-to\-ndo}
\fmtPOS{n}%
\fmtNC{5/6}
\fmtSense{drop, spot, blotch}
\fmtExampleRL{Ndoo\-wo wa\-de ei\-to\-ndo lya mai\-zi eree\-ro.}
\fmtExampleEN{There is not a single drop of water today.}

\fmtLemma{itoo\-ke}
\fmtCitation{(e)i\-too\-ke}
\fmtPOS{n}%
\fmtNC{5/6}
\fmtSense{banana}
\fmtExampleRL{Ba\-na\-yu\-ga\-nda ba\-ta\-ka mu\-no oku\-lya ama\-too\-ke.}
\fmtExampleEN{Ugandans really love eating matoke.}

\fmtLemma{itu\-bba}
\fmtCitation{(e)i\-tu\-bba}
\fmtPOS{n}%
\fmtNC{5/6}
\fmtSubentry(a)
\fmtSense{wooden plate}
\fmtExampleRL{Ama\-tu\-bba ga\-buu\-re ebi\-ro bi\-ni.}
\fmtExampleEN{Wooden plates are rare these days.}
\fmtSubentry(b)
\fmtSense{boat-shaped vessel used for making local brew}
\fmtExampleRL{Naa\-bya ei\-tu\-bba.}
\fmtExampleEN{Wash the vessel used for making the local brew.}

\fmtLemma{itu\-gu\-nda}
\fmtCitation{(e)i\-tu\-gu\-nda}
\fmtPOS{n}%
\fmtNC{5/6}
\fmtSense{\textit{Vangueria apiculata} (shrub or small tree with edible fruit)}
\fmtExampleRL{Ei\-tu\-gu\-nda ki\-ka kya ki\-ba\-la kya mu ki\-si\-ko eki\-ta ensi\-go egi\-gu\-mu.}
\fmtExampleEN{\textit{Vangueria apiculate} is a type of wild fruit with a very hard seed.}

\fmtLemma{itu\-ka}
\fmtCitation{(okw)i\-tu\-ka}
\fmtPOS{v}%
\fmtTrans{intr}
\fmtPFV{ituki\-re}
\fmtSense{to recur, to reappear, to happen repeatedly}
\fmtExampleRL{Obu\-lwai\-re bwa mu\-sai\-za oyo bwi\-tu\-ka bu\-li lu\-va\-nyu\-ma lwa kwe\-zi.}
\fmtExampleEN{That man’s illness recurs every month.}

\fmtLemma{itu\-ki\-rya}
\fmtCitation{(okw)i\-tu\-ki\-rya}
\fmtPOS{v}%
\fmtTrans{tr}
\fmtPFV{itu\-kii\-rye}
\fmtSense{to reopen (an investigation, inquiry, case), to revisit, to repeat, to revive}
\fmtExampleRL{Epo\-lii\-si etu\-ki\-rye omu\-sa\-ngo.}
\fmtExampleEN{The police have reopened the investigation.}

\fmtLemma{itu\-la}%
\fmtOne{1}
\fmtCitation{(okw)i\-tu\-la}
\fmtPOS{v}%
\fmtTrans{tr}
\fmtPFV{itwi\-re}
\fmtSubentry(a)
\fmtSense{to reopen (an investigation, inquiry, case), to revisit, to repeat, to revive}
\fmtExampleRL{Na\-bbai\-re mmai\-te nti omu\-sa\-ngo gwa\-wei\-re, ya\-gwi\-twi\-re bu\-yaa\-ka.}
\fmtExampleEN{I thought the case was closed, but he reopened it.}
\fmtSubentry(b)
\fmtSense{to exacerbate, to aggravate, to make worse}
\fmtExampleRL{Omwo\-jo yai\-twi\-re obu\-lwai\-re bwa maa\-mwe.}
\fmtExampleEN{The young man aggravated his mother's sickness.}

\fmtLemma{itu\-la}%
\fmtOne{2}
\fmtCitation{(okw)i\-tu\-la}
\fmtPOS{v}%
\fmtTrans{tr}
\fmtPFV{itwi\-re}
\fmtSense{to beat patterns (into bark cloth with a finely grooved mallet as the last step in making bark cloth)}
\fmtExampleRL{Itu\-la olu\-bu\-go olwo lu\-go\-nde.}
\fmtExampleEN{Beat patterns into bark cloth with a finely grooved mallet.}

\fmtLemma{itu\-la}%
\fmtOne{3}
\fmtCitation{(e)i\-tu\-la}
\fmtPOS{n}%
\fmtNC{5}
\fmtSense{type of mallet with small curves}
\fmtExampleRL{Nko\-le\-sya itu\-la okwi\-tu\-la.}
\fmtExampleEN{I use a mallet with small curves to soften the bark cloth.}

\fmtLemma{itu\-lu\-ku\-ka}
\fmtCitation{(okw)i\-tu\-lu\-ku\-ka}
\fmtPOS{v}%
\fmtTrans{intr}
\fmtPFV{itu\-lu\-ku\-ki\-re}
\fmtSubentry(a)
\fmtSense{(of water) to flow from (a container), to run out, to trickle down, to drip}
\fmtExampleRL{Amai\-zi ga\-kwi\-tu\-lu\-ku\-ka oku\-zwa omu ki\-do\-mo\-la.}
\fmtExampleEN{Water is running out of the jerry can.}
\fmtSubentry(b)
\fmtSense{(of tears) to flow}
\fmtSubentry(c)
\fmtSense{(of hair) to fall down}
\fmtSubentry(d)
\fmtSense{(of ghee, butter) to become clear}
\fmtExampleRL{Ama\-gi\-ta gai\-tu\-lu\-ku\-ki\-re ku\-sai mu\-no.}
\fmtExampleEN{The ghee has completely separated from the milk solids.}

\fmtLemma{itu\-lu\-la}
\fmtCitation{(okw)i\-tu\-lu\-la}
\fmtPOS{v}%
\fmtTrans{tr}
\fmtPFV{itu\-lwi\-re}
\fmtSense{to decant, to transfer, to tip, to pour into, to pour out, to drain out, to strain}
\fmtExampleRL{Akwi\-tu\-lu\-la amai\-zi omu ko\-nte\-yi\-na egi\-ndi.}
\fmtExampleEN{She is decanting the water into another container.}

\fmtLemma{itu\-ma}
\fmtCitation{(e)i\-tu\-ma}
\fmtPOS{n}%
\fmtNC{5}
\fmtSense{meanness, nastiness, unkindness}
\fmtExampleRL{Aba\-ntu bee\-yo ba\-ta n’ei\-tu\-ma.}
\fmtExampleEN{People from that area are bad-hearted.}

\fmtLemma{itu\-mbi}
\fmtCitation{(e)i\-tu\-mbi}
\fmtPOS{n}%
\fmtNC{5}
\fmtSense{midnight}
\fmtExampleRL{Ya\-ta\-ndi\-ki\-re oku\-ta\-na\-ka mu itu\-mbi.}
\fmtExampleEN{He started vomiting at midnight.}

\fmtLemma{itu\-ndi}
\fmtCitation{(e)i\-tu\-ndi}
\fmtPOS{n}%
\fmtNC{5}
\fmtSense{high demand}
\fmtExampleRL{Ati\-bu\-ni amai\-do ga\-li ku itu\-ndi.}
\fmtExampleEN{These days groundnuts are in high demand.}

\fmtLemma{itu\-ndi\-ro}
\fmtCitation{(e)i\-tu\-ndi\-ro rya ma\-tu\-fa\-t}
\fmtPOS{n}%
\fmtNC{5/6}
\fmtSense{petrol station, fuel station, gas station, filling station}
\fmtExampleRL{Ei\-tu\-ndi\-ro ly’a\-ma\-fu\-ta li\-kwai\-te omuu\-ro.}
\fmtExampleEN{The petrol station has caught fire.}

\fmtLemma{itu\-ndu\-baa\-li}
\fmtCitation{(e)i\-tu\-ndu\-baa\-li}
\fmtPOS{n}%
\fmtNC{5/6}
\fmtSense{tarpaulin, tarp}
\fmtExampleRL{Atu\-gu\-lii\-re ei\-tu\-ndu\-baa\-li erya ku\-le\-ja amai\-zi.}
\fmtExampleEN{He bought us a tarpaulin for tapping water.}

\fmtLemma{itu\-ndu\-ro}
\fmtCrossreference{\textnormal{see} isumu}

\fmtLemma{itu\-ngo}
\fmtCitation{(e)i\-tu\-ngo}
\fmtPOS{n}%
\fmtNC{5}
\fmtSubentry(a)
\fmtSense{riches, wealth}
\fmtExampleRL{Omwo\-jo aka\-nyi\-rye ei\-tu\-ngo.}
\fmtExampleEN{That boy has accumulated a lot of wealth.}
\fmtSubentry(b)
\fmtSense{domestic animal (cattle, sheep or goats)}
\fmtExampleRL{Omu\-sai\-za ali\-na ei\-tu\-ngo lya bi\-so\-lo.}
\fmtExampleEN{That man has many domestic animals.}

\fmtLemma{itwa\-le}
\fmtCitation{(e)i\-twa\-le}
\fmtPOS{n}%
\fmtNC{5/6}
\fmtSubentry(a)
\fmtSense{administrative unit in the Islamic faith at parish level}
\fmtExampleRL{Shei\-kh Mu\-sa ni\-ye akuu\-ra ei\-twa\-le lya Bu\-ko\-ma\-nsi\-mbi.}
\fmtExampleEN{Sheikh Musa is the one who leads the parish administrative unit for Muslims in Bukomansibi.}
\fmtSubentry(b)
\fmtSense{constituency}
\fmtExampleRL{Omu Ka\-yu\-nga mu\-li\-mu ama\-twa\-le asa\-tu.}
\fmtExampleEN{There are three constituencies in Kayunga.}
\fmtSubentry(c)
\fmtSense{colony, dependency, protectorate}
\fmtExampleRL{Ke\-nya lya\-li itwa\-le lya Bu\-nge\-re\-za.}
\fmtExampleEN{Kenya was a British colony.}

\fmtLemma{iwa}
\fmtCitation{(e)i\-wa}
\fmtPOS{n}%
\fmtNC{5/6}
\fmtSense{thorn, prickle, spike}
\fmtExampleRL{Ama\-wa ga\-nsu\-mi\-ti\-re nga nku\-zwa omu ndu\-mi\-ro.}
\fmtExampleEN{I was pricked by thorns on my way back from the garden.}

\fmtLemma{iwa\-ko}
\fmtCitation{(e)i\-wa\-ko}
\fmtPOS{n}%
\fmtNC{5/6}
\fmtSense{pregnancy (of an animal), conception}
\fmtExampleRL{Ombu\-li oyo ali\-na ei\-wa\-ko ti\-mu\-ku\-so\-bo\-la ku\-mu\-baa\-ga.}
\fmtExampleEN{That goat is pregnant, you cannot have it slaughtered.}

\fmtLemma{iwa\-ndii\-ki\-ro}
\fmtCitation{(e)i\-wa\-ndii\-ki\-ro}
\fmtPOS{n}%
\fmtNC{5/6}
\fmtSense{secretariat}
\fmtExampleRL{Ei\-wa\-ndii\-ki\-ro eri\-ku\-lu erya NRM li\-ri ku nzi\-ra ya Kya\-do\-ndo.}
\fmtExampleEN{The NRM secretariat is along the Kyadondo road.}

\fmtLemma{iwa\-ni\-ka}
\fmtCitation{(e)i\-wa\-ni\-ka}
\fmtPOS{n}%
\fmtNC{5/6}
\fmtSubentry(a)
\fmtSense{treasury}
\fmtExampleRL{Omu iwa\-ni\-ka lya Isa\-ba\-ruu\-li mu\-li\-mu empii\-ya zi\-ngi.}
\fmtExampleEN{There is a lot of money in the Isabaruuli's treasury.}
\fmtSubentry(b)
\fmtSense{mortuary, morgue}
\fmtExampleRL{Emi\-ra\-mbo egi\-ri omu iwa\-ni\-ka gi\-ku\-ju\-nda.}
\fmtExampleEN{The bodies in the mortuary are rotting.}

\fmtLemma{iwe\-se\-ro}
\fmtCitation{(e)i\-we\-se\-ro}
\fmtPOS{n}%
\fmtNC{5/6}
\fmtSense{metal fabrication workshop, blacksmith's workshop or plant}
\fmtExampleRL{Omu iwe\-se\-ro lya Mu\-ka\-la\-zi mu\-li\-mu enyi\-ge eza byo\-ma zi\-ngi.}
\fmtExampleEN{There is a number of metal doors in Mukalazi's metal fabrication workshop.}

\fmtLemma{iwo\-lu}
\fmtCitation{(e)i\-wo\-lu}
\fmtPOS{n}%
\fmtNC{5/6}
\fmtSense{leftovers}
\fmtExampleRL{Mpee\-rya oku iwo\-lu ndye\-ku.}
\fmtExampleEN{Give me the leftovers and I'll eat some.}

\fmtLemma{iwo\-ndo}
\fmtCitation{(e)i\-wo\-ndo}
\fmtPOS{n}%
\fmtNC{5/6}
\fmtSense{coconut-like wild fruit}
\fmtExampleRL{Enke\-mbo zi\-ta\-ka mu\-no oku\-lya ama\-wo\-ndo.}
\fmtExampleEN{Monkeys are fond of eating coconut-like wild fruit.}

\fmtLemma{iwoo\-bee\-si}
\fmtCitation{(e)i\-woo\-bee\-si}
\fmtPOS{n}%
\fmtNC{5/6}
\fmtSense{hook}
\fmtExampleRL{Ama\-wo\-bee\-si ga\-te\-ga ncwi.}
\fmtExampleEN{Hooks are used to catch fish.}

\fmtLemma{iwoo\-le}
\fmtCitation{(e)i\-woo\-le}
\fmtPOS{n}%
\fmtNC{5}
\fmtSense{stubbornness, obstinacy}
\fmtExampleRL{Ei\-woo\-le lya\-mwe lya\-mu\-lu\-gii\-re\-ku oku\-fee\-rwa omu\-lu\-mu.}
\fmtExampleEN{His stubbornness cost him his job.}

\fmtLemma{iwoo\-ryo}
\fmtCitation{(e)i\-woo\-ryo}
\fmtPOS{n}%
\fmtNC{5/6}
\fmtSense{court of justice}
\fmtExampleRL{Ya\-le\-ge\-re omu\-sa\-ngo omu iwoo\-ryo.}
\fmtExampleEN{He reported the case at the court of law.}

\fmtLemma{iwo\-woo\-lo}
\fmtCitation{(e)i\-wo\-woo\-lo}
\fmtPOS{n}%
\fmtNC{5/6}
\fmtSense{lung (of an animal)}
\fmtExampleRL{Nta\-ka mu\-no oku\-lya ama\-woo\-woo\-lo.}
\fmtExampleEN{I like eating animal lungs.}

\fmtLemma{iwu\-la}%
\fmtOne{1}
\fmtCitation{(okw)i\-wu\-la}
\fmtPOS{v}%
\fmtTrans{tr}
\fmtPFV{iwii\-re}
\fmtSense{to dig out, to scoop out, to excavate}
\fmtExampleRL{Yi\-wu\-la ei\-ta\-ka\-li oku\-zwa omu kii\-na tu\-bu\-mbe ebbu\-loo\-ka.}
\fmtExampleEN{Dig out the soil so that we can lay bricks.}

\fmtLemma{iwu\-la}%
\fmtOne{2}
\fmtCrossreference{\textnormal{see} ibura}

\fmtLemma{iwu\-li}
\fmtCitation{(e)i\-wu\-li}
\fmtPOS{n}%
\fmtNC{5/6}
\fmtSubentry(a)
\fmtSense{egg}
\fmtExampleRL{Onko\-ko ana\-gi\-re ama\-wu\-li gai\-ngi.}
\fmtExampleEN{The hen laid many eggs.}
\fmtSubentry(b)
\fmtFixedexpression{iwu\-li li\-zi\-rii\-rwa}
\fmtSense{unhatched egg}
\fmtExampleRL{Ei\-wu\-li eri\-zi\-rii\-rwa lya\-ti\-ki\-re.}
\fmtExampleEN{The unhatched egg cracked.}

\fmtLemma{iwu\-we}
\fmtCitation{(e)i\-wu\-we}
\fmtPOS{n}%
\fmtNC{5/6}
\fmtSense{lung (of a human being)}
\fmtExampleRL{Omu\-ntu a\-bba n’a\-ma\-wu\-we abi\-ri.}
\fmtExampleEN{A human being has two lungs.}

\fmtLemma{iya}%
\fmtOne{1}
\fmtCitation{(okw)i\-ya}
\fmtPOS{v}%
\fmtTrans{tr}
\fmtPFV{iyi\-re}
\fmtSubentry(a)
\fmtSense{to get, to obtain, to acqui\-re}
\fmtExampleRL{OKa\-wee\-si empii\-ya azii\-ya mu ku\-te\-ga bi\-na\-ge.}
\fmtExampleEN{Kaweesi gets money from fishing tilapia.}
\fmtSubentry(b)
\fmtSense{to take away, to remove}
\fmtExampleRL{Iya\-wo eki\-do\-mo\-la ki\-ni oki\-twa\-le omu nyu\-mba.}
\fmtExampleEN{Take away this jerry can and take it into the house.}
\fmtSubentry(c)
\fmtSense{to uproot, to root out}
\fmtExampleRL{Naa\-ni eya\-yii\-re endo\-kwai za\-nge oku\-zwa mu bbee\-di?}
\fmtExampleEN{Who uprooted my trees?}
\fmtSubentry(d)
\fmtSense{to harvest (by uprooting, \eg groundnut, mushroom)}
\fmtExampleRL{Baa\-bi\-re kwi\-ya mai\-do.}
\fmtExampleEN{They have gone to harvest groundnuts.}
\fmtSubentry(e)
\fmtSense{(in agriculture) to thin out, to trim}
\fmtExampleRL{Aya\-bi\-re kwi\-ya pa\-mba.}
\fmtExampleEN{He has gone to thin out the cotton plants.}
\fmtSubentry(f)
\fmtSense{to set free, to release}
\fmtExampleRL{Omu\-bbwe ba\-mwi\-yi\-re omu nko\-myo.}
\fmtExampleEN{The prisoner has been released from prison.}

\fmtLemma{iya}%
\fmtOne{2}
\fmtCitation{(okw)i\-ya}
\fmtPOS{v}%
\fmtTrans{tr}
\fmtPFV{iyi\-re}
\fmtSense{to toddle, to take the first step to walk}
\fmtExampleRL{Omwa\-na ata\-ndi\-ki\-re okwi\-ya ama\-gu\-lu.}
\fmtExampleEN{The baby has taken its first steps.}

\fmtLemma{iya}%
\fmtOne{3}
\fmtCitation{(e)i\-ya}
\fmtPOS{n}%
\fmtNC{5/6}
\fmtSense{shelf}
\fmtExampleRL{Ama\-duu\-ka ga\-bbaa\-mu ama\-ya.}
\fmtExampleEN{Shops have shelves.}

\fmtLemma{iya}%
\fmtOne{4}
\fmtCitation{(e)i\-ya}
\fmtPOS{n}%
\fmtNC{5/6}
\fmtSense{axe}
\fmtExampleRL{Ei\-ya lya mu\-bai\-zi lya\-buu\-re.}
\fmtExampleEN{The carpenter’s axe is lost.}

\fmtLemma{iyai}
\fmtPOS{interj}
\fmtSense{goodness (expression of surprise)}
\fmtExampleRL{Iyai! Omwa\-na anyi\-ru\-ki\-re\-ku.}
\fmtExampleEN{Goodness! The child has run away.}

\fmtLemma{iya\-ku}
\fmtCitation{(okw)i\-ya\-ku}
\fmtPOS{v}%
\fmtTrans{tr}
\fmtPFV{iyi\-re\-ku}
\fmtSubentry(a)
\fmtSense{to take away, to take part of, to take off a piece}
\fmtExampleRL{Yayi\-re\-ku ebi\-fi bi\-sa\-tu ebya nya\-ma n’a\-bi\-ga\-bi\-ra abaa\-li omu bwe\-taa\-vu.}
\fmtExampleEN{She took away three pieces of meat and shared them with the needy.}
\fmtSubentry(b)
\fmtSense{to remove (\eg a leader), to overthrow}
\fmtExampleRL{Baa\-mwi\-re\-ku ewoo\-fii\-si yaa\-mwe lwa bu\-nya\-ku\-li.}
\fmtExampleEN{He was removed from his office because of corruption.}

\fmtLemma{iya\-li}
\fmtCitation{(e)i\-ya\-li}
\fmtPOS{n}%
\fmtNC{5}
\fmtSubentry(a)
\fmtSense{envy}
\fmtExampleRL{Aba\-ntu aba\-ta n’ei\-ya\-li ti\-ba\-lya\-ba mu igu\-ru.}
\fmtExampleEN{The envious will never go to heaven.}
\fmtSubentry(b)
\fmtSense{jealousy}
\fmtExampleRL{Oi\-ba waa\-mwe ata n’ei\-ya\-li.}
\fmtExampleEN{Her husband is jealous. (lit.\,Her husband has jealousy.)}

\fmtLemma{iya\-mu}
\fmtCitation{(okw)i\-ya\-mu}
\fmtPOS{v}%
\fmtTrans{tr}
\fmtPFV{iyi\-re\-mu}
\fmtSense{to remove, to take off}
\fmtExampleRL{Iya\-mu enkai\-to zaa\-mu le\-ke oi\-ngi\-re omu\-zi\-gi\-ti.}
\fmtExampleEN{Remove your shoes and then enter the mosque.}

\fmtLemma{iya\-na}
\fmtCitation{(okw)i\-ya\-na}
\fmtPOS{v}%
\fmtTrans{intr}
\fmtPFV{iya\-ni\-re}
\fmtSense{to make oneself look important}
\fmtExampleRL{Lwa\-ki wee\-yi\-ya\-na?}
\fmtExampleEN{Why do you want to make yourself look important?}

\fmtLemma{iya\-nga}%
\fmtOne{1}
\fmtCitation{(e)i\-ya\-nga}
\fmtPOS{n}%
\fmtNC{5/6}
\fmtSubentry(a)
\fmtSense{indigenous group, ethnic group, race, tribe}
\fmtExampleRL{Uga\-nda eri\-na ama\-nga gai\-ngi.}
\fmtExampleEN{Uganda has many different indigenous groups.}
\fmtSubentry(b)
\fmtSense{nation, country, state}
\fmtExampleRL{Ei\-ya\-nga lya Ke\-nya lii\-zi\-re\-mu aka\-ca\-mbai\-tu\-ko.}
\fmtExampleEN{The country of Kenya is unstable.}

\fmtLemma{iya\-ngwe}
\fmtCitation{(e)i\-ya\-ngwe}
\fmtPOS{n}%
\fmtNC{5}
\fmtSense{noon, midday}
\fmtExampleRL{Lwa\-ki ota\-mbwi\-re omu iya\-ngwe?}
\fmtExampleEN{Why did you travel at noon?}

\fmtLemma{iya\-no}
\fmtCitation{(e)i\-ya\-no}
\fmtPOS{n}%
\fmtNC{5/6}
\fmtSubentry(a)
\fmtSense{taboo, culturally or socially unacceptable act}
\fmtExampleRL{Ei\-rai omu\-ka\-li oku\-lya ondu ya\-bba\-nga akoo\-re iya\-no.}
\fmtExampleEN{Long ago it was culturally inappropriate for a woman to eat a lungfish.}
\fmtSubentry(b)
\fmtSense{terrible event, catastrophe, tragedy}
\fmtExampleRL{Ga\-bba maa\-no oi\-te wa mwa\-na oku\-mu\-twe\-ka enda.}
\fmtExampleEN{It is a tragedy when a father impregnates his own daughter.}

\fmtLemma{iya\-wo}
\fmtCitation{(okw)i\-ya\-wo}
\fmtPOS{v}%
\fmtTrans{tr}
\fmtPFV{iyi\-re\-wo}
\fmtSense{to remove, to take away, to put away}
\fmtExampleRL{Iya\-wo ebi\-ta\-bo bya\-mu oku mee\-za.}
\fmtExampleEN{Remove your books from the table.}

\fmtLemma{iye}%
\fmtOne{1}
\fmtCitation{(e)i\-ye}
\fmtPOS{n}%
\fmtNC{5/6}
\fmtSense{army, regiment, armed force}
\fmtExampleRL{Ei\-ye lya Uga\-nda li\-te\-nde\-ke ku\-sai.}
\fmtExampleEN{Uganda's army is professionally trained.}

\fmtLemma{iye}%
\fmtOne{2}
\fmtPOS{interj}
\fmtSense{yes (expresses agreement)}
\fmtExampleRL{Iye, njee\-na nkwi\-za ku\-bbaa\-yo.}
\fmtExampleEN{Yes, I will also be there.}

\fmtLemma{iye\-mbe}%
\fmtOne{1}
\fmtCitation{(e)i\-ye\-mbe}
\fmtPOS{n}%
\fmtNC{5/6}
\fmtSense{fetish, spirit associated with a fetish, spell, hex}
\fmtExampleRL{Aba\-ntu aba\-mwe e Ka\-yu\-nga ba\-ta n’a\-me\-mbe.}
\fmtExampleEN{Some people in Kayunga have fetishes.}

\fmtLemma{iye\-mbe}%
\fmtOne{2}
\fmtCitation{(e)i\-ye\-mbe}
\fmtPOS{n}%
\fmtNC{5/6}
\fmtSense{horn}
\fmtExampleRL{Ame\-mbe ga nte wa\-nge ga\-so\-ngo\-li.}
\fmtExampleEN{My cow's horns are straight.}

\fmtLemma{iye\-ngo}
\fmtCitation{(e)i\-ye\-ngo}
\fmtPOS{n}%
\fmtNC{5/6}
\fmtSubentry(a)
\fmtSense{wave}
\fmtExampleRL{Ei\-ye\-ngo li\-ku\-bbi\-re akaa\-to ni kaa\-ti\-ka.}
\fmtExampleEN{The wave hit the canoe and it broke.}
\fmtSubentry(b)
\fmtSense{air wave}
\fmtExampleRL{Ama\-ra\-ka ga la\-di\-yo ga\-ka\-la mu ma\-ye\-ngo.}
\fmtExampleEN{Radio sounds are transmitted through air waves.}

\fmtLemma{iye\-rya}
\fmtCitation{(e)i\-ye\-rya}
\fmtPOS{n}%
\fmtNC{5/6}
\fmtSense{harvest, harvesting season}
\fmtExampleRL{Tu\-li mu iye\-rya lya me\-re.}
\fmtExampleEN{We are in the harvesting season.}

\fmtLemma{iyi\-mbe}
\fmtCrossreference{\textnormal{see} yimbe}

\fmtLemma{iyoo\-ro}
\fmtCitation{(e)i\-yoo\-ro}
\fmtPOS{n}%
\fmtNC{5/6}
\fmtSense{hardcore stone}
\fmtExampleRL{Ba\-lo\-nde\-re ama\-yoo\-ro oku\-zwa omu ki\-ro\-mbe.}
\fmtExampleEN{They picked hardcore stones from the stone quarry.}

\fmtLemma{iyo\-roo\-ro}
\fmtCitation{(e)i\-yo\-roo\-ro}
\fmtPOS{n}%
\fmtNC{5/6}
\fmtSense{nursery bed}
\fmtExampleRL{Ewai\-swe wa\-li\-yo ei\-yo\-roo\-ro lya mi\-cu\-ngwai.}
\fmtExampleEN{There is a nursery bed of orange trees in our area.}

\fmtLemma{iyu\-te}
\fmtCitation{(e)i\-yu\-te}
\fmtPOS{n}%
\fmtNC{5/6}
\fmtSense{boil, pustule}
\fmtExampleRL{Alwai\-re ei\-yu\-te oku ita\-ma.}
\fmtExampleEN{He has a boil on his cheek.}

\fmtLemma{iyuu\-ni}
\fmtCitation{(e)i\-yuu\-ni}
\fmtPOS{n}%
\fmtNC{5/6}
\fmtSense{yam}
\fmtExampleRL{Ama\-yuu\-ni ge ba\-su\-mbi\-re eree\-ro ti\-ga\-yii\-re.}
\fmtExampleEN{The yams they prepared today were not well cooked.}

\fmtLemma{iza}
\fmtCitation{(okw)i\-za}
\fmtPOS{v}%
\fmtTrans{intr}
\fmtPFV{izi\-re}
\fmtSubentry(a)
\fmtSense{to come, to appear, to show up}
\fmtExampleRL{Ai\-za bu\-li lwa Sa\-nde n’a\-nkya\-ra.}
\fmtExampleEN{Every Sunday, she comes here and visits me.}
\fmtSubentry(b)
\fmtSense{will, shall (used to build remote habitual future)}
\fmtExampleRL{Ebi\-ko\-lwa bya\-mwe ebya ki\-sa bi\-kwi\-za kwi\-zu\-ki\-rwa\-nga.}
\fmtExampleEN{His acts of kindness will always be remembered.}

\fmtLemma{iza\-za}
\fmtCitation{(e)i\-za\-za}
\fmtPOS{n}%
\fmtNC{5/6}
\fmtSense{shrine}
\fmtExampleRL{Ei\-za\-za lya Ba\-yi\-nda li\-ri ku Iru\-ba lya Na\-ki\-to\-ko\-lo.}
\fmtExampleEN{The Bayinda's shrine is found at Nakitokolo.}

\fmtLemma{izi\-moo\-ra}
\fmtCitation{(e)i\-zi\-moo\-ra}
\fmtPOS{n}%
\fmtNC{5/6}
\fmtSense{stretch mark}
\fmtExampleRL{Omu\-ka\-li oyo ali\-na ama\-zi\-moo\-ra oku nda.}
\fmtExampleEN{That woman has stretch marks on her stomach.}

\fmtLemma{izo}
\fmtCitation{(e)i\-zo}
\fmtPOS{adv}
\fmtSubentry(a)
\fmtSense{tomorrow, yesterday, day from now}
\fmtExampleRL{OJo\-y alii\-za izo.}
\fmtExampleEN{Joy will come tomorrow.}
\fmtSubentry(b)
\fmtFixedexpression{izo ma\-mbya}
\fmtSense{day after tomorrow, day before yesterday, two days from today}
\fmtExampleRL{Yai\-za izo ma\-mbya.}
\fmtExampleEN{He will come the day after tomorrow.}

\fmtLemma{izo\-bo\-ne}
\fmtCitation{(e)i\-zo\-bo\-ne}
\fmtPOS{n}
\fmtSense{colourfulness}
\fmtExampleRL{Emba\-ga izo\-bo\-ne ti lya\-ngu ku\-tee\-ka\-tee\-ka nga to\-li\-na se\-nte.}
\fmtExampleEN{A colourful wedding is not easy to organise when you do not have money.}

\fmtLemma{izu\-ba}%
\fmtOne{1}
\fmtCitation{(e)i\-zu\-ba}
\fmtPOS{n}%
\fmtNC{5/6}
\fmtSense{weed}
\fmtExampleRL{Ei\-zu\-ba lii\-ti\-re ebi\-duu\-ma.}
\fmtExampleEN{The weed was destroyed.}

\fmtLemma{izu\-ba}%
\fmtOne{2}
\fmtCitation{(e)i\-zu\-ba}
\fmtPOS{n}%
\fmtNC{5/6}
\fmtSense{well}
\fmtExampleRL{Ei\-zu\-ba lyai\-swe lya\-kaa\-re olwa kya\-nda.}
\fmtExampleEN{Our well has dried up because of the drought.}

\fmtLemma{izu\-ki\-ra}
\fmtCitation{(okw)i\-zu\-ki\-ra}
\fmtPOS{v}%
\fmtTrans{tr}
\fmtPFV{izu\-kii\-re}
\fmtSense{to remember, to recall}
\fmtExampleRL{Okwi\-zu\-ki\-ra kye na\-ku\-ko\-be\-re?}
\fmtExampleEN{Do you remember what I told you?}

\fmtLemma{izu\-ki\-rya}
\fmtCitation{(okw)i\-zu\-ki\-rya}
\fmtPOS{v}%
\fmtTrans{tr}
\fmtPFV{izu\-kii\-rye}
\fmtSense{to remind, to make one think of, to cause one to remember}
\fmtExampleRL{Taa\-yo ei\-ga\-ni okwi\-zu\-ki\-rya abaa\-na abo oku\-so\-ma.}
\fmtExampleEN{Make sure that you remind those children to study.}

\fmtLemma{izu\-ku\-rya}
\fmtCitation{(okw)i\-zu\-ku\-rya}
\fmtPOS{v}%
\fmtTrans{intr}
\fmtPFV{izu\-ku\-rye}
\fmtSense{to become a grandparent, to have grandchildren}
\fmtExampleRL{Ata\-ndi\-ki\-re okwi\-zu\-ku\-rya.}
\fmtExampleEN{He became a grand\-parent.}

\fmtLemma{izu\-la}
\fmtCitation{(okw)i\-zu\-la}
\fmtPOS{v}%
\fmtTrans{tr}
\fmtPFV{izwi\-re}
\fmtSense{to be full with, to be filled with}
\fmtExampleRL{Jaa\-ga ei\-zwi\-re amai\-zi.}
\fmtExampleEN{The jug is full of water.}

\fmtLemma{izu\-li}
\fmtCitation{\textnormal{(e.g.}\,kii\-zu\-li, gii\-zu\-li\textnormal{)}}
\fmtPOS{adj}
\fmtSense{full}
\fmtExampleRL{Mpa eki\-ko\-po kii\-zu\-li.}
\fmtExampleEN{Give me a full cup.}

\fmtLemma{izu\-lya}
\fmtCitation{(okw)i\-zu\-lya}
\fmtPOS{v}%
\fmtTrans{tr}
\fmtPFV{izwi\-rye}
\fmtSubentry(a)
\fmtSense{to fill, to make full, to charge}
\fmtExampleRL{Yai\-zwi\-rye ejaa\-ga ama\-tai.}
\fmtExampleEN{She filled the jug with milk.}
\fmtSubentry(b)
\fmtSense{to flood with, to overflow with, to become submerged with water}
\fmtExampleRL{OKa\-mpa\-la ai\-zu\-lya amai\-zi omu bi\-ro bya ike\-ndi.}
\fmtExampleEN{Kampala floods during the rainy season.}

\fmtLemma{izwe\-ra}
\fmtCitation{(okw)i\-zwe\-ra}
\fmtPOS{v}%
\fmtTrans{tr}
\fmtPFV{izwei\-re}
\fmtSense{(to lock with a security bar across the door}
\fmtExampleRL{Naa\-ni yai\-zwe\-ra olwi\-ge lwa\-nge?}
\fmtExampleEN{Who locked my door with a security bar?}

\fmtLemma{izwe\-ri}
\fmtCitation{(e)izwe\-ri}
\fmtPOS{adv}
\fmtSubentry(a)
\fmtSense{day after tomorrow, day before yesterday, two days from today}
\fmtExampleRL{Abaa\-na ba\-lya\-ba oku iso\-me\-ro izwe\-ri.}
\fmtExampleEN{The children will go to school the day after tomorrow.}
\fmtSubentry(b)
\fmtSense{recently, not long ago, a short time ago}
\fmtExampleRL{Yaa\-bi\-re izwe\-ri.}
\fmtExampleEN{He has left recently.}
\end{letter}

Vamale has a fine-tuned system of definiteness, using both definite and indefinite articles for specific reference, and the absence of articles for non-specific (i.e. less transitive) scenarios. This closed word class is called ``articles" in this grammar because its members precede nouns and mark them as (in)definite.
Vamale makes the wide-spread distinction between singular, dual and plural (see \Cref{tab:articles}), and what this study considers to be degrees of definiteness rather than specificity (an argument for this analysis follows in \sectref{ssec:WCArt_specificity}). Using a noun phrase without an article makes it non-specific, as seen in (\ref{ex:noArt2}) and (\ref{ex:Art2}). 

\ea\label{ex:noArt2}
\gll go=han can hnyimake thamo\\
 2\gl{sg}=walk \gl{subr} think woman\\
\glt \qu{You walk while thinking about a woman.} {[G4:22]}
\z


\ea\label{ex:Art2}
%\ili{}{}{} G4:22
\gll go=han can hnyimake i=thamo\\
 2\gl{sg}=walk \gl{subr} think \gl{def}.\gl{sg}=woman\\
\glt \qu{You walk while thinking about the woman.}
%\is{Subordination!Adverbial clause}
%\ea
%
%\gll e=xaleke apuli \textbf{a}=kon xhwi puaka 
% 1\gl{sg}=see man \gl{rel}=\gl{prog} eat pig
%\glt \qu{I see a man who is eating pork/a pig.}
%
%\a
%
%\gll e=xaleke apuli a=kon xhwi i=puaka 
% 1\gl{sg}=see man \gl{rel}=\gl{prog} eat \gl{def}.\gl{sg}=pig
%\glt \qu{I see a man who is eating the pig.}
%
\z

\begin{table}
	\caption{Articles in Vamale}
	\begin{tabular}{lll}
	\lsptoprule
		&\gl{spec} and \gl{def}& \gl{spec} and \gl{indf}\\\midrule
		\gl{sg} & \textit{i} & \textit{(e)ca} \\
		\gl{du} & \textit{mu} & \textit{muca}  \\
		\gl{pl} & \textit{li} / \textit{ni} & \textit{ca(been)} \\
	\lspbottomrule
	\end{tabular}
\label{tab:articles}
\end{table}

The article can combine with the stative verb \textit{se} \qu{one, same} and the noun \textit{been} \qu{peer} to mean \qu{the/some other}. This is further discussed in \sectref{sec:ise}.

%FZ: I am not sure this is the best way to put it -- do you mean "se" can be used predicatively (e.g. 'it is one') as well as attributively (like here, lit. 'the man who is one')? Does your theory of parts of speech work like that, viz. (suitable) verbs can occupy both syntactic(/pragmatic) slots? 
Vamale speakers can choose between using an article (the noun thus having a specific referent) (see \ref{ex:Art}), and not using an article (see \ref{ex:noArt}, \ref{ex:gen}), in which case the noun is generic. In oft-used expressions, the verb forms a compound with the noun.\footnote{See \textcite{ozanne-rivierre_verbal_2004} for a discussion of similar phenomena in other New Caledonian languages.}


\ea\label{ex:Art}
\gll e=xaleke i=apuli a=a=xhwi i=puaka\\
	1\gl{sg}=see \gl{def}.\gl{sg}=man	\gl{rel}=3\gl{sg}=eat	 \gl{def}.\gl{sg}=pig\\
\glt	\qu{I see the man who is biting the pig.}
\z

\ea\label{ex:noArt}
\gll		e=xaleke i=apuli 	a=a=xhwi puaka\\
		1\gl{sg}=see \gl{def}.\gl{sg}=man 	\gl{rel}=3\gl{sg}=eat pig\\
\glt		\qu{I see the man who is eating pork.} (not: the/a/some pig)
		\z

%\begin{comment}
%\a
%
%\gll e=xaleke apuli a=xhwi puaka	
%	}=see man	\gl{rel}=eat  pig
%\glt \qu{I see a man who is eating pork/a pig.}	
%
%\end{comment}

\ea\label{ex:gen}
\gll lu=xa-tena apuli\\
 3\gl{du}=\gl{hab}-understand person\\
\glt \qu{They understand people.}
\z



\subsection{The question of definiteness}
\label{ssec:WCArt_specificity}
\begin{sloppypar}
Vamale articles are split, as can be seen in \Cref{tab:articles}, not only according to their number, but along another axis, definiteness. 
%\subsection{The singular article \textit{i}.}	
 %who speaks Fwâi, Nemi, and understands Pije and Vamale. 
%I suggested a first translation, which was almost completely re-written by the workgroup. The entire text can be found in \Cref{text:beat}. 
Consider example (\ref{ex:t1}) from a French\hyp language legend, written by Yvonne Sahilé, and translated by the workgroup into Vamale. The example translates \qu{il y avait une tribu} (there was a tribe) using \textit{i} \qu{\gl{def}.\gl{sg}}, making it seem more specific than definite, since the listeners cannot yet be familiar with the tribe. My first suggestion was using an indefinite article \textit{eca}  (\ref{ex:wrongtipije}), which was refused. %The whole text is too long to be shown here, but see the document "conte". 
Now look at example (\ref{ex:t1}), where the tribe has never been mentioned before, compared to examples (\ref{ex:t2}) and (\ref{ex:t3}), where it has. All sentences use the same article. In (\ref{ex:t3}), \textit{li thamo} \qu{the women} are newly introduced to the narrative but marked with the definite \textit{li}. %It could be argued that there are always young women in a village, but the previous argument of saliency to the situation seems to justify an analysis as definite well enough.
\end{sloppypar}

\ea
\label{ex:wrongtipije}
%\ili{}{}{}  
\gll *Habu can vije vwa eca=ape-moo a= pwan jelan ehni i=jahoot-ca a= xhopwen.		\\
 long.ago in Tipije \gl{exist} \gl{indf}.\gl{sg}=\gl{loc}.\gl{nmlz}-stay \gl{rel}= on side \gl{dem}.\gl{prox}  \gl{def}.\gl{sg}=river-\gl{prox} \gl{rel}= big	\\
\glt  {[1, authors's attempt]}		
\z

\ea\label{ex:t1}
\gll 	Habu Can-Vije vwa \textbf{i}=apemoo a= pwa-n jela \textbf{i}=jahoot a= xhopwen.	\\
	long.ago Tipije \textit{exist} \gl{def}.\gl{sg}=tribe \gl{rel}= on-\gl{nspec} side \gl{def}.\gl{sg}=river \gl{rel}= big	\\
\glt  \qu{Long ago in the Tipije valley, there was a tribe on the bank of this great river.} (following the French original text)	{[1]}	
\z

\ea\label{ex:t2}
\gll 	Ca i=apemoo-ca le=vacuti \textbf{ca} daahma a= bwa xawe, ka yata-n Thêa Xa-vila	\\
	in \gl{def}.\gl{sg}=tribe-\gl{prox} 3\gl{pl}=erect some chief \gl{rel}= \gl{ipfv} young and name-3\gl{sg}.\gl{poss} T. \gl{nmlz}-dance	\\
\glt  \qu{In this tribe they erected a chief who was still young, and his name was Firstborn the Dancer.} (Thêa is a name commonly given to the firstborn son)	{[2]}	
\z

\ea\label{ex:t3}
\gll Ca \textbf{i}=apemoo vwa \textbf{li}=xawe thamo	\\
 in \gl{def}.\gl{sg}=tribe \gl{exist} \gl{def}.\gl{pl}=young woman	\\
\glt \qu{In the tribe were young women.} (indefinite, non-specific in the French original) {[3]}	
\z

This description analyzes \textit{i} \qu{\gl{def}.\gl{sg}} and \textit{li} \qu{\gl{def}.\gl{pl}} as definite articles, using \textcite{lyons_definiteness_1999}'s definition. Noun phrases marked with \textit{i} and \textit{li} are identifiable \parencite[1]{lyons_definiteness_1999}, albeit not necessarily familiar \parencite[3]{lyons_definiteness_1999}, see (\ref{ex:t2}) and (\ref{ex:t3}). Familiarity, to Lyons, is not an necessary feature for a construction to be definite \parencite[5]{lyons_definiteness_1999}. As Lyons discusses in the pages following that statement, the uniqueness of a referent, in total or relative to the context \parencite[8]{lyons_definiteness_1999}, or even ``the totality of the objects [\ldots] in the context which satisfy the description'' \parencite[11]{lyons_definiteness_1999}, can all be grounds for definiteness. 

Example (\ref{ex:t4-6}) describes a woman not previously mentioned, \textit{\textbf{eca} thamo a en maa-n} \qu{she who has a beautiful face}: indefinite but specific. Example (\ref{ex:t7}) describes a similar situation: In-Thu is a unique, identifiable character, not previously introduced. However, the relative clause modifying her is defining, and so preceded by \textit{i}, whereas the one in (\ref{ex:t4-6}) is not, and its modified noun takes \textit{eca}. %The use of definite \textit{i} instead of indefinite \textit{eca} may be a de-grammaticalized one, however: \textit{eca} is not attested before defining relative clauses, but can occur before an indefinite. 
Example (\ref{ex:t2}) seems to use \textit{ca} in an indefinite way, too, introducing a character who is then further specified and named. %But is it specific? They could have elected some chief, whoever he was, and only afterwards is it explained how he is called.
%\todo{zu spannend}

\ea\label{ex:t4-6}
\gll tha fe nyamaa-n ca-n dawee-le \textbf{eca} thamo a= en maa-n. ka a=xhani ma mwada-n.
Yata-n In Fwe\\
 \gl{ass}.3\gl{sg} take eye-3\gl{sg}.\gl{poss} in-\gl{nspec} between-3\gl{pl}.\gl{poss} \textbf{some} woman \gl{rel}= first face-3\gl{sg}.\gl{poss} and 3\gl{sg}=choose \gl{subr} wife-3\gl{sg}.\gl{poss} name-3\gl{sg}.\gl{poss} skin guettarda.speciosa \\
\glt \qu{Some woman among them \qu{caught his eye}, who was the most beautiful. And he chose her as his wife. Her name was Figtree Bark.}	{[GC:4-6]}
\z

\ea\label{ex:t7}%
(Note the \textit{i a yatan} construction, \qu{the who name-her})\\
\gll le=kiica ka meeka \textbf{li}=been thamo, ma ca-n e-dawee-le \textbf{i}=a= yata-n In Thu.	\\
 3\gl{pl}=jealous and all \gl{def}.\gl{pl}=other woman \gl{com} in-\gl{indf} \gl{mid}-between-3\gl{pl} \gl{def}.\gl{sg}=\gl{rel}= name-3\gl{sg}.\gl{poss} skin banyan\\
\glt \qu{All the other women were jealous, with among them all the one who was called Banyan Bark.} {[}GC:7]	
\z

%More examples from the same text:

%\ea
%\a
%
%\gll 	Kavi tha go xaleke i\textsubscript{def}=[e kon xaleke], na hmuun mae
%	but \gl{ass}=2\gl{sg} see \gl{def}.\gl{sg}=\gl{rel}.1\gl{sg} \gl{koon} see \gl{dem} smoke fire	
%\glt  \qu{But do you see what I'm seeing, that's firesmoke}		
%
%
%\a
%
%\gll 	Hmwakan vwa apuli\textsubscript{NSPEC} xada, gasu bo fwajimwa nyaanya
%	maybe \gl{exist} person up.there 1\gl{du}.\gl{incl} \gl{fut} ask mum	
%\glt  \qu{Maybe there are people up there, we'll ask Mum}	
%
%
%\a
%
%\gll 	Cala lu fwajimwe-a, a vwa e-vwaseekan ka In Fwe ko a caihna-n hapi na la i\textsubscript{def} xhoogo-nea		
%	when 3\gl{du} ask-3\gl{sg} 3\gl{sg} do \gl{ins}-sad \gl{sbj} I. F. because 3\gl{sg} know.\gl{inan} \gl{comp} \gl{dem} be.here \gl{def}.\gl{sg}=home-3\gl{sg}.\gl{poss}	
%\glt  \qu{When they asked her, Fig Bark was sad because she knew that this was her home.}		
%
%
%\a
%
%\gll 	Ka a sabe nya.sii-lu \qu{hmwakan vwa apuli\textsubscript{NSPEC} xada}	
%	and 3\gl{sg} lift.to.mouth give.hand-3\gl{du} maybe \gl{exist} people up.there	
%\glt  \qu{And she answered them ``maybe there are people up there"}		
%
%
%\a
%
%\ili{}{}{} Ungrammatical: *\textit{ca hmuu-n mae-aen}
%\gll 	Ca i wadan a e-thaloo-kan tha see ma i a vi, kavi a vi nyakolu hapi ca \textbf{eca} se wadan gase bo ta xaleke i hmuu-n mae-aen.		
%	in.\gl{def} \gl{def}.\gl{sg}=time \gl{rel} \gl{ord}-two-\gl{ord} \gl{ass}.3\gl{sg} cry \gl{com} \gl{def}.\gl{sg}=\gl{rel}.3\gl{sg} say, \gl{cnj} 3\gl{sg} say to-3\gl{du} that in some other time 1\gl{pl}.\gl{incl} \gl{fut} go.up see	\gl{def}.\gl{sg}=smoke-\gl{poss} fire \gl{dem}
%\glt  \qu{The second time she cried about what he said, but she told them that some other time(singular) we'll go up and/to look at that smoke.}		
%
%\z
%
%\ex
%
%\gll Ko i si-n thu cipa le xaleke \textbf{eca} mani a xhwatiin yata-n mathila ko a kon fwajimwake hapi na kai ni-ehni a le kon bwaa hma.	
% on \gl{def}.\gl{sg}=arm-\gl{poss} banyan \gl{neg} 3\gl{pl} see some bird \gl{rel} small name-3\gl{sg}.\gl{poss} M. but 3\gl{sg} \gl{koon} ask that \gl{dem}  who \gl{def}.\gl{pl}-\gl{prox}\gl{rel} 3\gl{pl} \gl{koon} \gl{ipfv} arrive	
%\glt \qu{They didn't see some small bird on the banyan branch called Mathila [\textit{Rhipidura} \textit{spilodera} \textit{verreauxi} Marie] but it was wondering who these people were who had just arrived.}	
%
%\z

Vamale distinguishes specific from generic participants using articles (except for pronouns and proper names). The articles mark number and definiteness, and, through their presence, specificity. The criteria for definite noun phrases include identifiability and uniqueness, as shown in (\ref{ex:t4-6}), where a woman is introduced as one of many, compared to the obligatorily definite introduction of the unique village in (\ref{ex:t1}).

\subsection{The articles \textit{ca} and \textit{eca} \qu{some}}
\label{ssec:eca}
The article \textit{eca} cannot co-occur with any of the definite articles \textit{i}, \textit{mu}, \textit{li}, nor with a demonstrative pronoun in the same phrase.\footnote{The adverb \textit{eca-ve} \qu{some-where} probably combines the stative verb \textit{ve} \qu{where? (immobile)} with the article, as does \textit{eca-se} \qu{some-one} (\ref{ex:eca}).} \Cref{tab:Pwpw} also suggests that \textit{ca}-forms are used as articles, with \textit{la} cognates in Hienghène-linkage languages.


	\ea \label{ex:eca}
	\gll	eca-aman \\
		some-thing \\
	\glt	\qu{something}% (\textit{aman} is also a placeholder for arguments of certain verbs, like \textit{xhwi aman} \qu{eat something})
	\z
	
	\ea
	%\ili{}{}{} \textit{se} can be inflected like a stative verb (\textit{se-a} \qu{he is alone; the only one})
	\gll	eca-se	\\
		some-one/other\\
	\glt	\qu{someone}
	\z
	
	\ea%
	(Vamale Usa)\\
	\gll eca-ve(n) xada	\\
	 some-where up.there	\\
	\glt \qu{somewhere up there}	
	\z


While \textit{eca} is used in compounds with singular meaning (\ref{ex:eca}), the distinction between singular and plural indefinite articles is becoming blurry. 
\textit{Ca} \qu{\gl{indf}.\gl{pl}} is distinguished by older speakers from \textit{eca} \qu{\gl{indf}.\gl{sg}}, but not anymore by many younger speakers, where the two forms are in relatively free variation. The extent to which this distinction is still important is illustrated in the examples (\ref{ex:ca_eca_old}). According to Jeo Kalène (40 years old),\footnote{Recorded on 12.11.2018, see the archive.} \textit{eca} is the singular and \textit{ca} the plural indefinite article (\ref{ex:ca_eca_old}). 

%Maybe example \ref{ex:ca_oven} is different from example \ref{ex:ca_field}, and more akin to \ref{ex:ca_xhuam}. In  \ref{ex:ca_oven} and \ref{ex:ca_xhuam}, \textit{ca} refers to parts of uncountable \qu{masses} (food in one case, things to put in her oven in the other). It is probably related to \textit{ca(-n)} \qu{in (it)} (see \ref{ex:t2}). In \ref{ex:ca_field}, \textit{ca} refers to a place, but more importantly, \textit{it does not modify a noun}. The clause in brackets is subordinate to \textit{hân fwadai} \qu{go look for it} and \textit{la} \qu{be here} is anaphoric, but does it refer to \textit{ca}, or something implied? \textit{ca} is a preposition for nouns with an article.


\ea \label{ex:ca_eca_old}
\gll tha vwa eca=apuli a= a=vwa hmwaena	\\
 \gl{ass} \gl{exist} some.\gl{sg}=person \gl{rel}= 3\gl{sg}=do thus.\gl{dist} \\
\glt	\qu{There is a person who does it like this.}
\z

\ea
(\textit{ca} and \textit{a} \qu{3\gl{sg}} cannot refer to the same participant)\\
\gll *tha vwa \textbf{ca}=apuli a= \textbf{a}=vwa hmwaena	\\
 \gl{ass} \gl{exist} some.\gl{pl}=man \gl{rel}= 3\gl{sg}=do thus.\gl{dist}	\\
\glt (for:\qu{There is a man who does thus.})	
\z

\ea
(\textit{eca} and \textit{le} cannot refer to the same participant.)\\
\gll *tha vwa \textbf{eca}=apuli a= \textbf{le}=vwa hmwaena	\\
 \gl{ass} \gl{exist} some.\gl{sg}=man \gl{rel}= 3\gl{pl}=do thus.\gl{dist}	\\
\glt (for: \qu{There is a/some man who does thus.})	
\z

\ea
\gll tha vwa ca=apuli a= le=vwa hmwaena	\\
 \gl{ass} \gl{exist} some.\gl{pl}=man \gl{rel}= 3\gl{pl}=do thus.\gl{dist}	\\
\glt	\qu{There are people who do it like this.}
\z


\ea \label{ex:ca_been}
\gll e=xaje ca=been\\
 1\gl{sg}=eat.juicy \gl{indf}.\gl{pl}=peer\\
\glt \qu{I ate some of them.} {[vamale-181127-jp\_nelemwa-1: 00:05:02]}
\z

%2018-12-17: 
\textit{Cabeen}, probably from \textit{ca been} \qu{some others, some of them}, see (\ref{ex:ca_been}), is for 25-year-old Jean-Philippe Oué the unambiguous plural form of \textit{eca}, whereas \textit{ca} is a free variant of \textit{eca}, but can also be used for the plural. A similar confusion is found in the examples below, which stem from the translated legend (the translators were all around 40–50 years old). The article in (\ref{ex:ca_oven}) could refer to plural or singular entities, but in (\ref{ex:ca_field}) would more refer signify a single place.


%\a
%
%\ili{}{}{} (context)
%\gll Fine bwaabwen pathaabua i juu mwa a mu hmaako-n xeen uvu, xeen xhwaeo, nyu ma juu-mani  
% count morning before \gl{def}.\gl{sg}=real house \gl{rel} 3\glu} find-\gl{nspec}	basket yam basket taro fish \gl{com} real-bird
%\glt  \qu{Every morning she found before the hut baskets of yam, taro, [she found] fish and \textit{notou} [huge, sacred forest pigeons \textit{Ducula goliath}]}		
%
\ea \label{ex:ca_oven}
\gll ma cika vuki-n ma a=xaahni \textbf{ca}=aman ma a=vwa tââ-n\\
 \gl{subr} \gl{neg}.\gl{exist} reason-\gl{poss} \gl{subr} 3\gl{sg}=look.for \gl{indf}.\gl{pl}=thing \gl{subr} 3\gl{sg}=make oven-3\gl{sg}.\gl{poss}	\\
\glt \qu{So that there was no reason for her to seek something to make her oven (with) (i.e. cook).}
\z

\ea \label{ex:ca_field}
\gll i=bwaabwen-an a=ja han fwadai \textbf{ca}\textsubscript{i}= {\ob}ma a=vwa nyangan-aman la\textsubscript{i}{\cb}\\
 \gl{def}.\gl{sg}=morning-\gl{poss}	3\gl{sg}=\gl{prf} go search.\gl{inan} some= \gl{subr} 3\gl{sg}=do garden-something be.here\\
\glt \qu{The next day (lit. its morning) she finally went to look for some place to make her field.}	
\z

Since number is not marked on nouns, both (\ref{ex:ca_oven}) above and (\ref{ex:ca_xhuam}) below may actually denote non-singular rather than plural, since both refer to non-singular, possibly uncountable referents.
%\todo{make up your mind viz. foodstuffs}
\ea \label{ex:ca_xhuam}
\gll		fe   ca=xhua-m 	\\		
		take \gl{art}.\gl{indf}.\gl{pl}=food-2\gl{pl}.\gl{poss}\\
\glt		\qu{Take some food (lit. Take some of your foodstuff).}
\z
%\a
%
%\gll		*fe   can  xhua-m 			
%		take in food-2\gl{pl}.\gl{poss}
%%\glt		%\qu{Take some food [lit. Take some of your foodstuff(s?)]}
%
%\z	
\subsection{Variation: \textit{li} vs \textit{ni}}
\label{ssec:ni}
%Whatever do you mean by this "distinctions of the past" ? It's opaque to me!



%The relevant data are shown below in table \ref{tab:pije_articles}.
%\begin{table}
%	\centering
%	\label{tab:pije_articles}
%	\caption{Articles in 1970s Pije.}\
%	\begin{tabular}{l|lll}
%		&	\multicolumn{2}{c}{\gl{def}}	&		\gl{indf} \\
%		&	\gl{spec}& \gl{nspec}&\\ 
%		SG& vin & vi & va\\
%		DU& [maan \textit{in Fwâi}] &maali & maala\\
%		PL& ni & li=& la\\
%	\end{tabular}
%	
%\end{table}

%Usa Vamale does not seem to use two different forms \textit{li} and \textit{ni} either, as the examples below show. 
%
\ea \label{ex:ni}
%% \ili{}{}{} 
\gll	ma 	le=tha-vwa ni ape-mae\\
	\gl{subr} 3\gl{pl}=strongly-do		ni		\gl{loc}.\gl{nmlz}-fire\\
\glt		\qu{And they start fires/make the fire places (not like before, they light too many fires now).} {[0:01:29 of vamale-171129-ecology (Usa)]}
\z
%
%
%\ea
%		
%\gll		vwa-ila-hn ea  go    ni   nyai-le  nya tha   le  moo xahut pedaa
%		do-pot-\gl{poss} 3\gl{sg}.\gl{poss} then n. child-3\gl{pl}.\gl{poss} put \gl{ass}=3\gl{pl} stay down  P.
%\glt \qu{...cook him, then their children ? They stayed down in Pindache.} 
%
%
%\ea
%	
%\gll		le 	tha 	vwa  tha  	vwa 	ni 		apuli a		thii mae
%		3\gl{pl} 	\gl{ass} 	do    \gl{ass} 	exist 	\gl{indf}.\gl{pl}? 	man \gl{rel}	light fire
%\glt		\qu{They do, there are people lighting fires}
%\ili{}{}{} 	>>>What I would have expected: \textit{Tha le vwa, tha vwa ni apuli}. The assertive is always used before the predicate in coastal Vamale, but maybe this is more flexible in Usa. Another possibility would be that \textit{le} is thematic and outside of the clause, and the structure is something like \textit{le\textsubscript{1}, [tha vwa\_\textsubscript{1}], [tha vwa ni apuli]}.
%	
%\z		
%One more form was recorded once:	\textit{kane} \gl{def}.\gl{pl}=in modern Pije (\textit{ni} in 1970s). Used by one person in Vamale once (her parents speak Pije), see example \ref{ex:kane}. This seems at the very least to indicate that the system is not as rigid as it used to be.
%\ex \label{ex:kane}
%		
%\gll		Juu 	vwa-wîî-n 		xat 	xhwii 	ka 	kane 	nyabu
%		very 	be-strong-3\gl{sg} 	sun 	bite 	\gl{sbj} k. 		mosquito
%\glt		\qu{[The] sun is very strong, the mosquitoes are biting" (Facebook message, speaker lives in Pije-dominant tribe but does not speak Pije)}
%
%\z	


\subsection{Other, related languages}	

Articles are a widespread feature of Oceanic languages \parencite[38]{lynch_oceanic_2002}. They are also common in Mainland New Caledonian languages, with the exception of Far Northern Nêlêmwa, which uses different verbal suffixes to dinstinguish specific from non-specific objects, and possibly Nyelâyu, where abbreviated demonstratives can act as articles \parencite[43]{ozanne-rivierre_nyelayu_1998}. All Northern languages use articles, from Jawe \parencite[255]{haudricourt_dictionnaire_1982} to Paicî \parencite[177]{rivierre_dictionnaire_1983}. Interestingly, though language contact and multilingualism were common and encouraged until at least the early 20\textsuperscript{th} century, article systems are not identical. Cèmuhî, for example, distinguishes nouns along personified/neutral and female/non-female axes \parencite[144]{rivierre_langue_1980}. The Hienghène systems distinguish definite, definite specific, and indefinite, where Vamale only has a definite/indefinite distinction (see \Cref{tab:hyehen_articles}). Bwatoo is described to have two plurals (but only one dual): an unmarked one, and a ``restricted" one, which is used for groups of known or cohesive elements \parencite[42]{rivierre_bwatoo_2006}, a feature which was not found in Vamale.

\begin{table}
	\caption{The article system in 1970 Hienghène languages \parencite[255]{haudricourt_dictionnaire_1982}}
	\begin{tabular}{l lll}
	\lsptoprule
			  &\multicolumn{3}{c}{Singular}\\
		      &\textsc{def}          & \textsc{def} \textsc{sp}                    & \textsc{indf}\\\midrule
		Pije  & \textit{vin}\footnote{This may be a complex form, with \textit{-n} \qu{\gl{nspec}}.} 
                             & \textit{vi}                 & \textit{va}               \\ 
		Fwâi  & \textit{ven} & \textit{veli}               & \textit{vera}             \\ 
		Nemi 1& \textit{vin} & \textit{vi}                 & \textit{va}               \\ 
		Nemi 2& \textit{ven} & \textit{vi}/\textit{veli}   &\textit{vera}/\textit{va}  \\ 
		Jawe  & \textit{nei} & \textit{di(i)}              & \textit{ya}               \\\midrule
		
		&\multicolumn{3}{c}{Dual}\\
			   &\textsc{def}          & \textsc{def} \textsc{sp}                    & \textsc{indf}\\\midrule
		Pije   &               &\textit{maali}   & \textit{maala}\\    
		Fwâi   & \textit{maan} & \textit{maali}  & \textit{maara}\\    
		Nemi 1 & \textit{maan} & \textit{maali}  & \textit{maara}\\    
		Nemi 2 & \textit{maan} & \textit{maali}  & \textit{maara}\\    
		Jawe   &               & \textit{deuli}  & \textit{deulixen}\\\midrule
		
		&\multicolumn{3}{c}{Plural}\\
		      &\textsc{def}          & \textsc{def} \textsc{sp}                    & \textsc{indf}\\\midrule
		Pije  & \textit{ni}   & \textit{li}                & \textit{la}\\
		Fwâi  & \textit{ngen} & \textit{ngeli}/\textit{li} & \textit{ngera}\\
		Nemi 1& \textit{ni}   & \textit{li}                & \textit{ra}\\
		Nemi 2& \textit{ngen} & \textit{ngeli}             & \textit{(nge)ra}\\
		Jawe  &               & \textit{deeli}             & \textit{deelixen}/\textit{yaxen}\\
		
		\lspbottomrule
	\end{tabular}
\label{tab:hyehen_articles}
\end{table}

The more archaic Vamale variety Usa kept \textit{v-} in its articles, which is found in Pwaamei, Pije as well, and was dropped in Vamale (see \Cref{tab:Pwpw}). \textit{Muu-hni} \qu{\gl{du}-\gl{prox}, those two} is frequent, and \textit{mu-ca} \qu{\gl{du}-\gl{indf}} accepted, following the Pwaamei logic of \textit{vaabu}\slash\textit{vaabu-ca} (see \Cref{tab:eca}). A combination of dual article and demonstrative suffix (in this case, a proximate visible one) is thus possible, and results in a demonstrative pronoun.

\begin{table}
		\caption{Pwaamei Hnaakâ (1a) / Pwaamei Yaak (1b) / Pwapwâ (2) / Bwatoo (3) / Usa (4) articles in: \parencite[94,95]{ozanne-rivierre_dictionnaire_} (1–2) and \parencite[42]{rivierre_bwatoo_2006} (3), fieldwork 2017 (4).}
	%\cite{ozanne-rivierre_dictionnaire_} 
	%\parencite[]{rivierre_bwatoo_2006}
	\begin{tabular}{l lllll}
	\lsptoprule
				& \multicolumn{5}{c}{\gl{def}}\\ 
				&1a             & 1b             & 2              & 3               & 4\\
				\midrule
		\gl{sg} & \textit{ve}   & \textit{vi}    & \textit{de}    &\textit{a / (a)ni} &\textit{vi(n)}   \\  
		\gl{du} & \textit{vaabu}& \textit{vaabu} & \textit{duuli} &?                &\textit{mu}   \\  
		\gl{pl} & \textit{ni/i} & \textit{ni/i}  &\textit{i/dili} &\textit{(le)ni}  &\textit{ni (li)} \\  
				\midrule
				& \multicolumn{5}{c}{\gl{indf}}\\
				& 1a& 1b & 2 &3 &4\\
				\midrule
		\gl{sg} & \textit{veca}    & \textit{vica}   & \textit{deca}   &?&\textit{veca} \\
		\gl{du} & \textit{vaabuca} & \textit{vaabuca}& \textit{duulica}&?&?  \\
		\gl{pl} & \textit{ca}      & \textit{ca}     & \textit{ca}     &?&?\\
	\lspbottomrule
	\end{tabular}
	\label{tab:Pwpw}
\end{table}


\begin{table}
	\centering

	\caption{Bwatoo articles}
	\begin{tabular}{lll}
	\lsptoprule
		\gl{sg} & \textit{a} & \textit{(a)ni} \\
		\gl{du}& \textit{(a)lu} & \textit{luni}\\
		restricted \gl{pl} & \textit{(a)le} & \textit{leni}\\
		\gl{pl} & \multicolumn{2}{c}{\textit{ni}} \\
	\lspbottomrule
	\end{tabular}
	\label{tab:Bwatoo_art}
\end{table}


\begin{table}
	\caption{Pwaamei\slash Pwapwâ articles in \citet{ozanne-rivierre_dictionnaire_}}
	\begin{tabular}{l lll lll}
	\lsptoprule
			   & \multicolumn{3}{c}{\gl{def}}   & \multicolumn{3}{c}{\gl{indf}}\\\cmidrule(lr){2-4}\cmidrule(lr){5-7}
			   & Pwm Hnaakâ & Yaak & Pwapwâ     & Pwm Hnaakâ & Yaak & Pwapwâ\\\midrule
		\gl{sg}& \textit{ve} & \textit{vi}      & \textit{de} &\textit{veca} & \textit{vica}& \textit{deca} \\
		\gl{du}& \textit{vaabu}& \textit{vaabu} & \textit{duuli} & \textit{vaabuca}&\textit{vaabuca}& \textit{duulica}  \\
		\gl{pl}& \textit{ni/i}& \textit{ni/i}   &\textit{i/dili} & \textit{ca}& \textit{ca}& \textit{ca}\\
	\lspbottomrule
	\end{tabular}
	\label{tab:eca}
\end{table}

%Things that can go between a noun phrase and its article: normally none (with the exception of the items discussed in \sectref{ssec:statVnomM}, and \textit{i-se} \qu{the other}, \textit{ca-been} \qu{the others} etc).

The definite articles \textit{i} and \textit{li} can derive relative clauses to nouns (\ref{ex:art_RC}).
\newpage

%\begin{multicolumns}{2}
		
	\ea \label{ex:art_RC}
		% \ili{}{}{} 
	\gll le=vwa ma le=thabilo li=a= le=fee-ko\\
	 3\gl{pl}=do \gl{subr} 3\gl{pl}=strike \gl{def}.\gl{pl}=\gl{rel}= 3\gl{pl}=take-2\gl{sg}.\gl{obj}\\
	\glt \qu{They want to kill those who took you.} {[B1:8]}
	\z
	
	
	\ea
	\gll na i=a= vwa wîî-n\\
	 \gl{dem} \gl{def}.\gl{sg}=\gl{rel}= \gl{exist} strength-3\gl{sg}.\gl{poss}\\
	\glt \qu{That's the strong one (among them).} {[JV:11]}
\z
%\end{multicolumn}



%Articles are not described in much detail in the literature so far, so to what extent the system presented here is extraordinary or typical will have to be answered in coming years.


\section{Demonstratives}
\is{Pronouns!Demonstrative pronouns}

Demonstrative pronouns have nominal status in the sense that they function syntactically as nouns including case marking, except that they cannot take articles. Vamale demonstratives distinguish proximal and distal (see \Cref{tab:demonstratives}), making the system somewhat simpler than the regional three-way average \parencite[38]{lynch_oceanic_2002}. They are a closed class of six forms, whose members are only partially transparent. All forms contain a distal or proximal suffix, and the dual forms still carry as a stem the dual article \textit{mu}. The plural forms' stem \textit{ni} is identical to the plural article in Usa Vamale and other Voh-Koné varieties \parencite[42]{rivierre_bwatoo_2006} discussed in \sectref{ssec:ni}, and is accepted by most speakers of Vamale as well. The segment \textit{e-} in the singular forms may derive from the singular article \textit{i} and the plural forms could be composed of plural article \textit{ni} and the singular, already lexicalised form. Demonstrative pronouns can serve both as topic and comment, as in (\ref{ex:WCdem}). 

\begin{table}
	\caption{Demonstrative pronouns}
	\begin{tabular}{lll}
	\lsptoprule
		& Proximal & Distal\\
		\midrule
		\gl{sg} & \textit{e-hni} & \textit{e-na}\\
		\gl{du} & \textit{muu-hni} & \textit{muu-na}\\
		\gl{pl} & \textit{ni-e-hni} & \textit{ni-e-na}\\	
	\lspbottomrule
	\end{tabular} 
\label{tab:demonstratives}
\end{table}



\ea \label{ex:WCdem}
\gll cahma \textbf{ehni} a=mu tua tua i=aman\\
 \gl{top} \gl{dem}.\gl{prox} 3\gl{sg}=\gl{freq} unwrap unwrap \gl{def}.\gl{sg}=thing \\
\glt \qu{But him, he was unwrapping, unwrapping the thing.} {[GS:76]}
\z


\ea
\gll li=thôa koon, li=e=paa vii \textbf{ehni} a= kon mo cahni\\
 \gl{def}.\gl{sg}=custom.object \gl{obl}-\gl{nspec} \gl{def}.\gl{pl}=1\gl{sg}=\gl{prf} say \gl{dem}.\gl{prox} \gl{rel}= \gl{prog} stay here\\
\glt \qu{The ceremonial objects I mentioned are these, which are lying here.} {[ET:1]}
\z 


%\subsection{\textit{na} \qu{\textsc{dem}}, \textit{ha} \qu{\textsc{dem.rep}}}

The demonstrative pronoun \textit{na} is special, because it can be used as a presentative (``this is Liline"), and to mark the comment of an equational clause (\ref{ex:na_topic1}, \ref{ex:na_topic2}). Since \textit{hni} does not exist (anymore), \textit{na} is neither proximal nor distal, and functions as a more neutral pronoun.


\ea\label{ex:na_topic1}
\gll {{\ob}}na{\cb}\textsubscript{{\upshape NP}} {\ob}vaang {\ob}hapi {\ob}{\ob}na{\cb}\textsubscript{{\upshape NP}}{\cb} {\ob}Lilin{\cb}\textsubscript{{\upshape NP}} a {\ob}{\ob}na=mwa{\cb}\textsubscript{{\upshape NP}} {\ob}Liiz{\cb}\textsubscript{{\upshape NP}}{\cb}\textsubscript{{\upshape clause}} a{\cb}\textsubscript{{\upshape\gl{subr}.clause}}{\cb}\\
 \gl{dem} unknown \gl{comp} \gl{dem} L. or \gl{dem}=\gl{rep} L. or\\
\glt \qu{It's unclear if it was Liline or if it was Lise or...} {[KG:472.1]}
\z

\ea\label{ex:na_topic2}
\gll Jacob tha juu xa-vee ma hmwaana. \textbf{na} hmwaana, go=xaleke?\\
 J. \gl{ass} real \gl{nmlz}.\gl{agt}-fuck \gl{subr} thus \gl{dem} thus 2\gl{sg}=see\\
\glt \qu{Jacob, it's bad for him if it's like this. That's how it is, you see?} {[KL:122.1]}
\z

\section{Nouns}
\label{sec:WC_nouns}
\is{Nouns}

Vamale nouns are defined in this grammar as syntactic units which can bear articles. Few other properties distinguish nouns from verbs, but nouns can be arguments to verbs, %(and as such often be counted)
most of them can be possessed (though possessive morphology shows overlaps with some verbal morphology, see \sectref{sec:StatV}), and though they do take some TAM marking, not all TAM marking is attested for nouns (examples include \textit{bwa balan} \qu{\gl{ipfv} \gl{cont}}, \textit{ja} \qu{\gl{prf}}). Nouns are an open class and are described in detail in \Cref{ChapterNouns}. Two noun classes are closed, however: classifiers and relational nouns.

\subsection{Classifiers}
\label{ssec:WCClass}
\is{Nouns!Classifiers}
Vamale possesses several types of classifiers, most prominently two types of relational classifiers, one type for food and drink items (\textit{xhua-} \qu{proteiny food}, \textit{ya-} \qu{starchy food}, \textit{udoo-} \qu{cold drink}, etc.), and one more type for parts of plants. Their occurrence is semantically specified. This class is described in more detail in \sectref{sec:CL}.
The descendant of the general possessive classifier POc *na has become a construct suffix \textit{-n} marking alienably possessible nouns. Another morpheme, \textit{ka}, marks alienable, semantically vaguely ``dynamic" relations (see \sectref{ssec:ka_CL}). Similar morphemes are called general classifiers by several authors \parencite{lichtenberk_oceanic_2009,lichtenberk_possessive_1985,lynch_historical_2000}. However, because \textit{ka} is syntactically different from typical classifiers (it is not the head of its phrase, it has no clear semantics and it is optional in many cases), I simply call it a linker here (following e.g. \cite{bril_ownership_2012}): a morpheme part of the head noun's noun phrase, introducing a modifier noun phrase.

\subsection{Relational nouns}
\label{ssec:WCPrepoNouns}
\is{Prepositions!Relational nouns}

Relational nouns are the functional equivalent of spatial prepositions in Vamale, meaning they are the head of a phrase. Their modifier\slash possessor is the location in which the noun phrase or verb phrase's referent is located (\ref{ex:pwa1}). The members of this closed class are possessed (mostly inalienably) and cannot take an article. They are included with nouns because of their historical relationship to nouns % xxaskAlexElias
 and their possessive morphology. The members are listed in \Cref{tab:relationalnouns}. They are inalienably possessed, except for \textit{cai-n} \qu{behind an animate entity}. 
 
\begin{table} 
\caption{Relational nouns}
\begin{tabular}{ll}
\lsptoprule
 	  \textit{xala-n} &\qu{under} \\ 
 	  \textit{cela-n} & \qu{next to} \\ 
 	  \textit{pwa-n} &\qu{on top of} \\ 
 	  \textit{pwan bwa-n}& \qu{on (top of)} \\ 
 	  \textit{cakebwa-n} &\qu{on the other side} \\ 
 	  \textit{(can) dawee-n} &\qu{(in-) between} \\ 
 	  \textit{ca-n} &\qu{in, at} \\ 
 	  \textit{can hawâ-n} &\qu{facing} \\ 
 	  \textit{ko-n}& \qu{on} \\ 
 	  \textit{pathabua-n} &\qu{before (spatial and temporal)}\\
	  \textit{cai-n} & \qu{behind an animate entity}\\   
\lspbottomrule
\end{tabular}
\label{tab:relationalnouns}
\end{table}

Some of the relational nouns are derived from nouns using a devoicing of the first consonant, e.g. \textit{bwa-n} \qu{head, top} $\rightarrow$ \textit{pwa-n} \qu{on top}, \textit{jela-n} \qu{side} $\rightarrow$ \textit{cela-n} \qu{next to}. This phenomenon is also described for Cèmuhî \parencite[37]{rivierre_langue_1980}.  Inalienably possessed relational nouns like \textit{xala-n} \qu{under} all lose their final \textit{-n} if followed by a specific argument (\ref{ex:pwa}). 


\ea\label{ex:pwa1}
%\ili{}{}{}
\gll a=xheela-ta pwa-n ye\\
 3\gl{sg}=crawl-go.up on-\gl{indf} tree\\
\glt \qu{He tree-climbs.}
\z

\ea\label{ex:pwa}
%\ili{}{}{}
\gll  a=xheela-ta pwa i=ye\\
 3\gl{sg}=crawl-go.up on \gl{def}.\gl{sg}=tree\\
\glt \qu{He climbs the tree/a tree.}
\z


\section{Independent personal pronouns}
\label{sec:WCPersPron}
\is{Pronouns!Personal pronouns}
\begin{table}
	\caption{S – free form}
	\begin{tabular}{l llll}
	\lsptoprule
		       & 1 (\gl{excl})&	1+ (\gl{incl})&	2&	3\\\midrule
		\gl{sg}&	\textit{yo}&&		\textit{go}&	\textit{ya}\\
		\gl{du}&	\textit{abu}&	\textit{gasu}&	\textit{gau}&	\textit{lu}\\
		\gl{pl}&	\textit{abe}&	\textit{gase}&	\textit{gavwe}&	\textit{le}\\
	\lspbottomrule
	\end{tabular}
	\label{tab:freePN1}
\end{table}

Vamale uses the eleven pronominal forms listed in \Cref{tab:freePN1}, distinguishing singular, dual, and plural, as well as inclusive and exclusive first persons. 
The forms are independent pronouns in the sense that they can take up the same slot as a nominal phrase. The difference to subject marker clitics is that the pronouns cannot co-occur with another noun phrase that has the same syntactic function. Subject index clitics are always present (except in imperatives, or in stative verbs with inanimate subjects). This means that, at least following the Uniqueness Condition asking for each grammatical relation to be assigned only once within its clause (after \cite[19]{kroeger_analyzing_2004}), subject indexing clitics (or ``bound pronouns") and free personal pronouns are not allomorphs. Compare to \Cref{tab:markers}. Pronouns can take subject markers (see example \ref{ex:PNcase}), but not articles.

\ea \label{ex:PNcase} 
\gll e=vii ka yo\\
 1\gl{sg}=say \gl{sbj} 1\gl{sg}\\
\glt \qu{I say.}
%\a
%
%\gll *i=mwa-n-le li=apuli
%
% \gl{def}.\gl{sg}=house-\gl{poss}-3\gl{sg}.\gl{poss} \gl{def}.\gl{pl}=person
%
%\glt (for: the house of the people)
%
%
\z


\section{Subject-indexing bound pronouns}
\label{sec:WCSubjProCl}
\is{Subject indexes}
As mentioned in \sectref{sec:WCPersPron}, Vamale uses proclitics derived from free pronominal forms, in order to index subject NPs on active verbs (see \Cref{tab:markers}). Note that stative verbs mark the intransitive subject much like alienable possessums do (compare \Cref{tab:PossSuffix1}). The clitics occur before active verbal, or nominal predicates, in the slot left of the TAM markers. The pair \textit{yo} \textasciitilde \textit{e=} \qu{1\gl{sg}} is the same in Cèmuhî \parencite[61]{rivierre_langue_1980}, but not in other Voh-Koné languages. The other markers are transparently Voh-Koné, and almost identical with the western varieties: \ort{zha} [ða] \textasciitilde \ort{a} [a] \qu{3\gl{sg}} is described for Bwatoo as well \parencite[31]{rivierre_bwatoo_2006}. The sound correspondence [ð] \textasciitilde [j] is a regular one between coastal and mountain varieties. Note that while the bound pronouns are mostly very similar to the free pronouns, they can co-occur in the same clause (\ref{ex:subject_index}), which is a sign that they do not have the same syntactic role: the free forms are true pronouns, whereas the bound pronouns have become grammaticalized to subject indexes on predicates.% the free form \textit{ya} \qu{3\gl{sg}} seems to be a loan from the Pije form \textit{nyang} or Fwâi \textit{yak} \qu{3\gl{sg}} \parencite[246]{haudricourt_dictionnaire_1982}. 

\begin{table} 
	\centering
	\caption{Subject and object markers for active and stative verbs}
	\begin{tabular}{lllll}
		\lsptoprule
		&	Free form	& \gl{a}=/\gl{S\textsubscript{A}}= & -\gl{S\textsubscript{P}} & -\gl{p}\\\midrule
		1\gl{sg} & \textit{io} & \textit{e} & \textit{-o(ng)} & \textit{-o} \\
		1\gl{du}.\gl{incl}& \textit{gasu} & \textit{gasu} & \textit{-gasu} & \textit{-kaeu}\\
		1\gl{pl}.\gl{incl} & \textit{gaa/gase} &\textit{ga(se)}&\textit{gaa}&\textit{-kaa}\\
		1\gl{du}.\gl{excl} & \textit{abu} & \textit{abu} & \textit{-abu} & \textit{-(a)bu}\\
		1\gl{pl}.\gl{excl} & \textit{abe}& \textit{abe} & \textit{-abe} & \textit{-(a)be}\\
		2\gl{sg} & \textit{go} &\textit{go} & \textit{-go} & \textit{-ko}\\
		2\gl{du} & \textit{gau} & \textit{gau} & \textit{-gau} & \textit{-kau}\\
		2\gl{pl} &\textit{gavwe}& \textit{gavwe} & \textit{-gavwe} & \textit{-kavwe}\\
		3\gl{sg} & \textit{ia} & \textit{a} & \textit{-(e)a} & \textit{-(e)a}\\
		3\gl{du} & \textit{lu} &\textit{lu} & \textit{-lu} & \textit{-lu}\\
		3\gl{pl} & \textit{le} & \textit{le} & \textit{-le} & \textit{-le}\\
		\lspbottomrule
	\end{tabular}
	\label{tab:markers}
\end{table}

\begin{table}
	\caption{Possessive suffix paradigms}
	\begin{tabular}{lc ccc}
		\lsptoprule
		& & inalienable I& inalienable Ib& alienable II \\\midrule
	\gl{sg} &	1&	\textit{-ng}&	\textit{-ong}&	\textit{-eong}\\
			&	2&	\textit{-m}&	\textit{-am}&	\textit{-go}\\
			&	3&	\textit{-n}&	\textit{-an}&	\textit{-ea}\\
			\midrule
	\gl{du} &	1\gl{incl}&	\textit{-ju}&	\textit{-aju}&	\textit{-gaeu}\\
			&	1\gl{excl}&	\textit{-bu}&	\textit{-abu}&	\textit{-abu}\\
			&	2&	\textit{-u}&	\textit{-au}&	\textit{-gau}\\
			&	3&	\textit{-lu}&	\textit{-alu}&	\textit{-lu}\\
	\midrule
	\gl{pl} &	1\gl{incl}&	\textit{-je}&	\textit{-aje}&	\textit{-gaa}\\
			&	1\gl{excl}&	\textit{-be}&	\textit{-abe}&	\textit{-abe}\\
			&	2&	\textit{-vwe}&	\textit{-avwe}&	\textit{-gavwe}\\
			&	3&	\textit{-le}&	\textit{-ale}&	\textit{-le}\\
	\lspbottomrule
	\end{tabular}
	\label{tab:PossSuffix1}
\end{table}

\ea \label{ex:subject_index}
\gll e=xale-le ka yo\\
 1\gl{sg}=see-3\gl{pl}.\gl{obj} \gl{sbj} 1\gl{sg}\\
\glt \qu{I see them.}
\z

\section{Prepositions}
\label{sec:WCPP}
\is{Prepositions}

Prepositions are derived from inalienable nominal forms (but do not take articles) and, similarly to relational nouns (see \sectref{ssec:WCPrepoNouns}), they can have generic or specific markers. There are only four prepositions: \textit{ko} \qu{lit. on} \qu{\gl{obl}}, \textit{si} \qu{for (human)} (illustrated in \ref{ex:si}), \textit{nya-si} \qu{lit. put/place-hand} \qu{for (human)} (see \ref{ex:nyasi2a}), \textit{nya-ko} \qu{lit. put/place-on}, \qu{for (all)}. Forms with \textit{ko} are lengthened when generic or anaphoric (i.e. \textit{koo-n}), and not followed by a NP. Since the markers cannot be modified, lack lexical content, and attach to noun phrases, I call them particles. A more detailed description is found in \sectref{ssec:nyako}.


\ea\label{ex:nyasi2a}
\gll tha lu=mata nyasi i=jamwa-n sohmu-n\\
 \gl{ass} 3\gl{du}=sing for \gl{def}.\gl{sg}=father-\gl{poss} study-\gl{nspec}\\
\glt \qu{They sing for the teacher.}
\z


\ea\label{ex:si}
\gll a=nya balan-o si li=thamo ka i=xa-vwa-o\\
 3\gl{sg}=put piece.of.length-bamboo \gl{ben} \gl{def}.\gl{pl}=woman \gl{sbj} \gl{def}.\gl{sg}=\gl{agt}.\gl{nmlz}-do-bamboo\\
\glt \qu{The bamboo cutter hands bamboo poles to the women.}
\z

\section{Subject marker \textit{ka}}
\label{sec:WC_sbjka}
\is{Case!Agentive}
The subject marker \textit{ka} marks the subject noun phrase (\ref{ex:kawc}). On the grounds that it is part of the noun phrase, but does not attach to any specific word class, that it is phonologically part of its host structure, and that it can itself not be modified or fronted, it is a proclitic. Obligatory for A arguments and optional for S (this is probably a more recent development), \textit{ka} can also be used to mark a focused possessor, though this is rare. A \textit{ka} marked noun phrase cannot be fronted. This distinguishes \textit{ka} from \gl{obl} markers and phrase heads \textit{ko} \qu{at, to}, \textit{nyako} \qu{at, to, for}, and \textit{nyasi} \qu{for}, which are hence considered both p- and g-words, whereas \textit{ka} is not only dependent on a noun phrase, but also on the latter's unmarked post-predicate position. See \sectref{ssec:ka} for a detailed description.

\ea \label{ex:kawc}
\gll go=thêên ka=go\\
2\gl{sg}=run \gl{sbj}=2\gl{sg}\\
\glt \qu{You run.}
\z

\section{Relativizer}
\label{sec:WCREL}
\is{Relative clause!Relativizer}
The relativizer \textit{a} introduces a relative clause subordinated to a noun phrase (\ref{ex:rel}). It is sometimes left out. It is probably related to \textit{a} \qu{3\gl{sg}}.\footnote{The relativizers in Cèmuhî are related to the plural article \textit{li} and proximal \textit{naa} \parencite[92]{rivierre_langue_1980}. The relativizer in Bwatoo is the same as in Vamale \parencite[473, 486]{rivierre_bwatoo_2006}.} Relative clauses can be derived to nouns by an article (\ref{ex:rela}).


\ea\label{ex:rel}
\gll e=thapi i=iila a= a=vwa\\
 1\gl{sg}=break \gl{def}.\gl{sg}=pot \gl{rel}= 3\gl{sg}=do\\
\glt \qu{I break the pot that he made.} {[G2:37]}
\z
 

\ea\label{ex:rela}
\gll li=a= le=vwa-sukin\\
 \gl{def}.\gl{pl}=\gl{rel}= 3\gl{pl}=do-price\\
\glt \qu{Those who shop.} {[AG1:81]}
\z

As the relativizer and the third person singular are both \textit{a}, they usually merge (\ref{ex:rel_merg}). 

\ea 
\label{ex:rel_merg}
\textit{exaleke iapuli axhwi-puaka} (from the dictionary)\\
\gll e=xaleke i=apuli \textbf{a}=\textbf{a}=xhwi puaka\\
 1\gl{sg}=see \gl{def}.\gl{sg}=man \gl{rel}=3\gl{sg}=eat pig\\
\glt \qu{I see the man who is eating pork.}
\z
%\a
%
%\ili{}{}{}  Normal relative clause
%\gll li=apuli a le=fee-ko
%
% \gl{art}.\gl{pl}=person \gl{rel} 3\gl{pl}=take-2\gl{sg}.\gl{obj}
%
%\glt \qu{The men who took you}
%
%



%\input{specific_parts_documents/i_and_ka/ka.tex}


\section{TAM markers}
\label{sec:TAM}
\is{TAM markers}
\is{Verb phrase!TAM markers}
Vamale has around a dozen morphemes used to situate an event relative to its temporal context (aspect), to the speaker moment (tense), and to reality in general (mood). Most of them are aspectual markers. Although Vamale traditionally does not seem to have expressed tense, this is nowadays done with \textit{bwa} \qu{\gl{ipfv}} and \textit{(b)o} \qu{\gl{irr}} for future tense. Most of the morphemes listed here and described in Chapter \ref{ChapterAspect} are used for several functions, and this is the case for all mood functions (e.g. \textit{balan} \qu{\gl{cont}} and \textit{bo} \qu{\gl{irr}}, but also \textit{ja} \qu{finally}). Syntactically speaking, all of the TAM markers proper are particles: though they directly precede the predicate (and follow the subject index proclitics), and though most cannot be fronted, they can be stressed. They can combine with other TAM markers to form either transparent or idiosyncratic new meanings (\ref{ex:bwakon}). Their meaning depends on the word class of the predicate's head and the aktionsart of the verb. 


\ea \label{ex:bwakon}
\gll e=bwa kon vii\\
 1\gl{sg}=\gl{ipfv} \gl{prog} say\\
\glt \qu{I am still saying.}
\z

Their basic members are \textit{bwa(n)} \qu{\gl{ipfv}}, \textit{pa} \qu{\gl{prf}}, \textit{ja} \qu{\gl{prf}}, \textit{(b)o} \qu{\gl{irr}, \gl{fut}}, \textit{balan} \qu{\gl{cont}}, and \textit{mu} \qu{\gl{freq}, \gl{iter}}. Some combinations are very common, but will be discussed in detail in §\ref{ChapterAspect}. \textit{Xa-} \qu{\gl{hab} (most likely from \gl{nmlz}.\gl{agt})}, as a prefix, is only semantically related to them. 

%\subsection{Phasal negation \textit{ban} \qu{not yet}}

The phasal negator \textit{ban} \qu{not yet} is a TAM marker as it takes a slot between the subject marker and the verb (\ref{ex:ban1}). It only occurs after the negator \textit{cipa}. \textit{Ban} may be related to \textit{bwa} \qu{\gl{ipfv}}, which in negated verb phrases has the allomorph \textit{bwan}. A reviewer suggested to analyze it as an incompletive aspect marker. This would simplify the analysis, but the term is more often used to describe actions which have begun but not been completed \parencite{overall_typology_2017}, which is not the case here.%\textit{ban} was elicited 

%\ea
%\a
%
%\ili{}{}{} 2018-07-06 transcriptions notes
%
%\gll cipa abe=ban xa-tena
%
% \gl{neg} 1\gl{pl}.\gl{excl}=not.yet \gl{hab}=listen
%
%\glt \qu{We've never heard (same meaning can be achieved without \textit{ban})}
%
%

\ea \label{ex:ban1}
\gll cipa go=ban han xaleke\\
 \gl{neg} 2\gl{sg}=not.yet go see\\
\glt \qu{You never went to see.} {[G4 16, 17]}
%\ea\label{ex:ban2}
%
%\ili{}{}{} G4 16, 17
%\gll cipa go han ban xaleke
%
% \gl{neg} 2\gl{sg} go not.yet see
%
%\glt \qu{You never went to see}
%
%
\z

\section{Verbs}% (take SUBJ= and TAM),  %how do I disambiguate this from other things? SUBJ= and TAM works for all predicates, really
\label{sec:WCVerbs}
\is{Verbs}
Verbs are by far the biggest word class and so diverse that a more in-depth description is provided in \Cref{ChapterVerbs}. 
There are active verbs and stative, the latter often describing more state-like, or patientive events. In Vamale, verbs distinguish themselves from other word classes by their participant indexes, which are proclitics for active verbs and suffixes for stative. 
While the proclitics will also attach to nominal predicates, they are optional for nouns, whereas verbs require them in non-imperative settings. Contrary to nouns, verbs cannot take articles. Verbs take derivational morphemes like causative \textit{fa-}, reflexive \textit{e-}, and attenuative \textit{the-}. The animacy of participants influences whether participants are indexed on the verb; inanimate referents do not trigger stative subject suffixes (\ref{ex:WCstat_inan}) nor object suffixes.  %\includegraphics[\width=\linewidth]{figures/alignment}
Participant indexing on verbs thus displays split-intransitive alignment: active verbs index transitive and intransitive subjects identically while distinguishing objects (\ref{ex:alignment1}, \ref{ex:alignment2}). Stative verbs do not index their invariably intransitive subject like active verbs do (\ref{ex:alignment3}), nor do their suffixes overlap with undergoer marking on transitive verbs (\ref{ex:alignment1}). An illustration is given in \Cref{fig:verb_alignment}.

\is{Alignment}
\begin{figure}
\begin{tikzpicture}
	\topnode{}
	\topleftnode[sa]{\gl{S\textsubscript{A}}}
	\toprightnode[sp]{\gl{S\textsubscript{P}}}
	\leftnode[a]{\gl{a}}
	\rightnode[p]{\gl{p}}
	\draw \convexpath{11.2pt}{a,sa};
	\draw (p) circle (11.2pt);
\end{tikzpicture}
% % 		\includegraphics[width=0.3\linewidth]{figures/verb_alignment}
\caption{Verbal alignment}
\label{fig:verb_alignment}
\end{figure}


	\ea 
\label{ex:alignment1}
	\gll e=xhwii-ko\\
	 1\gl{sg}=bite-2\gl{sg}.\gl{obj} \\
	\glt \qu{I bite you.}
	\z
	
	
	\ea\label{ex:alignment2}
	\gll e=thana\\
	 1\gl{sg}=wander\\
	\glt \qu{I wander around.}
	\z
	
	
	\ea\label{ex:alignment3}
	\gll sinu-go\\
	 suffer-2\gl{sg}\\
	\glt \qu{You are ill, you suffer.}
	\z
	
	
	\ea\label{ex:WCstat_inan}
	\gll sinu (i=xh-ong)\\
	 suffer (\gl{def}.\gl{sg}=leg-1\gl{sg}.\gl{poss})\\
	\glt \qu{(My leg) hurts.}
	\z

The main types of verbs are the following.
\begin{enumerate}
	\item \emph{active transitive}: The subject marker precedes the verb (except in the imperative), and the verb takes an argument (see \sectref{ssec:TransV}); \textit{{e=xale-ke/ko}} \qu{I see it\slash you}.
	
	\item \emph{active intransitive}: The subject marker precedes the verb, the verb does not take an argument. \textit{{E=moo}} \qu{I stay}, \textit{a=yajen} \qu{it shakes, trembles}, \textit{a=temineen} \qu{it floats}. The productive transitive suffix \textit{-ke} and the older \textit{-i} can often be added to derive a transitive form (see \sectref{ssec:ke_i}).
	
	\item \emph{stative}: stative verbs are directly followed by their subject marker. The latter are bound to the verb stem and cannot choose their host, which is why I analyze them as suffixes rather than clitics; \textbf{\textit{me-o}} \qu{I die}. While the proclitic subject markers of active verbs are obligatory for all subjects, the stative subject markers are only obligatory for human arguments. They are almost, but not completely, identical to the object markers found on transitive active verbs with animate objects that are not expressed as noun phrases (see \Cref{tab:markers}). Stative verbs cover a few semantically defined, closed groups of words.
	\begin{itemize}
	\sloppy
		\item numerals (\textit{see-a} \qu{be.one-3\gl{sg}}, \textit{thaloo-lu} \qu{be.two-3\gl{du}}, \textit{thien-le} \qu{be.three-3\gl{sg}})
		\item semantically ``patientive" verbs like \textit{sinu-ong} \qu{suffer-1\gl{sg}} \qu{I am sick/ I suffer}, \textit{xhwiiti-o koo-n} \qu{long.after-1\gl{sg} \gl{obl}-3\gl{sg}} \qu{I miss it}
		%\item demonstratives \textit{ehni-o} \qu{this is me, here I am}
		\item \textit{heeve-o/-go/-a} \qu{where-1\gl{sg}/-2\gl{sg}/-3\gl{sg}} \qu{where am I/are you/is s/he}
		\item verbs with ``adjectival meanings" like \textit{vun-go} \qu{blue/green-2\gl{sg}} \qu{you are blue/green}, \textit{xhopwe-} \qu{(be) grow(n)}, \textit{mapehno-le} \qu{they are few}
	\end{itemize}
	\item There is also a group of verbs that cannot occur alone. They are not transitive, nor can they take subject markers, and they occur before or after another, independent, verb. These bound roots (called roots because they take no morphology on their own) are further described in \sectref{ssec:MannerV}.%, in things like \textit{vwa-thuan-ke} \qu{do-well-\textsc{tr}} looks like a compound.
\end{enumerate}

\section{Adverbs}
\is{Adverbs}
\label{sec:WCAdverbs}
Vamale possesses a small class of adverbs. They occur at the end of a clause or phrase, are frequently fronted without a phrase (\ref{ex:Adv}), take neither articles nor any kind of possessive or inflectional morphology, and are, if at all, modified by the intensifiers described in \sectref{sec:WCIntensifiers}, \textit{juu} \qu{real, really, very}. As they modify verb, noun, and prepositional phrases, and as they can be fronted alone, this analysis considers them to be adjuncts. Most members are transparently derived from nouns or prepositional phrases. See \sectref{sec:Adv} for examples of their interaction with verb and noun phrases. Example (\ref{ex:Adv}) features two cases of fronted adverbs,  (\ref{ex:Adv2}) shows an adverb at the end of a verb phrase, and (\ref{ex:Adv3}) shows an adverb after noun phrase.\largerpage[-2]

\ea
\label{ex:Adv}
(Adverbs are in bold, brackets show phrases, the comma separates two clauses)\\
\gll ka {\ob}jethro{\cb} \textbf{canbwen} man \textbf{bwethalo} {\ob}le=cuut cahni ka ni=bee-m-ca{\cb}, \textbf{cahni} {\ob}ca-n xhoogo{\cb}\\
 \gl{cnj} J. yesterday \gl{com} two.days.ago 3\gl{pl}=stand here \gl{sbj} \gl{def}.\gl{pl}=peer-2\gl{sg}.\gl{poss}-\gl{prox} here in-\gl{nspec} home\\
\glt \qu{And Jethro, yesterday and the day before your relatives stood here, here at home.} {[CP1:29]}
\z


\ea\label{ex:Adv2}
\gll e=ha-mwa \textbf{canbwen}\\
 1\gl{sg}=go-\gl{rep} yesterday\\
\glt \qu{I went back yesterday.} {[B2:134]}
\z


\ea\label{ex:Adv3}
\gll na i=vaaya-n xayu \textbf{habu} ka\\
 \gl{dem} \gl{def}.\gl{sg}=work-\gl{poss} man before \gl{disc}\\
\glt \qu{This was a man's work back then, like.} {[AG1:160]}\\
\z

\subsection{Temporal adverbs}
\label{ssec:TempAdv}

Temporal adverbs are a closed class of words that can occur in a fronted position (\ref{ex:tempadv}). They cannot take articles. Temporal adverbs were almost all derived from nominals, compare \textit{bwaabwen} \qu{morning} is related to the adverb \textit{bwaabwen-an} \qu{in the morning after}. Members include \textit{ca-n-bwen} \qu{yesterday (lit. \qu{in-\gl{nspec}-night})}, \textit{naen} \qu{today/now}, \textit{xahmaen} \qu{tomorrow}\footnote{Proto (Southern) Oceanic *marani \parencite[314]{lynch_efate-erromango_2004}.}, \textit{jimin} \qu{late at night (after having fallen asleep)}, \textit{bwethaloo} \qu{two days ago}, \textit{thaloobwen} \qu{overmorrow (lit. `two nights')}, \textit{daboo-n bwen} \qu{midnight (`lit. puddle/lake of the night')} %19-07-18, page 34)
\textit{hnyanan} \qu{constantly (lit. `its breath')}, \textit{mati} \qu{earlier}, \textit{mu-bwen} \qu{early in the morning (lit. `little night')}, \textit{nyeet} \qu{when?}, \textit{ca-li-been} \qu{sometimes (lit. `among the others')}. 

\ea \label{ex:tempadv}
\gll na li=\textit{peintures} habu\\
 \gl{dem} \gl{def}.\gl{pl}=paint before\\
\glt \qu{It's the (style of) painting from the old days.} {[KG:21]}
\z

\subsection{Locative adverbs}
\label{ssec:WCLocAdv}
A closed class of words describes locations. They are mostly derived from movement verbs.\footnote{\textit{Patemwano} \qu{directly next to it}, \textit{ngangeno} \qu{close-by} in Pije \parencite[170]{haudricourt_dictionnaire_1982} and \textit{puput} \qu{behind (a building or a sizeable entity)} can be used in the same slots but do not possess the morphological combinatorics shown in \Cref{tab:LocAdv}.} They do not bear articles, can be fronted alone (\ref{ex:front_xahut}), and can modify verbs (\ref{ex:xahut1}) as well as nouns (\ref{ex:LocAdvPred}). Locative adverbs can be predicates, see (\ref{ex:LocAdvPred}). Contrary to relational nouns (\sectref{ssec:WCPrepoNouns}), locative adverbs do not form possessive relations with nouns, nor do they take generic \textit{-n}. \Cref{tab:LocAdv} shows a summary of the forms. For a more thorough account of space (see \sectref{ssec:spat_adv}).
%members: (\textit{xahan}, \textit{xahut}, \textit{xada}, \textit{xahnuut}, \textit{xahnuda}, \textit{nya-xahan} etc, possibly \textit{patemwano}

\begin{table}
	\caption{Locative adverbs}
	\fittable{\begin{tabular}{lllll}
		\lsptoprule
	                &      & Simple & \\
		Axis		& Verb & location & Close-by& Further away\\
		\midrule
		same-level& \textit{han}& \textit{xa-han}& \textit{nya-xa-han} & \textit{nya-an xa-han}\\
		downward & \textit{hut}& \textit{xa-hut} & \textit{nya-xa-hut} & \textit{nya-ut xa-hut}\\
		upward & \textit{ta} &	\textit{xa-da} & \textit{nya-xa-da} & \textit{nya-da xa-da} \\
		downstream & \textit{hnuut}& \textit{xa-hnuut}& \textit{nya-xa-hnuut} & \textit{nya-hnut xa-hnuut}\\
		upstream & \textit{hnuuda} & \textit{xa-hnuuda} & \textit{nya-xa-hnuuda}& \textit{nya-hnuda xa-hnuuda}\\
		\lspbottomrule
	\end{tabular}}
\label{tab:LocAdv}
\end{table}


\ea\label{ex:xahut1}
\gll go=moo xahut, go=xahut, go=hut xahut\\
 2\gl{sg}=stay below 2\gl{sg}=below 2\gl{sg}=go.down below\\
\glt \qu{You live down there, you're down there, you go down there.}
\z


\ea\label{ex:front_xahut}
\gll xahut, go=majit mati\\
 below 2\gl{sg}=rest earlier\\
\glt \qu{Down there, you were sleeping earlier.}
\z


\ea\label{ex:LocAdvPred}
\gll i=apuli (a=) xahut\\
 \gl{def}.\gl{sg}=man (3\gl{sg}=) below\\
\glt \qu{The man is down there.}
\z

\subsection{\textit{hman} \qu{also}}
\textit{hman} \qu{also} modifies verbs (\ref{ex:hmanV}), nouns (\ref{ex:hman}), and adverbs. Contrary to temporal and locative adverbs, \textit{hman} always comes after the modified word and cannot be fronted on its own. Similarly to \textit{mwa} (\sectref{sec:mwa}), \textit{hman} is used as a discourse marker as well, with a meaning of \qu{however} (\ref{ex:hmanDisc}).

\ea\label{ex:hmanV}
\gll tha gau=han tha gau tha gase=bo \textit{arriver} hman\\
 \gl{ass} 2\gl{du}=go \gl{ass} 2\gl{du} \gl{ass} 1\gl{pl}.\gl{incl}=\gl{irr} arrive also\\
\glt \qu{You go (despite the height of the steel beam), you two, we'll get (to the other side), too.} {[KG:9]}
\z

\ea\label{ex:hman}
\gll li=meeka i=\ldots  li=nyan-mwa ca-n {hman}\\
 \gl{def}.\gl{pl}=all \gl{def}.\gl{sg}=\ldots \gl{def}.\gl{pl}=inside-house in-\gl{ana} also\\
\glt \qu{All the rooms in the house as well.} {[KG: 30-1]}
\z

\ea\label{ex:hmanDisc}
\gll thake yavo kavi tha cipa xhwii hman\\
 throw fishing.line but \gl{ass} \gl{neg} bite also\\
\glt \qu{...threw out the fishing line but it didn't bite though.} {[GP:73]}
\z

\section{Complementizer \textit{hapi}}
\is{Subordination!Complementation}
Clauses that are complement to verbs of cognition (\ref{ex:hapi2}), opinion, and perception (\ref{ex:hapi3}), are introduced by the subordinator \textit{hapi}. Considering \textcite{lynch_oceanic_2002}'s observation that Oceanic languages tend to use a form related or identical to the word \qu{to say} to introduce complement clauses \parencite[53]{lynch_oceanic_2002}, that Voh-Koné languages changed \textit{p-} $\rightarrow$ \textit{v-}, and that Hienghène languages have \textit{peei} \qu{to say}, \qu{\gl{comp}} \parencite[260]{haudricourt_dictionnaire_1982}, postulating \textit{a=vii} \qu{s/he says} as the origin of \textit{hapi} seems plausible.


\ea\label{ex:hapi2}
\gll e=caihna-n hapi tha hmwaana\\
 1\gl{sg}=know-\gl{nspec} \gl{comp} \gl{ass} thus\\
\glt \qu{I know that it's like that.}
\z


\ea\label{ex:hapi3}
\gll sahnaang-eong hapi tha hmwaana\\
 not.understand-1\gl{sg} \gl{comp} \gl{ass} thus\\
\glt \qu{I'm not sure/I doubt that it's like that.}
\z

\section{Conjunctions}
\label{sec:Conj}
Vamale distinguishes two groups of conjunctions: those that link noun phrases, and those that link verb phrases as well as clauses. Clauses are defined by the presence of a predicate, which is in most cases a verb phrase. This book calls the conjunctions linking these ``verbal", to distinguish them from nominal ones.   

\subsection{Nominal conjunctions}
\label{ssec:NomConj}

Nominal conjunctions connect noun (phrases) and form a new constituent containing the connected noun phrases and the conjunction. Members include \textit{ma} `and\slash with' (\ref{ex:maconj}), \textit{ka} `on the other hand' (\ref{ex:kaconj}), \textit{hai} \textasciitilde \textit{a} \qu{or} (\ref{ex:haiconj1}) with its derivative \textit{hai...hai} \qu{either...or} (\ref{ex:haiconj}),  \textit{moko} \qu{more than}. \textit{Moko} may be complex and composed of \textit{moo} \qu{rest, reside} and \textit{ko} \qu{on} (\sectref{sec:Comp}). It has no Hienghène cognate. The other Voh-Koné varieties share the form, however. \citet[215]{rivierre_bwatoo_2006} suggest a makeup of \textit{mo} \qu{from}, as in \textit{e ha-me \textbf{mo} Tuo} \qu{I come \textbf{from} Touho}, and \textit{ko} \qu{on}. 


\ea\label{ex:maconj}
\gll i=wabatan ma i=xat\\
 \gl{def}.\gl{sg}=north.wind and \gl{def}.\gl{sg}=sun\\
\glt `the north wind and the sun'
\z


\ea\label{ex:kaconj}
\gll i=wabatan ka i=xat\\
 \gl{def}.\gl{sg}=north.wind and \gl{def}.\gl{sg}=sun\\
\glt `the north wind, and (on the other hand) the sun'
\z


\ea\label{ex:haiconj1}
\gll i=wabatan hai i=xat\\
 \gl{def}.\gl{sg}=north.wind or \gl{def}.\gl{sg}=sun\\
\glt`the north wind or the sun'
\z


\ea\label{ex:haiconj}
\gll hai i=wabatan hai i=xat\\
 or \gl{def}.\gl{sg}=north.wind or \gl{def}.\gl{sg}=sun\\
\glt \qu{either the north wind or the sun}
\z

\subsection{Verbal conjunctions}
\label{ssec:VerbConj}

\is{Coordination!Verbal conjunctions}
The set of verbal conjunctions is small and closed, and groups together some morphemes which only occur in this set, like \textit{kavi} \qu{but}, with words also present in other distributional classes. The meaning distinctions between \textit{kavi} \qu{but (introducing something in contrast with the former element)}, \textit{ko} \qu{but [introducing something unexpected)}, and \textit{ma} \qu{but (relaying something related but different)} are fine and depend on the context. Members include \textit{ka} `and' (\ref{ex:kaVC}), \textit{kavi} `but' (\ref{ex:kaviVC}), \textit{ma} `and, but' (\ref{ex:maVC}), \textit{hai} \goodtilde \textit{a} `or' (\ref{ex:haiVC}), \textit{ko} \qu{but}, \qu{because}, \textit{kona} \qu{furthermore}. 


\ea\label{ex:kaVC}
\gll le=hame ka le=siwa=mwa\\
 3\gl{pl}=come and 3\gl{pl}=return=\gl{rep}\\
\glt `They came and they left again.'
\z


\ea\label{ex:kaviVC}
\gll le=hame kavi le=siwa-mwa\\
 3\gl{pl}=come but 3\gl{pl}=return-\gl{rep}\\
\glt `They came but they left again.'
\z


\ea\label{ex:maVC}
\gll le=ha-me ma le=siwa-mwa\\
 3\gl{pl}=go-\gl{dir.cp} and 3\gl{pl}=return-\gl{rep}\\
\glt `They come and/in order to go.'
\z


\ea\label{ex:haiVC}
\gll le=ha-me hai le=siwa=mwa\\
 3\gl{pl}=go-\gl{dir.cp} or 3\gl{pl}=return=\gl{rep}\\
\glt `They come or they go.'
\z

\textit{Ka} is also used colloquially after a clause to ask for confirmation (\ref{ex:kaDisc}) (see \sectref{discourse ka}).

\ea \label{ex:kaDisc}
\gll i=apuli a=xahan ka?\\
 \gl{def}.\gl{sg}=person \gl{rel}=over.there \gl{disc}\\
\glt \qu{The guy over there, like?}
\z

%\subsubsection{Numeral coordinators \textit{na-bwa}, \textit{ko}}
\is{Coordination!Numeral coordinators}
Numbers are verbal, and \textit{(na)-bwa} \qu{plus} (possibly from \gl{dem}-\qu{head}, \ref{ex:nabwa}), and \textit{ko} \qu{times} (probably from \textit{ko} \qu{on}) are used to construct complex numbers (\ref{ex:nabwako}). 


\ea\label{ex:nabwa}
(nim a-bwa se)\\
\gll nim na-bwa se\\
 5 plus 1\\
\glt \qu{6}
\z


\ea\label{ex:nabwako}
\gll nim na-bwa se ko apuli nabwa nim na-bwa se\\
  5 plus 1 times man/20 plus 5 plus 1\\
\glt \qu{126}
\z

\section{Subordinators}

Vamale subordinators introduce a subordinated clause. They precede all other elements of said clause, and cannot occur without the clause, moving with it if fronted, see examples (\ref{ex:fronted}--\ref{ex:frontedend}). All those ending on -\textit{a} assimilate to following \textit{e=} \qu{1\gl{sg}}. They are proclitics.
Members include \textit{cala} `when' \textit{cama} `if', \textit{ma} `as\slash in order to', \textit{ko} `because', \textit{ko-ma} `so that',  \textit{ecupwa}  \qu{until}.\footnote{Possibly from \textit{e-cuut-pwa} \qu{\gl{refl}-stand-on}.}

\ea \label{ex:fronted}
\gll cel=e=hame go=pa yahan\\
 when=1\gl{sg}=come 2\gl{sg}=\gl{prf} leave\\
\glt `When I came, you had already left.'
\z


\ea
\gll cala go=hame e=pa yahan\\
 when 2\gl{sg}=come 1\gl{sg}=\gl{prf} leave\\
\glt `When you came, I had already left.'
\z


\ea
\gll cem=e hame go=pa yahan\\
 if/when=1\gl{sg} come 2\gl{sg}=\gl{prf} leave\\
\glt `If I come, you will already have left.' \\'Whenever I come, you already have left.'
\z


\ea
\gll cama go=hame e=pa yahan\\
 if/when 2\gl{sg}=come 1\gl{sg}=\gl{prf} leave\\
\glt `If you come, I will already have left.' \\ `Whenever you come, I already have left'
\z


\ea
\gll m=e hame go=pa yahan\\
 as=1\gl{sg} come 2\gl{sg}=\gl{prf} leave\\
\glt `As I come, you've already left.'
\z


\ea \label{ex:frontedend}
\gll ma le=fe, le=mu=xaahni\\
 as 3\gl{sg}=take, 3\gl{pl}=\gl{freq}=check\\
\glt \qu{When they take it, they check it.}
\z

Non-fronted examples, illustrated in (\ref{ex:non-fronted}), are the norm.

\ea \label{ex:non-fronted}
\gll tha=abe=saavi cama=abe icu-koo-n ko-n \textit{marché}\\
 \gl{ass}=1\gl{pl}.\gl{excl}=dig.up \gl{subr}=1\gl{pl}.\gl{excl} trade-\gl{obl}-\gl{ana} on-\gl{nspec} market\\
\glt `We dig (them) up when we sell them on the market.' {[AG1:22]}
\z


\ea
\gll le=thêên cala le=siwa-mwa\\
 3\gl{pl}=run when.\gl{real} 3\gl{pl}=return-\gl{rep}\\
\glt \qu{They ran when they went back.}
\z

\ea
\gll le=thêên ma le=yahan\\
  3\gl{pl}=run when.\gl{irr}/if 3\gl{pl}=leave\\
\glt `(Usually) they run when they leave.' / `They would run if they left.'
\z


\ea
\gll e=ha-me ma go=bwa=yahan\\
 1\gl{sg}=go-\gl{dir.cp} \gl{subr} 2\gl{sg}=\gl{ipfv}=leave\\
\glt `I come as as you leave.' / `I come if you leave.' / `I come so that you leave.'
%\a
%
%\gll le=ɣahni ma le=fe
%
% 3\gl{pl}=check in.order.to 3\gl{pl}=take
%
%\glt \qu{they're checking to take}
%
%
\z




\section{Negation markers}
\is{Negation}

The negation markers \textit{cipa} \qu{\gl{neg}} and \textit{cipii} \qu{\gl{proh}} share their scope over the entire following clause and their left-most position. The negation markers are not identical in their distribution and could, strictly speaking, be classified into two separate classes. Contrary to \textit{cipa} \qu{\gl{neg}} (\ref{ex:thacipa}), \textit{cipii} \qu{\gl{proh}} cannot take assertive \textit{tha}, nor \textit{na} \qu{\gl{foc}}. Furthermore, \textit{cipii} often omits the subject marker, which \textit{cipa} cannot do (\ref{ex:cipi}). This grammar will treat \textit{cipa} as a proclitic, because it integrates into the following verb phrase's stress structure, and assimilates phonologically to it as well (\ref{ex:cipa}).
%

	
	\ea\label{ex:thacipa}
	\gll (tha) cipa {\ob}go=bwaa=majit{\cb}?\\
	 (\gl{ass}) \gl{neg} 2\gl{sg}=\gl{ipfv}=sleep\\
	\glt \qu{Aren't you still asleep?}
	\z
	
	
	\ea\label{ex:cipa}
	[ˌci.pe.ˈmãn.ɟit]\\
	\gll cipa= e= majit\\
	 \gl{neg} 1\gl{sg}= sleep\\
	\glt \qu{I don't sleep.}
	\z
	
\ea \label{ex:cipi}
\gll cipii xaloo koo-ng hmwaahni (ka go)! \\
 \gl{proh} gaze \gl{obl}-1\gl{sg}.\gl{poss} thus \gl{sbj} 2\gl{sg}\\
\glt \qu{Don't look at me like that!}
\z

\section{Assertive \textit{tha}}
\label{sec:WCAssertive}
\is{Assertive \textit{tha}}

The assertive marker \textit{tha} is a proclitic that docks onto the predicates of non-imperative clauses, on the left-most position (\ref{ex:tha1}). \textit{Tha} assimilates to \textit{e=} \qu{1\gl{sg}}, like \textit{cipa} \qu{\gl{neg}} (\ref{ex:tha2}). %unlike \textit{na} \qu{\gl{dem}}, pro-clitic to predicates.



\begin{exe}[(999)]
\ex \label{ex:tha1}
\gll \textit{au lieu} ma tha bwa xhavwale i=\textit{copain}-ea vukin tha=a bo \textit{guide}-ea\\
 instead \gl{subr} \gl{ass} \gl{ipfv} wait \gl{def}.\gl{sg}=friend-3\gl{sg}.\gl{poss} reason \gl{ass}=3\gl{sg} \gl{irr} guide-3\gl{sg}.\gl{poss}\\
\glt \qu{Instead of waiting for his friend, because he would be his guide!} {[KG:497]}
\end{exe}


\ea\label{ex:tha2}
\gll cala th=e vwa-tau \\
 when \gl{ass}=1\gl{sg} do-impact\\
\glt \qu{when I fish} {[B3:3]}
\z

\section{Intensifiers}
\label{sec:WCIntensifiers}
\is{Intensifiers}

The two intensifiers \textit{juu} \qu{real, very} and \textit{vaa} \qu{(too) much} (most often preceded by \textit{juu}, but see \ref{ex:vaa}) cannot stand alone, are semantically vague (see \Cref{tab:juu}), and attach to the head of a phrase (be that a noun, an adverb, or anything else, \ref{ex:juu}) as closely as possible. Given that they can be stressed, they are analyzed as particles, though ``anti-clitic" may be a better term considering the fact that they syntactically depend on a host that can be nominal, verbal, or adverbial in nature. \textit{Juu} is also associated to \textit{bwa} \qu{\gl{ipfv}}, as described in \sectref{ssec:bwa_ju}.


\ea\label{ex:juu}
\gll a=juu hnyimake ka i=juu apuli, juu ca-n-bwen\\
 3\gl{sg}=very think \gl{sbj} \gl{def}.\gl{sg}=real person real in-\gl{nspec}-night\\
\glt `He thought hard, the real man, just yesterday.'
\z


\ea\label{ex:vaa}
\gll ma go=hmwaani vwasoon, ma go=hmwaani vaa...\\
 \gl{cond} 2\gl{sg}=like.this impossible \gl{cond} 2\gl{sg}=like.this too.much\\
\glt \qu{If you do it like this, it's impossible, and if you do it like this, it's too...} {[KG:140]}
%\a
%
%\gll na li=a le=ɣaleke
%
% DEM.PRED SPEC.PL REL 3PL=see
%
%\glt `It is they who watched.'
%
%
\z


\begin{table}
	\caption{Meanings of compounds with \textit{juu}}
	\begin{tabular}{lll}
	\lsptoprule
		Form & \multicolumn{2}{c}{Translation of}\\\cmidrule(lr){2-3}
		     & the 2nd morpheme & the whole\\
	\midrule
		\textit{juu han} & walk & \qu{walk barefoot}\\
		\textit{juu aman} & thing & \qu{important (adverb)}\\
		\textit{juu we} & water & \qu{drinking water} \\
		\textit{juu toot} & grass & \qu{thatching grass}\\
		\textit{juu o} & bamboo & \qu{building bamboo}\\
		\textit{juu mwa} & house & \qu{trad. house}\\
		\textit{juu mani} & bird & \qu{{notou} {[ducula goliath]}}\\
		\textit{juu apuli} & person & \qu{Kanak}\\
		\textit{juujuu} & & \qu{truth}\\
	\lspbottomrule
	\end{tabular}
\label{tab:juu}
\end{table}

The particle \textit{vaa}, depending on the word it modifies, means \qu{much (uncountable)} with non-human nouns (\ref{ex:vaa1}), %meaning \qu{many} for ants, in contrast to \textit{hmain}, \qu{many} for humans.
 intensifies the following verb, e.g. \textit{vaa thêên} \qu{strongly run}\slash\qu{run fast}, and in combination with \textit{ju} \qu{real, true}, it means \qu{too much}, as in (\ref{ex:vaa2}). 


\ea\label{ex:vaa1}
\gll e-vaa nya-da xa-da\\
 \gl{mid}-\gl{ints} towards-up.there \gl{loc}.\gl{adv}-up \\
\glt \qu{There are many (feral pigs) up there.} {[J3 16.1]}
\z


\ea \label{ex:vaa2}
\gll juu va vwasoon ma gase=vwa li=vaaya-n li=xhaohmu\\
 real much difficult \gl{comp} 1\gl{pl}.\gl{incl}=do \gl{def}.\gl{pl}=work-\gl{poss} \gl{def}.\gl{pl}=elder\\
\glt \qu{It's too hard for us to do the work of the elders.} {[KP:98]}
\z



\section{Repetitive \textit{mwa}}
\label{sec:WCRepetitive}
\is{Repetitive \textit{mwa}}

This class only has one member. \textit{Mwa} has rather different, related meanings, depending on the context. \textit{Mwa} can have the repetitive meaning \qu{again} (\ref{ex:mwa_even}), the restitutive \qu{back}, as well as \qu{also}, \qu{even}, \qu{on top of that}, or mark the preceding phrase as focused (see \sectref{sec:mwa} for a discussion). The deictic use of \textit{mwa} \qu{now} (\ref{ex:mwa}), seems to mostly anchor the listener's attention, similarly to \textit{mwa} \qu{even}, onto the noun phrase given, see (\ref{ex:mwa_now1a}). \textit{Mwa} is a particle that can dock onto any phrase preceding it (see \ref{ex:mwa_rep}).

\ea\label{ex:mwa_even}
\gll e=xaleke mwa\\
 1\gl{sg}=see \gl{rep}\\
\glt `I see again.', `I even see.'
\z

\ea
\label{ex:mwa_rep}
\gll e=vatipwe mwa nya-mwa si-m mwa i=mwani mwa\\
 1\gl{sg}=drop \gl{rep} give-\gl{rep} hand-2\gl{sg}.\gl{poss} \gl{rep} \gl{def}.\gl{sg}=money \gl{rep}\\
\glt \qu{I pass on to you too this money as well.}
\z


\ea\label{ex:mwa}
\gll hê na tha vwa li=wii-n. go le=vwa ibi-han li=nyamaan go tha le=ve-moo mwa, moo mwa. \\
 yes \gl{dem} \gl{ass} \gl{exist} \gl{def}.\gl{pl}=field-\gl{poss}.\gl{nspec} then 3\gl{pl}=do pinch-walk \gl{def}.\gl{pl}=eye then \gl{ass} 3\gl{pl}=\gl{mid}-stay \gl{rep} stay \gl{rep}\\
\glt \qu{Yes there were fields of it (macaranga vedeliana). And they'd go pinch the young sprouts. And those stay together now, stay.} {[KL:218-222]}
\z


\ea\label{ex:mwa_now1a}
\gll ya a=ja vwa \textbf{mwa} li=wee-n a=ta-\textbf{mwa} sibu li=sibu \textbf{mwa}. ja yabwat \textbf{mwa} sisuu \textbf{mwa}\\
 \gl{expl} 3\gl{sg}=\gl{prf} do \gl{rep} \gl{def}.\gl{pl}=water-\gl{poss}.\gl{nspec} 3\gl{sg}=go.up-\gl{rep} swell \gl{def}.\gl{pl}=swell \gl{rep} \gl{prf} dry \gl{rep} hard \gl{deict}\\
\glt \qu{And there's the sap that rises, swells, the swells there. It dries then, gets hard then.}
\z

In (\ref{ex:movV-mwa1}) and for all other movement verbs, as well as \textit{xhose} \qu{do again}, \textit{mwa} is analyzed as a suffix, i.e. as having fused with its host. First, \textit{mwa} assimilates to the root, which it does not do in other contexts.\footnote{\textit{Xhosepwa} suggests a dropped \textit{-t} or \textit{-p}. The Pije and Fwâi cognates \textit{khô-peei} \qu{?-say} \parencite[155]{haudricourt_dictionnaire_1982} could be a diachronic hint at a morphologically complex, old Vamale form.} Compare \textit{hut-mwa} $\rightarrow$ /hupʷa/ \qu{go back down}, to \textit{hut=mwa} \qu{go down again}.

%\begin{multicols}{2}
\ea \label{ex:movV-mwa1}
\gll go=ha-mwa-me\\
 2\gl{sg}=go=\gl{rep}=\gl{dir.cp}\\
\glt \qu{You return to me, you come back.}
\z


\ea
\gll go=ha-me mwa\\
 2\gl{sg}=go=\gl{dir.cp} \gl{rep}\\
\glt \qu{You come again.}
\z
%\end{multicols}

The particle also expresses repetition (\ref{ex:mwa_rep1a}), and deictically referring to something close spatially or recently mentioned (which is probably a derived meaning), as in (\ref{ex:mwa_rep1b}). See \sectref{sec:mwa} for a more detailed description. %Also means "even, on top of that". 


\ea\label{ex:mwa_rep1a}    
\gll e=tena mwa\textsuperscript{{\upshape REP}} i=hun-det\\
 1\gl{sg}=hear \gl{rep} \gl{def}.\gl{sg}=\gl{nmlz}-sound\\
\glt \qu{I hear the sound again.} {[JR:17]}
\z

\ea\label{ex:mwa_rep1b}
\gll xhose e=tena mwa\textsuperscript{{\upshape REP}} tha=a=bwa vwa det mwa\textsuperscript{{\upshape\gl{deict}}}\\
 again 1\gl{sg}-feel \gl{rep} \gl{ass}=3\gl{sg}=\gl{ipfv} do sound \gl{rep}\\
\glt \qu{Again I heard him make said (\textit{mwa}) noise.} {[JR:18]}
\z 


%intransitive und transitive stämme in vamale müssen nicht postuliert werden, bzw. kein unterschied zu intransitiven und transitiven wurzeln

\section{Interjections}
\is{Interjections}
Interjections do not integrate into clauses or phrases, and though at least \textit{hê} \qu{yes} can be derived to \textit{hêêke} \qu{to assent, to say yes}, and \textit{cika} \qu{\gl{neg}.\gl{exist}} is a commonly used impersonal verb, exclamations form a group through their uniquely individualistic behavior.
Members include \textit{ya} \qu{voilà, the result is there}, \textit{ûhû}\slash\textit{cika} \qu{no}, \textit{hat} \qu{strong negation}, \textit{hai} \qu{oh! (surprise, discovery)} and \textit{hê}\slash\textit{helong} \qu{yes}, as well as a growing class of swearwords. 

\section{Quantifiers}
\is{Noun phrase!Quantifiers}
Quantifiers are a tiny group of particles that are not inflected, directly preceding an (article) noun construction: \textit{mu} \qu{little}, \textit{jaa} \qu{many} (\ref{ex:jaa}), and \textit{ju-vaa} \qu{too much}, which is also attested as an intensifier in verb phrases (\ref{ex:juva}). Quantifiers denote number and are described in \sectref{ssec:Quant}. Other words have similar meanings, but are verbs, like \textit{hmai-} \qu{many}. One quantifier similarly integrates the noun phrase, but bears possessive suffixes: \textit{meeka-n} \qu{all}.



	\ea\label{ex:jaa}
	\gll ja apuli canbwen\\
	 many people yesterday\\
	\glt \qu{(there were) more people yesterday (than now)} {[B2 31.1]}
\z	
	
	\ea\label{ex:juva}
	\gll ju-vaa apuli\\
	 too.much person\\
	\glt \qu{too many people} {[B2:32]}
	\z
	


%\section{Conclusion}
%Does it even need to have a conclusion?
%\begin{quote}
%	Vamale wordos:\\
%	They differ like spring flowers\\
%	Behold the garden.\\
%\end{quote}
%\begin{table}
%\begin{tabular}{p{3cm}|c|c|c|c|c|c}
%takes / appears in & nouns & \multicolumn{2}{c}{verbs} & adverbs & \\
%\midrule
%&& active& stative&&&\\
%\midrule
%Articles & x &&&&&\\
%Possessive suffixes& x &&&& &\\
%S/A- markers and -O markers&  & x&&&&\\
%-S markers& && x&&&\\
%in NP& x &&& x& &\\
%in VP&& x& x&& x&\\
%\end{tabular}
%\caption{Word classes}
%\label{tab:WordClasses}
%\end{table}
