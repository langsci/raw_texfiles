\chapter{Word order} 
\label{ChapterWO} 


(subjekt), pronomen/subjektindex, modalpartikel, wurzel, objektindex/-ke, objekt, (ka, subjekt [markiert]?)

%todo{na: FOC oder THEM oder beides?}
%todo{li: spez/def?}

\pex
\label{ex:tree}
\a
\begingl
\gla tha abe [na [\textsubscript{NP}li= [\textsubscript{NP}a [le= xhwat xhwi [\textsubscript{NP}li= [\textsubscript{NP}nyu [\textsubscript{NP}a [\textsubscript{?P}xhopwen]]]]]]]]//
\glb \gl{ass}  1\gl{pl}.\gl{excl} \gl{na} \gl{def}.\gl{pl}= \gl{rel} 3\gl{pl}= almost eat \gl{def}.\gl{pl}= fish \gl{rel} big//
\glft \qu{We are the ones who almost eat/ate the big fish}//
\endgl

\a
\begingl
\gla  yo, e=bwa ja fa-xaleke i=ye ko i=apuli//
\glb 1\gl{sg}, 1\gl{sg}=\gl{ipfv} ja \gl{caus}-see \gl{def}.\gl{sg}=tree \gl{obl} \gl{def}=man//
\glft \qu{I have been making the man look at the tree.}//
\endgl
\xe

%todo{test examples}
The word order VOS is still used for stative verbs, especially with inanimate arguments.
\ex
\begingl
\gla sinu i=xho-ng//
\glb ill \gl{def}.\gl{sg}=leg-1.\gl{sg}.\gl{poss}//
\glft \qu{my leg hurts}//
\endgl
\xe


%\begin{figure}
%\begin{forest}
%for tree={
%fit=band,
%    parent anchor=south,
%    child anchor=north,
%    if n children=0{
%      font=\itshape,
%      tier=terminal,
%    }{},
%  }
%[S 
%	[?
%		[?
%			[tha]
%		]
%		[PN
%			[abe]
%		]
%	]
%	[?[?[na]]
%	[NP
%		[Det
%			[li{=}]]
%		[NP 
%			[REL
%				[a]]
%					[VP
%							[?
%								[le{=}]
%							]
%							[Adv	
%								[xhwat]] 
%							[VP	
%							[V
%								[xhwi] 	]			
%								[NP
%								[Det
%									[li{=}] ]
%									[N
%										[nyu]] 
%										[REL
%										[a] ]
%										[Adj
%											[xhopwen]
%										]
%								]		
%						]	
%						]		
%					]		
%			]
%		]
%		]
%		]
%]
%\end{forest}
%\caption{Tree-diagram of the sentence structure of example (\ref{ex:tree})}
%\label{fig:WOTree}
%\end{figure}

It appears with animate arguments especially when the agent is stressed (because of the \textit{ka}?). While the translation given tends to be a passive, nobody minds when I use \textit{ka}.

\pex
\a
\begingl
\gla e=ta//
\glb 1.\gl{sg}=go.up//
\glft \qu{I go up}//
\endgl
\a
\begingl
\gla go=ta//
\glb 2.\gl{sg}=go.up//
\glft \qu{You go up}//
\endgl
\a
\begingl
\gla Yo e=xale-ko//
\glb 1.\gl{sg} 1.\gl{sg}=see-2.\gl{sg}.\gl{obj}//
\glft \qu{Me, I see you}//
\endgl
\a
\begingl
\gla Go go=xale-o//
\glb 2.\gl{sg} 2.\gl{sg}=see-1.\gl{sg}.\gl{obj}//
\glft \qu{You, you see me}//
\endgl
\a
\begingl
\gla go e=xale-ko ka yo//
\glb 2.\gl{sg} 1.\gl{sg}=see-2.\gl{sg} \gl{agt} 1.\gl{sg}//
\glft \qu{You, I see you, I\footnote{ (formerly preferred, see Old Man Cakeo's Text)}}//
\endgl
\a
\begingl
\gla (e)=xale-ko//
\glb 1.\gl{sg}=see-2.\gl{sg}.\gl{obj}//
\glft \qu{I see you \footnote{(preferred now)}}//
\endgl
\xe 
-----> nominative/accusative system for free and bound pronouns, S\textsubscript{A} and A, and S\textsubscript{P} and P are marked the same, probably. Sinu-a, go xale-a, go pala.
\pex
\a
\begingl
\gla yo a=xale-o ka ya//
\glb 1.\gl{sg} 2.\gl{sg}=see-1.\gl{sg} \gl{agt} 3.\gl{sg}//
\glft \qu{Me, he saw me}//
\endgl 

\a
\begingl
\gla a=xale-o ka i=pupwaale//
\glb 3.\gl{sg}=see-1.\gl{sg} \gl{cnj} \gl{def}=Caucasian//
\glft \qu{The white guy sees me}//
\endgl


\a
\begingl
\gla a-xale-a i=pupwaale ka i=daahma//
\glb 3.\gl{sg}=see-3.\gl{sg}.\gl{obj} \gl{def}=Caucasian \gl{agt} \gl{def}=chief//
\glft \qu{The chief sees the White guy}//
\endgl

\a
\begingl
\gla *a=xale-a i=pupwaale i=daahma//
\glb 3.\gl{sg}=see-3.\gl{sg}.\gl{obj} \gl{def}=Caucasian \gl{def}=chief//
\glft \qu{(The chief sees the White guy)}//
\endgl

\a
\begingl
\gla a=majit i=pupwaale//
\glb 3.\gl{sg}=rest \gl{def}=Caucasian//
\glft \qu{The white guy sleeps}//
\endgl
\xe 


\section{Split S}
Nominative\slash absolutive system for nouns, traditionally. There is an S-split on verbs:

\pex 
\a 
\begingl 
\gla a=xale-o\textsuperscript{Sᵃ}//
\glb 3.\gl{sg}=see-1.\gl{sg}.\gl{obj}//
\glft \qu{He sees me.}//
\endgl 

\a 
\begingl 
\gla sinu-ong\textsuperscript{Sᵖ}//
\glb sick-1.\gl{sg}.\gl{sbj}//
\glft \qu{I am sick/dying.}//
\endgl 

\a 
\begingl 
\gla a=majit//
\glb 3.\gl{sg}=sleep//
\glft \qu{He sleeps.}//
\endgl

\a 
\begingl 
\gla {*}a=thathe//
\glb 3.\gl{sg}=wound//
\glft \qu{(he is wounded)}//
\endgl 
\xe

nominally, P and S are marked the same way, A is distinguished by \textit{ka} , or word order. 
\textit{ka}, however, also occurs with S\textsubscript{A}. It may not occur with S\textsubscript{P}. %todo{check}

\pex 
\a 
\begingl 
\gla A=xale-a ya ka ya//
\endgl 
\a 
\begingl 
\gla go=xale-ko go ka go//
\endgl
\xe

However, nowadays the word order is more often A V P than V P ka A, so that there is no difference between A and S, and we have a nominative/accusative system for nouns as well.

\pex
\a
\begingl
\gla E=holeke nyasi=m li=faati a go=vi thuan//
\glb 1.\gl{sg}=thank.for \gl{ben}-2.\gl{sg}.\gl{poss} \gl{def}.\gl{pl}=word \gl{rel} 2.\gl{sg}=say well//
\glft \qu{I thank you for the words, which you said well.}//
\endgl 
\a
\begingl
\gla e=holeke sika nyasi i=apuli a xhwata//
\glb 1.\gl{sg}=thank.for cigarette \gl{ben} \gl{def}.\gl{sg}=man \gl{rel} bald//
\glft \qu{I thank the bald man for the cigarette}//
\endgl 

\a
\begingl
\gla e=holeke nya-si-m i=xawakhan a siim-ea a go=nya// 
\glb 1\gl{sg}=thank put-hand-2.\gl{sg}.\gl{poss} \gl{def}=dog \gl{rel} mange-3\gl{sg}\gl{rel} 2\gl{sg}=give//
\glft \qu{I thank you for the mangy dog you gave [me]}//
%todo how to treat a 'rel', is it a clitic or not? what is the wordhood of the go, a, e things?
\endgl
\xe
One more thing: the argument markers can disappear, most prominently in lists of things happening, or when it is otherwise clear whom it happens to
