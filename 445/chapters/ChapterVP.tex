\chapter{Verb phrases} 
\label{ChapterVP} 
\is{Verb phrase}

This chapter describes verb phrases. After exploring what constitutes a verb phrase in Vamale (\sectref{sec:VPIntro}), various possible parts of the verb phrase are explored. This omits the head verb, which is discussed in \Cref{ChapterVerbs}. 
Verbs are mostly modified through analytical means, i.e. by adding words in the same phrase. Prefixes and suffixes that modify a verb, somewhat more rare, are treated under \sectref{sec:VPrefix}. This chapter will instead focus on negation (\sectref{sec:negation}), serial verb constructions (\sectref{sec:SVC}), bound verbal parts of the VP, and adverbs (\sectref{sec:Adv}). The last section of this chapter explores comparisons between verb phrases, because the coordinator is derived from a verb (\textit{moo ko} \qu{rest on}) (\sectref{sec:Comp}).

\section{What is a verb phrase?}
\label{sec:VPIntro}
A verb phrase in Vamale will be defined as the syntactic unit depending on a verb, i.e. which moves with the verb, disappears if the verb is replaced by a light verb like \textit{vwa} \qu{do} or \textit{hmwaana} \qu{do like this} (\ref{ex:VP}). The verb phrase includes the head verb, its subject and other arguments, and other verbs that stand in a more or less integrated relationship to the head verb, but are neither coordinated nor subordinated verb phrases. Whether TAM markers, subject indexes and other particles should count as part of the verb phrase may be debated, because they still occur with placeholder morphemes, but since the latter were treated under \Cref{ChapterVerbs} and the former are described in \Cref{ChapterAspect}, we shall concentrate on the elements to the right of the verb. A verb phrase has the following slots available:

\begin{figure}[H]
\raggedright
	\textsc{ass} \quad  TAM \quad \textsc{neg} \quad \textsc{sbj}= \quad  TAM  \quad Pre-V. \\
	 \quad  $\hookrightarrow$   \quad Verb \quad  (Verb) \quad  Manner-V.  \textsc{obj} \quad  \textsc{sbj} \quad  \textsc{obl} \quad  Adv-S\\ %\midrule
	% \quad  \quad  \quad  \quad  \quad  \quad  \quad  \quad \gl{obj} \quad  \gl{sbj} \quad  (\gl{obl}) \quad  (Adv-S)...\\
\end{figure}

Example (\ref{ex:VP}) shows a complex verb phrase, where \textit{vwa-suki-n} \qu{do-price-\gl{nspec}} \qu{to buy} is the main verb from which depends \textit{xaleke} \qu{to see, according to} with its oblique argument \textit{i mwani-n-eong} \qu{my money}. Example (\ref{ex:VP1}) shows the same sentence with the placeholder verb \textit{vwa} \qu{do}.

\ea \label{ex:VP}
\gll e=\textbf{vwa-suki-n} xaleke nyako i=mwani-n-eong\\ 
 1\gl{sg}=do-price-\gl{nspec} look \gl{obl} \gl{def}.\gl{sg}=money-\gl{poss}-1\gl{sg}.\gl{poss}\\ 
\glt \qu{I buy according to my means.}\\\relax
[vamale-181107-jpnelemwa-06\_LR 00:12:24-0:12:26]
\z


\ea\label{ex:VP1}
(The main verb is replaced by \textit{vwa}.)\\
\gll e=\textbf{vwa} xaleke nyako i=mwani-n-eong\\ 
 1\gl{sg}=do look \gl{obl} \gl{def}.\gl{sg}=money-\gl{poss}-1\gl{sg}.\gl{poss}\\ 
\glt \qu{I do it according to my means.}\\\relax
[vamale-181107-jpnelemwa-06\_LR 0:12:26-0:12:28]
\z

\section{Negation}
\label{sec:negation}
\is{Negation}

Negation in the narrow sense, i.e. the process which negates a predicate, uses the particle \textit{cipa}. The existence of something can be negated with the existential negation \textit{cika}, and absence has a dedicated stative verb \textit{ci(e)-a-n}. Disappearing is expressed with the related verb \textit{cii-}, e.g. \textit{cii-le} \qu{they disappear}. Whether the active verb \textit{ciba} \qu{refuse to take} is to be counted in this group of negating morphemes is unclear. Note that all of these forms, as well as the prohibitive \textit{cipii}, start with \textit{ci-}, from POc *\textit{tikai} \qu{not exist} \parencite[88]{lynch_oceanic_2002}, and ultimately PMP *\textit{(q)ati} \parencite[88]{lynch_oceanic_2002}. A table presenting the most important words is presented below, see \Cref{tab:neg_paradigms}.

\begin{table}
	\caption{Negative paradigms}
	\label{tab:neg_paradigms}
	\begin{tabular}{ccccc}
	\lsptoprule
		Number& Person & \textit{cipa} & \textit{ci-} & \textit{cika}\\
		&&\qu{not do} & \qu{be absent} & \qu{\gl{neg}.\gl{exist}}\\
		\midrule
\gl{sg} & 1  & \textit{cipe} & \textit{ci-e(o)}& \textit{cika}\\
		& 2 & \textit{cipa=go} & \textit{cia-ko} & \textit{cika}\\
		& 3 & \textit{cipa=a}& \textit{cia-a}& \textit{cika}\\
		\midrule
\gl{du} & 1 \gl{incl} & \textit{cipa=gasu} & \textit{cia-gasu}& \textit{cika} \\
		& 1 \gl{excl} & \textit{cip=abu}& \textit{cia-bu} & \textit{cika} \\
		& 2 & \textit{cipa=gau}& \textit{cia-gau}& \textit{cika}\\
		& 3& \textit{cipa=lu} & \textit{cia-lu}& \textit{cika}\\
		\midrule
\gl{pl} & 1 \gl{incl} & \textit{cipa=gase} & \textit{cia-gaa}& \textit{cika} \\
		& 1 \gl{excl} & \textit{cipa=be}&\textit{cia-be} & \textit{cika} \\
		& 2 & \textit{cipa=gavwe}&\textit{cia-vwe}& \textit{cika}\\
		& 3& \textit{cipa=le} & \textit{cie-le}& \textit{cika}\\
	\lspbottomrule
	\end{tabular}	
\end{table}

\subsection{Verbal negation \textit{cipa}} 
\is{Negation!Verbal negation}
The negating particle \textit{cipa} is the most common negator. It negates predicates in general, both nominal and verbal ones. This also applies to the other negative forms mentioned above, with the exception of the verb \textit{ci-(e)-a-n} \qu{be absent}. The negating particles occur at the very left of the clause, second only to the assertive \textit{tha} (\ref{ex:neg}). %Like aspect markers and subject indexes, \textit{cipa} can also precede nominal predicates. 

\ea\label{ex:neg}
\gll tha cipa=le=caihnan\\ 
\gl{ass} \gl{neg}=3\gl{pl}=know\\ 
\glt \qu{They don't know.}
\z

Contrary to other Voh-Koné varietes, Vamale assimilates the last vowel in \textit{cipa} to [e] in \textit{e=} \qu{1\gl{sg}}, yielding \textit{cip=e=} \qu{I don't}, as it does with \textit{cala} \qu{when}, \textit{ma} \qu{\gl{subr}}, and the assertive \textit{tha}: there is cross-boundary progressive assimilation, hinting at a proclitical status of \textit{cipa} that \textit{cika} does not have.

\subsection{Existential negation \textit{cika}}
\is{Negation!Existential negation}
The negative existential \textit{cika} is used to express that something does not exist (\ref{ex:cika2}). It is the opposite of \textit{vwa} \qu{exist}. There is no word for \qu{to have} in Vamale; this is expressed with a \gl{exist} X-\gl{poss} construction, i.e. by possessing an NP (\ref{ex:cika1}). \textit{cika} is also used as a more formal, or more marked, equivalent of \textit{ûhû} \qu{no}. %\todo{It might be interesting to take a look at the various articles on negation in Kanak lang in the volume edited by E. Hovdhaugen \& U. Mosel, Negation in Oceanic languages, Munich, lincom Europa (lincom Studies in Austronesian Linguistics 02). In particular the POc neg is mentioned as *tika (if  remember well)



\ea\label{ex:cika2}
\gll cika we\\ 
 \gl{neg}.\gl{exist} water\\ 
\glt \qu{There is no water.}
\z

\ea\label{ex:cika1}
\gll cika nyamaa-n\\ 
 \gl{neg}.\gl{exist} eye-\gl{poss}\\ 
\glt \qu{He is blind (there are no eyes of his).}
\z

\subsection{Other negative expressions and lexical items}
\label{ssec:other_neg}

\textit{cika} is also used to express a ``negative comitative" for inanimate nouns, using \textit{ko-} \qu{on} i.e. \qu{I don't have it on me}, see (\ref{ex:cika}). A dedicated stative verb expresses absence: \textit{ci-eo/cia-ko/ci-ea}. Like the dependent verbs discussed in \sectref{ssec:Verbs_n}, \textit{ci-} is able to take generic \textit{-n}, for non-specific and recently mentioned inanimate participants, see (\ref{ex:cian}). This latter form is also part of \textit{vacian} \qu{lose} which I assume was formerly complex: \textit{va(-)cia-n} \qu{lose (lit. make absent)}. While there are a few other forms that can be decomposed into \textit{va-} and a verb, e.g. \textit{va-tipwe} \qu{drop (lit. make-fall.\gl{tr})} and \textit{va-cut-i} \qu{raise (lit. make-stand-\gl{tr})}, this is not a productive word-formation process any more. %\todo{Is \textit{ci-} historically the negating verb and we have a \textit{ci-ka} \qu{neg-poss}?} 

\ea \label{ex:cika}
 \gll cika ca=mwani ko-ong\\ 
 \gl{neg}.\gl{exist} \gl{indf}.\gl{sg}=money on-1\gl{sg}\\ 
\glt \qu{I have no money with me.}
\z



\ea \label{ex:cian}
\gll li=xhaohmu habu cipa gase=mu vwa mwa ja mu cia-n mwa\\ 
 \gl{def}.\gl{pl}=elder long.ago \gl{neg} 1\gl{pl}.\gl{incl}=\gl{freq} do house \gl{prf} \gl{iter} absent-\gl{nspec} \gl{rep}\\ 
\glt \qu{The elders back in the day, we don't build houses anymore, this has progressively been lost, now.} (lit. \qu{this is absent, now}) {[KL:80]}
\z

Focused negation, used to negate one element rather than another (\qu{not \textit{sticks} but rocks}), is achieved by fronting (\ref{ex:foc_neg}), like other focusing strategies in Vamale: \textit{tha cipa sukaa kavi tapang} \qu{\gl{ass} \gl{neg} sugar but tobacco} \qu{this is not sugar but tobacco}. 

\ea \label{ex:foc_neg}
\gll cipa sinu kavi cika wîî-n xaleke\\ 
 \gl{neg} suffer but \gl{neg}.\gl{exist} strength-3\gl{sg}.\gl{poss} see\\ 
\glt \qu{It doesn't hurt but there's no strength in it, you see.} {[KG:101]}
\z




%@Bril: do I mark clitics everywhere in the examples? does it not suffice to say that they are clitics?

\section{Verbal elements of the verb phrase}

A verb phrase is composed of the head verb and its arguments, possibly other head verbs in the case of serial verb constructions, subordinate verbs, and adverbs (which are discussed in \sectref{sec:Adv}). Among complex verbs, two groups are distinguished by this analysis: serial verb constructions, where several main verbs co-exist in the same phrase, and asymmetrical verb strings, where a head verb is modified by dependent verbs. While there is considerable functional overlap between the two groups, only serial verb constructions describe a complex event by naming the simultaneous or consecutive actions, while verb strings are restricted to modifying a verb as an adverb would. A third group called \qu*{complex verb strings} posited for Nêlêmwa was not found in Vamale. %xxdoileavethisout

%\%%todo{:( Sehr schlechte terminologische Wahl, da die einschlägige (engl.) Literatur SVCs als Untertyp von komplexen Prädikaten sieht... Ich würde auf jeden Fall sie erwähnen, doch auch auf jeden Fall einen anderen Terminus für diesen zweiten Typ verwenden. Wie wär's mit:	Oberbegriff = Complex Predicate	Untertypen = 1) SVC, 2) Asymmetrical String, 3) Other Verbal Strings	Worin besteht übrigens die Asymmetrie? Könnte man für 2) was Anderes  in Betracht ziehen?		Ich verstehe bis jetzt überhaupt nicht, wie sich die SVCs von den anderen zwei komplexen Prädikaten unterscheiden, weder semantisch noch strukturell... :(}

Serial verb constructions (SVC) in Vamale are a common way to describe a complex event or to modify a verb. Serial verb constructions are defined by \textcite[169]{bril_complex_2004} as:
\begin{itemize}
	\item constituting a single predication/clause
	\item constituting a single prosodic entity
	\item sharing syntactic arguments, a single set of pronouns and arguments
	\item sharing TAM markers
	\item sharing illocutionary force
	\item sharing polarity
	\item referring to aspects of a single event
\end{itemize}
To this, \citeauthor{bril_complex_2004} adds a criterion of contiguity for Nêlêmwa, with directionals such as in (\ref{ex:contig}) counting as the only valid disruption of the verb chain \parencite[169]{bril_complex_2004}. This may be present in Vamale, but could not be distinguished safely from primed calques in elicitation contexts, and was not (to the knowledge of the author) found in unprompted speech. %This is also attested in Vamale (\ref{ex:contig2}).

\ea
%\ea
\label{ex:contig}
\gll I u â kuut \textbf{mwadu} axi axamalileny ebai nu malileny.\\ 
 3\gl{sg} \gl{prf} leave stand down.there see these2.\gl{dist} then coconut these2.\gl{dist}\\ 
\glt ‘He leaves, stands down there, sees these two coconuts.’
%\ea\label{ex:contig2}
%
%\gll xx
%
% 
%
%\glt
%
%
\z

\noindent There are three types of serial verb constructions common in Oceanic languages: \textit{nuclear}, \textit{core}, and \textit{adjunct} serialization \parencite[168]{bril_complex_2004}. 

\emph{Nuclear serialization} features contiguous verbs (sVVo) and are the most common type in New Caledonian languages. 
\emph{Core-layer serialization} has non\hyp contiguous verbs, either with the same subject, or with switch-subject constructions: sV sV(o) (I run I catch [him]) or sVo (s)V (I strike him [he] dies) \parencite[4]{bril_complex_2004c}. These meanings would be expressed in two clauses in Nêlêmwa, ``serial verb constructions have one single prime argument and one single patient [sVV(V)(o)]; [sV sV] patterns would constitute two independent clauses with a pause inbetween, not a serial construction." \parencite[168]{bril_complex_2004} They are not attested in Vamale either. In \emph{adjunct serialization}, a head verb has a second, adjunct verb. These adjunct verbs cannot head a verb phrase on their own. In this grammar, they are called ``manner verbs" and described in \sectref{ssec:MannerV}, as part of asymmetric verb strings. 
\ea \label{ex:siva_taeke}
\gll a=siva taeke i=oot\\ 
 3\gl{sg}=attach do.badly \gl{def}.\gl{sg}=rope\\ 
\glt \qu{He attached the rope badly.}\\\relax [vamale-181107-jpnelemwa-05: 00:01:06-00:01:07]
\z


\ea 
(Possibly a word play on the previous example)\\
\gll a=siva vee-ke i=oot\\ 
 3\gl{sg}=attach fuck-\gl{tr} \gl{def}.\gl{sg}=rope\\ 
\glt \qu{He attached the rope badly (by messing around).} {[vamale-181107-jpnelemwa-05: 00:01:11-00:01:12]}
\z

Another type of verb phrase including various verbs are asymmetric strings, called ``complex predicates"\footnote{As predicates can be formed with other word classes as well, I use a more explicitly verb-focused term.} elsewhere \parencite[168]{bril_complex_2004}. These can include a head and an adjunct as in (\ref{ex:siva_taeke}) (manner verbs are adjuncts), a head and a modal verb, e.g. \textit{thapoke mapeke} \qu{begin bright}, \qu{it is beginning to dawn} (preverbs are modal verbs), or a compound whose meaning is different from the meanings of its components \parencite[168]{bril_complex_2004}. A Vamale example of the latter would be \textit{moo han} \qu{stay go} \qu{be a nomad}, or \textit{han thêên} \qu{walk run} \qu{chase after}, see \sectref{ssec:CompoundV}. In any case, where serial verb constructions feature several main verbs, an asymmetric verb string is characterized by a head verb and a dependent one. Two classes of these are the preverbs and manner verbs discussed under \sectref{ssec:Preverbs} and \sectref{ssec:MannerV}, respectively.
%Anything that does not correspond to either of these criteria is not a SVC nor an asymmetrical verb string, but a xxverb string.  xxVerb strings are a diverse group including grammaticalized, or specialized verbs. Specialized verbs are still recognizably verbal but show no signs of categorical or morphophonological change, they are just verbs used for a function \parencite[194]{bril_complex_2004}. Grammaticalization shows semantic change. %vamale-181107-jpnelemwa-06: 00:05:53-00:06:03

\subsection{Serial verb constructions}
\label{sec:SVC}
\is{Verbs!Serial verb construction}

Serial verb constructions in Vamale are commonly used to link a motion to a purpose (\ref{ex:ta_controler}), with the motion verb always first, e.g. \textit{han moo} \qu{go stay} \qu{go to stay somewhere}. Most SVCs combine two verbs, but there are longer ones, e.g. (\ref{ex:wati}) and (\ref{ex:tame}). Both the motion and purpose are carried out by the same subject.
 
\ea \label{ex:ta_controler}
 % \langinfo{}{}{} 
\gll ta \textit{contrôler} li=carton ko-n jo, \textit{contrôler} meeka-n han han na bwaa ...\\ 
 go.up check \gl{def}.\gl{pl}=box \gl{obl}-\gl{nspec} chicken check everything-\gl{nspec} go go \gl{dem} \gl{ipfv} \\ 
\glt \qu{She went up to check the boxes of frozen chicken, check everything bit by bit/check everything while walking through it, it's still \ldots
} {[KG:473-747]}
\z


\ea\label{ex:wati}
\gll \ldots ma le=bwa hup-e wati thapi-bwa-le, watii-le koma bwa vwa ca=le=vwa-suki-aman\\ 
 \ldots \gl{subr} 3\gl{pl}=\gl{ipfv} go.down-\gl{dir.cp} chase.after smash-head-3\gl{pl}.\gl{poss}, chase.after-3\gl{pl} to \gl{ipfv} do \gl{indf}.\gl{pl}=3\gl{pl}=do-price-thing\\ 
\glt \qu{When they came north, chased and slew them, chased them to them [lit. do something that they would pay the price punish].} {[DT:12]}
\z


\ea\label{ex:tame}
%\langinfo{}{}{}  
\gll kona e=ta-me ta-me thaloot hup-wa sohmun\\ 
 \gl{cnj} 1\gl{sg}=go.up-\gl{dir.cp} go.up-\gl{dir.cp} appear go.down-\gl{rep} study\\ 
\glt \qu{Then I would go up, up, come out (on top of the hill), walk down again (on the other side) to study.} {[PE2:16]}
\z

A motion verb \textit{han} \qu{go}, \textit{hut} \qu{go down}, \textit{ta} \qu{go up}, \textit{hnuut} \qu{move upstream} etc. may also be introduced because it can host a centripetal suffix \textit{-me} (\ref{ex:hame_kito}), unlike non-motion verbs. This allows the construction to have a centripetal suffix. In (\ref{ex:oro}), \textit{hame} \qu{come} means that the drinks are aligned in such a way that the tastes change as one moves along them. Crucially, the movement does happen, meaning the motion verb is a full verb and not grammaticalized (for which see \sectref{ssec:gramm_mot_v}). 

\ea \label{ex:hame_kito}
\gll kona sili sahmwa ha-me sili sahmwa ha-me\\ 
 \gl{cnj} sew the.other.way go-\gl{dir.cp} sew the.other.way go-\gl{dir.cp} \\ 
\glt \qu{And then he backs up, backs up.} {[KG:467]}
\z

\ea \label{ex:oro}
\gll thapoke cuut \textit{aligner} li=\textit{parfum} ha-me hmwaani, \textit{tulem} li, \textit{aligner} ha-me li=\textit{oro} li=joakan aman\\ 
 begin stand align \gl{def}.\gl{pl}=flavor go-\gl{dir.cp} like.this tulem \gl{def}.\gl{pl} align go-\gl{dir.cp} \gl{def}.\gl{pl}=oro \gl{def}.\gl{pl}=big thing\\ 
\glt \qu{(So I) start and stand there and align the flavors while coming (out of the container), the \textit{tulem} sodas, the\ldots align the \textit{oro} sodas, the big ones.} {[KG:277-279]}
\z

%\ex
%
%\langinfo{}{}{} KG:399
%\gll a hân thêên tame a hân ah oui ka thêên hame
%
%
%
%\glt ah mais le conard il descend tu dis il descend ok mais il descend..
%
%
%\z

Motion verbs can also co-occur in an SVC, to describe the path of the subject more precisely. In (\ref{ex:thana_saxhuti}), the motion is first specified as aimless and slow (strolling), then a Ground is linked to it (beach) with a precise motion verb (follow). In (\ref{ex:tame}), however, the speaker describes in detail a long journey to school, with studying as a purpose in the end.  

\ea \label{ex:thana_saxhuti}
\gll a e-thana saxhuti hmeewan\\ 
 3\gl{sg} \gl{mid}-stroll follow sand\\ 
\glt \qu{She goes for a stroll along the beach.} {[181107 p.96]}
\z

Another use of SVCs is to link an action not to its direction, as described above, but to its manner. In (\ref{ex:vwa_xaleke}), a rather abstract action is done while considering the available funds.

\ea
\label{ex:vwa_xaleke}
%\langinfo{}{}{}  
\gll e=vwa xaleke nyako li=mwani-n-eong\\ 
 1\gl{sg}=do look for \gl{def}.\gl{pl}=money-\gl{poss}-1\gl{sg}.\gl{poss}\\ 
\glt \qu{I do (buy) according to my means.}\\\relax [vamale-181107-jpnelemwa-06\_LR 00:12:24-0:12:26]
\z

Other serial verb constructions are in muddy waters between true, clear SVCs and grammaticalized constructions (which would make the concerned construction an asymmetric verb string). Compare \textit{han sate-n} \qu{go,leave be.different-\gl{nspec}} \qu{leave separately} and \textit{saten han} \qu{leave afterwards}.\footnote{vamale-181107-jpnelemwa-06: 00:05:11- 00:05:46} While the first meaning is rather transparent, the second is more lexicalized. However, many common SVCs accumulate connotations, while retaining an ambiguity with the literal meaning.

\subsection{Asymmetrical verb strings}
\is{Verb phrase!Asymmetrical verb string}

The following section will first introduce so-called preverbs, bound verbs that modify the head verb by preceding it, amongst which a prominent example is \textit{se-me} \qu{all together (lit. same-all)}. Other bound parts of the verb phrase may include manner verbs, which are bound like the pre-verbs, but follow the head verb. TAM particles precede the predicate, not the verb phrase, and are discussed under \Cref{ChapterAspect}.

\subsubsection{Preverbs}
\label{ssec:Preverbs}
\is{Verb phrase!Preverbs}

Vamale has a series of elements which cannot nowadays be the head of a verb phrase. Some have nominal origins, like the ones in \Cref{tab:preverb}, while others have free verbal counterparts, often with a transitive meaning e.g. \textit{vataan} \qu{each}, \textit{vathanke} \qu{do separately, separate}. Preverbs occur directly before the verb root. The commonly used members include \textit{vataan} \qu{each} (\ref{ex:vatan}), \textit{xadaa} \qu{on the other hand} (\ref{ex:xadaa}), \textit{daa} \qu{do first} (\ref{ex:daa}), \textit{xhose} \qu{repeat} (\ref{ex:xhose}), \textit{xhwat} \qu{a little bit}\footnote{Compare \textit{xhwat apuli} \qu{little man}, \textit{xhwatin} \qu{be small}, \textit{xhwatiike} \qu{do slowly}.} (\ref{ex:xhwat}), \textit{xhopwe} \qu{on top of that}, \textit{balan} \qu{just} (\ref{ex:balan}). In addition to having different word-class status, some preverbs have homophonous equivalents with a different distribution, e.g. \textit{vataan}, which also occurs before nouns, and \textit{balan} exists as an aspect marker (\sectref{sec:balan}). The two preverbs \textit{mee} \qu{all} and its complex form \textit{se-me} \qu{all together} are discussed in the two sections to come, as examples of this multi-functionality.

\begin{table}
	\centering
	\caption{Preverbs and their free counterparts}
	
	\begin{tabular}{lll}
		\lsptoprule
		Form & Free form meaning & Preverb meaning\\
		\midrule
		\textit{xhwat} & \qu{small piece}& \qu{do a bit}\\
		\textit{balan} & \qu{part of long object}& \qu{just do; do anyway}\\
		\textit{xadaa} & \qu{turn to do something} & \qu{on the other hand, do unexpectedly}\\
	\lspbottomrule
	\end{tabular}
	\label{tab:preverb}
\end{table}


\ea \label{ex:vatan}
\gll {\ob}hmwaka li=buke{\cb} {\ob}tha=abe={\ob}vataan {\ob}vwa buke{\cb\cb\cb} {\ob}tha=abe=vataan cami{\cb}\\ 
 like \gl{def}.\gl{pl}=flower \gl{ass}=1\gl{pl}.\gl{excl}=each do flower \gl{ass}=1\gl{pl}.\gl{excl}=each plant\\ 
\glt \qu{Like the flowers, we each grow flowers, we each plant them.} {[AG1:68]}
\z

\ea \label{ex:xhose}
\gll xhose vii!\\ 
 repeat say\\ 
\glt \qu{Say (that) again!} {[GL:41]}
\z 

\ea \label{ex:xadaa}
\gll xadaa-go ma go=saxhuti eca=se\\ 
 turn-2\gl{sg} \gl{subr} 2\gl{sg}=narrate \gl{indf}.\gl{sg}=other\\ 
\glt \qu{It's your turn to tell another {[}story].} {[traditional end of a story]}
\z
\ea
\gll ko go={\ob}{\ob}xadaa siva{\cb} nya-xada{\cb} go xadaa hmwaani\\ 
 \gl{cnj} 2\gl{sg}=otherwise attach towards-up.there then otherwise like.this\\ 
\glt \qu{Then you attach it one way, and then, another way, you do like this.} {[KG:162]}
\z

\ea \label{ex:daa} %i'm guessing that tabe has i se carton as an object, regardless of da, and since it can be replaced by vwa, da is outside it. but hnyaut pu is an adverb and not tied to tabe, it could also come with vwa, so it is another part of the VP. da is closer to tabe than to hnyautpu, and both modify tabe.
\gll  go=bwa juu {\ob\ob}da {\ob}tabe i=se \textit{carton}{\cb\cb} {\ob}hnya-ut pu{\cb\cb} go=bwa fe i- aaa cana\\ 
 2\gl{sg}=\gl{ipfv} real first lift \gl{def}.\gl{sg}=one box \gl{prox}-move.down ground 2\gl{sg}=\gl{ipfv} take \gl{def}.\gl{sg} \gl{expl} vagina\\ 
\glt \qu{You just start by lifting one box from the ground, then you take the (next), aah shit...} {[KG:285]}
\z

\ea \label{ex:xhwat}
\gll lu=vwa ma lu=\textbf{xhwat} cu-vathan-ke li=see vuman nyu nyala ka muu-hni \\ 
 3\gl{du}=do \gl{subr} 3\gl{du}=a.bit stand-individually-\gl{tr} \gl{def}.\gl{pl}=same group fish there \gl{sbj} \gl{dem}.\gl{du}-\gl{prox}\\ 
\glt \qu{The two did so that they quickly surrounded the same school of fish there.} {[GP:57]}
\z

\ea \label{ex:balan}
\gll na i=s-ung tha balan hmwaani\\ 
 \gl{dem} \gl{def}.\gl{sg}=hand-1\gl{sg}.\gl{poss} \gl{ass} continue like.this\\ 
\glt \qu{It's my hand, it stayed [stiff] like this.} {[V]}
\z

\subsubsubsection{\textit{se-me(e)} \qu{together}}
\label{ssec:se-me}

The stative verb \textit{se} \qu{be one\slash same} has two meanings, depending on the context. One is the verbal form, predicatively used, meaning \qu{be alone, be one}. This is a comparatively rare function, shown in (\ref{ex:se}).
%\todo{What’s the difference betw. Mee and meeka ?}
\ea \label{ex:se}
\gll i=apuli a= se-a\\ 
 \gl{def}.\gl{sg}=person \gl{rel}= one-3\gl{sg}\\ 
\glt \qu{the person who is alone}
\z

When following a singular article, it means \qu{the other} or \qu{another}, see (\ref{ex:other_se1}). This is discussed in more detail in \sectref{sec:ise}.

\ea \label{ex:other_se1}
\gll hê vwa li=been xada a= mu moo ma ca-se thamo lu=moo ma ca-se xayu\\ 
 yes \gl{exist} \gl{def}.\gl{pl}=peer up.there \gl{rel}= \gl{iter} stay \gl{com} \gl{indf}.\gl{sg}-other woman 3\gl{du}=stay \gl{com} \gl{indf}.\gl{sg}-other man\\ 
\glt \qu{Yes there are some, up there, he'll stay with some other woman, she'll stay with some other man.} {[AG1:339]}
\z

The last meaning of \textit{se} is \qu{be one\slash same}. It is used either as a verbal predicate, with singular nouns (\ref{ex:se1}) as well as plural ones (\ref{ex:same_yam}). It can also be used before definite nouns to mean \qu{the same}, as in (\ref{ex:other_se2}) and (\ref{ex:se3}). 


\ea\label{ex:se1}
\gll meeka li=yavo, a=taemwi ka i=apuli a= \textbf{se}\\ 
 all \gl{def}.\gl{pl}=fishing.line 3\gl{sg}=grab \gl{sbj} \gl{def}.\gl{sg}=man \gl{rel}= same \\ 
\glt \qu{All the fishing rods are held by a single man.} (lit. \qu{All the fishing rods, a single man holds (them).}) {[J5:70]}
\z


\ea\label{ex:same_yam}
\gll lu=xhajake li=uvu a= see\\ 
 3\gl{du}=eat.starchy \gl{def}.\gl{pl}=yam \gl{rel}= one\\ 
\glt \qu{They eat the same yams.} {[J5:62]}
\z


\ea\label{ex:other_se2}\
\gll lu=moo ca se mwa-n-lu\\ 
 3\gl{du}=stay in one house-\gl{poss}-3\gl{du}.\gl{poss}\\ 
\glt \qu{They stay in the same house.} (lit. \qu{They stay in one house of theirs.}) {[J5:58]}
\z


\ea\label{ex:se3}
\gll i=that cipa xa-sivu ca la a= se la\\ 
 \gl{def}.\gl{sg}=wind \gl{neg} \gl{hab}-blow in location \gl{rel}= one location\\ 
\glt \qu{The wind does not always blow in the same place.} (lit. The wind is not a constant blower in a place that is one/the same place) {[J4:14]}
\z

Finally, \textit{se} can form a compound with the preverb \textit{me(e)} \qu{be all}; the result is both a pre-verb meaning \qu{do something all together} and a ``pre-noun" meaning \qu{all of X} (\ref{ex:se-me}--\ref{ex:meele}). Note that the former function is only attested for the compound \textit{se-me}, and only with free pronouns. %Adverbs tend not to precede nouns in noun phrases %(but see \textit{meeka-n} in (\ref{ex:meele})), the promiscuity of \textit{see-me} however is reminiscent of an adverb. 


\ea\label{ex:se-me}
%\langinfo{}{}{}  
\gll see-me gaa, gase=see-me vwa ka see-me gaa\\ 
 same-all 1\gl{pl}.\gl{incl} 1\gl{pl}.\gl{incl}=same-all do \gl{sbj} same-all 1\gl{pl}.\gl{incl} \\ 
\glt \qu{All together, we all do this together.} {[J6:1]}
\z


\ea \label{ex:me_abe}
\gll m=abe=bwa vwa \textit{nettoyage} h=abe thai li=\textit{vaisselle}-ea, ya, ja \textbf{me} abe mwa\\ 
 \gl{subr}=1\gl{pl}.\gl{excl}=\gl{ipfv} do cleaning \gl{top}.\gl{rep}=1\gl{pl}.\gl{excl} pick.up \gl{def}.\gl{pl}=dish-3\gl{sg}.\gl{poss} \gl{excl} \gl{prf} all 1\gl{pl}.\gl{excl} \gl{rep}\\ 
\glt \qu{When we do the clean up we'll pick up her dishes, well - we all as well.} {[AG1:428]}
\z


\ea\label{ex:meele}
\gll le=ha-me ka (*mee le) meeka le\\ 
 3\gl{pl}=go-\gl{dir.cp} \gl{sbj} all 3\gl{pl} all 3\gl{pl}\\ 
\glt \qu{They all come.} {[vamale-181127-jp\_nelemwa-1: 00:01:11-00:01:17]}
\z

\subsubsubsection{\textit{mee} \qu{all}}
\label{ssec:mee}

The morpheme \textit{me(e)} \qu{all} is likely the origin of the derived form \textit{mee-ka-n} \qu{everything, everywhere}\footnote{\textit{meeka-n} was probably structured \textit{mee ka-n}, with the \gl{s}/\gl{p}-marking clitic \textit{ka} \qu{\gl{abs}} found in nominalizations, further discussed under \sectref{kan}.} suggests it was originally a free verb (see \sectref{ssec:meekan} on the adverb and \sectref{ssec:meeka-n} on the quantifier). However, \textit{me(e)} is not attested as a head verb in Vamale, and western Voh-Koné languages only feature the nominalized form \textit{meeka-n \goodtilde meena-n} \parencite[211]{rivierre_bwatoo_2006}; no independent verb \textit{me} is attested in the language family today. \textit{me(e)} is mostly used preverbally, see (\ref{ex:me3}) and (\ref{ex:me1}), but it is also attested before a prepositional phrase, as in (\ref{ex:me1}). Interestingly, \textit{me} does not only add a meaning of \qu{to all do}, but can also, at least when the subject is singular, signify implicit plural referents ``all-ness", i.e. that all members of the referred-to group of referents are concerned (\ref{ex:me2}, \ref{ex:mePV}). This was only attested twice.
As a prefix, it occurs in combinations with demonstrative pronouns, as for \textit{me-ehni} \qu{all those} (\sectref{ssec:mehni}). This double occurrence as a pre-verb and a prefix on pronouns is unique in the language.
This grammar distinguishes the pre-verb and the prefix \textit{me(e)}, %like \textit{mwa} (\Cref{sec:mwa}) and \textit{juu} (\Cref{sec:WCIntensifiers}), 
and hypothesizes a verbal origin based on the derived form \textit{meeka-n}'s morphology.


\ea\label{ex:me3}
\gll i=se a= lu=\textbf{mee} hup-e ya a=bwa ta xale\\ 
 \gl{def}.\gl{sg}=other \gl{rel}= 3\gl{du}=all go.down-\gl{dir.cp} 3\gl{sg} 3\gl{sg}=\gl{ipfv} go.up look\\ 
\glt \qu{The other who came with (lit. the other that the two came down \textbf{together}) went up [in the gas station] to look around.} {[KG:471]}
\z


\ea\label{ex:me1}
\gll gaa=\textbf{me} ca i=mwani mwani mwani, cama li=xhaohmu habu, le, tha cika mwani-n-le\\ 
 1\gl{pl}.\gl{incl}=all in \gl{def}.\gl{sg}=money money money when \gl{def}.\gl{pl}=elder long.ago 3\gl{pl} \gl{ass} \gl{neg}.\gl{exist} money-\gl{poss}-3\gl{pl}.\gl{poss}\\ 
\glt \qu{We all are about money, money, money, but the elders back then, they didn't have money.} {[KP:102]}
\z

\ea \label{ex:me2}
\gll go=mee vwa ko na yamaan-go juu en-go, go, hê \\ 
 2\gl{sg}=all do because \gl{dem} unmotivated-2\gl{sg} real fine-2\gl{sg} 2\gl{sg} yes\\ 
\glt \qu{You did all that [punishments] because you were fed up [with school], you're a real fine one, yes.} {[PE2:30]}
\z


\ea \label{ex:mePV}
\gll ma gavwe vwa ehni a me vi\\ 
 \gl{subr} 2\gl{pl} do \gl{dem} \gl{rel}.3\gl{sg} all say\\ 
\glt \qu{May you do all he says.} (lit. May you do this, that he will all say) {[HC19:7]}
\z

\subsubsection{Manner verbs}
\label{ssec:MannerV}
\is{Verbs!Manner verbs}
\is{Verb phrase!Manner verbs}
\noindent
``Manner verbs" are, so to speak, halfway between full verbs and more bound morphemes like the prefixes described in \sectref{sec:Manner}. They can neither occur alone, nor can they occur in the first position of the verb phrase, or be fronted there. They cannot take arguments or subjects, and modify the head verb. They cannot add arguments to a verb phrase either. 
Members include \textit{thuan} \qu{do well} (\ref{ex:MannerV}), \textit{tatu} \qu{do quickly}, %is a suffix, innit, it's less free than the others \textit{-tae} 'do badly', 
\textit{sisipo} \qu{do together},\footnote{Diachronically related: \textit{nya-sipo-ke} \qu{put-together-\gl{tr}}.} \textit{xhwatiike} \qu{do quietly, softly}.


\ea  \label{ex:MannerV}
\gll e=holeke thuan i=vaaya a= gavwe=vwa sisipo\\ 
 1\gl{sg}=thank do.well \gl{def}.\gl{sg}=work \gl{rel}= 2\gl{pl}=do do.together\\ 
\glt \qu{I thank you for the work you did together.}
\z


\subsubsection{Grammaticalized motion verbs}
\label{ssec:gramm_mot_v}

A motion verb can come after verbs describing another action, but the former then has other functions than to contribute its lexical meaning. The motion verb may serve to express a spatial boundary as in (\ref{ex:xaleke_hmaca}) (compare this to (\ref{ex:xaleke_seen}), where a similar role is played by \textit{seen} \qu{border}), or to express a temporal boundary, such as in (\ref{ex:hame_naen}), where \textit{ha-me} \qu{go-\gl{dir.cp}} \qu{come} means \qu{until now, up to now}. Since nothing actually moves and the motion verb has no subject, the semantically idiosyncratic sequence of a main verb and a motion verb that modifies the former is an asymmetric verb string. Koch calls this function ``associated motion" \parencite{koch_associated_2021}.

\ea\label{ex:xaleke_hmaca}
\gll moo cahni gase=xaleke hma-ca-mwa xahan ko i=jahoot\\ 
 rest here 1\gl{pl}.\gl{incl}=see arrive-go.up-\gl{rep} over.there on \gl{def}.\gl{sg}=river\\ 
\glt \qu{From here we can see up to the river (and beyond).} {[181107 p.96]}
\z

\ea\label{ex:xaleke_seen}
% \langinfo{}{}{}
\gll moo cahni gase=xaleke seen xahan ko i=jahoot\\ 
 rest here 1\gl{pl}.\gl{incl}=see border over.there on \gl{def}.\gl{sg}=river\\ 
\glt \qu{From here we can see until the river (but not beyond).} {[181107 p.96]}
\z

\ea \label{ex:hame_naen}
\gll xethoo na la la, ha-mwa-me naen bwa vwa \textit{épuisettes}\\ 
  landing.net \gl{dem} be.here be.here go-\gl{rep}-\gl{dir.cp} now \gl{ipfv} \gl{exist} landing.net\\ 
\glt \qu{Landing nets, it's a recent thing, it has come about now that there are landing nets.} {[KL:188]}\\ 
\z

\subsubsection{Iterative \textit{han}}
\is{Verb phrase!Iterative \textit{han}}
\textit{han} is used to express iterativity, and can be used before, see (\ref{ex:modal_han1b}) and (\ref{ex:modal_han1a}), or after the other verbs (\ref{ex:modal_han2}). This is a rather grammaticalized function and although it latches onto serial verb constructions, the resulting whole qualifies as an asymmetric verb string (composed of an SVC and a modal verb).


\ea\label{ex:modal_han1b}
\gll gaa=vwa ma fa-pupwaale, gavwe=\textbf{han} pala thuan\\ 
 1\gl{pl}.\gl{incl}=do \gl{subr} speak.language-European 2\gl{pl}=go talk do.well\\ 
\glt \qu{...we busy ourselves with speaking French, you have come to speak it well.} {[HC19:42]}
\z


\ea\label{ex:modal_han1a}
\gll ko tha=ga=\textbf{han} pa xaleke naen\\ 
 \gl{cnj} \gl{ass}=1\gl{pl}.\gl{incl}=go \gl{pfv} see now\\ 
\glt \qu{And we see by now,...} {[HC19:40-42]}
\z


\ea \label{ex:modal_han2}
%\langinfo{}{}{}  
\gll e=bwa mu tena ha-mwa\\ 
 1\gl{sg}=\gl{ipfv} \gl{iter} hear go-\gl{rep}\\ 
\glt \qu{I heard about it all along.} {[Tipije]} 
\z

\section{Adverbs}
\label{sec:Adv}
\is{Adverbs}

Vamale does not productively derive adverbs from verbs, and instead commonly forms adverbial clauses with \textit{ca-n} \qu{in-\gl{nspec}} (see \sectref{sec:adv_can}). Some of these have become established expressions, e.g. \textit{can hawân} \qu{vis-à-vis (lit. in its possession\slash visible dependency)}.
Similarly to manner verbs, adverbs can modify a verb and are optional. Adverbs can also modify nouns, as well as occur alone at the edge of the clause, see (\ref{ex:adv}), where \textit{naen} \qu{now(adays)} is used at the end of the first clause, and after the conjunction, at the beginning of the following clause. 

\ea \label{ex:adv}
\gll  cipa hmwakan \textbf{naen}, ko \textbf{naen} a=xaahni eca=\textit{lit} a... eca=\textit{matelas} a \\ 
 \gl{neg} like-\gl{nspec} now \gl{cnj} now 3\gl{sg}=look.for \gl{indf}.\gl{sg}=bed \gl{cnj} \gl{indf}.\gl{sg}=mattress \gl{cnj}\\ 
\glt \qu{It's not like now, because now he looks for some bed or some mattress (instead of sleeping on a pandanus mat).} {[KP:40]}
\z

Semantically, most adverbs situate the action in time and space. They constitute a relatively small class. Among the temporal adverbs are \textit{naen} \qu{recently, now, later today}, \textit{xahmaen} \qu{tomorrow}, and \textit{can-bwen} \qu{yesterday (lit. at-night)} as well as \textit{mati} \qu{earlier today} and \textit{jimin} \qu{last night past bedtime}. Vamale has no dedicated words for \qu{always}, \qu{often}, \qu{sometimes}, or \qu{rarely}. These meanings are expressed using the frequentative particle, the habitual prefix, and other forms (\ref{ex:never}). For occasional occurrences, no matter the frequency, \textit{calibeen} \qu{sometimes, not always} is used. 

\ea \label{ex:never}
\gll  xa-pala hnyana kavi pa cipa=a pala\\ 
 \gl{hab}-talk constantly but \gl{pfv} \gl{neg}=3\gl{sg} talk\\ 
\glt \qu{He talks all the time but now he doesn't.} {[B2:145]}
\z

Spatial adverbs are as numerous as movement verbs, as they form with the locative prefix \textit{xa-}, or \textit{hnya-} \qu{towards} on a movement verb. Compare \textit{xa-hut} \qu{\gl{loc}.\gl{adv}-go.down (below)} and \textit{hnya-ut nya-xa-hut} \qu{down towards the general area down there}. There are also spatial adverbs not derived from verbs, such as \textit{xala-n} \qu{under} or \textit{paathabu-n} \qu{before, in front of}. There are other spatial adverbs derived from verbs that do not denote movement. One prominent example is \textit{meeka-n} \qu{everywhere}, ultimately derived from the stative verb \textit{mee} \qu{all} (further discussed in \sectref{ssec:meekan}).
%If \textit{xhwan} means \qu{barely, almost} and \textit{balan} means \qu{just now, almost, only a little} Either I fundamentally misunderstood something or their words mean \qu{close to a boundary} without clarifying on which end they are.

 
 \subsection{Spatial adverbs and proximity}
 \label{ssec:prox_Adv}
 \is{Space!Spatial adverbs}
 \is{Space!Proximity}
 
 Vamale distinguishes three main degrees of proximity via adverbs, which are unrelated to the demonstratives in \sectref{ssec:Prox}. The adverbs \textit{xa-hut} \qu{down there}, \textit{xa-da} \qu{up there}, and \textit{xa-han} \qu{over there} form the basis of this system, and do not by themselves specify the distance of the referent. 
 
 The closest ring, \textit{nya-ut/nya-da/nya-an} \qu{just there}, is composed of \textit{(h)nya} \qu{put, give} and a (sandhied) form of a motion verb. It is often accompanied by \textit{la} \qu{be.here}, i.e. \textit{nya-ut la} \qu{right here (below me)}. Things in this ring are at a hand's reach or figuratively so.  \textit{nya-an xa-han} \qu{farther than over there} is farther than either simple \textit{nya-an} or anything previously qualified as \textit{xa-han}. The last ring contains a repetitive morpheme \textit{mwa} \qu{\gl{rep}}: \textit{nya-a-mwa xa-han} \qu{even further over there}.
 %
 \subsection{\textit{meekan} \qu{everywhere}}
 \label{ssec:meekan}
A related form of \textit{mee} \qu{all} described above (\sectref{ssec:mee}) is the adverb \textit{meekan} \qu{everywhere}. It appears at the fringes of constituents: ex. (\ref{ex:everywhere}) stands between a verb and its adjunct \textit{can mwa} \qu{in the house}, and in (\ref{ex:meekan_adv}) it modifies a prepositional phrase. Unlike the quantifier \textit{meeka-n} (\sectref{ssec:meeka-n}), \textit{meekan} does not shed its \textit{-n} before specific noun phrases.
 
 \ea
 \label{ex:everywhere}
 % \langinfo{}{}{} 
 \gll vwa meekan ca i=mwa\\ 
  \gl{exist} everywhere in \gl{def}.\gl{sg}=house\\ 
 \glt \qu{It was everywhere in the house.} {[KG:26]}
 \z
 
 
 \ea \label{ex:meekan_adv}
 % \langinfo{}{}{} 
 \gll na wanke mae can mwa meekan\\ 
  \gl{dem} change light in house everywhere\\ 
 \glt \qu{The lights had changed everywhere in the house.} {[KG:84]}
  \z
 
\section{Comparison}
\label{sec:Comp}
\is{Verb phrase!Comparison}
Verb phrases can be coordinated for comparative purposes using \textit{moo ko}, literally \qu{rest on}, see (\ref{ex:moko}). 

\ea \label{ex:moko}
\gll gase=xadaa han-mwa can saten\\ moo-ko i=hun-moo-gaa\\ 
 1\gl{pl}.\gl{incl}=however walk-\gl{rep} \gl{adv}.\gl{subr} differently \\ \gl{cpr} \gl{def}.\gl{sg}=manner.\gl{nmlz}-stay-1\gl{pl}.\gl{incl}.\gl{poss}\\ 
\glt \qu{We, however, walk yet differently from our [real] culture.} {[RP:48]}
\z 

\textit{moo-ko} also means \qu{to come from}, as in (\ref{ex:mo_from}) (further discussed in \sectref{ssec:mo_ko}). Another construction yet is prefaced by \textit{thaloo} \qu{two} and means \qu{be of two sorts}, as in (\ref{ex:thaloo_moo_koon1}) and (\ref{ex:thaloo_moo_koon2}). Note that \textit{moo} \qu{stay} takes on a meaning of \qu{be} in the common nominalized form \textit{hun-moo} \qu{manner.\gl{nmlz}-stay} \qu{culture, nature}.

\ea\label{ex:mo_from}
\gll hmwakan cama thamo ha-me hoot, na le=o \textit{enseigner} ma api a=ha-me moo hoot\\ 
 perhaps if woman go-\gl{dir.cp} be.far \gl{dem} 3\gl{pl}=\gl{irr} teach if \gl{comp} 3\gl{sg}=go-\gl{dir.cp} stay be.far\\ 
\glt \qu{Perhaps if a woman comes from afar, they [the local women] will teach her, if she comes from afar.} {[AG1:429-430]}
\z


\ea\label{ex:thaloo_moo_koon1}
(Proverb)\\
\gll vi ka vwa, tha juu e-thaloo moo koo-n\\ 
 say \gl{cnj} do \gl{ass} really \gl{mid}-two stay on-\gl{nspec}\\ 
\glt \qu{Saying and doing are two different things.}
\z

\ea\label{ex:thaloo_moo_koon2}
\gll hê ena thaloo moo koo-n\\ 
 yes \gl{dem} two stay on-\gl{nspec}\\ 
\glt \qu{Yes exactly, there are two kinds [of bees].} {[KM:29]}
\z
