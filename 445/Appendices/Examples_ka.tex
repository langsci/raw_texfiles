\chapter{Examples for \textit{ka}} 
\label{Appendix_ka} 

\textit{ka}, in its unmarked, obligatory position, is used to disambiguate the S and the O in a VOS structure. It is ungrammatical to use example \ref{ex:noka} without \textit{ka} preceding \textit{li xavwatau}.

\pex
\a \label {ex:wifka}
\begingl
\gla le=xale-a i=puudo ka li=xa-vwa.tau	//
\glb 3.\gl{pl}=see-3.\gl{sg} \gl{agt} \gl{def}.\gl{sg}=whale \gl{def}.\gl{pl}=\gl{nmlz}-fish	//
\glft \qu{The fishermen saw the whale.}	//
\endgl 
\a \label{ex:noka}
\begingl
\gla *le=xale-a i=puudo li=xa-vwa.tau	//
%\glb	//
\glft *\qu{The fishermen saw the whale (intended)}	//
\endgl

\xe

The element following \textit{ka} has not necessarily already been introduced (see example \ref{ex:newka}).

\pex

\a \label{ex:newka}
\begingl
\glpreamble 2016-11-15 Le poulpe et le rat//
\gla koin a=kon a=bwa tena-a \textbf{ka} i=ibwen//
\glb While 3.\gl{sg}=\gl{koon} 3.\gl{sg}=\gl{bwa} hear-3.\gl{sg} \gl{agt} \gl{def}.\gl{sg}=squid//
\glft \qu{During this time [the rat's crying] he [the squid] heard him [the rat], did the squid}//
\endgl

\a
\begingl
\glpreamble Tipije 1//
\gla le=ja vatipwe-mwa kon ya tha pa ja me-a go tha le=ja thuat-mwa moo \textbf{ka} li=xhaomu ja tha le=ja ha-mwa-me// 
\glb 3.\gl{pl}=\gl{ja} drop-\gl{mwa} because 3.\gl{sg} \gl{ass} \gl{pa} \gl{ja} die-3.\gl{sg} then \gl{ass} 3.\gl{pl}=\gl{ja} emerge-\gl{mwa} stay \gl{agt} \gl{def}.\gl{pl}=elder \gl{ja} \gl{ass} 3.\gl{pl}=\gl{ja} go-\gl{mwa}-\gl{me}// 
\glft  \qu{They were released [again] because he was already dead, and they at last came out of prison again  and came back.}// 
\endgl 

\todo{check prosody for this one}


\a 
\begingl
\glpreamble 2017-09-01 Kito Nigai Thea//
\gla a=bwa hupwa ka i=jaati//
\glb 3.\gl{sg}=\gl{bwa} go.down-\gl{mwa} \gl{agt} \gl{def}.\gl{sg}=sea//
\glft \qu{The sea [tide] is about to go down/is going down}//
\endgl

\a 
\begingl
\glpreamble	2017-07-31 Kito Keela Nigai//
\gla	e-vaaya kon \textbf{ka} i=that	//
\glb	 \gl{refl}-move ? \gl{agt} \gl{def}.\gl{sg}=wind	//
\glft \qu{They are shaking in the wind}\\	\qu{Elles bougent avec le vent}	//
\endgl

\a 
\begingl
\glpreamble	2017-09-05 le vent du Nord et le soleil//
\gla	ha-me thaloot \textbf{ka} i=apuli a thabuna ca i=joakan sapwen	//
\glb	go-\gl{me} emerge \gl{agt} \gl{def}.\gl{sg}=man \gl{rel} wrap in \gl{def}.\gl{sg}=thick cloak//
\glft	\qu{A man wrapped in a thick cloak showed up}//
\endgl

\todo{add i soovan ka i xhaohmu}
\xe