% This file was converted to LaTeX by Writer2LaTeX ver. 1.4
% see http://writer2latex.sourceforge.net for more info
\documentclass[12pt]{article}
\usepackage[utf8]{inputenc}
\usepackage[T1]{fontenc}
\usepackage[english]{babel}
\usepackage{amsmath}
\usepackage{amssymb,amsfonts,textcomp}
\usepackage{array}
\usepackage{hhline}
\usepackage{hyperref}
\hypersetup{colorlinks=true, linkcolor=blue, citecolor=blue, filecolor=blue, urlcolor=blue}
% footnotes configuration
\makeatletter
\renewcommand\thefootnote{\arabic{footnote}}
\makeatother
\newcommand\textsubscript[1]{\ensuremath{{}_{\text{#1}}}}
% Text styles
\newcommand\textstyleStrong[1]{\textbf{#1}}
% Headings and outline numbering
\makeatletter
\renewcommand\subsection{\@startsection{subsection}{2}{0.0cm}{0in}{0.111in}{\normalfont\normalsize\fontsize{12pt}{14.4pt}\selectfont\rmfamily\raggedright}}
\renewcommand\@seccntformat[1]{\csname @textstyle#1\endcsname{\csname the#1\endcsname}\csname @distance#1\endcsname}
\setcounter{secnumdepth}{0}
\newcommand\@distancesection{}
\newcommand\@textstylesection[1]{#1}
\newcommand\@distancesubsection{}
\newcommand\@textstylesubsection[1]{#1}
\makeatother
\raggedbottom
% Paragraph styles
\renewcommand\familydefault{\rmdefault}
\newenvironment{styleStandard}{\setlength\leftskip{0cm}\setlength\rightskip{0cm plus 1fil}\setlength\parindent{0cm}\setlength\parfillskip{0pt plus 1fil}\setlength\parskip{0in plus 1pt}\writerlistparindent\writerlistleftskip\leavevmode\normalfont\normalsize\writerlistlabel\ignorespaces}{\unskip\vspace{0.111in plus 0.0111in}\par}
\newenvironment{stylelsSectioni}{\setlength\leftskip{0cm}\setlength\rightskip{0cm plus 1fil}\setlength\parindent{0cm}\setlength\parfillskip{0pt plus 1fil}\setlength\parskip{0.1665in plus 0.016649999in}\writerlistparindent\writerlistleftskip\leavevmode\normalfont\normalsize\fontsize{18pt}{21.6pt}\selectfont\bfseries\writerlistlabel\ignorespaces}{\unskip\vspace{0.0835in plus 0.00835in}\par}
\newenvironment{styleListParagraph}{\renewcommand\baselinestretch{1.0}\setlength\leftskip{0.5in}\setlength\rightskip{0in plus 1fil}\setlength\parindent{0in}\setlength\parfillskip{0pt plus 1fil}\setlength\parskip{0in plus 1pt}\writerlistparindent\writerlistleftskip\leavevmode\normalfont\normalsize\writerlistlabel\ignorespaces}{\unskip\vspace{0in plus 1pt}\par}
% List styles
\newcommand\writerlistleftskip{}
\newcommand\writerlistparindent{}
\newcommand\writerlistlabel{}
\newcommand\writerlistremovelabel{\aftergroup\let\aftergroup\writerlistparindent\aftergroup\relax\aftergroup\let\aftergroup\writerlistlabel\aftergroup\relax}
\newcounter{listWWNumxxiiileveli}
\newcounter{listWWNumxxiiilevelii}[listWWNumxxiiileveli]
\newcounter{listWWNumxxiiileveliii}[listWWNumxxiiilevelii]
\newcounter{listWWNumxxiiileveliv}[listWWNumxxiiileveliii]
\renewcommand\thelistWWNumxxiiileveli{\arabic{listWWNumxxiiileveli}}
\renewcommand\thelistWWNumxxiiilevelii{\alph{listWWNumxxiiilevelii}}
\renewcommand\thelistWWNumxxiiileveliii{\roman{listWWNumxxiiileveliii}}
\renewcommand\thelistWWNumxxiiileveliv{\arabic{listWWNumxxiiileveliv}}
\newcommand\labellistWWNumxxiiileveli{\thelistWWNumxxiiileveli.}
\newcommand\labellistWWNumxxiiilevelii{\thelistWWNumxxiiilevelii.}
\newcommand\labellistWWNumxxiiileveliii{\thelistWWNumxxiiileveliii.}
\newcommand\labellistWWNumxxiiileveliv{\thelistWWNumxxiiileveliv.}
\newenvironment{listWWNumxxiiileveli}{\def\writerlistleftskip{\setlength\leftskip{0.5in}}\def\writerlistparindent{}\def\writerlistlabel{}\def\item{\def\writerlistparindent{\setlength\parindent{-0.25in}}\def\writerlistlabel{\stepcounter{listWWNumxxiiileveli}\makebox[0cm][l]{\labellistWWNumxxiiileveli}\hspace{-0.635cm}\writerlistremovelabel}}}{}
\newenvironment{listWWNumxxiiilevelii}{\def\writerlistleftskip{\setlength\leftskip{1in}}\def\writerlistparindent{}\def\writerlistlabel{}\def\item{\def\writerlistparindent{\setlength\parindent{-0.25in}}\def\writerlistlabel{\stepcounter{listWWNumxxiiilevelii}\makebox[0cm][l]{\labellistWWNumxxiiilevelii}\hspace{-1.905cm}\writerlistremovelabel}}}{}
\newenvironment{listWWNumxxiiileveliii}{\def\writerlistleftskip{\setlength\leftskip{1.5in}}\def\writerlistparindent{}\def\writerlistlabel{}\def\item{\def\writerlistparindent{\setlength\parindent{-0.1252in}}\def\writerlistlabel{\stepcounter{listWWNumxxiiileveliii}\makebox[0cm][r]{\labellistWWNumxxiiileveliii}\hspace{-3.4919918cm}\writerlistremovelabel}}}{}
\newenvironment{listWWNumxxiiileveliv}{\def\writerlistleftskip{\setlength\leftskip{2in}}\def\writerlistparindent{}\def\writerlistlabel{}\def\item{\def\writerlistparindent{\setlength\parindent{-0.25in}}\def\writerlistlabel{\stepcounter{listWWNumxxiiileveliv}\makebox[0cm][l]{\labellistWWNumxxiiileveliv}\hspace{-4.4449997cm}\writerlistremovelabel}}}{}
\title{}
\author{Anna}
\date{2019-07-23}
\begin{document}
\title{\textsuperscript{Ditransitive constructions with DOM-ed direct objects in Romanian}}
\maketitle

\begin{styleStandard}
\textbf{Alexandra Cornilescu}
\end{styleStandard}

\begin{styleStandard}
\textbf{University of Bucharest}
\end{styleStandard}

\begin{styleStandard}
\textbf{\textit{Abstract}}\textit{. The paper discusses Romanian data that had gone unnoticed so far and investigates the differences of grammaticality triggered by DOM-ed DOs in ditransitive constructions, in binding configurations. Specifically, while a bare DO may bind a possessor contained in the IO, whether or not the IO is clitic doubled, a DOM-ed DO may bind into an undoubled IO, but cannot bind into an IO if the latter is clitic doubled. Grammaticality is restored if the DO is clitic doubled in its turn.}
\end{styleStandard}

\begin{styleStandard}
\textit{The focus of the paper is to offer a derivational account of ditransitive constructions, which accounts for these differences. The claim is that the grammaticality contrasts mentioned above result from the different feature structure of bare DOs compared with DOM-ed ones, as well as from the fact that DOM-ed DOs and IO have common features. DOM-ed DOs interfere with IOs since both are sensitive to the Animacy Hierarchy, and include a syntactic [Person] feature in their featural make-up. The derivational valuation of this feature by both objects may create locality problems.}
\end{styleStandard}

\begin{styleStandard}
\textbf{Keywords:}\textit{ dative, DOM, ditransitive construction, functional prepositions, binding}
\end{styleStandard}

\begin{stylelsSectioni}
1. Problem and aim
\end{stylelsSectioni}

\begin{styleStandard}
In this paper, I turn to data not discussed for Romanian so far and consider the differences of grammaticality triggered by DOM-ed DOs in ditransitive constructions, in \textit{binding} configurations.\footnote{I would like to express my gratitude for the wonderful help I got from the reviewers and the editors in finalizing the paper. Remaining errors are all mine.}
\end{styleStandard}

\begin{styleStandard}
Specifically\footnote{ Judgments on possessor binding in Romanian ditransitive constructions and some of the examples come from an experiment described in detail in Cornilescu, Dinu \& Tigău (CDT, 2017a). Unless otherwise specified, examples and acceptability judgments belong to the author.}, bare DOs easily bind a possessor contained in a dative IO, whether the latter is CD-ed or not, as in (1) - (2). The picture changes when the DO is DOM-ed. It is still possible for a DOM-ed DO to bind into an undoubled IO (3), but if the IO\textit{ is doubled, }the sentence is \textit{ungrammatical} (4). While co-occurrence of the DOM-ed DO with a dative clitic, results in ungrammaticality, if the DOM-ed \textsc{DO} is doubled, sentences are grammatical, again irrespective of the presence/absence of the dative clitic, as in examples (5) and (6).
\end{styleStandard}

\begin{styleStandard}
(1)\ \ Romanian (CDT 2017: 162)
\end{styleStandard}

\begin{styleStandard}
\ \ DP\textsubscript{theme}{\textgreater} DP\textsubscript{goal}
\end{styleStandard}

\begin{styleStandard}
\textit{Banca\ \ a retrocedat\ \ multe case}\textit{\textsubscript{i}}\textit{\ \ proprietarilor\ \  \ \ lor}\textit{\textsubscript{i}}\textit{\ \ de drept.}
\end{styleStandard}

\begin{styleStandard}
bank.the has returned\ \ many houses\ \ owners.the.\textsc{dat \ \ \ }their\ \ of right
\end{styleStandard}

\begin{styleStandard}
‘The bank returned the houses to their rightful owners.’
\end{styleStandard}

\begin{styleStandard}
(2)\ \ Romanian (CDT 2017a:162)
\end{styleStandard}

\begin{styleStandard}
DP\textsubscript{theme}{\textgreater}\textbf{cl-}DP\textsubscript{goal}
\end{styleStandard}

\begin{styleStandard}
\textit{Banca\ \ \ \ }\textbf{\textit{le}}\textit{\textsubscript{j}}\textit{=a \ \ \ \ retrocedat\ \ \ \ multe case}\textit{\textsubscript{i}}
\end{styleStandard}

\begin{styleStandard}
bank.the\ \ they.\textsc{dat}=has \ \ returned \ \ \ \ many houses\ \ 
\end{styleStandard}

\begin{styleStandard}
\textit{proprietarilor}\textit{\textsubscript{j}}\textit{\ \ \ \ lor}\textit{\textsubscript{i\ \ }}\textit{de drept.}
\end{styleStandard}

\begin{styleStandard}
owners.the.\textsc{dat}\ \ their \ \ of right
\end{styleStandard}

\begin{styleStandard}
‘The bank returned many houses to their rightful owners.’
\end{styleStandard}

\begin{styleStandard}
(3)\ \ Romanian
\end{styleStandard}

\begin{styleStandard}
DOM-ed DP\textsubscript{theme }{\textgreater} DP \textsubscript{goal}
\end{styleStandard}

\begin{styleStandard}
\textit{Comisia\ \ a repartizat\ \ pe mai \ \ \ mul[21B?]i \ \ medici}\textit{\textsubscript{i \ \ \ }}\textit{reziden[21B?]i\ \ }
\end{styleStandard}

\begin{styleStandard}
board.the\ \ has assigned\ \ \textsc{dom }more many medical residents\ \ 
\end{styleStandard}

\begin{styleStandard}
\textit{unor\ \ \ \ fo[219?]ti profesori\ \ \ \ de-ai lor}\textit{\textsubscript{i}}\textit{.}
\end{styleStandard}

\begin{styleStandard}
some.\textsc{dat}\ \ former professors \ \ of theirs.
\end{styleStandard}

\begin{styleStandard}
‘The board assigned several medical residents to some former professors of theirs.’
\end{styleStandard}

\begin{styleStandard}
(4)\ \ Romanian
\end{styleStandard}

\begin{styleStandard}
*DOM-ed DP\textsubscript{theme}{\textgreater}\textbf{cl}{}- DP\textsubscript{goal}
\end{styleStandard}

\begin{styleStandard}
\textit{*Comisia\ \ }\textbf{\textit{le}}\textit{=a \ \ repartizat\ \ pe\ \ mai mul[21B?]i medici}\textit{\textsubscript{i \ }}\textit{reziden[21B?]i}
\end{styleStandard}

\begin{styleStandard}
board.the\ \ they.\textsc{dat}=has assigned\ \ \textsc{dom }more many medical residents \ \ 
\end{styleStandard}

\begin{styleStandard}
\textit{unor\ \ \ \ fo[219?]ti\ \ profesori\ \ de-ai lor}\textit{\textsubscript{i}}\textit{.}
\end{styleStandard}

\begin{styleStandard}
some.\textsc{dat\ \ }former professors \ \ of theirs
\end{styleStandard}

\begin{styleStandard}
‘The board assigned several medical residents to some former professors of theirs.’
\end{styleStandard}

\begin{styleStandard}
(5)\ \ Romanian
\end{styleStandard}

\begin{styleStandard}
\textbf{cl}{}- DOM-ed DP\textsubscript{theme}{\textgreater} DP\textsubscript{goal}
\end{styleStandard}

\begin{styleStandard}
\textit{Comisia\ \ }\textbf{\textit{i}}\textit{=a repartizat\ \ \ \ pe \ mai \ mul[21B?]i \ medici}\textit{\textsubscript{i \ \ \ }}\textit{reziden[21B?]i}
\end{styleStandard}

\begin{styleStandard}
board.thethey.\textsc{acc=}has assigned \ \ \textsc{dom }more many medical residents
\end{styleStandard}

\begin{styleStandard}
\textit{unor \ \ \ \ fo[219?]ti\ \  \ \ \ \ \ \ \ profesori \ \ \ de-ai lor}\textit{\textsubscript{i}}\textit{.}
\end{styleStandard}

\begin{styleStandard}
some.\textsc{dat} former \ \ \ \ professors \ \ of theirs.
\end{styleStandard}

\begin{styleStandard}
‘The board assigned several medical residents to some former professors of theirs.’
\end{styleStandard}

\begin{styleStandard}
(6)\ \ Romanian
\end{styleStandard}

\begin{styleStandard}
\textbf{cl}{}- DOM-ed DP \textsubscript{theme}{\textgreater}\textbf{cl}{}-DP \textsubscript{goal }
\end{styleStandard}

\begin{styleStandard}
\textit{Comisia\ \ i=l=a=repartizat\ \  \ \ \ pe fiecare medic rezident}
\end{styleStandard}

\begin{styleStandard}
board.the\ \ she.\textsc{dat}=he.\textsc{Acc}=assigned \textsc{dom} each medical resident 
\end{styleStandard}

\begin{styleStandard}
\textit{unei foste profesoare\ \ \ \ a lui}.
\end{styleStandard}

\begin{styleStandard}
some.\textsc{dat }professor.F.\textsc{dat}\ \ his
\end{styleStandard}

\begin{styleStandard}
‘The board assigned each resident doctor to a former professor of his.’
\end{styleStandard}

\begin{styleStandard}
Critical is the difference between (2) and (4), and also between (4) and (5)-(6) where the DO is doubled.
\end{styleStandard}

\begin{styleStandard}
\textbf{\textit{The aim }}of the chapter is to offer a derivational account of ditransitive constructions, which accommodates these differences. We claim that the grammaticality contrasts above result from the different feature structure of bare DOs compared with DOM-ed ones, and from the fact that DOM-ed DOs and IOs need to check the same [Person] feature against the same functional head.
\end{styleStandard}

\begin{stylelsSectioni}
2. On Romanian dative DPs
\end{stylelsSectioni}

\subsection[2.1. Inflectional datives and the animacy hierarchy]{\textbf{2.1. Inflectional datives and the animacy hierarchy}}
\begin{styleStandard}
In Romanian nouns have \textit{inflectional dative morphology} and, additionally, exhibit \textit{prepositional marking}, employing the locative preposition \textit{la} ‘at’/’to’. An essential property of inflectional datives (=Inf-\textsc{dat}) is that they are highly sensitive to the animacy hierarchy (=AH) and have a \textit{higher cut-off point }than \textbf{\textit{la}}\textit{{}-}datives, as seen in (8). 
\end{styleStandard}

\begin{styleStandard}
(7)\ \ human\ \ {\textgreater}\ \ animate\ \ \ \ {\textgreater} inanimate
\end{styleStandard}

\begin{styleStandard}
(8)\ \ Romanian
\end{styleStandard}

\begin{styleStandard}
\ \ a.\ \ \textit{Am \ \ turnat\ \ vin \ \ la musafiri/ \ \ musafirilor}
\end{styleStandard}

\begin{styleStandard}
\ \ \ \ have.I \ \ poured \ \ wine \ \ at guests/ \ \ guests.the.\textsc{dat}
\end{styleStandard}

\begin{styleStandard}
\ \ \ \ ‘I poured wine to the guests.’
\end{styleStandard}

\begin{styleStandard}
\ \ b.\ \ \textit{Am dat\ \ \ \ apa\ \ la cai/ \ \ \ \ ?cailor}.
\end{styleStandard}

\begin{styleStandard}
\ \ \ \ have.I given \ \ water \ \ at horses/\ \ horses.the.\textsc{dat }
\end{styleStandard}

\begin{styleStandard}
\ \ \ \ ‘I poured water to the horses.’
\end{styleStandard}

\begin{styleStandard}
\ \ c.\ \ \textit{Am turnat\ \ apă\ \ la flori/ \ \ *?florilor.}
\end{styleStandard}

\begin{styleStandard}
\ \ \ \ have.I poured \ \ water \ \ at flowers \ \ flowers.the.\textsc{ dat }
\end{styleStandard}

\begin{styleStandard}
\ \ \ \ ‘I poured water to the flowers.’
\end{styleStandard}

\begin{styleStandard}
One theoretical difficulty that immediately arises is that of incorporating \textit{scalar concepts} like the AH or the definiteness hierarchy (= DefH) into the discrete binary system of a minimalist grammar. Richards (2008) argues that the AH and the DefH are semantic and pragmatic in nature and should be viewed as \textit{syntax-semantics interface phenomena}. Crucially, he proposes that nouns which are sensitive to these hierarchies should be lexically specified for a binary \textit{grammatical }[Person] feature (Rodríguez-Mondoñedo 2007 for Spanish). It is this [Person] feature which triggers the interpretation of a given NP along the two hierarchies, checking its position on the two scales. Nouns which accept the Inf-\textsc{dat} enter the derivation lexically marked as [+Person]. Since this is a syntactic feature, it must be checked during the derivation. 
\end{styleStandard}

\begin{styleStandard}
\textbf{2.2 On the internal structure of }\textbf{\textit{la}}\textbf{{}-datives}
\end{styleStandard}

\begin{styleStandard}
The preposition \textit{la} ‘at’/’to’ is not only a \textit{functional dative marker}, but it is also the core \textit{lexical preposition} of the location and movement frames. The lexical preposition \textit{la} assigns accusative case to its object, this accusative cannot be replaced by a dative, and, as correctly pointed out by both reviewers, accusative \textit{la-}phrases do not co-occur with dative clitics. All movement and location verbs may combine with lexical accusative \textit{la}{}-phrases, rejecting, however, dative \textit{la}{}-phrases. An example is the verb \textit{merge} ‘go’, which is compatible only with lexical \textit{la}, but not with functional dative \textit{la}. Substitution of the \textit{la}{}-phrase with a dative DP is impossible (9a), and a dative clitic is equally ungrammatical (9b).
\end{styleStandard}

\begin{styleStandard}
(9)\ \ Romanian
\end{styleStandard}

\begin{styleStandard}
\ a.\ \ \textit{Ion \ \ a mers\ \ \ \ la Maria/\ \ **Mariei.}
\end{styleStandard}

\begin{styleStandard}
\ \ \ \ Ion \ \ has gone \ \ at Maria.\textsc{acc}/ \ \ Maria.\textsc{dat}
\end{styleStandard}

\begin{styleStandard}
\ \ \ \ ‘Ion went to Maria.’
\end{styleStandard}

\begin{styleStandard}
b.\ \ \textit{*Ion\ \ îi merge \ \ (Mariei).}
\end{styleStandard}

\begin{styleStandard}
\ \ \ \ Ion \ \ she.\textsc{dat}=goes \ \ Maria.\textsc{dat}
\end{styleStandard}

\begin{styleStandard}
\ \ \ \ ‘Ion is going to Mary.’
\end{styleStandard}

\begin{styleStandard}
One specification is required at this point. Even for unaccusative verbs like\textit{ plăcea} ‘like’, which always select a dative Experiencer, either inflectional or prepositional, co-occurrence of a dative \textit{la-}phrase with a clitic is possible only in the third person; in the first and in the second person, the clitic may co-occur only with an inflectional dative strong pronoun, never with a prepositional dative, as apparent in (10b) below:
\end{styleStandard}

\begin{styleStandard}
(10)\ \ Romanian
\end{styleStandard}

\begin{styleStandard}
\ \ a.\ \ \textit{Cicolata\ \ le=place \ \ \ \ copiilor/\ \ la copii.}
\end{styleStandard}

\begin{styleStandard}
\ \ \ \ chocolate.the\ \ they.\textsc{dat}=like.3SG\ \ children.the.\textsc{dat}/ at children
\end{styleStandard}

\begin{styleStandard}
\ \ \ \ ‘Children like chocolate.’
\end{styleStandard}

\begin{styleStandard}
\ \ b.\ \ \textit{Ciocolata\ \ îmi=place \ \ şi mie/ \ \ \ \ *şi \ la mine.}
\end{styleStandard}

\begin{styleStandard}
\ \ \ \ chocolate.the\ \ I.\textsc{dat}=like.3SG \ also I.\textsc{dat }/\ \ also at me
\end{styleStandard}

\begin{styleStandard}
\ \ \ \ ‘I also like chocolate.’\ \ 
\end{styleStandard}

\begin{styleStandard}
Verbs in the movement frame do not behave uniformly regarding the realization of their Goal argument. While some never select a dative (e.g. \textit{merge} ‘go’), others (e.g. \textit{ajunge} ‘arrive’ or \textit{veni} ‘come’) may select a dative on condition that the Goal DP is [+Person]; the dative Goal is realized as a clitic, doubled by a strong pronoun or by a dative \textit{la-}phrase, provided that the clitic is third person, as already shown in (10). Thus, in (11a) the \textit{la}{}-phrase is lexical; in (11b), the Goal is a dative phrase realized as a clitic; the first person dative clitic can only be doubled by a dative strong pronoun, while the \textit{la}{}-phrase is out (11b’). The relevant example is however (11c), an attested Google example, where the Goal is a dative, and the dative clitic is doubled by a dative \textit{la}{}- phrase. As the comparison of (11a) and (11c) shows, the \textit{la-}phrase is interpreted as a dative only when it co-occurs with a dative clitic.
\end{styleStandard}

\begin{styleStandard}
(11)\ \ Romanian
\end{styleStandard}

\begin{styleStandard}
\ \ a.\ \ \textit{Pachetul\ \  a ajuns\ \ \ \ }\textbf{\textit{la mine/la Londra}}\textit{\ \ ieri.}
\end{styleStandard}

\begin{styleStandard}
\ \ \ \ parsel.the\ \ has arrived \ \ at I.\textbf{\textsc{acc}}\textbf{/ }at London\ \ yesterday
\end{styleStandard}

\begin{styleStandard}
\ \ \ \ ‘The parsel got to me/ to London yesterday.’
\end{styleStandard}

\begin{styleStandard}
\ \ b.\ \ \textit{Pachetul\ \ }\textbf{\textit{mi}}\textit{=a\ \ \ \ ajuns\ \ şi \ \ \ \ mie\ \ ieri.}
\end{styleStandard}

\begin{styleStandard}
\ \ \ \ parsel.the\ \ I.\textbf{\textsc{dat}}= has \ \ arrived \ \ also I.\textsc{dat}\ \ yesterday
\end{styleStandard}

\begin{styleStandard}
\ \ b.’\ \ \textit{Pachetul\ \ }\textbf{\textit{mi}}\textit{=a \ \ \ \ ajuns\ \ (*}\textbf{\textit{la mine)}}\textit{\ \ ieri.}
\end{styleStandard}

\begin{styleStandard}
\ \ \ \ parsel.the\ \ I.\textsc{dat}=has\ \ arrived\ \ (at.I.\textsc{acc})\ \ yesterday
\end{styleStandard}

\begin{styleStandard}
\ \ \ \ ‘The parsel\ \ got to me too yesterday’
\end{styleStandard}

\begin{styleStandard}
\ \ c.\ \ \textit{Acum \ \ }\textbf{\textit{le}}\textit{=au \ \ venit \ \ \ \ }\textbf{\textit{la mul[21B?]i}}\textit{\ \ \ \ deciziile}\ \ 
\end{styleStandard}

\begin{styleStandard}
now\ \ they.\textsc{dat}=have come\ \ at. many.\textsc{acc}\ \ decisions 
\end{styleStandard}

\begin{styleStandard}
\textit{de \ recalculare\ \ \ \ a pensiilor}
\end{styleStandard}

\begin{styleStandard}
of \ recalculation\ \ pensions.the.\textsc{gen}
\end{styleStandard}

\begin{styleStandard}
‘Now many have got their decisions for recalculation of their pensions.’
\end{styleStandard}

\begin{styleStandard}
In the rest of this section I examine the internal structure of the \textit{la}{}-phrase when it is a dative, i.e. when it is clitic-doubled. As a dative-marker \textit{la} is puzzling, since it is described as a “dative preposition”, but, as seen above in (9a), it clearly assigns accusative case to its complement (12). On the other hand, \textit{la}{}-phrases may take dative clitics (11c), and, as well-known, clitics and their associates always agree in Case. This suggests that, as a dative marker, \textit{la} simply assigns Case to a DP \textit{subcomponent }of the dative phrase, while the whole \textit{la}{}-phrase has \textit{an uninterpretable valued dative feature} (13), which agrees with the clitic’s Case feature. The marker \textit{la} has become an \textit{internal constituent} which extends the dative phrase (13), i.e. a K(ase) marker like the marker of DOM (Lopez 2011). An additional role of this morpheme is that of a category shifter, which reanalyzes the PP into a KP, therefore, an extended DP.
\end{styleStandard}

\begin{styleStandard}
The categorial change from P to K may be viewed as an instance of \textit{downward re-analysis} (Roberts \& Roussou 2003), likely to have occurred out of the need to improve the correspondence between syntactic features and their PF representation.
\end{styleStandard}

\begin{styleStandard}
(12)\ \ \ \ \ \  \ \ \ PP\ \ \ \ \ \ \ \ 
\end{styleStandard}

\begin{styleStandard}
\ \ \ \ qp
\end{styleStandard}

\begin{styleStandard}
P\ \ \ \ \ \ DP\ \ \ \ \ \ 
\end{styleStandard}

\begin{styleStandard}
\ \ \ \ [Case:\_\_\_]\ \ \ \ [\textit{u}Case:Acc]
\end{styleStandard}

\begin{styleStandard}
\ \ \ \ [Loc/Goal]\ \ \ \ ([\textit{i}Person])
\end{styleStandard}

\begin{styleStandard}
\ \ \ \ \textbf{la}
\end{styleStandard}

\begin{styleStandard}
(13)\ \ \ \ \ \ KP [Case:Dat]
\end{styleStandard}

\begin{styleStandard}
\ \ \ \ qp
\end{styleStandard}

\begin{styleStandard}
\ \ \ \ K\ \ \ \ \ \ DP
\end{styleStandard}

\begin{styleStandard}
\ \ \ \ [\textit{u}Case:Acc]\ \ \ \ [\textit{u}Case:Acc]
\end{styleStandard}

\begin{styleStandard}
\ \ \ \ [\textit{u}Case:Dat]
\end{styleStandard}

\begin{styleStandard}
\ \ \ \ \textbf{la}
\end{styleStandard}

\begin{styleStandard}
In time, there gradually emerged two different changes in the function of the Locative PP in (12). One has been the extension of \textit{la} from verbs that have Goals or Possessor-Goals in their a-structure (verbs of giving and throwing) to verbs that select Beneficiaries (e.g. verbs of creation, like \textit{face} ‘make, do’, \textit{coace} ‘bake’, etc.), and even verbs that select Maleficiary or Source, i.e. the opposite of Goal, (e.g. \textit{fura} ‘steal’). Thus the preposition \textit{la} widens its thematic sphere, but it is partly desemanticized, since the thematic content of the \textit{la}{}-phrase almost completely follows from the descriptive content of the selecting verb. Secondly, while any kind of DP may assume the Location/Goal ${\theta}${}-role, these extended interpretations (e.g. Beneficiary, Maleficiary) are compatible only with nouns high in the AH. As explained, such nouns grammaticalize their inherent human feature as a syntactic [Person] feature (Richards 2008).
\end{styleStandard}

\begin{styleStandard}
(14)\ \ Romanian
\end{styleStandard}

\begin{styleStandard}
Possessor-Goal
\end{styleStandard}

\begin{styleStandard}
\textit{Mama \ \ \ \ (le)=a \ \ \ \ dat\ \ prăjituri \ \ copiilor/\ \ }\textbf{\textit{la }}\textit{copii. }
\end{styleStandard}

\begin{styleStandard}
mother.the\ \ they.\textsc{dat}=has \ \ given \ \ cakes \ \ children.the.\textsc{dat}/ at children
\end{styleStandard}

\begin{styleStandard}
‘Mother gave cakes to the children.’
\end{styleStandard}

\begin{styleStandard}
(15)\ \ Romanian
\end{styleStandard}

\begin{styleStandard}
Beneficiary
\end{styleStandard}

\begin{styleStandard}
\textit{Mama \ \ \ \ (le)=a \ \ \ \ copt\ \ prăjituricopiilor/\ \ }\textbf{\textit{la}}\textit{copii.}
\end{styleStandard}

\begin{styleStandard}
\ \ mother.the\ \ they.\textsc{dat}=has \ \ baked \ \ cakes \ \ children.the.\textsc{dat}/at children
\end{styleStandard}

\begin{styleStandard}
\ \ ‘Mother baked cakes \textit{for }the children.’
\end{styleStandard}

\begin{styleStandard}
(16)\ \ Romanian
\end{styleStandard}

\begin{styleStandard}
Maleficiary/ Source
\end{styleStandard}

\begin{styleStandard}
\textit{Nişte vagabonzi le-au \ \ furat\ \ copiilor/\ \ }\textbf{\textit{la }}\textit{copii\ \ jucăriile. }
\end{styleStandard}

\begin{styleStandard}
\ \ some tramps\ \ they.cl.\textsc{dat-}have stolen children.the.\textsc{dat}/at children the toys
\end{styleStandard}

\begin{styleStandard}
\ \ ‘Some tramps stole the toys from the children.’
\end{styleStandard}

\begin{styleStandard}
At this point, there was an imperfect match between features and their exponents, since \textit{la} had partly lost its thematic content, and an obligatory syntactic [+Person] feature in the nominal matrix had no PF realization (12). This tension led to the re-analysis of \textit{la} as a PF exponent of the [Person] feature of the noun. As such \textit{la} becomes a higher K part of the nominal expression, where K is a spell-out of [\textit{i}Person]. Syntactically, K is a Probe that values an uninterpretable [\textit{u}Person:\_\_\_] feature of the DP through agreement (17).
\end{styleStandard}

\begin{styleStandard}
(17)\ \ \ \ 
\end{styleStandard}

\begin{styleStandard}
KP[\textit{i}Person,\textit{ i}${\varphi}$,${\pm}$Def, \textit{u}Case: \textsc{dat}]
\end{styleStandard}

\begin{styleStandard}
qp
\end{styleStandard}

\begin{styleStandard}
\ \ K\ \ \ \ \ \  \ \ \ \ \ DP
\end{styleStandard}

\begin{styleStandard}
\textbf{[}\textbf{\textit{i}}\textbf{Person]\ \  \ \ \ \ }wo
\end{styleStandard}

\begin{styleStandard}
\ \ [\textit{u}Case:\_\_\_]\ \ \ \ D\ \ \ \ NP
\end{styleStandard}

\begin{styleStandard}
\ \  [\textit{u}Case: \textsc{dat }]\ \ [+D\ \ \ \ [+N]
\end{styleStandard}

\begin{styleStandard}
\ \ \ \ {\textbar}\ \ \ \ [${\pm}$Def]\ \ [\textit{i}${\varphi}$]
\end{styleStandard}

\begin{styleStandard}
\ \ \ \ {\textbar}\ \ \ \ [\textit{u}${\varphi}$]\ \ \ \ \textbf{[}\textbf{\textit{u}}\textbf{Person]}
\end{styleStandard}

\begin{styleStandard}
\ \ \ \ \textbf{la}\ \ \ \ [\textit{u}Case:Acc]\ \ [+Animate]
\end{styleStandard}

\begin{styleStandard}
Compared to (12), representation in (17) is “simpler”, since the grammatical feature [\textit{i}Person], synchretically realized by N in (12) is realized as a separate lexical item in (17). 
\end{styleStandard}

\begin{styleStandard}
Like Inf-\textsc{dat}, \textit{la}{}-\textsc{dat} are sensitive to the AH\textit{, }but the selectional properties of \textit{la} are not identical to those of the dative inflection\textit{. }For instance, both types of datives are compatible with names of\textit{ corporate bodies (18), }but only Inf-\textsc{dat }are also felicitous with\textit{ abstract }nouns\textit{, la}{}-\textsc{dat }are not (19).
\end{styleStandard}

\begin{styleStandard}
(18)\ \ Romanian
\end{styleStandard}

\begin{styleStandard}
\textit{(Le)=a \ \ împărţit\ \ \ \ banii\ \ la nişte\ \ \ \ asociaţii caritabile /}
\end{styleStandard}

\begin{styleStandard}
\ \ (they.\textsc{dat}) has handed-out\ \ money.the at some \ \ \ associations charitable
\end{styleStandard}

\begin{styleStandard}
\ \ \textit{unor\ \ \ \ asociaţii\ \ caritabile.}
\end{styleStandard}

\begin{styleStandard}
some.\textsc{dat\ \ }associations \ \ charitable
\end{styleStandard}

\begin{styleStandard}
\ \ ‘He handed out the money to some charities.’
\end{styleStandard}

\begin{styleStandard}
(19)\ \ Romanian
\end{styleStandard}

\begin{styleStandard}
\textit{A supus\ \ \ \ proiectul\ \ *la atenţia \ \ bordului/}
\end{styleStandard}

\begin{styleStandard}
\ \ has submitted\ \ project.the\ \ at attention \ \ board.the.\textsc{gen}/ 
\end{styleStandard}

\begin{styleStandard}
\ \ \textit{aten[21B?]iei\ \ \ \ \ \ bordului.}
\end{styleStandard}

\begin{styleStandard}
attention.the.\textsc{dat\ \ }board.the.\textsc{gen}
\end{styleStandard}

\begin{styleStandard}
\ \ ‘He submitted the project to the board’s attention.’
\end{styleStandard}

\begin{styleStandard}
\textit{Conclusions so far}
\end{styleStandard}

\begin{styleStandard}
1. Nouns may come from the lexicon with an unvalued [\textit{u}Person] feature. 
\end{styleStandard}

\begin{styleStandard}
2. Dative \textit{la} is a K component which spells-out an [\textit{i}Person] feature, historically resulting through downward re-analysis of the homonymous [Location] preposition. K selects DPs which are [\textit{u}Person] and values their [\textit{u}Person] feature.
\end{styleStandard}

\begin{styleStandard}
3. A KP nominal expression is complex, since it contains a smaller DP. The K-head case-licenses the smaller DP. K also contains an \textit{uninterpretable valued} dative feature checked during the derivation.
\end{styleStandard}

\subsection[2.3 Why a clitic is sometimes required]{\textbf{2.3 Why a clitic is sometimes required}}
\begin{styleStandard}
In theory, like any functional head, the clitic should be a response to some internal need of the \textit{la}{}-phrase. It is plausible that dative \textit{la}, an [\textit{i}Person] spell-out, further eroded semantically, becoming an uninterpretable [\textit{u}Person], at least sometimes\footnote{ An important empirical generalization (Cornilescu 2017) regarding Romanian dative clitics is that they are obligatory for non-core datives, but optional for core datives. In the analysis proposed here, this means that the [Person] feature on dative KPs is uninterpretable by default and can be interpretable only for \textit{core datives}, which have the property of being s-selected by the verb.}. The KP continues to have all the features in (17), except that [Person] is uninterpretable (20).
\end{styleStandard}

\begin{styleStandard}
(20)\ \ KP[ \textit{u}Person, +D, ${\pm}$Def, \textit{i}${\varphi}$, \textit{u}Case: Dat]
\end{styleStandard}

\begin{styleStandard}
Given this feature structure a pronominal clitic is required to derivationally supply an [\textit{i}Person] feature. Clitics are known to be sensitive to features like [+D, +Def, …] and cannot combine with nominal projections smaller than DP. They may, however, combine with projections larger than DPs, i.e. KPs, where these features are specified, since they percolate from the D-head. 
\end{styleStandard}

\begin{styleStandard}
Concluding, \textit{la}+DP constituents are KPs, where K is a dative head. With verbs of movement and location including ditransitive ones, \textit{la} + DP are also still analyzable as PPs expressing Goal/ Location.
\end{styleStandard}

\begin{styleStandard}
\textbf{2.4. The internal structure of the inflectional dative phrase}
\end{styleStandard}

\begin{styleStandard}
The analysis of [\textit{la}\textit{\textsubscript{K}}] above suggests a parallel treatment for the dative morphology, K\textsubscript{dative}, which I will also consider a Person exponent. Nouns inflected for the dative are endowed with [\textit{u}Person\_\_], given their sensitivity to the AH. This feature is valued KP-internally, when K\textsubscript{dative} has an interpretable Person feature, i.e. K is [\textit{i}Person, Case-Dative\_\_\_]. Alternatively, if K’s semantic feature is bleached, then K\textsubscript{dative} is [\textit{u}Person] and simply realizes Case. In such situations, CD is obligatory and [\textit{u}Person] is checked KP-externally, using a clitic derivation. 
\end{styleStandard}

\begin{styleStandard}
Importantly, like\textit{ la-}\textsc{dat}\textit{, }Inf-\textsc{dat }are \textit{ambiguous between a KP and a PP categorization}. The PP analysis is, for example, required for adjectives that select Inf-\textsc{dat }complements (e.g. \textit{util} ‘useful’, \textit{folositor} ‘useful’, \textit{necesar} ‘necessary’). Since adjectives are not case-assigners, the Dative is licensed by a null preposition which finally incorporates into the adjective.
\end{styleStandard}

\begin{styleStandard}
Inside \textit{vP}, when the Inf-\textsc{dat} is CD-ed or, at least, may have been CD-ed, the Inf-\textsc{dat} is analyzable as a KP. However, when doubling is impossible, the Inf-\textsc{dat} \textit{must be projected as a PP, since otherwise it cannot check either Case or Person}. One example is that of sentences containing two Inf-\textsc{ dat phrases}, of which the higher \textit{must} be CD-ed and the lower \textit{cannot} be CD-ed (since they compete for the same \textit{v}P internal position at some point).
\end{styleStandard}

\begin{styleStandard}
(21)\ \ \textit{Ion\ \ }\textbf{\textit{şi}}\textit{=a\ \ vândut \ \ \ \ casa \ \ }\textbf{\textit{unor}}\textit{ \ \ \ \ \ rude/\ \  }\textbf{\textit{la }}\textit{nişte rude}.
\end{styleStandard}

\begin{styleStandard}
\ \ Ion \ \ he.\textsc{refl.}\textbf{\textsc{dat}}.=has sold house.the some.\textbf{\textsc{dat }}relatives/\textbf{at} some relatives
\end{styleStandard}

\begin{styleStandard}
\ \ ‘Ion sold his house to some relatives.’
\end{styleStandard}

\begin{styleStandard}
\textit{Some results}
\end{styleStandard}

\begin{styleStandard}
1. Datives inside \textit{v}P –whether \textit{la}{}-\textsc{ dat} or Inf-\textsc{ dat} - are uniformly either KPs or PPs.
\end{styleStandard}

\begin{styleStandard}
2. \textit{La-} and K\textsubscript{dative} are exponents of Person which encode sensitivity to the AH.
\end{styleStandard}

\begin{styleStandard}
3. When K is [\textit{i}Person], the Person feature of datives is checked KP-internally, while the Case feature is checked derivationally. The clitic is unnecessary and thus impossible.
\end{styleStandard}

\begin{styleStandard}
4. When K is [\textit{u}Person], the ultimate exponent of Person is the clitic, whose presence is mandatory.
\end{styleStandard}

\begin{styleStandard}
\ \ \textit{A consequence}
\end{styleStandard}

\begin{styleStandard}
Given the feature structure of datives [\textit{u/i} Person, \textit{u }Case: Dative], the applicative verb that licenses them should be endowed with the following features: V\textsubscript{appl}[ \textit{u}Person, \textit{u}Case:\_\_\_].
\end{styleStandard}

\begin{stylelsSectioni}
3. Briefly on the syntax of Romanian DOM
\end{stylelsSectioni}

\subsection[3.1. Background]{\textbf{3.1. Background}}
\begin{styleStandard}
The obligatory marker of Romanian DOM is the space preposition \textit{pe} ‘on’/ ‘towards’/ ‘against’, similar to Spanish \textit{a}. Unlike \textit{a, }however\textit{, pe }assigns accusative case to its object. Therefore, Romanian is not among the many languages where DOM-ed DO and IOs share the same dative/oblique case, sameness of case representing an explicit connection between the two (Manzini \& Franco 2016).
\end{styleStandard}

\begin{styleStandard}
One of the reviewers stresses that DOM \textit{pe} derives from the \textit{directional }uses of the Old Romanian (=OR) preposition \textit{p(r)e}, which was often used with directional/Goal verbs (e.g. \textit{striga} ‘call’, \textit{asculta} ‘listen to’, \textit{întreba} ‘ask’), as well as with verbs which entailed the presence of an opponent (e.g. \textit{lupta} ‘fight’), as in the following example: 
\end{styleStandard}

\begin{styleStandard}
(22)\ \ Old Romanian (Hill \& Mardale 2017: 395).
\end{styleStandard}

\begin{styleStandard}
\textit{au \ ascultat\ \ }\textbf{\textit{pre mine}}
\end{styleStandard}

\begin{styleStandard}
\ \ have listened \ \ \textsc{dom} me
\end{styleStandard}

\begin{styleStandard}
\ \ ‘they have listened to me’
\end{styleStandard}

\begin{styleStandard}
Significant research on the history of DOM has demonstrated that in OR the presence of the functional preposition \textit{p(r)e} was a means of upgrading the object, signaling a \textit{contrastive topic }interpretation (Hill 2013, Hill \& Mardale, 2017). Furthermore, in OR, \textit{p(r)e} was not restricted to animate nouns, as shown in (23) below:
\end{styleStandard}

\begin{styleStandard}
(23)\ \ Old Romanian\ \ (Hill \& Mardale 2017: 396)
\end{styleStandard}

\begin{styleStandard}
\textit{şi deaderă\ \ lui \ Iacov}\textbf{\textit{\ \ pre bozii\ \ }}\textit{cei\ \ striini} .
\end{styleStandard}

\begin{styleStandard}
\ \ and gave \ \ \textsc{dat }Jakob\ \ \textsc{dom }weeds.the \ the foreign
\end{styleStandard}

\begin{styleStandard}
\ \ ‘And they gave to Jakob the foreign weeds’
\end{styleStandard}

\begin{styleStandard}
In Modern Romanian (ModR), the noun classes compatible with DOM have been reduced to animate, predominantly [+human] nouns. This restriction is in line with the change in the discourse function of DOM, “from a marker of Contrastive Topic […] to a \textit{backgrounding device} for the [+human] noun in the discourse (Hill 2013:147)”. Thus in ModR, the most frequent discourse role of DOM-ed objects is that of \textit{familarity topic}, a role which is strengthened by the frequent association of DOM with CD (Hill \& Mardale 2017).
\end{styleStandard}

\begin{styleStandard}
Reinterpreting these important results in the framework of our analysis, it follows that although they do not share Case, DOM-ed DOs and IOs share other properties in Romanian, too. Thus, DOM is sensitive to the AH, which means that both DOM-ed DOs and IOs grammaticalize [Person].
\end{styleStandard}

\begin{styleStandard}
Similarly, the DOM marker \textit{pe} ‘on’/ ‘to(wards)’ can easily be analyzed as a K head (Lopez 2011, Hill \& Mardale 2017), a \textit{spell-out of Person}, behaving in all respects like dative \textit{la}, except that \textit{pe}{}-phrases check an accusative feature. Tentatively, the feature structure of \textit{pe}{}-KPs is as follows: [\textit{u/i}Person, \textit{u}Case:Acc]. When \textit{pe} selects the [\textit{u}Person] option, a clitic extends the KP, forming a chain that ultimately values the [\textit{u}Person] feature.\textit{ }
\end{styleStandard}

\begin{styleStandard}
In harmony with its familiar topic discourse role, DOM is also sensitive to the DefH (24), which arranges nominal expression by order of their referential stability. Thus, DOM is obligatory for personal pronouns and proper names, which are always referentially stable, it is felicitous but optional with definite and indefinite DPs, and it is impossible with determinerless nouns.
\end{styleStandard}

\begin{styleStandard}
(24)\ \ personal pronouns {\textgreater} proper names//{\textgreater} definite phrases {\textgreater} indefinite specific{\textgreater} indefinite non-specific {\textgreater} // bare plurals{\textgreater} bare singular.
\end{styleStandard}

\begin{styleStandard}
In its turn, CD is \textit{possible} and \textit{optional} for all accusative KPs, while being \textit{obligatory }only for \textit{personal pronouns}. Finally CD is not possible for bare DOs, i.e. it operates on KPs, not DPs, presumably because only KPs are marked for [Person].
\end{styleStandard}

\subsection[3.2. The syntax of DOM]{\textbf{3.2. The syntax of DOM}}
\begin{styleStandard}
As for the syntax of DOM, I have provisionally adapted to Romanian the analysis in Lopez (2011). Lopez maintains the classical view that accusative case originates in \textit{v}. In DOM languages there are two strategies of checking the accusative. Some objects remain \textit{in situ} and satisfy their Case requirement by \textit{incorporation} into the lexical verb V, which finally incorporates into \textit{v}. DOM-ed objects \textit{scramble} to the specifier of an $\alpha $P located between the little \textit{v} and the lexical VP, a position where they are directly probed by little \textit{v, }as in (25).
\end{styleStandard}

\begin{styleStandard}
\ (25)\ \ \ \ \textit{v}P
\end{styleStandard}

\begin{styleStandard}
\ \ ru
\end{styleStandard}

\begin{styleStandard}
\ \ Su\ \ \ \ v’
\end{styleStandard}

\begin{styleStandard}
\ \ \ \  \ ru
\end{styleStandard}

\begin{styleStandard}
\ \ \ \ \textit{v}\ \ \ \ $\alpha $P
\end{styleStandard}

\begin{styleStandard}
\ \ \ \ \ \ ru
\end{styleStandard}

\begin{styleStandard}
\ \ \ \ \ \ $\alpha $\ \ \ \ VP
\end{styleStandard}

\begin{styleStandard}
\ \ \ \ \ \ \ \  \ \ \ ru
\end{styleStandard}

\begin{styleStandard}
\ \ \ \ \ \ \ \  \ \ \ V\ \ \ \ DO
\end{styleStandard}

\begin{styleStandard}
The background assumption is that the grammar operates with nominals of different sizes (26), which may have different syntactic and semantic properties. 
\end{styleStandard}

\begin{styleStandard}
(26)\ \ KP\ \ {\textgreater}\ \ DP\ \ {\textgreater}\ \ //\ \ NumP\ \ {\textgreater} NP
\end{styleStandard}

\begin{styleStandard}
In Romanian the cut-off point between objects that scramble and objects that remain \textit{in situ} is the NumP: i.e. NumP and NPs remain \textit{in situ}, DPs may scramble, KPs \textit{must} scramble. On the semantic side, \textit{in situ} objects are interpreted as \textit{predicates,} objects that scramble are interpreted as \textit{arguments.}
\end{styleStandard}

\begin{stylelsSectioni}
4. Dative clitics and CD-Theory
\end{stylelsSectioni}

\subsection[4.1. On clitics]{\textbf{4.1. On clitics}}
\begin{styleStandard}
As already shown, with CD, both dative and accusative clitics select KPs [\textit{u}Person], showing sensitivity to the AH. Accusative clitics also observe the DefH. For instance they exclude bare quantifiers; in contrast, dative doubling is possible for \textit{any nominal} provided that it has an overt determiner (Cornilescu 2017).
\end{styleStandard}

\begin{styleStandard}
For the current analysis what matters most is that CD-ed DOs and IOs \textit{exit the v}P, passing through a \textit{v}P-periphery position which allows them to bind and out scope the subject in Spec, \textit{v}P (Dobrovie Sorin 1994, Tigău 2010, CDT 2017). Binding of the subject is impossible for undoubled objects. Thus in (27), the CD-ed dative \textit{fiecărui profesor ‘}every.\textsc{dat}professor\textit{’ }binds and outscopes the preverbal subject \textit{câte doi studenţi }‘some two students’\textit{. }Similarly, in (28), the post-verbal doubled DO may bind a possessive in the preverbal subject, but this is not possible for the undoubled DO. 
\end{styleStandard}

\begin{styleStandard}
(27) \ \ Romanian
\end{styleStandard}

\begin{styleStandard}
\textit{Câte doi studenţi\ \ i=au \ \ \ \ ajutat\ \ fiecărui profesor.}
\end{styleStandard}

\begin{styleStandard}
\ \ some two students \ \ he.\textsc{dat}=have \ \ helped \ \ each.dat professor
\end{styleStandard}

\begin{styleStandard}
\ \ ‘Each professor was helped by two students.’
\end{styleStandard}

\begin{styleStandard}
(28)\ \ Romanian
\end{styleStandard}

\begin{styleStandard}
a.\textit{ \ \ Muzica lor}\textit{\textsubscript{i}}\textit{\ \ îi =plictiseşte\ \ \ \ pe mulţi}\textit{\textsubscript{i}}\textit{/}\textit{\textsubscript{j}}
\end{styleStandard}

\begin{styleStandard}
\ \ \ \ music.the their\ \ they.\textsc{acc} bores\ \ \ \ \textsc{dom} many
\end{styleStandard}

\begin{styleStandard}
\ \ \ \ ‘Their own music bores many people.’
\end{styleStandard}

\begin{styleStandard}
\ \ b.\textit{ \ \ Muzica lor}\textit{\textsubscript{*i/j}}\textit{\ \ \ \ plictiseşte\ \ pe \ mulţi}\textit{\textsubscript{i}}\textit{.}
\end{styleStandard}

\begin{styleStandard}
\ \ \ \ music.the their\textsubscript{j\ \ \ \ }bores\ \ \ \ on \ many\textit{\textsubscript{i}}
\end{styleStandard}

\begin{styleStandard}
\ \ \ \ ‘Their music bores many people.’
\end{styleStandard}

\begin{styleStandard}
The identity of the \textit{v}P periphery projection through which clitics pass on the way to T is still under debate. Some researchers (e.g. Ciucivara 2009) propose that this is a projection where clitics check Case, while others argue that it is a PersonP at the \textit{v}P periphery (Belletti, 2004, Stegoveč, 2015), in whose specifier any [\textit{u}Person] nominal can value this feature (29). In line with the analysis above, I have adopted the second proposal. Since Person is an agreement feature, rather than an operator one, Spec, PersonP is an \textit{argumental position}. In conclusion, before going to the Person field above T (Ciucivara 2009), the clitic phrase reaches a \textit{Person P}, at the \textit{v}P periphery, above the subject constituent (29).
\end{styleStandard}

\begin{styleStandard}
(29)\ \ PersonP
\end{styleStandard}

\begin{styleStandard}
qp
\end{styleStandard}

\begin{styleStandard}
KP\ \ \ \ \ \ Person
\end{styleStandard}

\begin{styleStandard}
[\textit{u}Pers] \ \ \ \ \ \ \ \ \ \ qp
\end{styleStandard}

\begin{styleStandard}
\ \ \ \ Person\ \ \ \ \ \ \ \ \textit{v}P
\end{styleStandard}

\begin{styleStandard}
\ \ \ \ [\textit{i}Person]\ \ \ \ e
\end{styleStandard}

\begin{styleStandard}
\ \ \ \ \ \ \ \ Subject…
\end{styleStandard}

\subsection[4.2 A suitable clitic theory: Preminger 2016 ]{\textbf{4.2 A suitable clitic theory: Preminger 2016 }}
\begin{styleStandard}
Of the many available theories of CD, I have selected Preminger (2016), which is theoretically more motivated and also simpler; for instance, it does not require a big DP. Rather the starting point is a standard DP/KP. In Preminger’s interpretation, CD is an instance of \textit{long D-head movement}, as in (30). The D moves from its DP position and adjoins to little \textit{v}, skipping the V head (which is why this is an instance of long head movement).
\end{styleStandard}

\begin{styleStandard}
(30)\ \ \ \ \textit{v}P
\end{styleStandard}

\begin{styleStandard}
\ \ \ ei
\end{styleStandard}

\begin{styleStandard}
\ \ D\textsuperscript{0}{}-v\textsuperscript{0}\ \ \ \ VP
\end{styleStandard}

\begin{styleStandard}
\ \ \ \ ei
\end{styleStandard}

\begin{styleStandard}
\ \ \ \ V\ \ \ \ \textbf{DP}
\end{styleStandard}

\begin{styleStandard}
\ \ \ \ \ \ ei
\end{styleStandard}

\begin{styleStandard}
\ \ \ \ \ \ D\textsuperscript{0}\ \ \ \ NP
\end{styleStandard}

\begin{styleStandard}
What is specific to the CD chain is that \textit{both copies of D are pronounced}, the higher one is the clitic, the lower one is (part of) the associate DP. Pronunciation of two copies of a chain is allowed only if a phasal boundary is crossed (the DP boundary in (30)). The two copies are often phonologically distinct. 
\end{styleStandard}

\begin{stylelsSectioni}
5. On the syntax of ditransitives
\end{stylelsSectioni}

\subsection[5.1 Previous results]{\textbf{5.1 Previous results}}
\begin{styleStandard}
My analysis of ditransitives assumes the syntax of DOM above. For reasons presented in detail elsewhere (CDT 2017), I have adopted a \textit{classical derivational analysis} of the dative alternation (Harada \& Larson, 2009, Ormazabal \& Romero, 2017). Previous research on Romanian ditransitives (Diaconescu \& Rivero 2007), Cornilescu e.a. 2017) has brought to light several properties relevant for ditransitive binding configurations.
\end{styleStandard}

\begin{styleStandard}
a. Binding evidence points to the fact that in Romanian ditransitives the internal arguments show a Theme-over-Goal structure. Thus, as sentences (1) and (2) above indicate, the bare DO can bind, not only into an undoubled dative, but also into a doubled one. A Theme-over-Goal base configuration has also been argued for other Romance languages (see, for instance, Cepeda \& Cyrino (this volume) on Portuguese).
\end{styleStandard}

\begin{styleStandard}
b. In ditransitive constructions, the DO and IO show \textit{symmetric binding potential}, so that there is often an ambiguity between direct and inverse binding for the same pattern. The preferred reading is the one where the surface order corresponds to the direction of binding. For lack of space I will ignore these ambiguities in the analysis below.
\end{styleStandard}

\begin{styleStandard}
c. There is no difference between the CD-ed and the clitic-less constructions, as far as c-command configurational properties are concerned (CDT 2017), i.e. the DO and the IO have symmetric binding abilities \textit{irrespective of the presence of the clitic}.
\end{styleStandard}

\begin{styleStandard}
I claim that Romanian possesses a genuine alternation between a Prepositional Dative construction, similar to the \textit{to}{}-construction in English, and a pattern similar to DOC, where the dative is analyzed as a KP. In the Prepositional Dative construction, the P is \textit{null} and has the usual role of case-licensing its KP complement. If the null P incorporates, the dative is licensed by an applicative head with the features V\textsubscript{appl} [\textit{u}Person, \textit{u}Case:\_\_\_\_], for reasons explained in section 2.4 above.
\end{styleStandard}

\begin{styleStandard}
The \textit{focus of the analysis that follows }is to understand why some otherwise available binding configurations become degraded when the DO is DOM-ed.
\end{styleStandard}

\begin{styleStandard}
In order to bring out the contribution of DOM in ditransitive constructions, we compare derivations where the DO is a DP, not a KP, in which case it is not marked for [Person], with derivations in which the DO is DOM-ed, and has [Person] marking. The IO is also successively a PP, a KP, a cl+KP.
\end{styleStandard}

\subsection[5.2. The DO is a DP (i.e. it is not DOM{}-ed)]{\textbf{5.2. The DO is a DP (i.e. it is not DOM-ed)}}
\begin{styleStandard}
\textit{In the basic ditransitive configuration }the dative is a PP. This configuration, which corresponds to example (1) above \textit{unambiguously }expresses a Theme {\textgreater} Goal interpretation (well-attested). The null P checks Case, and K is [\textit{i}Person], irrespective of whether the IO is an Inf-\textsc{dat} or a \textit{la}{}-\textsc{dat}.
\end{styleStandard}

\begin{styleStandard}
(31)\ \ \ \ \ \ VP
\end{styleStandard}

\begin{styleStandard}
\ \ \ \ \ \ \ \ \ \ \ \ \ \ \ \ \ \ \ \ \ \ \ \ \ \ ep
\end{styleStandard}

\begin{styleStandard}
\ \ \ \ DP\textsubscript{theme}\ \ \ \  \ \ \ V’
\end{styleStandard}

\begin{styleStandard}
Case:[\textsc{acc}]\ \ \ \ to
\end{styleStandard}

\begin{styleStandard}
\ \ \ \ V\ \ \ \ PP
\end{styleStandard}

\begin{styleStandard}
\ \ \ \ \ \ ei
\end{styleStandard}

\begin{styleStandard}
\ \ \ \ \ \ P\ \ \ \ KP
\end{styleStandard}

\begin{styleStandard}
\ \ \ \ \ \ \ \ \ \ \ \ [${\varnothing}$]\ \ [\textit{u}Case:\textsc{dat}, \textit{i}Pers]
\end{styleStandard}

\begin{styleStandard}
When null P incorporates, as in (32), V\textsubscript{appl} [\textit{u}Pers, \textit{u}Case:\_\_\_] is projected. In (32), both nominals in the domain of V\textsubscript{appl} could value the Case feature of V\textsubscript{appl}, but only the Goal can value its [\textit{u}Pers\_\_\_] feature, since the Theme is a DP not marked for [Person]. Suppose a derivation is intended where the IO \textit{binds} and \textit{precedes} the DO, as in example (33) below. In this case, the DO need not move, while the IO should raise past it to Spec, Appl. This derivation is straightforward. V\textsubscript{appl }is allowed to case-license the Theme first, since V\textsubscript{appl} encounters the DO first, when it probes its domain. Next, adopting the locality theory in Dogget (2004) in order to maintain the standard direction of Agree, the Goal moves to an outer Spec,VP, above the Theme. In this position it can be probed by V\textsubscript{appl}, which thus values its own [\textit{u}Pers] feature. At the following step, the Goal KP moves further up to Spec, V\textsubscript{appl}P where it checks Case by Agree with little\textit{ v}. 
\end{styleStandard}

\begin{styleStandard}
(32)\ \ \ \ \textit{v}P
\end{styleStandard}

\begin{styleStandard}
\ \ ep
\end{styleStandard}

\begin{styleStandard}
\ \ \textit{v}\ \ \ \  \ \ \ \ \ \ V\textsubscript{appl}P
\end{styleStandard}

\begin{styleStandard}
\ \ \ \ \ \ ep
\end{styleStandard}

\begin{styleStandard}
\ \ \ \ V\textsubscript{appl}\ \ \ \ \ \ \ \ VP
\end{styleStandard}

\begin{styleStandard}
\ \ \ \ [\textit{u}Pers:\_\_, \textit{u}Case:\_\_]\ \ ep
\end{styleStandard}

\begin{styleStandard}
\ \ \ \ \ \ \ \ \ \ DP\textsubscript{theme}\ \ \ \ \ \ V’
\end{styleStandard}

\begin{styleStandard}
Case:\textsc{acc}]\ \ ei
\end{styleStandard}

\begin{styleStandard}
\ \ \ \ V\ \ \ \ \ \ KP\textsubscript{goal}
\end{styleStandard}

\begin{styleStandard}
\ \ \ \ \ \ \ \ \ \ P+V\ \ \ \ [\textit{u}Case:\textsc{dat}, \textit{i}Pers
\end{styleStandard}

\begin{styleStandard}
(33)\ \ Romanian (CDTb 2017: 201)
\end{styleStandard}

\begin{styleStandard}
IO before DO; IO {\textgreater} DO 
\end{styleStandard}

\begin{styleStandard}
\textit{\ \ Recep[21B?]ionerul\ \ arătă\ \ \ \ }\textbf{\textit{fiecărui turist}}\textbf{\textit{\textsubscript{i}}}\textit{\ \ \ \ camera }\textbf{\textit{lui}}\textbf{\textit{\textsubscript{i}}}\textit{,.}
\end{styleStandard}

\begin{styleStandard}
\ \ receptionist.the\ \ showed \ \ each.\textsc{dat} tourist \ \ room.the his.
\end{styleStandard}

\begin{styleStandard}
\ \ ‘The receptionist showed each tourist his room.’
\end{styleStandard}

\begin{styleStandard}
Cliticization is unnecessary, since the Goal is s-selected, and its Person feature is interpretable. Symmetric binding is predicted to be available, since in the initial structure Theme c-commands Goal, and in the derived structure(s) Goal c-commands Theme. Next we consider (34), where a CD-ed IO binds and precedes a bare DO. 
\end{styleStandard}

\begin{styleStandard}
(34)\ \ \textit{Statul\ \ \ \ }\textbf{\textit{le}}\textit{=a \ \ restituit\ \ \ \ }\textbf{\textit{foştilor proprietari}}\textit{\ \ \ \ }
\end{styleStandard}

\begin{styleStandard}
state.the t\ \ hey.\textsc{dat}=has returned \ \ former.the.\textsc{dat} \ \ \ \ \ \ owners 
\end{styleStandard}

\begin{styleStandard}
\textit{casele \ \ \ \ na[21B?]ionalizate}.
\end{styleStandard}

\begin{styleStandard}
houses.the\ \ natioanlized
\end{styleStandard}

\begin{styleStandard}
\ \ ‘The state returned the nationalized houses to their former owners.’
\end{styleStandard}

\begin{styleStandard}
The presence of the clitic shows that the dative KP is [\textit{u}Pers], as in (35). For the sake of simplicity I will again consider a derivation where the DO does not scramble and V\textsubscript{appl} checks its Case feature through Agree. At this point, both of the Goal’s features are unchecked, and V\textsubscript{appl }still has an unvalued [\textit{u}Person] feature.
\end{styleStandard}

\begin{styleStandard}
(35)\ \ \ \ \ \ ApplP
\end{styleStandard}

\begin{styleStandard}
qp
\end{styleStandard}

\begin{styleStandard}
\ \ Appl\ \ \ \ \ \ \ \ VP
\end{styleStandard}

\begin{styleStandard}
[\textit{u}Pers:\_\_] \ \ \ \ \ \ \ \ \ \ \ \ \ \ \ \ \ \ \ \ \ \ \ qp
\end{styleStandard}

\begin{styleStandard}
[\textit{u}Case:\_\_\_]\ \ \ \ DP\textsubscript{Theme}\ \ \ \ V’
\end{styleStandard}

\begin{styleStandard}
\ \ z-{}-{}-{}-{}-{}- [Case:\textsc{acc}]\ \ \ \ wp
\end{styleStandard}

\begin{styleStandard}
V\ \ \ \ \ \ KP\textsubscript{Goal} [\textit{u}Pers:\_\_\_, Case:\textsc{dat}]
\end{styleStandard}

\begin{styleStandard}
The Goal moves to a position (an outer specifier of VP) where it is accessible to V\textsubscript{appl }and there is Agree between V\textsubscript{appl} and the dative, which now shares a [\textit{u}Person] feature, but \textit{neither feature is deleted}, since both occurrences of the features are \textit{unvalued and uninterpretable}. The two features are related by agreement and count as instances of the same feature (Pesetsky \&Torrego 2007). As in the preceding derivation, the Goal raises to Spec, Appl and checks Case with little \textit{v}, but its [\textit{u}Person] feature is still unvalued. This is what \textit{forces movement to the PersonP, at the vP-periphery}, as in (36). When all the features of the Goal have been valued, the Goal undergoes cliticization.
\end{styleStandard}

\begin{styleStandard}
(36)\ \ PersP
\end{styleStandard}

\begin{styleStandard}
ei
\end{styleStandard}

\begin{styleStandard}
Pers\ \ \ \ \textit{v}P
\end{styleStandard}

\begin{styleStandard}
[\textbf{\textit{i}}\textbf{Pers}] \ ri
\end{styleStandard}

\begin{styleStandard}
\ \ DP\textsubscript{Goal}\ \ \ \ \textit{v}P
\end{styleStandard}

\begin{styleStandard}
\ \ [\textbf{\textit{u}}\textbf{Pers}] \ \ \ \ ro
\end{styleStandard}

\begin{styleStandard}
\ \ \ \ Su\ \ \ \ \textit{v}’
\end{styleStandard}

\begin{styleStandard}
\ \ \ \ \ \ ei
\end{styleStandard}

\begin{styleStandard}
\ \ \ \ \ \ \textit{v}\ \ \ \ V\textsubscript{Appl}P
\end{styleStandard}

\begin{styleStandard}
\ \ \ \ \ \ \ \ ei
\end{styleStandard}

\begin{styleStandard}
\ \ \ \ \ \ \ \ DP\textsubscript{Goal}\ \ \ \ V\textsubscript{Appl}’
\end{styleStandard}

\begin{styleStandard}
\ \ \ \ \ \ \ \ \ \ ei
\end{styleStandard}

\begin{styleStandard}
\ \ \ \ \ \ \ \ \ \ V\textsubscript{Appl}\ \ \ \ VP
\end{styleStandard}

\begin{styleStandard}
\ \ \ \ \ \ \ \ \ \ [\textit{u}Pers]ei
\end{styleStandard}

\begin{styleStandard}
\ \ \ \ \ \ \ \ \ \ z-{}-{}-DP\textsubscript{Goal}\ \ \ \ VP
\end{styleStandard}

\begin{styleStandard}
\ \ \ \ \ \ \ \ \ \ \ \ [\textit{u}Pers]ei
\end{styleStandard}

\begin{styleStandard}
\ \ \ \ \ \ \ \ \ \ \ \ \ \ DP\textsubscript{Theme}\ \ V
\end{styleStandard}

\begin{styleStandard}
\ \ \ \ \ \ \ \ \ \ \ \ \ \ \ \ ei
\end{styleStandard}

\begin{styleStandard}
\ \ \ \ \ \ \ \ \ \ \ \ \ \ \ \ V\ \ DP\textsubscript{Goal}
\end{styleStandard}

\begin{styleStandard}
CD was obligatory because the Goal’s Person feature could not be checked inside \textit{v}P.
\end{styleStandard}

\subsection[5.3 When DOM{}-ed themes and dative goals combine]{\textbf{5.3 When DOM-ed themes and dative goals combine}}
\begin{styleStandard}
Sentences with DOM and datives create locality problems, since both objects are KPs marked for the same [\textit{i/u} Person] feature and both may value the [\textit{u}Person] feature of V\textsubscript{appl}. The empirical facts are summed up in (37):
\end{styleStandard}

\begin{styleStandard}
(37)\ \ 
\end{styleStandard}

\begin{styleStandard}
a) A \textit{pe}{}-marked DO binds an undoubled IO without problems (sentence (3) above) 
\end{styleStandard}

\begin{styleStandard}
b) A \textit{pe}{}-marked DO cannot bind a CD-ed IO (sentence (4) above). 
\end{styleStandard}

\begin{styleStandard}
c) A CD-ed \textit{pe-}marked Object can bind an IO, irrespective of CD (sentences (5)-(6) above). 
\end{styleStandard}

\begin{styleStandard}
These facts follow from the analysis. A natural explanation for why a \textit{pe}{}-marked object can bind an IO (= (37a)) is that, in this case the IO stays low and \textit{may be (re)analyzed as a PP}, thus not competing with the DO. 
\end{styleStandard}

\begin{styleStandard}
(38)\ \ \ \ \textit{v}P
\end{styleStandard}

\begin{styleStandard}
\ \ \ \ \ \ \ \ \ qp
\end{styleStandard}

\begin{styleStandard}
\textit{v}\ \ \ \ \ \  \ \ \ \ \ ${\alpha}$P
\end{styleStandard}

\begin{styleStandard}
\ \ [\textit{u}Case:\_\_]\ \  \ \ qp
\end{styleStandard}

\begin{styleStandard}
\ \ \ \ \ \ KP\textsubscript{DO}\ \ \ \ \ \ ${\alpha}$P
\end{styleStandard}

\begin{styleStandard}
\ \ \ \ \ \ \textit{i}Pers\ \ \ \ eo
\end{styleStandard}

\begin{styleStandard}
\ \ \ \ \ \ \textit{u}Case:\textsc{acc}\ \ ${\alpha}$’\ \ \ \  \ VP
\end{styleStandard}

\begin{styleStandard}
\ \ \ \ \ \ \ \ \ \ \ \ eo
\end{styleStandard}

\begin{styleStandard}
\ \ \ \ \ \ \ \ \ \ \ \ {\textless}KP\textsubscript{DO}{\textgreater}\ \  \ \ \ \ \ \ V’
\end{styleStandard}

\begin{styleStandard}
\ \ \ \ \ \ \ \ \ \ \ \ \ \ eo
\end{styleStandard}

\begin{styleStandard}
\ \ \ \ \ \ \ \ \ \ \ \ \ \  \ \ \ \ \ \ V\ \ \ \ PP
\end{styleStandard}

\begin{styleStandard}
\ \ \ \ \ \ \ \ \ \ \ \ \ \  \ \ eo
\end{styleStandard}

\begin{styleStandard}
\ \ \ \ \ \ \ \ \ \ \ \ \ \  \ \ \ \ \ \ \ \ P\ \ \ \ KP\textsubscript{IO}
\end{styleStandard}

\begin{styleStandard}
The \textit{pe}{}-marked DO in (38) scrambles, and it is only for this reason that a landing site is projected between little \textit{v} and VP, as in Lopez’s analysis. The DO is [\textit{i}Person] and does not need to move beyond its case checking position (Spec, ${\alpha}$P). Let me turn to situations (37b-c) now. When the IO is CD-ed and there is DOM, the result is ungrammatical, as in sentence (4) above. A CD-ed \textit{pe}{}-object restores grammaticality, as in (5) above. Since CD-ed DOM-ed objects are unproblematic, it could be suggested that sentence (4) is ungrammatical because, at the current stage in the evolution of Romanian, \textit{pe}{}-DOs are well-formed only \textit{if they are also CD-ed}. The following Google example shows however, that CD is \textit{not obligatory} for \textit{pe}{}-DOs, except for personal pronouns.
\end{styleStandard}

\begin{styleStandard}
(39)\ \ Romanian (Google)
\end{styleStandard}

\begin{styleStandard}
\textit{Zavaidoc a \ \ tocmit\ \ }\textbf{\textit{pe \ un \ asasin}}\textit{\ \ care a \ \ injunghiat=o \ \ mortal }
\end{styleStandard}

\begin{styleStandard}
Zavaidoc has \ \ hired \ \ \textsc{dom }an assassin who has stabbed=she.\textsc{acc} mortally
\end{styleStandard}

\begin{styleStandard}
\textit{pe \ Zaraza}.
\end{styleStandard}

\begin{styleStandard}
\textsc{dom}Zaraza
\end{styleStandard}

\begin{styleStandard}
\ \ ‘Zavaidoc hired an assassin who stabbed Zaraza to death.’
\end{styleStandard}

\begin{styleStandard}
Therefore, the marginality of (4) cannot be attributed to the absence of the clitic, but to some kind of “interference” between the \textit{pe}{}-DOs and CD-ed IOs. I suggest that the problem concerns the locality of Agree, interfering with the feature structure of the two objects.
\end{styleStandard}

\begin{styleStandard}
Consider the following intermediate stage (40) in the derivation of sentences like (4). If the IO is CD-ed, then its Person feature is uninterpretable and the dative KP must check both Person and Case against appropriate functional heads. On the other hand the DOM-ed DO is [\textit{i}Pers] (since it does not need a clitic) and must only value its Case.
\end{styleStandard}

\begin{styleStandard}
When V\textsubscript{Appl} probes its c-command domain, the DOM-ed object is the first that it encounters, so V\textsubscript{Appl} agrees with the \textit{closer goal} and values its own Person and Case features and it further attracts the KP-DO to its Specifier, since, by assumption, DOM-ed DOs scramble (Lopez 2011). The IO is trapped in its merge position, and cannot check Case and Person anymore, so that the derivation crashes.
\end{styleStandard}

\begin{styleStandard}
(40)
\end{styleStandard}

\begin{styleStandard}
\ \ \textit{v}P
\end{styleStandard}

\begin{styleStandard}
eo
\end{styleStandard}

\begin{styleStandard}
\textit{v}\ \ \ \ V\textsubscript{Appl}P
\end{styleStandard}

\begin{styleStandard}
qp
\end{styleStandard}

\begin{styleStandard}
\ \ V\textsubscript{Appl\ \ }\ \ \ \ VP
\end{styleStandard}

\begin{styleStandard}
\ \ \textbf{\textit{u}}\textbf{Pers\ \ \ \ }qp
\end{styleStandard}

\begin{styleStandard}
\ \ \textit{u}Case:Acc\ \ KP\textsubscript{DO\ \ \ \ \ \ }V’
\end{styleStandard}

\begin{styleStandard}
\ \ !\ \ \ \ \textbf{\textit{i}}\textbf{Pers\ \ \ \ }eo
\end{styleStandard}

\begin{styleStandard}
\ \ z\_\_\_\_\_\_\_\_\_\_\_\_\_\_\textit{u}Case:Acc\ \ V\ \ \ \ KP\textsubscript{IO}
\end{styleStandard}

\begin{styleStandard}
\ \ \ \ \ \ \ \ \ \ \ \ \ \ [\textbf{\textit{u}}\textbf{Pers}]
\end{styleStandard}

\begin{styleStandard}
\ \ \ \ \ \ \ \ \ \ \ \ \ \ [\textit{u}Case:Dat]
\end{styleStandard}

\begin{styleStandard}
The problem disappears if the DO is CD-ed, as in sentences (5), (6) above. For simplicity’s sake I will examine sentences where the CD-ed \textit{pe}{}-DO binds an undoubled IO. In this case, the \textit{pe}{}-DP is endowed with an uninterpretable Person feature, which will be checked in the \textit{v}P periphery Person P, just as with datives.
\end{styleStandard}

\begin{styleStandard}
The accusative clitic’s role is syntactic: intuitively “it moves the Theme out of the Goal’s way” (Anagnostopoulou 2006). The DO moves to Spec, V\textsubscript{Appl}, a position where it can be probed by little \textit{v} which checks its accusative Case. Next it targets the PersonP at the \textit{v}P periphery, where it Agrees with the [\textit{i}Person] head and values [\textit{u}Person]. When all the DO’s features have been checked, cliticization is mandatory. The features of V\textsubscript{Appl} have not been valued yet and the IO is free to move to the outer Spec, VP, where the IO is probed by V\textsubscript{appl} checking its case. The IO, whose person feature is interpretable, values the [Person] feature of V\textsubscript{appl} and needs to raise no further. Resort to the Accusative clitic is a repair strategy: while the *DOM-ED DP \textsubscript{theme}{\textgreater}\textbf{cl}{}- DP \textsubscript{goal }pattern is severelydegraded\textsubscript{, }the pattern \textbf{cl}{}- DOM-ED DP \textsubscript{theme}{\textgreater}\textbf{cl}{}-DP \textsubscript{goal , }which differs from the preceding only through the presence of the accusative clitic, is fully grammatical.
\end{styleStandard}

\begin{styleStandard}
(41)\ \ 
\end{styleStandard}

\begin{styleStandard}
PersP
\end{styleStandard}

\begin{styleStandard}
\ \ \ \ \ ei
\end{styleStandard}

\begin{styleStandard}
Pers\ \ \ \ \textit{v}P
\end{styleStandard}

\begin{styleStandard}
[\textit{i}Pers] \ \ \ \ \ \ \ ei
\end{styleStandard}

\begin{styleStandard}
\ \ KP\textsubscript{theme}\ \ \ \ \ \ \textit{v}P
\end{styleStandard}

\begin{styleStandard}
\ \ [\textit{u}Person] \ \ ei
\end{styleStandard}

\begin{styleStandard}
\ \  \ \ [\textsc{acc}]\ \ DP\textsubscript{Agent}\ \ \ \ \textit{v}’
\end{styleStandard}

\begin{styleStandard}
\ \ \ \ \ \ \ \ ei
\end{styleStandard}

\begin{styleStandard}
\ \ \ \ \ \ \ \ \textit{v}\ \ \ \ V\textsubscript{appl}P
\end{styleStandard}

\begin{styleStandard}
\ \ \ \ \ \ \ \ [Case\_] \ \ \ \ wp
\end{styleStandard}

\begin{styleStandard}
\ \ \ \ \ \ \ \ \ \ KP\textsubscript{Theme}\ \ \ \ V’\textsubscript{appl}
\end{styleStandard}

\begin{styleStandard}
\ \ \ \ \ \ \ \ \ \ [\textit{u}Pers] \ \ \ \ \ \ \ \ \ \ \ \ ei
\end{styleStandard}

\begin{styleStandard}
\ \ \ \ \ \ \ \ \ \ [\textsc{acc}]\ \ V\textsubscript{Appl}\ \ \ \ VP
\end{styleStandard}

\begin{styleStandard}
\ \ \ \ \ \ \ \ \ \ \ \ \textit{u}Pers\ \ ei
\end{styleStandard}

\begin{styleStandard}
\ \ \ \ \ \ \ \ \ \ \ \ z\_\_\_\_\_\_KP\textsubscript{Goal}\ \ \ \ VP
\end{styleStandard}

\begin{styleStandard}
\ \ \ \ \ \ \ \ \ \ \ \ \ \ [\textit{i}Pers]\ \ ei
\end{styleStandard}

\begin{styleStandard}
\ \ \ \ \ \ \ \ \ \ \ \ \ \ KP\textsubscript{Theme}\ \ V’
\end{styleStandard}

\begin{styleStandard}
\ \ \ \ \ \ \ \ \ \ \ \ \ \ [\textit{u}Pers]ei
\end{styleStandard}

\begin{styleStandard}
\ \ \ \ \ \ \ \ \ \ \ \ \ \ \ \ V \ \ \ \ \ \ \ \ \ \ \ {\textless}KP\textsubscript{Goal}{\textgreater}
\end{styleStandard}

\begin{styleStandard}
\ \ \ \ \ \ \ \ \ \ \ \ \ \ \ \ \ \ [\textit{i}Pers]
\end{styleStandard}

\begin{stylelsSectioni}
6. Some theoretical implications of the analysis
\end{stylelsSectioni}

\begin{styleStandard}
Summing up the data we started with in (1) – (6) above and considering the categorial status of the arguments, as well as their (non)-clitic status, we obtain the patterns in (42). 
\end{styleStandard}

\begin{styleStandard}
(42)\ \  a.\ \ KP-DO\ \ \ \ *KP-IO/PP-IO
\end{styleStandard}

\begin{styleStandard}
b.\ \ Cl-KP\ \ \ \ \ \ KP IO
\end{styleStandard}

\begin{styleStandard}
c.\ \ Cl-KP\ \ \ \ \ \ Cl KP IO
\end{styleStandard}

\begin{styleStandard}
d.\ \ *KP-DO\ \ \ \ Cl-DP IO
\end{styleStandard}

\begin{styleStandard}
\ e.\ \ DP-DO\ \ \ \ (cl)-KP-IO 
\end{styleStandard}

\begin{styleStandard}
The critical property of the patterns is the need to check the [\textit{u}Person] against the Appl head. Sentences of type (42e), where the DO is a bare DP, which does not need to check Person, are fine irrespective of whether the IO is doubled or undoubled. In contrast, patterns (42a)-(42d) contain two nominals (KPs) that check Person, the DOM-ed direct object and the IO. These types of sentences rely on the configuration in (43), where the same Appl head should Agree with two arguments, a configuration familiar from the analysis of PCC effects (see Sheehan this volume and the references therein). 
\end{styleStandard}

\begin{styleStandard}
(43)\ \ Appl [\textit{u}Person]\ \ DOM DO [\textit{i}/\textit{u}Person]\ \ IO [\textit{i/u}Person]
\end{styleStandard}

\begin{styleStandard}
What differentiates between (42e) and (42a)-(42d) is that in (42a)- (42d), but not in (42e), not only the IO, but also the DO \textit{agrees with Appl.} Recall that according to Preminger (2017), PCC effects are likely to occur whenever the relevant DO agrees with \textit{v} or Appl. Indeed the distribution of the stars in (42a)- (42d) may be restated as a form of PCC, as also suggested for Spanish ditranstives with DOM, by Ormazabal \& Romero (2013).
\end{styleStandard}

\begin{styleStandard}
(44)\ \ \textit{PCC like effects in Romanian ditransitives}
\end{styleStandard}

\begin{styleStandard}
In a combination of DOM-ed DO and IO, the IO can be doubled (or a clitic) only if the DO is also doubled (or a clitic). 
\end{styleStandard}

\begin{styleStandard}
The admissible patterns in (42a)-(42d) fall in line with this generalization. Pattern (42a), where neither argument is provided with a clitic would be ungrammatical if the dative had been a KP[\textit{u}Person]. This ungrammaticality is not detected, since the dative is a second, locative argument and can be analyzed as a PP which checks the Case and Person feature of the DP, PP internally, as shown in the discussion of (38) above. Projection as a PP in (38) functions as a repair strategy. In the ungrammatical (42d), the undoubled DO blocks the lower clitic-doubled dative, preventing it from checking Person (and Case) and producing a PCC-like effect. Patterns (42b)-(42c) are fine since the DO and IO arguments check Person against different heads (Person P, ApplP, respectively), avoiding the problem of multiple arguments agreeing with the same head.
\end{styleStandard}

\begin{styleStandard}
Finally, the data analyzed in this paper provide further evidence for Sheehan’s (this volume) insight that PCC-like phenomena do not depend on (non)clitic status of the arguments, but on the emergence of a configuration of type (43). In the ungrammatical pattern (4)/(42d), the DO, in the intervener role, is not a clitic, only the IO is.
\end{styleStandard}

\begin{stylelsSectioni}
7. Conclusions
\end{stylelsSectioni}


\setcounter{listWWNumxxiiileveli}{0}
\begin{listWWNumxxiiileveli}
\item 
\begin{styleListParagraph}
DOM-ed DOs interfere with IOs since both are sensitive to AH, codified as [Person].
\end{styleListParagraph}
\item 
\begin{styleListParagraph}
The interaction of DOM-ed DO and IOs in Romanian is a classical locality problem based on the fact that the same applicative head matches two nominals in its c-command domain, regarding [Person]. The head agrees with the closer object, i.e. the DO. In such configurations, the IO must be a PP, i.e. it cannot be doubled.
\end{styleListParagraph}
\item 
\begin{styleListParagraph}
\ When the DO object is CD-ed, the IO may be a KP and accessing V\textsubscript{appl} and it may even be CD-ed. 
\end{styleListParagraph}
\end{listWWNumxxiiileveli}
\begin{styleStandard}
\textbf{References }
\end{styleStandard}

\begin{styleStandard}
Anagnostopoulou, Elena. 2006. \textit{Clitic doubling}. In Martin Everaert \& Henk van Riemsdijk, Henk (eds.), \textit{The Blackwell Companion to Syntax, }519-581. Malden: Blackwell. 
\end{styleStandard}

\begin{styleStandard}
Belletti, Adriana. 2004. Extended doubling and the VP periphery. \textit{Probus} 17(1). 1-36 
\end{styleStandard}

\begin{styleStandard}
Ciucivara, Oana. 2009. \textit{A syntactic analysis of pronominal clitic clusters in Romance}. New York: NYU dissertation.
\end{styleStandard}

\begin{styleStandard}
Cornilescu, Alexandra. 2017. Dative clitics and obligatory clitic doubling in Romanian. In Adina Dragomirescu et al. (eds.), \textit{Sintaxa ca mod de a fi. Omagiu Gabrielei Pană Dindelegan la aniversare}, 131-148. Bucureşti: Editura Universită[21B?]ii.
\end{styleStandard}

\begin{styleStandard}
Cornilescu, Alexandra, Anca Dinu \& Alina Tigău. 2017a. Experimental data on Romanian double object constructions \textstyleStrong{\textmd{\textit{Revue Roumaine de Linguistique}}\textmd{, 62(2). 157-177.}}
\end{styleStandard}

\begin{styleStandard}
Cornilescu, Alexandra, Anca Dinu \& Alina Tigău. 2017b. Romanian dative configurations: ditransitive verbs, a tentative analysis. \textstyleStrong{\textmd{\textit{Revue Roumaine de Linguistique}}\textmd{, 62(2). 179-206.}}
\end{styleStandard}

\begin{styleStandard}
Diaconescu, Rodica \& Maria Luisa Rivero. 2007. An applicative analysis of double object constructions in Romanian. \textit{Probus}. 19(2). 209-233
\end{styleStandard}

\begin{styleStandard}
Dobrovie-Sorin, Carmen. 1994. \textit{The syntax of Romanian}. Berlin: Mouton de Gruyter.
\end{styleStandard}

\begin{styleStandard}
Dogget, Teal. 2004. \textit{All things being unequal: locality in movement. }Cambridge, MA: MIT dissertation.
\end{styleStandard}

\begin{styleStandard}
Harada, Naomi \& Richard Larson. 2009. Datives in Japanese. In Ryosuke Shibagaki \& Reiko Vermeulen (eds.), \textit{Proceedings of the 5th Workshop on Altaic Formal Linguistics: MITWPL 54}, 109-120. Cambridge, MA: MITWPL.
\end{styleStandard}

\begin{styleStandard}
Hill, Virginia. 2013. The direct object marker in Romanian, a historical perspective. \textit{Journal of Australian Linguistics} 33(2). 140-151.
\end{styleStandard}

\begin{styleStandard}
Hill, Virginia \& Alexandru Mardale. 2017. On the interaction of differential object marking and clitic doubling in Romanian. \textit{Revue Roumaine de Linguistique }\textstyleStrong{\textmd{62(4).393-409.}}
\end{styleStandard}

\begin{styleStandard}
Lopez, Luis. 2011. \textit{Indefinite objects. Scrambling, choice functions, and differential marking}. Cambridge, Massachusetts: MIT
\end{styleStandard}

\begin{styleStandard}
Manzini Rita \&Ludovico Franco. 2016. Goal and DOM datives. \textit{Natural Language and Linguistic Theory}34(1). 197-204.
\end{styleStandard}

\begin{styleStandard}
Ormazabal Javier \& Juan Romero. 2013. Differential Object Marking, Case and Agreement. \textit{Borealis. An International Journal of Hispanic Linguistics }2: 211-239.
\end{styleStandard}

\begin{styleStandard}
Ormazabal Javier \& Juan Romero. 2017. Historical changes in Basque dative alternations: evidence for a P-based (neo) derivational analysis\textit{. Glossa} 2(1). 78.
\end{styleStandard}

\begin{styleStandard}
Pesetsky, David \& Esther Torrego. 2007. The syntax of valuation and the interpretability of features. In SiminKarimi, Vida Samiian, WendyK. Wilkins (eds.), \textit{Phrasal and Clausal Architecture: Syntactic Derivation and Interpretation}, 262-294. Amsterdam: John Benjamins.
\end{styleStandard}

\begin{styleStandard}
Rodríguez-Mondoñedo. 2007. \textit{The syntax of objects: Agree and differential object marking}. Storrs, Connecticut: University of Connecticut dissertation.
\end{styleStandard}

\begin{styleStandard}
Preminger, Omer. 2016. What the PCC tells us about “abstract” agreement, head movement and locality. Ms. University of Maryland.
\end{styleStandard}

\begin{styleStandard}
Richards Mark. 2008. Defective Agree, case alternation, and the prominence of person. \textit{Linguistische Arbeitsberichte} 86. 137-161
\end{styleStandard}

\begin{styleStandard}
Roberts Ian \& Anna Roussou 2003. \textit{Syntactic change}. Cambridge: Cambridge University Press.
\end{styleStandard}

\begin{styleStandard}
Stegoveč, Adrian. 2015. It’s not Case, it’s person! Reassessing the PCC and clitic restrictions in O’odham and Warlpiri. In Pocholo Umbal (ed.), \textit{Proceedings of the poster session of the 33rd West Coast Conference on Formal Linguistics: }Simon Fraser University \textit{Working Papers in Linguistics 5}, 131-140. 
\end{styleStandard}

\end{document}
