% This file was converted to LaTeX by Writer2LaTeX ver. 1.4
% see http://writer2latex.sourceforge.net for more info
\documentclass[12pt]{article}
\usepackage[utf8]{inputenc}
\usepackage[T1]{fontenc}
\usepackage[english]{babel}
\usepackage{amsmath}
\usepackage{amssymb,amsfonts,textcomp}
\usepackage{array}
\usepackage{hhline}
\usepackage{hyperref}
\hypersetup{colorlinks=true, linkcolor=blue, citecolor=blue, filecolor=blue, urlcolor=blue}
% footnotes configuration
\makeatletter
\renewcommand\thefootnote{\arabic{footnote}}
\makeatother
\newcommand\textsubscript[1]{\ensuremath{{}_{\text{#1}}}}
% Headings and outline numbering
\makeatletter
\renewcommand\section{\@startsection{section}{1}{0.0cm}{0in}{0.111in}{\normalfont\normalsize\fontsize{12pt}{14.4pt}\selectfont\rmfamily\raggedright}}
\renewcommand\@seccntformat[1]{\csname @textstyle#1\endcsname{\csname the#1\endcsname}\csname @distance#1\endcsname}
\setcounter{secnumdepth}{0}
\newcommand\@distancesection{}
\newcommand\@textstylesection[1]{#1}
\makeatother
\raggedbottom
% Paragraph styles
\renewcommand\familydefault{\rmdefault}
\newenvironment{stylelsAbstract}{\setlength\leftskip{0.5in}\setlength\rightskip{0.5in}\setlength\parindent{0in}\setlength\parfillskip{0pt plus 1fil}\setlength\parskip{0in plus 1pt}\writerlistparindent\writerlistleftskip\leavevmode\normalfont\normalsize\itshape\writerlistlabel\ignorespaces}{\unskip\vspace{0.111in plus 0.0111in}\par}
\newenvironment{stylelsSectioni}{\setlength\leftskip{0.25in}\setlength\rightskip{0in plus 1fil}\setlength\parindent{0in}\setlength\parfillskip{0pt plus 1fil}\setlength\parskip{0.1665in plus 0.016649999in}\writerlistparindent\writerlistleftskip\leavevmode\normalfont\normalsize\fontsize{18pt}{21.6pt}\selectfont\bfseries\writerlistlabel\ignorespaces}{\unskip\vspace{0.0835in plus 0.00835in}\par}
\newenvironment{styleStandard}{\setlength\leftskip{0cm}\setlength\rightskip{0cm plus 1fil}\setlength\parindent{0cm}\setlength\parfillskip{0pt plus 1fil}\setlength\parskip{0in plus 1pt}\writerlistparindent\writerlistleftskip\leavevmode\normalfont\normalsize\writerlistlabel\ignorespaces}{\unskip\vspace{0.111in plus 0.0111in}\par}
% List styles
\newcommand\writerlistleftskip{}
\newcommand\writerlistparindent{}
\newcommand\writerlistlabel{}
\newcommand\writerlistremovelabel{\aftergroup\let\aftergroup\writerlistparindent\aftergroup\relax\aftergroup\let\aftergroup\writerlistlabel\aftergroup\relax}
\newcounter{listWWNumxxiileveli}
\newcounter{listWWNumxxiilevelii}[listWWNumxxiileveli]
\newcounter{listWWNumxxiileveliii}[listWWNumxxiilevelii]
\newcounter{listWWNumxxiileveliv}[listWWNumxxiileveliii]
\renewcommand\thelistWWNumxxiileveli{\arabic{listWWNumxxiileveli}}
\renewcommand\thelistWWNumxxiilevelii{\arabic{listWWNumxxiileveli}.\arabic{listWWNumxxiilevelii}}
\renewcommand\thelistWWNumxxiileveliii{\arabic{listWWNumxxiileveli}.\arabic{listWWNumxxiilevelii}.\arabic{listWWNumxxiileveliii}}
\renewcommand\thelistWWNumxxiileveliv{\arabic{listWWNumxxiileveli}.\arabic{listWWNumxxiilevelii}.\arabic{listWWNumxxiileveliii}.\arabic{listWWNumxxiileveliv}}
\newcommand\labellistWWNumxxiileveli{\thelistWWNumxxiileveli.}
\newcommand\labellistWWNumxxiilevelii{\thelistWWNumxxiilevelii.}
\newcommand\labellistWWNumxxiileveliii{\thelistWWNumxxiileveliii.}
\newcommand\labellistWWNumxxiileveliv{\thelistWWNumxxiileveliv.}
\newenvironment{listWWNumxxiileveli}{\def\writerlistleftskip{\setlength\leftskip{0.5in}}\def\writerlistparindent{}\def\writerlistlabel{}\def\item{\def\writerlistparindent{\setlength\parindent{-0.25in}}\def\writerlistlabel{\stepcounter{listWWNumxxiileveli}\makebox[0cm][l]{\labellistWWNumxxiileveli}\hspace{-0.635cm}\writerlistremovelabel}}}{}
\newenvironment{listWWNumxxiilevelii}{\def\writerlistleftskip{\setlength\leftskip{1in}}\def\writerlistparindent{}\def\writerlistlabel{}\def\item{\def\writerlistparindent{\setlength\parindent{-0.25in}}\def\writerlistlabel{\stepcounter{listWWNumxxiilevelii}\makebox[0cm][l]{\labellistWWNumxxiilevelii}\hspace{-1.905cm}\writerlistremovelabel}}}{}
\newenvironment{listWWNumxxiileveliii}{\def\writerlistleftskip{\setlength\leftskip{1.5in}}\def\writerlistparindent{}\def\writerlistlabel{}\def\item{\def\writerlistparindent{\setlength\parindent{-0.1252in}}\def\writerlistlabel{\stepcounter{listWWNumxxiileveliii}\makebox[0cm][r]{\labellistWWNumxxiileveliii}\hspace{-3.4919918cm}\writerlistremovelabel}}}{}
\newenvironment{listWWNumxxiileveliv}{\def\writerlistleftskip{\setlength\leftskip{2in}}\def\writerlistparindent{}\def\writerlistlabel{}\def\item{\def\writerlistparindent{\setlength\parindent{-0.25in}}\def\writerlistlabel{\stepcounter{listWWNumxxiileveliv}\makebox[0cm][l]{\labellistWWNumxxiileveliv}\hspace{-4.4449997cm}\writerlistremovelabel}}}{}
\title{}
\author{Anna}
\date{2019-09-06}
\begin{document}
\title{\textsuperscript{Accusative/dative alternation of Catalan verbs with an experiencer object}}
\maketitle

\section{}
\section[Carles Royo]{\textbf{Carles Royo}}
\section[Universitat Rovira i Virgili]{\textbf{Universitat Rovira i Virgili}}
\begin{stylelsAbstract}
\textbf{Abstract.} Various Catalan psychological verbs that are part of causative sentences with an accusative experiencer (\textup{Els nens van molestar la Maria }or\textup{ La van molestar} ‘The kids annoyed Maria or They annoyed her’) alternate with stative sentences that change the sentence order and have a dative experiencer (\textup{A la Maria li molesta el teu caràcter }‘lit. To Maria your character is annoying’). Other psychological verbs, however, can form both types of sentence without changing the accusative morphology of the experiencer (\textup{Els nens van atabalar la Maria} or \textup{La van atabalar} ‘The kids overwhelmed Maria or They overwhelmed her’; \textup{A la Maria l’atabala el teu caràcter }‘lit. To Maria your character is overwhelming’). I argue that in stative sentences of all these verbs the experiencer is a real dative, regardless of its morphology (dative or accusative). Differential indirect object marking (DIOM) explains why accusative morphology is possible in these constructions.
\end{stylelsAbstract}


\setcounter{listWWNumxxiileveli}{0}
\begin{listWWNumxxiileveli}
\item 
\begin{stylelsSectioni}
Introduction
\end{stylelsSectioni}
\end{listWWNumxxiileveli}
\begin{styleStandard}
Since the first half of the 20\textsuperscript{th} century (cf. Ginebra 2003: 16; 2015: 147), some Catalan psychological verbs belonging to Belletti and Rizzi’s type II (1988) – which make sentences with an accusative experiencer or AcExp (1\textit{a})/(2\textit{a}) – have appeared with some frequency in both the written and spoken language with a change in sentence order and a dative experiencer (1\textit{b})/(2\textit{b}). This accusative/dative alternation has generated considerable academic debate. In most instances, the rules of the Institute of Catalan Studies (IEC) governing the Catalan language do not countenance this change in case marking, although the IEC’s new normative grammar (GIEC) and the changes introduced on 5 April 2017 to its online normative dictionary (DIEC2) accept the dative case marking – as well as the accusative – in some particular predicates: including the verbs \textit{encantar} ‘delight’, \textit{estranyar} ‘surprise’, \textit{molestar} ‘annoy’ and \textit{preocupar} ‘worry’.\footnote{ Before publication of the GIEC, the IEC accepted the intransitive nature of the verb \textit{interessar} ‘interest’ as well as an accusative case marking.}
\end{styleStandard}

\begin{styleStandard}
(1) a. Els nens van molestar \ \ \ \ \ \ \ la \ \ \ Maria \ \ \ \ \ \ \ \ (\textit{o} \ la \ \ \ \ \ \ \ \ \ \ \ \ \ \ \ \ \ \ van molestar).
\end{styleStandard}

\begin{styleStandard}
\ \ \ \ \ \ \ The kids annoyed.\textsc{aux+inf} \textsc{art} Maria.\textsc{acc} (\textit{or} her.\textsc{cl.acc}:3 annoyed.\textsc{aux+inf})
\end{styleStandard}

\begin{styleStandard}
\ \ \ \ \ \ b. A \ la \ \ \ Maria \ \ \ \ \ \ \ \ li \ \ \ \ \ \ \ \ \ molesten els nens.
\end{styleStandard}

\begin{styleStandard}
\ \ \ \ \ \ \ \ \ \ to \textsc{art }Maria.\textsc{dat} \textsc{cl.dat} annoy \ \ \ \ \ the kids
\end{styleStandard}

\begin{styleStandard}
(2) Cabré \& Mateu 1998: 77
\end{styleStandard}

\begin{styleStandard}
\ \ a. Les teves paraules la \ \ \ \ \ \ \ \ \ \ \ \ \ \ \ \ \ \ \ \ van \textit{sorprendre}, \ \ \ \ \ \ \ \textit{preocupar}, \textit{molestar} molt.
\end{styleStandard}

\begin{styleStandard}
\ \ \ \ \ \textsc{art} your \ words \ \ \ her.\textsc{cl.acc:3.f} surprised.\textsc{aux+inf} / worried \ \ / \ annoyed \ a lot
\end{styleStandard}

\begin{styleStandard}
\ \ b. Li \ \ \ \ \ \ \ \ \ \ \ \textit{sorprèn}, \ \ \ \textit{preocupa}, \textit{molesta} que la \ \ joventut d’ avui \ \ fumi \ \ \ tant.
\end{styleStandard}

\begin{styleStandard}
\textsc{\ \ \ \ \ \ \ cl.dat:3} surprises / worries \ / \ annoys \textsc{\ \ }that the youth \ \ \ \ of today smoke so much
\end{styleStandard}

\begin{styleStandard}
This change has not had a uniform impact on Catalan dialects. Moreover, notable differences often occur within each dialect and even in the use that a specific speaker makes of these predicates (cf. Cabré \& Mateu 1998: 70). Indeed, some predicates have become more entrenched than others, something that is irregularly reflected in several lexicographical collections in the Catalan language. It is common for AcExp verbs in Spanish to present this argument alternation (cf. Mendívil Giró 2005; Marín \& McNally 2011, among others). For this reason, psychological verbs that are used with dative constructions in Catalan, when they have traditionally been used with accusative constructions (AcExp), have often been regarded as syntactic calques of the Spanish; yet, some studies describe the change as being inherent to the Catalan language.
\end{styleStandard}

\begin{styleStandard}
This paper argues that in a stative sentence containing these verbs the experiencer is a real dative, not only when it presents the dative morphology, but also when it presents the accusative form (see also Cabré \& Fábregas and Ledgeway, Schifano \& Silvestri, this volume, about the different natures of datives). I also argue that the accusative morphology of such stative sentences is facilitated by a mechanism of differential indirect object marking (DIOM).
\end{styleStandard}

\begin{listWWNumxxiileveli}
\item 
\begin{stylelsSectioni}
Syntactico-semantic configuration of sentences with accusative and dative
\end{stylelsSectioni}
\end{listWWNumxxiileveli}
\begin{styleStandard}
Ynglès (1991) and Cabré \& Mateu (1998) point out that the syntactico-semantic configuration differs when some AcExp verbs are used with the accusative and when they are used with the dative: see the contrast in (3).\footnote{ For further information on the proof and examples that show that sentences such as that in (1\textit{a}){}-(2\textit{a}) are configured differently from those illustrated in (1\textit{b})/(2\textit{b}), see Royo (2017: §4.1).} In (1\textit{a}) and (2\textit{a}), three components of causative verbs imply a change of state: cause + process (change) + resulting state (cf. Levin \& Rappaport Hovav 1995; Cabré \& Mateu 1998; Rosselló 2008). The verb needs to be followed by an accusative in an eventive sentence of external causation and a neutral subject-verb-object (SVO) order. On the other hand, (1\textit{b}) and (2\textit{b}) do not have these three components, and the verb requires the dative in a stative sentence and a neutral object-verb-subject (OVS) order and clitic doubling (see also Fábregas \& Marín, this volume).
\end{styleStandard}

\begin{styleStandard}
(3) a. Els nens van molestar la Maria \textit{expressament} i \ \ els \ mestres també \textit{ho van fer}.
\end{styleStandard}

\begin{styleStandard}
\ \ \ \ \ \ \ \ \ \ \ \ \ \ \ \ \ \ \ \ \ \ \ \ \ \ \ \ \ \ \ \ \ \ \ \ \ \ \ \ \ \ \ \ \ \ \ \ \ \ \ \ \ \ \ \ \ \ \ \ on purpose \ \ and the teachers also \ \ \ it \ did.\textsc{aux+inf}
\end{styleStandard}

\begin{styleStandard}
\ \ \ \ \ \ b. *A la Maria li molesten els nens \textit{expressament} i \ \ \ els mestres \ també \textit{ho fan}.
\end{styleStandard}

\begin{styleStandard}
\ \ \ \ \ \ \ \ \ \ \ \ \ \ \ \ \ \ \ \ \ \ \ \ \ \ \ \ \ \ \ \ \ \ \ \ \ \ \ \ \ \ \ \ \ \ \ \ \ \ \ \ \ \ \ \ \ \ \ \ \ \ \ on purpose \ \ and the teachers also \ \ \ it \ \ do
\end{styleStandard}

\begin{styleStandard}
Two mechanisms help differentiate the causative structure in (1\textit{a})/(2\textit{a}) from the stative structure in (1\textit{b})/(2\textit{b}). On the one hand, their verbal aspect: the perfective aspect contributes to a causative interpretation while the imperfective aspect contributes to a stative interpretation; hence, there is a relation between the lexical aspect of the sentence (eventive or stative) and the verbal aspect of the predicate (perfective or imperfective). And, on the other, the sentence order: a neutral SVO order gives a causative interpretation and a neutral OVS order gives a stative interpretation.
\end{styleStandard}

\begin{styleStandard}
In line with Ynglès (1991), Cabré \& Mateu (1998), Rosselló (2008) and GIEC (§21.5\textit{b-c}) for Catalan, Pesetsky (1995) for English, Bouchard (1995) for French and Acedo-Matellán \& Mateu (2015) for Spanish, I consider that Catalan psychological verbs with an accusative experiencer (AcExp) generally cause a change of state:\footnote{ According to other authors, the characterization of these sentences is different or allows different structures: cf. Voorst (1992), Arad (1999), Landau (2010), Marín \& McNally (2011) and Fábregas (2015). Several authors, including Fábregas \& Marín (2012), Fábregas, Marín \& McNally (2012), Marín \& Sánchez Marco (2012), Ganeshan (2014) and Viñas-de-Puig (2014), study these constructions in their general analyses of the stative and eventive nature of Spanish sentences with psychological verbs (note Viñas-de-Puig do the same also with Catalan psychological verbs). Acedo-Matellán \& Mateu (2015: 83 (4)) also accept that these verbs cause a change of state in Spanish but point out that there is a less common construction of AcExp verbs with the accusative, that is, stative causative transitive (\textit{Este problema la ha preocupado desde siempre}).} in these sentences subjects are agents or inanimate causes and accusative experiencers are strictly speaking patients, even though conceptually they can be regarded as experiencers. I also concur with several authors who point out that the OVS stative construction of some AcExp Catalan verbs is the same as that of psychological verbs with a dative experiencer (DatExp, for example \textit{agradar} ‘like’; cf. Cabré \& Mateu 1998; Ramos 2004; Rosselló 2008; Cuervo 2010, among others): the subject is a stimulus or source of the psychological experience and the dative experiencer is not a patient, it does not undergo a change of state. What is more, clitic doubling occurs when the experiencer phrase appears in preverbal position.\footnote{ Acedo-Matellán \& Mateu (2015) have questioned this assumption in psychological verbs in Spanish and draw a distinction between DatExp verbs (unaccusative statives) and AcExp verbs that are constructed with the dative (unergative statives). For a discussion of this issue, see Royo (2017: §6.2.4.1).}
\end{styleStandard}

\begin{styleStandard}
These data suggest that many speakers need to change both the syntactical pattern of AcExp verbs and the sentence order when they use these verbs in a stative construction: the different semantic or lexical-aspectual interpretation of these sentences is reflected in the different syntactic configuration of constructions that contain Catalan AcExp verbs.\footnote{ Several authors claim that the change between causative and stative interpretation implies a change in the Spanish case marking, between accusative and dative respectively: cf. Fábregas (2015), Viñas-de-Puig (2017) and Ganeshan (2019).} According to Ginebra (2003: 14; 29-30), however, the examples in (4) show that Catalan can also denote a stative OVS construction without changing from the accusative to the dative with some predicates. These can be AcExp verbs (4\textit{a}) or non-psychological causative verbs that become psychological by means of a metaphorical expansion of the meaning (4\textit{b}) (the \textit{psych} \textit{constructions} described by Bouchard 1995). Therefore, the lexical nature of the verb plays an important role in the alternation since some verbs tend not to construct stative sentences with the dative.
\end{styleStandard}

\begin{styleStandard}
(4) Ginebra 2003: 29-30
\end{styleStandard}

\begin{styleStandard}
\ \ \ \ \ \ a. Al \ \ \ \ \ \ \ \ seu germà \ l’ \ \ \ \ \ \ \ \ \ \ \ \ \ \ \ \ atabala \ \ \ \ \ \ \ \ la \ \ nova responsabilitat.
\end{styleStandard}

\begin{styleStandard}
\ \ \ \ \ \ \ \ \ \ to+\textsc{art} his brother \textsc{cl.acc:3.m} overwhelms the new \ responsibility
\end{styleStandard}

\begin{styleStandard}
\ \ \ \ \ \ b. Al \ \ \ \ \ \ \ \ Xavier el \ \ \ \ \ \ \ \ \ \ \ \ \ \ \ \ destrossa aquesta tensió \ \ contínua.
\end{styleStandard}

\begin{styleStandard}
\ \ \ \ \ \ \ \ \ \ to+\textsc{art} Xavier \textsc{cl.acc:3.m} destroys \ \ this \ \ \ \ \ \ tension constant.\textsc{adj}
\end{styleStandard}

\begin{styleStandard}
What is more, with AcExp verbs such as those identified by Cabré \& Mateu (1998) – \textit{molestar}, \textit{preocupar}, \textit{sorprendre} (see (2)) – speakers may hesitate between accusative and dative case marking in OVS stative sentences. Some examples of this hesitation in a Catalan/Spanish bilingual newspaper are shown in (5). The print edition of the paper includes an OVS sentence with the verb \textit{preocupar} ‘worry’ that governs the accusative in Catalan (5\textit{a}) and the dative in Spanish (5\textit{b}); on the other hand, in the Catalan online edition the same sentence appears with a dative (5\textit{c}). Examples (6) and (7) show the same hesitation with the verb \textit{molestar} ‘annoy’, in the same news item reported by six media in Catalan on 5 December 2012: three use the accusative (6) and three the dative (7).\footnote{ The three sentences in the accusative use direct speech while the three in the dative use indirect speech, which may indicate that the person making the statement conceptualizes the verb differently from the journalists who report it.}
\end{styleStandard}

\begin{styleStandard}
(5) \textit{La Vanguardia}, 15 May 2015, p. 15 (headline)
\end{styleStandard}

\begin{styleStandard}
\ \ \ \ \ \ a. Per què a \ Ciu la \ \ \ \ \ \ \ \ \ \ \ \ \ \ \ preocupa Ciutadans\ \ [Catalan, printed version]
\end{styleStandard}

\begin{styleStandard}
\ \ \ \ \ \ \ \ \ \ why \ \ \ \ \ to Ciu \textsc{cl.acc:3.f} worries \ \ \ Ciutadans
\end{styleStandard}

\begin{styleStandard}
\ \ \ \ \ \ b. Por qué a \ Ciu le \ \ \ \ \ \ \ \ \ \ \ preocupa Ciutadans\ \ [Spanish, printed version]
\end{styleStandard}

\begin{styleStandard}
\textsc{\ \ \ \ \ \ \ \ \ \ \ \ \ \ \ \ }\ \ \ \ \ \ \ \ \ \ \ \ \ \ \ \ \ \ \ \ \ \textsc{cl.dat:3}
\end{styleStandard}

\begin{styleStandard}
\ \ \ \ \ \ c. Per què a \ Ciu li \ \ \ \ \ \ \ \ \ \ \ \ preocupa Ciutadans\ \ [Catalan, online version]
\end{styleStandard}

\begin{styleStandard}
\textsc{\ \ \ \ \ \ \ \ \ \ \ \ \ \ \ }\ \ \ \ \ \ \ \ \ \ \ \ \ \ \ \ \ \ \ \ \ \textsc{cl.dat:3}
\end{styleStandard}

\begin{styleStandard}
(6) a. VilaWeb (headline)
\end{styleStandard}

\begin{styleStandard}
\ \ \ \ \ \ \ \ \ \ Rigau: ‘A Wert el \ \ \ \ \ \ \ \ \ \ \ \ \ \ \ \ molesta l’ \ \ èxit \ \ \ \ \ \ del \ \ \ \ model \ d’ immersió’
\end{styleStandard}

\begin{styleStandard}
\ \ \ \ \ \ \ \ \ \ Rigau \ \ to Wert \textsc{cl.acc:3.m} annoys \ the success of-the model of immersion
\end{styleStandard}

\begin{styleStandard}
\ \ \ \ \ \ b. \textit{El Periódico de Catalunya} (headline)
\end{styleStandard}

\begin{styleStandard}
\ \ \ \ \ \ \ \ \ \ Rigau: “A Wert el \ \ \ \ \ \ \ \ \ \ \ \ \ \ \ molesta l’èxit de la \ \ immersió \ \ lingüística”
\end{styleStandard}

\begin{styleStandard}
\ \ \ \ \ \ \ \ \ \ \ \ \ \ \ \ \ \ \ \ \ \ \ \ \ \ \ \ \ \ \ \ \ \ \ \ \textsc{cl.acc:3.m} \ \ \ \ \ \ \ \ \ \ \ \ \ \ \ \ \ \ \ \ \ \ \ of the immersion language.\textsc{adj}
\end{styleStandard}

\begin{styleStandard}
\ \ \ \ \ \ c. \textit{Ara} (headline)
\end{styleStandard}

\begin{styleStandard}
\ \ \ \ \ \ \ \ \ \ Rigau: “A Wert, el que el \ \ \ \ \ \ \ \ \ \ \ \ \ \ \ \ molesta és l’èxit del model educatiu \ \ \ \ \ \ \ \ \ \ \ \ català”
\end{styleStandard}

\begin{styleStandard}
\ \ \ \ \ \ \ \ \ \ \ \ \ \ \ \ \ \ \ \ \ \ \ \ \ \ \ \ \ \ \ \ \ \ \ \ \ \ \ \ what \ \ \textsc{cl.acc:3.m} \ \ \ \ \ \ \ \ \ \ \ \ \ \ \ \ \ \ \ \ \ \ \ \ \ \ \ \ \ \ \ \ \ model educational.\textsc{adj} Catalan.\textsc{adj}
\end{styleStandard}

\begin{styleStandard}
(7) a. 3/24, www.ccma.cat (headline)
\end{styleStandard}

\begin{styleStandard}
\ \ \ \ \ \ \ \ \ \ \ Rigau creu \ \ \ \ \ \ que a \ \ Wert li \ \ \ \ \ \ \ \ \ \ \ \ molesta “l’ \ \ èxit” \ \ \ \ del \ \ \ \ \ model català
\end{styleStandard}

\begin{styleStandard}
\ \ \ \ \ \ \ \ \ \ \ Rigau believes that to Wert \textsc{cl.dat:3} annoys \ \ the success of-the model Catalan.\textsc{adj}
\end{styleStandard}

\begin{styleStandard}
\ \ \ \ \ \ b. diaridegirona.cat (headline)
\end{styleStandard}

\begin{styleStandard}
\ \ \ \ \ \ \ \ \ \ Rigau creu que a Wert li \ \ \ \ \ \ \ \ \ \ \ \ molesta “l’èxit” del model català
\end{styleStandard}

\begin{styleStandard}
\ \ \ \ \ \ \ \ \ \ \ \ \ \ \ \ \ \ \ \ \ \ \ \ \ \ \ \ \ \ \ \ \ \ \ \ \ \ \ \ \ \ \ \ \ \ \ \textsc{cl.dat:3}
\end{styleStandard}

\begin{styleStandard}
\ \ \ \ \ \ c. \textit{El Punt Avui}
\end{styleStandard}

\begin{styleStandard}
\ \ \ \ \ \ \ \ \ \ \ \ \ La \ titular \ \ \ d’ Ensenyament, creu que a Wert li \ \ \ \ \ \ \ \ \ \ \ “molesta” el \ model “d’ èxit” \ \ \ de l’ \ \ escola \ catalana.
\end{styleStandard}

\begin{styleStandard}
\ \ \ \ \ \ \ \ \ \ \ \ \ the minister of Education \ \ \ \ \ \ \ \ \ \ \ \ \ \ \ \ \ \ \ \ \ \ \ \ \ \ \ \ \ \ \ \ \textsc{cl.dat:3 }\ \ \ \ \ \ \ \ \ \ \ \ \ \ \ \ the model of success of the school Catalan.\textsc{adj}
\end{styleStandard}

\begin{styleStandard}
In fact, if in (1\textit{b}) and (2\textit{b}) we replace the dative clitic with the accusative clitic – \textit{A la Maria la molesten els nens};\textit{ (A ella)} \textit{La sorprèn, preocupa, molesta que la joventut d’avui fumi tant }– our discussion above about distinguishing these sentences from those in (1\textit{a}) and (2\textit{a}) is still valid: they are useful ways of characterizing both constructions differently, but they do not help determine the case marking.
\end{styleStandard}

\begin{styleStandard}
The ability of Catalan to construct a stative sentence with an AcExp verb and an accusative experiencer makes it necessary to analyse this accusative in those cases of hesitation with the dative (that is, in OVS stative sentences). We need to know whether the order of the sentences and clitic doubling in Catalan are sufficient to denote a lexical-aspectual change in the sentence or whether a change in case marking is also required.
\end{styleStandard}

\section{}
\begin{listWWNumxxiileveli}
\item 
\begin{stylelsSectioni}
Nature of the accusative and dative experiencer in OVS stative sentences
\end{stylelsSectioni}
\end{listWWNumxxiileveli}
\begin{styleStandard}
In the sentences in (1\textit{b})/(2\textit{b}) and (4)-(7), whether the verb governs the accusative or the dative, the subject is a stimulus of the emotion and the object is not a patient but an experiencer of the whole event in a more prominent structural position than that occupied by the stimulus. It can be shown that this experiencer argument, regardless of whether it is accusative or dative, is not a topicalized element and that it has properties of a subject: cf. examples \textit{a} and \textit{b} in (8)-(13). It behaves just like the experiencer in sentences with DatExp verbs such as \textit{agradar} ‘like’ (see the \textit{c} examples in (8)-(13)) and other canonical subjects (see the \textit{d} examples in (8) and (12) and example (10\textit{e})): it behaves quite differently from topicalized objects (see the \textit{d} examples in (9)-(11) and (13)).\footnote{ In examples (8){}-(13), as in the other examples employed in this paper, I conduct a descriptive rather than a prescriptive assessment.}
\end{styleStandard}

\begin{styleStandard}
The experiencer can link an anaphora in the subject (cf. Demonte 1989; Eguren \& Fernández Soriano 2004) (8), be modified by the adverb \textit{només} ‘only’ (cf. Cuervo 1999) (9), allow \textit{Wh-}extraction (cf. Belletti \& Rizzi 1988) (10), be an indefinite generalized quantifier in initial position (cf. Belletti \& Rizzi 1988; Masullo 1992; Cuervo 1999) (11), control the subject of an infinitive clause (cf. Campos 1999; Alsina 2008) (12) and it cannot be separated, in Catalan, by a comma from the rest of the sentence (cf. Ginebra 2003; 2005) (13).
\end{styleStandard}

\begin{styleStandard}
(8) a. OVS AcExp, \textbf{dative/accusative}
\end{styleStandard}

\begin{styleStandard}
\ \ \ \ \ \ \ \ \ \ A \ l’ \ \ \ Albert\textsubscript{i} \{\textbf{li}\textbf{\textsubscript{i}} \ \ \ \ \ \ \ \ \ \ \ / \textbf{el}\textbf{\textsubscript{i}}\} \ \ \ \ \ \ \ \ \ \ \ \ \ molesta aquesta fotografia de si mateix\textsubscript{i}.
\end{styleStandard}

\begin{styleStandard}
\ \ \ \ \ \ \ \ \ \ to \textsc{art} Albert \ \textbf{\textsc{cl.dat:3}} / \textbf{\textsc{cl.acc:3.m}} annoys this \ \ \ \ \ \ photo \ \ \ \ \ \ \ of himself
\end{styleStandard}

\begin{styleStandard}
\ \ \ \ \ \ b. OVS AcExp, \textbf{accusative}
\end{styleStandard}

\begin{styleStandard}
\ \ \ \ \ \ \ \ \ \ A l’Albert\textsubscript{i} \textbf{el}\textbf{\textsubscript{i}} \ \ \ \ \ \ \ \ \ \ \ \ \ \ \ \ neguiteja aquesta fotografia de si mateix\textsubscript{i}.
\end{styleStandard}

\begin{styleStandard}
\ \ \ \ \ \ \ \ \ \ \ \ \ \ \ \ \ \ \ \ \ \ \ \ \ \ \ \ \textbf{\textsc{cl.acc:3.m}} disturbs
\end{styleStandard}

\begin{styleStandard}
\ \ \ \ \ \ c. DatExp, \textbf{dative}
\end{styleStandard}

\begin{styleStandard}
\ \ \ \ \ \ \ \ \ \ A l’ \ \ \ \ Albert\textsubscript{i} \textbf{li}\textbf{\textsubscript{i}} \ \ \ \ \ \ \ \ \ \ \ \ \ agrada aquesta fotografia de si mateix\textsubscript{i}.
\end{styleStandard}

\begin{styleStandard}
\ \ \ \ \ \ \ \ \ \ \ \ \ \ \ \ \ \ \ \ \ \ \ \ \ \ \ \ \ \ \ \ \ \textbf{\textsc{cl.dat:3}} likes
\end{styleStandard}

\begin{styleStandard}
\ \ \ \ \ \ d. Subject, dative
\end{styleStandard}

\begin{styleStandard}
\ \ \ \ \ \ \ \ \ \ L’ \ \ Albert\textsubscript{i} envia una fotografia de si mateix\textsubscript{i/*j} a \ la \ \ \ \ Núria\textsubscript{j}.
\end{styleStandard}

\begin{styleStandard}
\ \ \ \ \ \ \ \ \ \ \textsc{art} Albert sends a \ \ \ \ photo \ \ \ \ \ \ \ of himself \ \ \ \ \ \ to \textsc{art} Nuria
\end{styleStandard}

\begin{styleStandard}
(9) a. OVS AcExp, \textbf{dative/accusati}\textbf{ve}
\end{styleStandard}

\begin{styleStandard}
\ \ \ \ \ \ \ \ \ \ Només a \ l’ \ \ \ \ Albert \{\textbf{li} \ \ \ \ \ \ \ \ \ \ \ \ \ / \textbf{el}\} \ \ \ \ \ \ \ \ \ \ \ \ \ \ molesta aquesta situació. 
\end{styleStandard}

\begin{styleStandard}
\ \ \ \ \ \ \ \ \ \ Only \ \ \ to \textsc{art} Albert \ \textbf{\textsc{cl.dat:3}} / \textbf{\textsc{cl.acc:3.m}} annoys \ this \ \ \ \ \ \ situation
\end{styleStandard}

\begin{styleStandard}
\ \ \ \ \ \ b. OVS AcExp, \textbf{accusative}
\end{styleStandard}

\begin{styleStandard}
\ \ \ \ \ \ \ \ \ \ Només a \ l’Albert \textbf{el} \ \ \ \ \ \ \ \ \ \ \ \ \ \ \ \ \ neguiteja aquesta situació.
\end{styleStandard}

\begin{styleStandard}
\ \ \ \ \ \ \ \ \ \ \ \ \ \ \ \ \ \ \ \ \ \ \ \ \ \ \ \ \ \ \ \ \ \ \ \ \ \ \ \textbf{\textsc{cl.acc:3.m}} disturbs
\end{styleStandard}

\begin{styleStandard}
\ \ \ \ \ \ c. DatExp, \textbf{dative}
\end{styleStandard}

\begin{styleStandard}
\ \ \ \ \ \ \ \ \ \ Només a l’Albert \textbf{li} \ \ \ \ \ \ \ \ \ \ \ \ \ agrada la \ \ cervesa.
\end{styleStandard}

\begin{styleStandard}
\ \ \ \ \ \ \ \ \ \ \ \ \ \ \ \ \ \ \ \ \ \ \ \ \ \ \ \ \ \ \ \ \ \ \ \ \ \ \textbf{\textsc{cl.dat:3}} likes \ \ \ \ the beer
\end{styleStandard}

\begin{styleStandard}
\ \ \ \ \ \ d. Topicalized dative
\end{styleStandard}

\begin{styleStandard}
\ \ \ \ \ \ \ \ \ \ \textsuperscript{?}Només a l’Albert \textbf{li} \ \ \ \ \ \ \ \ \ \ \ \ \ Ø vaig \ prendre \ el \ \ bolígraf.\footnote{ This sentence is acceptable with a stressed intonation: \textit{Només A L’ALBERT}…}
\end{styleStandard}

\begin{styleStandard}
\ \ \ \ \ \ \ \ \ \ \ \ \ \ \ \ \ \ \ \ \ \ \ \ \ \ \ \ \ \ \ \ \ \ \ \ \ \ \ \textbf{\textsc{cl.dat:3}} \ I \ took.\textsc{aux+inf} the pen
\end{styleStandard}

\begin{styleStandard}
(10) a. OVS AcExp, \textbf{dative/accusati}\textbf{ve}
\end{styleStandard}

\begin{styleStandard}
\ \ \ \ \ \ \ \ \ \ \ La \ situació \ que \ a \ l’ \ \ \ Albert \{\textbf{li} \ \ \ \ \ \ \ \ \ \ \ \ \ / \textbf{el}\} \ \ \ \ \ \ \ \ \ \ \ \ \ \ molesta és aquesta.
\end{styleStandard}

\begin{styleStandard}
\ \ \ \ \ \ \ \ \ \ \ the situation that to \textsc{art} Albert\textsc{ \ }\textbf{\textsc{cl.dat:3}} / \textbf{\textsc{cl.acc:3.m}} annoys \ is this
\end{styleStandard}

\begin{styleStandard}
\ \ \ \ \ \ \ b. OVS AcExp, \textbf{accusative}
\end{styleStandard}

\begin{styleStandard}
\ \ \ \ \ \ \ \ \ \ \ La \ situació \ que a \ l’Albert (\textbf{el}) \ \ \ \ \ \ \ \ \ \ \ \ \ \ \ \ neguiteja és aquesta.
\end{styleStandard}

\begin{styleStandard}
\ \ \ \ \ \ \ \ \ \ \ \ \ \ \ \ \ \ \ \ \ \ \ \ \ \ \ \ \ \ \ \ \ \ \ \ \ \ \ \ \ \ \ \ \ \ \ \ \ \ \ \ \ \ \ \ \ \textbf{\textsc{cl.acc}}\textbf{:3.}\textbf{\textsc{m}} disturbs
\end{styleStandard}

\begin{styleStandard}
\ \ \ \ \ \ \ c. DatExp, \textbf{dative}
\end{styleStandard}

\begin{styleStandard}
\ \ \ \ \ \ \ \ \ \ \ Els llibres que a \ l’ \ \ \ \ Albert (\textbf{li}) \ \ \ \ \ \ \ \ \ \ \ \ han agradat \ \ \ \ \ \ \ són aquests.
\end{styleStandard}

\begin{styleStandard}
\ \ \ \ \ \ \ \ \ \ \ the books that to \textsc{art} Albert\textsc{ \ }\textbf{\textsc{cl.dat:3}} liked.\textsc{aux+part} are \ these
\end{styleStandard}

\begin{styleStandard}
\ \ \ \ \ \ \ d. Topicalized dative
\end{styleStandard}

\begin{styleStandard}
\ \ \ \ \ \ \ \ \ \ \ \textsuperscript{??}Els llibres que a \ l’Albert (\textbf{li}) \ \ \ \ \ \ \ \ \ \ \ \ Ø he donat \ \ \ \ \ \ \ \ \ \ \ són aquests.
\end{styleStandard}

\begin{styleStandard}
\ \ \ \ \ \ \ \ \ \ \ \ \ \ \ \ \ \ \ \ \ \ \ \ \ \ \ \ \ \ \ \ \ \ \ \ \ \ \ \ \ \ \ \ \ \ \ \ \ \ \ \ \ \ \ \ \textbf{\textsc{cl.dat:3}} \ I \ gave.\textsc{aux+part}
\end{styleStandard}

\begin{styleStandard}
\ \ \ \ \ \ \ e. Preverbal subject
\end{styleStandard}

\begin{styleStandard}
\ \ \ \ \ \ \ \ \ \ \ Els llibres que l’Albert m’ \ \ \ \ \ \ \ \ \ \ ha donat \ \ \ \ \ \ \ \ \ \ \ són aquests.
\end{styleStandard}

\begin{styleStandard}
\ \ \ \ \ \ \ \ \ \ \ \ \ \ \ \ \ \ \ \ \ \ \ \ \ \ \ \ \ \ \ \ \ \ \ \ \ \ \ \ \ \ \ \ \ \ \ \ \textsc{cl.dat:1} gave.\textsc{aux+part}
\end{styleStandard}

\begin{styleStandard}
(11) a. OVS AcExp, \textbf{dative/accusati}\textbf{ve}
\end{styleStandard}

\begin{styleStandard}
\ \ \ \ \ \ \ \ \ \ \ A ningú \ \ \ \ (no) \{\textbf{li} \ \ \ \ \ \ \ \ \ \ \ \ \ / \textbf{el}\} \ \ \ \ \ \ \ \ \ \ \ \ \ \ \ molesta aquesta situació.
\end{styleStandard}

\begin{styleStandard}
\ \ \ \ \ \ \ \ \ \ \ to nobody (\textsc{neg}) \textbf{\textsc{cl.dat:3}} / \textbf{\textsc{cl.acc:3.m}} annoys \ this \ \ \ \ \ \ situation
\end{styleStandard}

\begin{styleStandard}
\ \ \ \ \ \ \ b. OVS AcExp, \textbf{accusative}
\end{styleStandard}

\begin{styleStandard}
\ \ \ \ \ \ \ \ \ \ \ A ningú (no) (\textbf{el}) \ \ \ \ \ \ \ \ \ \ \ \ \ \ \ neguiteja aquesta situació.
\end{styleStandard}

\begin{styleStandard}
\ \ \ \ \ \ \ \ \ \ \ \ \ \ \ \ \ \ \ \ \ \ \ \ \ \ \ \ \ \ \ \ \ \textbf{\textsc{cl.acc:3.m}}\textbf{ }disturbs
\end{styleStandard}

\begin{styleStandard}
\ \ \ \ \ \ \ c. DatExp, \textbf{dative}
\end{styleStandard}

\begin{styleStandard}
\ \ \ \ \ \ \ \ \ \ \ A ningú (no) \textbf{li} \ \ \ \ \ \ \ \ \ \ \ \ \ va agradar \ \ \ \ \ \ \ la \ \ pel·lícula.
\end{styleStandard}

\begin{styleStandard}
\ \ \ \ \ \ \ \ \ \ \ \ \ \ \ \ \ \ \ \ \ \ \ \ \ \ \ \ \ \ \ \ \textbf{\textsc{cl.dat:3}} liked.\textsc{aux+inf} the film\ \ 
\end{styleStandard}

\begin{styleStandard}
\ \ \ \ \ \ \ d. Topicalized dative
\end{styleStandard}

\begin{styleStandard}
\ \ \ \ \ \ \ \ \ \ \ *A ningú (no) \textbf{li} \ \ \ \ \ \ \ \ \ \ \ \ \ vaig donar \ \ \ \ \ \ el \ \ quadre.
\end{styleStandard}

\begin{styleStandard}
\ \ \ \ \ \ \ \ \ \ \ \ \ \ \ \ \ \ \ \ \ \ \ \ \ \ \ \ \ \ \ \ \ \ \textbf{\textsc{cl.dat:3}} gave.\textsc{aux+inf} the painting
\end{styleStandard}

\begin{styleStandard}
(12) a. OVS AcExp, \textbf{dative/accusati}\textbf{ve}
\end{styleStandard}

\begin{styleStandard}
\ \ \ \ \ \ \ \ \ \ \ A l’ \ \ \ \ Albert\textsubscript{i} \{\textbf{li}\textbf{\textsubscript{i}} \ \ \ \ \ \ \ \ \ \ \ \ / \textbf{el}\textbf{\textsubscript{i}}\} \ \ \ \ \ \ \ \ \ \ \ \ \ \ molesta PRO\textsubscript{i} parlar \ \ \ \ \ \ en públic.
\end{styleStandard}

\begin{styleStandard}
\ \ \ \ \ \ \ \ \ \ \ to \textsc{art} Albert \ \ \textbf{\textsc{cl.dat:3}} / \textbf{\textsc{cl.acc:3.m}} annoys \ \ \ \ \ \ \ \ \ \ speak.\textsc{inf} in public
\end{styleStandard}

\begin{styleStandard}
\ \ \ \ \ \ \ b. OVS AcExp, \textbf{accusative}
\end{styleStandard}

\begin{styleStandard}
\ \ \ \ \ \ \ \ \ \ \ A l’Albert\textsubscript{i} \textbf{el}\textbf{\textsubscript{i}} \ \ \ \ \ \ \ \ \ \ \ \ \ \ \ \ neguiteja PRO\textsubscript{i} parlar en públic.
\end{styleStandard}

\begin{styleStandard}
\ \ \ \ \ \ \ \ \ \ \ \ \ \ \ \ \ \ \ \ \ \ \ \ \ \ \ \ \ \textbf{\textsc{cl.acc:3.m}} disturbs
\end{styleStandard}

\begin{styleStandard}
\ \ \ \ \ \ \ c. DatExp, \textbf{dative}
\end{styleStandard}

\begin{styleStandard}
\ \ \ \ \ \ \ \ \ \ \ A l’Albert\textsubscript{i} \textbf{li}\textbf{\textsubscript{i}} \ \ \ \ \ \ \ \ \ \ \ \ agrada PRO\textsubscript{i} parlar en públic.
\end{styleStandard}

\begin{styleStandard}
\ \ \ \ \ \ \ \ \ \ \ \ \ \ \ \ \ \ \ \ \ \ \ \ \ \ \ \ \textbf{\textsc{cl.dat:3}} likes
\end{styleStandard}

\begin{styleStandard}
\ \ \ \ \ \ \ d. Subject
\end{styleStandard}

\begin{styleStandard}
\ \ \ \ \ \ \ \ \ \ \ L’Albert\textsubscript{i} vol \ \ \ \ PRO\textsubscript{i} arribar \ \ \ \ \ aviat.
\end{styleStandard}

\begin{styleStandard}
\ \ \ \ \ \ \ \ \ \ \ \ \ \textsc{\ \ \ \ \ \ \ \ \ \ \ \ \ \ \ \ \ }wants \ \ \ \ \ \ \ \ \ arrive.\textsc{inf} early
\end{styleStandard}

\begin{styleStandard}
(13) a. OVS AcExp, \textbf{dative/accusati}\textbf{ve}
\end{styleStandard}

\begin{styleStandard}
\ \ \ \ \ \ \ \ \ \ \ A l’ \ \ \ \ Albert\textsubscript{(*}, \textsubscript{/ Ø)} \{\textbf{li} \ \ \ \ \ \ \ \ \ \ \ \ \ / \textbf{el}\} \ \ \ \ \ \ \ \ \ \ \ \ \ \ \ molesta aquesta situació.
\end{styleStandard}

\begin{styleStandard}
\ \ \ \ \ \ \ \ \ \ \ to \textsc{art} Albert \ \ \ \ \ \ \ \ \ \ \textbf{\textsc{cl.dat:3}} / \textbf{\textsc{cl.acc:3.m}} annoys \ this \ \ \ \ \ \ situation
\end{styleStandard}

\begin{styleStandard}
\ \ \ \ \ \ \ b. OVS AcExp, \textbf{accusative}
\end{styleStandard}

\begin{styleStandard}
\ \ \ \ \ \ \ \ \ \ \ A l’Albert\textsubscript{(*}, \textsubscript{/ Ø)} \textbf{el} \ \ \ \ \ \ \ \ \ \ \ \ \ \ \ \ \ neguiteja aquesta situació.
\end{styleStandard}

\begin{styleStandard}
\ \ \ \ \ \ \ \ \ \ \ \ \ \ \ \ \ \ \ \ \ \ \ \ \ \ \ \ \ \ \ \ \ \ \ \ \ \textbf{\textsc{cl.acc:3.m}} disturbs
\end{styleStandard}

\begin{styleStandard}
\ \ \ \ \ \ \ c. DatExp, \textbf{dative}
\end{styleStandard}

\begin{styleStandard}
\ \ \ \ \ \ \ \ \ \ \ A l’Albert\textsubscript{(*}, \textsubscript{/ Ø)} \textbf{li} \ \ \ \ \ \ \ \ \ \ \ \ \ agrada aquesta situació.
\end{styleStandard}

\begin{styleStandard}
\ \ \ \ \ \ \ \ \ \ \ \ \ \ \ \ \ \ \ \ \ \ \ \ \ \ \ \ \ \ \ \ \ \ \ \ \textbf{\textsc{cl.dat:3}} likes
\end{styleStandard}

\begin{styleStandard}
\ \ \ \ \ \ \ d. Topicalized object
\end{styleStandard}

\begin{styleStandard}
\ \ \ \ \ \ \ \ \ \ \ (A) \ L’Albert\textsubscript{(},\textsubscript{)} \textbf{l’} \ \ \ \ \ \ \ \ \ \ \ \ \ \ \ \ Ø he vist \ \ \ \ \ \ \ \ \ \ \ \ \ que \ Ø \ plorava.
\end{styleStandard}

\begin{styleStandard}
\ \ \ \ \ \ \ \ \ \ \ \ \ \ \ \ \ \ \ \ \ \ \ \ \ \ \ \ \ \ \ \ \ \ \ \ \textbf{\textsc{cl.acc:3.m}} I saw.\textsc{aux+part} that he cried
\end{styleStandard}

\begin{listWWNumxxiileveli}
\item 
\begin{stylelsSectioni}
OVS sentences with AcExp verbs and an accusative experiencer
\end{stylelsSectioni}
\end{listWWNumxxiileveli}
\begin{styleStandard}
The analysis conducted in section §3 highlights the similarity between the dative experiencer in sentences with DatExp verbs and the experiencer object in OVS stative sentences with AcExp verbs, whether the morphology is dative or accusative. When the experiencer has accusative morphology, there is evidence to show that it is in fact a dative if we place it in sentence-initial position by using a relative pronoun (14\textit{a-b}) (adjectival relative clause and noun relative clause),\footnote{ In the examples, I do not consider the use of the relative often referred to as the \textit{relatiu popular} (cf. Ginebra 2005: §154-155), which is always marked with an asterisk.} an interrogative pronoun (14\textit{c-d}) (direct and indirect interrogative) or a determiner phrase (14\textit{e}). In this context, the experiencer can optionally take either the accusative or dative morphology in the corresponding agentive sentences with AcExp verbs (16), which is similar to how the person semantic object behaves in transitive sentences of non-psychological verbs, whether they are causative or not (17). But in stative sentences with AcExp verbs (14), the experiencer in initial position behaves like the dative experiencer in the corresponding sentences with DatExp verbs (15): it can only be dative, even though in (14) the morphology is still accusative clitic within the sentence (cf. Royo 2017: §4.3.4).
\end{styleStandard}

\begin{styleStandard}
To illustrate this contrast, the examples below are of stative sentences with an imperfective verbal aspect (14)-(15) and causatives and non-causative transitives with a perfective aspect (16)-(17). What is more, in (14) and (16) I use an AcExp verb that can easily be conceived as causative of change of state, such as \textit{atabalar} ‘overwhelm’, unlike other AcExp verbs such as \textit{molestar} ‘annoy’, which in some contexts can have the meaning of ‘desagradar molt’ (‘displease a lot’).
\end{styleStandard}

\begin{styleStandard}
(14) a. Ø \ \ \ \ \ \ \ \ És una persona \{\textbf{a \ qui} \ \ \ \ \ \ \ \ \ \ \ \ / *que\} \ \ \ \ \ \ (\textbf{l’}) \ \ \ \ \ \ \ \ \ atabala \ \ \ \ \ \ \ \ el \ \ record \ \ \ d’ aquell fracàs.
\end{styleStandard}

\begin{styleStandard}
\ \ \ \ \ \ \ \ \ \ \ \ \ He/She is \ a \ \ \ \ person \ \ \ \ to whom.\textsc{dat} / who.\textsc{acc} (\textsc{cl.acc}) overwhelms the memory of that \ \ \ \ failure
\end{styleStandard}

\begin{styleStandard}
\ \ \ \ \ \ \ \ \ b. \{\textbf{A qui \ \ \ \ \ \ \ \ \ }\ \ \ \ / *Qui\} \ \ \ \ \ \ (\textbf{l’}) \ \ \ \ \ \ \ \ atabala el \ record d’aquell fracàs és \{\textbf{a }/ *Ø\} la \ \ Maria.
\end{styleStandard}

\begin{styleStandard}
\ \ \ \ \ \ \ \ \ \ \ \ \ \ to whom.\textsc{dat} / \ \ who.\textsc{acc} (\textsc{cl.acc}) \ \ \ \ \ \ \ \ \ \ \ \ \ \ \ \ \ \ \ \ \ \ \ \ \ \ \ \ \ \ \ \ \ \ \ \ \ \ \ \ \ \ \ \ \ \ \ \ \ \ \ \ \ is\ \  \ to / \ Ø \ \textsc{art} Maria.\textsc{dat}
\end{styleStandard}

\begin{styleStandard}
\ \ \ \ \ \ \ c. \{\textbf{A qui} \ \ \ \ \ \ \ \ \ \ \ \ / *Qui\} \ \ \ \ \ \ \ (\textbf{l’}) \ \ \ \ \ \ \ \ atabala el record d’aquell fracàs?
\end{styleStandard}

\begin{styleStandard}
\ \ \ \ \ \ \ \ \ \ \ \ to whom.\textsc{dat} / \ \ who.\textsc{acc} (\textsc{cl.acc})
\end{styleStandard}

\begin{styleStandard}
\ \ \ \ \ \ \ d. Ø Voldria \ \ \ \ \ saber \ \ \ \ \ \{\textbf{a \ qui} \ \ \ \ \ \ \ \ \ \ \ \ / *qui\} \ \ \ \ \ \ \ \ (\textbf{l’}) \ \ \ \ \ \ \ \ atabala el record d’aquell fracàs.
\end{styleStandard}

\begin{styleStandard}
\ \ \ \ \ \ \ \ \ \ \ \ \ I \ would like to know \ \ to whom.\textsc{dat} / \ \ who.\textsc{acc} (\textsc{cl.acc})
\end{styleStandard}

\begin{styleStandard}
\ \ \ \ \ \ \ e. \{\textbf{A} / *Ø\} la \ \ \ Maria\textsubscript{(*},\textsubscript{)} \ \ \ \ \textbf{l’ \ \ \ \ \ \ \ \ \ }atabala el record d’aquell fracàs.
\end{styleStandard}

\begin{styleStandard}
\ \ \ \ \ \ \ \ \ \ \ \ to / \ \ Ø \ \ \textsc{art} Maria.\textsc{dat} \textsc{cl.acc}
\end{styleStandard}

\begin{styleStandard}
(15) a. Ø \ \ \ \ \ \ \ \ \ És una persona \{a \ qui \ \ \ \ \ \ \ \ \ \ \ \ / *que\} \ \ \ \ \ \ \ no \ \ \ \ \ \ \ (li) \ \ \ \ \ \ \ \ \ \ agrada el record d’aquell fracàs.
\end{styleStandard}

\begin{styleStandard}
\ \ \ \ \ \ \ \ \ \ \ \ \ \ He/She is \ a \ \ \ \ person \ \ \ \ to whom.\textsc{dat} / who.\textsc{acc} doesn’t (\textsc{cl.dat}) likes
\end{styleStandard}

\begin{styleStandard}
\ \ \ \ \ \ \ \ \ \ b. \{A qui \ \ \ \ \ \ \ \ \ \ \ \ / *Qui\} \ \ \ \ \ \ no (li) \ \ \ \ \ \ \ agrada el record d’aquell fracàs és \{a \ / *Ø\} la \ \ Maria.
\end{styleStandard}

\begin{styleStandard}
\ \ \ \ \ \ \ \ \ \ \ \ \ \ \ to whom.\textsc{dat} / who.\textsc{acc} \ \ \ \ \ \ (\textsc{cl.dat}) \ \ \ \ \ \ \ \ \ \ \ \ \ \ \ \ \ \ \ \ \ \ \ \ \ \ \ \ \ \ \ \ \ \ \ \ \ \ \ \ \ \ \ \ \ \ \ \ \ is \ \ to \ / \ \ Ø \textsc{art} Maria.\textsc{dat}
\end{styleStandard}

\begin{styleStandard}
\ \ \ \ \ \ \ \ c. \{A \ qui \ \ \ \ \ \ \ \ \ \ \ / *Qui\} \ \ \ \ \ \ \ no (li) \ \ \ \ \ \ \ \ \ agrada el record d’aquell fracàs?
\end{styleStandard}

\begin{styleStandard}
\ \ \ \ \ \ \ \ \ \ \ \ \ to whom.\textsc{dat} / \ who.\textsc{acc} \ \ \ \ \ \ (\textsc{cl.dat})
\end{styleStandard}

\begin{styleStandard}
\ \ \ \ \ \ \ \ \ \ d. Ø Voldria saber \{a \ qui \ \ \ \ \ \ \ \ \ \ \ \ \ / *qui\} \ \ \ \ \ \ \ no (li) \ \ \ \ \ \ \ \ \ agrada el record d’aquell fracàs.
\end{styleStandard}

\begin{styleStandard}
\ \ \ \ \ \ \ \ \ \ \ \ \ \ \ \ \ \ \ \ \ \ \ \ \ \ \ \ \ \ \ \ \ \ \ \ \ \ \ \ \ \ to whom.\textsc{dat} / \ \ who.\textsc{acc} \ \ \ \ \ (\textsc{cl.dat})
\end{styleStandard}

\begin{styleStandard}
\ \ \ \ \ \ \ \ e. \{A / *Ø\} la \ \ \ Maria\textsubscript{(*},\textsubscript{)} \ \ \ no li \ \ \ \ \ \ \ \ agrada el record d’aquell fracàs.
\end{styleStandard}

\begin{styleStandard}
\ \ \ \ \ \ \ \ \ \ \ \ \ \ to / \ \ Ø \ \textsc{art} Maria.\textsc{dat} \ \ \ \ \textsc{cl.dat}
\end{styleStandard}

\begin{styleStandard}
(16) \ a. Ø \ \ \ \ \ \ \ És una persona \{a \ qui \ \ \ \ \ \ \ \ \ \ \ \ (l’) \ \ \ \ \ \ \ \ \ / que\} \ \ \ \ \ \ \ Ø \ \ \ \ \ \ \ \ \ \ \ \ \ \ \ \ \ \ han atabalat \ \ \ \ \ \ \ \ \ \ \ \ \ \ \ \ \ \ \ \ \ contínuament amb insídies.
\end{styleStandard}

\begin{styleStandard}
\ \ \ \ \ \ \ \ \ \ \ \ \ \ \ \ He/She is \ a \ \ \ person \ \ to whom.\textsc{dat} (\textsc{cl.acc}) / who.\textsc{acc} somebody.\textsc{pl} overwhelmed.\textsc{aux}+\textsc{part} continuously with malicious acts
\end{styleStandard}

\begin{styleStandard}
\ \ \ \ \ \ \ \ \ \ \ b. \{A \ qui \ \ \ \ \ \ \ \ \ \ \ \ (l’) \ \ \ \ \ \ \ \ \ / Qui\} \ \ \ \ \ \ Ø han atabalat contínuament amb insídies és \{a \ \ \ \ \ \ \ \ \ \ \ \ \ / Ø\} la \ \ \ Maria.
\end{styleStandard}

\begin{styleStandard}
\ \ \ \ \ \ \ \ \ \ \ \ \ \ \ \ \ to whom.\textsc{dat} (\textsc{cl.acc}) / who.\textsc{acc} \ \ \ \ \ \ \ \ \ \ \ \ \ \ \ \ \ \ \ \ \ \ \ \ \ \ \ \ \ \ \ \ \ \ \ \ \ \ \ \ \ \ \ \ \ \ \ \ \ \ \ \ \ \ \ \ \ \ \ \ \ \ \ \ \ \ \ \ is \ \ to.\textsc{dom}\footnote{ Differential object marking or DOM (see Manzani, this volume).}\textsc{ }/ Ø \ \textsc{art} Maria.\textsc{acc}
\end{styleStandard}

\begin{styleStandard}
\ \ \ \ \ \ \ \ c. \{A \ qui \ \ \ \ \ \ \ \ \ \ \ \ (l’) \ \ \ \ \ \ \ \ \ / Qui\} \ \ \ \ \ \ Ø han atabalat amb aquestes insídies?
\end{styleStandard}

\begin{styleStandard}
\ \ \ \ \ \ \ \ \ \ \ \ \ to whom.\textsc{dat} (\textsc{cl.acc}) / who.\textsc{acc} \ \ \ \ \ \ \ \ \ \ \ \ \ \ \ \ \ \ \ \ \ \ \ \ \ \ \ \ \ \ \ \ these
\end{styleStandard}

\begin{styleStandard}
\ \ \ \ \ \ \ \ \ \ d. Ø Voldria saber \{a \ qui \ \ \ \ \ \ \ \ \ \ \ \ \ (l’) \ \ \ \ \ \ \ \ \ / qui\} \ \ \ \ \ \ \ Ø han atabalat amb aquestes insídies.
\end{styleStandard}

\begin{styleStandard}
\ \ \ \ \ \ \ \ \ \ \ \ \ \ \ \ \ \ \ \ \ \ \ \ \ \ \ \ \ \ \ \ \ \ \ \ \ \ \ \ \ \ to whom.\textsc{dat} (\textsc{cl.acc}) / who.\textsc{acc} \ \ \ \ \ \ \ \ \ \ \ \ \ \ \ \ \ \ \ \ \ \ \ \ \ \ \ 
\end{styleStandard}

\begin{styleStandard}
\ \ \ \ \ \ \ \ \ \ e. \{A \ \ \ \ \ \ \ \ / Ø\} la \ \ Maria\textsubscript{(},\textsubscript{)} \ \ \ \ \ Ø l’ \ \ \ \ \ \ \ \ han atabalat contínuament amb insídies.
\end{styleStandard}

\begin{styleStandard}
\ \ \ \ \ \ \ \ \ \ \ \ \ \ \ to.\textsc{dom} / Ø \ \textsc{art} Maria.\textsc{acc} \ \ \ \ \textsc{cl.acc}
\end{styleStandard}

\begin{styleStandard}
(17) a. Ø \ \ \ \ \ \ \ \ És una persona \{a \ qui \ \ \ \ \ \ \ \ \ \ \ \ \ (l’) \ \ \ \ \ \ \ \ \ \ / que\} \ \ \ \ \ \ \ Ø \ \ \ \ \ \ \ \ \ \ \ \ \ \ \ \ \ \ \ han \{mullat \ \ \ \ \ \ \ / vist\} \ \ \ \ \ \ \ \ \ \ \ \ \ \ \ \ \ \ \ \ amb una mànega.
\end{styleStandard}

\begin{styleStandard}
\ \ \ \ \ \ \ \ \ \ \ \ \ \ \ \ He/She is \ a \ \ \ \ person \ \ \ to whom.\textsc{dat} (\textsc{cl.acc}) / who.\textsc{acc} somebody.\textsc{pl }\{wet.\textsc{aux}+\textsc{part} / saw.\textsc{aux}+\textsc{part} \} with \ a \ \ \ hose
\end{styleStandard}

\begin{styleStandard}
\ \ \ \ \ \ \ \ \ \ b. \{A qui \ \ \ \ \ \ \ \ \ \ \ \ (l’) \ \ \ \ \ \ \ \ \ / Qui\} \ \ \ \ \ \ Ø han \{mullat / vist\} amb una mànega és \{a \ \ \ \ \ \ \ \ \ / Ø\} la \ \ Maria.
\end{styleStandard}

\begin{styleStandard}
\ \ \ \ \ \ \ \ \ \ \ \ \ \ \ to whom.\textsc{dat} (\textsc{cl.acc}) / who.\textsc{acc} \ \ \ \ \ \ \ \ \ \ \ \ \ \ \ \ \ \ \ \ \ \ \ \ \ \ \ \ \ \ \ \ \ \ \ \ \ \ \ \ \ \ \ \ \ \ \ \ \ \ \ \ \ \ \ \ \ \ \ \ \ \ \ is \ to.\textsc{dom} / Ø \textsc{art} Maria.\textsc{acc}
\end{styleStandard}

\begin{styleStandard}
\ \ \ \ \ \ \ \ c. \{A \ qui \ \ \ \ \ \ \ \ \ \ \ \ (l’) \ \ \ \ \ \ \ \ \ / Qui\} \ \ \ \ \ \ Ø han \{mullat / vist\} amb una mànega?
\end{styleStandard}

\begin{styleStandard}
\ \ \ \ \ \ \ \ \ \ \ \ \ \ to whom.\textsc{dat} (\textsc{cl.acc}) / who.\textsc{acc}
\end{styleStandard}

\begin{styleStandard}
\ \ \ \ \ \ \ \ \ \ d. Ø Voldria saber \{a \ qui \ \ \ \ \ \ \ \ \ \ \ \ \ (l’) \ \ \ \ \ \ \ \ \ / qui\} \ \ \ \ \ \ \ Ø han \{mullat / vist\} amb una mànega.
\end{styleStandard}

\begin{styleStandard}
\ \ \ \ \ \ \ \ \ \ \ \ \ \ \ \ \ \ \ \ \ \ \ \ \ \ \ \ \ \ \ \ \ \ \ \ \ \ \ \ \ \ to whom.\textsc{dat} (\textsc{cl.acc}) / who.\textsc{acc}
\end{styleStandard}

\begin{styleStandard}
\ \ \ \ \ \ \ \ e. \{A \ \ \ \ \ \ \ \ / Ø\} la \ \ \ Maria\textsubscript{(},\textsubscript{)} \ \ \ \ Ø l’ \ \ \ \ \ \ \ \ han \{mullat / vist\} amb una mànega.
\end{styleStandard}

\begin{styleStandard}
\ \ \ \ \ \ \ \ \ \ \ \ \ to.\textsc{dom} / Ø \ \ \textsc{art} Maria.\textsc{acc} \ \ \textsc{cl.acc}
\end{styleStandard}

\begin{styleStandard}
Bearing in mind that stative sentences of AcExp verbs are constructed with a real dative, regardless of the morphology of the experiencer clitic, I use the abbreviation Dat({\textgreater}{\textbar}{\textless}Ac)Exp to differentiate these constructions from both AcExp causatives and DatExp statives. The abbreviation can be used in cases of hesitation between the accusative and the dative form and, at the same time, to differentiate Dat({\textgreater}Ac)Exp when the morphology is dative and Dat({\textless}Ac)Exp when the morphology is accusative.
\end{styleStandard}

\begin{listWWNumxxiileveli}
\item 
\begin{stylelsSectioni}
Argument structure of stative sentences with AcExp verbs
\end{stylelsSectioni}
\end{listWWNumxxiileveli}
\begin{styleStandard}
According to Rosselló (2008: §S 13.3.6.2\textit{a-b }and §S 13.3.7.2\textit{b}) and GIEC (§21.2.2\textit{b} and §21.5\textit{a}), one characteristic of Catalan psychological verbs with an experiencer object (AcExp and DatExp) is that they can elide their object in the absolute use of the verb. Sentences with the absolute use of these predicates can express the property of a stimulus to affect a hypothetical experiencer, a stative construction with both DatExp verbs (18\textit{a}) and AcExp verbs (18\textit{b}), which in this case does not express an action.\footnote{ The GIEC (§21.2.2\textit{c}) points out that in absolute use those verbs that have an instrumental value (\textit{tallar} ‘cut’, \textit{obrir} ‘open’, \textit{tancar} ‘close’, \textit{tapar} ‘cover’, etc.), which like AcExp verbs are generally causative of change of state, express a property of the subject rather than a particular action.}
\end{styleStandard}

\begin{styleStandard}
(18) a. La \ xocolata \ agrada (‘és agradable’); La \ família importa (‘és important’). 
\end{styleStandard}

\begin{styleStandard}
\ \ \ \ \ \ \ \ \ \ \ the chocolate likes \ \ \ \ \ \ is \ pleasant \ \ \ \ \ \ the family \ matters \ \ \ \ is important
\end{styleStandard}

\begin{styleStandard}
\ \ \ \ \ \ \ \ \ b. Els nens molesten (‘són molestos’); El \ \ \ teu \ \ caràcter \ \ atabala \ \ \ \ \ \ (‘és atabalador’).
\end{styleStandard}

\begin{styleStandard}
\ \ \ \ \ \ \ \ \ \ \ \ \ the kids annoy \ \ \ \ \ \ \ \ are annoying \ \ \ \ \textsc{art} your character overwhelms is overwhelming
\end{styleStandard}

\begin{styleStandard}
Following Cuervo’s proposal (2003: §1.3.3.2 (25)) for verbs that she calls \textit{predicational statives}, all the sentences in (18) have an underlying stative unaccusative structure. For sentences with an experiencer, we need a functional head that introduces a dative with experiencer semantics and the characteristics of a subject in a hierarchically superior position and which relates it to the whole event that indicates a property of the stimulus: a high applicative head (external argument), with the dative in the position of specifier (cf. Pylkkänen 2008; Cuervo 2003, 2010; see also Cuervo, this volume) (19).\footnote{ Other authors explain the variability between the stative and the causative reading of these verbs without a high applicative head that introduces the experiencer in the stative construction (see Viñas-de-Puig 2014, 2017, and references therein). For example, Viñas-de-Puig proposes that in both readings the experiencer is licensed for a S\textit{v}\textsc{\textsubscript{exp}} head above the root, in a basic stative structure, which will take a causative reading by adding a S\textit{v}\textsc{\textsubscript{caus}} above the S\textit{v}\textsc{\textsubscript{exp}}.}
\end{styleStandard}

\begin{styleStandard}
(19) a. DatExp \ \ \ \ \ \ \ \ \ \ \ \ \ \ \ \ \ \ \ \ \ \ \ \ \ \ \ \ \ \ \ \ \ \ \ \ \ \ \ \ \ \ \ \ \ \ \ b. Dat({\textgreater}{\textbar}{\textless}Ac)Exp
\end{styleStandard}

\begin{styleStandard}
\ \ \ \ \ \ \ \ \ \ \ A la Maria li agrada la xocolata.\ \  \ \ \ \ \ \ \ \ \ \ A la Maria li/la molesten els nens.
\end{styleStandard}

\begin{styleStandard}
\ \ (lit.) ‘to Maria chocolate is pleasant’ \ \ \ \ \ (lit.) ‘to Maria kids are annoying’
\end{styleStandard}

\begin{styleStandard}
[Warning: Draw object ignored][Warning: Draw object ignored][Warning: Draw object ignored][Warning: Draw object ignored]ApplP\ \ \ \ \ \ \ \ \ \ ApplP
\end{styleStandard}

\begin{styleStandard}
\ \ \ DP\ \ \ \ \ \ \ \ \ \  \ \ DP
\end{styleStandard}

\begin{styleStandard}
[Warning: Draw object ignored][Warning: Draw object ignored][Warning: Draw object ignored][Warning: Draw object ignored][Warning: Draw object ignored][Warning: Draw object ignored][Warning: Draw object ignored][Warning: Draw object ignored][Warning: Draw object ignored]\textit{a la Maria\ \ \ \ \ \ \ \ a la Maria}
\end{styleStandard}

\begin{styleStandard}
Appl\ \ \ \  \textit{v}P\ \ \ \ \ \ Appl\ \ \ \  \textit{v}P
\end{styleStandard}

\begin{styleStandard}
[Warning: Draw object ignored]\textit{\ \  \ \ }\textbf{\textit{li}}\textit{\ \ \ \ \ \ \ \ \ \ }\textbf{\textit{li}}\textit{ / }\textbf{\textit{la}}
\end{styleStandard}

\begin{styleStandard}
\ \ \ \  DP\ \ \ \ \ \ \ \ \ \  DP
\end{styleStandard}

\begin{styleStandard}
[Warning: Draw object ignored][Warning: Draw object ignored][Warning: Draw object ignored][Warning: Draw object ignored]\textit{\ \ \ \ \ \ la xocolata\ \ \ \ \ \ \ \  \ \ \ \ \ \ \ \ \ els nens}
\end{styleStandard}

\begin{styleStandard}
\ \ \ \ \ \ \textit{v}\textsc{\textsubscript{be}}\ \  \ \ \ \ \ \ \ \ Root\ \ \ \ \ \  \textit{v}\textsc{\textsubscript{be}}\ \  \ \ \ \ \ \ \ \ Root
\end{styleStandard}

\begin{styleStandard}
\ \ \ \ \ \ ${\surd}$\textit{agrad-\ \ }\ \ \ \ \ \  \ \ \ \ \ ${\surd}$\textit{molest-}
\end{styleStandard}

\begin{styleStandard}
The unaccusative structure of (19\textit{a}) for DatExp verbs matches Belletti \& Rizzi’s characterisation of type-III predicates. The construction of (19\textit{b}), however, requires some additional clarifications. Apparently, we should reject an unaccusative structure with an accusative experiencer – and in Catalan we do not expect an accusative to be an external argument – but if we bear in mind that it is a superficial accusative and that it is really a dative (cf. §3 i §4), this objection disappears. We also need to be explain how some verbs can optionally use the accusative and dative forms (5)-(7), and other verbs the accusative form in OVS stative sentences, whether they are AcExp (4\textit{a}) or causative predicates with a metaphorical psychological meaning (4\textit{b}).
\end{styleStandard}

\begin{styleStandard}
In these sentences, the experiencer is a non-topicalized element with subject properties and a real dative, regardless of the form it takes. The syntactic mechanism that can explain sentences in which the experiencer has an apparent accusative morphology (20\textit{b}) is differential indirect object marking or DIOM (cf. Bilous 2011; Pineda 2016; Pineda in press; Pineda \& Royo 2017), which is not necessary when the clitic takes a dative morphology (20\textit{a}).
\end{styleStandard}

\begin{styleStandard}
(20) a. Dat({\textgreater})Exp \ \ \ \ \ \ \ \ \ \ \ \ \ \ \ \ \ \ \ \ \ \ \ \ \ \ \ \ \ \ \ \ \ \ \ \ b. Dat({\textless}Ac)Exp
\end{styleStandard}

\begin{styleStandard}
\ \ \ \ \ \ \ \ \ \ \ A la Maria \textbf{li} molesten els nens.\ \  \ \ \ \ A la Maria l’atabala el teu caràcter.
\end{styleStandard}

\begin{styleStandard}
\ \ (lit.) ‘to Maria kids are annoying’ \ \ \ (lit.) ‘to Maria your character is overwhelming’
\end{styleStandard}

\begin{styleStandard}
[Warning: Draw object ignored][Warning: Draw object ignored][Warning: Draw object ignored][Warning: Draw object ignored]ApplP\ \ \ \ \ \ \ \ \ \ ApplP
\end{styleStandard}

\begin{styleStandard}
\ \ \ DP\ \ \ \ \ \ \ \ \ \  \ \ DP
\end{styleStandard}

\begin{styleStandard}
[Warning: Draw object ignored][Warning: Draw object ignored][Warning: Draw object ignored][Warning: Draw object ignored][Warning: Draw object ignored][Warning: Draw object ignored][Warning: Draw object ignored][Warning: Draw object ignored][Warning: Draw object ignored]\textit{a la Maria\ \ \ \ \ \ \ \ a la Maria}
\end{styleStandard}

\begin{styleStandard}
Appl\ \ \ \  \textit{v}P\ \ \ \ \ \ Appl\ \ \ \  \textit{v}P
\end{styleStandard}

\begin{styleStandard}
[Warning: Draw object ignored]\textit{\ \  \ \ }\textbf{\textit{li}}\textit{\ \ \ \ \ \ \ \ \ \  \ }\textbf{\textit{l’}}\textbf{ }\textbf{\textsubscript{[DIOM]}}
\end{styleStandard}

\begin{styleStandard}
\ \ \ \  DP\ \ \ \ \ \ \ \ \ \  DP
\end{styleStandard}

\begin{styleStandard}
[Warning: Draw object ignored][Warning: Draw object ignored][Warning: Draw object ignored][Warning: Draw object ignored]\textit{\ \ \ \ \ \ \ \ \ \ els nens\ \ \ \ \ \ \ \  \ \ \ el teu caràcter}
\end{styleStandard}

\begin{styleStandard}
\ \ \ \ \ \ \textit{v}\textsc{\textsubscript{be}}\ \  \ \ \ \ \ \ \ \ Root\ \ \ \ \ \  \textit{v}\textsc{\textsubscript{be}}\ \  \ \ \ \ \ \ \ \ Root
\end{styleStandard}

\begin{styleStandard}
\ \ \ \ \ \ ${\surd}$\textit{molest-\ \ }\ \ \ \ \ \  \ \ \ \ \ ${\surd}$\textit{atabal-}
\end{styleStandard}

\begin{styleStandard}
The dative case marking of these sentences is congruent with the semantic and syntactic characteristics of the experiencer and with the function of the high applicative heads in a Romance language like Catalan. A DIOM accusative morphology would allow speakers to use these constructions with verbs that are difficult to conceive as stative, because in the minds of speakers they are closely related to verbs that cause a change of state (4). The morphological aspect of the experiencer depends on the lexical characteristics of the verb: even though the sentence is always stative, we can regard DIOM as being an anti-stativization mechanism in the minds of speakers. In this sense, it is significant that non-psychological causative verbs with a metaphorical psychological meaning present the superficial accusative form in OVS stative sentences (\textit{destrossar} ‘destroy’, \textit{enfonsar} ‘sink’). Like some psychological verbs (\textit{commoure} ‘move, touch’, \textit{esparverar} ‘terrify’),\footnote{ Ginebra (2003: 14; 29-30) offers more examples of OVS stative sentences of this type with a superficial accusative in both verb types, that is, psychological and non-psychological verbs with metaphorical psychological meaning.} they are verbs that speakers conceptualize habitually as being causative of change of state, unlike other verbs that more readily permit a stative conceptualization in certain contexts: for example, \textit{molestar} ‘annoy’, which can sometimes have the meaning of \textit{desagradar molt} (‘displease a lot').\footnote{ For an explanation of other factors that intervene so that an AcExp verb can participate in sentences such as Dat({\textless}Ac)Exp or Dat({\textgreater}Ac)Exp, see Royo (2017: §5).}
\end{styleStandard}

\begin{styleStandard}
This explanation takes into account the conceptual mechanisms that can, according to several authors, affect the construction of sentences and syntactic change: the speakers’ conception of the world (cf. Ramos 2002), the linguistic conception of particular communicative contexts (cf. Rosselló 2008) and the different conceptualisation of transitivity (cf. Ynglès 2011; Pineda 2012).
\end{styleStandard}

\begin{listWWNumxxiileveli}
\item 
\begin{stylelsSectioni}
Conclusions
\end{stylelsSectioni}
\end{listWWNumxxiileveli}
\begin{styleStandard}
The main argument presented in this article is that in stative sentences of Catalan AcExp predicates, the experiencer is a real dative. In stative sentences of some AcExp verbs and other non-psychological causative verbs with metaphorical psychological semantics, the experiencer may present an external accusative morphology by means of differential indirect object marking (DIOM). DIOM is the manifestation in the minds of speakers of their difficulty to conceive certain verbs as being stative or, in other words, of their tendency to conceive them as being causative of change of state.
\end{styleStandard}

\begin{styleStandard}
\textbf{Acknowledgments.} This study has been supported by research project FFI2014-56258-P (\textit{Ministerio de Economía y Competitividad}). I would like to thank Jaume Mateu for specific comments made in relation to this paper and Anna Pineda for encouraging me to present this research in public and to have it published.
\end{styleStandard}

\begin{styleStandard}
\textbf{References}
\end{styleStandard}

\begin{styleStandard}
Acedo Matellán, Víctor \& Jaume Mateu. 2015. Los verbos psicológicos: raíces especiales en estructuras corrientes. In Rafael Marín (ed.), \textit{Los verbos psicológicos}, 81-109. Madrid: Visor Libros.
\end{styleStandard}

\begin{styleStandard}
Alsina, Àlex. 2008. L’infinitiu. In Joan Solà, Maria-Rosa Lloret, Joan Mascaró \& Manuel Pérez Saldanya (dirs.), \textit{Gramàtica del català contemporani}, 3 vol., 4th ed. [2002], 2389-2454. Barcelona: Editorial Empúries.
\end{styleStandard}

\begin{styleStandard}
Arad, Maya. 1999. On “little $\upsilon $”. \textit{MIT Working Papers in Linguistics}, 33. 1-25.
\end{styleStandard}

\begin{styleStandard}
Belletti, Adriana \& Luigi Rizzi. 1988. Psych-Verbs and Theta-Theory. \textit{Natural Language and Linguistic Theory}, 6, 3. 291-352. 
\end{styleStandard}

\begin{styleStandard}
Bilous, Rostyslav. 2011. \textit{Transitivité et marquage d’objet différentiel}. PhD diss., University of Toronto.
\end{styleStandard}

\begin{styleStandard}
Bouchard, Denis. 1995. \textit{The Semantics of Syntax. A Minimalist Approach to Grammar}. Chicago, London: University of Chicago Press.
\end{styleStandard}

\begin{styleStandard}
Cabré, Teresa \& Jaume Mateu. 1998. Estructura gramatical i normativa lingüística: a propòsit dels verbs psicològics en català. \textit{Quaderns. Revista de traducció},\textit{ }2. 65-81.
\end{styleStandard}

\begin{styleStandard}
Campos, Héctor. 1999. Transitividad e intransitividad. In Ignacio Bosque \& Violeta Demonte (eds.), \textit{Gramática Descriptiva de la Lengua Española}, 3 vol, 1519-1574. Madrid: Espasa Calpe.
\end{styleStandard}

\begin{styleStandard}
Cuervo, M. Cristina. 1999. Quirky But Not Eccentric: Dative Subjects in Spanish. \textit{MIT Working Papers in Linguistics}, 34. 213-227.
\end{styleStandard}

\begin{styleStandard}
Cuervo, M. Cristina. 2003. \textit{Datives at large}. PhD diss., MIT.
\end{styleStandard}

\begin{styleStandard}
Cuervo, M. Cristina. 2010. La estructura de expresiones con verbos livianos y experimentante. In Marta Luján \& Mirta Groppi (eds.), \textit{Cuestiones gramaticales del español, últimos avances}, 194-206. Santiago de Chile: ALFAL.
\end{styleStandard}

\begin{styleStandard}
Demonte, Violeta (1989). \textit{Teoria sintáctica: de las estructuras a la rección}. Madrid: Síntesis. 
\end{styleStandard}

\begin{styleStandard}
DIEC2 = Institut d’Estudis Catalans. 2007. \textit{Diccionari de la llengua catalana}, 2nd ed. [1995]. Barcelona: Institut d’Estudis Catalans, Enciclopèdia Catalana, Edicions 62. Consult online at {\textless}http://dlc.iec.cat{\textgreater}. 
\end{styleStandard}

\begin{styleStandard}
Eguren, Luis \& Olga Fernández Soriano. 2004. \textit{Introducción a una sintaxis minimista}. Madrid: Gredos. 
\end{styleStandard}

\begin{styleStandard}
Fábregas, Antonio. 2015. No es experimentante todo lo que experimenta o cómo determinar que un verbo es psicológico. In Rafel Marín (ed.), \textit{Los predicados psicológicos}, 51-79. Madrid: Visor Libros.
\end{styleStandard}

\begin{styleStandard}
Fábregas, Antonio \& Rafael Marín. 2012. State nouns are Kimian states. In Irene Franco, Sara Lusini \& Andrés Saab (eds.), \textit{Romance Languages and Linguistic Theory 2010: Selected Papers from 'Going Romance' Leiden 2010}, 4, 41-64. Amsterdam, Philadelphia: John Benjamins Publishing Company.
\end{styleStandard}

\begin{styleStandard}
Fábregas, Antonio, Rafael Marín \& Louise McNally. 2012. From psych verbs to nouns. In Violeta Demonte \& Louise McNally (eds.), \textit{Telicity, Change, and State: A Cross-Categorial View of Event Structure}, 162-185. New York: Oxford University Press.
\end{styleStandard}

\begin{styleStandard}
Ganeshan, Ashwini. 2014. Revisiting Spanish ObjExp Psych Predicates. In Claire Renaud, Carla Ghanem, Verónica González López \& Kathryn Pruitt (eds.), \textit{Proceedings of WECOL 2013} (held at Arizona State University, Tempe campus, November 8-10, 2013), 73-84. Fresno: Department of Linguistics, California State University, Fresno.
\end{styleStandard}

\begin{styleStandard}
Ganeshan, Ashwini. 2019. Examining Animacy and Agentivity in Spanish Reverse-psych verbs. \textit{Studies in Hispanic and Lusophone Linguistics}, 12 (1). 1-33.
\end{styleStandard}

\begin{styleStandard}
GIEC = Institut d’Estudis Catalans. 2016. \textit{Gramàtica de la llengua catalana}. Barcelona: Institut d’Estudis Catalans.
\end{styleStandard}

\begin{styleStandard}
Ginebra, Jordi. 2003. El règim verbal i nominal. Ms.
\end{styleStandard}

\begin{styleStandard}
Ginebra, Jordi. 2005. \textit{Praxi lingüística. III. Criteris gramaticals i d’estil. Textos de normalització lingüística} 6. Tarragona: Servei Lingüístic de la Universitat Rovira i Virgili.
\end{styleStandard}

\begin{styleStandard}
Ginebra, Jordi. 2015. Neologia i gramàtica: entre el neologisme lèxic i el neologisme sintàctic. \textit{Caplletra}, 59. 137-157.
\end{styleStandard}

\begin{styleStandard}
Landau, Idan. 2010. \textit{The Locative Syntax of Experiencers}. Cambridge, Mass.: MIT Press.
\end{styleStandard}

\begin{styleStandard}
Levin, Beth \& Malka Rappaport Hovav. 1995. \textit{Unaccusativity. At the syntax-lexical semantics interface}. Cambridge, Mass.: MIT Press.
\end{styleStandard}

\begin{styleStandard}
Marín, Rafael \& Louise McNally. 2011. Inchoativity, change of state and telicity. \textit{Natural Language and Linguistic Theory}, 29. 467-502.
\end{styleStandard}

\begin{styleStandard}
Marín, Rafael \& Cristina Sánchez Marco. 2012. Verbos y nombres psicológicos: juntos y revueltos\textit{. Borealis–An International Journal of Hispanic Linguistics}, 1 (2). 91-108.
\end{styleStandard}

\begin{styleStandard}
Masullo, Pascual José. 1992. Quirky Datives in Spanish and the Non-Nominative Subject Parameter. In Andrea Kathol \& Jill Beckman (eds.), \textit{Proceedings of the 4th Meeting of SCIL, MITWPL}, 16. 89-104.
\end{styleStandard}

\begin{styleStandard}
Mendívil Giró, José Luis. 2005. El comportamiento variable de \textit{molestar}: \textit{A Luisa le molesta que la molesten}. In Gerd Wotjak \& Juan Cuartero Otal (eds.), \textit{Entre semántica léxica, teoría del léxico y sintaxis}, 261-272. Frankfurt: Peter Lang.
\end{styleStandard}

\begin{styleStandard}
Pesetsky, David. 1995. \textit{Zero Syntax.} \textit{Experiencers and Cascades}. Cambridge, Mass.: MIT Press.
\end{styleStandard}

\begin{styleStandard}
Pineda, Anna. 2012. Transitividad y afectación en el entorno lingüístico romance y eusquérico. In Xulio Viejo (coord.), \textit{Estudios sobre variación sintáctica peninsular. Seminariu de Filoloxía Asturiana}, Universidá d’Oviéu, 31-73. Oviéu: Trabe.
\end{styleStandard}

\begin{styleStandard}
Pineda, Anna. 2016. \textit{Les fronteres de la (in)transitivitat. Estudi dels aplicatius en llengües romàniques i basc}. Barcelona: Institut d’Estudis Món Juïc.
\end{styleStandard}

\begin{styleStandard}
Pineda, Anna. In press. From Dative to Accusative. An Ongoing Syntactic Change in Romance. \textit{Probus. International Journal of Romance Linguistics}.
\end{styleStandard}

\begin{styleStandard}
Pineda, Anna \& Carles Royo. 2017. Differential Indirect Object Marking in Romance (and How to Get Rid of it). \textit{Revue Roumaine de Linguistique}, 4. 445-462.
\end{styleStandard}

\begin{styleStandard}
Pylkkänen, Liina. 2008.\textit{ Introducing arguments}. Cambridge, Mass.: MIT Press.
\end{styleStandard}

\begin{styleStandard}
Ramos, Joan-Rafael. 2002. Factors del canvi sintàctic. In M. Antònia Cano, Josep Martines, Vicent Martines \& Joan J. Ponsoda (eds.), \textit{Les claus del canvi lingüístic} (\textit{Symposia Philologica }5), 397-428. Alacant: Institut Interuniversitari de Filologia Valenciana, Ajuntament de Nucia, Caja de Ahorros del Mediterráneo.
\end{styleStandard}

\begin{styleStandard}
Ramos, Joan-Rafael. 2004. El règim verbal: anàlisi contrastiva català-castellà. In Cesáreo Calvo, Emili Casanova \& Fco. Javier Satorre (eds.), \textit{Lingüística diacrònica contrastiva}, 119-139. València: Universitat de València.
\end{styleStandard}

\begin{styleStandard}
Rosselló, Joana. 2008. El SV, I: Verbs i arguments verbals. In Joan Solà, Maria-Rosa Lloret, Joan Mascaró \& Manuel Pérez Saldanya (dirs.), \textit{Gramàtica del català contemporani}, 3 vol., 4th ed. [2002], 1853-1949. Barcelona: Editorial Empúries.
\end{styleStandard}

\begin{styleStandard}
Royo, Carles. 2017. \textit{Alternança acusatiu/datiu i flexibilitat semàntica i sintàctica dels verbs psicològics catalans}. PhD diss., Universitat de Barcelona.
\end{styleStandard}

\begin{styleStandard}
van Voorst, Jan. 1992. The aspectual semantics of psychological verbs. \textit{Linguistics and Philosiphy}, 15 No. 1. 65-92.
\end{styleStandard}

\begin{styleStandard}
Viñas-de-Puig, Ricard. 2014. Predicados psicológicos y estructuras con verbo ligero: Del estado al evento. \textit{Revista de Lingüística Teórica y Aplicada}, 52 (2). 165-188.
\end{styleStandard}

\begin{styleStandard}
Viñas-de-Puig, Ricard. 2017. Psych predicates, light verbs, and phase theory: On the implications of case assignment to the experiencer in non-\textit{leísta} experience predicates. In Juan J. Colomina-Almiñana (ed.), \textit{Contemporary advances in theoretical and applied Spanish linguistics variation}, 201-224. Columbus, OH: The Ohio State University Press.
\end{styleStandard}

\begin{styleStandard}
Ynglès, M. Teresa. 1991. Les relacions semàntiques del cas datiu. In Jane White Albrecht, Janet Ann DeCesaris, Patricia V. Lunn \& Josep Miquel Sobrer (eds.), \textit{Homenantge a Josep Roca-Pons. Estudis de llengua i literatura}, 271-308. Barcelona: Publicacions de l’Abadia de Montserrat, Indiana University.
\end{styleStandard}

\begin{styleStandard}
Ynglès, M. Teresa. 2011. \textit{El datiu en català: una aproximació des de la lingüística cognitiva}. Barcelona: Publicacions de l’Abadia de Montserrat.
\end{styleStandard}
\end{document}
