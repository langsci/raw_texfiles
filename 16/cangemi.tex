\documentclass[ number=1
,series=labphon
,output=long
% ,draftmode
,url=http://langsci-press.org/catalog/book/16
,isbn=978-3-944675-01-5
]{LSP/langsci}
\usepackage{graphicx}	% image insertion
\usepackage{float}	% figure positioning
\usepackage{lscape}	% table/figure rotation
\usepackage{multirow}	% specific kind of tables
\usepackage[british,italian]{babel}	% hyphenation
\usepackage{LSP/lsp-styles/lsp-gb4e}	% leipzig glosses
\usepackage{setspace}	% and following: epigraphs
\renewcommand\epigraph[4]{
\vspace{1em}\hfill{}\begin{minipage}{#1}{\begin{spacing}{0.9}
\noindent\textit{#2}\end{spacing}
\vspace{1em}\small\hfill{}{#3}\\
\vspace{-2em}\begin{flushright}{#4}\end{flushright}}\vspace{2em}
\end{minipage}}
\title{Prosodic detail in Neapolitan Italian}
\author{Francesco Cangemi}
\typesetter{Francesco Cangemi}
\proofreader{Tom Gardner, Martin Hilpert, Michelle Natolo, Stephanie Natolo, Benedikt Singpiel, Siri Tuttle, Tamara Schmidt}
\BackTitle{Prosodic detail in Neapolitan Italian}
\BackBody{Recent findings on phonetic detail have been taken as supporting exemplar-based approaches to prosody. Through four experiments on both production and perception of both melodic and temporal detail in Neapolitan Italian, we show that prosodic detail is not incompatible with abstractionist approaches either. Specifically, we suggest that the exploration of prosodic detail leads to a refined understanding of the relationships between the richly specified and continuously varying phonetic information on one side, and coarse phonologically structured contrasts on the other, thus offering insights on how pragmatic information is conveyed by prosody.}
\lccode`\'=`\'
\hyphenation{
D'Im-pe-rio
}
\usepackage{booktabs}
\usepackage{floatrow}
\floatsetup[table]{capposition=top}
\newcommand{\mytoprule}{\midrule\toprule}
\newcommand{\mybottomrule}{\bottomrule\midrule}
\begin{document}
\selectlanguage{british}
\maketitle
\tableofcontents\enlargethispage{2em}
\mainmatter
\chapter{Einleitung}\label{1}

\epigraph{%
If it really were so that languages varied greatly in the complexity of subsystem \textit{X}, varied greatly in the complexity of subsystem \textit{Y}, and so on, yet for all languages the totals from the separate subsystems added together could be shown to come out the same, then I would not agree with Hockett in finding this unsurprising. To me it would feel almost like magic.\hfill\citep[2]{Sampson2009}}

\noindent Dieses Zitat illustriert, wie stark sich die Annahmen zur Komplexität in Sprachen zwischen dem 20. und dem 21. Jh. unterscheiden. Wenn man sich im 20. Jh. überhaupt mit eventuellen Komplexitätsunterschieden beschäftigt hat, ging man generell davon aus, dass unter dem Strich alle Sprachen gleich komplex sind (mit einigen Ausnahmen vor allem in der Variationslinguistik), was als \is{Equi-Complexity-Hypothese}\textit{Equi-Com\-ple\-xi\-ty-Hy\-po\-the\-se} bezeichnet wird. Dies kann sicher u.a. auch als Gegenreaktion auf Annahmen aus dem 19. Jh. interpretiert werden. Im 21. Jh. hingegen wurde die \is{Equi-Complexity-Hypothese}\textit{Equi-Com\-ple\-xi\-ty-Hy\-po\-the\-se} stark hinterfragt, woraus bereits zahlreiche Arbeiten vor allem aus der Typologie entstanden sind. Was jedoch bis jetzt fehlt, ist einerseits die Messung der Komplexität eines größeren Teilsystems und andererseits der Komplexitätsvergleich von eng verwandten Varietäten, die sich entweder synchron aus soziolinguistischer Perspektive oder diachron unterscheiden.

Eng verwandte Varietäten zu untersuchen, hat den Vorteil, dass diachrone Prozesse, wie z.\,B.\ Simplifizierung und Komplexifizierung, viel direkter betrachtet werden können \citep[22]{BaechlerSeiler2012}. Aufgrund der genetischen Verwandtschaft, die einen gemeinsamen historischen Ursprung impliziert, können diachrone Prozesse und ihre Wechselwirkungen innerhalb eines Systems A verglichen werden mit den diachronen Prozessen innerhalb eines anderen Systems B, das aber mit System A eng verwandt ist. Es stellt sich dann z.\,B.\ die Frage, weshalb zwei eng verwandte Varietäten trotz des gemeinsamen historischen Ursprungs (völlig) unterschiedliche Wege gehen.

Das Ziel dieser Arbeit ist, die absolute Komplexität der Nominalflexion in den folgenden Varietäten zu messen und zu vergleichen: Althochdeutsch, Mittelhochdeutsch, deutsche Standardsprache sowie siebzehn alemannische Dialekte aus dem höchstalemannischen, hochalemannischen, oberrheinalemannischen und schwäbischen Gebiet. Als Datengrundlage dienen \isi{Ortsgrammatiken}. Berücksichtigt wurden jene nominalen Wortarten, für die in allen zwanzig untersuchten Varietäten Beschreibungen existieren und die sich in ihrer Flexion unterscheiden: \isi{Substantiv}, \isi{Adjektiv}, \isi{Interrogativpronomen}, \isi{Personalpronomen}, einfaches \isi{Demonstrativpronomen}, \isi{Possessivpronomen}, bestimmter und \isi{unbestimmter Artikel}. Im Gegensatz zur Verbalflexion können also verschiedene Wortarten miteinander verglichen werden. Auch die Variationsbreite zwischen den Varietäten ist in der Nominalflexion grösser.

Die theoretischen Grundlagen bilden LFG (\is{Lexical-Functional Grammar (LFG)}Lexical-Functional Grammar) und die in\-fe\-ren\-tiel\-le-re\-a\-li\-sie\-ren\-de Morphologie. Aus diesen Modellen wird abgeleitet, was ein System komplexer bzw. simpler macht. Darauf auch die eigens entwickelte Methode zur Messung der Komplexität. Bei der Entwicklung der Messmethode standen drei Qualitätsmerkmale im Vordergrund: a) Die Messmethode muss so objektiv wie möglich sein; b) Die Messmethode muss auch kleinste Unterschiede von eng verwandten Varietäten messen können; c) Die Messmethode muss prinzipiell auf alle flektierenden Sprachen angewendet werden können. Im hier verwendeten Modell wird davon ausgegangen, dass Flexionsparadigmen durch sogenannte \isi{Realisierungsregeln} definiert sind. Folglich gilt, je mehr \isi{Realisierungsregeln} zur Definition der Flexionsparadigmen benötigt werden, desto komplexer ist das Flexionssystem. Schließlich soll überprüft werden, ob, und wenn ja, wie folgende Faktoren die \isi{Variation} in der Komplexität der Nominalflexion erklären können:

\begin{itemize}
\item 
Nimmt die Komplexität der Nominalflexion im Laufe der Zeit zu oder ab? Mit welchen soziolinguistischen Faktoren kann die Zu- oder Abnahme in Zusammenhang gebracht werden?
\item 
Unterscheiden sich die alemannischen \isi{Dialektgruppen} (Höchst-, Hoch-, Oberrheinalemannisch, Schwäbisch) in ihrer Komplexität?
\item 
Wie beeinflusst der Kontakt mit anderen Dialekten oder anderen Sprachen die Komplexität?
\item 
Wie beeinflussen die Standardisierung und die Kodifizierung die Komplexität?
\item 
Führt \isi{Isolation} (wenig Kontakt und geografische \isi{Isolation}) zu einer höheren oder niedrigeren Komplexität?
\end{itemize}

\noindent
Die vorliegende Arbeit ist wie folgt strukturiert. \chapref{2} gibt einen wissenschaftsgeschichtlichen Überblick über die Beschäftigung mit der linguistischen Komplexität (\sectref{2.1}) sowie über etliche Definitionen, Einflussfaktoren und Messmethoden, die zur linguistischen Komplexität vorgeschlagen und getestet wurden (\sectref{2.2}). \chapref{3} beantwortet die Fragen, weshalb hier gerade die Nominalflexion untersucht wird und was unter Komplexität verstanden wird (\sectref{3.1}). Des Weiteren werden die Hypothesen (\sectref{3.2}) und die untersuchten Varietäten (\sectref{3.3}) eingeführt. Im \chapref{4} werden die theoretischen Grundlagen vorgestellt (\sectref{4.1}), was ein System komplexer oder simpler macht (\sectref{4.2}) sowie die Methode zur Messung der absoluten Komplexität (\sectref{4.3}). \chapref{5} analysiert die Flexionsparadigmen und \isi{Realisierungsregeln} der hier untersuchten Varietäten. In \chapref{6} werden die Resultate vorgestellt, interpretiert und dadurch die Hypothesen überprüft. In \chapref{7} wird die vorliegende Arbeit kurz zusammengefasst, und zwar kombiniert mit einem Ausblick.

Außerdem folgen hier noch einige Anmerkungen zur Transkription und Verschriftlichung der untersuchten Varietäten. Die Daten basieren auf unterschiedlichen Ortgrammatiken und Beschreibungen der Varietäten, die verschiedene Transkriptionen (z.\,B.\ Teuthonista, Dieth etc.) verwenden und unterschiedlich präzise sind. Da diese Arbeit sich mit der nominalen Flexionsmorphologie beschäftigt, muss nur bezüglich der Flexionsaffixe auf phonetische Genauigkeit geachtet werden. Der einfacheren Lesbarkeit halber wird hier der Pho\-nem-Gra\-phem-Ent\-spre\-chung der deutschen Rechtschreibung gefolgt. Ausnahmen bilden Langvokale besonders in den alemannischen Dialekten, die mit einem Strich über dem betreffenden Vokal verschriftlicht werden (z.\,B.\ \=a). Des Weiteren gelten andere Regeln bezüglich der (vor allem auslautenden) e-Laute und a-Laute. Da gerade die untersuchten Dialekte im Auslaut unterschiedliche a- und e-Qualitäten unterscheiden, die verschiedene morphosyntaktische Eigenschaften kodieren, wird hier IPA gefolgt: [e], [ɛ], [æ], [a], [ɑ], [ɐ], [ə]. Aufgrund der unterschiedlichen Angaben in den \isi{Ortsgrammatiken} zur Lautqualität ist mit Ungenauigkeiten zu rechnen. Es wurde in dieser Arbeit jedoch darauf geachtet, dass, wenn in der \isi{Ortsgrammatik} zwei \isi{Affixe} unterschieden werden, dieser Unterschied auch in den Paradigmen aufgenommen wird. Dabei spielt die genaue Lautqualität der \isi{Affixe} keine ausschlaggebende Rolle, da hier die Flexionsmorphologie untersucht wird.

Bei der vorliegenden Monographie handelt es sich um eine überarbeitete Version meiner Dissertation, die von Prof.\ Dr.\ Guido Seiler und Prof.\ Dr.\ Martin Joachim Kümmel betreut und von der Albert-Ludwigs-Universität Freiburg angenommen wurde (promoviert am 23. Mai 2016). Schließlich möchte ich mich bei den folgenden Personen für ihre Unterstützung und die zahlreichen anregenden Diskussionen bedanken (alphabetisch gelistet): Ulrike Ackermann, Marco Angster, Manuela Baechler, Pia Bergmann, Antje Dammel, Jacopo Garzonio, Nikolay Khakimov, Martin Joachim Kümmel, Adriano Murelli, Johanna Nichols, Harald Noth, Simon Prentice, Simon Pröll, Javier Caro Reina, Lea Schäfer, Oliver Schallert, Guido Seiler, Peter Trudgill und Thilo Weber. Ein herzliches Dankeschön geht des Weiteren an Felix Kopecky für seine Unterstützung bezüglich Latex sowie an die LektorInnen.
\documentclass[output=paper,colorlinks,citecolor=brown]{langscibook}
\ChapterDOI{10.5281/zenodo.15682186}
\title[Language-independent and language-specific properties]{Language-independent and language-specific properties of semantic description: A case study on verbs of communication} 
\author{Svetlozara Leseva\orcid{0000-0001-8198-4555}\affiliation{Department of Computational Linguistics, Institute for Bulgarian Language, Bulgarian Academy of Sciences} and Ivelina Stoyanova\orcid{0000-0003-3771-435X}\affiliation{Department of Computational Linguistics, Institute for Bulgarian Language, Bulgarian Academy of Sciences}
}

\abstract{The study focuses on the properties of verb conceptual description in view of their linguistic universality and transferability of conceptual information across languages. Further, we present the semantic class of verbs of communication, the hierarchical organisation of frames and the corresponding frame elements. We consider the most prominent FrameNet frames evoking verbs of communication of higher frequency and make observations on the syntactic realisation of the frame elements in different valence patterns both in English and Bulgarian.}


\IfFileExists{../localcommands.tex}{
   \addbibresource{../localbibliography.bib}
   % add all extra packages you need to load to this file

\usepackage{tabularx,multicol}
\usepackage{url}
\urlstyle{same}

\usepackage{listings}
\lstset{basicstyle=\ttfamily,tabsize=2,breaklines=true}

\usepackage{langsci-basic}
\usepackage{langsci-optional}
\usepackage{langsci-lgr}
\usepackage{langsci-osl}
% \usepackage{./langsci/styles/langsci-lgr}
% \usepackage{./langsci/styles/langsci-osl}
% \usepackage{langsci-gb4e}

\usepackage{tikz}
\usetikzlibrary{patterns,calc}
\pgfdeclarepatternformonly{south east lines}{\pgfqpoint{-0pt}{-0pt}}{\pgfqpoint{3pt}{3pt}}{\pgfqpoint{3pt}{3pt}}{
    \pgfsetlinewidth{0.6pt}
    \pgfpathmoveto{\pgfqpoint{0pt}{3pt}}
    \pgfpathlineto{\pgfqpoint{3pt}{0pt}}
    \pgfpathmoveto{\pgfqpoint{.2pt}{-.2pt}}
    \pgfpathlineto{\pgfqpoint{-.2pt}{.2pt}}
    \pgfpathmoveto{\pgfqpoint{3.2pt}{2.8pt}}
    \pgfpathlineto{\pgfqpoint{2.8pt}{3.2pt}}
    \pgfusepath{stroke}}
    
\usepackage{stmaryrd}
\usepackage{wasysym}
\usepackage{multirow}
\usepackage{caption}
\usepackage{subcaption}
\usepackage{mathrsfs}
\usepackage{qtree}

\usepackage{linguex}


   %pminos do not split footnotes
% \interfootnotelinepenalty=10000 %Footnote in Laporte chapters has to be split SN


%\DeclareIndexNameFormat{default}{%
%\nameparts{#1}%
%\usebibmacro{index:name}%
%{\index[names]}%
%{\namepartfamily}%
%{\namepartgiveni}%
% {}% L1
% {}% L2
%{\namepartprefix}% generates spurious space L3
%{\namepartsuffix}% generates spurious space L4
%}

%  {\DeclareIndexNameFormat{default}{%
%     \usebibmacro{index:name}{\index[names]}{#1}{#3}{#5}{#7}}}

%\DeclareIndexNameFormat{default}{%
%  \usebibmacro{index:name}{\sindex[nom]}{#1}{#3}{#5}{#7}}

%\DeclareIndexNameFormat{default}{%
%  \usebibmacro{index:name}{\sindex[person]}{#1}{#3}{#5}{#7}}
%\DeclareIndexNameFormat{default}{%
%\nameparts{#1} \usebibmacro{index:name}{\sindex[person]]}{\namepartfamily}{‌​\namepartgiven}{\nam‌​epartprefix}{\namepa‌​rtsuffix}}

%\newcommand{\smiley}{:)}

%\renewbibmacro*{index:name}[5]{%
%\usebibmacro{index:entry}{#1}%
%{\iffieldundef{usera}{}{\thefield{usera}\actualoperator}\mkbibindexname{#2}{#3}{#4}{#5}}}

% \newcommand{\noop}[1]{}

%remove for final
%\overfullrule=1mm

\newcommand{\tobi}[2]}}
\renewcommand{\S}[1]{\tobi{#1}{\textsc{*}}}

% this volume references
% puts: [this volume]
% already defined: \citetv
%\newcommand{\citepv}[1]{(\citeauthor{#1} \citeyear*{#1} [this volume])}
\newcommand{\citealtv}[1]{\citeauthor{#1} \citeyear*{#1} [this volume]}

%parentheses around example number
\newcommand{\pref}[1]{(\ref{#1})}

% in-text examples

\newcommand{\lnex}[1]{\textit{#1}} %target lang word
\newcommand{\lnlit}[1]{(lit.: `#1')} %literal reading
\newcommand{\lnlat}[1]{(#1)} % latinization
\newcommand{\lntrans}[1]{`#1'} %translation
\newcommand{\lnexl}[2]%
{\lnex{#1}{} \lnlat{#2}} % ex with latinization
\newcommand{\lnexlat}[3]{\lnex{#1}{} \lnlat{#2}{} \lntrans{#3}} % ex with latinization and tranl.

%ch01
\newcommand{\co}[1]{\mbox{\textbf{#1}}}

%ch09

\newcommand{\cyrbulg}[1]{\begin{otherlanguage*}{bulgarian}#1\end{otherlanguage*}}


%ch10
\newcommand{\nlp}{{\small NLP}}
\newcommand{\mwe}{{\small MWE}}
\newcommand{\rae}{{\small RAE}}
\newcommand{\lvc}{{\small LVC}}
\newcommand{\pos}{{\small P}o{\small S}}
%\newcommand{\todo}[1]{ \textcolor{red}{#1} }

%\renewcommand{\labelenumi}{\theenumi}
%\ainamefmt{{vv}{ll}{, ff}{, jj}} % fullname

\newcommand{\biberror}[1]{{\color{red}#1}}

\newcommand{\osenovaitem}{--~}
   %% hyphenation points for line breaks
%% Normally, automatic hyphenation in LaTeX is very good
%% If a word is mis-hyphenated, add it to this file
%%
%% add information to TeX file before \begin{document} with:
%% %% hyphenation points for line breaks
%% Normally, automatic hyphenation in LaTeX is very good
%% If a word is mis-hyphenated, add it to this file
%%
%% add information to TeX file before \begin{document} with:
%% \include{localhyphenation}
\hyphenation{
    Beck-man
    Ngu-yen
    back-chan-nel
    back-chan-nels
    mo-not-o-nous
    ste-reo-typ-i-cal
}

\hyphenation{
    Beck-man
    Ngu-yen
    back-chan-nel
    back-chan-nels
    mo-not-o-nous
    ste-reo-typ-i-cal
}

   \boolfalse{bookcompile}
   \togglepaper[23]%%chapternumber
}{}

\begin{document}
\maketitle

\section{Introduction}\label{intro} 

In this paper we focus on combining the semantic description available for verbs in different lexical semantic resources (WordNet and FrameNet) which contain complementary semantic information \citep{Baker2009}. We discuss the aspects of universality of conceptual knowledge that enable the transfer of semantic and to a lesser extent syntactic information across resources and languages. Further, we analyse the language-specific properties of the semantic and syntactic description. We illustrate our findings in a case study on verbs of communication in English and Bulgarian.


For the purposes of the study we employ: (a) the Princeton WordNet, PWN \citep{Fellbaum1998}, and the Bulgarian WordNet \citep{koeva2021-wordnet}, and (b) FrameNet \citep{Baker1998,Ruppenhofer2016}. In particular, we centre on the information included in them and how they complement each other in terms of coverage of lexical units and with respect to the semantic and syntactic features of the description. While we use resources for English and Bulgarian, the principles adopted in this work are applicable to other languages for which a wordnet aligned with PWN is developed.%\footnote{The alignment at the synset level, usually implemented by means of synset IDs shared across wordnets enables the use of the mapping between the wordnet synsets and the FrameNet frames}.

There are several other resources relevant to our study, which provide background on the approaches for the extensive language-specific description of verb classes in comparison to developing cross-lingual and multilingual lexical and semantic resources. Further, their brief review sheds light on the possibilities for combining resources aiming at comprehensive description of lexical units. The functionalities and the additional information contained in these resources are summed up below.

VerbNet \citep{Kipper-Schuler2005,Kipper2008} provides substantial coverage of the English verb inventory and defines syntactic-semantic relations in an explicit way by means of predicate-argument structures (defined as configurations of thematic roles) with one-to-one linking to the syntactic category (type of phrase) and grammatical function (subject, object, etc.) of each argument expressed in terms of a relatively small number of syntactic frames. Selectional restrictions are defined for the thematic roles assigned to a verb’s arguments; these restrictions capture the semantic/ontological class of the nouns that express the arguments. However, although the verb classes describe the syntactic behaviour of verbs, many of the traditional thematic roles employed may be too general for an exhaustive semantic description and appropriate handling of the syntax-semantics interface, while the syntactic description is often biased towards English. Moreover, the overlap (and hence, the coverage of the existing mappings) between the WordNet synsets and the VerbNet classes is not large enough to provide sufficient data for analysis.

VerbAtlas \citep{di-fabio-etal-2019-verbatlas} is a lexical-semantic resource representing the semantic description of the verb synsets in BabelNet. BabelNet is a very large, richly populated multilingual semantic network (covering more than 500 languages) integrating lexicographic and encyclopaedic knowledge from WordNet and Wikipedia \citep{navigli-ponzetto-2010-babelnet}. Each verb synset in VerbAtlas is assigned a frame corresponding to its prototypical predicate-argument structure. Obligatory components are described using 26 semantic roles and the semantic restrictions governing their compatibility (116 types). A semantic annotation API with the frames described in it is also provided with the resource. 

Predicate Matrix \citep{lopez-de-lacalle-etal-2014-predicate} is a lexical resource resulting from the integration of several sources of predicate information: FrameNet, VerbNet, PropBank and WordNet, that have been previously aligned in Semlink.\footnote{\url{https://verbs.colorado.edu/semlink/}} \citep{Palmer2009} Predicate Matrix is compiled using advanced graph-based algorithms to extend the mapping coverage between resources. Additionally, by exploiting SemLink, new role mappings are inferred among the different predicate schemas. 

%Тhe SynSemClass lexicon\footnote{https://ufal.mff.cuni.cz/synsemclass} has marked a notable effort towards combining the rich semantic description in the Vallex dictionary family with conceptual and syntactic information from external semantic resources towards the creation of a multilingual contextually-based verb lexicon. The aim of the lexicon is to provide a resource of classes of verbs that compares their semantic roles as well as their syntactic properties \citep{Uresova2020a}. In addition, each entry is linked to FrameNet, WordNet, VerbNet, OntoNotes and PropBank, as well as the Czech VALLExample 

The alignments of WordNet and FrameNet have been proposed for different languages, such as Danish \citep{pedersen-etal-2018-danish}, Dutch \citep{Horak2008TheDO}, Korean \citep{Gilardi2018LearningTA}, among others. One of the challenges in mapping resources developed according to different methodologies is the coverage of the alignment between the units represented in them. For instance, the alignment between lexical units evoking particular frames in FrameNet and corresponding verbs in synonym sets in WordNet, achieves coverage of 30.5\% \citep{Stoyanova2019}. New methods have been proposed to increase the coverage by discovering suitable literals based on semantic relations with literals already described in semantic frames \citep{Burchardt2005}.  

Combining the semantic description of verbs from different resources has been proposed by \citet{Uresova2020a,Uresova2020b}. The result is a multilingual dictionary encoding a comprehensive description of the semantic classes of verbs and the semantic roles and syntactic properties of their arguments.\footnote{\url{https://ufal.mff.cuni.cz/synsemclass}} The project is also aimed at creating an ontology of events, processes and states, and for this purpose each dictionary entry is linked to its correspondences in FrameNet, WordNet, VerbNet, Ontonotes and PropBank, as well as the Valence Dictionary of Czech Verbs \citep{Lopatkova2016}, which represents the predicate-argument structure of each verb, its semantic class and the syntactic transformations (diatheses) in which it participates.


Our work on aligning conceptual resources relies on the notion of universality. We side with the idea that the conceptual description provided in the FrameNet frames is to a considerable degree language-independent, which makes it possible for it to be transferred and/or adapted from one language to another. We map the conceptual knowledge contained in FrameNet onto the Princeton WordNet and through it, onto the Bulgarian WordNet. We then go on to examine the feasibility of transferring the valence information %and syntactic patterns 
described for English to Bulgarian and the language-specific features that need to be addressed. The combination of semantic and syntactic information is seen as a possible way of transferring knowledge across languages (especially underresourced ones) by relying on the universality of the semantic description. 

The study is organised as follows. \sectref{sec:resources} briefly presents the lexical-se-mantic resources involved in the work as well as the corpora used for extracting examples illustrating the various syntactic realisations in English and Bulgarian. \sectref{sec:mappings} discusses the mapping of FrameNet frames onto WordNet synsets with a view to the universality of conceptual description as the main principle for cross-lingual transfer. \sectref{sec:communication} offers a detailed analysis of the semantic class of verbs of communication in terms of their conceptual structure and frame elements involved in the relevant frames. This analysis serves as a case study illustrating the main principles of universality as well as the language-specific features of syntactic realisation of frames. \sectref{sec:conclusion} draws conclusions based on the analysis and gives some directions for future work.


\section{Resources}\label{sec:resources}

Below we describe in brief the lexical semantic resources used in the study, focusing on their strengths and the ways of overcoming their possible limitations through integrating the information contained in them. We also describe the corpora serving as a source of examples, the methodology for extracting suitable examples and the annotation of frame elements and their syntactic realisation.
 
\subsection{Lexical-semantic resources}

\subsubsection{WordNet}

WordNet\footnote{\url{https://wordnet.princeton.edu/}} \citep{Miller1995,Fellbaum1998} is a large lexical database that represents comprehensively conceptual and lexical knowledge in the form of a network whose nodes denote cognitive synonyms (synsets) linked by means of a number of conceptual-semantic and lexical relations such as hypernymy, meronymy, antonymy, etc. WordNet provides extensive lexical coverage; the verbs presented in it are organised in 14,103 synsets (including verb synsets specific for Bulgarian). In this work, we use both the Princeton WordNet and the Bulgarian WordNet \citep{koeva2021-wordnet}, which are aligned at the synset level by means of unique synset identifiers. 

WordNet provides the most coarsely-grained semantic division in terms of a set of language-independent semantic primitives assigned to all the nouns and verbs in the resource. The verbs fall into 15 groups, such as verb.change (verbs describing change in terms of size, temperature, intensity, etc.), verb.cognition (verbs of mental activities or processes), verb.motion (verbs of change in the spatial position), verb.communication (verbs describing communication and information exchange), etc.\footnote{The division of the nouns and verbs into WordNet lexicographic files (reflecting the semantic primitive distinction) along with short definitions of the primitives are available at: \url{https://wordnet.princeton.edu/documentation/lexnames5wn}.}

Verb synsets are interrelated and form a hierarchical structure based on a troponymy relation which represents a manner relation and is to a great degree analogous to hypernymy; for example, in \textit{talk}.v – \textit{whisper}.v the second member of the pair refers to a particular, semantically more specified, manner of performing the action referred to by the first verb \citep{Fellbaum1999a}. 


\subsubsection{FrameNet}

FrameNet\footnote{\url{https://framenet.icsi.berkeley.edu/}}  \citep{Baker1998,Baker2008} is a lexical semantic resource which couches lexical and conceptual knowledge in the apparatus of frame semantics. Frames are conceptual structures describing types of objects, situations, or events along with their components -- frame elements \citep{Baker1998,Ruppenhofer2016}. Depending on their status, frame elements may be core, peripheral or extra-thematic \citep{Ruppenhofer2016}. We deal primarily with core frame elements, which instantiate conceptually essential components of a frame, and which in their particular configuration make a frame unique and different from other frames.

FrameNet frames are organised into a hierarchical network by means of a number of hierarchical and non-hierarchical frame-to-frame relations \citep[81--84]{Ruppenhofer2016}. Here we list the hierarchical relations, which bear most relevance to the internal structure of verb classes. These are: \FrameRelation{Inheritance} – a relationship between a parent frame and a more specific (child) frame, such that the child frame elaborates the parent frame; \FrameRelation{Uses} (also called “weak inheritance”) – a relationship between two frames where the first one makes reference in a very general kind of way to the structure of a more abstract, schematic frame; \FrameRelation{Perspective} – a relation indicating that a situation viewed as neutral may be specified by means of perspectivised frames that represent different possible points-of-view on the neutral state-of-affairs; \FrameRelation{Subframe} – a relation between a complex frame referring to sequences of states and transitions, each of which can itself be separately described as a frame, and the frames denoting these states or transitions. 


\subsection{Corpora}

\subsubsection{Semantically annotated corpora: SemCor and BulSemCor}

In order to explore the syntactic expression of the verbs and their participants, we study the usage examples available in two semantically annotated corpora -- the English SemCor and the Bulgarian semantically annotated corpus, BulSemCor, both of which are annotated with WordNet senses. 
 
SemCor (current version 3.0) \citep{miller-etal-1993-semantic,miller-etal-1994-using,landes1998} is compiled by the Princeton WordNet team and covers texts excerpted from the Brown Corpus. SemCor is supplied with POS and grammatical tagging and all open-class words (both single words and multiword expressions, as well as named entities) are semantically annotated by assigning each word a unique WordNet sense (synset ID). The corpus is the largest manually annotated corpus of this kind and amounts to a total of 226,040 sense annotations.

BulSemCor \citep{koeva-2006-bulsemcor,koeva-2011-bulsemcor} is designed according to the general methodology of the original SemCor and criteria for ensuring an appropriate coverage of contemporary general lexis. In addition to open-class words, BulSemCor includes annotation of prepositions, conjunctions, particles, pronouns and interjections; for that purpose the Bulgarian WordNet has been expanded with closed-class words \citep{koeva-2011-bulsemcor}. The size of the corpus is close to 100,000 annotated units. 

The size of the two corpora is not sufficient to provide enough evidence for many of the studied verbs so examples from other corpora have also been employed. 

\subsubsection{Bulgarian-English parallel corpus}

The Bulgarian-English Sentence- and Clause-Aligned Corpus (BulEnAC)\footnote{\url{https://dcl.bas.bg/en/resources\_list/bulenac/}} \citep{Koeva-et-al2012} is a parallel corpus of aligned Bulgarian and English sentences and clauses with annotation of the syntactic relation between clauses. The corpus contains 366,865 tokens (176,397 tokens in Bulgarian and 190,468 tokens in English). %The texts in BulEnAC cover five categories: administrative texts, fiction, journalism, science, informal texts. The Bulgarian subcorpus contains 14,667 sentences (with an average sentence length of 12.02 words), while the English subcorpus includes 15,718 sentences (with an average sentence length of 12.11 words). The average number of clauses per sentence is 1.67 for Bulgarian and 1.85 for English. 

The syntactic annotation of BulEnAC involves:
a) sentence and clause splitting;
b) annotation of the type of syntactic relation (coordinate or subordinate) between clauses;
c) marking of the elements that introduce the clause: conjunctions, complementisers, and punctuation.

BulEnAC is suitable for extracting parallel sentences that illustrate the use of particular verbs evoking the frames under study. Further, it facilitates the identification of corresponding translation equivalents within aligned clauses.

\subsubsection{The Bulgarian National Corpus}

The Bulgarian National Corpus is the largest corpus for Bulgarian: it consists of a monolingual (Bulgarian) part and 47 parallel corpora and amounts to 5.4 billion words. The Bulgarian part includes about 1.2 billion words of running text distributed in 240,000 text samples. The texts in the corpus reflect the state of the Bulgarian language predominantly in its written modality from the middle of the 20th century (1945) until the present day  \citep{Koeva2012}. The search engine developed for the exploration of the corpus allows the extraction of information according to complex grammatical criteria. We use the corpus to study the syntactic expression and the validity of the valence patterns described in \sectref{sec:communication} in addition to the examples extracted from the semantically disambiguated part of the corpus (BulSemCor).

\subsection{Motivation for combining WordNet and FrameNet}

It has long been acknowledged that combining WordNet with conceptual resources such as FrameNet results in more comprehensive semantic and syntactic representation of the lexical entries \citep{Baker2009,Schneider2012,das-etal-2014-frame}, thus expanding the possible applications of the resources for the purposes of syntactic and semantic parsing. Elaborating a bit on the discussion of the strengths and shortcomings of the different kinds of lexical semantic resources offered by \citet{Shi2005}, we may point out the following motivation for putting effort into their alignment. 

FrameNet provides a rich semantic description of the predicates using schematic representations (frames) of the configurations of “participants and props” (elements corresponding to the surrounding circumstances or other supporting facets of meaning, in the sense of \cite[7]{Ruppenhofer2016}) that define the situation described. The corpus of sentences annotated with explicit and implicit frame elements supplies empirical evidence about the syntactic realisations of semantic frames that is particularly valuable not only for linguistic generalisations about the target language (English) but also as a point of departure for making observations cross-linguistically. Besides the explicit syntactic expression, the annotators have marked non-overt but conceptually present frame elements retrievable from the immediate or the more general context (so-called null instantiations). However, while formulating ontological semantic types that classify lexical units, frames and frame elements and in the latter case denote the selectional restrictions imposed on the fillers of frame elements \citep[86]{Ruppenhofer2016}, FrameNet does not explicitly define the content of these semantic types (see \sectref{universality-restrictions}, which provides the authors' suggestions regarding that). In addition, FrameNet's coverage is limited both in terms of the lexical units included in the frames (i.e. there are lexical units pertaining to a frame that are not listed in it) and in terms of the parts of the lexicon encompassed by the system of frames, i.e. there are lexical units that cannot be described properly by the existing frames. Finally, as some of the frame elements are too finely-grained, certain generalisations across frames and frame elements might be missed.

WordNet ensures vast lexical coverage of the English lexicon structured and enriched with lexical and semantic information in the form of synset glosses, usage examples, notes on the usage or grammatical specificities, and a rich network of semantic relations. However, WordNet encodes no explicit semantic information about the participants in the situations described by the predicates and only limited information about their syntactic behaviour. 

The combination of the resources requires: (i) mapping of the units that correspond to each other in the resources, i.e. discovering the counterparts of the synsets' members among the lexical units in FrameNet and linking them to the frames they evoke; (ii) expanding the mapping by discovering new candidates in WordNet to be matched to the relevant frames.  
Such mapping procedures are discussed in \sectref{sec:mappings}. The limitations stemming from the lack of appropriate frames to describe certain parts of the lexicon need to be addressed by defining new frames.

The greater granularity of the frame elements in FrameNet (as compared with VerbNet, VerbAtlas and other resources) is handled, where necessary, by applying a shallow hierarchy derived from the hierarchical organisation of the frames and the inheritance relations defined between them \citep{Litkowski-2014-framenet}. %\footnote{\url{https://www.clres.com/clr/fetax.php}} 
Consider for instance the taxonomy of frame elements \fename{Air} > \fename{Fluid} > \fename{Theme} derived from the frame hierarchy \framename{Breathing} > \framename{Fluidic motion} > \framename{Motion} built on the frame-to-frame relation of \FrameRelation{Inheritance} between the three frames. In certain contexts and for certain tasks it may be more appropriate to make reference not to the most specific \fename{Air} but to \fename{Fluid} or even to \fename{Theme}, or vice versa. The maintaining of the different levels of granularity provides a more robust semantic description that is relatively resource- and theory-independent.

While genuinely beneficial, the mutual enrichment of WordNet and FrameNet is by no means trivial, as senses of the synsets and the lexical units that may be thought as equivalent may in fact not correspond well. The use of corpus occurrence and especially the study of annotated examples help in elucidating both theoretical and pragmatic aspects of the alignment between the resources and informs the judgments made in the course of the manual validation of the automatic assignment of frames to synsets. The case study presented in \sectref{sec:communication} may be viewed as the result of such analysis.

\section{Mapping between WordNet and FrameNet based on universal principles}\label{sec:mappings} 

Both resources have shown to be sufficiently language-independent as to provide an approximation at a description across typologically distinct languages. Both models have been transferred and adapted cross-linguistically. These include coordinated attempts to build multilingual resources or link existing independent resources through projects such as EuroWordNet \citep{Vossen2004} or Global WordNet \citep{mccrae-etal-2021-globalwordnet}, as well as Multilingual FrameNet \citep{Gilardi2018LearningTA}, among others. 

Our work expands on the notion of universality and cross-lingual applicability of lexical-semantic resources by linking the resources to each other and then transferring the language-independent (semantic and conceptual) description of English verbs in WordNet onto the Bulgarian lexical units in the Bulgarian WordNet.

\subsection{Universality of semantic inheritance relations between synsets and between frames}

The two resources have been aligned automatically by employing existing mappings (\cite{Tonelli2009}, \cite{Palmer2014}, among others) with additional implemented procedures for expansion and validation \citep{Leseva2018} and later refined \citep{Stoyanova2019,Leseva2020}; these procedures involve the mapping of FrameNet frames to WordNet synsets on the basis of the inheritance of conceptual features in hypernym trees, i.e., by assigning frames from hypernyms to hyponyms where possible and implementing a number of validation procedures based on the structural properties of the two resources, primarily the relations encoded in them. This has resulted in 13,104 automatic alignments, of which over 6,000 have been validated and corrected manually in the framework of this project and previous initiatives.

\figref{fig:01} illustrates a hypernym--hyponym pair of synsets, with the appropriate FrameNet frames assigned to them, which are themselves related by means of an inheritance relation (\framename{Cooking\_creation} being an elaboration of the mother frame \framename{Intentionally\_create}).

\begin{figure}
\includegraphics[width= \textwidth]{figures/uni_fig01.png}
\caption{Frame inheritance (\framename{Intentionally\_create} $\rightarrow$ \framename{Cooking\_creation}) as reflected in the hypernym relation (\textit{make, create} $\rightarrow$ \textit{cook}).}\label{fig:01}
\end{figure} 

\subsection{Universality of selectional restrictions}\label{universality-restrictions}

Part of the FrameNet frame elements are supplied with `semantic types' %(selectional restrictions) 
defining noun classes that narrow down the set of possible nouns that may be realised in the respective positions in the semantic frame. These semantic types are to a great degree relevant cross-linguistically, as they define ontological distinctions that underlie human cognition. To the best of our knowledge,  the list of the FrameNet types and the pertaining definitions have not been made available, but their semantic content can be intuitively construed by speakers from the relevant designations, such as Sentient, Physical object, etc.). As noted in \citet[86]{Ruppenhofer2016} most ontological semantic types ``correspond directly to synset nodes of WordNet, and can be mapped onto ontologies, e.g. Cyc or the Knowledge Graph''. The FrameNet semantic types form a semantic type hierarchy, which, however, does not necessarily correspond to that of WordNet or any other resource. Most of the frame-to-frame relations enable the propagation of the ontological semantic types of the parent frame and its frame elements down to the child frame and its frame elements \citep[99]{Ruppenhofer2016} as well as to the lexical units in the respective frame \citep[86]{Ruppenhofer2016}.
%These selectional preferences, however, are not aligned with concrete lexicalised classes, although they are relatively easy to map to a part of an ontology that would make it possible to populate them with instances. 
Using a linguistic taxonomy (moreover one implemented for numerous languages such as WordNet) to describe the selectional restrictions imposed by verbs on the nouns that fill the positions of their arguments has been proposed in different frameworks \citep{agirre-martinez2002,Koeva:2010}. While the particulars differ, the general idea is the same as the one adopted in FrameNet, i.e. to represent semantic constraints in the form of taxonomically definable classes.
%they can thus be defined as (a combination of) WordNet substructures, i.e. hypernym-hyponym trees. This type of formulation is facilitated by the hierarchical and largely cross-linguistically applicable definition of the WordNet structure.

\subsection{Universal and language-specific aspects of valence frames and syntactic realisation}

Through the alignment between frames and synsets, each verb in WordNet is associated with a number of valence patterns defined for the lexical units evoking a given frame in FrameNet. While the semantic component of the description is language-independent, the syntactic component is more language-specific as the realisation of the frame elements depends on the syntactic properties of each language. Even so, we assume that the valence patterns that underlie the syntactic expression are valid cross-linguistically to a considerable degree as they are grounded in human cognition and the conceptualisation of situations. More precisely, valence patterns describe ``the semantic and syntactic combinatory possibilities'', or valences of lexical units \citep[7]{Ruppenhofer2016}. They thus refer to the co-occurrence combinations of frame elements (both core and non-core) attested for each annotated lexical unit in the FrameNet annotated corpus.

The second, more language-specific level of syntactic description consists of the \emph{syntactic categories and grammatical functions} by which a particular frame element for a given lexical unit is expressed. Even at this level, for many (related) languages one can observe similar syntactic expression especially with respect to the participants that are selected as the subject and the object. A great degree of differentiation may be found at the level of certain grammatical peculiarities and constructions -- for instance, unlike English, Bulgarian lacks \textit{-ing} and infinitive clauses, so propositional complements will be realised as finite clauses; Bulgarian has impersonal verbs and subjectless sentences and does not make use of pleonastic subjects. Of course, there may be mismatches in the syntactic categories across languages, e.g. a certain frame element may be a direct object in one language and a prepositional object in another. Languages may also differ in terms of the overtness of syntactic information, i.e. the possibility to leave an obligatory element non-explicit (null instantiations retrievable from the context or the grammatical construction); the language-specific diatheses, constructions, word order, morphosyntactic features, etc. The inventory of means that introduce certain frame elements such as prepositions, conjunctions, wh-words, etc. may also vary across languages. 
 
The linking from the semantic level of the frame elements to the syntactic level of patterns of co-occurrence and syntactic categories in FrameNet is implemented in a straightforward manner by associating each frame element with a syntactic category and possibly a grammatical function -- e.g. subject (NP.Ext) and object (NP.Obj). 

Example \ref{ex:01} shows a partial representation of the valence patterns and the syntactic realisation of the verb \textit{teach} in the FrameNet frame \framename{Education\_teaching}.

\begin{exe}
 \ex  \label{ex:01}
     \begin{xlist}
         \ex \gll \fename{Teacher} \fename{Institution}\\
NP.Ext PP[\textit{at}] \\
         \ex \gll \fename{Teacher} \fename{Student} \fename{Subject}\\
NP.Ext NP.Obj PP[\textit{about}] \\
         \ex \gll \fename{Teacher} \fename{Student} \fename{Skill}\\
NP.Ext NP.Obj Sinterrog/VPto \\
     \end{xlist}
\end{exe}
 

To sum up, even though there may be typological cross-linguistic differences in the conceptualisation and expression of situations for many language pairs, English and Bulgarian including, there are also parallels that facilitate the transfer of information across languages at the semantic and possibly at the syntactic level. Even where direct transfer of the syntactic description is not justified, the valence patterns and the syntactic realisation lattices taken from FrameNet may serve as a point of departure in the analysis of the Bulgarian syntactic data: they help establishing what is valid or invalid in Bulgarian by comparing the syntactic properties of the Bulgarian verbs to those of their English counterparts and the example sentences in the resources.   

\section{A case study: Verbs of communication}\label{sec:communication}

Below we offer an analysis of a selection of verbs of communication as an illustration of the universal principles and the language-specific features of the adopted linguistic description. 

The domain of speech act verbs and their classification have been discussed by many authors (\cite{Wierzbicka1987}, \cite[202--211]{Levin1993}, \cite{Levin-et-al1997}, \cite{UrbanRuppenhofer2001}, \cite{Boas2010}, among others), including for Bulgarian (\cite{Nitsolova2008kompl,Penchev1998,Tisheva2000,Tisheva2004vapros,Koeva2021kompl}, among others). While previous work in this area has served to inform the current state of the linguistic knowledge about the semantic and syntactic properties of communication verbs, the analysis below is based primarily on our observations on the descriptions proposed in FrameNet for English and exploring and extending them to Bulgarian. 


First, we identify the ``basic'' frame which describes the general scenario or situation characterising the domain of communication in terms of the participants and circumstances involved and the relations among them \citep[16]{Johnson2001}. This general scenario is then elaborated in various ways in more specific frames. The semantic generalisations among such frames exhibiting different levels of abstraction and specialisation are typically cast in the form of frame-to-frame relations based on the inheritance among the semantic descriptions or parts of them. 


\begin{figure} 
\includegraphics[width=0.95\textwidth]{figures/uni_fig02.png}
\caption{The hierarchical organisation of FrameNet frames describing the verbs of communication.}\label{fig:hierarchy}
\end{figure}



The hierarchical organisation of the domain of communication verbs is presented in \figref{fig:hierarchy}.


Starting from this basic, or prototypical frame, we delve into several of the frames inheriting from it in order to show what kinds of processes are involved in the semantic specialisation and how this is reflected in the semantic description. The frames are selected based on the frequency of the verbs evoking them in the annotated data or with the objective to illustrate particular aspects of the analysis. For each such frame (including the prototypical one), we consider: (i) its semantics in terms of the frame definition, constellation of core frame elements that represent the main participants in the situation, and the relations among them, (ii) the syntactic expression of the frame elements, and (iii) the specifics of their realisation in Bulgarian as compared to English. The semantic and syntactic aspects referred to in (i) and (ii) are mostly taken for granted as represented in the FrameNet annotated corpus. In presenting each frame inheriting from the prototypical one, we do discuss how the conceptualisation of the basic frame is specialised or narrowed down and how this is reflected in the number of frame elements and the relationships among them. The main burden of our work is focused on (iii), i.e. the analysis of the syntactic expression of the frame elements as attested in the corpus compiled for Bulgarian. The valence patterns emerge from the annotated examples and are thus specified independently from the English data. The same holds for the syntactic information (syntactic function and syntactic category of the expressed frame elements). The tagset of categories is adapted from the FrameNet corpus so that the notations in the two annotated datasets are unified. 

Although there may be differences in the conceptualisation of situations across languages, we expect the semantic properties of the description to be largely shared between English and Bulgarian, as it has been shown by efforts undertaken for other languages (\sectref{intro}). Based on our preliminary observations, we also expect that at least part of the valence patterns will be relevant for both languages, i.e. the frame elements that tend to be expressed and the particular configurations in which they co-occur will be similar, even allowing for cross-lingual differences (such as the fact that Bulgarian, unlike English, is a pro-drop language). We then look at the syntactic expression of the patterns in terms of the grammatical function and the syntactic categories of the core frame elements and, where relevant, the possibility for their contextual construal (null instantiations).

We take as a point of departure the lattices of the frame elements and their syntactic realisations for certain verbs and the valence patterns of frame elements as described in the annotated FrameNet examples\footnote{\url{http://framenet.icsi.berkeley.edu/}} \citep{Burchardt2008}. %The dataset for English is supplemented with
In addition, below we also use examples from SemCor in order to illustrate the applicability of the FrameNet description independently of the annotation undertaken in the FrameNet corpus.

After analysing this information for English, we go on to observe to what extent it is applicable to Bulgarian. For this purpose, we have constructed a corpus of manually annotated examples extracted from BulSemCor and, where the number of examples is not sufficient, from the Bulgarian National Corpus. 
%For each frame, we study the verbs with sufficient number of attestations in the FrameNet annotated corpus and SemCor; the examples given for the English part of the analysis are sentences from these two corpora. Bulgarian examples are from BulSemCor and where necessary, we supplement these with manually selected examples from the Bulgarian National Corpus.

Each example sentence in the English and the Bulgarian dataset is annotated as shown in Example \ref{ex:annotation}. The English dataset consists of 93 verbs (lexical units in FrameNet) to which an appropriate communication-related frame is assigned. The verbs are aligned to 72 WordNet synsets. %using the WordNet to FrameNet mappings. %as outlined in \sectref{sec:mappings}. 
Each verb is supplied with a number of examples from the FrameNet corpus illustrating its valence patterns; the dataset contains a total of 4,525 illustration examples representing 863 different valence patterns. The annotation of each sentence in the Berkeley FrameNet corpus includes explicit annotation of the target word (in this case a verb) and the syntactic realisation of the frame elements.

The Bulgarian dataset covers 112 communication verbs (including aspectual pairs) across 63 WordNet synsets. As the corpus of annotated examples for Bulgarian is still work in progress, it is considerably smaller than the one for English: it contains 890 annotated sentences representing 136 different patterns. The annotation consists in labelling the sentence components with the frame elements they realise in a way consistent with the annotation in the Berkeley FrameNet. 

\begin{exe}
\ex \label{ex:annotation}
\begin{xlist}
\ex 
\glt FrameNet description: \textit{ask}.v `say something in order to obtain an answer or some information from someone', frame: \framename{Questioning}\\
WordNet alignment: \{\textit{ask}:4\} `address a question to and expect an answer from', synset ID: eng-30-00897746-v\\
BulNet alignment: \{\textit{питам}:2, \textit{попитвам}:1, \textit{попитам}:1, \textit{запитвам}:3, \textit{запитам}:3\}, synset ID: eng-30-00897746-v
\ex An adapted example from the FrameNet corpus with the relevant pattern:\\ 
{[\textit{They}]}$_{\feinsub{Com}}$  \textit{\textbf{ASKED}} [\textit{Rubbie}]$_{\feinsub{Addr}}$ [\textit{what she ate}]$_{\feinsub{Msg}}$.\\
\textbf{{[NP.Ext]}$_{\feinsub{Com}}$ VERB {[NP.Obj]}$_{\feinsub{Addr}}$ {[Sinterrog]}$_{\feinsub{Msg}}$}
\ex An annotated example from BulSemCor with the relevant pattern:\\
\gll[\textit{Престъпникът}]$_{\feinsub{Com}}$ \textit{\textbf{ПОПИТАЛ}} [\textit{полицая}]$_{\feinsub{Addr}}$ [\textit{дали може да си купи цигари}]$_{\feinsub{Msg}}$.\\
Criminal-\textsc{def} asked policeman-\textsc{def} {whether he could buy cigarettes}.\\
\glt `The criminal asked the policeman whether he could buy cigarettes.'\\
\textbf{{[NP.Ext]}$_{\feinsub{Com}}$ {[NP.Obj]}$_{\feinsub{Addr}}$ {[Sinterrog]}$_{\feinsub{Msg}}$} 
\end{xlist}
\end{exe} 


% \subsection{Hierarchical organisation of frames for verbs of communication}


%Here we study the verbs and frames that show important features of verbs of communication, illustrate their organisation and have sufficient number of attestations in the resources in order to allow reliable generalisations regarding the usage patterns they exhibit in text.

\subsection{The prototypical frame: \framename{Communication}}

As noted by \citet[108]{Johnson2001}, the frames in the domain of communication describe “verbal communication between people and inherit structure and frame elements from the higher-level frame \framename{Communication}”.  \framename{Communication} is thus the prototypical frame that represents the basic conceptual structure of the activity of communication as a configuration of five main interacting frame elements. This basic structure will be further elaborated (narrowed down, profiled or otherwise specialised) in the frames that inherit it.\footnote{By “inherit” we mean the relationships between the more general and the more specific frames between which the following hierarchical frame-to-frame relations hold: \FrameRelation{Inheritance}, \FrameRelation{Using}, \FrameRelation{Perspectivises}, \FrameRelation{Subframe}.} 

\begin{description}[font=\normalfont]
\item[Definition of the frame \framename{Communication}:] A \fename{Communicator} conveys a \fename{Message} to an \fename{Addressee}; the \fename{Topic} and \fename{Medium} of the communication may also be expressed. 
\end{description}

As described in the definition, the \framename{Communication} frame does not itself involve specification of the method of communication (speech, writing, gesture, etc.) but only the fact of it. The frames that inherit \framename{Communication} can add elaboration to the general idea in several ways: 

\begin{enumerate}[label=(\roman*)]
\item by specifying the \fename{Medium} in a variety of ways, such as the particular language (\textit{in French, in Russian}), or the physical entity or channel, e.g. a medium, technology, form, etc. (\textit{on the radio, in a letter, through the Messenger, in writing}). 
\item by specifying the manner of verbal communication according to various criteria such as loudness (e.g. \textit{shout}.v, \textit{whisper}.v); volubility and/or mood (e.g., \textit{babble}.v, \textit{rant}.v), distinctness (e.g., \textit{slur}.v, \textit{stutter}.v, \textit{mutter}.v), among many others;
\item specialisation may also mean that the more concrete frames inherit only part of the \framename{Communication} frame elements or do not inherit them in a straightforward manner. For example, \framename{Judgment\_communication} (which inherits from \framename{Statement}, in turn inheriting from \framename{Communication} according to the \FrameRelation{Using} relation, see \figref{fig:hierarchy} above) reinterprets the frame element \fename{Message} as a judgement on an \fename{Evaluee} according to a \fename{Reason}.
\end{enumerate}

The prototypical and the inheriting frames might exhibit a different construal of the relationship between certain frame elements. For instance (as pointed out in the description of \framename{Communication}), in the frame \framename{Chatting}, the \fename{Communicator} and \fename{Addressee} alternate their roles, and are often expressed by a single, plural NP, i.e. the relationship between them is not asymmetrical but reciprocal as they participate in the situation in the same way.

Another aspect of specialisation is the inability for overt expression of all the frame elements \citep[16]{Johnson2001}. For example, the lexical units \textit{talk}.v and \textit{speak}.v in the \framename{Statement} frame (which inherits \framename{Communication} according to the \FrameRelation{Using} relation) usually block the overt expression of \fename{Message}, although its existence is implied at the conceptual level (in their meaning). This is shown by the fact that in the annotated examples available for the two verbs the frame element \fename{Topic} is much more frequently expressed than \fename{Message}, although it is dependent on it (the topic characterises the message).

Another kind of elaboration is represented by the incorporation of frame elements \citep[164--165]{Jackendoff1990} whereby a certain frame element is integrated in the meaning of a verb as a result of which this frame element is usually left unexpressed \citep[30]{Ruppenhofer2016}. In the domain of \framename{Communication} the frame \framename{Communication\_means} describes situations that specify the concrete means with the aid of which communication takes place; the various \fename{Means} are thus incorporated in the meaning of the respective verbs, e.g. \textit{fax}.v, \textit{telephone}.v, \textit{email}.v.

The frame \framename{Communication} is evoked by a small number of verbs -- \textit{communicate}.v, \textit{convey}.v, \textit{indicate}.v, \textit{share}.v. Although pertaining to the prototypical frame, these verbs are not the most frequent ones associated with the activity of communicating, which are in fact described in more elaborate frames.  


\subsubsection{Prototypical frame elements in the domain of communication}

Below we present the prototypical frame elements of the \framename{Communication} frame as defined in FrameNet.

\begin{description}[font=\normalfont]\sloppy
\item[\fename{Communicator} (Semantic type: Sentient)] The sentient entity that uses language in the written or spoken modality to convey a \fename{Message} to the \fename{Addres\-see}. 
\item[\fename{Medium}] The physical or abstract setting in which the \fename{Message} is conveyed.
\item[\fename{Message} (Semantic type: Message)] A proposition or set of propositions that the \fename{Communicator} wants the \fename{Addressee} to believe or take for granted; in other words it is the content which is communicated. 
\item[\fename{Topic}]  The subject matter to which the \fename{Message} pertains.  It is thus a property of \fename{Message} \citep[17]{Johnson2001} and as a result its syntactic expression is also predetermined by the expression of the \fename{Message}.
\item[\fename{Addressee} (Semantic type: Sentient)]  The \fename{Addressee} is typically a person or organisation, etc. that receives a \fename{Message} from the \fename{Communicator}.\footnote{In the FrameNet frame \framename{Communication} the \fename{Addressee} is specified as a non-core element. However, we consider it is nonetheless implied in all examples from the FrameNet annotated corpus and thus analyse it in the set of prototypical frame elements.}
\end{description}

In the remainder of the chapter the data in the annotated corpora that are subject to analysis are organised as follows. We first show and discuss how each of the considered frame elements is realised at the level of the individual verbs evoking a given frame (the odd-numbered tables). This kind of presentation allows us to observe the expression of each frame element for each verb and the differences among verbs in the same frame. The data shown in the pairs of odd-numbered tables enable the comparison between English and Bulgarian and help in drawing conclusions about the correspondences and differences in the syntactic realisation between the two languages. These tables, however, do not represent the configurations of frame elements that actually occur in the annotated corpora. To illustrate those, we give a summarised list of the most characteristic valence patterns for each frame (i.e. the best-represented patterns in terms of numbers of examples) and the verbs that are observed in these configurations in the two languages (the even-numbered tables). The information in the subsequent odd- and even-numbered tables is thus complementary. Due to the currently insufficient number of examples even for many English verbs, we represent the valence patterns as an aggregate of the valence patterns for all verb,\footnote{In theory, the differences among the individual verbs are lost in this way, but since we do not have at our disposal large samples of annotated data for each verb, in practice, this is not relevant as the sparseness of data prevents us from making such detailed observations.} thus obtaining what we call generalised valence patterns. These give us an overall idea of the distribution of valence patterns across verbs and a point of departure for a more in-depth evidence based analysis.\footnote{The numbers in the tables for English are based on a version of the Berkeley FrameNet obtained in XML format in 2019.}

\subsubsection{Syntactic realisation of the \framename{Communication} frame elements}

The syntactic expression of the basic configuration of frame elements in the \framename{Communication} frame is exemplified in \tabref{tbl:communication-synt}.

\begin{table}
\fittable{\begin{tabular}{l rrrrrrrrr}
\lsptoprule
 & NP.Ext & NP.Obj & PP & AVP & NI & Clause & Quote & Other & Total\\
\midrule
\textit{communicate} &  &  &  &  &  &  &  &  & \\  
\fename{Communicator} & 39 &  &  &  & 5 &  &  &  & 44\\ 
\fename{Addressee} &  &  & 27 &  & 16 &  &  & 1 & 44\\ 
\fename{Message} & 3 & 23 &  &  & 14 & 1 &  &  & 41\\ 
\fename{Topic} &  & 1 & 3 &  & 1 &  &  & 1 & 6\\ 
\fename{Medium} & 2 &  & 3 &  &  &  &  & 1 & 6\\ 

\midrule
\textit{indicate} &  &  &  &  &  &  &  &  & \\  
\fename{Communicator} & 7 &  &  &  &  &  &  &  & 7\\ 
\fename{Addressee} &  &  &  &  & 8 &  &  &  & 8\\ 
\fename{Message} &  & 3 &  &  &  & 6 &  &  & 9\\ 
\fename{Medium} & 4 &  &  &  &  &  &  &  & 4\\ 

\midrule
%\textit{signal} &  &  &  &  &  &  &  &  & \\  
%\fename{Communicator} & 2 &  &  &  &  &  &  &  & 2\\ 
%\fename{Addressee} &  &  & 1 &  &  &  &  &  & 1\\ 
%\fename{Message} &  & 1 &  &  &  & 1 &  &  & 2\\ 
% \midrule
\textit{say} &  &  &  &  &  &  &  &  & \\  
\fename{Communicator} & 5 &  &  &  & 6 &  &  &  & 11\\ 
\fename{Medium} & 5 &  &  &  & 1 &  &  &  & 6\\ 
\fename{Message} & 6 &  &  &  &  & 9 & 2 &  & 17\\ 
\fename{Topic} &  &  & 1 &  & 1 &  &  &  & 2\\ 
%\textit{share} &  &  &  &  &  &  &  &  & \\  
%\fename{Communicator} & 1 &  &  &  &  &  &  &  & 1\\ 
%\fename{Message}&  & 1 &  &  &  &  &  &  & 1\\ 
%\fename{Topic} &  &  & 1 &  &  &  &  &  & 1\\ 
\lspbottomrule
 \end{tabular}}
 \caption{Syntactic expression of the \framename{Communication} frame elements of selected FrameNet lexical units. } 
    \label{tbl:communication-synt}
 \end{table}


\fename{Communicator} is expressed as the external argument, i.e. as a subject of the respective sentence or clause; as it is a sentient entity, it is realised as an NP. In a number of cases the frame element is realised as a definite null instantiation (DNI), i.e. it is retrievable from the previous context, or as a constructional null instantiation (CNI), where it is the grammatical construction that allows it to remain non-overt, e.g. in passive or infinitive clauses, etc.

Here and below, unless the distinction is specifically relevant, we consider INIs (indefinite null instantiations), CNIs (constructional null instantiations) and DNIs (definite null instantiations) as one category -- NI (null instantiations), together with the category INC (incorporated frame element), see \citet{petruck-2019-meaning}. The null instantiations are a very interesting category that merits a separate in-depth study. In particular, they may be considered as exponents of distinct properties, may stand for different syntactic categories and constituents with different grammatical functions, and respectively -- may participate in different valence patterns. However, the distinction among them is not trivial and especially the one between DNIs and INIs may require a broader context to be interpreted accurately. In addition, as this has not been the focus of study, sufficient number of examples and broad enough context has not been provided in the Bulgarian data.\footnote{The category `Other’ encompasses examples where a frame element is otherwise expressed. Due to the limited number of such instances, we omit them here.} 

With the verbs in this frame, \fename{Message} is typically realised as an object NP, as a complement clause (Example \ref{ex:02comm:a}) or as a quote. Quotes represent the content of the \fename{Message} as directly stated by the \fename{Communicator} in his or her own words, while clauses denote it as being retold by someone (such as in reported speech). A \fename{Message} realised as an NP constitute a nominalisation which rephrases its content in a more concise way or as a generalised idea. In about a third of the examples available for \textit{communicate}.v the \fename{Message} is annotated as an indefinite null instantiation (INI). This means that the verb is used intransitively: the \fename{Message} remains syntactically unexpressed and receives a certain typical interpretation without a specific discourse referent \citep{Ruppenhofer2016}, as in Example \ref{ex:02comm:b}. The INIs correspond to the activity use of certain types of verbs where the object remains implicit \citep{VanValinLaPolla1997}. 


The FrameNet examples show that \fename{Topic} is rarely expressed, with only several instances in the FrameNet corpus even for \textit{communicate}.v. Extrapolating from examples from other sources and the definition of the frame element, we may conclude that the \fename{Topic} is usually expressed as a prepositional phrase headed by the preposition \textit{about}. An alternative way of realising the \fename{Topic} is as a modifier of a noun expressing the \fename{Message} (Example \ref{ex:02comm:c}); such cases corroborate syntactically its semantic dependence on the \fename{Message} communicated. In the absence of an overt \fename{Message}, the \fename{Topic} may be expressed as an independent phrase (Example \ref{ex:02comm:d}); this is one of the typical patterns of its realisation as attested in the more specific communication frames.

\fename{Medium} is expressed either as a prepositional phrase, or as the subject in the case of a non-overt \fename{Communicator}.

\fename{Addressee} is either realised as a prepositional phrase or is left unexpressed, although its presence is always required conceptually as every act of communication is addressed to someone. Predominantly, the non-overt \fename{Addressee} frame elements are indefinite null instantiations (INI).



\begin{exe}
\ex \label{ex:02comm}
\begin{xlist}
\ex  \label{ex:02comm:a}
{[\textit{Iranian officials}]}$_{\feinsub{Com}}$ \textit{\textbf{INDICATE}} [\textit{that Iran would honor its safeguards agreement with the IAEA}]$_{\feinsub{Msg}}$ [\_]$_{\feinsub{Addr-INI}}$.
\ex \label{ex:02comm:b}
{[\textit{They}]}$_{\feinsub{Com}}$ \textit{can easily \textbf{COMMUNICATE}} [\_]$_{\feinsub{Msg-INI}}$ [\textit{with \\one another}]$_{\feinsub{Addr}}$.
\ex \label{ex:02comm:c}
{[\textit{The letter}]}$_{\feinsub{Med}}$ \textit{\textbf{COMMUNICATED}} [\textit{nothing}]$_{\feinsub{Msg}}$ [\textit{of her \\pleasure and love}]$_{\feinsub{Top}}$.
\ex \label{ex:02comm:d}
{[\textit{I}]}$_{\feinsub{Com}}$ \textit{\textbf{COMMUNICATED}} [\textit{with the Minister}]$_{\feinsub{Addr}}$ [\textit{on \\that issue}]$_{\feinsub{Top}}$.
\end{xlist}
\end{exe}



\subsubsection{\framename{Communication} valence patterns}

\framename{Communication} valence patterns are presented in \tabref{tbl:communication-valence}.

The most common valence pattern found in the data is represented as an expressed subject NP \fename{Communicator}, an object NP \fename{Message} and an \fename{Addressee} PP. The \fename{Message} is usually expressed and when it is not -- the \fename{Topic} may be realised (Example \ref{ex:02comm}). Due to the small number of examples, this last observation is not included in the table, but it is supported by the expression of the relevant frame element in the more specific frames. 



\begin{table}
    \begin{tabular}{lrl}
\lsptoprule
         Pattern  & \#  & verbs \\\midrule
{}[NP.Ext]$_{\feinsub{Com}}$ [PP]$_{\feinsub{Addr}}$ [NP]$_{\feinsub{Msg}}$  & 11 & \textit{communicate, signal} \\
{}[NP.Ext]$_{\feinsub{Com}}$ [PP]$_{\feinsub{Addr}}$ {[\_]}$_{\feinsub{Msg-INI}}$  & 7 & \textit{communicate} \\
{}[NP.Ext]$_{\feinsub{Msg}}$ {[\_]}$_{\feinsub{Com-CNI}}$ [Clause]$_{\feinsub{Msg}}$  & 5 & \textit{say} \\
{}[NP.Ext]$_{\feinsub{Com}}$ {[\_]}$_{\feinsub{Addr-INI}}$ [NP]$_{\feinsub{Msg}}$  & 5 & \textit{communicate} \\
{}[NP.Ext]$_{\feinsub{Com}}$ {[\_]}$_{\feinsub{Addr-INI}}$ {[\_]}$_{\feinsub{Msg-INI}}$  & 4 & \textit{communicate} \\
{}[NP.Ext]$_{\feinsub{Com}}$ [Clause]$_{\feinsub{Msg}}$  & 4 & \textit{indicate, say, signal} \\
%[NP.Ext]$_{\feinsub{Med}}$  & 3 & \textit{say} \\
{}[NP.Ext]$_{\feinsub{Msg}}$ [PP]$_{\feinsub{Addr}}$ {[\_]}$_{\feinsub{Com-CNI}}$  & 3 & \textit{communicate} \\
{}[NP.Ext]$_{\feinsub{Com}}$ {[\_]}$_{\feinsub{Addr-DNI}}$ [NP]$_{\feinsub{Msg}}$  & 3 & \textit{communicate, indicate} \\
%[NP.Ext]$_{\feinsub{Com}}$ {[\_]}$_{\feinsub{Addr-DNI}}$  & 2 & \textit{indicate} \\
%[NP.Ext]$_{\feinsub{Com}}$  & 2 & \textit{say} \\
{}[NP.Ext]$_{\feinsub{Com}}$ [PP]$_{\feinsub{Addr}}$ [NP]$_{\feinsub{Msg}}$ [PP]$_{\feinsub{Top}}$  & 2 & \textit{communicate} \\
{}[NP.Ext]$_{\feinsub{Med}}$ {[\_]}$_{\feinsub{Addr-INI}}$ [Clause]$_{\feinsub{Msg}}$  & 2 & \textit{indicate} \\
\lspbottomrule
\end{tabular}
    \caption{FrameNet valence patterns of \framename{Communication} verbs, their frequency in the FrameNet corpus and the verbs they appear with.}
    \label{tbl:communication-valence}
\end{table} 



\subsubsection{Syntactic realisation of the \framename{Communication} frame in Bulgarian}

%In Bulgarian, due to the small number of examples in BulSemCor a small excerpt of parallel examples was collected for the translation pair \textit{communicate} -- \textit{съобщавам}. 

The core frame elements are expressed in a similar way as in English: the \fename{Communicator} is realised as a subject, the \fename{Message} is an NP object or more rarely (although varying from verb to verb) a complement clause or a quote; if overt, the \fename{Addressee} is expressed as a prepositional phrase. The \fename{Topic} is syntactically explicit in about 20\% of the cases and, similarly to English, is realised as either a prepositional phrase that modifies a \fename{Message} head noun (Example \ref{ex:02commbg:a}) or independently in the absence of an overt \fename{Message} (Example \ref{ex:02commbg:b}); the number of examples is too small to make definitive conclusions, but both languages support this observation. 


\begin{exe}
\ex \label{ex:02commbg}
\begin{xlist}
\ex  \label{ex:02commbg:a}
\gll {[\textit{Те}]}$_{\feinsub{Com}}$ \textit{\textbf{СЪОБЩАВАТ}} [\textit{съответната} \textit{информация}]$_{\feinsub{Msg}}$ [\textit{за} \textit{дейността} \textit{си}]$_{\feinsub{Top}}$.\\
They communicate relevant information about activity-\textsc{def} REFL.\\
\glt
`They communicate relevant information about their activity.'
\ex  \label{ex:02commbg:b}
\gll {[\textit{Те}]}$_{\feinsub{Com}}$ \textit{\textbf{СЪОБЩАВАТ}} {[\_]}$_{\feinsub{Msg-INI}}$ [\textit{за} \textit{пристигането} \textit{си} \textit{на} \textit{гарата}]$_{\feinsub{Top}}$.\\
They communicate {} about arrival-\textsc{def} REFL at station-\textsc{def}.\\
\glt `They communicate about their arrival at the station.'
\ex  \label{ex:02commbg:c}
\gll {[\textit{Те}]}$_{\feinsub{Com}}$ \textit{\textbf{СЪОБЩАВАТ}} [\textit{на} \textit{Комисията}]$_{\feinsub{Addr}}$ [\textit{текста} \textit{на} \textit{разпоредбите}]$_{\feinsub{Msg}}$.\\
They communicate to Commission-\textsc{def} text-\textsc{def} of measures-\textsc{def}.\\
\glt `They communicate to the Commission the text of the measures.'
\ex  \label{ex:02commbg:d}
\gll {[\textit{Органите}]}$_{\feinsub{Com}}$ \textit{\textbf{СЪОБЩАВАТ}} [\textit{цялата} \textit{съществена} \textit{информация}]$_{\feinsub{Msg}}$ {[\_]}$_{\feinsub{Addr}}$.\\
Authorities-\textsc{def} communicate all essential information. {}\\
\glt `The authorities communicate all essential information.'
\ex \label{ex:02commbg:e}
\gll {[\textit{Страните}]}$_{\feinsub{Com}}$ \textit{\textbf{ПОСОЧВАТ}},  [\textit{че} \textit{информацията} \textit{не може} \textit{да} \textit{бъде} \textit{резюмирана}]$_{\feinsub{Msg}}$ {[\_]}$_{\feinsub{Addr}}$.\\
Parties-\textsc{def} indicate that information-\textsc{def} cannot to be summarised. {}\\
\glt `The parties indicate that the information cannot be summarised.'
\end{xlist}
\end{exe}

The syntactic realisation of the \framename{Communication} frame elements in Bulgarian is shown in \tabref{tbl:communication-synt-bg}.\footnote{In the Bulgarian annotated data the verbs are assigned a WordNet sense, so the corresponding Princeton WordNet synset serves as an English translation equivalent. As this information is not available to the readers, henceforth we have provided translation equivalents for the Bulgarian verbs.}
 
\begin{table}
\centering\footnotesize
\begin{tabular}{l rrrrrrrrr}
\lsptoprule
 & NP.Ext & NP.Obj & PP & AVP & NI & Clause & Quote & Other & Total\\
\midrule
%\multicolumn{10}{l}{\textit{общувам} }\\  
%\fename{Communicator} & 31 &  &  &  &  &  &  &  & 31\\ 
%\fename{Message} &  &  &  &  & 32 &  &  &  & 32\\ 
%\fename{Addressee} &  &  & 28 &  & 4 &  &  &  & 32\\ 
% \midrule
\multicolumn{10}{l}{\textit{споделям\slash споделя}}\\
`share'\\
\fename{Communicator} & 14 &  &  &  &  &  &  &  & 14\\ 
\fename{Message} &  & 11 &  &  & 2 & 1 &  &  & 14\\ 
\fename{Addressee} &  &  & 12 &  & 2 &  &  &  & 14\\ 
\fename{Topic} &  &  & 1 &  &  &  &  &  & 1\\ 

\midrule
\multicolumn{10}{l}{\textit{съобщавам\slash съобщя} }\\ 
`communicate'\\
\fename{Communicator} & 29 &  &  &  &  &  &  &  & 29\\ 
\fename{Message} &  & 22 &  &  & 6 & 2 &  &  & 30\\ 
\fename{Addressee} &  &  & 22 &  & 8 &  &  &  & 30\\ 
\fename{Medium} &  &  & 1 &  &  &  &  &  & 1\\ 
\fename{Topic} &  & 1 & 6 &  &  &  &  &  & 7\\ 

\midrule
\multicolumn{10}{l}{\textit{предавам\slash предам} }\\
`convey'\\
\fename{Communicator} & 48 &  &  &  &  &  &  &  & 48\\ 
\fename{Message} & 3 & 42 &  &  & 1 & 1 & 1 &  & 48\\ 
\fename{Addressee} &  &  & 28 &  & 19 &  &  &  & 47\\ 

\lspbottomrule
 \end{tabular}
 \caption{Syntactic expression of the \framename{Communication} frame elements in Bulgarian.} 
    \label{tbl:communication-synt-bg}
 \end{table}


The valence patterns in Bulgarian (\tabref{tbl:communication-valence-bg}) show similar preferences for the co-occurrence of frame elements; with both \fename{Message} and \fename{Addressee} expressed (Example \ref{ex:02commbg:c}) or with a realised \fename{Message} and a non-overt \fename{Addressee} (Example \ref{ex:02commbg:d}).

\begin{table}
\begin{tabularx}{\textwidth}{lrQ}
\lsptoprule
         Pattern  & \#  & verbs \\\midrule
{[NP.Ext]}$_{\feinsub{Com}}$ {[NP.Obj]}$_{\feinsub{Msg}}$ {[PP]}$_{\feinsub{Addr}}$ & 50 & \textit{предавам\slash предам, споделям\slash споделя, съобщавам\slash съобщя}\\
%{[NP.Ext]}$_{\feinsub{Com}}$ {[PP]}$_{\feinsub{Addr}}$ {[\_]}$_{\feinsub{Msg-INI}}$ & 27 & \textit{общувам}
%, предавам\slash предам
%\\
{[NP.Ext]}$_{\feinsub{Com}}$ {[NP.Obj]}$_{\feinsub{Msg}}$ {[\_]}$_{\feinsub{Addr-INI}}$ & 13 & \textit{предавам\slash предам, съобщавам\slash съобщя}\\
{[NP.Ext]}$_{\feinsub{Com}}$ {[NP.Obj]}$_{\feinsub{Msg}}$  {[\_]}$_{\feinsub{Addr-DNI}}$ & 9 & \textit{споделям\slash споделя, предавам\slash предам}\\
{[NP.Ext]}$_{\feinsub{Com}}$ {[PP]}$_{\feinsub{Addr}}$ {[PP]}$_{\feinsub{Top}}$   {[\_]}$_{\feinsub{Msg-INI}}$ & 4 & \textit{споделям\slash споделя, съобщавам\slash съобщя}\\
{[NP.Ext]}$_{\feinsub{Com}}$ {[PP]}$_{\feinsub{Addr}}$ {[\_]}$_{\feinsub{Msg-DNI}}$ & 3 & \textit{споделям\slash споделя, съобщавам\slash съобщя}\\
%{[NP.Ext]}$_{\feinsub{Com}}$ {[\_]}$_{\feinsub{Addr-DNI}}$ {[\_]}$_{\feinsub{Msg-INI}}$ & 3 & \textit{общувам}\\
{[NP.Ext]}$_{\feinsub{Com}}$ {[Clause]}$_{\feinsub{Msg}}$  {[\_]}$_{\feinsub{Addr-INI}}$ & 2 & \textit{съобщавам\slash съобщя}\\
%{[NP.Ext]}$_{\feinsub{Com}}$ {[PP]}$_{\feinsub{Addr}}$ {[Quote]}$_{\feinsub{Msg}}$ & 1 & \textit{предавам\slash предам}\\
%{[NP.Ext]}$_{\feinsub{Com}}$ {[PP]}$_{\feinsub{Top}}$ {[\_]}$_{\feinsub{Addr-INI}}$ {[\_]}$_{\feinsub{Msg-INI}}$ & 1 & \textit{съобщавам\slash съобщя}\\
%{[NP.Ext]}$_{\feinsub{Com}}$ {[NP.Obj]}$_{\feinsub{Msg}}$ {[PP]}$_{\feinsub{Addr}}$ {[PP]}$_{\feinsub{MEDIUM}}$ & 1 & \textit{съобщавам\slash съобщя}\\
%{[NP.Ext]}$_{\feinsub{Com}}$ {[NP.Obj]}$_{\feinsub{Msg}}$ {[PP]}$_{\feinsub{Top}}$ {[\_]}$_{\feinsub{Addr-INI}}$ & 1 & \textit{съобщавам\slash съобщя}\\
{[NP.Ext]}$_{\feinsub{Com}}$ {[NP.Obj]}$_{\feinsub{Msg}}$ {[PP]}$_{\feinsub{Addr}}$ {[PP]}$_{\feinsub{Top}}$ & 1 & \textit{съобщавам\slash съобщя}\\
\lspbottomrule
\end{tabularx}
    \caption{FrameNet valence patterns of \framename{Communication} verbs, their frequency in the Bulgarian dataset and the verbs they appear with. 
    English translation equivalents: \textit{предавам\slash предам} `convey', \textit{споделям\slash споделя} `share', \textit{съобщавам\slash съобщя} `communicate'.}
    \label{tbl:communication-valence-bg}
\end{table} 

In the Bulgarian data we have found only rare instances where there is an expressed \fename{Addressee} with non-overt \fename{Message} or \fename{Topic}, but this observation needs further corroboration from the data for this frame as well as for other related frames.

The \fename{Message} can also be expressed as a quote or a clausal complement; however, as Bulgarian lacks infinitives and \textit{-ing} clauses, clausal complements are realised as finite clauses (Example \ref{ex:02commbg:e}). 




\subsection{Frame \framename{Communication\_manner}}

\begin{description}[font=\normalfont]\sloppy
\item[Definition of the frame \framename{Communication\_manner}:] The words in this frame describe \fename{Manners} of verbal communication. Core frame elements: \fename{Speaker}, \fename{Message}, \fename{Topic}, \fename{Addressee}.
\end{description}

The \fename{Speaker} is a specific type of \fename{Communicator} who uses his or her voice to produce the \fename{Message}. Thus, apart from being a sentient being, it needs to be able to produce speech, e.g. is typically a person (Example \ref{ex:03manner:a}). The type of communication involves characteristics of individual organisms, so organisations are not typically realised as \fename{Speakers}, but groups of people can be (Example \ref{ex:03manner:b}).  


In particular, the verbs in the \framename{Communication\_manner} frame describe various manners of speaking or vocalising whereby a \fename{Speaker} conveys a \fename{Message} to the \fename{Addressee}. The focus is on the specifics of the articulation or vocalisation such as clarity, speed, loudness, etc. Thus, the \fename{Manner} of the communication is incorporated in the lexical meaning of the verb, e.g. \textit{whisper}.v `speak very softly using one's breath', \textit{babble}.v `talk rapidly and continuously', etc.; the \fename{Manner} can appear overtly when expressing additional manner meaning than the one incorporated by the verb (Example \ref{ex:03manner:e}). 

The \fename{Medium} of communication is peripheral to the conceptualisation of the frame and thus has a non-core status. 

The remaining core frame elements, i.e. \fename{Message} and \fename{Topic}, have the same specifics as in the \framename{Communication} frame.

\subsubsection{Syntactic realisation of the \framename{Communication\_manner} frame elements}
\largerpage

The syntactic expression of the basic configuration of frame elements in the \framename{Communication\_manner} frame is similar to the one in the \framename{Communication} frame, but there are differences that we point out below. Like \fename{Communicator}, the \fename{Speaker} is the external argument and is realised as the subject, which, under some contextually or constructionally grounded circumstances can be left implicit. 

Similarly to the same frame element in the \framename{Communication} frame, the \fename{Message} can be expressed as a subordinate clause (Example \ref{ex:03manner:c}), a quoted expression (Example \ref{ex:03manner:d}), or an NP object  that generalises over the type of information (Example \ref{ex:03manner:a}). In some cases the \fename{Message} can be unexpressed (Example \ref{ex:03manner:e}).

The \fename{Topic} is typically expressed as a prepositional phrase complement headed by `about’ (Example \ref{ex:03manner:f}).
%; this trend on a smaller scale due to the smaller number of examples is visible in the \framename{Communication} frame, where other types of expression were found as well; 
%In quite a few of the examples, this frame element is annotated as a null instantiation, since (as shown above) it is dependent on the \fename{Message} and is usually left implicit in its presence. 
An alternative type of pattern is for it to be left implicit (a null instantiation), especially in the presence of a \fename{Message}. As shown above, the two frame elements co-occur overtly primarily as an NP and a PP, where the \fename{Topic} PP should be treated as a modifier of the \fename{Message} NP (Example \ref{ex:03manner:g}).

The \fename{Addressee} is typically left non-overt but is always implied; otherwise it is expressed as a prepositional phrase (Examples \ref{ex:03manner:a}, \ref{ex:03manner:d}, \ref{ex:03manner:e}).

Among the verbs in this frame, certain differences may also be found. For instance, \textit{rave}.v and \textit{rant}.v tend to express overtly the \fename{Topic} more often than the \fename{Message} as compared with the purely manner verbs, which give preference to the \fename{Message} itself.


\begin{exe}
\ex \label{ex:03manner}
\begin{xlist}
\ex  \label{ex:03manner:a}
\glt {[\textit{Ann}]}$_{\feinsub{Com}}$ \textit{\textbf{WHISPERED}} [\textit{the question}]$_{\feinsub{Msg}}$   {[\textit{to Harry}]}$_{\feinsub{Addr}}$. 
\ex  \label{ex:03manner:b}
{[\textit{The crowd}]}$_{\feinsub{Com}}$ \textit{\textbf{CHANTED}} [\textit{my name}]$_{\feinsub{Msg}}$ {[\_]}$_{\feinsub{Addr-INI}}$.
\ex  \label{ex:03manner:c}
{[\textit{He}]}$_{\feinsub{Com}}$ \textit{\textbf{MUMBLED}} [\textit{that he was in a state of shock}]$_{\feinsub{Msg}}$   {[\_]}$_{\feinsub{Addr-INI}}$.
\ex  \label{ex:03manner:d}
{[\textit{`Change of plan,'}]}$_{\feinsub{Msg}}$ [\textit{Peter}]$_{\feinsub{Com}}$ \textit{\textbf{SHOUTED OUT}} {[\textit{to Kelly}]}$_{\feinsub{Addr}}$.
\ex  \label{ex:03manner:e}
{[\textit{I}]}$_{\feinsub{Com}}$ \textit{was \textbf{SINGING}}  [\_]$_{\feinsub{Msg-INI}}$ [\textit{happily}]$_{\feinsub{Manr}}$  
{[\textit{to myself}]}$_{\feinsub{Addr}}$.
\ex  \label{ex:03manner:f}
{[\textit{He}]}$_{\feinsub{Com}}$ \textit{was \textbf{RAVING}}  [\_]$_{\feinsub{Msg-INI}}$  
{[\textit{about Armageddon}]}$_{\feinsub{Top}}$ {[\_]}$_{\feinsub{Addr-INI}}$.
\ex  \label{ex:03manner:g}
{[\textit{He}]}$_{\feinsub{Com}}$ \textit{\textbf{MUMBLED}} [\textit{something}]$_{\feinsub{Msg}}$  
{[\textit{about something or other}]}$_{\feinsub{Top}}$.
\end{xlist}
\end{exe}

The specifics of the syntactic expression of the basic configuration of frame elements in the \framename{Communication\_manner} frame is exemplified in \tabref{tbl:communication-manner-synt}.

\begin{table}
\centering\footnotesize
\begin{tabular}{l rrrrrrrrr}
\lsptoprule
 & NP.Ext & NP.Obj & PP & AVP & NI & Clause & Quote & Other & Total\\ 
\midrule
%\multicolumn{10}{l}{\textit{babble} } \\  
%\fename{Speaker} & 38  &  &  &  &  &  &  &  & 38\\ 
%\fename{Addressee} &  &  & 2  &  & 36  &  &  &  & 38\\ 
%\fename{Topic} &  &  & 10  &  & 18  &  &  &  & 28\\ 
%\fename{Message} &  & 5  &  &  &  &  & 6  &  & 11\\ 
% \midrule
%\multicolumn{10}{l}{\textit{chant} } \\  
%\fename{Speaker} & 38  &  & 3  &  & 7  &  &  &  & 48\\ 
%\fename{Addressee} &  &  & 5  &  & 43  &  &  &  & 48\\ 
%\fename{Message} & 10  & 18  & 1  &  &  & 1  & 9  &  & 39\\ 
%\fename{Topic} &  &  &  &  & 9  &  &  &  & 9\\ 
% \midrule
%\multicolumn{10}{l}{\textit{mumble} } \\  
%\fename{Speaker} & 62  &  &  &  &  &  &  &  & 62\\ 
%\fename{Addressee} &  &  & 11  &  & 51  &  &  &  & 62\\ 
%\fename{Topic} &  &  & 11  &  & 12  &  &  &  & 23\\ 
%\fename{Message} &  & 16  &  &  &  & 6  & 22  &  & 44\\ 
% \midrule
\multicolumn{10}{l}{\textit{mutter} } \\  
\fename{Speaker} & 89  &  &  &  &  &  &  &  & 89\\ 
\fename{Addressee} &  &  & 21  &  & 68  &  &  &  & 89\\ 
\fename{Topic} &  &  & 20  &  & 14  &  &  &  & 34\\ 
\fename{Message} &  & 32  &  &  &  & 7  & 26  &  & 65\\ 

\midrule
%\multicolumn{10}{l}{\textit{rant} } \\  
%\fename{Speaker} & 26  &  &  &  &  &  &  &  & 26\\ 
%\fename{Addressee} &  &  & 4  &  & 22  &  &  &  & 26\\ 
%\fename{Topic} &  &  & 2  &  & 14  &  &  &  & 16\\ 
%\fename{Message} &  & 1  &  &  &  & 1  & 8  &  & 10\\ 
% \midrule
\multicolumn{10}{l}{\textit{rave} } \\  
\fename{Speaker} & 27  &  &  &  &  &  &  &  & 27\\ 
\fename{Addressee} &  &  & 3  &  & 24  &  &  &  & 27\\ 
\fename{Topic} &  & 3  & 13  &  & 8  &  &  &  & 24\\ 
\fename{Message} &  & 1  & 1  &  &  &  & 3  &  & 5\\ 

\midrule
\multicolumn{10}{l}{\textit{shout} } \\  
\fename{Addressee} & 3  &  & 38  &  & 83  &  &  &  & 124\\ 
\fename{Speaker} & 116  &  &  &  & 5  &  &  &  & 121\\ 
\fename{Topic} &  &  & 3  &  & 34  &  &  &  & 37\\ 
\fename{Message} & 2  & 38  & 5  &  &  & 15  & 26  &  & 86\\ 

\midrule
\multicolumn{10}{l}{\textit{sing} } \\  
\fename{Speaker} & 59  &  & 6  &  & 2  &  &  &  & 67\\ 
\fename{Addressee} &  &  & 8  &  & 58  &  &  &  & 66\\ 
\fename{Message} & 8  & 33  &  &  & 24  &  & 2  &  & 67\\ 
\fename{Topic} &  &  & 14  &  &  &  &  & 1 & 15\\ 

\midrule
\multicolumn{10}{l}{\textit{whisper} } \\  
\fename{Speaker} & 47  &  &  &  & 5  &  &  & 1 & 53\\ 
\fename{Addressee} &  &  & 16  &  & 37  &  &  &  & 53\\ 
\fename{Message} & 6  & 14  &  &  &  & 8  & 9  &  & 37\\ 
\fename{Topic} &  &  & 7  &  & 12  &  &  &  & 19\\ 
\lspbottomrule
 \end{tabular}
 \caption{Syntactic expression of the \framename{Communication\_manner} frame elements in selected FrameNet lexical units. } 
    \label{tbl:communication-manner-synt}
 \end{table}

 \subsubsection{\framename{Communication\_manner} valence patterns}

\tabref{tbl:communication-manner-valence} shows the prevalent valence patterns found with the verbs evoking the \framename{Communication\_manner} frame in the FrameNet annotated corpus.

The most frequent patterns include a canonical expression of the \fename{Speaker} as a subject NP and a \fename{Message} realised as either a direct quote, an object NP or a clausal complement. The valence patterns also show most frequently a non-overt or less often an expressed \fename{Addressee}.

\begin{table}
\begin{tabularx}{\textwidth}{lrQ}
\lsptoprule
         Pattern  & \#  & verbs \\
\midrule
{[NP.Ext]}$_{\feinsub{Spkr}}$ {[\_]}$_{\feinsub{Addr-INI}}$ {[Quote]}$_{\feinsub{Msg}}$  & 166 & \textit{rant, chant, slur, stutter, stammer, babble, chatter, rave, mumble, mutter, whisper, sing, shout}\\
% \midrule
{[NP.Ext]}$_{\feinsub{Spkr}}$ {[\_]}$_{\feinsub{Addr-INI}}$ {[\_]}$_{\feinsub{Top-INI}}$  & 156 & \textit{rant, chant, slur, stutter, stammer, babble, chatter, rave, mumble, mutter, whisper, shout}\\
% \midrule
{[NP.Ext]}$_{\feinsub{Spkr}}$ {[\_]}$_{\feinsub{Addr-INI}}$ {[NP.Obj]}$_{\feinsub{Msg}}$  & 146 & \textit{rant, chant, slur, stutter, stammer, babble, chatter, mumble, mutter, whisper, sing, shout}\\
% \midrule
{[NP.Ext]}$_{\feinsub{Spkr}}$ {[\_]}$_{\feinsub{Addr-INI}}$ {[PP]}$_{\feinsub{Top}}$  & 70 & \textit{rant, babble, chatter, rave, mumble, mutter, whisper, shout}\\
% \midrule
{[NP.Ext]}$_{\feinsub{Spkr}}$ {[PP]}$_{\feinsub{Addr}}$ {[\_]}$_{\feinsub{Top-INI}}$  & 48 & \textit{rant, chant, babble, chatter, rave, mumble, mutter, whisper, shout}\\
% \midrule
{[NP.Ext]}$_{\feinsub{Spkr}}$ {[\_]}$_{\feinsub{Addr-INI}}$ {[Clause]}$_{\feinsub{Msg}}$  & 41 & \textit{rant, chant, mumble, mutter, stutter, stammer, whisper, shout}\\
% \midrule
{[NP.Ext]}$_{\feinsub{Spkr}}$ {[PP]}$_{\feinsub{Addr}}$ {[NP.Obj]}$_{\feinsub{Msg}}$  & 34 & \textit{mumble, mutter, stutter, whisper, sing, shout}\\
% \midrule
{[NP.Ext]}$_{\feinsub{Spkr}}$ {[PP]}$_{\feinsub{Addr}}$ {[Quote]}$_{\feinsub{Msg}}$  & 31 & \textit{rant, chant, mumble, mutter, whisper, shout}\\
% \midrule
{[NP.Ext]}$_{\feinsub{Spkr}}$ {[\_]}$_{\feinsub{Addr-INI}}$ {[NP.Obj]}$_{\feinsub{Msg}}$ {[PP]}$_{\feinsub{Top}}$  & 21 & \textit{mumble, mutter, stammer, babble, sing, shout}\\
% \midrule
%{[NP.Ext]}$_{\feinsub{Msg}}$ {[\_]}$_{\feinsub{Addr-INI}}$ {[\_]}$_{\feinsub{Spkr-CNI}}$  & 17 & \textit{chant, whisper, sing, shout}\\
% \midrule
%{[NP.Ext]}$_{\feinsub{Spkr}}$ {[\_]}$_{\feinsub{Addr-INI}}$ {[\_]}$_{\feinsub{Msg-INI}}$ {[PP]}$_{\feinsub{Top}}$  & 11 & \textit{sing}\\
% \midrule
%{[NP.Ext]}$_{\feinsub{Msg}}$ {[\_]}$_{\feinsub{Addr-INI}}$ {[PP]}$_{\feinsub{Spkr}}$  & 9 & \textit{chant, sing}\\
% \midrule
%{[NP.Ext]}$_{\feinsub{Spkr}}$ {[\_]}$_{\feinsub{Addr-INI}}$ {[PP]}$_{\feinsub{Msg}}$  & 8 & \textit{chant, rave, shout}\\
% \midrule
%{[NP.Ext]}$_{\feinsub{Spkr}}$ {[\_]}$_{\feinsub{Addr-INI}}$ {[\_]}$_{\feinsub{Msg-INI}}$  & 7 & \textit{sing}\\
% \midrule
%{[NP.Ext]}$_{\feinsub{Spkr}}$ {[PP]}$_{\feinsub{Addr}}$ {[Clause]}$_{\feinsub{Msg}}$  & 6 & \textit{mouth, whisper, shout}\\
% \midrule
%{[NP.Ext]}$_{\feinsub{Spkr}}$ {[PP]}$_{\feinsub{Addr}}$ {[PP]}$_{\feinsub{Top}}$  & 5 & \textit{natter, chatter, whisper}\\
% \midrule
%{[NP.Ext]}$_{\feinsub{Spkr}}$ {[PP]}$_{\feinsub{Addr}}$ {[\_]}$_{\feinsub{Msg-INI}}$  & 5 & \textit{sing}\\
% \midrule
%{[NP.Ext]}$_{\feinsub{Spkr}}$ {[\_]}$_{\feinsub{Addr-INI}}$ {[NP.Obj]}$_{\feinsub{Top}}$  & 3 & \textit{chatter, rave}\\
% \midrule
%{[NP.Ext]}$_{\feinsub{Addr}}$ {[PP]}$_{\feinsub{Addr}}$ {[\_]}$_{\feinsub{Spkr-CNI}}$ {[\_]}$_{\feinsub{Top-INI}}$  & 3 & \textit{shout}\\
% \midrule
%{[NP.Ext]}$_{\feinsub{Spkr}}$ {[PP]}$_{\feinsub{Addr}}$ {[NP.Obj]}$_{\feinsub{Msg}}$ {[PP]}$_{\feinsub{Top}}$  & 2 & \textit{mutter, whisper}\\
% \midrule
%{[NP.Ext]}$_{\feinsub{Spkr}}$ {[\_]}$_{\feinsub{Addr-INI}}$ {[NP.Obj]}$_{\feinsub{Msg}}$ {[Quote]}$_{\feinsub{Msg}}$  & 2 & \textit{stammer, rave}\\
% \midrule
%{[NP.Ext]}$_{\feinsub{Spkr}}$ {[NP.Obj]}$_{\feinsub{Msg}}$  & 2 & \textit{mouth, sing}\\
\lspbottomrule
\end{tabularx}
\caption{FrameNet valence patterns of \framename{Communication\_manner} verbs, their frequency in the FrameNet corpus and the verbs they appear with.}
\label{tbl:communication-manner-valence}
\end{table} 


\subsubsection{Syntactic realisation of the \framename{Communication\_manner} frame in Bulgarian}

In a similar manner, in Bulgarian the \fename{Speaker} is realised as the external subject NP, while the \fename{Message} is expressed as a direct quote (Example \ref{ex:03mannerbg:a}), a finite complement clause (Example \ref{ex:03mannerbg:b}) or an NP Object (Example \ref{ex:03mannerbg:c}). The \fename{Topic} and the \fename{Addressee} are expressed as prepositional complements (Example \ref{ex:03mannerbg:d}).

\begin{exe}
\ex \label{ex:03mannerbg}
\begin{xlist}
\ex  \label{ex:03mannerbg:a}
\gll {[\_]}$_{\feinsub{Spkr-DNI}}$ \textit{Едва} \textit{\textbf{ПРОШЕПВАМ}}: [– \textit{За} \textit{какво} \textit{става дума}?]$_{\feinsub{Msg}}$ [\_]$_{\feinsub{Addr-INI}}$ \\
{} Hardly whisper.1sg: – About what {take place word}? {}
\\
\glt `I hardly whisper: – What is it about?'
\ex  \label{ex:03mannerbg:b}
\gll {[\_]}$_{\feinsub{Spkr-DNI}}$ \textit{\textbf{ПРОМЪРМОРВАМ}}, [\textit{че} \textit{отчаяно} \textit{искам} \textit{да} \textit{си} \textit{го} \textit{върна}]$_{\feinsub{Msg}}$ [\_]$_{\feinsub{Addr-INI}}$.\\
{} Mutter.1sg that desperately want.1sg to REFL it {get back}. {}\\
\glt `I mutter that I desperately want to get it back.'
\ex  \label{ex:03mannerbg:c}
\gll {[\textit{Аз}]}$_{\feinsub{Spkr}}$ \textit{\textbf{ИЗМЪНКАХ}} [\textit{някакъв} \textit{отговор}]$_{\feinsub{Msg}}$ [\_]$_{\feinsub{Addr}}$. \\
I stammered some reply. {}
\\
%\glt `I stammered some stupid reply.'
\ex  \label{ex:03mannerbg:d}
\gll {[\textit{Тя}]}$_{\feinsub{Spkr}}$ \textit{\textbf{ДРЪНКА}} [\textit{на} \textit{всички}]$_{\feinsub{Addr}}$ [\textit{за} \textit{мен}]$_{\feinsub{Top}}$.\\
She babbles to everyone about me.
\\
%\glt `She babbles about me to all.'
\ex  \label{ex:03mannerbg:e}
\gll {[\textit{Капитанът}]}$_{\feinsub{Spkr}}$ \textit{продължи} \textit{да} \textit{\textbf{КРЕЩИ}} [\textit{заповедите} \textit{си}]$_{\feinsub{Msg}}$
[\textit{за} \textit{разни} \textit{платна} \textit{и} \textit{въжета}]$_{\feinsub{Top}}$% [\_]$_{\feinsub{Addr}}$
. \\
Captain-\textsc{def} continued to shout orders REFL about some sails and ropes.\\
\glt `The captain continued shouting his orders about sails and ropes.'
\ex  \label{ex:03mannerbg:f}
\gll {[– \textit{Здравейте}]}$_{\feinsub{Msg}}$ – \textit{\textbf{ИЗМЪНКВАМ}} [\textit{аз}]$_{\feinsub{Spkr}}$ [\_]$_{\feinsub{Addr}}$  \textit{нерешително}.\\
{– Hello} – mumble I {} hesitantly.
\\ %tuka njmaa prevod, dali taka sme iskali
%\glt `– Hello, – I mumbled hesitantly.'
\end{xlist}
\end{exe}



\begin{table}
\footnotesize
\begin{tabular}{l rrrrrrrrr}
\lsptoprule
 & NP.Ext & NP.Obj & PP & AVP & NI & Clause & Quote & Other & Total\\
\midrule
%\multicolumn{10}{l}{\textit{смотолевям\slash смотолевя} }\\  
%\fename{Message} &  &  &  &  &  &  & 4 &  & 4\\ 
%\fename{Addressee} &  &  &  &  & 4 &  &  &  & 4\\ 
%\fename{Speaker} & 4 &  &  &  &  &  &  &  & 4\\ 
% \midrule
%\multicolumn{10}{l}{\textit{заеквам\slash заекна} }\\  
%\fename{Message} &  &  &  &  &  &  & 4 &  & 4\\ 
%\fename{Addressee} &  &  &  &  & 5 &  &  &  & 5\\ 
%\fename{Speaker} & 5 &  &  &  &  &  &  &  & 5\\ 
% \midrule
%\multicolumn{10}{l}{\textit{шушукам} }\\  
%\fename{Message} &  & 1 &  &  &  &  &  &  & 1\\ 
%\fename{Addressee} &  &  & 2 &  & 2 &  &  &  & 4\\ 
%\fename{Speaker} & 4 &  &  &  &  &  &  &  & 4\\ 
% \midrule
\multicolumn{10}{l}{\textit{шепна} }\\
`whisper'\\
\fename{Message} &  & 3 &  &  & 1 &  & 3 &  & 7\\ 
\fename{Addressee} &  &  & 6 &  & 2 &  &  &  & 8\\ 
\fename{Speaker} & 8 &  &  &  &  &  &  &  & 8\\ 

\midrule
\multicolumn{10}{l}{\textit{промърморвам\slash промърморя} }\\
\multicolumn{10}{l}{`mumble, mutter'}\\
\fename{Message} &  & 2 &  &  &  &  & 8 & 3 & 13\\ 
\fename{Addressee} &  &  & 3 &  & 10 &  &  &  & 13\\ 
\fename{Medium} &  &  & 1 &  &  &  &  &  & 1\\ 
\fename{Speaker} & 13 &  &  &  &  &  &  &  & 13\\ 

\midrule
%\multicolumn{10}{l}{\textit{мърморя} }\\  
%\fename{Message} &  & 3 &  &  &  &  &  &  & 3\\ 
%\fename{Addressee} &  &  & 2 &  & 2 &  &  &  & 4\\ 
%\fename{Speaker} & 4 &  &  &  &  &  &  &  & 4\\ 
% \midrule
%\multicolumn{10}{l}{\textit{бъбря} }\\  
%\fename{Message} &  &  &  &  &  &  & 2 &  & 2\\ 
%\fename{Addressee} &  &  & 2 &  & 2 &  &  &  & 4\\ 
%\fename{Speaker} & 4 &  &  &  &  &  &  &  & 4\\ 
% \midrule
\multicolumn{10}{l}{\textit{викам} }\\
`shout'\\
\fename{Message} &  & 3 &  &  & 1 &  & 23 & 4 & 31\\ 
\fename{Addressee} &  &  & 2 &  & 33 &  &  &  & 35\\ 
\fename{Medium} &  &  & 1 &  &  &  &  &  & 1\\ 
\fename{Speaker} & 35 &  &  &  &  &  &  &  & 35\\ 

\midrule
%\multicolumn{10}{l}{\textit{крещя} }\\  
%\fename{Message} &  & 3 &  &  &  &  & 3 &  & 6\\ 
%\fename{Addressee} &  &  & 1 &  & 7 &  &  &  & 8\\ 
%\fename{Speaker} & 8 &  &  &  &  &  &  &  & 8\\ 
% \midrule
\multicolumn{10}{l}{\textit{прошепвам\slash прошепна} }\\
`whisper'\\
\fename{Message} &  & 8 &  &  &  &  & 7 & 1 & 16\\ 
\fename{Addressee} &  &  & 6 &  & 10 &  &  &  & 16\\ 
\fename{Speaker} & 17 &  &  &  &  &  &  &  & 17\\ 

%\multicolumn{10}{l}{\textit{пошушвам\slash пошушна} }\\  
%\fename{Message} &  &  &  &  &  &  & 3 &  & 3\\ 
%\fename{Addressee} &  &  & 1 &  & 3 &  &  &  & 4\\ 
%\fename{Speaker} & 4 &  &  &  &  &  &  &  & 4\\ 
% \midrule
%\multicolumn{10}{l}{\textit{дрънкам} }\\  
%\fename{Message} &  & 2 &  &  & 1 &  &  & 2 & 5\\ 
%\fename{Addressee} &  &  & 2 &  & 7 &  &  &  & 9\\ 
%\fename{Topic} &  &  & 4 &  &  &  &  &  & 4\\ 
%\fename{Speaker} & 9 &  &  &  &  &  &  &  & 9\\ 
% \midrule
\lspbottomrule
 \end{tabular}
 \caption{Syntactic expression of the \framename{Communication\_manner} frame elements in Bulgarian.} 
    \label{tbl:communication-manner-synt-bg}
 \end{table}
 
\tabref{tbl:communication-manner-synt-bg} shows a selection of verbs in Bulgarian evoking the frame \framename{Communication\_manner}, while
\tabref{tbl:communication-manner-valence-bg} presents the most frequent valence patterns. The syntactic realisation is similar to English: strong preference for the overt expression of the \fename{Message} either together with the \fename{Addressee} or in its absence; realising the \fename{Topic} most often either in the absence of (Example \ref{ex:03mannerbg:d}) or as a modifier to the \fename{Message} (Example \ref{ex:03mannerbg:e}).

We can also note that at least for some manner verbs such as \textit{мърморя} `mumble, mutter’, \textit{мънкам} `stutter’, there is a marked trend of expressing the \fename{Message} as a quote rather than as a complement clause. 



\begin{table}
\begin{tabularx}{\textwidth}{lrQ}
\lsptoprule
         Pattern  & \#  & verbs \\\midrule
{[NP.Ext]}$_{\feinsub{Spkr}}$ {[Quote]}$_{\feinsub{Msg}}$ {[\_]}$_{\feinsub{Addr-INI}}$ & 48 & \textit{бъбря, викам, заеквам\slash заекна, крещя, пошушвам\slash пошушна, промърморвам\slash промърморя, прошепвам\slash прошепна, смотолевям\slash смотолевя, шепна}\\
{[NP.Ext]}$_{\feinsub{Spkr}}$ {[NP.Obj]}$_{\feinsub{Msg}}$ {[\_]}$_{\feinsub{Addr-INI}}$ & 13 & \textit{бърборя, викам, дрънкам, крещя, мърморя, прошепвам\slash прошепна, шепна, промърморвам\slash промърморя}\\
{[NP.Ext]}$_{\feinsub{Spkr}}$ {[\_]}$_{\feinsub{Addr-INI}}$ & 12 & \textit{бъбря, викам, дрънкам, заеквам\slash заекна, крещя, мърморя, шушукам}\\
{[NP.Ext]}$_{\feinsub{Spkr}}$ {[NP.Obj]}$_{\feinsub{Msg}}$ {[PP]}$_{\feinsub{Addr}}$ & 11 & \textit{викам, дрънкам, крещя, мърморя, прошепвам\slash прошепна, шепна, шушукам}\\
{[NP.Ext]}$_{\feinsub{Spkr}}$ {[PP]}$_{\feinsub{Addr}}$ {[Quote]}$_{\feinsub{Msg}}$ & 10 & \textit{бъбря, викам, прошепвам\slash прошепна, шепна, шушна, промърморвам\slash промърморя}\\
{[NP.Ext]}$_{\feinsub{Spkr}}$ {[Clause-that]}$_{\feinsub{Msg}}$ {[\_]}$_{\feinsub{Addr-INI}}$ & 6 & \textit{викам, дрънкам, шушна, промърморвам\slash промърморя}\\
%{[NP.Ext]}$_{\feinsub{Spkr}}$ {[PP]}$_{\feinsub{Addr}}$ & 5 & \textit{бъбря, бърборя, пошушвам\slash пошушна, шепна, шушукам}\\
%{[NP.Ext]}$_{\feinsub{Spkr}}$ {[\_]}$_{\feinsub{Addr-INI}}$ {[\_]}$_{\feinsub{Msg-INI}}$ & 2 & \textit{викам, дрънкам}\\
%{[NP.Ext]}$_{\feinsub{Spkr}}$ {[NP.Obj]}$_{\feinsub{Msg}}$ {[PP]}$_{\feinsub{MEDIUM}}$ {[\_]}$_{\feinsub{Addr-INI}}$ & 2 & \textit{викам, промърморвам\slash промърморя}\\
{[NP.Ext]}$_{\feinsub{Spkr}}$ {[PP]}$_{\feinsub{Top}}$ {[\_]}$_{\feinsub{Addr-INI}}$ & 2 & \textit{дрънкам}\\
\lspbottomrule
\end{tabularx}
    \caption{FrameNet valence patterns of \framename{Communication\_manner} verbs, their frequency in the Bulgarian dataset and the verbs they appear with.    
    English translation equivalents: \textit{бъбря} `babble, prattle', \textit{викам} `shout', \textit{дрънкам} `rattle, jabber', \textit{заеквам\slash заекна} `stammer, stutter', \textit{крещя} `shout, yell', \textit{мърморя, промърморвам\slash промърморя} `mumble, mutter', \textit{пошушвам\slash пошушна}, \textit{прошепвам\slash прошепна}, \textit{шепна}, \textit{шушна}, \textit{шушукам} `whisper', \textit{смотолевям\slash смотолевя} `mumble, falter'.} 
    \label{tbl:communication-manner-valence-bg}
\end{table} 


\subsection{Frame \framename{Statement}}
\largerpage[2]
\begin{description}[font=\normalfont]
\item[Definition of the frame \framename{Statement}:] A \fename{Speaker} addresses a \fename{Message} to some \fename{Addressee} using language. Instead of (or in addition to) a \fename{Speaker}, a \fename{Medium} may also be mentioned. Likewise, a \fename{Topic} may be stated instead of a \fename{Message}. Core frame elements: \fename{Speaker}, \fename{Message}, \fename{Medium}, \fename{Topic}; Non-core: \fename{Addressee}.
\end{description}

This frame represents the greatest number of verbs of speech, including many general lexis verbs such as \textit{say}.v, \textit{state}.v, \textit{declare}.v, \textit{speak}.v, \textit{report}.v, \textit{note}.v, etc. 

\subsubsection{Syntactic realisation of the \framename{Statement} frame elements}

The frame \framename{Statement} is an elaboration of the prototypical frame \framename{Communication} which specifies verbs for communication involving language. This is reflected by the fact that the \fename{Communicator} is conceptualised as the more specific \fename{Speaker}, which denotes the person who produces the message. Likewise, this The frame element \fename{Speaker} is realised as the external NP.

The \fename{Message} is  typically expressed either as a subordinate clause, an NP object, or a direct quote that represents the content being conveyed (Example \ref{ex:04statement:a}, \ref{ex:04statement:b}, \ref{ex:04statement:c}, respectively). There is a range of preferred realisations of the \fename{Message} with the different verbs in this frame: some of them have a stronger tendency to take a complement subordinate clause (e.g., \textit{claim}.v, \textit{suggest}.v, \textit{note}.v), while others show preference for an NP object (e.g., \textit{profess}.v, \textit{reiterate}.v, \textit{relate}.v) or a quote (e.g., \textit{exclaim}.v); in some cases the three realisations are equally likely (e.g., \textit{caution}.v). 

The \fename{Topic} is typically expressed as a prepositional phrase headed by different prepositions depending on the verb, e.g. (\textit{speak about him, speak of him, preach of heaven, comment on the protests, comment upon the economic conditions}), a trend inherited from the \framename{Communication} frame. Similarly to the frames discussed above, usually either the \fename{Message} or the \fename{Topic} is expressed; as expected, they may also occur together in a phrase (Example \ref{ex:04statement:b}), where the \fename{Topic} is syntactically dependent on the \fename{Message}. In addition, some verbs co-occur more readily with a \fename{Topic} rather than with a \fename{Message}, e.g. \textit{explain}.v (Example \ref{ex:04statement:d}).

As a peripheral frame element the \fename{Addressee} is often left non-overt although implied. When present, it is expressed as a prepositional phrase most frequently with the preposition `to' (Example \ref{ex:04statement:d}). In some cases it may be realised as an indirect object (Example \ref{ex:04statement:e}).


\begin{exe}
\ex \label{ex:04statement}
\begin{xlist}
\ex  \label{ex:04statement:a}
{[\textit{North} \textit{Korea}]}$_{\feinsub{Spkr}}$ \textit{\textbf{CLAIMED}} [\textit{it} \textit{had} \textit{no} \textit{intention} \textit{of} \textit{producing} \textit{nuclear} \textit{weapons}]$_{\feinsub{Msg}}$.
\ex  \label{ex:04statement:b}
{[\textit{He}]}$_{\feinsub{Spkr}}$ \textit{\textbf{SAID}} [\textit{little}]$_{\feinsub{Msg}}$ [\textit{about} \textit{the} \textit{case}]$_{\feinsub{Top}}$.
\ex  \label{ex:04statement:c}
{[\textit{He}]}$_{\feinsub{Spkr}}$ \textit{\textbf{ADDED}}: [`\textit{Eldorado is a brave venture}']$_{\feinsub{Msg}}$.
\ex  \label{ex:04statement:d}
{[\textit{Doc}]}$_{\feinsub{Spkr}}$ \textit{\textbf{EXPLAINED}} [\textit{the} \textit{injuries}]$_{\feinsub{Msg}}$ [\textit{to} \textit{the} \textit{police}]$_{\feinsub{Addr}}$.
\ex  \label{ex:04statement:e}
{[\textit{The} \textit{agency}]}$_{\feinsub{Spkr}}$ \textit{\textbf{WROTE}} [\textit{me}]$_{\feinsub{Addr}}$ [\textit{that} \textit{you} \textit{had}  \textit{moved}]$_{\feinsub{Msg}}$.
\ex  \label{ex:04statement:f}
{[\textit{The} \textit{letter}]}$_{\feinsub{Med}}$ \textit{\textbf{ALLEGED}} [\textit{serious} \textit{breaches} \textit{of} \textit{the} \textit{law}]$_{\feinsub{Msg}}$.
\end{xlist}
\end{exe}


The various specific configuration of frame elements as expressed by verbs in the \framename{Statement} frame are shown in \tabref{tbl:statement-synt}.

\largerpage
\begin{table}
\footnotesize
\begin{tabular}{l rrrrrrrrr}
\lsptoprule
 & NP.Ext & NP.Obj & PP & AVP & NI & Clause & Quote & Other & Total\\ 

\midrule
% \multicolumn{10}{l}{\textit{add} } \\  
%\fename{Speaker} & 24  &  &  &  &  &  &  & 1 & 25\\ 
%\fename{Addressee} &  &  & 1  &  &  &  &  &  & 1\\ 
%\fename{Message} & 1  & 6  &  &  & 1  & 7  & 10  &  & 25\\ 
%\fename{Medium} & 1  &  &  &  &  &  &  &  & 1\\ 
%\fename{Topic} &  &  & 3  &  &  &  &  &  & 3\\ 
% \midrule
\multicolumn{10}{l}{\textit{announce} } \\  
\fename{Speaker} & 44  &  & 3  &  & 5  &  &  & 1 & 53\\ 
\fename{Addressee} &  &  & 6  &  &  &  &  & 1 & 7\\ 
\fename{Message} & 8  & 20  &  &  &  & 24  & 6  &  & 58\\ 
\fename{Medium} & 3  &  & 2  &  &  &  &  &  & 5\\ 

\midrule
%\multicolumn{10}{l}{\textit{assert} } \\  
%\fename{Speaker} & 24  &  &  &  & 3  &  &  &  & 27\\ 
%\fename{Message} & 3  & 3  & 1  &  &  & 18  & 5  &  & 30\\ 
%\fename{Topic} &  &  & 1  &  &  &  &  &  & 1\\ 
%\fename{Medium} &  &  & 1  &  &  &  &  &  & 1\\ 
% \midrule
%\multicolumn{10}{l}{\textit{caution} } \\  
%\fename{Speaker} & 38  &  & 6  &  & 1  &  &  &  & 45\\ 
%\fename{Addressee} & 7  & 18  &  &  &  &  &  &  & 25\\ 
%\fename{Message} &  & 1  & 11  &  & 6  & 11  & 10  &  & 39\\ 
%\fename{Topic} &  &  & 7  &  &  &  &  &  & 7\\ 
% \midrule
%\multicolumn{10}{l}{\textit{comment} } \\  
%\fename{Speaker} & 26  &  & 2  &  & 3  &  &  &  & 31\\ 
%\fename{Addressee} &  &  & 4  &  &  &  &  &  & 4\\ 
%\fename{Message} &  &  & 1  & 1  &  & 4  & 6  &  & 12\\ 
%\fename{Topic} & 5  &  & 16  &  & 16  &  &  &  & 37\\ 
%\fename{Medium} & 1  &  & 3  &  &  &  &  &  & 4\\ 
% \midrule
\multicolumn{10}{l}{\textit{declare} } \\  
\fename{Speaker} & 58  &  &  &  & 7  &  &  &  & 65\\ 
\fename{Addressee} &  &  & 7  &  &  &  &  &  & 7\\ 
\fename{Message} & 7  & 32  & 6  &  &  & 17  & 15  & 7 & 84\\ 
%\fename{Medium} & 2  &  & 1  &  &  &  &  &  & 3\\ 
%\fename{Topic} &  &  & 1  &  &  &  &  & 1 & 2\\ 

\midrule
%\multicolumn{10}{l}{\textit{explain} } \\  
%\fename{Speaker} & 34  &  & 2  &  & 1  &  &  &  & 37\\ 
%\fename{Addressee} &  &  & 15  &  &  &  &  &  & 15\\ 
%\fename{Medium} & 1  &  & 4  &  &  &  &  &  & 5\\ 
%\fename{Topic} & 3  & 14  & 6  &  & 2  & 4  &  &  & 29\\ 
%\fename{Message} &  & 2  &  &  &  & 6  & 1  &  & 9\\ 
% \midrule
%\multicolumn{10}{l}{\textit{insist} } \\  
%\fename{Speaker} & 19  &  & 2  &  &  &  &  &  & 21\\ 
%\fename{Addressee} &  &  & 2  &  &  &  &  &  & 2\\ 
%\fename{Message} & 2  &  & 16  &  &  & 3  & 2  &  & 23\\ 
% \midrule
%\multicolumn{10}{l}{\textit{mention} } \\  
%\fename{Speaker} & 19  &  &  &  & 1  &  &  &  & 20\\ 
%\fename{Addressee} &  &  & 5  &  &  &  &  &  & 5\\ 
%\fename{Message} & 2  & 3  &  & 1  &  & 12  & 3  &  & 21\\ 
%\fename{Medium} & 1  &  & 4  &  &  &  &  &  & 5\\ 
%\fename{Topic} & 1  & 1  &  &  &  &  &  &  & 2\\ 
%
% \midrule
%\multicolumn{10}{l}{\textit{propose} } \\  
%\fename{Speaker} & 23  &  & 2  &  & 3  &  &  &  & 28\\ 
%\fename{Addressee} &  &  & 3  &  &  &  &  &  & 3\\ 
%\fename{Message} & 5  & 7  & 1  &  &  & 18  & 1  &  & 32\\ 
%\fename{Medium} & 3  &  & 1  &  &  &  &  &  & 4\\ 
%
% \midrule
%\multicolumn{10}{l}{\textit{recount} } \\  
%\fename{Speaker} & 31  &  & 1  &  & 5  &  &  &  & 37\\ 
%\fename{Addressee} &  &  & 8  &  &  &  &  &  & 8\\ 
%\fename{Message} & 6  & 23  &  &  &  & 7  & 3  &  & 39\\ 
%\fename{Medium} & 2  &  & 4  &  &  &  &  &  & 6\\ 
%\fename{Topic} &  &  & 1  &  &  &  &  &  & 1\\ 
%
% \midrule
%\multicolumn{10}{l}{\textit{reiterate} } \\  
%\fename{Speaker} & 24  &  & 2  &  & 1  &  &  &  & 27\\ 
%\fename{Addressee} &  &  & 1  &  &  &  &  &  & 1\\ 
%\fename{Message} & 2  & 17  &  &  &  & 3  & 4  &  & 26\\ 
%\fename{Medium} &  &  & 1  &  &  &  &  &  & 1\\ 
%\fename{Topic} & 1  &  &  &  &  &  &  &  & 1\\ 
%
% \midrule
%\multicolumn{10}{l}{\textit{relate} } \\  
%\fename{Speaker} & 24  &  & 1  &  &  &  &  &  & 25\\ 
%\fename{Addressee} &  &  & 1  &  &  &  &  &  & 1\\ 
%\fename{Message} & 1  & 14  &  & 1  &  & 7  & 2  &  & 25\\ 
%\fename{Topic} &  &  & 1  &  &  &  &  &  & 1\\ 
%
% \midrule
%\multicolumn{10}{l}{\textit{remark} } \\  
%\fename{Speaker} & 27  &  &  &  &  &  &  &  & 27\\ 
%\fename{Addressee} &  &  & 3  &  &  &  &  &  & 3\\ 
%\fename{Message} &  &  &  & 3  &  & 6  & 10  &  & 19\\ 
%\fename{Topic} &  &  & 12  &  &  &  &  &  & 12\\ 
%\fename{Medium} & 1  &  & 1  &  &  &  &  &  & 2\\ 
% \midrule
\multicolumn{10}{l}{\textit{report} } \\  
\fename{Speaker} & 54  &  & 1  &  & 19  &  &  &  & 74\\ 
\fename{Addressee} &  &  & 8  &  &  &  &  &  & 8\\ 
\fename{Message} & 19  & 20  & 2  & 1  & 1  & 44  & 2  &  & 89\\ 
\fename{Medium} & 9  &  & 5  &  & 1  &  &  & 1 & 16\\ 
\fename{Topic} & 2  &  & 5  &  & 1  &  &  &  & 8\\ 

\midrule
\multicolumn{10}{l}{\textit{say} } \\  
\fename{Message} & 14  & 22  & 1  & 4  & 2  & 49  & 33  &  & 125\\ 
\fename{Addressee} &  &  & 8  &  &  &  &  &  & 8\\ 
\fename{Speaker} & 90  & 1  &  &  & 14  &  &  &  & 105\\ 
\fename{Medium} & 9  &  & 10  &  & 1  &  &  &  & 20\\ 
\fename{Topic} &  &  & 10  &  & 1  &  &  & 1 & 12\\ 

\midrule
\multicolumn{10}{l}{\textit{state} } \\  
\fename{Speaker} & 38  &  &  &  &  &  &  &  & 38\\ 
\fename{Addressee} &  &  & 3  &  &  &  &  &  & 3\\ 
\fename{Message} & 3  & 8  & 2  &  &  & 19  & 13  &  & 45\\ 
\fename{Medium} & 3  &  & 1  &  & 3  &  &  &  & 7\\ 

\midrule
\multicolumn{10}{l}{\textit{suggest} } \\  
\fename{Speaker} & 27  &  & 2  &  & 4  &  &  &  & 33\\ 
\fename{Addressee} &  &  & 5  &  &  &  &  &  & 5\\ 
\fename{Message} & 3  & 5  &  & 3  &  & 21  & 5  &  & 37\\ 
\fename{Medium} & 4  &  & 4  &  &  &  &  &  & 8\\ 

\midrule
\multicolumn{10}{l}{\textit{talk} } \\  
\fename{Speaker} & 32  &  & 1  &  & 3  &  &  &  & 36\\ 
\fename{Topic} & 3  &  & 29  & 2  &  & 2  &  &  & 36\\ 
%\fename{Medium} & 1  &  &  &  &  &  &  &  & 1\\ 
\fename{Message} & 1  & 3  &  &  &  &  &  &  & 4\\ 

\midrule
\multicolumn{10}{l}{\textit{write} } \\  
\fename{Speaker} & 42  &  &  &  & 1  &  &  &  & 43\\ 
\fename{Addressee} &  & 2  & 4  &  &  & 1  &  &  & 7\\ 
\fename{Message} & 1  & 5  &  &  & 2  & 10  & 13  &  & 31\\ 
\fename{Topic} & 1  &  & 22  &  &  &  &  &  & 23\\ 
\fename{Medium} & 1  &  & 8  &  &  &  &  &  & 9\\ 

\lspbottomrule
%\multicolumn{10}{l}{\textit{confirm} } \\  
%\fename{Speaker} & 22  &  &  &  & 3  &  &  &  & 25\\ 
%\fename{Addressee} &  &  & 3  &  &  &  &  &  & 3\\ 
%\fename{Message} & 3  & 5  &  &  & 1  & 17  &  &  & 26\\ 
%\fename{Medium} & 1  &  & 3  &  &  &  &  &  & 4\\ 
% \midrule
%\multicolumn{10}{l}{\textit{note} } \\  
%\fename{Speaker} & 28  &  &  &  & 1  &  &  &  & 29\\ 
%\fename{Message} &  & 5  &  &  & 1  & 26  &  &  & 32\\ 
%\fename{Medium} & 5  &  & 4  &  &  &  &  &  & 9\\ 
%\fename{Topic} & 1  & 1  &  &  &  &  &  &  & 2\\ 
% \midrule
 \end{tabular}
 \caption{Syntactic expression of the \framename{Statement} frame elements in selected FrameNet lexical units.}
    \label{tbl:statement-synt}
 \end{table}


\subsubsection{\framename{Statement} valence patterns}

The prevalent valence patterns for verbs in the FrameNet frame \framename{Statement} are shown in \tabref{tbl:statement-valence}. The most typical ones include the canonical expression of the \fename{Speaker} as the external NP and the \fename{Message} as a subordinate clause, an object NP, or a quote.

Alternatively, the \fename{Medium} may occupy the position of the external argument with an implied generalised reading of the \fename{Speaker} which is left unexpressed (Example \ref{ex:04statement:f}). 
Similarly to many of the frames describing verbs of communication, instead of the \fename{Message} the \fename{Topic} may be realised, most often as a prepositional phrase.

The patterns involving the expression of an \fename{Addressee} are quite infrequent.
%The fact that talk and speak do allow the expression of the role Topic shows that the role Message is conceptually present, because Topic is a property of Messages.


\begin{table}[t]
    \centering\footnotesize
    \begin{tabularx}{\textwidth}{ lrQ }
\lsptoprule
         Pattern  & \#  & verbs \\
\midrule
{[NP.Ext]}$_{\feinsub{Spkr}}$ {[Clause]}$_{\feinsub{Msg}}$  & 281 & \textit{explain, note, declare, maintain, remark, mention, conjecture, reiterate, assert, preach, claim, attest, state, caution, write, add, allege, exclaim, say, suggest, insist, propose, announce, confirm, acknowledge, proclaim, reaffirm, report, pronounce}\\
{[NP.Ext]}$_{\feinsub{Spkr}}$ {[NP.Obj]}$_{\feinsub{Msg}}$  & 191 & \textit{explain, note, declare, tell, conjecture, reiterate, assert, preach, claim, speak, talk, state, caution, write, add, allege, exclaim, say, suggest, propose, announce, confirm, acknowledge, refute, proclaim, reaffirm, report}\\
{[NP.Ext]}$_{\feinsub{Spkr}}$ {[Quote]}$_{\feinsub{Msg}}$  & 143 & \textit{explain, gloat, declare, remark, observe, mention, reiterate, hazard, assert, preach, speak, attest, state, caution, write, add, allege, exclaim, say, pout, suggest, insist, propose, announce, proclaim, reaffirm, report}\\
{[NP.Ext]}$_{\feinsub{Spkr}}$ {[PP]}$_{\feinsub{Top}}$  & 83 & \textit{explain, gloat, preach, report, comment, remark, speak, talk, write}\\
{[NP.Ext]}$_{\feinsub{Medium}}$ {[Clause]}$_{\feinsub{Msg}}$  & 39 & \textit{note, declare, allege, say, suggest, propose, announce, confirm, acknowledge, proclaim, report, claim, state}\\
{[NP.Ext]}$_{\feinsub{Spkr}}$ {[PP]}$_{\feinsub{Addr}}$ {[NP.Obj]}$_{\feinsub{Msg}}$  & 28 & \textit{reiterate, declare, report, say, speak, state, suggest, propose, announce, mention}\\
{[NP.Ext]}$_{\feinsub{Spkr}}$ {[PP]}$_{\feinsub{Msg}}$  & 28 & \textit{profess, declare, preach, say, speak, describe, insist, caution}\\
%{[NP.Ext]}$_{\feinsub{Msg}}$ {[\_]}$_{\feinsub{Spkr-CNI}}$  & 28 & \textit{declare, allege, say, suggest, propose, announce, confirm, reiterate, assert, proclaim, preach, report, speak}\\
{[NP.Ext]}$_{\feinsub{Spkr}}$ {[PP]}$_{\feinsub{Addr}}$ {[Clause]}$_{\feinsub{Msg}}$  & 25 & \textit{add, explain, declare, allege, suggest, insist, propose, announce, mention, confirm, preach}\\
%{[NP.Ext]}$_{\feinsub{Msg}}$ {[Clause]}$_{\feinsub{Msg}}$ {[\_]}$_{\feinsub{Spkr-CNI}}$  & 21 & \textit{assert, allege, report, say}\\
{[NP.Ext]}$_{\feinsub{Spkr}}$ {[PP]}$_{\feinsub{Medium}}$ {[Clause]}$_{\feinsub{Msg}}$  & 20 & \textit{explain, note, acknowledge, allege, claim, say, state, suggest, write, mention}\\
{[NP.Ext]}$_{\feinsub{Medium}}$ {[NP.Obj]}$_{\feinsub{Msg}}$  & 20 & \textit{explain, note, proclaim, tell, allege, reaffirm, say, state, propose, announce, mention}\\
%{[NP.Ext]}$_{\feinsub{Msg}}$ {[PP]}$_{\feinsub{Spkr}}$  & 18 & \textit{profess, suggest, propose, announce, reiterate, preach, reaffirm, report, speak, talk, attest}\\\midrule
%{[NP.Ext]}$_{\feinsub{Spkr}}$ {[AVP]}$_{\feinsub{Msg}}$  & 16 & \textit{proclaim, report, remark, say, attest, suggest, mention}\\\midrule
%{[NP.Ext]}$_{\feinsub{Spkr}}$ {[NP.Obj]}$_{\feinsub{Top}}$  & 16 & \textit{explain, acknowledge, mention}\\\midrule
%{[NP.Ext]}$_{\feinsub{Spkr}}$ {[NP.Obj]}$_{\feinsub{Msg}}$ {[PP]}$_{\feinsub{Top}}$  & 13 & \textit{add, declare, allege, say}\\\midrule
%{[NP.Ext]}$_{\feinsub{Spkr}}$ {[PP]}$_{\feinsub{Addr}}$ {[Quote]}$_{\feinsub{Msg}}$  & 12 & \textit{declare, proclaim, exclaim, remark, say, insist, announce, mention}\\\midrule
%{[NP.Ext]}$_{\feinsub{Spkr}}$ {[PP]}$_{\feinsub{Addr}}$ {[PP]}$_{\feinsub{Top}}$  & 11 & \textit{explain, gloat, preach, remark, speak, write}\\\midrule
%{[NP.Ext]}$_{\feinsub{Spkr}}$  & 10 & \textit{add, maintain, allege, report, claim, speak, state}\\\midrule
%{[NP.Ext]}$_{\feinsub{Msg}}$ {[PP]}$_{\feinsub{MEDIUM}}$ {[\_]}$_{\feinsub{Spkr-CNI}}$  & 10 & \textit{confirm, assert, report, say, describe}\\\midrule
%{[NP.Ext]}$_{\feinsub{MEDIUM}}$ {[PP]}$_{\feinsub{Top}}$  & 7 & \textit{report, remark, talk, write}\\\midrule
%{[NP.Ext]}$_{\feinsub{Spkr}}$ {[PP]}$_{\feinsub{MEDIUM}}$ {[NP.Obj]}$_{\feinsub{Msg}}$  & 7 & \textit{confirm, reiterate, reaffirm, say, suggest, mention}\\\midrule
%{[NP.Ext]}$_{\feinsub{Spkr}}$ {[\_]}$_{\feinsub{Msg-DNI}}$  & 7 & \textit{note, say, speak, write}\\\midrule
%{[NP.Ext]}$_{\feinsub{Spkr}}$ {[NP.Obj]}$_{\feinsub{Msg}}$ {[Clause]}$_{\feinsub{Msg}}$  & 7 & \textit{declare, say, announce}\\\midrule
%{[NP.Ext]}$_{\feinsub{Spkr}}$ {[NP.Obj]}$_{\feinsub{Addr}}$ {[PP]}$_{\feinsub{Top}}$  & 6 & \textit{caution}\\\midrule
%{[NP.Ext]}$_{\feinsub{Spkr}}$ {[Quote]}$_{\feinsub{Msg}}$ {[Clause]}$_{\feinsub{Msg}}$  & 6 & \textit{assert, state, write}\\\midrule
%{[NP.Ext]}$_{\feinsub{Spkr}}$ {[NP.Obj]}$_{\feinsub{Addr}}$ {[Clause]}$_{\feinsub{Msg}}$  & 6 & \textit{address, tell, caution, write}\\\midrule
%{[NP.Ext]}$_{\feinsub{Spkr}}$ {[NP.Obj]}$_{\feinsub{Addr}}$ {[NP.Dep]}$_{\feinsub{Msg}}$  & 5 & \textit{address, tell}\\\midrule
%{[NP.Ext]}$_{\feinsub{Top}}$ {[\_]}$_{\feinsub{Spkr-CNI}}$ {[PP]}$_{\feinsub{Top}}$  & 5 & \textit{comment, talk}\\\midrule
%{[NP.Ext]}$_{\feinsub{Msg}}$ {[\_]}$_{\feinsub{Spkr-CNI}}$ {[PP]}$_{\feinsub{Top}}$  & 5 & \textit{say, speak}\\\midrule
%{[NP.Ext]}$_{\feinsub{Spkr}}$ {[Clause]}$_{\feinsub{Top}}$  & 5 & \textit{explain, talk}\\\midrule
%{[NP.Ext]}$_{\feinsub{Spkr}}$ {[PP]}$_{\feinsub{MEDIUM}}$ {[PP]}$_{\feinsub{Top}}$  & 5 & \textit{comment, write}\\\midrule
%{[NP.Ext]}$_{\feinsub{Spkr}}$ {[NP.Obj]}$_{\feinsub{Addr}}$ {[Quote]}$_{\feinsub{Msg}}$  & 5 & \textit{address, caution}\\\midrule
%{[NP.Ext]}$_{\feinsub{Spkr}}$ {[\_]}$_{\feinsub{Top-DNI}}$  & 5 & \textit{gloat, comment}\\\midrule
%{[NP.Ext]}$_{\feinsub{Spkr}}$ {[NP.Obj]}$_{\feinsub{Addr}}$ {[PP]}$_{\feinsub{Msg}}$  & 5 & \textit{address, caution}\\\midrule
\lspbottomrule
    \end{tabularx}
    \caption{FrameNet valence patterns of \framename{Statement} verbs, their frequency in the FrameNet corpus and the verbs they appear with.}
    \label{tbl:statement-valence}
\end{table} 


\subsubsection{Syntactic realisation of the \framename{Statement} frame in Bulgarian}
\largerpage
The syntactic realisation of the frame element configurations in Bulgarian closely resembles that in English. The \fename{Speaker} is usually realised as the external NP and can be a person, a group or an organisation (Example \ref{ex:04statementbg:a}, \ref{ex:04statementbg:b}). In some cases the \fename{Medium} can take the position of the external argument (Example \ref{ex:04statementbg:c}). 

The \fename{Message} is either a finite clause %introduced with conjunctions \textit{че} `that', \textit{да} `to' or an interrogative word 
(Example \ref{ex:04statementbg:a}), an object NP (Example \ref{ex:04statementbg:b}) or a direct quote (Example \ref{ex:04statementbg:f}). The \fename{Topic} rarely occurs together with the \fename{Message}, and it is usually a modifier of the \fename{Message} (Example \ref{ex:04statementbg:d}). The non-core \fename{Addressee} is mostly optional and is realised as a prepositional complement (Example \ref{ex:04statementbg:b}).



\begin{exe}
\ex \label{ex:04statementbg}
\begin{xlist}
\ex  \label{ex:04statementbg:a}
\gll {[\textit{Панайотов}]}$_{\feinsub{Spkr}}$ \textit{\textbf{ДОБАВИ}}, [\textit{че} \textit{лидер} \textit{на} \textit{бъдещата} \textit{партия} \textit{ще} \textit{е} \textit{Симеон}]$_{\feinsub{Msg}}$.
\\
Panayotov added that leader of future-\textsc{def} party will be Simeon.
\\
\glt `Panayotov added that Simeon will be the leader of the future party.'
\ex  \label{ex:04statementbg:b}
 \gll {[}\textit{Кредитните} \textit{институции}{]}$_{\feinsub{Spkr}}$ \textit{\textbf{ДЕКЛАРИРАХА}} [\textit{пред} \textit{властите}]$_{\feinsub{Addr}}$ [\textit{нарасналите} \textit{печалби}]$_{\feinsub{Msg}}$. 
\\
Credit institutions declared to authorities-\textsc{def} increased-\textsc{def} profits.
\\
\glt `Credit institutions declared increased profits to the authorities.'
\ex  \label{ex:04statementbg:c}
\gll {[}\textit{Неофициалните} \textit{статистики} \textit{за} {\textit{1999 г.}}{]}$_{\feinsub{Med}}$ \textit{\textbf{СОЧАТ}} [\textit{5000} \textit{посетители}]$_{\feinsub{Msg}}$.
\\
Unofficial-\textsc{def} statistics for 1999 report 5000 visitors.
\\
\glt `The unofficial statistics for 1999 state 5,000 visitors.'
\ex  \label{ex:04statementbg:d}
\gll {[\textit{Тези} \textit{лица}]}$_{\feinsub{Spkr}}$ \textit{\textbf{ИЗКАЗВАТ}} [\textit{пред} \textit{нас}]$_{\feinsub{Addr}}$ [\textit{неприятни} \textit{истини}]$_{\feinsub{Msg}}$ [\textit{за} \textit{смъртните} \textit{ни} \textit{врагове}]$_{\feinsub{Top}}$.
\\
{These persons} state to us unpleasant truths about mortal-\textsc{def} our enemies.
\\
\glt `These people state to us unpleasant truths about our mortal enemies.'
\newpage
\ex  \label{ex:04statementbg:e}
\gll {[В \textit{интервюто}]}$_{\feinsub{Med}}$ [\textit{Симеон}]$_{\feinsub{Spkr}}$ \textit{\textbf{ОБЯВИ}} [\textit{промяна} \textit{на} \textit{политическата} \textit{посока}]$_{\feinsub{Msg}}$.
\\
{In interview-\textsc{def}} Simeon announced change of political-\textsc{def} direction.
\\
\glt `In the interview Simeon announced a change in the political direction.'
\ex  \label{ex:04statementbg:f}
\gll {[}-- \textit{Тя} \textit{го} \textit{каза} \textit{просто} \textit{така}{]}$_{\feinsub{Msg}}$ -- \textit{\textbf{ДОБАВИ}} [\textit{Джени}]$_{\feinsub{Spkr}}$.
\\
--  She it said just so -- added Jenny.
\\
\glt `-- She said it just like that -- added Jenny.'
\end{xlist}
\end{exe}


\tabref{tbl:statement-synt-bg} shows some of the most frequent verbs in Bulgarian evoking the frame \framename{Statement}. The Bulgarian examples show similar patterns to the realisation of frame elements of the examples in the English dataset.

\tabref{tbl:statement-valence-bg} presents the most frequent valence patterns typical of the verbs evoking the \framename{Statement} frame in Bulgarian. Like in English, the most preferred realisations involve a subject \fename{Speaker} and a \fename{Message} expressed as an object NP, a clause or a quote.


\begin{table}
\centering\footnotesize
\begin{tabular}{l rrrrrrrrr}
\lsptoprule
 & NP.Ext & NP.Obj & PP & AVP & NI & Clause & Quote & Other & Total\\
\midrule
\multicolumn{10}{l}{\textit{обявявам\slash обявя} }\\
`announce'\\
\fename{Speaker} & 17 &  &  &  & 1 &  &  &  & 18\\ 
\fename{Message} &  & 4 &  &  &  & 12 & 1 & 1 & 18\\ 

\midrule
\multicolumn{10}{l}{\textit{твърдя} }\\ 
`claim'\\
\fename{Speaker} & 11 &  &  &  & 1 &  &  &  & 12\\ 
\fename{Message} &  &  &  &  &  & 10 & 2 &  & 12\\ 

\midrule
%\multicolumn{10}{l}{\textit{предлагам\slash предложа} }\\  
%\fename{Communicator} & 8 &  &  &  &  &  &  &  & 8\\ 
%\fename{Message} &  & 1 &  &  &  & 7 &  &  & 8\\ 
%\fename{Addressee} &  &  & 3 &  &  &  &  &  & 3\\ 
% \midrule
%\multicolumn{10}{l}{\textit{посочвам\slash посоча} }\\  
%\fename{Communicator} & 5 &  &  &  & 2 &  &  &  & 7\\ 
%\fename{Message} & 1 & 3 &  &  &  & 3 &  &  & 7\\ 
% \midrule
\multicolumn{10}{l}{\textit{коментирам} }\\  
`comment'\\
\fename{Speaker} & 8 &  &  &  &  &  &  &  & 8\\ 
\fename{Message} &  & 4 &  &  &  & 2 & 1 & 1 & 8\\ 

\midrule
%\multicolumn{10}{l}{\textit{оповестявам\slash оповестя} }\\  
%\fename{Communicator} & 1 &  &  &  &  &  &  &  & 1\\ 
%\fename{Message} &  & 1 &  &  &  &  &  &  & 1\\ 
% \midrule
%\multicolumn{10}{l}{\textit{отбелязвам\slash отбележа} }\\  
%\fename{Communicator} & 6 &  &  &  & 1 &  &  &  & 7\\ 
%\fename{Message} &  &  &  &  &  & 5 & 2 &  & 7\\ 
% \midrule
\multicolumn{10}{l}{\textit{добавям\slash добавя} }\\  
`add'\\
\fename{Speaker} & 10 &  &  &  &  &  &  &  & 10\\ 
\fename{Message} &  &  &  &  &  & 5 & 5 &  & 10\\ 

\midrule
\multicolumn{10}{l}{\textit{съобщавам\slash съобщя} }\\  
`announce'\\
\fename{Speaker} & 10 &  &  &  & 1 &  &  &  & 11\\ 
\fename{Message} &  & 4 &  &  &  & 2 & 5 &  & 11\\ 
\fename{Addressee} &  &  & 1 &  &  &  &  &  & 1\\ 

\midrule
\multicolumn{10}{l}{\textit{казвам\slash кажа} }\\
`say'\\
\fename{Speaker} & 47 &  &  &  & 1 &  &  &  & 48\\ 
\fename{Message} & 1 & 10 &  &  &  & 18 & 19 &  & 48\\ 
\fename{Addressee} &  &  & 4 &  &  &  &  &  & 4\\ 

\midrule
%\multicolumn{10}{l}{\textit{повтарям\slash повторя} }\\  
%\fename{Communicator} & 3 &  &  &  &  &  &  &  & 3\\ 
%\fename{Message} &  & 2 &  &  &  &  & 1 &  & 3\\ 
% \midrule
\multicolumn{10}{l}{\textit{обяснявам\slash обясня} }\\
`explain'\\
\fename{Speaker} & 14 &  &  &  & 2 &  &  &  & 16\\ 
\fename{Message} & 1 & 2 &  &  & 1 & 5 & 6 & 1 & 16\\ 
\fename{Addressee} &  &  & 6 &  &  &  &  &  & 6\\ 

\midrule
\multicolumn{10}{l}{\textit{заявявам\slash заявя} }\\
`state'\\
\fename{Speaker} & 17 &  &  &  &  &  &  &  & 17\\ 
\fename{Message} &  &  &  &  &  & 10 & 7 &  & 17\\ 
\fename{Addressee} &  &  & 4 &  &  &  &  &  & 4\\ 

\lspbottomrule
%\multicolumn{10}{l}{\textit{пиша} }\\  
%\fename{Communicator} & 6 &  &  &  &  &  &  &  & 6\\ 
%\fename{Message} &  &  &  &  &  & 1 & 5 &  & 6\\ 
%\fename{Topic} &  &  & 1 &  &  &  &  &  & 1\\ 
% \midrule
 \end{tabular}
 \caption{Syntactic expression of the \framename{Statement} frame elements in Bulgarian lexical units.  } 
    \label{tbl:statement-synt-bg}
 \end{table}

  \begin{table}
    \centering\footnotesize
    \begin{tabularx}{\textwidth}{ lrQ }
\lsptoprule
         Pattern  & \#  & verbs \\
\midrule
{[NP.Ext]}$_{\feinsub{Spkr}}$ {[Clause]}$_{\feinsub{Msg}}$ & 67 & \textit{добавям\slash добавя, заявявам\slash заявя, казвам\slash кажа, коментирам, обявявам\slash обявя, обяснявам\slash обясня, отбелязвам\slash отбележа, пиша, посочвам\slash посоча, предлагам\slash предложа, твърдя}\\

{[NP.Ext]}$_{\feinsub{Spkr}}$ {[Quote]}$_{\feinsub{Msg}}$ & 48 & \textit{добавям\slash добавя, заявявам\slash заявя, казвам\slash кажа, коментирам, обявявам\slash обявя, обяснявам\slash обясня, отбелязвам\slash отбележа, пиша, повтарям\slash повторя, съобщавам\slash съобщя, твърдя}\\

{[NP.Ext]}$_{\feinsub{Spkr}}$ {[NP.Obj]}$_{\feinsub{Msg}}$ & 29 & \textit{казвам\slash кажа, коментирам, обявявам\slash обявя, оповестявам\slash оповестя, повтарям\slash повторя, посочвам\slash посоча, предлагам\slash предложа, съобщавам\slash съобщя}\\

{[NP.Ext]}$_{\feinsub{Spkr}}$ {[Clause]}$_{\feinsub{Msg}}$  {[PP]}$_{\feinsub{Addr}}$ & 9 & \textit{заявявам\slash заявя, обяснявам\slash обясня, предлагам\slash предложа, съобщавам\slash съобщя}\\

{[NP.Ext]}$_{\feinsub{Spkr}}$ {[PP]}$_{\feinsub{Addr}}$  {[Quote]}$_{\feinsub{Msg}}$ & 5 & \textit{заявявам\slash заявя, казвам\slash кажа}\\

%{[Clause]}$_{\feinsub{Msg}}$ {[\_]}$_{\feinsub{Com-INI}}$ & 4 & \textit{отбелязвам\slash отбележа, посочвам\slash посоча, съобщавам\slash съобщя, твърдя}\\

%{[NP.Ext]}$_{\feinsub{Msg}}$ {[\_]}$_{\feinsub{Com-INI}}$ & 2 & \textit{казвам\slash кажа, посочвам\slash посоча}\\

{[NP.Ext]}$_{\feinsub{Spkr}}$ {[NP.Obj]}$_{\feinsub{Msg}}$   {[PP]}$_{\feinsub{Addr}}$ & 2 & \textit{обяснявам\slash обясня}\\

%{[NP.Ext]}$_{\feinsub{Msg}}$ {[PP]}$_{\feinsub{Addr}}$ {[\_]}$_{\feinsub{Com-INI}}$ & 1 & \textit{обяснявам\slash обясня}\\

%{[Sinterrog]}$_{\feinsub{Msg}}$ {[\_]}$_{\feinsub{Com-INI}}$ & 1 & \textit{обяснявам\slash обясня}\\

%{[NP.Ext]}$_{\feinsub{Com}}$ {[PP]}$_{\feinsub{Addr}}$ {[\_]}$_{\feinsub{Msg-INI}}$ & 1 & \textit{обяснявам\slash обясня}\\

%{[NP.Ext]}$_{\feinsub{Com}}$ {[Sinterrog]}$_{\feinsub{Msg}}$ & 1 & \textit{коментирам}\\

%{[NP.Ext]}$_{\feinsub{Com}}$ {[PP]}$_{\feinsub{Top}}$ {[Quote]}$_{\feinsub{Msg}}$ & 1 & \textit{пиша}\\

%{[Clause.Ext]}$_{\feinsub{Msg}}$ {[\_]}$_{\feinsub{Com-INI}}$ & 1 & \textit{обявявам\slash обявя}\\
\lspbottomrule
    \end{tabularx}
    \caption{FrameNet valence patterns of \framename{Statement} verbs, their frequency in the Bulgarian dataset and the verbs they appear with.
     English translation equivalents: \textit{добавям\slash добавя} `add', \textit{заявявам\slash заявя} `state', \textit{казвам\slash кажа} `say', \textit{коментирам} `comment', \textit{обявявам\slash обявя, оповестявам\slash оповестя, съобщавам\slash съобщя} `announce', \textit{обяснявам, обясня} `explain', \textit{отбелязвам\slash отбележа} `note', \textit{пиша} `write', \textit{повтарям\slash повторя} `reiterate', \textit{посочвам\slash посоча} `state', \textit{предлагам\slash предложа} `suggest'.}
    \label{tbl:statement-valence-bg}
\end{table} 


\subsection{Frame \framename{Telling}}

\begin{description}[font=\normalfont]
\item[The definition of the \framename{Telling} frame is:] A \fename{Speaker} addresses an \fename{Addressee} with  a \fename{Message}, which may be indirectly referred to as a \fename{Topic}. Instead of (or in addition to) a \fename{Speaker}, a \fename{Medium} may also be mentioned. Core frame elements: \fename{Speaker}, \fename{Addressee}, \fename{Message}, \fename{Medium}, \fename{Topic}.
\end{description}

The frame \framename{Telling} is evoked by a small number of frequently occurring verbs such as \textit{tell}.v, \textit{advise}.v, \textit{inform}.v, \textit{notify}.v, etc. The frame inherits from \framename{Statement} and its specialisation consists in the fact that it describes speech acts directed to a specific \fename{Addressee}. As a result this frame element is promoted to core status and with most verbs (\textit{inform}.v, \textit{advise}.v, \textit{confide}.v, \textit{notify}.v) is favoured for the direct object position. 

\subsubsection{Syntactic realisation of the \framename{Telling} frame elements}

The frame elements generally have the same characteristics as the ones in the \framename{Statement} frame from which they are inherited. The \fename{Speaker} usually takes the position of the external NP (Example \ref{ex:05tell:a}). Most often the \fename{Addressee} is expressed as an NP object (Example \ref{ex:05tell:b}) or in the case of \textit{tell}.v as an indirect object NP or a PP. %Both the \fename{Speaker} and the \fename{Addressee} are sentient beings able to produce and perceive language, respectively. They can also be groups or organisations (Example \ref{ex:05tell:b}).

\largerpage
The \fename{Message} is most often realised as a prepositional phrase, a subordinate clause or a quote (Example \ref{ex:05tell:b}, \ref{ex:05tell:c}, \ref{ex:05tell:a}, respectively). It may also take the position of an NP object, while the \fename{Addressee} is represented by a PP (Example \ref{ex:05tell:d}), a pattern which is actually favoured by the verb \textit{confide}.v. %A limited number of verbs, such as \textit{tell}, allow both the \fename{Message} and the \fename{Addressee} to assume the position of direct objects (Example \ref{ex:05tell:e}). 
Instead of the \fename{Message}, its \fename{Topic} may be realised as a prepositional phrase (Example \ref{ex:05tell:f}). 

\begin{exe}
\ex \label{ex:05tell}
\begin{xlist}
\ex  \label{ex:05tell:a}
\glt {[}`\textit{Take your bag and go},'{]}$_{\feinsub{Msg}}$ [\textit{Jake}]$_{\feinsub{Spkr}}$ \textit{\textbf{TOLD}} [\textit{her}]$_{\feinsub{Addr}}$.
\ex  \label{ex:05tell:b}
\glt {[}\textit{The police}{]}$_{\feinsub{Spkr}}$ \textit{didn't \textbf{INFORM}} [\textit{the British Consulate}]$_{\feinsub{Addr}}$ \newline [\textit{about his disappearance}]$_{\feinsub{Msg}}$. 
 \ex  \label{ex:05tell:c}
\glt {[}\textit{We}{]} \textit{have \textbf{NOTIFIED}} [\textit{Benoit}]$_{\feinsub{Addr}}$ [\textit{that Tweed is wanted}]$_{\feinsub{Msg}}$.
 \ex  \label{ex:05tell:d}
{[}\textit{She}{]}$_{\feinsub{Spkr}}$ \textit{\textbf{CONFIDED}} [\textit{her sadness}]$_{\feinsub{Msg}}$ [\textit{in Beth}]$_{\feinsub{Addr}}$.
 %\ex  \label{ex:05tell:e}
%\glt {[}\textit{I}{]}$_{\feinsub{Spkr}}$ \textit{told} [\textit{him}]$_{\feinsub{Addr}}$  [\textit{a bedtime story}]$_{\feinsub{Msg}}$.
 \ex  \label{ex:05tell:f}
\glt {[}\textit{He}{]}$_{\feinsub{Spkr}}$ \textit{will \textbf{ADVISE}} [\textit{you}]$_{\feinsub{Addr}}$ [\textit{on the inheritance tax}]$_{\feinsub{Top}}$.
\end{xlist}
\end{exe}



The various specific configurations of frame elements as expressed by verbs in the \framename{Telling} frame are shown in \tabref{tbl:telling-synt}.


\begin{table}
\centering\footnotesize
\begin{tabular}{l rrrrrrrrr}
\lsptoprule
 & NP.Ext & NP.Obj & PP & AVP & NI & Clause & Quote & Other & Total\\ 

\midrule
 \multicolumn{10}{l}{\textit{tell} } \\  
\fename{Speaker} & 90  & 1  & 9  &  & 14  &  &  &  & 114\\ 
\fename{Addressee} & 18  & 59  & 3  & 1  & 36  & 1  &  & 2 & 120\\ 
\fename{Topic} &  & 3  & 31  &  &  & 4  &  & 1 & 39\\ 
\fename{Message} & 5  & 11  & 9  & 3  & 13  & 35  & 6  & 8 & 90\\ 
\fename{Medium} & 10  &  & 2  &  &  &  &  &  & 12\\ 

\midrule
\multicolumn{10}{l}{\textit{inform} } \\  
\fename{Speaker} & 39  &  &  &  & 8  &  &  &  & 47\\ 
\fename{Addressee} & 8  & 37  &  &  & 2  &  &  &  & 47\\ 
\fename{Message} &  &  & 10  &  & 7  & 20  & 6  &  & 43\\ 
\fename{Medium} &  &  & 3  &  &  &  &  &  & 3\\ 
\fename{Topic} &  &  & 4  &  &  &  &  &  & 4\\ 

\midrule
\multicolumn{10}{l}{\textit{advise} } \\  
\fename{Speaker} & 59  &  & 1  &  & 6  &  &  &  & 66\\ 
\fename{Addressee} & 8  & 31  & 1  &  & 27  &  &  &  & 67\\ 
\fename{Message} &  & 3  & 7  &  &  & 29  & 8  &  & 47\\ 
\fename{Topic} &  &  & 19  &  & 1  &  &  &  & 20\\ 

\midrule
%\multicolumn{10}{l}{\textit{assure} } \\  
%\fename{Speaker} & 11  &  & 4  &  & 5  &  &  &  & 20\\ 
%\fename{Addressee} & 9  & 6  &  &  & 5  &  &  &  & 20\\ 
%\fename{Message} &  &  &  &  & 1  & 11  & 6  & 1 & 19\\ 
%\fename{Topic} &  &  &  &  & 1  &  &  &  & 1\\ 
% \midrule
\multicolumn{10}{l}{\textit{confide} } \\  
\fename{Speaker} & 45  &  &  &  & 1  &  &  &  & 46\\ 
\fename{Addressee} &  &  & 23  &  & 23  &  &  &  & 46\\ 
\fename{Message} & 1  & 23  &  &  & 4  & 14  & 4  &  & 46\\ 
\fename{Medium} &  &  & 1  &  &  &  &  &  & 1\\ 

\lspbottomrule
%\multicolumn{10}{l}{\textit{notify} } \\  
%\fename{Speaker} & 22  &  & 1  &  & 7  &  &  &  & 30\\ 
%\fename{Addressee} & 6  & 18  & 4  &  & 2  &  &  &  & 30\\ 
%\fename{Message} & 2  & 4  & 4  &  & 12  & 7  &  &  & 29\\ 
%\fename{Medium} &  &  & 5  &  &  &  &  &  & 5\\ 
%\fename{Topic} &  &  & 1  &  &  &  &  &  & 1\\ 
% \midrule
 \end{tabular}
 \caption{Syntactic expression of the \framename{Telling} frame elements in selected FrameNet lexical units. } 
    \label{tbl:telling-synt}
 \end{table}



\subsubsection{\framename{Telling} valence patterns}

The prevalent valence patterns for the verbs in the FrameNet frame \framename{Telling} are illustrated in \tabref{tbl:telling-valence}. These include the prototypical expression of the \fename{Speaker} as the external NP, usually with a direct object \fename{Addressee}, which may be left implicit and/or a \fename{Message} realised as a subordinate clause, a prepositional phrase or a quote; the \fename{Message} may also be implicit. A PP \fename{Topic} may co-occur with the \fename{Addressee} but usually not with the \fename{Message}.


\begin{table}
    \centering\footnotesize
    \begin{tabularx}{\textwidth}{ lrQ }
\lsptoprule
         Pattern  & \#  & verbs \\
\midrule
{[NP.Ext]}$_{\feinsub{Spkr}}$ {[NP.Obj]}$_{\feinsub{Addr}}$ {[Clause]}$_{\feinsub{Msg}}$  & 53 & \textit{inform, advise, tell, assure, notify}\\

{[NP.Ext]}$_{\feinsub{Spkr}}$ {[NP.Obj]}$_{\feinsub{Addr}}$ {[PP]}$_{\feinsub{Top}}$  & 30 & \textit{apprise, inform, advise, tell, notify}\\

{[NP.Ext]}$_{\feinsub{Spkr}}$ {[\_]}$_{\feinsub{Addr-DNI}}$ {[Clause]}$_{\feinsub{Msg}}$  & 26 & \textit{advise, confide, tell, assure}\\

{[NP.Ext]}$_{\feinsub{Spkr}}$ {[NP.Obj]}$_{\feinsub{Addr}}$ {[\_]}$_{\feinsub{Msg-DNI}}$  & 20 & \textit{inform, tell, assure, notify}\\

{[NP.Ext]}$_{\feinsub{Spkr}}$ {[NP.Obj]}$_{\feinsub{Addr}}$ {[PP]}$_{\feinsub{Msg}}$  & 20 & \textit{inform, advise, tell, notify}\\

{[NP.Ext]}$_{\feinsub{Spkr}}$ {[\_]}$_{\feinsub{Addr-DNI}}$ {[PP]}$_{\feinsub{Top}}$  & 17 & \textit{advise, tell}\\

{[NP.Ext]}$_{\feinsub{Spkr}}$ {[\_]}$_{\feinsub{Addr-DNI}}$ {[NP.Obj]}$_{\feinsub{Msg}}$  & 16 & \textit{advise, confide, tell}\\

{[NP.Ext]}$_{\feinsub{Spkr}}$ {[PP]}$_{\feinsub{Addr}}$ {[NP.Obj]}$_{\feinsub{Msg}}$  & 16 & \textit{advise, confide, tell, notify}\\

{[NP.Ext]}$_{\feinsub{Spkr}}$ {[\_]}$_{\feinsub{Addr-DNI}}$ {[Quote]}$_{\feinsub{Msg}}$  & 14 & \textit{advise, confide, assure}\\

{[NP.Ext]}$_{\feinsub{Spkr}}$ {[NP.Obj]}$_{\feinsub{Addr}}$ {[Quote]}$_{\feinsub{Msg}}$  & 11 & \textit{inform, tell, assure}\\

%{[NP.Ext]}$_{\feinsub{Spkr}}$ {[PP]}$_{\feinsub{Addr}}$ {[Clause]}$_{\feinsub{Msg}}$  & 4 & \textit{confide}\\

%{[NP.Ext]}$_{\feinsub{Spkr}}$ {[PP]}$_{\feinsub{Addr}}$ {[\_]}$_{\feinsub{Msg-INI}}$  & 4 & \textit{confide}\\

%{[NP.Ext]}$_{\feinsub{MEDIUM}}$ {[NP.Obj]}$_{\feinsub{Addr}}$ {[Clause]}$_{\feinsub{Msg}}$  & 4 & \textit{tell}\\

%{[NP.Ext]}$_{\feinsub{Spkr}}$ {[\_]}$_{\feinsub{Addr-DNI}}$ {[PP]}$_{\feinsub{Msg}}$  & 4 & \textit{advise, tell}\\

%{[NP.Ext]}$_{\feinsub{Spkr}}$ {[NP.Obj]}$_{\feinsub{Addr}}$ {[NP.Dep]}$_{\feinsub{Msg}}$  & 3 & \textit{tell, assure}\\
\lspbottomrule
\end{tabularx}
    \caption{FrameNet valence patterns of \framename{Telling} verbs, their frequency in the FrameNet corpus and the verbs they appear with.}
    \label{tbl:telling-valence}
\end{table} 



\subsubsection{Syntactic realisation and patterns in Bulgarian}
\largerpage
In a similar manner, in Bulgarian the \fename{Speaker} is realised as the external subject NP, while the \fename{Message} takes the position of an object NP, a subordinate clause or a quote (Example \ref{ex:05tellbg:a}, \ref{ex:05tellbg:b}, \ref{ex:05tellbg:c}). 

With some of the verbs in this frame, such as \textit{казвам, съобщавам} `tell, let know' the \fename{Addressee} assumes the position of the indirect object as the receiver to whom the message is directed (Example \ref{ex:05tellbg:b}), while with verbs such as \textit{уведомявам} `notify, inform', \textit{информирам, осведомявам} `inform' it is realised as an NP object (Example \ref{ex:05tellbg:d}); the \fename{Addressee} may also be null instantiated %considered as generalised public or audience and can be an indefinite null instantiation 
(Example \ref{ex:05tellbg:e}). 

\largerpage
\begin{exe}
\ex \label{ex:05tellbg}
\begin{xlist}
\ex  \label{ex:05tellbg:a}
\gll  {[}\_{]}$_{\feinsub{Spkr-DNI}}$ \textit{Искам} \textit{да} [\textit{ви}]$_{\feinsub{Addr}}$ \textit{\textbf{СЪОБЩЯ}} [\textit{една} \textit{тъжна} \textit{вест}]$_{\feinsub{Msg}}$. \\
{} Want.1sg to you.2pl-DAT tell one sad news.
\\
\glt `I want to tell you some sad news.'
\ex  \label{ex:05tellbg:b}
\gll {[}\textit{Всеки} \textit{българин}{]}$_{\feinsub{Spkr}}$ \textit{ще} [\textit{ти}]$_{\feinsub{Addr}}$ \textit{\textbf{КАЖЕ}} [\textit{каквото} \textit{е} \textit{чул} \textit{от} \textit{майка} \textit{си}]$_{\feinsub{Msg}}$. \\
Every Bulgarian will you.2sg-DAT tell whatever has heard from mother  REFL.
\\
\glt `Every Bulgarian will tell you whatever he has heard from his mother.'
\ex  \label{ex:05tellbg:c}
\gll  {[}\textit{Не} \textit{са} \textit{намерили} \textit{Санса}{]}$_{\feinsub{Msg}}$ -- \textit{учтиво} [\textit{го}]$_{\feinsub{Addr}}$ \textit{\textbf{УВЕДОМИ}} [\textit{чичо} \textit{му}]$_{\feinsub{Spkr}}$.\\
Not have found Sansa -- politely him informed uncle his.
\\
\glt `They have not found Sansa -- his uncle informed him politely.'
\ex  \label{ex:05tellbg:d}
\gll  {[}\_{]}$_{\feinsub{Spkr-DNI}}$ \textit{Трябва} \textit{да} \textit{\textbf{ОСВЕДОМЯ}} [\textit{читателя}]$_{\feinsub{Addr}}$ [\textit{за} \textit{тайната} \textit{интрига}]$_{\feinsub{Top}}$.
\\
{} Need.1sg to inform reader-\textsc{def} about secret-\textsc{def} plot.
\\
\glt `I need to inform the reader about the secret plot.'
\ex  \label{ex:05tellbg:e}
\gll {[}\textit{Пенсионерите} \textit{да} \textit{избягват} \textit{навалиците}{]}$_{\feinsub{Msg}}$, \textit{\textbf{СЪВЕТВА}}  [\textit{г-жа} \textit{Ненова}]$_{\feinsub{Spkr}}$ [\_]$_{\feinsub{Addr-INI}}$.\\
Elderly-\textsc{def} to avoid crowds,  advises  Mrs Nenova. {}
\\
\glt `The elderly should avoid crowds, Mrs Nenova advises.' 
\end{xlist}
\end{exe}

\begin{table}
\centering\footnotesize
\begin{tabular}{l rrrrrrrrr}
\lsptoprule
 & NP.Ext & NP.Obj & PP & AVP & NI & Clause & Quote & Other & Total\\ 

\midrule
\multicolumn{10}{l}{\textit{уверявам\slash уверя} }\\
`assure'\\
\fename{Message} &  &  & 1 &  &  & 24 & 6 &  & 31\\ 
\fename{Addressee} &  & 31 &  &  &  &  &  &  & 31\\ 
\fename{Speaker} & 31 &  &  &  &  &  &  &  & 31\\ 

\midrule
%\multicolumn{10}{l}{\textit{осведомявам\slash осведомя} }\\  
%\fename{Message} &  &  &  &  & 2 & 2 &  &  & 4\\ 
%\fename{Addressee} &  & 4 &  &  &  &  &  &  & 4\\ 
%\fename{Topic} &  &  & 2 &  &  &  &  &  & 2\\ 
%\fename{Speaker} & 4 &  &  &  &  &  &  &  & 4\\ 
% \midrule
\multicolumn{10}{l}{\textit{съобщавам\slash съобщя} }\\
`tell, let know'\\
\fename{Message} &  & 3 &  &  &  & 2 &  &  & 5\\ 
\fename{Addressee} &  &  & 5 &  &  &  &  &  & 5\\ 
\fename{Speaker} & 5 &  &  &  &  &  &  &  & 5\\ 

\midrule
\multicolumn{10}{l}{\textit{уведомявам\slash уведомя} }\\ 
`inform, notify'\\
\fename{Message} &  &  &  &  & 5 & 5 & 3 &  & 13\\ 
\fename{Addressee} & 1 & 15 &  &  &  &  &  &  & 16\\ 
\fename{Topic} &  &  & 4 &  &  &  &  &  & 4\\ 
\fename{Speaker} & 15 &  &  &  & 1 &  &  &  & 16\\ 

\midrule
\multicolumn{10}{l}{\textit{казвам\slash кажа} }\\  
`tell'\\
\fename{Message} &  & 11 &  &  &  & 15 & 6 &  & 32\\ 
\fename{Addressee} &  &  & 32 &  &  &  &  &  & 32\\ 
\fename{Speaker} & 32 &  &  &  &  &  &  &  & 32\\ 

\lspbottomrule
%\multicolumn{10}{l}{\textit{посъветвам} }\\  
%\fename{Message} &  &  &  &  &  & 3 & 2 &  & 5\\ 
%\fename{Addressee} &  & 3 &  &  &  &  &  &  & 3\\ 
%\fename{Speaker} & 5 &  &  &  &  &  &  &  & 5\\ 
% \midrule
 \end{tabular}
 \caption{Syntactic expression of the \framename{Telling} frame elements in Bulgarian. } 
    \label{tbl:telling-synt-bg}
 \end{table}

\tabref{tbl:telling-synt-bg} presents the most frequent verbs in Bulgarian evoking the frame \framename{Telling}, while
\tabref{tbl:telling-valence-bg} shows the typical valence patterns. The \fename{Message} and the \fename{Addressee} tend to co-occur syntactically, while the \fename{Topic} is expressed more rarely.


\begin{table}
    \centering\footnotesize
    \begin{tabularx}{\textwidth}{ lr Q}
\lsptoprule
         Pattern  & \#  & verbs \\
\midrule
{[NP.Ext]}$_{\feinsub{Spkr}}$ {[Clause]}$_{\feinsub{Msg}}$ {[NP.Obj]}$_{\feinsub{Addr}}$ & 32 & \textit{осведомявам\slash осведомя, уверявам\slash уверя, уведомявам\slash уведомя}\\

{[NP.Ext]}$_{\feinsub{Spkr}}$ {[Clause]}$_{\feinsub{Msg}}$ {[PP]}$_{\feinsub{Addr}}$ & 15 & \textit{казвам\slash кажа, съобщавам\slash съобщя}\\

{[NP.Ext]}$_{\feinsub{Spkr}}$ {[NP.Obj]}$_{\feinsub{Msg}}$ {[PP]}$_{\feinsub{Addr}}$ & 14 & \textit{казвам\slash кажа, съобщавам\slash съобщя}\\

{[NP.Ext]}$_{\feinsub{Spkr}}$ {[NP.Obj]}$_{\feinsub{Addr}}$ {[Quote]}$_{\feinsub{Msg}}$ & 9 & \textit{уверявам\slash уверя, уведомявам\slash уведомя}\\

{[NP.Ext]}$_{\feinsub{Spkr}}$ {[PP]}$_{\feinsub{Addr}}$ {[Quote]}$_{\feinsub{Msg}}$ & 6 & \textit{казвам\slash кажа}\\

{[NP.Ext]}$_{\feinsub{Spkr}}$ {[NP.Obj]}$_{\feinsub{Addr}}$ {[PP]}$_{\feinsub{Top}}$  {[\_]}$_{\feinsub{Msg-INI}}$ & 5 & \textit{осведомявам\slash осведомя, уведомявам\slash уведомя}\\

{[NP.Ext]}$_{\feinsub{Spkr}}$ {[Quote]}$_{\feinsub{Msg}}$  {[\_]}$_{\feinsub{Addr-INI}}$ & 4 & \textit{информирам, посъветвам}\\

%{[NP.Ext]}$_{\feinsub{Spkr}}$ {[NP.Obj]}$_{\feinsub{Addr}}$ {[]}$_{\feinsub{Msg}}$ & 3 & \textit{уведомявам\slash уведомя}\\
\lspbottomrule
    \end{tabularx}
    \caption{FrameNet valence patterns of the frame \framename{Telling}, their frequency in the Bulgarian dataset and the verbs they appear with.
     English translation equivalents: \textit{информирам, осведомявам\slash осведомя} `inform', \textit{казвам, съобщавам\slash съобщя} `tell, let know', \textit{посъветвам} `advise', \textit{уведомявам\slash уведомя} `notify', \textit{уверявам\slash уверя} `assure'.}
    \label{tbl:telling-valence-bg}
\end{table} 



\subsection{Frame \framename{Judgment\_communication}}

\begin{description}[font=\normalfont]
\item[Definition of the frame \framename{Judgment\_communication}:] A \fename{Communicator} communicates a judgement of an \fename{Evaluee} to an \fename{Addressee}. The judgement may be positive (e.g. \textit{praise}.v) or negative (e.g. \textit{criticise}.v). Core frame elements: \fename{Communicator}, \fename{Expressor}, \fename{Reason}, \fename{Medium}, \fename{Topic}, \fename{Evaluee}; Non-core: \fename{Addressee}.
\end{description}

The frame \framename{Judgment\_communication} inherits from both the \framename{Statement} and the \framename{Judgment} frame (weak inheritance through the \FrameRelation{Uses} frame-to-frame relation). Verbs included in this frame concern acts of speech which also convey judgement on a certain topic, the \fename{Evaluee}. The frame elaborates on the frame \framename{Statement} most notably in the interpretation of the \fename{Message} as a judgement on a complex state-of-affairs concerning an additional participant, represented by the frame element \fename{Evaluee}. The \fename{Evaluee} can be a person, an object, an action or any topic (Example \ref{ex:06judgment:a}, \ref{ex:06judgment:b}, \ref{ex:06judgment:f}). The judgement may be positive, e.g. \textit{praise}.v, \textit{commend}.v, \textit{acclaim}.v, or negative, e.g. \textit{criticise}.v, \textit{condemn}.v, \textit{denounce}.v; its value is encoded by the verb. In addition, the frame element \fename{Reason} denotes the argumentation for the judgement. The \fename{Addressee} is a non-core frame element, reflecting the fact that the judgement regarding the \fename{Evaluee} may but need not be intended for another participant. %the \fename{Addressee}, or it can be expressed in general terms and hence null instantiated \fename{Addressee} (Example \ref{ex:06judgment:a}, \ref{ex:06judgment:d}).

\subsubsection{Syntactic realisation of the \framename{Judgment\_communication} frame elements}

%The syntactic expression of the  configuration of core frame elements in the frame \framename{Judgment\_communication} resembles the realisation of the verbs in the frame \framename{Statement} which it inherits. 

The frame \framename{Judgment\_communication} specifies the more general frame element \fename{Communicator} rather than inheriting the \fename{Speaker} from the \framename{Statement} frame. The reason for this is that the frame also includes verbs which represent communication acts that are more general or complex than speech acts, e.g. \textit{belittle}.v, \textit{ridicule}.v. 

The \fename{Communicator} is usually realised as the external argument and can be represented by a person, a group or an organisation  (Example \ref{ex:06judgment:a}, \ref{ex:06judgment:b}).  

%In this frame the message inherent in any communication act is reconceptualised as a construction involving the \fename{Evaluee}  and the \fename{Reason} for the judgement. 
The \fename{Evaluee} is most often expressed in the position of the NP direct object (Example \ref{ex:06judgment:a}, \ref{ex:06judgment:b}, \ref{ex:06judgment:c}), while the \fename{Reason} can be a prepositional phrase headed by prepositions such as \textit{for, of, as} (Example \ref{ex:06judgment:c}, \ref{ex:06judgment:d}, \ref{ex:06judgment:f}). Instead of the \fename{Reason}, a \fename{Topic} can be present (Example \ref{ex:06judgment:e}).

The \fename{Addressee}, whenever overt, is expressed as a prepositional phrase (Example \ref{ex:06judgment:b}). 

The \fename{Expressor} is rare with verbs evoking this frame and usually represents a body part or an action performed by the \fename{Communicator} in order to convey the judgment (Example \ref{ex:06judgment:g}). 

\begin{exe}
\ex \label{ex:06judgment}
\begin{xlist}
\ex  \label{ex:06judgment:a}
{[}\textit{Frank}{]}$_{\feinsub{Com}}$ \textit{\textbf{RIDICULED}} [\textit{everything}]$_{\feinsub{Eval}}$.
\ex  \label{ex:06judgment:b}
{[}\textit{Jon}{]}$_{\feinsub{Com}}$ \textit{\textbf{BELITTLED}} [\textit{Madie}]$_{\feinsub{Eval}}$ {[}\textit{to her colleagues}{]}$_{\feinsub{Addr}}$.
\ex  \label{ex:06judgment:c}
{[}\textit{Georgia}{]}$_{\feinsub{Com}}$ \textit{has \textbf{ACCUSED}} [\textit{Russian troops}]$_{\feinsub{Eval}}$ [\textit{of backing separa\-tists}]$_{\feinsub{Reas}}$.
\ex  \label{ex:06judgment:d}
{[}\textit{I}{]}$_{\feinsub{Com}}$ \textit{have \textbf{PRAISED}} [\textit{her}]$_{\feinsub{Eval}}$ [\textit{for her work}]$_{\feinsub{Reas}}$.
\ex  \label{ex:06judgment:e}
{[}\textit{He}{]}$_{\feinsub{Com}}$ \textit{\textbf{CRITICISED}} [\textit{the president}]$_{\feinsub{Eval}}$ [\textit{over his decision to go to \linebreak war}]$_{\feinsub{Top}}$.
\ex  \label{ex:06judgment:f}
{[}\textit{The conservatives}{]}$_{\feinsub{Com}}$ \textit{\textbf{DENOUNCED}} {[}\textit{the proposed reforms}{]}$_{\feinsub{Eval}}$ [\textit{as an attempt to distract voters}]$_{\feinsub{Reas}}$.
\ex  \label{ex:06judgment:g}
{[}\textit{His glance}{]}$_{\feinsub{Exr}}$ \textit{\textbf{DENIGRATED}} [\textit{her attempt at humour}]$_{\feinsub{Eval}}$.
\end{xlist}
\end{exe}

\tabref{tbl:judgment-synt} shows some of the frequent verbs of the frame and the realisation of their frame elements.

\begin{table}
\centering\footnotesize
\begin{tabular}{l rrrrrrrrr}
\lsptoprule
 & NP.Ext & NP.Obj & PP & AVP & NI & Clause & Quote & Other & Total\\ 

\midrule
%\multicolumn{10}{l}{\textit{accuse} } \\  
%\fename{Communicator} & 32  &  & 7  & 1  & 16  &  &  &  & 56\\ 
%\fename{Evaluee} & 24  & 32  &  &  &  &  &  &  & 56\\ 
%\fename{Reason} &  &  & 51  &  & 5  &  &  &  & 56\\ 
%\fename{Medium} &  &  & 1  &  &  &  &  &  & 1\\ 
% \midrule
%\multicolumn{10}{l}{\textit{commend} } \\  
%\fename{Communicator} & 29  &  & 3  &  & 5  &  &  &  & 37\\ 
%\fename{Addressee} &  &  & 3  &  &  &  &  &  & 3\\ 
%\fename{Evaluee} & 6  & 22  &  &  & 3  &  &  & 6 & 37\\ 
%\fename{Reason} & 2  & 7  & 21  &  & 5  &  &  &  & 35\\ 
% \midrule
\multicolumn{10}{l}{\textit{condemn} } \\  
\fename{Communicator} & 105  &  & 21  &  & 11  &  &  &  & 137\\ 
\fename{Evaluee} & 32  & 103  & 1  &  & 2  &    &    &  & 138\\ 
\fename{Medium} & 1  &  & 5  &  &  &  &  &  & 6\\ 
\fename{Reason} &  & 4  & 44  &  & 90  &   &  &  & 138\\ 

\midrule
\multicolumn{10}{l}{\textit{criticize} } \\  
\fename{Communicator} & 88  &  & 15  &  & 47  &  &  &  & 150\\ 
\fename{Addressee} &  &  & 1  &  &  &  &  &  & 1\\ 
\fename{Evaluee} & 74  & 71  &  &  & 9  &    &    & 1 & 155\\ 
\fename{Reason} &  & 4  & 87  &  & 59  &    &  &  & 150\\ 
\fename{Topic} &  & 1  & 3  &  &  &  &  &  & 4\\ 
\fename{Medium} & 5  &  & 3  &  &  &  &  &  & 8\\ 

\midrule
%\multicolumn{10}{l}{\textit{denounce} } \\  
%\fename{Communicator} & 45  &  & 16  &  & 12  &  &  &  & 73\\ 
%\fename{Addressee} &  &  & 4  &  &  &  &  &  & 4\\ 
%\fename{Evaluee} & 29  & 38  &  &  &  & 5  & 2  &  & 74\\ 
%\fename{Reason} &  &  & 39  & 1  & 24  & 2  &  &  & 66\\ 
%\fename{Medium} &  &  & 5  &  &  &  &  &  & 5\\ 
% \midrule
\multicolumn{10}{l}{\textit{praise} } \\  
\fename{Communicator} & 50  &  & 12  &  & 18  &  &  &  & 80\\ 
\fename{Evaluee} & 27  & 49  &  &  &  &    &    & 4 & 80\\ 
\fename{Reason} & 3  & 2  & 38  &  & 34  &    &  &  & 77\\ 
\fename{Medium} &  &  & 5  &  &  &  &  &  & 5\\ 

\midrule
\multicolumn{10}{l}{\textit{ridicule} } \\  
\fename{Communicator} & 14  &  & 16  &  & 16  &  &  &  & 46\\ 
\fename{Evaluee} & 38  & 9  &  &  &  &    &  &  & 47\\ 
\fename{Medium} & 1  &  & 2  &  &  &  &  &  & 3\\ 
\fename{Reason} &  & 1  & 13  &  & 33  &    &  &  & 47\\ 

\lspbottomrule
%\multicolumn{10}{l}{\textit{castigate} } \\  
%\fename{Evaluee} & 16  & 36  &  &  &  &  &  &  & 52\\ 
%\fename{Communicator} & 35  &  & 5  &  & 11  &  &  & 1 & 52\\ 
%\fename{Medium} &  &  & 1  &  &  &  &  &  & 1\\ 
%\fename{Reason} &  &  & 38  &  & 14  &  &  &  & 52\\ 
% \midrule
 \end{tabular}
 \caption{Syntactic expression of the \framename{Judgment\_communication} frame elements in selected FrameNet lexical units. } 
    \label{tbl:judgment-synt}
 \end{table}


\subsubsection{\framename{Judgment\_communication} valence patterns}

The valence patterns characteristic for verbs in the FrameNet frame \framename{Judgment\_\linebreak communication} are presented in \tabref{tbl:judgment-valence}. The most common ones involve a \fename{Communicator} as the external argument, a direct object NP \fename{Evaluee}, and an  either overtly expressed or implicit \fename{Reason} or much more rarely a \fename{Topic}.

\begin{table}
    \centering\footnotesize
    \begin{tabularx}{\textwidth}{ lrQ }
\lsptoprule
         Pattern  & \#  & verbs \\
\midrule
{[NP.Ext]}$_{\feinsub{Com}}$ {[NP.Obj]}$_{\feinsub{Eval}}$ {[PP]}$_{\feinsub{Reas}}$  & 263 & \textit{accuse, deprecate, denigrate, censure, castigate, condemn, ridicule, commend, belittle, denounce, praise, damn, criticize, execrate, mock}\\

{[NP.Ext]}$_{\feinsub{Com}}$ {[NP.Obj]}$_{\feinsub{Eval}}$   {[\_]}$_{\feinsub{Reas-DNI}}$  & 138 & \textit{accuse, deprecate, denigrate, censure, ridicule, commend, castigate, acclaim, belittle, condemn, denounce, praise, damn, criticize}\\

%{[NP.Ext]}$_{\feinsub{Com}}$ {[Clause]}$_{\feinsub{Eval}}$ {[\_]}$_{\feinsub{Reas-DNI}}$  & 38 & \textit{damn, deprecate, denigrate, ridicule, condemn, denounce, praise}\\

{[NP.Ext]}$_{\feinsub{Com}}$ {[NP.Obj]}$_{\feinsub{Eval}}$ {[\_]}$_{\feinsub{Reas-INI}}$  & 25 & \textit{criticize, denigrate, mock, castigate, condemn, denounce}\\

%{[NP.Ext]}$_{\feinsub{Com}}$ {[NP.Obj]}$_{\feinsub{Eval}}$ \newline {[Clause]}$_{\feinsub{Reas}}$  & 17 & \textit{criticize, denigrate, ridicule, condemn, denounce, praise}\\

%{[NP.Ext]}$_{\feinsub{Com}}$ {[\_]}$_{\feinsub{Eval-DNI}}$ {[\_]}$_{\feinsub{Reas-DNI}}$  & 15 & \textit{remonstrate, scoff}\\

%{[NP.Ext]}$_{\feinsub{Com}}$ {[NP.Obj]}$_{\feinsub{Eval}}$  & 14 & \textit{extol, commend, denounce, praise}\\

%{[NP.Ext]}$_{\feinsub{Com}}$ {[2nd]}$_{\feinsub{Eval}}$ {[NP.Obj]}$_{\feinsub{Reas}}$  & 13 & \textit{damn, criticize, denigrate, censure, commend, praise}\\

%{[NP.Ext]}$_{\feinsub{Com}}$ {[Clause]}$_{\feinsub{Eval}}$ {[\_]}$_{\feinsub{Reas-INI}}$  & 9 & \textit{criticize}\\

{[NP.Ext]}$_{\feinsub{Com}}$ {[NP.Obj]}$_{\feinsub{Eval}}$ {[PP]}$_{\feinsub{Top}}$  & 7 & \textit{slam, charge, criticize}\\
\lspbottomrule
    \end{tabularx}
    \caption{FrameNet valence patterns of \framename{Judgment\_communication} verbs, their frequency in the FrameNet corpus and the verbs they appear with.}
    \label{tbl:judgment-valence}
\end{table} 
 

\subsubsection{Syntactic realisation of the frame \framename{Judgement\_communication} in Bulgarian}

%The syntactic realisation of the verbs from the \framename{Judgement\_communication} frame in Bulgarian contains the same core frame elements. %and exhibits more varied valence patterns. 

The \fename{Communicator} is expressed as the external NP (Example \ref{ex:06judgmentbg:a}). The \fename{Evaluee} can be any concrete or abstract entity, quality, property, etc., whose properties are being evaluated, and is usually realised as the direct NP object (Example \ref{ex:06judgmentbg:a}, \ref{ex:06judgmentbg:b}) or as a prepositional phrase for a limited number of verbs such as \textit{подигравам се} `mock, ridicule' in (Example \ref{ex:06judgmentbg:e}). 

The \fename{Reason} is expressed as a prepositional phrase with a range of prepositions such as \textit{за, в, на} (Example \ref{ex:06judgmentbg:c}, \ref{ex:06judgmentbg:d}, \ref{ex:06judgmentbg:f}), or more rarely as a clause (Example \ref{ex:06judgmentbg:b}) or a direct quote (Example \ref{ex:06judgmentbg:g}). In some cases the \fename{Evaluee} and the \fename{Reason} can be expressed jointly (Example \ref{ex:06judgmentbg:f}).% na en syshto jointly i direct

The \fename{Addressee} is rarely expressed and is realised as a prepositional phrase (Example \ref{ex:06judgmentbg:g}).


\begin{exe}
\ex \label{ex:06judgmentbg}
\begin{xlist}
\ex  \label{ex:06judgmentbg:a}
\gll {[}\textit{Нашето} \textit{посолство}{]}$_{\feinsub{Com}}$ \textit{\textbf{ОСЪДИ}} [\textit{разрушаването} \textit{на} \textit{храма} \textit{в} \textit{Скопие}]$_{\feinsub{Eval}}$ {[}\_{]}$_{\feinsub{Reas-INI}}$. 
\\
Our embassy condemned destruction-\textsc{def} of church-\textsc{def} in Skopje. {}
 \\
 \glt `Our embassy condemned the destruction of the church in Skopje.'
\ex  \label{ex:06judgmentbg:b}
\gll {[}\_{]}$_{\feinsub{Com-DNI}}$ \textit{Не} \textit{могат} \textit{да} [\textit{ме}]$_{\feinsub{Eval}}$ \textit{\textbf{ОБВИНЯВАТ}}, [\textit{че} \textit{съм} \textit{ги} \textit{оскърбил}]$_{\feinsub{Reas}}$.
 \\
 {} Not can.3pl to me accuse that have them offended.
 \\
 \glt `They cannot accuse me of offending them.'
\ex  \label{ex:06judgmentbg:c}
\gll {[}\textit{България}{]}$_{\feinsub{Com}}$ [\textit{ни}]$_{\feinsub{Eval}}$ \textit{\textbf{ПРОКЛИНА}} [\textit{за} \textit{нещастията} \textit{си}]$_{\feinsub{Reas}}$.
 \\
Bulgaria us condemns for misfortunes-\textsc{def}  REFL.
 \\
 \glt `Bulgaria condemns us for its misfortunes.'
\ex  \label{ex:06judgmentbg:d}
\gll {[}\_{]}$_{\feinsub{Com-DNI}}$ \textit{\textbf{ОБВИНЯВАШЕ}} [\textit{ме}]$_{\feinsub{Eval}}$ [\textit{в} \textit{коравосърдечие}]$_{\feinsub{Reas}}$.
 \\
{} Accused me in cold-heartedness. {}
 \\
 \glt `He/she accused me of cold-heartedness.'
\ex  \label{ex:06judgmentbg:e}
\gll {[}\textit{Ти}{]}$_{\feinsub{Com}}$ \textit{\textbf{ПОДИГРАВАШ}} \textit{ли} \textit{\textbf{СЕ}} [\textit{с} \textit{мен}]$_{\feinsub{Eval}}$?
 \\
 You mock QST REFL with me?
 \\
 \glt `Are you mocking me?'
\ex  \label{ex:06judgmentbg:f}
\gll {[}\textit{Мускетарите}{]}$_{\feinsub{Com}}$ \textit{се} \textit{\textbf{ПОДИГРАВАХА}} [\textit{на} \textit{кривите} \textit{му} \textit{крака}]$_{\feinsub{Eval+Reas}}$.
 \\
Musketeers-\textsc{def}  REFL ridiculed for bow-\textsc{def} his legs.
 \\
 \glt `The musketeers ridiculed him for his bow legs.'
\ex  \label{ex:06judgmentbg:g}
\gll {[}--\textit{Много} \textit{е} \textit{наблюдателна}{]}$_{\feinsub{Reas}}$ -- \textit{\textbf{ПОХВАЛИ}} [\textit{я}]$_{\feinsub{Eval}}$ [\textit{той}]$_{\feinsub{Com}}$ [\textit{на} \textit{другите}]$_{\feinsub{Addr}}$.
 \\
 --Very is.3sg observant -- praised her he to others-\textsc{def}.
 \\
 \glt `-- She is very observant -- he praised her to the others.'
\end{xlist}
\end{exe}

\begin{table}
\centering\footnotesize
\begin{tabular}{l rrrrrrrrr}
\lsptoprule
 & NP.Ext & NP.Obj & PP & AVP & NI & Clause & Quote & Other & Total\\
\midrule
%\multicolumn{10}{l}{\textit{присмивам се} }\\  
%\fename{Communicator} & 6 &  &  &  &  &  &  &  & 6\\ 
%\fename{Evaluee} &  &  & 6 &  &  &  &  &  & 6\\ 
%\fename{Reason} &  &  & 1 &  &  &  &  &  & 1\\ 
% \midrule
\multicolumn{10}{l}{\textit{похвалвам\slash похваля} }\\  
`praise'\\
\fename{Communicator} & 16 &  &  &  &  &  &  &  & 16\\ 
\fename{Evaluee} &  & 15 &  &  & 1 &  &  &  & 16\\ 
\fename{Reason} &  &  & 2 &  &  &  &  &  & 2\\ 

\midrule
%\multicolumn{10}{l}{\textit{критикувам} }\\  
%\fename{Communicator} & 5 &  &  &  &  &  &  &  & 5\\ 
%\fename{Evaluee} &  & 5 &  &  &  &  &  &  & 5\\ 
% \midrule
%\multicolumn{10}{l}{\textit{хваля} }\\  
%\fename{Communicator} & 8 &  &  &  &  &  &  &  & 8\\ 
%\fename{Evaluee} &  & 8 &  &  &  &  &  &  & 8\\ 
%\fename{Reason} &  &  &  &  &  & 1 &  &  & 1\\ 
% \midrule
%\multicolumn{10}{l}{\textit{анатемосвам\slash анатемосам} }\\  
%\fename{Communicator} & 2 &  &  &  &  &  &  &  & 2\\ 
%\fename{Evaluee} &  & 1 &  &  & 1 &  &  &  & 2\\ 
% \midrule
%\multicolumn{10}{l}{\textit{отричам\slash отрека} }\\  
%\fename{Communicator} & 3 &  &  &  &  &  &  &  & 3\\ 
%\fename{Evaluee} &  & 3 &  &  &  &  &  &  & 3\\ 
% \midrule
%\multicolumn{10}{l}{\textit{проклинам\slash прокълна} }\\  
%\fename{Communicator} & 4 &  &  &  &  &  &  &  & 4\\ 
%\fename{Evaluee} &  & 4 &  &  &  &  &  &  & 4\\ 
%\fename{Reason} &  &  & 1 &  &  &  &  &  & 1\\ 
% \midrule
\multicolumn{10}{l}{\textit{обвинявам\slash обвиня} }\\  
`blame'\\
\fename{Communicator} & 12 &  &  &  &  &  &  &  & 12\\ 
\fename{Evaluee} &  & 11 &  &  & 1 &  &  &  & 12\\ 
\fename{Reason} &  &  & 5 & 1 &  & 2 &  &  & 8\\ 

\midrule
%\multicolumn{10}{l}{\textit{възхвалявам\slash възхваля} }\\  
%\fename{Communicator} & 3 &  &  &  &  &  &  &  & 3\\ 
%\fename{Evaluee} &  & 3 &  &  &  &  &  &  & 3\\ 
%\fename{Reason} &  &  &  &  &  &  &  & 1 & 1\\ 
% \midrule
\multicolumn{10}{l}{\textit{подигравам се\slash подиграя се} }\\  
`mock, ridicule'\\
\fename{Communicator} & 15 &  &  &  &  &  &  &  & 15\\ 
\fename{Medium} &  &  & 1 &  &  &  &  &  & 1\\ 
\fename{Evaluee} &  &   & 13 &  & 2 &  &  &  & 15\\ 
\fename{Reason} &  &   & 2 &  & 1 &  &  &  & 3\\ 

\lspbottomrule
 \end{tabular}
 \caption{Syntactic expression of the \framename{Judgment\_communication} frame elements in Bulgarian.} 
    \label{tbl:judgment-synt-bg}
 \end{table}
 
\subsubsection{\framename{Judgment\_communication} valence patterns in Bulgarian}

 The valence patterns for the Bulgarian verbs in this frame are presented in \tabref{tbl:judgment-valence-bg}. Similarly to English, the most typical ones involve the expression of the \fename{Communicator} and the \fename{Evaulee} and possibly the \fename{Reason}; in the dataset there have not been cases of \fename{Topic}.
 
\begin{table}[t]
\small
\begin{tabularx}{\textwidth}{ lrQ }
\lsptoprule
Pattern  & \#  & verbs \\
\midrule
{[NP.Ext]}$_{\feinsub{Com}}$ {[NP.Obj]}$_{\feinsub{Eval}}$ \newline {[\_]}$_{\feinsub{Reas-DNI/INI}}$  & 57 & \textit{величая, виня, възхвалявам\slash възхваля, иронизирам, клеветя, критикувам, кълна, обвинявам\slash обвиня, омаловажавам\slash омаловажа, осъждам\slash осъдя, отричам\slash отрека, подценявам\slash подценя, порицавам\slash порицая, похвалвам\slash похваля, прославям\slash прославя, славя, хваля}\\

{[NP.Ext]}$_{\feinsub{Com}}$ {[PP]}$_{\feinsub{Eval}}$  {[\_]}$_{\feinsub{Reas-DNI/INI}}$& 28 & \textit{гавря се, заяждам се\slash заям се, подигравам се\slash подиграя се, присмивам се\slash присмея се}\\

{[NP.Ext]}$_{\feinsub{Com}}$ {[NP.Obj]}$_{\feinsub{Eval}}$   {[PP]}$_{\feinsub{Reas}}$ & 12 & \textit{заклеймявам\slash заклеймя, иронизирам, обвинявам\slash обвиня, подигравам\slash подиграя, порицавам\slash порицая, похвалвам\slash похваля, проклинам\slash прокълна}\\

%{[NP.Ext]}$_{\feinsub{Com}}$ {[\_]}$_{\feinsub{Eval-INI}}$ & 8 & \textit{похвалвам\slash похваля, подигравам\slash подиграя, обвинявам\slash обвиня, заяждам се\slash заям се, заклеймявам\slash заклеймя}\\

%{[NP.Ext]}$_{\feinsub{Com}}$ {[Clause]}$_{\feinsub{Reas}}$ {[NP.Obj]}$_{\feinsub{Eval}}$ & 3 & \textit{обвинявам\slash обвиня, хваля}\\

%{[NP.Ext]}$_{\feinsub{Com}}$ {[PP]}$_{\feinsub{Eval}}$ {[PP]}$_{\feinsub{Reas}}$ & 2 & \textit{подигравам се, присмивам се}\\

%{[NP.Ext]}$_{\feinsub{Com}}$ {[AVP]}$_{\feinsub{Reas}}$ {[NP.Obj]}$_{\feinsub{Eval}}$ & 2 & \textit{виня, обвинявам\slash обвиня}\\
\lspbottomrule
    \end{tabularx}
    \caption{FrameNet valence patterns of \framename{Judgment\_communication} verbs, their frequency in the Bulgarian dataset and the verbs they appear with.
     English translation equivalents: \textit{величая, възхвалявам\slash възхваля} `extol', \textit{виня, обвинявам\slash обвиня} `blame', \textit{гавря се} `deride', \textit{заклеймявам\slash заклеймя} `condemn', \textit{заяждам се\slash заям се} `criticise',  \textit{иронизирам} `ironise', \textit{клеветя} `denigrate', \textit{критикувам} `criticise', \textit{кълна} `damn', \textit{омаловажавам\slash омаловажа} `belittle', \textit{осъждам\slash осъдя} `judge', \textit{отричам\slash отрека} `denounce', \textit{подценявам\slash подценя} `disparage', \textit{подигравам се\slash подиграя се, присмивам се\slash присмея се} `mock, ridicule', \textit{порицавам\slash порицая} `castigate', \textit{похвалвам\slash похваля, хваля} `commend, praise' \textit{прославям\slash прославя, славя} `laud'.}
    \label{tbl:judgment-valence-bg}
\end{table} 


\newpage
\subsection{Frame \framename{Questioning}}
\begin{description}[font=\normalfont]
\item[Definition of the frame \framename{Questioning}:] A \fename{Speaker} asking an \fename{Addressee} a question, which represents the \fename{Message}, calling for a reply. Core frame elements: \fename{Speaker}, \fename{Message}, \fename{Addressee}, \fename{Topic}.
\end{description}

%This frame inherits \framename{Communication} through weak inheritance (via the relation Using). This is shown by the different configuration of the relations among the core frame elements. The main difference as compared with many of the related frames, including \framename{Communication}, is that -- as the definition shows -- the speech activity is directed not to the content of the proposition, i.e. the \fename{Message}, in this case -- a question requiring a reply, but to the \fename{Addressee}, who needs to provide the reply.

\subsubsection{Syntactic realisation of the \framename{Questioning} frame elements}

The semantic specification of the core frame elements is similar to those in the other related frames. As questioning is a purposeful action, the \fename{Speaker} is necessarily a person or an organisation. The \fename{Speaker} is the external argument projected as a subject NP. 

The central role of the \fename{Addressee} is reflected in the fact that it is a frame element that is typically expressed as the direct object NP (except for \textit{inquire}.v and some uses of \textit{ask}.v where it can be expressed as a prepositional complement headed by \textit{of}).

Except for a small number of occurrences with the same verbs, i.e. \textit{inquire}.v and \textit{ask}.v, where it takes the direct object position, \fename{Message} is typically expressed as a direct quote or an embedded question. 

The \fename{Topic} is either expressed as a prepositional complement or remains implied but non-overt syntactically. 

The verbs evoking the frame \framename{Questioning} are divided along two lines:\\ (i) whether they tend to express the \fename{Message} over the \fename{Topic} or vice versa; \\
(ii) whether they tend to leave the \fename{Addressee} unexpressed if it is understood from the context or not.
\largerpage
\begin{sloppypar}
With respect to the first criterion, the valence patterns for the verbs in the frame clearly show that the \fename{Message} and the \fename{Topic} rarely co-occur. %as opposed to what was shown for other frames. 
Out of the verbs listed in this frame, \textit{grill}.v, \textit{interrogate}.v, \textit{question}.v and \textit{quiz}.v strongly favour the \fename{Topic} (Example \ref{ex07question:a}, \ref{ex07question:b}), with much rarer occurrences of the \fename{Message}, usually in the form of a direct quotation (Example \ref{ex07question:c}); at least in the annotated corpus the two frame elements do not co-occur with these verbs. %This means that these verbs prefer to refer to the subject matter of the question rather than to its content. 
\end{sloppypar}
 
The remaining verbs: \textit{ask}.v, \textit{inquire}.v, \textit{query}.v tend to express the content of the question, i.e. the \fename{Message} rather than its subject matter, the \fename{Topic}, but \fename{Topics} do occur. Besides, the two frame elements can co-occur provided that the \fename{Message} is not realised by a clause, compare (Example \ref{ex07question:d} and Example \ref{ex07question:e}) or a quote. With the verb \textit{inquire}.v, the \fename{Message} may be realised not only as a clause or a quote but also (though rarely) as a prepositional complement (Example \ref{ex07question:f}). In addition, both \textit{inquire}.v and \textit{ask}.v allow the \fename{Message} to be expressed as an object NP (Example \ref{ex07question:d}, \ref{ex07question:h}). This pattern is typical of \textit{ask}.v and rare for \textit{inquire}.v. In such cases the \fename{Addressee} is expressed as an indirect (Example \ref{ex07question:h}) or a prepositional object (see Example \ref{ex07question:i}, which is a rephrase of Example \ref{ex07question:h}).

As regards the second distinction, the same verbs that favour \fename{Topics} over \fename{Messages} -- \textit{grill}.v, \textit{interrogate}.v, \textit{question}.v and \textit{quiz}.v -- show preference to expressing the \fename{Addressee} as an object NP, rather than leaving it implicit (Example \ref{ex07question:j}). As shown in \tabref{tbl:questioning-valence}, they tend to realise the \fename{Addressee} together with the \fename{Topic} (expressed as a PP headed by \textit{about}). When the \fename{Message} is expressed, the \fename{Addressee} is often left out.

%The second group is further divided -- on the one hand \textit{inquire}, \textit{query} prefer to leave the \fename{Addressee} implicit, while \textit{ask} shows both patterns with a prevalence of the DNI interpretation. If the \fename{Addressee} of \textit{ask} is non-overt, the \fename{Message} may be a quote, an embedded clause or more rarely -- an object NP (Example \ref{ex07question:k}).



\begin{exe} 
\ex \label{ex07question}
\begin{xlist}
\ex \label{ex07question:a}
 {[}\textit{Journalists}{]}$_{\feinsub{Spkr}}$ \textit{\textbf{GRILLED}} [\textit{Mr. Major}]$_{\feinsub{Addr}}$ [\textit{about \\Maastricht}]$_{\feinsub{Top}}$.
\ex  \label{ex07question:b}
{[}\textit{She}{]}$_{\feinsub{Spkr}}$ \textit{\textbf{QUESTIONED}} [\textit{him}]$_{\feinsub{Addr}}$ [\textit{about his aspirations}]$_{\feinsub{Top}}$.
\ex  \label{ex07question:c}
{[}\textit{I}{]}$_{\feinsub{Spkr}}$ \textit{\textbf{QUIZZED}} [\textit{him}]$_{\feinsub{Addr}}$: [\textit{``Who are you?''}]$_{\feinsub{Msg}}$.
\ex  \label{ex07question:d}
{[}\textit{You}{]}$_{\feinsub{Spkr}}$ \textit{\textbf{ASK}} [\textit{many questions}]$_{\feinsub{Msg}}$ [\textit{about her}]$_{\feinsub{Top}}$ \\ {[\_]}$_{\feinsub{Addr-DNI}}$.
\ex  \label{ex07question:e}
{[}\textit{The clerk}{]}$_{\feinsub{Spkr}}$ \textit{\textbf{INQUIRED}} [\textit{\_}]$_{\feinsub{Addr-DNI}}$ [\textit{if it would be \\cash}]$_{\feinsub{Msg}}$.
\ex  \label{ex07question:f}
{[}\textit{He}{]}$_{\feinsub{Spkr}}$ \textit{\textbf{INQUIRED}} [\_]$_{\feinsub{Addr-DNI}}$ [\textit{as to their where-\\abouts}]$_{\feinsub{Msg}}$.
\ex  \label{ex07question:g}
{[}\textit{I}{]}$_{\feinsub{Spkr}}$ \textit{did not \textbf{INQUIRE}} [\textit{the reason}]$_{\feinsub{Msg}}$ [\_]$_{\feinsub{Addr-DNI}}$.
\ex  \label{ex07question:h}
{[}\textit{They}{]}$_{\feinsub{Spkr}}$ \textit{\textbf{ASKED}} [\textit{the newcomer}]$_{\feinsub{Addr}}$ [\textit{his name}]$_{\feinsub{Msg}}$.
\ex  \label{ex07question:i}
{[}\textit{They}{]}$_{\feinsub{Spkr}}$ \textit{\textbf{ASKED}} [\textit{the name}]$_{\feinsub{Msg}}$ [\textit{of the newcomer}]$_{\feinsub{Addr}}$.
\ex  \label{ex07question:j}
{[}\textit{They}{]}$_{\feinsub{Spkr}}$ \textit{\textbf{QUESTIONED}} [\textit{the convict}]$_{\feinsub{Addr}}$ [\textit{about the \\money}]$_{\feinsub{Top}}$.
\ex  \label{ex07question:k}
{[}``\textit{Why not?''}{]}$_{\feinsub{Msg}}$ \textit{\textbf{QUERIED}} [\textit{she}]$_{\feinsub{Spkr}}$ {[\_]}$_{\feinsub{Addr-DNI}}$.
\end{xlist}
\end{exe}

\tabref{tbl:questioning-synt} shows some of the frequent verbs of the frame and the realisation of their frame elements.

\begin{table}
\centering\footnotesize
\begin{tabular}{l rrrrrrrrr}
\lsptoprule
 & NP.Ext & NP.Obj & PP & AVP & NI & Clause & Quote & Other & Total\\ 

\midrule
%\multicolumn{10}{l}{\textit{grill} } \\  
%\fename{Speaker} & 22  &  & 5  &  & 5  &  &  &  & 32\\ 
%\fename{Addressee} & 10  & 22  &  &  &  &  &  &  & 32\\ 
%\fename{Message} &  &  & 1  &  &  &  &  &  & 1\\ 
%\fename{Topic} &  &  & 14  &  & 17  &  &  &  & 31\\ 
% \midrule
\multicolumn{10}{l}{\textit{inquire} } \\  
\fename{Speaker} & 37  &  &  &  &  &  &  &  & 37\\ 
\fename{Addressee} &  &  & 5  &  & 32  &  &  &  & 37\\ 
\fename{Message} &  & 3  & 1  &  &  & 5  & 18  &  & 27\\ 
\fename{Topic} &  &  & 10  &  &  &  &  &  & 10\\ 

\midrule
%\multicolumn{10}{l}{\textit{interrogate} } \\  
%\fename{Speaker} & 12  &  & 7  &  & 5  &  &  &  & 24\\ 
%\fename{Addressee} & 12  & 12  &  &  &  &  &  &  & 24\\ 
%\fename{Topic} &  &  & 3  &  & 21  &  &  &  & 24\\ 
% \midrule
%\multicolumn{10}{l}{\textit{query} } \\  
%\fename{Speaker} & 23  &  &  &  &  &  &  &  & 23\\ 
%\fename{Addressee} &  & 1  &  &  & 22  &  &  &  & 23\\ 
%\fename{Message} &  &  &  &  &  & 1  & 21  &  & 22\\ 
%\fename{Topic} &  &  & 1  &  &  &  &  & 1 & 2\\ 
% \midrule
\multicolumn{10}{l}{\textit{question} } \\  
\fename{Speaker} & 34  &  & 4  &  & 9  &  &  &  & 47\\ 
\fename{Addressee} & 13  & 29  &  &  & 5  &  &  &  & 47\\ 
\fename{Message} &  &  &  &  &  &  & 5  &  & 5\\ 
\fename{Topic} &  &  & 25  &  & 17  &  &  &  & 42\\ 

\midrule
%\multicolumn{10}{l}{\textit{quiz} } \\  
%\fename{Speaker} & 22  &  & 4  &  & 2  &  &  &  & 28\\ 
%\fename{Addressee} & 6  & 21  &  &  & 1  &  &  &  & 28\\ 
%\fename{Message} &  &  &  &  &  &  & 2  &  & 2\\ 
%\fename{Topic} &  &  & 16  &  & 9  &  &  & 1 & 26\\ 
% \midrule
\multicolumn{10}{l}{\textit{ask} } \\  
\fename{Speaker} & 68  &  &  &  & 8  &  &  &  & 76\\ 
\fename{Addressee} & 7  & 27  &  &  & 35  &  &  &  & 69\\ 
\fename{Message} & 2  & 7  & 3  &  & 5  & 26  & 18  &  & 61\\ 
\fename{Topic} &  & 4  & 8  &  &  & 1  &  & 1 & 14\\ 

\lspbottomrule
 \end{tabular}
 \caption{Syntactic expression of the \framename{Questioning} frame elements in selected FrameNet lexical units. } 
    \label{tbl:questioning-synt}
 \end{table}


\subsubsection{\framename{Questioning} valence patterns}

The valence patterns (\tabref{tbl:questioning-valence}) show the tendency outlined above: the preference for expressing the \fename{Addressee} together with the \fename{Topic} or to leave it non-overt when the focus is on the \fename{Message} (i.e. it is syntactically expressed).

\begin{table}
    \centering\footnotesize
    \begin{tabularx}{\textwidth}{ lrQ }
\lsptoprule
         Pattern  & \#  & verbs \\
\midrule
{[NP.Ext]}$_{\feinsub{Spkr}}$ {[\_]}$_{\feinsub{Addr-DNI}}$ {[Quote]}$_{\feinsub{Msg}}$  & 55 & \textit{quiz, inquire, question, query, ask}\\
{[NP.Ext]}$_{\feinsub{Spkr}}$ {[NP.Obj]}$_{\feinsub{Addr}}$ {[PP]}$_{\feinsub{Top}}$  & 48 & \textit{quiz, interrogate, question, ask, grill}\\
{[NP.Ext]}$_{\feinsub{Spkr}}$ {[NP.Obj]}$_{\feinsub{Addr}}$ {[\_]}$_{\feinsub{Top-DNI/INI}}$  & 38 & \textit{quiz, grill, interrogate, question}\\
{[NP.Ext]}$_{\feinsub{Spkr}}$ {[NP.Obj]}$_{\feinsub{Addr}}$ {[Clause]}$_{\feinsub{Msg}}$  & 13 & \textit{ask}\\
{[NP.Ext]}$_{\feinsub{Spkr}}$ {[\_]}$_{\feinsub{Addr-DNI}}$ {[PP]}$_{\feinsub{Top}}$  & 12 & \textit{inquire, ask}\\
{[NP.Ext]}$_{\feinsub{Spkr}}$ {[\_]}$_{\feinsub{Addr-DNI}}$ {[Clause]}$_{\feinsub{Msg}}$  & 10 & \textit{inquire, query, ask}\\
%{[NP.Ext]}$_{\feinsub{Addr}}$ {[\_]}$_{\feinsub{Spkr-CNI}}$ {[\_]}$_{\feinsub{Top-DNI}}$  & 9 & \textit{interrogate, question}\\
%{[NP.Ext]}$_{\feinsub{Addr}}$ {[PP]}$_{\feinsub{Spkr}}$ {[\_]}$_{\feinsub{Top-INI}}$  & 5 & \textit{quiz, grill}\\
\lspbottomrule
    \end{tabularx}
    \caption{FrameNet valence patterns of \framename{Questioning} verbs, their frequency in the FrameNet corpus and the verbs they appear with.}
    \label{tbl:questioning-valence}
\end{table} 

\subsubsection{Syntactic realisation of \framename{Questioning} in Bulgarian}


Most of the Bulgarian counterparts are derived from the basic \framename{Questioning} verb \textit{питам} `ask' -- \textit{попитвам}, \textit{запитвам} `ask', \textit{разпитвам} `ask, question, grill', \textit{пре-\newline питвам} `quiz, query'. Typically, either the \fename{Message} or the \fename{Topic} is expressed (Example \ref{ex:07questionbg:a}, \ref{ex:07questionbg:b}). The two may co-occur only if the \fename{Message} is nominalised, usually by means of any of a small inventory of pronouns such as \textit{нещо} `something, anything', \textit{нищo} `nothing', \textit{това} `this, that' or some other expressions %or another pronoun nominalising a proposition and some other words 
(Example \ref{ex:07questionbg:c}). If the \fename{Message} is expressed otherwise, most often as a quote or an embedded clause, the two frame elements typically do not co-occur. The \fename{Topic} is expressed as a PP headed by the prepositions \textit{за} or \textit{относно} `about, regarding', while the \fename{Addressee} occupies the direct object position -- NP.Obj (Examples \ref{ex:07questionbg:a}, \ref{ex:07questionbg:b}, \ref{ex:07questionbg:c}).


The predominant valence patterns in Bulgarian are similar (\tabref{tbl:questioning-valence-bg}), although the data show that the \fename{Addressee} co-occurs more frequently with \fename{Message} (Example \ref{ex:07questionbg:d}) than in English.

\begin{exe}
\ex \label{ex:07questionbg}
\begin{xlist}
\ex \label{ex:07questionbg:a}
\gll {[}-- \textit{Какво} \textit{мислиш}?{]}$_{\feinsub{Msg}}$  -- \textit{\textbf{ПОПИТА}} [\textit{я}]$_{\feinsub{Addr}}$  [\textit{той}]$_{\feinsub{Spkr}}$.\\
-- What think.2sg? -- asked her he.
 \\
 \glt `-- What do you think? -- he asked her.'
\ex \label{ex:07questionbg:b}
 \gll  {[}\textit{Тя}{]}$_{\feinsub{Spkr}}$  [\textit{го}]$_{\feinsub{Addr}}$  \textit{\textbf{РАЗПИТВА}} [\_]$_{\feinsub{Msg}}$   [\textit{за} \textit{личния} \textit{му} \textit{живот}]$_{\feinsub{Top}}$.\\
  She him inquires {} about personal-\textsc{def}  his life.
  \\
 \glt `She inquires him about his personal life.'
\ex \label{ex:07questionbg:c}
\gll {[}\_{]}$_{\feinsub{Spkr}}$  \textit{Ще} [\textit{те}]$_{\feinsub{Addr}}$  \textit{\textbf{ПОПИТАМ}} [\textit{нещо}]$_{\feinsub{Msg}}$   [\textit{за} \textit{Арон}]$_{\feinsub{Top}}$.\\
{}  Will you ask.1sg something about Aaron.
 \\
 \glt `I will ask you something about Aaron.'
\ex \label{ex:07questionbg:d}
\gll {[}\textit{Престъпникът}{]}$_{\feinsub{Spkr}}$  \textit{\textbf{ПОПИТАЛ}} [\textit{полицая}]$_{\feinsub{Addr}}$  [\textit{дали} \textit{може} \textit{да} \textit{си} \textit{купи} \textit{цигари}]$_{\feinsub{Msg}}$.
 \\
 Criminal-\textsc{def} asked policeman-\textsc{def} whether can to  himself buy cigarettes.
 \\
 \glt `The criminal asked the policeman whether he could buy cigarettes.'
\end{xlist}
\end{exe}

The most frequent verbs and the syntactic realisation of the  frame elements of \framename{Questioning} is shown in \tabref{tbl:questioning-synt-bg}.

\begin{table}
\centering\footnotesize
\begin{tabular}{l rrrrrrrrr}
\lsptoprule
 & NP.Ext & NP.Obj & PP & AVP & NI & Clause & Quote & Other & Total\\ 

\midrule
\multicolumn{10}{l}{\textit{попитвам\slash попитам} }\\  
`ask'\\
\fename{Message} &  & 1 &  &  & 1 &   20& & 7 & 29\\ 
\fename{Addressee} &  & 34 &  &  &  &  &  &  & 34\\ 
\fename{Topic} &  &  & 2 &  &  &  &  &  & 2\\ 
\fename{Speaker} & 57 &  &  &  &  &  &  &  & 57\\ 

\midrule
\multicolumn{10}{l}{\textit{запитвам\slash запитам} }\\  
`ask, question' \\
\fename{Message} &  &  &  &  &  &   6& & 14 & 20\\ 
\fename{Addressee} &  & 16 &  &  &  &  &  &  & 16\\ 
\fename{Topic} &  &  & 3 &  &  &  &  &  & 3\\ 
\fename{Speaker} & 23 &  &  &  &  &  &  &  & 23\\ 

\midrule
\multicolumn{10}{l}{\textit{питам} }\\
`ask'\\
\fename{Message} & 1 & 1 &  &  & 2 &   6& & 15 & 25\\ 
\fename{Addressee} &  & 16 &  &  & 3 &  &  &  & 19\\ 
\fename{Topic} &  &  & 5 &  &  &  &  &  & 5\\ 
\fename{Speaker} & 29 &  &  &  & 1 &  &  &  & 30\\ 

\lspbottomrule
%\multicolumn{10}{l}{\textit{разпитвам\slash разпитам} }\\  
%\fename{Message} &  &  &  &  & 1 &  &  &  & 1\\ 
%\fename{Addressee} &  & 14 &  &  & 2 &  &  &  & 16\\ 
%\fename{Topic} &  &  & 4 &  &  &  &  &  & 4\\ 
%\fename{Speaker} & 17 &  &  &  &  &  &  &  & 17\\ 
% \midrule
 \end{tabular}
 \caption{Syntactic expression of the \framename{Questioning} frame elements in Bulgarian. } 
    \label{tbl:questioning-synt-bg}
 \end{table}

\begin{table}
    \centering\footnotesize
    \begin{tabularx}{\textwidth}{ lrQ }
\lsptoprule
         Pattern  & \#  & verbs \\
\midrule
{[NP.Ext]}$_{\feinsub{Spkr}}$ {[NP.Obj]}$_{\feinsub{Addr}}$ {[Sinterrog]}$_{\feinsub{Msg}}$ & 25 & \textit{питам, запитвам\slash запитам, попитвам\slash попитам}\\

{[NP.Ext]}$_{\feinsub{Spkr}}$ {[NP.Obj]}$_{\feinsub{Addr}}$ {[\_]}$_{\feinsub{Msg-INI}}$ & 25 & \textit{разпитвам\slash разпитам}\\

{[NP.Ext]}$_{\feinsub{Spkr}}$ {[Quote]}$_{\feinsub{Msg}}$ {[\_]}$_{\feinsub{Addr-INI}}$ & 20 & \textit{питам, запитвам\slash запитам, попитвам\slash попитам, поинтересувам се}\\

{[NP.Ext]}$_{\feinsub{Spkr}}$ {[NP.Obj]}$_{\feinsub{Addr}}$ {[Quote]}$_{\feinsub{Msg}}$ & 14 & \textit{питам, запитвам\slash запитам, попитвам\slash попитам}\\

{[NP.Ext]}$_{\feinsub{Spkr}}$ {[Sinterrog]}$_{\feinsub{Msg}}$ {[\_]}$_{\feinsub{Addr}}$ & 13 & \textit{интересувам се, питам, запитвам\slash запитам, попитвам\slash попитам, полюбопитствам}\\

%{[NP.Ext]}$_{\feinsub{Spkr}}$ {[]}$_{\feinsub{Msg}}$ & 11 & \textit{попитвам\slash попитам, разпитвам\slash разпитам}\\

{[NP.Ext]}$_{\feinsub{Spkr}}$ {[NP.Obj]}$_{\feinsub{Addr}}$ {[PP]}$_{\feinsub{Top}}$ & 9 & \textit{питам, запитвам\slash запитам, попитвам\slash попитам, разпитвам\slash разпитам}\\

%{[NP.Ext]}$_{\feinsub{Spkr}}$ {[PP]}$_{\feinsub{Top}}$ {[\_]}$_{\feinsub{Addr-INI}}$ & 6 & \textit{питам, поинтересувам се, разпитвам\slash разпитам}\\

%{[NP.Ext]}$_{\feinsub{Spkr}}$ {[NP.Obj]}$_{\feinsub{Addr}}$ {[\_]}$_{\feinsub{Msg-INI}}$ & 3 & \textit{попитвам\slash попитам, разпитвам\slash разпитам, питам}\\

%{[NP.Ext]}$_{\feinsub{Spkr}}$ {[NP.Obj]}$_{\feinsub{Addr}}$ & 3 & \textit{разпитвам\slash разпитам, питам}\\
\lspbottomrule
    \end{tabularx}
    \caption{FrameNet valence patterns of the frame \framename{Questioning}, their frequency in the Bulgarian dataset and the verbs they appear with.
    English translation equivalents: \textit{питам, запитвам\slash запитам, попитвам\slash попитам} `ask, question', \textit{интересувам се\slash поинтересувам се} `inquire', \textit{разпитвам\slash разпитам} `question, grill, interrogate'.}
    \label{tbl:questioning-valence-bg}
\end{table} 


\subsection{Frame \framename{Communication\_response}}  

\begin{description}[font=\normalfont]
\item[Definition of the frame \framename{Communication\_response}:] A \fename{Speaker} communicates a reply or response, a \fename{Message}, to some prior communication or action, the \fename{Trigger}. Core frame elements: \fename{Speaker}, \fename{Message}, \fename{Trigger}, \fename{Addressee}, \fename{Topic}.
\end{description}

The \framename{Communication\_response} frame inherits from the frame \framename{Communication}. It elaborates on the prototypical frame by introducing a new frame element, the \fename{Trigger}, which requires a response, expressed as the \fename{Message}. %is in fact a response to the \fename{Trigger}.

\subsubsection{Syntactic realisation of \framename{Communication\_response} frame elements}


The \fename{Speaker} inherits the frame element \fename{Communicator} which exhibits the same characteristics and behaviour as in the other frames in the domain, and is realised most often as an external NP. 

The \fename{Trigger} is the prior communication or action to which a response is given. It can be implicit, or overtly expressed either as an NP object or as a prepositional complement (Examples \ref{ex:08response:a}, \ref{ex:08response:b}).

The \fename{Message} is not necessarily expressed when the \fename{Trigger} is present (Examples \ref{ex:08response:a}, \ref{ex:08response:b}). When the \fename{Message} is realised, it predominantly takes the form of an embedded clause (Example \ref{ex:08response:c}) or a direct quote (Example \ref{ex:08response:d}).

Although rarely, the \fename{Trigger} and the \fename{Message} may co-occur (Example \ref{ex:08response:g}).

The \fename{Addressee} is the person to whom the response is directed. When expressed, it occurs as a prepositional phrase introduced by the preposition `to’ (Example \ref{ex:08response:e}) or as an indirect object (Example \ref{ex:08response:f}).

The \fename{Topic} is possible but rare with verbs from this frame.

\begin{exe}
\ex \label{ex:08response}
\begin{xlist}
\ex \label{ex:08response:a}
{[}\textit{Sue}{]}$_{\feinsub{Com}}$  \textit{\textbf{ANSWERED}} [\textit{the question}]$_{\feinsub{Trig}}$.
\ex \label{ex:08response:b}
{[}\textit{The US}{]}$_{\feinsub{Com}}$  \textit{has not \textbf{RESPONDED}} [\textit{to the offer}]$_{\feinsub{Trig}}$.
\ex \label{ex:08response:c}
{[}\textit{Blanche}{]}$_{\feinsub{Com}}$  \textit{\textbf{RESPONDED}} [\textit{that the police were talking to \newline everyone}]$_{\feinsub{Msg}}$.
\ex \label{ex:08response:d}
{[}`\textit{Does it matter}?']$_{\feinsub{Msg}}$  [\textit{she}]$_{\feinsub{Com}}$  \textit{\textbf{COUNTERED} defeatedly}.
\ex \label{ex:08response:e}
{[}\textit{Sue}{]}$_{\feinsub{Com}}$  \textit{\textbf{RESPONDED}} [\textit{to Bob}]$_{\feinsub{Addr}}$  \textit{immediately}. 
\ex \label{ex:08response:f}
{[}\textit{The senator}{]}$_{\feinsub{Com}}$  \textit{took the floor to \textbf{ANSWER}} [\textit{critics of the \\deal}]$_{\feinsub{Addr}}$. 
\ex \label{ex:08response:g}
{[}`\textit{Does it matter}?']$_{\feinsub{Msg}}$ \textit{\textbf{REPLIED}} [\textit{she}]$_{\feinsub{Com}}$   [\textit{to his question}]$_{\feinsub{Trig}}$. 
\end{xlist}
\end{exe}

\tabref{tbl:response-synt} shows the syntactic realisations of verbs evoking the frame \framename{Communication\_response}.

\begin{table}
\centering\footnotesize
\begin{tabular}{l rrrrrrrrr}
\lsptoprule
 & NP.Ext & NP.Obj & PP & AVP & NI & Clause & Quote & Other & Total\\ 

\midrule
\multicolumn{10}{l}{\textit{answer} } \\  
\fename{Speaker} & 31  &  & 1  &  & 2  &  &  &  & 34\\ 
\fename{Addressee} &  & 1  &  &  & 30  &  &  & 3 & 34\\ 
\fename{Message} &  & 2  & 4  &  & 21  & 2  & 5  &  & 34\\ 
\fename{Trigger} & 3  & 11  &  &  & 18  & 1  &  &  & 33\\ 

\midrule
\multicolumn{10}{l}{\textit{reply} } \\  
\fename{Speaker} & 69  &  &  &  &  &  &  &  & 69\\ 
\fename{Addressee} &  &  & 4  &  & 65  &  &  &  & 69\\ 
\fename{Message} &  &  & 4  &  & 21  & 11  & 27  & 6 & 69\\ 
\fename{Trigger} &  &  & 13  &  & 56  &  &  &  & 69\\ 

\midrule
\multicolumn{10}{l}{\textit{respond} } \\  
\fename{Speaker} & 25  &  &  &  &  &  &  &  & 25\\ 
\fename{Addressee} &  &  & 1  &  & 23  &  &  &  & 24\\ 
\fename{Message} &  &  & 3  & 1  & 4  & 4  & 13  &  & 25\\ 
\fename{Trigger} &  &  & 1  &  & 24  &  &  &  & 25\\ 

\midrule
\multicolumn{10}{l}{\textit{retort} } \\  
\fename{Speaker} & 46  &  &  &  &  &  &  &  & 46\\ 
\fename{Addressee} &  &  & 1  &  & 45  &  &  &  & 46\\ 
\fename{Message} &  & 2  & 1  &  & 2  & 21  & 23  &  & 49\\ 
\fename{Trigger} &  &  & 1  &  & 45  &  &  &  & 46\\ 

\lspbottomrule
 \end{tabular}
 \caption{Syntactic expression of the \framename{Communication\_response} frame elements for selected FrameNet lexical units. } 
    \label{tbl:response-synt}
 \end{table}

\subsubsection{\framename{Communication\_response} valence patterns}

\tabref{tbl:response-valence} illustrates the valence patterns that characterise the verbs in the frame \framename{Communication\_response}. The most frequent pattern has the \fename{Message} realised as a direct quote, followed by the pattern with an embedded clause or a PP. The \fename{Trigger} is expressed in fewer instances and in such cases the \fename{Addressee} and the \fename{Message} remain non-overt.


\begin{table}
    \centering\footnotesize
    \begin{tabularx}{\textwidth}{ lrQ }
\lsptoprule
         Pattern  & \#  & verbs \\
\midrule
{[NP.Ext]}$_{\feinsub{Spkr}}$ {[\_]}$_{\feinsub{Addr-DNI}}$ {[Quote]}$_{\feinsub{Msg}}$  {[\_]}$_{\feinsub{Trig-DNI}}$  & 83 & \textit{answer, rejoin, counter, reply, respond, retort}\\

{[NP.Ext]}$_{\feinsub{Spkr}}$ {[\_]}$_{\feinsub{Addr-DNI}}$ {[Clause]}$_{\feinsub{Msg}}$   {[\_]}$_{\feinsub{Trig-DNI}}$  & 34 & \textit{answer, rejoin, counter, reply, respond, retort}\\

%{[NP.Ext]}$_{\feinsub{Spkr}}$ {[\_]}$_{\feinsub{Addr-DNI}}$ {[\_]}$_{\feinsub{Msg-INI}}$ {[\_]}$_{\feinsub{Trigger-DNI}}$  & 19 & \textit{answer, reply, respond, retort}\\

{[NP.Ext]}$_{\feinsub{Spkr}}$ {[\_]}$_{\feinsub{Addr-DNI}}$ {[PP]}$_{\feinsub{Msg}}$ {[\_]}$_{\feinsub{Trig-DNI}}$  & 14 & \textit{answer, counter, reply, respond}\\

{[NP.Ext]}$_{\feinsub{Spkr}}$ {[\_]}$_{\feinsub{Addr-DNI}}$ {[\_]}$_{\feinsub{Msg-INI}}$ {[PP]}$_{\feinsub{Trig}}$  & 10 & \textit{reply}\\

%{[NP.Ext]}$_{\feinsub{Spkr}}$ {[\_]}$_{\feinsub{Addr-DNI}}$ {[2nd]}$_{\feinsub{Msg}}$ {[\_]}$_{\feinsub{Trigger-DNI}}$  & 10 & \textit{counter, reply}\\

{[NP.Ext]}$_{\feinsub{Spkr}}$ {[\_]}$_{\feinsub{Addr-DNI}}$ {[\_]}$_{\feinsub{Msg-INI}}$ {[NP.Obj]}$_{\feinsub{Trig}}$  & 7 & \textit{answer}\\
\lspbottomrule
    \end{tabularx}
    \caption{FrameNet valence patterns of \framename{Communication\_response} verbs, their frequency in the FrameNet corpus and the verbs they appear with.}
    \label{tbl:response-valence}
\end{table} 

\subsubsection{Syntactic realisation of \framename{Communication\_response} frame in Bulgarian}

In Bulgarian the syntactic realisation of the frame is similar to English. The \fename{Message} most often appears as an embedded clause (Example \ref{ex:08responsebg:a}) or  as a direct quote (Example \ref{ex:08responsebg:b}), and in some cases as a direct object (Example \ref{ex:08responsebg:d}) or a prepositional phrase (Example \ref{ex:08responsebg:c}). The \fename{Trigger} is realised as a prepositional phrase (Example \ref{ex:08responsebg:e}). %The \fename{Trigger} can co-occur with the \fename{Message} (Example \ref{ex:08responsebg:c}).

\begin{exe}
\ex \label{ex:08responsebg}
\begin{xlist} 
\ex \label{ex:08responsebg:a}
\gll {[}\textit{Той}{]}$_\fename{Spkr}$ [\textit{ми}]$_\fename{Addr}$ \textit{\textbf{ОТГОВОРИ}}, [\textit{че} \textit{няма} \textit{да} \textit{отиде}]$_\fename{Msg}$. 
\\
He me answered that not to go. 
\\
\glt `He answered me that he won't go.'
\ex \label{ex:08responsebg:b}
\gll {[}\textit{Студентът}{]}$_\fename{Spkr}$ \textit{\textbf{ОТГОВОРИЛ}}: [-- \textit{Професоре}, \textit{забравих}!]$_\fename{Msg}$
\\
Student-\textsc{def} responded: -- Professor, forgot.1sg!
\\
\glt `The student responded: -- Professor, I forgot!'
\ex \label{ex:08responsebg:c}
\gll {[}\textit{Мнозинството}{]}$_\fename{Spkr}$ \textit{\textbf{ОТВРЪЩА}} [\textit{с} \textit{надменни} \textit{приказки}]$_\fename{Msg}$ [\textit{за} \textit{своята} \textit{безалтернативност}]$_\fename{Top}$.
\\
Majority-\textsc{def} answers with arrogant words about their-REFL {lack of prospects}].
\\
\glt `The majority answers with arrogant words about their lack of prospects.'
\ex \label{ex:08responsebg:d}
\gll {[}\textit{Той}{]}$_\fename{Spkr}$ \textit{не} \textit{\textbf{ОТГОВОРИ}} [\textit{нищо}]$_\fename{Msg}$. 
\\
He not responded nothing. 
\\
\glt `He did not respond anything.'
\ex \label{ex:08responsebg:e}
\gll {[}\textit{На} \textit{този} \textit{въпрос}{]}$_\fename{Trig}$ \textit{ще} \textit{\textbf{ОТГОВОРИ}} [\textit{министър}-\textit{председателят}]$_\fename{Spkr}$. 
\\
To this question will answer {prime minister}-\textsc{def}.
\\
\glt `The prime minister will answer this question.'
\end{xlist}
\end{exe}

The most frequent verbs evoking the frame \framename{Communication\_response} and the realisation of their frame elements are shown in \tabref{tbl:response-synt-bg}. The associated valence patterns are presented in \tabref{tbl:response-valence-bg}.

\begin{table}
\centering\footnotesize
\begin{tabular}{l rrrrrrrrr}
\lsptoprule
 & NP.Ext & NP.Obj & PP & AVP & NI & Clause & Quote & Other & Total\\ 

\midrule
\multicolumn{10}{l}{\textit{отвръщам\slash отвърна} }\\  
`reply'\\
\fename{Trigger} &  &  & 2 &  & 4 &  &  &  & 6\\ 
\fename{Message} &  & 2 &  &  & 5 &  & 23 & 5 & 35\\ 
\fename{Addressee} &  &  & 14 &  & 21 &  &  &  & 35\\ 
\fename{Manner} &  &  & 2 &  &  &  &  & 3 & 5\\ 
\fename{Speaker} & 35 &  &  &  &  &  &  &  & 35\\ 

\midrule
\multicolumn{10}{l}{\textit{отговарям\slash отговоря} }\\  
`answer, reply'\\
\fename{Trigger} &  &  & 15 &  & 15 &  &  &  & 30\\ 
\fename{Message} &  & 3 &  &  & 26 &  & 21 & 16 & 66\\ 
\fename{Addressee} &  &  & 23 &  & 43 &  &  &  & 66\\ 
\fename{Medium} &  &  & 7 &  &  &  &  &  & 7\\ 
\fename{Manner} &  &  & 2 &  &  &  &  & 12 & 14\\ 
\fename{Speaker} & 66 &  &  &  &  &  &  &  & 66\\ 

\lspbottomrule
 \end{tabular}
 \caption{Syntactic expression of the \framename{Communication\_response} frame elements in Bulgarian. } 
    \label{tbl:response-synt-bg}
 \end{table}

\begin{table}
    \centering\footnotesize
    \begin{tabularx}{\textwidth}{ lrQ }
\lsptoprule
         Pattern  & \#  & verbs \\
\midrule
{[NP.Ext]}$_{\feinsub{Spkr}}$ {[Quote]}$_{\feinsub{Msg}}$ {[\_]}$_{\feinsub{Addr-DNI}}$ {[\_]}$_{\feinsub{Trig-DNI}}$ & 22 & \textit{контрирам, отвръщам\slash отвърна, отговарям\slash отговоря}\\

{[NP.Ext]}$_{\feinsub{Spkr}}$ {[PP]}$_{\feinsub{Addr}}$ {[Quote]}$_{\feinsub{Msg}}$ {[\_]}$_{\feinsub{Trig-DNI}}$ & 12 & \textit{отвръщам\slash отвърна, отговарям\slash отговоря}\\

{[NP.Ext]}$_{\feinsub{Spkr}}$ {[Clause]}$_{\feinsub{Msg}}$ {[\_]}$_{\feinsub{Addr-DNI}}$ {[\_]}$_{\feinsub{Trig-DNI}}$ & 10 & \textit{отвръщам\slash отвърна, отговарям\slash отговоря}\\

{[NP.Ext]}$_{\feinsub{Spkr}}$ {[Clause]}$_{\feinsub{Msg}}$ {[PP]}$_{\feinsub{Addr}}$ {[\_]}$_{\feinsub{Trig-DNI}}$ & 9 & \textit{отвръщам\slash отвърна, отговарям\slash отговоря}\\

{[NP.Ext]}$_{\feinsub{Spkr}}$ {[PP]}$_{\feinsub{Trig}}$ {[\_]}$_{\feinsub{Addr-DNI}}$   {[\_]}$_{\feinsub{Msg-INI}}$ & 8 & \textit{отвръщам\slash отвърна, отговарям\slash отговоря}\\

%{[NP.Ext]}$_{\feinsub{Spkr}}$ {[AdvP]}$_{\feinsub{MANNER}}$ {[Quote]}$_{\feinsub{Msg}}$ {[\_]}$_{\feinsub{Addr-DNI}}$ {[\_]}$_{\feinsub{Trigger-DNI}}$ & 5 & \textit{отговарям\slash отговоря, отвръщам\slash отвърна}\\

%{[NP.Ext]}$_{\feinsub{Spkr}}$ {[PP]}$_{\feinsub{Addr}}$ {[\_]}$_{\feinsub{Msg-INI}}$ & 5 & \textit{отвръщам\slash отвърна, отговарям\slash отговоря}\\

%{[NP.Ext]}$_{\feinsub{Spkr}}$ {[NP.Obj]}$_{\feinsub{Msg}}$ {[\_]}$_{\feinsub{Addr-DNI}}$ & 4 & \textit{отговарям\slash отговоря, отвръщам\slash отвърна}\\

%{[NP.Ext]}$_{\feinsub{Spkr}}$ {[AdvP]}$_{\feinsub{MANNER}}$ {[PP]}$_{\feinsub{Trigger}}$ {[\_]}$_{\feinsub{Addr-DNI}}$ {[\_]}$_{\feinsub{Msg-INI}}$ & 3 & \textit{отговарям\slash отговоря}\\

%{[NP.Ext]}$_{\feinsub{Spkr}}$ {[PP]}$_{\feinsub{MEDIUM}}$ {[\_]}$_{\feinsub{Addr-DNI}}$ {[\_]}$_{\feinsub{Msg-INI}}$ {[\_]}$_{\feinsub{Trigger-DNI}}$ & 2 & \textit{отговарям\slash отговоря}\\

%{[NP.Ext]}$_{\feinsub{Spkr}}$ {[AdvP]}$_{\feinsub{MANNER}}$ {[PP]}$_{\feinsub{Addr}}$ {[Quote]}$_{\feinsub{Msg}}$ {[\_]}$_{\feinsub{Trigger-DNI}}$ & 2 & \textit{отговарям\slash отговоря}\\
\lspbottomrule
\end{tabularx}
\caption{FrameNet valence patterns of the frame \framename{Communication\_response}, their frequency in the Bulgarian dataset and the verbs they appear with.
English translation equivalents: \textit{контрирам} `counter', \textit{отвръщам\slash отвърна, отговарям\slash отговоря} `answer, reply, counter, retort'.}
\label{tbl:response-valence-bg}
\end{table} 




\section{Conclusions}\label{sec:conclusion}

In this paper we have discussed thе universal features of the conceptual description of verbs which is transferable across languages. We illustrate our analysis with examples from the class of verbs of communication with a view to their use in English and Bulgarian.


The universality of the semantic relations of inheritance (from a more generalised to a more specialised entity) underlies the hierarchical organisation of both the FrameNet frames and the WordNet synsets. The configuration of frame elements describing the behaviour of verbs evoking a particular frame are also language-independent, as well as the semantic restrictions determining their selection. Moreover, we have shown that the principles of syntactic realisation of the frame elements as represented by the generalised valence patterns are also valid to a large degree across different languages. For Bulgarian and English we have established substantial correspondence in both the valence patterns and the syntactic categories and grammatical functions by which frame elements are expressed.

Further, we have outlined some basic language-specific properties of the syntactic realisation of semantic frames and their corresponding frame elements. 
In some cases the two languages give different preference to the overt expression of particular frame elements. For example, the \fename{Topic} is more frequent in English and rarely expressed with Bulgarian communication verbs (e.g., evoking the frames \framename{Statement} and \framename{Communication\_manner}). %and in Bulgarian it most often occurs as a modifier to \fename{Message}. 
We also observe differences in the syntactic realisation of particular frame elements due to the distinct syntactic properties of the two languages. For example, Bulgarian lacks infinitives and \textit{-ing} clauses, so clausal complements expressing the frame element \fename{Message} are finite clauses. Differences at the syntactic level between Bulgarian and English are also found between verbs considered as translation equivalents (belonging to corresponding synsets in Bulgarian and English). For example, with the verb \textit{ridicule} (evoking the frame \framename{Judgment\_communication}) the \fename{Evaluee} is expressed predominantly as a direct object, while the Bulgarian verb \textit{подигравам се} realises it as an indirect object due to the fact that reflexive verbs do not take a direct object.

The analysis confirms the assumption that a large part of a verb’s semantic valency and syntactic behaviour is predictable from its lexical meaning and the semantic class it belongs to. The various semantic classifications of verbs focus on different semantic and/or syntactic properties, relying mostly on theoretical analysis or expert intuition rather than on authentic corpus data. A study based on corpus analysis and statistical observations on the frequency of valence patterns could provide more reliable evidence for the behaviour of verbs, in particular in view of cross-linguistic studies. Moreover, this will confirm the validity of the cross-linguistic analysis and the universality of semantic and syntactic features.

%The cross-linguistic analysis of conceptual structure, as well as the developed resources, could be applied to boost annotation with FrameNet frames (e.g., FATE -- FrameNet-Annotated Textual Entailment, \cite{Burchardt2008}).

In our work on describing the conceptual and syntactic properties of Bulgarian verbs, we have found the applicability of the conceptual description encoded in the FrameNet frames to be to a great extent language-independent and transferrable cross-linguistically, even if in some cases adjustments may be necessary. Given the fact that the alignment between equivalent senses in the wordnets developed for different languages is ensured by means of shared identification numbers with the original Princeton WordNet, the conceptual information from FrameNet is mappable across languages via WordNet.\footnote{For a list of existing wordnets in the world, see \url{http://globalwordnet.org/resources/wordnets-in-the-world/}.}

\section*{Abbreviations}

\begin{tabularx}{.5\textwidth}{lQ}
\scshape Addr & \fename{Addressee}\\
\scshape Auth & \fename{Author}\\
BulEnAC & Bulgarian-English \\ & Sentence- and Clause- \\ & Aligned Corpus\\
BulSemCor & Semantically annotated \\ & corpus for Bulgarian\\
CNI & Constructional null \\ & instantiation\\
\scshape Com & \fename{Communicator}\\
\scshape Cont & \fename{Content}\\
DNI & Definite null \\ & instantiation\\
\scshape Eval & \fename{Evaluee}\\
\scshape Exr & \fename{Expressor}\\
\end{tabularx}%
\begin{tabularx}{.49\textwidth}{lQ}
INI & Indefinite null instantiation\\
\scshape Manr & \fename{Manner}\\
\scshape Med & \fename{Medium}\\
\scshape Msg & \fename{Message}\\
N or n & Noun\\
NP & Noun phrase\\
PP & Prepositional phrase\\
PWN & Princeton WordNet\\
\scshape Reas & \fename{Reason}\\
SemCor & Semantically annotated \\ & corpus for English\\
\scshape Spkr & \fename{Speaker}\\
\scshape Top & \fename{Topic}\\
\scshape Trig & \fename{Trigger}\\
V or v & Verb
\end{tabularx}


\section*{Acknowledgements}

This research is carried out as part of the project \emph{Enriching Semantic Network WordNet with Conceptual Frames} funded by the Bulgarian National Science Fund, Grant Agreement No. KP-06-H50/1 from 2020.


{\sloppy\printbibliography[heading=subbibliography,notkeyword=this]}
\end{document}

\chapter{Hypothesen und Forschungsdesign}\label{3}

Dieses Kapitel steckt den Rahmen der vorliegenden Arbeit ab. In \sectref{3.1} wird argumentiert, weshalb die Komplexität der Nominalflexion und welche Arten von Komplexität gemessen werden. In \sectref{3.2} werden die Hypothesen vorgestellt sowie anhand welcher Varietäten diese Hypothesen überprüft werden. Die Charakteristika der untersuchten Varietäten werden in \sectref{3.3} präsentiert.

\section{Absolute Komplexität in der Nominalflexion}\label{3.1}

Das Ziel dieses Kapitels ist es, zuerst zu erklären, weshalb hier die  der nominalen Flexionsmorphologie gemessen wird (und z.\,B.\ nicht die verbale Flexionsmorphologie). In einem zweiten Schritt wird erläutert, welche Arten der Komplexität untersucht werden, und zwar auf der Basis der Einteilung von \citet{Miestamo2008} sowie \citet{Rescher1998} und \citet{Sinnemäki2011}. Diese Arten der Komplexität wurden in \sectref{2.2.5} eingeführt.

\subsection{Nominalflexion}\label{3.1.1}

In der vorliegenden Arbeit soll die Komplexität der nominalen Flexionsmorphologie gemessen werden, und zwar in den folgenden Wortarten: \isi{Substantive}, starke und schwache \isi{Adjektive}, \isi{Personalpronomen}, \isi{Interrogativpronomen}, bestimmter und \isi{unbestimmter Artikel}, einfaches \isi{Demonstrativpronomen} und \isi{Possessivpronomen}. Begründet wird diese Auswahl der Wortarten in \sectref{4.3.2}. Im Gegensatz zur verbalen Flexionsmorphologie bietet die nominale Flexionsmorphologie vor allem zwei Vorteile: Erstens ist die Variationsbreite zwischen den Varietäten größer und zweitens können innerhalb des nominalen Bereichs unterschiedliche Wortarten miteinander verglichen werden.\largerpage[1.5]

Die große Variationsbreite in der Nominalflexion kommt besonders dadurch zustande, dass es keinen diachronen Hauptprozess gibt, der in allen der hier untersuchten Varietäten gewirkt und gleiche Resultate verursacht hat.\footnote{Von einem der wichtigsten diachronen Prozesse in der Verbalflexion jedoch, nämlich vom Präteritumschwund, sind alle alemannischen Dialekte gleichermaßen betroffen. Eine Ausnahme davon bildete die Sprachinsel Saley/Salecchio, in der heute jedoch kein Alemannisch mehr gesprochen wird (herzlichen Dank an Oliver Schallert für diesen Hinweis).} Ein vermeintlicher prominenter Hauptprozess betrifft die Setzung des germanischen Initialakzents und die daraus resultierende Schwächung der Vollvokale im Nebenton, was zum Verlust von Kasusmarkierung (besonders am \isi{Substantiv} und \isi{Adjektiv}) geführt hat. Diese Hypothese könnte zutreffen, vergleicht man z.\,B.\ die althochdeutsche Kasusmorphologie (mit /\textit{a}/, /\textit{e}/, /\textit{i}/, /\textit{o}/ und /\textit{u}/ im Nebenton, \citealt[61]{Braune2004}) mit jener der nieder- und hochalemannischen Dialekte (mit /\textit{e}/, /\textit{i}/ und /\textit{ə}/ im Nebenton, \citealt[102--103]{Caro2011}): Während Althochdeutsch am \isi{Substantiv} morphologisch vier \isi{Kasus} unterscheidet (im Singular Maskulin und Neutrum mit dem Instrumental sogar fünf), sind in den meisten nieder- und hochalemannischen Dialekten alle \isi{Kasus} am \isi{Substantiv} zusammengefallen. Aber auch beide logisch möglichen Gegenbeispiele sind zu beobachten, die dagegen sprechen, dass die Setzung des Initialakzents, die Schwächung der Nebentonvokale und der Kasusverlust direkt miteinander zusammenhängen müssen, wie dies in vielen Sprachgeschichten zur deutschen Sprache suggeriert wird. Erstens weist der höchstalemannische Dialekt des Sensebezirks zwar die Vollvokale /\textit{a}/, /\textit{ə}/, /\textit{ɪ}/ und /\textit{ʊ}/ im Nebenton auf, die Kasusmarkierung am \isi{Substantiv} ist jedoch abgebaut \citep[116, 179–190]{Henzen1927}. Im Gegensatz dazu hat die deutsche Standardsprache nur Schwa im Nebenton (außer in Fremd- und Lehnwörtern), dafür ist im Gegensatz zum Sensebezirk die Markierung einiger \isi{Kasus} am \isi{Substantiv} erhalten (wenn auch nur sehr geringfügig) \citep[158--169]{Eisenberg2006}. Es kann also erstens festgehalten werden, dass die Setzung des Initialakzents zur Schwächung der Nebentonsilben führen kann, aber nicht muss.\footnote{Vgl. auch die kritische Diskussion dieser wenig hinterfragten Annahme in Caro Reina (2011: 103–105).} Weitere Beispiele dafür sind Finnisch, Ungarisch und Tschechisch, die ebenfalls einen Initialakzent haben, in denen jedoch die Nebensilben stabil sind.\footnote{Vielen Dank an Martin Kümmel für diesen Hinweis.} Zweitens kann die Ursache für die Nicht-Mar\-kie\-rung von \isi{Kasus} die Zentralisierung der Nebensilben sein. Ein Beispiel dafür ist die deutsche Standardsprache, ein Gegenbeispiel der höchstalemannische Dialekt des Sensebezirks. Dieser Dialekt hat zwar die Kasusmarkierung am \isi{Substantiv} abgebaut (trotz des Erhalts der Vollvokale in der Nebensilbe), markiert jedoch interessanterweise \isi{Kasus} am \isi{Adjektiv} (vgl. Paradigmen 7 und 27). Die Ursache für den Abbau der Kasusmarkierung am \isi{Substantiv} in diesem Dialekt ist also nicht in der Phonologie zu finden, sondern im Umbau im System der Kasusmarkierung, folglich innerhalb der Flexionsmorphologie.\largerpage[-2]\pagebreak

Der zweite Grund, weshalb die nominale und nicht die verbale Flexionsmorphologie untersucht wird, ist die Tatsache, dass die nominale Flexionsmorphologie unterschiedliche Wortarten aufweist. Synchron kann somit die Komplexität der verschiedenen Wortarten eines Sprachsystems miteinander verglichen werden (z.\,B.\ Komplexität des bestimmten und \isi{unbestimmten Artikels} in Varietät A), aber auch die Komplexität einer Wortart in den verschiedenen Sprachsystemen (z.\,B.\ die Komplexität des \isi{bestimmten Artikels} in Varietät A und Varietät B). Diachron sind in den deutschen Varietäten vor allem Determinierer grammatikalisiert worden. Man kann also fragen, ob diese neuen Wortarten die Flexionsmorphologie insgesamt komplexer gemacht haben oder ob Ausgleichstendenzen zwischen den Wortarten zu beobachten sind (wird z.\,B.\ die Artikelsetzung immer durch einen Abbau der Kasusmarkierung am \isi{Substantiv} kompensiert?). Die nominale Flexionsmorphologie bietet also sozusagen eine größere Angriffsfläche, über die verschiedenen Prozesse und Mechanismen innerhalb eines morphologischen Systems Antworten zu bekommen.

\subsection{Komplexität}\label{3.1.2}

In der vorliegenden Arbeit ist mit Komplexität stets absolute Komplexität gemeint. Welche linguistischen Phänomene für welche Sprecher-/Hörergruppen (L1/L2) schwierig oder aufwändig sind (= relative Komplexität), interessiert hier nicht. Im Fokus steht das linguistische System selbst. Die grobe Grundidee dahinter lautet, dass ein System umso komplexer ist, je mehr Elemente es aufweist. Die Einteilung in absolute und relative Komplexität wurde von \citet{Miestamo2008} eingeführt und wird in \sectref{2.2.5} erläutert. \citet{Rescher1998} liefert in seiner philosophischen Auseinandersetzung mit dem Konzept Komplexität eine detailliertere Taxonomie, welche von \citet{MiestamoSinnemäkiKarlsson2008} auf linguistische Phänomene übertragen wurde. Diese verschiedenen Arten an Komplexität werden in \sectref{2.2.5} ausgeführt. Hier soll nun dargestellt werden, welche Arten von Komplexität in der vorliegenden Arbeit untersucht werden. Dazu ist zuerst zu überlegen, welche Phänomene ein System mehr oder weniger komplex machen. Um dabei eine maximal mögliche Objektivität zu erreichen, ist ein theoriegeleitetes Vorgehen zu bevorzugen.

Die theoretischen Annahmen und wie diese zur Messung von Komplexität operationalisiert werden können, wird in \sectref{4.1} ausführlich vorgestellt. Für dieses Kapitel ist vorerst nur wichtig, dass die \is{Lexical-Functional Grammar (LFG)}Lexical-Functional Grammar (LFG) und die in\-fe\-ren\-tiel\-le-re\-a\-li\-sie\-ren\-de Morphologie die theoretische Grundlage bilden. \citet{Stump2001} schlägt zur Erfassung von Morphologie sogenannte \isi{Realisierungsregeln} (RR) vor, die ausschließlich die Form (und nicht die Funktion) definieren, und zwar nicht nur von \isi{Affixen}, sondern überhaupt die Form eines Wortes, also z.\,B.\ auch Wur\-zel-/Stamm\-mo\-di\-fi\-ka\-tio\-nen. Bezogen auf die Flexionsmorphologie sind RRs Instruktionen/Regeln zum Aufbau eines Paradigmas. Hinsichtlich der Komplexität gilt: Je mehr RRs ein System hat, desto komplexer ist dieses System. Basierend auf \citeauthor{Rescher1998}s \citeyearpar{Rescher1998} Klassifikation wird hier folglich generative Komplexität gemessen.

Welche linguistischen Phänomene die Flexionsmorphologie nun mehr oder weniger komplex machen, wird in \sectref{4.2} ausführlich dargestellt und von \sectref{4.1} (theoretische Grundlage und Operationalisierung dieser zur Messung von Komplexität) abgeleitet. Um aber zu verstehen, mit welcher Art von Komplexität wir uns befassen, werden die wichtigsten Punkte, die die Flexionsmorphologie komplexer machen, hier zusammengefasst. (in Klammern stehen die Arten der Komplexität):

\begin{enumerate}
\item 
Anzahl grammatischer Eigenschaften, die in der Flexion unterschieden und overt markiert werden, z.\,B.\ Anzahl \isi{Kasus}, \isi{Genera} etc. (konstitutionelle und taxonomische Komplexität).
\item 
Anzahl Allomorphe, z.\,B.\ Anzahl Pluralallomorphe (konstitutionelle und organisationelle Komplexität).
\item 
Mehrfachausdruck derselben Funktion, z.\,B.\ \textit{Wäld}-\textit{er} (Plural wird durch \isi{Umlaut} und \isi{Suffix} ausgedrückt; konstitutionelle und organisationelle Komplexität).
\item 
Bestimmte Art von \isi{Synkretismus}, nämlich wenn die Werte von mehr als zwei Features variieren; wird z.\,B.\ das \isi{Suffix} -\textit{er} der starken Adjektivflexion im Genitiv Feminin Singular und Genitiv Plural (keine Genusunterscheidung) suffigiert, so sind zwei RRs nötig (die Werte von \isi{Genus} und \isi{Numerus} variieren; organisationelle Komplexität).\footnote{Detailliert beschrieben wird dieser Fall von \isi{Synkretismus} in \sectref{4.1.3.3}.}
\end{enumerate}

Wie bereits erwähnt, wird hier generative (also epistemische) Komplexität gemessen, denn je mehr Instruktionen (d.h. RRs) gebraucht werden, um ein Flexionsparadigma zu produzieren, desto komplexer ist ein Flexionssystem. Betrachtet man jedoch, für welche Phänomene RRs benötigt werden (wie oben 1.–4. aufgelistet), so befassen wir uns mit den ontologischen Arten der Komplexität: Konstitutionelle, taxonomische und organisationelle Komplexität. Mit konstitutioneller Komplexität ist die Größe des Inventars gemeint, mit taxonomischer Komplexität die Anzahl kodierter Unterscheidungen in einem System. Je mehr Kategorien also unterschieden (1.) und je häufiger diese Kategorien overt markiert werden (2. und 3.), desto komplexer ist ein System. Unter der organisationellen Komplexität versteht man allgemein formuliert alles, was dem Eins-Zu-Eins-Verhältnis zwischen Form und Funktion widerspricht. Dazu gehört Allomorphie, d.h., eine Funktion kann durch verschiedene Formen ausgedrückt werden (z.\,B.\ unterschiedliche \isi{Suffixe} für Plural). Ein Extremfall von Allomorphie ist, wenn eine Funktion nicht nur unterschiedlich ausgedrückt werden kann, sondern wenn sie am selben Wort unterschiedlich markiert wird (z.\,B.\ Mehrfachausdruck von Plural, \textit{Wäld}-\textit{er}). Aber auch der umgekehrte Fall, d.h., Homonymie, kommt vor: Eine Form hat mehrere Funktionen. Dies betrifft den beschriebenen \isi{Synkretismus} (4.). Es liegt eine Form vor (-\textit{er}), die jedoch unterschiedliche Funktionen hat, die nicht einheitlich (d.h. durch eine RR) erfasst werden können.

In dieser Arbeit wird also unter Komplexität Folgendes verstanden:

\begin{itemize}
\item 
Kompositionelle Komplexität (konstitutionelle und taxonomische Komplexität): Je mehr Kategorien unterschieden werden und je größer das Inventar an Markierungsmöglichkeiten ist, desto komplexer ist das Flexionssystem.
\item 
Organisationelle Komplexität: Allomorphie und Homonymie führen zu einer höheren Komplexität.
\item 
Generative Komplexität: Kompositionelle und organisationelle Komplexität werden automatisch durch \isi{Realisierungsregeln} erfasst (wie diese in \sectref{4.1.3} definiert sind). Weil \isi{Realisierungsregeln} eine Art Instruktion zum Aufbau eines Paradigmas darstellen, wird also auch generative Komplexität gemessen.
\end{itemize}

\section{Hypothesen}\label{3.2}

Das Ziel dieses Kapitels ist, die Fragestellungen und Hypothesen vorzustellen sowie die Varietäten des Samples anhand derer diese Hypothese überprüft werden. Die Varietäten selbst werden im \sectref{3.3} beschreiben. Die Hypothesen basieren vorwiegend auf Trudgills Arbeiten, deren wichtigste Überlegungen und Resultate bereits im \sectref{2.2.3} vorgestellt wurden. Aus diesem Grund werden diese hier nur kurz wiederaufgegriffen und nicht detailliert diskutiert. Das Kapitel ist nach folgenden Faktoren gegliedert: \isi{Diachronie} (\sectref{3.2.1}), \isi{Dialektgruppe} (\sectref{3.2.2}), Kontakt (\sectref{3.2.3}), \isi{Standardvarietät} (\sectref{3.2.4}) und \isi{Isolation} (\sectref{3.2.5}).

\subsection{Diachronie}\label{3.2.1}

Es stellt sich hier die Frage, ob Sprachen im Laufe der Zeit tendenziell komplexer oder einfacher werden oder ob ihre Komplexität konstant bleibt. Alle drei Hypothesen sind in der Linguistik vertreten. Vertreter der \is{Equi-Complexity-Hypothese}\textit{Equi-Com\-ple\-xi\-ty-Hy\-po\-the\-se} nehmen an, dass zwar die Komplexität einzelner Ebenen der Grammatik variieren kann. Verrechnet man jedoch die Komplexität der verschiedenen linguistischen Beschreibungsebenen, sind die Sprachen immer gleich komplex. Ein oft genanntes Beispiel sind die angenommenen Ausgleichstendenzen zwischen der Morphologie und der Syntax: Was in der Morphologie nicht ausgedrückt wird, wird in der Syntax kodiert und umgekehrt. Dies impliziert also, dass Sprachen aus diachroner Perspektive immer gleich komplex bleiben. Eine weitere prominente Hypothese ist, dass Sprachen und vor allem ihre Morphologie kontinuierlich an Komplexität verlieren. Sie fußt wohl vorwiegend auf dem Wissen über alte indogermanische Sprachen, die im Vergleich zu modernen indogermanischen Sprachen eine relativ reiche Flexionsmorphologie aufweisen. Schließlich konnte vor allem die soziolinguistische Typologie zeigen, dass Sprachen auch diachrone Komplexifizierung aufweisen (vgl. \sectref{2.2.1}, \sectref{2.2.3} und \sectref{2.2.4}), d.h., dass Sprachen komplexer werden können. Zurzeit liegt noch keine Messung der Gesamtkomplexität einer Sprache vor, weil zahlreiche theoretische und methodische Fragen offenstehen (vgl. \sectref{2.1.2}). Es konnte jedoch nachgewiesen werden, dass einzelne Phänomene oder Teilsysteme synchron (im Vergleich mit anderen Sprachen/Varietäten) oder diachron komplexer sind. Es handelt sich dabei vorwiegend um Sprachen, die von einer Sprachgemeinschaft mit bestimmten Charakteristika gesprochen werden: klein, isoliert, mit einem dichten Netzwerk und wenig L2-Ler\-nern. Sprachen dieser Sprachgemeinschaften können eine höhere Komplexität aufweisen, die nicht auf Sprachkontakteffekte zurückgeführt werden kann. Die höhere Komplexität wird dadurch erklärt, dass es in diesen Sprachgemeinschaften einfacher ist, einen Wandel des phonologisch eher unnatürlichen Typs sowie den Zuwachs an morphologischen Kategorien durchzuführen und zu erhalten (\citealt[11]{Trudgill1996} und \citealt[109]{Trudgill2009}; ausführlich vorgestellt in \sectref{2.2.3}). Trudgill nennt dies spontane Komplexifizierung \citep[71]{Trudgill2011}. Einen zweiten Typ von Komplexifizierung bezeichnet Trudgill als \textit{Additive Borrowing} \citep[27]{Trudgill2011}. Es handelt sich dabei um Elemente oder Kategorien, die von der einen in die andere Sprache übernommen werden, ohne dass in der übernehmenden Sprache bereits existierende Elemente oder Kategorien substituiert werden. Dies kommt in Sprachen von Sprachgemeinschaften vor, die sich in einer langzeitigen koterritorialen Kontaktsituation befinden, in der folglich Kinder zweisprachig aufwachsen \citep[34]{Trudgill2011}.

In dieser Arbeit gehe ich \textit{erstens} davon aus, dass Sprachen bezüglich ihrer Komplexität aus diachroner Perspektive variieren (Zunahme und Abnahme von Komplexität), da die \is{Equi-Complexity-Hypothese}\textit{Equi-Com\-ple\-xi\-ty-Hy\-po\-the\-se} als allgemeine Tendenz verworfen werden kann (vgl. Diskussion \sectref{2.1.2}). Es soll also untersucht werden, wie sich Komplexität diachron entwickelt und wie diese unterschiedlichen Entwicklungen erklärt werden können. Dazu wird Althochdeutsch mit Mittelhochdeutsch verglichen sowie Alt-/Mittelhochdeutsch mit modernen alemannischen Dialekten und mit der deutschen Standardsprache. \textit{Zweitens} kann spontane Komplexifizierung in Dialekten beobachtet werden, die von einer kleinen, \isi{isolierten Sprachgemeinschaft} mit engen Netzwerken und wenig L2-Ler\-nern gesprochen wird. In diesem Sample trifft diese Definition am besten auf folgende Dialekte zu (zur Begründung, weshalb diese Dialekte als isoliert gelten, s. \sectref{3.3.3}): Issime (Höchstalemannisch), Visperterminen (Höchstalemannisch), Jaun (Höchstalemannisch), Vorarlberg (Hochalemannisch), Huzenbach (Schwäbisch), Münstertal (Oberrheinalemannisch, Elsass). \textit{Drittens} können \textit{Additive Borrowings} in bi- oder multilingualen Sprachgemeinschaften erwartet werden. Zu den bi- oder multilingualen Sprachgemeinschaften gehören in diesem Sample die Sprachinseldialekte Issime (Höchstalemannisch), Petrifeld (Schwäbisch) und Elisabethtal (Schwäbisch) wie auch die elsässischen Dialekte (Oberrheinalemannisch).

\subsection{Dialektgruppen}\label{3.2.2}

Die alemannischen Dialekte werden in die folgenden Gruppen eingeteilt: Höchstalemannisch, Hochalemannisch, Oberrheinalemannisch, Bodenseealemannisch und Schwäbisch, wobei die drei letzten Gruppen auch unter dem Begriff Niederalemannisch subsumiert werden können. Da für das Bodenseealemannische keine vollständige Beschreibung der nominalen Flexionsmorphologie vorliegt, wird es im weiteren Verlauf nicht berücksichtigt. Diese Einteilung der alemannischen Dialekte basiert vorwiegend auf phonologischen Phänomenen, z.\,B.\ Nasalausfall vor Frikativ (Höchst- vs. Hochalemannisch), \textit{k}{}-Verschiebung (Hoch- vs. Niederalemannisch), frühneuhochdeutsche Diphthongierung (Oberrheinalemannisch vs. Schwäbisch). Ergänzt wird diese Klassifikation durch morphologische, morphosyntaktische und lexikalische Eigenschaften, z.\,B.\ Flexion der prädikativen \isi{Adjektive} (Höchst- vs. Hochalemannisch), verbaler Einheitsplural -\textit{e}/-\textit{et} sowie die Lexeme \textit{Matte}/\textit{Wiese} ‘Wiese’ (Oberrheinalemannisch vs. Schwäbisch). Es stellt sich also die Frage, ob diese Einteilung der alemannischen Dialekte sich auch in der strukturellen Komplexität dieser Dialekte widerspiegelt. Somit könnte diese Klassifikation durch einen weiteren Faktor ergänzt werden.

Da die höchstalemannischen Dialekte eine reiche Flexionsmorphologie und die niederalemannischen Dialekte eine schlanke Flexionsmorphologie aufweisen, gehe ich von folgender Komplexitätshierarchie aus: Höchstalemannisch > Hochalemannisch > Niederalemannisch. Innerhalb des Niederalemannischen können weiter Oberrheinalemannisch und Schwäbisch miteinander verglichen werden. Da es für das badische Oberrheinalemannisch nur eine Grammatik gibt, die eine vollständige Beschreibung der nominalen Flexionsmorphologie enthält, macht es wenig Sinn, das badische Oberrheinalemannisch mit dem elsässischen Oberrheinalemannisch oder dem Schwäbischen zu vergleichen.

\subsection{Kontakt}\label{3.2.3}

\isi{Sprachkontakt} kann sowohl zur Komplexifizierung als auch zur Simplifizierung linguistischer Systeme führen. Im vorangehenden Kapitel wurde gezeigt, dass in Sprachen bi- oder multilingualer Sprachgemeinschaften \textit{Additive Borrowings} gefunden werden, wodurch die Komplexität einer Sprache erhöht wird. Wie bereits erwähnt wurde, trifft dies auf folgende Dialekte zu: Issime (Höchstalemannisch), Petrifeld (Schwäbisch) und Elisabethtal (Schwäbisch) wie auch auf die elsässischen Dialekte (Oberrheinalemannisch). Es müssen hier jedoch zwei Einschränkungen gemacht werden. Erstens entwickelte sich eine bilinguale Sprachgemeinschaft im Elsass erst nach der Französischen Revolution und vor allem nach dem 2. Weltkrieg. Zweitens ist über die Sprachkompetenz in der Kontaktsprache der Sprecher von Petrifeld und Elisabethtal (zur Zeit der Publikation der hier verwendeten \isi{Ortsgrammatiken}) nichts Genaues bekannt. Bei Issime hingegen handelt es sich um eine alte Sprachinsel (seit dem 13. Jh.) und Informationen zur Kontaktsituation und Sprachkompetenz sind vorhanden. Ausführlich dargestellt und diskutiert wird dies in \sectref{3.3.3} und \sectref{6.3.1}.

Simplifizierung hingegen kommt in Sprachen von großen Sprachgemeinschaften mit vielen Kontakten, losen Netzwerken und vielen L2-Ler\-nern vor (\citealt[146]{Trudgill2011}; ausführlich vorgestellt in \sectref{2.2.3}). In diesen Sprachen werden besonders Irregularitäten, Redundanzen und Opazität reduziert \citep[101]{Trudgill2009}. Um diese Hypothese der Simplifizierung zu überprüfen, lassen sich bezüglich der alemannischen Dialekte die Stadt- mit den Landdialekte vergleichen.\footnote{Standardvarietäten und Nicht-Stan\-dard\-va\-ri\-e\-tä\-ten werden in \sectref{3.2.4} verglichen, \isi{geografisch isolierte} und nicht isolierte Dialekte in \sectref{3.2.5}.} Es kann davon ausgegangen werden, dass die Sprachgemeinschaft einer Stadt größer ist und mehr Kontakte, eher lose Netzwerke und mehr L2-Ler\-ner aufweist, als dies auf dem Land der Fall ist. Aus dem hier untersuchten Sample (vgl. \tabref{table3.1}) werden für das Hochalemannische Bern und Zürich mit Vorarlberg verglichen, für das Schwäbische Stuttgart mit Bad Saulgau und Huzenbach, für das Oberrheinalemannische Colmar mit Münstertal, Elsass (Ebene) und Kaiserstuhl. Für Städte im höchstalemannischen Gebiet liegen keine \isi{Ortsgrammatiken} vor, die die Flexionsmorphologie aller untersuchten Wortarten beschreiben. Der Vergleich Stadt–Land wird innerhalb derselben \isi{Dialektgruppe} vorgenommen, da die Flexionsmorphologie mancher \isi{Dialektgruppen} eine höhere Komplexität aufweist als die Flexionsmorphologie anderer \isi{Dialektgruppen} (vgl. \sectref{6.2}).

\subsection{Standardvarietät}\label{3.2.4}

Bezüglich der Standardsprache widersprechen sich die Erwartungen, weil sowohl Simplifizierung als auch Komplexifizierung möglich sind. Laut \citet{Ferguson1959} ist zu erwarten, dass Standardvarietäten (\textit{High} \textit{Varieties}) eine höhere Komplexität aufweisen als Nicht-Stan\-dard\-va\-ri\-e\-tä\-ten (\textit{Low} \textit{Varieties}) (\citealt[333]{Ferguson1959}; vgl. \sectref{2.1.2}). Des Weiteren ist denkbar, dass Kategorien durch Kodifizierung besser konserviert werden können. Ein Beispiel hierfür ist der Erhalt des Genitivs in der deutschen Standardsprache, während dieser in den meisten Dialekten abgebaut wurde. Zudem können auch kulturelle Faktoren eine Rolle spielen. Beispielsweise zeigen sich die Varietäten im südlichen Teil des deutschsprachigen Raumes sehr progressiv, z.\,B.\ durch die Syn- und Apokope von \textit{e} in der Flexion \citep[275]{vonPolenz2013}. Grammatiker des 17. und 18. Jahrhunderts forderten jedoch, das -\textit{e} zur Unterscheidung von \isi{Numerus} am \isi{Substantiv} beizubehalten \citep[275]{vonPolenz2013}. Solche Bemühungen sind auch auf bestimmte kulturelle Kontexte zurückzuführen: „Nach der Reformationszeit, im Zusammenhang mit der stärkeren überregionalen Sprachvereinheitlichung, ist eine deutliche Tendenzwende zu beobachten: Schriftsteller, Korrektoren, Drucker und Sprachgelehrte bemühen sich um schreib- und und drucksprachliche Restitutionen von sprechsprachlich längst geschwundenen Flexionsendungen“ \citep[275]{vonPolenz2013}.

Vor dem Hintergrund der Diskussion im vorangehenden Kapitel ist jedoch zu erwarten, dass eine Standardsprache eine geringere Komplexität als ein Dialekt aufweist. Es wurde dargestellt, dass die Sprachen von großen Sprachgemeinschaften mit einem losen Netzwerk, vielen Kontakten und vielen L2-Ler\-nern dazu tendieren, strukturell einfacher zu werden. Noch besser als die in \sectref{3.2.3} erwähnten Stadtdialekte treffen Standardvarietäten auf diese Definition zu. Dies gilt besonders bezüglich der L2-Ler\-ner. Erstens lernt, wer Deutsch als Fremdsprache erwirbt, meistens zumindest die deutsche Standardsprache, wobei der Erwerb eines Dialektes bzw. einer Regionalsprache nicht ausgeschlossen ist. Dies trifft vorwiegend auf Deutschland und Österreich zu, in geringerem Maße auf die Schweiz (besonders bezogen auf den ungesteuerten Spracherwerb). Zweitens lernen viele Deutsch-Muttersprachler einen Dialekt oder eine Regionalsprache als L1, in der Schweiz gilt dies für alle Deutsch-Muttersprachler. In diesen Fällen kann die Standardsprache wohl nicht als Fremdsprache angesehen werden, trotzdem fängt der Erwerb der Standardsprache etwas später als der L1 an. Ein weiterer Grund, weshalb bei einer Standardsprache eine geringere Komplexität zu erwarten ist, liegt in der Natur der Standardisierung. Das Ziel eines Standards ist die Vereinheitlichung und Vereinfachung, wobei die Vereinfachung u.a. als Konsequenz aus der Vereinheitlichung resultieren kann. Bei der Standardisierung einer Sprache wird bestimmt, welche Varianten richtig oder falsch sind, was zu einer Variantenreduzierung führt \citep[231]{vonPolenz1999}. Welche Varianten als richtig und falsch klassifiziert werden, hängt von unterschiedlichen kultur- und wissenschaftsgeschichtlichen Faktoren ab, worauf hier nicht weiter eingegangen wird. Schließlich ist hier noch zu bemerken, dass Deutsch vor allem im 19. und 20. Jahrhundert stark standardisiert wurde, was vorwiegend politische und soziale Ursachen hat \citep[232]{vonPolenz1999}. In derselben Zeit konnte sich die deutsche Standardsprache auch aus sozialer und regionaler Sicht ausbreiten, u.a. aufgrund von „Industrialisierung, Verstädterung, Bevölkerungsmischung, […] Fernverkehr[…], […] Verschriftlichung des täglichen Lebens“ \citep[232]{vonPolenz1999}, aber auch aufgrund der Bildung von Nationalstaaten und der Alphabetisierung der Bevölkerung \citep[233]{vonPolenz1999}. Die deutsche Standardsprache hat folglich eine deutlich geringere diachrone Tiefe als die Dialekte.

Aus dieser Diskussion scheint es plausibler anzunehmen, dass die strukturelle Komplexität einer Standardsprache geringer ist als jene von Nicht-Stan\-dard\-spra\-chen. Dies schließt jedoch nicht den Erhalt von einzelnen Kategorien aus, die in den Dialekten abgebaut wurden (z.\,B.\ Genitiv). Um zu überprüfen, ob Standardsprachen weniger komplex sind als Nicht-Stan\-dard\-spra\-chen, wird die deutsche Standardsprache mit den hier untersuchten alemannischen Dialekten verglichen.

\subsection{Isolation}\label{3.2.5}

In den beiden vorangehenden Kapiteln wurde gezeigt, dass Sprachen innerhalb eines bestimmten Typs von Sprachgemeinschaft zur Simplifizierung tendieren. Sprachen hingegen, die von kleinen, stabilen Sprachgemeinschaften mit einem engen Netzwerk, wenig \isi{Sprachkontakt} und kaum L2-Ler\-nern (= \isi{sozial isoliert}, vgl. \sectref{2.2.3}) gesprochen werden, sind laut \citet{Trudgill2011} tendenziell komplexer \citep[146–147]{Trudgill2011}. Es handelt sich dabei um kleine, stabile Sprachgemeinschaften . Diese Sprachgemeinschaften können zusätzlich auch \isi{geografisch isoliert} sein, womit in den bisherigen Studien meistens Sprachen bzw. Sprachgemeinschaften in den Bergen gemeint sind (vgl. \sectref{2.2.4}). Sprachen solcher Sprachgemeinschaften tendieren erstens dazu, ihre strukturelle Komplexität in größerem Umfang zu erhalten, da der Sprachwandel langsamer abläuft \citep[103]{Trudgill2011}. Zweitens kann auch spontane Komplexifizierung beobachtet werden (\citealt[71]{Trudgill2011}, vgl. \sectref{3.2.1} zur \isi{Diachronie}).

Um diese Hypothese zu überprüfen, bietet das hier untersuchte Sample zwei Möglichkeiten (vgl. \tabref{table3.1}). \textit{Erstens} können die Landdialekte den Stadtdialekten gegenübergestellt werden, wofür bereits in \sectref{3.2.3} argumentiert wurde. \textit{Zweitens} werden Dialekte, die zusätzlich \isi{geografisch isoliert} sind, mit geografisch nicht isolierten Dialekten verglichen, d.h. Stadtdialekte und geografisch nicht isolierte Landdialekte vs. \isi{geografisch isolierte} Landdialekte.\footnote{Inwiefern welche Dialekte \isi{geografisch isoliert} bzw. nicht isoliert sind, wird in \sectref{3.3} dargestellt.} Dies geschieht innerhalb derselben \isi{Dialektgruppe}, weil die Flexionsmorphologie gewisser \isi{Dialektgruppen} komplexer ist als die Flexionsmorphologie anderer \isi{Dialektgruppen} (vgl. \sectref{6.2}). Für die vier hier untersuchten alemannischen \isi{Dialektgruppen} (Höchstalemannisch, Hochalemannisch, Oberrheinalemannisch, Schwäbisch) liegen vollständige Beschreibungen der nominalen Flexionsmorphologie von geografisch isolierten und nicht-isolierten Dialekten vor.\\

% \textbf{\tabref{table3.1}: Die untersuchten alemannischen Dialekte}\\

\begin{table}
\caption{Die untersuchten alemannischen Dialekte}\label{table3.1}
\begin{tabular}{lllc}
\lsptoprule
{DG} & {Dialekt} & {Stadt/Land} & {geogr. isoliert}\\\midrule
h-st\il{Höchstalemannisch} & Issime & Land & \ding{52}\\
& Visperterminen & Land & \ding{52}\\
& Jaun & Land & \ding{52}\\
& Sensebezirk & Land & \ding{55}\\
& Uri & Land & \ding{55}\\
hoch\il{Hochalemannisch} & Vorarlberg & Land & \ding{52}\\
& Zürich & Stadt & \ding{55}\\
& Bern & Stadt & \ding{55}\\
schw\il{Schwäbisch} & Huzenbach & Land & \ding{52}\\
& Bad Saulgau & Land & \ding{55}\\
& Stuttgart & Stadt & \ding{55}\\
& Petrifeld & Land & \ding{55}\\
& Elisabethtal & Land & \ding{55}\\
oberr\il{Oberrheinalemannisch} & Münstertal & Land & \ding{52}\\
& Elsass (Ebene) & Land & \ding{55}\\
& Colmar & Stadt & \ding{55}\\
& Kaiserstuhl & Land & \ding{55}\\
\lspbottomrule
\end{tabular}
\end{table}

\subsection{Zusammenfassung}\label{3.2.6}

In der Folge werden die Hypothesen nochmal zusammengefasst. Gleichzeitig wird gezeigt, welche Vergleiche zur Überprüfung der Hypothesen angestellt werden.\\

\noindent
Diachrone Zunahme und/oder Abnahme von Komplexität, d.h., wie sich Komplexität diachron entwickelt:
\begin{itemize}
\item Ahd. vs. Mhd.
\item Ahd./Mhd. vs. alemannische Dialekte und deutsche Standardsprache: Ob AHD oder MHD für den Vergleich mit einer bestimmten modernen Varietät gewählt wird, ist abhängig davon, auf welche Stufe diese Varietät zurückgeführt werden kann, vgl. \sectref{6.1.1}).
\end{itemize}

\noindent
\isi{Dialektgruppen} unterscheiden sich in ihrer Komplexität, und zwar:
\begin{itemize}
\item Höchstalemannisch ist komplexer als Hochalemannisch und Hochalemannisch ist komplexer als Niederalemannisch.
\end{itemize}

\noindent
Spontane Komplexifizierung wird in isolierten Dialekten erwartet:
\begin{itemize}
\item Ahd./Mhd. vs. isolierte alemannische Dialekte (diachroner Vergleich).
\item Nicht isolierte vs. isolierte alemannische Dialekte, innerhalb derselben \isi{Dialektgruppe} (synchroner Vergleich).
\end{itemize}

\noindent
\isi{Additive Borrowings} werden in mehrsprachigen Sprachgemeinschaften erwartet:
\begin{itemize}
\item Ahd./Mhd. und Alemannisch einer einsprachigen Sprachgemeinschaft vs. Alemannisch einer zwei- oder mehrsprachigen Sprachgemeinschaft.
\end{itemize}

\noindent
Wird ererbte Komplexität erhalten oder ist Simplifizierung festzustellen:
\begin{itemize}
\item Geografisch nicht isolierte Dialekte simplifizieren eher, während \isi{geografisch isolierte} Dialekte eher ererbte Komplexität erhalten (Vergleich innerhalb derselben \isi{Dialektgruppe}).
\item Stadtdialekte simplifizieren, Landdialekte bewahren ererbte Komplexität stärker (Vergleich innerhalb derselben \isi{Dialektgruppe}).
\item Deutsche Standardsprache simplifiziert, alemannische Dialekte erhalten ererbte Komplexität stärker.
\end{itemize}

\section{Varietäten}\label{3.3}

In diesem Kapitel werden alle analysierten Varietäten vorgestellt und die verwendeten (Orts-) Grammatiken aufgeführt. Bezüglich der alemannischen Dialekte wird zudem erörtert, ob es sich um einen Stadt- oder Landdialekt handelt sowie ob und weshalb der Dialekt \isi{geografisch isoliert} ist oder nicht. Es werden zuerst die älteren Stufen des Deutschen präsentiert (\sectref{3.3.1}), dann die deutsche Standardsprache (\sectref{3.3.2}) und schließlich die alemannischen Dialekte (\sectref{3.3.3}).

Es sei des Weiteren darauf hingewiesen, dass die Grammatiken sich in ihrer Art der Beschreibung unterscheiden. Für die Messung der Komplexität konnten die Paradigmen folglich nicht verbatim aus den Grammatiken übernommen werden. Vielmehr wurden die relevanten Informationen den Grammatiken entnommen und daraus neue Paradigmen nach einheitlichen Kriterien erstellt. Welche diese Kriterien sind und wie dabei genau vorgegangen wurde, wird in \sectref{4.3.2} und \sectref{5} erklärt.

\subsection{Ältere Stufen des Deutschen}\label{3.3.1}
\subsubsection{Althochdeutsch}

Unter Althochdeutsch versteht man die älteste Stufe des Deutschen. Der Beginn wird mit der 2. Lautverschiebung im späten 6. Jahrhundert angesetzt und das Ende mit der Abschwächung der Endsilbenvokale in der 2. Hälfte des 11. Jahrhunderts, wobei jedoch eine fortwährende schriftliche Überlieferung erst Ende des 8. Jahrhunderts beginnt \citep[1]{Braune2004}. Das Althochdeutsche verfügt über keine überdachende Koiné, sondern besteht aus dem Oberdeutschen (Alemannisch und Bairisch), dem Mitteldeutschen (Rhein- und Mittelfränkisch) und dem Ostfränkischen, das den Übergang zwischen Ober- und Mitteldeutsch bildet \citep[1, 6]{Braune2004}.

Als grammatische Beschreibung des Althochdeutschen werden die beiden\linebreak Bände der althochdeutschen Referenzgrammatik von Braune (2004, Laut- und Formenlehre) und von \citet[Syntax]{Schrodt2004} verwendet. Diese Grammatik bildet eine Art ‘Normalalthochdeutsch’ ab, basierend vorwiegend auf dem althochdeutschen Tatian, also auf einer ostfränkischen Handschrift aus dem 8. Jh. \citep[6]{Braune2004}. Es handelt sich dabei folglich um ein schriftbasiertes Normalalthochdeutsch, das in dieser Form nie existiert hat. Trotzdem gibt es etliche Gründe, weshalb diese Grammatik für die Analyse herangezogen werden kann. Erstens wäre jene Grammatik am idealsten, die einen althochdeutschen Ortsdialekt zu einem bestimmten Zeitpunkt beschreibt (z.\,B.\ St. Gallen um 830). Eine solche Beschreibung liegt jedoch zum gegenwärtigen Zeitpunkt nicht vor. Zweitens ist die Grammatik von \citet{Braune2004} und \citet{Schrodt2004} die ausführlichste und umfassendste Beschreibung des Althochdeutschen. Drittens beschreibt sie das Althochdeutsche zu einem bestimmten Zeitpunkt, nämlich das Althochdeutsche des 9. Jahrhunderts. Gibt es durch Sprachwandel bedingte \isi{Variation}, werden die älteren und jüngeren Varianten genannt. Viertens basiert diese Grammatik zwar auf einem ostfränkischen Dialekt, dialektale Besonderheiten werden jedoch ebenfalls angegeben \citep[6]{Braune2004}. Fünftens sind auch die \isi{Ortsgrammatiken} zu den alemannischen Dialekten nicht uneingeschränkt repräsentativ für den gesamten Ort, da meist nur wenige Sprecher befragt wurden. Diese Ortgrammatiken stellen folglich ebenfalls nur einen Ausschnitt dar und können deshalb mit der althochdeutschen Grammatik von \citet{Braune2004} und \citet{Schrodt2004} verglichen werden.

Da \citet{Braune2004} ältere und jüngere Formen sowie dialektale Charakteristika unterscheidet, ist noch zu klären, welche Varianten für die vorliegende Analyse verwendet werden. Werden ältere und jüngere Formen genannt, wird die ältere Form gewählt. Zum Beispiel: Instr.Sg.m./n. = -\textit{u}, Nom.Sg.f. = -\textit{u}, Dat.Sg.f. = -\textit{u}/-\textit{eru}/\textit{iru}, Dat.Sg.m/n. = -\textit{emu}/\textit{imu}, Nom.Sg.f. = -\textit{u}, Dat.Pl. = V\textit{m}. Werden ober- von mitteldeutschen Varianten oder alemannische von Varianten anderer Dialekte unterschieden, wird die oberdeutsche bzw. alemannische Variante bevorzugt. Zum Oberdeutschen, wozu das Alemannische, das Ostfränkische und das Bairische gehören, ist noch zu erwähnen, dass die beiden Dialekte in althochdeutscher Zeit deutlich weniger Unterschiede aufweisen als später \citep[7]{Braune2004}.

\subsubsection{Mittelhochdeutsch}

Das Mittelhochdeutsche beginnt mit der Schwächung der Endsilbenvokale (Mitte 11. Jh.) und endet neben anderen Phänomenen mit der frühneuhochdeutschen Diphthongierung und Monophthongierung (Mitte 14. Jh., wobei der Anfang der Diph- und Monophthongierung früher anzusetzen ist) \citep[18–21]{Paul2007}. Wie das Althochdeutsche ist auch das Mittelhochdeutsche dialektal gegliedert und weist keine überdachende\largerpage Varietät auf \citep[34]{Paul2007}. Zudem zeigen die mittelhochdeutschen Handschriften auch stilistische Unterschiede \citep[11]{Paul2007}. Folglich bildet die mittelhochdeutsche Referenzgrammatik von \citet{Paul2007} ebenfalls ein Konstrukt, das vorwiegend auf dem sogenannten klassischen Mittelhochdeutsch (ca. 1170–1250) basiert \citep[10]{Paul2007}. Aus denselben Gründen, die bereits für das Althochdeutsche dargestellt wurden, kann diese Grammatik trotzdem für diese Arbeit verwendet werden.

\subsection{Deutsche Standardsprache}\label{3.3.2}
\largerpage
Die Daten für die deutsche Standardsprache stammen aus der deskriptiven Grammatik von \citet{Eisenberg2006}, die eine vollständige Wort- und Satzgrammatik enthält. Es handelt sich dabei um die vorwiegend geschriebene Variante der Standardsprache. Eine umfassende Grammatik der gesprochenen Standardsprache gibt es meines Wissens nicht. Außerdem ist die gesprochene Standardsprache stets regional gefärbt, d.h., für die vorliegende Analyse bräuchte es nicht nur eine Beschreibung für den alemannischen Raum, sondern auch mindestens je eine für die unterschiedlichen Staaten des alemannischen Raumes. Umfassende und detaillierte Beschreibungen der nominalen Flexionsmorphologie dieser regionalen Varietäten der deutschen Standardsprache existieren jedoch nicht. Um klarer zu machen, um welche Unterschiede es sich u.a. handelt, wird hier ein Beispiel angefügt. In der Schweiz werden üblicherweise die vollen Formen des \isi{unbestimmten Artikels} benutzt (z.\,B.\ \textit{eine Birne}), während in weiten Teilen Deutschlands (teils auch im alemannischsprachigen Raum) die reduzierten Formen verwendet werden (z.\,B.\ \textit{ne Birne}). Dass die deutsche Standardsprache in der Schweiz oft als ʻSchriftspracheʼ bezeichnet wird, kommt also (aus unterschiedlichen Gründen) nicht von ungefähr: Die deutsche Standardsprache in der Schweiz orientiert sich relativ stark an der Schrift. Da es also regionale Unterschiede in der deutschen Standardsprache gibt, aber keine vollständigen und detaillierten Beschreibungen für das hier untersuchte Gebiet, wird die geschriebene deutsche Standardsprache in dieser Arbeit herangezogen, da es sich um jene \isi{Standardvarietät} handelt, die von allen Sprechern des alemannischen Raums verwendet wird, wenn auch von einigen von ihnen vorwiegend schriftlich.

Neben diesem praktischen gibt es jedoch noch einen soziolinguistischen\linebreak Grund, weshalb die geschriebene Standardsprache herbeigezogen wird. Die\linebreak Sprachgemeinschaft, die diese Standardsprache verwendet (schriftlich wie mündlich), passt genau auf \citeauthor{Trudgill2011}s \citeyearpar{Trudgill2011} Definition des einen Typs von Sprachgemeinschaft: große Sprachgemeinschaft mit losen Netzwerken, viel Kontakt und vielen L2-Ler\-nern. Die Standardsprache fungiert als Lingua franca innerhalb des deutschsprachigen Raumes, ist normalerweise Unterrichtssprache und jene Variante des Deutschen, die im Unterricht für Deutsch als Fremdsprache vermittelt wird.

\subsection{Alemannische Dialekte}\label{3.3.3}

Die alemannischen Dialekte werden in fünf Gruppen eingeteilt, wie bereits in \sectref{3.2.2} dargestellt wurde: Höchstalemannisch, Hochalemannisch, Bodenseeale-\linebreak mannisch, Oberrheinalemannisch und Schwäbisch. Da für das Bodenseealemannische keine Grammatik mit einer vollständigen Beschreibung der nominalen Flexionsmorphologie existiert, wird diese Gruppe hier ausgeschlossen. Das Oberrheinalemannische kann weiter in ein elsässisches und ein badisches Oberrheinalemannisch unterteilt werden, da auch in der Vergangenheit das elsässische Oberrheinalemannisch deutlich stärker und länger dem französischen Einfluss ausgesetzt war.

Welche Dialekte und Grammatiken ausgewählt wurden, hat unterschiedliche Gründe. Die Grammatiken betreffend war wichtig, dass sie die Flexionsmorphologie aller hier untersuchten Wortarten beschreiben: \isi{Substantive}, \isi{Adjektive}, \isi{Personalpronomen}, \isi{Interrogativpronomen}, einfaches \isi{Demonstrativpronomen}, \isi{Possessivpronomen}, bestimmter und \isi{unbestimmter Artikel} (vgl. \sectref{4.3.2}). Würde für einen Dialekt die Komplexität einer Wortart mehr oder weniger gemessen werden, würde dies die Resultate verzerren. Bezüglich der Dialekte sollte erstens möglichst das gesamte alemannische Gebiet abgedeckt werden, d.h. alle Gruppen und alle Staaten. Mit Ausnahme des Bodenseealemannischen konnte dies erreicht werden (vgl. \tabref{table3.2}). Zweitens verfügt idealerweise jede Gruppe über Stadt- und Landdialekte sowie über \isi{geografisch isolierte} und nicht isolierte Dialekte. Dies ist für das Hochalemannische, für das elsässische Oberrheinalemannische und für das Schwäbische gewährleistet. Für das badische Oberrheinalemannische konnte nur eine Grammatik gefunden werden, die die Flexionsmorphologie aller analysierten Wortarten beschreibt. Für das Höchstalemannische gibt es keine Grammatiken eines Stadtdialekts. Drittens sollen auch alemannische Sprachinseln einbezogen werden. Dazu konnten \isi{Ortsgrammatiken} zu den Dialekten von Issime (Höchstalemannisch) sowie von Petrifeld und Elisabethtal (beide Schwäbisch) gefunden werden. Was bei der Wahl der Grammatiken nicht berücksichtigt werden konnte, war, dass alle Grammatiken in derselben Zeit entstanden sind. Die älteste verwendete Grammatik stammt aus dem Jahr 1886, die jüngste aus dem Jahr 1999. Die meisten Grammatiken sind jedoch aus der ersten Hälfte des 20. Jh., wodurch die Vergleichbarkeit wieder besser gegeben ist.

\begin{table}
\caption{Die untersuchten alemannischen Dialekte und ihre sprachexternen Eigenschaften}\label{table3.2}
\small\begin{tabularx}{\textwidth}{lllcXX}
\lsptoprule
{DG} & {Dialekt} & \multicolumn{1}{p{1cm}}{Stadt\slash\newline Land} & \multicolumn{1}{p{.75cm}}{geogr.\newline isoliert} & {Staat} & {Quellen}\\\midrule
h-st\il{Höchstalemannisch} & Issime & Land & \ding{52} & Italien & \citet{Zürrer1999} \citet{Perinetto1981}\\
& Visperterminen & Land & \ding{52} & Schweiz & \citet{Wipf1911}\\
& Jaun & Land & \ding{52} & Schweiz & \citet{Stucki1917}\\
& Sensebezirk & Land & \ding{55} & Schweiz & \citet{Henzen1927}\\
& Uri & Land & \ding{55} & Schweiz & \citet{Clauß1929}\\
hoch\il{Hochalemannisch} & Vorarlberg & Land & \ding{52} & Österreich & \citet{Jutz1925}\\
& Zürich & Stadt & \ding{55} & Schweiz & \citet{Weber1987}\\
& Bern & Stadt & \ding{55} & Schweiz & \citet{Marti1985}\\
schw & Huzenbach & Land & \ding{52} & Deutschland & \citet{Baur1967}\\
& Bad Saulgau & Land & \ding{55} & Deutschland & \citet{Raichle1932}\\
& Stuttgart & Stadt & \ding{55} & Deutschland & \citet{Frey1975}\\
& Petrifeld & Land & \ding{55} & Rumänien & \citet{Moser1937}\\
& Elisabethtal & Land & \ding{55} & Georgien & \citet{Žirmunskij1928/29}\\
oberr\il{Oberrheinalemannisch} (Baden) & Kaiserstuhl & Land & \ding{55} & Deutschland & \citet{Noth1993}\\
oberr (Elsass) & Münstertal & Land & \ding{52} & Frankreich & \citet{Mankel1886}\\
 & Elsass (Ebene) & Land & \ding{55} & Frankreich & \citet{Beyer1963}\\
 & Colmar & Stadt & \ding{55} & Frankreich & \citet{Henry1900}\\
\lspbottomrule
\end{tabularx}
\end{table}

In der Folge werden alle 17 alemannischen Dialekte bzw. die Orte, in denen sie gesprochen werden, kurz vorgestellt. Dabei liegt der Fokus vor allem auf der Darstellung, inwiefern ein Dialekt bzw. Ort \isi{geografisch isoliert} ist und zu welchem Typ Sprachgemeinschaft nach  \citeauthor{Trudgill2011}s \citeyearpar{Trudgill2011} Definition der Dialekt gehört (vgl. \sectref{2.2.3}, \sectref{3.2.3} und \sectref{3.2.5}.). Die Reihenfolge entspricht jener in \tabref{table3.2}.

\subsubsection{Issime}

Issime ist eine der höchstalemannischen Walser Sprachinseln in den Alpen des Aostatals (Italien), die im 13. Jh. entstanden sind. Genauer liegt Issime im Lystal, einem Seitental des Aostatals, auf 953 m.ü.M \citep[25]{Zürrer1999} und hat 432 Einwohner \citep{Issime2013}. Der Ort kann also als topografisch isoliert gelten, denn zudem „liessen sich [die Walser Kolonisten] in unwirtlichen Höhen nieder“, wie z.\,B.\ an Steilhängen, Terrassen, schmalen Hangleisten etc. \citep[31]{Zürrer1999}. Der Kontakt der Sprachinseln im Aostatal zum Deutschwallis wurde erst Ende des 19. Jh. unterbrochen, als Schienen- und Straßennetze gebaut und die alten Saumpässe zum Wallis vernachlässigt wurden \citep[28]{Zürrer1999}. Dies gilt jedoch nicht für Issime, das schon immer kaum Beziehungen zum Wallis hatte. Vielmehr gab es in Issime eine Teilassimilation an die Umgebungsgesellschaft \citep[28]{Zürrer1999}. Des Weiteren war in Issime Deutsch nie Schriftsprache oder Sprache der Schule und der Kirche \citep[29--30]{Zürrer1999}. Dazu kommt, dass die Walser Sprachinseln nur sehr wenig Kontakt untereinander hatten \citep[28]{Zürrer1999}.

Der Kontakt zu den benachbarten romanischen Sprachen hat dazu geführt, dass die Bewohner von Issime mehrsprachig sind. Zu den im Aostatal gesprochenen romanischen Sprachen gehören Piemontese, Dialekte des Franco-Provençal, sowie die Standardvarietäten des Französischen und Italienischen (beide offizielle Sprachen der autonomen Region Aostatal), wobei vor allem Standardfranzösisch und Franco-Provençal eine große Rolle gespielt haben \citep[28]{Zürrer1999}. Der alemannische Dialekt wird nur innerhalb von Issime und nur unter Alemannisch-Muttersprachlern gesprochen \citep[37]{Zürrer1999}. Beispielsweise wird sogar mit den Gressoneyer Walsern (13km entferntes Dorf, das ebenfalls eine Walser\linebreak Sprachinsel ist) auf Italienisch oder Piemontesisch gesprochen \citep[37]{Zürrer1999}.

Dass sich der Dialekt von Issime in besonderem Maße von den übrigen Dialekten entfernt hat, zeigen auch Zürrers Ausführungen auf Grundlage eigner Erfahrungen. Im Rahmen der SDS-Aufnahmen in den 1960er Jahren führte Zürrer als Schweizer Hochalemannischsprecher die Interviews in Issime auf Französisch \citep[38]{Zürrer1999}.

Für Issime kann also festgehalten werden, dass es sich dabei um eine kleine, \isi{isolierte Sprachgemeinschaft} handelt, deren Sprache nur innerhalb des Dorfes und nur mit Muttersprachlern gesprochen wird. Durch die Mehrsprachigkeit sind jedoch auch \textit{Additive Borrowings} zu erwarten. Ein Fall ist tatsächlich unter den \isi{Personalpronomen} zu finden (vgl. \sectref{5.3.1}).

\subsubsection{Visperterminen}

Vispterterminen ist ein Dorf in den Walliser Alpen (Kanton Wallis, Schweiz), das auf 1378 m.ü.M. liegt und zur Zeit der Erhebung ca. 600 Einwohner hatte \citep[1]{Wipf1911}. Noch heute führt nur eine Straße nach Visperterminen, die kurz nach dem Ort endet. Das Ziel von Wipf war, den Dialekt „eines möglichst abgelegenen, noch nicht von dem großen Touristenstrome ergriffenen Walliser Dorfes“ zu beschreiben \citep[1]{Wipf1911}. Außerdem beschränkten sich die Kontakte außerhalb des Dorfes vorwiegend auf einige Einkäufe in Visp und Stalden (beide im Tal gelegen) und auf den Militärdienst in Sitten, Brig (beide Wallis) und Chur (Graubünden) \citep[1--2]{Wipf1911}. Wie Issime kann also auch Visperterminen zu den kleinen und \isi{isolierten Sprachgemeinschaften} gezählt werden. Im Gegensatz zu Issime jedoch wird Visperterminen Alemannisch von einer einsprachigen Sprachgemeinschaft gesprochen und ist vom Deutschen überdacht.

\subsubsection{Jaun}

Jaun ist ein Bergdorf in den Freiburger Voralpen (Kanton Freiburg, Schweiz) und liegt auf 1030 m.ü.M. \citep[1]{Stucki1917}. Zur Zeit der Erhebung hatte das Dorf 802 Einwohner \citep[10]{Stucki1917}. Jaun ist das vorletzte Dorf des Jauntals. Dahinter liegt nur Abläntschen, ein Bergdorf mit etwa 100 Einwohnern, das der Gemeinde Saanen (Kanton Bern) zugeteilt ist und dessen Dialekt zum westlichen Berner Oberländischen gehört \citep[2--3]{Stucki1917}. Verbunden mit Jaun war es durch ein „schlechtes Fahrsträßchen“ \citep[2]{Stucki1917}. Mit Ausnahme von sehr geringem Handel und der Post (eine Person aus Abläntschen holte täglich die Post in Jaun ab), gab es nur wenig Kontakt zwischen den beiden Dörfern. Die Bewohner von Abläntschen tätigten die Behördengänge in Saanen (Kanton Bern), wohin sie auch neben Zweisimmen (Kanton Bern) zu den großen Märkten gingen \citep[3]{Stucki1917}. Für die Bewohner von Jaun ist Bulle (französischsprachig, Kanton Freiburg) \citep[1]{Stucki1917} der Ort für Behördengänge und Märkte. Schließlich trennte die unterschiedliche Religion die beiden Dörfer: Während die Bewohner von Jaun mehrheitlich katholisch sind, sind die Bewohner von Abläntschen protestantisch, was zur damaligen Zeit einen engen Kontakt zwischen den Dörfern verhinderte \citep[3 ,9]{Stucki1917}. Mit zwei weiteren deutschsprachigen Dörfern ist Jaun durch Schotterwege oder Pfade über Pässe (Scheitelpunkt über 1500 m) verbunden: Boltigen (Simmental, Kanton Bern) und Plaffeien (Sensebezirk, Kanton Freiburg) \citep[2--3]{Stucki1917}. Es wäre anzunehmen, dass Jaun gerade mit Plaffeien viel Kontakt hatte: ähnlicher Dialekt, Kanton Freiburg, katholisch. Nach Plaffeien führte jedoch nur ein Pfad, der als „beschwerlich [und] durchweg schlecht unterhalten“ beschrieben wird \citep[3]{Stucki1917}, wohin noch heute nur ein Wanderweg führt. Die Dörfer, die im Tal vor Jaun liegen, wie auch der gesamte Bezirk Greyerz, zu dem Jaun gehört, sind französischsprachig. Seit 1875 (also 42 Jahre bevor Stuckis Grammatik erschien) verbindet „eine gute Fahrstraße“ das Dorf Jaun talabwärts mit den französischsprachigen Orten Charmey, Broc und vor allem Bulle \citep[1]{Stucki1917}. Jaun kann also definitiv als ein kleiner, isolierter Ort gelten, dessen Dialekt zur damaligen Zeit vorwiegend im Dorf selbst gesprochen wurde (die meisten Kontakte nach außen führten ins französischsprachige Gebiet) und der praktisch keine L2-Ler\-ner hat.

\subsubsection{Sensebezirk}\largerpage

Die Grammatik von \citet{Henzen1927} erfasst den Dialekt des Sensebezirks (hauptsächlich katholisch) sowie der neun katholischen Gemeinden der Pfarrei Gurmels im Seebezirk des Kantons Freiburg (Schweiz). Im weiteren Verlauf der Arbeit wird der Einfachheit halber auf dieses gesamte Gebiet mit \textit{Sensebezirk} referiert. Dieses Gebiet ist zwar klar begrenzt: Im Süden die Voralpen, im Westen das französischsprachige Gebiet, im Osten und Norden hochalemannisches berndeutsches Gebiet. Zum Berner Gebiet gab es im Gegensatz zu heute nur wenig Kontakte, weil dieses hauptsächlich protestantisch war \citep[2]{Henzen1927}. Die Mobilität innerhalb dieses Gebietes war jedoch relativ hoch, und zwar vor allem durch Heirat und durch die Kleinbauern, die ihren Wohnort oft wechseln mussten \citep[9]{Henzen1927}. Nur in einigen Gemeinden des voralpinen Oberlandes ist dies deutlich weniger ausgeprägt \citep[8--9]{Henzen1927}. Dies führte dazu, dass die Dialekte der einzelnen Dörfer keine sehr großen Unterschiede aufwiesen. \citet{Henzen1927} spricht von einem „Mischdialekt“ \citep[1]{Henzen1927}. Des Weiteren ist die Stadt Freiburg aus der Erhebung ausgeschlossen, einbezogen wurden nur Dörfer mit durchschnittlich 838 Einwohnern \citep[8]{Henzen1927}. Es handelt sich hierbei also um einen Landdialekt, der im Vergleich zu Issime, Visperterminen und Jaun jedoch nicht isoliert ist.

\subsubsection{Uri}\largerpage

Die Grammatik von \citet{Clauß1929} umfasst das Reußtal und seine Seitentäler (ohne Talschaft Urseren) im Kanton Uri (Schweiz). Es handelt sich dabei um ein in den Alpen zwischen Urnersee und Gotthard liegendes Tal \citep[1]{Clauß1929}. Laut \citet{Clauß1929} kann das untersuchte Gebiet linguistisch als weitgehend einheitlich angesehen werden. Ein geschlossenes Gebiet ist vor allem das Reußtal, die Dialekte der Seitentäler weisen, wenn überhaupt, vor allem phonologische Unterschiede auf \citep[11--12]{Clauß1929}. Bei den erhobenen Orten handelt es sich um Dörfer. Zwar wurde auch Altdorf, der Hauptort des Kantons Uri, einbezogen, der jedoch zum Zeitpunkt der Erhebung um 1920 nur ca. 4000 Einwohner zählte (Historisches Lexikon der Schweiz ~\citeyear{LexA2011}, ~\citetitle{LexA2011}).

Schließlich stellt sich noch die Frage nach der Isoliertheit. Seit mindestens der römischen Zeit wird der Gotthard als Verkehrsweg genutzt, wenn auch kein kontinuierlicher Verkehrsfluss nachgewiesen werden kann (Historisches Lexikon der Schweiz ~\citeyear{LexG2011},~\citetitle{LexG2011}). Um 1200 entstand die erste Brücke über die Schöllenen, zwischen 1166/1176 und 1230 wurde eine Kapelle auf der Passhöhe eingeweiht (Historisches Lexikon der Schweiz ~\citeyear{LexG2011},~\citetitle{LexG2011}). 1707 wurde der erste Tunnel der Alpen auf der Urner Seite gebaut (Historisches Lexikon der Schweiz ~\citeyear{LexG2011},~\citetitle{LexG2011}). Zwischen 1810 und 1831 wurde der Pass für Kutschen (im Winter für Postschlitten) fahrbar gemacht (Historisches Lexikon der Schweiz ~\citeyear{LexG2011},~\citetitle{LexG2011}). Folglich kann vor allem das Reußtal nicht als isoliert gelten Bei diesem Untersuchungsareal handelt es sich also um ein ländliches, nicht isoliertes Gebiet.

\subsubsection{Vorarlberg}

Im weiteren Verlauf dieser Arbeit wird dieses Gebiet \textit{Vorarlberg} genannt, obwohl nur der Süden Vorarlbergs (ohne die Walsertäler), aber zusätzlich das Fürstentum Liechtenstein von \citet{Jutz1925} erhoben wurden. Zum Untersuchungsgebiet des südlichen Vorarlberg gehören der Walgau (unteres Illtal), das Montafon (oberes Illtal) und das Klostertal \citep[3]{Jutz1925}. Größere Städte gab es zu Beginn des 20. Jh. in diesem Gebiet nicht. Bezüglich Kontakte/\isi{Isolation} beschreibt \citet{Jutz1925} vorwiegend das Ill- und Klostertal. Zwischen dem Walgau und dem Rheintal existierte eine alte Straße von Rankweil über Göfis \citep[6]{Jutz1925}, wobei \citet{Jutz1925} aber nicht spezifiziert, was er genau mit ‘alt’ meint. Er merkt jedoch an, dass der Dialekt des Walgau von jenem des Rheintals beeinflusst ist \citep[3]{Jutz1925}. Kaum Einfluss gab es von Feldkirch her, da der Verkehrsweg durch die Illschlucht erst relativ jung war \citep[5]{Jutz1925}. Die Dialekte im Montafon und im Klostertal waren kaum vom Rheintal beeinflusst \citep[3]{Jutz1925}. Vielmehr sieht \citet{Jutz1925} für das gesamte Gebiet einen wachsenden schriftsprachlichen Einfluss durch Schule, Kirche, Industrie und Tourismus, der jedoch vorwiegend den Wortschatz betraf \citep[4--5]{Jutz1925}. Allgemein kann jedoch das Ill- und Klostertal zur damaligen Zeit durch seine geografische Lage als abgeschieden gelten \citep[3]{Jutz1925}. Auch für dieses Gebiet kann also von einer eher kleinen, \isi{isolierten Sprachgemeinschaft} ausgegangen werden, und zwar vor allem auch im Vergleich zu den beiden anderen hochalemannischen Dialekten (Bern und Zürich), die hier analysiert werden. Wichtig ist in diesem Zusammenhang noch \citeauthor{Jutz1925}ʼ \citeyearpar{Jutz1925} Anmerkung, dass „der Grundzug der Maa. [= Mundarten] überall derselbe ist“ und dass auch höher gelegene Orte berücksichtigt werden, da diese einen archaischeren Dialekt aufweisen \citep[7, 9]{Jutz1925}. Gibt es Unterschiede zwischen den Gebieten, so sind diese markiert. In diesem Fall werden in dieser Arbeit die Varianten des Montafon und des Klostertals verwendet, da diese abgeschiedener als das Walgau liegen.

Schließlich ist noch zu erwähnen, dass in der vorliegenden Arbeit Vorarlberg zum Hochalemannischen gezählt wird. \citet{Jutz1925} ordnet dieses Gebiet dem Niederalemannischen zu, weil germ. \textit{k} im Anlaut nur bis zur Affrikata und nicht weiter zum Frikativ verschoben ist (\citealt[9]{Jutz1925}, vgl. auch VALTS, Bd. III, Karten 40–53). Da aber das Niederalemannische (z.\,B.\ Bodenseealemannisch und Oberrheinalemannisch) germ. \textit{k} im Anlaut nicht verschoben hat (SSA, Karten II 105.00–105.05), wird das von \citet{Jutz1925} untersuchte Gebiet zum Hochalemannischen gerechnet. Ein weiteres Abgrenzungskriterium sind die gerundeten Vordervokale, deren Isoglosse im äußersten Norden von Vorarlberg verläuft. \footnote{Herzlichen Dank an Oliver Schallert für diesen Hinweis.}

\subsubsection{Zürich}

Als die Grammatik von \citet{Weber1987} erschien, hatte die Stadt Zürich bereits 383.568 Einwohner \citep{Zürich2015}. Zürich liegt im Schweizer Mittelland, d.h., sie ist topografisch nicht isoliert. Diese Sprachgemeinschaft kann also zum Typ der großen Sprachgemeinschaften mit viel Kontakt gezählt werden. Der Dialekt von Zürich gehört zu den östlichen hochalemannischen Dialekten \citep[51–67]{Hotzenköcherle1984}. 

\subsubsection{Bern}

Auch die Sprachgemeinschaft der Stadt Bern kann zum Typ der großen Sprachgemeinschaften mit viel Kontakt gerechnet werden. Wie Zürich liegt auch Bern im Schweizer Mittelland, es gibt also keine topografischen Gegebenheiten, die Bern isolieren. In der Zeit der Publikation von \citeauthor{Marti1985}s \citeyearpar{Marti1985} Grammatik hatte die Stadt Bern 145.254 Einwohner (Zahlen für das Jahr 1980, \citealt{Bern2014}). Da diese Grammatik deutlich später erschienen ist als die meisten anderen Grammatiken, die für diese Arbeit berücksichtigt wurden, ist es wichtig, die Zahlen aus der ersten Hälfte des 20. Jh. für Bern zu betrachten: 90.937 Einwohner im Jahr 1910, 111.783 Einwohner im Jahr 1930. Die Sprachgemeinschaft kann also auch für die erste Hälfte des 20. Jh. zum anfangs genannten Typ gezählt werden. Der Dialekt von Bern ist wie jener von Zürich Teil des Hochalemannischen, und zwar des westlichen Hochalemannisch \citep[51--67]{Hotzenköcherle1984}.

\subsubsection{Huzenbach}\largerpage[1]

Huzenbach liegt zwischen 450 und 950 m.ü.M. im oberen Murgtal im Schwarzwald (Deutschland) und kann mit seinen 753 Einwohnern als Dorf in einer ländlichen Gegend gelten \citep{Baiersbronn2015}.\footnote{Ältere Daten zu den Einwohnern konnten leider nicht gefunden werden.} Huzenbach gehört zur Gemeinde Baiersbronn, von der es jedoch 12 km entfernt situiert ist. Obwohl heute eine Bundesstraße durch das Murgtal führt, war der obere Teil dieses Tals lange Zeit isoliert. Huzenbach liegt an keinem der vier älteren Hauptübergänge durch den Schwarzwald (z.\,B.\ Römerstraße) \citep[28–29]{Baur1967}. Erst am Ende des 18. Jhs. wurde das hintere mit dem vorderen Murgtal durch eine Straße nach Gernsbach verbunden \citep[29]{Baur1967}. Am Anfang des 20. Jhs. wurde das ganze Tal ans Eisenbahnnetz angeschlossen, was die Ansiedlung von Industrie und den Tourismus im Murg- und Kinzigtal förderte \citep[30]{Baur1967}. Im Vergleich zu den beiden schwäbischen Orten Bad Saulgau und Stuttgart kann Huzenbach folglich als klein und isoliert gelten.

\subsubsection{Bad Saulgau}

Bad Saulgau liegt im östlichen Teil des Landkreises Sigmaringen (Ba\-den-Würt\-tem\-berg, Deutschland), zwischen Biberach und dem Bodensee. Mit 17.080 Einwohnern\footnote{Vgl. Fußnote 6.} und einer mittleren Besiedlungsdichte \citep{Bund2014} gilt der Ort als halbstädtisch (vgl. \citealt{Eurostat2011}). Die Grammatik von \citet{Raichle1932} berücksichtigt neben Bad Saulgau auch etliche Dörfer in der Umgebung der Kleinstadt. Im Gegensatz zu Stuttgart handelt es sich also um eine eher ländliche Gegend, die jedoch im Vergleich mit Huzenbach nicht als isoliert gelten kann.

\subsubsection{Stuttgart}

Stuttgart ist die Landeshauptstadt von Ba\-den-Würt\-tem\-berg. Bei der Publikation der Grammatik von Frey im Jahr 1975 hatte Stuttgart 628.598 Einwohner \citep{Stuttgart2015}. Aufgrund ihrer Bevölkerungsdichte gilt sie als städtisch ( \citealt{Bund2014}, vgl. \citealt{Eurostat2011} ). Wie die Sprachgemeinschaften von Bern und Zürich kann auch jene von Stuttgart zu den großen Sprachgemeinschaften mit vielen Kontakten und losen Netzwerken gezählt werden.

\subsubsection{Petrifeld}

Petrifeld (rum. Petrești, ung. Mezőpetri) liegt im Sathmargebiet. Dieses gehörte zu Ungarn und ging 1919 an Rumänien \citep[14]{Moser1937}. In diesem Gebiet gab es 33 schwäbische Siedlungen, wovon drei nach 1919 zu Ungarn gehörten und 30 zu Rumänien \citep[15--16]{Moser1937}. Die Bevölkerungszahl pro Dorf lag zwischen 600 und 2000. Die schwäbischen Einwanderer waren überwiegend katholisch \citep[16]{Moser1937} und stammten aus dem Gebiet zwischen Donau, Iller und Bodensee \citep[Karte 1 und 2]{Moser1937}, also genau aus jenem Gebiet, in dem auch Bad Saulgau liegt. In den rumänischen Bezirken Sălai und Sătmar gaben 1927 ca. 6\% der Bewohner an, deutscher Herkunft zu sein \citep[15--16]{Moser1937}. Die schwäbischen Dörfer selbst waren jedoch vollständig oder vorwiegend deutsch \citep[16]{Moser1937}, wobei es aber große Unterschiede in der Kenntnis des Schwäbischen zwischen den Bewohnern der einzelnen Dörfer gab. Durchschnittlich führten ca. 2/3 der Bevölkerung an, Schwäbisch oder Standarddeutsch zu können \citep[17]{Moser1937}.

Die Grammatik von \citet{Moser1937} beschreibt den Dialekt von Petrifeld, der stellvertretend für das gesamte Sathmargebiet steht, denn die Dialekte aller Orte wiesen große Ähnlichkeiten auf \citep[23]{Moser1937}. Es kann also angenommen werden, dass es Kontakte zwischen den schwäbischsprachigen Dörfern gab. Detaillierte Infomationen zu diesen möglichen Kontakten liegen jedoch nicht vor. Petrifeld gehörte zur Westgruppe, welche eine geschlossenen Siedlungsgruppe bildete und von ungarischen und rumänischen Dörfern umgeben war \citep[17]{Moser1937}. Gegründet wurde Petrifeld 1740–1741 \citep[19]{Moser1937}. 1930 zählte das Dorf 1588 Bewohner, wobei folgende Muttersprachen angegeben wurden: 1276 Bewohner Deutsch, 264 Bewohner Ungarisch, 30 Bewohner Romani, 18 Bewohner Rumänisch \citep[61]{Varga2002}. Laut \citet{Moser1937} beeinflussten das Rumänische und Ungarische vor allem den Schwäbischen Wortschatz \citep[102]{Moser1937}, wobei die Lehnwörter zumeist phonologisch integriert wurden \citep[103]{Moser1937}. Einfluss übte besonders das Ungarische als Verwaltungs- und Handelssprache, weniger das Rumänische \citep[103]{Moser1937}.

Wir haben es hier also mit einer mehrheitlich ländlichen Gegend zu tun, die jedoch geografisch nicht als isoliert gelten kann. Da \citet{Moser1937} keine genauen Angaben macht, wie kompetent die deutschsprachigen Bewohner in der ungarischen und rumänischen Sprache waren, ist bezüglich möglicher \textit{Additive Borrowings} keine Voraussage möglich. Vielmehr wird sich zeigen, dass der Dialekt von Petrifeld zwar rumänische und ungarische Lehnwörter aufweist, jedoch keine \textit{Additive Borrowings} in der Flexionsmorphologie.

\subsubsection{Elisabethtal}

Die Grammatik von \citet{Žirmunskij1928/29} basiert auf den Dialekten der Dörfer Katharinenfeld (georg. Bolnissi) und Elisabethtal (georg. Asureti) \citep[38]{Žirmunskij1928/29}, die sich im heutigen Georgien befinden. Bei den Bewohnern handelte es sich um Schwaben aus dem Neckartal, ungefähr zwischen Stuttgart und Esslingen im Norden, Tübingen und Reutlingen im Süden \citep[56]{Žirmunskij1928/29}. Diese waren Pietisten, welche sich von der lutherischen Kirche getrennt hatten \citep[39]{Žirmunskij1928/29}. Neben religiösen waren es besonders wirtschaftliche Gründe, die sie zum Auswandern bewogen: große Bevölkerung, Krieg, relativ hohe Steuerlast, schlechte Ernten, Hungersnot etc. \citep[41]{Žirmunskij1928/29}. Die Auswanderung in die Südukraine und in den Südkaukasus fand zwischen 1816 und 1819 statt \citep[39]{Žirmunskij1928/29}. Wie viele Personen nach Katharinenfeld und Elisabethtal kamen, nennt \citet{Žirmunskij1928/29} nicht. Jedoch gibt er an, dass ins südkaukasische Gebiet 2629 Menschen kamen, die sich auf 7 Orte verteilten \citep[42]{Žirmunskij1928/29}. Es kann also festgehalten werden, dass es sich um kleine Dörfer gehandelt hat. Laut \citet{Schrenk1997} lebten in den 1860er Jahren 851 Personen in Elisabethtal \citep[202–203]{Žirmunskij1928/29}. Des Weiteren sind keine Details über den Kontakt zwischen den Kolonien in Georgien bekannt. Die Menschen lebten aber zumindest nicht völlig isoliert voneinander, da die Kolonien einen gemeinsamen Fonds für gemeinnützige Zwecke hatten, in den jede Gemeinde einbezahlte \citep[198]{Schrenk1997}. Des Weiteren verfügten sie über eine gemeinsame Synode \citep[199]{Schrenk1997}.

Laut \citet{Žirmunskij1928/29} werden diese schwäbischen Mundarten von der\linebreak deutschen geschriebenen Standardsprache beeinflusst \citep[58]{Žirmunskij1928/29}. Neben einigen russischen Lehnwörtern (vgl. \citealt[52]{Žirmunskij1928/29}) konnten in der Flexionsmorphologie keine \textit{Additive Borrowings} gefunden werden. Dasselbe wurde bereits für Petrifeld festgestellt. Im Gegensatz zu diesen beiden Sprachinseln gibt es jedoch im Dialekt von Issime ein \textit{Additive Borrowing} (vgl. \sectref{5.3.1}). Dies könnte dadurch erklärt werden, dass Issime (13. Jh.) eine deutlich ältere Sprachinsel ist als Petrifeld (18. Jh.) und Elisabethtal (19. Jh.).

Schließlich ist noch zu erwähnen, dass \citet{Žirmunskij1928/29} die Varianten sowohl für Elisabethtal als auch für Katharinenfeld angibt, wenn sich diese voneinander unterscheiden. In diesem Fall wird in der vorliegenden Arbeit die Variante von Elisabethtal aufgenommen, da die Mundart von Elisabethtal die am besten erhaltene war \citep[58]{Žirmunskij1928/29}. Wir können also festhalten, dass es sich bei Elisabethtal wie bei Petrifeld um ein Dorf in einer ländlichen Gegend handelte, welches geografisch nicht isoliert und wohl auch sozial zumindest nicht völlig isoliert war.

\subsubsection{Kaiserstuhl}

Die Grammatik von \citet{Noth1993} basiert auf dem Dialekt von Rotweil (heute: politische Gemeinde Vogtsburg-Oberrotweil) \citep[293]{Noth1993}. Rotweil liegt an der westlichen Seite des Kaiserstuhls (Ba\-den-Würt\-tem\-berg, Deutschland), ca. 25 km von Freiburg entfernt. Zum Rhein sind es etwa 5 km, der gleichzeitig die Staatsgrenze zwischen Frankreich und Deutschland bildet. Im weiteren Verlauf der Arbeit wird auf diesen Dialekt immer mit \textit{Kaiserstuhl} referiert. Da Rotweil nicht im Zentrum, sondern an der Peripherie des Kaiserstuhlgebirges liegt, kann Rotweil als nicht isoliert gelten. Die Gemeinde Vogtsburg-Oberrotweil hat 5737 Einwohner, weist eine geringe Besiedlungsdichte auf und kann folglich als ländlich charakterisiert werden (Statistisches Bundesamt).

\subsubsection{Münstertal}

Die Grammatik von \citet{Mankel1886} beschreibt den Dialekt des Münstertals, das in den Vogesen unweit von Colmar liegt (Elsass, Frankreich). Das Haupttal führt von Colmar bis nach Münster, dazwischen liegen etwa 15 km \citep[1]{Mankel1886}. Bei Münster gabelt sich das Tal in das sogenannte Großtal und Kleintal, welche 8–9 km lang sind \citep[1]{Mankel1886}. Beschrieben wird die Grammatik des Großtals, da dieses am meisten von den übrigen elsässischen Dialekten abweicht \citep[2]{Mankel1886}. Alle drei Täler, aber besonders das Großtal können als isoliert gelten, denn die Bewohner dieser Täler hatten aufgrund der Berge kaum Kontakt mit anderen Gebieten \citep[1]{Mankel1886}. \citet{Mankel1886} weist darauf hin, dass sich wegen dieser \isi{Isolation} die Dialekte „eigenartig ausgebildet“ haben \citep[1]{Mankel1886}. Übrigens führt heute zwar eine \textit{Route Nationale} durch das Haupt- und Kleintal, jedoch nur eine \textit{Route Départementale} durch das Großtal. Des Weiteren handelt es sich bei den Orten in allen drei Tälern um Dörfer. Die erhobenen Orte im Großtal hatten im Jahr 1886 zwischen 996 und 1091 Einwohner, Münster (Hauptort und größter Ort des Münstertals) 1886 wie auch heute ca. 5.000 Einwohner \citep{MotteVouloirSarrabezolles2015}. Die erhobenen Dörfer des Großtals sind: Mühlbach (frz. Muhlbach-sur-Munster), Breitenbach (frz. Breitenbach-Haut-Rhin), Metzeral und Sondernach \citep[2]{Mankel1886}.

Die Sprachgemeinschaft im Großtal kann also als klein, isoliert, mit wenig Kontakten und engen Netzwerken charakterisiert werden. Wichtig ist hier noch zu erwähnen, dass die Variante des Großtals in die vorliegende Arbeit übernommen wurde, wenn zwei Varianten für das Großtal und das Haupt-/Kleintal angegeben werden.

Neben dem Dialekt des Münstertals gehören zum untersuchten Sample auch der Dialekt von Colmar und jener der elsässischen Rheinebene, welche in den nachfolgenden Kapiteln vorgestellt werden. Da heutzutage die Sprecher elsässischer Dialekte mindestens bilingual sind, stellt sich die Frage, wie lange dies schon der Fall ist. Deswegen wird in der Folge ein Teil der elsässischen Sprachgeschichte skizziert, wobei die Ausführungen äußerst kurz gefasst sind, denn eine ausführliche Darstellung würde den Rahmen dieser Arbeit sprengen.

Während der Völkerwanderung siedelten sich Alemannen und Franken im heutigen Elsass an, das vorher romanischsprachiges (und evtl. keltisches) Gebiet war \citep[2]{Lösch1997}. Zum Königreich Frankreich gehörte das Elsass erst ab 1681. Außer dem Adel und Teilen des gehobenen Bürgertums, die Französisch konnten, wurden im Elsass weiterhin alemannische Dialekte gesprochen \citep[7]{Lösch1997}. Erst nach der Französischen Revolution kann eine Französisierung des Gebiets durch eine auch repressive Sprachpolitik festgestellt werden \citep[7--10]{Lösch1997}. Trotzdem hielten sich die elsässischen Dialekte wie auch die deutsche Standardsprache, wohl auch, weil es keine Schulpflicht gab und Messen auf Deutsch gehalten wurden \citep[10--11]{Lösch1997}. 1871–1918 gehörte das Elsass zum Deutschen Kaiserreich. Die Zeit bis 1914 ist von einer gewissen sprachlichen Liberalität geprägt, da es je nach Gemeinde deutsche und französische Schulen gab und auch Literatur und Zeitungen in beiden Sprachen vertrieben werden konnte \citep[12]{Lösch1997}. Während des 1. Weltkrieges durfte nur noch auf Deutsch unterrichtet werden \citep[16]{Lösch1997}. In der Zeit von 1871 bis 1918 war jedoch vor allem das Bürgertum zweisprachig, Bauern und Arbeiter vorwiegend deutschsprachig \citep[13]{Lösch1997}. 1918–1940 gehörte das Elsass wieder zu Frankreich. Bis 1927 wurde an den Schulen ausschließlich auf Französisch unterrichtet, ab 1927 war Deutsch „in eingeschränktem Maße in den Schulen wieder zugelassen“ \citep[18]{Lösch1997}. 1940 wurde das Elsass de facto dem Deutschen Reich einverleibt \citep[20]{Lösch1997}. Die Nationalsozialisten führten eine „Entfranzösisierungskampagne“ \citep[20]{Lösch1997}, woraus u.a. resultierte, dass Französischsprachige vertrieben wurden und Deutsch die einzige Unterrichtssprache war \citep[21]{Lösch1997}. 1944 wurde das Elsass „von amerikanischen und französischen Verbänden […] zurückerobert“ und gehörte nach Kriegsende wieder zu Frankreich \citep[21]{Lösch1997}. Französisch war wieder Sprache des Unterrichts und des öffentlichen Lebens, Deutsch spielte in den Medien kaum noch eine Rolle \citep[22]{Lösch1997}. Dadurch dehnte sich Französisch auch stärker in den privaten und familiären Bereich aus \citep[25]{Lösch1997}. Erst in den 1980er Jahren im Zuge der Regionalisierung Frankreichs änderte sich die Situation, „Schulen, Radio und Fernsehen sollten den Regionalsprachen eröffnet werden“ \citep[26]{Lösch1997}. Daraus kann geschlossen werden, dass sich eine deutsch-französische Zweisprachigkeit erst seit der Französischen Revolution allmählich ausbreitete. Diese doch eher kurze Zeit ist vielleicht auch ein Grund, weshalb in der nominalen Flexionsmorphologie der elsässischen Dialekte keine \textit{Additive Borrowings} gefunden werden können. Ähnliches wurde bereits in Bezug auf Petrifeld und Elisabethtal beobachtet. Demgegenüber hat Issime eine viel längere Geschichte der Mehrsprachigkeit (seit dem 13. Jh.).

\subsubsection{Elsass (Ebene)}

Die Grammatik von \citet{Beyer1963} beschreibt die nominale Flexionsmorphologie des gesamten Elsass (Frankreich). Gibt es zwischen Regionen Unterschiede, werden diese genannt und geografisch eingeordnet. In diesen Fällen wurden für die hier vorgestellte Auswertung jeweils die Varianten des Zentrums (im Gegensatz zu nördlichen und südlichen Varianten) gewählt, da dieses Gebiet ungefähr zwischen dem Kaiserstuhl und dem Münstertal liegt.

Welche Orte genau erhoben wurden, ist aus der sonst sehr detaillierten Beschreibung nicht ersichtlich. Der größte Teil des Gebiets liegt jedoch in der Rheinebene (weswegen im weiteren Verlauf darauf mit \textit{Elsass (Ebene)} referiert wird), in der neben den beiden größeren Städten Strasbourg und Mulhouse und mittleren Städten wie Colmar auch viele Dörfer liegen. Da 180 Orte erhoben wurden \citep[15]{Beyer1963}, ist also davon auszugehen, dass es sich bei der Mehrheit der Orte um Dörfer handelt. Bei diesem Gebiet handelt es sich folglich nicht um ein isoliertes, aber um ein eher ländliches Gebiet.

\subsubsection{Colmar}

Colmar ist eine Stadt im Elsass (Frankreich), die im oben beschriebenen Gebiet Elsass (Ebene) liegt. Im Gegensatz zum Gebiet Elsass (Ebene), für das vorwiegend der Dialekt der Dörfer erhoben wurde, handelt es sich bei Colmar um eine Stadt. Im Jahr 1901 (die hier verwendete Grammatik erschien 1900) hatte Colmar 36.844 Einwohner \citep{MotteVouloirSarrabezolles2015}. Die Sprachgemeinschaft in Colmar kann um 1900 als eine große, nicht \isi{isolierte Sprachgemeinschaft} mit vielen Kontakten und losen Netzwerken charakterisiert werden.
% ####################################################################

\setcounter{chapter}{3}
\chapter{Multi-level selection and language systematicity}\is{multi-level selection}\is{systematicity}
\label{c:experiment1}

\section{Introduction}
The baseline experiments of the previous chapter looked at how analog\is{analogy}y could be exploited for the generalization of case marker\is{case!case marking}s for covering semantic roles. The experiments focused on the development of these semantic roles in isolation of each other in order to identify the diagnostic\is{learning strategies!diagnostics}s, repair\is{learning strategies!repair strategies}s and alignment strateg\is{alignment strategy}ies that make the emergence\is{formation} of such roles possible. However, the behaviour and functionality of case marker\is{case!case marking}s can only be fully understood when they are studied in relation to the other elements in their linguistic context. In other words: case marker\is{case!case marking}s have to be investigated in relation to the patterns in which they occur. This chapter therefore presents experiments in which case marker\is{case!case marking}s can be combined in larger patterns.
 
The next section first gives a brief overview of pattern formation\is{formation}\is{pattern formation} in language and operationalizes one strategy of pattern formation\is{pattern formation} in the form of diagnostic\is{learning strategies!diagnostics}s, repair\is{learning strategies!repair strategies}s and alignment strateg\is{alignment strategy}ies. \sectref{s:pattern-exp-1} implements this operationalization and shows that the ``systematicity\is{systematicity}'' of the artificial languages gets lost once smaller linguistic units are starting to combine into larger patterns. In this section I will also briefly discuss other experiments in the field in which the problem of systematicity\is{systematicity} occurs but is either overlooked or misinterpreted by the experimenter. The next section then presents the results of another experiment that uses the more complex alignment strateg\is{alignment strategy}y of multi-level selection\is{multi-level selection} to overcome this problem. Three variation\is{variation}s of multi-level selection\is{multi-level selection} are implemented and compared to each other in terms of systematicity\is{systematicity} and coherence. The insights of these experiments are ported to experiments involving analog\is{analogy}y and the formation\is{formation} of semantic roles in \sectref{s:pattern-exp-3}. \sectref{s:pattern-exp-4} finally offers a first step towards simulations involving the formation\is{formation} of syntactic cases (corresponding to stage 3 in \sectref{s:stage3}). Even though stage 3 is not fully accomplished yet, this section offers a clear idea of the work that needs to be undertaken in order to form syntactic cases.

\section{Pattern formation}
\label{s:pattern-formation}

\subsection{Overview}
One crucial aspect of grammaticalization (see Martin Haspelmath's definition in \sectref{s:assumptions}) is the evolution towards tighter structures and a lesser degree of freedom. For example, lexical items develop into more grammatical items and become part of (larger) constructions. Within these constructions or patterns, the freedom of the individual parts is restricted and depends on the pattern as a whole. This would explain why for example an allative\is{case!allative} case marker\is{case!case marking} only makes sense in a motion-pattern. However, linguistic items that become part of a larger construction may still have a life on their own in their original sense, a phenomenon traditionally known as ``layering\is{synchron\is{synchronic}ic layering}'' \citep[124--126]{hopper93grammaticalization}. For example the preposition\is{preposition} {\em like} can still be used for indicating similarity while at the same time it can be used as a marker\is{case!case marking} for introducing reported speech:

\ea
She looks nothing {\em like} her father.
\item And he was {\em like} ``Oh that is so not true!''
\z

In the following subsection I will briefly touch upon some phenomena of grammaticalization involving pattern formation\is{pattern formation} and offer an analysis which is somewhat different from the traditional linguistic approach. I will support my analysis through other examples of patterns and idiom\is{idiom}s in language. In the next subsection, I will then offer an operationalization of my analysis in terms of diagnostic\is{learning strategies!diagnostics}s and repair\is{learning strategies!repair strategies} strategies for the artificial agents that will be used in the experiments in this chapter.



\subsection{Pattern formation in language}
\is{pattern formation}

\subsubsection{Negation in French} 
A very good example of the development of a lexical item into a part of a grammatical structure can be found in French\is{French} negation\is{negation}. Traditionally, the development of negation\is{negation} particles (also known as ``Jespersen's cycle\is{Jespersen's cycle}'') is defined in terms of a cycle of reanalysis\is{reanalysis} -- analog\is{analogy}y (generalization) -- reanalysis\is{reanalysis} \cite[65--66]{hopper93grammaticalization}:

\begin{enumerate}  \item[1.]   Negation in French\is{French} originally only involved {\em ne} before the verb\is{verb}:
 



\ea
\gll Il ne va.\\
he {\textsc{neg}} go.3{\sg}.{\prs}\\
\glt `He doesn't go.'\\
\z

  \item[2.]   In the context of motion verb\is{verb}s, {\em ne} could optionally be reinforc\is{reinforcement}ed by the noun\is{noun} {\em pas} `step':
 


\ea
\gll Il ne va (pas).\\
he {\textsc{neg}} go.3{\sg}.{\prs} (step)\\
\glt `He doesn't go (a step).'\\
\z

 \item[3.]  The word {\em pas} is reanalyzed as a negator particle in the construction [{\em ne} Vmotion {\em pas}];
 

  \item[4.]   The particle {\em pas} is exten\is{extension}ded analog\is{analogy}ically to non-motion verb\is{verb}s as well:
 

 

\ea
\gll Il ne sait pas.\\
he {\textsc{neg}} know.3{\sg}.{\prs} {\textsc{neg}}\\
\glt `He doesn't know.'\\
\z


   \item[5.]  The particle {\em pas} is then reanalyzed as an obligatory part of the construction [{\em ne} V {\em pas}];
 

 \item[6.]  In spoken French\is{French}, {\em ne} is reanalyzed to become optional and is eventually lost:

\ea
\gll Il sait pas.\\
he know.3{\sg}.{\prs} {\textsc{neg}}\\
\glt `He doesn't know.'\\
\z 

\end{enumerate}
 

\subsubsection{Reanalysis versus pattern formation}
\is{formation}
\is{pattern formation}
 Reanalysis is essentially a hearer-based analysis of this developmental cycle in which the hearer interprets the underlying structure of an utterance in another way than was intended by the speaker. Reanalysis is traditionally understood as  ``change in the structure of an expression or class of expressions that does not involve any immediate or intrinsic modification of its surface manifestation''  \citep[58]{langacker77syntactic}. 
Even though reanalysis\is{reanalysis} is a plausible mechanism for step 3, its main problem is that it is invisible from the outside. \citet{hopper93grammaticalization} write that for  ``the French\is{French} negator {\em pas}, we would not know that reanalysis\is{reanalysis} had taken place at stage [3] without the evidence of the working generalization at stage [4]''  (p. 66). As \citet{haspelmath98does} points out, however, this means that reanalysis\is{reanalysis} cannot explain how the new use of {\em pas} got propagat\is{propagation}ed and accepted in the speech community\is{speech population} unless all speakers are assumed to make the same reanalysis\is{reanalysis} at roughly the same time, which is very implausible. As I will explain more thoroughly in \sectref{s:actualization}, reanalysis\is{reanalysis} needs to be accompanied by other mechanisms in order to account for the empirical data.

I propose a different and simpler mechanism for step 3 that is in line with the general approach of usage-based model\is{usage-based model}s of language: pattern formation\is{formation}\is{pattern formation}. If a certain group of words occur frequently enough together, they are stored as a new unit in the linguistic inventor\is{linguistic inventory}y. This means that the language user now knows two {\bfseries competing} constructions in the case of motion verb\is{verb}s: [{\em ne} V] and [{\em ne} Vmotion {\em pas}]. This approach of pattern formation\is{pattern formation} may seem redundant from the point of view of inventory size, but it may optimize linguistic processing because a pattern is a ``pre-compiled'' chunk that is readily available for use, whereas otherwise the language user needs to compose the structure over and over again. Since pattern formation\is{pattern formation} is a relatively ``simple'' operation for optimizing processing, we can assume within a usage-based model\is{usage-based model} that most language users will do this spontaneously for all recurrent patterns in the language as opposed to a collective operation of reanalysis\is{reanalysis}. Once a pattern is stored in memory, it can start a life on its own and diverge from its original usage. Steps 1--5 in the negation\is{negation} cycle can thus be reinterpreted as follows in a more speaker-based analysis:


\begin{enumerate}  
\item[1.]  Negation in French\is{French} originally only involved {\em ne} before the verb\is{verb}: 
 
\ea
\gll Il ne va.\\
he {\textsc{neg}} go.3{\sg}.{\prs}\\
\glt `He doesn't go.'\\
\z

 \item[2.]  The speakers of French\is{French} start to reinforc\is{reinforcement}e the negation\is{negation} particle {\em ne} in some situations to put more emphasis on the negation\is{negation} or to solve communicative problems. In the context of motion verb\is{verb}s, the reinforc\is{reinforcement}ement is achieved through the noun\is{noun} {\em pas} `step', whereas in other contexts such as verb\is{verb}s of visual perception, negation\is{negation} is reinforc\is{reinforcement}ed through {\em point} `point':
 
\ea
\gll Il ne va (pas).\\
he {\textsc{neg}} go.3{\sg}.{\prs} (step)\\
\glt `He doesn't go (a step).'\\

\item
\gll Il ne voit (point).\\
he {\textsc{neg}} see.3{\sg}.{\prs} (point)\\
\glt `He doesn't see (a point).'\\
\z

 \item[3.]   The frequent use of these reinforc\is{reinforcement}ement noun\is{noun}s leads to the creation of readily available patterns which co-exist\is{co-existence} (and compete with) the standard negation\is{negation} construction;

 \item[4.]  The new patterns are exten\is{extension}ded analog\is{analogy}ically to non-motion verb\is{verb}s as well and start to compete with each other and with the old negation\is{negation} construction for becoming the new default negation\is{negation};


 \item[5.]  The construction [{\em ne} V {\em pas}] wins the competit\is{competition}ion and becomes the new default construction for negation\is{negation}. Other competit\is{competition}ors using different particles either disappear or take up their own semantic niche ({\em ne ... point} `nothing' (old-fashioned), {\em ne ... plus} `no more', {\em ne ... rien} `nothing', {\em ne ... jamais} `never', {\em ne ... gu\`{e}re} `almost nothing', etc.). The old negation\is{negation} construction gets lost except for some archaic uses in writing.
\end{enumerate}

 

\subsubsection{Idioms}
 Evidence for pattern formation\is{pattern formation} as opposed to reanalysis\is{reanalysis} can be found in idiom\is{idiom}s. Idiomatic expressions have always been problematic for traditional linguistic theories that take a modular approach to language and assume a sharp distinction between convention\is{convention}al-lexical items and systematic-syntactic rules. Faced with such problematic issues, usage-based model\is{usage-based model}s and particularly construction grammars  ``grew out of a concern to find a place for idiom\is{idiom}atic expressions in the speaker's knowledge of a grammar of their language''  \citep[225]{croft04cognitive}. Idioms range from highly idiom\is{idiom}atic expressions to more schematic constructions \citep[][chapter 9]{croft04cognitive}:

\ea
by and large; no can do; be that as it may; make believe; so far so good
\item kick the bucket; pull a fast one; spill the beans
\item to answer the door; wide awake; bright red; to blow one's nose
\item the bigger the better; the louder you shout, the sooner they will serve you
%\item what's a girl like you doing in a place like this
\z

No theory of grammaticalization that I am aware of explains idiom\is{idiom}s such as {\em so far so good} or {\em by and large} in terms of reanalysis\is{reanalysis} of the words that make up the idiom\is{idiom}. Similarly, compound noun\is{noun}s are given their own lexical entry rather than introducing a notion of `synchron\is{synchronic}ic layering\is{synchron\is{synchronic}ic layering}' \citep[124--126]{hopper93grammaticalization} over the original words caused by reanalysis\is{reanalysis}. Also pattern formation\is{formation}\is{pattern formation} on other levels of language (e.g. reoccurring syllables, morphemes, etc.) are never treated as synchron\is{synchronic}ic layers on top of one entry in the linguistic inventor\is{linguistic inventory}y. Reanalysis is therefore used in an ad-hoc way, or as \citet{haspelmath98does} writes,  ``as one pleases''  (p. 341).

By taking pattern formation\is{formation}\is{pattern formation} seriously, meaning that many redundant copies exist in memory, a simpler alternative exists for the ad-hoc mechanism of reanalysis\is{reanalysis}. Just as there is no reason for differentiating `core case marker\is{case!case marking}s' from `peripheral semantic case marker\is{case!case marking}s' (see \sectref{s:stage4}), the language user makes no difference between fully idiom\is{idiom}atic expressions such as {\em by and large} and more grammatical constructions such as [{\em ne} ... {\em pas}]. The only difference between them is that the more schematic constructions were exten\is{extension}ded and generaliz\is{generalization}ed to new uses whereas the more idiom\is{idiom}atic expressions remained unchanged depending on communicative needs in language use and frequency\is{frequency} effects. This usage-based approach\is{usage-based model} naturally leads to the continuum\is{syntax-lexicon continuum} of linguistic items as observed in natural languages.

One problem with the alternative hypothesis is that it is invisible from the outside just like reanalysis\is{reanalysis} is. This is where computational models can prove their worth: they can {\bfseries demonstrate} the consequences of each alternative hypothesis and show what kind of cognitive apparatus is needed for both. Additional evidence can then be gathered from other disciplines such as psycholinguistics to determine which cognitive architecture is most plausible. So even though computational modeling cannot predict actual language change\is{language change}, they can demonstrate the effects of proposed mechanisms and help to fill in the blanks when there is a lack of empirical data.

\subsection{Operationalizing pattern formation}
\is{formation}
\is{pattern formation}
\label{s:operationalizing-patterns}

The above idea of pattern formation\is{formation}\is{pattern formation} needs to be implemented in terms of diagnostic\is{learning strategies!diagnostics}s and repair\is{learning strategies!repair strategies} strategies that make use of information that is locally available to the agents. Consider the reaction network\is{reaction network} of \figref{f:reaction1} in which an agent used two constructions which subsequently license\is{license}d node-2 and node-3 in the network and which license\is{license}s the utterance {\em jack -bo push block -ka}:

\begin{figure}[htb]
\centerline{\includegraphics[width=0.8\textwidth]{Chapter4/figs/reaction1}}
  \caption[A reaction network\is{reaction network} as a source for pattern formation\is{formation}\is{pattern formation}]{An agent's reaction network\is{reaction network} is the source for pattern formation\is{pattern formation}. If the agents have to apply two constructions to license\is{license} an utterance (production) or a meaning (parsing), they will create a pattern based on the applied constructions. This pattern has the same functionality as the constructions but only requires one step.}
   \label{f:reaction1}
\end{figure}

Suppose that the agent is in production mode. In this case node-1 is the coupled feature structure\is{feature structure!coupled feature structure} which was license\is{license}d after unify\is{unify and merge}ing and merging the lexical entries for {\em jack, push} and {\em block}. Next, the speaker has to unify\is{unify and merge} and merge two constructions for marking the two participants of the push-event which license\is{license}s node-3. In a next step, which is not shown in the figure, the agent will unify\is{unify and merge} and merge the morphological rules. As indicated in the figure, this reaction network\is{reaction network} forms the basis for a new pattern (which will be construction-3). In principle this pattern should combine the entire reaction network\is{reaction network} including the lexical entries, but for convenience's sake the agents will only make a pattern which combines the functionality of constructions 1 and 2, as shown in \figref{f:new-pattern}.
\begin{figure}[htb]
\centerline{\includegraphics[width=0.8\textwidth]{Chapter4/figs/new-pattern}}
  \caption[A new pattern]{The two constructions that were used during processing are combined into a new construction. The agents keep a link between the new construction and the constructions that were used for creating it.}
   \label{f:new-pattern}
\end{figure}

\newpage
The new construction is stored in the linguistic inventor\is{linguistic inventory}y with information about its origins: the agents keep a link between the new pattern and the constructions that were used for creating it. If the speaker has to produce the same meaning again, the new construction now forms an alternative path in the reaction network\is{reaction network}. The speaker will prefer this new path because it is faster in processing (one step can be skipped) and the links between the constructions can be used for giving priority to larger constructions if they unify\is{unify and merge} and merge. This new reaction network\is{reaction network} is illustrated in \figref{f:reaction2}.

\begin{figure}
\centerline{\includegraphics[width=0.8\textwidth]{Chapter4/figs/reaction2}}
  \caption[A reaction network\is{reaction network} with the new pattern]{The new construction now offers the agent an alternative path in the reaction network\is{reaction network}. Since the new pattern yields the same coupled feature structure\is{feature structure!coupled feature structure} as node-3 in only one step, it is faster and therefore preferred. The links between the three constructions are used to give larger patterns priority if they unify\is{unify and merge} and merge.}
   \label{f:reaction2}
\end{figure}

Apart from creating the new pattern, not much needs to be changed in the linguistic inventor\is{linguistic inventory}y apart from the fact that the agents have to link the new construction to the lexical entries that are compatible with it. The agents will not do this in one sweep but postpone this task until processing: lexical entries are only linked to the new construction instance by instance if this is required during a language game\is{language game}. The mechanism works entirely the same: the agent wants to unify\is{unify and merge} and merge two constructions and wants to optimize processing by creating a pattern. This time, however, no new pattern needs to be created because there is already one. The pattern thus exten\is{extension}ds its use to a new verb\is{verb} as well. The newly-made construction looks as follows:
\largerpage[-1]

\begin{lstlisting}
<Construction: construction-3
((?top-unit
   (sem-subunits (== ?unit-a ?unit-b ?unit-c)))
 (?unit-a
   (sem-frame (== (sem-role-1 ?unit-b ?obj-x)
                  (sem-role-2 ?unit-c ?obj-y))))
 (?unit-b
   (referent ?obj-x))
 (?unit-c
   (referent ?obj-y))
 ((J ?unit-b NIL)
   (sem-role sem-role-1))
 ((J ?unit-c NIL)
   (sem-role sem-role-2)))
<==>
((?top-unit
   (syn-subunits (== ?unit-a ?unit-b ?unit-c)))
 (?unit-a
   (syn-frame (== (syn-role-1 ?unit-b)
                  (syn-role-2 ?unit-c))))
 (?unit-b
   (syn-role syn-role-1))
 (?unit-c
   (syn-role syn-role-2)))>
\end{lstlisting}


\noindent To summarize, the agents are equipped with the following diagnostic\is{learning strategies!diagnostics} and repair\is{learning strategies!repair strategies} strategy in all the experiments in this chapter:

\begin{enumerate}
\item {\bfseries Diagnostic:} If two constructions are used together for licensing a node in the network, report an opportunity for optimizing processing (both for production and parsing);
\item {\bfseries Repair strategy:} If there is a problem of processing effort: 
\begin{enumerate}
\item If a larger construction already exists for the same mapping, create a link between the lexical entry and the construction;
\item Else combine the two constructions into a new construction and keep a link between them.
\end{enumerate}
\end{enumerate}

During processing, the link between constructions is used for giving priority to larger constructions. They can also be used for consolidation\is{consolidation} as I will show in sections \ref{s:pattern-exp-2} and \ref{s:pattern-exp-3}. There are, however, no inheritance\is{inheritance} links: all relevant information is stored in the constructions themselves and no additional aspects are inherited from other constructions.

\section{Experiment 1: individual selection without analogy}
\is{analogy}
\label{s:pattern-exp-1}

\subsection{Overview}
Before immediately picking up the experiments where the previous chapter left off, the influence of the diagnostic\is{learning strategies!diagnostics} and repair\is{learning strategies!repair strategies} strategy for pattern formation\is{formation}\is{pattern formation} is first tested for stage 2 in the development of case marker\is{case!case marking}s: the invention and adoption of specific marker\is{case!case marking}s.

\subsection{Experimental set-up}

The experimental set-up for experiment 1 is entirely the same as the one in baseline experiment 2c but this time the new diagnostic\is{learning strategies!diagnostics} and repair\is{learning strategies!repair strategies} strategy for pattern formation\is{formation}\is{pattern formation} are added to the agents. The set-up can be briefly summarized as follows:

\begin{itemize}
\item The population\is{speech population} consists of 10 agents that engage in description game\is{language game!description game}s;
\item The meaning space is the same one as detailed in \tabref{t:events} and all event type\is{event type}s occur with the same frequency\is{frequency};
\item The agents have two diagnostic\is{learning strategies!diagnostics}s: detecting unexpressed variable equal\is{variable equality}ities and the new diagnostic\is{learning strategies!diagnostics} detecting whether two constructions were applied during processing;
\item The agents have two repair\is{learning strategies!repair strategies} strategies: one for inventing and learning verb\is{verb}-specific marker\is{case!case marking}s and one for combining them into a larger construction;
\item The agents use an alignment strateg\is{alignment strategy}y of direct competit\is{competition}ion which I will further call `individual selection'. This means that the hearer increases the confidence\is{confidence} scores of successfully applied constructions by 0.1 and decreases the scores of their direct competit\is{competition}ors by 0.1. The speaker does not perform score updating.
\end{itemize}

From the above follows that the agents will have to create and converge on one construction for each possible combination of meanings. There are thirty individual participant roles that need a single-participant construction, eighteen combinations of two participant roles and three combinations of three participant roles. Since the agents have no analog\is{analogy}y, the target number of constructions should be 51 (the sum of all these possibilities). All the combinations can be verified in the Appendix.

\subsection{Results and discussion}

\begin{figure}[tb]
\centerline{\includegraphics[width=0.85\textwidth]{Chapter4/figs/size2a}}
  \caption[Experiment 1: number of constructions with individual selection]{This graph shows the average number of constructions in a population\is{speech population} of ten agents in experiment 1. In this set-up the agents succeed in converging on an optimal inventory size -- given their cognitive abilities -- of 30 single-argument constructions, 18 two-argument constructions and 3 three-argument constructions. The graph here indicates that there is still an average of 19 two-argument constructions but this competit\is{competition}ion also gets resolved if more language game\is{language game}s are played.}
   \label{f:size1}
\end{figure}
\begin{figure}[htb]
\centerline{\includegraphics[width=0.85\textwidth]{Chapter4/figs/effort2a}}
  \caption[Experiment 1: success, effort and coherence]{This graph shows average communicative success\is{communicative success}, cognitive effort\is{cognitive effort} and meaning-form coherence in a population\is{speech population} of ten agents in experiment 1. The results show that the agents succeed in reaching 100\% communicative success\is{communicative success} and reducing the cognitive effort\is{cognitive effort} needed for communication. Meaning-form coherence reaches almost 100\% with only competit\is{competition}ion between one or two forms that is still undecided.}
   \label{f:effort1}
\end{figure}

\subsubsection{Results}
 The experimental set-up was tested in ten series of 16.000 language game\is{language game}s. By looking at the same measures as in the baseline experiments, the simulations seem to yield successful results at first sight. \figref{f:size1} plots the average number of constructions in the population\is{speech population}. Here, the agents have almost reached the optimal state in terms of linguistic inventor\is{linguistic inventory}y. Only in the case of two-argument constructions there are additional language game\is{language game}s needed for deciding on the competit\is{competition}ion between one or two surviving constructions. Acquiring the constructions happens quite fast (in less than 3.000 games), but alignment takes much more time than was needed in the baseline experiments. This is due to the individual selection alignment strateg\is{alignment strategy}y: if a pattern was used, only competing patterns are punished through lateral inhibition\is{lateral inhibition}. The individual marker\is{case!case marking}s or rather the single-argument constructions they occur in are not considered during consolidation\is{consolidation}.
\begin{figure}[p]
\centerline{\includegraphics[width=0.7\textwidth]{Chapter4/figs/direct-coherence-no-analogy-1000}}
  \caption[Experiment 1: snapshot after 1.000 games]{This diagram gives a snapshot of the average coherence in a population\is{speech population} of 10 agents after 1.000 language game\is{language game}s using the direct selection alignment strateg\is{alignment strategy}y. Each circle stands for a particular meaning (see the Appendix), for example circles 4 and 5 stand for `appear-1' and `appear-2'. The lines between circles means that the meanings combine into compositional meanings, for example circle 31 means the combination `appear-1 appear-2'. The darker the circle is colour\is{colour}ed, the more agents prefer the same case marker\is{case!case marking}(s) for covering this meaning. A full line between circles means that both meanings are covered using the same marker\is{case!case marking}s (= systematic), a dotted line means that a different form is preferred for the same meaning (= unrelated). The diagram shows that for most meanings only half of the population\is{speech population} prefer the same form and that in many cases there is no systematic choice for a certain case marker\is{case!case marking}.}
   \label{f:1-coherence-1000}
\end{figure}

The long alignment period is also illustrated in \figref{f:effort1}, which displays average communicative success\is{communicative success}, cognitive effort\is{cognitive effort} and meaning-form coherence. The fact that communicative success\is{communicative success} rapidly rises to 100\% within 4.000 language game\is{language game}s and that cognitive effort\is{cognitive effort} drops to zero between 6.000 and 8.000 language game\is{language game}s suggests that the agents have learned all the variation\is{variation}s floating around in their population\is{speech population}. However, meaning-form coherence takes much longer to rise to its maximum which is again due to the alignment strateg\is{alignment strategy}y. Coherence reaches almost 100\% after 16.000 games with only competit\is{competition}ion going on for one or two cases of two-argument constructions. This competit\is{competition}ion will in the end also be resolved after additional language game\is{language game}s.

The longer alignment period is however not the most fundamental problem with the artificial languages that are formed by the agents. A closer examination of them shows that all meaning-form mappings that they agree on are totally arbitrary. The problem is illustrated in Figures \ref{f:1-coherence-1000} and \ref{f:1-coherence-7000} which give a snapshot of convergence and coherence in one simulation after 1.000 and 7.000 language game\is{language game}s respectively. Each meaning or combination of meanings (see the Appendix) is represented as a circle. For example, the meaning `approach-1' is represented as circle 4 and meaning `approach-2' is represented as circle 5. Lines between circles indicate that the meaning of one circle is a combination of the meanings of the other circles. For example, circle 31 combines `approach-1' and `approach-2'. The colour\is{colour} of the circles represents the number of agents that prefer the most frequent form in the population\is{speech population} for that particular word. A white circle means that there is either no form yet for this meaning or that there is no form which is preferred by more than one agent. A black circle means that all ten agents prefer the same form for this meaning. If all the circles are black, the agents have reached 100\% convergence. If the lines between the circles are full lines, the same participant role is expressed by the same marker\is{case!case marking} across constructions. If however the line is dotted, there is a different form for the same meaning.

This can best be illustrated through an example. The circle for meaning 4 (approach-1) indicates that there are 4  or 5 agents in the population\is{speech population} which prefer the same form for marking this participant role at this stage of the simulation. For circles 5 (approach-2) and 31 (approach-1 approach-2), there are two or three agents that prefer the same form. The dotted lines between the circles, however, indicate that the most frequent pattern for circle 31 uses different marker\is{case!case marking}s than the single-argument constructions for circles 4 and 5:

\ea
\gll jack -lich approach\\
jack approach-1 approach\\
\glt `Jack approaches (someone)'.\\

\item
\gll jill -sut approach\\
jill approach-2 approach\\
\glt `(Someone) approaches Jill'.\\

\item
\gll jill -xa jack -zuih approach\\
jill approach-2 jack approach-1 approach\\
\glt `Jack approaches Jill'.\\
\z

\newpage
\figref{f:1-coherence-7000} shows that after 7.000 language game\is{language game}s, the agents have almost converged on a form for every meaning, but the problem of systematicity\is{systematicity} remains: in half of the cases, a different case marker\is{case!case marking} is winning the competit\is{competition}ion on the level of single-argument constructions than the one(s) winning on the other levels. The figure also shows that in most of the cases where there is no systematic use of a form for the same meaning, convergence is also still not complete. This is in contrast to the meanings which (accidentally) arrived at the same form across constructions. Here we see mostly black circles meaning that all agents prefer the same convention\is{convention}.

\begin{figure}[p]
\centerline{\includegraphics[width=0.7\textwidth]{Chapter4/figs/direct-coherence-no-analogy-7000}}
  \caption[Experiment 1: snapshot after 7.000 games]{This diagram gives a snapshot of the average coherence in a population\is{speech population} of 10 agents after 7.000 language game\is{language game}s using the direct selection alignment strateg\is{alignment strategy}y. The agents have reached convergence for most meanings by now, but these form-meaning mappings are not always systematically related to each other. For example, the meanings related to 49 were pretty consistent in their meaning-form mappings after 1.000 games, but have now become totally unrelated to each other: for each possible combination a new form is introduced to cover the same meaning. In all the cases where there is no systematicity\is{systematicity}, the convergence is not complete yet.}
   \label{f:1-coherence-7000}
\end{figure}


\subsubsection{Discussion}
 The results clearly indicate that the agents are not capable of constructing a systematic language. The reason for this is that all constructions are basically treated as independent linguistic items. This means that once a larger pattern is created, it starts living its own life without influencing or being influenced by the constructions that were used to create it. This results in some case marker\is{case!case marking}s losing the competit\is{competition}ion for marking a certain participant role on the level of single-argument constructions but still becoming the most successful one as part of a larger pattern. In all the simulations, this happened in 40 to 60\% of the cases (see \figref{f:systematicity2}).

The fact that in more than half of the cases the same marker\is{case!case marking} wins the competit\is{competition}ion on all levels is due to the small meaning space of the experiment and the fact that patterns are always created by combining the most successful constructions at a given point in the simulation. In fact, the agents can continue to create new patterns for a certain combination of participant roles even if they already know other patterns for it. For example, it may happen that on a lower level the average confidence\is{confidence} scores of a new combination becomes more successful than the confidence\is{confidence} score of the patterns. In this case the agents will still innovat\is{innovation}e which gives a slight advantage to those patterns that are in line with the most successful constructions of a lower level. As the results show, however, this is not enough.

Since natural languages are also not fully regular, it is important to see whether the lack of systematicity\is{systematicity} in the experiments is relevant for the many exceptions and sub-regularities found in natural languages. The answer is no: for most if not all irregular forms and sub-regularities in natural language, either a systematic origin can be found through diachron\is{diachronic}ic changes or through external pressures such as language contact\is{language contact}. For example, the -ed-participle in English\is{English} did not manage to exten\is{extension}d its use to all past tense\is{tense}s as can be observed in irregular verb\is{verb}s such as {\em to sing} and {\em to give}. These strong verb\is{verb}s are however remnants of completely regular classes of verb\is{verb}s in Proto-Indo-European that were able to survive thanks to their high token frequency\is{token frequency}\is{frequency}. Despite all sociological factors, historical incidents, language contact\is{language contact}, and other kinds of exceptions, natural languages succeed remarkably well in developing systematicity\is{systematicity} spanning over many constructions, as for example word order in English\is{English}. Given the abstraction\is{abstraction}s and scaffold\is{scaffold}s of the present experiments, the agents should thus be capable of developing a fully systematic language without any problems.

This leaves us the question of how systematicity\is{systematicity} can be achieved. As said before, all systematic form-meaning mappings have been formed by accident due to the small world and the nature of the innovat\is{innovation}ion mechanism. For true systematicity\is{systematicity}, however, the agents need to be able to recognize relations between constructions rather than treating them as a list of independent units. This would mean that if a particular construction is successful, its systematically related constructions should also (perhaps indirectly) benefit from its success. In \sectref{s:pattern-exp-2} I will introduce a biologically inspired mechanism that can be exploited to achieve this effect: multi-level selection\is{multi-level selection}.

\subsection{The problem of systematicity in other work}
\is{systematicity}
\label{s:problem-systematicity}

As to my knowledge, the problem of systematicity\is{systematicity} has never been reported before in the field of the origins\is{origins} and evolution\is{evolution!cultural evolution} of language. This does not mean, however, that the problem never existed. In this section, I will give a brief overview of some prior work in the field in which the problem was either overlooked or in which it could not occur due to experimental assumption\is{assumption}s.

\largerpage[-1]
\subsubsection{Exemplar-based simulations}
\is{exemplar-based models}
 One computational simulation which is closely related to the work in this book is presented by \citet{batali02negotiation}. Batali investigates how a multi-agent population\is{speech population} can form a recursi\is{recursion}ve communication system by using exemplars stored in memory. This work can be categorized as a `problem-solving\is{problem-solving} model' because these exemplars have to be agreed upon in locally situated interactions. Each exemplar has a confidence\is{confidence} score which is increased and decreased according to similar lateral inhibition\is{lateral inhibition} dynamics as in the simulations of the previous section. The type of learner is thus the same one as the agents in this book: they build their language instance by instance in a bottom-up and redundant fashion. Batali's agents only keep exemplars and all generalization in the model is captured by directly manipulating these exemplars during processing. \figref{f:batali} gives an example of an exemplar composed of two smaller ones \citep[exemplar 5.1.2.a]{batali02negotiation}.

\begin{figure} 
\centerline{\includegraphics[width=0.75\textwidth]{Chapter4/figs/batali}}
  \caption[An exemplar \citep{batali02negotiation}]{A complex exemplar from \citet{batali02negotiation}. The exemplar features a compositional meaning with `argument maps' to the smaller exemplars that take care of variable equal\is{variable equality}ities in the meaning.}
   \label{f:batali}
\end{figure}

Batali does not use event-specific variables as I do in this book but assumes a simple three-way contrast between arguments 1, 2 and 3. For example the meaning ((snake 1) (sang 1)) translates to something like `the snake sang', whereas ((snake 1) (sang 2)) would mean something like `there was a snake and something sang.' Event structure\is{event structure} is stored immediately in the exemplar but can be overridden by argument maps between complex exemplars and their subcomponents. For example, the argument map `1:2' translates a meaning like (rat 1) to (rat 2). These argument maps are also stored as part of the complex exemplar. Apart from these argument maps between complex exemplars and their components, all exemplars are unrelated and listed in the memory.

The agents then engage in a series of description game\is{language game!description game}s. They are able to invent new words for new meanings and they are capable of combining existing words into larger patterns or breaking up a pattern again into smaller parts. The ultimate goal of the agents is two-fold: (a) agree on a shared lexicon\is{lexicon} for all the single meanings (e.g. cat, fox, chase, etc.) and (b) agree on a way to mark event structure\is{event structure} through the argument maps (i.e. marking the difference between arguments 1, 2 and 3). The simulations make use of a single generation of agents.

The results indicate that the agents gradually reach communicative success\is{communicative success} and that they agree on the same exemplars. Goal (a) is therefore definitely reached. However, the results show that event structure\is{event structure} is not always marked in the same way: all the simulations end up using specific ordering for each exemplar (even though they may involve the same meanings) and using `empty' words that accidentally evolved into marker\is{case!case marking}s for argument mappings. The agents thus do not succeed in agreeing on a systematic way of distinguishing participant `1' from participants `2' and `3'. The agents thus cannot generaliz\is{generalization}e argument mapping to new predicates such as (give 1 2 3) and have to negotiat\is{negotiation}e event structure\is{event structure} for each word separately. This lack of systematicity\is{systematicity} is not noted by Batali as a problem and the use of the empty words is wrongly interpreted as corresponding to argument marker\is{case!case marking}s in natural languages.


\subsubsection{Probabilistic grammars}
 Another experiment in which the systematicity\is{systematicity} problem is overlooked is reported by \citet[chapter 10]{depauw02grael}. De Pauw investigates how rudimentary principles of syntax can emerge from distributional aspects of communication rather than from the interface\is{syntax-semantics interface} between syntax and semantics. This exclusive focus on syntax is different from the work in this book (even though some semantics is smuggled into De Pauw's simulations in the disctinction between animate\is{animacy} and non-animate\is{animacy} objects which results in different distributional patterns). Similar assumption\is{assumption}s to this book are the heavy use of memory (even more so by De Pauw), a bottom-up and redundant formation\is{formation} of the language and a predefined lexicon\is{lexicon} in order to focus exclusively on the topic of interest. The population\is{speech population} in De Pauw's simulations is dynamic in the sense that there is a generational turn-over, but no linguistic information is transmitted genetically from one generation to the next.

The agents engage in a series of language game\is{language game}s in which they communicate about objects or events. If there are several objects, the agents can choose between six word orders: SVO, SOV, VSO, VOS, OSV or OVS. De Pauw therefore does not distinguish between verb\is{verb}-specific participant roles, but only assumes a two-way contrast between the subject\is{syntactic role!subject} (S) and the object\is{syntactic role!object} (O). The agents start without any preference for a particular word order so variation\is{variation} naturally occurs in the population\is{speech population}. The alignment strateg\is{alignment strategy}y of the agents is simply storing bigrams or frequencies of co-occur\is{co-occurrence}rences and performing statistical inducti\is{induction}on on top of those bigrams \citep[362]{depauw02grael}:

\begin{figure}[h]
\centerline{\includegraphics[width=\textwidth]{Chapter4/figs/depauw}}
  \caption[Bigram probabilities \citep{depauw02grael}]{The agents in \citet{depauw02grael} store co-occur\is{co-occurrence}rence frequencies and use these bigram-probabilities to decide on a preferred ordering.}
   \label{f:depauw}
\end{figure}

During the simulations, the agents rapidly converge on fixed word order on simple relations. However, as De Pauw notes, there are no general tendencies in terms of a general word order. Each `verb' rather has its own preferred ordering. For more complex relations, the agents do not always reach coherence. De Pauw concludes that the agents therefore reside in a local maximum and that they evolve from one local maximum of convergence to the next. De Pauw argues that this is not a shortcoming of the model but rather its greatest asset: whereas other models in the field are looking for the state of convergence,  ``language itself never converges and constantly adapts to a changing environment and seems to be driven by chaotic elements, introducing a large degree of randomness in language both from a synchron\is{synchronic}ic, as well as a diachron\is{diachronic}ic point of view''  (p. 378).

There are however no such chaotic elements present in De Pauw's model which should prevent the agents from reaching complete coherence. The degree of randomness in his simulations seems to stem from the systematicity\is{systematicity} problem: by only looking at bigram probabilities, it is to be expected that there is an arbitrary word order which is verb\is{verb}-specific. In the case of more complex predicates, the preferred order depends on a combination of various bigrams which increases the randomness because the probabilities of these bigrams are constantly changing so it becomes much harder to agree on a fixed order for these complex meanings. De Pauw dismisses the possibility that the degree of convergence is the maximum that can be expected from the population\is{speech population}, but this is in fact the only correct conclusion. Given the cognitive capabilities of the agents, convergence could only increase if the input would be more structured. In certain machine learning\is{machine learning} tasks, there is already a lot of structure present in the learning data so bigrams can be successfully used for making some predictions. In the case of language formation\is{formation}, however, agents have to start from scratch so there is no structure spanning multiple levels yet that can be induced.

De Pauw's concluding remark is that it is empirically impossible to know whether the agents succeed in  ``expressing the proper (agent, patient) relationship, or if it is just a side-effect of beneficial bigram probability distributions''  (p. 376). I would argue, however, that since the agents are not endowed with the capacity of relating bigrams to each other but solely rely on these probabilities, all tendencies in word order are in fact a side-effect of the bigrams. The conclusion is that \citet{depauw02grael}, just like \citet{batali02negotiation}, misinterpreted experimental results because the problem of systematicity\is{systematicity} was not noticed.


\subsubsection{Iterated Learning Models}
\is{Iterated Learning Model (ILM)}
 So far I only discussed models that featured lazy learner\is{lazy learner}s: agents which postpone generalization until processing time and which shape their language in a step-by-step fashion. As opposed to lazy learner\is{lazy learner}s there are `eager learner\is{eager learner}s'. Eager learner\is{eager learner}s try to look for generalizations (and abstraction\is{abstraction}s) before it is actually needed in processing and work on the complete inventory. Eager learner\is{eager learner}s typically discard the examples that can be deriv\is{derivation}ed from a rule and thus try to optimize the inventory size. If the problem of systematicity\is{systematicity} also occurs with eager learner\is{eager learner}s, then we know that the problem is not exclusive to the usage-based approach\is{usage-based model} proposed in this book. 

In the field of artificial language evolution\is{artificial language evolution}, especially Iterated Learning Model\is{Iterated Learning Model (ILM)}s feature agents that loop through their inventory after each interaction in order to make abstraction\is{abstraction}s. In \sectref{s:impact} I will draw a thorough comparison between my experimental results and those of \citet{moy06case}, who investigated the same topic using the Iterated Learning Model\is{Iterated Learning Model (ILM)} so I will not go into details here. As a quick preview, I can already give away one of the conclusions which is that the problem of systematicity\is{systematicity} also occurs in Iterated Learning Model\is{Iterated Learning Model (ILM)}s. This does not only happen in Moy's experiments, but also in the simulations reported by \citet{kirby00syntax} and \citet{smith03iterated} even though these models feature complete meaning transfer and a population\is{speech population} of only two agents.

The conclusion is the same as for the other simulations reported in this section: the problem of systematicity\is{systematicity} goes by unnoticed in most Iterated Learning Model\is{Iterated Learning Model (ILM)}s, but becomes very apparent in \citet{moy06case}. The problem occurs for the same reasons as in all the other experiments: the agents only behave `systematic' during innovat\is{innovation}ion and learning, but then treat all linguistic items as an unstructured list of unrelated elements. So either there is no adequate model yet that avoids the problem of systematicity\is{systematicity} or the problem is {\bfseries not restricted to the type of learner}. In the latter case, the problem seems to be caused by the fact that the linguistic inventor\is{linguistic inventory}y is unstructured.


\subsubsection{Other models}
 Finally, there are many models that investigate certain aspects of grammar in which the problem of systematicity\is{systematicity} does not occur such as \citet{debeule07compositionality, debeule06emergence, nowak99evolution, steels06how-grammar}; etc. I will take the simulations by \citet{debeule06emergence} as an example of why these models don't have the problem. The conclusions of this brief discussion extend to all the other models on grammar as well.

De Beule \& Bergen investigate the competit\is{competition}ion between holistic and compositional utterances. In case of compositional utterances, one could expect the problem of systematicity\is{systematicity} to pop up, but it doesn't. The reason is that De Beule \& Bergen designed their experiment in such a way that the agents had prior knowledge about what kind of categories and constructions to expect: individual words are immediately tagged with a certain syntactic category and grammatical constructions are fully schematic from the start. One construction can hence be used for all possible combinations of competing individual words that are tagged with the same category and remains agnostic as to which words should win the competit\is{competition}ion. The experiment thus made a clear separation between the lexicon\is{lexicon} on the one hand and grammatical constructions on the other; and it did not offer the agents the possibility of in-between patterns or idiom\is{idiom}s.

This is not a criticism of the model per se: given the fact that De Beule \& Bergen only intended to focus on competit\is{competition}ion between holistic and compositional utterances, the design choice is justified in which the competit\is{competition}ion dynamics can be clearly investigated on each level. As such the experiment can be interpreted as investigating a prerequisite of grammar rather than the emergence\is{formation} of {\em actual} grammar. For the scope of this book, this experimental design is thus not warranted: the barrier between fully idiom\is{idiom}atic items and fully schematic items needs to be broken down.

\section{Experiment 2: multi-level selection without analogy}
\is{multi-level selection}
\is{analogy}
\label{s:pattern-exp-2}

\subsection{Overview}
In the previous section I demonstrated the problem of systematicity\is{systematicity} that occurs during the emergence\is{formation} of a language if the agents treat all entries in their linguistic inventor\is{linguistic inventory}y as unrelated individuals and if their language comprises multiple layers of organization. An alignment strateg\is{alignment strategy}y involving the individual selection of constructions leads to completely arbitrary form-meaning pairs whereas natural languages show greater cohesion and a higher degree of systematically related constructions. Even in idiom\is{idiom}s such as {\em he kicked the bucket}, some degree of schematicity is present such as the conjugation of the verb\is{verb}. The agents therefore need a new alignment strateg\is{alignment strategy}y in which the success of one construction may have an impact on the success of other related constructions.

In this section I will present an experiment which features new alignment strateg\is{alignment strategy}ies that are inspired by the notion of `multi-level selection\is{multi-level selection}' in evolutionary biology \citep{wilson94group}. Multi-level selection\is{multi-level selection} (formerly known as `group selection') acknowledges the fact that groups or other higher-level entities can act as `vehicles' for selection. In this view, not all aspects of groups are reduced to by-products of individual (and usually selfish) interactions. In other words, being part of a group can increase the selectionist advantage of individuals.% One example from biology concerns the origins of chromosomes. Individual genes started to be combined into larger units (chromosomes). The genes are on the one hand replicators in their own right, undergoing competit\is{competition}ion, but they are also part of the larger replicating unit of chromosomes \citep{maynard-smith93origin}.

Natural languages are clear instances of organisms with a hierarchical functional organization which can be conceived as `groups within groups'. Competition is going on at multiple levels of this organization: between synonyms for becoming dominant in expressing a particular meaning, between idiom\is{idiom}atic patterns that group a number of words, between different syntactic and semantic categories competing for a role in the grammar, between ways in which a syntactic category is marked, etc. Multi-level selection\is{multi-level selection}s therefore seems to be readily applicable to language as well.

\subsection{Experimental set-up}

The most important requirement for implementing multi-level selection\is{multi-level selection} is that the agents have to be capable themselves of recognizing relations between linguistic items. This is in fact not so difficult to achieve: in \sectref{s:operationalizing-patterns} I explained that the agents keep a link between larger constructions and the constructions that were used for creating them. These links can now be used for implementing multi-level selection\is{multi-level selection}. Three different alignment strateg\is{alignment strategy}ies have been implemented for comparison:

\begin{itemize}
\item {\bfseries Top-down selection}: if the game was a success, the hearer will not only reward the constructions that were applied during processing, but also all the related constructions on a lower level. The confidence\is{confidence} scores of all the competit\is{competition}ors of these constructions are decreased through lateral inhibition\is{lateral inhibition}.
\item {\bfseries Bottom-up selection}: If the game was a success, the hearer will not only increase the score of the applied constructions, but also the scores of all the related constructions on a higher level. All the competing constructions are punished.
\item {\bfseries Multi-level selection\is{multi-level selection}}: If the game was a success, the hearer will not only increase the score of the applied constructions, but also the scores of all related constructions. All the competing constructions are punished through lateral inhibition\is{lateral inhibition}.
\end{itemize}

Retrieving related constructions is performed recursi\is{recursion}vely. For example, if a three-argument construction was applied using the top-down selection alignment strateg\is{alignment strategy}y, its two sub-components are retrieved (a two-argument and a single-argument construction) as well as the two sub-components of the two-argument construction. The hearer thus increases the scores of five constructions. The competit\is{competition}ors are all the direct competit\is{competition}ors of these five constructions.


During processing, only the scores of the applied constructions are taken into account and not of the whole group of related constructions. The group selection dynamics therefore only matter during consolidation\is{consolidation}. The rest of the set-up is the same as for experiment 1.

\subsection{Results and discussion}



The three alignment strateg\is{alignment strategy}ies were compared to each other and to experiment 1 in ten series of 10.000 language game\is{language game}s.


\subsubsection{Results}
\begin{figure}[t]
\centerline{\includegraphics[width=\textwidth]{Chapter4/figs/systematicity-vs-2}}
  \caption[Experiment 2: systematicity]{This graph compares the performance\is{performance} in terms of systematicity\is{systematicity} of four experimental set-ups. systematicity\is{systematicity} fluctuates around 50\% in the baseline case where there is only individual selection (experiment 1). With the bottom-up selection strategy the agents improve the systematicity\is{systematicity} rate to 80\% but then get stuck. Top-down selection leads to full systematicity\is{systematicity} in some of the runs, but most of the simulations feature some `frozen accidents' as well. Only the multi-level selection\is{multi-level selection} strategy leads to full systematicity\is{systematicity} in all the series after about 7.000 language game\is{language game}s.}
   \label{f:systematicity2}
\end{figure}

 \figref{f:systematicity2} illustrates the amount of systematicity\is{systematicity} in all four alignment strateg\is{alignment strategy}ies. The graph shows that the three alignment strateg\is{alignment strategy}ies involving multiple levels all improve on the baseline of individual selection of experiment 1. With the alignment strateg\is{alignment strategy}y of individual selection, systematicity\is{systematicity} fluctuates between 40 and 60\% depending on how `lucky' the agents were. The behaviour of the other three strategies is much more consistent over the ten series. The graph shows that bottom-up selection allows the agents to improve systematicity\is{systematicity} to 80\% but there they are faced with `frozen accidents' as well. The top-down selection improves systematicity\is{systematicity} even further and allows the agents to reach full systematicity\is{systematicity} in some of the runs. However, in most cases, there were still two or three unsystematic patterns left. Only the multi-level selection\is{multi-level selection} strategy led to full systematicity\is{systematicity} in all the simulations.

\begin{figure}[t]
\centerline{\includegraphics[width=\textwidth]{Chapter4/figs/coherence-vs-2}}
  \caption[Experiment 2: meaning-form coherence]{Since the systematicity\is{systematicity} graph only takes the most frequent forms into account, meaning-form coherence has to be checked in order to verify whether all the agents have converged on the same form-meaning pairs. We see that after 10.000 language game\is{language game}s, only the top-down and the multi-level selection\is{multi-level selection} alignment strateg\is{alignment strategy}ies have already reached complete coherence. Multi-level selection\is{multi-level selection} slightly outperforms top-down selection but not significantly so. In the case of bottom-up and individual selection, the agents need additional language game\is{language game}s for reaching coherence.}
   \label{f:coherence2}
\end{figure}

Since the measure of systematicity\is{systematicity} only looks at the most frequent forms floating in a population\is{speech population}, it needs to be complemented with meaning-form coherence to verify whether {\em all} the agents converge on the same preferences. \figref{f:coherence2} therefore compares the performance\is{performance} of the four alignment strateg\is{alignment strategy}ies in terms of coherence. From the results of experiment 1 we already knew that in the case of individual selection, alignment takes longer than 10.000 language game\is{language game}s. The coherence line for bottom-up selection runs almost parallel with it and does not improve on it in terms of convergence speed. The only two strategies that reach convergence within 10.000 games are top-down and multi-level selection\is{multi-level selection}. Full coherence in the case of top-down selection however does not mean full systematicity\is{systematicity}, as was shown in \figref{f:systematicity2}.

\begin{figure}[t]
\centerline{\includegraphics[width=0.87\textwidth]{Chapter4/figs/size2d}}
  \caption[Experiment 2: number of constructions with multi-level selection\is{multi-level selection}]{This graph shows the average number of constructions known by an agent using the multi-level selection\is{multi-level selection} alignment strateg\is{alignment strategy}y. Compared to individual selection, multi-level selection\is{multi-level selection} allows the agents to discard competit\is{competition}ors much more rapidly: there are significantly less variation\is{variation}s floating around in the population\is{speech population}. For example, the peak of single-argument constructions is about 60 instead of 80 in experiment 1. Also the alignment phase happens much faster.}
   \label{f:size2d}
\end{figure}


 
\figref{f:size2d} shows the average number of constructions in the ten series involving the alignment strateg\is{alignment strategy}y of multi-level selection\is{multi-level selection}. The graph confirms the fact that the agents converge significantly faster on an optimal number of constructions than in experiment 1. At the peak of competing constructions, there are about 60 single-argument constructions, 30 two-argument constructions and 6 three-argument constructions or an average of two competing constructions for each possible meaning. This is much less than in experiment 1 (see \figref{f:size1}) which featured peaks of 80 single-argument, 50 two-argument and 10 three-argument constructions. Also alignment happens much faster: with multi-level selection\is{multi-level selection}, the agents align after 6.000 games as opposed to 14.000 language game\is{language game}s or more if the agents use individual selection.

\largerpage[-2]
Figures \ref{f:2d-coherence-1000} and \ref{f:2d-coherence-7000} offer a snapshot of the most frequent forms in a population\is{speech population} using the multi-level selection\is{multi-level selection} alignment strateg\is{alignment strategy}y. Both snapshots confirm the results indicated by the coherence and systematicity\is{systematicity} graphs. \figref{f:2d-coherence-1000} shows already much more dark grey circles than \figref{f:1-coherence-1000} featuring individual selection, indicating that for most meanings there is already a majority of agents preferring the same form. The preferred forms are also to a higher degree systematically related to each other than in experiment 1, even for the more complex patterns. All black circles in the Figure feature meanings which are related to other meanings, which suggests that multi-level selection\is{multi-level selection} indeed favours groups of related items. The snapshot in \figref{f:2d-coherence-7000} only shows black circles, which means that all agents in the population\is{speech population} prefer the same form for that particular meaning. There are also only full lines between the circles indicating that the same case marker\is{case!case marking}s are consistently used across patterns. This result significantly improves over the earlier results with individual selection.

\begin{figure}[p]
\centerline{\includegraphics[width=0.7\textwidth]{Chapter4/figs/multilevel-coherence-no-analogy-1000}}
  \caption[Experiment 2: snapshot after 1.000 games (multi-level selection\is{multi-level selection})]{This diagram gives a snapshot of the average coherence in a population\is{speech population} of 10 agents after 1.000 language game\is{language game}s using the multi-level selection\is{multi-level selection} alignment strateg\is{alignment strategy}y. Compared to the simulations using direct selection, the agents seem to be converging more rapidly on meaning-form mappings and have already settled on 11 of them. Whereas there was no convergence at all yet for the more complex meanings 49--51 in the simulations using the direct selection alignment strateg\is{alignment strategy}y, here they are already shared by a majority of the population\is{speech population}. This suggests that multi-level selection\is{multi-level selection} speeds up the convergence dynamics significantly especially for related meanings. For the meanings in the bottom right, there is less systematicity\is{systematicity} and hence convergence takes longer time.}
   \label{f:2d-coherence-1000}
\end{figure}

\begin{figure}[p]
\centerline{\includegraphics[width=0.7\textwidth]{Chapter4/figs/multilevel-coherence-no-analogy-7000}}
  \caption[Experiment 2: snapshot after 7.000 games (multi-level selection\is{multi-level selection})]{This diagram gives a snapshot of the average coherence in a population\is{speech population} of 10 agents after 7.000 language game\is{language game}s using the multi-level selection\is{multi-level selection} alignment strateg\is{alignment strategy}y. All the circles are black which means that the entire population\is{speech population} prefers the same meaning-form mappings. The language is also fully systematic so the agents have converged on 30 case marker\is{case!case marking}s which can be combined into 51 different constructions. The results indicate that even for instance-based\is{exemplar-based models} learners/innovat\is{innovation}ors systematicity\is{systematicity} can be reached by keeping a link between newly acquired constructions and the constructions that were used for learning or creating them.}
   \label{f:2d-coherence-7000}
\end{figure}


\subsubsection{Discussion}
 The results show that the agents reach full systematicity\is{systematicity} if each level in the linguistic inventor\is{linguistic inventory}y can have an influence on the competit\is{competition}ion in other levels. It is now important to understand why this is the case and why the other alignment strateg\is{alignment strategy}ies did not yield full systematicity\is{systematicity}.

First of all, bottom-up selection doesn't improve on the results of experiment 1 in terms of convergence speed and it doesn't lead to a systematic mapping of meaning to form across patterns. A closer examination of the the alignment strateg\is{alignment strategy}y reveals that the relatively high systematicity\is{systematicity} of 80\% is due to the higher frequency\is{frequency} of single-argument constructions. The frequent use of these constructions each time has repercussions for the competit\is{competition}ion between larger constructions whereas this is not the case in the other direction. If there would be more patterns than single-argument constructions, the improvement would therefore be less high. The patterns that resist the bottom-up selection can do so because the competit\is{competition}ion is not fully decided yet on a lower level (each time reinforc\is{reinforcement}ing competing patterns) and there is additional competit\is{competition}ion on the level of the patterns themselves which may have different winners than those on lower levels. In the end, though, the bottom-up strategy should lead to full systematicity\is{systematicity} but only very slowly and because the smaller constructions are more frequent.

The top-down strategy is affected by frequency\is{frequency} as well: competit\is{competition}ion between the larger constructions has significant impact on lower levels because it can increase the scores of up to six constructions while at the same time punishing all competit\is{competition}ors. However, since the smaller constructions are actually more frequent than the larger ones, some divergent competit\is{competition}ion pathways may resist this influence from the patterns and survive nevertheless. As the results indicate, this in fact happens in most of the simulations. Top-down selection thus improves systematicity\is{systematicity} significantly, but it is affected by the frequency\is{frequency} of the various levels of linguistic items and it is therefore no guarantee of full systematicity\is{systematicity}.

Finally, the agents can achieve full systematicity\is{systematicity} through multi-level selection\is{multi-level selection}. This strategy allows the competit\is{competition}ion of each level to influence the competit\is{competition}ion on others and given its n-directionality, it is not (or less) dependent on differences in frequency\is{frequency}. Moreover, the agents do not need to differentiate between a `higher' and a `lower' level but can treat all links between constructions on equal footing. The results of experiment 2 confirm earlier results on multi-level selection\is{multi-level selection} and systematicity\is{systematicity} reported by \citet{steels07multilevel}. In these experiments, which involved a scale-up in convergence space, multi-level selection\is{multi-level selection} outperforms the other strategies even more significantly.

\section{Experiment 3: multi-level selection with analogy}
\is{multi-level selection}
\is{analogy}
\label{s:pattern-exp-3}

\subsection{Overview}
Similarly to the previous experiments, experiment 3 investigates how baseline experiment 3 can be extended with a diagnostic\is{learning strategies!diagnostics} and repair\is{learning strategies!repair strategies} for pattern formation\is{formation}\is{pattern formation}. Since the previous experiments identified the problem of systematicity\is{systematicity}, experiment 3 first of all needs to verify whether the conclusions of the second experiment also hold for the new set-up in which the agents are capable of performing analog\is{analogy}ical reasoning over events. I will then report an experiment that adapts the algorithm for multi-level selection\is{multi-level selection} to the token-frequency\is{frequency} alignment strateg\is{alignment strategy}y of baseline experiment 3d (see \sectref{s:base3}).

\subsection{Experimental set-up}

Experiment 3 features the same experimental set-up as baseline experiment 3 with the addition of a diagnostic\is{learning strategies!diagnostics} and repair\is{learning strategies!repair strategies} strategy for pattern formation\is{formation}\is{pattern formation}. To summarize:

\begin{itemize}
\item The population\is{speech population} consists of 10 agents that engage in description game\is{language game!description game}s;
\item The meaning space is the same one as detailed in \tabref{t:events} and all event type\is{event type}s occur with the same frequency\is{frequency};
\item The agents have two diagnostic\is{learning strategies!diagnostics}s: detecting unexpressed variable equal\is{variable equality}ities and the new diagnostic\is{learning strategies!diagnostics} detecting whether two constructions were applied during processing;
\item The agents have two repair\is{learning strategies!repair strategies} strategies: one for inventing and learning new verb\is{verb}-specific marker\is{case!case marking}\is{case!case marking}s and one for combining these marker\is{case!case marking}\is{case!case marking}s into a larger construction. The invention and learning strategy also includes the possibility of exten\is{extension}ding and reusing\is{reuse} existing marker\is{case!case marking}\is{case!case marking}s through analog\is{analogy}ical reasoning. The algorithm for analog\is{analogy}y is the same as in baseline experiment 3 and only looks at individual marker\is{case!case marking}\is{case!case marking}s.
\end{itemize}

The experiment has been tested using five different alignment strateg\is{alignment strategy}ies. The first four strategies are individual selection, top-down selection, bottom-up selection and multi-level selection\is{multi-level selection} using the same fine-grained lateral inhibition\is{lateral inhibition} mechanism as used in baseline experiment 3c. This means that competit\is{competition}ion is only held at the level of the co-occur\is{co-occurrence}rence links between a construction and a lexical entry rather than at the level of the constructions themselves. The algorithms can be summarized as follows:

\begin{itemize}
\item {\bfseries Individual selection:} This is the exact same set-up as baseline experiment 3c. If a game was successful, the hearer will increase the score of the co-occur\is{co-occurrence}rence link between the applied lexical entry and the applied construction(s) by 0.1. He will also decrease the scores of the competing links by 0.1. The score of a link is always between 0 (high uncertainty) and 1 (high confidence\is{confidence}).
\item {\bfseries Top-down selection:} In this strategy, the hearer will not only increase the score of the relevant co-occur\is{co-occurrence}rence link, but also the score of all the co-occur\is{co-occurrence}rence links that link the lexical entry to the smaller constructions which are related to the applied construction. The scores of the competit\is{competition}ors of these links are decreased.
\item {\bfseries Bottom-up selection:}  In this strategy, the hearer increases the score of the relevant link and of all the co-occur\is{co-occurrence}rence links that link the lexical entry to the larger constructions which are related to the applied constructions. All competit\is{competition}ors of these links are punished.
\item {\bfseries Multi-level selection\is{multi-level selection}:} In this strategy, the hearer increases the scores of the relevant co-occur\is{co-occurrence}rence links and of all the links which link the lexical entry to constructions that are related to the applied constructions.
\end{itemize}

The fifth experimental set-up does not involve lateral inhibition\is{lateral inhibition} but implements {\bfseries multi-level selection\is{multi-level selection} and memory decay\is{memory decay}}. In this set-up, the hearer will not only increase the frequency\is{frequency} score of the applied constructions, but also that of all the related constructions by 1. The frequency\is{frequency} scores have no upper limit, so the higher the score, the more entrench\is{entrenchment}ed the construction is. After an agent has individually engaged in 200 language game\is{language game}s, the frequency\is{frequency} scores of all the items in the inventory are decreased.

In all five set-ups, the speaker will use the co-occur\is{co-occurrence}rence links to speed up processing. This means that not the entire inventory of constructions is considered, but only those constructions which are linked to the lexical entry. Links can be added through co-occur\is{co-occurrence}rence. When the speaker is faced with multiple hypotheses, he will choose the construction which either has the strongest co-occur\is{co-occurrence}rence link with the lexical entry (in the first four set-ups) or the one with the highest token frequency\is{token frequency}\is{frequency} (in the fifth set-up). During processing, only the scores of individual competit\is{competition}ors are taken into account. All simulations have been run in 10 series of 12.000 language game\is{language game}s.

\subsection{Results and discussion}

In this section, the first four set-ups are again compared to each other to demonstrate the reoccurrence of the problem of systematicity\is{systematicity}. The fourth set-up (multi-level selection\is{multi-level selection} with lateral inhibition\is{lateral inhibition}) is then compared more thoroughly to the fifth set-up using multi-level selection\is{multi-level selection} and memory decay\is{memory decay}. Finally, this section offers a closer look at one language evolved using the fifth set-up.

\begin{figure}[t]
\centerline{\includegraphics[width=\textwidth]{Chapter4/figs/systematicity3}}
  \caption[Experiment 3: systematicity]{This graph compares systematicity\is{systematicity} in the first four experimental set-ups of experiment 3. The results confirm those of experiment 2. Even though there is potentially less variation\is{variation} because of the reuse\is{reuse} of existing marker\is{case!case marking}s, individual selection stagnates at 50\% systematicity\is{systematicity}. Bottom-up selection increases systematicity\is{systematicity} beyond 80\% but also gets stuck. Top-down selection manages to reach full systematicity\is{systematicity} in more simulations than in experiment 2 because of the smaller variation\is{variation} space, but does not guarantee full systematicity\is{systematicity}. Only multi-level selection\is{multi-level selection} reaches systematicity\is{systematicity} in all the simulations and does so significantly faster than the other alignment strateg\is{alignment strategy}ies.}
   \label{f:systematicity3}
\end{figure}


\subsubsection{Results}
 \figref{f:systematicity3} compares the first four experimental set-ups to each other in terms of systematicity\is{systematicity}. The lines indicating systematicity\is{systematicity} for each set-up show the same behaviour as those in experiment 2 (\figref{f:systematicity2}). The first set-up using the alignment strateg\is{alignment strategy}y of individual selection fluctuates between 45 and 60\% systematicity\is{systematicity} and stops evolving after 6.000 language game\is{language game}s. The bottom-up strategy reaches more than 80\% systematicity\is{systematicity} after 8.000 language game\is{language game}s. In some simulations, this strategy leads up to 90\% but never to maximum systematicity\is{systematicity}. Top-down selection performs a bit better than in experiment 2 due to the fact that the agent's capacity of reusing\is{reuse} existing marker\is{case!case marking}s leads to a smaller variation\is{variation} space so `frozen accidents' are less likely. Yet, as the results show, some simulations still involve an unsystematic convention\is{convention} and reaching systematicity\is{systematicity} takes a longer time than the multi-level selection\is{multi-level selection} strategy. The latter strategy is again the only one which leads to full systematicity\is{systematicity} in all the simulations.

\begin{figure} 
\centerline{\includegraphics[width=\textwidth]{Chapter4/figs/coherence3}}
  \caption[Experiment 3: coherence]{This graph compares the first four set-ups of experiment 3 in terms of meaning-form coherence. The graph shows that multi-level and top-down selection perform equally well and reach 100\% between 6.000 and 8.000 language game\is{language game}s. Bottom-up and individual selection again run almost parallel and reach coherence faster than in experiment 2 because of the smaller variation\is{variation} space.}
   \label{f:coherence3}
\end{figure}

The four set-ups also confirm the results of experiment 2 in terms of coherence. \figref{f:coherence3} shows that bottom-up selection and individual selection again run almost parallel in terms of convergence. This time the agents reach coherence faster because of the smaller variation\is{variation} space. Multi-level and top-down selection also perform equally well and reach coherence between 6.000 and 8.000 language game\is{language game}s.

\begin{figure}[b]
\centerline{\includegraphics[width=\textwidth]{Chapter4/figs/markers3}}
  \caption[Experiment 3: number of markers]{This graph shows the average number of marker\is{case!case marking}s known by each agent using the strategy of multi-level selection\is{multi-level selection} with lateral inhibition\is{lateral inhibition} (the fourth set-up of experiment 3). The results show that the generalization rate of the agents is not impressive: only 3--5 semantic roles survive the competit\is{competition}ion as opposed to 18--25 specific marker\is{case!case marking}s.}
   \label{f:markers3}
\end{figure}

\figref{f:markers3} gives an indication of the kinds of languages that are formed in the population\is{speech population} if the agents use the fourth set-up (multi-level selection\is{multi-level selection} with lateral inhibition\is{lateral inhibition}). The graph shows that the generalization rate of the agents is not really impressive: only three to five generaliz\is{generalization}ed roles survive the competit\is{competition}ion. Moreover, these roles only cover two or maximally three participant roles. This is clear from the fact that there are still 18 to 25 specific marker\is{case!case marking}s floating around in the population\is{speech population}.

\begin{figure}[p]
\centering
\begin{tabular}{c}
{\includegraphics[width=0.8\textwidth]{Chapter4/figs/systematicity3b}}
\\
{\includegraphics[width=0.8\textwidth]{Chapter4/figs/markers3-decay}}
\end{tabular}
\caption[Experiment 3: results for set-up 4]{The top graph shows communicative success\is{communicative success}, cognitive effort\is{cognitive effort}, systematicity\is{systematicity} and meaning-form coherence in the set-up using multi-level selection\is{multi-level selection} and decay. The bottom graph shows the average number of marker\is{case!case marking}s in the same set-up.}
\label{f:multi-level-decay}
\end{figure}

These results can be compared to the performance\is{performance} of the fifth set-up (multi-level selection\is{multi-level selection} with decay) which is illustrated in \figref{f:multi-level-decay}. The top graph shows the results for communicative success\is{communicative success}, cognitive effort\is{cognitive effort}, meaning-form coherence and systematicity\is{systematicity}. As the graph indicates, the agents succeed in reaching full systematicity\is{systematicity} somewhere between 8.000 and 12.000 language game\is{language game}s, which is a bit slower than the alignment strateg\is{alignment strategy}y using lateral inhibition\is{lateral inhibition}. The bottom graph shows the average number of marker\is{case!case marking}s floating around in the population\is{speech population}. Here, we see that the average number of specific marker\is{case!case marking}s has made a significant drop from 18--25 marker\is{case!case marking}s to only 9. The number of semantic roles shifts from simulation to simulation between 4 and 6. The semantic roles also tend to be more general categories than in the simulations using lateral inhibition\is{lateral inhibition}.


Here is a list of marker\is{case!case marking}s and the participant roles they cover in one of the simulations (from more general to more specific):

\begin{itemize}
\item {\em -kad:} object-1, approach-1, fall-1, touch-1, move-outside-1
\item {\em -fuir:}  grasp-2, hide-2, move-inside-2, touch-2, walk-to-1
\item {\em -kazo:} approach-2, fall-2, grasp-1, hide-1, walk-to-2
\item {\em -hesa:} move-1, take-3
\item {\em -ti:} visible-1, take-1
\item {\em -qiwo:} move-inside-1, give-3
\item {\em -fen:} distance-decreasing-1
\item {\em -rem:} distance-decreasing-2
\item {\em -gaeh:} move-outside-2
\largerpage
\item {\em -wupu:} cause-move-on-1
\item {\em -chuiw:} cause-move-on-2
\item {\em -nuip:} cause-move-on-3
\item {\em -tu:} give-1
\item {\em -hozae:} give-2
\item {\em -fut:} take-2
\end{itemize}

The above marker\is{case!case marking}s occur systematically across patterns for marking the same parti\-ci\-pant roles. In this specific example, the agents succeeded in reaching coherence and systematicity\is{systematicity} after only 8.000 language game\is{language game}s. When we compare the results to those of baseline experiment 3d, roughly the same level of generalization is reached.

\begin{figure}[p]
\centerline{\includegraphics[width=0.75\textwidth]{Chapter4/figs/network}}
  \caption[Experiment 3: a partial network of set-up 5]{This figure gives a partial network for one agent in one of the simulations using multi-level selection\is{multi-level selection} with decay. In the middle there are three constructions which are systematically related to each other. The constructions are also linked with lexical entries. The links act both as co-occur\is{co-occurrence}rence links for optimizing processing and as fusion\is{fusion} links for integrating the participant roles of the lexical entries with the semantic roles of the constructions. The networks are constructed in a stepwise fashion as a response to communicative needs.}
   \label{f:network}
\end{figure}

Finally, \figref{f:network} looks inside the linguistic inventor\is{linguistic inventory}y of a single agent and offers a partial network of the agent's knowledge of its language. The figure concentrates on three constructions (in the middle) which are related to each other (indicated by the dotted line). The relation between the constructions indicate that construction-27 was created as a pattern of construction-2 and construction-10. The figure also shows all the lexical entries that are convention\is{convention}ally associated with these constructions. The links between the lexical entries and the constructions are used for optimizing processing: instead of trying out all the constructions in memory, only the linked constructions are considered. Links can be added as part of a problem-solving\is{problem-solving} process during communication or pruned if the co-occur\is{co-occurrence}rence is not a successful one. Some redundant co-occur\is{co-occurrence}rence links may survive in the inventory. The links can also be seen as fusion\is{fusion} links: they are annotated with information on how the participant roles can be fused with the semantic roles of the construction. This annotation is however not used by the agents themselves but for the clarity of interpretation for the experimenter. The actual fusion\is{fusion} is taken care of by the unification\is{unification} of the potential valents of the lexical entry with the actual valency of the construction.


\subsubsection{Discussion}
 The results of experiment 3 confirm the problem of systematicity\is{systematicity} that was uncovered in the other experiments. Here too, the strategy using multi-level selection\is{multi-level selection} was the only one to yield fully systematic languages. The systematicity\is{systematicity} rate was in each of the first four set-ups comparable to the rates in experiment 2. This may come as a surprise since the variation\is{variation} space is potentially smaller because the agents can exten\is{extension}d the use of existing marker\is{case!case marking}s rather than inventing new ones all the time.

A closer look at the number of marker\is{case!case marking}s in the fourth set-up (multi-level selection\is{multi-level selection} with lateral inhibition\is{lateral inhibition}), however, pointed to the reason for the small differences in systematicity\is{systematicity} rates: only a very small number of semantic roles survived the competit\is{competition}ion compared to a large number of specific marker\is{case!case marking}s. This means that the generalization rate is not really impressive so the number of variation\is{variation}s is not much smaller in these simulations than was the case in experiment 2. The resulting number of specific mar\-kers versus semantic roles corresponds to the results obtained in baseline experiment 3c and also the reason for the result is the same: since the competit\is{competition}ion is held at the level of co-occur\is{co-occurrence}rence links and not at the level of constructions, the type frequency\is{type frequency}\is{frequency} of a marker\is{case!case marking} does not translate into a larger category gravity\is{category gravity}. Competition is held only in a local context so specific mar\-kers have an equally high chance of winning as more generaliz\is{generalization}ed semantic roles.

The results of the fifth set-up (multi-level selection\is{multi-level selection} with memory decay\is{memory decay}) also confirm the results of the baseline experiments and improve significantly in terms of generalization over the simulations using the fine-grained lateral inhibition\is{lateral inhibition} dynamics. The number of specific categories has dropped by half and larger, more general semantic roles have a selectionist advantage because they occur more often. The generalization rate roughly matches the performance\is{performance} of baseline experiment 3d.

The consistency in number of generaliz\is{generalization}ed semantic roles and specific marker\is{case!case marking}s across the baseline experiments and the pattern experiments indicate that this is the maximum generalization rate that the agents can reach. Possible improvements would have to come from two sources:

\begin{enumerate}
\item The structure of the world: The capacity of analog\is{analogy}ical reasoning is heavily dependent on the structure of the world environment in which communication takes place. If the agents have to communicate about lots of events which show recurrent patterns in terms of visual primitives, they will be able to detect more analog\is{analogy}ies. If, however, the world is totally unstructured, the agents will come up with more specific marker\is{case!case marking}s than general semantic roles.
\item The capacity of analog\is{analogy}ical reasoning can be made more flexible. At the moment, the agents make a sharp distinction between what is analog\is{analogy}ous and what is not. A possible relaxation could be to only care about whether the mapping between a source\is{semantic role!source} role and a target role is discriminating enough for identifying the target role as well. Another possibility would be to use a similarity or a distance metric instead of the more rigid structural mapping that the agents currently use.
\end{enumerate}

Experiment 3 also shows the potential power of the combination of analog\is{analogy}ical reasoning, pattern formation\is{pattern formation} and multi-level selection\is{multi-level selection} respectively. First of all, analog\is{analogy}ical reasoning over the linguistic inventor\is{linguistic inventory}y can lead to an increasing generalization rate in the population\is{speech population}, as is also shown in several instance-based\is{exemplar-based models} approaches to language \citep{daelemans05memory, skousen89analogical}. These models also argue that a language can look rule-based from outside whereas in fact the generalization is distributed over the linguistic items in the inventory. This experiment shows that this observation also holds true for the case of the emergence\is{emergence} of grammar if multi-level selection\is{multi-level selection} is applied. Finally, the formation\is{formation} of patterns to improve processing can have dramatic effects on the grammar: the patterns increase the survival chances of its related items; and they may potentially exten\is{extension}d their use as well in later interactions. 

So far, I did not spend much attention to the efficiency of the formalization of argument realization proposed in Chapter \ref{c:ar}. In all the experiments presented in this Chapter and the previous one, the representation has proven to be flexible enough to deal with the enormous amount of {\bfseries uncertainty} that is inherent to the emergence\is{emergence} of new grammar convention\is{convention}s. In this case, the agents needed a flexible way of integrating lexical entries with constructions of various degrees of entrench\is{entrenchment}ment. Instead of copying all the possible case frames into a new entry, the formalization allowed the agents to constantly `mould' their lexical entries until a stable set of convention\is{convention}s had been negotiat\is{negotiation}ed. In this way, the competit\is{competition}ion of case marker\is{case!case marking}s could be held exclusively at the level of constructions instead of creating an additional competit\is{competition}ion on the lexical level for how these lexical entries should be integrated with the constructions. The lexical entries also integrated as easily with verb\is{verb}-specific construction\is{construction!verb-specific construction}s as with verb\is{verb}-class specific constructions.

The formalization was also flexible enough to deal with {\bfseries multiple argument realization}: the agents were capable of integrating a single lexical entry into multiple patterns or constructions without the need for deriv\is{derivation}ational rules or additional copies in the lexicon\is{lexicon}. Moreover, the lexical entries do not need to `profile' their participant roles: the actual valency of a verb\is{verb} is determined by the construction it integrates with. Preferences for certain patterns of argument realization could be captured in this formalization by assigning a frequency\is{frequency} score to the co-occur\is{co-occurrence}rence links of the lexical entries and the constructions that they are convention\is{convention}ally associated with.

One aspect that is still absent in the experiment is how the functions of the case marker\is{case!case marking}s start to influence each other once they start combining into patterns. At this moment, the meanings of the marker\is{case!case marking}s stay the same and patterning only influences their survival chances. Future work would thus have to include a way for the patterns themselves to evolve, which would also require the analog\is{analogy}y to use the patterns as the source domain for innovat\is{innovation}ion rather than focusing exclusively on single marker\is{case!case marking}s.

Including the patterns into the search domain could however lead to a huge hypo\-the\-sis space and a complexity measure is needed to verify whether the algorithm for analog\is{analogy}y can scale up to larger worlds while maintaining a reasonable processing time. If not, a possible alternative could be a nearest-neighbour\is{nearest-neighbour} algorithm which has already been successfully applied to various tasks in natural language processing\is{natural language processing}. In a comparison between Royal Skou\-sen's {\em Analogical Modeling\is{Analogical Modeling}} \citep{skousen89analogical} and {\em Memory-Based Language Processing\is{Memory-Based Language Processing}} \citep{daelemans05memory}, \citet{daelemans02comparing} shows that a relatively simple and efficient nearest-neighbour\is{nearest-neighbour} learner yields comparable and sometimes even better results than the costly algorithm  of Analogical Modeling\is{Analogical Modeling}. This observation seriously challenges the more traditional approach to analog\is{analogy}y and is highly relevant for the discussion of this work as well.

\section{Towards syntactic cases}
\label{s:pattern-exp-4}


\subsection{Overview}
\largerpage
The experiments so far have dealt with stages 1 to 3 in the development of case marker\is{case!case marking}s (see Chapter 1). The next step is the introduction of syntactic roles that group together two or more semantic roles. In this section, I will introduce a first experiment that investigates how the transition to stage 4 can be achieved and what can be learned from the results. I will then use the grammatical square\is{grammatical square} as a roadmap for future experiments.

\subsection{A first experiment}

As I argued in sections \ref{s:stage3} and \ref{s:stage4}, syntactic roles impose even more abstraction\is{abstraction} on the conceptualization\is{conceptualization} of specific events than semantic roles do. In natural languages, syntactic roles typically emerge when a category gradually starts exten\is{extension}ding its use until two cases merge into one class. In this first tentative experiment, I will scaffold\is{scaffold} the merger of cases and assume that a case marker\is{case!case marking} can exten\is{extension}d its use by subcategorization rather than by merging two roles. I will make this assumption\is{assumption} more clear in the following paragraphs.


\subsubsection{Experimental set-up}
 The experiment features the exact same set-up as the fourth set-up in experiment 3: the agents are capable of reusing\is{reuse} a marker\is{case!case marking} through analog\is{analogy}y and combining the marker\is{case!case marking}s into larger patterns. They employ the alignment strateg\is{alignment strategy}y of multi-level selection\is{multi-level selection} using lateral inhibition\is{lateral inhibition}. The main question of the experiment is whether the agents are capable of aligning their grammars, which form an abstract intermediary layer of semantic and syntactic roles which are not directly observable by the other agents.

The novelty of this particular set-up involves the idea of `reusing\is{reuse} as much as possible'. Roughly speaking, a speaker will reuse\is{reuse} an existing marker\is{case!case marking} even if it is not analog\is{analogy}ous to the target role, on the condition that the marker\is{case!case marking} does not cover a conflicting participant role yet. The algorithm is operationalized as follows:

\begin{enumerate}
\item If the speaker diagnosed a problem of unexpressed variable equal\is{variable equality}ities, he tries to repair\is{learning strategies!repair strategies} the problem.
\item The speaker checks whether he already has a marker\is{case!case marking} which can be reuse\is{reuse}d:
\begin{enumerate}
\item Take all known marker\is{case!case marking}s. Markers are ordered according to type frequency\is{type frequency}\is{frequency}, that is, from more general (i.e. covering the most participant roles) to less general (i.e. covering the least participant roles). Loop through the marker\is{case!case marking}s until a solution is found:
\begin{enumerate}
\item Take all the semantic roles that are covered by the marker\is{case!case marking}, also ordered according to type frequency\is{type frequency}\is{frequency}.
\item Loop through the semantic roles until a role is found which is analog\is{analogy}ous to the target role. If so, return the analog\is{analogy}y.
\end{enumerate}
\item If a solution is found, return the analog\is{analogy}y. If not...
\begin{enumerate}
\item Take the most general marker\is{case!case marking} which does not cover another participant role of the same event yet;
\item Create a new role for the target role and make it a subcategory of the chosen marker\is{case!case marking}.
\end{enumerate}
\item If there are no marker\is{case!case marking}s yet or if no marker\is{case!case marking} can be found that does not already cover a conflicting participant role, create a new marker\is{case!case marking}.
\end{enumerate}
\item The speaker creates the necessary rules and/or links in the inventory.
\end{enumerate}

The hearer learns a marker\is{case!case marking} in a similar way. If he observes a marker\is{case!case marking} in a new situation, he will first try to retrieve an analog\is{analogy}ous role covered by that marker\is{case!case marking}. If he cannot retrieve the analog\is{analogy}y, he creates a new specific role and makes it a subcategory of the marker\is{case!case marking}. It will often happen that the speaker used a marker\is{case!case marking} which according to the hearer already covered a conflicting participant role. For example, the speaker uses {\em -bo} for marking `approach-1', whereas the hearer has already observed this marker\is{case!case marking} for covering `approach-2'. In this case, the hearer will nevertheless create a new specific role for the new use as a possible subcategory of the marker\is{case!case marking}. The fine-grained alignment strateg\is{alignment strategy}y of the experimental set-up is flexible enough to rule out which participant role should win the competit\is{competition}ion (unless another marker\is{case!case marking} takes over).

The main idea behind this innovat\is{innovation}ion and learning strategy is that the speaker will be reluctant to invent a new marker\is{case!case marking}. Rather, he will reuse\is{reuse} a marker\is{case!case marking} as long as it is discriminating a participant role in an event from the other roles. This means that, in principle, the agents should suffice in using three different marker\is{case!case marking}s. This experimental set-up has been implemented in a two-agent simulation and a five-agent simulation.
\begin{figure}[p]
\centerline{\includegraphics[width=0.9\textwidth]{Chapter4/figs/two-agent}}
  \caption[Syntactic roles: the mapping for two agents]{This diagram shows the mapping between participant roles and semantic roles; and between semantic roles and syntactic roles in the two-agent simulation on the emergence\is{emergence} of syntactic roles. Since there are no variation\is{variation}s in the simulation, both agents align their grammars perfectly. The results are taken as the baseline for the multi-agent simulations.}
   \label{f:two-agent}
\end{figure}


\subsubsection{Results}
 The two-agent simulation features no variation\is{variation} in the population\is{speech population} so the agents have no problems in aligning their grammars since they are endowed with the same algorithm of analog\is{analogy}ical reasoning. The resulting grammar is illustrated in \figref{f:two-agent} which shows the mapping between participant, semantic and syntactic roles.

The diagram shows that the two agents indeed agree upon three case marker\is{case!case marking}s for covering all thirty participant roles. The results even seem to improve on the marker\is{case!case marking}s that were formed in baseline experiment 3a: ten semantic roles have been formed as opposed to six specific roles (which are also called `sem-roles' in the experiment for convenience's sake). On the other hand, the baseline experiment featured two semantic roles which covered six and four participant roles respectively, whereas the semantic roles in this case reach a maximum of four.

In terms of inventory size, the agents require 16 single-argument, 16 two-argument and 3 three-argument constructions. The fact that the number of two- and three-argument constructions is almost the same as in the experiments using no analog\is{analogy}y is due to the fact that only in two cases, a larger construction can also be fused with two different lexical entries. For example, {\em approach} and {\em fall} integrate with the same three constructions. So even though single-argument constructions can group several participant roles together, the patterns do not succeed in going beyond a verb\is{verb}-specific use.

The results of the two-agent simulation can now be taken as the baseline for the multi-agent simulations. A similar snapshot of the multi-agent simulation is presented in \figref{f:multi-agent} and \figref{f:multi-agent2}. These diagrams present the internal mappings between participant, semantic and syntactic roles from two different agents in the same population\is{speech population}. Both agents (as well as the other agents in the population\is{speech population}) converge on a coherent set of meaning-form mappings.

A closer look at the two diagrams shows that the agents have converged on five different case marker\is{case!case marking}s. Two of these marker\is{case!case marking}s are participant role-specific, while the others group together multiple roles. Compared to the two-agent simulation, there is a significant loss in number of semantic roles: the first agent has constructed five semantic roles and the second agent has constructed six semantic roles. This leaves 18 and 17 specific roles respectively. Most of the semantic roles also only cover two participant roles.
\begin{figure}[p]
\centerline{\includegraphics[width=0.9\textwidth]{Chapter4/figs/multi-agent}}
  \caption[Syntactic roles: the mapping for a single agent (1)]{The mapping between participant, semantic and syntactic roles in a single agent in the multi-agent simulation.}
   \label{f:multi-agent}
\end{figure}
\begin{figure}[p]
\centerline{\includegraphics[width=0.9\textwidth]{Chapter4/figs/multi-agent2}}
  \caption[Syntactic roles: the mapping for a single agent (2)]{This diagram shows the internal mapping as known by another agent in the population\is{speech population}.}
   \label{f:multi-agent2}
\end{figure}

A comparison of the semantic roles of both agents also shows that they do not align their internal categorization. For example, the agent in \figref{f:multi-agent} groups `move-outside-2', `walk-to-1', `fall-2' and `grasp-2' together. The other agent has a similar category but has constructed a separate role for `fall-2'. The other agent has also created a semantic role for covering `move-inside-2' and `move', whereas these are two distinct categories for the first agent.


\subsubsection{Discussion of the two-agent simulation}
 The agents in the two-agent simulation were capable of improving over baseline experiment 3a in terms of semantic roles, single-argument constructions and the number of marker\is{case!case marking}s. The improvement is due to the fact that the limited use of marker\is{case!case marking}s guides the search space more strongly during innovat\is{innovation}ion. In the previous experiments, each semantic role had its own case marker\is{case!case marking} so the chances that they had the same type frequency\is{type frequency}\is{frequency} were quite high. In this case, the speaker would always randomly choose which semantic role to exten\is{extension}d. In this new simulation, the type frequency\is{type frequency}\is{frequency} of the syntactic role was more important which led to a faster divergence between the productivity\is{productivity} rate of the cases. This means that semantic roles which would otherwise miss exten\is{extension}sion due to random choice now have more weight to categorize new participant roles.

An interesting side-effect of the innovat\is{innovation}ion algorithm is that there are two syntactic roles which cover almost exclusively semantic roles while a third role acts as a waste basket category for three participant roles which are all three part of events featuring three participants. This maps onto the distinction between agents and patients (and subject\is{syntactic role!subject}s and object\is{syntactic role!object}s) that is made by most of the languages in the world. Most theories of language assume a (near-)universal distinction between agents and patients to be given (either based on a universal conceptual space or on Universal Grammar\is{Universal Grammar}). This first tentative experiment suggests an alternative hypothesis: the distinction could emerge as a side-effect of communicative goals because language users want to make grammatically and communicatively relevant distinctions. In most of the cases, two or three syntactic roles suffice. The exten\is{extension}sion of case marker\is{case!case marking}s and the merger of semantic role could thus spontaneously lead to `core' cases.

\newpage 
A final remark considering the two-agent simulation is that there is no gain in terms of inventory size with respect to larger constructions. This is due to the fact that only a couple of constructions share the same semantic roles which combine into the same larger construction. As I suggested during the discussion of experiment 3, the newly formed patterns themselves should be considered by the analog\is{analogy}y as well. This could further increase the generalization and productivity\is{productivity} rate of the agents and could be an additional drive towards a prototypical agent-patient distinction.
\\
\\
\subsubsection{Discussion of the multi-agent simulation}
 The results of the multi-agent simulation shows that the alignment of an indirect and multilayered grammatical mapping is no trivial issue: the number of semantic roles constructed by each agent drops significantly and there are differences in how the internal mapping of each agent is organized. This is basically a problem of feedback: there is too much variation\is{variation} floating around in the population\is{speech population} for the agents to successfully retrieve the analog\is{analogy}y meant by the speaker. Additional feedback could consist of alternative agnat\is{agnation}ing structures that could be exploited for constructing semantic roles. This, however, would require the capacity of dynamically updating the function or meaning of the semantic roles, which the agents do not have (also see the concluding remarks in \sectref{s:base3}).

\subsection{The grammar square: a roadmap for further work}
\is{grammatical square}
\label{s:future}


The first experiments towards syntax showed that once the mapping between semantic roles and syntactic roles becomes indirect and polysemous, the agents are faced with a complex coordination problem\is{coordination problem}: the abstract layer of semantic and syntactic roles is not directly observable from the outside and the agents have no means of finding a shared categorization. Yet the alignment of this internal categorization is crucial in order to preserve a good productivity\is{productivity} and generalization rate for reaching communicative success\is{communicative success} in future interactions. In order to take this step, the experimental set-up needs to be expanded. We can use the grammar square (repeated in \figref{f:square-bis}), as a guidance for identifying which efforts need to be made in order to solve this coordination problem.

\begin{figure} 
\centerline{\includegraphics[scale=0.7]{Chapter4/figs/case}}
  \caption[The grammatical square\is{grammatical square}]{The grammar square\is{grammatical square} can be read as a roadmap for future research. Each mapping in the square is dependent on the context, so experiments should investigate which mechanisms and conditions can lead to such an indirect mapping.}
   \label{f:square-bis}
\end{figure}

\subsubsection{Mapping participant roles onto semantic roles}
 Participant roles of a particular event can map onto several semantic roles in natural languages. This is clear in the following sentences, in which {\em the window} first plays the role of patient, then some entity which undergoes a change of state, and finally some stative entity:

\ea
He broke the window.
\label{e:aspect-1}
\item The window broke.
\label{e:aspect-2}
\item The window was broken.
\label{e:aspect-3}
\z

In the experiments presented in this book, such functional variation\is{variation} was impossible: during conceptualization\is{conceptualization}, the agents always profile \is{event profile}the complete participant role in the event structure\is{event structure}. This could lead to sentences similar as example \ref{e:aspect-1}. However, in the other two sentences, only a subpart of the participant role is profiled\is{event profile}: example \ref{e:aspect-2} profiles the change of state whereas example \ref{e:aspect-3} profiles the resulting state of the participant.

In order to achieve the same functional variation\is{variation}, the agents would thus have to be able to include aspect\is{aspect} (and tense\is{tense}) distinctions in their conceptualization\is{conceptualization}. For this, the conceptualization\is{conceptualization} algorithm and the algorithm for analog\is{analogy}y would have to be changed in order to include the hierarchical structure of the event descriptions and the time stamps provided by the event recognition\is{event recognition} system. On the level of the interaction pattern, the agents would somehow have to be able to get sufficient feedback in order to recognize and learn the relevant aspect\is{aspect}ual distinctions. The emergence\is{emergence} of grammatical marker\is{case!case marking}s for aspect\is{aspect} (and tense\is{tense}) is no trivial matter and goes well beyond the scope of this book.


\subsubsection{Mapping semantic roles onto syntactic roles}
 The mapping between semantic roles and syntactic roles can already be multilayered in nature without taking information structure\is{information structure} into account: in examples \ref{e:aspect-2} and \ref{e:aspect-3}, the distinction is not one of active\is{construction!active construction} versus passive\is{construction!passive construction}, but rather one of aspect\is{aspect}. In order to see such alternations emerging, the idea of reuse\is{reuse} can be exploited again. As the agents develop their grammars, they increase their expressive power. From the moment they want to express aspect\is{aspect}ual distinctions as well, they could try to reuse\is{reuse} the existing grammatical system instead of inventing some new strategy. This puts pressure on the existing convention\is{convention}s and may lead to additional abstraction\is{abstraction}s in the form of syntactic roles.

Another way to investigate the emergence\is{emergence} of syntactic roles would be to endow the agents with the capacity of dynamically updating the representation of their categories. If categories are not fixed but dynamic, there is a risk of `category leakage\is{category leakage}' as observed in natural languages: some categories start expanding their use which could lead to the merger of two semantic roles into one case. Typically, the more frequent cases would start exten\is{extension}ding their usage which could gradually lead to prototypical agent and patient categories.


\subsubsection{Mapping syntactic roles onto case markers}
\is{case!case marking}
 In the experiments, there is a one-to-one relation between syntactic roles and case marker\is{case!case marking}s. In natural case grammar\is{case!case grammar}s, however, there are often paradigms\is{case!case paradigm} of related marker\is{case!case marking}s that together cover a particular case. These marker\is{case!case marking} alternations typically indicate grammatical distinctions in terms of gender\is{gender} and number\is{number}. Evolving this kind of variation\is{variation} would thus require a need for marking grammatically relevant distinctions between arguments.

Even more intriguing are case systems\is{case!case system} where the same marker\is{case!case marking}s can be used to indicate different cases. For example, Latin\is{Latin} uses the inflection\is{inflection} {\em -um} to mark nominative case in neuter singular\is{number!singular} words ({\em bellum} `war') and accusative case in masculine singular\is{number!singular} words ({\em dominum} `master') \citep[4--5]{blake94case}. This means that case marker\is{case!case marking}s somehow manage to grow a paradigmatic\is{case!case paradigm} case system\is{case!case system} which goes beyond the borders of individual cases. Applied to the experiments, this would mean lifting the present assumption\is{assumption} that competing case marker\is{case!case marking}s are always competing with each other for marking a particular participant role. Instead, agents should be allowed to accept competit\is{competition}ion and variation\is{variation} at each possible mapping in the grammatical square\is{grammatical square}. This may lead to a credit assignment problem\is{credit assignment problem} in which the agents can never have complete certainty about which mapping in the grammatical square\is{grammatical square} was relevant during a particular innovat\is{innovation}ion.

From the above discussion it should be clear that scaling up the experiments towards richer syntax and grammar is not a trivial matter. It would include research into the emergence\is{emergence} of aspect\is{aspect} and tense\is{tense} distinctions, a dynamic representation of linguistic categories, and allowing competit\is{competition}ion on all aspects of the grammatical square\is{grammatical square}. A scale-up to include information structure\is{information structure} as well would involve expansion of the language game\is{language game} model to larger dialogue\is{dialogue}s, which creates the need for additional capacities such as episodic memory\is{episodic memory}, scoping and coordination issues and possibly anaphora resolution\is{anaphora resolution}.
\chapter{Flexionsparadigmen und Realisierungsregeln}\label{5}

Ziel dieses Kapitels ist, die in \chapref{4} eingeführte Theorie und Messmethode auf die in dieser Arbeit untersuchten Varietäten zu übertragen. Da sich die Phänomene der analysierten Kategorien in den verschiedenen Varietäten wiederholen, wird nicht jedes Paradigma und jedes System an RRs jeder Varietät einzeln besprochen. Vielmehr geht es darum, exemplarisch die Typen an RRs für jede Kategorie und die Typen an Problemen und deren Lösung vorzustellen. Es soll also an Beispielen dargestellt werden, wie die Varietäten dieses Samples mit der in \chapref{4} eingeführten Theorie und Messmethode analysiert wurden. Die vollständigen Paradigmen und RRs der hier untersuchten Varietäten befinden sich im Anhang A bzw. im Anhang B.

Die Typen von Phänomenen, Problemen und Analysen wiederholen sich besonders innerhalb derselben Kategorie. Deshalb ist dieses Kapitel nach Kategorien gegliedert: \isi{Substantive} (\sectref{5.1}), \isi{Adjektive} (\sectref{5.2}), \isi{Personalpronomen} (\sectref{5.3}), \isi{Interrogativpronomen} (\sectref{5.4}), \isi{bestimmter Artikel}/\isi{Demonstrativpronomen} (\sectref{5.5}) und \isi{unbestimmter Artikel}/\isi{Possessivpronomen} (\sectref{5.6}). Es soll hier noch darauf hingewiesen werden, dass am Ende des Abschnitts \sectref{5.1} (\sectref{5.1.6}) anhand eines Beispiels die Systematisierungsarbeit gezeigt wird, d.h., die Analyseschritte von den Angaben aus den Quellen zum Paradigma und vom Paradigma zu den RRs. Dies wird anhand der Substantivflexion von Jaun vorgestellt.

Vorausgeschickt werden hier zwei grundsätzliche Überlegungen, die alle folgenden Kapitel betreffen: Erstens die Definition der Konzepte \isi{Wurzel} und Stamm und zweitens die Trennbarkeit eines Wortes in seine \isi{Wurzel} und \isi{Affixe}.

Erstens soll kurz wiederholt werden, was unter \isi{Wurzel} und Stamm verstanden wird. Dies wurde in \sectref{4.1.3.3} ausführlich erklärt. Dabei sind zwei grundsätzliche Annahmen in Erinnerung zu rufen: a) Nur \isi{Wurzeln} stehen im \isi{Radikon}, welche ausschließlich die Form darstellen und keine Bedeutung tragen; b) Flexion ist ein rein phonologischer Prozess, Flexionsaffixe tragen ebenfalls keine Bedeutung. Durch die Flexion, d.h. durch die RRs, wird ein flektiertes Wort von einer \isi{Wurzel} abgeleitet. Diese Idee wird auf die Stammbildung übertragen, denn auch bei der Stammbildung wird von einer \isi{Wurzel} eine neue Form abgeleitet. Bei der Flexion und Stammbildung handelt es sich also formal um dieselben Prozesse. Beispielsweise kann von der \isi{Wurzel} \textit{kind} der Pluralstamm \textit{kind}-\textit{ər} gebildet werden, an den weiteres Material suffigiert werden kann (z.B. \textit{kind}-\textit{ər}-\textit{n}), aber nicht muss. Des Weiteren ist festzuhalten, dass nicht von jeder \isi{Wurzel} zuerst ein Stamm gebildet werden muss, bevor Flexionsaffixe angehängt werden können. Das Wort \textit{kind}-\textit{əs} entsteht also dadurch, dass die \isi{Realisierungsregel} \mbox{RR \textsubscript{C, \{\textsc{case:gen}, \textsc{num:sg}\}, \textsc{n[}\textsc{ic:}1} \textsubscript{${\veebar}$}\textsubscript{ 2} \textsubscript{${\veebar}$}\textsubscript{ 3} \textsubscript{${\veebar}$}\textsubscript{ 5} \textsubscript{${\veebar}$}\textsubscript{ 9]} ($\langle$X,$\sigma$$\rangle$) = \textsubscript{def} $\langle$X\textit{s}ˊ,$\sigma$ $\rangle$} von der \isi{Wurzel} \textit{kind} das Wort \textit{kind}-\textit{əs} ableitet.

Zweitens können die \isi{Wurzel} und die \isi{Affixe} in den Wörtern der einen Wortarten voneinander getrennt werden, in den Wörtern anderer Wortarten jedoch nicht, wenn man sie synchron analysiert, was in dieser Arbeit gemacht wird. \isi{Substantive}, \isi{Adjektive} und \isi{Possessivpronomen} sind in eine \isi{Wurzel} und \isi{Affixe} dividierbar. Die \isi{Wurzel} steht im \isi{Radikon}, die \isi{Affixe} und \is{Modifikation}Modifikationen des Stammes oder der \isi{Wurzel} werden durch RRs bestimmt. Die Formen der übrigen Wortarten jedoch (\isi{Personalpronomen}, \isi{Interrogativpronomen}, \isi{Demonstrativpronomen}, bestimmter und \isi{unbestimmter Artikel}) können aus synchroner Perspektive nicht weiter in eine \isi{Wurzel} und \isi{Affixe} aufgeteilt werden. Die RRs definieren folglich die gesamte Form. Dies stellt weder ein Problem für das der Messmethode zugrunde liegende Modell noch für die Messmethode selbst dar, da sowohl die \isi{Wurzeln} aus dem \isi{Radikon} als auch die Flexion (d.h. die RRs) nur die Form und nicht die Funktion/Bedeutung eines Wortes definieren. Beim \isi{unbestimmten Artikel} bilden folgende Varietäten eine Ausnahme, in denen eine \isi{Wurzel} von \isi{Suffixen} getrennt werden kann: Mittelhochdeutsch und die deutsche Standardsprache sowie die Dialekte von Visperterminen und Issime. Muss bei einer Wortart einer Varietät die gesamte Form stipuliert werden, wird dies in den einzelnen Kapiteln noch einmal kurz erwähnt.

\section{Substantive}\label{5.1}

\subsection{Flexionsklassen}\label{5.1.1}\label{5.1.1.}

Wie bereits in \sectref{4.3.2} erwähnt wurde, werden die \isi{Flexionsklassen} als eine Art Instruktion gesehen, wie die RRs miteinander kombiniert werden. Das Flexionssystem der \isi{Substantive} kann man sich also als ein Inventar an RRs und als eine Menge an Kombinationen dieser RRs vorstellen. Dabei gilt: Werden z.B. die RRs auf fünf unterschiedliche Weisen kombiniert, weist das System fünf \isi{Flexionsklassen} auf. Nicht nur die RRs tragen folglich zur Komplexität der Flexion bei, sondern auch die \isi{Flexionsklassen}. Denn es kommt vor, wie in \sectref{4.3.2} dargestellt, dass wenn eine Varietät A und eine Varietät B genau dieselbe Anzahl RRs aufweisen, Varietät A diese zehn Mal kombiniert, Varietät B aber nur acht Mal. Die \isi{Flexionsklassen} haben also keine Funktion oder Bedeutung im System, sie stellen lediglich die Anzahl unterschiedlicher Kombinationen der RRs dar.

Es stellt sich nun die Frage, wann eine neue \isi{Flexionsklasse} eröffnet wird. Erstens unterscheiden sich zwei \isi{Flexionsklassen} in mindestens einer RR, egal um welche Flexionsart es sich handelt (\isi{Affix}, Wur\-zel-/Stamm\-al\-ter\-na\-tion etc.). Weist eine \isi{Flexionsklasse} im Gegensatz zu allen anderen \isi{Flexionsklassen} keine overte Markierung auf, wird auch für diesen Fall eine \isi{Flexionsklasse} angenommen. Die \isi{Flexionsklassen} werden also weder nach germanischen Stämmen (z.B. \citealt{Braune2004}) noch nach Deklinationstypen (z.B. \citealt{EisenbergGelhausHenneSittaWellmann1998}) eingeteilt. Ob folglich eine neue \isi{Flexionsklasse} eröffnet wird oder nicht, hängt ausschließlich davon ab, ob diese sich in mindestens einer RR von einer anderen unterscheidet. Beispielsweise ist in der deutschen Standardsprache für die Lexeme, die ihren Plural auf -\textit{ər} bilden (\textit{Kind}{}-\textit{Kindər}, \textit{Wald}{}-\textit{Wäldər}), nur eine \isi{Flexionsklasse} anzusetzen (und nicht zwei). Denn alle Wörter mit einem -\textit{ər} als Pluralsuffix werden umgelautet. Wendet man den \isi{Umlaut} auf \textit{Wald} an, entsteht \textit{Wäld}, wendet man ihn auf \textit{Kind} an, passiert nichts, da \textit{i} nicht umgelautet werden kann. Im Gegensatz dazu sind im Issime Alemannischen zwei \isi{Flexionsklassen} für Wörter anzusetzen, die ihren Plural mit dem \isi{Suffix} -\textit{er} bilden: \textit{l}{\textit{a}}\textit{m}-\textit{l}{\textit{a}}\textit{mmer} ʻLammʼ, \textit{l}{\textit{a}}\textit{n}-\textit{l}{\textit{e}}\textit{nner} ʻLandʼ \citep[164]{Zürrer1999}. Denn nicht alle Wörter mit -\textit{er} als Pluralsuffix lauten den Wurzelvokal um, auch wenn dies möglich wäre. Die beiden \isi{Flexionsklassen} unterscheiden sich also dadurch, dass die eine eine RR für den \isi{Umlaut} hat, die andere nicht. Des Weiteren gibt es auch keine Ober- und Unterklassen, wie dies z.B. bei der Einteilung nach Stämmen üblich ist (z.B. ja- und wa-Stäm\-me als Teil der a-Stäm\-me). Alle \isi{Flexionsklassen} stehen also gleichberechtigt nebeneinander und tragen in gleicher Weise zur Komplexität bei. Deswegen können die \isi{Flexionsklassen} nummeriert werden, ohne dass die Nummer etwas über die \isi{Flexionsklasse} aussagt. Auch andere Symbole wären vorstellbar, wie z.B. die Buchstaben A-Z. Aus praktischen Gründen werden jedoch in dieser Arbeit Zahlen verwendet. Zweitens wird nur dann eine \isi{Flexionsklasse} eröffnet, wenn mindestens zwei Lexeme nach dieser \isi{Flexionsklasse} flektieren. Drittens werden in einer \isi{Flexionsklasse} nur jene Zellen angenommen, die in der Flexion der \isi{Substantive} tatsächlich unterschieden werden. Z.B. macht das Althochdeutsche im Singular und Plural durchaus Kasusunterschiede (Paradigma 1), während im Alemannischen des Sensebezirks sowohl im Singular als auch im Plural alle \isi{Kasus} zusammengefallen sind (Paradigma 7). Da also im Alemannischen des Sensebezirks \isi{Kasus} durch overte Markierung zwar im Pronomen, im \isi{Adjektiv} und in den Determinierern (Paradigma 27, 47, 67, 87, 107), aber nicht im \isi{Substantiv} unterschieden wird, gibt es auch keinen Grund, im Substantivparadigma Zellen für die \isi{Kasus} anzusetzen. Viertens werden Varianten, die phonologisch erklärt werden können, nicht berücksichtigt, da ausschließlich die Komplexität der Flexionsmorphologie gemessen werden soll. Z.B. eröffnen die Varianten -\textit{es}/-\textit{s} des Genitiv Singular in der deutschen Standardsprache keine neuen \isi{Flexionsklassen}, da diese \isi{Variation} phonologisch bedingt ist \citep[224-225]{EisenbergGelhausHenneSittaWellmann1998}.

\subsection{Realisierungsregeln für Affixe}\label{5.1.2}

Dieses Kapitel hat zum Ziel, die RRs zur Suffigierung vorzustellen. Die RRs für die Wur\-zel-/Stamm\-al\-ter\-na\-tio\-nen werden im nachfolgenden Kapitel präsentiert. In den hier untersuchten Varietäten können drei Kategorien durch \isi{Suffixe} kodiert werden: \isi{Numerus}, \isi{Kasus} und Possessiv. Zuerst wird hier also gezeigt, wie die RRs aussehen, die ein \isi{Suffix} für \isi{Numerus} und \isi{Kasus} definieren. Zweitens wird begründet, weshalb in den meisten Varietäten ein Possessiv-S angenommen wird. Schließlich können mehrere \isi{Suffixe} in derselben Zelle des Paradigmas stehen. Drittens wird also gezeigt, welche Art von RRs freie \isi{Variation} adäquat erfasst.

\noindent
{Numerus- und Kasussuffixe}: Bei den \isi{Affixen} der Substantivflexion in den untersuchten Varietäten handelt es sich einzig um \isi{Suffixe}. Diese \isi{Suffixe} drücken sowohl \isi{Kasus} wie auch \isi{Numerus} aus, wie die folgende RR\footnote{Die Notation der RRs wurde in \sectref{4.1.3} eingeführt.} für die deutsche Standardsprache zeigt:

% \ea%20
\begin{exe}
\exr{ex:key:20}
 RR \textsubscript{C}, \textsubscript{\{\textsc{case:dat}, \textsc{num:pl}\}}, \textsubscript{\textsc{n[}\textsc{ic:} 1} \textsubscript{\tiny $\veebar$}\textsubscript{ 2} \textsubscript{\tiny $\veebar$}\textsubscript{ 3} \textsubscript{\tiny $\veebar$}\textsubscript{ 4} \textsubscript{\tiny $\veebar$}\textsubscript{ 5} \textsubscript{\tiny $\veebar$}\textsubscript{ 6} \textsubscript{\tiny $\veebar$}\textsubscript{ 7} \textsubscript{${\veebar}$}\textsubscript{ 8]} ($\langle$X,$\sigma$ $\rangle$) = \textsubscript{def} $\langle$X\textit{ən}ˊ,$\sigma$ $\rangle$\\
\end{exe}

\noindent
{Possessiv-S}: Neben \isi{Numerus} und \isi{Kasus} weisen fast alle Dialekte und die deutsche Standardsprache ein besitzanzeigendes s-\isi{Suffix} auf, das hier Possessiv-S genannt wird und nicht mit dem Genitiv-S verwechselt werden darf. Dass das Possessiv-S angenommen werden muss, hat zwei Gründe. Erstens weisen die Dialekte, in denen es vorkommt, keine Genitivsuffixe auf, d.h., in diesen Dialekten gibt es keinen Genitiv. Zweitens wird das Possessiv-S nur an Eigennamen und Berufsbezeichnungen suffigiert, aber sowohl an Maskulina als auch an Feminina. Deswegen ist auch für die deutsche Standardsprache ein Possessiv-S anzusetzen (z.B. \textit{Annas Schwester}), weil ein Genitiv-S nie an Feminina suffigiert werden kann. Die RR für das Possessiv-S sieht wie folgt aus:

\usecounter{equation}\setcounter{equation}{45}
\ea%46
\label{ex:key:46}
 RR \textsubscript{C,} \textsubscript{\{\textsc{poss:yes}\}}\textsubscript{,} \textsubscript{\textsc{n[}\textsc{proper noun}]} ($\langle$X,$\sigma$ $\rangle$) = \textsubscript{def} $\langle$X\textit{s}ˊ,$\sigma$ $\rangle$\\
\z

Schließlich ist im Alt- und Mittelhochdeutschen \textsc{freie Variation} zu beobachten, und zwar wird eine Zelle durch zwei Formen definiert, wobei die eine aus der \isi{Wurzel} und einem \isi{Suffix} besteht und die andere nur aus der \isi{Wurzel}. Dazu werden zwei RRs (\ref{ex:key:47} + \ref{ex:key:48}) benötigt. Da beide RRs gleich spezifisch sind und in demselben \isi{Block} stehen, definieren beide RRs die Zelle Dativ Singular der \isi{Flexionsklasse} 7 (vgl. Diskussion zur \isi{Variation} und Bedingung \ref{ex:key:29} in \sectref{4.1.3.3}).

\ea%47
\label{ex:key:47}
 RR \textsubscript{C,} \textsubscript{\{\textsc{case:dat}, \textsc{num:sg}\},} \textsubscript{\textsc{n[}\textsc{ic:}7]} ($\langle$X,$\sigma$ $\rangle$) = \textsubscript{def} $\langle$X\textit{e}ˊ,$\sigma$ $\rangle$
\z

\ea%48
\label{ex:key:48}
 RR \textsubscript{C,} \textsubscript{\{\textsc{case:dat,} \textsc{num:sg}\}}, \textsubscript{\textsc{n[}\textsc{ic:}7]} ($\langle$X,$\sigma$ $\rangle$) = \textsubscript{def} $\langle$Xˊ,$\sigma$ $\rangle$
\z

Das Beispiel dazu stammt ebenfalls aus dem Althochdeutschen. Der Dativ Singular der \isi{Flexionsklasse} 7 weist sowohl \isi{Wurzel} + \isi{Suffix} -\textit{e} \REF{ex:key:47} als auch nur \isi{Wurzel} \REF{ex:key:48} auf, z.B. \textit{fater} und \textit{fater-e} \citep[214]{Braune2004}. Auf den ersten Blick sieht die RR \REF{ex:key:48} wie die RR \textit{Identity Function Default} aus (vgl. \sectref{4.1.3.2}, die hier wiederholt wird:

% \ea%26
\begin{exe}
\exr{ex:key:26}
\isi{Identity Function Default} […]\\
RR\textsubscript{n,\{\},U} ($\langle$X,$\sigma $$\rangle$) = \textsubscript{def} ($\langle$X,$\sigma$ $\rangle$). \citep[53]{Stump2001}
\end{exe}      

Die RR \textit{Identity Function Default} definiert, dass, wenn in einem \isi{Block} für ein bestimmtes Set an morphosyntaktischen Eigenschaften keine RR gefunden wird, mit der \isi{Wurzel} nichts passiert. Die RR \REF{ex:key:48} dagegen besagt, dass die Form für den Dativ Singular der \isi{Flexionsklasse} 7 der Form der \isi{Wurzel} entspricht. In \sectref{4.1.3} wurde gezeigt, dass die RRs in einem \isi{Block} miteinander in Konkurrenz stehen und dass immer jene Regel eine bestimmte Zelle definiert, die am spezifischsten für diese Zelle ist. Ist eine Zelle mit einer Form (durch eine RR) gefüllt, ist diese Zelle für weitere potentielle RRs aus demselben \isi{Block} blockiert. Sind jedoch zwei oder mehrere RRs für eine bestimmte Zelle gleich spezifisch, finden beide RRs Anwendung und folglich definieren zwei Formen diese Zelle. Stehen also in einer Zelle zwei Formen, eine bestehend aus \isi{Wurzel}+\isi{Suffix} und eine nur aus der \isi{Wurzel}, muss die RR für die \isi{Wurzel} gleich spezifisch sein wie jene für die Form \isi{Wurzel}+\isi{Suffix}. Da nun die RR \textit{Identity Function Default} \REF{ex:key:26} weniger spezifisch ist als die RR \REF{ex:key:47}, blockiert die RR \REF{ex:key:47} die RR \REF{ex:key:26}. Weil aber sowohl \textit{fatere} wie auch \textit{fater} in der Zelle Dativ Singular stehen, muss eine RR für \textit{fater} angenommen werden, die genauso spezifisch ist wie die RR für \textit{fatere}, was durch die RRs \REF{ex:key:47} und \REF{ex:key:48} gewährleistet ist.

Wichtig ist hier also festzuhalten, dass die Formen, die in freier \isi{Variation} stehen, durch gleich komplexe RRs repräsentiert sind, und zwar unabhängig davon, ob z.B. zwei \isi{Suffixe} in derselben Zelle stehen oder ein \isi{Suffix} und eine \isi{Wurzel}. Würden wir eine freie \isi{Variation} (z.B. \isi{Suffix}/\isi{Suffix}) als komplexer ansehen als eine andere freie \isi{Variation} (z.B. \isi{Suffix}/\isi{Wurzel}), wäre zu definieren, um wie viel die eine mehr oder weniger komplex ist. Das heißt, man würde von einer theoriegeleiteten Komplexitätsmessung, die nicht absolut, aber maximal möglich objektiv ist, zu einer eher von der Intuition geleiteten Komplexitätsmessung übergehen. Des Weiteren wird hier jede untersuchte Varietät ausschließlich streng synchron analysiert, was für die freie \isi{Variation} folgende Konsequenz hat. Zu einem bestimmten Zeitpunkt kann es freie \isi{Variation} geben und ob diese freie \isi{Variation} sich stabilisiert oder ob die eine oder andere Form sich durchsetzt, kann synchron nicht eruiert werden, auch wenn wir aus diachronen Daten das Wissen dazu haben. Synchron sind zwei (oder mehr) Formen in derselben Zelle des Paradigmas, die beide durch RRs definiert werden müssen. Auf das Beispiel aus dem Althochdeutschen bezogen, bedeutet das also Folgendes. Wir wissen, dass die alte Form \textit{fater} lautet, die neue Form \textit{fatere} \citep[214]{Braune2004}. Es handelt sich diachron also um einen Aufbau an Komplexität: Im ältesten Althochdeutsch gibt es keine RR für die Zelle Dativ Singular, da die Defaultform \textit{fater} verwendet wird; im jüngeren Althochdeutsch gibt es hingegen eine RR für die Zelle Dativ Singular, da die suffigierte Form \textit{fater}-\textit{e} verwendet wird. Für den Übergang zwischen diesen beiden Stadien sind, wie oben dargestellt, zwei RRs anzusetzen. Genau dasselbe gilt für den umgekehrten Fall, wenn ein Wandel von einer suffigieren Form (eine RR) zu einer Form ohne Markierung (keine RR) vorliegt: Es gibt einen Zeitpunkt, in dem beide Formen grammatisch sind, weshalb auch zwei RRs angenommen werden müssen, um die Formen in der Zelle zu definieren. Freie \isi{Variation} führt folglich immer zu einer höheren Komplexität, unabhängig davon, ob das Endprodukt des Wandels von einer Form A zu einer Form B höher oder niedriger in seiner Komplexität ist. Denn synchron stehen zwei (oder mehr) Formen in einer Zelle des Paradigmas, welche durch RRs definiert werden müssen.

\subsection{Realisierungsregeln für Wurzel-/Stammalternationen}\label{5.1.3}

In diesem Kapitel werden die RRs zu den Wur\-zel-/Stamm\-al\-ter\-na\-tio\-nen vorgestellt. Die in diesem Sample vorkommenden Wur\-zel-/Stamm\-al\-ter\-na\-tio\-nen können drei Typen zugeordnet werden: \is{Modifikation}Modifikation eines Vokals oder eines Konsonanten in der \isi{Wurzel}, Wurzelerweiterung (woraus Stämme entstehen) und \is{Subtraktion}Subtraktion. Den Defaultstamm bildet die \isi{Wurzel}, wobei es sich um die Form des Nominativs Singular handelt.

Zu den \is{Modifikation}\textsc{Modifikationen} der \isi{Wurzel} gehören drei Phänomene: \isi{Umlaut}, Diphthongierung und Velarisierung. In allen hier untersuchten Varietäten wird der \isi{Umlaut} zur Pluralmarkierung verwendet. In \sectref{4.1.3.2} wurde die RR für den \isi{Umlaut} eingeführt, die in \REF{ex:key:12} wiederholt ist:

% \ea%12
\begin{exe}
\exr{ex:key:12}
 RR \textsubscript{A,} \textsubscript{\{\textsc{num:pl}\},} \textsubscript{\textsc{n[}\textsc{ic:} 1} \textsubscript{\tiny $\veebar$}\textsubscript{ 3} \textsubscript{\tiny $\veebar$}\textsubscript{ 7} \textsubscript{\tiny $\veebar$}\textsubscript{ 8]} ($\langle$X,$\sigma$ $\rangle$) = \textsubscript{def} $\langle$Ẍˊ,$\sigma$ $\rangle$
\end{exe}

Dies ist die Regel für die deutsche Standardsprache, in der jeder Wurzelvokal genau einen \isi{Umlaut} hat. In vielen untersuchten Dialekten\footnote{Dies gilt für das Alemannische des Sensebezirks, von Uri, Zürich, Bern, Saulgau, Petrifeld, Elisabethtal, des Münstertals und des Elsass (Ebene).} wird jedoch ein Primär- und ein Sekundärumlaut zur Pluralmarkierung unterschieden, wobei der Primärumlaut nur vom Wurzelvokal \textit{a} gebildet wird. Das Alemannische von Zürich weist zum Primär- und Sekundärumlaut noch einen zweiten Sekundärumlaut auf (vgl. \tabref{table5.1}). Zum Wurzelvokal \textit{a} lautet der Primärumlaut \textit{e}, der erste Sekundärumlaut \textit{æ}, der zweite Sekundärumlaut \textit{ö}. Der zweite Sekundärumlaut kommt im Gegensatz zu den anderen nur in Langvokalen vor. Trotzdem muss auch dieser definiert werden, da lange Wurzelvokale auch den Primär- oder ersten Sekundärumlaut zeigen.

%{\tabref{table5.1}: \isi{Umlaut} im Alemannischen von Zürich (basierend auf \citealt[111-119]{Weber1987})}

\begin{table}
\caption{Umlaut im Alemannischen von Zürich (basierend auf \citealt[111-119]{Weber1987})}\label{table5.1}
\begin{tabular}{lll}
\lsptoprule
{Singular} & {Plural} & {Umlaut}\\
\midrule
gascht ‘Gast’ & gescht & Primärumlaut\\
schl\=ag ‘Schlag’ & schl\=eg & Primärumlaut\\
bank ‘Bank’ & bænk & Sekundärumlaut 1\\
rom\=an ‘Roman’ & romǣn & Sekundärumlaut 1\\
sal\=at ‘Salat’ & salȫt & Sekundärumlaut 2\\
\lspbottomrule
\end{tabular}
\end{table}

Da Primär- und Sekundärumlaut zur Pluralmarkierung synchron nicht mehr phonologisch erklärt werden können, ist die \isi{Variation} in der Morphologie zu verorten und folglich durch RRs auszudrücken. Für das Alemannische von Zürich werden also drei RRs für den \isi{Umlaut} benötigt:

\ea%49
\label{ex:key:49}
 RR \textsubscript{A,} \textsubscript{\{\textsc{num:pl}\},} \textsubscript{\textsc{n[}\textsc{ic:} 2} \textsubscript{\tiny $\veebar$}\textsubscript{ 5} \textsubscript{\tiny $\veebar$}\textsubscript{ 6]} ($\langle$X,$\sigma$ $\rangle$) = \textsubscript{def} $\langle$Ẍˊ,$\sigma$ $\rangle$
\z

\ea%50
\label{ex:key:50}
 RR \textsubscript{A,} \textsubscript{\{\textsc{num:pl}\},} \textsubscript{\textsc{n[}\textsc{ic:} 1} \textsubscript{\tiny $\veebar$}\textsubscript{ 4]} ($\langle$X,$\sigma$ $\rangle$) = \textsubscript{def} $\langle$Ẍ[\textit{a} $\rightarrow$ \textit{e}]ˊ,$\sigma$ $\rangle$
\z

\ea%51
\label{ex:key:51}
 RR \textsubscript{A,} \textsubscript{\{\textsc{num:pl}\},} \textsubscript{\textsc{n[}\textsc{ic:} 3]} ($\langle$X,$\sigma$ $\rangle$) = \textsubscript{def} $\langle$Ẍ[\textit{\=a} $\rightarrow$ \textit{ȫ}]ˊ,$\sigma$ $\rangle$\\
\z

Die RR \REF{ex:key:49} bildet den ersten Sekundärumlaut, den Default-\isi{Umlaut}, nach dem auch alle anderen Wurzelvokale umgelautet werden. Lautet der Wurzelvokal \textit{a}, werden noch zwei spezifischere Regeln gebraucht, nämlich eine für den Primärumlaut \REF{ex:key:50} und eine für den zweiten Sekundärumlaut \REF{ex:key:51}.

Ist einer bestimmten \isi{Flexionsklasse} eine RR zugeordnet, die einen \isi{Umlaut} bildet, werden alle Wörter dieser \isi{Flexionsklasse} umgelautet, wenn dies möglich ist. In der deutschen Standardsprache z.B. stehen alle Lexeme, die einen Plural auf -\textit{ər} bilden und umgelautet werden, in derselben \isi{Flexionsklasse} (\textit{Wald}-\textit{Wäldər}, \textit{Kind}-\textit{Kindər}). Wörter wie \textit{Kind} haben keine eigene \isi{Flexionsklasse}, denn Wörter, die den Plural auf -\textit{ər} bilden, werden immer umgelautet. Wird der \isi{Umlaut} auf \textit{Kind} angewendet, passiert mit dem Wurzelvokal nichts, denn \textit{i} kann nicht umgelautet werden. Im Gegensatz dazu sind z.B. im Alemannischen von Issime zwei \isi{Flexionsklassen} für den Plural auf -\textit{er} nötig, denn nicht alle Wörter, deren Wurzelvokal umlautbar wäre, werden auch umgelautet: \textit{lam}-\textit{lammer} ‘Lamm’, \textit{lan}-\textit{lenner} ‘Land, Dorf’ \citep[164]{Zürrer1999}. Deswegen sind für die Substantivflexion von Issime zwei \isi{Flexionsklassen} für den Plural auf -\textit{er} anzusetzen (vgl. \isi{Flexionsklassen} 10 und 11 in Paradigma 4).  

Schließlich ist noch zu entscheiden, ob es sich im Althochdeutschen um einen phonologisch oder morphologisch bedingten \isi{Umlaut} handelt. Dies betrifft die \isi{Flexionsklassen} 3 und 14 (\textit{gast}/\textit{gesti} und \textit{anst}/\textit{ensti}; i-Stäm\-me) sowie die \isi{Flexionsklasse} 9 (\textit{lamb}/\textit{lembir}; iz-/az-Stämme) (vgl. Paradigma 1 im Anhang A). Es wird hier davon ausgegangen, dass der \isi{Umlaut} sowohl phonologisch als auch bereits morphologisch bedingt ist. In den \isi{Flexionsklassen} 3 und 14 zeigen alle Formen im Plural einen \isi{Umlaut}, also nicht nur jene, die in der auf die \isi{Wurzel} folgenden Silbe ein \textit{i} aufweisen (z.B. \textit{enst}-\textit{i} Nominativ/Akkusativ Plural, \textit{enst}-\textit{in} Dativ Plural), sondern auch jene Formen, die kein nachfolgendes \textit{i} haben (z.B. \textit{enst}-\textit{o} Genitiv Plural). Ursprünglich stand auch im Genitiv Plural ein \textit{i} (\textit{enst}-\textit{io}), welches aber nur noch in der frühesten Phase des Althochdeutschen belegt ist \citep[201]{Braune2004}. Es empfiehlt sich hier jedoch nicht, Belege aus der frühesten Phase zu berücksichtigen, da es nur äußerst wenige Belege gibt und diese geringe Menge sich folglich für eine Gesamtanalyse nicht anbietet. Der \isi{Umlaut} im Genitiv Plural (\textit{enst}-\textit{o}) muss synchron also morphologisch bedingt sein. Zwei Analysen sind folglich möglich: a) Der \isi{Umlaut} markiert den Genitiv Plural, b) der \isi{Umlaut} markiert den gesamten Plural. Da Letzteres klar wahrscheinlicher ist als Ersteres (der \isi{Umlaut} setzt sich mehr und mehr als Pluralmarker durch), wird für den gesamten Plural ein morphologischer \isi{Umlaut} angenommen. Anders sieht dies im Singular der \isi{Flexionsklasse} 14 aus, für die von einem phonologisch bedingten \isi{Umlaut} ausgegangen wird: \textit{anst} (Nominativ/Akkusativ Singular), \textit{enst}-\textit{i} (Dativ/Genitiv Singular). Da im Singular nur dann ein \isi{Umlaut} auftritt, wenn in der nachfolgenden Silbe ein \textit{i} steht, und dieser auch später nicht morphologisiert wird, kann angenommen werden, dass es sich dabei auch im Althochdeutschen ausschließlich um einen phonologisch bedingten \isi{Umlaut} handelt. Folglich ist für das Althochdeutsche eine RR für den Pluralumlaut anzusetzen (morphologisch bedingt), nicht jedoch für den \isi{Umlaut} im Singular (phonologisch bedingt).

Damit sind die Ausführungen zum \isi{Umlaut} abgeschlossen. Zum Thema \is{Modifikation}Modifikation gehören noch die Diphthongierung und die Velarisierung, welche im Folgenden beschrieben werden.

Das Alemannische des Münstertals markiert den Plural u.a. durch Diphthongierung (vgl. Paradigma 18). Dies trifft nur auf nasalierte, lange Wurzelvokale zu: \textit{p\~{\=a}t} ‘Band’(Singular), \textit{pain} (Plural) \citep[43]{Mankel1886}. Der Diphthong wird durch folgende RR gebildet (die Subtraktion von \textit{t} wird weiter unten diskutiert, RR \REF{ex:key:59}), wobei < { ̑}> für Diphthongierung steht:

\ea%52
\label{ex:key:52}
 RR \textsubscript{A,} \textsubscript{\{\textsc{num:pl}\},} \textsubscript{\textsc{n[}\textsc{ic:} 7]} ($\langle$X,$\sigma$ $\rangle$) = \textsubscript{def} $\langle$X̑ˊ,$\sigma$ $\rangle$\\
\z

Schließlich gehört zu den \is{Modifikation}Modifikationen noch die Velarisierung. Im Alemannischen des Elsass (Ebene) wird in den \isi{Flexionsklassen} 6 (ohne \isi{Umlaut}) und 7 (mit \isi{Umlaut}) der auslautende Konsonant velarisiert\footnote{\citet{Beyer1963} spricht von Palatalisierung \citep[63]{Beyer1963}. Es ist jedoch anzunehmen, dass es sich dabei eher um velare Nasale handelt.} (vgl. Paradigma 20): \textit{hund}-\textit{hung} ‘Hund’, \textit{hand}-\textit{hæng} ‘Hand’ \citep[63]{Beyer1963}. Dies wird durch folgende RR ausgedrückt:

\ea%53
\label{ex:key:53}
 RR \textsubscript{B,} \textsubscript{\{\textsc{num:pl}\},} \textsubscript{\textsc{n[}\textsc{ic:} 6} \textsubscript{\tiny $\veebar$}\textsubscript{ 7]} ($\langle$X,$\sigma$ $\rangle$) = \textsubscript{def} $\langle$X *\textit{nd} $\rightarrow$ \textit{ŋ}ˊ,$\sigma$ $\rangle$ \\
\z

Der zweite Fall von Wur\-zel-/Stamm\-al\-ter\-na\-tion ist die \textsc{Wurzelerweiterung}. Es handelt sich hier zwar um \isi{Suffixe}. Da diese \isi{Suffixe} jedoch die \isi{Wurzel} erweitern (z.B. Pluralstamm), an die weitere \isi{Suffixe} angehängt werden können (z.B. Kasussuffixe), wird dieser Prozess Wurzelweiterung genannt und in diesem Kapitel zu den Wur\-zel-/Stamm\-mo\-di\-fi\-ka\-tio\-nen behandelt.

Die \isi{Flexionsklasse} 3 von Visperterminen weist einen Singular- und einen Pluralstamm auf, wobei der Pluralstamm durch die Suffigierung mit -\textit{m} entsteht: \textit{ar-o} ‘Arm’ (Nominativ Singular), \textit{arm-a} (Nominativ Plural) \citep[122]{Wipf1911}. Wir können hier von einer Wurzelerweiterung und einem Pluralstamm sprechen, da die \isi{Flexionsklasse} 3 im Plural dieselben Kasussuffixe aufweist wie die \isi{Flexionsklasse} 1: \textit{tag}-\textit{a}, \textit{ar}-\textit{m}-\textit{a} (Nominativ/Akkusativ Plural), \textit{tag}-\textit{u}, \textit{ar}-\textit{m}-\textit{u} (Dativ Plural), \textit{tag}-\textit{o}, \textit{ar}-\textit{m}-\textit{o} (Genitiv Plural). In der \isi{Flexionsklasse} 9 wird im Plural ein \textit{n} suffigiert, jedoch nur im Dativ Plural: \textit{hor-u} ‘Horn’ (Nominativ Plural), \textit{horn-u} (Dativ Plural) \citep[130]{Wipf1911}. \textit{Arm}- wird durch die RR \REF{ex:key:54} definiert, \textit{horn}- durch die RR \REF{ex:key:55}.

\ea%54
\label{ex:key:54}
 RR \textsubscript{A,} \textsubscript{\{\textsc{num:pl}\},} \textsubscript{\textsc{n[}\textsc{ic:} 3]} ($\langle$X,$\sigma$ $\rangle$) = \textsubscript{def} $\langle$X\textit{m}ˊ,$\sigma$ $\rangle$
\z

\ea%55
\label{ex:key:55}
 RR \textsubscript{A,} \textsubscript{\{\textsc{case:dat}, \textsc{num:pl}\},} \textsubscript{\textsc{n[}\textsc{ic:} 9]} ($\langle$X,$\sigma$ $\rangle$) = \textsubscript{def} $\langle$X\textit{n}ˊ,$\sigma$ $\rangle$
\z

Die RR \REF{ex:key:54} stellt also den Pluralstamm zur Verfügung (\textit{arm}-), an den weitere Kasusmarker suffigiert werden können (z.B. Nominativ Plural \textit{arm}-\textit{a}, Dativ Plural \textit{arm}-\textit{u}, vgl. Paradigma 5, \citealt[120]{Wipf1911}). Identisch sehen alle RRs für das Pluralsuffix aus, egal ob danach noch weitere Kasussuffixe angehängt werden oder nicht. Ob nach dem Pluralsuffix noch weitere Kasussuffixe folgen, ergibt sich automatisch daraus, ob nach dem \isi{Block} für das Pluralsuffix noch weitere \isi{Blöcke} für Kasussuffixe kommen. Einen etwas komplexeren Fall stellt das \textit{n} dar: Es kann sowohl ein Pluralsuffix sein (z.B. in Issime \textit{uav}-\textit{n}-\textit{a} ‘Ofen’, \citealt[164]{Zürrer1999}) als auch eingefügt werden, um einen Hiatus zu vermeiden (z.B. \textit{chötti}-\textit{n}-\textit{i} ‘Kette’, \citealt[164]{Zürrer1999}). In welchen Varietäten und \isi{Flexionsklassen} was zutrifft, wird in \sectref{5.1.4} dargestellt.

Im Althochdeutschen variiert der Stamm der Diminutiva, die drei Stämme aufweisen: a) Der Default-Stamm (=\isi{Wurzel}) endet auf ein \textit{\=i} (Nominativ und Akkusativ Singular, \textit{chindil\=i} ‘Kind’); b) folgt ein Monophthong, wird ein \textit{n} eingefügt (\textit{chindil\=in}-); c) folgt ein Diphthong, wird \textit{\=i} getilgt (\textit{chindil}-) (vgl. Paradigma 1; \citealt[187]{Braune2004}). Die Tilgung wird weiter unten diskutiert. Der n-Einschub tritt in den \isi{Kasus} Dativ und Genitiv Singular und Plural auf, in denen ein \isi{Suffix} an die \isi{Wurzel} angehängt wird, das mit einem Monophthong anlautet. Dadurch entsteht die Abfolge \textit{\=i}+Monophthong. Es könnte also angenommen werden, dass der n-Einschub der Hiattilgung dient. Erstens werden jedoch üblicherweise im Althochdeutschen \textit{h}, \textit{j} oder \textit{w} zur Hiattilgung verwendet \citep[49]{Armborst1979}. Zweitens ist diachron im Alemannischen \textit{n} im Auslaut weggefallen (\textit{chindil\=i}), im Fränkischen jedoch erhalten (\textit{chindil\=in}) \citep[187]{Braune2004}. Im Alemannischen des 9. Jh. ist diese \isi{Variation} also in der Morphologie zu verorten:

\ea%56
\label{ex:key:56}
 RR \textsubscript{D,} \textsubscript{\{\textsc{case:dat}} \textsubscript{\tiny $\veebar$}\textsubscript{ \GEN\},} \textsubscript{\textsc{n[}\textsc{ic:} 19]} ($\langle$X,$\sigma$ $\rangle$) = \textsubscript{def} $\langle$X\textit{n/}V\_Vˊ,$\sigma$ $\rangle$\\
\z

Die RR \REF{ex:key:56} definiert: Füge im Dativ und Genitiv Singular und Plural der \isi{Flexionsklasse} 19 intervokalisch ein \textit{n} ein. Dies zeigt, dass die Kontextbedingungen für den n-Einschub erst durch die Suffigierung entstehen. Diese RR ist folglich erst in \isi{Block} D anzusetzen, also nach den \isi{Blöcken} B und C, die RRs für Numerus- und Kasussuffixe enthalten (Diskussion zur Abfolge von RRs und \isi{Blöcken} vgl. \sectref{4.1.3.3}).

Drittens sind noch die \is{Subtraktion}\textsc{Subtraktionen} zu behandeln. Die \is{Subtraktion}Subtraktion betrifft sowohl Vokale als auch Konsonanten: \is{Subtraktion}Subtraktion der auslautenden Vokale im Plural des Althochdeutschen und in etlichen Dialekten, \is{Subtraktion}\textit{t}{}-Subtraktion im Alemannischen von Münstertal und die wa-/w\=o-Stäm\-me im Alt- und Mittelhochdeutschen.

Es wurde bereits dargestellt, dass die Diminutiva des Althochdeutschen einen Default-Stamm (=\isi{Wurzel}) auf \textit{\=i} aufweisen (z.B. \textit{chindil\=i}). Dieses \textit{\=i} wird getilgt, wenn ein Diphthong folgt, was im Nominativ und Akkusativ Plural geschieht (z.B. \textit{chindil\=i}-\textit{iu} $\rightarrow$ \textit{chindil}-\textit{iu}). Da auch hier der Kontext zur Tilgung erst gegeben ist, nachdem suffigiert wurde, steht diese RR in \isi{Block} D:

\ea%57
\label{ex:key:57}
 RR \textsubscript{D,} \textsubscript{\{\textsc{case:nom}} \textsubscript{\tiny $\veebar$}\textsubscript{ \textsc{acc}, \textsc{num:pl}\},} \textsubscript{\textsc{n[}\textsc{ic:} 19]} ($\langle$X,$\sigma$ $\rangle$) = \textsubscript{def} $\langle$X \textit{*\=\i} $\rightarrow$ ø\textit{/}\_VVˊ,$\sigma$ $\rangle$\\
\z

Ein ähnlicher Fall der Vokalsubtraktion kommt in vielen alemannischen Dialekten vor. In diesen Dialekten ist im Plural ein wiederkehrendes Muster zu beobachten: An eine \isi{Wurzel}, die auf einen Vokal auslautet, wird ein vokalisches Pluralsuffix angehängt. In solchen Fällen ist in den alemannischen Dialekten zu erwarten, dass ein \textit{n} zur Hiattilgung eingefügt wird, was nicht nur innerhalb eines Wortes, sondern auch über die Wortgrenze hinweg üblich ist, wie dieses Beispiel zeigt:

\ea%58
    \label{ex:key:58}
\gll \textit{W\=are} \hspace{2em} \textit{n} \hspace{2em} \textit{er}\\
wart {} Hiattustilgung {} ihr\\

\citep[321]{Noth1993}
\z

Die Default-Strategie der Phonologie zur Hiat-Tilgung ist also der n-Einschub. Dasselbe wäre also zu erwarten, wenn einer vokalisch auslautenden \isi{Wurzel} ein vokalisches \isi{Suffix} angehängt wird. Im Alemannischen von Huzenbach z.B. enden die Diminutiva auf -\textit{le}, denen ein -\textit{ə} im Plural suffigiert wird. Der Plural lautet aber nicht *\textit{heislenə}, sondern \textit{heislə} ‘Häuschen’ \citep[98]{Baur1967}. Der auslautende Vokal der \isi{Wurzel} muss also im Plural getilgt werden, wenn ein -\textit{ə} folgt. Dies wird durch folgende RR ausgedrückt, die bereits in \sectref{4.1.3.3} eingeführt wurde:

% \ea%44
\begin{exe}
\exr{ex:key:44}
 RR \textsubscript{C, \{\textsc{num:pl}\}, \textsc{n[}\textsc{ic:} 4} \textsubscript{\tiny $\veebar$} \textsubscript{5]} ($\langle$X,$\sigma$ $\rangle$) = \textsubscript{def} $\langle$X *\textit{e} $\rightarrow$ ø/\_əˊ,$\sigma$ $\rangle$
\end{exe}

Auch hier entstehen die Kontextbedingungen erst durch die Suffigierung, weshalb die RR für die Wur\-zel-/Stamm\-al\-ter\-na\-tion in \isi{Block} C steht und die RR der Suffigierung in \isi{Block} B. RRs von diesem Typ gibt es in den folgenden alemannischen Dialekten: Elisabethtal, Kaiserstuhl, Saulgau, Huzenbach, Sensebezirk, Stuttgart, Uri und Petrifeld. Viele alemannische Dialekte weisen diese Wurzelalternation auch im Zusammenhang mit dem Pluralsuffixen des Typs -\textit{ənə} auf, welche im nächsten Kapitel erörtert werden.

Im Gegensatz zu vielen alemannischen Dialekten hat die deutsche Standardsprache keine hiatvermeidende n-Epenthese, noch braucht sie eine RR wie in \REF{ex:key:44}. Auf den ersten Blick suggerieren aber Wörter, deren Singular auf \textit{o} oder \textit{a} auslautet, etwas anderes: \textit{Konto}, \textit{Pizza} (Nominativ Singular), \textit{Kont}-\textit{ən}, \textit{Pizz}-\textit{ən} (Nominativ Plural). Die Tilgung von \textit{o} im Plural wird jedoch von der Phonologie verursacht. Wird an \textit{Konto} die Endung -\textit{ən} suffigiert, können prinzipiell folgende Möglichkeiten entstehen: a) [kon.'to.ən], b) ['kon.to.ən]. In der deutschen Standardsprache wird in einem Simplex die Pänultima akzentuiert, sofern sie „einen akzentuierfähigen Vokal - also nicht /ə/ - enthält, der auch nicht im \isi{Hiat} mit einem folgenden Vokal steht. In diesen letzteren Fällen trifft der Akzent auf die Antepänultima“ \citep[186]{Kohler1995}. Es ist also zuerst von b) mit dem Akzent auf der Antepänultima auszugehen. Des Weiteren wirkt die Regel, dass \isi{Substantive} im Plural (außer -\textit{s}) auf einen zweisilbigen Fuß enden \citep[ 61-62, 106-109]{Wiese1996}: „[…] [W]ith the exception of nouns taking +\textit{s} as the plural marker, nouns in the plural are such that the last syllable must be a schwa syllable, while the preceding syllable is stressed“ \citep[61]{Wiese1996}. Wie beispielsweise *\textit{Schwestərən} kein zulässiger Plural ist, sondern \textit{Schwestərn}, muss auch in *\textit{Kontoən} eine Silbe gekürzt werden. Diese Analyse wirft viele Fragen auf, z.B. weshalb \textit{o} und nicht \textit{ə} getilgt wird, wozu keine detaillierte Analyse gefunden werden konnte. Relevant für diese Arbeit ist jedoch nur, ob es sich dabei um einen morphologischen oder phonologischen Mechanismus handelt. Wie ausgeführt wurde, gibt es klare Indizien dafür, dass die Tilgung phonologisch bzw. phonotaktisch bedingt ist, weshalb dafür keine RR angenommen werden muss.

In einem alemannischen Dialekt ist auch Konsonantensubtraktion zu beobachten. Im Alemannischen des Münstertals bilden Wörter mit einem langen, nasalierten Wurzelvokal den Plural durch Diphthongierung des Wurzelvokals (RR (\ref{ex:key:52})): \textit{p\~{\=a}t} ‘Band’ (Singular), \textit{pain} (Plural) \citep[43]{Mankel1886}. Zusätzlich wird der Plural durch die \is{Subtraktion}Subtraktion des auslautenden \textit{t} markiert. Folgende RR definiert, dass im Plural der \isi{Flexionsklasse} 7 das auslautende \textit{t} getilgt wird:

\ea%59
\label{ex:key:59}
 RR \textsubscript{C,} \textsubscript{\{\textsc{num:pl}\},} \textsubscript{\textsc{n[}\textsc{ic:} 7]} ($\langle$X,$\sigma$ $\rangle$) = \textsubscript{def} $\langle$X *\textit{t} $\rightarrow$ ø/\_\#ˊ,$\sigma$ $\rangle$\\
\z

Schließlich sind in diesem Kontext noch die alt- und mittelhochdeutschen wa-/w\=o-Stäm\-me zu diskutieren (vgl. \tabref{table5.2}). In beiden Varietäten weisen die wa-/w\=o-Stäm\-me dieselben \isi{Suffixe} wie die a-Stäm\-me auf, folglich ist für beide Stämme nur eine \isi{Flexionsklasse} anzunehmen. Der einzige Unterschied in beiden Varietäten besteht darin, dass die \isi{Wurzel} auf \textit{w} endet, wenn ein \isi{Suffix} folgt. Im Auslaut wird das \textit{w} im Althochdeutschen vokalisiert, im Mittelhochdeutschen getilgt. Die \isi{Variation} kann im Althochdeutschen also phonologisch erklärt werden, im Mittelhochdeutschen jedoch nicht. Deshalb braucht es für das Mittelhochdeutsche eine RR, die diese Tilgung durchführt. Im Althochdeutschen steuert die Phonologie diese \isi{Variation}, folglich ist keine RR nötig. In der Folge soll nun genauer darauf eingegangen werden, weshalb die \isi{Variation} im Althochdeutschen phonologisch und im Mittelhochdeutschen morphologisch bedingt ist.

%{\tabref{table5.2}: Wa-/w\=o-Stäm\-me im Alt- und Mittelhochdeutschen (\citealt[193]{Braune2004}, \citealt[143, 189]{Paul2007}}\\

\begin{table}
\caption{Wa-/w\=o-Stämme im Alt- und Mittelhochdeutschen (\citealt[193]{Braune2004}, \citealt[143, 189]{Paul2007})}\label{table5.2}
\begin{tabular}{llll}
\lsptoprule
\multicolumn{2}{c}{{Althochdeutsch}} & \multicolumn{2}{c}{{Mittelhochdeutsch}}\\\cmidrule(lr){1-2}\cmidrule(lr){3-4}
{\textsc{nom}.\textsc{sg}.} & {\textsc{gen.sg.}} & \textsc{nom.sg.} & \textsc{gen.sg.}\\
\midrule
sn\=eo & sn\=ew-es & sn\=e & sn\=ew-es\\
horo & horaw-es & hor & horw-es\\
tou & touw-es & tou & touw-es\\
\lspbottomrule
\end{tabular}
\end{table}

Im Althochdeutschen ist von einer \isi{Wurzel} auf -\textit{w} auszugehen. Steht kein \isi{Suffix} (wie z.B. im Nominativ Singular), tritt das \textit{w} in den Auslaut und wird zu \textit{o} vokalisiert. Diese phonologische Regel dauert im Althochdeutschen fort \citep[68]{VoylesBarrack2014}. Auf der einen Seite weisen zwar auch andere germanische Sprachen ähnliche Prozesse auf, was dafür sprechen würde, dass es sich dabei um ein gemeingermanisches Phänomen handelt. Die Varianten sind aber nicht in allen germanischen Varietäten gleich verteilt, was bedeutet, dass diese Regel in unterschiedlichen Varianten in den einzelnen Sprachen fortdauert. Beispielsweise weist das Nordgermanische dieselbe Verteilung wie das Althochdeutsche auf, im Gotischen und Altenglischen jedoch ist das \textit{w} im Auslaut nach Langvokal erhalten, z.B. Altenglisch \textit{sn\=aw} ‘Schnee’ (Nominativ Singular), \textit{sn\=awes} (Genitiv Singular) \citep[18--19]{KraheMeid1967}.

Weiter stellt sich für das Althochdeutsche die Frage, weshalb aus \textit{w} ein \textit{o} entstanden ist. Dass auslautende Konsonanten zu \textit{o} vokalisiert werden, ist ein durchaus bekanntes Phänomen. Beispielsweise wird im Kroatischen auslautendes \textit{l} zu \textit{o} vokalisiert \citep[84]{Holzer2007}. Für das Althochdeutsche kann folgender Wandel postuliert werden: \textit{w} > \textit{u} > \textit{o}. Dafür spricht, dass alle anderen germanischen Sprachen in diesen Positionen ein \textit{u} aufweisen, das Altsächsische \textit{u} und \textit{o} \citep[18--19]{KraheMeid1967}. Zusätzlich sind hier zwei weitere Wandelerscheinungen wichtig, die wohl parallel abgelaufen sind. Germanisch *\textit{au} vor \textit{h} und besonders vor Dentalen ist zu Althochdeutsch \textit{\=o} geworden, und zwar über die (wenn auch spärlich) belegte Zwischenstufe \textit{ao} \citep[47]{Braune2004}. Parallel dazu läuft der Wandel von auslautendem \textit{ao} zu \textit{\=o}, wobei \textit{ao} aus \textit{aw} entstanden ist \citep[48]{Braune2004}: \textit{fraw\=er}/\textit{frao} > \textit{fr\=o} \citep[112]{Braune2004}. Folglich verhält sich *\textit{au}C wie *\textit{aw}. Schließlich können die Fälle, in denen durch die Vokalisierung \textit{eu} entsteht (*\textit{sn\=ew} > *\textit{sn\=eu} > \textit{sn\=eo}), dadurch erklärt werden, dass im Westgermanischen \textit{eu} im Auslaut zu \textit{eo} wird \citep[68, 167]{VoylesBarrack2014}.

Geklärt werden muss noch, woher das \textit{a} in \textit{horawes} ‘Schmutz’ und der Diphthong in \textit{tou} ‘Tau’ stammen. Beim \textit{a} in \textit{horawes} handelt es sich um einen Sprossvokal, der zwischen \textit{r}+\textit{h}, \textit{r}+\textit{l}, \textit{r}+\textit{w}, \textit{l}+\textit{w} und \textit{s}+\textit{w} eingefügt wird \citep[71]{Braune2004}. Anstelle von \textit{a} können auch \textit{o} oder \textit{e} stehen, wobei die Wahl des Sprossvokals ebenfalls phonologisch bedingt ist: „Der entstehende Vokal erscheint als \textit{a} oder (bes. vor \textit{w}) als \textit{o}, nimmt aber häufig auch die Form eines nebenstehenden Vokals an, wobei in der Regel die Endsilbenvokale, seltener die Stammsilbenvokale, maßgebend sind […]“ \citep[71]{Braune2004}. Auch für den Einschub und die \isi{Variation} des Sprossvokals ist also die Phonologie verantwortlich. Der diphthongische Auslaut von \textit{tou} geht hingegen auf ein geminiertes \textit{w} zurück \citep[109]{Braune2004}. Dieses geminierte \textit{w} wird im Auslaut vereinfacht und „es bleibt also nur der erste Teil des \textit{ww}, der am Silbenschluss einen Diphthong bildet […]“ \citep[111-112]{Braune2004}. Auch dafür gibt es folglich eine phonologische Erklärung, weshalb es in der Morphologie nicht berücksichtigt werden muss.

Im Mittelhochdeutschen ist die \isi{Variation} nicht phonologisch bedingt, denn im Gegensatz zum Althochdeutschen wird im Mittelhochdeutschen ein Segment (d.h. \textit{w}) getilgt und nicht vokalisiert. Aus synchroner Sicht ist diese Tilgung phonologisch nicht voraussagbar, weil auslautendes \textit{w} im Mittelhochdeutschen meistens, aber nicht immer getilgt wird \citep[123]{VoylesBarrack2014}. Geht man vom umgekehrten Fall, d.h., von einer \isi{Wurzel} ohne \textit{w} aus, gibt es ebenfalls keine phonologische Erklärung. Denn \textit{w} steht in den unterschiedlichsten phonologischen Kontexten: intervokalisch wie auch nach verschiedenen Konsonanten. Folglich kann nicht vorausgesagt werden, in welchen phonologischen Kontexten ein \textit{w} eingesetzt werden muss. Diese Gründe sprechen dafür, dass es sich dabei um eine Wurzelalternation handelt, die durch eine RR ausgedrückt wird, welche an die entsprechenden \isi{Flexionsklassen} gebunden ist:

\ea%60
\label{ex:key:60}
 RR \textsubscript{D,} \textsubscript{\{\},} \textsubscript{\textsc{n[}\textsc{ic:} 1} \textsubscript{\tiny $\veebar$}\textsubscript{ 3]} ($\langle$X,$\sigma$ $\rangle$) = \textsubscript{def} $\langle$X *\textit{w} $\rightarrow$ ø/\_\#ˊ,$\sigma$ $\rangle$\\
\z

Diese RR definiert, dass in den \isi{Flexionsklassen} 1 und 3 \textit{w} getilgt wird, sobald es in den Auslaut tritt. Es wird also von einer \isi{Wurzel} auf \textit{w} ausgegangen. Obwohl in der \isi{Flexionsklasse} 1 \textit{w} im Dativ und Genitiv Singular und im gesamten Plural getilgt wird, in der \isi{Flexionsklasse} 3 jedoch nur im Dativ und Genitiv Singular und Plural, kann diese RR für beide \isi{Flexionsklassen} verwendet werden. Denn die Präsenz oder Absenz von \textit{w} ist davon abhängig, ob \textit{w} am Wortende oder im Wortinneren steht, aber nicht abhängig von den morphosyntaktischen Eigenschaften.

In den hier untersuchten alemannischen Dialekten sind wa-/w\=o-Stäm\-me nur in Issime erhalten (vgl. Paradigma 4, \isi{Flexionsklasse} 8; \citealt[163]{Zürrer1999}). Obwohl das \textit{w} in Issime nur intervokalisch auftritt, handelt es sich dabei nicht um ein Element zur Hiatvermeidung. Wie bereits oben dargestellt wurde, wird in den alemannischen Dialekten ein \textit{n} eingefügt, um einen \isi{Hiat} zu vermeiden. Da das \textit{w} in Issime ausschließlich im Plural erscheint, kann es synchron folglich als Pluralmarker analysiert werden. Es wird also durch eine RR suffigiert, wie diese in \REF{ex:key:54} vorgestellt wurde.

\subsection{{Suffixe des Typs -\textit{ənə} }{und \textit{n}} {zur Hiatvermeidung}}\label{5.1.4}

In diesem Kapitel werden drei mit \textit{n} in Zusammenhang stehende Phänomene diskutiert. Erstens ist \textit{n} nur in Issime ein Pluralmarker. Zweitens wird \textit{n} in der Substantivflexion vieler Dialekte zur Vermeidung eines \isi{Hiats} verwendet, was in einigen Dialekten mit einer Schwächung der Mittelsilbe verbunden ist. Beides ist phonologisch bedingt, weshalb es keine RRs dafür braucht. Drittens weisen fünf Dialekte eine Pluralendung des Typs -\textit{ənə} auf (-\textit{ənə}, -\textit{əni}, -\textit{inə} etc.) und vier davon verwenden \textit{n} zur Hiatvermeidung. Bei diesen vier Dialekten ist also zusätzlich zu zeigen, weshalb sowohl ein \textit{n} zur Hiatvermeidung als auch eine Pluralendung des Typs -\textit{ənə} anzunehmen ist. Wie für alle anderen Pluralsuffixe ist auch für die \isi{Suffixe} des Typs -\textit{ənə} eine RR anzusetzen. In drei der fünf Dialekte, die ein Pluralsuffix des Typs -\textit{ənə} aufweisen, wird eine Silbe reduziert, was jedoch phonotaktisch bedingt ist (also keine RR). Punkt zwei und drei werden zusammen erörtert, indem jedes betroffene Paradigma einzeln analysiert wird.

\begin{description}
 \item[\textit{N} als Pluralmarker:]
Wie in vielen anderen Dialekten dient auch in Issime \textit{n} zur Hiatvermeidung. Gleichzeitig weist Issime aber auch ein -\textit{n} als Pluralsuffix auf: \textit{uav}-\textit{e} ‘Ofen’ (Nominativ Singular), \textit{uav}-\textit{n}-\textit{a} (Nominativ Plural), \textit{uav}-\textit{n}-\textit{e} (Dativ Plural) (vgl. \isi{Flexionsklasse} 2 in Paradigma 4, \citealt[164]{Zürrer1999}). Hier wird mit dem \textit{n} klar kein \isi{Hiat} vermieden (es steht nicht intervokalisch), sondern ein Plural markiert, was besonders der Vergleich des Nominativ Singular mit dem Dativ Plural zeigt. Ähnliches kann für den Dialekt von Jaun beobachtet werden: \textit{h\=ar} ʻHaarʼ (Nom/Akk.Sg.), \textit{h\=ar}{}-\textit{ən}{}-\textit{i} (Nom/Akk.Pl.) (vgl. Paradigma 6).

\item[N-Einschub zur Hiatvermeidung und Pluralendung des Typs \textit{-ənə}:] Nun werden folgende Phänomene zusammen betrachtet: n-Einschub zur Hiatvermeidung, die daraus resultierende Zentralisierung des Mittelsilbenvokals (\textit{blatti} > *\textit{blatti}-\textit{n}-\textit{i} > \textit{blattə}-\textit{n}-\textit{i}, ‘Blatt’, Jaun, \citealt[267]{Stucki1917}); Plural des Typs -\textit{ənə}, dadurch verursachte Tilgung des auslautenden Wurzelvokals aufgrund\linebreak phonotaktischer Restriktionen (\textit{wiərde} > * \textit{wiərde}-\textit{ənə} > \textit{wiərd}-\textit{ənə}, ‘Wirtin’, Huzenbach, \citealt[97]{Baur1967}). Die folgenden Tabellen \tabref{table5.3}, \ref{table5.4} und \ref{table5.5} geben eine Übersicht darüber, in welchen Dialekten welche Phänomene zusammen auftreten. Es werden zuerst die Dialekte der \tabref{table5.3} besprochen (von links nach rechts), welche ein \textit{n} zur Hiatvermeidung und eine damit einhergehende Zentralisierung der Mittelsilbe aufweisen, und dann jene der \tabref{table5.4}, deren Mittelsilbe nicht zentralisiert wird. Schließlich wird auf den Dialekt von Huzenbach eingegangen (\tabref{table5.5}), der als einziger in der Substantivflexion kein \textit{n} zur Hiatvermeidung aufweist, aber eine Pluralendung des Typs -\textit{ənə}.
\end{description}

%{\tabref{table5.3}: Dialekte mit \textit{n} (Hiatvermeidung) und Zentralisierung der Mittelsilbe}\\

\begin{table}[p]
\caption{Dialekte mit \textit{n} (Hiatvermeidung) und Zentralisierung der Mittelsilbe}\label{table5.3}
\resizebox{\textwidth}{!}{\begin{tabular}{p{3cm}*{6}{l}}
\lsptoprule
 & Jaun & Sensebezirk & Uri & Vorarlberg & Saulgau & Stuttgart\\\midrule
n (Hiatus) & + & + & + & + & + & +\\
Mittelsilbenzen\-tralisierung & + & + & + & + & + & +\\
Plural Typ -\textit{ənə} & - & + & + & - & - & -\\
Phonotaktisch \mbox{bedingte Tilgung} einer Silbe & - & - & - & - & - & -\\
\lspbottomrule
\end{tabular}}
\end{table}

%{\tabref{table5.4}: Dialekte mit \textit{n}} {(Hiatvermeidung), aber ohne Zentralisierung der Mittelsilbe}\\

\begin{table}[p]
\caption{Dialekte mit \textit{n} (Hiatvermeidung), aber ohne Zentralisierung der Mittelsilbe}\label{table5.4}
\resizebox{\textwidth}{!}{\begin{tabular}{p{3cm}*{6}{l}}
\lsptoprule
 & Issime & Visperterminen & Bern & Petrifeld & Elisabethtal & Kaiserstuhl\\\midrule
n (Hiatus) & + & + & + & + & + & +\\
Mittelsilbenzen\-tralisierung & - & - & - & - & - & -\\
Plural Typ -\textit{ənə} & - & - & - & + & - & +\\
Phonotaktisch \mbox{bedingte Tilgung} einer Silbe & - & - & - & + & - & +\\
\lspbottomrule
\end{tabular}}
\end{table}

%{\tabref{table5.5}: Dialekte ohne \textit{n}} {(Hiatvermeidung)}\\

\begin{table}[p]
\caption{Dialekt ohne \textit{n} (Hiatvermeidung)}\label{table5.5}
\begin{tabularx}{\textwidth}{Xl}
\lsptoprule
 & Huzenbach\\\midrule
n (Hiatus) & -\\
Mittelsilbenzen\-tralisierung & -\\
Plural Typ -\textit{ənə} & +\\
Phonotaktisch bedingte Tilgung einer Silbe & +\\
\lspbottomrule
\end{tabularx}
\end{table}

% Diese drei Tabellen sollten alle zusammen auf einer Querformat-Seite stehen.

In der Substantivflexion von \textbf{Jaun} (Paradigma 6) gehören Maskulina, Neutra und Feminina, deren \isi{Wurzel} auf \textit{i} endet, zu den \isi{Flexionsklassen} 11 (Maskulina und Neutra) bzw. 12 (Feminina). Ihre Flexionsendungen entsprechen denen der \isi{Flexionsklassen} 11 und 12 (vgl. \tabref{table5.6}). Da ihre \isi{Wurzel} jedoch vokalisch auslautet, muss ein \textit{n} eingeschoben werden, wenn das \isi{Suffix} vokalisch anlautet. Außerdem wird schwachtoniges \textit{i} zu \textit{ə}, wenn es im Wortinlaut steht \citep[159-164]{Stucki1917}.

%{\tabref{table5.6}: \textit{N}} {zur Hiatvermeidung in Jaun (basierend auf \citealt[255-272]{Stucki1917}}\\

\begin{table}
\caption{\textit{N} zur Hiatvermeidung in Jaun (basierend auf \citealt[255-272]{Stucki1917})}\label{table5.6}
\resizebox{\textwidth}{!}{\begin{tabular}{lllllll}
\lsptoprule
{\textsc{fk}} & \multicolumn{2}{c}{{\textsc{singular}}}  & \multicolumn{4}{c}{{\textsc{plural}}} \\\cmidrule(lr){2-3}\cmidrule(lr){4-7}
& {\NOM/\AKK/\DAT} & {\GEN} & {\NOM} & {\AKK} & {\DAT} & {\GEN}\\
\midrule
11 & bet ‘Bett’  & bet-s & bet-i & bet-i & bet-ə & bet-ə\\
& blatti ‘Teller’  & blatti-s & blattə-n-i & blattə-n-i & blattə-n-ə & blattə-n-ə\\
\midrule
12 & fr\=ag ‘Frage’  & fr\=ag & fr\=ag-i & fr\=ag-i & fr\=ag-ə & fr\=ag-ə\\
& schieri ‘Schere’ & schieri & schierə-n-i & schierə-n-i & schierə-n-ə & schierə-n-ə\\
\lspbottomrule
\end{tabular}}
\end{table}

Im Gegensatz dazu bilden \textit{h\=ar}/\textit{h\=ar}-\textit{ən}-\textit{i} ‘Haar’ (\isi{Flexionsklasse} 4), \textit{tǖr}/\textit{tǖr}-\textit{ən}-\textit{i} ‘Tür’ (\isi{Flexionsklasse} 14),\largerpage[2]  \textit{matt}-\textit{a}/\textit{matt}-\textit{ən}-\textit{i} ‘Wiese’ (\isi{Flexionsklasse} 16) eigene \isi{Flexionsklassen}. Zwar weisen auch sie dieselben Kasusendungen wie die \isi{Flexionsklassen} 11 und 12 auf: -\textit{s} (Genitiv Singular, nur Maskulina und Neutra), -\textit{i} (Nominativ und Akkusativ Plural), -\textit{ə} (Dativ und Genitiv Plural). Zusätzlich wird aber der Plural durch das \isi{Suffix} -\textit{ən} markiert, was auf die \isi{Flexionsklassen} 11 und 12 nicht zutrifft. Für die \isi{Flexionsklassen} 4, 14 und 16 wäre auch ein Plural auf -\textit{əni} (Nominativ und Akkusativ) und -\textit{ənə} (Dativ und Genitiv) denkbar. Bei dieser Analyse müssten aber zwei neue \isi{Suffixe} angenommen werden (-\textit{əni}, -\textit{ənə}), da sie in keiner anderen \isi{Flexionsklasse} vorkommen. Bei der Analyse mit -\textit{ən} als Pluralsuffix braucht es zusätzlich nur dieses \isi{Suffix}, denn die \isi{Suffixe} -\textit{i} und -\textit{ə} existieren bereits in anderen \isi{Flexionsklassen}. Die Analyse mit -\textit{ən} als Pluralsuffix benötigt also eine RR weniger als die Analyse mit -\textit{əni}/-\textit{ənə}, weshalb sie zu bevorzugen ist.

Im Alemannischen des \textbf{Sensebezirks} verhalten sich die \isi{Substantive}, deren\linebreak \isi{Wurzel} auf \textit{i} auslautet, ähnlich wie jene in Jaun. Die Feminina der auf \textit{i} auslautenden \isi{Substantive} bilden den Plural auf -\textit{ə} (\isi{Flexionsklasse} 3), die Maskulina und Neutra auf -\textit{i} (\isi{Flexionsklasse} 7) (vgl. Paradigma 7): \textit{schwechi}/\textit{schwechə}-\textit{n}-\textit{ə} ‘Schwäche’, \textit{blatti}/\textit{blattə}-\textit{n}-\textit{i} ‘Teller’ \citet[188, 186]{Henzen1927}. \isi{Kasus} wird in diesem Dialekt am \isi{Substantiv} nicht unterschieden. Auch hier wird also an eine \isi{Wurzel}, die auf einen Vokal endet, ein vokalisches Pluralsuffix angehängt, folglich muss ein \textit{n} eingeschoben werden. Wie in Jaun wird auch hier das \textit{i} zu \textit{ə} gesenkt, wenn es im Wortinlaut steht \citep[124]{Henzen1927}. Des Weiteren existiert aber auch das Pluralsuffix -\textit{əni}, das also eine eigene \isi{Flexionsklasse} bildet: \textit{nɛts}/\textit{nɛts}-\textit{əni} ‘Netz’. Da die \isi{Wurzeln} der \isi{Substantive} dieser \isi{Flexionsklasse} konsonantisch auslauten, ist \textit{ə} kein gesenktes \textit{i} und \textit{n} kein Einschub zur Hiatvermeidung.

Im Gegensatz zu den Dialekten von Jaun und des Sensebezirks unterscheidet sich in \textbf{Uri} die Flexion der \isi{Substantiven}, die auf \textit{i} auslauten, in den drei \isi{Genera}. Die Maskulina (\textit{briali} ‘schreiende Person’) gehören zur \isi{Flexionsklasse} 4, die Neutra (\textit{bekchi} ‘Becken’) zur \isi{Flexionsklasse} 11 und die Feminina (\textit{schn\=idəri} ‘Schneiderin’) bilden eine eigene \isi{Flexionsklasse} \REF{ex:key:7} (vgl. Paradigma 8 zusammengefasst in \tabref{table5.7}). Die Maskulina haben dieselben Flexionsendungen wie in \isi{Flexionsklasse} 4. Per Default wird bei einem Hiatus ein \textit{n} eingeschoben, was eine Senkung des \textit{i} zu \textit{ə} als Konsequenz hat \citep[106]{Clauß1929}. Die Feminina bilden eine eigene \isi{Flexionsklasse}. Zwar weisen sie dieselben Flexionsendungen auf wie die \isi{Flexionsklasse} 4; würden aber die Feminina in die \isi{Flexionsklasse} 4 eingeteilt, dann entstünde im Plural *\textit{schn\=idərə}-\textit{n}-\textit{a}, da von der Phonologie per Default ein \textit{n} eingeschoben wird. In den Feminina wird das auslautende \textit{i} der \isi{Wurzel} aber nicht reduziert, sondern getilgt. Da es sich dabei um keinen Prozess handelt, der im gesamten System stattfindet (im Gegensatz zum n-Einschub), ist die Tilgung in der Morphologie zu verorten und folglich durch eine RR zu definieren, wie diese in \sectref{5.1.3} (Subtraktion) vorgestellt wurden:

\ea%61
\label{ex:key:61}
 RR \textsubscript{D,} \textsubscript{\{\textsc{num:pl}\},} \textsubscript{\textsc{n[}\textsc{ic:} 7} \textsubscript{\tiny $\veebar$}\textsubscript{ 11]} ($\langle$X,$\sigma$ $\rangle$) = \textsubscript{def} $\langle$X *\textit{i} $\rightarrow$ ø/\_Vˊ,$\sigma$ $\rangle$\\
\z

Diese RR gilt nur für die Feminina, nicht für die Maskulina. Wären die Feminina und Maskulina in der \isi{Flexionsklasse} 4 und die RR \REF{ex:key:61} der \isi{Flexionsklasse} 4 zugeordnet, würde diese RR auch auf die Maskulina angewendet, wodurch *\textit{brial}-\textit{ɐ} entstehen würde. Die kürzeste Beschreibung dieses Systems, die alle Formen korrekt definiert, erreicht man also, wenn die Maskulina zur \isi{Flexionsklasse} 4 gehören und die Feminina eine eigene \isi{Flexionsklasse} bilden.

%{\tabref{table5.7}: \textit{N}} {(Hiatvermeidung) und \isi{Suffix} -\textit{ənə}} {in Uri (basierend auf Clauß 1929: 173-185})\\

\begin{table}
\caption{\textit{N} (Hiatvermeidung) und Suffix -\textit{ənə} in Uri (basierend auf \citealt[173-185]{Clauß1929})}\label{table5.7}
\begin{tabular}{llll}
\lsptoprule
{\textsc{fk}} & {\textsc{singular}} & \multicolumn{2}{c}{{\textsc{plural}}} \\\cmidrule(lr){3-4}
&  & {\NOM/\AKK} & {\DAT}\\
\midrule
4 & chnacht ‘Knecht’ & chnacht-ɐ & chnacht-ɐ\\
& \mbox{briali ‘schreiende Person’} & brialə-n-ɐ & brialə-n-ɐ\\
\midrule
7 & \mbox{schn\=idəri ‘Schneiderin’} & schn\=idər-ɐ & schn\=idər-ɐ\\
\midrule
11 & bet ‘Bett’ & bet-i & bet-ənɐ\\
& bekchi ‘Becken’ & bekch-i & bekch-ənɐ\\
\midrule
12 & nets ‘Netz’ & nets-i & nets-ɐ\\
\lspbottomrule
\end{tabular}
\end{table}

Die Zuordnung der Neutra zur \isi{Flexionsklasse} 11 hängt auch damit zusammen, dass die \isi{Flexionsklassen} 11 und 12 unterschieden werden. Um den Fall besser zu verstehen, wird nun zuerst die Analyse eingeführt, die hier verwendet wird. Da eine alternative Analyse denkbar ist, aber nur auf den ersten Blick ökonomischer erscheint, wird auch diese vorgestellt und es wird gezeigt, weshalb sie zu verwerfen ist. Für die hier eingeführte Analyse der \isi{Flexionsklassen} 11 und 12 werden folgende RR gebraucht (RR \REF{ex:key:61} bereits oben, hier wiederholt):

\ea%62
\label{ex:key:62}
 RR \textsubscript{C,} \textsubscript{\{\textsc{case:nom}} \textsubscript{\tiny $\veebar$}\textsubscript{ \AKK, \textsc{num:pl}\},} \textsubscript{\textsc{n[}\textsc{ic:} 11} \textsubscript{\tiny $\veebar$}\textsubscript{ 12]} ($\langle$X,$\sigma$ $\rangle$) = \textsubscript{def} $\langle$X\textit{i}ˊ,$\sigma$ $\rangle$
\z

\ea%63
\label{ex:key:63}
 RR \textsubscript{C,} \textsubscript{\{\textsc{case:dat}, \textsc{num:pl}\},} \textsubscript{\textsc{n[}\textsc{ic:} 11]} ($\langle$X,$\sigma$ $\rangle$) = \textsubscript{def} $\langle$X\textit{ənɐ}ˊ,$\sigma$ $\rangle$
\z

\ea%64
\label{ex:key:64}
 RR \textsubscript{C,} \textsubscript{\{\textsc{case:dat}, \textsc{num:pl}\},} \textsubscript{\textsc{n[}\textsc{ic:} 1} \textsubscript{\tiny $\veebar$} \textsubscript{2} \textsubscript{\tiny $\veebar$}\textsubscript{ 3} \textsubscript{\tiny $\veebar$} \textsubscript{4} \textsubscript{\tiny $\veebar$} \textsubscript{5} \textsubscript{\tiny $\veebar$} \textsubscript{6} \textsubscript{\tiny $\veebar$} \textsubscript{7} \textsubscript{${\veebar}$} \textsubscript{8} \textsubscript{${\veebar}$} \textsubscript{9} \textsubscript{${\veebar}$} \textsubscript{10} \textsubscript{${\veebar}$}\textsubscript{ 12]} ($\langle$X,$\sigma$ $\rangle$) = \textsubscript{def} $\langle$X\textit{ɐ}ˊ,$\sigma$ $\rangle$
\z

% \ea%61
\begin{exe}
\exr{ex:key:61}
 RR \textsubscript{D,} \textsubscript{\{\textsc{num:pl}\},} \textsubscript{\textsc{n[}\textsc{ic:} 7} \textsubscript{\tiny $\veebar$}\textsubscript{ 11]} ($\langle$X,$\sigma$ $\rangle$) = \textsubscript{def} $\langle$X *\textit{i} $\rightarrow$ ø/\_Vˊ,$\sigma$ $\rangle$
\end{exe}

Die RR \REF{ex:key:62} definiert, dass der Plural der \isi{Flexionsklassen} 11 und 12 auf -\textit{i} lautet, die RR \REF{ex:key:63}, dass der Dativ Plural der \isi{Flexionsklasse} 11 auf -\textit{ənɐ} lautet, die RR \REF{ex:key:64}, dass der Dativ Plural aller \isi{Flexionsklassen} außer 11 auf -\textit{ɐ} endet. Die Neutra mit einer auf \textit{i} auslautenden \isi{Wurzel} gehören zur \isi{Flexionsklasse} 11. Wie in \isi{Flexionsklasse} 7 wird auch hier das \textit{i} der \isi{Wurzel} getilgt. Die RR \REF{ex:key:61} kann also für die \isi{Flexionsklassen} 7 und 11 verwendet werden.

Die zu verwerfende Analyse geht von der Beobachtung aus, dass der Dativ Plural aller \isi{Flexionsklassen} auf -\textit{ɐ} endet (vgl. Paradigma 8). Im Gegensatz zur \isi{Flexionsklasse} 12 wird in \isi{Flexionsklasse} 11 agglutiniert: \textit{bet}/\textit{bet}-\textit{i}/\textit{bet}-\textit{i}-\textit{n}-\textit{ɐ}>\textit{bet}-\textit{ə}-\textit{n}-\textit{ɐ} ‘Bett’. Ein \textit{n} wird eingeschoben, um den \isi{Hiat} zu vermeiden und das \textit{i} wird inlautend gesenkt. Auf den ersten Blick wäre also die RR \REF{ex:key:63} nicht nötig. Um aber die korrekten Formen der beiden \isi{Flexionsklassen} 11 und 12 zu definieren, müsste aus der RR \REF{ex:key:62} zwei RRs gemacht werden, nämlich eine für die agglutinierende \isi{Flexionsklasse} 11 \REF{ex:key:65} und eine für die \isi{Flexionsklasse} 12 \REF{ex:key:66}:

\ea%65
\label{ex:key:65}
 *RR \textsubscript{B,} \textsubscript{\{\textsc{num:pl}\},} \textsubscript{\textsc{n[}\textsc{ic:} 11]} ($\langle$X,$\sigma$ $\rangle$) = \textsubscript{def} $\langle$X\textit{i}ˊ,$\sigma$ $\rangle$
\z

\ea%66
\label{ex:key:66}
 *RR \textsubscript{C,} \textsubscript{\{\textsc{case:nom}} \textsubscript{\tiny $\veebar$}\textsubscript{ \AKK, \textsc{num:pl}\},} \textsubscript{\textsc{n[}\textsc{ic:} 12]} ($\langle$X,$\sigma$ $\rangle$) = \textsubscript{def} $\langle$X\textit{i}ˊ,$\sigma$ $\rangle$
\z

Die RR \REF{ex:key:65} besagt, dass -\textit{i} im gesamten Plural suffigiert wird, was nur für die \isi{Flexionsklasse} 11, aber nicht für 12 zutrifft. Mit dieser Analyse würde also die RR \REF{ex:key:63} wegfallen, dafür wird eine zusätzliche RR für das Pluralsuffix -\textit{i} gebraucht. Problematisch wird diese Analyse aber erst, wenn die Neutra mit einer auf \textit{i} auslautenden \isi{Wurzel} betrachtet werden (\textit{bekchi} ‘Becken’). Mit den RRs \REF{ex:key:65}, \REF{ex:key:66} und (\REF{ex:key:64}, für alle \isi{Flexionsklassen}) entstehen folgende Formen, unabhängig davon, zu welcher \isi{Flexionsklasse} diese Neutra gezählt werden: *\textit{bekchə}-\textit{n}-\textit{i}, \textit{bekchə}-\textit{n}-\textit{ɐ} (n-Einschub und Senkung des \textit{i} zu \textit{ə}). Nimmt man die RR \REF{ex:key:61} dazu, entstehen diese Formen: \textit{bekch}-\textit{i}, *\textit{bekch}-\textit{ɐ}. Damit mit den RRs \REF{ex:key:65}, \REF{ex:key:66} und (\REF{ex:key:64}, für alle \isi{Flexionsklassen}) die Formen der Neutra korrekt definiert werden, müsste eine zusätzliche RR für die \is{Subtraktion}Subtraktion angenommen werden:

\ea%67
\label{ex:key:67}
 *RR \textsubscript{D,} \textsubscript{\{\textsc{num:pl}\},} \textsubscript{\textsc{n[}\textsc{ic:} 11]} ($\langle$X,$\sigma$ $\rangle$) = \textsubscript{def} $\langle$X *\textit{i} $\rightarrow$ ø/\_iˊ,$\sigma$ $\rangle$
\z

Mit dieser RR wird das \textit{i} der \isi{Wurzel} getilgt, wenn das Pluralsuffix -\textit{i} folgt, womit der n-Einschub verhindert wird (\textit{bekch}-\textit{i}). Da der Kontext dieser RR auf ein nachfolgendes \textit{i} beschränkt ist, entsteht der korrekte Dativ Plural \textit{bekchə}-\textit{n}-\textit{ɐ} (insofern die RR \REF{ex:key:61} auf die \isi{Flexionsklasse} 7 beschränkt wird). Da die erste Analyse mit vier RRs auskommt (\ref{ex:key:61}--\ref{ex:key:64}), für die zweite jedoch fünf RRs nötig sind (\REF{ex:key:61}, \ref{ex:key:64}--\ref{ex:key:67}), ist die erste Analyse zu bevorzugen.

Die Dialekte von \textbf{Vorarlberg, Saulgau und Stuttgart} können zusammen betrachtet werden. Für alle drei Dialekte ist eine \isi{Flexionsklasse} mit einem Pluralsuffix auf -\textit{ə} anzusetzen: für Vorarlberg \isi{Flexionsklasse} 6 (Paradigma 9), für Saulgau \isi{Flexionsklasse} 4 (Paradigma 13), für Stuttgart \isi{Flexionsklasse} 5 (Paradigma 14). In diese \isi{Flexionsklassen} gehören im Dialekt von Vorarlberg jene \isi{Substantive}, deren \isi{Wurzel} auf ein \textit{i} auslautet (\textit{kchöchi}/\textit{kchöchə}-\textit{n}-\textit{ə} ‘Köchin’ \citealt[252-253]{Jutz1925}), im Dialekt von Saulgau und Stuttgart \isi{Substantive} mit einer \isi{Wurzel} auf \textit{e} (Saulgau: \textit{deke}/\textit{dekə}-\textit{n}-\textit{ə} ‘Decke’ \citep[105]{Raichle1932}; Stuttgart: \textit{wəide}/\textit{wəidə}-\textit{n}-\textit{ə} ‘Wirtin’ \citep[151]{Frey1975}). Da in keinem dieser Dialekte konsonantisch auslautende \isi{Wurzeln} einen Plural mit -\textit{ənə} haben, müsste also eine zusätzliche \isi{Flexionsklasse} angesetzt werden, würde man bei den genannten Wörtern von einem Plural auf -\textit{ənə} ausgehen. Dies würde die Beschreibung unnötigerweise komplexer machen, zumal diese Fälle auch problemlos anderen \isi{Flexionsklassen} zugeordnet werden können. In allen drei Dialekten wird ein \textit{n} zur Hiatvermeidung verwendet, wodurch der Auslaut der \isi{Wurzel} ins Wortinnere tritt und dort gesenkt wird (\citealt[175]{Jutz1925}, \citealt[64]{Raichle1932}). In Saulgau und Stuttgart wird eine weitere \isi{Flexionsklasse} für \isi{Substantive} angenommen, deren \isi{Wurzel} auf \textit{e} endet (z.B. Stuttgart: \textit{dischle}/\textit{dischl}-\textit{ə} ‘Tischlein’). Die Begründung entspricht jener für die \isi{Flexionsklassen} 4 und 7 in Uri. Im Gegensatz zu den oben geschilderten Fällen wird hier der auslautende Vokal der \isi{Wurzel} nicht als Konsequenz des n-Einschubs gesenkt, sondern der Vokal getilgt. Aus den bereits für den Dialekt von Uri genannten Gründen (\textit{schn\=idəri}/\textit{schn\=idər}-\textit{a} ‘Schneiderin’) muss eine separate \isi{Flexionsklasse} angenommen werden, wenn der auslautende Wurzelvokal getilgt wird.

Nun werden die Dialekte aus der \tabref{table5.4} besprochen. Die Dialekte von \textbf{Issime, Visperterminen, Bern und Elisabethtal} weisen nicht nur ein \textit{n} zur Hiatvermeidung, aber keine Schwächung der Mittelsilbe auf, sondern haben auch kein Pluralsuffix des Typs -\textit{ənə}. Deswegen können sie hier auch zusammen behandelt werden. Aus den Tabellen \ref{table5.8} und \ref{table5.9} kann entnommen werden, dass bei Wörtern, deren \isi{Wurzeln} auf einen Vokal auslauten, ein \textit{n} eingeschoben wird, wenn ein vokalisches \isi{Suffix} folgt. Da bereits \isi{Flexionsklassen} mit denselben Pluralsuffixen für auf Konsonant auslautende \isi{Wurzeln} angesetzt werden müssen, können Wörter mit vokalisch auslautenden \isi{Wurzeln} denselben \isi{Flexionsklassen} zugeordnet werden. In Visperterminen beispielsweise hat die \isi{Flexionsklasse} 12 folgenden Satz an \isi{Suffixen}: -\textit{sch} (Genitiv Singular), -\textit{i} (Nominativ und Akkusativ Plural), -\textit{u} (Dativ Plural), -\textit{o} (Genitiv Plural). Dies entspricht dem Suffixsatz der Maskulina und Neutra mit einer auf \textit{i} auslautenden \isi{Wurzel} haben, nur dass ein \textit{n} aus phonologischen Gründen eingeschoben wird. Des Weiteren kann festgestellt werden, dass der auslautende Vokal der \isi{Wurzel} nicht verändert wird, wenn \textit{n} + \isi{Suffix} folgen. Außerdem muss im Dialekt von Elisabethtal zusätzlich eine \isi{Flexionsklasse} für jene \isi{Substantive} angenommen werden, deren \isi{Wurzel} auf -\textit{ə} auslautet. Wie in den Dialekten von Uri (\textit{schn\=idəri}/\textit{schn\=idər}-\textit{ɐ} ‘Schneiderin’), Vorarlberg, Saulgau und Stuttgart handelt es sich auch hierbei um eine Tilgung des auslautenden Vokals der \isi{Wurzel}, was durch eine RR abgebildet wird. Schließlich sind in Issime zwei \isi{Flexionsklassen} (16 und 17) anzusetzen, da sie einen Pluralsuffix -\textit{in} haben. Derselbe Fall, der oben erörtert wurde, tritt in Jaun auf (Pluralsuffix -\textit{ən}). Wie für Jaun ist auch für Issime die Analyse mit -\textit{in} als Pluralsuffix kürzer. Andere \isi{Flexionsklassen} weisen denselben Satz an Pluralsuffixen auf: -\textit{i} (Nominativ und Akkusativ Plural), -\textit{u} (Dativ und Genitiv Plural). Zusätzlich muss also nur ein Pluralsuffix -\textit{in} angenommen werden, während in einer Analyse -\textit{ini}/-\textit{inu} zwei RRs gebraucht würden.

%{\tabref{table5.8}: \textit{N}} {zur Hiatvermeidung in Issime und Visperterminen (basierend auf \citealt[144-205]{Zürrer1999}} {und \citealt[119-134]{Wipf1911} }\\

\begin{table}
\caption{\textit{N} zur Hiatvermeidung in Issime und Visperterminen (basierend auf \citealt[144-205]{Zürrer1999} und \citealt[119-134]{Wipf1911}}\label{table5.8}
\resizebox{\textwidth}{!}{\begin{tabular}{l@{ }l*{5}{l}} 
\lsptoprule
\textsc{fk} & \multicolumn{2}{c}{\textsc{singular}} & \multicolumn{4}{c}{\textsc{plural}} \\\cmidrule(lr){2-3}\cmidrule(lr){4-7}
& \textsc{nom/akk/dat} & \textsc{gen} & \textsc{nom} & \textsc{akk} & \textsc{dat} & \textsc{gen}\\
\midrule
\multicolumn{7}{l}{Issime}\\\midrule
9 (\textsc{m+n}) & bet ‘Bett’ & bet-sch & bet-i & bet-i & bet-u & bet-u\\
& berri ‘Beere’ & berri-sch & berri-n-i & berri-n-i & berri-n-u & berri-n-u\\
14 (\textsc{f}) & aksch ‘Axt’ & aksch & aksch-i & aksch-i & aksch-u & aksch-u\\
& chötti ‘Kette’ & chötti & chötti-n-i & chötti-n-i & chötti-n-u & chötti-n-u\\
16 & schuld ‘Schuld’ & schuld & schuld-in-i & schuld-in-i & schuld-in-u & schuld-in-u\\
17 & nacht ‘Nacht’ & nacht & necht-in-i & necht-in-i & necht-in-u & necht-in-u\\
\midrule
\multicolumn{7}{l}{Visperterminen} \\\midrule
12 (\textsc{m+n}) & ber ‘Beere’ & ber-sch & ber-i & ber-i & ber-u & ber-o\\
& redli ‘Rad’ & redli-sch & redli-n-i & redli-n-i & redli-n-u & redli-n-o\\
14 (\textsc{f}) & farb ‘Farbe’ & farb & farb-e & farb-e & farb-u & farb-o\\
& welbi ‘Wölbung’ & welbi & welbi-n-e & welbi-n-e & welbi-n-u & welbi-n-o\\
\lspbottomrule
\end{tabular}}
\end{table}

%{\tabref{table5.9}: \textit{N}} {zur Hiatvermeidung in Bern und Elisabethtal (basierend auf \citealt[82-90]{Marti1985} und \citealt[50-52]{Žirmunskij1928/29}}\\

\begin{table}
\caption{\textit{N} zur Hiatvermeidung in Bern und Elisabethtal (basierend auf \citealt[82-90]{Marti1985} und \citealt[50-52]{Žirmunskij1928/29})}\label{table5.9}
\begin{tabular}{lll} 
\lsptoprule
\textsc{fk} & \textsc{singular} & \textsc{plural}\\
\midrule
Bern &  & \\\midrule
6 & h\=as ‘Haase’ & h\=as-ə\\
& tantə ‘Tante’ & tantə-n-ə\\
& bürdi ‘Bürde, viel’ & bürdi-n-ə\\
\midrule
Elisabethtal &  & \\\midrule
8 & schuld ‘Schuld’ & schuld-ɐ\\
& khuchə ‘Küche’ & khuchə-n-ɐ\\
11 & biəblə ‘Bübchen’ & biəbl-ɐ\\
\lspbottomrule
\end{tabular}
\end{table}

Die Dialekte von \textbf{Petrifeld und des Kaiserstuhls} weisen ebenfalls ein \textit{n} zur Hiatvermeidung und keine Senkung des auslautenden Wurzelvokals auf, wenn er im Wortinneren steht. Dies wird aus den Tabellen \ref{table5.10} und \ref{table5.11} bezüglich der \isi{Flexionsklassen} 4 ersichtlich. Im Gegensatz zu Issime, Visperterminen, Bern und Elisabethtal haben Petrifeld und Kaiserstuhl jedoch Pluralsuffixe des Typs -\textit{ənə}. Es ist phonologisch nicht erklärbar, wie in Petrifeld der auslautende Wurzelvokal \textit{e} zu \textit{i} werden soll bzw. \textit{i} zu \textit{ɐ} im Kaiserstuhl, wenn man davon ausgeht, dass ein \textit{n} zur Hiatvermeidung eingefügt wird. Deswegen ist ein Pluralsuffix -\textit{inə} bzw. -\textit{ɐnɐ} anzunehmen. Des Weiteren kann bezüglich der \isi{Flexionsklasse} 9 von Petrifeld nicht argumentiert werden, dass \textit{e} automatisch von der Phonologie zu \textit{ə} gesenkt wird, da die Abfolge -\textit{enə} möglich ist \citep[42]{Moser1937}. Folglich handelt es sich bei -\textit{ənə} um ein Pluralsuffix. In den \isi{Flexionsklassen} 7 und 9 von Petrifeld wird aber der auslautende Vokal der \isi{Wurzel} getilgt, wenn ein Pluralsuffix folgt, was durch eine RR definiert wird, wie bereits oben für Uri (\textit{schn\=idəri}/\textit{schn\=idər}-\textit{a} ‘Schneiderin’) und weitere Dialekte gezeigt wurde.

%{\tabref{table5.10}: \textit{N}} {zur Hiatvermeidung und Plural des Typs -\textit{ene}} {in Petrifeld (basierend auf \citealt[59-62]{Moser1937}:}\\

\begin{table}
\caption{\textit{N} zur Hiatvermeidung und Plural des Typs -\textit{ene} in Petrifeld (basierend auf \citealt[59-62]{Moser1937})}\label{table5.10}
\begin{tabular}{lll}
\lsptoprule
{\textsc{fk}} & {\textsc{singular}} & {\textsc{plural}}\\
\midrule
4 & bek ‘Bäcker’ & bek-ə\\
& glokə ‘Glocke’ & glokə-n-ə\\
\midrule
8 & kh\=inege ‘Königin’ & kh\=ineg-inə\\
\midrule
9 & kheche ‘Köchin’ & khech-ənə\\
\midrule
7 & -le & -lə\\
\lspbottomrule
\end{tabular}
\end{table}

%{\tabref{table5.11}: \textit{N}} {zur Hiatvermeidung und Plural des Typs -\textit{ene}} {in Kaiserstuhl (basierend auf \citealt[359-373]{Noth1993}:}\\

\begin{table}
\caption{\textit{N} zur Hiatvermeidung und Plural des Typs -\textit{ene} in Kaiserstuhl (basierend auf \citealt[359-373]{Noth1993})}\label{table5.11}
\begin{tabular}{lll}
\lsptoprule
{\textsc{fk}} & {\textsc{singular}} & {\textsc{plural}}\\
\midrule
4 & grab ‘Grab’ & grab-ɐ\\
& dandɐ ‘Tante’ & dandɐ-n-ɐ\\
& bhatzianti ‘Patientin’ & bhatzianti-n-ɐ\\
\midrule
5 & ghuchi ‘Küche’ & ghuch-ɐnɐ\\
\lspbottomrule
\end{tabular}
\end{table}

\textbf{Huzenbach} stellt im Vergleich zu allen anderen Dialekten einen Sonderfall dar (Tabellen \ref{table5.5} und \ref{table5.12}). Würde man die Substantivflexion gleich wie in den übrigen Dialekten analysieren, dann würde daraus Folgendes resultieren. Wörter wie \textit{wiərde} ‘Wirtin’ \citep[97]{Baur1967}, deren \isi{Wurzeln} auf \textit{e} auslauten und die den Plural auf -\textit{ə} bilden, gehören in die \isi{Flexionsklasse} 4. Ein \textit{n} wird eingefügt, um den \isi{Hiat} zu vermeiden, und \textit{e} wird im Wortinneren zu \textit{ə} gesenkt \citep[75-78]{Baur1967}. Für Wörter wie \textit{heisle} ‘Häuschen’ \citep[98]{Baur1967} muss eine neue \isi{Flexionsklasse} angenommen werden, da der auslautende Vokal der \isi{Wurzel} im Plural getilgt wird. Mit dieser Analyse müsste jedoch für die movierten Feminina wie \textit{naiəre}/\textit{naiərnə} ‘Näherin’ \citep[97]{Baur1967} eine eigene \isi{Flexionsklasse} angesetzt werden (mit einem Pluralsuffix auf -\textit{nə}), was mit der hier verwendeten Analyse nicht nötig ist. Wörter, deren auslautender Wurzelvokal im Plural getilgt wird (\textit{heisle}), gehören zur \isi{Flexionsklasse} 4, die den Plural ebenfalls auf -\textit{ə} bildet. Die Tilgung wird durch eine RR \is{Subtraktion}(Subtraktion) gewährleistet, die dieser \isi{Flexionsklasse} zugeordnet ist. Dieselbe RR wird in der \isi{Flexionsklasse} 6 verwendet. Auch hier fällt der Wurzelauslaut \textit{e} weg, wenn ein \isi{Suffix} (-\textit{ənə}) folgt.

%{\tabref{table5.12}: Plural des Typs -\textit{ene}} {in Huzenbach (basierend auf \citealt[92-98]{Baur1967}}\\

\begin{table}
\caption{Plural des Typs -\textit{ene} in Huzenbach (basierend auf \citealt[92-98]{Baur1967})}\label{table5.12}
\begin{tabular}{lll}
\lsptoprule
{\textsc{fk}} & {\textsc{singular}} & {\textsc{plural}}\\
\midrule
4 & dan ‘Tanne’ & dan-ə\\
& heisle ‘Häuschen’ & heisl-ə\\
\midrule
6 & wiərde ‘Wirtin’ & wiərd-ənə\\
& naiəre ‘Näherin’ & naiər-nə\\
\lspbottomrule
\end{tabular}
\end{table}

Mit dieser Analyse benötigt man keine zusätzliche \isi{Flexionsklasse} für die movierten Feminina, sondern sie können der \isi{Flexionsklasse} 6 zugeordnet werden. Wie bereits dargestellt wurde, wird in dieser \isi{Flexionsklasse} im Plural -\textit{ənə} suffigiert und der auslautende Vokal der \isi{Wurzel} getilgt: \textit{naiəre}→\textit{naiəre}-\textit{ənə}→\textit{naiər}-\textit{ənə} ‘Näherin’. Daraus entsteht ein Wort, auf dessen betonte Wurzelsilbe drei unbetonte Silben folgen. Da dies im gesamten System nicht möglich ist, wird von der Phonologie eine Silbe gekürzt (\textit{naiər}-\textit{nə} ‘Näherin’). Dazu braucht es folglich keine RR und die movierten Feminina können in die \isi{Flexionsklasse} 6 eingruppiert werden. Dass diese drei Silben unbetont sind, ist gesichert, da ihre Nuklei aus einem Schwa bestehen. Nicht gesichert ist jedoch, dass in einem solchen Fall eine Silbe automatisch gekürzt wird, da dieses Phänomen in diesem Dialekt noch nicht untersucht worden ist. Solche Restriktionen sind aber aus anderen Varietäten des Deutschen bekannt. Beispielsweise lauten in der deutschen Standardsprache Wörter auf einen Trochäus (minimaler Fuß = Zweisilber) bzw. Daktylus (maximaler Fuß = Dreisilber) aus \citep[130, 135]{Eisenberg2006}. Dies reguliert u.a., ob im Dativ Plural der \isi{Substantive} ein -\textit{n} oder -\textit{en} suffigiert wird \citep[167-169]{Eisenberg2006}. Dasselbe Phänomen wie in Huzenbach ist auch in den Dialekten von Petrifeld und Kaiserstuhl zu beobachten. In Petrifeld gehören movierte Feminina zur \isi{Flexionsklasse} 9 (vgl. \tabref{table5.10}), in der im Plural -\textit{ənə} suffigiert und der auslautende Vokal der \isi{Wurzel} getilgt wird. Bei den movierten Feminina wird zusätzlich eine Silbe gekürzt, was phonologisch bedingt ist: \textit{schaidəre}→\textit{schaidəre}-\textit{ənə}→\textit{schaidər}-\textit{ənə}→\textit{schaidərnə} ‘Schneiderin’. Genau dasselbe passiert im Dialekt des Kaiserstuhls, in dem movierte Feminina zur \isi{Flexionsklasse} 5 gehören (vgl. \tabref{table5.11}): \textit{lährəri}→\textit{lährəri}-\textit{ana}→\textit{lährər}-\textit{ana}→\textit{lährərna} ‘Lehrerin’.

\subsection{Blöcke}\label{5.1.5}

Wie in \sectref{4.1.3.2} dargestellt wurde, sind die RRs in \isi{Blöcken} organisiert. Dadurch wird gewährleistet, dass die RR in der richtigen Reihenfolge angewendet werden. Beispielsweise ist beim standarddeutschen Wort \textit{Bild-ər-n} (Dativ Plural) wichtig, dass zuerst aus der \isi{Wurzel} \textit{bild} der Plural \textit{bild-ər} entsteht und dann, auf \textit{bild-ər} basierend, der Dativ Plural \textit{bild-ər-n}. Wäre die Reihenfolge der RRs nicht definiert, könnte daraus auch *\textit{bild-n-ər} resultieren. Innerhalb eines Blockes konkurrieren die RRs miteinander, RRs unterschiedlicher \isi{Blöcke} hingegen nicht. Dadurch kann beispielsweise definiert werden, dass zur Pluralmarkierung maximal ein \isi{Suffix} verwendet wird, indem alle Pluralsuffixe im selben \isi{Block} stehen.

Alle hier untersuchten Varietäten entsprechen einem der vier Systeme an \isi{Blöcken} in den Tabellen \ref{table5.13} und \ref{table5.14}. \tabref{table5.13} zeigt jene Varietäten, die mindestens einen \isi{Umlaut} (\isi{Block} A) und Pluralsuffixe (\isi{Block} B) haben; in einigen Dialekten wird zudem subtrahiert (\isi{Block} C). Dass die \isi{Umlaute} und Pluralsuffixe in zwei verschiedenen \isi{Blöcken} stehen, hat zwei Gründe. Erstens stehen die \isi{Umlaute} einerseits und die Pluralsuffixe andererseits miteinander in Konkurrenz, wenn eine Flexion mehrere \isi{Umlaute} und Pluralsuffixe aufweist. Dies beschränkt die Anzahl der \isi{Umlaute} und Pluralsuffixe auf je eine Markierung pro Wort. Zwei \isi{Umlaute} oder zwei Pluralsuffixe kommen also nicht vor. Zweitens definieren diese \isi{Blöcke}, dass ein Wort sowohl einen \isi{Umlaut} als auch ein Pluralsuffix haben kann. Neben den RRs für die \isi{Umlaute} und die Pluralsuffixe braucht es in vielen Dialekten RRs für die \is{Subtraktion}Subtraktion (\isi{Block} C). Außer im Dialekt von Münstertal handelt es sich dabei in allen Dialekten um die Tilgung des auslautenden Vokals der \isi{Wurzel}, wenn ein Vokal folgt. In der Substantivflexion von Münstertal wird im Plural ein auslautendes \textit{t} getilgt.

Des Weiteren wird in einer \isi{Flexionsklasse} im Münstertal im Plural der Wurzelvokal diphthongiert. Da jedoch nicht in derselben \isi{Flexionsklasse} diphthongiert und umgelautet wird, kann die RR für die Diphthongierung ebenfalls in \isi{Block} A stehen (zusammen mit den \isi{Umlauten}). Schließlich wird in den meisten Dialekten in possessiven Kontexten bei Eigennamen ein -\textit{s} suffigiert. Da dieses \isi{Suffix} nur im Singular verwendet wird, kann die RR dazu in \isi{Block} B stehen. Denn die Pluralsuffixe sind auf den Plural und die Possessivsuffixe auf den Singular beschränkt, wodurch diese RRs nicht miteinander in Konflikt treten. Gleiches gilt für die Akkusativ/Dativ Singular Markierung im Dialekt von Elisabethtal. Im Dialekt von Vorarlberg positionieren sich die Pluralsuffixe sowie das \isi{Suffix} für den Dativ Plural in \isi{Block} B, denn ein sie werden nie an dasselbe Wort angehängt (vgl. Paradigma 9).

%{\tabref{table5.13}: \isi{Blöcke} der Substantivflexion der Varietäten ohne Kasusmarkierung im Plural}\\

\begin{table}
\caption{Blöcke der Substantivflexion der Varietäten ohne Kasusmarkierung im Plural}\label{table5.13}
\resizebox{\textwidth}{!}{\begin{tabular}{llll}
\lsptoprule
\isi{Block} A & \isi{Umlaut} & \isi{Block} A & \isi{Umlaut}\\
\isi{Block} B & \isi{Suffixe} Plural & \isi{Block} B & \isi{Suffixe} Plural\\
\isi{Block} C & RR Subtraktion &  & \\
\midrule
\multicolumn{2}{p{.5\linewidth}}{Gilt für folgende Varietäten:} & \multicolumn{2}{p{.5\linewidth}}{Gilt für folgende Varietäten:}\\
\multicolumn{2}{p{.5\linewidth}}{Sensebezirk (Poss-S), Huzenbach (Poss-S), Saulgau (Poss-S), Stuttgart, Petrifeld (Poss-S) (Dat.Sg.), Elisabethtal (Akk/Dat.Sg.), Kaiserstuhl (Poss-S), Münstertal (Poss-S) (Diphthong)}
& \multicolumn{2}{p{.5\linewidth}}{Bern (Poss-S), Vorarlberg (Poss-S) (Dat.Pl.), Colmar, Elsass (Ebene)}\\
\lspbottomrule
\end{tabular}}
\end{table}

%{\tabref{table5.14}: \isi{Blöcke} der Substantivflexion der Varietäten mit Kasusmarkierung im Plural}\\

\begin{table}
\caption{Blöcke der Substantivflexion der Varietäten mit Kasusmarkierung im Plural}\label{table5.14}
\resizebox{\textwidth}{!}{\begin{tabular}{llll}
\lsptoprule
\isi{Block} A & \isi{Umlaut} & \isi{Block} A & \isi{Umlaut}\\
\isi{Block} B & \isi{Suffixe} Plural & \isi{Block} B & \isi{Suffixe} Plural\\
\isi{Block} C & \isi{Suffixe} \isi{Kasus} & \isi{Block} C & \isi{Suffixe} \isi{Kasus}\\
\isi{Block} D & RR Subtraktion &  & \\
\midrule
\multicolumn{2}{p{.5\linewidth}}{Gilt für folgende Varietäten:} & \multicolumn{2}{p{.5\linewidth}}{Gilt für folgende Varietäten:}\\
\multicolumn{2}{p{.5\linewidth}}{Althochdeutsch, Mittelhochdeutsch, Uri (Poss-S)} & \multicolumn{2}{p{.5\linewidth}}{Issime, Visperterminen, Jaun, Zürich, Standard}\\
\lspbottomrule
\end{tabular}}
\end{table}

Auf die \isi{Blöcke} in \tabref{table5.14} treffen genau dieselben Beobachtungen zu. Die RR für die \is{Subtraktion}Subtraktion definieren im Althochdeutschen und im Dialekt von Uri, dass der auslautende Vokal der \isi{Wurzel} bei Suffigierung getilgt wird, im Mittelhochdeutschen, dass auslautendes \textit{w} getilgt wird. Die \isi{Blöcke} in \tabref{table5.14} unterscheiden sich von jenen in \tabref{table5.13} nur darin, dass ein zusätzlicher \isi{Block} für die Kasussuffixe angenommen werden muss. Der \isi{Block} mit den Pluralsuffixen steht vor dem \isi{Block} mit den Kasussuffixen. Dies definiert, dass, wenn im Plural \isi{Numerus} und \isi{Kasus} separat markiert werden, zuerst \isi{Numerus} und dann \isi{Kasus} ausgedrückt wird (z.B. \textit{Bild}-\textit{ər}-\textit{n}). Für die Kasussuffixe wird nur ein \isi{Block} und nicht je einer pro \isi{Numerus} benötigt. Denn entweder treten diese RR nicht direkt miteinander in Konkurrenz, da sie für den Singular oder für den Plural definiert sind, oder es handelt sich um einen \isi{Synkretismus}. In diesem Fall bleibt das Feature \isi{Numerus} in der RR unterspezifiziert. Dies kann an der \isi{Flexionsklasse} 4 des Althochdeutsch gut dargestellt werden (vgl. Paradigma 1). Betrachtet man nur diese \isi{Flexionsklasse}, sind folgende RR anzusetzen:

\ea%68
\label{ex:key:68}
 RR \textsubscript{C,} \textsubscript{\{\textsc{case:nom}} \textsubscript{\tiny $\veebar$}\textsubscript{ \AKK\},} \textsubscript{\textsc{n[}\textsc{ic:} 4]} ($\langle$X,$\sigma$ $\rangle$) = \textsubscript{def} $\langle$X\textit{i}ˊ,$\sigma$ $\rangle$
\z

\ea%69
\label{ex:key:69}
 RR \textsubscript{C,} \textsubscript{\{\textsc{case:dat}, \textsc{num:sg}\},} \textsubscript{\textsc{n[}\textsc{ic:} 4]} ($\langle$X,$\sigma$ $\rangle$) = \textsubscript{def} $\langle$X\textit{e}ˊ,$\sigma$ $\rangle$
\z

\ea%70
\label{ex:key:70}
 RR \textsubscript{C,} \textsubscript{\{\textsc{case:gen}, \textsc{num:sg}\},} \textsubscript{\textsc{n[}\textsc{ic:} 4]} ($\langle$X,$\sigma$ $\rangle$) = \textsubscript{def} $\langle$X\textit{es}ˊ,$\sigma$ $\rangle$
\z

\ea%71
\label{ex:key:71}
 RR \textsubscript{C,} \textsubscript{\{\textsc{case:dat}, \textsc{num:pl}\},} \textsubscript{\textsc{n[}\textsc{ic:} 4]} ($\langle$X,$\sigma$ $\rangle$) = \textsubscript{def} $\langle$X\textit{im}ˊ,$\sigma$ $\rangle$
\z

\ea%72
\label{ex:key:72}
 RR \textsubscript{C,} \textsubscript{\{\textsc{case:gen}, \textsc{num:pl}\},} \textsubscript{\textsc{n[}\textsc{ic:} 4]} ($\langle$X,$\sigma$ $\rangle$) = \textsubscript{def} $\langle$X\textit{o}ˊ,$\sigma$ $\rangle$
\z

RR \REF{ex:key:68} definiert, dass im Nominativ und Akkusativ Singular und Plural (\isi{Numerus} unterspezifiziert) -\textit{i} suffigiert wird. Die morphosyntaktischen Eigenschaften der RRs (\ref{ex:key:69}-\ref{ex:key:72} sind so definiert, dass die RR nicht gegeneinander in Konkurrenz treten.

\subsection{Von der Ortsgrammatik zu den Paradigmen und Realisierungsregeln}\label{5.1.6}

In diesem Kapitel soll die Systematisierungsarbeit anhand der Substantivflexion im Dialekt von Jaun vorgestellt werden. Es wird gezeigt, wie in dieser Arbeit aufgrund der Angaben in einer \isi{Ortsgrammatik} ein Paradigma erstellt wird und wie die RRs dieses Paradigmas formuliert werden. Eine Beschreibung dieser Systematisierungsarbeit mit allen Details (d.h. jede einzelne Entscheidung in allen 20 untersuchten Varietäten) kann hier nicht geleistet werden. Vielmehr sollen hier die wichtigsten Herausforderungen und Fragen anhand von repräsentativen Beispielen thematisiert werden.

{Grundsätzlich gilt:} \isi{Ortsgrammatiken} liefern die Datengrundlage der Dialekte in dieser Arbeit. Jede \isi{Ortsgrammatik} systematisiert die sprachlichen Daten auf unterschiedliche Weise. Damit aber die verschiedenen Varietäten miteinander verglichen werden können, müssen auf der Basis der Informationen in den \isi{Ortsgrammatiken} nach einheitlichen Regeln Paradigmen erstellt werden. Auf der Grundlage dieser Paradigmen wiederum werden die RRs hergeleitet.

Beide Schritte, also von den Quellen zum Paradigma und vom Paradigma zu den RRs, sind mit viel Analysen der Sprachdaten verknüpft. Als Hauptfragen ergeben sich: a) Ist die Modifikation einer Wortform phonologisch oder morphologisch bedingt? b) Wie wird segmentiert, d.h., wo fängt ein \isi{Suffix} an und wo hört es auf? c) Was ist das Minimum an \isi{Flexionsklassen}, die angenommen werden müssen? d) Welche \isi{Affixe} können in einer RRs zusammengefasst werden, sind also \isi{Synkretismen}, und welche nicht? Exemplarisch sollen nun die beiden Schritte (\isi{Ortsgrammatik} $\rightarrow$ Paradigma, Paradigma $\rightarrow$ RRs) anhand der Substantivflexion von Jaun dargestellt werden.

{Von der \isi{Ortsgrammatik} zum Paradigma:} Die Seiten zu den \isi{Substantiven} in der Grammatik des Dialekts von Jaun \citep{Stucki1917} sind in \citet{Stucki2011} abgebildet. Vergleicht man diesen Auszug in \citet{Stucki2011} mit dem Paradigma 6, das auf diesem Auszug basiert, wird offensichtlich, dass nicht einfach abgeschrieben werden kann, sondern dass die Daten analysiert und neu systematisiert werden müssen. Im Auszug der \isi{Ortsgrammatik} fällt erstens auf, dass kaum Paradigmen vorhanden sind. Ebenso gibt es keine expliziten Angaben und Erklärungen zu Fragen der Segmentierung der \isi{Affixe} sowie, ob ein \isi{Affix} morphosyntaktische Funktionen kodiert oder ob es sich dabei um eine phonologisch bedingte Modifikation der Wortform handelt. Zweitens werden die \isi{Substantive} in drei \isi{Genera} und in starke oder schwache Flexion eingeteilt, was sechs \isi{Flexionsklassen} ergeben würde. Wie aus dem Paradigma 6 jedoch ersichtlich wird, müssen für den Dialekt von Jaun 16 \isi{Flexionsklassen} angenommen werden. Drittens ist oft die Rede von germanischen Stämmen, die jedoch zum Teil schon für das Althochdeutsche problematisch sind (z.B. u-Stämme) und für die heutigen Dialekte nicht mehr zur Kategorisierung herangezogen werden können. Aus diesen Gründen und auch, weil jede \isi{Ortsgrammatik} eine eigene Definition von \isi{Flexionsklassen} besitzt, welche in den \isi{Ortsgrammatiken} jedoch nicht explizit thematisiert wird, werden in dieser Arbeit die \isi{Flexionsklassen} nach einheitlichen Regeln bestimmt, wie diese in \sectref{5.1.1} vorgestellt wurden. Folglich besteht das Erstellen der Paradigmen nicht in einem einfachen Abschreiben, sondern die Daten aus der \isi{Ortsgrammatik} bedürfen einer detaillierten und umfassenden Analyse, um die Paradigmen zu erstellen. Dies soll nun genauer dargestellt werden, indem u.a. auch auf Teilanalysen aus den vorangehenden Kapiteln Bezug genommen wird.

In \citeauthor{Stucki1917}s \citeyearpar{Stucki1917} Grammatik werden die Wurzelvokale mit ihren \textsc{Umlauten} gelistet und mit Beispielen belegt (s. \citealt[§200c]{Stucki2011}). Weiter diskutiert wird dies jedoch nicht, wie z.B., dass \textit{a} zwei umgelautete Entsprechungen hat ([ɛ] und [æ]), diese dem Primär- und Sekundärumlaut entsprechen, was in den heutigen Dialekten aber nicht mehr phonologisch bedingt ist und folglich Teil der Morphologie ist. Wie in \sectref{5.1.3} \is{Modifikation}(Modifikationen) gezeigt wurde, werden dafür also zwei verschiedene \isi{Flexionsklassen} benötigt. Aus demselben Grund müssen für die \isi{Substantive}, die den Plural auf -\textit{ər} bilden, zwei \isi{Flexionsklassen} angenommen werden: Es gibt \isi{Substantive} mit dem Plural auf -\textit{ər}, die den Wurzelvokal umlauten, und andere, die den Wurzelvokal nicht umlauten. Dass dies nicht mit der Phonologie begründet werden kann, steht nicht in der \isi{Ortsgrammatik} (vgl. \citealt[§205.2]{Stucki2011}), sondern beruht auf der hier vorgenommenen Analyse der Beispiele in der \isi{Ortsgrammatik}. Wann eine neue \isi{Flexionsklasse} angenommen wird, basiert auf der in dieser Arbeit verwendeten Definition von \isi{Flexionsklassen} (vgl. \sectref{5.1.1}), die nicht jener der \isi{Ortsgrammatik} entspricht.

Eine weitere Frage ist jene der \textsc{Segmentierung}, welche sich erstens gut anhand des Plurals der \isi{Flexionsklassen} 4, 14 und 16 illustrieren lässt (vgl. Paradigma 6). Die \isi{Ortsgrammatik} von Jaun gibt als Pluralmarker -\textit{əni} und -\textit{ənə} an (vgl. \citealt[§206, §211 und §212.2]{Stucki2011}). Eine weitere mögliche Analyse wäre, -\textit{ən} als Pluralmarker, -\textit{i} als Marker für Nominativ/Akkusativ Plural und -\textit{ə} für Dativ/Genitiv Plural anzunehmen, zumal -\textit{i} und -\textit{ə} auch in den \isi{Flexionsklassen} 11 und 15 für dieselben \isi{Kasus} verwendet werden (vgl. Paradigma 6). Wie in \sectref{5.1.4} dargestellt wurde, ist die zweite Analyse (-\textit{ən}-\textit{i} und -\textit{ən}-\textit{ə}) ökonomischer, da eine RR weniger angenommen werden muss, weswegen die zweite mögliche Analyse und nicht jene der \isi{Ortsgrammatik} ausgewählt wird. Betroffen vom Problem der Segmentierung ist zweitens auch der Singular der \isi{Flexionsklassen} 6, 15 und 16 (vgl. Paradigma 6; \citealt[§203.2.b.β und §212]{Stucki2011}). Für die \isi{Flexionsklasse} 6 kann von einer \isi{Wurzel} \textit{chaschtə} ‘Kasten’ und nicht \textit{chascht}-\textit{ə} ausgegangen werden, da alle Zellen des Paradigmas über diese Form verfügen. Für den Nominativ/Akkusativ Plural darf angenommen werden, dass, wenn zwei zentralisierte Vokale mit derselben Quantität und Qualität aufeinandertreffen, diese verschmelzen (*\textit{chaschtə}-\textit{ə} > \textit{chascht}-\textit{ə}). Bezüglich der \isi{Flexionsklassen} 15 und 16 sind auch für den Singular \isi{Suffixe} anzunehmen: \textit{tsung}-\textit{a} (Nominativ/Akkusativ Singular), \textit{tsung}-\textit{ə} (Dativ/Genitiv Singular + Plural), \textit{tsung}-\textit{i} (Nominativ/Akkusativ Plural). Würde man \textit{tsunga} als \isi{Wurzel} annehmen, müsste man erklären, weshalb das auslautende \textit{a} getilgt wird, wenn ein \textit{ə} oder \textit{i} suffigiert wird. Außerdem wäre dafür eine RR nötig, da diese Tilgung kein phonologischer Automatismus sein kann, denn der hiatvermeidende Automatismus ist die n-Epenthese. Aufgrund dieses Paradigmas scheint es also plausibler, eine RR für den Nominativ/Akkusativ Singular anzusetzen, die ein -\textit{a} suffigiert, als eine RR, die ein wurzelauslautendes \textit{a} tilgt. Dazu, wie diese \isi{Substantive} zu segmentieren sind, finden sich keine Angaben in der \isi{Ortsgrammatik}. Die Segmentierung muss selbst vorgenommen werden. Besonders aufwändig in der Analyse sind drittens die \isi{Substantive} auf -\textit{i}, z.B.: \textit{schieri} (Sg.), \textit{schierəni} (Nominativ/Akkusativ Plural), \textit{schierənə} (Dativ/Genitiv Plural) ʻSchereʼ (\citealt[§206 und §211]{Stucki2011}). Die \isi{Ortsgrammatik} von Jaun beschreibt diese knapp als auf \textit{i} auslautende \isi{Substantive}, die den Plural auf -\textit{əni} bilden (\citealt[§206 und §211]{Stucki2011}). Es stellen sich hier folgende Fragen: a) Ist das \textit{i} im Singular Teil der \isi{Wurzel} oder ein \isi{Suffix}? b) Lautet der Plural -\textit{əni}/-\textit{ənə} (neue \isi{Flexionsklasse}) oder -\textit{ən}-\textit{i}/-\textit{ən}-\textit{ə} (wie \isi{Flexionsklassen} 4, 14, 16)? c) Wenn \textit{i} Teil der \isi{Wurzel} ist, wäre auch eine Zentralisierung von \textit{i} vorstellbar, wenn es im Inlaut steht, d.h., wenn ein \isi{Suffix} folgt (\textit{schieri} > \textit{schierə}-\textit{ni}). Dann hätte man es mit einem phonologischen Prozess zu tun, für den folglich keine RRs angesetzt werden müssen. Angenommen es handelt sich bei i>ə um einen phonologischen Prozess, dann ist weiter der Frage nachzugehen, ob der Plural -\textit{ni}/-\textit{nə}  oder -\textit{i}/-\textit{ə} (mit n-Epenthese zur Tilgung des \isi{Hiats}) lautet. Wie in \sectref{5.1.4} dargestellt wurde (vgl. auch \tabref{table5.6}), ist es am plausibelsten und ökonomischsten, von einer \isi{Wurzel} auf \textit{i} auszugehen und von einem Plural auf -\textit{i}/-\textit{ə} mit n-Epenthese zur Tilgung des \isi{Hiats} und mit Zentralisierung von \textit{i} zu \textit{ə}. Die n-Epenthese und die Zentralisierung von \textit{i} zu \textit{ə} sind phonologische Mechanismen (herauszufinden anhand des Teils zur Phonologie in der \isi{Ortsgrammatik}), d.h., sie finden immer und automatisch statt, folglich müssen dafür keine RRs angenommen werden. Daraus ergibt sich weiter, dass der Plural -\textit{i}/-\textit{ə} lautet und dass das \textit{i} zur \isi{Wurzel} gehört. Des Weiteren muss für auf \textit{i} auslautende \isi{Substantive} keine eigene \isi{Flexionsklasse} angesetzt werden, denn diese \isi{Substantive} funktionieren aufgrund dieser Analyse exakt wie \isi{Substantive}, für die auf jeden Fall eigene \isi{Flexionsklassen} angenommen werden müssen (vgl. \tabref{table5.6} in \sectref{5.1.4}).

Anhand dieser Beispiele wird klar, wie vielschrittig der Weg von der \isi{Ortsgrammatik} bis zum Paradigma erfordert. Das Ziel ist also, mithilfe der Menge an Daten in der \isi{Ortsgrammatik} eine linguistisch adäquate Analyse durchzuführen und somit eine adäquate Beschreibung zu erhalten. Gleichzeitig soll die Beschreibung, d.h. das Paradigma, so ökonomisch, also so kurz wie möglich ausfallen, um Redundanzen zu vermeiden, die aus der Beschreibung entstehen und nicht aus dem Sprachsystem resultieren.

{Vom Paradigma zu den RRs:} Nun wird noch gezeigt, wie vorgegangen wird, um auf der Basis des Paradigmas 6 die RRs zu bestimmen. Als erstes muss die Anzahl der \isi{Blöcke} eruiert werden. Der Dialekt von Jaun unterscheidet \isi{Umlaute} sowie \isi{Suffixe}, die nur \isi{Numerus} markieren, und \isi{Suffixe}, die nur \isi{Kasus} markieren. Wie in \sectref{5.1.5} gezeigt wurde, sind also drei \isi{Blöcke} anzunehmen, um die Wortform korrekt aufzubauen: \isi{Block} A für \isi{Umlaute}, \isi{Block} B für Numerussuffixe, \isi{Block} C für Kasussuffixe.

Schritt für Schritt werden nun die RR eingeführt. Da ein Primär- und Sekundärumlaut unterschieden wird, braucht es dafür zwei RRs in \isi{Block} A:

\ea%73
\label{ex:key:73}
 RR \textsubscript{A, \{\textsc{num:pl}\}, \textsc{n[}\textsc{ic:}1} \textsubscript{\tiny $\veebar$}\textsubscript{ 6} \textsubscript{\tiny $\veebar$}\textsubscript{ 9} \textsubscript{\tiny $\veebar$}\textsubscript{ 13]} ($\langle$X,$\sigma$ $\rangle$) = \textsubscript{def} $\langle$Ẍˊ,$\sigma$ $\rangle$
\z

\ea%74
\label{ex:key:74}
 RR \textsubscript{A, \{\textsc{num:pl}\}, \textsc{n[}\textsc{ic:}3]} ($\langle$X,$\sigma$ $\rangle$) = \textsubscript{def} $\langle$Ẍ[\textit{a} $\rightarrow$ \textit{e}]ˊ,$\sigma$ $\rangle$
\z

Es folgen in \isi{Block} B die RRs, die nur Plural markieren. Es handelt sich dabei um drei \isi{Suffixe}, nämlich -\textit{ən}, -\textit{ə} und -\textit{ər}:

\ea%75
\label{ex:key:75}
 RR \textsubscript{B, \{\textsc{num:pl}\}, \textsc{n[}\textsc{ic:}4} \textsubscript{\tiny $\veebar$}\textsubscript{ 14} \textsubscript{\tiny $\veebar$}\textsubscript{ 16]} ($\langle$X,$\sigma$ $\rangle$) = \textsubscript{def} $\langle$X\textit{ən}ˊ,$\sigma$ $\rangle$
\z

\ea%76
\label{ex:key:76}
 RR \textsubscript{B, \{\textsc{num:pl}\}, \textsc{n[}\textsc{ic:}7]} ($\langle$X,$\sigma$ $\rangle$) = \textsubscript{def} $\langle$X\textit{ə}ˊ,$\sigma$ $\rangle$
\z

\ea%77
\label{ex:key:77}
 RR \textsubscript{B, \{\textsc{num:pl}\}, \textsc{n[}\textsc{ic:}9} \textsubscript{\tiny $\veebar$}\textsubscript{ 10]} ($\langle$X,$\sigma$ $\rangle$) = \textsubscript{def} $\langle$X\textit{ər}ˊ,$\sigma$ $\rangle$
\z

In \isi{Block} C stehen die RRs für \isi{Kasus}. Aus dem Paradigma 6 ist ersichtlich, dass im Plural Nominativ und Akkusativ immer zusammenfallen. Für beide \isi{Kasus} ist also nur eine RR nötig. Die drei Allomorphe sind -\textit{a}, -\textit{ə} und -\textit{i}:

\ea%78
\label{ex:key:78}
 RR \textsubscript{C, \{\textsc{case:nom}} \textsubscript{\tiny $\veebar$}\textsubscript{ \textsc{acc},\textsc{num:pl}\}, \textsc{n[}\textsc{ic:}1]} ($\langle$X,$\sigma$ $\rangle$) = \textsubscript{def} $\langle$X\textit{a}ˊ,$\sigma$ $\rangle$
\z

\ea%79
\label{ex:key:79}
 RR \textsubscript{C, \{\textsc{case:nom}} \textsubscript{\tiny $\veebar$}\textsubscript{ \textsc{acc},\textsc{num:pl}\}, \textsc{n[}\textsc{ic:}5} \textsubscript{\tiny $\veebar$}\textsubscript{ 6]} ($\langle$X,$\sigma$ $\rangle$) = \textsubscript{def} $\langle$X\textit{ə}ˊ,$\sigma$ $\rangle$
\z

\ea%80
\label{ex:key:80}
 RR \textsubscript{C, \{\textsc{case:nom}} \textsubscript{\tiny $\veebar$}\textsubscript{ \textsc{acc},\textsc{num:pl}\}, \textsc{n[}\textsc{ic:}4} \textsubscript{\tiny $\veebar$}\textsubscript{ 11} \textsubscript{\tiny $\veebar$}\textsubscript{ 12} \textsubscript{\tiny $\veebar$}\textsubscript{ 14} \textsubscript{\tiny $\veebar$}\textsubscript{ 15} \textsubscript{\tiny $\veebar$}\textsubscript{ 16]} ($\langle$X,$\sigma$ $\rangle$) = \textsubscript{def} $\langle$X\textit{i}ˊ,$\sigma$ $\rangle$
\z

Auch der Dativ und der Genitiv Plural werden nicht unterschieden. Es gibt zwei Allomorphe, nämlich -\textit{ə} und -\textit{nə}:

\ea%81
\label{ex:key:81}
 RR \textsubscript{C, \{\textsc{case:dat}} \textsubscript{\tiny $\veebar$}\textsubscript{ \GEN,\textsc{num:pl}\}, \textsc{n[}\textsc{ic:}1} \textsubscript{\tiny $\veebar$}\textsubscript{ 2} \textsubscript{\tiny $\veebar$}\textsubscript{ 3} \textsubscript{\tiny $\veebar$}\textsubscript{ 4} \textsubscript{\tiny $\veebar$}\textsubscript{ 8} \textsubscript{\tiny $\veebar$}\textsubscript{ 9} \textsubscript{\tiny $\veebar$}\textsubscript{ 10} \textsubscript{${\veebar}$}\textsubscript{ 11} \textsubscript{${\veebar}$}\textsubscript{ 12} \textsubscript{${\veebar}$}\textsubscript{ 13} \textsubscript{${\veebar}$}\textsubscript{ 14} \textsubscript{${\veebar}$}\textsubscript{ 15} \textsubscript{${\veebar}$}\textsubscript{ 16]} ($\langle$X,$\sigma$ $\rangle$) = \textsubscript{def} $\langle$X\textit{ə}ˊ,$\sigma$ $\rangle$
\z

\ea%82
\label{ex:key:82}
 RR \textsubscript{C, \{\textsc{case:dat}} \textsubscript{\tiny $\veebar$}\textsubscript{ \GEN,\textsc{num:pl}\}, \textsc{n[}\textsc{ic:}5} \textsubscript{\tiny $\veebar$}\textsubscript{ 6]} ($\langle$X,$\sigma$ $\rangle$) = \textsubscript{def} $\langle$X\textit{nə}ˊ,$\sigma$ $\rangle$
\z

Im Singular wird der Genitiv in den \isi{Flexionsklassen} 1 bis 10 mit dem \isi{Suffix} -\textit{s} markiert:

\ea%83
\label{ex:key:83}
 RR \textsubscript{C, \{\textsc{case:gen},\textsc{num:sg}\}, \textsc{n[}\textsc{ic:}1} \textsubscript{\tiny $\veebar$}\textsubscript{ 2} \textsubscript{\tiny $\veebar$}\textsubscript{ 3} \textsubscript{\tiny $\veebar$}\textsubscript{ 4} \textsubscript{\tiny $\veebar$}\textsubscript{ 5} \textsubscript{\tiny $\veebar$}\textsubscript{ 6} \textsubscript{\tiny $\veebar$}\textsubscript{ 7} \textsubscript{\tiny $\veebar$}\textsubscript{ 8} \textsubscript{${\veebar}$}\textsubscript{ 9} \textsubscript{${\veebar}$}\textsubscript{ 10} \textsubscript{${\veebar}$}\textsubscript{ 11]} ($\langle$X,$\sigma$ $\rangle$) = \textsubscript{def} $\langle$X\textit{s}ˊ,$\sigma$ $\rangle$\\
\z

Schließlich müssen noch die RRs für den Singular der \isi{Flexionsklassen} 15 und 16 bestimmt werden. Es stellen sich hier einige Fragen bezüglich der \isi{Synkretismen}. Das \isi{Suffix} -\textit{a} kommt sowohl im Nominativ/Akkusativ Singular der \isi{Flexionsklassen} 15 und 16 als auch im Nominativ/Akkusativ Plural vor. der \isi{Flexionsklasse} 1. Da sich jedoch diese \isi{Suffixe} in mehr als einer Eigenschaft unterscheiden (detailliert beschrieben in \sectref{4.1.3.3}), nämlich im \isi{Numerus} und \isi{Flexionsklasse}, braucht es dafür zwei RRs: eine RRs für den ominativ/Akkusativ Plural der \isi{Flexionsklasse} 1 (\ref{ex:key:78}) und eine für den Nominativ/Akkusativ Singular der \isi{Flexionsklassen} 15 und 16:

\ea%84
\label{ex:key:84}
 RR \textsubscript{C, \{\textsc{case:nom}} \textsubscript{\tiny $\veebar$}\textsubscript{ \textsc{acc},\textsc{num:sg}\}, \textsc{n[}\textsc{ic:}15} \textsubscript{\tiny $\veebar$}\textsubscript{ 16]} ($\langle$X,$\sigma$ $\rangle$) = \textsubscript{def} $\langle$X\textit{a}ˊ,$\sigma$ $\rangle$\\
\z

Der Dativ/Genitiv Singular der \isi{Flexionsklassen} 15 und 16 fällt mit dem Dat/Gen.Pl. derselben \isi{Flexionsklassen} zusammen (-\textit{ə}). Sie unterscheiden sich also nur in einer Eigenschaft, nämlich \isi{Numerus}, folglich bräuchte es nur eine RR. Es gibt also zwei Möglichkeiten, denn alle \isi{Flexionsklassen} außer den \isi{Flexionsklassen} 5 und 6 bilden den Dativ/Genitiv Plural ebenfalls auf -\textit{ə}: a) eine RR für Dativ/Genitiv Singular und Dativ/Genitiv Plural der \isi{Flexionsklassen} 15 und 16 sowie eine RR für den Dativ/Genitiv Plural aller \isi{Flexionsklassen} außer 15 und 16 sowie 5 und 6 (anderes \isi{Suffix}), b) eine RR für Dativ/Genitiv Singular der \isi{Flexionsklassen} 15 und 16 sowie eine für den Dativ/Genitiv Plural aller \isi{Flexionsklassen} außer der \isi{Flexionsklassen} 5 und 6. Welche Analyse gewählt wird, hat also keinen Einfluss auf die Anzahl RRs. Es stellt sich folglich die Frage, was morphologisch gesehen adäquater erscheint. Mit zwei Ausnahmen markieren alle \isi{Flexionsklassen} den Dativ/Genitiv Plural mit dem \isi{Suffix} -ə, aber nur die \isi{Flexionsklassen} 15 und 16 zeigen einen \isi{Synkretismus} zwischen dem Dativ/Genitiv Singular und dem Dativ/Genitiv Plural. Deswegen wird hier die Analyse b) gewählt:

\ea%85
\label{ex:key:85}
 RR \textsubscript{C, \{\textsc{case:dat}} \textsubscript{\tiny $\veebar$}\textsubscript{ \GEN,\textsc{num:sg}\}, \textsc{n[}\textsc{ic:}15} \textsubscript{\tiny $\veebar$}\textsubscript{ 16]} ($\langle$X,$\sigma$ $\rangle$) = \textsubscript{def} $\langle$X\textit{ə}ˊ,$\sigma$ $\rangle$\\
\z

\section{Adjektive}\label{5.2}

\subsection{Allgemeines und Realisierungsregeln, starke und schwache Flexion}\label{5.2.1}

\subsubsection{Starke und schwache Flexion} Alle der hier untersuchten Varietäten unterscheiden in den Adjektiven eine starke und schwache Flexion. In welchem syntaktischen Kontext eine stark oder schwach flektierte Form verwendet wird, regelt die Syntax. Da die Distribution also syntaktisch bedingt ist, muss die Morphologie lediglich die Formen zur Verfügung stellen. Dies ist auch der Grund, weshalb keine sogenannte gemischte Flexion angenommen wird, wie dies manchmal in Bezug auf die Standardsprache gemacht wird: Mit den Paradigmen der starken und schwachen Flexion sind bereits alle Formen vorhanden.

\subsubsection{Definition der Form} In allen Varietäten dieses Samples wird in der Adjektivflexion ausschließlich suffigiert. Eine Ausnahme bilden nur die wa-/w\=o-Stäm\-me im Mittelhochdeutschen, auf die im folgenden Kapitel eingegangen wird. Die morphosyntaktischen Eigenschaften, die in den RRs definiert werden müssen, sind die folgenden: Numerus, Kasus, Genus, starke oder schwache Flexion. Mit Ausnahmen des Mittelhochdeutschen werden keine Blöcke benötigt, da nie mehr als ein Suffix auftritt. Speziell hervorzuheben ist hier nur, dass alle höchstalemannischen Dialekte (außer Visperterminen) im Plural der starken Flexion Genus unterscheiden. Dies kann sicher als Archaismus gewertet werden, da auch das Alt- und Mittelhochdeutsche eine Genusunterscheidung im Plural der starken Flexion (Althochdeutsch teils auch in der schwachen Flexion) aufweisen. In allen anderen der hier untersuchten alemannischen Dialekten sind die Genera im Plural zusammengefallen.

\subsubsection{Beispiel} Anhand der Adjektivflexion von Issime soll nun illustriert werden, wie ein System an RRs für die Adjektivflexion aussieht. \tabref{table5.15} zeigt das Paradigma der starken und schwachen Adjektivflexion in Issime, \REF{ex:key:86}-\REF{ex:key:96} die dazugehörigen RRs. 

%{\tabref{table5.15}: Starke und schwache Adjektivflexion in Issime anhand des Lexems \textit{naw}} {‘neu’ (\citealt[90-97]{Perinetto1981}, \citealt[267-268]{Zürrer1999}}\\

\begin{table}
\caption{Starke und schwache Adjektivflexion in Issime anhand des Lexems \textit{naw} ‘neu’ (\citealt[90-97]{Perinetto1981}, \citealt[267-268]{Zürrer1999})}\label{table5.15}
\resizebox{\textwidth}{!}{\begin{tabular}{*{9}{l}}
\lsptoprule
\multicolumn{9}{l}{{stark}}\\
& \multicolumn{4}{c}{\textsc{singular}} & \multicolumn{4}{c}{\textsc{plural}}\\\cmidrule(lr){2-5}\cmidrule(lr){6-9}
& \NOM & \AKK & \DAT & \GEN & \NOM & \AKK & \DAT & \GEN\\\midrule
\textsc{m} & naw-e & naw-e & naw-e & naw-s & naw-ø & naw-ø & naw-ø & naw-er\\
\textsc{n} & naw-s & naw-s & naw-s & naw-s & naw-i & naw-i & naw-i & naw-er\\
\textsc{f} & naw-ø & naw-ø & naw-ø & naw-er & naw-ø & naw-ø & naw-ø & naw-er\\
\midrule
\multicolumn{9}{l}{{schwach}} \\
& \multicolumn{4}{c}{\textsc{singular}} & \multicolumn{4}{c}{\textsc{plural}}\\\cmidrule(lr){2-5}\cmidrule(lr){6-9}
& \NOM & \AKK & \DAT & \GEN & \NOM & \AKK & \DAT & \GEN\\\midrule
\textsc{m} & naw-e & naw-e & naw-e & naw-e & \multirow{3}{*}{naw-u} & \multirow{3}{*}{naw-u} & \multirow{3}{*}{naw-e} & \multirow{3}{*}{naw-u}\\
\textsc{n} & naw-ø & naw-ø & naw-e & naw-e &  &  &  & \\
\textsc{f} & naw-u & naw-u & naw-u & naw-u &  &  &  & \\
\lspbottomrule
\end{tabular}}
\end{table}

\ea%86
\label{ex:key:86}
 RR \textsubscript{A,} \textsubscript{\{\textsc{case:nom}} \textsubscript{\tiny $\veebar$}\textsubscript{ \AKK} \textsubscript{\tiny $\veebar$}\textsubscript{ \DAT, \textsc{num:sg}, \textsc{gend:m}\},} \textsubscript{\textsc{adj[]}} ($\langle$X,$\sigma$ $\rangle$) = \textsubscript{def} $\langle$X\textit{e}ˊ,$\sigma$ $\rangle$
\z

\ea%87
\label{ex:key:87}
 RR \textsubscript{A,} \textsubscript{\{\textsc{case:nom}} \textsubscript{\tiny $\veebar$}\textsubscript{ \AKK} \textsubscript{\tiny $\veebar$}\textsubscript{ \DAT, \textsc{num:sg}, \textsc{gend:n}\},} \textsubscript{\textsc{adj[strong]}} ($\langle$X,$\sigma$ $\rangle$) = \textsubscript{def} $\langle$X\textit{s}ˊ,$\sigma$ $\rangle$
\z

\ea%88
\label{ex:key:88}
 RR \textsubscript{A,} \textsubscript{\{\textsc{case:gen}, \textsc{num:sg}, \textsc{gend:m}} \textsubscript{\tiny $\veebar$}\textsubscript{ \textsc{n}\},} \textsubscript{\textsc{adj[strong]}} ($\langle$X,$\sigma$ $\rangle$) = \textsubscript{def} $\langle$X\textit{s}ˊ,$\sigma$ $\rangle$
\z

\ea%89
\label{ex:key:89}
 RR \textsubscript{A,} \textsubscript{\{\textsc{case:gen}, \textsc{num:sg}, \textsc{gend:m}} \textsubscript{\tiny $\veebar$}\textsubscript{ \textsc{n}\},} \textsubscript{\textsc{adj[weak]}} ($\langle$X,$\sigma$ $\rangle$) = \textsubscript{def} $\langle$X\textit{e}ˊ,$\sigma$ $\rangle$ 
\z

\ea%90
\label{ex:key:90}
 RR \textsubscript{A,} \textsubscript{\{\textsc{case:dat}, \textsc{num:sg}, \textsc{gend:n}\},} \textsubscript{\textsc{adj[weak]}} ($\langle$X,$\sigma$$\rangle$) = \textsubscript{def} $\langle$X\textit{e}ˊ,$\sigma$$\rangle$
\z

\ea%91
\label{ex:key:91}
 RR \textsubscript{A,} \textsubscript{\{\textsc{case:gen}, \textsc{num:sg}, \textsc{gend:f}\},} \textsubscript{\textsc{adj[strong]}} ($\langle$X,$\sigma$$\rangle$) = \textsubscript{def} $\langle$X\textit{er}ˊ,$\sigma$$\rangle$
\z

\ea%92
\label{ex:key:92}
 RR \textsubscript{A,} \textsubscript{\{\textsc{num:sg}, \textsc{gend:f}\},} \textsubscript{\textsc{adj[weak]}} ($\langle$X,$\sigma$$\rangle$) = \textsubscript{def} $\langle$X\textit{u}ˊ,$\sigma$$\rangle$
\z

\ea%93
\label{ex:key:93}
 RR \textsubscript{A,} \textsubscript{\{\textsc{case:nom}} \textsubscript{\tiny $\veebar$}\textsubscript{ \AKK} \textsubscript{\tiny $\veebar$}\textsubscript{ \GEN, \textsc{num:pl}\},} \textsubscript{\textsc{adj[weak]}} ($\langle$X,$\sigma$$\rangle$) = \textsubscript{def} $\langle$X\textit{u}ˊ,$\sigma$$\rangle$
\z

\ea%94
\label{ex:key:94}
 RR \textsubscript{A,} \textsubscript{\{\textsc{case:dat}, \textsc{num:pl}\},} \textsubscript{\textsc{adj[weak]}} ($\langle$X,$\sigma$$\rangle$) = \textsubscript{def} $\langle$X\textit{e}ˊ,$\sigma$$\rangle$
\z

\ea%95
\label{ex:key:95}
 RR \textsubscript{A,} \textsubscript{\{\textsc{case:gen}, \textsc{num:pl}\},} \textsubscript{\textsc{adj[strong]}} ($\langle$X,$\sigma$$\rangle$) = \textsubscript{def} $\langle$X\textit{er}ˊ,$\sigma$$\rangle$
\z

\ea%96
\label{ex:key:96}
 RR \textsubscript{A,} \textsubscript{\{\textsc{case:nom}} \textsubscript{\tiny $\veebar$}\textsubscript{ \AKK} \textsubscript{\tiny  $\veebar$}\textsubscript{ \DAT, \textsc{num:pl}, \textsc{gend:n}\},} \textsubscript{\textsc{adj[strong]}} ($\langle$X,$\sigma$$\rangle$) = \textsubscript{def} $\langle$X\textit{i}ˊ,$\sigma$$\rangle$
\z

Aus den RRs wird ersichtlich, dass keine RR anzusetzen ist, wenn dem \isi{Adjektiv} keine Endung suffigiert wird (im Paradigma -ø). Hier wirkt die RR \textit{Identitiy Function Default} (vgl. RR \REF{ex:key:26} in \sectref{4.1.3.2}, die definiert, dass an der \isi{Wurzel} keine Veränderungen vorgenommen werden, wenn keine RR vorhanden ist). Diese RR muss für jede Varietät dieses Samples angenommen werden. Des Weiteren zeigen die RRs (\ref{ex:key:86}--\ref{ex:key:96}), dass nicht nur Kasus- und Genussynkretismen abgebildet werden können (z.B. Kasussynkretismus RR \REF{ex:key:86} mittels Disjunktion, \REF{ex:key:92} mittels Unterspezifikation; Genussynkretismus RR \REF{ex:key:88} mittels Disjunktion, \REF{ex:key:95} mittels Unterspezifikation), sondern auch jener \isi{Synkretismus}, wenn starke und schwache Flexion nicht unterschieden werden (\REF{ex:key:86}, mittels Unterspezifikation). Weisen die starke und schwache Flexion für ein bestimmtes Bündel an morphosyntaktischen Einheiten identische Formen auf, kann die Art der Flexion (stark/schwach) unterspezifiziert bleiben. Dasselbe gilt, wenn \isi{Numerus} nicht unterschieden würde, was in der Adjektivflexion von Issime nicht vorkommt.

\subsection{Wa-/w\=o-Stämme}\label{5.2.2}

Wa-/w\=o-Stäm\-me sind nur im Alt- und Mittelhochdeutschen vorhanden, die wie die wa-/w\=o-Stäm\-me der \isi{Substantive} behandelt werden (vgl. \sectref{5.1.3}). Im Althochdeutschen ist die Vokalisierung von \textit{w} zu \textit{o} phonologisch bedingt (also keine RR), während im Mittelhochdeutschen die Tilgung des \textit{w} aus Mangel an einer möglichen synchronen phonologischen Erklärung in der Morphologie zu verorten ist.

Aus \tabref{table5.16} wird ersichtlich, dass im Althochdeutschen die wa-/w\=o-Stäm\-me dieselben Flexionsendungen aufweisen wie die a-Stäm\-me. Dafür, dass es sich beim auslautenden \textit{o} im Nominativ Singular der wa-/w\=o-Stäm\-me nicht um eine Flexionsendung handelt, sprechen zwei Gründe. Erstens weisen auch a-Stäm\-me keine Endung auf (\textit{blint} ‘blind’) und da das Set an \isi{Suffixen} der wa-/w\=o-Stäm\-me jedem der a-Stäm\-me entspricht, würde es nur wenig Sinn machen, im Nominativ von einer Ausnahme auszugehen. Vielmehr wissen wir aus der Diskussion zu den \isi{Substantiven}, dass \textit{w} zu \textit{o} vokalisiert wird, wenn es im Auslaut steht. Es passt also ins Gesamtsystem, wenn man annimmt, dass aus *\textit{garw} > \textit{garo} ‘bereit’ entsteht. Zweitens wird die \isi{Wurzel} (\textit{garw}-) verwendet, wenn ein \isi{Suffix} folgt, d.h. auch ein \isi{Suffix} auf -\textit{o}. Dafür spricht die Form \textit{garawu} des Instrumentals. Die w- und a-Stäm\-me fallen folglich zusammen und die Vokalisierung ist phonologisch bedingt, weshalb dafür keine RR nötig ist. Der Sprossvokal \textit{a} (\textit{gara}\textit{w}-) und weshalb \textit{w} zu \textit{o} (und nicht zu \textit{u}) vokalisiert wird, wurde genauer in \sectref{5.1.3} besprochen.

%{\tabref{table5.16}: Wa-/w\=o-Stämme im Althochdeutschen am Beispiel der Lexeme \textit{garo}} {‘bereit’ und \textit{blint}} {‘blind’ \citep[220, 225]{Braune2004}}\\

\begin{table}
\caption{ Wa-/w\=o-Stämme im Althochdeutschen am Beispiel der Lexeme \textit{garo} ‘bereit’ und \textit{blint} ‘blind’ \citep[220, 225]{Braune2004}}\label{table5.16}
\resizebox{\textwidth}{!}{\begin{tabular}{llllll}
\lsptoprule
\multicolumn{6}{l}{starke Flexion, Singular, Maskulin, wa-/w\=o-Stämme}\\
& {\NOM} & {\AKK} & {\DAT} & {\GEN} & {\INSTR}\\\midrule
{wa-/w\=o-Stämme} & garaw-\=er/garo & garaw-an & garaw-emo & garaw-es & garaw-u\\
{a-Stämme} & blint-\=er/blint & blint -an & blint -emo & blint -es & blint -u\\
\lspbottomrule
\end{tabular}}
\end{table}

Im Mittelhochdeutschen ist die \isi{Variation} nicht phonologisch bedingt, was bereits in \sectref{5.1.3} erörtert wurde. Es wurde gezeigt, dass weder eine Tilgung von \textit{w} noch eine Einfügung von \textit{w} phonologisch voraussagbar ist. Folglich ist die \isi{Variation} durch eine RR zu definieren. Wie bei den \isi{Substantiven} wird auch bei den \isi{Adjektiven} davon ausgegangen, dass aus dem \isi{Radikon} auf \textit{w} auslautende \isi{Wurzeln} kommen. Nachdem durch RRs Endungen suffigiert worden sind (\isi{Block} A), wird in einem zweiten \isi{Block} (B) eine RR benötigt, die \textit{w} tilgt, wenn es im Auslaut steht:

\ea%97
\label{ex:key:97}
 RR \textsubscript{B,} \textsubscript{\{\},} \textsubscript{\textsc{adj[]}} ($\langle$X,$\sigma$ $\rangle$) = \textsubscript{def} $\langle$X *\textit{w} $\rightarrow$ ø/\_\#ˊ,$\sigma$ $\rangle$\\
\z

\subsection{Freie Variation}\label{5.2.3}

In etlichen der hier untersuchten Varietäten kommen zwei verschiedene Formen in derselben Zelle des Paradigmas vor. Da die Distribution nicht weiter erklärt werden kann, ist von freier \isi{Variation} auszugehen. Es können prinzipiell zwei Arten von freier \isi{Variation} unterschieden werden, deren RRs aber gleich aussehen: Erstens weist eine Zelle des Paradigmas zwei \isi{Suffixe} auf, zweitens weist eine Zelle ein \isi{Suffix} und zusätzlich nur die \isi{Wurzel} auf. Dies soll anhand der starken Flexion im Althochdeutschen dargestellt werden.

\tabref{table5.17} zeigt, dass an die a-Stäm\-me im Nominativ Singular -\textit{er} (pronominale Endung) oder auch nichts (nominal) suffigiert werden kann. An die i-Stäm\-me können -\textit{er} (pronominale Endung) oder -\textit{i} (nominale Endung) suffigiert werden.

%{\tabref{table5.17}: Freie \isi{Variation} in der starken Flexion des Althochdeutschen \citep[220, 223]{Braune2004}}\\

\begin{table}
\caption{Freie Variation in der starken Flexion des Althochdeutschen \citep[220, 223]{Braune2004}}\label{table5.17}
\begin{tabular}{ll}
\lsptoprule
& \textsc{nom.sg.}\\\midrule
{a-Stämme} & blint-\=er / blint ‘blint’\\
{i-Stämme} & m\=ar-\=er / m\=ar-i ‘berühmt’\\
\lspbottomrule
\end{tabular}
\end{table}

Wie bereits für die \isi{Substantive} dargestellt (vgl. \sectref{5.1.2}), wird auch eine RR für Fälle wie \textit{blint} benötigt, welche gleich spezifisch sein muss wie die RR für das \isi{Suffix} -\textit{er} (\textit{blint-er}). Nur so können beide RRs angewendet werden und definieren zwei Formen für dieselbe Zelle des Paradigmas (vgl. \sectref{4.1.3.3}).

Schließlich ist noch interessant zu erwähnen, dass freie \isi{Variation} mit zwei \isi{Suffixen} nur im Althochdeutschen und im Alemannischen von Huzenbach vorkommt (vgl. \tabref{table5.18}). Die \isi{Variation} mit \isi{Wurzel} und \isi{Suffix} ist häufiger: Sie tritt Alt-, Mittelhochdeutsch, Jaun, Sensebezirk, Stuttgart, Kaiserstuhl, Colmar auf. Außerdem sind von der freien \isi{Variation} ausschließlich Nominativ und/oder Akkusativ Singular und/oder Plural betroffen. Tritt freie \isi{Variation} im Nominativ und Akkusativ Plural auf, ist sie auf das Feminin beschränkt. Im Singular können keine Verallgemeinerung bezüglich des \isi{Genus} gemacht werden.\is{Suffix}

%{\tabref{table5.18}: Freie \isi{Variation} in den untersuchten Varietäten}\\
\begin{table}
\caption{ Freie Variation in den untersuchten Varietäten}\label{table5.18}
\begin{tabularx}{\textwidth}{lQX}
\lsptoprule
Varietät & {Suffix}\slash Stamm & {Suffix}\slash {Suffix}\\\midrule
Althochdeutsch & stark: \textsc{nom.sg.m.+n.+f.}; \textsc{akk.sg.n.} & stark: \textsc{nom.sg.m.+n.+f.}; \textsc{akk.sg.n.}\\
Mittelhochdeutsch & stark: \textsc{nom.sg.m.+n.+f.}; \textsc{akk.sg.n.} & \\
Jaun & stark: \textsc{nom./akk.pl.f.} schwach: \textsc{nom./akk.sg.m.+n.} &  \\
Sensebezirk & stark: \textsc{nom./akk.pl.f.} & \\
Huzenbach & & stark: \textsc{nom.sg.m.} \\
Stuttgart & schwach: \textsc{nom./akk.sg.f.}  & \\
Kaiserstuhl & stark: \textsc{nom./akk.sg.n.} & \\
Colmar & schwach:    \textsc{nom./akk.sg.n.+f.} & \\
\lspbottomrule
\end{tabularx}
\end{table}



\section{Personalpronomen}\label{5.3}

\subsection{Allgemeines und Realisierungsregeln}\label{5.3.1}

{Morphosyntaktische Eigenschaften}: In den \isi{Personalpronomen} werden folgende morphosyntaktischen Eigenschaften unterschieden: \isi{Numerus}, \isi{Kasus}, Person, \isi{Genus}, be\-tont/un\-be\-tont, be\-lebt/un\-be\-lebt. Die \isi{Personalpronomen} aller Varietäten in diesem Sample differenzieren \isi{Numerus}, \isi{Kasus}, Person und \isi{Genus} in der 3. Person Singular. Eine Genusunterscheidung in der 3. Person Plural ist nur im Alt- und Mittelhochdeutschen sowie im Alemannischen von Issime und Jaun (beide\pagebreak[4] Höchstalemannisch) zu beobachten. Alle Varietäten außer der Standardsprache weisen jeweils ein Paradigma für das betonte und unbetonte \isi{Personalpronomen} auf, wobei diese unterschiedlich vollständig sind. Darauf wird in \sectref{5.3.2} genauer eingegangen. \isi{Belebtheit} wird nur in der 3. Person Singular Neutrum und nur in einigen Dialekten unterschieden. Dies wird in \sectref{5.3.3} erörtert.

{Definition der Form}: Die Formen der \isi{Personalpronomen} aller Varietäten müssen durch RRs definiert werden, weil sie nicht weiter unterteilbar sind, d.h., nicht weiter unterteilbar in eine \isi{Wurzel} und \isi{Affixe}. Somit sind die \isi{Personalpronomen} vergleichbar mit Kasus- oder Numerussuffixen der \isi{Substantive} oder \isi{Adjektive}. Dies stellt weder für das der Messmethode zugrunde liegende Modell noch für die Messmethode selbst ein Problem dar, wie dies am Anfang dieses Kapitels beschrieben wurde. Jede Form jeder Zelle wird also durch eine RR definiert. Folglich befinden sich alle RR in demselben \isi{Block}. Eine Ausnahme hiervon bilden die Doppelformen des Plurals im Alemannischen von Issime. Diese sollen im folgenden Abschnitt beschrieben werden. Außerdem soll anhand des Plurals der \isi{Personalpronomen} von Issime ein System an RRs für das \isi{Personalpronomen} gezeigt werden.

{Zusammengesetzte Formen in Issime}: \tabref{table5.19} stellt das Paradigma des betonten \isi{Personalpronomens} im Plural von Issime dar. Jede Person weist sowohl eine einfache Form (z.B. \textit{wir}) als auch eine Doppelform auf (z.B. \textit{wirendri}), welche in unterschiedlichen Kontexten verwendet werden \citep[216-221]{Zürrer1999}. Die Doppelform setzt sich aus dem einfachen \isi{Personalpronomen} und dem Indefinitpronomen \textit{andere} zusammen. Die einfache Form ist aus dem Althochdeutschen ererbt, die Doppelform aus dem Piemontesischen, Frankoprovenzalischen, gesprochenen Französischen und/oder Italienischen nachgebildet \citep[215]{Zürrer1999}, z.B. Piemontesisch \textit{noj-autri} \citep[72]{BreroBertodatti1988}. Wie in \sectref{3.3.3} beschrieben wurde, werden alle vier Sprachen im Aostatal gesprochen. Im Italienischen und Frankoprovenzalischen gibt es solche Doppelformen in der 1. und 2. Person Plural, im Französischen und Piemontesischen auch in der 3. Person Plural \citep[215]{Zürrer1999}. Im alemannischen Dialekt von Issime wurde nicht nur das Muster zur Bildung solcher Doppelformen übernommen, sondern auch in das bereits vorhandene Kasussystem integriert.

%{\tabref{table5.19}: Betontes Pluralparadigma des \isi{Personalpronomens} in Issime \citep[206-312]{Zürrer1999}}\\

\begin{table}
\caption{Betontes Pluralparadigma des Personalpronomens in Issime \citep[206--312]{Zürrer1999}}\label{table5.19}
\begin{tabular}{lllll}
\lsptoprule
{\textsc{person}} & {\textsc{nom}} & {\textsc{akk}} & {\textsc{dat}} & {\textsc{gen}}\\
\midrule
& \multicolumn{4}{c}{einfaches Personalpronomen}\\\midrule
1. & wir & ündsch & ündsch & ündsch-uru\\
2. & ir & auw & auw & auw-uru\\
3. & dschi & dschi & ürj-u & ürj-u, ürj-uru\\
\midrule
& \multicolumn{4}{c}{zusammengesetztes Personalpronomen}\\\midrule
1. & wir-endri & ündsch-endri & ündsch-enandre & ündsch-erandru\\
2. & ir-endri & auw-endri & auw-enandre & auw-erandru\\
3. & dschi-endri & dschi-endri & ürj-enandre & ürj-erandru\\
\lspbottomrule
\end{tabular}
\end{table}

Da die Doppelformen aus zwei Teilen bestehen, werden auch zwei \isi{Blöcke} benötigt, damit z.B. \textit{wir-endri} definiert ist und nicht \textit{endri-wir}. Des Weiteren sind alle Paradigmen auf ein Minimum zu reduzieren, was auch für die Doppelformen gilt. Beispielsweise wäre der Dativ in vier \isi{Affixe} einteilbar: einfache Form + \textit{en} + \textit{andr} + \textit{e} (vier RRs). Kürzer fällt jedoch die Beschreibung aus, wenn man von der einfachen Form + \textit{enandre} ausgeht (zwei RRs). Da \textit{enandre} im gesamten Dativ unabhängig von Person vorkommt, kann \textit{enandre} als Dativmarker analysiert werden. Die Analyse einfache Form + \textit{enandre} bildet also den Dativ adäquat ab und ist die kürzeste Beschreibung.

Es sollen nun die RRs für den Plural des betonten \isi{Personalpronomens} von Issime gezeigt werden. Aus Gründen der Anschaulichkeit entspricht das System an RRs nicht ganz jenem, das zur Komplexitätsmessung verwendet wird (vgl. RRs im Anhang B), da die unbetonten \isi{Personalpronomen} hier ausgelassen werden. Zuerst sind die einfachen Formen zu definieren, die auch in den Doppelformen vorkommen:

\ea%98
\label{ex:key:98}
 RR \textsubscript{A,} \textsubscript{\{\textsc{case:nom}, \textsc{num:pl}, \textsc{pers:1}\},} \textsubscript{\textsc{pron.pers[stress:+]}} ($\langle$X,$\sigma$$\rangle$) = \textsubscript{def} $\langle$\textit{wir}ˊ,$\sigma$$\rangle$
\z

\ea%99
\label{ex:key:99}
 RR \textsubscript{A,} \textsubscript{\{\textsc{case:nom}, \textsc{num:pl}, \textsc{pers:2}\},} \textsubscript{\textsc{pron.pers[stress:+]}} ($\langle$X,$\sigma$$\rangle$) = \textsubscript{def} $\langle$\textit{ir}ˊ,$\sigma$$\rangle$
\z

\ea%100
\label{ex:key:100}
 RR \textsubscript{A,} \textsubscript{\{\textsc{case:nom}} \textsubscript{\tiny $\veebar$}\textsubscript{ \AKK, \textsc{num:pl}, \textsc{pers:3}\},} \textsubscript{\textsc{pron.pers[stress:+]}} ($\langle$X,$\sigma$$\rangle$) = \textsubscript{def} $\langle$\textit{dschi}ˊ,$\sigma$ $\rangle$
\z

\ea%101
\label{ex:key:101}
 RR \textsubscript{A,} \textsubscript{\{\textsc{case: akk}}\textsubscript{ ${\veebar}$}\textsubscript{ \DAT} \textsubscript{\tiny $\veebar$}\textsubscript{ \GEN, \textsc{num:pl}, \textsc{pers:1}\},} \textsubscript{\textsc{pron.pers[stress:+]}} ($\langle$X,$\sigma$$\rangle$) = \textsubscript{def} $\langle$\textit{ündsch}ˊ,$\sigma$$\rangle$
\z

\ea%102
\label{ex:key:102}
 RR \textsubscript{A,} \textsubscript{\{\textsc{case: akk}}\textsubscript{ ${\veebar}$}\textsubscript{ \DAT} \textsubscript{\tiny $\veebar$}\textsubscript{ \GEN, \textsc{num:pl}, \textsc{pers:2}\},} \textsubscript{\textsc{pron.pers[stress:+]}} ($\langle$X,$\sigma$$\rangle$) = \textsubscript{def} $\langle$\textit{auw}ˊ,$\sigma$$\rangle$
\z

\ea%103
\label{ex:key:103}
 RR \textsubscript{A,} \textsubscript{\{\textsc{case: dat}} \textsubscript{\tiny $\veebar$}\textsubscript{ \GEN, \textsc{num:pl}, \textsc{pers:3}\},} \textsubscript{\textsc{pron.pers[stress:+]}} ($\langle$X,$\sigma$$\rangle$) = \textsubscript{def} $\langle$\textit{ürj}ˊ,$\sigma$$\rangle$
\z

\noindent
Es folgen die Dativ- und Genitivsuffixe für die einfache Form (\isi{Block} B):

\ea%104
\label{ex:key:104}
 RR \textsubscript{B,} \textsubscript{\{\textsc{case: dat}} \textsubscript{\tiny $\veebar$}\textsubscript{ \GEN, \textsc{num:pl}, \textsc{pers:3}\},} \textsubscript{\textsc{pron.pers[stress:+}, \textsc{form:simple]}} ($\langle$X,$\sigma$ $\rangle$) = \textsubscript{def} $\langle$X\textit{u}ˊ,$\sigma$ $\rangle$
\z

\ea%105
\label{ex:key:105}
 RR \textsubscript{B,} \textsubscript{\{\textsc{case: gen}, \textsc{num:pl}\},} \textsubscript{\textsc{pron.pers[stress:+}, \textsc{form:simple]}} ($\langle$X,$\sigma$ $\rangle$) = \textsubscript{def} $\langle$X\textit{uru}ˊ,$\sigma$ $\rangle$
\z

Auch die RRs für den zweiten Teil der Doppelformen stehen in \isi{Block} B. Dies ist möglich, da in den RRs die Form als \textit{simple} oder \textit{composed} klar definiert ist sowie der zweite Teil der Doppelformen und die Dativ-/Genitivsuffixe nie zusammen an dasselbe Wort suffigiert werden:

\ea%106
\label{ex:key:106}
 RR \textsubscript{B,} \textsubscript{\{\textsc{case:} \textsc{nom}} \textsubscript{\tiny $\veebar$}\textsubscript{ \AKK, \textsc{num:pl}\},} \textsubscript{\textsc{pron.pers[stress:+}, \textsc{form:composed]}} ($\langle$X,$\sigma$$\rangle$) = \textsubscript{def} $\langle$X\textit{endri}ˊ,$\sigma$$\rangle$
\z

\ea%107
\label{ex:key:107}
 RR \textsubscript{B,} \textsubscript{\{\textsc{case: dat}, \textsc{num:pl}\},} \textsubscript{\textsc{pron.pers[stress:+}, \textsc{form:composed]}} ($\langle$X,$\sigma$$\rangle$) = \textsubscript{def} $\langle$X\textit{enandre}ˊ,$\sigma$$\rangle$
\z

\ea%108
\label{ex:key:108}
 RR \textsubscript{B,} \textsubscript{\{\textsc{case: gen}, \textsc{num:pl}\},} \textsubscript{\textsc{pron.pers[stress:+}, \textsc{form:composed]}} ($\langle$X,$\sigma$$\rangle$) = \textsubscript{def} $\langle$X\textit{erandru}ˊ,$\sigma$$\rangle$
\z

In den RRs (\ref{ex:key:105}--\ref{ex:key:108}) bleibt Person unterspezifiziert, da diese RRs die 1.-3. Person definieren. Dasselbe gilt für die RRs (\ref{ex:key:98}--\ref{ex:key:103}), die die Form (\textit{simple}/\textit{composed}) nicht spezifizieren. Weil auch Betonung ein binärer Parameter ist, verhält er sich wie z.B. \isi{Numerus} und Form: Weist eine Zelle für das betonte und unbetonte \isi{Personalpronomen} dieselbe Form auf, ist der Parameter Betonung unterspezifiziert.

Schließlich muss noch hervorgehoben werden, dass die einfachen und zusammengesetzten Formen nicht in freier \isi{Variation} (d.h. beide Formen in einer Zelle) sind, sondern jeweils ein eigenes Paradigma bilden (vgl. \tabref{table5.19}). Würden beide Formen in derselben Zelle stehen, müsste für alle einfachen Formen, an die in \isi{Block} B kein weiteres Material suffigiert wird, eine RR in \isi{Block} B angesetzt werden, die definiert, dass nichts suffigiert wird (vgl. Diskussion zur freien \isi{Variation} in den Abschnitten \sectref{5.1.2} und \sectref{5.2.3}). Das würde die Beschreibung des Systems deutlich verlängern, also Komplexität hinzufügen. Um dies verständlicher zu machen, sollen anhand der 1. Person Plural Nominativ beide Möglichkeiten (freie \isi{Variation}, zwei Paradigmen) aufgezeigt werden.

Angenommen, die einfache (\textit{wir}) und zusammengesetzte (\textit{wirendri}) Form stünden in derselben Zelle, dann sind folgende RRs nötig:\largerpage[2]

% \ea%98
\begin{exe}[(98)]
\exr{ex:key:98}
 RR \textsubscript{A,} \textsubscript{\{\textsc{case:nom}, \textsc{num:pl}, \textsc{pers:1}\},} \textsubscript{\textsc{pron.pers[stress:+]}} ($\langle$X,$\sigma$$\rangle$) = \textsubscript{def} $\langle$\textit{wir}ˊ,$\sigma$$\rangle$
\end{exe}

\begin{exe}[(106.1)]%106.1
\exspecial{106} \label{ex:key:106.1}
RR \textsubscript{B,} \textsubscript{\{\textsc{case:} \textsc{nom}} \textsubscript{\tiny $\veebar$}\textsubscript{ \AKK, \textsc{num:pl}\},} \textsubscript{\textsc{pron.pers[stress:+}, \textsc{form:composed]}} ($\langle$X,$\sigma$$\rangle$) = \\\textsubscript{def} $\langle$X\textit{endri}ˊ,$\sigma$$\rangle$
\end{exe}

\begin{exe}[(106.2)]%106.2
\exspecial{106} \label{ex:key:106.2}
RR \textsubscript{B,} \textsubscript{\{\textsc{case:} \textsc{nom}} \textsubscript{\tiny $\veebar$}\textsubscript{ \AKK, \textsc{num:pl}\},} \textsubscript{\textsc{pron.pers[stress:+}, \textsc{form:simple]}} ($\langle$X,$\sigma$$\rangle$) = \textsubscript{def} $\langle$Xˊ,$\sigma$$\rangle$
\end{exe}

In \isi{Block} A wird definiert, dass in der Zelle Nominativ \textit{wir} steht \REF{ex:key:98}, in \isi{Block} B, dass der zusammengesetzten Form -\textit{endri} suffigiert wird \REF{ex:key:106.1} und dass der einfachen Form nichts suffigiert wird \REF{ex:key:106.2}. Würde die RR \REF{ex:key:106.2} nicht angenommen, wäre in der Zelle Nominativ keine einfache Form definiert. Es braucht also drei RRs, damit gewährleistet ist, dass in derselben Zelle eine einfache und eine zusammengesetzten Form stehen.

Angenommen, die einfachen und zusammengesetzten \isi{Personalpronomen} haben jeweils ein eigenes Paradigma, so sind nur die RRs \REF{ex:key:98} und \REF{ex:key:106} nötig (vgl. \tabref{table5.18}). Die RR \REF{ex:key:98} definiert \textit{wir} für beide Paradigmen (der Parameter \textit{Form} ist unterspezifiziert) und die RR \REF{ex:key:106} das \isi{Suffix} -\textit{endri} für das zusammengesetzte \isi{Personalpronomen}. Benötigt werden also nur zwei RRs, weshalb diese Analyse zu bevorzugen ist.

\subsection{Betont und unbetont}\label{5.3.2}

Alle der hier untersuchten Varietäten außer der Standardsprache haben betonte und unbetonte \isi{Personalpronomen}, d.h., in einem betonten Kontext wird eine andere Form des \isi{Personalpronomens} verwendet als in einem unbetonten Kontext. Bei der unbetonten Variante handelt es sich meistens um eine phonetisch reduzierte Form im Vergleich zur betonten Form. Die unbetonten Formen können also als \textit{special clitics} bezichnet werden \citep[510--511]{ZwickyPullum1983}. Da diese phonetische Reduktion jedoch nicht in der gesamten Sprache gilt, es sich also nicht um Regeln handelt, die automatisch auf das gesamte System angewendet werden, bilden die unbetonten \isi{Personalpronomen} ein zusätzliches Paradigma neben jenem der betonten \isi{Personalpronomen}. Sie sind folglich ebenfalls durch RRs zu definieren. Formalisiert wird die Betonung durch den Parameter \textit{Stress} (Betonung), welcher die Ausprägung + oder - haben kann.

Bereits das Alt- und Mittelhochdeutsche weisen unbetonten \isi{Personalpronomen} auf, jedoch nur in der 3. Person Singular und Plural (vgl. Paradigmen 41 und 42). Alle alemannischen Dialekte mit Ausnahme von Elisabethtal haben aber ein vollständiges Paradigma mit unbetonten \isi{Personalpronomen}, d.h. auch in der 1. und 2. Person Singular und Plural. Dies kann als Ausbau und Grammatikalisierung der Kategorie unbetonter \isi{Personalpronomen} interpretiert werden, während die Standardsprache diese Kategorie abgebaut hat. Erstaunlicherweise existieren im Dialekt von Elisabethtal unbetonte Formen nur in der 3. Person Singular Maskulin und Neutrum.\largerpage[2] Da alle anderen schwäbischen Dialekte dieses Samples ein vollständiges Paradigma für die unbetonten \isi{Personalpronomen} haben, kann dies als Abbau gewertet werden.\footnote{Denkbar wäre natürlich auch, dass die Beschreibung unvollständig ist. Da jedoch \citeauthor{Žirmunskij1928/29}s \citeyearpar{Žirmunskij1928/29} Ausführungen sonst eine sehr hohe Genauigkeit aufweisen, kann davon ausgegangen werden, dass auch diese adäquat ist.}

\subsection{Belebt und unbelebt}\label{5.3.3}

{Dialekte}: Einige alemannische Dialekte unterscheiden in der 3. Person Singular Neutrum zwischen belebt und unbelebt. Es handelt sich um folgende Dialekte: Jaun, Sensebezirk, Uri (also alle höchstalemannischen Dialekte außer der Walser Dialekte), Bern und Zürich (also alle hochalemannischen Dialekte außer Vorarlberg), Kaiserstuhl und Elsass (Ebene) (Oberrheinalemannisch). In keinem der schwäbischen Dialekte kommt dieses Phänomen vor. Die Dialekte von Jaun, Bern und des Sensebezirks differenzieren \isi{Belebtheit} im betonten und unbetonten Paradigma, die Dialekte von Uri, Zürich, des Kaiserstuhls und des Elsass (Ebene) nur im betonten Paradigma.

{Das System}: In den Dialekten, die belebte und unbelebte Formen unterscheiden, wird das Neutrum verwendet, um sich auf weibliche Menschen zu beziehen (für männliche Menschen wird das Maskulin verwendet). In den Walser Dialekten, die belebt und unbelebt nicht differenzieren, bezieht man sich mit dem Neutrum auf weibliche und männliche Menschen. \tabref{table5.20} zeigt das Paradigma der 3. Person Singular Neutrum belebt und unbelebt des Dialektes von Bern.

%{\tabref{table5.20}: 3. Person Singular Neutrum belebt und unbelebt in Bern \citep[92-97]{Marti1985}}\\

\begin{table}
\caption{3. Person Singular Neutrum belebt und unbelebt in Bern \citep[92-97]{Marti1985}}\label{table5.20}
\begin{tabular}{lllll} 
\lsptoprule
&  & {\NOM} & {\AKK} & {\DAT}\\
\midrule
{Singular} & {3.\textsc{n}.unbelebt} & ǣs & ǣs & \=im\\
& {3.\textsc{n}.belebt} & ǣs & \=ins & \=im\\
\lspbottomrule
\end{tabular}
\end{table}

Die unbelebten Formen sind jene, die für eine 3. Person Singular Neutrum zu erwarten sind: Nominativ und Akkusativ fallen zusammen, während der Dativ eine eigene Form hat. In der belebten Form werden auch Nominativ und Akkusativ unterschieden. Man kann sich nun fragen, ob die belebten Formen zum Neutrum gehören oder ob sie neben Maskulin, Feminin und Neutrum ein eigenes \isi{Genus} bilden. Die grundsätzlichere Frage lautet also, mit welchem Kriterium \isi{Genus} ermittelt werden kann. \citet{Corbett1991} schlägt Kongruenz vor: „[…] the determining criterion of gender is agreement […]. Saying that a language has three genders implies that there are three classes of nouns which can be distinguished syntactically by the agreements they take“ \citep[4]{Corbett1991}. Da die Wörter, die durch die belebten Pronomen ersetzt werden können, Kongruenz im Neutrum aufweisen, gehören die belebten Pronomen zum Neutrum. Bevor versucht wird, dieses Phänomen einzuordnen und zu erklären, wird nun zuerst gezeigt, woher die Akkusativform des belebten Neutrums stammen könnte und anschließend, weshalb der Dialekt des Sensebezirks einen Sonderfall darstellt.

{Ursprung des belebten Akkusativs}: Es stellt sich also die Frage, woher die Form des Akkusativs Neutrum belebt stammt. Eine mögliche Erklärung ist, dass sie vom Akkusativ Maskulin der 3. Person Singular abgeleitet ist, da ihre Formen die größten Ähnlichkeiten aufweisen, was in \tabref{table5.21} zusammengefasst ist. Der einzige Unterschied zwischen den beiden Formen ist das auslautende \textit{s} im Neutrum. Es könnte davon ausgegangen werden, dass \textit{s} als eine Art Default-Marker für das Neutrum fungiert (z.B. \textit{das}, \textit{meinəs}, \textit{schönəs} etc.) und somit an die Maskulinform angehängt wird, um eine Neutrumform zu bilden. Dies entspricht der Erklärung, die \citet{Stucki1917} in Bezug auf den Dialekt von Jaun gibt \citep[281]{Stucki1917}.

%{\tabref{table5.21}: Herkunft der Akkusativform der 3. Person Singular Neutrum unbelebt}\\

\begin{table}
\caption{Herkunft der Akkusativform der 3. Person Singular Neutrum unbelebt}\label{table5.21}
\begin{tabular}{lll} 
\lsptoprule
& \textsc{3.sg.m.akk} & \mbox{\textsc{3.sg.n.}unbelebt.\AKK}\\
\midrule
Jaun & ẽ & ẽs\\
Uri & inɐ & inəss\\
Zürich & in & ins\\
Bern & \=in & \=ins\\
Kaiserstuhl & \=inɐ & \=inəs\\
Elsass (Ebene) & inə & inəs\\
\lspbottomrule
\end{tabular}
\end{table}

Alle Dialekte außer das Alemannische des Sensebezirks, die \isi{Belebtheit} unterscheiden, weisen parallele Formen zum Paradigma von Bern auf, d.h. unbelebt Nominativ=Akkusativ${\neq}$Dativ und belebt Nominativ${\neq}$Akkusativ${\neq}$Dativ. Dabei handelt es sich bei den unbelebten Formen um jene für ein Neutrum zu erwartende Formen und die belebten Formen weisen die in \tabref{table5.21} gelisteten Akkusativformen auf. 

{Das System im Dialekt des Sensebezirks}: Das Paradigma des Sensebezirks\linebreak weicht davon ab (vgl. \tabref{table5.22}). Zwar wird auch hier zwischen belebten und unbelebten Formen im Neutrum unterschieden, aber ein anderer Wandel ist hier eingetreten: In den meisten nominalen Wortarten wurde die Akkusativ- durch die Dativform ersetzt \citep{Bucheli2010}. Dieser Wandel ist in die Formen des unbelebten Paradigmas eingedrungen, jedoch nicht in jene des belebten Paradigmas. Im belebten Paradigma ist also der alte Akkusativ erhalten, während er im unbelebten Paradigma durch den Dativ ersetzt wurde. Dieser Dialekt hat folglich keine spezielle Form für den belebten Akkusativ grammatikalisiert. Zusammenfassend kann man also festhalten, dass im Dialekt des Sensebezirks der Akkusativ der 3. Person Singular Neutrum unbelebt wie der Dativ lautet und der Akkusativ der 3. Person Singular Neutrum belebt wie der Nominativ.

%{\tabref{table5.22}: 3. Person Singular Neutrum belebt und unbelebt im Sensebezirk \citep[196-198]{Henzen1927}}\\

\begin{table}
\caption{3. Person Singular Neutrum belebt und unbelebt im Sensebezirk \citep[196-198]{Henzen1927}}\label{table5.22}
\begin{tabular}{lllll} 
\lsptoprule
&  & {\NOM} & {\AKK} & {\DAT}\\
\midrule
{Singular} & \textsc{3.n.}unbelebt & æs & \=im & \=im\\
& \textsc{3.n.}belebt & æs & æs & \=im\\
\lspbottomrule
\end{tabular}
\end{table}

{RRs und Parameter \textit{Animacy}}: Wenn also \isi{Belebtheit} unterschieden wird, muss dies durch die RRs definiert werden. Neben den Parametern zur Betonung und zu den morphosyntaktischen Eigenschaften muss folglich noch ein Parameter \isi{Belebtheit} spezifiziert werden (ANIM für \textit{Animacy}). Folgende RRs bilden das Paradigma der \tabref{table5.22} ab:

\ea%109
\label{ex:key:109}
 RR \textsubscript{A,} \textsubscript{\{\textsc{case: nom}, \textsc{num:sg}, \textsc{pers:3}, \textsc{gend:n}\},} \textsubscript{\textsc{pron.pers[stress:+]}} ($\langle$X,$\sigma$ $\rangle$) = \textsubscript{def} $\langle$X\textit{æs}ˊ,$\sigma$ $\rangle$
\z

\ea%110
\label{ex:key:110}
 RR \textsubscript{A,} \textsubscript{\{\textsc{case: dat}, \textsc{num:sg}, \textsc{pers:3}, \textsc{gend:n}\},} \textsubscript{\textsc{pron.pers[stress:+]}} ($\langle$X,$\sigma$$\rangle$) = \textsubscript{def} $\langle$X\textit{\=im}ˊ,$\sigma$$\rangle$
\z

\ea%111
\label{ex:key:111}
 RR \textsubscript{A,} \textsubscript{\{\textsc{case: akk}, \textsc{num:sg}, \textsc{pers:3}, \textsc{gend:n}, \textsc{anim:-}\},} \textsubscript{\textsc{pron.pers[stress:+]}} ($\langle$X,$\sigma$$\rangle$) = \textsubscript{def} $\langle$X\textit{\=im}ˊ,$\sigma$$\rangle$
\z

\ea%112
\label{ex:key:112}
 RR \textsubscript{A,} \textsubscript{\{\textsc{case: akk}, \textsc{num:sg}, \textsc{pers:3}, \textsc{gend:n}, \textsc{anim:+}\},} \textsubscript{\textsc{pron.pers[stress:+]}} ($\langle$X,$\sigma$$\rangle$) = \textsubscript{def} $\langle$X\textit{æs}ˊ,$\sigma$$\rangle$
\z

Es soll nun ein Versuch vorgenommen werden, dieses Phänomen zu erklären, da es meines Wissens in keinem anderen deutschen Dialekt vorkommt. Dazu soll zuerst ein Vergleich mit älteren und jüngeren Sprachstufen angestellt werden, die mit dem Deutschen eng verwandt sind. Anschließend werden die Ergebnisse erklärt.

{Sprachvergleich}: In der niederländischen und englischen Standardsprache fallen in der 3. Person Singular Neutrum des \isi{Personalpronomens} der Nominativ und der Akkusativ zusammen, engl. \textit{it}, niederl. \textit{het} \citep[36]{GabrielRoodzant2010}. Mit dem Neutrum werden unbelebte Entitäten pronominalisiert, mit dem Maskulin und Feminin wird auf belebte Entitäten verwiesen, engl. \textit{he}\slash\textit{she}, niederl. \textit{hij}\slash \textit{zij} \citep[36]{GabrielRoodzant2010}. Interessanterweise wird in den belebten \isi{Personalpronomen} zwischen Subjekt- und Objektkasus unterschieden, in den unbelebten Pronomen jedoch nicht: engl. \textit{he}/\textit{him}, \textit{she}/\textit{her}, \textit{it}/\textit{it}; niederl. \textit{hij}/\textit{hem}, \textit{zij}/\textit{haar}, \textit{het}/\textit{het} \citep[36]{GabrielRoodzant2010}. 

Die skandinavischen Standardsprachen können in zwei Gruppen geteilt werden. Zur ersten gehören Isländisch und Färöisch, deren \isi{Personalpronomen} der 3. Person Singular mit drei \isi{Genera} und ohne Unterscheidung von \isi{Belebtheit} wie die deutsche Standardsprache funktionieren (\citealt[76]{Pétursson1981}; \citealt[200]{BarnesWeyhe2002}). Zur zweiten Gruppe zählen Norwegisch (Bokmål), Schwedisch und Dänisch (\citealt[317--332]{FaarlundLieVannebo1997}; \citealt[128--136]{HolmesHinchliffe1994}; \citealt[141-155]{AllanHomlmesLundskær-Nielsen1995}). In diesen Sprachen wird in der 3. Person Singular des \isi{Personalpronomens} zwischen belebt und unbelebt unterschieden. Bei den belebten \isi{Personalpronomen} wird weiter zwischen Maskulin und Feminin (Sexus) differenziert (z.B. norw. \textit{han}/\textit{hun}), bei den unbelebten \isi{Personalpronomen} zwischen Utrum und Neutrum (\isi{Genus}) (z.B. norw. \textit{den}/\textit{det}) \citep[317]{FaarlundLieVannebo1997}. Des Weiteren weist in den belebten \isi{Personalpronomen} der Objektkasus eine andere Form auf als der Subjektkasus, während in den unbelebten \isi{Personalpronomen} Subjekt- und Objektkasus zusammenfallen (wie das Englische und das Niederländische).

Die älteren Stufen Gotisch, Altnordisch, Altenglisch, Altsächsisch und Althochdeutsch haben drei \isi{Genera}, funktionieren wie Deutsch, Isländisch und Färöisch und weisen im Neutrum eine einheitliche Form für den Nominativ und Akkusativ auf \citep[54]{KraheMeid1967}. Im Altnordischen stammt die Form für die 3. Person Singular Neutrum aus dem \isi{Demonstrativpronomen} ( \citealt[54]{KraheMeid1967}, \citealt[108]{Gutenbrunner1951} ).

Die hier untersuchten germanischen Sprachen können also in zwei Gruppen eingeteilt werden. Erstens verfügen die alten germanischen Stufen wie auch\linebreak Deutsch, Isländisch und Färöisch in der 3. Person Singular des \isi{Personalpronomens} über drei \isi{Genera} und machen keine zusätzliche Unterscheidung zwischen belebt und unbelebt. Zweitens differenzieren Norwegisch, Schwedisch, Dänisch, Englisch und Niederländisch zuerst zwischen belebt und unbelebt. Anschließend wird innerhalb der belebten \isi{Personalpronomen} Sexus (Maskulin und Feminin) kodiert und innerhalb der unbelebten \isi{Personalpronomen} \isi{Genus} (Utrum und Neutrum). Die Genusunterscheidung trifft nur auf die skandinavischen Sprachen zu. In dieser zweiten Gruppe unterscheidet außerdem das belebte \isi{Personalpronomen} einen Subjekt- von einem Objektkasus, das unbelebte \isi{Personalpronomen} jedoch nicht.

Die modernen romanischen Sprachen verfügen über ein Zwei-Genus-System (Maskulin und Feminin), Latein jedoch über ein Drei-Genus-System wie u.a. die deutsche Standardsprache. Im Neutrum wird die Form \textit{id} für Nominativ und Akkusativ verwendet \citep[90]{Touratier2013}. Dasselbe gilt für das Neugriechische \citep[95]{HoltonMackridgePhilippaki-Warburton2002} wie auch für das Altkirchenslawische \citep[50]{Trunte2005}. Das Altgriechische verwendet für den Nominativ und die obliquen \isi{Kasus} unterschiedliche Pronomen \citep[92]{Smyth1984}, weshalb es hier für den Vergleich nicht berücksichtigt werden kann. Bereits für das Urindogermanische mit einem Drei-Genus-System wird von einem \isi{Synkretismus} zwischen Nominativ und Akkusativ im Neutrum ausgegangen \citep[67]{Tichy2004}. Man nimmt aber des Weiteren an, dass dem Drei-Genus-System ein Zwei-Genus-System vorausgegangen ist:

\begin{quote}
Diese [Zweiheit] bestand vermutlich auf der einen Seite aus einer Klasse A, wo eine Nom./Akk.-Dif\-fe\-ren\-zie\-rung möglich war ( […] vom Sprecher als Träger einer Verbalhandlung vorstellbar, agensfähig), auf der anderen Seite aus einer Klasse B, wo dies gerade ausgeschlossen war ( […] vom Sprecher als Träger einer Verbalhandlung nicht vorstellbar, nicht agensfähig). \citep[323]{Meier-Brügger2010}
\end{quote}

Im Gegensatz zum Altkirchenslawischen weist die Substantivflexion der modernen slawischen Sprachen einen parallelen Fall zu den \isi{Belebtheit} unterscheidenden alemannischen Dialekten auf, was kurz anhand des Russischen gezeigt werden soll. Das Russische vefügt über drei \isi{Genera} (Maskulin, Feminin, Neutrum) und unterscheidet be\-lebt/un\-be\-lebt im Maskulin Singular und im Plural aller \isi{Genera} (\tabref{table5.23}). Dabei variiert die Akkusativform: Handelt es sich um belebte Entitäten, lautet der Akkusativ wie der Genitiv, handelt es sich um unbelebte Entitäten, lautet der Akkusativ wie der Nominativ.

%{\tabref{table5.23}: Russische Substantivflexion, belebt und unbelebt (gekürztes Paradigma aus \citet[166]{Corbett1991})}\\

\begin{table}
\caption{Russische Substantivflexion, belebt und unbelebt (gekürztes Paradigma aus \citealt[166]{Corbett1991})}\label{table5.23}
\resizebox{\textwidth}{!}{\begin{tabular}{lllllll} 
\lsptoprule
& {\textit{student}} {(m)} & {\textit{dub}} {(m)} & {\textit{sestra}} {(f)} & {\textit{škola}} {(f)} & {\textit{čudovišče} }{(n)} & {\textit{vino} }{(n)}\\
& {ʻStudentʼ} & {ʻEicheʼ} &  {ʻSchwesterʼ} &  {ʻSchuleʼ} &  {ʻMonsterʼ} & {ʻWeinʼ}\\
\midrule
\multicolumn{7}{c}{{\textsc{singular}}}\\\midrule
{\NOM} & student & dub & sestr-a & škol-a & čudovišč-e & vin-o\\
{\AKK} & student-a & dub & sestr-u & škol-u & čudovišč-e & vin-o\\
{\GEN} & student-a & dub-a & sestr-y & škol-y & čudovišč-a & vin-a\\
\midrule
\multicolumn{7}{c}{{\textsc{Plural}}}\\\midrule
{\NOM} & student-y & dub-y & sestr-y & škol-y & čudovišč-a & vin-a\\
{\AKK} & student-ov & dub-y & sester & škol-y & čudovišč & vin-a\\
{\GEN} & student-ov & dub-ov & sester & škol & čudovišč & vin\\
\lspbottomrule
\end{tabular}}
\end{table}

In den alemannischen Dialekten und in den slawischen Sprachen stellt die Kategorie \isi{Belebtheit} eine Neuerung dar. In den alemannischen Dialekten taucht dies im \isi{Personalpronomen} auf, in den slawischen Sprachen in den \isi{Substantiven}. Übrigens weist im Russischen der Akkusativ in der 3. Person Singular des \isi{Personalpronomens} dieselbe Form wie der Genitiv auf und beide \isi{Kasus} unterschieden sich vom Nominativ. Darauf kann aber an dieser Stelle nicht weiter eingegangen werden. Die alemannischen Dialekte und das Russische besitzen drei \isi{Genera} und ein Subgenus \isi{Belebtheit}, wobei das Subgenus im Russischen im Maskulin Singular und im Plural aller \isi{Genera} vorkommt, in den alemannischen Dialekten nur im Neutrum Singular. Des Weiteren beinhaltet \isi{Belebtheit} im Russischen Menschen wie Tiere, während \isi{Belebtheit} in den alemannischen Dialekten nur Menschen einschließt. Bezüglich der Form verwendet das Russische eine bereits vorhandene Form, nämlich die Genitivform (für Akkusativ belebt). Die alemannischen Dialekte haben für den belebten Akkusativ eine neue Form hervorgebracht. Eine Ausnahme bildet diesbezüglich der Dialekt des Sensebezirks, der wie das Russische eine bereits vorhandene Form benutzt: Der Akkusativ der unbelebten Form lautet wie der Dativ, der Akkusativ der belebten Form wie der Nominativ. Besonders auffällig ist, dass in beiden Sprachen jeweils nur der Akkusativ in Abhängigkeit von \isi{Belebtheit} variiert. Nimmt man noch das Urindogermanische und jene germanischen Sprachen hinzu, die im \isi{Personalpronomen} be\-lebt/un\-be\-lebt unterscheiden, kann dies allgemeiner zusammengefasst werden: Sprachen, die in bestimmten Wortarten eine Kategorie \isi{Belebtheit} aufweisen, unterscheiden in diesen Wortarten Nominativ und Akkusativ (bzw. Subjekt- und Objektkasus), wenn der Referent belebt ist, während diese \isi{Kasus} zusammenfallen, wenn der Referent unbelebt ist. Dies bedarf einer Erklärung.

{Erklärungsversuch}: Einen interessanten Ansatz bietet \citet{Comrie1996}. Er geht davon aus, dass in einer Transitivkonstruktion das Agens einen hohen Grad an \isi{Belebtheit} und das Patiens einen niedrigen Grad an \isi{Belebtheit} aufweist. Verfügt nun das Patiens einen hohen Grad an \isi{Belebtheit}, liegt ein markierter Fall vor, der auch formal markiert werden muss \citep[128]{Comrie1996}. Da in Transitivkonstruktionen (der Nominativ-Akkusativ-Sprachen) der prototypische \isi{Kasus} für Patiens der Akkusativ ist, muss in diesen Sprachen also der Akkusativ belebt besonders markiert werden. In dieselbe Richtung geht \citet{Bossong1998}: Das prototypische Subjekt ist belebt, das prototypische Objekt unbelebt, da das Subjekt die Handlung ausführt, während das Objekt die Handlung erfährt \citep[201]{Bossong1998}. In diesen Fällen muss das Objekt nicht speziell markiert werden. Wenn ein Objekt jedoch durch seine inhärente Semantik, indem es belebt ist, ein potentielles Subjekt darstellt, „[…] il s’avère nécessaire de lui conférer une marque spécifique permettant de le distinguer du sujet sans ambiguïté [et] parce qu’il[…] correspond[…] moins bien à la sémantique de l’objet prototypique […]“ \citep[202]{Bossong1998}. Dies nennt \citet{Bossong1998} differentielle Objektmarkierung. Das erklärt also, weshalb in den alemannischen Dialekten im Neutrum belebt der Akkusativ vom Nominativ unterschieden wird, jedoch nicht im Neutrum unbelebt. Interessanterweise widerspricht das System vom Dialekt des Sensebezirks Bossongs differentiellen Objektmarkierung, denn dieser Dialekt zeigt einen Nominativ-Akkusativ-\isi{Synkretismus} im belebten Paradigma und eine Nominativ-Akkusativ-Unterscheidung im unbelebten Paradigma. Aufgrund dieser Beobachtungen wird klar, dass dies weiterer Beschäftigung und Analysen bedarf.

Des Weiteren übernimmt \citet{Bossong1998} \citeauthor{Silverstein1976}s \citeyearpar{Silverstein1976} Belebtheitshierarchie, in der Deiktika am belebtesten sind \citep[203]{Bossong1998}. Diese Belebtheitshierarchie kann also erklären, weshalb in den alemannischen Dialekten \isi{Belebtheit} am \isi{Personalpronomen}, nicht aber an anderen nominalen Kategorien markiert wird.\footnote{Es sei hier nur darauf hingewiesen, dass in den Dialekten des Sensebezirks und von Jaun der Akkusativ und der Dativ des \isi{Personalpronomens} mit einem vorausgehenden \textit{i}{}- markiert werden (also eine Distinktion Objekt vs. Subjekt), während in den übrigen Determinierern nur der Dativ mit einem vorausgehenden \textit{i}{}- markiert wird \citep[85]{Seiler2003}.}\largerpage

Schließlich ist noch zu klären, weshalb in den alemannischen Dialekten das \isi{Personalpronomen} der 3. Person Singular im Maskulin und Neutrum belebt eine Akkusativform hat, die sich von der Nominativform unterscheidet, das Feminin aber nicht. Oder andersherum gefragt: Weshalb kommt diese Neuerung (neue Akkusativform, die sich vom Nominativ unterscheidet) nur im Neutrum, aber nicht im Feminin vor? Die prototypischen \isi{Genera} für belebte Entitäten sind Maskulin und Feminin, für unbelebte Entitäten das Neutrum. Stehen nun auch belebte Entitäten im Neutrum, müssen diese Fälle und besonders der Akkusativ dieser Fälle speziell markiert werden. Dies widerspricht jedoch den Ergebnissen aus dem Singular Neutrum in der russischen Substantivflexion. Diese Diskussion ließe sich noch vertiefen, zumal vieles ungeklärt bleibt oder widersprüchlich ist. Beispielsweise wird nicht klar, weshalb gerade die Walser Dialekte, in denen mit dem Neutrum auf männliche wie weibliche Menschen referiert wird, im Neutrum keine Kategorie \isi{Belebtheit} grammatikalisiert haben. Diese Diskussion würde hier jedoch zu weit führen und ist Aufgabe zukünftiger Forschung.

\subsection{Freie Variation}\label{5.3.4}

Freie \isi{Variation} kommt im Alt- und Mittelhochdeutschen sowie in der Standardsprache bezüglich der \isi{Personalpronomen} nicht vor. Im Gegensatz dazu ist freie \isi{Variation} ein durchaus übliches Phänomen in den alemannischen Dialekten. Nur die Dialekte von Saulgau, Colmar, des Münstertals und des Kaiserstuhls weisen keine freie \isi{Variation} in den \isi{Personalpronomen} auf. Mit Abstand am häufigsten kommt freie \isi{Variation} im Nominativ vor, gefolgt vom Akkusativ und Genitiv. Am seltensten ist freie \isi{Variation} in Dativ zu beobachten. Beispielsweise lautet im Dialekt des Sensebezirks die 1. Person Plural Nominativ \textit{wiər} und \textit{miər} \citep[196]{Henzen1927}.

Wie bereits mehrmals dargestellt wurde (\sectref{4.1.3.3}, \sectref{5.1.2}, \sectref{5.2.3}), müssen bei freier \isi{Variation} die RRs gleich spezifisch sein. Weist eine Zelle des Paradigmas mehr als eine Form auf, müssen die RRs gleich spezifisch gestaltet und demselben \isi{Block} zugewiesen werden. Nur so wird gewährleistet, dass zwei RRs zwei Formen für dieselbe Zelle definieren.

\section{Interrogativpronomen}\label{5.4}

{Morphosyntaktische Eigenschaften}: Im \isi{Interrogativpronomen} \textit{wer}/\textit{was} werden \isi{Kasus} und \isi{Belebtheit} unterschieden. Zwar entsprechen die belebten Formen formal einem Maskulin und die unbelebten einem Neutrum. Da aber mit \textit{wer} auf belebte Entitäten verwiesen wird und mit \textit{was} auf unbelebte, wird in den RRs nicht \isi{Genus} sondern ±belebt definiert. Wie beim \isi{Personalpronomen} lautet der Parameter \textit{Animacy} und wird in den RR mit ANIM abgekürzt.\largerpage

{Definition der Form}: Wie die \isi{Adjektive} (Ausnahme: Mittelhochdeutsch) weist auch das \isi{Interrogativpronomen} maximal ein \isi{Suffix} auf. Es muss also keine Abfolge von Affigierungen durch verschiedene \isi{Blöcke} definiert werden. Des Weiteren macht es keinen Unterschied bezüglich der Anzahl RRs, ob durch die RR die ganze Form des \isi{Interrogativpronomens} definiert wird (z.B. \textit{wer}, \textit{wen} etc.) oder ob dieses in \isi{Wurzel} und \isi{Suffix} geteilt wird (z.B. \textit{w}-\textit{er}, \textit{w}-\textit{en} etc.). Daraus resultieren zwei Konsequenzen für die RRs: Durch die RRs wird die gesamte Form definiert und alle RRs stehen im selben \isi{Block}.

{Freie Variation}: Auch im \isi{Interrogativpronomen} kommt \isi{Variation} vor, d.h., eine Zelle des Paradigmas hat zwei Formen. Wie bereits referiert und analog zu den oben besprochenen Kategorien (\sectref{4.1.3.3}, \sectref{5.1.2}, \sectref{5.2.3}, \sectref{5.3.4}) sind für beide Formen gleich spezifische RRs anzunehmen. Beispielsweise verfügt der Akkusativ Singular belebt im Dialekt von Jaun über zwei Formen, nämlich \textit{wær} und \textit{wɛm} (vgl. \tabref{table5.24}). Zwar ist die Distribution syntaktisch bedingt: \textit{Wær} in freier Verwendung, \textit{wɛm} nach Präposition \citep[285]{Stucki1917}, trotzdem muss die Morphologie für eine Zelle des Paradigmas beide Formen bilden. Folglich braucht es dafür zwei gleich spezifische RRs.

{Beispiel}: Als Beispiel für ein System an RRs bezüglich des \isi{Interrogativpronomens} werden in der Folge die RRs für das \isi{Interrogativpronomen} von Jaun gelistet. \tabref{table5.24} zeigt das Paradigma des \isi{Interrogativpronomens} von Jaun, (\ref{ex:key:113}--\ref{ex:key:116}) die dazugehörigen RRs.

%{\tabref{table5.24}: \isi{Interrogativpronomen} von Jaun \citep[285-286]{Stucki1917}}\\

\begin{table}
\caption{Interrogativpronomen von Jaun \citep[285-286]{Stucki1917}}\label{table5.24}
\begin{tabular}{llll} 
\lsptoprule
& {\NOM} & {\AKK} & {\DAT}\\\midrule
{belebt} & wær & wær/wemm & wɛm\\
{unbelebt} & was & was & wɛm\\
\lspbottomrule
\end{tabular}
\end{table}

RRs \REF{ex:key:113} und \REF{ex:key:114} definieren den \isi{Synkretismus} zwischen Nominativ und Akkusativ, RR \REF{ex:key:115} jenen zwischen belebt und unbelebt im Dativ. Schließlich wird in RR \REF{ex:key:116} die zweite Form des Akkusativs belebt bestimmt.

\ea%113
\label{ex:key:113}
 RR \textsubscript{A,} \textsubscript{\{\textsc{case:} \textsc{nom}} \textsubscript{\tiny $\veebar$}\textsubscript{ \AKK, \textsc{anim:+}\},} \textsubscript{PRON.INTER} ($\langle$X,$\sigma$ $\rangle$) = \textsubscript{def} $\langle$\textit{wær}ˊ,$\sigma$ $\rangle$
\z

\ea%114
\label{ex:key:114}
 RR \textsubscript{A,} \textsubscript{\{\textsc{case:} \textsc{nom}} \textsubscript{\tiny $\veebar$}\textsubscript{ \AKK, ANIM:-\},} \textsubscript{PRON.INTER} ($\langle$X,$\sigma$ $\rangle$) = \textsubscript{def} $\langle$\textit{was}ˊ,$\sigma$ $\rangle$
\z

\ea%115
\label{ex:key:115}
 RR \textsubscript{A,} \textsubscript{\{\textsc{case: dat}\},} \textsubscript{PRON.INTER} ($\langle$X,$\sigma$ $\rangle$) = \textsubscript{def} $\langle$\textit{wɛm}ˊ,$\sigma$ $\rangle$
\z

\ea%116
\label{ex:key:116}
 RR \textsubscript{A,} \textsubscript{\{\textsc{case: akk}, \textsc{anim:+}\},} \textsubscript{PRON.INTER} ($\langle$X,$\sigma$ $\rangle$) = \textsubscript{def} $\langle$\textit{wɛm}ˊ,$\sigma$ $\rangle$
\z

\section{Bestimmter Artikel / Demonstrativpronomen}\label{5.5}

\subsection{Allgemeines und Realisierungsregeln}\label{5.5.1}

{Zwei Wortarten, eine Kategorie}: Der \isi{bestimmte Artikel} und das einfache \isi{Demonstrativpronomen} bilden zusammen eine Kategorie. Wie in \sectref{4.3.2} erörtert wurde, hat dies zwei Gründe. Erstens sollen die Paradigmen auf ein Minimum reduziert werden, da nur so unterschiedliche Paradigmen aus unterschiedlichen Grammatiken verglichen werden können. Dies ist am besten zu erreichen, indem ähnliche oder sogar gleiche Paradigmen Teil derselben Kategorie sind. Da unter allen untersuchten Wortarten der \isi{bestimmte Artikel} und das \isi{Demonstrativpronomen} die größten Ähnlichkeiten aufweisen, formen sie eine Kategorie. Zweitens ist der \isi{bestimmte Artikel} aus dem \isi{Demonstrativpronomen} entstanden.

{Definition der Form}: Wie in den \isi{Personalpronomen} sind die Formen des \isi{bestimmten Artikels} und des \isi{Demonstrativpronomens} synchron nicht weiter unterteilbar, weswegen die gesamte Form durch die RR definiert wird (vgl. \sectref{5.3.1}). Da unterschiedliche \isi{Affixe} nicht aneinandergereiht sind und ihre Abfolge also nicht definiert werden muss, stehen alle RRs in demselben \isi{Block}.

Als Beispiel für ein System an RRs soll hier der Dialekt von Jaun herangezogen werden (\tabref{table5.25}, RRs (\ref{ex:key:118}--\ref{ex:key:136})), an dem in diesem Unterkapitel unterschiedliche Eigenschaften der Kategorie \isi{bestimmter Artikel}/\isi{Demonstrativpronomen} illustriert werden. Die zu unterscheidenden morphosyntaktischen Eigenschaften sind \isi{Kasus}, \isi{Numerus}, \isi{Genus}.

%{\tabref{table5.25}: Bestimmter Artikel und \isi{Demonstrativpronomen} von Jaun \citep[282-283]{Stucki1917}}\\

\begin{table}
\caption{Bestimmter Artikel und Demonstrativpronomen von Jaun \citep[282-283]{Stucki1917}}\label{table5.25}
\begin{tabular}{lllllllll}
\lsptoprule
 \multicolumn{5}{c}{{bestimmter Artikel}}  & \multicolumn{4}{c}{{Demonstrativpronomen}}\\\cmidrule(lr){1-5}\cmidrule(lr){6-9}
& {\NOM} & {\AKK} & {\DAT} & {\GEN} &  & {\NOM} & {\AKK} & {\DAT}\\\midrule
\textsc{m.sg} & dər & dər/ə & dəm/əm & ts & \textsc{m.sg} & dær & dær & dɛm\\
\textsc{n.sg} & ts & ts & dəm/əm & ts &     \textsc{n.sg} & das & das & dɛm\\
\textsc{f.sg} & di/t & di/t & dər & dər &   \textsc{f.sg} & di & di & dɛr\\
\textsc{pl} & di/t & di/t & də & dər &      \textsc{m./n.pl} & di & di & dɛnə\\
&  &  &  &  & \textsc{f.pl} & diu & diu & dɛnə\\
\lspbottomrule
\end{tabular}
\end{table}

Die Kategorie \isi{bestimmter Artikel}/\isi{Demonstrativpronomen} ist mit DET1 kodiert. Sowohl bei dieser Kategorie als auch bei der Kategorie \isi{unbestimmter Artikel}/\isi{Possessivpronomen} handelt es sich um Determinierer, die jeweils große Ähnlichkeiten in ihren Formen, aber nicht in ihrer Semantik aufweisen. Da es hier\linebreak jedoch nur um die Form geht, kann die Kategorie \isi{bestimmter Artikel}\slash\is{Demonstrativpronomen}Demonstra-\linebreak tivpronomen mit DET1 und die Kategorie \isi{unbestimmter Artikel}\slash Possessivprono-\linebreak men\is{Possessivpronomen} mit DET2 (\sectref{5.6}) abgekürzt werden. Fallen die Formen des \isi{bestimmten Artikels} und des \isi{Demonstrativpronomens} zusammen, muss der Parameter DET1 nicht weiter definiert werden. Weisen die beiden Wortarten zwei unterschiedliche Formen auf, ist die Wortart zu spezifizieren.

{Beispiel}: Für beide Fälle finden sich Beispiele im Dialekt von Jaun. Der Nominativ Singular des \isi{bestimmten Artikels} lautet \textit{dər} und des \isi{Demonstrativpronomens} \textit{dær} \citep[282]{Stucki1917}. Es sind also zwei unterschiedliche Formen, die durch zwei RRs bestimmt werden müssen (vgl. RRs (\ref{ex:key:118}) und (\ref{ex:key:130})). Im Gegensatz dazu weisen der \isi{bestimmte Artikel} und das \isi{Demonstrativpronomen} im Nominativ/Akkusativ Singular Feminin dieselbe Form auf, nämlich \textit{di} \citep[282]{Stucki1917}. Für diese Form ist folglich nur eine RR nötig:

\ea%117
\label{ex:key:117}
 RR \textsubscript{A,} \textsubscript{\{\textsc{case:nom}} \textsubscript{\tiny $\veebar$}\textsubscript{ \textsc{acc}}\textsubscript{, \textsc{num:sg}, \textsc{gend:f}\},} \textsubscript{DET1} ($\langle$X,$\sigma$ $\rangle$) = \textsubscript{def} $\langle$\textit{di}ˊ,$\sigma$ $\rangle$ \\
\z

In allen anderen Zellen weisen die beiden Wortarten unterschiedliche Formen auf (vgl. \tabref{table5.25}), worauf in \sectref{5.5.3} noch eingegangen wird. Die \isi{Variation} im Akkusativ Singular Maskulin und im Nominativ/Akkusativ Singular Feminin und Plural wird in \sectref{5.5.5} referiert. Es kann aber schon vorweggenommen werden, dass ihre Distribution syntaktisch bedingt ist und die Formen durch RRs definiert werden müssen. Ein immer wiederkehrendes Phänomen in dieser Kategorie sind Kasus- und Genussynkrestismen (z.B. RR \ref{ex:key:118}, \ref{ex:key:124}). Des Weiteren wird \isi{Genus} nicht spezifiziert, wenn die Formen aller drei \isi{Genera} zusammenfallen (RR \ref{ex:key:136}). Es folgen hier die RRs des \isi{bestimmten Artikels} (dazu gehört auch die oben eingeführte RR \REF{ex:key:117}, die für die Wortart unterspezifiziert ist):

\ea%118
\label{ex:key:118}
 RR \textsubscript{A,} \textsubscript{\{\textsc{case:nom}} \textsubscript{\tiny $\veebar$}\textsubscript{ \textsc{acc}}\textsubscript{, \textsc{num:sg}, \textsc{gend:m}\},} \textsubscript{\textsc{det1[art.def]}} ($\langle$X,$\sigma$ $\rangle$) = \textsubscript{def} $\langle$\textit{də}\textit{r}ˊ,$\sigma$ $\rangle$
\z

\ea%119
\label{ex:key:119}
 RR \textsubscript{A,} \textsubscript{\{\textsc{case:}}\textsubscript{\textsc{acc}}\textsubscript{, \textsc{num:sg}, \textsc{gend:m}\},} \textsubscript{\textsc{det1[art.def]}} ($\langle$X,$\sigma$ $\rangle$) = \textsubscript{def} $\langle$\textit{ə}ˊ,$\sigma$ $\rangle$
\z

\ea%120
\label{ex:key:120}
 RR \textsubscript{A,} \textsubscript{\{\textsc{case:nom}} \textsubscript{\tiny $\veebar$}\textsubscript{ \textsc{acc}}\textsubscript{, \textsc{num:sg}, \textsc{gend:n}\},} \textsubscript{\textsc{det1[art.def]}} ($\langle$X,$\sigma$ $\rangle$) = \textsubscript{def} $\langle$\textit{ts}ˊ,$\sigma$ $\rangle$
\z

\ea%121
\label{ex:key:121}
 RR \textsubscript{A,} \textsubscript{\{\textsc{case:nom}} \textsubscript{\tiny $\veebar$}\textsubscript{ \textsc{acc}}\textsubscript{, \textsc{num:sg}, \textsc{gend:f}\},} \textsubscript{\textsc{det1[art.def]}} ($\langle$X,$\sigma$ $\rangle$) = \textsubscript{def} $\langle$\textit{t}ˊ,$\sigma$ $\rangle$
\z

\ea%122
\label{ex:key:122}
 RR \textsubscript{A,} \textsubscript{\{\textsc{case:nom}} \textsubscript{\tiny $\veebar$}\textsubscript{ \textsc{acc}}\textsubscript{, \textsc{num:pl}\},} \textsubscript{\textsc{det1[art.def]}} ($\langle$X,$\sigma$ $\rangle$) = \textsubscript{def} $\langle$\textit{t}ˊ,$\sigma$ $\rangle$
\z

\ea%123
\label{ex:key:123}
 RR \textsubscript{A,} \textsubscript{\{\textsc{case:nom}} \textsubscript{\tiny $\veebar$}\textsubscript{ \textsc{acc}}\textsubscript{, \textsc{num:pl}\},} \textsubscript{\textsc{det1[art.def]}} ($\langle$X,$\sigma$ $\rangle$) = \textsubscript{def} $\langle$\textit{di}ˊ,$\sigma$ $\rangle$
\z

\ea%124
\label{ex:key:124}
 RR \textsubscript{A,} \textsubscript{\{\textsc{case:dat}, \textsc{num:sg}, \textsc{gend:m}} \textsubscript{\tiny $\veebar$} \textsubscript{\scshape n}\textsubscript{\},} \textsubscript{\textsc{det1[art.def]}} ($\langle$X,$\sigma$ $\rangle$) = \textsubscript{def} $\langle$\textit{də}\textit{m}ˊ,$\sigma$ $\rangle$
\z

\ea%125
\label{ex:key:125}
 RR \textsubscript{A,} \textsubscript{\{\textsc{case:dat}, \textsc{num:sg}, \textsc{gend:m}} \textsubscript{\tiny $\veebar$} \textsubscript{\scshape n}\textsubscript{\},} \textsubscript{\textsc{det1[art.def]}} ($\langle$X,$\sigma$ $\rangle$) = \textsubscript{def} $\langle$\textit{əm}ˊ,$\sigma$ $\rangle$
\z

\ea%126
\label{ex:key:126}
 RR \textsubscript{A,} \textsubscript{\{\textsc{case:gen}, \textsc{num:sg}, \textsc{gend:m}} \textsubscript{\tiny $\veebar$} \textsubscript{\scshape n}\textsubscript{\},} \textsubscript{\textsc{det1[art.def]}} ($\langle$X,$\sigma$ $\rangle$) = \textsubscript{def} $\langle$\textit{ts}ˊ,$\sigma$ $\rangle$
\z

\ea%127
\label{ex:key:127}
 RR \textsubscript{A,} \textsubscript{\{\textsc{case:dat}} \textsubscript{\tiny ${\veebar}$} \textsubscript{\GEN, \textsc{num:sg}, \textsc{gend:f}\},} \textsubscript{\textsc{det1[art.def]}} ($\langle$X,$\sigma$ $\rangle$) = \textsubscript{def} $\langle$\textit{dər}ˊ,$\sigma$ $\rangle$
\z

\ea%128
\label{ex:key:128}
 RR \textsubscript{A,} \textsubscript{\{\textsc{case:dat}, \textsc{num:pl}\},} \textsubscript{\textsc{det1[art.def]}} ($\langle$X,$\sigma$ $\rangle$) = \textsubscript{def} $\langle$\textit{də}ˊ,$\sigma$ $\rangle$
\z

\ea%129
\label{ex:key:129}
 RR \textsubscript{A,} \textsubscript{\{\textsc{case:gen}, \textsc{num:pl}\},} \textsubscript{\textsc{det1[art.def]}} ($\langle$X,$\sigma$ $\rangle$) = \textsubscript{def} $\langle$\textit{dər}ˊ,$\sigma$ $\rangle$
\z
\noindent
Die RRs des \isi{Demonstrativpronomens} sind die folgenden (plus die für die Wortart unterspezifizierte RR \ref{ex:key:117}):

\ea%130
\label{ex:key:130}
 RR \textsubscript{A,} \textsubscript{\{\textsc{case:nom}} \textsubscript{\tiny $\veebar$}\textsubscript{ \textsc{acc}}\textsubscript{, \textsc{num:sg}, \textsc{gend:m}\},} \textsubscript{\textsc{det1[pron.dem]}} ($\langle$X,$\sigma$ $\rangle$) = \textsubscript{def} $\langle$\textit{dæ}\textit{r}ˊ,$\sigma$ $\rangle$
\z

\ea%131
\label{ex:key:131}
 RR \textsubscript{A,} \textsubscript{\{\textsc{case:nom}} \textsubscript{\tiny $\veebar$}\textsubscript{ \textsc{acc}}\textsubscript{, \textsc{num:sg}, \textsc{gend:n}\},} \textsubscript{\textsc{det1[pron.dem]}} ($\langle$X,$\sigma$ $\rangle$) = \textsubscript{def} $\langle$\textit{das}ˊ,$\sigma$ $\rangle$
\z

\ea%132
\label{ex:key:132}
 RR \textsubscript{A,} \textsubscript{\{\textsc{case:nom}} \textsubscript{\tiny $\veebar$}\textsubscript{ \textsc{acc}}\textsubscript{, \textsc{num:pl}, \textsc{gend:m}} \textsubscript{\tiny $\veebar$}\textsubscript{ \textsc{n}\},} \textsubscript{\textsc{det1[pron.dem]}} ($\langle$X,$\sigma$ $\rangle$) = \textsubscript{def} $\langle$\textit{di}ˊ,$\sigma$ $\rangle$
\z

\ea%133
\label{ex:key:133}
 RR \textsubscript{A,} \textsubscript{\{\textsc{case:nom}} \textsubscript{\tiny $\veebar$}\textsubscript{ \textsc{acc}}\textsubscript{, \textsc{num:pl}, \textsc{gend:f}\},} \textsubscript{\textsc{det1[pron.dem]}} ($\langle$X,$\sigma$ $\rangle$) = \textsubscript{def} $\langle$\textit{diu}ˊ,$\sigma$ $\rangle$
\z

\ea%134
\label{ex:key:134}
 RR \textsubscript{A,} \textsubscript{\{\textsc{case:dat}, \textsc{num:sg}, \textsc{gend:m}} \textsubscript{\tiny $\veebar$}\textsubscript{ \textsc{n}\},} \textsubscript{\textsc{det1[pron.dem]}} ($\langle$X,$\sigma$ $\rangle$) = \textsubscript{def} $\langle$\textit{dɛm}ˊ,$\sigma$ $\rangle$
\z

\ea%135
\label{ex:key:135}
 RR \textsubscript{A,} \textsubscript{\{\textsc{case:dat}, \textsc{num:sg}, \textsc{gend:f}\},} \textsubscript{\textsc{det1[pron.dem]}} ($\langle$X,$\sigma$ $\rangle$) = \textsubscript{def} $\langle$\textit{dɛr}ˊ,$\sigma$ $\rangle$
\z

\ea%136
\label{ex:key:136}
 RR \textsubscript{A,} \textsubscript{\{\textsc{case:dat}, \textsc{num:pl}\},} \textsubscript{\textsc{det1[pron.dem]}} ($\langle$X,$\sigma$ $\rangle$) = \textsubscript{def} $\langle$\textit{dɛnə}ˊ,$\sigma$ $\rangle$
\z

\subsection{Possessiv-Artikel \textit{s}}\label{5.5.2}

Wie in \sectref{5.1.2} gezeigt wurde, ist in der Substantivflexion einiger Varietäten ein Possessiv-s anzunehmen, das an Eigennamen und Berufsbezeichnungen sowohl im Maskulin als auch im Feminin suffigiert wird. Einige dieser Dialekte haben einen speziellen Artikel, nämlich \textit{ts} oder \textit{s}, der ausschließlich in diesen possessiven Kontexten verwendet wird. Es handelt sich dabei um folgende Dialekte: Uri (Paradigma 88), Vorarlberg (Paradigma 89), Huzenbach (Paradigma 92), Saulgau (Paradigma 93), Petrifeld (Paradigma 95) und Kaiserstuhl (Paradigma 97).

Analog zur RR \REF{ex:key:46} zum Possessiv-S in der Substantivflexion (vgl. \sectref{5.1.2}) ist auch für den \isi{Possessivartikel} eine RR anzunehmen. Sie bestimmt, dass dieser Artikel nur in Possessivkontexten vorkommt. Folgende RR zeigt diese für den Dialekt von Uri:

\ea%137
\label{ex:key:137}
 RR \textsubscript{A,} \textsubscript{\{POSS:+, \textsc{num:sg}\},} \textsubscript{\textsc{det1[art.def]}} ($\langle$X,$\sigma$ $\rangle$) = \textsubscript{def} $\langle$\textit{ts}ˊ,$\sigma$ $\rangle$ \\
\z

Die Kategorie Possessiv trifft weder eine Genus- noch Kasusunterscheidung und tritt nur im Singular auf. Sie steht also nicht in Konkurrenz zu den genus- und kasusspezifizierten Formen, sondern definiert eigene Zellen im Paradigma (vgl. z.B. Paradigma 88 für den Dialekt von Uri).

\subsection{Diachrone Differenzierung des bestimmten Artikels und des Demonstrativpronomens}\label{5.5.3}

Wie bereits in \sectref{4.3.2} dargestellt, ist der \isi{bestimmte Artikel} aus dem einfachen \isi{Demonstrativpronomen} entstanden. Für das Althochdeutsche ist noch von keinem grammatikalisierten \isi{bestimmten Artikel} auszugehen \citep[24]{Schrodt2004}, sondern noch von einem einfachen \isi{Demonstrativpronomen}. Das Mittelhochdeutsche verfügt über einen \isi{bestimmten Artikel}, dessen Gebrauch aber teils von jenem der modernen deutschen Standardsprache abweicht \citep[380-381]{Paul2007}.

Die Formen des \isi{bestimmten Artikels} und des einfachen \isi{Demonstrativpronomens} im Mittelhochdeutschen und in der deutschen Standardsprache fallen vollständig zusammen (vgl. Paradigmen 82 und 83). Eine Ausnahme bildet hier nur der Dativ Singular Maskulin und Neutrum in der deutschen Standardsprache, der nach Präpositionen eine andere Form zeigt (vgl. \sectref{5.5.5}). Im Gegensatz dazu weisen alle alemannischen Dialekte für die beiden Wortarten unterschiedliche Paradigmen auf (z.B. \tabref{table5.25} für Jaun). Dass synchron der \isi{bestimmte Artikel} der alemannischen Dialekte eine andere Form als das \isi{Demonstrativpronomen} zeigt und es sich bei dieser Form diachron um eine reduzierte Form des \isi{Demonstrativpronomens} handelt, ist ein Symptom dafür, dass die Grammatikalisierung des Artikels in diesen Dialekten weiter fortgeschritten ist als im Mittelhochdeutschen und in der Standardsprache. Dabei können jene Dialekte, in denen keine Formen identisch sind, von den Dialekten unterschieden werden, in denen einige wenige Zellen für die beiden Wortarten gleiche Formen haben. Die Dialekte, in denen die beiden Wortarten zu hundert Prozent verschiedene Formen aufweisen, sind: Visperterminen, Uri, Zürich, Huzenbach, Saulgau, Stuttgart, Petrifeld, Elisabethtal, Kaiserstuhl, Münstertal und Colmar. Die Dialekte, in denen sich die beiden Wortarten einzelne Formen teilen, sind in \tabref{table5.26} gelistet, wie auch die Zellen der beiden Wortarten, die zusammenfallen. Betroffen vom Zusammenfall sind unterschiedliche Zellen. Am häufigsten jedoch weist der Dativ Singular Feminin des \isi{bestimmten Artikels} und des \isi{Demonstrativpronomens} dieselben Formen auf.

% {\tabref{table5.26}: Gemeinsame Formen des \isi{bestimmten Artikels} und des Demonstrativpronomens}

\begin{table}
\caption{Gemeinsame Formen des bestimmten Artikels und des Demonstrativpronomens}\label{table5.26}
\begin{tabular}{l>{\scshape}l}
\lsptoprule
{Dialekt} & {\upshape zusammengefallene Zellen}\\
\midrule
Issime & dat./gen.sg.f, gen.pl\\
Jaun & nom./akk.sg.f\\
Sensebezirk & nom./akk.sg.f, nom./akk.pl\\
Vorarlberg & dat.sg.m./n./f\\
Bern & dat.sg.f\\
Elsass (Ebene) & nom./akk.sg.m., dat.sg.f\\
\lspbottomrule
\end{tabular}
\end{table}

\subsection{Freie Variation im bestimmten Artikel und Demonstrativpronomen}\label{5.5.4}

In der Hälfte der untersuchten Varietäten kommt freie \isi{Variation} im \isi{bestimmten Artikel} und/oder im \isi{Demonstrativpronomen} vor. In beiden Wortarten sind davon viele unterschiedliche Zellen des Paradigmas betroffen, im Dativ Singular Maskulin und Neutrum des \isi{bestimmten Artikels} jedoch kommt sie häufiger vor als in den übrigen Zellen.

Für die RRs der freien \isi{Variation} gilt dasselbe wie für die RRs der bis hier diskutierten Kategorien (vgl. \sectref{5.1.2}, \sectref{5.2.3}, \sectref{5.3.4}, \sectref{5.4}): Die RRs der beiden Varianten müssen gleich spezifisch sein, damit sie einander bei der Definition der Form für eine bestimmte Zelle nicht blockieren. Theoretisch begründet wurde dies in \sectref{4.1.3.3} Als Bespiel fungiert hier das \isi{Demonstrativpronomen} von Colmar, in dem Nominativ und Akkusativ Singular Maskulin jeweils zwei Formen aufweisen, nämlich \textit{tar} und \textit{ta} \citep[83]{Henry1900}. Damit für eine Zelle zwei Formen definiert sind, werden zwei gleich spezifische RRs benötigt wie in \REF{ex:key:138} und \REF{ex:key:139} bezüglich des Paradigmas von Colmar.

\ea%138
\label{ex:key:138}
 RR \textsubscript{A,} \textsubscript{\{\textsc{case:nom}} \textsubscript{\tiny $\veebar$}\textsubscript{ \textsc{acc}}\textsubscript{, \textsc{num:sg}, \textsc{gend:m}\},} \textsubscript{\textsc{det1[pron.dem]}} ($\langle$X,$\sigma$ $\rangle$) = \textsubscript{def} $\langle$\textit{tar}ˊ,$\sigma$ $\rangle$
\z

\ea%139
\label{ex:key:139}
 RR \textsubscript{A,} \textsubscript{\{\textsc{case:nom}} \textsubscript{\tiny $\veebar$}\textsubscript{ \textsc{acc}}\textsubscript{, \textsc{num:sg}, \textsc{gend:m}\},} \textsubscript{\textsc{det1[pron.dem]}} ($\langle$X,$\sigma$ $\rangle$) = \textsubscript{def} $\langle$\textit{ta}ˊ,$\sigma$ $\rangle$
\z

\subsection{Syntaktisch bedingte Variation im bestimmten Artikel}\label{5.5.5}

Von der freien \isi{Variation} ist die syntaktisch bedingte \isi{Variation} zu unterscheiden und innerhalb dieser zwei Faktoren: a) Der \isi{bestimmte Artikel} ist nur Teil einer NP oder diese NP gehört zu einer PP; b) der Artikel steht vor einem \isi{Adjektiv} oder einem \isi{Substantiv}. Im Gegensatz zur freien \isi{Variation} ist von der syntaktisch bedingten \isi{Variation} also nur der \isi{bestimmte Artikel} betroffen. Ob es sich um freie oder syntaktisch bedingte \isi{Variation} handelt, hat keinen Einfluss auf das Aussehen der RR, denn die Morphologie stellt nur die Formen zur Verfügung. Wie die Formen distribuiert sind, ist Aufgabe der Syntax, d.h., die Regeln der Distribution dürfen nicht in den RRs definiert werden. Trotzdem soll die syntaktische bedingte \isi{Variation} kurz besprochen werden, denn wenn in Abhängigkeit von der syntaktischen Umgebung unterschiedliche Artikel verwendet werden, impliziert dies, dass die Morphologie unterschiedliche Formen dieses Artikels definieren muss. In der Folge wird zuerst auf die Abfolge Artikel + \isi{Adjektiv} und dann auf die Abfolge Präposition + Artikel + \isi{Substantiv} vs. Artikel + \isi{Substantiv} eingegangen. Eine Übersicht bietet \tabref{table5.27}.

{Artikel + Adjektiv}: Einige alemannische Dialekte weisen im Nominativ und Akkusativ Singular Feminin sowie im Nominativ und Akkusativ Plural zwei Formen des \isi{bestimmten Artikels} auf, wobei es sich immer um die Varianten \textit{di} und \textit{d} handelt (vgl. Paradigmen 86, 87, 88, 89, 91). \textit{Di} wird vor einem \isi{Adjektiv} verwendet, \textit{d} vor einem \isi{Substantiv} (die Qualität von \textit{d} kann je nach Dialekt variieren), z.B. \textit{t Han} ʻdie Handʼ, \textit{di grosi Han} ʻdie große Handʼ \citep[187, 191, 200]{Henzen1927}. Diese \isi{Variation} ist nur in den hoch- und höchstalemannischen Dialekten zu finden, mit Ausnahme der Walser Dialekte und des Dialektes von Zürich. Zwar zeigen auch die schwäbischen Dialekte von Huzenbach (Paradigma 92) und Petrifeld (Paradigma 95) \isi{Variation}, aber nur im Nominativ und Akkusativ Plural. Außerdem finden sich in den Grammatiken dieser schwäbischen Dialekte keine Angaben zur Distribution dieser Varianten, weshalb von freier \isi{Variation} ausgegangen werden kann.

{(Präposition +) Artikel + Substantiv}: Die Form des Artikels variiert auch in Abhängigkeit davon, ob die NP Teil einer PP ist oder nicht, d.h., ob dem Artikel eine Präposition vorangeht oder nicht. Von dieser \isi{Variation} betroffen sind der Akkusativ Singular Maskulin und der Dativ Singular Maskulin/Neutrum, die nun nacheinander besprochen werden.

Im Akkusativ Singular Maskulin steht eine Form des Typs \textit{der} oder \textit{de}, wenn dem Artikel keine Präposition vorangeht, eine Form des Typs \textit{e}, wenn dem Artikel eine Präposition vorangeht. Beispiel: \textit{dər} \textit{w\=i} ʻden Weinʼ, \textit{für} \textit{ə} \textit{w\=i} ʻfür den Weinʼ \citep[283]{Stucki1917}. Dies kommt in den hoch- und höchstalemannischen Dialekten (außer den Walser Dialekten und dem Dialekt von Zürich) vor, also in genau denselben Dialekten, die bereits die \isi{Variation} vor \isi{Adjektiv} aufgewiesen haben (vgl. Paradigmen 86, 87, 88, 89. 91). Auch im Dialekt von Saulgau ist dieselbe \isi{Variation} zu beobachten, wobei jedoch nach der Präposition die Form \textit{n} (und nicht \textit{ə}) steht (Paradigma 93). Schließlich zeigt der Dialekt des Elsass (Ebene) eine ähnliche \isi{Variation}: \textit{Dər} steht, wenn keine Präposition vorangeht, \textit{dər} oder \textit{də}, wenn eine Präposition vorangeht (Paradigma 100).

Dieselbe syntaktisch bedingte \isi{Variation} (±Präposition + Artikel), aber im Dativ Singular Maskulin/Neutrum weisen folgende Varietäten auf: die drei elsässischen Dialekte (Paradigmen 98, 99, 100), die höchstalemannischen Dialekte außer Issime (Paradigmen 85, 86, 87, 88), der Dialekt von Saulgau (Paradigma 93) und die deutsche Standardsprache (Paradigma 83). Beim Artikel, der nach einer Präposition steht, handelt es sich um eine reduzierte Form jenes Artikels, dem keine Präposition vorangeht, z.B. \textit{dum}/\textit{um} (Sensebezirk, \citealt[200]{Henzen1927}). Eine phonologische Erklärung kann jedoch nicht gefunden werden, da keine Vokalcluster o.ä. auftreten. Anders sieht dies aus, wenn man nur die Abfolge Präposition + Artikel betrachtet: Lautet die Präposition vokalisch aus und der Artikel vokalisch an, wird der anlautende Vokal des Artikels getilgt. Im Dialekt des Sensebezirks z.B. lautet der Artikel \textit{um}, wenn die Präposition konsonantisch auslautet, aber \textit{m}, wenn die Präposition vokalisch auslautet \citep[200]{Henzen1927}. Ebenfalls phonologisch bedingt ist die \isi{Variation} im Dialekt von Zürich: Geht dem Artikel keine oder eine konsonantisch auslautende Präposition voraus, lautet der Artikel \textit{əm}, geht dem Artikel eine vokalisch auslautende Präposition voraus, lautet der Artikel \textit{m} \citep[103]{Weber1987}. Da die Varianten phonologisch bedingt sind, müssen für diese Formen keine RRs angenommen werden. Im Gegensatz zum Dialekt von Zürich ist im Dialekt von Visperterminen die \isi{Variation} nicht phonologisch bedingt. Die beiden Varianten \textit{dum} (keine Präposition oder konsonantisch auslautende Präposition) und \textit{m} (vokalisch auslautende Präposition) lauten beide konsonantisch an \citet[141-142]{Wipf1911}. Beide Formen sind also durch RRs zu definieren. Schließlich weisen folgende Dialekte im Dativ Singular Maskulin/Neutrum freie \isi{Variation} auf: Elisabethtal (Paradigma 96), Vorarlberg (Paradigma 89) und Jaun (Paradigma 86, zusätzlich zur syntaktisch bedingten \isi{Variation}).

Zusammenfassend kann festgehalten werden, dass in den Dialekten von Uri, Jaun und des Sensebezirks (alle höchstalemannisch) syntaktisch bedingte \isi{Variation} besonders häufig vorkommt. In etwas geringerem Maße sind die hochalemannischen Dialekte (außer Zürich), die elsässischen Dialekte sowie der schwäbische Dialekt von Saulgau betroffen.

% % {\tabref{table5.27}: Syntaktisch bedingte \isi{Variation} im bestimmten Artikel}\\

\begin{table}
\caption{Syntaktisch bedingte Variation im bestimmten Artikel}\label{table5.27}
\begin{tabular}{p{2.25cm}p{4cm}p{4cm}}
\lsptoprule
{± Präposition\newline + best. Artikel:} & \textsc{akk.sg.m} & \textsc{dat.sg.m/n}\\
\midrule
{Dialekte:} & 
• Höchstalemannisch\newline\hspaceThis{•} (außer Walser)\newline • Hochalemannisch\newline\hspaceThis{•} (außer Zürich)\newline • Saulgau\newline • Elsass (Ebene) & 
• Höchstalemannisch\newline\hspaceThis{•} (außer Issime)\newline • Saulgau\newline • Elsässischen Dialekte\newline • Standardsprache\\
\midrule
{best. Artikel\newline ±\isi{Adjektiv}:} & \multicolumn{2}{l}{\textsc{nom/akk.sg.f + nom/akk.pl}}\\
\midrule
{Dialekte:} & \multicolumn{2}{l}{• Höchstalemannisch (außer Walser)}\\ & \multicolumn{2}{l}{• Hochalemannisch (außer Zürich)}\\
\lspbottomrule
\end{tabular}
\end{table}
%AUFZÄHLUNG SIEHT SELTSAM AUS

\section{Unbestimmter Artikel / Possessivpronomen}\label{5.6}

\subsection{Allgemeines und Realisierungsregeln}\label{5.6.1}

{Zwei Wortarten, eine Kategorie}: Der \isi{unbestimmte Artikel} und das \isi{Possessivpronomen} bilden zusammen eine Kategorie, was in \sectref{4.3.2} bereits begründet wurde und hier nur kurz wiederholt wird. Erstens weist die Flexion dieser Wortarten im Vergleich zu den anderen Wortarten die größten Ähnlichkeiten auf. Wenn die Paradigmen auf ein Minimum reduziert werden sollen, ist dies am besten zu erreichen, wenn Wortarten mit ähnlicher Flexion dieselbe Kategorie bilden. Zweitens hat sich die Flexion dieser Wortarten aus diachroner Sicht parallel entwickelt.

{Parameter DET}: In den RRs wird die Kategorie des \isi{unbestimmten Artikels} und des \isi{Possessivpronomens} mit DET2 kodiert. Wie in \sectref{5.5.1} zum \isi{bestimmten Artikel} und zum \isi{Demonstrativpronomen} gezeigt wurde, handelt es sich bei den Wortarten der Kategorien DET1 und DET2 um Determinierer, die jeweils in ihrer Form, aber nicht in ihrer Semantik große Ähnlichkeiten zeigen. Deshalb können diese Kategorien mit DET1 und DET2 kodiert werden.

Eine weitere allgemeine Beobachtung betrifft das Inventar an Wortarten. Alle der hier untersuchten Varietäten verfügen über einen \isi{unbestimmten Artikel} und ein \isi{Possessivpronomen}. Davon weicht nur das Althochdeutsche ab, das über keinen grammatikalisierten \isi{unbestimmten Artikel} verfügt \citep[26]{Schrodt2004}. Im weiteren Verlauf dieses Unterkapitels wird für beide Wortarten beschrieben, ob und welche Arten von \isi{Affixen} und Wur\-zel-/Stamm\-al\-ter\-na\-tio\-nen sie aufweisen, welche \isi{Blöcke} anzunehmen und welche morphosyntaktischen Eigenschaften zu unterscheiden sind. Diese Fragen werden zuerst für den \isi{unbestimmten Artikel} und dann für das \isi{Possessivpronomen} beantwortet. Abgeschlossen wird dieses Unterkapitel mit einigen Bemerkungen zu \isi{Synkretismen}.

Der \textsc{unbestimmte Artikel} weist in vielen untersuchten Varietäten keine \isi{Affixe} auf, was in der Diskussion der vorangehenden Wortarten schon  dargestellt wurde. In diesen Varietäten definieren die RRs also die gesamte Form. Auch sind keine \isi{Blöcke} nötig. Folgende RR definiert die Form des \isi{unbestimmten Artikels} im Dativ Singular Maskulin/Neutrum im Dialekt des Kaiserstuhls, der \textit{imɐ} lautet \citep[376]{Noth1993}:

\ea%140
\label{ex:key:140}
 RR \textsubscript{A,} \textsubscript{\{\textsc{case:dat}}\textsubscript{, \textsc{num:sg}, \textsc{gend:m}} \textsubscript{\tiny $\veebar$} \textsubscript{\scshape n}\textsubscript{\},} \textsubscript{\textsc{det2[art.indef]}} ($\langle$X,$\sigma$ $\rangle$) = \textsubscript{def} $\langle$\textit{imɐ}ˊ,$\sigma$ $\rangle$ \\
\z

Ausnahmen bilden folgende Phänomene: \isi{Suffixe} und \isi{Präfixe}. Präfigierung sowie ihre RRs und \isi{Blöcke} werden in \sectref{5.6.5} vorgestellt. Präfigierung gibt es in den folgenden Dialekten: Jaun, Sensebezirk, Uri, Zürich, Bern. Suffigierung kommt im Mittelhochdeutschen, in der deutschen Standardsprache und in den Walser Dialekten (Issime und Visperterminen) vor. Die RRs entsprechen dem Typ, der bereits für die \isi{Substantive} und die \isi{Adjektive} eingeführt wurde: Die \isi{Wurzel}\linebreak stammt aus dem \isi{Radikon}, die RRs definieren die \isi{Suffixe}. Dies ist am \isi{unbestimmten Artikel} \textit{ein-ən} (Akkusativ Singular Neutrum) in der deutschen Standardsprache exemplifiziert:

\ea%141
\label{ex:key:141}
 RR \textsubscript{A,} \textsubscript{\{\textsc{case:}}\textsubscript{\textsc{acc}}\textsubscript{, \textsc{num:sg}, \textsc{gend:m}\},} \textsubscript{\textsc{det2[]}} ($\langle$X,$\sigma$ $\rangle$) = \textsubscript{def} $\langle$X\textit{ən}ˊ,$\sigma$ $\rangle$\\
\z

Die RR von Varietäten, deren \isi{unbestimmter Artikel} suffigiert wird, stehen im selben \isi{Block}. Mehrere \isi{Blöcke} sind nicht nötig, da jede Zelle maximal ein \isi{Suffix} aufweist, die Abfolge von \isi{Suffixen} muss also nicht definiert werden. Für Varietäten mit präfigiertem unbestimmtem Artikel müssen zwei \isi{Blöcke} angenommen werden, was in \sectref{5.6.5} erörtert wird. Die morphosyntaktischen Eigenschaften, die im \isi{unbestimmten Artikel} unterschieden werden müssen, sind \isi{Kasus} und \isi{Genus}. \isi{Numerus} ist nur im Mittelhochdeutschen zu bestimmen, weil nur das Mittelhochdeutsche im Plural einen \isi{unbestimmten Artikel} hat.

Beim \textsc{Possessivpronomen} kommt die \isi{Wurzel} aus dem \isi{Radikon}, die \isi{Affixe} werden durch RRs definiert. Das \isi{Possessivpronomen} verfügt ausschließlich über \isi{Suffixe}. Zusätzlich alterniert die \isi{Wurzel} in den folgenden Varietäten: Issime, Vorarlberg, Zürich, Huzenbach, Saulgau und Petrifeld. Wie und in welchen Kontexten die \isi{Wurzeln}/Stämme variieren und wie die RRs aussehen, wird in \sectref{5.6.8} besprochen. Verfügt eine Varietät im \isi{Possessivpronomen} nur über \isi{Suffixe}, ist nur ein \isi{Block} nötig, weil nie mehr als ein \isi{Suffix} angehängt wird. Kommen zu den \isi{Suffixen} noch Wur\-zel-/Stamm\-al\-ter\-na\-tio\-nen hinzu, sind zwei \isi{Blöcke} anzunehmen, worauf in \sectref{5.6.8} genauer eingegangen wird.

Die meisten modernen Dialekte in diesem Sample haben im \isi{Possessivpronomen} unterschiedliche Suffixparadigmen, und zwar abhängig davon, an welches \isi{Possessivpronomen} suffigiert wird. Z.B. weist der Nominativ/Akkusativ Singular Feminin im Dialekt des Kaiserstuhls ein \isi{Suffix} -\textit{i} auf, aber nur im \isi{Possessivpronomen} der 3. Person Singular Feminin und der 3. Person Plural (\textit{ir}-\textit{i}, \citealt[382]{Noth1993}). In allen anderen \isi{Possessivpronomen} steht im Nominativ/Akkusativ Singular Feminin kein \isi{Suffix} (\textit{mi}, \citealt{Noth1993}: 380). In den RRs für die \isi{Possessivpronomen} sind also nicht nur die morphosynaktischen Eigenschaften des \isi{Suffixes} zu bestimmen, sondern auch jene des \isi{Possessivpronomens}. Darauf wird in \sectref{5.6.6} noch genauer eingegangen. Die RRs für das genannte Beispiel sehen also wie folgt aus:

\ea%142
\label{ex:key:142}
 RR \textsubscript{A,} \textsubscript{\{\textsc{case:} \textsc{nom}} \textsubscript{\tiny $\veebar$}\textsubscript{ \textsc{acc}}\textsubscript{, \textsc{num:sg}, \textsc{gend:f}\},} \textsubscript{\textsc{det2[pron.poss}, \textsc{pers:3}, \textsc{num:sg}, \textsc{gend:f}]} ($\langle$X,$\sigma$$\rangle$) = \textsubscript{def} $\langle$X\textit{i}ˊ,$\sigma$$\rangle$
\z

\ea%143
\label{ex:key:143}
 RR \textsubscript{A,} \textsubscript{\{\textsc{case:} \textsc{nom}} \textsubscript{\tiny $\veebar$}\textsubscript{ \textsc{acc}}\textsubscript{, \textsc{num:sg}, \textsc{gend:f}\},} \textsubscript{\textsc{det2[pron.poss}, \textsc{pers:3}, \textsc{num:pl}]} ($\langle$X,$\sigma$$\rangle$) = \textsubscript{def} $\langle$X\textit{i}ˊ,$\sigma$$\rangle$
\z

In den geschwungenen Klammern werden die morphosyntaktischen Eigenschaften des \isi{Suffixes} definiert. In den eckigen Klammern stehen weitere Angaben zur Kategorie DET2, wie Wortart und weitere Angaben zu dieser Wortart, d.h., um welches \isi{Possessivpronomen} es sich handelt. Welche Dialekte unterschiedliche Suffixparadigmen für welche \isi{Possessivpronomen} haben, wird in \sectref{5.6.6} gezeigt.

Schließlich werden die \textsc{Synkretismen} in der Kategorie DET2 wie bei den übrigen Kategorien erfasst. Die RR \REF{ex:key:140} zeigt einen Genussynkretismus, die RR \REF{ex:key:142} einen Kasussynkretismus. Wird \isi{Numerus} nicht unterschieden, bleibt \isi{Numerus} unterspezifiziert. Dasselbe gilt auch für DET2: Weisen der \isi{unbestimmte Artikel} und das \isi{Possessivpronomen} dieselbe Form auf, so muss DET2 nicht weiter spezifiziert werden (z.B. RR \ref{ex:key:141}).

\subsection{Diachrone Differenzierung der Paradigmen des unbestimmten Artikels und des Possessivpronomens}\label{5.6.2}

Die deutsche Standardsprache hat ein Set an \isi{Suffixen} für den \isi{unbestimmten Artikel} und das \isi{Possessivpronomen} (vgl. Paradigma 103). Bereits im Mittelhochdeutschen weist die Flexion des \isi{unbestimmten Artikels} und des \isi{Possessivpronomens} keine Unterschiede auf, denn beide werden stark flektiert (\citealt[216-217]{Paul2007}, Paradigma 102). Wenn dem \isi{Possessivpronomen} aber ein Artikel vorangeht, kann es stark oder schwach flektiert werden \citep[369]{Paul2007}.

Im Gegensatz zum Mittelhochdeutschen und zur deutschen Standardsprache verfügen die hier untersuchten alemannischen Dialekte über zwei separate Paradigmen für den \isi{unbestimmten Artikel} und das \isi{Possessivpronomen} (vgl. Paradigmen 104-120), wobei in einigen Dialekten die Formen einzelner Zellen der beiden Wortarten identisch sind. Keine \isi{Synkretismen} zwischen den beiden Wortarten sind in folgenden Dialekten zu finden: Issime, Jaun, Sensebezirk, Uri, Bern, Elisabethtal, Kaiserstuhl, Münstertal und Elsass. Sind Zellen der beiden Wortarten identisch, so betrifft das (mit einer Ausnahme) stets den Dativ Singular Feminin, und zwar in folgenden Dialekten: Vorarlberg, Zürich, Huzenbach, Saulgau, Stuttgart, Petrifeld und Colmar. Dies tangiert also vor allem östliche hochalemannische Dialekte, die schwäbischen Dialekte (mit Ausnahme der Sprachinsel Elisabethtal) und einen elsässischen Dialekt. Auch im Dialekt von Visperterminen sind einige Zellen der beiden Wortarten gleich, aber im Nominativ und Akkusativ Singular Neutrum sowie im Dativ und Genitiv Singular Maskulin und Neutrum.

Es kann also festgehalten werden, dass die alemannischen Dialekte (mit Ausnahme einiger wenigen Zellen) - im Gegensatz zum Mittelhochdeutschen, aber vor allem im Gegensatz zur deutschen Standardsprache - für den \isi{unbestimmten Artikel} und für das \isi{Possessivpronomen} zwei unterschiedliche Paradigmen grammatikalisiert haben. Zur Verdeutlichung soll dies am Beispiel des Nominativs Singular Feminin illustriert werden: Im Dialekt von Visperterminen lautet der \isi{unbestimmte Artikel} \textit{a}, das \isi{Possessivpronomen} \textit{m\=in}-\textit{i}, in der deutschen Standardsprache \textit{ein}-\textit{ə} bzw. \textit{mein}-\textit{ə}. Für die deutsche Standardsprache ist also nur ein Satz an \isi{Suffixen} für beide Wortarten nötig, für den Dialekt von Visperterminen jedoch zwei. Die Grammatikalisierung dieser beiden Wortarten ist folglich in den alemannischen Dialekten weiter fortgeschritten als in der deutschen Standardsprache. Die gleichen Beobachtungen wurden in \sectref{5.5.3} bezüglich des \isi{bestimmten Artikels} und des \isi{Demonstrativpronomens} gemacht.

\subsection{Unbestimmter Artikel und Possessivpronomen: Freie Variation}\label{5.6.3}

Wie für alle bis hierhin besprochenen Wortarten ist freie \isi{Variation} auch im \isi{unbestimmten Artikel} und im \isi{Possessivpronomen} zu finden. Dabei kann unterschieden werden, ob \isi{Wurzel} + \isi{Suffix} und nur die \isi{Wurzel} zueinander in freier \isi{Variation} stehen oder ob zwei \isi{Suffixe} in derselben Zelle vorkommen. Diese Typen an freier \isi{Variation} kommen in der Adjektivflexion ebenfalls vor (vgl. \sectref{5.2.3}). \tabref{table5.28} gibt einen Überblick darüber, welcher Typ freier \isi{Variation} in welcher Wortart und in welchen Dialekten zu beobachten ist. Dabei kann festgehalten werden, dass es sich um ein weit verbreitetes Phänomen handelt.

% {\tabref{table5.28}: Freie \isi{Variation} im \isi{unbestimmten Artikel} und im Possessivpronomen}\\

\begin{table}
\caption{Freie Variation im unbestimmten Artikel und im Possessivpronomen}\label{table5.28}
\begin{tabularx}{\textwidth}{XXX} 
\lsptoprule
& {unbestimmter Artikel} & {Possessivpronomen}\\
\midrule
{freie \isi{Variation} \isi{Suffix}/-ø} & Mhd & Ahd, Mhd, Jaun, Sensebezirk, Uri, Saulgau, Stuttgart\\
\midrule
{freie \isi{Variation} \isi{Suffix}/ Suffix} & Visperterminen, Sensebezirk, Uri, Vorarlberg, Zürich, Bern, Huzenbach, Saulgau, Petrifeld & Sensebezirk, Vorarlberg, Bern, Huzenbach, Stuttgart, Kaiserstuhl\\
\lspbottomrule
\end{tabularx}
\end{table}

Im \isi{Possessivpronomen} sind alle \isi{Kasus}, \isi{Genera} und beide \isi{Numeri} von freier \isi{Variation} betroffen. Es kann keine Tendenz festgestellt werden. Dies ist jedoch bezüglich des \isi{unbestimmten Artikels} möglich, wie \tabref{table5.29} zeigt. Freie \isi{Variation} kommt im \isi{unbestimmten Artikel} nur im Singular und besonders häufig im Dativ vor.

% {\tabref{table5.29}: Zellen des \isi{unbestimmten Artikels} mit freier Variation}\\

\begin{table}
\caption{Zellen des unbestimmten Artikels mit freier Variation}\label{table5.29}
\begin{tabularx}{\textwidth}{>{\scshape}XX}
\lsptoprule
{\upshape Zellen mit freier \isi{Variation} im\newline \isi{unbestimmten Artikel}:} & {Dialekte:}\\
\midrule
dat.sg.m/n/f & Sensebezirk, Vorarlberg, Zürich, Bern, Saulgau\\
dat.sg.m/n & Visperterminen, Uri\\
nom.sg.m & Huzenbach, Petrifeld\\
akk.sg.m & Petrifeld\\
akk.sg.n & Mittelhochdeutsch\\
\lspbottomrule
\end{tabularx}
\end{table}

Wie schon mehrmals gezeigt, müssen die RRs gleich spezifisch sein, damit sie dieselbe Zelle des Paradigmas definieren können. Dies wurde in \sectref{4.1.3.3} theoretisch begründet. Folglich muss auch die \isi{Wurzel} stipuliert werden, wenn diese mit \isi{Wurzel} + \isi{Suffix} in freier \isi{Variation} steht. Nur so wird gewährleistet, dass beide Formen dieselbe Zelle definieren. Dies wird an den folgenden RRs für das \isi{Possessivpronomen} der 1. und 2. Person Singular im Dialekt des Sensebezirks illustriert (vgl. Paradigma 107). Die RRs \REF{ex:key:144} und \REF{ex:key:145} definieren -ø/-\textit{ər (}z.B. \textit{mi}, \textit{minər}\footnote{-\textit{N}- dient synchron der Hiatbeseitigung und wird von der Phonologie automatisch eingefügt. Genauer diskutiert wird dies in \sectref{5.6.8}.} im Nominativ und Akkusativ Plural), die RRs \REF{ex:key:146} und \REF{ex:key:147} -\textit{um}/-\textit{m} (z.B. \textit{mim}, \textit{minum}\footnote{Vgl. vorangehende Fußnote.} im Dativ Singular Maskulin und Neutrum) \citep[198]{Henzen1927}.

\ea%144
\label{ex:key:144}
 RR \textsubscript{A,} \textsubscript{\{\textsc{case:} \textsc{nom}} \textsubscript{\tiny $\veebar$}\textsubscript{ \textsc{acc}}\textsubscript{, \textsc{num:pl}\}}\textsubscript{,} \textsubscript{\textsc{det2[pron.poss}, \textsc{pers:1}} \textsubscript{\tiny $\veebar$} \textsubscript{\textsc{2}}\textsubscript{, \textsc{num:sg}]} ($\langle$X,$\sigma$ $\rangle$) = \textsubscript{def} $\langle$Xˊ,$\sigma$ $\rangle$
\z

\ea%145
\label{ex:key:145}
 RR \textsubscript{A,} \textsubscript{\{\textsc{case:} \textsc{nom}} \textsubscript{\tiny $\veebar$}\textsubscript{ \textsc{acc}}\textsubscript{, \textsc{num:pl}\}}\textsubscript{,} \textsubscript{\textsc{det2[pron.poss}, \textsc{pers:1}} \textsubscript{\tiny $\veebar$} \textsubscript{\textsc{2}}\textsubscript{, \textsc{num:sg}]} ($\langle$X,$\sigma$ $\rangle$) = \textsubscript{def} $\langle$X\textit{ər}ˊ,$\sigma$ $\rangle$
\z

\ea%146
\label{ex:key:146}
 RR \textsubscript{A,} \textsubscript{\{\textsc{case: dat}, \textsc{num:sg}, \textsc{gend: m}} \textsubscript{\tiny $\veebar$} \textsubscript{\textsc{n}\}}\textsubscript{,} \textsubscript{\textsc{det2[pron.poss}, \textsc{pers:1}} \textsubscript{\tiny $\veebar$} \textsubscript{\textsc{2}}\textsubscript{, \textsc{num:sg}]} ($\langle$X,$\sigma$ $\rangle$) = \textsubscript{def}   $\langle$X\textit{um}ˊ,$\sigma$ $\rangle$
\z

\ea%147
\label{ex:key:147}
 RR \textsubscript{A,} \textsubscript{\{\textsc{case: dat}, \textsc{num:sg}, \textsc{gend: m}} \textsubscript{\tiny $\veebar$} \textsubscript{\textsc{n}\}}\textsubscript{,} \textsubscript{\textsc{det2[pron.poss}, \textsc{pers:1}} \textsubscript{\tiny $\veebar$} \textsubscript{\textsc{2}}\textsubscript{, \textsc{num:sg}]} ($\langle$X,$\sigma$ $\rangle$) = \textsubscript{def}   $\langle$X\textit{m}ˊ,$\sigma$ $\rangle$
\z

\subsection{Unbestimmter Artikel: Syntaktisch bedingte Variation}\label{5.6.4}

{(Präposition +) Artikel + Substantiv}: Wie im \isi{bestimmten Artikel} (vgl. \sectref{5.5.5}) gibt es auch im \isi{unbestimmten Artikel} syntaktisch bedingte \isi{Variation}. Im Gegensatz zum \isi{bestimmten Artikel} kommt diese jedoch nur im Kontext ± vorangehende Präposition vor. Im \isi{bestimmten Artikel} betrifft dies die Zellen Akkusativ Singular Maskulin und Dativ Singular Maskulin/Neutrum, im \isi{unbestimmten Artikel} ebenfalls Akkusativ und Dativ, aber jeweils alle drei \isi{Genera}. In der Folge wird zuerst die \isi{Variation} im Dativ und anschließend jene im Akkusativ vorgestellt.

{Dativ}: In allen Dialekten fallen im Dativ Singular die Formen des Maskulins und des Neutrums zusammen, das Feminin hat eine eigene Form, z.B. \textit{æmə}/\textit{əmə} (mask. und neut.), \textit{ænərə}/\textit{ənərə} (fem.) (Münstertal, \citealt[45]{Mankel1886}). Alle Dialekte, deren \isi{unbestimmter Artikel} syntaktisch bedingte \isi{Variation} zeigt, haben zwei Formen: eine für den Kontext, wenn keine Präposition vorangeht, eine für den Kontext, wenn eine konsonantisch auslautende Präposition vorangeht. Im Dialekt des Münstertals wird \textit{æmə}/\textit{ænərə} verwendet, wenn dem \isi{unbestimmten Artikel} keine Präposition vorangeht, \textit{əmə}/\textit{ənərə}, wenn dem \isi{unbestimmten Artikel} eine konsonantisch auslautende Präposition vorangeht \citep[45-46]{Mankel1886}. Diese \isi{Variation} kann phonologisch nicht erklärt werden. In allen betroffenen Dialekten sind folglich im Dativ für Maskulin/Neutrum und Feminin jeweils zwei \isi{unbestimmte Artikel} anzunehmen, d.h. zwei RRs. Des Weiteren kann die Form des \isi{unbestimmten Artikels}, der einer vokalisch auslautenden Präposition folgt, stets von einer der beiden beschriebenen Formen abgeleitet werden. Im Dialekt von Münstertal lautet der \isi{unbestimmte Artikel} nach vokalisch auslautender Präposition \textit{mə}/\textit{nərə} \citep[45]{Mankel1886}. Um einen \isi{Hiat} zu vermeiden, wird also dem \isi{unbestimmten Artikel} nach vokalisch auslautender Präposition die erste Silbe getilgt: \textit{zu}-\textit{əmə} $\rightarrow$ \textit{zu}-\textit{mə}. Interessanterweise wird hier jedoch nicht die Defaultstrategie verwendet, um einen \isi{Hiat} zu vermeiden, nämlich das Einfügen von \textit{n}. Man würde also eigentlich erwarten: *\textit{zu}-\textit{n}-\textit{əmə}. Da hier nicht die Defaultstratiegie eingesetzt wird, die für die gesamte Phonologie gilt (n-Epenthese), sondern eine Silbe getilgt wird, muss dies durch eine RR definiert werden. Man benötigt aber nicht für jede Form des \isi{unbestimmten Artikels} nach auf Vokal auslautender Präposition eine RR. Weil die erste Silbe einer bereits vorhandenen Form getilgt wird, ist nur eine RR nötig, die diese Tilgung definiert (vgl. Anhang B: Elsass (Ebene) RR \REF{ex:key:81}, Münstertal RR \REF{ex:key:81}, Zürich RR \REF{ex:key:95}, Bern RR \REF{ex:key:111}, Uri RR \REF{ex:key:124}, Sensebezirk RR \REF{ex:key:124}, Jaun RR \REF{ex:key:136}). Folglich stehen diese Tilgungsregeln auch im letzten \isi{Block}.

\tabref{table5.30} gibt eine Übersicht, in welchen Dialekten im Dativ des \isi{unbestimmten Artikels} syntaktisch bedingte \isi{Variation} auftritt. Die Dialekte von Petrifeld und Colmar weisen nach konsonantisch und vokalisch auslautender Präposition denselben Artikel auf. In allen anderen Dialekten wird der \isi{unbestimmte Artikel} nach vokalisch auslautender Präposition (PP V) entweder von jener Form abgeleitet, die einer konsonantisch auslautenden Präposition folgt (PP K), oder von jener Form, der keine Präposition vorausgeht (NP).

% {\tabref{table5.30}: Formen des \isi{unbestimmten Artikels} bei syntaktisch bedingter \isi{Variation} im Dativ}\\

\begin{table}
\caption{Formen des unbestimmten Artikels bei syntaktisch bedingter Variation im Dativ}\label{table5.30}
\begin{tabularx}{\textwidth}{l*{5}{X}} 
\lsptoprule
& \multicolumn{3}{c}{\textsc{dat.m+n}} & \multicolumn{2}{c}{\textsc{dat.f}}\\\cmidrule(lr){2-4}\cmidrule(lr){5-6}
& \textsc{np} & \textsc{pp (k)} & \textsc{pp (v)} & \textsc{np} & \textsc{pp (k)}\\
\midrule
Elsass (Ebene) & imə & əmə & mə & inərə & ərə\\
% % \midrule
Münstertal & æmə & əmə & mə & ænərə & ənərə\\
% % \midrule
Colmar & eme & me & me & enre & re\\
% % \midrule
Petrifeld & imə & əmə & əmə & inrə & rə\\
% % \midrule
Zürich & əmənə & əmə & mənə, mə & ənərə & ərə\\
% % \midrule
Bern & əmənə, əmnə, əmə & əmənə, əmnə, əmə & mənə & ənərə, ərə & ənərə, ərə\\
% % \midrule
Uri & ɑmənɐ, ɑmɐ & əmɐ & mənɐ, mɐ & ɑnərɐ & ərɐ\\
% % \midrule
Sensebezirk & (a,i)məna, (a,i)ma & əməna & məna, ma & (a,i)nəra, əra & ənəra\\
% % \midrule
Jaun & əmənə & əmənə & mənə & andərə & ərə\\
\lspbottomrule
\end{tabularx}
\end{table}

Der \isi{unbestimmte Artikel} im Dativ zeigt also syntaktisch bedingte \isi{Variation} in den höchstalemannischen Dialekten (außer den Walser Dialekte), in den elsässischen Dialekten sowie im Dialekt von Bern, Zürich und Petrifeld (vgl. \tabref{table5.30}). Der \isi{bestimmte Artikel} weist syntaktisch bedingte \isi{Variation} in den höchstalemannischen Dialekten (außer Issime), in den elsässischen Dialekten sowie im Dialekt von Saulgau (vgl. \tabref{table5.27}). Es kann also festgehalten werden, dass syntaktisch bedingte \isi{Variation} im Dativ vor allem ein Phänomen der höchstalemannischen Dialekte (außer den Walser Dialekte) und der elsässischen Dialekte ist (vgl. \tabref{table5.31}).

Schließlich zählen etliche Grammatiken unterschiedliche Varianten für den Dativ auf, ohne jedoch auf ihre Distribution einzugehen. Wie beim \isi{bestimmten Artikel} wird in diesen Fällen auch beim \isi{unbestimmten Artikel} davon ausgegangen, dass es sich um freie \isi{Variation} handelt. Zum Beispiel hat der Dialekt von Bern drei Varianten: \textit{əmənə}, \textit{əmnə}, \textit{əmə} (\citealt[79]{Marti1985}; vgl. auch \tabref{table5.30}). Dies betrifft folgende Dialekte: Visperterminen (nur Maskulin/Neutrum), Sensebezirk, Bern, Vorarlberg, Saulgau.

{Akkusativ}: Wie der \isi{bestimmte Artikel} variiert auch der \isi{unbestimmte Artikel} im Akkusativ Singular, jedoch in allen \isi{Genera}, während im \isi{bestimmten Artikel} nur das Maskulin davon betroffen ist. Im Akkusativ Singular fallen Maskulin und Feminin zusammen, das Neutrum hat eine eigene Form, z.B. \textit{a}/\textit{əna} (Maskulin/Feminin), \textit{as}/\textit{ənas} (Neutrum) (Sensebezirk, \citealt[194]{Henzen1927}). Die kürzeren Formen \textit{a}/\textit{as} werden verwendet, wenn dem \isi{unbestimmten Artikel} keine Präposition vorangeht, die längeren Formen \textit{əna}/\textit{ənas}, wenn eine Präposition vorangeht. Bei den längeren Formen handelt es sich um präfigierte Formen, auf deren Bildung im nachfolgenden Abschnitt \sectref{5.6.5} noch eingegangen wird. Es kann aber schon festgehalten werden, dass für die kürzeren und längeren Formen jeweils RRs benötigt werden, um ihre Form zu definieren. Lautet die Präposition konsonantisch aus, so folgen die genannten Formen \textit{əna}/\textit{ənas}. Lautet die Präposition vokalisch aus, wird der anlautende Vokal des \isi{unbestimmten Artikels} getilgt (\textit{na}/\textit{nas}), um einen \isi{Hiat} zu vermeiden. Im Gegensatz zum Dativ ist im Akkusativ der \isi{unbestimmte Artikel} nach vokalisch auslautender Präposition immer von jener Form abzuleiten, welche einer konsonantisch auslautenden Präposition folgt. Wie beim Dativ des \isi{unbestimmten Artikels} ist für die Tilgung der ersten Silbe auch im Akkusativ eine RR im letzten \isi{Block} anzunehmen, da nicht die Defaultstrategie zur Hiatvermeidung verwendet wird (n-Epenthese), sondern die erste Silbe getilgt wird. Genauer gesagt, reicht eine RR für den Akkusativ und den Dativ, da eine RR für beide \isi{Kasus} definiert werden kann, indem beide \isi{Kasus} bei den morphosyntaktischen Eigenschaften in der RR angegeben werden (vgl. Liste an RRs oben bezüglich des Dativs).

Wie beim \isi{unbestimmten Artikel} im Dativ ist die \isi{Variation} im Akkusativ ebenfalls in den Dialekten Jaun, Sensebezirk und Uri vorhanden (höchstalemannisch, außer den Walser Dialekte). Dazu kommen die Dialekte von Bern und Zürich. \tabref{table5.31} fasst die \isi{Variation} im bestimmten und \isi{unbestimmten Artikel} zusammen. Die \isi{Variation} ±Präposition+Artikel ist sowohl im Akkusativ als auch im Dativ im bestimmten und \isi{unbestimmten Artikel} zu beobachten, wobei im unbestimmten alle \isi{Genera} betroffen sind, während dies im \isi{bestimmten Artikel} nur für das Maskulin und Neutrum gilt. Ausschließlich der \isi{bestimmte Artikel} variiert in Abhängigkeit eines nachfolgenden \isi{Adjektivs}, und zwar nur im Nominativ/Akkusativ Singular Feminin und im Nominativ/Akkusativ Plural. Bezüglich der Dialekte kann ein höchstalemannisches Zentrum an Artikelvariation beobachtet werden: Die höchstalemannischen Dialekte mit Ausnahme der Walser Dialekte weisen alle fünf Variationen auf. Als mögliche Hypothese könnte das Fehlen dieser \isi{Variation} in den Walser Dialekten damit erklärt werden, dass diese Dialekte als isoliert und archaisch gelten können und somit resistenter gegenüber Neuerungen sind. Die hochalemannischen Dialekte zeigen \isi{Variation} vor allem im Akkusativ (weniger im Dativ) sowie im Kontext ±\isi{Adjektiv}. Demgegenüber variieren die elsässischen Dialekte vorwiegend im Dativ. Schließlich weist der Dialekt von Saulgau \isi{Variation} nur im \isi{bestimmten Artikel} im Kontext ±Präposition auf (Akkusativ und Dativ), jedoch nicht im \isi{unbestimmten Artikel}. Das Gegenteil gilt für den Dialekt von Zürich: Nur der \isi{unbestimmte Artikel} variiert, aber ebenfalls im Akkusativ und Dativ. Dies sind doch erstaunliche areale Muster, welchen in einer weiteren Arbeit nachgegangen werden sollte.

% {\tabref{table5.31}: Syntaktisch bedingte \isi{Variation} im bestimmten und unbestimmten Artikel}\\

\begin{table}
\caption{Syntaktisch bedingte Variation im bestimmten und unbestimmten Artikel}\label{table5.31}
\begin{tabular}{>{\raggedright}p{2.75cm}p{4cm}p{4cm}}
\lsptoprule
± Präposition\newline + unbest. Artikel & \textsc{akk.sg.m/n/f} & \textsc{dat.sg.m/n/f}\\
\midrule
{Dialekte:} & • Höchstalemannisch\newline\hspaceThis{•} (außer Walser)\newline • Bern\newline • Zürich & 
• Höchstalemannisch\newline\hspaceThis{•} (außer Walser)\newline • Elsässische Dialekte\newline • Bern\newline • Zürich\newline • Petrifeld\\
\midrule
{± Präposition\newline + best. Artikel:} & \textsc{akk.sg.m} & \textsc{dat.sg.m/n}\\
\midrule
{Dialekte:} & • Höchstalemannisch\newline\hspaceThis{•} (außer Walser)\newline • Hochalemannisch\newline\hspaceThis{•} (außer Zürich)\newline • Saulgau\newline • Elsass (Ebene)& • Höchstalemannisch\newline\hspaceThis{•} (außer Issime)\newline • Elsässische Dialekte\newline • Saulgau\newline • Standard\\
\midrule
{best. Artikel\newline ±\isi{Adjektiv}:} & \multicolumn{2}{l}{\textsc{nom/akk.sg.f + nom/akk.pl}}\\
\midrule
{Dialekte:} & \multicolumn{2}{l}{• Höchstalemannisch (außer Walser)} \\& \multicolumn{2}{l}{ • Hochalemannisch (außer Zürich)}\\
\lspbottomrule
\end{tabular}
\end{table}
%AUFZÄLUNG SIEHT SELTSAM AUS

\subsection{Unbestimmter Artikel: Präfixe}\label{5.6.5}

Im vorangehenden Kapitel wurde gezeigt, dass die präfigierten Formen des \isi{unbestimmten Artikels} im Akkusativ in einem bestimmten syntaktischen Kontext verwendet werden, und zwar nach einer Präposition. Außerdem kommen diese Formen nur in den folgenden Dialekten vor: Jaun, Sensebezirk, Uri, Zürich, Bern. Hier geht es darum, die Bildung ihrer Form zu erfassen.

Dass es sich um präfigierte Formen handelt, demonstriert ein Vergleich des Akkusativs mit dem Nominativ. \tabref{table5.32} zeigt die Nominativ- und Akkusativformen jener Dialekte, die einen präfigierten Akkusativ aufweisen. Die Nominativform für das Maskulin/Feminin und das Neutrum ist in der Akkusativform beinhaltet. Alle drei \isi{Genera} und alle fünf Dialekte zeigen im Akkusativ dasselbe \isi{Affix}, das an die Nominativform präfigiert wird: -\textit{ən}/-\textit{n}.

% {\tabref{table5.32}: Präfigierte unbestimmte Artikel}\\

\begin{table}
\caption{Präfigierte unbestimmte Artikel}\label{table5.32}
\resizebox{\textwidth}{!}{\begin{tabular}{*{7}{l}}
\lsptoprule
{Dialekt} & \multicolumn{2}{c}{\textsc{nom.sg}} & \multicolumn{4}{c}{\textsc{akk.sg}}\\\cmidrule(lr){2-3}\cmidrule(lr){4-7}
& \textsc{m/f} & \textsc{n} & \multicolumn{2}{c}{\scshape m/f} & \multicolumn{2}{c}{\scshape n}\\
&  &  & \mbox{-präfigiert} & \mbox{+präfigiert} & \mbox{-präfigiert} & \mbox{+präfigiert}\\
\midrule
Jaun & a & as & a & ən-a & as & ən-as\\
Sensebezirk & a & as & a & ən-a & as & ən-as\\
Uri & ɐ & ɐs & ɐ & ən{}-ɐ & ɐs & ən-ɐs\\
Zürich & ən & əs & ən & ən-ən & əs & ən-əs\\
Bern & ə & əs & ə & n-ə & əs & n-əs\\
\lspbottomrule
\end{tabular}}
\end{table}

In diesen Dialekten ist also im Akkusativ Singular aller \isi{Genera} eine RR für das \isi{Präfix} anzunehmen. Wie in \sectref{5.6.1} besprochen wurde, muss die gesamte Form des \isi{unbestimmten Artikels} durch RR definiert werden, was in \isi{Block} A passiert (RRs \ref{ex:key:148} und \ref{ex:key:149}). Die \isi{Präfixe} stehen dann in einem weiteren \isi{Block} B. Folgende RRs exemplifizieren dies am Dialekt des Sensebezirks, wobei \isi{Genus} in RR \REF{ex:key:150} unterspezifiziert bleibt, da in allen \isi{Genera} präfigiert wird:

\ea%148
\label{ex:key:148}
 RR \textsubscript{A,} \textsubscript{\{\textsc{case:nom}} \textsubscript{\tiny $\veebar$}\textsubscript{ \textsc{acc}}\textsubscript{, \textsc{num:sg}, \textsc{gend:m}} \textsubscript{\tiny $\veebar$}\textsubscript{ \textsc{f}\},} \textsubscript{\textsc{det2[art.indef]}} ($\langle$X,$\sigma$ $\rangle$) = \textsubscript{def} $\langle$X\textit{a}ˊ,$\sigma$ $\rangle$
\z

\ea%149
\label{ex:key:149}
 RR \textsubscript{A,} \textsubscript{\{\textsc{case:nom}} \textsubscript{\tiny $\veebar$}\textsubscript{ \textsc{acc}}\textsubscript{, \textsc{num:sg}, \textsc{gend:n}\},} \textsubscript{\textsc{det2[art.indef]}} ($\langle$X,$\sigma$ $\rangle$) = \textsubscript{def} $\langle$X\textit{as}ˊ,$\sigma$ $\rangle$
\z

\ea%150
\label{ex:key:150}
 RR \textsubscript{B,} \textsubscript{\{\textsc{case:}}\textsubscript{\textsc{acc}}\textsubscript{, \textsc{num:sg}\},} \textsubscript{\textsc{det2[art.indef]}} ($\langle$X,$\sigma$ $\rangle$) = \textsubscript{def} $\langle$\textit{ən}Xˊ,$\sigma$ $\rangle$
\z

\subsection{Possessivpronomen: Unterschiedliche Paradigmen}\label{5.6.6}

Wie schon in \sectref{5.6.1} erwähnt wurde, weisen die \isi{Possessivpronomen} unterschiedliche Suffixparadigmen auf, je nachdem um welches \isi{Possessivpronomen} es sich handelt. In diesen Fällen ist in den RRs also auch das \isi{Possessivpronomen} zu definieren, wie dies anhand der RRs \REF{ex:key:142} und \REF{ex:key:143} illustriert wurde. Dies gilt für fast alle alemannischen Dialekte dieses Samples, jedoch nicht für das Alt- und Mittelhochdeutsche sowie die deutsche Standardsprache. Diese drei Varietäten verfügen über nur ein Paradigma an \isi{Suffixen} für alle \isi{Possessivpronomen}. Im Alt- und Mittelhochdeutschen jedoch kommen noch zwei unveränderliche \isi{Possessivpronomen} hinzu, nämlich die 3. Person Singular Feminin und die 3. Person Plural, welche keine \isi{Suffixe} aufweisen. Darauf wird in \sectref{5.6.7} noch genauer eingegangen. Ebenfalls nur ein Paradigma haben die elsässischen Dialekte wie auch die Dialekte von Huzenbach und Visperterminen, wobei Visperterminen zusätzlich dieselben unveränderlichen \isi{Possessivpronomen} wie das Alt- und Mittelhochdeutsche aufweist. Auch die Dialekte von Issime, Jaun und des Sensebezirks haben diese unveränderlichen \isi{Possessivpronomen}, die in der anschließenden Diskussion dieses Kapitels nicht berücksichtigt werden.

Es verfügen also alle Dialekte (außer Visperterminen, Huzenbach, elsässische Dialekte) über unterschiedliche Paradigmen in Abhängigkeit des \isi{Possessivpronomens} (vgl. \tabref{table5.33}). Zwei Paradigmen haben Vorarlberg, Zürich, Stuttgart, Elisabethtal, drei Paradigmen Jaun, Sensebezirk, Uri, Saulgau, Kaiserstuhl, Petrifeld, vier Paradigmen Issime und Bern. Ein areales Muster ist also nicht zu erkennen.

Dafür gibt es ein Muster bezüglich der Art des \isi{Possessivpronomens}. In all diesen Dialekten bilden die \isi{Possessivpronomen} der 1. und 2. Person Singular sowie der 3. Person Singular Maskulin und Neutrum zusammen ein Paradigma, das sich von jenem der 1. und 2. Person Plural unterscheidet, welche ihrerseits ebenfalls zusammen ein Paradigma formen. Die einzige Ausnahme davon ist der Dialekt von Bern.

Für die \isi{Possessivpronomen} der 3. Person Singular Feminin und der 3. Person Plural sind (mit ein paar Ausnahmen) drei Typen festzustellen: Entweder sind sie unveränderlich, bilden zusammen ein drittes Paradigma oder gehören zum Paradigma der 1./2. Person Plural. Dies soll nun genauer beschrieben werden.

Mehrheitlich formen die Pronomen der 3. Person Singular Feminin und der 3. Person Plural zusammen ein drittes Paradigma (Uri, Saulgau und Kaiserstuhl) oder sie funktionieren gleich wie die Pronomen der 1. und 2. Person Plural (Zürich, Stuttgart und Elisabethtal). Jaun und Sensebezirk haben unveränderliche \isi{Possessivpronomen}. In Issime ist die 3. Person Singular Feminin unveränderlich, die 3. Person Plural bildet ein eigenes Paradigma. Dies gilt auch für Petrifeld, wobei hier die 3. Person Singular Feminin mit der 1. und 2. Person Plural geht. Im Dialekt von Vorarlberg wird die 3. Person Singular Feminin und die 3. Person Plural analytisch gebildet, nämlich mit dem \isi{Demonstrativpronomen} + der 3. Person Singular Maskulin/Neutrum \citep[275]{Jutz1925}. Da diese Formen bereits durch RRs zur Verfügung stehen, müssen sie nicht nochmal definiert werden. Bern entspricht keiner der benannten Regelmäßigkeiten.

% {\tabref{table5.33}: Paradigmen des \isi{Possessivpronomens} in den Dialekten}\\

\begin{table}
\caption{Paradigmen des Possessivpronomens in den Dialekten}\label{table5.33}
\begin{tabularx}{\textwidth}{>{\scshape}X>{\scshape}X>{\scshape}X>{\scshape}XX}
\lsptoprule
1.2.sg, 3.sg.m/n & 1.2.pl & \mbox{3.sg.f, 3.pl} ({\upshape unveränderlich}) &  & Jaun, Sensebezirk\\
\midrule
1.2.sg, 3.sg.m/n & 1.2.pl & 3.sg.f ({\upshape \mbox{unveränder}\-lich}) & 3.pl   & Issime\\
\midrule                                                           
1.2.sg, 3.sg.m/n & 1.2.pl & 3.sg.f, 3.pl &                   & Uri, Saulgau, Kaiserstuhl\\
\midrule                                                           
1.2.sg, 3.sg.m/n & 1.-3.pl, 3.sg.f &  &                       & Zürich, Stuttgart, Elisabethtal\\
\midrule                                                           
1.2.sg, 3.sg.m/n & 1.2.pl,3.sg.f & 3.pl &                    & Petrifeld\\
\midrule                                                           
1.2.sg, 3.sg.m/n & 1.2.pl &  &                                 & Vorarlberg\\
\midrule                                                           
1.sg & 2.sg, 3.sg.m/n & 1.pl & 2.3.pl, 3.sg.f               & Bern\\
\lspbottomrule
\end{tabularx}
\end{table}

\subsection{Possessivpronomen: Unveränderliche Formen}\label{5.6.7}

Alt- und Mittelhochdeutsch sowie alle höchstalemannischen Dialekte außer Uri haben in der 3. Person Singular Feminin und in der 3. Person Plural unveränderliche \isi{Possessivpronomen}, z.B. \textit{ira} (3. Person Singular Feminin), \textit{iro} (3. Person Plural) (Althochdeutsch, \citealt[245]{Braune2004}). Diese \isi{Possessivpronomen} flektieren also nicht nach \isi{Kasus}, \isi{Numerus} und \isi{Genus}. Der Dialekt von Issime weist ein solches \isi{Possessivpronomen} nur in der 3. Person Singular Feminin auf, das \isi{Possessivpronomen} der 3. Person Plural flektiert (vgl. \tabref{table5.33} und Paradigma 104). In all diesen Varietäten wurde das unveränderliche \isi{Possessivpronomen} aus dem Genitiv des \isi{Personalpronomens} der 3. Person Singular Feminin bzw. der 3. Person Plural übernommen.

Für diese unveränderlichen \isi{Possessivpronomen} sind keine RRs nötig. Ihre\linebreak Form stammt aus dem \isi{Radikon} wie die \isi{Wurzeln} aller anderen \isi{Possessivpronomen}. Beispielsweise muss für das \isi{Possessivpronomen} \textit{mein}-\textit{ən} (Akkusativ Singular Maskulin, deutsche Standardsprache) nur das \isi{Suffix} durch RRs definiert werden, die \isi{Wurzel} \textit{mein} kommt aus dem \isi{Radikon}. Wird in einer Zelle nicht suffigiert, z.B. \textit{mein} (Nominativ Singular Maskulin, deutsche Standardsprache), wird keine RR benötigt, da die \isi{Wurzel} per Default eingefügt wird (vgl. \sectref{4.1.3.2}, RR \textit{Identity Function Default}). Ebenfalls per Default füllen die unveränderlichen \isi{Possessivpronomen} ihre Paradigmen.

Eine Ausnahme bildet hier nur der Dialekt des Sensebezirks. In diesem Dialekt hat das unveränderliche \isi{Possessivpronomen} folgende Formen: \textit{\=ira}/\textit{\=iras} (3. Person Singular Feminin), \textit{\=irə}/\textit{\=irəs} (3. Person Plural). Auch diese flektieren nicht nach \isi{Numerus}, \isi{Kasus} und \isi{Genus}, weisen jedoch zwei Varianten auf, nämlich eine mit -\textit{s} suffigierte und eine ohne \isi{Suffix}. Es liegt also freie \isi{Variation} des Typs \isi{Wurzel}/\isi{Wurzel}+\isi{Suffix} vor, für den zwei gleich spezifische RRs nötig sind, damit beide RRs dieselbe Zelle definieren (vgl. Diskussion zur freien \isi{Variation} in den \isi{Substantiven} \sectref{5.1.2}, \isi{Adjektiven} \sectref{5.2.3} und \isi{Possessivpronomen} \sectref{5.6.3} sowie die theoretische Begründung \sectref{4.1.3.3}). Für die 3. Person Singular Feminin und die 3. Person Plural im Dialekt des Sensebezirks müssen also folgende RRs angesetzt werden:\largerpage

\ea%151
\label{ex:key:151}
 RR \textsubscript{A,} \textsubscript{\{\},} \textsubscript{\textsc{det2[pron.poss}, \textsc{pers:3}, \textsc{num:sg}, \textsc{gend:f}]} ($\langle$X,$\sigma$$\rangle$) = \textsubscript{def} $\langle$Xˊ,$\sigma$$\rangle$
\z

\ea%152
\label{ex:key:152}
 RR \textsubscript{A,} \textsubscript{\{\},} \textsubscript{\textsc{det2[pron.poss}, \textsc{pers:3}, \textsc{num:sg}, \textsc{gend:f}]} ($\langle$X,$\sigma$$\rangle$) = \textsubscript{def} $\langle$X\textit{s}ˊ,$\sigma$$\rangle$
\z

\ea%153
\label{ex:key:153}
 RR \textsubscript{A,} \textsubscript{\{\},} \textsubscript{\textsc{det2[pron.poss}, \textsc{pers:3}, \textsc{num:pl}]} ($\langle$X,$\sigma$$\rangle$) = \textsubscript{def} $\langle$Xˊ,$\sigma$$\rangle$
\z

\ea%154
\label{ex:key:154}
 RR \textsubscript{A,} \textsubscript{\{\},} \textsubscript{\textsc{det2[pron.poss}, \textsc{pers:3}, \textsc{num:pl}]} ($\langle$X,$\sigma$$\rangle$) = \textsubscript{def} $\langle$X\textit{s}ˊ,$\sigma$$\rangle$
\z

\subsection{Possessivpronomen (und unbestimmter Artikel): Wur\-zel-/Stamm\-al\-ter\-na\-tio\-nen}\label{5.6.8}

Hier werden zwei Wur\-zel-/Stamm\-al\-ter\-na\-tio\-nen im \isi{Possessivpronomen} vorgestellt. Für den ersten Typ sind keine RRs anzunehmen, die Wur\-zel-/Stamm\-al\-ter\-na\-tion wird von der Phonologie gesteuert (n-Einschub zur Hiatvermeidung). Der zweite Typ zeichnet sich dadurch aus, dass die Alternationen durch RRs definiert werden müssen.\largerpage

{N-Einschub zur Hiatvermeidung}: In vielen alemannischen Dialekten hat das \isi{Possessivpronomen} der 1. und 2. Person Singular sowie der 3. Person Singular Maskulin/Neutrum das auslautende \textit{n} verloren. Es taucht jedoch wieder auf, wenn der \isi{Wurzel} ein vokalisch anlautendes \isi{Suffix} folgt, z.B. \textit{mi} (Nominativ/Akkusativ Singular Maskulin), \textit{min}-\textit{a} (Nominativ/Akkusativ Singular Maskulin), \textit{mi}-\textit{s} (Genitiv Singular Maskulin/Neutrum) (Jaun, \citealt[284]{Stucki1917}). Aus synchroner Sicht kann also festgehalten werden, dass ein \textit{n} eingefügt wird, wenn die \isi{Wurzel} vokalisch auslautet und das \isi{Suffix} vokalisch anlautet. Wie bereits in \sectref{5.1.4} gezeigt wurde, wird in den alemannischen Dialekten ein \textit{n} zur Hiatvermeidung per Default eingefügt. Da also die Phonologie den \isi{Hiat} automatisch beseitigt, müssen dafür keine RRs angenommen werden. Dies trifft auf folgende Dialekte zu: Jaun, Sensebezirk, Uri, Vorarlberg, Bern, Saulgau, Stuttgart, Petrifeld, Elisabethtal, Kaiserstuhl und Colmar.

Dasselbe ist im \isi{unbestimmten Artikel} des Dialekts von Visperterminen zu beobachten. Folgt dem \isi{unbestimmten Artikel} ein vokalisch anlautendes \isi{Suffix}, wird ein \textit{n} eingeschoben: \textit{a} (Nominativ Singular Maskulin), \textit{a}-\textit{s} (Genitiv Singular Maskulin), \textit{an}-\textit{um} (Dativ Singular Maskulin) \citep[137]{Wipf1911}.

{Nicht phonologisch bedingte Wur\-zel-/Stamm\-al\-ter\-na\-tio\-nen}: Bei den in der Folge diskutierten Fällen können die Wur\-zel-/Stamm\-al\-ter\-na\-tio\-nen nicht durch allgemeine phonologische Regeln erklärt werden, die für das gesamte System gelten. Deswegen sind dafür RRs anzusetzen. Bei diesen  Wur\-zel-/Stamm\-al\-ter\-na\-tio\-nen handelt es sich sowohl um Einfügungen als auch um Tilgungen. Des Weiteren sind sie entweder vom phonologischen Kontext oder vom morphosyntaktischen Kontext abhängig. Eine Übersicht gibt \tabref{table5.34}. Solche  Wur\-zel-/Stamm\-al\-ter\-na\-tio\-nen kommen in folgenden Dialekten vor, die anschließend in dieser Reihenfolge erörtert werden: Saulgau, Issime, Zürich, Huzenbach, Vorarlberg und Petrifeld.

%{\tabref{table5.34}: \isi{Wurzel-/Stammalternationen} im \isi{Possessivpronomen} der Dialekte}\\
              
\begin{table}
\caption{Wurzel-/Stammalternationen im Possessivpronomen der Dialekte}\label{table5.34}
\begin{tabularx}{\textwidth}{lXl}
\lsptoprule
\multicolumn{3}{c}{{Abhängig vom phonologischen Kontext}}\\\cmidrule(lr){1-3}
Prozess & Dialekt & Beispiel\\
\midrule
Einfügen von K (\textit{r}) & Saulgau & \textit{əisə}, \textit{əisər}-\textit{e}\\
Tilgen von V (\textit{ə}, \textit{u}) & Saulgau,\newline Issime & \textit{iərə}, \textit{iər}-\textit{e} \textit{üriu}, \textit{üri}-\textit{er}\\
Tilgen von K (\textit{n}) & Issime, Zürich & \textit{mein}, \textit{mei}-\textit{s} \textit{m\=in}, \textit{m\=i}-\textit{s}\\
\raggedright Verlust Nasalierung & Huzenbach & \textit{mãẽ}, \textit{mae}-\textit{ərə}\\

\midrule
 \multicolumn{3}{c}{{Abhängig vom morphosyntaktischen Kontext}}\\\cmidrule(lr){1-3}
Prozess & Dialekt & Beispiel\\
\midrule
Einfügen von K (\textit{n}) & Petrifeld & \textit{m\=ai}, \textit{m\=ain}-\textit{rə}\\
Tilgen von K (\textit{r}) & Petrifeld, Vorarlberg & \textit{\=ais}\textit{r}, \textit{\=ais}-\textit{ǝm} \textit{üsər}, \textit{üsə}\\
\raggedright Tilgen eines Segments (\textit{nə}) & Petrifeld & \textit{\=iərnə}, \textit{\=iər}-\textit{əm}\\
\lspbottomrule
\end{tabularx}
\end{table}

Im Dialekt von \textbf{Saulgau} wird im Pronomen der 1. und 2 Person Plural ein \textit{r} eingefügt, wenn ein vokalisch anlautendes \isi{Suffix} folgt. Die \isi{Wurzel} der 1. Person Plural ist \textit{əisə}, lautet das \isi{Suffix} vokalisch an, wird die \isi{Wurzel} durch ein \textit{r} erweitert, z.B. \textit{əisər}-\textit{e} (Plural) \citep[118]{Raichle1932}. Im \isi{Possessivpronomen} der 3. Person Singular Feminin und der 3. Person Plural hingegen wird der auslautende Vokal getilgt, wenn ein vokalisch anlautendes \isi{Suffix} folgt: z.B. \textit{iərə} (=\isi{Wurzel}), \textit{iər}-\textit{e} (Plural) \citep[119]{Raichle1932}. Für beide Fälle sind RRs nötig:

\ea%155
\label{ex:key:155}
 RR \textsubscript{B,} \textsubscript{\{\}}\textsubscript{,} \textsubscript{\textsc{det2[pron.poss}, \textsc{pers:1}} \textsubscript{\tiny $\veebar$}\textsubscript{\textsc{2}}\textsubscript{, \textsc{num:pl}]} ($\langle$X,$\sigma$ $\rangle$) = \textsubscript{def} $\langle$X\textit{r}/\_Vˊ,$\sigma$ $\rangle$
\z

\ea%156
\label{ex:key:156}
 RR \textsubscript{B,} \textsubscript{\{\}}\textsubscript, \textsubscript{\textsc{det2[pron.poss}, \textsc{pers:3}, \textsc{num:pl}]} ($\langle$X,$\sigma$ $\rangle$) = \textsubscript{def} $\langle$X *\textit{ə} $\rightarrow$ ø/\_Vˊ,$\sigma$ $\rangle$
\z

\ea%157
\label{ex:key:157}
 RR \textsubscript{B,} \textsubscript{\{\}}\textsubscript, \textsubscript{\textsc{det2[pron.poss}, \textsc{pers:3}, \textsc{num:sg}, \textsc{gend:f}]} ($\langle$X,$\sigma$ $\rangle$) = \textsubscript{def} $\langle$X *\textit{ə} $\rightarrow$ ø/\_Vˊ,$\sigma$ $\rangle$
\z

Das \isi{Suffix} löst also die Wurzelalternation aus. Deswegen sind die \isi{Suffixe} in \isi{Block} A, die RR für die Wurzelalternationen in \isi{Block} B. Dies gilt auch für alle folgenden Fälle, in denen der phonologische Kontext der Auslöser der Wurzelalternation ist.

In \textbf{Issime} lautet das \isi{Possessivpronomen} der 3. Person Plural \textit{üriu}. Folgt ein vokalisch anlautendes \isi{Suffix}, wird das auslautende \textit{u} getilgt, z.B. \textit{üri}-\textit{er} \citep[84]{Perinetto1981}:

\ea%158
\label{ex:key:158}
 RR \textsubscript{B,} \textsubscript{\{\}}\textsubscript{,} \textsubscript{\textsc{det2[pron.poss}, \textsc{pers:3}, \textsc{num:pl}]} ($\langle$X,$\sigma$ $\rangle$) = \textsubscript{def} $\langle$X *\textit{u} $\rightarrow$ ø/\_Vˊ,$\sigma$ $\rangle$ \\
\z

Ebenfalls im Dialekt von Issime enden die \isi{Possessivpronomen} der 1. und 2. Person Singular sowie der 3. Person Singular Maskulin und Neutrum auf ein \textit{n}. Folgt ihnen ein konsonantisch anlautendes \isi{Suffix}, wird das \textit{n} der \isi{Wurzel} getilgt, z.B. \textit{mein}, \textit{mei}-\textit{s} \citep[83]{Perinetto1981}:

\ea%159
\label{ex:key:159}
 RR \textsubscript{B,} \textsubscript{\{\}}\textsubscript{,} \textsubscript{\textsc{det2[pron.poss}, \textsc{pers:3}, \textsc{num:sg}, \textsc{gend: m}} \textsubscript{\tiny $\veebar$}\textsubscript{ \textsc{n}}\textsubscript{]} ($\langle$X,$\sigma$ $\rangle$) = \textsubscript{def} $\langle$X *\textit{n} $\rightarrow$ ø/\_Kˊ,\mbox{$\sigma$ $\rangle$}
\z

\ea%160
\label{ex:key:160}
 RR \textsubscript{B,} \textsubscript{\{\}}\textsubscript{,} \textsubscript{\textsc{det2[pron.poss}, \textsc{pers:1}} \textsubscript{\tiny $\veebar$}\textsubscript{\textsc{2}, \textsc{num:sg}]} ($\langle$X,$\sigma$ $\rangle$) = \textsubscript{def} $\langle$X *\textit{n} $\rightarrow$ ø/\_Kˊ,$\sigma$ $\rangle$
\z

Dieselbe \isi{Variation} kommt im Dialekt von \textbf{Zürich} vor. Auch sind die gleichen \isi{Possessivpronomen} betroffen, weshalb die RRs hier nicht gelistet werden.

Im Dialekt von \textbf{Huzenbach} ist im \isi{Possessivpronomen} der 1. und 2. Person Singular sowie der 3. Person Singular Maskulin und Neutrum der Diphthong der \isi{Wurzel} nasaliert. Diese Nasalierung geht verloren, wenn das \isi{Suffix} vokalisch anlautet, z.B. \textit{mãẽ}, \textit{mae}-\textit{ərə} \citep[104]{Baur1967}. Die RRs definieren, dass die nasalierten Vokale ihre Nasalierung verlieren, wenn ein Vokal folgt:

\ea%161
\label{ex:key:161}
 RR \textsubscript{B,} \textsubscript{\{\}}\textsubscript{,} \textsubscript{\textsc{det2[pron.poss}, \textsc{pers:1}} \textsubscript{\tiny $\veebar$}\textsubscript{\textsc{2}, \textsc{num:sg}]} ($\langle$X,$\sigma$ $\rangle$) = \textsubscript{def} $\langle$X [+V, +nasal] $\rightarrow$ \mbox{[+V, -nasal]/\_Vˊ,$\sigma$ $\rangle$}
\z

\ea%162
\label{ex:key:162}
 RR \textsubscript{B,} \textsubscript{\{\}}\textsubscript{,} \textsubscript{\textsc{det2[pron.poss}, \textsc{pers:3}, \textsc{num:sg}, \textsc{gend: m}} \textsubscript{\tiny $\veebar$}\textsubscript{ \textsc{n}}\textsubscript{]} ($\langle$X,$\sigma$ $\rangle$) = \textsubscript{def} $\langle$X [+V, +nasal] $\rightarrow$ [+V, -nasal]/\_Vˊ,$\sigma$ $\rangle$
\z

Nun werden die  Wur\-zel-/Stamm\-al\-ter\-na\-tio\-nen besprochen, die von morphosyntaktischen Eigenschaften abhängig sind. Im Dialekt von Vorarlberg weist das \isi{Possessivpronomen} der 1. und 2. Person Plural im Nominativ/Akkusativ Singular Neutrum zwei Formen auf, nämlich \textit{üsə} und \textit{üsər} \citep[276]{Jutz1925}. Vergleicht man diese beiden Formen mit den Formen der übrigen Zellen, ist festzustellen, dass in allen anderen Zellen die Form \textit{üsər} als \isi{Wurzel} angenommen werden kann. Die Tilgung des \textit{r} ist also durch RRs auszudrücken. Gleichzeitig muss aber für die Zelle Nominativ/Akkusativ Singular Neutrum die Form \textit{üsər} durch eine gleich spezifische RRs definiert werden, weil nur so beide Formen in derselben Zelle stehen. Deswegen stehen auch beide Zellen in \isi{Block} A:

\ea%163
\label{ex:key:163}
 RR \textsubscript{A,} \textsubscript{\{\textsc{case:nom}} \textsubscript{\tiny $\veebar$}\textsubscript{ \textsc{acc}, \textsc{num:sg}, \textsc{gend:n}\},} \textsubscript{\textsc{det2[pron.poss}, \textsc{pers:1}} \textsubscript{\tiny $\veebar$}\textsubscript{ 2, \textsc{num:pl}]} ($\langle$X,$\sigma$ $\rangle$) = \textsubscript{def} $\langle$Xˊ,$\sigma$ $\rangle$
\z

\ea%164
\label{ex:key:164}
 RR \textsubscript{A,} \textsubscript{\{\textsc{case:nom}} \textsubscript{\tiny $\veebar$}\textsubscript{ \textsc{acc}, \textsc{num:sg}, \textsc{gend:n}\},} \textsubscript{\textsc{det2[pron.poss}, \textsc{pers:1}} \textsubscript{\tiny $\veebar$}\textsubscript{ 2, \textsc{num:pl}]} ($\langle$X,$\sigma$ $\rangle$) = \textsubscript{def} $\langle$X *\textit{r} $\rightarrow$ øˊ,$\sigma$ $\rangle$
\z

Auch im Dialekt von \textbf{Petrifeld} wird das auslautende \textit{r} der \isi{Wurzel} in einer bestimmten morphosyntaktischen Umgebung getilgt. Dies betrifft die \isi{Possessivpronomen} der 1. und 2. Person Plural wie auch der 3. Person Singular Feminin. Das \textit{r} wird im Dativ Singular Maskulin/Neutrum getilgt: \textit{\=aisr} (=\isi{Wurzel}), \textit{\=ais}-\textit{ǝm} (Dativ Singular Maskulin/Neutrum) \citep[65]{Moser1937}. Man könnte annehmen, dass das \textit{r} wegfällt, wenn ein Vokal folgt. Dies trifft aber nicht zu, was folgende Formen zeigen: \textit{\=aisr}-\textit{ǝ} (Dativ Singular Feminin), \textit{\=aisr}-\textit{e} (Plural) \citep[65]{Moser1937}. Die \isi{Variation} in der \isi{Wurzel} ist also vom morphosyntaktischen Kontext abhängig. In den genannten \isi{Possessivpronomen} sind folglich für den Dativ Singular Maskulin/Neutrum zwei RRs nötig: eine definiert das \isi{Suffix} -\textit{ǝm}, die andere tilgt \textit{r}. Diese RRs müssen in zwei verschiedenen \isi{Blöcken} stehen. Ständen sie im selben \isi{Block}, würden sie zwei Formen definieren (vgl. \sectref{5.6.3}). Die RR für das \isi{Suffix} ist in \isi{Block} B verortet, die RR für die Wur\-zel-/Stamm\-al\-ter\-na\-tion in \isi{Block} C (\isi{Block} A wird unten diskutiert). Die RRs für die Wur\-zel-/Stamm\-al\-ter\-na\-tion sehen wie folgt aus:

\ea%165
\label{ex:key:165}
 RR \textsubscript{C,} \textsubscript{\{\textsc{case:dat}, \textsc{num:sg}, \textsc{gend:m}} \textsubscript{\tiny $\veebar$}\textsubscript{ \textsc{n}\},} \textsubscript{\textsc{det2[pron.poss}, \textsc{pers:1}} \textsubscript{\tiny $\veebar$}\textsubscript{ 2, \textsc{num:pl}]} ($\langle$X,$\sigma$ $\rangle$) = \textsubscript{def} $\langle$X *\textit{r} $\rightarrow$ øˊ,$\sigma$ $\rangle$
\z

\ea%166
\label{ex:key:166}
 RR \textsubscript{C,} \textsubscript{\{\textsc{case:dat}, \textsc{num:sg}, \textsc{gend:m}} \textsubscript{\tiny $\veebar$}\textsubscript{ \textsc{n}\},} \textsubscript{\textsc{det2[pron.poss}, \textsc{pers:3}, \textsc{num:sg}, \textsc{gend:f}]} ($\langle$X,$\sigma$ $\rangle$) = \textsubscript{def} $\langle$X *\textit{r} $\rightarrow$ øˊ,$\sigma$ $\rangle$
\z

Ebenfalls im Dialekt von Petrifeld wird im \isi{Possessivpronomen} der 3. Person Plural ein ganzes Segment getilgt, wenn ein \isi{Suffix} folgt: \textit{\=iərnə} (=\isi{Wurzel}), \textit{\=iər}-\textit{əm} (Dativ Singular) \citep[65-66]{Moser1937}). Die Definition über die morphosyntaktischen Eigenschaften scheint hier also auf den ersten Blick nicht nötig zu sein. Da aber ausschließlich im Dativ Singular ein \isi{Suffix} auftaucht (alle anderen Zellen weisen keine \isi{Suffixe} auf), kann der Kontext am einfachsten über die morphosyntaktischen Eigenschaften beschrieben werden:

\ea%167
\label{ex:key:167}
 RR \textsubscript{C,} \textsubscript{\{\textsc{case:dat}, \textsc{num:sg}, GEND:\},} \textsubscript{\textsc{det2[pron.poss}, \textsc{pers:3}, \textsc{num:pl}]} ($\langle$X,$\sigma$ $\rangle$) = \textsubscript{def} $\langle$X *\textit{nə} $\rightarrow$ øˊ,$\sigma$ $\rangle$ \\
\z

Schließlich muss für Petrifeld noch ein n-Einschub stipuliert werden. Die \isi{Possessivpronomen} der 1. und 2. Person Singular sowie der 3. Person Singular Maskulin/Neutrum lauten vokalisch aus. Wird ihnen ein vokalisches \isi{Suffix} angehängt, wird per Default ein \textit{n} eingeschoben: \textit{m\=ai} (=\isi{Wurzel}), \textit{m\=ain}-\textit{e} (Plural) \citep[65]{Moser1937}. Im Dativ Singular Feminin wird jedoch ebenfalls ein \textit{n} eingeschoben, obwohl das \isi{Suffix} konsonantisch anlautet: \textit{m\=ain}-\textit{rə} \citep[65]{Moser1937}. Dies muss also durch eine RRs in \isi{Block} A definiert werden, die übrigen \isi{Suffixe} stehen in \isi{Block} B. Nur so kann gewährleistet werden, dass zuerst -\textit{n} und dann - \textit{rə} suffigiert wird (\textit{m\=ai}-\textit{n}-\textit{rə}):

\ea%168
\label{ex:key:168}
 RR \textsubscript{A,} \textsubscript{\{\textsc{case:dat}, \textsc{num:sg}, \textsc{gend:f}\},} \textsubscript{\textsc{det2[pron.poss}, \textsc{pers:3}, \textsc{num:sg}, \textsc{gend:m}} \textsubscript{\tiny $\veebar$}\textsubscript{ \textsc{n}]} ($\langle$X,$\sigma$$\rangle$) = \textsubscript{def} $\langle$X\textit{n}ˊ,$\sigma$$\rangle$
\z

\ea%169
\label{ex:key:169}
 RR \textsubscript{A,} \textsubscript{\{\textsc{case:dat}, \textsc{num:sg}, \textsc{gend:f}\},} \textsubscript{\textsc{det2[pron.poss}, \textsc{pers:1}} \textsubscript{\tiny $\veebar$}\textsubscript{ 2, \textsc{num:sg}]} ($\langle$X,$\sigma$$\rangle$) = \textsubscript{def} $\langle$X\textit{n}ˊ,$\sigma$$\rangle$
\z

\section{Synopse}\label{5.7}

In den vorangehenden Kapiteln wurden exemplarisch jene Phänomene sowie ihre Probleme und Lösungen vorgestellt, die in den Varietäten dieser Arbeit vorkommen. Diese Analyse soll hier kurz zusammengefasst werden, und zwar nach Phänomenen und nicht nach Wortarten. Daraus ergeben sich drei Themenblöcke: Zugehörigkeit eines Phänomens zur Morphologie oder zur Phonologie (\sectref{5.7.1}) bzw. zur Morphologie oder zur Syntax (\sectref{5.7.2}) und Typen morphologischer Probleme (\sectref{5.7.3}).

\subsection{Morphologie vs. Phonologie}\label{5.7.1}

Eine zentrale Frage ist, ob ein bestimmtes Phänomen zur Morphologie oder zur Phonologie gehört. Handelt es sich um einen Prozess, der im gesamten Sprachsystem gilt, also per Default ausgeführt wird, so gehört dieser zur Phonologie. Kommt dieser Prozess jedoch nur in einem morphologisch definierbaren Bereich vor, wird er zur Morphologie gezählt. Ein ausgezeichnetes Beispiel sind die wa-/w\=o{}-Stäm\-me bzw. deren Reste im Alt- und Mittelhochdeutschen sowie im Dialekt von Issime. Im Althochdeutschen ist die \isi{Variation} zwischen \textit{sn\=e}\textit{o} ‘Schnee’ und \textit{sn\=e}\textit{wes} phonologisch bedingt, weil es eine Regel gibt, die auslautendes \textit{w} immer zu \textit{o} vokalisiert (vgl. \sectref{5.1.3} \is{Subtraktion}Subtraktion). Im Gegensatz dazu wird im Mittelhochdeutschen auslautendes \textit{w} erstens getilgt und zweitens ist diese Tilgung synchron nicht vorauszusagen, weil auslautendes \textit{w} nicht immer getilgt wird (vgl. \sectref{5.1.3} \is{Subtraktion}Subtraktion). Diese Tilgung gehört also zur Morphologie und wird durch eine an bestimmte \isi{Flexionsklassen} gebundene RR definiert (vgl. \sectref{5.1.3} \is{Subtraktion}Subtraktion). Schließlich konnte für den Dialekt von Issime gezeigt werden, dass es sich bei \textit{w} um einen Pluralmarker handelt, der ebenfalls durch eine RR bestimmt wird (vgl. \sectref{5.1.3} \is{Subtraktion}Subtraktion).

Ein ähnliches Phänomen ist bezüglich des \textit{n} zu finden. In den alemannischen Dialekten wird \textit{n} eingeschoben, um einen \isi{Hiat} zu vermeiden, und zwar nicht nur innerhalb eines Wortes sondern auch zwischen Wörtern. Die Phonologie infigiert also automatisch ein \textit{n} zur Vermeidung eines \isi{Hiats}. Im Dialekt von Issime hingegen stellt \textit{n} auch eine Pluralmarker da, der folglich zur Morphologie gehört (vgl. \sectref{5.1.4}).

In der Folge sollen nun die wichtigsten Phänomene aufgezählt werden, die erstens aus synchroner Sicht phonologisch nicht erklärt werden können (sie gehören zur Morphologie) und zweitens die synchron phonologisch voraussagbar sind (sie gehören zur Phonologie).

\subsubsection{Synchron phonologisch nicht erklärbar}

\begin{description}
\item [\isi{Substantive}:] Dazu gehört das bereits erwähnte \textit{w} im Mittelhochdeutschen wie auch die althochdeutschen Diminutive, in denen das auslautende \textit{\=i} vor Diphthong getilgt wird (vgl. \sectref{5.1.3} \is{Subtraktion}Subtraktion). Außerdem zählen hierzu auch die Fälle in den alemannischen Dialekten, in denen der wurzelauslautende Vokal getilgt wird, wenn ein vokalisch anlautendes \isi{Suffix} folgt. Weil per Default die Phonologie den entstandenen \isi{Hiat} durch den Einschub von \textit{n} beheben würde, gehört die Tilgung des auslautenden Wurzelvokals zur Morphologie (vgl. \sectref{5.1.4}).
\item [\isi{Possessivpronomen}:] Bei den \isi{Possessivpronomen} werden ebenfalls an der \isi{Wurzel} Elemente suffigiert oder subtrahiert (meistens nachdem ein Flexionssuffix angehängt wurde), ohne dass diese phonologisch erklärbar sind (ausführlich diskutiert in \sectref{5.6.8}).
\item [Bestimmter und \isi{unbestimmter Artikel} (in vielen Dialekten):] Im Akkusativ und Dativ nimmt der Artikel eine andere Form an, je nachdem ob ihm eine Präposition vorausgeht oder nicht. Geht dem Artikel eine Präposition voraus, weist der Artikel eine reduzierte Form auf. Diese Reduktion kann aber nicht allgemein für das ganze System beschrieben werden, weshalb auch die reduzierte Form von der Morphologie definiert werden muss (vgl. \sectref{5.5.5} und \sectref{5.6.4}).
\item [Bestimmter Artikel vs. \isi{Demonstrativpronomen} (in allen Dialekten):] Beim bestim-\linebreak mten Artikel handelt es sich um eine reduzierte Form des \isi{Demonstrativpronomens}. Auch diese kann nicht durch einheitliche phonologische Regeln beschrieben werden. Sie muss also von RRs definiert werden (vgl. \sectref{5.5.3}).
\item [Betontes und unbetontes \isi{Personalpronomen} (in allen Dialekten):] Die Form des unbetonten \isi{Personalpronomens} ist ebenfalls eine reduzierte Form des betonten \isi{Personalpronomens}. Auch hier agieren die phonologischen Regeln synchron nicht mehr (vgl. \sectref{5.3.2}).
\end{description}


\subsubsection{Synchron phonologisch/phonotaktisch erklärbare Phänomene}

\begin{description}
\item[\isi{Substantive} der deutschen Standardsprache:] Wird -\textit{s} oder -\textit{n} suffigiert, können diese \isi{Suffixe} als -\textit{s}/-\textit{n} oder -\textit{əs}/-\textit{ən} auftreten (gilt in allen \isi{Kasus} und beiden \isi{Numeri}). Diese \isi{Variation} ist phonologisch und phonotaktisch bedingt, gilt also für das gesamte System (z.B. auch in der Verbflexion) (vgl. \sectref{5.1.1}).
\item[Mittelsilbensenkung in den \isi{Substantiven} (in vielen Dialekten):] In vielen Dialekten\linebreak wird das auslautende -\textit{i} zu -\textit{e} oder -\textit{ə} gesenkt, wenn es in den Inlaut tritt. Vollvokale in der Mittelsilbe sind in diesen Dialekten nicht möglich (vgl. \sectref{5.1.4}).
\item[Kürzung der Endungen des Typs -\textit{ən}\textit{ə} (\isi{Substantive}) (in drei Dialekten):] Durch die\linebreak Suffigierung von -\textit{ən}\textit{ə} folgen der betonten Wurzelsilben drei unbetonte Silben. Aus phonotaktischen Gründen wird eine unbetonte Silbe getilgt, weil Wörter nur auf einen Trochäus oder einen Daktylus auslauten dürfen (vgl. \sectref{5.1.4}).
\end{description}

\subsection{Morphologie vs. Syntax}\label{5.7.2}

Neben phonologischen sind auch syntaktische Phänomene von morphologisch-\linebreak en zu trennen. Es geht vor allem darum, dass die Morphologie Formen zur Verfügung stellt, deren Distribution jedoch von der Syntax geregelt wird. Beispielsweise definieren RRs stark und schwach flektierte \isi{Adjektive}. Wie diese jedoch im Syntagma verteilt sind, wird durch syntaktische Regeln bestimmt (vgl. \sectref{5.2.1}).

Ein weiteres Beispiel findet sich bei den Artikeln. Der \isi{bestimmte Artikel} weist unterschiedliche Formen auf, je nachdem, ob er vor einem \isi{Adjektiv} steht oder nicht, oder ob ihm eine Präposition vorausgeht oder nicht. Der \isi{unbestimmte Artikel} variiert nur in Abhängigkeit von der Präsenz und Absenz einer vorausgehenden Präposition. Zwar ist die Distribution syntaktisch bedingt; die Formen der Artikel müssen jedoch definiert werden, und zwar durch RRs (vgl. \sectref{5.5.5} und \sectref{5.6.4}).

\subsection{Morphologische Phänomene}\label{5.7.3}

In diesem Kapitel\largerpage[2] sollen noch einige morphologische Phänomene zusammengefasst werden, die vor allem \is{nicht-konkatenative Flexion}nicht-konkatenativ sind (\sectref{5.7.3.1} und \sectref{5.7.3.2}) oder dem 1-zu-1-Verhältnis zwischen Form und Funktion widersprechen (\sectref{5.7.3.3} und \sectref{5.7.3.4}). Des Weiteren soll die Definition der \isi{Flexionsklasse} resümiert werden, da diese von den meisten Definitionen abweicht (\sectref{5.7.3.5}). Schließlich werden die diachron neu entstandenen Kategorien aufgelistet (\sectref{5.7.3.6}).

\subsubsection{Wurzel + Realisierungsregeln vs. nur Realisierungsregeln}\label{5.7.3.1}

In den folgenden Wortarten kann eine \isi{Wurzel} von \isi{Affixen} getrennt werden: \isi{Substantive}, \isi{Adjektive} und \isi{Possessivpronomen}. Die \isi{Wurzel} stammt aus dem \isi{Radikon}, von der durch RRs flektierte Wörter abgeleitet werden können. In den übrigen Wortarten (\isi{Personalpronomen}, \isi{Interrogativpronomen}, \isi{Demonstrativpronomen}, bestimmter und \isi{unbestimmter Artikel}) ist eine Trennung von \isi{Wurzel} und \isi{Affixen} nicht möglich. Eine Ausnahme bildet der \isi{unbestimmte Artikel} im Mittelhochdeutschen und in der deutschen Standardsprache sowie in den Dialekten von Visperterminen und Issime. Kann ein flektiertes Wort nicht in eine \isi{Wurzel} und \isi{Affixen} dividiert werden, wird die gesamte Form durch RRs definiert (vgl.\chapref{5}, \sectref{5.3.1}, \sectref{5.4}, \sectref{5.5.1}, \sectref{5.6.1}).

\subsubsection{Nicht-konkatenative Morphologie}\label{5.7.3.2}

Hierzu gehören \is{Modifikation}Modifikationen und \is{Subtraktion}Subtraktionen an der \isi{Wurzel}, woraus neue Stämme entstehen. Innerhalb der in\-fe\-ren\-tiel\-len-re\-a\-li\-sie\-ren\-den Morphologie und anhand der RRs können \is{nicht-konkatenative Flexion}nicht-konkatenative Phänomene problemlos adäquat erfasst werden, was in den Abschnitten \sectref{4.1.2} und \sectref{4.1.3} erörtert wurde.

Zur \is{Modifikation}\textsc{Modifikation} in den \isi{Substantiven} gehören der \isi{Umlaut}, die Diphthongierung (nur Münstertal) und die Velarisierung (nur Elsass (Ebene)). Durch diese \is{Modifikation}Modifikationen werden Pluralstämme abgeleitet (vgl. \sectref{5.1.3}).

Auch durch \is{Subtraktion}\textsc{Subtraktionen} entstehen neue Stämme. Im mittelhochdeutschen \isi{Substantiv} und \isi{Adjektiv} wird wurzelauslautendes \textit{w} getilgt, wenn es in den Auslaut tritt. Im Dialekt von Münstertal fällt das \textit{t} der \isi{Wurzel} im Plural weg. Zudem wird in vielen Dialekten der auslautende Vokal der \isi{Wurzel} getilgt, wenn ein vokalisch anlautendes \isi{Suffix} folgt (z.B. \textit{heisle} ’Häuschen‘ (Sg.), \textit{heisl}-\textit{ə} (Pl.) \citep[98]{Baur1967}). Auch in den althochdeutschen Diminutiva fällt das auslautende -\textit{\=i} der \isi{Wurzel} weg, wenn das \isi{Suffix} mit einem Diphthong anlautet. Wie in \sectref{5.7.1} sowie genauer in den Abschnitten \sectref{5.1.3} und \sectref{5.2.2} erklärt wurde, handelt es sich dabei nicht um voraussagbare phonologische Prozesse. Weitere \is{Subtraktion}Subtraktionen gibt es in den \isi{Possessivpronomen} (vgl. \sectref{5.6.8}).

Bemerkenswert ist hier nicht nur die \is{Subtraktion}Subtraktion an sich, sondern auch die Tatsache, dass die Bedingung für die \is{Subtraktion}Subtraktion erst durch das \isi{Suffix} gegeben ist. Die RR muss also definieren, in welcher phonologischen Umgebung was subtrahiert wird. Des Weiteren steht diese \is{Subtraktion}Subtraktions-RR in jenem \isi{Block}, der auf den \isi{Block} der RR für das \isi{Suffix} folgt. Nur so ist gewährleistet, dass zuerst suffigiert wird, woraus die Bedingung für die \is{Subtraktion}Subtraktion entsteht (vgl. \sectref{5.1.3} und \sectref{4.1.3.2}). In den \isi{Possessivpronomen} sind außerdem auch Fälle zu beobachten, in denen die \isi{Wurzel} erweitert wird, wenn ein \isi{Suffix} angehängt wird (vgl. \sectref{5.6.8}).

\subsubsection{Synkretismen}\label{5.7.3.3}

Es wurde gezeigt, dass \isi{Synkretismen} im \isi{Numerus}, \isi{Kasus} sowie \isi{Genus} auftreten. Außerdem gibt es auch \isi{Synkretismen} zwischen den Wortarten einer Kategorie. Beispielsweise unterscheidet die deutsche Standardsprache die Form des \isi{bestimmten Artikels} nicht von der Form des einfachen \isi{Demonstrativpronomens}. \isi{Synkretismen} können von den RRs auf zwei verschiedene Arten erfasst werden. Fallen alle Features einer morphosyntaktischen Eigenschaft bzw. einer Kategorie zusammen (z.B. keine Kasusunterscheidung), bleibt diese morphosyntaktische Eigenschaft bzw. Kategorie unterspezifiziert. Fallen einige Features zusammen (z.B. Nominativ und Akkusativ), werden diese durch eine ausschließende Disjunktion erfasst (Zeichen ${\veebar}$) (vgl. z.B. \sectref{5.2.1}).

Dieser Typ von \isi{Synkretismus} macht also ein System einfacher, da weniger RRs benötigt werden. Ein zweiter Typ von \isi{Synkretismus} hingegen macht das System komplexer, da er nicht durch eine einzige RR definiert werden kann. Dies ist der Fall, wenn die Features von mehr als einer morphosyntaktischen Eigenschaft/Kategorie variieren. Beispielsweise muss in der starken Adjektivflexion der deutschen Standardsprache das \isi{Suffix} -\textit{ər} (Dativ und Genitiv Feminin Singular sowie Genitiv Plural) durch zwei RRs bestimmt werden, da \isi{Numerus}, \isi{Kasus} und \isi{Genus} variieren (ausführlich diskutiert in \sectref{4.1.3.3}).

\subsubsection{Freie Variation}\label{5.7.3.4}

Alle Formen einer Zelle des Paradigmas müssen definiert werden, d.h. also auch, wenn mehr als eine Form vorhanden ist. In den hier untersuchten Varietäten kommen zwei Typen an freien Varianten vor. Im ersten Typ sind beide Formen suffigiert, wobei die beiden RRs gleich spezifisch sein und im selben \isi{Block} stehen müssen. Nur so können zwei Formen für dieselbe Zelle bestimmt werden (vgl. \sectref{4.1.3.3}). Im zweiten Typ ist eine Form suffigiert, die andere Form besteht nur aus der \isi{Wurzel}. Hier sind ebenfalls zwei RRs nötig: Eine definiert das \isi{Suffix}, die andere definiert, dass mit der \isi{Wurzel} nichts passiert. Würde nur die suffigierte Form definiert werden, würde diese ein zusätzliches Einfügen der \isi{Wurzel} in dieselbe Zelle verhindern. Dass beide Varianten in derselben Zelle stehen, ist nur dann gewährleistet, wenn zwei gleichspezifische RRs desselben \isi{Blocks} für dieselbe Zelle Formen definieren (vgl. \sectref{4.1.3.3}, Beispiel u.a. \sectref{5.1.2}).

\subsubsection{Zugehörigkeit Flexionsklassen}\label{5.7.3.5}

In den Abschnitten \sectref{5.1.1} und \sectref{5.1.4} wurde gezeigt, was in dieser Arbeit unter \isi{Flexionsklasse} verstanden wird. Die wichtigsten Punkte sollen hier noch einmal zusammengefasst werden:

\begin{itemize}
\item 
\isi{Flexionsklassen} sind eine Art Instruktion, wie die RRs miteinander kombiniert werden, d.h., sie zeigen die Anzahl Kombinationen an RRs.
\item 
Folglich werden nur jene Kategorien unterschieden, die auch durch die RRs unterschieden werden. Z.B. wird in der Substantivflexion im Dialekt des Kaiserstuhls kein \isi{Kasus} markiert.
\item 
Zwei \isi{Flexionsklassen} unterscheiden sich in mindestens einer RR. Sie werden also weder nach Stämmen oder Deklinationstypen eingeteilt, noch gibt es Ober- und Unterklassen.
\item 
Eine \isi{Flexionsklasse} hat mindestens zwei Lexeme.
\end{itemize}

\subsubsection{Neue Kategorien}\label{5.7.3.6}

Vergleicht man das Alt- und Mittelhochdeutsche mit den modernen Varietäten, stellt man fest, dass neue Kategorien entstanden sind. Diese sollen in der Folge kurz zusammengefasst werden. Die Zusammenfassung ist nach Wortarten gegliedert: Artikel, \isi{Possessivpronomen} und \isi{Personalpronomen}.

Einen grammatikalisierten \textsc{Artikel} gibt es im Althochdeutschen nicht. Das Mittelhochdeutsche und die deutsche Standardsprache verfügen über einen bestimmten und einen \isi{unbestimmten Artikel}. Die Flexion der Artikel weist jedoch keine Unterschiede zur Flexion des Demonstrativ- bzw. \isi{Possessivpronomens} auf. Es ist also nur ein Satz an RRs für den \isi{bestimmten Artikel}/\isi{Demonstrativpronomen} und ein Satz für den \isi{unbestimmten Artikel}/\isi{Possessivpronomen} nötig. Für die Dialekte benötigt man jedoch für jede der vier Wortarten einen Satz an RRs. Schließlich wurde gezeigt, dass viele Dialekte im Akkusativ und/oder Dativ zwei Artikelvarianten pro Zelle haben, wobei diese syntaktisch distribuiert sind. Trotzdem sind für beide Varianten RRs anzunehmen, da nur RRs Formen definieren (vgl. \sectref{5.5.3} und \sectref{5.6.2}).

Die Flexion des \textsc{Possessivpronomens} im Mittelhochdeutschen und in der deutschen Standardsprache ist also identisch mit der Flexion des \isi{unbestimmten Artikels}. In den Dialekten sind zwei separate Paradigmen anzunehmen. Zusätzlich weisen die meisten Dialekte im \isi{Possessivpronomen} unterschiedliche Paradigmen auf, und zwar in Abhängigkeit davon, um welches \isi{Possessivpronomen} es sich handelt (vgl. \sectref{5.6.6}).

Es wurde auch gezeigt, dass alle Dialekte im \textsc{Personalpronomen} ein betontes und unbetontes Paradigma aufweisen (vgl. \sectref{5.3.2}). Des Weiteren unterscheiden einige Dialekte im Neutrum der 3. Person Singular im Akkusativ eine belebte und eine unbelebte Form (vgl. \sectref{5.3.3}). Schließlich hat der Dialekte von Issime im Plural einfache und zusammengesetzte Formen (vgl. \sectref{5.3.1}).
\documentclass[output=paper,colorlinks,citecolor=brown]{langscibook}
\ChapterDOI{10.5281/zenodo.15682194}
\author{Hristina Kukova\orcid{0000-0001-9938-5462}\affiliation{Department of Computational Linguistics, Institute for Bulgarian Language, Bulgarian Academy of Sciences}}

\title{Frame semantics and verbs of emotion}

\abstract{The intersection of lexical semantics and syntax has been an important area of linguistics for some time. Verbs as the core of the lexicon are key to exploring the interaction between syntax and semantics and to understanding the nature of the lexicon. The study focuses on verbs of emotion in the Bulgarian language and their frame semantics. An overview of theoretical and empirical observations forms the general aim of the study. Neutral, positive and negative verbs of emotion are discussed and the results are summarised. The analysis is based on the semantic and partly morphological information of the lexical units from the WordNet (\cite{Fellbaum1998}) as well as on the semantic and syntactic features with which the investigated emotion verbs are represented in the FrameNet (\cite{Baker1998}, \cite{Ruppenhofer2016}). Five semantic frames are documented, which were selected due to their high frequency and the wide variety of lexical units they are evoked by. The description includes grammatical features of the lexical units, semantic and syntactic restrictions that verbs impose on the frame elements, and the assignment of the frame elements to a WordNet noun synset or subtree that reflects the realisation of the frame elements in context. The status of the frame elements, which is essential for the realisation of a lexical unit, is retrieved from FrameNet.}




\IfFileExists{../localcommands.tex}{
  \addbibresource{../localbibliography.bib} 
   % add all extra packages you need to load to this file

\usepackage{tabularx,multicol}
\usepackage{url}
\urlstyle{same}

\usepackage{listings}
\lstset{basicstyle=\ttfamily,tabsize=2,breaklines=true}

\usepackage{langsci-basic}
\usepackage{langsci-optional}
\usepackage{langsci-lgr}
\usepackage{langsci-osl}
% \usepackage{./langsci/styles/langsci-lgr}
% \usepackage{./langsci/styles/langsci-osl}
% \usepackage{langsci-gb4e}

\usepackage{tikz}
\usetikzlibrary{patterns,calc}
\pgfdeclarepatternformonly{south east lines}{\pgfqpoint{-0pt}{-0pt}}{\pgfqpoint{3pt}{3pt}}{\pgfqpoint{3pt}{3pt}}{
    \pgfsetlinewidth{0.6pt}
    \pgfpathmoveto{\pgfqpoint{0pt}{3pt}}
    \pgfpathlineto{\pgfqpoint{3pt}{0pt}}
    \pgfpathmoveto{\pgfqpoint{.2pt}{-.2pt}}
    \pgfpathlineto{\pgfqpoint{-.2pt}{.2pt}}
    \pgfpathmoveto{\pgfqpoint{3.2pt}{2.8pt}}
    \pgfpathlineto{\pgfqpoint{2.8pt}{3.2pt}}
    \pgfusepath{stroke}}
    
\usepackage{stmaryrd}
\usepackage{wasysym}
\usepackage{multirow}
\usepackage{caption}
\usepackage{subcaption}
\usepackage{mathrsfs}
\usepackage{qtree}

\usepackage{linguex}


   %pminos do not split footnotes
% \interfootnotelinepenalty=10000 %Footnote in Laporte chapters has to be split SN


%\DeclareIndexNameFormat{default}{%
%\nameparts{#1}%
%\usebibmacro{index:name}%
%{\index[names]}%
%{\namepartfamily}%
%{\namepartgiveni}%
% {}% L1
% {}% L2
%{\namepartprefix}% generates spurious space L3
%{\namepartsuffix}% generates spurious space L4
%}

%  {\DeclareIndexNameFormat{default}{%
%     \usebibmacro{index:name}{\index[names]}{#1}{#3}{#5}{#7}}}

%\DeclareIndexNameFormat{default}{%
%  \usebibmacro{index:name}{\sindex[nom]}{#1}{#3}{#5}{#7}}

%\DeclareIndexNameFormat{default}{%
%  \usebibmacro{index:name}{\sindex[person]}{#1}{#3}{#5}{#7}}
%\DeclareIndexNameFormat{default}{%
%\nameparts{#1} \usebibmacro{index:name}{\sindex[person]]}{\namepartfamily}{‌​\namepartgiven}{\nam‌​epartprefix}{\namepa‌​rtsuffix}}

%\newcommand{\smiley}{:)}

%\renewbibmacro*{index:name}[5]{%
%\usebibmacro{index:entry}{#1}%
%{\iffieldundef{usera}{}{\thefield{usera}\actualoperator}\mkbibindexname{#2}{#3}{#4}{#5}}}

% \newcommand{\noop}[1]{}

%remove for final
%\overfullrule=1mm

\newcommand{\tobi}[2]}}
\renewcommand{\S}[1]{\tobi{#1}{\textsc{*}}}

% this volume references
% puts: [this volume]
% already defined: \citetv
%\newcommand{\citepv}[1]{(\citeauthor{#1} \citeyear*{#1} [this volume])}
\newcommand{\citealtv}[1]{\citeauthor{#1} \citeyear*{#1} [this volume]}

%parentheses around example number
\newcommand{\pref}[1]{(\ref{#1})}

% in-text examples

\newcommand{\lnex}[1]{\textit{#1}} %target lang word
\newcommand{\lnlit}[1]{(lit.: `#1')} %literal reading
\newcommand{\lnlat}[1]{(#1)} % latinization
\newcommand{\lntrans}[1]{`#1'} %translation
\newcommand{\lnexl}[2]%
{\lnex{#1}{} \lnlat{#2}} % ex with latinization
\newcommand{\lnexlat}[3]{\lnex{#1}{} \lnlat{#2}{} \lntrans{#3}} % ex with latinization and tranl.

%ch01
\newcommand{\co}[1]{\mbox{\textbf{#1}}}

%ch09

\newcommand{\cyrbulg}[1]{\begin{otherlanguage*}{bulgarian}#1\end{otherlanguage*}}


%ch10
\newcommand{\nlp}{{\small NLP}}
\newcommand{\mwe}{{\small MWE}}
\newcommand{\rae}{{\small RAE}}
\newcommand{\lvc}{{\small LVC}}
\newcommand{\pos}{{\small P}o{\small S}}
%\newcommand{\todo}[1]{ \textcolor{red}{#1} }

%\renewcommand{\labelenumi}{\theenumi}
%\ainamefmt{{vv}{ll}{, ff}{, jj}} % fullname

\newcommand{\biberror}[1]{{\color{red}#1}}

\newcommand{\osenovaitem}{--~}
   %% hyphenation points for line breaks
%% Normally, automatic hyphenation in LaTeX is very good
%% If a word is mis-hyphenated, add it to this file
%%
%% add information to TeX file before \begin{document} with:
%% %% hyphenation points for line breaks
%% Normally, automatic hyphenation in LaTeX is very good
%% If a word is mis-hyphenated, add it to this file
%%
%% add information to TeX file before \begin{document} with:
%% \include{localhyphenation}
\hyphenation{
    Beck-man
    Ngu-yen
    back-chan-nel
    back-chan-nels
    mo-not-o-nous
    ste-reo-typ-i-cal
}

\hyphenation{
    Beck-man
    Ngu-yen
    back-chan-nel
    back-chan-nels
    mo-not-o-nous
    ste-reo-typ-i-cal
}

   \boolfalse{bookcompile}
   \togglepaper[23]%%chapternumber
}{}

\begin{document}
\maketitle

\section{Introduction} 
The aim of this study is to present the emotion verbs of the Bulgarian vocabulary. We apply the methodology of frame semantics to outline different constructions in which verbs of emotion are involved. We also use the BulNet semantic network to extract their characteristic meanings. Therefore, the verbs under investigation are presented in specific WordNet synsets containing lexical and morphological information. We then describe each predicate within the semantic frame it evokes, together with its frame elements (FEs) and their selectional restrictions, which are expressed in terms of specific WordNet synsets or subtrees. We assume that the lower levels (hyponyms) of the selected subtree can also occupy the FE position.

In this study, we adopt a usage-based approach and provide evidence for the importance of context in semantic analysis and frame profiling. The analysis of the corpus data contributes to the development of a theoretically and empirically coherent approach to describing the semantic and syntactic features of verb classes.

The main aims of this study are: (i) to systematise the main theoretical findings on emotion verbs; (ii) to analyse semantic frames and their frame elements; (iii) to demonstrate how syntactic realisations can be predicted by lexical semantics within a given verb class; (iv) to highlight the importance of the interaction between semantics (lexical-semantic properties) and syntax (syntactic behaviour).
 

We rely on \citegen{Levin1993} study, which categorises verbs of psychological state into four major subclasses based on both intuitive semantic grouping and participation in valency alternations. We consider the transitivity / intransitivity of the verbs \textit{желая} `wish' / \textit{страхувам се} `fear' and the possibility of taking the \fename{Experiencer} as a grammatical subject -- \textit{обичам} `love', or object -- \textit{харесва ми} `appeal to' in a sentence to further subdivide them. This division is reflected in a verb’s evoking
  \framename{Experiencer\_focused\_emotion} -- \textit{завиждам} `envy', or \framename{Stimulate\_emotion} and \framename{Cause\_to\_experience}  semantic frames -- \textit{изненадвам} `surprise', \textit{дразня} `annoy'. 
 

If we take \textit{съжалявам} `regret' as an example of a verb that evokes the \framename{Contrition} frame, we can see that in most cases the position of the \fename{Experiencer} is occupied in context  by the synsets eng-30-00007846-n: \{\textit{човек}\}  `person' or eng-30-07950920-n: \{\textit{социална група}\} `social group'. The \fename{Action} FE ``marks expressions that indicate a prior action that the \fename{Experiencer} has come to feel bad about'' and can be encoded both as a PP (with the prepositions \textit{за} `for' and \textit{заради} `because of') or as a clause (with the help of the conjunctions \textit{че} `that', \textit{задето} `for' and the interrogative pronouns \textit{как} `how', \textit{къде} `where', \textit{какво} `what', \textit{кой} `who', \textit{колко} `how much/many'. The prepositions \textit{за} `for' and \textit{заради}  `because of' in turn take an object, which can vary between the following synsets: eng-30-00029378-n: \{\textit{събитие}\} `event', eng-30-00030358-n: \{\textit{действие}\} `act',  eng-30-00037396-n: \{\textit{действие}\} `action', or eng-30-05770926-n:  \{\textit{умствена дейност}\} `thought process'. We take into consideration the possible selectional restrictions a verb imposes on its frame elements and group verbs further into subclasses.
 

The study is based on corpus data; unless otherwise stated, the examples are taken from the Bulgarian National Corpus \citep{Koeva2012}.

The rest of the paper is organised as follows.
Section \ref{ch6:sec:2} deals with the notion of conceptual frame and its preconditions. The resources used are explained in detail. Section \ref{ch6:sec:3} outlines previous studies and motivation. The focus is on the description of the class of verbs of emotion and the different criteria for categorisation. In the same section and throughout the paper, the differences in classification systems serve as a basis for distinguishing between subclasses of verbs of emotion. Section \ref{ch6:sec:4} gives an overview of the linguistic descriptions of the Bulgarian verbs of emotion and their special features. Section \ref{ch6:sec:5} deals with the semantic features of verbs of emotion in Bulgarian. It includes descriptions of different semantic frames and their frame elements. Section \ref{ch6:sec:6} summarises the results of this study and concludes the paper.

 

\section{Resources} \label{ch6:sec:2}


WordNet is a lexical-semantic network suitable for machine processing that was originally developed at Princeton University by a team led by George Miller (\cite{Miller1995}, \cite{Fellbaum1998}). The Bulgarian version of WordNet -- BulNet -- contains more than 100,000 synsets (\cite{koeva2021a}).

Although BulNet was used to represent the semantic and paradigmatic features of the predicates, the most important resource for their ``semantic and syntactic combination possibilities is FrameNet \citep[7]{Ruppenhofer2016}. FrameNet was launched in 1997 under the guidance of Charles Fillmore (\cite{Baker1998}) and is essential for both theoretical linguistic research and practical natural language processing.

Semantic frames represent the conceptual structure of an event or object and its participants.
Frame elements can be regarded as semantic roles. They can be core and non-core elements, the former being essential for the realisation of the respective semantic frame, while the latter are mostly descriptive (in terms of time, place, etc.). Lexical units are lemmas that describe a situation (frame). Each meaning of a word is encoded as a separate lexical unit and evokes a different semantic frame.

As Koeva and Doychev state, ``a Conceptual frame defines a unique set of syntagmatic relations between verb synsets representing the frame and noun synsets expressing the frame elements'' (\cite[203]{koeva-doychev-2022-ontology}). Based on the information we extract from the WordNet about a verb meaning and the syntactic and semantic restrictions it imposes on its FrameNet frame elements, we create a grid of possible combinations. All analysed verbs are considered separately in each sense, and their frame elements can be an NP, PP, AdvP, AccCl (obligatory accusative clitic), DatCl (obligatory dative clitic) or a clause element (S or small clause). We use the web-based system BulFrame to create and visualise conceptual frames \citep{koeva-doychev-2022-ontology}.
 
In order to provide an exhaustive description of the Bulgarian verbal lexical units, the following information was used:

\begin{itemize}
\item[(a)] From FrameNet: core and non-core frame elements, their semantic types, the sets of verbal lexical units associated with a given semantic frame and the Inheritance relation between frames.

\item[(b)] From WordNet: hypernym-hyponym relations, which organise synsets for nouns and verbs in hypernym trees, and the semantic classes to which these synsets belong.
 \end{itemize}

The web-based system BulFrame, developed at the Department of Computational Linguistics of the Bulgarian Academy of Sciences, is used to create, edit, view and review the conceptual frames \citep{koeva-doychev-2022-ontology}.

Most of the language material was taken from the Bulgarian National Corpus (\cite{Koeva2012}), which was created at the Institute for Bulgarian Language ``Prof. Lyubomir Andreychin''. The Bulgarian National Corpus consists of a monolingual part containing 240,000 texts or 1.2 billion words and 47 parallel corpora.

\section{Previous studies and motivation} \label{ch6:sec:3}


\subsection{Methodology}


S. Koeva points out the need for a formal description of syntagmatic relations in WordNet (\cite{Fellbaum1998}, \cite{koeva2021a}) by introducing the notion of conceptual frame to define a set of verbs that have unique syntagmatic relations to nouns (\cite [182]{svetla2021towards}). Leseva et al. have also explored the possibility of integrating data from WordNet, FrameNet and VerbNet and proposed a system of semantic relations that reflects thematic relations between predicates and their potential arguments in the context of WordNet' (\cite{leseva2018integrating}). Our approach, which is based on frame semantics (\cite{Baker1998}, \cite{Koeva2010FN}), combines both the abstract syntactic level and the projection of semantic relations onto the corresponding frame elements.

Since frame analysis is very sensitive and error-prone, decision-making is delegated to human experts. To facilitate the process, we have chosen the following procedure (described in detail in the chapter \textit{Universality of semantic frames versus specificity of conceptual frames} in this volume).


\begin{description}
\item[Step 1:] We select the relevant verb meaning (literal) that evokes a particular fra-me from a set of synsets.
\item[Step 2:] We check whether all core frame elements of the frame are relevant for Bulgarian and/or whether additional frame elements should be included. We can either choose from the existing FEs where appropriate or insert a completely new one and give it a name.
\item[Step 3:] We define the possible selectional restrictions by (a) choosing from a list of noun synsets for NPs; (b) specifying the prepositions for PPs; (c) specifying the conjunctions that can introduce the dependent clauses.
\end{description}

Frame-semantic analysis with its flexibility and versatility can contribute to a number of NLP tasks and applications and to improving language understanding.
\largerpage[2]
 
\begin{itemize}
\item[(i)] Frame semantics provides a framework for semantic role labelling, i.e. identifying and labelling the different roles that entities play in a sentence. This process is crucial for tasks such as question answering, information extraction and text understanding.

\item[(ii)] Sentiment analysis. Frame semantics helps to better understand and analyse the emotions and attitudes expressed in a text. By capturing the semantic frames associated with the sentiment, sentiment analysis models can understand the implicit information in a more nuanced way.

\item[(iii)] Text classification. By considering semantic frames and their associated meanings, models can identify the implicit information in context and capture the intended meaning of a text, leading to more accurate and nuanced text classification.

\item[(iv)] Machine translation. Frame semantics helps to transfer meaning from one language to another by capturing the semantic frames and their semantic roles. This approach goes beyond word-to-word translation and ensures that the intended meaning of the source sentence is preserved in the target language, resulting in more accurate translations. In addition, metrics based on frame semantics, e.g. \citep{czulo2019designing}, have been proposed for machine translation evaluation, e.g. \citep{czulo2019designing}.

\item[(v)] Information retrieval and question answering. Frame semantics helps to improve search engine results and question-answering systems. By understanding the frames and semantic roles in queries and documents, these systems can retrieve more relevant information and provide accurate answers by matching semantic frames and roles.

\item[(vi)] Building knowledge graphs. Frame semantics is useful in building knowledge graphs by identifying the relationships between entities based on semantic frames and their FEs. It helps in organising and representing the structured knowledge from an unstructured text and contributes to tasks such as knowledge extraction and knowledge representation.
 \end{itemize}

Frame semantics also plays a crucial role in corpus research. It provides a framework for analysing and understanding the meaning and structure of texts within a given corpus. It can influence corpus research in various ways:

\begin{itemize}
\item[(vii)] Semantic analysis. By identifying and labelling semantic frames and their FEs, corpus studies can uncover patterns and relationships between entities, actions and events, leading to a deeper understanding of the underlying semantics within the corpus.

\item[(viii)] Semantic annotation. Frame semantics provides a systematic approach for annotating corpora with semantic information. Corpus studies can use frame-based annotation schemes to label frames and their FEs in texts, which enables more detailed analysis and in turn facilitates the development of machine learning models for various NLP tasks.

\item[(ix)] Comparative studies. Frame semantics enables comparative studies of different corpora or subsets within a corpus. Researchers can analyse variations in the use of frames in different genres, time periods or languages and find out how meaning and semantic structure differ in distant contexts. This helps to analyse linguistic and cultural differences, diachronic changes and genre-specific semantic patterns.

\item[(x)] Semantic similarity and clustering. By applying frame semantics to corpus studies, researchers can measure semantic similarity and cluster texts based on their frame-based representations. This facilitates tasks such as document clustering, topic modelling and information retrieval, where a better understanding of the semantic relationships between texts is essential.

\item[(xi)] Corpus-based lexical semantics. Frame semantics helps with corpus-based studies of lexical semantics. By analysing lexical items in the context of semantic frames and their FEs, corpus studies can uncover the nuances and contextual meanings associated with words, leading to the identification of polysemy, homonymy and semantic shifts within the corpus.

\item[(xii)] Corpus-based frame compilation. Corpus studies contribute to the compilation of frame databases or resources. By analysing large corpora, researchers can identify recurring semantic frames, frame-triggering lexical units and their roles, which serve as valuable data for building or extending frame resources.
 \end{itemize}

Overall, frame semantics provides a rich representation of the meaning and structure of language that enables NLP models to gain a deeper understanding of texts and perform a variety of tasks more effectively. It also provides a theoretical and practical basis for corpus studies, allowing researchers to delve deeper into the semantics of texts, compare different corpora, uncover patterns and improve our understanding of language structure and meaning within a given corpus.
 

\subsection{Verbs of emotion}

Emotions can be defined as experiences or states triggered off by a certain event, situation, action, other people, our thoughts, expectations and plans (\cite[155]{belaj2011cognitive}). In view of this phenomenon we attempt to relate the complexity of the syntax of emotions to the variety of their semantics as demonstrated in Section 1. 
 

In one recently published psychological encyclopedic manual (\cite[218]{Strickland2000}) emotions are defined as ``a reaction, both psychological and physical, subjectively experienced as strong feelings, many of which prepare the body for immediate action. In contrast to moods, which are generally longer lasting, emotions are transitory, with relatively well-defined beginnings and endings. They also have valence, meaning that they are either positive or negative. Subjectively, emotions are experienced as passive phenomena. Even though it is possible to exert a measure of control over one’s emotions, they are not initiated – they happen to people.''

 

As far as the linguistic field is concerned, there have been published a number of studies dealing with the description of emotion words, starting with  \citet{anna1971kocha}, \citet[57]{wierzbicka1972semantic}, \citet[142]{wierzbicka1980lingua}, \citet{wierzbicka1986human} and \citet{иорданская1970попытка}, \citet{iordanskaja1973tentative}, \citet{iordanskaja1986russian}. Wierzbicka was the first to observe that unlike other language groups, Slavic languages tend to use verbs to speak of emotions, which holds true for the Bulgarian language as well. Her early works include attempts to formalise emotions, defining emotion words in natural language and referring to typical situations that evoke particular emotional states. Both Wierzbicka and Iordanskaja put forward the concept of evaluation of the situation by X for the description of emotion words in linguistic semantics.  \citet{зализняк1983семантика}, \citet{зализняк1985функциональная} deals with what she calls ``predicates of internal state'', establishing the distinction of the assertion and presupposition in their definitions. \citet{Lakoff1987} and  \citet{kovecses1988language} pay attention to the uses of emotion expressions and metaphors in a given language, in order to describe a conceptual model of the corresponding emotion -- as it is perceived and expressed in actual speech.


Another widely disputed issue throughout the studies of verbs of emotion and specifically among Slavic authors is the verbs’ reflexivity or mediality. The most influential account of the Slavic verbs under discussion is offered by \citet{wierzbicka1988semantics}, \citet{wierzbicka1995everyday}. The author states that these verbs in Russian and Polish with -sja and \textit{-się} respectively are reflexive forms on the basis that they indicate ``emotions to which people `give themselves' almost voluntarily and which they outwardly express'' (\cite[253]{wierzbicka1988semantics}). As the author claims, expressing emotions by reflexive verbs implies that they are ``treated not as arising by themselves but by the speaker’s conscious thoughts about the event'' \citep[22]{wierzbicka1995everyday}. Moreover, she outlines the syntactic distinction between voluntary (with \fename{Experiencer} in nominative and the \textit{-sja} verb), involuntary (with dative \fename{Experiencer} and an adverbial predicative) and neutral emotion (with nominative \fename{Experiencer} and an adjectival predicative) (\cite[253--254]{wierzbicka1988semantics}). A. Bedkowska-Kopzcyk challenges Wierzbicka’s views and considers this type of verbs in Slovene middle voice verbs (\cite{bkedkowska2014verbs}).


As far as Bulgarian language is concerned, the particle \textit{се} can be involved in rather complex relations between words and constructions. It can represent both a word-forming and a morphological marker and can bear a passive (Example \ref{ch6:ex1a}), a medial (Example \ref{ch6:ex1b}) or a reflexive (Example \ref{ch6:ex1c}) meaning \citep[100--103]{tisheva2022положителни}. 

\begin{exe} 
\ex  \label{ch6:ex1} 
\begin{xlist}
\ex \label{ch6:ex1a} 
\gll
\textit{Пациент-ът} \textit{не} \textit{трябва} \textit{да} \textit{\uppercase{\textbf{се безпокои-Ø}}} \textit{(от никого)}. \\
patient-DEF.M not should to {REFL disturb-3.SG.PRS} {(by nobody)}\\
\glt `The patient should not be disturbed (by anybody).' 
\ex \label{ch6:ex1b} 
\gll
\textit{Пациент-ът} \textit{не} \textit{трябва} \textit{да} \textit{\uppercase{\textbf{се безпокои-Ø}}}. {(да изпитва безпокойство)} \\
patient-DEF not should to {REFL disturb-3.SG.PRS} {(to experience worry)}\\
\glt `The patient should not worry. (experience worries)'
\ex \label{ch6:ex1c} 
\gll
\textit{Син-ът} \textit{ми} \textit{вече} \textit{\uppercase{\textbf{се МИЕ-Ø}}} \textit{сам-Ø}. \\
son-DEF my already {REFL wash-3.SG.PRS} alone-M.SG\\
\glt `My son can already wash himself on his own.'
\end{xlist}
\end{exe}

 \citet[76]{tisheva2022syntactic} also address this polemical issue in their research on syntactic characteristics of emotion predicates\footnote{The authors explore both verbs and other constructions, based on adjectives, adverbs or nouns (predicatives).} in Bulgarian. According to the authors ``\textit{se} is a marker for middle voice construction and does not indicate reflexivity, it occupies the direct object position and those verbs could have only PP or a complement clause as their second argument.''



Since the current study focuses on the semantic and syntactic features of the verbs under discussion together with their possible complements as imposed by the verb sense, we will not deal with this particular aspect of the verb description. As in most cases the verbs used with and without reflexive  \textit{се} involve literals from different WordNet synsets, they will have different meanings and, respectively, heterogeneous frame elements’ restrictions. 


A large number of studies have been carried out on different language material in the last 20 years involving emotion verbs, their organisation in FrameNet and their semantic specifications. 



Taking emotion concepts as a basis, Ruppenhofer describes the evolution and the development of FrameNet analyses over time due to application-oriented goals. Taking into account different linguistic theories and approaches (dimensional, categorical, meaning-oriented, etc.), the author illustrates how fine-gra\-ined distinctions of lexical units lead to formulating new semantic frames or dividing one frame into two (\cite{ruppenhofer2018treatment}). The explanation of the steps and motivation underlying the conceptualisation and the development of the frame organisation holds a specific value for emotion frame descriptions and their detailed understanding. Thus, the \framename{Experiencer\_Subj} and \framename{Experiencer\_Obj} verbs were initially grouped by valence criteria whereas in the latest version the semantics of the verbs is also considered. 

\citet{subirats2003surprise} compare the Spanish lexical units with those of English in order to work out the similarities and differences in the lexicalisation patterns of the two languages. They use the annotation of Spanish verbs with the help of FrameNet frames to summarise the different syntactic realisations.
Since the Bulgarian grammatical and syntactic realisation has many more similarities with Spanish than with English, it was particularly useful to learn about their experience.\footnote{It shows closeness in agreement, the formation of questions, negation, the use of prepositions and, above all, word order.}


\citet{subirats2004spanish} presented the Spanish FrameNet and the web application that processes bilingual information and facilitates the comparison of the semantic structures of two lexicons.

\citet{ghazi2015detecting} make an attempt to automatically recognise the emotion \fename{Stimulus}. They assemble a dataset with manually labelled emotion stimuli and then apply sequential learning methods to a complementary dataset that does not contain labelled stimuli.

All of these studies form the basis for our research and have influenced the observations we will make in the central part of this chapter, in which we will examine the nature of emotion categorisation and the way it is formally reflected in grammatical and semantic structure, particularly in emotion-verb complement constraints.

\subsection{Classifications}

The typological description of emotion verbs has also proved interesting for various authors in different studies. In this section we give a brief overview of their approaches.
  
Based on emotion words in general, \citet{kovecses2003metaphor} proposes a division into expressive and descriptive emotion words, whereby he categorises emotionally charged comments and expressions of agreement and disagreement in the first group, while in the second group he categorises terms that denote a specific emotional experience. \citet [115]{tisheva2021наблюдения} also distinguishes between the lexical and grammatical means for the emotional attitude of the speaker/writer on the one hand and the naming of emotional states, relationships or evaluations on the other. In view of this subdivision, we will only deal with the descriptive emotion words in the following.

\citet[115]{tisheva2021наблюдения} claims that duration is a fundamental concept to draw the line between emotions and feelings. According to the author, ``emotions are spontaneous reactions to certain internal or external stimuli, while feelings are more permanent and enduring and always involve an evaluation of the object to which they are directed''.

Most linguistic classifications are based on the above-mentioned psychological aspects of emotions and divide them into positive and negative emotions depending on their basic tone. \citet{scherer2005emotions} recognises three characteristic features of emotions, namely: intensity, duration and the ability to evoke a reaction, and creates a typology of affective phenomena as presented below:
 
\begin{itemize}
\item[(a)] emotion: a relatively brief response to an external or internal \fename{Stimulus}  event, e.g. \textit{angry}, \textit{sad}, \textit{joyful}, \textit{fearful}, \textit{ashamed}, \textit{desperate}, 

\item[(b)] mood: a diffuse affect state characterised by low intensity but relatively long duration, often without apparent cause, e.g. \textit{cheerful}, \textit{gloomy}, \textit{depres-sed}, 

\item[(c)] interpersonal stance: affective stance taken toward another person in a specific interaction, e.g. \textit{distant}, \textit{warm}, \textit{supportive}, \textit{contemptuous},

\item[(d)] attitude: relatively enduring, effectively colored beliefs, preferences, and predispositions towards objects or persons, e.g. \textit{liking}, \textit{hating}, \textit{desiring}, 

\item[(e)] personality traits: emotionally laden, stable personality dispositions and behavior tendencies, typical for a person, e.g. \textit{nervous}, \textit{hostile}, \textit{jealous}, \textit{envious}. 
 \end{itemize}

\citet{ляшевская2011онтологические}, on the other hand, classify verbs of emotion on the basis of their semantic structure and the consistency of the verbal operational functors contained in each meaning. Thus, they categorise the verbs in question into five different groups: Event, Feeling, Attitude, State and Feature.

In the present study, we will not focus so much on the semantic differentiation, but rather on the syntactic realisation of the verbs and  the semantic specificity of their FEs, which plays a crucial role in the frame-semantic analysis.

In terms of their grammatical features, \citet{johnson1989language} refer to two types of emotion verbs (they also speak of emotion nouns and emotion adjectives): \textbf{emotional relations}, e.g. \textit{to love}, \textit{to fear}, and \textbf{causatives}, e.g. \textit{to annoy}, \textit{to frighten}. This observation is consistent with the two types frequently described in the linguistic literature. Syntactic structures in which the \fename{Experiencer} is the subject encode the emotional relation verb class, while the structures in which the \fename{Experiencer} is encoded as the grammatical object denote the causative verb class. The former are known across languages as \textbf{Subject-}\fename{Experiencer} verbs (SE), while the latter are known as \textbf{Object-}\fename{Experiencer} verbs (OE) \citep{dowty1991thematic,levin2005argument}. \citet{Fellbaum1999a} follows this line of linguistic description by saying that emotion predicates ``fall into two grammatically distinct classes: those whose subject is the animate \fename{Experiencer} and whose object (if any) is the \fename{Source} (\textit{fear, miss, adore, love, despise}); and those whose object is the animate \fename{Experiencer} and whose subject is the \fename{Source} (\textit{amuse, charm, encourage, anger})''.

The main subdivision in the Slavic languages follows the definition of the two groups of emotion verbs based on the syntactic expression of the \fename{Experiencer} as subject or direct or indirect object \citep [55]{Croft1993CaseMA}, \citep [21]{ovsjannikova2013encoding}, \citep [75]{tisheva2022syntactic}.

Based on these observations, three main subtypes are generally distinguished for Slavic languages: (i) SE verbs (Example \ref{ch6:ex2a}), (ii) OE verbs with the \fename{Experiencer} in the accusative case (Example \ref{ch6:ex2b}), and (iii) OE verbs with the \fename{Experiencer} in the dative case (Example \ref{ch6:ex2c}). This fact has been maintained by a number of Slavic linguists: for Russian – \citet{sonnenhauser2010event}, for Polish – \citet{bialy2005polish} and \citet{rozwadowska2007various}, for Bulgarian – \citet{slabakova1996bulgarian}, among others.


\begin{exe} 
\ex  \label{ch6:ex2} 
\begin{xlist}
\ex\label{ch6:ex2a} 
\gll \textit{Аз} \textit{наистина} \textit{\uppercase{\textbf{харесвам}}} \textit{вампир-и-те.} \\
I really like-1.SG.PRS vimpire-PL.DEF\\
\glt `I really  like  vampires.'
\ex\label{ch6:ex2b} 
\gll \textit{Тази} \textit{постоянна} \textit{светлина} \textit{почва} \textit{да} \textit{ме} \textit{{\textbf{ДРАЗНИ-Ø}}.} \\
this-F.SG constant-F.SG light start-3.SG.PRS to I-ACC annoy-3.SG.PRS\\
\glt `This constant light is starting to  annoy me.'
\ex\label{ch6:ex2c} 
\gll \textit{Мисля,} \textit{че} \textit{това} \textit{му} \textit{\uppercase{\textbf{харесва-Ø}}.} \\
think-1.SG.PRS that it he-DAT appeal-3.SG.PRS\\
\glt `I think he likes it.'
\end{xlist}
\end{exe}



Some authors also observe the possibility of forming diathetic verb pairs in which the \fename{Stimulus}-subject verb is transitive, while its counterpart \fename{Experien\-cer}-subject is an intransitive reflexive verb marked with a reflexive pronoun or the suffix \citep [121]{ovsjannikova2020instrumental}. \citet{koevaсистема} introduces the system of diatheses and alternations for Bulgarian.


\section{Bulgarian verbs of emotion} \label{ch6:sec:4}

The Bulgarian verbs of emotion, traditionally considered part of the larger psychological class of verbs, form an intriguing set. In her 2008 study, \citet [265]{ницолова2008проблематика} proposes to consider verbs such as \textit{обичам} `love', \textit{мразя} `hate', \textit{нена\-виждам} `detest' and others as ``mental predicates for emotional attitude''. \citet[62--63]{koeva2019complements} further subdivides them into predicates for emotional reaction or evaluation, (i) which are expressed by verbs \textit{харесвам} `like', \textit{съжалявам} `regret', \textit{радвам се} `be glad', \textit{страхувам се} `fear', \textit{тревожа се} `worry' or (ii) constructions like \textit{благодарен съм} `be grateful', \textit{яд ме е} `be mad', \textit{срам ме е} `be ashamed', \textit{тъжно ми е} `be sad'.

When looking at the argument structure of verbs and predicative expressions for emotions in the Bulgarian language, \citet{dineva2000} states that there are four types, namely: (i) one-argument constructions, realising only an \fename{Experiencer}, such as \textit{вълнувам се} `be excited', \textit{тъжно ми е} `be sad', \textit{страх ме е} `be afraid', \textit{спокоен съм} `be calm'; (ii) two-argument structures, expressing the \fename{Experiencer} and the \fename{Stimulus} with causative verbs, such as \textit{радвам} `rejoice', \textit{натъ\-жaвам} `sadden', \textit{ядосвам} `make angry', \textit{изненадвам} `surprise', or (iii) the \fename{Experiencer} and the Object with verbs for attitude such as \textit{обичам} `love', \textit{уважавам} `respect', \textit{харесвам} `like', \textit{ценя} `appreciate', \textit{обожавам} `adore' and (iv) verbs with three arguments -- an \fename{Experiencer} and alternating arguments, expressing the \fename{Stimulus} and the \fename{Object} \textit{Книгата ми харесва.} (I like the book.)  – \textit{Харесвам книга\-та.} ({The book appeals to me.})

\citet [102]{tisheva2022положителни} divides the verbs of emotion into two groups based on the semantic role of the subject in the sentence: subject\hyp Stimulus verbs and subject\hyp Experiencer verbs, which thus form conversive pairs (\textit{безпокоя – безпокоя се} `wor\-ry', \textit{радвам -- радвам се} `rejoice', \textit{обиждам – обиждам се} `insult' and so on). The state verbs of emotion involved in these oppositions are reflexive in form and therefore intransitive.  As a rule, the emotion state verbs with \textit{се} take the \fename{Stimulus} as PP, while the causatives encode the \fename{Stimulus} as NP. Tisheva notes that not all emotion verbs fall into these pairs. A number of authors state that verbs such as \textit{боя се} `be afraid', \textit{наслаждавам се} `enjoy', \textit{страхувам се} `fear' are not used without the reflexive \textit{се}, while \textit{тъжа} `grieve', \textit{тъгувам} `sorrow', \textit{тържествувам} `triumph' do not have a counterpart with \textit{се} \citep [24]{коева1996класификация}, \citep [232]{Nitsolova2008}, \citep[101--102]{tisheva2022положителни}. This is one of the reasons why the common semantic model comprising subject and object of emotion cannot be expressed with a universal structural equivalence drawn between the causatives and \textit{се}-verbs. 


\citet [394]{tisheva2022syntactic} note that these conversive pairs can represent one and the same situation and have two identical valences, although they are occupied by different actants. Causative predicates transfer the semantic role of the \fename{Experiencer} to the direct object, while stative predicates attribute it to the subject. According to the authors, state verbs of emotion in Bulgarian are considered primary predicates and causative predicates are considered semantically derived predicates, folowing the Van Valin and LaPolla's classification of predicates \citep{VanValinLaPolla1997}.

\citet [70]{Stamenov2021} categorises the Bulgarian verbs for internal psych experiences into 12 groups based on the semantic roles that each of them requires. In addition to verbs of emotion, Stamenov's structural classification also includes verbs of mentality and perception. Of the 12 types outlined by the author, we have singled out 7 that contain verbs of emotion (at least one) and are relevant for the purpose of our study:

\begin{itemize}
\item[(i)] Intransitive verbs whose lexical meaning expresses the inseparable unity of the \fename{Actor} and the \fename{Experiencer}: \textit{копнея} `crave', \textit{тъжа} `grieve';

\item[(ii)] Transitive verbs with and \fename{Experiencer} and \fename{Stimulus} or \fename{Object}: \textit{оби\-чам} `love', \textit{мразя} `hate', \textit{обожавам} `adore', \textit{харесвам} `like';

\item[(iii)] Verbs, expressing the semantic structure of a \fename{Stimulus} and a specific effect (on the \fename{Experiencer}) with a predicate of the CAUSE + DEVERBAL NOUN type: \textit{възторгвам} / \textit{възторгвам се} `enrapture' / `go into raptures' –\textit{ предизвиквам} / \textit{преживявам възторг} `cause / experience rapture', \textit{възхи\-щавам} / \textit{възхищавам се} `cause admiration' / `admire', \textit{вълнувам} / \textit{вълну\-вам се} `excite' / `experience excitement';

\item[(iv)] Verbs with the possibility of attaching clitics – \textit{домъчнява (ми)} `start to feel unhappy', \textit{причернява (ми)} `start to feel unwell' – \textit{Четенето ми доскуча\-ва} (\textit{Reading makes me bored}) / \textit{Доскучавам на Петьо с въпросите си} (\textit{I'm boring Petyo with my questions});

\item[(v)] Verbs for ambient inner state with obligatory accusative \fename{Experiencer}: \textit{дострашава ме} `start to feel scared', \textit{доядява ме} `start to feel angry'.

\item[(vi)] Verbs for inner psychological state with obligatory dative \fename{Experiencer}, which allow for a second indirect object \fename{Stimulus} / \fename{Theme}: \textit{дожалява ми} `start to feel pity', \textit{докривява ми} `start to feel sad';

\item[(vii)] Reflexiva tantum verbs, in which the verb action is directed back to its subject \fename{Experiencer}: \textit{любувам се} `revel', \textit{срамувам се} `feel ashamed', \textit{страху\-вам се} `fear';
\end{itemize}

The author also points out that the different meanings of the verbs in his classification can be categorised into different groups.

In \sectref{ch6:sec:5} we analyse five of these subclasses of verbs, taking into account their frequency and distribution in Bulgarian. First, we analyse the top hypernyms of the general emotion verbs \{\textit{изпитвам, чувствам}\} / \{\textit{feel, experience}\}. Then we analyse the general transitive verbs that express an emotion and an attitude, such as \textit{обичам} `love', \textit{мразя} `hate', \textit{харесвам} `like'. We also deal with causatives \textit{веселя} `rejoice', \textit{радвам} `gladden', \textit{плаша} `scare', which are subdivided on the basis of the entity that evokes the emotion -- either an \fename{Agent} or a \fename{Stimulus}. And finally, we introduce the stative and inchoative verbs, which are the causative verbs’ middle-voice counterparts formally expressed with the verb and the reflexive \textit{се} -- \textit{веселя се} `rejoice (oneself)', \textit{радвам се} `gladden (oneself)', \textit{плаша се} `scare (oneself)'. We used the WordNet definitions where it was necessary to distinguish between different senses.



\section{Frames and semantic features}\label{ch6:sec:5}

The \fename{Experiencer} and the \fename{Stimulus} are the two obligatory participants in an emotion event. The \fename{Experiencer} is at the centre of much research and is known to necessarily involve a sentient participant -- usually a human or an animate being. Much less attention has been paid to the behaviour and syntactic expression of the \fename{Stimulus}, as its multifaceted nature can hardly be specified. For the predicates under consideration in our work the \fename{Stimulus}  affects the \fename{Experiencer}, changing the emotions he or she experiences. This general scenario determines the emotion verbs and specifies both the possible syntactic structures within a sentence and the morphological and semantic restrictions imposed on the situation participants. 

In the present study, we will look at some of the most common verbs of emotion. The study is based on their semantic frame representation, which builds on the FrameNet and WordNet structures. For this reason, the different subclasses of emotion verbs will presented below with brief definitions taken from FrameNet when appropriate and slightly modified when not.
 
We will outline first the core frame elements within the \framename{Emotions} frame as most of the semantic frames under study inherit them by virtue of the relations between frames. The definition of the \framename{Emotions} non-lexical frame is that ``An \fename{Experiencer} has a particular emotional \fename{State}, which may be described in terms of a specific \fename{Stimulus} that provokes it, or a \fename{Topic} which categorises the kind of \fename{Stimulus}. Rather than expressing the \fename{Experiencer} directly, it may (metonymically) have in its place a particular \fename{Event} (with participants who are \fename{Experiencers} of the emotion) or an \fename{Expressor} (a body-part of gesture which would give an indication of the \fename{Experiencer}'s state to an external observer)''. Both the core and the non-core Emotion frame elements are presented in \tabref{tab:emotionfe}.
 

We will consider five main semantic frames that demonstrate the syntactic specificity of five subclasses of emotion verbs. First, we will analyse the top hy\-pernyms of the emotion verbs  {\textit{изпитвам}:1, \textit{чувствам}:1} ({\textit{feel}:8, \textit{experience}:3}), which are represented by the \framename{Feelings} frame  (\sectref{ch6:sec:feeling}). In \sectref{ch6:sec:expfocem} we will deal with the \framename{Experienced\_focused\_emotion} frame, which is comprised of transitive verbs, expressing attitude. Thirdly, we will explore the \framename{Cause\_to\_experience} and \framename{Stimulate\_emotion} frames, which represent causative verbs of emotion, having  an \fename{Agent} or a \fename{Stimulus} as a subject (\sectref{ch6:sec:stemcaex}). And finally, in \sectref{ch6:sec:emdir}, we will examine the \framename{Emotion\_directed} frame, which represents the stative and inchoative verbs, formed from the causatives from \sectref{ch6:sec:stemcaex} and the reflexive \textit{се}.

\begin{table}
\begin{subtable}{\textwidth}
\caption{Core frame elements}
\begin{tabularx}{\textwidth}{lQ}
\lsptoprule
 \fename{Event} & The occasion or happening in which \fename{Experiencers} in a certain emotional state participate. \\
 \fename{Experiencer}  & The person or sentient entity who experiences or feels the emotion. \\ 
 \fename{Expressor} & It marks expressions that indicate body part, gesture or other expression of the \fename{Experiencer}  that reflects his or her emotional state. \\
 \fename{State} & The abstract noun that describes a more lasting experience by the \fename{Experiencer}. \\
 \fename{Stimulus}  & The person, event, or state of affairs that evokes the emotional response in the \fename{Experiencer}. \\ 
 \fename{Topic} & The general area in which the emotion occurs. It indicates a range of possible \fename{Stimulus}. \\ \lspbottomrule
\end{tabularx}
\end{subtable}\medskip\\
\begin{subtable}{\textwidth}
\caption{Non-core frame elements}
\begin{tabularx}{\textwidth}{lQ}
\lsptoprule
 \fename{Degree} & The extent to which the \fename{Experiencer}'s emotion deviates from the norm for the emotion. \\
 \fename{Empathy-target} & The individual or individuals with which the \fename{Experiencer}  identifies emotionally and thus shares their emotional response. \\
 \fename{Reason}\slash \fename{Explanation} & The \fename{Explanation} is the explanation for why the \fename{Stimulus}  evokes a certain emotional response. \\
 \fename{Manner} & Any description of the way in which the \fename{Experiencer}  experiences the \fename{Stimulus}  which is not covered by more specific FEs. Manner may also describe a state of the \fename{Experiencer}  that affects the details of the emotional experience. \\
 \fename{Parameter} &  A domain in which the \fename{Experiencer}  experiences the \fename{Stimulus}. \\
\lspbottomrule
\end{tabularx}
\end{subtable}
\caption{The \framename{Emotion} frame elements.}
\label{tab:emotionfe}
\end{table}

\subsection{\framename{Feeling}}\label{ch6:sec:feeling}

\begin{description}[font=\normalfont]
\item[{Definition}:] In this frame an \fename{Experiencer} experiences an \fename{Emotion} or is in an \fename{Emotional\_state}. There can also be an \fename{Evaluation} of the internal experiential state.
\end{description}
The verbs \textit{чувствам} `feel'  and \textit{изпитвам} `experience' typically evoke this semantic frame. The only synset that contains those verbs and is marked by the semantic prime \textit{verb.emotion} is presented in Example \ref{ch6:ex3} and is illustrated in Example \ref{ch6:ex:04}:

\begin{exe}  
\ex  \label{ch6:ex3}
\begin{xlist}
\ex %a. 
BG \{\textit{изпитвам}; \textit{изпитам}; \textit{чувствам}; \textit{почувствам}; \textit{преживея}; \textit{прежи\-вявам}; \textit{осезавам}\} (`изживявам емоционално състояние или афект (по отношение на някого или нещо)’)
\ex %b. 
EN \{\textit{feel}; \textit{experience}\} (`undergo an emotional sensation or be in a particular state of mind’)
\end{xlist}
\end{exe}

\begin{exe}
\ex   \label{ch6:ex:04} 
\gll \textit{Започна-ли} \textit{да} \textit{ЧУВСТВА-Т} \textit{любов} \textit{към} \textit{венерианк-и-те} \textit{и} \textit{същ-а-та} \textit{загриженост}, \textit{както} \textit{и} \textit{към} \textit{себе си}. \\
Start-3.PL.PST.INFR to FEEL-3.PL.PRS love towards Venusian-PL-DEF.PL and same-F-DEF.F concern as also towards oneself \\
\glt `\textit{They started to feel love towards the Venusian women and the same concern they felt towards themselves}.' 

\end{exe}

 

The core frame elements are \fename{Emotion}, \fename{Emotional\_state}, \fename{Evaluation}, and \fename{Experiencer}. The observations on Bulgarian material show that the transitive verbs (\textit{чувствам} `feel', \textit{изпитвам} `experience' only take an \fename{Experiencer} and \fename{Emotion} FEs, whereas intransitive ones (\textit{чувствам се} `feel (oneself)') encode an \fename{Experiencer} and an \fename{Emotional\_state} or \fename{Evaluation} (in rare cases). 
 

\fename{\textbf{Emotion}} -- the \fename{Emotion} is the feeling that the \fename{Experiencer} experiences. There are a lot more examples with \textit{изпитвам} `experience' than with \textit{чувствам} `feel' with a direct object position filled with a hyponym of the \{emotion\} synset (Examples \ref{ch6:ex:05} and \ref{ch6:ex:06}). 

\begin{exe} 
\ex  \label{ch6:ex:05} 
  %  \settowidth \jamwidth{(bg)}  
\gll \textit{Често} \textit{\textbf{ИЗПИТВА-М}} [\textit{завист}]$_{\feinsub{Emot}}$ \textit{към} \textit{човешк-и-те} \textit{съществ-а}. \\
often feel-1.SG.PRS envy to human-PL-DEF.PL being-PL \\  %\jambox{(bg)}
\glt `I often feel envious of human beings.' 
\end{exe}

\begin{exe} 
\ex  \label{ch6:ex:06} 
   % \settowidth \jamwidth{(bg)}  
\gll \textit{Елейн} \textit{изобщо} \textit{не} \textit{\textbf{ИЗПИТВА-ШЕ}} [\textit{гордост}]$_{\feinsub{Emot}}$. \\ 
Eleyn {at all} not {feel-3.SG.IPFV} pride \\  %\jambox{(bg)}
\glt `Elaine felt no pride at all.' 
\end{exe}

\fename{\textbf{Experiencer}} -- the \fename{Experiencer} experiences the \fename{Emotion} or is in the \fename{Emotional\_state}. The position of the \fename{Experiencer} is generally occupied by a literal belonging to the eng-30-00007846-n: \{\textit{person}\} synset or its hyponyms. It is expressed in a sentence by an NP, which functions as a subject. It can also be used metaphorically with a part of the body, usually the \textit{heart}, which has the potential to serve as an expressor of one’s feelings (Example \ref{ch6:ex:07}).

\begin{exe} 
\ex  \label{ch6:ex:07} 
  %  \settowidth \jamwidth{(bg)} 
\gll [...] \textit{и} \textit{да} \emph{каж-е} \textit{онова}, \textit{кое-то} \textit{\textbf{ЧУВСТВА-Ø}}  [\textit{сърце-то} \textit{ѝ}]$_{\feinsub{Body\_part}}$.\\
{}[...] and to tell-3.SG.PRS that-N which-DEF.N feel-3.SG.PRS heart-DEF.N her \\ % \jambox{(bg)}
\glt `[...] and to say what her heart feels.' 
\end{exe}

The \textit{se} counterpart of \textit{чувствам} `feel' – \textit{чувствам се} `feel (oneself)' is reflexive concerning its form and, respectively, intransitive. It does not take a direct object and, therefore, does not encode an \fename{Emotion}. In order to realise its meaning, it needs the other core frame element -- \fename{Emotional\_state}.


\fename{\textbf{Emotional\_state}} -- the \fename{Emotional\_state} is the state the \fename{Experiencer} is in. The \fename{Emotional\_state} can be expressed by an adjective/participle as in Examples \ref{ch6:ex:08} and \ref{ch6:ex:09} describing the \fename{Experiencer} or by an adverb (Example \ref{ch6:ex:10}), indicating the manner in which the \fename{Experiencer} feels.

\begin{exe} 
\ex  \label{ch6:ex:08} 
%    \settowidth \jamwidth{(bg)} 
\gll \textit{\textbf{ЧУВСТВА-М СЕ}}  [\textit{свободен-Ø}]$_{\feinsub{Emos}}$. \\ 
feel-1.SG.PRS  free-SG.M \\ % \jambox{(bg)}
\glt `I feel free.' 
\end{exe}

\begin{exe} 
\ex  \label{ch6:ex:09} 
    \settowidth \jamwidth{(bg)} 
\gll \textit{В} \textit{момент-а} \textit{тя} \textit{\textbf{СЕ ЧУВСТВА-ШЕ}}  [\textit{предад-ен-а}]$_{\feinsub{Emos}}$.   \\ 
in {moment-DEF.M} she feel-3.SG.IPFV betray-PTCP-F \\ % \jambox{(bg)}
\glt `Right now she felt betrayed.' 
\end{exe}

\begin{exe} 
\ex  \label{ch6:ex:10} 
 %   \settowidth \jamwidth{(bg)} 
\gll \textit{Дали} \textit{ще} \textit{\textbf{СЕ ЧУВСТВА-Ш}}  [\textit{отвратително}]$_{\feinsub{Emos}}$, \textit{може}.   \\ 
whether will feel-2.SG.PRS disgustingly, maybe \\ % \jambox{(bg)}
\glt `Will you feel disgusted, maybe.' 
\end{exe}

\subsection{\framename{Experiencer\_focused\_emotion}}\label{ch6:sec:expfocem}

\begin{description}[font=\normalfont]
\item[Definition:] The words in this frame describe an \fename{Experiencer}'s emotions with respect to some \fename{Content}. Although the \fename{Content} may refer to an actual, current state of affairs, quite often it refers to a general situation which causes the emotion.
\end{description}

\framename{Experiencer\_focused\_emotion} is a semantic frame that encodes the \fename{Experiencer} as a subject and the \fename{Content} as a direct object and is well-represented in Bulgarian. This semantic frame encompasses verbs like \textit{харесвам} `like', \textit{обичам} `love', \textit{мразя} `hate', \textit{ненавиждам} `detest', \textit{обожавам} `adore', \textit{съжалявам} `feel sorry', \textit{презирам} `despise', among others. As \citet[117]{tisheva2021наблюдения} specifies ``unlike the usage of some mental predicates (\cite[264]{ницолова2008проблематика}), with the verb \textit{обичам} `love' the negation does not affect the choice of lexical elements that can occupy the syntactic positions, but only the interpretation of the meaning of the whole sentence''. This observation can be spread over the verbs comprising this subclass with the exception of \textit{ненавиждам} `detest', which bears negation within its structure and does not allow for a second negative element. 
 


The core frame elements within the \framename{Experiencer\_focused\_emotion} in FrameNet are the \fename{Experiencer}, the \fename{Content}, the \fename{Event} and the \fename{Topic}. For the sake of the description of Bulgarian verbs we will use a modified frame, taking into consideration only the \fename{Experiencer} and the \fename{Content} as the other two core frame elements are generally combined with other parts of speech. The \fename{Event} is generally expressed by noun phrases and the \fename{Topic}, which gives additional information, was not found in the Bulgarian corpus examples with the verbs under discussion. That is why the latter two core FE will not be discussed here. 
 

We will consider firstly the two main meanings of the verb \textit{обичам} `love' as reflected in the BulNet lexical-semantic resource (Examples \ref{ch6:ex:11} and \ref{ch6:ex:12}). They both encode the \fename{Content} as a direct object or as a \textit{да}-clause. We believe that the other verbs from the group follow the same syntactic constructions.

\begin{exe} 
\ex  \label{ch6:ex:11} 
\begin{xlist}
\ex %a. 
BG \{\textit{обичам}\} (`изпитвам силна привързаност и симпатии към ня\-кого или свързаност с и удоволствие от нещо')\\
\textit{Тя обича шефа си и работи усърдно за него}; \textit{Обичам френската кухня}.
\ex %b. 
EN \{\textit{love}\} (`have a great affection or liking for’)\\
\textit{She loves her boss and works hard for him}; \textit{I love French cuisine}.
\end{xlist}
\end{exe}\textit{}

\begin{exe} 
\ex  \label{ch6:ex:12} 
\begin{xlist}
\ex %a. 
BG \{\textit{обичам}\} (`харесвам много, изпитвам удоволствие от нещо’)\\ \textit{Обичам да готвя}. 
\ex %b. 
EN \{\textit{love}\} (`get pleasure from’)\\
\textit{I love cooking}.
\end{xlist}
\end{exe}

\fename{\textbf{Experiencer}} -- The \fename{Experiencer} experiences the emotion or other internal state. The \fename{Experiencer} FE position is generally filled with a subtree of \{{\textit{person}}\} (Example \ref{ch6:ex:13}), but can also be encoded as \{{\textit{animal}}\} and its hyponyms (Example \ref{ch6:ex:14}). 

\begin{exe} 
\ex  \label{ch6:ex:13} 
  %  \settowidth \jamwidth{(bg)} 
\gll [\textit{Победител-ят}]$_{\feinsub{Exp}}$ \textit{\textbf{ОБИЧА-ШЕ}}  \textit{да} \textit{вкарва} \textit{гол-ове}.   \\ 
winner-DEF.M love-3.SG.IPFV to score goal-PL \\  %\jambox{(bg)}
\glt `The winner loved to score goals.' 
\end{exe}

\begin{exe}  
\ex  \label{ch6:ex:14}
 %   \settowidth \jamwidth{(bg)} 
\gll [\textit{Куче-то}]$_{\feinsub{Exp}}$ \textit{\textbf{ОБИЧА-Ø}}  \textit{стариц-и-те}.   \\ 
dog-DEF.N love-3.SG.PRS {old woman-PL-DEF.PL} \\ 
%\jambox{(bg)}
\glt `The dog loves old women.' 
\end{exe}

The subject \fename{Experiencer} is usually expressed by a singular noun, as shown in Examples \ref{ch6:ex:13} and \ref{ch6:ex:14}. If the subject \fename{Experiencer} is in the plural, it denotes a specific group that acts as the collective subject of the emotion, as shown in the Example \ref{ch6:ex:15}. 

\begin{exe} 
\ex  \label{ch6:ex:15} 
 %   \settowidth \jamwidth{(bg)} 
\gll [\textit{Магьосниц-и-те}]$_{\feinsub{Exp}}$ \textit{\textbf{ОБОЖАВА-Т}}  \textit{неразкри-ти-те} \textit{тайн-и}.   \\ 
magician-PL-DEF.PL adore-3.PL.PRS undiscover-PTCP.PL-DEF.PL secret-PL \\ % \jambox{(bg)}
\glt `Magicians adore undiscovered secrets.' 
\end{exe}

In addition, the \fename{Experiencer}'s position is often occupied by \{\textit{душа}\} `soul' or \{\textit{сърце}\} `heart'  synsets together with expressive modifiers or quantifiers to reveal the point where the feeling is concentrated (Example \ref{ch6:ex:16}). 

\begin{exe}
\ex   \label{ch6:ex:16} 
%    \settowidth \jamwidth{(bg)} 
\gll [\textit{Душа-та} \textit{ми}]$_{\feinsub{Exp}}$ {\textit{до болка}} \textit{те} \textit{\textbf{ОБИЧА-Ø}}.   \\ 
soul-DEF.F my {to pain} {you-ACC} {love-3.SG.PRS}
\\ % \jambox{(bg)}
\glt `My soul loves you painfully.' 
\end{exe}

Examples with metonymic shifts of the type \textit{Barcelona loves partying} (meaning the people of Barcelona) actually show that a great number of lexical units can possibly occupy a certain position if they can express the same semantic role. Our aim is to outline the syntactic regularities and that is why such occasional examples lie beyond the scope of this study. 



\textbf{\fename{Content}} -- \fename{Content} is what the \fename{Experiencer}'s feelings or experiences are directed towards or based upon. The \fename{Content} differs from a \fename{Stimulus}  because the \fename{Content} is not construed as being directly responsible for causing the emotion. The \fename{Content} FE is commonly expressed by a noun of the \{\textit{person}\} or \{\textit{animal}\} subtrees, as is shown in Example \ref{ch6:ex:17}, but the position of this FE can generally be occupied by any \{\textit{entity}\} hyponym, alluding to a specific human being (e.g. one's voice, as shown in Examples \ref{ch6:ex:18} and \ref{ch6:ex:19}). 


\begin{exe} 
\ex  \label{ch6:ex:17} 
%    \settowidth \jamwidth{(bg)} 
\gll \textit{Искрено} {\textit{да \textbf{ОБИЧА-Ø}}} [\textit{родител-я}]$_{\feinsub{Cont}}$   [...]  \\ 
{sincerely} {to love-3.SG.PRS} {parent-DEF.M} [...]
\\ % \jambox{(bg)}
\glt `To sincerely love the parent [...]'

\ex   \label{ch6:ex:18} 
 %   \settowidth \jamwidth{(bg)} 
\gll  {\textit{\textbf{ОБИЧА-М}}} [\textit{звук-а} \textit{на} \textit{глас-а} \textit{ти}]$_{\feinsub{Cont}}$!  \\ 
{love-1.SG.PRS} {sound-DEF.M} {of} {voice-DEF.M} {your}
\\ % \jambox{(bg)}
\glt `I love the sound of your voice!'
  
\ex   \label{ch6:ex:19} 
%    \settowidth \jamwidth{(bg)} 
\gll \textit{Библиотекар-ят} {\textit{\textbf{ОБИЧА-ШЕ}}} [\textit{театър-а}]$_{\feinsub{Cont}}$.  \\ 
{librarian-DEF.M} {love-3.SG.IPFV} theatre-DEF.M
\\ % \jambox{(bg)}
\glt `The librarian loved theatre.'
\end{exe}

There are examples in the corpus where the \fename{Content} is also conveyed metaphorically as in Example \ref{ch6:ex:20}, where the glass actually symbolises the \fename{Experiencer}’s attitude towards drinking.


\begin{exe} 
\ex  \label{ch6:ex:20} 
 %   \settowidth \jamwidth{(bg)} 
\gll \textit{Чичо Тошко} {\textit{\textbf{ОБИЧА-ШЕ}}} [\textit{чашка-та}]$_{\feinsub{Cont}}$ [...] \\ 
{uncle Toshko} {love-3.SG.IPFV} {glass-DEF.F}  [...]
\\ % \jambox{(bg)}
\glt `Uncle Toshko loved to drink [...]'
\end{exe}

The \fename{Content} of the emotion can also be expressed in Bulgarian by a subordinate clause, introduced by the conjunction \textit{да} (Example \ref{ch6:ex:21}), interrogative pronoun \textit{как} (Example \ref{ch6:ex:22}) and relative pronoun \textit{когато} (Example \ref{ch6:ex:23}). When the object position is filled with a clause, there is no structural dependency between the arguments of the predicates in the main and the subordinate clause. 

\begin{exe} 
\ex  \label{ch6:ex:21} 
%    \settowidth \jamwidth{(bg)} 
\gll \textit{Повече} \textit{ѝ} \textit{\textbf{ХАРЕСВА-ШЕ}} [\textit{да прочете-Ø}]$_{\feinsub{Cont}}$ [...] \\ 
{More} {she-DAT} {LIKE-3.SG.IPFV} {to read-3.SG.PRS} [...]
\\ % \jambox{(bg)}
\glt `She liked more to read [...]'

\ex   \label{ch6:ex:22} 
 %   \settowidth \jamwidth{(bg)} 
\gll  \textit{И} \textit{\textbf{МРАЗЕ-ШЕ}} [\textit{как} \textit{я}
\textit{гледа-ш}]$_{\feinsub{Cont}}$ [...] \\ 
{and}  {hate-3.SG.IPFV} {how} {she-ACC} {look\_at}-2.SG.PRS [...]
\\  %\jambox{(bg)}
\glt `And she hated how you looked at her [...]'
 
\ex   \label{ch6:ex:23} 
 %   \settowidth \jamwidth{(bg)} 
\gll  \textit{Хора-та} \textit{\textbf{ОБИЧА-Т}} [\textit{когато} \textit{някой} \textit{се нужда-е} \textit{от} \textit{свобода}]$_{\feinsub{Cont}}$ [...] \\ 
{person-PL.DEF} {love-3.PL.PRS} when somebody need-3.SG.PRS of freedom  [...]
\\ % \jambox{(bg)}
\glt `People like it when someone needs freedom [...]'
\end{exe} 

This usage should be distinguished from the one where \textit{когато}-clause is used for conflicting circumstances as in Example \ref{ch6:ex:24}. The \fename{Content} position in this sentence is filled with a direct object accusative pronoun \textit{ме} `me'.

\begin{exe} 
\ex  \label{ch6:ex:24} 
 %   \settowidth \jamwidth{(bg)} 
\gll [...] \textit{как} {\textit{може-ш}} {да} [{\textit{ме}}]$_{\feinsub{Cont}}$ \textit{\textbf{ОБИЧА-Ш}}, \textit{когато} \textit{едва} \textit{снощи} \textit{се срещнахме}  [...] \\ 
{}[...] {how} {can-2.SG.PRS} {to} {I-ACC} {love-2.SG.PRS}, {when} only last\_night {REFL met-1.PL.PST} [...]
\\ % \jambox{(bg)}
\glt `[...] how can you love me when we met only last night [...]'
\end{exe} 

English verbs show similar usage to Example \ref{ch6:ex:23} when projecting the non-core frame element \fename{Circumstances} with the help of a finite  \textit{wh}-complement, which is typically preceded by a pronominal object (Example \ref{ch6:ex:25}).


\begin{exe}  
\ex  \label{ch6:ex:25}
\textit{I \textbf{HATE} it when you do that}.
\end{exe}

An adjunct \textit{когато}-clause is also used in Example \ref{ch6:ex:26}, as \textit{обичам} `love' takes a complement \textit{да}-clause,  which occupies the position of the \fename{Content}. The proximity or remoteness of a phrase/clause and the verb does not affect the logical structure of the sentence.

\begin{exe} 
\ex   \label{ch6:ex:26}
%    \settowidth \jamwidth{(bg)} 
\gll \textit{Не} {\textit{\textbf{ОБИЧА-М}}}, {\textit{когато}} {\textit{ѝ}}  {\textit{говор-я}} {\textit{колко}} {\textit{много}} {\textit{я}} {\textit{обича-м}}, [\textit{тя да мълчи-Ø}]$_{\feinsub{Cont}}$. \\ 
{not} {love-1.SG.PRS} {when} {she-DAT} {speak-1.SG.PRS} {how} {much} {she-ACC} {love-1.SG.PRS}, {she to be silent-3.SG.PRS}
\\% \jambox{(bg)}
\glt `I don't like her keeping silent when I tell her how much I love her.'
\end{exe} 

Finally, we are going to examine a more specific sense of the verb \textit{обичам} `love' as in Example \ref{ch6:ex:27}, as it demonstrates high frequency of usage. 

\begin{exe} 
\ex   \label{ch6:ex:27}
\begin{xlist}
\ex  \label{ch6:ex:27a} %a. 
BG \{\textit{обичам}\} (`влюбен съм, изпитвам любов към някого’) \\
\textit{Тя искрено обичаше съпруга си}. 
\ex  \label{ch6:ex:27b}%b. 
EN \{\textit{love}\} (`be enamored or in love with’)\\
\textit{She loves her husband deeply}.
\end{xlist}
\end{exe}

In this particular meaning of the verb, the positions of the\fename{Experiencer} and \fename{Content} are semantically restricted to the synset \{\textit{person}:1\} and its hyponyms. Furthermore, as \citet [124]{tisheva2021наблюдения} states, they should reflect a single individual, so that both FEs should be expressed by singular nouns. If expressed with a plural form, the FEs consider a collective image of a specific group. The use of the definite form in singular or plural usually indicates a generic use. The Example \ref{ch6:ex:28} comes from the work of Tisheva.
 
\begin{exe} 
\ex   \label{ch6:ex:28}
  %  \settowidth \jamwidth{(bg)} 
\gll \textit{Майка-та} {\textit{\textbf{ОБИЧА-Ø}}} {\textit{дец-а-та}} {\textit{си}}.\\ 
{mother-DEF.F} {love-3.SG.PRS} {child-PL-DEF.PL} {REFL-POSS}
\\  %\jambox{(bg)}
\glt `A mother loves her children.'
\end{exe}


\subsection{\framename{Stimulate\_emotion} and \framename{Cause\_to\_experience}}\label{ch6:sec:stemcaex}

We will consider these two semantic frames and the lexical units that evoke them in parallel, since they show great similarities in terms of sentence structure and situation participants and differ only with respect to one of the frame elements. Both frames denote two core frame elements that are expressed conventionally.

\framename{Stimulate\_emotion}’s definition is ``Some phenomenon (the \fename{Stimulus}) provokes a particular emotion in an \fename{Experiencer}.'' Its core frame elements are an \fename{Experiencer} and a \fename{Stimulus}, defined as follows:

\begin{description}[font=\normalfont]
\item[\textbf{\fename{Experiencer}}:] the \fename{Experiencer} reacts emotionally or psychologically to the \fename{Stimulus}.
\item[\textbf{\fename{Stimulus}}:] the \fename{Stimulus} is the event or entity which brings about the emotional or psychological state of the \fename{Experiencer}.
\end{description}

Within the \framename{Cause\_to\_experience} frame an \fename{Experiencer} and an \fename{Agent} can be pointed out as core frame elements and the definition of the frame is ``An \fename{Agent} intentionally seeks to bring about an internal mental or emotional state in the \fename{Experiencer}''. 

\begin{description}[font=\normalfont]
\item[\textbf{\fename{Agent}}:] the \fename{Agent} is an external argument of the target word and purposefully arouses an emotional state.
\item[\textbf{\fename{Experiencer}}:] the \fename{Experiencer} is the person the \fename{Agent} causes to have a particular emotional state.
\end{description}

The semantic and syntactic restrictions of the frame element \fename{Experiencer} are identical for both semantic frames. It is an \textit{animate being} (Example \ref{ch6:ex:29}), but most of the time the position is represented by a \{\textit{person}\} NP.
\begin{exe} 
\ex  \label{ch6:ex:29} 
 %   \settowidth \jamwidth{(bg)} 
\gll \textit{Изведнъж} {\textit{\textbf{СТРЯСКА-МЕ}}} [\textit{заек}]$_{\feinsub{Exp}}$ [...] \\ 
{suddenly} {startle-1.PL.PRS} {rabbit} [...]
\\%  \jambox{(bg)}
\glt `Suddenly, we startle a rabbit [...]'
\end{exe}

An interesting case are the examples with the explicit presence of the \fename{Stimulus} of the emotion and an unexpressed \fename{Experiencer} (Example \ref{ch6:ex:30}). In her study of the predicative construction \textit{it is known}, \citet [175]{ницолова2001значение} notes that ``the place of \fename{Experiencer} in the semantic structure is actually occupied by a variety of epistemic subjects. The set of epistemic subjects includes at least the speaker himself, who also wants to include the hearer''. This observation can also be applied to the unexpressed \fename{Experiencer} of the causative predicates of emotion: the object is present in the semantic structure of the predicate and represents a plurality of individuals.

\begin{exe} 
\ex  \label{ch6:ex:30} 
  %  \settowidth \jamwidth{(bg)} 
\gll [\textit{Москва}]$_{\feinsub{Age}}$  {\textit{\textbf{ПЛАШИ-Ø}}}, \textit{че} \textit{ще} \textit{разкрие-Ø} \textit{истина-та}  [...] \\ 
{Moscow} {threaten-3.SG.PRS} {that} {will} {reveal-3.SG.PRS} {truth-DEF.F} [...]
\\% \jambox{(bg)}
\glt `Moscow threatens to reveal the truth [...]'
\end{exe}

Both semantic frames can be evoked by verbs such as \textit{ужасявам} `terrify', \textit{пла\-ша} `scare', \textit{разстройвам} `upset', \textit{веселя} `rejoice', \textit{радвам} `gladden', \textit{успокоявам} `comfort', \textit{вълнувам} `excite', \textit{забавлявам} `entertain', \textit{стряскам} `startle', which are causative and, correspondingly, transitive. The position of the direct object is taken by the \fename{Experiencer}, while the subject can be either animate or inanimate. If the source of the emotion is animate, it receives an agent-like interpretation and refers to the frame \framename{Cause\_to\_experience}; however, if it is inanimate, it is projected as the \fename{Stimulus} of the emotion and belongs to the frame \framename{Stimulate\_emotion}.

Within the frame \framename{Cause\_to experience} the \fename{Agent} can only be presented with the synset {\textit{person}} or its hyponyms. The frame \framename{Stimulate\_emotion} can encode all \{\textit{entity}\} hyponyms in its subject position, with the exception of \textit{person}. In addition, the \fename{Stimulus} can also be encoded as a clause. As \citet [18]{коева2021към} notes, the complementiser in Bulgarian is represented by the conjunction or conjunction-like words such as \textit{че}, \textit{да}, \textit{как} and \textit{дето}. This applies to predicates of emotion whose complement clauses representing the \fename{Stimulus} are generally introduced by one of these complementisers.
Some of the verbs allow all types of clauses, while some verbs in the corpus show no use with some of them. \tabref{tab:distributioncausative} shows the distribution of possible conjunctions with the predicates as represented in the Bulgarian National Corpus.\footnote{We have documented the results of the corpus-based search, although the values do not always match our linguistic intuition.}

\begin{table}
    \begin{tabular}{l *4{c}} 
    \lsptoprule
         verbs& \textit{че}~`\textit{that}' & \textit{да}~`\textit{to}'&  \textit{как}~`\textit{how}' & \textit{дето}~`\textit{as/for/that}'{\footnote{According to the Dictionary of Bulgarian Language \textit{дето} is a conjunction formed by an adverb or a relative pronoun. It has a variety of functions in a sentence, that is why more than one possible translation is presented in the table.}}\\ 
         \midrule
        \textit{ужасявам} `\textit{terrify}'&  +&  +&  −& −\\
        \textit{плаша} `\textit{scare}'&  +&  +&  −& +\\
        \textit{разстройвам} `\textit{upset}'&  −&  −&  −& −\\
        \textit{веселя} `\textit{rejoice}'&  −&  −&  −& −\\
        \textit{радвам} `\textit{gladden}'&  +&  +&  +& +\\
        \textit{успокоявам} `\textit{comfort}'&  +&  +&  −& −\\
        \textit{вълнувам} `\textit{excite}'&  +&  −&  +& −\\
        \textit{забавлявам} `\textit{entertain}'&  +&  +&  −& −\\
        \textit{стряскам} `\textit{startle}'&  −&  −&  −& −\\
    \lspbottomrule
    \end{tabular}
    \caption{The distribution of causative verbs and possible complementisers.}
    \label{tab:distributioncausative}
\end{table}

The results from the corpus search show that \textit{веселя} `rejoice', \textit{разстройвам} `upset' and \textit{стряскам} `startle' can have only a NP in the subject position. \tabref{tab:distributioncausative} shows that \textit{радвам} `gladden' is the only verb that allows for all four conjunctions. \textit{Ужасявам} `terrify', \textit{успокоявам} `comfort' and \textit{забавлявам} `entertain' can have \textit{че}- and \textit{да}-clauses (Example \ref{ch6:ex:31}), but do not show usage with the other two complementisers. \textit{Плаша} `scare' can be used with \textit{че}-, \textit{да}- and \textit{детo}-constructions in the subject position, and \textit{вълнувам} `excite' – with \textit{чe} and \textit{как} complementisers (Example \ref{ch6:ex:32}).


\begin{exe}  
\ex  \label{ch6:ex:31}
  %  \settowidth \jamwidth{(bg)}
\gll \textit{Винаги} {\textit{го}} {\textit{\textbf{ЗАБАВЛЯВА-ШЕ}}}, [{\textit{че}} {\textit{араб-и-те}} {\textit{им}} {\textit{вярва-ха}}]$_{\feinsub{Stim}}$. \\ 
{always}  {he-ACC} {entertain-3.SG.IPFV} {that} {Arab-PL-DEF.PL} {they-DAT} {believe-3.PL.IPFV}
\\ % \jambox{(bg)}
\glt `It always entertained him that the Arabs believed them.'
  
\ex  \label{ch6:ex:32}
%    \settowidth \jamwidth{(bg)} 
\gll \textit{Изобщо} {\textit{не}} {\textit{ме}} {\textit{\textbf{ВЪЛНУВА-Ø}}} [{\textit{как}} {\textit{изглежда-Ø}}]$_{\feinsub{Stim}}$! \\
{at all} {not}  {I-ACC} {CARE-3.SG.PRS} {how} {look like-3.PL.PRS}
\\ % \jambox{(bg)}
\glt `I don't care at all what it looks like!'
\end{exe}

The \fename{Stimulus}  clause can also be introduced with the intensifying modifier \textit{колко} `how much / many' as in Example \ref{ch6:ex:33}. 

\begin{exe} 
\ex  \label{ch6:ex:33} 
 %   \settowidth \jamwidth{(bg)} 
\gll \textit{\textbf{УЖАСЯВА-Ø}} {\textit{ме}}  [{\textit{колко}} {\textit{е}} {\textit{сериозен-Ø}}]$_{\feinsub{Stim}}$. \\ 
{terrify-3.SG.PRS} {I-ACC} {how much} {be-3.SG.PRS} {serious-M.SG}
\\  %\jambox{(bg)}
\glt `It terrifies me how serious he is.'
\end{exe}

In addition to the subject clauses (Examples \ref{ch6:ex:31} and \ref{ch6:ex:33}), the frame element \fename{Stimulus} can also be introduced by a \textit{c}-PP (Examples \ref{ch6:ex:34}). In this case, the subordinate clause applies to the PP and not to the verb.

\begin{exe} 
\ex  \label{ch6:ex:34} 
  %  \settowidth \jamwidth{(bg)} 
\gll \textit{Неведнъж} \textit{я} {\textit{беше \textbf{СТРЯСКА-Л}}} [{\textit{с}} {\textit{това}}, {\textit{което}} {\textit{знае-ше}}]$_{\feinsub{Stim}}$. \\ 
{not\_once} {she-ACC} {startle-3.SG.PLUSQ} {with} {this} {which} {know-3.SG.IPFV}
\\  %\jambox{(bg)}
\glt `Not once had he startled her with what he knew.'
\end{exe}

\subsection{\framename{Emotion\_directed}}\label{ch6:sec:emdir}

The frame \framename{Emotion\_directed} includes stative and inchoative subject\hyp\fename{Experiencer} psych verbs, which are characterised by the reflexive-by-form \textit{се} and a middle-voice use. It comprises of verbs such as \textit{ужасявам се} `feel/become terrified', \textit{плаша се} `fear',\textit{ разстройвам се} `feel/become upset', \textit{веселя се} `rejoice', \textit{радвам се} `be glad', \textit{успокоявам се} `calm down', \textit{вълнувам се} `be excited', \textit{забав\-лявам се} `entertain', \textit{стряскам се} `be startled' and others. We consider the above verbs to be the \textit{се} counterparts of the verbs we analysed in \sectref{ch6:sec:stemcaex}.

 
\begin{description}[font=\normalfont]
\item[Definition:] this frame describes an \fename{Experiencer} who is feeling or experiencing a particular emotional response to a \fename{Stimulus} or about a \fename{Topic}. There can also be \fename{Circumstances} FE under which the response occurs or a \fename{Reason} why the \fename{Stimulus} evokes the particular response in the \fename{Experiencer}.
\end{description}

The core frame elements are \fename{Event}, \fename{Experiencer}, \fename{Expressor}, \fename{Reason}, \fename{State}, \fename{Stimulus}, \fename{Topic}. We will slightly modify this semantic frame for the Bulgarian verbs by excluding the FEs \fename{Event}, \fename{Expressor} and \fename{State}, as we do not describe adjectives or nouns that evoke the semantic frame.

\fename{\textbf{Experiencer}} -- The \fename{Experiencer} is the person or sentient entity that experiences or feels the emotions. 

We found no examples in the corpus with complement clauses in subject position. When a subject is explicitly present in the sentence, the syntactic realisations of \fename{Experiencer} consist mainly of noun phrases of the subtree \textit{person}. There are rare cases with animate non-persons, which in this case belong to the subtree \textit{animate being} (Example \ref{ch6:ex:35}). As we have already mentioned, Bulgarian as a pro-drop language allows the subject position to be empty.

\begin{exe} 
\ex  \label{ch6:ex:35} 
 %   \settowidth \jamwidth{(bg)} 
\gll \textit{Животн-и-те} {\textit{\textbf{СЕ ПЛАШЕ-ХА}}}, {\textit{но}} {\textit{назад}} {\textit{не}} {\textit{може-хме}} {\textit{да}} {\textit{се върне-м}}. \\ 
{animal-PL-DEF.PL} {scare-3.PL.IPFV} {but} {back} {not} {can-1.PL.IPFV} {to} {go back-1.PL.PRS}
\\ % \jambox{(bg)}
\glt `The animals were scared, but we could not go back.'
\end{exe}

Metonymic transfers make it possible for non-animate objects to take the position of the subject, although few cases illustrated that type in the corpus (Examples \ref{ch6:ex:36} and \ref{ch6:ex:37}).

\begin{exe} 
\ex  \label{ch6:ex:36} 
 %   \settowidth \jamwidth{(bg)} 
\gll \textit{Как} \textit{\textbf{СЕ ВЕСЕЛИ-Ø}} [{\textit{град-ът}}]$_{\feinsub{Exp}}$? \\ 
{how} {rejoice-3.SG.PRS} {city-DEF.M}
\\ % \jambox{(bg)}
\glt `How does the city have fun?'
\ex  \label{ch6:ex:37}  
   % \settowidth \jamwidth{(bg)} 
\gll [...] \textit{търсене-то} \textit{беше} \textit{прекрат-ен-о} \textit{и} [{\textit{село-то}}]$_{\feinsub{Exp}}$ \textit{\textbf{СЕ УСПОКОИ-Ø}}. \\ 
{}[...] {search-DEF} {was} {call off-PTCP-N} {and} {village-DEF.N} {calm down-3.SG.PST}
\\ % \jambox{(bg)}
\glt `[...] the search was called off and the village calmed down.'
\end{exe}

\fename{\textbf{Stimulus}} -- The \fename{Stimulus} is the person, event, or state of affairs (excluding \fename{Reason}) that evokes the emotional response in the \fename{Experiencer}. As the last example (Example \ref{ch6:ex:37}) shows, the \fename{Stimulus} of the emotion can be syntactically unexpressed. When this element of the emotional scenario is expressed, it is traditionally projected into a subordinate clause. 

As in \sectref{ch6:sec:stemcaex} we checked all the possible combinations of verbs and complementisers and present them in Table 3.\footnote{We have documented the results from the corpus-based search, although the values are not always consistent with our own intuition
.} The differences from causative verbs’ usage are encircled.

\begin{table}
    \begin{tabular}{ l *4{c} } 
    \lsptoprule
         verbs& \textit{че} &  \textit{да} &  \textit{как} & \textit{дето} \\
              & `\textit{that}' & `\textit{to}' & `\textit{how}' & `\textit{as/for/that}' \\ 
         \midrule
        \textit{ужасявам се} `\textit{feel terrified}'&  +&  +&  ⊕& −\\ 
        \textit{плаша се} `\textit{fear}'&  +&  +&  ⊕& ⊖\\ 
        \textit{разстройвам се} `\textit{feel upset}'&  −&  −&  −& −\\ 
        \textit{веселя се} `\textit{rejoice}'&  ⊕&  −&  −& −\\ 
        \textit{радвам се} `\textit{be glad}'&  +&  +&  +& +\\ 
        \textit{успокоявам се} `\textit{calm down}'&  +&  ⊖&  −& −\\ 
        \textit{вълнувам се} `\textit{be excited}'&  +&  ⊕&  +& −\\ 
        \textit{забавлявам се} `\textit{entertain}'&  +&  +&  −& −\\ 
        \textit{стряскам се} `\textit{be startled}'&  ⊕&  −&  −& −\\ 
      \lspbottomrule
    \end{tabular}
    \caption{The distribution of \textit{се}-verbs and possible complementisers.}
    \label{tab:distributionse}
\end{table}

The verbs \textit{ужасявам се} `feel/become terrified', \textit{плаша се} `fear', \textit{вълнувам се} `be excited' show examples with the first three complement types as marked in \tabref{tab:distributionse} (Example \ref{ch6:ex:38}). \textit{Разстройвам се} `feel/become upset' does not take subordinate clauses of any type. \textit{Веселя се} `rejoice', \textit{успокоявам се} `calm down' and \textit{стряскам се} `be startled' allow clause complements with \textit{че} only (Example \ref{ch6:ex:39}). The compatibility of \textit{радвам се} `be glad' with subordinate conjunctions is significantly wider -- it can be used with all four of them, according to our empirical material (Example \ref{ch6:ex:40}). And finally, \textit{забавлявам се} `entertain' can be used with the two most frequent conjunctions \textit{че} and \textit{да} (Example \ref{ch6:ex:41}), but not with the other two.

\begin{exe} 
\ex  \label{ch6:ex:38} 
  %  \settowidth \jamwidth{(bg)} 
\gll \textit{\textbf{УЖАСЯВА-Ø СЕ}} [{\textit{да}} {\textit{е}} {\textit{далеч}} {\textit{от}} {\textit{теб}}]$_{\feinsub{Stim}}$. \\ 
{be terrified-3.SG.PRS} {to} {be-3.SG.PRS} {far} {from} {you-ACC}
\\ % \jambox{(bg)}
\glt `(He) is terrified of being away from you.'

\ex \label{ch6:ex:39} 
   % \settowidth \jamwidth{(bg)} 
\gll [...] \textit{а} \textit{пи-хме} \textit{и} \textit{слага-хме} \textit{трапез-и} \textit{да} \textit{\textbf{СЕ ВЕСЕЛИ-М}}, [\textit{че} \textit{те} \textit{си отидо-ха}]$_{\feinsub{Stim}}$. \\ 
{}[...] {but} {drink-1.PL.PST} {and} {set-1.PL.PST} {table-PL} {to} {rejoice-1.PL.PRS} {that} {they} {go away-3.PL.PST}
\\  %\jambox{(bg)}
\glt `[...] but drank and set tables to rejoice that they had gone.'

\ex   \label{ch6:ex:40}
  %%%  \settowidth \jamwidth{(bg)} 
\gll [...] \textit{че} \textit{я} \textit{обича-Ø} \textit{много} \textit{и} \textit{\textbf{СЕ РАДВА-Ø}}  [\textit{дето} \textit{всичко} \textit{свърши-Ø}]$_{\feinsub{Stim}}$ [...] \\ 
{}[...] {that} {she-ACC} {love-3.SG.PRS} {much} {and} {be glad-3.SG.PRS} {that} {everything} {end-3.SG.PST} [...]
\\ % \jambox{(bg)}
\glt `[...] that he loves her very much and is glad that everything ended [...]'

\ex  \label{ch6:ex:41}  
 %   \settowidth \jamwidth{(bg)} 
\gll \textit{И} \textit{тя} \textit{\textbf{СЕ ЗАБАВЛЯВА-ШЕ}}  [\textit{да} \textit{ме} \textit{учи-Ø}]$_{\feinsub{Stim}}$. \\ 
{and} {she} {entertain-3.SG.IPFV} {to} {I-ACC} {teach-3.SG.PRS}
\\  %\jambox{(bg)}
\glt `And she had fun teaching me.'
\end{exe}

In the corpus, there are occasional cases in which the complement clause is introduced with the intensifier \textit{колко} `how much / many' (Example \ref{ch6:ex:42}).

\begin{exe}
\ex   \label{ch6:ex:42} 
  %  \settowidth \jamwidth{(bg)} 
\gll [...] \textit{и} \textit{\textbf{СЕ РАДВА-ХА}}  [\textit{колко} \textit{хубав-Ø} \textit{обещава-Ø} \textit{да бъде} \textit{ден-ят}]$_{\feinsub{Stim}}$. \\ 
{}[...] {and} {rejoice-3.PL.IPFV} {how\_much} {fine-M.SG} {promise-3.SG.PRS} {to be-3.SG.PRS} {day-DEF.M}
\\ % \jambox{(bg)}
\glt `[...] and rejoiced at how fine the day promised to be.'
\end{exe}


In addition to a subordinate clause, the \fename{Stimulus} can also be introduced by a dative clitic argument (Example \ref{ch6:ex:43}) or a \textit{на}-, \textit{за}-, \textit{от}- or \textit{с}-PP (Example \ref{ch6:ex:44}).

\begin{exe} 
\ex  \label{ch6:ex:43} 
 %   \settowidth \jamwidth{(bg)} 
\gll [...] \textit{да} [\textit{му}]$_{\feinsub{Stim}}$ \textit{\textbf{СЕ РАДВА-М}}   \textit{скришом}.\\ 
{}[...] {to} {he-DAT} {rejoice-1.SG.PRS} {secretly}.
\\ % \jambox{(bg)}
\glt `[...] and enjoyed him secretly.'

\ex \label{ch6:ex:44} 
 %   \settowidth \jamwidth{(bg)} 
\gll \textit{Обикновено} \textit{\textbf{СЕ УСПОКОЯВА-МЕ}} [\textit{с} \textit{известн-ия} \textit{факт}]$_{\feinsub{Stim}}$ [...].\\ 
{usually} {calm down-1.PL.PRS} {with} {known-DEF.M} {fact} [...].
\\ % \jambox{(bg)}
\glt `We usually calm down at the well-known fact [...].'
\end{exe}

Negative-emotion verbs --~ \textit{ужасявам се} `feel / become terrified', \textit{плаша се} `fear', \textit{pазстройвам се} `feel / become upset', tend to take an \textit{от}-PP, while positive ones prefer \textit{на}- or \textit{за}-PPs. 



The \textit{с}-PP appears a lot more frequently when denoting another individual or individuals, who the \fename{Experiencer} shares emotional response with. Within the frame structure it is marked as an \fename{Empathy\_target} and is a non-core frame element (Example \ref{ch6:ex:45}). 


\begin{exe} 
\ex   \label{ch6:ex:45}
 %   \settowidth \jamwidth{(bg)} 
\gll \textit{Върв-и} \textit{да} \textit{\textbf{СЕ ЗАБАВЛЯВА-Ш}} [\textit{с} \textit{Едуард}]$_{\feinsub{EmpT}}$.\\ 
{go-SG.IMP} {to} {entertain-2.SG.PRS} {with} {Edward}
\\ % \jambox{(bg)}
\glt `Go have fun with Edward.'
\end{exe}


Another possible syntactic construction within this semantic frame is that both the \fename{Stimulus} and the \fename{Reason} appear together in one sentence. In these cases, the \fename{Stimulus} is expressed by a PP and the \fename{Reason} by a complement clause. Koeva points out that in these cases an internal left dislocation is observed -- an argument from the subordinate clause can appear in object position with the main predicate. It can also be expressed explicitly in the subordinate clause and is coreferent with the object in the main clause. No such examples were found in the corpus, but there are some on the internet (Example \ref{ch6:ex:46}).

\begin{exe}  
\ex  \label{ch6:ex:46}
  %  \settowidth \jamwidth{(bg)} 
\gll \textit{\textbf{РАДВА-М СЕ}} [\textit{на} \textit{дец-а-та}]$_{\feinsub{Stim}}$ [\textit{че} \textit{ходя-т} \textit{на} \textit{училище} \textit{с} \textit{удоволствие}]$_{\feinsub{Reas}}$.\\ 
{be glad-1.SG.PRS}  {to} {child-PL-DEF.PL} {that} {go-3.PL.PRS} {to} {school} {with} {pleasure}
\\ % \jambox{(bg)}
\glt `I am happy for the children that they attend school with pleasure.'
\end{exe}


 


\section{Conclusions} \label{ch6:sec:6}

This study is devoted to the representation of the semantic and syntactic behaviour of verbs of emotion and their arguments. %The resources used were described in \sectref{ch6:sec:2}, where their main characteristics and structures were discussed. The methodology of the work was outlined in \sectref{ch6:sec:3}, followed by a description of the class of emotion verbs and different typological approaches dealing with them.

A number of the most common emotion verbs were selected for the study and their semantic frames were discussed. The main focus was on five semantic frames, namely \framename{Feeling}, \framename{Experiencer\_focused\_emotion}, 
\framename{Cause\_to\_experience},
\framename{Stimulate\_emotion} and \framename{Emotion\_directed}.

A smaller number of semantic frames (e.g. \framename{Worry}, \framename{Fear}, \framename{Emotion\_heat} and others), which comprise fewer lexical units, were not considered in our study and will be analysed in the future.

All semantic frames investigated were characterised in terms of the lexical units which evoke them; their core frame elements and the possible representations they may have in terms of their syntactic and semantic expression. The frame \framename{Feeling} was presented with its core frame elements \fename{Experiencer}, \fename{Emotion}, \fename{Emotional\_state} and \fename{Evaluation}. It was found that the transitive verbs encode an \fename{Emotion} as a direct object, while the intransitive \textit{чувствам се} `feel (oneself)' includes the \fename{Emotional\_state} or \fename{Evaluation} in the sentence. The \framename{Experiencer\_focused\_emotion} was slightly modified with regard to the description of the Bulgarian verbs and the \fename{Content} and \fename{Experiencer} were adopted as core frame elements. Various options for the encoding of \fename{Content} were presented. The semantic frames \framename{Stimulate\_emotion} and \framename{Cause\_to\_experience} had similar characteristics: they both contain causative verbs and have two core frame elements, one of which is the \fename{Experiencer}. The second core frame element is semantically expressed as \fename{Stimulus} in the first semantic frame and as \fename{Agent} in the second. The potential conjunctions, interrogative and relative pronouns that can introduce a frame element were searched for in the corpus, and the results were presented as examples and listed in a table for clarification. The frame \framename{Emotion\_directed} comprises the middle-voice equivalents of the stative and inchoative verbs evoking \framename{Stimulate\_emotion} and \framename{Cause\_to\_experience} frames.
 
Lexical units of the frame \framename{Feeling} are neutral with respect to the emotion they denote, and their complements express the positive or negative connotation. The verbs themselves carry the semantics of a positive or negative emotion within the other four semantic frames discussed above.

To summarise, each semantic frame consists of a collection of frame elements that represent the semantic components or roles associated with it. The role of each frame element within a particular semantic frame is crucial for the accurate representation of the semantic structure and frame conceptualisation. FEs help to capture the relations, roles and interactions between the different participants and components within a semantic frame. They provide a detailed representation of the conceptual content of a frame and enable a more precise and nuanced linguistic analysis and understanding. The semantic analysis of the frame elements of the frame \framename{Emotions} and its five analysed subframes enables a prediction of the arguments of a semantic frame with respect to the specified linguistic constraints. The corresponding facets of the scenario represented for each semantic frame are a set of possible values from an inverted tree or subtree of WordNet. Sorting possible semantic components of words into groups of common semantic type (hypernyms) is in contrast to analysing the semantic argument structure of sentences based on specific words.

This in-depth analysis and manual approach to assigning semantic and syntactic information to the core frame elements provides new insights and a deeper understanding of the syntactic behaviour of verbs and their environment. Although the manual review and selection is quite time-consuming, one of the strengths of the method is that it involves precise alignment of data from different resources which are quite asymmetric for automatic alignment.


 

 
\section*{Abbreviations}
\begin{multicols}{2}
\begin{tabbing}
MMMM \= Obligatory\kill
 AccCl \> Obligatory accusative clitic \\
  \textsc{Age} \> Agent \\
  \textsc{Cont} \> Content\\
 AdvP \> Adverbial phrase \\
 DatCl \> Obligatory dative clitic \\
\scshape Emos \> \fename{Emotional\_state}\\
\scshape Emot \> \fename{Emotion}\\
\scshape EmpT \> \fename{Empathy\_target}\\
\scshape Exp \> \fename{Experiencer}\\
 FE \> Frame element  \\
 NP \> Noun phrase \\
 PP \> Prepositional phrase \\
\scshape Reas \> \fename{Reason}\\
 S \> Subordinate clause \\
\scshape Stim \> \fename{Stimulus}\\
\end{tabbing}
\end{multicols}




\section*{Acknowledgements}

This research is carried out as part of the project \emph{Enriching Semantic Network WordNet with Conceptual Frames} funded by the Bulgarian National Science Fund, Grant Agreement No. KP-06-H50/1 from 2020.



{\sloppy\printbibliography[heading=subbibliography,notkeyword=this]}
\end{document}

\backmatter
\ohead{Bibliography}


%\bibliography{biblio}

\begin{thebibliography}{284}
\providecommand{\natexlab}[1]{#1}
\providecommand{\url}[1]{#1}
\providecommand{\urlprefix}{}
\expandafter\ifx\csname urlstyle\endcsname\relax
  \providecommand{\doi}[1]{doi:\discretionary{}{}{}#1}\else
  \providecommand{\doi}{doi:\discretionary{}{}{}\begingroup
  \urlstyle{rm}\Url}\fi

\bibitem[{Abercrombie(1967)}]{abercrombie1967elements}
Abercrombie, David. 1967.
\newblock \emph{Elements of general phonetics}.
\newblock Edinburgh: Edinburgh University Press.

\bibitem[{Adjarian(1899)}]{adjarian1899explosives}
Adjarian, Hrachia. 1899.
\newblock Les explosives de l'ancien arménien étudiées dans les dialectes
  modernes.
\newblock \emph{La Parole: Revue internationale de Rhinologie, Otologie,
  Laryngologie et Phonétique expérimentale} 119--127.

\bibitem[{{Albano Leoni}(2006)}]{albanoleoni2006statuto}
{Albano Leoni}, Federico. 2006.
\newblock Lo statuto del fonema.
\newblock In Stefano Gensini \& Martone Arturo (eds.), \emph{Il linguaggio:
  {T}eorie e storia delle teorie. {I}n onore di {L}ia {F}ormigari}, 281--303.
  Napoli: Liguori.

\bibitem[{Anderson et~al.(1984)Anderson, Pierrehumbert \&
  Liberman}]{anderson1984synthesis}
Anderson, Mark, Janet Pierrehumbert \& Mark Liberman. 1984.
\newblock Synthesis by rule of {E}nglish intonation patterns.
\newblock In \emph{Proceedings of 9th {I}nternational {C}onference of
  {A}coustics, {S}peech and {S}ignal {P}rocessing}, vol.~9, 77--80. San Diego.

\bibitem[{Andr\'{e} et~al.(2003)Andr\'{e}, Ghio, Cav\'{e} \&
  Teston}]{andre2003perceval}
Andr\'{e}, Carine, Alain Ghio, Christian Cav\'{e} \& Bernard Teston. 2003.
\newblock Perceval: {A} computer-driven system for experimentation on auditory
  and visual perception.
\newblock In Daniel Recasens, Maria~Josep Sol\'{e} \& Joaquín Romero (eds.),
  \emph{Proceedings of the 15th {I}nternational {C}ongress of {P}honetic
  {S}ciences}, 1421--1424. Barcelona.

\bibitem[{Angelini et~al.(1993)Angelini, Brugnara, Falavigna, Giuliani, Gretter
  \& Omologo}]{angelini1993baseline}
Angelini, Bianca, Fabio Brugnara, Daniele Falavigna, Diegl Giuliani, Roberto
  Gretter \& Maurizio Omologo. 1993.
\newblock A baseline of a speaker independent continuous speech recognizer of
  {I}talian.
\newblock In \emph{Proceedings of the 3rd {European Conference on Speech
  Communication and Technology}}, 847--850. Berlin.

\bibitem[{Arndt(1960)}]{arndt1960modal}
Arndt, Walter. 1960.
\newblock Modal particles in {R}ussian and {G}erman.
\newblock \emph{Word} 16. 323--336.

\bibitem[{Atterer \& Ladd(2004)}]{atterer2004phonetics}
Atterer, Michaela \& Robert Ladd. 2004.
\newblock On the phonetics and phonology of segmental anchoring of {F}0:
  {E}vidence from {G}erman.
\newblock \emph{Journal of Phonetics} 32. 177--197.

\bibitem[{{Audacity Development Team}(2006)}]{audacity2006audacity}
{Audacity Development Team}. 2006.
\newblock Audacity: {F}ree audio editor and recorder.
\newblock Computer program, retrieved from http://audacity.sourceforge.net/.

\bibitem[{Avesani(1990)}]{avesani1990contribution}
Avesani, Cinzia. 1990.
\newblock A contribution to the synthesis of {I}talian intonation.
\newblock In \emph{Proceedings of the 1st {International Conference on Spoken
  Language Processing}}, 833--836. Kobe.

\bibitem[{Balota(1994)}]{balota1994visual}
Balota, David. 1994.
\newblock Visual word recognition.
\newblock In Matthew Traxler \& Morton Gernsbacher (eds.), \emph{Handbook of
  psycholinguistics}, 334--357. San Diego: Academic Press.

\bibitem[{Batliner \& M{\"o}bius(2006)}]{batliner2005prosodic}
Batliner, Anton \& Batliner M{\"o}bius. 2006.
\newblock Prosodic models, automatic speech understanding, and speech
  synthesis: {T}owards the common ground?
\newblock In William Barry, Wim {van Dommelen} \& Jacques Koreman (eds.),
  \emph{The integration of phonetic knowledge in speech technology}, 21--44.
  Dordrecht: Kluwer.

\bibitem[{Batliner et~al.(2001)Batliner, M{\"o}bius, M{\"o}hler, Schweitzer \&
  N{\"o}th}]{batliner2001prosodic}
Batliner, Anton, Bernd M{\"o}bius, Gregor M{\"o}hler, Antje Schweitzer \& Elmar
  N{\"o}th. 2001.
\newblock Prosodic models, automatic speech understanding, and speech
  synthesis: {T}owards the common ground.
\newblock In Paul Dalsgaard, Børge Lindberg \& Henrik Benner (eds.),
  \emph{Proceedings of the 7th {European Conference on Speech Communication and
  Technology}}, vol.~4, 2285--2288. Aalborg.

\bibitem[{Beckman(1996)}]{beckman1996parsing}
Beckman, Mary. 1996.
\newblock The parsing of prosody.
\newblock \emph{Language and Cognitive Processes} 11(1-2). 17--68.

\bibitem[{Beckman(1997)}]{beckman1997typology}
Beckman, Mary. 1997.
\newblock A typology of spontaneous speech.
\newblock In Yoshinori Sagisaka, Nick Campbell \& Norio Higuchi (eds.),
  \emph{Computing prosody: {C}omputational models for processing spontaneous
  speech}, 7--26. Dordrecht, Heidelberg, London, New York: Springer.

\bibitem[{Bigi \& Hirst(2012)}]{bigi2012speech}
Bigi, Brigitte \& Daniel Hirst. 2012.
\newblock {SP}eech {P}honetization {A}lignment and {S}yllabification ({SPPAS}):
  {A} tool for the automatic analysis of speech prosody.
\newblock In Qiuwu Ma, Hongwei Ding \& Daniel Hirst (eds.), \emph{Proceedings
  of the 5th {I}nternational {C}onference on {S}peech {P}rosody}, vol.~1,
  19--22. Shanghai: Tongji University Press.

\bibitem[{Black \& Hunt(1996)}]{black1996generating}
Black, Alan \& Andrew Hunt. 1996.
\newblock Generating f0 contours from {ToBI} labels using linear regression.
\newblock In \emph{Proceedings of the 4th {International Conference on Spoken
  Language Processing}}, vol.~3, 1385--1388. Philadelphia.

\bibitem[{Blesser(1972)}]{blesser1972speech}
Blesser, Barry. 1972.
\newblock Speech perception under conditions of spectral transformation: {I}.
  {P}honetic characteristics.
\newblock \emph{Journal of Speech and Hearing Research} 15(1). 5--41.

\bibitem[{Boersma \& Weenink(2008)}]{boersma2008praat}
Boersma, Paul \& David Weenink. 2008.
\newblock Praat: {D}oing phonetics by computer.
\newblock Computer program, retrieved from http://www.praat.org/.

\bibitem[{Bolinger(1951)}]{bolinger1951intonation}
Bolinger, Dwight. 1951.
\newblock Intonation: {L}evels versus configurations.
\newblock \emph{Word} 7. 199--210.

\bibitem[{Bolinger(1964)}]{bolinger1964intonation}
Bolinger, Dwight. 1964.
\newblock Intonation as a universal.
\newblock In Horace Lunt (ed.), \emph{Proceedings of the 9th {I}nternational
  {C}ongress of {L}inguists}, 833--848. The Hague: Mouton.

\bibitem[{Bolinger(1989)}]{bolinger1989intonation}
Bolinger, Dwight. 1989.
\newblock \emph{Intonation and its uses: {M}elody in grammar and discourse}.
\newblock Palo Alto: Stanford University Press.

\bibitem[{Brooks(1978)}]{brooks1978nonanalytic}
Brooks, Lee. 1978.
\newblock Nonanalytic concept formation and memory for instances.
\newblock In Eleanor Rosch \& Barbara Lloyd (eds.), \emph{Cognition and
  categorization}, 170--211. Hillsdale: Erlbaum.

\enlargethispage{\baselineskip}
\bibitem[{Browman \& Goldstein(1986)}]{browman1986articulatory}
Browman, Catherine \& Louis Goldstein. 1986.
\newblock Towards an articulatory phonology.
\newblock \emph{Phonology Yearbook} 3. 219--252.

\bibitem[{Brunetti et~al.(2010)Brunetti, D'Imperio \&
  Cangemi}]{brunetti2010prosodic}
Brunetti, Lisa, Mariapaola D'Imperio \& Francesco Cangemi. 2010.
\newblock On the prosodic marking of contrast in {R}omance sentence topic:
  {E}vidence from {N}eapolitan {I}talian.
\newblock In \emph{Proceedings of the 5th {I}nternational {C}onference on
  {S}peech {P}rosody}, Chicago.

\bibitem[{Bruni(1992)}]{bruni1992italiano}
Bruni, Francesco. 1992.
\newblock \emph{L'italiano nelle regioni. {L}ingua nazionale e identità
  regionali}.
\newblock Torino: Utet.

\bibitem[{B\"{u}ring(1997)}]{buring1997meaning}
B\"{u}ring, Daniel. 1997.
\newblock \emph{The meaning of topic and focus: {T}he 59th street bridge
  accent}.
\newblock London, New York: Routledge.

\bibitem[{Bybee(2001)}]{bybee2001phonology}
Bybee, Joan. 2001.
\newblock \emph{Phonology and language use}.
\newblock Cambridge: Cambridge University Press.

\bibitem[{Bybee(2006)}]{bybee2006usage}
Bybee, Joan. 2006.
\newblock From usage to grammar: {T}he mind's response to repetition.
\newblock \emph{Language} 82(4). 711--733.

\bibitem[{Campbell \& Mokhtari(2003)}]{campbell2003voice}
Campbell, Nick \& Parham Mokhtari. 2003.
\newblock Voice quality: {T}he 4th prosodic dimension.
\newblock In Daniel Recasens, Maria~Josep Sol\'{e} \& Joaquín Romero (eds.),
  \emph{Proceedings of the 15th {I}nternational {C}ongress of {P}honetic
  {S}ciences}, 2417--2420. Barcelona.

\bibitem[{Cangemi(2009)}]{cangemi2009phonetic}
Cangemi, Francesco. 2009.
\newblock Phonetic detail in intonation contour dynamics.
\newblock In Stephan Schmid, Michael Schwarzenbach \& Dieter Studer (eds.),
  \emph{La dimensione temporale del parlato: {P}roceedings of the 5th
  {C}onference of {A}ssociazione {I}taliana di {S}cienze della {V}oce},
  325--334. Torriana: EDK.

\bibitem[{Cangemi et~al.(2011)Cangemi, Cutugno, Ludusan, Seppi \&
  Van~Compernolle}]{cangemi2011automatic}
Cangemi, Francesco, Francesco Cutugno, Bogdan Ludusan, Dino Seppi \& Dirk
  Van~Compernolle. 2011.
\newblock Automatic {S}peech {S}egmentation for {I}talian ({ASSI}): {T}ools,
  models, evaluation and application.
\newblock In Barbara Gili~Fivela, Antonio Stella, Luigia Garrapa \& Mirko
  Grimaldi (eds.), \emph{Contesto comunicativo e variabilità nella produzione
  e percezione della lingua: {P}roceedings of the 7th {C}onference of
  {A}ssociazione {I}taliana di {S}cienze della {V}oce}, Roma: Bulzoni.

\bibitem[{Cangemi \& D'Imperio(2011{\natexlab{a}})}]{cangemi2011local}
Cangemi, Francesco \& Mariapaola D'Imperio. 2011{\natexlab{a}}.
\newblock Local speech rate differences between questions and statements in
  italian.
\newblock In Wai-Sum Lee \& Eric Zee (eds.), \emph{Proceedings of the 17th
  {I}nternational {C}ongress of {P}honetic {S}ciences}, 392--395. Hong Kong:
  City University of Hong Kong.

\bibitem[{Cangemi \& D'Imperio(2011{\natexlab{b}})}]{cangemi2011prosodia}
Cangemi, Francesco \& Mariapaola D'Imperio. 2011{\natexlab{b}}.
\newblock Prosodia oltre la f0: {T}empo e modalità.
\newblock In Barbara Gili~Fivela, Antonio Stella, Luigia Garrapa \& Mirko
  Grimaldi (eds.), \emph{Contesto comunicativo e variabilità nella produzione
  e percezione della lingua: {P}roceedings of the 7th {C}onference of
  {A}ssociazione {I}taliana di {S}cienze della {V}oce}, Roma: Bulzoni.

\bibitem[{Cangemi \& D'Imperio(2013)}]{cangemiFORTHtempo}
Cangemi, Francesco \& Mariapaola D'Imperio. 2013.
\newblock Tempo and the perception of sentence modality.
\newblock \emph{Laboratory Phonology} 4(1). 191--219.

\enlargethispage{\baselineskip}
\bibitem[{Cangemi \& D'Imperio(forthcoming)}]{cangemiFORTHbeyond}
Cangemi, Francesco \& Mariapaola D'Imperio. forthcoming.
\newblock Beyond f0: {S}entence modality and speech rate.
\newblock In Joaquín Romero \& Maria Riera (eds.), \emph{Selected papers from
  the 5th {C}onference on {P}honetics and {P}honology in {I}beria}, Amsterdam:
  John Benjamins.

\bibitem[{Caputo(1994)}]{caputo1994intonazione}
Caputo, Maria~Rosaria. 1994.
\newblock L'intonazione delle domande s{\`i}/no in un campione di italiano
  parlato.
\newblock In \emph{Proceedings of the 4th {Gruppo di Fonetica Sperimentale
  Workshop}}, 9--18. Torino.

\bibitem[{Caputo(1996)}]{caputo1996presupposizione}
Caputo, Maria~Rosaria. 1996.
\newblock Presupposizione, fuoco, modalità e schemi melodici.
\newblock In \emph{Proceedings of the 24th {National Congress of Associazione
  Italiana di Acustica}}, 49--54. Trento.

\bibitem[{Caputo \& D'Imperio(1995)}]{caputo1995possibile}
Caputo, Maria~Rosaria \& Mariapaola D'Imperio. 1995.
\newblock Verso un possibile sistema di trascrizione prosodica dell’italiano:
  {C}enni preliminari.
\newblock In \emph{Proceedings of the 5th workshop of {Gruppo di Fonetica
  Sperimentale}}, 71--83. Trento.

\bibitem[{Charles-Luce(1985)}]{charlesluce1985word}
Charles-Luce, Jan. 1985.
\newblock Word-final devoicing in {G}erman and the effects of phonetic and
  sentential contexts.
\newblock \emph{Journal of Phonetics} 13. 309--324.

\bibitem[{Charles-Luce \& Dinnsen(1987)}]{charlesluce1987reanalysis}
Charles-Luce, Jan \& Daniel Dinnsen. 1987.
\newblock A reanalysis of {C}atalan devoicing.
\newblock \emph{Journal of Phonetics} 15(2). 187--190.

\bibitem[{Chomsky(1965)}]{chomsky1965aspects}
Chomsky, Noam. 1965.
\newblock \emph{Aspects of the theory of syntax}.
\newblock Cambridge: MIT Press.

\bibitem[{Church \& Schacter(1994)}]{church1994perceptual}
Church, Barbara \& Daniel Schacter. 1994.
\newblock Perceptual specificity of auditory priming: {I}mplicit memory for
  voice intonation and fundamental frequency.
\newblock \emph{Journal of Experimental Psychology: Learning, Memory, and
  Cognition} 20(3). 521--533.

\bibitem[{Cole \& Shattuck-Hufnagel(2011)}]{cole2011phonology}
Cole, Jennifer \& Stefanie Shattuck-Hufnagel. 2011.
\newblock The phonology and phonetics of perceived prosody: {W}hat do listeners
  imitate?
\newblock In \emph{Proceedings of the 12th {Annual Conference of the
  International Speech Communication Association}}, 969--972. Firenze.

\bibitem[{Coleman(2003)}]{coleman2003discovering}
Coleman, John. 2003.
\newblock Discovering the acoustic correlates of phonological contrasts.
\newblock \emph{Journal of Phonetics} 31(3-4). 351--372.

\bibitem[{Cooper et~al.(1952)Cooper, Delattre, Liberman, Borst \&
  Gerstman}]{cooper1952experiments}
Cooper, Franklin, Pierre Delattre, Alvin Liberman, John Borst \& Louis
  Gerstman. 1952.
\newblock Some experiments on the perception of synthetic speech sounds.
\newblock \emph{Journal of the Acoustical Society of America} 24(6). 597--606.

\bibitem[{Cooper \& Paccia-Cooper(1980)}]{cooper1980syntax}
Cooper, William \& Jeanne Paccia-Cooper. 1980.
\newblock \emph{Syntax and speech}.
\newblock Cambridge: Harvard University Press.

\bibitem[{Dahan et~al.(2002)Dahan, Tanenhaus \& Chambers}]{dahan2002accent}
Dahan, Delphine, Michael Tanenhaus \& Craig Chambers. 2002.
\newblock Accent and reference resolution in spoken-language comprehension.
\newblock \emph{Journal of Memory and Language} 47(2). 292--314.

\bibitem[{De~Dominicis(2010)}]{dedominicis2010interrogative}
De~Dominicis, Amedeo. 2010.
\newblock Interrogative e assertive in un corpus dialettale recuperato
  ({B}omarzo).
\newblock In Francesco Cutugno, Pietro Maturi, Renata Savy, Giovanni Abete \&
  Iolanda Alfano (eds.), \emph{Parlare con le persone, parlare alle macchine:
  {L}a dimensione interazionale della comunicazione verbale: {P}roceedings of
  the 6th {C}onference of {A}ssociazione {I}taliana di {S}cienze della {V}oce},
  Torriana: EDK.


\enlargethispage{2\baselineskip}
\bibitem[{De~Mauro(1970)}]{demauro1970storia}
De~Mauro, Tullio. 1970.
\newblock \emph{Storia linguistica dell'{I}talia unita (nuova edizione)}.
\newblock Laterza: Laterza.

\bibitem[{Del~Giudice et~al.(2007)Del~Giudice, Shosted, Davidson, Salihie \&
  Arvaniti}]{delgiudice2007comparing}
Del~Giudice, Alex, Ryan Shosted, Kathryn Davidson, Mohammad Salihie \& Amalia
  Arvaniti. 2007.
\newblock Comparing methods for locating pitch ``elbows''.
\newblock In Jürgen Trouvain \& William Barry (eds.), \emph{Proceedings of the
  16th {I}nternational {C}ongress of {P}honetic {S}ciences}, 1117--1120.
  Saarbr\"{u}cken.

\bibitem[{Delattre(1966)}]{delattre1966dix}
Delattre, Pierre. 1966.
\newblock Les dix intonations de base du fran\c{c}ais.
\newblock \emph{The French Review} 40(1). 1--14.

\bibitem[{Delattre et~al.(1955)Delattre, Liberman \&
  Cooper}]{delattre1955acoustic}
Delattre, Pierre, Alvin Liberman \& Franklin Cooper. 1955.
\newblock Acoustic loci and transitional cues for consonants.
\newblock \emph{Journal of the Acoustical Society of America} 27(4). 769--773.

\bibitem[{D'Imperio(1995)}]{dimperio1995timing}
D'Imperio, Mariapaola. 1995.
\newblock Timing differences between prenuclear and nuclear pitch accents in
  {I}talian.
\newblock \emph{Journal of the Acoustical Society of America} 98(5). 2894.

\bibitem[{D'Imperio(1996)}]{dimperio1996caratteristiche}
D'Imperio, Mariapaola. 1996.
\newblock Caratteristiche di timing degli accenti nucleari in parlato italiano
  letto.
\newblock In \emph{Proceedings of the 24th {National Congress of Associazione
  Italiana di Acustica}}, 55--60. Trento.

\bibitem[{D'Imperio(1997{\natexlab{a}})}]{dimperio1997breadth}
D'Imperio, Mariapaola. 1997{\natexlab{a}}.
\newblock Breadth of focus, modality, and prominence perception in {N}eapolitan
  {I}talian.
\newblock \emph{Working Papers in Linguistics -- Ohio State University} 50.
  19--39.

\bibitem[{D'Imperio(1997{\natexlab{b}})}]{dimperio1997narrow}
D'Imperio, Mariapaola. 1997{\natexlab{b}}.
\newblock Narrow focus and focal accent in the {N}eapolitan variety of
  {I}talian.
\newblock In Antonis Botinis, Georgios Kouroupetroglou \& George Carayiannis
  (eds.), \emph{Intonation: {T}heory, models and applications. {Proceedings of
  an European Conference on Speech Communication and Technology Workshop}},
  87--90. Athens.

\bibitem[{D'Imperio(1999)}]{dimperio1999tonal}
D'Imperio, Mariapaola. 1999.
\newblock Tonal structure and pitch targets in {I}talian focus constituents.
\newblock In John Ohala (ed.), \emph{Proceedings of the 14th {I}nternational
  {C}ongress of {P}honetic {S}ciences}, 1757--1760. San Francisco: University
  of California.

\bibitem[{D'Imperio(2000)}]{dimperio2000role}
D'Imperio, Mariapaola. 2000.
\newblock \emph{The role of perception in defining tonal targets and their
  alignment}: Columbus: The Ohio State University dissertation.

\bibitem[{D'Imperio(2001)}]{dimperio2001focus}
D'Imperio, Mariapaola. 2001.
\newblock Focus and tonal structure in neapolitan italian.
\newblock \emph{Speech Communication} 33(4). 339--356.

\bibitem[{D'Imperio(2002)}]{dimperio2002italian}
D'Imperio, Mariapaola. 2002.
\newblock Italian intonation: {A}n overview and some questions.
\newblock \emph{Probus} 14(1). 37--69.

\bibitem[{D'Imperio(2003)}]{dimperio2003tonal}
D'Imperio, Mariapaola. 2003.
\newblock Tonal structure and pitch targets in {I}talian focus constituents.
\newblock \emph{Catalan Journal of Linguistics} 2. 55--65.


%%\enlargethispage{\baselineskip}
\bibitem[{D'Imperio \& Cangemi(2009)}]{dimperio2009interplay}
D'Imperio, Mariapaola \& Francesco Cangemi. 2009.
\newblock The interplay between tonal alignment and rise shape in the
  perception of two {N}eapolitan rising accents.
\newblock Talk presented at the 4th Conference on Phonetics and Phonology in
  Iberia, Gran Canaria, Spain.

\enlargethispage{\baselineskip}
\bibitem[{D'Imperio \& Cangemi(2011)}]{dimperio2011phrasing}
D'Imperio, Mariapaola \& Francesco Cangemi. 2011.
\newblock Phrasing, register level downstep and partial topic constructions in
  {N}eapolitan {I}talian.
\newblock In Christoph Gabriel \& Conxita Lle\'{o} (eds.), \emph{Intonational
  phrasing in {R}omance and {G}ermanic: {C}ross-linguistic and bilingual
  studies}, 75--94. Amsterdam: John Benjamins.

\bibitem[{D'Imperio et~al.(2008)D'Imperio, Cangemi \&
  Brunetti}]{dimperio2008phonetics}
D'Imperio, Mariapaola, Francesco Cangemi \& Lisa Brunetti. 2008.
\newblock The phonetics and phonology of contrastive topic constructions in
  {I}talian.
\newblock Poster presented at the 3rd Conference on Tone and Intonation in
  Europe, Lisbon, Portugal.

\bibitem[{D'Imperio et~al.(2005)D'Imperio, Elordieta, Frota, Prieto \&
  Vig\`{a}rio}]{dimperio2005intonational}
D'Imperio, Mariapaola, Gorka Elordieta, Sónia Frota, Pilar Prieto \& Marina
  Vig\`{a}rio. 2005.
\newblock Intonational phrasing in {R}omance: {T}he role of syntactic and
  prosodic structure.
\newblock In Sónia Frota, Marina Vig\`{a}rio \& Maria Freitas (eds.),
  \emph{Prosodies}, 59--97. Berlin, New York: Mouton de Gruyter.

\bibitem[{D'Imperio \& Gili~Fivela(2003)}]{dimperio2003levels}
D'Imperio, Mariapaola \& Barbara Gili~Fivela. 2003.
\newblock How many levels of phrasing? {E}vidence from two varieties of
  {I}talian.
\newblock In John Local, Richard Ogden \& Rosalind Temple (eds.), \emph{Papers
  in {L}aboratory {P}honology}, vol.~6, 38--57. Cambridge: Cambridge University
  Press.

\bibitem[{D'Imperio \& House(1997)}]{dimperio1997perception}
D'Imperio, Mariapaola \& David House. 1997.
\newblock Perception of questions and statements in {N}eapolitan {I}talian.
\newblock In George Kokkinakis, Nikos Fakotakis \& Evangelos Dermatas (eds.),
  \emph{Proceedings of the 5th {European Conference on Speech Communication and
  Technology}}, 251--254. Rhodes.

\bibitem[{D'Imperio et~al.(2007)D'Imperio, Petrone \&
  Nguyen}]{dimperio2007effects}
D'Imperio, Mariapaola, Caterina Petrone \& Noël Nguyen. 2007.
\newblock Effects of tonal alignment on lexical identification in {I}talian.
\newblock In Tomas Riad \& Carlos Gussenhoven (eds.), \emph{Tones and tunes:
  {E}xperimental studies in word and sentence prosody}, vol.~2, 79--106.
  Berlin: de Gruyter.

\bibitem[{Dinnsen \& Charles-Luce(1984)}]{dinnsen1984phonological}
Dinnsen, Daniel \& Jan Charles-Luce. 1984.
\newblock Phonological neutralization, phonetic implementation and individual
  differences.
\newblock \emph{Journal of Phonetics} 12(1). 49--60.

\bibitem[{Dinnsen \& {Garcia Zamor}(1971)}]{dinnsen1971three}
Dinnsen, Daniel \& Maria {Garcia Zamor}. 1971.
\newblock The three degrees of vowel length in {G}erman.
\newblock \emph{Research on Language \& Social Interaction} 4(1). 111--126.

\bibitem[{Dmitrieva et~al.(2010)Dmitrieva, Jongman \&
  Sereno}]{dmitrieva2010phonological}
Dmitrieva, Olga, Allard Jongman \& Joan Sereno. 2010.
\newblock Phonological neutralization by native and non-native speakers: {T}he
  case of {R}ussian final devoicing.
\newblock \emph{Journal of Phonetics} 38(3). 483--492.

\bibitem[{Dombrowski \& Niebuhr(2005)}]{dombrowski2005acoustic}
Dombrowski, Ernst \& Oliver Niebuhr. 2005.
\newblock Acoustic patterns and communicative functions of phrase-final f0
  rises in {G}erman: {A}ctivating and restricting contours.
\newblock \emph{Phonetica} 62(2-4). 176--195.

\bibitem[{Dryer(2011)}]{wals-2011-116}
Dryer, Matthew. 2011.
\newblock Polar questions.
\newblock In Matthew Dryer \& Martin Haspelmath (eds.), \emph{{The World Atlas
  of Language Structures Online}}, Munich: Max Planck Digital Library.

\bibitem[{Duncan(1972)}]{duncan1972signals}
Duncan, Starkey. 1972.
\newblock Some signals and rules for taking speaking turns in conversations.
\newblock \emph{Journal of Personality and Social Psychology} 23(2). 283--292.

\enlargethispage{\baselineskip}
\bibitem[{Eefting(1991)}]{eefting1991effect}
Eefting, Wieke. 1991.
\newblock The effect of ‘‘information value’’and
  ‘‘accentuation’’ on the duration of {D}utch words, syllables, and
  segments.
\newblock \emph{Journal of the Acoustical Society of America} 89(1). 412--424.

\bibitem[{Elman \& McClelland(1988)}]{elman1988cognitive}
Elman, Jeffrey \& James McClelland. 1988.
\newblock Cognitive penetration of the mechanisms of perception: {C}ompensation
  for coarticulation of lexically restored phonemes.
\newblock \emph{Journal of Memory and Language} 27(2). 143--165.

\bibitem[{Ernestus(2014)}]{ernestusacoustic}
Ernestus, Mirjam. 2014.
\newblock Acoustic reduction and the roles of abstractions and exemplars in
  speech processing.
\newblock \emph{Lingua} 142. 27--41.

\bibitem[{Ernestus \& Baayen(2006)}]{ernestus2006functionality}
Ernestus, Mirjam \& Harald Baayen. 2006.
\newblock The functionality of incomplete neutralization in {D}utch: {T}he case
  of past-tense formation.
\newblock In Louis Goldstein, Douglas Whalen \& Catherine Best (eds.),
  \emph{Papers in {L}aboratory {P}honology}, vol.~8, 27--49. Cambridge:
  Cambridge University Press.

\bibitem[{Faber(1992)}]{faber1992phonemic}
Faber, Alice. 1992.
\newblock Phonemic segmentation as epiphenomenon: {E}vidence from the history
  of alphabetic writing.
\newblock In Pamela Downing, Susan Lima \& Michael Noonan (eds.), \emph{The
  linguistics of literacy}, 111--134. Amsterdam: John Benjamins.

\bibitem[{Face(2001)}]{face2001focus}
Face, Timothy. 2001.
\newblock Focus and early peak alignment in {S}panish intonation.
\newblock \emph{Probus} 13(2). 223--246.

\bibitem[{Farnetani \& Kori(1991)}]{farnetani1990rhytmic}
Farnetani, Edda \& Shiro Kori. 1991.
\newblock Rhytmic structure in {I}talian noun phrases: {A} study on vowel
  durations.
\newblock \emph{Phonetica} 47. 50--65.

\bibitem[{Firth(1948)}]{firth1948prosodies}
Firth, John. 1948.
\newblock Sounds and prosodies.
\newblock \emph{Transactions of the Philological Society} 47(1). 127--152.

\bibitem[{Flemming(1997)}]{flemming1997phonetic}
Flemming, Edward. 1997.
\newblock Phonetic detail in phonology: {T}owards a unified account of
  assimilation and coarticulation.
\newblock In Keiichiro Suzuki \& Dirk Elzinga (eds.), \emph{Proceedings of the
  1995 {Southwestern Workshop in Optimality Theory (SWOT)}}, Tucson.

\bibitem[{Flemming(2001)}]{flemming2001scalar}
Flemming, Edward. 2001.
\newblock Scalar and categorical phenomena in a unified model of phonetics and
  phonology.
\newblock \emph{Phonology} 18(1). 7--44.

\bibitem[{Fourakis \& Iverson(1984)}]{fourakis1984incomplete}
Fourakis, Marios \& Gregory Iverson. 1984.
\newblock On the ‘incomplete neutralization’ of {G}erman final obstruents.
\newblock \emph{Phonetica} 41(3). 140--149.

\bibitem[{Frick(1995)}]{frick1995accepting}
Frick, Robert. 1995.
\newblock Accepting the null hypothesis.
\newblock \emph{Memory \& Cognition} 23(1). 132--138.

\bibitem[{Frota et~al.(2007)Frota, D'Imperio, Elordieta, Prieto \&
  Vig\`{a}rio}]{frota2007phonetics}
Frota, Sónia, Mariapaola D'Imperio, Gorka Elordieta, Pilar Prieto \& Marina
  Vig\`{a}rio. 2007.
\newblock The phonetics and phonology of intonational phrasing in {R}omance.
\newblock In Pilar Prieto, Joan Mascar\`{o} \& Maria~Josep Sol\'{e} (eds.),
  \emph{Segmental and prosodic issues in {R}omance phonology}, 131--154.
  Amsterdam: John Benjamins.

%\enlargethispage{\baselineskip}
\bibitem[{Fujisaki \& Hirose(1982)}]{fujisaki1982modelling}
Fujisaki, Hiroya \& Keikichi Hirose. 1982.
\newblock Modelling the dynamic characteristics of voice fundamental frequency
  with application to analysis and synthesis of intonation.
\newblock In Shiro Hattori \& Kazuko Inoue (eds.), \emph{Proceedings of 13th
  {International Congress of Linguists}}, 57--70. Tokyo.

\bibitem[{Gili~Fivela(2004)}]{gilifivela2004phonetics}
Gili~Fivela, Barbara. 2004.
\newblock \emph{The phonetics and phonology of intonation: {T}he case of {P}isa
  {I}talian}: Pisa: Scuola Normale Superiore dissertation.

\bibitem[{Gili~Fivela(2006)}]{gilifivela2006coding}
Gili~Fivela, Barbara. 2006.
\newblock The coding of target alignment and scaling in pitch accent
  transcription.
\newblock \emph{Italian Journal of Linguistics} 18(1). 189--221.

\bibitem[{Gili~Fivela(2008)}]{gilifivela2008intonation}
Gili~Fivela, Barbara. 2008.
\newblock \emph{Intonation in production and perception: {T}he case of {P}isa
  {I}talian}.
\newblock Alessandria: Edizioni dell'{O}rso.

\bibitem[{Gili~Fivela \& D'Imperio(2003)}]{gilifivela2003tonal}
Gili~Fivela, Barbara \& Mariapaola D'Imperio. 2003.
\newblock Tonal alignment of prenuclear accents in {I}talian.
\newblock Poster presented at the 2nd Conference on Tone and Intonation in
  Europe, Santorini, Greece.

\bibitem[{Goldinger(1996)}]{goldinger1996words}
Goldinger, Stephen. 1996.
\newblock Words and voices: {E}pisodic traces in spoken word identification and
  recognition memory.
\newblock \emph{Journal of Experimental Psychology: Learning, Memory, and
  Cognition} 22(5). 1166--1183.

\bibitem[{Goldinger et~al.(1991)Goldinger, Pisoni \&
  Logan}]{goldinger1991nature}
Goldinger, Stephen, David Pisoni \& John Logan. 1991.
\newblock On the nature of talker variability effects on recall of spoken word
  lists.
\newblock \emph{Journal of Experimental Psychology: Learning, Memory, and
  Cognition} 17(1). 152--162.

\bibitem[{Goldman(2011)}]{goldman2011easyalign}
Goldman, Jean-Philippe. 2011.
\newblock Easy{A}lign: {A}n automatic phonetic alignment tool under {P}raat.
\newblock In \emph{Proceedings of the 12th {Annual Conference of the
  International Speech Communication Association}}, Firenze.

\bibitem[{Grice(1991)}]{grice1991intonation}
Grice, Martine. 1991.
\newblock The intonation of interrogation in two varieties of {S}icilian
  {I}talian.
\newblock In \emph{Proceedings of the 12th {I}nternational {C}ongress of
  {P}honetic {S}ciences}, vol.~5, 210--213. Aix-en-Provence.

\bibitem[{Grice(1995)}]{grice1995intonation}
Grice, Martine. 1995.
\newblock \emph{The intonation of interrogation in {P}alermo {I}talian:
  {I}mplications for intonation theory}.
\newblock T\"{u}bingen: Niemeyer.

\bibitem[{Grice \& Baumann(2002)}]{grice2002deutsche}
Grice, Martine \& Stefan Baumann. 2002.
\newblock Deutsche {I}ntonation und {GToBI}.
\newblock \emph{Linguistische Berichte} 191. 267--298.

\bibitem[{Grice et~al.(2005{\natexlab{a}})Grice, Baumann \&
  Benzm\"{u}lller}]{grice2005german}
Grice, Martine, Stefan Baumann \& Ralf Benzm\"{u}lller. 2005{\natexlab{a}}.
\newblock German intonation in autosegmental-metrical phonology.
\newblock In Sun-Ah Jun (ed.), \emph{Prosodic typology: {T}he phonology of
  intonation and phrasing}, 55--83. Oxford: Oxford University Press.

\bibitem[{Grice et~al.(2005{\natexlab{b}})Grice, D'Imperio, Savino \&
  Avesani}]{grice2005strategy}
Grice, Martine, Mariapaola D'Imperio, Michelina Savino \& Cinzia Avesani.
  2005{\natexlab{b}}.
\newblock Towards a strategy for labelling varieties of italian.
\newblock In Sun-Ah Jun (ed.), \emph{Prosodic typology: {T}he phonology of
  intonation and phrasing}, 362--389. Oxford: Oxford University Press.

\bibitem[{Grice et~al.(2000)Grice, Ladd \& Arvaniti}]{grice2000place}
Grice, Martine, Robert Ladd \& Amalia Arvaniti. 2000.
\newblock On the place of phrase accents in intonational phonology.
\newblock \emph{Phonology} 17(2). 143--185.

\bibitem[{Gubian et~al.(2010)Gubian, Cangemi \& Boves}]{gubian2010automatic}
Gubian, Michele, Francesco Cangemi \& Lou Boves. 2010.
\newblock Automatic and data driven pitch contour manipulation with functional
  data analysis.
\newblock In \emph{Proceedings of the 5th {I}nternational {C}onference on
  {S}peech {P}rosody}, Chicago.

\enlargethispage{\baselineskip}
\bibitem[{Gubian et~al.(2011)Gubian, Cangemi \& Boves}]{gubian2011joint}
Gubian, Michele, Francesco Cangemi \& Lou Boves. 2011.
\newblock Joint analysis of f0 and speech rate with functional data analysis.
\newblock In \emph{Proceedings of 36th {I}nternational {C}onference of
  {A}coustics, {S}peech and {S}ignal {P}rocessing}, 4972--4975. Prague.

\bibitem[{Gussenhoven(1984)}]{gussenhoven1984grammar}
Gussenhoven, Carlos. 1984.
\newblock \emph{On the grammar and semantics of sentence accents}.
\newblock Dordrecht: Foris.

\bibitem[{Gussenhoven(2004)}]{gussenhoven2004phonology}
Gussenhoven, Carlos. 2004.
\newblock \emph{The phonology of tone and intonation}.
\newblock Cambridge: Cambridge University Press.

\bibitem[{Gussenhoven(2006)}]{gussenhoven2006experimental}
Gussenhoven, Carlos. 2006.
\newblock Experimental approaches to establishing discreteness of intonational
  contrasts.
\newblock In Stefan Sudhoff, Denisa Lenertov\'{a}, Roland Meyer, Sandra
  Pappert, Petra Augurzky, Ina Mleinek, Nicole Richter \& Johannes
  Schlie\ss{}er (eds.), \emph{Methods in empirical prosody research}, 321--334.
  Berlin: De Gruyter.

\bibitem[{Haan(2002)}]{haan2002speaking}
Haan, Judith. 2002.
\newblock \emph{Speaking of questions}.
\newblock Utrecht: LOT.

\bibitem[{Harris(1955)}]{harris1955phoneme}
Harris, Zellig. 1955.
\newblock From phoneme to morpheme.
\newblock \emph{Language} 31(2). 190--222.

\bibitem[{Hawkins(2003)}]{hawkins2003roles}
Hawkins, Sarah. 2003.
\newblock Roles and representations of systematic fine phonetic detail in
  speech understanding.
\newblock \emph{Journal of Phonetics} 31(3-4). 373--405.

\bibitem[{Hawkins(2010)}]{hawkins2010phonetic}
Hawkins, Sarah. 2010.
\newblock Phonetic variation as communicative system: {P}erception of the
  particular and the abstract.
\newblock In C\'{e}cile Fougeron, Barbara K\"{u}hnert, Mariapaola D'Imperio \&
  Nathalie Vall\'{e}e (eds.), \emph{Papers in {L}aboratory {P}honology},
  vol.~10, 479--510. Cambridge: Cambridge University Press.

\bibitem[{Hawkins(2011)}]{hawkins2011phonetic}
Hawkins, Sarah. 2011.
\newblock Does phonetic detail guide situation-specific speech recognition?
\newblock In Wai-Sum Lee \& Eric Zee (eds.), \emph{Proceedings of the 17th
  {I}nternational {C}ongress of {P}honetic {S}ciences}, 9--18. Hong Kong: City
  University of Hong Kong.

\bibitem[{Hawkins \& Nguyen(2003)}]{hawkins2003effects}
Hawkins, Sarah \& Noël Nguyen. 2003.
\newblock Effects on word recognition of syllable-onset cues to syllable-coda
  voicing.
\newblock In John Local, Richard Ogden \& Rosalind Temple (eds.), \emph{Papers
  in {L}aboratory {P}honology}, vol.~6, 38--57. Cambridge: Cambridge University
  Press.

\bibitem[{Hawkins \& Nguyen(2004)}]{hawkins2004influence}
Hawkins, Sarah \& Noël Nguyen. 2004.
\newblock Influence of syllable-coda voicing on the acoustic properties of
  syllable-onset /l/ in {E}nglish.
\newblock \emph{Journal of Phonetics} 32(2). 199--231.

\bibitem[{Hawkins \& Smith(2001)}]{hawkins2001polysp}
Hawkins, Sarah \& Rachel Smith. 2001.
\newblock Polysp: {A} polysystemic, phonetically-rich approach to speech
  understanding.
\newblock \emph{Italian Journal of Linguistics} 13. 99--188.

\bibitem[{Hawkins \& Warren(1991)}]{hawkins1991factors}
Hawkins, Sarah \& Paul Warren. 1991.
\newblock Factors affecting the given-new distinction in speech.
\newblock In \emph{Proceedings of the 12th {I}nternational {C}ongress of
  {P}honetic {S}ciences}, 66--69. Aix-en-Provence: Universit\'{e} de Provence.

\bibitem[{Heinrich et~al.(2010)Heinrich, Flory \&
  Hawkins}]{heinrich2010influence}
Heinrich, Antje, Yvonne Flory \& Sarah Hawkins. 2010.
\newblock Influence of english r-resonances on intelligibility of speech in
  noise for native {E}nglish and {G}erman listeners.
\newblock \emph{Speech Communication} 52(11). 1038--1055.

\bibitem[{Henriksen(2012)}]{henriksen2012intonation}
Henriksen, Nicholas. 2012.
\newblock The intonation and signaling of declarative questions in {M}anchego
  {P}eninsular {S}panish.
\newblock \emph{Language and Speech} 55(4).

\enlargethispage{\baselineskip}
\bibitem[{Hintzman(1986)}]{hintzman1986schema}
Hintzman, Douglas. 1986.
\newblock Schema abstraction in a multiple-trace memory model.
\newblock \emph{Psychological Review} 93(4). 411--428.

\bibitem[{Hirschberg \& Ward(1992)}]{hirschberg1992influence}
Hirschberg, Julia \& Gregory Ward. 1992.
\newblock The influence of pitch range, duration, amplitude and spectral
  features on the interpretation of the rise-fall-rise intonation contour in
  {E}nglish.
\newblock \emph{Journal of {P}honetics} 20. 241--251.

\bibitem[{Hooper(1976)}]{hooper1976word}
Hooper, Joan. 1976.
\newblock Word frequency in lexical diffusion and the source of
  morphophonological change.
\newblock In William Christie (ed.), \emph{Current progress in historical
  linguistics. {P}roceedings of the 2nd {I}nternational {C}onference on
  {H}istorical {L}inguistics}, 96--105. Amsterdam: North-Holland.

\bibitem[{House(1990)}]{house1990tonal}
House, David. 1990.
\newblock \emph{Tonal perception in speech}.
\newblock Lund: Lund University Press.

\bibitem[{House(1997)}]{house1997perceptual}
House, David. 1997.
\newblock Perceptual thresholds and tonal categories.
\newblock \emph{Phonum} 4. 179--182.

\bibitem[{Hualde(2002)}]{hualde2002intonation}
Hualde, José. 2002.
\newblock Intonation in {S}panish and the other {Ibero-Romance} languages:
  {O}verview and status quaestionis.
\newblock In Caroline Wiltshire \& Joaquim Camps (eds.), \emph{Romance
  phonology and variation}, 101--116. Amsterdam: John Benjamins.

\bibitem[{Huddleston(1994)}]{huddleston1994contrast}
Huddleston, Rodney. 1994.
\newblock The contrast between interrogatives and questions.
\newblock \emph{Journal of Linguistics} 30(2). 411--39.

\bibitem[{{Institut f\"{u}r Phonetik und digitale
  Sprachverarbeitung}(1994)}]{ipds1994kcrs}
{Institut f\"{u}r Phonetik und digitale Sprachverarbeitung}. 1994.
\newblock The {K}iel {C}orpus of {R}ead {S}peech.
\newblock CD rom.

\bibitem[{Isa{\v{c}}enko \& Sch{\"a}dlich(1970)}]{isavcenko1970untersuchungen}
Isa{\v{c}}enko, Aleksandr \& Hans~Joachim Sch{\"a}dlich. 1970.
\newblock \emph{Untersuchungen {\"u}ber die deutsche {S}atzintonation}.
\newblock Berlin: Mouton.

\bibitem[{Jackendoff(1972)}]{jackendoff1972semantic}
Jackendoff, Ray. 1972.
\newblock \emph{Semantic interpretation in generative grammar}.
\newblock Cambridge: MIT Press.

\bibitem[{Jakobson et~al.(1952)Jakobson, Fant \&
  Halle}]{jakobson1952preliminaries}
Jakobson, Roman, Gunnar Fant \& Morris Halle. 1952.
\newblock Preliminaries to speech analysis: {T}he distinctive features.
\newblock MIT {A}coustics {L}aboratory technical report.

\bibitem[{Jassem \& Richter(1989)}]{jassem1989neutralization}
Jassem, Wiktor \& Lutosława Richter. 1989.
\newblock Neutralization of voicing in {P}olish obstruents.
\newblock \emph{Journal of Phonetics} 17(4). 317--326.

\bibitem[{Johnson(1997)}]{johnson1997speech}
Johnson, Keith. 1997.
\newblock Speech perception without speaker normalization.
\newblock In Keith Johnson \& John Mullennix (eds.), \emph{Talker variability
  in speech processing}, 145--165. San Diego: Academic Press.

\bibitem[{Johnson \& Mullennix(1997)}]{johnson1997complex}
Johnson, Keith \& John Mullennix. 1997.
\newblock Complex representations used in speech processing.
\newblock In Keith Johnson \& John Mullennix (eds.), \emph{Talker variability
  in speech processing}, 1--8. San Diego: Academic Press.

\bibitem[{Jun(1993)}]{jun1993phonetics}
Jun, Sun-Ah. 1993.
\newblock \emph{The phonetics and phonology of {K}orean prosody}: Columbus: The
  Ohio State University dissertation.

\bibitem[{Jusczyk \& Luce(2002)}]{jusczyk2002speech}
Jusczyk, Peter \& Paul Luce. 2002.
\newblock Speech perception and spoken word recognition: {P}ast and present.
\newblock \emph{Ear and Hearing} 23(1). 2--40.

\enlargethispage{\baselineskip}
\bibitem[{Kelly \& Local(1986)}]{kelly1986long}
Kelly, John \& John Local. 1986.
\newblock Long-domain resonance patterns in {E}nglish.
\newblock In \emph{Proceedings of the {International Conference on Speech
  Input/Output, Techniques and Applications}}, 304--308. London.

\bibitem[{Kirsner et~al.(1994)Kirsner, {van Heuven} \& {van
  Bezooijen}}]{kirsner1994interaction}
Kirsner, Robert, Vincent {van Heuven} \& Renée {van Bezooijen}. 1994.
\newblock Interaction of particle and prosody in the interpretation of factual
  {D}utch sentences.
\newblock In Reineke Bok-Bennema \& Crit Cremers (eds.), \emph{Linguistics in
  the {N}etherlands}, 107--118. Amsterdam: John Benjamins.

\bibitem[{Klatt(1973)}]{klatt1973discrimination}
Klatt, Dennis. 1973.
\newblock Discrimination of fundamental frequency contours in synthetic speech:
  {I}mplications for models of pitch perception.
\newblock \emph{Journal of the Acoustical Society of America} 53(1). 8--16.

\bibitem[{Klatt(1979)}]{klatt1979speech}
Klatt, Dennis. 1979.
\newblock Speech perception: {A} model of acoustic-phonetic analysis and
  lexical access.
\newblock \emph{Journal of Phonetics} 7. 279--312.

\bibitem[{Kleber et~al.(2010)Kleber, John \&
  Harrington}]{kleber2010implications}
Kleber, Felicitas, Tina John \& Jonathan Harrington. 2010.
\newblock The implications for speech perception of incomplete neutralization
  of final devoicing in {G}erman.
\newblock \emph{Journal of Phonetics} 38(2). 185--196.

\bibitem[{Kohler(1987)}]{kohler1987categorical}
Kohler, Klaus. 1987.
\newblock Categorical pitch perception.
\newblock In \emph{Proceedings of the 11th {I}nternational {C}ongress of
  {P}honetic {S}ciences}, vol.~5, 331--333. Tallin: Academy of Sciences.

\bibitem[{Kohler(1991)}]{kohler1991model}
Kohler, Klaus. 1991.
\newblock A model of {G}erman intonation.
\newblock \emph{Arbeitsberichte des Instituts f\"{u}r Phonetik der
  Universit\"{a}t Kiel} 25. 295--360.

\bibitem[{Kopkalli(1993)}]{kopkalli1993phonetic}
Kopkalli, Handan. 1993.
\newblock \emph{A phonetic and phonological analysis of final devoicing in
  {T}urkish}: Ann Arbor: University of Michigan dissertation.

\bibitem[{Kretschmer(1938)}]{kretschmer1938ursprung}
Kretschmer, Paul. 1938.
\newblock Der {U}rsprung des {F}ragetons \& {F}ragesatzes.
\newblock In \emph{Scritti in onore di alfredo trombetti}, 27--50. Milano:
  Hoepli.

\bibitem[{Kruschke(1992)}]{kruschke1992alcove}
Kruschke, John. 1992.
\newblock {ALCOVE}: {A}n exemplar-based connectionist model of category
  learning.
\newblock \emph{Psychological Review} 99(1). 22--44.

\bibitem[{Ladd(1996)}]{ladd1996intonational}
Ladd, Robert. 1996.
\newblock \emph{Intonational phonology}.
\newblock Cambridge: Cambridge University Press.

\bibitem[{Ladd(2008)}]{ladd2008intonational}
Ladd, Robert. 2008.
\newblock \emph{Intonational phonology (2nd edition)}.
\newblock Cambridge: Cambridge University Press.

\bibitem[{Ladd \& Schepman(2003)}]{ladd2003sagging}
Ladd, Robert \& Astrid Schepman. 2003.
\newblock ``{S}agging transitions'' between high pitch accents in {E}nglish:
  {E}xperimental evidence.
\newblock \emph{Journal of Phonetics} 31(1). 81--112.

\bibitem[{Ladefoged(2000)}]{ladefoged2000course}
Ladefoged, Peter. 2000.
\newblock \emph{A course in phonetics (4th edition)}.
\newblock Boston: Heinle \& Heinle.

\bibitem[{Ladefoged \& Broadbent(1957)}]{ladefoged1957information}
Ladefoged, Peter \& Donald Broadbent. 1957.
\newblock Information conveyed by vowels.
\newblock \emph{Journal of the Acoustical Society of America} 29(1). 98--104.

\bibitem[{Lehiste(1970)}]{lehiste1970suprasegmentals}
Lehiste, Ilse. 1970.
\newblock \emph{Suprasegmentals}.
\newblock Cambridge: MIT Press.

\bibitem[{Levi \& Bruno(2010)}]{levi2010priming}
Levi, Susannah \& Jennifer Bruno. 2010.
\newblock Priming at the level of phonetic detail: {E}vidence from voice onset
  time.
\newblock \emph{Journal of the Acoustical Society of America} 127(3). 1853.

\bibitem[{Liberman(1979)}]{liberman1979intonational}
Liberman, Mark. 1979.
\newblock \emph{The intonational system of {E}nglish}.
\newblock New York: Garland.

\bibitem[{Licklider(1952)}]{licklider1952process}
Licklider, Joseph. 1952.
\newblock On the process of speech perception.
\newblock \emph{Journal of the Acoustical Society of America} 24(6). 590--594.

\bibitem[{Lindgren(1965)}]{lindgren1965machine}
Lindgren, Nilo. 1965.
\newblock Machine recognition of human language.
\newblock \emph{Spectrum} 2(3). 114--136.

\bibitem[{Lisker(1986)}]{lisker1986voicing}
Lisker, Leigh. 1986.
\newblock Voicing in {E}nglish: {A} catalogue of acoustic features signaling
  /b/ versus /p/ in trochees.
\newblock \emph{Language and Speech} 29. 3--11.

\bibitem[{Lisker \& Abramson(1964)}]{lisker1964crosslanguage}
Lisker, Leigh \& Arthur Abramson. 1964.
\newblock A cross-language study of voicing in initial stops: {A}coustical
  measurements.
\newblock \emph{Word} 20(3). 384--422.

\bibitem[{Local(2003{\natexlab{a}})}]{local2003phonetics}
Local, John. 2003{\natexlab{a}}.
\newblock Phonetics and talk-in-interaction.
\newblock In Daniel Recasens, Maria~Josep Sol\'{e} \& Joaquín Romero (eds.),
  \emph{Proceedings of the 15th {I}nternational {C}ongress of {P}honetic
  {S}ciences}, 115--118. Barcelona.

\bibitem[{Local(2003{\natexlab{b}})}]{local2003variable}
Local, John. 2003{\natexlab{b}}.
\newblock Variable domains and variable relevance: {I}nterpreting phonetic
  exponents.
\newblock \emph{Journal of Phonetics} 31(3-4). 321--339.

\bibitem[{Luce \& McLennan(2005)}]{luce2005spoken}
Luce, Paul \& Conor McLennan. 2005.
\newblock Spoken word recognition: {T}he challenge of variation.
\newblock In David Pisoni \& Robert Remez (eds.), \emph{The handbook of speech
  perception}, 590--609. Hoboken: Wiley-Blackwell.

\bibitem[{Luce et~al.(2003)Luce, McLennan \&
  Charles-Luce}]{luce2003abstractness}
Luce, Paul, Conor McLennan \& Jan Charles-Luce. 2003.
\newblock Abstractness and specificity in spoken word recognition: {I}ndexical
  and allophonic variability in long-term repetition priming.
\newblock In Jeffrey Bowers \& Chad Marsolek (eds.), \emph{Rethinking implicit
  memory}, 145--165. Oxford: Oxford University Press.

\bibitem[{Lyberg(1981)}]{lyberg1981observations}
Lyberg, Bertil. 1981.
\newblock Some observations on the vowel duration and the fundamental frequency
  contour in {S}wedish utterances.
\newblock \emph{Journal of Phonetics} 9. 261--272.

\bibitem[{Lyons(1977)}]{lyons1977semantics}
Lyons, John. 1977.
\newblock \emph{Semantics}.
\newblock Cambridge: Cambridge University Press.

\bibitem[{Magno~Caldognetto et~al.(1978)Magno~Caldognetto, Ferrero, Lavagnoli
  \& Vagges}]{magnocaldognetto1978f0}
Magno~Caldognetto, Emanuela, Franco Ferrero, Carlo Lavagnoli \& Kyriaki Vagges.
  1978.
\newblock F0 contours of statements, yes-no questions, and wh-questions of two
  regional varieties of {I}talian.
\newblock \emph{Journal of Italian Linguistics} 3(1). 57--66.

\bibitem[{{Manaster Ramer}(1996)}]{manaster1996letter}
{Manaster Ramer}, Alexis. 1996.
\newblock A letter from an incompletely neutral phonologist.
\newblock \emph{Journal of Phonetics} 24(4). 477--489.

\bibitem[{Marchese(1978)}]{marchese1978atlas}
Marchese, Lynell. 1978.
\newblock \emph{Atlas linguistique {K}ru: {E}ssai de typologie}.
\newblock Abidjan: Institut de Linguistique Appliqu\'{e}e.

\bibitem[{Marslen-Wilson \& Tyler(1980)}]{marslen1980temporal}
Marslen-Wilson, William \& Lorraine Tyler. 1980.
\newblock The temporal structure of spoken language understanding.
\newblock \emph{Cognition} 8(1). 1--71.

\bibitem[{Marslen-Wilson \& Welsh(1978)}]{marslen1978processing}
Marslen-Wilson, William \& Alan Welsh. 1978.
\newblock Processing interactions and lexical access during word recognition in
  continuous speech.
\newblock \emph{Cognitive Psychology} 10(1). 29--63.

\bibitem[{Mascar{\'o}(1987)}]{mascaro1987underlying}
Mascar{\'o}, Joan. 1987.
\newblock Underlying voicing recoverability of finally devoiced obstruents in
  {C}atalan.
\newblock \emph{Journal of Phonetics} 15. 183--186.

\bibitem[{Matthews(1993)}]{matthews1993grammatical}
Matthews, Peter. 1993.
\newblock \emph{Grammatical theory in the {U}nited {S}tates from {B}loomfield
  to {C}homsky}.
\newblock Cambridge: Cambridge University Press.

\bibitem[{Maturi(1988)}]{maturi1988intonazione}
Maturi, Pietro. 1988.
\newblock L'intonazione delle frasi dichiarative ed interrogative nella
  variet\`{a} napoletana dell'italiano.
\newblock \emph{Rivista Italiana di Acustica} 12. 13--30.

\bibitem[{McClelland(1981)}]{mcclelland1981retrieving}
McClelland, James. 1981.
\newblock Retrieving general and specific information from stored knowledge of
  specifics.
\newblock In \emph{Proceedings of the 3rd {Annual Conference of the Cognitive
  Science Society}}, 170--172. Berkeley.

\bibitem[{McClelland \& Elman(1986)}]{mcclelland1986trace}
McClelland, James \& Jeffrey Elman. 1986.
\newblock The {TRACE} model of speech perception.
\newblock \emph{Cognitive Psychology} 18(1). 1--86.

\bibitem[{McClelland \& Rumelhart(1985)}]{mcclelland1985distributed}
McClelland, James \& David Rumelhart. 1985.
\newblock Distributed memory and the representation of general and specific
  information.
\newblock \emph{Journal of Experimental Psychology: General} 114(2). 159--197.

\bibitem[{Medin \& Schaffer(1978)}]{medin1978context}
Medin, Douglas \& Marguerite Schaffer. 1978.
\newblock Context theory of classification learning.
\newblock \emph{Psychological Review} 85(3). 207--238.

\bibitem[{Michelas(2011)}]{michelas2011caracterisation}
Michelas, Amandine. 2011.
\newblock \emph{Caract\'{e}risation phon\'{e}tique et phonologique du syntagme
  interm\'{e}diaire en fran\c{c}ais: {D}e la production \`{a} la perception}:
  Aix-en-Provence: Universit\'{e} de Provence dissertation.

\bibitem[{M{\"o}hler(2001)}]{mohler2001improvements}
M{\"o}hler, Gregor. 2001.
\newblock Improvements of the {PaIntE} model for f0 parametrization.
\newblock Research Papers from the Phonetics Lab, IMS Universit\"{a}t
  Stuttgart.

\bibitem[{M{\"o}hler \& Conkie(1998)}]{mohler1998parametric}
M{\"o}hler, Gregor \& Alistair Conkie. 1998.
\newblock Parametric modeling of intonation using vector quantization.
\newblock In \emph{Proceedings of the 3rd {ESCA Workshop on Speech Synthesis}},
  311--316. Jenolan Caves.

\bibitem[{Moulines \& Charpentier(1990)}]{moulines1990pitchsyncronous}
Moulines, Eric \& Francis Charpentier. 1990.
\newblock Pitch-synchronous waveform processing techniques for text-to-speech
  synthesis using diphones.
\newblock \emph{Speech Communication} 9(5). 453--467.

\bibitem[{Nash \& Mulac(1980)}]{nash1980intonation}
Nash, Rose \& Anthony Mulac. 1980.
\newblock The intonation of verifiability.
\newblock In Linda Waugh \& Cornelis {van Schoonevenld} (eds.), \emph{The
  melody of language: {I}ntonation and prosody}, 219--242. Baltimore:
  University Park Press.

\bibitem[{Neukom(1995)}]{neukom1995description}
Neukom, Lukas. 1995.
\newblock \emph{Description grammaticale du nateni ({B}{\'e}nin): {S}yst{\`e}me
  verbal, classification nominale, phrases complexes, textes}.
\newblock Z\"{u}rich: Universit\"{a}t Z\"{u}rich.

\bibitem[{Nguyen et~al.(2009)Nguyen, Wauquier \& Tuller}]{nguyen2009dynamical}
Nguyen, Noël, Sophie Wauquier \& Betty Tuller. 2009.
\newblock The dynamical approach to speech perception: {F}rom fine phonetic
  detail to abstract phonological categories.
\newblock In François Pellegrino, Egidio Marsico, Ioana Chitoran \& Christophe
  Coup{\'e} (eds.), \emph{Approaches to phonological complexity}, 193--217.
  Berlin: Mouton de Gruyter.

\enlargethispage{\baselineskip}
\bibitem[{Niebuhr(2007)}]{niebuhr2007categorical}
Niebuhr, Oliver. 2007.
\newblock Categorical perception in intonation: {A} matter of signal dynamics?
\newblock In Cyril Goutte, Nicola Cancedda, Marc Dymetman \& George Foster
  (eds.), \emph{Proceedings of the 8th {Annual Conference of the International
  Speech Communication Association}}, 642--645. Antwerp.

\bibitem[{Niebuhr \& Ambrazaitis(2006)}]{niebuhr2006alignment}
Niebuhr, Oliver \& Gilbert Ambrazaitis. 2006.
\newblock Alignment of medial and late peaks in {G}erman spontaneous speech.
\newblock In Rüdiger Hoffmann \& Hansjörg Mixdorff (eds.), \emph{Proceedings
  of the 3rd {I}nternational {C}onference on {S}peech {P}rosody}, 161--164.
  Dresden.

\bibitem[{Niebuhr et~al.(2011)Niebuhr, D'Imperio, Gili~Fivela \&
  Cangemi}]{niebuhr2011shapers}
Niebuhr, Oliver, Mariapaola D'Imperio, Barbara Gili~Fivela \& Francesco
  Cangemi. 2011.
\newblock Are there ``shapers'' and ``aligners''? {I}ndividual differences in
  signalling pitch accent category.
\newblock In Wai-Sum Lee \& Eric Zee (eds.), \emph{Proceedings of the 17th
  {I}nternational {C}ongress of {P}honetic {S}ciences}, 120--123. Hong Kong:
  City University of Hong Kong.

\bibitem[{Niebuhr \& Pfitzinger(2010)}]{niebuhr2010pitchaccent}
Niebuhr, Oliver \& Hartmut Pfitzinger. 2010.
\newblock On pitch-accent identification: {T}he role of syllable duration and
  intensity.
\newblock In \emph{Proceedings of the 5th {I}nternational {C}onference on
  {S}peech {P}rosody}, Chicago.

\bibitem[{Nosofsky(1986)}]{nosofsky1986attention}
Nosofsky, Robert. 1986.
\newblock Attention, similarity, and the identification--categorization
  relationship.
\newblock \emph{Journal of Experimental Psychology: General} 115(1). 39--61.

\bibitem[{Nosofsky(1988)}]{nosofsky1988exemplar}
Nosofsky, Robert. 1988.
\newblock Exemplar-based accounts of relations between classification,
  recognition, and typicality.
\newblock \emph{Journal of Experimental Psychology: Learning, Memory, and
  Cognition} 14(4). 700--708.

\bibitem[{Noth et~al.(2000)Noth, Batliner, Kie\ss{}ling, Kompe \&
  Niemann}]{noth2000verbmobil}
Noth, Elmar, Anton Batliner, Andreas Kie\ss{}ling, Ralf Kompe \& Heinrich
  Niemann. 2000.
\newblock Verbmobil: {T}he use of prosody in the linguistic components of a
  speech understanding system.
\newblock \emph{Speech and Audio Processing} 8(5). 519--532.

\bibitem[{{O'Dell} \& Port(1983)}]{odell1983discrimination}
{O'Dell}, Michael \& Robert Port. 1983.
\newblock Discrimination of word-final voicing in {G}erman.
\newblock \emph{Journal of the Acoustical Society of America} 73. S31.

\bibitem[{Ohala(1983)}]{ohala1983cross}
Ohala, John. 1983.
\newblock Cross-language use of pitch: {A}n ethological view.
\newblock \emph{Phonetica} 40(1). 1--18.

\bibitem[{Ohala(1984)}]{ohala1984ethological}
Ohala, John. 1984.
\newblock An ethological perspective on common cross-language utilization of f0
  of voice.
\newblock \emph{Phonetica} 41(1). 1--16.

\bibitem[{Ohala(1990)}]{ohala1990interface}
Ohala, John. 1990.
\newblock There is no interface between phonology and phonetics: {A} personal
  view.
\newblock \emph{Journal of Phonetics} 18(2). 153--172.

\bibitem[{Oldfield(1966)}]{oldfield1966things}
Oldfield, Richard. 1966.
\newblock Things, words and the brain.
\newblock \emph{The Quarterly Journal of Experimental Psychology} 18(4).
  340--353.

%\enlargethispage{\baselineskip}
\bibitem[{Osgood et~al.(1957)Osgood, Suci \&
  Tannenbaum}]{osgood1957measurement}
Osgood, Charles, George Suci \& Percy Tannenbaum. 1957.
\newblock \emph{The measurement of meaning}.
\newblock Urbana: University of Illinois Press.

\bibitem[{Palmeri et~al.(1993)Palmeri, Goldinger \&
  Pisoni}]{palmeri1993episodic}
Palmeri, Thomas, Stephen Goldinger \& David Pisoni. 1993.
\newblock Episodic encoding of voice attributes and recognition memory for
  spoken words.
\newblock \emph{Journal of Experimental Psychology: Learning, Memory, and
  Cognition} 19(2). 309--328.

\bibitem[{Pandelaere \& Dewitte(2006)}]{pandelaere2006question}
Pandelaere, Mario \& Siegfried Dewitte. 2006.
\newblock Is this a question? {N}ot for long: {T}he statement bias.
\newblock \emph{Journal of Experimental Social Psychology} 42(4). 525--531.

\bibitem[{Peterson(1952)}]{peterson1952information}
Peterson, Gordon. 1952.
\newblock The information-bearing elements of speech.
\newblock \emph{Journal of the Acoustical Society of America} 24(6). 629--637.

\bibitem[{Peterson \& Barney(1952)}]{peterson1952control}
Peterson, Gordon \& Harold Barney. 1952.
\newblock Control methods used in a study of the vowels.
\newblock \emph{Journal of the Acoustical Society of America} 24(2). 175--184.

\bibitem[{Petrone(2008)}]{petrone2008role}
Petrone, Caterina. 2008.
\newblock \emph{Le r{\^o}le de la variabilit{\'e} phon{\'e}tique dans la
  repr{\'e}sentation des contours intonatifs et de leur sens}: Aix-en-Provence:
  Universit{\'e} de Provence dissertation.

\bibitem[{Petrone \& D'Imperio(2008)}]{petrone2008tonal}
Petrone, Caterina \& Mariapaola D'Imperio. 2008.
\newblock Tonal structure and constituency in {N}eapolitan {I}talian:
  {E}vidence for the accentual phrase in statements and questions.
\newblock In Plinio Barbosa, Sandra Madureira \& César Reis (eds.),
  \emph{Proceedings of 4th {I}nternational {C}onference on {S}peech {P}rosody},
  301--304. Campinas.

\bibitem[{Petrone \& D'Imperio(2009)}]{petrone2009tonal}
Petrone, Caterina \& Mariapaola D'Imperio. 2009.
\newblock Is tonal alignment interpretation independent of methodology?
\newblock In Maria Uther, Roger Moore \& Stephen Cox (eds.), \emph{Proceedings
  of the 10th {Annual Conference of the International Speech Communication
  Association}}, 2459--2462. Brighton.

\bibitem[{Petrone \& D'Imperio(2011)}]{petrone2011tones}
Petrone, Caterina \& Mariapaola D'Imperio. 2011.
\newblock From tones to tunes: {E}ffects of the f0 prenuclear region in the
  perception of {N}eapolitan statements and questions.
\newblock In Sónia Frota, Gorka Elordieta \& Pilar Prieto (eds.),
  \emph{Prosodic categories: {P}roduction, perception and comprehension},
  207--230. Dordrecht, Heidelberg, London, New York: Springer.

\bibitem[{Petrone \& Niebuhr(2014)}]{petrone2014intonation}
Petrone, Caterina \& Oliver Niebuhr. 2014.
\newblock On the intonation in {G}erman intonation questions: {T}he role of the
  prenuclear region.
\newblock \emph{Language and Speech} 57(1). 108--46.

\bibitem[{Pfitzinger(2001)}]{pfitzinger2001phonetische}
Pfitzinger, Hartmut. 2001.
\newblock \emph{Phonetische {A}nalyse der {S}prechgeschwindigkeit}: Munich:
  Ludwig-Maximilians-Universit{\"a}t M{\"u}nchen dissertation.

\bibitem[{Pierrehumbert(1980)}]{pierrehumbert1980phonology}
Pierrehumbert, Janet. 1980.
\newblock \emph{The phonology and phonetics of {E}nglish intonation}:
  Cambridge, MA: Massachussets Institut of Technology dissertation.

\bibitem[{Pierrehumbert(1981)}]{pierrehumbert1981synthesizing}
Pierrehumbert, Janet. 1981.
\newblock Synthesizing intonation.
\newblock \emph{Journal of the Acoustical Society of America} 70(4). 985--995.

\bibitem[{Pierrehumbert(1990)}]{pierrehumbert1990phonological}
Pierrehumbert, Janet. 1990.
\newblock Phonological and phonetic representation.
\newblock \emph{Journal of Phonetics} 18(3). 375--394.

%\enlargethispage{\baselineskip}
\bibitem[{Pierrehumbert(2001)}]{pierrehumbert2001exemplar}
Pierrehumbert, Janet. 2001.
\newblock Exemplar dynamics: {W}ord frequency, lenition and contrast.
\newblock In Joan Bybee \& Paul Hopper (eds.), \emph{Frequency and the
  emergence of linguistic structure}, 137--158. Amsterdam: John Benjamins.

\bibitem[{Pierrehumbert \& Beckman(1988)}]{pierrehumbert1988japanese}
Pierrehumbert, Janet \& Mary Beckman. 1988.
\newblock \emph{Japanese tone structure}.
\newblock Cambridge: MIT Press.

\bibitem[{Pierrehumbert et~al.(2000)Pierrehumbert, Beckman \&
  Ladd}]{pierrehumbert2000conceptual}
Pierrehumbert, Janet, Mary Beckman \& Robert Ladd. 2000.
\newblock Conceptual foundations of phonology as a laboratory science.
\newblock In Noel Burton-Roberts, Philip Carr \& Gerard Docherty (eds.),
  \emph{Phonological knowledge: {C}onceptual and empirical issues}, 273--303.
  Oxford: Oxford University Press.

\bibitem[{Pierrehumbert \& Hirschberg(1990)}]{pierrehumbert1990meaning}
Pierrehumbert, Janet \& Julia Hirschberg. 1990.
\newblock The meaning of intonational contours in the interpretation of
  discourse.
\newblock In Philip Cohen, Jerry Morgan \& Martha Pollack (eds.),
  \emph{Intentions in communication}, 271--311. Cambridge: MIT Press.

\bibitem[{Pierrehumbert \& Steele(1989)}]{pierrehumbert1989categories}
Pierrehumbert, Janet \& Shirley Steele. 1989.
\newblock Categories of tonal alignment in {E}nglish.
\newblock \emph{Phonetica} 46(3). 181--196.

\bibitem[{Pike(1945)}]{pike1945intonation}
Pike, Kenneth. 1945.
\newblock \emph{The intonation of {A}merican {E}nglish}.
\newblock Ann Arbor: University of Michigan Press.

\bibitem[{Podi(1995)}]{podi1995esquisse}
Podi, Napo. 1995.
\newblock \emph{Esquisse comparative de l'akasilimi et du basaal}: Grenoble:
  Universit\'{e} de Grenoble III dissertation.

\bibitem[{Port(1996)}]{port1996discreteness}
Port, Robert. 1996.
\newblock The discreteness of phonetic elements and formal linguistics:
  {R}esponse to {A}. {M}anaster {R}amer.
\newblock \emph{Journal of Phonetics} 24(4). 491--512.

\bibitem[{Port(2006)}]{port2006graphical}
Port, Robert. 2006.
\newblock The graphical basis of phones and phonemes.
\newblock In Murray Munro \& Ocke-Schwen Bohn (eds.), \emph{Second language
  speech learning: {T}he role of language experience in speech perception and
  production}, 349--365. Amsterdam: John Benjamins.

\bibitem[{Port \& Crawford(1989)}]{port1989incomplete}
Port, Robert \& Penny Crawford. 1989.
\newblock Incomplete neutralization and pragmatics in {G}erman.
\newblock \emph{Journal of Phonetics} 17(4). 257--282.

\bibitem[{Port et~al.(1981)Port, Mitleb \& O'Dell}]{port1981neutralization}
Port, Robert, Fares Mitleb \& Michael O'Dell. 1981.
\newblock Neutralization of obstruent voicing in {G}erman is incomplete.
\newblock \emph{Journal of the Acoustical Society of America} 70. S13.

\bibitem[{Port \& O'Dell(1985)}]{port1985neutralization}
Port, Robert \& Michael O'Dell. 1985.
\newblock Neutralization of syllable-final voicing in {G}erman.
\newblock \emph{Journal of Phonetics} 13(4). 455--471.

\bibitem[{Prieto et~al.(2005)Prieto, D'Imperio \&
  Gili~Fivela}]{prieto2005pitch}
Prieto, Pilar, Mariapaola D'Imperio \& Barbara Gili~Fivela. 2005.
\newblock Pitch accent alignment in {R}omance: {P}rimary and secondary
  associations with metrical structure.
\newblock \emph{Language and Speech} 48(4). 359--396.

\bibitem[{Pye(1986)}]{pye1986word}
Pye, Susan. 1986.
\newblock Word-final devoicing of obstruents in {R}ussian.
\newblock \emph{Cambridge Papers in Phonetics and Experimental Linguistics} 5.
  1--10.

\bibitem[{{R Development Core Team}(2008)}]{r2008r}
{R Development Core Team}. 2008.
\newblock R: {A} language and environment for statistical computing.
\newblock Computer program, retrieved from http://www.R-project.org/.

\bibitem[{Repp(1979)}]{repp1979relative}
Repp, Bruno. 1979.
\newblock Relative amplitude of aspiration noise as a voicing cue for
  syllable-initial stop consonants.
\newblock \emph{Language and Speech} 22(2). 173--189.

\bibitem[{Rialland(1984)}]{rialland1984fini}
Rialland, Annie. 1984.
\newblock Le fini/l'infini ou l'affirmation/l'interrogation en moba (langue
  volta{\"i}que parl{\'e}e au {N}ord-{T}ogo).
\newblock \emph{Studies in African Linguistics} supp. 9. 258--261.

\enlargethispage{2\baselineskip}
\bibitem[{Rialland(2007)}]{rialland2007question}
Rialland, Annie. 2007.
\newblock Question prosody: {A}n {A}frican perspective.
\newblock In Tomas Riad \& Carlos Gussenhoven (eds.), \emph{Tones and tunes:
  {E}xperimental studies in word and sentence prosody}, vol.~2, 35--62. Berlin:
  de Gruyter.

\bibitem[{Rietveld \& Gussenhoven(1987)}]{rietveld1987perceived}
Rietveld, Toni \& Carlos Gussenhoven. 1987.
\newblock Perceived speech rate and intonation.
\newblock \emph{Journal of Phonetics} 15(3). 273--285.

\bibitem[{Rossi(1971)}]{rossi1971seuil}
Rossi, Mario. 1971.
\newblock Le seuil de glissando ou seuil de perception des variations tonales
  pour les sons de la parole.
\newblock \emph{Phonetica} 23(1). 1--33.

\bibitem[{R{\"o}ttger et~al.(2011)R{\"o}ttger, Winter \&
  Grawunder}]{rottger2011robustness}
R{\"o}ttger, Timo, Bodo Winter \& Sven Grawunder. 2011.
\newblock The robustness of incomplete neutralization in {G}erman.
\newblock In Wai-Sum Lee \& Eric Zee (eds.), \emph{Proceedings of the 17th
  {I}nternational {C}ongress of {P}honetic {S}ciences}, 1722--1725. Hong Kong:
  City University of Hong Kong.

\bibitem[{Ryalls et~al.(1994)Ryalls, Le~Dorze, Lever, Ouellet \&
  Larfeuil}]{ryalls1994effects}
Ryalls, John, Guylaine Le~Dorze, Nathalie Lever, Lisa Ouellet \& Céline
  Larfeuil. 1994.
\newblock The effects of age and sex on speech intonation and duration for
  matched statements and questions in {F}rench.
\newblock \emph{Journal of the Acoustical Society of America} 95(4).
  2274--2276.

\bibitem[{Sabatini(1985)}]{sabatini1985italiano}
Sabatini, Francesco. 1985.
\newblock L'italiano dell'uso medio: {U}na realt{\`a} tra le variet{\`a}
  linguistiche italiane.
\newblock In Günter Holtus \& Edgar Radtke (eds.), \emph{Gesprochenes
  {I}talienisch in {G}eschichte und {G}egenwart}, 154--184. T{\"u}bingen:
  Gunter Narr.

\bibitem[{Sadock \& Zwicky(1985)}]{sadock1985speech}
Sadock, Jerrold \& Arnold Zwicky. 1985.
\newblock Speech act distinctions in syntax.
\newblock In Timothy Shopen (ed.), \emph{Language typology and syntactic
  description}, 155--196. Cambridge: Cambridge University Press.

\bibitem[{Savino(1997)}]{savino1997ruolo}
Savino, Michelina. 1997.
\newblock \emph{Il ruolo dell'intonazione nell'interazione comunicativa:
  {A}nalisi strumentale delle domande polari in un corpus di dialoghi spontanei
  (varieta' di {B}ari)}: Bari: Universit{\`a}/{P}olitecnico di {B}ari
  dissertation.

\bibitem[{Savino(2012)}]{savino2012intonation}
Savino, Michelina. 2012.
\newblock The intonation of polar questions in {I}talian: {W}here is the rise?
\newblock \emph{Journal of the International Phonetic Association} 42. 23--48.

\bibitem[{Savy \& Cutugno(2009)}]{savy2009clips}
Savy, Renata \& Francesco Cutugno. 2009.
\newblock {CLIPS: D}iatopic, diamesic and diaphasic variations in spoken
  {I}talian.
\newblock In \emph{Proceedings of the 5th {Corpus Linguistics Conference}},
  Liverpool.

\bibitem[{Schacter \& Church(1992)}]{schacter1992auditory}
Schacter, Daniel \& Barbara Church. 1992.
\newblock Auditory priming: {I}mplicit and explicit memory for words and
  voices.
\newblock \emph{Journal of Experimental Psychology: Learning, Memory, and
  Cognition} 18(5). 915--930.

\bibitem[{Scherer et~al.(1984)Scherer, Ladd \& Silverman}]{scherer1984vocal}
Scherer, Klaus, Robert Ladd \& Kim Silverman. 1984.
\newblock Vocal cues to speaker affect: {T}esting two models.
\newblock \emph{Journal of the Acoustical Society of America} 76(5).
  1346--1356.

\bibitem[{Schouten(1985)}]{schouten1985identification}
Schouten, Marten Egbertus~Hendrik. 1985.
\newblock Identification and discrimination of sweep tones.
\newblock \emph{Attention, Perception, \& Psychophysics} 37(4). 369--376.

\enlargethispage{\baselineskip}
\bibitem[{Schweitzer et~al.(2010{\natexlab{a}})Schweitzer, Calhoun,
  Sch{\"u}tze, Schweitzer \& Walsh}]{schweitzer2010relative}
Schweitzer, Katrin, Sasha Calhoun, Hinrich Sch{\"u}tze, Antje Schweitzer \&
  Michael Walsh. 2010{\natexlab{a}}.
\newblock Relative frequency affects pitch accent realisation: {E}vidence for
  exemplar storage of prosody.
\newblock In Marija Tabain, Janet Fletcher, David Grayden, John Hajek \& Andy
  Butcher (eds.), \emph{Proceedings of the 13th {Australasian International
  Conference on Speech Science and Technology}}, 62--65. Melbourne.

\bibitem[{Schweitzer et~al.(2011)Schweitzer, Walsh, Calhoun \&
  Sch{\"u}tze}]{schweitzer2011prosodic}
Schweitzer, Katrin, Michael Walsh, Sasha Calhoun \& Hinrich Sch{\"u}tze. 2011.
\newblock Prosodic variability in lexical sequences: {I}ntonation entrenches
  too.
\newblock In Wai-Sum Lee \& Eric Zee (eds.), \emph{Proceedings of the 17th
  {I}nternational {C}ongress of {P}honetic {S}ciences}, 1778--1781. Hong Kong:
  City University of Hong Kong.

\bibitem[{Schweitzer et~al.(2009)Schweitzer, Walsh, M{\"o}bius, Riester,
  Schweitzer \& Sch{\"u}tze}]{schweitzer2009frequency}
Schweitzer, Katrin, Michael Walsh, Bernd M{\"o}bius, Arndt Riester, Antje
  Schweitzer \& Hinrich Sch{\"u}tze. 2009.
\newblock Frequency matters: {P}itch accents and information status.
\newblock In Diana McCarthy \& Shuly Wintner (eds.), \emph{Proceedings of the
  12th {Conference of the European Chapter of the Association for Computational
  Linguistics}}, 728--736. Athens.

\bibitem[{Schweitzer et~al.(2010{\natexlab{b}})Schweitzer, Walsh, M{\"o}bius \&
  Sch{\"u}tze}]{schweitzer2010frequency}
Schweitzer, Katrin, Michael Walsh, Bernd M{\"o}bius \& Hinrich Sch{\"u}tze.
  2010{\natexlab{b}}.
\newblock Frequency of occurrence effects on pitch accent realisation.
\newblock In Takao Kobayashi, Keikichi Hirose \& Satoshi Nakamura (eds.),
  \emph{Proceedings of the 11th {Annual Conference of the International Speech
  Communication Association}}, 138--141. Makuhari.

\bibitem[{Sekiguchi(2006)}]{sekiguchi2006effects}
Sekiguchi, Takahiro. 2006.
\newblock Effects of lexical prosody and word familiarity on lexical access of
  spoken {J}apanese words.
\newblock \emph{Journal of Psycholinguistic Research} 35(4). 369--384.

\bibitem[{Sergeant \& Harris(1962)}]{sergeant1962sensitivity}
Sergeant, Russell \& Donald Harris. 1962.
\newblock Sensitivity to unidirectional frequency modulation.
\newblock \emph{Journal of the Acoustical Society of America} 34(10).
  1625--1628.

\bibitem[{Shattuck-Hufnagel \& Turk(1996)}]{shattuck1996prosody}
Shattuck-Hufnagel, Stefanie \& Alice Turk. 1996.
\newblock A prosody tutorial for investigators of auditory sentence processing.
\newblock \emph{Journal of Psycholinguistic Research} 25(2). 193--247.

\bibitem[{Sheffert et~al.(2002)Sheffert, Pisoni, Fellowes \&
  Remez}]{sheffert2002learning}
Sheffert, Sonya, David Pisoni, Jennifer Fellowes \& Robert Remez. 2002.
\newblock Learning to recognize talkers from natural, sinewave, and reversed
  speech samples.
\newblock \emph{Journal of Experimental Psychology: Human Perception and
  Performance} 28(6). 1447--1469.

\bibitem[{Shriberg et~al.(1998)Shriberg, Stolcke, Jurafsky, Coccaro, Meteer,
  Bates, Taylor, Ries, Martin \& Van Ess-Dykema}]{shriberg1998can}
Shriberg, Elizabeth, Andreas Stolcke, Daniel Jurafsky, Noah Coccaro, Marie
  Meteer, Rebecca Bates, Paul Taylor, Klaus Ries, Rachel Martin \& Carol Van
  Ess-Dykema. 1998.
\newblock Can prosody aid the automatic classification of dialog acts in
  conversational speech?
\newblock \emph{Language and Speech} 41(3-4). 443--492.

%\enlargethispage{\baselineskip}
\bibitem[{Slowiaczek \& Dinnsen(1985)}]{slowiaczek1985neutralizing}
Slowiaczek, Louisa \& Daniel Dinnsen. 1985.
\newblock On the neutralizing status of {P}olish word-final devoicing.
\newblock \emph{Journal of Phonetics} 13(3). 325--341.

\bibitem[{Slowiaczek \& Szymanska(1989)}]{slowiaczek1989perception}
Slowiaczek, Louisa \& Helena Szymanska. 1989.
\newblock Perception of word-final devoicing in {P}olish.
\newblock \emph{Journal of Phonetics} 17(3). 205--212.

\bibitem[{Smith(2002)}]{smith2002prosodic}
Smith, Caroline. 2002.
\newblock Prosodic finality and sentence type in {F}rench.
\newblock \emph{Language and Speech} 45(2). 141--178.

\bibitem[{Smith \& Medin(1981)}]{smith1981categories}
Smith, Edward \& Douglas Medin. 1981.
\newblock \emph{Categories and concepts}.
\newblock Cambridge: Harvard University Press.

\bibitem[{Smith et~al.(2012)Smith, Baker \& Hawkins}]{smith2012phonetic}
Smith, Rachel, Rachel Baker \& Sarah Hawkins. 2012.
\newblock Phonetic detail that distinguishes prefixed from pseudo-prefixed
  words.
\newblock \emph{Journal of Phonetics} 40(5). 689--705.

\bibitem[{Smith \& Hawkins(2000)}]{smith2000allophonic}
Smith, Rachel \& Sarah Hawkins. 2000.
\newblock Allophonic influences on word-spotting experiments.
\newblock In Anne Cutler, James McQueen \& Rian Zondervan (eds.), \emph{{ISCA
  Tutorial and Research Workshop on Spoken Word Access Processes}}, 139--142.
  Nijmegen: Max-Planck-Gesellschaft zur Förderung der Wissenschaften.

\bibitem[{Sobrero(1992)}]{sobrero1992italiano}
Sobrero, Alberto. 1992.
\newblock \emph{L'italiano di oggi}.
\newblock Roma: Istituto della Enciclopedia Italiana.

\bibitem[{Standing et~al.(1970)Standing, Conezio \&
  Haber}]{standing1970perception}
Standing, Lionel, Jerry Conezio \& Ralph Haber. 1970.
\newblock Perception and memory for pictures: {S}ingle-trial learning of 2500
  visual stimuli.
\newblock \emph{Psychonomic Science} 19(2). 73--74.

\bibitem[{Stevens(1960)}]{stevens1960model}
Stevens, Kenneth. 1960.
\newblock Toward a model for speech recognition.
\newblock \emph{Journal of the Acoustical Society of America} 32(1). 47--55.

\bibitem[{Stevens(2004)}]{stevens2004invariance}
Stevens, Kenneth. 2004.
\newblock Invariance and variability in speech: {I}nterpreting acoustic
  evidence.
\newblock In Janet Slifka, Sharon Manuel \& Melanie Matthies (eds.),
  \emph{Proceedings of {From Sound to Sense Workshop}}, vol.~B, 77--85.
  Cambridge: MIT Press.

\bibitem[{Swerts et~al.(1999)Swerts, Avesani \&
  Krahmer}]{swerts1999reaccentuation}
Swerts, Marc, Cinzia Avesani \& Emiel Krahmer. 1999.
\newblock Reaccentuation or deaccentuation: {A} comparative study of {I}talian
  and {D}utch.
\newblock In John Ohala (ed.), \emph{Proceedings of the 14th {I}nternational
  {C}ongress of {P}honetic {S}ciences}, 1541--144. San Francisco: University of
  California.

\bibitem[{{'t Hart}(1976)}]{thart1976psychoacoustic}
{'t Hart}, Johan. 1976.
\newblock Psychoacoustic backgrounds of pitch contour stylisation.
\newblock \emph{IPO -- Annual Progress Report} 11. 11--19.

\bibitem[{{'t Hart} et~al.(1990){'t Hart}, Collier \&
  Cohen}]{tHart1990perceptual}
{'t Hart}, Johan, Rene Collier \& Antonie Cohen. 1990.
\newblock \emph{A perceptual study of intonation: {A}n experimental-phonetic
  approach}.
\newblock Cambridge: Cambridge University Press.

\bibitem[{Taljaard \& Bosch(1988)}]{taljaard1988handbook}
Taljaard, Petrus \& Sonja Bosch. 1988.
\newblock \emph{Handbook of {I}sizulu}.
\newblock Hatfield, Pretoria: J.L. van Schaik.

\bibitem[{Tenpenny(1995)}]{tenpenny1995abstractionist}
Tenpenny, Patricia. 1995.
\newblock Abstractionist versus episodic theories of repetition priming and
  word identification.
\newblock \emph{Psychonomic Bulletin \& Review} 2(3). 339--363.

\bibitem[{Theodore(2009)}]{theodore2009characteristics}
Theodore, Rachel. 2009.
\newblock \emph{Some characteristics of talker-specific phonetic detail}:
  Boston: Northeastern University dissertation.

\bibitem[{Trager \& Smith(1951)}]{trager1951outline}
Trager, Leonard \& Henry Smith. 1951.
\newblock \emph{An outline of {E}nglish structure}.
\newblock Norman: Battenburg Press.

\bibitem[{Trouvain(2004)}]{trouvain2004tempo}
Trouvain, Jürgen. 2004.
\newblock \emph{Tempo variation in speech production: {I}mplications for speech
  synthesis}: Saarbr{\"u}cken: Saarland University dissertation.

\bibitem[{Trubetzkoy(1939)}]{trubeckoj1939grundzuege}
Trubetzkoy, Nikolaus. 1939.
\newblock \emph{Grundz\"{u}ge der {P}honologie}.
\newblock Prag: Travaux du cercle linguistique de Prague.

\bibitem[{Turk et~al.(2006)Turk, Nakai \& Sugahara}]{turk2006acoustic}
Turk, Alice, Satsuki Nakai \& Mariko Sugahara. 2006.
\newblock Acoustic segment durations in prosodic research: {A} practical guide.
\newblock In Stefan Sudhoff, Denisa Lenertov\'{a}, Roland Meyer, Sandra
  Pappert, Petra Augurzky, Ina Mleinek, Nicole Richter \& Johannes
  Schlie\ss{}er (eds.), \emph{Methods in empirical prosody research}, 1--28.
  Berlin: De Gruyter.

\bibitem[{Uldall(1964)}]{uldall1964dimensions}
Uldall, Elizabeth. 1964.
\newblock Dimensions of meaning in intonation.
\newblock In David Abercrombie (ed.), \emph{In honour of {D}aniel {J}ones},
  271--279. London: Longmans.

\bibitem[{{van Alphen} \& McQueen(2006)}]{vanalphen2006effect}
{van Alphen}, Petra \& James McQueen. 2006.
\newblock The effect of voice onset time differences on lexical access in
  {D}utch.
\newblock \emph{Journal of Experimental Psychology: Human Perception and
  Performance} 32(1). 178--196.

\bibitem[{{van Heerden} \& Barnard(2007)}]{vanheerden2007speech}
{van Heerden}, Charl~Johannes \& Etienne Barnard. 2007.
\newblock Speech rate normalization used to improve speaker verification.
\newblock \emph{South African Institute of Electrical Engineers} 98(4).
  136--140.

\bibitem[{{van Heuven} \& Haan(2000)}]{vanheuven2000phonetic}
{van Heuven}, Vincent \& Judith Haan. 2000.
\newblock Phonetic correlates of statement versus question intonation in
  {D}utch.
\newblock In Antonis Botinis (ed.), \emph{Intonation: {A}nalysis, modelling and
  technology}, 119--144. Dordrecht: Kluwer.

\bibitem[{{van Heuven} \& Haan(2002)}]{vanheuven2002temporal}
{van Heuven}, Vincent \& Judith Haan. 2002.
\newblock Temporal distribution of interrogativity markers in {D}utch: {A}
  perceptual study.
\newblock In Carlos Gussenhoven \& Natasha Warner (eds.), \emph{Papers in
  {L}aboratory {P}honology}, vol.~7, 61--86. Berlin: Mouton de Gruyter.

\bibitem[{{van Heuven} \& {van Zanten}(2005)}]{vanheuven2005speech}
{van Heuven}, Vincent \& Ellen {van Zanten}. 2005.
\newblock Speech rate as a secondary prosodic characteristic of polarity
  questions in three languages.
\newblock \emph{Speech Communication} 47(1). 87--99.

\bibitem[{{van Santen} \& M{\"o}bius(2000)}]{vansanten2000quantitative}
{van Santen}, Jan \& Bernd M{\"o}bius. 2000.
\newblock A quantitative model of f0 generation and alignment.
\newblock In Antonis Botinis (ed.), \emph{Intonation: {A}nalysis, modelling and
  technology}, 269--288. Dordrecht: Kluwer.

\bibitem[{Walsh et~al.(2008)Walsh, Schweitzer, M{\"o}bius \&
  Sch{\"u}tze}]{walsh2008examining}
Walsh, Michael, Katrin Schweitzer, Bernd M{\"o}bius \& Hinrich Sch{\"u}tze.
  2008.
\newblock Examining pitch-accent variability from an exemplar-theoretic
  perspective.
\newblock In Janet Fletcher, Deborah Loakes, Roland Gocke, Denis Burnham \&
  Michael Wagner (eds.), \emph{Proceedings of the 9th {Annual Conference of the
  International Speech Communication Association}}, 877--880. Brisbane.

\newpage
\bibitem[{Warner et~al.(2006)Warner, Good, Jongman \&
  Sereno}]{warner2006orthographic}
Warner, Natasha, Erin Good, Allard Jongman \& Joan Sereno. 2006.
\newblock Orthographic vs. morphological incomplete neutralization effects.
\newblock \emph{Journal of Phonetics} 34(2). 285--293.

\bibitem[{Warner et~al.(2004)Warner, Jongman, Sereno \&
  Kemps}]{warner2004incomplete}
Warner, Natasha, Allard Jongman, Joan Sereno \& Rachèl Kemps. 2004.
\newblock Incomplete neutralization and other sub-phonemic durational
  differences in production and perception: {E}vidence from {D}utch.
\newblock \emph{Journal of Phonetics} 32(2). 251--276.

\bibitem[{Wells(1945)}]{wells1945pitch}
Wells, Rulon. 1945.
\newblock The pitch phonemes of {E}nglish.
\newblock \emph{Language} 21. 27--39.

\bibitem[{West(1999)}]{west1999perception}
West, Paula. 1999.
\newblock Perception of distributed coarticulatory properties of english /l/
  and /r/.
\newblock \emph{Journal of Phonetics} 27(4). 405--426.

\bibitem[{Wheeler(1970)}]{wheeler1970processes}
Wheeler, Daniel. 1970.
\newblock Processes in word recognition.
\newblock \emph{Cognitive Psychology} 1(1). 59--85.

\bibitem[{Williams \& Stevens(1972)}]{williams1972emotions}
Williams, Carl \& Kenneth Stevens. 1972.
\newblock Emotions and speech: {S}ome acoustical correlates.
\newblock \emph{Journal of the Acoustical Society of America} 52(4).
  1238--1250.

\bibitem[{Winter \& R{\"o}ttger(forthcoming)}]{winterFORTHnature}
Winter, Bodo \& Timo R{\"o}ttger. forthcoming.
\newblock The nature of incomplete neutralization in {G}erman: {I}mplications
  for laboratory phonology.
\newblock \emph{Grazer Linguistische Studien} .

\bibitem[{Wood(1973)}]{wood1973speech}
Wood, Sidney. 1973.
\newblock Speech tempo.
\newblock In \emph{Working papers}, vol.~9, 99--147. Lund: Department of
  Linguistics, Lund University.

\bibitem[{Xu(2005)}]{xu2005speech}
Xu, Yi. 2005.
\newblock Speech melody as articulatorily implemented communicative functions.
\newblock \emph{Speech Communication} 46(3). 220--251.

\bibitem[{Zeng et~al.(2004)Zeng, Martin \& Boulakia}]{zeng2004tones}
Zeng, XiaoLi, Philippe Martin \& Georges Boulakia. 2004.
\newblock Tones and intonation in declarative and interrogative sentences in
  {M}andarin.
\newblock In \emph{Proceedings of the {International Symposium on Tonal Aspects
  of Languages}}, 235--238. Beijing.

\end{thebibliography}



\clearpage

\phantomsection%this allows hyperlink in ToC to work
\addcontentsline{toc}{chapter}{Name index}
\ohead{Name index}
% \pdfbookmark[0]{Index}{Index}

%\pdfbookmark[1]{Name index}{Name index}
\printindex[aut]

% 
% \phantomsection%this allows hyperlink in ToC to work
% \addcontentsline{toc}{chapter}{Language index}
% \ohead{Language index}
%\pdfbookmark[1]{Language index}{Language index}
% \printindex[lan]


\phantomsection%this allows hyperlink in ToC to work
\addcontentsline{toc}{chapter}{Subject index}
\ohead{Subject index}
%\pdfbookmark[1]{Subject index}{Subject index}
\printindex
\end{document}
