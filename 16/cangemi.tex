\documentclass[ number=1
,series=labphon
,output=long
% ,draftmode
,url=http://langsci-press.org/catalog/book/16
,isbn=978-3-944675-01-5
]{LSP/langsci}
\usepackage{graphicx}	% image insertion
\usepackage{float}	% figure positioning
\usepackage{lscape}	% table/figure rotation
\usepackage{multirow}	% specific kind of tables
\usepackage[british,italian]{babel}	% hyphenation
\usepackage{LSP/lsp-styles/lsp-gb4e}	% leipzig glosses
\usepackage{setspace}	% and following: epigraphs
\renewcommand\epigraph[4]{
\vspace{1em}\hfill{}\begin{minipage}{#1}{\begin{spacing}{0.9}
\noindent\textit{#2}\end{spacing}
\vspace{1em}\small\hfill{}{#3}\\
\vspace{-2em}\begin{flushright}{#4}\end{flushright}}\vspace{2em}
\end{minipage}}
\title{Prosodic detail in Neapolitan Italian}
\author{Francesco Cangemi}
\typesetter{Francesco Cangemi}
\proofreader{Tom Gardner, Martin Hilpert, Michelle Natolo, Stephanie Natolo, Benedikt Singpiel, Siri Tuttle, Tamara Schmidt}
\BackTitle{Prosodic detail in Neapolitan Italian}
\BackBody{Recent findings on phonetic detail have been taken as supporting exemplar-based approaches to prosody. Through four experiments on both production and perception of both melodic and temporal detail in Neapolitan Italian, we show that prosodic detail is not incompatible with abstractionist approaches either. Specifically, we suggest that the exploration of prosodic detail leads to a refined understanding of the relationships between the richly specified and continuously varying phonetic information on one side, and coarse phonologically structured contrasts on the other, thus offering insights on how pragmatic information is conveyed by prosody.}
\lccode`\'=`\'
\hyphenation{
D'Im-pe-rio
}
\usepackage{booktabs}
\usepackage{floatrow}
\floatsetup[table]{capposition=top}
\newcommand{\mytoprule}{\midrule\toprule}
\newcommand{\mybottomrule}{\bottomrule\midrule}
\begin{document}
\selectlanguage{british}
\maketitle
\tableofcontents\enlargethispage{2em}
\mainmatter
\documentclass[output=paper,colorlinks,citecolor=brown]{langscibook}
\ChapterDOI{10.5281/zenodo.15682184}
\title{Universality of semantic frames versus specificity of conceptual frames} 
\author{Svetla Koeva\orcid{0000-0001-5947-8736}\affiliation{Department of Computational Linguistics, Institute for Bulgarian Language, Bulgarian Academy of Sciences}}

\abstract{FrameNet is a semantic network that links semantic frames, each evoked by a set of lexical units and consisting of frame elements (with semantic types, definitions and relations) that outline the semantic structure of the frame, as well as frame-to-frame relations and annotations that illustrate the syntactic realisation of the frame elements.

The Bulgarian FrameNet is based on the FrameNet and at the same time offers the possibility to encode language-specific semantic structures, either by replicating or reconstructing existing semantic frames or by introducing new frames. An abstract representation, called \emph{superframe}, is developed to replicate language-independent information (at least for English and Bulgarian) from semantic frames. In the Bulgarian FrameNet, the \emph{conceptual frames} inherit either all or part of the language-independent information from the semantic frames via the superframes and may contain additional language-specific data to represent scenarios evoked by the Bulgarian lexical units. Each conceptual frame is extended by a set of nouns that represent the lexical realisations of the frame elements corresponding to the target lexical units.

The study presents the FrameNet, the creation of FrameNets for other languages, the motivation for introducing the conceptual frames and the superframes, which combine the semantic and conceptual frames in a \emph{multilingual network}, and the structure of the Bulgarian FrameNet, which includes Lexical, Grammatical, Frame and Syntactic sections (valence patterns). 

The overall aim is to present our approach to the identification and transfer of \emph{language-universal} knowledge from the FrameNet semantic frames, universal in the sense that it applies to both English and Bulgarian, and the definition and integration of \emph{language-specific components} of the conceptual frames for Bulgarian (as compared to English).
}

\IfFileExists{../localcommands.tex}{
   \addbibresource{../localbibliography.bib}
   \usepackage{langsci-optional}
\usepackage{langsci-gb4e}
\usepackage{langsci-lgr}

\usepackage{listings}
\lstset{basicstyle=\ttfamily,tabsize=2,breaklines=true}

%added by author
% \usepackage{tipa}
\usepackage{multirow}
\graphicspath{{figures/}}
\usepackage{langsci-branding}

   
\newcommand{\sent}{\enumsentence}
\newcommand{\sents}{\eenumsentence}
\let\citeasnoun\citet

\renewcommand{\lsCoverTitleFont}[1]{\sffamily\addfontfeatures{Scale=MatchUppercase}\fontsize{44pt}{16mm}\selectfont #1}
  
   %% hyphenation points for line breaks
%% Normally, automatic hyphenation in LaTeX is very good
%% If a word is mis-hyphenated, add it to this file
%%
%% add information to TeX file before \begin{document} with:
%% %% hyphenation points for line breaks
%% Normally, automatic hyphenation in LaTeX is very good
%% If a word is mis-hyphenated, add it to this file
%%
%% add information to TeX file before \begin{document} with:
%% \include{localhyphenation}
\hyphenation{
affri-ca-te
affri-ca-tes
an-no-tated
com-ple-ments
com-po-si-tio-na-li-ty
non-com-po-si-tio-na-li-ty
Gon-zá-lez
out-side
Ri-chárd
se-man-tics
STREU-SLE
Tie-de-mann
}
\hyphenation{
affri-ca-te
affri-ca-tes
an-no-tated
com-ple-ments
com-po-si-tio-na-li-ty
non-com-po-si-tio-na-li-ty
Gon-zá-lez
out-side
Ri-chárd
se-man-tics
STREU-SLE
Tie-de-mann
}
   \boolfalse{bookcompile}
   \togglepaper[1]%%chapternumber
}{}

\begin{document}
\maketitle

\section{Introduction} 

FrameNet is a semantic network that links semantic frames, each evoked by a set of lexical units and consisting of frame elements (with semantic types, definitions and relations) that outline the semantic structure of the frame \citep{Fillmore2003,fillmore2010}. It also includes frame-to-frame relations and contains syntactic and semantic annotations of examples that illustrate the syntactic realisation of frame elements.

The study presents the structure of the Bulgarian FrameNet, which is based on two basic principles: maintaining consistency with FrameNet and providing a mechanism for encoding semantic structures that either replicate or reconstruct the existing semantic frames or are completely new. To achieve this, an abstract level of representation, the \emph{superframe}, is introduced, which contains the \emph{language-independent information} inherited from the semantic frames. To represent the semantic structure of Bulgarian lexical units that evoke the same situation, property or process, an abstract structure, the \emph{conceptual frame}, is introduced, which is influenced by the semantic frames of FrameNet. The conceptual frame:

\begin{itemize}
\item applies only to lexical units described by the same set of core frame elements, which in turn have the same syntactic realisation and lexical compatibility;
\item is extended by nouns that can form semantically valid phrases with the verbal lexical units that evoke the frame.
\end{itemize}

A superframe is linked to exactly one semantic frame, while a superframe can be connected to one or more conceptual frames. Three models of correspondence between a conceptual frame and a superframe are described: (a) equivalence, (b) partial equivalence and (c) no equivalence. The superframe is introduced to ensure alignment with language-independent information from FrameNet that is valid for at least two languages, English and Bulgarian, while conceptual frames are used to delineate semantic and syntactic differences in conceptual descriptions of Bulgarian lexical units. This representation enables the integration of Bulgarian into a global network that captures both unique semantic and syn- tactic features of individual languages as well as language-independent features that may apply to a large group of languages.

In the following sections, we present the structure of FrameNet, the creation of FrameNets for other languages, the motivation for the introduction of superframes and conceptual frames, and the relations between the two abstract structures. This is followed by an overview of the structure of the Bulgarian FrameNet, which comprises four sections: Lexical, Grammatical, Frame and Syntactic, all of which are integrated into a web-based data management system called BulFrame \citep{koeva-doychev-2022-ontology}. This system facilitates the manual evaluation and visualisation of the Bulgarian FrameNet.\footnote{https://dcl.bas.bg/bulframe/}

We describe the components that make up the Lexical, Grammatical, Frame, and Syntactic section of each conceptual frame and present their components, sources and associated data. The Bulgarian lexical units are provided with additional grammatical, lexical and semantic information. The frame elements in the Bulgarian FrameNet are associated to nouns that are suitable for collocations with verbal lexical units that evoke the corresponding frame. Based on annotated examples, each frame element is linked to the relevant syntactic categories, grammatical roles and labels for implicit use related with its lexical representation.

The contributions of the study are as follows: (a) formulation of an abstract structure, the superframe, to connect the semantic and conceptual frames in a cross-linguistic network; (b) identification of language-independent knowledge (for at least two languages, in our case for English and Bulgarian) in the semantic frames of FrameNet for transfer to the superframes; (c) definition of conceptual frames based on the structure of the semantic frames of FrameNet and their extension with components containing additional lexical and grammatical information; (d) associating the conceptual frame elements with sets of nouns that can be collocated contextually with the lexical units evoking the frame; and (e) developing the network of conceptual frames valid for Bulgarian, containing both language-independent information from the corresponding semantic frames and language-specific information for Bulgarian.

The BulFrame system for editing, evaluating and visualising data as well as the results of the annotation are presented in various studies, e.g. in \citet{skoeva2024} and in \citet{koeva-doychev-2022-ontology} and the other contributions in this volume.

\section{Semantic and syntactic representations in FrameNet}
FrameNet is based on the theory of Frame semantics \citep{fillmore1982frame,fillmore1976frame,fillmore1976frame,fillmore1985frames,Fillmore+2006+373+400,Fillmore+2008,fillmore2010}, going beyond the general semantic roles of Case Grammar \citep{fillmore1968case}.

The central idea of Frame semantics is that word meanings are described in relation to semantic frames, which are schematic representations of the conceptual structures and patterns of beliefs, practices, institutions, images, etc. that provide a foundation for meaningful communication within a particular speech community \citep[235]{Fillmore2003}. Semantic frames are defined more concisely as schematic representations of speakers' knowledge of the situations or states of affairs that underlie the meanings of lexical items \citep[130]{fillmore2007valency}. A frame-bearing lexical unit evokes a frame, and a valency description of a specific lexical unit presents the ways the semantic valents are expressed in the sentence built around the frame-bearing unit \citep[131]{fillmore2007valency}. 

FrameNet is a collection of semantic frames (each evoked by a set of lexical units associated with valency patterns) that represent conceptual-semantic and syntactic descriptions based on the annotation of examples. The semantic frame in FrameNet includes the following components: the frame name; the informal definition of the situation represented by the frame; a specification for the semantic type of the frame (optional); the set of frame elements (core and non-core: peripheral, extrathematic and core-unexpressed); a specification for the relations between frame elements, if any; a specification for frame-to-frame relations, if any; and the lexical units that evoke the frame. 

The frame element information includes the name of the frame element, its informal definition, the semantic type (optional) and examples illustrating the use of the frame element (optional). The information on the lexical units includes a definition, the semantic type (optional), examples and annotation in the examples of the frame elements as well as the grammatical categories and grammatical functions of their syntactic realisations.

Frame semantics thus links lexical units with both linguistic and conceptual information. The linguistic information consists of the frames as predicate classes, the sets of frame elements associated with them and their valency patterns. The conceptual information comprises the descriptions of situations and their participants as well as the relations between the frames \citep{sikos2018framenets}.


Two types of criteria were used to formulate the semantic frames \citep[11–17]{Ruppenhofer2016}: a checklist of features and other principles such as paraphrases and alternative answers to a question.


The checklist of features includes \citep[12–14]{Ruppenhofer2016}: 
\begin{itemize}
    \item the same number and type of frame elements for all lexical units; 
    \item the same set of stages and transitions (sub-events) shared by lexical units, i.e. unlike the verb \textit{decapitate}, the verb \textit{shoot} can be used to report the event of firing and hitting at a person, but it does not entail that the person dies, thus the two verbs should belong to different frames;
    \item the same participants' point of view, i.e. since the verb \textit{buy} takes the point of view of the \fename{Buyer} and the verb \textit{sell} takes the point of view of the \fename{Seller}, they belong to different semantic frames;
    \item the same interrelations between frame elements for all lexical units, i.e. a \fename{Purpose} expressed with the verb \textit{buy} relates to the \fename{Buyer}, a \fename{Purpose} expressed with the verb \textit{sell} relates to the \fename{Seller}, and the different relations indicate participation in different semantic frames;
    \item the same presuppositions, expectations, and concomitants of the target lexical units, i.e, the verb \emph{receive} presupposes a willing \fename{Agent} participating as a \fename{Donor} while the verb \emph{take} does not;
    \item the similar basic denotation of the lexical units (similarity of type);
    \item the similar pre-specifications given to frame elements by frame-evoking lexical units, i.e. verbs such as \emph{crowd}, \emph{flock}, \emph{pour}, \emph{stream}, etc. are part of the frame \framename{Mass\_motion} but not of the frame \framename{Self\_motion}  since they require that the moving entity is a \framename{Mass\_theme}, which generally consists of many individuals.
\end{itemize}

The development of frames is also based on the \emph{paraphrasability} (or near-paraphrasability) of lexical units: whether one lexical unit can be more or less successfully replaced by another, while evoking the same frame and the same configuration of frame elements. A semantic frame can be evoked by synonyms, near synonyms, antonyms, derivationally related lexical units, hypernyms, or hyponyms. For example, the  verb \textit{hate} with the definition `feel intense dislike for or a strong aversion towards' is a synonym of the verb \textit{detest} with the definition `dislike intensely' in the semantic frame \framename{Experiencer\_focused\_emotion}.  Both verbs 
have a hypernym \textit{dislike} with the definition `feel distaste for or hostility towards', the verb \textit{resent} with the definition `feel bitterness or indignation at' is also a hyponym of {\textit{dislike}} and it has hyponyms such as \textit{abhor}, \textit{abominate}, and  \textit{despise}. On the other hand, verbs like \textit{excrate}, \textit{contemn}, \textit{scorn}, \textit{disdain} are not presented in the frame as of September 2024. \textbf{Multiword expressions} are also included, albeit relatively rarely.

As in dictionaries, the lexical units of FrameNet are provided with \emph{definitions}, which were either taken from the Concise Oxford Dictionary, 10th Edition (courtesy of Oxford University Press) or written by the FrameNet developers \citep[9]{Ruppenhofer2016}.

In FrameNet, the frame elements are classified according to how central they are in a particular frame, whereby three levels are distinguished: core, peripheral and extrathematic.
A \textbf{core frame element} is an element that is necessary for the central meaning of the frame \citep[133]{fillmore2007valency} and represents a conceptually essential component of a frame and distinguishes the frame from others \citep[23]{Ruppenhofer2016}. \textbf{Peripheral frame elements} mark such notions as \fename{Time}, \fename{Place}, \fename{Manner}, \fename{Means} \fename{Degree} and the like. They do not distinguish between different frames and can be instantiated in any semantically suitable frame \citep[24]{Ruppenhofer2016}.
\textbf{Extrathematic frame elements} are understood as not conceptually belonging to the frames they appear in. They are part of other abstract frames and situate the event against the backdrop of another event \citep[133]{fillmore2007valency}.
The \textbf{Core-unexpressed} property refers to frame elements that function as core frame elements but do not appear in descendants of that frame. In child frames, however, the Core-unexpressed frame element is absorbed by the lexical units in the frame and cannot be represented individually \citep[25]{Ruppenhofer2016}.

In FrameNet, some formal properties, typically co-present, are taken into account when selecting the core frame elements. The core frame element \citep[23–24]{Ruppenhofer2016}:
\begin{itemize}

\item{should be specified openly};
\item{receives a definite interpretation if it is omitted (in the sentence \emph{John arrived} a certain frame element -- \fename{Goal} (location) -- is understood; \fename{Goal} is therefore a core frame element)};
\item{has no formal marking (its interpretation depends entirely on the target: i.e. frame elements that can be subject or object in a simple active sentence in English, or has an idiosyncratic formal marking (i.e. the preposition \textit{on} in \textit{depend on} has no semantic meaning)}.
\end{itemize}

Although some of the names of the frame elements correspond to the names of the semantic roles, the names of the frame elements only serve a mnemonic purpose \citep[237]{Fillmore2003}. The definitions of frame elements are statements that express the semantics of a particular frame element in relation to the target lexical unit (and possibly in relation to other frame elements).

It has been established that the frame elements are not necessarily independent of each other. 
Some groups of frame elements behave like sets (called Core Sets), since the existence of any member of the set is sufficient to fulfil the semantic valency of the predicator \citep[25]{Ruppenhofer2016}. For example, \fename{Source}, \fename{Path} and \fename{Goal} core frame elements in motion frames form a \textbf{Core Set} in the sense that only one or two (rarely all three) frame elements can occur in a sentence without violating the semantic structure.

The relation \FrameRelation{Requires} is coded if the occurrence of a core frame element presupposes that another core frame element also occurs. The relation \textit{Excludes} is observed if one of the frame elements from a group of conceptually related frame elements occurs and no other frame element from this group can occur \citep[26]{Ruppenhofer2016}. For example, the frame elements \fename{Goal} and \fename{Item} complement each other in the frame \framename{Attaching} and exclude the frame element \fename{Items}:

 \begin{exe}
 \ex  \label{ch01:ex:01}
  \textit{The robber \textbf{TIED}} [\textit{Harry}]$_{\feinsub{Item}}$ [\textit{to the chair}]$_{\feinsub{Goal}}$.
 \ex  \label{ch01:ex:02}
  \textit{The robber \textbf{TIED}} [\textit{Harry's ankles}]$_{\feinsub{Items}}$ \textit{together}.
\end{exe}


The FrameNet frames are linked by a system of nine \emph{frame-to-frame relations}, seven of which fall into three groups: Generalisation, Event structure, and Systematic \citep[806-807]{fillmore2010}.  
FrameNet can therefore be seen as a semantic net (or a set of small semantic nets) whose nodes represent the semantic frames and whose arcs represent the (semantic) relations between the frames.

Generalisation relations are \FrameRelation{Inheritance}, \FrameRelation{Perspective on} and \FrameRelation{Using}. In the relation \FrameRelation{Inheritance} (represented by directed (asymmetric) relations \FrameRelation{Inherits from} and \FrameRelation{Is Inherited by}), the frame elements of the parent frame are bound to the frame elements of the child frame, whereby the names of the child frame elements can be different. The semantics of the child frame is therefore a subtype of the semantics of the parent frame, and the child frame can contain additional frame elements \citep[330]{fillmore2010}. For example, the \FrameRelation{Inheritance} relation exists between the frame \framename{Revenge} and the frame \framename{Rewards\_and\_Punishment} because the frame \framename{Revenge} involves one person inflicting punishment on another, as in its parent frame, the frame \framename{Rewards\_and\_Punishment}. However, the frame \framename{Revenge} is explicitly different from the frame \framename{Rewards\_and\_Punishments} as it is outside institutional or judicial control \citep[330]{fillmore2010}.

It is also asserted that the \FrameRelation{Inheritance} relation corresponds to the \FrameRelation{is-a} relation in ontologies and that every semantic fact about the parent frame must correspond to an equally specific or more specific fact about the child \citep[80]{Ruppenhofer2016}. The complexity of the \FrameRelation{Inheritance} relation can manifest itself in different ways \citep[81]{Ruppenhofer2016}: parent and child frames can have different extrathematic frame elements; a child frame can have frame elements that are not present in the parent frame or such that are extrathematic in the parent frame; a child frame often does not express the parent frame elements of type \fename{Core-unexpressed}; a frame element of a child frame can be mapped to two frame elements of the parent frame; etc.

\tabref{tab:my_label1} illustrates the \FrameRelation{Inheritance} relation between the frame \framename{Experiencer\_focused\_emotion} and its successor frames: \framename{Desiring} and \framename{Mental\_stimulus\_exp\_\linebreak focus}.
\largerpage[2]

\begin{table}
\begin{tabular}{lccc}
\lsptoprule
Frames & \rotatebox{90}{\parbox{3cm}{\framename{Experiencer\_focused\_emotion}}} 
       & \rotatebox{90}{\framename{Desiring}} 
       & \rotatebox{90}{\parbox{3cm}{\framename{Mental\_stimulus\_exp\_focus}}} \\
\midrule
\multicolumn{4}{l}{\fename{{Core Frame elements} (FEs})}\\
\fename{Experiencer} & Yes &  Yes & Yes \\
\fename{Content} & Yes & \fename{Focal\_participant};   &   \fename{Stimulus}\\
                 &     &  \fename{Event}  & \\
\fename{Topic}  & Yes  & & Yes\\
\fename{Event} & Yes  &  \fename{Location\_of\_event}  &  \\
\midrule
\multicolumn{4}{l}{\fename{{Core Unexpressed} FEs}} \\
\fename{Expressor}   & Yes  &  & Yes (core) \\
\fename{State}  & Yes   &  & Yes (core) \\
\midrule
\multicolumn{4}{l}{\fename{{Peripheral} FEs}}\\
\fename{Degree}  & Yes & Yes & Yes \\
\fename{Manner}  & Yes & Yes & Yes \\ 
\fename{Time}  & Yes & Yes & Yes \\
\fename{Explanation}  & Yes & Yes & Yes  \\
\fename{Circumstances}  & Yes & & Yes \\
\fename{Parameter}  & Yes  &  & Yes  \\
\fename{Empathy\_target}  &  &  & Yes  \\
\fename{Duration}  &  & Yes & Yes \\
\fename{Purpose\_of\_event}  &  & Yes &  \\
\fename{Role\_of\_focal\_} &  & Yes &  \\
\quad \fename{participant} & \\
\fename{Time\_of\_event} &  & Yes &  \\
\fename{Place} &  & Yes &  \\
\lspbottomrule
\end{tabular}
\caption{The \FrameRelation{Inheritance} relation between the frame \framename{Experiencer\_focused\_emotion} and its successor frames, expressed by frame elements}
\label{tab:my_label1}
\end{table}

As the example shows, the relations between the frame elements of the frames connected via the Inheritance relation are quite complex: omission of a core frame element, i.e. the frame element \fename{Topic} in the frame \framename{Desiring}; specification of child frame elements, which is indicated by the names of the frame elements, i.e. \fename{Stimulus} in the frame \framename{Mental\_stimulus\_exp\_focus}, defined as ``the person, event or state of affairs that evokes the emotional response in the \fename{Experiencer}", corresponding to the more general frame element \fename{Content} in the frame \framename{Experiencer\_focused\_emotion}, defined as ``what the Experiencer's feelings or experiences are directed towards or based upon; the \fename{Content} differs from a stimulus because the \fename{Content} is not construed as being directly responsible for causing the emotion". In addition, the \fename{Content} can be expressed by one or both of the frame elements \fename{Focal\_participant} (``the entity that the \fename{Experiencer} wishes to be affected by some \fename{Event}") and \fename{Event} (``the change that the \fename{Experiencer} would like to see") in the frame \framename{Desiring}; etc. Although the frame elements of the parent frame are by and large retained in the child frames linked by \FrameRelation{Inheritance}, the example shows that some frame elements of the parent frame can be omitted in the child frame.

 
The relation \FrameRelation{Perspective on} (represented by directed (asymmetric) relations \FrameRelation{Perspectivises} and \FrameRelation{Is Perspectivised in}) encodes the different perspectives on an abstract event \citep[867]{fillmore2010}. The use of this relation indicates the existence of at least two different possible points of view on the neutral frame. The commercial transaction scenario, where buying and selling are seen as different perspectives on the transfer of goods (\framename{Commerce\_goods\_transfer}) and paying and accepting money are seen as different perspectives on the transfer of money (\framename{Commerce\_money\_transfer}), is an example that is frequently analysed in the FrameNet literature. 
It has also been shown that frames with perspectives are often non-lexical and abstract \citep[131]{osswald2014jr}.

In the relation \FrameRelation{Using} (with its members: \FrameRelation{Uses} and \FrameRelation{Is Used by}), the child frame is dependent on the background knowledge provided by the parent frame; at least some of the core frame elements of the parent frame are bound to child frame elements, but not all \citep[330]{fillmore2010}. The following example illustrates this: the frame \framename{Being\_attached} with a definition `An \fename{Item} is attached by a \fename{Handle}, via a \fename{Connector}, to a \fename{Goal}, or \fename{Items} are attached to each other' is \FrameRelation{Used by} the frame \framename{Being\_detached} with the definition `An \fename{Item}  is detached from a \fename{Source}, or \fename{Items} are detached from each other'.

 \begin{exe}
 \ex  \label{ch01:ex:03}
  \textit{It seems that} [\textit{the nits}]$_{\feinsub{Item}}$ \textit{are \textbf{ATTACHED}}  [\textit{to the hair}]$_{\feinsub{Goal}}$.
 \ex  \label{ch01:ex:04}
  \textit{I feel like} [\textit{my head}]$_{\feinsub{Item}}$  \textit{is \textbf{DETACHED} }  [\textit{from the rest of my body}]$_{\feinsub{Source}}$.
\end{exe}


\textbf{Event structure} relations are \textit{Subframe} and \textit{Precedes}  \citep[867]{fillmore2010}. 

\textit{Subframe} relation (\textit{Subframe of} and \textit{Has Subframe(s)}) is used when the child frame is expressed as a sub-event of a more complex parent event. For example, the frame \framename{Criminal\_process} has four subframes: \framename{Arraignment}, \framename{Arrest}, \framename{Sentencing}, and \framename{Trial}.

\textit{Precedes} relation (\textit{Precedes} and \textit{Is Preceded by}) indicates that there is a temporal order between the frames: the parent frame precedes the child frame. For example, the frame \framename{Employment\_continue} \textit{Precedes} the frame \framename{Employment\_end} and \textit{Is Preceded by} the frame \framename{Employment\_start}.

\FrameRelation{Causative of} and \FrameRelation{Inchoative of} are  \textbf{Syntactic} relations \citep[331]{fillmore2010}. In the relation \FrameRelation{Causative of},  the parent frame represents the causative  that corresponds to the child frame. In the relation \FrameRelation{Inchoative of},  the parent frame  represents the inchoative and the child represents the stative. For example, the frame \framename{Cause\_to\_fragment} is related to the frame \framename{Breaking\_apart} by the relation \FrameRelation{Causative of}. The frame \framename{Cause\_to\_fragment} has an \fename{Agent} as part of its conceptual core structure, while the frame \framename{Breaking\_apart} does not and expresses the \fename{Agent} as an oblique.

Furthermore, if there are groups of frames that are similar and should be carefully distinguished, each of the frames in question has a \FrameRelation{See Also} relation with a representative member of the group; \FrameRelation{Metaphor} is a relation between a source frame and a target frame in which many or all of the lexical units of the target frame are at least partially understood in terms of the source frame \citep[85]{Ruppenhofer2016}.
               
According to Fillmore, the implementation of Frame semantics in FrameNet should lead to correct frame-to-frame relations, including generalisations about how syntactic roles are assigned to arguments that depend on the more abstract inherited schemas \citep[157]{fillmore2007valency}. Developing a consistent relational structure of frames with different degrees of abstraction is a key challenge for the FrameNet approach, as certain case studies show \citep[153]{osswald2014jr}. At the same time, the addition of new frame-to-frame relations together with proposals for distinguishing subtypes within existing relations \citep[12--19]{sikos2018framenets} emphasises both the complexity of the conceptual information presented and the potential for its extension.

\figref{fig:F-to-FR} provides an overview of the connectedness between frames in FrameNet.

\begin{figure}
%   \includegraphics[width=\textwidth]{figures/F-to-F.png}
  \includegraphics[width=\textwidth]{figures/F-to-F.pdf}
  \caption{The immediate frame-to-frame relations of the semantic frame \framename{Arriving}. Red arrows \FrameRelation{Inheritance}, black --  \FrameRelation{Precedes}, green -- \FrameRelation{Using}, blue -- \FrameRelation{Subframe}, the direction is parent-child, the dashed lines show inverse relations.
  }
  \label{fig:F-to-FR}
\end{figure}

\newpage
\textbf{The semantic types} in FrameNet are used for \citep[86]{Ruppenhofer2016}:

\begin{itemize}
 \item{Marking of frames for their function}.
    \item{Specification of the basic typing of fillers for frame elements.}
 \item{Marking important dimensions of semantic variation between the lexical units in a frame.}
\end{itemize}

Lexical units, frames and frame elements are categorised according to \textbf{ontological semantic types}. For example, the semantic type [Region] is assigned to the lexical unit \emph{island}.n in the frame \framename{Natural\_features}, while the type [Body of water] is assigned to the lexical unit  \emph{bay}.n.

For frames, the semantic type indicates that each lexical unit of the frame can be labelled with an equivalent or more specific type. For example, the frame \framename{Clothing} has the semantic type [Artefact]. Consequently, all its lexical units denote artefacts, i.e. \textit{boot}.n, \textit{cape}.n, \textit{dress}.n, etc. \citep[422--423]{Lonneker-RodmanB09}.

Semantic types for frame elements classify the type of filler that is to appear as a frame element. Not all frame elements (and frames) have a specific semantic type, and in general semantic types are too broad, so they lack precision when it comes to conveying actual constraints on lexical combinations. For example, certain frame elements within the semantic frame \framename{Experiencer\_focused\_emotion} have rather general semantic types: \fename{Content} with the semantic type [Content]; \fename{Event} with the semantic type [State of affairs]; \fename{Experiencer} with the semantic type [Sentient]; \fename{Degree} with the semantic type [Degree]; \fename{Explanation} with the semantic type [State of affairs]; \fename{Manner} with the semantic type [Manner]; \fename{Time} with the semantic type [Time].
On the other hand, some frame elements such as \fename{Topic}, \fename{Expressor}, \fename{State} are not specified with a semantic type.

\textbf{Framal types} are applied to frames. The type [Non-lexical] characterises fra\-mes that have no lexical units but are used to semantically connect frames in a network. The type [Non-perspectivized] is used for frames that consist of a large number of lexical units that are connected by a common scene as a background. These frames usually lack a consistent set of frame elements for the targets, a consistent assignment of time to events or players and a consistent point of view between the targets, e.g. the frame \framename{Performers\_and\_roles}, which contains lexical units as different as \textit{co-star}.v, \textit{feature}.v and \textit{as}.prep \citep[87]{Ruppenhofer2016}.

\textbf{Annotations of examples} (originally mainly from the British National Corpus) are provided for lexical units. The annotations show the variety of syntactic manifestations of individual frame elements in the corpus (including zero realisations), together with the patterns of frame element realisations in sentences \citep[132]{fillmore2007valency}.

The syntactic annotation includes the labelling of \textbf{grammatical categories} and the \textbf{grammatical functions} of sentence constituents in relation to a particular target lexical unit.

The principal grammatical functions are External, Object and Dependent; the \newline  other grammatical functions are Appositive, Modifier, Head, Genitive and Quantifier, which are particularly important for nouns \citep[135]{fillmore2007valency}. The  grammatical function External corresponds not only to the subject of a finite sentence but also to the phrases that stand for the subject function of non-finite verbs, e.g., the controllers of subject roles in Raising and Equi constructions and subordinated participial constructions, and to the primary arguments of frame-bearing nouns and predicatively used adjectives \citep[135]{fillmore2007valency}. The function Dependent is used for all other dependents of a verbal predicate (other than External and Object).

The annotated examples show that some frame elements are restricted to certain parts of speech, suggesting that it might be a slightly different scene and raising questions such as: Is there a difference between frame elements of targets from different parts of speech that evoke one and the same frame, and what is the inheritance relation for targets from different parts of speech?

The top-down approach to frame creation and annotation is described as follows \citep[418--419]{Lonneker-RodmanB09}:

\begin{itemize}
 \item Selection of a semantic domain and outline of the frames involved.
 \item Definition of the frames and their frame elements and selection of the lexical units, each with a short definition.
 \item Determination of the principal syntactic patterns and extraction of examples for each pattern from a large corpus.
 \item Annotation of a sufficient number of examples to prove all relevant syntactic realisations of each frame element. FrameNet has extended its annotation to continuous text. In full-text annotation, all content words are annotated, leading to the addition of new lexical units within existing frames and (less frequently) the creation of new frames.
 \item Development of the FrameNet annotation view and the lexical entry view.
\end{itemize}

As pointed out, the semantic and syntactic descriptions in Frame\-Net differ from other lexical resources in several ways \citep[129]{fillmore2007valency}, including: (a) its reliance on corpus evidence; (b) its foundation on knowledge of the cognitive (semantic) frames that motivate and underlie the meanings of each lexical unit; (c) its recognition of various types of discrepancies between lexical units on the semantic level and patterns of syntactic form; and (d) its provision of the means of assigning partial interpretations to frame elements that are conceptually present but syntactically unexpressed.

\section{FrameNets for other languages} 

FrameNet has been largely extended to other languages \citep{boas2009multilingual}, such as Spanish \citep{Rggeberg2003SurpriseSF,Subirats+2009+135+162}, Japanese \citep{Ohara2004TheJF,ohara-2012-semantic},  German \citep{aljoscha2009}, 
 Chinese \citep{You2005BuildingCF}, Italian \citep{lenci-etal-2010-building}, Swedish \citep{Borin-Lars2010-110368}, Brazilian Portuguese \citep{torrent2014multilingual}, 
 French \citep{candito-etal-2014-developing}, Hebrew \citep{Hayoun2016TheHF}, Danish \citep{pedersen-etal-2018-danish}, Czech \citep{materna-pala-2010-using}, and many others \citep{framenet-2020-international}.

When creating lexical-semantic networks, two basic approaches are usually used: the expand model and the merge model \citep[716]{vossen1996right}.  The first approach is to translate the lexical units, their definitions and (possibly) usage examples from one language (usually English) into another and to transfer (and manually or semi-automatically check) all the relations between the lexical units as well as the remaining semantic information.

The task of FrameNets for other languages, which are created by the expand model (i.e. by searching for translation equivalents of language units), is to encode the language-specific features that can be expressed both semantically (by the number and relations of the frame elements) and at the grammatical level. In general, it can be said that many differences at the semantic level between languages are due to their different grammatical structures and, to a lesser extent, to the encoding of different features of the real world.

For example, the Spanish FrameNet describes the meaning of lexical units by drawing directly on the frames already constructed for English and analysing the grammatical constructions in which these lexical units are instantiated \citep[136]{Subirats+2009+135+162}.  If the English frames are not compatible with the Spanish language, the inconsistencies are resolved by restructuring the frames.

The second approach is to merge existing language resources for a particular language with other lexico-semantic resources for another language (usually English). One example is the Czech FrameNet, which was created by linking the independently developed Verbalex (a lexicon of verb valency for Czech) with the FrameNet \citep{materna-pala-2010-using}. The independent development of FrameNets may face the problem of achieving sufficient overlap in lexical coverage while maintaining language-specific properties.

It was found that there are two primary strategies for FrameNet development: a lemma-by-lemma strategy, that provides annotations that reflect the overall ambiguity of a given lemma within a target corpus, and a frame-by-frame strategy, that enforces the coherence of annotations within a frame \citep[1373]{candito-etal-2014-developing}.

The frame-by-frame approach, which is used by most FrameNets, takes into account the entire lexical diversity available for the expression of a frame \citep[1373]{candito-etal-2014-developing}. However, only the senses of a particular lemma related to covered frames are taken into account, and these senses are not necessarily the most frequent.

The lemma-by-lemma strategy considers different lemma senses for which there is often no frame in the English FrameNet, including rare senses. During the development of the German FrameNet (SALSA), each instance of a lemma in a corpus was annotated and tested for a FrameNet frame. Proto-frames were created for lemmas that could not be defined by existing frames. The proto-frames contain a single lexical unit and are not coupled with frame-to-frame relations \citep[213]{aljoscha2009}.

Some FrameNets are built entirely by experts (manually), both the mapping to English and the semantic and syntactic annotation, while others rely on automatic or semi-automatic mapping or annotation, possibly using post-validation, such as the Italian FrameNet \citep{lenci-etal-2010-building}.

Some of the linguistic issues that have arisen in the development of other FrameNets have been discussed by \citet{boas2009multilingual}: degree of overlapping cross-lingual polysemy, differences in lexicalisation patterns, measurement of paraphrase relations (words that evoke a particular meaning may differ in different sentences) and translation equivalence.

FrameNet is used extensively for the development of multilingual resources, and two general approaches to FrameNet integration can be distinguished: either building on the English FrameNet infrastructure as a foundation \citep{Boas+2009+59-100,Rggeberg2003SurpriseSF} or by (semi-)automatically creating frame-based multilingual resources \citep{10.1007/978-3-319-41552-9_35,torrent-etal-2014-copa}. The first approach uses the semantic frames as interlingual representations to connect different parallel lexicon fragments and involves several steps:

\begin{itemize}
\item Removing all language-specific information for English, including lemma, parts of speech and annotated sentences, and retaining only the information that is not specific to English -- frames, frame-to-frame relations, frame elements and frame element relations.
\item Repopulating the database to create a non-English FrameNet \citep[72]{Boas+2009+59-100}.
\end{itemize}

The (semi-)automatic creation of FrameNet-like resources involves the use of existing linguistic frameworks or corpora to extract semantic frames, frame elements and their relations to each other. Computational methods are employed to automatically identify frames in large datasets and annotate examples. This process includes the extraction of frame elements and the creation of frame-to-frame relations. The aim is to create a comprehensive lexical-semantic resource, similar to FrameNet, with minimal manual intervention.

The development of FrameNet for languages other than English has shown that many frames, especially those for common human behaviours such as \textit{drinking}, \textit{eating} and \textit{sleeping}, are relevant in multiple languages despite the presence of numerous language-specific valency patterns \citep[78]{baker-lorenzi-2020-exploring}. The different languages have adhered to the Berkeley FrameNet model to varying degrees: German, French, Swedish and Chinese FrameNet have deviated further from it by either adding many new frames or/and modifying existing ones, while Spanish, Japanese and Brazilian-Portuguese FrameNet have closely followed the original FrameNet and used FrameNet frames as templates \citep[78]{baker-lorenzi-2020-exploring}. The Multilingual FrameNet project \citep{baker-etal-2018-frame} investigates the relations between frames in different languages and alignments between FrameNets. There are different approaches to calculate the similarity of frames to create cross-lingual alignments: alignment by translation of lexical units, alignment by frame names, alignment by similarity of frame elements, alignment by similarity of distribution of lexical units, etc. \citep[79–80]{baker-lorenzi-2020-exploring}.

In this study we outline the basic principles for the development of the Bulgarian FrameNet, relying on language-independent information from the semantic frames while taking into account the language-specific features of Bulgarian. We can characterise the model for the development of the Bulgarian FrameNet as a \textbf{semi-automatic expand model}, since the automatic mapping of lexical units from semantic frames is applied to the Bulgarian WordNet \citep{LesevaStoyanova2020}, but both the automatic mapping of translation equivalents and the semi-automatic compilation of extended semantic and grammatical information for Bulgarian are evaluated manually.

The most important steps in the creation of the Bulgarian FrameNet can be summarised as follows:
\begin{itemize}
 \item Semi-automatic identification of lexical units (verbs) belonging to the general lexicon of Bulgarian;
 \item Identification of semantic frames suitable for describing situations evoked by the selected Bulgarian lexical units;
 \item Import of relevant language-independent information (valid at least for English and Bulgarian) from FrameNet semantic frames into superframes and conceptual frames within the system for the development of the Bulgarian FrameNet, BulFrame;
 \item Semi-autоmatic population of conceptual frames with relevant Bulgarian lexical units and related lexical, grammatical and semantic information;
 \item FrameNet-based annotations of examples to illustrate the valency patterns of the selected lexical units;
 \item Manual evaluation of the information in conceptual frames based on the annotation and potential reconstruction of conceptual frames, leading to the development of multiple conceptual frames associated with a superframe.
\end{itemize}

\section{Introducing superframes and conceptual frames in Bulgarian FrameNet} 

The endeavours to create the Bulgarian FrameNet have a history of about 20 years, the origins of which go back to predecessors such as the Bulgarian Valence Dictionary and the Semantic-Syntactic Dictionary of Bulgarian \citep{Koevaetal2003}. Originally, the resources focussing on frame-like semantic and syntactic descriptions were exclusively centred on Bulgarian, without establishing correspondences with FrameNet.

In the following phase, appropriate semantic frames were selected manually and language-independent information was extracted from these frames. This information was then supplemented with Bulgarian lexical units evoking the corresponding frames, and relevant examples were annotated \citep{koeva-2010-lexicon}. However, this endeavour was fraught with challenges, as there were no suitable means of maintaining correspondence with the semantic frames while providing options for reconstructing the semantic frame structures required for an adequate representation of some Bulgarian lexical units. Further challenges were to encode the translation equivalence between Bulgarian and English lexical units and to ensure the consistency of the FrameNet-like annotation with respect to the Bulgarian grammatical structure.

In its current stage, the Bulgarian FrameNet comprises two abstract semantic structures: a superframe and a conceptual frame, and it contains lexical units (accompanied by comprehensive lexical, semantic and grammatical information) that evoke conceptual frames, as well as valency patterns derived from authentic examples.

The main motivation for introducing superframes and conceptual frames is to facilitate the inclusion of language-specific information while ensuring consistency and  alignment with the relevant semantic frames.

Superframes establish abstract mappings between semantic frames in FrameNet and their counterparts in Bulgarian, thus forming a bridge between semantic resources. Conceptual frames (linked with a superframe) encode relevant information for Bulgarian, which may overlap in whole or in part with that for English (\figref{fig:frames}).
\largerpage
\begin{figure}
 \includegraphics[width=\textwidth]{figures/F1-cropped.pdf}
 \caption{The correspondence between Berkeley semantic frames, superframes and conceptual frames for Bulgarian.}
  \label{fig:frames}
\end{figure}

\subsection{Superframes}  
  

Introducing a mediating abstract layer, such as the level of superframes, enables the alignment of the appropriate components in Bulgarian FrameNet with FrameNet semantic frames, while allowing some others to retain their specificity. Superframes are constructed by removing all language-specific information for English, including lexical units that evoke the frames and their parts of speech, and retaining only non-specific information -- semantic frames, their semantic types and definitions, frame-to-frame relations, frame elements, their semantic types and definitions, frame element relations, and administrative information such as frame and frame element names. Superframes therefore contain language-independent information that can apply to at least two languages, in this case English and Bulgarian.


In principle, a superframe may be constructed based on semantic frames for languages other than English for which a FrameNet is developed. This means that when a conceptual frame is developed based on Bulgarian data for which no appropriate  superframe exists, a new superframe may be constructed, retaining only language-independent information in it.

This strategy aims to establish a seamless connection with FrameNet while enabling the identification and description of language-specific conceptualisations that are unique to Bulgarian and, if necessary, splitting a semantic frame into two or more conceptual frames, each characterised by different levels of reconstruction. An equivalence relation is established between the language-independent information in a semantic frame and the language-independent information in a superframe.

Conceptual frames are used to introduce script-like descriptions that are relevant to Bulgarian and that may be wholly or partially analogous to the information for English or provide unique information relevant to the Bulgarian conceptual description. A superframe can therefore be linked to one or more conceptual frames. However, there can be at most one conceptual frame whose components are connected to the language-independent components of the superframe via an equivalence relation and to the semantic frame via the latter. The remaining conceptual frames are connected to the superframe by partial equivalence relations, that can be tracked to determine which components of the conceptual frames are equivalent to the corresponding components in the superframe and which are not. In some cases, only one conceptual frame for Bulgarian can be associated with a particular superframe.

The relations between superframes mirror the relations between semantic frames in FrameNet. At the current stage of development of the Bulgarian FrameNet, there is rarely a need to introduce a conceptual frame that is not linked to an existing superframe, and its mirroring as a superframe is not accompanied by the introduction of new frame-to-frame  relations. Such changes to the FrameNet network, if they become necessary in the future, should be made with a high degree of consensus.

\subsection{Conceptual frames}  
A \emph{conceptual frame} can be defined (similarly to the semantic frame) as an abstract structure that describes a certain type of situation or event together with its actors and properties \citep[7]{koeva2020}. The conceptual frame is characterised by frame elements and relations between them and is complemented by a set of nouns that are compatible with the lexical units that evoke the frame. 

A specific conceptual frame in the Bulgarian FrameNet is evoked by a group of lexical units, which  (as of September 2024) are exclusively verbs.

\emph{Conceptual frames} have a frame name, a definition, a semantic type, frame elements and relations between frames. \emph{Frame elements} have a name, a definition, a semantic type, a core status and relations to each other: Core Sets, Excludes, Requires. This information is inherited from the semantic frames (via superframes) if they are already defined for English, and validated for Bulgarian by annotation.
 
Our motivation for employing superframes and conceptual frames is based on the following arguments:

\begin{description}
    \item[Argument 1:] Not all lexical units that evoke a given semantic frame exhibit the same semantic structure, which may lead to different syntactic behaviour.
 \end{description}   
As part of the comprehensive FrameNet approach to conceptual description, we want to distinguish groups of lexical units with equivalent semantic and syntactic properties. Therefore, we adhere to the principle that the \emph{semantic description of lexical units associated with a given conceptual frame is achieved by using the same number and type of core frame elements}. This approach does not change the structure of semantic frames, as many conceptual frames can be associated to a semantic frame (by a superframe). Therefore, there is often no one-to-one correspondence between a FrameNet semantic frame and a conceptual frame, as there are differences in conceptualisation between languages. The abstract superframe connects conceptual frames that express the same scene (one fully, the other partially) as the  FrameNet semantic frame. The omission, rare addition and status change of core frame elements within the conceptual frames associated with a semantic frame is justified by the annotation of examples. As for the equivalent syntactic properties, they are only related to the equivalent semantic properties, i.e. to the number and type of core frame elements, but not to the possibilities of expressing one and the same frame element in different syntactic ways, e.g. by a prepositional phrase or a clause.

For example, the Bulgarian verbs \textit{настанявам} (sit-IPFV, `am sitting'), \textit{настаня} (sit-PFV, `sit'), with the definition `determine, show someone a place to sit or lie down and help him/her to take it' (or comparison: the definition of \textit{sit}.v in FrameNet is `cause to sit or be placed (somewhere)') evokes the frame \framename{Placing}, which encompasses core frame elements such as \fename{Agent}, \fename{Theme} and \fename{Goal}. The \fename{Agent} is in a Core Set with the core frame element \fename{Cause}, and each of them controls the \fename{Theme} by placing it in a location, the \fename{Goal}.
For the semantic description of the Bulgarian verbs \textit{настанявам}, \textit{настаня} in the conceptual frame \fename{Placing}, only the frame element \fename{Agent} is relevant, while the frame element \fename{Cause} is omitted as semantically incompatible.

 \begin{exe}
 \ex  \label{ch01:ex:05}
 \gll \textit{Тогава}  [\textit{тя}]$_{\feinsub{Agent}}$   \textit{\textbf{НАСТАНИ}}  [\textit{майка си}]$_{\feinsub{Theme}}$  [\textit{в удобното кресло}]$_{\feinsub{Goal}}$.\\
 Then {she}  set {her mother} {in the comfortable armchair}. \\
 \glt `Then she sat her mother in the comfortable armchair.' 
 \end{exe}
 
\begin{exe}
 \ex  \label{ch01:ex:06}
\gll *\textit{Тогава} [\textit{вятърът}]$_{\feinsub{Cause}}$  \textit{\textbf{НАСТАНИ}} [\textit{майка ѝ}]$_{\feinsub{Theme}}$  [\textit{в удобното кресло}]$_{\feinsub{Goal}}$.\\
Then {the wind}  set {her mother} {in the comfortable armchair}. \\
\glt `{Then the wind sat her mother in the comfortable armchair}.' 
\end{exe}

 
Another example is the Bulgarian imperfect verbs from the frame \framename{Self\_Motion}, describing a scene in which a being moves in a certain way: \textit{ходя} (walk-IPFV, `am waking') `move by walking';  \textit{разхождам се} (walk-IPFV, `am waking') `walk somewhere outdoors'; \textit{плувам} (swim-IPFV, `am swimming') `for living organisms -- move on the water surface or in the water by certain movements of the body'.

These lexical units imply very little in terms of source and direction, and there is no reason to include the frame elements \fename{Source},  \fename{Goal} and \fename{Direction} in their semantic description as core elements.\footnote{A deeper semantic analysis will show that verbs such as \textit{walk}, \textit{swim}, etc. are typical activity verbs, but when used with an explicitly expressed \fename{Goal}, they can be regarded as  accomplishment verbs.} This is in contrast to the derivatively related perfective verbs, in whose semantic structure these frame elements can be core elements: \textit{преплувам} (swim across--PFV, `swim across')  `for humans or animals - by swimming cross a body of water or reach a certain place to which I am led'; \textit{доплувам} (swim up--PFV, `swim up') `swim to a certain place'.

 \begin{exe}
 \ex  \label{ch01:ex:07}
 \gll  [\textit{Момчето}]$_{\feinsub{Self\_mover}}$   \textit{\textbf{ПЛУВА}}  [\textit{в реката}]$_{\feinsub{Area}}$.\\
{The boy}  swim-IPFV {in the river}. \\
 \glt `The boy is swimming in the river.' 
 \end{exe}
 
  \begin{exe}
 \ex  \label{ch01:ex:08}
 \gll  [\textit{Момчето}]$_{\feinsub{Self\_mover}}$   \textit{\textbf{ДОПЛУВА}}  [\textit{до брега}]$_{\feinsub{Goal}}$.\\
{The boy}  swim\_up-PFV {to the shore}. \\
 \glt `The boy swam to the shore.' 
 \end{exe}

\begin{description}
    \item[Argument 2:] In contrast to English and other languages, a large number of diatheses in Bulgarian are associated with a lexical and/or morphological change of the source verb and are part of the lexicon in dictionaries.
\end{description}
Our second reason relates to the inclusion of verbal diatheses in FrameNet. In FrameNet, there is no specific list of verbal diatheses that a semantic frame encompasses. However, certain details in the annotation instructions indicate that diatheses associated with a particular predicate are considered part of the frame to which the basic diathesis belongs.  For example, no additional frame is formulated for the word \emph{sell} to account for usages such as \emph{Those boots sell well} that deprofile and make generic one or more of the prominent participants, in this case the \fename{Seller} \citep[12]{Ruppenhofer2016}. A similar approach also applies to passive sentences.

In cases where the semantic roles (the relation of an argument to the predicate or, in other words, of a frame element to the situation evoked by the lexical unit) do not change, the diatheses can undoubtedly be interpreted within a single frame, even if some frame elements remain implicit. We refer to such diatheses as syntactic, e.g. the participial passive and syntactic reciprocals in Bulgarian. However, if the semantic role of at least one frame element changes as a result of the diathetic alternation (we call such diatheses lexical), there are reasons to reconstruct the semantic frame in a new conceptual frame.

In Bulgarian there are several lexical diatheses (\textit{se} passive, impersonal participle passive, impersonal \textit{se} passive, middle, anticausative, lexical reciprocal, optative, impersonal optative, “oblique” subject and property of the “oblique” subject \citep[153--155]{Koeva2022} and some others. The lexical diatheses can either be structure-preserving (i.e. the number of frame elements remains the same, but at least one of the frame elements is given a new semantic role) or structure-changing (whereby the number of frame elements changes).

\begin{itemize}
\item  \emph{Structure-preserving} diatheses in Bulgarian are optatives and lexical reciprocals.  In optative diathesis, the semantic role  of a core frame element, the source subject, is changed, which is accompanied by a change in its grammatical role. In lexical reciprocal diathesis, the semantic roles of two core frame elements (source subject and object) as well as the grammatical role and the syntactic category of the source object change.

\item \emph{Structure-changing} diatheses in Bulgarian: the impersonal passives (both impersonal participle and  impersonal \textit{se}-passive), the impersonal optatives, the middles, the anticausatives, the “oblique” subjects and the property of “oblique” subjects -- show a reduction of semantic role as follows: both the source subject and the source object in the impersonal passives, the source subject in the middles and anticausatives and the source object in the impersonal optatives and “oblique” subjects. The reduction of the semantic role can be accompanied by a change in the semantic role, the grammatical role and the syntactic category of a remaining frame element. 

\end{itemize}

The problem of the representation of lexical diatheses in the structure of the Bulgarian FrameNet is even more complicated because: (a) in some of them the change of frame elements is connected with the appearance of a new frame element which is not present in the source diathesis; (b) some of them have a regular character, i.e. if certain lexical, morphological and syntactic features are present in the source diathesis, the formation of a certain lexical diathesis follows. An example of a regularly occurring lexical diathesis in which a new frame element appears that is not part of the structure of the source diathesis is optative diathesis in Bulgarian (which expresses a wish or desire to carry out the state or process evoked by the source diathesis).
The optative diathesis in Bulgarian is characterised by the following general features: The semantic role (frame element) of the canonical subject changes from \fename{Agent} (the name of the frame element can be different in FrameNet, e.g. \fename{Reader}) to \fename{Experiencer}, while that of the canonical object (if the verb is transitive) does not. At the morphological level, the optative verb is characterised by a reduction of the verb paradigm to the third person singular and plural and by the conjunction of the verb with the marker \textit{се }(self, `oneself'). The agentive source subject has the selectional restriction \textit{person} (animate), the object -- the selectional restriction \textit{inanimate}, and the source verb should be in the imperfective aspect (primary or derived from a primary imperfective verb).

 \begin{exe}
 \ex  \label{ch01:ex:00}
 \gll  [\textit{Момчето}]$_{\feinsub{Reader}}$   \textit{\textbf{ЧЕТЕ}}  [\textit{книга}]$_{\feinsub{Text}}$.\\
{The boy}  read--IPFV {book}. \\
 \glt `The boy is reading a book.' 
 \end{exe}
 
  \begin{exe}
 \ex  \label{ch01:ex:10}
 \gll   \textit{\textbf{ЧЕТЕ}} [\textit{му}]$_{\feinsub{Experiencer}}$   \textit{\textbf{СЕ}} [\textit{книга}]$_{\feinsub{Text}}$.\\
 read-IPFV {him} self {book}. \\
 \glt `He feels like reading a book.' 
 \end{exe}
 
Although the meanings of the modified verbs in lexical diatheses differ and there are morphological (the lemma form), grammatical (the change of syntactic categories and grammatical roles in the realisation of one or two frame elements) and semantic differences (the change in the number and type of frame elements or semantic roles), most lexical diatheses in Bulgarian are formed by regular language rules and can be predicted just like the syntactic ones. For those that involve the introduction of a new core frame element, such as the optative, there are only technical solutions to mark the option during annotation, i.e. by the name of the frame element: Agent-to-Experiencer and by the syntactic category and grammatical role of the word or phrase (otherwise, all conceptual frames that allow optative diathesis and other diatheses with similar regular alternations must be downgraded). However, numerous diatheses, e.g. lexical reciprocals and anticausatives, are not only formed regularly when certain conditions are met by the source diathesis, but they are also used very frequently and as such have become part of the lexicon in Bulgarian dictionaries. For such verbs there is a reason to present them in a separate conceptual frame in relation to their source verbs.

\textbf{Lexical reciprocals} are defined as ``words with an inherent reciprocal meaning'' \citep[14]{Nedjalkov2007}. There are some unmarked reciprocal predicates in Bulgarian: \textit{приличам на} (resemble, `look like'); reciprocal predicates with a reciprocal marker \textit{се}: \textit{състезавам се} (compete with, `compete with someone'), and reciprocal predicates that are a derived reciprocal diathesis: \textit{прегръщам се} (hug with, `to hold someone at the same time as he/she holds me') derived from the source diathesis \textit{прегръщам} (hug, `to put one or two arms around someone or something and hug him/her to my chest'). The verbs \textit{прегръщам} and \textit{прегръщам се} are presented in two conceptual frames under the superframe \framename{Manipulation}, which is connected to the semantic frame \framename{Manipulation}. Since the meaning is reciprocal, but the reciprocity is not realised syntactically by a reciprocal pronoun and a plural subject as in \textit{The boy and the girl hold each other}, the frame elements are both \fename{Agent} and \fename{Entity} with a different focus on one of the two.

\begin{exe}
 \ex  \label{ch01:ex:11}
 \gll  [\textit{Момчето}]$_{\feinsub{Agent}}$   \textit{\textbf{ПРЕГРЪЩА}}  [\textit{момичето}]$_{\feinsub{Entity}}$.\\
{The boy}  hold--IPFV {the girl}. \\
 \glt `The boy hugs the girl.' 
 \end{exe}

 \begin{exe}
 \ex  \label{ch01:ex:12}
 \gll  [\textit{Момчето}]$_{\feinsub{Agent-Entity}}$   \textit{\textbf{СЕ ПРЕГРЪЩА}}  [\textit{с момичето}]$_{\feinsub{Entity-Agent}}$.\\
{The boy}  hold--IPFV {with the girl}. \\
 \glt `The boy hugs with the girl.' 
 \end{exe}

Another example of diathesis that is regularly listed in dictionaries is \textbf{anticausative} diathesis, which is also known as inchoative, causative-inchoative or ergative diathesis \citep[27]{Levin:93}. In this diathesis, the semantic role of the source subject is reduced and the semantic and grammatical role of the source object is changed. For example, the verbs \textit{късам} (tear, `to cut something into pieces') and \textit{къса се} (is torn--3SG--3PL, `to tall into pieces') are members of the causative-anticausative diathesis and are located in separate conceptual frames under the superframe \framename{Cutting}, which is connected to the semantic frame \framename{Cutting}.

\begin{exe}
 \ex  \label{ch01:ex:13}
 \gll  [\textit{Съседката}]$_{\feinsub{Agt}}$   \textit{\textbf{КЪСА}}  [\textit{салфетката}]$_{\feinsub{Item}}$ [\textit{на парчета}]$_{\feinsub{Pie}}$.\\
{The neighbor}  cut-IPFV {the napkin} {into pieces} \\
 \glt `The neighbor is tearing the napkin into pieces.' 
 \end{exe}
 
\begin{exe}
 \ex  \label{ch01:ex:14}
\gll  [\textit{Салфетката}]$_{\feinsub{Item}}$   \textit{\textbf{СЕ}} \textit{\textbf{КЪСА}} [\textit{на парчета}]$_{\feinsub{Pie}}$.\\
{The napkin}  itself is\_cut-IPFV.3SG {into pieces} \\
\glt `{The napkin is torn into pieces}.'
\end{exe}

In fact, the anticausatives in FrameNet are housed in separate semantic frames, which are linked to the corresponding causative frames via the frame-to-frame relations \FrameRelation{Inchoative of} and \FrameRelation{Causative of}. However, the Inchoative frames are not yet fully represented in the FrameNet. In order to maintain the FrameNet structure, no new semantic frame is introduced for the anticausative (inchoative) diathesis, but an inchoative conceptual frame is established which, together with the causative conceptual frame, is connected to the causative superframe and via this to the causative semantic frame. This approach maintains the structure of the semantic frames without altering it, yet effectively reflects the distinctions in both the semantic and syntactic structures of the verbs that evoke them in Bulgarian.

The so-called autocausative diathesis can be seen as a variant of the anticausative diathesis, with the difference that the verb is bound to an animate subject that causes its own activity. A large proportion of autocausative verbs have become part of the lexical structure of Bulgarian and are not perceived as a regular product of autocausative diathesis, as is actually the case.

\begin{exe}
 \ex  \label{ch01:ex:15}
 \gll  [\textit{Бащата}]$_{\feinsub{Agent}}$   \textit{\textbf{ДЪРЖИ}}  [\textit{детето}]$_{\feinsub{Entity}}$ \textit{за ръка}.\\
{The father}  hold--IPFV {the child} {for the hand} \\
 \glt `The father is holding the child's hand.' 
 \end{exe}
 
\begin{exe}
 \ex  \label{ch01:ex:16}
\gll  [\textit{Детето}]$_{\feinsub{Protagonist}}$   \textit{\textbf{СЕ}} \textit{\textbf{ДЪРЖИ}} \textit{за ръката на баща си.}\\
{The child}  itself is\_hold-IPFV.3SG {for his father's hand}. \\
\glt `{The child is holding his father's hand}.'
\end{exe}

In such examples, the autocausative marker \textit{се} (se, `oneself') in Bulgarian is not the short form of a reflexive pronoun, but a lexical marker that is part of the verb. For verbs such as \textit{държа се} (hold, `hold on to something with my hands, as a support to keep my balance so that I don't topple over or fall') separate conceptual frames are created (if a corresponding meaning is not lexicalised in English and there is no semantic frame), which are linked to the respective causative semantic frame by a superframe.

\begin{description}
    \item[Argument 3:] Conceptual frames differ from semantic frames in that the frame elements of the conceptual frame are associated with a number of lexical units through which they can potentially be realised.
\end{description}
Each core element of the conceptual frame is connected to a set of nouns that are compatible with the verbs that evoke the frame. The set can contain only one noun, several nouns or a large number of nouns linked by semantic relations at the lexical level (synonymy, antonymy) or by hierarchical conceptual relations (hyperonymy, hyponymy). For example, the verb \textit{варя} (boil) with the definition `WN: cook food in very hot or boiling water' from the frame \framename{Apply\_heat} is characterised by four frame elements: \fename{Cook}, \fename{Food}, \fename{Container} and \fename{Heating instrument}, and for each of these elements the synset (one or more) from the Bulgarian WordNet that dominates the nouns suitable for collocations is specified.

\begin{itemize}
    \item   \fename{Cook}: pro-drop, NP, subject,  eng-30-00007846-n: \textit{person} 
   \item   \fename{Food}: optional, NP, object, eng-30-07555863-n: \textit{food};   eng-30-07649854-n: \textit{meat};  eng-30-07775375-n: \textit{fish};  eng-30-07707451-n: \textit{vegetable}
  \item \fename{Container}: optional, PP, object (\textit{в} `in'),  eng-30-03990474-n: \textit{pot}
 \item  \fename{Heating instrument}: optional, PP, object (\textit{в} `of', \textit{на} `on'), \newline eng-30-08581699-n: \textit{hearth}; eng-30-03543254-n: \textit{stove};  eng-30-03343560-n): \textit{fire}

\end{itemize}

\subsubsection{Levels of equivalence between conceptual frames and superframes}

Three general cases can be outlined. A superframe may be suitable for adoption as a conceptual frame if it reflects the semantics of at least one Bulgarian lexical unit. Some modifications to the semantic structure of the superframe may be required to match the Bulgarian data; these changes relate to the number and type of frame elements. It may also be necessary to develop new conceptual frames to describe language-specific data.
Thus, with regard to the use of the language-independent information from the semantic frames (which apply at least to English and Bulgarian), several cases may arise in relation to the superframes and conceptual frames: \emph{equivalence}, \emph{partial equivalence} and \emph{no equivalence}.

An \FrameRelation{equivalence} relation is observed when the abstract semantic representation of a superframe is copied into a conceptual frame to describe a scene evoked by a particular Bulgarian lexical unit (or units).
For example, the semantic frame \framename{Breaking\_apart} with the definition `A \fename{Whole} breaks apart into \fename{Pieces}, resulting in the loss of the \fename{Whole} (and in most cases no piece that has a separate function)' applies to English lexical units: \textit{break apart}, \textit{break}, \textit{crumble}, \textit{fragment}, \textit{shatter}, \textit{snap}, \textit{splinter}, as well as to their Bulgarian translation equivalents: \textit{чупи се},\textit{счупи се}, \textit{счупва се}, \textit{разпада се}, \textit{разпадне се}, \textit{разтрошава се}, \textit{разтроши се}, \textit{строшава се}, \textit{строши се}. In both languages, an equivalent situation can be represented, which is evoked by translation equivalents and expressed by the same number and type of frame elements: \fename{Whole} and \fename{Parts}. In Bulgarian FrameNet, the respective conceptual frame is thus constructed through a superframe.

\begin{exe}
 \ex  \label{ch01:ex:17}
\gll  [\textit{Стъклената кана}]$_{\feinsub{Whole}}$   \textit{\textbf{СЕ СЧУПИ}} [\textit{на множество малки парченца}]$_{\feinsub{Pieces}}$.\\
{Тhe glass jug}  break-PFV.3SG {into many small pieces}. \\
\glt `{The glass jug broke into many small pieces}.'
\end{exe}

The same procedure applies if a semantic frame is defined in FrameNet that is suitable for describing a Bulgarian verb or verbs, but its translation equivalent is not available in FrameNet. For example, the semantic frame \framename{Breaking\_apart} is also suitable for describing the verbs \textit{пръсна се} `burst', \textit{пръсва се} `is bursting' with the definition `WN: of a solid body or object -- to split, break apart suddenly and with force into parts', which are hyponyms of the verbs \textit{счупи се}, \textit{счупва се}. In this case, the lexical units are added to the Bulgarian conceptual frame.

\begin{exe}
 \ex  \label{ch01:ex:18}
\gll  [\textit{Балонът}]$_{\feinsub{Whole}}$   \textit{\textbf{СЕ ПРЪСНА}} \textit{неочаквано} [\textit{на парчета}]$_{\feinsub{Pieces}}$.\\
{The bubble}  burst-PFV.3SG {unexpectedly} {into pieces}. \\
\glt `{The bubble burst unexpectedly into pieces}.'
\end{exe}

A relation of \FrameRelation{partial equivalence} is observed when a semantic frame defined in FrameNet is only partially suitable for describing a Bulgarian verb or verbs. In such cases, an equivalent superframe is defined for the language-independent information and the corresponding conceptual frame is reconstructed. This reconstruction can include the exclusion or addition (rarely) of a frame element. It is also possible to change the core status of a frame element. For example, the frame \framename{Breaking\_apart} is partially suitable for describing the verbs \textit{пробива се} `is breaking through', \textit{пробие се} `break through' with a definition `WN: of a solid body or object -- suffer a breach of integrity by stabbing or piercing with a sharp object', whereby only one frame element is realised: \fename{Whole}. The new conceptual frame is connected to the same superframe as the conceptual frame evoked by verbs such as \textit{счупи се} "breaking apart".


\begin{exe}
 \ex  \label{ch01:ex:19}
\gll  [\textit{Гумата}]$_{\feinsub{Whole}}$   \textit{\textbf{СЕ ПРОБИ}}  \textit{на две места}.\\
{The tyre}  puncture-IPFV.3SG {in two places}. \\
\glt `{The tyre has punctured in two places}.'
\end{exe}


The abstract semantic structure \emph{superframe} is introduced to maintain the relation to a semantic frame and to combine semantically related verbs that do not have exactly the same meaning and the same semantic, morphological and syntactic features. A superframe corresponds to a semantic frame from FrameNet and connects a group of conceptual frames that share all or part of the semantic information of the respective semantic frame in FrameNet. The conceptual frames associated with a particular superframe are identified by the name of the corresponding semantic frame and an additional unique name after one of the verbs that evoke the conceptual frame, e.g. \framename{Breaking\_apart\_пробива\_се}.

The relation \FrameRelation{no equivalence} occurs when a semantic frame that is suitable for describing a Bulgarian verb is not defined in FrameNet and a new superframe and a conceptual frame must be defined. This can happen for two reasons:

\begin{itemize}
    \item The concept exists in English, but the corresponding semantic frame has not yet been created in FrameNet, for example, the verb \textit{golf}.v.
    \item  The concept is not conceptualised in English; for example, \textit{захърквам} (am snoring-IPFV, `start snoring'); \textit{захъркам} (snore-PFV, `start snoring');  \textit{за\-търсвам} (am looking for--IPFV, `start looking for');  \textit{затърся} (look for--PFV, `start looking for').
\end{itemize}

\begin{exe}
 \ex  \label{ch01:ex:20}
 \gll  [\textit{Той}]$_{\feinsub{Sound\_source}}$  \textit{моментално}  \textit{\textbf{ЗАХЪРКА}}   \textit{в хотела}.\\
{He}  {instantly} start\_snore-PST.3SG {in the hotel}. \\
\glt `{He instantly started snoring in the hotel}.'
 \end{exe}

The language-independent information in the semantic frames, which is  inherited by the superframes, is located on the conceptual and semantic level. This includes the definitions of the frames, the relations between the frames, the number and types of the frame elements, their definitions, semantic types, core status and relations. Administrative information, such as the name of the frame and the names of the frame elements, is also inherited. In addition, conceptual frames may contain information that is language-specific or potentially language-uni\-versal but has not yet been integrated into FrameNet.

Conceptual frames also contain sets of nouns that are suitable for collocations with the target verb. In addition, some information, such as definitions of lexical units, semantic relations between the concepts they denote, semantic classes of lexical units, grammatical information such as verb aspects and administrative information such as identification numbers, is taken from WordNet.

\section{Structure of the Bulgarian FrameNet} 

The structure in Bulgarian FrameNet associated with each lexical unit consists of the following sections: \textbf{Administrative}, \textbf{Lexical}, \textbf{Grammatical}, \textbf{Frame} and \textbf{Syntactic}.

\textbf{Administrative} information ensures the unambiguous interpretation of lexical units and frames. The WordNet ILI \citep{Vossen2004} serves several purposes: it acts as a unique identifier in both Bulgarian FrameNet and Bulgarian WordNet (BulNet), as it is the primary identifier for lexical units in Bulgarian FrameNet, it indicates the mapping to the corresponding synset (concept) in Princeton WordNet, it relates synonyms and labels the word senses associated with the lexical units. The names of the semantic frames are unique and are transferred to both the superframes and the conceptual frames, and the combination of frame name and frame element name is also unique.

Lexical units are provided with lexical and semantic information (lemma, part of speech, lexical type -- indication whether it is a multiword expression or not, sense definition, semantic class of the lexical unit, semantic relations to other verbs, if any, and some stylistic or usage labels) in the \textbf{Lexical section} and with grammatical information (verb aspect, transitivity and the range of grammatical subjects) in the \textbf{Grammatical section}.

The \textbf{Frame section} contains information about the frame definition, the frame elements, their definitions and relations, their semantic types and the semantic classes of the noun synsets that are suitable for pairing with the lexical units that evoke the frame.

Grammatical categories and grammatical functions encode the syntactic realisation (valency pattern) in the \textbf{Syntactic section}, as supported by the annotation.

The sources of inheritance and uniqueness of information in the Bulgarian FrameNet are schematised in the \tabref{tab:my_label2}.

\begin{table}
    \begin{tabular}{llll}
    \lsptoprule
     &  & FrameNet (FN) & BulFrame (BF)  \\\midrule
     \multicolumn{4}{l}{Admin. information}\\
     & Frame name  & FN  & FN or BF\\
     &  FE name & FN &  FN or BF \\
     & Verb ID  & No & WordNet (WN) \\
     \multicolumn{4}{l}{Lexical information}\\
     & Lemma type & word, MWE &  word, MWE\\
     & POS & V, N, Adj, Adv & V \\
     & Definition & FN & BF\\
     & Semantic class & No & WN \\
     & Stylistic note & No & BulNet (BWN)\\
     & Semantic type & FN &  FN\\
     & Semantic relations & No &  WN\\
     \multicolumn{4}{l}{Grammatical information}\\
     & Verb Aspect & No & BWN \\
     & Transitivity & No & BF \\
     & Personality & No & BF \\
     \multicolumn{4}{l}{Frame information}\\
     & Frame definition & FN & FN or BF \\
     & Frame-to-Frame relations & FN & BF\\
     & Frame elements  & FN & FN or BF \\
     & FE Core status & FN & FN or BF \\
     & FE definition & FN &  FN or BF  \\
     & FE type & FN & FN or BF \\
     & FE relations & FN & FN or BF \\  
     & V-to-N compatibility & No & BF \\
     \multicolumn{4}{l}{Syntactic information}\\
     & Grammatical category  & FN  & BF \\
     & Grammatical function &  FN & BF \\
     & Implicitness & FN & BF \\
    \lspbottomrule
    \end{tabular}
    \caption{Source of information in Bulgarian FrameNet}
    \label{tab:my_label2}
\end{table}

\subsection{Lexical section} 

Following FrameNet, a \textbf{lexical unit} is defined as a pairing of a word with a sense \citep[235]{Fillmore2003} (expressed by lemma and definition). The FrameNet assumption that each sense of a polysemous word belongs to a different semantic frame is followed, and it also applies to homonyms. For example, the Bulgarian verb \emph{деля} (divide) with the definition `WN: make a division or separation; FN: separate into parts or groups' evokes the frame \framename{Separating}, while the verb \emph{деля} (share) with the definition `WN: use jointly or in common; FN: to use something jointly with another sentient being' evokes the frame \framename{Sharing}. There are five meanings in the Dictionary of Bulgarian Language\footnote{\scriptsize\url{https://ibl.bas.bg/rbe/lang/bg/деля}} that are close to the meaning of the verb \emph{дeля} `divide', and four meanings in the Bulgarian wordnet;\footnote{\scriptsize\url{https://dcl.bas.bg/bulnet/}} in both sources the granularity of meaning is thus high, suggesting that such words may belong to separate conceptual frames that are related to one or more superframes.

In contrast to English, homonymy between lemmas from different parts of speech occurs less frequently in Bulgarian, but it does exist, e.g. the verb \textit{сушa} (dry, `WN: to remove moisture and make dry') and the noun \textit{суша} (land, `WN: the solid part of the earth's surface').

In this phase of the development of the Bulgarian FrameNet, we focussed on 5,074 verbs, which were selected according to quantitative and qualitative criteria and a heuristic according to which the criteria are applied \citep[207–208]{koeva-doychev-2022-ontology}. The criteria include presence in the Age of Acquisition Test -- the school level at which a word (the meaning of a word) must be learnt or mastered \citep{dale1981,goodman_dale_li_2008,morrison1997}; presence in WordNet Base concepts \citep[12--14]{Vossen1998}, aiming for maximum overlap and compatibility between the wordnets of multiple languages; root distance (the number of nodes) of a synset to the root of the local tree (the hierarchical substructure in WordNet in which the corresponding synset is contained); relative frequency in the Bulgarian National Corpus \citep{Koeva2012}, in Bulgarian textbooks from first to fourth grade and in a Bulgarian dictionary for primary school children.

\tabref{tab:my_label3} shows the language-independent and language-specific information provided in the Lexical section of the Bulgarian FrameNet.

\begin{table}
    \begin{tabular}{lll}
     \lsptoprule
     Type of information & FrameNet  & BulFrame  \\\midrule
     Semantic class & No & Language-independent\\
     Stylistic and usage notes & No & Both specific or independent\\
     Semantic relations & No & Language-independent\\
     \lspbottomrule
    \end{tabular}
    \caption{Language-independent and language-specific information in Lexical section}
    \label{tab:my_label3}
\end{table}

This information either shows the systematic semantic relations between the  concepts denoted by the lexical units or serves as classifying meta-information indicating the affiliation of the lexical units to certain semantic, stylistic or usage classes.

\subsubsection{Lemma}

In Bulgarian grammar, it is assumed that the lemma is the highest unmarked word form, i.e. the form in which there are no morphematically expressed grammatical categories (with the exception of the verbs \citep[20]{kutsarov2007}, where the lemma is the form of the first person singular present tense, while the grammatically most bare form is the third person singular present tense).

The lemma for certain verb classes with a restricted paradigm, such as impersonal verbs, is the present tense in the third person singular. For other verb classes with a restricted paradigm, however, the first person singular is chosen as the lemma in dictionaries, even if it is not used with the specified word sense. For example, \textit{тека} (flow-1SG.PRS, `I am flowing'), is used in dictionaries as a lemma instead of \textit{тече} (flow-3SG.PRS, `it is flowing'). In the Bulgarian FrameNet, the lemma is defined as the first member of the word paradigm actively used in the language \citep[25]{Koeva2008}, and for personal verbs the lemma is the first person singular, present tense; for impersonal and third-personal verbs the lemma is the third person singular, present tense; and for plural personal verbs the lemma is the first person plural, present tense \citep[19]{Koeva2010}.

\subsubsection{Multiword expressions} 

There are many different classifications for multiword expressions (\cite{baldwin2010multiword,constant-etal-2017-survey}), of which we have chosen the following classification for verbal multiword expressions in Bulgarian:

\begin{description}
\item[Semi-fixed:] The number of constituents is fixed, but these constituents can undergo certain paradigmatic changes within certain grammatical categories; the order of constituents can change, although there is a preferred word order; and there is room for insertions from restricted groups of words, i.e., the multiword expression \textit{гушна букета} (hug-PRS.1SG the-bouquet, `to kick the bucket').
\end{description}
In this context, we can distinguish different types of personal verbs whose lemma is formed with a reflexive-in-form particles. These include personal reflexiva tantum \textit{se} verbs, such as \emph{спирам се} (stop-1SG oneself, `am stopping'); personal reflexiva tantum \textit{si} verbs, such as \emph{спомням си} (remember-1SG oneself, `remember'); personal reciproca tantum \textit{se} verbs, such as \emph{състезавам се} (compete-1SG with someone, `compete with'); and personal reciproca tantum \textit{si} verbs, such as \emph{пиша си} (write-1SG with someone, `correspond with').

This group also includes third-personal verbs that may or may not form their lemma with a ``reflexive'' particle. These are:  third-personal accusativa tantum verbs, such as \emph{мързи ме} (lazy-3SG me-ACC.1SG, `I am lazy'); third-personal dativa tantum verbs, such as \emph{хрумнe ми} (occur-3SG me-DAT.1SG, `it occurs to me'); and impersonal reflexiva dativa tantum verbs, such as \emph{гади ми се} (sick-3SG me-ACC.1SG oneself, `I feel sick'). In these verbs, one of the frame elements must obligatorily be expressed by a personal pronominal clitic (accusative or dative), but it is conventionally regarded as part of the lemma, since the other forms express no meaning without it.
\begin{description}
\item[Non-fixed:] Its constituents can undergo morphological changes, undergo changes in word order and accommodate variable elements in their composition, e.g. \textit{изнасям лекция} (give-3SG lecture, `to give a lecture'), \textit{лекция ще изнасям утре} (give-FUT.3SG lecture tomorrow, `I will give a lecture tomorrow').
\end{description}

Most constructions with support verbs belong to this group. In FrameNet, it is assumed that support (light) verbs are selected by a frame-bearing noun: \textit{say a prayer} = \textit{pray} vs. *\textit{give a prayer} and \textit{give a speech} = \textit{speak} vs. *\textit{say a speech} \citep[244]{Fillmore2003}. For example, \emph{give a lecture} is part of the frame \framename{Speak\_on\_Topic}, evoked by the lexical unit \emph{lecture}.n. An important consequence of this analysis is the annotation in FrameNet of support verb subjects as frame elements relative to the noun.

The relevance of the verb to the support construction has been demonstrated as the support verbs can determine the semantic role that a particular constituent takes in a sentence \citep[244]{Fillmore2003}. For example, in the first sentence below, the grammatical subject is \fename{Patient}, while in the second sentence the grammatical subject is \fename{Agent}.

\begin{exe}
 \ex  \label{ch01:ex:21}
 \gll \textit{В миналото}  [\textit{той}]$_{\feinsub{Patient}}$  \textit{\textbf{Е ИМАЛ $^S$$^u$$^p$$^p$}} [\textit{катастрофа}]$_{\feinsub{Undesirable\_event}}$   \textit{с камион}.\\
 {In the past} {he}  {have-PST.3SG} {accident}  {with a truck}. \\
\glt `{In the past he had an accident with a truck.}'
\ex  \label{ch01:ex:22}
 \gll \textit{Днес}  [\textit{той}]$_{\feinsub{Agent}}$  \textit{\textbf{НАПРАВИ $^S$$^u$$^p$$^p$}}  \textit{\textbf{катастрофа}}  \textit{с камиона}.\\
{Today} {he} {make-PST.3SG} {accident} {with the truck}. \\
\glt `{Today he has made an accident with the truck.}'
\end{exe}

Both multiword expressions serve as synonyms for the Bulgarian verb \textit{ка\-тастрофирам}. In the first example, it corresponds to the meaning  `having an accident',  in the second example to the meaning of `making an accident'. The observations indicating that constructions with support verbs often have a single verb synonym, as well as the fact that support constructions can be considered sentence predicates, provide convincing evidence for the inclusion of support constructions as verbal multiword expressions in the Bulgarian FrameNet. The solution is that the multiword expressions with support verbs are considered to evoke the respective noun frame and are annotated in the same way as in English. The difference is that the entire multiword expression is added as a lexical unit with its own meaning. Thus, the first multiword expression with a support verb becomes part of the conceptual frame  \framename{Catastrophe}, which is linked to the superframe and the semantic frame \framename{Catastrophe}, while a new conceptual frame and superframe is created for the second, since there is currently no suitable semantic frame that can be copied.

\subsubsection{Definition} 

The definition serves to explain the meaning of a verb in a way that clearly distinguishes it from other meanings of the same word. These definitions are taken from the Bulgarian WordNet \citep[57--58]{koeva2021}, in which verbs of the imperfective and perfective aspect are intentionally presented as synonyms, accompanied by a common definition to preserve the structure of the Princeton WordNet. An appropriate definition should reflect the category verb aspect and the morphological features of the verbs (the limited person paradigm as third person, impersonal and plural personal); therefore, some of the definitions in the Bulgarian FrameNet need to be modified. For example, the following two verbs: \emph{излитам} (take off-IPFV.1SG, `am taking off') and \emph{излетя} (take off-PFV.1SG, `take off') are described with one definition in Bulgarian WordNet: 
`за летателни и космически апарати или под. -- отделям се от земята и започвам да летя' (of an aircraft and spacecraft or sub. -- leave-IPFV.1SG the ground and begin to fly).

The definition was modified to describe the meaning of the imperfective and perfective verbs that are used in the third person only:

\ea \emph{излита} \hspace{0.5cm} take off-IPFV.3SG \hspace{0.5cm}  \textit{It is taking off}.
\glt `за летателни и космически апарати или под. -- отделя се от земята и започва да лети'
\glt `of an aircraft, spacecraft, etc., leaves the ground and begins to fly' 
\ex \emph{излети}  \hspace{0.5cm}  take off-PFV.3SG \hspace{0.5cm}  \textit{It takes off.}
\glt `за летателни и космически апарати или под. -- да се отдели от земята и започне да лети' 
\glt `of an aircraft, spacecraft, etc., to leave the ground and begin to fly'.
\z

\subsubsection{Semantic type, semantic class, stylistic and usage labels}

Lexical units can be tagged with three different types from various sources: \emph{semantic type} from FrameNet, \emph{semantic class} from WordNet, and \emph{stylistic and usage labels} from Bulgarian WordNet.

FrameNet applies a number of \textbf{semantic types} to lexical units, with most types reserved for nouns and adjectives; however, all types available for verbs can be added to Bulgarian lexical units where appropriate. For example, \textit{see}.v with the definition `COD: perceive with the eyes', which evokes the frame \framename{Perception\_experience}, and \textit{glance}.v with the definition `COD: take a brief or hurried look', which evokes the frame \framename{Perception\_active}, have the semantic type \emph{Visual\_modality}, which can be transferred to the corresponding Bulgarian lexical units.

The synsets (or the individual words that make them up) in Princeton WordNet are organised into \emph{semantic classes} (primitives) that represent basic concepts that serve as distinct roots of different hierarchies. Verbs within these hierarchies are grouped under common semantic classes \citep[47]{fellbaum1993}, such as \textit{bodily care and functions}, \textit{change}, \textit{cognition}, \textit{communication}, \textit{competition}, \textit{consumption}, \textit{contact}, \textit{creation}, \textit{emotion}, \textit{motion}, \textit{perception}, \textit{possession}, \textit{social interaction}, \textit{weather verbs}, \textit{state}.
Each verb in the Bulgarian FrameNet is assigned a semantic class from the WordNet. For example, the verb \textit{искам} (wish, `order politely; express a wish') has the semantic class verb.emotion and the verb \textit{моля} (request, `ask a person to do something') has the semantic class verb.communication.

\textbf{Labels} or notes (also from the Bulgarian WordNet) are assigned to the corresponding lexical units to indicate various features, such as non-standard usage, figurative meanings, obsolete terms, informal usage and more. This labelling scheme reflects various distinctions in language usage: \textbf{belonging to non-stan\-dard lexis}, which includes dialectal words, slang or vernacular terms; \textbf{words with unfavourable connotations}; \textbf{use in a specific functional style}, such as colloquial, poetic, literary or technical terms; \textbf{historical period of use}, which distinguishes between obsolete, historical and newly coined words; \textbf{the expressive properties of the word}, such as pejorative meanings, augmentative or diminutive forms; \textbf{frequency of use}, which indicates whether a word is rare; \textbf{the nuances in use}, such as figurative meanings \citep[55]{koeva2021}. Stylistic marking usually excludes words from the core vocabulary, and although the labelling comes from the Bulgarian WordNet, the number of marked verbs is not large.

While the stylistic and usage labels are language-specific, the semantic classes are largely language-independent. A single lexical unit can be given several labels if they characterise it in different ways.

\subsubsection{Semantic relations}

The \textbf{lexical units} that evoke a conceptual frame can be one or more linked to each other by lexical relations (synonymy, antonymy) and/or hierarchical semantic relations at the conceptual level (hypernymy, troponymy, entailment).

The \emph{semantic relations} are inherited from the (Bulgarian) WordNet. Taxonomic relations for verbs are inverse and transitive (\emph{has a troponym} and \emph{has a hypernym}, or (\emph{has a subevent} and \emph{is a subevent of}).
Non-hierarchical relations are: symmetrical, irreflexive and non-transitive (\emph{antonymy}); symmetrical, irreflexive and Euclidean (\emph{also see}, \emph{verb group}). 

The semantic relations in WordNet are defined between synsets, and in Bulgarian WordNet verbs with different lexico-grammatical aspect are unnaturally grouped in one synset. The following general rules have been implemented to split verbs with different  aspect in the Bulgarian FrameNet, adopting the semantic relations for the concepts they express:

\begin{itemize}
 \item \emph{Troponymy}, \emph{Hypernymy} and \emph{Antonymy} connect either imperfective or perfective verbs.
 \item \emph{Also see} and \emph{Verb group} link verbs regardless of their  aspect.
\end{itemize}

Semantic relations between lexical units are usually language-independent, with the exception of relations that reflect culturally and historically specific concepts. By integrating the semantic relations of WordNet into the Bulgarian FrameNet, the scope of the semantic description is extended, which enables the application of certain evaluation heuristics.

\figref{fig:F11.png} shows an example of the Lexical section for the verb \textit{варя} (boil, `FN: cook by immersing in boiling water'), evoking the  conceptual frame \framename{Apply\_heat\_варя}.

\begin{figure}[ht]
  \includegraphics[width=\textwidth]{figures/FI11.png}
  \caption{Illustration of Lexical section in the BulFrame system.}
  \label{fig:F11.png}
\end{figure}

\subsection{Grammatical section}

The conceptual frame integrates morphological details specific to each verb: the \textbf{verb aspect}; the \textbf{range of grammatical subjects} that a predicate can select, such as nouns, all personal pronouns, third person pronouns only, subject clauses or none; and the \textbf{range of grammatical objects} that a predicate can select, including nouns, personal accusative pronouns, complement clauses or none. The verb aspect is determined by the morphological structure of verbs, which includes lexico-grammatical prefixes and grammatical suffixes.
The choice of grammatical subject is related to the person of the verb, while the presence of grammatical object depends to a certain extent on the transitivity of the verb.

These features (the ranges of grammatical subjects and objects) are closely related to the semantic and grammatical structures of a particular language. Even though some of the grammatical features are common to typologically related languages, they cannot be considered language-independent.
In the Slavic languages, including Bulgarian, the lexical manifestation of the verb aspect is evident, in contrast to English, where the continuous and perfect aspect are indicated by an auxiliary verb together with a present participle and a past participle respectively.
The range of grammatical subjects and objects in FrameNet is illustrated by annotation and summaries of valency patterns, provided that the annotation covers all contextual realisations within a language. In the Bulgarian FrameNet, the verb aspect and the ranges of grammatical subject and object selection are explicitly indicated for each verbal lexical unit. In addition, these features are also confirmed by the annotation and the valency patterns.

\tabref{tab:my_label4} summarises the language-specific information in the Grammatical section of the conceptual frame.

\begin{table}
    \begin{tabular}{lll}
    \lsptoprule
       Type of information & FrameNet  & BulFrame  \\\midrule
       Verb aspect & No & language-specific\\
       Range of grammatical subjects & language-specific & language-specific\\
       Range of grammatical objects & language-specific & language-specific\\
     \lspbottomrule
    \end{tabular}
    \caption{Language-specific information in Grammatical section}
     \label{tab:my_label4}
 \end{table} 

 \newpage
The combination of the values of the three categories (as well as some basic semantic information for the frame elements \fename{Agent} and \fename{Experiencer} such as  human and animate, and their grammatical roles) determines the formation of verbal diatheses in Bulgarian.

\subsubsection{Verb aspect}

As in other Slavic languages, it is disputed whether perfective and imperfective verbs express different lexical meanings or whether they are different forms of the same word. Some scholars such as \citet[137,193]{andreychin1944}, \citet[546]{kutsarov2007}, \citet[247]{Nitsolova2008} favour the former view, while others such as \citet[189]{Maslov}, \citet[8--9]{Stankov} argue for the latter. However, the prevailing evidence favours the first assumption and suggests that perfective and imperfective verbs in Bulgarian express different lexical and grammatical characteristics and thus represent different lexical units.

Bulgarian perfective and imperfective verbs typically exhibit overt morphological distinctions. Aspectual derivation in Bulgarian involves two main processes: perfectivisation and imperfectivisation. These derivations are formed by adding a prefix, a suffix or both to the verb root.

Certain verbs have a morphologically non-derived imperfective form (imperfectiva tantum) and can form perfective counterparts through prefixation. Conversely, perfective verbs can become imperfective through suffixation, a process known as secondary imperfectivisation. As a result, many Bulgarian verbs have aspect triplets.

 \begin{exe}
 \ex  \label{ch01:ex:23}
 \glll \textit{водя}   \hspace{0.4cm}  \textit{\textbf{из}веда}  \hspace{0.4cm} \textit{\textbf{из}вежд\textbf{ам}}\\
 take-IPFV  \hspace{0.4cm} take\_out-PER   \hspace{0.4cm} take\_out-IPFV \\
  \textit{`am taking to'}  \hspace{0.4cm} \textit{`take out'}  \hspace{0.4cm}   \textit{`am taking out'} \\
 \end{exe}
 
A smaller group of verbs are morphologically non-derived perfective verbs (perfectiva tantum), which produce imperfective counterparts through suffixation. Both non-derived perfective and derived imperfective verbs can form new prefixed verbs, whereby the lexico-grammatical aspect of the source verb is retained.

\begin{exe}
 \ex  \label{ch01:ex:24}
 \glll \textit{кажа} \hspace{1.2cm} \textit{каз\textbf{вам}}\\
say-PFV  \hspace{1cm} say-IPFV\\
 \textit{`say'} \hspace{1.3cm} \textit{`am saying'} \\

 \ex  \label{ch01:ex:25}
 \glll  \textit{\textbf{из}кажа} \hspace{1.0cm} \textit{\textbf{из}каз\textbf{вам}}\\
express-PFV  \hspace{0.3cm} express-IPFV\\
  \textit{`express'} \hspace{0.9cm} \textit{`am expressing'}\\
 \end{exe} 

In Bulgarian WordNet, each verb is assigned a label that indicates its lexico-grammatical aspect. For example: perfective verb -- \textit{запея} `start singing'; imperfective verb -- \textit{запявам} `starting singing'; simultaneous perfective and imperfective verb -- \textit{пенсионирам} `retire'; imperfective verb without perfective equivalent -- \textit{вали} `it is raining'; perfective verb without imperfective equivalent -- \textit{повярвам} `believe'. The labels indicating the aspect of the verbs in the Bulgarian WordNet are transferred to the Bulgarian FrameNet.

The difference in verb aspect is reflected lexically, morphologically and syntactically, which determines the explicit definition of the aspect values in the Bulgarian FrameNet. Here are some examples of morphological differences \citep[256]{Nitsolova2008}:

\begin{itemize}
 
\item The verbs with perfective aspect lack the so-called independent present tense (the present tense forms can only be used in subordinate clauses and their temporal meaning depends on the main verb), the present participles (both active and passive) and the negative imperative forms.

\item The derivational potential also differs between perfective and imperfective verbs. Perfective verbs cannot form certain types of deverbal nouns or nouns that denote occupations.

\end{itemize}

The syntactic differences can be summarised as follows \citep[56]{Koeva2022}:

\begin{itemize}

\item The perfective aspect has a direct influence on the syntactic realisation of verb complements. Direct objects of perfective verbs cannot remain implicit, and perfective verbs cannot serve as complements of phase predicates.

\item The perfective aspect of the verbs also reflects the restrictions in the formation of diatheses, as the perfective verbs in Bulgarian do not form middles, optatives or impersonals.

\end{itemize}

\subsubsection{Range of grammatical subjects}

With regard to the categories person and number, each verb in the Bulgarian FrameNet is categorised in one of four ways: \textbf{personal} (for first, second and third person singular and plural), \textbf{impersonal} (restricted to third person singular), \textbf{third-personal} (for third person both singular and plural) and \textbf{plural-personal} (for first, second and third person plural) \citep[33]{Koeva2010}. This categorisation is compatible with transitivity: Personal verbs can be both transitive and intransitive, while impersonal and third-personal verbs are typically intransitive, with the exception of accusativa tantum verbs.

Furthermore, the range of grammatical subjects corresponds to the possibilities of occupying the subject position: personal verbs require a subject (noun, noun phrase, substantive, pronoun, or subject clause); impersonal verbs do not require an argument in the subject position; third-personal verbs admit only a third-person subject (noun, noun phrase, substantive, pronoun); and plural-personal verbs require a plural subject (noun, noun phrase, substantive, pronoun).

\subsubsection{Verb transitivity}

Transitivity (and intransitivity) result from a specific syntactic realisation of a core frame element: a noun phrase that fulfils the grammatical role of a direct object.

Most languages have a number of transitivity classes of verbs. A typical pattern (which occurs in English, Bulgarian and many other languages) is given by \citet[4]{Dixon-2000}:
\begin{itemize}
\item strictly intransitive verbs that only occur in an intransitive clause;
\item strictly transitive verbs that only occur in a transitive clause (in such a case, the position of the direct object in the Bulgarian FrameNet is marked with the label obligatory);
\item ambitransitive verbs (or labile) that occur in both an intransitive and a transitive clause (in such a case, the position of the direct object in the Bulgarian FrameNet is marked with the label optional).
\end{itemize}

Within the two main groups of verbs (transitive and intransitive), the subclasses in Bulgarian are further subdivided on the basis of their lexical properties. This subdivision takes into account whether the verbs form a multiword expression with a ``reflexive'' particle and/or whether they obligatorily combine with a pronominal ``accusative'' or ``dative'' clitic \citep[34]{Koeva2010}.

In diatheses, it is common that the number of core frame elements (subjects and objects) either decreases or that the semantic roles of these core frame elements change. Many diatheses mainly concern transitive verbs and their transformation into intransitives, in which the original object takes on the grammatical role of the subject.

\subsection{Frame section}

The parts of the Frame section have different origins: some are inherited from FrameNet (in our case through a superframe), others are built according to the FrameNet structure, and another part is specific to the organisation of conceptual frames.

The correspondence with the language-independent semantic description in FrameNet (which is proven to be valid for at least two languages: English and Bulgarian) is documented in the conceptual frames by the superframes. There are three primary scenarios: two of them involve some form of equivalence (equivalence or partial equivalence) with a superframe, another does not. In the case of equivalence, the superframe is copied into a conceptual frame. In the case of partial equivalence, the language-independent information from FrameNet is reconstructed in a conceptual frame. If there is no equivalence, a new conceptual frame is developed and the language-independent information is integrated as a new superframe.

The FrameNet-related parts (inherited, (re)constructed or newly constructed) include frame elements together with their names, status (core, non-core and extrathematic), definitions, semantic types and relations.

The reconstruction of an inherited language-independent part may involve the reduction of a (core) frame element, the occasional insertion of a new frame element, or the change of status of a core or a peripheral frame element. Some relations between frame elements can either be reduced or redefined.

In addition to specifying semantic types for frame elements, a set of nouns is defined by using one or more noun synsets in WordNet that dominate the hypernym trees in which the nouns compatible with the target lexical units are represented.

\tabref{tab:my_label5} contains both language-independent and language-specific information in the Frame section, which is explained in more detail below.

\begin{table}
    \begin{tabular}{lll}
    \lsptoprule
       Type of information & FrameNet & BulFrame  \\\midrule
       Frame elements & LI or specific & LI or specific\\
       FE relations & LI or specific & LI or specific\\
       FE noun fillers  & No & LI or specific\\
     \lspbottomrule
    \end{tabular}
  \caption{Language-independent (LI) and language-specific information in Frame section}
   \label{tab:my_label5}
\end{table}  


\subsubsection{Frame elements}

Conceptual frames consist of frame elements equipped with name, definition, semantic type, core status, internal relations between the frame elements and information about the nouns with which these frame elements can be expressed. The names, definitions and semantic types of frame elements are adopted by FrameNet without additional specifications if they are suitable for Bulgarian. Only in rare cases, when a new frame (or a new frame element) is proposed, are they created from scratch.

So far, only the core frame elements relevant to Bulgarian have been included in the conceptual frames. In most cases where there is overlap between the core frame elements in the conceptual descriptions for both languages (English and Bulgarian), this correlation also extends to the peripheral frame elements.

\subsubsubsection{Core vs. peripheral frame elements}

When constructing the Bulgarian FrameNet, it is necessary to assess whether (core) frame elements are applicable to the semantic structure of the Bulgarian lexical units. The task is to determine whether a particular frame element is relevant, and if so, whether it is a core or a peripheral, and in some cases there is a core status shift of frame elements in Bulgarian descriptions. A (core) frame element may be omitted or modified to better fit the description of the Bulgarian lexical units.

The distinction between core and peripheral frame elements in FrameNet is based on the following properties of core frame elements: overtly specified, unambiguous interpretation when omitted, and without formal marking.

If we compare Bulgarian with English, the overt specification of frame elements cannot be regarded as a formal feature of subjects. In contrast to English, subject omission in Bulgarian can occur in combination with all verbal categories, not only with the imperative. Bulgarian is a null subject language and allows subject omission due to its rich verb inflectional morphology, which indicates person, number and in some verb categories also the gender of the omitted subject. The use of null subjects in Bulgarian in the first and second person is not grammatically and contextually restricted, while the choice between explicit or implicit subject in the third person may depend on the context of the discourse.

The conditions for direct null objects differ from the conditions for null subjects. Similar to some other languages, direct null objects in Bulgarian can only be observed with transitive imperfective verbs. Since the imperfective verb may not imply the result of an activity, the object can be omitted. A null object is permissible if it is understood by the lexical meaning of the verb, mentioned earlier in the discourse or implied by the context. In cases where support verbs are used, the object cannot be omitted even with transitive imperfective verbs:

 \begin{exe}
 \ex[]{\label{ch01:ex:26}
\emph{Влязох в стаята, където Иван четеше \textbf{книга}, преди да заспи.}
  \glt `{I entered the room where Ivan was reading a book before falling asleep.}'}
 
 \ex[]{\label{ch01:ex:27}
\emph{  Влязох в стаята, където Иван четеше, преди да заспи.}
  \glt `{I entered the room where Ivan was reading  before falling asleep.}'}
 
 \ex[]{\label{ch01:ex:28}
 \emph{ Влязох в стаята, където Иван вземаше \textbf{хапчета за сън}, преди да заспи.}
 \glt `{I entered the room where Ivan was taking sleeping pills before falling asleep.}'}
 
 \ex[*]{\label{ch01:ex:29}
\emph{ Влязох в стаята, където Иван вземаше, преди да заспи.}
  \glt `{I entered the room where Ivan was taking before falling asleep.}'}
\end{exe} 
 

Indirect objects and nominal or prepositional adverbials that express a core element of the frame can also remain implicit if they can be derived from the wider context.

It has been pointed out that arguments at the semantic level can be obligatory or non-obligatory, and truly optional semantic arguments are distinguished from obligatory semantic arguments. For example, in \textit{He kicked the pumpkin} (\textit{down the stairs}) the phrase \textit{down the stairs} is a realisation of an optional semantic argument, while in \textit{He threw the pumpkin} (\textit{down the stairs}) it is an obligatory semantic argument \citep[174--176]{Culicover2005}. Both verbs allow the optional expression of the \emph{path of motion}. If such a \emph{path} is not expressed with \textit{kick}, the direct object does not have to experience any movement. With \textit{throwing}, on the other hand, something is set in motion. Even if the \emph{path} expression is omitted, it is therefore semantically implicit. In the verb \textit{throw}, the \emph{path of motion} is therefore an obligatory semantic argument that remains implicit at the syntactic level.

In Bulgarian, there is no general formal marking for grammatical roles such as subject and object. Only the personal pronominal clitics have forms for the nominative, accusative and dative. For indirect objects and prepositional adverbials, the range of permissible prepositions can be specified. Given the many options for omitting sentence constituents and the limited use of formal case markers, the unambiguous semantic interpretation of omitted core constituents therefore remains the primary formal feature for their differentiation in Bulgarian.

The challenges in determining the core status of frame elements for Bulgarian verbs can be twofold: firstly, the categorisation of indirect objects that do not serve as core frame elements, and secondly, the categorisation of prepositional and noun adverbial phrases that serve as core frame elements.

It has been pointed out that adverbial phrases as core frame elements typically accompany verbs describing situations in which temporal or spatial elements represent frame-internal information -- ``information that fills in details of the internal structure of an event ... as opposite to the information about the setting of incidental attending circumstances of that event, the frame-external information'' \citep[159]{fillmore1994}.

  \begin{exe}
 \ex  \label{ch01:ex:30}
 \textit{The show started \textbf{at five o'clock}.}
\ex  \label{ch01:ex:31}
\textit{The performance lasted \textbf{five hours}.}
\ex  \label{ch01:ex:32}
\textit{John lives \textbf{in Sofia}.}
\ex  \label{ch01:ex:33}
\textit{The vase stays \textbf{at the table}.}
 \end{exe} 
  

\subsubsubsection{Basic instances of restructuring frame elements}

Since the core frame elements were taken from the FrameNet, the main task in building the Bulgarian FrameNet is to assess whether these frame elements are applicable to the semantic structure of the Bulgarian lexical units. In rare cases, a peripheral frame element may be elevated to the status of a core frame element in the Bulgarian FrameNet. Conversely, a core frame element can be omitted or modified in order to adapt it to the semantic description of the situation evoked by the Bulgarian lexical units.

An example of the lack of a core frame element in the Bulgarian FrameNet is the semantic frame \framename{Awareness}, which contains core frame elements such as \fename{Cognizer}, \fename{Content} and \fename{Topic}, the latter two forming a Core Set.
The Bulgarian verb \textit{вярвам} (believe, `COD: feel sure of the truth of') evokes the conceptual frame \framename{Awareness}, in which the frame element \fename{Topic} is not involved. In other words, when describing the verb \textit{вярвам}, only the frame element \fename{Content} is relevant within the conceptual frame \framename{Awareness}, while the frame element \fename{Topic} is omitted.

 \begin{exe}
 \ex  \label{ch01:ex:34}
 \gll \textit{Вчера}   [\textit{той}]$_{\feinsub{Experiencer}}$    \textit{\textbf{ВЯРВАШЕ}}  [\textit{в своята гениалност}]$_{\feinsub{Content}}$. \\
Yesterday {he}  believe-PST  {in his genius}. \\
 \glt `Yesterday he believed in his genius.'
 \end{exe} 


 \begin{exe}
 \ex  \label{ch01:ex:35}
 \gll  [DNI]$_{\feinsub{Experiencer}}$   \textit{\textbf{ВЯРВАM}},  [\textit{че това е верният път}]$_{\feinsub{Content}}$. \\
 {I-dropped} believe-PRS  {that this is the right way}. \\
 \glt `I believe that this is the right way.'
 \end{exe} 
 
Another example in the Bulgarian FrameNet that illustrates the transformation of a peripheral frame element into a core frame element is the frame element \fename{Goods} in the frame \framename{Robbery}, which is defined as follows: A \fename{Perpetrator} wrongs a \fename{Victim} by taking something (\fename{Goods}) away from him. Bulgarian verbs such as \textit{обирам} (rob-IPFV, `am robbing'), \textit{обера} (rob-PFV, `rob') evoke the frame \framename{Robbery}, and in their syntactic realisation the element \fename{Goods} receives a unique interpretation and is therefore considered a core frame element.

 \begin{exe}
 \ex  \label{ch01:ex:36}
 \gll \textit{Тогава}   [\textit{скитникът}]$_{\feinsub{Perpetrator}}$    \textit{\textbf{ОБРА}}  [\textit{дома на капитана}]$_{\feinsub{Source}}$. \\
 Then  {the tramp} rob-PST  {the captain's house}. \\
 \glt `Then the tramp robbed the captain's house.'
 \end{exe} 

 \begin{exe}
 \ex  \label{ch01:ex:37}
 \gll \textit{Тогава}  [\textit{скитникът}]$_{\feinsub{Perpetrator}}$   \textit{\textbf{ОБРА}}  [\textit{накитите}]$_{\feinsub{Goods}}$  
 [\textit{от къщата}]$_{\feinsub{Source}}$. \\
 Then  {the tramp}  rob-PST   {the ornaments} {from the house}. \\
 \glt `Then the tramp stole the ornaments from the house.'
 \end{exe} 
 

The Bulgarian verbs resulting from lexical reciprocal diathesis, in which the reciprocal meaning is expressed at the lexical level in both singular and plural forms, are an example of a case in which the semantic roles of the core elements of the frame involved in the source are shifted within the derived diathesis. For example, the verbs \textit{пиша} (write-PER, `provide information to someone through letters') and \textit{пиша си} (write-PER oneself-REFL, `exchange information with someone through letters') are both parts of the frame \framename{Text\_creation}, which is defined as follows: An \fename{Author} creates a \fename{Text}, either written, such as a letter, or spoken, such as a speech, that contains meaningful linguistic tokens and may have a particular \fename{Addressee} in mind. The source diathesis evokes a situation with the core frame elements \fename{Author} and \fename{Addressee}, while the derived diathesis with the reciprocal meaning evokes a situation with the core frame elements \fename{Author1\_Addressee2} and \fename{Author2\_Addressee1}.

\begin{exe}
 \ex  \label{ch01:ex:38}
 \gll  [\textit{Поетът}]$_{\feinsub{Author}}$   \textit{\textbf{ПИШЕ}}    [\textit{на своята любима}]$_{\feinsub{Addressee}}$. \\
 {The poet} write-PRS  {to his beloved}. \\
 \glt `The poet writes to his beloved.'
 \end{exe} 
 
\begin{exe}
 \ex  \label{ch01:ex:39}
 \gll   [Поетът]$_{\feinsub{Author1\_Addressee2}}$  \textit{\textbf{СИ ПИШЕ}} [\textit{със своята любима}]$_{\feinsub{Author2\_Addressee1}}$. \\
 {The poet}  oneself\_write-PRS {with his beloved}. \\
 \glt `The poet corresponds with his beloved.'
 \end{exe} 
  

\subsubsubsection{Frame element relations}

The relations between frame elements are inherited from FrameNet. If a semantic description of a scene is suitable for English and Bulgarian, then the generalisations for the relations between the frame elements should also apply to both languages (as a general rule). For example, in the frame \framename{Manipulation}, which describes the manipulation of an \fename{Entity} by an \fename{Agent}, the \textit{Core Set} is defined between the frame elements \fename{Agent} and \fename{Bodypart\_of\_agent}.


 \begin{exe}
 \ex  \label{ch01:ex:40}
  \gll  \textit{Бабата} \textit{започна}   \textit{да}  \textit{го}  \textbf{\textit{МАСАЖИРА}} [\textit{с ръце}]$_{\feinsub{Bodypart\_of\_agent}}$. \\
 The\_old\_lady  start.3SG-PST to  he-DAT   massage {with hand-PL.}\\
 \glt `The old lady started massaging him with her hands.'
 \end{exe} 
 
\begin{exe}
 \ex  \label{ch01:ex:41}
 \gll \textit{Бавно}  [\textit{ръцете}]$_{\feinsub{Bodypart\_of\_agent}}$  \textit{\textbf{МАСАЖИРАХА}}  \textit{врата}  \textit{и}  \textit{гърба}  \textit{му}. \\
 Slowly the\_hand.PL massage-3PL neck  and  back his. \\
 \glt `Slowly, the hands massaged his neck and back.'
 \end{exe} 
  

The definition of Core Sets (each member of the set is sufficient to fulfil the semantic valency of the predicator) allows the inclusion of diatheses \textit{Oblique} subject, as members of one and the same conceptual frame. The alternations that fall into this group are: Natural force subject, Instrument subject, Locatum subject, Raw Material Subject \citep[79–83]{Levin:93}. The ``oblique subject'' diatheses are realised when the semantic role of the subject does not change and the semantic role of the prepositional object is reduced, but the source noun of the prepositional phrase is realised as a derived subject \citep[148]{Koeva2022}.

In many cases, it is sufficient to define a Core Set because it means the realisation of one frame element instead of another, but it also defines their co-occurrence, which is not possible with the relation \textit{Exclusion}.
 
The frame elements \fename{Content} (The \fename{Content} is the entity that evokes a cognitive reaction for the \fename{Experiencer}) and \fename{Experiencer} (The \fename{Experiencer} is newly aware of the \fename{Content}) from the frame \framename{Enter\_awareness} are in the relation \textit{Requires}. The frame element \fename{Content} cannot appear without the frame element \fename{Experiencer}. The example is for with the lexical unit \textit{хрумне ми} (it occurs to me, `FN: suddenly become known (to someone)’).


\begin{exe}
 \ex  \label{ch01:ex:42}
 \gll \textit{\textbf{ХРУМНА}}   [\textit{ми}]$_{\feinsub{Exp}}$  [\textit{една}  \textit{идея}]$_{\feinsub{Content}}$. \\
 Occur  I-DAT an  idea. \\
 \glt `An idea occurred to me.'
 \end{exe} 
 
\begin{exe}
 \ex  \label{ch01:ex:43}
 \gll \textit{\textbf{ХРУМНА}}   [\textit{му}]$_{\feinsub{Exp}}$  [\textit{да}   \textit{излезе}  \textit{от} \textit{стаята}]$_{\feinsub{Content}}$. \\
 Occur  he-DAT to leave  from  the\_room. \\
 \glt `It occurred to him to leave the room.'
 \end{exe} 
 
An example of the additional coding of relations between frame elements are the relations \textit{Requires} between the frame elements \fename{Whole} and \fename{Parts} as well as between \fename{Part} and \fename{Part2}, while \fename{Part1} \textit{Excludes} \fename{Parts} and \fename{Part2} \textit{Excludes} \fename{Parts} in the frame \framename{Becoming\_separated} with the definition: A \fename{Whole} separates into \fename{Parts}, or one part of a whole, called \fename{Part1}, becomes separate from the remaining portion, \fename{Part2}.

\subsubsection{Noun frame element fillers}

FrameNet enables the characterisation of `role fillers' by semantic types of frame elements, which should be largely constant across all uses. However, not all frame elements are provided with a semantic type or the semantic types are too general. For this reason, we have decided to specify as far as possible the set of nouns that are suitable to represent a particular frame element in a sentence and combine with the target verb.

Several attempts have been made in this direction. Compared to FrameNet, another lexical-semantic resource based on frame semantics, VerbAtlas, uses fewer semantic roles (frame elements) (25) and many more semantic types (selectional preferences) (116), expressed in terms of WordNet synsets \citep[627]{di-fabio-etal-2019-verbatlas}. Selectional preferences were manually chosen from a set of 116 macro-concepts defined by WordNet synsets whose hyponyms are considered likely candidates for the corresponding argument slot \citep[627]{di-fabio-etal-2019-verbatlas},  a strategy similar to the previous one based instead on algorithms \citep{agirre-martinez-2001-learning}. The comparison of the interpretation of the verb \textit{heаr} in FrameNet and VerbAtlas shows that the semantic frame in FrameNet \framename{Perception\_experience} is more general (it includes any kind of perception), while the frame in VerbAtlas includes verbs with a different but narrower meaning, limited to auditory perception. In FrameNet there is a frame element that refers to the body part through which the perception occurs, and in VerbAtlas there are two semantic roles: \emph{Stimulus} and \emph{Source}, with an equal selectional preference: \emph{entity} (the difference remains unclear). Both resources contain frame elements (semantic roles) for the \emph{Perceiver} and the perceived auditory \emph{Phenomenon}. It is very difficult to define sets of verb-noun combinations, regardless of whether automatic or corpus observation methods are used, due to the objective difficulties that figurative but acceptable usage may always entail.

In another approach, for a dictionary containing FrameNet-based data for English, Brazilian Portuguese and Spanish, domain-specific ontologies are used to impose semantic constraints on the frame elements \citep{Hauck}. 
In the context of the Brazilian FrameNet, each core frame element undergoes an analysis based on the aspect of the scene it represents, resulting in the mapping of one or more frames to the frame element \citep{Torrent}. Only frames that stand for events, states, attributes and relations are eligible for frame element to frame relations. The information available from the definition or semantic type is used to determine the type of concept it refers to (e.g. person, place, event) and to identify the highest level frame that represents it. In this way, the FrameNet is enriched with additional semantic information by linking the conceptual structures that make it up. The approach seems to be similar to the introduction of morphosemantic relations between verb and noun synsets in WordNet; however, this extension is not applicable as nouns have not yet been included in the Bulgarian FrameNet.

The accepted approach is to select the topmost synset (or conjunction of topmost synsets) from the Bulgarian WordNet that dominates all corresponding noun synsets for a given frame element. The nouns in WordNet are divided into 25 semantic classes \citep[16]{Miller1990}, which, being general, can be subdivided into subclasses. For example, within the semantic class {food}, the subclass {beverage} can be introduced for nouns associated with verbs such as \textit{stir}, \textit{sip}, \textit{drink}, \textit{lap}, and so on. Such a representation aims to specify the organisation of concepts in an ontological structure that allows inheritance between semantic classes in the hierarchy and ensures a more precise specification of compatibility between verbs and nouns. One way to extend the WordNet semantic class repository is to map the WordNet synsets to an existing hierarchy of semantic types, e.g. the Corpus Pattern Analysis (CPA) types \citep{hanks2004}. The extension of the WordNet semantic classes with the CPA semantic types is done manually by matching the CPA semantic types with the WordNet synsets and selecting the most suitable ones \citep{koeva-etal-2018-mapping}. Initially, less than 100 semantic classes were used; however, as the number of lexical units and conceptual frames increased, so did the number of semantic classes defined, reaching 377 in September 2024.

The aim is to provide representative information about the collocations between verbs and nouns by extracting corpus evidence. Both extremely vague and overly specific descriptions are avoided. For example, the verbs \textit{bomb}.v and \textit{attack}.v are part of the frame \framename{Attack} (An \fename{Assailant} physically attacks a \fename{Victim}), and the frame element \fename{Victim} with the verb \textit{bomb}.v is characterised by the hyponyms {region}: eng-30-08630985-n and {building}: eng-30-02913152-n. Combinations such as \textit{bombed the island}, \textit{village}, \textit{meadow}, \textit{bank}, \textit{building} etc. are therefore supplied, while collocations such as \textit{bombed the ship}, \textit{boat} and \textit{army}, which have a lower frequency, are not supplied. For the verb \textit{attack}.v, the specification is: {\textit{settlement}}: eng-30-08672562-n, {\textit{building}}: eng-30-02913152-n, {\textit{military unit}}: eng-30-08198398-n, {\textit{defensive structure}}: eng-30-03171356-n. In FrameNet, the frame element \fename{Victim} is represented by the semantic type \emph{Sentient}, which is not particularly representative for the verbs \textit{bomb}.v and \textit{attack}.v. In the VerbAtlas, the verbs \textit{bomb} and \textit{attack} are part of the frame \textbf{bomb-attack}, and the corresponding semantic role is \textbf{patient} with the selective preference \textbf{object}, which is an overgeneralised representation. Instead, our approach is to identify patterns of nouns that are combined with specific verbs from semantically relevant frames.

Another example is the verbs used in Bulgarian to convey information to the recipient (indirect object), such as \textit{казвам, съобщавам} (say, `convey an information, an opinion, an instruction, etc.'); \textit{разказвам} (tell; narrate, `communicate a story, a fairy tale, etc.'); \textit{обяснявам, разяснявам} (explain, `clarify (something) to someone by describing it in more detail'), etc. All these verbs are assigned to the semantic class verb.communication in WordNet and belong to the semantic frame \framename{Statement} with the definition: This frame contains verbs (and nouns) that communicate the act of a \fename{Speaker} to address a \fename{Message} to some \fename{Addressee} using \fename{Language}. Similarly, a \fename{Topic} can be specified instead of a \fename{Message}.
To illustrate the proposed approach, the corresponding noun fillers for the frame elements \fename{Speaker} and \fename{Message} are shown in the frame \framename{Statement}  evoked by the verb \textit{обяснявам} (explain).

The fillers for the frame element \fename{Speaker} are either nouns that are assigned to the semantic class noun.person in WordNet, or non-sentient nouns whose meaning can express unions of persons, such as \textit{party, ministry, organisation, company}, etc., which denote organisations that are responsible for specific functions, policies or services. In this context, such nouns embody abstract concepts of administrative powers, policy formulation, regulatory oversight, etc., which do not refer to physical, tangible entities, but to the collective functions and responsibilities related to human activities. Therefore, in this case, appropriate nouns in the WordNet synsets should be classified based on their ability to express the collective functions of people.

Regarding the noun fillers of the frame element \fename{Message}, it should be noted that these are nouns that are classified as noun.communication or noun.cognition in WordNet. However, these nouns differ in how they express communication and cognition. Therefore, it is important to develop a technique to eliminate those nouns that cannot be collocated with the verb \textit{обяснявам} `explain' as direct objects. The synset {communication} with the definition `something that is communicated by or to or between people or groups'  is at the top of the hierarchy  of nouns labelled with the semantic class noun.communication. However, this meaning is too abstract to serve as a filler for the frame element \fename{Message}. The same applies to the top synset in the hierarchy, which is labelled with the semantic class noun.cognition: {cognition; knowledge; noesis} with the definition `the psychological result of perception, learning and reasoning'. Although the hyponyms of these synsets are suitable in most cases, some inappropriate synsets appear in the respective subtrees: \emph{receipt} -- `an acknowledgment (usually tangible) that a payment has been made'; \emph{mail} -- `the bags of letters and packages that are transported by the postal service'; and \emph{publication} -- `the communication of something to the public; making information generally known', among others. All non-combinable nouns are concrete and contained in synsets labelled noun.communication. One possible strategy to restrict them is to add another level of noun classification: abstract and concrete nouns.

Two approaches must therefore be combined to correctly determine the appropriate noun classes to fill the positions of the frame elements of a given verb:

\begin{itemize}

\item Selection of the most appropriate synset or the most suitable combination of synsets in the hypernym hierarchy that semantically dominate the corresponding nouns.
 \item Introduction of additional elementary semantic types and classification of nouns based on these types so that correct generalisations can be made. These types include \textit{collective, abstract, concrete} and \textit{agentive}.
 
\end{itemize}

\figref{fig:FF} shows an example of how the information for conceptual frames is stored in Bulgarian FrameNet.

\begin{figure}[ht]
  \includegraphics[width=\textwidth]{figures/FF.png}
  \caption{An extract from the conceptual frame \framename{Applied\_heat\_варя}.}
  \label{fig:FF}
\end{figure}


\subsection{Syntactic section (Valency patterns)}

The Syntactic section comprises two components: the specification of \emph{grammatical categories} and the assignment of \emph{grammatical functions} to syntactic phrases, which are realisations of frame elements. In addition, it contains specifications for implicit realisations and sets of suitable prepositions for prepositional phrases as well as sets for suitable clause-linking components.

Valence patterns associated with frame elements of specific conceptual frames are language-specific, although a comparison with English may show that some patterns are applicable to both languages.

\tabref{tab:my_label6} represents language-specific information in the Syntactic section.

\begin{table}
    \centering
    \begin{tabular}{lll}
    \lsptoprule
      Type of information & Semantic frame  & Conceptual frame  \\\midrule
      Implicitness & language-specific & language-specific\\
      Grammatical category & language-specific & language-specific\\
      Grammatical function & language-specific & language-specific\\
      \lspbottomrule
    \end{tabular}
    \caption{Type of information in the Syntactic section}
     \label{tab:my_label6}
  \end{table}  

The main source for the annotation and extraction of the valency patterns is the Bulgarian sense-annotated corpus \citep{2012-Bulgarian-Sense-annotated}. It was selected due to its relatively extensive coverage, which includes 17,041 verbs with 6,612 unique senses. The high variance in the corpus is achieved by selecting about 100 word excerpts from each of the 500 texts of the Bulgarian Brown Corpus. These samples are selected based on the density of the most frequently occurring open-class lemmas, applying heuristics to ensure a fair representation across different parts of speech and a broader coverage of lemmas. In addition, the corpus is characterised by the fact that all the words it contains have been manually annotated for their senses from the Bulgarian WordNet. So far, the full annotation includes verbs related to communication, contact, change, perception, emotion and movement, with a total of about 2,500 unique meanings \citep{skoeva2024}. The annotation led to the grouping of lexical units (as of September 2024) into 551 conceptual frames, which are linked to 247 semantic frames.

For the realisations of each frame element, a \textbf{syntactic description} is provided on the basis of the annotation. For example, the verb \textit{режа} (cut, `WN: separate with or as if with an instrument') from the frame \framename{Cutting} is characterised by the following syntactic description of the frame elements:
\begin{itemize}
    \item \fename{Agent}: pro-drop, NP, Subject
    \item \fename{Object}: optional, NP, Object
    \item \fename{Parts}:  optional, PP, Object (\textit{на} `of')
    \item \fename{Instrument}: optional, PP, Object (\textit{с} `with')
\end{itemize}

Individual words are annotated for the part of speech and dependency relations corresponding to grammatical functions (roles) are automatically assigned by the Natural Language Processing Pipeline \citep{koeva-etal-2020-natural}, so that the grammatical categories and universal dependencies are automatically annotated before the correspondences between grammatical categories, grammatical roles and frame elements are manually annotated.

\subsubsection{Grammatical categories}

The \textbf{grammatical categories} in the syntactic realisation of the core frame elements in Bulgarian are: NP (noun phrase), PP (prepositional phrase), S (clause), AP (adjective phrase), ACCCL (obligatory accusative clitic) and DATCL (obligatory dative clitic). Noun phrase stands for nouns, substantives (lexical and syntactical), pronouns or noun phrases, also coordinative. A prepositional phrase can have a simple structure (a preposition and a noun, a pronoun or a substantive) or a complex structure (a preposition and a noun phrase). The adjective phrase, the noun phrase and the prepositional phrase can be a realisation of a small clause. Obligatory accusative or dative clitics are frame element fillers of multiword verbs, although as such they are part of the complex lemma. Since in such cases the accusative or dative clitics are instances of core frame elements, we repeat the clitics both in the lemma and in the valency patterns.

There can be more than one valency pattern for a single verb with a unique meaning. To account for this, two strategies are used:
\begin{itemize}
 \item A given frame element can have more than one type of realisation, e.g. a noun phrase or a clause;
 \item In some conceptual frames, special relations may exist between the competing realisations of frame elements.
\end{itemize}

 For example, the lexical unit \textit{убеждавам} (persuade) is part of the frame \framename{Suasion} which has a frame element \fename{Topic} with a definition: The general item or items that are the focus of the \fename{Content} of the \fename{Speaker}'s message.
 The frame element \fename{Topic} can have different syntactic realisations: a prepositional phrase or a clause.

 
\begin{exe}
 \ex  \label{ch01:ex:44}
 \textit{Днес} [\textit{все повече предприемачи}]$_{\feinsub{Spkr}}$ \textit{\textbf{УБЕЖДАВАТ}} [\textit{клиентите}]$_{\feinsub{Addr}}$ \\
$[$PP \textit{в предимствата на търговската марка}]$_{\feinsub{Topic}}$. \\
 `\textit{Nowadays, more and more entrepreneurs are convincing customers in the advantages of the trademark}.'
 \end{exe} 

 
\begin{exe}
 \ex  \label{ch01:ex:45}
 \textit{Днес} [\textit{все повече предприемачи}]$_{\feinsub{Spkr}}$ \textit{\textbf{УБЕЖДАВАТ}} [\textit{клиентите}]$_{\feinsub{Addr}}$,  \\
$[$S \textit{че търговската марка има предимства}]$_{\feinsub{Top}}$. \\
 `\textit{Nowadays, more and more entrepreneurs are convincing customers that the trademark has advantages}.'
 \end{exe} 

In the annotation, various prepositions and clause-linking phrases can be collected for a specific frame element, which is expressed as a prepositional phrase. The types of clauses vary according to the type of linking, whether by interrogative pronouns or conjunctions; the complex linking phrases or complementisers are annotated and summarised accordingly. The combinations of prepositions and clause types form clusters within different verb classes, each associated with conceptual frames belonging to different semantic domains.

We have implemented an approach that complements the annotation to encode possible valency patterns, even if they do not yet occur in the corpus examples. A different approach would significantly reduce their number. For example, the lexical unit \textit{разказвам} (`tell, narrate; communicate a story, tale, etc.') from the conceptual frame \framename{Statement} is linked to 22 valency patterns that only take the core frame elements into account. These patterns include options for expressing the frame element \fename{Speaker} with a Definite Null Instantiation or a noun (a noun, a noun phrase or a pronoun, with the exception of possessive and reflexive pronouns): the frame element \fename{Message} with Indefinite Null Instantiation, a noun (a noun, a noun phrase, an accusative personal pronoun clitic) or a complement clause; the frame element \fename{Addressee} with an Indefinite Null Instantiation or a prepositional phrase introduced by the preposition \textit{на} `to' or replaced by a dative personal pronoun clitic, and the frame element \fename{Topic} with Indefinite Null Instantiation or a prepositional phrase introduced by the preposition \textit{за} `about'.

\subsubsubsection{Null instantiations of frame elements}

The phrases expressing the frame elements are: (a) (in rare cases in Bulgarian) obligatorily explicit or (b) non-obligatory explicit, which means that the potential for syntactic realisation of the phrase is present, but its explicitness is not obligatory because it is understood from the context in a broader sense (verb morphology, preceding text, extra-linguistic information, etc.). A special case is a pronominal drop in the subject position.

In FrameNet, the annotation for zero instantiation corresponds to the alternatives for the omission of core frame elements in Bulgarian FrameNet: pro-drop subjects and implicit (optional) objects. If the missing part is understood in the linguistic or conversational context, this is called Definite Null Instantiation in FrameNet and corresponds to pro-drop subjects. Indefinite Null Instantiation is indicated by the absence of objects in verbs such as \textit{eat}, \textit{read}, \textit{drink}, etc., i.e. in cases where transitive verbs are used intransitively. As in FrameNet, construct-related omitted constituents can also occur here, such as the omitted subject of imperative sentences, the omitted agent of passive sentences, etc.

\subsubsection{Grammatical functions}

The \textbf{grammatical functions} used in Bulgarian FrameNet are subject, direct object, indirect object, adverbial, subject clause, object clause, adverbial clause and small clause. Compared to FrameNet, they are more detailed, especially with regard to the clauses.

The frame elements associated with the subject of Bulgarian verbs can be characterised as follows: They can have an explicitly or implicitly expressed subject with a complete paradigm; alternatively, they can have a subject explicitly or implicitly expressed in the third person; or they can have no subject at all. The frame elements corresponding to the objects of Bulgarian verbs can be classified as follows: with a single NP object; with an NP object and a complement clause; with an NP object and PP objects, regardless of their number; with an NP object, PP objects (regardless of their number) and a complement clause; with PP objects, regardless of their number; with PP objects (regardless of their number) and a complement clause; with a complement clause; and without objects. In addition, an AdvP predicate modifier, SC (small clause) NP, SC (small clause) PP and SC (small clause) AP can occur as realisations of some frame elements.

While the grammatical functions are largely predictable due to the nature of the frame elements and the grammatical categories of their syntactic realisations, the encoding of this information mainly illustrates the manifestation of different types of diatheses registered for a lexical unit.

Our approach aims to maximise coverage by explicitly encoding the grammatical functions for potential syntactic realisations of a given frame element alongside the annotation process.

\figref{fig:FFF} provides an overview of the encoding of syntactic information (valency patterns) in the Bulgarian FrameNet.

\begin{figure}[ht]
  \includegraphics[width=\textwidth]{figures/FFF.png}
  \caption{Illustration for a part of the Syntactic section in the BulFrame system.}
  \label{fig:FFF}
\end{figure}

\section{Conclusions}

The study presents the semantic frames of FrameNet on the basis of Charles J. Fillmore's frame semantics theory and outlines the main assumptions underlying the development of the Bulgarian FrameNet.

A core premise of our research is that the semantic frames developed for English as part of the Berkeley FrameNet project can also be applied to the semantic analysis of Bulgarian. In other words, we claim that semantic frames represent a language-independent repository of semantic descriptions in which the language-independent components of the semantic frames are integrated into abstract structures known as superframes.

Most semantic frames can actually be used for analysing Bulgarian, especially through the use of superframes and conceptual frames that facilitate the alignment of semantic frames with Bulgarian data, including idiosyncratic differences and lexical diatheses. The main components of this description, lexical units and frame elements, are enriched in the following way: lexical units are enriched with grammatical, lexical and semantic information, such as semantic classes and semantic relations, while frame elements are associated with noun phrases that represent how they can be realised and combined with the target lexical units.

The Bulgarian FrameNet will be integrated into the network of equivalent or identical linguistic descriptions for other languages by using the language-universal information in the FrameNet to describe Bulgarian lexical units.

The unified representation of semantic and syntactic information is important for the system analysis, description and use of Bulgarian, both for natural language processing and as a source of linguistic knowledge. The syntactic patterns resulting from the realisation of frame elements can be studied in order to draw classificatory and typological conclusions about Bulgarian verb classes. This also applies to the patterns of prepositions and clause types and, to the highest degree, to the semantic classes of nouns that coexist with the target lexical units.

\section*{Abbreviations}

\begin{multicols}{2}
\begin{tabbing}
MMMM \= Adjective\kill
A or a \> Adjective \\
ACCCL \> Obligatory accusative clitic \\
\scshape Addr \> \fename{Addressee} \\
AdvP \> Adverbial phrase \\
AP \> Adjectival phrase \\
%Auth \> \fename{Author} \\
BF \> BulFrame \\
%BodP \> \fename{Body\_part} \\
%Cont \> \fename{Content} \\
CPA \> Corpus Pattern Analysis \\
DATCL \> Obligatory dative clitic \\
%Exp \> \fename{Experiencer} \\
FE \> Frame element \\
FN \> FrameNet \\
ILI \> Inter-lingual index \\
IPFV \> Imperfective \\
LI \> Language-independent \\
LS \> Language-specific \\
MWE \> Multiword expression \\
N or n \> Noun \\
NP \> Noun phrase \\
PFV \> Perfective \\
%Perp \> \fename{Perpetrator} \\
PP \> Prepositional phrase \\
S \> Subordinate clause \\
SC \> Small clause \\
SPKR \> {Speaker} \\
%Src \> \fename{Source} \\
%Thm \> \fename{Theme} \\
TOPIC \> {Topic} \\
V or v \> Verb \\
WN \> WordNet \\
\end{tabbing}
\end{multicols}

\section*{Acknowledgements}

This research is carried out as part of the project \emph{Enriching Semantic Network WordNet with Conceptual Frames} funded by the Bulgarian National Science Fund, Grant Agreement No. KP-06-H50/1 from 2020.

%\section*{Contributions}
%John Doe contributed to conceptualisation, methodology, and validation.
%Jane Doe contributed to the writing of the original draft, review, and editing.

%\section*{Contributions}
%John Doe contributed to conceptualisation, methodology, and validation.
%Jane Doe contributed to the writing of the original draft, review, and editing.

{\sloppy\printbibliography[heading=subbibliography,notkeyword=this]}
\end{document}

\chapter{Strukturelle Komplexität in der Linguistik}\label{2}

\section{Grundsätzliche Überlegungen zur linguistischen Komplexität}\label{2.1}

\subsection{Wissenschaftsgeschichtlicher Überblick}\label{2.1.1}

\citet[1--5]{Kusters2003} und \citet{Sampson2009} geben einen ausgezeichneten Überblick darüber, ob, und wenn ja, auf welche Art und Weise, die Komplexität der Sprache in der Linguistik des 19. und 20. Jh. eine Rolle gespielt hat. Deswegen werden hier nur die wichtigsten Punkte kurz zusammengefasst, was natürlich eine gewisse Verallgemeinerung als Konsequenz hat. Die Übersicht ist chronologisch organisiert.

Im philosophischen Idealismus und der Romantik des 19. Jh. wurde das vermeintlich hohe Niveau europäischer Kultur mit der vorgeblich hohen Komplexität europäischer Sprachen in Verbindung gebracht \citep[2]{Kusters2003}. \citet{Humboldt1836} bringt dies wie folgt auf den Punkt: „Die Sprache ist gleichsam die äußerliche Erscheinung des Geistes der Völker; ihre Sprache ist ihr Geist und ihr Geist ihre Sprache […]“ \citep[53]{Humboldt1836}. Der angenommene Zusammenhang zwischen Sprache und Denken kann ebenfalls mit Humboldt illustriert werden, der davon ausging, dass flektierende Sprachen sich für komplexe Gedanken besser eignen als andere Sprachtypen \citep[3]{Kusters2003}. Mit den Junggrammatikern gegen Ende des 19. Jh. hatten Denken und Kultur keine so große Bedeutung mehr in der Sprache, wie dies im Idealismus und in der Romantik der Fall war. Sprache wurde als Naturphänomen (vgl. Ausnahmslosigkeit der Lautgesetze) angesehen und die Verwandtschaften zwischen den Sprachen (vgl. Stammbaumtheorie) standen im Vordergrund \citep[3]{Kusters2003}. Auch die Strukturalisten interessierten sich (zumindest in erster Linie) nicht für die Komplexität in der Sprache. Vielmehr ging es darum, ein umfassendes Instrument zu haben, um alle Sprachen beschreiben zu können \citep[4]{Kusters2003}. In der generativen Grammatik spielte Komplexität nur eine geringe Rolle, da Sprache als Teil der menschlichen Biologie und nicht der Kultur gesehen wird \citep[6]{Sampson2009}. Deswegen steht die I-Language im Fokus der Betrachtung, die E-Language wird für die linguistische Analyse als eher unwichtig angesehen. Ziel ist es, Übereinstimmungen zwischen Sprachen und den Sprachen zugrunde liegenden Universalien herauszuarbeiten \citep[4]{Kusters2003}. Unterschiede zwischen den Sprachen werden durch Unterschiede in der Derivation erklärt (z.\,B.\ Transformationen, Move und Merge etc.). Kürzlich erschienene Bände zeigen jedoch, dass linguistische Komplexität vermehrt auch innerhalb von generativen Modellen analysiert wird (z.\,B.\ \citealt{Culicover2013}, \citealt{TrotzkeBayer2015}). Die Aufsätze in dem von \citet{TrotzkeBayer2015} herausgegebene Band beschäftigen sich vor allem mit syntaktischer Komplexität (z.\,B.\ Rekursion, Derivation, Einbettung etc.) und deren Erwerb, die Monografie von \citet{Culicover2013} mit dem Sprachwandel, Spracherwerb und Prozessierung und deren Zusammenhang bzw. Einfluss auf linguistische Komplexität.

Im 20. Jh. stand also die Komplexität der Sprache nicht im Fokus des Interesses. Vielmehr galt, verallgemeinernd ausgedrückt, dass sich Sprachen in ihrer Komplexität nicht unterscheiden, was als \is{Equi-Complexity-Hypothese}\textit{Equi-Com\-ple\-xi\-ty-Hy\-po\-the\-se} bezeichnet wird. Dies kann auch als Gegenreaktion auf die Annahmen aus dem 19. Jh. verstanden werden. Gleichzeitig sprachen besonders Variationslinguisten im 20. Jh. immer wieder von Komplexitätsunterschieden, z.\,B.\ \citeauthor{Ferguson1959}s \citeyearpar{Ferguson1959} Unterscheidung zwischen \textit{High} und \textit{Low Varieties}. Darauf wird im anschließenden Kapitel noch genauer eingegangen.

Im 21. Jh. ist ein wachsendes Interesse an struktureller Komplexität in der Sprache zu beobachten. Dazu sind bereits etliche Aufsatzsammlungen und Dissertationen entstanden, wie, um nur einige wenige zu nennen, \citet{Kusters2003}, \citet{MiestamoSinnemäkiKarlsson2008}, \citet{SampsonGilTrudgill2009}, \citet{Sinnemäki2011}, \citet{SzmrecsanyiKortmann2012}. Dabei stehen vor allem folgende Fragen im Zentrum: Sind alle Sprachen gleich komplex? Wird höhere Komplexität in einem Subsystem (z.\,B.\ Morphologie) durch niedrigere Komplexität in einem anderen Subsystem (z.\,B.\ Syntax) ausgeglichen? Wie kann Komplexität gemessen werden? Wie können Unterschiede in der Komplexität zwischen Sprachen erklärt werden? \citep{BaechlerSeiler2016a}. Warum die linguistische Komplexitätsforschung seit wenigen Jahren und relativ plötzlich so viel Aufwind erfahren hat, liegt vermutlich in der klaren, expliziten Entkoppelung der früher angenommenen Verbindung zwischen besonders komplex und besonders wertvoll, d.h., dass eine vor allem flexionsmorphologisch besonders komplexe Sprache auch besonders wertvoll sei. Versteht man jedoch eine Sprache als ein System von Regeln, die vorgeben, wie das Lexikon aufgebaut wird und wie mit diesem Lexikon Sätze generiert werden, müsste man vom Gegenteil ausgehen.\largerpage[2] Denn ein solches System zur Informationsverarbeitung wäre wohl dann besonders elegant, wenn es mit sehr wenigen Regeln (also möglichst geringer Komplexität) auskommt, folglich sehr effizient ist.

\subsection{Hypothese der gleichen Komplexität (\textit{Equi-Complexity-Hypothese})}\label{2.1.2}

Im Strukturalismus des 20. Jh. wurde generell angenommen – insofern es überhaupt diesbezügliche Äußerungen gibt – dass sich Sprachen in ihrer Gesamtkomplexität nicht signifikant unterscheiden. Diese Hypothese wird \is{Equi-Complexity-Hypothese}\textit{Equi-Com\-ple\-xi\-ty-Hy\-po\-the\-se} genannt. Ihr prominentester Vertreter ist wohl \citet{Hockett1958}, der feststellt, dass „[…] impressionistically it would seem that the total grammatical complexity of any language, counting both morphology and syntax, is about the same as that of any other. This is not surprising, since all languages have about equally complex jobs to do […]“ \citep[180]{Hockett1958}. Wenn man also die Komplexität aller Subsysteme messen würde, käme unter dem Strich heraus, dass alle Sprachen ungefähr gleich komplex sind. \citet{Hockett1958} sagt jedoch nichts dazu, welche die relevanten Subsysteme sind und wie deren Komplexität gemessen werden kann. Sehr verwundert und pointiert äußert sich \citet{Sampson2009} über den angenommenen Ausgleichsmechanismus in Sprachen:

\begin{quote}
If it really were so that languages varied greatly in the complexity of subsystem \textit{X}, varied greatly in the complexity of subsystem \textit{Y}, and so on, yet for all languages the totals from the separate subsystems added together could be shown to come out the same, then I would not agree with Hockett in finding this unsurprising. To me it would feel almost like magic. \citep[2--3]{Sampson2009}
\end{quote}

\noindent
Obwohl im 20. Jh. die \is{Equi-Complexity-Hypothese}\textit{Equi-Com\-ple\-xi\-ty-Hy\-po\-the\-se} dominierte, sprachen besonders Variationslinguisten immer wieder von Komplexitätsunterschieden zwi-\linebreak schen Sprachen. Der einflussreichste Vertreter ist hier u.a. wohl \citet{Ferguson1959}, der \textit{High Varieties} von \textit{Low Varieties} differenziert: „One of the most striking differences between H[igh] and L[ow] in the defining languages is in the grammatical structure: H has grammatical categories not present in L and has an inflectional system of nouns and verbs which is much reduced or totally absent in L“ \citep[333]{Ferguson1959}. Es bleibt aber offen, wie diese Unterschiede quantifiziert werden können. In der Folge sollen nun einige Studien vorgestellt werden, die zeigen, dass es Ausgleichstendenzen zwischen den Subsystemen innerhalb einer Sprache gibt, wie auch solche Studien, die das Gegenteil belegen. Diese Liste versteht sich als kurzer Überblick mit Fokus auf den Resultaten und erhebt keinen Anspruch auf Vollständigkeit.

Sowohl \citet{Juola2008} als auch \citet{EhretSzmrecsanyi2016} verwenden ein informationstheoretisches Maß (Kolmogorov-Komplexität und Ziv-Lem\-pel-Kom\-ple\-xi\-tät), um Komplexität zu messen, womit die Menge an Informationen und Redundanzen ermittelt werden kann \citep[93]{Juola2008}. Konkret wird Komplexität durch das Verzerren bestimmter Subsysteme und durch Komprimierung gemessen. Die Datengrundlage bilden große Textkorpora in sechs Sprachen: bei \citet{Juola2008} die Bibel, bei \citet{EhretSzmrecsanyi2016} das Markusevangelium. \citet{Juola2008} kann zeigen, dass sich Sprachen in ihrer Komplexität nicht wesentlich unterscheiden \citep[106]{Juola2008}, während die Resultate von \citet{EhretSzmrecsanyi2016} mit einer vergleichbaren Methode das Gegenteil demonstrieren \citep[78]{EhretSzmrecsanyi2016}. Beide Studien können aber mit genau derselben Messmethode einen Ausgleich zwischen morphologischer und syntaktischer Komplexität nachweisen (\citealt[104]{Juola2008}; \citealt[79--80]{EhretSzmrecsanyi2016}).

\citet{Sinnemäki2008} untersucht in 50 Sprachen die Markierung von Agens und Patiens, wozu vier Strategien existieren, aber nur die letzten drei als strukturelle Strategien berücksichtigt werden: Lexikon, Wortstellung, \textit{Head Marking} und \textit{Dependent Marking} \citep[68]{Sinnemäki2008}. Gemessen wird, wie oft eine bestimmte Kodierungsstrategie verwendet wird, um Agens und Patiens zu unterscheiden \citep[72]{Sinnemäki2008}. \citet{Sinnemäki2008} stellt fest, dass in einigen Subdomänen Ausgleichstendenzen zu beobachten sind, vor allem verallgemeinernd zwischen Wortstellung und \textit{Dependent Marking} \citep[84--85]{Sinnemäki2008}. Die meisten potentiellen Korrelationen jedoch, die geprüft wurden, waren äußerst gering oder inexistent, weshalb Trade-Offs als allgemeines Prinzip verworfen werden können \citep[84]{Sinnemäki2008}.

\citet{Shosted2006} überprüft in 32 Sprachen einen möglichen Ausgleich zwischen morphologischer und phonologischer Komplexität. Dazu zählt er die Anzahl\linebreak möglicher Silben einer Sprache und die Anzahl der Marker in der Verbflexion (als eine Art Synthesegrad) \citep[9–17]{Shosted2006}. Die Korrelation zwischen den Messresultaten ist leicht positiv, aber statistisch nicht signifikant \citep[1]{Shosted2006}.

Auch \citet{Nichols2009} konnte keine negativen Korrelationen zwischen verschiedenen Komponenten der Grammatik feststellen \citep[119]{Nichols2009}. Sie untersucht 68 Sprachen und fünf Komponenten: Phonologie (u.a. Anzahl der Qualitätsunterschiede der Vokale), Synthese (u.a. Anzahl am Verb markierter Kategorien), Klassifikation (u.a. Genuskongruenz), Syntax (u.a. Anzahl unterschiedlicher Abfolgen von nominalen und pronominalen Argumenten sowie dem Verb) und Lexikon (u.a. Anzahl Suppletivpaare in neun Paaren Vollverb/kausatives Verb) \citep[113]{Nichols2009}.

Einen Ausgleich zwischen Subsystemen der Grammatik konnten nur die informationstheoretisch basierten Methoden nachweisen. Diese Methoden ermöglichen zwar im Prinzip, gesamte Subsysteme zu messen.\largerpage Sie sind jedoch probabilistisch und erlauben keinen Einblick in die Details der Ausgleiche.\largerpage Dies wäre jedoch wichtig, um die eventuellen Ausgleichsmechanismen zu verstehen und um ein Ausgleichsmuster abstrahieren zu können, denn so wäre dann ein Vergleich dieser Muster von verschiedenen Sprachen möglich. Des Weiteren könnten diese Ausgleichsmuster und ihr Vergleich für die linguistische Theoriebildung nutzbar gemacht werden. In diese Richtung gehen eher Studien wie jene von \citet{Sinnemäki2008} und \citet{Nichols2009}. Das Problem dieser Methoden jedoch ist, dass immer nur ein Ausschnitt einer Sprache oder eines Subsystems analysiert werden kann. Dafür ermöglichen sie aber einen detaillierteren Einblick in die inneren Vorgänge der Sprachen.

Dies tritt in Konflikt mit der \is{Equi-Complexity-Hypothese}\textit{Equi-Com\-ple\-xi\-ty-Hy\-po\-the\-se}, die implizit verlangt, dass die Gesamtkomplexität einer Sprache gemessen werden soll, d.h. alle Subsysteme. Erstens gibt es aber keinen Konsens über die Taxonomie der linguistischen Subsysteme. Zweitens stellt sich die Frage, wie die Komplexität der einzelnen Subsysteme gemessen werden kann, sodass eine Verrechnung der Resultate die Gesamtkomplexität darstellt. \citet{Miestamo2008} bezeichnet dies als das Problem der Repräsentativität und als das Problem der Vergleichbarkeit:

\begin{quote}
The problem of representativity means that no metric can pay attention to all aspects of grammar that are relevant for measuring global complexity. Even if this were theoretically possible, it would be beyond the capacities of the mortal linguist to exhaustively count all grammatical details of the languages studied […]. The problem of comparability is about the difficulty of comparing different aspects of grammar in a meaningful way, and especially about the impossibility of quantifying their contributions to global complexity. \citep[30]{Miestamo2008}
\end{quote}

\noindent\largerpage
Beispielsweise stellt sich die Frage, wie viele Unterscheidungen im Aspektsystem gleich komplex sind und wie viele Unterscheidungen im Tempussystem \citep[7]{Miestamo2006}. Daraus kann geschlossen werden, dass (zumindest vorerst) die Messung einzelner Subsysteme oder Teile von Subsystemen zu bevorzugen ist. \citet{Miestamo2008} nennt dies \textit{Local Complexity} im Gegensatz zur \textit{Global} oder \textit{Overall} \textit{Complexity}, womit die Gesamtkomplexität einer Sprache gemeint ist \citep[29]{Miestamo2008}. Viel grundsätzlicher geht es darum, dass nur Vergleichbares verglichen werden kann. \citet{Miestamo2008} plädiert dafür, formale Aspekte der Grammatik (z.\,B.\ morphologische Systeme) von den funktionalen Aspekten (z.\,B.\ Kodierung von Tempus, Aspekt) zu unterscheiden und nur innerhalb dieser Bereiche einen Komplexitätsvergleich zwischen Sprachen vorzunehmen \citep[31]{Miestamo2008}. Ein gutes Beispiel dafür ist die oben zitierte Arbeit von \citet{Sinnemäki2008}. Dabei wird die Markierung von\largerpage Argumenten gemessen, wofür die Sprache unterschiedliche Mittel (z.\,B.\ morphologische und syntaktische) zur Verfügung hat.

\subsection{Hat Komplexität eine Funktion in der Sprache?}\label{2.1.3}

Die \is{Equi-Complexity-Hypothese}\textit{Equi-Com\-ple\-xi\-ty-Hy\-po\-the\-se}, die vorwiegend auf den Strukturalismus zurückgeht, nimmt an, dass alle Sprachen gleich komplex sind und höhere Komplexität in einem Subsystem durch niedrigere Komplexität in einem anderen Subsystem kompensiert wird. Dies impliziert, dass Komplexität eine Funktion innerhalb der Sprache erfüllt. In \sectref{2.1.1} wurde gezeigt, dass besonders in der ersten Hälfte des 19. Jhs. davon ausgegangen wurde, dass die strukturelle Komplexität einer Sprache mit der (u.a. kulturellen) Entwicklung ihrer Sprecher korreliert. Verallgemeinernd kann man also Folgendes festhalten. Sowohl im 19. Jh. als auch im 20. Jh. wurde angenommen, dass strukturelle Komplexität eine Funktion in der Sprache innehat. Im 19. Jh. wurde diese Annahme verwendet, um die angebliche Fortschrittlichkeit besonders Europas zu erklären. Im 20. Jh. ging man davon aus, dass die kommunikativen Anforderungen an eine Sprache immer dieselben sind, folglich alle Sprachen den gleichen Grad an Komplexität aufweisen müssen. Ein Beispiel dafür ist die bereits oben zitierte Aussage von \citet{Hockett1958}, der behauptet, dass Ausgleichstendenzen in der Komplexität von verschiedenen Subsystemen zu erwarten sind, „[…] since all languages have about equally complex jobs to do […]“ \citep[180]{Hockett1958}. Dem entgegnet \citet{Sampson2009} mit einer ganz grundsätzlichen Überlegung: „[…] but it seems to me very difficult to define the job which grammar does in a way that is specific enough to imply any particular prediction about grammatical complexity“ \citep[2]{Sampson2009}.

Es ist also sicher nicht möglich, jene Aufgaben umfassend zu beschreiben, die eine Sprache erfüllen muss. Noch grundlegender ist jedoch die Frage, ob eine komplexe Aufgabe zwangsläufig mit einer komplexen Grammatik gelöst werden muss: Weshalb sollten komplexe Konzepte auch nur durch eine komplexe Grammatik ausgedrückt werden können und umgekehrt, einfache Konzepte durch eine einfache Grammatik? Können wir doch mit derselben Sprache sowohl einfache wie auch schwierige Dinge auf eine mehr oder weniger komplexe Art formulieren. Folglich gibt es zwischen der Komplexität der Grammatik und der Komplexität der Dinge, die wir tun, keinen Zusammenhang. Dass Grammatik oft eben gerade nicht funktional ist, bringt \citet{Gil2009} wie folgt auf den Punkt:

\begin{quote}
These facts cast doubt on a central tenet of most functionalist approaches to language, in accordance with which grammatical complexity is there to enable us to communicate the messages we need to get across. In spite of overwhelming evidence showing that diachronic change can be functionally motivated, the fact remains that language is hugely dysfunctional. Just think of all the things that it would be wonderful to be able to say but for which no language comes remotely near to providing the necessary expressive tools. For example, it would be very useful to be able to describe the face of a strange person in such a way that the hearer would be able to pick out that person in a crowd or a police line-up. But language is completely helpless for this task, as evidenced by the various stratagems police have developed, involving skilled artists or, more recently, graphic computer programs, to elicit identifying facial information from witnesses – in this case a picture actually being worth much more than the proverbial thousand words. Yet paradoxically, alongside all the things we’d like to say but can’t, language also continually forces us to say things that we don’t want to say; this happens whenever an obligatorily marked grammatical category leads us to specify something we would rather leave unspecified. English, famously, forces third person singular human pronouns to be either masculine or feminine; but in many contexts we either don’t know the person’s gender or actually wish to leave it unspecified […]. \citep[32]{Gil2009}
\end{quote}

\noindent
Strukturelle Komplexität hat folglich nichts mit der Effizienz und Expressivität einer Sprache als Mittel zur Kommunikation zu tun \citep[2]{Miestamo2006}. Auch ist sie kein Symptom für Zivilisation und Fortschrittlichkeit. Vielmehr kann Grammatik als ein System gesehen werden, dessen \isi{Variation} in der Komplexität primär systeminterne Ursachen hat:

\begin{quote}
Rather than having evolved in order to enable us to survive, sail boats, and do all the other things that modern humans do, most contemporary grammatical complexity is more appropriately viewed as the outcome of natural processes of self-organization whose motivation is largely or entirely system-internal. In this respect, grammatical complexity may be no different from complexity in other domains, such as anthropological complexity, economics, biology, chemistry, and cosmology, which have been suggested to be governed by general laws of nature pertaining to the evolution of complex systems. \citep[32--33]{Gil2009}
\end{quote}

\noindent
Wenn Grammatik also ein sich selbst organisierendes System ist, dann besteht ebenfalls die Möglichkeit, dass auch die Komplexität der Grammatik verschiedener Sprachen unterschiedlich hoch ist. Die grammatischen Systeme von Sprachen mögen zwar untereinander mehr Ähnlichkeiten aufweisen als im Vergleich mit z.\,B.\ chemischen Systemen. Trotzdem gibt es keinen Grund, weshalb die grammatischen Systeme aller Sprachen gleich komplex sein sollten.

\section{Strukturelle Komplexität: Definitionen, Messmethoden und Einflussfaktoren}\label{2.2}%%%\rohead{\thesection\hspace{0.5em}Definitionen, Messmethoden und Einflussfaktoren}
\sectionmark{Definitionen, Messmethoden und Einflussfaktoren}

Dieser Abschnitt gibt einen Überblick über die Arbeiten, die Unterschiede in der strukturellen Komplexität u.a. durch sprachexterne Faktoren zu erklären versuchen. Alle bereits publizierten Untersuchungen zu diskutieren, würde den hiesigen Rahmen sprengen. Vielmehr soll ein Ausschnitt jener Studien vorgestellt werden, die sich mit den sprachexternen Faktoren beschäftigen, welche für die vorliegende Arbeit zentral sind: kleine, \isi{isolierte Sprachgemeinschaften} mit wenig \isi{Sprachkontakt} und einem engen Netzwerk vs. große, nicht \isi{isolierte Sprachgemeinschaften} mit viel \isi{Sprachkontakt} (vielen L2-Ler\-nern) und losen Netzwerken. Unterschiedliche Arbeiten dazu werden in den  \sectref{2.2.3} und \sectref{2.2.4} vorgestellt. Zuerst sollen jedoch die ersten publizierten Überlegungen zu möglichen Einflussfaktoren kurz eingeführt werden (\sectref{2.2.1}). Anschließend werden Arbeiten erörtert, in denen sprachexterne Faktoren herangezogen werden, die in der vorliegenden Arbeit nicht im Fokus stehen, die aber die Diskussion über den Zusammenhang zwischen struktureller Komplexität und sprachexternen Faktoren beeinflusst haben (\sectref{2.2.2}). Es handelt sich dabei um die Faktoren Alter, Geschlecht, Schicht, Region und \isi{Bevölkerungsgröße}. In diesem Überblick stehen folgende Fragen im Vordergrund: Was wird unter struktureller Komplexität verstanden und wie kann diese gemessen werden? Welche Sprachen/Varietäten und welche linguistischen Beschreibungsebenen werden untersucht? Wie werden die Unterschiede in der Komplexität von Sprachen erklärt? In einem abschließenden Abschnitt wird zuerst die mittlerweile etablierte Unterscheidung zwischen absoluter und relativer Komplexität vorgestellt. Anschließend wird eine detailliertere Unterscheidung von Komplexitätstypen eingeführt, die auf den Philosophen \citet{Rescher1998} zurückgeht und von \citet{MiestamoSinnemäkiKarlsson2008} für linguistische Phänomene adaptiert wurde. Zuletzt werden die in den vorangehenden Abschnitten \sectref{2.2.1}–\sectref{2.2.4} verwendeten Definitionen struktureller Komplexität zusammengefasst und kategorisiert (\sectref{2.2.5}).

\subsection{Erste Überlegungen zu möglichen Einflussfaktoren}\label{2.2.1}

Überlegungen zu möglichen sprachexternen Faktoren, die mit der \isi{Variation} in der strukturellen Komplexität zusammenhängen könnten, sind schon relativ alt. Einen Überblick über die ersten Hypothesen in diesem Zusammenhang bieten \citet{BaechlerSeiler2016a}, woran ich mich in der Folge orientiere.

Der vielleicht erste Linguist, der soziale und geografische Faktoren mit der Höhe struktureller Komplexität in Verbindung brachte, ist Roman \citet{Jakobson1929}. Er untersuchte das phonologische System ukrainischer Dialekte und stellte Folgendes fest: Zentrale, sich ausbreitende Dialekte, die von einer homogenen Sprachgemeinschaft gesprochen werden, haben ein kleineres Vokalinventar als die Dialekte an der Peripherie des Sprachgebiets \citep[73]{Jakobson1929}.

\begin{quote}
Cette différence est due, en premier lieu à la tendance conservatrice qui est caractéristique des parlers de la périphérie, et en second lieu à des différences fonctionnelles. Il n’est pas rare d’observer que la tendance à simplifier le système phonologique croît à mesure que grandit le rayon d’emploi d’un dialecte, avec la plus grande hétérogénéité des sujets parlant la langue généralisée. On n’a pas encore, en linguistique, prêté assez attention à la différence essentielle de structure et d’évolution qui existe entre les parlers gravitant vers le rôle de ϰoɩν\'η ou langue commune, et ceux d’usage purement local. \citep[73]{Jakobson1929}
\end{quote}

\begin{quote}
[Dieser Unterschied ist an erster Stelle durch die konservierende Tendenz bedingt, die charakteristisch für die Sprachen an der Peripherie ist, und an zweiter Stelle durch funktionale Unterschiede. Es ist nicht selten zu beobachten, dass die Tendenz, das phonologische System zu vereinfachen, zunimmt, je mehr das Areal, in dem der Dialekt gesprochen wird, größer wird, und mit einer heterogenen Sprachgemeinschaft, die die Lingua franca spricht. In der Linguistik hat man dem wesentlichen Unterschied in der Struktur und der Evolution noch nicht genügend Aufmerksamkeit\linebreak geschenkt, der zwischen den Sprachen existiert, die in Richtung Koiné oder Gemeinsprache tendieren, und jenen, die ausschließlich lokal verwendet werden.] [meine Übersetzung]
\end{quote}

\citet{Jakobson1929} beschreibt hier also geografische und soziale Faktoren, die auf die Funktion einer Sprache Einfluss haben, was wiederum auf den Sprachwandel wirkt. \citeauthor{Hymnes1975}’ \citeyearpar{Hymnes1975} Beobachtungen weisen in eine ähnliche Richtung wie jene von \citet{Jakobson1929}, wobei bei \citet{Hymnes1975} soziale Faktoren der Sprachgemeinschaft im Vordergrund stehen. Er stellt fest, dass kleine Sprachen mit einem engen Netzwerk eine höhere strukturelle Komplexität aufweisen: „This latter process may have something to do with the fact that the surface structures of languages spoken in small, cheek-by-jowl communities so often are markedly complex, and the surface structures of languages spoken over wide ranges less so“ \citep[50]{Hymnes1975}.

Auch \citet{Werner1975} beobachtet Ähnliches wie \citet{Jakobson1929}. Er vergleicht kleine und \isi{isolierte Sprachgemeinschaften}, die Jakobsons peripheren\linebreak Sprachen entsprechen, mit Sprachen mit viel Kontakt, die Jakobsons sich ausbreitenden Sprachen gleichkommen. Bezüglich der möglichen Korrelation zwischen diesen Sprachgemeinschaftstypen und der strukturellen Komplexität einer Sprache stellt \citet{Werner1975} dasselbe wie \citet{Jakobson1929} fest: 

\begin{quote}
Es ist – vermute ich – ganz allgemein ein Kennzeichen kleinerer, isolierter Sprachgemeinschaften, dass sie lange komplizierte Regelsysteme bewahren; großräumige Sprachkontakte und damit verbundene Interferenzen fordern dagegen die Analogien, wie sie ja auch von Kindern und Ausländern gerne gemacht werden. \citep[791]{Werner1975}
\end{quote}

Eine der ersten Arbeiten, in der explizit ein möglicher Zusammenhang zwischen sprachexternen Faktoren und struktureller Komplexität systematischer geprüft wird, ist \citeauthor{Braunmüller1984}s \citeyear{Braunmüller1984} erschienener Aufsatz. Er untersucht die Flexionsmorphologie des Isländischen, Färöischen und Friesischen. Dabei handelt es sich um Sprachen, die nicht nur wenig \isi{Sprachkontakt} aufweisen (mit Nicht-Mut\-ter\-sprach\-lern wird in einer anderen Sprache kommuniziert), sondern auch \isi{geografisch isoliert} sind: Alle drei Sprachen werden auf Inseln gesprochen, Friesisch zusätzlich an der Küste Schleswig-Holsteins. \citet{Braunmüller1984} zeigt, dass viele unterschiedliche phonologische Regeln gleichzeitig wirken und so Opazität im Paradigma verursachen. Die so entstandene paradigmatische Opazität wird nur sehr wenig durch Analogien ausgeglichen, was besonders bei kleinen und isolierten Sprachen vorkommt \citep[49]{Braunmüller1984}. Dies wird vor allem durch zwei Charakteristika dieser Sprachgemeinschaften begünstigt. Erstens findet der Kontakt mit Nicht-Mut\-ter\-sprach\-lern in einer anderen Sprache statt, sodass diese Sprachen nur sehr selten als L2 erlernt werden und sie nicht als Koiné dienen \citep[49]{Braunmüller1984}. Dies hat zur Konsequenz, dass Sprachen von kleinen und \isi{isolierten Sprachgemeinschaften} nur wenig durch andere Sprachen beeinflusst werden und kaum L2-Sim\-pli\-fi\-zie\-rungen aufweisen. Zweitens sind kleine und \isi{isolierte Sprachgemeinschaften} sehr homogen, weswegen die Sprache wenig strenger Normierung ausgesetzt, die in den Sprachwandel eingreifen könnte \citep[49]{Braunmüller1984}. \citet{Braunmüller1984} bringt also frühere Beobachtungen zu möglichen Einflussfaktoren zusammen: kleine Sprachgemeinschaften mit einem engen Netzwerk und hoher Homogenität, wenige L2-Ler\-ner, kleinräumiger Gebrauch der Sprache, geografische \isi{Isolation} und Peripheralität. 

\subsection{Sprachexterne Faktoren}\label{2.2.2}

In diesem Abschnitt sollen drei Studien vorgestellt werden, die den Zusammenhang zwischen sprachexternen Faktoren und struktureller Komplexität untersuchen. Bei \citet{Sampson2001} handelt es sich bei den sprachexternen Faktoren um Alter, Geschlecht, Schicht und Region, bei \citet{HayBauer2007} sowie bei \citet{Sinnemäki2009} um Bevölkerungsgrößen.

\citet{Sampson2009} prüft den hypothetischen Zusammenhang zwischen syntaktischer Komplexität und Alter, Geschlecht, Schicht sowie Region, wobei es sich also um demografische Daten handelt. Die Datengrundlage bildet sein eigens erstelltes Korpus ‘Christine’, das spontansprachliche Äußerungen enthält und auf dem British National Corpus basiert \citep[57--58]{Sampson2009}. Syntaktische Komplexität wird als die Einbettungstiefe definiert: „For a sentence to be ‘simple’ or ‘complex’ in traditional grammatical parlance refers to whether or not it contains subordinate clause(s)“ \citep[58]{Sampson2009}. Da in der Spontansprache der Anfang und das Ende von Sätzen nicht immer eindeutig bestimmt werden können, wird die Einbettungstiefe auf der Ebene des Wortes definiert: „The present research treats degree of embedding as a property of individual words. Each word is given a score representing the number of nodes in the CHRISTINE ‘lineage’ of that word […] which are labeled with clause categories“ \citep[59]{Sampson2001}. In einem ersten Durchgang fließen die Äußerungen aller Informanten in die Analyse ein, in einem zweiten Durchgang nur die Äußerungen jener Informanten, die älter als 16 Jahre sind. Zwischen den fünf untersuchten Regionen (Süd- und Nordengland, Wales, Schottland, Nordirland) gibt es keinen signifikanten Unterschied in der Komplexität, unabhängig davon, ob die unter 16-Jährigen mit eingeschlossen sind oder nicht \citep[62, 64, 66]{Sampson2009}. Dasselbe gilt bezüglich der sozialen Schicht \citep[64]{Sampson2009}; die Resultate werden aber annähernd signifikant, wenn die unter 16-Jährigen ausgeschlossen werden (von mehr zu weniger komplex): ausgebildet-handwerklich, leitende und technische Berufe, ausgebildet-nicht handwerklich, teils ausgebildet/ohne Ausbildung \citep[67]{Sampson2009}. Dieses doch eher erstaunliche Ergebnis die soziale Schicht betreffend wird dadurch erklärt, dass die Angaben zur Schicht in diesen Daten am wenigsten verlässlich sind \citep[67]{Sampson2009}. Frauen produzieren etwas häufiger komplexe Sätze (signifikant) als Männer \citep[65]{Sampson2009}. Werden jedoch nur die über 16-Jährigen berücksichtigt, ist das Resultat nicht mehr signifikant \citep[66]{Sampson2009}. Das signifikante Ergebnis wird als Problem der Datengrundlage gedeutet, denn in diesem Sample sind mehr Männer als Frauen \citep[66--67]{Sampson2009}. Die verschiedenen Altersgruppen unterscheiden sich signifikant in der syntaktischen Komplexität, wobei mehr komplexe Äußerungen produziert werden, je älter die Informanten sind \citep[65]{Sampson2009}. Von der Annahme ausgehend, dass sich die Informanten bis 13 Jahre noch im Spracherwerbsprozess befinden, werden in einem weiteren Test die unter 13-Jährigen ausgeschlossen (\citealt{Sampson2009}: 67–70). Auch in diesem Test zeigt sich, dass die syntaktische Komplexität mit dem Alter zunimmt \citep[70]{Sampson2009}. \citet{Sampson2009} interpretiert, „that (while the evidence is not overwhelming) increase in average grammatical complexity of speech appears to be a phenomenon that does \textit{not} terminate at puberty, but continues throughout life“ \citep[70]{Sampson2009}.

\citet{Sinnemäki2009} misst die Komplexität in der Markierung von Agens und Patiens in 50 Sprachen und prüft den Zusammenhang zwischen Komplexität und \isi{Bevölkerungsgröße}. Zur Markierung von Agens und Patiens stehen drei Strategien zur Verfügung: \textit{Head Marking} (am Verb), \textit{Dependent Marking} (an Agens oder Patiens) und die Wortstellung \citep[130]{Sinnemäki2009}. Komplexität wird als die Verletzung des Eine-Bedeutung-Eine-Form-Prinzips definiert. Von diesem Prinzip gibt es zwei Abweichungen, welche die Komplexität erhöhen: Verletzung der Ökonomie und Verletzung der Distinktheit \citep[132]{Sinnemäki2009}. Wird das Prinzip der Ökonomie verletzt, wird zu viel markiert, d.h., die Sprache verwendet mehr als eine Strategie, um Agens und Patiens zu markieren, welche nicht komplementär verteilt sind \citep[133]{Sinnemäki2009}. Wird das Prinzip der Distinktheit verletzt, wird zu wenig markiert, d.h., die Sprachen „use only one strategy but in limited contexts, or allow a lot of syncretism in the head or dependent marking paradigms“ \citep[133]{Sinnemäki2009}. \citet{Sinnemäki2009} listet und zählt die Sprachen, die dem Eine-Bedeutung-Eine-Form-Prinzip entsprechen, und jene Sprachen, die davon abweichen. Die Resultate können wie folgt zusammengefasst werden. Erstens nimmt die Anzahl der Sprachen, die dem Eine-Bedeutung-Eine-Form-Prinzip entsprechen, parallel zur Größe der Sprachgemeinschaft zu \citep[135]{Sinnemäki2009}. Zweitens tendieren Sprachen, die von weniger als 10.000 Sprechern gesprochen werden, dazu, entweder Ökonomie oder Distinktheit zu verletzen. Dabei nimmt die Anzahl der Sprachen, die Ökonomie oder Distinktheit verletzen, ab, wenn die Anzahl der Sprecher zunimmt \citep[135]{Sinnemäki2009}. Da es problematisch ist zu bestimmen, was eine große oder kleine Sprachgemeinschaft ist, wird mit demselben Sample ein weiterer Test gemacht. Dazu werden Grenzwerte bestimmt: Angefangen wird bei 250 Sprechern, der nächste Grenzwert ist das Doppelte des vorangehenden Grenzwertes usw. \citep[135]{Sinnemäki2009}. Es kann festgestellt werden, dass Sprachen, die von einer Sprachgemeinschaft gesprochen werden, die kleiner oder gleich groß wie ein Grenzwert ist, eher Distinktheit oder Ökonomie verletzen. Des Weiteren tendieren Sprachen, die von einer Sprachgemeinschaft gesprochen werden, die größer als ein Grenzwert ist, dem Eine-Bedeutung-Eine-Form-Prinzip zu entsprechen \citep[136--138]{Sinnemäki2009}. Zusammengefasst kann also festgehalten werden, dass Sprachen einer kleineren Sprachgemeinschaft in der Markierung von Agens und Patiens höhere Komplexität aufweisen als Sprachen einer größeren Sprachgemeinschaft.

\citet{HayBauer2007} prüfen einen Zusammenhang zwischen der Größe des Phoneminventars einer Sprache und der Größe der Sprachgemeinschaft. Das Sample beträgt 216 Sprachen \citep[388]{HayBauer2007}. Die Größe des Phoneminventars wird durch die Anzahl folgender Phonemkategorien gemessen: Basismonophthonge (unterscheiden sich in ihrer Qualität), Extramonophthonge (Unterscheidung der Länge und Nasalierung), Diphthonge, Obstruenten und Sonoranten \citep[389]{HayBauer2007}. Getestet wird ein Zusammenhang zwischen diesen einzelnen Kategorien und der \isi{Bevölkerungsgröße} wie auch zwischen der Gesamtgröße des Phoneminventars und der \isi{Bevölkerungsgröße}. Es wird eine positive Korrelation sowohl zwischen jeder Phonemkategorie und der \isi{Bevölkerungsgröße} als auch zwischen der Gesamtgröße des Phoneminventars und der \isi{Bevölkerungsgröße} festgestellt \citep[389--390]{HayBauer2007}. Wichtig ist hier auch die Beobachtung, dass das Vokalinventar und das Konsonanteninventar jedoch nicht miteinander korrelieren \citep[391]{HayBauer2007}. Des Weiteren handelt es sich hier um eine statistische Tendenz. Es gibt also auch Sprachen, die dieser Tendenz zuwiderlaufen, wie z.\,B.\ das Färöische, das von relativ wenigen Sprechern gesprochen wird, aber ein großes Phoneminventar aufweist \citep[390]{HayBauer2007}. In einem weiteren Test wird untersucht, ob diese Tendenz verursacht wird von einem Zusammenhang zwischen Sprachfamilien und der \isi{Bevölkerungsgröße} \citep[391]{HayBauer2007}. Tatsächlich hat die Sprachfamilie einen Einfluss, z.\,B.\ haben die indgermanischen Sprachen die größten Phoneminventare, während die austronesischen Sprachen die kleinsten Phoneminventare aufweisen \citep[392]{HayBauer2007}. Die \isi{Bevölkerungsgröße} ist jedoch neben der Sprachfamilie ein zusätzlicher, signifikanter Prädiktor für die Größe des Phoneminventars \citep[392]{HayBauer2007}. Weshalb Sprachen mit kleinen Sprachgemeinschaften auch kleine Phoneminventare aufweisen, kann nicht definitiv erklärt werden. \citet{HayBauer2007} diskutieren verschiedene Erklärungsversuche, auf die hier nicht weiter eingegangen werden kann. Eine Hypothese soll hier aber kurz vorgestellt werden. \citet{Trudgill2004a} geht davon aus, dass kleine Sprachgemeinschaften entweder große oder kleine Phoneminventare aufweisen, während große Sprachgemeinschaften eher mittlere Phoneminventare favorisieren (\citealt{Trudgill2004a}: 317, zitiert aus \citealt[396]{HayBauer2007}). Große Phoneminventare werden dadurch erklärt, dass kleine Sprachgemeinschaften die Fähigkeit haben „to encourage continued adherence to norms from one generation to another, however complex they may be“ \citep[317]{Trudgill2004a}, zitiert aus \citealt[396]{HayBauer2007}). Kleine Phoneminventare dagegen werden auf das gemeinsame Wissen kleiner Sprachgemeinschaften zurückgeführt: „initial small community size […] would have led in turn to tight social networks, which would have implied large amounts of shared background information – a situation in which communication with relatively low level of phonological redundancy would have been relatively tolerable“ (\citealt[720]{Trudgill2002}: 720, zitiert aus \citealt[396]{HayBauer2007}). Den Zusammenhang zwischen phonologischer Komplexität, d.h. kleinem Inventar, und kleiner Sprachgemeinschaft illustriert \citet{Trudgill2011} anhand des Hawaiischen (basierend auf \citealt{Maddieson1984}), das fünf Vokale, acht Konsonanten, eine CVCV-Sil\-ben\-struk\-tur und nur 162 mögliche Silben hat:

\begin{quote}
My suggestion is that possessing only a small number of available syllables – and therefore a relatively small amount of redundancy – may, other things being equal, lead to greater communicative and/or cognitive difficulty because of a lack of contrastive possibilities. I suggest that while this lack of contrastive possibilities is entirely unproblematical for native speakers, languages such as Hawai’ian will cause difficulties for non-natives. Languages with very small phoneme inventories cause problems of \textit{memory load} for foreign learners – they are L2 difficult. […] The problem lies in the relative lack of distinctiveness between one vocabulary item and another, due to the necessarily high proportion of usage of possible syllables […]”. \citep[124]{Trudgill2011}
\end{quote}

\subsection{Soziale Isolation vs. Sprachkontakt}\label{2.2.3}

In diesem Abschnitt werden einige Studien vorgestellt, die strukturelle Komplexität in Sprachgemeinschaften mit viel und wenig \isi{Sprachkontakt} untersuchen. Dabei werden Sprachgemeinschaften mit wenig \isi{Sprachkontakt} als \isi{isolierte Sprachgemeinschaften} bezeichnet. Es handelt sich hier nur um einen exemplarischen Ausschnitt, in dem die wichtigsten Richtungen aufgezeigt werden sollen.

Peter Trudgill hat eine Vielzahl an Arbeiten veröffentlicht, die sich mit dem Zusammenhang zwischen struktureller Komplexität und dem Typ der Sprachgemeinschaft beschäftigen. In \citet{BaechlerSeiler2016a} wurden die wichtigsten Resultate zusammengefasst, was hier in leicht abgeänderter Form übernommen wird. Trudgill unterscheidet vor allem zwei Typen von Sprachgemeinschaften. Der eine Typ wird wie bei \citet{Braunmüller1984} durch geringe \isi{Bevölkerungsgröße}, geografische und vor allem soziale \isi{Isolation} charakterisiert \citep{Trudgill1992}. Mit sozialer \isi{Isolation} sind wenig \isi{Sprachkontakt} und wenige L2-Ler\-ner gemeint. Zusätzlich berücksichtigt \citet{Trudgill2011} aber auch interne Charakteristika einer Sprachgemeinschaft: enge soziale Netzwerke, hohe soziale Stabilität und eine große Menge gemeinsamen Wissens \citep[146]{Trudgill2011}. Folglich gibt es auf der einen Seite kleine, \isi{geografisch isolierte}, stabile Sprachgemeinschaften mit einem engen Netzwerk und wenig \isi{Sprachkontakt}, welche hier \textit{isolierte Sprachgemeinschaften} genannt werden, und auf der anderen Seite große, geografisch nicht isolierte, sich verändernde Sprachgemeinschaften mit losen Netzwerken und viel \isi{Sprachkontakt}, welche hier als \textit{nicht isoliert} gekennzeichnet werden. Diese zwei Typen von Sprachgemeinschaften bilden natürlich die beiden Pole, die Ausprägung der Charakteristika (z.\,B.\ hohe vs. geringe Bevölkerungszahl) wie auch die Charakteristika selbst können in unterschiedlicher Kombination auftreten \citep[147]{Trudgill2011}. Einfachheitshalber wird hier jedoch weiterhin von den beiden Extremen, d.h. von isolierten und nicht \isi{isolierten Sprachgemeinschaften} gesprochen. Zusammengefasst geht Trudgill davon aus, dass \isi{isolierte Sprachgemeinschaften} dazu tendieren, höhere strukturelle Komplexität aufzuweisen und nicht \isi{isolierte Sprachgemeinschaften} dazu, geringere strukturelle Komplexität zu zeigen. Die Gründe dafür sieht Trudgill im Sprachwandel, dessen Mechanismen und Ergebnisse abhängig vom Typ Sprachgemeinschaft sind, was hier kurz ausgeführt werden soll. Trudgill nimmt an, dass \isi{isolierte Sprachgemeinschaften} höhere strukturelle Komplexität haben, weil in diesen Sprachgemeinschaften die Wahrscheinlichkeit groß ist, „to find not only the preservation of complexity but also an increase in complexity, i.e. irregularity, opacity, syntagmatic redundancy, and non-borrowed morphological categories“ \citep[64]{Trudgill2011}. Sowohl der Erhalt von Komplexität als auch die Zunahme an Komplexität können durch Charakteristika des Sprachwandels erklärt werden, der durch die Struktur der Sprachgemeinschaft beeinflusst ist. Der Erhalt von Komplexität liegt besonders an drei Faktoren. Erstens ist es einfacher für kleine Sprachgemeinschaften mit wenig \isi{Sprachkontakt} und engen sozialen Netzwerken „to enforce and reinforce the learning and use of irregularities“ \citep[204]{Trudgill1992} und deshalb „to enforce and reinforce the learning and use of complexities by children and adolescents“ \citep[13]{Trudgill1996}. Zweitens unterscheidet sich das Tempo des Sprachwandels in Abhängigkeit vom Typ der Sprachgemeinschaft: „In small, isolated, stable communities, linguistic change will be slower“ \citep[103]{Trudgill2011}. Drittens sind kleine und \isi{isolierte Sprachgemeinschaften} weniger vom Sprachwandel betroffen bzw. die „[g]eographically peripheral varieties which have been least subject to dialect contact most strongly resist [...] language change leading to simplification“ \citep[6]{Trudgill1996}. Wenn jedoch Sprachwandel stattfindet, so sind kleine, \isi{isolierte Sprachgemeinschaften} mit engen Netzwerken in höherem Maße fähig einerseits, „to push through, enforce and sustain changes of a less natural or usual phonological type“ \citep[11]{Trudgill1996}, und andererseits „[to] promote the spontaneous growth of morphological categories“ \citep[109]{Trudgill2009}. Allgemein kann folglich davon ausgegangen werden, dass in diesen Sprachgemeinschaften der Sprachwandel eher Komplexifizierung verursacht \citep[103]{Trudgill2011}. \citet{Trudgill2011} nennt diesen Typ der Komplexifizierung „spontaneous, non-additive complexification“ \citep[71]{Trudgill2011}. Es kann also festgehalten werden, dass kleine, \isi{geografisch isolierte}, stabile Sprachgemeinschaften mit einem engen Netzwerk und wenig \isi{Sprachkontakt} dazu tendieren, Komplexität zu erhalten wie auch zu erhöhen. Im Gegensatz dazu tendieren große, geografisch nicht isolierte, sich verändernde Sprachgemeinschaften mit losen Netzwerken und viel \isi{Sprachkontakt} dazu, ihre Grammatik zu vereinfachen, „because high irregularity, low transparency, and high levels of redundancy make for difficulties of learning and remembering for adolescent and adult learner-speakers“ \citep[101]{Trudgill2009}. Es kann aber noch eine weitere Art des Kontakts beobachtet werden, nämlich „long-term co-territorial contact situations involving child bilingualism“ \citep[34]{Trudgill2011}. In dieser Kontaktsituation kommt Komplexifizierung in der Form von \textit{Additive Borrowings} vor \citep[27]{Trudgill2011}. Dabei werden neue Elemente oder Kategorien von der einen in die andere Sprache übernommen, ohne dass in der übernehmenden Sprache bereits existierende Elemente oder Kategorien ersetzt werden \citep[27]{Trudgill2011}.

\citet{Kusters2003} untersucht in seiner Dissertation den möglichen Zusammenhang zwischen dem Wandel der Komplexität in der Verbflexion und dem Wandel einer Sprachgemeinschaft von Typ 1 zu Typ 2. Typ 1 hat vor allem L1-Ler\-ner, ist sprecher-orientiert, d.h., der Hörer kann vermuten, was gesagt wird, und Sprecher sowie Hörer verfügen über ein großes gemeinsames Wissen. Des Weiteren hat die Sprache in Gemeinschaften des Typs 1 eine symbolische Funktion: Ausdruck von Identität, ästhetische Funktion etc. \citep[41]{Kusters2003}. Typ-2-Sprachgemeinschaften sind genau das Gegenteil: viele L2-Ler\-ner, kommunikative Funktion (primäres Ziel ist der Austausch von Informationen) und hörer-orientiert, d.h. die Bedürfnisse des Hörers (z.\,B.\ klare, explizite Artikulation) stehen im Vordergrund \citep[41]{Kusters2003}. Die Komplexität der Verbflexion wird definiert als die Phänomene, die für Außenstehende schwierig sind \citep[403]{Kusters2003}. Genauer werden drei Prinzipien herangezogen: Ökonomie, Transparenz und Isomorphie. Werden diese Prinzipien verletzt, gilt die Verbflexion als komplexer: Das Prinzip der Ökonomie ist verletzt, wenn eine Sprache viele Flexionskategorien aufweist; das Prinzip der Transparenz ist verletzt durch Fusion, Fission, Allomorphie und Homonymie; das Prinzip der Isomorphie ist verletzt bei nicht gleichbleibender Affixabfolge in verschiedenen Domänen \citep[403]{Kusters2003}. Untersucht werden Sprachen aus vier typologisch unterschiedlichen Sprachfamilien: Arabische Varietäten, skandinavische Sprachen, Quechua und Swahili \citep[403]{Kusters2003}. Für jede dieser Sprachfamilien gibt es Varietäten, deren Sprachgemeinschaft eher dem Typ 1 entspricht und solche, deren Sprachgemeinschaft eher dem Typ 2 entspricht. So kann geprüft werden, ob die Komplexität in der Verbflexion abnimmt, wenn die Sprachgemeinschaft sich von Typ 1 zu Typ 2 wandelt \citep[45]{Kusters2003}. Kurz zusammengefasst kann \citet{Kusters2003} zeigen, dass eine Sprache ökonomischer wird, wenn ihre Sprachgemeinschaft von einem Typ 1 zu einem Typ 2 wird: „[T]he number of categories and category combinations indeed decreases correspondingly in these communities“ \citep[357]{Kusters2003}. Dies hat zur Folge, dass diese Sprachen auch transparenter werden \citep[357]{Kusters2003}. Im Gegensatz dazu scheint Isomorphie keine große Rolle zu spielen \citep[357]{Kusters2003}. Es kann also ein Zusammenhang zwischen dem Typ einer Sprachgemeinschaft und den Phänomenen, die in der Verbflexion vorkommen, also der Komplexität der Verbflexion, beobachtet werden.

\citet{McWhorter2001} vergleicht in seinem ausführlichen und viel beachteten Aufsatz die strukturelle Komplexität einer Kreolsprache (Saramaccaans) mit zwei alten Sprachen, wobei die eine (Tzes) eher einen synthetischen, die andere (Lahu) eher einen analytischen Sprachbau aufweist. Komplexität wird durch Überspezifikation definiert, d.h. die overte Markierung phonetischer, morphologischer, syntaktischer und semantischer Unterscheidungen, die über die kommunikativen Notwendigkeiten hinausgehen \citep[125]{McWhorter2001}. Bestimmt wird der Grad an Überspezifikation mit der Beschreibungslänge eines Systems: „[S]ome grammars might be seen to require lengthier descriptions in order to characterize even the basics of their grammar than others“ \citep[134--135]{McWhorter2001}. \citet{McWhorter2001} geht dabei aber qualitativ und nicht quantitativ vor, d.h., er macht eine Analyse zu mehr oder weniger Markierung, es werden jedoch für die verschiedenen linguistischen Ebenen keine Zahlen ermittelt und am Schluss miteinander verrechnet. Untersucht wird Überspezifikation in vier Bereichen. Erstens ist das phonologische System komplexer, je mehr markierte Phoneme das System aufweist \citep[135]{McWhorter2001}. Mit markiert sind selten vorkommende Phoneme gemeint (z.\,B.\ Klicks, hintere gerundete Vokale etc.), wobei es hier nicht um komplexe Artikulation, sondern um Implikation geht: Eine Sprache mit markierten Phonemen hat auch unmarkierte und ist somit komplexer als eine Sprache mit unmarkierten Phonemen \citep[135]{McWhorter2001}. Zweitens wird die Syntax komplexer, je mehr Regeln verarbeitet werden müssen, wie beispielsweise Asymmetrien zwischen Matrix- und Nebensatz (z.\,B.\ Verb\-zweit-/Verb\-end\-stel\-lung), Ergativ/Absolutiv und Nominativ/Akkusativ in derselben Sprache usw. \citep[136]{McWhorter2001}. Drittens wird die Grammatik komplexer, wenn feine semantische und/oder pragmatische Unterscheidungen overt kodiert werden und dieser Ausdruck grammatikalisiert ist \citep[136]{McWhorter2001}. Ein Beispiel hierfür gibt Koasati (Muskogee-Sprache, Nordamerika), das spezielle pronominale \isi{Affixe} hat, die nur mit stativen Verben verwendet werden \citep[137]{McWhorter2001}. Viertens führt die Flexionsmorphologie generell zu Überspezifikation. Hier werden vor allem drei Bereiche hervorgehoben. A) Morphophonologie und Suppletion: Die Morphophonologie verursacht Prozesse, die phonetisch nur wenig vorausgesagt werden können (z.\,B.\ \isi{Umlaut}) und Suppletion \citep[137]{McWhorter2001}. B) Allomorphie in der Flexion und arbiträre Allomorphie: Zu Ersterem gehören Unterscheidungen zwischen Substantiv- oder Verbklassen, die von der Flexion kodiert werden; zum zweiten Flexion, die mit der \isi{Wurzel} gelernt und gespeichert werden muss (z.\,B.\ Aspektpaare im Russischen) \citep[138]{McWhorter2001}. C) Die Markierung von Kongruenz erhöht ebenfalls die strukturelle Komplexität \citep[138]{McWhorter2001}. In einem ersten Schritt vergleicht \citet{McWhorter2001} die Kreolsprache Saramaccaans, die Ende des 17. Jh. entstanden ist, mit der synthetischen Sprache Tsez, eine nakho-daghestanische Sprache aus dem Nordkaukasus. Dabei beobachtet er, dass Tsez häufiger als Saramaccaans Überspezifikation zeigt und erklärt dies wie folgt: „[B]y virtue of the fact that they were born as pidgins, and thus stripped of almost all features unnecessary to communication, and since then have not existed as natural languages for a long enough time for diachronic drift to create the weight of ʻornamentʼ that encrusts older languages“ \citep[125]{McWhorter2001}. Man könnte also annehmen, dass der Unterschied in der Komplexität zwischen analytischen Sprachen und Kreolsprachen nicht so groß ausfällt \citep[143--144]{McWhorter2001}. Dazu wird Saramaccaans mit Lahu verglichen. Das Resultat des Vergleichs zeigt genau das Gegenteil von dem, was man erwarten würde: „It is demonstrated that this complexity differential remains robust even when creoles are compared with older languages lacking inflection“ \citep[125]{McWhorter2001}. Zusammengefasst kann man also festhalten, dass die strukturelle Komplexität einer Sprache mit der Geschichte ihrer Entstehung und ihres Wandels zusammenhängt.

\citet{SzmrecsanyiKortmann2009} untersuchen die morphologische/morphosyntaktische Komplexität vieler unterschiedlicher englischer Varietäten, die in folgende Typen eingeteilt werden können: traditionelle L1-Va\-rie\-tä\-ten mit wenig \isi{Sprachkontakt} (z.\,B.\ East Anglia English), L1-Va\-rie\-tä\-ten mit viel \isi{Sprachkontakt} (z.\,B.\ Australian English), Englisch-basierte Pidgins und Kreols (z.\,B.\ Tok Pisin) und L2-Va\-rie\-tä\-ten (z.\,B.\ Hong Kong English) \citep[64–65]{SzmrecsanyiKortmann2009}. Die strukturelle Komplexität wird anhand von vier Parametern definiert und gemessen: Ornamentale Regeln, Schwierigkeiten beim L2-Er\-werb, Grammatizität/Redundanz, Irregularitäten \citep[64–65]{SzmrecsanyiKortmann2009}. Ornamentale Regeln sind solche, die Unterscheidungen und Asymmetrien hinzufügen, ohne dass sie einen kommunikativen oder funktionalen Vorteil bringen, z.\,B.\ \isi{Genus} \citep[68]{SzmrecsanyiKortmann2009}. Die Schwierigkeit beim L2-Er\-werb wird durch den Grad gemessen, zu dem eine bestimmte Varietät nicht jene Phänomene aufweist, „that L2 acquisition research has shown to recur in interlanguage varieties“ \citep[69]{SzmrecsanyiKortmann2009}. Dazu zählen beispielsweise die fehlende Markierung der Vergangenheit bei regelmäßigen Verben oder die fehlende Inversion. Die Merkmale dieser zwei Parameter sind binär, d.h., entweder kommen sie vor oder nicht (Liste in \citealt[69–71]{SzmrecsanyiKortmann2009}). Die Komplexität der vier Typen von Varietäten wird folglich durch das durchschnittliche Vorkommen dieser Merkmale berechnet. Die Datengrundlage hierfür bildet der \textit{World Atlas of Morphosyntactic \isi{Variation} in English} \citep[65--66]{SzmrecsanyiKortmann2009}. Die Datengrundlage für die Parameter Grammatizität und Irregularität sind verschiedene Korpora, die jedoch keine Pidgins und Kreols beinhalten \citep[67]{SzmrecsanyiKortmann2009}. Dies ermöglicht, die strukturelle Komplexität durch die \isi{Textfrequenz} zu messen. Grammatizität/Redundanz beinhaltet die synthetische Grammatizität, d.h. die gebundenen grammatischen Morpheme und die analytische Grammatizität, d.h. die freien grammatischen Morpheme \citep[71--72]{SzmrecsanyiKortmann2009}. Für die synthetische Grammatizität wird ein Syntheseindex berechnet: Die Datenbasis bilden 15.000 Tokens und der Syntheseindex ist die Prozentzahl der gebundenen grammatischen Morpheme pro 1.000 Tokens \citep[72]{SzmrecsanyiKortmann2009}. Auf die gleiche Weise wird der Analyseindex berechnet und die Komplexität der Grammatizität bildet die Summe des Synthese- und Analyseindex \citep[72]{SzmrecsanyiKortmann2009}. Die Irregularität wird ebenfalls durch die \isi{Textfrequenz} der gebundenen grammatischen Morpheme gemessen, jedoch getrennt nach regelmäßigen und unregelmäßigen Allomorphen \citep[74]{SzmrecsanyiKortmann2009}. Genauer wird ein Transparenzindex ermittelt, und zwar durch den Anteil (in Prozent) der regelmäßigen Allomorphe an allen gebundenen grammatischen Allomorphen \citep[74]{SzmrecsanyiKortmann2009}. Es sollen nun kurz die wichtigsten Resultate zusammengefasst werden. Am meisten ornamentale Regeln haben die traditionellen L1-Va\-rie\-tä\-ten, gefolgt von L1-Kon\-takt\-va\-rie\-tä\-ten, Pi\-dgins/Kreols, L2-Va\-rie\-tä\-ten, d.h. „[…] ornamental complexity is clearly a function of the degree of contact […]“ \citep[69]{SzmrecsanyiKortmann2009}. Die Ergebnisse des Parameters L2-Schwie\-rig\-kei\-ten zeigen Folgendes (von am meisten zu am wenigsten L2-Merk\-ma\-le): Pidgins/Kreols > L1-Kon\-takt\-va\-rie\-tä\-ten > traditionelle L1-Va\-rie\-tä\-ten > L2-Va\-rie\-tä\-ten. Diese Resultate sind zu erwarten, jedoch mit Ausnahme der L2-Va\-rie\-tä\-ten, wofür eine Erklärung im Parameter Grammatizität gefunden wird \citep[71]{SzmrecsanyiKortmann2009}. Den höchsten Grammatizitätsindex zeigen die traditionellen L1-Va\-rie\-tä\-ten, danach kommen die L1-Kon\-takt\-va\-rie\-tä\-ten und schließlich die L2-Va\-rie\-tä\-ten \citep[73]{SzmrecsanyiKortmann2009}. Erstens zeigt dies, dass „a history of contact and adult language learning can eliminate certain types of redundancy“ \citep[73]{SzmrecsanyiKortmann2009}. Zweitens ziehen L2-Spre\-cher nicht die einfacheren den komplexeren Merkmale vor, sondern bevorzugen Null-Markierung. Dies erklärt, weshalb die L2-Va\-rie\-tä\-ten besonders wenige Merkmale aufweisen, die als L2-Merk\-ma\-le identifiziert wurden \citep[73]{SzmrecsanyiKortmann2009}. Drittens ist schließlich besonders interessant, dass es zwischen dem Analyse- und dem Syntheseindex keine Ausgleichstendenzen gibt bzw. dass diese Indizes sogar positiv miteinander korrelieren \citep[74]{SzmrecsanyiKortmann2009}. Auch der vierte Parameter weist erwartete Resultate auf. Am transparentesten sind die L2-Va\-rie\-tä\-ten, gefolgt von den L1-Kon\-takt\-va\-rie\-tä\-ten und zuletzt von den traditionellen L1-Va\-rie\-tä\-ten \citep[75]{SzmrecsanyiKortmann2009}. Dies zeigt, dass ausgeprägter \isi{Sprachkontakt} (d.h. Spracherwerb von Erwachsenen), dazu führt, Unregelmäßigkeiten abzubauen \citep[75]{SzmrecsanyiKortmann2009}. Zusammengefasst erweist sich also, „that variety type is a powerful predictor of complexity variance“, weshalb \isi{Sprachkontakt} bezüglich des Grads struktureller Komplexität eine zentrale Rolle spielt \citep[76]{SzmrecsanyiKortmann2009}.

\citet{MaitzNémeth2014} übernehmen den Forschungsaufbau von \citet{SzmrecsanyiKortmann2009} und übertragen ihn auf deutsche Varietäten. Die untersuchten Varietäten sind Zimbrisch (traditionelle L1-Va\-rie\-tä\-t), die deutsche Standardsprache (L1-Kon\-takt\-va\-rie\-tät), Kiche Duits und Unserdeutsch (Pidgin/Kreol) \citep[7--9]{MaitzNémeth2014}. Strukturelle Komplexität wird durch die Parameter Synthetizität, Analytizität, Grammatizität und Irregularität definiert \citep[6--7]{MaitzNémeth2014}. Die Datengrundlage bilden gesprochensprachliche Korpora. Daraus werden 3.000 Wörter zufällig ausgewählt, welche in drei Subsamples à 1.000 Wörter eingeteilt sind \citep[9]{MaitzNémeth2014}. Synthetizität wird durch den Anteil der Wörter am Korpus gemessen (in Prozent), die durch gebundene grammatische Morpheme markiert sind \citep[10]{MaitzNémeth2014}. Dazu gehören die Kasus- und Numerusmarkierung am Nomen, Pronomen und Artikel, die Nu\-me\-rus- und Genusmarkierung und die Komparativformen des \isi{Adjektivs}, die Tempusmarkierung und die Markierung der Person am Verb \citep[10]{MaitzNémeth2014}. Analytizität wird durch den Anteil an Funktionswörtern am Korpus ermittelt, z.\,B.\ Determinierer, Konjunktionen, Auxiliarverben etc. \citep[11--12]{MaitzNémeth2014}. Die Grammatizität besteht aus der Summe der Synthetizität und der Analytizität \citep[12]{MaitzNémeth2014}. Schließlich wird die Irregularität anhand der durchschnittlichen \isi{Textfrequenz} von unregelmäßigen Markern berechnet, wozu z.\,B.\ Stammmodifikationen und Suppletion gehören \citep[12--13]{MaitzNémeth2014}. Die Parameter Synthetizität, Analytizität und Grammatizität zeigen dieselben Resultate: Die höchste Komplexität zeigt Zimbrisch (traditionelle L1-Va\-rie\-tät), gefolgt von der Standardsprache (L1-Kon\-takt\-va\-rie\-tät) und zuletzt von Kiche Duits/Unserdeutsch (Pidgin/Kreol), wobei der Unterschied nur zwischen Zimbrisch und der Standardsprache sowie zwischen Zimbrisch und Kiche Duits/Unserdeutsch signifikant ist, jedoch nicht zwischen der Standardsprache und Kiche Duits/Unserdeutsch \citep[11--13]{MaitzNémeth2014}. Gleiches gilt für den Parameter Irregularität, nur dass hier auch der Unterschied zwischen der Standardsprache und Kiche Duits/Unserdeutsch signifikant ist \citep[14]{MaitzNémeth2014}. Des Weiteren wird eine positive Korrelation zwischen Synthetizität und Analytizität gefunden, was gegen die oft angenommene Ausgleichstendenz zwischen synthetischen und analytischen Phänomenen spricht \citep[15]{MaitzNémeth2014}. Auch eine positive Korrelation zwischen Grammatizität und Irregularität kann beobachtet werden \citep[15]{MaitzNémeth2014}. Folglich ist ein hoher Grad an Irregularität typisch für morphosyntaktisch komplexe Varietäten \citep[17]{MaitzNémeth2014}. Es kann also festgehalten werden, dass \isi{Sprachkontakt}, d.h. Spracherwerb von Erwachsenen, zu Simplifizierungen in der Grammatik führt. \citet{MaitzNémeth2014} geben aber ebenfalls zu bedenken, dass auch weitere Faktoren die Morphologie vereinfachen können, wie beispielsweise phonologischer Wandel (z.\,B.\ Schwächung der Nebensilbe) \citep[20]{MaitzNémeth2014}. Gleichzeitig können jedoch ebenfalls Faktoren wirken, die Simplifizierung hemmen, wie z.\,B.\ die Attitüden gegenüber präskriptiven Normen \citep[21]{MaitzNémeth2014}.
 
Auch \citet{Schreier2016} geht davon aus, dass \isi{Sprachkontakt} die strukturelle Komplexität einer Sprache oder einer Varietät beeinflussen kann. Im Gegensatz jedoch zu den bis hier erörterten Studien kann er zeigen, dass Kontakt nicht nur zu Simplifizierung, sondern auch zu Komplexifizierung führt, und zwar in derselben Varietät zur selben Zeit \citep[139, 148-151]{Schreier2016}. Auch die (so\-zio-)lin\-gu\-is\-ti\-schen Faktoren differenziert er genauer, wobei es sich um folgende handelt: a) die Eigenschaften der Systeme, die miteinander in Kontakt stehen; b) das soziale Fundament des Kontakts, wozu auch demografische Daten und die Besiedlung gehören; c) die Intensität des Kontakts, wie z.\,B.\ die Stärke von Substrateffekten \citep[139]{Schreier2016}. Es geht vor allem darum, „to do justice to the multifaceted and interwoven (social and linguistic) phenomena of contact-induced change“ \citep[139]{Schreier2016}. Vor allem kann \citet{Schreier2016} zeigen, dass die Dichotomie – Varietäten mit viel Kontakt werden einfacher, während Varietäten mit wenig Kontakt komplexer werden – zu kurz greift. Als erstes wird anhand der Reduktion von Konsonantenclustern dargestellt, dass nicht die Intensität des Sprachkontakts entscheidend ist, sondern die systemischen und typologischen Unterschiede zwischen den Sprachen im Kontakt: Der starke Kontakt von zwei Varietäten, die beide komplexe Konsonantencluster aufweisen, hat kaum einen Effekt, während ein weniger starker Kontakt mit einer Varietät mit einfacheren Konsonantenclustern einen höheren Effekt hat \citep[144]{Schreier2016}. Dass starker Kontakt zu Simplifizierungen führt, trifft folglich nur auf unterschiedliche Systeme zu, nicht aber, wenn die Systeme große Ähnlichkeiten aufweisen \citep[144--145]{Schreier2016}. Zweitens zeigt \citet{Schreier2016}, dass das System der \isi{Personalpronomen} von Varietäten mit wenig Kontakt relativ regelmäßig, also wenig komplex ist und dass jenes von Varietäten mit viel Kontakt höhere Komplexität aufweist \citep[145--147]{Schreier2016}. Beispielsweise hat Tok Pisin in der 1. Person Plural zusätzlich die Unterscheidung inklusiv/exklusiv grammatikalisiert \citep[147]{Schreier2016}. Drittens ist festzustellen, dass Tristan da Cunha Englisch weder eine rein \textit{Low-Contact-} noch eine rein \textit{High-Contact-Varietät} ist und gleichzeitig Simplifizierung und Komplexifizierung aufweist. Je nachdem, welches Zeitfenster man sich anschaut, handelt es sich bei Tristan da Cunha Englisch um eine Varietät mit viel oder mit wenig Kontakt \citep[148]{Schreier2016}. Beispiele für Simplifizierung sind die Ausgleiche im Paradigma von \textit{sein} (\textit{is} vs. \textit{was} unabhängig von Person und \isi{Numerus}) oder die hohe Rate an Vereinfachung komplexer Konsonantencluster \citep[149--150]{Schreier2016}. Komplexifizierung zeigt sich z.\,B.\ darin, dass der dentale Frikativ vier Realisierungen hat: dentale Frikative, labiodentale Frikative und labiodentale Plosive sowie Sibilanten \citep[150]{Schreier2016}. Schließlich kann noch angefügt werden, dass Tristan da Cunha Englisch \isi{Archaismen} erhalten hat und gleichzeitig aber auch \isi{Innovationen} aufweist \citep[150--151]{Schreier2016}. Aufgrund dieser Resultate kritisiert \citet{Schreier2016} besonders drei Aspekte der bisherigen Modelle. Erstens ist es schwierig, Varietäten in die binäre Unterscheidung zwischen Varietäten mit viel Kontakt und wenig Kontakt einzuteilen \citep[151--152]{Schreier2016}. Zweitens berücksichtigen die Modelle die \isi{Diachronie} zu wenig, denn „isolation effects may override contact effects over time“ \citep[152]{Schreier2016}. Drittens können Komplexifizierung und Simplifizierung gleichzeitig auftreten \citep[153]{Schreier2016}. \citet{Schreier2016} schlägt ein Modell vor, das wenig Kontakt und viel Kontakt als die beiden Pole auf einem Kontinuum ansetzt \citep[153--154]{Schreier2016}. Dieses Modell trägt der Tatsache Rechnung, dass eine Varietät komplexe und einfache Phänomene gleichzeitig aufweisen kann.

\subsection{Geografische Isolation}\label{2.2.4} 

Im vorangehenden Kapitel wurden einige Studien skizziert, in denen mit dem Begriff \isi{Isolation}, wenn er verwendet wird, vorwiegend soziale \isi{Isolation} gemeint ist, d.h. Sprachgemeinschaften mit wenig \isi{Sprachkontakt}, wenig L2-Spre\-chern etc. Vor allem bei \citet{Jakobson1929}, \citet{Braunmüller1984} und \citet[u.a.]{Trudgill2011} impliziert \isi{Isolation} jedoch meistens auch geografische \isi{Isolation}. In diesem Kapitel sollen drei Untersuchungen vorgestellt werden, die sich explizit mit geografischer \isi{Isolation} beschäftigen, nämlich \citet{Nichols1992,Nichols2016} und \citet{Garzonio2016}. Obwohl diese Studien erst Ende des 20. Jh.\slash Anfang des 21. Jh. entstanden sind, weist bereits \citet{Jakobson1929} darauf hin, dass die ukrainischen Dialekte mit einem großen Vokalinventar sich eher an der Peripherie des ukrainischen Sprachgebiets befinden: „Cette différence [entre les systèmes de voyelles] est due, en premier lieu à la tendance conservatrice qui est caractéristique des parlers de la périphérie, et en second lieu à des différences fonctionnelles“ (\citealt[73]{Jakobson1929}, ausführliches Zitat in \sectref{2.2.1}).

Die wichtigsten Aspekte aus \citet{Nichols1992} für das Ziel dieser Arbeit wurden in \citet{BaechlerSeiler2016a} zusammengefasst. Die anschließenden Ausführungen folgen dieser Zusammenfassung. Nichols ist wahrscheinlich die erste Linguistin, die die mögliche Korrelation zwischen geografischer \isi{Isolation} und struktureller Komplexität explizit untersucht hat. Ihr wichtigstes Ziel ist, typologische Eigenschaften und Geografie in Beziehung zu setzen: „In describing the distribution of types and typological features we can often make active use of geography as a predictive factor. This means viewing the languages of a region as a population and demonstrating a correlation between the location or type of the region and the distribution of traits within the population or between populations“ \citep[12]{Nichols1992}. \citet{Nichols1992} unterscheidet u.a. zwischen \textit{Residual} und \textit{Spread Zones}, wobei ersterer als isoliertes Areal und letzterer als nicht isoliertes Areal bezeichnet werden kann. \textit{Spread Zones} sind wie folgt charakterisiert:

% \ea%1
%     \label{ex:key:1}
%     \gll\\
%         \\
%     \glt
%     \z
\begin{quote}


(1) Little genetic diversity

% \ea%2
%     \label{ex:key:2}
%     \gll\\
%         \\
%     \glt
%     \z

(2) Low structural diversity

% \ea%3
%     \label{ex:key:3}
%     \gll\\
%         \\
%     \glt
%     \z

(3) The language families present in the spread zone are shallow.

% \ea%4
%     \label{ex:key:4}
%     \gll\\
%         \\
%     \glt
%     \z

(4) Rapid spread of languages or language families and consequent language succession.

% \ea%5
%     \label{ex:key:5}
%     \gll\\
%         \\
%     \glt
%     \z

(5) Classic dialectal-geographical area with innovating center and conservative periphery.

% \ea%6
%     \label{ex:key:6}
%     \gll\\
%         \\
%     \glt
%     \z

(6) No net long-term increase in diversity. A spread zone is a long-lasting phenomenon, but it preserves little linguistic evidence of its history.

% \ea%7
%     \label{ex:key:7}
%     \gll\\
%         \\
%     \glt
%     \z

(7) The spreading language serves as a lingua franca for the entire area or a large part of it.


\citep[16--17]{Nichols1992}
\end{quote}

\noindent
Im Gegensatz zu den \textit{Spread Zones} weisen die \textit{Residual Zones} folgende typischen Charakteristika auf:

\begin{quote}


% \ea%1
%     \label{ex:key:1}
%     \gll\\
%         \\
%     \glt
%     \z

(1) High genetic density.

% \ea%2
%     \label{ex:key:2}
%     \gll\\
%         \\
%     \glt
%     \z

(2) High structural diversity.

% \ea%3
%     \label{ex:key:3}
%     \gll\\
%         \\
%     \glt
%     \z

(3) The language families [...] are deep.

% \ea%4
%     \label{ex:key:4}
%     \gll\\
%         \\
%     \glt
%     \z

(4) No appreciable spread of languages or families. No language succession.

% \ea%5
%     \label{ex:key:5}
%     \gll\\
%         \\
%     \glt
%     \z

(5) No clear center of innovation.

% \ea%6
%     \label{ex:key:6}
%     \gll\\
%         \\
%     \glt
%     \z

(6) Accretion of languages and long-term net increase in diversity.

% \ea%7
%     \label{ex:key:7}
%     \gll\\
%         \\
%     \glt
%     \z

(7) No lingua franca [...] for the entire area; local bilingualism or multilingualism is the main means of inter-ethnic communication.

\citep[21]{Nichols1992}
\end{quote}

Das wichtigste Resultat von Nicholsʼ Studie bezüglich der möglichen Korrelation zwischen \isi{Isolation} und struktureller Komplexität ist, dass „[r]esidual zones show relatively high complexity, equal to or greater than that of their respective continents. Spread zones show somewhat lower average complexity, equal to or lower than that of their respective continents“ \citep[192]{Nichols1992}. \citet{Nichols1992} nimmt aber auch an, dass es innovative und konservative Gegenden innerhalb der \textit{Residual} und \textit{Spread Zones} gibt. In einer \textit{spread Zone} kann ein innovatives Zentrum und eine konservative Peripherie gefunden werden \citep[17]{Nichols1992}. In bergigen Regionen, die als \textit{residual Zones} gelten können, gibt es \isi{Innovationen} eher an der Peripherie, d.h. im Tal oder in der Ebene, und \isi{Archaismen} im Zentrum, d.h. im Gebirge \citep[14]{Nichols1992}.

Dass \textit{Spread} und \textit{Residual} \textit{Zones} tatsächlich innovative und konservative Areale aufweisen, kann \citet{Nichols2016} in einer weiteren Untersuchung zeigen. Dazu werden Sprachen aus dem Ostkaukasus und aus der eurasischen Steppe analysiert. Bei den Sprachen aus dem Ostkaukasus handelt es sich um solche aus der nakho-daghestanischen Sprachfamilie \citep[120--121]{Nichols2016}. Aus der eurasischen Steppe werden 22 Sprachen aus folgenden Sprachfamilien herangezogen, wobei einige Sprachen keiner Sprachfamilie zugeordnet werden können (es werden nur die Sprachfamilien bzw. die isolierten Sprachen aufgezählt): Turksprachen, Mongolisch, Tungusisch, Tibetisch, Koreanisch, Japanisch, Ainu, Nivkh, Tschuk\-tschisch-Kam\-tscha\-da\-lisch, Es\-ki\-mo-A\-leu\-tisch, Jukagir, Jenissej, östliches Uralisch \citep[129]{Nichols2016}. Eingeteilt werden die Sprachen in zwei Kategorien, und zwar in sich ausbreitende Sprachen (mit vielen L2-Ler\-nern, Lingua Franca) und in isolierte, sich nicht ausbreitende Sprachen. Gemessen wird die strukturelle Komplexität durch die Größe des Inventars (d.h. die Anzahl der Elemente, z.\,B.\ die Anzahl der Phoneme, Flexionskategorien, Wortabfolgen etc.) und durch den Grad der Opazität (z.\,B.\ Suppletion, Sandhi, Mehrfachausdruck, Synthese etc.) \citep[118]{Nichols2016}. Die Größe des Inventars wird auf verschiedenen linguistischen Ebenen ermittelt, nämlich Phonologie, Synthese, Klassifizierer, Syntax und Lexikon, und Opazität durch \isi{Genus} und Deklination \citep[137]{Nichols2016}. Eine genaue Liste der untersuchten linguistischen Variablen ist aus \citet[137]{Nichols2016} zu entnehmen. Quantifiziert wird auf zwei Arten, und zwar je nachdem, ob es sich um eine binäre Variable handelt oder nicht. Auf der phonologischen Ebene kann beispielsweise die Anzahl der kontrastiven obstruenten Arten der Artikulation einfach gefunden werden, während Töne eine binäre Variable darstellen, die mit +1 für das Vorhandensein dieser Variable zu Buche schlägt und mit +0 für das nicht Vorhandensein \citep[137]{Nichols2016}. Für den Ostkaukasus kann \citet{Nichols2016} zeigen, dass Opazität mit isolierten Sprachen korreliert, während Transparenz und eine geringe \isi{Inventargröße} in sich ausbreitenden Sprachen zu beobachten sind \citep[129]{Nichols2016}. Da es sich um ein gebirgiges Gebiet handelt, kann \isi{Isolation} durch geografische Höhe quantifiziert werden \citep[129]{Nichols2016}. Diese Resultate sind aufgrund des soziolinguistischen Kontextes dieser Region zu erwarten, die kurz erörtert werden. Die Sprachen, die in der Ebene oder am Fuße eines Berges situiert sind, breiten sich bergaufwärts aus und weisen deshalb viele L2-Spre\-cher auf \citep[123--124]{Nichols2016}. In den Bergdörfern hingegen sind enge Netzwerke und kaum L2-Ler\-ner der isolierten Sprache zu beobachten \citep[123--124]{Nichols2016}. Neben den nakho-daghestanischen Sprachen des Ostkaukasus untersucht \citet{Nichols2016} auch unterschiedliche Sprachen der eurasischen Steppe. In dieser Gruppe weisen sich ausbreitende Sprachen eine geringe \isi{Inventargröße} auf, während isolierte Sprachen an der Peripherie des Gebiets einen hohen Grad an Opazität haben \citep[129--130]{Nichols2016}. Die eurasische Steppe kann eine lange Geschichte an Ausbreitung und Überlagerung unterschiedlicher Sprachen vorweisen. Sie hatten also immer viele L2-Ler\-ner, was zur Simplifizierung der Grammatik führt \citep[118--119]{Nichols2016}. Den Sprachen an der Peripherie (besonders im Norden) fehlen jüngste Ausbreitungsphasen, oder sie haben nie welche gehabt, d.h., diese Sprachen verfügten über längere Zeitabschnitte, in denen sich natürliche Komplexifizierung durch L1-Ü\-ber\-tra\-gung entwickeln konnte \citep[118--119]{Nichols2016}. Es kann also festgehalten werden, dass geringe \isi{Inventargröße} vor allem in sich ausbreitenden Sprachen gefunden werden kann, die zur Kommunikation zwischen verschiedenen Gesellschaften dienen, und hohe Opazität in isolierten Sprachen, die generell nicht als L2 gelernt werden \citep[132--133]{Nichols2016}.

\citet{Garzonio2016} untersucht die syntaktische Komplexität von Entscheidungsfragesätzen sowie w-Fragesätzen, und zwar im üblichen System nicht-alpiner norditalienischer Dialekte sowie in den folgenden alpinen ita\-lo-ro\-ma\-ni\-schen Dialekten: Monnese, Bellunese und Mendrisiotto \citep[95]{Garzonio2016}. Alle drei Dialekte liegen im alpinen Gebiet und können im Gegensatz zu vielen anderen norditalienischen Dialekten als isoliert gelten, wobei Monnese noch etwas isolierter ist als Bellunese und Mendrisiotto \citep[95, 102]{Garzonio2016}. Die syntaktische Komplexität wird anhand zweier Parameter gemessen: Komplexität der Derivation und die Anzahl freier Varianten für dieselbe Funktion (= Optionalität) \citep[98--99]{Garzonio2016}. Die Komplexität der Derivation wird anhand der Anzahl Move- und Mergeoperationen quantifiziert: je mehr Move- und Mergeoperationen, desto komplexer die Derivation \citep[98--99]{Garzonio2016}. Die Resultate sind in \tabref{table2.1} zusammengefasst.

% \textbf{Tabelle 2.1: Resultate syntaktischer Komplexität, Move/Merge \citep[111--112]{Garzonio2016}}\\

\begin{table}
\caption{Resultate syntaktischer Komplexität, Move/Merge \citep[111--112]{Garzonio2016} }\label{table2.1}
\begin{tabular}{lll}
\lsptoprule
\multicolumn{3}{l}{\mbox{Komplexität der Derivation:}}\\
 & Entscheidungsfragesätze & w-Fragesätze \\\midrule
\mbox{norditalienische Dialekte} & 1/1 & 2/1 \\
Monnese & 1/1 & 2/1 \\
Bellunese & 1/1 & 1/1 \\
Mendrisiotto & 0/1 & 1/1 \\
\midrule
\multicolumn{3}{l}{Optionalität:}\\
 & & w-Fragesätze   \\\midrule
\mbox{norditalienische Dialekte} & & 1  \\
Monnese & & 3  \\
Bellunese & & 1   \\
Mendrisiotto & & 5 (3)   \\
\lspbottomrule
\end{tabular}
\end{table}

Bezüglich der Optionalität zeigt sich, dass Monnese und Mendrisiotto deutlich komplexer sind. Dies hat vor allem damit zu tun, dass in diesen Dialekten das w-Element verdoppelt werden und dann links und/oder rechtsperipher stehen kann \citep[105, 108-111]{Garzonio2016}. Mendrisiotto weist zudem im Gegensatz zu den übrigen Varietäten des Samples in den w-Elementen drei verschiedene Formen auf: \textit{cusè}, \textit{cusa}, \textit{sa} \citep[109]{Garzonio2016}. Die Komplexität der Derivation betreffend sind die nicht-alpinen norditalienischen Dialekte und Monnese komplexer als Bellunese und Mendrisiotto. Garzonio kann jedoch zeigen, dass in Bellunese und Mendrisiotto die w-Fragesätze mit einer speziellen Semantik höhere Komplexität aufweisen, nämlich zwei Move- und zwei Mergeoperationen \citep[108, 111]{Garzonio2016}. Schließlich sind zwei interessante Beobachtungen hervorzuheben. Erstens hat Monnese eine \textit{tun}-Periphrase grammatikalisiert, die jedoch in der Derivation nicht zu höherer Komplexität führt \citep[103--104]{Garzonio2016}. \isi{Isolation} hat in diesem Fall also nicht höhere syntaktische Komplexität hervorgebracht, sondern eine andere Syntax \citep[113]{Garzonio2016}. Zweitens weist Medrisiotto in der Derivation die geringste, aber in der Optionalität die höchste Komplexität auf \citep[111--114]{Garzonio2016}. Zusammengefasst kann also Folgendes festgehalten werden: Erstens sind isolierte Dialekte nicht komplexer als nicht-isolierte bezüglich der derivationellen Komplexität, jedoch können sie einen hohen Grad an Optionalität aufweisen; zweitens variiert die derivationelle Komplexität zwischen verwandten Varietäten nur wenig; drittens können sich Fragesatztypen in ihrer Komplexität unterscheiden (am extremsten im Mendrisiotto) \citep[114]{Garzonio2016}.

\subsection{Arten von Komplexität}\label{2.2.5}

Hier soll nun zuerst die etablierte Unterscheidung zwischen absoluter und relativer Komplexität vorgestellt werden, die von \citet{Miestamo2008} vorgeschlagen wird. Danach wird eine detailliertere Einteilung der Typen struktureller Komplexität erörtert, die auf den Philosophen \citet{Rescher1998} zurückgeht und von \citet{MiestamoSinnemäkiKarlsson2008} für linguistische Phänomene adaptiert wurde. Schließlich werden die Definitionen struktureller Komplexität der vorangehenden Kapitel zusammengefasst und kategorisiert.

Neben lokaler und globaler Komplexität (vgl. \sectref{2.1.2}) unterscheidet \citet{Miestamo2008} zwischen absoluter und relativer Komplexität. Im relativen Ansatz wird Komplexität durch den Aufwand oder Schwierigkeit für den Sprecher definiert, d.h., „how difficult a phenomenon is to process (encode/decode) or learn. The more costly or difficult a linguistic phenomenon is, the more complex it is“ \citep[25]{Miestamo2008}. Es stellt sich also immer die Frage, für wen ein linguistisches Phänomen schwierig oder aufwändig ist, wie z.\,B.\ Hörer, Sprecher, L1- oder L2-Ler\-ner. Dabei ist zu beachten, dass ein linguistisches Phänomen für eine Gruppe (z.\,B.\ Hörer) Aufwand oder Schwierigkeiten verursacht, während dasselbe Phänomen für eine andere Gruppe (z.\,B.\ Sprecher) einfach ist \citep[25]{Miestamo2008}. Beschäftigt man sich mit absoluter Komplexität, steht ausschließlich das linguistische System im Vordergrund. Die Grundidee dahinter ist, je mehr Elemente ein System hat, desto komplexer ist es \citep[24]{Miestamo2008}. Um die Idee der Anzahl Elemente genereller zu fassen, greift \citet{Miestamo2008} auf Konzepte der \isi{Informationstheorie} zurück (z.\,B.\ Shannon-, Kolmogorov-Komplexität) und schlägt, basierend auf \citet{Dahl2004}, folgende Definition absoluter Komplexität vor:

\begin{quote}
[T]he complexity of a linguistic phenomenon may be measured in terms of the length of the description of that phenomenon; the longer a description a phenomenon requires, the more complex it is […] A less complex phenomenon can be compressed to a shorter description without losing information. On a high level of abstraction we may say that we are still dealing with the number of parts in a system, but these parts are now the elements that constitute the description of the system. \citep[24--25]{Miestamo2008}
\end{quote}

In Arbeiten zur strukturellen Komplexität sind bereits unzählige Übersichten zusammengestellt worden, die resümieren, was genau mit struktureller Komplexität gemeint ist und welche Arten struktureller Komplexität in Sprachen vorgefunden werden (z.\,B.\ \citealt[10--12]{SzmrecsanyiKortmann2012}). Im Gegensatz dazu gibt \citet{Rescher1998} in seiner philosophischen Auseinandersetzung mit dem Thema Komplexität eine breitere Definition und Klassifikation der Arten von Komplexität. Komplexität definiert er folgendermaßen:

\begin{quote}
Complexity is first and foremost a matter of the number and variety of an item’s constituent elements and of the elaborateness of their interrelational structure, be it organizational or operational. Any sort of system or process – anything that is a structured whole consisting of interrelated parts – will be to some extent complex. Accordingly, all manner of things can be more or less complex: natural objects (plants or river systems), physical artifacts (watches or sailboats), mind-engendered processes (languages or instructions), bodies of knowledge, and so on. \citep[1]{Rescher1998}
\end{quote}

Des Weiteren unterscheidet \citep[8--16]{Rescher1998} zwischen unterschiedlichen Typen von Komplexität. Diese Unterscheidung wurde von \citet{MiestamoSinnemäkiKarlsson2008} übernommen und für die Kategorisierung linguistischer Phänomene angepasst \citep[VIII–IX]{MiestamoSinnemäkiKarlsson2008}. Davon erstellt \citet{Sinnemäki2011} eine modifizierte Version, die in \figref{table2.2} wiedergegeben ist und welche weiter unten anhand der \figref{table2.3} genauer erörtert wird.

% \textbf{Tabelle 2.2: Arten der Komplexität (\citealt[9]{Rescher1998}; \citealt[23]{Sinnemäki2011}}\\

\begin{figure}
\caption{Arten der Komplexität (\citealt[9]{Rescher1998}; \citealt[23]{Sinnemäki2011})}\label{table2.2}
\noindent\fbox{\parbox{\textwidth}{\selectlanguage{english}\begin{enumerate}[label=\arabic*.,listparindent=0pt,leftmargin=.5cm,rightmargin=.5cm]
\item Epistemic modes
\begin{enumerate}[label=\Alph*.]
	\item Formulaic complexity
	\begin{enumerate}[label=\alph*.]
		\item Descriptive complexity: length of the account that must be given to provide an adequate description of a given system.
		\item Generative complexity: length of the set of instructions that must be given to provide a recipe for producing a given system.
		\item Computational complexity: amount of time and effort involved in resolving a problem.
	\end{enumerate}
\end{enumerate}
\item Ontological modes
	\begin{enumerate}[label=\Alph*.]
		\item Compositional complexity
		\begin{enumerate}[label=\alph*.]
			\item Constitutional complexity: number of constituent elements (e.g., in terms of the number of phonemes, morphemes, words, or clauses).
			\item Taxonomic complexity (or heterogeneity): variety of constituent elements, that is, the number of different kinds of components (e.g., tense-aspect distinctions, clause types).
		\end{enumerate}
		\item Structural complexity
		\begin{enumerate}[label=\alph*.]
			\item Organizational complexity: variety of ways of arranging components in different modes of interrelationship (e.g., phonotactic restrictions, variety of distinctive word orders).
			\item Hierarchical complexity: elaborateness of subordination relationships in the modes of inclusion and subsumption (e.g., recursion, intermediate levels in lexical-semantic hierarchies).
		\end{enumerate}
	\end{enumerate}
\item Functional complexity
	\begin{enumerate}[label=\Alph*.]
		\item Operational complexity: variety of modes of operation or types of functioning (e.g., cost-related differences concerning the production and comprehension of utterances).
		\item Nomic complexity: elaborateness and intricacy of the laws governing a phenomenon (e.g., anatomical and neurological constraints on speech production; memory restrictions).
	\end{enumerate}
\end{enumerate}}}
\end{figure}

Vergleicht man diese Arten von Komplexität und \citeauthor{Miestamo2008}s \citeyearpar{Miestamo2008} Einteilung in absolute und relative Komplexität, kann die epistemische und die ontologische Komplexität der absoluten Komplexität zugeteilt werden und die funktionale Komplexität der relativen Komplexität.

Abschließend soll nun versucht werden, die in diesem Kapitel diskutierten Arbeiten in die Kategorisierung der Komplexitätsarten von \citet{Rescher1998} und \citet{Sinnemäki2011} einzuordnen. Eine tabellarische Übersicht gibt \figref{table2.3}, die die Komplexitätsarten auslässt, welche in den erörterten Arbeiten keine Rolle spielen, da es das Ziel ist, einen Überblick zu bieten. Deswegen werden linguistische Konzepte und Prinzipien einzelnen Komplexitätsarten zugeordnet, was natürlich eine Verallgemeinerung zur Konsequenz hat.

Alle Studien mit Ausnahme von \citet{Sampson2001} und \citet{Garzonio2016} beruhen mehr oder weniger explizit auf der Grundidee der deskriptiven Komplexität: Je länger die Beschreibung der Grammatik ist, desto komplexer ist die Grammatik. Ausschließlich \citet{Garzonio2016} operationalisiert die generative Komplexität, um die Komplexität einer Sprache zu messen. Dazu zählt er die Anzahl von Move- und Mergeoperationen, wobei es sich um Instruktionen zur Ableitung (also Produktion) wohlgeformter Sätze handelt. Die ontologischen Arten bilden den Schwerpunkt in der bisherigen Beschäftigung mit linguistischer Komplexität. Diese werden unterteilt in kompositionelle und strukturelle Komplexität. Zur kompositionellen Komplexität gehören die Größe des Inventars (konstitutionelle Komplexität) und die Anzahl Unterscheidungen (und deren Kodierung), die in einem System gemacht werden (taxonomische Komplexität). Die \isi{Inventargröße} wird vorwiegend in der Phonologie untersucht, wie z.\,B.\ das \mbox{Phon-/}Pho\-nem\-in\-ven\-tar (\citealt{Jakobson1929}, \citealt{HayBauer2007}, \citealt{Nichols2016}, \citealt{Schreier2016}). Dazu gehört aber auch die Komplexität der Konsonantencluster bei \citet{Schreier2016} sowie die markierten Phoneme bei \citet{McWhorter2001}, da es hier um eine Implikation geht: Ein Sprachsystem, das markierte Phoneme hat, hat auch unmarkierte und folglich ein größeres Inventar. Zur \isi{Inventargröße} kommt bei \citet{Nichols2016} neben der Größe des Phoneminventars die Anzahl unterschiedlicher Präfixtypen dazu. Bei allen anderen Variablen, die \citet{Nichols2016} unter \isi{Inventargröße} subsummiert, handelt es sich nach \citeauthor{Rescher1998}s \citeyearpar{Rescher1998} und \citeauthor{Sinnemäki2011}s \citeyearpar{Sinnemäki2011} Einteilung eher um taxonomische Komplexität. Damit ist gemeint, dass je mehr Unterscheidungen (und Kodierungen dieser Unterscheidungen) gemacht werden, desto komplexer ist ein System. Beispielsweise ist ein System, das \isi{Genus} unterscheidet, komplexer als eines, das \isi{Genus} nicht unterscheidet, und je mehr \isi{Genera} ein System unterscheidet, desto komplexer ist es ebenfalls. Dazu gehören die meisten Variablen bei \citet{Nichols2016}, die ornamentalen Regeln bei \citet{SzmrecsanyiKortmann2009} sowie der Verstoß gegen das Ökonomieprinzip bei \citet{Kusters2003}. Auch \citeauthor{McWhorter2001}s \citeyearpar{McWhorter2001} Konzept der Überspezifikation kann zur taxonomischen Komplexität gerechnet werden und insbesondere sein dritter Parameter, nämlich die overt kodierten feinen semantischen und/oder pragmatischen Unterscheidungen. Schließlich zählen bei \citet{Schreier2016} die Grammatikalisierung von Exklusiv/Inklusiv in den \isi{Personalpronomen} wie auch die Ausgleiche im Verbalparadigma (Verlust von Person- und Numerusunterscheidung, also Simplifizierung) dazu. In der strukturellen Komplexität wird zwischen der organisationellen und der hierarchischen Komplexität unterschieden. Die organisationelle Komplexität beinhaltet jene Phänomene, die das 1-zu-1-Verhältnis zwischen Form und Bedeutung verletzen \citep[25]{Sinnemäki2011}. Bei \citet{Sinnemäki2009} ist das die Verletzung von Ökonomie und Distinktheit, d.h., eine Funktion (hier Markierung von Agens und Patiens) wird zu viel oder zu wenig markiert. Auch die Grammatizität bei \citet{MaitzNémeth2014} sowie \citet{SzmrecsanyiKortmann2009} könnte dazugezählt werden. Die Grammatizität wird durch die \isi{Textfrequenz} freier und gebundener grammatischer Morpheme gemessen. Damit können jedoch weder Aussagen gemacht werden, ob eine Funktion zu viel oder zu wenig kodiert wird, noch, welche und wie viele Funktionen kodiert werden. Das Konzept der Grammatizität ist also nur schwer in Reschers Kategorisierung einzuordnen. Des Weiteren gehören zur organisationelle Komplexität alle Phänomene, die unter dem Konzept Opazität/Transparenz zusammengefasst werden können: morphophonologische Regeln, Suppletion, Allomorphie, Homonymie, Fusion, Irregularität etc. Arbeiten dazu stammen von \citet{Braunmüller1984}, \citet{McWhorter2001}, \citet{Kusters2003}, \citet{SzmrecsanyiKortmann2009}, \citet{MaitzNémeth2014} sowie \citet{Nichols2016}. Zur organisationellen Komplexität wird auch die Anzahl unterschiedlicher Wortabfolgen gerechnet (bei \citealt{McWhorter2001}, \citealt{Nichols2016}). Schließlich untersucht \citet{Sampson2001} hierarchische Komplexität, indem die Einbettungstiefe gemessen wird. Wie bereits oben erwähnt, zählen die epistemische und die ontologische Komplexität zur absoluten Komplexität und die funktionale Komplexität zur relativen Komplexität. Zur funktionalen Komplexität gehört die operationelle Komplexität, womit u.a. der Aufwand und die Schwierigkeiten in der Produktion und Rezeption der Sprache durch unterschiedliche Gruppen (z.\,B.\ Sprecher, Hörer, L1- oder L2-Ler\-ner) gemeint sind. Dies spielt besonders bei \citet{Kusters2003} eine zentrale Rolle, bei \citet{SzmrecsanyiKortmann2009} betrifft dies den zweiten von vier Parametern. Die vorliegende Arbeit beschäftigt sich mit der generativen Komplexität, die genauer durch konstitutionelle, taxonomische und organisationelle Komplexität definiert ist. Genauer wird dies im anschließenden Kapitel erläutert.

%\textbf{Tabelle 2.3: Einordnung der diskutierten Arbeiten in die Komplexitätsarten nach \citet{Rescher1998} und \citet{Sinnemäki2011} }

\begin{figure}
\caption{Einordnung der diskutierten Arbeiten in die Komplexitätsarten nach \citet{Rescher1998} und \citet{Sinnemäki2011}}\label{table2.3}
\noindent\fbox{\parbox{\textwidth}{\selectlanguage{english}\begin{enumerate}[label=\arabic*.,listparindent=0pt,leftmargin=.5cm,rightmargin=.5cm]
\item Epistemic modes
\begin{enumerate}[label=\Alph*.]
	\item Formulaic complexity
	\begin{enumerate}[label=\alph*.]
		\item Descriptive complexity: all (except \citealt{Sampson2001} and \citealt{Garzonio2016} ) 
		\item Generative complexity: \citet{Garzonio2016}  
	\end{enumerate}
\end{enumerate}
\item Ontological modes
	\begin{enumerate}[label=\Alph*.]
	\item Compositional complexity
	\begin{enumerate}[label=\alph*.]
		\item Constitutional complexity: \citet{Jakobson1929}, \citet{McWhorter2001}, \citet{HayBauer2007}, \citet{Nichols2016}, \citet{Schreier2016} 
		\item Taxonomic complexity (or heterogeneity): \citet{McWhorter2001}, \citet{Kusters2003}, \citet{Sinnemäki2009}, \citet{SzmrecsanyiKortmann2009}, \citet{MaitzNémeth2014}, \citet{Nichols2016}, \citet{Schreier2016}  
	\end{enumerate}
	\item Structural complexity
	\begin{enumerate}[label=\alph*.]
		\item Organizational complexity: transparency/opacity: \citet{Braunmüller1984}, \citet{McWhorter2001}, \citet{Kusters2003}, \citet{Sinnemäki2009}, \citet{Nichols2016}; word order: \citet{McWhorter2001}, \citet{Nichols2016} 
		\item Hierarchical complexity: \citet{Sampson2001}  
	\end{enumerate}
\end{enumerate}
\item Functional complexity
	\begin{enumerate}[label=\Alph*.]
	\item Operational complexity: \citet{Kusters2003}, \citet{SzmrecsanyiKortmann2009} 
\end{enumerate}

\end{enumerate}}}
\end{figure}
\chapter[Perception of melodic detail]{Perception of melodic detail}\label{sec3}
In the previous chapter we documented regularities in speakers' productions which are not accounted for by the Autosegmental-Metrical (AM) model of intonational phonology. Moreover, it appears that this previously unacknowledged phonetic information allows for the elaboration of metrics which are more effective than traditional AM indices in mirroring pragmatic contrasts. However, regularity in production and robust mirroring of pragmatic contrasts do not give sufficient reason to include phonetic information into phonological representations. The additional requirement that has to be met is actual use of this phonetic information in perception. If phonetic information is regularly produced but not perceived and parsed (or, more radically, not perceived at all), then it could be deemed a by-product of other contrasts, and its inclusion in a minimalist phonological representation would not be justified. As we saw in  Section~\ref{sec242}, enriching phonological representations is a complex operation. For this reason, before prospecting any modification in the inventory or in the grammar of the intonational phonology of Neapolitan Italian (NI), an evaluation of the perceptual role of the regularities found in production is in order.

\section{Introduction}\label{sec31}
Even if the perception of intonation has witnessed a growing interest from the research community since the 1960s, studies which concentrate on the perception of phonologically salient dynamic proprieties of \textit{f0} contours are rare.\is{f0 dynamics} Such studies seem to bring together the two main threads which have characterized research on intonation perception, and which can also be (although very loosely) arranged diachronically. Early studies focussed on the psychoacoustics of pitch perception, attempting to define how listeners deal with fundamental frequency modulations over time (at least since \citealt{sergeant1962sensitivity}). This research agenda, combined with the relative unavailability of stable procedures for speech resynthesis,\footnote{Linear Predictive Coding (\textit{LPC}) based resynthesis was only available in 1970s, while Pitch-Synchronous OverLapp-Add (\textit{PSOLA}) based methods are more recent.} motivates, at least partially, the pervasive use of simple non-speech signals as the basis for stimuli construction.\is{resynthesis} However, it soon became evident that the pure tones used in early psychoacoustically-oriented research were but a first step towards the study of the specificities of the speech signal. The use of speech-like material in pitch perception investigations \citep{rossi1971seuil,klatt1973discrimination,thart1976psychoacoustic,schouten1985identification} was instrumental in orienting research on the relationships between spectral content and \textit{f0} variation \citep{house1990tonal,house1997perceptual}, permitting a shift from psychoacoustic to proper linguistic research.

While the psychoacoustic approach was yielding its first mature fruits, research on the structure of intonational categories had gone its first steps. A new insight on intonation perception came from studies on categorical perception of segmental contrasts (see Section~\ref{sec111}): the exploration of the viability of the categorical perception \citep{kohler1987categorical} for intonation as well started a line of studies in which intonational categories are central. Different paradigms for exploring the warping of perceptual space in intonation were proposed and tested, from classification and discrimination to imitation \citep{pierrehumbert1989categories}, semantic differential (from  \citealt{osgood1957measurement} through \citealt{uldall1964dimensions} to \citealt{kirsner1994interaction}) and indirect identification (context matching, \citealt{nash1980intonation}), up to eye-tracking \citep{dahan2002accent}. It seems that, unlike the one focussing on the pychoacoustics of pitch perception, this research thread on intonational categories still draws linguists' attention, as recent work questioning the viability of categorical perception for intonation \citep{gussenhoven2006experimental,niebuhr2007categorical} shows.

Studies on the role of dynamic \textit{f0} contour cues in perception bring together these two lines of research, in that they rest at the same time on a deep understanding of how much detail in the signal is psychoacoustically perceptible and of how intonational categories can be identified.
 
In this section we will review two such studies, dealing with NI \citep{petrone2011tones} and with Northern Standard German \citep{petrone2014intonation}, respectively. Both studies analyze prenuclear falls and show that sentence modality contrasts are not exclusively cued by the intonational nucleus.\is{sentence modality} 

The first builds on production results of \citet[see Section~\ref{sec241}]{petrone2008tonal}\is{Neapolitan Italian}, in which the authors accounted for different shapes in prenuclear falls by suggesting a tonal contrast (H in questions and L in statements) for the tone associated to the right edge of the Accentual Phrase domain. A subsequent study \citep{petrone2011tones} performs two experiments on the perceptual role of the Accentual Phrase tone, establishing the hypothesis that listeners do not rely exclusively on nuclear pitch accent contrasts in order to classify questions and statements. The first experiment is based on an identification task using gated stimuli (see Figure~\ref{fig301}). Utterances are cut after the prenuclear pitch accent (PREN condition) and after the alleged Accentual Phrase tone (AP condition), and they were presented to listeners along with a control set of uncut stimuli (NUCL condition) for classification as either Statements or Questions. 

\begin{figure}
\centering
\resizebox{\linewidth}{!}{\includegraphics{images/301.png}}
\caption{Schematized representation of the stimulus manipulation (three conditions: PREN, AP, NUCL) for the sentence \textit{La mamma vuole vedere la Rina}, uttered as a narrow focus statement with late focus (top panel), and as a yes/no question (bottom panel). From Petrone (2008).}
\label{fig301}\end{figure}

Results show that the classification scores are already above chance level for the PREN condition, and that in the AP condition classification is even more robust for statements, but not for questions.\footnote{This latter finding is the starting point for the second experiment, in which a semantic differential task is used to assess the nature of the meaning (attitudinal or pragmatic) conveyed by the AP tone.} The authors suggest that the absence of a significant improvement in question identification from the PREN to the AP condition could be due to the fact that the \textit{f0} contour stretching from the prenuclear peak to the H Accentual Phrase tone has a characteristically concave shape if compared to the convex shape of the interpolation between the peak and the L Accentual Phrase in statements. If listeners relied on this difference in \textit{f0} contours to classify stimuli, the same performances would indeed be expected for PREN and AP condition. It is interesting to note that a difference in the shape of the fall, described with different tonal specifications for the AP right edge, actually seems to be perceived even in stimuli gated before the AP itself. In a radical perspective, this finding could be taken as evidence against the adequacy of a representation based on AP tones, and rather supporting the hypothesis that the shape of the interpolation between tonal targets is relevant in itself: in the concluding remarks, \citeauthor{petrone2011tones} themselves acknowledge the need for a closer examination of dynamic proprieties of the fall.

The perceptual role of prenuclear falls has been investigated for Northern Standard German as well \citep{petrone2014intonation}.\is{German} In their first experiment, syntactically declarative Subject-Verb-Object sentences were uttered as questions or statements, using different nuclear configurations: statements all had a L- L\% utterance-final fall combined with one of three different pitch accents (H+L*, H* and L*+H in AM terms or early, medial and late peak in KIM terms; see Section~\ref{sec242}); questions had an L*+H pitch accent combined with either a final fall (L- L\%) or rise (H- H\%). The working-hypothesis transcription for the prenuclear fall was H*, but some phonetic differences between questions and statements can be spotted in the slope and the shape of the fall (see Figure~\ref{fig302}).

\begin{figure}
\centering
\resizebox{\linewidth}{!}{\includegraphics{images/302.png}}
\caption{Phonological analyses and \textit{f0} contours of the five naturally produced stimuli \textit{Katherina sucht 'ne Wohnung}; (1)-(3) were produced as statements, (4)-(5) were produced as questions. The syllables \textit{-ri-} and \textit{Woh-} that showed the prenuclear and nuclear accents are delimited by horizontal lines. Readapted from Petrone and Niebuhr (2014), Fig. 2.}
\label{fig302}\end{figure}

Natural utterances were gated after the Subject and the Verb and presented (along with control uncut items) to listeners for a semantic differential task on three scales, namely ‘astonished --- not astonished', ‘questioning --- not questioning', ‘uncertain --- certain'. The esults show that phonetic information in prenuclear falls is used by listeners in order to classify questions and statements: sentences sound more astonished, uncertain and questioning when uttered as questions, even when gated right after the Subject, that is before the nuclear configuration. This is not to say that the nucleus itself plays no role at all: complete utterances yield stronger responses towards the astonished, uncertain and questioning pole of the semantic scales.\footnote{The increase is maximal when compared with stimuli gated after the Verb. For these cases, the authors suggest a Frequency Code based explanation for listeners' bias towards statement-like responses, given that \textit{f0} on the Verb is a low plateau. However, a syntax based bias could also contribute to this result.} This finding leads the authors to claim that phonetic detail in \textit{f0} contours can even be spotted in the nuclear pitch accent labelled as L*+H, which would have a more convex rise in statements and a more concave rise in questions. However, this evidence is not compelling, since the availability to the listener of a L*+H pitch accent seem to induce more question-like responses for original statements as well.

The second experiment aimed at precisely identifying which of the melodic properties in the prenuclear region are actually responsible for the shift in the listeners' perception. Stimuli with resynthesized peak alignment, fall slope and fall shape in the prenuclear region were used in a context matching task. The results show that alignment, slope and shape interact in cueing sentence modality: question classification, for example, is strongest with early aligned peak, shallow fall slopes and concave shapes. On the basis of this evidence and of similar findings reported by \citet{petrone2011tones} for NI, \citeauthor{petrone2014intonation} underline the necessity of accounting for \textit{f0} dynamic information in phonological representations of prenuclear regions in German, through a different specification of either the prenuclear pitch accents (to be differentiated via trailing tones) or an alleged edge tone of a new prosodic domain (as in the NI analysis).

The two studies we reviewed in this section are both concerned with the prenuclear region, where shape differences can be attributed to either pitch accents or edge tones, as we have seen in the discussion of the German data. In the following section, we will report on two experiments on the perception of the nuclear rise shape differences documented in Section~\ref{sec2} \citep{dimperio2009interplay}.\is{pitch accent} Focussing on nuclear rises has the advantage of discarding any account of shape differences based on additional edge tones, thus enabling a more straightforward phonological modelling of the data. Both experiments (henceforth E1 and E2) are based on a categorization task and use stimuli resynthesized from utterances contained in the \textit{Tre Grazie} corpus (see Section~\ref{sec221}).

\section{Experiment 1}\label{sec32}
\subsection{Background}\label{sec3200}
The general aim of E1 was to test whether or not pitch accent classification is affected by dynamic intonational cues.\is{f0 dynamics} Specifically, building on the regularities found in speakers' productions (see Section~\ref{sec2}), we tested the perceptual role of rise shape in the contrast between nuclear accents of partial topic statements (SPT)\is{partial topic} and narrow focus questions (QNF).\is{narrow focus} If rise shape differences are consistently produced by speakers and reliably used by listeners, then phonological representations of pitch accents should include this phonetic information: rise shape would qualify as prosodic detail in the sense of useful information not yet encoded in abstract accounts of intonation. In order to test this hypothesis, we devised a forced-choice categorization task in which we manipulated rise shape (from concave to convex), asking subjects to classify items as questions or statements. If listeners use rise shape information in classification, we expect more question responses for stimuli with a more convex rise and more statement responses for stimuli with a more concave rise.

In addition to rise shape, we decided to test the impact of another cue, namely the scaling of the elbow following the peak in the stressed syllable.\footnote{L1 alignment, in addition, was manipulated in order to test whether the entire rise is aligned later in questions; see \citet{dimperio2003tonal}. Since this issue is not relevant here, we will not comment it any further.\label{footnoteL1}} As we discussed in Section~\ref{sec241}, \citet{petrone2008tonal} showed that productions of focus-final statements and questions in NI are characterized by postnuclear falls having different shapes, being more concave in questions and more convex in statements. They suggested that this contrast can be accounted for by a different tonal specification of a new prosodic domain, the Accentual Phrase, which would bear an H tone in questions (hence the concave fall) and an L tone in statements (hence the convex fall). Our corpus, on the other hand, focusses on shape differences in nuclear rises, and moreover on a different pragmatic contrast, namely the one between partial topic statements and narrow focus questions. However, through an examination of the \textit{Tre Grazie} corpus, \citet{dimperio2011phrasing} show that shape can also vary in postnuclear falls, with Partial Topics being characterized by an intermediate shape between concave questions and convex statements. In \citeauthor{petrone2008tonal}'s \citeyearpar{petrone2008tonal} terms, SPT would also have an Accentual Phrase break, whose tonal specification could be transcribed as !H in order to account for the three way contrast with statements (L) and questions (H). Independent of the phonological analysis, it is true that, at least in some cases, postnuclear falls present different shapes in SPT and QNF, as Figure~\ref{fig201} shows. For this reason, along with the rise shape discussed above, 
the fall shape factor was included in the design of the perception experiment. In order to be able to tell apart the contribution of these two factors, we tested them both individually and jointly, thus creating three different manipulation sets (i.e. rise, fall, both rise and fall).

\subsection{Hypotheses}\label{sec320}
E1 allows us to test two hypotheses concerning the classification as narrow focus questions or partial topic statements of trisyllabic stimuli bearing a nuclear accent:

\begin{description}
   \item[H1:] \textit{identification is affected by rise shape}. The production experiment reported in Section~\ref{sec2} showed different interpolation forms between the tonal targets composing a rising accent. According to the null hypothesis, these differences are to be considered redundant phonetic information. The alternative hypothesis is that rise shape is perceptible and indeed used in classification.
   \item[H2:] \textit{identification is affected by fall shape}. Production data on prenuclear accents indicate that questions and statements are characterized by different fall shapes. According to the null hypothesis, these findings can not be extended to nuclear accents.
\end{description}

Given current limitations in our understanding of trading relationships between supposed phonetic details in different dimensions, we will restrain from explicitly formulating hypotheses on the joint manipulation of rise and fall.

Support for the alternative hypotheses will be evaluated by fitting subjects' responses to a Logit Mixed Model and by gauging the statistical significance of the factor \textit{Stimulus step} (from allegedly SPT-like to QNF-like) for each of the two manipulation sets, namely rise shape (for H1) and fall shape (or scaling of the postaccentual elbow for H2). The significance level was set to $<$ 0.05.

\subsection{Method}\label{sec321}
The first forced-choice categorization task involved 22 native speakers of NI, mainly undergraduate science students with no training in linguistics. The experiment took place in a silent room, using a personal computer and a professional headphone set.\footnote{We would like to thank Franco Cutugno for allowing us to use the facilities at \textit{DSFMN} (Dipartimento di Scienze Fisiche, Matematiche e Naturali), Naples University ``Federico II'', as well as Bogdan Ludusan for his assistance.} Subjects listened to audio stimuli and had to identify them as either Questions or Statements. They were asked to put their index fingers in resting position above one of the two designated computer keys, each bearing a coloured sticker label (blue on far left of the keyboard, red on the right). The colour code was reminded throughout the whole experiment by on-screen instructions which associated colours to the Italian labels \textit{Domanda} `Question' and \textit{Risposta} `Answer', and was counterbalanced across speakers. Stimulus presentation and response recording were managed by the software \textit{Perceval} \citep{andre2003perceval}. The task lasted about 30 minutes and subjects were allowed to take a break between any of the five blocks.

Each block was composed of 28 experimental resynthesized items and 18 natural control items. We used 2 repetitions of the first word from 9 utterances recorded by a single speaker in the \textit{Tre Grazie} corpus as control items. Two SPT and two QNF utterances of the two sentences \textit{Valeria viene alle nove} and \textit{Amelia dorme da nonna} were cut after the Subject, yielding 8 of the control items.\footnote{See Section~\ref{sec221} for more details on the test sentences (6-8) and the pragmatic contexts (9-10).} The ninth control item also served as the starting point for resynthesizing experimental items, and consisted in the first word of the sentence \textit{Milena lo vuole amaro}, uttered as a SPT. The control items, along with the full utterances they were extracted from, were used in the training phase to make sure that subjects understood the task and the labels employed. This was particularly important, since SPT utterances are not as prototypical of the Statement category as QNF utterances are of the Question category (see Section~\ref{sec323}). Control items were also used to determine a baseline for correct classification of resynthesized items (see Section~\ref{sec322}). 

Experimental stimuli were built by modifying the melodic properties of the base natural stimulus described above (\textit{Milena} as SPT) using the \textit{PSOLA} algorithm \citep{moulines1990pitchsyncronous} embedded in \textit{Praat} \citep{boersma2008praat}.\is{resynthesis} The first manipulation consisted in resynthesizing a base experimental stimulus with ambiguous \textit{f0} features between those of a SPT and a QNF (see Figure~\ref{fig303}, solid line). To achieve this, we calculated values for scaling and alignment with respect to the stressed vowel’s boundaries of the \textit{f0} peak (H) and the inflection points on its left (L1) and its right (L2) (see Figure~\ref{fig303}, dotted line) for the base natural stimulus. Then we averaged these values with the ones extracted from a QNF utterance of the same speaker for the same sentence (see Figure~\ref{fig303}, dashed line). 

\begin{figure}[p]
\centering
\includegraphics[height=.3\textheight]{images/303.png}
\caption{Base stimulus averaging.}
\label{fig303}\end{figure}


\begin{figure}[p]
\centering
\resizebox{\linewidth}{!}{\includegraphics{images/304.png}}
\caption{Sketches of the 4 manipulation sets (1: L1 alignment; 2: rise shape; 3: L2 scaling; 4: rise shape and L2 scaling), aligned with the name \textit{Milena}.}
\label{fig304}\end{figure}


The resulting stimulus was used as the starting point for further \textit{f0} manipulations. Four sets of stimuli were created by manipulating three dimensions and one of their combinations (see Figure~\ref{fig304} and Table~\ref{tab31}). The three dimensions were L1 alignment, rise shape and L2 scaling, whereas the fourth set combined rise shape and L2 scaling manipulation. Each of the 4 sets was composed by 7 steps which were equally spaced in frequency as expressed in Hz.\footnote{In the first set (which we will not discuss any further; see Footnote~\ref{footnoteL1}) we manipulated L1 alignment, so steps were rather equally spaced in time.} Values of the ambiguous base stimulus were assigned to the central step (n. 4); values of the original stimuli were assigned to penultimate steps in the two directions (i.e. SPT = 2 and QNF = 6). This allowed us to create 4 sets which, for each parameter, went from an overtly SPT characterized stimulus (n. 1) to an overt QNF (n. 7). 


All in all, 3080 responses were gathered for experimental items (22 subjects x 5 blocks x 4 sets x 7 steps) and 1980 for control items (22 subjects x 5 blocks x 2 repetitions x 9 stimuli).


\begin{table}[t]
\centering
\begin{tabular}{c c c}
\mytoprule
Set & Feature(s) & Steps \\
\midrule
1 & L1 alignment & 15 ms \\
2 & Rise midpoint & 15 Hz \\
3 & L2 scaling & 10 Hz \\
4 & Rise midpoint and L2 scaling & 15 and 10 Hz \\
\mybottomrule
\end{tabular}
\caption{Summary of manipulations}
\label{tab31}\end{table}


\subsection{Results}\label{sec322}
Responses to control stimuli show that listeners found the task very difficult. During the training phase, no subject experienced difficulties in classifying the uncut stimuli. However, during the test phase, subjects had to classify 90 (5 blocks x 9 items x 2 repetitions) natural stimuli cut after the first word. Results show that only 5 out of 22 listeners managed to make a reliable distinction (above 60\% of correct classification) between unresynthesized SPTs and QNFs.

Figure~\ref{fig305} (left panel) plots the frequency of Statement responses (y-axis) against the seven manipulation steps (x-axis) for the three individual sets (rise curvature, L2 scaling, rise curvature and L2 scaling), pooled across all subjects. The results do not show the expected trend to higher Statement responses for the first (1-3) manipulation steps. 

\begin{figure}[hbpt]
\centering
\includegraphics[height=.3\textheight]{images/305.png}
\caption{Experiment I. Frequency of Statement responses for the three relevant manipulations sets (2-4 in Table 3.1). Data for all subjects (left panel) and for the 5 subjects with best performance on control items (right panel).}
\label{fig305}\end{figure}

Given the results of the control items classification, we decided to plot separately the responses to experimental stimuli for the five subjects with the highest control performance, in order to ascertain whether the degree of sensitivity to resynthesis was different across subjects. However, as Figure~\ref{fig305} (right panel) shows, no trends are discernible for this subgroup either. Statistical analyses are omitted, since the visual inspection of the results clearly indicates no effect of any of the experimental treatments on subjects' responses, for both groups of subjects. The inspection of the two panels in Figure~\ref{fig305}, however, shows that the five ``reliable'' subjects had a slight bias towards the Statement response, irrespective of the dimension, the direction and the intensity of the manipulation (see Section~\ref{sec33} for discussion).

\subsection{Discussion}\label{sec323}
Results show that subjects do not use L1 alignment, rise shape or L2 scaling as cues for the classification of trisyllabic stimuli as narrow focus questions or partial topic statements. The interpretation of negative results being an epistemologically complex operation, in the following we will concentrate on some hypotheses accounting for this outcome, and on further testing needed in order to validate them.

First of all, it is possible that our manipulations did not involve phonetic information actually used in classification. In this case, the differences in rise shape we found to be consistent in production (see Section~\ref{sec2}) could be deemed perceptually irrelevant by-products of other paradigmatic options, as the tonal specification of the accentual phrase \citep{petrone2008tonal} or the compression of postnuclear register \citep{dimperio2011phrasing}. This perspective would constitute evidence for the appropriateness of a strongly abstractionist approach to intonation, in which some of the phonetic information contained in the signal, even if consistently present, is actually discarded in the mapping phase.

Going a step further, we could also hypothesize that our manipulations are not only unused in classification, but are also not perceptible at all. In this case, rise shape differences could absolutely not qualify as prosodic detail, but only as side effects of phonetic implementation. This hypothesis could be readily tested with a discrimination task; however, informal testing by three NI native speakers suggested that different steps along the manipulated dimensions are indeed discriminable. For this reason, we cannot rule out without further testing the option that task-related issues affected the validation of the research hypothesis. 

One of the issues which might have played this role is the difficulty of the task itself. The poor performances in correct classification of natural control stimuli seem to strengthen this view. If only 5 out of 22 subjects managed to make a somewhat reliable (above 60\%) classification of natural stimuli, we can not expect high performances on experimental items either. It is true, after all, that no subject experienced any difficulties in correctly classifying the uncut utterances from which the stimuli were excerpted. Test items, however, consisted in a single trisyllabic word, and control items consisted in one of two trisyllabic words as well. Each subject had to listen to 230 short and similar items. Moreover, 150 out of these 230 items consisted in various forms of the same one proper name used for the experimental stimuli (namely \textit{Milena}). During informal post-experiment interviews, nine subjects stated that one of the names in the test was very frequent, and almost all of them reported to have found the test frustrating and boring for this very reason.

However, as anticipated in the Method section (see Section~\ref{sec321}), additional difficulties might have arisen from the interaction between the category labels used for classification (namely \textit{Question} and \textit{Answer}) and the particular pragmatic contrast under examination (i.e. question narrow focus vs statement partial topic).\is{compositionality}\is{partial topic} Even if we made sure that subjects could make a reliable association between stimuli and labels during the training phase, it is nonetheless true that QNF and SPT differ in how strongly they can represent questions and statements respectively. SPT, in particular, can be thought of as partial answer --- that is a statement, but one calling for an integration. QNF, on the contrary, can be seen as more prototypically representing the question category.

In retrospect, the negative results presented in this section could have been determined by a variety of more or less controllable experimental factors. For this
\enlargethispage{1em}
reason we devised and ran a second experiment, before dismissing the hypothesis of the perceptual relevance of rise shape in pitch accent categorization altogether.

\section{Experiment 2}\label{sec33}
\subsection{Background}\label{sec3300}
E2 was meant to reduce the impact of all the task-related factors which could have hindered the appreciation of the perceptual relevance of rise shape in E1. As discussed in the preceding section, the task might have been made more difficult by the excessive presence of the test word in each block. This was due to the fact that four (sets of) cues were manipulated, each in seven steps. For E2 then, we decided to test the most relevant feature alone, namely rise curvature.

More importantly, the task might have suffered by the association of SPT and QNF with the labels \textit{Answer} and \textit{Question}, since partial topics are characterized by openness and non-conclusiveness, thus not qualifying as prototypical statements. Instead of modifying the labels, since issues in compositionality of pragmatic meaning could still have affected classification choices (see Section~\ref{sec243}), we decided to use a more clear-cut contrast on the meaning side. Narrow focus questions were contrasted to Narrow Focus Statements (\textit{SNF}),\is{narrow focus} thus permitting a more straightforward association with our two labels. It is true that nuclear pitch accents in SNF are characterized by an earlier peak alignment than both QNFs and SPTs, and that shape proprieties in SNF have not been directly contrasted with those from QNF. However, an informal examination of the \textit{Tre Grazie} corpus showed that SNF exhibit a concave rise, as in the case of SPT (see also examples from \citealt{dimperio2008phonetics}).\footnote{SPTs appear to share features of QNF and SNF in both substance and meaning: on the phonetic side, they are characterized by QNF peak alignment and SNF rise shape, while on the pragmatic side they share SNF sentence modality and QNF openness. An utterly compositional approach to intonational meaning could suggest a direct link between the two sets. Given the arguments exposed in Section~\ref{sec243}, we will not pursue this hypothesis here.} Moreover, in the perspective of research on prosodic detail, the fact that SNF and QNF have different peak alignment could actually represent an asset. Since the role of peak alignment in question-statement classification has been shown to be crucial (\citealt{dimperio2002italian}, among others), it is reasonable to hypothesize that if rise shape is also a cue to sentence modality, its role will be ancillary to stronger cues such as peak alignment.\is{resynthesis} By creating stimuli with ambiguous timing of the peak and by manipulating rise shape, we have the opportunity to test if listeners rely on prosodic detail in the very condition in which they are supposed to do so, namely when other stronger and already acknowledged cues are not available. E2 will then test the hypothesis that classification of utterances as narrow focus question or statement will be influenced by the scaling of the midpoint of the rise when peak timing is ambiguous. 

For E2, a last improvement was devised. Results from E1 showed that subjects with higher correct classification rate of control stimuli had a consistent bias towards more \textit{Answer} responses for experimental stimuli (see Section~\ref{sec322}).\footnote{Bias towards statement responses is not infrequent in various percceptual tasks: see \citet{petrone2014intonation} for a discussion of the possible statistical or cognitive \citep{pandelaere2006question} reasons behind the phenomenon.} Recall that experimental items were created by manipulating an original SPT stimulus, which we tried to de-characterize by averaging out some of its melodic proprieties with those extracted from a QNF stimulus. However, listeners might also have paid attention to other features in the original stimulus, for example details along other prosodic cues, such as intensity, duration, or even voice quality or segmental proprieties. This hypothesis can be tested by using two stimuli (extracted from sentences uttered in different pragmatic contexts) instead of one as a basis for further manipulations. Support to this hypothesis would disclose the possibility of investigating prosodic detail not only within intonation contours, but also along other prosodic dimensions, with potentially severe implications on phonological modelling.

\subsection{Hypotheses}\label{sec330}
E2 thus allows us to test two hypotheses concerning the classification as questions or statements of trisyllabic stimuli with ambiguous peak alignment of the nuclear accent:

\begin{description}
   \item[H1:] \textit{identification is affected by rise shape}. The production experiment reported in Section~\ref{sec2} showed different interpolation paths between the tonal targets composing a rising accent. According to the null hypothesis, the negative findings of E1 in Section~\ref{sec32} suggest that these differences are to be considered redundant phonetic information. The alternative hypothesis is that rise shape is perceptible and indeed used in classification, and that the negative findings of E1 are due to a number of confounding factors in the administered task.
   \item[H2:] \textit{identification is affected by non-melodic cues}. The responses of the most reliable listeners in E1 had a bias towards the category of the stimulus used as a source for resynthesis. Since \textit{f0} was made ambiguous between the two tested categories, phonetic information other than \textit{f0} must be recoverable and indeed used for classification. According to the null hypothesis, intonational cues alone are at work in pitch accent categorization.
\end{description}

Support for the alternative hypotheses will be evaluated by fitting subjects' responses to a Logit Mixed Model and by gauging the statistical significance of the factors \textit{Stimulus step} (that is, degrees of concavity or convexity in rise shape manipulation) and \textit{Base stimulus} (for stimuli resynthesized starting from either a Question or a Statement), for H1 and H2 respectively. Significance level was set to $<$ 0.05.

\subsection{Method}\label{sec331}
15 Neapolitan Italian native speakers took part in the second forced-choice categorization task. They were mainly undergraduate students from various faculties of Naples' University, and none had training in prosody and intonation. They performed the task using their own computers and headphones, after downloading 5 soundfiles (one for each block) and 1 textfile (the answer sheet) from a website.
Subjects had to listen to the soundfiles and write their answer on the textfile; they were asked not to pause during blocks, but no restrictions were given as for pauses between blocks. In each block, stimuli were separated by 5 seconds of silence, during which subjects were supposed to write down their answer. No subject reported problems in performing this operation within the time they were given. Each of the 5 blocks was 5 minutes long, so the entire experiment lasted approximately 30 minutes. 

As mentioned above, E2 differs from E1 in (1) the use of Narrow Focus instead of Partial Topic, (2) the manipulation of two base stimuli instead of one and (3) the manipulation of a single dimension (namely rise shape) instead of four. However, as for E1, stimuli consisted in utterances recorded for the \textit{Tre Grazie} corpus.\footnote{See Section~\ref{sec321} and especially Section~\ref{sec221} for details on the elicitation procedure. SNF utterances were preceded by a contextualizing question suggesting a wrong instantiation for the Subject position, as in ``Is it Mary the one who arrives at 9?'' preceding the target sentence \textit{Valeria viene alle nove}.}

Using SNF instead of SPT affected the creation of the ambiguous stimuli to be used as the basis for further manipulations. Unlike SPTs, SNF have a peak aligned earlier than QNF, a phonetic property mirrored in the different analyses and transcriptions of Narrow Focus accents in questions (L*+H, see Section~\ref{sec212}) and statements (L+H*). Averaging the peak alignment required a 2 ms manipulation for E1 and a 15 ms manipulation for E2. However, informal testing from three NI native speakers confirmed that the resulting stimuli sounded both ambiguous and natural, thus qualifying as viable bases for further manipulations.

The use of two base stimuli instead of one did not lead to an increase in the number of experimental stimuli because instead of manipulating four (sets of) cues we only manipulated one. For E2 there were 14 test items (2 base stimuli x 1 dimension x 7 steps), i.e. half of those used in E1 (1 base stimulus x 4 dimensions x 7 steps). This allowed us to increase from 18 to 34 the number of control natural items. Control items were composed by the two unresynthesized base stimuli and by two repetitions of 16 trisyllabic Subjects extracted from the \textit{Tre Grazie} corpus. This time, we excerpted the names from 4 utterances of 2 different sentences in the 2 pragmatic contexts. Again, the first names used for control items were different from the one used for test items, but in this case the percentage of the three first names was perfectly balanced.\footnote{7 steps x 1 set x 2 ambiguous bases + 2 natural bases = \textbf{16} \textit{Valeria} as experimental items, and 4 utterances x 2 contexts x 2 repetitions = \textbf{16} for both \textit{Amelia} and \textit{Milena} as control items.}, whereas the test name was almost four times as frequent as each of the other two in E1.\footnote{7 steps x 4 sets x 1 ambiguous base + 2 repetitions x 1 natural base = \textbf{30} \textit{Milena} as experimental items, and 2 utterances x 2 contexts x 2 repetitions = \textbf{8} for both \textit{Amelia} and \textit{Valeria} as control items.} Subjects participating to E1 spontaneously reported a certain degree of sensitivity to the relative frequency of the test items' name (see Section~\ref{sec323}). Thus, for E2 stimuli we rescaled the relative frequencies of the three names. Moreover, after the test we asked subjects whether they considered one of the three names to be frequent than the others, but none reported any noticeable skew.

As for the experimental items, we modified the curve index by shifting the height of rise midpoint in 7 steps, identical to the procedure for the second set of E1, using the \textit{PSOLA} algorithm \citep{moulines1990pitchsyncronous} embedded in \textit{Praat} \citep{boersma2008praat}.\is{resynthesis} This was done for both ambiguous base stimuli. The central step (n. 4) was assigned a value that corresponded to a linear interpolation between L1 and H; steps from 3 to 1 had progressively higher height values, corresponding to a progressively increasing concave interpolation, and steps from 5 to 7 had progressively lower height values, corresponding to a more and more convex interpolation (see Figure~\ref{fig306}). Step size was 15 Hz, as determined through the use of actual values of rise midpoint in the two base stimuli (used as penultimate in both direction) and the number of steps.

\begin{figure}[t]
\centering
\includegraphics[height=.4\textheight]{images/306.png}
\caption{\textit{f0} contours and averaged phone segmentation for original stimuli (dashed: statement; dotted: question) and resynthezied items (numbers indicate continuum's ends).}
\label{fig306}\end{figure}

We gathered responses for 1050 experimental items (15 subjects x 5 blocks x 7 steps x 2 base stimuli) and 2550 control items (15 subjects x 5 blocks, each composed by 2 base stimuli and 2 repetitions of 16 natural stimuli).

\subsection{Results}\label{sec332}
Responses to control stimuli show that listeners found this task far easier than the preceding one. We had 170 control stimuli for each subject (5 blocks x (2 base stimuli + 2 repetitions x 16 natural stimuli)), and this time only one speaker out of 15 did not reach the 60\% correct response threshold (17 out of 22 in E1). 

The top panel of Figure~\ref{fig307} shows the observed responses to experimental stimuli. Percent of question responses is plotted on the y-axis and step in manipulation on the x-axis (1 being the most concave and 7 the most convex). Results for items created from the two base stimuli are plotted separately (solid line: question, dashed line: statement). Results show a trend to more question responses for more convex rises. Observed responses to experimental stimuli were fitted to a Logit Mixed Model, in which \textit{Stimulus step} (1 to 7) and \textit{Base stimulus} (Statement or Question) were chosen as fixed factors, while \textit{Subjects} was assigned random status (with variable slope and intercept). Results of the model are shown in Figure~\ref{fig307}, bottom panel.

\begin{figure}
\centering
\resizebox{0.9\linewidth}{!}{\includegraphics{images/307.png}}
\caption{Experiment II. Observed (top panel) and estimated (bottom panel) question response frequency (y-axis) as a function of manipulation step (x-axis, 1 being very concave and 7 being very convex) and grouped according to base stimulus (solid line: question; dashed line: statement).}
\label{fig307}\end{figure}

\textit{Stimulus step} and \textit{Base stimulus} proved to be highly significant (respectively, beta= 0.252,  z=3.1, p $<$ 0.002 and beta= -1.196, z=-3.3, p $<$ 0.001), while the interaction between the two factors was not significant (z= -0.018, p= 0.98), indicating that the slopes relative to the two base stimuli are not significantly different. 

\enlargethispage{\baselineskip}
This means that, although a Stimulus step effect can be recovered for both continua, items obtain consistently different scores according to the base stimulus from which they are resynthetized. As Figure~\ref{fig307} (right panel) shows, the Question-based stimuli always elicited more Question responses than the respective Statement-based stimuli. Moreover, if for S-based stimuli there is an actual shift in perception,\footnote{From less than 40\% Q-responses for step 1 to more than 60\% Q-responses for step 7, through about 50\% Q-responses for the intermediate step 4.} Q-based stimuli only display a strengthening of Q-responses.\footnote{From more than 60\% for step 1 to more than 90\% for step 7.}

\subsection{Discussion}\label{sec333}

Results from E2 suggest that NI listeners do use rise shape information in order to classify stimuli with ambiguous peak alignment as either questions or statements. Lower rise midpoints (i.e. convex rises) cue more question responses while higher rise midpoints (i.e. concave rises) cue more statement responses, independently of the nature of the stimulus used as starting point for the resynthesis.

As the results for control items show, subjects found the task involving QNF and SNF far easier than the one involving QNF and SPT. 14 subjects out of 15 had a correct classification rate on control stimuli of above 60\%, indicating that judgements on trisyllabic stimuli are reliable. This shows that the low performances recorded for E1 are indeed due to the specific pragmatic contrast under examination rather than to the task itself (see Section~\ref{sec323}), even if it is reasonable to assume that the rescaling of first name proportions also had a positive effect on the listeners' attention (see Section~\ref{sec331}).

Performance rates on control stimuli show that task results are reliable, but the magnitude of the shift in the subjects' responses to experimental stimuli indicate that rise shape is not a primary cue.\is{f0 dynamics} While resynthesis of peak alignment can yield up to a 90\% shift in subjects' responses to a question-statement classification task \citep[§3, among others]{dimperio2000role}, in our experiment the shift is only around 30\%. This is consistent with the role of shape information as prosodic detail: in normal conditions, listeners would rely on peak alignment information.\is{prosodic detail} When this information is removed (through averaging in resynthesis), subsidiary information from rise shape can partially surrogate the disambiguation load.

Our results also show that the nature of the base stimulus has a significant effect on subjects' responses. Items resynthesized from a question or statement base always elicited more question or statement responses, respectively. We could speculate that the procedure employed to de-characterize base stimuli (see Section~\ref{sec321}) did not really produce ambiguous items, as it only involved averaging out tone scaling and peak alignment. In more general terms, the procedure only involved \textit{f0} manipulations, but we cannot exclude that modality contrasts could also be signalled by cues other than \textit{f0}, such as spectral or rhythmic cues \citep[§5, among others]{niebuhr2010pitchaccent,dimperio2000role}. With regard to the issue of prosodic detail, this finding is particularly interesting: phonological representations of intonation could be underspecified not only with respect to details of phonetic information relative to the fundamental frequency, but also with respect to other prosodic cues.

\section{General discussion}\label{sec34}
The two experiments reported in this chapter aimed at evaluating the perceptual role of shape differences in rising nuclear accents, in order to determine whether the phonological representation of pitch accents must be enriched with dynamic \textit{f0} information. In summary, we could say that E1 and E2 bring mixed evidence to our research question, but they also point towards its broadening. 

In the first experiment we asked our listeners to classify short manipulated stimuli as QNF or SPT. Manipulations were carried out according to the production evidence found in Section~\ref{sec2} (where rises were more convex in Questions and more concave in Statements), under the hypothesis that regularities in production are exploited in perception. The results show that classification is not affected by rise shape manipulation, thus invalidating our research hypothesis. However, the results could have been affected by a number of factors linked to the nature of the task itself, rather than actually mirroring a total lack of significant effects. We devised a follow-up experiment (E2) in order to reduce the impact of some of these possibly confounding factors. From an epistemological point of view, the evaluation of negative results is a very complex operation, which requires the exploration of several repair strategies. In the following, we discuss three improvements we did not apply to E2 (Section~\ref{sec341}; see Section~\ref{sec331} for a discussion of the adopted ones). Then we turn to the main finding of E2, namely the influence of base stimulus category on subjects' responses, and discuss how it reshapes the research question we tackled so far (Section~\ref{sec342}).

\subsection{Possible task improvements}\label{sec341}
To begin with, a first remark is that shape differences cannot be used in classification if they are not perceived at all. That is, if negative results to the classification task proposed in E1 had been complemented by negative results for a parallel discrimination task, dismissing the hypothesis of the perceptual relevance of shape differences would have been easier. We thus proceeded to an informal evaluation of the discriminability of pitch accents with different rise shapes. Given the limited pool of available subjects for this additional task and the encouraging results of the informal exploration (in which three out of three listeners reported to ``hear a difference'' between concave and convex rises), we decided to concentrate our efforts in devising a second classification task. The epistemological assumption behind this choice was that positive results to a second task could have allowed us for a clearer interpretation of the negative results to the first.

Another possible improvement could have been the use of a different task. We have seen in the introduction that identification is not the only task available to the researcher: indeed, semantic differential and context matching both have already been used for the exploration of the perceptual role of prosodic detail. When discussing the results from E1, we stated that one of the possible confounding factor was the use of the labels \textit{Question} and \textit{Answer} for the classification of narrow focus questions and partial topic statements. Given the particularly open pragmatic value of SPT, it is possible that the two labels would not represent equally well the two categories. In E2, we overrode this issue by using narrow focus statements instead of SPT, thus resetting the symmetry between categories and labels. However, it could be noted that the use of a context matching task would have eluded the categories and label issue from its very roots. Upon listening to the excerpted short stimuli, subjects could have been asked to either choose the most appropriate completion between the ones typed on screen, or to rate the appropriateness of one single completion. We could also have avoided the use of possibly misleading labels through the use of a semantic differential task. However, which semantic scales would have allowed us for a clear characterization of QNF and SPT? As we said above, SPT share with Q(NF) a certain degree of openness, in that they qualify as both giving information and suggesting that more information is needed. This feature could have complicated the positioning of SPT on the other hand of QNF on any given semantic scale. For these reasons, we decided to drop the SPT context altogether, in favour of a more straightforward pragmatic contrast. This obviously introduced an asymmetry between the exploration of production and perception data on two different contrasts (production of SPT vs QNF in Section~\ref{sec2}; perception of SNF vs QNF in Section~\ref{sec3}, E2). However, this choice also had the advantage of situating our experiment in the broader frame of other studies on phonetic detail along contrasts in sentence modality, as the work on prenuclear falls in both NI and Northern Standard German we discussed in the introduction (Section~\ref{sec31}).

One last possible improvement to the task was the use of stimuli modified in order to maximize the listeners' attention to the possibly relevant cues under investigation. The use of degraded stimuli, for example, has proven useful in the exploration of the role of phonetic detail in speaker recognition \citep{sheffert2002learning}. However, as responses to control items in E1 show, no further complexification of task or stimuli was possible, since the subjects already reported serious difficulties in performing the original task, even with simply excerpted (and thus not resynthesized) stimuli.

While the choice of not using degraded stimuli or a semantic differential task was motivated by the reasons we exposed above, the options of a discrimination task or of a context matching had no clear drawbacks, and were discarded only out of the relative unavailability of NI native speakers at the Université de Provence. 

\subsection{A broader research question}\label{sec342}
The perceptual evaluation of shape differences in nuclear rises was instrumental in deciding whether phonological representations of pitch accents should be enriched with dynamic melodic information. The experiments reported in this chapter bring mixed evidence to this research question. On the one hand, the shape differences attested in the SPT vs QNF contrast in production were not found to be used in perception. On the other hand, we documented an effect of rise shape on classification in the SNF vs QNF contrast, a contrast for which we had not documented shape differences in production yet. Perhaps the most linear way to complement our positive perception results was to verify the presence of rise shape differences in production for SNF vs QNF as well. However, we felt that E2 yielded evidence for a phenomenon which could reshape our initial research question altogether: if listeners responses are biased by the nature of the base stimulus even when \textit{f0} contours are made the same (see the offset between the two curves in Figure~\ref{fig307}), then phonological categories could need an enrichment not only with respect to melodic information, but also to phonetic information along different dimensions, such as intensity, duration, voice quality or spectral composition. 

Research on intonational phonology has long acknowledged that cues other than \textit{f0} could play a role in the signalling of post-lexical meaning \citep{hirschberg1992influence}: base-related effects have been reported in perceptual experiments since \citet{dimperio2000role}, and have recently made the object of direct investigation \citep{niebuhr2010pitchaccent}. If other phonetic cues are involved in coding and decoding prosodic categories, the question of whether phonological representations of pitch accents should include dynamic \textit{f0} detail could be generalized and reshaped as to ask whether phonological approaches to intonation should include non-\textit{f0} information. That is, investigating prosodic detail would mean not only to concentrate on unacknowledged features (such as shape) for acknowledged dimension (\textit{f0}), but to unacknowledged features themselves. 
Throughout the remeinder of this book we will verify the potential importance of one of these dimensions, namely tempo. Therefore, a discussion as to whether or not the shape differences explored in this chapter should be accommodated into phonological descriptions of intonation needs to be postponed. If duration or intensity variations have to be included in the representation of pitch accents, the restructuring will be deeper than if only new features for already existing dimensions had to be added. For this reason, before suggesting any account of if and how shape differences should be included in phonological representations of pitch accents, we will turn to the potential role of other prosodic cues as prosodic detail.

\section{Conclusion}\label{sec35}
In this chapter we evaluated the perceptual role of rise shape as documented for production in Section~\ref{sec2}. Two experiments based on identification tasks showed that listeners perceive the difference between concave and convex nuclear rises, and that they exploit it for classification purposes. While the first experiment failed to show that classification of Partial Topic Statements and Narrow Focus Questions is affected by rise shape, evidence from the second experiment shows that these negative results might be due to task-related issues. This is because, as soon as stimuli are made ambiguous with respect to the main cue of peak alignment, listeners do use rise shape information in order to classify them as Narrow Focus Questions or Statements. 

While not being explicitly tested, a strong effect of the nature of the stimulus used as base for \textit{f0} manipulations was also found in both experiments. In the first experiment, where we only used stimuli resynthesized from a (Partial Topic) Statement base, the listeners who showed the best performance on control stimuli showed a consistent Statement bias for test items. In the second experiment we decided to resynthesize test stimuli from both (Narrow Focus) Question and Statements. Responses from all subjects displayed a strong base effect: for a given step on the continuum in the manipulation of shape proprieties, stimuli resynthesized from a base question always elicited more question responses. That is, while looking for evidence for phonetic detail within a single prosodic dimension (rise shape within \textit{f0} contours), we found reason to believe that other entire prosodic dimensions (duration, intensity, voice quality, spectral proprieties) could represent a source for phonetic detail as well.

\enlargethispage{\baselineskip}
Given the possibility of a radical restructuring of phonological representations for pitch accents due to the inclusion of new prosodic dimensions in intonational phonology, our investigation of the role of dynamic melodic detail risks to qualify as minimalist and inconclusive. For these reasons, before returning to the issue of the enrichment of phonological categories, in the next experimental chapters we turn to the evaluation of the role of one of these possible additional dimension, namely tempo, both from a production (Section~\ref{sec4}) and a perception (Section~\ref{sec5}) viewpoint.

\documentclass[output=paper,colorlinks,citecolor=brown]{langscibook}
\ChapterDOI{10.5281/zenodo.15682190}

\author{Svetlozara Leseva\orcid{0000-0001-8198-4555}\affiliation{Department of Computational Linguistics, Institute for Bulgarian Language, Bulgarian Academy of Sciences}}

\title[The conceptualisation of the route: Non-directed and directed motion]{The conceptualisation of the route: Non-directed and directed motion verbs in Bulgarian and English} 

\abstract{This chapter offers an analysis of non-directed and directed motion verbs from a frame semantics perspective through exploring the semantic description and syntactic realisation of the frame elements of several semantic frames in FrameNet. The study is focused on the conceptualisation and syntactic expression of the elements of the route along which motion occurs: \fename{Goal} (the final part of the route), \fename{Source} (the initial part of the route) and \fename{Path} (the middle part of the route) in English and Bulgarian by studying the syntactic expression of the corresponding frame elements in FrameNet. The research questions explored in the chapter deal with the prominent aspects in the semantics of the verbs evoking a particular semantic frame, the syntactic expression of the relevant frame elements, syntactic explicitness and implicitness. The empirical evidence provided by the FrameNet corpus is compared with a sample of annotated Bulgarian examples. The observations made throughout the chapter are brought in the perspective of linguistic hypotheses put forward in the literature: in particular, the goal-over-source hypothesis and the proposal that motion verbs tend to co-occur with expressions that align with the part of the trajectory of motion that is most prominent in their semantics.  
}

\IfFileExists{../localcommands.tex}{
   \addbibresource{../localbibliography.bib}
   \usepackage{langsci-optional}
\usepackage{langsci-gb4e}
\usepackage{langsci-lgr}

\usepackage{listings}
\lstset{basicstyle=\ttfamily,tabsize=2,breaklines=true}

%added by author
% \usepackage{tipa}
\usepackage{multirow}
\graphicspath{{figures/}}
\usepackage{langsci-branding}

   
\newcommand{\sent}{\enumsentence}
\newcommand{\sents}{\eenumsentence}
\let\citeasnoun\citet

\renewcommand{\lsCoverTitleFont}[1]{\sffamily\addfontfeatures{Scale=MatchUppercase}\fontsize{44pt}{16mm}\selectfont #1}
  
   %% hyphenation points for line breaks
%% Normally, automatic hyphenation in LaTeX is very good
%% If a word is mis-hyphenated, add it to this file
%%
%% add information to TeX file before \begin{document} with:
%% %% hyphenation points for line breaks
%% Normally, automatic hyphenation in LaTeX is very good
%% If a word is mis-hyphenated, add it to this file
%%
%% add information to TeX file before \begin{document} with:
%% \include{localhyphenation}
\hyphenation{
affri-ca-te
affri-ca-tes
an-no-tated
com-ple-ments
com-po-si-tio-na-li-ty
non-com-po-si-tio-na-li-ty
Gon-zá-lez
out-side
Ri-chárd
se-man-tics
STREU-SLE
Tie-de-mann
}
\hyphenation{
affri-ca-te
affri-ca-tes
an-no-tated
com-ple-ments
com-po-si-tio-na-li-ty
non-com-po-si-tio-na-li-ty
Gon-zá-lez
out-side
Ri-chárd
se-man-tics
STREU-SLE
Tie-de-mann
}
   \boolfalse{bookcompile}
   \togglepaper[23]%%chapternumber
}{}

\usepackage{longtable}

\begin{document}
\maketitle

\section{Introduction}\label{ch4:intro}
  
This chapter deals with the semantic and syntactic description of motion verbs in Bulgarian (as compared with English) with respect to: their semantics as described in terms of semantic frames; the conceptualisation of parts of the trajectory of motion and the corresponding frame elements; the syntactic realisation of the major frame elements as reflected in corpora. 

The study is based on the description of verbs in FrameNet \citep{Baker1998} as lexical units evoking particular frames, defined themselves as schematic representations of situations in terms of the configurations of participants and props that constitute their meaning. The syntactic description will be focused on the main patterns of syntactic expression of the most essential frame elements in and across the selected motion frames. The proposed account aims at capturing the semantic and syntactic properties of Bulgarian verbs of motion against a more universal background.

%The theoretical grounding of these frames originates in the theory of Frame Semantics \citep{Fillmore2003,Ruppenhofer2016}. In addition, semantic frames specify the syntactic valence patterns associated with lexical units that belong to the relevant frame on the basis of corpus annotation. 

The analysed verbs are selected from the Bulgarian WordNet \citep{koeva2021-wordnet}, which were associated with FrameNet frames \citep{LesevaStoyanova2020} and further aligned with verbs in other resources where possible \citep{2022-Linked-Resources-towards-}. As a result, the verb synsets in the Bulgarian WordNet are mapped to FrameNet frames (one frame per synset), making it possible for observations to be made on the basis of the semantic representation available for English verbs. The FrameNet model has been widely adopted for building similar descriptions of the lexis of a number of typologically diverse languages -- German \citep{burchardt-etal-2006-salsa}, Dutch \citep{Vossen2018}, Danish \citep{pedersen-etal-2018-danish}, Swedish \citep{Borin-Lars2010-110368}, Latvian \citep{Gruzitis-EtAl:2018:IFNW},  French \citep{candito-etal-2014-developing}, Spanish \citep{Subirats+2009+135+162}, Brazilian Portuguese \citep{TimponiTorrent2018Chapter4T}, Chinese \citep{You2005BuildingCF}, Japanese \citep{Ohara2004TheJF}, Korean \citep{DBLP:conf/i-semantics/HahmKWWSKPHC14}, among others. For a more comprehensive description of the existing framenets and the Multilingual FrameNet annotation initiative,\footnote{\url{https://www.globalframenet.org/}} cf. \citet{Gilardi2018LearningTA}. FrameNet’s theoretical framework has been adopted for Bulgarian and extended into a model accounting for language-specific features, including verb aspect, semantic and syntactic diatheses and syntactic alternations. The concept was implemented in the development of the Bulgarian FrameNet \citep{Koeva2008-framenet, Koeva2010-framenet,svetla2021towards}.

This chapter will specifically address: (i) those aspects in the semantics of the verbs evoking the studied frames that are cast as any of the frame elements describing the motion of an entity along a trajectory; (ii) the syntactic expression of the relevant frame elements and the conditions predetermining their syntactic explicitness or implicitness. The empirical evidence provided by the examples in the FrameNet corpus will be studied against a sample of annotated Bulgarian examples, thus testing the cross-lingual validity of the theoretical and practical observations and drawing parallels or distinctions where appropriate. The observations made throughout the chapter will be analysed from the perspective of linguistic hypotheses that have been put forward in the literature: in particular: the goal-over-source hypothesis and the proposal that motion verbs tend to co-occur with expressions that align with the most prominent aspect of the trajectory of motion encoded in their semantics.

\section{Motion verbs}\label{motion-verbs}

The semantic representation of motion verbs has been the focus of a multitude of studies. One of the major distinctions in the verb lexis is the one between manner and result, which are usually viewed as complementary notions, i.e. verbs lexicalise either one or the other \citep{Levin2015}. In the domain of motion this differentiation criterion takes the form of a distinction between the expression (and possibly the conflation) of manner and path. It has been extensively studied by \citet{Talmy1985,Talmy1991,Talmy2000}, who offered a typology %of the so-called verb-framed and satellite-framed languages, 
characterising languages in terms of the lexicalisation patterns of motion events: in satellite-framed languages, verbs usually encode manner, while the path of the movement is encoded outside the verb (base) by satellites such as adverbial particles (but also prepositions and prefixes); in verb-framed languages, the path is expressed by the verbs, and manner is either omitted or realised by means of an adverbial expression. The discovery of finer typological distinctions across languages with respect to motion expressions has led to the refinement of the original Talmian typology in the works of a number of authors (\cite{Aske1989,Slobin1996b,Papafragou2002,Ibarretxe-Antunano2004,Slobin2004,Filipovic2007,Beavers2010,Croft2010}, among others). The interest in the elements that make up the trajectory, or path, of the motion (\cite[162]{Jackendoff1983}, \cite[57]{Talmy1985}, \cite[275]{Lakoff1987}) has been reflected in numerous studies on the lexical encoding and syntactic expression of these elements in co-occurrence with the verb %(e.g. \fename{Goal}-profiling verbs tend to co-occur with \fename{Goal}-PPs, \fename{Source}-profiling verbs with \fename{Source}-PPs) 
(\cite{Rohde2001,Rakhilina2004,StefanowitschRohde2004,Cristobal2010,Kopecka2010}, to mention but a few). A related line of research has been the study of the bias with respect to the expression of one path element over another in and across languages (\cite{Ikegami1987,DirvenVerspoor1998,StefanowitschRohde2004,WalchliZuniga2006,Verkerk2017}, among others).

%The aspects of motion verbs such as 
The distinction between manner and path of motion and the expression and profiling of different sections of the path, have been the prime focus of many other studies. For instance, \citet{Viberg2015} proposes a verb typology with respect to the expression of the endpoint of motion in Swedish in comparison with Eng-\newline lish, German, French and Finnish. In her study \citet{Kopecka2010} explores lexicalisation patterns of manner of motion verbs in Polish, while \citet{Lozinska2018} delves into the expression of path and manner in Polish and Russian in contrastive terms. \citet{Taremaa2017,Taremaa2021} has explored motion verbs in Estonian, focusing on the expression of source, goal, path, location and direction with both manner of motion verbs and source- and goal-profiling verbs. 

Various authors have previously adopted the FrameNet approach in the analy-\newline sis of motion verbs. \citet{Viberg2008} proposes a study of Swedish verbs of motion in a vehicle; the verbs have been analysed from a cross-linguistic perspective and with respect to their encoding in FrameNet. \citet{Cristobal2010} provides a detailed analysis of \framename{Arriving} verbs in English and Spanish. \citet{Imani2020} study a selection of manner of motion verbs in English and contrast them with their counterparts in Persian. 

A number of studies in these lines of research have been dedicated to Bulgarian motion verbs. \citet{tchizmarova:2015} analyses several verbs with respect to the way they divide the space of linear motion, including the co-occurrence with directional phrases. %, in particular: the source-and-path oriented \textit{отивам} (`go'), the path-and-goal oriented \textit{идвам, дойда} (`come'), the source-oriented \textit{заминавам} and \textit{тръгвам}, the path-oriented \textit{ходя, вървя} (`walk') and the goal-oriented \textit{пристигам} (`arrive'). 
\citet{Lindsey2011} and \citet{speed:2015} explore the preference for and distribution of manner and path verbs in Bulgarian in contrast with other Balkan and Slavic (Balkan and non-Balkan) languages and come to the conclusion that, as suggested for Modern Greek, Bulgarian does not conform to one of the two Talmian typological patterns of conflating motion. In her work \citet{Pantcheva-2007,Pantcheva2007} centres on prefixation involving directional prefixes in Bulgarian and how this process affects event structure and syntactic structure, as part of a cross-linguistic study on directional expressions \citep{Pantcheva2011}.

A small number of FrameNet-based studies dealing with Bulgarian motion verbs have also been published, usually focusing on a small selection of predicates and their description in FrameNet, possibly supported by corpus data. For instance, \citet{2010-Formal-Description-of-Som} offer an analysis of Bulgarian verbs of non-directed motion, while
%while \citep{nedelcheva:2018} explore corpus data excerpted for the verbs \textit{enter} and \textit{go into}. \
\citet{nestorova:2009} discusses several transitive verbs involving the relocation of masses of people (\textit{populate} verbs).

This chapter's contribution lies in delivering an analysis of a selection of a non-directed and directed motion verbs in Bulgarian as compared with their English counterparts implemented through the adoption of the descriptive devices developed within the Berkeley FrameNet project and applying them to Bulgarian. The proposed methodology provides a solid foundation for cross-linguistic study of the semantic and syntactic properties of verbs.

\section{The organisation of FrameNet}

\subsection{Semantic frames and frame elements}

FrameNet \citep{Baker1998,Baker2008} is a lexical resource which couches lexical and conceptual knowledge in %a framework originating in 
the theory of Frame Semantics \citep{Johnson2001,Fillmore2003:ch4,Ruppenhofer2016}. A semantic frame is a ``script-like structure of inferences, linked by linguistic convention to the meanings of linguistic units -- in our case, lexical items. Each frame identifies a set of frame elements (FEs) -- participants and props in the frame. A frame semantic description of a lexical item identifies the frames which underlie a given meaning and specifies the ways in which FEs, and constellations of FEs, are realised in structures headed by the word” \citep[9]{Johnson2001}. Each frame in FrameNet is represented by means of a definition that describes schematically the situation and the way in which at least the most essential FEs are involved in it. Each FE is also supplied with a definition that further clarifies its semantics and its interaction with other FEs. Frame elements have different status reflecting their role in the description of a given semantic frame: core, peripheral or extra-thematic \citep[19--20]{Ruppenhofer2016}. A core FE is ``one that instantiates a conceptually necessary component of a frame, while making the frame unique and different from other frames” \citep[23]{Ruppenhofer2016}. Peripheral FEs make reference to notions such as Time, Place, Manner, Means, Degree, etc. Extra-thematic FEs characterise an event against a backdrop of another state of affairs, either of an actual event or state of the same type (e.g. the FE \fename{Iteration}), or by evoking a larger frame within which the reported state of affairs is embedded. A frame in FrameNet is linked to the meanings of a set of linguistic items, called lexical units (LUs). Each LU is thus a pairing of a word and a meaning whose conceptual semantics is represented by the frame (so that the LU is said to evoke the relevant frame). Below, reference will be made mainly to core FEs as the ones that are most essential to the description of the different frames.

The observations presented below are based on the data in the Berkeley FrameNet requested in 2017. For the sake of consistency, in the course of this work the data have been checked against the online version of the resource.\footnote{The official Berkeley FrameNet has migrated to: \url{http://berkeleyfn.framenetbr.ufjf.br/}. The online searchable database is available for browsing at \url{https://framenet2.icsi.berkeley.edu/}.} 


\subsection{Frame-to-frame relations}

FrameNet frames are organised in a network by means of a number of hierarchical and non-hierarchical frame-to-frame relations \citep[81--84]{Ruppenhofer2016}. Four of them denote hierarchical relationships that bear relevance to the internal organisation of a particular semantic domain of the lexis and will be discussed below. \FrameRelation{Inheritance} is a relation between a more general (parent) frame and a more specific (child) frame where ``each semantic fact about the parent must correspond to an equally specific or more specific fact about the child'' \citep[81–82]{Ruppenhofer2016}, i.e. there should be a strict correspondence between entities, frame elements, frame relations and semantic characteristics in the parent and the child frame \citep{Petruck2015}. Examples of this relation in the context of the studied domain are represented by the frames \framename{Self\_motion}, \framename{Fluidic\_motion}, etc. (see Fig. \ref{ch4:fig:01}, p. \pageref{ch4:fig:01}), which share the main configuration of frame elements defined for the parent frame \framename{Motion}, but in addition provide a further specification of the \fename{Theme} as an entity moving under its own power and will, i.e. a \fename{Self\_\linebreak mover} (in \framename{Self\_motion}), or as a \fename{Fluid} (in \framename{Fluidic\_motion}).
\FrameRelation{Using}, also defined as weak \FrameRelation{Inheritance} \citep{Petruck2015}, is a relation between a parent frame and a child frame in which only some of the FEs in the parent have a corresponding entity in the child, and if such exist, they are more specific \citep{Petruck2012}. In the studied domain, an instance of such a relation exists between \framename{Motion} and its child \framename{Operate\_vehicle}. Like \framename{Motion}, the more specific frame describes the translational motion of a \fename{Theme} from a \fename{Source} to a \fename{Goal} along a \fename{Path}, but elaborates on it by introducing further frame elements: an \fename{Agent}, who controls the movement, and a \fename{Carrier}, which is the actual object carrying the \fename{Theme}. \FrameRelation{Perspective} is a relation where a more abstract situation viewed as neutral may be specified by means of perspectivised semantic frames that represent ``different possible points-of-view on the neutral frame'' \citep[82]{Ruppenhofer2016}. For instance, the frames \framename{Operate\_vehicle} and \framename{Ride\_vehicle} perspectivise different facets of the idea of moving by means of a vehicle described in \framename{Using\_\linebreak vehicle} according to the involvement of a person, who is being transported, as either the driver/operator or as a passenger. \FrameRelation{Subframe} captures the relationship between a complex frame referring to ``sequences of states and transitions, each of which can itself be separately described as a frame'' and the frames denoting these states or transitions \citep[83–84]{Ruppenhofer2016}. For example, the frames \framename{Arriving} and \framename{Departing} are defined as subframes of \framename{Traversing}, as they describe the initial and the final stage of the translational movement that results in a moving entity's change of location.

A comprehensive description of all the frame-to-frame relations with more examples is provided in \citet{Ruppenhofer2016}.

\section{English and Bulgarian data employed in the study}

\subsection{FrameNet and WordNet as a source for the inventory of motion verbs}

The inventory of English verbs and their semantic and syntactic description used in the study is directly derived from the description of the lexical units in the studied semantic frames in the Berkeley FrameNet, as well as the lattices summarising the valence patterns attested in the FrameNet corpus, including the particular syntactic realisation of the FEs in terms of their syntactic category and syntactic function. The corpus is also used as a source for the examples illustrating the realisation of the English verbs.\footnote{For brevity some of the examples throughout the paper will be adapted.}

%in FATE (FrameNet-Annotated Textual Entailment)\footnote{http://framenet.icsi.berkeley.edu/} \citep{Burchardt2008}. The dataset for English is supplemented with examples from SemCor in order to illustrate the usage of particular verb senses and verb literals.

%After analysing this information for English, we turn to observe to what extent it is applicable for Bulgarian. For the purposes of the study, we attempt to construct a similar resource for Bulgarian with manually selected and validated examples from BulSemCor and additionally, where the number of examples is not sufficient, from the Bulgarian National Corpus.
%Taking as a point of departure the semantic analysis for the English verbs, I verify the validity of the observations for Bulgarian, focusing on the parallels and differences, where relevant. 

The semantic frames are adopted from the Berkeley FrameNet without changes, but where relevant, comments regarding the set of frame elements are made. The Bulgarian verbs are studied independently but in comparison with their Eng-\newline lish counterparts, taking as a point of departure the relevant motion frames and the valence patterns described in the Berkeley FrameNet. This approach has been adopted to facilitate the description of the motion verbs in the Bulgarian FrameNet whose original concept was laid out in \citet{Koeva2008-framenet, Koeva2010-framenet} and further elaborated in \citet{svetla2021towards}, as well as in Chapter 1, this volume. The Bulgarian FrameNet is implemented within BulFrameNet \citep{koeva-doychev-2022-ontology}, a dedicated web-based system allowing the comprehensive description of the semantic and syntactic properties of verbs. The study of the valence patterns of the motion verbs and the syntactic expression of their semantic participants presented below was undertaken specifically as part of the work on the Bulgarian FrameNet. 

The set of Bulgarian motion verbs discussed in the chapter is extracted from the Bulgarian WordNet, a lexical-semantic net modelled on the Princeton WordNet (PWN). PWN \citep{Miller1995,Fellbaum1998} is a large lexical database for English that comprehensively represents conceptual and lexical knowledge in the form of a network whose nodes denote cognitive synonyms (synsets) connected through a number of conceptual-semantic and lexical relations such as hypernymy, meronymy, antonymy, etc. The synsets in the Bulgarian Wordnet  have been developed by translation and adaptation of the PWN counterparts, and the corresponding synsets in the two wordnets are related to each other through unique interlingual identifiers (which also provide links to the respective synsets in all other wordnets that support them). Thus, the lexical and conceptual knowledge is aligned cross-linguistically. In the course of its creation the Bulgarian WordNet has been expanded so as to cover all the synsets included in PWN (117,659 in total) by means of automatic translation followed by manual editing and enrichment (currently 85,954 synsets have been manually validated). The Bulgarian WordNet includes language-specific lexicalisations (synsets with no correspondence in PWN) as well as synsets describing closed-class words: prepositions, conjunctions, pronouns, particles, interjections; as a result it has amounted to 121,282 synsets altogether. It thus provides substantial coverage of the language's lexis, including verbs (forming a total of 14,103 synsets). In addition, BulNet has developed a number of language-specific characteristics, including notation of verb aspect. The current state of the Bulgarian WordNet is detailed in \citet{koeva2021-wordnet}.\footnote{The Bulgarian WordNet may be browsed at: \url{dcl.bas.bg/bulnet/}.} 

WordNet and FrameNet were aligned automatically using several previous mappings coupled with additional procedures for expansion and validation. In particular, the following were employed: (i) direct mappings provided within FrameNet \citep{Baker2009}, eXtendedWordFrameNet \citep{Laparra2010} and MapNet \citep{Tonelli2009}, supplemented with (ii) indirect mapping through VerbNet \citep{Palmer2009,Palmer2014}. This resulted in 4,306 unique WordNet synsets to FrameNet frame mappings, a coverage of 30.5\% of the verb synsets \citep[110]{LesevaStoyanova-CIT:2020}. A number of procedures inspired by ideas proposed in \citet{lopez-de-lacalle-etal-2014-predicate} and especially in \citet{Burchardt2005} were implemented towards the  improvement and extension of the mapping coverage. These procedures, described in \citet{Leseva2018} and further refined in \citet{StoyanovaLeseva2019,LesevaStoyanova2020}, are specifically based on exploring the structural properties of the two resources, such as: (i) the assumption that as verbs in a synset denote the same or very similar meaning, they are likely to evoke the same semantic frame; (ii) the hierarchical relational structure of the two resources based on the notion of inheritance from a more general to a more specific synset or frame. As a result, in general, more specific concepts should be associated with the frame of their hypernym(s) or with more specific frames elaborating on (and possibly inheriting from) this frame, although various divergences occur in practice. Part of the other relations among frames and among synsets were also cast as validation procedures. The main mapping mechanism involved: (i) manual assignment of semantic frames to root verb synset to ensure greater accuracy at the highest hierarchical level and to reduce error propagation down the tree; (ii) automatic assignment of the hypernym’s frame onto hyponyms which were not previously mapped; and (iii) verification and improvement of the assignments by applying the validation procedures. In this way, the coverage of the automatic mapping has been gradually increased to 94\% \citep[115–116]{LesevaStoyanova-CIT:2020}. Due to various peculiarities of the structure of WordNet or lack of appropriate frames in FrameNet (as part of the lexis has not yet been described by frames), the automatic assignment has been undergoing manual validation, so far covering almost 50\% of the mapping (over 6,000 synsets).  
 
The FrameNet-to-WordNet alignment together with the alignment between the Princeton WordNet and the Bulgarian WordNet has enabled the association of Bulgarian verbs with a FrameNet semantic description. This possibility is founded on the assumption that although the construal of the semantic description of situations across languages (as well as across resources) often differs, the major semantic aspects represent shared conceptual knowledge. Such an assumption underlies the development of both wordnets and framenets for other languages besides English, as well as the Global FrameNet initiative and Shared Annotation Task  (cf. \sectref{ch4:intro}). The genealogical and typological similarities between English and Bulgarian have also made it possible to base the syntactic description of the Bulgarian verbs of motion on the one provided for their English counterparts in the Berkeley FrameNet. Similar ideas have been pursued by other authors who have adopted a FrameNet-oriented approach to the semantic and syntactic analysis for languages other than English (cf. \sectref{motion-verbs}). The analysis below has been specifically informed by previous work on Bulgarian change \citep{StoyanovaLeseva:21a} and stative verbs \citep{LesevaSoyanova:2022}.

The English and the Bulgarian verbs included in the analysis are members of synsets that have been assigned one of several FrameNet frames belonging to the motion domain. In order to be selected, they had to meet the following requirements: (i) pertain to the general lexis; (ii) have a corresponding LU in FrameNet with a sufficient number of annotated sentences (20+). This means that synsets such as \{walk:1\} `use one's feet to advance; advance by steps' and \{run:34\} `move fast by using one's feet, with one foot off the ground at any given time', which have as correspondences the LUs \textit{walk.v} and \textit{run.v} in the \framename{Self\_motion} frame are included in the analysis, while ones such as \{lollop:1\} `walk clumsily and with a bounce' and \{hare:1\} `run quickly, like a hare' are not. These requirements have been adopted for the following reasons: general-lexis verbs are more likely to be represented in BulSemCor (see \sectref{annotated-data}), hence more Bulgarian examples would be available for them; the representation in the FrameNet corpus provides the pool of examples for English.


\subsection{Annotated examples}\label{annotated-data}

The statistics and analysis for English are based on the annotated sentences available for the respective verbs in the Berkeley FrameNet. 

The examples of the usage of the selected Bulgarian verbs are extracted from BulSemCor \citep{koeva-2006-bulsemcor,koeva-2011-bulsemcor} -- a 100,000-word corpus designed according to the overall methodology of the English SemCor \citep{miller-etal-1993-semantic,miller-etal-1994-using,landes1998}, further adapted by using criteria for ensuring an appropriate coverage of contemporary Bulgarian general lexis. As BulSemCor is manually annotated with wordnet senses, it provides disambiguated examples for the studied verbs. Where the number of examples is not sufficient, they have been supplemented with sentences from the Bulgarian National Corpus, a corpus of 1.2 billion words of running Bulgarian text distributed in 240,000 text samples spanning the second half of the 20th century and the beginning of the 21st century \citep{Koeva2012}. As the corpus is not sense-disambiguated, the examples have been selected manually so as to correspond to the studied senses. 

The Bulgarian examples extracted from the different corpora have been annotated so that the sentence components that syntactically realise the core frame elements related to motion are explicitly marked in a similar fashion to the annotation in the Berkeley FrameNet corpus. The selection covers 893 annotated clauses or sentences distributed as follows across the selected semantic frames: \framename{Motion} -- 149; \framename{Self\_motion} -- 262; \framename{Arriving} -- 182; \framename{Departing} -- 178; \framename{Traversing} -- 122.

%which represents the contemporary state of the language and is the largest corpus of Bulgarian, amounting to more than 1.2 billion words for Bulgarian (5.4 billion with its parallel corpora. 

\section{The domain of Motion}

\subsection{Organising semantic domains}

As suggested by \citet[16]{Johnson2001}, the lexicon pertaining to a semantic domain is hierarchically organised in a number of semantic frames of different abstraction and specialisation related through the frame-to-frame relations that capture semantic generalisations existing across frames. Thus, as pointed out in the work cited, for many semantic domains, there is one general frame that describes the common aspects of the more specific frames. It may be posited that at the conceptual level all (or most) frames in a domain share this basic structure consisting of a configuration of FEs that defines the distinctive meaning of the domain. The mechanisms that organise such a part of the lexis involve various changes in this prototypical structure that reflect the various ways in which specialisation within the domain occurs: (i) not all frames allow the overt expression of all FEs -- some of them may be blocked from overt expression, although they are conceptually necessary and implied in the meaning of the lexical units; (ii) more specific frames may exclude some FEs or demote them to non-core status; for instance, in the \fename{Goal}-profiling \framename{Arriving} and the \fename{Source}-profiling \framename{Departing} frames the FEs describing the remaining parts of the route are regarded as non-core; (iii) more specific frames may further narrow down the semantic properties of one or more of the FEs as compared with their counterparts in the more general frame (for instance, impose stricter selectional restrictions on the expressions realising the FEs): e.g. the moving entity is defined as the FE \fename{Fluid} in the \framename{Fluidic\_motion} frame, and as \fename{Mass\_theme} in the \framename{Mass\_motion} frame which both inherit from the \framename{Motion} frame (Fig. \ref{ch4:fig:01}, p. \pageref{ch4:fig:01}); (iv) more specific frames may include other FEs besides the ones describing the general frame, may change perspective, incorporate or profile a certain FE. An example of a semantic elaboration resulting in the introduction of a new FE is the specification of the vehicle which ``holds or conveys'' the traveller (the FE \fename{Mode\_of\_transportation}) in the \framename{Travel} frame (Fig. \ref{ch4:fig:01}).

The observations below are based on the theoretical and practical motivations described in \citet{Johnson2001}, \citet{Petruck2012}, \citet{Petruck2015}, \citet{Ruppenhofer2016} and the definitions, comments and frame-to-frame relations in FrameNet.

\subsection{General organisation of the domain of motion}

The lexis denoting movement is most broadly divided between translational and non-translational (or self-contained) motion with respect to some background or location. %\cite[35–26]{Talmy!!!}. 
Based on the definitions in FrameNet,\footnote{\url{http://framenet2.icsi.berkeley.edu/fnReports/data/frameIndex.xml?frame=Motion}} in the first case a moving entity typically starts at some location, moves through space along a trajectory and ends up in another location. Non-translational motion\footnote{\url{https://framenet2.icsi.berkeley.edu/fnReports/data/frameIndex.xml?frame=Moving_in_place}} involves the movement of an entity or parts of it with respect to some fixed location or landmark, without undergoing motion in space or without a significant alteration of configuration or shape. Translational motion is most broadly described by the non-lexicalised frame \framename{Motion\_scenario}, which is further perspectivised by several frames, two of which, \framename{Motion} and \framename{Traversing}, form the core of the translational motion domain. Non-translational movement is described by the \framename{Moving\textunderscore in\textunderscore place} frame and its causative counterpart \framename{Cause\_to\_move\_in\_place}, which are evoked by LUs such as \textit{rock, shake, twirl}, e.g. \textit{The earth shook} vs. \textit{He shook the remote control}. In what follows below, the focus will be on translational motion.

Another major division in the domain of motion is between (i) self-induced motion that a moving entity undergoes on its own -- under its own physical power, due to some internal cause, physical forces, features of the relief, etc., and (ii) caused motion that is brought about by an external participant that may be an animate, volitional \fename{Agent} or a non-animate \fename{Cause}. The prototypical semantic frame that organises the lexis of this type of translational motion is \framename{Cause\_motion}, which is indirectly integrated in the \framename{Motion\_scenario} through its causative relation to \framename{Motion} (i.e. \framename{Cause\_motion} Is\_causative\_of \framename{Motion}). The frames related to \framename{Cause\_motion} include \framename{Bringing} (e.g. \textit{bring, carry, transport}), \framename{Placing} (e.g. \textit{bottle, load, pocket}), \framename{Filling} (e.g. \textit{fill, flood}), \framename{Removing} (e.g. \textit{extract, remove}), \framename{Emptying} (e.g. \textit{empty, purge}), as well as some frames (e.g. \framename{Cause\_fluidic\_motion}) that have counterparts in the non-causative domain (\framename{Fluidic\_motion}). %In the remaining part of the chapter I will analyse frames from the \framename{Motion} domain.

%As defined in the Motion frame, specialisation of the vocabulary in this domain may also involve ``assumptions about the shape-properties, etc., of any of the places involved (insert, extract)”. Such assumptions are reflected in the subdomain of caused motion in frames such as Placing (bottle, enclose, put), Removing (clear, remove, strip).

As suggested in the description of the \framename{Motion} frame, a complex area in the vocabulary of motion is the depiction of the relation of \fename{Vehicles} to the moving entity and other participants. In the cases where the moving entity cannot be expressed, the LUs denoting the motion of vehicles are treated as evoking the \framename{Self\_motion} frame and the vehicles are annotated as \fename{Self\_movers}. When the \fename{Vehicle} is profiled as being operated by a \fename{Driver}, the relevant LUs evoke the frame \framename{Operate\_vehicle}; the \fename{Driver} may be construed very generally as the one controlling the vehicle, as attested by verbs such as \textit{bicycle, canoe, paddle, skate}, along with verbs involving special qualifications or skills such as \textit{drive, fly, sail, taxi}. The situation where the moving entities are passengers that are transported by means of the \fename{Vehicle} which is not under their control, is described by the \framename{Ride\_vehicle} frame (\textit{bus, hitchhike, ride, sail}).

Another type of elaboration in the motion domain described in the definition of the frame or reflected in the frame-to-frame relational structure refers to properties of the manner of motion, which basically stem from prominent features of the moving entity. One such feature is the requirement for the moving entity to be a living being whose body moves on its own, using its own energy, as in \framename{Self\_motion} (e.g. \textit{jog, limp, run, walk}), and semantic frames inheriting from it such as \framename{Cotheme} (\textit{accompany, lead, track}) and \framename{Travel} (\textit{journey, tour, voyage}). Further salient distinctions based on the types of entities involved in the motion and the specific manner of motion typical of them is reflected in the definition of several frames such as: \framename{Fluidic\_motion} (e.g. \textit{cascade, ooze, stream}), describing the motion of liquids; \framename{Mass\_motion} (e.g. \textit{crowd, swarm, throng}), which refers to the motion of a collective of individuals (a \fename{Mass\_theme}) as one entity; \framename{Motion\_noise} (e.g. \textit{buzz, roar, whir}), specified according to the type of noise the moving entity produces; \framename{Light\_movement} (e.g. \textit{gleam, shine, twinkle}), describing the emission and movement of light, etc. 

Another facet of motion has to do with the feature of directionality, which is lexically encoded in some LUs (e.g. \textit{descend, fall, rise} in the \framename{Motion\_directional} frame). Directed motion is also described in semantic frames that profile parts of the path along which an entity moves, such as its initial (\framename{Departing}) or final stage (\framename{Arriving}). 

In the remaining part of the chapter the analysis will be focused specifically on non-directed and directed motion verbs as represеnted by semantic frames such as \framename{Motion} and \framename{Self\_motion} on the one hand, and \framename{Traversing}, \framename{Arriving} and \framename{Departing} on the other, drawing parallels between the semantics and syntactic expression of the relevant frame elements.

The overall organisation of the domain of self-induced translational motion is partially represented in \figref{ch4:fig:01}.

\begin{figure}
\includegraphics[width=0.9\textwidth]{figures/Motion1.jpg}
\caption{The Motion hierarchy: green lines denote the relation of \FrameRelation{Inheritance}; red lines show the relation \FrameRelation{Perspectivises}; blue lines correspond to the \FrameRelation{Subframe of} relation; magenta lines denote the relation \FrameRelation{Using}.}\label{ch4:fig:01}
\end{figure} 


\subsection{\framename{Motion}}

\subsubsection{Semantic description of the \framename{Motion} frame}

%Below I will present in brief the prototypical frame of the semantic domain, i.e. \framename{Motion}, and the main criteria that further organise the verb lexis pertaining to the domain. Ideally, the principles of organisation are reflected in the relational structure of the relevant subsection of FrameNet (i.e. through frame-to-frame relations), although the relation may not always be direct. In outlining these criteria, I use the description proposed in the Motion frame and the frame-to-frame relations.

The \framename{Motion} frame describes the changing of spatial location of a \fename{Theme} understood in the classical sense of Gruber (\cite[27--31]{Gruber1965}, \cite[29]{Jackendoff1972}) as an entity that moves. More precisely, the LUs in this frame involve the translational motion of entities that are either not (capable of) moving under their own power or are underspecified for this feature. %in the lexical meaning of the relevant LUs: 
Therefore, many of the definitions of the motion frames for which this property is relevant note that the \fename{Theme} is frequently a living being moving on its own but need not be. Generalising over FrameNet examples such as the following, one can infer that the motion may be induced by various factors: (i) an outside force: [\textit{The black dust}]$_{\feinsub{Thm}}$ \textit{began \textbf{BLOWING OFF}} [\textit{the tailings lake}]$_{\feinsub{Src}}$; %(FN\footnote{The examples marked as (FN) are taken from the relevant frame in FrameNet.})
(ii) the \fename{Theme}’s own momentum:  \textit{It} \textit{fell on the floor and [\textit{\_}]$_{\feinsub{Thm}}$ \textit{\textbf{ROLLED}} [\textit{towards Uncle Mick’s feet}}]$_{\feinsub{Goal}}$; (iii) some internal process: [\textit{Tears}]$_{\feinsub{Thm}}$ \textit{\textbf{ROLLED}} [\textit{down my cheeks}]$_{\feinsub{Path}}$, etc., but it is represented with respect to the involvement of the \fename{Theme} in it, regardless of the cause that has brought it about. Volitional or self-directed motion is elaborated in some of the frames inheriting from \framename{Motion}, \framename{Self\_motion} and its descendants in particular. The remaining core FEs of the \framename{Motion} frame describe various elements or properties of the path\footnote{When used with a capital letter, \fename{Path} would mean the frame element; in small letters, path would be used in its accepted meaning in the literature, i.e. the medial part of the route traversed by a \textbf{Figure} (\cite[26]{Fillmore1971},  \cite[275]{Lakoff1987}, among  others). The term “route” will be used instead of “path” to refer to the line of movement that comprises all the three elements: \fename{Source}, \fename{Path} and \fename{Goal}.} that the moving entity moves along. 

\begin{description}[font=\normalfont]
\item[Definition of the \framename{Motion} frame:] Some entity, the \fename{Theme}, starts out in one place (\fename{Source}), and ends up in some other place (\fename{Goal}), having covered some space between the two (\fename{Path}). Alternatively, the \fename{Area} or \fename{Direction} in which the \fename{Theme} moves or the \fename{Distance} covered may be mentioned.\footnote{The frame definitions are taken from FrameNet: \url{https://framenet.icsi.berkeley.edu/framenet_search}.}

\item[Core FEs in the \framename{Motion} frame:] \fename{Theme}, \fename{Source}, \fename{Goal}, \fename{Path}, \fename{Area}, \fename{Direction}, \fename{Distance}.
\end{description}

The \fename{Theme}\footnote{The definitions of the FEs are taken from the description of the \framename{Motion} frame, with further elaboration on their semantic specification informed by the annotated examples studied in the paper.} is defined as ``the entity that changes location''. An important feature of this FE noted in the FrameNet description is that it need not be a \fename{Self\_mover}, that is, it need not be capable of moving on its own and by its own power or will.
%\footnote{The definitions of FEs are based on the definitions across related frames so as to provide a consistent and more comprehensive description.} is the entity that changes location either by moving on its own and/or under its own power or by being moved, carried, etc. by another entity or force. An important feature of this FE is that it may but need not be capable of moving on its own and by its own power or will (which distinguishes it from \fename{Self\_mover}). Its semantic specification includes \textbf{animate beings} and \textbf{physical objects}.
 Its semantic specification includes animate beings, physical objects, substances, etc.

The \fename{Source} is ``the location the \fename{Theme} occupies initially before its change of location''. It may refer to geological and other natural formations, geographical points, celestial bodies; physical objects, including man-made structures, such as buildings, constructions, facilities, other objects occupying space, etc.


%a location or an entity occupying space that serves as the starting point or landmark where the moving entity is at before it starts changing location. 
%Its semantic specification spans various \textbf{locations}, including geological and other natural formations, geographical points, celestial bodies; \textbf{physical objects}, including man-made structures, such as buildings, constructions, facilities, etc., but also any other object occupying space.

The \fename{Goal} is ``the location the \fename{Theme} ends up in'' as a result of the motion. It has the same semantic specification as the \fename{Source}.

The \fename{Path} refers to ``(a part of) the ground over which the \fename{Theme} travels or to a landmark by which the \fename{Theme} travels''. Its semantic specification encompasses locations, including geological and other natural formations, geographical points, celestial bodies; physical objects, including man-made structures and other objects occupying space that may be construed as having extent along which the motion takes place; extents of various media, such as water, air, etc. 


%is any description of a trajectory of motion which is neither a \fename{Source} nor a \fename{Goal} and more specifically refers to (a part of) the ground over which the moving entity travels or to a landmark by which it travels.


The \fename{Area} identifies the setting where ``the \fename{Theme}'s movement takes place without a specified \fename{Path}''. A notable consequence of the lack of a single linear trajectory is that the \fename{Area} cannot co-occur with \fename{Source}, \fename{Goal} and \fename{Path}, i.e. it is defined in an ‘excludes’ relation to each of them as well as to the FEs \fename{Distance} and \fename{Direction} which provide additional details referring to the translational motion in space. 
Like \fename{Path}, the semantic specification of \fename{Area} refers to locations, physical objects, other objects occupying space, various media, such as water, air, etc., which, however, may be construed as comprising some spatial expanse in, over or around which the motion takes place in an irregular fashion.

The \fename{Direction} %indicates the general spatial orientation %of the motion of the moving entity along a line from the deictic centre towards a (possibly implicit) reference point that is neither the \fename{Goal} of the posture change nor a landmark along the way of the moving part of the body; it characterises the \fename{Path} of motion according to some system of orientation; often \fename{Direction} is defined with reference to the canonical orientation of the moving entity, or the orientation imposed by an implicit observer. 
indicates ``motion along a line from the deitic center towards a reference point (which may be implicit) that is neither the \fename{Goal} of the posture change nor a landmark along the way of the moving part of the body. Often \fename{Direction} is defined with reference to the canonical orientation of the Protagonist, or the orientation imposed by an implicit observer''. %For some semantic frames it may be further specified, e.g. with \framename{Motion\_directional} it is the direction of motion of the \fename{Theme}, which is most often imposed by gravity). 
The semantic specification of this FE includes directions, such as compass points (north, east, south, west), body relative directions (left, right, back, front, backward, forward, up, down), coordinates, etc.

The \fename{Distance} encompasses expressions that characterise ``the extent of the motion'' covered by the \fename{Theme}. Its semantic specification includes distance denotations expressed either in various systems of measurement or as relative distances (farther, closer), etc.  

The basic configuration of the core FEs of the \framename{Motion} frame and the interaction among them determines the overall semantic specification of the prototypical notion of motion, which is subject to various modifications and elaborations in the more specific motion frames.

The syntactic expression of the semantic configuration of the \framename{Motion} frame will be discussed in terms of: (i) the (typical) syntactic projections of each core FE, in particular its syntactic (phrasal) category and grammatical relation; (ii) the most frequent valence patterns, i.e. the various frequent combinations in which the core FEs co-occur in the annotated FrameNet corpus. For English, both types of data are extracted from the summaries provided for each LU in FrameNet; the Bulgarian counterparts are analysed in comparison with the descriptions available for English and tested against the corpus of annotated examples created for Bulgarian (see \sectref{annotated-data} above).

\subsubsection{Verbs evoking the \framename{Motion} frame}

Тhe \framename{Motion} frame is evoked by a couple of basic verbs of inherently directed motion \citep{Levin1993}, such as \textit{come} and \textit{go}, as well as by verbs of non-directed motion. Within the second class, some predicates, such as \textit{move} and \textit{travel}, describe the general idea of moving through space, while others, for instance \textit{blow}, \textit{drift}, \textit{float}, \textit{circle}, \textit{roll}, denote various types of manner of motion; part of these verbs, e.g. \textit{meander}, \textit{spiral}, \textit{weave}, \textit{wind}, \textit{zigzag}, involve complex trajectories. 


\subsubsection{Syntactic realisation of the frame elements in the \framename{Motion} frame}

\tabref{tab:4:motion-synt} below illustrates the syntactic realisation of several verbs evoking the \framename{Motion} frame, chosen according to the following criteria: (i) having a sufficient number of attestations in the FrameNet corpus, thus allowing for more reliable observations; (ii) illustrating distinct syntactic patterns with respect to the expression of the FEs denoting the different parts or features of the route of movement.

The \fename{Theme} is typically projected in the subject position and the remaining core FEs are expressed primarily as prepositional (PP) or, more rarely, as adverbial phrases (marked as AVP). Some of the verbs also allow object NPs, especially as a realisation of \fename{Path}: [\textit{She}]$_{\feinsub{Thm}}$ \textit{\uppercase{\textbf{circles}}} [\textit{the taxi}]$_{\feinsub{Path}}$. In addition, the core FEs expressing elements or aspects of the route, may be conceptually present but left syntactically non-overt if they are known or retrievable from the previous text (definite null instantiations, DNIs) or implied from a broader context but without a referent in the previous text (indefinite null instantiations, INIs) or if the grammatical construction requires them to be left non-explicit (constructional null instantiations, CNIs), cf. \citep[28--30]{Ruppenhofer2016}. 

Several preliminary observations are relevant at this point. Usually, the route is conceived as a tripartite spatial extent consisting of an initial part, \fename{Source}, a medial part, \fename{Path}\footnote{Various names have been applied to this part; here I adopt the name of the relevant FrameNet frame element.}, and a final part, \fename{Goal} (\cite[162]{Jackendoff1983}, \cite[57]{Talmy1985}, \cite[275]{Lakoff1987}). As mentioned in \sectref{motion-verbs}, studies on the co-occurrence of directional phrases with motion verbs have shown that verbs tend to express the element of the route that is most prominent in their semantics. It has been convincingly demonstrated for many languages that \fename{Goal}-oriented verbs tend to co-occur with \fename{Goal} phrases (\cite[22--24]{Rakhilina2004}, \cite[255-257]{StefanowitschRohde2004}, \cite[174--178]{Taremaa2017}, among others), and \fename{Source}-oriented verbs co-occur with \fename{Source} phrases (\cite[22--23]{Rakhilina2004}, \cite[255-257]{StefanowitschRohde2004}, \cite[160--164]{Taremaa2017}), see also the analyses proposed in studies such as the ones by \citet{Cristobal2010,Kopecka2010}. %\citep{Cristobal2010,Kopecka2010}.  %(Example \ref{ex:01} a).

In addition, it has been posited that there is a marked cross-lingual asymmetry with respect to the expression of the \fename{Source} and the \fename{Goal}, known as the goal-over-source principle. This proposition suggests that \fename{Goals} are expressed more frequently, using more fine-grained linguistic devices than \fename{Sources} (\cite{Ikegami1987,WalchliZuniga2006,Verkerk2017}, among others). One of the explanations for this peculiarity offered in the literature is that \fename{Goals} bear higher information value with respect to the complete conceptualisation of motion \citep[249]{StefanowitschRohde2004}; for an extensive overview of the discussion, see \citet{Verkerk2017}. However, as noted above, the preference for one type of phrase over another depends on the semantics of the verb, specifically, whether the verb conceptualises the motion in terms of a route or not \citep{StefanowitschRohde2004}. 

Unlike \fename{Source}- and \fename{Goal}-phrases, \fename{Path}-phrases %characterise the motion through space without 
do not express directionality %are inherently non-directional 
(see also \cite[31]{Pantcheva2011}). Considering that many of the verbs in the \framename{Motion} frame describe non-directed motion, the \fename{Path} should be the most prominent phase of motion inherent in their semantics and hence will be favoured for syntactic expression, unless some other aspect of motion turns out to be more prominent. Respectively, we should expect that the inherently directed-motion verbs -- \textit{go} and \textit{come} -- favour \fename{Goal}-phrases. \tabref{tab:4:motion-synt} confirms these expectations, which are further corroborated by the co-occurrence patterns in \tabref{tab:4:motion-valence-framenet}: the ones involving \fename{Paths} are the most frequent among the top ranking patterns (Column 1) and have the greatest number of occurrences (Column 2) across the greatest number of verbs (Column 3).

Looking at individual verbs, a couple of trends may be noted with respect to the prevalence of expression of the route-related FEs (i.e. all the core FEs, excluding the \fename{Theme}) (\tabref{tab:4:motion-synt}).

\fename{Path}:~ Several~ verbs~ show~ marked~ preference~ for~ expressing syntactically \fename{Paths} over any other route-related FEs. These include \textit{move} as well as a number of manner of motion verbs: \textit{weave}, %and a number of other, less represented\footnote{By representation I mean the number of annotated examples in the FrameNet corpus.} verbs: 
\textit{circle}, \textit{glide}, \textit{meander}, \textit{wind}, \textit{zigzag}. %\textit{Move} specifies neither manner, nor directed motion. The latter fact accounts for the modest number of expressed \fename{Goals} and \fename{Sources}. By comparison, in half of the annotated sentences the verb co-occurs with a syntactically expressed \fename{Path}. Additionally, in a substantial number of cases (21) the \fename{Path} is considered to be understood in a general context, although not having a specific discourse referent (the so-called indefinite null instantiation) \citep[28-29]{Ruppenhofer2016}, and is thus conceptually present but syntactically non-overt. 
Among them \textit{glide} denotes qualitative features of the movement (effortlessness) and the contact with the surface along which the motion takes place. \textit{Weave}, \textit{wind}, \textit{meander} and \textit{zigzag}\footnote{There are other verbs, e.g. \textit{undulate} and \textit{spiral}, that possibly behave in a like manner, but the number of occurrences is too small to make a judgment.} describe complex vacillating or snake-like movement along a more or less linear route or general direction, while \textit{circle} refers to a circular trajectory. In all these cases the \fename{Path} -- including its form, landmarks, etc. --  is the default spatial dimension according to which the movement is characterised.
%with reference to its relation to the objects of the surrounding locale (e.g. \textit{across the valley, along the river}), the form or extent of the route, the interaction of the movement with the route (e.g. \textit{weave, zigzag}), etc.

The second most prominent aspect of motion with the verbs in the \framename{Motion} frame is the end-point of the route, the \fename{Goal}. It is usually less frequent than \fename{Path}, to the exception of the inherently directed motion verb \textit{go} (cf. also \citet[253--254]{StefanowitschRohde2004}, for which the prevalence of \fename{Goal}- over \fename{Path}-phrases is roughly 4:1. 

%\begin{table}
%\centering
{\small
\begin{longtable}{l ccccccccc}
\caption{\label{tab:4:motion-synt}Syntactic expression of the \framename{Motion} FEs in FrameNet}\\
\lsptoprule
  & NP.Ext & NP.Obj & PP & AVP & NI & Clause & Other & Total\\ \midrule \endfirsthead
\midrule
  & NP.Ext & NP.Obj & PP & AVP & NI & Clause & Other & Total\\ \midrule \endhead
\multicolumn{9}{l}{\textit{move} } \\*
\fename{Theme} & 67  &  &  &  &  &  &  & 67\\*
\fename{Area} &  &  & 2  &  &  &  &  & 2\\*
\fename{Source} &  &  & 5  & 2  &  &  &  & 7\\*
\fename{Path} &  & 3  & 30  & 5  & 21  &  &  & 59\\*
\fename{Goal} &  &  & 5  &  & 1  &  &  & 6\\*
\fename{Direction} &  &  &  & 1  &  &  &  & 1\\*
\fename{Distance} &  &  &  &  &  &  & 1 & 1\\\midrule
\multicolumn{9}{l}{\textit{go} } \\*
\fename{Theme} & 64  &  &  &  & 1  &  &  & 65\\*
\fename{Area} &  & 1  & 1  &  &  &  &  & 2\\*
\fename{Source} &  &  & 1  & 1  & 3  &  &  & 5\\*
\fename{Path} &  &  & 8  & 2  & 1  &  &  & 11\\*
\fename{Goal} & 1  & 2  & 30  & 8  & 6  & 1  &  & 48\\*
\fename{Direction} &  &  & 1  & 5  &  &  & 1 & 7\\*
\fename{Distance} &  &  & 1  & 1  & 1  &  &  & 3\\\midrule
\multicolumn{9}{l}{\textit{drift} } \\*
\fename{Theme} & 39  &  &  &  &  &  &  & 39\\*
\fename{Area} &  &  & 2  &  &  &  &  & 2\\*
\fename{Source} &  &  & 9  &  &  &  &  & 9\\*
\fename{Path} &  &  & 15  & 4  & 4  &  &  & 23\\*
\fename{Goal} &  &  & 8  & 1  &  &  & 1 & 10\\*
\fename{Distance} &  &  &  & 1  &  &  & 1 & 2\\
\midrule
%\multicolumn{9}{l}{\textit{glide} } \\
%\fename{Theme} & 12  &  &  &  &  &  & 1 & 13\\
%\fename{Path} &  &  & 7  & 2  & 1  &  &  & 10\\
%\fename{Source} &  &  & 1  & 2  &  &  &  & 3\\
%\midrule
%\multicolumn{9}{l}{\textit{blow} } \\
%\fename{Theme} & 21  &  &  &  &  &  & 7 & 28\\
%\fename{Area} &  & 1  & 1  & 1  &  &  &  & 3\\
%\fename{Path} &  &  & 6  & 2  & 3  &  &  & 11\\
%\fename{Source} &  &  & 7  & 1  &  &  &  & 8\\
%\fename{Goal} &  & 1  & 6  & 1  &  &  &  & 8\\
%\midrule
\multicolumn{9}{l}{\textit{float} } \\*
\fename{Theme} & 43  &  &  &  &  &  &  & 43\\*
\fename{Area} &  &  & 13  & 1  &  &  &  & 14\\*
\fename{Source} &  &  & 4  & 1  &  &  &  & 5\\*
\fename{Path} &  &  & 13  & 3  & 2  &  &  & 18\\*
\fename{Goal} &  &  & 8  &  &  &  &  & 8\\*
\fename{Distance} &  &  &  & 1  &  &  &  & 1\\\midrule
%\multicolumn{9}{l}{\textit{coast} } \\
%\fename{Theme} & 6  &  &  &  &  &  &  & 6\\
%\fename{Path} &  & 2  & 2  & 1  & 1  &  &  & 6\\
%\fename{Source} &  & 1  &  &  &  &  &  & 1\\
%\midrule
\multicolumn{9}{l}{\textit{roll} } \\*
\fename{Theme} & 31  &  &  &  &  &  &  & 31\\*
\fename{Area} &  &  & 1  &  &  &  &  & 1\\*
\fename{Source} &  &  & 1  & 1  &  &  &  & 2\\*
\fename{Path} &  & 3  & 17  & 2  &  &  &  & 22\\*
\fename{Goal} &  &  & 9  & 1  &  & 1  &  & 11\\\midrule
%\multicolumn{9}{l}{\textit{soar} } \\
%\fename{Theme} & 8  &  &  &  &  &  &  & 8\\
%\fename{Area} &  &  &  & 1  &  &  &  & 1\\
%\fename{Path} &  & 1  &  & 1  & 2  &  &  & 4\\
%\fename{Goal} &  &  & 3  &  &  &  &  & 3\\
%\midrule
%\multicolumn{9}{l}{\textit{fly} } \\
%\fename{Theme} & 11  &  &  &  &  &  &  & 11\\
%\fename{Area} &  &  & 1  &  & 1  &  &  & 2\\
%\fename{Distance} &  &  &  & 1  &  &  & 1 & 2\\
%\fename{Path} &  &  & 6  &  & 1  &  &  & 7\\
%\fename{Goal} &  &  & 2  &  &  &  &  & 2\\
%\midrule
\multicolumn{9}{l}{\textit{slide} } \\*
\fename{Theme} & 26  &  &  &  &  &  &  & 26\\*
\fename{Area} &  &  & 3  &  &  &  &  & 3\\*
\fename{Source} &  &  & 1  &  &  &  & 1 & 2\\*
\fename{Path} &  &  & 9  &  &  &  &  & 9\\*
\fename{Goal} &  &  & 10  &  &  &  &  & 10\\*
\fename{Direction} &  &  & 1  & 5  &  &  &  & 6\\\midrule
\multicolumn{9}{l}{\textit{swerve} } \\*
\fename{Theme} & 27  &  &  &  &  &  &  & 27\\*
\fename{Area} &  &  & 2  &  & 4  &  &  & 6\\*
\fename{Source} &  &  & 4  &  &  &  &  & 4\\*
\fename{Path} &  &  & 10  &  &  &  &  & 10\\*
\fename{Goal} &  &  & 2  &  &  &  &  & 2\\*
\fename{Direction} &  &  & 5  & 2  &  &  &  & 7\\\midrule
%\multicolumn{9}{l}{\textit{snake} } \\
%\fename{Theme} & 11  &  &  &  &  &  &  & 11\\
%\fename{Area} &  &  & 1  &  & 1  &  &  & 2\\
%\fename{Direction} &  &  & 2  & 1  &  &  &  & 3\\
%\fename{Path} &  &  & 2  &  &  &  &  & 2\\
%\fename{Source} &  &  & 5  & 1  &  &  &  & 6\\
%\fename{Goal} &  &  & 1  &  &  &  &  & 1\\
%\midrule
%\multicolumn{9}{l}{\textit{meander} } \\
%\fename{Theme} & 11  &  &  &  &  &  &  & 11\\
%\fename{Area} &  &  &  & 1  & 2  &  &  & 3\\
%\fename{Goal} &  &  & 3  &  &  & 1  &  & 4\\
%\fename{Path} &  &  & 7  &  &  &  &  & 7\\
%\midrule
%\multicolumn{9}{l}{\textit{undulate} } \\
%\fename{Theme} & 1  &  &  &  &  &  &  & 1\\
%\fename{Path} &  &  & 1  &  &  &  &  & 1\\
%\midrule
\multicolumn{9}{l}{\textit{weave} } \\*
\fename{Theme} & 27  &  &  &  &  &  &  & 27\\*
\fename{Area} &  &  & 3  &  &  &  &  & 3\\*
\fename{Source} &  &  & 1  &  &  &  &  & 1\\*
\fename{Path} &  & 1  & 20  &  & 1  &  &  & 22\\*
\fename{Goal} &  &  & 1  &  &  &  &  & 1\\*
\fename{Direction} &  &  & 4  &  &  &  &  & 4\\
%\multicolumn{9}{l}{\textit{wind} } \\
%\fename{Theme} & 8  &  &  &  &  &  &  & 8\\
%\fename{Direction} &  &  & 2  &  &  &  &  & 2\\
%\fename{Path} &  &  & 8  &  &  &  &  & 8\\
%\midrule
%\multicolumn{9}{l}{\textit{zigzag} } \\
%\fename{Theme} & 13  &  &  &  &  &  &  & 13\\
%\fename{Area} &  &  & 1  & 1  &  &  &  & 2\\
%\fename{Goal} &  &  &  & 1  &  &  &  & 1\\
%\fename{Path} &  &  & 6  &  & 4  &  &  & 10\\
%\fename{Source} &  &  & 2  &  &  &  &  & 2\\
%\midrule
%\multicolumn{9}{l}{\textit{circle} } \\
%%\fename{Theme} & 16  &  &  &  &  &  &  & 16\\
%\fename{Area} &  &  & 1  & 1  &  &  &  & 2\\
%\fename{Distance} &  &  &  & 2  &  &  &  & 2\\
%\fename{Path} &  & 8  & 1  &  & 6  &  &  & 15\\
%\fename{Direction} &  &  &  & 2  &  &  &  & 2\\
%\midrule
%\multicolumn{9}{l}{\textit{spiral} } \\
%\midrule
%\multicolumn{9}{l}{\textit{swing} } \\
%\fename{Theme} & 3  &  &  &  &  &  &  & 3\\
%\fename{Area} &  &  &  & 1  &  &  &  & 1\\
%\fename{Direction} &  &  & 1  &  &  &  &  & 1\\
%\fename{Path} &  &  & 1  &  &  &  &  & 1\\
%\fename{Goal} &  &  & 1  &  &  &  &  & 1\\
%\fename{Source} &  &  & 1  &  &  &  &  & 1\\
%\midrule
%\multicolumn{9}{l}{\textit{travel} } \\
%\fename{Theme} & 2  &  &  &  &  &  &  & 2\\
%\fename{Goal} &  &  & 1  &  &  &  &  & 1\\
%\fename{Source} &  &  & 1  &  &  &  &  & 1\\
%\midrule
%\multicolumn{9}{l}{\textit{come} } \\
%\fename{Theme} & 5  &  &  &  &  &  &  & 5\\
%\fename{Direction} &  &  & 2  &  &  &  &  & 2\\
%\fename{Distance} &  &  &  & 1  &  &  &  & 1\\
%\fename{Goal} &  &  & 1  &  &  &  &  & 1\\
%\fename{Source} &  &  & 1  &  &  &  &  & 1\\
\lspbottomrule
\end{longtable}
  }
%\end{table}

%The predominant valence patterns in Bulgarian are similar, although it seems from the data that the \textbf{Addressee} co-occurs readily with \textbf{Messages} (last example above).


The other verbs tend to express either the \fename{Path} or another aspect of the route, usually with prevalence of the former.

\fename{Path} or \fename{Goal}: The preference of either \fename{Path}- or \fename{Goal}-phrases (but not any other type of phrase) is typical of the verb \textit{roll}: the \fename{Path}-expressions outnumber the \fename{Goal}-expressions by two to one. 

\fename{Path}, \fename{Goal} or \fename{Source}: \textit{Blow} and \textit{drift} %and \textit{float} 
exhibit preference for either \fename{Path}- or \fename{Goal}-phrases but also tend to express the \fename{Source} more often than most of the remaining verbs evoking the \framename{Motion} frame. However, with \textit{drift}, \fename{Path} is the predominant type of expression, with \fename{Goals} and \fename{Sources} being half as few, while with \textit{blow} the three parts of the route are represented equally. 

The three other motion FEs -- \fename{Direction}, \fename{Distance} and \fename{Area} -- are not usually discussed separately in the literature. By virtue of its definition, \fename{Area} occurs in competition with the elements of the route. The rationale is that it describes motion encompassing an expanse that is not construed in terms of a discreet trajectory. \fename{Areas} are not equally represented across manner of motion verbs as a whole, but are typical for some of them. \fename{Direction} and \fename{Distance} are represented by just a few examples across different verbs and have much poorer inventories. They still do need to be considered as separate FEs, as (i) some verbs incorporate them (e.g. \textit{descend}, \textit{rise} incorporate \fename{Direction}), and (ii) they may be independently expressed syntactically (Example \ref{ex01}). 


%\begin{exe} \label{ex:01} 
%\ex  \label{bg:diff-vb}
%    \settowidth \jamwidth{(bg)} 
%\gll {[\ ]}$_{\text{THEME-CNI}}$ \textit{вървете} [\textit{в} \textit{комуникационната} %\textit{зала}]$_{\feinsub{Goal}}$. \\

\begin{exe}
\ex  \label{ex01}
  %  \settowidth \jamwidth{(en)} 
[\textit{The storm}]$_{\feinsub{Thm}}$ \textit{\textbf{was MOVING}} [\textit{north}]$_{\feinsub{Dir}}$ [\textit{along the coast}]$_{\feinsub{Path}}$.
 %\jambox{(en)}
\end{exe}

\fename{Path}, \fename{Direction} or \fename{Area}: This pattern is represented by the verb \textit{swerve}, which describes motion involving a complex route characterised by an abrupt change in direction from an imaginary straight line or course. Respectively, it tends to co-occur with \fename{Path}-expressions as well as with ones denoting the newly assumed \fename{Direction}. As this kind of motion may encompass a broader spatial region, the FE \fename{Area} is also more frequently expressed than with other verbs. 

\fename{Path}, \fename{Area} or \fename{Goal}: The verb \textit{float} denotes a manner of motion which is brought about by the movement of a fluid. As this type of motion tends to encompass an expanse of the medium where it takes place, \fename{Area}-expressions are much more typical than with the rest of the verbs evoking the frame -- almost on a par with \fename{Paths} and more than \fename{Goals}.

\fename{Path}, \fename{Goal} or \fename{Direction}: This pattern is exemplified by the verb \textit{slide}. While it is expected to co-occur with \fename{Path} (like \textit{glide}), the verb also shows a tendency to express directionality either by means of \fename{Goal}-phrases, which in the data are represented on par with \fename{Paths}, or by means of the FE \fename{Direction}. 


\subsubsection{FrameNet valence patterns}

\tabref{tab:4:motion-valence-framenet} sums up the most frequent valence patterns represented among verbs evoking the \framename{Motion} frame, understood as combinations of FEs which co-occur syntactically, including null instantiations.\footnote{Non-core FEs are not considered in the analysis.}

The patterns corroborate the prominence of the \fename{Path} FE expressed predominantly as a prepositional phrase, followed by indefinite null instantiations (INIs), noun phrases and adverbial phrases. The second most frequent pattern involves the \fename{Goal}, followed by  \fename{Area}- and \fename{Source}-phrases. It is also notable that the simultaneous expression of two route-related FEs is much rarer.

\begin{table}
   \begin{tabularx}{\textwidth}{lrQ} 
   \lsptoprule
   Pattern & \# & Verbs \\ \midrule\relax
{[NP.Ext]}$_{\feinsub{Thm}}$ {[PP]}$_{\feinsub{Path}}$  & 134 & \textit{move, meander, go, roll, snake, float, undulate, zigzag, coast, fly, slide, swerve, glide, blow, circle, weave, drift, wind} \\ \relax
{[NP.Ext]}$_{\feinsub{Thm}}$ {[PP]}$_{\feinsub{Goal}}$  & 60 & \textit{move, fly, slide, meander, go, roll, soar, swerve, come, blow, float, drift}\\ 
{[NP.Ext]}$_{\feinsub{Thm}}$ {[\_]}$_{\feinsub{Path-INI}}$  & 40 & \textit{coast, move, go, soar, glide, blow, float, circle, weave, drift, zigzag}\\ 
{[NP.Ext]}$_{\feinsub{Thm}}$ {[PP]}$_{\feinsub{Area}}$  & 29 & \textit{move, fly, slide, go, roll, snake, swerve, blow, float, weave, drift}\\ 
{[NP.Ext]}$_{\feinsub{Thm}}$ {[PP]}$_{\feinsub{Src}}$  & 28 & \textit{move, slide, snake, swerve, come, glide, blow, float, drift, zigzag}\\ 
{[NP.Ext]}$_{\feinsub{Thm}}$ {[NP.Obj]}$_{\feinsub{Path}}$  & 14 & \textit{coast, move, roll, soar, circle, weave}\\ 
{[NP.Ext]}$_{\feinsub{Thm}}$ {[PP]}$_{\feinsub{Dir}}$  & 11 & \textit{swerve, come, weave}\\ 
{[NP.Ext]}$_{\feinsub{Thm}}$ {[AVP]}$_{\feinsub{Path}}$  & 11 & \textit{move, soar, glide, blow, float, drift}\\ 
{[NP.Ext]}$_{\feinsub{Thm}}$ {[AVP]}$_{\feinsub{Goal}}$  & 9 & \textit{go, roll, blow}\\ 
\lspbottomrule
\end{tabularx}
    \caption{FrameNet valence patterns of \framename{Motion} verbs}
    \label{tab:4:motion-valence-framenet}
\end{table} 

\subsubsection{Syntactic realisation of \framename{Motion} verbs in Bulgarian}

The list of Bulgarian verbs evoking the \framename{Motion} frame includes the Bulgarian counterparts of the verbs considered above. In particular, it features (i) a couple of verbs of directed motion which belong to the central part of the motion lexis -- \textit{идвам} `come’, \textit{отивам} `go’ and their perfective aspect counterparts\footnote{For brevity only the imperfective members of aspectual pairs will be listed in the text. The annotated examples include verbs of both aspects, where such exist.} (on the deictic aspects of these verbs, cf.  \cite{Nitsolova1984,Trifonova1982,Stanisheva1985}, among others), and (ii) a number of non-directed motion verbs, predominantly ones describing various manners of motion, such as \textit{вия се} `wind’, `weave’, \textit{духам} `blow’, \textit{летя} `fly’, \textit{лъкатуша} `meander’, \textit{нося се} `drift’, `float’, \textit{плувам}, \textit{плавам} `float’, \textit{кръжа}, \textit{обикалям} `circle’, \textit{търкалям се} `roll’, etc., as well as ones denoting the general idea of moving through space, such as \textit{движа се} `move’, `locomote’ and \textit{пътувам} `travel’.

A selection of corpus examples has been collected for several of them (verbs having correspondences among the English predicates represented in \tabref{tab:4:motion-synt}), and annotated with the core FEs: \textit{вия се} `wind’, `weave’, \textit{движа се} `move’, \textit{нося се} `drift’, `float’, \textit{отивам} `go’, \textit{търкалям се} `roll’. Although on a smaller scale, the results, shown in \tabref{tab:4:motion-synt-bg}, are consistent with the observations on the FrameNet corpus. In particular, \textit{отивам} shows a very strong preference for \fename{Goal}-phrases similarly to \textit{go} (Example \ref{ex:02}), while the rest of the verbs (Examples \ref{ex:03}--\ref{ex:06}) favour \fename{Paths}, with different proportions of other FEs, in particular \fename{Areas} for \textit{нося се} `float’, `drift’ and \fename{Goals} for \textit{търкалям се} `roll’.

\begin{exe}
\ex \label{ex:02} 
  %  \settowidth \jamwidth{(bg)} 
\gll \emph{[Те]}$_{\feinsub{Thm}}$ \textit{\textbf{ОТИВАТ}} \textit{право} [\textit{в} \textit{печатницата}]$_{\feinsub{Goal}}$. \\
They go-PRS.3PL straight to printer's-DEF. \\
\glt `They are going straight to the printer's.' 
\ex \label{ex:03} 
\gll [\textit{Кучетата}]$_{\feinsub{Thm}}$ {\textit{\textbf{СЕ}} \textit{\textbf{ДВИЖАТ}}} [\textit{по} \textit{мекия} \textit{сняг}]$_{\feinsub{Path}}$. \\
Dogs-DEF move-PRS.3PL across soft-DEF snow. \\
\glt `The dogs are moving across the soft snow.' 
\ex \label{ex:04} 
\gll [\textit{Топката}]$_{\feinsub{Thm}}$ {\textit{\textbf{СЕ}} \textit{\textbf{ТЪРКАЛЯ}}} [\textit{по} \textit{тревата}]$_{\feinsub{Path}}$. \\
Ball-DEF roll-PRS.3PL across grass-DEF. \\
\glt `The ball is rolling across the grass.' 
\ex \label{ex:05} 
\gll [\textit{Колата}]$_{\feinsub{Thm}}$ {\textit{\textbf{СЕ}} \textit{\textbf{ВИЕШЕ}}} [\textit{по} \textit{завоите}]$_{\feinsub{Path}}$. \\
Car-DEF wind-PST.3SG along turns-DEF. \\
\glt `The car was winding along the turns of the road.'
%\gll [\textit{Колата}]$_{\feinsub{Thm}}$ {\textit{се} \textit{виеше}} [\textit{по} \textit{завоите}]$_{\feinsub{Path}}$ [\textit{до} \textit{тях}]$_{\feinsub{Goal}}$. \\
%[Shadow-DEF]$_{\feinsub{Thm}}$ slid [across snow-DEF]$_{\feinsub{Path}}$ [to them]$_{\feinsub{Goal}}$. \\
\ex \label{ex:06} 
\gll [\textit{Туфа водорасли}]$_{\feinsub{Thm}}$ {\textit{\textbf{СЕ}} \textit{\textbf{НОСИ}}} [\textit{във} \textit{водата}]$_{\feinsub{Area}}$. \\
Clump-INDF seaweed float-PRS.3SG on water-DEF. \\
\glt `A clump of seaweed is floating on the water.' 
%\ex \label{ex:07} 
%\gll [\textit{От} \textit{Центъра}]$_{\feinsub{Src}}$ \textit{духаха} [\textit{облаци}]$_{\feinsub{Thm}}$. \\
%[From Centre-DEF]$_{\feinsub{Src}}$ blew [clouds]$_{\feinsub{Thm}}$. \\
%\glt `Clouds blew from the Centre.' 
%\ex \label{ex:08} 
%\gll [\textit{Градът}]$_{\feinsub{Thm}}$ \textit{плава} [\textit{по} \textit{езерото}]$_{\feinsub{Area}}$. \\ 
%[City-DEF]$_{\feinsub{Thm}}$ floats [on lake-DEF]$_{\feinsub{Area}}$. \\
%\glt `The city floats on the lake.' 
% \jambox{(bg)}
\end{exe}

{\small
\begin{longtable}{l ccccccccc}   
\caption{\label{tab:4:motion-synt-bg}Syntactic expression of the \framename{Motion} FEs in Bulgarian}\\
 \lsptoprule
  & NP.Ext & NP.Obj & PP & AVP & NI & Clause & Other & Total\\ \midrule \endfirsthead
  \midrule
  & NP.Ext & NP.Obj & PP & AVP & NI & Clause & Other & Total\\ \midrule \endhead
\multicolumn{9}{l}{\textit{вия се} `wind’, `weave’ }\\*
\fename{Theme} & 26 &  &  &  &  &  &  & 26\\*
\fename{Area} &  &  & 5 &  &  &  &  & 5\\*
\fename{Source} &  &  & 4 &  &  &  &  & 4\\*
\fename{Path} &  &  & 10 &  &  &  &  & 10\\*
\fename{Goal} &  &  & 3 &  &  &  &  & 3\\*
\fename{Direction} &  &  &  & 1 &  &  &  & 1\\
%\fename{Manner} &  &  & 1 & 1 &  &  &  & 2\\ 
 \midrule
\multicolumn{9}{l}{\textit{нося се} `float’, `drift’}\\*
\fename{Theme} & 32 &  &  &  &  &  &  & 32\\*
\fename{Area} &  &  & 9 &  &  &  &  & 9\\*
\fename{Source} &  &  & 5 & 1 &  &  &  & 6\\*
\fename{Path} &  &  & 8 & 1 &  &  &  & 9\\*
\fename{Goal} &  &  & 6 &  &  &  &  & 6\\*
\fename{Direction} &  &  & 2 & 1 &  &  &  & 3\\
 \midrule
\multicolumn{9}{l}{\textit{движа се} `move’ }\\*
\fename{Theme} & 31 &  &  &  &  &  &  & 31\\*
\fename{Area} &  &  & 1 &  &  &  &  & 1\\*
\fename{Source} &  &  & 1 &  &  &  &  & 1\\*
\fename{Path} &  &  & 18 & 1 &  &  &  & 19\\*
\fename{Goal} &  &  & 2 &  &  &  &  & 2\\*
\fename{Direction} &  &  & 3 &  &  &  &  & 3\\
%\fename{Manner} &  &  & 2 & 3 &  &  &  & 5\\*
\midrule
\multicolumn{9}{l}{\textit{търкалям се} `roll’}\\*
\fename{Theme} & 30 &  &  &  &  &  &  & 30\\*
\fename{Source} &  &  & 2 &  &  &  &  & 2\\*
\fename{Path} &  &  & 19 &  &  &  &  & 19\\*
\fename{Goal} &  &  & 7 &  &  &  &  & 7\\*
\fename{Direction} &  &  &  & 2 &  &  &  & 2\\
%\fename{Manner} &  &  &  & 2 &  &  &  & 2\\ 
 \midrule
%\multicolumn{9}{l}{\textit{търкулна се} }\\*
%\fename{Path} &  &  & 2 &  &  &  &  & 2\\*
%\fename{Goal} &  &  & 1 &  &  &  &  & 1\\*
%\fename{Theme} & 4 &  &  &  &  &  &  & 4\\ 
% \midrule
%\multicolumn{9}{l}{\textit{въртя се} }\\*
%\fename{Path} &  &  & 1 &  &  &  &  & 1\\*
%\fename{Theme} & 1 &  &  &  &  &  &  & 1\\ 
% \midrule
\multicolumn{9}{l}{\textit{отивам\slash отида} `go’ }\\*
\fename{Theme} & 28 &  &  &  &  &  &  & 28\\*
\fename{Source} &  &  & 1 &  &  &  &  & 1\\*
\fename{Path} &  &  & 2 &  &  &  &  & 2\\*
\fename{Goal} &  &  & 15 & 4 & 1 &  &  & 20\\*
\fename{Direction} &  &  &  & 3 &  &  &  & 3\\*
%\fename{Manner} &  &  &  & 1 &  &  &  & 1\\*
\fename{Distance} &  &  &  & 2 &  &  &  & 2\\*
 \lspbottomrule
\end{longtable}
}

\subsubsection{Valence patterns in the Bulgarian dataset}

The valence patterns in the Bulgarian dataset, represented in \tabref{tab:4:motion-valence-bg}, show similar results to the ones in the FrameNet corpus: in particular, a prevalence of patterns exhibiting PP \fename{Paths}, followed by a more modest representation of \fename{Goals} and \fename{Areas}. Among the several top valence patterns, combinations of \fename{Sources} and \fename{Goals} are also found.

\begin{table}
   \begin{tabularx}{\textwidth}{lrQ}
  \lsptoprule
   Pattern & \# & Verbs \\ \midrule
{[NP.Ext]}$_{\feinsub{Thm}}$ {[PP]}$_{\feinsub{Path}}$ & 55 & \textit{вия се, въртя се, движа се, нося се, търкалям се, отивам\slash отида}\\ 
{[NP.Ext]}$_{\feinsub{Thm}}$ {[PP]}$_{\feinsub{Goal}}$ & 25 & \textit{вия се, движа се, нося се, търкалям се, отивам\slash отида}\\ 
{[NP.Ext]}$_{\feinsub{Thm}}$ {[PP]}$_{\feinsub{Area}}$ & 15 & \textit{вия се, движа се, нося се}\\ 
{[NP.Ext]}$_{\feinsub{Thm}}$ {[\_]}$_{\feinsub{Path-INI}}$& 10 & \textit{вия се, движа се, нося се, отивам\slash отида, търкалям се}\\ 
{[NP.Ext]}$_{\feinsub{Thm}}$ {[PP]}$_{\feinsub{Src}}$ & 6 & \textit{вия се, нося се}\\ 
{[NP.Ext]}$_{\feinsub{Thm}}$ {[AVP]}$_{\feinsub{Dir}}$ & 5 & \textit{нося се, търкалям се, отивам\slash отида}\\ 
{[NP.Ext]}$_{\feinsub{Thm}}$ {[PP]}$_{\feinsub{Goal}}$ {[PP]}$_{\feinsub{Src}}$ & 5 & \textit{движа се, нося се, търкалям се, отивам\slash отида}\\ 
{[NP.Ext]}$_{\feinsub{Thm}}$ {[PP]}$_{\feinsub{Dir}}$ & 4 & \textit{движа се, нося се}\\ 
{[NP.Ext]}$_{\feinsub{Thm}}$ {[AVP]}$_{\feinsub{Goal}}$ & 4 & \textit{отивам\slash отида}\\ 
%{[NP.Ext]}$_{\feinsub{Thm}}$ {[AVP]}$_{\text{MANNER}}$ & 4 & \textit{движа се, търкалям се}\\ 
%{[NP.Ext]}$_{\feinsub{Thm}}$ {[PP]}$_{\text{MANNER}}$ & 3 & \textit{вия се, движа се}\\ 
%{[NP.Ext]}$_{\feinsub{Thm}}$ {[AVP]}$_{\feinsub{Distance}}$ & 2 & \textit{отивам\slash отида}\\ 
%{[NP.Ext]}$_{\feinsub{Thm}}$ {[AVP]}$_{\text{MANNER}}$ {[PP]}$_{\feinsub{Path}}$ & 2 & \textit{търкалям се, отивам\slash отида}\\ 
%{[NP.Ext]}$_{\feinsub{Thm}}$ {[AVP]}$_{\feinsub{Path}}$ & 2 & \textit{движа се, нося се}\\ 
%{[NP.Ext]}$_{\feinsub{Thm}}$ {[AVP]}$_{\feinsub{Src}}$ & 1 & \textit{нося се}\\ 
%{[NP.Ext]}$_{\feinsub{Thm}}$ {[PP]}$_{\feinsub{Dir}}$ {[PP]}$_{\feinsub{Goal}}$ {[PP]}$_{\feinsub{Src}}$ & 1 & \textit{нося се}\\ 
%{[NP.Ext]}$_{\feinsub{Thm}}$ {[AVP]}$_{\feinsub{Dir}}$ {[PP]}$_{\feinsub{Path}}$ & 1 & \textit{търкалям се}\\ 
%{[NP.Ext]}$_{\feinsub{Thm}}$ {[AVP]}$_{\text{MANNER}}$ {[PP]}$_{\feinsub{Goal}}$ {[PP]}$_{\feinsub{Src}}$ & 1 & \textit{вия се}\\ 
%{[NP.Ext]}$_{\feinsub{Thm}}$ {[\_]}$_{\feinsub{Goal-DNI}}$ & 1 & \textit{отивам\slash отида}\\ 
%{[NP.Ext]}$_{\feinsub{Thm}}$ {[AVP]}$_{\feinsub{Dir}}$ {[PP]}$_{\feinsub{Goal}}$ & 1 & \textit{вия се}\\ 
\lspbottomrule
\end{tabularx}
    \caption{FrameNet valence patterns of \framename{Motion} verbs in Bulgarian}
    \label{tab:4:motion-valence-bg}
\end{table}

%\section{Frames in the Motion domain}

\subsection{\framename{Self\_motion}}

\framename{Self\_motion} is an elaboration of the \framename{Motion} frame (and related to it by means of an \textbf{Inheritance} relation) which involves a similar configuration of core FEs with some further restrictions. 

\subsubsection{Semantic description of the \framename{Self\_Motion} frame}

Frame definition: The \fename{Self\_mover}, a living being, moves under its own direction along a \fename{Path}. Alternatively or in addition to \fename{Path}, an \fename{Area}, \fename{Direction}, \fename{Source}, or \fename{Goal} for the movement may be mentioned.

The most important distinction and the one that primarily motivates the differentiation between \framename{Motion} and \framename{Self\_motion} is the capability of the \fename{Self\_mover} to change location by exercising their own will and power by the coordinated movement of their bodies,\footnote{\url{https://framenet2.icsi.berkeley.edu/fnReports/data/frameIndex.xml?frame=Self_motion}} which is not a necessity with the \framename{Motion} \fename{Theme}. By metaphorical extension, \fename{Self\_movers} may be self-directed entities such as vehicles. %Self\_motion inherits both the frames Motion and Intentionally\_act, which also points to the agent-like quality of its main participant. 
The remaining core FEs in this frame are the ones defining the elements and aspects of the route of movement.\footnote{\fename{Distance} is not defined as a core FE, but will be treated on a par with its equivalent in the mother frame.} %(\fename{Source}, \fename{Goal}, \fename{Path}), \fename{Direction} and \fename{Distance}

Core~ FEs~ in~ the~ \framename{Self\_motion}~ frame:~ \fename{Self\_mover},~ \fename{Source},~ \fename{Goal}, \fename{Path}, \fename{Area}, \fename{Direction}, \fename{Distance}.
\fename{Self\_mover} is the entity (living being or a vehicle) that changes location under its own power and direction. Its semantic specification includes \textbf{animate beings} and \textbf{vehicles}. The remaining core FEs have the same semantic specification as their counterparts in the \framename{Motion} frame from which they are inherited. 
%: \fename{Source}, \fename{Goal}, \fename{Path}, \fename{Area}, \fename{Direction}, \fename{Distance}.

%The various manners of motion specified by the verbs in this frame characterise  

%\subsubsubsection*{\textbf{Syntactic properties of the \framename{Self\_motion} frame}}

\subsubsection{Verbs evoking the \framename{Self\_motion} frame}

Unlike its parent frame, \framename{Self\_motion} prototypically describes individuals capable of applying their own will and bodies to perform the motion. The verbs thus encode various aspects of motion impossible for inanimate beings. These involve modes of motion: (i) characteristic of different organisms, e.g. \textit{fly}, \textit{swim}, \textit{crawl}, \textit{slither}, \textit{walk}, etc.; (ii) requiring different configuration of the body: \textit{slouch}, \textit{shoulder}; (iii) (lack of ) purposefulness: \textit{roam}, \textit{saunter}, \textit{wander}; (iv) intent: \textit{prowl}, \textit{hike}, \textit{hitchhike}; (v) different kinds of steps, speed, weight or force applied: \textit{mince}, \textit{scurry}, \textit{shuffle}, \textit{plod}, \textit{trample}, \textit{run}, \textit{jog}, \textit{hop}, etc. 

\subsubsection{Syntactic realisation of the frame elements in the \framename{Self\_motion} frame}

The expression of the core FEs according to syntactic categories and syntactic function is similar to those of the corresponding FEs in the \framename{Motion} frame. The \fename{Self\_mover} is realised as the external argument; the remaining core FEs are typically realised as prepositional or adverbial phrases. %\fename{Distance} may also be expressed by measurement NPs. 

%The syntactic realisations of the various locative aspects of motion shows that they are not equal with respect to the verbs’ preference for their expression. 

\tabref{tab:4:self-motion-synt} illustrates the syntactic expression of the core FEs for several English verbs with the highest number of attestations in the FrameNet corpus. The verbs evoking these semantic frames further extend the observations made for the \framename{Motion} frame with respect to the tendency for the various verbs to co-occur with motion expressions. Overall, the \fename{Path} is the prevalent FE to be expressed, followed by \fename{Goals}, \fename{Areas} and \fename{Sources} in descending order (see the valence patterns in \tabref{tab:4:self-motion-valence-framenet}).

\begin{description}[font=\normalfont]
\item[\fename{Path}:] Several verbs exhibit a strong preference for \fename{Paths} over any other core FE: \textit{amble}, \textit{drive}, \textit{make}, \textit{plod}.
\item[\fename{Path or Goal}:] Verbs that show preference to co-occur with either of these FEs can be further distinguished into two patterns.

\item[The first one is \fename{Path} > \fename{Goal}:] In this case, the examples with \fename{Path} show prevalence, amounting to around or even more than half of the examples, and \fename{Goals} usually account for a quarter to a third, rarely more, see \textit{hop} in \tabref{tab:4:self-motion-synt}. This pattern is further illustrated by \textit{hurry}, \textit{jog}, \textit{limp}, \textit{lumber}, \textit{lunge}, \textit{lurch}, \textit{proceed}, \textit{skip}, \textit{stagger}, \textit{stroll}, \textit{stumble}, \textit{swagger}, \textit{totter}, \textit{trot}, \textit{trumble}, \textit{trek}.

\item[\fename{Path} = \fename{Goal}:] With the second pattern, there is no marked preference for one FE over the other, as exemplified by \textit{walk} (\tabref{tab:4:self-motion-synt}). Other verbs which pattern in a similar way are: \textit{barge}, %\textit{bustle},
\textit{clamber}, \textit{dash}, \textit{head}, \textit{hasten}, \textit{pad}, \textit{romp}, \textit{sidle}, \textit{toddle}, \textit{wade}.
       
\item[\fename{Path}, \fename{Goal} or \fename{Source}:] This pattern is distinguished from the second subgroup of the previous one empirically on the basis of the greater ratio of \fename{Sources} against the overall number of examples for each of the verbs. The verbs in this group tend to co-occur with expressions denoting any of the three parts of the route more consistently than the remaining verbs evoking the \framename{Self\_motion} fra\-me. As already shown in the \framename{Motion} frame, the frequency of each of these FEs is not equal across verbs. In this group one finds that \fename{Paths} account for half to up to two-thirds of the examples, \fename{Goals} -- for a quarter to a third of the examples, \fename{Sources} -- usually for a fifth to a quarter of the examples, as illustrated by \textit{crawl} (\tabref{tab:4:self-motion-synt}), \textit{creep}, \textit{dart}, \textit{march}, \textit{saunter}, \textit{scamper}, \textit{scramble}, \textit{shuffle}, \textit{spring}, \textit{sprint}, \textit{stride}, \textit{trudge}. Another variation is represented by the verbs \textit{lope}, \textit{leap}, \textit{jump}, where \fename{Paths} account for half or more of the examples, and \fename{Goals} and \fename{Sources} are on a par, about one third of the instances. %\textit{stomp} - on a par all of them
%\textit{scuttle}, \textit{dart}

\item[\fename{Paths} = \fename{Goals} or \fename{Source}:] This pattern shows no marked difference between \fename{Paths} and \fename{Goals} with a weaker preference for \fename{Sources}: \textit{climb} (\tabref{tab:4:self-motion-synt}), \textit{rush}, \textit{scuttle}.

\item[\fename{Goal}:] A couple of verbs, such as \textit{file} and \textit{pounce} show marked preference for \fename{Goal}-expressions over all other motion-related FEs. 

\item[\fename{Goal} or \fename{Path}:] These verbs tend to co-occur with both \fename{Goals} and \fename{Paths} with a prevalence of the former (about a half of the examples) to the latter (around a third of the examples): \textit{steal}, \textit{run} (\tabref{tab:4:self-motion-synt}).
							
\item[\fename{Goal}, \fename{Path} or \fename{Source}:] This combination is exemplified by verbs such as \textit{troop}, \textit{sneak} (\tabref{tab:4:self-motion-synt}), \textit{stalk}. The \fename{Goals} amount to half or more of the instances, while the \fename{Paths} and \fename{Sources} are fewer: around one-third of the examples for \fename{Path} and a quarter for \fename{Source} with \textit{troop}, and equally distributed between the two FEs for \textit{sneak} and \textit{stalk}. %
\end{description}

The FE \fename{Area} usually alternates with expressions denoting one or another element or aspect of the route of a moving entity and as a whole accounts for far fewer cases than \fename{Paths} and \fename{Goals} in the frame. For some verbs, however, it is either the preferred motion expression or is much more frequent than with most verbs. This characteristic is typical of verbs that describe motion that encompasses or spreads over a larger region or expanse.

\begin{description}[font=\normalfont]
\item[\fename{Area}:] The verbs \textit{traipse} and \textit{skulk} show a much more marked preference for \fename{Areas} than for other motion-related FEs: half of the instances for \textit{traipse}, two-thirds for \textit{skulk}.

\item[\fename{Area} or \fename{Path}:] Other verbs tend to co-occur with either \fename{Areas} or \fename{Paths} with a prevalence of the former (half or more of the examples) to the latter (around one-third of the examples): \textit{prance}, \textit{prowl}, \textit{roam} (\tabref{tab:4:self-motion-synt}).
%area - 1/2 or more %path - around a third

\item[\fename{Path} or \fename{Area}:] The opposite is observed with \textit{strut} and \textit{flit} where \fename{Paths} are preferred (between half and two-thirds of the examples) to \fename{Areas} (a quarter of the examples).
											
\item[\fename{Path}, \fename{Area} or \fename{Goal}:] This pattern shows prevalence of \fename{Paths} (with half or more of the instances), a substantial number (a quarter to one-third) of \fename{Areas} and a smaller number (one-sixth to one-fifth of the examples) of \fename{Goals}: \textit{dance}, \textit{pace}, \textit{swim} (\tabref{tab:4:self-motion-synt}), \textit{tread}, \textit{tramp}. In the case of \textit{fly} the number of \fename{Areas} and \fename{Goals} is equal.

%\textit{bustle} % path, goal 1/3; area 1/5
            
\item[\fename{Path}, \fename{Goal}, \fename{Area} or \fename{Source}:] This pattern shows prevalence of \fename{Paths} (ar\-ound half of the examples), with various distributions (between one-fifth and one-third) of the other three FEs: \textit{scurry}, \textit{slither}, \textit{waddle}, \textit{wander} (\tabref{tab:4:self-motion-synt}).
\end{description}

A couple of verbs, such as \textit{flounce} and \textit{storm}, show preference to \fename{Sources} over other motion-related FEs. 

%\fename{Direction}

   
%stagger, skip					
            
%\begin{table}
%\centering
{\footnotesize
\begin{longtable}{l ccccccccc}
 \caption{\label{tab:4:self-motion-synt}Syntactic expression of the \framename{Self\_motion} FEs in FrameNet}
    \\
    \lsptoprule
& NP.Ext & NP.Obj & PP & AVP & NI & Clause & Other & Total\\ \midrule\endfirsthead
\midrule
& NP.Ext & NP.Obj & PP & AVP & NI & Clause & Other & Total\\ \midrule\endhead
\multicolumn{9}{l}{\textit{climb} } \\*
\fename{Self\_mover} & 115  &  &  &  &  &  &  & 115\\*
\fename{Area} &  & 1  & 2  &  & 2  &  &  & 5\\*
\fename{Source} &  &  & 21  &  &  &  &  & 21\\*
\fename{Path} &  & 1  & 46  & 3  & 4  &  &  & 54\\*
\fename{Goal} &  &  & 59  & 1  &  &  &  & 60\\
\midrule
\multicolumn{9}{l}{\textit{crawl} } \\*
\fename{Self\_mover} & 140  &  &  &  &  &  &  & 140\\*
\fename{Area} &  &  & 18  &  & 4  &  &  & 22\\*
\fename{Source} &  &  & 23  & 6  &  &  &  & 29\\*
\fename{Path} &  &  & 58  & 7  & 9  &  & 1 & 75\\*
\fename{Goal} &  &  & 31  & 4  &  & 1  &  & 36\\
\midrule
%\multicolumn{9}{l}{\textit{dance} } \\		
%\fename{Self\_mover} & 80  &  &  &  &  &  &  & 80\\		
%\fename{Area} &  &  & 23  & 3  & 1  &  & 1 & 28\\		
%\fename{Source} &  &  & 8  & 4  &  &  &  & 12\\		
%\fename{Path} &  &  & 46  &  &  &  &  & 46\\		
%\fename{Goal} &  &  & 15  &  &  &  &  & 15\\		
%\midrule
%\multicolumn{9}{l}{\textit{dart} } \\		
%\fename{Self\_mover} & 72  &  &  &  &  &  &  & 72\\		
%\fename{Area} &  &  & 12  &  & 1  &  &  & 13\\		
%\fename{Goal} &  &  & 25  & 2  &  &  &  & 27\\		
%\fename{Source} &  &  & 20  & 4  &  &  &  & 24\\		
%\fename{Path} &  &  & 26  & 6  &  &  &  & 32\\		
%\midrule
%\multicolumn{9}{l}{\textit{dash} } \\		
%\fename{Self\_mover} & 62  &  &  &  &  &  &  & 62\\		
%\fename{Area} &  &  & 8  &  &  &  &  & 8\\		
%\fename{Path} &  &  & 29  & 2  &  &  &  & 31\\		
%\fename{Goal} &  &  & 25  & 4  &  &  &  & 29\\		
%\fename{Source} &  &  & 12  & 1  &  &  &  & 13\\		
%\midrule
%\multicolumn{9}{l}{\textit{hobble} } \\		
%\fename{Self\_mover} & 73  &  &  &  &  &  &  & 73\\		
%\fename{Area} &  &  & 15  &  &  &  & 1 & 16\\		
%\fename{Goal} &  &  & 26  & 3  &  &  & 1 & 30\\		
%\fename{Path} &  &  & 21  &  & 4  &  &  & 25\\		
%\fename{Source} &  &  & 8  & 3  &  &  &  & 11\\		
%\midrule
\multicolumn{9}{l}{\textit{hop} } \\*
\fename{Self\_mover} & 103  &  &  &  &  &  &  & 103\\*
\fename{Area} &  &  & 14  &  & 1  &  &  & 15\\*
\fename{Source} &  &  & 9  & 2  &  &  &  & 11\\*
\fename{Path} &  &  & 50  & 8  & 2  &  & 2 & 62\\*
\fename{Goal} &  &  & 28  & 4  &  &  &  & 32\\
\midrule
\multicolumn{9}{l}{\textit{hurry} } \\		
\fename{Self\_mover} & 74  &  &  &  &  &  &  & 74\\		
\fename{Source} &  &  & 10  & 2  &  &  &  & 12\\
\fename{Path} &  &  & 41  & 7  & 2  &  &  & 50\\		
\fename{Goal} &  &  & 28  & 8  &  &  &  & 36\\		
\midrule
%\multicolumn{9}{l}{\textit{leap} } \\		
%\fename{Self\_mover} & 85  &  &  &  &  & 1  &  & 86\\		
%\fename{Area} &  &  & 8  &  & 2  &  &  & 10\\		
%\fename{Source} &  &  & 22  &  &  &  &  & 22\\		
%\fename{Path} &  &  & 26  & 4  & 6  &  & 7 & 43\\		
%\fename{Goal} &  &  & 21  & 1  &  &  & 2 & 24\\		
%\midrule
%\multicolumn{9}{l}{\textit{limp} } \\		
%\fename{Self\_mover} & 64  &  &  &  &  &  &  & 64\\		
%\fename{Area} &  &  & 7  &  & 1  &  &  & 8\\		
%\fename{Goal} &  &  & 16  & 4  &  &  & 1 & 21\\		
%\fename{Path} &  &  & 22  & 2  & 7  &  &  & 31\\		
%\fename{Source} &  &  & 6  & 2  &  &  &  & 8\\		
%\midrule
%\multicolumn{9}{l}{\textit{march} } \\		
%\fename{Self\_mover} & 96  &  & 1  &  &  &  &  & 97\\		
%\fename{Area} &  &  & 5  &  & 1  &  &  & 6\\		
%\fename{Goal} &  &  & 27  & 1  &  &  & 2 & 30\\		
%\fename{Path} & 1  &  & 47  & 9  & 9  &  & 2 & 68\\		
%\fename{Source} &  &  & 17  & 3  &  &  &  & 20\\		
%\midrule
%\multicolumn{9}{l}{\textit{plod} } \\		
%\fename{Self\_mover} & 79  &  &  &  &  & 3  &  & 82\\		
%\fename{Area} &  &  & 6  &  &  &  &  & 6\\		
%\fename{Path} &  &  & 45  & 6  & 16  &  & 1 & 68\\		
%\fename{Source} &  &  & 3  & 1  &  &  &  & 4\\		
%\fename{Goal} &  &  & 9  & 3  &  &  &  & 12\\		
%\midrule
\multicolumn{9}{l}{\textit{roam} } \\*
\fename{Self\_mover} & 66  &  &  &  &  &  &  & 66\\*
\fename{Area} &  & 13 & 26  & 1  & 5  &  & & 45\\*
\fename{Source} &  &  & 1  &  &  &  &  & 1\\*
\fename{Path} &  &  & 13  & 2  & 3  &  &  & 18\\*
\fename{Goal} &  &  & 2  & 2  &  &  &  & 4\\
\midrule
\multicolumn{9}{l}{\textit{run} } \\*
\fename{Self\_mover} & 64  &  &  &  &  &  &  & 64\\*
\fename{Area} &  &  & 1  & 3  & 2  &  &  & 6\\*
\fename{Source} &  &  & 3  & 1  & 2  &  &  & 6\\*
\fename{Path} &  &  & 16  &  & 3  &  &  & 19\\*
\fename{Goal} &  &  & 16  & 3  & 9  &  &  & 28\\*
\fename{Direction} &  &  & 4  & 2  & 2  &  &  & 8\\
\midrule
%\multicolumn{9}{l}{\textit{rush} } \\		
%\fename{Self\_mover} & 146  &  &  &  &  &  &  & 146\\		
%\fename{Area} &  &  & 9  &  &  &  &  & 9\\		
%\fename{Path} &  &  & 60  & 10  & 3  &  &  & 73\\		
%\fename{Goal} &  &  & 54  & 14  &  &  &  & 68\\		
%\fename{Source} &  &  & 24  & 4  &  &  &  & 28\\		
%\fename{Direction} &  &  & 1  & 1  &  &  &  & 2\\		
%\midrule
%\multicolumn{9}{l}{\textit{scamper} } \\		
%\fename{Self\_mover} & 72  &  &  &  &  &  &  & 72\\		
%\fename{Area} &  &  & 10  &  &  &  &  & 10\\		
%\fename{Path} &  &  & 27  & 2  & 6  &  & 2 & 37\\		
%\fename{Goal} &  &  & 23  & 2  &  &  &  & 25\\		
%\fename{Source} &  &  & 11  & 4  &  &  &  & 15\\		
%\midrule
%\multicolumn{9}{l}{\textit{scramble} } \\		
%\fename{Self\_mover} & 86  &  &  &  & 1  &  &  & 87\\		
%\fename{Area} &  &  & 2  & 1  &  &  &  & 3\\		
%\fename{Path} & 1  & 3  & 33  &  & 16  &  &  & 53\\		
%\fename{Source} &  &  & 15  & 1  &  &  &  & 16\\		
%\fename{Goal} &  & 1  & 21  & 4  &  &  &  & 26\\		
%\midrule
%\multicolumn{9}{l}{\textit{scurry} } \\		
%\fename{Self\_mover} & 90  &  &  &  &  &  &  & 90\\		
%\fename{Area} &  &  & 19  &  &  &  &  & 19\\		
%\fename{Path} &  &  & 34  & 2  & 8  &  &  & 44\\		
%\fename{Goal} &  &  & 22  & 1  &  &  &  & 23\\		
%\fename{Source} &  &  & 10  & 6  &  &  &  & 16\\		
%\midrule
%\multicolumn{9}{l}{\textit{scuttle} } \\		
%\fename{Self\_mover} & 72  &  &  &  &  &  &  & 72\\		
%\fename{Area} &  &  & 7  &  &  &  &  & 7\\		
%\fename{Path} &  &  & 23  & 2  & 6  &  &  & 31\\		
%\fename{Source} &  &  & 18  & 3  &  &  &  & 21\\		
%\fename{Goal} &  &  & 25  & 2  &  &  &  & 27\\		
%\midrule
%\multicolumn{9}{l}{\textit{shuffle} } \\		
%\fename{Self\_mover} & 91  &  &  &  &  &  &  & 91\\		
%\fename{Area} &  &  & 9  &  &  &  &  & 9\\		
%\fename{Path} &  &  & 32  & 6  & 7  &  &  & 45\\		
%\fename{Goal} &  &  & 28  & 2  &  &  &  & 30\\		
%\fename{Source} &  &  & 15  & 4  &  &  &  & 19\\		
%\midrule
%\multicolumn{9}{l}{\textit{skip} } \\		
%\fename{Self\_mover} & 82  &  &  &  &  &  &  & 82\\		
%\fename{Area} &  &  & 8  &  &  &  &  & 8\\		
%\fename{Path} &  &  & 53  & 1  & 2  &  & 1 & 57\\		
%\fename{Goal} &  &  & 19  & 3  &  &  &  & 22\\		
%\fename{Source} &  &  & 7  & 6  &  &  &  & 13\\		
%\midrule
\multicolumn{9}{l}{\textit{sneak} } \\*
\fename{Self\_mover} & 68  &  &  &  &  &  &  & 68\\*
\fename{Area} &  &  & 2  &  &  &  &  & 2\\*
\fename{Source} &  &  & 17  & 4  &  &  &  & 21\\*
\fename{Path} &  &  & 20  &  &  &  &  & 20\\*
\fename{Goal} &  & 1  & 37  & 6  &  &  &  & 44\\
\midrule
%\multicolumn{9}{l}{\textit{sprint} } \\		
%\fename{Self\_mover} & 75  &  &  &  &  &  &  & 75\\		
%\fename{Path} &  &  & 45  & 2  & 6  & 1  &  & 54\\		
%\fename{Source} &  &  & 15  & 2  &  &  &  & 17\\		
%\fename{Goal} &  & 1  & 14  & 2  &  &  &  & 17\\		
%\midrule
%\multicolumn{9}{l}{\textit{stagger} } \\		
%\fename{Self\_mover} & 135  &  &  &  &  &  &  & 135\\		
%\fename{Area} &  &  & 12  & 4  &  &  &  & 16\\		
%\fename{Path} &  &  & 29  & 14  & 29  &  &  & 72\\		
%\fename{Goal} &  &  & 44  & 4  &  & 1  &  & 49\\		
%\fename{Source} &  &  & 20  & 2  &  &  &  & 22\\		
%\midrule
%\multicolumn{9}{l}{\textit{step} } \\		
%\fename{Self\_mover} & 153  &  &  &  &  &  &  & 153\\		
%\fename{Area} &  &  & 4  &  &  &  &  & 4\\		
%\fename{Path} &  & 1  & 54  & 12  & 1  & 2  &  & 70\\		
%\fename{Direction} &  &  &  & 1  &  &  &  & 1\\		
%\fename{Source} &  &  & 38  & 4  &  &  &  & 42\\		
%\fename{Goal} &  & 2  & 74  & 8  & 1  &  &  & 85\\		
%\midrule
%\multicolumn{9}{l}{\textit{stride} } \\		
%\fename{Self\_mover} & 113  &  &  &  &  &  &  & 113\\		
%\fename{Area} &  &  & 7  &  &  &  &  & 7\\		
%\fename{Path} &  &  & 66  & 9  & 8  &  &  & 83\\		
%\fename{Source} &  &  & 20  & 6  &  &  &  & 26\\		
%\fename{Goal} &  & 1  & 29  & 2  &  &  &  & 32\\		
%\midrule
%\multicolumn{9}{l}{\textit{stroll} } \\		
%\fename{Self\_mover} & 86  &  &  &  &  &  &  & 86\\		
%\fename{Area} &  &  & 12  &  &  &  &  & 12\\		
%\fename{Path} &  & 2  & 48  & 3  & 1  &  &  & 54\\		
%\fename{Goal} &  & 1  & 18  & 3  &  &  &  & 22\\		
%\fename{Source} &  &  & 7  & 1  &  &  &  & 8\\		
%\midrule
%\multicolumn{9}{l}{\textit{stumble} } \\		
%\fename{Self\_mover} & 69  &  & 1  &  &  &  &  & 70\\		
%\fename{Area} &  &  & 3  & 1  &  &  &  & 4\\		
%\fename{Goal} &  & 1  & 19  & 1  &  &  &  & 21\\		
%\fename{Source} &  &  & 8  & 2  &  &  &  & 10\\		
%\fename{Path} &  & 2  & 35  & 2  & 10  &  &  & 49\\		
%\midrule
\multicolumn{9}{l}{\textit{swim} } \\*
\fename{Self\_mover} & 259  &  &  &  & 1  &  &  & 260\\*
\fename{Area} & 1  & 2  & 54  & 3  & 1  &  &  & 61\\*
\fename{Source} &  &  & 19  & 14  &  &  &  & 33\\*
\fename{Path} &  & 5  & 95  & 25  & 42  &  &  & 167\\*
\fename{Goal} &  & 1  & 45  & 3  & 1  & 1  &  & 51\\*
\fename{Direction} &  &  &  & 1  &  &  &  & 1\\
\midrule
\multicolumn{9}{l}{\textit{walk} } \\*
\fename{Self\_mover} & 102  &  &  &  &  &  &  & 102\\*
\fename{Area} &  &  & 9  & 2  & 4  &  & 1 & 16\\*
\fename{Source} &  &  & 17  &  & 1  &  &  & 18\\*
\fename{Path} &  & 2  & 36  &  & 6  &  & 1 & 45\\*
\fename{Goal} &  &  & 29  & 2  & 1  & 3  & 7 & 42\\*
\fename{Direction} &  &  & 3  & 3  & 1  &  &  & 7\\
\midrule
%\multicolumn{9}{l}{\textit{tramp} } \\		
%\fename{Self\_mover} & 79  &  &  &  &  &  &  & 79\\		
%\fename{Area} &  & 1  & 15  &  & 1  &  & 6 & 23\\		
%\fename{Path} &  &  & 43  & 2  & 4  &  &  & 49\\		
%\fename{Goal} &  &  & 13  & 2  &  & 1  &  & 16\\		
%\fename{Source} &  &  & 5  & 3  &  &  &  & 8\\		
%\midrule
\multicolumn{9}{l}{\textit{wander} } \\*
\fename{Self\_mover} & 81  &  &  &  &  &  &  & 81\\*
\fename{Area} &  &  & 17  &  & 3  &  &  & 20\\*
\fename{Source} &  &  & 12  & 5  &  &  &  & 17\\*
\fename{Path} &  &  & 33  & 4  & 4  &  &  & 41\\*
\fename{Goal} &  &  & 27  & 2  &  &  &  & 29\\
\lspbottomrule
%\multicolumn{9}{l}{\textit{head} } \\		
%\fename{Self\_mover} & 60  &  &  &  &  &  &  & 60\\		
%\fename{Goal} &  & 1  & 32  & 2  &  &  &  & 35\\		
%\fename{Path} &  & 6  & 26  & 7  &  &  &  & 39\\		
%\fename{Direction} &  &  & 1  & 1  &  &  &  & 2\\		
%\fename{Source} &  &  & 4  &  &  &  &  & 4\\		
%\midrule
\end{longtable}
 }
%\end{table}


\subsubsection{FrameNet valence patterns}
\begin{sloppypar}
The valence patterns exhibited in the \framename{Motion} frame are confirmed on a larger scale by \framename{Self\_motion}, in particular the prevalence of \fename{Path}-expressions over \fename{Goals}, \fename{Areas} and \fename{Sources} in descending order. It is worth noting that the number of the second most frequent pattern as compared with the most frequent one is higher than for \framename{Motion} verbs (66\% and 45\%, respectively) i.e. \fename{Goal} expressions are found more frequently as compared to \fename{Path} expressions with \framename{Self\_motion} verbs. In addition, the most frequent patterns involving two motion-related FEs are \fename{Goal} + \fename{Path}  and \fename{Goal} + \fename{Source} representing about 19\% and 11\% of the number of the most frequent pattern; this ratio is much greater than for \framename{Motion}, where the pattern \fename{Goal} + \fename{Path} amounts to 6\% of the most frequent one. An interesting hypothesis to test on this amount of data would be whether this observation ties with animacy and/or agentivity.
\end{sloppypar}

\begin{table}
\small
\begin{tabularx}{\textwidth}{lrQ} 
\lsptoprule
 Pattern & \# & Verbs \\ \midrule
{[NP.Ext]}$_{\feinsub{SMov}}$ {[PP]}$_{\feinsub{Path}}$  & 1576 & \textit{stumble, mince, lurch, frolic, stride, climb, tramp, scurry, trip, stalk, rip, burrow, strut, roam, dance, prowl, jump%, straggle, tiptoe, proceed, amble%,pace, taxi, canter, hike, venture, promenade, wander, drive, flit, gambol, sidle, trudge, hasten, hop, skim, skip, stagger, dart, sashay, waltz, file, waddle, leap, caper, storm, slink, totter, sleepwalk, march, fly, swing, slip, shuffle, slalom, hobble, sprint, lope, run, lunge, stroll, bustle, slosh, swagger, crawl, scramble, spring, pad, tread, trek, edge, romp, trundle, sneak, scuttle, vault, barge, creep, dash, swim, traipse, wriggle, jog, prance, clamber, slog, sail, steal, meander, scamper, scoot, toddle, troop, parade, saunter, head, limp, trot, skulk, stomp, make, lumber, slither, plod, bound, clomp, rush, step, hurry, wade, flounce, walk
}\\

{[NP.Ext]}$_{\feinsub{SMov}}$ {[PP]}$_{\feinsub{Goal}}$  & 1035 & \textit{stumble, mince, lurch, stride, climb, tramp, scurry, trip, stalk, rip, burrow, strut, roam, press%, dance, prowl, jump, straggle, tiptoe, proceed, amble, pace, taxi, canter, hike, venture, wander, drive, flit, sidle, trudge, hasten, hop, skip, pounce, stagger, dart, waltz, file, waddle, leap, storm, slink, totter, sleepwalk, march, fly, swing, shuffle, hobble, sprint, lope, run, lunge, stroll, bustle, swagger, crawl, scramble, spring, pad, tread, trek, edge, romp, trundle, sneak, scuttle, vault, barge, creep, mosey, dash, swim, traipse, wriggle, repair, jog, clamber, sail, steal, meander, scamper, scoot, toddle, troop, saunter, head, limp, trot, skulk, stomp, lumber, slither, plod, bound, rush, step, hurry, wade, flounce, walk
}\\

{[NP.Ext]}$_{\feinsub{SMov}}$ {[PP]}$_{\feinsub{Area}}$  & 599 & \textit{stumble, hobble, mince, lurch, lope, frolic, stroll, bustle, stride, swagger, crawl, scramble, climb, spring, tramp%, pad%, tread, trek, scurry, trip, romp, stalk, rip, trundle, strut, roam, sneak, dance, prowl, scuttle, jump, tiptoe, barge, amble, pace, creep, canter, hike, mosey, wander, dash, swim, traipse, flit, gambol, bop, jog, prance, clamber, slog, trudge, gallivant, steal, scamper, hop, toddle, skip, parade, stagger, saunter, dart, limp, trot, waddle, leap, skulk, stomp, lumber, slither, plod, bound, slink, clomp, totter, rush, sleepwalk, march, fly, swing, step, shuffle, wade, flounce, walk
}
\\

{[NP.Ext]}$_{\feinsub{SMov}}$ {[PP]}$_{\feinsub{Src}}$  & 415 & \textit{stumble, hobble, mince, lurch, sprint, lope, run, lunge, stroll, bustle, stride, slosh, swagger, crawl, scramble%, climb%, spring, tramp, pad, scurry, trip, romp, stalk, trundle, strut, roam, sneak, dance, scuttle, vault, jump, straggle, tiptoe, barge, proceed, creep, venture, mosey, wander, dash, swim, traipse, flit, sidle, prance, clamber, trudge, sail, steal, hasten, scamper, scoot, hop, toddle, skip, troop, pounce, stagger, saunter, dart, limp, waltz, file, trot, waddle, leap, skulk, stomp, storm, lumber, slither, plod, bound, slink, totter, rush, march, fly, swing, step, hurry, shuffle, wade, flounce, walk
}\\

{[NP.Ext]}$_{\feinsub{SMov}}$   {[\_]}$_{\feinsub{Path-INI}}$  & 375 & \textit{stumble, hobble, lurch, sprint, run, lunge, stroll, stride, swagger, crawl, scramble, climb, spring, tramp, pad%, tread, trek, scurry, romp, trundle, strut, roam, prowl, scuttle, straggle, tiptoe, proceed, amble, pace, taxi, canter, venture, promenade, wander, drive, swim, traipse, gambol, jog, prance, clamber, slog, trudge, sail, hasten, scamper, scoot, hop, toddle, skip, troop, stagger, saunter, limp, trot, waddle, leap, skulk, lumber, slither, plod, slink, totter, rush, sleepwalk, march, fly, step, hurry, shuffle, wade, walk
}\\

{[NP.Ext]}$_{\feinsub{SMov}}$ {[PP]}$_{\feinsub{Goal}}$ {[PP]}$_{\feinsub{Path}}$  & 297 & \textit{stumble, hobble, mince, lurch, sprint, lope, stroll, stride, swagger, crawl, scramble, climb, spring, tramp, pad%, tread%, trek, scurry, stalk, sneak, dance, prowl, scuttle, vault, tiptoe, barge, proceed, amble, creep, hike, wander, dash, swim, sidle, jog, clamber, slog, trudge, steal, hasten, meander, scamper, hop, toddle, skip, troop, pounce, stagger, saunter, head, dart, limp, waltz, trot, waddle, leap, stomp, storm, lumber, slither, plod, totter, rush, march, fly, swing, step, hurry, shuffle, wade, walk
}\\

{[NP.Ext]}$_{\feinsub{SMov}}$ {[AVP]}$_{\feinsub{Path}}$  & 187 & \textit{stumble, lurch, frolic, lunge, stroll, bustle, stride, swagger, crawl, climb, spring, tread, trek, scurry, trip, trundle, strut%, roam, prowl, scuttle, jump, straggle, tiptoe, pace, creep, wander, dash, drive, swim, wriggle, clamber, slog, trudge, hasten, scamper, hop, stagger, saunter, slop, head, dart, limp, trot, waddle, leap, lumber, slither, plod, bound, slink, totter, rush, march, step, hurry, shuffle, wade
}\\

{[NP.Ext]}$_{\feinsub{SMov}}$ {[PP]}$_{\feinsub{Goal}}$ {[PP]}$_{\feinsub{Src}}$  & 175 & \textit{stumble, hobble, lurch, sprint, lope, lunge, stroll, bustle, stride, crawl, scramble, climb, spring, pad, trek, edge%, scurry%, trip, romp, trundle, strut, sneak, dance, scuttle, jump, tiptoe, proceed, pace, creep, venture, wander, dash, swim, traipse, clamber, trudge, sail, hasten, scamper, scoot, hop, toddle, troop, stagger, saunter, head, dart, sashay, waltz, file, trot, waddle, leap, stomp, storm, lumber, slither, plod, bound, slink, totter, rush, march, swing, step, hurry, shuffle, wade, walk
}\\

{[NP.Ext]}$_{\feinsub{SMov}}$ {[AVP]}$_{\feinsub{Goal}}$  & 149 & \textit{stumble, hobble, lurch, sprint, lope, run, lunge, stroll, bustle, crawl, scramble, climb, spring, tramp, pad, edge%, scurry%, stalk, roam, sneak, scuttle, jump, tiptoe, pace, creep, hike, venture, wander, dash, swim, repair, jog, clamber, trudge, steal, hasten, scamper, scoot, hop, skip, troop, stagger, saunter, head, dart, limp, waltz, leap, skulk, stomp, storm, lumber, slither, plod, bound, rush, march, swing, step, hurry, shuffle, wade, walk
}\\

{[NP.Ext]}$_{\feinsub{SMov}}$ {[AVP]}$_{\feinsub{Src}}$  & 113 & \textit{stumble, hobble, sprint, lope, stroll, bustle, stride, swagger, crawl, scramble, tramp, pad, scurry, romp, stalk, sneak%, dance, scuttle, tiptoe, amble, pace, taxi, creep, whisk, venture, wander, drive, swim, traipse, flit, trudge, steal, hasten, scamper, hop, skip, stagger, saunter, dart, sashay, limp, waltz, trot, waddle, stomp, lumber, slither, slink, totter, rush, march, fly, step, hurry, shuffle, flounce
}\\

{[NP.Ext]}$_{\feinsub{SMov}}$ {[PP]}$_{\feinsub{Path}}$ {[PP]}$_{\feinsub{Path}}$  & 107 & \textit{stumble, hobble, lurch, sprint, lope, lunge, stroll, bustle, stride, crawl, scramble, climb, spring, tramp, pad, trek%, scurry%, trundle, strut, dance, scuttle, tiptoe, proceed, amble, pace, creep, wander, dash, swim, jog, clamber, trudge, skip, stagger, saunter, head, dart, sashay, trot, waddle, leap, lumber, slither, plod, totter, rush, march, hurry, shuffle, wade
}\\
%
{[NP.Ext]}$_{\feinsub{SMov}}$ {[PP]}$_{\feinsub{Path}}$ {[PP]}$_{\feinsub{Src}}$  & 96 & \textit{jog, prance, clamber, trudge, lurch, scamper, sprint, hop, lope, troop, stagger, stroll, bustle, stride, swagger, crawl, climb%, spring, head, tramp, trot, waddle, leap, dance, prowl, scuttle, jump, straggle, tiptoe, barge, lumber, slither, bound, creep, slink, rush, march, wander, step, dash, shuffle, swim, walk
%
}\\ 
\lspbottomrule
%{[NP.Ext]}$_{\feinsub{SMov}}$ {[\_]}$_{\text{AREA-INI}}$  & 50 & \textit{prance, lurch, hop, parade, crawl, climb, dart, tramp, limp, waddle, burrow, leap, roam, dance, prowl, caper, amble%, pace, canter, march, promenade, wander, drive, swim, walk
%}\\ \midrule
%{[NP.Ext]}$_{\feinsub{SMov}}$ {[NP.Dep]}$_{\feinsub{Area}}$  & 34 & \textit{tramp, tread, trudge, trot, pace, roam, drive, dance, prowl}\\ \midrule
%{[NP.Ext]}$_{\feinsub{SMov}}$ {[PP]}$_{\feinsub{Goal}}$ {[AVP]}$_{\feinsub{Path}}$  & 34 & \textit{wriggle, slither, sail, steal, lurch, creep, lunge, stagger, stride, totter, rush, march, crawl, climb, head, waltz, romp%, wander, trundle, burrow, step, wade, swim, jump
%}\\ \midrule
%{[NP.Ext]}$_{\feinsub{SMov}}$ {[PP]}$_{\feinsub{Area}}$ {[PP]}$_{\feinsub{Area}}$  & 33 & \textit{straggle, slither, trudge, hop, creep, skip, frolic, stroll, crawl, dart, wander, trundle, skulk, strut, dash, roam, wade, dance, flounce, swim, jump, traipse}\\ \midrule
%{[NP.Ext]}$_{\feinsub{SMov}}$ {[AVP]}$_{\feinsub{Path}}$ {[PP]}$_{\feinsub{Path}}$  & 32 & \textit{lurch, scamper, sprint, hop, lunge, stagger, bustle, stride, crawl, spring, head, dart, tramp, trek, trip, trot, leap, strut, prowl, barge, proceed, plod, pace, rush, march, wander}\\ \midrule
%{[NP.Ext]}$_{\feinsub{SMov}}$ {[PP]}$_{\feinsub{Path}}$ {[AVP]}$_{\feinsub{Src}}$  & 31 & \textit{stumble, hobble, clamber, trudge, run, stride, saunter, crawl, dart, tramp, scurry, waddle, trundle, stomp, sneak, scuttle%, make, tiptoe, slither, amble, pace, fly, wander, dash, shuffle
%}\\ \midrule
%{[NP.Ext]}$_{\feinsub{SMov}}$ {[NP.Dep]}$_{\feinsub{Path}}$  & 26 & \textit{tiptoe, tread, trek, promenade, trot, scamper, leap, skip, wade, make, vault, walk}\\ \midrule
%{[NP.Ext]}$_{\feinsub{SMov}}$ {[PP]}$_{\feinsub{Goal}}$ {[AVP]}$_{\feinsub{Src}}$  & 25 & \textit{wriggle, hobble, sidle, plod, scamper, skip, slink, rush, march, venture, stalk, wander, skulk, step, hurry, sneak, drive, dance, swim}\\ \midrule
%{[NP.Ext]}$_{\feinsub{SMov}}$ {[NP.Obj]}$_{\feinsub{Path}}$  & 25 & \textit{stumble, sidle, skim, stroll, swagger, scramble, climb, spring, head, step, make, swim, walk}\\ \midrule
%{[NP.Ext]}$_{\feinsub{SMov}}$  & 22 & \textit{spring, take to the air, run, lunge, stroll, walk}\\ \midrule
%{[NP.Ext]}$_{\feinsub{SMov}}$ {[PP]}$_{\feinsub{Goal}}$ {[PP]}$_{\feinsub{Path}}$ {[PP]}$_{\feinsub{Src}}$  & 21 & \textit{clamber, lurch, hop, stroll, stride, rush, climb, spring, tramp, fly, scurry, step, dash, hurry, shuffle, sneak, dance, swim, walk}\\ \midrule
%{[NP.Ext]}$_{\feinsub{SMov}}$ {[AVP]}$_{\feinsub{Area}}$  & 21 & \textit{stumble, clamber, lurch, run, stagger, scramble, slop, fly, roam, dance, prowl, swim, walk}\\ \midrule
%{[NP.Ext]}$_{\feinsub{SMov}}$ {[AVP]}$_{\feinsub{Path}}$ {[PP]}$_{\feinsub{Src}}$  & 20 & \textit{wriggle, storm, slither, pace, sprint, creep, troop, stagger, rush, saunter, march, head, dart, tramp, step, swim, jump}\\ \midrule
%{[NP.Ext]}$_{\feinsub{SMov}}$ {[AVP]}$_{\feinsub{Dir}}$  & 18 & \textit{head, edge, file, shrink, amble, rip, run, press, walk}\\ \midrule
%{[NP.Ext]}$_{\feinsub{SMov}}$ {[AVP]}$_{\feinsub{Goal}}$ {[PP]}$_{\feinsub{Src}}$  & 13 & \textit{dart, pad, jog, trot, slink, troop, dash, sneak, stride, rush, scuttle, swim} 
%\\ \midrule							 \midrule
\end{tabularx}
\caption{FrameNet valence patterns of \framename{Self\_Motion} verbs}
\label{tab:4:self-motion-valence-framenet}
\end{table} 

\subsubsection{Verbs evoking the \framename{Self\_motion} frame in Bulgarian}

Many of the Bulgarian verbs that evoke the \framename{Self\_motion} frame are manner of motion simplex verbs. This aligns with the fact that, overall, \fename{Self\_motion} describes the idea of movement without profiling any of the route-related aspects of motion. Other verbs, such as \textit{втурвам се} `rush’, \textit{отправям се} `head’, `make’, \textit{спускам се} `dash’, `dart’, \textit{налитам}, \textit{хвърлям се} `barge’, etc., involve directed motion or the initial phase of motion rather than manner. Such verbs usually come in aspectual pairs. 

\subsubsection{Syntactic realisation of \framename{Self\_motion} verbs in Bulgarian}

\tabref{tab:4:selfmotionbg} shows the results for several frequent Bulgarian verbs with correspondence in \tabref{tab:4:self-motion-synt}. For the English data  \fename{Goals} are found in competition with \fename{Paths} and other motion-related FEs, either in fewer numbers, but still well-represented across many verbs, or in greater numbers than the other FEs co-occurring with the respective predicates. The most notable difference found in the Bulgarian sample is the lower frequency of \fename{Goals} as compared with the data in the FrameNet corpus.

The verbs \textit{бродя} `roam’, `wander’, \textit{вървя} `walk’, \textit{катеря се} `climb’, \textit{плувам} `swim’, \textit{пълзя} `crawl’, \textit{тичам} `run’ (Examples \ref{ex:09}--\ref{ex:14}) all show a lower occurrence of \fename{Goals}, whereas in English \textit{climb} and \textit{walk} co-occur equally with both FEs and \textit{run} shows preference for \fename{Goals} over \fename{Paths}.  

\begin{exe}
\ex \label{ex:09}
  %  \settowidth \jamwidth{(bg)} 
\gll [\textit{Ноа}]$_{\feinsub{SMov}}$ \textit{\textbf{ТИЧАШЕ}} [\textit{към} \textit{него}]$_{\feinsub{Goal}}$. \\
Noah run-PST.3SG towards him. \\
\glt `Noah was running towards him.' 
%\fename{Goal}: 
%[\textit{Той}]$_{\feinsub{SMov}}$ \textit{тичаше} [\textit{към} \textit{нас}]$_{\feinsub{Goal}}$\\
%{[\textit{He}]}$_{\text{SELF-MOVER}}$ \textit{was} \textit{running} [\textit{towards} \textit{us}]$_{\feinsub{Goal}}$.
\ex \label{ex:10}
%\fename{Path}: 
\gll [\textit{Той}]$_{\feinsub{SMov}}$ \textit{\textbf{ТИЧАШЕ}} [\textit{по} \textit{дългия} \textit{коридор}]$_{\feinsub{Path}}$.\\
He run-PST.3SG down long-DEF hall. \\
\glt `He was running down the long hall.' 
%ю\fename{Area}: 
%\fename{Direction}: 
\ex \label{ex:11}
\gll [\textit{Той}]$_{\feinsub{SMov}}$ {\textit{\textbf{СЕ}} \textit{\textbf{КАТЕРИ}}} 
 [\textit{по} \textit{хълма}]$_{\feinsub{Path}}$.\\
He climb-PRS.3SG up hill-DEF. \\
\glt `He is climbing up the hill.'
\ex \label{ex:12}
%\fename{Source}: 
\gll [\textit{Те}]$_{\feinsub{SMov}}$ \textit{\textbf{БРОДЕХА}} [\textit{по} \textit{коридорите}]$_{\feinsub{Path}}$.\\
They wander-PST.3PL along corridors-DEF. \\
\glt `They wandered along the corridors.' 
\ex \label{ex:13}
\gll [\textit{Назгулите}]$_{\feinsub{SMov}}$ \textit{\textbf{БРОДЯТ}} [\textit{по} \textit{земята}]$_{\feinsub{Area}}$.\\
Nazgul-DEF roam-PRS.3PL across earth-DEF. \\
\glt `The Nazgul roam the earth.'
\ex \label{ex:14}
%\fename{Distance}: 
\gll [\textit{Той}]$_{\feinsub{SMov}}$ \textit{\textbf{ПЛУВАШЕ}} [\textit{из} \textit{бурното} \textit{море}]$_{\feinsub{Area}}$. \\
He swim-PST.3SG across stormy-DEF sea. \\
\glt `He was swimming in the stormy seas.'
%\ex
%\gll [\textit{Фен}]$_{\feinsub{SMov}}$ \textit{тича} [\textit{6 км}]$_{\feinsub{Distance}}$ \textit{след} \textit{автобуса}.\\
%[Fan-INDF]$_{\feinsub{SMov}}$ ran [6km]$_{\feinsub{Distance}}$ behind bus-DEF. \\
%\glt `A fan ran 6 km behind the bus.'
%\end{xlist}
% \jambox{(bg)}
\end{exe}

This is at least partly predictable: while English manner of motion verbs express directionality by means of \fename{Goal}- or \fename{Source}-phrases or particles with a similar meaning, the corresponding Slavic (Bulgarian) simplex verbs may also derive new verbs with a directional meaning through prefixation (\cite{Beavers2010,Lindsey2011,Pantcheva-2007,Pantcheva2007,Pantcheva2011,speed:2015}, among many others). While simplex verbs can realise directionality by means of route-related phrases, the derived prefixed verbs profile the relevant aspect of the route and encode it in their lexical structure; the two types of verbs may be used interchangeably in certain contexts, but not in others. As a cursory illustration of this point, consider the verb in (Example \ref{ex:224:a}), whose directional meaning cannot be expressed by the simplex verb it is derived from; hence the expression \textit{в стаята} in (Example \ref{ex:224:b}) cannot be interpreted as the \fename{Goal} (marked by an asterisk); still, it will have an English correspondence of manner of motion verb + a directional phrase. 

\begin{exe}
\ex \label{ex:224}
\begin{xlist}
\ex  \label{ex:224:a}
  %  \settowidth \jamwidth{(bg)} 
\gll [\textit{Птицата}]$_{\feinsub{SMov}}$ \textit{\textbf{ВЛИТА}} [\textit{в} \textit{стаята}]$_{\feinsub{Goal}}$. \\
Bird-DEF fly-PRS.3SG into room-DEF. \\
\glt `The bird flies into the room.'
\ex  \label{ex:224:b}
\gll *[\textit{Птицата}]$_{\feinsub{SMov}}$ \textit{\textbf{ЛЕТИ}} [\textit{в} \textit{стаята}]$_{\feinsub{Goal}}$. \\
Bird-DEF fly-PRS.3SG into room-DEF. \\
\glt `The bird flies into the room.'
\end{xlist}
\end{exe}


{\footnotesize
\begin{longtable}{l ccccccccc}   
\caption{Syntactic expression of the \framename{Self\_motion} FEs in Bulgarian}\label{tab:4:selfmotionbg}\\
 \midrule
  & NP.Ext & NP.Obj & PP & AVP & NI & Clause & Other & Total\\ \midrule\endfirsthead
\midrule
  & NP.Ext & NP.Obj & PP & AVP & NI & Clause & Other & Total\\
  \midrule\endhead
\multicolumn{9}{l}{\textit{пълзя} `crawl’ }\\*
\fename{Self\_mover} & 40 &  &  &  &  &  &  & 40\\*
\fename{Area} &  &  & 4 & 2 &  &  &  & 6\\*
\fename{Path} &  &  & 19 & 1 &  &  &  & 20\\*
\fename{Goal} &  &  & 8 & 1 &  &  &  & 9\\*
\fename{Direction} &  &  &  & 2 &  &  &  & 2\\ 
 \midrule
%\multicolumn{9}{l}{\textit{плавам} }\\*
%\fename{Area} &  &  & 1 &  &  &  &  & 1\\*
%\fename{Self\_mover} & 2 &  &  &  &  &  &  & 2\\ 
% \midrule
\multicolumn{9}{l}{\textit{катеря се} `climb’ }\\*
\fename{Self\_mover} & 48 &  &  &  &  &  &  & 48\\*
\fename{Area} &  &  & 2 &  &  &  &  & 2\\*
\fename{Source} &  &  & 1 &  &  &  &  & 1\\* 
\fename{Path} &  &  & 22 & 1 &  &  &  & 23\\*
\fename{Goal} &  &  & 14 & 2 &  &  &  & 16\\
%\fename{Manner} &  &  &  & 1 &  &  &  & 1\\*
 \midrule
\multicolumn{9}{l}{\textit{бродя} `roam’, `wander’ }\\*
\fename{Self\_mover} & 39 &  &  &  &  &  &  & 39\\*
\fename{Area} &  &  & 18 & 2 &  &  &  & 20\\*
\fename{Source} &  &  & 1 &  &  &  &  & 1\\*
\fename{Path} &  &  & 13 &  &  &  &  & 13\\*
\fename{Goal} &  &  & 1 &  &  &  &  & 1\\
%\fename{Manner} &  &  &  & 2 &  &  &  & 2\\*
\midrule
\multicolumn{9}{l}{\textit{плувам} `swim’ }\\*
\fename{Self\_mover} & 37 &  &  &  &  &  &  & 37\\*
\fename{Area} &  &  & 9 & 3 &  &  &  & 12\\*
\fename{Path} &  &  & 6 & 1 &  &  &  & 7\\*
\fename{Goal} &  &  & 5 &  &  &  &  & 5\\
%\fename{Manner} &  &  &  & 3 &  &  &  & 3\\ 
 \midrule
\multicolumn{9}{l}{\textit{тичам} `run’ }\\*
\fename{Self\_mover} & 42 &  &  &  &  &  &  & 42\\*
\fename{Area} &  &  & 3 &  &  &  &  & 3\\*
\fename{Source} &  &  & 1 &  &  &  &  & 1\\*
\fename{Path} &  &  & 23 & 1 &  &  &  & 24\\*
\fename{Goal} &  &  & 11 & 1 &  &  &  & 12\\
%\fename{Manner} &  &  & 3 & 1 &  &  &  & 4\\*
 \midrule
\multicolumn{9}{l}{\textit{вървя} `walk’ }\\*
\fename{Self\_mover} & 40 &  &  &  &  &  &  & 40\\*
\fename{Area} &  &  &  & 1 &  &  &  & 1\\*
\fename{Path} &  &  & 16 & 2 &  &  &  & 18\\*
\fename{Goal} &  &  & 2 & 3 &  &  &  & 5\\*
\fename{Direction} &  &  & 1 & 1 &  &  &  & 2\\*
%\fename{Manner} &  &  & 2 & 4 &  &  &  & 6\\ 
 \midrule
\end{longtable}
}

The above observations are confirmed by the distribution of the patterns involving \fename{Paths}, \fename{Goals} and \fename{Areas} in the Bulgarian dataset (\tabref{tab:4:selfmotion-valence-bg}).

\begin{table}
   \begin{tabularx}{\textwidth}{lrQ} 
   \lsptoprule
   Pattern & \# & Verbs \\ \midrule
{[NP.Ext]}$_{\feinsub{SMov}}$ {[PP]}$_{\feinsub{Path}}$ & 86 & \textit{вървя, плувам, пълзя, бродя, тичам, катеря се}\\ 
{[NP.Ext]}$_{\feinsub{SMov}}$ {[\_]}$_{\feinsub{Path-INI}}$ & 45 & \textit{вървя, плувам, пълзя, бродя, тичам, катеря се}\\ 
{[NP.Ext]}$_{\feinsub{SMov}}$ {[PP]}$_{\feinsub{Area}}$ & 37 & \textit{плувам, пълзя, бродя, тичам, катеря се}\\ 
{[NP.Ext]}$_{\feinsub{SMov}}$ {[PP]}$_{\feinsub{Goal}}$ & 35 & \textit{вървя, плувам, пълзя, тичам, катеря се}\\ 
{[NP.Ext]}$_{\feinsub{SMov}}$ {[AVP]}$_{\feinsub{Area}}$ & 6 & \textit{плувам, пълзя, бродя}\\ 
{[NP.Ext]}$_{\feinsub{SMov}}$ {[AVP]}$_{\feinsub{Goal}}$ & 6 & \textit{вървя, пълзя, катеря се}\\ 
%{[NP.Ext]}$_{\feinsub{SMov}}$ {[AVP]}$_{\text{MANNER}}$ {[PP]}$_{\feinsub{Path}}$ & 5 & \textit{вървя, бродя}\\ 
{[NP.Ext]}$_{\feinsub{SMov}}$ {[AVP]}$_{\feinsub{Path}}$ & 5 & \textit{вървя, плувам, пълзя, катеря се}\\ 
%{[NP.Ext]}$_{\feinsub{SMov}}$ {[AVP]}$_{\text{MANNER}}$ & 5 & \textit{вървя, плувам, тичам, катеря се}\\ 
%{[NP.Ext]}$_{\feinsub{SMov}}$ {[PP]}$_{\text{MANNER}}$ & 3 & \textit{вървя, тичам}\\  %
%{[NP.Ext]}$_{\feinsub{SMov}}$ {[PP]}$_{\feinsub{Goal}}$ {[PP]}$_{\feinsub{Src}}$ & 3 & %\textit{бродя, тичам, катеря се}\\ 
%{[NP.Ext]}$_{\feinsub{SMov}}$ {[PP]}$_{\feinsub{Goal}}$ {[PP]}$_{\feinsub{Path}}$ & 2 & %\textit{тичам, катеря се}\\ 
%{[NP.Ext]}$_{\feinsub{SMov}}$ {[PP]}$_{\text{MANNER}}$ {[PP]}$_{\feinsub{Path}}$ & 2 & \textit{тичам}\\ 
%{[NP.Ext]}$_{\feinsub{SMov}}$ {[AVP]}$_{\feinsub{Dir}}$ & 2 & \textit{вървя, пълзя}\\ %
%{[NP.Ext]}$_{\feinsub{SMov}}$ {[AVP]}$_{\feinsub{Area}}$ {[AVP]}$_{\text{MANNER}}$ & 1 & %\textit{плувам}\\ 
%{[NP.Ext]}$_{\feinsub{SMov}}$ {[AVP]}$_{\feinsub{Path}}$ {[PP]}$_{\feinsub{Goal}}$ & 1 & \textit{тичам}\\ 
%{[NP.Ext]}$_{\feinsub{SMov}}$ {[AVP]}$_{\feinsub{Dir}}$ {[PP]}$_{\feinsub{Path}}$ & 1 & %\textit{пълзя}\\ 
%{[NP.Ext]}$_{\feinsub{SMov}}$ {[PP]}$_{\feinsub{Dir}}$ {[PP]}$_{\feinsub{Path}}$ & 1 & %\textit{вървя}\\ 
%{[NP.Ext]}$_{\feinsub{SMov}}$ {[AVP]}$_{\feinsub{Area}}$ {[PP]}$_{\feinsub{Path}}$ & 1 & \textit{вървя}\\ 
%{[NP.Ext]}$_{\feinsub{SMov}}$ {[AVP]}$_{\feinsub{Goal}}$ {[PP]}$_{\feinsub{Path}}$ & 1 & \textit{тичам}\\ 
\lspbottomrule
\end{tabularx}
    \caption{Valence patterns of \framename{Self\_motion} verbs in Bulgarian}
    \label{tab:4:selfmotion-valence-bg}
\end{table}


\subsection{\framename{Traversing}, \framename{Arriving}, \framename{Departing}}\label{traversing}

These semantic frames narrow down the idea of motion through profiling aspects of the general motion schema corresponding to elements of the route along which the moving object changes location: the initial stage of the motion corresponding to the \fename{Source}; the end-stage -- associated with the \fename{Goal}, or the middle stage -- corresponding to the \fename{Path}, cf. \citep[16]{Johnson2001}. Borrowed from \citet{Langacker1987}, profiling is understood as ``the representation of the foregrounded part of a frame, the participant, prop, phase or moment which figures centrally in the semantic interpretation of the sentence within which the frame is evoked” \citep[16]{Fillmore2001}. 

\fename{Goal}-profiling LUs (e.g., \textit{arrive}, \textit{reach}) evoke the semantic frame \framename{Arriving};  \fename{Source}\hyp profiling LUs (e.g. \textit{leave}, \textit{depart}) evoke the \framename{Departing} frame; \fename{Path}\hyp profiling LUs, such as \textit{traverse}, \textit{cross} correspond to the \framename{Traversing} frame. \framename{Arriving} and \framename{Departing} are defined as subframes of \framename{Traversing}: as such, each of them describes a state or transition in the conceptualisation of a complex situation referring to the sequence of transitions from the \fename{Source}, through the \fename{Path}, to the \fename{Goal}. \framename{Departing} and \framename{Arriving} are related to each other by means of the Precedes relation. %, which, as implemented in FrameNet, means that \framename{Departing} temporally precedes (and is prerequisite for) \framename{Arriving}.

The profiling of a given FE is associated with the fact that the respective FE is central to the meaning and is always conceptually implied even if not necessarily overtly realised. In such cases it is often retrievable from the context and is thus understood and annotated as a definite null instantiation (DNI).


\subsubsection{\framename{Arriving}}

\subsubsection{Semantic description of the \framename{Arriving} frame}

\framename{Arriving} describes directed motion towards an end point which is part of the lexical encoding of the relevant LUs: i.e. the verbs evoking the frame are \fename{Goal}-oriented verbs of inherently directed motion.   

\begin{description}[font=\normalfont]
\item[Definition of the frame \framename{Arriving}:] An object, \fename{Theme}, moves in the direction of a \fename{Goal}. The \fename{Goal} may be expressed or it may be understood from the context, but it is always implied by the verb itself. 

\item[Core frame elements:] \fename{Theme}, \fename{Goal}. The core FEs of the \framename{Arriving} frame represent a subset of the core FEs of the  \framename{Traversing} frame of which it is a subframe. The FEs share the definition and semantic properties of their correspondences in the \framename{Motion} frame. %(inherited from the \framename{Motion\_scenario} frame) 
The profiling of the \fename{Goal} results in the backgrounding or exclusion of the remaining elements that form part of the core FEs of \framename{Traversing}. \fename{Source} and \fename{Path} become peripheral, while \fename{Path\_shape}, \fename{Distance}, \fename{Direction}, \fename{Area}, as defined in FrameNet, are not conceptually present in the scenario described by this frame.
%there are examples...
\end{description}


\subsubsection{Verbs evoking the \framename{Arriving} frame}

The verbs evoking the \framename{Arriving} frame form a central part of the lexis of \fename{Goal}-directed motion: \textit{appear}, \textit{approach}, \textit{arrive}, \textit{come}, \textit{crest}, \textit{descend (on)}, \textit{enter}, \textit{get}, \textit{hit}, \textit{make it}, \textit{make}, \textit{reach}, \textit{return}, \textit{visit}.

\subsubsection{Syntactic realisation of the frame elements in the \framename{Arriving} frame}

The syntactic realisation of the frame elements in the \framename{Arriving} frame as represented in the FrameNet corpus are illustrated in \tabref{tab:4:arriving-synt}. The \fename{Theme} is projected as a subject, while depending on the verb, the \fename{Goal} may be expressed as either a prepositional or adverbial phrase -- e.g. \textit{arrive}, \textit{come}, \textit{return}, \textit{get}, \textit{make it}, or as a direct object (NP.Obj) -- e.g. \textit{approach}, \textit{enter}, \textit{reach}, \textit{visit}.

{\footnotesize
\begin{longtable}{l ccccccccc}
\caption{\label{tab:4:arriving-synt}Syntactic expression of the \framename{Arriving} FEs in FrameNet}\\
\lsptoprule
  & NP.Ext & NP.Obj & PP & AVP & NI & Clause & Other & Total\\ \midrule \endfirsthead
  \midrule
    & NP.Ext & NP.Obj & PP & AVP & NI & Clause & Other & Total\\ \midrule \endhead
\textit{approach} &&&&&&&&\\*
\fename{Theme} & 36 &  & 1 &  &   &  &  & 37 \\*
\fename{Goal} & 1 & 29 &  & 1 & 6   &  &  & 37 \\ 
\midrule
\textit{arrive} &&&&&&&&\\*
\fename{Theme} & 81 &  &  &  &    &  &  & 81 \\*
\fename{Goal} & 1 &  & 31 & 11 & 35   &  &  & 78 \\ 
\midrule
\textit{come} &&&&&& &&\\*
\fename{Theme} & 119 &  &  &  &    &  &  & 119 \\*
\fename{Goal} &  & 2 & 44 & 16 & 50 & 3 & 2 & 117 \\ 
 \midrule
\textit{enter} &&&&&& &&\\*
\fename{Theme} & 30 & 1 &  &  & 3 &  &  & 34 \\*
\fename{Goal} & 3 & 17 &  & & 10 &  & 1 & 31 \\ 
 \midrule
\textit{return} &&&&&& &&\\*
\fename{Theme} & 48 &  &  &  & 3 &  &  & 48 \\*
\fename{Goal} & & 1 & 17 & 9 & 21 & 1 & & 49 \\ 
 \midrule
\textit{visit} &&&&&& &&\\*
\fename{Theme} & 24 &  & 3 &  & 2 &  &  & 29 \\*
\fename{Goal} & 5 & 14 & 2 &  & 7 & 1 & 1 & 29 \\ 
 \midrule
\textit{reach} &&&&&& &&\\*
\fename{Theme} & 50 & 1 &  &  & 7 &  &  & 58 \\*
\fename{Goal} & 7 & 48 &  & 1 &  &  &  & 56 \\ 
 \midrule
\textit{get} &&&&&& &&\\*
\fename{Theme} & 35 &  &  &  &  &  &  & 35 \\*
\fename{Goal} & &  & 16 & 12 & 7 &  & 1 & 36 \\ 
 \midrule
\textit{make it} &&&&&& &&\\*
\fename{Theme} & 22 &  &  &  &  &  &  & 22 \\*
\fename{Goal} & & & 12 & 1 & 9 &  &  & 22 \\ 
 \lspbottomrule
 \end{longtable}
}  
%\end{table}

The possibility for leaving the \fename{Goal} non-overt as reflected in the considerable number of definite null instantiations (NIs in the table), stems from the fact that with some verbs this FE often receives a definite interpretation as the %place where the speaker is located 
deictic centre and its identity is thus implied even without previous reference. 
This is typical for \textit{come} and to a lesser degree for \textit{arrive} due to their deictic nature. In this respect they are clearly distinct from \textit{reach}, %(which does not imply the identity of the \fename{Goal} as any place, including the location of the speaker) 
 \textit{approach}, \textit{visit}, \textit{get} and \textit{make it}, which usually express the \fename{Goal}, as it need not be identical to the deictic centre. Examples (\ref{ex:222:a}, \ref{ex:222:b}) illustrate this point.

\begin{exe}
\ex \label{ex:222}
\begin{xlist}
\ex[]{\label{ex:222:a}
  %  \settowidth \jamwidth{(en)} 
[\textit{She}]$_{\feinsub{Thm}}$ \textit{\textbf{REACHED}} [\textit{Rome}]$_{\feinsub{Goal}}$ [\textit{via Assisi}]$_{\feinsub{Path}}$.} 
\ex[*]{\label{ex:222:b}
  %  \settowidth \jamwidth{(en)} 
[\textit{She}]$_{\feinsub{Thm}}$ \textit{\textbf{REACHED}}. 
  %\jambox{(en)}
}
 \end{xlist}
\end{exe}

%\begin{exe}
%\ex  \label{ex223}
  %  \settowidth \jamwidth{(en)} 
%*[\textit{She}]$_{\feinsub{Thm}}$ \textit{reached}. 
 %\jambox{(en)}
%\end{exe}

\subsubsection{FrameNet valence patterns}

In line with the above observations, syntactically implicit \fename{Goals} represent half of the aggregated number of the \fename{Goal}-phrases (\tabref{tab:4:arriving-valence-framenet}). There is a considerable number of NP \fename{Goals}, which accounts for the fact that a great deal of the verbs are transitive. In addition, AVPs are much more prominent: they make up for a third of the prepositional \fename{Goal}-phrases, while in \framename{Self\_motion} their number is 15\% of the number of \fename{Goal}-PPs.

\begin{table}
    \begin{tabularx}{\textwidth}{lrQ}
    \lsptoprule
     Pattern & \# & Verbs \\ 
     \midrule
{}[NP]$_{\feinsub{Thm}}$ [\ ]$_{\feinsub{Goal-DNI}}$&  144& \textit{appear}, \textit{approach}, \textit{arrive}, \textit{come}, \textit{enter}, \textit{return}, \textit{visit}, \textit{get}, \textit{make it} \\
{}[NP]$_{\feinsub{Thm}}$ [NP]$_{\feinsub{Goal}}$ &  126&  \textit{approach}, \textit{enter}, \textit{visit}, \textit{reach}, \textit{make}, \textit{crest}, \textit{hit} \\ 
{}[NP]$_{\feinsub{Thm}}$ [PP]$_{\feinsub{Goal}}$& 121 & \textit{arrive}, \textit{come}, \textit{return}, \textit{visit}, \textit{get}, \textit{make it}, \textit{descend (on)}, \textit{appear} \\
{}[NP]$_{\feinsub{Thm}}$ [AVP]$_{\feinsub{Goal}}$ &  46&  \textit{approach}, \textit{arrive}, \textit{come}, \textit{return}, \textit{reach}, \textit{get}, \textit{make it}\\
\lspbottomrule
%[NP]$_{\feinsub{Thm}}$ [PP/AdvP]$_{\feinsub{Path}}$& 3& влизам, идвам, навлизам \\\midrule  [NP]$_{\feinsub{Thm}}$ [PP/AdvP]$_{\feinsub{Goal}}$    \newline
 % [PP/AdvP]$_{\feinsub{Path}}$& 2& достигам, стигам \\\midrule
%[NP]$_{\feinsub{Thm}}$ [PP/AdvP]$_{\feinsub{Goal}}$  \newline
% [PP/AdvP]$_{\feinsub{Src}}$& 2& пристигам \\\midrule
%[NP]$_{\feinsub{Thm}}$ [\ ]$_{\feinsub{Goal-DNI}}$    \newline
% [PP/AdvP]$_{\feinsub{Path}}$& 1& пристигам\\\midrule
%[NP]$_{\feinsub{Thm}}$ [\ ]$_{\feinsub{Goal-DNI}}$   \newline
 %[PP/AdvP]$_{\feinsub{Src}}$& 1& идвам\\\midrule
    \end{tabularx}
    \caption{FrameNet valence patterns of \framename{Arriving} verbs}
    \label{tab:4:arriving-valence-framenet}
\end{table} 


\subsubsection{Syntactic realisation of \framename{Arriving} verbs in Bulgarian}

The basic verbs evoking the \framename{Arriving} frame form a small but central part of the lexis of directed motion: \textit{влизам} `enter’, \textit{връщам се} `return’, \textit{добирам се} `make it’, \textit{доближавам}, \textit{доближавам се} `approach’, \textit{достигам} `reach’, \textit{завръщам се} `return’, \textit{наближавам}, \textit{приближавам}, \textit{приближавам се} `approach’, \textit{идвам}, \textit{ида} `come’, \textit{пристигам} `arrive’, \textit{стигам} `reach’, \textit{посещавам} `visit’, \textit{прибирам се} `go home’. To the exception of \textit{посещавам}, which requires object NP \fename{Goals}, and \textit{доближавам}, \textit{наближавам}, \textit{приближавам}, \textit{достигам}, \textit{стигам} -- which take either an object NP or a PP\slash AVP, the rest of the verbs select a PP/AVP complement. In this respect the Bulgarian verbs differ from their English counterparts, many of which take an object \fename{Goal} complement.



\begin{longtable}{l ccccccccc}   
\caption{Syntactic expression of the \framename{Arriving} FEs in Bulgarian} 
    \label{tab:4:arriving-synt-bg}
 \\ \lsptoprule
  & NP.Ext & NP.Obj & PP & AVP & NI & Clause & Other & Total\\ \midrule \endfirsthead
  \midrule
  & NP.Ext & NP.Obj & PP & AVP & NI & Clause & Other & Total\\ \midrule \endhead
%\multicolumn{9}{l}{\textit{добера} }\\*
%\fename{Goal} &  &  & 2 &  &  &  &  & 2\\*
%\fename{Theme} & 2 &  &  &  &  &  &  & 2\\ 
% \midrule
%\multicolumn{9}{l}{\textit{ида} }\\*
%\fename{Goal} &  &  & 3 & 1 &  &  &  & 4\\*
%\fename{Theme} & 4 &  &  &  &  &  &  & 4\\ 
% \midrule
%\multicolumn{9}{l}{\textit{приближавам\slash приближа} }\\*
%\fename{Goal} &  &  & 5 &  &  &  &  & 5\\*
%\fename{Theme} & 5 &  &  &  &  &  &  & 5\\ 
% \midrule
\multicolumn{9}{l}{\textit{отивам\slash отида} `go’}\\*
\fename{Theme} & 10 &  &  &  &  &  &  & 10\\* 
\fename{Goal} &  &  & 6 & 1 & 3 &  &  & 10\\*
 \midrule
\multicolumn{9}{l}{\textit{достигам\slash достигна} `reach’}\\*
\fename{Theme} & 4 &  &  &  &  &  &  & 4\\* 
\fename{Goal} &  & 2 & 2 &  &  &  &  & 4\\
 \midrule
%\multicolumn{9}{l}{\textit{наближавам\slash наближа} }\\*
%\fename{Goal} &  & 1 &  &  &  &  &  & 1\\*
%\fename{Theme} & 1 &  &  &  &  &  &  & 1\\ 
% \midrule
\multicolumn{9}{l}{\textit{идвам\slash дойда} `come’}\\*
\fename{Theme} & 20 &  &  &  &  &  &  & 20\\*
\fename{Goal} &  &  & 11 &  & 9 &  &  & 20\\
 \midrule
\multicolumn{9}{l}{\textit{пристигам\slash пристигна} `arrive’}\\*
\fename{Theme} & 21 &  &  &  &  &  &  & 21\\* 
\fename{Goal} &  &  & 10 &  & 11 &  &  & 21\\
% \midrule
%\multicolumn{9}{l}{\textit{завърна} }\\*
%\fename{Goal} &  &  & 2 &  &  &  &  & 2\\*
%\fename{Theme} & 2 &  &  &  &  &  &  & 2\\ 
\midrule
\multicolumn{9}{l}{\textit{стигам\slash стигна} `reach’}\\*
\fename{Theme} & 15 &  &  &  &  &  &  & 15\\* 
\fename{Goal} &  & 2 & 11 & 2 &  &  &  & 15\\
 \midrule
%\multicolumn{9}{l}{\textit{посещавам\slash посетя} }\\*
%\fename{Goal} &  &  &  &  &  &  & 1 & 1\\*
%\fename{Theme} & 1 &  &  &  &  &  &  & 1\\ 
% \midrule
\multicolumn{9}{l}{\textit{влизам\slash вляза} `enter’}\\*
\fename{Theme} & 23 &  &  &  &  &  &  & 23\\*
\fename{Goal} &  &  & 13 &  & 10 &  &  & 23\\
 \midrule
%\multicolumn{9}{l}{\textit{навлизам\slash навляза} }\\*
%\fename{Goal} &  &  & 1 &  &  &  &  & 1\\*
%\fename{Theme} & 1 &  &  &  &  &  &  & 1\\ 
% \midrule
%\multicolumn{9}{l}{\textit{завръщам} }\\*
%\fename{Goal} &  &  & 2 &  &  &  &  & 2\\*
%\fename{Theme} & 2 &  &  &  &  &  &  & 2\\ 
% \midrule
%\multicolumn{9}{l}{\textit{прибирам\slash прибера} }\\*
%\fename{Goal} &  &  & 2 & 1 &  &  &  & 3\\*
%\fename{Theme} & 3 &  &  &  &  &  &  & 3\\ 
% \midrule
%\multicolumn{9}{l}{\textit{доближавам\slash доближа} }\\*
%\fename{Goal} &  & 1 & 1 &  &  &  &   & 2\\*
%\fename{Theme} & 2 &  &  &  &  &  &  & 2\\ 
% \midrule
\multicolumn{9}{l}{\textit{връщам се\slash върна се} `return’}\\*
\fename{Theme} & 14 &  &  &  &  &  &  & 14\\* 
\fename{Goal} &  &  & 7 & 4 & 3 &  &  & 14\\
 \lspbottomrule
\end{longtable}

As expected, the \fename{Goal}-PPs (Example \ref{ex:18:a}) predominate over NPs  (Example \ref{ex:18:b}) and AVPs (Example \ref{ex:18:c}) as shown in \tabref{tab:4:arriving-valence-bg}. The possibility of leaving the \fename{Goal} syntactically unexpressed if it is construable from the context (Example \ref{ex:18:d}) is underrepresented in the sample of annotated examples.

\begin{exe}
\ex \label{ex:18}
\begin{xlist}
\ex \label{ex:18:a}
\gll [\textit{Вражески} \textit{кораби}]$_{\feinsub{Thm}}$ \textit{\textbf{ИДВАТ}} [\textit{към} \textit{вас}]$_{\feinsub{Goal}}$. \\
Hostile aircraft come-PRS.3PL towards you. \\
\glt `Hostile spacecraft are coming your way.'
\ex \label{ex:18:b}
\gll \textit{Следобед} [\textit{те}]$_{\feinsub{Thm}}$ \textit{\textbf{ДОСТИГНАХА}} [\textit{брега}]$_{\feinsub{Goal}}$. \\
In-afternoon-DEF they reach-PST.3PL coast-DEF. \\
\glt `They reached the coast in the afternoon.'
\ex \label{ex:18:c}
\gll [\textit{Никой}]$_{\feinsub{Thm}}$ \textit{не} \textit{\textbf{СЕ}} \textit{\textbf{ВРЪЩА}} [\textit{тук}]$_{\feinsub{Goal}}$. \\
Nobody not REFL return-PRS.3SG-NEG here. \\
\glt `No one returns here.'
\ex \label{ex:18:d}
\gll [\textit{\ }]$_{\feinsub{Thm}}$ \textit{\textbf{ПРИСТИГАТЕ}} [\textit{\ }]$_{\feinsub{Goal}}$ \textit{точно} \textit{навреме,} \textit{докторе}! \\
{} Arrive-PRS.2PL {} just {on time}, doctor! \\
\glt `You arrive just on time, doctor!'
\end{xlist}
\end{exe}


\begin{table}
   \begin{tabularx}{\textwidth}{ lrQ } 
   \lsptoprule
   Pattern & \# & Verbs \\ \midrule
{[NP.Ext]}$_{\feinsub{Thm}}$ {[PP]}$_{\feinsub{Goal}}$ & 78 & \textit{влизам\slash вляза, връщам се\slash върна се, добирам се\slash добера се, доближавам (се)\slash доближа (се), идвам\slash дойда, достигам\slash достигна, завръщам се, завърна се, ида, навлизам\slash навляза, отивам\slash отида, прибирам се\slash прибера се, приближавам (се)\slash приближа (се), пристигам\slash пристигна, стигам\slash стигна}\\ 
{[NP.Ext]}$_{\feinsub{Thm}}$ {[\_]}$_{\feinsub{Goal}}$ & 36 & \textit{влизам\slash вляза, връщам се\slash върна се, идвам\slash дойда, отивам\slash отида}\\ 
{[NP.Ext]}$_{\feinsub{Thm}}$ {[AVP]}$_{\feinsub{Goal}}$ & 9 & \textit{връщам се\slash върна се, ида, отивам\slash отида, прибирам се\slash прибера се, стигам\slash стигна}\\ 
{[NP.Ext]}$_{\feinsub{Thm}}$ {[NP]}$_{\feinsub{Goal}}$ & 7 & \textit{доближавам\slash доближа, достигам\slash достигна, наближавам\slash наближа, посещавам\slash посетя, стигам\slash стигна}\\ 
\lspbottomrule
\end{tabularx}
    \caption{FrameNet valence patterns of \framename{Arriving} verbs in Bulgarian}
    \label{tab:4:arriving-valence-bg}
\end{table}


\subsubsection{\framename{Departing}}

\subsubsection{Semantic description of the \framename{Departing} frame}

\framename{Departing} describes directed motion away from a starting point, which is encoded in the lexical meaning of the respective LUs.

\begin{description}[font=\normalfont]
\item[Definition of the frame \framename{Departing}:] An object (the \fename{Theme}) moves away from a \fename{Source}. The \fename{Source} may be expressed or it may be understood from context, but its existence is always implied by the departing word itself.

\item[Core frame elements:] \fename{Theme}, \fename{Source}
\end{description}

Being a subframe of \framename{Traversing} that describes the other end point of translational motion, the description of the \framename{Departing} frame mirrors that of \framename{Arriving}, but the profiled FE is the \fename{Source}. The profiling results in the backgrounding of the \fename{Goal} and the \fename{Path} to peripheral FEs and the removal of the remaining route FEs present in the description of \framename{Traversing} (\fename{Path\_shape}, \fename{Distance}, \fename{Direction}, \fename{Area}) from the scenario described by \framename{Departing}.

\subsubsection{Verbs evoking the \framename{Departing} frame}

The basic verbs that evoke the \framename{Departing} frame form a central part of the lexis of \fename{Source}-oriented directed motion: \textit{decamp}, \textit{depart}, \textit{disappear}, \textit{emerge}, \textit{escape}, \textit{exit}, \textit{leave}, \textit{skedaddle}, \textit{vamoose}, \textit{vanish}.

\subsubsection{Syntactic realisation of the frame elements in the \framename{Departing} frame}

The syntactic realisation of the frame elements in the \framename{Departing} frame as represented in the FrameNet corpus examples are illustrated in \tabref{tab:4:departingsynt}. The \fename{Theme} is projected as a subject (NP.Ext), while depending on the verb the \fename{Source} may be expressed as either a prepositional or an adverbial phrase, e.g. \textit{disappear}, \textit{emerge}, \textit{vanish}, on the one hand, or as a direct object (NP.Obj), on the other: \textit{depart}, \textit{escape}, \textit{exit}, \textit{leave}.

Unlike \framename{Arriving} predicates, which show a distinct preference to either NP or PP/AVP \fename{Goals}, the FrameNet data for \framename{Departing} point to different distribution of NP and PP \fename{Sources} across the verbs (\tabref{tab:4:departingsynt}), compare \textit{depart}, where the two types of phrases are equally distributed and \textit{leave}, which favours NP.Obj.

The \framename{Departing} verbs show a similar tendency to leave the profiled element unexpressed (less prominent for the verb \textit{leave}) if it is retrievable from the wider context and/or the movement away takes place with reference to the speaker (i.e. the deictic centre). 

While some \framename{Arriving} verbs, such as \textit{arrive}, \textit{come}, \textit{get} and \textit{return} tend to express the \fename{Goal} as either a PP or an AVP, the \framename{Departing} verbs hardly opt for AVPs, at least in the FrameNet corpus. 

\begin{table}
\caption{Syntactic expression of the \framename{Departing} FEs in FrameNet} 
\label{tab:4:departingsynt}
\begin{tabular}{l cccccccc}   
\lsptoprule
  & NP.Ext & NP.Obj & PP & AVP & NI & Clause & Other & Total\\ \midrule
  %\multicolumn{9}{l}{\textit{decamp} } \\		
%\fename{Theme} & 13  &  &  &  &  &  &  & 13\\		
%\fename{Source} &  &  & 1  &  & 12  &  &  & 13\\		
%\midrule
\multicolumn{9}{l}{\textit{depart} } \\*
\fename{Theme} & 77  &  &  & 1  &  &  &  & 78\\*
\fename{Source} &  & 14  & 11  &  & 52  &  &  & 77\\		
\midrule
\multicolumn{9}{l}{\textit{disappear} } \\*
\fename{Theme} & 120  &  &  &  &  &  &  & 120\\*
\fename{Source} &  &  & 8  &  & 111  &  &  & 119\\		
\midrule
\multicolumn{9}{l}{\textit{escape} } \\*
\fename{Theme} & 16  &  &  &  &  &  &  & 16\\*
\fename{Source} &  & 4  & 2  & 1  & 9  &  &  & 16\\		
\midrule
%\multicolumn{9}{l}{\textit{vamoose} } \\		
%\midrule
\multicolumn{9}{l}{\textit{vanish} } \\*
\fename{Theme} & 69  &  & 1  &  &  &  &  & 70\\*
\fename{Source} &  &  & 12  &  & 57  &  &  & 69\\		
\midrule
%\multicolumn{9}{l}{\textit{skedaddle} } \\		
%\midrule
\multicolumn{9}{l}{\textit{exit} } \\*
\fename{Theme} & 32  &  &  &  &  &  &  & 32\\*
\fename{Source} &  & 5  & 5  &  & 21  & 1  &  & 32\\		
\midrule
\multicolumn{9}{l}{\textit{leave} } \\*
\fename{Theme} & 90  & 1  &  &  &  &  &  & 91\\*
\fename{Source} &  & 45  & 7  & 4  & 29  &  & 3 & 88\\		
\lspbottomrule
%\multicolumn{9}{l}{\textit{emerge} } \\		
%\fename{Theme} & 36  &  &  &  &  &  &  & 36\\		
%\fename{Source} &  &  & 12  &  & 24  &  &  & 36\\		
%\midrule
\end{tabular}
\end{table}

\subsubsection{FrameNet valence patterns}

While the \fename{Goal-DNIs} of the \framename{Arriving} verbs represent 33\% of the overall number of \fename{Goals}, syntactically implicit \fename{Sources} are the prevalent pattern, making up for 63\% of the aggregated number of the patterns with \fename{Source}-phrases (\tabref{tab:4:departing-valence-framenet}). In other words, judging from these data, \fename{Goal}-profiling verbs express syntactically the profiled element twice as frequently as do \fename{Source}-profiling verbs. This observation supports the goal-over-source asymmetry.

The number of the patterns with NP and PP \fename{Sources} is similar, while, as noted above, AVPs, are poorly represented (\tabref{tab:4:departing-valence-framenet}).


\begin{table}
   \begin{tabularx}{\textwidth}{ lrQ } 
   \lsptoprule
   Pattern & \# & Verbs \\ \midrule
   {[NP.Ext]}$_{\feinsub{Thm}}$ {[\_]}$_{\feinsub{Src-DNI}}$  & 312 & \textit{decamp, exit, leave, emerge, disappear, depart, escape, vanish}\\ 
{[NP.Ext]}$_{\feinsub{Thm}}$ {[NP.Obj]}$_{\feinsub{Src}}$  & 68 & \textit{exit, leave, depart, escape}\\ 
{[NP.Ext]}$_{\feinsub{Thm}}$ {[PP]}$_{\feinsub{Src}}$  & 58 & \textit{decamp, exit, leave, emerge, disappear, depart, escape, vanish}\\ 
{[NP.Ext]}$_{\feinsub{Thm}}$ {[AVP]}$_{\feinsub{Src}}$  & 5 & \textit{leave, escape}\\ 
\lspbottomrule
%{[NP.Ext]}$_{\feinsub{Thm}}$  & 3 & \textit{leave, disappear}\\ \midrule
%{[NP.Ext]}$_{\feinsub{Thm}}$ {[NP.Dep]}$_{\feinsub{Src}}$  & 3 & \textit{leave}\\ \midrule
%{[NP.Ext]}$_{\feinsub{Thm}}$ {[\_]}$_{\feinsub{Src-DNI}}$ {[PP]}$_{\feinsub{Thm}}$  & 1 & \textit{vanish}\\ \midrule
%{[NP.Ext]}$_{\feinsub{Thm}}$ {[\_]}$_{\feinsub{Src-DNI}}$ {[AVP]}$_{\feinsub{Thm}}$  & 1 & \textit{depart}\\ \midrule
%{[NP.Ext]}$_{\feinsub{Thm}}$ {[\_]}$_{\feinsub{Src-DNI}}$ {[NP.Obj]}$_{\feinsub{Thm}}$  & 1 & \textit{leave}\\ \midrule
%{[NP.Ext]}$_{\feinsub{Thm}}$ {[Clause]}$_{\feinsub{Src}}$  & 1 & \textit{exit}\\ \midrule
\end{tabularx}
    \caption{FrameNet valence patterns of \framename{Departing} verbs}
    \label{tab:4:departing-valence-framenet}
\end{table}

\subsubsection{Syntactic realisation of \framename{Departing} verbs in Bulgarian}

The Bulgarian verbs evoking the \framename{Departing} frame represent the central lexis of \fename{Source}-oriented directed motion verbs: \textit{заминавам} `depart’, \textit{избягвам} `escape’, \textit{излизам} `exit’, \textit{изчезвам} `disappear’, \textit{напускам} `leave’, \textit{отивам си} `leave’, `go home’, \textit{тръгвам} `leave’, `depart’, \textit{отдалечавам се} `move away’, etc. Most of the Bulgarian \framename{Departing} verbs take a PP or an AVP complement, with few exceptions, such as \textit{напускам} `leave’, which takes an NP.Obj complement.


\begin{table}
\caption{Syntactic expression of the \framename{Departing} FEs in Bulgarian} 
    \label{tab:4:departing-synt-bg}
\begin{tabular}{l ccccccccc}   
\lsptoprule
  & NP.Ext & NP.Obj & PP & AVP & NI & Clause & Other & Total\\ \midrule
\multicolumn{9}{l}{\textit{напускам\slash напусна} `leave’}\\*
\fename{Theme} & 39 &  &  &  &  &  &  & 39\\*
\fename{Source} &  & 38 &  &  & 1 &  &  & 39\\ 
 \midrule
\multicolumn{9}{l}{\textit{тръгвам\slash тръгна} `leave’}\\*
\fename{Theme} & 36 &  &  &  &  &  &  & 36\\*
\fename{Source} &  &  &  &  & 36 &  &  & 36\\*
\fename{Goal} &  &  & 10 & 1 &  &  &  & 11\\*
\fename{Direction} &  &  & 1 & 1 &  &  &  & 2\\
\midrule
\multicolumn{9}{l}{\textit{заминавам\slash замина} `depart’}\\*
\fename{Theme} & 40 &  &  &  &  &  &  & 40\\*
\fename{Source} &  &  & 1 & 1 & 38 &  &  & 40\\*
\fename{Path} &  &  & 1 &  &  &  &  & 1\\*
\fename{Goal} &  &  & 20 &  &  &  &  & 20\\*
\fename{Distance} &  &  & 1 &  &  &  &  & 1\\ 
 \midrule
\multicolumn{9}{l}{\textit{излизам\slash изляза} `exit’}\\*
\fename{Theme} & 39 &  &  &  &  &  &  & 39\\*
\fename{Source} &  &  & 15 & 1 & 23 &  &  & 39\\* 
\fename{Goal} &  &  & 6 & 3 &  & 2 &  & 11\\
 \lspbottomrule
\end{tabular}
\end{table}

\begin{table}
   \begin{tabularx}{\textwidth}{ lrQ } 
   \lsptoprule
   Pattern & \# & Verbs \\ \midrule
{[NP.Ext]}$_{\feinsub{Thm}}$ {[\_]}$_{\feinsub{Src-DNI}}$ & 53 & \textit{заминавам\slash замина, излизам\slash изляза, напускам\slash напусна, тръгвам\slash тръгна}\\ 
{[NP.Ext]}$_{\feinsub{Thm}}$ {[NP]}$_{\feinsub{Src}}$ & 38 & \textit{напускам\slash напусна}\\ 
{[NP.Ext]}$_{\feinsub{Thm}}$ {[PP]}$_{\feinsub{Goal}}$ {[\_]}$_{\feinsub{Src-DNI}}$ & 35 & \textit{заминавам\slash замина, излизам\slash изляза, тръгвам\slash тръгна}\\ 
{[NP.Ext]}$_{\feinsub{Thm}}$ {[PP]}$_{\feinsub{Src}}$ & 15 & \textit{заминавам\slash замина, излизам\slash изляза}\\ 
{[NP.Ext]}$_{\feinsub{Thm}}$ {[AVP]}$_{\feinsub{Goal}}$ {[\_]}$_{\feinsub{Src-DNI}}$ & 4 & \textit{излизам\slash изляза, тръгвам\slash тръгна}\\ 
%{[NP.Ext]}$_{\feinsub{Thm}}$ {[Clause]}$_{\feinsub{Goal}}$ {[\_]}$_{\feinsub{Src-DNI}}$ & 2 & \textit{излизам\slash изляза}\\ \midrule
%{[NP.Ext]}$_{\feinsub{Thm}}$ {[AVP]}$_{\feinsub{Src}}$ & 2 & \textit{заминавам\slash замина, излизам\slash изляза}\\ \midrule
%{[NP.Ext]}$_{\feinsub{Thm}}$ {[PP]}$_{\feinsub{Path}}$ {[\_]}$_{\feinsub{Src-DNI}}$ & 1 & \textit{заминавам\slash замина}\\ \midrule
%{[NP.Ext]}$_{\feinsub{Thm}}$ {[PP]}$_{\feinsub{Goal}}$ {[PP]}$_{\feinsub{Src}}$ & 1 & \textit{излизам\slash изляза}\\ \midrule
%{[NP.Ext]}$_{\feinsub{Thm}}$ {[AVP]}$_{\feinsub{Dir}}$ {[\_]}$_{\feinsub{Src-DNI}}$ & 1 & \textit{тръгвам\slash тръгна}\\ \midrule
%{[NP.Ext]}$_{\feinsub{Thm}}$ {[PP]}$_{\feinsub{Dir}}$ {[\_]}$_{\feinsub{Src-DNI}}$ & 1 & \textit{тръгвам\slash тръгна}\\ \midrule
%{[NP.Ext]}$_{\feinsub{Thm}}$ {[PP]}$_{\feinsub{Distance}}$ {[\_]}$_{\feinsub{Src-DNI}}$ & 1 & \textit{заминавам\slash замина}\\ \midrule
\lspbottomrule
\end{tabularx}
    \caption{FrameNet valence patterns of \framename{Departing} verbs in Bulgarian}
    \label{tab:4:departing-valence-bg}
\end{table}

The data in \tabref{tab:4:departing-valence-bg} support the observations that apart from NP \fename{Sources} (Example \ref{ex:19:c}), the profiled element of the \framename{Departing} frame
 (Example \ref{ex:19:a}) tends to be left out, i.e. it is usually interpreted from the previous or the general context (Example \ref{ex:19:b}).

 In addition, while the peripheral frame element \fename{Source} in the \framename{Arriving} frame is rarely expressed (in fact not present in the data), the peripheral frame element \fename{Goal} in the \framename{Departing} frame (Example \ref{ex:19:d}) was found to be quite frequently expressed and was thus annotated in the Bulgarian examples: in fact, it has as  many occurrences as the profiled FE \fename{Source} (\tabref{tab:4:departing-valence-bg}).

\begin{exe}
\ex  \label{ex:19}
\begin{xlist}
\ex  \label{ex:19:a}
\gll [\textit{\ }]$_{\feinsub{Thm}}$ \textit{Не} \textit{\textbf{ИЗЛИЗАЙ}} [\textit{от} \textit{къщи}]$_{\feinsub{Src}}$. \\
{} Not go-out-IMP.2SG out-of house-DEF.\\
\glt `Don't leave the house.'
\ex\label{ex:19:b}
\gll [\textit{Той}]$_{\feinsub{Thm}}$ \textit{\textbf{ЗАМИНА}} [\ ]$_{\feinsub{Src}}$ \textit{на} \textit{сутринта}. \\
He leave-PST.3SG {} on morning-DEF. \\
\glt `He departed on the following morning.'
\ex\label{ex:19:c}
\gll [\textit{Тя}]$_{\feinsub{Thm}}$ \textit{\textbf{НАПУСНА}} [града]$_{\feinsub{Src}}$ \textit{завинаги}. \\
She leave-PST.3SG city-DEF {for good}. \\
\glt `She left the city for good.'
\ex \label{ex:19:d}
\gll \textit{След} \textit{завършването} [\textit{той}]$_{\feinsub{Thm}}$ \textit{\textbf{ЗАМИНА}} [\ ]$_{\feinsub{Src}}$ [\textit{за} \textit{Париж}]$_{\feinsub{Goal}}$. \\
 After graduating, he leave-PST.3SG {} for Paris. \\
\glt `After his graduation he left for Paris.'
\end{xlist}
\end{exe}

Another fact that emerged from the data is that, even though \fename{Direction} and \fename{Distance} are not specified in the \framename{Arriving} and the \framename{Departing} frame, there are examples that suggest that these FEs are part of the description of the two semantic frames, even if with a peripheral status (Example \ref{ex:20} and Example \ref{ex:21}, respectively).


\begin{exe}
\ex \label{ex:20}
\begin{xlist}
\ex \label{ex:20:а}
\gll [\textit{Корабът}]$_{\feinsub{Thm}}$ \textit{\textbf{ЗАМИНАВА}} [\ ]$_{\feinsub{Src}}$ [\textit{на} \textit{юг}]$_{\feinsub{Dir}}$. \\
Ship-DEF leave-PRS.3SG  {} to south. \\
\glt `The ship leaves south.'
\ex \label{ex:20:b}
\gll [\textit{Тя}]$_{\feinsub{Thm}}$ \textit{\textbf{ЗАМИНА}} [\textit{на 3000 км}]$_{\feinsub{Dist}}$ [\textit{от} \textit{дома}]$_{\feinsub{Src}}$. \\
 She leave-PST.3SG to 3000 km from home. \\
\glt `She went (to live) 3,000 km away from home.'
\end{xlist}
\end{exe}

\begin{exe}
\ex  \label{ex:21}
\begin{xlist}
\ex  \label{ex:21:а}
\gll [\textit{Те}]$_{\feinsub{Thm}}$ \textit{\textbf{ПРИСТИГАТ}} [\ ]$_{\feinsub{Goal}}$ [\textit{от} \textit{юг}]$_{\feinsub{Dir}}$. \\
They arrive-PRS.3PL {} from south. \\
\glt `They arrive from the south.'
\ex \label{ex:21:b}
\gll [\textit{\ }]$_{\feinsub{Thm}}$ \textit{\textbf{ИДВАХА}} [\textit{тук}]$_{\feinsub{Goal}}$ [\textit{отдалече}]$_{\feinsub{Dist}}$. \\
They come-PST.3PL here {from  far away}. \\
\glt `They came here from far away.'
\end{xlist}
\end{exe}


\subsubsection{Traversing}
\subsubsection{Semantic description of the \framename{Traversing} frame}

Traversing represents the complex situation of the motion of a \fename{Theme} with respect to the different locations constituting the route.

\begin{description}[font=\normalfont]
\item[Definition of the frame \framename{Traversing}:] A \fename{Theme} changes location with respect to a salient place, which can be expressed by a \fename{Source}, \fename{Path}, \fename{Goal}, \fename{Area}, \fename{Direction}, \fename{Path\_shape}, or \fename{Distance}.
\end{description}

The frame profiles the middle section of the trajectory of motion of a moving entity, i.e. the \fename{Path}. Its core FEs include the \fename{Path} itself, as well as  elements that represent either an alternative expression of the idea of space covered by the moving entity (such as \fename{Area}) or a characteristic feature of the \fename{Path}. These features may include: \fename{Direction}, which adds the dimension of spatial orientation to the non-directional \fename{Path}; \fename{Distance}, i.e. the length or extent of the trajectory between the starting and the end point; \fename{Path}\_shape -- the form of the \fename{Path}. All of the core FEs that describe the \framename{Traversing} frame are inherited from the most abstract motion frame \framename{Motion\_scenario} which is perspectivised by \framename{Traversing}. 

\subsubsection{Verbs evoking the \framename{Traversing} frame}

As with \framename{Arriving} and \framename{Departing}, there are just a small number of mainly non-derived verbs that evoke the frame: \textit{ascend}, \textit{circle}, \textit{crisscross}, \textit{cross}, \textit{descend}, \textit{hop}, \textit{jump}, \textit{leap}, \textit{mount}, \textit{pass}, \textit{skirt}, \textit{traverse}.


\subsubsection{Syntactic realisation of the frame elements in the \framename{Traversing} frame}

\tabref{tab:4:traversing-synt} illustrates the syntactic expression for a selection of \framename{Traversing} verbs.  The \fename{Theme} is projected as the subject. Among the motion-related FEs, usually it is the profiled \fename{Path} that is expressed syntactically; its favoured realisation is either as a direct object NP, e.g. \textit{ascend}, \textit{cross}, \textit{descend}, \textit{skirt}, or as a prepositional (or adverbial) phrase, e.g. \textit{pass} and \textit{leap}. It can also be left unexpressed (DNI), although the number of unexpressed \fename{Paths} is much fewer than that of the profiled FEs of the \framename{Arriving} and the \framename{Departing} frame.

When the \fename{Area} is expressed, it may also take the place of the direct object: for most of the verbs, these are single occurrences, except for \textit{circle} and \textit{crisscross}: their semantics are consistent with motion along an irregular trajectory over an extended region, which predetermines their preference for the \fename{Area} over the \fename{Path}.  

\fename{Sources} and \fename{Goals} are expressed as prepositional or adverbial phrases; \fename{Direction}, \fename{Distance}, sometimes \fename{Area} (when not an object), although represented by just a few examples, are realised likewise. A small number of exceptions is found with \textit{descend}, where some \fename{Distance}s and \fename{Direction}s are annotated as NP objects (e.g. \textit{descended 300 m}).
\largerpage
\begin{table}
\caption{Syntactic expression of the \framename{Traversing} FEs in FrameNet} \label{tab:4:traversing-synt}\footnotesize
\begin{tabular}{l cccccccc}
 \lsptoprule
  & NP.Ext & NP.Obj & PP & AVP & NI & Clause & Other & Total\\ \midrule
\multicolumn{9}{l}{\textit{traverse} } \\*
\fename{Theme} & 13  &  & 2  &  & 2  &  &  & 17\\*
\fename{Area} &  & 1  &  &  &  &  &  & 1\\*
\fename{Source} &  &  & 2  &  &  &  &  & 2\\*
\fename{Path} & 4  & 4  & 3  &  & 4  & 1  &  & 16\\*
\fename{Goal} &  &  & 6  &  &  &  &  & 6\\*
\fename{Path\_shape} &  &  &  &  & 17  &  &  & 17\\*
\fename{Distance} &  &  &  &  &  &  & 1 & 1\\
%\midrule
%\multicolumn{9}{l}{\textit{skirt} } \\
%\fename{Theme} & 12  &  &  &  &  &  &  & 12\\
%\fename{Path} &  & 8  & 3  &  &  &  & 1 & 12\\
%\fename{Direction} &  &  &  &  &  &  & 1 & 1\\
%%\fename{Goal} &  &  & 2  &  &  &  &  & 2\\
%\midrule
%\multicolumn{9}{l}{\textit{mount} } \\
%\fename{Theme} & 5  &  &  &  &  &  &  & 5\\
%\fename{Path} &  & 3  &  &  & 2  &  &  & 5\\
%\fename{Goal} &  & 2  &  &  &  &  &  & 2\\
\midrule
\multicolumn{9}{l}{\textit{descend} } \\*
\fename{Theme} & 35  &  &  &  & 1  &  &  & 36\\*
\fename{Source} &  &  & 5  &  &  &  &  & 5\\*
\fename{Path} & 1  & 17  & 8  &  & 3  &  &  & 29\\*
\fename{Goal} &  &  & 9  &  &  &  &  & 9\\*
\fename{Path\_shape} &  &  & 1  &  &  &  &  & 1\\*
\fename{Direction} &  &  & 1  & 1  &  &  &  & 2\\*
\fename{Distance} &  & 1  &  & 1  &  &  & 1 & 3\\
\midrule
\multicolumn{9}{l}{\textit{cross} } \\*
\fename{Theme} & 53  &  & 2  &  & 2  &  &  & 57\\*
\fename{Area} &  & 1  &  &  &  &  &  & 1\\*
\fename{Source} &  &  & 4  &  &  &  &  & 4\\*
\fename{Path} & 4  & 26  & 6  & 4  & 16  &  &  & 56\\*
\fename{Goal} &  &  & 16  &  &  &  & 1 & 17\\*
\fename{Direction} &  &  & 4  & 1  &  &  &  & 5\\
\midrule
%\multicolumn{9}{l}{\textit{crisscross} } \\
%\fename{Theme} & 13  &  &  &  &  &  &  & 13\\
%\fename{Area} &  & 13  &  &  &  &  &  & 13\\
%\fename{Path\_shape} &  &  &  &  & 13  &  &  & 13\\
%\midrule
%\multicolumn{9}{l}{\textit{ascend} } \\
%\fename{Theme} & 15  &  &  &  &  &  &  & 15\\
%\fename{Path} &  & 10  & 1  &  & 2  &  &  & 13\\
%\fename{Goal} &  &  & 3  &  & 1  &  &  & 4\\
%\fename{Source} &  &  & 1  &  &  &  & 1 & 2\\
%\midrule
%\multicolumn{9}{l}{\textit{jump} } \\
%\fename{Theme} & 5  &  &  &  &  &  &  & 5\\
%\fename{Area} &  & 1  &  &  &  &  &  & 1\\
%\fename{Path} &  &  & 3  &  &  &  &  & 3\\
%\fename{Source} &  &  & 1  &  &  &  &  & 1\\
%\midrule
\multicolumn{9}{l}{\textit{pass} } \\*
\fename{Theme} & 20  &  &  &  &  &  &  & 20\\*
\fename{Area} &  & 1  &  &  &  &  &  & 1\\*
\fename{Source} &  &  & 1  &  &  &  &  & 1\\*
\fename{Path} &  & 3  & 14  &  &  &  &  & 17\\*
\fename{Direction} &  &  & 1  &  &  &  &  & 1\\
\midrule
\multicolumn{9}{l}{\textit{circle} } \\*
\fename{Theme} & 22  &  &  &  &  &  &  & 22\\*
\fename{Area} &  & 9  &  &  & 6  &  &  & 15\\*
\fename{Path} &  & 1  & 3  &  &  &  &  & 4\\*
\fename{Direction} &  &  & 1  & 1  &  &  &  & 2\\
\lspbottomrule
%\multicolumn{9}{l}{\textit{hop} } \\
%\fename{Theme} & 6  &  &  &  &  &  &  & 6\\
%\fename{Goal} &  &  & 3  &  &  &  &  & 3\\
%\fename{Path} &  &  & 3  &  &  &  &  & 3\\
%\fename{Source} &  &  & 1  &  &  &  &  & 1\\
%\midrule
%\multicolumn{9}{l}{\textit{leap} } \\
%\fename{Theme} & 12  &  &  &  &  &  &  & 12\\
%\fename{Path} &  &  & 10  &  &  &  &  & 10\\
%\fename{Goal} &  &  & 1  &  &  &  &  & 1\\
%\fename{Distance} &  &  & 1  &  &  &  &  & 1\\
%\midrule
\end{tabular}
\end{table}

\fename{Path\_shapes} are almost always implied in the semantics of the verbs but are rarely expressed (as PPs/AVPs).

\subsubsection{FrameNet valence patterns}

The most frequent valence patterns (\tabref{tab:4:traversing-valence-framenet}) show in even more explicit terms that across the different verbs evoking the frame, the non-overt realisation of the \fename{Path} is much rarer, especially when compared with the profiled elements of \framename{Traversing}'s subframes, while NPs and PPs are both well-represented, with variations across the different verbs. Another fact worth noting is that out of the remaining FEs, the \fename{Goal} is the preferred one to be expressed.

\begin{table}
   \begin{tabularx}{\textwidth}{ lrQ }
   \lsptoprule
   Pattern & \# & Verbs \\ \midrule
   {[NP.Ext]}$_{\feinsub{Thm}}$ {[NP.Obj]}$_{\feinsub{Path}}$  & 48 & \textit{descend, ascend, skirt, pass, cross, circle, mount}\\ 
{[NP.Ext]}$_{\feinsub{Thm}}$ {[PP]}$_{\feinsub{Path}}$  & 40 & \textit{descend, ascend, skirt, pass, cross, hop, leap, circle, jump}\\ 
{[NP.Ext]}$_{\feinsub{Thm}}$ {[PP]}$_{\feinsub{Goal}}$ {[NP.Obj]}$_{\feinsub{Path}}$  & 14 & \textit{descend, ascend, skirt, cross}\\ 
{[NP.Ext]}$_{\feinsub{Thm}}$ {[NP.Obj]}$_{\feinsub{Area}}$ {[\_]}$_{\feinsub{Path\_shape-INC}}$  & 13 & \textit{crisscross}\\ 
{[NP.Ext]}$_{\feinsub{Thm}}$ {[\_]}$_{\feinsub{Path-DNI}}$  & 11 & \textit{descend, ascend, cross}\\ 
{[NP.Ext]}$_{\feinsub{Thm}}$ {[NP.Obj]}$_{\feinsub{Area}}$  & 10 & \textit{pass, circle, jump}\\ 
{[NP.Ext]}$_{\feinsub{Thm}}$ {[PP]}$_{\feinsub{Src}}$  & 6 & \textit{descend, hop, jump}\\ 
{[NP.Ext]}$_{\feinsub{Thm}}$ {[\_]}$_{\feinsub{Area-DNI}}$  & 6 & \textit{circle}\\ 
{[NP.Ext]}$_{\feinsub{Thm}}$ {[PP]}$_{\feinsub{Goal}}$ {[\_]}$_{\feinsub{Path-DNI}}$  & 5 & \textit{cross}\\ \lspbottomrule
%{[NP.Ext]}$_{\feinsub{Thm}}$  & 5 & \textit{descend, pass, cross, circle}\\ \midrule
%{[NP.Ext]}$_{\feinsub{Thm}}$ {[PP]}$_{\feinsub{Goal}}$  & 4 & \textit{descend, hop, leap}\\ \midrule
%{[NP.Ext]}$_{\feinsub{Thm}}$ {[PP]}$_{\feinsub{Path}}$ {[PP]}$_{\feinsub{Src}}$  & 3 & \textit{descend, pass, cross}\\ \midrule
\end{tabularx}
    \caption{FrameNet valence patterns of \framename{Traversing} verbs}
    \label{tab:4:traversing-valence-framenet}
\end{table} 

\subsubsection{Syntactic realisation of \framename{Traversing} verbs in Bulgarian}

The central part of the Bulgarian verbs evoking the \framename{Traversing} frame includes predicates such as \textit{минавам} `pass’, \textit{кръстосвам} `crisscross’, \textit{качвам се}, \textit{качвам} `ascend’, \textit{слизам}, \textit{спускам се} `descend’, as well as several verbs produced through derivation, though not necessarily transparent in the contemporary language: \textit{изкачвам се}, \textit{изкачвам} `ascend’, \textit{обикалям}, \textit{заобикалям} `circle’, `skirt’, \textit{пресичам}, \textit{прекосявам} `cross’, `traverse’, \textit{преминавам} `pass’, `pass over’.

In addition, there are a lot of Bulgarian verbs that represent lexicalisations of \fename{Path}-profiling formed by means of prefixation primarily from manner of motion verbs, which will be discussed in the next subsection along with similarly formed \fename{Goal}-profiling and \fename{Source}-profiling verbs. 

\tabref{tab:4:traversing-synt-bg} illustrates the syntactic realisation of several Bulgarian verbs evoking the frame \framename{Traversing}. It can be noted that, like in English, for different verbs the preferred expression of the \fename{Path} may either be a direct object NP, e.g. \textit{пресичам}, \textit{прекосявам} `cross’, `traverse’ or a prepositional (or adverbial) phrase (Example \ref{ex:22:а}), e.g. \textit{пресека} `cross’. \tabref{tab:4:traversing-valence-bg} shows that some of the verbs that may be used both transitively and intransitively, favour the transitive (NP.Obj) realisation. The profiled element tends to be syntactically expressed, rather than left non-overt.

\begin{table}
\caption{Syntactic expression of the \framename{Traversing} FEs in Bulgarian} 
\label{tab:4:traversing-synt-bg}
\begin{tabular}{l cccccccc}   
  \lsptoprule
  & NP.Ext & NP.Obj & PP & AVP & NI & Clause & Other & Total\\ \midrule
\multicolumn{9}{l}{\textit{пресичам\slash пресека} `traverse’ }\\*
\fename{Theme} & 39 &  &  &  &  &  &  & 39\\* 
\fename{Path} &  & 33 & 5 &  & 1 &  &  & 39\\*
\fename{Goal} &  &  & 2 &  &  &  &  & 2\\
 \midrule
\multicolumn{9}{l}{\textit{изкачвам\slash изкача} `ascend’ }\\*
\fename{Theme} & 12 &  &  &  &  &  &  & 12\\*
\fename{Path} &  & 12 &  &  &  &  &  & 12\\*
\fename{Goal} &  &  & 4 &  &  &  &  & 4\\ 
 \midrule
\multicolumn{9}{l}{\textit{прекосявам\slash прекося} `cross’ }\\*
\fename{Theme} & 40 &  &  &  &  &  &  & 40\\* 
\fename{Path} &  & 36 & 4 &  &  &  &  & 40\\
 \midrule
\multicolumn{9}{l}{\textit{преминавам\slash премина} `pass’ }\\*
\fename{Theme} & 15 &  &  &  &  &  &  & 15\\* 
\fename{Path} &  & 4 & 11 &  &  &  &  & 15\\
 \midrule
\multicolumn{9}{l}{\textit{изкачвам се\slash изкача се} `ascend’ }\\*
\fename{Theme} & 9 &  &  &  &  &  &  & 9\\*
\fename{Path} &  &  & 4 &  & 5 &  &  & 9\\*
\fename{Goal} &  &  & 6 &  &  &  &  & 6\\ 
 \lspbottomrule
\end{tabular}
\end{table}

%\newpage

The verbs \textit{качвам}, \textit{изкачвам} `ascend’ are always transitive (Example \ref{ex:22:b}), while \textit{качвам се}, \textit{изкачвам се} `ascend’ and \textit{спускам се}, \textit{слизам} `descend’ are always intransitive (Examples \ref{ex:22:c}, \ref{ex:22:d}).

\begin{exe}
\ex  \label{ex:22}
\begin{xlist}
\ex  \label{ex:22:а}
\gll [\textit{Те}]$_{\feinsub{Thm}}$ \textit{\textbf{ПРЕСЯКОХА}} [\textit{(през)} \textit{двора}]$_{\feinsub{Path}}$. \\
They cross-PST.3PL (through) yard-DEF. \\
\glt `The boys crossed the yard.'
\ex\label{ex:22:b}
\gll [\textit{Те}]$_{\feinsub{Thm}}$ \textit{\textbf{ИЗКАЧИХА}} [\textit{планината}]$_{\feinsub{Path}}$. \\
They climb-PST.3PL mountain-DEF. \\
\glt `They climbed the mountain.'
\ex \label{ex:22:c}
\gll [\textit{Те}]$_{\feinsub{Thm}}$ {\textit{\textbf{СЕ}} \textit{\textbf{ИЗКАЧИХА}}} [\textit{по} \textit{планината}]$_{\feinsub{Path}}$. \\
They climb-REFL.PST.3PL on mountain-DEF. \\
\glt `They climbed the mountain.'
\ex \label{ex:22:d}
\gll [\textit{Те}]$_{\feinsub{Thm}}$ {\textit{\textbf{СЛИЗАТ}}} [\textit{по} \textit{стълбите}]$_{\feinsub{Path}}$. \\
They climb-PRS.3PL down stairs-DEF. \\
\glt `They descended the stairs.'
\end{xlist}
\end{exe}

%\begin{exe}
%\ex \label{ex:23}
%\begin{xlist}
%\ex \label{ex:23:а}
%\gll [\textit{Те}]$_{\feinsub{Thm}}$ \textit{прекосиха} [\textit{тревата}]$_{\feinsub{Path}}$. \\
%[They]$_{\feinsub{Thm}}$ crossed [meadow-DEF]$_{\feinsub{Path}}$. \\
%\glt `They crossed the meadow.'
%\ex \label{ex:23:b}
%\gll [\textit{\_}]$_{\feinsub{Thm}}$ \textit{Прекосих} [\textit{Колумбия} \textit{надлъж} % \textit{и} \textit{нашир}]$_{\feinsub{Path}}$. \\
%[\ ]$_{\feinsub{Thm}}$ traversed [Columbia far and wide]$_{\feinsub{Path}}$. \\
%\glt `They traversed Columbia far and wide.'
%\end{xlist}
%\end{exe}


%In reality the two FEs are difficult to distinguish in the context except with verbs whose semantics provide additional information. Such verbs are, for instance, descend and circle. Descend incorporates the FE \fename{Direction}, so it implies directional motion along a route, i.e. the FEs it takes as direct objects are usually \fename{Path}s (or other aspects of such motion, see the example of \fename{Distance} as a direct object NP above); circle involves circular motion around an expanse, which is better construed as \fename{Area}.

Although on a very small scale due to the size of the sample, the valence patterns show that \fename{Goals} are also realised syntactically (\tabref{tab:4:traversing-valence-bg} and Example \ref{ex:24:a}). \fename{Sources} (Example \ref{ex:24:c}) and \fename{Directions} (Example \ref{ex:24:b}) as well as combinations of motion-related FEs (Example \ref{ex:24:c}) are also attested as individual occurrences in the data.

\begin{table} 
   \begin{tabularx}{\textwidth}{ lrQ } 
   \lspbottomrule
   Pattern & \# & Verbs \\ \midrule
{[NP.Ext]}$_{\feinsub{Thm}}$ {[NP.Obj]}$_{\feinsub{Path}}$ & 79 & \textit{прекосявам\slash прекося, пресичам\slash пресека, преминавам\slash премина, изкачвам\slash изкача}\\ 
{[NP.Ext]}$_{\feinsub{Thm}}$ {[PP]}$_{\feinsub{Path}}$ & 23 & \textit{прекосявам\slash прекося, пресичам\slash пресека, преминавам\slash премина, изкачвам се\slash изкача се}\\ 
{[NP.Ext]}$_{\feinsub{Thm}}$ {[NP.Obj]}$_{\feinsub{Path}}$ {[PP]}$_{\feinsub{Goal}}$ & 6 & \textit{пресичам\slash пресека, изкачвам\slash изкача}\\ 
{[NP.Ext]}$_{\feinsub{Thm}}$ {[PP]}$_{\feinsub{Goal}}$ {[\_]}$_{\feinsub{Path-DNI}}$ & 5 & \textit{изкачвам се\slash изкача се}\\ 
%{[NP.Ext]}$_{\feinsub{Thm}}$ {[PP]}$_{\feinsub{Goal}}$ {[PP]}$_{\feinsub{Path}}$ & 1 & \textit{изкача се}\\ 
%{[NP.Ext]}$_{\feinsub{Thm}}$ {[\_]}$_{\feinsub{Path-DNI}}$ & 1 & \textit{пресичам\slash пресека}\\ 
\lspbottomrule
\end{tabularx}
    \caption{FrameNet valence patterns of \framename{Traversing} verbs in Bulgarian}
    \label{tab:4:traversing-valence-bg}
\end{table}
 
\begin{exe}
\ex \label{ex:24}
\begin{xlist}
\ex \label{ex:24:a}
\gll [\textit{Тя}]$_{\feinsub{Thm}}$ \textit{\textbf{ПРЕКОСИ}} [\textit{полето}]$_{\feinsub{Path}}$ [\textit{до} \textit{крепостта}]$_{\feinsub{Goal}}$. \\
She cross-PST.3SG field-DEF to fortress-DEF. \\
\glt `She crossed the field towards the fortress.'
\ex  \label{ex:24:b}
\gll [\textit{Той}]$_{\feinsub{Thm}}$ \textit{\textbf{ПРЕКОСИ}} [\textit{залата}]$_{\feinsub{Path}}$ [\textit{по} \textit{посока} \textit{на}  \textit{вратата}]$_{\feinsub{Dir}}$. \\
He cross-PST.3SG hall-DEF in direction of door-DEF. \\
\glt `He crossed the hall towards the door.'
\ex  \label{ex:24:c}
\gll [\textit{Тя}]$_{\feinsub{Thm}}$ \textit{\textbf{ПРЕСЕЧЕ}} [\textit{моста}]$_{\feinsub{Path}}$ [\textit{от} \textit{мидълсекския}]$_{\feinsub{Src}}$ [\textit{към} \textit{сърейския} \textit{бряг}]$_{\feinsub{Goal}}$. \\
Тя traverse-PST.3SG bridge-DEF from Middlesex-DEF to Surrey-DEF shore. \\
\glt `She traversed the bridge from the Middlesex to the Surrey shore.'
\end{xlist}
\end{exe}

The verb \textit{слизам} `descend’ can also co-occur with \fename{Distances} that may be expressed as measurement NPs (Example \ref{ex:25}).
%may be annotated as NP objects (on the example of the English descended 300 m). However, as spuskam se (always intransitive as transitivity is blocked by the reflexive particle se \cite{}) shows such NPs cannot be direct objects, but rather measurement NPs; even more they cannot passivise. 

\begin{exe}
\ex  \label{ex:25}
%\begin{xlist}
%\ex  \label{ex:25:а}
\gll [\textit{Те}]$_{\feinsub{Thm}}$ {\textit{\textbf{СЕ}} \textit{\textbf{СПУСКАТ}}} [\textit{300} \textit{м}]$_{\feinsub{Dist}}$. \\
They {climb-PRS.3PL down} 300 m. \\
\glt `They descend 300 m.'
%\end{xlist}
\end{exe}


\subsubsection{Derivation of directional motion verbs}\label{prefixes}

It has been well-established in the literature that part of the verbal prefixes in the Slavic languages yield (resultative) prefixed verbs when attached to unprefixed (simplex) verbs (\cite{Beavers2010,Pantcheva-2007,Pantcheva2007,Pantcheva2011, Palmer2009,SpencerZaretskaya1998,Svenonius2005}, among many others), see also \citet[178--184]{VanValinLaPolla1997} for other languages. Regardless of the theoretical framework adopted and the specifics of the treatment of such verbs, the mechanism involves a verb with a simple internal (event, lexical semantic, logical) structure to which a prefix is attached so as to add a resultative subevent, thus producing a verb describing a more complex eventuality. 

A typical example in the domain of motion is the prefixation of manner of motion verbs using directional prefixes, which, depending on the prefix, leads to the formation of \fename{Goal}-profiled, \fename{Source}-profiled or \fename{Path}-profiled predicates. As noted earlier, besides the verbs discussed in the previous sections, most of which are underived verbs with a primary directional motion meaning, there are a number of prefixed predicates derived mainly from simplex manner of motion verbs (belonging themselves to frames such as \framename{Motion}, \framename{Self\_motion}, \framename{Fluidic\_motion}, among others), which also evoke the frames \framename{Traversing}, \framename{Arriving} and \framename{Departing}, and possibly other frames profiling the elements of the route of motion. 

\begin{table}
\footnotesize
\begin{tabularx}{\textwidth}{ p{\widthof{\textbf{Self\_motion }}} QQQ} 
\lsptoprule
\framename{Self\_motion} & \fename{Source}-profiled & \fename{Goal}-profiled & \fename{Path}-profiled \\
\midrule
\textit{летя} `fly’ & \textit{отлитам} `fly away’, \textit{излитам}, \textit{политам} `fly off’, `take off’ & \textit{долитам} `fly (up) to’, \textit{влитам} `fly into’ & \textit{прелитам} `fly over’ \\
\midrule
\textit{хвърча} `fly’ & \textit{отхвърчавам} `fly away’, \textit{изхвърчавам} `fly off’ & \textit{дохвърчавам} `fly (up) to’ & \textit{прехвърчавам} `fly over’ \\
\midrule
\textit{бягам} `run’ & \textit{избягвам} `run away’ & \textit{добягвам} `run (up) to’ (dialect) & \textit{пробягвам}, \textit{пребягвам} `run’, `cover distance by running'

\textit{пребягвам} `run across’ \\
\midrule
\textit{тичам} `run’ & \textit{изтичвам} `run out of’ & \textit{дотичвам} `run (up) to’ & \textit{претичвам} `cover distance by running' \\
\midrule
\textit{пълзя} `crawl’ & \textit{изпълзявам} `crawl out’ & \textit{допълзявам} `crawl (up) to’

\textit{пропълзявам} `crawl in’, `crawl onto’

\textit{впълзявам} `crawl into’ & \textit{препълзявам}, \textit{пропълзявам} `crawl across’, \textit{пропълзявам} `cover distance by crawling' \\
\midrule
\textit{скачам} `jump’ & \textit{изскачам} `jump out’ & \textit{доскачам} `jump (up) to’ & \textit{прескачам} `jump’, `pass over’ \\
\midrule
\textit{плувам} `swim’ & \textit{изплувам} `swim up’, `swim to the surface’ & \textit{доплувам} `swim (up) to’

\textit{вплувам} `swim into’ & \textit{преплувам} `swim across’

\textit{проплувам} `cover distance by swimming' \\
\midrule
\textit{нижа се} `file’ & \textit{изнизвам се} `file out’ &  &  \\
\midrule
\textit{газя} `wade’ &  & \textit{догазвам} `wade (up) to’ & \textit{изгазвам}, \textit{прегазвам} `pass through some substance by wading'

\textit{изгазвам} `cross by wading' \\
\midrule
\textit{танцувам} `dance’ &  & \textit{дотанцувам} `dance (up) to’ &  \\
\midrule
\textit{клатушкам се} `totter’ &  & \textit{доклатушквам се} `totter (up) to’ &  \\
\midrule
\textit{куцам}, \textit{куцукам} `limp’ &  & \textit{докуцвам}, \textit{докуцуквам} `limp (up) to’ &  \\
\lspbottomrule
\end{tabularx}
\caption{Prefixal derivation of directed motion verbs from manner of motion verbs in Bulgarian} 
\label{tab:4:big-table-of-prefixed-verbs}
\end{table}

\tabref{tab:4:big-table-of-prefixed-verbs} shows the productivity of this pattern. The inventory of verbs evoking semantic frames profiling different elements of the route, is much richer than in English, where similar meanings may be encoded either by manner of motion verbs which have developed a more complex event structure and meaning (Example \ref{ex:26:a}) or by means of certain syntactic constructions (Example \ref{ex:26:b}). %The consequence for the FrameNet organisation is that senses such as fly the Atlantic should be accommodated in one way or another. The Bulgarian verbs without correspondences in English which evoke the respective frames need to be included.

\begin{exe}
\ex  \label{ex:26}
\begin{xlist}
\ex  \label{ex:26:a}
[\textit{He}]$_{\feinsub{SMov}}$ \textit{was the first to \textbf{FLY}} [\textit{the Atlantic}]$_{\feinsub{Path}}$.
\ex  \label{ex:26:b}
[\textit{He}]$_{\feinsub{SMov}}$ \textit{\textbf{LIMPED}} [\textit{to the store}]$_{\feinsub{Goal}}$.
\end{xlist}
\end{exe}

\section{Conclusions}

%I discussed the principles of the internal organisation of the lexis of motion and how these principles project into a system of related frames. This was followed by a more detailed analysis of the definition of several related frames in terms of how frame-to-frame relations are implemented and how inheritance and other frame relations yield certain configurations of FEs (partly) inherited from more general frames; the focus was on showing what specialisations too place in the more specific frames.

In this chapter particular attention has been paid to the expression of the FEs that define the elements of the route traversed (\fename{Source}, \fename{Goal}, \fename{Path}) or region covered (\fename{Area}) by the moving entity and prominent aspects of the route such as the \fename{Distance} it spans, the \fename{Direction} it takes or the form it has (\fename{Path\_shape}).

I showed and commented on the semantic specification, syntactic expression and valence patterns typical of manner of motion and directed motion verbs by analysing the examples in the FrameNet corpus and expanding the observations to Bulgarian examples. 

Manner of motion verbs tend to express the \fename{Path} over the \fename{Goal} and especially over the \fename{Source}, but the particular distribution of the various patterns varies across verbs. \fename{Path} is especially prominent where complex notions of motion or trajectory are involved. 
%They also show preference for the peripheral FE Manner.

The data corroborate the observations made in the literature, that all other things being equal, there is a bias for expressing \fename{Goals} over \fename{Sources}, a tendency which has been studied for many typologically distinct languages. In particular, if the verbs do not profile a particular aspect of the route, they tend to express \fename{Goals} over \fename{Sources}, the intuition being that motion through space involves getting to some place, even with manner of motion verbs, and that, in this respect, the end point of the motion is a more salient feature than the starting point. 

Across verbs that profile a particular aspect of the route, the profiled FE is the one that tends to be expressed, i.e. \fename{Source}-profiling verbs co-occur more frequently with \fename{Source} expressions than verbs that do not profile this FE, \fename{Goal}-profiling verbs co-occur with \fename{Goal} expressions. While these two aspects have been of primary interest in the linguistic literature, similar observations may be made for \fename{Path} and to a lesser extent for \fename{Area} (as the examples are fewer), judging from the data. 

\fename{Distances} and \fename{Directions} are rarely expressed and at least in some cases they show to be syntactically, as well as semantically dependent on the \fename{Path}, as they represent elaborations on certain aspects of it (deictic or geographical orientation or the length of the route covered). 

Other elements of the route may be expressed besides or instead of the profiled one. %\fename{Goal}-profiling (\framename{Arriving}) verbs and 
\fename{Source}-profiling (\framename{Departing}) verbs tend to realise \fename{Goals} or \fename{Paths}, but the preference for one over the other varies across verbs and the examples are not always definitive. \fename{Path}-profiling verbs tend to favour \fename{Goals} over \fename{Sources}. In addition, the following was observed in the Bulgarian data: peripheral \fename{Goals} may be expressed on a par with profiled \fename{Sources}. Each of these observations warrants further investigation, especially with respect to the frequency and means of expression (including the available inventories) of various FE combinations within and across verbs and frames.

While only marked in passing, the productivity of prefixal derivation as a mechanism of deriving directed motion verbs from other motion verbs, especially from manner verbs, in Bulgarian (and other Slavic and non-Slavic languages) points to the need for these verbs to be systematically addressed within the FrameNet structure. This may also result in the definition of frame-to-frame relations that account for this systematicity. 

\section*{Abbreviations}
\begin{multicols}{2}
\begin{tabbing}
MMMM \= Adverbial phrase\kill
AVP \> Adverbial phrase\\
CNI \> Constructional null \\ \> instantiation\\
DEF \> Definite form\\
\scshape Dir \> \fename{Direction}\\
\scshape Dist \> \fename{Distance}\\
DNI \> Definite null instantiation\\
FE \> frame element\\
IMPF \> Imperfective aspect\\
INDF \> Indefinite form\\
INI \> Indefinite null instantiation\\
LU \> Lexical unit\\
NEG \> Negative form\\
NP \> Noun phrase\\
NP.Ext \> Subject NP\\
NP.Obj \> Object NP\\
PL \> Plural\\
PP \> Prepositional phrase\\
PRS \> Present tense\\
PST \> Past tense\\
PWN \> Princeton WordNet\\
SG \> Singular\\
\scshape SMov \> \fename{Self\_mover}\\
\scshape Src \> \fename{Source}\\
\scshape Thm \> \fename{Theme}
\end{tabbing}
\end{multicols}

\section*{Acknowledgements}

This research is carried out as part of the project \emph{Enriching Semantic Network WordNet with Conceptual Frames} funded by the Bulgarian National Science Fund, Grant Agreement No. KP-06-H50/1 from 2020.

{\sloppy\printbibliography[heading=subbibliography,notkeyword=this]}
\end{document}


% ####################################################################

\setcounter{chapter}{4}
\chapter{Impact on Artificial Language Evolution\is{artificial language evolution} and Linguistic Theory}
\label{c:impact}
\label{c:impact-linguistics}

The time has now come to weave through the theoretical foundations and experimental results to reflect on the contributions of this book to artificial language evolution\is{artificial language evolution} and linguistics. Section \ref{s:impact} deals with the first part of this reflection by comparing the results of this book to those obtained in a recent study on case marking\is{case!case marking} in the Iterated Learning Model\is{Iterated Learning Model (ILM)}, one of the most widely adopted approaches in the field of artificial language evolution\is{artificial language evolution}. The comparison shows that the cognitive-functional\is{cognitive-functional approach} approach outperforms the Iterated Learning Model\is{Iterated Learning Model (ILM)} and that the work in this book is a significant step forwards in the field.

The other three sections of this chapter deal with the contributions of this book to linguistics. More specifically, I believe this book can have an impact in three domains. First of all, the formalism proposed in Chapter \ref{c:ar} is the first computational implementation of \is{event structure}argument structure ever in a construction grammar framework. In section \ref{s:comp-formalism}, I will compare it to an upcoming alternative in Sign-Based Construction Grammar\is{construction grammar!Sign-Based Construction Grammar}. A second contribution has to do with the structure of the linguistic inventor\is{linguistic inventory}y. The problem of systematicity\is{systematicity} and multi-level selection\is{selection}\is{multi-level selection} is new to linguistics and may have an impact on how we should conceive the constructicon. I will discuss this matter from the viewpoint of construction grammars and usage-based model\is{usage-based model}s of language in section \ref{s:comp-constructicon}. Finally, the experiments in this book and prior work in the field provide alternative evidence in recent debates in linguistic typology\is{linguistic typology} and grammaticalization\is{grammaticalization}. These debates concern the status of semantic map\is{semantic maps}s and thematic hier\is{thematic hierarchy}archies as universals of human cognition, and the appropriateness of reanalysis\is{reanalysis} as a mechanism for explaining grammaticalization\is{grammaticalization}. Section \ref{s:comp-semmaps} introduces the debates and illustrates how experiments on artificial language evolution\is{artificial language evolution} can propose a novel way of thinking about the issues at hand.

\section{Pushing the state-of-the-art}
\label{s:impact}

The significance of scientific research can only be fully appreciated by comparing it to other studies in its domain. In this section, I will illustrate how the work in this book advances the state-of-the-art in artificial language evolution\is{artificial language evolution} by comparing it to a recent study by \citet{moy06case} who investigated the emergence\is{emergence} of a case grammar\is{case!case grammar} in the Iterated Learning Model\is{Iterated Learning Model (ILM)} (ILM\is{Iterated Learning Model (ILM)}), at present one of the most widely adopted models in the field. The comparison reveals some fundamental problems with the ILM\is{Iterated Learning Model (ILM)} and shows that a cognitive-functional\is{cognitive-functional approach} approach is the most fruitful way for moving the experiments on artificial language evolution\is{artificial language evolution} towards greater complexity, expressiveness\is{expressiveness} and realism. In the following subsections, I will summarize four series of experiments reported by Moy and discuss how and why the work in this book yields better results. Finally, I will give a general overview of both approaches.

\subsection{Experiment 1: a primitive case system?}

The main objective of Moy's work is to expand the Iterated Learning Model\is{Iterated Learning Model (ILM)} as presented by \citet{kirby02learning} in order to study the emergence\is{emergence} of a case grammar\is{case!case grammar}. Kirby's original experiments investigated how a recursi\is{recursion}ve word-order syntax can emerge as a side-effect of the cultural transmission \is{transmission}of language from one generation to the next without the need for communicative pressures \citep[the so-called `function independence principle\is{function independence principle}', ][]{brighton05cultural}. Moy's first series of experiments are a replication of these simulations.
\\
\\
\noindent{\bfseries The experiment.} The experiment features a population\is{speech population} of two agents: one `adult' speaker and one `child' learner. The adult speaker has to produce a number of utterances that are observed by the learner. The adult speaker has an innovat\is{innovation}ion strategy which allows him to invent a random holistic\is{holistic versus compositional languages} word for each new meaning if he does not know a word yet. The child learner is equipped with a Universal Grammar\is{Universal Grammar} in the form of an inducti\is{induction}on algorithm and will try to induce as much grammatical rules as possible. The child will thus overgeneraliz\is{generalization}e the input provided by the speaker which causes language change\is{language change} as illustrated in the child-based model in Chapter 1. After some time, the adult agent `dies' and the learner becomes the new adult speaker so its grammar becomes the new convention\is{convention}. A new child learner is then introduced into the population\is{speech population}. This population\is{speech population} turnover is iterated thousands of times.

The ILM\is{Iterated Learning Model (ILM)} hypothesizes that the development of grammar is triggered by the Poverty of the Stimulus\is{Poverty of the Stimulus} or the learning bottleneck\is{bottleneck}: since children cannot observe all possible utterances in a language, there is a pressure on language to become more learnable. The linguistic inventor\is{linguistic inventory}y should therefore evolve to an optimal size for a given meaning space. The meaning space consists here of five predicates and five objects, which can combine into simple two-argument events such as {\em loves(john, mary)}. In total there are 100 possible events. The optimal inventory size consists of 11 rewrite-rules: one abstract rule for word order\is{word order}, and ten rules for each word.

One challenge for a Universal Grammar\is{Universal Grammar} mechanism is to provide the agents with a strategy for filtering the `correct' input from the `wrong' input. In these experiments, child learners will especially be confronted with conflicting input at the beginning because the language of the adult speaker is still holistic and unstructured. The ILM\is{Iterated Learning Model (ILM)} solves this problem by ignoring all variation\is{variation}. For the learner, this means that once a rule has been induced, all conflicting input is neglected. On top of that, the agents are endowed with a deterministic parser which always picks the first matching rule in the list. Competing rules are therefore never considered because they are lower in the list. This assures a one-to-one mapping between meaning and form, which is crucial for the ILM\is{Iterated Learning Model (ILM)} to work properly \citep{smith03transmission}. The deterministic parser also allows the agents to rely on word order\is{word order} for distinguishing the semantic role\is{semantic role}s of events.
\\
\\
\noindent{\bfseries Results.} The results of the simulations show that the agents start inventing holistic\is{holistic versus compositional languages} utterances, which leads to an unstructured language in the first generations. After several hundreds of generations, the language becomes more and more regular and thus learnable due to the overgeneralization of linguistic input by the child learners. These overgeneralizations become the new grammar of the language once the child replaces the adult speaker. Moy notes, however, that not all languages evolve to the optimal size of 11 rules, but that {\em ``a significant number of the runs converged on a larger grammar with 16 rules''} (p. 113). These grammars contain two distinct noun\is{noun} categories: one for the agent and one for the patient\is{semantic role!patient}. Here is an example of such a grammar which uses an SOV word order\is{word order}:

\ea
\label{g:moy-grammar}
\begin{tabular}{l c l}
{\em s/} [P, X, Y] & $\rightarrow$  & 3/X, i, 1/Y, 2/P
\\ 1/anna & $\rightarrow$ & i, p, l
\\ 1/kath & $\rightarrow$ & c, s
\\ 1/mary & $\rightarrow$ & t, a
\\ 1/john & $\rightarrow$ & j, e
\\ 1/pete & $\rightarrow$ & h
\\ 3/kath & $\rightarrow$ & a, k, f
\\ 3/pete & $\rightarrow$ & a, u, f
\\ 3/mary & $\rightarrow$ & t, s
\\ 3/anna & $\rightarrow$ & g
\\ 3/john & $\rightarrow$ & p
\\ 2/kisses & $\rightarrow$ & t
\\ 2/hates & $\rightarrow$ & z, s
\\ 2/loves & $\rightarrow$ & m, q, j
\\ 2/adores&  $\rightarrow$ & u, i
\\ 2/sees & $\rightarrow$ & m, y
\\ \citep[p. 113]{moy06case}
\end{tabular}
\z

In the above grammar, the agents have to use a different word for the same object depending on which semantic role\is{semantic role} it plays in the event:

\ea
\gll p i ta mqj \\
john {[EMPTY]} mary loves \\
\glt `John loves Mary.' \\

\item
\gll ts i je mqj \\
mary {[EMPTY]} john loves \\
\glt `Mary loves John.' \\
\z

This kind of `suboptimal' grammar is not exclusive to Moy's replication experiment: it also occurs in Kirby's original simulations. For example, \citet{kirby99syntax} suggests that the two distinct noun\is{noun} categories can be considered as case-marked nominals. In the discussion of the replicating experiment, Moy seems to follow this hypothesis:

\begin{quote}
Could we view such a grammar as exhibiting some form of primitive case system\is{case!case system}, in that it is possible to distinguish subject\is{syntactic role!subject} forms of noun\is{noun}s from object\is{syntactic role!object}s, rather than using the same form for both? This is analog\is{analogy}ous perhaps to highly irregular forms of case found in some languages, such as the English\is{English} pronoun\is{pronoun}s {\em I, me, we} and {\em us}, where the nominative\is{case!nominative} forms, ({\em I} and {\em we}) used to the [sic] represent the subject\is{syntactic role!subject} of a sentence, have no morphological relationship to the accusative\is{case!accusative} forms used for object\is{syntactic role!object}s ({\em me} and {\em us}). \citep[p. 114]{moy06case}
\end{quote}

\noindent{\bfseries Discussion.} As I argued in section \ref{s:problem-systematicity}, considering the above grammar as some kind of primitive case system\is{case!case system} is an over-interpretation of experimental results. While there are many attested examples in natural languages in which a word form depends on the linguistic context (e.g. many Slavic languages such as Russian\is{Russian} use different lexical entr\is{lexical entry}ies for verb\is{verb}s depending on aspect\is{aspect}), they do not come about as frozen accidents of the learning mechanism. As for the English\is{English} pronoun\is{pronoun}s, they are remnants of a stage in the development of English\is{English} where the grammar had a fully productive case system\is{case!case system}. Both Moy and Kirby thus fail to identify the problem of systematicity\is{systematicity} that occurs in the experiments.

This observation illustrates the importance of a strong dialogue\is{dialogue} with linguistics. Most research in the field so far has contented itself with shallow comparisons to natural languages. It is however crucial to go into more empirical details in order to appreciate the enormous complexity involved in grammatical phenomena in natural languages. In this book, I tried to offer such an appreciation of case marker\is{case!case marking}s in the first Chapter. A better understanding of the developmental pathways of case marker\is{case!case marking}s helped to uncover the problem of systematicity\is{systematicity} in this book and prevented me from simply concluding that the lack of systematicity\is{systematicity} could be mapped onto some of the more exotic case alignment systems\is{case!case system} found in the world's languages (see Figure \ref{f:wals}). Other first steps towards domain-specific dialogue\is{dialogue}s exist for colour\is{colour} terms \citep{steels05coordinating}, spatial\is{spatial language} language \citep{loetzsch08typological} and vowel systems\is{vowel systems} \citep{deboer99self}.

In this book, I solved the problem of frozen accidents through multi-level selection\is{selection}\is{multi-level selection}. I implemented multi-level selection\is{selection}\is{multi-level selection} as an alignment strateg\is{alignment strategy}y which is therefore tightly coupled to communicative success\is{communicative success}. In the ILM\is{Iterated Learning Model (ILM)}, however, communicative success\is{communicative success} has no impact on the behaviour of the agents so multi-level selection\is{selection}\is{multi-level selection} cannot be readily applied here. Second, multi-level selection\is{selection}\is{multi-level selection} only makes sense if there is variation\is{variation} in the model, but the ILM\is{Iterated Learning Model (ILM)} avoids variation\is{variation} as much as possible so once a frozen accident occurs, it is hard to get rid of it.

Even though it is unwarranted to equate the frozen accidents in the ILM\is{Iterated Learning Model (ILM)} experiments with case systems\is{case!case system} in natural languages, Moy's work takes an interesting turn by asking how the ILM\is{Iterated Learning Model (ILM)} can be expanded so that it favours these `suboptimal' languages. If Moy succeeds in demonstrating the factors that {\em systematically} lead to the emergence\is{emergence} of such languages, the experiments could still come up with relevant pressures for evolving a case language.

\subsection{Experiment 2: dealing with variation\is{variation}}
\label{s:moy-communication}

In order to encourage the emergence\is{emergence} of a primitive case system\is{case!case system}, \citet[chapter 5]{moy06case} experiments with various modifications of Kirby's parser that make it less deterministic and which allows for variation\is{variation} in word order\is{word order}. The hypothesis is that if the agents can no longer rely on word order\is{word order} for distinguishing agents from patient\is{semantic role!patient}s in the events, they will start learning grammatical rules with case-like properties.
\\
\\
\noindent{\bfseries Experimental results.} Moy first tries to introduce variation\is{variation} in the word order\is{word order} by allowing the agents to randomly choose among conflicting rules or by reshuffling the inventory. In all the simulations, however, the agents fail to converge on a compositional grammar: the linguistic inventor\is{linguistic inventory}ies of the agents are very large and too much conflicts are known for reaching a regular language. Moy argues that this is due to the fact that compositionality can only emerge in the ILM\is{Iterated Learning Model (ILM)} if variation\is{variation} in meaning-form mappings is ignored: learners will not consider any new variation\is{variation} anymore once they have associated a certain meaning with a certain form, and the deterministic parser excludes the use of variation\is{variation}s. The same conclusion has also been suggested by \citet{smith03transmission}, but as opposed to Moy who rejects this unrealistic assumption\is{assumption}, Smith argues that natural languages have such a bias\is{learning bias} towards one-to-one mappings as well.

Moy thus points to a fundamental problem with the ILM\is{Iterated Learning Model (ILM)}: in order to allow variation\is{variation} in word order\is{word order}, the agents need to be capable of parsing and producing competing rules. Yet, in order for the grammar to emerge at all, the ILM\is{Iterated Learning Model (ILM)} expects that a single variation\is{variation} is maintained. Moy notes that in order to dampen the search space, the agents need a way to prefer some variation\is{variation}s over other ones. She therefore endows the agents with an alignment strateg\is{alignment strategy}y which takes the frequency\is{frequency} of rules into account. The more frequent the rule, the higher the probability that the agent will select it for production. The alignment strateg\is{alignment strategy}y allows the agents to reach a compositional language again. It does not, however, lead to an increase in the number of primitive case grammar\is{case!case grammar}s. In additional experimental set-ups, Moy allows even more word order\is{word order} freedom by reshuffling sentences before they are actually produced, but this again does not lead to a preference for the primitive case grammar\is{case!case grammar}. Finally, manipulating the size of the transmission \is{transmission}bottleneck (i.e. the number of utterances that the child learner observes during a lifetime) does not yield significant results either.

Moy concludes that the small effect of free word order\is{word order} and the bottleneck\is{bottleneck} size is due to the fact that the agents do not have to reach mutual understanding: if the free word order\is{word order} cannot be parsed by the hearer, it will simply be ignored so it will not lead to a change in the linguistic inventor\is{linguistic inventory}y. In other words, the child agent does not really care about converging on the same language as that of the adult speaker but learns whatever hypotheses its inducti\is{induction}on algorithm comes up with. Moy argues that the agents thus have no need for disambiguating semantic role\is{semantic role}s, which prevents them from reaching a case-like grammar.
\\
\\
\noindent{\bfseries Discussion.} Moy's experiments reveal a number of fundamental shortcomings of the ILM\is{Iterated Learning Model (ILM)}: first of all, the agents have no way of dealing with variation\is{variation}. This problem did not surface in Kirby's prior work because he implemented a bias\is{learning bias} towards one-to-one mappings that excludes the possibility of competit\is{competition}ors. Moreover, the ILM\is{Iterated Learning Model (ILM)} is typically implemented using only two agents, so no competing rules can ever be introduced in the population\is{speech population}. Moy's results clearly show that there is no connection whatsoever between the grammar that is acquired by the learner and the grammar of the speaker: in fact, the learner will apply a `first come, first serve' approach to learning grammar in which the first successful parse leads to a fixed entry in the inventory. This means that the agents in Kirby's models did not {\em learn} to mark the distinction between semantic role\is{semantic role}s, but that they have this distinction already built in.

Moy rightfully notices that the agents need some kind of (alignment) strategy in order to reach a regular language. She introduces an utterance-based\is{selection!utterance-based selection} strategy which counts the number of occurrences of rules. Equipped with this alignment strateg\is{alignment strategy}y, the agents are capable of producing and parsing multiple word order\is{word order}s, but this has hardly any effect on the language of the agents. Moy then rightfully concludes that the agents need {\bfseries communicative pressures}: since the child learner does not care about communicative success\is{communicative success}, any grammar inducti\is{induction}on will do. The experiments thus show that the ILM's `function independence principle\is{function independence principle}' cannot lead to grammar (at least not a case grammar\is{case!case grammar}) unless by accident or through a Universal Grammar\is{Universal Grammar}. The transmission \is{transmission}bottleneck is therefore not a sufficient trigger for marking event structure\is{event structure} through grammar.
\\
\\
\noindent{\bfseries Problems with multi-agent simulations.} Another way in which the problems of the ILM\is{Iterated Learning Model (ILM)} can be demonstrated is to scale up the experiments to multi-agent population\is{speech population}s. This has indeed been attempted by \citet{smith03language}. Smith \& Hurford reached the following conclusions:

\begin{enumerate}
\item The agents fail to align their grammars because they have no alignment strateg\is{alignment strategy}y and different agents will come up with different innovat\is{innovation}ions and generaliz\is{generalization}ations;
\item As a result, learners are presented with inconsistent training data;
\item The eager abstraction\is{abstraction} algorithm has disastrous consequences for the performance\is{performance} of the agents.
\end{enumerate}

Smith \& Hurford consider the option of allowing the learners to keep multiple hypotheses. Since the agents are however `eager learners', their abstraction\is{abstraction}s harm their performance\is{performance}: abstraction\is{abstraction}s made by one agent are not always the same as the abstraction\is{abstraction}s made by another one so the agents never reach a shared language. This means that the agents would have to maintain all possible grammars, which rapidly becomes intractable. Another problem is that the agents need to find out which grammar is `best'. Smith \& Hurford refute the possibility of a cost system (similar to the lateral inhibition\is{lateral inhibition} strategies proposed in many problem-solving\is{problem-solving} models) on the grounds that it is `rather ad hoc'. The solution offered by Smith \& Hurford is however at least ad hoc as most cost systems are: they implement strong production bias\is{learning bias}es coupled to `smart pruning' in order to reduce the number of hypotheses.

However, variation\is{variation} is a fundamental property of language. Introducing  additional production bias\is{learning bias}es is a way to put less weight on the cultural evolution\is{evolution!cultural evolution} of language and more on the genetic endowment of the agents, which is exactly the contrary of what the ILM\is{Iterated Learning Model (ILM)}s try to show. Moreover, there now exist mathematical models of lateral inhibition\is{lateral inhibition} dynamics which solve the `rather ad hoc' status of cost systems \citep{baronchelli06sharp, devylder07evolution}. Also the alignment strateg\is{alignment strategy}y based on token frequency\is{token frequency}\is{frequency} proposed in this book is rooted in proposals made in cognitive-functional\is{cognitive-functional approach} models of language.

The real problem for the ILM\is{Iterated Learning Model (ILM)} is however that such a cost system only works in a bottom-up, instance per instance learning of the grammar. Otherwise it would indeed lead to an intractable search space containing all possible grammars. In order to avoid innate bias\is{learning bias}es towards one-to-one mappings, it is therefore necessary to introduce {\bfseries an utterance-based\is{selection!utterance-based selection} selection\is{selection}ist system} rather than a grammar-based selection\is{selection}ist system. The work in this book demonstrates how such a bottom-up approach can lead to systematic languages if analog\is{analogy}y is used in combination with multi-level selection\is{selection}\is{multi-level selection}.

\subsection{Experiment 3: implementing communicative pressures}

Following the conclusion that the agents need additional communicative pressures in order to form a primitive case grammar\is{case!case grammar}, \citet[chapter 6]{moy06case} presents a series of experiments in which ambiguous word order\is{word order}ings occur. The hypothesis is that this ambiguity\is{ambiguity} will lead the agents to prefer rules that use different noun\is{noun} categories to mark the distinction between semantic role\is{semantic role}s.
\\
\\
\noindent{\bfseries Experimental results.} In a first set-up, Moy implements an `inversion procedure' that swaps the ordering of words for which the agent already has a rule. This procedure thus guarantees ambiguity\is{ambiguity} during learning. The results are however disastrous: most of the runs fail to reach any kind of regularity at all. Moy assigns the reason for this failure to the fact that the child learner indeed faces ambiguity\is{ambiguity}, but that he still does not need to reach mutual understanding with the speaker. The hearer thus keeps ignoring the ambiguous utterances if they do not fit the grammar rules that have already been acquired.

In an attempt to fix this problem, Moy implements a learner that does not tolerate ambiguity\is{ambiguity} and an interaction script that forces the speaker to introduce unambiguous utterances: the meaning that was parsed by the hearer is compared to the intended meaning of the speaker. If there is a mismatch, the speaker has to come up with an alternative verb\is{verb}alization. However, this implementation does not lead to success either: in most cases, the speaker does not have an alternative way to verb\is{verb}alize an utterance so he will invent a new holistic\is{holistic versus compositional languages} string. Since the ILM\is{Iterated Learning Model (ILM)} features meaning transfer, the hearer will each time learn this new utterance. The result is that there is constant innovat\is{innovation}ion in the simulations so the agents never reach an `optimal' language. Additional attempts, such as punishing ambiguous utterances, do not yield improvement either.
\\
\\
\noindent{\bfseries Discussion.} Even though Moy's experiments started from a correct observation, the `communicative pressures' that are needed for a case grammar\is{case!case grammar} were not operationalized and implemented in a satisfying way. The main problem is that the agents in her experiments are not truly communicating. First of all, the speaker's output was randomly changed by an artificial procedure in order to create ambiguities for the hearer. This is already highly problematic because it would require some malicious mind-reading from the speaker's part, but there are other more serious issues: the speaker does not have any communicative goal at all. He just produces an utterance but does not care about whether this utterance had the intended effect in the hearer's mind. Other simulations that emphasize the importance of communication have all come to the conclusion that the speaker must try to produce an utterance in such a way as to improve the chance of being understood by the hearer \citep{smith03intelligent, steels03language}. In this book, this was operationalized in the form of re-entrance\is{re-entrance} (see Chapter \ref{c:base}).

From the part of the hearer, there is the same problem. `Ambiguity' in Moy's model does not mean that the hearer did not understand what the speaker said because there is meaning transfer and the hearer can even compare his parsed meaning to the speaker's intended meaning. The problem is that the hearer does not attempt to align his grammar to that of the speaker at all: a mismatch in meanings means the rejection of the speaker's utterance. This means that the hearer stubbornly sticks to his induced grammar rules which may well be completely different than the grammar of the adult speaker because of the greedy inducti\is{induction}on algorithm. This problem does not occur in the experiments in this book because the agents try to find out what the speaker's intentions were and want to conform to the convention\is{convention}s of the population\is{speech population}.

In short, none of the agents ever actually try to reach communicative success\is{communicative success}. As I argued in Chapter \ref{c:base}, language is an {\bfseries inferential coding system\is{inferential coding system}} in which the language users are assumed to be intelligent enough to make innovat\is{innovation}ions that can be understood by the hearers, and in which hearers can make {\bfseries abduction\is{abduction}s} about the speaker's intended meaning. In Moy's experiments, the agents only want to get rid of internal inconsistencies rather than trying to converge on the same grammar.

\subsection{Experiment 4: more innate knowledge}

In a final series of experiments, \citet[chapter 7]{moy06case} tries to address the problem of the `suboptimal' primitive case system\is{case!case system} in a different way. She starts from the observation that the grammar inducer {\em ``is not capable of} effectively {\em learning inflection\is{inflection}al grammars''} (p. 206). For example, the default inducer would fail to notice the inflection\is{inflection}al marker\is{case!case marking}s in an utterance such as {\em johnalovesmaryb}: instead of recognizing {\em -a} as an indicator of the subject\is{syntactic role!subject} and {\em -b} as an indicator of the object\is{syntactic role!object}, the learner will induce them as an integral part of the word form. Moy therefore proposes several implementations that attempt to solve this problem. I will discuss one of these attempts in which he implements a richer meaning representation.
\\
\\
\noindent{\bfseries Experimental results.} In all of the previous experiments, the semantic role\is{semantic role}s of agent and patient\is{semantic role!patient} were implicit in the meaning so the inducer wasn't able to extract them. Moy therefore decides to make the meaning representation richer by explicitly adding the roles of `actor' and `actedon'. The meaning [loves, john, mary] would thus become [[act, loves], [actor, john], [actedon mary]] (p. 217). The `optimal' case marking\is{case!case marking} language should have 15 rules: one top-level rule, ten lexical entr\is{lexical entry}ies, two marker\is{case!case marking}s and two rules to combine the marker\is{case!case marking}s with the noun\is{noun} categories. Despite the explicit representation of semantic role\is{semantic role}s, however, not all simulations led to satisfying results. There were two kinds of problematic cases:

\begin{itemize}
\item One type of grammar again features two completely distinct noun\is{noun} categories with different words for the same meaning. The inducer failed to recognize inflection\is{inflection}al marker\is{case!case marking}s for the agent\is{semantic role!agent} and the patient\is{semantic role!patient}.
\item A second type of languages {\em did} have inflection\is{inflection}al marker\is{case!case marking}s, but they also featured two different lexical entr\is{lexical entry}ies for the noun\is{noun}s. An example of such a grammar is:

\ea
\label{g:moy-grammar2}
\begin{tabular}{l c l}
s/ [P, X, Y] & $\rightarrow$  & 1/Y, 13/X, 6/P
\\ 13/ [A, B] & $\rightarrow$  & 16/B, 15/A
\\ 1/ [C, D] & $\rightarrow$  & 3/C, 8/D
\\ 15/ actor & $\rightarrow$  & i
\\ 3/ actedon & $\rightarrow$  & h
\\ 16/ john & $\rightarrow$  & v, s
\\ 16/ jane & $\rightarrow$  &  k, h
\\ 8/ john & $\rightarrow$  & e, v, s, n
\\ 8/jane & $\rightarrow$  & s
\\ 6/ [E, F] & $\rightarrow$  & 14/F, 7/E
\\ 7/ act & $\rightarrow$  & i, b
\\ 14/ adores & $\rightarrow$  & y, n, k
\\ 14/loves & $\rightarrow$  & j, v
\\ 14/sees & $\rightarrow$  & h, m, l
\\ \citep[p. 230]{moy06case}
\end{tabular}
\z
\end{itemize}

This would lead to sentences such as: 

\ea
\gll h evsn kh i jv ib \\
actedon john jane actor loves act \\
\glt `Jane loves John.' \\

\item
\gll h s vs i jv ib \\
actedon jane john actor loves act \\
\glt `John loves Jane.' \\
\z

\noindent{\bfseries Discussion.} Moy's experiments suggest that the Iterated Learning Model\is{Iterated Learning Model (ILM)} cannot overcome its bias\is{learning bias} towards one-to-one mappings, which is confirmed in the fact that experimenters working with the model increasingly turn to innate constraints for explaining grammar emergence\is{emergence}: Moy explicitly includes two semantic role\is{semantic role}s in the meaning space, and \citet{smith03transmission} and \citet{smith03language} implement explicit bias\is{learning bias}es towards one-to-one mappings. This is highly problematic since grammatical categories are clearly multifunctional. It also means that the ILM\is{Iterated Learning Model (ILM)} studies have given up on their original objectives: instead of demonstrating that grammar can evolve through {\em cultural} selection\is{selection}, a strong Language Acquisition Device\is{Language Acquisition Device} is built in. 

The experiments in this book do not presuppose such a bias\is{learning bias} towards one-to-one mappings and in fact offer {\bfseries the first multi-agent simulations ever featuring polysemous categories}. By taking communicative pressures seriously and by providing the agents with a richer cognitive apparatus including analog\is{analogy}y and multi-level selection\is{selection}\is{multi-level selection}, this book demonstrated how agents can deal with the variety and uncertainty that is inherent to multi-agent simulations, how they can self-organiz\is{self-organization}e and coordinate a grammar involving multifunctional semantic role\is{semantic role}s, and how they can reuse\is{reuse} the same linguistic items in multiple patterns of argument realization\is{argument realization}.

What is also striking about Moy's results is the fact that even though there are only two agents at each given time and even though the learner is equipped with a highly specialized learning mechanism, the grammars can still get stuck in a state which is not fully systematic. This suggests that (a) a learning bottleneck\is{bottleneck} is not a sufficient pressure for reaching a systematic language, and that (b) an innate mechanism that seriously restricts the space of possible grammars is not sufficient for dealing with variation\is{variation}. The only way to avoid this unsystematic state, as I argued in Chapter \ref{c:experiment1}, is to assume a cognitive-functional\is{cognitive-functional approach} view on grammar in which agents are given credit for possessing the right skills and alignment strateg\is{alignment strategy}ies to arrive at a shared and systematic communication system.

\subsection{Summary: case marker\is{case!case marking}s serve communication}


\begin{centering}
\begin{table}[t]
\begin{tabular}{| l | l | l |}
\hline
 & {\bfseries Iterated Learning} & {\bfseries This book}
\\
\hline
{\bfseries Triggers} & - Poverty of the Stimulus\is{Poverty of the Stimulus} & - Communicative success\is{communicative success}
\\ {\bfseries of grammar} & - Learning bottleneck\is{bottleneck} & - Reducing cognitive effort\is{cognitive effort}
\\ & - Function independence & - Increasing expressiveness\is{expressiveness}
\\
\hline
{\bfseries Learner} & - Eager learner & - Lazy learner
\\ & - Greedy inducti\is{induction}on & - Careful abstraction\is{abstraction}
\\ & - Top-down & - Bottom-up
\\
\hline
{\bfseries Language}  & - Hearer-based innovat\is{innovation}ion & - Speaker-based innovat\is{innovation}ion
\\ {\bfseries change\is{language change}} & - Mismatch in learning & - Adoption by hearer
\\& & - Propagation in population
\\
\hline
{\bfseries Grammati-} & - Analysis only & - Analog\is{analogy}y and patterns
\\ {\bfseries calization} & - Holistic strategy & - Continuity principle (reuse\is{reuse})
\\ & - Start without language & - Start from lexicon\is{lexicon}
\\
\hline
{\bfseries Inventory} & - Rewrite rules & - Constructions
\\ & - Ordered but unstructured & - Structured through usage
\\
\hline
{\bfseries Meanings} & - Two-way contrast & - Event-specific meaning
\\ & - Semantic role\is{semantic role}s given & - No prior semantic role\is{semantic role}s
\\
\hline
{\bfseries Population} & - Two-agent simulation & - Multi-agent simulation
\\ & - Generational turnover & - Single generation
\\ & - Speakers vs learners & - Peer-to-peer interactions
\\
\hline
\end{tabular}
\caption[\citet{moy06case} versus this book]{This table summarizes the main differences between \citet{moy06case} and the work in this book.}
\label{t:moy}
\end{table}
\end{centering}

The ILM\is{Iterated Learning Model (ILM)} and the problem-solving\is{problem-solving} approach both argue that grammar evolves through cultural evolution\is{evolution!cultural evolution} but both models diverge significantly in terms of assumption\is{assumption}s and hypotheses (see Table \ref{t:moy}). The most important difference is that the ILM\is{Iterated Learning Model (ILM)} tries to explain as much grammatical structure as possible as a side-effect of the cultural transmission \is{transmission}of language from one generation to the next, whereas the problem-solving\is{problem-solving} approach assumes that grammatical development is triggered by the need to optimize communicative success\is{communicative success}. This leads to two different types of learners: in the ILM\is{Iterated Learning Model (ILM)} the learner needs strong innate constraints as opposed the problem-solving\is{problem-solving} approach where the learner is equipped with a rich cognitive apparatus for detecting and solving communicative problems.

Since the learner in the ILM is endowed with some kind of Universal Grammar, the model has a lot of difficulties with handling inconsistent input data and variation. The problem-solving\is{problem-solving} model of this book, on the other hand, does not face those difficulties. It features a redundant and bottom-up approach to language and an utterance-based\is{selection!utterance-based selection} selection\is{selection}ist system in which careful abstraction\is{abstraction} is possible despite the enormous uncertainty about the convention\is{convention}s in the speech community\is{speech population}. The agents in this book do not assume one-to-one mappings and the experiments offer the first multi-agent simulations ever which involve polysemous categories.

Another achievement of this book is that it detected and solved the problem of systematicity\is{systematicity}. I showed that the problem also occurs in Iterated Learning Model\is{Iterated Learning Model (ILM)}s (and other models, see Chapter \ref{c:experiment1}) but that it remained unnoticed so far. I argued that this was due to an underestimation of the complexity of natural language phenomena and a shallow comparison between natural and artificial languages. The work in this book is more firmly rooted in linguistic theory and offers a model which is closer to attested cases of grammaticalization\is{grammaticalization}.

It should be noted that both approaches are not mutually exclusive. The problem-solving\is{problem-solving} approach naturally incorporates all the learning constraints that are focused upon in the ILM\is{Iterated Learning Model (ILM)} but goes much further in terms of learning and innovat\is{innovation}ion strategies in order to optimize communicative success\is{communicative success}. Expanding the model to multi-generational population dynamics is indeed technically quite trivial and has already been successfully demonstrated in many experiments \citep[e.g.][]{debeule08emergence, steels99spatially}. Expanding the ILM\is{Iterated Learning Model (ILM)} to a multi-agent population, however, is much more problematic.


\section{Argument structure and construction grammar}
\label{s:comp-formalism}

In Chapter \ref{c:ar} and \citet{vantrijp08argumentsstruktur}, I proposed a formalization of argument realization\is{argument realization} in Fluid Construction Grammar\is{construction grammar!Fluid Construction Grammar}. Even though my proposal was explicitly implemented for supporting experiments on artificial language evolution\is{artificial language evolution}, its ideas are relevant for formalisms of natural languages as well. Within the family of construction grammars, however, no alternative computational implementation has been reported yet in a peer-reviewed publication that can handle both parsing and production. As such, \citet{vantrijp08argumentsstruktur} offers the first computational implementation of \is{event structure}\is{construction!argument structure construction}argument structure constructions within a construction grammar framework.

This doesn't mean that no other work has been done yet: at the moment this book was written, a different proposal was still in the process of being worked out (but not implemented) in Sign-Based Construction Grammar\is{construction grammar!Sign-Based Construction Grammar} \citep[SBCG\is{construction grammar!Sign-Based Construction Grammar},][]{fillmore08construction}, a formalism that combines HPSG\is{Head-Driven Phrase Structure Grammar} \citep{sag94hpsg} with Berkeley Construction Grammar \citep[BCG,][]{kay99grammatical}. Since the draft proposals are closely related to the more construction-oriented approaches in HPSG\is{Head-Driven Phrase Structure Grammar}, I will assume here that SBCG\is{construction grammar!Sign-Based Construction Grammar} can be implemented and applied in a {\em computational} formalism as well. In the remainders of this section, I will briefly illustrate how SBCG\is{construction grammar!Sign-Based Construction Grammar} deals with \is{event structure}argument structure \citep[based on][]{fillmore08construction, kay05argument, michaelis06complementation,sag07sbcg}. I will then illustrate this for the English\is{English} ditransitive\is{ditransitive clause} based on \citet{kay05argument} and then compare it to my own representation of \is{event structure}argument structure.

\subsection{Argument structure in BCG\is{construction grammar!Berkeley Construction Grammar} and SBCG\is{construction grammar!Sign-Based Construction Grammar}}
\label{s:comp-bcg}

Early representations of \is{event structure}argument structure in a construction grammar framework \citep[such as][]{goldberg95construction} and the representation used in this book propose \is{event structure}\is{construction!argument structure construction}argument structure constructions in which the meaning can be seen as a skeletal or schematic event type\is{event type} such as X-CAUSES-Y-TO-MOVE. These constructions then have to unify\is{unify and merge} or fuse with the lexical entr\is{lexical entry}y of the verb\is{verb}. In later versions of BCG\is{construction grammar!Berkeley Construction Grammar} and in SBCG\is{construction grammar!Sign-Based Construction Grammar}, however, \is{event structure}\is{construction!argument structure construction}argument structure constructions are implemented as two-level deriv\is{derivation}ational rules that have a `mother\is{mother and daughter components}'-component (MTR) and a `daughter'-component (DTR). The DTR unifies with the lexical entr\is{lexical entry}y of the verb\is{verb} and is complemented by the MTR. The SBCG\is{construction grammar!Sign-Based Construction Grammar} proposal looks very much like lexical rule\is{lexical rule}s in lexicalis\is{lexicalist}t accounts.

The SBCG\is{construction grammar!Sign-Based Construction Grammar} approach is in the same spirit as \citet{goldberg95construction} in the sense that minimal lexical entr\is{minimal lexical entry}\is{lexical entry}ies are assumed for verb\is{verb}s which have to be complemented by \is{event structure}\is{construction!argument structure construction}argument structure constructions. SBCG\is{construction grammar!Sign-Based Construction Grammar} however goes a step further and allows \is{event structure}\is{construction!argument structure construction}argument structure constructions to {\em override} the default behaviour of a verb\is{verb} \citep{michaelis06complementation, sag07sbcg}. \citet{sag07sbcg} writes that SBCG\is{construction grammar!Sign-Based Construction Grammar} has a feature called `\is{event structure}argument structure' (ARG-ST) which encodes the valence of a verb\is{verb}. This feature is a structured list which is coupled to the `Accessibility Hierarchy\is{Accessibility Hierarchy}' of \citet{keenan77noun}: the first argument maps onto subject\is{syntactic role!subject}, the second onto the direct object\is{syntactic role!object}, etc. The rank-based listing of arguments is chosen to eliminate the need for explicit features such as `subject\is{syntactic role!subject}' and `object\is{syntactic role!object}'.

Different argument realization\is{argument realization} patterns, such as the active\is{construction!active construction}-passive\is{construction!passive construction} alternation, are represented through different values for the ARG-ST list. SBCG\is{construction grammar!Sign-Based Construction Grammar} has two different ways to implement these differences: either a deriv\is{derivation}ational construction overrides the default ARG-ST of the verb\is{verb} (as is the case in passivization) or there is lexical under-specification (for example in locative alternation\is{locative alternation}s). In addition to the feature ARG-ST, SBCG\is{construction grammar!Sign-Based Construction Grammar} also has a feature VAL(ence) which is a list of the syntactic elements that a linguistic expression yet has to combine with. Sag gives the example of the verb\is{verb} phrase {\em persuaded me to go}, which takes the VAL list < NP > because it still needs to combine with a subject\is{syntactic role!subject} NP. The clause {\em my dad persuaded me to go} takes an empty VAL list because it doesn't need to combine with any other argument anymore. Only lexical constructions have both ARG-ST and VAL features. Phrases, clauses, NPs, and other items only have empty VAL lists. 

\subsection{An example: the ditransitive\is{ditransitive clause} construction}

I will now illustrate \is{event structure}\is{construction!argument structure construction}argument structure constructions in SBCG\is{construction grammar!Sign-Based Construction Grammar} based on \citet{kay05argument}'s analysis of the English\is{English} ditransitive\is{ditransitive clause}. Kay's proposal is somewhat different than the most recent developments of the SBCG\is{construction grammar!Sign-Based Construction Grammar} architecture, but the underlying ideas are the same (Kay, pers. comm.). It shares many aspects with the approach offered by \citet{goldberg95construction}, such as the assumption\is{assumption} that there is a default and minimal lexical entr\is{minimal lexical entry}\is{lexical entry}y for each verb\is{verb}. For example, the lexical entr\is{lexical entry}y of the verb\is{verb} {\em to bake} contains two minimally required arguments (a baker and a thing that is baked). \is{event structure}Argument structure constructions can add arguments such as the beneficiary\is{semantic role!beneficiary} in {\em He baked him a cake}. 

Goldberg sees \is{event structure}\is{construction!argument structure construction}argument structure constructions as larger patterns carrying grammatical meanings which have to be `fused' with the meaning of the lexical entr\is{lexical entry}ies. Kay, however, proposes \is{event structure}\is{construction!argument structure construction}argument structure constructions which are more like lexical constructions with a `mother\is{mother and daughter components} constituent' and a single daughter. The daughter unifies with a lexical entr\is{lexical entry}y and is elaborated by the mother\is{mother and daughter components} constituent. Applied to the English\is{English} ditransitive\is{ditransitive clause} construction, Kay proposes three `maximal recipient\is{semantic role!recipient} constructions' which all three inherit from a more schematic `Abstract Recipient Construction\is{construction!Recipient Construction}':

\ea
\Tree[.{Abstract Recipient Construction\is{construction!Recipient Construction}} {Direct RC\\({\em give, throw, ...})} {Intended RC\\({\em bake, buy, ...})} {Modal RC\\({\em refuse, promise, allow, ...})} ]
\z

These constructions are represented in a unification\is{unification}-based grammar as detailed in \citet{kay99grammatical}, and which is similar to analyses of \is{event structure}argument structure in HPSG\is{Head-Driven Phrase Structure Grammar} \citep{sag94hpsg}. The Abstract Recipient Construction is shown in Figure \ref{f:arc}. The construction shows all information which is common to all English\is{English} ditransitive\is{ditransitive clause} constructions. The lower box represents the DTR constituent, which needs to unify\is{unify and merge} with the lexical entr\is{lexical entry}y of the verb\is{verb}. As can be seen, its valence list contains an NP (which plays the role of `actor') and an under-specified argument (...). The top box represents the MTR constituent which complements the lexical entr\is{lexical entry}y of the verb\is{verb} with the `recipient\is{semantic role!recipient}' role. The `list'-feature displays the construction's primary semantic frame\is{frame!semantic frame} ({\em an intentional act} which has an actor and an undergoer) and additionally an {\em intended result} which still needs to be specified by another frame. Kay proposes that the intended result in the Direct RC is the `receive frame' (see below) whereas in the other sub-constructions it is not.

\begin{figure}[t]
\Tree[.{\begin{avm}
\[ syn & \[ cat v, lex +, min- \]
\\ sem\|cont & \@0 \[ handle & \@1 \\ index & \@4 \\ list & \< \[ {\em intnl-act} \\ handle & \@1 \\ actor & \@2 \\ undergoer & \@3 \\ intnd-rslt & {\em handle} \\ event & \@4 \], \[{\em receive} \\ handle & {\em handle} \\ theme & \@3 \\ recipient\is{semantic role!recipient} & \@5 \\ event & {\em handle} \] , ... \> \]
\\ valence & \< \[ syn & NP \\ instance & \@2 \], \[ syn & NP \\ gf & obl \\ instance & \@3 \], \[ syn & NP \\ instance & \@5 \] \>
\]
\end{avm}} {\begin{avm}
\[ syn & \[ cat v, lex + \]
\\ sem\|cont & \@0
\\ valence & \< \[ syn & NP \\ instance & \@2 \], ...\> 
\]
\end{avm}} ]
\caption{The Abstract Recipient Construction proposed by \citet{kay05argument}.}
\label{f:arc}
\end{figure}

The Direct RC corresponds to \citet{goldberg95construction}'s `central sense' of the ditransitive\is{ditransitive clause} as in {\em He gave him a book}. The main frame of the MTR component is a CAUSE-MOVE act which is defined as a subtype of an intentional act. The receive frame is unified with the intended result of the main frame indicating that there was an actual transfer of possession. The daughter's valence list indicates that it takes verb\is{verb}s which have at least two arguments (an actor and an undergoer). The Direct RC is illustrated in Figure \ref{f:drc}.

\begin{figure}[t]
\Tree[.{\begin{avm}
\[ syn & \[ cat v, lex +, min- \]
\\ sem\|cont & \@0 \[ handle & \@1 \\ index & \@4 \\ list & \< \[ {\em cause-to-move} \\ handle & \@1 \\ actor & \@2 \\ undergoer & \@3 \\ intnd-rslt & \@6 \\ event & \@4 \], \[{\em receive} \\ handle & \@6 \\ theme & \@3 \\ recipient\is{semantic role!recipient} & \@5 \\ event & \@4 \] \> \]
\\ valence & \< \[ syn & NP \\ instance & \@2 \], \[ syn & NP \\ gf & obl \\ instance & \@3 \], \[ syn & NP \\ instance & \@5 \] \>
\]
\end{avm}} {\begin{avm}
\[ syn & \[ cat v, lex + \]
\\ sem\|cont & \@0
\\ valence & \< \[ syn & NP \\ instance & \@2 \], \[ syn & NP \\ instance & \@3 \], ...\> 
\]
\end{avm}} ]
\caption{The Direct Recipient Construction proposed by \citet{kay05argument}.}
\label{f:drc}
\end{figure}

In the Intended RC, for handling utterances such as {\em He baked him a cake}, there is no actual transfer of possession entailed but only the intention of transfer. A second reason for positing a different construction for the Intended RC is that it cannot occur in the passive\is{construction!passive construction}. Kay therefore includes an explicit stipulation in the construction which states that it cannot combine with a deriv\is{derivation}ational passive\is{construction!passive construction} construction. All the remaining senses of the ditransitive\is{ditransitive clause} are grouped together in the Modal RC. This construction is similar to the Intended RC in that it does not entail\is{entailment} actual transfer either, but it is different in the sense that it doesn't require {\em beneficiary\is{semantic role!beneficiary}} semantics. Kay argues that specific meanings are contributed by the verb\is{verb}s themselves so no additional constructions need to be posited. 

\subsection{Discussion and comparison}
\label{s:active-passive}

\noindent{\bfseries Derivational versus non-deriv\is{derivation}ational constructions.} The most remarkable distinction between argument realization\is{argument realization} in Sign-Based Construction Grammar\is{construction grammar!Sign-Based Construction Grammar} and Fluid Construction Grammar\is{construction grammar!Fluid Construction Grammar} is that FCG\is{construction grammar!Fluid Construction Grammar} adopts the cognitive-functional tradition of construction-based approaches in which argument structure constructions have skeletal meanings that need to unify\is{unify and merge} or fuse with the semantics of the lexical entr\is{lexical entry}ies of the verb\is{verb}s. FCG\is{construction grammar!Fluid Construction Grammar} could thus be said to implement a `fusion\is{fusion}' process similar to proposals made by \citet{goldberg95construction}. SBCG\is{construction grammar!Sign-Based Construction Grammar}, on the other hand, has given up on this kind of analysis and moved closer towards HPSG\is{Head-Driven Phrase Structure Grammar} by using (almost lexical) deriv\is{derivation}ational rules which feature two components: a mother\is{mother and daughter components} and a daughter.

These different approaches are the result of different solutions to the same problem: multiple argument realization. Both SBCG\is{construction grammar!Sign-Based Construction Grammar} and FCG\is{construction grammar!Fluid Construction Grammar} try to solve the problem through the notion of `potential syntactico-semantic arguments' \citep{sag07sbcg} or what I called `potential valent\is{valency!potential valents}s' in Chapter \ref{c:ar}. This word `potential' refers to the fact that a lexical entr\is{lexical entry}y can combine with multiple argument realization\is{argument realization} patterns. However, the implementation of this `potential' is fundamentally different in both formalisms.

SBCG\is{construction grammar!Sign-Based Construction Grammar} starts from the traditional assumption\is{assumption} that lexical entr\is{lexical entry}ies have a fixed predicate-frame\is{predicate-frame} which is implemented in the verb's VAL(ence) list. In order to still cope with multiple argument realization\is{argument realization} patterns, this VAL list either needs to be under-specified or overridden by deriv\is{derivation}ational rules. This capacity of overwriting the predicate-frame\is{predicate-frame} of a verb\is{verb} is in fact the only difference with traditional lexicalis\is{lexicalist}t accounts. FCG\is{construction grammar!Fluid Construction Grammar}, on the other hand, does not assume a minimal lexical entr\is{minimal lexical entry}\is{lexical entry}y: a linguistic item merely lists its potential from which more grammatical constructions can select the actual valen\is{valency!actual valency}cy. In other words, the meaning of the verb\is{verb} strongly influences its morpho-syntactic realization but its actual valen\is{valency!actual valency}cy is still dependent on the other constructions that combine with it.

The difference can be easily explained through an analog\is{analogy}y to mathematics which I borrowed and adapted from \citet{michaelis06complementation}. Michaelis writes that constructions have the possibility to change the associations within an arithmetic sequence like {\em 2 x (3 + 4)} to the sequence {\em (2 x 3) + 4}, which would yield different results ({\em 14} and {\em 10}). The individual numbers, however, denote the same value in each sequence. Michaelis' analog\is{analogy}y does not quite fit, however, since in SBCG\is{construction grammar!Sign-Based Construction Grammar} a number would be listed with a minimal ARG-ST (for example saying that `2' has to be used in a sum). SBCG\is{construction grammar!Sign-Based Construction Grammar} therefore does more than storing the entry `2' with its denotation and would need a deriv\is{derivation}ational rule which overrides the specification of `2'. FCG\is{construction grammar!Fluid Construction Grammar}, on the other hand, would list the number `2' and state that it can be potentially used in sums, divisions and other functions without actually committing to a single operation. The construction would then pick what it needs from the number.
\\
\\
\noindent{\bfseries Evidence from corpus-linguistics.} Technically speaking, the implementation differences between SBCG\is{construction grammar!Sign-Based Construction Grammar} and FCG\is{construction grammar!Fluid Construction Grammar} do not really matter. The main problem with deriv\is{derivation}ational constructions, however, is the assumption\is{assumption} that some senses of lexical items are more central or more basic than others. Traditionally, these `minimal entr\is{minimal lexical entry}ies' are however based on intuition rather than empirical data. For example, what is the minimal entr\is{minimal lexical entry}y for a verb\is{verb} such as {\em to give} if the following examples are taken into account:

\ea
He gave him the book.
\item He was given the book.
\item He gave blood.
\item Give it!
\item Give it to me!
\item Give me the book!
\z

More examples can easily be found. The point is however that FCG\is{construction grammar!Fluid Construction Grammar} would have no real preference for either pattern (except perhaps as the result of frequency\is{frequency} and priming\is{priming} effects) since the lexical entr\is{lexical entry}y does not contain a fixed predicate-frame\is{predicate-frame}. SBCG\is{construction grammar!Sign-Based Construction Grammar} would list {\em give} as a three-place predicate even though numerous sentences can be observed in which not all three arguments are present. Both formalisms thus make different predictions as to the frequency\is{frequency} of argument realization\is{argument realization} patterns: FCG\is{construction grammar!Fluid Construction Grammar} allows for different frequency\is{frequency} patterns for each verb\is{verb} individually (which is captured through co-occur\is{co-occurrence}rence links between lexical entr\is{lexical entry}ies and constructions), whereas SBCG\is{construction grammar!Sign-Based Construction Grammar} predicts a most basic use of a verb\is{verb} with deriv\is{derivation}ed and therefore less frequent uses.

These predictions can be verified through careful corpus studies. A good example is the active\is{construction!active construction}-passive\is{construction!passive construction} alternation. In FCG\is{construction grammar!Fluid Construction Grammar}, the passive\is{construction!passive construction} is an \is{event structure}\is{construction!argument structure construction}argument structure construction in its own right which stands on equal footing with active\is{construction!active construction} \is{event structure}\is{construction!argument structure construction}argument structure constructions. In SBCG\is{construction grammar!Sign-Based Construction Grammar}, on the other hand, the passive\is{construction!passive construction} construction is a deriv\is{derivation}ational construction, which needs to overwrite the default active\is{construction!active construction} VAL.

The relationship between active\is{construction!active construction} and passive\is{construction!passive construction} has been investigated by \citet{stefanowitsch03collostructions} in a `collostructional\is{collostructional analysis} analysis'. Collostructional\is{collostructional analysis} analysis combines statistical data of co-occur\is{co-occurrence}rences between words with a close attention to the constructions in which these words occur. This method allows for a detailed analysis of the relations between words and constructions. Stefanowitsch \& Gries use a slightly extended collostructional\is{collostructional analysis} analysis for investigating various alternations among which the active\is{construction!active construction}-passive\is{construction!passive construction} one. The results of their study shows that there are clear semantically motivat\is{semantic motivation}ed classes of distinctive collexemes\is{collexemes} for both the active\is{construction!active construction} and the passive\is{construction!passive construction}. The most distinctive collexemes\is{collexemes} with respect to the active\is{construction!active construction} voice were {\em to have} along with emotional-mental stative verb\is{verb}s such as {\em think, say, want} and {\em mean}. With respect to the passive\is{construction!passive construction} voice, there is a clear class of verb\is{verb}s that {\em ``overwhelmingly encode processes that cause\is{causality} the patient\is{semantic role!patient} to come to be in a relatively permanent end state''} (p. 110), such as {\em base, concern} and {\em use}.

Stefanowitsch \& Gries thus confirm prior claims by \citet{pinker89learnability} and conclude that the passive\is{construction!passive construction} construction is primarily a semantic construction rather than a construction that is mainly used for marking differences in information structure\is{information structure}. In other words, there are no empirical grounds for assuming that the active\is{construction!active construction} construction is basic and that the passive\is{construction!passive construction} construction has to be deriv\is{derivation}ed from it. This observation is highly problematic for the lexical deriv\is{derivation}ations in SBCG\is{construction grammar!Sign-Based Construction Grammar} but can be captured nicely by FCG\is{construction grammar!Fluid Construction Grammar}.
\\
\\
\noindent{\bfseries Thematic hierarchy.} The above observations also make the ranked ordering of the ARG-ST of SBCG\is{construction grammar!Sign-Based Construction Grammar} highly problematic: many verb\is{verb}s apparently prefer the passive\is{construction!passive construction} construction and therefore violate the ranking more frequently than they follow it. Moreover, the universality of the thematic hier\is{thematic hierarchy}archy has become a matter of big debate due to the serious empirical problems with such hierarchies \citep{levin05argument}. In fact, many researchers in HPSG\is{Head-Driven Phrase Structure Grammar} have turned to macro-roles\is{macro-roles} or other constructs to get rid of the unsatisfactory thematic hier\is{thematic hierarchy}archies \citep{davis00linking}.

FCG\is{construction grammar!Fluid Construction Grammar} does not assume any notion of thematic hier\is{thematic hierarchy}archies, macro-roles\is{macro-roles} or universal linking rules. Instead, preference patterns for argument linking and realization emerge as a side-effect of analog\is{analogy}y in innovat\is{innovation}ion and multi-level selection\is{selection}\is{multi-level selection}: analog\is{analogy}ical reasoning explains why existing linguistic items get reuse\is{reuse}d in new situations and multi-level selection\is{selection}\is{multi-level selection} assures a growing systematicity\is{systematicity} in which linguistic items combine into a structured linguistic inventor\is{linguistic inventory}y. Recurrent patterns are hypothesized to be captured by the distributed constructions which show systematicity\is{systematicity} in their classification behaviour. I will come back to this matter in more detail in the other sections of this chapter.
\\
\\
\noindent{\bfseries The emergence\is{emergence} of \is{event structure}\is{construction!argument structure construction}argument structure constructions.} A third problem with the architecture of SBCG\is{construction grammar!Sign-Based Construction Grammar} is that it would be very hard to implement in the scenario of an emergent language or in which the first emergence\is{emergence} of grammar takes place. If there is no grammar yet, there are simply no convention\is{convention}s to build a grammar upon. Language users would nevertheless have to agree on the `basic' \is{event structure}argument structure of a lexical entr\is{lexical entry}y despite the many conflicting variation\is{variation}s floating around in the population (which are often more frequent than what would be intuitively speaking the `minimal' entry). Then they would have to agree somehow that all possible alternations are in fact deriv\is{derivation}ational constructions. On top of that, these deriv\is{derivation}ational constructions have to be nicely ordered: for example, the passive\is{construction!passive construction} alternation comes after the ditransitive\is{ditransitive clause} deriv\is{derivation}ation which comes after the lexical entr\is{lexical entry}y. Again, FCG\is{construction grammar!Fluid Construction Grammar} seems to be more flexible in this respect.

\section{Analog\is{analogy}y, multi-level selection\is{selection}\is{multi-level selection} and the constructicon}
\label{s:comp-constructicon}

In generativ\is{generative grammar}e grammar, the problem of systematicity\is{systematicity} has never been an issue since all categories and grammar rules are assumed to be innate. In construction grammars and usage-based model\is{usage-based model}s of language, however, the linguistic inventor\is{linguistic inventory}y is supposed to be acquired in a bottom-up fashion so more attention has been given to how this inventory should be structured. In this section, I will give a brief overview of the most important proposals in construction grammar based on \citet[p. 262--290]{croft04cognitive}. Next, I will argue that construction grammars need to take multi-level selection\is{selection}\is{multi-level selection} into account in how they conceive the relations between constructions in the linguistic inventor\is{linguistic inventory}y or the `constructicon'.

\subsection{The organization of the linguistic inventory}

\citet{croft04cognitive} write that the structured, linguistic inventor\is{linguistic inventory}y of a speaker {\em ``is usually represented by construction grammars in terms of a {\bfseries taxonomic network} of constructions. Each construction constitutes a {\bfseries node} in the taxonomic network of constructions}'' (p. 262). In other words, constructions are related to each other through taxonomy links or instance link\is{instance link}s which describe a relationship of schematicity between two constructions. The following example shows a taxonomic relation between the idiom\is{idiom} {\em The X-er, the Y-er\is{construction!the X-er, the Y-er construction}} and an {\em instance} of that more schematic construction:

\ea
\label{e:net1}
\Tree[.{[The X-er, the Y-er\is{construction!the X-er, the Y-er construction}]} {[The bigger they come, the harder they fall.]} ]
\\ \citep[p. 263, example 3]{croft04cognitive}
\z

The rule of thumb for deciding when a construction has its independent node in the network is when not all aspects of the construction's semantics or syntax can be deriv\is{derivation}ed from its subparts or from more schematic constructions. For example, the idiom\is{idiom} {\em to kick the bucket} has its own representation in the network because its meaning cannot be deriv\is{derivation}ed from the individual words in combination with a schematic transitive construction. Most but not all theories assume that {\em to kick the bucket} is also part of the inheritance\is{inheritance} network but which locally overrides the default behaviour of the more schematic constructions:

\ea
\label{e:net2}
\Tree [.{[Verb\is{verb}Phrase]} [.{[Verb\is{verb} Obj]} [.[{[{\em kick} Obj]} {[{\em kick} [{\em the bucket}]]} ] ] ]
\citep[p. 263, example 4]{croft04cognitive}
\z

Depending on how much  redundancy\is{redundancy} the theory allows, frequent instances can be kept as well even though a more schematic construction may already exist. Finally, sentences usually feature {\em multiple inheritance\is{inheritance}}: constructions often only offer a `partial specification' of the grammatical structures of their daughter constructions. Croft \& Cruse give the example {\em I didn't sleep}, which inherits from both the [Subject - Intransitive Verb\is{verb}] construction and the [Subject Auxiliary-n't Verb\is{verb}] construction (p. 264, example 6).

The most influential construction grammars all assume the above organization of the linguistic inventor\is{linguistic inventory}y. Croft \& Cruse discuss four of them: Berkeley Construction Grammar \citep{kay99grammatical}, the Lakoff/Goldberg model\is{construction grammar!Lakoff/Goldberg model} \citep{goldberg95construction}, Cognitive Grammar\is{construction grammar!Cognitive Grammar} \citep{langacker87foundations} and Radical Construction Grammar \citep{croft01radical}. The latter three are also considered to be usage-based model\is{usage-based model}s of language. Croft \& Cruse compare the different models based on a couple of questions, of which the following two are directly relevant for our discussion (p. 265, questions (iii) and (iv)):

\begin{enumerate}
\item What sorts of relations are found between constructions?
\item How is grammatical information stored in the construction taxonomy?
\end{enumerate}

\noindent{\bfseries Berkeley Construction Grammar.} In the discussion of Berkeley Construction Grammar (BCG) in section \ref{s:comp-bcg} I already briefly mentioned that BCG\is{construction grammar!Berkeley Construction Grammar} features an inheritance\is{inheritance} network for organizing the linguistic inventor\is{linguistic inventory}y. Unlike examples \ref{e:net1} and \ref{e:net2}, however, BCG\is{construction grammar!Berkeley Construction Grammar} does not allow for any kind of  redundancy\is{redundancy}. It is a complete inheritance\is{inheritance} model in which information is only represented once and at the highest, most schematic level possible. This also means that BCG\is{construction grammar!Berkeley Construction Grammar} does not require all constructions to be symbolic units (i.e. form-meaning mappings): they can be entirely syntactic or semantic as well.

BCG therefore captures all information in terms of taxonomy links. Since no information is stored more than once, parts of constructions can in fact be children of other parent constructions. The network thus not only has instance link\is{instance link}s between constructions, but also between parents and parts of other constructions.
\\
\\
\noindent{\bfseries The Lakoff/Goldberg model\is{construction grammar!Lakoff/Goldberg model}.} The model proposed by \citet{lakoff87woman} and \citet{goldberg95construction} focuses more on the categorization relations that may exist between constructions. Next to the taxonomy/instance link\is{instance link}s, Goldberg also proposes a meronom\is{meronomy}ic or subpart link (p. 78) and a `polysemy\is{polysemy}' link (p. 38). The subpart link is different from the BCG\is{construction grammar!Berkeley Construction Grammar} subpart links: in BCG\is{construction grammar!Berkeley Construction Grammar}, a subpart is a complete instance of a more schematic construction, whereas Goldberg sees subpart links as constructions which are subparts of larger constructions but nevertheless have an independent representation in the inventory. The `polysemy\is{polysemy}' links are links between constructions that have the same syntactic specification but different semantics.

One important aspect of the Lakoff/Goldberg model\is{construction grammar!Lakoff/Goldberg model} is the notion of a prototype\is{prototype} and (metaphorical) exten\is{extension}sion. For example, if constructions are related through polysemy\is{polysemy} links, there is always a `central sense' assumed. For the English\is{English} ditransitive\is{ditransitive clause}, this is the sense of actual transfer as in {\em I gave him a book}. Goldberg and Lakoff propose a somewhat different model when it comes to metaphorical exten\is{extension}sion: Goldberg assumes that metaphorical exten\is{extension}sion involves a superordinate schema from which the central sense and the exten\is{extension}stion(s) are instances; Lakoff does not assume such a schema.

The type of inheritance\is{inheritance} in the Lakoff/Goldberg model\is{construction grammar!Lakoff/Goldberg model} is different from BCG\is{construction grammar!Berkeley Construction Grammar} in the sense that an instance is allowed to locally overwrite some information that is normally inherited from a higher schema. For example, a schematic category such as BIRD may contain the feature FLIES, but this is not true for penguins. In the penguin-category, the inheritance\is{inheritance} of FLIES is therefore blocked by local specifications. This solution is also handy when there is conflicting information in the case of multiple inheritance\is{inheritance}: the instance is then assumed to be represented as a full entry in the inventory.
\\
\\
\noindent{\bfseries Cognitive Grammar\is{construction grammar!Cognitive Grammar}.} Langacker's Cognitive Grammar\is{construction grammar!Cognitive Grammar} (CG) is regarded by most cognitive linguists as some form of construction grammar because it shares many of its assumption\is{assumption}s and objectives. Langacker assumes that a category typically has a prototypical member or a set of members and that new instances are categorized by exten\is{extension}sion from the prototype\is{prototype}s. Next to this model of prototype\is{prototype}s and exten\is{extension}sion, CG also allows for a more schematic unit which subsumes the prototype\is{prototype} and its exten\is{extension}sions. This view comes closest to the model of exten\is{extension}sion through analog\is{analogy}y that I operationalized in Chapter \ref{c:base}.

The organization of the linguistic inventor\is{linguistic inventory}y is dependent on language use. The entrench\is{entrenchment}ment or independent representation of a linguistic item is hypothesized to depend on its token frequency\is{token frequency}\is{frequency}: if a unit occurs frequently enough, it is stored in memory. Productivity\is{productivity} of a linguistic unit goes hand in hand with its exten\is{extension}sion through language use: if a (prototypical) category gets exten\is{extension}ded to new situations, it increases its type frequency\is{type frequency}\is{frequency} and hence its productivity\is{productivity}. As said before, categories can form a network based on prototypical members (instances) and non-prototypical members, but there are also abstraction\is{abstraction}s which are related to their members through taxonomy links.
\\
\\
\noindent{\bfseries Radical Construction Grammar.} The word `radical' in Radical Construction Grammar (RCG\is{construction grammar!Radical Construction Grammar}) comes from the fact that RCG\is{construction grammar!Radical Construction Grammar} does not assume constructions to be built from atomic categories such as noun\is{noun}s or verb\is{verb}s, but rather that the construction is the atomic unit of language. All other categories are defined in terms of the constructions they occur in. Categories are thus assumed to be construction- and language-specific. For example, the transitive construction and intransitive construction are hypothesized to contain two different verb\is{verb} categories: the transitive verb\is{verb} and the intransitive verb\is{verb}. The superordinate category Verb\is{verb} is seen as a linguistic abstraction\is{abstraction} over those two categories \citep[p. 287--288]{croft04cognitive}:

\ea
\Tree[.{[MVerb\is{verb}-TA]} {[IntrSbj IntrV]} {[TrSbj TrV TrObj]} ]
\z

In the above example, the label MVerb\is{verb} is used to indicate that this is a morphological construction (TA stands for Tense\is{tense} and Aspect\is{aspect}). The superordinate abstraction\is{abstraction} can only be made if it is linguistically motivat\is{semantic motivation}ed. For example, both transitive and intransitive verb\is{verb}s can be marked for tense\is{tense} and aspect\is{aspect} so both categories should be able to occur in those Tense-Aspect constructions. In short, RCG\is{construction grammar!Radical Construction Grammar} is a strongly non-reductionis\is{reductionism}t approach as opposed to BCG\is{construction grammar!Berkeley Construction Grammar}.

RCG\is{construction grammar!Radical Construction Grammar} features the same taxonomy links as other construction grammars and also allows for redundant information according to the principles of usage-based model\is{usage-based model}s of language. One other important aspect of RCG\is{construction grammar!Radical Construction Grammar} is that it is based on the `semantic map\is{semantic maps}' model (see section \ref{s:comp-semmaps}). In this model, all constructions are hypothesized to map onto contiguous regions in `conceptual space\is{conceptual space}' which is assumed to be universal. Finally, syntactic structures are defined as language-specific units but in relation to `syntactic space\is{syntactic space}' which aims at typologically comparing the world's languages.

\subsection{The inventor\is{linguistic inventory}y in Fluid Construction Grammar\is{construction grammar!Fluid Construction Grammar}}

Before I start the comparison between Fluid Construction Grammar\is{construction grammar!Fluid Construction Grammar} and the above theories, I would like to emphasize again that FCG\is{construction grammar!Fluid Construction Grammar} takes a design stance\is{design stance} towards the emergence of grammar and that it therefore only implements mechanisms that are experimentally demonstrated to be necessary requirements. The fact that FCG\is{construction grammar!Fluid Construction Grammar} does not make the same abstraction\is{abstraction}s or does not feature the same complex mechanisms for organizing the linguistic network therefore does not mean that they are refuted, but only that they are not necessary (yet) for the level of complexity that is reached in current simulations. On the other hand, FCG\is{construction grammar!Fluid Construction Grammar} can show which proposals stand the computationally rigid test in less complex languages. Secondly, by demonstrating novel but necessary mechanisms in those less complex languages, FCG\is{construction grammar!Fluid Construction Grammar} can show which ideas are currently being overlooked by linguistic theories.
\\
\\
\noindent{\bfseries The emergence of linguistic categories.} With respect to the `atomic' building blocks of a grammar in emergence, FCG\is{construction grammar!Fluid Construction Grammar} is closest related to Radical Construction Grammar. From an evolutionary point-of-view, it is more natural to think of constructions or form-meaning mappings as the atomic units in language and that other categories are dependent on the organization of these constructions. For example, the experiments do not feature an explicit category for noun\is{noun}s or verb\is{verb}s, yet all words can be used without any problem in \is{event structure}\is{construction!argument structure construction}argument structure constructions. Further categorizations should be functionally motivat\is{semantic motivation}ed. For example, if the agents should also worry about tense\is{tense} and aspect\is{aspect} marking, they might need additional generaliz\is{generalization}ations over their existing constructions.

This scenario is attractive in many ways. First, the agents do not need to agree on a set of building blocks such as noun\is{noun}s or verb\is{verb}s before they can start combining them into sentences. Instead they keep on constructing new categories on the fly but only when this optimizes communication and thus when it is functionally motivat\is{semantic motivation}ed. This approach also seems to fit natural languages better since it is impossible to come up with an abstract rule that can be applied to all parts of speech of a language. Finally, this approach also suits my proposal of potential valent\is{valency!potential valents}s for linguistic items: the freer and typically lexical items can be potentially used in many different constructions, whereas the more grammaticalized, tightened constructions typically decide on the actual valen\is{valency!actual valency}cy of a linguistic expression.

FCG\is{construction grammar!Fluid Construction Grammar} therefore rejects the reductionis\is{reductionism}t approach of BCG\is{construction grammar!Berkeley Construction Grammar} (and SBCG\is{construction grammar!Sign-Based Construction Grammar}). Reductionis\is{reductionism}t approaches are still dominant in linguistics as a result of a desire for maximizing `storage parsimon\is{parsimony}y' in the linguistic inventor\is{linguistic inventory}y. \citet[p. 278]{croft04cognitive}, however, point to psychological evidence that suggests that storage parsimon\is{parsimony}y is a cognitively implausible criterion for modeling the linguistic inventor\is{linguistic inventory}y. Language users rather seem to store a lot of redundant information which requires more memory but which optimizes  `computing parsimon\is{parsimony}y' because not all information has to be computed online.
\\
\\
\noindent{\bfseries Innovation through analog\is{analogy}y and pattern formation\is{pattern formation}\is{formation}.} Fluid Construction Grammar\is{construction grammar!Fluid Construction Grammar} also subscribes the usage-based model\is{usage-based model} and argues that innovat\is{innovation}ion occurs through analog\is{analogy}ical reasoning. In the experiments of this book, I implemented an innovat\is{innovation}ion strategy in which the productivity\is{productivity} of a category is related to its type frequency\is{type frequency}\is{frequency} and which is therefore similar to proposals made in Cognitive Grammar\is{construction grammar!Cognitive Grammar}. FCG\is{construction grammar!Fluid Construction Grammar} also allows for careful abstraction\is{abstraction} in which an instance link\is{instance link} is created between the more abstract category and the specific instances that are compatible with it. A second drive for innovat\is{innovation}ion is pattern formation\is{pattern formation}\is{formation}: frequently co-occur\is{co-occurrence}ring utterances are stored as independent units in memory. This is also completely in line with usage-based model\is{usage-based model}s that take token frequency\is{token frequency}\is{frequency} as an indicator of entrench\is{entrenchment}ment. The newly formed patterns themselves may be exten\is{extension}ded through analog\is{analogy}y as well.

One salient feature of FCG\is{construction grammar!Fluid Construction Grammar} is that all innovat\is{innovation}ion occurs in a stepwise fashion. If a careful abstraction\is{abstraction} is made, it is at that moment only valid for the instances that were used in creating the abstraction\is{abstraction}. The newly formed category therefore does not automatically exten\is{extension}d its use to other situations: an explicit link in the network has to be created during other interactions. For pattern formation\is{pattern formation}\is{formation} as well, links are kept between the newly created pattern and its subparts. All the links in the inventory are used for optimizing linguistic processing: instead of considering the entire memory, only linked constructions are unified and merged. Only when this strategy leads to communicative problems, the language user will try to adapt the inventory through analog\is{analogy}y.
\\
\\
\noindent{\bfseries Multi-level selection\is{selection}\is{multi-level selection} in the emergence\is{emergence} of language systematicity\is{systematicity}.} The work in this book has also uncovered the problem of systematicity\is{systematicity} which has so far been overlooked by all linguistic theories. The usage-based model\is{usage-based model}s presented in the previous section mainly focus on a top-down inheritance\is{inheritance} network and seem to assume that this suffices for reaching and maintaining systematicity\is{systematicity} if the network is combined with an innovat\is{innovation}ion strategy based on type frequency\is{type frequency}\is{frequency} and productivity\is{productivity}.

The experiments in Chapter \ref{c:experiment1}, however, demonstrate that this is not the case. Next to an innovat\is{innovation}ion strategy which systematically reuse\is{reuse}s productive and successful items of the inventory, language users need an alignment strateg\is{alignment strategy}y based on multi-level selection\is{selection}\is{multi-level selection} to further streamline their inventories and keep the generaliz\is{generalization}ation rate of their language high. The experiments demonstrated that a top-down strategy does not suffice but that the success and evolution\is{evolution!cultural evolution} of specific instances must also have a way to influence the more schematic constructions in the network. The networks therefore need {\bfseries systematicity\is{systematicity} links} rather than (only) taxonomy links.
\\
\\
\noindent{\bfseries On the status of inheritance\is{inheritance} networks.} The experiments on multi-level selection\is{selection}\is{multi-level selection} show how a linguistic network similar to the one proposed in Radical Construction Grammar could gradually emerge: a nonreductionis\is{reductionism}t approach is taken in which each construction has its specific categories. However, FCG\is{construction grammar!Fluid Construction Grammar} does not make explicit generaliz\is{generalization}ations over these constructions as is done in RCG\is{construction grammar!Radical Construction Grammar}, but rather keeps systematicity\is{systematicity} links which are used by the multi-level selection\is{selection}\is{multi-level selection} alignment strateg\is{alignment strategy}y. There is no need for an inheritance\is{inheritance} network and all utterances are license\is{license}d by unify\is{unify and merge}ing and merging fully specified constructions.

This architecture suffices for the kinds of experiments performed in this book and only further work can show whether additional abstraction\is{abstraction}s and perhaps inheritance\is{inheritance} networks are really needed. A serious challenge to these kinds of abstraction\is{abstraction}s and inheritance\is{inheritance} networks is posed by the successful application of instance-based\is{exemplar-based models} models in natural language processing\is{natural language processing} such as Memory-Based Language Processing \citep{daelemans05memory} and Analog\is{analogy}ical Modeling \citep{skousen89analogical}. Another challenge for inheritance\is{inheritance} networks, I believe, is that they might require abstraction\is{abstraction}s that are too greedy and therefore harmful to the communicative success\is{communicative success} of language users, especially in experiments on the emergence of grammar. As becomes very clear in such experiments, agents have to deal with an enormous amount of uncertainty about the convention\is{convention}s in their population. It might very well turn out to be that a fully redundant model (with or without careful abstraction\is{abstraction}) using multi-level selection\is{selection}\is{multi-level selection} is a more adequate model.

\section{Linguistic typology\is{linguistic typology} and grammaticalization\is{grammaticalization}}
\label{s:comp-semmaps}

The previous two sections mainly dealt with the relations between construction grammar and the experiments in this book. In this section, I will discuss how the methodology of artificial language evolution\is{artificial language evolution} can provide novel insights to the fields of grammaticalization\is{grammaticalization} and linguistic typology\is{linguistic typology}.

\subsection{The status of semantic map\is{semantic maps}s}
\label{s:semantic-maps}

Semantic maps have offered linguists an appealing and empirically rooted methodology for visualizing the multifunctional nature of grammatical categories and for describing recurrent structural patterns in how these functions relate to each other. Consider the following examples in which various functions of the English\is{English} preposition\is{preposition} {\em to} are illustrated along with some corresponding examples from the French\is{French} preposition\is{preposition} {\em \`{a}} and the German\is{German} dative\is{case!dative} case \citep[taken from][example 2, p. 212 and example sentences on p. 213--215]{haspelmath03geometry}:

\ea
English\is{English} preposition\is{preposition} {\em to}:
\label{e:to-dative}
\begin{tabbing}
a. \hspace{0,3cm}    \= Goethe went to Leipzig as a student.  \hspace{1cm} \= (direction\is{semantic role!direction})\\
b. \> Eve gave the apple to Adam. \> (recipient\is{semantic role!recipient})\\
c. \> This seems outrageous to me. \> (experiencer\is{semantic role!experiencer})\\
d. \> I left the party early to get home in time. \> (purpose\is{semantic role!purpose})\\
e. \> This dog is (mine/*to me). \> (predicative possessor\is{semantic role!predicative possessor}\is{semantic role!possessor})\\
d. \> I'll buy a bike (for/*to) you. \> (beneficiary\is{semantic role!beneficiary}) \\
e. \> That's too warm (for/*to) me. \> (dative\is{case!dative} judicantis\is{semantic role!dative judicantis})\\
\end{tabbing}
\item French\is{French} preposition\is{preposition} {\em \`{a}}:
\\
\gll Ce chien est \`{a} moi.\\
this dog is.3SG to me\\
\glt `This dog is mine.' (predicative possessor\is{semantic role!predicative possessor}\is{semantic role!possessor})\\

\item German\is{German} dative\is{case!dative} case:
\\
\gll Es ist mir zu warm.\\
it is.3SG I.DAT too warm\\
\glt `It's too warm for me.' (dative\is{case!dative} judicantis\is{semantic role!dative judicantis})\\
\z

\noindent{\bfseries The universality of semantic map\is{semantic maps}s.} Instead of listing the various functions of a grammatical morpheme or `gram', semantic map\is{semantic maps}s offer a {\em ``geometrical representation of functions in ``conceptual/semantic'' space that are linked by connecting lines and thus constitute a network'' \citep[p. 213]{haspelmath03geometry}.} Figure \ref{f:semmap-to} gives an example of a semantic map\is{semantic maps} which shows some typcal functions for the dative\is{case!dative} case. This map features a network of seven nodes which each represent a grammatical function. The Figure also illustrates that the English\is{English} preposition\is{preposition} {\em to} covers four of these functions (as was shown in example \ref{e:to-dative}): purpose\is{semantic role!purpose}, direction\is{semantic role!direction}, recipient\is{semantic role!recipient} and experiencer\is{semantic role!experiencer}. It does not cover the functions beneficiary\is{semantic role!beneficiary}, predicative possessor\is{semantic role!predicative possessor}\is{semantic role!possessor} (if preposition\is{preposition}al verb\is{verb}s are not counted as in {\em the dog belongs to me}) and dative\is{case!dative} judicantis\is{semantic role!dative judicantis}.
\begin{figure}[t]
\centerline{\includegraphics[width=\textwidth]{Chapter5/figs/semmap-to}}
  \caption[A semantic map\is{semantic maps} of dative\is{case!dative} functions for {\em to} \citep{haspelmath03geometry}]{This partial semantic map\is{semantic maps} compares the French\is{French} preposition\is{preposition} {\em \`{a}} to the English\is{English} preposition\is{preposition} {\em to} with respect to which typical dative\is{case!dative} functions they cover. Non-dative\is{case!dative} functions are ignored in this map \citep[adapted from][figures 8.1 and 8.2, p. 213 and 215]{haspelmath03geometry}.}
   \label{f:semmap-to}
\end{figure}

Semantic maps depend crucially on cross-linguistic research. For example, a node in the network is only added if at least one language is found which makes the distinction. Haspelmath gives the example of direction\is{semantic role!direction} versus recipient\is{semantic role!recipient} (p. 217). Based on English\is{English} and French\is{French}, which use one preposition\is{preposition} for
both functions, this distinction could not be made. However, German\is{German} uses {\em zu} or {\em nach} for direction\is{semantic role!direction}, whereas it uses the dative\is{case!dative} case for recipient\is{semantic role!recipient}. A large sample set of languages is therefore needed to uncover all the uses of a gram.

Another important aspect of semantic map\is{semantic maps}s is the connection between nodes in the network. The map must represent these nodes in a contiguous area on the map. Haspelmath writes that based on the English\is{English} preposition\is{preposition} {\em to}, for example, the following three orders could be possible for purpose\is{semantic role!purpose}, direction\is{semantic role!direction} and recipient\is{semantic role!recipient} (p. 217, example 4):

\eal
\ex[]{purpose\is{semantic role!purpose} -- direction\is{semantic role!direction} -- recipient\is{semantic role!recipient}}
\ex[]{direction\is{semantic role!direction} -- purpose\is{semantic role!purpose} -- recipient\is{semantic role!recipient}}
\ex[]{ direction\is{semantic role!direction} -- recipient\is{semantic role!recipient} -- purpose\is{semantic role!purpose}}
\zl

Again, data from other languages are taken into account for choosing which option can be eliminated. Since the French\is{French} preposition\is{preposition} {\em \`{a}} cannot be used for marking purpose\is{semantic role!purpose}, option (b) cannot represent a contiguous space in the network. The German\is{German} preposition\is{preposition} {\em zu} eliminates option (c) because it can express purpose\is{semantic role!purpose} and direction\is{semantic role!direction}, but not recipient\is{semantic role!recipient}. The direct connections between functions on the semantic map\is{semantic maps} are important because they are hypothesized to be universal:

\begin{quote}
Semantic maps not only provide an easy way of formulating and visualizing differences and similarities between individual languages, but they can also be seen as a powerful tool for discovering universal semantic features that characterize the human language capacity. Once a semantic map\is{semantic maps} has been tested on a sufficiently large number of languages [...] from different parts of the world, we can be reasonably confident that it will indeed turn out to be universal. \citep[p. 232]{haspelmath03geometry} 
\end{quote}

This view is shared by many other linguists, among whom Bill Croft. Croft's {\em Semantic Map Connectivity Hypothesis\is{Semantic Map Connectivity Hypothesis}} \citep[p. 96]{croft01radical} states that the functions of a particular construction will always cover functions that are connected regions in {\em conceptual space\is{conceptual space}}. In other words, grammatical categories are language-particular, but they are based on a universal conceptual/semantic space.

The universality of semantic map\is{semantic maps}s is however an issue of debate. For example, \citet{cysouw08building} reports on his attempts at making a satisfying map for person marking. He concludes that there is no single `universal' semantic map\is{semantic maps}. Instead, different semantic map\is{semantic maps}s are possible depending on the level and granularity of the analysis. Cysouw therefore calls for using semantic map\is{semantic maps}s as a tool for modeling attested linguistic variety and as a way to predict {\em probable} languages rather than {\em possible} languages by weighting the function nodes in the network depending on the number of attested cases.

Cysouw thus points to a serious problem of the semantic map\is{semantic maps} hypothesis: what grain-size is acceptab\is{acceptability}le for making semantic map\is{semantic maps}s? For example, \citet{haspelmath03geometry} uses functions such as `recipient\is{semantic role!recipient}' and `beneficiary\is{semantic role!beneficiary}' as primitive categories for his analysis. However, these functions are language-specific and no grammatical category has been demonstrated to cover all possible instantiations of such a function. Instead, languages tend to have many exceptions, irregularities or a redundant overlap in categories that mark that function. For example, `recipient\is{semantic role!recipient}' and `beneficiary\is{semantic role!beneficiary}' not only occur with the preposition\is{preposition}s {\em to} and {\em for} respectively, they can also take the first object\is{syntactic role!object} position in the English\is{English} ditransitive\is{ditransitive clause}.

The universality hypothesis therefore faces a problem of circularity. On the one hand, semantic map\is{semantic maps}s are hypothesized to represent universal conceptual space\is{conceptual space}; on the other hand, that conceptual space\is{conceptual space} is based on an analysis which ignores language-internal differences and irregularities, and the languages that do not mark any differences are still assumed to have the same underlying functions.

Artificial language evolution\is{artificial language evolution} could demonstrate an alternative hypothesis to explain the universal tendencies in grammatical marking. In problem-solving\is{problem-solving} models such as this book, grammatical evolution\is{evolution!cultural evolution} is a consequence of distributed processes whereby language users shape and reshape their language. The main challenge is therefore to find out what these processes are and under what circumstances they could create the kind of semantic map\is{semantic maps}s that are observed for human languages. The hypothesis is that these processes suffice for the emergence\is{emergence} of semantic map\is{semantic maps}s and that conceptual space\is{conceptual space} is dynamically configured in co-evolution with grammar. Semantic maps of different languages will naturally show similarities and differences depending on whether they followed the same evolutionary pathways or not.
\\
\\
\noindent{\bfseries Prior work on concept\is{concept formation} emergence.} As mentioned in section \ref{s:history-of-research}, prior work in the field has already demonstrated how a population\is{speech population} of agents could self-organiz\is{self-organization}e a shared ontology through communicative interactions. \citet{steels97constructing} reports the first experiments in which conceptualization\is{conceptualization} and lexicon\is{lexicon} emergence are coupled to each other. In the experiment, a population\is{speech population} of artificial agents take turns in playing `guessing game\is{language game!guessing game}s': the speaker chooses one of the objects in the context to talk about and wants to draw the hearer's attention to it by saying a word. The game is a success if the hearer points to the correct object. If the game fails, the speaker will point to the intended topic and the hearer tries to guess what the speaker might have meant with his word. The agents start without any language and even without an ontology. Instead, they are equipped with several sensory channels for perceiving the objects in their environment.

At the start of a game, two agents are randomly chosen from the population\is{speech population} to act as a speaker and as a hearer. The speaker chooses an object from the context to talk about and needs to conceptualize a meaning which discriminates the topic from the other objects in the context. For example, if the topic is a green ball and there are also three red balls in the context, then the topic's colour\is{colour} would be a good discriminating feature. At the beginning of the experiment, the agents have no concepts yet so the speaker has to create a new one. He will do so by taking the minimal set of features that can discriminate the topic from the other objects. The speaker will then invent a new word for this concept\is{concept formation} or meaning and transmit it to the hearer.

The hearer will in turn experience a communicative problem: He does not know the word that was used by the speaker. The game thus fails, but the speaker points to the intended object. The hearer then tries to retrieve the intended meaning through the same discrimination game\is{discrimination game}. Often there are many different sets of discriminating features possible, but if the agents play a sufficient amount of language game\is{language game}s with each other, they come to an agreement on what the form-meaning pairs are in their language and thus also reach a shared conceptual space\is{conceptual space}. Similar experiments have been successfully performed in the domains of colour\is{colour} terms \citep{steels05coordinating} and spatial\is{spatial language} language \citep{steels08perspective-alignment}, and they have been scaled up to large meaning spaces \citep{wellens08coping}.
\begin{figure}[t]
\centerline{\includegraphics[scale=0.7]{Chapter5/figs/sem-map1}}
  \caption[Two `semantic map\is{semantic maps}s' from the experiments]{This diagram compares two different artificial grammars with respect to two categories in each of them. The languages were formed using the final set-up of experiment 3 (section \ref{s:pattern-exp-3}). Even though the agents did not have a continuous conceptual space\is{conceptual space} in advance, it is nevertheless possible to draw a primitive semantic map\is{semantic maps} afterwards.}
   \label{f:semmap-1}
\end{figure}
\\
\\
\noindent{\bfseries The contribution of this book.} All of the above experiments confirm that communicative success\is{communicative success} can be a driving force for constructing an ontology of meaningful distinctions and that language can be used as a way to agree on a shared ontology among a population\is{speech population} of autonomous embodied artificial agents. These experiments, however, have only focused on concept-and-lexicon\is{lexicon} emergence so far, and the systematic relations between words have not been investigated yet. The experiments in this book, however, have polysemous semantic role\is{semantic role}s so they form an ideal starting point for testing the alternative hypothesis.

Figure \ref{f:semmap-1} illustrates how analog\is{analogy}ical reasoning can be responsible for constructing coherent classes of semantic role\is{semantic role}s. The diagrams shows two semantic map\is{semantic maps}s for two languages that were formed in the last set-up of experiment 3 as described in section \ref{s:pattern-exp-3} (multi-level selection\is{selection}\is{multi-level selection} with memory decay\is{memory decay} and pattern formation\is{pattern formation}\is{formation}). Both diagrams show that it is possible to draw a primitive semantic map\is{semantic maps} which compares the semantic role\is{semantic role}s of both languages. For example, in one language the marker\is{case!case marking} {\em -mepui} can be used for covering four participant role\is{participant role}s. Three of them (grasp-1, touch-1 and take-2) overlap with a semantic role\is{semantic role} of a different language. A similar observation counts for the two semantic role\is{semantic role}s in the second semantic map\is{semantic maps}.

A comparison of the formed artificial languages suggests that grammaticalization\is{grammaticalization} processes can be visualized as a movement or change in connected regions of a continuous domain as a side-effect of analog\is{analogy}ical reasoning: exten\is{extension}sion of a category happens when new situations are encountered which are closely related to the existing categories. This shows that semantic map\is{semantic maps}s could in principle be the result of dynamic processes involving analog\is{analogy}y rather than starting from universal conceptual space\is{conceptual space}.

The alternative proposed here needs further investigation and essentially requires a significant scale-up in terms of the meaning space and world environment as well as the conceptualization\is{conceptualization} capabilities of the agents. The present results are however encouraging and the proposed alternative has the advantage that it is more adaptive and open-ended to a changing environment: a universal conceptual space\is{conceptual space} would still require some mechanism of mapping culture-specific developments (such as buying and selling, driving cars, and steering airplanes) onto a prewired structure. If the alternative hypothesis is followed, semantic map\is{semantic maps}s would thus not point to a universal map of human cognition but rather to recurrent patterns in human experience and preferred developmental pathways followed by dynamic categorization mechanisms.

\subsection{Thematic hierarchies in case systems\is{case!case system}}
\label{s:comp-thematic}

\begin{figure}[t]
\centerline{\includegraphics[width=\textwidth]{Chapter5/figs/wals.jpg}}
  \caption[Alignment of case marking of full noun\is{noun} phrases \citep{comrie05wals}]{The alignment of case marking\is{case!case marking} of full noun\is{noun} phrases \citep[98]{comrie05wals}.}
   \label{f:wals}
\end{figure}

Many linguistic theories assume that argument linking is governed by a universal thematic hierarchy \citep[e.g.][]{dik97functional, fillmore68case, givon01syntax, jackendoff90semantic, keenan77noun}. However, empirical evidence shows that such hierarchies can offer tendencies at best, and that they cannot be considered as innate knowledge \citep{levin05argument}. Even for language-specific argument linking patterns, no satisfying hierarchy\is{hierarchy} has been found yet.

The question of how language systematicity\is{systematicity} can ever arise becomes a big issue if no universal hierarchy\is{hierarchy} can be found, especially if no Universal Grammar\is{Universal Grammar} is assumed. The map in Figure \ref{f:wals}, for example, shows the alignment of case marking\is{case!case marking} of full noun\is{noun} phrases across 190 languages. It clearly demonstrates strong systematicity\is{systematicity} in the marking of `core arguments' in these languages. \citet{comrie05wals} distinguishes five different systems\is{case!case system} (I count the two variants of nominative\is{case!nominative}-accusative\is{case!accusative} systems\is{case!case system} as one):

\begin{itemize}
\item Neutral: the subject\is{syntactic role!subject} of intransitive clause\is{intransitive clause}s (S) is marked in the same way as both the subject\is{syntactic role!subject} (A) and object\is{syntactic role!object} (P) of transitive clause\is{transitive clause}s. Example: Mandarin.
\item Nominative-accusative\is{case!accusative}: A and S are marked in the same way (nominative\is{case!nominative} marking). P is marked differently (accusative\is{case!accusative} marking). Example: Latvian\is{Latvian}.
\item Ergative-absolutive: S and P are marked in the same way (ergative marking), A is marked differently (absolutive marking). Example: Hunzib\is{Hunzib}.
\item Tripartite\is{case!tripartite case system}: S, A and P are all marked differently. Example: Hindi\is{Hindi}.
\item Active-inactive\is{active-stative languages}\is{construction!active construction}: There is a different marker\is{case!case marking} for an agentive S (aligning with A) and a patient\is{semantic role!patient}ive S (aligning with P). Example: Georgian\is{Georgian}. 
\end{itemize}

The answer for most linguists is again sought in universals. For example, \citet{croft98event} assumes a universal conceptual space\is{conceptual space} and universal linking rules for mapping arguments to core syntactic cases. The problem here is again that the proposals only work for analyses that do not go beyond the crude representation of case marking\is{case!case marking} systems\is{case!case system} as presented in Figure \ref{f:wals}. Closer studies show that the proposed systems\is{case!case system} are again only tendencies in each language and that there are lots of exceptions to the `default' alignment of case marking\is{case!case marking}. Also the typological variation\is{variation} across languages is greater than suggested by the traditional SAP-system of core arguments \citep{mithun05beyond}.
\\
\\
\noindent{\bfseries Analog\is{analogy}y, pattern formation\is{pattern formation}\is{formation} and multi-level selection\is{selection}\is{multi-level selection}.} In the case of thematic hier\is{thematic hierarchy}archies, a similar alternative can be devised based on the distributed processes whereby language users shape and reshape their language for communication. As I argued in Chapter \ref{c:base}, generaliz\is{generalization}ation of grammatical categories arises as a side-effect within inferential coding system\is{inferential coding system}s: language users want to increase their communicative success\is{communicative success} and when speakers have to solve a problem or innovat\is{innovation}e, they will try to do this in such a way that the intended communicative effect is still reached. By exploiting analog\is{analogy}y, the speaker can hook the new situation up to previous convention\is{convention}s which are probably known by the hearer as well. The hearer can then retrieve the intended meaning through the same mechanisms of analog\is{analogy}ical reasoning.

As categories get reuse\is{reuse}d more often, they increase their type frequency\is{type frequency}\is{frequency} and hence their productivity\is{productivity}. An additional factor that boosts the success of such a category is when it starts to form patterns or groups with other elements in the inventory. A multi-level selection\is{selection}\is{multi-level selection} alignment strateg\is{alignment strategy}y then assures that certain categories can also survive and reoccur in multiple levels of the linguistic network which again increases their frequency\is{frequency} and chances of survival. Multi-level selection\is{selection}\is{multi-level selection} could thus explain how different constructions align their categories with each other as demonstrated in the map in Figure \ref{f:wals}.

Preferences in argument linking, as predicted by thematic hier\is{thematic hierarchy}archies, could thus gradually emerge as a side-effect of these mechanisms: as certain categories become more and more dominant and productive, they can start to exten\is{extension}d their use across patterns and eventually evolve into prototypical subject\is{syntactic role!subject} and object\is{syntactic role!object} categories (as I also suggested in section \ref{s:stage4}). The many subregularities that are observed in languages are no problem in this model and are in fact predicted because everything has to emerge in a bottom-up fashion. Further experiments on the emergence of syntactic cases could thus be the starting point for modeling this alternative to thematic hier\is{thematic hierarchy}archies.

\subsection{A redundant approach to grammaticalization\is{grammaticalization}}
\label{s:actualization}

A third debate in which artificial language evolution\is{artificial language evolution} can offer novel insights is grammaticalization\is{grammaticalization} theory. As I already mentioned in section \ref{s:stage2}, one of the problems of grammaticalization\is{grammaticalization} is that linguists can usually only detect language change\is{language change} once the processes of grammaticalization\is{grammaticalization} have already taken place. It is therefore difficult to hypothesize what mechanisms should be proposed to explain such changes especially since the consequences of communicative interactions in larger population\is{speech population}s are often overlooked. Multi-agent simulations can thus demonstrate which mechanisms are better suited for dealing with innovat\is{innovation}ions, variation\is{variation}s, and propagat\is{propagation}ions of linguistic convention\is{convention}s.
\\
\\
\noindent{\bfseries Reanalysis and actualization\is{actualization}.} Diachronic reanalysis\is{reanalysis} has taken a foreground position in traditional grammaticalization\is{grammaticalization} theory. For example, \citet{hopper93grammaticalization} write: {\em ``Unquestionably, reanalysis\is{reanalysis} is the most important mechanism for grammaticalization\is{grammaticalization}''} (p. 32). Reanalysis is understood as a {\em ``change in the structure of an expression or class of expressions that does not involve any immediate or intrinsic modification of its surface manifestation}'' \citep[p. 59]{langacker77syntactic}. In other words, reanalysis\is{reanalysis} is not noticeable from the surface form but only has consequences for the grammar at a later stage. Many theories therefore posit another mechanism called `actualization' that maps out the consequences of reanalysis\is{reanalysis} \citep{timberlake77reanalysis}.

Reanalysis is typically illustrated by the grammaticalization\is{grammaticalization} of {\em be going to} into {\em gonna} \citep[p. 2--4]{hopper93grammaticalization}. In an older use of {\em be going to}, {\em to} was part of a purposive direction\is{semantic role!direction}al complement\is{complement} as in {\em I am going to marry Bill} meaning `I am going/travelling in order to marry Bill'. At a later stage, {\em to} is hypothesized to be reanalysed as belonging to {\em be going} instead of to the complement\is{complement}. In other words, rebracketing\is{rebracketing} of the structure has taken place from [[I] [am going] [to marry Bill]] to [[I] [am going to] [marry Bill]].

Reanalysis\is{reanalysis} has recently been challenged. \citet{haspelmath98does} writes that reanalysis\is{reanalysis} does not entail\is{entailment} a loss of autonomy which is typical for grammaticalization\is{grammaticalization} and that grammaticalization\is{grammaticalization} is (almost exclusively) unidirectional instead of bidirectional\is{bidirectionality} as predicted by reanalysis\is{reanalysis}. Haspelmath also rejects the combination of reanalysis\is{reanalysis} with actualization\is{actualization} which is often used as a way to assign gradualness to reanalysis\is{reanalysis} (p. 340--341). Actualization makes `reanalysis\is{reanalysis}' as a mechanism impossible to verify and it requires speakers to know at least two analyses of the same construction to account for both the old and the new behaviour. Actualization also does not explain how innovat\is{innovation}ions might propagat\is{propagation}e. Haspelmath's comparison of `grammaticalization' and `reanalysis\is{reanalysis}' is summarized in Table \ref{t:grammaticalization}.
\begin{table}
\centerline{\begin{tabular}{l l}
\hline
{\em Grammaticalization}\hspace{2,5cm} & {\em Reanalysis}
\\
\hline
loss of autonomy / substance & no loss of autonomy / substance
\\
gradual & abrupt
\\
unidirectional & bidirectional\is{bidirectionality}
\\
no ambiguity\is{ambiguity} & ambiguity\is{ambiguity} in the input structure
\\
due to language use & due to language acquisition\is{acquisition}
\\
\hline
\end{tabular}}
\caption[Major differences between grammaticalization\is{grammaticalization} and reanalysis \citep{haspelmath98does}]{This table shows the major differences between grammaticalization\is{grammaticalization} and reanalysis\is{reanalysis} \citep[p. 327, Table 1]{haspelmath98does}.}
\label{t:grammaticalization}
\end{table}

Despite the problems of reanalysis\is{reanalysis}, it seems hard to conceive an alternative process that could explain certain changes. Haspelmath suggests that formal theories should implement the gradience of membership of word classes in some way such as {\em ``V$_{1.0}$ for ordinary verb\is{verb}s, V$_{.7}$/P$_{.3}$ for preposition\is{preposition}-like verb\is{verb}s (e.g.} considering{\em) and so on''} (p. 330). Even though gradience is indeed an important matter, such a proposal cannot capture the fact that the old use of a linguistic item and its new function can co-exist\is{co-existence} for hundreds of years in a language. The alternative that I would propose is  redundancy\is{redundancy} and pattern formation\is{pattern formation}\is{formation} along the lines of my example for French\is{French} predicate negation\is{negation} in Chapter \ref{c:experiment1}. Applied to the example of {\em gonna}, this alternative would simply state that the frequent co-occur\is{co-occurrence}rence of the words {\em be going to} led to the creation of a pattern for optimizing linguistic processing. Once this pattern is created, it may start evolving on its own which allows it to gradually drift away from the original use of the words.
\\
\\
\noindent{\bfseries Example: the English\is{English} verb\is{verb}al gerund\is{gerund}.} To illustrate this alternative approach, I will briefly take a look at the English\is{English} verb\is{verb}al gerund\is{gerund} which historically developed from a deverbal nominalization\is{nominalization} and which later acquired more and more verb\is{verb}al properties. I will show examples of this development taken from \citet{fanego04reanalysis} and summarize how he describes this grammaticalization\is{grammaticalization} process in terms of `reanalysis\is{reanalysis}' and `actualization'. Next, I will argue for a simpler model based on  redundancy\is{redundancy} and pattern formation\is{pattern formation}\is{formation}.

The English\is{English} gerund\is{gerund} is a unique category in European languages in the sense that it is a third type of verb\is{verb}al complement\is{complement} besides  to-infinitive\is{to-infinitive}s (example \ref{e:gerund1}) and finite\is{finiteness} clauses (\ref{e:gerund2}). The present-day English\is{English} gerund\is{gerund} has the following verb\is{verb}al properties: it can take a direct object\is{syntactic role!object} (\ref{e:gerund3}), it can be modified by adverb\is{adverb}s (\ref{e:gerund4}), it can mark tense\is{tense}, aspect\is{aspect} and voice distinctions (\ref{e:gerund5}), it can be negated using the predicate negator {\em not} (\ref{e:gerund6}) and it can take a subject\is{syntactic role!subject} in a case other than the genitive\is{case!genitive} (\ref{e:gerund7}).

\ea
\label{e:gerund1}
I just called \emph{to say} `I love you'.
\item Just tell him \emph{we're not interested anymore}.
\label{e:gerund2}
\item By writing \emph{a book}, he managed to face all his inner demons.
\label{e:gerund3}
\item My \emph{quietly} leaving before anyone noticed.
\label{e:gerund4}
\item The necessity of \emph{being loved} is a driving force in our lives.
\label{e:gerund5}
\item My \emph{not} leaving the room caused a stir.
\label{e:gerund6}
\item We should prevent \emph{the treaty} taking effect.
\label{e:gerund7}
\z

Studies on the emergence\is{emergence} and evolution\is{evolution!cultural evolution} of the gerund\is{gerund} suggest that it developed from a deverbal nominalization\is{nominalization} construction, similar to phrases such as \emph{the writing of a book} \citep{tajima85syntactic}. This nominalization\is{nominalization} lacked the aforementioned verb\is{verb}al properties, which can be illustrated with a similar nominalization\is{nominalization} construction in Dutch\is{Dutch}: example \ref{e:gerund8} shows that the nominalized \emph{bewerking} `adaptation' cannot be complement\is{complement}ed by a direct object\is{syntactic role!object} (as is possible with the English\is{English} gerund\is{gerund}). Instead, it requires the genitival preposition\is{preposition} \emph{van} `of' (a). Example (c) shows that speakers of Dutch\is{Dutch} need to combine a preposition\is{preposition}al noun\is{noun} phrase with some kind of  to-infinitive\is{to-infinitive} to express one of the functions carried by the English\is{English} gerund\is{gerund}.

\eal
\label{e:gerund8}
\ex[ ]{ \gll de	bewerking 	van 		het 	stuk\\
the	adaptation	of	the	piece\\
\glt `the adaptation of the play'
}
\ex[ ]{ \gll *door 	bewerking 	het 	stuk\\
by	adaptation	 the	piece\\}
\ex[ ]{ \gll door	het	stuk	te	bewerken\\
by	the	piece	to	adapt\\
\glt `by adapting the play'}
\zl

From this kind of deverbal nominalization\is{nominalization}, the English\is{English} gerund\is{gerund} probably evolved according to the following steps \citep[\citealp{tajima85syntactic}, summary and examples taken from][]{fanego04reanalysis}:

\begin{enumerate}
\item Around 1200, the deverbal nominalization\is{nominalization} \emph{-ing} began taking adverb\is{adverb}ial modifiers of all kinds:

\ea
Of \th i comyng at domesday\\
{\em `Of your coming \emph{at doomsday}.'}
\z

\item The first examples with direct object\is{syntactic role!object}s have been attested around 1300.

\ea
yn feblyng \th e body with moche fastyng
\\ {\em `in weakening the body by too much abstinence'}
\z

\item In the Early Modern English\is{English} period, other verb\is{verb}al features are increasingly found, such as distinctions of voice and tense\is{tense}. From Late Modern English\is{English} on, gerund\is{gerund}s also start to take subject\is{syntactic role!subject}s:

\ea 
he was war of hem comyng and of here malice \\
{\em `he was informed of them coming and of their wickedness'}
\z
\end{enumerate}

\citet{fanego04reanalysis} argues that these changes are best understood as reanalysis\is{reanalysis} of a nominal structure to a (more) verb\is{verb}al one (p. 26). This requires the speaker's ability to recognize multiple structural analyses since the `old' and the `new' use co-exist\is{co-existence}ed for a long time. The following examples show how the nominal analysis and the more verb\is{verb}al structure could be used together around 1300, whereas nowadays the nominal structure is unacceptab\is{acceptability}le unless there is a determiner\is{determiner}:

\ea
\label{e:gerund-last}
Sain Jon was ... bisi In ordaining of priestes, and clerkers, And in planning kirc werkes.\\
{\em `Saint John was ... busy ordaining priests and clerics, and in planning church works.'}

\item the ordaining of priests / the planning of works
\item ordaining priests / planning works
\item *ordaining of priests / *planning of works
\z

In order to account for the gradualness of the change, Fanego suggests a reanalysis-plus-actualization model. He acknowledges \citet{haspelmath98does}'s criticism on this model that it is still not gradual enough and he proposes that the gerund\is{gerund} should be regarded as a hybrid category which is partly noun\is{noun} and partly verb\is{verb}. To summarize, Fanego suggests that the development of the various uses of the English\is{English} gerund\is{gerund} involved (a) reanalysis\is{reanalysis} and (b) actualization\is{actualization} using Haspelmath's proposal for gradient categories. 
\\
\\
\noindent{\bfseries Problems with Fanego's account.} Fanego's analysis of the development of English\is{English} requires complex cognitive operations from the part of the speaker that do not seem entirely justified. First of all, in order to reconcile reanalysis\is{reanalysis} with the data, he needs to call on the process of actualization\is{actualization}. However, as \citet{haspelmath98does} already noted, actualization\is{actualization} {\em ``waters down the notion of reanalysis\is{reanalysis}, because it allows one to posit non-manifested reanalysis\is{reanalysis} as one pleases''} (p. 341). It also seems contradictory to propose reanalysis\is{reanalysis}, which is essentially an abrupt and discrete process, together with Haspelmath's gradient categories. Mechanisms such as semantic bleaching\is{semantic bleaching}, analog\is{analogy}y and exten\is{extension}sion could explain a gradual shift from a nominal category to a more verb\is{verb}-like category just as well without evoking reanalysis\is{reanalysis}.

A second problem has to do with the idea of a gradient category, that is, analyzing the gerund\is{gerund} as some hybrid category which is let's say 20\% nominal and 80\% verb\is{verb}al. This kind of analysis treats the Gerund as a single category in the grammar whereas Fanego himself distinguishes at least three different types existing today, each with their own particular syntactic behaviours:

\begin{itemize}
\item Type 1: gerund\is{gerund}s lacking determiner\is{determiner}s (e.g. {\em by writing it})
\item Type 2: gerund\is{gerund}s taking determiner\is{determiner}s (e.g. {\em the writing of the letter})
\item Type 3: verb\is{verb}al gerund\is{gerund} (e.g. {\em the people living in this town})
\end{itemize}

\noindent{\bfseries A model based on  redundancy\is{redundancy}.} Reanalysis is a mechanism which is based on mismatches in learning. In the case of the English\is{English} gerund\is{gerund}, Fanego writes that the first gerund\is{gerund}s to take verb\is{verb}al traits were the ones occurring in constructions without determiner\is{determiner}s (p. 19--20). However, the lack of determiner\is{determiner}s is in itself not necessarily a reason for reanalyzing a grammatical structure, especially since determiner\is{determiner}s were not at all obligatory in many noun\is{noun} phrases in Old and Middle English\is{English} \citep[p. 172--174]{traugott92syntax}. One can therefore reverse the question and ask why some uses of the gerund\is{gerund} resisted the spread of determiner\is{determiner}s. In other words: is there a {\em functional} explanation for the development of the gerund\is{gerund}?

A first step in the alternative hypothesis is to accept  redundancy\is{redundancy}: language users store many instances in memory so rather than looking for a single category which leads to multiple structural analyses, speakers are assumed to store many instances in memory. Actual change in the system only takes place if one of these redundant instances gets exten\is{extension}ded. No layering\is{synchron\is{synchronic}ic layering} or complex mechanisms for disambiguity\is{ambiguity} are needed since there are still enough instances left that cover the older use of a particular form. Redundancy thus requires a far more simple cognitive model than the reanalysis-and-actualization approach and treats each use of the gerund\is{gerund} as a construction in its own right.

Instead of reanalysis\is{reanalysis} and mismatches in learning, the alternative hypothesis assumes that some patterns or instances exten\is{extension}d their usage for a communicative reason. Fanego lists several possible sources (p. 11--17): first of all, the {\em-ing}-form of nominalization\is{nominalization}s was in competit\is{competition}ion with the Old English\is{English} present participle {\em -ende} (which still exists in Dutch\is{Dutch}, for example). {\em -ing} became dominant by the fifteenth century and thus increased its frequency\is{frequency}. Along with this competit\is{competition}ion, the productivity\is{productivity} of {\em -ing} also increased from a limited number of verb\is{verb}s to an almost fully productive schema. A third possible source could be the fact that the English\is{English}  to-infinitive\is{to-infinitive} has resisted the combinations with other preposition\is{preposition}s than {\em to} and {\em for to}. This created a gap in the usage of the infinitive which could be filled by the gerund\is{gerund} (or conversely, the expansion of the gerund\is{gerund} prevented the infinitive from filling this gap itself). Other sources are influences from French\is{French} and the co-occur\is{co-occurrence}rence of the gerund\is{gerund} with a genitive\is{case!genitive} phrase.

The point here is not to find {\em the} source for the development of the English\is{English} gerund\is{gerund} but rather to illustrate that many possible sources can be identified and that they all probably played some role. It is therefore fruitful to see language as a selection\is{selection}ist system in which all linguistic items compete for a place in the inventory. Due to multi-level selection\is{selection}\is{multi-level selection}, categories can become more dominant across patterns which is what seemed to have happened with the gerund\is{gerund}: it increased its productivity\is{productivity}, won the competit\is{competition}ion against {\em -ende} for marking participles and hence became more frequent and successful.

\citet{haspelmath98does} also criticized reanalysis\is{reanalysis} for failing to explain the strong unidirectional tendency of grammaticalization\is{grammaticalization}. In a system of multi-level selection\is{selection}\is{multi-level selection}, this could be explained due to the fact that once linguistic items become part of larger patterns or occur in multiple constructions, they are no longer fully independent of those constructions. The benefit of belonging to larger groups is that each item's individual survival chances increase, but the possible downside could be that the original use becomes structurally ambiguous or that it loses its distinctiveness. This would weaken its position and leaves the possibility for other items to conquer its space. In other words, there are always two factors influencing survival of a linguistic item: frequency\is{frequency} and function.
\\
\\
\noindent{\bfseries Back to the computational model.} The above analysis is only an illustration of how computational modeling could inspire linguists to come up with alternative hypotheses. The grammaticalization\is{grammaticalization} model of  redundancy\is{redundancy} that I presented here mainly comes from the observation that variation\is{variation} in a population\is{speech population} is an extremely challenging problem and that it is very difficult for a population\is{speech population} to reach a shared and coherent language without losing generaliz\is{generalization}ation accuracy. Moreover, the design stance\is{design stance} can offer mechanisms and operationalizations that are simpler than the processes that are often proposed in verb\is{verb}al theories.

That said, the experiments presented in this book have not yet offered any proof that an analysis such as the one proposed here can actually work. However, they did show that a redundant and bottom-up approach {\em can} deal with high degrees of uncertainty in the development of a grammar whereas no such model exists (yet) for reanalysis\is{reanalysis}. The comparison between the Iterated Learning Model\is{Iterated Learning Model (ILM)} (which essentially relies on reanalysis\is{reanalysis}) and this book has shown that a usage-based approach\is{usage-based model} performs significantly better than a reanalysis\is{reanalysis} model. This does not mean that reanalysis\is{reanalysis} does not exist or that it cannot be operationalized, but it poses some serious challenges to the effectiveness and explanatory power of the mechanism.

% ####################################################################



\chapter{Conclusion}\label{sec6}
\epigraph{1.25in}{In substance lies\\A form that's pure\\That is all lies\\I'm not so sure}{Nino Logoratti}{}

The working hypothesis which animated this book was that insights from research on phonetic detail at the prosodic level can be usefully incorporated into phonological models of intonation. This is consistent with our historist understanding of phonetic detail as systematically produced and perceived phonetic information which is not yet included in abstract phonological representations.\is{phonetic detail} Under such perspective, if phonological categories are flexible enough to be enriched with phonetic information which proves to be systematically produced and perceptually relevant, phonetic detail is not only consistent with exemplar-based approaches, but can also lead to a refinement of accounts based on abstractionist assumptions. 

In this book I explored whether and how one particular abstractionist model of intonation, the Autosegmental-Metrical (\textit{AM}) framework, should account for detailed phonetic information in \textit{f0} contours and durational patterns. The evidence gathered in the experimental chapters will be reviewed in the next section (Section~\ref{sec61}), by grouping results according to their relevance to production or perception and to intonation or tempo (see Table~\ref{tab11} in Section~\ref{sec13}). I will then provide a brief overview of the tools for the exploration of prosodic detail developed or fine-tuned across the various experimental chapters, thus grouping the methodological outcomes of this work in Section~\ref{sec62}. I will conclude by discussing the wider theoretical implications of our findings and commenting on the polyvalence of prosodic detail, which can be accommodated in both exemplar-based (Section~\ref{sec631}) and abstractionist (Section~\ref{sec632}) accounts of prosody.

\section{Summary of findings}\label{sec61}
Evidence from the experimental chapters points to the need of a partial enrichment of phonological categories in the AM framework. By examining functional contrasts between narrow focus questions and both partial topic statements (Section~\ref{sec2} and Section~\ref{sec32}) and narrow focus statements (Section~\ref{sec33} and Sections~\ref{sec4}--\ref{sec5}), we have found that some phonetic detail in the shape of \textit{f0} contours should be included in abstract representations of tunes (Sections~\ref{sec2}--\ref{sec3}), whereas phonetic information about durational patterns can indeed be regarded as negligible detail (Sections~\ref{sec4}--\ref{sec5}), at least for this contrast and in this variety. In the following subsections, instead of presenting results for individual studies as in the experimental chapters, we group them according to the phonetic dimension involved (melodic detail, Sections~\ref{sec2}--\ref{sec3}, see Section~\ref{sec611}; temporal detail, Sections~\ref{sec4}--\ref{sec5}, see Section~\ref{sec612}) and to the mechanisms explored (production, Sections~\ref{sec2}--\ref{sec4}, and perception, Sections~\ref{sec3}--\ref{sec5}, see Section~\ref{sec613}).

\subsection{Intonation}\label{sec611}
In the AM framework, continuous phonetic information relative to \textit{f0} contours is discretized into phonological tunes. Tunes are composed by a series of tonal events, namely pitch accents and boundary tones, which are phonetically represented by points in the \textit{f0}-time plane. As a result, \textit{f0} contours between such tonal events are considered as context-determined, inferable by rule, and, ultimately, phonologically irrelevant.\is{f0 dynamics} However, we provided in Section~\ref{sec2} some evidence for systematically produced differences in the \textit{f0} contour between the two tones composing the nuclear pitch accents in narrow focus questions (\textit{QNF}) and partial topic statements (\textit{SPT}) in Neapolitan Italian (NI). Both pitch accents are realized phonetically as a rise which begins at stressed syllable onset and reaches its peak at the end of the stressed vowel. Alignment and scaling of rise start and end are not significantly different in the two contexts, but the \textit{f0} contour between the two is: the rise is more convex in QNF and more concave in SPT.\is{partial topic} If \textit{f0} contours had to be reduced to a sequence of points on the \textit{f0}-time plane connected by irrelevant interpolations, there would be no way to account for these observed regularities in production. 

As Section~\ref{sec32} shows, however, differences in \textit{f0} rise shape do not seem to be used as a perceptual cue to the contrast between QNF and SPT. We resynthesized stimuli at different points along a continuum of rise shape, ranging from very concave to very convex. Listeners' responses to a two-alternatives forced-choice identification task showed no correlation with stimulus manipulation. That is, it seems that differences in rise shape, while consistently produced, are not always used as a perceptual cue to pragmatic contrasts. Under these circumstances, pitch accent internal rise shape can not be considered as phonetic detail, and does not need to be included in the phonological representations of nuclear pitch accents in order to contrast QNFs and SPTs. Nuclear pitch accents in both contexts might use the prosodic transcription already suggested in the literature for QNF, namely L*+H. The contrast between the two contexts is rather expressed at the tune level, by different paradigmatic options in terms of boundary tones and postnuclear pitch accents.

The fact that rise shape does not play a perceptual role in contrasting QNF and SPT does not mean that rise shape is phonologically irrelevant altogether. In Section~\ref{sec33} we examined the perceptual role of rise shape differences in another pragmatic contrast, the one between QNF and narrow focus statements (SNF), whose nuclear rises are also more concave compared to questions.\is{sentence modality} It has been long acknowledged that the contrast between QNF and SNF is primarily signalled by differences in tonal alignment - that is, in the synchronization of \textit{f0} movements with the segmental string. As we said above, in QNF the \textit{f0} peak is reached at the end of the stressed vowel; in SNF, on the other hand, the peak is reached around the stressed vowel midpoint, and the pitch accent is accordingly transcribed as L+H*. We hypothesized that, if alignment information were made ambiguous, rise shape could have been the only cue for listeners to rely on. A two-alternative forced choice identification task of stimuli with ambiguous alignment showed that listeners do use phonetic information in rise shape when categorizing (narrow focus) questions and statements.

These findings do not necessarily have to impact the conventions in use for prosodic transcription in the AM framework. We can still continue to label NI nuclear pitch accents as L*+H in questions and as L+H* in statements, as long as we acknowledge that these are used as shortcuts to richer phonetic descriptions.\is{pitch accent} This might not always be the case, as shown by \citeauthor{petrone2011tones}'s \citeyearpar{petrone2011tones} work on phonetic information in the prenuclear region, according to which a new structural position (a phrase accent) is required to account for sentence modality contrasts. It is important to stress that the exploration of melodic detail is consistent with different outcomes, ranging from the validation of information reduction (as we have seen in the case of QNF vs SPT rises) to an enrichment of phonetic representations which does not affect transcription conventions (as in the case of QNF vs SNF rises) and to the suggestion of different structural interpretations (as in the case of prenuclear falls across sentence modality). The interest of studying phonetic detail lies indeed in this rich range\enlargethispage{1em} of solutions which can be suggested for the research questions it raises.

\subsection{Tempo}\label{sec612}
Phonetic information fed into phonological categories in the AM framework is not only reduced with respect to the discretization of \textit{f0} contours into a sequence of contrastive tonal events and irrelevant transition. Information is also reduced by concentrating on \textit{f0} contours alone, thus discarding information on other dimensions, such as duration, intensity, voice quality and spectral proprieties. We thus tested whether sentence modality contrasts (again QNF vs SNF) are characterized acoustically by differences along other dimensions, and whether eventual differences are used as perceptual cues. We decided to focus on the temporal dimension, since in the last ten years the literature on sentence modality contrasts has shown that questions and statements often differ with respect to either global measures of speech rate or local measures in the duration of linguistic units of various sizes, ranging from segments to phonological words.\is{tempo}

We thus collected two corpora of sentences uttered as both questions and statements, by controlling focus placement as well. Results of a first experiment (Section~\ref{sec43}) show that global utterance duration and thus speech rate do not vary across sentence modality. Whereas \citet{vanheuven2005speech} suggested that questions might display a universal trend to faster speech rate just as they show a trend to higher pitch, our findings are rather consistent with language-specific encoding of sentence modality contrasts.\is{frequency code} Differences between questions and statements along the temporal dimension, however, were found when analyzing our corpora in more detail. If overall utterance duration is the same, segmental durations have been found to vary in the two conditions. In particular, statements have longer initial segments, whereas the final segment (a vowel in our corpora) is systematically longer in questions. The magnitude of these effects is not negligible, especially for final vowel in questions, which are about 20 ms longer than in statements. 

The existent differences in segmental duration within utterance of the same global duration suggested the use of an integrated metric for the evaluation of durational patterns. In a second experiment (Section~\ref{sec44}) we thus adapted the algorithm proposed by \citet{pfitzinger2001phonetische} in order to capture local variations of speech rate.\is{Local Phone Rate} This allowed us to show that speech rate indeed follows different patterns across sentence modality, being globally increasing in statements and decreasing in questions.

A subsequent experiment (Section~\ref{sec5}) was devised in order to establish whether durational differences at the segmental level were also consistently used as perceptual cues, in which case they should be considered as relevant prosodic detail and be somehow incorporated into phonological representations of sentence modality contrasts. We had to manipulate durational patterns independently of \textit{f0} contours, which represent by themselves a very strong cue to sentence modality contrasts. As in the case of the perceptual study on melodic detail in QNF vs SNF pitch accents (Section~\ref{sec33}), in which peak alignment information was made unavailable in order to evaluate the role of rise shape, in the study of durational patterns we manipulated the test stimuli so as to have an ambiguous \textit{f0} contour. In addition, unlike the previous experiment on melodic detail, we also manipulated durational patterns in utterances with clear question or statement intonation.\is{resynthesis} This enabled us to assess the perceptual importance of temporal information, by testing whether it is used constantly and independently (that is, in addition to intonational cues) or only when other primary cues are not available. 

A two-alternative forced-choice identification task showed however that listeners' responses are not affected by manipulations of durational patterns, not even when intonation was made ambiguous. These results are not consistent with the hypothesis that temporal detail is evaluated as a perceptual cue in its own right. This finding has been replicated in a shorter experiment, in which subjects only listened to intonationally ambiguous stimuli, in order to maximize their attention on temporal cues. However, the fact that listeners responses are not affected by temporal manipulations does not entail that durational differences are not processed at all. Listeners might perceive durational information but ultimately discard it when intonational cues have been evaluated. For this reason, we measured reaction times to stimuli with either congruous or incongruous cues on the melodic and temporal level. Stimuli with congruous information (e.g. with statement-like \textit{f0} contour and durational pattern) were predicted to elicit faster responses than stimuli with incongruous information (e.g.with statement-like \textit{f0} contour and question-like durational pattern). However, this prediction was not borne out either. Reaction times are only slightly longer when intonation is ambiguous - a fact which contributes to show that, in NI sentence modality contrasts, durational information is negligible detail.

\subsection{Production and perception}\label{sec613}
We are thus faced with an extremely interesting pattern of results, where production experiments show consistent acoustic differences in both melodic detail (between QNF and SPT nuclear rises) and temporal detail (in durational patterns across sentence modality), but perceptual experiments fail in attesting their use as perceptual cues. Of course, we cannot exclude that our negative results in perception are due to poor methodological choices in the set-up of the experiments. However, especially in the case of the perception of temporal detail, the conditions for appropriate testing were probably met (see Sections~\ref{sec541}--\ref{sec542} for discussion). According to \citeauthor{frick1995accepting}'s \citeyearpar{frick1995accepting} ``good effort criterion'', we should even accept the null hypothesis of no durational information in phonological categories for sentence modality contrasts in NI, rather than simply stating that the alternative hypotheses are not supported.\is{good effort} In any case, this does not allow us to conclude that durational patterns play no role at all in the perception of any contrast in any language, and thus we cannot exclude that phonetic information at the temporal level is stored and used in perception of post-lexical contrasts, as exemplar-based approaches would predict.

Our goal, however, is not to rule out the possibility of an ``exemplar prosody'' altogether. We rather aim to show that evidence from both production and perception is needed when working on prosodic detail, from either an abstractionist or an exemplarist viewpoint. Quite recently, \citet{nguyen2009dynamical} observed that ``much of the available evidence for long-term storage of FPD in the mental lexicon comes from studies of speech production''. The observation is even more true for research on exemplar prosody, which deals exclusively with production data, as we will see shortly (Section~\ref{sec631}). This is understandable, since research in this field is still very young. But when suggesting a new understanding of phonological structures, evaluating the impact of phonetic detail on perception is no less important. This is clearly shown, for example, by research on incomplete neutralization, dealing with allegedly neutralized phonological contrasts which are still reflected by surface phonetic differences. 

For example, a devoicing process has been said to neutralize voicing contrasts in domain-final obstruents in German, thus making \textit{Rat} (advice) and \textit{Rad} (wheel) homophones.\footnote{Among the vast bibliography on the topic, see \citet{port1981neutralization,odell1983discrimination,charlesluce1985word,port1985neutralization,port1989incomplete,port1996discreteness,kleber2010implications,rottger2011robustness,winterFORTHnature}. Similar phenomena have been explored in other languages, such as Dutch \citep{warner2004incomplete,warner2006orthographic,ernestus2006functionality}, Russian \citep{pye1986word,dmitrieva2010phonological}, Polish \citep{slowiaczek1985neutralizing,slowiaczek1989perception} and Catalan \citep{dinnsen1984phonological,charlesluce1987reanalysis}. We exclude from our review the seminal paper by \citet{dinnsen1971three}, which was unfortunately not available to us. Experimental results or theoretical arguments against incomplete neutralization are provided by \citet{fourakis1984incomplete,mascaro1987underlying,jassem1989neutralization,kopkalli1993phonetic,manaster1996letter}.} However, subtle sub-phonemic durational differences can be found in speakers' production of underlying voiceless and devoiced obstruents.\is{incomplete neutralization} Along the phonetic continua of vowel duration, burst duration and closure voicing, devoiced obstruents are somewhere in between the extremes occupied by voiced and voiceless sounds. Crucially to our discussion, the perceptual role of this consistently produced phonetic detail has been investigated since the very first studies on incomplete neutralization - that is, at least since \citet{port1981neutralization}. Constant methodological improvements enabled the exclusion of possible experimental confounds, as in the case of orthography-induced biases \citep{rottger2011robustness}. Likewise, determining whether such contrasts are perceptible is instrumental in deciding of their functional role: whereas \citet{port1981neutralization} first thought that ``this `semicontrast' must be nearly useless in conversation'', \citet{ernestus2006functionality} recently suggested that incomplete neutralization might be ``a subphonemic cue to past-tense formation'' in Dutch. Ultimately, it is this long-term exploration of both production and perception mechanisms which allowed researchers to recast the incomplete neutralization issue in abstractionist/exemplarist terms, as in \citet{kleber2010implications}. We hope that our investigation of prosodic detail, however far from conclusive, might at least demonstrate that the recent work on exemplar prosody based on production evidence (see Section~\ref{sec631}) must be necessarily complemented by a thorough examination of perceptual mechanisms.

\section{Tools for prosodic detail research}\label{sec62}\is{ASSI}
Besides suggesting a potentially interesting research topic, we also aimed at providing some experimental tools which might be useful in its exploration. This was particularly needed in the case of the study of temporal detail, which has not been analyzed in the literature as thoroughly as melodic detail. However, the tools briefly presented in the experimental chapters on temporal detail might also prove relevant in the study of prosodic detail in general. 

\subsection{Automatic Speech Segmentation for Italian}\label{sec621}
The study of temporal detail in production required the collection of a matrix with a great number of segment durations (Section~\ref{sec422}). Pooling data from the \textit{Orlando} and the \textit{Danser} corpora (see Section~\ref{sec421}), we had to segment 2376 utterances, each composed of 8 CV syllables. With more than 35.000 segmental boundaries to be placed, manual segmentation was simply not an option. However, tools for automatic segmentation of Italian were not available either.\footnote{\textit{EasyAlign} \citep{goldman2011easyalign} only works with French, English, Brazilian Portuguese, Spanish and Taiwan Min, while \textit{SPPAS} \citep{bigi2012speech}, which works with French, English, Italian and Chinese, was only released after our experiment was planned, executed and published.} Our solution has been to develop our own tool for Italian forced alignment, \textit{ASSI} \citep{cangemi2011automatic}. In forced alignment, audio files are segmented according to an orthographic transcription and a phonetized lexicon provided by the user. The first is a plain text file containing for each row an audio file name and its orthographically transcribed content, as in (1) for the \textit{Danser} transcription file:

\begin{description}
   \item[(1)] {\tt Q1BD1.wav danilo\_vola\_da\_roma}
\end{description}\label{ex61}

\noindent The second contains for each row an orthographic word-form and a phonetic transcription of the expected pronunciation (variants are allowed), as in (2) for the \textit{Orlando} lexicon file: 

\begin{description}
   \item[(2)] {\tt ralego r:a:l:[e:/E:]g:o:}
\end{description}\label{ex62}

Forced alignment is especially suited for the segmentation of read speech, since for this kind of data the experimenter can provide an orthographic transcription with no effort. Moreover, when working on sentence modality and/or focus placement in NI, which only use prosodic cues to express these contrasts, the use of forced alignment is even more indicated: the same orthographic transcription based on the same phonetized lexicon can be used for a variety of experimental items. For example, the \textit{Orlando} corpus contained three sentences composed by 16 segments. These were uttered in the six combinations between the two levels of the sentence modality factor (question, statement) and the three levels of the narrow focus placement factor (on subject, verb or object). Each of 30 speakers read three repetitions of three 16-segment sentences uttered in six contexts, thus requiring the extraction of 25.920 phone durations in total. This was accomplished by providing a single lexicon file with the phonetic transcription of \textit{six} words, and a single transcription file containing orthographic transcriptions for \textit{three} sentence types.

\subsection{Multi-parametric continuous resynthesis}\label{sec622}\is{resynthesis}
Whereas the exploration of temporal detail in production required a tool which merely speeded up an already existing procedure (viz. manual segmentation), to test our hypotheses on the perception of temporal detail we had to resynthesize stimuli using a new procedure altogether (see Section~\ref{sec52}). The procedure is partly based on \citet{gubian2010automatic,gubian2011joint}, and through a set of \textit{Praat} \citep{boersma2008praat} and \textit{R} \citep{r2008r} scripts yields input files for the \textit{PSOLA} \citep{moulines1990pitchsyncronous} resynthesis engine in \textit{Praat}. Given two utterances, the question and statement version of a same sentence, we needed to resynthesize each one using (1) durational patterns and/or \textit{f0} contours of the other one (\textit{cross-modality} resynthesis) and (2) ambiguous durational patterns and/or \textit{f0} contours between the two (\textit{ambiguous} resynthesis).

However, as we have seen discussing NI intonation (see ~\ref{sec123}), the synchronization of \textit{f0} movements with the segmental string is crucial in signalling sentence modality contrasts. This means that, as far as cross-modality resynthesis is concerned, intonational and temporal cues must be jointly manipulated: one cannot simply extract the \textit{f0} of the first utterance and use it to resynthesize the second. Similarly, segmental durations cannot be modified without transforming \textit{f0} contours as well. Thus we extracted \textit{f0} contours and segmental durations for each utterance, then the two \textit{f0} contours were time-warped by aligning their corresponding segmental boundaries. This landmark registration procedure creates two intermediate contours having identical underlying phone durations. These intermediate contours can be combined with actual durational patterns and thus be ready to be resynthesized onto an actual utterance. 

The results of cross-modality resynthesis are particularly satisfying. As we have seen in Section~\ref{sec541}, listeners' responses to stimuli resynthesized by applying question \textit{f0} contours onto statement bases are not significantly different from listeners' responses to natural questions. Informal testing shows encouraging results in the resynthesis of other contrasts as well, as for example in the case of focus placement, and even when the lexical material is different between the two sentences. For example, \textit{f0} contour and durational pattern of a (prepositional) \textit{object} narrow focus statement utterance of the sentence \textit{Danilo vola da Roma} were used to resynthesize a \textit{subject} narrow focus statement utterance of the sentence \textit{Serena vive da Lara} (see Section~\ref{sec4212}), yielding a stimulus which was identified as having narrow focus on the object. In this case, performances could be even improved by adding manipulation of landmark-registered intensity contours, which could be easily included as an additional module in the resynthesis procedure. Of course, these excellent results are at least in part motivated by the use of sentences with identical metrical and syllable structures at both ends of the resynthesis procedure. However, we believe that very different sentences could also be used, if phonologically motivated assumptions guided the choice of the landmarks to be registered.

Our second goal was the creation of ambiguous stimuli, with respect to \textit{f0} contours and/or durational patterns. This was achieved by averaging phone durations, in the case of durational patterns, and by averaging intermediate \textit{f0} contours (i.e. landmark-registered contours expressed in normalized time with identical underlying phone durations), prior to resynthesis. By using a simple average, we obtained \textit{acoustically} ambiguous stimuli. These stimuli would have also been perceptually ambiguous only if the perceptual space between questions and statements was perfectly linear. Unsurprisingly, this proved not to be the case (for a discussion of how this affected the interpretation of our results, see Section~\ref{sec541}): responses to stimuli with acoustically ambiguous intonation had a significant question bias, probably due to the postnuclear region.\footnote{The subject narrow focus utterances used in the experiment have an audible fall in the question condition (see Figure~\ref{fig201}, bottom panel) and flat \textit{f0} contour in the statement condition, which are respectively transcribed as !H* and !H+L*. However, in postnuclear position even slight \textit{f0} falls (as those in the acoustically ambiguous condition) can be salient, and thus bias the listener towards the question response. The impact of the postnuclear region on acoustical and perceptual ambiguity can thus be easily tested by using gated or object-focussed stimuli.} What is relevant to the present discussion is that our resynthesis procedure allows us to address very explicitly the issue of the difference between acoustically and perceptually ambiguous stimuli in a multidimensional prosodic space. In this sense, this procedure could prove a useful tool in the investigation of questions which are not directly addressed in this book.

\section{Theoretical implications}\label{sec63}
In this final section we interpret our findings on prosodic detail in Neapolitan Italian by discussing their relevance for recent exemplar-based approaches to prosody focussing on frequency effects in production (Section~\ref{sec631}). We conclude by highlighting that a close examination of phonetic detail is necessary for the construction of phonologically adequate categories (Section~\ref{sec632}): neither excessive unanalyzed phonetic information nor bony minimalist abstract categories are viable options when prosody is analyzed in production and perception.

\subsection{Exemplar prosody}\label{sec631}\is{exemplar prosody}
As we said in the introductory pages (Section~\ref{sec113}), an exemplar-based approach to prosody would provide a natural setting for the accommodation of prosodic detail. Let us flesh out this statement in this section. According to Johnson, \begin{quote}an exemplar is an association between a set of auditory properties and a set of category labels. The auditory properties are output from the peripheral auditory system, and the set of category labels includes any classification that may be important to the perceiver, and which was available at the time that the exemplar was stored - for example, the linguistic value of the exemplar, the gender of the speaker, the name of the speaker, and so on. \cite[147]{johnson1997speech}\end{quote}
As we explained above (see Section~\ref{sec111}), in this approach the normalization phase is no longer necessary: new prompts activate both ``linguistic value'', thus feeding word recognition, and indexical information (e.g. the gender and name of the speaker), thus feeding talker recognition. But what happens if \textit{f0} contours are stored as part of the auditory properties set, and pragmatic or information structure contrasts are stored as part of the category labels set (specifically, its ``linguistic value'')? By enriching exemplars with information on both the substantial and the functional sides, the model could perform talker recognition, word recognition and extraction of post-lexical meaning at the same time.

Recent research has addressed the issue of whether \textit{f0} contours are stored into exemplars and connected to post-lexical meaning.\footnote{For storage of fundamental frequency information connected to lexical contrasts, see \citet{sekiguchi2006effects}.} Most work has focussed on frequency effects in pitch accent production. Exemplar models have been extended to production since \citet{pierrehumbert2001exemplar}. In her most basic model, \begin{quote}the decision to produce a given category is realized through activation of that label. The selection of a phonetic target, given the label, may be modelled as a random selection of an exemplar from the cloud of exemplars associated with the label.\cite[§3.1]{pierrehumbert2001exemplar}\end{quote}
By positing activation of a region in the exemplar cloud rather than that of a single exemplar, the model can account for entrenchment effects, according to which productions become less variable with practice. In this case, phonetic variability is expected to decrease when the cloud is denser because the exemplars are produced and perceived with higher frequency.

Recent work by Katrin Schweitzer combines the hypothesis of \textit{f0} contour storage with predictions on entrenchment. In her model, \begin{quote}during speech production a speaker selects a stored exemplar as a production target. Assuming that pitch accents can be stored with the word, the speaker would select an exemplar that matches not only the intended word but also the intended pitch accent. \cite[138]{schweitzer2010frequency}\end{quote}
If, instead of selecting a single exemplar, a whole region of exemplars is activated, as in Pierrehumbert's refined model, entrenchment would predict that more frequent pitch accents are less variable. In the last few years, a number of corpus studies has used parametrized descriptions of pitch accents (based on \textit{PaIntE}, see \citealt{mohler1998parametric,mohler2001improvements}) to explore whether phonetic variability is affected by frequency of occurrence. The results seem to provide mixed evidence, ranging from the absence of frequency effects in German \citep{walsh2008examining}, to more prominent \textit{f0} movements in frequent word/pitch accent combinations \citep{schweitzer2010frequency} and to entrenchment in English collocations \citep{schweitzer2011prosodic}. In sum, even if the authors conclude that ``there is still a great deal to be understood about how lexicalised storage interacts with `top-down' information in the production of prosody'' \cite[4]{schweitzer2011prosodic}, these results are taken as supporting an exemplar-based view of prosody.

\subsection{Substance, form and function}\label{sec632}
This approach is surely intriguing, and we are persuaded that it will receive a great deal of attention in the coming years. Its elaboration, however, might benefit from a close inspection of its theoretical underpinnings, in order to rule out possible aporetic developments. At this point, it must be clear that in a model where exemplars are conceived as associations between \textit{f0} information and post-lexical function labels, there is no longer room for phonological representations, which are at best redundant. Substance is no longer linked to function by abstract forms, but rather through activation of exemplars using similarity functions. This is indeed the perspective of the so-called \textit{functional} approaches to prosody \citep{shriberg1998can,noth2000verbmobil,batliner2001prosodic}, which suggest that formal entities such as ``the unfortunate notion of pitch accent'' should be pruned by Occam's razor \citep[25]{batliner2005prosodic}.\is{pitch accent} 

However, two possible objections arise at this point. The first is that, as \citet[20]{ladd2008intonational} puts it, ``whether we should adopt a `phonological' approach to intonation is not primarily a matter of taste, but an \textit{empirical question}''. Or, in other words, in addition to dealing with previously unaccounted-for phenomena (such as frequency effects), an exemplar-based approach to prosody must also account for phenomena which have already been framed in phonological terms.\footnote{A few examples might be the role of accentedness in discourse structure \citep{hawkins1991factors}, the disentangling of linguistic and paralinguistic meaning \citep{scherer1984vocal}, and evidence from imitation studies \citep{cole2011phonology}.} This objection, of course, is nothing more than an empirical argument: in \citeauthor{smith1981categories}'s \citeyearpar[33]{smith1981categories} terms, ``it is a statement about what has happened so far, not about what can happen''. And since functional models have been seriously explored for only a decade, we surely cannot consider empirical arguments as conclusive.

The second objection is perhaps more cogent. It regards the covert use of phonological forms, even when the general system architecture is claimed to posit a direct link between phonetic substance and post-lexical meaning. In the discussion of frequency of usage in exemplar-based models of prosody, for example, we have seen that its effects have been explored in terms of phonetic variability of pitch accents. That is, even the arguments adduced in favour of exemplar dynamics eventually posit somehow abstract categories. At this point, it is unclear whether exemplars actually link phonetic substance to post-lexical categorical labels or rather to pitch accents - that is, forms which themselves bridge substance and functions.\footnote{This state of affair is already exemplified by the titles of relevant studies in this perspective, such as ``Relative frequency affects pitch accent realisation'' \citep{schweitzer2010relative} and ``Frequency of occurrence effects on pitch accent realisation'' \citep{schweitzer2010frequency}.} An operationalized version of abstractions corresponding to pitch accents also seems to be required by current text-to-speech systems. In \citeauthor{vansanten2000quantitative}' \citeyear[278]{vansanten2000quantitative} quantitative model, for example, phonetic differences between \textit{f0} contours interpreted as having the same function are accounted for by time warping of a common \textit{template}.

It is unclear, at this point, whether formal representations of prosody can really be dismissed, even in exemplar-based approaches. Functional approaches criticize AM-like phonological approaches to intonation because \begin{quote}The classical phonological concept of the Prague school has been abandoned in contemporary intonation models, namely that phonemes - be they segmental or suprasegmental - should only be assumed if these units make a difference in meaning. This functional point of view has given way to more formal criteria such as economy of description. Thus, the decision on the descriptive units is not based on differences in meaning but on formal criteria, and only afterwards are functional differences sought that can be described with these formal units. \cite[§1.1]{batliner2005prosodic}\end{quote}
However, it is crucial to stress out that, in principle, ``formal criteria'' \textit{include} consideration of contrasts in meaning. And that, in that very same Prague school, ``phonemes - be they segmental or suprasegmental'' are \textit{formal} entities. In this sense, despite their claims, functional approaches do not actually advocate for the suppression of phonological representations and of inventories of formal units. 

It is true that, in AM accounts of intonation, the mapping between forms and functions is often confusing. In discussing prenuclear fall shape across sentence modalities in NI (see Section~\ref{sec242}) and German (see Section~\ref{sec31}), we have seen a clear example of how meaningful phonetic detail can be accommodated by enriching either the tonal inventory or the sequential grammar. We agree with proponents of functional approaches that if a single pragmatic contrast is expressed by a given phonetic difference, having two competing phonological analyses is symptomatic of the unstable state of the formal descriptions available. But again, we must acknowledge that this objection is nothing more than an empirical argument: it does not prove that phonological representations are useless, but rather that they are not yet capable of providing a unified account. 

Ultimately, the central point is that if decisions on the descriptive units must be based on differences in meaning, then we need to start from a catalogue of different functions. But as we have seen in the discussion of the gap between segmental and supra-segmental phonology (Section~\ref{sec243}), there is no such thing as a theory-independent set of post-lexical functions. Moreover, individual theories of pragmatic and discourse meaning use as \textit{explicans} the very same set of phenomena which is \textit{explicandum} in a functional account of intonation.\footnote{One example is the case of B-accents in \citet{jackendoff1972semantic}.} The risk of circularity here is very high. Theories of intonation must acknowledge the need of a constant exchange between evidence provided by research on substance and by research on function: the formal level is indeed the central processor which permits the incorporation of insights coming from both directions.

Frequency effects on pitch accent variability shows that research on prosodic detail in production can provide arguments supporting an exemplar-based view of prosody, but also that no framework for the study of intonation has actually dismissed a somehow abstract level of representation altogether. Our results on the perception of prosodic detail support this view by showing that some consistently produced phonetic information does not function as a cue to some post-lexical contrasts. The strongest interpretation of these results is that an abstract representation in terms of phonological categories is useful and necessary in the study of intonation. However, current phonological models need to be refined with respect to both the richness of phonetic specification (as in the case of the nuclear rise shape across sentence modalities) and in the mechanisms used to link substance with function (as in the case of the competing analyses of prenuclear falls across sentence modalities). According to the minimalist interpretation, on the other hand, no inferences are drawn about the role of abstract forms in intonation, but whenever phonological categories are indeed assumed, they must be thoroughly explored in production and perception to rule out the exclusion of useful prosodic detail.

\enlargethispage{1em}In any case, the exploration of prosodic detail appears to be crucial for both exemplar-based and abstractionist approaches to intonation, and will probably provide the common ground for their integration into a truly hybrid model.
\backmatter
\ohead{Bibliography}


%\bibliography{biblio}

\begin{thebibliography}{284}
\providecommand{\natexlab}[1]{#1}
\providecommand{\url}[1]{#1}
\providecommand{\urlprefix}{}
\expandafter\ifx\csname urlstyle\endcsname\relax
  \providecommand{\doi}[1]{doi:\discretionary{}{}{}#1}\else
  \providecommand{\doi}{doi:\discretionary{}{}{}\begingroup
  \urlstyle{rm}\Url}\fi

\bibitem[{Abercrombie(1967)}]{abercrombie1967elements}
Abercrombie, David. 1967.
\newblock \emph{Elements of general phonetics}.
\newblock Edinburgh: Edinburgh University Press.

\bibitem[{Adjarian(1899)}]{adjarian1899explosives}
Adjarian, Hrachia. 1899.
\newblock Les explosives de l'ancien arménien étudiées dans les dialectes
  modernes.
\newblock \emph{La Parole: Revue internationale de Rhinologie, Otologie,
  Laryngologie et Phonétique expérimentale} 119--127.

\bibitem[{{Albano Leoni}(2006)}]{albanoleoni2006statuto}
{Albano Leoni}, Federico. 2006.
\newblock Lo statuto del fonema.
\newblock In Stefano Gensini \& Martone Arturo (eds.), \emph{Il linguaggio:
  {T}eorie e storia delle teorie. {I}n onore di {L}ia {F}ormigari}, 281--303.
  Napoli: Liguori.

\bibitem[{Anderson et~al.(1984)Anderson, Pierrehumbert \&
  Liberman}]{anderson1984synthesis}
Anderson, Mark, Janet Pierrehumbert \& Mark Liberman. 1984.
\newblock Synthesis by rule of {E}nglish intonation patterns.
\newblock In \emph{Proceedings of 9th {I}nternational {C}onference of
  {A}coustics, {S}peech and {S}ignal {P}rocessing}, vol.~9, 77--80. San Diego.

\bibitem[{Andr\'{e} et~al.(2003)Andr\'{e}, Ghio, Cav\'{e} \&
  Teston}]{andre2003perceval}
Andr\'{e}, Carine, Alain Ghio, Christian Cav\'{e} \& Bernard Teston. 2003.
\newblock Perceval: {A} computer-driven system for experimentation on auditory
  and visual perception.
\newblock In Daniel Recasens, Maria~Josep Sol\'{e} \& Joaquín Romero (eds.),
  \emph{Proceedings of the 15th {I}nternational {C}ongress of {P}honetic
  {S}ciences}, 1421--1424. Barcelona.

\bibitem[{Angelini et~al.(1993)Angelini, Brugnara, Falavigna, Giuliani, Gretter
  \& Omologo}]{angelini1993baseline}
Angelini, Bianca, Fabio Brugnara, Daniele Falavigna, Diegl Giuliani, Roberto
  Gretter \& Maurizio Omologo. 1993.
\newblock A baseline of a speaker independent continuous speech recognizer of
  {I}talian.
\newblock In \emph{Proceedings of the 3rd {European Conference on Speech
  Communication and Technology}}, 847--850. Berlin.

\bibitem[{Arndt(1960)}]{arndt1960modal}
Arndt, Walter. 1960.
\newblock Modal particles in {R}ussian and {G}erman.
\newblock \emph{Word} 16. 323--336.

\bibitem[{Atterer \& Ladd(2004)}]{atterer2004phonetics}
Atterer, Michaela \& Robert Ladd. 2004.
\newblock On the phonetics and phonology of segmental anchoring of {F}0:
  {E}vidence from {G}erman.
\newblock \emph{Journal of Phonetics} 32. 177--197.

\bibitem[{{Audacity Development Team}(2006)}]{audacity2006audacity}
{Audacity Development Team}. 2006.
\newblock Audacity: {F}ree audio editor and recorder.
\newblock Computer program, retrieved from http://audacity.sourceforge.net/.

\bibitem[{Avesani(1990)}]{avesani1990contribution}
Avesani, Cinzia. 1990.
\newblock A contribution to the synthesis of {I}talian intonation.
\newblock In \emph{Proceedings of the 1st {International Conference on Spoken
  Language Processing}}, 833--836. Kobe.

\bibitem[{Balota(1994)}]{balota1994visual}
Balota, David. 1994.
\newblock Visual word recognition.
\newblock In Matthew Traxler \& Morton Gernsbacher (eds.), \emph{Handbook of
  psycholinguistics}, 334--357. San Diego: Academic Press.

\bibitem[{Batliner \& M{\"o}bius(2006)}]{batliner2005prosodic}
Batliner, Anton \& Batliner M{\"o}bius. 2006.
\newblock Prosodic models, automatic speech understanding, and speech
  synthesis: {T}owards the common ground?
\newblock In William Barry, Wim {van Dommelen} \& Jacques Koreman (eds.),
  \emph{The integration of phonetic knowledge in speech technology}, 21--44.
  Dordrecht: Kluwer.

\bibitem[{Batliner et~al.(2001)Batliner, M{\"o}bius, M{\"o}hler, Schweitzer \&
  N{\"o}th}]{batliner2001prosodic}
Batliner, Anton, Bernd M{\"o}bius, Gregor M{\"o}hler, Antje Schweitzer \& Elmar
  N{\"o}th. 2001.
\newblock Prosodic models, automatic speech understanding, and speech
  synthesis: {T}owards the common ground.
\newblock In Paul Dalsgaard, Børge Lindberg \& Henrik Benner (eds.),
  \emph{Proceedings of the 7th {European Conference on Speech Communication and
  Technology}}, vol.~4, 2285--2288. Aalborg.

\bibitem[{Beckman(1996)}]{beckman1996parsing}
Beckman, Mary. 1996.
\newblock The parsing of prosody.
\newblock \emph{Language and Cognitive Processes} 11(1-2). 17--68.

\bibitem[{Beckman(1997)}]{beckman1997typology}
Beckman, Mary. 1997.
\newblock A typology of spontaneous speech.
\newblock In Yoshinori Sagisaka, Nick Campbell \& Norio Higuchi (eds.),
  \emph{Computing prosody: {C}omputational models for processing spontaneous
  speech}, 7--26. Dordrecht, Heidelberg, London, New York: Springer.

\bibitem[{Bigi \& Hirst(2012)}]{bigi2012speech}
Bigi, Brigitte \& Daniel Hirst. 2012.
\newblock {SP}eech {P}honetization {A}lignment and {S}yllabification ({SPPAS}):
  {A} tool for the automatic analysis of speech prosody.
\newblock In Qiuwu Ma, Hongwei Ding \& Daniel Hirst (eds.), \emph{Proceedings
  of the 5th {I}nternational {C}onference on {S}peech {P}rosody}, vol.~1,
  19--22. Shanghai: Tongji University Press.

\bibitem[{Black \& Hunt(1996)}]{black1996generating}
Black, Alan \& Andrew Hunt. 1996.
\newblock Generating f0 contours from {ToBI} labels using linear regression.
\newblock In \emph{Proceedings of the 4th {International Conference on Spoken
  Language Processing}}, vol.~3, 1385--1388. Philadelphia.

\bibitem[{Blesser(1972)}]{blesser1972speech}
Blesser, Barry. 1972.
\newblock Speech perception under conditions of spectral transformation: {I}.
  {P}honetic characteristics.
\newblock \emph{Journal of Speech and Hearing Research} 15(1). 5--41.

\bibitem[{Boersma \& Weenink(2008)}]{boersma2008praat}
Boersma, Paul \& David Weenink. 2008.
\newblock Praat: {D}oing phonetics by computer.
\newblock Computer program, retrieved from http://www.praat.org/.

\bibitem[{Bolinger(1951)}]{bolinger1951intonation}
Bolinger, Dwight. 1951.
\newblock Intonation: {L}evels versus configurations.
\newblock \emph{Word} 7. 199--210.

\bibitem[{Bolinger(1964)}]{bolinger1964intonation}
Bolinger, Dwight. 1964.
\newblock Intonation as a universal.
\newblock In Horace Lunt (ed.), \emph{Proceedings of the 9th {I}nternational
  {C}ongress of {L}inguists}, 833--848. The Hague: Mouton.

\bibitem[{Bolinger(1989)}]{bolinger1989intonation}
Bolinger, Dwight. 1989.
\newblock \emph{Intonation and its uses: {M}elody in grammar and discourse}.
\newblock Palo Alto: Stanford University Press.

\bibitem[{Brooks(1978)}]{brooks1978nonanalytic}
Brooks, Lee. 1978.
\newblock Nonanalytic concept formation and memory for instances.
\newblock In Eleanor Rosch \& Barbara Lloyd (eds.), \emph{Cognition and
  categorization}, 170--211. Hillsdale: Erlbaum.

\enlargethispage{\baselineskip}
\bibitem[{Browman \& Goldstein(1986)}]{browman1986articulatory}
Browman, Catherine \& Louis Goldstein. 1986.
\newblock Towards an articulatory phonology.
\newblock \emph{Phonology Yearbook} 3. 219--252.

\bibitem[{Brunetti et~al.(2010)Brunetti, D'Imperio \&
  Cangemi}]{brunetti2010prosodic}
Brunetti, Lisa, Mariapaola D'Imperio \& Francesco Cangemi. 2010.
\newblock On the prosodic marking of contrast in {R}omance sentence topic:
  {E}vidence from {N}eapolitan {I}talian.
\newblock In \emph{Proceedings of the 5th {I}nternational {C}onference on
  {S}peech {P}rosody}, Chicago.

\bibitem[{Bruni(1992)}]{bruni1992italiano}
Bruni, Francesco. 1992.
\newblock \emph{L'italiano nelle regioni. {L}ingua nazionale e identità
  regionali}.
\newblock Torino: Utet.

\bibitem[{B\"{u}ring(1997)}]{buring1997meaning}
B\"{u}ring, Daniel. 1997.
\newblock \emph{The meaning of topic and focus: {T}he 59th street bridge
  accent}.
\newblock London, New York: Routledge.

\bibitem[{Bybee(2001)}]{bybee2001phonology}
Bybee, Joan. 2001.
\newblock \emph{Phonology and language use}.
\newblock Cambridge: Cambridge University Press.

\bibitem[{Bybee(2006)}]{bybee2006usage}
Bybee, Joan. 2006.
\newblock From usage to grammar: {T}he mind's response to repetition.
\newblock \emph{Language} 82(4). 711--733.

\bibitem[{Campbell \& Mokhtari(2003)}]{campbell2003voice}
Campbell, Nick \& Parham Mokhtari. 2003.
\newblock Voice quality: {T}he 4th prosodic dimension.
\newblock In Daniel Recasens, Maria~Josep Sol\'{e} \& Joaquín Romero (eds.),
  \emph{Proceedings of the 15th {I}nternational {C}ongress of {P}honetic
  {S}ciences}, 2417--2420. Barcelona.

\bibitem[{Cangemi(2009)}]{cangemi2009phonetic}
Cangemi, Francesco. 2009.
\newblock Phonetic detail in intonation contour dynamics.
\newblock In Stephan Schmid, Michael Schwarzenbach \& Dieter Studer (eds.),
  \emph{La dimensione temporale del parlato: {P}roceedings of the 5th
  {C}onference of {A}ssociazione {I}taliana di {S}cienze della {V}oce},
  325--334. Torriana: EDK.

\bibitem[{Cangemi et~al.(2011)Cangemi, Cutugno, Ludusan, Seppi \&
  Van~Compernolle}]{cangemi2011automatic}
Cangemi, Francesco, Francesco Cutugno, Bogdan Ludusan, Dino Seppi \& Dirk
  Van~Compernolle. 2011.
\newblock Automatic {S}peech {S}egmentation for {I}talian ({ASSI}): {T}ools,
  models, evaluation and application.
\newblock In Barbara Gili~Fivela, Antonio Stella, Luigia Garrapa \& Mirko
  Grimaldi (eds.), \emph{Contesto comunicativo e variabilità nella produzione
  e percezione della lingua: {P}roceedings of the 7th {C}onference of
  {A}ssociazione {I}taliana di {S}cienze della {V}oce}, Roma: Bulzoni.

\bibitem[{Cangemi \& D'Imperio(2011{\natexlab{a}})}]{cangemi2011local}
Cangemi, Francesco \& Mariapaola D'Imperio. 2011{\natexlab{a}}.
\newblock Local speech rate differences between questions and statements in
  italian.
\newblock In Wai-Sum Lee \& Eric Zee (eds.), \emph{Proceedings of the 17th
  {I}nternational {C}ongress of {P}honetic {S}ciences}, 392--395. Hong Kong:
  City University of Hong Kong.

\bibitem[{Cangemi \& D'Imperio(2011{\natexlab{b}})}]{cangemi2011prosodia}
Cangemi, Francesco \& Mariapaola D'Imperio. 2011{\natexlab{b}}.
\newblock Prosodia oltre la f0: {T}empo e modalità.
\newblock In Barbara Gili~Fivela, Antonio Stella, Luigia Garrapa \& Mirko
  Grimaldi (eds.), \emph{Contesto comunicativo e variabilità nella produzione
  e percezione della lingua: {P}roceedings of the 7th {C}onference of
  {A}ssociazione {I}taliana di {S}cienze della {V}oce}, Roma: Bulzoni.

\bibitem[{Cangemi \& D'Imperio(2013)}]{cangemiFORTHtempo}
Cangemi, Francesco \& Mariapaola D'Imperio. 2013.
\newblock Tempo and the perception of sentence modality.
\newblock \emph{Laboratory Phonology} 4(1). 191--219.

\enlargethispage{\baselineskip}
\bibitem[{Cangemi \& D'Imperio(forthcoming)}]{cangemiFORTHbeyond}
Cangemi, Francesco \& Mariapaola D'Imperio. forthcoming.
\newblock Beyond f0: {S}entence modality and speech rate.
\newblock In Joaquín Romero \& Maria Riera (eds.), \emph{Selected papers from
  the 5th {C}onference on {P}honetics and {P}honology in {I}beria}, Amsterdam:
  John Benjamins.

\bibitem[{Caputo(1994)}]{caputo1994intonazione}
Caputo, Maria~Rosaria. 1994.
\newblock L'intonazione delle domande s{\`i}/no in un campione di italiano
  parlato.
\newblock In \emph{Proceedings of the 4th {Gruppo di Fonetica Sperimentale
  Workshop}}, 9--18. Torino.

\bibitem[{Caputo(1996)}]{caputo1996presupposizione}
Caputo, Maria~Rosaria. 1996.
\newblock Presupposizione, fuoco, modalità e schemi melodici.
\newblock In \emph{Proceedings of the 24th {National Congress of Associazione
  Italiana di Acustica}}, 49--54. Trento.

\bibitem[{Caputo \& D'Imperio(1995)}]{caputo1995possibile}
Caputo, Maria~Rosaria \& Mariapaola D'Imperio. 1995.
\newblock Verso un possibile sistema di trascrizione prosodica dell’italiano:
  {C}enni preliminari.
\newblock In \emph{Proceedings of the 5th workshop of {Gruppo di Fonetica
  Sperimentale}}, 71--83. Trento.

\bibitem[{Charles-Luce(1985)}]{charlesluce1985word}
Charles-Luce, Jan. 1985.
\newblock Word-final devoicing in {G}erman and the effects of phonetic and
  sentential contexts.
\newblock \emph{Journal of Phonetics} 13. 309--324.

\bibitem[{Charles-Luce \& Dinnsen(1987)}]{charlesluce1987reanalysis}
Charles-Luce, Jan \& Daniel Dinnsen. 1987.
\newblock A reanalysis of {C}atalan devoicing.
\newblock \emph{Journal of Phonetics} 15(2). 187--190.

\bibitem[{Chomsky(1965)}]{chomsky1965aspects}
Chomsky, Noam. 1965.
\newblock \emph{Aspects of the theory of syntax}.
\newblock Cambridge: MIT Press.

\bibitem[{Church \& Schacter(1994)}]{church1994perceptual}
Church, Barbara \& Daniel Schacter. 1994.
\newblock Perceptual specificity of auditory priming: {I}mplicit memory for
  voice intonation and fundamental frequency.
\newblock \emph{Journal of Experimental Psychology: Learning, Memory, and
  Cognition} 20(3). 521--533.

\bibitem[{Cole \& Shattuck-Hufnagel(2011)}]{cole2011phonology}
Cole, Jennifer \& Stefanie Shattuck-Hufnagel. 2011.
\newblock The phonology and phonetics of perceived prosody: {W}hat do listeners
  imitate?
\newblock In \emph{Proceedings of the 12th {Annual Conference of the
  International Speech Communication Association}}, 969--972. Firenze.

\bibitem[{Coleman(2003)}]{coleman2003discovering}
Coleman, John. 2003.
\newblock Discovering the acoustic correlates of phonological contrasts.
\newblock \emph{Journal of Phonetics} 31(3-4). 351--372.

\bibitem[{Cooper et~al.(1952)Cooper, Delattre, Liberman, Borst \&
  Gerstman}]{cooper1952experiments}
Cooper, Franklin, Pierre Delattre, Alvin Liberman, John Borst \& Louis
  Gerstman. 1952.
\newblock Some experiments on the perception of synthetic speech sounds.
\newblock \emph{Journal of the Acoustical Society of America} 24(6). 597--606.

\bibitem[{Cooper \& Paccia-Cooper(1980)}]{cooper1980syntax}
Cooper, William \& Jeanne Paccia-Cooper. 1980.
\newblock \emph{Syntax and speech}.
\newblock Cambridge: Harvard University Press.

\bibitem[{Dahan et~al.(2002)Dahan, Tanenhaus \& Chambers}]{dahan2002accent}
Dahan, Delphine, Michael Tanenhaus \& Craig Chambers. 2002.
\newblock Accent and reference resolution in spoken-language comprehension.
\newblock \emph{Journal of Memory and Language} 47(2). 292--314.

\bibitem[{De~Dominicis(2010)}]{dedominicis2010interrogative}
De~Dominicis, Amedeo. 2010.
\newblock Interrogative e assertive in un corpus dialettale recuperato
  ({B}omarzo).
\newblock In Francesco Cutugno, Pietro Maturi, Renata Savy, Giovanni Abete \&
  Iolanda Alfano (eds.), \emph{Parlare con le persone, parlare alle macchine:
  {L}a dimensione interazionale della comunicazione verbale: {P}roceedings of
  the 6th {C}onference of {A}ssociazione {I}taliana di {S}cienze della {V}oce},
  Torriana: EDK.


\enlargethispage{2\baselineskip}
\bibitem[{De~Mauro(1970)}]{demauro1970storia}
De~Mauro, Tullio. 1970.
\newblock \emph{Storia linguistica dell'{I}talia unita (nuova edizione)}.
\newblock Laterza: Laterza.

\bibitem[{Del~Giudice et~al.(2007)Del~Giudice, Shosted, Davidson, Salihie \&
  Arvaniti}]{delgiudice2007comparing}
Del~Giudice, Alex, Ryan Shosted, Kathryn Davidson, Mohammad Salihie \& Amalia
  Arvaniti. 2007.
\newblock Comparing methods for locating pitch ``elbows''.
\newblock In Jürgen Trouvain \& William Barry (eds.), \emph{Proceedings of the
  16th {I}nternational {C}ongress of {P}honetic {S}ciences}, 1117--1120.
  Saarbr\"{u}cken.

\bibitem[{Delattre(1966)}]{delattre1966dix}
Delattre, Pierre. 1966.
\newblock Les dix intonations de base du fran\c{c}ais.
\newblock \emph{The French Review} 40(1). 1--14.

\bibitem[{Delattre et~al.(1955)Delattre, Liberman \&
  Cooper}]{delattre1955acoustic}
Delattre, Pierre, Alvin Liberman \& Franklin Cooper. 1955.
\newblock Acoustic loci and transitional cues for consonants.
\newblock \emph{Journal of the Acoustical Society of America} 27(4). 769--773.

\bibitem[{D'Imperio(1995)}]{dimperio1995timing}
D'Imperio, Mariapaola. 1995.
\newblock Timing differences between prenuclear and nuclear pitch accents in
  {I}talian.
\newblock \emph{Journal of the Acoustical Society of America} 98(5). 2894.

\bibitem[{D'Imperio(1996)}]{dimperio1996caratteristiche}
D'Imperio, Mariapaola. 1996.
\newblock Caratteristiche di timing degli accenti nucleari in parlato italiano
  letto.
\newblock In \emph{Proceedings of the 24th {National Congress of Associazione
  Italiana di Acustica}}, 55--60. Trento.

\bibitem[{D'Imperio(1997{\natexlab{a}})}]{dimperio1997breadth}
D'Imperio, Mariapaola. 1997{\natexlab{a}}.
\newblock Breadth of focus, modality, and prominence perception in {N}eapolitan
  {I}talian.
\newblock \emph{Working Papers in Linguistics -- Ohio State University} 50.
  19--39.

\bibitem[{D'Imperio(1997{\natexlab{b}})}]{dimperio1997narrow}
D'Imperio, Mariapaola. 1997{\natexlab{b}}.
\newblock Narrow focus and focal accent in the {N}eapolitan variety of
  {I}talian.
\newblock In Antonis Botinis, Georgios Kouroupetroglou \& George Carayiannis
  (eds.), \emph{Intonation: {T}heory, models and applications. {Proceedings of
  an European Conference on Speech Communication and Technology Workshop}},
  87--90. Athens.

\bibitem[{D'Imperio(1999)}]{dimperio1999tonal}
D'Imperio, Mariapaola. 1999.
\newblock Tonal structure and pitch targets in {I}talian focus constituents.
\newblock In John Ohala (ed.), \emph{Proceedings of the 14th {I}nternational
  {C}ongress of {P}honetic {S}ciences}, 1757--1760. San Francisco: University
  of California.

\bibitem[{D'Imperio(2000)}]{dimperio2000role}
D'Imperio, Mariapaola. 2000.
\newblock \emph{The role of perception in defining tonal targets and their
  alignment}: Columbus: The Ohio State University dissertation.

\bibitem[{D'Imperio(2001)}]{dimperio2001focus}
D'Imperio, Mariapaola. 2001.
\newblock Focus and tonal structure in neapolitan italian.
\newblock \emph{Speech Communication} 33(4). 339--356.

\bibitem[{D'Imperio(2002)}]{dimperio2002italian}
D'Imperio, Mariapaola. 2002.
\newblock Italian intonation: {A}n overview and some questions.
\newblock \emph{Probus} 14(1). 37--69.

\bibitem[{D'Imperio(2003)}]{dimperio2003tonal}
D'Imperio, Mariapaola. 2003.
\newblock Tonal structure and pitch targets in {I}talian focus constituents.
\newblock \emph{Catalan Journal of Linguistics} 2. 55--65.


%%\enlargethispage{\baselineskip}
\bibitem[{D'Imperio \& Cangemi(2009)}]{dimperio2009interplay}
D'Imperio, Mariapaola \& Francesco Cangemi. 2009.
\newblock The interplay between tonal alignment and rise shape in the
  perception of two {N}eapolitan rising accents.
\newblock Talk presented at the 4th Conference on Phonetics and Phonology in
  Iberia, Gran Canaria, Spain.

\enlargethispage{\baselineskip}
\bibitem[{D'Imperio \& Cangemi(2011)}]{dimperio2011phrasing}
D'Imperio, Mariapaola \& Francesco Cangemi. 2011.
\newblock Phrasing, register level downstep and partial topic constructions in
  {N}eapolitan {I}talian.
\newblock In Christoph Gabriel \& Conxita Lle\'{o} (eds.), \emph{Intonational
  phrasing in {R}omance and {G}ermanic: {C}ross-linguistic and bilingual
  studies}, 75--94. Amsterdam: John Benjamins.

\bibitem[{D'Imperio et~al.(2008)D'Imperio, Cangemi \&
  Brunetti}]{dimperio2008phonetics}
D'Imperio, Mariapaola, Francesco Cangemi \& Lisa Brunetti. 2008.
\newblock The phonetics and phonology of contrastive topic constructions in
  {I}talian.
\newblock Poster presented at the 3rd Conference on Tone and Intonation in
  Europe, Lisbon, Portugal.

\bibitem[{D'Imperio et~al.(2005)D'Imperio, Elordieta, Frota, Prieto \&
  Vig\`{a}rio}]{dimperio2005intonational}
D'Imperio, Mariapaola, Gorka Elordieta, Sónia Frota, Pilar Prieto \& Marina
  Vig\`{a}rio. 2005.
\newblock Intonational phrasing in {R}omance: {T}he role of syntactic and
  prosodic structure.
\newblock In Sónia Frota, Marina Vig\`{a}rio \& Maria Freitas (eds.),
  \emph{Prosodies}, 59--97. Berlin, New York: Mouton de Gruyter.

\bibitem[{D'Imperio \& Gili~Fivela(2003)}]{dimperio2003levels}
D'Imperio, Mariapaola \& Barbara Gili~Fivela. 2003.
\newblock How many levels of phrasing? {E}vidence from two varieties of
  {I}talian.
\newblock In John Local, Richard Ogden \& Rosalind Temple (eds.), \emph{Papers
  in {L}aboratory {P}honology}, vol.~6, 38--57. Cambridge: Cambridge University
  Press.

\bibitem[{D'Imperio \& House(1997)}]{dimperio1997perception}
D'Imperio, Mariapaola \& David House. 1997.
\newblock Perception of questions and statements in {N}eapolitan {I}talian.
\newblock In George Kokkinakis, Nikos Fakotakis \& Evangelos Dermatas (eds.),
  \emph{Proceedings of the 5th {European Conference on Speech Communication and
  Technology}}, 251--254. Rhodes.

\bibitem[{D'Imperio et~al.(2007)D'Imperio, Petrone \&
  Nguyen}]{dimperio2007effects}
D'Imperio, Mariapaola, Caterina Petrone \& Noël Nguyen. 2007.
\newblock Effects of tonal alignment on lexical identification in {I}talian.
\newblock In Tomas Riad \& Carlos Gussenhoven (eds.), \emph{Tones and tunes:
  {E}xperimental studies in word and sentence prosody}, vol.~2, 79--106.
  Berlin: de Gruyter.

\bibitem[{Dinnsen \& Charles-Luce(1984)}]{dinnsen1984phonological}
Dinnsen, Daniel \& Jan Charles-Luce. 1984.
\newblock Phonological neutralization, phonetic implementation and individual
  differences.
\newblock \emph{Journal of Phonetics} 12(1). 49--60.

\bibitem[{Dinnsen \& {Garcia Zamor}(1971)}]{dinnsen1971three}
Dinnsen, Daniel \& Maria {Garcia Zamor}. 1971.
\newblock The three degrees of vowel length in {G}erman.
\newblock \emph{Research on Language \& Social Interaction} 4(1). 111--126.

\bibitem[{Dmitrieva et~al.(2010)Dmitrieva, Jongman \&
  Sereno}]{dmitrieva2010phonological}
Dmitrieva, Olga, Allard Jongman \& Joan Sereno. 2010.
\newblock Phonological neutralization by native and non-native speakers: {T}he
  case of {R}ussian final devoicing.
\newblock \emph{Journal of Phonetics} 38(3). 483--492.

\bibitem[{Dombrowski \& Niebuhr(2005)}]{dombrowski2005acoustic}
Dombrowski, Ernst \& Oliver Niebuhr. 2005.
\newblock Acoustic patterns and communicative functions of phrase-final f0
  rises in {G}erman: {A}ctivating and restricting contours.
\newblock \emph{Phonetica} 62(2-4). 176--195.

\bibitem[{Dryer(2011)}]{wals-2011-116}
Dryer, Matthew. 2011.
\newblock Polar questions.
\newblock In Matthew Dryer \& Martin Haspelmath (eds.), \emph{{The World Atlas
  of Language Structures Online}}, Munich: Max Planck Digital Library.

\bibitem[{Duncan(1972)}]{duncan1972signals}
Duncan, Starkey. 1972.
\newblock Some signals and rules for taking speaking turns in conversations.
\newblock \emph{Journal of Personality and Social Psychology} 23(2). 283--292.

\enlargethispage{\baselineskip}
\bibitem[{Eefting(1991)}]{eefting1991effect}
Eefting, Wieke. 1991.
\newblock The effect of ‘‘information value’’and
  ‘‘accentuation’’ on the duration of {D}utch words, syllables, and
  segments.
\newblock \emph{Journal of the Acoustical Society of America} 89(1). 412--424.

\bibitem[{Elman \& McClelland(1988)}]{elman1988cognitive}
Elman, Jeffrey \& James McClelland. 1988.
\newblock Cognitive penetration of the mechanisms of perception: {C}ompensation
  for coarticulation of lexically restored phonemes.
\newblock \emph{Journal of Memory and Language} 27(2). 143--165.

\bibitem[{Ernestus(2014)}]{ernestusacoustic}
Ernestus, Mirjam. 2014.
\newblock Acoustic reduction and the roles of abstractions and exemplars in
  speech processing.
\newblock \emph{Lingua} 142. 27--41.

\bibitem[{Ernestus \& Baayen(2006)}]{ernestus2006functionality}
Ernestus, Mirjam \& Harald Baayen. 2006.
\newblock The functionality of incomplete neutralization in {D}utch: {T}he case
  of past-tense formation.
\newblock In Louis Goldstein, Douglas Whalen \& Catherine Best (eds.),
  \emph{Papers in {L}aboratory {P}honology}, vol.~8, 27--49. Cambridge:
  Cambridge University Press.

\bibitem[{Faber(1992)}]{faber1992phonemic}
Faber, Alice. 1992.
\newblock Phonemic segmentation as epiphenomenon: {E}vidence from the history
  of alphabetic writing.
\newblock In Pamela Downing, Susan Lima \& Michael Noonan (eds.), \emph{The
  linguistics of literacy}, 111--134. Amsterdam: John Benjamins.

\bibitem[{Face(2001)}]{face2001focus}
Face, Timothy. 2001.
\newblock Focus and early peak alignment in {S}panish intonation.
\newblock \emph{Probus} 13(2). 223--246.

\bibitem[{Farnetani \& Kori(1991)}]{farnetani1990rhytmic}
Farnetani, Edda \& Shiro Kori. 1991.
\newblock Rhytmic structure in {I}talian noun phrases: {A} study on vowel
  durations.
\newblock \emph{Phonetica} 47. 50--65.

\bibitem[{Firth(1948)}]{firth1948prosodies}
Firth, John. 1948.
\newblock Sounds and prosodies.
\newblock \emph{Transactions of the Philological Society} 47(1). 127--152.

\bibitem[{Flemming(1997)}]{flemming1997phonetic}
Flemming, Edward. 1997.
\newblock Phonetic detail in phonology: {T}owards a unified account of
  assimilation and coarticulation.
\newblock In Keiichiro Suzuki \& Dirk Elzinga (eds.), \emph{Proceedings of the
  1995 {Southwestern Workshop in Optimality Theory (SWOT)}}, Tucson.

\bibitem[{Flemming(2001)}]{flemming2001scalar}
Flemming, Edward. 2001.
\newblock Scalar and categorical phenomena in a unified model of phonetics and
  phonology.
\newblock \emph{Phonology} 18(1). 7--44.

\bibitem[{Fourakis \& Iverson(1984)}]{fourakis1984incomplete}
Fourakis, Marios \& Gregory Iverson. 1984.
\newblock On the ‘incomplete neutralization’ of {G}erman final obstruents.
\newblock \emph{Phonetica} 41(3). 140--149.

\bibitem[{Frick(1995)}]{frick1995accepting}
Frick, Robert. 1995.
\newblock Accepting the null hypothesis.
\newblock \emph{Memory \& Cognition} 23(1). 132--138.

\bibitem[{Frota et~al.(2007)Frota, D'Imperio, Elordieta, Prieto \&
  Vig\`{a}rio}]{frota2007phonetics}
Frota, Sónia, Mariapaola D'Imperio, Gorka Elordieta, Pilar Prieto \& Marina
  Vig\`{a}rio. 2007.
\newblock The phonetics and phonology of intonational phrasing in {R}omance.
\newblock In Pilar Prieto, Joan Mascar\`{o} \& Maria~Josep Sol\'{e} (eds.),
  \emph{Segmental and prosodic issues in {R}omance phonology}, 131--154.
  Amsterdam: John Benjamins.

%\enlargethispage{\baselineskip}
\bibitem[{Fujisaki \& Hirose(1982)}]{fujisaki1982modelling}
Fujisaki, Hiroya \& Keikichi Hirose. 1982.
\newblock Modelling the dynamic characteristics of voice fundamental frequency
  with application to analysis and synthesis of intonation.
\newblock In Shiro Hattori \& Kazuko Inoue (eds.), \emph{Proceedings of 13th
  {International Congress of Linguists}}, 57--70. Tokyo.

\bibitem[{Gili~Fivela(2004)}]{gilifivela2004phonetics}
Gili~Fivela, Barbara. 2004.
\newblock \emph{The phonetics and phonology of intonation: {T}he case of {P}isa
  {I}talian}: Pisa: Scuola Normale Superiore dissertation.

\bibitem[{Gili~Fivela(2006)}]{gilifivela2006coding}
Gili~Fivela, Barbara. 2006.
\newblock The coding of target alignment and scaling in pitch accent
  transcription.
\newblock \emph{Italian Journal of Linguistics} 18(1). 189--221.

\bibitem[{Gili~Fivela(2008)}]{gilifivela2008intonation}
Gili~Fivela, Barbara. 2008.
\newblock \emph{Intonation in production and perception: {T}he case of {P}isa
  {I}talian}.
\newblock Alessandria: Edizioni dell'{O}rso.

\bibitem[{Gili~Fivela \& D'Imperio(2003)}]{gilifivela2003tonal}
Gili~Fivela, Barbara \& Mariapaola D'Imperio. 2003.
\newblock Tonal alignment of prenuclear accents in {I}talian.
\newblock Poster presented at the 2nd Conference on Tone and Intonation in
  Europe, Santorini, Greece.

\bibitem[{Goldinger(1996)}]{goldinger1996words}
Goldinger, Stephen. 1996.
\newblock Words and voices: {E}pisodic traces in spoken word identification and
  recognition memory.
\newblock \emph{Journal of Experimental Psychology: Learning, Memory, and
  Cognition} 22(5). 1166--1183.

\bibitem[{Goldinger et~al.(1991)Goldinger, Pisoni \&
  Logan}]{goldinger1991nature}
Goldinger, Stephen, David Pisoni \& John Logan. 1991.
\newblock On the nature of talker variability effects on recall of spoken word
  lists.
\newblock \emph{Journal of Experimental Psychology: Learning, Memory, and
  Cognition} 17(1). 152--162.

\bibitem[{Goldman(2011)}]{goldman2011easyalign}
Goldman, Jean-Philippe. 2011.
\newblock Easy{A}lign: {A}n automatic phonetic alignment tool under {P}raat.
\newblock In \emph{Proceedings of the 12th {Annual Conference of the
  International Speech Communication Association}}, Firenze.

\bibitem[{Grice(1991)}]{grice1991intonation}
Grice, Martine. 1991.
\newblock The intonation of interrogation in two varieties of {S}icilian
  {I}talian.
\newblock In \emph{Proceedings of the 12th {I}nternational {C}ongress of
  {P}honetic {S}ciences}, vol.~5, 210--213. Aix-en-Provence.

\bibitem[{Grice(1995)}]{grice1995intonation}
Grice, Martine. 1995.
\newblock \emph{The intonation of interrogation in {P}alermo {I}talian:
  {I}mplications for intonation theory}.
\newblock T\"{u}bingen: Niemeyer.

\bibitem[{Grice \& Baumann(2002)}]{grice2002deutsche}
Grice, Martine \& Stefan Baumann. 2002.
\newblock Deutsche {I}ntonation und {GToBI}.
\newblock \emph{Linguistische Berichte} 191. 267--298.

\bibitem[{Grice et~al.(2005{\natexlab{a}})Grice, Baumann \&
  Benzm\"{u}lller}]{grice2005german}
Grice, Martine, Stefan Baumann \& Ralf Benzm\"{u}lller. 2005{\natexlab{a}}.
\newblock German intonation in autosegmental-metrical phonology.
\newblock In Sun-Ah Jun (ed.), \emph{Prosodic typology: {T}he phonology of
  intonation and phrasing}, 55--83. Oxford: Oxford University Press.

\bibitem[{Grice et~al.(2005{\natexlab{b}})Grice, D'Imperio, Savino \&
  Avesani}]{grice2005strategy}
Grice, Martine, Mariapaola D'Imperio, Michelina Savino \& Cinzia Avesani.
  2005{\natexlab{b}}.
\newblock Towards a strategy for labelling varieties of italian.
\newblock In Sun-Ah Jun (ed.), \emph{Prosodic typology: {T}he phonology of
  intonation and phrasing}, 362--389. Oxford: Oxford University Press.

\bibitem[{Grice et~al.(2000)Grice, Ladd \& Arvaniti}]{grice2000place}
Grice, Martine, Robert Ladd \& Amalia Arvaniti. 2000.
\newblock On the place of phrase accents in intonational phonology.
\newblock \emph{Phonology} 17(2). 143--185.

\bibitem[{Gubian et~al.(2010)Gubian, Cangemi \& Boves}]{gubian2010automatic}
Gubian, Michele, Francesco Cangemi \& Lou Boves. 2010.
\newblock Automatic and data driven pitch contour manipulation with functional
  data analysis.
\newblock In \emph{Proceedings of the 5th {I}nternational {C}onference on
  {S}peech {P}rosody}, Chicago.

\enlargethispage{\baselineskip}
\bibitem[{Gubian et~al.(2011)Gubian, Cangemi \& Boves}]{gubian2011joint}
Gubian, Michele, Francesco Cangemi \& Lou Boves. 2011.
\newblock Joint analysis of f0 and speech rate with functional data analysis.
\newblock In \emph{Proceedings of 36th {I}nternational {C}onference of
  {A}coustics, {S}peech and {S}ignal {P}rocessing}, 4972--4975. Prague.

\bibitem[{Gussenhoven(1984)}]{gussenhoven1984grammar}
Gussenhoven, Carlos. 1984.
\newblock \emph{On the grammar and semantics of sentence accents}.
\newblock Dordrecht: Foris.

\bibitem[{Gussenhoven(2004)}]{gussenhoven2004phonology}
Gussenhoven, Carlos. 2004.
\newblock \emph{The phonology of tone and intonation}.
\newblock Cambridge: Cambridge University Press.

\bibitem[{Gussenhoven(2006)}]{gussenhoven2006experimental}
Gussenhoven, Carlos. 2006.
\newblock Experimental approaches to establishing discreteness of intonational
  contrasts.
\newblock In Stefan Sudhoff, Denisa Lenertov\'{a}, Roland Meyer, Sandra
  Pappert, Petra Augurzky, Ina Mleinek, Nicole Richter \& Johannes
  Schlie\ss{}er (eds.), \emph{Methods in empirical prosody research}, 321--334.
  Berlin: De Gruyter.

\bibitem[{Haan(2002)}]{haan2002speaking}
Haan, Judith. 2002.
\newblock \emph{Speaking of questions}.
\newblock Utrecht: LOT.

\bibitem[{Harris(1955)}]{harris1955phoneme}
Harris, Zellig. 1955.
\newblock From phoneme to morpheme.
\newblock \emph{Language} 31(2). 190--222.

\bibitem[{Hawkins(2003)}]{hawkins2003roles}
Hawkins, Sarah. 2003.
\newblock Roles and representations of systematic fine phonetic detail in
  speech understanding.
\newblock \emph{Journal of Phonetics} 31(3-4). 373--405.

\bibitem[{Hawkins(2010)}]{hawkins2010phonetic}
Hawkins, Sarah. 2010.
\newblock Phonetic variation as communicative system: {P}erception of the
  particular and the abstract.
\newblock In C\'{e}cile Fougeron, Barbara K\"{u}hnert, Mariapaola D'Imperio \&
  Nathalie Vall\'{e}e (eds.), \emph{Papers in {L}aboratory {P}honology},
  vol.~10, 479--510. Cambridge: Cambridge University Press.

\bibitem[{Hawkins(2011)}]{hawkins2011phonetic}
Hawkins, Sarah. 2011.
\newblock Does phonetic detail guide situation-specific speech recognition?
\newblock In Wai-Sum Lee \& Eric Zee (eds.), \emph{Proceedings of the 17th
  {I}nternational {C}ongress of {P}honetic {S}ciences}, 9--18. Hong Kong: City
  University of Hong Kong.

\bibitem[{Hawkins \& Nguyen(2003)}]{hawkins2003effects}
Hawkins, Sarah \& Noël Nguyen. 2003.
\newblock Effects on word recognition of syllable-onset cues to syllable-coda
  voicing.
\newblock In John Local, Richard Ogden \& Rosalind Temple (eds.), \emph{Papers
  in {L}aboratory {P}honology}, vol.~6, 38--57. Cambridge: Cambridge University
  Press.

\bibitem[{Hawkins \& Nguyen(2004)}]{hawkins2004influence}
Hawkins, Sarah \& Noël Nguyen. 2004.
\newblock Influence of syllable-coda voicing on the acoustic properties of
  syllable-onset /l/ in {E}nglish.
\newblock \emph{Journal of Phonetics} 32(2). 199--231.

\bibitem[{Hawkins \& Smith(2001)}]{hawkins2001polysp}
Hawkins, Sarah \& Rachel Smith. 2001.
\newblock Polysp: {A} polysystemic, phonetically-rich approach to speech
  understanding.
\newblock \emph{Italian Journal of Linguistics} 13. 99--188.

\bibitem[{Hawkins \& Warren(1991)}]{hawkins1991factors}
Hawkins, Sarah \& Paul Warren. 1991.
\newblock Factors affecting the given-new distinction in speech.
\newblock In \emph{Proceedings of the 12th {I}nternational {C}ongress of
  {P}honetic {S}ciences}, 66--69. Aix-en-Provence: Universit\'{e} de Provence.

\bibitem[{Heinrich et~al.(2010)Heinrich, Flory \&
  Hawkins}]{heinrich2010influence}
Heinrich, Antje, Yvonne Flory \& Sarah Hawkins. 2010.
\newblock Influence of english r-resonances on intelligibility of speech in
  noise for native {E}nglish and {G}erman listeners.
\newblock \emph{Speech Communication} 52(11). 1038--1055.

\bibitem[{Henriksen(2012)}]{henriksen2012intonation}
Henriksen, Nicholas. 2012.
\newblock The intonation and signaling of declarative questions in {M}anchego
  {P}eninsular {S}panish.
\newblock \emph{Language and Speech} 55(4).

\enlargethispage{\baselineskip}
\bibitem[{Hintzman(1986)}]{hintzman1986schema}
Hintzman, Douglas. 1986.
\newblock Schema abstraction in a multiple-trace memory model.
\newblock \emph{Psychological Review} 93(4). 411--428.

\bibitem[{Hirschberg \& Ward(1992)}]{hirschberg1992influence}
Hirschberg, Julia \& Gregory Ward. 1992.
\newblock The influence of pitch range, duration, amplitude and spectral
  features on the interpretation of the rise-fall-rise intonation contour in
  {E}nglish.
\newblock \emph{Journal of {P}honetics} 20. 241--251.

\bibitem[{Hooper(1976)}]{hooper1976word}
Hooper, Joan. 1976.
\newblock Word frequency in lexical diffusion and the source of
  morphophonological change.
\newblock In William Christie (ed.), \emph{Current progress in historical
  linguistics. {P}roceedings of the 2nd {I}nternational {C}onference on
  {H}istorical {L}inguistics}, 96--105. Amsterdam: North-Holland.

\bibitem[{House(1990)}]{house1990tonal}
House, David. 1990.
\newblock \emph{Tonal perception in speech}.
\newblock Lund: Lund University Press.

\bibitem[{House(1997)}]{house1997perceptual}
House, David. 1997.
\newblock Perceptual thresholds and tonal categories.
\newblock \emph{Phonum} 4. 179--182.

\bibitem[{Hualde(2002)}]{hualde2002intonation}
Hualde, José. 2002.
\newblock Intonation in {S}panish and the other {Ibero-Romance} languages:
  {O}verview and status quaestionis.
\newblock In Caroline Wiltshire \& Joaquim Camps (eds.), \emph{Romance
  phonology and variation}, 101--116. Amsterdam: John Benjamins.

\bibitem[{Huddleston(1994)}]{huddleston1994contrast}
Huddleston, Rodney. 1994.
\newblock The contrast between interrogatives and questions.
\newblock \emph{Journal of Linguistics} 30(2). 411--39.

\bibitem[{{Institut f\"{u}r Phonetik und digitale
  Sprachverarbeitung}(1994)}]{ipds1994kcrs}
{Institut f\"{u}r Phonetik und digitale Sprachverarbeitung}. 1994.
\newblock The {K}iel {C}orpus of {R}ead {S}peech.
\newblock CD rom.

\bibitem[{Isa{\v{c}}enko \& Sch{\"a}dlich(1970)}]{isavcenko1970untersuchungen}
Isa{\v{c}}enko, Aleksandr \& Hans~Joachim Sch{\"a}dlich. 1970.
\newblock \emph{Untersuchungen {\"u}ber die deutsche {S}atzintonation}.
\newblock Berlin: Mouton.

\bibitem[{Jackendoff(1972)}]{jackendoff1972semantic}
Jackendoff, Ray. 1972.
\newblock \emph{Semantic interpretation in generative grammar}.
\newblock Cambridge: MIT Press.

\bibitem[{Jakobson et~al.(1952)Jakobson, Fant \&
  Halle}]{jakobson1952preliminaries}
Jakobson, Roman, Gunnar Fant \& Morris Halle. 1952.
\newblock Preliminaries to speech analysis: {T}he distinctive features.
\newblock MIT {A}coustics {L}aboratory technical report.

\bibitem[{Jassem \& Richter(1989)}]{jassem1989neutralization}
Jassem, Wiktor \& Lutosława Richter. 1989.
\newblock Neutralization of voicing in {P}olish obstruents.
\newblock \emph{Journal of Phonetics} 17(4). 317--326.

\bibitem[{Johnson(1997)}]{johnson1997speech}
Johnson, Keith. 1997.
\newblock Speech perception without speaker normalization.
\newblock In Keith Johnson \& John Mullennix (eds.), \emph{Talker variability
  in speech processing}, 145--165. San Diego: Academic Press.

\bibitem[{Johnson \& Mullennix(1997)}]{johnson1997complex}
Johnson, Keith \& John Mullennix. 1997.
\newblock Complex representations used in speech processing.
\newblock In Keith Johnson \& John Mullennix (eds.), \emph{Talker variability
  in speech processing}, 1--8. San Diego: Academic Press.

\bibitem[{Jun(1993)}]{jun1993phonetics}
Jun, Sun-Ah. 1993.
\newblock \emph{The phonetics and phonology of {K}orean prosody}: Columbus: The
  Ohio State University dissertation.

\bibitem[{Jusczyk \& Luce(2002)}]{jusczyk2002speech}
Jusczyk, Peter \& Paul Luce. 2002.
\newblock Speech perception and spoken word recognition: {P}ast and present.
\newblock \emph{Ear and Hearing} 23(1). 2--40.

\enlargethispage{\baselineskip}
\bibitem[{Kelly \& Local(1986)}]{kelly1986long}
Kelly, John \& John Local. 1986.
\newblock Long-domain resonance patterns in {E}nglish.
\newblock In \emph{Proceedings of the {International Conference on Speech
  Input/Output, Techniques and Applications}}, 304--308. London.

\bibitem[{Kirsner et~al.(1994)Kirsner, {van Heuven} \& {van
  Bezooijen}}]{kirsner1994interaction}
Kirsner, Robert, Vincent {van Heuven} \& Renée {van Bezooijen}. 1994.
\newblock Interaction of particle and prosody in the interpretation of factual
  {D}utch sentences.
\newblock In Reineke Bok-Bennema \& Crit Cremers (eds.), \emph{Linguistics in
  the {N}etherlands}, 107--118. Amsterdam: John Benjamins.

\bibitem[{Klatt(1973)}]{klatt1973discrimination}
Klatt, Dennis. 1973.
\newblock Discrimination of fundamental frequency contours in synthetic speech:
  {I}mplications for models of pitch perception.
\newblock \emph{Journal of the Acoustical Society of America} 53(1). 8--16.

\bibitem[{Klatt(1979)}]{klatt1979speech}
Klatt, Dennis. 1979.
\newblock Speech perception: {A} model of acoustic-phonetic analysis and
  lexical access.
\newblock \emph{Journal of Phonetics} 7. 279--312.

\bibitem[{Kleber et~al.(2010)Kleber, John \&
  Harrington}]{kleber2010implications}
Kleber, Felicitas, Tina John \& Jonathan Harrington. 2010.
\newblock The implications for speech perception of incomplete neutralization
  of final devoicing in {G}erman.
\newblock \emph{Journal of Phonetics} 38(2). 185--196.

\bibitem[{Kohler(1987)}]{kohler1987categorical}
Kohler, Klaus. 1987.
\newblock Categorical pitch perception.
\newblock In \emph{Proceedings of the 11th {I}nternational {C}ongress of
  {P}honetic {S}ciences}, vol.~5, 331--333. Tallin: Academy of Sciences.

\bibitem[{Kohler(1991)}]{kohler1991model}
Kohler, Klaus. 1991.
\newblock A model of {G}erman intonation.
\newblock \emph{Arbeitsberichte des Instituts f\"{u}r Phonetik der
  Universit\"{a}t Kiel} 25. 295--360.

\bibitem[{Kopkalli(1993)}]{kopkalli1993phonetic}
Kopkalli, Handan. 1993.
\newblock \emph{A phonetic and phonological analysis of final devoicing in
  {T}urkish}: Ann Arbor: University of Michigan dissertation.

\bibitem[{Kretschmer(1938)}]{kretschmer1938ursprung}
Kretschmer, Paul. 1938.
\newblock Der {U}rsprung des {F}ragetons \& {F}ragesatzes.
\newblock In \emph{Scritti in onore di alfredo trombetti}, 27--50. Milano:
  Hoepli.

\bibitem[{Kruschke(1992)}]{kruschke1992alcove}
Kruschke, John. 1992.
\newblock {ALCOVE}: {A}n exemplar-based connectionist model of category
  learning.
\newblock \emph{Psychological Review} 99(1). 22--44.

\bibitem[{Ladd(1996)}]{ladd1996intonational}
Ladd, Robert. 1996.
\newblock \emph{Intonational phonology}.
\newblock Cambridge: Cambridge University Press.

\bibitem[{Ladd(2008)}]{ladd2008intonational}
Ladd, Robert. 2008.
\newblock \emph{Intonational phonology (2nd edition)}.
\newblock Cambridge: Cambridge University Press.

\bibitem[{Ladd \& Schepman(2003)}]{ladd2003sagging}
Ladd, Robert \& Astrid Schepman. 2003.
\newblock ``{S}agging transitions'' between high pitch accents in {E}nglish:
  {E}xperimental evidence.
\newblock \emph{Journal of Phonetics} 31(1). 81--112.

\bibitem[{Ladefoged(2000)}]{ladefoged2000course}
Ladefoged, Peter. 2000.
\newblock \emph{A course in phonetics (4th edition)}.
\newblock Boston: Heinle \& Heinle.

\bibitem[{Ladefoged \& Broadbent(1957)}]{ladefoged1957information}
Ladefoged, Peter \& Donald Broadbent. 1957.
\newblock Information conveyed by vowels.
\newblock \emph{Journal of the Acoustical Society of America} 29(1). 98--104.

\bibitem[{Lehiste(1970)}]{lehiste1970suprasegmentals}
Lehiste, Ilse. 1970.
\newblock \emph{Suprasegmentals}.
\newblock Cambridge: MIT Press.

\bibitem[{Levi \& Bruno(2010)}]{levi2010priming}
Levi, Susannah \& Jennifer Bruno. 2010.
\newblock Priming at the level of phonetic detail: {E}vidence from voice onset
  time.
\newblock \emph{Journal of the Acoustical Society of America} 127(3). 1853.

\bibitem[{Liberman(1979)}]{liberman1979intonational}
Liberman, Mark. 1979.
\newblock \emph{The intonational system of {E}nglish}.
\newblock New York: Garland.

\bibitem[{Licklider(1952)}]{licklider1952process}
Licklider, Joseph. 1952.
\newblock On the process of speech perception.
\newblock \emph{Journal of the Acoustical Society of America} 24(6). 590--594.

\bibitem[{Lindgren(1965)}]{lindgren1965machine}
Lindgren, Nilo. 1965.
\newblock Machine recognition of human language.
\newblock \emph{Spectrum} 2(3). 114--136.

\bibitem[{Lisker(1986)}]{lisker1986voicing}
Lisker, Leigh. 1986.
\newblock Voicing in {E}nglish: {A} catalogue of acoustic features signaling
  /b/ versus /p/ in trochees.
\newblock \emph{Language and Speech} 29. 3--11.

\bibitem[{Lisker \& Abramson(1964)}]{lisker1964crosslanguage}
Lisker, Leigh \& Arthur Abramson. 1964.
\newblock A cross-language study of voicing in initial stops: {A}coustical
  measurements.
\newblock \emph{Word} 20(3). 384--422.

\bibitem[{Local(2003{\natexlab{a}})}]{local2003phonetics}
Local, John. 2003{\natexlab{a}}.
\newblock Phonetics and talk-in-interaction.
\newblock In Daniel Recasens, Maria~Josep Sol\'{e} \& Joaquín Romero (eds.),
  \emph{Proceedings of the 15th {I}nternational {C}ongress of {P}honetic
  {S}ciences}, 115--118. Barcelona.

\bibitem[{Local(2003{\natexlab{b}})}]{local2003variable}
Local, John. 2003{\natexlab{b}}.
\newblock Variable domains and variable relevance: {I}nterpreting phonetic
  exponents.
\newblock \emph{Journal of Phonetics} 31(3-4). 321--339.

\bibitem[{Luce \& McLennan(2005)}]{luce2005spoken}
Luce, Paul \& Conor McLennan. 2005.
\newblock Spoken word recognition: {T}he challenge of variation.
\newblock In David Pisoni \& Robert Remez (eds.), \emph{The handbook of speech
  perception}, 590--609. Hoboken: Wiley-Blackwell.

\bibitem[{Luce et~al.(2003)Luce, McLennan \&
  Charles-Luce}]{luce2003abstractness}
Luce, Paul, Conor McLennan \& Jan Charles-Luce. 2003.
\newblock Abstractness and specificity in spoken word recognition: {I}ndexical
  and allophonic variability in long-term repetition priming.
\newblock In Jeffrey Bowers \& Chad Marsolek (eds.), \emph{Rethinking implicit
  memory}, 145--165. Oxford: Oxford University Press.

\bibitem[{Lyberg(1981)}]{lyberg1981observations}
Lyberg, Bertil. 1981.
\newblock Some observations on the vowel duration and the fundamental frequency
  contour in {S}wedish utterances.
\newblock \emph{Journal of Phonetics} 9. 261--272.

\bibitem[{Lyons(1977)}]{lyons1977semantics}
Lyons, John. 1977.
\newblock \emph{Semantics}.
\newblock Cambridge: Cambridge University Press.

\bibitem[{Magno~Caldognetto et~al.(1978)Magno~Caldognetto, Ferrero, Lavagnoli
  \& Vagges}]{magnocaldognetto1978f0}
Magno~Caldognetto, Emanuela, Franco Ferrero, Carlo Lavagnoli \& Kyriaki Vagges.
  1978.
\newblock F0 contours of statements, yes-no questions, and wh-questions of two
  regional varieties of {I}talian.
\newblock \emph{Journal of Italian Linguistics} 3(1). 57--66.

\bibitem[{{Manaster Ramer}(1996)}]{manaster1996letter}
{Manaster Ramer}, Alexis. 1996.
\newblock A letter from an incompletely neutral phonologist.
\newblock \emph{Journal of Phonetics} 24(4). 477--489.

\bibitem[{Marchese(1978)}]{marchese1978atlas}
Marchese, Lynell. 1978.
\newblock \emph{Atlas linguistique {K}ru: {E}ssai de typologie}.
\newblock Abidjan: Institut de Linguistique Appliqu\'{e}e.

\bibitem[{Marslen-Wilson \& Tyler(1980)}]{marslen1980temporal}
Marslen-Wilson, William \& Lorraine Tyler. 1980.
\newblock The temporal structure of spoken language understanding.
\newblock \emph{Cognition} 8(1). 1--71.

\bibitem[{Marslen-Wilson \& Welsh(1978)}]{marslen1978processing}
Marslen-Wilson, William \& Alan Welsh. 1978.
\newblock Processing interactions and lexical access during word recognition in
  continuous speech.
\newblock \emph{Cognitive Psychology} 10(1). 29--63.

\bibitem[{Mascar{\'o}(1987)}]{mascaro1987underlying}
Mascar{\'o}, Joan. 1987.
\newblock Underlying voicing recoverability of finally devoiced obstruents in
  {C}atalan.
\newblock \emph{Journal of Phonetics} 15. 183--186.

\bibitem[{Matthews(1993)}]{matthews1993grammatical}
Matthews, Peter. 1993.
\newblock \emph{Grammatical theory in the {U}nited {S}tates from {B}loomfield
  to {C}homsky}.
\newblock Cambridge: Cambridge University Press.

\bibitem[{Maturi(1988)}]{maturi1988intonazione}
Maturi, Pietro. 1988.
\newblock L'intonazione delle frasi dichiarative ed interrogative nella
  variet\`{a} napoletana dell'italiano.
\newblock \emph{Rivista Italiana di Acustica} 12. 13--30.

\bibitem[{McClelland(1981)}]{mcclelland1981retrieving}
McClelland, James. 1981.
\newblock Retrieving general and specific information from stored knowledge of
  specifics.
\newblock In \emph{Proceedings of the 3rd {Annual Conference of the Cognitive
  Science Society}}, 170--172. Berkeley.

\bibitem[{McClelland \& Elman(1986)}]{mcclelland1986trace}
McClelland, James \& Jeffrey Elman. 1986.
\newblock The {TRACE} model of speech perception.
\newblock \emph{Cognitive Psychology} 18(1). 1--86.

\bibitem[{McClelland \& Rumelhart(1985)}]{mcclelland1985distributed}
McClelland, James \& David Rumelhart. 1985.
\newblock Distributed memory and the representation of general and specific
  information.
\newblock \emph{Journal of Experimental Psychology: General} 114(2). 159--197.

\bibitem[{Medin \& Schaffer(1978)}]{medin1978context}
Medin, Douglas \& Marguerite Schaffer. 1978.
\newblock Context theory of classification learning.
\newblock \emph{Psychological Review} 85(3). 207--238.

\bibitem[{Michelas(2011)}]{michelas2011caracterisation}
Michelas, Amandine. 2011.
\newblock \emph{Caract\'{e}risation phon\'{e}tique et phonologique du syntagme
  interm\'{e}diaire en fran\c{c}ais: {D}e la production \`{a} la perception}:
  Aix-en-Provence: Universit\'{e} de Provence dissertation.

\bibitem[{M{\"o}hler(2001)}]{mohler2001improvements}
M{\"o}hler, Gregor. 2001.
\newblock Improvements of the {PaIntE} model for f0 parametrization.
\newblock Research Papers from the Phonetics Lab, IMS Universit\"{a}t
  Stuttgart.

\bibitem[{M{\"o}hler \& Conkie(1998)}]{mohler1998parametric}
M{\"o}hler, Gregor \& Alistair Conkie. 1998.
\newblock Parametric modeling of intonation using vector quantization.
\newblock In \emph{Proceedings of the 3rd {ESCA Workshop on Speech Synthesis}},
  311--316. Jenolan Caves.

\bibitem[{Moulines \& Charpentier(1990)}]{moulines1990pitchsyncronous}
Moulines, Eric \& Francis Charpentier. 1990.
\newblock Pitch-synchronous waveform processing techniques for text-to-speech
  synthesis using diphones.
\newblock \emph{Speech Communication} 9(5). 453--467.

\bibitem[{Nash \& Mulac(1980)}]{nash1980intonation}
Nash, Rose \& Anthony Mulac. 1980.
\newblock The intonation of verifiability.
\newblock In Linda Waugh \& Cornelis {van Schoonevenld} (eds.), \emph{The
  melody of language: {I}ntonation and prosody}, 219--242. Baltimore:
  University Park Press.

\bibitem[{Neukom(1995)}]{neukom1995description}
Neukom, Lukas. 1995.
\newblock \emph{Description grammaticale du nateni ({B}{\'e}nin): {S}yst{\`e}me
  verbal, classification nominale, phrases complexes, textes}.
\newblock Z\"{u}rich: Universit\"{a}t Z\"{u}rich.

\bibitem[{Nguyen et~al.(2009)Nguyen, Wauquier \& Tuller}]{nguyen2009dynamical}
Nguyen, Noël, Sophie Wauquier \& Betty Tuller. 2009.
\newblock The dynamical approach to speech perception: {F}rom fine phonetic
  detail to abstract phonological categories.
\newblock In François Pellegrino, Egidio Marsico, Ioana Chitoran \& Christophe
  Coup{\'e} (eds.), \emph{Approaches to phonological complexity}, 193--217.
  Berlin: Mouton de Gruyter.

\enlargethispage{\baselineskip}
\bibitem[{Niebuhr(2007)}]{niebuhr2007categorical}
Niebuhr, Oliver. 2007.
\newblock Categorical perception in intonation: {A} matter of signal dynamics?
\newblock In Cyril Goutte, Nicola Cancedda, Marc Dymetman \& George Foster
  (eds.), \emph{Proceedings of the 8th {Annual Conference of the International
  Speech Communication Association}}, 642--645. Antwerp.

\bibitem[{Niebuhr \& Ambrazaitis(2006)}]{niebuhr2006alignment}
Niebuhr, Oliver \& Gilbert Ambrazaitis. 2006.
\newblock Alignment of medial and late peaks in {G}erman spontaneous speech.
\newblock In Rüdiger Hoffmann \& Hansjörg Mixdorff (eds.), \emph{Proceedings
  of the 3rd {I}nternational {C}onference on {S}peech {P}rosody}, 161--164.
  Dresden.

\bibitem[{Niebuhr et~al.(2011)Niebuhr, D'Imperio, Gili~Fivela \&
  Cangemi}]{niebuhr2011shapers}
Niebuhr, Oliver, Mariapaola D'Imperio, Barbara Gili~Fivela \& Francesco
  Cangemi. 2011.
\newblock Are there ``shapers'' and ``aligners''? {I}ndividual differences in
  signalling pitch accent category.
\newblock In Wai-Sum Lee \& Eric Zee (eds.), \emph{Proceedings of the 17th
  {I}nternational {C}ongress of {P}honetic {S}ciences}, 120--123. Hong Kong:
  City University of Hong Kong.

\bibitem[{Niebuhr \& Pfitzinger(2010)}]{niebuhr2010pitchaccent}
Niebuhr, Oliver \& Hartmut Pfitzinger. 2010.
\newblock On pitch-accent identification: {T}he role of syllable duration and
  intensity.
\newblock In \emph{Proceedings of the 5th {I}nternational {C}onference on
  {S}peech {P}rosody}, Chicago.

\bibitem[{Nosofsky(1986)}]{nosofsky1986attention}
Nosofsky, Robert. 1986.
\newblock Attention, similarity, and the identification--categorization
  relationship.
\newblock \emph{Journal of Experimental Psychology: General} 115(1). 39--61.

\bibitem[{Nosofsky(1988)}]{nosofsky1988exemplar}
Nosofsky, Robert. 1988.
\newblock Exemplar-based accounts of relations between classification,
  recognition, and typicality.
\newblock \emph{Journal of Experimental Psychology: Learning, Memory, and
  Cognition} 14(4). 700--708.

\bibitem[{Noth et~al.(2000)Noth, Batliner, Kie\ss{}ling, Kompe \&
  Niemann}]{noth2000verbmobil}
Noth, Elmar, Anton Batliner, Andreas Kie\ss{}ling, Ralf Kompe \& Heinrich
  Niemann. 2000.
\newblock Verbmobil: {T}he use of prosody in the linguistic components of a
  speech understanding system.
\newblock \emph{Speech and Audio Processing} 8(5). 519--532.

\bibitem[{{O'Dell} \& Port(1983)}]{odell1983discrimination}
{O'Dell}, Michael \& Robert Port. 1983.
\newblock Discrimination of word-final voicing in {G}erman.
\newblock \emph{Journal of the Acoustical Society of America} 73. S31.

\bibitem[{Ohala(1983)}]{ohala1983cross}
Ohala, John. 1983.
\newblock Cross-language use of pitch: {A}n ethological view.
\newblock \emph{Phonetica} 40(1). 1--18.

\bibitem[{Ohala(1984)}]{ohala1984ethological}
Ohala, John. 1984.
\newblock An ethological perspective on common cross-language utilization of f0
  of voice.
\newblock \emph{Phonetica} 41(1). 1--16.

\bibitem[{Ohala(1990)}]{ohala1990interface}
Ohala, John. 1990.
\newblock There is no interface between phonology and phonetics: {A} personal
  view.
\newblock \emph{Journal of Phonetics} 18(2). 153--172.

\bibitem[{Oldfield(1966)}]{oldfield1966things}
Oldfield, Richard. 1966.
\newblock Things, words and the brain.
\newblock \emph{The Quarterly Journal of Experimental Psychology} 18(4).
  340--353.

%\enlargethispage{\baselineskip}
\bibitem[{Osgood et~al.(1957)Osgood, Suci \&
  Tannenbaum}]{osgood1957measurement}
Osgood, Charles, George Suci \& Percy Tannenbaum. 1957.
\newblock \emph{The measurement of meaning}.
\newblock Urbana: University of Illinois Press.

\bibitem[{Palmeri et~al.(1993)Palmeri, Goldinger \&
  Pisoni}]{palmeri1993episodic}
Palmeri, Thomas, Stephen Goldinger \& David Pisoni. 1993.
\newblock Episodic encoding of voice attributes and recognition memory for
  spoken words.
\newblock \emph{Journal of Experimental Psychology: Learning, Memory, and
  Cognition} 19(2). 309--328.

\bibitem[{Pandelaere \& Dewitte(2006)}]{pandelaere2006question}
Pandelaere, Mario \& Siegfried Dewitte. 2006.
\newblock Is this a question? {N}ot for long: {T}he statement bias.
\newblock \emph{Journal of Experimental Social Psychology} 42(4). 525--531.

\bibitem[{Peterson(1952)}]{peterson1952information}
Peterson, Gordon. 1952.
\newblock The information-bearing elements of speech.
\newblock \emph{Journal of the Acoustical Society of America} 24(6). 629--637.

\bibitem[{Peterson \& Barney(1952)}]{peterson1952control}
Peterson, Gordon \& Harold Barney. 1952.
\newblock Control methods used in a study of the vowels.
\newblock \emph{Journal of the Acoustical Society of America} 24(2). 175--184.

\bibitem[{Petrone(2008)}]{petrone2008role}
Petrone, Caterina. 2008.
\newblock \emph{Le r{\^o}le de la variabilit{\'e} phon{\'e}tique dans la
  repr{\'e}sentation des contours intonatifs et de leur sens}: Aix-en-Provence:
  Universit{\'e} de Provence dissertation.

\bibitem[{Petrone \& D'Imperio(2008)}]{petrone2008tonal}
Petrone, Caterina \& Mariapaola D'Imperio. 2008.
\newblock Tonal structure and constituency in {N}eapolitan {I}talian:
  {E}vidence for the accentual phrase in statements and questions.
\newblock In Plinio Barbosa, Sandra Madureira \& César Reis (eds.),
  \emph{Proceedings of 4th {I}nternational {C}onference on {S}peech {P}rosody},
  301--304. Campinas.

\bibitem[{Petrone \& D'Imperio(2009)}]{petrone2009tonal}
Petrone, Caterina \& Mariapaola D'Imperio. 2009.
\newblock Is tonal alignment interpretation independent of methodology?
\newblock In Maria Uther, Roger Moore \& Stephen Cox (eds.), \emph{Proceedings
  of the 10th {Annual Conference of the International Speech Communication
  Association}}, 2459--2462. Brighton.

\bibitem[{Petrone \& D'Imperio(2011)}]{petrone2011tones}
Petrone, Caterina \& Mariapaola D'Imperio. 2011.
\newblock From tones to tunes: {E}ffects of the f0 prenuclear region in the
  perception of {N}eapolitan statements and questions.
\newblock In Sónia Frota, Gorka Elordieta \& Pilar Prieto (eds.),
  \emph{Prosodic categories: {P}roduction, perception and comprehension},
  207--230. Dordrecht, Heidelberg, London, New York: Springer.

\bibitem[{Petrone \& Niebuhr(2014)}]{petrone2014intonation}
Petrone, Caterina \& Oliver Niebuhr. 2014.
\newblock On the intonation in {G}erman intonation questions: {T}he role of the
  prenuclear region.
\newblock \emph{Language and Speech} 57(1). 108--46.

\bibitem[{Pfitzinger(2001)}]{pfitzinger2001phonetische}
Pfitzinger, Hartmut. 2001.
\newblock \emph{Phonetische {A}nalyse der {S}prechgeschwindigkeit}: Munich:
  Ludwig-Maximilians-Universit{\"a}t M{\"u}nchen dissertation.

\bibitem[{Pierrehumbert(1980)}]{pierrehumbert1980phonology}
Pierrehumbert, Janet. 1980.
\newblock \emph{The phonology and phonetics of {E}nglish intonation}:
  Cambridge, MA: Massachussets Institut of Technology dissertation.

\bibitem[{Pierrehumbert(1981)}]{pierrehumbert1981synthesizing}
Pierrehumbert, Janet. 1981.
\newblock Synthesizing intonation.
\newblock \emph{Journal of the Acoustical Society of America} 70(4). 985--995.

\bibitem[{Pierrehumbert(1990)}]{pierrehumbert1990phonological}
Pierrehumbert, Janet. 1990.
\newblock Phonological and phonetic representation.
\newblock \emph{Journal of Phonetics} 18(3). 375--394.

%\enlargethispage{\baselineskip}
\bibitem[{Pierrehumbert(2001)}]{pierrehumbert2001exemplar}
Pierrehumbert, Janet. 2001.
\newblock Exemplar dynamics: {W}ord frequency, lenition and contrast.
\newblock In Joan Bybee \& Paul Hopper (eds.), \emph{Frequency and the
  emergence of linguistic structure}, 137--158. Amsterdam: John Benjamins.

\bibitem[{Pierrehumbert \& Beckman(1988)}]{pierrehumbert1988japanese}
Pierrehumbert, Janet \& Mary Beckman. 1988.
\newblock \emph{Japanese tone structure}.
\newblock Cambridge: MIT Press.

\bibitem[{Pierrehumbert et~al.(2000)Pierrehumbert, Beckman \&
  Ladd}]{pierrehumbert2000conceptual}
Pierrehumbert, Janet, Mary Beckman \& Robert Ladd. 2000.
\newblock Conceptual foundations of phonology as a laboratory science.
\newblock In Noel Burton-Roberts, Philip Carr \& Gerard Docherty (eds.),
  \emph{Phonological knowledge: {C}onceptual and empirical issues}, 273--303.
  Oxford: Oxford University Press.

\bibitem[{Pierrehumbert \& Hirschberg(1990)}]{pierrehumbert1990meaning}
Pierrehumbert, Janet \& Julia Hirschberg. 1990.
\newblock The meaning of intonational contours in the interpretation of
  discourse.
\newblock In Philip Cohen, Jerry Morgan \& Martha Pollack (eds.),
  \emph{Intentions in communication}, 271--311. Cambridge: MIT Press.

\bibitem[{Pierrehumbert \& Steele(1989)}]{pierrehumbert1989categories}
Pierrehumbert, Janet \& Shirley Steele. 1989.
\newblock Categories of tonal alignment in {E}nglish.
\newblock \emph{Phonetica} 46(3). 181--196.

\bibitem[{Pike(1945)}]{pike1945intonation}
Pike, Kenneth. 1945.
\newblock \emph{The intonation of {A}merican {E}nglish}.
\newblock Ann Arbor: University of Michigan Press.

\bibitem[{Podi(1995)}]{podi1995esquisse}
Podi, Napo. 1995.
\newblock \emph{Esquisse comparative de l'akasilimi et du basaal}: Grenoble:
  Universit\'{e} de Grenoble III dissertation.

\bibitem[{Port(1996)}]{port1996discreteness}
Port, Robert. 1996.
\newblock The discreteness of phonetic elements and formal linguistics:
  {R}esponse to {A}. {M}anaster {R}amer.
\newblock \emph{Journal of Phonetics} 24(4). 491--512.

\bibitem[{Port(2006)}]{port2006graphical}
Port, Robert. 2006.
\newblock The graphical basis of phones and phonemes.
\newblock In Murray Munro \& Ocke-Schwen Bohn (eds.), \emph{Second language
  speech learning: {T}he role of language experience in speech perception and
  production}, 349--365. Amsterdam: John Benjamins.

\bibitem[{Port \& Crawford(1989)}]{port1989incomplete}
Port, Robert \& Penny Crawford. 1989.
\newblock Incomplete neutralization and pragmatics in {G}erman.
\newblock \emph{Journal of Phonetics} 17(4). 257--282.

\bibitem[{Port et~al.(1981)Port, Mitleb \& O'Dell}]{port1981neutralization}
Port, Robert, Fares Mitleb \& Michael O'Dell. 1981.
\newblock Neutralization of obstruent voicing in {G}erman is incomplete.
\newblock \emph{Journal of the Acoustical Society of America} 70. S13.

\bibitem[{Port \& O'Dell(1985)}]{port1985neutralization}
Port, Robert \& Michael O'Dell. 1985.
\newblock Neutralization of syllable-final voicing in {G}erman.
\newblock \emph{Journal of Phonetics} 13(4). 455--471.

\bibitem[{Prieto et~al.(2005)Prieto, D'Imperio \&
  Gili~Fivela}]{prieto2005pitch}
Prieto, Pilar, Mariapaola D'Imperio \& Barbara Gili~Fivela. 2005.
\newblock Pitch accent alignment in {R}omance: {P}rimary and secondary
  associations with metrical structure.
\newblock \emph{Language and Speech} 48(4). 359--396.

\bibitem[{Pye(1986)}]{pye1986word}
Pye, Susan. 1986.
\newblock Word-final devoicing of obstruents in {R}ussian.
\newblock \emph{Cambridge Papers in Phonetics and Experimental Linguistics} 5.
  1--10.

\bibitem[{{R Development Core Team}(2008)}]{r2008r}
{R Development Core Team}. 2008.
\newblock R: {A} language and environment for statistical computing.
\newblock Computer program, retrieved from http://www.R-project.org/.

\bibitem[{Repp(1979)}]{repp1979relative}
Repp, Bruno. 1979.
\newblock Relative amplitude of aspiration noise as a voicing cue for
  syllable-initial stop consonants.
\newblock \emph{Language and Speech} 22(2). 173--189.

\bibitem[{Rialland(1984)}]{rialland1984fini}
Rialland, Annie. 1984.
\newblock Le fini/l'infini ou l'affirmation/l'interrogation en moba (langue
  volta{\"i}que parl{\'e}e au {N}ord-{T}ogo).
\newblock \emph{Studies in African Linguistics} supp. 9. 258--261.

\enlargethispage{2\baselineskip}
\bibitem[{Rialland(2007)}]{rialland2007question}
Rialland, Annie. 2007.
\newblock Question prosody: {A}n {A}frican perspective.
\newblock In Tomas Riad \& Carlos Gussenhoven (eds.), \emph{Tones and tunes:
  {E}xperimental studies in word and sentence prosody}, vol.~2, 35--62. Berlin:
  de Gruyter.

\bibitem[{Rietveld \& Gussenhoven(1987)}]{rietveld1987perceived}
Rietveld, Toni \& Carlos Gussenhoven. 1987.
\newblock Perceived speech rate and intonation.
\newblock \emph{Journal of Phonetics} 15(3). 273--285.

\bibitem[{Rossi(1971)}]{rossi1971seuil}
Rossi, Mario. 1971.
\newblock Le seuil de glissando ou seuil de perception des variations tonales
  pour les sons de la parole.
\newblock \emph{Phonetica} 23(1). 1--33.

\bibitem[{R{\"o}ttger et~al.(2011)R{\"o}ttger, Winter \&
  Grawunder}]{rottger2011robustness}
R{\"o}ttger, Timo, Bodo Winter \& Sven Grawunder. 2011.
\newblock The robustness of incomplete neutralization in {G}erman.
\newblock In Wai-Sum Lee \& Eric Zee (eds.), \emph{Proceedings of the 17th
  {I}nternational {C}ongress of {P}honetic {S}ciences}, 1722--1725. Hong Kong:
  City University of Hong Kong.

\bibitem[{Ryalls et~al.(1994)Ryalls, Le~Dorze, Lever, Ouellet \&
  Larfeuil}]{ryalls1994effects}
Ryalls, John, Guylaine Le~Dorze, Nathalie Lever, Lisa Ouellet \& Céline
  Larfeuil. 1994.
\newblock The effects of age and sex on speech intonation and duration for
  matched statements and questions in {F}rench.
\newblock \emph{Journal of the Acoustical Society of America} 95(4).
  2274--2276.

\bibitem[{Sabatini(1985)}]{sabatini1985italiano}
Sabatini, Francesco. 1985.
\newblock L'italiano dell'uso medio: {U}na realt{\`a} tra le variet{\`a}
  linguistiche italiane.
\newblock In Günter Holtus \& Edgar Radtke (eds.), \emph{Gesprochenes
  {I}talienisch in {G}eschichte und {G}egenwart}, 154--184. T{\"u}bingen:
  Gunter Narr.

\bibitem[{Sadock \& Zwicky(1985)}]{sadock1985speech}
Sadock, Jerrold \& Arnold Zwicky. 1985.
\newblock Speech act distinctions in syntax.
\newblock In Timothy Shopen (ed.), \emph{Language typology and syntactic
  description}, 155--196. Cambridge: Cambridge University Press.

\bibitem[{Savino(1997)}]{savino1997ruolo}
Savino, Michelina. 1997.
\newblock \emph{Il ruolo dell'intonazione nell'interazione comunicativa:
  {A}nalisi strumentale delle domande polari in un corpus di dialoghi spontanei
  (varieta' di {B}ari)}: Bari: Universit{\`a}/{P}olitecnico di {B}ari
  dissertation.

\bibitem[{Savino(2012)}]{savino2012intonation}
Savino, Michelina. 2012.
\newblock The intonation of polar questions in {I}talian: {W}here is the rise?
\newblock \emph{Journal of the International Phonetic Association} 42. 23--48.

\bibitem[{Savy \& Cutugno(2009)}]{savy2009clips}
Savy, Renata \& Francesco Cutugno. 2009.
\newblock {CLIPS: D}iatopic, diamesic and diaphasic variations in spoken
  {I}talian.
\newblock In \emph{Proceedings of the 5th {Corpus Linguistics Conference}},
  Liverpool.

\bibitem[{Schacter \& Church(1992)}]{schacter1992auditory}
Schacter, Daniel \& Barbara Church. 1992.
\newblock Auditory priming: {I}mplicit and explicit memory for words and
  voices.
\newblock \emph{Journal of Experimental Psychology: Learning, Memory, and
  Cognition} 18(5). 915--930.

\bibitem[{Scherer et~al.(1984)Scherer, Ladd \& Silverman}]{scherer1984vocal}
Scherer, Klaus, Robert Ladd \& Kim Silverman. 1984.
\newblock Vocal cues to speaker affect: {T}esting two models.
\newblock \emph{Journal of the Acoustical Society of America} 76(5).
  1346--1356.

\bibitem[{Schouten(1985)}]{schouten1985identification}
Schouten, Marten Egbertus~Hendrik. 1985.
\newblock Identification and discrimination of sweep tones.
\newblock \emph{Attention, Perception, \& Psychophysics} 37(4). 369--376.

\enlargethispage{\baselineskip}
\bibitem[{Schweitzer et~al.(2010{\natexlab{a}})Schweitzer, Calhoun,
  Sch{\"u}tze, Schweitzer \& Walsh}]{schweitzer2010relative}
Schweitzer, Katrin, Sasha Calhoun, Hinrich Sch{\"u}tze, Antje Schweitzer \&
  Michael Walsh. 2010{\natexlab{a}}.
\newblock Relative frequency affects pitch accent realisation: {E}vidence for
  exemplar storage of prosody.
\newblock In Marija Tabain, Janet Fletcher, David Grayden, John Hajek \& Andy
  Butcher (eds.), \emph{Proceedings of the 13th {Australasian International
  Conference on Speech Science and Technology}}, 62--65. Melbourne.

\bibitem[{Schweitzer et~al.(2011)Schweitzer, Walsh, Calhoun \&
  Sch{\"u}tze}]{schweitzer2011prosodic}
Schweitzer, Katrin, Michael Walsh, Sasha Calhoun \& Hinrich Sch{\"u}tze. 2011.
\newblock Prosodic variability in lexical sequences: {I}ntonation entrenches
  too.
\newblock In Wai-Sum Lee \& Eric Zee (eds.), \emph{Proceedings of the 17th
  {I}nternational {C}ongress of {P}honetic {S}ciences}, 1778--1781. Hong Kong:
  City University of Hong Kong.

\bibitem[{Schweitzer et~al.(2009)Schweitzer, Walsh, M{\"o}bius, Riester,
  Schweitzer \& Sch{\"u}tze}]{schweitzer2009frequency}
Schweitzer, Katrin, Michael Walsh, Bernd M{\"o}bius, Arndt Riester, Antje
  Schweitzer \& Hinrich Sch{\"u}tze. 2009.
\newblock Frequency matters: {P}itch accents and information status.
\newblock In Diana McCarthy \& Shuly Wintner (eds.), \emph{Proceedings of the
  12th {Conference of the European Chapter of the Association for Computational
  Linguistics}}, 728--736. Athens.

\bibitem[{Schweitzer et~al.(2010{\natexlab{b}})Schweitzer, Walsh, M{\"o}bius \&
  Sch{\"u}tze}]{schweitzer2010frequency}
Schweitzer, Katrin, Michael Walsh, Bernd M{\"o}bius \& Hinrich Sch{\"u}tze.
  2010{\natexlab{b}}.
\newblock Frequency of occurrence effects on pitch accent realisation.
\newblock In Takao Kobayashi, Keikichi Hirose \& Satoshi Nakamura (eds.),
  \emph{Proceedings of the 11th {Annual Conference of the International Speech
  Communication Association}}, 138--141. Makuhari.

\bibitem[{Sekiguchi(2006)}]{sekiguchi2006effects}
Sekiguchi, Takahiro. 2006.
\newblock Effects of lexical prosody and word familiarity on lexical access of
  spoken {J}apanese words.
\newblock \emph{Journal of Psycholinguistic Research} 35(4). 369--384.

\bibitem[{Sergeant \& Harris(1962)}]{sergeant1962sensitivity}
Sergeant, Russell \& Donald Harris. 1962.
\newblock Sensitivity to unidirectional frequency modulation.
\newblock \emph{Journal of the Acoustical Society of America} 34(10).
  1625--1628.

\bibitem[{Shattuck-Hufnagel \& Turk(1996)}]{shattuck1996prosody}
Shattuck-Hufnagel, Stefanie \& Alice Turk. 1996.
\newblock A prosody tutorial for investigators of auditory sentence processing.
\newblock \emph{Journal of Psycholinguistic Research} 25(2). 193--247.

\bibitem[{Sheffert et~al.(2002)Sheffert, Pisoni, Fellowes \&
  Remez}]{sheffert2002learning}
Sheffert, Sonya, David Pisoni, Jennifer Fellowes \& Robert Remez. 2002.
\newblock Learning to recognize talkers from natural, sinewave, and reversed
  speech samples.
\newblock \emph{Journal of Experimental Psychology: Human Perception and
  Performance} 28(6). 1447--1469.

\bibitem[{Shriberg et~al.(1998)Shriberg, Stolcke, Jurafsky, Coccaro, Meteer,
  Bates, Taylor, Ries, Martin \& Van Ess-Dykema}]{shriberg1998can}
Shriberg, Elizabeth, Andreas Stolcke, Daniel Jurafsky, Noah Coccaro, Marie
  Meteer, Rebecca Bates, Paul Taylor, Klaus Ries, Rachel Martin \& Carol Van
  Ess-Dykema. 1998.
\newblock Can prosody aid the automatic classification of dialog acts in
  conversational speech?
\newblock \emph{Language and Speech} 41(3-4). 443--492.

%\enlargethispage{\baselineskip}
\bibitem[{Slowiaczek \& Dinnsen(1985)}]{slowiaczek1985neutralizing}
Slowiaczek, Louisa \& Daniel Dinnsen. 1985.
\newblock On the neutralizing status of {P}olish word-final devoicing.
\newblock \emph{Journal of Phonetics} 13(3). 325--341.

\bibitem[{Slowiaczek \& Szymanska(1989)}]{slowiaczek1989perception}
Slowiaczek, Louisa \& Helena Szymanska. 1989.
\newblock Perception of word-final devoicing in {P}olish.
\newblock \emph{Journal of Phonetics} 17(3). 205--212.

\bibitem[{Smith(2002)}]{smith2002prosodic}
Smith, Caroline. 2002.
\newblock Prosodic finality and sentence type in {F}rench.
\newblock \emph{Language and Speech} 45(2). 141--178.

\bibitem[{Smith \& Medin(1981)}]{smith1981categories}
Smith, Edward \& Douglas Medin. 1981.
\newblock \emph{Categories and concepts}.
\newblock Cambridge: Harvard University Press.

\bibitem[{Smith et~al.(2012)Smith, Baker \& Hawkins}]{smith2012phonetic}
Smith, Rachel, Rachel Baker \& Sarah Hawkins. 2012.
\newblock Phonetic detail that distinguishes prefixed from pseudo-prefixed
  words.
\newblock \emph{Journal of Phonetics} 40(5). 689--705.

\bibitem[{Smith \& Hawkins(2000)}]{smith2000allophonic}
Smith, Rachel \& Sarah Hawkins. 2000.
\newblock Allophonic influences on word-spotting experiments.
\newblock In Anne Cutler, James McQueen \& Rian Zondervan (eds.), \emph{{ISCA
  Tutorial and Research Workshop on Spoken Word Access Processes}}, 139--142.
  Nijmegen: Max-Planck-Gesellschaft zur Förderung der Wissenschaften.

\bibitem[{Sobrero(1992)}]{sobrero1992italiano}
Sobrero, Alberto. 1992.
\newblock \emph{L'italiano di oggi}.
\newblock Roma: Istituto della Enciclopedia Italiana.

\bibitem[{Standing et~al.(1970)Standing, Conezio \&
  Haber}]{standing1970perception}
Standing, Lionel, Jerry Conezio \& Ralph Haber. 1970.
\newblock Perception and memory for pictures: {S}ingle-trial learning of 2500
  visual stimuli.
\newblock \emph{Psychonomic Science} 19(2). 73--74.

\bibitem[{Stevens(1960)}]{stevens1960model}
Stevens, Kenneth. 1960.
\newblock Toward a model for speech recognition.
\newblock \emph{Journal of the Acoustical Society of America} 32(1). 47--55.

\bibitem[{Stevens(2004)}]{stevens2004invariance}
Stevens, Kenneth. 2004.
\newblock Invariance and variability in speech: {I}nterpreting acoustic
  evidence.
\newblock In Janet Slifka, Sharon Manuel \& Melanie Matthies (eds.),
  \emph{Proceedings of {From Sound to Sense Workshop}}, vol.~B, 77--85.
  Cambridge: MIT Press.

\bibitem[{Swerts et~al.(1999)Swerts, Avesani \&
  Krahmer}]{swerts1999reaccentuation}
Swerts, Marc, Cinzia Avesani \& Emiel Krahmer. 1999.
\newblock Reaccentuation or deaccentuation: {A} comparative study of {I}talian
  and {D}utch.
\newblock In John Ohala (ed.), \emph{Proceedings of the 14th {I}nternational
  {C}ongress of {P}honetic {S}ciences}, 1541--144. San Francisco: University of
  California.

\bibitem[{{'t Hart}(1976)}]{thart1976psychoacoustic}
{'t Hart}, Johan. 1976.
\newblock Psychoacoustic backgrounds of pitch contour stylisation.
\newblock \emph{IPO -- Annual Progress Report} 11. 11--19.

\bibitem[{{'t Hart} et~al.(1990){'t Hart}, Collier \&
  Cohen}]{tHart1990perceptual}
{'t Hart}, Johan, Rene Collier \& Antonie Cohen. 1990.
\newblock \emph{A perceptual study of intonation: {A}n experimental-phonetic
  approach}.
\newblock Cambridge: Cambridge University Press.

\bibitem[{Taljaard \& Bosch(1988)}]{taljaard1988handbook}
Taljaard, Petrus \& Sonja Bosch. 1988.
\newblock \emph{Handbook of {I}sizulu}.
\newblock Hatfield, Pretoria: J.L. van Schaik.

\bibitem[{Tenpenny(1995)}]{tenpenny1995abstractionist}
Tenpenny, Patricia. 1995.
\newblock Abstractionist versus episodic theories of repetition priming and
  word identification.
\newblock \emph{Psychonomic Bulletin \& Review} 2(3). 339--363.

\bibitem[{Theodore(2009)}]{theodore2009characteristics}
Theodore, Rachel. 2009.
\newblock \emph{Some characteristics of talker-specific phonetic detail}:
  Boston: Northeastern University dissertation.

\bibitem[{Trager \& Smith(1951)}]{trager1951outline}
Trager, Leonard \& Henry Smith. 1951.
\newblock \emph{An outline of {E}nglish structure}.
\newblock Norman: Battenburg Press.

\bibitem[{Trouvain(2004)}]{trouvain2004tempo}
Trouvain, Jürgen. 2004.
\newblock \emph{Tempo variation in speech production: {I}mplications for speech
  synthesis}: Saarbr{\"u}cken: Saarland University dissertation.

\bibitem[{Trubetzkoy(1939)}]{trubeckoj1939grundzuege}
Trubetzkoy, Nikolaus. 1939.
\newblock \emph{Grundz\"{u}ge der {P}honologie}.
\newblock Prag: Travaux du cercle linguistique de Prague.

\bibitem[{Turk et~al.(2006)Turk, Nakai \& Sugahara}]{turk2006acoustic}
Turk, Alice, Satsuki Nakai \& Mariko Sugahara. 2006.
\newblock Acoustic segment durations in prosodic research: {A} practical guide.
\newblock In Stefan Sudhoff, Denisa Lenertov\'{a}, Roland Meyer, Sandra
  Pappert, Petra Augurzky, Ina Mleinek, Nicole Richter \& Johannes
  Schlie\ss{}er (eds.), \emph{Methods in empirical prosody research}, 1--28.
  Berlin: De Gruyter.

\bibitem[{Uldall(1964)}]{uldall1964dimensions}
Uldall, Elizabeth. 1964.
\newblock Dimensions of meaning in intonation.
\newblock In David Abercrombie (ed.), \emph{In honour of {D}aniel {J}ones},
  271--279. London: Longmans.

\bibitem[{{van Alphen} \& McQueen(2006)}]{vanalphen2006effect}
{van Alphen}, Petra \& James McQueen. 2006.
\newblock The effect of voice onset time differences on lexical access in
  {D}utch.
\newblock \emph{Journal of Experimental Psychology: Human Perception and
  Performance} 32(1). 178--196.

\bibitem[{{van Heerden} \& Barnard(2007)}]{vanheerden2007speech}
{van Heerden}, Charl~Johannes \& Etienne Barnard. 2007.
\newblock Speech rate normalization used to improve speaker verification.
\newblock \emph{South African Institute of Electrical Engineers} 98(4).
  136--140.

\bibitem[{{van Heuven} \& Haan(2000)}]{vanheuven2000phonetic}
{van Heuven}, Vincent \& Judith Haan. 2000.
\newblock Phonetic correlates of statement versus question intonation in
  {D}utch.
\newblock In Antonis Botinis (ed.), \emph{Intonation: {A}nalysis, modelling and
  technology}, 119--144. Dordrecht: Kluwer.

\bibitem[{{van Heuven} \& Haan(2002)}]{vanheuven2002temporal}
{van Heuven}, Vincent \& Judith Haan. 2002.
\newblock Temporal distribution of interrogativity markers in {D}utch: {A}
  perceptual study.
\newblock In Carlos Gussenhoven \& Natasha Warner (eds.), \emph{Papers in
  {L}aboratory {P}honology}, vol.~7, 61--86. Berlin: Mouton de Gruyter.

\bibitem[{{van Heuven} \& {van Zanten}(2005)}]{vanheuven2005speech}
{van Heuven}, Vincent \& Ellen {van Zanten}. 2005.
\newblock Speech rate as a secondary prosodic characteristic of polarity
  questions in three languages.
\newblock \emph{Speech Communication} 47(1). 87--99.

\bibitem[{{van Santen} \& M{\"o}bius(2000)}]{vansanten2000quantitative}
{van Santen}, Jan \& Bernd M{\"o}bius. 2000.
\newblock A quantitative model of f0 generation and alignment.
\newblock In Antonis Botinis (ed.), \emph{Intonation: {A}nalysis, modelling and
  technology}, 269--288. Dordrecht: Kluwer.

\bibitem[{Walsh et~al.(2008)Walsh, Schweitzer, M{\"o}bius \&
  Sch{\"u}tze}]{walsh2008examining}
Walsh, Michael, Katrin Schweitzer, Bernd M{\"o}bius \& Hinrich Sch{\"u}tze.
  2008.
\newblock Examining pitch-accent variability from an exemplar-theoretic
  perspective.
\newblock In Janet Fletcher, Deborah Loakes, Roland Gocke, Denis Burnham \&
  Michael Wagner (eds.), \emph{Proceedings of the 9th {Annual Conference of the
  International Speech Communication Association}}, 877--880. Brisbane.

\newpage
\bibitem[{Warner et~al.(2006)Warner, Good, Jongman \&
  Sereno}]{warner2006orthographic}
Warner, Natasha, Erin Good, Allard Jongman \& Joan Sereno. 2006.
\newblock Orthographic vs. morphological incomplete neutralization effects.
\newblock \emph{Journal of Phonetics} 34(2). 285--293.

\bibitem[{Warner et~al.(2004)Warner, Jongman, Sereno \&
  Kemps}]{warner2004incomplete}
Warner, Natasha, Allard Jongman, Joan Sereno \& Rachèl Kemps. 2004.
\newblock Incomplete neutralization and other sub-phonemic durational
  differences in production and perception: {E}vidence from {D}utch.
\newblock \emph{Journal of Phonetics} 32(2). 251--276.

\bibitem[{Wells(1945)}]{wells1945pitch}
Wells, Rulon. 1945.
\newblock The pitch phonemes of {E}nglish.
\newblock \emph{Language} 21. 27--39.

\bibitem[{West(1999)}]{west1999perception}
West, Paula. 1999.
\newblock Perception of distributed coarticulatory properties of english /l/
  and /r/.
\newblock \emph{Journal of Phonetics} 27(4). 405--426.

\bibitem[{Wheeler(1970)}]{wheeler1970processes}
Wheeler, Daniel. 1970.
\newblock Processes in word recognition.
\newblock \emph{Cognitive Psychology} 1(1). 59--85.

\bibitem[{Williams \& Stevens(1972)}]{williams1972emotions}
Williams, Carl \& Kenneth Stevens. 1972.
\newblock Emotions and speech: {S}ome acoustical correlates.
\newblock \emph{Journal of the Acoustical Society of America} 52(4).
  1238--1250.

\bibitem[{Winter \& R{\"o}ttger(forthcoming)}]{winterFORTHnature}
Winter, Bodo \& Timo R{\"o}ttger. forthcoming.
\newblock The nature of incomplete neutralization in {G}erman: {I}mplications
  for laboratory phonology.
\newblock \emph{Grazer Linguistische Studien} .

\bibitem[{Wood(1973)}]{wood1973speech}
Wood, Sidney. 1973.
\newblock Speech tempo.
\newblock In \emph{Working papers}, vol.~9, 99--147. Lund: Department of
  Linguistics, Lund University.

\bibitem[{Xu(2005)}]{xu2005speech}
Xu, Yi. 2005.
\newblock Speech melody as articulatorily implemented communicative functions.
\newblock \emph{Speech Communication} 46(3). 220--251.

\bibitem[{Zeng et~al.(2004)Zeng, Martin \& Boulakia}]{zeng2004tones}
Zeng, XiaoLi, Philippe Martin \& Georges Boulakia. 2004.
\newblock Tones and intonation in declarative and interrogative sentences in
  {M}andarin.
\newblock In \emph{Proceedings of the {International Symposium on Tonal Aspects
  of Languages}}, 235--238. Beijing.

\end{thebibliography}



\clearpage

\phantomsection%this allows hyperlink in ToC to work
\addcontentsline{toc}{chapter}{Name index}
\ohead{Name index}
% \pdfbookmark[0]{Index}{Index}

%\pdfbookmark[1]{Name index}{Name index}
\printindex[aut]

% 
% \phantomsection%this allows hyperlink in ToC to work
% \addcontentsline{toc}{chapter}{Language index}
% \ohead{Language index}
%\pdfbookmark[1]{Language index}{Language index}
% \printindex[lan]


\phantomsection%this allows hyperlink in ToC to work
\addcontentsline{toc}{chapter}{Subject index}
\ohead{Subject index}
%\pdfbookmark[1]{Subject index}{Subject index}
\printindex
\end{document}
