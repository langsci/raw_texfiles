\chapter{Introduction} \label{Introduction}



%Gemination: what is it. Morpho-phonolo pehneomeonen
In English, \isi{affixation} may lead to the adjacency of two identical consonants across a morpheme boundary. When in a derivative the final consonant of the prefix and the first consonant of the base are the same, a phonological double consonant emerges (see examples \ref{example gemination un}--\ref{example gemination dis}). The same happens when the first consonant of the suffix and the last consonant of the base are the same (see example \ref{example gemination ly}). I will call these phonological double consonants \is{morphological gemination} \newterm{morphological geminates}.


\begin{exe} 
	\ex \label{example gemination un}  \makebox[1cm][l]{\prefix{un}:}  \prefix{un}\textit{natural}, \prefix{un}\textit{known}
	\ex \label{example gemination in}  \makebox[1cm][l]{\prefix{in}:}  \prefix{in}\textit{numerous}, \prefix{im}\textit{mortal}
	\ex \label{example gemination dis} \makebox[1cm][l]{\prefix{dis}:} \prefix{dis}\textit{satisfy}, \prefix{dis}\textit{solution}
	\ex \label{example gemination ly}  \makebox[1cm][l]{\suffix{ly}:}  \textit{real}\suffix{ly}, \textit{sole}\suffix{ly}
\end{exe}

 There are two possibilities for the \isi{phonetic realization} of \is{morphological gemination} morphological geminates: Either the phonological double is realized with a longer duration than a phonological singleton (\isi{gemination}), or it is of the same duration as a singleton consonant (\isi{degemination}). It is, however, yet unclear in which cases we find \isi{gemination}, and in which we find \isi{degemination}.
 
%Not much investigated but just claims are made
There are numerous claims about the pattern of \isi{gemination} in English \isi{affixation} in the literature (see, for example, \citealt[141]{Wijk.1966}; \citealt[255]{OConnor.1973}; \citealt[18]{Mohanan.1986}; \citealt[251]{Ladefoged.1993}; \citealt{Roach.2011,Wells.2008}; \citealt[1055f.]{CohenGoldberg.2013}), but there is hardly any evidence for these claims. %Large-scale empirical research on \isi{gemination} in English has been lacking. 
Only four studies have empirically investigated \isi{gemination} in English affixed words: \cite{Kaye.2005,Oh.2012,Oh.2013} and \cite{Kotzor.2016}. Due to methodological issues and the small scale of the studies, their empirical findings are not sufficient to explain the \isi{gemination} pattern of English affixational geminates.



%Beause it is a morpho-phonolo phenomoen spanning morph bundaries very well suited to investigate the interface
As \isi{gemination} in English \isi{affixation} can be regarded as a morpho-phonological process which is mirrored on the phonetic level, explaining its pattern is of high theoretical importance for morpho-phonological approaches which discuss the role of phonetics in phonology and morphology. Finding out which factors govern \isi{gemination} in English \isi{affixation} can reveal important insights about \is{morpho-phonological interface} the interplay between morphology, phonology and phonetics\is{morpho-phonetic interface}.

% The theoires
One can distinguish between two major branches of morpho-phonological approaches. The first one can be categorized as rule based and  categorical in nature, while the second one is founded on the assumption that processes are gradient and dependent on the properties of individual words. 
Both types of approaches assume \is{boundary strength} morphological boundary strength to affect the \isi{phonetic realization} of complex words. 
It is generally assumed that weaker boundaries lead to more phonetic \isi{reduction}, while stronger boundaries lead to less \isi{reduction}. 
The two types of approaches deviate, however, in how they conceptualize these boundaries. In turn, they differ in their predictions about how morphological boundaries affect the \isi{phonetic realization} of complex words, including the \isi{phonetic realization} of morphological \is{morphological gemination}geminates.
 

Categorical approaches like \isi{Lexical Phonology} (cf., for example, \citealt{Kiparsky.1982,Mohanan.1986}) assume \isi{boundary strength} to depend on affixes. Affixes belong to different \is{lexical stratum}{lexical strata} which determine the phonological relation between an affix and its base. This relation is reflected on the phonetic level. 
For the phenomenon of {gemination} it is predicted that \is{level 1 affix}level 1 affixes, such as \is{in-}\prefix{in}, are separated from their base by a weak morphological boundary and hence \is{degemination}{degeminate}. \is{level 2 affix}Level 2 affixes, such as\is{un-} \prefix{un}, in contrast, \is{gemination}{geminate} due to the strong morphological boundary which they feature. 

Gradient probabilistic approaches, on the other hand, would expect factors which are related to individual derivatives to govern \isi{gemination}. The \isi{Morphological Segmentability Hypothesis} (\citealt{Hay.2003}), for example, claims that the \isi{decomposability} of a word determines the \isi{boundary strength} between the affix and its base. This strength is assumed to be mirrored in \isi{phonetic detail}, such as the duration and \isi{reduction} of boundary adjacent segments. Applied to \isi{gemination}, one would thus expect that more \is{decomposability}{decomposable} words display longer consonant durations (\isi{gemination}), while less \is{decomposability}{decomposable} words display shorter durations (\isi{degemination}). 


In this book, I will test the predictions for \isi{morphological gemination} made by various approaches to the morpho-phonological and the \isi{morpho-phonetic interface}. 
On the one hand, I will test the predictions made by formal linguistic theories, which are mostly categorical in nature. On the other, I will test predictions which are derived from psycholinguistic approaches, which are mostly gradient in nature. Furthermore, I will test some general assumptions about the realization of complex words, as proposed by different \is{speech production model}{models of speech production}.

I will investigate \isi{morphological gemination} with the five English affixes \prefix{un}, negative \prefix{in}, locative \prefix{in}, \prefix{dis} and adverbial \suffix{ly}. The \isi{gemination} pattern of each affix will be investigated in a corpus and an experimental study. 
By finding out which approach can account best for the \isi{gemination} pattern of English affixed words, important implications about the interplay between morphology, \is{morpho-phonetic interface}\is{morpho-phonological interface}{phonology and phonetics} can be drawn.


%Structure

The book is structured as follows.
 In \chapref{Gemination}, I will give an overview of the phenomenon \newterm{gemination}. I will introduce key terminology, discuss the \isi{phonological representation} of {geminates} and summarize previous work on \isi{gemination}. I will mainly focus on \isi{morphological gemination} in English. 
 In \chapref{affixes}, I will turn to the five affixes investigated in this book. I will describe the characteristics of each affix and compare them in a qualitative analysis. 
  In  \chapref{Theory}, I will discuss the three investigated fields of morpho-phonological and morpho-phonetic approaches: Formal linguistic theories, psycholinguistic approaches to \isi{morphological processing} and theories of speech production. I will summarize the main aspects of each field, discuss the most important theories in the field, and deduce the predictions each theory makes for \isi{gemination} with the five affixes under investigation. 
  These predictions will then be tested in a corpus study and an experimental study. The studies will be discussed in Chapters~\ref{General Method}--\ref{Conclusion}.
   While in \chapref{General Method} the general methodology underlying both studies will be described, 
  \chapref{Corpus Studies} will focus on the methodology, analyses and results of the corpus study, and 
  \chapref{Experimental Studies} will focus on the methodology, analyses and results of the experimental study.
   In \chapref{Conclusion}, the results of both studies will be summarized and discussed with regard to the approaches discussed in \chapref{Theory}.
  In \chapref{final conclusion} a final conclusion will be given.\footnote{Earlier versions of parts of Chapters~\ref{Gemination},~\ref{General Method} and~\ref{Corpus Studies} have been previously published in \cite{BenHedia.2017}. They were only minimally altered for the present book. The pertinent chapters and sections will be identified by a footnote.}
