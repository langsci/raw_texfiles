%!TEX root = 3-P_Masterdokument.tex
%!TEX encoding = UTF-8 Unicode

\chapter{The architecture of Paunaka words}\label{chap:Architecture}

This chapter starts with a description of what I mean when talking about a Paunaka “word” (\sectref{sec:Word}). It goes on to a discussion of the structure of words in \sectref{sec:AffixesAndClitics} and grammatical markers that can attach to words belonging to various word classes in \sectref{sec:LowSelectivityMarkers}. The general word formation processes found in the language are summarised in \sectref{sec:Classifiers} to \sectref{sec:CompoundingIncorporation}. At the end of this chapter, \sectref{sec:POS} provides a short characterisation of the different parts of speech that can be identified in Paunaka.

\section{A word about “word”}\label{sec:Word}

In this grammar, I make use of the notion of “word”, since I believe a grammatical description without this concept would unnecessarily complicate the matter. The words I identified as such are separated from each other by spaces, just as in writing with Greek-derived alphabets \citep[33]{Haspelmath2011}. However, “word” is not an unproblematic concept. \citet[]{DixAikh2003} remind us that we have to be careful to distinguish phonological from grammatical words, and \citet[]{Haspelmath2011} even showed that there are no criteria that can be implied in a cross-linguistically valid definition of “word”. How do I come to my decisions as to how strings of sounds are instantiations of different words?

Actually, I did not “examine wordhood using test batteries” \citep[60]{Haspelmath2011}, i.e. I did not apply different kinds of available criteria that have been proposed in the definition of word and considered the results. My decisions were rather based on intuition. Intuition does not come out of the blue, of course. In doing fieldwork with a language previously unknown to me, the first attempts to transcribe the utterances produced by speakers and sort them into different words were heavily influenced by prosodic criteria, such as stress assignment and pauses. This was rather perceived unconsciously, because from the beginning, I had been busier trying to decipher the meaning of the utterances than phonological structure, due to my personal preference for certain kinds of linguistic topics over others. Nonetheless, in the very first step, I considered purely phonological words. With growing knowledge on how different parts of the identified words (i.e. the morphemes) realise different meanings and how these morphemes are ordered with respect to each other, these criteria became ever more important in the analysis. I am, of course, biased in my decisions by decisions made by others for all the languages I have acquired, tried to learn or studied in my life, as well as the scholarly literature on languages both related and not related to Paunaka. This may sound trivial, but I think it is necessary to be clear about it. Thus what constitutes a word in this grammar (as indicated by being put between spaces) is determined by a mixture of grammatical and phonological criteria as well as tradition in Arawakan or Amazonian linguistics. Other researchers with different backgrounds and/or applying other criteria may arrive at different results. Thus a “word” in this work is a convenient solution that is apt, I believe, to describe grammatical properties of Paunaka. The question\is{clitic|(} whether the grammatical and phonological word coincides largely depends on how the status of certain markers is analysed, which are phonologically bound to words belonging to different word classes.\is{word class} Does this mean they are clitics? And if so, is it a logical consequence that they are grammatical words of their own? I pursue this issue further in the following section.


\section{Roots, stems, affixes and clitics}\label{sec:AffixesAndClitics}\is{affix|(}

A word can be composed of different layers in Paunaka. This especially applies to verbs,\is{verb} where we can often distinguish roots,\is{verbal root} stems,\is{verbal stem} derivational affixes,\is{derivation|(} and inflectional markers.\is{inflection} I hardly make use of the concept of “clitic” in this work; the reasons will become apparent throughout this section.\is{clitic|)} 

Root and affix have been defined as two kinds of morphemes, one of which encodes lexical, the other grammatical meaning. Although this distinction is not always unproblematic, especially as to what is the semantic basis of lexical vs. grammatical meaning \citep[cf.][]{Croft2000}, it is a widely accepted one. Roots can encode very basic lexical meaning, which is enhanced by derivation. A root together with all derivational processes is a stem, which may still be different from a full word form \citep[cf.][523]{Mel’cuk2000}. In particular, verb stems\is{verbal stem} can be fairly complex.\footnote{I basically follow the analysis of \citet[217]{Danielsen2007}, who identifies several layers in the composition of a verb in \isi{Baure}. Nonetheless, I do not make use of her notion of “verb base”, since “base” is usually understood as the basic form to which derivational processes apply, while it is general inflectional morphology that follows the \isi{Baure} “verb base”. I acknowledge, though, that the last suffix\is{suffix|(} (especially the so-called thematic suffixes\is{thematic suffix}) of a verb stem\is{verbal stem} works on a different layer than the derivational suffixes preceding it. See \figref{fig:VerbTemplate-1} in \sectref{sec:LowSelectivityMarkers} for the schema of an active verb’s composition.\is{suffix|)}}

Derivational processes can be marked by derivational affixes, but there are also other means, e.g. \isi{reduplication} (see \sectref{sec:Reduplication}). In addition, \isi{compounding} and \isi{incorporation} as non-derivational processes can alter the meaning of a word, while attachment of  classifiers\is{classifier} lies on the edge between compounding and derivation. Some derivational processes are clear-cut. This is the case whenever we find a derivational affix together with various roots, where it exhibits the same meaning or has the same function. There are other cases of totally opaque derivational processes, i.e. we can identify a root and a stem, but not the affixes or processes that lead from the root to the stem. Most derivational affixes follow the root, i.e. Paunaka is mainly suffixing\is{suffix} in this regard.

Consider (\getref{ex:VerbStems-1}), which shows the composition of four related verb stems.\is{verbal stem|(} The root\is{verbal root} is \textit{-pa} ‘die’ in all cases. This root is minimally accompanied by the thematic suffix\is{thematic suffix|(} \textit{-ku} (irrealis \textit{-ka}). The meaning of this thematic suffix is very opaque, but it only appears on active verbs\is{active verb|(} and it has cognates in the most closely related languages. From the verb stem \textit{-paku} ‘die’, the verb \textit{-kupaku} ‘kill’ can be derived by a \isi{causative} prefix \textit{ku-}. The stem \textit{-kupaku} is the base for insertion of the \isi{extension applicative} suffix \textit{-i}, which yields \textit{-kupaiku} ‘slaughter’. This again is the base for another derivation whose result is \textit{-kupaikechu} with the pluractional meaning ‘kill all’ or ‘kill everybody’. However in this case, it is not clear which affixes are involved. It is possible that the last sequence \textit{-chu} of \textit{-kechu} is the other thematic suffix \textit{-chu} ‘\textsc{th}2’, and it is also possible that the /k/ is a remnant from the more frequent thematic suffix \textit{-ku}. Nonetheless, the function of \textit{ke} (or \textit{e}) remains unclear. There is only one other verb stem with a similar derivation, the related \textit{-paikechu} ‘all die’, and it is not possible to derive new verbs with \textit{-kechu}.

\ea\label{ex:VerbStems-1}
\begingl
\glpreamble -paku, -kupaku, -kupaiku, -kupaikechu\\
\gla pa-ku ku-pa-ku ku-pa-i-ku ku-pa-i-ke-chu\\
\glb die-\textsc{th}1 \textsc{caus}-die-\textsc{th}1 \textsc{caus}-die-\textsc{ext}-\textsc{th}1 \textsc{caus}-die-\textsc{ext}-?-\textsc{th}2?\\
\glft ‘verb stems: die, kill, slaughter, kill all/everybody’
\endgl
\xe

Another example is (\getref{ex:VerbStems-2}). In this case, \textit{-yunu} ‘go’ has no thematic suffix,\footnote{This verb has cognates in other Arawakan languages.} and it seems we are dealing with a root here.\is{verbal root} If we consider, however, that there is also \textit{-yuiku} ‘walk’, which looks as if it is related to \textit{-yunu}, we might conclude that the root must be only \textit{-yu}. But unlike a thematic suffix like the one we have found in (\getref{ex:VerbStems-1}) above, the syllable \textit{nu} does not detach from \textit{-yunu} in any case of inflection, and we cannot assign any specific function to this syllable. There are two more possible derivations though that seem to build on \textit{-yu} only, \textit{-yupu} and \textit{-yupunu}, both meaning ‘go out, come out, sprout’.

\ea\label{ex:VerbStems-2}
\begingl
\glpreamble -yunu, -yuiku, -yupu, -yupunu\\
\gla yunu yu-i-ku yu-pu yu-punu\\
\glb go go?-\textsc{ext}-\textsc{th}1 go?-\textsc{dloc} go?-\textsc{am.prior}\\
\glft ‘verb stems: go, walk, come out, come out’
\endgl
\xe
\is{thematic suffix|)}

The exact derivational relation between root\is{verbal root} and stem is thus not always clear, especially if there do not happen to be several related verb stems.\footnote{Note that \citet[]{BaptistaWallin1965} in their tagmemic grammar of \isi{Baure} identified five recurring final syllables found on active verbs’ roots, which belong to their ‘fv-fund’ class, i.e. the ‘formative-fundament’ class of verbs (see \citealt[42, 49-50]{BaptistaWallin1965} for examples). One of these syllables happens to be \textit{no}, which is the \isi{Baure} equivalent of Paunaka \textit{nu}. It may thus be possible to analyse the structure of a verb even more deeply, but I did not undertake this task.}\is{active verb|)} Furthermore, since most verb stems are totally lexicalised,\is{lexicalisation} it is not always possible to determine which part of the meaning is realised by which part exactly. As a citation form in text and interlinear glosses, I have therefore chosen the stem and not the root.\is{verbal stem|)} There are some exceptions to this, mainly if a \isi{classifier} or incorporated noun\is{incorporation} is part of the verb stem, which is not generally lexicalised on it. This happens relatively rarely, and thus I believe it is worth highlighting these cases. More information on the composition of verb stems is given in Chapter \ref{sec:Verbs}.\is{derivation|)}

The verb stems of (\getref{ex:VerbStems-1}) and (\getref{ex:VerbStems-2}) are not complete words, nor are they citation forms used by the speakers themselves, e.g. in offering a translation for a Spanish word. Verbs\is{verb} need to minimally inflect\is{inflection|(} for person\is{person marking} and \isi{reality status} (RS).\footnote{Actually according to my own analysis, RS marking of active verbs fuses with the last suffix or vowel of the stem\is{verbal stem} or of the marker directly following the stem, so that the forms in (\getref{ex:VerbStems-1}) and (\getref{ex:VerbStems-2}) can be analysed to already contain RS inflection, which is realis in this case by absence of irrealis marking. The irrealis forms of (\getref{ex:VerbStems-1}) are \textit{-paka}, \textit{-kupaka}, \textit{-kupaika}, \textit{-kupaikecha} and of  (\getref{ex:VerbStems-2}) \textit{-yuna}, \textit{-yuika}, \textit{-yupa}, and \textit{-yupuna}. This will be explained in more detail below, and especially in \sectref{sec:RealityStatus}.} Cross-linguistically, inflection is often carried out by inflectional affixes. However, the case is not totally clear in Paunaka, which may be claimed to use mainly clitics\is{clitic} to encode inflection (see discussion below in this section). Regardless of whether inflectional markers are analysed as affixes or clitics, most of them follow the stem. 

The boundaries between \isi{derivation} and inflection are not sharp, both processes can rather be defined as two points on a continuum \citep[261]{Croft2000}. In Paunaka, this is especially clear for all kinds of markers that attach to the edge of an \isi{active verb} stem\is{verbal stem} replacing or directly following the thematic suffixes\is{thematic suffix} (if applicable, as we have just seen in (\getref{ex:VerbStems-2}), not all active verb stems have a thematic suffix), thus those markers which fuse with RS marking.\is{reality status}\is{inflection|)}

\is{transcategorial morphology|(}
Turning to the notion of “clitic”\is{clitic|(} now, this term has been applied to a wide range of phenomena \citep[388]{Nevis2000}, which is not surprising because “no rigorously defined cross-linguistically applicable concept of clitic seems to exist” \citep[276]{Haspelmath2015}.

The most important feature to distinguish a clitic from an affix on which most authors agree is their non- (or low) selectivity or promiscuous attachment, i.e. clitics can attach to words belonging to more than one \isi{word class}. However, this seems to be at most a necessary condition of clitichood, rather than a sufficient one, if we consider, for instance, the way \citet[176]{BickelNichols2007} formulate the matter: “An element is a clitic only if it can attach to hosts of diverse categories”.\footnote{Low selectivity is not even a necessary condition for some authors; consider, for instance, \citet[503]{ZwickyPullum1983}: “Clitics \textit{can} exhibit a low degree of selection with respect to their host” (emphasis added) or \citet[44]{Aikhenvald2003a} for a similar statement.} As for Paunaka, there are many markers with low selectivity, but if this criterion is only a necessary condition, it does not automatically follow that they are clitics. Considering other criteria of clitics that have been proposed in the literature does not help. \citet[37]{SpencerLuis2012} propose seven properties,\footnote{The criteria read as follows: “(1) Clitics express functional (inflectional) categories or discourse functions. (2) Clitics are generally unstressed (and unstressable). (3) Clitics require a host to attach to. (4) Clitics show low selectivity towards their host (promiscuous attachment). (5) Clitics typically appear in rigidly ordered clusters (templates). (6) Clitics and clitic clusters often have different syntax from fully-
fledged words. A particularly common phenomenon is the 2P clitic (cluster), in which the clitics have to be placed after the first constituent (word/phrase) of the phrase or clause they relate to. (7) Pronominal clitics often serve as the argument of the verb, but in some languages the clitics can be doubled by full noun phrases, giving the appearance of subject-verb or object-verb agreement” \citep[37]{SpencerLuis2012}.} only two of them set clitics apart from affixes, one being their low selectivity and the other one a possible fixed position in the clause, most commonly the second position.\footnote{As for their property of clitics being unstressable, the authors contradict themselves by devoting an entire section to the subject of clitics and stress assignment, showing that markers identified as clitics may well fall under the rules of stress assignment in some languages \citep[cf.][84--92]{SpencerLuis2012}.} None of Paunaka’s markers with promiscuous attachment are confined to a specific syntactic position inside the clause.

\citet[277]{Haspelmath2015} states that “there is no single set of properties that always uniquely identifies clitics and distinguishes them from affixes”.\footnote{Consider also \citet[503--504]{ZwickyPullum1983}, who name six differences between affixes and clitics, but all of them very vaguely formulated. \citet[52]{Sadock1991} states that clitics “are characterized” by 15 criteria, but at the same time, he is convinced that there are markers which are clitics although they differ from most of these criteria \citep[55]{Sadock1991}. \citet[43]{Aikhenvald2003b} also lists 15 criteria, but explicitly states that clitics “can be characterised in terms of” these criteria, thus she does not offer any \textit{defining} criteria either.} This is also reflected in the way in which low selectivity has been treated in research on Arawakan languages\is{Arawakan languages|(}.

In applying her proposed criteria to characterise clitics, \citet[]{Aikhenvald2003b} analyses some markers of low selectivity as clitics and others as affixes in Tariana, an Arawakan language distantly related to Paunaka. She carefully explains her analysis as clitics for those markers she identifies as such, but does not offer an explanation why she dismisses the analysis as clitic for the other markers that can be attached to different parts of speech, such as the privative and attributive markers, as well as the person markers \citep[73]{Aikhenvald2003b}.\footnote{This is not to say that I do not agree with her, presupposed first, that we want to make a distinction between affixes and clitics in the analysis (which may make sense in the case of Tariana) and second, that affixes can show low selectivity as well.}
%person markers and privative and attributive work similarly as in Paunaka, I would come to the same conclusion, since person markers have different functions on nouns and on verbs and the privative and attributive marker are derivational rather than inflectional morphemes. 

Person markers\is{person marking|(} are an interesting case in the Arawakan languages. They are cognate throughout the whole family, at least in parts of the paradigm, and they can be used to mark possessors\is{possessor} on nouns and S/A\is{subject} participants on verbs in general \citep[]{Danielsen2014}. Thus they can be claimed to show low selectivity. In descriptions of individual languages, some researchers have defined them as affixes and some as clitics. Person markers behave differently throughout the family, e.g. with regard to co-occurring or alternating with pronouns,\is{pronoun} and this may be a decisive factor in the choice of one or the other analysis. Yet there is some insecurity about their status. Concerning Alto Perené, a language of the Kampan branch, thus more closely related to Paunaka than aforementioned Tariana, \citet[101]{Mihas2015} states: “The syntactic status of person/possessor morphemes remains a difficult issue. In my doctoral thesis (Mihas 2010), I classified them as phrasal clitics. Additional evidence, collected during the last years, indicates they are more affix-like in their behavior”. %She then goes on to discuss the criteria which make them resemble clitics and those that point to status of an affix. 

Another solution to the matter has been proposed by \citet[]{Facundes2000} for Apurinã of the Purus branch (also more closely related to Paunaka than Tariana). Disagreeing with the sharp distinction between clitics and affixes, mainly because he objects to the view that clitics are subject to the rules of syntax and not morphology, he proposes a language-specific class of “special bound formatives” which contains the person markers, but also others like the oblique marker and TAME markers. All of them show low selectivity and some may float in the clause \citep[431]{Facundes2000}. They partly encode concepts similar to the ones expressed by the Paunaka markers with low selectivity.\is{person marking|)} 

I think the previous short discussion has sufficiently shown that defining clitics is not straigthforward in general and also specifically as regards Arawakan languages.\is{Arawakan languages|)} Many have noted this before me. I thus decided not to make use of the notion of “clitic” in this grammar and consequently not to use equal signs (=) as opposed to dashes (-). There is only one exception: nominal demonstratives\is{nominal demonstrative} can occasionally be realised as a phonetically reduced, unstressed form attached to a preceding word. If this happens, the process may be referred to as cliticisation. It is different from the other cases dealt with below (see \sectref{sec:LowSelectivityMarkers}), because there is a corresponding free form that would occur in the very same position of the clause.\footnote{This would count as a “simple clitic” in the terminology of \citet[]{Zwicky1977}.} With regards to terminology in explanatory texts, I make use of the neutral term “marker”. A markers is a grammatical morphemes that marks a feature of a  grammatical category and is phonologically attached to a word. This includes affixes, clitics (if one wants to maintain the concept) and borderline cases.\is{affix|)} 

The following section is dedicated to those markers with promiscuous attachment in terms of their possible hosts and loci, as well as their possible different functions on words belonging to different word classes. More complete descriptions of their semantics are found in subsequent chapters. As for the organisation of these chapters, I decided to describe person and number marking separately for nouns and verbs. Diminutives are described in the chapter on nouns, because they primarily relate to nouns, even if they are attached to another part of speech. Correspondingly, TAME markers are described in the chapter about verbs, because they primarily relate to predicates, and most predicates are verbs. In addition, degree markers are also found in the chapter about verbs.

%Although it may be useful in individual languages to set up a special sub-category of clitic to capture various distributional or formal generalisations, there is no obvious sense in which clitics represent a uniform, universal category. Rather, a complete grammatical description of a language has to establish for each element, and sometimes for each class of contexts in which an element is used, just what clitic-like or non-clitic-like properties that element has \citep[11]{SpencerLuis2012}

\section{Markers with low selectivity}\label{sec:LowSelectivityMarkers}
\is{verb|(}

This section gives an overview of the markers that can occur with different parts of speech. Since active verbs\is{active verb} are usually among the classes of words that markers with low selectivity attach to, \figref{fig:VerbTemplate-1} provides a schematic overview of the structure of such a verb. Markers that only occur on verbs are marked in grey, and they will not be considered in this section. Reduplication as a “nonlinear formative” \citep[183]{BickelNichols2007} will not be considered here either.

\begin{sidewaysfigure}
\centering
%\includegraphics[width=\textwidth]{figures/VerbTemplate-1.png}
\includegraphics[width=\textwidth]{figures/VerbTemplate-1-new.pdf}
\caption{Template of an active verb}
\label{fig:VerbTemplate-1}
\end{sidewaysfigure}


Everything marked with an asterisk * must be expressed on a verb: \isi{subject} marking, verb stem\is{verbal stem} and reality status (RS) marking\is{reality status} are always obligatory, although RS marking fuses with either the last \isi{suffix} of the stem or one of the markers directly following the stem. As for the \isi{object} markers and the \isi{plural} marker, they are only demanded under specific conditions: if the verb has an SAP object, an object marker occurs\is{person marking} obligatorily, and if it has a human\is{animacy} third person plural \isi{subject} or \isi{object}, the \isi{plural} marker has to be used. Not all of the markers can co-occur on a verb: only one \isi{aspect} marker is possible (\textsc{incmp}, \textsc{prsp}, \textsc{dsc}, \textsc{iam}) and only one that encodes epistemic \isi{modality} (\textsc{uncert}, \textsc{ded}). Intensive (\textsc{ints})\is{intensifier} and \isi{limitative} (\textsc{lim}1 and \textsc{lim}2) markers are not compatible either. There is often no morphology at all besides the obligatory categories of person/number\is{person marking} and RS.\is{reality status} If other markers are added, there are usually maximally two of them, though exceptions are found, which shows that the possible number of markers is not restricted to two. For some markers, namely \textsc{lim}1, \textsc{lim}2, and \textsc{emph}2, it was not possible to determine their exact position in the template with certainty due to lack of examples. This is why they are marked with ?. Some markers can appear in two different slots, which is indicated by a superscript number, the one closer to the stem taking the number 1 in these cases. Besides the ones overtly marked by these numbers, some other markers have occasionally also been found in other slots. Since most of them are disyllabic and encode only one feature, they are easily recognisable as meaningful units. This makes them manipulable. If, for instance, a speaker has forgotten to insert a marker in its usual slot, she can just add it later. In real speech, we thus find verbs with orders of markers that deviate from the template. Nonetheless, in elicitation such forms are not accepted. I believe that this general recognisability and 1:1 correspondence between form and meaning is a prerequisite for these markers being able to attach promiscuously to words of different classes. At the same time, the high number of low selectivity markers shows us that in Paunaka (and presumably also in other Arawakan languages that have similar systems) the distinction between different parts of speech plays\is{word class} a more minor role than it does in other languages. This also becomes apparent when looking at syntactic relations: nouns\is{noun} (and also adjectives)\is{adjective} can be predicates (see Chapter \ref{sec:NonVerbalPredication}), and verbs (in headless relative clauses)\is{relative relation} can act as arguments\is{argument} (see \sectref{sec:NonVerbalPredication}). It is a reduction of linguistic effort if the same (or very similar) notions can be expressed by one marker instead of two or more.

The following sections are dedicated to different morphemes with low selectivity.


\subsection{Classifiers}\label{sec:AffClCLF}\is{classifier|(}

Classifiers combine with verbs, nouns\is{noun|(} and adjectives\is{adjective|(} in Paunaka. They derive nouns\is{derivation} from verb stems\is{verbal stem} and nouns with new meanings from other nouns.\is{nominal stem} The combination of noun or verb stem with a specific classifier is often totally lexicalised\is{lexicalisation} in these cases.\is{noun|)} When combined with adjectives and verbs, classifiers specify some properties of a referent, mostly shape and consistency.\is{adjective|)} 

(\getref{ex:FirstCLF-1}) shows a noun and a verb stem lexicalised with the classifier \textit{-pa}, which is used with dusty things, flour, small particles and the like. In the case of the noun in (\getfullref{ex:FirstCLF-1.1}), the stem never occurs without a classifier (\textit{*mute}),\footnote{There is also \textit{muteji} ‘mud’ with the same noun stem and a different classifier.} while in the case of the verb in (\getfullref{ex:FirstCLF-1.2}), the root \textit{-yÿti} ‘set on fire’ also occurs without a classifier in the stem \textit{-yÿtiku} ‘set (a pot) on fire’.

\ea\label{ex:FirstCLF-1}
  \ea\label{ex:FirstCLF-1.1}
\begingl
\glpreamble mutepa\\
\gla mute-pa\\
\glb earth-\textsc{clf:}particle\\
\glft ‘earth, dust’
\endgl
  \ex\label{ex:FirstCLF-1.2}
\begingl
\glpreamble -yÿtipajiku\\
\gla yÿti-pa-ji-ku\\
\glb set.on.fire-\textsc{clf:}particle-\textsc{intsv}-\textsc{th}1\\
\glft ‘cook chicha’
\endgl
\z
\xe

Classifiers occur inside the verb stem.\is{verbal stem} Their nature is not inflectional in Paunaka, they can be considered purely derivational devices\is{derivation} on the edge of being lexical, such that if a classifier derives a new noun, the process resembles \isi{compounding}. Thus due to their position close to the root\is{verbal root} and their derivational function, I consider them a special type of lexical \isi{affix} in spite of their promiscuous attachment.

Classifiers are described in more detail in \sectref{sec:Classifiers} below.
\is{classifier|)}

\subsection{Associated motion and regressive markers}\label{sec:AffClAM}\is{inflection|(}\is{associated motion|(}

Among the associated motion (AM) markers, the ones expressing concurrent associated motion have been found once on a noun and once on a numeral in the corpus. In all other cases, the concurrent motion markers attach to verbs. (\getref{ex:FirstAM-1}) shows the use of the cislocative concurrent motion marker on a numeral, and (\getref{ex:FirstAM-2}) is the corresponding expression with a verb.\footnote{Note that although the last morpheme in (\getref{ex:FirstAM-1}) is glossed as a third person marker, it belongs to the numeral whose simple form without any further morphology is \textit{chÿnachÿ} ‘one’. This numeral is probably composed of \textit{chÿ-na-chÿ} 3-\textsc{clf:}general-3. As for the different forms of the AM marker, its form is actually \textit{-(CV)kÿupunu}, i.e. it is optionally accompanied by a reduplicated\is{reduplication} syllable.}

\ea\label{ex:FirstAM-1}
\begingl
\glpreamble chÿnanakÿupunuchÿ\\
\gla chÿna-nakÿupunu-chÿ\\
\glb one-\textsc{am.conc.cis}-3\\
\glft ‘she came alone’
\endgl
\trailingcitation{[cux-120410ls.173]}
\xe

\ea\label{ex:FirstAM-2}
\begingl
\glpreamble pipÿsisikÿupunu\\
\gla pi-pÿsisi-kÿupunu\\
\glb 2\textsc{sg}-be.alone-\textsc{am.conc.cis}\\
\glft ‘you came alone’
\endgl
\trailingcitation{[mrx-c120509l.023]}
\xe

I believe the very few examples of AM markers occurring with words other than verbs do not challenge their status as suffixes,\is{suffix} but rather show the flexibility of the whole system of inflectional markers: given that there is an appropriate semantic context, markers can be employed to express an inflectional notion, no matter which word class a particular word may belong to. 

AM markers are actually on the edge of being derivational\is{derivation} and inflectional. They definitely add semantic content to the verb; however, there is a small paradigm of different mutually-exclusive markers. AM markers can either replace a \isi{thematic suffix} of a stem\is{verbal stem|(} or follow it,\footnote{Replacement of a stem suffix has not been found with the subsequent motion marker, but this one is almost out of use.} with no apparent difference in meaning (see (\getref{ex:new23-punu1}) and (\getref{ex:new23-punu2}) respectively) – with the difference in direction (‘come’ in (\getref{ex:new23-punu1}) and ‘go’ in (\getref{ex:new23-punu2})) being bound to the semantics of the marker and not to its position on the verb. If an AM marker is present on the verb, it becomes the locus of RS inflection,\is{reality status} and the verb stem ends in default /u/ in this case. Since RS is the most important inflectional category besides person, this is meaningful and shows the deep integration of an AM marker with the stem.\is{verbal stem|)}

\ea\label{ex:new23-punu1}
\begingl
\glpreamble asamaipunu eka paunaka\\
\gla e-semai-punu eka paunaka\\
\glb 2\textsc{pl}-search-\textsc{am.prior} \textsc{dem}a Paunaka\\
\glft ‘you came in search of Paunaka’
\endgl
\trailingcitation{[uxx-p110825l.098]}
\xe

\ea\label{ex:new23-punu2}
\begingl
\glpreamble nisemaikupunu echÿu bakajane\\
\gla ni-semaiku-punu echÿu baka-jane\\
\glb 1\textsc{sg}-search-\textsc{am.prior} \textsc{dem}b cow-\textsc{distr}\\
\glft ‘I went to look for the cows’
\endgl
\trailingcitation{[mxx-n101017s-2.072]}
\xe

The regressive marker\is{regressive/repetitive|(} is derived\is{derivation} from an AM marker. It encodes regressive motion on motion verbs and repetition on any other word. Besides verbs, it has been found on nouns and adverbs. If used with verbs, the regressive marker is the place for RS inflection,\is{reality status} just like the AM markers are, i.e. it is placed very close to the stem.\is{verbal stem} The regressive marker has several allomorphs and sometimes forms a phonological word of its own. On some occasions, it occurs on the verb and another word in the clause; sometimes its placement seems to correspond to its Spanish equivalent \textit{de nuevo} ‘again’. One example of the latter is given in (\getref{ex:FirstAM-3}).

\ea\label{ex:FirstAM-3}
\begingl
\glpreamble titukanube eka beteapunuku...\\
\gla ti-itu-uka-nube eka bi-etea-punuku\\
\glb 3i-master-\textsc{add.irr}-\textsc{pl} \textsc{dem}a 1\textsc{pl}-language-\textsc{reg}\\
\glft ‘they will also learn our language again...’
\endgl
\trailingcitation{[mxx-x110917.18]}
\xe
\is{regressive/repetitive|)}

A detailed discussion about associated motion and related categories is found in \sectref{sec:AssociatedMotion}.\is{associated motion|)}


\subsection{TAME markers}\label{sec:AffClTAME}
\is{tense|(}\is{aspect|(}\is{modality|(}\is{evidentiality|(}

The tense, aspect, modality and evidentiality (TAME) markers show low selectivity insofar as they can minimally attach to verbal and non-verbal predicates alike,\is{non-verbal predication} just as is usual in \isi{Arawakan languages} \citep[cf.][13]{Overall2018}.\footnote{The positions of the two alternating \isi{optative} markers \textit{-yuini} and \textit{-jÿti} on a verb could not be established due to lack of data. While \textit{-yuini} can attach to predicates as well as to the \isi{negative particle}, \textit{-jÿti} can only attach to predicates. These predicates need not be verbal predicates though.} Consider (\getref{ex:new23-prsp-1}), in which the \is{prospective|(} prospective marker \textit{-bÿti} attaches to a verb, and compare with (\getref{ex:new23-prsp-2}), where it occurs on a non-verbal predicate, a loan from Spanish.\footnote{Note that verbs borrowed\is{borrowing} from Spanish are mostly integrated into Paunaka discourse as non-verbal predicates,\is{non-verbal predication} which can be recognised by a different position of the subject marker and use of a different irrealis marker\is{non-verbal irrealis marker} if applicable, see \sectref{sec:borrowed_verbs} for more information.}

\ea\label{ex:new23-prsp-1}
\begingl
\glpreamble tebitakunubebÿti\\
\gla ti-ebitaku-nube-bÿti\\
\glb 3i-clear-\textsc{pl}-\textsc{prsp}\\
\glft ‘they first ploughed’
\endgl
\trailingcitation{[jxx-p120515l-2.113]}
\xe

\ea\label{ex:new23-prsp-2}
\begingl
\glpreamble pikichupa pario abansaunÿinabÿti nipikeikiu\\
\gla pi-kichupa pario abansau-nÿ-ina-bÿti ni-pikeik-i-u\\
\glb 2\textsc{sg}-wait.\textsc{irr} some advance-1\textsc{sg}-\textsc{irr.nv}-\textsc{prsp} 1\textsc{sg}-knot-\textsc{subord}-\textsc{real}\\
\glft ‘wait a bit until I have advanced my knotting (of the hammock)’
\endgl
\trailingcitation{[rxx-e181022le]}
\xe
\is{prospective|)}

Some can also be found on the \isi{negative particle} \textit{kuina} and/or on non-predica\-tively used adverbs\is{adverb} in the clause as in (\getref{ex:new23-uncert}), in which \textit{tanÿma} ‘now’ bears the uncertainty marker.
 
\ea\label{ex:new23-uncert}
\begingl
\glpreamble tipajÿkutu tanÿmakena\\
\gla ti-pajÿku-tu tanÿma-kena\\
\glb 3i-stay-\textsc{iam} now-\textsc{uncert}\\
\glft ‘he probably keeps staying (here) now’
\endgl
\trailingcitation{[mqx-p110826l.092]}
\xe
  
The \isi{uncertainty} marker \textit{-kena} ‘\textsc{uncert}’ and the remote marker\is{remote past|(} \textit{-bane} ‘\textsc{rem}’ have identical free forms (\textit{kena} and \textit{bane}), and the \isi{iamitive} marker \textit{-tu} ‘\textsc{iam}’ probably goes back to the \isi{adverb} \textit{metu} ‘ready, already’, but has an extended meaning. Consider (\getref{ex:FirstTAME-1}), which exemplifies the free and the bound use of the remote marker \textit{-bane}. Both co-occur in one clause here, but this is not necessary in general. The other TAME markers do not show any similarity with free forms.

\ea\label{ex:FirstTAME-1}
\begingl
\glpreamble bane kuina takikikÿbane chenekÿ\\
\gla bane kuina ti-a-ki-ki-kÿ-bane chenekÿ\\
\glb \textsc{rem} \textsc{neg} 3i-\textsc{irr}-be.many-\textsc{rdpl}-\textsc{clf:}bounded-\textsc{rem} way\\
\glft ‘in the old days the way was not wide’
\endgl
\trailingcitation{[rxx-p181101l-2.067]}
\xe

The remote marker has another peculiarity. When attached to human\is{animacy} nouns,\is{noun} it usually specifies that the referent has passed away, as in (\getref{ex:new23-deceased}).\is{deceased marking} As for the possibility of other TAME markers being used referentially, this is much less clear.%\footnote{The non-verbal irrealis marker can be used to express nominal irrealis (see \sectref{NominalRS}), and a referential use of TAME markers is definitely possible in Mojeño Trinitario \citep[cf.][]{Rose2017a}. There are some cases including the uncertainty marker \textit{-kena} or the deductive marker \textit{-yenu}, but they include non-verbal predication.} 

\ea\label{ex:new23-deceased}
\begingl 
\glpreamble metu tepakutu nÿabane\\
\gla metu ti-paku-tu nÿ-a-bane\\ 
\glb already 3i-die-\textsc{iam} 1\textsc{sg}-father-\textsc{rem}\\ 
\glft ‘my late father had died’
\endgl
\trailingcitation{[rxx-e120511l.169]}
\xe
\is{remote past|)}

Most TAME markers are occasionally used more than once in a clause;\footnote{Exceptions are the \isi{prospective} (\textsc{prsp}), \isi{avertive} (\textsc{avert}) and \isi{deductive} (\textsc{ded}) markers, which have only been found once per clause in the corpus, and possibly the \isi{optative} markers.} one example with the reportive marker being attached to both the verb and the object NP is given in (\getref{ex:FirstTAME-2}).
 

 %one example with a iamitive marker occurring twice is given in (\getref{ex:FirstTAME-2222}), where the iamitive is attached to an adverb and to a Spanish verb which is borrowed as a non-verbal predicate.
 %\ea\label{ex:FirstTAME-2222}
%\begingl
%\glpreamble nakatu pensaitu eka aitubuchepÿimÿnÿ\\
%\gla naka-tu pensai-tu eka aitubuchepÿi-mÿnÿ\\
%\glb here-\textsc{iam} think-\textsc{iam} \textsc{dem}a boy-\textsc{dim}\\
%\glft ‘here now, the little boy is thinking’\\
%\endgl
%\trailingcitation{(mox-a110920l-2 058)}
%\xe

\ea\label{ex:FirstTAME-2}
\begingl
\glpreamble tumuji nÿkÿikimÿnÿji yÿtÿuku\\
\gla ti-umu-ji nÿkÿiki-mÿnÿ-ji yÿtÿuku\\
\glb 3i-take-\textsc{rprt} pot-\textsc{dim}-\textsc{rprt} food\\
\glft ‘she took her little pot with food, it is said’
\endgl
\trailingcitation{[mox-n110920l.061]}
\xe

Most TAME markers occur quite remotely from the verb stem\is{verbal stem} and follow the \isi{object} and \isi{plural} markers. They thus definitely violate the Affix ordering hierarchy proposed by \citet[521]{Booij2010}, which is given in \figref{fig:AffixOrderingHierarchy}.

\begin{figure}[!ht]
\centering
Voice > Aspect > Tense > Agreement 
\caption{Affix ordering hierarchy after \citet[521]{Booij2010}}
\label{fig:AffixOrderingHierarchy}
\end{figure}

This hierarchy indicates that tense and aspect affixes are usually closer to the stem than agreement markers.\footnote{Note, however, that person markers are not strictly\is{person marking} agreement markers in Paunaka. I suppose that the hierarchy is equally true for person indexes \citep[for the term “index” see][]{Haspelmath2013}.} As can be seen in (\getref{ex:FirstTAME-3}), however, the iamitive follows the 1\textsc{sg} object marker.

\ea\label{ex:FirstTAME-3}
\begingl
\glpreamble tekichunÿtu\\
\gla ti-ekichu-nÿ-tu\\
\glb 3i-invite-1\textsc{sg}-\textsc{iam}\\
\glft ‘she invited me (food)’
\endgl
\trailingcitation{[jmx-e090727s.171]}
\xe

There are two exceptions to this placement remote from the stem.\is{verbal stem} First, the \isi{discontinuous} marker, meaning ‘(not) anymore’, can either precede or follow the \isi{plural} marker. I do not know what lets the speaker choose one or the other option. Both combinations are rare in the corpus. The difference does not depend on the plural marker belonging to the subject or object participant. %with -nube relating to subject: i tanÿma kuina tanabunube pimiyanube: jxx-p120430l-2.547

\ea\label{ex:FirstTAME-4}
\begingl
\glpreamble kuina nichupuikabunube\\
\gla kuina ni-chupuika-bu-nube\\
\glb \textsc{neg} 1\textsc{sg}-know.\textsc{irr}-\textsc{dsc}-\textsc{pl}\\
\glft ‘I don’t know them anymore’
\endgl
\trailingcitation{[rxx-e181022le]}
\xe


\ea\label{ex:FirstTAME-5}
\begingl
\glpreamble kuina tisamuikanubebu\\
\gla kuina ti-samuika-nube-bu\\
\glb \textsc{neg} 3i-listen.\textsc{irr}-\textsc{pl}-\textsc{dsc}\\
\glft ‘they don’t listen anymore’
\endgl
\trailingcitation{[jxx-e190210s-01]}
\xe

Second, the \isi{incompletive} marker precedes not only the \isi{plural}, but also the \isi{distributive} marker in its second possible slot, and is thus even closer to the verb stem\is{verbal stem} than the \isi{discontinuous} marker.\footnote{It could be deduced that it must also precede the object markers, since they occur after the second possible slot for the distributive marker, thus this case would be in compliance with the agreement hierarchy. The \isi{incompletive} marker is (semantically) stativising though, which may be the reason that object markers are never found on a verb with \isi{incompletive} aspect, and thus we cannot make any statements about the hierarchy.}

\ea\label{ex:FirstTAME-6}
\begingl
\glpreamble tujikukuÿjaneyu\\
\gla ti-ujiku-kuÿ-jane-yu\\
\glb 3i-suckle-\textsc{incmp}-\textsc{distr}-\textsc{ints}\\
\glft ‘they still suckle a lot’
\endgl
\trailingcitation{[rxx-e120511l.364]}
\xe

In summary, most TAME markers follow the ones connected to person\is{person marking} (i.e. person, \isi{plural}, \isi{distributive}, \isi{diminutive}), but two of them always or sometimes precede them. Thus the class of TAME markers is not uniform regarding the position inside the verb template. More information about TAME markers can be found in \sectref{sec:OperationsPredicates}.
\is{evidentiality|)}\is{modality|)}\is{aspect|)}\is{tense|)}

\subsection{Person markers}\label{sec:AffClPerson}
\is{person marking|(}
\is{possessor|(}

There are two sets of related person markers, one preceding stems and the other one following stems. One of the two third-person markers, \textit{ti-} ‘3i’, only appears on verbs and only precedes the stem,\is{verbal stem} thus this one certainly qualifies as a \isi{prefix}. All other person markers occur on verbs, nouns and sometimes words of other parts of speech. I first consider person markers preceding stems. On verbs they mark subjects, as in (\getfullref{ex:firstPersonMarkers.1}), and on nouns they mark possessors, as in (\getfullref{ex:firstPersonMarkers.2}).

\ea\label{ex:firstPersonMarkers}
  \ea\label{ex:firstPersonMarkers.1}
\begingl
\glpreamble piniku\\
\gla pi-niku\\
\glb 2\textsc{sg}-eat\\
\glft ‘you ate’
\endgl
  \ex\label{ex:firstPersonMarkers.2}
\begingl
\glpreamble pijinepÿi\\
\gla pi-jinepÿi \\
\glb 2\textsc{sg}-daughter\\
\glft ‘your daughter’
\endgl
\z
\xe
\is{possessor|)}

Regarding the person markers following the stem, they mark objects\is{object|(} on verbs,\is{verbal stem} (\getfullref{ex:secondPersonMarkers.1}) and subjects\is{subject|(} on non-verbal predicates\is{non-verbal predication} of different types, e.g. nouns, as in (\getfullref{ex:secondPersonMarkers.2}).

\ea\label{ex:secondPersonMarkers}
  \ea\label{ex:secondPersonMarkers.1}
\begingl
\glpreamble pisimune\\
\gla pi-simu-ne\\
\glb 2\textsc{sg}-find-1\textsc{sg}\\
\glft ‘you found me’
\endgl
\newpage
  \ex\label{ex:secondPersonMarkers.2}
\begingl
\glpreamble baichane\\
\gla baicha-ne\\
\glb orphan-1\textsc{sg}\\
\glft ‘I am an orphan’
\endgl
\z
\xe

\is{subject|)}
\is{object|)}

As is apparent from (\getref{ex:firstPersonMarkers}) and (\getref{ex:secondPersonMarkers}), person markers do show low selectivity, but the functions encoded differ for nouns and verbs. This may be more apparent for the ones preceding the stem, because on verbs they mark arguments while on nouns they mark possessors, although as Danielsen (2021, p.c.) rightly remarks, in both cases “it is all about the primary argument”. The person markers following the stem both mark secondary arguments, but there is a very clear distinction as to the syntactic function of the \isi{argument}. 

Person markers do not alternate with pronouns\is{pronoun} in Paunaka. In applying the terminology by \citet[]{Haspelmath2013}, they are also called “indexes” in this grammar. 

More information on argument indexing on verbs can be found in \sectref{sec:NumberPersonVerbs}. Encoding of possession is discussed in \sectref{sec:Possession}, and argument indexing on non-verbal predicates is among the topics of \sectref{sec:NonVerbalPredication}.

\is{person marking|)}

\subsection{Plural marker}\label{sec:AffClPlural}
\is{plural|(}

The plural marker \textit{-nube} is found on nouns and verbs, as well as on numerals\is{numeral} and nominal demonstratives to express non-singularity of human\is{animacy} referents, as in (\getref{ex:FirstPlural-1}).\footnote{Although it is not attested in the corpus, I would assume that it can occur with the adjective \textit{kana} ‘this size’ as well, because this adjective can take person markers.}

\ea\label{ex:FirstPlural-1}
  \ea
\begingl
\glpreamble apimiyanube\\
\gla apimiya-nube\\
\glb girl-\textsc{pl}\\
\glft ‘girls’
\endgl
  \ex
\begingl
\glpreamble tiyununube\\
\gla ti-yunu-nube\\
\glb 3i-go-\textsc{pl}\\
\glft ‘they went’
\endgl
  \ex
\begingl
\glpreamble ruschÿnube\\
\gla ruschÿ-nube\\
\glb two-\textsc{pl}\\
\glft ‘two (people)’
\endgl
  \ex
\begingl
\glpreamble echÿunube\\
\gla echÿu-nube\\
\glb \textsc{dem}b-\textsc{pl}\\
\glft ‘they’
\endgl
\z
\xe

The plural marker combines with the third person markers\is{person marking} \textit{ti-} and \textit{chÿ-}/\textit{chi-}. Both are underspecified for number; if plural number of a human participant indexed by a person marker is encoded, the plural marker is simply added after the stem.\is{verbal stem} The plural marker thus occurs in the same contexts in which (third) person markers occur (see \sectref{sec:AffClPerson} above). 

As for \isi{possession} marking, if both possessor and possessed are human\is{animacy} third person, there is ambiguity between belonging to the possessor and belonging to the possessed. Consider the following examples: in (\getfullref{ex:FirstPlural-2.1}) the plural marker belongs to the possessed, in (\getfullref{ex:FirstPlural-2.2}) it belongs to the possessor, but in (\getfullref{ex:FirstPlural-2.3}) it can belong to the possessed, to the possessor, or to both. It is only the context that can clarify what precisely is meant.

\ea\label{ex:FirstPlural-2}
  \ea\label{ex:FirstPlural-2.1}
\begingl
\glpreamble nijinepÿinube\\
\gla ni-jinepÿi-nube\\
\glb 1\textsc{sg}-daughter-\textsc{pl}\\
\glft ‘my daughters’
\endgl
  \ex\label{ex:FirstPlural-2.2}
\begingl
\glpreamble chiputrerunenube\\
\gla chi-putreru-ne-nube\\
\glb 3-pasture-\textsc{possd}-\textsc{pl}\\
\glft ‘their pasture’
\endgl
  \ex\label{ex:FirstPlural-2.3}
\begingl
\glpreamble chijinepÿinube\\
\gla chi-jinepÿi-nube\\
\glb 3-daughter-\textsc{pl}\\
\glft ‘his/her daughters’\\or: ‘their daughter’\\or: ‘their daughters’
\endgl
\z
\xe

The plural marker on verbs is also ambiguous as to whether it refers to a third person plural \isi{subject} or \isi{object}. If the verb is \isi{intransitive}, it can only belong to the subject marker, as in (\getfullref{ex:FirstPlural-3.1}). If the subject of a \isi{transitive} verb is a speech act participant (SAP), it can only mark the object, see (\getfullref{ex:FirstPlural-3.2}). However, if both \isi{subject} and \isi{object} of a \isi{transitive} verb are human\is{animacy} third persons, it is again only the context that can clarify who is the plural participant (see (\getfullref{ex:FirstPlural-3.3})).

\ea\label{ex:FirstPlural-3}
  \ea\label{ex:FirstPlural-3.1}
\begingl
\glpreamble tiyununube\\
\gla ti-yunu-nube\\
\glb 3i-go-\textsc{pl}\\
\glft ‘they go’
\endgl
  \ex\label{ex:FirstPlural-3.2}
\begingl
\glpreamble nichupuikunube\\
\gla ni-chupuiku-nube\\
\glb 1\textsc{sg}-know-\textsc{pl}\\
\glft ‘I know them’
\endgl
  \ex\label{ex:FirstPlural-3.3}
\begingl
\glpreamble chimunube\\
\gla chi-imu-nube\\
\glb 3-see-\textsc{pl}\\
\glft ‘he/she sees them’\\or: ‘they see him/her’\\or: ‘they see them’
\endgl
\z
\xe

If a clause has both an index on the verb and a conominal NP,\is{conomination} the plural marker usually occurs on both the verb and the NP,\is{noun phrase} as in (\getref{ex:agr-aff}).

\ea\label{ex:agr-aff}
\begingl
\glpreamble suntabunube chumunubetu labionyae\\
\gla suntabu-nube chÿ-umu-nube-tu labion-yae\\
\glb soldier-\textsc{pl} 3-take-\textsc{pl}-\textsc{iam} plane-\textsc{loc}\\
\glft ‘the soldiers (i.e. policemen) took her to the plane’
\endgl
\trailingcitation{[jxx-p120430l-1.232]}
\xe

In complementation,\is{complement relation} however, there is no such agreement marking: if the subject has a human third-person referent, the plural marker often occurs only once, namely on the complement verb as in (\getref{ex:FirstPlural-4}). It sometimes also attaches to both verbs, but it can only occur on the complement-taking verb alone if the complement verb has a different subject, i.e. where there is no other subject to agree with, as in (\getref{ex:2023-FirstPlural}).

\ea\label{ex:FirstPlural-4}
\begingl
\glpreamble tisachu tinikanube\\
\gla ti-sachu ti-nika-nube\\
\glb 3i-want 3i-eat.\textsc{irr}-\textsc{pl}\\
\glft ‘they want to eat’
\endgl
\trailingcitation{[jxx-e190210s-01]}
\xe

\ea\label{ex:2023-FirstPlural}
\begingl
\glpreamble tisachunube bitupupuna echÿu betea\\
\gla ti-sachu-nube bi-itu-pupuna echÿu bi-etea\\
\glb 3i-want-\textsc{pl} 1\textsc{pl}-master-\textsc{reg.irr} \textsc{dem}b 1\textsc{pl}-language\\
\glft ‘they want us to learn our language again’
\endgl
\trailingcitation{[ump-p110815sf.130-133]}
\xe

%tisachunube tichemupuna bétea, jxx-x110916.04

The plural marker also occurs only once if a plural noun is modified\is{modification} by a demonstrative as in (\getref{ex:FirstPlural-5}). It can be attached to demonstratives as well, but in those cases there is usually no noun.

\ea\label{ex:FirstPlural-5}
\begingl
\glpreamble echÿu jentenube\\
\gla echÿu jente-nube\\
\glb \textsc{dem}b man-\textsc{pl}\\
\glft ‘these men’
\endgl
\trailingcitation{[mxx-p110825l.045]}
\xe

For Proto-Arawakan,\is{Arawakan languages} a separate third person plural index \textit{*na} has been postulated,\footnote{See \citet[505, 513]{Danielsen2011} for a summary of different approaches to reconstruct Proto-Arawakan person markers.} but there are a number of languages that encode plural separately from third person \citep[]{Danielsen2014}.\is{person marking|(} The plural marker was probably first used with nouns exclusively and was subsequently extended to number marking together with the underspecified third person markers. We can state that its function is to express plural number of human nouns (and demonstratives referring to humans) or plural number of the third person markers. Subject,\is{subject} \isi{object} and possessor marking derive from the latter. On non-verbal predicates\is{non-verbal predication} with human third person referents, it can also specify the number of a subject without this being bound to person marking, as in (\getref{ex:FirstPlural-6}), which builds on the non-verbal third person expression for ‘come’ (see \sectref{sec:Kapunu} for more information on this topic).\footnote{Note that third person marking is largely restricted to markers preceding the stem. Third person markers following the stem only occur in a few specific contexts, including speech verbs\is{speech verb} and subordinate (deranked) verbs\is{deranked verb} (see \sectref{sec:3_suffixes}). Non-verbal predicates\is{non-verbal predication} index subjects by person markers following the stem (see \sectref{sec:NonVerbalPredication}), thus third person subjects are not marked on them.}\is{person marking|)}

\ea\label{ex:FirstPlural-6}
\begingl
\glpreamble kapununube naka ...\\
\gla kapunu-nube naka\\
\glb come-\textsc{pl} here\\
\glft ‘they came here...’
\endgl
\trailingcitation{[jmx-e090727s.322]}
\xe

The marker has some properties of an \isi{agreement} affix, such as being attached to both noun and verb as in (\getref{ex:agr-aff}), but it fails to indicate \isi{agreement} in other cases, like complementation (\getref{ex:FirstPlural-4}) and \isi{modification} of a noun by a demonstrative (\getref{ex:FirstPlural-5}).


More information about plural marking is given in \sectref{sec:NumberNouns} for nouns and in \sectref{sec:Verbs_3PL} for verbs as a part of person marking .


%chisikanube = her arms
%chimaletanenube = their suitcases
%chitapikinenube = their travel supplies
%chiputrerunenube = su potrero (mass noun?)
\is{plural|)}

\subsection{Distributive marker}\label{sec:AffClDistributive}\is{distributive|(}

The distributive marker \textit{-jane} is used to mark plural number of non-human\is{animacy} referents. Theoretically, the same kind of ambiguities described in \sectref{sec:AffClPlural} above also apply to the distributive marker, but, being optional, it is generally employed less frequently than the \isi{plural} marker. There are, for instance, no examples in the corpus in which \textit{-jane} would mark a plural possessor, but this is possibly due to pragmatic reasons, because non-human\is{animacy} entities are in general less likely to be construed as possessors. In addition, contrary to the plural marker, if an NP conominates\is{conomination} an argument indexed on a verb, \textit{-jane} often appears only on the NP argument\is{noun phrase} or only on the verb, the latter being shown in (\getref{ex:new23-janeV}). It may, however, also be attached to both NP and verb. If the non-verbal \isi{copula}\is{non-verbal predication} \textit{kaku} is the predicate of the clause, \textit{-jane} usually attaches to the noun\is{noun} only; this is the case in (\getref{ex:FirstDistr-2}).

\ea\label{ex:new23-janeV}
\begingl 
\glpreamble tibÿkupujaneji bakayayae baka\\
\gla ti-bÿkupu-jane-ji bakaya-yae baka \\ 
\glb 3i-enter-\textsc{distr}-\textsc{rprt} enclosure-\textsc{loc} cow\\ 
\glft ‘the cows went into the enclosure, it is said’
\endgl
\trailingcitation{[jxx-p151016l-2]}
\xe

\ea\label{ex:FirstDistr-2}
\begingl 
\glpreamble kakutu naka bakajane\\
\gla kaku-tu naka baka-jane\\ 
\glb exist-\textsc{iam} here cow-\textsc{distr}\\ 
\glft ‘the cows are here now’
\endgl
\trailingcitation{[mxx-n151017l-1.55]}
\xe

When attached to verbs, \textit{-jane} can occur in two slots with no difference in meaning. One of these slots is right on the edge of the stem,\is{verbal stem} even before RS marking, which is very tightly connected with verbs. In this case, the distributive marker replaces the stem-closing \isi{thematic suffix} \textit{-ku} (if the verb in question usually takes this suffix). (\getref{ex:FirstDistr-1}) shows the two possible positions of the distributive marker on an \isi{active verb}; in (\getfullref{ex:FirstDistr-1.1}) it follows the thematic suffix, and in (\getfullref{ex:FirstDistr-1.2}) it replaces it.\footnote{As for the question why there is a realis marker on (\getfullref{ex:FirstDistr-1.2}) and apparently no realis marker on (\getfullref{ex:FirstDistr-1.1}), see \sectref{sec:VerbalRS}.}

\ea\label{ex:FirstDistr-1}
  \ea\label{ex:FirstDistr-1.1}
\begingl
\glpreamble tujikujane\\
\gla ti-ujiku-jane\\
\glb 3i-suckle-\textsc{distr}\\
\glft ‘they suckle’
\endgl
  \ex\label{ex:FirstDistr-1.2}
\begingl
\glpreamble tujijaneu\\
\gla ti-uji-jane-u\\
\glb 3i-suckle-\textsc{distr}-\textsc{real}\\
\glft ‘they suckle’
\endgl
\z
\xe


More information about distributive marking on nouns can be found in \sectref{sec:NounPL-jane} and on verbs as part of person marking in \sectref{sec:Verbs_3PL}. In addition, \textit{-jane} is also occasionally found on demonstratives and it may be found on adjectives, too, although there is only one example that includes a borrowed adjective in the corpus, which is given here as (\getref{ex:new23-yellow}).

\ea\label{ex:new23-yellow}
\begingl
\glpreamble echÿu amariyujane\\
\gla echÿu amariyu-jane\\
\glb \textsc{dem}b yellow-\textsc{distr}\\
\glft ‘the yellow ones (speaking about piranha)’
\endgl
\trailingcitation{[cux-c120414ls-2.097]}
\xe
\is{distributive|)}

\subsection{Collective marker}\label{sec:AffClCollective}\is{collective|(}

The collective marker \textit{-ji} is highly selective, but for semantic reasons rather than for word class. It has been found on nouns,\is{noun} adjectives\is{adjective} and stative verbs\is{stative verb} (which is why it does not show up in the verb template in \figref{fig:VerbTemplate-1}, since this template shows an active verb).\footnote{There seems to be one exception to this: the verb for ‘fight’ has a \isi{reciprocal} suffix followed by the collective marker (see (\getref{ex:FirstLim-2}) below). I have not found any other active verb taking the collective marker, which is why I neglect this single case here. } Regarding adjectives\is{adjective} and stative verbs, it is used only with expressions of colour and size, often in combination with a \isi{classifier} or incorporated noun.\is{incorporation} (\getref{ex:FirstCol-1}) is an example of an adjective and (\getref{ex:FirstCol-2}) of a stative verb bearing the collective marker.

\ea\label{ex:FirstCol-1}
\begingl
\glpreamble temetapuji\\
\gla teme-tapu-ji\\
\glb big-\textsc{clf:}scales-\textsc{col}\\
\glft ‘big ones (referring to piranhas here, but it can also refer to other fish species)’
\endgl
\trailingcitation{[cux-c120414ls-2.014]}
\xe 

\ea\label{ex:FirstCol-2}
\begingl
\glpreamble tisururukiji\\
\gla ti-sururu-ki-ji\\
\glb 3i-be.clear-\textsc{clf:}spherical-\textsc{col}\\
\glft ‘they are all light-coloured’
\endgl
\trailingcitation{[jxx-e150925l-1.164]}
\xe

Some nouns\is{noun} referring to small animals that occur in groups or swarms show up in the collective rather than in the singular form, especially some fish species. The same is true for \textit{-muki-ji} ‘hair’. Strikingly, the collective marker also obligatorily occurs on the kinship terms for ‘mother’, ‘father’, ‘brother’, ‘sister’, ‘sibling (of same sex)’ and ‘son’ if the possessor, the possessed or both have plural referents,\is{possession} as well as on the plural-only noun \textit{sesejinube} ‘children’. The collective marker then precedes the \isi{plural} marker, as in (\getref{ex:FirstCol-3}). Other nouns do not usually take the collective marker.

\ea\label{ex:FirstCol-3}
\begingl
\glpreamble chipijijinube\\
\gla chi-piji-ji-nube\\
\glb 3-sibling-\textsc{col}-\textsc{pl}\\
\glft ‘his/her siblings (of the same sex)’\\or: ‘their sibling’\\or: ‘their siblings’
\endgl
\xe

More information about collective marking on nouns can be found in \sectref{sec:Collective} and on verbs in \sectref{sec:Verbs_3PL}. Some more examples with adjectives taking the collective marker are scattered through \sectref{sec:Adjectives}.
\is{collective|)}

\subsection{Diminutive}\label{sec:AffClDiminutive}\is{diminutive|(}

The diminutive marker \textit{-mÿnÿ} can attach to nouns, adjectives, verbs, and more rarely nominal demonstratives and numerals. It can express smallness, but also emotional affection (sympathy, pity, modesty), and the latter is often more important in the choice than mere size. The diminutive usually expresses that the speaker has some emotional affection towards one specific participant in the clause, that one participant is small or both. It does not matter to which word it attaches; in (\getref{ex:FirstDim-1}) it attaches to a noun, in (\getref{ex:FirstDim-2}) to a stative verb, but in both cases it refers to a participant of the clause.


\ea\label{ex:FirstDim-1}
\begingl
\glpreamble kaku kabemÿnÿ naka\\
\gla kaku kabe-mÿnÿ naka\\
\glb exist dog-\textsc{dim} here\\
\glft ‘here’s a little dog’
\endgl
\trailingcitation{[mox-a110920l-2.007]}
\xe


\ea\label{ex:FirstDim-2}
\begingl
\glpreamble kuina tatÿkemiumÿnÿ\\
\gla kuina ti-a-tÿkemiu-mÿnÿ\\
\glb \textsc{neg} 3i-\textsc{irr}-be.quiet-\textsc{dim}\\
\glft ‘it (i.e. the little baby) doesn’t calm down’
\endgl
\trailingcitation{[jxx-e120430l-4.04]}
\xe


The diminutive is also marginally used to attenuate a verb’s meaning, which is, of course, another extension from smallness and affection marking. However, it acquires an extra function on verbs, different from the ones encoded when it is attached to nouns. One example is given in (\getref{ex:FirstDim-3}), which could be produced to describe that somebody is eating a small quantity of food, but also that somebody is eating at least a bit of light food again after having been ill, although she might still not be eating properly. In the first scenario, the diminutive refers to the \isi{object} participant, while the second scenario is a case of attenuation.

\ea\label{ex:FirstDim-3}
\begingl
\glpreamble tinikumÿnÿtu\\
\gla ti-niku-mÿnÿ-tu\\
\glb 3i-eat-\textsc{dim}-\textsc{iam}\\
\glft ‘he is already eating a bit’
\endgl
\trailingcitation{[rxx-e181024l.116]}
\xe

The diminutive marker is described in more detail in \sectref{sec:Diminutives}.
\is{diminutive|)}

\subsection{Degree markers}\label{sec:AffClDegree}
\is{degree marker|(}

The category of degree markers subsumes the intensive marker (or \isi{intensifier}) \textit{-yu}, the \isi{limitative} markers \textit{-jiku} and \textit{-yÿchi}, additive\is{additive} \textit{-uku}, and the emphatic markers\is{emphatic} \textit{-ja} and \textit{-kene}. They occur in different slots of the verb, some closer, some more remote from the stem.\is{verbal stem}

The intensifier\is{intensifier|(} \textit{-yu} can be translated with ‘very’ or ‘a lot’ most of the time. It mainly occurs on predicates, which may be verbs, adjectives,\is{adjective} quantifiers\is{quantifier} or nouns.\is{nominal predicate} Thus, while it is not selective to word class, it is selective to syntactic function. One example with a verbal predicate is given in (\getref{ex:FirstINTS-1}) and one with an adjective in (\getref{ex:FirstINTS-2}).

\ea\label{ex:FirstINTS-1}
\begingl
\glpreamble tinikunubeyu\\
\gla ti-niku-nube-yu\\
\glb 3i-eat-\textsc{pl}-\textsc{ints}\\
\glft ‘they ate a lot’
\endgl
\trailingcitation{[jxx-e190210s-01]}
\xe

\ea\label{ex:FirstINTS-2}
\begingl
\glpreamble michanikiyu yÿtÿuku\\
\gla michaniki-yu yÿtÿuku\\
\glb delicious-\textsc{ints} food\\
\glft ‘the food is very delicious’
\endgl
\trailingcitation{[jxx-p120430l-2.035]}
\xe


There is one other context in which the intensive marker occurs. This is different from the uses described above, because it does not involve plain predication:\footnote{The nouns involved can possibly be understood as predicates of a headless relative clause,\is{relative relation} i.e. (\getref{ex:FirstINTS-5}) translates as ‘how are you, who are my dear brothers and sisters’.} the intensive marker can also attach to kinship terms to produce a polite or affective address form, as in (\getref{ex:FirstINTS-5}). 

\ea\label{ex:FirstINTS-5}
\begingl
\glpreamble ¿michae? nipijiyue netinejiyue\\
\gla micha-e ni-piji-yu-e nÿ-etine-ji-yu-e\\
\glb good-2\textsc{pl} 1\textsc{sg}-sibling-\textsc{ints}-2\textsc{pl} 1\textsc{sg}-sister-\textsc{col}-\textsc{ints}-2\textsc{pl}\\
\glft ‘how are you, my dear brothers and sisters?’
\endgl
\trailingcitation{[mxx-x110916]}
\xe

\is{intensifier|)}
% tejujumiyubi = está con mucha pena de nosotros, rmx-c121126s.39

The two \isi{limitative} markers are \textit{-jiku} ‘\textsc{lim}1’ and \textit{-yÿchi} ‘\textsc{lim}2’. They indicate exhaustive \isi{focus}, i.e. they translate as ‘only, just’. According to Paunaka speakers, there is no difference in meaning, and both markers can have wide and narrow scope (for examples see \sectref{sec:Limitatives}). The \isi{additive} marker \textit{-uku} is used to express additive focus and can be translated as ‘also, too’. Like English adverbs, these focus markers are relatively free in selecting a word, depending on \isi{focus}, not on word class. They seem to attach to different slots in the active verb; however, it was not possible to determine their exact place due to lack of sufficient examples in which other markers co-occur. It seems to be the case that one of them precedes the object and \isi{plural} markers and the other one follows them (compare (\getref{ex:FirstLim-1}) and (\getref{ex:FirstLim-2})). Actually, these are the only two examples in the corpus in which the \isi{limitative} markers combine with plural or object markers on a verb.

\ea\label{ex:FirstLim-1}
\begingl
\glpreamble echÿu tichupuikujikunÿ\\
\gla echÿu ti-chupuiku-jiku-nÿ\\
\glb \textsc{dem}b 3-know-\textsc{lim}1-1\textsc{sg}\\
\glft ‘she knew only me’
\endgl
\trailingcitation{[ump-p110815sf.011]}
\xe

\ea\label{ex:FirstLim-2}
\begingl
\glpreamble teukukujinubeyÿchi\\
\gla ti-eu-kuku-ji-nube-yÿchi\\
\glb 3i-hit-\textsc{rcpc}-\textsc{col}-\textsc{pl}-\textsc{lim}2\\
\glft ‘they are only fighting with each other’
\endgl
\trailingcitation{[jxx-e120516l-1.089]}
\xe

Two examples of the limitative markers on words other than verbs follow; (\getref{ex:FirstLim3}) shows the use of \textit{-jiku} with a \isi{preposition}, (\getref{ex:FirstLim4}) the use of \textit{-yÿchi} with a noun.

\ea\label{ex:FirstLim3}
\begingl
\glpreamble chitÿpijikunube\\
\gla chi-tÿpi-jiku-nube\\
\glb 3-\textsc{obl}-\textsc{lim}1-\textsc{pl}\\
\glft ‘it was only for them’
\endgl
\trailingcitation{[mxx-p181027l-1.028]}
\xe

\ea\label{ex:FirstLim4}
\begingl
\glpreamble maneyÿchi biyunu asaneti\\
\gla mane-yÿchi bi-yunu asaneti\\
\glb morning-\textsc{lim}2 1\textsc{pl}-go field\\
\glft ‘we just went to the field in the mornings’
\endgl
\trailingcitation{[rxx-p181101l-2.147]}
\xe

The additive marker\is{additive|(} most often goes on the predicate, but it can also attach to other constituents of a clause, as in (\getref{ex:FirstAdd-1}).

\ea\label{ex:FirstAdd-1}
\begingl
\glpreamble nÿtiuku kuina nichupa\\
\gla nÿti-uku kuina ni-chupa\\
\glb 1\textsc{sg.prn}-\textsc{add} \textsc{neg} 1\textsc{sg}-know.\textsc{irr}\\
\glft ‘me too, I don’t know it’ (i.e. ‘I don’t know it either’)
\endgl
\trailingcitation{[cux-c120414ls-2.238]}
\xe

On verbs, the additive marker occurs directly after the stem\is{verbal stem} of an active verb. If the verb stem ends in a \isi{thematic suffix}, the additive marker even inflects for RS.\is{reality status} When the verb stem\is{verbal stem} does not end in a thematic suffix, RS is marked on the stem and the additive follows directly. Compare the two irrealis verbs in (\getfullref{ex:FirstAdd-2}); the verb in (\getref{ex:FirstAdd-2.1}) has a thematic suffix, the one in (\getref{ex:FirstAdd-2.2}) has none.

\ea\label{ex:FirstAdd-2}
  \ea\label{ex:FirstAdd-2.1}
\begingl
\glpreamble ninikuka\\
\gla ni-niku-uka\\
\glb 1\textsc{sg}-eat-\textsc{add.irr}\\
\glft ‘I will eat, too’
\endgl
  \ex\label{ex:FirstAdd-2.2}
\begingl
\glpreamble niyunauku\\
\gla ni-yuna-uku\\
\glb 1\textsc{sg}-go.\textsc{irr}-\textsc{add}\\
\glft ‘I will go, too’
\endgl
\z
\xe

The additive also occurs prior to inflectional suffixes like the \isi{middle voice} marker \textit{-bu}, as in (\getref{ex:FirstAdd-3}).

\ea\label{ex:FirstAdd-3}
\begingl
\glpreamble pikubiakukubu\\
\gla pi-kubiaku-uku-bu\\
\glb 2\textsc{sg}-be.tired-\textsc{add}-\textsc{mid}\\
\glft ‘you are tired, too’
\endgl
\trailingcitation{[cux-c120414ls-2.329]}
\xe\is{additive|)} 

There are two emphatic markers.\is{emphatic|(} One of them, \textit{-kene} ‘\textsc{emph}2’, occurs very sporadically and will thus not be considered further here. The other one has the form \textit{-ja} with the gloss ‘\textsc{emph}1’. It can attach to words of different classes and exhibit wide or narrow scope in the clause. One example is given in (\getref{ex:FirstEmph-1}), where the emphatic marker attaches to the free form of the uncertainty marker. The translation of the question as referring to the manner of an event is due to the context.


\ea\label{ex:FirstEmph-1}
\begingl
\glpreamble ¿kenaja? kuina nimua\\
\gla kena-ja kuina ni-imua\\
\glb \textsc{uncert}-\textsc{emph}1 \textsc{neg} 1\textsc{sg}-see.\textsc{irr}\\
\glft ‘how might it have happened? I have not seen it’
\endgl
\trailingcitation{[rxx-e181021les]}
\xe

\is{emphatic|)}
\is{degree marker|)}
\is{verb|)}

More information about the degree markers is given in \sectref{sec:MiscellaneousMarkers}.


\subsection{Non-verbal irrealis marker}\label{sec:AffCl_nvirr}
\is{non-verbal irrealis marker|(}

The non-verbal irrealis marker \textit{-ina} is selective in a negative way: it is never found on verbs. In \isi{non-verbal predication}, it can attach to a range of words from different classes, e.g. the \isi{copula} \textit{kaku} as in (\getref{ex:new23-kaku}), verbs borrowed\is{borrowing} from Spanish that are integrated as non-verbs as in (\getref{ex:new23-forget}) (and see \sectref{sec:borrowed_verbs} on the topic of verbs borrowed from Spanish), and adjectives\is{adjective} as in (\getref{ex:new23-weather}), the latter being rare.

\ea\label{ex:new23-kaku}
\begingl
\glpreamble kuina kakuina Krara\\
\gla kuina kaku-ina Krara\\
\glb \textsc{neg} exist-\textsc{irr.nv} Clara\\
\glft ‘Clara is not here’
\endgl
\trailingcitation{[cux-c120510l-1.199]}
\xe

\ea\label{ex:new23-forget}
\begingl
\glpreamble eka nijinepÿi kuina arbidauna\\
\gla eka ni-jinepÿi kuina arbidau-ina\\
\glb \textsc{dem}a 1\textsc{sg}-daughter \textsc{neg} forget-\textsc{irr.nv}\\
\glft ‘my daughter doesn’t forget [me]’
\endgl
\trailingcitation{[jxx-p110923l-1.215]}
\xe

\ea\label{ex:new23-weather}
\begingl
\glpreamble michamuenatu\\
\gla michamue-ina-tu\\
\glb of.good.weather-\textsc{irr.nv}-\textsc{iam}\\
\glft ‘the sky will be nice again now‘ (i.e. ‘it won’t rain anymore’)
\endgl
\trailingcitation{[jxx-p120515l-2.269]}
\xe

As for its occurrence on nouns,\is{noun} there is a predicative and a referential use of the non-verbal irrealis marker. It is used predicatively if the noun is the predicate of the clause. This is similar to the use of an \isi{adjective} as predicate as in  (\getref{ex:new23-weather}). Nonetheless, if a verbal predicate is present, the non-verbal irrealis marker signals non-existence of an \isi{argument} or adverbial of the clause independent of predication. This is explained in detail in \sectref{NominalRS}.

\is{non-verbal irrealis marker|)}
\is{inflection|)}
\is{clitic|)} 
\is{transcategorial morphology|)}

The following sections are dedicated to some affixes and mechanisms found in word formation.


\section{Classifiers}\label{sec:Classifiers}\is{classifier|(}

Classifiers constitute a special class of lexical suffix\is{suffix} in Paunaka. Most of them have CV shape. They combine with stems of nouns, adjectives, and verbs (see \sectref{sec:Nouns_CLF} for combinations of classifiers with nouns, \sectref{sec:Adjectives} for adjectives and \sectref{sec:StativeVerbs_CLF} and \sectref{sec:CLF_ActiveVerbs} for verbs).\footnote{A lot of ink has been spilled in the attempt to grasp the function and meaning of classifiers in general, but the type of classifier system found in the Bolivian Arawakan\is{Southern Arawakan} languages is not convincingly covered by these approaches. A detailed discussion of this issue for the Baure system can be found in \citet[174--177]{Terhart2016}. \citet{Rose2019b} and \citet{RoseVanlinden2022} offer a description of different functions of Mojeño Trinitario. The Paunaka system is in tendency similar, but less open and less productive than the Baure and especially the Trinitario system.} Classifiers do not occur with numerals in Paunaka, which may be important to mention explicitly, because this is the only word class with which classifiers obligatorily occur in the related \isi{Baure} and the \isi{Mojeño languages}, see more on this below.

Since classifiers do not show up very often in Paunaka, except for nouns that are lexicalised with a classifier, it is not entirely clear how many classifiers there are. The ones I could identify are summarised in \tabref{table:CLFs}. Most of these classifiers have a semantic basis in shape, which is typical for classifiers in general (\citealp[cf.][300--301]{Allan1977}; \citealt[273]{Aikhenvald2003}). 



\begin{table}[htbp]
\caption{Classifiers}
\label{table:CLFs}
\small
\begin{tabularx}{\textwidth}{llQQ}
\lsptoprule
Form & Gloss & Examples & Description \cr
\midrule
\textit{-be} & \textsc{cfl:}pointed & \textit{-mube} ‘comb’, \textit{kusaube} ‘hook’, \textit{esebe} ‘thorn, sting’& pointed objects \cr%\textit{tÿmuabe} ‘needle’ 
\tablevspace
\textit{-e} & \textsc{clf:}water & \textit{-bÿtuekubu} ‘fall into water’ & only found with active verbs where it always refers to obliques\cr
\tablevspace
\textit{-i} & \textsc{clf:}fruit & \textit{bÿrÿsÿi} ‘guava’, \textit{yÿkÿi} ‘pot’ & also possessed noun (\textit{chÿi} ‘fruit’), possibly also used for round containers \cr
\tablevspace
\textit{-ji} & \textsc{clf:}soft.mass & \textit{muteji} ‘loam, mud’, \textit{kuaji} ‘fishing net’%\textit{tikeji} ‘intestines’ , \textit{mÿuji} ‘clothes’ \textit{sakiji}? ‘grass sp. (Imperata brasiliensis)’ or: collective marker, 
& soft masses, dough \cr
\tablevspace
\textit{-ke} & \textsc{clf:}cylindrical & \textit{amuke} ‘corn’, \textit{kÿike} ‘peanut’ \textit{yÿkÿke} ‘tree, stick, wood’ & cylindrical objects: seeds, sticks, totally lexicalised and not transparent for the speakers; a homophonous affix is sometimes used for places like \textit{anÿke} ‘up’, \textit{jamuike} ‘pampa’\cr
\tablevspace
\textit{-ki} & \textsc{clf:}spherical & \textit{-kakaki} ‘nape’, \textit{-kiyuraki} ‘brain’, \textit{ukabaki} ‘beetle sp.’ & more frequent in incorporation, often with reference to head or insects \cr
\tablevspace
\textit{-kÿ} & \textsc{clf}:bounded & \textit{kimenukÿ} ‘woods’ \textit{kÿpenukÿ} ‘water hole, hollow’ & objects are perceived as a container, when combined with \textsc{clf}\cr
\tablevspace
\textit{-na} & \textsc{clf:}general & \textit{chÿnachÿ} ‘one’, \textit{michana} ‘nice’ & only found with adjectives and the numeral ‘one’\cr
\tablevspace
\textit{-pa} & \textsc{clf:}particle & \textit{yÿbapa} ‘flour’, \textit{mutepa} ‘earth, dust’, \textit{kuyepa} ‘salt’ & dusty things, particles, also steam; a homophonous suffix is found on \textit{-jimunepa} ‘ribs’, \textit{-musipa} ‘eyelashes’ and \textit{tÿmuepa} ‘knife’ \cr %\textit{simapa} ‘ashes’, \textit{ikupa} ‘cloud’
\midrule
\end{tabularx}
\end{table}

\begin{table}\ContinuedFloat
\small
\begin{tabularx}{\textwidth}{llQQ}
\midrule
Form & Gloss & Examples & Description \cr
\midrule
\textit{-pai} & \textsc{clf:}ground & \textit{-bÿtupaiku} ‘fall’, \textit{nekupai} ‘outside, yard’, \textit{-ubiupai} ‘home, own land, village’ & mainly found with verbs, where it usually refers to obliques\cr
\tablevspace
\textit{-pe} & \textsc{clf:}flat & \textit{kÿnupe} ‘fish sp.’, \textit{churupepe} ‘butterfly’ & flat, relatively rigid objects, more frequent in incorporation  \cr %, ??\textit{ijÿupe} ‘spindle’ 
\tablevspace
\textit{-pi} & \textsc{clf:}long.flexible & \textit{jupipi} ‘liana’, \textit{kechuepi} ‘worm’, \textit{kusepi} ‘thread’ & long (two-dimensional) and rather flexible objects \cr %??\textit{kÿjÿpi} ‘manioc’, \textit{kujubipi} ‘liana sp.’
\tablevspace
\textit{-tapu} & \textsc{clf:}scales & \textit{temetapuji} ‘big fish’ & the noun \textit{-tapu} means ‘scales’, but on adjectives it refers to animals with scales or a carapace (fish, tortoise, armadillo)\cr
\tablevspace
\textit{-umu} & \textsc{clf:}liquid & \textit{patabiumu} ‘cane juice’, \textit{ipitiumu} ‘honey’ & liquids \cr
\lspbottomrule
\end{tabularx}


\end{table}



All of these classifiers, except for \textit{-tapu} ‘\textsc{clf:}scales’, have cognate forms in at least one of the \isi{Mojeño languages}, and most of them also in \isi{Baure}. The classifier \textit{-tapu} derives from\is{grammaticalisation} an identical noun \textit{-tapu} ‘scales’; however, when used as a classifier, it refers to animals with scales (or a carapace), not to the scales themselves, as in (\getref{ex:clfclf-1}).

\ea\label{ex:clfclf-1}
\begingl
\glpreamble kaku mutemetapuji\\
\gla kaku muteme-tapu-ji\\
\glb exist big-\textsc{clf:}scales-\textsc{col}\\
\glft ‘there are big ones (piranhas)’
\endgl
\trailingcitation{[cux-c120414ls-2.015]}
\xe

Many objects fall into the same classes in all Bolivian Arawakan\is{Southern Arawakan} languages, but there are also some differences. To give just one example, the classifier for pointed objects is \textit{-be} ‘\textsc{clf:}pointed’ in Paunaka. Trinitario\is{Mojeño Trinitario} has a cognate form \textit{-ve}, but the classifier seems to be absent from Baure and Ignaciano. Instead of this, in both languages (and partly also in Trinitario), pointed objects are classified with \textit{-po}/\textit{-pa}, i.e. a form identical to the classifier for dusty things. In Paunaka, some objects that are possibly conceived as pointed are also formed with final \textit{-pa}. As for \textit{-jimunepa} ‘ribs’ and \textit{-musipa} ‘eyelashes’, there is also a special parallel arrangement of single longish items. This does not seem to hold for \textit{tÿmuepa} ‘knife’.\footnote{One could think of the serrated blade as indicating a similar arrangement, but it is not clear whether the word was originally derived for knives with serrated or smooth blades or both. Other objects with parallel arrangement take a different classifier (\textit{-be}: \textit{-mube} ‘comb’) or no classifier at all (\textit{-kaba} ‘palm leaf’).} Thus we find that most pointed objects are classified by \textit{-be} in Paunaka and by \textit{-pa}/\textit{-po} in Ignaciano\is{Mojeño Ignaciano} and \isi{Baure}, but a few of them are also classified by \textit{-pa} in Paunaka. There seem to be several (temporal) layers of how classifiers come into being and go out of use again, some shared by all languages, some only by a subset of them or restricted to a single language. This results in several possible classes an item can be assigned to and in differences between the related languages.

It has already been mentioned that numerals\is{numeral} do not take classifiers in Paunaka, but suspiciously, the numeral \textit{chÿnachÿ} ‘one’ has a syllable \textit{na}, as does the related adjective \textit{punachÿ} ‘other’. \isi{Baure} and the \isi{Mojeño languages} have a neutral or unspecified classifier, which has the form \textit{-no} in \isi{Baure} and \textit{-na} in Trinitario and Ignaciano \citep[148]{Terhart2016}. This classifier is also used with human referents, and this is why it has been glossed as ‘\textsc{clf:}human’ by \citet[148]{Danielsen2007}. This very same classifier can be used with numerals to replace any other more specific classifier, and indeed, this is very frequently the case in these languages. Thus \citet[]{Rose2019b} speaks of a generic classifier. In Paunaka, there are also a few adjectives\is{adjective|(} and stative verbs\is{stative verb|(} that have a sequence \textit{na} in their most neutral form, and some of them change exactly this syllable for a specific classifier or a noun denoting a body or plant part. This is why I postulate that there is a default or general classifier \textit{-na} in Paunaka, cognate to the forms found in the other Bolivian Arawakan\is{Southern Arawakan} languages.
 The words containing the general classifier are summarised in \tabref{table:na-CLF}.

\begin{table}[htbp]
\caption[Words containing the presumed classifier \textit{-na}]{Words containing the presumed classifier \textit{-na}}

\begin{tabular}{lll}
\lsptoprule
Word & Gloss & POS \cr
\midrule
\textit{chÿnachÿ} & one & numeral\cr
\textit{punachÿ} & other & \isi{adjective}\cr
\textit{(mu)temena} & big & adjective\cr
\textit{michana} & beautiful & adjective\cr
\textit{kana} & be of this size (showing) & adjective\cr
\textit{-ÿnai} & be long, be tall & stative verb\cr
\textit{-sabana} & be big, be fat & stative verb\cr
\lspbottomrule
\end{tabular}

\label{table:na-CLF}
\end{table}\is{stative verb|)}\is{adjective|)}


Processes related to classifiers include \isi{compounding} and \isi{incorporation} (see \sectref{sec:CompoundingIncorporation}).

%There is also a noun or adverb \textit{nekupai} ‘outside, yard’ which contains the classifier.
%
%\ea\label{ex:clfclf-2}
%\begingl
%\glpreamble tisÿeipai apuke\\
%\gla ti-sÿei-pai apuke\\
%\glb 3i-be.cold-\textsc{clf:}ground ground\\
%\glft ‘the ground is cold’\\
%\endgl
%\trailingcitation{[jxx-e150925l-1.085]}
%\xe
\is{classifier|)}

\section{Repetition and reduplication}\label{sec:Reduplication}
\is{reduplication|(}

Repetition is found on stems of verbs, nouns, and adjectives, but true reduplication is only found on verbs, i.e. “the repetition of morphemes or parts of morphemes by which a new morpheme with a new, related meaning is created, or by which a grammatical function is systematically expressed” \citep[2]{GomezVoort2014}. Reduplication can express iterative, durative, and intensive \isi{aktionsart} on active verbs\is{active verb} (see \sectref{sec:ActiveVerbs_RDPL}) and possibly inchoative \isi{aktionsart} on stative verbs\is{stative verb} (see \sectref{sec:StativeVerbs_RDPL}). These categories, among others, have been identified by \citet[19]{Rubino2005} to be expressed by reduplication cross-linguistically. Consider (\getref{ex:new23-see}), which includes the verb \textit{-imu} ‘see’, and compare with (\getref{ex:new23-look}), which has the related stem \textit{-imumuku} ‘look, watch’, including a reduplicated syllable (+ thematic suffix \textit{-ku}).


\ea\label{ex:new23-see}
\begingl
\glpreamble pero eka nipiji chimu\\
\gla pero eka ni-piji chÿ-imu\\
\glb but \textsc{dem}a 1\textsc{sg}-sibling 3-see\\
\glft ‘but my sister saw it’
\endgl
\trailingcitation{[jxx-p120430l-2.047]}
\xe

\ea\label{ex:new23-look}
\begingl
\glpreamble timumuku echÿu chipeu mase\\
\gla ti-imumuku echÿu chi-peu mase\\
\glb 3i-look \textsc{dem}b 3-animal squirrel\\
\glft ‘he is looking at his squirrel’
\endgl
\trailingcitation{[dxx-d120416s.057]}
\xe


Most nouns with a repeated syllable are totally lexicalised.\is{lexicalisation} If we consider the few stems\is{nominal stem} that have been found with and without repeated syllables, we must note that the ones including repetition also bear the collective\is{collective|(} marker (see \sectref{sec:RDPL_Nouns}). The collective marker also triggers repetition of the general \isi{classifier} on adjectives (see \sectref{sec:Adjectives})). In any case, the meaning conveyed by repetition plus collective\is{collective|)} marking on nouns is distributivity.

There is almost exclusively progressive partial reduplication (and repetition), and it is usually restricted to duplication. However, a few cases of triplication exist in verbs, e.g. \textit{-japipipiku} ‘wag tail’ and \textit{-pÿsisisiku} ‘smoke, smoulder’.

A reduplicated syllable occurs in \isi{continuous} and optionally also in concurrent \isi{associated motion} marking, but is accompanied by additional material. The forms are \textit{-CViku} for \isi{continuous} marking, while the concurrent associated motion markers have the forms \textit{-(CV)kuÿ} and \textit{-(CV)kÿupunu} respectively, with CV standing for the reduplicated syllable. 
Reduplication is not obligatory in this case, thus we can assume that there are two affixes \textit{-kÿu} and \textit{-kÿu(punu)}, which can be accompanied by reduplication of the last syllable of the preceding stem (see \sectref{sec:AMconcurrent}). Reduplication does not add any semantic content to the markers. The cislocative concurrent motion marker \textit{-kÿupunu} is only sometimes accompanied by a reduplicated syllable, but the  translocative \is{associated motion} \textit{-kuÿ} is accompanied by a reduplicated syllable most of the times. Thus at least for the latter, we can possibly speak of “automatic reduplication”, defined as “reduplication that is obligatory in combination with another affix” \citep[18]{Rubino2005}, although reduplication is not obligatory but only highly preferred in this case.

The case of continuous\is{continuous|(} marking is harder to classify. The reduplicated syllable is accompanied by material which presumably consists of two separate markers, \textit{-i} and \textit{-ku/-ka}. Those markers occur also elsewhere in the stem, \textit{-i} as an \is{extension applicative} and \textit{-ku/-ka} as a \isi{thematic suffix}. It is the combination of reduplication with these two markers that expresses the continuous reading as in (\getref{ex:FirstRDPL-1}). For more examples of continuous marking, see \sectref{sec:ActiveVerbs_RDPL}.\is{continuous|)}

\ea\label{ex:FirstRDPL-1}
\begingl
\glpreamble tikusabenunuiku chisabenu\\
\gla ti-kusabenu-nuiku chi-sabenu\\
\glb 3i-play.flute-\textsc{cont} 3-flute\\
\glft ‘he was playing the flute’
\endgl
\trailingcitation{[mox-n110920l.049]}
\xe

\is{reduplication|)}

\section{Compounding, incorporation and derivation}\label{sec:CompoundingIncorporation}
\is{compounding|(}

Compounding and \isi{incorporation} are both minor processes in Paunaka, at least as far as productivity of these processes is concerned. A number of body part terms result from compounding, but productive compounding is largely restricted to inalienably possessed\is{inalienability} nouns denoting plant parts attaching to plant names. One example is given in (\getref{ex:new23-compound}). 

\ea\label{ex:new23-compound}
\begingl
\glpreamble rupinupune\\
\gla rupinu-pune\\
\glb banana.sp-leaf\\
\glft ‘banana leaf’
\endgl
\trailingcitation{[mxx-e120415ls.053]}
\xe
\is{compounding|)}

\is{incorporation|(}
Incorporation is largely restricted to plant- and body-part terms. The latter also belong to the class of inalienably possessed\is{inalienability} nouns at large. These nouns can incorporate into active and stative verb stems\is{verbal stem} and combine with adjectives.\is{adjective}\footnote{The last process may also be classified as a kind of compounding, but I believe that adjectives are closer to verbs than to nouns.} Some stative verbs have lexicalised\is{lexicalisation} with a body-part term, and some of them even do not have forms without the incorporated noun. Ohers do, but have developed idiosyncratic meanings \citep[cf.][]{TerhartDanielsenBODY}. (\getref{ex:new23-incorporation}) illustrates a verb combining with the same noun \textit{-pune} ‘leaf’ as (\getref{ex:new23-compound}) above.

\ea\label{ex:new23-incorporation}
\begingl
\glpreamble bipÿrupune\\
\gla bi-pÿru-pune\\
\glb 1\textsc{pl}-burn-leaf\\
\glft ‘we roasted the leaves’
\endgl
\trailingcitation{[rxx-p181101l-2.223]}
\xe
\is{incorporation|)}

As regards derivation, some processes show a certain productivity and are thus an\-a\-lys\-able. Examples have already been given as (\getref{ex:VerbStems-1}) and (\getref{ex:VerbStems-2}) in \sectref{sec:AffixesAndClitics} above. As far as derivational processes are an\-a\-lys\-able, they are discussed in the individual chapters on nouns, verbs, and other parts of speech. Other processes are totally opaque, often to the degree that they are not recognisable anymore. 


\section{Parts of speech}\label{sec:POS}
\is{word class|(}

This section provides a very short overview of the different parts of speech. More complete descriptions are given in the following chapters. The most important distinction is between verbs as typical predicates and nouns as typical arguments of a clause. All other classes can be considered minor. Pronouns\is{pronoun} and demonstratives\is{demonstrative} are closed classes. All others are open, and they can gain new members by \isi{borrowing}, \isi{grammaticalisation}, \isi{derivation} etc. However, some classes certainly acquire new members more easily than others.

\subsection{Nouns}\label{sec:POS_Nouns}
\is{noun|(}

Nouns typically express “the most time-stable concepts” \citep[33]{Payne1997} in a language, and this is also the case in Paunaka.

Paunaka distinguishes three classes of nouns:\is{possession|(} inalienably possessed, alienably possessed and non-possessable. Inalienably\is{inalienability} possessed nouns obligatorily take a person marker that expresses the possessor.\is{person marking} However, non-possessed nouns can be derived\is{derivation} from some of them. Alienably\is{alienability} possessed nouns do not need a possessor, they can stand on their own. Some alienably possessed nouns can take a possessor marker without any further change, while others need to derive a possessable form first. Non-possessable\is{non-possessability} nouns are never marked for possession directly, nor is there any derivational device that could derive possessable nouns from non-possessables. Some of them can, however, be juxtaposed to a possessable noun to indicate possession.\is{possession|)}

Plural\is{plural} marking is obligatory for human\is{animacy} nouns with a plural marker \textit{-nube} and optional for non-human\is{animacy} nouns that instead take the \isi{distributive} marker \textit{-jane} or \isi{collective} \textit{-ji}. 
Other categories associated with nouns are \isi{diminutive}, \isi{nominal irrealis}, “deceased”\is{deceased marking} marking and possibly nominal past.\is{nominal tense} In the clause, nouns mainly function as arguments.\is{argument} They head NPs,\is{head} where they can be modified\is{modification} by demonstratives,\is{nominal demonstrative} numerals,\is{numeral} and sometimes also by adjectives and quantifiers.\is{quantifier} There is no case marking, but some locative relations are expressed by attaching \textit{-yae} to a noun. A few other \isi{oblique} relations are expressed with the help of prepositions\is{preposition} preceding the noun. 

Nouns can also be used predicatively and then marked for reality status (RS), aspect, tense, modality, and evidentiality.

New nouns can be formed by \isi{compounding}, attachment of classifiers\is{classifier} or \isi{nominalisation} of verbs with the suffix \textit{-kene}, the latter being rare. Nouns can also easily be borrowed\is{borrowing} from Spanish and \isi{Bésiro}. Borrowed nouns inflect\is{inflection} just like Paunaka nouns. Most of them are alienably possessed.\is{alienability}
\is{noun|)}

\subsection{Verbs}\label{sec:POS_Verbs}
\is{verb|(}

Verbs typically express the “least time-stable concepts” \citep[47]{Payne1997}. Members of the word class of verbs are mostly used as predicates.

Verbs obligatorily inflect\is{inflection|(} for person\is{person marking} and RS,\is{reality status} i.e. they never appear as bare stems.\is{verbal stem} The position of \isi{irrealis} marking is decisive for the classification of a verb as either stative or active: on stative verbs,\is{stative verb} realis is unmarked (\getfullref{ex:Stative.1}), while irrealis is marked by a \isi{prefix} \textit{a-} (\getfullref{ex:Stative.2}). Active verbs\is{active verb} have realis RS if none of the suffixes that can mark RS is realised in its irrealis form, i.e. all of these suffixes end in the vowel \textit{u} (\getfullref{ex:Active.1}). Irrealis is indicated by one of these suffixes taking a final vowel \textit{a} instead (\getfullref{ex:Active.2}). 

\ea\label{ex:Stative}
  \ea\label{ex:Stative.1}
\begingl
\glpreamble tikutiu\\
\gla ti-kutiu\\
\glb 3i-be.ill\\
\glft ‘he is ill’
\endgl
%\trailingcitation{[jxx-p110923l-1.042]}
  \ex\label{ex:Stative.2}
\begingl
\glpreamble kuina takutiu\\
\gla kuina ti-a-kutiu\\
\glb \textsc{neg} 3i-\textsc{irr}-be.ill\\
\glft ‘he is not ill’
\endgl
%\trailingcitation{[jxx-p120430l-2.437]}
\z
\xe


\ea\label{ex:Active}
  \ea\label{ex:Active.1}
\begingl
\glpreamble  nimukupunu\\
\gla ni-muku-punu\\
\glb 1\textsc{sg}-sleep-\textsc{am.prior}\\
\glft ‘I go to sleep’
\endgl
%\trailingcitation{[jxx-e141024s-3.3]}
  \ex\label{ex:Active.2}
\begingl
\glpreamble nimukupuna\\
\gla ni-muku-puna\\
\glb 1\textsc{sg}-sleep-\textsc{am.prior.irr}\\
\glft ‘I will go to sleep’
\endgl
%\trailingcitation{[ump-p110815sf.462]}
\z
\xe
\is{inflection|)}

In general, stative verbs\is{stative verb|(} are intransitive. However, there are a few that are stative by their stem and RS placement, but transitive by inflection for person; i.e. they can take object markers. Active verbs\is{active verb|(} can be intransitive, transitive or ditransitive, the latter being rare. The verb stems\is{verbal stem} of many active verbs end in one of the two thematic suffixes \textit{-ku} and \textit{-chu}; stative verb stems do not carry thematic suffixes.

There are two mechanisms that re-arrange the syntactic roles of verbal arguments and only occur with active verbs: causative and benefactive. Middle voice is also restricted to active verbs. There are a number of deponent middle verbs. Semantically, many middle verbs are very similar to stative verbs. Passive\is{passive} voice does not exist. A difference between \isi{transitive} and intransitive verbs (active and stative ones alike) is that one of the two third person markers,\is{person marking} \textit{chÿ-}, can only be combined with transitive verbs, since it encodes 3>3 relations.\is{active verb|)}\is{stative verb|)}

Aspect,\is{inflection|(} tense, reported evidentiality, and modality can optionally be expressed on all verbs. The category of associated motion lies on the edge between inflection and derivation. In addition, the diminutive is also frequently found on verbs. There is no compounding of two or more verb stems, but we find incorporation of nouns, and attachment of classifiers to verb stems.\is{inflection|)}\is{verbal stem}

With regard to borrowing\is{borrowing|(} of verbs, it is often past participles that are borrowed from Spanish, which are then integrated into Paunaka as non-verbal predicates \citep[cf.][]{Terhart_subm}.\is{non-verbal predication|(} This is somewhat problematic, since these words do not neatly fit into any of the word classes postulated for Paunaka. They are definitely not verbs, because they do not inflect\is{inflection} like verbs. They are not nouns either. Their main function is predication and they are thus hardly ever used as arguments.\is{argument}\footnote{One exception is \textit{trabaku} ‘work’, which can be used as a predicate or an argument.} I often simply speak of non-verbal predicates when referring to these words, but the term obviously has shortcomings, since it can also refer to any other type of predicate that is not a verb. It is also possible to verbalise these participles (or other verbal stems from Spanish) by attaching the \isi{thematic suffix} \textit{-chu} and inflect them like regular Paunaka verbs.\footnote{Note that both \isi{Baure} and Trinitario\is{Mojeño Trinitario} integrate Spanish verbs into their verbal systems by cognate suffixes.}\is{non-verbal predication|)}\is{borrowing|)}

Verbs can also figure as arguments\is{argument} without any derivation. All they need is to be accompanied by a nominal demonstrative. Such a construction can be analysed as a headless relative clause.\is{relative relation}
\is{verb|)}

\subsection{Other parts of speech}\label{sec:POS_Others}

There are personal pronouns\is{personal pronoun} for first and second person singular and plural, but not for the third person. To refer to a third person pronominally, a nominal demonstrative\is{nominal demonstrative|(} or \isi{topic pronoun} can be used.\footnote{Topic pronouns are used among other things to topicalise third person referents; they are described in more detail in \sectref{sec:FocPron}.} There is also a topic pronoun for inanimate obliques.\is{oblique} Three nominal demonstratives can be distinguished. One of them, \textit{nechÿu}, is exclusively used in locative contexts and could also be defined as an adverb, since it has nominal and adverbial demonstrative functions.\is{nominal demonstrative|)}

There are only a few adjectives,\is{adjective|(} since most properties and qualities that may be encoded by adjectives in other languages are expressed by stative verbs in Paunaka. The adjective \textit{micha} ‘good’ and its derivations show up quite frequently in the corpus, as do \textit{(mu)temena} ‘big’ and the \isi{demonstrative adjective} \textit{kana} ‘this size’, which is accompanied by a gesture. All other adjectives are used very rarely.\is{adjective|)}

All numerals\is{numeral|(} except for \textit{chÿnachÿ} ‘one’ are borrowed\is{borrowing} from Spanish. The numerals \textit{ruschÿ} ‘two’ and \textit{treschÿ} ‘three’ carry a third person marker\is{person marking} \textit{-chÿ}, which is relatively strongly fixed on the numeral, just as with \textit{chÿnachÿ}. Numerals higher than three can also attach \textit{-chÿ}, but the higher the number, the less likely it is that the person marker appears.\is{numeral|)} Some of the quantifiers\is{quantifier} are borrowed from \isi{Bésiro}. They hardly ever modify nouns, but are rather used predicatively or modify a verb.

Among the adverbs\is{adverb} are words that express spatial, temporal, aspectual\is{temporal/aspectual} and modal\is{modal} relations. 

There are a number of connectives,\is{connective} some borrowed\is{borrowing} from Spanish and others of presumable Paunaka origin \citep[cf.][]{DanielsenTerhart2015}. Regarding their word class, some of them might be defined as particles and others as adverbs, but they are described together in one section due to their common function.

Four prepositions\is{preposition} could be identified. They may possibly derive\is{derivation} from verbs,\is{verb} two of them being highly grammaticalised\is{grammaticalisation} and two less so. There is one \isi{general oblique} preposition \textit{(-)tÿpi}; the others are used with \isi{source}, accompaniment\is{comitative} and instrument or cause\is{instrument/cause} expressions.
\is{word class|)}

In the following chapter, minor parts of speech are presented in more detail.


