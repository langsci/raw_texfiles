%!TEX root = 3-P_Masterdokument.tex
%!TEX encoding = UTF-8 Unicode

\section{Connectives}\label{sec:Conjunctions}\is{connective|(}

In the grammaticography of standard average European (SAE) languages, words with a linking function are often divided into the class of conjunctions (or particles) and the class of adverbs.\is{adverb} A number of words have a clause linking function\is{complex sentence|(} in Paunaka, but as for their part of speech classification,\is{word class} this is less clear than it might be in some SAE languages, and this is why they are called “connectives” in this grammar. They are all discussed together in this section. As for \textit{tÿpi} ‘\textsc{obl}’ (or: ‘for’), this is a \isi{preposition} rather than an adverb or a conjunction, but it still has a linking function and is thus included in this description. \tabref{table:Connectives} lists the connective words of Paunaka. At least one example of the use of each of these connectives is given in this section.%, for more examples the reader is referred to Chapter \ref{sec:ComplexClauses} on complex clauses.

\begin{table}[htbp]
\caption{Connective words}
\small
\begin{tabularx}{\textwidth}{p{2cm}lQQ}
\lsptoprule
Word & Translation & Class & Comment \cr
\midrule
\textit{che(je)puine} & because & cause & \cr
\textit{chijikiu} & however, in contrast & adversative & \cr
\textit{depue}\slash \textit{repue} & afterwards, then & temporal & loan from Spanish \cr
\textit{entonses} & so, thus, then & consecutive, temporal & loan from Spanish \cr
\textit{i} & and & conjunctive & loan from Spanish \cr
\textit{kue} & if & conditional &\cr
\textit{masa} & lest & apprehensional & possibly loan from Spanish\cr
\textit{nechikue}\slash \textit{nechukue} & therefore, thus & consecutive & \cr %might be nechÿkue!
\textit{o} & or & disjunctive & loan from Spanish\cr
\textit{pero} & but & adversative & loan from Spanish\cr
\textit{porke} & because & cause & loan from Spanish\cr
\textit{te} & then (‘\textsc{seq}’) & sequential & \cr
\textit{tÿpi} & for (‘\textsc{obl}’) & purpose &\cr
\lspbottomrule
 \end{tabularx}

\label{table:Connectives}
\end{table}

Many connectives have been borrowed\is{borrowing|(} from Spanish: the three typical SAE coordinating conjunctions \textit{pero} ‘but’, \textit{o} ‘or’ and \textit{i} ‘and’ can be found in Paunaka,\is{coordination} but also the subordinating \isi{consecutive}/temporal \textit{entonses} ‘so, thus, then’, temporal \textit{depue} ‘afterwards, then’ and causal\is{cause} \textit{porke} ‘because’.\is{subordination} This is not surprising given that all connectives are prone to being borrowed, since they fulfil an important role in managing the processing of discourse and speaker-addressee interaction in general, which leads to less attention towards the actual form used and finally to long-term borrowing \citep[194]{Matras2009}. As for \textit{masa} ‘lest’, this connective could also be of Spanish origin, deriving from \textit{más} ‘more’, which is sometimes also used as an \isi{adversative} connective ‘but’. \textit{Masa} is used exclusively in \isi{apprehensional} clauses (and together with the frustrative marker in warnings\is{admonitive}) today, but it seems to be the case that it was used in \isi{adversative} coordination in former times (see \sectref{sec:AdversativeCoordination}). However, \textit{masa} could also be related to \isi{Mojeño Ignaciano} \textit{machu} ‘caution, beware of...’ (Rose 2021, p.c.).\is{borrowing|)}

Most connectives do not only link clauses, but also connect a piece of information to the previous discourse, i.e. they occur at the beginning of intonationally\is{intonation} independent clauses, sometimes after a pause \citep[140--141]{DanielsenTerhart2015}.

\newpage
The connective \textit{nechikue} (or less frequently \textit{nechukue}) is possibly related to the demonstrative \textit{nechÿu} ‘\textsc{dem}c’ (see \sectref{sec:DemPron}). It presents an event as a consequence\is{consecutive} of another, preceding event. One example is given in (\ref{ex:nechikue-1}), where Miguel speaks about how the family moved from \isi{Altavista} to Santa Rita (not directly, but moving to several other villages and settlements elsewhere, before settling down there):

\ea\label{ex:nechikue-1}
\begingl
\glpreamble nauku echÿu ÿne sepitÿjiku kuina chitupuna bitÿpi, nechukue biyunuku naka \\Naranjito\\
\gla nauku echÿu ÿne sepitÿ-jiku kuina chi-tupuna bi-tÿpi nechukue bi-yunuku naka Naranjito\\
\glb there \textsc{dem}b water small-\textsc{lim}1 \textsc{neg} 3-reach.\textsc{irr} 1\textsc{pl}-\textsc{obl} therefore 1\textsc{pl}-go.on here Naranjito\\
\glft ‘there was little water, it wasn’t enough for us, therefore we went on to Naranjito’
\endgl
\trailingcitation{[mxx-p110825l.062-064]}
\xe

% bichÿnumitu. nechikue bijechÿpunu naka, rxx-e120511l.170-171
% we were sad (when our father died), therefore we moved here

There is another \isi{consecutive} connective, \textit{entonses}, a loan\is{borrowing} from Spanish \textit{entonces} ‘so, thus, then’. Its \isi{consecutive} force is weaker than that of \textit{nechikue}, but stronger than that of \isi{sequential} \textit{te} ‘then’. It introduces events that happen subsequently to previously mentioned events. However, these events are not only connected temporally, but the second one is also a consequence\is{consecutive} from the other. On the other hand, the event introduced by \textit{nechikue} is most often temporally subsequent to the priorly mentioned event, but not necessarily so \citep[142]{DanielsenTerhart2015}. In (\ref{ex:entonces}), Miguel uses \textit{entonses}. He has just explained that they put four baking trays with rice bread into the oven in total, and he goes on to tell us that they do not fit in altogether, but two or three at a time.

\ea\label{ex:entonces}
\begingl
\glpreamble  pero ruschÿ banaiu entonses banaukupunuku punachÿ ruschÿ o treschÿ\\ purtukupunuku\\
\gla pero ruschÿ bi-ana-i-u entonses bi-anau-uku-punuku punachÿ ruschÿ o treschÿ purtuku-punuku\\
\glb but two 1\textsc{pl}-make-\textsc{subord}-\textsc{real} thus 1\textsc{pl}-make-\textsc{add}-\textsc{reg} other two or three put.in-\textsc{reg}\\
\glft ‘but having made two, then we made another two or three and put them in again’
\endgl
\trailingcitation{[mxx-e120415ls.097]}
\xe

The connective \textit{te} marks sequence. It can attach to the end of the first clause or to the beginning of the second as signalled by \isi{intonation}: there may be a pause preceding or following it. (\ref{ex:good-house}) stems from an account by Juana about her daughter who had fallen down badly and had to stay in hospital, until:

\ea\label{ex:good-house}
\begingl
\glpreamble metu michaupupunutu te tiyunupunutu chubiuyae\\
\gla metu micha-upupunu-tu te ti-yunupunu-tu chÿ-ubiu-yae\\
\glb already good-\textsc{reg}-\textsc{iam} \textsc{seq} 3i-go.back-\textsc{iam} 3-house-\textsc{loc}\\
\glft ‘once she had recovered, then she could go back home’
\endgl
\trailingcitation{[jxx-p110923l-1.477]}
\xe

The borrowed\is{borrowing} adverb \textit{depue} (from Span. \textit{después} ‘after’)\footnote{It is also pronounced \textit{despue}, \textit{depues}, \textit{despues}, \textit{re(s)pue(s)} or \textit{te(s)pue(s)}. The /s/ is mostly realised as [h] in coda position in Lowland Bolivian Spanish and may even delete completely in word-final position \citep[35]{Mendoza2015} and there are different degrees of complete deletion of it in the Paunaka word. The change /d/ → /ɾ/ is regular in integration of Spanish loans, while /t/ is rather unexpected, but this variant is also the least frequent one.} is a connective insofar as it always refers to something that has been mentioned before. It usually occurs at the beginning of a clause, as in (\ref{ex:depue}), but is also often combined with the connective \textit{i} ‘and’.

\ea\label{ex:depue}
\begingl
\glpreamble naukubane bubiu depue bijechikumÿnÿ naka\\
\gla nauku-bane bi-ubiu depue bi-jechiku-mÿnÿ naka\\
\glb there-\textsc{rem} 1\textsc{pl}-house afterwards 1\textsc{pl}-move-\textsc{dim} here\\
\glft ‘there was our house before, then we moved here’
\endgl
\trailingcitation{[rxx-e120511l.168]}
\xe

In the following example, Juana describes how her in-law had escaped from the police, which was trying to arrest him. The episode of his escape ends by him hiding in the woods. Then a new episode in the life of this man begins with his moving to San Ignacio. This beginning of a new episode is expressed by \textit{i depue}.

\ea\label{ex:depue-2}
\begingl
\glpreamble ... max nauku kimenukÿ tiyunu. i depue tiyunu San Inacio\\
\gla max nauku kimenu-kÿ ti-yunu i depue ti-yunu {San Inacio}\\
\glb more there woods-\textsc{clf:}bounded 3i-go and afterwards 3i-go {San Ignacio}\\
\glft '... deeper into the woods he went. And after that he went to San Ignacio’
\endgl
\trailingcitation{[jxx-p120430l-2.056-058]}
\xe

%\ea\label{ex:}
%\begingl
%\glpreamble i kapunuji echÿu kechue nechÿu i despues tiberiukubuji tiyunuji\\
%\gla i kapunu-ji echÿu kechue nechÿu i despues ti-beriuku-bu-ji ti-yunu-ji\\
%\glb and come-\textsc{rprt} snake \textsc{dem}c and after.that 3i-turn-\textsc{mid}-\textsc{rprt} 3i-go-\textsc{rprt}\\
%\glft ‘and there came the snake, it is said, and then it turned and went away, it is said’\\
%\endgl
%\trailingcitation{[jxx-p120515l-2.169]}
%\xe

In conditional clauses,\is{temporal overlap/condition|(} as well as in temporal clauses, the connective \textit{kue} ‘if, when’ is often found to introduce the antecedent (or: protasis), the clause encoding the condition, as can be seen in (\ref{ex:kue-1}), where Juana talks about her hair care with palm fruit oil.

\ea\label{ex:kue-1}
\begingl
\glpreamble pero kue netuka kÿsi eka kuyae tejekupubu\\
\gla pero kue nÿ-etuka kÿsi eka kuyae ti-jekupu-bu\\
\glb but if 1\textsc{sg}-put.\textsc{irr} cusi \textsc{dem}a totaí 3i-lose-\textsc{mid}\\
\glft ‘but if I put \textit{cusi} or \textit{totaí} [oil] (on my hair), it gets lost (i.e. the white colour)’
\endgl
\trailingcitation{[jxx-d181102l.50]}
\xe
\is{temporal overlap/condition|)}

The causal\is{cause|(} connective \textit{che(je)puine} is used by Miguel and María S., but not by Juana. Miguel uses \textit{chejepuine}, María S. the shorter form \textit{chepuine}. The connective signals that an event is seen as a reason or cause for another event. In (\ref{ex:chejepuine-1}), María S. explains why she does not remember much of her living in \isi{Altavista}.

\ea\label{ex:chejepuine-1}
\begingl
\glpreamble pero kuina nichupa micha chepuine sepitÿkuÿnÿ nibÿsÿu tukiu nauku\\
\gla pero kuina ni-chupa micha chepuine sepitÿ-kuÿ-nÿ ni-bÿsÿu tukiu nauku\\
\glb but \textsc{neg} 1\textsc{sg}-know.\textsc{irr} good because small-\textsc{incmp}-1\textsc{sg} 1\textsc{sg}-come from there\\
\glft ‘but I don’t remember it (i.e. living in Altavista) well, because I was still a child when I came from there’
\endgl
\trailingcitation{[rxx-p181101l-2.005]}
\xe

Instead of \textit{che(je)puine}, Juana makes use of a loan\is{borrowing} from Spanish: \textit{porke} from \textit{porque} ‘because’. This connective is also found with other speakers. In (\ref{ex:porque-1:0505}), Juana reports what the owner of the house where she lived with her daughter’s family had told her daughter.

\ea\label{ex:porque-1:0505}
\begingl
\glpreamble “¡esemaika juchubu ejecheka! porke kopaunatu nubiu”, tikechu\\
\gla e-semaika juchubu e-jecheka porke kopau-ina-tu nÿ-ubiu ti-kechu\\
\glb 2\textsc{pl}-search.\textsc{irr} where 2\textsc{pl}-move.\textsc{irr} because use-\textsc{irr.nv}-\textsc{iam} 1\textsc{sg}-house 3i-say\\
\glft ‘“look for where to move, because I want to use my house for myself!” he said’
\endgl
\trailingcitation{[jxx-p120430l-1.397]}
\xe
\is{cause|)}

If a speaker uses the connective \textit{chijikiu} ‘however’, she establishes a contrast between two events. The connective introduces independent clauses, as in (\ref{ex:chijikiu}) from the story about the fox and the jaguarundi, where the drunken fox is killed by dogs, while the smart jaguarundi has escaped onto a tree, where he is safe. The story was mainly told by Miguel, but this was an intervention by Juana.

\newpage

\ea\label{ex:chijikiu}
\begingl
\glpreamble chikupaku kupisaÿrÿ. chijikiu tisepiu kaku anÿke\\
\gla chi-kupaku kupisaÿrÿ chijikiu tisepiu kaku anÿke\\
\glb 3-kill fox however jaguarundi exist up\\
\glft ‘they killed the fox. However, the jaguarundi was up (in the tree)’
\endgl
\trailingcitation{[jmx-n120429ls-x5.443-445]}
\xe

The other \isi{adversative} connective is \textit{pero} ‘but’, which has been borrowed\is{borrowing} from Spanish. Like \textit{chijikiu}, it can introduce independent clauses, but it can also link two clauses to form a complex sentence, which is the case in (\ref{ex:stay-nowater}). This example comes from Miguel telling me about the beginning of settlement in Santa Rita.

\ea\label{ex:stay-nowater}
\begingl
\glpreamble nebutu naka bipajÿkiu pero kuinauku eka ÿneina bitÿpi\\
\gla nebu-tu naka bi-pajÿk-i-u pero kuina-uku eka ÿne-ina bi-tÿpi\\
\glb 3\textsc{obl.top.prn}-\textsc{iam} here 1\textsc{pl}-stay-\textsc{subord}-\textsc{real} but \textsc{neg}-\textsc{add} \textsc{dem}a water-\textsc{irr.nv} 1\textsc{pl}-\textsc{obl}\\
\glft ‘from that point on we stayed here, but there was no water for us either’
\endgl
\trailingcitation{[mxx-p110825l.060]}
\xe

The connective \textit{masa} ‘lest’ is used to form apprehensional\is{apprehensional|(} clauses. (\ref{ex:lestripe}) is given here to exemplify this. It comes from the same description as (\ref{ex:kue-1}) above and provides the answer to my question why Juana puts \textit{totaí} oil on her head.


\ea\label{ex:lestripe}
\begingl
\glpreamble ja tÿpi eka betuka bichÿtiyae masa eka tayutu eka bimukiji; masa takipÿpa\\
\gla ja tÿpi eka bi-etuka bi-chÿti-yae masa eka ti-a-yu-tu eka bi-muki-ji masa ti-a-kipÿpa\\
\glb \textsc{afm} \textsc{obl} \textsc{dem}a 1\textsc{pl}-put.\textsc{irr} 1\textsc{pl}-head-\textsc{loc} lest \textsc{dem}a 3i-\textsc{irr}-be.ripe-\textsc{iam} \textsc{dem}a 1\textsc{pl}-hair-\textsc{col} lest 3i-\textsc{irr}-be.white\\
\glft ‘well, for this we put it on our heads, lest our hair gets ripe (i.e. grey); lest it gets white’
\endgl
\trailingcitation{[jxx-d181102l.05-07]}
\xe
\is{apprehensional|)}

%komoraubinatu (...) masa arbiraubina chija echÿu = you will be accommodating (...) lest you forget something, jxx-p120515l-2.276-278

The positive counterpart of \textit{masa} ‘lest’ is \textit{tÿpi}, which is a preposition to mark different kinds of obliques\is{general oblique} (see \sectref{sec:adp-tÿpi}), but can also be used to introduce \isi{purpose} clauses, in which case it can be translated as ‘to, in order to’. (\ref{ex:feed-animals}) was produced by Juana in telling me how she and her sister María S. spent time together speaking Paunaka. Irrealis RS is due to a habitual reading of the sentence.

\newpage
\ea\label{ex:feed-animals}
\begingl
\glpreamble bupupuna ubiaeyae tÿpi chinika takÿra, chinika upuji, chinika ..., tÿpi aumuena\\
\gla bi-upupuna ubiae-yae tÿpi chi-nika takÿra chi-nika upuji chi-nika tÿpi aumue-ina\\
\glb 1\textsc{pl}-bring.back.\textsc{irr} house-\textsc{loc} \textsc{obl} 3-feed.\textsc{irr} chicken 3-feed.\textsc{irr} duck 3-feed.\textsc{irr} \textsc{obl} chicha-\textsc{irr.nv}\\
\glft ‘we brought it (the corn) back to the house (from the field) to feed the chicken, feed the ducks, feed the ..., for chicha’
\endgl
\trailingcitation{[jxx-p120430l-1.055]}
\xe

Finally, there are also \textit{i} ‘and’ and \textit{o} ‘or’, both borrowed\is{borrowing} from Spanish and used in \isi{conjunctive} and \isi{disjunctive} coordination, respectively. Two examples follow.

In (\ref{ex:i-new}), the connective \textit{i} introduces a new intonation unit and thus attaches the clause to the preceding discourse. It is from the story about the two men who meet the devil in the woods as told by Miguel. This is how the disaster begins: one man answers the devil, who anounces his arrival by shouting.

\ea\label{ex:i-new}
\begingl
\glpreamble i chinachÿ echÿu chikompanyerone chijakupu echÿu tiyÿbui\\
\gla i chinachÿ echÿu chi-kompanyero-ne chi-jakupu echÿu ti-yÿbui\\
\glb and one \textsc{dem}b 3-companion-\textsc{possd} 3-receive \textsc{dem}b 3i-shout\\
\glft ‘and one of the companions answered the one who shouted’
\endgl
\trailingcitation{[mxx-n101017s-1.021]}
\xe

(\ref{ex:plantainoryuca}) was produced by María S. to exemplify the use of the word \textit{ubupunu} ‘carry’. This word had been unknown to me, so I asked her what it meant.\footnote{Actually, rather than ‘carry’, which is the meaning I proposed and which was affirmed by María S., this verb might be the very same as the one in (\ref{ex:feed-animals}), a derivation of the verb \textit{-upunu} ‘bring’ with the associated motion marker \textit{-punu} to yield ‘bring back’ (see also \sectref{sec:punu}).}

\ea\label{ex:plantainoryuca}
\begingl
\glpreamble nubupuna tukiu asaneti, nubupuna ubiaeyae, nubupuna merÿ o kÿjÿpi \\
\gla ni-ubupuna tukiu asaneti ni-ubupuna ubiae-yae ni-ubupuna merÿ o kÿjÿpi\\
\glb 1\textsc{sg}-carry.\textsc{irr} from field 1\textsc{sg}-carry.\textsc{irr} house-\textsc{loc} 1\textsc{sg}-carry.\textsc{irr} plantain or manioc\\
\glft ‘I carry it from the field, I carry it home, I carry plantain or manioc’
\endgl
\trailingcitation{[rxx-e181020le]}
\xe

%chepuine bikuti chikeuchi bijibÿkia, rxx-e120511l.384
%depue nipabentecha chiukeuchi niyÿseikia amuke arusu, rxx-e181022le
%tanaubu echÿu chubiu kuina tateane keuchi faltau plata, rxx-e120511l.115


%nijirebÿkeuchi nipikiu = se fruncia mi cara por tener miedo, rxx-e141230s.086


%eka ajumerku tÿpi piyunia nauku unekuyae reunion, mux-c110810l.012

The topic of connectives will be taken up again in Chapter \ref{sec:ComplexClauses}, where different kinds of clause combinations are described in more detail.\is{complex sentence|)}\is{connective|)} At this place, a discussion of the two major word classes follows, starting with nouns in the following chapter.

