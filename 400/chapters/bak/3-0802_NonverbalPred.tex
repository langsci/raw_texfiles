%!TEX root = 3-P_Masterdokument.tex
%!TEX encoding = UTF-8 Unicode

\section{Non-verbal predication}\label{sec:NonVerbalPredication}
\is{declarative clause|(}
\is{non-verbal predication|(}

There are many clauses with non-verbal predicates. They belong to different semantic types of non-verbal predication with partly different construction types. As for the latter we find \isi{juxtaposition} of the predicate and the \isi{subject} NP,\is{noun phrase} usage of the non-verbal \isi{copula} \textit{kaku} and other strategies. The semantic types of non-verbal predication comprise the ones typically found in the literature: equative, proper inclusion,
attributive, location, existential and possessive, to use the terminology by \citet[ch. 6]{Payne1997}.\footnote{\citet[6]{Overall2018} use the term “identification” instead of “equative” and “categorisation” instead of “proper inclusion”. They further deviate from \citet[]{Payne1997} in that they split the attributive type in two subtypes encoding permanent versus temporary property, but this is not of concern for Paunaka. See also \citet[]{Dryer2007} for yet another similar classification.} In addition, some other semantic types are found in Paunaka, which have been called “minor types”: genitive and benefactive,\is{genitive/benefactive clause} quantification, similative \is{similative clause}  \citep[246--249]{Dryer2007},\footnote{The author actually uses the term “simulative”, while “similative” is the one proposed by \citet[]{HaspelmathBuchholz1998}.} and also locomotion \citep[113]{Payne1997}, the latter being restricted to third person cislocative motion.\is{motion predicate} Strikingly, verbs are often borrowed\is{borrowing} from Spanish as non-verbal predicates, too. \tabref{table:OverviewNV-Pred} shows how semantic types and construction types correlate. All semantic types marked by an asterisk can also be expressed by a verbal strategy. 

\begin{table}[htbp] 
\caption{Semantic types and construction types in non-verbal predication}

\begin{tabular}{ll}
\lsptoprule
Semantic type & Construction type\cr
\midrule
equative & juxtaposition\cr
proper inclusion & juxtaposition\cr
attributive & juxtaposition\cr
quantification & juxtaposition\cr
genitive/benefactive & juxtaposition \cr
location* & juxtaposition / copula\cr
existential & copula\cr
possessive*  & copula\cr
similative  & other \cr
(3rd person) locomotion* & other\cr
borrowed verbs* & other\cr
\lspbottomrule
\end{tabular}

\label{table:OverviewNV-Pred}
\end{table}


Many semantic types in the table are assigned to the juxtaposition construction and a few to the one including a \isi{copula}; however, this is a simplification of the issue. As for \isi{juxtaposition}, a \isi{subject} NP\is{noun phrase} does not necessarily co-occur with the predicate just as in verbal predication. In non-verbal predication this can be related to the subject being topical\is{topic} and/or the predicate taking person indexes. As for the location,\is{locative clause} existential,\is{existential clause} and possessive\is{possessive clause} non-verbal types, they include a copula in positive clauses, but do not need a \isi{copula} in negative clauses\is{negation} where its use is often related to emphasis.

Before describing the different constructions in more detail, I want to provide a short illustration of the properties in which non-verbal predicates differ from verbal ones. Two factors are involved, \isi{reality status} and \isi{person marking}.\is{inflection|(}

\is{non-verbal irrealis marker|(}
Realis is completely unmarked in non-verbal predication, but irrealis is marked. It is triggered by the same parameters that are also relevant for irrealis marking on verbs minus the imperative (see \sectref{RS:parameters}), but there is a specific irrealis marker \textit{-ina} that only occurs on non-verbal words.

Consider (\ref{ex:NV-irr-1}), in which the \isi{nominal predicate} is negated and thus has irrealis RS. The example was produced by María S. and referred to some puppies in her yard which had apparently been brought to Santa Rita although they were still sucklings, thus they were condemned to die.

\ea\label{ex:NV-irr-1}
\begingl
\glpreamble kuina chÿenuina tukiu uneku\\
\gla kuina chÿ-enu-ina tukiu uneku\\
\glb \textsc{neg} 3-mother-\textsc{irr.nv} from town\\
\glft ‘they don’t have a mother, they are from town’
\endgl
\trailingcitation{[rxx-e120511l.363]}
\xe

The non-verbal irrealis marker is the very same morpheme that is also used as a \isi{nominal irrealis} marker (see \sectref{NominalRS}). There is sometimes substantial functional overlap between both, as in (\ref{ex:NV-IRR-cuento}), where the nominal predicate has \isi{future reference}, but could also well be analysed as a nominal future.

The example is taken from a syncretistic creation story told by Juana. Jesus is about to marry María Eva and God tells them:

\ea\label{ex:NV-IRR-cuento}
\begingl 
\glpreamble “eka pimaina i eka piyenuina”\\
\gla eka pi-ima-ina i eka pi-yenu-ina\\ 
\glb \textsc{dem}a 2\textsc{sg}-husband-\textsc{irr.nv} and \textsc{dem}a 2\textsc{sg}-wife-\textsc{irr.nv}\\ 
\glft ‘“this one will be your husband and this one will be your wife”’\\or:‘“this is your future husband and this is your future wife”’\\ 
\endgl
\trailingcitation{[jxx-n101013s-1.368-369]}
\xe

Irrealis marking is applicable to distinguish most cases of verbal and non-verbal predication. However, some words do not inflect for irrealis, e.g. the demonstrative adverbs.\is{non-verbal irrealis marker|)}

The second criterion\is{alignment|(} to distinguish both kinds of predicates is the position of the \isi{subject} indexes.\is{person marking|(}\is{argument|(} It has been shown in \sectref{sec:NumberPersonVerbs} (and also in \sectref{sec:ExpressionSubjects} and \sectref{sec:DeclClausesOBL}) that verbs index subjects with person markers preceding the stem and objects with person markers following the stem. Non-verbal predicates, however, index subjects with person markers following the stem, while the position preceding the stem is retained for possessors. The person markers are identical to the ones used on verbs.\footnote{This type of \isi{split-S marking} dependent on parts of speech\is{word class} has been described in detail for \isi{Baure} in contrast with another Arawakan language\is{Arawakan languages} in which the split depends on other factors \citep[cf.][]{Danielsen_Granadillo2008}.}\is{alignment|)}

The subject indexes on non-verbal predicates are summarised in \tabref{table:NVP_Person}. Compare to \tabref{table:VerbsPerson_all} in \sectref{sec:NumberPersonVerbs}, which summarises the person indexes used with verbs.

\begin{table}[htbp] 
\caption{Subject indexes on non-verbal predicates}

\begin{tabular}{ll}
\lsptoprule
 Person & Index \cr
\midrule
1\textsc{sg} & \textit{-ne / -nÿ} \cr
2\textsc{sg} & \textit{-bi / -pi} \cr
3\textsc{sg} & / \cr
1\textsc{pl} & \textit{-bi} \cr
2\textsc{pl} & \textit{-e} \cr
3\textsc{pl} & (\textit{-nube}) \cr
3\textsc{distr} & (\textit{-jane}) \cr
\lspbottomrule
\end{tabular}

\label{table:NVP_Person}
\end{table}

(\ref{ex:NV-pers-1}) is an example in which a person marker indexes the subject on a \isi{nominal predicate}. It comes from Isidro who was talking about getting old and grey with Miguel and contrasted this to my age.

\ea\label{ex:NV-pers-1}
\begingl
\glpreamble pimiyakuÿbi\\
\gla pimiya-kuÿ-bi\\
\glb girl-\textsc{incmp}-2\textsc{sg}\\
\glft ‘you are still young’
\endgl
\trailingcitation{[mdx-c120416ls.152]}
\xe

%nisimubi, pisimune

Position of the subject index is not always applicable as an indicator for non-verbal predication. The use of third person markers that follow the stem is very restricted on verbs (see \sectref{sec:3_suffixes}), and they do not occur on non-verbal predicates, so that there is no subject index for the third person.\footnote{A \isi{plural} or \isi{distributive} marker can be used, though, if the subject is a third person plural, but it is not always possible to distinguish referential and predicative use, i.e. it is arguable whether \textit{-nube} and \textit{-jane} relate to subject marking or rather to number marking of a noun or nominal demonstrative. Regarding the locomotion predicate\is{motion predicate} \textit{kapunu} ‘come’ and borrowed verbs\is{borrowing} integrated as non-verbal predicates, these markers relate to subjects.} In addition, indexing the subject is sometimes optional, e.g. in equative clauses,\is{equative/proper inclusion clause} and some words generally do not take person markers, e.g. the demonstrative adverbs.
\is{person marking|)}\is{inflection|)}

As for the copula \textit{kaku},\is{copula|(} it only relates to third-person referents. The same is true for the non-verbal locomotion predicate\is{motion predicate} \textit{kapunu} ‘come’. The form of both of them includes a first syllable \textit{ka} which, as I have argued in \sectref{sec:DemPron}, could be the same deictic root that we find in the demonstratives\is{demonstrative} \textit{eka} ‘\textsc{dem}a’ and \textit{naka} ‘here’. This would explain their restriction to the third person. However, as for the copula, there is also a suffix \textit{-uku} used exclusively with personal pronouns\is{personal pronoun} (i.e. with first and second person reference) in non-verbal locative predication, see \sectref{sec:LocativePredicates} below.\is{copula|)} This suffix might well be related to the final syllable \textit{ku} of the copula \textit{kaku}. The expression of locomotion, however, uses a verbal strategy whenever there is an SAP subject.\is{argument|)}

The remainder of this section is structured by semantic and construction type. Equative, proper inclusion\is{equative/proper inclusion clause} and attributive clauses\is{attributive clause} overlap to a large degree and are thus described together in \sectref{sec:PropInclEquatAttr}. Predication of quantification is the topic of \sectref{sec:Quantification}. Genitive and benefactive clauses are discussed in \sectref{sec:GenitiveBenfactivePreds}. All of these semantic types are formed by \isi{juxtaposition} of predicate and \isi{subject}. Predication of location can also be expressed by \isi{juxtaposition} or with the help of a \isi{copula}. Due to restriction of the \isi{copula} to third person referents, there are different strategies for SAP. This is explained in detail in \sectref{sec:LocativePredicates}. The existential construction is described in \sectref{sec:Existentials}, and possessive clauses, which can be considered to belong to the existential construction, in \sectref{sec:PossessiveClauses}. \sectref{sec:SimilativePreds} is about the similative construction and \sectref{sec:Kapunu} describes the use of the non-verbal third person cislocative predicate. Finally, \sectref{sec:borrowed_verbs} shows how verbs from Spanish are integrated into Paunaka as non-verbal predicates.



\subsection{Equative, proper inclusion and attributive}\label{sec:PropInclEquatAttr}
\is{juxtaposition|(}
\is{equative/proper inclusion clause|(}

The two semantic types of equation and proper inclusion usually include nouns\is{noun} that serve as predicates,\is{nominal predicate} the difference among them being that equation expresses that the \isi{subject} of the clause is an entity that is identical to the entity specified by the predicate, while proper inclusion encodes that the \isi{subject} is member of a class which is specified by the predicate \citep[114]{Payne1997}. There are some languages, in which both types are encoded differently, but the languages in South and Central America commonly use the same construction for both types \citep[7]{Overall2018}. Since Paunaka does not obligatorily mark \isi{definiteness} on NPs, some sentences are ambiguous as to the question whether they represent equation or proper inclusion, and one example of this ambiguity is presented in (\ref{ex:ambi-eqprop}) below.

The attributive type\is{attributive clause|(} is bound to adjectives\is{adjective|(} having the role of predicates. However, in Paunaka most property concepts are expressed by stative verbs. Only a few words can felicitously be defined as adjectives. In addition, some properties, especially age, are predicated by nouns\is{noun} (see \sectref{sec:Adjectives}). This is where proper inclusion and attributive predication semantically overlap. In addition, attributive clauses are cross-linguistically often also identical to proper inclusion and equative clauses in structure \citep[120]{Payne1997}, and this is also the case in Paunaka.\is{adjective|)} 

There are three possibilities how to encode all three types: first, the \isi{subject} NP\is{noun phrase} and the predicate are juxtaposed, second, the subject is indexed on the predicate, and third, subject NP and predicate are juxtaposed AND the subject is indexed on the predicate. (\ref{ex:Equative-1}), which stems from elicitation with Miguel, shows all of these possibilities. 
 
\ea\label{ex:Equative-1}
  \ea
 \begingl 
\glpreamble piti nÿa \\
\gla piti nÿ-a\\ 
\glb 2\textsc{sg.prn} 1\textsc{sg}-father\\ 
\glft ‘you are my father’\\ 
\endgl
  \ex
 \begingl 
\glpreamble nÿabi\\
\gla nÿ-a-bi\\ 
\glb 1\textsc{sg}-father-2\textsc{sg}\\ 
\glft ‘you are my father’\\ 
\endgl
  \ex
 \begingl
\glpreamble piti nÿabi\\
\gla piti nÿ-a-bi\\
\glb 2\textsc{sg.prn} 1\textsc{sg}-father-2\textsc{sg}\\
\glft ‘you are my father’
\endgl
\trailingcitation{[mxx-e090728s-3.088-090]}
\z
\xe

If speakers choose the juxtaposition construction,\is{word order|(} the predicate usually precedes the \isi{subject} NP unless the subject is expressed by a \isi{pronoun}. In the latter case, the pronoun precedes the predicate. This is also the preferred word order in Trinitario’s\is{Mojeño Trinitario} non-verbal clauses \citep[75]{Rose2018}, and it mirrors the one found in verbal clauses, where non-emphasised subjects usually follow the verb, but not if they are pronominal (see \sectref{sec:WordOrder}).\is{word order|)} A few examples with the order predicate -- subject NP follow. (\ref{ex:equa-1}) and (\ref{ex:equa-2}) are equative clauses, (\ref{ex:propi-1}) is an example of proper inclusion, and (\ref{ex:attri-1}) and (\ref{ex:attri-2}) have adjectival predicates and are thus attributive clauses.\is{attributive clause|)}

(\ref{ex:equa-1}) comes from Juana who interrupted her speech, when she recognised the wasp close to her.

\ea\label{ex:equa-1}
\begingl
\glpreamble ¡aij jane echÿu!\\
\gla aij jane echÿu\\
\glb \textsc{intj} wasp \textsc{dem}b\\
\glft ‘aiy, this is a wasp!’
\endgl
\trailingcitation{[jxx-p120430l-2.478]}
\xe

(\ref{ex:equa-2}) is a similar example from Miguel, who first thought the bed of the boy in the \isi{frog story} was a church.

\ea\label{ex:equa-2}
\begingl
\glpreamble chubiukena bia eka naka\\
\gla chÿ-ubiu-kena bia eka naka\\
\glb 3-house-\textsc{uncert} God \textsc{dem}a here\\
\glft ‘this one here might be a church’
\endgl
\trailingcitation{[mox-a110920l-2.019]}
\xe

In (\ref{ex:propi-1}), Juana tells me that the water spirit whom their grandparents met on their journey back home from Moxos was a woman, a fact that becomes important a bit later in the story.

\ea\label{ex:propi-1}
\begingl
\glpreamble i seunube echÿu ue\\
\gla i seunube echÿu ue\\
\glb and woman \textsc{dem}b water.spirit\\
\glft ‘and the water spirit was a woman’
\endgl
\trailingcitation{[jxx-p151016l-2.157]}
\xe
\is{equative/proper inclusion clause|)}

\is{attributive clause|(}
(\ref{ex:attri-1}) was produced by Juana as a confirmation of what I had said before. It is about the house of her daughter in Santa Cruz.

\ea\label{ex:attri-1}
\begingl
\glpreamble ja temena ubiyae\\
\gla ja temena ubiyae\\
\glb \textsc{afm} big house\\
\glft ‘yes, the house is big’
\endgl
\trailingcitation{[jxx-p120430l-1.414]}
\xe

(\ref{ex:attri-2}) was a statement by María C. about her favourite drink.

\ea\label{ex:attri-2}
\begingl
\glpreamble michaniki aumue\\
\gla michaniki aumue\\
\glb delicious chicha\\
\glft ‘chicha tastes good’
\endgl
\trailingcitation{[uxx-p110825l.257]}
\xe
\is{attributive clause|)}

\is{equative/proper inclusion clause|(}
If the information given in the sentence is about the name of somebody, we actually find both orders:\is{word order} predicate – subject NP as in (\ref{ex:Equative2}) and subject NP – predicate as in (\ref{ex:name-1}). This may be related to the structure of the corresponding sentence in Spanish. Both examples come from Juana, the first one is about her father, the second about one of her daughters.

\ea\label{ex:Equative2}
\begingl 
\glpreamble Kwachu chija\\
\gla Kwachu chi-ija\\ 
\glb Juan 3-name\\ 
\glft ‘his name was Juan’\\ 
\endgl
\trailingcitation{[jxx-p120515l-1.125]}
\xe

\ea\label{ex:name-1}
\begingl
\glpreamble chija Gladys\\
\gla chi-ija Gladys\\
\glb 3-name Gladys\\
\glft ‘her name is Gladys’
\endgl
\trailingcitation{[jxx-p110923l-2.059]}
\xe

There are a few more examples from Juana in which the order of predicate and subject is reversed, all of them either include proper names or contrastive topics\is{topic} as in (\ref{ex:propi-rev}). Note, however, that irrealis marking is on the subject (\textit{punachina}) in this case although it is related to the predicate (\textit{jente}).

\ea\label{ex:propi-rev}
\begingl
\glpreamble mm rusxhunubechÿ chichechajimÿnÿnube, kana, punachina jente\\
\gla mm rusxhu-nube-chÿ chi-checha-ji-mÿnÿ-nube kana punachÿ-ina jente\\
\glb \textsc{intj} two-\textsc{pl}-3 3-son-\textsc{col}-\textsc{dim}-\textsc{pl} this.size other-\textsc{irr.nv} man\\
\glft ‘mhm, she has two children, [one is] of this size (showing with hands) and the other one will be a man (soon)’
\endgl
\trailingcitation{[jxx-p110923l-1.241]}
\xe

The following examples show that a pronominal subject precedes the non-verbal predicate. (\ref{ex:EqPoss-nube}) was elicited from Miguel.

\ea\label{ex:EqPoss-nube}
\begingl 
\glpreamble nÿti chÿenunube\\
\gla nÿti chÿ-enu-nube\\ 
\glb 1\textsc{sg.prn} 3-mother-\textsc{pl}\\ 
\glft ‘I am their mother’\\ 
\endgl
\trailingcitation{[mxx-e090728s-3.081]}
\xe

(\ref{ex:Prn-eq-1}) comes from María narrating the story of the fox and the jaguar. The fox makes the jaguar believe that the reflection of the moon in the water was a wheel of cheese.

\ea\label{ex:Prn-eq-1}
\begingl
\glpreamble “chibu echÿu kesu”\\
\gla chibu echÿu kesu\\
\glb 3\textsc{top.prn} \textsc{dem}b cheese\\
\glft ‘“this is the cheese”’
\endgl
\trailingcitation{[rxx-n120511l-1.037]}
\xe

The non-verbal predicate can also be a prepositional phrase as in (\ref{ex:EqClause-2}), which is a statement by Clara about her origin.

\ea\label{ex:EqClause-2}
\begingl 
\glpreamble nÿti tukiu nauku Santa Rita\\
\gla nÿti tukiu nauku {Santa Rita}\\ 
\glb 1\textsc{sg.prn} from there {Santa Rita}\\ 
\glft ‘I am from Santa Rita’\\ 
\endgl
\trailingcitation{[cxx-e121130s.011]}
\xe

(\ref{ex:PropIncl-1}) and (\ref{ex:propi-2}) are two additional examples in which juxtaposition is accompanied by subject marking on the predicate. In (\ref{ex:PropIncl-1}), María C. talks about herself.

\ea\label{ex:PropIncl-1}
\begingl 
\glpreamble nÿti juberÿpunÿmÿnÿ\\
\gla nÿti juberÿpu-nÿ-mÿnÿ\\ 
\glb 1\textsc{sg.prn} old.woman-1\textsc{sg}-\textsc{dim}\\ 
\glft ‘poor me, I am an old woman’\\ 
\endgl
\trailingcitation{[uxx-p110825l.038]}
\xe

(\ref{ex:propi-2}) is a confirmation of a sentence I had produced. It comes from María S.

\ea\label{ex:propi-2}
\begingl
\glpreamble biti paunakabi\\
\gla biti paunaka-bi\\
\glb 1\textsc{pl.prn} Paunaka-1\textsc{pl}\\
\glft ‘we are Paunakas’
\endgl
\trailingcitation{[rmx-e150922l.103]}
\xe

There are also some examples in which the person marker alone is used as subject expression.\is{person marking} Since there is no third person marker to index subjects of non-verbal predicates, third person singular subjects can be completely unmarked in non-verbal predication as in (\ref{ex:equa-3}).

(\ref{ex:ambi-eqprop}) comes from Juana who told me about her duty in \isi{Altavista}. Depending on the noun’s \isi{definiteness}, this sentence could be an example of proper inclusion (if it is indefinite) or as equative (if it is definite). NPs are not obligatorily marked for definiteness, so that only context or general knowledge can be used to distinguish both. In this case, I do not know whether Juana was the only cook or one among several (I guess the latter was the case, but this is speculation). 

\ea\label{ex:ambi-eqprop}
\begingl
\glpreamble asta nÿti niyunu, kosinerunÿ\\
\gla asta nÿti ni-yunu kosineru-nÿ\\
\glb until 1\textsc{sg.prn} 1\textsc{sg}-go cook-1\textsc{sg}\\
\glft ‘even I went, I was a/the cook’
\endgl
\trailingcitation{[jxx-p120515l-2.085-086]}
\xe

(\ref{ex:attri-3}) is another example with an adjective serving as predicate. It is the adjective with which one usually answers small-talk questions for one’s health and condition, but in this case María S.’  statement is not about herself but about me (in helping me formulate an adequate answer).

\ea\label{ex:attri-3}
\begingl
\glpreamble michachaikubi\\
\gla micha-chaiku-bi\\
\glb good-\textsc{cont}-2\textsc{sg}\\
\glft ‘you are fine’
\endgl
\trailingcitation{[rmx-e150922l.016]}
\xe

(\ref{ex:equa-3}) comes from Juana who provides some additional information about a woman she was talking about.

\ea\label{ex:equa-3}
\begingl
\glpreamble chikomarne Miyel\\
\gla chi-komar-ne Miyel\\
\glb 3-fellow-\textsc{possd} Miguel\\
\glft ‘she is Miguel’s fellow’
\endgl
\trailingcitation{[jxx-p120430l-2.342]}
\xe


Negation\is{negation|(} is rarely found among the semantic types described in this section, most of the examples I found in the corpus were elicited. If the subject is an SAP, the same negative particle we find in standard negation is used. Consider (\ref{ex:not-mother-1}) and (\ref{ex:not-mother-2}) which were elicited from Miguel and Juana respectively.

\ea\label{ex:not-mother-1}
\begingl
\glpreamble piti kuina nÿenuina\\
\gla piti kuina nÿ-enu-ina\\
\glb 2\textsc{sg.prn} \textsc{neg} 1\textsc{sg}-mother-\textsc{irr.nv}\\
\glft ‘you are not my mother’
\endgl
\trailingcitation{[rmx-e150922l.100]}
\xe

\ea\label{ex:not-mother-2}
\begingl
\glpreamble kuina nÿenubina\\
\gla kuina nÿ-enu-bi-ina\\
\glb \textsc{neg} 1\textsc{sg}-mother-2\textsc{sg}-\textsc{irr.nv}\\
\glft ‘you are not my mother’
\endgl
\trailingcitation{[jxx-p150920l.052]}
\xe

If the subject is a third person, there is a different strategy: a negative third person pronoun\is{pronoun!negative pronoun}\is{negation!negative pronoun} is used, composed of the third person index \textit{chÿ-} and the \isi{non-verbal irrealis marker} \textit{-ina}. It thus resembles both the \isi{topic pronoun} \textit{chibu} and the negative particle \textit{kuina}. Two examples are given below.

(\ref{ex:neg-eq-1}) comes from elicitation with María S.

\ea\label{ex:neg-eq-1}
\begingl
\glpreamble chÿina nÿenuina\\
\gla chÿina nÿ-enu-ina\\
\glb 3\textsc{neg.prn} 1\textsc{sg}-mother-\textsc{irr.nv}\\
\glft ‘she is not my mother’
\endgl
\trailingcitation{[rmx-e150922l.099]}
\xe

(\ref{ex:neg-eq-2}) stems from Miguel’s description of the \isi{frog story}. It refers to the picture on which the boy realises that what he was holding were not branches of a bush, but a deer’s antler.

\ea\label{ex:neg-eq-2}
\begingl
\glpreamble chÿinatu kÿkejina\\
\gla chÿina-tu yÿkÿke-ji-ina\\
\glb 3\textsc{neg.prn}-\textsc{iam} stick-\textsc{col}-\textsc{irr.nv}\\
\glft ‘now it wasn’t branches’
\endgl
\trailingcitation{[mox-a110920l-2.129]}
\xe

I have also found a few examples in which \textit{kuina} is used with third person referents in non-verbal predication. One of them is (\ref{ex:neg-eq-4}) from elicitation with Juana. She was asked to translate ‘he is not my brother’ but instead she gave the form for a sibling of the same sex, i.e. a sister in this case.

\ea\label{ex:neg-eq-4}
\begingl
\glpreamble kuina nipijina\\
\gla kuina ni-piji-ina\\
\glb \textsc{neg} 1\textsc{sg}-sibling-\textsc{irr.nv}\\
\glft ‘she is not my sister’
\endgl
\trailingcitation{[jxx-p150920l.053]}
\xe
\is{negation|)}
\is{equative/proper inclusion clause|)}

\subsection{Quantification}\label{sec:Quantification}
\is{quantification clause|(}

Predication of a quantity is achieved in the same way as equative, proper inclusion\is{equative/proper inclusion clause} and attributive predication, the only difference being that the predicate is a \isi{numeral} or quantifier\is{quantifier|(} in this case. However, since it seems to be a type rarely mentioned in the literature \citep[cf.][61]{Rose2018}, it deserves being treated in a bit more detail here. 

As is the case with the other types described above, a subject NP can be juxtaposed to the numeral or quantifier or a person marker\is{person marking} can index the subject.

(\ref{ex:QuantP-2}) was elicited from Juana.

\ea\label{ex:QuantP-2}
\begingl 
\glpreamble musumenubetu chimajinubetu\\
\gla musume-nube-tu chi-ima-ji-nube-tu\\ 
\glb many-\textsc{pl}-\textsc{iam} 3-husband-\textsc{col}-\textsc{pl}-\textsc{iam}\\ 
\glft ‘she has had many husbands (lit.: her husbands are many already)’\\ 
\endgl
\trailingcitation{[jmx-e090727s.076]}
\xe
\is{quantifier|)}

(\ref{ex:QuantP-5}) also comes from Juana who speaks about the duration of her grandson’s studies at university.

\ea\label{ex:QuantP-5}
\begingl 
\glpreamble ruschÿtu anyo\\
\gla ruschÿ-tu anyo\\ 
\glb two-\textsc{iam} year\\ 
\glft ‘it is two years now (that he is in university)’\\ 
\endgl
\trailingcitation{[jxx-p110923l-1.185]}
\xe

(\ref{ex:QuantP-3}) is a statement by Miguel after we came back from visiting José. Miguel was bitten by many ticks, while I only suffered a few tick bites.

\ea\label{ex:QuantP-3}
\begingl 
\glpreamble parikiyu samuchujane\\
\gla pariki-yu samuchu-jane\\ 
\glb many-\textsc{ints} tick.sp-\textsc{pl.nh}\\ 
\glft ‘there are a lot of ticks’\\ 
\endgl
\trailingcitation{[mrx-c120509l.148]}
\xe

(\ref{ex:QuantP-1}) is an example in which the subject is indexed on the numeral. It is a statement by Juana about the number of the Supepí sisters, not counting the ones who had already passed away.

\ea\label{ex:QuantP-1}
\begingl
\glpreamble i nÿti, Maria, Krara, tresxhecheikubimÿnÿ tanÿma\\
\gla i nÿti Maria Krara tresxhe-cheiku-bi-mÿnÿ tanÿma\\
\glb and 1\textsc{sg.prn} María Clara three-\textsc{cont}-1\textsc{pl}-\textsc{dim} now\\
\glft ‘and me, María, Clara, we are only three now’
\endgl
\trailingcitation{[jxx-p120430l-2.352-353]}
\xe

One peculiarity of numerals\is{numeral|(} acting as predicates is that after a \isi{plural} marker, they can take a third person marker\is{person marking} following the stem. This marker is usually part of the numeral, but it undergoes lenition if the plural marker or some other morpheme (as in (\ref{ex:QuantP-1})) is added. A third person marker is then attached. Consider (\ref{ex:quant-2-1}) in which María S. first uses the numeral in juxtaposition to a nominal demonstrative and then again, in repetition of the predication on with plural and third person marker attached to it. By doing so, she corrected her own priorly uttered statement that I had three children.

\ea\label{ex:quant-2-1}
\begingl
\glpreamble ruschÿkena ekanube rusxhunubechÿ\\
\gla ruschÿ-kena eka-nube rusxhu-nube-chÿ\\
\glb two-\textsc{uncert} \textsc{dem}a-\textsc{pl} two-\textsc{pl}-3\\
\glft ‘they are probably two, they are two’
\endgl
\trailingcitation{[rmx-e150922l.078]}
\xe

(\ref{ex:three-3}) is from Juana and has a similar context as (\ref{ex:QuantP-1}) above, only that this time the sentence is about third person subjects and she counts the men in. Sadly to say, one of them has passed away since then.

\ea\label{ex:three-3}
\begingl
\glpreamble trexenubechÿ seunubenube i ruxhnubechÿ jentenube\\
\gla trexe-nube-chÿ seunube-nube i ruxh-nube-chÿ jente-nube\\
\glb three-\textsc{pl}-3 woman-\textsc{pl} and two-\textsc{pl}-3 men-\textsc{pl}\\
\glft ‘the women are three and the men are two’
\endgl
\trailingcitation{[jxx-p120515l-2.239]}
\xe

Finally, (\ref{ex:only-child}) shows a quantification clause with the numeral \textit{chÿnachÿ} ‘one’ to which the limitative marker \textit{-jiku} is attached. It is a question by María C. about the number of my children, when I first came to Santa Rita in 2011.

\ea\label{ex:only-child}
\begingl
\glpreamble ¿chÿnajiku pichecha?, kuina punachÿina\\
\gla chÿna-jiku pi-checha kuina punachÿ-ina\\
\glb one-\textsc{lim}1 2\textsc{sg}-son \textsc{neg} other-\textsc{irr.nv}\\
\glft ‘you have only one child?, there is no other’
\endgl
\trailingcitation{[uxx-p110825l.242-244]}
\xe
\is{numeral|)}
\is{quantification clause|)}


\subsection{Genitive and benefactive predication}\label{sec:GenitiveBenfactivePreds}\is{genitive/benefactive clause|(}

Two further minor types in non-verbal predication have been called genitive and benefactive by \citet[248]{Dryer2007}.

Genitive predication is different from possessive predication in that the existence of an item is presupposed, and the information conveyed is its relation to a possessor, while in possessive clauses,\is{possessive clause} the possessor is presupposed and the predication is about relating an item to it. Structurally, genitive predication is a subtype of equation or proper inclusion.\is{equative/proper inclusion clause} It only differs from them in that its focus is the possessive relation rather than identification of any kind.

There are a few examples in the corpus which build on the general relational noun\is{relational noun|(} \textit{-yae} as a predicate, which may be extended by a possessor NP if it has a third person possessor. A subject NP can be juxtaposed, but is omitted most of the times, since the subject is usually topical.\is{topic} The subject is never indexed on the predicate, because it always has an inanimate third person referent. All of these examples clearly focus on possessive relations rather than on identification.

(\ref{ex:GenP-2}) was elicited from Clara.

\ea\label{ex:GenP-2}
\begingl
\glpreamble niyae echÿu lote\\
\gla ni-yae echÿu lote\\
\glb 1\textsc{sg}-\textsc{grn} \textsc{dem}b plot\\
\glft ‘the plot is mine’
\endgl
\trailingcitation{[cux-c120414ls-1.104]}
\xe

(\ref{ex:GenP-3}) comes from Juana who talked about the different names a specific manor has had during the decades. Retiro was the place where Juan Ch., the consultant of Riester, used to live.

\ea\label{ex:GenP-3}
\begingl
\glpreamble aa Retiro estansiane chiyaebane mm Aurerio Castedo\\
\gla aa Retiro estansia-ne chi-yae-bane mm {Aurerio Castedo}\\
\glb \textsc{intj} Retiro manor-\textsc{possd}? 3-\textsc{grn}-\textsc{rem} \textsc{intj} {Aurelio Castedo}\\
\glft ‘ah as for the manor Retiro, it was mm Aurelio Castedo’s’
\endgl
\trailingcitation{[jxx-p120430l-2.019]}
\xe

(\ref{ex:GenP-1}) was elicited from Juana.

\ea\label{ex:GenP-1}
\begingl 
\glpreamble kuina niyaena, chiyae nima\\
\gla kuina ni-yae-ina chi-yae ni-ima\\ 
\glb \textsc{neg} 1\textsc{sg}-\textsc{grn}-\textsc{irr} 3-\textsc{grn} 1\textsc{sg}-husband\\ 
\glft ‘it is not mine (the sombrero), it is my husband's‘\\ 
\endgl
\trailingcitation{[jxx-e081025s-1.123]}
\xe

(\ref{ex:GenP-4}) comes from Miguel and refers to something that was mine. I cannot say what it was in retrospect, because there is no video-recording.

\ea\label{ex:GenP-4}
\begingl
\glpreamble eka piyae\\
\gla eka pi-yae\\
\glb \textsc{dem}a 2\textsc{sg}-\textsc{grn}\\
\glft ‘this is yours’
\endgl
\trailingcitation{[mrx-c120509l.030]}
\xe

\is{relational noun|)}

Benefactive predicates are built on the \isi{general oblique} preposition \textit{tÿpi}. Just like genitive predication, it can be considered a subtype of the equative or proper inclusion type.\is{equative/proper inclusion clause} Unlike in equative or proper inclusion, the subject usually precedes the predicate,\is{word order} which may be due to influence of Spanish word order (but see (\ref{ex:BenP-3}) where the subject follows due to emphasis on an exclusive benefactive relation).

(\ref{ex:BenP-1}) and (\ref{ex:BenP-2}) were both produced by Juana in elicitation.

\ea\label{ex:BenP-1}
\begingl 
\glpreamble eka pitÿpi\\
\gla eka pi-tÿpi\\ 
\glb \textsc{dem}a 2\textsc{sg}-\textsc{obl}\\ 
\glft ‘this is for you’\\ 
\endgl
\trailingcitation{[jmx-e090727s.067]}
\xe

\ea\label{ex:BenP-2}
\begingl
\glpreamble eka punachÿ tÿpi piati\\
\gla eka punachÿ tÿpi pi-ati\\
\glb \textsc{dem}a other \textsc{obl} 2\textsc{sg}-brother\\
\glft ‘the other one is for your brother’
\endgl
\trailingcitation{[jmx-e090727s.063]}
\xe

An example with a negated benefactive predicate is (\ref{ex:BenP-4}), which comes from Juana telling the story about the origin of some plants. This story mixes with the biblical creation story, so it is actually Jesus who speaks here, telling a monkey that the corn is not meant for it.

\ea\label{ex:BenP-4}
\begingl
\glpreamble “kuina pitÿpina”\\
\gla kuina pi-tÿpi-ina\\
\glb \textsc{neg} 2\textsc{sg}-\textsc{obl}-\textsc{irr.nv}\\
\glft ‘“it is not for you”’
\endgl
\trailingcitation{[jxx-n101013s-1.872]}
\xe

The long example of (\ref{ex:BenP-3}) comes from Miguel and has several benefactive predications in a row. He explains here why the pupils in \isi{Altavista} had wooden plates to write on in the old days: because paper was reserved for the \textit{karay}.

\ea\label{ex:BenP-3}
\begingl
\glpreamble kaku pero kuina, chitÿpijiku eka kayaraunube echÿu ajumerku kuadernu, \\chitÿpijikunube, chitÿpi eka jentenube naka o komunidades kuina\\
\gla kaku pero kuina chi-tÿpi-jiku eka kayarau-nube echÿu ajumerku kuadernu chi-tÿpi-jiku-nube chi-tÿpi eka jente-nube naka o komunidades kuina\\
\glb exist but \textsc{neg} 3-\textsc{obl}-\textsc{lim}1 \textsc{dem}a karay-\textsc{pl} \textsc{dem}b paper notebook 3-\textsc{obl}-\textsc{lim}1-\textsc{pl} 3-\textsc{obl} \textsc{dem}a man-\textsc{pl} here or communities \textsc{neg}\\
\glft ‘there was, but no, the paper and notebooks were only for the \textit{karay}, only for them, not for the people here or the communities’
\endgl
\trailingcitation{[mxx-p181027l-1.027-029]}
\xe

Calling this type of non-verbal predication “benefactive” suggests that the beneficiary is human or at least animate,\is{animacy} but we also find constructions of this type in which we have inanimate “beneficiaries”. Again, these examples resemble Spanish resemble Spanish word order very much, except that a copula is missing. Two examples follow.

(\ref{ex:BenP-5}) comes from the listing of several plants by María C. to teach me some vocabulary. I could not find out which tree \textit{kupaju} refers to.

\ea\label{ex:BenP-5}
\begingl
\glpreamble kupajumÿnÿ tÿpi bubiu\\
\gla kupaju-mÿnÿ tÿpi bi-ubiu\\
\glb tree.sp-\textsc{dim} \textsc{obl} 1\textsc{pl}-house\\
\glft ‘the \textit{kupaju} wood is for our houses’
\endgl
\trailingcitation{[uxx-p110825l.229]}
\xe

(\ref{ex:BenP-6}) was produced by Juana, when she told me about a house that they considered renting.

\ea\label{ex:BenP-6}
\begingl
\glpreamble ... pero mil bolivianos tÿpi entero ubiae\\
\gla pero mil bolivianos tÿpi entero ubiae\\
\glb but 1000 bolivianos \textsc{obl} whole house\\
\glft ‘...but it is 1000 bolivianos for the whole house’
\endgl
\trailingcitation{[jxx-p120430l-1.365-369]}
\xe

%kaku +
%-mÿnÿ
%-nube
%-jane
%-tu
%-kuÿ
%-ina
%-kena
%-ini
%-yenu
%-ji
%-jiku
%-yu
%-uku
%-kuiku?
\is{genitive/benefactive clause}
\is{juxtaposition|)}

\subsection{Location}\label{sec:LocativePredicates}
\is{locative clause|(}

Paunaka has a \isi{verb} to express location, \textit{-ubu} ‘be, live’,\is{copula|(} but it is mainly used with temporally stable locations like the place where somebody lives. In reference to temporary locations, speakers often prefer a non-verbal strategy. 

If the \isi{subject} of a locative predication is a third person, the copula \textit{kaku} ‘exist’ can be used. Just like the gloss suggests, \textit{kaku} is also found in existential predication.\is{existential clause|(} Indeed, there is semantic overlap between locative and existential predication. According to \citet[9]{Creissels2014a}, both provide different perspectives on how to encode the relationship between a figure and a ground, with locative predication tracking the figure and existential predication tracking the ground. 

In Paunaka, this distinction can be reflected in a different word order\is{word order|(} of the two constructions. In locative predication the locative expression directly follows the copula. If there is a conominal subject,\is{conomination} the latter can precede the copula as in (\ref{ex:LocP-1}) or follow the locative phrase as in (\ref{ex:LocP-3}). In existential predication, however, it is the subject that directly follows the copula. A locative expression can occur in these clauses but is not mandatory.\is{word order|)}\is{existential clause|)}

(\ref{ex:LocP-1}) comes from Juana who was speaking about several people in her social network, and told me where they lived.\footnote{This would actually be a context in which the verb \textit{-ubu} ‘be, live’ could well be used. I do not know why Juana preferred the copula here, but I suspect it has to do with the copula being much more frequent than the verb, thus gradually being extended to contexts of permanent location, too.}

\ea\label{ex:LocP-1}
\begingl 
\glpreamble i nikumarne kaku nauku Conceyae\\
\gla i ni-kumare-ne kaku nauku Conce-yae\\ 
\glb and 1\textsc{sg}-fellow-\textsc{possd} exist there Concepción-\textsc{loc}\\ 
\glft ‘and my fellow is there in Concepción’\\ 
\endgl
\trailingcitation{[jxx-p110923l-2.133]}
\xe

(\ref{ex:LocP-3}) comes from the story about the enchanted cowherd told by Miguel. The spirit of the hill had taken away the cows to his world in the hill. When the cowherd finds out, he informs his wife, and she replies:

\ea\label{ex:LocP-3}
\begingl
\glpreamble “kakutu chiyikikiyae echÿu bakajane kakunubetu nauku”\\
\gla kaku-tu chiyikiki-yae echÿu baka-jane kaku-nube-tu nauku\\
\glb exist-\textsc{iam} hill-\textsc{loc} \textsc{dem}b cow-\textsc{distr} exist-\textsc{pl}-\textsc{iam} there\\
\glft ‘“the cows are in the hill now, they are there now”’
\endgl
\trailingcitation{[mxx-n151017l-1.64]}
\xe

I deliberately stated above that the difference between locative and existential predication\is{existential clause|(} \textit{can} be reflected in different \isi{word order}, because this is not always the case. Due to information structure, the locative expression can also sometimes precede the predicate. This is the case in (\ref{ex:loc-exi-1}), in which Juana connects to my statement that Federico was in Buenos Aires with the following:

\ea\label{ex:loc-exi-1}
\begingl
\glpreamble aa Buenos Aires, aa, nauku kaku nijinepÿi\\
\gla aa {Buenos Aires} aa nauku kaku ni-jinepÿi\\
\glb \textsc{intj} {Buenos Aires} \textsc{intj} there exist 1\textsc{sg}-daughter\\
\glft ‘ah Buenos Aires, ah, my daughter is there’
\endgl
\trailingcitation{[jxx-p110923l-1.104-107]}
\xe

In this case, it is only the context and general knowledge that helps to distinguish locative from existential\is{existential clause|)} and also possessive predication.\is{possessive clause} With general knowledge I refer to the fact that Juana most probably presupposed that I know she has a daughter, so that an existential or possessive reading is excluded. For the sake of simplicity and because it is often not totally clear what exactly a speaker had in mind when producing a sentence, I will not consider more examples like (\ref{ex:loc-exi-1}) here.

In locative predication, it is common that the \isi{subject} is topical\is{topic} and thus not conominated, as in the following examples. (\ref{ex:LocP-2}) comes from Juana and is about my cell phone. The first person possessor on the noun is related to the fact that this sentence was produced to correct my pronunciation.

\ea\label{ex:LocP-2}
\begingl 
\glpreamble kaku nipusaneyae\\
\gla kaku ni-pusane-yae \\ 
\glb exist 1\textsc{sg}-bag-\textsc{loc}\\ 
\glft ‘it is in my bag’\\ 
\endgl
\trailingcitation{[jxx-p110923l-2.040]}
\xe

(\ref{ex:LocP-5}) was produced by María S. and is about Juana. The reason for her being in Santa Cruz was given before: she cares for her grandchildren and cooks.

\ea\label{ex:LocP-5}
\begingl
\glpreamble nechikue kaku nauku Santa Cruz\\
\gla nechikue kaku nauku {Santa Cruz}\\
\glb therefore exist there {Santa Cruz}\\
\glft ‘that’s why she is there in Santa Cruz’
\endgl
\trailingcitation{[rxx-e120511l.120]}
\xe

(\ref{ex:LocP-6}) comes from an elicitation session with wooden toy figures. It is Miguel’s answer to Alejo’s question where the wooden toy was.

\ea\label{ex:LocP-6}
\begingl
\glpreamble hm, kaku naka mÿbanejiku eka ubiae\\
\gla hm kaku naka mÿbane-jiku eka ubiae\\
\glb \textsc{intj} exist here close-\textsc{lim}1 \textsc{dem}a house\\
\glft ‘hm, it is here, close to the house’
\endgl
\trailingcitation{[mtx-e110915ls.47]}
\xe

So far, all examples were about the location of a third person referent. SAP referents cannot combine with the copula \textit{kaku}.\footnote{A probable explanation for this incompatibility is provided in \sectref{sec:Kapunu} below. Note that there is one counter-example in the corpus, which comes from Juana.} There are two alternative ways to predicate location of a first or second person: first, the locative copular suffix \textit{-uku} can be attached to a \isi{personal pronoun}. This suffix is exclusively found with personal pronouns. Second, the personal pronoun and the locative expression can be juxtaposed\is{juxtaposition} without any further marking of the relation between them. In any case, the personal pronoun comes first and the locative expression follows.\is{word order}

(\ref{ex:LocP-7}) to (\ref{ex:LocP-8}) illustrate the use of the locative suffix, and (\ref{ex:LocP1sg}) and (\ref{ex:LocP2sg}) the juxtaposition strategy.

(\ref{ex:LocP-7}) comes from the creation story as told by Juana. God has just asked Jesus where he was, thus the latter answers:

\ea\label{ex:LocP-7}
\begingl
\glpreamble “nÿtiuku naka”\\
\gla nÿti-uku naka\\
\glb 1\textsc{sg.prn}-\textsc{prn.loc} here\\
\glft ‘“I am here”’
\endgl
\trailingcitation{[jxx-n101013s-1.467]}
\xe

(\ref{ex:LocP-bitiuku}) comes from the recordings by Riester and is about Juan Ch. and his sister being the only ones of their family in Retiro.

\ea\label{ex:LocP-bitiuku}
\begingl 
\glpreamble rusxujikubinube bitiuku nakaja\\
\gla rusxu-jiku-bi-nube biti-uku naka-ja\\ 
\glb two-\textsc{lim}1-1\textsc{pl}-\textsc{pl} 1\textsc{pl.prn}-\textsc{prn.loc} here-\textsc{emph}1\\ 
\glft ‘only the two of us are here’\\ 
\endgl
\trailingcitation{[nxx-p630101g-1.165]}
\xe

(\ref{ex:LocP-8}) was elicited from Juana. Note, however, that María S. does not accept the combination of the locative-marked pronoun\is{personal pronoun} with the adverb \textit{nauku}. According to her, it can only combine with the proximate \textit{naka} ‘here’.

\ea\label{ex:LocP-8}
\begingl
\glpreamble pitiuku nauku\\
\gla piti-uku nauku\\
\glb 2\textsc{sg.prn}-\textsc{prn.loc} there\\
\glft ‘you were there’
\endgl
\trailingcitation{[jxx-p150920l.108]}
\xe

(\ref{ex:LocP1sg}) comes from Isidro talking with Swintha. He contrasts his state of being at the place of conversation (here) in the first clause with his wife being alone on their field, which is expressed in the second clause with a verbal predicate.

\ea\label{ex:LocP1sg}
\begingl 
\glpreamble nÿti naka, tipÿisisikubu nauku\\
\gla nÿti naka ti-pÿisisikubu nauku\\ 
\glb 1\textsc{sg.prn} here 3i-be.alone there\\ 
\glft ‘I am here, she is alone there’\\ 
\endgl
\trailingcitation{[dxx-d120416s.167]}
\xe

(\ref{ex:LocP2sg}) was produced by Juana, re-narrating what happened when we wanted to meet, but she came late. When she just left home, I had already arrived at the zoo, so I had to wait for her for quite some time there.

\ea\label{ex:LocP2sg}
\begingl 
\glpreamble i piti nauku zoolojikayae\\
\gla i piti nauku zoolojika-yae\\ 
\glb and 2\textsc{sg.prn} there zoo-\textsc{loc}\\ 
\glft ‘and you were there at the zoo’\\ 
\endgl
\trailingcitation{[jxx-p110923l-2.044]}
\xe

Negation\is{negation|(} is achieved by the same negative particle \textit{kuina} that we also find in verbal clauses (see \sectref{sec:Negation}). In negated clauses, the copula takes the \isi{non-verbal irrealis marker} as in (\ref{ex:LocP-9}) or is omitted as in (\ref{ex:LocP-10}), with the latter being less common. All examples of negated locative predication I found in the corpus have third person referents.

(\ref{ex:LocP-9}) comes from Juana reporting what Miguel’s daughter had said when she asked her about her father.

\ea\label{ex:LocP-9}
\begingl
\glpreamble “kuina kakuina, tiyunu Santa Kuru”\\
\gla kuina kaku-ina ti-yunu {Santa Kuru}\\
\glb \textsc{neg} exist-\textsc{irr.nv} 3i-go {Santa Cruz}\\
\glft ‘“he is not here, he went to Santa Cruz”'
\endgl
\trailingcitation{[jxx-e150925l-1.126]}
\xe

(\ref{ex:LocP-10}) was produced by Miguel in telling the \isi{frog story} and it refers to the frog, which has left its glass.

\ea\label{ex:LocP-10}
\begingl
\glpreamble kuinabutu naka\\
\gla kuina-bu-tu naka\\
\glb \textsc{neg}-\textsc{dsc}-\textsc{iam} here\\
\glft ‘it is not here anymore’
\endgl
\trailingcitation{[mox-a110920l-2.039]}
\xe
\is{negation|)}

(\ref{ex:LocP-11}) comes from Miguel. It is a description of the first picture of the \isi{frog story}. He uses a locative clause first to introduce the boy into the discourse: the adverb \textit{naka} ‘here’ directly follows the copula.  Then he uses two existential clauses to introduce two additional referents, the dog and the glass. In these latter cases, the subjects follow the copula and the adverbs come last in the clause. Existential predication is the topic of the next section.

\ea\label{ex:LocP-11}
\begingl
\glpreamble kaku naka eka sepitÿmÿnÿ, kaku kabemÿnÿ naka, kakuku eka tachumÿnÿkena eka naka\\
\gla kaku naka eka sepitÿ-mÿnÿ kaku kabe-mÿnÿ naka kaku-uku eka tachu-mÿnÿ-kena eka naka\\
\glb exist here \textsc{dem}a child-\textsc{dim} exist dog-\textsc{dim} here exist-\textsc{add} \textsc{dem}a small.pot-\textsc{dim}-\textsc{uncert} \textsc{dem}a here\\
\glft ‘the boy is here, here is a little dog and here is what I suppose is a small pot’
\endgl
\trailingcitation{[mox-a110920l-2.006-007]}
\xe
\is{locative clause|)}

\subsection{Existentials}\label{sec:Existentials}
\is{existential clause|(}

Existential and locative clauses\is{locative clause} overlap in that both often express a spatial relation between a figure and a ground. Both prototypically encode “\textit{episodic} spatial relationships between a \textit{concrete} entity conceived as \textit{movable} (the figure) and another concrete entity (the ground) conceived as occupying a fixed position in the space, or at least as being \textit{less easily movable} than the figure” \citep[10]{Creissels2014a}.
 Existential clauses, however, provide a different perspective on the spatial relation as locative clauses do,\is{locative clause} i.e. a perspective from the ground, not the figure \citep[9, 18]{Creissels2014a}.
%“selection of the ground as the perspectival center in clauses encoding figure-ground relationships” (18)
This is why they do not serve as “adequate answers to questions about the location of an entity, but can be used to identify an entity present at a certain location” \citep[2]{Creissels2014a}.

(\ref{ex:Exi-1}) and (\ref{ex:Exi-2}) are examples for prototypical existential clauses resembling the ones presented by \citet[]{Creissels2014a}: they have an indefinite referent,\is{definiteness} which is the moveable figure on a relatively fixed ground. Both examples were elicited, (\ref{ex:Exi-1}) comes from María S., (\ref{ex:Exi-2}) from Juana.

\ea\label{ex:Exi-1}
\begingl
\glpreamble kaku jike nikusepineyae\\
\gla kaku jike ni-kusepi-ne-yae\\
\glb exist fly 1\textsc{sg}-thread-\textsc{possd}-\textsc{loc}\\
\glft ‘there is a fly on my thread’
\endgl
\trailingcitation{[rxx-e181024l.092]}
\xe

\ea\label{ex:Exi-2}
\begingl
\glpreamble kaku ÿne chÿupekÿyae keyu\\
\gla kaku ÿne chÿ-upekÿ-yae keyu\\
\glb exist water 3-place.under-\textsc{loc} snail\\
\glft ‘there is water under the snail’
\endgl
\trailingcitation{[jcx-e090727s.035]}
\xe


Thus an existential construction exists in Paunaka, and in many cases it can be distinguished from locative predication\is{locative clause} by placement of the subject directly following the copula.\is{word order}\footnote{Note, however, that manipulation of word order does not count as “a dedicated existential construction” but is rather analysed as an equivalent by \citet[19]{Creissels2014a}.} This, however, is not the only ambit of this construction type, and probably even not its main one. Consider (\ref{ex:Exi-33}), which shows that the existential construction is not restricted to indefinite referents\is{definiteness|(} in Paunaka.\footnote{Note that although indefiniteness is often explicitly or implicitly involved in the definition of an existential construction, \citet[4]{Creissels2014a} has shown that in some languages existential constructions are not restricted to indefinite subjects.} It was produced by María C. when it was about to rain. The hammock is mentioned here for the first time. This is probably the reason why María C. chose an existential construction rather than a locative one with the spatial expression following the copula directly.

\ea\label{ex:Exi-33}
\begingl
\glpreamble kaku niyumaji nekupai\\
\gla kaku ni-yumaji nekupai\\
\glb exist 1\textsc{sg}-hammock outside\\
\glft ‘my hammock is outside (i.e. there is my hammock outside)’
\endgl
\trailingcitation{[cux-120410ls.258]}
\xe

A second example with a definite subject is (\ref{ex:Exi-3}) from Juana. The cows she is speaking about were already well established in the story, which was about her grandparents buying cows in Moxos. However, they had not been mentioned for some time. The existential construction is thus used here to re-establish the cows as a topic. The location mentioned in this sentence is an enclosure by a hut where Juana’s grandparents slept on their way home, so it can be considered a temporary, episodic location rather than a permanent one.

\ea\label{ex:Exi-3}
\begingl
\glpreamble i kaku baka bakayayae\\
\gla i kaku baka bakaya-yae\\
\glb and exist cow enclosure-\textsc{loc}\\
\glft ‘and the cows were in the enclosure (i.e. there were the cows in the enclosure)’
\endgl
\trailingcitation{[jxx-p151016l-2.185]}
\xe


The Paunaka existential construction also relates to two types that \citet[]{Creissels2014} explicitly distinguishes from existential predication, although he recognises that they are related and in some languages encoded by the same construction.

First, the Paunaka construction serves a presentative function, i.e. the introduction of participants into the discourse \citep[cf.][15]{Creissels2014a}. Actually, the presentative and the “prototypical” existential construction both introduce an indefinite referent into the discourse and thus only differ in presence or absence of a locative expression in the clause. It is because of this presentative type that existential predication partly overlaps with possessive predication\is{possessive clause} in Paunaka (see \sectref{sec:PossessiveClauses}). 

Second, the existential construction in Paunaka also encodes “habitual presence of an entity at some place” \citet[14]{Creissels2014a}. The actual place does not have to be overtly expressed in this case if it is identifiable from the context or identical to the deictic centre. This is to say that if I speak of existential predication (or an existential clause or construction) in this work, this includes also presentatives as well as expressions of \isi{habitual} presence. My usage of the term is thus more conform with the broader definition of existential predication given by \citet[123--125]{Payne1997} or \citet[240--244]{Dryer2007}.  

A last word about the notion of “subject”\is{subject} is necessary. I agree with \citet[9]{Overall2018} who state that “[t]he indefinite participant introduced in the existential construction often lacks some of the grammatical properties of a prototypical subject, but even so, there is usually no other argument available as a candidate to be the subject”.

With this general background in mind, I turn to a few more examples of existential predication in Paunaka now.

With (\ref{ex:exist-1}), Juana introduced the arroyo close to Santa Rita into the discourse as a place where the young people go swimming. This is another example of a definite subject being introduced by an existential clause.

\ea\label{ex:exist-1}
\begingl 
\glpreamble nauku Santa Ritayae kaku echÿu chÿkÿ\\
\gla nauku {Santa Rita}-yae kaku echÿu chÿkÿ\\ 
\glb there {Santa Rita}-\textsc{loc} exist \textsc{dem}b arroyo\\ 
\glft ‘there in Santa Rita, there is this arroyo’\\ 
\endgl
\trailingcitation{[jxx-a120516l-a.571]}
\xe
\is{definiteness|)}

(\ref{ex:Exi-4}) is a typical beginning of a story by Miguel. A participant is introduced into the discourse here. There is no locative expression, but the whole story is posited in the \isi{remote past} by the remote marker \textit{-bane} being attached to the copula.

\ea\label{ex:Exi-4}
\begingl
\glpreamble kakubaneji chÿnachÿ jente i tipÿkubai\\
\gla kaku-bane-ji chÿnachÿ jente i ti-pÿkubai\\
\glb exist-\textsc{rem}-\textsc{rprt} one man and 3i-be.lazy\\
\glft ‘once upon a time there was a man, it is said, and he was lazy’
\endgl
\trailingcitation{[mox-n110920l.011]}
\xe

The copula can also take the \isi{iamitive} marker to contrast the state of existence of a referent with the time prior to this existence. This is the case in (\ref{ex:Exi-5}) from Miguel. He was talking about the history of Santa Rita and his own personal history and had just abbreviated his more detailed account by simply telling me that several years turned by until:

\ea\label{ex:Exi-5}
\begingl
\glpreamble i kakutu echÿu nuebo presidente de Bolivia\\
\gla i kaku-tu echÿu {nuebo presidente de Bolivia}\\
\glb and exist-\textsc{iam} \textsc{dem}b {new president of Bolivia}\\
\glft ‘and then there was this new president of Bolivia’
\endgl
\trailingcitation{[mxx-p110825l.035]}
\xe

(\ref{ex:exist-tu}) comes from Juana and is an example of habitual presence. The subject of this clause is complex. It is an equative clause\is{equative/proper inclusion clause} with two NPs in juxtaposition, which both mean ‘pot’. The difference is that \textit{nÿkÿiki} is a Paunaka word, and because of its native origin it is here associated with traditional (clay) pots. \textit{Uyetaki} is a loan from Spanish \textit{olleta} ‘pot’ with the \isi{classifier} for round objects \textit{-ki} attached to it. The Spanish loan is associated with modern pots made of aluminium.

\ea\label{ex:exist-tu}
\begingl 
\glpreamble metu kakutu eka nÿkÿiki uyetaki\\
\gla metu kaku-tu eka nÿkÿiki uyeta-ki\\ 
\glb already exist-\textsc{iam} \textsc{dem}a pot aluminium.pot-\textsc{clf}:spherical\\ 
\glft ‘now there are these modern pots (i.e. the pots that are aluminium pots)’\\ 
\endgl
\trailingcitation{[jxx-d110923l-2.41]}
\xe

Another sentence representing habitual presence is (\ref{ex:Exi-6}) from María S. Although a location is specified here, this sentence is not understood as encoding an episodic spatial relation, because the general context was that the Supepí siblings had left their old house after their father passed away, and it was only their mother who stayed in the old house, permanently.

\ea\label{ex:Exi-6}
\begingl
\glpreamble depue kakukuÿbane nÿenubane primero nubiu nauku\\
\gla depue kaku-kuÿ-bane nÿ-enu-bane primero nÿ-ubiu nauku\\
\glb afterwards exist-\textsc{incm}-\textsc{rem} 1\textsc{sg}-mother-\textsc{rem} first 1\textsc{sg}-house there\\
\glft ‘afterwards there was still my late mother in my first house there long time ago’
\endgl
\trailingcitation{[rxx-e120511l.172]}
\xe

I have found but one example in the corpus, in which an existential predication is about a non-third-person referent. It is given in (\ref{ex:Exi-7}). Just like locative clauses, this sentence is realised without a copula. It comes from María S. telling me that it was only her family that lived in the specific place they used to live before most of the siblings moved to Santa Rita, to Concepción or elsewhere.

\ea\label{ex:Exi-7}
\begingl
\glpreamble kuina, bitiyÿchi nauku\\
\gla kuina biti-yÿchi nauku\\
\glb \textsc{neg} 1\textsc{pl.prn}-\textsc{lim}2 there\\
\glft ‘no, it was only us there’
\endgl
\trailingcitation{[rxx-p181101l-2.130]}
\xe

Negative existential clauses\is{negation|(} require the negative particle \textit{kuina}. They can be formed with or without a copula. The copula usually shows up if the \isi{subject} is not conominated,\is{conomination} as in (\ref{ex:neg-exist-cond}) or in negative answers. The non-verbal irrealis marker is always attached to negated \textit{kaku} in this case.

(\ref{ex:neg-exist-cond}) was elicited from Miguel and referred to the fact that Federico had bought some food for our picnic in \isi{Altavista}.

\ea\label{ex:neg-exist-cond}
\begingl 
\glpreamble kue kuina tiyÿseika, kuina kakuina naka\\
\gla kue kuina ti-yÿseika kuina kaku-ina naka\\ 
\glb if \textsc{neg} 3i-buy.\textsc{irr} \textsc{neg} exist-\textsc{irr.nv} here\\ 
\glft ‘if he hadn’t bought it, there wouldn’t be anything (to eat) here’\\ 
\endgl
\trailingcitation{[mxx-n120423lsf-X.45]}
\xe

Otherwise, the use of the copula in negative clauses is rare, although it does occur sometimes as in (\ref{ex:kuina-kakuina-1}). This example was elicited from Miguel, but it seems to over-emphasise the non-existence a bit.

\ea\label{ex:kuina-kakuina-1}
\begingl 
\glpreamble kuina kakuina menonitanube\\
\gla kuina kaku-ina menonita-nube\\ 
\glb \textsc{neg} exist-\textsc{irr.nv} Menonite-\textsc{pl}\\ 
\glft ‘there are no Menonites (in Beni)’\\ 
\endgl
\trailingcitation{[jmx-e090727s.357]}
\xe

Usually, the negative existential clause can do without a copula, as in examples (\ref{ex:Exi-8}) and (\ref{ex:Exi-9}).

(\ref{ex:Exi-8}) comes from Juana, who told me about the circumstances of the encounter of María S. and her husband with a snake (or water spirit) in the reservoir of Santa Rita.

\ea\label{ex:Exi-8}
\begingl
\glpreamble tipÿsisikubunube kuina kristianunubeina\\
\gla ti-pÿsisikubu-nube kuina kristianu-nube-ina\\
\glb 3i-be.alone-\textsc{pl} \textsc{neg} person-\textsc{pl}-\textsc{irr.nv}\\
\glft ‘they were alone, there were no people’
\endgl
\trailingcitation{[jxx-p120515l-2.145]}
\xe

(\ref{ex:Exi-9}) is a summary of the climax of the story about the jaguar and the fox narrated by María S. The fox had made the jaguar believe that the reflection of the moon in the water was a wheel of cheese and the jaguar had drowned in trying to get hold of the suspected cheese.

\ea\label{ex:Exi-9}
\begingl
\glpreamble kuina kesuina, kujejiku\\
\gla kuina kesu-ina kuje-jiku\\
\glb \textsc{neg} cheese-\textsc{irr.nv} moon-\textsc{lim}1\\
\glft ‘there wasn’t any cheese, it was only the moon,’
\endgl
\trailingcitation{[rxx-n120511l-1.044]}
\xe
\is{negation|)}

\subsection{Possessive clauses}\label{sec:PossessiveClauses}
\is{possessive clause|(}

The non-verbal possessive clause is a type of existential clause (see \sectref{sec:Existentials} above). Just like in the latter, in positive possessive clauses, there is a copula directly followed by the subject, while in negative clauses,\is{negation} the copula can be omitted.\is{word order} The difference to existential clauses is that the subject is marked as possessed in some way. A locative expression is not required in possessive clauses, but as we have just seen, a locative expression does not necessarily occur in existential clauses either. The main reason to describe possessive clauses in its own section here is that unlike existence, possession can also be expressed by a verbal strategy which builds on a \isi{verb} composed of the \isi{attributive prefix} \textit{ku-} and a \isi{nominal stem} (see \sectref{sec:AttributiveVerbs} for the verbal expression of possession).

(\ref{ex:Possi-1}) to (\ref{ex:PossP-3}) show non-verbal possessive predication build on an inalienably possessed noun in positive clauses.

(\ref{ex:Possi-1}) comes from the creation story as told by Juana. It is Jesus who had a field.

\ea\label{ex:Possi-1}
\begingl
\glpreamble kaku chisane\\
\gla kaku chi-sane\\
\glb exist 3-field\\
\glft ‘he had a field’ (lit.: ‘there was his field’)
\endgl
\trailingcitation{[jxx-n101013s-1.555]}
\xe

(\ref{ex:Possi-2}) was produced by Clara who was trying to remember the name of the fish that bites.

\ea\label{ex:Possi-2}
\begingl
\glpreamble kaku chija echÿu\\
\gla kaku chi-ija echÿu\\
\glb exist 3-name \textsc{dem}b\\
\glft ‘it has a name’
\endgl
\trailingcitation{[cux-c120414ls-1.217]}
\xe

With (\ref{ex:PossP-3}), María C. made a judgement about the capacity of Clara’s daughters to learn Paunaka.

\ea\label{ex:PossP-3}
\begingl 
\glpreamble kaku pijinejinube pero kuina puero chitanube\\
\gla kaku pi-jine-ji-nube pero kuina puero chi-ita-nube\\ 
\glb exist 2\textsc{sg}-daughter-\textsc{col}-\textsc{pl} but \textsc{neg} can 3-master.\textsc{irr}-\textsc{pl}\\ 
\glft ‘you have daughters, but they can't figure it out (to speak Paunaka)’\\ 
\endgl
\trailingcitation{[cux-c120414ls-2.265]}
\xe
% maybe this is an exitential??


The inalienably possessed noun may also be derived by the possessed marker as in (\ref{ex:Possi-3}). This is often the case with Spanish loans. The example comes from María S. telling the story about how the tortoise got its carapace. She did not want to leave her house to welcome new-born Jesus, because she had a shop in that house.

\ea\label{ex:Possi-3}
\begingl
\glpreamble pimua, kaku chibentane\\
\gla pi-imua kaku chi-benta-ne\\
\glb 2\textsc{sg}-see.\textsc{irr} exist 3-shop-\textsc{possd}\\
\glft ‘you see, she had a shop’
\endgl
\trailingcitation{[rxx-n121128s.17]}
\xe

%The example was produced by Juana to provide some information about an old lady she once met in Candelaria.

%\ea\label{ex:Possi-33}
%\begingl
%\glpreamble kaku chibastunemÿnÿtu\\
%\gla kaku chi-bastun-ne-mÿnÿ-tu\\
%\glb exist 3-walking.cane-\textsc{possd}-\textsc{dim}-\textsc{iam}\\
%\glft ‘she already had a walking cane’\\
%\endgl
%\trailingcitation{[jxx-p120515l-1.220]}
%\xe

(\ref{ex:Possi-4}) is an example of a negative possessive clause without copula. It is a statement by María C. about being all alone, without any siblings.

\ea\label{ex:Possi-4}
\begingl
\glpreamble kuina nÿatimÿnÿina nipijina \\
\gla kuina nÿ-ati-mÿnÿ-ina ni-piji-ina\\
\glb \textsc{neg} 1\textsc{sg}-brother-\textsc{dim}-\textsc{irr.nv} 1\textsc{sg}-sibling-\textsc{irr.nv}\\
\glft ‘I don’t have a brother or sister’
\endgl
\trailingcitation{[uxx-p110825l.074]}
\xe

The possessive clause can contain a locative expression as in (\ref{ex:PossP-1}) and (\ref{ex:PossP-2}).

(\ref{ex:PossP-1}) was elicited from Juana.

\ea\label{ex:PossP-1}
\begingl 
\glpreamble ¿kaku pubiu nauku pisaneyae?\\
\gla kaku pi-ubiu nauku pi-sane-yae\\ 
\glb exist 2\textsc{sg}-house there 2\textsc{sg}-field-\textsc{loc}\\ 
\glft ‘do you have a house at your field?’\\ 
\endgl
\trailingcitation{[jmx-e090727s.352]}
\xe

(\ref{ex:PossP-2}) also comes from Juana who was making a statement about her daughter here.

\ea\label{ex:PossP-2}
\begingl 
\glpreamble kaku ruschÿ chilotene nauku\\
\gla kaku ruschÿ chi-lote-ne nauku\\ 
\glb exist two 3-plot-\textsc{possd} there\\ 
\glft ‘she has two plots there’\\ 
\endgl
\trailingcitation{[jxx-p110923l-1.421]}
\xe

Instead of marking the possession directly on the noun, the \isi{possessor} can also be expressed by a person-marked\is{person marking} preposition directly following the possessed entity. The causal and instrumental\is{instrument/cause} preposition \textit{-keuchi} is used if the possessed is a concrete object, the \isi{general oblique} preposition \textit{-tÿpi} for possession of more abstract entities, usually some temporal units. In the latter case, it is arguable whether the clause can be analysed as a possessive clause at all or rather counts as existential, depending on the question whether temporal units can be considered as being possessed. This, however, is a philosophical rather than a linguistic question, because there is no difference in structure between existential and possessive clauses anyway.

(\ref{ex:PossP-keuchi-1}) is particularly interesting, because it contains a kind of \isi{secondary possession} of an inalienably possessed noun. This noun, \textit{chÿeche} ‘meat’, has a third person marker\is{person marking} by default if used to denote meat as an edible good (\textit{chÿ-eche} 3-flesh). The third person marker can be replaced by an SAP person marker in reference to the flesh of the body (e.g. \textit{nÿ-eche} ‘my flesh’). Since there is already a person marker on the noun denoting ‘meat’, attachment of a second possessor marker is blocked and another way of expressing the possessor is needed. In possessive predication, this is achieved by using the preposition \textit{-keuchi} which carries the person marker of the possessor. The example comes from Juan Ch. who was recorded by Riester and speaks about consequences of a successful hunting expedition.

\ea\label{ex:PossP-keuchi-1}
\begingl 
\glpreamble tanÿmapaiku kaku chÿeche nikeuchi nubiuyae tÿpi chÿnachÿ semana\\
\gla tanÿma-paiku kaku chÿeche ni-keuchi nÿ-ubiu-yae tÿpi chÿnachÿ semana\\ 
\glb now-\textsc{punct} exist meat 1\textsc{sg}-\textsc{ins} 1\textsc{sg}-house-\textsc{loc} \textsc{obl} one week\\ 
\glft ‘right now I have meat for one week in my house’\\ 
\endgl
\trailingcitation{[nxx-a630101g-1.56]}
\xe

(\ref{ex:PossP-keuchi-2}) shows the same construction. In this case, I think it might also be possible to derive an inalienably possessed noun,\footnote{Animals are in general not directly possessable with a few exceptions concerning parasites. In the case of shells, however, I can imagine that a possessed form could be derived at least if reference is not to the mussel as an animal but to its shell as is the case in (\ref{ex:PossP-keuchi-2}). This remains to be verified.} but the possessive relation between a shell and a possessor is not a permanent one, unlike the relation to family members, fields or walking canes. Thus a construction with \textit{-keuchi} is preferred. The sentence comes from Miguel who was asking Juana about a special kind of shell which they use to polish pottery before burning.

\ea\label{ex:PossP-keuchi-2}
\begingl 
\glpreamble ¿pero kaku nauku sipÿ pikeuchi?\\
\gla pero kaku nauku sipÿ pi-keuchi\\ 
\glb but exist there shell 2\textsc{sg}-\textsc{ins}\\ 
\glft ‘but do you have shells (for polishing clay) there?’\\ 
\endgl
\trailingcitation{[jmx-d110918ls-1.098]}
\xe

(\ref{ex:Poss-no-banana}) is a negative possessive clause including \textit{-keuchi}. It comes from María C. who said this to me regretfully, because I had told her that my little daughter wanted a plantain.\footnote{Actually I had intended to tell María C. that my daugther \textit{liked} the plantains, when she was crawling around pointing to things and uttering one-word clauses. Apparently, she found some plantains particularly interesting.}

\ea\label{ex:Poss-no-banana}
\begingl 
\glpreamble kuinachu merÿna nikeuchi\\
\gla kuina-chÿu? merÿ-ina ni-keuchi\\ 
\glb \textsc{neg}-\textsc{dem}b? plantain-\textsc{irr.nv} 1\textsc{sg}-\textsc{ins}\\ 
\glft ‘I don’t have plantains’\\ 
\endgl
\trailingcitation{[uxx-p110825l.173]}
\xe

Turning to the use of \textit{-tÿpi} in possessive predication now, consider (\ref{ex:Possi-5}). It was elicited from Isidro and is about the age of an invented person.

\ea\label{ex:Possi-5}
\begingl 
\glpreamble metu kakutu nobenta anyo chitÿpi\\
\gla metu kaku-tu nobenta anyo chi-tÿpi\\ 
\glb already exist-\textsc{iam} ninety year 3-\textsc{obl}\\ 
\glft ‘she was already 90 years old’\\ 
\endgl
\trailingcitation{[dxx-d120416s.203]}
\xe

(\ref{ex:Possi-6}) is from the recordings made by Riester with Juan Ch.

\ea\label{ex:Possi-6}
\begingl
\glpreamble tanÿma uchuini kaku tiempo nitÿpi\\
\gla tanÿma uchuine? kaku tiempo ni-tÿpi\\
\glb now just.now? exist time-\textsc{irr.nv} 1\textsc{sg}-\textsc{obl}\\
\glft ‘now I have time’
\endgl
\trailingcitation{[nxx-p630101g-1.012]}
\xe

(\ref{ex:Possi-7}) is a negated version of (\ref{ex:Possi-6}) and comes from María S. who provides the reason why she has not finished knotting a hammock.

\ea\label{ex:Possi-7}
\begingl
\glpreamble kuina tiempoina nÿtÿpi\\
\gla kuina tiempo-ina nÿ-tÿpi\\
\glb \textsc{neg} time-\textsc{irr.nv} 1\textsc{sg}-\textsc{obl}\\
\glft ‘I didn’t have time’
\endgl
\trailingcitation{[rxx-e181022le]}
\xe

Another negative possessive clause including \textit{-tÿpi} comes from Juan C. who was talking about the past with Miguel and stated here that he had a very hard life.

\ea\label{ex:Exi-10}
\begingl
\glpreamble tijainube tijainube kuina ruminkuina nitÿpi\\
\gla tijai-nube tijai-nube kuina ruminku-ina ni-tÿpi\\
\glb day-\textsc{pl} day-\textsc{pl} \textsc{neg} Sunday-\textsc{irr.nv} 1\textsc{sg}-\textsc{obl}\\
\glft ‘every single day (I worked), there was no Sunday for me’
\endgl
\trailingcitation{[mqx-p110826l.467]}
\xe

There are also some cases of formally existential clauses, i.e. clauses that do not include any marking of possession, but imply a possessive relation nonetheless. One example is given in (\ref{ex:Possi-8}), where existence of rice implies possession of rice. It comes from Miguel.

\ea\label{ex:Possi-8}
\begingl
\glpreamble kue kaku arusu banau pan de arroz\\
\gla kue kaku arusu bi-anau {pan de arroz}\\
\glb if exist rice 1\textsc{pl}-make {rice bread}\\
\glft ‘when there is rice, we make rice bread’
\endgl
\trailingcitation{[mxx-d120411ls-1a.042]}
\xe

\is{possessive clause|)}
\is{existential clause|)}
\is{copula|)}

\subsection{Similative and related construction}\label{sec:SimilativePreds}
\is{similative clause|(}

The similative construction is relatively simple in Paunaka. It includes a comparee, a standard marker and a standard \citep[cf.][]{HaspelmathBuchholz1998}, that is, there are clauses of the type ‘X is like Y’ in Paunaka, in which \textit{X} is the comparee, \textit{like} the standard marker and \textit{Y} the standard. The standard marker is \textit{nena} ‘like, resemble, be like’ in Paunaka. Specific parameters of comparison are not included in the cosntruction. Thus sentences equivalent to ‘She is as old as me’, which I will call “equality sentences”, do not exist.\footnote{\citet[277]{HaspelmathBuchholz1998} use the term “equative construction”, but this term is already applied to a different construction here (see \sectref{sec:PropInclEquatAttr} above).} Such concepts are rather expressed by more than one clause, which do not convey exactly the same meaning.\footnote{This is also in line with what \citet[]{Rose2019c} found for Trinitario.\is{Mojeño Trinitario}} There are a few cases that resemble sentences like ‘She jumps like a frog’. This is the kind of sentence that has been described under the realm of “similative construction” by \citet[277]{HaspelmathBuchholz1998}, I thus apply the term a bit differently here. I believe though that such concepts are also often expressed by two clauses. Spanish influence may play a role here in that originally biclausal structures are re-interpreted as monoclausal, because \textit{nena} is used as a translational equivalent of \textit{como} ‘like’, \textit{parece} ‘it seems, resembles’ and \textit{igual que} ‘equal to’.\footnote{Because the question will probably arise at this point as to how comparative constructions look like in Paunaka, and since there is no other place in the grammar, where they would be described, I give a very short summary here: Comparative constructions include the adverb \textit{max} ‘more, most’, a loan\is{borrowing} from Spanish \textit{más} which has the same meaning. The adverb is placed before the word expressing a property or quality. This may be an adjective, a noun or a stative verb. Comparative constructions never include a standard, i.e. the item that is surpassed. This is rather deduced from the context. (\ref{ex:comp-adj}) is an example with an adjective and thus also a case of non-verbal predication. It comes from Juana who was conversing with María S.

\ea\label{ex:comp-adj}
\begingl
\glpreamble amukeyu max michaniki\\
\gla amukeyu max michaniki\\
\glb soft.corn more delicious\\
\glft ‘soft corn is more delicious’
\endgl
\trailingcitation{[jrx-c151001lsf-11.184]}
\xe}

Depending on topicality,\is{topic} it is possible that the comparee or the standard are not overtly expressed in the similative clause, but if there is a standard NP, it follows the standard marker \textit{nena} ‘like, resemble, be like’ directly.\is{word order} I think it is generally not possible to index a subject directly, but there are a few counter-examples in the corpus. It is common though that the \isi{additive} marker is added to \textit{nena}, and in this case, a subject index can follow the additive marker (see below in this section).

(\ref{ex:nena-2}) is an example in which both comparee and standard are expressed by NPs. It comes from María C. who uses a \isi{Bésiro} word to refer to a specific tree with dark, blood-like resin.

\ea\label{ex:nena-2}
\begingl 
\glpreamble echÿu tokoxhirx nena iti\\
\gla echÿu tokoxhirx nena iti\\ 
\glb \textsc{dem}b tree.sp like blood\\ 
\glft ‘the (resin of the) \textit{tokoxhirxh} tree is like blood’\\ 
\endgl
\trailingcitation{[ump-p110815sf.366]}
\xe

The comparee can also follow the standard as in (\ref{ex:nena-1}), which is from the data collected by Riester in the 1960s. Juan Ch. compares the \textit{pututu} soup with chicha here, i.e. the soup is not well garnished.

\ea\label{ex:nena-1}
\begingl 
\glpreamble nenayu aumue bijiemÿnÿjini\\
\gla nena-yu aumue abijie-mÿnÿ-ji-ini\\ 
\glb like-\textsc{ints} chicha pututu-\textsc{dim}-\textsc{rprt}-\textsc{frust}\\ 
\glft ‘the so-called \textit{pututu} soup is (thin) like chicha’\\ 
\endgl
\trailingcitation{[nxx-p630101g-2.58]}
\xe

(\ref{ex:nena-3}) comes from Juana and is a description of the spirit of the water, with whom her grandparents had an unpleasant encounter on their way back home from Moxos, where they had bought cows.

\ea\label{ex:nena-3}
\begingl
\glpreamble kananaji chikebÿke, nenayuji kuje chibÿke\\
\gla kanana-ji chi-kebÿke nena-yu-ji kuje chi-bÿke\\
\glb this.size-\textsc{rprt} 3-eye like-\textsc{ints}-\textsc{rprt} moon 3-face\\
\glft ‘she had big eyes, her face was like the moon, it is said’
\endgl
\trailingcitation{[jxx-p151016l-2.091]}
\xe

In (\ref{ex:nena-4}), there are two juxtaposed clauses. The comparee is expressed in the first clause, a possessive one. The similative clause follows, the comparee is not repeated. The fact that the parameter is the age has to be deduced from the context. The sentence comes from Juana who was talking about her relatives.

\ea\label{ex:nena-4}
\begingl
\glpreamble i kaku echÿu chichechapÿi nena eka nisinepÿi\\
\gla i kaku echÿu chi-chechapÿi nena eka ni-sinepÿi\\
\glb and exist \textsc{dem}b 3-son like \textsc{dem}a 1\textsc{sg}-grandchild\\
\glft ‘and she had a son, who was like my grandson (in age)’
\endgl
\trailingcitation{[jxx-p120430l-2.163]}
\xe

In (\ref{ex:nena-6}), the comparee is incorporated\is{incorporation} into the verb that precedes the similative clause. The sentence comes from Juana who reported what the old lady she met in Candelaria long ago said when some chicha dripped on her face.

\ea\label{ex:nena-6}
\begingl
\glpreamble “nijirebÿketu nenayu chÿbÿke iyu”\\
\gla ni-jire-bÿke-tu nena-yu chÿ-bÿke iyu\\
\glb 1\textsc{sg}-wrinkle-face-\textsc{iam} like-\textsc{ints} 3-face monkey\\
\glft ‘“my face wrinkled, it looks like the face of a monkey”’
\endgl
\trailingcitation{[jxx-p120515l-1.075]}
\xe

On the other hand, in (\ref{ex:nena-5}), it is the standard which is not expressed. The sentence comes from a conversation between María S. and Juana. They were just talking about keeping ducks and Juana had mentioned that ducks are dirty, because they just squat and defecate everywhere and their excrements are liquid like diarrhea. María S. adds to this:

\ea\label{ex:nena-5}
\begingl
\glpreamble kuina nenaina echÿu gansojane tirÿrÿ chisikuji\\
\gla kuina nena-ina echÿu ganso-jane ti-rÿrÿ chi-sikuji\\
\glb \textsc{neg} like-\textsc{irr.nv} \textsc{dem}b goose-\textsc{distr} 3i-be.hard 3-excrement\\
\glft ‘the geese are not like them, their poo is hard’
\endgl
\trailingcitation{[jrx-c151001lsf-11.040]}
\xe

(\ref{ex:nena-6}) and (\ref{ex:nena-5}) come close to what was originally defined as the similative construction \citep[cf.][277]{HaspelmathBuchholz1998}. I have found one example which comes even closer. It was produced by María S. when I was eliciting examples with the associated motion marker. Apparently, she found the idea of simultaneously moving and eating quite funny, saying:

\ea\label{ex:nena-7}
\begingl
\glpreamble ninikukukÿu nena mura\\
\gla ni-niku-kukÿu nena mura\\
\glb 1\textsc{sg}-eat-\textsc{am.conc.tr} like horse\\
\glft ‘I walk eating like a horse’
\endgl
\trailingcitation{[rmx-e150922l.066]}
\xe

It is not clear to me whether (\ref{ex:nena-7}) is still a biclausal sentence or can be considered a monoclausal one. What becomes apparent in any case is that there is no subject marker on \textit{nena}, although the comparee is a first person. Consider also (\ref{ex:nena-8}), which comes from Miguel who addressed Juana. The latter had just loaded a big bag full of loam onto her head.

\ea\label{ex:nena-8}
\begingl
\glpreamble nenayu mutuÿ\\
\gla nena-yu mutuÿ\\
\glb like-\textsc{ints} termite\\
\glft ‘you look like a termite’
\endgl
\trailingcitation{[jmx-d110918ls-1.112]}
\xe

There are, however, also a few examples in the corpus with a subject marker added to the standard marker.\is{person marking} One of them is (\ref{ex:nena-9}), which comes from Miguel, when he was telling the story about the ants that are happy, when a boy is born, because when he is on a trip, he drops little crumbs of food that they can eat. The boy’s being on a trip is compared to our situation, because we were currently on a trip to \isi{Altavista}. Note that the verb \textit{-chubiku} ‘stroll’ is mostly used to denote hunting trips, which is probably why Miguel felt the need to specify what he wanted to say by use of a Spanish loan \textit{pasea} ‘stroll’.

\ea\label{ex:nena-9}
\begingl
\glpreamble tiyuna tichubikupa tiyuna paseana nenabi biti tanÿmapaiku\\
\gla ti-yuna ti-chubiku-pa ti-yuna pasea-ina nena-bi biti tanÿma-paiku\\
\glb 3i-go.\textsc{irr} 3i-stroll-\textsc{dloc.irr} 3i-go.\textsc{irr} stroll-\textsc{irr.nv} like-1\textsc{pl} 1\textsc{pl.prn} now-\textsc{punct}\\
\glft ‘he will go on a hunting trip, he will go on a jaunt like we are doing right now’
\endgl
\trailingcitation{[mxx-n120423lsf-X.14-15]}
\xe

It is common to add the \isi{additive} marker \textit{-uku} to the standard marker and then attach a person marker.\is{person marking} In (\ref{ex:nena-10}), Miguel uses the standard marker in this way to make a comparative statement to what Juan C. had said before. Both of them suffered lack of water in former times.

\ea\label{ex:nena-10}
\begingl
\glpreamble nenaukubi nauku Santa Rita kuina ÿneina\\
\gla nena-uku-bi nauku {Santa Rita} kuina ÿne-ina\\
\glb like-\textsc{add}-1\textsc{pl} there {Santa Rita} \textsc{neg} water-\textsc{irr.nv}\\
\glft ‘we didn’t have water either in Santa Rita’ (lit.: ‘like us, too, there in Santa Rita was no water’)
\endgl
\trailingcitation{[mqx-p110826l.103]}
\xe

In (\ref{ex:nena-12}), Juana compares her own state of being full to mine. I had just said before that I was ready with eating.

\ea\label{ex:nena-12}
\begingl 
\glpreamble nenaukunÿ metu\\
\gla nena-uku-nÿ metu\\ 
\glb like-\textsc{add}-1\textsc{sg} already\\ 
\glft ‘me, too, I am finished’\\ 
\endgl
\trailingcitation{[jxx-p120515l-2.262]}
\xe

One last example with \textit{nenauku} follows, including the complete statement. This is the closest possible equivalent to equality sentences in other languages. Juana speaks about how much her foster child and her daughter love her.

\ea\label{ex:nena-11}
\begingl
\glpreamble tesabichunÿ micha nimijÿna, \textup{(pause)} nijinepÿi Gladys nenauku, tisumachune micha\\
\gla ti-esabichu-nÿ micha ni-mijÿna ni-jinepÿi Gladys nena-uku ti-sumachu-ne micha\\
\glb 3i-estimate-1\textsc{sg} good 1\textsc{sg}-foster.child 1\textsc{sg}-daughter Gladys like-\textsc{add} 3i-want-1\textsc{sg} good\\
\glft ‘my foster child estimates me a lot, (pause) my daughter Gladys, too, she likes me a lot’
\endgl
\trailingcitation{[jxx-p110923l-1.212-214]}
\xe
\is{similative clause|)}

\subsection{Locomotion of third person}\label{sec:Kapunu}
\is{motion predicate|(}
\is{non-verbal motion clause|(}

Cislocative locomotion of third person participants is usually expressed with a non-verbal strategy in Paunaka. It builds on the word \textit{kapunu} ‘come’. There is also a \isi{verb} \textit{-bÿsÿu} ‘come’, but it is hardly ever used with a third person subject. Consider (\ref{ex:new23-come}), which clearly shows that \textit{kapunu} is not a verb. There is no person index on the predicate and the irrealis marker is \textit{-ina}.\is{non-verbal irrealis marker} The sentence was produced by Juana on my first visit to hers in 2015.

\ea\label{ex:new23-come}
\begingl
\glpreamble tajaitu kapunuina Maria\\
\gla tajaitu kapunuina Maria\\
\glb tomorrow come-\textsc{irr} María\\
\glft ‘María will come tomorrow’
\endgl
\trailingcitation{[jxx-p150920l.009]}
\xe

Non-verbal predication includes stativity, so it may sound strange that the volitional action of motion is expressed non-verbally. There is, however, a connection between locomotion and stativity and this is expressed in some way in several languages of very different language families around the world \citep[]{Payne2008}. The similarity derives from locomotion predicates encoding a change of place, situation or scene which is analogous to the change of state encoded by other stative predicates \citep[249]{Payne2008}. \citet[57, 113]{Payne1997} further states that locomotion may even be expressed non-verbally in some languages.

This does still not explain why non-verbal expression of locomotion is restricted to cislocative motion of third person participants in Paunaka. A look at closely related Trinitario\is{Mojeño Trinitario|(} sheds some light on this issue. Mojeño Trinitario has a non-verbal predication type called “motion-presentationals” by \citet[68]{Rose2018a}. This construction is used to introduce new participants into the discourse, just like the existential construction does,\is{existential clause} but with an additional notion of movement onto the scene. That is, while the existential construction can be often translated by ‘there/here is ...’, the motion-presentational construction expresses meanings like ‘there/here comes ...’. Both constructions are based on a personal or demonstrative pronoun in Trinitario to which a suffix (existential or motion copula) is added. 

As I have argued in \sectref{sec:DemPron} and \sectref{sec:NonVerbalPredication} above, the first syllable \textit{ka} of \textit{kapunu} is most probably a \is{demonstrative} root with third person reference. The rest is the \isi{associated motion} marker \textit{-punu} that encodes prior motion to and away from the deictic centre on non-motion verbs, but has exclusively cislocative semantics if combined with motion verbs (see \sectref{sec:punu}). Unlike in Trinitario, this marker never combines with personal pronouns\is{personal pronoun} in Paunaka, thus there are no non-verbal expressions for motion of first or second persons.

I would suggest that just like in the Trinitario case,\is{Mojeño Trinitario|)} a construction with \textit{kapunu} was once used to introduce new participants into the discourse only, but at some point, use of the predicate became independent from the presentational function. It thus developed into the default cislocative motion predicate for third person referents. This means that nowadays \textit{kapunu} can also occur in questions,\is{interrogative clause} it can be negated etc.\is{negation}

Consider (\ref{ex:kapunu-3}). The subject is not conominated here and it is not the place of arrival but of precedence that is of importance here. This shows that this is not a presentational construction anymore. The sentence was elicited from Juana.

\ea\label{ex:kapunu-3}
\begingl 
\glpreamble kapununube tukiu tÿbane\\
\gla kapunu-nube tukiu ti-ÿbane\\ 
\glb come-\textsc{pl} from 3i-be.far\\ 
\glft ‘they came from far away’\\ 
\endgl
\trailingcitation{[jmx-e090727s.320]}
\xe

In (\ref{ex:kapunu-new}), \textit{kapunu} is part of the antecedent clause of a conditional sentence. Thus no presentation is implied here. The sentence also comes from Juana, who was talking about a possible visit of her daughter to hers.

\ea\label{ex:kapunu-new}
\begingl
\glpreamble kue kapunuina parauna kuatruchÿ kuje\\
\gla kue kapunu-ina parau-ina kuatruchÿ kuje\\
\glb if come-\textsc{irr.nv} stop-\textsc{irr.nv} four month\\
\glft ‘if she comes, she stays four months’
\endgl
\trailingcitation{[jxx-p110923l-1.425]}
\xe

Nonetheless, there are also cases in which a presentational function is notable as in (\ref{ex:kapunu-2}), a statement by Clara about the weather.

\ea\label{ex:kapunu-2}
\begingl 
\glpreamble mm, kapunu ÿku\\
\gla mm kapunu ÿku\\ 
\glb mh come rain\\ 
\glft ‘mh, rain is coming’\\ 
\endgl
\trailingcitation{[cux-120410ls.257]}
\xe

 
Some more examples follow. (\ref{ex:kapunu-4}) comes from Miguel’s account about the history of Santa Rita.

\ea\label{ex:kapunu-4}
\begingl
\glpreamble i depueskuku, kuina naejumibu chijakena anyokena, kapunu padre Xeinaldo\\
\gla i depues-uku? kuina nÿ-a-ejumi-bu chija-kena anyo-kena kapunu {padre Xeinaldo}\\
\glb and afterwards-\textsc{add}? \textsc{neg} 1\textsc{sg}-\textsc{irr}-remember-\textsc{dsc} what-\textsc{uncert} year-\textsc{uncert} come {Father Reinaldo}\\
\glft ‘and also afterwards, I don’t remember anymore in which year, Father Reinaldo came’
\endgl
\trailingcitation{[mxx-p110825l.150-151]}
\xe

(\ref{ex:kapunu-new-2}) also comes from Miguel, who was conversing with Juana.\footnote{Don is a
respectful form of address in Spanish, which is used a lot in the region.}

\ea\label{ex:kapunu-new-2}
\begingl
\glpreamble rumingo kapunu unekoyae echÿu don Mario\\
\gla rumingo kapunu uneku-yae echÿu don Mario\\
\glb Sunday come town-\textsc{loc} \textsc{dem}b \textsc{hon} Mario\\
\glft ‘on Sunday, don Mario came to town’
\endgl
\trailingcitation{[jmx-c120429ls-x5.141]}
\xe

Unlike the copula \textit{kaku}, \textit{kapunu} is never omitted in negated sentences.\is{negation|(} One example is given in (\ref{ex:kapunuIRR-2}). It comes from Juana who was disappointed that her daughter did not visit her over Christmas.

\ea\label{ex:kapunuIRR-2}
\begingl 
\glpreamble kuina kapunuina nijinepÿi\\
\gla kuina kapunu-ina ni-jinepÿi\\ 
\glb \textsc{neg} come-\textsc{irr.nv} 1\textsc{sg}-daughter\\ 
\glft ‘my daughter didn’t come’\\ 
\endgl
\trailingcitation{[jxx-p120430l-1.317]}
\xe
\is{negation|)}

Just like other motion predicates (see \sectref{sec:Repetition}), \textit{kapunu} has a regressive\is{regressive/repetitive} \isi{derivation}, which is \textit{kapupunu} ‘come back’. This is illustrated by (\ref{ex:kapupunu}), which comes from María S. and is about me.\footnote{Actually, I \textit{did} come back, but only later, when my second child was born and had grown a little.} 

\ea\label{ex:kapupunu}
\begingl 
\glpreamble tichÿunumi kuina kapupunuinabu naka\\
\gla ti-chÿnumi kuina kapupunu-ina-bu naka\\ 
\glb 3i-be.sad \textsc{neg} come.back-\textsc{irr}-\textsc{dsc} here\\ 
\glft ‘she is sad, because she doesn’t come back here anymore’
\endgl
\trailingcitation{[rxx-e121128s-1.020]}
\xe
\is{non-verbal motion clause|)}
\is{motion predicate|)}

The existence of a non-verbal predicate with active semantics may have played a role in non-verbal integration of verbs borrowed from Spanish into Paunaka. This is the topic of the following section.


\subsection{Borrowed verbs}\label{sec:borrowed_verbs}\is{borrowing|(}
\is{non-verbal clause with borrowed verbs|(}
Although verbs borrowed from Spanish can be verbalised and then be used just like normal active verbs\is{active verb} in Paunaka (see \sectref{sec:ActiveVerbs_TH}), this is not the preferred pattern.\footnote{This section is based on \citet[]{Terhart_subm}, but provides some additional examples.} Speakers rather rely on integrating borrowed verbs as non-verbal predicates. No light verb is needed in order to accommodate these non-verbal predicates.\footnote{A light verb is a verb with relatively general semantics, such as ‘do’, which is used as a kind of auxiliary together with an uninflected form of the borrowed verb \citep[102]{Wohlgemuth2009}.} They are rather treated as if they were nouns or adjectives (see \sectref{sec:PropInclEquatAttr} above), i.e. they take person markers that follow the predicate to index the \isi{subject}\is{person marking|(} and the \isi{non-verbal irrealis marker} \textit{-ina} in contexts that demand irrealis RS.

This can be seen in (\ref{ex:PTCP-Person}), where the borrowed form \textit{komorau}, from Spanish \textit{acomodar} ‘accomodate, arrange’, takes a second person singular marker following the predicate to index the subject and the non-verbal irrealis marker for future reference. It comes from Juana and refers to me packing my stuff shortly before I would fly back to Germany.

\ea\label{ex:PTCP-Person}
\begingl 
\glpreamble metu komoraubinatu\\
\gla metu komorau-bi-ina-tu\\ 
\glb already accommodate-2\textsc{sg}-\textsc{irr.nv}-\textsc{iam}\\ 
\glft ‘you are already going to arrange (your stuff)’\\ 
\endgl
\trailingcitation{[jxx-p120515l-2.275]}
\xe
\is{person marking|)}

In most cases the input form, i.e. the form of the original verb which is borrowed \citep[cf.][]{Wohlgemuth2009}, is based on a Spanish past participle in \textit{-ado}, which is pronounced [ao̪] in Eastern Bolivia. Some examples are listed in \tabref{table:ptcps_ado}.\footnote{In addition, borrowed participles are sometimes also used adverbially like in Spanish, consider (\ref{ex:Borri-fn}) which has \textit{purau} from \textit{apurar(se)} ‘hurry up’ : 

\ea\label{ex:Borri-fn}
\begingl
\glpreamble purau tikubu\\
\gla purau ti-kubu\\
\glb hurry 3i-bath\\
\glft ‘she bathed quickly’
\endgl
\trailingcitation{[jxx-p120515l-2.152]}
\xe}

\begin{table}
\caption{Paunaka loans of past participles in \textit{-ado}}

\begin{tabularx}{\textwidth}{lllQ}
\lsptoprule
{Spanish infinitive} & {Spanish participle} & {Paunaka loan} & {Translation}\cr
\midrule
\textit{apostar} & \textit{apostado} & \textit{apostau} & bet\cr
\textit{ayudar} & \textit{ayudado} & \textit{ayurau} & help\cr
\textit{comenzar} & \textit{comenzado} & \textit{komensau} & begin\cr
\textit{ganar} & \textit{ganado} & \textit{kanau} & win\cr
\textit{mandar} & \textit{mandado} & \textit{mandau} & send\cr
\textit{multiplicar} & \textit{multiplicado} & \textit{multiplikau} & multiply\cr
\textit{olvidar} & \textit{olvidado} & \textit{arbidau/arbirau} & forget\cr
\textit{pasar} & \textit{pasado} & \textit{pasau} & pass by; happen, pass\cr
\textit{regalar} & \textit{regalado} & \textit{regalau} & give as present\cr
\lspbottomrule
\end{tabularx}

\label{table:ptcps_ado}
\end{table}

For verbs that do not provide participles in \textit{-ado}, i.e. the ones with an infinitive ending in \textit{er} and \textit{ir}, there exist different strategies. Either a past participle ending in \textit{-ido} is the input form -- Paunaka then borrows a form ending in \textit{-iru} or \textit{-iu} --, or a reduced infinitive is borrowed. The reduced infinitive is the Spanish infinitive minus the final \textit{r}, a form that is often borrowed by American languages in contact with Romance languages \citep[170]{Wohlgemuth2009}. In addition, there are some minor strategies encountered with only one or two verbs, e.g. the predicate \textit{trabaku} ‘work’ seems to be derived from the noun \textit{trabajo} ‘work’. It can also be used nominally in Paunaka (e.g. a possessed form can be derived\is{derivation} by addition of the possessed marker \textit{-ne}).

In \citet[8]{Terhart_subm}, I developed the hypothesis that the predicate \textit{kompirau} ‘share, invite’ has evolved from the Spanish verb \textit{compartir} ‘share’, whose participle is \textit{compartido}. In my argumentation, speakers would have metathesised the vowels of the last two syllables of the Spanish infinitive in order to arrive at the form \textit{kompirau}, yielding first *\textit{compirtar} and then a participle *\textit{compirtado} with the preferred ending in \textit{-ado}. However, as Nikulin (2020, p.c.) has pointed out, the input verb is possibly not \textit{compartir} ‘share’ but \textit{convidar} ‘invite’ with the participle \textit{convidado}, thus no metathesis is involved. I deem it possible that both input verbs merged in the Paunaka predicate. This would explain the sequence /mp/ in \textit{kompirau} as well as the fact that both meanings ‘share’ and ‘invite’ can be realised by this form. \textit{Kompirau} is verbalised in most cases, but one example in which it is used non-verbally is (\ref{ex:Borri-1}). Note that the predicate irregularly takes a third person marker in this case. This is usually excluded in non-verbal predication. The example comes from Miguel telling the story about the cowherd and the spirit of hill. After the spirit has taken away the cows of the man, the latter agrees to reside with the spirit in his world. Towards the end of the story, the spirit suggests that the cows can be given to the people of a village. In order to bring the cows there, the cowherd needs some help.

\ea\label{ex:Borri-1}
\begingl 
\glpreamble tupunuji kompirauchituji sinko jentenube\\
\gla ti-upunu-ji kompirau-chi-tu-ji sinko jente-nube\\ 
\glb 3i-bring-\textsc{rprt} share-3-\textsc{iam}-\textsc{rprt} five man-\textsc{pl}\\ 
\glft ‘he brought five men to share (the workload), it is said’\\ 
\endgl
\trailingcitation{[mxx-n151017l-1.81]}
\xe

Another interesting feature of borrowed non-verbal predicates is their possibility to be used transitively.\is{transitivity} The \isi{object} is expressed by an NP in this case. An example is (\ref{ex:Borri-4}), which comes from Miguel’s account about how he learned to calculate. It was a young man doing military service together with Miguel who taught him, but first of all, Miguel had to register for military service:

\ea\label{ex:Borri-4}
\begingl
\glpreamble konsegiunÿtu echÿu niribretane\\
\gla konsegiu-nÿ-tu echÿu ni-ribreta-ne\\
\glb obtain-1\textsc{sg}-\textsc{iam} \textsc{dem}b 1\textsc{sg}-military.registration.document-\textsc{possd}\\
\glft ‘I obtained my military registration document’
\endgl
\trailingcitation{[mxx-p181027l-1.114]}
\xe


In (\ref{ex:PTCP-OBJ-HUM}), a theme object\is{patient/theme} is expressed by an NP and a recipient\is{recipient|(} participant is additionally added to the clause with the help of the oblique preposition \textit{-tÿpi}.\is{general oblique} This example comes from María S. and is about her plans to write back to me after I had sent her greetings via Swintha. She actually produced this sentence in Spanish first and translated it on request.

\ea\label{ex:PTCP-OBJ-HUM}
\begingl 
\glpreamble mandaubina karta chitÿpiuku\\
\gla mandau-bi-ina karta chi-tÿpi-uku\\ 
\glb send-1\textsc{pl}-\textsc{irr.nv} letter 3-\textsc{obl}-\textsc{add}\\ 
\glft ‘we will send her a letter, too’\\ 
\endgl
\trailingcitation{[rxx-e121128s-1.115]}
\xe


(\ref{ex:Borri-2}) has another recipient participant that is encoded with the oblique preposition.\is{general oblique} This is a sentence by Juana about some coffee from Argentina which some friends of her daughter had given her. Note that the borrowed verb does not take a plural marker here, which is unusual, since there is a plural subject (which is clear from the context).

\ea\label{ex:Borri-2}
\begingl
\glpreamble regalau nitÿpi\\
\gla regalau ni-tÿpi\\
\glb give.as.present 1\textsc{sg}-\textsc{obl}\\
\glft ‘they gave it to me as a present’
\endgl
\trailingcitation{[jxx-e120430l-4.29]}
\xe
\is{recipient|)}

Borrowed non-verbal predicates can be used in complex clauses. In (\ref{ex:PTCP-complement}), \textit{trabaku} ‘work’ is the complement of a desiderative verb, in (\ref{ex:PTCP-OBJ}) we have a construction that resembles the serial verb construction but with a non-verbal predicate as the second predicate (i.e. a serial predicate construction).

\ea\label{ex:PTCP-complement}
\begingl 
\glpreamble nijinepÿi kuina tisacha trabakuneina\\
\gla ni-jinepÿi kuina ti-sacha trabaku-ne-ina\\ 
\glb 1\textsc{sg}-daughter \textsc{neg} 3i-want work-1\textsc{sg}-\textsc{irr.nv}\\ 
\glft ‘my daughter doesn't want me to work’\\ 
\endgl
\trailingcitation{[jxx-n101013s-1.193-194]}
\xe

\ea\label{ex:PTCP-OBJ}
\begingl 
\glpreamble eka semana niyuna kontratauneina chÿnachÿ makina\\
\gla eka semana ni-yuna kontratau-ne-ina chÿnachÿ makina\\ 
\glb \textsc{dem}a week 1\textsc{sg}-go.\textsc{irr} engage-1\textsc{sg}-\textsc{irr.nv} one machine\\ 
\glft ‘this week I will hire a machine’\\ 
\endgl
\trailingcitation{[jxx-p120515l-2.106]}
\xe


There is also one modal non-verbal predicate\is{modality|(} borrowed from the Spanish modal verb \textit{poder} ‘can, be able to’. In Paunaka, its form is \textit{puero},\is{knowledge/ability predicate|(} and it has possibly been borrowed via \isi{Bésiro}, which has a noun \textit{puéru} ‘possibility’ and a verb \textit{puérux} ‘can, be able to’, which is derived from that noun \citep[cf.][]{Sans2011}.

\textit{Puero} is exceptional insofar as that it usually does not take subject indexes,\is{person marking} although a few cases with a first person singular marker do occur. If used together with another predicate, it is also not necessarily marked for irrealis in irrealis contexts.\is{non-verbal irrealis marker} If used alone (e.g. as an answer to a question), it does take the irrealis marker in these contexts. \textit{Puero} is primarily used in negative contexts,\is{negation|(} since irrealis alone is sufficient to indicate a permissive or abilitive reading in Paunaka. In negation of a permissive or abilitive constructions, however, two factors trigger irrealis marking, so that speakers may feel the need to be more explicit about the modal meaning. This is reminiscent of the \isi{doubly irrealis construction} found in other Arawakan languages \citep[cf.][271]{Michael2014}, the difference being that in Paunaka, the fact that two parameters trigger irrealis is expressed by a lexical rather than morphological means. Two examples of \textit{puero} follow, one with and the other one without irrealis marking on \textit{puero}. Both come from María S.

(\ref{ex:Borri-6}) was elicitated. It refers to an imagined old man.

\ea\label{ex:Borri-6}
\begingl
\glpreamble kuina pueroina tiyuna asaneti\\
\gla kuina puero-ina ti-yuna asaneti\\
\glb \textsc{neg} can-\textsc{irr.nv} 3i-go.\textsc{irr} field\\
\glft ‘he cannot go to the field’
\endgl
\trailingcitation{[rxx-e181022le]}
\xe

(\ref{ex:Borri-5}) is a statement by María S. about herself. She had a bad knee by that time.

\ea\label{ex:Borri-5}
\begingl
\glpreamble kuina puero niyuika kasi\\
\gla kuina puero ni-yuika kasi\\
\glb \textsc{neg} can 1\textsc{sg}-walk.\textsc{irr} almost\\
\glft ‘I almost cannot walk’
\endgl
\trailingcitation{[rxx-e181017l.011]}
\xe
\is{negation|)}
\is{knowledge/ability predicate|)}

In addition, there is also \textit{tiene ke} ‘must’ (from Span. \textit{tiene que} ‘he/she/it has to’), but this one is used very infrequently. It also occurs in \isi{Bésiro} \citep[cf. Bésiro text examples in][47--70]{Sans2013}. One example is (\ref{ex:must-borr-1}) from Clara, who refers to the excursion to \isi{Altavista} which Swintha, Federico and I had planned.

\ea\label{ex:must-borr-1}
\begingl
\glpreamble pero esachu eyuna tiene ke tiyunakena Miyel\\
\gla pero e-sachu e-yuna {tiene ke} ti-yuna-kena Miyel\\
\glb but 2\textsc{pl}-want 2\textsc{pl}-go.\textsc{irr} must 3i-go.\textsc{irr}-\textsc{uncert} Miguel\\
\glft ‘but if you want to go, Miguel probably has to go as well’
\endgl
\trailingcitation{[cux-c120414ls-1.139]}
\xe
\is{modality|)}

The use of borrowed verbs as non-verbal predicates is surprising, because it links the encoding of events and actions to non-verbal predication, although this is usually closely connected to verbal predication (e.g. \citealt[189, 244]{Langacker1987}; \citealt[140, 142]{Frawley1992}; \citealt[82--83]{VanValinLaPolla1997}; \citealt[52]{Givon2001}).

Considering native structures only, non-verbal predication in Paunaka covers stative relationships, but the insertion of borrowed Spanish verbs has extended the semantic scope to include also active relationships. It might be the case though that prior to this, the non-verbal predicate \textit{kapunu} ‘come’ was already used with active semantics, thus facilitating the integration of borrowed verbs in a similar way. Furthermore, the integration of borrowed verbs as non-verbal predicates might be an areal feature. Consider the case of Bésiro.\is{Bésiro|(} In this language, verbs obligatorily take prefixes to index the subject and they take enclitics to index objects. Nominal and adjectival predicates take enclitics to index a subject, and so do some borrowed verbs. Between the input form and the enclitic, a suffix \textit{-bo} is inserted (Sans 2012, p.c.). Note, however, that Bésiro borrows reduced infinitives instead of participles, and some borrowed verbs seem to be verbalised rather than used non-verbally \citep[cf. Bésiro texts in][47--70]{Sans2013}. It remains unclear, for the time being, how frequent the borrowing of verbs as non-verbal predicates is in Bésiro.\is{Bésiro|)}

As for the borrowing of participles, this input form could also have been preferred by speakers of the Chapacuran language Kitemoka \citep[cf. ex. KIT1 739 in][96]{Wienold2012}. With only one example of a borrowed Spanish verb in the Kitemoka corpus and the little knowledge about Kitemoka in general, we cannot, unfortunately, make any statements about verbal or non-verbal character of the borrowed item.\is{borrowing|)}
\is{non-verbal clause with borrowed verbs|)}
\is{non-verbal predication|)}

While this chapter has focused on different kinds of declarative clauses up to here,\is{declarative clause|)} the remaining two sections are dedicated to other speech acts: directives and interrogatives. The next section starts with a discussion of imperatives and other kinds of directives.


