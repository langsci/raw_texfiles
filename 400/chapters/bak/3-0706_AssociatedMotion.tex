%!TEX root = 3-P_Masterdokument.tex
%!TEX encoding = UTF-8 Unicode

\section{Associated motion, associated path, and regressive/ repetitive}\label{sec:AssociatedMotion}\is{associated motion|(}

In addition to having motion verbs, Paunaka possesses some means to express motion through morphemes attached to other verbs, e.g. the marker \textit{-punu}, which encodes motion prior to the event expressed by the verb, as in (\ref{ex:go-sleep}) from elicitation with Miguel.

\ea\label{ex:go-sleep}
\begingl 
\glpreamble ubiayae nimukupunu\\
\gla ubiae-yae ni-muku-punu\\ 
\glb house-\textsc{loc} 1\textsc{sg}-sleep-\textsc{am.prior}\\ 
\glft ‘I went to the house to sleep’\\ 
\endgl
\trailingcitation{[jmx-e090727s.348]}
\xe

%chijaririkupunuji kupisaÿrÿ, jmx-n120429ls-x5.452 Miguel

The morphological expression of motion on a verb has been treated under the term of “associated motion” (AM) in the literature and is the topic of this section. Furthermore, the \isi{regressive/repetitive} marker is described here because it derives from an AM marker.\is{derivation} The category of AM is not expressed very frequently in general.

AM markers occur very close to the verb stem.\is{verbal stem} On active verbs, they directly follow the \isi{thematic suffix}.\is{active verb} The prior motion marker can even replace the \isi{thematic suffix}, see \figref{fig:VerbTemplate-Active} in the introduction to this chapter. This and the fact that AM markers are the locus of RS\is{reality status} inflection (see \sectref{sec:RealityStatus}) make them reminiscent of derivational\is{derivation} affixes, but AM is commonly described as an inflectional category in South American languages \citep[cf.][]{Guillaume2016}. AM markers form a small paradigm in Paunaka and this suggests that they are indeed inflectional. They are listed in \tabref{table:AM_markers}, which additionally shows the other markers dealt with in this section.

\begin{table}[htbp]
\caption{Associated motion and related markers}

\begin{tabularx}{\textwidth}{llQ}
\lsptoprule
Suffix (realis) & Gloss & Function \\
\midrule
\textit{-(CV)kÿu} & \textsc{am.conc.tr} & concurrent translocative motion\cr
\textit{-(CV)kÿupunu} & \textsc{am.conc.cis} & concurrent cislocative motion\cr
\textit{-punu} & \textsc{am.prior} & prior motion (translocative and cislocative), directional (cislocative)\cr
\textit{-nÿmu} & \textsc{am.subs}? & possibly subsequent motion\cr
\textit{-pu} & \textsc{dloc} & dislocative, marks purpose verb of motion-cum-purpose constructions, adds path component to semantics of verb\cr
\textit{-punuku} & \textsc{reg} & regressive (motion back to a point) on motion verbs, repetitive on other verbs\cr
\lspbottomrule
\end{tabularx}

\label{table:AM_markers}
\end{table}

The prior motion marker, the subsequent motion marker, and the dislocative\is{dislocative|(} marker have cognates in other \isi{Arawakan languages} of the Southern, Kampan and Purus branches \citep[cf.][131--138 for an overview]{Guillaume2016}, and a suffix \mbox{\textit{*-ape}} ‘directional, arriving, approaching, motion’ has been reconstructed for Proto-Arawakan \citep[cf.][380]{Payne1991}. This could be the source of the prior motion marker and/or the dislocative marker; however, Rose (2021, p.c.) also reports an obsolete verbal root \textit{*-po} ‘go’ in Mojeño.\is{Mojeño languages}\is{dislocative|)} The concurrent motion markers of Paunaka do not have cognate forms in any other Southern or more distantly related Arawakan language as far as I know.\footnote{Trinitario\is{Mojeño Trinitario} has several concurrent motion markers, but none of them is related to a sequence \textit{-kÿu} \citep[cf.][135]{Rose2015}.}
% Proto-Arawakan: \citet[381]{Payne1991}:  \textit{*-ane} ‘directional, leaving’ ;  \textit{*-ake} \textit{(-akʰe)} for ‘directional, go to do X’.
% Trinitario: \citet[140]{Rose2015}: \textit{-pono} ‘reversive interrupted motion marker’, i.e. it encodes prior motion to a point where the action is performed and subsequent motion back
% Old Baure: \citet[14--15]{Magio1880} and \citet[72]{AsisCoparcari1880}:  \textit{-pono} or \textit{-pone}, irrealis \textit{-pana}, \textit{-pa} or \textit{-papo}  ‘translocative prior motion’, "supine marker"
% Baure: \citet[257--258]{Danielsen2007}: \textit{-wana} ‘subsequent motion’
% Nanti:  Michael 2008:266: \textit{-apanaa} ‘in passing’ and \textit{-apanu} 'in passing (round-trip trajectory)’ 

%order of suffixes: bÿku-pa-nube but -bÿ-jane-pu! cf. mrx-e150219s.101

The term “associated motion” was first used in the description of Australian languages \citep[cf.][]{Koch1984,Wilkins1991}, and the grammatical category has recently also been described for several languages in South America \citep[e.g.][]{Guillaume2000,Guillaume2013,Guillaume2016,Fabre2013,Vuillermet2013,Rose2015} and other parts of the world \citep[e.g.][]{OConnor2007,Jacques2013}.

\citet[]{Guillaume2013,Guillaume2016} offers two definitions of the category that can be categorised as a broad and a narrow definition. According to the broad definition, “[a]n AM marker is a grammatical morpheme that is associated with the verb and that has among its possible functions the coding of translational motion” \citep[92]{Guillaume2016}. The narrow definition defines AM markers as “grammatical markers that attach to non-motion verbs and specify that the verb action occurs against the background of a motion event with a specific orientation in space” \citep[131]{Guillaume2013}.

The distinction into broad and narrow definitions works well for Paunaka: the two markers of concurrent motion can be defined as AM markers according to the narrow definition, see \sectref{sec:AMconcurrent}, but the prior motion marker can only be defined as an AM marker according to the broad definition, see \sectref{sec:punu}. There is one marker that possibly encodes subsequent motion, which is described in \sectref{sec:SubsequentMotion}. In addition, the \isi{dislocative} marker very much resembles the prior motion marker because it appears in the same grammatical contexts. Nonetheless, it cannot be defined an AM marker with certainty because it does not itself encode motion, but rather emphasises the path and goal component of a motion event or adds such a component. However, one of the speakers also uses the marker to encode translocative prior motion, when the intention to do something in another place is foregrounded and motion plays a minor role. This is discussed in more detail in \sectref{sec:PA}.

Unlike other Amazonian languages \citep[cf.][]{Guillaume2016}, Paunaka has no markers to encode motion of the object. The markers described in this section exclusively encode motion of the \isi{subject}. The only exception are two verbs, in which two of the markers, \textit{-punu} and \textit{-pu}, are lexicalised.\is{lexicalisation} Those two verbs encode motion of the object and are described in \sectref{sec:OBJ-AM}.

\citet[83]{Guillaume2016} proposed two hierarchies for AM marking, which are given in \figref{fig:AMHierarchy}.

\begin{figure}[!ht]
\centering
\begin{enumerate}
\item motion of the subject > motion of the object
\item prior motion > concurrent motion > subsequent motion
\end{enumerate}
\caption{AM hierarchies after \citet[83]{Guillaume2016}}
\label{fig:AMHierarchy}
\end{figure}

The first scale thus definitely holds for Paunaka: All markers encode motion of the \isi{subject}, motion of the object being restricted to two specific verbs.
The second scale also holds for Paunaka; however, Paunaka has more fine-grained expressions for concurrent motion than for prior motion. It shares this characteristic with Trinitario.\is{Mojeño Trinitario}\footnote{Trinitario, according to the analysis by \citet[140]{Rose2015}, does not have markers to express prior motion at all except for \textit{-pono}, which encodes prior and subsequent motion together. \citet[108, 117--118]{Guillaume2016} nonetheless counts Trinitario’s \textit{-pono} as a prior motion marker and argues that it may rather be motion to a temporary location than return motion that is encoded.} 

This chapter concludes with \sectref{sec:Repetition}, which deals with the different possibilities to mark regression and repetition on predicates, with the regressive marker definitely being derived from the prior motion marker.


\subsection{Concurrent motion}\label{sec:AMconcurrent}

Both \textit{-(CV)kÿu} and \textit{-(CV)kÿupunu} can be clearly defined as AM markers, because they are not used for any other purpose than AM marking. They both encode concurrent motion, i.e. motion that happens simultaneously with the action expressed by the predicate, but differ in deixis.

The marker \textit{-(CV)kÿu} consists of a sequence \textit{kÿu} that is attached to the reduplicated\is{reduplication} last syllable of the verb stem in the majority of cases. Since most active verb stems end in the \isi{thematic suffix} \textit{-ku}, the most frequent form of the concurrent motion marker is \textit{-kukÿu}, but there are also cases in which another last syllable is reduplicated as is the case in (\ref{ex:kukÿu-manner-1}) below. The reduplicated syllable may also be dropped so that the bare form \textit{-kÿu} shows up, but there are only very few occurrences of this in the corpus. %NOTE TO SELF: it might be the case that the thematic suffix -chu is also dropped, but there are not enough examples (muyayakÿu = andar despacingo)

The marker \textit{-(CV)kÿu} is used to express concurrent translocative motion, i.e. motion away from the deictic centre.\footnote{Other terms to express motion away from the deictic centre are itive and andative.} It attaches to non-motion verbs, active and stative ones alike. 

(\ref{ex:kukÿu-non.mot-1}) shows the marker attached to an active non-motion verb. There is no (other) motion verb in this clause, the only marker of motion being the AM marker, which encodes that the actions of talking and moving (walking in this case) happen simultaneously. The sentence comes from Juana’s description of their grandparents’ journey to Moxos and back home. They had bought cows in Moxos.

\ea\label{ex:kukÿu-non.mot-1}
\begingl 
\glpreamble chichujikukukÿunube chipeunube baka\\
\gla chi-chujiku-kukÿu-nube chi-peu-nube baka\\ 
\glb 3-speak-\textsc{am.conc.tr}-\textsc{pl} 3-animal-\textsc{pl} cow\\ 
\glft ‘they went talking to their cows’\\ 
\endgl
\trailingcitation{[jxx-p151016l-2]}
\xe


In addition, the marker can also attach to  manner of motion verbs,\is{motion predicate|(} where it adds a certain translational reading, as in the following example from the same story as the prior example. There were heavy rainfalls and the grandparents came to an arroyo, which they had to cross swimming.

\ea\label{ex:kukÿu-manner-1}
\begingl 
\glpreamble tubuejijikÿubu pasaunube\\
\gla ti-ubueji-jikÿu-bu pasau-nube\\ 
\glb 3i-swim-\textsc{am.conc.tr}-\textsc{mid} pass-\textsc{pl}\\ 
\glft ‘swimming they went and passed (the arroyo)’\\ 
\endgl
\trailingcitation{[jxx-p151016l-2]}
\xe

The verb \textit{-ubueji} ‘swim’ in the example above is a stative verb\footnote{It is composed of the defective verb \textit{-ubu} ‘be, live’, the \isi{classifier} \textit{-e} ‘\textsc{clf}:water’, and the sequence \textit{-ji} of unclear origin.} and could therefore arguably be defined as a verb that expresses a stative location ‘be in water’ rather than motion ‘swim’. However, the AM marker also attaches to clearly active  manner of motion verbs like \textit{-bÿbÿku} ‘fly’ in (\ref{ex:kukÿu-manner-2}) and \textit{-yuiku} ‘walk’ in (\ref{ex:kukÿu-manner-3}).

(\ref{ex:kukÿu-manner-2}) comes from the story by Miguel about a lazy man. At the end of the story, he sacrifices himself, cutting off his arms and legs, and lets his son throw him into a pond. From there, he rises to the sky as a comet, which comes back every year.\footnote{As for the marker \textit{-ni} at the end of the \isi{demonstrative verb} \textit{chikuyeni}, it is a deictic element attaching to demonstratives\is{demonstrative} of different word classes that seems to add some emphasis. It occurs very infrequently and is predominantly used by María C., but also sometimes by others. Its exact meaning or function could not be determined.}

\ea\label{ex:kukÿu-manner-2}
\begingl 
\glpreamble kada anyo kue pimua echÿu pasauna chikuyeni tibÿbÿkukukÿu anÿke...\\
\gla kada anyo kue pi-mua echÿu pasau-ina chi-kuye-ni ti-bÿbÿku-kukÿu anÿke\\ 
\glb each year if 2\textsc{sg}-see.\textsc{irr} \textsc{dem}b pass-\textsc{irr.nv} 3-be.like.this-\textsc{deict} 3i-fly-\textsc{am.conc.tr} up\\ 
\glft ‘every year, when you see something passing by there like moving and flying above...’\\ 
\endgl
\trailingcitation{[mox-n110920l.128]}
\xe
\is{motion predicate|)}

The AM marker in the previous example expresses that the flight follows a certain route, which is also encoded by the Spanish loan \textit{pasau}, which has several translations into English, of which ‘pass by’ is the most accurate here. Compare this example to (\ref{ex:random-flight}), where Juana does not emphasise a route but rather the ability of the bird to fly, and thus does not use the concurrent AM marker on the predicate.

\ea\label{ex:random-flight}
\begingl 
\glpreamble i echÿu yÿnÿ tibÿbÿku anÿke\\
\gla i echÿu yÿnÿ ti-bÿbÿku anÿke\\ 
\glb and \textsc{dem}b jabiru 3-fly up\\ 
\glft ‘and the jabiru flies up in the sky’\\ 
\endgl
\trailingcitation{[jxx-a120516l-a.253]}%non-e
\xe

The focus on route is prevalent in the following example, where Miguel describes the motion of a wooden toy cow that I move along my notebook, not randomly, but following a route from one point on the notebook to another.

\ea\label{ex:kukÿu-manner-3}
\begingl 
\glpreamble mm, tiyuikukukÿu\\
\gla mm ti-yuiku-kukÿu\\ 
\glb \textsc{intj} 3i-walk-\textsc{am.conc.tr}\\ 
\glft ‘hm, it goes walking’\\ 
\endgl
\trailingcitation{[mox-e110914l-1.150]}
\xe

Irrealis is expressed on the AM marker rather than on the verb stem. The irrealis form of the marker is \textit{-(CV)kÿa} as can be seen in (\ref{ex:kukÿa-1}) (second verb in the example), where irrealis is due to future reference of the predicate. The sentence comes from the recordings by Riester with Juan Ch.

\ea\label{ex:kukÿa-1}
\begingl 
\glpreamble nipubÿchupunu ruschÿmÿnÿ charutu chijibÿkukukÿa\\
\gla ni-pu-bÿu-chu-punu ruschÿ-mÿnÿ charutu chi-jibÿku-kukÿa\\ 
\glb 1\textsc{sg}-give-hand-\textsc{th}2-\textsc{am.prior} two-\textsc{dim} cigar 3-smoke-\textsc{am.conc.tr.irr}\\ 
\glft ‘I came and put in his hand two cigars so that he would go smoking them’\\ 
\endgl
\trailingcitation{[nxx-p630101g-1.035-036]}
\xe


The marker \textit{-(CV)kÿupunu} similarly attaches to non-motion and  manner of motion verbs,\is{motion predicate} the difference from \textit{-(CV)kÿu} being the deixis expressed: while \textit{-(CV)kÿu} is translocative, \textit{-(CV)kÿupunu} is cislocative (or venitive). \textit{-(CV)kÿupunu} can be decomposed into an optional reduplicated syllable,\is{reduplication} the sequence \textit{kÿu} and \textit{-punu}, which is itself an AM marker used to express prior motion among other things, see \sectref{sec:punu}.\footnote{Alternatively, \textit{-(CV)kÿupunu} could also be analysed as a conventionalised sequence of two markers, the concurrent motion marker \textit{-(CV)kÿu} and the prior motion marker \textit{-punu} in its directional sense.} 

The following example shows the cislocative concurrent motion marker on a non-motion verb realised as a headless relative clause (\sectref{sec:HeadlessRC}). The “shouters” are moving towards the deictic centre. In this case this is a village, or some people within this village, who are visited by some of their former neighbours that had been enchanted by the spirit of the hill. The villagers help them drive the cows (shouting), but cannot see them, because they are invisible, having turned into ghosts. The sentence comes from a story by Miguel.

\ea\label{ex:kÿupunu-non.mot-2}
\begingl 
\glpreamble tosetuji chisamunubetuji echÿu tiyÿbuikÿupununubetuji\\
\gla tose-tu-ji chi-samu-nube-tu-ji echÿu ti-yÿbui-kÿupunu-nube-tu-ji\\ 
\glb noon-\textsc{iam}-\textsc{rprt} 3-hear-\textsc{pl}-\textsc{iam}-\textsc{rprt} \textsc{dem}b 3i-shout-\textsc{am.conc.cis}-\textsc{pl}-\textsc{iam}-\textsc{rprt}\\ 
\glft ‘when it turned twelve, they heard the ones who came shouting, it is said’ \\ 
\endgl
\trailingcitation{[mxx-n151017l-1.86]}
\xe
 
In the following example, (\ref{ex:kÿupunu-stative}), the cislocative concurrent motion marker is attached to a stative verb. Miguel is commenting about his brother José, whom Swintha and Miguel wanted to visit at his home, which is outside Santa Rita’s village centre. The fact that people do not pass by randomly gives José certain freedom to dress or undress according to his mood – and the weather, of course.

\ea\label{ex:kÿupunu-stative}
\begingl 
\glpreamble kapunu, tisukerepÿikÿupunu\\
\gla kapunu ti-sukere-pÿi-kÿupunu\\ 
\glb come 3i-be.naked-body-\textsc{am.conc.cis}\\ 
\glft ‘he comes, naked he comes’\\ 
\endgl
\trailingcitation{[mox-c110926s-1.107]}
\xe

There are no cases of \isi{irrealis} marking on the cislocative concurrent motion marker in the corpus, but I suppose this is a coincidence.
 
Both concurrent motion markers express associated motion in the narrower sense as “grammatical markers that attach to non-motion verbs and specify that the verb action occurs against the background of a motion event with a specific orientation in space” \citep[131]{Guillaume2013}. The only point in which they deviate from the definition is that they also attach to  manner of motion verbs.\is{motion predicate} However, manner of motion verbs are also among the targets of AM markers in \isi{Mojeño Trinitario} \citep[131]{Rose2015}. An investigation of the ability of (some?) AM markers to attach to  manner of motion verbs would thus be an interesting topic for future research.


\subsection{Prior motion}\label{sec:punu}

The marker \textit{-punu} (irrealis \textit{-puna}) has among its functions the expression of prior motion in relation to the predicate. Furthermore, the marker is lexicalised\is{lexicalisation} on a number of motion verbs,\is{motion predicate} and one non-verbal motion predicate.\is{non-verbal predication} It can also be used as a directional and as a purpose marker together with another motion predicate.

If \textit{-punu} attaches to a verb stem,\is{verbal stem} the thematic suffix \is{thematic suffix|(} \textit{-ku} is often dropped as in (\ref{ex:punu-no-ku}). The verb is commonly realised as \textit{-niku} ‘eat’, but appears here without the thematic suffix. The action of eating follows the action of coming; thus it is an example of \textit{-punu} with the function of expressing prior motion.

The example comes from Clara, who suggests that María C. could speak a little Paunaka with her son, using for instance this sentence: 

\ea\label{ex:punu-no-ku}
\begingl 
\glpreamble pinipuna nichechapÿibi\\
\gla pi-ni-puna ni-chechapÿi-bi\\ 
\glb 2\textsc{sg}-eat-\textsc{am.prior} 1\textsc{sg}-son-2\textsc{sg}\\ 
\glft ‘come and eat, my dear son’\\ 
\endgl
\trailingcitation{[cux-c120414ls-2.302]}
\xe

In (\ref{ex:go-cow}), however, the AM marker follows the thematic suffix of the verb \textit{-semaiku} ‘search, look for’. This example comes from Miguel telling Swintha what he was doing the afternoon. Some cows had escaped from the enclosure.

\ea\label{ex:go-cow}
\begingl
\glpreamble nisemaikupunu echÿu bakajane\\
\gla ni-semaiku-punu echÿu baka-jane\\
\glb 1\textsc{sg}-search-\textsc{am.prior} \textsc{dem}b cow-\textsc{distr}\\
\glft ‘I went to look for the cows’
\endgl
\trailingcitation{[mxx-n101017s-2.072]}
\xe\is{thematic suffix|)}


(\ref{ex:punu-no-ku}) and (\ref{ex:go-cow}) also show that the direction of motion is not specified. That is, motion can be cislocative, as is the case in (\ref{ex:punu-no-ku}), as well as translocative, as we have seen in (\ref{ex:go-cow}).

I give two more examples that show the AM function of the suffix with different directions, cislocative in (\ref{ex:punu-cis-1}) and translocative in (\ref{ex:punu-trans-1}).

(\ref{ex:punu-cis-1}) comes from a correction session with Juana. She re-enacts what she heard Juan Ch. say in one of the recordings by Riester.

\ea\label{ex:punu-cis-1}
\begingl 
\glpreamble micha, kumare, etibupuna\\
\gla micha kumare e-tibu-puna\\ 
\glb good fellow 2\textsc{pl}-sit.down-\textsc{am.prior.irr}\\ 
\glft ‘I am fine, fellow, come and sit down’\\ 
\endgl
\trailingcitation{[jxx-p120430l-2.039]}
\xe

In (\ref{ex:punu-trans-1}), a cislocative reading is not possible; the sentence refers to Clara’s going to her son's building plot to remove the weed there. It was uttered by María C. At the time of the utterance, we were all sitting in Clara’s yard. María C. had not accompanied Clara to the building plot, nor is there any connection of hers to the building plot that would make a cislocative reading plausible.

\ea\label{ex:punu-trans-1}
\begingl 
\glpreamble chiyaemÿnÿ lote chisupunu, chubiuna chichechapÿi\\
\gla chi-yae-mÿnÿ lote chi-isu-punu chÿ-ubiu-ina chi-chechapÿi\\ 
\glb 3-\textsc{grn}-\textsc{dim} plot 3-weed-\textsc{am.prior} 3-house-\textsc{irr.nv} 3-son\\ 
\glft ‘she went to weed his building plot, for the future house of her son’\\ 
\endgl
\trailingcitation{[cux-c120414ls-1.98-101]}
\xe

As an AM marker, \textit{-punu} is usually attached to non-motion verbs, but I have also found it on the demonstrative adverb \textit{naka}. Note that the marker is realised with an additional initial \textit{u} in this case, as in (\ref{ex:nakaupunu}). This is a reply by Miguel to Juan C.’s statement that he had come back after having lived in another place for fifteen years.

\ea\label{ex:nakaupunu}
\begingl 
\glpreamble nakaupunu Naranjitoyae\\
\gla naka-upunu Naranjito-yae\\ 
\glb here-\textsc{am.prior} Naranjito-\textsc{loc}\\ 
\glft ‘(you) came here, to Naranjito’\\ 
\endgl
\trailingcitation{[mqx-p110826l.451]}
\xe

The suffix has lexicalised\is{lexicalisation} with a number of motion verbs.\is{motion predicate|(} Many of these motion verbs are never found without it, but some can replace \textit{-punu} with the \isi{dislocative} marker \textit{-pu} (see \sectref{sec:PA}). \tabref{table:Lexicalised_punu} presents the verbs that obligatorily take \textit{-punu}.

\begin{table}[htbp]
\caption{Verbs lexicalised with the prior motion marker}

\begin{tabularx}{\textwidth}{lQQ}
\lsptoprule
Verb & Translation & Comment\cr
\midrule
\textit{(-bÿku-punu)} & enter & \textit{-bÿku-pu} is more frequent\cr
\textit{-bÿsÿu-punu} & come & \textit{-bÿsÿu} ‘come’\cr
\textit{-e-punu} & take & occasionally \textit{-e-pu} \cr%translocative!
\textit{-etu-punu} & leave sth. with s.o., take/bring sth. to a place & \textit{-etuku} ‘put’ \cr%translocative
\textit{ka-punu} & come & non-verbal predicate, see \sectref{sec:Kapunu} \cr
\textit{-kuje-punu} & get hold of &\cr%check translation! jekupu = perder, olvidar
\textit{-ne-punu} & see so. going & motion of object, see \sectref{sec:OBJ-AM}\cr
\textit{-tu-punu} & reach (tr.) & \cr
\textit{-tu-punu-bu} & arrive (intr.) & \cr
\textit{-u-punu} & bring & \textit{-u} is an existential root, but does not occur on its own \cr
\textit{-yunu-punu} & go back & \textit{-yunu} ‘go’; \textit{-yunupu} ‘go to’; \textit{-yunupunuku} ‘go back’\cr
\textit{-yu-punu} & go/come out, exit & occasionally \textit{-yu-pu}; related to \textit{-yunu} ‘go’ and \textit{-yuiku} ‘walk’ \cr
\lspbottomrule
\end{tabularx}

\label{table:Lexicalised_punu}
\end{table}

Since most verb forms listed in the table do not exist without the prior motion marker, it is impossible to tell which semantic feature it adds to the stem. It is clear, however, that some verbs describe motion away from the deictic centre (i.e. translocative), like \textit{-epunu} ‘take’, and some towards the deictic centre (i.e. cislocative), like \textit{-upunu} ‘bring’.

The verbs that stick out here are the two translational motion verbs \textit{-bÿsÿupunu} and \textit{-yunupunu}. Both have counterparts without the AM marker, \textit{-bÿsÿu} ‘come’ and \textit{-yunu} ‘go’, respectively. According to the speakers, there is no difference between \mbox{\textit{-bÿsÿu}} without and \textit{-bÿsÿupunu} with the marker, but the latter is the preferred form that sounds “better”. This is surprising, since addition of \textit{-punu} to motion verbs usually denotes a movement back to a place, but I have indeed encountered examples with \textit{-bÿsÿupunu}, in which no return is implied. It occurs twice as often as \textit{-bÿsÿu} in the corpus. One difference between the two verbs that I could make out is that the \isi{irrealis} form \textit{-bÿsÿupuna} is clearly preferred over \textit{-bÿsÿa}, although there are a few examples with the latter, too.

(\ref{ex:bÿsÿu}) is an example with the verb \textit{-bÿsÿu} and (\ref{ex:bÿsÿupunu}) shows the use of \textit{-bÿsÿupunu}. Both examples describe some motion from “there” to “here” and there are absolutely no differences in meaning between the two verbs.

In (\ref{ex:bÿsÿu}), Juana is speaking about a relocation in Santa Cruz.

\ea\label{ex:bÿsÿu}
\begingl 
\glpreamble tukiu nauku tanÿma bibÿsÿu naka\\
\gla tukiu nauku tanÿma bi-bÿsÿu naka\\ 
\glb from there now 1\textsc{pl}-come here\\ 
\glft ‘from there we came here now’\\ 
\endgl
\trailingcitation{[jxx-p110923l-1.182]}
\xe

(\ref{ex:bÿsÿupunu}) comes from Miguel and also describes a relocation, from \isi{Altavista} to Santa Rita.

\ea\label{ex:bÿsÿupunu}
\begingl 
\glpreamble bichubibikiu tukiu nauku bibÿsÿupunu naka\\
\gla bi-chubibik-i-u tukiu nauku bi-bÿsÿu-punu naka\\ 
\glb 1\textsc{pl}-stroll-\textsc{subord}-\textsc{real} from there 1\textsc{pl}-come-\textsc{am.prior} here\\ 
\glft ‘moving from there, we came here’\\ 
\endgl
\trailingcitation{[mxx-p110825l.181]}
\xe


The meaning of \textit{-yunupunu} is lexicalised\is{lexicalisation} as ‘go back’, opposed to \textit{-yunu} ‘go’. The point of origin towards which the motion is directed is the home of the speaker in most cases, but it can also be another point, from which the referent has departed before.\footnote{See also \sectref{sec:Repetition}, where \textit{-yunupunu} ‘go back (returning to origin)’ is contrasted with \textit{-yunupunuku} ‘go back (to non-origin or departing)’.} 
 This can be seen in (\ref{ex:back-up}). At this point of the story, the grandparents of Juana try to cross the arroyo with their cows, but the water spirit has risen from the water and threatens them, so that they have to return and climb up the slope again, which they achieve by pulling themselves up grabbing twigs or roots that grow in the arroyo.\footnote{The stem of the verb is also very interesting; a detailed analysis of it is found in \sectref{sec:ActiveVerbs_Combi}, ex. (\ref{ex:act-combi-2}).}

\ea\label{ex:back-up}
\begingl 
\glpreamble tijatÿtÿkeikukukÿubunubeji yÿkÿke tiyunupununube\\
\gla ti-jatÿtÿkeiku-kukÿu-bu-nube-ji yÿkÿke ti-yunupunu-nube\\ 
\glb 3i-pull.sticks-\textsc{am.conc.tr}-\textsc{mid}-\textsc{pl}-\textsc{rprt} stick 3i-go.back-\textsc{pl}\\ 
\glft ‘going pulling themselves up with the help of sticks (i.e. twigs and roots), they went back (up the slope), it is said’\\ 
\endgl
\trailingcitation{[jxx-p151016l-2]}
\xe

The meaning of \textit{-yunupunu} is partly decomposable into the meanings of the prior motion marker \textit{-punu} and the verb \textit{-yunu} as ‘come to a point and go’; however, the fact that the verb only describes actions of going back and not of going further is something that is not predictable from its parts. This is rather a case of regressive marking; however, \textit{-yunu} can also combine with the regressive marker\is{regressive/repetitive} \textit{-punuku} with a slightly different meaning (see discussion in \sectref{sec:Repetition}).

The meaning of return is also realised by \textit{-punu}, when it attaches to any other verb that can be interpreted to include motion: After having managed to climb up the slope again (see (\ref{ex:back-up}) above), Juana’s grandparents clutch a tree, so that the wind cannot blow them back into the arroyo. The verb stem \textit{-jatÿku} translates as ‘pull’. In combination with \textit{-puna}, it could, of course, mean ‘come and pull’ here, but since ‘pull’ can be understood to include motion, it is rather the direction of this pulling back to the arroyo which is expressed here.

\ea\label{ex:punu-DIR-2}
\begingl 
\glpreamble nebuji eka tujubeiku kuinabu chijatÿkupunanubetu nauku\\
\gla nebu-ji eka tujubeiku kuina-bu chi-jatÿku-puna-nube-tu nauku\\ 
\glb 3\textsc{obl.top.prn}-\textsc{rprt} \textsc{dem}a wind \textsc{neg}-\textsc{dsc} 3-pull-\textsc{am.prior.irr}-\textsc{pl}-\textsc{iam} there\\ 
\glft ‘so that the wind could not pull them back there anymore, it is said’\\ 
\endgl
\trailingcitation{[jxx-p151016l-2]}
\xe

In addition, I found one example of a translational motion verb with \textit{-punu} in the corpus, in which \textit{-punu} functions as a directional that specifies that the action is directed towards the deictic centre, the place where the speaker, María S., currently lives and where she uttered the sentence. In this case, \textit{-punu} cannot be interpreted as motion back to a point because the family had not lived at this point before as far as I know. The actions of coming and moving thus happen simultaneously, see (\ref{ex:punu-DIR-1}).

\ea\label{ex:punu-DIR-1}
\begingl 
\glpreamble bichÿnumitu – nechikue bijechÿpunu naka\\
\gla bi-chÿnumi-tu nechikue bi-jechÿ-punu naka\\ 
\glb 1\textsc{pl}-be.sad-\textsc{iam} therefore 1\textsc{pl}-move-\textsc{am.prior} here\\ 
\glft ‘we were sad (when our father died) – therefore we moved here’\\ 
\endgl
\trailingcitation{[rxx-e120511l.170-171]}
\xe


Last but not least,\is{purpose|(} \textit{-punu} can also attach to a non-motion verb which is combined with another motion predicate in the clause.\is{motion predicate|)} In these constructions, \textit{-punu} very much resembles the \isi{dislocative} marker \textit{-pu}, which is frequently found in translocative motion-cum-purpose constructions,\is{motion-cum-purpose construction} see \sectref{sec:PA}. 

(\ref{ex:cis-purpose-1}) to (\ref{ex:punu-with-ku}) exemplify this with different cislocative predicates. It is possible that this double-marking of motion can be attributed to Spanish influence, because there is no way of encoding motion morphologically in Spanish. The speakers may thus feel the need to be more explicit. In any case, constructions like these could be the beginning of re-interpretation of the AM marker as a purpose marker. This is what possibly happened to the \isi{dislocative} marker before.

(\ref{ex:cis-purpose-1}) was a statement by María C. directed to me.

\ea\label{ex:cis-purpose-1}
\begingl 
\glpreamble pibÿsÿu naka pisamupunu paunaka\\
\gla pi-bÿsÿu naka pi-samu-punu paunaka\\ 
\glb  2\textsc{sg}-come here 2\textsc{sg}-hear-\textsc{am.prior} Paunaka\\ 
\glft ‘you came here to learn (lit.: hear) Paunaka’\\ 
\endgl
\trailingcitation{[uxx-p110825l.036]}
\xe

(\ref{ex:cis-purpose-1}) comes from Juana, who speaks about a criminal in-law of hers. He was a fugitive and only came home at night.

\ea\label{ex:cis-purpose-2}
\begingl 
\glpreamble las dies yuti titupunubu timukupunaji nechÿu\\
\gla {las dies} yuti ti-tupunubu ti-muku-puna-ji nechÿu\\ 
\glb {at ten o’clock} night 3i-arrive 3i-sleep-\textsc{am.prior.irr}-\textsc{rprt} \textsc{dem}c\\ 
\glft ‘at ten o’clock at night, he arrived in order to sleep there, it is said’\\ 
\endgl
\trailingcitation{[jxx-p120430l-2.115]}
\xe

In (\ref{ex:punu-with-ku}), Juana speaks about the men of Santa Rita, who were clearing land for a German lady in exchange for the construction of the reservoir.

\ea\label{ex:punu-with-ku}
\begingl 
\glpreamble tuse kapupununubeinatu tinikupunanube\\
\gla tuse kapupunu-nube-ina-tu ti-niku-puna-nube\\ 
\glb noon come.back-\textsc{pl}-\textsc{irr.nv}-\textsc{iam} 3i-eat-\textsc{am.prior.irr}-\textsc{pl}\\ 
\glft ‘at noon they would come back to eat’\\ 
\endgl
\trailingcitation{[jxx-p120515l-2.187]}
\xe


As has been shown, the marker \textit{-punu} has two to three different functions: on non-motion verbs, it functions as an AM marker that encodes prior motion without a specified direction; as a directional on motion verbs,\is{motion predicate} it encodes regressive deixis and possibly also non-regressive motion towards the deictic centre. In combination with another motion predicate in the clause, it seems to marks purpose. It could therefore be defined as an AM marker according to the broad definition by \citet[92]{Guillaume2016}, because \textit{one} of its functions is the expression of translational motion.\is{purpose|}

In addition to the marker \textit{-punu},\is{suppletive imperative|(} there is an imperative particle \textit{pana} ‘come!’ that might go back to an old \isi{irrealis} form of the marker that preserved vowel harmony (see Footnote \ref{fn:VowelHarmony} in \sectref{sec:VerbalRS}).\footnote{In addition, \textit{pana} is also the second person singular irrealis form of \textit{-anau} ‘make’ (\textit{pi-ana} 2\textsc{sg}-make.\textsc{irr}), but this seems to be a coincidence.} (\ref{ex:pana-COME}) shows the imperative venitive particle, for more examples see \sectref{sec:SuppletiveImperatives}. The sentences come from Juana and reproduce a warning directed to María S. by her husband, when they were bathing in the reservoir of Santa Rita.

\ea\label{ex:pana-COME}
\begingl
\glpreamble “¡pana naka! ¡kechue echÿu!”\\
\gla pana naka kechue echÿu\\
\glb come.\textsc{imp} here snake \textsc{dem}b\\
\glft ‘“come here! that’s a snake!”
\endgl
\trailingcitation{[jxx-p120515l-2.164]}
\xe
\is{suppletive imperative|)} 


\subsection{Subsequent motion}\label{sec:SubsequentMotion}

In addition to the concurrent and prior AM markers, there is possibly also a subsequent motion marker \textit{-nÿmu}/\textit{-nÿma}. This marker appeared once Juana’s account of her grandparents’ journey and was translated by her as ‘sleep in a hut, on the way’. This translation could imply subsequent motion because sleeping on the way, on a long journey that takes several days, implies that one moves on the next day.

\ea\label{ex:SUBSMOT-1}
\begingl
\glpreamble timukunÿmunube juchubu kaku ÿne\\
\gla ti-muku-nÿmu-nube juchubu kaku ÿne\\
\glb 3i-sleep-\textsc{am.subs}?-\textsc{pl} where exist water\\
\glft ‘they spent the night where water was (and went)’
\endgl
\trailingcitation{[jxx-e150925l-1.197]}
\xe

\isi{Mojeño Trinitario} has a cognate marker \textit{-numo}, which is analysed as a subsequent motion marker by \citet[]{Rose2015}.\footnote{In addition, \isi{Baure} has a subsequent motion marker \textit{-wana} \citep[cf.][257]{Danielsen2007}, but the relation to this form is not as straightforward as to the Trinitario one.} Among the translations for the suffix she found in the grammar by \citet[]{Marban1894}, there is ‘do on the way’ besides subsequent motion ‘do and go’. In current Trinitario,\is{Mojeño Trinitario} speakers translate the morpheme as ‘do sth. first’, which is also the translation found for the Ignaciano\is{Mojeño Ignaciano} cognate \textit{-numa} \citep[141--142]{Rose2015}.

(\ref{ex:SUBSMOT-2}) is from an elicitation session, in which I asked María S. whether some verbs with added \textit{-nÿmu} were possible and what they meant. I asked for the form \textit{nikubunÿmu} ‘I bathe (and go)’, and María S. produced the following sentence to provide me with an example of how the form could be used. It contained an explicit motion verb that encoded motion back home after the action of the verb marked with \textit{-nÿmu} is completed.

\ea\label{ex:SUBSMOT-2}
\begingl
\glpreamble niyunupunatu ubiayae nikubunÿmutu\\
\gla ni-yunupuna-tu ubiae-yae ni-kubu-nÿmu-tu\\
\glb 1\textsc{sg}-go.back.\textsc{irr}-\textsc{iam} house-\textsc{loc} 1\textsc{sg}-bathe-\textsc{am.subs}?-\textsc{iam}\\
\glft ‘I will go back to the house now, I have bathed now (and go)’
\endgl
\trailingcitation{[rxx-e181018le-a]}
\xe

With other verbs, the speaker rather provided a translation that included ‘over there’ (which does not exclude the possibility that it is a subsequent motion marker) and she would not accept all examples that I tried to produce with this marker. It is in any case a rare marker and does not seem to be included in the (very) active repertoire of grammatical morphemes of the speakers.


\subsection{Associated path}\label{sec:PA}\is{dislocative|(}
%Trajectory with an endpoint (cf. Craig 1993:33)

Of the markers described in this section, the dislocative suffix \textit{-pu}/\textit{-pa} has the most diverse functions. It occurs predominantly in expressions of motion with purpose\is{motion-cum-purpose construction|(} that are formed with two verbs, the motion verb\is{motion predicate} and the purpose verb. The dislocative attaches to the purpose verb in this case, as in (\ref{ex:new23-dloc}), where María S. describes a duty of her life as a child and adolescent. 

\ea\label{ex:new23-dloc}
\begingl
\glpreamble biyuna bisupa\\
\gla bi-yuna bi-isu-pa\\
\glb 1\textsc{pl}-go.\textsc{irr} 1\textsc{pl}-weed-\textsc{dloc.irr}\\
\glft ‘we had to go to weed’
\endgl
\trailingcitation{[rxx-p181101l-2.149]}
\xe
\is{motion-cum-purpose construction|)}

In addition, the dislocative marker is lexicalised\is{lexicalisation} with a number of motion verbs\is{motion predicate} and possibly also with some non-motion verbs. The general motion verb \textit{-yunu} ‘go’ has a variant \textit{-yunupu} ‘go to’ derived\is{derivation} with this suffix. I found it very hard to come to grips with it until I finally understood that a) my assumption that the marker occurred on certain stative verbs was simply false and resulted from some misheard forms and b) my main consultants on this topic, Juana and María S., use the marker quite differently. I will explain this in more detail below.

In the preliminary stage of analysis, I decided to name the marker “dislocative”, following the terminology of \citet[]{OConnor2007}, who uses the term to describe a morpheme in Lowland Chontal of Oaxaca (Tequistlatecan).\footnote{The language is also known as Huamelultec, and should not be confused with the Mayan language also called Chontal.} According to \citet[112]{OConnor2007}, “the dislocative is oriented to the Goal or ending point of motion away”. It is “concerned with situating the event at a not-here location”.
Since this definition holds for some of the constructions found with the marker, I will stick to the term, although in other contexts “intentional” or “purposive” would be a better gloss.\footnote{In addition, the term “dislocative” is also used in the description of Onondonga, a Northern Iroquian language spoken in North America, and defined as follows: “The dislocative indicates agentive movement or intention on the part of the subject to undertake the event denoted by the verb” \citep[66]{Barrie2015}.}

Although the dislocative marker occurs in different kinds of motion expressions, it does not by itself encode motion (although motion may be implied in its use by María S.). It rather highlights the path or goal of a motion event or adds such a component. It could therefore be defined as belonging to a category of “associated path”, although this term was originally used to cover the functions of some grammatical markers that encode AM plus some more path-related meanings \citep[cf.][297--298]{Simpson2002}. Despite highlighting a path or a goal, it is not a directional either, since it does not encode any specific direction, but rather the existence of a direction with an endpoint.

In the remainder of this section, I give an overview of the different kinds of occurrences and functions of the dislocative. I start the description with a list of verbs that obligatorily take the dislocative marker in \tabref{table:dislocative}.\is{motion predicate|(}

\begin{table}[htbp] 
\caption{Motion verbs lexicalised with the dislocative suffix}

\begin{tabularx}{\textwidth}{llQ}
\lsptoprule
Verb & Translation & Comment\cr
\midrule
\textit{-benu-pu} & lie down, be born & one occurrence of each \textit{-benu} and \textit{-benu-punu} in corpus\cr
\textit{-bÿku-pu} & enter & \textit{-bÿku-punu} is also possible\cr%bÿchÿ/eku = salir	
\textit{-jinu-pu} & flow, flood &  \cr%\textit{-jinu} ‘breast of animal’ is a noun 
\textit{-ku-pu} & go down, descend &\cr%-punu = subir
\textit{-neku-pu} & see so. coming & motion of object, see \sectref{sec:OBJ-AM} \cr
\textit{-yunu-pu} & go to & \textit{-yunu} ‘go’, \textit{-yunu-punu} ‘go back’\cr
(\textit{-yu-pu}) & go/come out, exit & more frequent form is \textit{-yu-punu} \cr
\lspbottomrule
\end{tabularx}

\label{table:dislocative}
\end{table}

Besides motion verbs, some verbs that encode motion in a more abstract way are lexicalised\is{lexicalisation} with the dislocative marker, like \textit{-jakupu} ‘receive, answer’ whose meaning is about exchange, although this exchange may include more (receive) or less concrete (answer) motion. Some other verbs do not seem to encode motion at all. It is unclear at the moment whether they take the same dislocative marker or a homophonous suffix with a yet undetermined meaning. These verbs are listed in \tabref{table:non-motVpu}.\is{motion predicate|)}


\begin{table}[htbp] 
\caption{Non-motion verbs with \textit{-pu}}

\begin{tabularx}{\textwidth}{llQ}
\lsptoprule
Verb & Translation & Comment\cr
\midrule
\textit{-chu-pu} & know & \textit{-chupu} is used for ‘know a fact’ and mostly negated\cr%-chupuiku; -(i)chuna
\textit{-jaku-pu} & receive, answer & \cr
\textit{-jeku-pu} & lose & \cr
\textit{-jeku-pu-umi} & forget & \cr
\textit{-jiku-pu} & swallow & \cr
\textit{-kÿchu-pu} & wait & \cr
\textit{-seku-pu} & suit & \textit{-seku} ‘dig hole’ \cr
\textit{-tu-pu} & find, meet & \textit{-tu-punu} ‘reach’ \cr
\lspbottomrule
\end{tabularx}

\label{table:non-motVpu}
\end{table}

%yu-puna salir de abajo para arriba, also yu-pu
%check: mejikupu = perder %kujeupu /irr:akujeupu = aborrecer a alguien vs. kujemu

As for the addition of the dislocative marker to the basic motion verb\is{motion predicate|(} \textit{-yunu} ‘go’, the resulting form \textit{-yunupu} ‘go to’ obligatorily takes a goal as its object, and thus the dislocative marker functions as a kind of applicative\is{applicative|(} here, as in (\ref{ex:yunupu-1}) from Miguel, who made a statement about his brother.

\ea\label{ex:yunupu-1}
\begingl 
\glpreamble kuina Jose, tiyunupu uneku\\
\gla kuina Jose ti-yunu-pu uneku\\ 
\glb \textsc{neg} José 3i-go-\textsc{dloc} town\\ 
\glft ‘José isn’t here, he went to town’\\ 
\endgl
\trailingcitation{[mxx-d110813s-2.013]}
\xe

A clause with \textit{-yunu} can already contain an explicit goal, which usually carries the locative marker \is{locative marker|(} \textit{-yae} to mark it as an oblique in this case. This can be seen in (\ref{ex:yunu-GOAL1}), where the endpoint of the translational motion encoded by the predicate is the field of the addressed person. Miguel was talking with José in this case.
 
\ea\label{ex:yunu-GOAL1}
\begingl 
\glpreamble ¿pero piyunu pisaneyae?\\
\gla pero pi-yunu pi-sane-yae\\ 
\glb but 2\textsc{sg}-go 2\textsc{sg}-field-\textsc{loc}\\ 
\glft ‘but did you go to your field?’\\ 
\endgl
\trailingcitation{[mox-c110926s-1.185]}
\xe

However, if the focus is not on the goal, but on the action of going (somewhere previously established), there is no need to express a goal in the clause. Note that in those cases, the goal is either retrievable through the context or unimportant, as is the case with (\ref{ex:yunu-pure}) from Miguel’s story about the fox and the jaguar. The jaguar wants to punish the fox, but the latter has already gone away. The goal is not of importance here, but rather the fact that the fox has left the scene.

\ea\label{ex:yunu-pure}
\begingl
\glpreamble kuina kakuina kupisaÿrÿ, tiyunutu\\
\gla kuina kaku-ina kupisaÿrÿ ti-yunu-tu\\
\glb \textsc{neg} exist-\textsc{irr.nv} fox 3i-go-\textsc{iam}\\
\glft ‘the fox wasn’t there, he had already gone’
\endgl
\trailingcitation{[jmx-n120429ls-x5.167-168]}
\xe

In contrast, if the dislocative suffix is added to \textit{-yunu}, mention of an explicit goal is obligatory and the goal is not marked with the locative marker \is{locative marker|)}, see (\ref{ex:yunupu-1}) above, where the goal is a common noun, and (\ref{ex:yunupu-2}) where the goal is a toponym.\footnote{Note that, although there is usually a locative marker on the goal NPs of the verb \textit{-yunu} ‘go’, \textit{-yae} is sometimes dropped. This is especially true if the goal is a toponym, but also in some other cases, like %(\ref{ex:yunu-GOAL-2}) and 
 (\ref{ex:yunu-GOAL-3}).

%\ea\label{ex:yunu-GOAL-2}
%\begingl 
%\glpreamble tiyunu España\\
%\gla ti-yunu España\\ 
%\glb 3i-go Spain\\ 
%\glft ‘she went to Spain’\\ 
%\endgl
%\trailingcitation{[jxx-p120430l-1.185]}
%\xe

\ea\label{ex:yunu-GOAL-3}
\begingl 
\glpreamble kuina niyuna asaneti, pasayu sache\\
\gla kuina ni-yuna asaneti pasa-yu sache\\ 
\glb \textsc{neg} 1\textsc{sg}-go.\textsc{irr} field a.lot-\textsc{ints} sun\\ 
\glft ‘I didn’t go to the field, there was much sun’\\ 
\endgl
\trailingcitation{[rmx-e150922l.004]}%non-elicited!
\xe
} The latter example is also from Miguel who describes a passage of his life.

\ea\label{ex:yunupu-2}
\begingl 
\glpreamble tukiuku naka biyunupu Naranjito\\
\gla tukiu-uku naka bi-yunu-pu Naranjito\\ 
\glb from-\textsc{add} here 1\textsc{pl}-go-\textsc{dloc} Naranjito\\ 
\glft ‘from here again we went to Naranjito’\\ 
\endgl
\trailingcitation{[mxx-p110825l.181]}
\xe

Note that the dislocative suffix is added to \textit{-yunu} most often, when the goal is \textit{uneku} ‘town’, which is always equivalent to Concepción in the corpus.\footnote{In an elicitation session, Juana rejected \textit{Santa Kuru}, ‘the city of Santa Cruz’, as a possible goal for \textit{-yunupu}, but \textit{asaneti} ‘field’ was an appropriate goal according to her. María S. would only accept \textit{uneku} ‘town’ (equivalent to Concepción) as a possible goal for \textit{-yunupu}. It is possible that \textit{-yunupu} is only applicable to places in the closer, more familiar surrounding of the speakers, places where they go regularly or places that are within walking distance.}

Other motion verbs lexicalised with the dislocative marker do not assign object status to goal arguments.\is{argument} The verb \textit{-bÿkupu} ‘enter’ takes a goal argument marked with the locative marker, see (\ref{ex:enter-GOAL}), and in the only occurrence of \textit{-jinupu} ‘flow’ with a goal, this is introduced by \textit{tÿpi} ‘\textsc{obl}’, which can be translated as ‘towards’ in this case, see (\ref{ex:flow-GOAL}). Therefore, the dislocative marker cannot be described as a general applicative to promote \isi{object} status to goal arguments, but this is rather a peculiarity of the combination of the verb stem \textit{-yunu} ‘go’ with the marker.\is{applicative|)}

(\ref{ex:enter-GOAL}) comes from Juana’s account about her grandparents’ journey with their cows. When they rested, they let their cows in enclosures.

\ea\label{ex:enter-GOAL}
\begingl 
\glpreamble tibÿkupujaneji bakayayae baka\\
\gla ti-bÿkupu-jane-ji bakaya-yae baka \\ 
\glb 3i-enter-\textsc{distr}-\textsc{rprt} enclosure-\textsc{loc} cow\\ 
\glft ‘the cows went into the enclosure, it is said’\\ 
\endgl
\trailingcitation{[jxx-p151016l-2]}
\xe

(\ref{ex:flow-GOAL}) is also from Juana speaking about the water that comes from Naranjito.

\ea\label{ex:flow-GOAL}
\begingl 
\glpreamble echÿuni, jaa, nauku tijinupu tÿpi echÿu chÿkÿ\\
\gla echÿu-ni jaa nauku ti-jinupu tÿpi echÿu chÿkÿ\\ 
\glb \textsc{dem}b-\textsc{deict} \textsc{afm} there 3i-flow \textsc{obl} \textsc{dem}b arroyo\\ 
\glft ‘from there, yes, there it flows towards the arroyo’\\ 
\endgl
\trailingcitation{[jxx-p120515l-2.129-131]}
\xe
\is{motion predicate|)}

There are a few rare cases in the corpus where the dislocative marker attaches to verbs that do not express locomotion, like \textit{kebu} ‘rain’. Compare  (\ref{ex:kebu-ohne-dir}), which has no dislocative marker and simply gives information about the weather, with (\ref{ex:kebu-DIR}). The latter example expresses that the rain not only fell randomly, but was directed towards the school, i.e. entered the school, which is the important information in this clause.

(\ref{ex:kebu-ohne-dir}) comes from the same context as (\ref{ex:enter-GOAL}) above.

\ea\label{ex:kebu-ohne-dir}
\begingl 
\glpreamble tikebu ÿku\\
\gla ti-kebu ÿku\\ 
\glb 3i-rain rain\\ 
\glft ‘it rained’\\ 
\endgl
\trailingcitation{[jxx-p151016l-2]}
\xe

(\ref{ex:kebu-DIR}) was produced by Miguel when he told me about the history of Santa Rita. They had built a school, but apparently, the roof had a leak. 

\ea\label{ex:kebu-DIR}
\begingl 
\glpreamble kuina tamichana echÿu, tikebupu echÿu\\
\gla kuina ti-a-michana echÿu ti-kebu-pu echÿu\\ 
\glb \textsc{neg} 3i-\textsc{irr}-nice \textsc{dem}b 3i-rain-\textsc{dloc} \textsc{dem}b\\ 
\glft ‘it wasn’t good, it dripped in’\\ 
\endgl
\trailingcitation{[mxx-p110825l.089]}
\xe

The example in (\ref{ex:kebu-DIR}) was produced by Miguel and verified by Juana (although she understood it that way, that the rain flooded the school rather than dripping in). María S., however, claimed this form was ungrammatical, which probably has to do with her usage of the suffix as a marker of motion to town, see below.

Most frequently, the dislocative marker is used in expressions of motion with purpose,\is{purpose|(} in which there is a separate \isi{motion predicate} and the dislocative attaches to the verb expressing the purpose. I call this type of construction the motion-cum-purpose construction \is{motion-cum-purpose construction|(} (or MCPC). (\ref{ex:new23-dloc}) at the beginning of this section is an example of the MCPC, furthermore (\ref{ex:go-DIR-1}) and (\ref{ex:pu-1}) illustrate this use.

(\ref{ex:go-DIR-1}) comes from Miguel’s account of the past. He cites the old \textit{patrón} of \isi{Altavista}, when he had to let the workers go after the agrarian reform of 1952.

\ea\label{ex:go-DIR-1}
\begingl 
\glpreamble “¡eyuna esamaikupa juchubu ubiuye!"\\
\gla e-yuna e-semaiku-pa juchubu ubiu-yae\\ 
\glb 2\textsc{pl}-go.\textsc{irr} 2\textsc{pl}-search-\textsc{dloc.irr} where house-\textsc{loc}\\ 
\glft ‘“go to look for where to live!"’\\ 
\endgl
\trailingcitation{[mxx-p110825l.052]}
\xe

(\ref{ex:pu-1}) comes from María S.

\ea\label{ex:pu-1}
\begingl 
\glpreamble niyunu nisane nisupu\\
\gla ni-yunu ni-sane ni-isu-pu\\ 
\glb 1\textsc{sg}-go 1\textsc{sg}-field 1\textsc{sg}-weed-\textsc{dloc}\\ 
\glft ‘I went to my field to weed’\\ 
\endgl
\trailingcitation{[rxx-e120511l.033]}
\xe


Besides the MCPC, a \isi{serial verb construction} can be used to express the purpose of motion. \sectref{sec:SVC_and_MCPC} elaborates on this topic. Other kinds of purposes, which do not include motion, are expressed differently.\is{purpose|)}
%\emph{TO DO: check motion-cum-purpose constructions with DS again with María and Juana!!, with -pu probably only possible with same subject? (-> müller 2013:154-155)}

If the motion verb\is{motion predicate} is realis, the second verb can take either the realis or irrealis form\is{reality status} of the dislocative marker to yield a perfective or imperfective\is{perfective/imperfective} reading of the action encoded by the second verb, see also \sectref{sec:Prospectiveness} and \sectref{sec:SVC_and_MCPC}.\footnote{\label{fn:pu_Baure_Trinitario}This is a very interesting fact in comparison with the other Bolivian Arawakan languages.\is{Southern Arawakan} Trinitario\is{Mojeño Trinitario} has several AM markers. The marker \textit{-opo} attaches to pronouns to yield a non-verbal predication of ‘come’ \citep[139]{Rose2015}, but \textit{-po} is a perfective marker on verbs \citep[82]{Rose2014}. \isi{Baure} has lost RS, but maintains two markers \textit{-po} and \textit{-pa}. While \textit{-po} is a perfective marker \citep[262]{Danielsen2007}, \textit{-pa}’s main function is intentional marking, although it can also encode direction or AM \citep[221--223]{Admiraal2016}. \citet[]{Danielsen2012} proposes the idea that both markers have developed from only one. If we consider the constructions in modern Paunaka, we can speculate that both the perfective\is{perfective/imperfective} and the intentional reading of the \isi{Baure} marker may have developed out of motion-cum-purpose constructions similar to the ones present in Paunaka.}


In addition to \textit{-yunu}, other motion verbs\is{motion predicate} can also be combined with a second verb that optionally takes the dislocative marker. Among them is the manipulative (caused motion) verb\is{manipulative verb} \textit{-bÿche(i)ku} ‘send, order’. In this case, both verbs have a different subject. One example is (\ref{ex:send-DIR-1}), which was elicited from Miguel.

\ea\label{ex:send-DIR-1}
\begingl 
\glpreamble ebÿcheku tiyÿseikupa\\
\gla e-bÿcheku ti-yÿseiku-pa\\ 
\glb 2\textsc{pl}-order 3i-buy-\textsc{dloc.irr}\\ 
\glft ‘you sent him to buy it’\\ 
\endgl
\trailingcitation{[mxx-n120423lsf-X.41]}
\xe

(\ref{ex:nabi-DIR}) is an MCPC with the \isi{suppletive imperative} \textit{nabi} ‘go!’. It comes from Juana telling me what her mother used to say.


\ea\label{ex:nabi-DIR}
\begingl 
\glpreamble “¡nabi epuikupa!, temetapujiyu kÿpu”\\
\gla nabi e-epuiku-pa teme-tapu-ji-yu kÿpu\\ 
\glb go.\textsc{imp} 2\textsc{pl}-fish-\textsc{dloc.irr} big-\textsc{clf:}scales-\textsc{col}-\textsc{ints} sardine\\ 
\glft ‘“go fishing! the sardines are very big”’\\ 
\endgl
\trailingcitation{[jxx-e150925l-1.160]}
\xe

There is also one example of this construction type with the \isi{hortative} particle \textit{jaje}, in (\ref{ex:jaje-DIR}). The interpretation of this construction as encoding motion could result from actual motion implied in this particular example, generalised motion in hortative expressions (compare e.g. English and Spanish) or from an analogy with the Spanish \isi{hortative} expression \textit{vamos a} ‘let's go to’ from the verb \textit{ir} ‘go’. The example was elicited from Juana.

\ea\label{ex:jaje-DIR}
\begingl 
\glpreamble ¡jaje binebÿkupa keyu binika!\\
\gla jaje bi-nebÿku-pa keyu bi-nika\\ 
\glb \textsc{hort} 1\textsc{pl}-collect-\textsc{dloc.irr} snail 1\textsc{pl}-eat.\textsc{irr}\\ 
\glft ‘let's go to collect freshwater snails to eat’\\ 
\endgl
\trailingcitation{[jxx-e081025s-1.171]}
\xe

The specific use of the dislocative marker by María S. is either derived from such an MPCP or represents a former stage, in which an additional motion predicate\is{motion predicate|(} was not necessary in general; she also allows purpose-of-motion constructions\is{purpose} without a motion verb. The purpose verb takes the dislocative marker and this alone conveys prior motion to a nearby place, mostly Concepción as in (\ref{ex:pu-2}), but also the field as in (\ref{ex:new23-pur}).

The context of (\ref{ex:pu-2}) was that María S. told me that she was tired from weeding:

\ea\label{ex:pu-2}
\begingl 
\glpreamble nechikue niyÿbamukeikupu uneku\\
\gla nechikue ni-yÿbamukeiku-pu uneku\\ 
\glb therefore 1\textsc{sg}-husk-\textsc{dloc} town\\ 
\glft ‘therefore I went to husk (rice in machine) in town’\\ 
\endgl
\trailingcitation{[rxx-e120511l.035]}
\xe

(\ref{ex:new23-pur}) is an elicited example, a statement about an imaginary old man.

\ea\label{ex:new23-pur}
\begingl
\glpreamble kuina pueroinabu tisupabu\\
\gla kuina puero-ina-bu ti-isu-pa-bu\\
\glb \textsc{neg} can-\textsc{irr.nv}-\textsc{dsc} 3i-weed-\textsc{dloc.irr}-\textsc{dsc}\\
\glft ‘he cannot go to weed anymore’
\endgl
\trailingcitation{[rxx-e181022le]}
\xe

In contrast, Juana once told me that precisely this usage of the dislocative marker without motion predicate was incorrect (she did not make reference to her sister’s usage of the marker, her comment was just meant to correct me).\is{motion predicate|)} We can thus state that there are important differences in the individual use of the dislocative marker.
\is{motion-cum-purpose construction|)}\is{dislocative|)}

\subsection{Motion of the object}\label{sec:OBJ-AM}
\is{object|(}

Paunaka does not have special markers that encode the motion of the object, unlike other South American languages \citep[cf.][]{Guillaume2016}. Nonetheless, there is one \isi{verbal root} \textit{-ne} that is combined with either \textit{-punu} or \textit{-pu}, the result being verb stems that encode the motion of the object. The root \textit{-ne} is not used on its own, nor is there any verb stem \textit{*-neku} or \textit{*-nechu}. The verb \textit{-nekupu} ‘see somebody coming’ is composed of the root \textit{-ne}, the \isi{thematic suffix} \textit{-ku} and the \isi{dislocative} marker \textit{-pu} and expresses cislocative concurrent motion of the object. One example is given in (\ref{ex:nekupu-1}), which is about the appearance of the spirit of water before Juana’s grandparents on their journey home from Moxos.

\ea\label{ex:nekupu-1}
\begingl 
\glpreamble nenayuji eka chinekupunubeji echÿu pariki chikebÿkeji\\
\gla nena-yu-ji eka chi-nekupu-nube-ji echÿu pariki chi-kebÿke-ji\\ 
\glb like-\textsc{ints}-\textsc{rprt} \textsc{dem}a 3-see.coming-\textsc{pl}-\textsc{rprt} \textsc{dem}b many 3-eye-\textsc{rprt}\\ 
\glft ‘it seems that they saw a pair of eyes approaching them, it is said’\\ 
\endgl
\trailingcitation{[jxx-p151016l-2]}
\xe

If the object is an SAP, it is indexed on the verb as in (\ref{ex:nekupu-2}), which was elicited from María S.

\ea\label{ex:nekupu-2}
\begingl 
\glpreamble nÿnekupubi\\
\gla nÿ-nekupu-bi\\ 
\glb 1\textsc{sg}-see.coming-2\textsc{sg}\\ 
\glft ‘I see you coming’\\ 
\endgl
\trailingcitation{[mrx-e150219s.088]}
\xe

If the root \textit{-ne} combines with \textit{-punu} ‘\textsc{am.prior}’, there is no thematic suffix on the verb stem.\footnote{Note that \textit{-punu} is often attached to the verb stem without the thematic suffix, see \sectref{sec:punu}.} The verb \textit{-nepunu} ‘see somebody going’ expresses translocative concurrent motion of the object. However, the verb is also translated as ‘see somebody over there’ by the speakers. Distance could be the more important information in this verb in comparison to motion, although the latter is probably also part of its semantics. When María S. was asked in an elicitation session what the object of the verb could be doing “over there”, the first thing that came to her mind was motion away from the scene (going to the toilet or going to wash clothes). 

(\ref{ex:nepunu-1}) is from an account of the imagined theft of a young woman in the recordings from the 1960s by Riester. Juan Ch. fears that the grandfather of the woman could catch them, therefore he proposes:

\ea\label{ex:nepunu-1}
\begingl 
\glpreamble bumupaika tanÿma, kuina chinepunabu\\
\gla bi-umu-paika tanÿma kuina chi-nepuna-bu\\ 
\glb 1\textsc{pl}-take-\textsc{punct.irr} now \textsc{neg} 3-see.going.\textsc{irr}-\textsc{dsc}\\ 
\glft ‘we’ll take her quickly now, so that he doesn’t see her going anymore’\\ 
\endgl
\trailingcitation{[nxx-a630101g-3.066]}
\xe

There are no other monoverbal constructions, to my knowledge, that express associated motion of the object. Strikingly, while \textit{-punu} is normally more closely connected with cislocative motion and \textit{-pu} with translocative motion (see previous \sectref{sec:AMconcurrent}, \sectref{sec:punu}, and \sectref{sec:PA}), in combination with the root \textit{-ne} and encoding the motion of the object, the directions reverse: \textit{-nekupu} encodes cislocative motion and \textit{-nepunu} translocative motion.
\is{object|)}

\subsection{Regressive and repetitive}\label{sec:Repetition}
\is{regressive/repetitive|(}

The regressive marker does not encode motion by itself, and therefore it does not count as an AM marker, although it is related in form to the prior motion marker and/or disclocative marker (see \sectref{sec:punu} and \sectref{sec:PA} above).\is{derivation} Regressive is a term used in the description of some Kampan \isi{Arawakan languages} for a morpheme that “indicates motion from some point back to a salient point of origin” on motion verbs\is{motion predicate} (\citealt[256]{Michael2008}, and see also \citealt[42]{Payne_et_al1982}). The regressive suffix of Nanti -- similar to the Paunaka marker -- additionally exhibits a meaning of “repetition of the action (or return to the state)” on non-motion verbs \citep[256]{Michael2008},\footnote{The abilitive reading of the regressive found in Nanti \citep[cf.][257]{Michael2008} is absent from Paunaka.} see (\ref{ex:new23-again}) and (\ref{ex:swing-again}) respectively, which both come from a story by Miguel in which the main character, a lazybones who dodges making a field, returns to a place in the woods after having lunch with his wife. There he swings on a liana and plays the flute as he has already done in the morning.

\ea\label{ex:new23-again}
\begingl
\glpreamble kupai tiyunupunukuji\\
\gla kupai ti-yunu-punuku-ji\\
\glb afternoon 3i-go-\textsc{reg}-\textsc{rprt}\\
\glft ‘in the afternoon, he went (there) again, it is said’
\endgl
\trailingcitation{[mox-n110920l.041]}
\xe

\ea\label{ex:swing-again}
\begingl 
\glpreamble tebibikupunukubuji echÿu kujubipiyae\\
\gla ti-ebibiku-punuku-bu-ji echÿu kujubipi-yae\\ 
\glb 3i-swing-\textsc{reg}-\textsc{mid}-\textsc{rprt} \textsc{dem}b liana.sp-\textsc{loc}\\ 
\glft ‘he swung himself again on the liana, it is said’\\ 
\endgl
\trailingcitation{[mox-n110920l.042]}
\xe

Paunaka’s regressive marker has several allomorphs that are added to the stem: \textit{-(u)pupunu}, \textit{-upunuku} or \textit{-(u)pupunuku}. They possibly consist of the \isi{dislocative} marker \textit{-pu} followed by the prior motion marker \textit{-punu} or of the prior motion marker with a reduplicated\is{reduplication} first syllable. The source for the sequence /ku/ could either be the additive marker \textit{-uku} or the \isi{thematic suffix} \textit{-ku}. The whole sequence is lexicalised as encoding return or repetition. I could not find any general rules that would determine the choice of any of the allomorphs. However, stems that end in a diphthong\is{vowel sequence} tend to prefer \textit{-pupunuku}. In elicitation on the regressive marker, both María S. and Juana only accepted the form \textit{-pupunuku}. %-> Jxx-e181031l-3 / rxx-e181018le-a oder -b
Irrealis is marked on the regressive marker, the irrealis variants are \textit{-(u)pupuna}, \textit{-upunuka}, and \textit{-(u)pupunuka}, respectively. Sometimes, the marker \textit{(p)upunuku} is produced as a separate phonological word following the verb after a short pause. However, the fact that irrealis is marked on it suggests that it nonetheless belongs to the preceding verb in these cases. Very infrequently, it also occurs in other positions of the clause, then it resembles an adverb or repetitive particle.

In spontaneous speech, the marker has not been combined with a person marker encoding the object in my corpus; however, elicitation showed that the object marker follows the regressive marker, as in the following example from María S.

\ea\label{ex:dog-bite-again}
\begingl
\glpreamble tinijabakupunukunÿtu kabe\\
\gla ti-nijabaku-punuku-nÿ-tu kabe\\
\glb 3i-bite-\textsc{reg}-1\textsc{sg}-\textsc{iam} dog\\
\glft ‘the dog bit me again’
\endgl
\trailingcitation{[rxx-e181018le-a]}
\xe

Motion predicates\is{motion predicate} lexicalised with the prior motion marker (see \tabref{table:Lexicalised_punu} in \sectref{sec:punu}) replace  \textit{-punu} with one of the allomorphs for regressive marking. %(or by reduplicating the syllable \textit{pu} and/or adding \textit{ku}/\textit{ka} after \textit{nu}). 
The most frequent allomorph on the motion verb \textit{-bÿsÿu} ‘come’ is \textit{-pupunuku}, closely followed by \textit{-pupunu}. There are only a few instances of \textit{-punuku} attached to \textit{-bÿsÿu}.
In any of the cases, the regressive implies, first, that the subject has been at the place in question before, and second, that the subject has been away from the place in question for some time before the action of coming happens. The return point is often the current location of the speaker; however, this location can be relatively small, like the speaker’s home, or as big as the whole country of Bolivia depending on the perspective that is taken. In stories or personal narratives about the past, the return point is usually a salient place that has been mentioned before or is otherwise important and retrievable from the context.

(\ref{ex:come-back-home}) is an utterance from Miguel reported by his sister Juana, who was waiting for him at her house the previous day because Miguel had promised to visit her again. This was at the time when Juana was still living in Santa Cruz. Miguel had joined me and my colleague Swintha on a trip to Santa Cruz and we had visited his sister at her house together.

\ea\label{ex:come-back-home}
\begingl 
\glpreamble “nibÿsÿupupunuka”, tikechu\\
\gla ni-bÿsÿu-pupunuka ti-kechu\\ 
\glb 1\textsc{sg}-come-\textsc{reg.irr} 3i-say\\ 
\glft ‘“I’ll come back”, he said’\\ 
\endgl
\trailingcitation{[jxx-p120430l-1.124]}
\xe

In the following example, the return is to the vicinity of Concepción and Santa Rita rather than to the village of San Miguelito de la Cruz, where the sentence was produced. The village itself was only founded after people had left Naranjito. Since Miguel, who produced this sentence, has a high historical consciousness, especially about the history of the Paunaka people in the 20th century, I suppose that this fact is well-known to him. %Miguel addressed me with this sentence providing a summary of what Juan C. had said, this is why third person is used.

\ea\label{ex:Naranjito-back-here}
\begingl 
\glpreamble teneikunube Naranjito tibÿsÿupupunube naka\\
\gla ti-eneiku-nube Naranjito ti-bÿsÿu-pupunu-nube naka\\ 
\glb 3i-live-\textsc{pl} Naranjito 3i-come-\textsc{reg}-\textsc{pl} here\\ 
\glft ‘they lived in Naranjito, then they came back here’\\ 
\endgl
\trailingcitation{[mqx-p110826l.084]}
\xe


In (\ref{ex:back-to-Bolivia}), the general return to Bolivia is meant rather than a return to the city of Santa Cruz, where I was sitting and conversing with Juana. The house she is speaking of is not in Santa Cruz, but in Concepción, and in order to see it, I would have to come to Concepción.

\ea\label{ex:back-to-Bolivia}
\begingl 
\glpreamble de repente nauku tamichatu nubiu te cuando kue ebÿsÿupupunuka\\
\gla {de repente} nauku ti-a-micha-tu nÿ-ubiu te cuando kue e-bÿsÿu-pupunuka\\ 
\glb {maybe} there 3i-\textsc{irr}-good-\textsc{iam} 1\textsc{sg}-house \textsc{seq} when if 2\textsc{pl}-come-\textsc{reg.irr}\\ 
\glft ‘maybe my house there is already ready (lit.: good), when you come back’\\ 
\endgl
\trailingcitation{[jxx-p120430l-1.427]}
\xe 

The regressive form of the non-verbal predicate\is{non-verbal motion clause} \textit{kapunu} ‘come’ is \textit{kapupunu} ‘come back’. Very few examples of \textit{kapupunuku} are also found.

An example of the use of \textit{kapupunu} is (\ref{ex:come-back-2}), where the point of return is identical to the speaker’s position. In a conversation with Swintha, María S. speculates about when I would come back to Bolivia. At that time I was pregnant and had not planned to go to the field in the near future.

\ea\label{ex:come-back-2}
\begingl 
\glpreamble kakuinatukena mediu anyutu tÿpi chichecha kapupunuina\\
\gla kaku-ina-tu-kena mediu anyu-tu tÿpi chi-checha kapupunu-ina\\ 
\glb exist-\textsc{irr.nv}-\textsc{iam}-\textsc{uncert} half year-\textsc{iam} \textsc{obl} 3-son come.back-\textsc{irr.nv}\\ 
\glft ‘maybe when her child is half a year old, she will come back’\\ 
\endgl
\trailingcitation{[rxx-e121128s-1.054]}
\xe

The regressive marker can also be attached to \textit{-yunu} ‘go’, and the resulting form \mbox{\textit{-yunupunuku}} semantically resembles \textit{-yunupunu} (see \sectref{sec:punu} above), because both describe motion back to a point of origin.\footnote{Note that, besides \textit{-yunu} ‘go’, \textit{-yunupu} ‘go to’, \textit{-yunupunu} ‘go back’ and \textit{-yunu-punuku} ‘go back’, there is also \textit{-yunuku} ‘go on’, which is possibly composed of the verb \textit{-yunu} and the \isi{thematic suffix} \textit{-ku}. This verb is used to encode motion that is continued after an interruption, e.g. when someone is travelling and the journey takes various days.}  However, there are differences between the two. The verb \textit{-yunupunu} implies a return to a basis, mostly the home, but also other more permanent locations associated with the referent. In contrast, the verb \textit{-yunupunuku} is used if the place of return is associated with a less permanent stay; it is thus usually not the home or another basis. Closely connected to this is that \textit{-yunupunu} focuses on return and \textit{-yunupunuku} on departure. In some cases departure indeed seems to be more important than non-permanent stay. \figref{fig:yunupunu} illustrates this difference.

\begin{figure}[!ht]

%\includegraphics[width=.5\textwidth]{figures/yunupunu-4.png}
\includegraphics[width=.5\textwidth]{figures/yunupunu-4-new.pdf}
\caption{Different semantics of \textit{-yunupunu} and \textit{-yunupunuku}}
\label{fig:yunupunu}
\end{figure}


In the following example, which was elicited from María S., both verbs are used, and the difference is obvious: the return to home is expressed by \textit{-yunupuna}, while the return to the field is expressed by \textit{-yunupunuka}.

\ea\label{ex:less-perm-1}
\begingl
\glpreamble niyunupunatu nubiuyae depue kupeina niyunupunuka asaneti\\
\gla ni-yunupuna-tu nÿ-ubiu-yae depue kupei-ina ni-yunu-punuka asaneti\\
\glb 1\textsc{sg}-go.back.\textsc{irr}-\textsc{iam} 1\textsc{sg}-house-\textsc{loc} afterwards afternoon-\textsc{irr.nv} 1\textsc{sg}-go-\textsc{reg.irr} field\\
\glft ‘I will go back to my house now, after that, in the afternoon, I will go back to the field’
\endgl
\trailingcitation{[rxx-e181020le]}
\xe

Another example is (\ref{ex:less-perm-2}) from Juana, who described the life of her criminal in-law hiding away from the police. He only came home to sleep:

\ea\label{ex:less-perm-2}
\begingl
\glpreamble chÿnachÿ kuje trajinau de yutitu kapupunu i tijaikenekÿu tiyunupunukuji kimenukÿ\\
\gla chÿnachÿ kuje trajinau {de yuti}-tu kapupunu i tijaikenekÿu ti-yunu-punuku-ji kimenu-kÿ\\
\glb one moon commute {at night}-\textsc{iam} come.back and dawn 3i-go-\textsc{reg}-\textsc{rprt} woods-\textsc{clf:}bounded\\
\glft ‘he commuted for a month, at night he came back home and at dawn he went back to the woods, it is said’
\endgl
\trailingcitation{[jxx-p120430l-2.119-123]}
\xe

In (\ref{ex:short-term-stay}), Juana describes how people from Santa Rita plan to come to Concepción (for the inauguration of the multi-purpose hall by Evo Morales) with two light lorries that will bring them there, return to Santa Rita only for a short stop and bring more people.

\ea\label{ex:short-term-stay}
\begingl 
\glpreamble ruschÿ kamion kapununubeina i tiyunupunuka kapunuina punachÿ tropa naka\\
\gla ruschÿ kamion kapunu-nube-ina i ti-yunu-punuka kapunu-ina punachÿ tropa naka\\ 
\glb two lorry come-\textsc{pl}-\textsc{irr.nv} and 3i-go-\textsc{reg.irr} come-\textsc{irr.nv} other pack here\\ 
\glft ‘in two lorries, they come, and they (the lorries) go back, and another pack (of people) comes here’\\ 
\endgl
\trailingcitation{[jxx-p150920l.080]}
\xe

The importance of the point of departure is well illustrated by the next example. Juana tells about her children who came to visit her at the end of a year. When the feast days were over, they went back home. Emotionally important for the speaker in this moment is the fact that her children are leaving her, that they are going away from her home, and not so much that they are returning to their own home (given that they will arrive there healthy).

\ea\label{ex:children-go-back}
\begingl 
\glpreamble i despues pasau la nabidad, anyo nuebo te tiyunupunukunube\\
\gla i despues pasau {la nabidad} {anyo nuebo} te ti-yunu-punuku-nube\\ 
\glb and after pass {Christmas} {New Year} \textsc{seq} 3i-go-\textsc{reg}-\textsc{pl}\\ 
\glft ‘and then, when Christmas and New Year had passed, they went again’\\ 
\endgl
\trailingcitation{[jxx-p120430l-1.315]}
\xe

A similar example comes from Miguel who was speaking about Swintha going to Baures (in the department of Beni):

\ea\label{ex:Swintha-back-Baures}
\begingl
\glpreamble punachina semana tiyunupunukatu, tiyuna Beni\\
\gla punachÿ-ina semana ti-yunu-punuka-tu ti-yuna Beni\\
\glb other-\textsc{irr.nv} week 3i-go-\textsc{reg.irr}-\textsc{iam} 3i-go.\textsc{irr} Beni\\
\glft ‘next week, she will leave again, she will go to Beni’
\endgl
\trailingcitation{[mxx-d110813s-2.043-044]}
\xe


On stative non-motion predicates,\is{stative verb}\is{non-verbal predication} the regressive marker signals the resumption and restoration of a state. In the following example, Clara comments on the state of the former estate \isi{Altavista}, which is now nicely restored (as a centre for studies of the dry forest) after having suffered decay when the time of indigenous debt labour was over.

\ea\label{ex:Altavista-regresseive}
\begingl 
\glpreamble tanÿma michanapupunuku\\
\gla tanÿma michana-pupunuku\\ 
\glb now beautiful-\textsc{reg}\\ 
\glft ‘it is nice and neat again’ \\ 
\endgl
\trailingcitation{[cux-c120414ls-1.152]}
\xe

The regressive marker also appears on comments about health recovery as in (\ref{ex:health-regressive}), where Juana speaks about her son-in-law.

\ea\label{ex:health-regressive}
\begingl 
\glpreamble echÿu tetukapu te tamichaupupunu\\
\gla echÿu ti-etuka-pu te ti-a-micha-upupunu\\ 
\glb \textsc{dem}b 3i-put.\textsc{irr}-\textsc{mid} \textsc{seq} 3i-\textsc{irr}-good-\textsc{reg}\\ 
\glft ‘if he gives himself this (injection), he will be healthy (lit.: good) again’\\ 
\endgl
\trailingcitation{[jxx-p110923l-1.068]}
\xe

Active non-motion verbs\is{active verb} can also take the regressive. When the regressive marker is added to a \isi{motion predicate}, the return to a place undoes the previous motion away from this point or the state of being-away from this point. Similarly, in (\ref{ex:take-stitches-out}) the regressive expresses that the action of stitching the wound is undone or the return to a stitch-less state. It comes from Miguel who speaks about an operation.

\ea\label{ex:take-stitches-out}
\begingl 
\glpreamble chibeupupunanube echÿu kusepimÿnÿ\\
\gla chi-beu-pupuna-nube echÿu kusepi-mÿnÿ\\ 
\glb 3-take.away-\textsc{reg.irr}-\textsc{pl} \textsc{dem}b thread-\textsc{dim}\\ 
\glft ‘they’ll take his stitches (lit.: thread) out again’\\ 
\endgl
\trailingcitation{[mqx-p110826l.330]}
\xe

Furthermore, the regressive marker can also be used to signal the repetition of an action. The repetition occurs once and in a different situation, i.e. it is different from iterative repetition as defined by \citet[97]{Mueller2013}.\footnote{The definition by \citet[97]{Mueller2013} centres around five obligatory characteristics, among them repetition of the verbal action and event-internal immediate repetition. Those two characteristics do not fit the semantics of the Paunaka marker, e.g. in (\ref{ex:take-stitches-out}), the action of taking out the stitches is not repeated, the important information being the return to a state. In all examples including motion verbs,\is{motion predicate} motion is repeated, but not immediately, and this is also true for the non-motion action of (\ref{ex:swing-again}) above.}

%pero tajaituji tikebupunuka, jxx-e120516l-1.102

%punachÿ tijai takuestapupunuku = el otro dia azoteó otra vez, mxx-p181027l-1.083

%chijakupupunuku = le contestó de nuevo, mxx-n101017s-1 025


An example of this is (\ref{ex:drink-again}) from a story by Miguel (and Juana), in which the already tipsy jaguarundi invites the fox to drink more chicha with him after the fox has shamed him by stating that he knows 25 jumps, while the jaguarundi had to admit to knowing only one.

\ea\label{ex:drink-again}
\begingl 
\glpreamble “bueno, ¡beupupunuka!"\\
\gla bueno b-eu-pupunuka\\ 
\glb well 1\textsc{pl}-drink-\textsc{reg.irr}\\ 
\glft ‘“well, let’s drink again!”’\\ 
\endgl
\trailingcitation{[jmx-n120429ls-x5.371]}
\xe

The close connection between motion of return and repetition is not only present in Nanti and Paunaka, but also in \isi{Baure}, where a preverbal particle \textit{avik} marking repetition and a homophonous adverb with the meaning ‘again’ are derived from the verb \textit{-avik} ‘return’ \citep[283]{Danielsen2007}.\footnote{The distinction between the preverbal particle and the adverb is that the position of the adverb is relatively free, while the particle only appears directly preceding the verb and thus resembles verb serialisation \citep[283]{Danielsen2007}.} \is{associated motion|)}It would be interesting to find out whether such a connection is more widespread in general, but this is a topic for future research.\is{regressive/repetitive|)} For the time being, this work continues with a description of the middle voice.

