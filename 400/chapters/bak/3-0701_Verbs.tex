%!TEX root = 3-P_Masterdokument.tex
%!TEX encoding = UTF-8 Unicode


\chapter[The verb and morphology on predicates]{The verb and morphology on predicates}\label{sec:Verbs}
\is{verb|(}

Verbs can be fairly complex in Paunaka; however, the only categories that are obligatorily marked on verbs are person/number\is{person marking} and \isi{reality status} (RS). Complexity can increase through derivational processes\is{derivation} inside the stem and through addition of inflectional formatives.\is{inflection} There are also processes at the edge of the verb stem,\is{verbal stem} which can be considered borderline cases between derivation and inflection.

Two main classes of verbs, stative\is{stative verb|(} and active\is{active verb}, can be distinguished by a different strategy of RS marking.\is{reality status} The position of the \isi{subject} marker is identical in stative and active verbs, i.e. the person marker encoding the subject always precedes the verb stem.\is{person marking}

\figref{sec:SimpleStativeVerbs} shows the template of a stative verb. Contrary to active verbs, the position of the markers following the stem could not be established because there are few examples in which two or even more of them are combined. Thus all markers that have been found on stative verb stems in the corpus are simply given in alphabetical order.

\begin{figure}[!ht]
%\includegraphics[width=.7\textwidth]{figures/VerbTemplate-Stative-2.png}
\includegraphics[width=\textwidth]{figures/VerbTemplate-Stative-2-new.pdf}
\caption{Template of a stative verb}
\label{fig:VerbTemplate-Stative}
\end{figure}\is{agglutination}

Stative verbs are \isi{intransitive}\is{valency} and encode typical stative relations like colour, knowledge, emotion, and temperature. There are simple and derived stative verbs. Interestingly, the \isi{incorporation} of body part terms into active verb stems often results in stative verb stems. The stems of stative verbs do not carry a \isi{thematic suffix}. Realis RS\is{realis} is not marked on stative verbs, while \isi{irrealis} is marked by a prefix \textit{a-}, which directly precedes the stem. There are also a number of active verbs\is{active verb} that encode stative relations, all of them are deponent middle verbs,\is{middle voice} i.e. they only occur in the middle form (see \sectref{sec:Middle_voice}).\is{stative verb|)}


\figref{fig:VerbTemplate-Active}, a slightly adapted repetition of \figref{fig:VerbTemplate-1} in \sectref{sec:LowSelectivityMarkers}, shows the template of an \isi{active verb}, including morphology inside and outside of the stem. As has been mentioned before (see \sectref{sec:AffClTAME}), the position of the \isi{optative} markers could not be established because there are simply not enough examples in the corpus. Superscript numbers show different possible positions for some markers.

\begin{figure}[!ht]

%\includegraphics[width=\textwidth]{figures/VerbTemplate-Active-2.png}
\includegraphics[width=\textwidth]{figures/VerbTemplate-Active-2-new.pdf}
\caption{Template of an active verb}
\label{fig:VerbTemplate-Active}
\end{figure}\is{agglutination}

Active verbs\is{active verb} can be \isi{intransitive}, \isi{transitive} or \isi{ditransitive}.\is{valency} Many, but not all of them, have a \isi{thematic suffix} that marks the stem boundary.\is{reality status|(} Realis RS\is{realis} is signalled by absence of any marker including irrealis \isi{inflection}; note that irrealis can be marked on different markers, but generally only once. The default/realis realisation of these markers is with a vowel \textit{u}. In \isi{irrealis} RS, \textit{u} changes to \textit{a}. In some cases, there is also a proper suffix \textit{-a} that signals irrealis RS that alternates with a suffix \textit{-u} for realis RS. A more elaborate explanation follows in \sectref{sec:RealityStatus}.\is{reality status|)}

Paunaka has no \isi{passive}, but there is middle voice. TAME marking is optional and not restricted to verbs. The markers expressing these categories are also found on non-verbal predicates\is{non-verbal predication} and sometimes on other constituents in the clause. Another category is associated motion (AM), the expression of motion events that happen before, simultaneously with and possibly also after the event expressed by the verb. Related to this category is the expression of associated path and repetitive.

The remainder of this chapter is organised roughly by verb structure, from inner to outer processes: \sectref{sec:StativeVerbs} to \sectref{sec:Valency} deal with verb stems including derivations inside and at the edge of stems. This is followed by sections on the most important inflectional categories, person/number in \sectref{sec:NumberPersonVerbs} and reality status in \sectref{sec:RealityStatus}. AM markers are discussed in \sectref{sec:AssociatedMotion}. This is followed by a section on the middle voice in \sectref{sec:Middle_voice}. TAME marking is the topic of \sectref{sec:OperationsPredicates}. Finally, some degree markers are presented in \sectref{sec:MiscellaneousMarkers}. Similarly to TAME markers, they do not only attach to verbs (although one of them, the additive marker, is deeply integrated into the verb structure as will become apparent).


\section{Stative verb stems}\label{sec:StativeVerbs}\is{stative verb|(}
\is{intransitive|(}

Stative verb stems express emotion, cognition, temperature, colour, consistency, taste and some other properties, qualities and non-permanent states. 

They are composed a bit differently from active ones, since they never include a \isi{thematic suffix} and do not take aktionsart suffixes. Unlike in other \isi{Arawakan languages}, however, stative verbs index subjects\is{subject} just like active verbs \citep[cf.][]{Danielsen_Granadillo2008}.\is{person marking} The main difference lies in a different location for irrealis marking: stative verbs take an irrealis prefix, while realis is completely unmarked.

A small number of stems are stative by their inflection for \isi{irrealis} preceding the verb stem, see \sectref{sec:VerbalRS}, but transitive by their ability to either be combined with an \isi{object} NP\is{noun phrase} or even index an object by a person marker.\is{person marking} They are further described in \sectref{sec:TransitiveStativeV}.


\subsection{Simplex stative verbs}\label{sec:SimpleStativeVerbs}

\is{verbal root|(}
Most stative verbs are or appear to be underived or “simplex”. However, it is possible that they are the result of non-productive or singular derivational processes. We can sometimes recognise semantic similarities between phonologically similar stems, e.g. \textit{-kutiu} ‘be ill’ must be derived from \textit{-kuti} ‘hurt’ (which is also stative), \textit{-michainu} ‘be pleased’ from the adjective \textit{micha} ‘good’, and \textit{-(i)chuna} ‘know, be capable’ is certainly related to the active verb \textit{-chupu} ‘know (a fact)’, but there are no regular patterns underlying these derivations.

A number of stative verb stems start with the syllable \textit{ku}. There is indeed a derivational process to derive stative verbs from nouns that includes a prefix \textit{ku-}\is{attributive prefix} (see \sectref{sec:AttributiveVerbs}), but not in every case is the derivation transparent such that the part of the verb stem following \textit{ku} would be recognisable as an independent stem.

The verb \textit{-pisÿ} ‘be black’ seems to be borrowed\is{borrowing} from \isi{Bésiro}, and it inflects like an ordinary stative verb.

\tabref{table:stat-simple} shows some additional stative verbs. Many of them describe properties, both of prototypically human and prototypically inanimate referents; others describe states.

\begin{table}
\caption[Stative verb stems]{Stative verb stems}

\begin{tabular}{ll}
\lsptoprule
Stem & Gloss \\
\midrule
\textit{-ima} & be cooked \\
\textit{-kipÿpa} & be white\\
\textit{-kubÿu} & be drunk \\
\textit{-kujemu} & be angry \\
\textit{-kujÿma} & have fever \\
\textit{-mÿra} & be dry \\
\textit{-pÿkubai} & be lazy\\
\textit{-sabana} & be fat\\
\textit{-sakue} & be salty \\
\textit{-sÿei} & be cold \\
\textit{-tibe} & be sweet\\
\textit{-tÿkÿmiu} & be quiet \\
\textit{-yu} & be ripe \\
\textit{-yÿsi} & be hot, be warm\\
\lspbottomrule
\end{tabular}

\label{table:stat-simple}
\end{table}

%
%probably also: -kunipa = be hungry, -kumuyu = be dirty, -sabana = be fat, -tÿrÿrÿ = be hard, sachukubÿ = be hard-working, -sÿbe? - -asÿbe = sufrir  only irr in corpus, timupike = it is dark, -sakue = salty
%
%but active verbs: be full = kupunÿku/-ka, be moldy = burujiku/-ka?
%
\is{verbal root|)}

\subsection{Verbs that end in \textit{-umi}}\label{sec:StativeVerbs-mi}
\is{verbal stem|(}
\is{derivation|(}

A few stative verbs end in the sequence \textit{umi}. However, to my knowledge, only two of them have a counterpart without this final sequence, and it is not possible to derive new words with this root.\is{nominal root} All verbs that end in \textit{-umi} denote emotions and feelings that are typically associated with humans.\footnote{Another number of other bodily sensations and emotions are expressed by deponent middle verbs\is{middle voice} (see \sectref{sec:Middle_voice}).} The root probably derives from\is{lexicalisation} an obsolete noun with the meaning ‘heart’; this noun had the form \textit{-omiri} in Old Mojeño\is{Mojeño languages} (Rose 2021, p.c.). Note that some \isi{Baure} verbs denoting emotions and physical feelings include a root \textit{-’in(o)} that is also found in the noun \textit{etko’in} ‘heart’ \citep[231]{Danielsen2007}, and it is quite possible that Paunaka once had the same pattern of word formation. \tabref{table:stat-MI} lists the verb stems that end in \textit{-umi}.

\begin{table}
\caption[Verbs that end in \textit{umi}]{Verbs that end in \textit{umi}}

\begin{tabular}{lll}
\lsptoprule
Stative stem & Gloss & Related active stem \\
\midrule
\textit{-chÿnumi} & be sad & \\
\textit{-eju(ju)mi} & remember, miss so. & \\
\textit{-jekupumi} & forget & \textit{-jekupu} ‘lose’\\
\textit{-yayaumi} & be happy & \\
\textit{-yÿsÿumi} & feel hot & \textit{-yÿsi} ‘be hot’\\
\lspbottomrule
\end{tabular}

\label{table:stat-MI}
\end{table}

%
\subsection{Attributive verbs}\label{sec:AttributiveVerbs}\is{attributive prefix|(}

Verbs can be derived from nouns with the prefix \textit{ku-}, which directly precedes the \isi{nominal stem}. The prefix goes back to Proto-Ara\-wa\-kan \textit{*ka-} \citep[377]{Payne1991} and has cognates in many \isi{Arawakan languages} \citep[cf.][95]{Aikhenvald1999}.\footnote{\citet[95]{Aikhenvald1999} calls this prefix ‘relative-attributive’; in other papers, she also uses the single terms ‘attributive’  and 'relative’. % attributive in \citep[cf.][449]{Aikhenvald2006}
Outside Arawakan linguistics, the term “attributive verb” is also used for a noun-modifying verb, but this is not what is meant here. The negative counterpart of the attributive prefix, the \isi{privative} prefix \citep[cf.][285]{Michael2014b}, is almost absent from Paunaka, where there are only a few fixed forms.} The prefix is probably related to the homophonous \isi{causative} prefix. Both have in common that they increase \isi{valency} by one participant \citep[+VAL1, cf.][289]{Danielsen2014a}. For causative derivations see \sectref{sec:Causative}.

Only possessable nouns\is{possession} can be combined with \textit{ku-} (see \sectref{sec:Possession}). In the case of alienable nouns, the derived form including the possessed marker \textit{-ne} is chosen, see (\ref{ex:ATTR-al-1}) and (\ref{ex:ATTR-al-2}) below.

The meaning of attributive verbs is classically described as ‘have X, be with X’, but in Paunaka, some attributive verbs have more active meanings (‘make X, do with X’), and a small number of attributive verbs can even be used transitively, though they remain stative by position of \isi{irrealis} marking.

To start with, consider some examples which show the classical, possessive function of the attributive prefix: (\ref{ex:ATTR-inal-1}) to (\ref{ex:attrib-1}).

In (\ref{ex:ATTR-inal-1}), Miguel is talking about Federico, whom he had jokingly called a relative of theirs (the men in Santa Rita usually do not have beards).

\ea\label{ex:ATTR-inal-1}
\begingl
\glpreamble tikujiyumama biparientene\\
\gla ti-ku-jiyumama bi-pariente-ne\\
\glb 3i-\textsc{attr}-beard 1\textsc{pl}-relative-\textsc{possd}\\
\glft ‘our relative has a beard’
\endgl
\trailingcitation{[ump-p110815sf.669]}
\xe

(\ref{ex:ATTR-inal-2}) comes from María S., who could not cultivate her field in 2018 because of problems with her knee.

\ea\label{ex:ATTR-inal-2}
\begingl
\glpreamble kuina nakuesanebu\\
\gla kuina nÿ-a-ku-esane-bu\\
\glb \textsc{neg} 1\textsc{sg}-\textsc{irr}-\textsc{attr}-field-\textsc{dsc}\\
\glft ‘I don’t have a field anymore’
\endgl
\trailingcitation{[rxx-e181017l]} %non-el.!
\xe

(\ref{ex:attrib-1}) comes from Miguel explaining a word to me: \textit{pÿsi}, ‘spirit of the hill’.

\ea\label{ex:attrib-1}
\begingl
\glpreamble pÿsi chija bitÿpi echÿu tikubiunube naka chiyikikeyae\\
\gla pÿsi chi-ija bi-tÿpi echÿu ti-ku-ubiu-nube naka chiyikike-yae\\
\glb pÿsi 3-name 1\textsc{pl}-\textsc{obl} \textsc{dem}b 3i-\textsc{attr}-house-\textsc{pl} here hill-\textsc{loc}\\
\glft ‘\textit{pÿsi} we call the ones who have their houses in the hills (i.e. the owners of the hills)’
\endgl
\trailingcitation{[mxx-n151017l-1.28]}
\xe

In (\ref{ex:attrib-2}), the derived verb does not have a stative meaning, but rather indicates a change of state. This example comes from the recordings by Riester with Juan Ch., who has just stated that it is not good to harvest corn during new moon, and now provides the reason. Note that the \isi{connective} \textit{che(je)puine} ‘because’\is{cause} sounds more like \textit{mapuine} or \textit{bapuine} in all of his speech.

\ea\label{ex:attrib-2}
\begingl
\glpreamble mapuine takukane uchu i chibuka chikane kuinabu binikeneina\\
\gla chejepuine ti-a-ku-kane uchu i chi-buka chi-kane kuina-bu bi-ni-kene-ina\\
\glb because 3i-\textsc{irr}-\textsc{attr}-maggot \textsc{uncert.fut} and 3-finish.\textsc{irr} 3-maggot \textsc{neg}-\textsc{dsc} 1\textsc{pl}-eat-\textsc{nmlz}-\textsc{irr.nv}\\
\glft ‘because then it may get maggots and if the maggots finish it, we do not have any food anymore’
\endgl
\trailingcitation{[nxx-a630101g-1.59-61]}
\xe

In (\ref{ex:ATTR-al-1}) and (\ref{ex:ATTR-al-2}), the meaning of the derived verb is active, ‘do X’.\footnote{Particularly interesting in this regard is the verb \textit{-kuyenu} (\textit{-ku-yenu} \textsc{attr}-wife), which can mean ‘have a wife’, ‘marry’ or ‘have sex’.}

In (\ref{ex:ATTR-al-1}), María C. speaks about the old days in \isi{Altavista}, when she had to work at night.

\ea\label{ex:ATTR-al-1}
\begingl
\glpreamble yuti nikuyuine pan de arroz\\
\gla yuti ni-ku-yui-ne {pan de arroz}\\
\glb night 1\textsc{sg}-\textsc{attr}-bread-\textsc{possd} {rice bread}\\
\glft ‘at night I baked rice bread’
\endgl
\trailingcitation{[cux-c120510l-1.031]}
\xe

(\ref{ex:ATTR-al-2}) is an utterance by the drunken fox in the story of the fox and the jaguarundi as told by Miguel.

\ea\label{ex:ATTR-al-2}
\begingl
\glpreamble “nÿsachutu eka nakusunine”\\
\gla nÿ-sachu-tu eka nÿ-a-ku-suni-ne\\
\glb 1\textsc{sg}-want-\textsc{iam} \textsc{dem}a 1\textsc{sg}-\textsc{irr}-chant-\textsc{possd}\\
\glft ‘“I want to sing now”’
\endgl
\trailingcitation{[jmx-n120429ls-x5.380]}
\xe

The attributive verb \textit{-kuyae} ‘have, possess, own’ is always realised with a third person marker\is{person marking} following the stem. It is possibly the case that it is always used referentially (the restriction “possibly” is due to the fact that in some cases referential and predicative use is hardly distinguishable, e.g. in questions).\is{interrogative clause}

(\ref{ex:attrib-3}) comes from Juana, who speaks about gold that is sometimes found in the woods. The owner is a spirit in this case.

\ea\label{ex:attrib-3}
\begingl
\glpreamble kaku tikuyaechÿ\\
\gla kaku ti-ku-yae-chÿ\\
\glb exist 3i-\textsc{attr}-\textsc{grn}-3\\
\glft ‘there is an owner’
\endgl
\trailingcitation{[jxx-p151020l-2]}
\xe

In an elicitation session on possessive questions,\is{content question} María S. also used the third person marker \textit{-chÿ} on the attributive verb in many of the questions she formed, like the following:

\ea\label{ex:attrib-4}
\begingl
\glpreamble ¿chija tikupeuchÿ?\\
\gla chija ti-ku-peu-chÿ\\
\glb what 3i-\textsc{attr}-animal-3\\
\glft ‘who is the owner of the animal?’
\endgl
\trailingcitation{[rxx-e201231f.02]}
\xe

In the same elicitation session, she also formed some possessive verbs without the attributive prefix. This happened more than once, which is why I find it worth mentioning here, though I cannot say whether other speakers would accept such a verb form.\footnote{There is not a single question about possession produced in spontaneous speech in the corpus.}

\ea\label{ex:attrib-5}
\begingl
\glpreamble ¿chija tipeuchÿ?\\
\gla chija ti-peu-chÿ\\
\glb what 3i-animal-3\\
\glft ‘who is the owner of the animal?'
\endgl
\trailingcitation{[rxx-e201231f.06]}
\xe

On the other hand, I have also heard the attributive verb with ‘house’, taking the \isi{thematic suffix}, which is a suffix of active verbs (see \sectref{sec:ActiveVerbs_TH}). This may be a special case though because the noun for house, \textit{-ubiu}, derives from\is{lexicalisation} a verb (see \sectref{sec:Subordination-i}). Note that the final /u/ of the noun gets lost in the attributive verb.

(\ref{ex:attrib-6}) shows the use of the attributive verb including the thematic suffix. Juana had just told me that a relative of hers lives in a huge house, in which the rooms downstairs are rented:

\ea\label{ex:attrib-6}
\begingl
\glpreamble i nauku anÿke nebu chubiu tikubiku\\
\gla i nauku anÿke nebu chÿ-ubiu ti-ku-ubiku\\
\glb and there up 3\textsc{obl.top.prn} 3-house 3i-\textsc{attr}-reside?\\
\glft ‘and up there, there is the flat of the owner of the house’
\endgl
\trailingcitation{[jxx-p120430l-1.416]}
\xe 

In general, though not always having stative semantics, attributive verbs are intransitive. Usually, they cannot take person markers\is{person marking} to index an object. In the few cases in which an object is logically possible, it can be expressed periphrastically, as in the case of the verb \textit{-kuetea} ‘tell’ from \textit{-etea} ‘language, word’, where the addressee is expressed by an \isi{oblique} form.\footnote{Actually in the case of \textit{-kuetea}, there are a few examples in the corpus in which the verb takes the third person marker \textit{chÿ-}, which is only used with transitive verbs. They can be considered exceptional.}

In (\ref{ex:attrib-7}), Clara uses such a construction. She was addressing Swintha, helping her formulate what she wanted to say: that Federico already told us about the death of María C.’s husband.

\ea\label{ex:attrib-7}
\begingl
\glpreamble tikuetea etÿpi\\
\gla ti-ku-etea e-tÿpi\\
\glb 3i-\textsc{attr}-language 2\textsc{pl}-\textsc{obl}\\
\glft ‘he told you’
\endgl
\trailingcitation{[cux-120410ls.100]}
\xe

%-kunipa
%-kuyae
%-kubiku
%-kupeu
%-kupanakune = have panacú, carry panacú
%-kuchecha = have child
%-kujauki = wear sombrero
%-kujibÿ = blossom
%kukipujiabÿke = wash face
%-kuesta = estar azotado
%& -kuetea = tell
%& -kukane = infest with vermin
%kuch(a|e)b(u)eji = work ku-chabu!
% kubijai = play -> stative?
%kuima and kuyenu
\is{attributive prefix|)}


\subsection{The verbal root \textit{-ÿ} ‘be long’}\label{sec:StativeVerbs_long}

A peculiarity of the \isi{verbal root} \textit{-ÿ} ‘be long’ is that it cannot occur on its own, but needs some additional material that specifies the kind of length that is expressed by the stem. All stems presented in \tabref{table:stat-long-verbs} are lexicalised\is{lexicalisation} and only occur with the third person marker \textit{ti-} ‘3i’.

\begin{table}
\caption{Verbs with the root \textit{-ÿ} ‘be long’}

\begin{tabular}{lll}
\lsptoprule
Stem & Gloss & Related to \\
\midrule
\textit{-ÿbane} & be far & \textit{-bane} ‘\textsc{rem}’\\
\textit{-ÿbutu} & long time & \textit{-bu} ‘\textsc{mid}’ or ‘\textsc{dsc}’ + \textit{-tu} ‘\textsc{iam}’\\
\textit{-ÿpenu} & be deep & \textit{epenue} ‘hole’\\
\lspbottomrule
\end{tabular}

\label{table:stat-long-verbs}
\end{table}

In addition, there is the verb stem \textit{-ÿnai} ‘be tall’, which most probably consists of the root \textit{-ÿ}, the general \isi{classifier} \textit{-na} and an \textit{i} of unknown origin. This verb can also be inflected for first and second person as in (\ref{ex:be-tall}).

\ea\label{ex:be-tall}
\begingl
\glpreamble nÿti nÿjÿku micha, nÿnai\\
\gla nÿti nÿ-jÿku micha nÿ-ÿnai\\
\glb 1\textsc{sg.prn} 1\textsc{sg}-grow good 1\textsc{sg}-be.tall\\
\glft ‘I grew well, I am tall’
\endgl
\trailingcitation{[jxx-p150920l.057]}%el.
\xe


The \textit{-nai} part of this verb can be replaced by body-part nouns to specify that a specific body part is long, see (\ref{ex:long-leg-1}), or the body-part noun is placed before \textit{na} and \textit{i} is dropped as in (\ref{ex:long-leg-2}). It is not clear at this point whether both forms are grammatical. (\ref{ex:long-leg-1}) was elicited together with similar examples with other body parts, while (\ref{ex:long-leg-2}) was uttered spontaneously by the same speaker, Juana. She speaks about different kinds of frogs\is{frog story} in that case.

\ea\label{ex:long-leg-1}
\begingl
\glpreamble tÿjabuji\\
\gla ti-ÿ-jabu-ji\\
\glb 3i-be.long-leg-\textsc{col}\\
\glft ‘it has long legs’
\endgl
\trailingcitation{[jxx-e150925l-1.047]}
\xe


\ea\label{ex:long-leg-2}
\begingl
\glpreamble echÿu punachÿ tÿjabubunajane\\
\gla echÿu punachÿ ti-ÿ-jabu-bu-na-jane\\
\glb \textsc{dem}b other 3i-be.long-leg-\textsc{rdpl}-\textsc{clf:}general?-\textsc{distr}\\
\glft ‘the other one has long legs’
\endgl
\trailingcitation{[jxx-a120516l-a.463]}
\xe

%tÿkeisi, tÿchuka
%tÿpupai?? be a lot? be wide (landscape?)
%


\subsection{Reduplication}\label{sec:StativeVerbs_RDPL}
\is{reduplication|(}

Some stative verb stems always have a reduplicated syllable, among them \textit{-sururu} ‘be clear, be light-coloured, be white’ and \textit{-yayaumi} ‘be happy’. Only a handful of stative verb stems can occur both with and without a reduplicated syllable. Since there are very few examples, I cannot say for sure what the effect of reduplication is on these verb stems, but it probably intensifies the meaning. This is the case in (\ref{ex:RDPL-stat}), where the verb expresses that the fox got quiet forever because he was killed by some dogs after having sung loudly for quite some time and thus attracting the dogs’ attention. The sentence comes from Juana, an intervention into Miguel’s telling of the story about the fox and the jaguar.

\ea\label{ex:RDPL-stat}
\begingl
\glpreamble titÿkÿkÿmiu kupisaÿrÿ\\
\gla ti-tÿkÿ-kÿ-miu kupisaÿrÿ\\
\glb 3i-be.quiet-\textsc{rdpl}-be.quiet fox\\
\glft ‘the fox got very quiet’
\endgl
\trailingcitation{[jmx-n120429ls-x5.448]}
\xe

In the case of the verb \textit{-eju(ju)mi} ‘remember’, the form without reduplication is preferred in negative statements,\is{negation} and the one including reduplication in positive statements, in which the verb is used transitively in most cases,\footnote{It seems to be the case that, when used transitively, this verb does not inflect for \isi{irrealis} at all, even in negative contexts, which is highly unusual.} though this cannot be generalised at all.

(\ref{ex:RDPL-stat-1}) is an example of the form not including reduplication. It comes from elicitation with María S.

\ea\label{ex:RDPL-stat-1}
\begingl
\glpreamble kuina naejumibu\\
\gla kuina nÿ-a-ejumi-bu\\
\glb \textsc{neg} 1\textsc{sg}-\textsc{irr}-remember-\textsc{dsc}\\
\glft ‘I don’t remember anymore’
\endgl
\trailingcitation{[rxx-e181022le]}
\xe

In (\ref{ex:RDPL-stat-2}), the form including the reduplicated syllable is used transitively. The example also comes from María S., who speaks with her brother Miguel about me. Swintha, who recorded this, had travelled to Bolivia for a second time in 2012, while I stayed in Germany.

\ea\label{ex:RDPL-stat-2}
\begingl
\glpreamble tichÿnumiji tejujumibi\\
\gla ti-chÿnumi-ji ti-eju-ju-mi-bi\\
\glb 3i-be.sad-\textsc{rprt} 3i-remember-\textsc{rdpl}-remember-1\textsc{pl}\\
\glft ‘she says she is sad, because she misses (lit.: remembers) us’
\endgl
\trailingcitation{[rmx-c121126s.19]}
\xe

%tamÿraraji = the clothes are drying, but here also irrealis
% ejujumi
%ji(yu)mama
\is{reduplication|)}

\subsection{Classifiers}\label{sec:StativeVerbs_CLF}\is{classifier|(}

Not all stative verbs can take classifiers, only the ones that express non-human\is{animacy} qualities or states. The classifier refers to the \isi{subject} of the stative verb. Some examples follow. 

In (\ref{ex:CLF-stat-1}), the classifier \textit{-umu} for liquid things is added to the verb stem \textit{-sÿei} ‘be cold’, thus providing the information that the cold thing in question is a liquid. Juana speaks about the good properties of clayware here.

\ea\label{ex:CLF-stat-1}
\begingl
\glpreamble tisÿeimu ÿne yÿpikÿ\\
\gla ti-sÿei-umu ÿne yÿpi-kÿ\\
\glb 3i-be.cold-\textsc{clf:}liquid water jar-\textsc{clf:}bounded\\
\glft ‘the water stays cold inside the jar’
\endgl
\trailingcitation{[jxx-p120430l-2.601]}
\xe

In (\ref{ex:CLF-stat-2}) the classifier \textit{-pa} for dusty things specifies that what is dry is something that consists of small particles, in this case the earth. It comes from Juana who speaks about the search for water in the old days, before the reservoir was constructed.

\ea\label{ex:CLF-stat-2}
\begingl
\glpreamble timÿrapa epuke\\
\gla ti-mÿra-pa epuke\\
\glb 3i-be.dry-\textsc{clf:}particle ground\\
\glft ‘the earth in the ground was dry’
\endgl
\trailingcitation{[jxx-p120515l-2.020]}
\xe

In (\ref{ex:CLF-stat-3}), the classifier \textit{-pai} is used with the stative verb stem \textit{-yÿsi} ‘be hot’. Note that the first verb, which is an active verb, also takes \textit{-pai} here to refer to the same concept (see discussion below in \sectref{sec:CLF_ActiveVerbs}). The sentence was produced by María C. as a warning directed to me. People believe that one can get seriously ill when sitting on a hot surface.

\ea\label{ex:CLF-stat-3}
\begingl
\glpreamble titibubupaikumÿnÿ tiyÿsipai echÿu pijinepÿimÿnÿ\\
\gla ti-tibubu-pai-ku-mÿnÿ ti-yÿsi-pai echÿu pi-jinepÿi-mÿnÿ\\
\glb 3i-sit-\textsc{clf:}ground-\textsc{th}1-\textsc{dim} 3i-be.hot-\textsc{clf:}ground \textsc{dem}b 2\textsc{sg}-daughter-\textsc{dim}\\
\glft ‘your little daughter is sitting on the hot ground’
\endgl
\trailingcitation{[uxx-p110825l.166]}
\xe

The general classifier can also be attached to stative verb stems. This is the case in (\ref{ex:CLF-stat-4}), where Juana translated to Paunaka on request what she had said before in Spanish. She speaks about different types of chicken here.

\ea\label{ex:CLF-stat-4}
\begingl
\glpreamble kuina tinijanea eka tikipÿpanaji, eka tisina tipisÿna tinijaneu entero amuke\\
\gla kuina ti-ni-jane-a eka ti-kipÿpa-na-ji eka ti-si-na ti-pisÿ-na ti-ni-jane-u entero amuke\\
\glb \textsc{neg} 3i-eat-\textsc{distr}-\textsc{irr} \textsc{dem}a 3i-be.white-\textsc{clf:}general-\textsc{col} \textsc{dem}a 3i-be.red-\textsc{clf:}general 3i-be.black-\textsc{clf:}general 3i-eat-\textsc{distr}-\textsc{real} whole corn\\
\glft ‘the white ones don’t eat it, the red and black ones eat the whole corn kernels (i.e. without crushing them)’
\endgl
\trailingcitation{[jxx-e150925l-1.143-144]}
\xe


%check: -miu = sky, water, on top of sth.??

%sÿipai, yÿsipai, sururupai, tisipai, pujepai = plano
% tikiumu = harto

%tisipi tisururupi tikipÿpana kipÿpaki -> tisi = adj or v?
\is{derivation|)}
\is{classifier|)}

\subsection{Incorporation}\label{sec:StativeVerbs_INC}\is{incorporation|(}

Stative verbs denoting qualities can be combined with body- or plant-part terms to confine the expressed quality to the specific part of the referent. Some examples follow.

(\ref{ex:INC-stat-3}) comes from the recordings by Riester with Juan Ch., who speaks about an imagined theft of a young woman here.

\ea\label{ex:INC-stat-3}
\begingl
\glpreamble titÿrÿrÿpetemÿnÿ micha\\
\gla ti-tÿrÿrÿ-pete-mÿnÿ micha\\
\glb 3i-be.hard-vagina-\textsc{dim} good\\
\glft ‘she has a hard vagina (i.e. she is a virgin)’
\endgl
\trailingcitation{[nxx-a630101g-3.023]}
\xe

(\ref{ex:INC-stat-6}) was produced by Juana. It comes from her narration of the story about the fox and the jaguar together with Miguel and refers to the state of the jaguar’s body after he had drowned some months before because of being deceived by the fox. 

\ea\label{ex:INC-stat-6}
\begingl
\glpreamble chijikiuji isini tisikererekebetu metu tibÿrutu\\
\gla chijikiu-ji isini ti-sikere-re-kebe-tu metu ti-bÿru-tu\\
\glb however-\textsc{rprt} jaguar 3i-be.naked-\textsc{rdpl}-tooth-\textsc{iam} already 3i-be.rotten-\textsc{iam}\\
\glft ‘however, the jaguar was stripped to his teeth, he was already rotten, it is said’
\endgl
\trailingcitation{[jmx-n120429ls-x5.288]}
\xe

The following two examples were elicited: (\ref{ex:INC-stat-4}) comes from Miguel, while (\ref{ex:INC-stat-5}) has the same verb but with a plant-part term and comes from Juana.

\ea\label{ex:INC-stat-4}
\begingl
\glpreamble tisururubÿke\\
\gla ti-sururu-bÿke\\
\glb 3i-be.clear-face\\
\glft ‘she is pale-faced’
\endgl
\trailingcitation{[mdx-c120416ls.137]}%el.
\xe

\ea\label{ex:INC-stat-5}
\begingl
\glpreamble tisururupune\\
\gla ti-sururu-pune\\
\glb 3i-be.clear-leaf\\
\glft ‘white leaf’
\endgl
\trailingcitation{[jxx-e150925l-1.180]}
\xe

Interestingly, the combination of verbal roots\is{verbal root} with body-part nouns often results in stative verb stems. This is illustrated in (\ref{ex:INC-stat-1}), where the root \textit{-ja} ‘be open’ (which also forms part of the active verb stem \textit{-jajaku} ‘be wide’) is combined with \textit{-naba} ‘inside of the mouth’. The \isi{irrealis} prefix shows that the verb is stative.

(\ref{ex:INC-stat-1}) comes from the same story as (\ref{ex:INC-stat-6}) above, but at this point, the jaguar is still alive. When he obeys and opens his mouth, the vulture whom he had caught flies away and defecates into his open mouth.

\ea\label{ex:INC-stat-1}
\begingl
\glpreamble “¡pajanaba!”\\
\gla pi-a-ja-naba\\
\glb 2\textsc{sg}-\textsc{irr}-open-mouth.inside\\
\glft ‘“open your mouth!”’
\endgl
\trailingcitation{[jmx-n120429ls-x5.197]}
\xe

A number of semantically extended stative verbs stems contain the incorporated body part \textit{-bÿke} ‘face’. Many of them have to do with sight \citep[267]{TerhartDanielsenBODY}. One example is given in (\ref{ex:INC-stat-2}), where the active stem\is{active verb} \textit{-imu} ‘see’ combines with \textit{-bÿke}, and the resulting form \textit{-imubÿke} is a stative verb with the meaning ‘see well, be capable of seeing’. It comes from María C.

\ea\label{ex:INC-stat-2}
\begingl
\glpreamble kuinabu naimubÿkebu\\
\gla kuina-bu nÿ-a-imu-bÿke-bu\\
\glb \textsc{neg}-\textsc{dsc} 1\textsc{sg}-\textsc{irr}-see-face-\textsc{dsc}\\
\glft ‘I can’t see (well) anymore’
\endgl
\trailingcitation{[uxx-p110825l.013]}
\xe


%\ea\label{ex:}
%\begingl
%\glpreamble tijanababaikuji echÿu isini\\
%\gla ti-ja-naba-baiku-ji echÿu isini\\
%\glb 3i-open-mouth-\textsc{cont}-\textsc{rprt} \textsc{dem}b jaguar\\
%\glft 'he opened his mouth’\\
%\endgl
%\trailingcitation{[jmx-n120429ls-x5.207]}
%\xe
%---> ja is a stative verb! -baiku is the continuous form

\is{incorporation|)}
\is{verbal stem|)}

All derivational processes of the stative verb stem that are productive and/or traceable have been described now.\is{intransitive|)}\is{stative verb|)} The following section is dedicated to the active verb stem.

\section{Active verbs}\label{sec:ActiveVerbs}\is{active verb|(}

Active verb stems minimally consist of a verb root.\is{verbal root} Some of these roots do not require any further material to be ready for inflection, others take a \isi{thematic suffix}. Reduplication,\is{reduplication} aktionsart suffixes,\is{intensive aktionsart} classifiers,\is{classifier} and incorporated nouns\is{incorporation} can occur between root and \isi{thematic suffix}.

\subsection{Simplex active verb stems}\label{sec:SimplexActiveVerbs}
\is{verbal root|(}

Some active verbs stems are identical to the verbal root, i.e. they do not take a \isi{thematic suffix}. These simplex verbs can be mono- or disyllabic. It is also possible that some of them have an additional initial \isi{syllable} \textit{i}. Since stem-initial high front vowels merge with the vowel of the person marker, it is impossible to say whether a stem starts with \textit{i} or not, unless there exists a stative \isi{derivation}\is{stative verb} of the stem. The \isi{irrealis} form of the stative derivation can reveal an initial \textit{i}, as is the case with the verb stem \textit{-imu} ‘see’ with the stative derivation being \textit{-imubÿke} ‘see well’ as in (\ref{ex:INC-stat-2}) in \sectref{sec:StativeVerbs_INC} above.

Tables \ref{table:BareVerbStems1} and \ref{table:BareVerbStems2} list some verb stems that are identical to verb roots, i.e. they do not have thematic suffixes. A peculiarity of the forms in \tabref{table:BareVerbStems1} is that they do not mark \isi{irrealis} by change of the last \textit{u} to \textit{a}, but by addition of \textit{-a}.\is{reality status|(} Comparison with \isi{Mojeño Trinitario} reveals that at least some of these stems must have had an additional final syllable in a prior state of the language; three cognate stems could be identified: \textit{-imo’o} ‘see’, \textit{-iso’o} ‘weed’ and \textit{-iyo’o} ‘cry’ (Rose 2021, p.c.).\footnote{The comparison equally shows that these stems begin with \textit{i}, as is also in concordance with the \isi{stress} patterns of these verbs (see \sectref{sec:Stress}).}

\begin{table}
\caption{Verb stems without a thematic suffix with addition of irrealis marker \textit{-a}}

\begin{tabular}{lll}
\lsptoprule
Realis form & Irrealis form & Gloss \\
\midrule
\textit{-imu} & \textit{-imua} & see \\
\textit{-isu} & \textit{-isua} & weed \\
\textit{-iyu} & \textit{-iyua} & cry; sing, shout (animals) \\
\textit{-ju} & \textit{-jua} & urinate\\
\textit{-maku} & \textit{-makua} & bury\\
\textit{-pu} & \textit{-pua} & give \\
\textit{-tibu} & \textit{-tibua} & sit down \\
\lspbottomrule
\end{tabular}

\label{table:BareVerbStems1}
\end{table}

\begin{table}
\caption{Verb stems without a thematic suffix with change of the last vowel to \textit{a} for irrealis RS}

\begin{tabular}{lll}
\lsptoprule
Realis form & Irrealis form & Gloss \\
\midrule
\textit{-beu} & \textit{-bea} & take away, take out \\
\textit{-benu} & \textit{-bena} & lie down, fall \\
\textit{-bÿsÿu} & \textit{-bÿsÿa} & come \\
\textit{-chemu} & \textit{-chema} & stand up, rise \\
\textit{-eu} & \textit{-ea} & 1. drink 2. hit \\
\textit{-umu} & \textit{-uma} & take \\
\textit{-samu} & \textit{-sama} & hear \\
\textit{-yunu} & \textit{-yuna} & go \\
\lspbottomrule
\end{tabular}

\label{table:BareVerbStems2}
\end{table}

\is{reality status|)}

%-yu --> -yuyuiku
%-imu --> imumuku
%-yunu --> yuiku
%-samu --> samuiku
%-eu --> eiku
%-umu --> umeiku
%-chupu --> chupuiku
%-isu
%-beu
%-chemu -> chemumuiku, chepapaku (be awake)
%-benu
%-tibu
%
\is{verbal root|)}

\subsection{Verb stems with a thematic suffix}\label{sec:ActiveVerbs_TH}
\is{verbal stem|(}
\is{derivation|(}
\is{thematic suffix|(}

There are two thematic suffixes in Paunaka, \textit{-ku} and \textit{-chu}, which both have cognate forms in the other \isi{Southern Arawakan} languages. 
%Terena p. 132-134

Thematic suffixes are only found on active verbs, which is why \citet[380]{Rose2014b} calls the Trinitario\is{Mojeño Trinitario} cognates, which she suggests are allomorphs, an “active suffix”. \citet[240--244]{Danielsen2007} argues that the \isi{Baure} cognate of \textit{-ku} is an absolute suffix, which interacts with transitivity, and the cognate of \textit{-chu} is an applicative. Paunaka’s \textit{-chu} could also be interpreted as an applicative \is{applicative|(} marker, but contrary to \isi{Baure}, we do not find many verb roots that can take both suffixes.\footnote{In \isi{Baure}, both suffixes may even occur together on a single verb in some specific cases (Danielsen 2020, p.c.).} The only exception of an active verb I know of is the root \textit{-su} ‘write’, which can be combined with the \isi{extension applicative} suffix \textit{-i} and \textit{-ku} into the stem \textit{-suiku} ‘write’, as well as with the suffix \textit{-chu} into \textit{-suchu} ‘write something’.\footnote{The last form has also been found with additional \textit{-i}, as \textit{-suichu}. In one context, this verb was used in the sense of ‘write down’ (a name in that case), in the other it meant ‘enroll’ and was realised with a first person singular object.} This pair of verb stems suggests an interpretation of \textit{-chu} as an applicative. Additional support comes from the derivation of active from stative verbs,  e.g. \textit{-jÿchu} ‘light (fire)’ from \textit{-ijÿe} ‘burn’ and \textit{-eimachu} ‘cook until done’ from \textit{-ima} ‘be cooked, done’, but this kind of active verb stem derivation also seems to be used very infrequently. I thus decided to call both suffixes ‘thematic suffixes’, \textit{-ku} being glossed as ‘\textsc{th}1’ and  \textit{-chu} as ‘\textsc{th}2’.\is{applicative|)}

\tabref{table:Thematic1} lists verb stems that are usually realised with the thematic suffix \textit{-ku} and \tabref{table:Thematic2} list verb stems with \textit{-chu}. The thematic suffix \textit{-ku} occurs much more frequently than \textit{-chu}.

\begin{table}
\caption{Verb stems with the thematic suffix \textit{-ku}}

\begin{tabular}{lll}
\lsptoprule
Realis form & Irrealis form & Gloss \\
\midrule
\textit{-buku} & \textit{-buka} & finish\\
\textit{-muku} & \textit{-muka} & sleep\\
\textit{-jÿku} & \textit{-jÿka} & grow\\
\textit{-maku} & \textit{-maka} & bury\\
\textit{-marÿku} & \textit{-marÿka} & cut\\
\textit{-niku} & \textit{-nika} & eat\\
\textit{-paku} & \textit{-paka} & die\\
\textit{-punaku} & \textit{-punaka} & give\\
\textit{-ramuku} & \textit{-ramuka} & thunder\\
\textit{-seku} & \textit{-seka} & dig hole\\
\lspbottomrule
\end{tabular}

\label{table:Thematic1}
\end{table}


\begin{table}
\caption{Verb stems with the thematic suffix \textit{-chu}}

\begin{tabular}{lll}
\lsptoprule
Realis form & Irrealis form & Gloss \\
\midrule
\textit{-akachu} & \textit{-akacha} & lift\\
\textit{-ekichu} & \textit{-ekicha} & invite\\
\textit{-kechu} & \textit{-kecha} & say\\
\textit{-kipuchu} & \textit{-kipucha} & wash\\
\textit{-kusachu} & \textit{-kusacha} & fish with hook, angle\\
\textit{-sachu} & \textit{-sacha} & want\\
\textit{-sumachu} & \textit{-sumacha} & want, like\\
\textit{-yÿseuchu} & \textit{-yÿseucha} & greet\\
\lspbottomrule
\end{tabular}

\label{table:Thematic2}
\end{table}
%-eteumichu = avisar

If Spanish loans\is{borrowing|(} are verbalised, it is usually the suffix \textit{-chu} that is attached to the stem.\footnote{More frequently, however, verbs borrowed from Spanish are integrated as non-verbal predicates,\is{non-verbal predication} see \sectref{sec:borrowed_verbs}.} Loan verb integration by this suffix is also found in \isi{Mojeño Trinitario} and \isi{Baure} \citep[4]{Terhart_subm}. 

(\ref{ex:loan-chu}) shows a verb borrowed from Spanish. The verbal root \textit{-ayurau}, derived from the Spanish participle \textit{ayudado} ‘helped’, takes the thematic suffix \textit{-chu} to yield the stem \textit{-ayurauchu} ‘help’.

\ea\label{ex:loan-chu}
\begingl
\glpreamble tayurauchunÿ\\
\gla ti-ayurau-chu-nÿ\\
\glb 3i-help-\textsc{th}2-1\textsc{sg}\\
\glft ‘she helps me’
\endgl
\trailingcitation{[mxx-n101017s-2.054]}
\xe
\is{borrowing|)}

The suffix \textit{-chu} is also found in some irregular derivations, e.g. the pluractional \textit{-paikechu} ‘all die’ from \textit{-paku} ‘die’ and equally \textit{-kupaikechu} ‘kill all’ from \textit{-kupaku} ‘kill’.\is{thematic suffix|)}

%
\subsection{The extension applicative}\label{sec:EXTApplicative}\is{aktionsart|(}\is{extension applicative|(}

The extension applicative suffix \textit{-i} (‘\textsc{ext}’) directly precedes the thematic suffixes,\is{thematic suffix} usually \textit{-ku}, sometimes also \textit{-chu}. This suffix is completely lexicalised\is{lexicalisation} with the verb stems and thus hard to grasp. The term “extension applicative” comes from \citet[]{Danielsen2014a} who describes a cognate form of this suffix for \isi{Baure}.\footnote{The same \isi{Baure} suffix was called “durative” before \citep[cf.][232--234]{Danielsen2007}.} According to her analysis, the suffix has at least two functions: first, it can decrease the \isi{valency} of a transitive verb by deriving a durative form, and second, it can change the semantic role of the object \citep[297, 299]{Danielsen2014a}.\footnote{A third possible function identified by \citet[300]{Danielsen2014a}, change of direction of an event (exemplified by the pair ‘buy’ and ‘sell’ for \isi{Baure}), is absent from Paunaka, as far as I can tell.}

In Paunaka, verbs with the extension applicative suffix can be \isi{transitive} or \isi{intransitive}. Many of them are atelic,\is{telicity} but not all of them. There is a huge number of verbs which take the marker, so many that I first thought I was dealing with a (morpho-)phonological rule (prepalatalisation) rather than with a morpheme. There are, however, a number of verbs that are never realised with \textit{-i}. 

Some verb stems with the extension suffix and the thematic suffix \textit{-ku} are listed in \tabref{table:AKT-i}.

\begin{table}
\caption{Verb stems with the extension applicative and thematic suffix \textit{-ku}}

\begin{tabular}{ll}
\lsptoprule
Verb stem (realis) & Gloss \\
\midrule
\textit{-beiku} & lie\\
\textit{-bÿcheiku} & send so., make so. do\\
\textit{-majaiku}  & bark\\
%\textit{-mesumeiku}  & teach\\
\textit{-musuiku}  & wash clothes\\
\textit{-nÿnÿiku} & live, be alive\\
\textit{-epuiku} & fish with net\\
\textit{-semaiku}  & search, look for\\
\textit{-sipuiku}  & pay\\
\textit{-suiku}  & write\\
%\textit{-yÿbamukeiku}  & mill corn\\
\textit{-yÿbuiku}  & shout\\
\textit{-yÿseiku} & buy\\
\lspbottomrule
\end{tabular}

\label{table:AKT-i}
\end{table}

%\tabref{table:AKT-i-chu} lists those stems with the applicative marker and the thematic suffix \textit{-chu}. Some of the latter include a syllable \textit{pu}, which is most probably another derivational suffix, but I was not able to identify its meaning. Some can also alternatively be realised with \textit{-ku}.
%
%\begin{table}
%\caption{Verb stems with the extension applicative and thematic suffix \textit{-chu}}
%
%\begin{tabularx}{\textwidth}{LL}
%\lspbottomrule
%Verb stem (realis) & Gloss \\
%\lspbottomrule
%\textit{-buichu} & slash-and-burn\\
%\textit{-jabipuichu} & pluck (birds)\\
%\textit{-musupuichu} & wash for so.\\
%\textit{-papapuichu} & make clay ball \\
%\lspbottomrule
%\end{tabularx}
%
%\label{table:AKT-i-chu}
%\end{table}
%-sabeichuyÿkukÿa = andar por orilla

Only a few related verbs differ in presence or absence of \textit{-i}, which makes the analysis very difficult. The pairs I have found in the corpus are given in \tabref{table:AKT-i-deriv}.\footnote{Note that the pair \textit{-kuyeneu} and \textit{-kuyeneiku} ‘visit’ strictly speaking belongs to the stative verbs\is{transitive stative verb} by composition of the stem and irrealis marking, but \textit{-kuyeneu} is transitive nonetheless, and \textit{-kuyeneiku} takes the thematic suffix, which usually occurs on active verbs only, see \sectref{sec:StativeVerbs} above. There is possibly another pair: \textit{-yÿseiku} ‘buy’ and \textit{-yÿseuchu} ‘greet’. The first of them is probably borrowed from Guarayu \textit{yusei} ‘want or wish sth. edible’. It is not clear whether there is any derivational relation between the two stems. The similarity could also be coincidental.} Among them, we find some that suggest a durative derivation, like \textit{-samu} ‘hear’ and \textit{-samuiku} ‘listen’, so this speaks for an analysis as an aktionsart suffix. Others rather differ in involvement of a human being in the event, although this human being is not always involved in the same type of stem, i.e. either the one without or the one with the applicative, and it does not necessarily have to be realised as an \isi{object}, as e.g. in the pair \textit{-umu} ‘take’ (whose object can be human or non-human) and \textit{-umeiku} ‘steal’, whose object is usually non-human, but a human is necessarily involved as the victim of the theft. It is thus not precisely an applicative which changes the semantic role of the object.

\begin{table}
\caption{Related verb stems without and with the extension applicative suffix}

\begin{tabularx}{\textwidth}{lllQ}
\lsptoprule
Simple stem & Gloss & Stem with applicative & Gloss \\
\midrule
\textit{-chupu} & know (a fact) & \textit{-chupuiku} & know so./sth.\\
\textit{-etuku} & put & \textit{-etuichu} & put into sth.\\
\textit{-jajaku} & be wide & \textit{-jajaiku} & extend \\
\textit{-kupaku} & kill & \textit{-kupaiku} & slaughter\\
\textit{-kuyeneu} & visit so. & \textit{-kuyeneiku} & visit a place, stroll around\\
\textit{-samu} & hear & \textit{-samuiku} & listen\\
\textit{-umu} & take & \textit{-umeiku} & steal\\
\textit{-yunu} & go & \textit{-yuiku} & walk\\
\lspbottomrule
\end{tabularx}

\label{table:AKT-i-deriv}
\end{table}\is{extension applicative|)}


\subsection{Intensive aktionsart}\label{sec:IntensiveAktionsart}\is{intensive aktionsart|(}

The suffix \textit{-ji} shows up on many verb stems whose semantics includes an intensive degree, and so it can be analysed as an aktionsart suffix. It is only found on active verbs with the \isi{thematic suffix} \textit{-ku} and directly precedes it. Just as is the case with extension applicative \textit{-i}, there are not many verb stems that can be realised with or without the suffix, so that it is hard to determine the exact meaning of it. It is also possible that the sequence \textit{ji} is just part of the verb root, especially with the shorter, trisyllabic stems.\footnote{This is the analysis I prefer for the verb \textit{-majiku} ‘crawl’, considering that the \isi{classifier} \textit{-pai} ‘\textsc{clf:}ground’ may be added \textit{after} the syllable \textit{ji} (with additional \isi{reduplication} of that syllable, see (\ref{ex:act-combi-1}) in \sectref{sec:ActiveVerbs_Combi}), while in the stem \textit{-yÿtipajiku} ‘make chicha’, the \isi{classifier} \textit{-pa} ‘\textsc{clf:}particle’ precedes the syllable \textit{ji}, which is in this case analysed as the aktionsart suffix.} There is also a homophonous \isi{classifier}\textit{-ji} for soft masses (see \sectref{sec:Classifiers}), which has to be distinguished from the aktionsart suffix. Both occur in different slots, theoretically, but in all actual verb forms found in the corpus, they both directly precede the thematic suffix, so that knowledge about position inside the stem does not help in this case. \tabref{table:AKT-ji} lists some verbs which do or may contain the intensive suffix \textit{-ji}.

\begin{table}
\caption{Verb stems with the aktionsart suffix \textit{-ji}}

\begin{tabular}{ll}
\lsptoprule
Verb stem (realis) & Gloss \\
\midrule
\textit{-chujiku} & harvest corn\\
\textit{-kerajiku} &  break (in two pieces)\\
\textit{-kupujiku} &  1. meet 2. come or go down, descend\\
\textit{-kurumejiku} &  pierce\\
\textit{-kuyajijiku} &  laugh, laugh about\\
\textit{-nejiku} &  leave so. or sth.\\
\textit{-rabajiku} &  break (into several pieces) \\
\textit{-tÿyajiku} &  grind in mortar\\
\textit{-ujiku} &  suckle\\
\textit{-yejiku} &  tear out/off, pluck, harvest\\
\textit{-yÿtipajiku} &  make chicha\\
\lspbottomrule
\end{tabular}

\label{table:AKT-ji}
\end{table}


Verb stems that can be realised with or without the aktionsart suffix are listed in \tabref{table:AKT-ji-deriv}.\footnote{The two homographical verb stems \textit{-chujiku} in Tables \ref{table:AKT-ji} and \ref{table:AKT-ji-deriv} are distinguished by different \isi{stress}: /ˈʧuhiku/ ‘speak, talk’, /ʧuˈhiku/ ‘harvest corn’.} They all include more changes than absence vs. presence of the aktionsart suffix. The same statement made for the extension applicative also holds for the intensive aktionsart suffix: in general, it either always forms part of the stem or never, with only few exceptions.

\begin{table}
\caption{Related verb stems without and with the intensive aktionsart suffix}

\begin{tabularx}{\textwidth}{lllQ}
\lsptoprule
Simple stem & Gloss & Intensive form & Gloss \\
\midrule
\textit{-bikÿku} & throw & \textit{-bikÿjiku} & throw away\\
\textit{-chupu} & know & \textit{-chujiku} & speak, talk\\
%\textit{-jatÿku} & grab & \textit{jatÿjiku} & throw weed out of the water\\
\textit{-kutikubu} & run & \textit{-kutijiku} & escape, flee\\
\textit{-teku} & call, invite & \textit{-tetejiku} & call shouting, call iteratively\\
\textit{-yÿbaiku} & grind, mill & \textit{-yÿbajiku} & grind (manioc, clay etc.) \\
\lspbottomrule
\end{tabularx}

\label{table:AKT-ji-deriv}
\end{table}
\is{intensive aktionsart|)}\is{aktionsart|)}

\subsection{Verbs with \textit{-nÿ}}\label{sec:ActiveVerbs_around}

If something or someone is completely surrounded, the suffix \textit{-nÿ} ‘around’ is likely to form part of the verb stem. For a few stems, there is no corresponding form without the suffix. \tabref{table:DIR-nÿ-deriv} gives some verb stems that contain the suffix, in comparison to stems without it where possible.

\begin{table}
\caption{Related verb stems without and with the suffix \textit{-nÿ}}

\begin{tabularx}{\textwidth}{lllQ}
\lsptoprule
Simple form & Gloss & Derived form & Gloss \\
\midrule
\textit{-bejaiku} & take away & \textit{-bejanÿku} & unlock (i.e. take away complete surrounding)\\
\textit{-juku} & pour solid things & \textit{-jukunÿku} & bury (i.e. put earth around someone)\\
&  & \textit{-kupunÿku} & be full (with food)\\
 &  & \textit{-pitanÿku} & embrace\\
\textit{-rataku} & press & \textit{-ratanÿku} & lock (i.e. completely surround) \\
\textit{-rÿtÿku} & tie to sth. & \textit{-rÿtÿnÿku} & tie up\\
\lspbottomrule
\end{tabularx}

\label{table:DIR-nÿ-deriv}
\end{table}
%-pujunÿpaiku = empujar -> ø
% -pujÿnÿku = empujar
%tibÿkupujaneiji bakayayae baka kakujane chiratanÿkuji trankera, jxx-p151016l-2.166, they went into the barn and when the cows were in there, they locked with bolt

\subsection{Reduplication and continuous marking}\label{sec:ActiveVerbs_RDPL}
\is{reduplication|(}

In some of the verb stems of Paunaka, we find a syllable of the verbal root reduplicated. These verbs often encode iterative or durative, and sometimes intensive events, so that reduplication seems to encode the same semantic notion as the applicative\is{extension applicative} and aktionsart \is{intensive aktionsart} suffixes \textit{-i} and \textit{-ji} (see \sectref{sec:EXTApplicative} and \sectref{sec:IntensiveAktionsart}). As far as I know, there is only one verb root with more than one reduplicated syllable, which is \textit{-musimusiku} ‘blink, wink’ (with a related noun \textit{-musipa} ‘eyelash’). The great majority of verb stems are instances of partial reduplication of a single syllable.
Verbs with a reduplicated root syllable almost exclusively take the \isi{thematic suffix} \textit{-ku}. Some of these verb stems are listed in \tabref{table:RDPL}, including two that repeat the syllable twice.


\begin{table}
\caption{Verb stems with a reduplicated root syllable}

\begin{tabular}{ll}
\lsptoprule
Verb stem (realis) &  Gloss \\
\midrule
\textit{-bibiku} &  swing in hammock\\
\textit{-buririku} & fall out (hair)\\
\textit{-bururuku} &  boil\\
\textit{-bÿbÿku} &  fly\\
\textit{-nÿnÿiku} &  live, be alive\\
\textit{-pÿsisikubu} &  be alone\\
\textit{-pÿsisisiku} &  smoke (intr.)\\
\textit{-terere(i)kubu} & be afraid\\
\textit{-yapipipiku} &  wag tail\\
\lspbottomrule
\end{tabular}

\label{table:RDPL}
\end{table}

%akakachu = alzar? está alzando?
%-bemususuku = häuten
%-jaririku = arrastrar
%-jibÿbÿku = smoke -> jibÿku
%-keraraku = quebrar -> keraku
%-muyayachu = be slow  --> check IRR!
%nijababauku = bite several times, nijabaku
%-pÿrÿrÿpeubu -> stative
%-tetejiku = call
%yayaumi -> stative
%-titiuku = tie -> check
%-niratajijiku = chew
%-japipiku (-> tijapipiku) = blitzen -> stative?
%-jajaiku = ?
%-rechechechepaiku = bambolear

Only a handful of verbs are also found without the reduplicated syllable. They are listed in \tabref{table:RDPL-deriv}.

\begin{table}
\caption{Related verb stems without and with reduplication of a root syllable}

\begin{tabularx}{\textwidth}{lllQ}
\lsptoprule
Simple form & Gloss & Derived form & Gloss \\
\midrule
\textit{-benu} & lie down & \textit{-benunuku(bu)} & lie\\
\textit{-chubiku} & stroll, hunt & \textit{-chubibiku} & stroll, hunt\\
\textit{-chujiku} & speak, talk & \textit{-chujijiku} & talk, converse\\
\textit{-imu} & see & \textit{-imumuku} & look\\
\textit{-keraku} & break  & \textit{-keraraku} & break (sth. thick, heavy)\\
\textit{-rÿtÿku} & tie & \textit{-rÿtÿtÿku} & tie wrapping around several times \\
\textit{-samu} & hear, understand & \textit{-samumuku} & listen\\
\lspbottomrule
\end{tabularx}

\label{table:RDPL-deriv}
\end{table}
%bebeku = ventear -> stative??

In addition to the forms in Tables \ref{table:RDPL} and \ref{table:RDPL-deriv}, where the reduplicated syllable belongs to the verbal root, reduplication is also found on the last syllable of the verb stem in continuous\is{continuous|(} marking. The reduplicated syllable is followed by the extension suffix \textit{-i} and the thematic suffix \is{thematic suffix|(} \textit{-ku}, and the whole sequence \textit{-CViku} stands for continuous aspect. Note that it is not always clear whether we are dealing with reduplication of a root or a stem syllable, since they are identical for all those verb stems that do not end in a thematic suffix. For those verbs, it is not clear whether the form derived by reduplication + \textit{-i} + \textit{-ku} should be analysed as continuous forms or as verbs with a reduplicated stem syllable and the \isi{extension applicative} suffix with the latter triggering the addition of the thematic suffix \textit{-ku}. All “continuous” verbs derived from verbs without a thematic suffix show a high degree of lexicalisation\is{lexicalisation|(}, while in most cases where there is a thematic suffix on the underived verb stem (and thus the continuous marker is \textit{-kuiku}),\is{thematic suffix|)} the continuous form seems to be used for inflectional\is{inflection|(} rather than derivational purposes, e.g. progressive marking, mirroring the use of Spanish gerunds. 

Continuous marking often but not always goes along with \isi{middle voice} (see \sectref{sec:Middle_voice}). It is sometimes also found on non-verbal predicates.\is{non-verbal predication}
\tabref{table:RDPL-cont} shows the most frequent lexicalised continuous forms. An example of the inflectional use of the continuous is given in (\ref{ex:Cont-infl-2}).

\begin{table}
\caption{Related verb stems without and with continuous marking}

\begin{tabularx}{\textwidth}{lllQ}
\lsptoprule
Simple form & Gloss & Derived form & Gloss \\
\midrule
\textit{-chabu} & do & \textit{-chabubuiku} & do\\
\textit{-chemu} & stand up & \textit{-chemumuiku(bu)} & stand\\
\textit{-eiku} & follow, be behind & \textit{-eikukuiku} & chase\\
% &  & \textit{sinunuiku} & look\\
\textit{-tibu} & sit down  & \textit{-tibubuiku(bu)} & sit\\
\textit{-ubu} & be, live & \textit{-ububuiku} & be, live (for a longer time?) \\
\textit{-iyu} & cry & \textit{-iyuyuiku(bu)} & cry; shout, sing (of animals)\\
\lspbottomrule
\end{tabularx}

\label{table:RDPL-cont}
\end{table}
\is{lexicalisation|)}

%kupukuiku
%mukukukubu
%papapuichu = 
%-japapaiku = desramar (ceniza)
%-chepapaiku = recordar = sepapaiku
%-mejepapaiku = decidir

The example comes from Juana and refers to her little parrot, which we were watching.

\ea\label{ex:Cont-infl-2}
\begingl
\glpreamble tinikukuikumÿnÿ tikunipa\\
\gla ti-niku-kuiku-mÿnÿ ti-kunipa\\
\glb 3i-eat-\textsc{cont}-\textsc{dim} 3i-be.hungry\\
\glft ‘it is eating, it is hungry’
\endgl
\trailingcitation{[jxx-e110923l-2.007]}
\xe
\is{continuous|)}
\is{inflection|)}


Another process that optionally but frequently includes reduplication on the edge of a verb stem is concurrent motion marking\is{associated motion} (see \sectref{sec:AMconcurrent}). This can probably be considered a case of automatic reduplication, where the presence of a marker triggers reduplication on another morpheme without reduplication itself adding any meaning \citep[cf.][18]{Rubino2005}. However, as \citet[3, Footnote 1]{GomezVoort2014} rightly remark, “it may be hard to determine that the involved reduplication does not contribute in any way to the meaning of the resulting construction”.

Last but not least, the regressive and repetitive\is{regressive/repetitive} marker has an allomorph with a repeated syllable, but this syllable does not belong to the verb stem (see \sectref{sec:Repetition}). 
\is{reduplication|)}


\subsection{Classifiers}\label{sec:CLF_ActiveVerbs}\is{classifier|(}

Most of the classifiers are more commonly attached to stative than to active verb stems. However, some examples of \isi{transitive} verbs including classifiers are found in my corpus. The classifier then refers to the \isi{object} or to an \isi{oblique}. 

In (\ref{ex:CLF-OBJ}), the classifier \textit{-pe} indicates that the object is a flat thing, a fish in this case. The classifier is also found on the noun \textit{kÿnupe} ‘fish sp.’. The sentence comes from Juana who was thinking about catching delicious fish.\footnote{A larger excerpt of the conversation from which this example is taken is given in the appendix.}

\ea\label{ex:CLF-OBJ}
\begingl
\glpreamble nibÿrupekaini kÿnupe\\
\gla ni-bÿru-pe-ka-ini kÿnupe\\
\glb 1\textsc{sg}-suck-\textsc{clf:}flat-\textsc{th}1\textsc{.irr}-\textsc{frust} fish.sp\\
\glft ‘I would suck the (juice out of the) \textit{cupacá} fish’
\endgl
\trailingcitation{[jrx-c151001fls-9.62]}
\xe

An example of a classifier referring to the goal of an action is given in (\ref{ex:CLF-goal}), where the stem \textit{-etukiku} ‘put on head’ is derived from \textit{-etuku} ‘put’. It was elicited from María S.

\ea\label{ex:CLF-goal}
\begingl
\glpreamble netukikapu nupukene\\
\gla nÿ-etu-ki-ka-pu nÿ-upukene\\
\glb 1\textsc{sg}-put-\textsc{clf:}spherical-\textsc{th}1\textsc{.irr}-\textsc{mid} 1\textsc{sg}-load\\
\glft ‘I put my load on my head’
\endgl
\trailingcitation{[rxx-e181020le]}%el.
\xe

%tetukikubu chichÿtiyae metó su cabeza jxx-a120516la.75
%teiti chija chiratakikutu chichÿti jxx-a120516la.81
%aja pechukikapu = se echó en su cabeza, jxx-p151016l-2.243

%chujimeiku = read from chujiku = speak

%te- tekumuyumutu = estaba muy turbio el agua, mxx-p110825l.170

A peculiarity of the classifiers \textit{-pai} ‘\textsc{clf:}ground' and \textit{-e} ‘\textsc{clf:}water’ in this regard is first that they are found in \isi{transitive} and \isi{intransitive} verbs alike, and second that they only refer to obliques.\is{oblique|(}\footnote{There is one counterexample in the corpus in which \textit{-pai} most probably refers to an object. It comes from Miguel telling the story about the lazy man, who tells his wife:
\ea\label{ex:pai-OBJ}
\begingl
\glpreamble “bueno niyunabÿti nebitakupai”\\
\gla bueno ni-yuna-bÿti nÿ-ebitaku-pai\\
\glb well 1\textsc{sg}-go.\textsc{irr}-\textsc{prsp} 1\textsc{sg}-clear-\textsc{clf:}ground\\
\glft ‘“well, I am going to go to clear the ground (for a field)”’
\endgl
\trailingcitation{[mox-n110920l.020]}
\xe} Regarding their reference to obliques, this is reminiscent of the behaviour of the cognate classifiers in \isi{Baure} and \isi{Mojeño Ignaciano} (\citealp[cf.][156]{Terhart2016}; \citealt[271]{OlzaZubiri2004}), but contrary to \isi{Mojeño Trinitario} \citep[cf.]{Rose2019b,Rose2020}. In the latter language, however, the possibility of the cognate forms referring to objects has possibly changed relatively recently considering that \citet[176, 205]{Gill1957} still describes \textit{-e} as a directional and \textit{-pue} (corresponding to Paunaka \textit{-pai}) as a classifier that refers to obliques only. I had initially opted for an analysis of \textit{-pai} and \textit{-e} as directionals rather than classifiers as well. However, \textit{-pai} can refer to S arguments\is{subject} of \isi{intransitive} verbs (see (\ref{ex:CLF-stat-3}) in \sectref{sec:StativeVerbs_CLF} above) and -- albeit rarely -- it is used to derive nouns, thus it is better analysed as a classifier with the peculiarity of relating primarily to obliques when attached to verbs. As for \textit{-e}, it occurs in the same slot as classifiers and the fact the cognate morpheme of Trinitario\is{Mojeño Trinitario} can nowadays also refer to S arguments of intransitive and O arguments of transitive verbs can be taken as an indication that it is considered as part of the system of classifiers by the speakers rather than as a different category.\footnote{The classifier is actually analysed as an allomorph of \textit{-omo} in Trinitario\is{Mojeño Trinitario} \citep[464]{Rose2019b}, the cognate form of Paunaka’s classifier \textit{-umu} ‘\textsc{clf:}liquid’.} I thus tentatively include \textit{-e} in the list of classifiers (see \sectref{sec:Classifiers}), although it deviates from the other ones. Furthermore, the term “directional” would also be misleading, because it is not only directions (goal, source), but also static locations that \textit{-e} refers to -- the same is true for the classifier \textit{-pai}.\is{oblique|)}

\tabref{table:DIR-pai-deriv} lists the verb forms that have been encountered with the classifier \textit{-pai}; where possible these forms are contrasted to the ones not taking the classifier.

\begin{table}
\caption{Related verb stems without and with the classifier \textit{-pai}}

\begin{tabularx}{\textwidth}{lllQ}
\lsptoprule
Simple form & Gloss & Derived form & Gloss \\
\midrule
\textit{-akachu} & lift, hold, grab & \textit{-akapaiku} & lift from the floor\\
\textit{-bebeiku} & lie & \textit{-be(be)paiku} & lie on the floor\\
 &  & \textit{-bekupaiku} & hang down\\
 &  & \textit{-bÿtupaiku} & fall, fall down\\
\textit{-chemu} & stand up & \textit{-chemupaiku} & stand up from the floor\\
\textit{-etuku} & put & \textit{-etupaiku} & put on the floor\\
\textit{-jatÿku} & pull & \textit{-jatÿpaiku} & pull down\\
\textit{-kubu} & descend, go down & \textit{-kubupaiku} & descend, go down\\
\textit{-seku} & dig hole & \textit{-sekupaiku} & sow with an awl (i.e. dig holes in the ground)\\
& & \textit{-tabipaiku} & appear\\
\textit{-tibubuiku(bu)} & sit & \textit{-tibupaiku} & sit on the floor\\
\textit{-ubu} & be, live & \textit{-ubupaiku apuke} & be born\\
\lspbottomrule
\end{tabularx}

\label{table:DIR-pai-deriv}
\end{table}
%-beriupaiku = return -> beriuku = return
%-chubibipaiku = andar??? -> chubibiku = andar
%-jipaiku = jump (down??) -> jipuku
%-jatÿpaiku = agarrar (abajo??) -> jatÿku


An example of \textit{-pai} is given in (\ref{ex:clf-pai-act-2}), where Juana explains to me how the tool they use for for sowing works:

\ea\label{ex:clf-pai-act-2}
\begingl
\glpreamble tisekupaikukukÿapu bebukatu arusu\\
\gla ti-seku-pai-ku-kukÿa-pu bi-ebuka-tu arusu\\
\glb 3i-dig.hole-\textsc{clf:}ground-\textsc{th}1-\textsc{am.conc.tr.irr}-\textsc{mid} 1\textsc{pl}-sow.\textsc{irr}-\textsc{iam} rice\\
\glft ‘it moves digging holes in the ground, we sow the rice’
\endgl
\trailingcitation{[jxx-p120515l-2.042]}
\xe

\tabref{table:DIR-e-deriv} lists verb stems including \textit{-e}.\footnote{In addition, there is also one stative verb with the classifier: \textit{-ubueji} ‘swim’ derived from the defective verb \textit{-ubu} ‘be, live’. All other verbs that take \textit{-e} are active.}


\begin{table}
\caption{Related verb stems without and with the classifier \textit{-e}}

\begin{tabular}{llll}
\lsptoprule
Simple form & Gloss & Derived form & Gloss \\
\midrule
\textit{-bikÿku} & throw & \textit{-bikÿechu} & throw into water\\
\textit{-bÿtupaikubu} & fall down & \textit{-bÿtuekubu} & fall into water\\
%\textit{-chuku} & pour liquid & \textit{-chukueku}/\textit{-chukuechu} & put into water\\ -> OR: be in center??
\textit{-purtuku} & put into & \textit{-purutueku} & submerge\\
\textit{-tibubuikubu} & sit & \textit{-tibuekubu} & sit in water\\
\lspbottomrule
\end{tabular}

\label{table:DIR-e-deriv}
\end{table}
%jukueku = put into water??
%ubichueku = ??
%jipunueku = jump into water??
%kachukuejiku

(\ref{ex:clf-e-act}) shows the use of such a verb. It comes from Miguel telling the story about the boy and the frog\is{frog story}.

\ea\label{ex:clf-e-act}
\begingl
\glpreamble  titibuekapu ÿne eka peÿ\\
\gla  ti-tibu-e-ka-pu ÿne eka peÿ\\
\glb 3i-sit-\textsc{clf:}water-\textsc{th}1.\textsc{irr}-\textsc{mid} water \textsc{dem}a frog\\
\glft ‘the frog sits in water’
\endgl
\trailingcitation{[mox-a110920l-2.014]}
\xe

\citet[]{Rose2019b} shows that (some) classifiers have the ability to promote obliques to object status in Trinitario.\is{Mojeño Trinitario} Examples like (\ref{ex:clf-e-act}) above, where \textit{ÿne} ‘water’ is used without a locative marker, may suggest that the same could be true for Paunaka. On the other hand, the verb \textit{titibuekapu} in this example bears the middle marker (just like the form without the classifier). Middle verbs are notionally \isi{intransitive} (see \sectref{sec:Middle_voice}), thus \textit{ÿne} cannot be its \isi{object}. Equally in (\ref{ex:CLF-goal}), the object of the derived verb seems to be the very same object of the non-derived verb. I would thus tentatively reject this analysis for the Paunaka case for the time being. Note that when there is a verb with a classifier referring to an \isi{oblique}, NPs co-nominating this oblique occur rarely, so that more data would be necessary to come to a definite conclusion about this point.
\is{classifier|)}
\is{derivation|)}


\subsection{Incorporation}\label{sec:INC_ActiveVerbs}\is{incorporation|(}

Not only classifiers, but also noun stems\is{nominal stem} can form part of active verb stems. This process is commonly known as incorporation. In Paunaka, it is almost exclusively body-part terms that are incorporated with a few exceptions. Two examples of verb stems with incorporated body parts are given below. In total, incorporation plays a minor role in present-day Paunaka. 

In (\ref{ex:beu-INC}), the verb \textit{-bemusuku} ‘skin, take off skin’ is derived from \textit{-beu} ‘take away (or take out/off)’. The example comes from Miguel telling the story about the two men and the devil. The men were successful in hunting, and they decide to stay the night in the woods and prepare some of the meat from the hunt.  


\ea\label{ex:beu-INC}
\begingl
\glpreamble chibemusukunube\\
\gla chi-be-musu-ku-nube\\
\glb 3-take.away-skin-\textsc{th}1-\textsc{pl}\\
\glft ‘they skinned them (the pigs)’
\endgl
\trailingcitation{[mxx-n101017s-1.016]}
\xe

Incorporation of body parts triggers \isi{possessor raising}. This becomes evident in (\ref{ex:INC-PR}), where the possessor of the incorporated body part is expressed as the direct \isi{object} of the verb by a first person plural marker that follows the stem. The example comes from Juana’s account of her encounter with two old ladies at a party in Candelaria long ago. She cites one of the ladies here:

\ea\label{ex:INC-PR}
\begingl
\glpreamble “titikubÿkeubitu isipau”\\
\gla ti-tiku-bÿke-u-bi-tu isipau\\
\glb 3i-drip-face-\textsc{real}-1\textsc{pl}-\textsc{iam} strong.chicha\\
\glft ‘“he dropped strong chicha on our face”’
\endgl
\trailingcitation{[jxx-p120515l-1.073]}
\xe

Incorporation of the body part noun \textit{-bÿke} ‘face’ is peculiar because – unlike in the example above – many verb stems lexicalised\is{lexicalisation} with this noun show semantic extension and often change of verb class, with the resulting verbs being stative (see \sectref{sec:StativeVerbs_INC}).

One exception to body part incorporation is found in the verb stem \textit{-yÿba\-mukeiku} ‘husk’ with the incorporated noun \textit{-muke} ‘seed’. The verb is used to express the husking of rice in a mortar or machine. There is also a verb stem \textit{-yÿbapaku} ‘grind’ with the \isi{classifier} \textit{-pa} for small particles and floury, dusty things, besides two verb stems \textit{-yÿbaiku} ‘grind’ and \textit{-yÿbajiku} ‘grind (with some strength)’ with the applicative\is{extension applicative} and aktionsart suffixes \textit{-i} and \textit{-ji} (see \sectref{sec:EXTApplicative} and \sectref{sec:IntensiveAktionsart}).

The other exception is the relational noun \is{relational noun|(} \textit{-(i)ne} ‘top’ (see also \sectref{sec:Locative}), which is always realised as \textit{-ne} when incorporated into verbs. As far as I know, this is the only one of the relational nouns that can incorporate into verb stems. \tabref{table:DIR-ne-deriv} lists the verb stems including the relational noun that I have found in the corpus in comparison with the bare form. %In one case, that of \textit{uchuneku} ‘put liquid on top, ’, I have not found a corresponding bare form, but one with a classifier yielding \textit{uchukiku} ‘put liquid on head’. -> chuku = put liquid into

\begin{table}
\caption{Related verb stems without and with the relational nouns \textit{-ne}}

\begin{tabularx}{\textwidth}{lllQ}
\lsptoprule
Simple form & Gloss & Derived form & Gloss \\
\midrule
\textit{-chuku} & pour liquid & \textit{-uchuneku} & pour liquid on top (e.g. give water to plants)\\
\textit{-etuku} & put & \textit{-etuneku} & put on back\\
\textit{-kupachu} & step & \textit{-kupanejiku} & step on\\
\textit{-ubu} & be, live & \textit{-ubune(i)ku} & ride, sit on (horse)back\\
\lspbottomrule
\end{tabularx}

\label{table:DIR-ne-deriv}
\end{table}

The relational noun often refers to the back, probably because the back of an animal is considered as its top and this is then also transferred to the human body.\footnote{Note that \citet[464]{Rose2019b} analyses \textit{-ne} as a \isi{classifier} in Trinitario\is{Mojeño Trinitario} with the core member of concepts that the form is used with being ‘back’.} One example is (\ref{ex:relational-verb-inc}), which was elicited from María S.

\ea\label{ex:relational-verb-inc}
\begingl
\glpreamble netunekapubÿti nupukene\\
\gla nÿ-etu-ne-ka-pu-bÿti nÿ-upukene\\
\glb 1\textsc{sg}-put-top-\textsc{th}1\textsc{.irr}-\textsc{mid}-\textsc{prsp} 1\textsc{sg}-load\\
\glft ‘I am just going to put my load on my back’
\endgl
\trailingcitation{[rxx-e181020le]}
\xe

\is{relational noun|)}


\subsection{Combination of derivational processes inside the verb stem}%check, this could also go together with stative verbs??
\label{sec:ActiveVerbs_Combi}
\is{derivation|(}

Some of the processes described in the preceding sections can be combined, others cannot. Classifiers\is{classifier} and incorporated nouns seem to exclude each other. Interestingly, there is one example in which there are two incorporated nouns: the relational \textit{-ne} ‘top’\is{relational noun} and a body part term. This is shown in (\ref{ex:INC-DIR}).

This example comes from the story about the fox and the jaguar as told by María S. With the stone tied to his hands, the jaguar is pushed into the water and drowns.

\ea\label{ex:INC-DIR}
\begingl
\glpreamble chirÿtÿnebuÿchuji mai echÿu kupisairÿ echÿu isini\\
\gla chi-rÿtÿ-ne-buÿ-chu-ji mai echÿu kupisairÿ echÿu isini\\
\glb 3-tie-top-hand-\textsc{th}2-\textsc{rprt} stone \textsc{dem}b fox \textsc{dem}b jaguar\\
\glft ‘the fox tied a stone on top of the jaguar’s hands, it is said’
\endgl
\trailingcitation{[rxx-n120511l-1.037]}
\xe

Reduplication\is{reduplication} is probably compatible with all other processes. In (\ref{ex:RDPL-INC}), reduplication is combined with incorporation, and the reduplicated syllable belongs to the noun stem in this case.

The sentence comes from Juana who speaks about an in-law of hers.

\ea\label{ex:RDPL-INC}
\begingl
\glpreamble chibemususukutuji baka\\
\gla chi-be-musu-su-ku-tu-ji baka\\
\glb 3-take.away-skin-\textsc{rdpl}-\textsc{th}1-\textsc{iam}-\textsc{rprt} cow\\
\glft ‘he had skinned the cow, it is said’
\endgl
\trailingcitation{[jxx-p120430l-2.081]}
\xe
\is{incorporation|)}

The following example shows the combination of \isi{reduplication} with a \isi{classifier}. In this case, a syllable of the root is reduplicated. María C. speaks about herself. She went crawling because she was severely ill, since a sorcerer had put a spell on her.

\ea\label{ex:act-combi-1}
\begingl
\glpreamble nimajijipaikukukÿuni\\
\gla ni-maji-ji-pai-ku-kukÿu-ni\\
\glb 1\textsc{sg}-crawl-\textsc{rdpl}-\textsc{clf:}ground-\textsc{th}1-\textsc{am.conc.tr}-\textsc{deic}\\
\glft ‘I went crawling’
\endgl
\trailingcitation{[ump-p110815sf.306]}
\xe
%susupaiku = carpir

Finally, the last example in this section presents a verb with a reduplicated\is{reduplication} root syllable, a \isi{classifier}, and the applicative suffix\is{extension applicative} \textit{-i}. It comes from Juana and is about her grandparents who had arrived at an arroyo on their journey back home from Moxos. Since heavy rainfalls and the appearance of a water spirit make it impossible to cross, they have to climb up the slope again, grabbing twigs and roots of plants for support.

\ea\label{ex:act-combi-2}
\begingl 
\glpreamble tijatÿtÿkeikukukÿubunubeji yÿkÿke\\
\gla ti-jatÿ-tÿ-ke-i-ku-kukÿu-bu-nube-ji yÿkÿke \\ 
\glb 3i-pull-\textsc{rdpl}-\textsc{clf}:cylindrical-\textsc{ext}-\textsc{th}1-\textsc{am.conc.tr}-\textsc{mid}-\textsc{pl}-\textsc{rprt} stick\\ 
\glft ‘they went pulling themselves up with the help of sticks (i.e. twigs and roots), it is said’\\ 
\endgl
\trailingcitation{[jxx-p151016l-2]}
\xe\is{active verb|)}
%--> this example (in its complete form) is repeated in the AM chapter as <ex:back-up>
\is{derivation|)}

In the following section, some processes occurring at the edge of the verb stem are described. All of them change the verb’s valency.


\section{Adjusting valency}\label{sec:Valency}
\is{valency|(}

Many active verbs\is{active verb|(} are ambitransitive,\is{ambitransitivity} i.e. they can be used intransitively and transitively without any overt change of the verb stem. Only the choice of the third person marker\is{person marking} (see \sectref{sec:3Marking}) possibly sheds light on the valency. There are also some verbs that are stative\is{stative verb} by their stem, but can be used transitively nonetheless. Furthermore, two affixes increase the valency of an active verb: \isi{causative} and \isi{benefactive}. Both of them do not occur very frequently in my corpus.

Valency of active verbs can be decreased by \isi{reciprocal} marking, which expresses that the participants act on each other. Paunaka has no \isi{reflexive} marker proper, but reflexive is one of the functions of the middle marker\is{middle voice} \textit{-bu}, see \sectref{sec:Middle_voice}. 

All processes described in this section are found on the edge of verb stems, i.e. following or replacing the \isi{thematic suffix} (if the verb in question has one, see \sectref{sec:SimplexActiveVerbs} and \sectref{sec:ActiveVerbs_TH}). They arguably derive new stems that are the locus of RS marking.\is{reality status} This sets them apart from middle voice, which is clearly marked outside the verb stem, following the RS marking. Therefore, I decided to describe \isi{causative}, \isi{benefactive}, and \isi{reciprocal} here as derivational processes\is{derivation} and provide an extra section for middle marking.\is{middle voice}

While the \isi{causative} markers precede the verb stem,  \isi{benefactive} and \isi{reciprocal} follow.

\subsection{Ambitransitivity}\label{sec:ambitransitivity}\is{ambitransitivity|(}

Paunaka’s \isi{transitive} verbs are ambitransitive insofar as a non-human\is{animacy} third person \isi{object} does not have to be expressed in the clause, neither by an index, nor by an NP. This may be the case if the object of the verb is accessible so that it does not need to be mentioned overtly. In other cases, a specific \isi{object} is not needed because a general statement is being made.

This is the case in (\ref{ex:ambi-1}), in which María S. states that she has not sown (at all) yet. The reason was that it had not yet rained enough. 

\ea\label{ex:ambi-1}
\begingl
\glpreamble kuinakuÿ nebuka\\
\gla kuina-kuÿ nÿ-ebuka\\
\glb \textsc{neg}-\textsc{incmp} 1\textsc{sg}-sow.\textsc{irr}\\
\glft ‘I haven’t sown yet’
\endgl
\trailingcitation{[rmx-e150922l.023]}
\xe

(\ref{ex:ambi-2}) presents the same verb \textit{ebuku} ‘sow, plant, grow’, this time with an object expressed by an NP. It comes from Miguel telling Swintha what he had done the day before.

\ea\label{ex:ambi-2}
\begingl
\glpreamble nebuku kÿjÿpi\\
\gla n-ebuku kÿjÿpi\\
\glb 1\textsc{sg}-sow manioc\\
\glft ‘I planted manioc’
\endgl
\trailingcitation{[mxx-n101017s-2.017]}
\xe

%kuina nebukabu arusu = ya no siembro arroz, rxx-e181017l

When used intransitively, ambitransitive verbs with third person subjects always take the person prefix \textit{ti-}.\is{person marking} When used transitively, some of them can use either \textit{ti-} or \textit{chÿ-}, sometimes even in the same sentence.\footnote{With human objects, \textit{chÿ-} is obligatory; with non-human objects, both \textit{ti-} and \textit{chÿ-} are possible. Third person objects are not indexed on verbs with first and second person subjects with a few exceptions, see \sectref{sec:3Marking}.} This is illustrated by (\ref{ex:ambi-3}), which builds on the same verb stem \textit{-ebuku} ‘sow, plant, grow’ as the examples above. It was produced in elicitation by María S.

\ea\label{ex:ambi-3}
\begingl
\glpreamble jaja ja amuke tebuku, pero kÿjÿpi kuina chebuka\\
\gla jaja ja amuke ti-ebuku pero kÿjÿpi kuina chÿ-ebuka\\
\glb \textsc{afm} \textsc{afm} corn 3i-sow but manioc \textsc{neg} 3-sow.\textsc{irr}\\
\glft ‘yes, yes, he sowed corn, but he did not plant manioc’
\endgl
\trailingcitation{[rxx-e181024l]}%semi-el.
\xe

There are other ambitransitive verbs that always take \textit{chÿ-} when used transitively (with a third person subject). The verb \textit{-piku} means ‘be afraid’ when used intransitively and ‘fear’ when used transitively. The transitive forms always occur with \textit{chÿ-} in my corpus, even if the object is inanimate as in (\ref{ex:ambi-4}), which was elicited from Juana.

\ea\label{ex:ambi-4}
\begingl
\glpreamble chipiku ÿne\\
\gla chi-piku ÿne\\
\glb 3-be.afraid water\\
\glft ‘it (the dog) is afraid of the water’
\endgl
\trailingcitation{[jxx-a120516l-a.376]}%el.
\xe

A verb stem with very high frequency is \textit{-niku} ‘eat’. Interestingly, the \isi{causative} form of this verb, ‘feed, give food’, has exactly the same form \textit{-niku}.\footnote{It might be possible that the causative stem was once derived with a causative prefix \textit{i-}, which is also found in \isi{Baure} on the corresponding cognate verb stem \citep[cf.][291, Footnote 11]{Danielsen2014a}. At some point in time, all person markers\is{person marking|(} of the Paunaka language developed allomorphs with high front vowels, which in the current stage of the language occur (among other contexts) if the first vowel of the stem is a high front vowel. An exception is the seldom used second person plural subject which is always indexed with \textit{e-}, unless the marker merges with the following vowel of a vowel-initial verb stem.\is{person marking|)} Nonetheless, the hypothesised difference between the form of the non-causative and the causative derivation was not perceivable any longer in all other persons (e.g. in the first person singular between \mbox{\textit{ni-niku}} and \textit{ni-i-niku}). It remains to be checked whether there is still a difference between the forms in the second person plural. I have only found the verb being used with the meaning of ‘eat’ in this context. It should also be noted here that there is no difference in \isi{stress} placement between the non-causative and causative form, so that nothing speaks for an invisible prefix synchronically.} 
Both the non-causative and the \isi{causative} verb can take \textit{ti-} and \textit{chÿ-} in the third person. Thus, in some contexts they are hardly distinguishable – at least for me, although the speakers do not seem to share my confusion. Note that speakers can resort to the use of other verbs to solve this problem, e.g. I have heard \textit{-ekichu} ‘invite (food)’ being used to describe feeding of a baby.\is{active verb|)}\is{ambitransitivity|)}

\subsection{Transitive stative verbs}\label{sec:TransitiveStativeV}\is{stative verb|(}\is{transitive stative verb|(} 

Stative verbs are usually intransitive, but a few of them can nonetheless take an object.\is{object|(} First of all, this can be achieved by adding an NP\is{noun phrase} or a verbal complement to the clause while \isi{person marking} on the verb does not index any object. This is what we find with \textit{-(i)chuna}\is{knowledge/ability predicate|(} ‘be capable, know’. (\ref{ex:TSV-1}) shows the usage of \textit{-(i)chuna} together with an NP that constitutes an object; examples of this verb taking verbal complements can be found in §\ref{sec:CC_KnowledgeAbility}.

The example comes from Juana, who was telling me about her encounter with two old ladies at a party in Candelaria. At first, they did not know that Juana understood their talk in Paunaka, but they finally noticed it when Juana laughed about their comments. The next day, one lady says to Juana:

\ea\label{ex:TSV-1}
\begingl
\glpreamble “bien pichunayenu paunaka"\\
\gla bien pi-ichuna-yenu paunaka\\
\glb well 2\textsc{sg}-be.capable-\textsc{ded} Paunaka\\
\glft ‘“you must know Paunaka well”’
\endgl
\trailingcitation{[jxx-p120515l-1.187]}
\xe
\is{knowledge/ability predicate|)}

The verbs \textit{-kuyeneu} ‘visit’ and \textit{-eju(ju)mi} ‘remember’ can both take person markers to index an object. This is shown in (\ref{ex:TSV-2}) and (\ref{ex:TSV-3}), and another example of \textit{-eju(ju)mi} with object index is (\ref{ex:RDPL-stat-2}) in \sectref{sec:StativeVerbs_RDPL} above.

(\ref{ex:TSV-2}) was elicited from Juana. Actually I had asked for an expression to tell her that I would visit her if she came to Spain. Juana, however, answered with a request directed to me instead.

\ea\label{ex:TSV-2}
\begingl
\glpreamble ¡nabi pakuyeneunÿ nauku!\\
\gla nabi pi-a-kuyeneu-nÿ nauku\\
\glb go.\textsc{imp} 2\textsc{sg}-\textsc{irr}-visit-1\textsc{sg} there\\
\glft ‘go visit me there!’
\endgl
\trailingcitation{[jxx-p110923l-1.268]}
\xe

In (\ref{ex:TSV-3}), María C. tells me to greet Juana when I go and visit her in Santa Cruz.

\ea\label{ex:TSV-3}
\begingl
\glpreamble pikecha kapo chejumiyu echÿu Maria Cuasase Choma pikecha\\
\gla pi-kecha kapo chÿ-ejumi-yu echÿu {Maria Cuasase Choma} pi-kecha\\
\glb 2\textsc{sg}-say.\textsc{irr} ? 3-remember-\textsc{ints} \textsc{dem}b {María Cuasase Choma} 2\textsc{sg}-say.\textsc{irr}\\
\glft ‘tell her that María Cuasase Choma misses her, tell her’
\endgl
\trailingcitation{[uxx-e120427l.072]}
\xe

The most important transitive stative verb in terms of frequency is \textit{-kuye} ‘be like this’. This is a manner demonstrative verb\is{demonstrative verb|(} \citep[cf.][]{Guerin2015} and it has cognates in the \isi{Mojeño languages} (Rose 2021, p.c.). In my corpus, the verb only shows up with the marker \textit{chi-}, i.e. with an index encoding both a third person subject and object (see \sectref{sec:3Marking}); however, I have never tried to elicit forms with a first or second person index, and so for the time being I would not exclude that this is possible.

The demonstrative verb is used anaphorically in most occurrences, i.e. it refers back to a proposition of the preceding discourse, and it is often used to close a discourse topic \citep[cf.][173]{Guerin2015}. \textit{Chikuye} forms a whole clause on its own in most occurrences, as in (\ref{ex:chikuye-0}). If there is an additional verb, this is usually marked as subordinate,\is{deranked verb} see (\ref{ex:chikuye-1}). In (\ref{ex:chikuye-2}), the irrealis form is shown.

(\ref{ex:chikuye-0}) was produced by Miguel to approve something Juan C. had said.

\ea\label{ex:chikuye-0}
\begingl
\glpreamble jaa, chikuye\\
\gla jaa chi-kuye\\
\glb \textsc{afm} 3-be.like.this\\
\glft ‘yes, this is how it is’
\endgl
\trailingcitation{[mqx-p110826l.277]}
\xe

In (\ref{ex:chikuye-1}), the verb is used to provide a summary of what María S. had told me about her childhood before.

\ea\label{ex:chikuye-1}
\begingl
\glpreamble chikuye bijÿkiu\\
\gla chi-kuye bi-jÿk-i-u\\
\glb 3-be.like.this 1\textsc{pl}-grow-\textsc{subord}-\textsc{real}\\
\glft ‘this is how we grew up’
\endgl
\trailingcitation{[rxx-p181101l-2.215]}
\xe

%chikuye bejikiumÿne ÿne, jxx-p120515l-2.067 = así sacamos agua

Finally, (\ref{ex:chikuye-2}) is about the bad behaviour of \textit{karay} towards Juana’s grandparents.

\ea\label{ex:chikuye-2}
\begingl
\glpreamble pero kuina chakuye eka chejumikene eka kayaraunube\\
\gla pero kuina chi-a-kuye eka chÿ-ejumikene eka kayaraunu-nube\\
\glb but \textsc{neg} 3-\textsc{irr}-be.like.this \textsc{dem}a 3-thought \textsc{dem}a karay-\textsc{pl}\\
\glft ‘but the reasoning of \textit{karay} is not like that’
\endgl
\trailingcitation{[jxx-e150925l-1.256]}%non-el.
\xe
\is{object|)}

There is also a \isi{question word} \textit{chikuyena} or \textit{kuyena} derived from the verb, which means ‘how?’ or ‘why?’ (see \sectref{sec:Q_chikuyena}). Sometimes, however, these forms are simply used as a variant of \textit{chikuye}. There are also a few cases in the corpus in which the stem \textit{kuye} shows up without a person marker.\is{demonstrative verb|)}\is{transitive stative verb|)}\is{stative verb|)} 


\subsection{Causative}\label{sec:Causative}\is{active verb|(}\is{causative|(}
\is{derivation|(}

In causative derivations, one participant, the causer, is added to the semantic structure of the verb. The causer is expressed as the \isi{subject} argument of the predicate. The \isi{agent} of the non-causative verb becomes the causee of the causative verb and is expressed as the \isi{object}. I have only found causative derivations of \isi{intransitive} verbs in the corpus, the majority of them being active verbs. 

Causative derivation is not very productive. Nonetheless, two different causative prefixes could be identified, \textit{ku-} and \textit{mi-}. Both of them have cognates in several other \isi{Arawakan languages} \citep[cf.][293]{Aikhenvald2002}.\footnote{The causative and the attributive prefixes\is{attributive prefix} may be related, also in Proto-Arawakan, where both are reconstructed as \textit{*ka-}, see also \sectref{sec:AttributiveVerbs}.} %\citep[cf.][102]{Wise1990}. 
\citet[295]{Danielsen2014a} claimed for \isi{Baure} that the cognate causative prefix \textit{ko-} derives causative states, while verbs derived with the cognate prefix \textit{imo-} express caused actions. I cannot confirm this same distinction for Paunaka. However, my impression is that the prefix \textit{ku-} is predominantly used with telic\is{telicity} verb stems. Causative verbs derived with \textit{mi-} have only been found in Juana’s speech in the corpus. Both derivations are rare so that I do not want to make any absolute claims about them here.

The prefix \textit{ku-} is found on the highly lexicalised\is{lexicalisation} verb \textit{-kupaku} ‘kill’ from \textit{-paku} ‘die’, as in (\ref{ex:ku-caus-2}), which was elicited from José.

\ea\label{ex:ku-caus-2}
\begingl
\glpreamble tikupakubi ue\\
\gla ti-ku-paku-bi ue\\
\glb 3i-\textsc{caus}1-die-1\textsc{pl} water.spirit\\
\glft ‘the water spirit kills us’
\endgl
\trailingcitation{[oxx-m110814sf-2]}%el.
\xe

Another example with the causative prefix \textit{ku-} is given in (\ref{ex:ku-caus-1}), which comes from Juana’s description of one picture of the \isi{frog story} \citep[]{Mayer2003}, where some bees (or wasps) chase the dog and the frightened boy falls from the tree.\footnote{The boy is frightened by an owl that comes out of a hole in the tree, but the speaker interpreted the picture in the way that it was the wasps that made the boy fall.} The causative verb \textit{-kubenupu} usually means ‘fell (trees)’, while \textit{-benupu} means ‘fall down’, but is more often used with the more specific meaning ‘be born’.

\ea\label{ex:ku-caus-1}
\begingl
\glpreamble i naka tikijane jane, aa, chikubenuputu\\
\gla i naka ti-kijane jane aa chi-ku-benupu-tu\\
\glb and here 3i-be.many wasp \textsc{intj} 3-\textsc{caus}1-fall.down-\textsc{iam}\\
\glft ‘and here are a lot of wasps, ah, they have made him fall’
\endgl
\trailingcitation{[jxx-a120516l-a.130]}%semi-el.
\xe

(\ref{ex:ku-caus-3}) is an example with the transitive verb \textit{-kurabajiku} ‘break’, a causative derivation of  \textit{-rabajiku} ‘break (intr.)’. It was elicited from María S. and referred to an imagined broken pot.

\ea\label{ex:ku-caus-3}
\begingl
\glpreamble eka ÿba tikurabajikuchÿ\\
\gla eka ÿba ti-ku-rabajiku-chÿ\\
\glb \textsc{dem}a pig 3i-\textsc{caus}1-break-3\\
\glft ‘the pig broke it’
\endgl
\trailingcitation{[rxx-e181024l.029]}
\xe


Other causative verb stems derived with \textit{ku-} that I found in the corpus include \textit{-kuchepaku} ‘wake up (tr.)’ from \textit{-chepaku} ‘wake up (intr.)’, \mbox{\textit{-kujapÿku}} ‘fill (tr.)’ from \mbox{\textit{-japÿku}} ‘fill (intr.)’ and \textit{-kurÿrÿku} ‘make burn, light fire’ from \textit{-rÿrÿku} ‘burn’. 

The use of the other causative prefix, \textit{mi-}, is illustrated by the following examples. As has been stated above, all of them come from Juana.

The verb in (\ref{ex:mi-caus-1}) builds on the verb stem \textit{-benu} ‘lie down’ (which is related to \mbox{\textit{-benupu}} ‘fall’ in (\ref{ex:ku-caus-1}) above). Juana tells about her sister, who was severely injured and put into the hammock by her son.

\ea\label{ex:mi-caus-1}
\begingl
\glpreamble chimibenu yumaji\\
\gla chi-mi-benu yumaji\\
\glb 3-\textsc{caus}2-lie.down hammock\\
\glft ‘he laid her down in the hammock’
\endgl
\trailingcitation{[jxx-p120430l-2.194]}%non-el.
\xe

Other verbs that take \textit{mi-} are \textit{-mijÿku} ‘raise, bring up’ from \textit{-jÿku} ‘grow’ and \textit{-mikuchu} ‘bathe so.’ from \textit{-kubu} ‘bathe, take a bath’. The latter has a lexicalised\is{lexicalisation|(} middle marker,\is{middle voice|(} which gets detached and replaced by a thematic suffix in causative derivation.\footnote{This is a bit surprising, because the middle marker is the locus of irrealis marking in the non-causative form of the verb, with the irrealis form being \textit{-kuba}. This sets this verb apart from other deponent middle verbs which mark irrealis on the verb stem, and suggests that it has become completely fixed. It may only be due to minimal stem requirements though, because causative derivation suggests that the middle marker\is{middle voice|)} is still transparent. The other possibility is that the causative form is also completely lexicalised.\is{lexicalisation|)}}  % unlike Baure, where the wo of kowo does not get detached in transitive bathing see ex. 8 and 11 on p.176 of Swintha's grammar

%\ea\label{ex:}
%\begingl
%\glpreamble nÿti nimijÿku\\
%\gla nÿti ni-mi-jÿku\\
%\glb 1\textsc{sg.prn} 1\textsc{sg}-\textsc{caus}-grow\\
%\glft ‘I raised him’\\
%\endgl
%\trailingcitation{[jxx-p110923l-1.171]}
%\xe

(\ref{ex:mi-caus-2}) has the active verb \textit{-chuku} ‘pour liquid, empty a container of liquid’ with the \isi{classifier} \textit{-ki} for spherical things (thus resulting in \textit{-chukiku}), which is further causativised with \textit{mi-}. This example was elicited.

\ea\label{ex:mi-caus-2}
\begingl
\glpreamble timichukikanÿ\\
\gla ti-mi-chu-ki-ka-nÿ\\
\glb 3i-\textsc{caus}2-pour.liquid-\textsc{clf:}spherical-\textsc{th}1\textsc{.irr}-1\textsc{sg}\\
\glft ‘she baptised me’ (lit.: ‘she poured liquid on my head’)
\endgl
\trailingcitation{[jxx-e081025s-1.052]}
\xe

The elicited verb in (\ref{ex:attr-caus}) is a case of double derivation: first, an attributive verb\is{attributive prefix} \textit{-kumÿu} ‘wear clothes, put on clothes’ is derived from the noun \textit{-mÿu} ‘clothes’ with the prefix \textit{ku-} (see \sectref{sec:AttributiveVerbs}), and this verb stem is causativised with the prefix \textit{mi-}. The resulting verb is active and carries the thematic suffix \textit{-chu} (see \sectref{sec:ActiveVerbs_TH}).

\ea\label{ex:attr-caus}
\begingl
\glpreamble nimikumÿuchabÿti\\
\gla ni-mi-ku-mÿu-cha-bÿti\\
\glb 1\textsc{sg}-\textsc{caus}2-\textsc{attr}-clothes-\textsc{th}2\textsc{.irr}-\textsc{prsp}\\
\glft ‘I am going to dress him’
\endgl
\trailingcitation{[jxx-e150925l-1.101]}%el.
\xe\is{active verb|)}

Speakers often prefer a periphrastic expression for causative relations.\is{manipulative verb|(} If volition and movement is involved in the action, i.e. cause and effect take place in different locations \citep[cf.][182]{Payne1997}, they use a complement construction including the manipulative verb \textit{-bÿche(i)ku} ‘send, order’, one example of which is given here. It was elicited from Miguel. More examples can be found in \sectref{sec:CC_Manipulative}.

\ea\label{ex:caus-peri-1}
\begingl
\glpreamble nibÿchekabi pisupa\\
\gla ni-bÿcheka-bi pi-isu-pa\\
\glb 1\textsc{sg}-order.\textsc{irr}-2\textsc{sg} 2\textsc{sg}-weed-\textsc{dloc.irr}\\
\glft ‘I will send you to weed’
\endgl
\trailingcitation{[mxx-e160811sd.298]}
\xe
\is{manipulative verb|)}

The other possibility is preferred if no volition is involved. It builds on the instrument and cause preposition\is{instrument/cause} \textit{-keuchi} (see \sectref{sec:adp-keuchi}). Consider (\ref{ex:caus-peri-2}), which is about making something fall, just like (\ref{ex:ku-caus-1}) above. It also comes from a description of the \isi{frog story}, but this sentence was produced by Miguel and referred to another picture, the one on which the dog has made the beehive (or: wasp nest) fall.

\ea\label{ex:caus-peri-2}
\begingl
\glpreamble tibÿtupaikubutu chikeuchi echÿu kabe eka chubiu eka jane\\
\gla ti-bÿtupaikubu-tu chi-keuchi echÿu kabe eka chÿ-ubiu eka jane\\
\glb 3i-fall.down-\textsc{iam} 3-\textsc{ins} \textsc{dem}b dog \textsc{dem}a 3-house \textsc{dem}a wasp\\
\glft ‘the wasp nest fell down because of the dog’
\endgl
\trailingcitation{[mox-a110920l-2.091]}
\xe

A similar example comes from María S. who had just stated that smoking is bad and now provides the reason:

\ea\label{ex:caus-peri-3}
\begingl
\glpreamble chepuine bikutiu chikeuchi bijibÿkia\\
\gla  chepuine bi-kutiu chi-keuchi bi-jibÿk-i-a\\
\glb because 1\textsc{pl}-be.ill 3-\textsc{ins} 1\textsc{pl}-smoke-\textsc{subord}-\textsc{irr}\\
\glft ‘because we get ill by smoking’
\endgl
\trailingcitation{[rxx-e120511l.384]}
\xe
 
%ti- ti- ti- tikutitu chichÿti eka chÿchÿ ee niuma tikutitu nichÿti chikeuchi ÿku, jxx-p151016l-2.239
\is{causative|)}

\subsection{Benefactive}\label{sec:Benefactive}\is{active verb|(}\is{applicative|(}\is{benefactive|(}

The benefactive marker occurs on the edge of the stem of active verbs, following the \isi{thematic suffix} -- if the verb stem in question includes one -- but preceding or fusing with RS marking.\is{reality status} The benefactive marker thus has the realis form \textit{-inu} and the irrealis form \textit{-ina}. It has cognates in the other \isi{Southern Arawakan} languages. In benefactive derivation, added SAP participants are indexed by a person marker following the stem, i.e. as the \isi{object} of the active verb.\footnote{Remember that third person objects are usually not marked by third person markers that follow the stem, see \sectref{sec:NumberPersonVerbs}.} 

(\ref{ex:BEN-4first}) has a second person plural subject and a first person singular object. The sentence comes from Juana who cited her deceased sister here, who was in prison and demanded to see her daughter.

\ea\label{ex:BEN-4first}
\begingl
\glpreamble “¡epuninane nijinepÿimÿnÿ!”\\
\gla e-epun-ina-ne ni-jinepÿi-mÿnÿ\\
\glb 2\textsc{pl}-take-\textsc{ben.irr}-1\textsc{sg} 1\textsc{sg}-daughter-\textsc{dim}\\
\glft ‘“take my daughter to me!”’
\endgl
\trailingcitation{[jxx-p120430l-2.101]}
\xe

In my data, the added participant is always a \isi{recipient}. Verbs of the type ‘do instead of somebody else doing it’ or ‘do because it is good for somebody’ are absent, so that the suffix could be called a recipient applicative more precisely. The marker \textit{-inu} is not found to encode any malefactive relations, i.e. actions that are done to the detriment of the recipient. 
%I decided to gloss this marker as a benefactive nonetheless (instead of e.g. recipient applicative) for comparative reasons: it has cognates in the other Southern Arawakan languages that derive real benefactives. %Terena Vol. 1:112
It is due to the relation with other Arawakan languages that I prefer the term benefactive nonetheless. 

Benefactive verbs with a third person subject always take the third person marker \textit{chÿ-},\is{person marking|(} no matter whether the theme\is{patient/theme} is animate or inanimate and whether a first or second person recipient\is{recipient|(} is indexed. This marker is reserved for 3>3 relationships on \isi{transitive} and \isi{ditransitive} (non-subordinate) verbs (see \sectref{sec:3Marking}). Note that non-derived \isi{ditransitive} verbs take the marker \textit{chÿ-} if they also take a third person object \is{object|(} marker \textit{-chÿ} (see \sectref{sec:3_suffixes}), but not if they have an SAP object. Benefactive verbs do not take the marker \textit{-chÿ} though. We can therefore state that there is a kind of double object marking on benefactive verbs, but they do not behave like other \isi{ditransitive} verbs regarding person marking.\is{person marking|)}

In (\ref{ex:BEN-1}) we find a benefactive derivation of the verb \textit{-upunu} ‘bring’. The recipient is marked by a person index as the object of the verb and the third person marker \textit{chÿ-} is used obligatorily, also with inanimate themes,\is{patient/theme} a photo in this case, which we had brought to María C. Apparently, she had already been waiting for this photo. 

\ea\label{ex:BEN-1}
\begingl
\glpreamble metu nÿmayu chupuninunÿnube\\
\gla metu nÿmayu chÿ-upun-inu-nÿ-nube\\
\glb already just 3-bring-\textsc{ben}-1\textsc{sg}-\textsc{pl}\\
\glft ‘they just finally brought it to me’
\endgl
\trailingcitation{[cux-c120410ls.134]}%non-el.
\xe

Another example with double object marking is (\ref{ex:BEN-2}), which was elicited from Clara.

\ea\label{ex:BEN-2}
\begingl
\glpreamble chiyÿseikinubi\\
\gla chi-yÿseik-inu-bi\\
\glb 3-buy-\textsc{ben}-2\textsc{sg}\\
\glft ‘he bought it for you’
\endgl
\trailingcitation{[cxx-e120410ls-2.006]}%el.
\xe
\is{object|)}

 (\ref{ex:BEN-3}) illustrates a benefactive derivation with a first person subject and a third person recipient object. The latter is not marked on the verb. The theme object\is{patient/theme} is expressed by an NP, the benefactive recipient is not expressed overtly at all, and it is the benefactive verb itself that provides the information that there is a recipient. The example is an excerpt from Juana’s citation of what Miguel’s daughter had told her.

\ea\label{ex:BEN-3}
\begingl
\glpreamble “i netukinu pan”\\
\gla i nÿ-etuk-inu pan\\
\glb and 1\textsc{sg}-put-\textsc{ben} bread\\
\glft ‘“and I served him some bread”’
\endgl
\trailingcitation{[jxx-e150925l-1.129]}%non-el.
\xe
\is{recipient|)}

Only very few verb stems take the benefactive suffix in spontaneous speech, though a few more could be elicited from Clara and Miguel; the latter provided the verb forms in (\ref{ex:BEN-4second}).

\ea\label{ex:BEN-4second}
\begingl
\glpreamble tisipuikinane, tiyÿtikinane\\
\gla ti-sipuik-ina-ne ti-yÿtik-ina-ne\\
\glb 3i-pay-\textsc{ben.irr}-1\textsc{sg} 3i-set.on.fire-\textsc{ben.irr}-1\textsc{sg}\\
\glft ‘he/she will pay for me, he/she will cook for me'
\endgl
\trailingcitation{[mxx-e181024l]}
\xe\is{active verb|)}

Juana would not produce benefactive verbs in elicitation, but rather found other ways of expressing a benefactive relation, as in (\ref{ex:BEN-5}) and (\ref{ex:BEN-6}):

\ea\label{ex:BEN-5}
\begingl
\glpreamble eka nipiji tiyÿseiku, tiyÿseiku nauku, kapunu tipunÿ\\
\gla eka ni-piji ti-yÿseiku ti-yÿseiku nauku kapunu ti-pu-nÿ\\
\glb \textsc{dem}a 1\textsc{sg}-sibling 3i-buy 3i-buy there come 3i-give-1\textsc{sg}\\
\glft ‘my sister bought it, she bought it there, she came and gave it to me’
\endgl
\trailingcitation{[jxx-e191021e-2]}
\xe

\ea\label{ex:BEN-6}
\begingl
\glpreamble metu pisutu nitÿpi\\
\gla metu pi-isu-tu ni-tÿpi\\
\glb already 2\textsc{sg}-weed-\textsc{iam} 1\textsc{sg}-\textsc{obl}\\
\glft ‘you already weeded for me’
\endgl
\trailingcitation{[jxx-e191021e-2]}
\xe

Especially the \isi{general oblique} preposition \textit{(-)tÿpi} of this last example is used a lot to encode benefactive relations,\is{beneficiary} much more than any derived benefactive verb (see \sectref{sec:adp-tÿpi} for more information on the preposition).\is{benefactive|)}\is{applicative|)}


\subsection{Reciprocal}\label{sec:RCPC}
\is{active verb|(}
\is{reciprocal|(}

Paunaka’s reciprocal marker is rarely found in natural discourse, but the speakers have no difficulty in producing reciprocal forms in elicitation. The form of the marker is \mbox{\textit{-kuku}}, irrealis \textit{-kuka}.\is{reality status} It goes back to Proto-Arawakan\is{Arawakan languages} \textit{*-kaka} \citep[cf.][293]{Aikhenvald2002}, and cognates are also found in all other \isi{Southern Arawakan} languages \citep[cf.][428]{deCarvalhoPAU}. The reciprocal marker is either added after the thematic suffix \is{thematic suffix|(} \textit{-ku} or replaces it, depending on the speaker.\footnote{Interestingly, in \isi{Baure} the reciprocal marker is analysed as reduplication of the thematic (or absolute) suffix \textit{-ko} \citep[cf.][296]{Danielsen2014a}. Thus in some rare cases, verb stems with the applicative suffix \textit{-cho} add a single \textit{-ko} in reciprocal derivations (Danielsen 2020, p.c.). In Paunaka, addition of the full form \textit{-kuku} after the thematic suffix instead of replacing it points in the opposite direction: the reciprocal marker is perceived as a fixed unit.}\is{thematic suffix|)} Since the additive marker \textit{-uku}/\textit{-uka} (see \sectref{sec:Additive}) can also directly follow the stem-closing suffix, many verbs have a sequence \textit{kuku} or \textit{kuka} which is not related to the reciprocal marker.

The reciprocal marker can only be attached to \isi{transitive} verb stems. Third person subjects then always take the person marker \textit{ti-} regardless of animacy, and \textit{chÿ-} is not possible (compare \sectref{sec:3Marking}).\is{person marking} This is a clear sign that valency of the verb decreases if the reciprocal marker is added.

Consider the following pair of examples: In (\ref{ex:no-RCPC}), the speaker uses the marker \textit{chÿ-} to indicate that a third person acts on another third person. In (\ref{ex:RCPC-el}), \textit{chÿ-} is not possible. Both examples were elicited from Juana.

\ea\label{ex:no-RCPC}
\begingl
\glpreamble eka kabe chinijabaku takÿra\\
\gla eka kabe chi-nijabaku takÿra\\
\glb \textsc{dem}a dog 3-bite chicken\\
\glft ‘the dog bit the hen’
\endgl
\trailingcitation{[jxx-e110923l-1.117]}
\xe

\ea\label{ex:RCPC-el}
\begingl
\glpreamble tinijabakukuku kabe\\
\gla ti-nijabaku-kuku kabe\\
\glb 3i-bite-\textsc{rcpc} dog\\
\glft ‘the dogs are biting each other’
\endgl
\trailingcitation{[jxx-e081025s-1.571]}
\xe

The reciprocal marker can also be used on verbs with a first person plural marker as in (\ref{ex:RCPC-3}), which was also elicited from Juana.

\ea\label{ex:RCPC-3}
\begingl
\glpreamble bipitanÿkukuka\\
\gla bi-pitanÿku-kuka\\
\glb 1\textsc{pl}-embrace-\textsc{rcpc.irr}\\
\glft ‘we embrace each other’
\endgl
\trailingcitation{[jxx-e150925l-1.114]}
\xe


The following example (\ref{ex:RCPC-2}) with a reciprocal marker was produced spontaneously by María S., who was making fun of an embracing couple by stating that they should rather go fishing.

\ea\label{ex:RCPC-2}
\begingl
\glpreamble bupuna echÿu tipitanÿikukunube tepuikanube\\
\gla bi-upuna echÿu ti-pitanÿi-kuku-nube ti-epuika-nube\\
\glb 1\textsc{pl}-bring.\textsc{irr} \textsc{dem}b 3i-embrace-\textsc{rcpc}-\textsc{pl} 3i-fish.\textsc{irr}-\textsc{pl}\\
\glft ‘let’s bring the ones that embrace each other so that they fish’
\endgl
\trailingcitation{[jrx-c151001fls-9.58]}
\xe

If the verb \textit{-eu} ‘hit, fight’ is combined with the reciprocal marker, the \isi{collective} marker \textit{-ji} is added. This seems to be peculiar to this one verb, since it is not found in other reciprocal derivations. One example is given in (\ref{ex:RCPC-4}), which was elicited from María S.

\ea\label{ex:RCPC-4}
\begingl
\glpreamble teukukujinube chajechubu chÿa\\
\gla ti-eu-kuku-ji-nube chÿ-ajechubu chÿ-a\\
\glb 3i-fight-\textsc{rcpc}-\textsc{col}-\textsc{pl} 3-\textsc{com} 3-father\\
\glft ‘with his father, they were fighting with each other’
\endgl
\trailingcitation{[rxx-e141230s.195]}
\xe

If we compare examples (\ref{ex:RCPC-3}) and (\ref{ex:RCPC-2}) above, we see that the reciprocal marker can occur in two different possible slots: that of the thematic suffix,\is{thematic suffix|(} which usually closes the stem, thus replacing this suffix, and the one following the thematic suffix. There are a few more markers that can occur either in the place of the thematic suffix or in a position following it, namely the associated motion\is{associated motion|(} markers and the \isi{distributive} marker, but their treatment is postponed.\is{thematic suffix|)} Since they do not affect the valency of a verb, if we adopt the view that derivation and inflection can be seen as a continuum \citep[cf.][261]{Croft2000}, then \isi{causative}, \isi{benefactive} and \isi{reciprocal} markers are clearly on the derivational side, but AM markers and especially the \isi{distributive} marker can be placed a little further to the inflection side.\is{associated motion|)}

\is{valency|)}
\is{reciprocal|)}
\is{active verb|)}
\is{derivation|)}
\is{verbal stem|)}

The following two sections deal with the most important inflectional categories of the verb, person (\sectref{sec:NumberPersonVerbs})\footnote{This one actually includes a discussion on the distributive marker as part of number marking.} and reality status (\sectref{sec:RealityStatus}).


%yÿbaiku =mill
%yÿbajiku = mill (yuca, barro)
%yÿbamukeiku = peel rice, husk rice
%yÿbapaku = mill
%nouns: yÿbainu, yÿbapa, yÿbauke
%
%niku
%nijabaku
%nirataku = 1x
%niratajiku =mit Kraft
%nisapiku = sting
%kunipa
%
%adj: michaniki
%
%
%
%AM markers lexicalysed
%upunu
%epunu
%
%
%Verbs in word formation: all processes inside the stem, nominalisation, 
%Verbs as predicates: irrealis, associated motion?, reciprocal?, benefactive?, voice, TAME, cross-referencing
%verbs as arguments: relative clauses
%
%
%bikÿku = throw away
%abikÿku = hold
%
%
%kupachu = pisar
%kupaikechu = matar a todos
%


