%!TEX root = 3-P_Masterdokument.tex
%!TEX encoding = UTF-8 Unicode

\section{Number}\label{sec:NumberNouns}

Plural marking is obligatory with human\is{animacy|(} referents. There is one \isi{plural} marker \textit{-nube}, which is largely restricted to human nouns (see \sectref{sec:NounsPL-nube}). In addition, the \isi{distributive} marker \textit{-jane} can be used to signal non-singularity of non-human referents, usually animate ones.\is{animacy|)} It is described in \sectref{sec:NounPL-jane}. The \isi{collective} marker \textit{-ji} is used with nouns of two different semantic classes: things which are little individuated, since they occur in masses or swarms and kinship terms with the plural marker (used for both plural kin and plural possessors) (see \sectref{sec:Collective}). Although distributives and collectives\isi{collective} are not part of the number system according to \citet[117, 119, 120]{Corbett2000}, they are certainly semantically related, since they also provide information about quantity. This is why they are all subsumed under the heading of “number” here. All three markers are also found on verbs\is{verb} (see \sectref{sec:Verbs_3PL}).

\subsection{The plural marker}\label{sec:NounsPL-nube}
\is{plural|(}

The plural marker \textit{-nube} is obligatory with non-singular human nouns.\is{animacy} An example of such a constellation is given in (\ref{ex:pimiyanube}), where the noun \textit{(a)pimiya} ‘girl, young woman’ takes this marker.
Juana is speaking about the production of traditional clay pots here.

\ea\label{ex:pimiyanube}
\begingl 
\glpreamble i tanÿma kuina tanabunube pimiyanube\\
\gla i tanÿma kuina ti-ana-bu-nube pimiya-nube\\ 
\glb and now \textsc{neg} 3i-make.\textsc{irr}-\textsc{dsc}-\textsc{pl} girl-\textsc{pl}\\ 
\glft ‘and today the young women don’t make them any more’\\ 
\endgl
\trailingcitation{[jxx-p120430l-2.547]}
\xe

(\ref{ex:DEMa-noAGR}) comes from Miguel who was happy that Swintha knew a word he had forgotten because:

\ea\label{ex:DEMa-noAGR}
\begingl 
\glpreamble tiyÿsebÿkeunÿnube eka aitubuchepÿinube naka unekuyae\\
\gla ti-yÿsebÿkeu-nÿ-nube eka aitubuchepÿi-nube naka uneku-yae\\ 
\glb 3i-ask-1\textsc{sg}-\textsc{pl} \textsc{dem}a boy-\textsc{pl} here town-\textsc{loc}\\ 
\glft ‘the boys here in town asked me (about it)’\\ 
\endgl
\trailingcitation{[mdx-c120416ls.121]}
\xe

The noun \textit{aitubuche} ‘boy, young man’ in (\ref{ex:DEMa-noAGR}) is a loan from Bésiro, and the plural marker can also be used with Spanish loans. An example is (\ref{ex:kristi-1}) with a plural-marked version of the noun \textit{kristianu} ‘person’, borrowed from the Spanish noun \textit{cristiano} ‘Christian person’. The sentence comes from the recordings of the 1960s with Juan Ch., who introduced his playing the flute with a few words.

\ea\label{ex:kristi-1}
\begingl 
\glpreamble ¡esamu!, kristianunube\\
\gla e-samu kristianu-nube\\ 
\glb 2\textsc{pl}-hear person-\textsc{pl}\\ 
\glft ‘listen, people!’\\ 
\endgl
\trailingcitation{[nxx-a630101g-2.002]}
\xe

The Spanish word \textit{gente} ‘people’ is borrowed as a countable noun \textit{jente} ‘man’ into Paunaka. (\ref{ex:jentenube}) shows an occurrence of this noun with the plural marker. It comes from Juana telling about the work of the people of Santa Rita in exchange for the construction of their reservoir.

\ea\label{ex:jentenube}
\begingl 
\glpreamble tropanube eka jentenube trabakunube\\
\gla tropa-nube eka jente-nube trabaku-nube\\ 
\glb pack-\textsc{pl} \textsc{dem}a man-\textsc{pl} work-\textsc{pl}\\ 
\glft ‘the men worked in packs’\\ 
\endgl
\trailingcitation{[jxx-p120515l-2.112]}
\xe

Two nouns contain the plural marker as lexicalised\is{lexicalisation|(} part of the stem.\is{nominal stem} One is \textit{mu\-pÿi\-nube} ‘devil’. The noun is composed of the privative\is{privative|(} marker \textit{mu-},\footnote{The privative marker is not productive anymore in Paunaka, but it can be traced back to Proto-Arawakan \textit{*ma-}. Many \isi{Arawakan languages} have (productive) reflexes of this prefix \citep[276]{Michael2014b}.\is{privative|)}} the body-part term \textit{-pÿi} ‘body’ and the plural marker, signifying thus ‘the ones without body’, a term that presumably goes back to pre-Christian belief in spirits. The other noun is \textit{seunube} ‘woman’. I cannot offer any explanation why the plural marker lexicalised with the root\is{nominal root} \textit{*seu}. The plural of \textit{seunube} ‘woman’ is \textit{seunubenube} ‘women’, see (\ref{ex:Plurili-1}). Juana counts the Supepí siblings here.

\ea\label{ex:Plurili-1}
\begingl
\glpreamble trexenubechÿ seunubenube i ruxhnubechÿ jentenube\\
\gla trexe-nube-chÿ seunube-nube i ruxh-nube-chÿ jente-nube\\
\glb three-\textsc{pl}-3 woman-\textsc{pl} and two-\textsc{pl}-3 men-\textsc{pl}\\
\glft ‘the women are three and the men are two’
\endgl
\trailingcitation{[jxx-p120515l-2.239]}
\xe
\is{lexicalisation|)}

There is also one plural-only noun: \textit{sesejinube} ‘children’. The corresponding singular forms are either \textit{sepitÿ} ‘small, child’ or the gender-specific \textit{(a)pimiya} ‘girl, young woman’ and \textit{aitubuche} ‘boy, young man’, see (\ref{ex:pimiyanube}) and (\ref{ex:DEMa-noAGR}) above.

Plural is usually marked on both the noun and the verb if the noun conominates\is{conomination} a subject or object index. There is thus a kind of \isi{agreement} in number between verb and noun, see (\ref{ex:pimiyanube}) and (\ref{ex:DEMa-noAGR}) above.

The plural marker \textit{-nube} does not occur on non-human nouns\is{animacy|(} with few exceptions. First, anthropomorphic characters in narratives can take the plural marker. However, in my data I only found this for verbs (see \sectref{sec:Verbs_3PL}).\footnote{The reason for this is that there is hardly any story in which two or more animals or other anthropomorphic characters would be of the same species or kind, so that there is not much possibility for plural marking on a noun referring to them (like ‘the foxes’). Of course, they could also be denominated by another noun that does not specify the species like ‘the friends’, but this is not the case in the stories I collected. There is one interesting example from a story told by María S. about how the tortoise obtained its carapace, in which there is a mismatch between plural marking on the verb and \isi{distributive} marking on the noun, see (\ref{ex:mismatch}); however, this cannot be generalised.

\ea\label{ex:mismatch}
\begingl
\glpreamble te chisamunubetuji eka ubechajane\\
\gla te chi-samu-nube-tu-ji eka ubecha-jane\\
\glb \textsc{seq} 3-hear-\textsc{pl}-\textsc{iam}-\textsc{rprt} \textsc{dem}a sheep-\textsc{distr}\\
\glft ‘then the sheep heard it, it is said’
\endgl
\trailingcitation{[rxx-n121128s.10]}
\xe}

Second, a few inanimate nouns occasionally take the plural marker. The noun \textit{anyo} ‘year’, a loan\is{borrowing} from Spanish \textit{año}, is such a case, which can be seen in (\ref{ex:anyonube}), where Juana talks about her mother who was ill for a long time.

\ea\label{ex:anyonube}
\begingl 
\glpreamble tibenunukubu yumaji anyonube\\
\gla ti-benunuku-bu yumaji anyo-nube\\ 
\glb 3i-lie-\textsc{mid} hammock year-\textsc{pl}\\ 
\glft ‘she lay in the hammock for years’\\ 
\endgl
\trailingcitation{[jxx-p120430l-2.501]}
\xe

Furthermore, the noun \textit{ubiae} ‘house’ can take the plural marker when reference is to multiple houses, see (\ref{ex:ubiyaenube}). This noun is a special case, though, because it derives\is{derivation} from a verb (\textit{-ubu} ‘be, live’), thus the use of the plural marker may be a relict of subject number marking. Juan C. talks about his village, San Miguelito de la Cruz, in this example.

\ea\label{ex:ubiyaenube}
\begingl 
\glpreamble kakiu nechÿu pario ubiaenube\\
\gla kakiu nechÿu pario ubiae-nube\\ 
\glb exist.\textsc{subord}? \textsc{dem}c some house-\textsc{pl}\\ 
\glft ‘there are some houses’\\ 
\endgl
\trailingcitation{[mqx-p110826l.182]}%kakiu
\xe

In addition, \textit{ubiae} can also take the \isi{distributive} marker \textit{-jane} as in (\ref{ex:ubiyaejane}), where Miguel talks with Alejo and Polonia about the current state of \isi{Altavista}.

\ea\label{ex:ubiyaejane}
\begingl 
\glpreamble i tanÿmatu echÿu ubiaejane kuinabutu\\
\gla i tanÿma-tu echÿu ubiae-jane kuina-bu-tu\\ 
\glb and now-\textsc{iam} \textsc{dem}b house-\textsc{distr} \textsc{neg}-\textsc{dsc}-\textsc{iam}\\ 
\glft ‘and now the houses do not exist anymore’\\ 
\endgl
\trailingcitation{[mty-p110906l.200-201]}
\xe

There are not many occurrences of \textit{ubiae} with the plural marker in my corpus and even less with the \isi{distributive} marker, which is connected to the fact that non-human nouns do not have to be marked for number at all.\is{animacy|)} 

The nominal demonstratives can take the plural marker, when used pronominally, as in (\ref{ex:Plurili-2}), which was elicited from Miguel. If they modify the noun, there is usually no plural marking on the demonstratives, see also \sectref{sec:NP}.

\ea\label{ex:Plurili-2}
\begingl
\glpreamble echÿunube tichujijikubunube\\
\gla echÿu-nube ti-chujijiku-bu-nube\\
\glb \textsc{dem}b-\textsc{pl} 3i-talk-\textsc{mid}-\textsc{pl}\\
\glft ‘they are chatting’
\endgl
\trailingcitation{[mrx-e150219s.011]}
\xe
\is{plural|)}

\subsection{The distributive marker}\label{sec:NounPL-jane}\is{distributive|(}

The plural marker cannot be used with non-human nouns,\is{animacy} but there is another marker, \textit{-jane}, used mainly to express plurality of animals, as in (\ref{ex:kabejane}). It was produced by Miguel, but in the story he was telling, it is uttered by the jaguarundi, who warns his companion, the drunken fox, to stop singing lest he calls the attention of the dogs. While the fox and the jaguarundi are anthropomorphic characters and thus subject of plural marking, the dogs are not; they behave like dogs and they do not speak but bark. Use of a distributive form makes clear that there are several dogs that could harm them, thus marking the situation extremely dangerous.

\ea\label{ex:kabejane}
\begingl 
\glpreamble “¡tch xhhh, kaku kabejane naka, kaku kabejane naka!”\\
\gla tch xhhh kaku kabe-jane naka kaku kabe-jane naka\\ 
\glb \textsc{intj} \textsc{intj} exist dog-\textsc{distr} here exist dog-\textsc{distr} here\\ 
\glft ‘“shh, shhh, there are dogs around here, there are dogs around here!”’\\ 
\endgl
\trailingcitation{[jmx-n120429ls-x5.381]}
\xe

The marker is called “distributive marker” in this grammar, although this term might be a bit misleading. According to \citet[112]{Corbett2000}, the primary function of distributive marking on nouns is to “spread (distribute) various entities over various locations or over various sorts (types)”. In current Paunaka, the function of the distributive is rather to express \textit{overtly} that there are various non-humans tokens,\is{animacy|(} since number of non-human entities does not have to be specified at all. I had priorly just glossed the marker as a non-human \isi{plural} until I noticed that there are a few cases in which \textit{-jane} occurs together with \textit{-nube}. In these cases, there would be a semantic mismatch if \textit{-jane} was analysed as a non-human plural marker.

I have found three such cases. First of all, there is a \isi{question word} \textit{kajane} ‘how many’ and a quantifying stative verb \textit{-kijane} ‘be many’, where the marker is a lexicalised\is{lexicalisation} part of the stem.\footnote{The root \textit{ka} of \textit{kajane} is probably a \isi{demonstrative} element to which \textit{-jane} is added, see \sectref{sec:DemPron}. The composition of \textit{-kijane} is opaque, the element \textit{*ki} could not be identified.} Both add the plural marker when they refer to quantities of humans. In addition, \textit{-jane} also shows up in the plural form \textit{pujane(nube)} ‘others’ of the singular form \textit{punachÿ} ‘other’. Distributives encode distinctiveness or individuation of referents \citep[116]{Corbett2000}, each member of a group is perceived individually in contrast to perceiving plurality as a unit. I suppose this may have once been the primary function of \textit{-jane}, and this is still well visible in the \isi{question word} \textit{kajane} ‘how many’. Asking for a number presupposes that each member of a group is counted individually. Nonetheless, the primary function of the distributive marker in current-day Paunaka is plural-marking of non-human referents.\is{plural} It is never attached to human nouns nor to verbs in reference to humans. Among the possible non-human referents, it is more commonly found with animate than with inanimate nouns and bigger, more individuated animals, like dogs, cows, and to a lesser extent pigs, are more likely to be marked by the distributive than smaller and less individuated animals like chicken and fish.\is{animacy|)}

In (\ref{ex:bakajane}), the distributive marker attaches to \textit{baka} ‘cow’. The example comes from Juana who was telling me about the journey of her grandparents back home from Moxos. They had bought cows there. It is a long way from Moxos to the Chiquitania, which the grandparents went by foot. They slept in huts or temporary shelters and let the cows in enclosures they found along the way.

\ea\label{ex:bakajane}
\begingl 
\glpreamble kaku eka bakayayae eka bakajane\\
\gla kaku eka bakaya-yae eka baka-jane\\ 
\glb exist \textsc{dem}a enclosure-\textsc{loc} \textsc{dem}a cow-\textsc{distr}\\ 
\glft ‘the cows were in the enclosure’\\ 
\endgl
\trailingcitation{[jxx-p151016l-2.030]}
\xe

Contrary to the plural marker, \textit{-jane} usually occurs only once in a clause, either on the predicate or on the NP conominating\is{conomination} subject or object, with some exceptions. Which factors determine the choice of either predicate or NP\is{noun phrase} taking the distributive marker remains to be investigated.\footnote{It seems to be the case that distributive marking on the \isi{copula} \textit{kaku} ‘exist’ is generally avoided though not impossible.} Thus, there is usually no \isi{agreement} in \textit{-jane} between the NP and the predicate, although a few counterexamples exist. All examples in this section show the use on the NP.

%In most of the cases in which the distributive is used to signal non-singularity of animals, the NP is unmarked and \textit{-jane} occurs on the verb. However, sometimes \textit{-jane} is marked on the noun, but not on the verb as in (\ref{ex:no-agr-jane-3}).

 (\ref{ex:no-agr-jane-3}) is another example with dogs, it comes from Juana who was telling me about her own dogs.

\ea\label{ex:no-agr-jane-3}
\begingl 
\glpreamble tichaneikune eka kabejane\\
\gla ti-chaneiku-ne eka kabe-jane\\ 
\glb 3i-care.for-1\textsc{sg} \textsc{dem} dog-\textsc{distr}\\ 
\glft ‘the dogs protect me’\\ 
\endgl
\trailingcitation{[jxx-e150925l-1.093]}
\xe

(\ref{ex:distri-3}) is an example of the distributive marker on the word for ‘pig’ and was elicited from Miguel.

\ea\label{ex:distri-3}
\begingl
\glpreamble tibÿjaneupuku ÿbajane\\
\gla ti-bÿ-jane-u-pu-uku ÿba-jane\\
\glb 3i-go.in-\textsc{distr}-\textsc{real}-\textsc{dloc}-\textsc{add} pig-\textsc{distr}\\
\glft ‘the pigs also go inside’
\endgl
\trailingcitation{[mrx-e150219s.102]}
\xe

In (\ref{ex:distri-1}), there are two inanimate\is{animacy} nouns with plural referents, the ‘stones’ and the ‘adobe bricks’; however, only the first one takes the distributive marker. The example comes from Miguel who told me about the construction of the school building in Santa Rita a long time ago.

\ea\label{ex:distri-1}
\begingl
\glpreamble entonses bisemaikutu echÿu maijane banautu echÿu arubi\\
\gla entonses bi-semaiku-tu echÿu mai-jane bi-anau-tu echÿu arubi\\
\glb thus 1\textsc{pl}-search-\textsc{iam} \textsc{dem}b stone-\textsc{distr} 1\textsc{pl}-make-\textsc{iam} \textsc{dem}b adobe\\
\glft ‘thus we looked for stones, we made adobe bricks’
\endgl
\trailingcitation{[mxx-p110825l.114]}
\xe

An example from Juana with two inanimate\is{animacy} distributive-marked nouns is (\ref{ex:distri-2}), in which the flower boxes they have in Cotoca, a small city famous for its ceramics, are compared to jars.

\ea\label{ex:distri-2}
\begingl
\glpreamble kaku echÿu maseterojane nena yÿpijanemÿnÿ\\
\gla kaku echÿu masetero-jane nena yÿpi-jane-mÿnÿ\\
\glb exist \textsc{dem}b flower.box-\textsc{distr} like jar-\textsc{distr}-\textsc{dim}\\
\glft ‘there are flower boxes that look like jars'
\endgl
\trailingcitation{[jxx-p120430l-2.616]}
\xe
\is{distributive|)}

\subsection{The collective marker}\label{sec:Collective}\is{collective|(}
The marker \textit{-ji}, which can best be interpreted as a collective marker, since it is found on a number of nouns that occur in uncountable collections or groups like \textit{-mukiji} ‘hair’ and  \textit{mÿiji} ‘grass’. It is used with certain plant parts perceived as a collection rather than as countable objects like \textit{yÿkÿkekeji} ‘branches, twigs’,\footnote{As noted in \sectref{sec:RDPL_Nouns}, the collective marker sometimes causes repetition of a preceding syllable for an unknown reason.} \textit{chipuneji} ‘leaves’ and \textit{chÿiji} ‘fruits’, and on names of small fish species that occur in swarms like \textit{kÿnupeji} ‘fish sp.’ and \textit{turukeji} ‘fish sp.’ (these fish are called \textit{cupacá} and \textit{tayoca} in local Spanish) as in (\ref{ex:turukeji}), where Clara describes the fish, which are small but fat.

\ea\label{ex:turukeji}
\begingl 
\glpreamble tisabananaji echÿu turukeji\\
\gla ti-sabana-na-ji echÿu turuke-ji\\ 
\glb 3i-be.fat-\textsc{rep}-\textsc{col} \textsc{dem}b fish.sp-\textsc{col}\\ 
\glft ‘the \textit{tayoca} fish are fat’\\ 
\endgl
\trailingcitation{[cux-c120414ls-2.152]}
\xe

(\ref{ex:colli-1}) is a statement by Juana about her mother’s hair.

\ea\label{ex:colli-1}
\begingl
\glpreamble michana chimukijimÿnÿ nÿenubane\\
\gla michana chi-muki-ji-mÿnÿ nÿ-enu-bane\\
\glb nice 3-hair-\textsc{col}-\textsc{dim} 1\textsc{sg}-mother-\textsc{rem}\\
\glft ‘my late mother had beautiful hair’
\endgl
\trailingcitation{[jxx-d181102l.47]}
\xe


It is not always easy to distinguish the collective marker from one of homo\-nyms, especially the classifier\is{classifier|(} for soft masses (e.g. dough, mud), see \sectref{sec:Classifiers}. Both theoretically occur in different slots, see \figref{fig:NounTemplate} above,\footnote{This is the case e.g. in collective marked \textit{kÿnupeji}; the fish name is \textit{kÿnupe} in Paunaka %(\textit{Bujurquina oenolaemus}??
 with the classifier \textit{-pe} for flat things.} but since the nouns derived with the classifier denote (soft) masses, collective marking is not applicable to them. \textit{Muteji} ‘loam, mud’ is certainly a soft mass and \textit{-mukiji} a collection of ‘hair’, but what about \textit{-mÿuji} ‘clothes’ – is this a soft mass or a non-countable collection of individual pieces of garment? There may be a substantial semantic overlap in some cases.\footnote{Note also that \isi{Baure} has a similar marker \textit{-je} which was glossed ‘distributive’ by \citet[155--156]{Danielsen2007}, but has a collective function as well, resembling the Paunaka one (Danielsen 2021, p.c.). For the \isi{Mojeño languages}, on the other hand, the form \textit{-ji} was analysed as a classifier for amorphous items, applied among other things to “grass, leaves, small branches” \citep[17]{Rose2020}, i.e. items which I have analysed as including the collective marker. Nonetheless, Rose (2021, p.c.) confirms that a collective marker \textit{-ji} also exists in Trinitario\is{Mojeño Trinitario} and that some of the nouns priorly analysed to be built on the classifier may actually rather include the collective marker. The problem of distinguishing both markers remains for both languages, Paunaka and Trinitario.\is{Mojeño Trinitario}}\is{classifier|)}

The collective marker also shows up on kinship terms if the possessed kin or the possessor\is{possessor|(} is plural (or both).\is{plural|(} The first scenario is a case of regular plural marking on human nouns\is{animacy} (see \sectref{sec:NounsPL-nube} above), the other one relates to regular possessor marking, where addition of the plural marker to a noun bearing the third person marker\is{person marking} compensates for the non-existence of a specific third person plural person marker (see \sectref{sec:Possession}).

My hypothesis is that the collective was once used in addition to the plural marker if the possessed kin was the plural referent, while no collective marker showed up, when only the possessor was plural. There are two examples in my corpus that hint at this. In those examples, there is no collective marker, and in both, plural reference is to the possessor while the possessed kin is singular. The first one, (\ref{ex:Kin-no-ji-1}), was produced by Miguel in elicitation, the second one, (\ref{ex:Kin-no-ji-2}), occurred in spontaneous speech of María C.

\ea\label{ex:Kin-no-ji-1}
\begingl 
\glpreamble nÿti chÿenunube\\
\gla nÿti chÿ-enu-nube\\ 
\glb 1\textsc{sg.prn} 3-mother-\textsc{pl}\\ 
\glft ‘I am their mother’\\ 
\endgl
\trailingcitation{[mxx-e090728s-3.081]}
\xe

\ea\label{ex:Kin-no-ji-2}
\begingl 
\glpreamble chibu chÿanube\\
\gla chibu chÿ-a-nube\\ 
\glb 3\textsc{top.prn} 3-father-\textsc{pl}\\ 
\glft ‘he is their father’\\ 
\endgl
\trailingcitation{[cux-c120414ls-1.114]}
\xe

In most cases, however, the collective marker also occurs if there is a singular possessed kin and a plural possessor. (\ref{ex:Kin-ji-PL-1}) shows this. Like (\ref{ex:Kin-no-ji-2}), it was also produced by María C. in spontaneous speech and expresses exactly the same constellation: a plural possessor with a singular possessed kin (though with a different kinship term). Nonetheless, the collective marker is used together with the plural marker in this case.

\ea\label{ex:Kin-ji-PL-1}
\begingl 
\glpreamble chÿenujinube ekanube\\
\gla chÿ-enu-ji-nube eka-nube\\ 
\glb 3-mother-\textsc{col}-\textsc{pl} \textsc{dem}a-\textsc{pl}\\ 
\glft ‘the mother of them’\\ 
\endgl
\trailingcitation{[cux-c120410ls.124]}
\xe

In an elicitation session with María S. about this topic (rxx-e151021l-1), she explicitly confirmed that the collective marker is used with both third person plural possessors and plural possessed kins. Even more so, omission of the marker leads to ungrammatical forms according to her. This was verified by the phrases she produced in the elicitation with some playmobil toys that represented mothers and daughters in different constellations.

(\ref{ex:Kin-ji-PL-2}) is another example with a third person plural possessor and a singular possessed kin. Both the plural and the collective marker are used again. It is a description by Juana of a photo on which my husband cradles both our daughters in his arms. 

\ea\label{ex:Kin-ji-PL-2}
\begingl 
\glpreamble i eka tanÿma chumu chÿajinube, chakachunube chÿajinube\\
\gla i eka tanÿma chÿ-umu chÿ-a-ji-nube ch-akachu-nube chÿ-a-ji-nube\\ 
\glb and \textsc{dem}a now 3-take 3-father-\textsc{col}-\textsc{pl} 3-lift-\textsc{pl} 3-father-\textsc{col}-\textsc{pl}\\ 
\glft ‘and (on) this one, now their father takes them, he lifts them’\\ 
\endgl
\trailingcitation{[jxx-p141024s-1.26]}
\xe

The following examples have singular possessors and plural possessed kin. The plural marker thus relates to the possessed kin in these cases. (\ref{ex:POSS-ji-PL-1}) has a third person singular possessor, whereas in (\ref{ex:POSS-ji-PL-2}) the possessor is first person singular.

In (\ref{ex:POSS-ji-PL-1}), \textit{chipijijinube} ‘his brothers’ is the possessed noun with the singular possessor and plural possessed. The example comes from Juana telling me about the life of her sister.

\ea\label{ex:POSS-ji-PL-1}
\begingl
\glpreamble tikutijikuji chima, chajechubu chipijijinube tikutijikunubeji kimenukÿ \\
\gla ti-kutijiku-ji chi-ima, chÿ-ajechubu chi-piji-ji-nube ti-kutijiku-nube-ji kimenu-kÿ \\
\glb 3i-flee-\textsc{rprt} 3-husband 3-\textsc{com} 3-sibling-\textsc{col}-\textsc{pl} 3i-flee-\textsc{pl}-\textsc{rprt} woods-\textsc{clf:}bounded\\
\glft  ‘her husband fled, together with his brothers he fled to the woods, it is said’
\endgl
\trailingcitation{[jxx-p120430l-2.086-087]}
\xe

(\ref{ex:POSS-ji-PL-2}) is from María C.

\ea\label{ex:POSS-ji-PL-2}
\begingl 
\glpreamble nichechajinube kakunube uneku\\
\gla ni-checha-ji-nube kaku-nube uneku\\ 
\glb 1\textsc{sg}-son-\textsc{col}-\textsc{pl} exist-\textsc{pl} town\\ 
\glft ‘my children live in town’\\ 
\endgl
\trailingcitation{[uxx-p110825l.075]}
\xe
\is{plural|)}
\is{possessor|)}


Plural kinship terms do not take the collective marker if they include the noun (or suffix) \textit{-pÿi}, which literally means ‘body’, but is rather used to express \isi{endearment} (see \sectref{sec:Compounding}). The nouns \textit{-jinepÿi} ‘daughter’ and \textit{-sinepÿi} ‘grandchild’ are lexicalised\is{lexicalisation} with \textit{-pÿi}, and since it is not detachable, these nouns cannot take the collective marker when pluralised. An example is (\ref{ex:daughter-PL}), where Clara speaks about her plans to spread the use of Paunaka.

\ea\label{ex:daughter-PL}
\begingl 
\glpreamble nisachu nimeisumeikanube nijinepÿinube\\
\gla ni-sachu ni-meisumeika-nube ni-jinepÿi-nube\\ 
\glb 1\textsc{sg}-want 1\textsc{sg}-teach.\textsc{irr}-\textsc{pl} 1\textsc{sg}-daughter-\textsc{pl}\\ 
\glft ‘I want to teach it to my daughters’\\ 
\endgl
\trailingcitation{[cux-c120414ls-2.323]}
\xe

Other kinship terms can also take \textit{-pÿi}, but usually occur without it, when pluralised. Thus, in the singular we mostly find \textit{-chechapÿi} ‘son, child’, but in the plural, it is usually only \textit{-checha}. An example was already given in (\ref{ex:POSS-ji-PL-2}) above. There are a few exceptions though, where \textit{-chechapÿi} is pluralised without the collective marker, as in (\ref{ex:chechapÿi}), which comes from a story told by Miguel about a lazy man. His wife is angry with him, when she finds out that he did not do the work he was supposed to do, so she refuses to give him food.

\ea\label{ex:chechapÿi}
\begingl 
\glpreamble chÿnikujikutu chichechapÿinube\\
\gla chÿ-niku-jiku-tu chi-chechapÿi-nube\\ 
\glb 3-feed-\textsc{lim}1-\textsc{iam} 3-son-\textsc{pl}\\ 
\glft ‘she only gave food to her children’\\ 
\endgl
\trailingcitation{[mox-n110920l.081]}
\xe

Strikingly, there is also one utterance in my corpus, given here as (\ref{ex:jinejinube}), in which detachment of \textit{-pÿi} in the \isi{plural} also applies to \textit{-jinepÿi} ‘daughter’, although \textit{*-jine} does not exist as an independent noun stem in Paunaka, contrary to \textit{-checha} ‘son, child, egg, offspring’. This shows that the underlying process of alternating \textit{-pÿi} and \textit{-ji} is transparent for the speakers. The sentence comes from María C. and is about the supposed incapability of Clara’s daughters to learn Paunaka.

\ea\label{ex:jinejinube}
\begingl 
\glpreamble kaku pijinejinube pero kuina puero chitanube\\
\gla kaku pi-jine-ji-nube pero kuina puero chi-ita-nube\\ 
\glb exist 2\textsc{sg}-daughter-\textsc{col}-\textsc{pl} but \textsc{neg} can 3-master.\textsc{irr}-\textsc{pl}\\ 
\glft ‘you have daughters, but they can’t figure it out’\\ 
\endgl
\trailingcitation{[cux-c120414ls-2.265]}
\xe

Apart from kinship terminology, the collective marker also forms part of the plural-only word \textit{sesejinube} ‘children’.
\is{collective|)}

The following sections are dedicated to other kinds of inflectional morphology, nominal irrealis and deceased. Both of these categories provide information about the existence of an entity at reference or utterance time.

