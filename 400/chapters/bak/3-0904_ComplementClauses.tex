%!TEX root = 3-P_Masterdokument.tex
%!TEX encoding = UTF-8 Unicode


\section{Complement relations}\label{sec:ComplementClauses}\is{complement relation|(}

From a semantic-functional view, in a complement relation one event entails that there is reference to another one \citep[95]{Cristofaro2003}. If complementation is defined as a syntactic relation, it is the “syntactic situation that arises when a notional sentence or predication is an argument of a predicate” \citep[52]{Noonan2007}.

The question arises whether the complement clause (CC) can be analysed as an argument \is{argument|(} of a predicate in Paunaka. The most typical instances of CCs are not marked for dependency. They cannot be indexed on the matrix verb nor be substituted by a pronoun. 

One example is given in (\ref{ex:CC-1}). Both the complement-taking verb and the complement verb are fully inflected for person and RS; they could occur independently in exactly this way.

In this sentence from the story about the two men and the devil, Miguel quotes what the devil says.

\ea\label{ex:CC-1}
\begingl
\glpreamble “nisachu \textup{[}nÿnika\textup{]}”\\
\gla ni-sachu nÿ-nika\\
\glb 1\textsc{sg}-want 1\textsc{sg}-eat.\textsc{irr}\\
\glft ‘“I want to eat”’
\endgl
\trailingcitation{[mxx-n101017s-1.035]}
\xe

If the clause is not an argument of the matrix verb, there is no complement clause at all according to the definition by \citet{Dixon2006}. Instead of this, he speaks of “complementation strategies”. For reasons of convenience, I will nonetheless continue to speak of complement clauses and use the abbreviation CC in reference to the complement predicate and its arguments and modifiers.\is{argument|)}

One complementation strategy described by \citet[34--35]{Dixon2006} and \citet[87--92]{Noonan2007} is the serial verb construction\is{serial verb construction|(} (SVC). As has been stated in \sectref{sec:SVC_and_MCPC}, a crucial feature in the definition of SVCs is that they are monoclausal. A test for monoclausality is negation\is{negation|(} and, indeed, a CC cannot be treated as a syntactically independent clause, because it lacks independent negatability. The matrix clause and the CC can only be negated together. The negative particle is placed before the complement-taking verb and has scope over both predicates, which can be seen in (\ref{ex:want-eat}). There is no way to negate the CC alone. Complementation is thus achieved by an asyndetic integrating\is{syndesis/asyndesis} strategy in Paunaka (see \sectref{sec:AsyndeticJuxtaposition}).

In (\ref{ex:want-eat}), Juana reports what she said to her daughter. She did not want to eat for being sad about the death of her sister.

\ea\label{ex:want-eat}
\begingl
\glpreamble kuina nisacha \textup{[}ninika\textup{]}\\
\gla kuina ni-sacha ni-nika\\
\glb \textsc{neg} 1\textsc{sg}-want.\textsc{irr} 1\textsc{sg}-eat.\textsc{irr}\\
\glft ‘I don’t want to eat’
\endgl
\trailingcitation{[jxx-p120430l-2.239]}
\xe
\is{negation|)}

However, two other features claimed to be decisive for SVCs in complementation contexts do not hold for Paunaka. First of all, it is not necessary in all types of complementation that the two verbs involved have the same \isi{subject} unlike what has been suggested by \citet[34]{Dixon2006}. Consider (\ref{ex:new-want-take}), in which Juana speaks about her daughter who lived in Argentina by that time.

\ea\label{ex:new-want-take}
\begingl
\glpreamble nisachu tumane Buenos Aires\\
\gla ni-sachu ti-uma-ne {Buenos Aires}\\
\glb 1\textsc{sg}-want 3i-take.\textsc{irr}-1\textsc{sg} {Buenos Aires}\\
\glft ‘I want her to take me to Buenos Aires’
\endgl
\trailingcitation{[jxx-e120516l-1.023]}
\xe

Second, in contrast to what has been proposed by \citet[3]{Aikhenvald2018} for SVCs in general, the CC can be left unexpressed if the reference is sufficiently clear, e.g. in an answer to a question\is{answer to question|(} as in the question-answer pair in (\ref{ex:want-photo}). This is an example where Swintha had a question to Miguel that was translated to Paunaka by Juana on her request. It is about a photo she had taken of him.

\ea\label{ex:want-photo}
  \ea
\begingl
\glpreamble ¿pisachu tipunakapi echÿu chÿbutune?\\
\gla pi-sachu ti-punaka-pi echÿu chÿ-butu-ne\\
\glb 2\textsc{sg}-want 3i-give.\textsc{irr}-2\textsc{sg} \textsc{dem}b 3-photo-\textsc{possd}\\
\glft ‘do you want her to give you her photo?’
\endgl
  \ex
\begingl
\glpreamble ja, nÿsachu\\
\gla ja nÿ-sachu\\
\glb \textsc{afm} 1\textsc{sg}-want\\
\glft ‘yes, I want’
\endgl
\trailingcitation{[jmx-e090727s.039-040]}
\z
\xe
\is{answer to question|)}
\is{serial verb construction|)}


%also: nanau chepuine timesumeikunebane nÿenu nechikue nichuna, rxx-181022le

Besides serialisation, \citet[87]{Noonan2007} mentions another possible construction type for CCs that are not arguments, i.e. the paratactic construction, in which the CC is syntactically independent from the MC. However, this construction is ruled out, because, as has been shown above, the predicates are not independently negatable. Thus the question how to syntactically classify the CC in Paunaka remains open for the time being.

Up to this point, only sentences with the most frequent complement-taking verb \textit{-sachu} ‘want’ have been considered. Paunaka has a small number of verbs that take unmarked clausal complements like the ones shown in (\ref{ex:CC-1})–(\ref{ex:new-want-take}). Among those verbs are some secondary verbs, i.e. verbs that only take clausal complements, and some primary verbs, those that can also take objects\is{object} expressed by a person index on the verb and/or an NP \citep[cf.][9]{Dixon2006}. Some complement-taking verbs with different subject CCs index the shared argument as an object, while others do not. Although dependency is not overtly marked, the fact that the \isi{complement verb} is dependent on the matrix verb can be deduced from its restrictions on RS marking:\is{reality status|(} Some complement-taking verbs only take same-RS complements, others only irrealis complements. Only reported speech CCs stick out here, since their RS is not predetermined.\is{reality status|)} 

In some special cases, verbs that usually take unmarked CCs can also take CCs with a \isi{deranked verb}. This is probably bound to some extraordinary circumstances that are not inherent in the relation between the complement-taking verb and its CC. For some verbs, there seems to be a certain variability though. They can take both balanced\is{finite verb} and deranked verbs in their CCs. This may be bound to the generally sparse occurrence in discourse of these verbs. There are also some verbs that, if they occur with a clausal complement at all, only allow a CC with a deranked verb. Finally, I have also found a few cases in which a CC with a balanced verb is introduced by a demonstrative.

The remainder of this section is organised as follows. I first show some more examples of unmarked CCs in \sectref{sec:Unmarked_CCs}. \sectref{sec:CCs_deranked} is dedicated to the discussion of cases in which CCs with deranked verbs show up. Finally, \sectref{sec:CC_CMPL} takes a look at CCs that include the demonstrative \textit{eka} as a complementiser.

Throughout this section, CCs are given in square brackets. If an argument that is shared by the complement-taking predicate and the complement predicate is conominated, it occurs outside the square brackets, i.e. as if belonging primarily to the main clause. This is motivated by a similar analysis of English CCs in which the subject of the CC is encoded as an object of the complement-taking verb (e.g. \textit{I hear \underline{him} \textup{[}singing\textup{]}.}) For Paunaka, however, this is an arbitrary decision, since there is nothing on the NP that would point towards it belonging to either of the two clauses.


\subsection{Unmarked complement clauses}\label{sec:Unmarked_CCs}
\is{finite verb|(}

The predicates that generally take unmarked CCs are summarised in \tabref{table:ComplementPredicates-1}. It is probably not an exhaustive listing: there may be more which simply do not occur very often. All predicates in this table have been found with unmarked CCs at least three times in the corpus. They belong to different semantic classes and have different properties of how they combine with a CC.  

\begin{table}[htbp]
\caption{Predicates that take unmarked CCs}
\small
\begin{tabularx}{\textwidth}{QlQQQQ}
\lsptoprule
Category & Predicate & Translation & \textsc{subj} & CC arg. & Restriction \\
& & & of CC & index on MC & on RS of CC\\

\midrule
desiderative & \textit{-sachu} & want &  SS \& DS & no & \textsc{irr} \\
& \textit{-sumachu} & want, like & SS \& DS & no & \textsc{irr} \\
knowledge \& ability & \textit{-(i)chuna} & know, be capable & SS & no  & as in MC \\
 & \textit{puero} & can, be able & SS & no & as in MC\\
manipulative & \textit{-bÿche(i)ku}  & send, order & DS & S/A & \textsc{irr}\\
perception & \textit{-samu} & hear & DS & S/A &  \textsc{irr}\\
utterance & \textit{-kechu} & say, tell & SS \& DS & S/A & no \\
\lspbottomrule
\end{tabularx}

\label{table:ComplementPredicates-1}
\end{table}

Only \textit{-sachu} ‘want’, \textit{-bÿche(i)ku} ‘send, order’ and the non-verbal \textit{puero}\is{knowledge/ability predicate}\is{non-verbal predication} ‘can’ can be considered secondary predicates. The others are able to take nominal objects, too. 

Both\is{knowledge/ability predicate} \textit{-(i)chuna} ‘know, be capable’ and \textit{puero} ‘can’ only take CCs with same subjects,\is{subject|(} while \textit{-bÿche(i)ku} ‘send, order’ and \textit{-samu} ‘hear’ always have CCs with different subjects. These latter ones also index an argument of the CC as their \isi{object}, namely the S or A of the CC. The CCs of the desiderative verbs \textit{-sachu} ‘want’ and \textit{-sumachu} ‘want, like’ predominantly have coreferential subjects, but can also have a different subject. As for \textit{-kechu} ‘say’, it can take CCs with both coreferential and different subjects, and in the latter case, the subject of the CC can be indexed on the verb as its object.\is{subject|)}

Most complement-taking predicates restrict the RS\is{reality status|(} of the predicate in the CC, but in a different way. The verbs \textit{-sachu} ‘want’, \textit{-sumachu} ‘want, like’, \textit{-bÿche(i)ku} ‘send, order’ and also \textit{puero} ‘can’ necessarily take irrealis CCs, while \textit{-(i)chuna}\is{knowledge/ability predicate} ‘know, be capable’ and \textit{-samu} ‘hear’ demand the CC to have equal RS. The verb \textit{-kechu} ‘say’ sticks out here, since it does not place any restriction on the RS of the CC.\is{reality status|)}

The following sections illustrate the use of CCs. I proceed by semantic category of the complement-taking predicates, because the predicates belonging to one category share several of their features in complementation as can be seen in \tabref{table:ComplementPredicates-1}. Most frequently, we find CCs of desideratives. They are explained in \sectref{sec:CC_Desideratives}. CCs of all other categories are much rarer, but some generalisations on CCs of knowledge and ability predicates are drawn in \sectref{sec:CC_KnowledgeAbility} nonetheless. CCs of the manipulative verb \textit{-bÿche(i)ku} are illustrated in \sectref{sec:CC_Manipulative}, \sectref{sec:CC_Perception} shows the use of the perception verb \textit{-samu} ‘hear’ with CCs, and finally, \sectref{sec:CC_Utterance} is about CCs of the utterance verb \textit{-kechu} ‘say’.


\subsubsection{Desiderative verbs}\label{sec:CC_Desideratives}\is{desiderative verb|(}

There are two desiderative verbs that take CCs, \textit{-sachu} ‘want’ and \textit{-sumachu} or \textit{-sumechu} ‘want, like’. The first of them, \textit{-sachu}, only takes clausal complements and is more frequent. The other one can also take nominal objects. When taking CCs, the verbs can be used largely interchangeably; \textit{-sumachu} also expresses liking, but more so, when there is a nominal object. On the other hand, \textit{-sachu} occasionally also states that something is imminent. This use is also found with the corresponding desiderative verb \textit{querer} in local Spanish.\footnote{\citet[860]{Kuteva2019} call this kind of use of a desiderative verb “proximative”.} (\ref{ex:want-rain}) is one example of the use of \textit{-sachu} to express imminence. I will then proceed with examples that show the desiderative use of \textit{-sachu} and \textit{-sumachu}.

(\ref{ex:want-rain}) was produced by Clara when the sky got grey and it got windy.

\ea\label{ex:want-rain}
\begingl
\glpreamble tisachutu \textup{[}tikeba\textup{]}\\
\gla ti-sachu-tu ti-keba\\
\glb 3i-want-\textsc{iam} 3i-rain.\textsc{irr}\\
\glft ‘it wants to rain now’, i.e. ‘it is about to rain’
\endgl
\trailingcitation{[cux-120410ls.253]}
\xe

As can be seen in (\ref{ex:want-rain}) above and also in the following two examples, (\ref{ex:want-sleep}) and (\ref{ex:want-write}), as well as in the rest of the examples in this section, the RS of the complement clause is always \isi{irrealis}, even if the complement-taking verb has realis RS.\is{reality status}

 In (\ref{ex:want-sleep}), Isidro describes a picture of a puzzle with a boy and a squirrel.

\ea\label{ex:want-sleep}
\begingl
\glpreamble tisachutu \textup{[}timuka\textup{]}\\
\gla ti-sachu-tu ti-muka\\
\glb 3i-want-\textsc{iam} 3i-sleep.\textsc{irr}\\
\glft ‘he wants to sleep now’
\endgl
\trailingcitation{[dxx-d120416s.086]}
\xe

In (\ref{ex:want-write}), Miguel explains to Juan C. what the plans and aims of the PDP team are in working with the Paunaka people. The CC is part of a \isi{cleft} construction in this case, see \sectref{sec:Clefts} for further information.

\ea\label{ex:want-write}
\begingl
\glpreamble chibu echÿu tisumachunube \textup{[}tisuikanube\textup{]}\\
\gla chibu echÿu ti-sumachu-nube ti-suika-nube\\
\glb 3\textsc{top.prn} \textsc{dem}b 3i-want-\textsc{pl} 3i-write.\textsc{irr}-\textsc{pl} \\
\glft ‘this is what they want to write’
\endgl
\trailingcitation{[mqx-p110826l.688]}
\xe

The previous three examples had CCs and main clauses with shared subjects. The following two examples illustrate the use of \textit{-sachu} with CCs with different subjects.

The main clause in (\ref{ex:wantDS2}) has a third person singular subject marked on the verb by the prefix \textit{ti-}, and the CC has a second person singular subject expressed by \textit{pi-} on the complement verb.\is{complement verb|(} The subject of the complement verb is not indexed on the desiderative verb as an object (the index would be \textit{-bi} in that case). The sentence was produced by Juana when I visited her at her house in Santa Cruz and received a call from my husband, who had accompanied me to Bolivia in 2011.

\ea\label{ex:wantDS2}
\begingl
\glpreamble tisachutu \textup{[}piyunupuna\textup{]}\\
\gla ti-sachu-tu pi-yunupuna\\
\glb 3i-want-\textsc{iam} 2\textsc{sg}-go.back.\textsc{irr}\\
\glft ‘now he wants you to go back’
\endgl
\trailingcitation{[jxx-e110923l-2.032]}
\xe


In (\ref{ex:wantDS1}), the desiderative verb has a first person singular subject, while the CC has a weather verb that takes a third person subject. There is another subordinate clause with a deranked verb that provides the reason for the wish expressed in the CC, see \sectref{sec:EmbeddedAC_bare} for this kind of adverbial clause. With (\ref{ex:wantDS1}), José explains why he is happy to hear thundering.

\ea\label{ex:wantDS1}
\begingl
\glpreamble nÿsachutu \textup{[}tikeba\textup{]} nebukia kÿjÿpimÿnÿ\\
\gla nÿ-sachu-tu ti-keba nÿ-ebuk-i-a kÿjÿpi-mÿnÿ\\
\glb 1\textsc{sg}-want-\textsc{iam} 3i-rain.\textsc{irr} 1\textsc{sg}-sow-\textsc{subord}-\textsc{irr} manioc-\textsc{dim}\\
\glft ‘I want it to rain now, so that I can plant my manioc seedlings’
\endgl
\trailingcitation{[mox-c110926s-1.208]}
\xe


Objects\is{object} are indexed on complement verbs, just like they are indexed on independent verbs (see \sectref{sec:NumberPersonVerbs}). In (\ref{ex:want-take}), the complement verb has a first person object as indexed with \textit{-ne}. The example stems from Juana reporting what her grandfather said when he was hassled by a spirit at night on his journey home from Moxos.

\ea\label{ex:want-take}
\begingl
\glpreamble “tisachu \textup{[}tumane\textup{]}”\\
\gla ti-sachu ti-uma-ne\\
\glb3i-want 3i-take.\textsc{irr}-1\textsc{sg} \\
\glft ‘“she wants to take me”’
\endgl
\trailingcitation{[jxx-p151016l-2.217]}
\xe


In (\ref{ex:wantDS3}), the complement verb takes the marker \textit{chÿ-} to signal that there is a third person object, which is non-human in this case, referring to a Juana’s own death which she had mentioned before. She expresses her anger about her daughter not visiting her.

\ea\label{ex:wantDS3}
\begingl
\glpreamble kuina nisacha \textup{[}chichupa\textup{]}\\
\gla kuina ni-sacha chi-chupa\\
\glb \textsc{neg} 1\textsc{sg}-want.\textsc{irr} 3-know.\textsc{irr}\\
\glft ‘I don’t want her to know it (i.e. if I die)’
\endgl
\trailingcitation{[jxx-p120430l-1.319]}
\xe

If the \isi{object} is conominated, it always follows the complement verb.\is{complement verb|)} One case of this is (\ref{ex:want-take-her}). The transitive verb of the CC takes the third person marker \textit{chÿ-} to indicate that there is a third person object and this object is additionally expressed by an NP which follows the verb.

The sentence is the conclusion of Juana’s telling how problematic it was for her daughter in Spain to work and care for her child during working hours. The desiderative verb has a frustrative marker, because the whole attempt of her daughter to bring her sister to Spain as a nanny failed.

\ea\label{ex:want-take-her}
\begingl
\glpreamble nechukue tisachuini \textup{[}chuma chipiji\textup{]}\\
\gla nechukue ti-sachu-ini chÿ-uma chi-piji\\
\glb therefore 3i-want-\textsc{frust} 3-take.\textsc{irr} 3-sibling\\
\glft ‘that’s why she would have liked to take her sister’
\endgl
\trailingcitation{[jxx-p110923l-1.374]}
\xe

An example with a conominated object in a CC taken by \textit{-sumachu} is (\ref{ex:eat-mill}). In this case, the CC is part of an antecedent in an unmarked conditional construction (see \sectref{sec:AsyndeticSubordination}). The object NP \textit{bijie semiya} ‘pututu with grain’ follows the verb \textit{binika}. Actually, it may seem that the second noun \textit{semiya} rather belongs to the consequent clause for semantic reasons, but intonation suggested that it was part of the object NP of the CC. This sentence was produced by María C. to explain to us what a grinding bowl is used for. \textit{Pututu} is a soup with corn.

\ea\label{ex:eat-mill}
\begingl
\glpreamble bisumachamÿnÿ \textup{[}binika bijie semiya\textup{]} ayÿbaka nechÿu\\
\gla bi-sumacha-mÿnÿ bi-nika bijie semiya a-yÿbaka nechÿu\\
\glb 1\textsc{pl}-want.\textsc{irr}-\textsc{dim} 1\textsc{pl}-eat.\textsc{irr} pututu grain 2\textsc{pl}-grind.\textsc{irr} \textsc{dem}c\\
\glft ‘if we want to eat \textit{pututu} with grain, you can grind it there’
\endgl
\trailingcitation{[cux-c120414ls-2.217]}
\xe


In (\ref{ex:want-teach}), the verb \textit{-mesumeiku} ‘teach’ is used ditransitively.\is{ditransitive} The teacher and the one who is taught are indexed on the verb, which is what we normally find with this verb, but the theme\is{patient/theme} that is taught is usually understood from the context or expressed in another clause. In this case, a sentence elicited form Miguel, the theme is expressed by an NP and placed after the \isi{complement verb}.

\ea\label{ex:want-teach}
\begingl
\glpreamble nÿsumachuyu \textup{[}pimesumeikanÿ echÿu petea\textup{]} \\
\gla nÿ-sumachu-yu pi-mesumeika-nÿ echÿu pi-etea \\
\glb 1\textsc{sg}-want-\textsc{ints} 2\textsc{sg}-teach-1\textsc{sg} \textsc{dem}b 2\textsc{sg}-language\\
\glft ‘I really want you to teach me your language’
\endgl
\trailingcitation{[mxx-e090728s-3.055]}
\xe

Oblique NPs and adverbs also follow the \isi{complement verb}, like the locative-marked noun \textit{yÿkÿyae} ‘on(to) fire’ in (\ref{ex:pot-cook-222}), in which Juana tells me what she wants to do with her clay pot, when it is ready.

\ea\label{ex:pot-cook-222}
\begingl
\glpreamble i despues nisumecha \textup{[}netuka yÿkÿyae\textup{]} niyÿtikapu\\
\gla i despues ni-sumacha nÿ-etuka yÿkÿ-yae ni-yÿtikapu\\
\glb and afterwards 1\textsc{sg}-want.\textsc{irr} 1\textsc{sg}-put.\textsc{irr} fire-\textsc{loc} 1\textsc{sg}-cook.\textsc{irr}\\
\glft ‘and after that I want to put it on fire in order to cook’
\endgl
\trailingcitation{[jxx-d110923l-4.14]}
\xe


The next example has an adverb following the complement verb. It is from a story about the clever fox tricking the naive jaguar. The sentence in (\ref{ex:tiger-died}) is what the fox spitefully says to the jaguar who drowns in a pond at the end, as reported by Juana. Frustrative is due to failure of the jaguar’s attempt to eat the fox.

\ea\label{ex:tiger-died}
\begingl
\glpreamble “pimua, pisachuini \textup{[}pinikanÿ uchuineini\textup{]}, tanÿma te pipakatu tanÿma”, tikechu\\
\gla pi-imua pi-sachu-ini pi-nika-nÿ uchuine-ini tanÿma te pi-paka-tu tanÿma ti-kechu\\
\glb 2\textsc{sg}-see.\textsc{irr} 2\textsc{sg}-want-\textsc{frust} 2\textsc{sg}-eat.\textsc{irr}-1\textsc{sg} just.now-\textsc{frust} now \textsc{seq} 2\textsc{sg}-die.\textsc{irr}-\textsc{iam} now 3i-say\\
\glft ‘“you see, you just wanted to eat me, and now, you will die now”, he said’
\endgl
\trailingcitation{[jmx-n120429ls-x5.279]}
\xe


Subjects\is{subject} are only seldom conominated in CCs,\is{conomination} because speakers usually construe their talk in a way that the subject of a CC is topical\is{topic} so that a subject index is sufficient. A few cases of conominated subjects are found nonetheless and in this case, the subject NP usually follows\is{word order} the \isi{complement verb} but may also precede it. In (\ref{ex:want-jump}) and (\ref{ex:want-bite}) the subject NP follows the complement verb. In (\ref{ex:Nacha}) and (\ref{ex:eka-eka}), the conominal subject precedes the desiderative verb for emphasis.

(\ref{ex:want-jump}) comes from Miguel describing the \isi{frog story} and refers to the picture on which the dog jumps against the tree with the beehive.

\ea\label{ex:want-jump}
\begingl
\glpreamble i naka tisachutu \textup{[}tjipuka\textup{]} echÿu kabe\\
\gla i naka ti-sachu-tu ti-jipuka echÿu kabe\\
\glb and here 3i-want-\textsc{iam} 3i-jump.\textsc{irr} \textsc{dem}b dog\\
\glft ‘and here the dog wants to (i.e. is about to) jump’
\endgl
\trailingcitation{[mtx-a110906l.091]}
\xe

(\ref{ex:want-bite}) was produced by Juana in telling me how her sister once had an encounter with a snake or water spirit in the reservoir of Santa Rita.

\ea\label{ex:want-bite}
\begingl
\glpreamble mÿbanejikuji tÿpi Maria tisachutÿini \textup{[}chinijabaka\textup{]} kechue\\
\gla mÿbanejiku-ji tÿpi Maria ti-sachu-tÿini chi-nijabaka kechue\\
\glb close-\textsc{rprt} \textsc{obl} María 3i-want-\textsc{avert} 3-bite.\textsc{irr} snake\\
\glft ‘being close to María, the snake almost wanted to bite her, it is said’
\endgl
\trailingcitation{[jxx-p120515l-2.161]}
\xe

In (\ref{ex:Nacha}), the subject introduces into the discourse a new participant whom I do not know so that quite a long expression is necessary, which is placed prominently in the first position of the sentence to make sure I do not miss whom Juana is talking about. She states that a relative of hers does not want to travel to Europe.

\ea\label{ex:Nacha}
\begingl
\glpreamble echÿu nikumadrene nauku Concecion, komadre Nacha, kuina tisacha \textup{[}tiyuna\textup{]} tÿbaneyu\\
\gla echÿu ni-kumadre-ne nauku Concecion komadre Nacha kuina ti-sacha ti-yuna ti-ÿbane-yu\\
\glb \textsc{dem}b 1\textsc{sg}-fellow-\textsc{possd} there Concepción fellow Nacha \textsc{neg} 3i-want.\textsc{irr} 3i-go.\textsc{irr} 3i-be.far-\textsc{ints}\\
\glft ‘my fellow there in Concepción, fellow Nacha, doesn’t want to go, because it is very far’
\endgl
\trailingcitation{[jxx-p120430l-1.175]}
\xe

In (\ref{ex:eka-eka}) placement of the demonstrative before the desiderative verb is for contrastive focus. The sentence refers to two frogs Juana points at on one of the last pictures in the \isi{frog story}.


\ea\label{ex:eka-eka}
\begingl
\glpreamble eka titibubuikubu i eka tisachu \textup{[}tijipuka\textup{]}\\
\gla eka ti-tibubuiku-bu i eka ti-sachu ti-jipuka\\
\glb \textsc{dem}a 3i-sit-\textsc{mid} and \textsc{dem}a 3i-want 3i-jump.\textsc{irr}\\
\glft ‘this one is sitting and this one wants to (i.e. is about to) jump’
\endgl
\trailingcitation{[jxx-a120516l-a.521-522]}
\xe


I have found but two examples in the corpus in which the subject of the desiderative verb is placed between the complement-taking and the \isi{complement verb}. In both examples, complement-taking and complement verbs have different subjects.\is{word order}

(\ref{ex:dog-cat}) was elicited from Juana. It is the only example in the corpus in which the desiderative and the \isi{complement verb} have two different third person subjects. Both subjects are conominated and the NPs follow their respective verb. The context of this sentence is that we were sitting in Juana’s yard and making a recording, when her cat and her dog began fighting about a bone.

\ea\label{ex:dog-cat}
\begingl
\glpreamble kuina tisacha eka kabe \textup{[}tinika eka michi\textup{]}\\
\gla kuina ti-sacha eka kabe ti-nika eka michi\\
\glb \textsc{neg} 3i-want.\textsc{irr} \textsc{dem}a dog 3i-eat.\textsc{irr} \textsc{dem}a cat\\
\glft ‘the dog doesn’t want the cat to eat it’
\endgl
\trailingcitation{[jxx-e120430l-4.18]}
\xe

(\ref{ex:Lena-wants}) comes from Miguel. The whole sentence was produced with relatively long pauses between the desiderative verb, the subject and the rest of the sentence, which is possibly due to its strange, non-preferred structure. It was directed to Juan C. with the aim to introduce me to him and explain what I wanted. \textit{Donya}  (Span. \textit{doña}) is a respectful form of address in Spanish, which is used a lot in the region.

\ea\label{ex:Lena-wants}
\begingl
\glpreamble tisachu donya Lena \textup{[}pichujika pario chikuyenakena pubupaikiu naka\textup{]}, ...\\
\gla ti-sachu donya Lena pi-chujika pario chikuyena-kena pi-ubupaik-i-u naka\\
\glb 3i-want \textsc{hon} Lena 2\textsc{sg}-speak.\textsc{irr} some how-\textsc{uncert} 2\textsc{sg}-be.born-\textsc{subord}-\textsc{real} here\\
\glft ‘doña Lena wants you to speak a bit about how you were born here, ...’
\endgl
\trailingcitation{[mqx-p110826l.004-007]}
\xe
\is{desiderative verb|)}

\subsubsection{Knowledge and ability predicates}\label{sec:CC_KnowledgeAbility}\is{knowledge/ability predicate|(}

The complement-taking predicates presented in this section are \textit{-(i)chuna} ‘know, be capable, be able’ and \textit{puero} ‘can, be able’.\footnote{There are two more knowledge verbs that are probably related to \textit{-(i)chuna}, \textit{-chupu} ‘know (a fact)’ and \textit{-chupuiku} ‘know or get to know (a person)’. The first of them \textit{-chupu} was found taking a CC a few times, but mostly occurred in elicitation after a researcher had started a sentence with this verb herself. Regarding \textit{-chupuiku}, there is one non-elicited example in the corpus with a CC. It is given in (\ref{ex:know-town}). María S. states here that going to town was not part of her daily life when she was a child. Today, it is fairly normal that people from Santa Rita regularly visit Concepción because of better possibilities of transportation. 

\ea\label{ex:know-town}
\begingl
\glpreamble kuina bichupuika \textup{[}biyuna uneku\textup{]}\\
\gla kuina bi-chupuika bi-yuna uneku\\
\glb \textsc{neg} 1\textsc{pl}-know.\textsc{irr} 1\textsc{pl}-go.\textsc{irr} town\\
\glft ‘going to town was unknown to us’
\endgl
\trailingcitation{[rxx-p181101l-2.164]}
\xe

In the remainder of this section, both verbs will be ignored due to lack of (further) convincing examples to prove their ability of taking clausal complements.}
 Both occur more often in negative clauses\is{negation} than in positive ones and both have certain peculiarities when combining with a CC as will become apparent in the following discussion. They have in common that they can only take a CC if the \isi{subject} is coreferential. That they are combined with a CC at all may be ultimately due to Spanish influence. This is quite obvious for \textit{puero}, which is borrowed\is{borrowing} from the Spanish modal verb\is{modality} \textit{poder} ‘can’.
  
The verb \textit{-(i)chuna} means ‘know, be capable, be able’ in the sense of having acquired a capacity by doing something regularly or having done something before. Being morphologically intransitive and stative,\is{transitive stative verb}  this verb should not be able to take neither nominal objects\is{object} nor clausal complements, but we find both in addition to it being used intransitively. I thus suspect that it underwent semantic shift, possibly due to an analogy to a Spanish or Bésiro verb. It still inflects like a stative verb though, with the irrealis marker being prefixed to the verb stem. Like the desiderative verbs (see \sectref{sec:CC_Desideratives} above), it always takes the third person marker \textit{ti-} if the subject is a third person,\is{person marking} and never \textit{chÿ-}. The RS of the CC always has to match the one of \textit{-(i)chuna}.\is{reality status}

One example of a negative clause with \textit{-(i)chuna} is (\ref{ex:know-1-1}). The knowledge verb takes a prefix to mark irrealis and the \isi{complement verb} also has irrealis RS.
The sentence comes from María S. who helped me formulate what I wanted to say. Talking about some tobacco leaves she was drying, I wanted to tell her that I do not smoke, but did not know the word for ‘smoke’. The sentence in (\ref{ex:know-1-1}) was the second one that came to her mind. She first used \textit{puero} ‘can’ as a complement-taking predicate (see (\ref{ex:can-1}) below), but found \textit{-(i)chuna} more appropriate and thus corrected herself.

\ea\label{ex:know-1-1}
\begingl
\glpreamble kuina paichuna \textup{[}pijibÿka\textup{]}\\
\gla kuina pi-a-ichuna pi-jibÿka\\
\glb \textsc{neg} 2\textsc{sg}-\textsc{irr}-be.capable 2\textsc{sg}-smoke.\textsc{irr}\\
\glft ‘you are not capable of smoking’ (in the sense of: ‘you have never tried smoking’ or ‘you don’t have the habit of smoking’)
\endgl
\trailingcitation{[rxx-e120511l.381]}
\xe

If the CC has a conominated object, it follows the \isi{complement verb}, as in (\ref{ex:know-1-3}), in which María C. expresses what she believes to be the reason for me buying bread made by a German lady who lived in Concepción at that time.

\ea\label{ex:know-1-3}
\begingl
\glpreamble kuina achuna \textup{[}enika eka pan de aroj\textup{]}\\
\gla kuina e-a-chuna e-nika eka {pan de aroj}\\
\glb \textsc{neg} 2\textsc{pl}-\textsc{irr}-be.capable 2\textsc{pl}-eat.\textsc{irr} \textsc{dem}a {rice bread}\\
\glft ‘you are not capable of eating rice bread’ (in the sense of: ‘you don’t have the habit of eating it, you don’t know how good it is’)
\endgl
\trailingcitation{[uxx-e120427l.128]}
\xe

Another example with an object following the \isi{complement verb} is (\ref{ex:know-pot}), in which two CCs are coordinated. This sentence is a statement by Juana, when she described how to make a clay pot and contrasted her knowledge to that of the young women today.

\ea\label{ex:know-pot}
\begingl
\glpreamble kuina taichunanube \textup{[}tananube nÿkÿiki tananube yÿpi\textup{]}\\
\gla kuina ti-a-ichuna-nube ti-ana-nube nÿkÿiki ti-ana-nube yÿpi\\
\glb \textsc{neg} 3i-\textsc{irr}-be.capable-\textsc{pl} 3i-make.\textsc{irr}-\textsc{pl} pot 3i-make.\textsc{irr}-\textsc{pl} jar\\
\glft ‘they don’t know how to make pots and how to make jars’
\endgl
\trailingcitation{[jxx-p120430l-2.549]}
\xe


The object of the complement-taking verb can also be a relative clause as in (\ref{ex:know-write}) (see \sectref{sec:HeadlessRC} for this kind of relative clause). This is a positive sentence with \textit{-(i)chuna}, the verb thus has realis RS and the \isi{complement verb} consequently has realis RS, too. The sentence comes from Juana and refers to Tiburcio, a deceased relative of hers who was a very good story-teller never forgetting any stories.

\ea\label{ex:know-write}
\begingl
\glpreamble nenayu tichuna \textup{[}tisuiku eka tumuyubu chikuetea\textup{]}\\
\gla nena-yu ti-ichuna ti-suiku eka tumuyubu chi-kuetea\\
\glb like-\textsc{ints} 3i-be.capable 3i-write \textsc{dem}a all 3-tell\\
\glft ‘it seemed as if he could write down everything he narrated’
\endgl
\trailingcitation{[jmx-n120429ls-x5.042]}
\xe


In all examples I found that have a conominated subject, the subject precedes the knowledge verb.\is{word order}

%In (\ref{ex:know-burn}), another example for a positive sentence, the subject is conominated by the first person singular pronoun \textit{nÿti}, which precedes \textit{-(i)chuna}. The sentence comes from María C. and followed her statement in (\ref{ex:know-1-3}), contrasting her knowledge of making rice bread with my lack of habit eating it.
%
%\ea\label{ex:know-burn}
%\begingl
%\glpreamble nÿti nichuna \textup{[}nejÿkine pan de aroj\textup{]}\\
%\gla nÿti ni-chuna nÿ-jÿkine {pan de aroj}\\
%\glb 1\textsc{sg.prn} 1\textsc{sg}-be.capable 1\textsc{sg}-bake {rice bread}\\
%\glft ‘I know how to bake rice bread’\\
%\endgl
%\trailingcitation{[uxx-e120427l.131]}
%\xe
%%%THIS ACTUALLY SEEMS TO BE A NOMINALISED VERB, WOW!!!!

(\ref{ex:know-1-2}) has a first person singular pronoun preceding the negated knowledge verb. This sentence was triggered by me asking Juana for the word for ‘swim’. She told me the word and gave several examples how to use it, then she stated that she cannot swim in (\ref{ex:know-1-2}).

\ea\label{ex:know-1-2}
\begingl
\glpreamble nÿti kuina naichuna \textup{[}nabueji\textup{]}\\
\gla nÿti kuina nÿ-a-ichuna nÿ-a-ubueji\\
\glb 1\textsc{sg.prn} \textsc{neg} 1\textsc{sg}-\textsc{irr}-be.capable 1\textsc{sg}-\textsc{irr}-swim \\
\glft ‘I can’t swim’
\endgl
\trailingcitation{[jxx-a120516l-a.561]}
\xe

In (\ref{ex:know-1-4}) the third person subject is conominated by the demonstrative \textit{eka}. It was a translation by Juana of what she and Miguel had just discussed in Spanish that it would be good if somebody came to write down their language, because the speakers cannot do that themselves.

\ea\label{ex:know-1-4}
\begingl
\glpreamble michayu, jimu eka tichuna \textup{[}tisuiku\textup{]}\\
\gla micha-yu jimu eka ti-ichuna ti-suiku\\
\glb good-\textsc{ints} \textsc{mir} \textsc{dem}a 3i-be.capable 3-write\\
\glft ‘this is very good, you know, that one knows to write’
\endgl
\trailingcitation{[jmx-e090727s.028]}
\xe

I have only found two examples in which a conominated subject is a noun, and in both cases this noun precedes the knowledge verb.

(\ref{ex:know-write-2}) was elicited from Juana.

\ea\label{ex:know-write-2}
\begingl
\glpreamble nijinepÿi taichunatu \textup{[}tisuika\textup{]}, profesuruina uchu\\
\gla ni-jinepÿi ti-a-ichuna-tu ti-suika profesuru-ina uchu\\
\glb 1\textsc{sg}-daughter 3i-\textsc{irr}-be.capable 3i-write.\textsc{irr} teacher-\textsc{irr.nv} \textsc{uncert.fut}\\
\glft ‘when my daughter will know how to write, some day she will be a teacher’
\endgl
\trailingcitation{[jxx-p150920l.068]}
\xe

(\ref{ex:know-speak}) was produced by Juana in the speech she gave at the workshop on Paunaka in 2011. The CC is part of another CC of the desiderative verb \textit{-sachu} ‘want’, but a non-canonical one, compare to the examples given in \sectref{sec:CC_Desideratives} above. It is probably influenced by the way she would have phrased it in Spanish (\textit{más bien quisiera que sea así que...}).

\ea\label{ex:know-speak}
\begingl
\glpreamble mas bien nisachumÿnÿini \textup{[}eka chakuyena eka nÿuchikupÿimÿnÿ taichuna \textup{[}tichujika eka betea\textup{]}\textup{]}\\
\gla {mas bien} ni-sachu-mÿnÿ-ini eka chÿ-a-kuye-na eka nÿ-uchikupÿi-mÿnÿ ti-a-ichuna ti-chujika eka bi-etea\\
\glb {rather} 1\textsc{sg}-want-\textsc{dim}-\textsc{frust} \textsc{dem}a 3-\textsc{irr}-be.like.this-? \textsc{dem}a 1\textsc{sg}-grandchild-\textsc{dim} 3i-\textsc{irr}-be.capable 3i-speak.\textsc{irr} \textsc{dem}a 1\textsc{pl}-language\\
\glft ‘I would rather like it to be that way that my grandson knows to speak our language’
\endgl
\trailingcitation{[jxx-x110916.24-25]}
\xe


Besides \textit{-(i)chuna}, there is also \textit{puero}\is{modality|(} ‘can’, from the Spanish modal verb \textit{poder} borrowed\is{borrowing|(} into Paunaka as a non-verbal predicate\is{non-verbal predication|(} possibly via \isi{Bésiro}, which has a noun \textit{puerux} ‘ability’. Since Paunaka does not have modal verbs like Spanish, I opt for an analysis as a complement-taking predicate. \textit{Puero} itself is remarkable, because unlike other non-verbal predicates borrowed from Spanish, it does not inflect for person\is{person marking} and it does not obligatorily take an \isi{irrealis} marker if it is combined with a CC and irrealis is already marked on the complement predicate (see \sectref{sec:borrowed_verbs}).\is{non-verbal predication|)} There is, however, nothing special about the CC: it is a clause with a balanced predicate that is juxtaposed to the matrix clause.

 \textit{Puero} is mostly found in negative clauses.\is{negation|(} This is probably bound to the semantic parameters encoded by \isi{irrealis} RS in Paunaka (see \sectref{sec:RealityStatus}): ability is among the things encoded by irrealis as well as negation. If an ability is negated, there are two parameters that trigger irrealis marking. Other languages may use a \isi{doubly irrealis construction} in this case, while Paunaka makes use of a borrowed predicate that encodes ability. Both parameters – negation and ability – are thus overtly expressed. There are also a few cases where we find \textit{puero} being used in positive sentences, an indication that this word is beginning to take over the function of marking ability from irrealis RS, at least in part. The verb in the CC is still \isi{irrealis} in a positive sentence. I will start with a few negative sentences and show the positive ones afterwards.\is{negation|)}

In both (\ref{ex:cannot-walk}) and (\ref{ex:can-1}), \textit{puero} is not inflected for irrealis despite being negated. The complement verbs\is{complement verb} are irrealis though.

In (\ref{ex:cannot-walk}), Juana makes a statement about her ill son-in-law.

\ea\label{ex:cannot-walk}
\begingl
\glpreamble kuina puero \textup{[}tiyuika\textup{]}\\
\gla kuina puero ti-yuika\\
\glb \textsc{neg} can 3i-walk.\textsc{irr}\\
\glft ‘he cannot walk’
\endgl
\trailingcitation{[jxx-p110923l-1.048]}
\xe
\is{borrowing|)}

(\ref{ex:can-1}) is what María S. first said in helping me to express that I do not smoke, before she decided to use the verb \textit{-(i)chuna} instead, see  (\ref{ex:know-1-1}) above.

\ea\label{ex:can-1}
\begingl
\glpreamble kuina puero \textup{[}pijibÿka\textup{]}\\
\gla kuina puero pi-jibÿka\\
\glb \textsc{neg} can 2\textsc{sg}-smoke.\textsc{irr}\\
\glft ‘you can’t smoke’
\endgl
\trailingcitation{[rxx-e120511l.379]}
\xe

In the following two examples, irrealis is also marked on \textit{puero}. (\ref{ex:cannot-field}) was produced by María S. in an elicitation session.

\ea\label{ex:cannot-field}
\begingl
\glpreamble kuina tiyuna, kuina pueroina \textup{[}tiyuna asaneti\textup{]}\\
\gla kuina ti-yuna kuina puero-ina ti-yuna asaneti\\
\glb \textsc{neg} 3i-go.\textsc{irr} \textsc{neg} can-\textsc{irr.nv} 3i-go.\textsc{irr} field\\
\glft ‘he (an old man) doesn’t go (there), he cannot go to his field’
\endgl
\trailingcitation{[rxx-e181022le]}
\xe

In (\ref{ex:cannot-walk-2}), Miguel completed a sentence I started to make about my 15-month-old daughter, whom we were watching.\footnote{Actually, I am not totally sure about the placement of the right square bracket here. It depends on whether one analyses the verb \textit{tipÿsisikapu} ‘she is alone’ or ‘she does it alone’ to modify the complement verb or the whole sentence. I decided for an analysis in which it modifies the whole sentence, but the other option would work as well, I believe.}

\ea\label{ex:cannot-walk-2}
\begingl
\glpreamble kuina pueroina \textup{[}tiyuika\textup{]} tipÿsisikapu\\
\gla kuina puero-ina ti-yuika ti-pÿsisikapu\\
\glb \textsc{neg} can-\textsc{irr.nv} 3i-walk.\textsc{irr} 3i-be.alone.\textsc{irr}\\
\glft ‘she cannot walk on her own’
\endgl
\trailingcitation{[mxx-e110820ls.008]}
\xe

Among the positive sentences with \textit{puero} is (\ref{ex:can-walk-grab}), which is from the same situation as (\ref{ex:cannot-walk-2}) above. Although the sentence is not negated, the \isi{complement verb} has irrealis RS.

\ea\label{ex:can-walk-grab}
\begingl
\glpreamble puero \textup{[}tiyuika\textup{]} chÿjatÿtÿika eka punachÿ\\
\gla puero ti-yuika chÿ-jatÿtÿika eka punachÿ\\
\glb can 3i-walk.\textsc{irr} 3-pull.\textsc{irr} \textsc{dem}a other\\
\glft ‘she can walk if another person pulls her (i.e. holds her hand)’
\endgl
\trailingcitation{[mxx-e110820ls.012]}
\xe

Another positive sentence with an irrealis \isi{complement verb} is (\ref{ex:sell-little}). In this example, the object of the complement verb is preposed to \textit{puero}, which is highly unusual. Paunaka allows OV structures if the O shall be emphasised (see \sectref{sec:WordOrder}), but we do not normally find this in complementation (i.e. O[VV]).\is{word order}

The sentence comes from Juana, who earns her money by selling food. However, since she has never learnt to calculate, she sometimes does not know how much change she has to give. Only small amounts of money are easy to handle for her.

\ea\label{ex:sell-little}
\begingl
\glpreamble \textup{[}echÿu chepitÿjikumÿnÿyÿchi\textup{]} puero \textup{[}nipabenteika\textup{]}\\
\gla echÿu chepitÿ-jiku-mÿnÿ-yÿchi puero ni-pabenteika\\
\glb \textsc{dem}b small-\textsc{lim}1-\textsc{dim}-\textsc{lim}2 can 1\textsc{sg}-sell.\textsc{irr}\\
\glft ‘only small things, I can sell’
\endgl
\trailingcitation{[jxx-e110923l-2.150]}
\xe

I want to conclude with (\ref{ex:cannot-sit}), which also shows the use of \textit{puero} in a positive sentence. It comes from María S. and the context is that I wanted to elicit some spatial expressions, so I asked to translate the sentence ‘the chicken sits on the chair’, which amused Miguel and María S., because, as María S. rightly objected, chicken cannot sit. I thus tried again with ‘step on’ instead of ‘sit’. This works better as she states:

\ea\label{ex:cannot-sit}
\begingl
\glpreamble aa puero \textup{[}chikupachajiku nechÿu\textup{]} kuina titibua\\
\gla aa puero chi-kupacha-jiku nechÿu kuina ti-tibua\\
\glb \textsc{intj} can 3-step.on.\textsc{irr}-\textsc{lim}1 \textsc{dem}c \textsc{neg} 3i-sit.down.\textsc{irr}\\
\glft ‘ah, it can only step on it, but it cannot sit down’
\endgl
\trailingcitation{[rmx-e150922l.129]}
\xe
\is{modality|)}
\is{knowledge/ability predicate|)}
%kuina pueroinakuÿ chichujika = tadavía no puede hablar, mxx-e110820ls.030
%juberÿpunÿtu kuina puero trabakuinabo, uxx-p110825l.203
%pue kuina puero tinikabu tebiyuku kuina puero tea jaja, jxx-p120430l-2.366


\subsubsection{The manipulative verb \textit{-bÿche(i)ku}}\label{sec:CC_Manipulative}\is{manipulative verb|(}


Manipulative predicates have been defined as such predicates that “express a relation between an \isi{agent} or a situation which functions as a cause, an affectee, and a resulting situation. The affectee must be a participant in the resulting situation” \citep[136]{Noonan2007}.

There is one manipulative predicate in Paunaka, \textit{-bÿche(i)ku} ‘send, order’. %In addition, the speech verb \textit{-kechu} ‘say, tell’ can sometimes be used as a manipulative, too, but only with a deranked verb (see \sectref{sec:CCs_deranked}). %There is no predicate to express permission to my knowledge, this would probably be expressed by the use of irrealis.
The RS of the complement predicate is always \isi{irrealis}, regardless of whether the situation described by the complement predicate has been realised by the time that the sentence is produced.

One of the very few non-elicited examples with this verb is (\ref{ex:send-1}). It shows that, unlike with desiderative verbs (see \sectref{sec:CC_Desideratives}), the conominated subject of the CC follows the complement-taking verb.\is{word order} This is due to the fact that it is expressed as its proper object, as will become apparent in the examples below, where we find indexes on the verb.\footnote{It is unclear whether the verb \textit{-bÿche(i)ku} can occur without a CC, since almost all examples are elicited translations of Spanish sentences. They were formulated in such a way that they necessarily take a CC. This should be investigated further.} The affectee is also expressed as the subject of the CC, and the \isi{complement verb} thus takes the person marker \textit{ti-}. Note that the second verb in this sentence, \textit{tipÿsisiku} ‘be alone’, has realis RS, thus it probably belongs to the main clause, not to the CC, which has irrealis RS. I do not know why it does not take the middle voice marker in this case, as it usually does.

The sentence was produced by Clara, who was sitting with us chatting. Clara earns her money by baking bread. Her daughter, apparently, when she went to school, had told Clara to ask her other daughter to help her make the bread. That was before we arrived and consequently her daughter was doing all the work alone now. Clara found it funny and at the same time felt a bit guilty about making her daughter do all the work alone.\footnote{The daughter grinds manioc to bake rice bread, which is made with a dough of manioc and rice and often gets a cheese topping.}

\ea\label{ex:send-1}
\begingl
\glpreamble nibÿchekutu nijinepÿi tipÿsisiku \textup{[}tiyÿbajika kÿjÿpi\textup{]}\\
\gla ni-bÿcheku-tu ni-jinepÿi ti-pÿsisiku ti-yÿbajika kÿjÿpi\\
\glb 1\textsc{sg}-order-\textsc{iam} 1\textsc{sg}-daughter 3i-be.alone 3i-grind.\textsc{irr} manioc\\
\glft ‘I ordered my daughter to grind the manioc all by herself’
\endgl
\trailingcitation{[cux-120410ls.223]}
\xe


A similar example is (\ref{ex:CC-3-3}). The shared conominated argument, which is the object of the manipulative verb and subject of the \isi{complement verb}, is placed after the manipulative verb.\is{word order} This sentence was elicited from Juana.

\ea\label{ex:CC-3-3}
\begingl
\glpreamble ¡pibÿchekaji kabe \textup{[}chinijabaka echÿu tumei\textup{]}!\\
\gla pi-bÿcheka-ji kabe chi-nijabaka echÿu ti-umei\\
\glb 2\textsc{sg}-order-\textsc{imp} dog 3-bite.\textsc{irr} \textsc{dem}b 3i-steal\\
\glft ‘give the dog the command to bite the thief!’
\endgl
\trailingcitation{[jxx-e191021e-2]}
\xe

In (\ref{ex:send-run}), the shared argument is not conominated, but it is encoded as the object of the manipulative verb by use of the third person marker \textit{chÿ-}, which is used for specific 3>3 relationships. This example was elicited from María S. Irrealis of the manipulative verb is probably due this sentence being elicited without any context.

\ea\label{ex:send-run}
\begingl
\glpreamble chibÿchekatu \textup{[}tikutikapu\textup{]}\\
\gla chi-bÿcheka-tu ti-kutikapu\\
\glb 3-order.\textsc{irr}-\textsc{iam} 3i-run.\textsc{irr}\\
\glft ‘(s)he makes him run’
\endgl
\trailingcitation{[rxx-e141230s.123]}
\xe

(\ref{ex:send-dloc}) shows two things. First, the shared first person singular argument is indexed on the matrix verb as an object. And second, if movement is implied in the order, the \isi{dislocative} marker (see \sectref{sec:PA}) can be attached to the \isi{complement verb}. This shows how closely related this type of CC is to a \isi{motion-cum-purpose construction}, which has been analysed to encode an adverbial relation (see \sectref{sec:MotionCumPurpose}). The example was elicited from Miguel.


\ea\label{ex:send-dloc}
\begingl
\glpreamble nijinepÿi tibÿchekunÿ \textup{[}niyÿseikupa chÿeche\textup{]}\\
\gla ni-jinepÿi ti-bÿcheku-nÿ ni-yÿseiku-pa chÿeche\\
\glb 1\textsc{sg}-daughter 3i-order-1\textsc{sg} 1\textsc{sg}-buy-\textsc{dloc.irr} meat\\
\glft ‘my daughter sent me to buy meat’
\endgl
\trailingcitation{[mxx-e160811sd.291]}%el.
\xe

A similar example was elicited from Juana, where she preferred a CC containing a complete motion-cum-purpose construction including the motion verb \textit{-yunu}. Note that the motion verb has realis RS in this case. This may be a mistake or a sign that we are dealing with two coordinated clauses here rather than with a CC: ‘I sent her and she went to buy cinnamon’. This issue cannot be resolved at the moment for lack of data.

\ea\label{ex:send-MCPC}
\begingl
\glpreamble nÿbÿcheku \textup{[}tiyunu tiyÿseikupa eka kanela\textup{]}\\
\gla nÿ-bÿcheku ti-yunu ti-yÿseiku-pa eka kanela\\
\glb 1\textsc{sg}-order 3i-go 3i-buy-\textsc{dloc.irr} \textsc{dem}a cinnamon\\
\glft ‘I made her go to buy cinnamon’
\endgl
\trailingcitation{[jxx-e191021e-2]}
\xe 

%mxx-e160811sd.295, tibÿchikune, pibÿchekune nisupa
\is{manipulative verb|)}


\subsubsection{The perception verb \textit{-samu}}\label{sec:CC_Perception}\is{perception verb|(}

Perception verbs do not often take CCs as arguments, I have only found a few examples with the verb \textit{-samu} ‘hear’.  
According to \citet[143]{Noonan2007}, in perception complementation “semantically it is the entire event, not the argument coded as the matrix direct object, that is perceived”. It is difficult to determine whether a person perceives an entire event or primarily one participant in that event if there are no formal linguistic means to express the difference. If the shared argument of the main clause and the CC is conominated,\is{conomination|(} the CC has exactly the same structure as an unmarked relative clause\is{relative relation|(} (see \sectref{sec:UnmarkedRC}). 


This can be seen in (\ref{ex:hear-partridge}): first comes the complement-taking verb, its object directly follows it, then comes a clause which could be either a CC or a relative clause modifying the object.

The sentence is from Juana’s narration about the journey of her grandparents from Moxos. Her grandfather has had an unpleasant encounter with a water spirit at night and hardly slept, so that the singing of the partridge in the very early morning is a relief for him.

\ea\label{ex:hear-partridge}
\begingl
\glpreamble chisamutu mukÿkÿi \textup{[}tiyutu\textup{]}\\
\gla chi-samu-tu mukÿkÿi ti-iyu-tu\\
\glb 3-hear-\textsc{iam} partridge 3i-cry-\textsc{iam}\\
\glft ‘he already heard the partridge singing’ (or: ‘he already heard the partridge that sang’)
\endgl
\trailingcitation{[jxx-p151016l-2.212]}
\xe


If the object is not conominated, a difference to relative clauses is notable.

Consider (\ref{ex:hear-2}), in which the complement-taking perception verb is itself a complement of a desiderative verb, but this is not of concern at the moment. Let us consider the perception verb and its complement. The object of \textit{-samu} ‘hear’ is indexed by the plural marker \textit{-nube} and the CC directly follows.

The sentence was elicited from Juana. 

\ea\label{ex:hear-2}
\begingl
\glpreamble kuina nisacha \textup{[}nisamanube \textup{[}chichujijikabunube\textup{]}\textup{]}\\
\gla kuina ni-sacha ni-sama-nube chi-chujijika-bu-nube\\
\glb \textsc{neg} 1\textsc{sg}-want.\textsc{irr} 1\textsc{sg}-hear.\textsc{irr}-\textsc{pl} 3-talk.\textsc{irr}-\textsc{mid}-\textsc{pl}\\
\glft ‘I don’t want to hear them talking’
\endgl
\trailingcitation{[jxx-e190210s-01]}
\xe


A headless relative clause on the other hand is usually introduced by a demonstrative,\is{nominal demonstrative|(} see \sectref{sec:HeadlessRC}. Thus (\ref{ex:hear-shout}) is analysed as containing a headless relative clause not a CC, with the relative clause being underlined.\footnote{Note that the plural marker on the verb belongs to the subject in this case, not to the object as in (\ref{ex:hear-2}) above.} 

(\ref{ex:hear-shout}) comes from a story told by Miguel about two men who meet the devil in the woods. The devil approaches the men shouting.

\ea\label{ex:hear-shout}
\begingl
\glpreamble chisamunube \underline{echÿu tiyÿbuiku} kimenukÿ\\
\gla chi-samu-nube echÿu ti-yÿbuiku kimenu-kÿ\\
\glb 3-hear-\textsc{pl} \textsc{dem} 3i-shout woods-\textsc{clf:}bounded\\
\glft ‘they heard the one who shouts in the woods’
\endgl
\trailingcitation{[mxx-n101017s-1.020]}
\xe
\is{conomination|)}

For lack of a demonstrative preceding the clause, (\ref{ex:hear-1}) can then again be analysed as a CC, but a word of caution is necessary here: many weather and environment concepts are expressed by verbs in Paunaka. These verbs can be integrated into a clause like nouns if they are formed as a headless relative clause. However, if such a weather or environment verb is more often expressed as a nominal constituent of a sentence (be that an argument or an adjunct) than as a predicate, use of the demonstrative can become optional and the verb may ultimately turn into a noun. This is what has happened with \textit{tijai} ‘day’, actually meaning ‘it is light’, see \sectref{sec:UnmarkedRC}. The verb \textit{tiramuku} ‘it thunders’ in this example was also translated to Spanish with a noun by José.

This sentence was produced by Miguel, when he and Swintha once visited José and a thunderstorm was approaching. 

\ea\label{ex:hear-1}
\begingl
\glpreamble bisamukutu \textup{[}tiramuku\textup{]}\\
\gla bi-samu-uku-tu ti-ramuku\\
\glb 1\textsc{pl}-hear-\textsc{add}-\textsc{iam} 3i-thunder\\
\glft ‘we already hear it thunder, too’
\endgl
\trailingcitation{[mox-c110926s-1.176]}
\xe\is{nominal demonstrative|)}

This is an exhaustive listing of all cases of \textit{-samu} being combined with an unmarked clause in the corpus, so that it is not possible to come up with a good conclusion about the difference between CCs of perception verbs and relative clauses modifying the object of a perception verb or in the case of a headless relative clause becoming the object itself.\is{relative relation|)}
\is{perception verb|)}

\subsubsection{The utterance verb \textit{-kechu}}\label{sec:CC_Utterance}\is{utterance verb|(}

The \isi{speech verb} \textit{-kechu} ‘say’ can take a CC to report what others have said, i.e. in indirect speech. This is not very common in Paunaka, as speakers clearly prefer to cite directly what others said, i.e. in direct speech. They may also express that something is reported by using the hearsay/reportive marker\is{evidentiality} \textit{-ji} (see \sectref{sec:Evidentiality}). Nonetheless, indirect speech occurs from time to time. 

CCs of \textit{-kechu} are different from all other CCs in that their RS\is{reality status} is not predetermined by the construction, but depends on its realisation in relation to the point in time of the original utterance, i.e. the one that is cited. Thus we have irrealis RS in (\ref{ex:say-rep-1}), because at the time Swintha made her statement, the event had not been realised.

The example comes from Miguel. He just talked for the sake of the recording, narrating what he was thinking about, and this is why he uses third person markers to refer to Swintha, who was accompanying him. The sentence is about Swintha’s return to Germany.

\ea\label{ex:say-rep-1}
\begingl
\glpreamble tiyunupunuka Alemania te pero tikechu \textup{[}tibÿsÿupupunuka punachinakena kuje o punachina anyo\textup{]}\\
\gla ti-yunu-punuka Alemania te pero ti-kechu ti-bÿsÿu-pupunuka punachÿ-ina-kena kuje o punachÿ-ina anyo\\
\glb 3i-go-\textsc{reg.irr} Germany \textsc{seq} but 3i-say 3i-come-\textsc{reg.irr} other-\textsc{irr.nv}-\textsc{uncert} month or other-\textsc{irr.nv} year\\
\glft ‘she will go back to Germany, but she said that she would come back, maybe the other month or the other year’
\endgl
\trailingcitation{[mxx-d110813s-2.049-050]}
\xe

That use of RS depends on the point in time the original utterance was made becomes more clear in (\ref{ex:say-rep-2}). Juana speaks about the past here. The reported event has long been realised by the time she produced the sentence, but not by the time the reported utterance was made. In the meeting she talks about, people of Santa Rita should discuss a proposal of a lady who came to the region, a proposal that they make pasture for her in exchange for her taking charge of construction of the reservoir.

\ea\label{ex:say-rep-2}
\begingl
\glpreamble i tikechunube \textup{[}tananube reunion\textup{]}\\
\gla i ti-kechu-nube ti-ana-nube reunion\\
\glb and 3i-say-\textsc{pl} 3i-make-\textsc{pl} meeting\\
\glft ‘and she told them to make a meeting’
\endgl
\trailingcitation{[jxx-p120515l-2.077]}
\xe

In the following example, realis RS is used in the CC, which contains another CC, because the statement held at the time the original utterance was made. The frustrative marker is placed on the verb due to Juana’s evaluation of her daughter’s wish to be unrealisable. Juana simply does not want what her daughter wants: accompany her to Europe. Irrealis RS of verb in the second CC is due to its own complement-taking verb, desiderative \textit{-sachu} (see \sectref{sec:CC_Desideratives} above).

\ea\label{ex:say-rep-3}
\begingl
\glpreamble nijinepÿi tikechu ukuinebu \textup{[}tisachuini \textup{[}tumane\textup{]}\textup{]}\\
\gla ni-jinepÿi ti-kechu ukuinebu ti-sachu-ini ti-uma-ne\\
\glb 1\textsc{sg}-daughter 3i-say some.time.ago 3i-want-\textsc{frust} 3i-take.\textsc{irr}-1\textsc{sg}\\
\glft ‘some time ago my daughter said that she wants to take me’
\endgl
\trailingcitation{[jxx-e120430l-4.44-45]}
\xe

(\ref{ex:coord-purp-2}) has indirect speech inside a direct quotation.
The example comes from Miguel telling me about his experience in school. His teacher is the one who is cited by using direct speech and in this direct speech Miguel is requested to tell something to his father, namely, that he helps him obtain the things he needs in school. The utterance verb \textit{-kechu} inside the quoted speech has a third person marker following the stem, which we also find on speech verbs introducing direct speech. This marker marks the verb as \isi{ditransitive}, and we can thus deduce that the CC can theoretically be indexed on the verb as an \isi{argument},\footnote{It is a unclear though whether the addressee or the clause is indexed by \textit{-chÿ}, see discussion in \sectref{sec:3_suffixes}.} but this is by no means obligatory as can be seen in all other examples given in this section.

\ea\label{ex:coord-purp-2}
\begingl
\glpreamble “pikechuchÿji echÿu pia \textup{[}tisemaika echÿu yÿkÿke tana taurapachumÿnÿ\textup{]} i nebu pisuikia”, tikechu\\
\gla pi-kechu-chÿ-ji echÿu pi-a ti-semaika echÿu yÿkÿke ti-ana taurapachu-mÿnÿ i nebu pi-suik-i-a ti-kechu\\
\glb 2\textsc{sg}-say-3-\textsc{imp} \textsc{dem}b 2\textsc{sg}-father 3i-search.\textsc{irr} \textsc{dem}b wood 3i-make.\textsc{irr} board-\textsc{dim} and 3\textsc{obl.top.prn} 2\textsc{sg}-write-\textsc{subord}-\textsc{irr} 3i-say\\
\glft ‘“tell your father to look for wood to make a small board and on that one you can write”, he said’
\endgl
\trailingcitation{[mxx-p181027l-1.022]}
\xe
\is{utterance verb|)}

%tikechu doña Sinthia tupuna echÿu pichai, mux-c110810l.106
\is{finite verb|)}

\subsection{Complement clauses with deranked verbs}\label{sec:CCs_deranked}
\is{complement verb|(}
\is{deranked verb|(}

In addition to unmarked CCs, there are a number of sentences in the corpus that have clausal complements with a deranked verb. The reasons for this may be different. First of all, if a verb that usually takes an unmarked CC is combined with a deranked verb, there may be something unusual involved in the relation between both verbs. In the examples I found in the corpus, this was usually a greater amount of force or control on the subject of the complement verb by the subject of the complement-taking verb. However, many of the examples come from elicitation, so it is not clear how natural this overt marking of the subordinate relation is in general.

Consider (\ref{ex:CC-der-1}), in which the complement verb is \textit{-niku}. This verb can mean both ‘eat’ and ‘feed, give food’, and it might be this ambiguity that led Miguel use a subordinate verb to signal that the interpretation cannot be straightforward (‘I don’t want you to eat it’), but that there is some force implied. Note also that the subordinate verb irregularly takes realis RS.

\ea\label{ex:CC-der-1}
\begingl
\glpreamble kuina nisacha \textup{[}pinikiuchÿ\textup{]}\\
\gla kuina ni-sacha pi-nik-i-u-chÿ\\
\glb \textsc{neg} 1\textsc{sg}-want.\textsc{irr} 2\textsc{sg}-feed-\textsc{subord}-\textsc{real}-3\\
\glft ‘I don’t want you to make him eat’
\endgl
\trailingcitation{[mxx-e090728s-3.007]}
\xe


(\ref{ex:CC-der-2}), another example elicited from Miguel, has a perception verb, \textit{-mumuku} ‘look, watch’, combined with a CC containing a deranked verb.\footnote{Actually this verb, as well as any other verb of sight, does not occur with an unmarked CC in the corpus, but given the fact that \textit{-samu} ‘hear’ does take unmarked CCs, I consider it realistic that verbs of sight can theoretically take unmarked CCs, too.} In this case, the perception verb does not express mere perception, but rather has the overtone that the subject of the matrix predicate takes care that the subject of the complement verb fulfils the task of reading.

\ea\label{ex:CC-der-2}
\begingl
\glpreamble nimumuka \textup{[}chichujimeikia\textup{]}\\
\gla ni-mumuka chi-chujimeik-i-a\\
\glb 1\textsc{sg}-watch.\textsc{irr} 3-read-\textsc{subord}-\textsc{irr}\\
\glft ‘I take care of her reading’
\endgl
\trailingcitation{[mxx-e120415ls.149]}
\xe

A similar example is (\ref{ex:tell-sleep}), elicited from María S. The sentence has two CCs: the first one is unmarked, the second one has a deranked verb. The second one implies force: the first person subject was not simply spoken to, but ordered to sleep against her will.

\ea\label{ex:tell-sleep}
\begingl
\glpreamble ¡kuinaini nisacha \textup{[}nÿmuka\textup{]} piti pikechunÿ \textup{[}nimukia\textup{]}!\\
\gla kuina-ini ni-sacha nÿ-muka piti pi-kechu-nÿ ni-muk-i-a\\
\glb \textsc{neg}-\textsc{frust} 1\textsc{sg}-want.\textsc{irr} 1\textsc{sg}-sleep 2\textsc{sg.prn} 2\textsc{sg}-say-1\textsc{sg} 1\textsc{sg}-sleep-\textsc{subord}-\textsc{irr}\\
\glft ‘I didn’t want to sleep, you told me to sleep!’
\endgl
\trailingcitation{[rxx-e181024l]}
\xe

One non-elicited sentence that contains \textit{-kechu} ‘say’ together with a CC containing a deranked verb is (\ref{ex:tell-dog}). It is also analysed as including an order. The sentence stems from Miguel’s description of the \isi{frog story} and was produced when he was looking at the picture in which the boy and the dog look over a log and finally meet the frog again. It is not clear to me why he uses the instrumental preposition\is{instrument/cause} \textit{chikeuchi}. Maybe it is because the log, or tree/wood as he calls it, is not upright but lying in the water.

\ea\label{ex:tell-dog}
\begingl
\glpreamble tikechuchÿjiku \textup{[}chipunaikiu chikeuchi yÿkÿke\textup{]}\\
\gla ti-kechu-chÿ-jiku chi-punaik-i-u chi-keuchi yÿkÿke\\
\glb 3i-say-3-\textsc{lim}1 3-go.up-\textsc{subord}-\textsc{real} 3-\textsc{ins} tree\\
\glft ‘he only told him to go up on the tree’
\endgl
\trailingcitation{[mox-a110920l-2.176-179]}
\xe

Note, however, that some of the examples with an unmarked CC of the same utterance verb can also be read as encoding an order rather than mere report, see \sectref{sec:CC_Utterance}. It is also possible that some verbs allow a certain variation in taking a CC with a balanced or a deranked verb.\footnote{This seems to be the case in \isi{Baure}, where the verb in a CC can either be balanced or occur as a participle, often without any difference in meaning \citep[424]{Danielsen2007}.} 

There may also be some variation between speakers. For example, in elicitation on causative relations, Miguel often but not always chose a deranked verb in combination with the \isi{manipulative verb} \textit{-bÿche(i)ku}  as in (\ref{ex:CC-3-3-2}) and (\ref{ex:CC-der-123}). Note that the deranked verb in (\ref{ex:CC-der-123}) exceptionally takes the third person marker\is{person marking} \textit{ti-} (see \sectref{sec:Subordination-i}). For examples with balanced verbs in the CC see \sectref{sec:CC_Manipulative}.

\ea\label{ex:CC-3-3-2}
\begingl
\glpreamble nibÿchekubi \textup{[}pupuniachÿ nichechapÿi tukiu nauku terminal\textup{]}\\
\gla ni-bÿcheku-bi pi-upun-i-a-chÿ ni-chechapÿi tukiu nauku terminal\\
\glb 1\textsc{sg}-order-2\textsc{sg} 2\textsc{sg}-bring-\textsc{subord}-\textsc{irr}-3 1\textsc{sg}-son from there bus.station\\
\glft ‘I sent you to pick up my son at the bus station’
\endgl
\trailingcitation{[mxx-e160811sd.301-302]}
\xe

%pibuichikune niyejikia arusu, mxx-e120415ls.135

\ea\label{ex:CC-der-123}
\begingl
\glpreamble nibÿcheku nichechapÿi \textup{[}tupuniapi\textup{]}\\
\gla ni-bÿcheku ni-chechapÿi ti-upun-i-a-pi\\
\glb 1\textsc{sg}-order 1\textsc{sg}-son 3i-bring-\textsc{subord}-\textsc{irr}-2\textsc{sg}\\
\glft ‘I sent my son to pick you up’
\endgl
\trailingcitation{[mxx-e160811sd.307]}
\xe

In a short check about the verb \textit{-muyayachu} ‘be slow, do slowly’, which had just popped up in elicitation, María S. combined the verb first with a deranked verb as in (\ref{ex:stat-i-2}), and two minutes later with a balanced verb, see (\ref{ex:statnoi-2}). It is unclear whether one or the other construction is more frequent, because there is not enough data.

\ea\label{ex:stat-i-2}
\begingl
\glpreamble nimuyayachutu \textup{[}ninikia\textup{]}\\
\gla ni-muyayachu-tu ni-nik-i-a\\
\glb 1\textsc{sg}-do.slowly-\textsc{iam} 1\textsc{sg}-eat-\textsc{subord}-\textsc{irr}\\
\glft ‘I eat slowly’
\endgl
\trailingcitation{[rmx-e150922l.086]}
\xe

\ea\label{ex:statnoi-2}
\begingl
\glpreamble nimuyayachu \textup{[}niyÿtikapu\textup{]}\\
\gla ni-muyayachu ni-yÿtikapu\\
\glb 1\textsc{sg}-do.slowly 1\textsc{sg}-cook.\textsc{irr}\\
\glft ‘I cook slowly’
\endgl
\trailingcitation{[rmx-e150922l.091]}
\xe

It is unclear at the moment what exactly triggers the use of a deranked verb if a balanced verb is possible too. 

There are, however, some verbs that only take CCs with deranked verbs. We can assume that these verbs usually do not take clausal complements at all and in the rare cases that there is a demand to combine them with a clause, choosing a deranked verb with its more nominal characteristics is the most adequate solution. Possibly every \isi{transitive} verb can take a deranked one as an \isi{object}, I will only present two examples with verbs that could be assumed to take CCs due to their semantics. 

The verb \textit{-itu} ‘master, manage, cope with, learn’ encodes knowledge and ability, but unlike other verbs of this category (see \sectref{sec:CC_KnowledgeAbility} above) does not usually combine with a clause. In the rare cases it does, the complement verb has to be deranked, as in (\ref{ex:know-speak-222}). Note that \textit{-itu} is very rarely used in general. The context of the following example is as follows: Clara had stated that she wants to teach her daughter Paunaka, because the latter wants to learn it. Clara is sure that she can learn it quickly if she wants to. Note that the subordinate verb also takes the third person\is{person marking} prefix \textit{ti-} here.

\ea\label{ex:know-speak-222}
\begingl
\glpreamble kue tisacha \textup{[}tichujika\textup{]}, un ratoyÿchi chita \textup{[}tichujikia\textup{]}\\
\gla kue ti-sacha ti-chujika {un rato}-yÿchi chÿ-ita ti-chujik-i-a\\
\glb if 3i-want.\textsc{irr} 3i-speak.\textsc{irr} {in a while}-\textsc{lim}2 3-master.\textsc{irr} 3i-speak-\textsc{subor}-\textsc{irr}\\
\glft ‘if she wants to speak, she can quickly learn to speak’
\endgl
\trailingcitation{[cux-c120414ls-2.327]}
\xe

Another verb that combines with a CC containing a deranked verb is \textit{-buku} ‘finish’. This verb can take a conominal NP in O function, but usually does not take clausal complements. One exception is (\ref{ex:finish-hammock}), which is an elicited sentence about making a hammock. Knotting refers to the technique of weaving. The women finish their hammocks by knotting the wefts for adornment.

\ea\label{ex:finish-hammock}
\begingl
\glpreamble nakayenetu nÿbuka \textup{[}nipikeikiu\textup{]}\\
\gla nakayenetu nÿ-buka ni-pikeik-i-u\\
\glb almost 1\textsc{sg}-finish.\textsc{irr} 1\textsc{sg}-knot-\textsc{subord}-\textsc{real}\\
\glft ‘I have almost finished my knotting (of the hammock)’
\endgl
\trailingcitation{[rxx-e181022le]}
\xe
\is{deranked verb|)}
\is{complement verb|)}


\subsection{The use of \textit{eka} as a complementiser}\label{sec:CC_CMPL}
\is{nominal demonstrative|(}\is{complementiser|(}

There are a few cases in which the demonstrative \textit{eka} seems to be used as a complementiser in CCs containing balanced predicates.\is{finite verb} Not every case in which we find \textit{eka} between a complement-taking predicate and a CC is necessarily a complementiser though, as the demonstrative is also used as a filler in hesitation. All of the examples presented in this section were selected because they do not show signs of hesitation, like repeated starts, pauses, or use of the question word \textit{¿chija?} (which is also used as a filler in hesitation), so it is relatively certain that \textit{eka} really functions as a complementiser there.

%+ use of eka after chikuyena?? \ref{sec:Q_chikuyena}

It may be the case that use of a complementiser is due to influence of Spanish, but comparison with the most closely related languages reveals that in Trinitario,\is{Mojeño Trinitario} an article optionally introduces unmarked CCs \citep[91]{Rose2014a}, while in \isi{Baure} an article or a demonstrative can introduce CCs with a participle, although this is not obligatory \citep[424]{Danielsen2007}.\footnote{In addition, \isi{Baure} makes use of the multifunctional connective \textit{apo} as a complementiser \citep[393]{Danielsen2007}.} This does not exclude the possibility that the use of an article or demonstrative in these languages is also due to influence of Spanish, but then again, it is striking that all three languages use a determiner rather than anything else. In Spanish, the complementiser \textit{que} is identical to the relative pronoun and to the question word for ‘what’.\footnote{Note, however, that a demonstrative may be used in relative clauses as well in Paunaka, especially in headless relative clauses. We find predominantly \textit{echÿu} there, but \textit{eka} is also used sometimes, see \sectref{sec:RelativeClauses}.}


In (\ref{ex:want-sing}), both the matrix and the \isi{complement verb} have a first person subject and the complement verb is intransitive so that there is no \isi{argument} in the sentence to which \textit{eka} could refer. Therefore the demonstrative must be analysed as a complementiser belonging to the CC. The sentence is from the story about the fox and the jaguarundi as narrated by Miguel. It is an utterance of the drunken fox.

\ea\label{ex:want-sing}
\begingl
\glpreamble “nÿsachutu \textup{[}eka nakusunine\textup{]}”\\
\gla nÿ-sachu-tu eka nÿ-a-kusunine\\
\glb 1\textsc{sg}-want-\textsc{iam} \textsc{dem}a 1\textsc{sg}-\textsc{irr}-sing\\
\glft ‘“I want to sing now”’
\endgl
\trailingcitation{[jmx-n120429ls-x5.380]}
\xe

The case is less clear in all other examples, since they have third person arguments that the demonstrative could theoretically refer to.

Consider (\ref{ex:DEMO}), which was elicited from Miguel. The \isi{complement verb} has a third person subject, so the demonstrative could be analysed to conominate this subject. However, if we compare to the other CCs of desiderative verbs, we find that subject NPs never precede the verb in a CC\is{word order} (see \sectref{sec:CC_Desideratives}), so that this must be a case of \textit{eka} being used as a complementiser, too. In addition, if this was a subject in preverbal position, i.e. in a position reserved for arguments with special discourse status, it would be emphasised by intonation, but the opposite is the case. This is also true for the rest of the examples.

\ea\label{ex:DEMO}
\begingl
\glpreamble kuina nÿsumacha \textup{[}eka takutiu\textup{]}\\
\gla kuina nÿ-sumacha eka ti-a-kutiu\\
\glb \textsc{neg} 1\textsc{sg}-want.\textsc{irr} \textsc{dem}a 3i-\textsc{irr}-be.ill\\
\glft ‘I don’t want him to get ill’
\endgl
\trailingcitation{[mxx-e090728s-3.013]}
\xe

(\ref{ex:hear-dead}) was produced by Clara and directed to María C., who is a bit hard of hearing, in order to tell her that Swintha and I already knew that her husband had passed away.

\ea\label{ex:hear-dead}
\begingl
\glpreamble tisamuikunube \textup{[}eka tepaku eka pima\textup{]}\\
\gla ti-samuiku-nube eka ti-paku eka pi-ima\\
\glb 3i-listen-\textsc{pl} \textsc{dem}a 3i-die \textsc{dem}a 2\textsc{sg}-husband\\
\glft ‘they heard that your husband is dead’
\endgl
\trailingcitation{[cux-120410ls.096]}
\xe

In (\ref{ex:pensau-2}), Juana actually had some trouble with wording, but the last part, the CC including the demonstrative, was uttered without hesitation. The sentence expresses what she believed to be the reason for Federico going to Miguel’s daughter’s house, which I had just been telling her before.

\ea\label{ex:pensau-2}
\begingl
\glpreamble pensau \textup{[}eka chayuraucha\textup{]}\\
\gla pensau eka chÿ-ayuraucha\\
\glb think \textsc{dem}a 3-help.\textsc{irr}\\
\glft ‘he thought he could help him’
\endgl
\trailingcitation{[jxx-e120516l-1.094]}
\xe

%i kuina eka trabakuina, mox-n110920l.039

%Finally, there are also a few examples in which \textit{te} is used as a complementiser in a similar fashion.
%
%\ea\label{ex:}
%\begingl
%\glpreamble tisachu te tatÿbÿkÿmiu\\
%\gla ti-sachu te ti-a-tÿbÿkÿmiu\\
%\glb 3i-want \textsc{seq} 3i-\textsc{irr}-be.quiet\\
%\glft ‘he wants him to be quiet’\\
%\endgl
%\trailingcitation{[jmx-n120429ls-x5.383]}
%\xe
%

%kuinayuini te tikeba, jxx-e110923l-2.054
%yutituji te tiyÿbuituji mupÿinube = at night the devil screamed, rxx-n120511l-2.23


%chikuye eka nÿ- nÿ- pueroyÿchi eka nitu pario, mxx-p181027l-1.156
%pero nÿsachuini eka nana eka nÿpujikia eka amukeyu, translation unclear, jrx-c151001lsf-11.182
%pero kuina naichuna eka eka nipuika eka kampio, jxx-e110923l-2.148
%pisamu eka ti- timajaikupunukutu echÿu kabe chitÿpijiku eka takÿrajane = escuchas está ladrando otra vez el perro nomás a las gallinas, mox-a110920l-1



%general clause separator:?

%chima kaku eka pobre kristianonube eka chubuinubetu te tiyeseikumÿnenubetu chipeu baka te, ??vieron?? que había gente pobre, ya viejita que se había comprado vacas, jxx-e150925l-1.257


CCs introduced by a demonstrative resemble headless relative clauses a lot. It is then only through context and possibly due to the semantics of the main clause verb that both can be distinguished.\is{complementiser|)}\is{nominal demonstrative|)}\is{complement relation|)} Relative clauses are the topic of the next section.








