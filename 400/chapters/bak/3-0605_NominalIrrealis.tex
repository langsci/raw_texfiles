%!TEX root = 3-P_Masterdokument.tex
%!TEX encoding = UTF-8 Unicode

\section{Nominal irrealis}\label{NominalRS}
\is{nominal irrealis|(}

Nominal irrealis is a category that has not been widely described up to now. It is reminiscent of the better-known category of \isi{nominal tense} \citep[]{NordlingerSadler2004} (aka nominal temporal markers cf. \citealt[]{Tonhauser2008}) – a relatively widespread feature in South American languages (\citealt[158--163]{Aikhenvald2012}; \citealt[258]{Campbell2012}).

Nominal irrealis is marked by attaching \textit{-ina} to the noun in question. The very same marker also figures as an irrealis marker in \isi{non-verbal predication}. There is sometimes considerable overlap between both functions (see \sectref{sec:NonVerbalPredication}), but there are also enough cases in which nominal irrealis can well be distinguished from predication. These cases are described in this section.

In nominal irrealis, \textit{-ina} indicates that an entity is non-existent or that it has not come into existence yet. This is in compliance with the function it fulfils in predication (see \sectref{sec:RealityStatus}). Regarding syntax, I have found irrealis-marked objects\is{object} and obliques,\is{oblique} but no irrealis-marked subjects.

Consider (\ref{ex:IRR-OBJ-1}) from Juan C. Irrealis marking is due to the non-existence of a pair of trousers in the possession of the speaker here, because his \textit{patrón} refused to give him one. Before the land reform of 1952, many people in the Chiquitania worked in a debt-bondage relation on the haciendas of big landowners, \textit{patrones}, who were supposed to “pay” their workers in kind (see \sectref{sec:Century18-20}). Depending on the character or mood of the \textit{patrón}, people were often paid badly or not paid at all.

\ea\label{ex:IRR-OBJ-1}
\begingl 
\glpreamble kuina tipunakane nikasuneina\\
\gla kuina ti-punaka-ne ni-kasune-ina\\ 
\glb \textsc{neg} 3i-give.\textsc{irr}-1\textsc{sg} 1\textsc{sg}-trousers-\textsc{irr.nv}\\ 
\glft ‘he didn’t give me my supposed trousers’\\ 
\endgl
\trailingcitation{[mqx-p110826l.458]}
\xe

Note that, if the pair of trousers in question existed, the object would not be marked for irrealis. Imagine, for example, a situation, in which a child wants to put on a pair of trousers, but the mother refuses to give it to her because it has just been washed and is drying, or because the child is supposed to wear that pair of trousers on Sunday service. In this situation, which was described to Miguel in elicitation, the statement of the child would be as in (\ref{ex:noIRR-OBJ-1}) and use of irrealis on the object would be incorrect.

\ea\label{ex:noIRR-OBJ-1}
\begingl 
\glpreamble nÿenu kuina tipunakane nikasune\\
\gla nÿ-enu kuina ti-punaka-ne ni-kasune\\ 
\glb 1\textsc{sg}-mother \textsc{neg} 3i-give.\textsc{irr}-1\textsc{sg} 1\textsc{sg}-trousers\\ 
\glft ‘my mother didn't give me my trousers’\\ 
\endgl
\trailingcitation{[mxx-e160811sd.039]}
\xe

(\ref{ex:IRR-functive}) is from the personal account about Juana’s daughter who once wanted to emigrate to Spain to live with her sister as a nanny. The plan never worked out, this is why irrealis and frustrative is used on the predicate (see \sectref{sec:Frust_avertive_optatiev}). The irrealis on the oblique noun with the function of functive \citep[cf.][]{Creissels2014} complies with this reading: the job as an attendant was never accomplished despite the strong expectation of the people involved.

\ea\label{ex:IRR-functive}
\begingl 
\glpreamble i tiyunaini arsaroremÿnÿina tÿpi chisobrinonemÿnÿ\\
\gla i ti-yuna-ini arsarore-mÿnÿ-ina tÿpi chi-sobrino-ne-mÿnÿ\\ 
\glb and 3i-go.\textsc{irr}-\textsc{frust} attendant-\textsc{dim}-\textsc{irr.nv} \textsc{obl} 3-nephew-\textsc{possd}-\textsc{dim}\\ 
\glft ‘and she would have gone as an attendant of her nephew’\\ 
\endgl
\trailingcitation{[jxx-p120430l-1.188]}
\xe

In (\ref{ex:IRR-transformative}), the field that is talked about by Miguel has not been made at all because the main character of the story is a lazybones who prefers swinging in his hammock and playing the flute to the hard physical work of wresting a field from the woods.

\ea\label{ex:IRR-transformative}
\begingl 
\glpreamble kuinaji tana pario chisaneina\\
\gla kuina-ji ti-ana pario chi-sane-ina\\ 
\glb \textsc{neg}-\textsc{rprt} 3i-make.\textsc{irr} some 3-field-\textsc{irr.nv}\\ 
\glft ‘he didn't do anything for his (supposed) field, it is said’\\ 
\endgl
\trailingcitation{[mox-n110920l.012]}
\xe

The previous examples have shown the use of the nominal irrealis to express non-existence in absolute terms. Moreover, the non-existence of the entity marked with \textit{-ina} was contrary to the expectation of the people involved in all cases. This can be called the “negative use”\is{negation} of nominal irrealis. The other semantic context found to be expressed by nominal irrealis is \isi{future reference}. (\ref{ex:IRR-OBJ-2}) is from a story by Miguel about ants and trees and their relation to humans: Trees are sad, when a boy is born because once he grows up, he fells trees for making his field. The irrealis-marked object \textit{chisaneina} ‘his (future) field’ has not come into existence by the reference time of the clause, which is the birth of the boy.

\ea\label{ex:IRR-OBJ-2}
\begingl 
\glpreamble chejepuine echÿu aitubuchepÿi tijÿkatu, tiyunaji tebitaka chisaneina\\
\gla chejepuine echÿu aitubuchepÿi ti-jÿka-tu ti-yuna-ji ti-ebitaka chi-sane-ina\\ 
\glb because \textsc{dem}b boy 3i-grow.\textsc{irr}-\textsc{iam} 3i-go.\textsc{irr}-\textsc{rprt} 3i-clear.\textsc{irr} 3-field-\textsc{irr.nv}\\ 
\glft ‘because once the boy has grown up, he will go and clear his future field, it is said’\\ 
\endgl
\trailingcitation{[mxx-n120423lsf-X.28]}
\xe

Irrealis nouns also occur in purposive expressions. In (\ref{ex:IRR-NOM-purposive}) the aim of the action is additionally marked by the oblique preposition \textit{tÿpi}. It comes from María S. 

\ea\label{ex:IRR-NOM-purposive}
\begingl 
\glpreamble niyunu niyÿbamukeikupu arusu tÿpi niyitÿina\\
\gla ni-yunu ni-yÿbamukeiku-pu arusu tÿpi ni-yÿti-ina\\ 
\glb 1\textsc{sg}-go 1\textsc{sg}-husk-\textsc{dloc} rice \textsc{obl} 1\textsc{sg}-food-\textsc{irr.nv}\\ 
\glft ‘I went to husk rice (in machine in town) for my (future, not yet made) food’\\ 
\endgl
\trailingcitation{[rxx-e120511l.024-025]}
\xe

Nominal irrealis specifies the non-existence of the entity in question, be it absolute or only at reference time. It is not related to the semantics of RS marking of the predicate and should therefore be independent of RS marking of the predicate. Nonetheless, I have found only two examples of irrealis-marked objects in combination with a realis predicate in my corpus. Both come from Juana, and both are about houses.

In (\ref{ex:IRR-OBJ-REAL}), although the action of building the house is completed as signalled by the realis predicate and the general past setting of the story, the irrealis marker on the object is used to express that one of the main characters (Jesus in this case) does not live in the house by reference time, it is his future house,\is{future reference} the one where he is going to live after marrying his wife.

\ea\label{ex:IRR-OBJ-REAL}
\begingl 
\glpreamble tanau chubiuna\\
\gla ti-anau chÿ-ubiu-ina\\ 
\glb 3i-make 3-house-\textsc{irr.nv}\\ 
\glft ‘he made his house (where he was going to live)’\\ 
\endgl
\trailingcitation{[jxx-n101013s-1.552]}
\xe

In (\ref{ex:nomirr-1}), Juana speaks about ongoing construction of the future house of one of her daughters.

\ea\label{ex:nomirr-1}
\begingl
\glpreamble ja’a puakenechÿ tanaunube chubiunubeina\\
\gla ja’a pu-akene-chÿ ti-anau-nube chÿ-ubiu-nube-ina\\
\glb \textsc{afm} other-non.vis.side-3 3i-make-\textsc{pl} 3-house-\textsc{pl}-\textsc{irr.nv}\\
\glft ‘yes, on the other side (of the street) they are making their future house’
\endgl
\trailingcitation{[jxx-p110923l-2.154]}
\xe


Last, non-verbal irrealis is also frequently found on temporal nouns or adverbs\is{temporal/aspectual} to trigger a future reading,\is{future reference} as in (\ref{ex:sabaruina}), which comes from Juana, who was talking about a visit of her brother at her other brother’s house. The latter was not at home.

\ea\label{ex:sabaruina}
\begingl 
\glpreamble sabaruina kapunuina\\
\gla sabaru-ina kapunu-ina\\ 
\glb saturday-\textsc{irr.nv} come-\textsc{irr.nv}\\ 
\glft ‘he will come on Saturday’\\ 
\endgl
\trailingcitation{[jxx-p120430l-2.411]}
\xe

In complex NPs,\is{noun phrase} nominal irrealis is marked only once, i.e. it is not a feature of agreement between a noun and its modifier. This can be seen in (\ref{ex:punachina}), in which the irrealis marker only occurs on the modifier \textit{punachÿ} ‘other’, but not on the noun \textit{semana} ‘week’. The sentence comes from Miguel who was talking about Swintha here.

\ea\label{ex:punachina}
\begingl 
\glpreamble punachina semana tiyunupunukatu\\
\gla punachÿ-ina semana ti-yunu-punuka-tu\\ 
\glb other-\textsc{irr.nv} week 3i-go-\textsc{reg.irr}-\textsc{iam}\\ 
\glft ‘next week she will leave again’\\ 
\endgl
\trailingcitation{[mxx-d110813s-2.043]}
\xe

The whole construction in (\ref{ex:punachina}) resembles the local Spanish expression \textit{la otra semana} ‘the other week’, which can refer either to the preceding or coming week. In Paunaka, the distinction is made by using either an irrealis-marked NP for the coming week or no irrealis marker for reference to the preceding week.

While nominal irrealis relates to the non-existence of an entity,\is{nominal irrealis|)} the markers described in the next section tell us about ceased existence, more precisely, about the fact that somebody is already deceased.
