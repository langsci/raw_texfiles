%!TEX root = 3-P_Masterdokument.tex
%!TEX encoding = UTF-8 Unicode

\section{Person and number} \label{sec:NumberPersonVerbs}
\is{inflection|(}
\is{person marking|(}

Verbs take the very same person markers as nouns do: on nouns they index possessors and subjects in \isi{non-verbal predication} (see \sectref{sec:Possession} and \sectref{sec:NonVerbalPredication}), on verbs they index subjects and objects.\is{argument|(} The occurrence of the same set of person markers on nouns and verbs is common among \isi{Arawakan languages} (\citealp[cf.][89]{Aikhenvald1999}; \citealt[176]{Aikhenvald2012}). There is, however, one person marker that only occurs with verbs in Paunaka, the third person marker \textit{ti-}, which is also found in the \isi{Mojeño languages}. It is described in detail in \sectref{sec:3Marking}.

\tabref{table:VerbsPerson_all} summarises the subject\is{subject|(} and object\is{object|(} indexes found on verbs.\footnote{Remember that the variation of the vowel in the first person singular and third person marker is largely determined by the following vowel, although the correspondence is not perfect, see \sectref{sec:Possession}. The variation in the first person singular object index is free. In the second person singular object index, RS marking\is{reality status} determines the allomorph, see \sectref{sec:1_2Marking} below.}

\begin{table}
\caption{Person marking on verbs}

\begin{tabular}{lll}
\lsptoprule
&  Subject & Object \cr
\midrule
1\textsc{sg} & \textit{nÿ-/ni-} & \textit{-ne/-nÿ} \cr
2\textsc{sg} & \textit{pi-} & \textit{-bi/-pi} \cr
3 & \textit{ti-} & \textit{(-chÿ)} \cr
1\textsc{pl} & \textit{bi-} & \textit{-bi} \cr
2\textsc{pl} & \textit{e-} & \textit{-e} \cr
3\textsc{pl} (human) & \textit{ti-…-nube} & \textit{-nube}  \cr
\midrule
3>3 & \multicolumn{2}{c}{\textit{chÿ-/chi-}}\cr
3(\textsc{pl})>3(\textsc{pl}) & \multicolumn{2}{c}{\textit{chÿ-…-nube}} \cr
\lspbottomrule
\end{tabular}

\label{table:VerbsPerson_all}
\end{table}
\is{object|)}


Subject markers\is{alignment|(} precede the verb stems of both main classes, which are stative\is{stative verb} and active\is{active verb}.\footnote{In a number of more distantly related \isi{Arawakan languages}, stative verbs take suffixes to mark subjects \citep[cf.][212]{Aikhenvald2012}, this is why I emphasise this fact here.} The latter includes both intransitive and transitive verbs. Indexing the subject is in general obligatory on all verbs, but in rapid speech, person markers are sometimes dropped. This partly depends on the speaker -- both Marías drop person markers more often than other speakers do --, and partly on the marker – the one most likely to be dropped is the third person marker \textit{ti-}.\footnote{Note that the person marker \textit{ti-} is generally optional in Ignaciano\is{Mojeño Ignaciano} if the verb stem starts with a consonant \citep[482]{OlzaZubiri2004}.} An NP\is{noun phrase} referring to the subject can co-occur, but this is not necessary, thus the NP can be defined as an optional conominal\is{conomination} and the person markers as cross-indexes according to the terminology used by \citet[213, 219]{Haspelmath2013}.\is{subject|)} 

First and second person objects\is{object} are indexed by a person marker following the stem.\is{verbal stem} In this regard, Paunaka is a typical Arawakan language\is{Arawakan languages} \citep[cf.][]{Danielsen2014} exhibiting nominative-accusative alignment.\is{alignment|)} Third person markers that follow the stem occur rarely and only under specific circumstances, which will be discussed in \sectref{sec:3_suffixes}. The index \textit{chÿ-}, however, expresses 3>3 relationships on verbs. That means that in general, third person singular objects are only indexed on the verb if there is also a third person subject, i.e. jointly with the subject by \textit{chÿ-}. This marker is used obligatorily if the object is human, but it is optional with non-human objects.\is{object}

Both third person markers \textit{ti-} and \textit{chÿ-} are underspecified for number. The \isi{plural}, \isi{distributive} or \isi{collective} marker can be used to encode that a third person subject or object is non-singular.

The following sections describe subject and object marking in more detail, starting in \sectref{sec:1_2Marking} with the description of how first and second person arguments are indexed. Subsequently, I attend to the third person in \sectref{sec:3Marking}, with \sectref{sec:Verbs_3PL} being devoted to third person plural indexing. \sectref{sec:3_suffixes} provides an overview of the contexts in which a third person marker can follow a verb (or other) stem.



\subsection{First and second person}\label{sec:1_2Marking}
\is{subject|(}

Every verb obligatorily takes a subject marker. For first and second person subjects, there are four indexes preceding the stem, and they encode first singular, second singular, first plural, and second plural respectively. \tabref{table:VerbsPerson1_2} shows these indexes.

\begin{table}
\caption{1st and 2nd person subject indexes}

\begin{tabular}{lll}
\lsptoprule
& \textsc{sg} & \textsc{pl} \cr
\midrule
1 & \textit{nÿ-/ni-} & \textit{bi-} \cr
2 & \textit{pi-} & \textit{e-} \cr
\lspbottomrule
\end{tabular}

\label{table:VerbsPerson1_2}
\end{table}

The first and second person markers are found on \isi{transitive} and \isi{intransitive} verbs alike. Concerning the intransitive verbs, both active and stative verbs take indexes that precede the stem. (\ref{ex:ni-INT-stat-1}) shows a first person marker on a stative intransitive verb, in (\ref{ex:ni-INT-act}) the marker indexes the subject of an active intransitive verb and in (\ref{ex:ni-TRANS}) it is used with a transitive verb.

\ea\label{ex:ni-INT-stat-1}
\begingl 
\glpreamble nÿnai\\
\gla nÿ-ÿnai\\ 
\glb 1\textsc{sg}-be.long\\ 
\glft ‘I am tall’\\ 
\endgl
\trailingcitation{[rmx-e150922l.149]}
\xe

\ea\label{ex:ni-INT-act}
\begingl 
\glpreamble niyunu\\
\gla ni-yunu\\ 
\glb 1\textsc{sg}-go\\ 
\glft ‘I went’\\ 
\endgl
\trailingcitation{[jxx-p120430l-2.246]}
\xe

\ea\label{ex:ni-TRANS}
\begingl 
\glpreamble nisimubi\\
\gla ni-simu-bi\\ 
\glb 1\textsc{sg}-find-2\textsc{sg}\\ 
\glft ‘I found you’\\ 
\endgl
\trailingcitation{[mrx-c120509l.043]}
\xe

\tabref{table:InflV-i-t-1-2} shows one intransitive and one transitive verb with first and second person singular and plural subject markers.

\begin{table}
\caption{1st and 2nd person subject marking on an intransitive and a transitive verb}

\begin{tabular}{lllll}
\lsptoprule
 & \textit{-yuiku} & ‘walk’  & \textit{-satÿku} & ‘cut’ \cr
\midrule
1\textsc{sg} & \textit{niyuiku} & ‘I walk’ & \textit{nÿsatÿku} & ‘I cut it’  \cr	
2\textsc{sg} & \textit{piyuiku} & ‘you (\textsc{sg}) walk’ & \textit{pisatÿku} & ‘you (\textsc{sg}) cut it’ \cr
1\textsc{pl}  & \textit{biyuiku} & ‘we walk’ & \textit{bisatÿku} & ‘we cut it’ \cr
2\textsc{pl} & \textit{eyuiku} & ‘you (\textsc{pl}) walk’ & \textit{esatÿku} & ‘you (\textsc{pl}) cut it’\cr
\lspbottomrule
\end{tabular}

\label{table:InflV-i-t-1-2}
\end{table}
\is{subject|)}

First and second person objects\is{object|(} are indexed by markers that resemble the subject markers in form, but have a different position, i.e they follow the stem. \tabref{table:VerbsOBJPerson1_2} shows the markers that are used to index first and second person objects on verbs. The two forms of the object index for the first person singular occur in free variation. The second person singular index is realised as \textit{-bi} (pronounced /vi/) after the default/\isi{realis} form of a marker, and as \textit{-pi} after the \isi{irrealis} form.\footnote{This is reminiscent of the pattern found with the middle marker,\is{middle voice} see \sectref{sec:Middle_voice}.} The markers for first person plural and second person singular are thus identical in many cases.

\begin{table}
\caption{1st and 2nd person object indexes}

\begin{tabular}{lll}
\lsptoprule
& \textsc{sg} & \textsc{pl} \cr
\midrule
1 & \textit{-ne/-nÿ} & \textit{-bi} \cr
2 & \textit{-bi/-pi} & \textit{-e} \cr
\lspbottomrule
\end{tabular}

\label{table:VerbsOBJPerson1_2}
\end{table}


First and second person objects are obligatorily marked on the verb, i.e. the object indexes cannot be replaced by personal pronouns.\is{personal pronoun} What is more, personal pronouns cannot even co-occur (see \sectref{sec:PersPron}). The following examples show the first and second person object markers on different verbs. 

In (\ref{ex:OBJ1SG}), the object is a first person singular. 

\ea\label{ex:OBJ1SG}
\begingl 
\glpreamble kuina pisimuane\\
\gla kuina pi-simua-ne\\ 
\glb \textsc{neg} 2-find.IRR-1\textsc{sg}\\ 
\glft ‘you didn’t meet me’\\ 
\endgl
\trailingcitation{[mrx-c120509l.046]}
\xe

In (\ref{ex:OBJ2SG-2}), the object is a second person singular. Since the index follows the default/realis variant of the stem, the allomorph \textit{-bi} is used.

\ea\label{ex:OBJ2SG-2}
\begingl
\glpreamble nekichubi ÿne\\
\gla nÿ-ekichu-bi ÿne\\
\glb 1\textsc{sg}-invite-2\textsc{sg} water\\
\glft ‘I invite you for some water (i.e. I share my water with you)’
\endgl
\trailingcitation{[jmx-e090727s.163]}
\xe

(\ref{ex:OBJ2SG}) has a second person singular object realised with the voiceless allomorph after irrealis marking.

\ea\label{ex:OBJ2SG}
\begingl 
\glpreamble biyÿseuchapi\\
\gla bi-yÿseucha-pi\\ 
\glb 1\textsc{pl}-greet.\textsc{irr}-2\textsc{sg}\\ 
\glft ‘we want to greet you’\\ 
\endgl
\trailingcitation{[rxx-e121128s-2.007]}
\xe

The object of (\ref{ex:OBJ1PL}) is first person plural.

\ea\label{ex:OBJ1PL}
\begingl 
\glpreamble pimesumeikubi\\
\gla pi-mesumeiku-bi\\ 
\glb 2\textsc{sg}-teach-1\textsc{pl}\\ 
\glft ‘you taught us’\\ 
\endgl
\trailingcitation{[oxx-e120414ls-1a.157]}%el.
\xe

In (\ref{ex:OBJ2PL}), the object is second person plural. 

\ea\label{ex:OBJ2PL}
\begingl 
\glpreamble nichupuikue\\
\gla ni-chupuiku-e\\ 
\glb 1\textsc{sg}-know-2\textsc{pl}\\ 
\glft ‘I met (lit.: know) you’\\ 
\endgl
\trailingcitation{[cux-c120410ls.011]}
\xe
\is{object|)}


\subsection{Third person}\label{sec:3Marking}
\is{object|(}
\is{subject|(}

There are two different third person indexes on verbs, \textit{ti-} and \textit{chÿ-} (also realised as \textit{chi-)}. Only the latter also occurs on nouns, where it indexes a \isi{possessor} (see \sectref{sec:Possession}); \textit{ti-} is attached to verbs exclusively. The marker \textit{ti-} only indexes the subject, while \textit{chÿ-} is used to encode two third person arguments, a subject and an object. The gloss chosen for \textit{ti-} is ‘3i’ and for \textit{chÿ-}, it is ‘3’.\footnote{The ‘i’ in ‘3i’ is a vestige of the first assumption in analysis of the two markers that \textit{ti-} would be a marker for intransitive verbs \citep[cf.][506]{Danielsen2011}. This analysis proved to be incorrect and it would certainly be more precise in the context of verbs to gloss \textit{ti-} as ‘3’ and \textit{chÿ-} as ‘3>3’, but the disadvantage is that \textit{chÿ-} – and only \textit{chÿ-} – is also used on nouns to mark a third person possessor. In addition, it occasionally occurs as \textit{-chÿ}, thus following the stem. In these cases, the gloss ‘3>3’ makes no sense. I prefer having one gloss for one marker over preciseness of glosses in the specific context of person marking on verbs. For that reason, I stick to ‘3i’ for the marker \textit{ti-} and ‘3’ for the marker \textit{chÿ-}.} Both indexes do not distinguish \isi{gender} and are underspecified for number, but there are other means to express plurality of an argument (see \sectref{sec:Verbs_3PL}). Third person markers follow a stem only marginally, see \sectref{sec:3_suffixes}.

Let us first have a look at \textit{ti-}.\is{animacy|(} This marker is obligatory with \isi{intransitive} verbs and with transitive verbs with SAP objects, and it can also be used with many \isi{transitive} verbs that have non-human third person objects, see (\ref{ex:ti-intrans})--(\ref{ex:tiOBJ-nhum}).

(\ref{ex:ti-intrans}) has a stative intransitive verb. It comes from Juana speaking about her son-in-law. That fact that the subject is a male person is thus deduced from the context; this information is not given by the third person marker.

\ea\label{ex:ti-intrans}
\begingl 
\glpreamble tikutiu\\
\gla ti-kutiu\\ 
\glb 3i-be.ill\\ 
\glft ‘he is ill’\\ 
\endgl
\trailingcitation{[jxx-p110923l-1.043]}
\xe

In (\ref{ex:ti-intrans-2}) from the same speaker, there is a middle verb, which is also intransitive, and \textit{ti-} clearly relates to the conominated female person.

\ea\label{ex:ti-intrans-2}
\begingl
\glpreamble i titupunubu nijinepÿi\\
\gla i ti-tupunubu ni-jinepÿi\\
\glb and 3i-arrive 1\textsc{sg}-daughter\\
\glft ‘and my daughter arrived’
\endgl
\trailingcitation{[jxx-p120430l-1.267]}
\xe

(\ref{ex:tiOBJ1SG}) has a transitive verb with a first person singular object and was elicited from María S. The imagined third person subject was a dog in this case, but it was not conominated.

\ea\label{ex:tiOBJ1SG}
\begingl 
\glpreamble tinijabakunÿ\\
\gla ti-nijabaku-nÿ\\ 
\glb 3i-bite-1\textsc{sg}\\ 
\glft ‘it bites me’\\ 
\endgl
\trailingcitation{[rxx-e181018le]}
\xe

In (\ref{ex:tiOBJ1PL}), there is a first person plural object. It comes from María S. who speaks about smoking.

\ea\label{ex:tiOBJ1PL}
\begingl
\glpreamble tikupakabi\\
\gla ti-kupaka-bi\\
\glb 3i-kill.\textsc{irr}-1\textsc{pl}\\
\glft ‘it can kill us’
\endgl
\trailingcitation{[rxx-e120511l.385]}
\xe

If there is a non-human third person object as in (\ref{ex:tiOBJ-inanim}), \textit{ti-} may also be used. This sentence was elicited from Juana and it is again the context which determines that a male person is meant.

\ea\label{ex:tiOBJ-inanim}
\begingl 
\glpreamble tikeburiku amuke\\
\gla ti-keburiku amuke\\ 
\glb 3i-remove.grain corn\\ 
\glft ‘he removed grains of the corn cobs’\\ 
\endgl
\trailingcitation{[jxx-e110923l-1.050]}
\xe

In (\ref{ex:tiOBJ-nhum}) from Juana, there is an animate but non-human object, thus \textit{ti-} can be used.

\ea\label{ex:tiOBJ-nhum}
\begingl
\glpreamble tikupaiku baka\\
\gla ti-kupaiku baka\\
\glb 3i-slaughter cow\\
\glft ‘she slaughtered a cow’
\endgl
\trailingcitation{[jxx-p120515l-2.097]}
\xe
 
The other third person marker, \textit{chÿ-}, overtly expresses that a third person subject acts on another third person object. It is thus neither found on \isi{intransitive} verbs nor on \isi{transitive} ones with SAP objects.\footnote{An exception to this is \isi{ditransitive} verbs derived with the \isi{benefactive} suffix, see \sectref{sec:Benefactive}. Benefactive verbs are rare and will thus not be considered further in this discussion.} Just like \textit{ti-}, the marker does not specify gender or number.\footnote{In this regard, the Paunaka system of person marking is different from the one found in the \isi{Mojeño languages}, which have one unspecified third person prefix that is cognate to \textit{ti-} and occurs in largely the same contexts, but contrast this unspecified prefix to a whole set of 3>3 person markers that distinguish gender and number \citep[cf.][]{Rose2011a}.} 

Use of \textit{chÿ-} is obligatory if the object is human. This can be seen in the following three examples, which all have human objects. In (\ref{ex:3on3-1}), a conominal subject NP follows the verb, (\ref{ex:3on3-2}) has an object NP, and in (\ref{ex:3on3-3}) there is neither a subject nor an object NP. 

(\ref{ex:3on3-1}) comes from Juana who talks about her daughter who lives in Spain.

\ea\label{ex:3on3-1}
\begingl 
\glpreamble chumu chima\\
\gla chÿ-umu chi-ima\\ 
\glb 3-take 3-husband\\ 
\glft ‘her husband took her (with him)’\\ 
\endgl
\trailingcitation{[jxx-p110923l-1.240]}
\xe

%chinisapikutu eka jane 'los petos le picaron’ jxx-a120516l-a.109
%chimumukuji chipeu kabe 'his dog is looking at it‘ jxx-a120516l-a.026

(\ref{ex:3on3-2}) is from the story about the lazy man as told by Miguel. The main character has cut off his limbs at this point of the story and is put into a basket by his son to be carried away.

\ea\label{ex:3on3-2}
\begingl 
\glpreamble chipurtuku echÿu chÿa\\
\gla chi-purtuku echÿu chÿ-a\\ 
\glb 3-put.in \textsc{dem} 3-father\\ 
\glft ‘he put his father into it (i.e. the basket)’\\ 
\endgl
\trailingcitation{[mox-n110920l.120]}
\xe

(\ref{ex:3on3-3}) comes from Juana telling the creation story. It is God who puts María Eva outside, into the world.

\ea\label{ex:3on3-3}
\begingl 
\glpreamble chetuku nauku nekupai\\
\gla chÿ-etuku nauku nekupai\\ 
\glb 3-put there outside\\ 
\glft ‘he put her there outside’\\ 
\endgl
\trailingcitation{[jxx-n101013s-1.359]}
\xe

(\ref{ex:INAN-ANIM}) shows one of the rare cases in which an inanimate subject acts on an animate object. Since the object is a human third person, \textit{chÿ-} is used. The sentence comes from Juana who was telling me how her grandparents tried to cross an arroyo on their journey back home from Moxos, but had to climb up the slope again and hold on to a tree there:
 
 \ea\label{ex:INAN-ANIM}
\begingl 
\glpreamble nebuji eka tujubeiku kuinabu chijatÿkupunanubetu nauku\\
\gla nebu-ji eka tujubeiku kuina-bu chi-jatÿkupuna-nube-tu nauku\\ 
\glb 3\textsc{obl.top.prn}-\textsc{rprt} \textsc{dem}a wind \textsc{neg}-\textsc{dsc} 3-pull.back-\textsc{pl}-\textsc{iam} there\\ 
\glft ‘so that the wind could not pull them back there anymore, it is said’\\ 
\endgl
\trailingcitation{[jxx-p151016l-2.120-121]}
\xe
\is{subject|)}

If the object is non-human, both \textit{ti-} ‘3i’ and \textit{chÿ-} ‘3’ can be used. (\ref{ex:tiOBJ-inanim}) and (\ref{ex:tiOBJ-nhum}) already showed the use of \textit{ti-} with non-human objects. (\ref{ex:3dog-eat}) has is a non-human object and \textit{chÿ-} is used. Juana speaks about a bird of prey here that once stole her dog.

\ea\label{ex:3dog-eat}
\begingl
\glpreamble chikuye chumu chiniku kabe\\
\gla chi-kuye chÿ-umu chi-niku kabe\\
\glb 3-be.like.this 3-take 3-eat dog\\
\glft ‘it was like this, it took and ate the dog’
\endgl
\trailingcitation{[jxx-a120516l-a.199]}
\xe\is{animacy|)}

\citet[480]{Rose2011a} found out that in Trinitario,\is{Mojeño Trinitario} the choice between an unspecified and a 3>3-marking third person marker depends on definiteness,\is{definiteness|(} i.e. the unspecified marker is only used with indefinite objects. Considering (\ref{ex:tiOBJ-inanim}), (\ref{ex:tiOBJ-nhum}) and (\ref{ex:3dog-eat}), Paunaka may have a tendency towards a similar use. Since Paunaka does not have articles, I could not find out in each and every case whether an NP was definite or indefinite in the perception of the speaker. What is for sure is that \textit{ti-} can also be used if a demonstrative is used together with a noun. This is the case in (\ref{ex:ti-DEF-3}) and (\ref{ex:ti-DEF-2}).

In (\ref{ex:ti-DEF-3}), definiteness of the object is signalled by the use of the demonstrative \textit{eka} in the NP and by the fact that the noun is possessed. Nonetheless, \textit{ti-} is used on the verb. The sentence comes from Miguel telling a story about ants that are happy about the travel supplies of a young man.

\ea\label{ex:ti-DEF-3}
\begingl
\glpreamble tumu eka chitapikine\\
\gla ti-umu eka chi-tapiki-ne\\
\glb 3i-take \textsc{dem}a 3-travel.supplies-\textsc{possd}\\
\glft ‘he takes his travel supplies’
\endgl
\trailingcitation{[mxx-n120423lsf-X.22]}
\xe

%\ea\label{ex:ti-DEF-1}
%\begingl 
%\glpreamble i tumeikuji eka kesu\\
%\gla i t-umeiku-ji eka kesu \\ 
%\glb and 3i-steal-\textsc{rprt} \textsc{dem}a cheese\\ 
%\glft ‘and he had stolen the cheese, it is said’\\ 
%\endgl
%\trailingcitation{[jmx-n120429ls-x5.237]}
%\xe

(\ref{ex:ti-DEF-2}) is another example of a definite object in combination with the \textit{ti-} marker, but in this case, the other demonstrative, \textit{echÿu}, is used. I have stated in \sectref{sec:DemPron} (Footnote \ref{fn:useDEMb}) that the use of the demonstrative \textit{echÿu} is possibly extended to indefinite NPs; however, in (\ref{ex:ti-DEF-2}), the NP is definitely definite. There is only one reservoir in Santa Rita, and Juana knows that I know that. In addition, she had already mentioned the reservoir shortly before.

\ea\label{ex:ti-DEF-2}
\begingl 
\glpreamble tanautu echÿu tajau\\
\gla ti-anau-tu echÿu atajau\\ 
\glb 3i-make-\textsc{iam} \textsc{dem}b reservoir\\ 
\glft ‘now she made the reservoir’\\ 
\endgl
\trailingcitation{[jxx-p120515l-2.083]}
\xe

It is harder to find examples in which \textit{chÿ-} is combined with an indefinite object. (\ref{ex:chÿ-indef}) might be one case: the stone that Juana is talking about has not been mentioned before, and thus it is probably indefinite (though specific). Nonetheless, since the stone plays an important role in the story, she knows the story well and her brother Miguel, who was co-telling the story, also knows it well, it is also possible that the NP is indeed definite, referring to ‘the stone (that we all know)’. The stone in question is tied to the jaguar’s paws here, and he is subsequently thrown into water, where he drowns.

\ea\label{ex:chÿ-indef}
\begingl
\glpreamble i chetuku mai\\
\gla i chÿ-etuku mai\\
\glb and 3-put stone\\
\glft ‘and he put a stone (in the bonds)’
\endgl
\trailingcitation{[jmx-n120429ls-x5.259]}
\xe

Definiteness as the (only) deciding factor can thus be excluded.\is{definiteness|)} 

In the case of the verb \textit{-piku} ‘be afraid, fear’, the marker \textit{chÿ-} is used to signal a \isi{transitive} and \textit{ti-} an \isi{intransitive} reading regardless of \isi{animacy} and \isi{definiteness} of possible objects.\footnote{Ambitransitivity\is{ambitransitivity} is widespread in Paunaka, but normally a transitive reading does not necessarily entail use of \textit{chÿ-}. I cannot think of any other verb in which the distinction intransitive – transitive is signalled solely and consistently by different third person indexes. The verb \textit{-piku} in its transitive sense ‘fear’ is of course special, since the subject has little control over the event expressed by the verb.\is{transitivity} It is thus rather atypical in transitivity \citep[cf.][252]{HopperThompson1980} so that we could expect a less transitive expression here, i.e. the index \textit{ti-}, which does not imply the presence of an object. However, it might also be the other way round: since the verb is atypical in transitivity, there may be a need to enforce the \isi{transitive} reading by using \textit{chÿ-}.} Compare (\ref{ex:be.afraid}) with the intransitive verb with (\ref{ex:fear}) and (\ref{ex:fear-2}) with transitive verbs.

(\ref{ex:be.afraid}) comes from María S. who told me the story about the two hunters who meet the devil in the woods. One of them interacts with the devil and is taken away to be eaten in the end, while the other hides away in a tree and stays there until the devil leaves the scene. He does not try to save his friend:

\ea\label{ex:be.afraid}
\begingl 
\glpreamble janeka tiyuna, tipiku\\
\gla janeka ti-yuna ti-piku\\ 
\glb never 3i-go 3i-be.afraid\\ 
\glft ‘he never went, he was afraid’\\ 
\endgl
\trailingcitation{[rxx-n120511l-2.61]}
\xe

(\ref{ex:fear}) and (\ref{ex:fear-2}) both come from Juana describing pictures of the \isi{frog story}. In the first example, it is the boy who feels fear, in the second one the dog. In both cases, there is a concrete thing that they fear.

\ea\label{ex:fear}
\begingl 
\glpreamble chipiku jane\\
\gla chi-piku jane\\ 
\glb 3-be.afraid wasp\\ 
\glft ‘he fears the wasps’\\ 
\endgl
\trailingcitation{[jxx-a120516l-a.125]}
\xe

\ea\label{ex:fear-2}
\begingl
\glpreamble chipiku ÿne\\
\gla chi-piku ÿne\\
\glb 3-be.afraid water\\
\glft ‘it is afraid of the water’
\endgl
\trailingcitation{[jxx-a120516l-a.376]}
\xe

Some \isi{transitive} verbs are preferably used with \textit{chÿ-}, among them \textit{-umu} ‘take’, \textit{-imu} ‘see’, and \textit{-tupu} ‘find, meet’. Others are rather indexed with \textit{ti-} like \textit{-anau} ‘make’ and \textit{-yÿseiku} ‘buy’. This is certainly connected to frequency effects: ‘make’ and ‘buy’ hardly ever take human\is{animacy} objects, while ‘take’, ‘see’, and ‘find, meet’ can readily have human objects and are thus often realised with \textit{chÿ-} for that reason. By extension, \textit{chÿ-} is then also used with non-human objects more frequently. This is not absolute, though, since speakers may still opt for the the other marker as in  (\ref{ex:ti-DEF-3}) above where \textit{-umu} is combined with \textit{ti-} or  (\ref{ex:chi-house-irr}) below, in which \textit{-anau} is combined with \textit{chÿ-} .

There are not only differences among verbs, but also differences between the speakers: Juana generally produces more forms with \textit{chÿ-} than Miguel. But even if she often uses \textit{chÿ-}, she may also opt for \textit{ti-}. When she once told me a story about her grandparents’ journey to Moxos to buy some cows, she used \textit{ti-} on most of the verbs that describe an action of her grandparents upon the cows, as in (\ref{ex:ti-vaca}). However, prompted by questions about some words that I did not understand, she re-told the story some days later, adding the description of how her grandparents met an evil spirit on their way back home, and this time she used the marker \textit{chÿ-} more often for describing actions of her grandparents upon their cows, as in (\ref{ex:chi-vaca}). The cows can be considered definite\is{definiteness} in both examples.

\ea\label{ex:ti-vaca}
\begingl 
\glpreamble i enteraukena kuje ke te kapupununubetu te tupunanubetu baka\\
\gla i enterau-kena kuje ke te kapupunu-nube-tu te ti-upuna-nube-tu baka\\ 
\glb and whole-\textsc{uncert} moon that \textsc{seq} come.back-\textsc{pl}-\textsc{iam} \textsc{seq} 3i-bring.\textsc{irr}-\textsc{pl}-\textsc{iam} cow\\ 
\glft ‘and it took them a whole month to come back and bring the cows’\\ 
\endgl
\trailingcitation{[jxx-e150925l-1.206]}
\xe

\ea\label{ex:chi-vaca}
\begingl
\glpreamble kapupunutu niuma chupunu chipeu baka\\
\gla kapupunu-tu ni-uma chÿ-upunu chi-peu baka\\
\glb come.back-\textsc{iam} 1\textsc{sg}-grandfather 3-bring 3-animal cow\\
\glft ‘my grandfather came back and brought his cows along’
\endgl
\trailingcitation{[jxx-p151016l-2.259]}
\xe

An important difference between the two examples\is{reality status|(} is that the verb in (\ref{ex:ti-vaca}) is realis, while in (\ref{ex:chi-vaca}), it is irrealis.\footnote{Thanks to Françoise Rose (2021, p.c.) for pointing this out to me.} Irrealis is among the factors that signal low \isi{transitivity} \citep[252]{HopperThompson1980}. However, it is not the case that every irrealis verb with a non-human\is{animacy} object would take \textit{ti-}. Consider \ref{ex:chi-house-irr}, which comes from the same speaker and has an irrealis verb that takes the index \textit{chÿ-} (and is also one of the relatively few examples of the verb \textit{-anau} ‘make’ taking this marker). Juana speaks about her daughter in Spain, who still owns a plot in Bolivia that she could use for building a house.

\ea\label{ex:chi-house-irr}
\begingl
\glpreamble kapunuina chana chubiuna\\
\gla kapunu-ina chÿ-ana chÿ-ubiu-ina\\
\glb come-\textsc{irr.nv} 3-make.\textsc{irr} 3-house-\textsc{irr.nv}\\
\glft ‘if she comes, she can make her future house’
\endgl
\trailingcitation{[jxx-p120430l-1.298-299]}
\xe

%kuina chetukanube eka yÿtiÿukumÿnÿ eka yÿtÿuku chitÿpijane, jrx-c151001lsf-11.063
%chiratanÿkanube te tiyÿtikapunube, jxx-p151016l-2.058

A few more examples in the corpus combine an irrealis verb with \textit{chÿ-}: consider (\ref{ex:Real-Irr-3}), which was produced by María S. in elicitation and has two clauses that give information about sown and planted crops using the same verb. Strikingly, in the first clause the realis verb is indexed with \textit{ti-}, and in the second clause the negated irrealis verb carries the marker \textit{chÿ-}.

\ea\label{ex:Real-Irr-3}
\begingl
\glpreamble jaja ja amuke tebuku, pero kÿjÿpi kuina chebuka\\
\gla jaja ja amuke ti-ebuku pero kÿjÿpi kuina chÿ-ebuka\\
\glb \textsc{afm} \textsc{afm} corn 3i-sow but manioc \textsc{neg} 3-sow.\textsc{irr}\\
\glft ‘yes, yes, he sowed corn, but he did not plant manioc’
\endgl
\trailingcitation{[rxx-e181024l]}%semi-el.
\xe

Thus irrealis as the only decisive factor for choosing one or the other marker can definitely be ruled out.\is{reality status|)} I would rather suggest that several factors are involved, and more profound knowledge about discourse structure is needed to explain each and every choice.

How different factors trigger the choice of one or the other of the two third person indexes is summarised in \figref{fig:3Person}.

\begin{figure}
%     \includegraphics[width=\textwidth]{figures/3MarkingGraph-3.jpg}
%%Mit rechteckigen Rahmen, auskommentiert
%     \begin{forest}
%       [predicate,draw,rectangle,inner sep=2pt,name=predicate
%         [non-verbal,draw,rectangle,inner sep=2pt
%             [$\emptyset$,edge={->}]
%         ]
%         [verbal,draw,rectangle,inner sep=2pt,name=verbal
%             [intransitive,draw,rectangle,inner sep=2pt
%                 [ti-,edge={->}]
%             ]
%             [transitive,draw,rectangle,inner sep=2pt,name=transitive
%                 [OBJ~SAP,draw,rectangle,inner sep=2pt
%                     [ti-,edge={->}]
%                 ]
%                 [OBJ~3rd person,draw,rectangle,inner sep=2pt,name=OBJ3rd
%                     [OBJ non-human,draw,rectangle,inner sep=2pt
%                         [ti-,edge={->}]
%                         [chÿ-,edge={->}]
%                     ]
%                     [OBJ~hum,draw,rectangle,inner sep=1pt
%                         [chÿ-,edge={->}]
%                     ]
%                 ]
%             ]
%         ]
%       ]
%        \node [inner sep=1pt,minimum width=30mm,anchor=south east,fill=gray!20!white,draw,ellipse,yshift=-3mm,right=2mm of predicate]{\small part of speech\strut};
%        \node [inner sep=1pt,minimum width=30mm,anchor=south east,fill=gray!20!white,draw,ellipse,yshift=-3mm,right=2mm of verbal]{\small valency\strut};
%        \node [inner sep=1pt,minimum width=30mm,anchor=south east,fill=gray!20!white,draw,ellipse,yshift=-3mm,right=2mm of transitive]{\small object\strut};
%        \node [inner sep=1pt,minimum width=30mm,anchor=south east,fill=gray!20!white,draw,ellipse,yshift=-3mm,right=2mm of OBJ3rd]{\small humanness\strut};
%     \end{forest}

        \begin{forest}
      [predicate,inner sep=2pt,name=predicate
        [non-verbal,inner sep=2pt
            [$\emptyset$,edge={-latex}]
        ]
        [verbal,inner sep=2pt,name=verbal
            [intransitive,inner sep=2pt
                [ti-,edge={-latex}]
            ]
            [transitive,inner sep=2pt,name=transitive
                [OBJ~SAP,inner sep=2pt
                    [ti-,edge={-latex}]
                ]
                [OBJ~3rd person,inner sep=2pt,name=OBJ3rd
                    [OBJ non-human,inner sep=2pt
                        [ti-,edge={-latex}]
                        [chÿ-,edge={-latex}]
                    ]
                    [OBJ~hum,inner sep=1pt
                        [chÿ-,edge={-latex}]
                    ]
                ]
            ]
        ]
      ]
       \node [inner sep=1pt,minimum width=30mm,anchor=south east,fill=gray!20!white,draw,ellipse,yshift=-3mm,right=2mm of predicate]{\scriptsize part of speech\strut};
       \node [inner sep=1pt,minimum width=30mm,anchor=south east,fill=gray!20!white,draw,ellipse,yshift=-3mm,right=2mm of verbal]{\scriptsize valency\strut};
       \node [inner sep=1pt,minimum width=30mm,anchor=south east,fill=gray!20!white,draw,ellipse,yshift=-3mm,right=2mm of transitive]{\scriptsize object\strut};
       \node [inner sep=1pt,minimum width=30mm,anchor=south east,fill=gray!20!white,draw,ellipse,yshift=-3mm,right=2mm of OBJ3rd]{\scriptsize humanness\strut};
    \end{forest}

    \caption{Third person marking}
\label{fig:3Person}
\end{figure}
\is{object|)}

\subsection{Third person plural}\label{sec:Verbs_3PL}

Both third person markers are underspecified for number, i.e. both singular and \isi{plural} subjects take \textit{ti-} and \textit{chÿ-}. However, the plural marker, the \isi{distributive} marker, and the \isi{collective} marker can be attached to the verb to indicate that one of the participants is non-singular. All of them also occur on nouns (see \sectref{sec:NumberNouns}). It is not uncommon among \isi{Arawakan languages} to have a third person marker that is not specified for number and a separate marker expressing plural. In Danielsen’s (\citeyear[]{Danielsen2014}) sample, 15 out of 38 languages have “separate gender/person and plural marking”, while 17 do not show this distinction, and for six languages, the question remains unclear.


 \subsubsection{Third person plural subject marking with \textit{-nube}}\is{plural|(}\is{subject|(} If the third person plural subject is human,\is{animacy} the marker \textit{-nube} is added to the verb obligatorily. The following examples show third person plural subjects on different kinds of verbs with different \isi{valency} and different types of objects. In (\ref{ex:ti-nube-INTR}), there is an intransitive verb and thus \textit{ti-} is used. It comes from an utterance by Miguel.
 
 
\ea\label{ex:ti-nube-INTR}
\begingl 
\glpreamble tikujemunubetu\\
\gla ti-kujemu-nube-tu\\ 
\glb 3i-be.angry-\textsc{pl}-\textsc{iam}\\ 
\glft ‘they were already angry’\\ 
\endgl
\trailingcitation{[jmx-c120429ls-x5.225]}
\xe

The verb in (\ref{ex:ti-bi-nube}) has a third person plural subject acting on an SAP object. The marker indexing the SAP object always precedes the plural marker. The example comes from the recordings by Riester with Juan Ch. The \textit{patrones} are the ones who might kill their workers if they tried to escape.

\ea\label{ex:ti-bi-nube}
\begingl 
\glpreamble tikupakabinube\\
\gla ti-kupaka-bi-nube\\ 
\glb 3i-kill.\textsc{irr}-1\textsc{pl}-\textsc{pl}\\ 
\glft ‘they would kill us’\\ 
\endgl
\trailingcitation{[nxx-p630101g-1.182]}
\xe

In (\ref{ex:chi-nube}), there is a third person plural subject. It is not a human subject in this case, but an anthropomorphic one:\is{animacy} fox and jaguarundi are acting, the two main characters of a story. They see a pot with strong chicha, the third person singular inanimate object\is{animacy} of this clause (and they decide to get drunk). In this example, Miguel chose the index \textit{chÿ-} to encode the 3>3 relationship. An example with a human\is{animacy} third person plural subject acting on a non-human third person object being encoded by the \textit{ti-}marker has already been given in (\ref{ex:ti-vaca}) above.

\ea\label{ex:chi-nube}
\begingl 
\glpreamble pero chimukunubeji echÿu barerekiji tijapÿkubu isipau\\
\gla pero chi-imu-uku-nube-ji echÿu barereki-ji ti-japÿku-bu isipau\\ 
\glb but 3-see-\textsc{add}-\textsc{pl}-\textsc{rprt} \textsc{dem}b pot-\textsc{rprt} 3i-fill-\textsc{mid} strong.chicha\\ 
\glft ‘but they also saw a pot filled with strong chicha’\\ 
\endgl
\trailingcitation{[jmx-n120429ls-x5.328]}
\xe
\is{subject|)}


 \subsubsection{Third person plural object marking with \textit{-nube}}\is{object|(} Not only human\is{animacy} third person plural subjects, but also human third person plural objects can be encoded by \textit{-nube}. It is thus not always clear which of the arguments is the plural participant. In this respect, Paunaka differs from the \isi{Mojeño languages}, where the plural marker can only relate to the subject \citep[474--475]{Rose2011a}.\footnote{Note, however, that \citet[382-384, Footnote 1]{Facundes2000} reports the same kind of potential ambiguity, which he found in some varieties of Apurinã, an Arawakan language\is{Arawakan languages} more distantly related to Paunaka.} 
 
 If the subject is an SAP, the only participant to which \textit{-nube} can refer is obviously the object, as in (\ref{ex:ni-nube}), which is taken from an utterance by Juana.
  
 \ea\label{ex:ni-nube}
\begingl 
\glpreamble nichupuikunube\\
\gla ni-chupuiku-nube\\ 
\glb 1\textsc{sg}-know-\textsc{pl}\\ 
\glft ‘I met them’\\ 
\endgl
\trailingcitation{[jxx-p120515l-1.218]}
\xe

However, if both subject and object have third person referents, ambiguity arises.\footnote{Remember that there is no case-marking on nouns, so that even if a singular or plural NP conominates the person indexes, it could theoretically refer to the subject or the object alike.} Only context and general knowledge can clarify which one is the plural argument.

In the following example, Juana describes a photo to my colleague Swintha on which my husband is carrying both my daughters in his arms. It is therefore clear that a singular subject acts on a plural object.

\ea\label{ex:3sg-3pl-1}
\begingl 
\glpreamble chakachunube chima\\
\gla chÿ-akachu-nube chi-ima\\ 
\glb 3-lift-\textsc{pl} 3-husband\\ 
\glft ‘her husband lifts them’\\ 
\endgl
\trailingcitation{[jxx-p141024s-1.31]}
\xe

The next example from Miguel is sufficiently clear because of world knowledge: it is the \textit{patrón}, the liege lord, who was supposed to pay his many workers; any other interpretation would be odd.

\ea\label{ex:3sg-3pl-2}
\begingl 
\glpreamble kuina chisiupuchanube eka patron\\
\gla kuina chi-siupucha-nube eka patron\\ 
\glb \textsc{neg} 3-pay.\textsc{irr}-\textsc{pl} \textsc{dem}a patrón\\ 
\glft ‘the \textit{patrón} didn’t pay them’\\ 
\endgl
\trailingcitation{[mxx-p110825l.042]}
\xe

On the other hand, in (\ref{ex:SubjisPL}), the subject is plural. Juana had just narrated that her brother felt dizzy and fell. It was thus only a single person who was lifted and taken away and the plural marker must thus refer to the subject. % Note to self: This seems to be one of the cases in which the plural marker is used to encode an unspecific indefinite subject just like in Spanish.

\ea\label{ex:SubjisPL}
\begingl
\glpreamble chakachunube chumunubeji nauku\\
\gla chÿ-akachu-nube chÿ-umu-nube-ji nauku\\
\glb 3-lift-\textsc{pl} 3-take-\textsc{pl}-\textsc{rprt} there\\
\glft ‘they lifted him and took him there, it is said’
\endgl
\trailingcitation{[jxx-p120430l-2.444]}
\xe

It is also possible that both subject and object are plural as is the case in (\ref{ex:pl-pl}), which has two conominal NPs that encode subject and object. The first NP only consists of a modifier, but this is nothing special in Paunaka.\footnote{There are no discontinuous NPs and there is usually no agreement in number inside an NP, thus it is clear that we are indeed dealing with two separate NPs here, see \sectref{sec:NP}. The sentence exhibits SVO order, which is common if two arguments are conominated (see \sectref{sec:WordOrder}).}  Both NPs are marked for plural. It is not important in this case to which of the arguments the plural marker belongs; it could be any of the two. There is no double plural marking on verbs.\footnote{But see below for possible co-occurrence of plural and distributive marker.}

The sentence comes from Juana who told me what her daughter in Spain had said to her. This daughter wanted to convince Juana to visit her.

\ea\label{ex:pl-pl}
\begingl 
\glpreamble “pujanenube chumunube chÿenujinube”\\
\gla pu-jane-nube chÿ-umu-nube chÿ-enu-ji-nube\\ 
\glb other-\textsc{distr}-\textsc{pl} 3-take-\textsc{pl} 3-mother-\textsc{col}-\textsc{pl}\\ 
\glft ‘“others take their mothers (to Spain for a visit)”’\\ 
\endgl
\trailingcitation{[jxx-e120516l-1.030]}
\xe
\is{object|)}
\is{plural|)}


\subsubsection{Third person plural subject marking with \textit{-jane}}\is{distributive|(}\is{subject|(} If the third person plural subject is non-human,\is{animacy} the marker \textit{-jane} can be used optionally to establish plural reference. (\ref{ex:ti-jane-1}) shows this with three verbs. It was produced by Juana and is about some ducklings that we were watching.
 
 \ea\label{ex:ti-jane-1}
\begingl 
\glpreamble tiyunujane kosinayae tinikupajane teajane ÿne\\
\gla ti-yunu-jane kosina-yae ti-niku-pa-jane ti-ea-jane ÿne\\ 
\glb 3i-go-\textsc{distr} kitchen-\textsc{loc} 3i-eat-\textsc{dloc.irr}-\textsc{distr} 3i-drink.\textsc{irr}-\textsc{distr} water\\ 
\glft ‘they go into the kitchen to eat and drink water’\\ 
\endgl
\trailingcitation{[jxx-e150925l-1.116]}
\xe

The distributive marker \textit{-jane} can occur in two different morphological slots of the verb. First, it can attach to the complete verb stem including the thematic suffix \is{thematic suffix|(} (see \sectref{sec:ActiveVerbs_TH}), and the other possibility is to replace the thematic suffix by \textit{-jane}. Obviously, this possibility only exists for verbs that have a thematic suffix.\footnote{The verb \textit{-yunu} ‘go’ might be exceptional in this regard because the last syllable of the stem (\textit{nu}) can be deleted instead. However, there is only one example of this in the corpus, which comes from the recordings made by Riester in the 1960s.} If the distributive marker deletes the thematic suffix, the reality status suffix follows the distributive marker (see (\ref{ex:jane-eat-2}), (\ref{ex:ti-jane-2}) and also \sectref{sec:VerbalRS}).

In order to compare both forms directly, consider the following two examples with the verb \textit{-niku} ‘eat’. In (\ref{ex:jane-eat-1}), the full verb stem \textit{-niku} including the thematic suffix \textit{-ku} is used and folowed by \textit{-jane}. The example is from the same context as (\ref{ex:ti-jane-1}) above; the ducklings were approaching us, coming back from eating grass somewhere. (\ref{ex:jane-eat-2}) shows the use of the distributive marker in the slot that is usually occupied by the thematic suffix. Thus only the root\is{verbal root} \textit{-ni} shows up here. The example comes from elicitation with María S. which aimed at finding out whether there is a difference between the verbs for ‘eat’ and ‘feed’.\footnote{There is no difference: the verb \textit{-niku} has both meanings in present-day Paunaka.}

\ea\label{ex:jane-eat-1}
\begingl
\glpreamble tinikujane mÿiji\\
\gla ti-niku-jane mÿiji\\
\glb 3i-eat-\textsc{distr} grass\\
\glft ‘they ate grass’\\ 
\endgl
\trailingcitation{[jxx-e150925l-1.115]}
\xe

\ea\label{ex:jane-eat-2}
\begingl
\glpreamble tinijaneutu\\
\gla ti-ni-jane-u-tu\\
\glb 3i-eat-\textsc{distr}-\textsc{real}-\textsc{iam}\\
\glft ‘they already ate’
\endgl
\trailingcitation{[rxx-e141230s.038]}
\xe

It is tempting to analyse the latter form as the older, more conservative one and the one in which the distributive follows the thematic suffix as an innovation which regulates the pattern by making distributive marking more reminiscent of \isi{plural} marking. Nonetheless, since we do not have any old texts to compare with, this remains speculative.

In any case, María S. prefers the form without thematic suffix, while the other speakers prefer the other one.\footnote{María S. in general uses more forms with deleted thematic suffixes. This is also true for the associated motion markers, see \sectref{sec:AssociatedMotion}.} When asked, all speakers accept both forms. There is no difference in meaning according to them, and I could not find any difference either. 

Another example with the distributive marker produced by María S. is given in (\ref{ex:ti-jane-2}): the verb stem\is{verbal stem} is \textit{-yuiku} ‘walk’, but the thematic suffix \textit{-ku} is deleted, and \textit{-jane} is inserted in the slot. It comes from elicitation and is about some recently hatched chicks.

\ea\label{ex:ti-jane-2}
\begingl 
\glpreamble tiyuijaneatu\\
\gla ti-yui-jane-a-tu\\ 
\glb 3i-walk-\textsc{distr}-\textsc{irr}-\textsc{iam}\\ 
\glft ‘they are about to walk’\\ 
\endgl
\trailingcitation{[rxx-e121128s-1.035]}
\xe


If the object is an SAP, the object marker follows the distributive marker. Only the form of the verb in which \textit{-jane} deletes the thematic suffix has been found in the corpus.\is{thematic suffix|)} Note that there is not necessarily agreement in number between the conominal\is{conomination} subject and the verb. Usually, \textit{-jane} is only attached to either the verb or the noun, but more frequently found on the verb only (see also \sectref{sec:NounPL-jane}). 


In (\ref{ex:jane-SAPOBJ}), the distributive marker is only found on the verb. It is combined with a person marker indexing the object in this case. This sentence was elicited from María S.

\ea\label{ex:jane-SAPOBJ}
\begingl
\glpreamble tinisapijaneunÿ anibÿ\\
\gla ti-nisapi-jane-u-nÿ anibÿ\\
\glb 3i-sting-\textsc{distr}-\textsc{real}-1\textsc{sg} mosquito\\
\glft ‘the mosquitos stung me’
\endgl
\trailingcitation{[rxx-e181101l-1]}
\xe

A similar example from the same elicitation session is (\ref{ex:jane-SAPOBJ}), but this time, the distributive marker additionally occurs in the NP.

\ea\label{ex:jane-SAPOBJ-2}
\begingl
\glpreamble tichaneijaneunÿ nipeujane kabe\\
\gla ti-chanei-jane-u-nÿ ni-peu-jane kabe\\
\glb 3i-care.for-\textsc{distr}-\textsc{real}-1\textsc{sg} 1\textsc{sg}-animal-\textsc{distr} dog\\
\glft ‘my dogs protect me’
\endgl
\trailingcitation{[rxx-e181101l-1]}
\xe

As has been mentioned above, there are no examples in the corpus with SAP object marking and \textit{-jane} being attached after the \isi{thematic suffix}. The distributive marker is rather not realised on the verb at all in these cases, as in (\ref{ex:no-janeSAP}), which was a spontaneous utterance from Juana.

\ea\label{ex:no-janeSAP}
\begingl
\glpreamble tichaneikune eka kabejane\\
\gla ti-chaneiku-ne eka kabe-jane\\
\glb 3i-care.for-1\textsc{sg} \textsc{dem}a dog-\textsc{distr}\\
\glft ‘the dogs protects me’
\endgl
\trailingcitation{[jxx-e150925l-1.093]}
\xe

(\ref{ex:bite-jane}) is an example of a transitive verb with a non-human animate\is{animacy} plural subject. Since the object is an anthropomorphic character (from the same narrative that (\ref{ex:chi-nube}) above is taken from), the marker \textit{chÿ-} is used. The drunken fox is bitten here, bitten to death by the dogs of the owner of the house where he stole the strong chicha. The dogs are not anthropomorphic but behave like dogs behave.

\ea\label{ex:bite-jane}
\begingl 
\glpreamble chinijababakujanetu kabe\\
\gla chi-nijababaku-jane-tu kabe\\ 
\glb 3-bite-\textsc{distr}-\textsc{iam} dog\\ 
\glft ‘the dogs bit him’\\ 
\endgl
\trailingcitation{[jmx-n120429ls-x5.435]}
\xe
\is{subject|)}

\subsubsection{Third person plural object marking with \textit{-jane}}\is{object|(} The distributive marker can optionally index non-human third person plural objects. This is only found with animate referents.\is{animacy} It expresses a higher degree of individuation in this case. Individuation is context-dependent, i.e. animals may sometimes be perceived as individual beings and on other occasions individuation is not necessary.
 
(\ref{ex:pi-jane}) shows a case in which a second person subject acts on a non-human animate plural object which is marked by \textit{-jane}. It comes from elicitation with María S. Actually, it is the question to the answer that has been given in (\ref{ex:jane-eat-2}) above.
 
 \ea\label{ex:pi-jane}
\begingl 
\glpreamble ¿pinijaneutu?\\
\gla pi-ni-jane-u-tu\\ 
\glb 2\textsc{sg}-feed-\textsc{distr}-\textsc{real}-\textsc{iam}\\ 
\glft ‘have you fed them?’\\ 
\endgl
\trailingcitation{[rxx-e141230s.037]}
\xe

(\ref{ex:see-frogs}) is from Juana, the subject is a third person and she chose \textit{ti-} as a subject marker, which is combined with \textit{-jane} as an object index. The sentence is from her description of the \isi{frog story}.

\ea\label{ex:see-frogs}
\begingl
\glpreamble i echÿu kabe timumuku, timumukujane peÿ\\
\gla i echÿu kabe ti-imumuku ti-imumuku-jane peÿ\\
\glb and \textsc{dem}b dog 3i-look 3i-look-\textsc{distr} frog\\
\glft ‘and the dog is looking, it is looking at the frogs’
\endgl
\trailingcitation{[jxx-a120516l-a.429]}
\xe

In (\ref{ex:chi-jane}), the subject is also a third person. In this case, \textit{chÿ-} is chosen as a marker. The sentence was produced by Juana in elicitation.

\ea\label{ex:chi-jane}
\begingl 
\glpreamble chikupakujane upuji\\
\gla chi-kupaku-jane upuji\\ 
\glb 3-kill-\textsc{distr} duck\\ 
\glft ‘it (the dog) kills ducks’\\ 
\endgl
\trailingcitation{[jxx-e081025s-1.552]}
\xe


If we compare these last examples with (\ref{ex:jane-eat-1}) and (\ref{ex:bite-jane}) above, it is ambiguous which of the participants is non-singular just as in the case of plural marking. Thus in (\ref{ex:bite-jane}) \textit{-jane} refers to the subject, and in (\ref{ex:chi-jane}) it refers to the object. It is again the context or general knowledge that is needed to understand which argument is plural.
 
Both markers \textit{-nube}\is{plural|(} and \textit{-jane} can also co-occur, and in this case, both of them can refer to the subject and to the object. The distributive marker always precedes the plural marker, regardless of who is the subject and who is the object. (\ref{ex:jane-nube}) shows this case. It was elicited from María S.

\ea\label{ex:jane-nube}
\begingl 
\glpreamble cheikukuijaneunube\\
\gla ch-eikukui-jane-u-nube\\ 
\glb 3-chase-\textsc{distr}-\textsc{real}-\textsc{pl}\\ 
\glft ‘they chase them’ (i.e. either the children chase the dogs or the dogs chase the children)\\ 
\endgl
\trailingcitation{[rxx-e141230s.211]}
\xe
 
In real contexts, however, this kind of ambiguity hardly exists. First, verbs which do not favour a semantic interpretation of either the human or the non-human participant\is{animacy|(} being the subject are rare. Second, utterances such as (\ref{ex:jane-nube}) are usually embedded in some context, which helps identifying the subject and the object. And most importantly, I have not encountered a single example of a verb taking both \textit{-jane} and \textit{-nube} outside of elicitation.\is{plural|)}

The principles underlying the choice of person marker and plural/distributive marker if there is a non-singular third person referent are summarised in \figref{fig:3PL}.

\begin{figure}
%
 \begin{itemize}
\item If the subject is a human third person plural, we can find:
 \begin{itemize}
\item \textit{ti-...-nube} on intransitive verbs and transitive verbs with SAP and non-human objects
\item \textit{chÿ-...nube} on transitive verbs with third person objects
\end{itemize}

\item If the subject is non-human third person plural, we can find:
 \begin{itemize}
\item \textit{ti-} or \textit{ti-...-jane} on intransitive verbs and on transitive verbs with SAP or non-human objects
\item \textit{chÿ-} or \textit{chÿ-...-jane} with third person objects
\end{itemize}

\item If the object is a human third person plural, we find:
 \begin{itemize}
\item \textit{-nube} on the verb
\end{itemize}

\item If the object is a non-human third person plural, we can find:
 \begin{itemize}
\item no distributive marking on verb
\item \textit{-jane} on the verb
\end{itemize}

\item Third person plural subject and object markings combine, thus we also find \textit{chÿ-...-nube} for verbs with a non-human third person plural subject and a human plural object, as well as \textit{chÿ-...-jane...-nube}.
\end{itemize}

\caption{Indexes of third person non-singular participants}
\label{fig:3PL}\is{animacy|)}
%
\end{figure}

%chinijabajane
%tinijabaijane
%tujijaneu
%cheikukuijaneu
%tipurtujaneu
%tikubijaijane
\is{object|)}
\is{distributive|)}


\subsubsection{Collective marking on verbs}\label{CollectiveVerbs}\is{collective|(}

In addition to \textit{-nube} ‘\textsc{pl}’ and \textit{-jane} ‘\textsc{distr}’, a third marker can encode non-singularity on verbs, the collective marker \textit{-ji}. It occurs on some stative verbs\is{stative verb}  that encode properties if the subject appears in a mass or swarm. Since all property verbs are \isi{intransitive}, the collective marker always refers to the \isi{subject}. There are not many examples in the corpus and they all refer to fish. Agreement\is{agreement} in collective marking between verb and noun is not obligatory. However, there can be agreement as is the case in (\ref{ex:turukeji}), which was presented in \sectref{sec:Collective}. (\ref{ex:white-sardines}) is an example where there is no agreement in collective marking between verb and conominal subject.

\ea\label{ex:white-sardines}
\begingl 
\glpreamble tisururupeji kÿpu\\
\gla ti-sururu-pe-ji kÿpu\\ 
\glb 3i-be.light.coloured-\textsc{clf}:flat-\textsc{col} sardine\\ 
\glft ‘the sardines are light-coloured’\\ 
\endgl
\trailingcitation{[jxx-e150925l-1.162]}
\xe

Collective markers on verbs never refer to objects\is{object} as far as I can tell. However, it is not easy to distinguish the collective marker from the reportive marker\is{evidentiality} (see \sectref{sec:Evidentiality}), since both have the same form. Although they fill different slots in the verb template, both often just occur on the right edge of the actual verb form. It might thus be possible that I mistakenly took as a reportive marker which was indeed a collective marker in some cases.

There is one active verb on which the collective marker is found apparently. This is the verb \textit{-eu} ‘hit, fight’ together with the \isi{reciprocal} marker \textit{-kuku}, see (\ref{ex:father-fight}), which was already presented in \sectref{sec:RCPC}. There are only very few examples of this construction in the corpus.

\ea\label{ex:father-fight}
\begingl
\glpreamble teukukujinube chajechubu chÿa\\
\gla ti-eu-kuku-ji-nube chÿ-ajechubu chÿ-a\\
\glb 3i-fight-\textsc{rcpc}-\textsc{col}-\textsc{pl} 3-\textsc{com} 3-father\\
\glft ‘with his father, they were fighting with each other’
\endgl
\trailingcitation{[rxx-e141230s.195]}
\xe
\is{collective|)}

\subsection{Third person markers following the stem}\label{sec:3_suffixes}
\is{object|(}

Third person markers that follow the stem only rarely occur on verbs, but there are some exceptions. 
First of all, the \isi{ditransitive} verb \textit{-punaku} ‘give’ may possibly take \textit{-chÿ} as a marker to index a \isi{recipient} or theme.\is{patient/theme} There are two examples in the corpus that point to this. However, those two examples are not totally clear for several reasons (grammatical structure, pronunciation, lexicon) and there are not enough contrastive examples to compare with. I give one of them here nonetheless, but acknowledge that the issue needs additional research.\footnote{There are three different (related) verbs with the meaning ‘give’, \textit{-pu}, \textit{-punaku} and \textit{-puiku}. The first two of them can index a first or second person \isi{recipient} as an object (i.e. by a marker that follows the stem), while the third one does not seem to include a \isi{recipient} argument in its semantic structure. However, this still remains to be verified.} (\ref{ex:ditr-new-3}) is not clear in structure regarding use of irrealis and realis and the function of \textit{chija} as an indefinite pronoun or hesitation marker. In addition, I could also not determine the exact meaning of the verb \textit{-yubunu}. My colleague Lena S. elicited the sentence ‘I will give a present to my mother’ from Juana. The latter needed several attempts to formulate an adequate Paunaka translation, one of them containing the verb \textit{-punaku} ‘give’ with the third person marker \textit{-chÿ}:

\ea\label{ex:ditr-new-3}
\begingl
\glpreamble niyunabÿti nipunakumÿnÿchÿ chija eka niyubuniachÿ\\
\gla ni-yuna-bÿti ni-punaku-mÿnÿ-chÿ chija eka ni-yubun-i-a-chÿ\\
\glb 1\textsc{sg}-go.\textsc{irr}-\textsc{prsp} 1\textsc{sg}-give-\textsc{dim}-3 what \textsc{dem}a 1\textsc{sg}-be.worth?-\textsc{subord}-\textsc{irr}-3\\
\glft ‘I am about to go to give her her present’
\endgl
\trailingcitation{(jxx-e191021e-2)}
\xe

Third person markers that follow the stem most frequently occur on speech verbs,\is{speech verb|(} or more precisely on one specific speech verb, \textit{-kechu} ‘say’. This verb is used in connection with direct speech. The verb with \textit{-chÿ} may precede or follow the quoted speech and both third person indexes, \textit{ti-} and \textit{chÿ-} occur on these verbs, as well as SAP indexes. One example with a speech verb following the direct speech is given below. Juana cites the husband of her sister in this case, who was a criminal and treated his family badly.

\ea\label{ex:SPEECH-1}
\begingl 
\glpreamble “¡patÿkemiu!”, chikechuchiji chichechapÿi\\
\gla pi-a-tÿkemiu chi-kechu-chi-ji chi-chechapÿi\\ 
\glb 2\textsc{sg}-\textsc{irr}-be.quiet 3-say-3-\textsc{rprt} 3-son\\ 
\glft ‘“shut up!” he said to his son, it is said’\\ 
\endgl
\trailingcitation{[jxx-p120430l-2.181]}
\xe

Speech verbs have been identified as a special type of trivalent verbs in \isi{Mojeño Trinitario} \citep[]{Rose2011b}. This is probably also the best analysis for the Paunaka verb \textit{-kechu}, and it relates to the probable possibility of \textit{-punaku} ‘give’ taking third person markers after the stem (see above). Let us have a closer look at \textit{-kechu} ‘say’.

First of all, this verb can be used without any object index. The subject can be first, second or third person, and in the latter case, the marker \textit{ti-} is used relatively consistently as in (\ref{ex:SPEECH-4}), in which Clara cites what her brother José had said to her the day before.

\ea\label{ex:SPEECH-4}
\begingl
\glpreamble “niyuna nisemaika kringanube”, tikechu\\
\gla ni-yuna ni-semaika kringa-nube ti-kechu\\
\glb 1\textsc{sg}-go.\textsc{irr} 1\textsc{sg}-search gringa-\textsc{pl} 3i-say\\
\glft ‘“I am going to search for the gringas”, he said’
\endgl
\trailingcitation{[cux-c120510l-1.127]}
\xe

%i nikechu: pubremÿne kabe, jxx-a120516l-a.209

A first or second person addressee\is{addressee|(} can be indexed on the verb as an object. If the agent is a third person, \textit{ti-} is used to index the subject. This can be seen in (\ref{ex:SPEECH-5}), where Miguel tells me what his fellow said to him when he decided to teach him to calculate.

\ea\label{ex:SPEECH-5}
\begingl
\glpreamble tikechunÿ: “Miyel, nimesumeikapi echÿu”\\
\gla ti-kechu-nÿ Miyel ni-mesumeika-pi echÿu\\
\glb 3i-say-1\textsc{sg} Miguel 1\textsc{sg}-teach.\textsc{irr}-2\textsc{sg} \textsc{dem}b\\
\glft ‘he said to me: “Miguel, I am going to teach it to you”’
\endgl
\trailingcitation{[mxx-p181027l-1.119]}
\xe

When there is a third person addressee, the third person marker \textit{-chÿ} is normally used.\is{addressee|)} (\ref{ex:SPEECH-3}) provides one example, here in combination with a first person singular marker that indexes the \isi{agent}. It comes from Miguel’s account about the past of Santa Rita. He narrates how he talked with brother Bendelín, who helped in constructing the school of the village.

\ea\label{ex:SPEECH-3}
\begingl
\glpreamble entonses nÿjakupu nikechuchÿ “bueno...”\\
\gla entonses nÿ-jakupu ni-kechu-chÿ bueno\\
\glb thus 1\textsc{sg}-receive 1\textsc{sg}-say-3 well\\
\glft ‘so I accepted, I told him: “well...”’
\endgl
\trailingcitation{[mxx-p110825l.113]}
\xe

Strikingly, if the \isi{agent} is a third person, too, we usually find both \textit{chÿ-} and \textit{-chÿ} on the verb, as in the introductory example (\ref{ex:SPEECH-1}) above. Given the fact that \textit{chÿ-} already indexes two third person participants, it seems that three arguments are encoded in this case: the \isi{agent}, the \isi{addressee} and possibly the utterance. Since \textit{-chÿ} indexes the \isi{addressee} if there is an SAP \isi{agent}, I would suggest that it also encodes the \isi{addressee} in this relation and \textit{chÿ-} encodes \isi{agent} and utterance.

(\ref{ex:SPEECH-6}) offers a second example of the constellation with apparently three third person arguments being encoded. It comes from a different speaker than (\ref{ex:SPEECH-1}), has an \isi{agent} that is conominated (instead of the \isi{addressee}) and also shows the speech verb preceding the quoted speech. Miguel starts to tell me here what the \textit{patrón} said to Marco Choma, one of the founders of Santa Rita, when he finally let the indigenous people free in the 1950s.

\ea\label{ex:SPEECH-6}
\begingl
\glpreamble entonses chikechuchÿ echÿu chipatrone: “bueno...”\\
\gla entonses chi-kechu-chÿ echÿu chi-patron-ne bueno\\
\glb thus 3-say-3 \textsc{dem}b 3-patrón-\textsc{possd} well\\
\glft ‘so the \textit{patrón} said to him: “well...”
\endgl
\trailingcitation{[mxx-p110825l.026]}
\xe

To complete the picture: There are also a few occurrences of the verb \textit{-kechu} taking \textit{chÿ-} without any index following the stem and a few more where we find the prefix \textit{ti-} being combined with \textit{-chÿ}. This is mainly true for Miguel’s speech, but sometimes also found with Juana. An example is given in (\ref{ex:SPEECH-7}). It comes from Miguel’s story about the lazy man. This is the moment that the wife of the lazybones finds out that he was betraying her when he said he was going to work in the woods to supply them with food:

\ea\label{ex:SPEECH-7}
\begingl
\glpreamble i titupunubuji echÿu chiyenu i tikechuchÿji: “aja chikuyeje echÿu pitrabakune”, tikechuchi\\
\gla i ti-tupunubu-ji echÿu chi-yenu i ti-kechu-chÿ-ji aja chi-kuye-ja? echÿu pi-trabaku-ne ti-kechu-chi\\
\glb and 3i-arrive-\textsc{rprt} \textsc{dem}b 3-wife and 3i-say-3-\textsc{rprt} \textsc{intj} 3-be.like.this-\textsc{emph}1? \textsc{dem}b 2\textsc{sg}-work-\textsc{possd} 3i-say-3\\
\glft ‘and his wife arrived there, it is said, and said to him, it is said: “aha, this is your work”, she said to him’
\endgl
\trailingcitation{[mox-n110920l.071-072]}
\xe

In terms of its meaning, the verb form in (\ref{ex:SPEECH-7}) seems to be an equivalent to the ones in (\ref{ex:SPEECH-1}) and (\ref{ex:SPEECH-6}) above. However, it is this precisely this construction, i.e. the combination of the markers \textit{ti-} and \textit{-chÿ}, that we sometimes also find with other verbs.
\is{speech verb|)}

%Juana cites her own words that she had said to her brother the day before.
%
%\ea\label{ex:SPEECH-2}
%\begingl
%\glpreamble i nikechuchÿ “aa nikichupapi tajaitu”\\
%\gla i ni-kechu-chÿ aa ni-kichupa-pi tajai-tu \\
%\glb and 1\textsc{sg}-sai-3 \textsc{intj} 1\textsc{sg}-wait-2\textsc{sg} tomorrow-\textsc{iam}\\
%\glft ‘and I told him “ah, I will expect you tomorrow”‘\\
%\endgl
%\trailingcitation{[jxx-p120430l-1.127]}
%\xe

%-kechu as nomination verb: ciervo bikechuchi, pero ciervo tukiu kastelyanoyae, mtx-a110906l.167-169

As for these verbs, the \textit{-chÿ} sometimes also occurs in focus\is{focus|(} constructions of the argument focus type, i.e. where the proposition encoded by the sentence is already known to the person with whom the speaker is interacting, but the identity of one of the arguments is unknown to her \citep[cf.][228]{Lambrecht1994}. It may also be the case that the identity is known, but needs to be singled out, emphasised. The latter is the case in the following example which comes from the recordings made by Riester. Juan Ch. is contrasting the hard work of the indigenous people with the carelessness of the \textit{patrón}.

\ea\label{ex:bichÿ}
\begingl
\glpreamble pesau bitÿpi bitrabakune mapuine biti bebukutuchÿ\\
\gla pesau bi-tÿpi bi-trabaku-ne chejepuine biti bi-ebuku-tu-chÿ\\
\glb hard 1\textsc{pl}-\textsc{obl} 1\textsc{pl}-work-\textsc{possd} because 1\textsc{pl.prn} 1\textsc{pl}-sow-\textsc{iam}-3\\
\glft ‘our work is hard for us, because WE are the ones who sow’
\endgl
\trailingcitation{[nxx-p630101g-1.088]}
\xe

In (\ref{ex:tichi-1}), Juana speaks about her deceased brother Cristóbal.

\ea\label{ex:tichi-1}
\begingl
\glpreamble eka chiserebrone chibu tikupakuchÿ\\
\gla eka chi-serebro-ne chibu ti-kupaku-chÿ\\
\glb \textsc{dem}a 3-brain-\textsc{possd} 3\textsc{top.prn} 3i-kill-3\\
\glft ‘his brain, THIS is what killed him’
\endgl
\trailingcitation{[jxx-p120430l-2.383]}
\xe

Most cases of these focus constructions including \textit{-chÿ} include a pronoun\is{pronoun} and the focused argument is the \isi{subject} of the verb. However, it is also possible that a noun is used instead of a pronoun or that the focused argument is the object. (\ref{ex:tichi-2}) and (\ref{ex:tichi-3}) provide examples for this. (\ref{ex:tichi-2}) was elicited from María S. and contains a subject expressed by a noun.

\ea\label{ex:tichi-2}
\begingl
\glpreamble eka pimiyapÿi tikurabajikuchÿ nÿnikÿiki\\
\gla eka pimiyapÿi ti-kurabajiku-chÿ nÿ-nikÿiki\\
\glb \textsc{dem}a girl 3i-break-3 1\textsc{sg}-pot\\
\glft ‘this GIRL broke my pot’
\endgl
\trailingcitation{[rxx-e181024l]}
\xe

(\ref{ex:tichi-3}) is from the same recording as (\ref{ex:tichi-1}) above and also deals with the death of a family member: Juana’s sister, who died in a hospital and had to be picked up there.\footnote{Directly preceding this sentence, Juana uses the word \textit{-sienu} to refer to her. I do not know what this word means, but a bit later in the recording, it becomes clear that Juana is talking about her sister.}

\ea\label{ex:tichi-3}
\begingl
\glpreamble chibuyenu tirekojechuchÿ tukiu naukutu\\
\gla chibu-yenu ti-rekojechu-chÿ tukiu nauku-tu\\
\glb 3\textsc{top.prn}-\textsc{ded} 3i-pick.up-3 from there-\textsc{iam}\\
\glft ‘SHE must be the one whom they picked up there’
\endgl
\trailingcitation{[jxx-p120430l-2.295]}
\xe

There are not many examples of argument focus constructions of this type. It is more common to use a \isi{cleft} construction, which can convey the same pragmatic function, see \sectref{sec:Clefts}.\is{focus|)}

Finally, another context in which \textit{-chÿ} occasionally occurs is deranked verbs\is{deranked verb|(} marked by \textit{-i}. Subordination with the suffix \textit{-i} is described in detail in \sectref{sec:Subordination-i}. One example of a subordinate verb taking \textit{-chÿ} is given in (\ref{ex:juchubu-sub-3}), which has a question word, \textit{juchubu} ‘where’, that often takes subordinate predicates. For more examples see \sectref{sec:Q_juchubu}. This sentence comes from Miguel’s telling of the story about the two men and the devil. One man has given meat to the devil, but the latter does not fill up, so when the meat is finished, he asks for the skins (and subsequently for the heads and finally, he eats the man).


\ea\label{ex:juchubu-sub-3}
\begingl 
\glpreamble “¿juchubu ebikÿjikiuchÿ chimusuji echÿu ÿbajane?”\\
\gla juchubu e-bikÿjik-i-u-chÿ chi-musuji echÿu ÿba-jane\\ 
\glb where 2\textsc{pl}-throw.away-\textsc{subord}-\textsc{real}-3 3-skin \textsc{dem}b pig-\textsc{distr}\\ 
\glft ‘“where did you throw the pigs’ skins?”’\\ 
\endgl
\trailingcitation{[mxx-n101017s-1.055-056]}
\xe

\is{deranked verb|)}
\is{object|)}
\is{argument|)}
\is{person marking|)}

The third person marker \textit{-chÿ} is possibly also found as a (relatively) fixed part of numerals\is{numeral} (see \sectref{sec:Numerals}). At this point, we leave person marking behind and turn to the second major inflectional category of the verb: reality status.

