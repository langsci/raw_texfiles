%!TEX root = 3-P_Masterdokument.tex
%!TEX encoding = UTF-8 Unicode

\section{Interrogative clauses}\label{sec:Questions}
\is{interrogative clause|(}

While declarative clauses typically assert information, the main function of interrogative clauses is to request information \citep[294]{Payne1997}. Two main types of interrogative clauses can be distinguished. Polar questions seek an affirmation or negation\is{negation} of information already given in the question \citep[291]{Koenig2007}. They can be distinguished from declarative sentences by intonation in Paunaka. Content questions seek information about a participant in an event or some circumstances in the event. They build on a question word. Question words always precede the verb, i.e. they are placed in \isi{focus} position\is{word order} (see \sectref{sec:WordOrder}).

In this section, I first describe polar questions in \sectref{sec:PolarQuestions} and then turn to content questions in \sectref{sec:ContentQuestions}.


\subsection{Polar questions}\label{sec:PolarQuestions}
\is{polar question|(}

The only feature that distinguishes polar questions from declarative sentences is intonation.\is{intonation|(} In polar questions, pitch rises towards the end of the utterance, sometimes considerably, sometimes only slightly.

%special intonation pattern most frequent type of building polar questions (König & Siemund: 292), usually with rising intonation (ebd.)

To start with, consider (\ref{ex:Q1}). The question was produced by Juana and directed to me. It was uttered with a very high pitch towards the end.

\ea\label{ex:Q1}
\begingl
\glpreamble ¿pisachu pinika yÿtÿuku?\\
\gla pi-sachu pi-nika yÿtÿuku\\
\glb 2\textsc{sg}-want 2\textsc{sg}-eat.\textsc{irr} food\\
\glft ‘do you want to eat some food?’
\endgl
\trailingcitation{[jxx-d110923l-2.45]}
\xe

\figref{fig:pitch-Q} shows the pitch analysis for (\ref{ex:Q1}), for which I used Praat.\footnote{Developed by Paul Boersma and David Weenink, Phonetic Sciences, University of Amsterdam, see: https:\\www.fon.hum.uva.nl/praat/ Accessed 2021-04-16}

\begin{figure}

\includegraphics[width=\textwidth]{figures/QuestionPitch.pdf}
\caption{Pitch analysis of the question \textit{¿pisachu pinika yÿtÿuku?}}
\label{fig:pitch-Q}

\end{figure}

\is{intonation|)}

A question can consist of a single verb as in (\ref{ex:Q2}), where María S. asked me whether I had met her sister Clara earlier that day.

\ea\label{ex:Q2}
\begingl
\glpreamble ¿pisimuku?\\
\gla pi-simuku\\
\glb 2\textsc{sg}-find\\
\glft ‘did you meet her?’
\endgl
\trailingcitation{[rxx-e120511l.083]}
\xe

%pichupuiku nauku?, rxx-e120511l.255

(\ref{ex:Q3}) is a question about a third person. It refers to Federico who had rented an apartment in Concepción.

\ea\label{ex:Q3}
\begingl
\glpreamble ¿eka Federico tepajÿka nauku?\\
\gla eka Federico ti-pajÿka nauku\\
\glb \textsc{dem}a Federico 3i-stay.\textsc{irr} there\\
\glft ‘Federico will stay there?’
\endgl
\trailingcitation{[jxx-p110923l-1.089]}
\xe

A polar question can also be formed with a non-verbal predicate.\is{non-verbal predication} The greeting formula in Paunaka is actually a question for one’s condition and builds on the adjective \textit{micha} ‘good’. (\ref{ex:Q7}) is an example, where Juana produced this formula to teach Swintha.

\ea\label{ex:Q7}
\begingl
\glpreamble ¿michabi?\\
\gla micha-bi\\
\glb good-2\textsc{sg}\\
\glft ‘how are you?’ (lit.: ‘are you well?’)
\endgl
\trailingcitation{[jxx-n101013s-1.081]}
\xe

It is not really expected to provide information about ones condition when one is asked the question in (\ref{ex:Q7}). In order to ask for the condition of somebody, speakers use a slightly different wording attaching the \isi{continuous} marker to the adjective, see (\ref{ex:Q8}).\footnote{I cannot provide a recorded source for this question, since I simply do not have a recording of it. Greeting usually took place before I started my recording device.}

\ea\label{ex:Q8}
\begingl
\glpreamble ¿michachaikubi?\\
\gla micha-chaiku-bi\\
\glb good-\textsc{cont}-2\textsc{sg}\\
\glft ‘how are you?’
\endgl
%\trailingcitation{[]}
\xe

The question in (\ref{ex:Q4}) includes the non-verbal existential copula \textit{kaku}. In this specific case, the question rather expresses surprise than a request for information, since the information has been given before; Miguel had already told María S. that we were attacked by little ticks on the way to José’s house. 

\ea\label{ex:Q4}
\begingl
\glpreamble ¿kakutu samuchu?\\
\gla kaku-tu samuchu\\
\glb exist-\textsc{iam} tick\\
\glft ‘there are ticks already?’
\endgl
\trailingcitation{[mrx-c120509l.149]}
\xe

A polar question can also include a negative particle, and in that case it usually includes some greater amount of previous knowledge or some presupposition as  in the following examples.

(\ref{ex:Q5}) was elicited from María S. for the purpose that I could ask her about her knee, which I knew had hurt the days before.

\ea\label{ex:Q5}
\begingl
\glpreamble ¿kuina takutibu pisÿikuke?\\
\gla kuina ti-a-kuti-bu pi-sÿikuke\\
\glb \textsc{neg} 3i-\textsc{irr}-hurt-\textsc{dsc} 2\textsc{sg}-knee\\
\glft ‘doesn’t your knee hurt anymore?
\endgl
\trailingcitation{[rxx-e181022le]}
\xe

(\ref{ex:Q6}) is a negative question\is{negation} produced by Juana. She was talking about Cotoca, a town in the vicinity of Santa Cruz, and it was probably my reaction to what she had just said before that let her suppose that I had never visited the place.

\ea\label{ex:Q6}
\begingl
\glpreamble ¿kuina piyuna?\\
\gla kuina pi-yuna\\
\glb \textsc{neg} 2\textsc{sg}-go.\textsc{irr}\\
\glft ‘you haven’t gone there?’
\endgl
\trailingcitation{[jxx-p120430l-2.551]}
\xe

When answering\is{answer to question|(} a question, the predicate is usually repeated, i.e. Paunaka speakers use “verb-echo answers”\is{verb} \citep[3]{Holmberg2016}. There is an affirmative particle in Paunaka, which is \textit{jaa}, \textit{ja’a} or the like, but it does not suffice as an answer. 

One example of a question-answer pair is (\ref{ex:QA-2}), where María C. asks whether Pedro knows the man she was speaking about, a sorcerer, and Pedro affirms that he knows him.


\ea\label{ex:QA-2}
  \ea\label{ex:QA-2.1}
\begingl
\glpreamble \textup{u:} ¿pichupuiku?\\
\gla pi-chupuiku\\
\glb 2\textsc{sg}-know\\
\glft ‘do you know him?’
\endgl
  \ex\label{ex:QA-2.2}
\begingl
\glpreamble \textup{p:} nichupuiku\\
\gla ni-chupuiku\\
\glb 1\textsc{sg}-know\\
\glft ‘I know him (i.e. yes)’
\endgl
\trailingcitation{[ump-p110815sf.553-554]}
\z
\xe


Another question-answer pair is given in (\ref{ex:QA-1}). The question was asked by Miguel, when he helped Juana digging for loam for her clay pot in the vicinity of Santa Rita. The shell he asks for is used to pull up and smoothen the clay rolls, \textit{nauku} ‘there’ refers to Juana’s house in Santa Cruz, where she lived at that time.

\ea\label{ex:QA-1}
  \ea
\begingl
\glpreamble \textup{m:} ¿pero kaku nauku sipÿ pikeuchi?\\
\gla pero kaku nauku sipÿ pi-keuchi\\
\glb but exist there shell 2\textsc{sg}-\textsc{ins}\\
\glft ‘but do you have shells there?’ (lit.: ‘are there shells with regard to you?’)
\endgl
  \ex
\begingl
\glpreamble \textup{j:} kaku\\
\gla kaku\\
\glb exist\\
\glft ‘there are’
\endgl
\trailingcitation{[jmx-d110918ls-1.098-099]}
\z
\xe

In (\ref{ex:QA-3}) the affirmative particle accompanies the repeated verb in the answer. This example stems from Miguel’s narration about his time in school. It is a citation of a question of his teacher and the answer of one pupil.

\ea\label{ex:QA-3}
  \ea
\begingl
\glpreamble “¿pichuputu eka pitareane?”\\
\gla pi-chupu-tu eka pi-tarea-ne\\
\glb 2\textsc{sg}-know-\textsc{iam} \textsc{dem}a 2\textsc{sg}-excercise-\textsc{possd}\\
\glft ‘“do you know your exercise now?”'
\endgl
  \ex
\begingl
\glpreamble “jaa, nÿchuputu”\\
\gla jaa nÿ-chupu-tu\\
\glb \textsc{afm} 1\textsc{sg}-know-\textsc{iam}\\
\glft ‘“yes, I know it”’
\endgl
\trailingcitation{[mxx-p181027l-1.047]}
\z
\xe

In an answer to a negative question,\is{negation|(} the negative particle occurs together with the predicate to confirm the negative alternative, i.e. Paunaka exhibits a polarity-based system \citep[cf.][140]{Holmberg2016}. An example is (\ref{ex:QA-new23}) from Juana who reproduced what her brother asked her when their other brother had passed away.

\ea\label{ex:QA-new23}
  \ea
\begingl
\glpreamble “¿kuina pisama eka mensaje?”\\
\gla kuina pi-sama eka mensaje\\
\glb \textsc{neg} 2\textsc{sg}-hear.\textsc{irr} \textsc{dem}a message\\
\glft ‘“haven’t you heard (i.e. received) the message?”'
\endgl
  \ex
\begingl
\glpreamble “kuina nisama”\\
\gla kuina ni-sama\\
\glb \textsc{neg} 2\textsc{sg}-hear.\textsc{irr}\\
\glft ‘“no, I haven’t heard it”’
\endgl
\trailingcitation{[jxx-p120430l-2.266-267]}
\z
\xe

In this case, it also seems to be possible to omit the verb and only use the negative particle, as in (\ref{ex:QA-new23-2}) from elicitation with Juana. However, all examples I have stem from elicitation or imagined or remembered dialogues reported by a single speaker, I have not found a single example of a question – answer pair including a negative question that comes from natural conversation between two Paunaka speakers. Thus it remains to be checked whether they employ the same patterns in real conversation.

\ea\label{ex:QA-new23-2}
  \ea
\begingl
\glpreamble ¿kuina pekicha?\\
\gla kuina pi-ekicha\\
\glb \textsc{neg} 2\textsc{sg}-invite.\textsc{irr} \\
\glft ‘didn’t you give him anything (to eat or drink)?'
\endgl
  \ex
\begingl
\glpreamble kuina\\
\gla kuina\\
\glb \textsc{neg}\\
\glft ‘no’
\endgl
\trailingcitation{[jmx-e090727s.179]}
\z
\xe
\is{negation|)}

A negative question can be answered positively by repeating the predicate, which usually has \isi{realis} RS in the answer (if irrealis is not demanded by another factor, e.g. future time reference). This is the case in (\ref{ex:QA-4}), which is a little imagined conversation by  Juana and was triggered by me asking questions about lurking (in trying to make sense of the dislocative marker). This made Juana think about hunting and catching animals.

\ea\label{ex:QA-4}
  \ea
\begingl
\glpreamble ¿kuina tituika? \\
\gla kuina ti-tuika\\
\glb \textsc{neg} 3i-hunt.\textsc{irr}\\
\glft ‘he didn’t catch any (animals)?
\endgl
  \ex
\begingl
\glpreamble tituiku, unya\\
\gla ti-tuiku unya\\
\glb 3i-hunt gray.brocket\\
\glft ‘he caught one, a gray brocket’
\endgl
\trailingcitation{[jxx-e110923l-1.059-060]}
\z
\xe
\is{answer to question|)}

%pikubiakubu max nÿmayu bi- etupunubu tukiu nauku Santa Rita? cux-c120414ls-2.330, 
%
%
%¿pisachu pinika yÿtÿuku? tekomperauchapi eka nijinepÿi jxx-d110923l-2.45 -> gute, deutliche Fragekontur
%¿ee pisachu pinika eka mutu?, jxx-p120430l-2.640 -> viel weniger deutlich!
%
%¿pisachu pibena yumaji? ¿pisachu pebikapu?
%Quieres mecerte (en la hamaca)? Jxx-p150920l.017-018
%
%eka Federico tepajÿka nauku? jxx-p110923l-1.089
%
%
%
%
%
%¿pisachu pea kape?, jxx-e120430l-3a
\is{polar question|)}


\subsection{Content questions}\label{sec:ContentQuestions}
\is{content question|(}

Content questions “receive answers that provide the kind of information specified by the interrogative word” \citep[291]{Koenig2007}.\is{answer to question} Content questions build on question words\is{question word|(} in Paunaka, which are given in \tabref{table:QuestionWords}. The question words are always placed in the first position, i.e. they occupy the position of the sentence that is associated with emphasis.

\begin{table}
\caption{Question words}
\begin{tabularx}{\textwidth}{l>{\raggedright\arraybackslash}p{2.2cm}QQ}
\lsptoprule
Question word & Translation & Category & Source\\
\midrule
\textit{chija} & what, who, whom & subject, object, action, identity & \textit{chi-ija} 3-name ‘his/her/its name’?\\
\textit{(chi)kuyena} & how, why & manner, reason & based on manner verb \textit{-kuye} ‘be like this’\\
\textit{kena} & what about & generic & uncertainty marker \textit{kena}\\
\textit{juchubu} & where, when & location, time & \textit{j-u(-)chu-bu}? \\
\textit{(u)kajane} & how many & quantity & with distributive marker \textit{-jane} \\
\lspbottomrule
 \end{tabularx}
\label{table:QuestionWords}
\end{table}

Question words often, but not always, combine with \isi{focus} expressions. This may be a special form of the transitive verb including a third person marker\is{person marking} following the verb stem, a relative clause\is{relative relation} or a \isi{deranked verb}, see \sectref{sec:3_suffixes}, \sectref{sec:Clefts} and \sectref{sec:Subordination-i} for more information about these constructions. The different question words have different possibilities of combining with one or the other of them. In addition, “plain” finite verbs\is{finite verb} can also be used in questions.

Sometimes, question words take \textit{-chÿ}, \textit{-chÿu} or \textit{-chu}. It is not entirely clear to me, what this form is, and whether it is always the same marker only pronounced differently. As for \textit{-chÿ}, it is found on \textit{chija} and \textit{(u)kajane}. This might be the third person marker.\is{person marking} The form \textit{-chÿu}, which may be a cliticised\is{clitic} form of the demonstrative\is{nominal demonstrative} \textit{echÿu}, occurs on \textit{chija}, \textit{juchubu} and \textit{(chi)kuyena}, \textit{-chu} is only found on \textit{juchubu}. I gloss \textit{-chÿ} as ‘3’, i.e. as third person marker on \textit{(u)kajane}, every other occurrence of \textit{-chÿ}, \textit{-chÿu} or \textit{-chu} is glossed as ‘\textsc{dem}b?’, regardless of its actual pronunciation.\footnote{\isi{Mojeño Trinitario} has a “restrictive clitic” \textit{-chu} (Rose 2021, p.c.). Thus in the future, thorough comparison with this language could shed light on the precise function of the Paunaka form(s) found on question words.}\is{question word|)}

The remainder of this section is structured as follows: \sectref{sec:Q_chija} is about questions for subject and object participants as well as actions, \sectref{sec:Q_juchubu} deals with questions about locations and points in time. In \sectref{sec:Q_chikuyena}, questions for reason and manner are presented, and \sectref{sec:Q_kajane} is about requesting quantities. Finally, \sectref{sec:Q_kena} describes some very general questions based on the uncertainty marker. 

\subsubsection{Questions for persons and things}\label{sec:Q_chija}

The question word \is{question word|(} \textit{chija} ‘what, who’ is used to form different kinds of questions for a referent: it can be used to request for human and non-human entities, animate and non-animate alike. It can be combined with a verbal or non-verbal predicate\is{non-verbal predication} and the requested participant can be a \isi{subject} or an \isi{object} of a verbal clause, one of the constituents of an existential\is{existential clause} or equative clause\is{equative/proper inclusion clause} or an action.

The question word itself could possibly derive from the noun \textit{-ija} ‘name’ with a third person possessor, but in any case it is totally grammaticalised\is{grammaticalisation} and may thus be called an interrogative \isi{pronoun}.\footnote{In addition, \textit{chija} also serves as an indefinite pronoun, see \sectref{sec:IndefinitePronouns}.} This becomes apparent in questions about the name of somebody, in which the word form \textit{chija} doubles, as in (\ref{ex:what-name}), which was produced by María C. to obtain some information about my family.

\ea\label{ex:what-name}
\begingl
\glpreamble ¿chija chija penu?\\
\gla chija chi-ija pi-enu\\
\glb what 3-name 2\textsc{sg}-mother\\
\glft ‘what is your mother’s name?’
\endgl
\trailingcitation{[uxx-p110825l.144]}
\xe

Sometimes, \textit{-chÿu} or \textit{-chÿ} is attached to \textit{chija}, which could be a cliticised nominal demonstrative (\textit{echÿu}). Like the question word itself, attachment of \textit{-chÿu} seems to be relatively grammaticalised,\is{grammaticalisation} because a free demonstrative can co-occur, as in (\ref{ex:what-this}), which represents Juana’s reaction as reported by herself, when she was offered frogs to eat.

\ea\label{ex:what-this}
\begingl
\glpreamble ¿chijachÿu echÿu?\\
\gla chija-chÿu echÿu\\
\glb what-\textsc{dem}b? \textsc{dem}b\\
\glft ‘what is this?’
\endgl
\trailingcitation{[jxx-a120516l-a.479]}
\xe
\is{question word|)}

When asking for an action, speakers make use of the verb \textit{-chabu} ‘do’, which almost exclusively occurs in questions.\footnote{The verb may also be nominalised \textit{-chabukene} ‘deeds, behaviour’, but is not used as a verbal predicate in declarative sentences.} 

The question in (\ref{ex:what-do-1}) belongs to the repertoire of exchange of pleasantries, when meeting each other, like (\ref{ex:Q7}) and (\ref{ex:Q8}) above. It was produced by Isidro when meeting Swintha.

\ea\label{ex:what-do-1}
\begingl
\glpreamble ¿chija pichabu?\\
\gla chija pi-chabu\\
\glb what 2\textsc{sg}-do\\
\glft ‘what are you doing?’
\endgl
\trailingcitation{[mdx-c120416ls.005]}
\xe

The verb can also be used to request what others are doing as in (\ref{ex:what-do-2}), in which María S. asks about her brother.

\ea\label{ex:what-do-2}
\begingl
\glpreamble ¿chija chichabu Miyel?\\
\gla chija chi-chabu Miyel\\
\glb what 3-do Miguel\\
\glft ‘what is Miguel doing?’
\endgl
\trailingcitation{[rxx-e120511l.337]}
\xe

Apart from \textit{-chabu} ‘do’, \textit{chija} is not often combined with a plain \isi{finite verb}, but a few examples occurred nonetheless. Two are given here. 

(\ref{ex:what-eat}) is a question the jaguar asks the fox, when he finds him eating cheese in the story told by María S.

\ea\label{ex:what-eat}
\begingl
\glpreamble “¿chija piniku?”\\
\gla chija pi-niku\\
\glb what 2\textsc{sg}-eat\\
\glft ‘“what are you eating?“’
\endgl
\trailingcitation{[rxx-n120511l-1.031]}
\xe

In (\ref{ex:who-die}), the uncertainty marker on the question word tells us that we are dealing with a rhetorical question, that the one who asks does not expect the addressee to know the answer. Juana reports here what she asked her brother, when he told her that a family member had died, but could not say whom it was, because the message he received to tell him about the death was not clear.

\ea\label{ex:who-die}
\begingl
\glpreamble ¿chijakena tepaku?\\
\gla chija-kena ti-paku\\
\glb what-\textsc{uncert} 3i-die\\
\glft ‘who may have died?’
\endgl
\trailingcitation{[jxx-p120430l-2.285]}
\xe


If the verb is \isi{transitive}, a third person marker\is{person marking} can follow the verb stem, a construction reserved to express argument \isi{focus}. This is the case in the following questions.

(\ref{ex:who-break}) was elicited from María S.

\ea\label{ex:who-break}
\begingl
\glpreamble ¿chija tikurabajikuchÿ nÿnikÿiki?\\
\gla chija ti-kurabajiku-chÿ nÿ-nikÿiki\\
\glb what 3i-break-3 1\textsc{sg}-pot\\
\glft ‘who broke my pot?’
\endgl
\trailingcitation{[rxx-e181024l]}
\xe

(\ref{ex:who-take}) comes from Juana telling the story about how the silk floss tree obtained its big belly-like trunk: it swallowed all of Jesus’ corn, who asks for the fate of this supply of corn here.

\ea\label{ex:who-take}
\begingl
\glpreamble “¿chija tumuchÿ?”\\
\gla chija ti-umu-chÿ\\
\glb what 3i-take-3\\
\glft ‘“who took it?”’
\endgl
\trailingcitation{[jxx-n101013s-1.663]}
\xe

Questions for a possessor \is{possessor|(} are a subtype of this kind of questions, since they all build on a verb.\is{attributive prefix|(} The verb is composed of the attributive prefix \textit{ku-} (see \sectref{sec:AttributiveVerbs}) and either \textit{-peu} ‘animal’, as in (\ref{ex:Q-own-1}), or \textit{-yae} ‘\textsc{grn}’, as in (\ref{ex:Q-own-2}) and usually takes the third person marker\is{person marking} \textit{-chÿ}. The examples were elicited from María S. Both \textit{-peu} and \textit{-yae} also play a role in possession marking of non-possessable nouns (see \sectref{sec:Non-possessables}).

\ea\label{ex:Q-own-1}
\begingl
\glpreamble ¿chija tikupeuchÿ ÿba?\\
\gla chija ti-kupeu-chÿ ÿba\\
\glb what 3i-have.animal-3 pig\\
\glft ‘whose pig is this?’
\endgl
\trailingcitation{[rxx-e201231f.08]}
\xe

\ea\label{ex:Q-own-2}
\begingl
\glpreamble ¿chijakena tikuyaechÿkenatu San Jorge?\\
\gla chija-kena ti-kuyae-chÿ-kena-tu {San Jorge}\\
\glb what-\textsc{uncert} 3i-own-3-\textsc{uncert}-\textsc{iam} {San Jorge}\\
\glft ‘who may be the owner of (the estate) San Jorge now?’
\endgl
\trailingcitation{[rxx-e201231f.34]}
\xe

The usage of an attributive verb based on either \textit{-peu} or \textit{-yae} can be considered the main strategy of asking for a possessor, but alternatively, any other noun can also be added to the attributive prefix to derive\is{derivation} a verb of possession, as in (\ref{ex:Q-own-3}), which was also elicited from María S.

\ea\label{ex:Q-own-3}
\begingl
\glpreamble ¿chijakena tikumÿubanechÿ eka mÿuji?\\
\gla chija-kena ti-kumÿu-bane-chÿ eka mÿu-ji\\
\glb what-\textsc{uncert} 3i-have.garment-\textsc{rem}-3 \textsc{dem}a clothes-\textsc{clf:}soft.mass\\
\glft ‘whose garment may this have been before?’
\endgl
\trailingcitation{[rxx-e201231f.44]}
\xe
\is{possessor|)}
\is{attributive prefix|)}

Finally, \textit{chija} often combines with a relative clause,\is{relative relation} especially when \textit{-kena} is attached. In many relative clauses, the \isi{verb} is totally unmarked, but they can be recognised by being introduced with a demonstrative (see \sectref{sec:HeadlessRC}). Combination of \textit{chija} with a relative clause can be considered a subtype of cleft\is{cleft|(} construction (see \sectref{sec:Clefts}) and reflects the structure of the corresponding question in Spanish: \textit{¿qué/quién será que...?} ‘what/who could it be that...?’.

In (\ref{ex:Q-cleft-1}), Juana starts telling the story about the fox and the jaguarundi, but interrupts herself, because she does not remember which animal the fox met.

\ea\label{ex:Q-cleft-1}
\begingl
\glpreamble i chitupukuku echÿu ¿chijachÿukena echÿu chitupuku?\\
\gla i chi-tupuku-uku echÿu chija-chÿu-kena echÿu chi-tupuku\\
\glb and 3-meet-\textsc{add} \textsc{dem}b what-\textsc{dem}b?-\textsc{uncert} \textsc{dem}b 3-meet\\
\glft ‘and he also met the, what was it that he met?’
\endgl
\trailingcitation{[jmx-n120429ls-x5.301-302]}
\xe

(\ref{ex:Q-cleft-2}) was elicited from María S. The relative clause builds on a non-verbal predicate borrowed from Spanish.

\ea\label{ex:Q-cleft-2}
\begingl
\glpreamble ¿chijakena eka pasau chitÿpi? kuina kapunuinabu \\
\gla chija-kena eka pasau chi-tÿpi kuina kapunu-ina-bu\\
\glb what-\textsc{uncert} \textsc{dem}a pass 3-\textsc{obl} \textsc{neg} come-\textsc{irr.nv}-\textsc{dsc}\\
\glft ‘what may have happened to him that he doesn’t come anymore?’
\endgl
\trailingcitation{[rxx-e181022le]}
\xe

(\ref{ex:Q-cleft-3}) was elicited from Juana. Actually, she was asked to translate “why are the cows afraid?”, but she formed the question differently.

\ea\label{ex:Q-cleft-3}
\begingl
\glpreamble tipikujane eka bakajane, ¿chijakena eka cheikukuikujane?\\
\gla ti-piku-jane eka baka-jane chija-kena eka chÿ-eikukuiku-jane\\
\glb 3i-be.afraid-\textsc{distr} \textsc{dem}a cow-\textsc{distr} what-\textsc{uncert} \textsc{dem}a 3-chase-\textsc{distr}\\
\glft ‘the cows are afraid, what may it be that chases them?’
\endgl
\trailingcitation{[jxx-a110923l.18]}
\xe

Occasionally, in questions with a relative clause,\is{relative relation} the Spanish relativiser \textit{ke} (Span. \textit{que}) shows up instead of a demonstrative, as in (\ref{ex:Q-cleft-4}). This question was asked by Juana, when we had requested a story from her and Miguel.

\ea\label{ex:Q-cleft-4}
\begingl
\glpreamble ¿chijakena ke bakueteachikena?\\
\gla chija-kena ke bi-a-kuetea-chi-kena\\
\glb what-\textsc{uncert} \textsc{rel} 1\textsc{pl}-\textsc{irr}-tell-3-\textsc{uncert}\\
\glft ‘what can we tell her?’
\endgl
\trailingcitation{[jmx-n120429ls-x5.046]}
\xe
\is{cleft|)}

\subsubsection{Questions for locations and time}\label{sec:Q_juchubu}

Questions for location and points in time are formed with the question word\is{question word|(} \textit{juchubu}. The composition of the word is quite opaque, it may be decomposed into \textit{j-u-chu-bu}, with an existential or locative root \textit{-u}, also found in the defective verb \textit{-ubu} ‘be, live’, the thematic suffix \textit{-chu} and the middle marker \textit{-bu}, so it possibly goes back to a verb denoting existence or location at a place. It may also be related to the \isi{uncertain future} particle \textit{uchu} (see \sectref{sec:UncertainFuture}).  As for the first part \textit{j-}, this prefix is also found on the \isi{mirative} particle \textit{jimu} ‘you see, you know, right?’ (see Footnote \ref{fn:mirative} in \sectref{sec:Frust_avertive_optatiev}), but it is not productive in any way. The middle marker\is{middle voice} \textit{-bu} in \textit{juchubu} is dropped in a few examples and sometimes \textit{-chÿu} or \textit{-chu} is attached to the question word. Like \textit{chija} \textit{juchubu} can also be used as an indefinite pronoun, see \sectref{sec:IndefinitePronouns}.

In questions for location, the question word \textit{juchubu} is usually combined with the \isi{copula} \textit{kaku} or the defective verb \textit{-ubu} ‘be, live’.  The first two examples are formed with the non-verbal existential copula \textit{kaku}.\is{question word|)}

(\ref{ex:where-1}) was produced by Miguel in an elicitation session, in which he and Alejo had two identical sets of wooden toys. The arrangement of toys Alejo saw was given and Miguel was supposed to arrange his set of wooden toys in an identical way by asking questions.

\ea\label{ex:where-1}
\begingl
\glpreamble ¿juchubu kaku echÿu yÿkÿke?\\
\gla juchubu kaku echÿu yÿkÿke?\\
\glb where exist \textsc{dem}b tree\\
\glft ‘where is the tree?’
\endgl
\trailingcitation{[mtx-e110915ls.19]}
\xe

In (\ref{ex:where-2}), which comes from the story about the fox and the jaguar told by María S., the jaguar asks the fox, where he had obtained the cheese he was eating. There is no NP denoting the cheese, reference is sufficiently clear from the context.

\ea\label{ex:where-2}
\begingl
\glpreamble “¿juchubu kaku?”\\
\gla juchubu kaku\\
\glb where exist\\
\glft “where is some?”
\endgl
\trailingcitation{[rxx-n120511l-1.033]}
\xe

(\ref{ex:where-3}) and (\ref{ex:where-4}) are examples of the use of the verb \textit{-ubu} in a question for a location. In (\ref{ex:where-3}), the location of a third person participant is requested. The verb thus takes a third person marker, more precisely \textit{chÿ-}, since this verb is never found with \textit{ti-}. The example comes from Juana’s story about how the floss silk tree obtained its big trunk. The question is asked by Jesus in the story in order to obtain information about his supply of corn (see also (\ref{ex:who-take}) above).

\ea\label{ex:where-3}
\begingl
\glpreamble “¿juchubu chubu neumuka?”\\
\gla juchubu chÿ-ubu neumuka\\
\glb where 3-be supply\\
\glft ‘“where is the supply (of corn)?”’
\endgl
\trailingcitation{[jxx-n101013s-1.659]}
\xe

In (\ref{ex:where-4}), there is second person reference. It is Miguel’s translation of the book title of the \isi{frog story} by \citet[]{Mayer2003} “Frog, where are you?".

\ea\label{ex:where-4}
\begingl
\glpreamble ¿juchubu pubu peÿyubi?\\
\gla juchubu pi-ubu peÿ-yu-bi\\
\glb where 2\textsc{sg}-be frog-\textsc{ints}-2\textsc{sg}\\
\glft ‘where are you, dear frog?’
\endgl
\trailingcitation{[mox-a110920l-2.197]}
\xe

It seems that any other verb except for \textit{-ubu} usually occurs in deranked form\is{deranked verb} when combined with \textit{juchubu}, although a few exceptions of this are found.

(\ref{ex:where-5}) is such an exception. It has a dynamic finite verb and was directed to José whom Miguel and Swintha met, when they were just on the way to his house to visit him.

\ea\label{ex:where-5}
\begingl
\glpreamble ¿juchubu piyuna?\\
\gla juchubu pi-yuna\\
\glb where 2\textsc{sg}-go.\textsc{irr}\\
\glft ‘where are you going?’
\endgl
\trailingcitation{[mox-c110926s-1.132]}
\xe

In contrast, the following questions are built on a deranked verb. The deranked form of a verb contains the “subordinate” suffix \textit{-i}. This form occurs in several contexts of subordination but not exclusively, see \sectref{sec:Subordination-i} and \sectref{sec:AdverbialModification}.

(\ref{ex:where-6}) was produced by Juana to ask me about the route of my flight back to Germany.

\ea\label{ex:where-6}
\begingl
\glpreamble ¿juchubu piyunia tukiu naka, Argentina?\\
\gla juchubu pi-yun-i-a tukiu naka Argentina\\
\glb where 2\textsc{sg}-go-\textsc{subord}-\textsc{irr} from here Argentina\\
\glft ‘where will you go from here, to Argentina?’
\endgl
\trailingcitation{[jxx-e120516l-1.111]}
\xe

(\ref{ex:where-7}) is from the story about the two men and the devil. Having eaten all meat the men hunted, the devil is still hungry and asks for the pigs’ heads.

\ea\label{ex:where-7}
\begingl
\glpreamble “¿juchubu ebikÿjikiuchÿ echÿu chichÿtijane ÿba?”\\
\gla juchubu e-bikÿjik-i-u-chÿ echÿu chi-chÿti-jane ÿba\\
\glb where 2\textsc{pl}-throw.away-\textsc{subord}-\textsc{real}-3 \textsc{dem}b 3-head-\textsc{distr} pig\\
\glft ‘“where did you throw the pigs’ heads?”’
\endgl
\trailingcitation{[mxx-n101017s-1.046-048]}
\xe

(\ref{ex:where-8}) was translated on request in an elicitation session with Miguel and Juana. Note that Miguel attaches \textit{-chÿu} to the question word (\getfullref{ex:where-8.1}), while Juana does not (\getfullref{ex:where-8.2}). It is not clear whether there is a difference in meaning.

\ea\label{ex:where-8}
  \ea\label{ex:where-8.1}
\begingl
\glpreamble \textup{m:} ¿juchubuchÿu pimukiu?\\
\gla juchubu-chÿu pi-muk-i-u\\
\glb where-\textsc{dem}b? 2\textsc{sg}-sleep-\textsc{subord}-\textsc{real}\\
\glft ‘where do you sleep?’
\endgl
  \ex\label{ex:where-8.2}
\begingl
\glpreamble \textup{j:} ¿juchubu pimukiu? \\
\gla juchubu pi-muk-i-u\\
\glb where 2\textsc{sg}-sleep-\textsc{subord}-\textsc{real}\\
\glft ‘where do you sleep?’
\endgl
\trailingcitation{[jmx-e090727s.362-363]}
\z
\xe

Finally, (\ref{ex:where-9}) also has \textit{-chÿu} on the question word but is combined with a finite verb. The example comes from the recordings made by Riester and reflects the hopelessness of Juan Ch. who knows he is treated badly by his \textit{patrón}, but sees no alternative to staying with him nonetheless.

\ea\label{ex:where-9}
\begingl
\glpreamble ¿juchubuchÿukena biyuna?\\
\gla juchubu-chÿu-kena bi-yuna\\
\glb where-\textsc{dem}b?-\textsc{uncert} 1\textsc{pl}-go.\textsc{irr}\\
\glft ‘where could we go?’
\endgl
\trailingcitation{[nxx-p630101g-1.176]}
\xe

Requesting a location can be considered the primary function of \textit{juchubu}, but it may also be used to ask for a point in time. In the latter case, \textit{juchubu} is usually combined with a word with temporal meaning. In (\ref{ex:when-1}), this is the word \textit{tijai} ‘day’. In this case, the shorter form \textit{juchu} is used, which lacks the middle marker. This form occurs infrequently in the corpus without any notable functional or semantic difference to \textit{juchubu}. The question was asked by María C. in seeking information about which day the workshop on Paunaka would be held.

\ea\label{ex:when-1}
\begingl
\glpreamble ¿juchu tijai?\\
\gla juchu tijai\\
\glb where day\\
\glft ‘what day?’
\endgl
\trailingcitation{[mux-c110810l.015]}
\xe

%juchubu tijai pitupunubu, María S., rmx-e150922l.002

In (\ref{ex:when-2}), \textit{juchubu} combines with \textit{uchu}, a particle denoting an uncertain and in most cases remote future (see \sectref{sec:UncertainFuture}). Juana asked this question, when she was telling me about Cotoca and felt like going there together with me.

\ea\label{ex:when-2}
\begingl
\glpreamble ¿juchubukena uchukena biyuna nauku?\\
\gla juchubu-kena uchu-kena bi-yuna nauku\\
\glb where-\textsc{uncert} \textsc{uncert.fut}-\textsc{uncert} 1\textsc{pl}-go.\textsc{irr} there\\
\glft ‘when may we go there?’
\endgl
\trailingcitation{[jxx-p120430l-2.557]}
\xe

(\ref{ex:when-3}) was produced by Juana in an elicitation session in an imagined beginning of a conversation. Note that the adverb \textit{uchuine} ‘just now', which denotes a point in time some time ago on the same day, has a continuous marker attached here and a third person marker which follows the stem and thus resembles a verb. I do not know why this is the case.

\ea\label{ex:when-3}
\begingl
\glpreamble naka, ¿juchubu chuineneikuchÿ pibÿsÿu?\\
\gla naka juchubu uchuine-neiku-chÿ pi-bÿsÿu\\
\glb here where just.now-\textsc{cont}-3 2\textsc{sg}-come\\
\glft ‘how much time has passed since you came here?’
\endgl
\trailingcitation{[jxx-e150925l-1.038]}
\xe

Finally, sometimes there is no temporal expression in combination with \textit{juchubu}; usually, this is the case when it is clear enough from the context or due to combination with the verb that a point in time is requested instead of a location. This is the case in the last example in this section, which comes from elicitation with Miguel and has the structure of a \isi{cleft} construction.

\ea\label{ex:when-4}
\begingl
\glpreamble ¿juchubukena echÿu pibÿsÿupupunuka?\\
\gla juchubu-kena echÿu pi-bÿsÿu-pupunuka\\
\glb where-\textsc{uncert} \textsc{dem}b 2\textsc{sg}-come-\textsc{reg.irr}\\
\glft ‘when is it that you come back?’
\endgl
\trailingcitation{[mxx-e090728s-1.48]}
\xe

\subsubsection{Questions for manner and cause}\label{sec:Q_chikuyena}

Questions for manner and reason are formed with \textit{(chi)kuyena}.\is{question word|(} Reason may be an extension of manner, as the overlap resembles the overlap between the instrumental and causal\is{instrument/cause} function of the preposition \textit{-keuchi} (see \sectref{sec:adp-keuchi}). The question word derives from the manner verb \textit{-kuye} ‘be like this’,\is{demonstrative verb} which always takes the third person marker \textit{chi-} in my corpus, even though it is a stative verb by position of irrealis marking (i.e. its irrealis form is \textit{chakuye} – see also \sectref{sec:TransitiveStativeV} –, but this one does not occur in questions). When used as a question word, \textit{-na} is added to \textit{chikuye} yielding \textit{chikuyena}.\footnote{\textit{Chikuyena} or \textit{kuyena} is also occasionally found in non-interrogative contexts, but \textit{chikuye} (or \textit{kuye}) without \textit{-na} is never used as a question word.} It is not clear what kind of suffix this is; it could be the general \isi{classifier} \textit{-na}, which we also find with some adjectives (see \sectref{sec:Adjectives}) or – less probably given the verbal origin – the non-verbal irrealis marker \textit{-ina}. The third person marker is sometimes dropped, thus we also find \textit{kuyena} in questions. 


The question word most often combines with a verb that is introduced by the demonstrative \textit{eka},\is{nominal demonstrative} a construction that we also sometimes find in complementation\is{complement relation} (see \sectref{sec:CC_CMPL}). The verb may be finite\is{finite verb} or deranked.\is{deranked verb} Sometimes, however, no demonstrative is included. \is{question word|)} I will start with some examples that show the use of the question word in requesting manner and then turn to some examples which illustrate its use in questions for reason.

(\ref{ex:Qkuyena-1}) was elicited from María S. It has a deranked verb which is not introduced by a demonstrative.


\ea\label{ex:Qkuyena-1}
\begingl
\glpreamble ¿kuyena panaiuchi yumaji?\\
\gla kuyena pi-ana-i-u-chi yumaji\\
\glb how 2\textsc{sg}-make-\textsc{subord}-\textsc{real}-3 hammock\\
\glft ‘how do you make the hammock?’
\endgl
\trailingcitation{[rxx-e181022le]}
\xe

A deranked verb combined with a demonstrative is found in (\ref{ex:Qkuyena-3}) from Juana telling the story about the fox and the jaguar. This is a question the hungry jaguar asks the fox when he finds him eating cheese.

\ea\label{ex:Qkuyena-3}
\begingl
\glpreamble “¿chikuyena eka pitiuchi eka?” \\
\gla chikuyena eka pi-it-i-u-chi eka\\
\glb how \textsc{dem}a 2\textsc{sg}-master-\textsc{subord}-\textsc{real}-3 \textsc{dem}a\\
\glft ‘“how did you get this?”’
\endgl
\trailingcitation{[jmx-n120429ls-x5.245-246]}
\xe

(\ref{ex:Qkuyena-3}) has a finite verb introduced with \textit{eka}. The sentence produced by Miguel, when he helped Juana dig for loam at a place close to Santa Rita. The remote marker \textit{-bane} on the verb indicates that he knows that she has known this place long before.

\ea\label{ex:Qkuyena-2}
\begingl
\glpreamble ¿chikuyena eka pitupubanechÿ eka muteji? \\
\gla chikuyena eka pi-tupu-bane-chÿ eka muteji\\
\glb how \textsc{dem}a 2\textsc{sg}-find-\textsc{rem}-3 \textsc{dem}a loam\\
\glft ‘how did you once find the loam?’
\endgl
\trailingcitation{[jmx-d110918ls-1.013]}
\xe

(\ref{ex:Qkuyena-4}) could theoretically also be analysed as a request for manner, but it is a rhetorical question in this case, so no answer is expected. In this example, Juana uses finite verbs that are not introduced by \textit{eka}. She told me about a place in the woods, where they wash in a big hollow rock. Apparently, the place is watched over by a spirit, because the clothes are often blown away and disappear in the woods, which are difficult to access with all the plants growing there.

\ea\label{ex:Qkuyena-4}
\begingl
\glpreamble kimenu nauku ¿kuyena biyuna bisemaika bimÿu nauku?\\
\gla kimenu nauku kuyena bi-yuna bi-semaika bi-mÿu nauku?\\
\glb woods there how 1\textsc{pl}-go.\textsc{irr} 1\textsc{pl}-search.\textsc{irr} 1\textsc{pl}-clothes there\\
\glft ‘there are the woods, how could we go and look for our clothes there?’
\endgl
\trailingcitation{[jxx-p151020l-2]}
\xe

The next example clearly does not request manner, because it is about the non-realisation of an action. Here, \textit{-chÿ} is attached to the question word; however, this is less frequent with \textit{(chi)kuyena} than with \textit{chija} or \textit{juchubu}, see \sectref{sec:Q_chija} and \sectref{sec:Q_juchubu} above. Like in (\ref{ex:Qkuyena-4}) above, a finite verb is used, this time in combination with the demonstrative. In Juana’s report, she said this sentence to her daughter, when the latter did not pick up her other daughter at the airport in Spain. Her other daughter was finally deported and had to fly back to Bolivia.

\ea\label{ex:Qkuyena-5}
\begingl
\glpreamble ¿chikuyenachÿ eka kuina piyuna?\\
\gla chikuyena-chÿu eka kuina pi-yuna\\
\glb how-\textsc{dem}b? \textsc{dem}a \textsc{neg} 2\textsc{sg}-go.\textsc{irr}\\
\glft ‘how could you NOT go?’\\or: ‘why didn’t you go?’
\endgl
\trailingcitation{[jxx-p110923l-1.308]}
\xe

(\ref{ex:Qkuyena-7}) allows a manner and reason reading. The question has no verb, but an equative clause is attached to the question word. It was produced by Clara and refers to Swintha’s dreadlocks.

\ea\label{ex:Qkuyena-7}
\begingl
\glpreamble ¿kuyenakena eka chimukiji eka?\\
\gla kuyena-kena eka chi-muki-ji eka\\
\glb how-\textsc{uncert} \textsc{dem}a 3-hair-\textsc{col} \textsc{dem}a\\
\glft ‘how can her hair be like this?’\\or: ‘why is her hair like this?’
\endgl
\trailingcitation{[cux-c120414ls-2.345]}
\xe

The last two examples in this section rather request reason than manner and both make use of deranked verbs.

(\ref{ex:Qkuyena-6}) comes from the story about the fox and the jaguar as told by Miguel. This is what the jaguar asks the vulture, when he discovers that the vulture, who was supposed to watch over the fox, let him escape.

\ea\label{ex:Qkuyena-6}
\begingl
\glpreamble “¿chikuyena eka pikujikiuchi eka kupisaÿrÿ?”\\
\gla chikuyena eka pi-kujik-i-u-chi eka kupisaÿrÿ\\
\glb how \textsc{dem}a 2\textsc{sg}-let.go-\textsc{subord}-\textsc{real}-3 \textsc{dem}a fox\\
\glft ‘“why did you let the fox go?”’
\endgl
\trailingcitation{[jmx-n120429ls-x5.179]}
\xe


Finally, (\ref{ex:Qkuyena-8}) was elicited from Miguel to be able to ask him questions about the process of baking rice bread that we had filmed.

\ea\label{ex:Qkuyena-8}
\begingl
\glpreamble ¿chikuyena eka penukiuchÿ echÿu merÿpune naka latakÿye?\\
\gla chikuyena eka pi-nuk-i-u-chÿ echÿu merÿ-pune naka lata-kÿ-yae\\
\glb how \textsc{dem}a 2\textsc{sg}-put-\textsc{subord}-\textsc{real}-3 \textsc{dem}b plantain-leave here metal.sheet-\textsc{clf:}bounded-\textsc{loc}\\
\glft ‘why do you put plantain leaves on the baking tray?’
\endgl
\trailingcitation{[mxx-e120415ls.058]}
\xe


\subsubsection{Questions for quantities}\label{sec:Q_kajane}

Questions for quantities build on the question word\is{question word|(} \textit{kajane} ‘how many’. It is composed of a root \textit{ka-} and the \isi{distributive} marker \textit{-jane}. As for \textit{ka-}, this is possibly the same root we find in the copula \textit{kaku} and the non-verbal motion predicate \textit{kapunu} ‘come’. This root might be related to the demonstrative \textit{eka} (see \sectref{sec:DemPron}). In some cases, \textit{-chÿ} is added to the question word. Unlike the similar sequence added to other question words, this is never pronounced \textit{-chÿu}, so that there is no reason to believe that it could be a demonstrative. Consequently \textit{-chÿ} is glossed as a third person marker here.\is{person marking}
Sometimes \textit{u-} is placed before the question word and in that case, the marker \textit{-chÿ} always follows, yielding \textit{ukajanechÿ}.\footnote{The word patterns with \textit{punachÿ} ‘other’, which is also sometimes given as \textit{upunachÿ}. When \textit{u-} is added, those words have the regular \isi{iambic} pattern of polysyllabic words (see \sectref{sec:IambicPattern}), while \textit{kajane} and \textit{punachÿ} are both irregularly stressed\is{stress} on the first syllable. This hints at \textit{u} having been a fixed part of the root once.}\is{question word|)}

There are not many examples in the corpus which stem from one of the speakers (there are some more examples that were produced by the researchers though). All but one refer to quantities of countable things.

The only example in which the quantity of a solid object is requested is (\ref{ex:qquan-1}), which was elicited from Miguel, when Swintha wanted to ask him how many baking trays of rice bread he had baked.

\ea\label{ex:qquan-1}
\begingl
\glpreamble ¿kajane latajane?\\
\gla kajane lata-jane\\
\glb how.many metal.sheet-\textsc{distr}\\
\glft ‘how many baking trays?’
\endgl
\trailingcitation{[mxx-e120415ls.093]}
\xe

The question word is used to request age, as in (\ref{ex:qquan-2}), which was elicited from Isidro.

\ea\label{ex:qquan-2}
\begingl
\glpreamble ¿kajane anyu pitÿpi?\\
\gla kajane anyo pi-tÿpi\\
\glb how.many year 2\textsc{sg}-\textsc{obl}\\
\glft ‘how old are you?’
\endgl
\trailingcitation{[dxx-d120416s.073]}
\xe

The same question was translated by María S. using \textit{ukajanechÿ}, see (\ref{ex:qquan-22}). It is not clear whether there is any difference to (\ref{ex:qquan-2}).

\ea\label{ex:qquan-22}
\begingl
\glpreamble ¿ukajanechÿtu anyo pitÿpi?\\
\gla ukajane-chÿ-tu anyo pi-tÿpi?\\
\glb how.many-3-\textsc{iam} year 2\textsc{sg}-\textsc{obl}\\
\glft ‘how old are you?’
\endgl
\trailingcitation{[rxx-e151017l]}
\xe

(\ref{ex:qquan-3}) comes from the story about the fox and the jaguarundi. The fox asks how many different jumps the jaguarundi knows and the latter answers him that he knows only one, while the fox brags about knowing twenty-five. Nonetheless, it is this one jump that saves the jaguarundi later; he escapes on a tree, while the fox is killed by dogs.

\ea\label{ex:qquan-3}
\begingl
\glpreamble “¿kajane eka piyae lanse?” tikechuchÿji\\
\gla kajane eka pi-yae lanse ti-kechu-chÿ-ji\\
\glb how.many \textsc{dem}a 2\textsc{sg}-\textsc{grn} jump 3i-say-3-\textsc{rprt}\\
\glft ‘“how many jumps do you know?” he said to him, it is said’
\endgl
\trailingcitation{[jmx-n120429ls-x5.355]}
\xe

The one example in which \textit{kajane} rather refers to a non-countable noun is given in (\ref{ex:qquan-4}).  Although the semantics of the \isi{distributive} marker rather seems to impede requesting quantity of a \isi{mass noun}, this example suggests that it is possible. This is ultimately a matter of semantic extension.
Juana asks for the quantity of money that I had to pay for my flight to Germany.

\ea\label{ex:qquan-4}
\begingl
\glpreamble ¿kajanechÿ eka tÿmue?\\
\gla kajane-chÿ eka tÿmue\\
\glb how.many-3 \textsc{dem}a money\\
\glft ‘how much money did you pay?’
\endgl
\trailingcitation{[jxx-p120430l-1.157]}
\xe

\subsubsection{Questions of the ‘what about’ type}\label{sec:Q_kena}\is{uncertainty|(}

The uncertainty marker \textit{kena} is not precisely a question word,\is{question word|(} but can be used to form very general questions. Just like the question word in the other kinds of content questions, \textit{kena} is placed in \isi{focus} position. It is usually followed by an NP\is{noun phrase} or sometimes by a relative clause,\is{relative relation} but not directly by a verb.\is{word order} \textit{Kena} signals that there is some uncertainty about a referent. This kind of question can be translated to English with ‘what about X?’. Depending on the context, \textit{kena} can be used to request the identity of someone, her disposition, health, activity etc.\is{question word|)} Some examples follow.

(\ref{ex:name-father}) was produced by María C. to follow up on questions about my parents. Shortly before, she had asked about the name of my mother, see (\ref{ex:what-name}), so it is clear that the information she is seeking is a name here, too.

\ea\label{ex:name-father}
\begingl
\glpreamble ¿kena pia?\\
\gla kena pi-a\\
\glb \textsc{uncert} 2\textsc{sg}-father\\
\glft ‘what about your father?’
\endgl
\trailingcitation{[uxx-p110825l.152]}
\xe

(\ref{ex:qkena-1}) is the counter question to (\ref{ex:qquan-3}). The jaguarundi has replied to the fox, telling him that he knows one jump, and now he wants to know about the repertoire of jumps of the fox.

\ea\label{ex:qkena-1}
\begingl
\glpreamble “¿i kenabi?” chikechuchÿji echÿu tisepiu\\
\gla i kena-bi chi-kechu-chÿ-ji echÿu tisepiu\\
\glb and \textsc{uncert}-2\textsc{sg} 3-say-3-\textsc{rprt} \textsc{dem}b jaguarundi\\
\glft ‘“and what about you?”, said the jaguarundi to him, it is said’
\endgl
\trailingcitation{[jmx-n120429ls-x5.360-361]}
\xe

(\ref{ex:qkena-2}) was produced by Juana when turning a page of the book with the \isi{frog story} and having a first look at the picture.

\ea\label{ex:qkena-2}
\begingl
\glpreamble ¿kena naka?\\
\gla kena naka\\
\glb \textsc{uncert} here\\
\glft ‘what do we have here?’
\endgl
\trailingcitation{[jxx-a120516l-a.033]}
\xe

In (\ref{ex:qkena-3}), Juana ponders about her daughter in Argentina, whom she has not seen in a while.

\ea\label{ex:qkena-3}
\begingl
\glpreamble ¿kenaja nijinepÿi?\\
\gla kena-ja ni-jinepÿi\\
\glb \textsc{uncert}-\textsc{emph}1 1\textsc{sg}-daughter\\
\glft ‘how may my daughter be doing?’
\endgl
\trailingcitation{[jxx-e120516l-1.022]}
\xe

Finally, (\ref{ex:Syn-REL-rep}) is an example in which \textit{kena} combines with a relative clause. María C. asks Miguel about a leaflet with information about the Paunaka Documentation Project, when Miguel had told her that she would receive another one.

\ea\label{ex:Syn-REL-rep}
\begingl
\glpreamble ¿i kena echÿu chinejiku ukuinebu?\\
\gla i kena echÿu chi-nejiku ukuinebu\\
\glb and \textsc{uncert} \textsc{dem}b 3-leave some.time.ago\\
\glft ‘and what about the one he left some days ago?’
\endgl
\trailingcitation{[mux-c110810l.131]}
\xe
\is{uncertainty|)}
\is{content question|)}
\is{interrogative clause|)}



This is the end of the description of simple clauses. The chapter that follows is about different kinds of combinations of clauses and predicates.
