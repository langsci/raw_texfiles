%!TEX root = 3-P_Masterdokument.tex
%!TEX encoding = UTF-8 Unicode

\chapter{The noun and the NP}\label{chapter:Nouns}
\is{noun|(}

This chapter is about nouns and NPs in their primary, referential use. Nouns can also act as predicates, which is described in \sectref{sec:NonVerbalPredication}.%\footnote{In non-verbal predication, nouns can also take TAME markers, which are described in \ref{sec:OperationsPredicates}.} 

First of all, the composition of simplex and complex noun stems is discussed in \sectref{sec:SimplexNouns} and \sectref{sec:ComplexNouns}. There are different processes for deriving\is{derivation} a complex noun from a \isi{nominal root}, among them repetition, \isi{compounding} and addition of classifiers.\is{classifier} Nominalisation\is{nominalisation} of verbs is a marginal strategy.

Three different classes of nouns can be distinguished by their interaction with \isi{possession} marking: inalienable, alienable and non-possessable nouns. Possession marking with these three types of nouns is the topic of \sectref{sec:Possession}. The second main division in nouns concerns human and non-human nouns\is{animacy} and their possibilities to combine with \isi{plural} and other number markers. This is described in \sectref{sec:NumberNouns}. There is no grammatical \isi{gender}.

Nominal irrealis is the topic of \sectref{NominalRS}. This process is comparable to the better-known \isi{nominal tense} marking, but in \isi{nominal irrealis} marking, the referent is non-existent but presupposed. “Deceased”\is{deceased marking} is a category marked on kinship terms and personal names in reference to people who have passed away (see \sectref{sec:Deceased}). This could be considered a case of specialised \isi{nominal tense}. The diminutive is discussed in \sectref{sec:Diminutives}. It occurs not only on nouns, but also on verbs and other parts of speech,\is{word class} but it most often relates to a referent, not to predication.

There is no core-case marking (flagging). A few prepositions\is{preposition} encode \isi{oblique} relations, and there is a general \isi{locative marker} \textit{-yae}. In order to express more specific spatial notions, special locative noun stems are used in \isi{juxtaposition} with a noun referring to the ground. This is the topic of \sectref{sec:Locative}. Information on the NP is given wherever it seems relevant to the topic discussed, but \sectref{sec:NP} summarises all this information and provides a unified description of the NP.

A schema of the noun including all markers that can attach to it in referential use is given in \figref{fig:NounTemplate}. The \isi{locative marker} occurs in two different slots. This is related to the fact that it precedes the diminutive\is{diminutive} marker and follows the \isi{plural} marker, while the diminutive always precedes the plural marker. Unfortunately, there are no examples in the corpus in which all three (diminutive, plural and locative) occur on the noun.\footnote{Examples with locative and plural or locative and diminutive are also rare.}

\begin{figure}[!ht]
\centering
%\includegraphics[width=\textwidth]{figures/NounTemplate.png}
\includegraphics[width=\textwidth]{figures/NounTemplate-new.pdf}
\caption{Template of a noun}
\label{fig:NounTemplate}
\end{figure}\is{agglutination}

\section{The simplex noun}\label{sec:SimplexNouns}
\is{nominal stem|(}

Compared to verbs, nouns have significantly less internal complexity. The great majority of simplex (i.e. non-derived non-composite) noun stems are di- or trisyllabic. In addition, there are a few mono- and tetrasyllabic stems. Some of the latter have a specific phonological structure (see \sectref{sec:WordStructure_subsection}). Table  \ref{table:noun-stems-simple} shows some non-possessed simplex noun stems with different numbers of syllables.

\begin{table}[htbp]
\caption{Simplex noun stems}

\begin{tabular}{lll}
\lsptoprule
& Noun stem & Gloss \cr
\midrule
Monosyllabic & \textit{mai} & stone\cr
& \textit{peÿ} & frog\cr
& \textit{yui} & bread\cr
Disyllabic & \textit{jimu} & fish\cr
& \textit{kuje} & moon\cr
& \textit{ÿku} & rain \cr
Trisyllabic & \textit{kimenu} & woods\cr
& \textit{kÿjÿpi} & manioc\cr
& \textit{uneku} & town\cr
Tetrasyllabic & \textit{ajumerku} & paper\cr
& \textit{kupisaÿrÿ} & fox\cr
& \textit{urupunu} & red brocket \cr
\lspbottomrule
\end{tabular}

\label{table:noun-stems-simple}
\end{table}

Among nouns, we find many loans from Spanish,\is{borrowing|(} especially nouns denoting objects and concepts that were introduced to the area by \textit{karay} in different points in time, such as \textit{baka} ‘cow’ from \textit{vaca}, \textit{arusu} ‘rice’ from \textit{arroz}, \textit{anyo} ‘year’ from \textit{año}, \textit{kupeta} ‘gun’ from \textit{escopeta}, etc. Some of them are phonologically more integrated than others, which may be a hint that they are older.

In addition, there are also a number of nouns borrowed from Bésiro\is{Bésiro|(}. When they are similar to Paunaka’s native nouns in phonemic and syllabic structure, they were mostly not recognised by me, since I do not speak Bésiro. As far as I can tell from checking a word list with 705 entries by \citet[]{Sans2010} and the vocabulary lists compiled by \citet[]{Pinto2010}, they are surprisingly low in number,\footnote{I did not check the complete dictionary by \citet[]{FussRiester1986}, which is of a different variety anyway.} but a more in-depth study may reveal that there are actually more loans.\footnote{For instance, I immediately identified a few more loans when reading the article by \citet[]{Nikulin2019} about Proto-Chiquitano.} A number of nouns with a Spanish origin must have entered Paunaka via Bésiro, easily detectable if they contain the sounds [ʂ] and [ʃ], which are not part of the phonemic inventory of native Paunaka words. One of these loans is \textit{remonixhi} ‘lemon’ from Bésiro \textit{nermónixhi} from Spanish \textit{limón}. The Bésiro noun contains the prefix \textit{n-}, which is preposed to nominal roots starting with a vowel\footnote{The Spanish word \textit{limón} does not start with a vowel. However, it is likely that the process of borrowing was as follows: Bésiro speakers replaced Spanish /l/, which is not part of the phonemic inventory of Bésiro, by /ɾ/. Since /ɾ/ does not usually occur word-initially, an epenthetic vowel was inserted or the syllable was metathesised and then the nasal prefix was attached (compare \citealt[]{Sans2010,Sans2013} for Bésiro phonology).} and the “general case” suffix \mbox{\textit{-xhi}} \citep[20]{Sans2013}. Both are typical for Bésiro nouns. While the prefix is detached in Paunaka – and the first two sounds metathesised again, reflecting the Spanish original –, the suffix is maintained. This cannot be considered a general pattern though, and there is only a small number of nouns with final \textit{xhi} (or similar sounds) in Paunaka. Most loanwords are integrated differently.\footnote{For instance, the noun ‘orange’, Spanish \textit{naranja}, is \textit{narankaxÿ} in Paunaka, with the [ʂ] sound suggesting that it was borrowed via Bésiro, too, while ‘tangerine’ is \textit{mantarina} from Spanish \textit{mandarina}.}\is{Bésiro|)}

Two animal terms that are widespread in the area are found in Paunaka as well, where they have the specific forms \textit{takÿra} ‘chicken, hen’ and \textit{kabe} ‘dog’. \textit{Merÿ} ‘plantain’ and \textit{patabi} ‘sugarcane’ are probably borrowed from Guarayu.\is{borrowing|)} Besides borrowing, new lexemes are created by word formation processes. This is the topic of the following section.

%loans from Spanish:jente, baka, arusu
%loans from Bésiro: aitubuche, mukianka, xhikuera, xhabu
%widespread terms takÿra and kabe
%Guarayo? merÿ, patabi

\section{The complex noun}\label{sec:ComplexNouns}

There are several processes that produce complex nouns in Paunaka, but none of them is very productive. A few nouns with repeated syllables are found (see \sectref{sec:RDPL_Nouns}). Compounding\is{compounding} is largely restricted to plant parts (this is the topic of \sectref{sec:Compounding}). Attachment of classifiers\is{classifier} to a noun or verb stem is similar to compounding (this is discussed in \sectref{sec:Nouns_CLF}). A different derivational pattern is found with some parts of body parts (see \sectref{sec:BodyPartofPartDerivation}). Finally, some nouns are derived from verbs with nominalisers (see \sectref{sec:MorphologyNominalisation}).

\subsection{Repetition}\label{sec:RDPL_Nouns}

Some noun stems contain repeated syllables, but this is hardly productive, and thus does not fall under the concept of \isi{reduplication} 
(\citealp[cf.][13]{Rubino2005}; \citealt[2]{GomezVoort2014}). \tabref{table:noun-stems-RDPL} shows some noun stems with repeated syllables. 

\begin{table}[htbp]
\caption{Repetition in noun stems}

\begin{tabular}{ll}
\lsptoprule
Noun stem & Gloss \cr
\midrule
\textit{barereki} & (clay) pot\cr
\textit{churupepe} & butterfly\cr
\textit{jupipi} & liana\cr
\textit{mimi} & mum (endearment form)\cr
\textit{pujukeke} & patasca (food)\cr
\textit{pÿrÿsÿsÿ} & armadillo sp.\cr
\textit{-sÿsÿ} & nose\cr
\textit{-tabubuji} & branches\cr
\textit{tupapana} & soursop \cr
\textit{yeye} & granny (endearment form), old lady \cr
\textit{yÿkÿkekeji} & branches\cr
\lspbottomrule
\end{tabular}

\label{table:noun-stems-RDPL}
\end{table}

%sÿrÿpÿtÿtÿ = picaflor ; sepitekÿrÿrÿ = picaflor; sumurukuku 'bird sp.'

Repetition of more than one syllable is extremely rare. One example with repetition of the last two syllables is \textit{pichikurakura} ‘thrush-like wren’, a bird species with the scientific name Cam\-py\-lo\-rhyn\-chus turdinus, which seems to be named after the sound it makes when singing. Another one is \textit{pasipasi} ‘sand fly sp.’, which might be onomatopoetic,\is{onomatopoeia} too, considering the buzzing sound these insects make. This is the complete list of nouns with more than one repeated syllable I found in the corpus. As for \textit{pasipasi}, it is also one of the few examples of full repetition. Most other cases of full repetition are disyllabic words, like the \isi{endearment} forms \textit{mimi} ‘mum, mother’ and \textit{yeye} ‘granny, grandmother, old lady’.

\largerpage
In most cases, there is no corresponding noun without the repeated syllable. However, corresponding to two words at the bottom of the table, \textit{-tabubuji} and \textit{yÿkÿkekeji}, there are also \textit{yÿkÿke} ‘tree, wood, stick’ and \textit{-tabu} ‘branch, twig’. Both words including the repetition express a multitude of branches and twigs in the crown of a tree. Repetition is probably triggered by the \isi{collective} marker in this case (see \sectref{sec:Collective}). There is also repetition of the general classifier on adjectives if the collective marker is added, see \sectref{sec:Adjectives}.


\subsection{Compounding}\label{sec:Compounding}\is{compounding|(}
\largerpage
Compounding is not very productive in Paunaka, the only exception being compounds of a plant name and a plant part, especially leaves of plants.\footnote{The noun or \isi{classifier} for fruits (\textit{-i}) is possibly too short to be recognised as a proper stem.\is{nominal stem} Miguel produced some compounds with \textit{-i} in elicitation, but his sister María S. did not and she even claimed that the compounds produced by her brother were incorrect. There is at least one generally agreed upon compound with \textit{-i}, \textit{ichÿi} ‘tree calabash (Crescentia cujete)’. The general word for ‘fruit’ is \textit{chÿi}, which is easily decomposable into the third person marker \textit{chÿ-} and the noun \textit{-i}.} The plant name occurs in N1 position and it modifies\is{modification} the plant part in N2 position. The nouns in N2 position are always inalienably possessed,\is{inalienability} as in (\ref{ex:plant-leaves}).

\ea\label{ex:plant-leaves}
  \ea\label{ex:plant-leaves.1}
\begingl
\glpreamble merÿpune\\
\gla merÿ-pune\\
\glb plantain-leaf\\
\glft ‘plantain leaf’
\endgl
  \ex\label{ex:plant-leaves.2}
\begingl
\glpreamble santiapune\\
\gla santia-pune\\
\glb watermelon-leaf\\
\glft ‘watermelon leaf’
\endgl
  \ex\label{ex:plant-leaves.3}
\begingl
\glpreamble amekaba\\
\gla ame-kaba\\
\glb palm.sp-palm.leaf\\
\glft ‘leaf of \textit{motacú} palm (\textit{Attalea princeps})’
\endgl
  \ex\label{ex:plant-leaves.4}
\begingl
\glpreamble kuyaekaba\\
\gla kuyae-kaba\\
\glb palm.sp-palm.leaf\\
\glft ‘leaf of \textit{totaí} palm (\textit{Acrocomia aculeata})’
\endgl
\z
\xe

In addition, some seeds of plants can be expressed by compounds, as in (\ref{ex:plant-seeds}).


\ea\label{ex:plant-seeds}
  \ea\label{ex:plant-seeds.1}
\begingl
\glpreamble kÿikemuke\\
\gla kÿike-muke\\
\glb peanut-seed\\
\glft ‘peanut seed’
\endgl
  \ex\label{ex:plant-seeds.2}
\begingl
\glpreamble eniyemuke\\
\gla eniye-muke\\
\glb achiote-seed\\
\glft ‘achiote seed’
\endgl
\trailingcitation{\citep[11, 19]{Sell2019}}
\z
\xe


As for other plant parts, speakers rather use complex NPs\is{noun phrase} with a possessed\is{possession} form of the plant part followed by the plant name, as in (\ref{ex:plant-part-NP}), which is from Juana’s telling of the \isi{frog story}.

\ea\label{ex:plant-part-NP}
\begingl
\glpreamble chÿabÿkÿkutu chitabu yÿkÿke\\
\gla chÿ-abÿkÿku-tu chi-tabu yÿkÿke\\
\glb 3-hold-\textsc{iam} 3-branch tree\\
\glft ‘he is holding to a branch of a tree’
\endgl
\trailingcitation{[jxx-a120516l-a.162]}
\xe

The word for ‘chicken egg’ is a compound, too (see (\ref{ex:chicken-egg})), but it can be considered a lexicalised compound,\is{lexicalisation} since other eggs of animals are rather expressed periphrastically, as in (\ref{ex:turtle-egg}). %The compound only denotes eggs, not little chicks. In reference to the latter, a diminutive marker is attached to the noun for ‘chicken’ yielding \textit{takÿramÿnÿ}.

\ea\label{ex:chicken-egg}
\begingl
\glpreamble takÿrachecha\\
\gla takÿra-checha\\
\glb chicken-son\\
\glft ‘chicken egg’
\endgl
\xe

\ea\label{ex:turtle-egg}
\begingl
\glpreamble chichecha kipÿ\\
\gla chi-checha kipÿ\\
\glb 3-son tortoise\\
\glft ‘tortoise egg’
\endgl
%\trailingcitation{[rxx-e121128s-1.086]}
\xe


There are many compounds consisting of a human noun,\is{endearment|(}  most of the times a kinship term, in N1 position, and the possessed noun \textit{-pÿi} ‘body’ in N2 position. In these constructions, \textit{-pÿi} does not alter the lexical meaning of the compound. It rather signals affection or sympathy for the N1, similar to a \isi{diminutive}.\footnote{Note that \textit{-pÿi} and the diminutive marker \textit{-mÿnÿ} are not mutually exclusive. Quite the contrary is true: Human noun compounds with \textit{-pÿi} in N2 position often take diminutive marking, too.} It is mainly used with human nouns denoting people of younger age than the speaker, although, when asked, one speaker claimed that it is also possible to use a compound with \textit{-pÿi} in reference to older people than oneself, like the own mother or father. Since the semantic connection of these compounds to the body-part term is totally opaque, \textit{-pÿi} could also be analysed as a derivational suffix.\footnote{Interestingly, in Apurinã, an Arawakan language\is{Arawakan languages} in Brazil, a similar process might be at work: \citet[15]{Pickering2009} provides a gloss ‘private, esteemed’ for a form \textit{mane}, which is translated as ‘body of’ by \citet[265]{Facundes2000}. The latter nevertheless disagrees with Pickering’s analyses stating that its only meaning is ‘body’. The form seems to occur as a fixed part of two lexemes, one kinship term and one other human noun.}
 In (\ref{ex:buddy-body}), some examples for human nouns composed with \textit{-pÿi} are given.\footnote{\textit{Aitubuche} ‘boy, young man’ is a loan from \isi{Bésiro}. I am actually not sure whether it is used in present-day Bésiro. It is not included in the word list by \citet[]{Sans2011} nor in the dictionary by \citet[]{FussRiester1986}, but \citet[479]{Adelaar2004} offer the form \textit{aɨtoboti} ‘his stepson’, which they found in historical data. This, I believe, is the source of the Paunaka noun.} In the remainder of this work, the forms are usually not decomposed in examples.

\ea\label{ex:buddy-body}
  \ea\label{ex:buddy-body.1}
\begingl
\glpreamble nichechapÿi\\
\gla ni-checha-pÿi\\
\glb 1\textsc{sg}-son-body\\
\glft ‘my son’
\endgl
  \ex\label{ex:buddy-body.2}
\begingl
\glpreamble aitubuchepÿi\\
\gla aitubuche-pÿi\\
\glb boy-body\\
\glft ‘boy’
\endgl
  \ex\label{ex:buddy-body.3}
\begingl
\glpreamble apimiyapÿi\\
\gla apimiya-pÿi\\
\glb girl-body\\
\glft ‘girl’
\endgl
  \ex\label{ex:buddy-body.4}
\begingl
\glpreamble nÿatipÿi\\
\gla nÿ-ati-pÿi\\
\glb 1\textsc{sg}-brother-body\\
\glft ‘my brother (of a woman)’
\endgl
\z
\xe

When the \isi{collective} marker is added, \textit{-pÿi} is detached from most forms, but it has completely lexicalised\is{lexicalisation} with the words \textit{-jinepÿi} ‘daughter’ and \textit{-sinepÿi} ‘grandchild’, i.e. these words are never found without \textit{-pÿi}. Consequently, they do not take the collective marker (see \sectref{sec:Collective}).
The first part of \textit{-jinepÿi}, \textit{*jine}, has cognates in other Southern Arawakan languages, the addition of \textit{-pÿi} is an innovation that is only found in Paunaka.% Baure \textit{-jin}, Kinikinao \textit{-ihine} \textit[146]{Souza2008}
\is{endearment|)} 

Finally, a lot of lexicalised\is{lexicalisation} compounds are found among body part terms, especially for bones and hair \citep[255]{TerhartDanielsenBODY}. While \textit{-jiyu} ‘body hair’ can also occur in non-compound forms,\footnote{The noun is inalienably possessed and thus underlies the conditions specified in \sectref{sec:Inalienables}.} the sequence \textit{-chupu/-chupea}, which is found in compounds denoting bones, is never used as a non-compound lexeme. 

In body part compounds, strikingly, the order of modifier\is{modification} and modified noun seems to be reversed, with N1 being the semantic \isi{head}. However, we cannot be sure, how these body parts were perceived originally, so, instead of ‘face hair’, the beard could have also been perceived as ‘hair face’, i.e. the part of the face with hair, so that the question which part is the semantic \isi{head} cannot be resolved here. (\ref{ex:jiyu}) lists a few examples for compounds with \textit{-jiyu} and (\ref{ex:chupu}) with \textit{-chupu}.

\ea\label{ex:jiyu}
  \ea\label{ex:jiyu.1}
\begingl
\glpreamble chijiyumama\\
\gla chi-jiyu-mama\\
\glb 3-hair-jaw\\
\glft ‘his jaw beard’
\endgl
  \ex\label{ex:jiyu.2}
\begingl
\glpreamble chijiyutaka\\
\gla chi-jiyu-taka\\
\glb 3-hair-armpit\\
\glft ‘his/her armpit hair’
\endgl
  \ex\label{ex:jiyu.3}
\begingl
\glpreamble chijiyunÿkÿ\\
\gla chi-jiyu-nÿkÿ\\
\glb 3-hair-mouth\\
\glft ‘his mustache’
\endgl
\z
\xe

\ea\label{ex:chupu}
  \ea\label{ex:chupu.1}
\begingl
\glpreamble chichupupiÿnÿ\\
\gla chi-chupu-piÿnÿ\\
\glb 3-bone-neck\\
\glft ‘his/her cervicals’
\endgl
  \ex\label{ex:chupu.2}
\begingl
\glpreamble chichuputÿi\\
\gla chi-chupu-tÿi\\
\glb 3-bone-anus\\
\glft ‘his/her tailbone’
\endgl
  \ex\label{ex:chupu.3}
\begingl
\glpreamble chichupukekÿ\\
\gla chi-chupu-kekÿ\\
\glb 3-bone-back.of.animal\\
\glft ‘his/her spine’
\endgl
\z
\xe

Both \textit{-jiyu} ‘hair’ and \textit{-chupu} ‘bone’ (in this specific case with the slightly different form \mbox{\textit{-chupea}}) can also be combined in the word for eyebrow, literally ‘hair bone face’ (or ‘hairy part of the boney part of the face’), see (\ref{ex:eyebrow}).

\ea\label{ex:eyebrow}
\begingl 
\glpreamble chijiyuchupeabÿke\\
\gla chi-jiyu-chupea-bÿke\\ 
\glb 3-hair-bone-face\\ 
\glft ‘his/her eyebrow’\\ 
\endgl
\xe

The body part noun \textit{-bÿke} ‘face’ forms part of the exocentric compounds denoting cardinal directions \citep[266]{TerhartDanielsenBODY}. The expressions are given in (\ref{ex:cardinals}).\footnote{There is no noun \textit{*kuju}, but the form may be related to \textit{tujubeiku} ‘wind’, which is most probably a \isi{verb} structurally. Note that the wind usually comes from the North, unless it is cold wind from the South (\textit{tisÿeipu} ‘south wind, cold weather coming from the South’).}

\ea\label{ex:cardinals}
  \ea\label{ex:cardinals.1}
\begingl
\glpreamble manebÿke\\
\gla mane-bÿke\\
\glb morning-face\\
\glft ‘East’
\endgl
  \ex\label{ex:cardinals.2}
\begingl
\glpreamble kupeibÿke\\
\gla kupei-bÿke\\
\glb afternoon-face\\
\glft ‘West’
\endgl
  \ex\label{ex:cardinals.3}
\begingl
\glpreamble kuju-bÿke\\
\gla kuju-bÿke\\
\glb wind?-face\\
\glft ‘North’
\endgl
  \ex\label{ex:cardinals.4}
\begingl
\glpreamble tisÿeibÿke\\
\gla ti-sÿei-bÿke\\
\glb 3i-be.cold-face\\
\glft ‘South’
\endgl
\z
\xe
\is{compounding|)}

\subsection{Derivation of nouns with classifiers}\label{sec:Nouns_CLF}\is{derivation|(}\is{classifier|(}

Classifiers can derive nouns from other nouns and verbs.\is{verb} They are often completely lexicalised\is{lexicalisation|(} on the noun and cannot be detached. There is, for example, the word pair \textit{mutepa} ‘dust, earth’ and \textit{muteji} ‘loam, mud’. Both must derive from a stem \textit{*mute}, but there is no such noun (or verb) in the language – at least not synchronically.\is{lexicalisation|)} A list of all classifiers I could identify is given in \sectref{sec:Classifiers}.

(\ref{ex:noun-clf}) and (\ref{ex:verb-clf}) list words in which the derivational process is still transparent, because the stems\is{nominal stem} are also found without classifiers. The nouns in (\ref{ex:noun-clf}) are results of the combination of a noun with a classifier, while (\ref{ex:verb-clf}) lists two nouns which are derived from verb stems with the help of classifiers.

\ea\label{ex:noun-clf}
  \ea\label{ex:noun-clf.1}
\begingl
\glpreamble yÿkÿke\\
\gla yÿkÿ-ke\\
\glb fire-\textsc{clf:}cylindrical\\
\glft ‘tree, stick’
\endgl
  \ex\label{ex:noun-clf.2}
\begingl
\glpreamble chicheneumu\\
\gla chi-chene-umu\\
\glb 3-breast-\textsc{clf:}liquid\\
\glft ‘milk’
\endgl
  \ex\label{ex:noun-clf.3}
\begingl
\glpreamble yÿbapa jimupa\\
\gla yÿbapa jimu-pa\\
\glb flour fish-\textsc{clf:}particle\\
\glft ‘fish flour’
\endgl
  \ex\label{ex:noun-clf.4}
\begingl
\glpreamble kechuepi\\
\gla kechue-pi\\
\glb snake-\textsc{clf:}long.flexible\\
\glft ‘worm’
\endgl
\z
\xe

\ea\label{ex:verb-clf}
  \ea\label{ex:verb-clf.1}
\begingl
\glpreamble nijikupupi\\
\gla ni-jikupu-pi\\
\glb 1\textsc{sg}-swallow-\textsc{clf:}long.flexible\\
\glft ‘my gullet’
\endgl
  \ex\label{ex:verb-clf.2}
\begingl
\glpreamble chimukuji\\
\gla chi-muku-ji\\
\glb 3-sleep-\textsc{clf:}soft.mass\\
\glft ‘its nest’
\endgl
\z
\xe

There are also some idiosyncratic forms. The word \textit{ÿneumu} may be composed of \textit{ÿne} ‘water’ and \textit{-umu}, the classifier for liquids. Its meaning, however, is ‘inside the water’.\footnote{It could also include the same suffix found in \textit{anÿmu} ‘sky’ (opposed to \textit{anÿke} ‘up, above’), but then the first \textit{u} of \textit{ÿneumu} would be unexplained.} This word seems to be quite old, it was possibly already used in the Proto language of the Bolivian Arawakan\is{Southern Arawakan} languages, considering that we find a similar word in Old Mojeño,\is{Mojeño languages}\footnote{Old Mojeño is the variety of Mojeño that was documented by the Jesuits in the late 17th century in a grammar and catechism \citep[]{Marban1894}.} <uneamukû> ‘inside of the water’ \citep[94]{Marban1894}, although this one includes the boundedness classifier <-kû>  – whose cognate form \textit{-kÿ} is also found in the corpus on \textit{ÿneumu} once. The latter is a reflex of Proto-Arawakan\is{Arawakan languages} \textit{*-Vku} \citep[cf.][384]{Payne1991}. It expresses boundedness, i.e. nouns with this classifier are perceived as having boundaries, as a kind of “container”. It is a special case of classifier, since it is often used in locative expressions only, see \sectref{sec:Locative}, but also found lexicalised\is{lexicalisation} with some nouns outside of such contexts, e.g. the word \textit{chenekÿ} ‘way, path’ occurs with \textit{-kÿ} in locative and non-locative contexts.

I have claimed elsewhere \citep[178--180]{Terhart2016} that the two word formation processes of \isi{compounding} two noun stems and combination of a noun with a classifier can be seen as two ends of a continuum, where nouns have a more concrete lexical meaning and can typically occur on their own (or with a person marker if they are inalienably possessed), and classifiers have a broader meaning, mostly based on shape, and can never stand on their own, which makes them more reminiscent of derivational affixes,\is{affix} yet with a relatively concrete meaning.
\is{classifier|)}

\subsection{Derivation of parts of body parts}\label{sec:BodyPartofPartDerivation}

There is one derivational prefix \textit{ke-}, which attaches to a few body part terms to derive a part of this body part \citep[]{TerhartDanielsenBODY}. The process is not productive and restricted to the nouns listed in (\ref{ex:ke-deriv}).

\ea\label{ex:ke-deriv}
  \ea
 \begingl 
\glpreamble nibÿke – nikebÿke\\
\gla ni-bÿke ni-ke-bÿke\\ 
\glb 1\textsc{sg}-face 1\textsc{sg}-\textsc{der}-face\\ 
\glft ‘my face – my eye(s)’\\ 
\endgl
  \ex
 \begingl
\glpreamble nibuÿ – nikebuÿ\\
\gla ni-buÿ ni-ke-buÿ\\ 
\glb 1\textsc{sg}-hand 1\textsc{sg}-\textsc{der}-hand\\ 
\glft ‘my hand(s) – my finger(s)’\\ 
\endgl
  \ex
 \begingl
\glpreamble nibu – nikeibu\\
\gla ni-ibu ni-ke-ibu\\
\glb 1\textsc{sg}-foot 1\textsc{sg}-\textsc{der}-foot\\
\glft ‘my foot (feet) – my toe(s)’
\endgl
\z
\xe

The same derivation pattern seems to be at work in derivation of the word ‘tail’ from ‘wing’ as in (\ref{ex:ke-deriv-animal}), although the semantic relationship between those animal body parts is not the same as for the human body parts in (\ref{ex:ke-deriv}), since a tail is not a part of the wings. An animal does not even necessarily have to have wings in order to have a tail.

\ea\label{ex:ke-deriv-animal}
\begingl 
\glpreamble chisi – chikeisi\\
\gla chÿ-isi chi-ke-isi\\ 
\glb 3-wing 3-\textsc{der}-wing\\ 
\glft ‘its wing(s) – its tail’\\ 
\endgl
\xe


\subsection{Nominalisation}\label{sec:MorphologyNominalisation}
\is{nominalisation|(}

A few nouns result from nominalisation with the suffix \textit{-kene}, they are listed in (\ref{ex:NomiNouns-1}). All of them are inalienably possessed\is{inalienability} and given here with the first person plural possessor. They are objective nouns \citep[cf.][]{ComrieThompson2007}, i.e. the patient\is{patient/theme} is nominalised.


\ea\label{ex:NomiNouns-1}
  \ea\label{ex:NomiNouns-1.1}
\begingl
\glpreamble bejumikene\\
\gla bi-ejumi-kene\\
\glb 1\textsc{pl}-remember-\textsc{nmlz}\\
\glft ‘our thoughts’
\endgl
  \ex\label{ex:NomiNouns-1.2}
\begingl
\glpreamble bupukene\\
\gla bi-upu(nu)-kene\\
\glb 1\textsc{pl}-bring-\textsc{nmlz}\\
\glft ‘our load’
\endgl
  \ex\label{ex:NomiNouns-1.3}
\begingl
\glpreamble bichabukene\\
\gla bi-chabu-kene\\
\glb 1\textsc{pl}-do-\textsc{nmlz}\\
\glft ‘our actions/deeds’
\endgl
\z
\xe

In addition, further nouns that may or may not contain the nominaliser are \textit{kuchepukene} ‘sorcerer’, \textit{-akene}/\textit{-ekene} ‘non-visible side’, in both cases no form without \textit{-kene} is known to me.\footnote{Rose (2021, p.c.) relates \textit{kuchepukene} ‘sorcerer’ to the noun for ‘bone’ as in \isi{Mojeño Trinitario}, where the form of the nominaliser is \textit{-giene} (while Ignaciano\is{Mojeño Ignaciano} has \textit{-kene} \citep[cf.][663--672]{OlzaZubiri2004}). ‘Bone’ is \textit{eupe} or \textit{-upeji} in Paunaka, thus the relation is not straightforward in this language. She further relates \textit{-akene} ‘non-visible side’ to a different etymon given the fact that there are several forms in Trinitario\is{Mojeño Trinitario} containing the sequence \textit{giene} that mean ‘follow’ or ‘behind’.} The word \textit{tijaikenekÿu} ‘dawn’ is derived from \textit{tijai} ‘it is light, day’, seemingly with \textit{-kene} and the translocative concurrent motion marker\is{associated motion} \textit{-kÿu}, but this does not make much sense to me, since this marker usually encodes motion away from the scene (‘the light that is going’?). Note that \textit{tijai} is formally a \isi{verb}, although used like a noun in most cases (see \sectref{sec:UnmarkedRC}).

One further example of a nominalised verb form used referentially (as S of a non-verbal predicate) has been found in the data collected by Riester: (\ref{ex:NMLZ-r3}), which is about scarcity of food due to a drought. 

\ea\label{ex:NMLZ-r3}
\begingl
\glpreamble nechikue sepitÿjiku tanÿma eka binikeneina\\
\gla nechikue sepitÿ-jiku tanÿma eka bi-ni-kene-ina\\
\glb therefore small-\textsc{lim}1 now \textsc{dem}a 1\textsc{pl}-eat-\textsc{nmlz}-\textsc{irr.nv}\\
\glft ‘therefore we have little (possible) food now’
\endgl
\trailingcitation{[nxx-a630101g-1.38-39]}
\xe

The form \textit{-nikene}\is{non-verbal irrealis marker|(} ‘food’ has not been found with the speakers I worked with. And there is even more to it: when working on parts of the recordings by Riester together with Miguel, Juana and María S., they repeated the nominalised verb \textit{binikeneina} as \textit{binikukeneina}, i.e. including a \isi{thematic suffix}, see \sectref{sec:ActiveVerbs_TH}. This is not trivial, since the thematic suffix is the place of \isi{reality status} marking on active verbs (see \sectref{sec:RealityStatus}), while nouns take a different irrealis marker (see \sectref{NominalRS} and \sectref{sec:NonVerbalPredication}).  %(e.g. in rxx-e181024l)
There were a few more instances in the corpus where speakers used \textit{-nikukene}, all of them taking the non-verbal irrealis marker \textit{-ina}. One of them is (\ref{ex:emphi-3nmlz}) which is taken from a story by Miguel. The wife of the main character, who is very lazy, asks him to make a field, because they do not have food.

\ea\label{ex:emphi-3nmlz}
\begingl
\glpreamble “panajachÿu pario eka pisaneina kuina binikukeneina"\\
\gla pi-ana-ja-chÿu pario eka pi-sane-ina kuina bi-niku-kene-ina\\
\glb 2\textsc{sg}-make.\textsc{irr}-\textsc{emph}1-\textsc{dem}b some \textsc{dem}a 2\textsc{sg}-field-\textsc{irr.nv} \textsc{neg} 1\textsc{pl}-eat-\textsc{nmlz}-\textsc{irr.nv}\\
\glft ‘“make something for your field, we do not have any food”’
\endgl
\trailingcitation{[mox-n110920l.015]}
\xe
\is{non-verbal irrealis marker|)}

In addition to the nominaliser, there is a homophonous \isi{emphatic} marker \textit{-kene}, which is equally rare (see \sectref{sec:EmphMarker}).

A few nouns seem to be derived from verbs by a suffix \textit{-e}. I have found only three examples, which are given in (\ref{ex:NomiNouns-2}), all of them with the first person plural possessor. I first thought they were deranked verbs (see \sectref{sec:Subordination-i}) with a somehow inarticulate RS suffix, but there is a difference: the subordinating suffix \textit{-i} comes after the thematic suffix, and here, the thematic suffix is detached, the /i/ is part of the verb stem. Just like the derivation with \textit{-kene}, this process does not seem to be productive in Paunaka considering the small number of words with this suffix in the corpus.\footnote{It is noticeable that all verb stems begin with \textit{yÿ}. As for \textit{-yÿtiku} ‘set on fire (to cook)’ and \mbox{\textit{-yÿtipajiku}} ‘make/cook chicha’, they are certainly related, the second one containing the \isi{classifier} \textit{-pa} for dusty things and the intensive aktionsart suffix \textit{-ji}. The verb \textit{-yÿseiku} ‘buy’, however, could be a loan from Guarayu (Danielsen 2021, p.c.).} 


\ea\label{ex:NomiNouns-2}
  \ea\label{ex:NomiNouns-2.1}
\begingl
\glpreamble biyÿtie\\
\gla bi-yÿti-e\\
\glb 1\textsc{pl}-set.on.fire-\textsc{nmlz}\\
\glft ‘our food’
\endgl
  \ex\label{ex:NomiNouns-2.2}
\begingl
\glpreamble biyÿtipajie\\
\gla bi-yÿtipaji-e\\
\glb 1\textsc{pl}-make.chicha-\textsc{nmlz}\\
\glft ‘our chicha (still being cooked)’
\endgl
  \ex\label{ex:NomiNouns-2.3}
\begingl
\glpreamble biyÿseie\\
\gla bi-yÿsei-e\\
\glb 1\textsc{pl}-buy-\textsc{nmlz}\\
\glft ‘our purchas’
\endgl
\z
\xe

We can conclude that nominalisation is a very rare process of word formation in Paunaka. This is because speakers rather use headless relative clauses\is{relative relation} in those contexts in which a nominalised verb would be expected in other languages.
\is{nominalisation|)}\is{derivation|)}\is{nominal stem|)}

We will leave the inner constituency of nouns now and have a look at nominal inflection, starting from the next section, which is about possession marking.




