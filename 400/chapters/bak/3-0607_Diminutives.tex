%!TEX root = 3-P_Masterdokument.tex
%!TEX encoding = UTF-8 Unicode



\section{Diminutive}\label{sec:Diminutives}\is{diminutive|(}\is{derivation|(}

Paunaka has one diminutive marker, \textit{-mÿnÿ}. It can occur on nouns and verbs,\is{verb} as in (\ref{ex:new23-dim}), which was elicited from María S. and shows both of this. The diminutive also occasionally occurs on words belonging to other classes. 

\ea\label{ex:new23-dim}
\begingl
\glpreamble tibebeikubumÿnÿ michimÿnÿ\\
\gla ti-bebeiku-bu-mÿnÿ michi-mÿnÿ\\
\glb 3i-lie-\textsc{mid}-\textsc{dim} cat-\textsc{dim}\\
\glft ‘the little cute cat is lying (on a chair)’
\endgl
\trailingcitation{[rxx-e181024l.066]}
\xe

The form of the diminutive marker, \textit{-mÿnÿ}, reminds me of the diminutive found in Guarayu. Guarayu has \textit{mini} among other diminutives, which can be added to verbs and adjectives, in addition to nouns \citep[13]{Hoeller1932a}. Paunaka’s diminutive marker is also similar to the Trinitario\is{Mojeño Trinitario} one \textit{-samini} and \citet[174--175]{Rose2018} proposed that both forms are cognates. Note however, that Paunaka’s /ɨ/ is a reflex of \textit{*u} of a common ancestor language, and the process of fronting did not take place in the Mojeño languages \citep[cf.][418]{deCarvalhoPAU}. If the forms are cognates, we have to assume an unconditioned shift from \textit{*i} to /ɨ/ in Paunaka. The same holds for the hypothesis that the Paunaka diminutive is related to the Guarayu form. In Guarasu, another Tupi-Guarani language closely related to Guarayu, there is a diminutive marker \textit{-m\'{ɨ}nɨ} \citep[437]{RamirezAL2017}, which is identical to the Paunaka one, but we do not know whether the languages were in contact.

Besides the basic meaning of smallness, the diminutive marker can also express emotional values like affection and compassion\is{endearment} and attenuation, often for reasons of politeness or modesty. This extension from the core meaning has been reported to occur very frequently cross-linguistically \citep[535, 558]{Jurafsky1996}. Even if the diminutive is not attached to a noun, but to an adjective\is{adjective} or a verb,\is{verb} it is associated with a noun (or its referent) in most cases. Sometimes it can also attenuate the verb’s meaning. It is, however, often impossible to distinguish diminutive notions belonging to the referent from purely predicative attenuation, because a small, modest or pitied referent usually causes little action. In many cases, I just do not know what the speaker exactly wanted to express with the diminutive. This is why I decided not to treat the diminutive in different chapters -- unlike other markers of \isi{transcategorial morphology} like the person and number markers, the non-verbal irrealis and the remote past marker, where the different functions are more easily distinguished (at least in some cases).\footnote{I have explained possible cases of overlap in the preceding sections.} To compensate for this, the subsections on diminutive marking are ordered by word class. First, examples of diminutives on nouns are given in \sectref{sec:Diminuitves_Nouns}, while \sectref{sec:Diminuitves_Verbs} discusses use of the marker on verbs. In \sectref{sec:Diminuitves_OtherPOS}, occurrences of \textit{-mÿnÿ} with other parts of speech are presented.

The use of diminutives is not only very common in Paunaka, but also in Bolivian Spanish, where it has largely the same functions \citep[cf.][38]{Mendoza2015}, but not the same distribution, i.e. it cannot be used on verbs.

%Other than Ignaciano \textit{-chicha}, the Paunaka diminutive \textit{-mÿnÿ} does NOT appear in the same slot as the classifier and possessed marker.

\subsection{Diminutives on nouns}\label{sec:Diminuitves_Nouns}

Sometimes, the diminutive marker \textit{-mÿnÿ} clearly expresses its core meaning of smallness. This is often the best interpretation, when it is added to inanimate nouns. An example is (\ref{ex:dim-small}) from Miguel. The woman, a character of a narrative, brings some food to her husband, who is supposed to be working in the woods. She uses a small pot for transportation, not one of those huge ones that are sometimes used for cooking. 

\ea\label{ex:dim-small}
\begingl 
\glpreamble tumuji nÿkÿikimÿnÿji yÿtÿuku\\
\gla ti-umu-ji nÿkÿiki-mÿnÿ-ji yÿtÿuku\\ 
\glb 3i-take-\textsc{rprt} pot-\textsc{dim}-\textsc{rprt} food\\ 
\glft ‘she took the small pot of food, it is said’\\ 
\endgl
\trailingcitation{[mox-n110920l.058-059]}
\xe

Another example, in which smallness is the factor expressed by the diminutive is (\ref{ex:dimi-1}) from Juana, where she describes one of the last pictures of the \isi{frog story} including many little frogs.\footnote{The last \textit{-ji} of \textit{chÿenuji} which is glossed here as collective could also be the reportive marker.}

\ea\label{ex:dimi-1}
\begingl
\glpreamble aa peÿjanemÿnÿ cheikukukÿujanetuji chÿenuji\\
\gla aa peÿ-jane-mÿnÿ chÿ-eiku-kukÿu-jane-tu-ji chÿ-enu-ji\\
\glb \textsc{intj} frog-\textsc{distr}-\textsc{dim} 3-follow-\textsc{am.conc.tr}-\textsc{distr}-\textsc{iam}-\textsc{rprt} 3-mother-\textsc{col} \\
\glft ‘ah, the little frogs are following their mother, it is said’
\endgl
\trailingcitation{[jxx-a120516l-a.440]}
\xe

Whenever a diminutive is added to a noun denoting a child or a small animal, it is hard to say whether the speaker uses it only because the referent is small, or also to convey certain affection for the referent.\is{endearment}  Consider (\ref{ex:dim-small-aff}), which is from the same story as (\ref{ex:dim-small}) above. After the woman has discovered the deception of her husband, who had pretended he was making a field, the man decides to sacrifice himself by cutting off his limbs. He takes his little son with him, so that the latter can carry his limbless father and throw him into a well from where the lazy man rises as a comet.

\ea\label{ex:dim-small-aff}
\begingl 
\glpreamble chumuji chichechapÿimÿnÿ\\
\gla chÿ-umu-ji chi-chechapÿi-mÿnÿ\\ 
\glb 3-take-\textsc{rprt} 3-son-\textsc{dim}\\ 
\glft ‘he took his little son, it is said’\\ 
\endgl
\trailingcitation{[mox-n110920l.089]}
\xe

The diminutive can also be used for a small amount of something. Consider example (\ref{ex:dim-small-amount}), in which María C. describes that she only has little corn left to prepare chicha, her preferred beverage, though it cannot be excluded that the speaker also uses the diminutive to express self-pity about that fact.

\ea\label{ex:dim-small-amount}
\begingl 
\glpreamble kakumÿnÿ amukemÿnÿ te tibukapu echÿu te kuinabu nea aumue\\
\gla kaku-mÿnÿ amuke-mÿnÿ te ti-buka-pu echÿu te kuina-bu nÿ-ea aumue\\ 
\glb exist-\textsc{dim} corn-\textsc{dim} \textsc{seq} 3i-finish.\textsc{irr}-\textsc{mid} \textsc{dem}b \textsc{seq} \textsc{neg}-\textsc{dsc} 1\textsc{sg}-drink.\textsc{irr} chicha\\ 
\glft ‘there is little corn and when it will be finished, then I cannot drink chicha anymore’\\ 
\endgl
\trailingcitation{[ump-p110815sf.693]}
\xe

Finally, there are also cases, in which no smallness is involved and the only possible reading is one of emotional evaluation. This is the case in (\ref{ex:dim-pity-1}), where María C. feels pity for herself.

\ea\label{ex:dim-pity-1}
\begingl 
\glpreamble nÿti juberÿpunÿmÿnÿ\\
\gla nÿti juberÿpu-nÿ-mÿnÿ\\ 
\glb 1\textsc{sg.prn} old.woman-1\textsc{sg}-\textsc{dim}\\ 
\glft ‘poor me, I am an old woman’\\ 
\endgl
\trailingcitation{[uxx-p110825l.038]}
\xe

Affection is not necessarily for the referent of the noun that bears the diminutive marker, but can also be for the \isi{possessor} of that noun. Preceding the cited clause in (\ref{ex:dim-pity-2}), Juana explained that the two old ladies she was talking about have passed away a long time ago. They had been old already when she first met them. Juana’s speech is full of diminutives in reference to the old ladies, and in (\ref{ex:dim-pity-2}), she adds one to a possessed item, the walking cane of one of the ladies.

\ea\label{ex:dim-pity-2}
\begingl 
\glpreamble kaku chibastunemÿnÿtu, mhm, chiyuikiumÿnÿ\\
\gla kaku chi-bastun-ne-mÿnÿ-tu mhm chi-yuik-i-u-mÿnÿ\\ 
\glb exist 3-cane-\textsc{possd}-\textsc{dim}-\textsc{iam} \textsc{intj} 3-walk-\textsc{subord}-\textsc{real}-\textsc{dim}\\ 
\glft ‘she already had a cane, mhm, for walking’\\ 
\endgl
\trailingcitation{[jxx-p120515l-1.220-221]}
\xe

%other example: nisapatunemÿne, mqx-p110826l.514: Juan C. expresses his modesty in the wish for shoes that his patrón is supposed to give to him, it does not mean that the shoes are small or particularly nice -> not a good example though, because it is not a complete sentence



\subsection{Diminutives on verbs}\label{sec:Diminuitves_Verbs}
\is{verb|(}

The diminutive on verbs fulfils largely the same functions as on nouns and can also attenuate the meaning of the verb. As had been mentioned above, it is often hard to decide what exactly the speaker had in mind, when she used a diminutive.

(\ref{ex:dimi-2}) and (\ref{ex:dim-V-1}) are two sentences elicited from María S. and referring to a small chick of hers, which was given water by her grandchild. The diminutive expresses that the chick is small, that it is cute or that she feels empathy for it, or all of this together. In (\ref{ex:dimi-2}) the diminutive refers to the \isi{object} of the clause and in (\ref{ex:dim-V-1}) to the \isi{subject}.

\ea\label{ex:dimi-2}
\begingl
\glpreamble tekichamÿnÿ ÿne\\
\gla ti-ekicha-mÿnÿ ÿne\\
\glb 3i-invite.\textsc{irr}-\textsc{dim} water\\
\glft ‘she gives it water’
\endgl
\trailingcitation{[rmx-e150922l.051]}
\xe

\ea\label{ex:dim-V-1}
\begingl 
\glpreamble tibiyukumÿnÿ takÿra\\
\gla ti-biyuku-mÿnÿ takÿra\\ 
\glb 3i-be.thirsty-\textsc{dim} chicken\\ 
\glft ‘the chick is thirsty’\\ 
\endgl
\trailingcitation{[rmx-e150922l.054]}
\xe

The attenuation of a verb’s meaning is prevalent in (\ref{ex:attenuation-dim}), where Juana tells me that on her grandparents’ journey back home from Moxos  the sun started to shine a bit again after heavy rainfalls.

\ea\label{ex:attenuation-dim}
\begingl 
\glpreamble tukiu nechÿu chikebiuji, las sinkotuji tijayekamÿnÿji sache\\
\gla tukiu nechÿu chi-keb-i-u-ji {las sinko}-tu-ji ti-jayeka-mÿnÿ-ji sache\\ 
\glb from \textsc{dem}c 3-rain-\textsc{subord}-\textsc{real}-\textsc{rprt} {at five o’clock}-\textsc{iam}-\textsc{rprt} 3i-shine.\textsc{irr}-\textsc{dim}-\textsc{rprt} sun\\ 
\glft ‘from then on it was raining, it is said, until at five the sun started to shine a bit’\\ 
\endgl
\trailingcitation{[jxx-p151016l-2.122]}
\xe

Sometimes, a diminutive occurs on a verb to make an \isi{imperative} more polite, as is the case in (\ref{ex:imperative-dim}), where María C. tells Clara what we had said to her when she visited us in the hotel we stayed at.\footnote{Note that this recording has not been archived because it contains sensitive data.}

\ea\label{ex:imperative-dim}
\begingl
\glpreamble pibenamÿnÿ naka yumaji\\
\gla pi-bena-mÿnÿ naka yumaji\\
\glb 2\textsc{sg}-lie.down.\textsc{irr}-\textsc{dim} here hammock\\
\glft ‘lie down here in the hammock’
\endgl
\trailingcitation{[cux-c120510l-1.141]}
\xe

Attenuation can also be due to modesty, as in (\ref{ex:dim-modesty}), where Juana does not want to boast about her knowledge of Paunaka. It is her imagined or remembered answer in a remembered dialogue with the two old ladies also mentioned in (\ref{ex:dim-pity-2}) after they found out that she was a speaker of Paunaka.

\ea\label{ex:dim-modesty}
\begingl 
\glpreamble nichujikumÿnÿ, yeyeini kuina tichujikane kasteyano\\
\gla ni-chujiku-mÿnÿ yeye-ini kuina ti-chujika-ne kasteyano\\ 
\glb 1\textsc{sg}-speak-\textsc{dim} granny-\textsc{dec} \textsc{neg} 3i-speak-1\textsc{sg} Spanish\\ 
\glft ‘I speak it a little, my late grandmother didn’t speak Spanish with me’\\ 
\endgl
\trailingcitation{[jxx-p120515l-1.166]}
\xe



%courtesy:
%\ea\label{ex:}
%\begingl 
%\glpreamble ¡peatumÿnÿ!\\
%\gla \\ 
%\glb \\ 
%\glft ‘drink! you can drink, now!’\\ 
%\endgl
%\trailingcitation{[jxx-p150920l.001]}
%\xe  --> mÿnÿ hier nach -tu, oder ist das bei Verben anders?

\is{verb|)}

\subsection{Diminutives on other parts of speech}\label{sec:Diminuitves_OtherPOS}

Diminutives can also occasionally be added to other parts of speech. They can attach to the few adjectives that exist in Paunaka, and infrequently also to pronouns and demonstratives (usually the nominal demonstratives, but one time in the corpus also to the demonstrative adverb \textit{naka} ‘here’).

Because of the emotional value of the diminutive, \textit{-mÿnÿ} can also be added to the adjective\is{adjective|(} \textit{(mu)temena} ‘big’. This is the case in (\ref{ex:dim-ADJ}). It is not clear, though, whether Juana uses the diminutive to express her pity for some, already grown, ducks that died in her absence, because none of her family members fed them, or to attenuate the meaning of the predicate as ‘big, but small’ = ‘biggish’.\footnote{Thanks to Swintha Danielsen for pointing out this second possibility.}


\ea\label{ex:dim-ADJ}
\begingl 
\glpreamble pero temenanajimÿnÿtu\\
\gla pero temena-na-ji-mÿnÿ-tu\\ 
\glb but big-\textsc{rep}-\textsc{col}-\textsc{dim}-\textsc{iam}\\ 
\glft ‘but they were already big, the poor ones’\\or: ‘but they were already biggish’\\ 
\endgl
\trailingcitation{[jrx-c151001lsf-11.071]}
\xe
\is{adjective|)}

(\ref{ex:PersPron-3}) is a statement by María C. about herself, in which the diminutive is added to the first person singular pronoun,\is{personal pronoun} because she pities herself.

\ea\label{ex:PersPron-3}
\begingl
\glpreamble nÿtimÿnÿ baichane, kuina nÿana kuina nenuina\\
\gla nÿti-mÿnÿ baicha-ne kuina nÿ-a-ina kuina nÿ-enu-ina\\
\glb 1\textsc{sg.prn}-\textsc{dim} orphan-1\textsc{sg} \textsc{neg} 1\textsc{sg}-father-\textsc{irr.nv} \textsc{neg} 1\textsc{sg}-mother-\textsc{irr.nv}\\
\glft ‘poor me, I am an orphan, I don’t have a father, I don’t have a mother’
\endgl
\trailingcitation{[uxx-p110825l.071]}
\xe

In (\ref{ex:dimi-4}), a \isi{numeral} carries the diminutive for attenuation. The sentence comes from María S.

\ea\label{ex:dimi-4}
\begingl
\glpreamble chÿnamÿnÿchÿ nipeu ÿba\\
\gla chÿna-mÿnÿ-chÿ ni-peu ÿba\\
\glb one-\textsc{dim}-3 1\textsc{sg}-animal pig\\
\glft ‘I have a single pig’
\endgl
\trailingcitation{[rxx-e181024l.059]}
\xe

\is{derivation|)}


While the diminutive can thus attach to various parts of speech,\is{diminutive|)} the locative marker, which is described in the following section, exclusively occurs with nouns.


