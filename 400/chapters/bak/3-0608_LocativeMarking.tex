%!TEX root = 3-P_Masterdokument.tex
%!TEX encoding = UTF-8 Unicode

\section{Locative marking}\label{sec:Locative}
\is{oblique|(}
\is{locative marker|(}

There is one general locative marker, \textit{-yae}, which possibly originated as a relational noun\is{grammaticalisation} (cf. \citealt[]{Rose2019a} and see also \sectref{sec:Non-possessables}).\footnote{According to the analysis by \citet[]{Rose2019a}, this relational noun developed into a universal preposition in Trinitario.\is{Mojeño Trinitario} A related form \textit{-ye} also occurs in \isi{Baure} \citep[cf.][150]{Danielsen2007}. Paunaka shares with \isi{Baure} that the root is used as a locative marker on nouns, and it shares with Trinitario that it is used as a relational noun in possessive constructions.\is{possessive clause}} The marker attaches exclusively to nouns that express spatial relations in a clause, more precisely relations of place, goal, and source. It is not found on adverbs. In slow speech, the marker is pronounced \textit{-yae}, but it can be reduced to \textit{-ye} or \textit{-ya} in rapid speech. 

(\ref{ex:loc-place}) is an expression of a place, (\ref{ex:loc-goal}) presents the locative marker on a goal, and (\ref{ex:loc-source}) on a source expression. In (\ref{ex:loc-place}), Juana speaks about her daughter who had badly fallen down, was treated in hospital and still had to stay in bed afterwards.

\ea\label{ex:loc-place}
\begingl
\glpreamble \textup{place:}\\pero tibenunukubu chikamaneyae\\
\gla pero ti-benunuku-bu chi-kama-ne-yae\\
\glb but 3i-lie-\textsc{mid} 3-bed-\textsc{possd}-\textsc{loc}\\
\glft ‘but she kept lying in her bed’
\endgl
\trailingcitation{[jxx-p110923l-1.485]}
\xe

(\ref{ex:loc-goal}) was a conjecture of María S. about what her brother was going to do.

\ea\label{ex:loc-goal}
\begingl
\glpreamble \textup{goal:}\\tiyunakena chisaneyae\\
\gla ti-yuna-kena chi-sane-yae\\ 
\glb 3i-go.\textsc{irr}-\textsc{uncert} 3-field-\textsc{loc}\\ 
\glft ‘maybe he wants to go to his field’\\ 
\endgl
\trailingcitation{[rxx-e120511l.348]}
\xe

(\ref{ex:loc-source}) was produced by Juana in telling me about the death of some of her siblings. She went to the funeral of her brother by public transportation, which is carried out by vans or small buses in Bolivia, called \textit{micros} in Spanish.

\ea\label{ex:loc-source}
\begingl
\glpreamble \textup{source:}\\nikupu tukiu mikroyae\\
\gla ni-kupu tukiu mikro-yae\\
\glb 1\textsc{sg}-go.down from microbus-\textsc{loc}\\
\glft ‘I got off the microbus’
\endgl
\trailingcitation{[jxx-p120430l-2.465]}
\xe

The locative marker may be dropped under certain conditions. Most importantly, if the spatial relation includes a \isi{toponym}, \textit{-yae} is often absent, see (\ref{ex:new23-loc1}); this is also true for the noun \textit{uneku} ‘town’ which is toponym-like\is{toponym}, since it usually refers to Concepción, as in (\ref{ex:new23-loc2}).

(\ref{ex:new23-loc1}) comes from Juana, who was telling me how hard it was to obtain water before the reservoir was made in Santa Rita.

\ea\label{ex:new23-loc1}
\begingl
\glpreamble bupunu ÿne Santa Rita\\
\gla bi-upunu ÿne {Santa Rita}\\
\glb 1\textsc{pl}-bring water {Santa Rita}\\
\glft ‘we brought water to Santa Rita’
\endgl
\trailingcitation{[jxx-p120515l-2.054]}
\xe

(\ref{ex:new23-loc2}) is a comment from María S. about her siblings who moved away from Santa Rita.

\ea\label{ex:new23-loc2}
\begingl
\glpreamble depue tepajÿkunubetu uneku\\
\gla depue ti-epajÿku-nube-tu uneku\\
\glb afterwards 3i-stay-\textsc{pl}-\textsc{iam} town\\
\glft ‘then they stayed in town’
\endgl
\trailingcitation{[rxx-p181101l-2.264]}
\xe

If the verb in a goal expression is \textit{-yunu} ‘go’, it is also not uncommon that the locative marker is missing on the noun denoting the goal, as in (\ref{ex:new23-loc3}) from Clara. She was explaining us that her daughter was baking bread alone at that moment, thus this sentence is about her other daughter’s physical presence in school, not the general enrollment in school.

\ea\label{ex:new23-loc3}
\begingl
\glpreamble punachÿ tiyunu xhikuera\\
\gla punachÿ ti-yunu xhikuera\\
\glb other 3i-go school\\
\glft ‘the other (sister) went to school’
\endgl
\trailingcitation{[cux-120410ls.220]}
\xe

In source expressions, the preposition \textit{tukiu}\is{source|(} is needed, and \textit{-yae} can be considered optional in this case, thus in (\ref{ex:new23-loc4}) from Juana, no locative marker is necessary on the noun. She spoke about her daughter (and other people) who had come back from Spain.

\ea\label{ex:new23-loc4}
\begingl
\glpreamble tikubupaikunubetu tukiu labion\\
\gla ti-kubupaiku-nube-tu tukiu labion\\
\glb 3i-go.down-\textsc{pl}-\textsc{iam} from plane\\
\glft ‘they disembarked from the plane’
\endgl
\trailingcitation{[jxx-p120430l-1.266]}
\xe
\is{source|)}

%nubupuna tukiu asaneti nubupuna ubiaeyae, rxx-e181020le

All of these examples would equally work well if the locative marker was attached to the noun. Furthermore, locative marking can also target human nouns, as is the case in (\ref{ex:loc-human}), which comes from Miguel who was talking about Swintha.

\ea\label{ex:loc-human}
\begingl
\glpreamble paseaubÿti nauku baurenyonubeyae\\
\gla paseau-bÿti nauku baurenyo-nube-yae\\
\glb stroll-\textsc{prsp} there Baure.person-\textsc{pl}-\textsc{loc}\\
\glft ‘she is going to travel to the Baure people’
\endgl
\trailingcitation{[mxx-d110813s-2.066]}
\xe

The marker \textit{-yae} can also figure as an instrumental marker in cases, in which the preposition \textit{en} ‘in’ would be used in Spanish, e.g. for motion by a vehicle. This is the case in (\ref{ex:loc-Span-1}), where Swintha and I were discussing our little excursion to \isi{Altavista} with María C. and Clara. Altavista is far away from Santa Rita if one has to walk, but nicely reachable by bike as Clara recognises here.

\ea\label{ex:loc-Span-1}
\begingl
\glpreamble pero un ratoyÿchi eyuna bisikletayae\\
\gla pero {un rato}-yÿchi e-yuna bisikleta-yae\\
\glb but {a while}-\textsc{lim}2 2\textsc{pl}-go.\textsc{irr} bicyle-\textsc{loc}\\
\glft ‘but it only takes you a little while if you go by bike’
\endgl
\trailingcitation{[cux-c120414ls-1.155]}
\xe

Other cases of semantic extension of locative marking that resemble the ones in Spanish are exemplified by (\ref{ex:loc-Span-2}) and (\ref{ex:loc-Span-3}). There is no extension to temporal expressions though.

In (\ref{ex:loc-Span-2}), María C. construes the inside of her head as the place containing knowledge.

\ea\label{ex:loc-Span-2}
\begingl
\glpreamble kakutu pario nÿchÿtiyaemÿnÿ pario\\
\gla kaku-tu pario nÿ-chÿti-yae-mÿnÿ pario\\
\glb exist-\textsc{iam} some 1\textsc{sg}-head-\textsc{loc}-\textsc{dim} some\\
\glft ‘I have a lot (knowledge) in my head, a lot’
\endgl
\trailingcitation{[uxx-p110825l.095]}
\xe

(\ref{ex:loc-Span-3}) comes from Juana and is about words in different languages.

\ea\label{ex:loc-Span-3}
\begingl
\glpreamble jaja, kastelyanoyae ciervo, pero naka neteayae kaku chijaini\\
\gla jaja kastelyano-yae ciervo pero naka nÿ-etea-yae kaku chi-ija-ini\\
\glb \textsc{afm} Spanish-\textsc{loc} ciervo but here 1\textsc{sg}-language-\textsc{loc} exist 3-name-\textsc{frust}\\
\glft ‘yes, it is \textit{ciervo} (=deer) in Spanish, but it has a name in my language (that I don’t remember)’
\endgl
\trailingcitation{[jxx-a120516l-a.231-233]}
\xe

Despite of these possible extensions, the locative marker is mainly applied to spatial relations of different kinds. In (\ref{ex:loc-on}), there is contact from above (‘on’), in (\ref{ex:loc-at}) the spatial relation is one of closeness and can be translated with ‘at’, and in (\ref{ex:loc-in}), the figure (i.e. the subject of the clause) is inside a location (‘in’).

(\ref{ex:loc-on}) was elicited from María S.

\ea\label{ex:loc-on}
\begingl
\glpreamble tibebeikubu siyayae\\
\gla ti-bebeiku-bu siya-yae\\
\glb 3i-lie-\textsc{mid} chair-\textsc{loc}\\
\glft ‘it (the cat) is lying on the chair’
\endgl
\trailingcitation{[rxx-e181024l]}%el.
\xe

(\ref{ex:loc-at}) comes from the same session. It referred to a pig which I had asked for, since it was suddenly not in the yard anymore.

\ea\label{ex:loc-at}
\begingl
\glpreamble tiyunu tisemaiku yÿtie atajauyae\\
\gla ti-yunu ti-semaiku yÿtie atajau-yae\\
\glb 3i-go 3i-search food water.reservoir-\textsc{loc}\\
\glft ‘it (the pig) went to look for food at the reservoir’
\endgl
\trailingcitation{[rxx-e181024l]}%non-el.
\xe


(\ref{ex:loc-in}) was elicited from Miguel.

\ea\label{ex:loc-in}
\begingl
\glpreamble kaku kÿjÿpi ubiaeye\\
\gla kaku kÿjÿpi ubiae-yae\\
\glb exist manioc house-\textsc{loc}\\
\glft ‘there is manioc in the house’
\endgl
\trailingcitation{[mxx-e160811sd.073]}%el.
\xe

The locative marker alone thus expresses the most expected spatial relations and its interpretation as ‘on’, ‘at’ or ‘in’ largely depends on the spatial dimensions of the noun denoting the ground (i.e. the location) and the habits or properties of the figure \citep[cf.][69]{Admiraal2016}. In order to be more specific or for the expression of unusual relations, speakers can make use of two different strategies: either a more precise locative noun is derived from the noun denoting the ground or a complex NP\is{noun phrase} is used which contains a possessed \isi{relational noun}\is{possession} and a possessor denoting the ground.

For the expression of complete containment, a “container” noun is derived\is{derivation|(} by adding the “bounded” classifier\is{classifier|(} \textit{-kÿ} or attaching the locative stem \textit{-j(ÿ)ekÿ} ‘inside’, which can most probably be classified as a \isi{nominal stem}. There are differences between the resulting nouns.

Not every noun can take the classifier \textit{-kÿ}, the majority are containers anyway. The difference is that without the classifier, they are perceived as manipulable objects, with \textit{-kÿ} they denote locations. The locative marker is usually added to the derived noun. An example is given in (\ref{ex:clf-loc-1}), which contains \textit{tachukÿyae} ‘inside the small pot’. It comes from Miguel describing the pictures of the \isi{frog story}.

\ea\label{ex:clf-loc-1}
\begingl
\glpreamble i naka chipurutukutu eka kabe chichÿti naka eka tachukÿyae\\
\gla i naka chi-purutuku-tu eka kabe chi-chÿti naka eka tachu-kÿ-yae\\
\glb and here 3-put.in-\textsc{iam} \textsc{dem}a dog 3-head here \textsc{dem}a small.pot-\textsc{clf:}bounded-\textsc{loc}\\
\glft ‘and here the dog has stuck its head into the small pot here’
\endgl
\trailingcitation{[mox-a110920l-2.052]}
\xe

Another example, which also comes from Miguel re-telling the \isi{frog story} (but on another occasion), is given in (\ref{ex:clf-loc-2}), and this time the classifier \textit{-kÿ} adds the important information that the action is performed in relation to the inside of the boot.

\ea\label{ex:clf-loc-2}
\begingl
\glpreamble chimumukuji chijachÿukena kakukena nauku botakÿyae\\
\gla ch-imumuku-ji chija-chÿu-kena kaku-kena nauku bota-kÿ-yae\\
\glb 3-look-\textsc{rprt} what-\textsc{dem}b-\textsc{uncert} exist-\textsc{uncert} there boot-\textsc{clf:}bounded-\textsc{loc}\\
\glft ‘he is looking what may be there in his boot, it is said’
\endgl
\trailingcitation{[mtx-a110906l.043-046]}
\xe

The noun \textit{kimenu} ‘woods’ is also frequently found with the classifier, when it conveys the idea of a place that somebody goes to or acts in. Unlike the other nouns with \textit{-kÿ}, it is often used without the locative marker. In (\ref{ex:loci-1}), however, \textit{-yae} is attached to the noun. The example comes from Miguel who told me and José the story about a lazy man.

\ea\label{ex:loci-1}
\begingl
\glpreamble titupunubuji kimenukÿyae tisemaikuji echÿu kujubipi\\
\gla ti-tupunubu-ji kimenu-kÿ-yae ti-semaiku-ji echÿu kujubipi\\
\glb 3i-arrive-\textsc{rprt} woods-\textsc{clf:}bounded-\textsc{loc} 3i-find-\textsc{rprt} \textsc{dem}b liana.sp\\
\glft ‘when he arrived in the woods, he found a liana, it is said’
\endgl
\trailingcitation{[mox-n110920l.025]}
\xe
\is{classifier|)}
\is{derivation|)}

More emphasis is attained through attachment of the locative stem \textit{-j(ÿ)ekÿ} ‘inside’, which is related to the free noun \textit{nujekÿ} ‘inside’.\footnote{\textit{Nujekÿ} ‘inside’ and its antonym \textit{nekupai} ‘outside, yard’ are never set in relation to another noun.} Its relation to \textit{-kÿ} is similar to the relation of ‘in(to)’ to ‘inside (of)’. Nominal compounds\is{compounding} with \textit{-j(ÿ)ekÿ} can be followed by the locative marker, as in (\ref{ex:inside-loc}), but this is not always the case, see (\ref{ex:inside-no-loc}). Both examples were elicited, (\ref{ex:inside-loc}) from Miguel and (\ref{ex:inside-no-loc}) from María S.

\ea\label{ex:inside-loc}
\begingl
\glpreamble nipurtuka jurnujÿekÿyae\\
\gla ni-purtuka jurnu-jÿekÿ-yae\\
\glb 1\textsc{sg}-put.in.\textsc{irr} oven-inside-\textsc{loc}\\
\glft ‘I will put it inside the oven’
\endgl
\trailingcitation{[mxx-e120415ls.105]}%el.
\xe

\ea\label{ex:inside-no-loc}
\begingl
\glpreamble tibÿkupu kabe kosinajÿekÿ\\
\gla ti-bÿkupu kabe kosina-jÿekÿ\\
\glb 3i-enter dog kitchen-inside\\
\glft ‘the dog goes into the kitchen’
\endgl
\trailingcitation{[rxx-e181021les.105]}%el.
\xe

The noun \textit{ÿne} ‘water’ does not combine with \textit{-j(ÿ)ekÿ}, maybe because it is not perceived as a container. There is a special expression for ‘inside of the water’, which is \textit{ÿneumu(kÿ)} (see \sectref{sec:Nouns_CLF}), while ‘above/on the water’ is \textit{ÿnemiuke}. These are unique non-productive derivations.\footnote{I could elicit \textit{mutejimiuke} ‘on/above the mud’, but no other word I tried to form accordingly was accepted by the speakers. There is another singular locative expression, \textit{mainekÿke} ‘on/above the stone’, which contains the relational noun \textit{-(i)ne} ‘top’ and seemingly also \textit{-kÿ}. Both words contain a suffix \textit{-ke}, which may be the one we find on nouns denoting places (see \sectref{sec:Classifiers}).}

For any relation other than “inside”,\is{possession|(} the other strategy mentioned above is used:\is{relational noun|(} an inalienably\is{inalienability} possessed locative noun stem\is{nominal stem} expressing the specific relation is juxtaposed\is{juxtaposition} to the noun denoting the ground which acts as a possessor. The locative marker is attached to the relational noun in this case. However, the nouns \textit{-upekÿ} ‘place under’ and \mbox{\textit{-akene/-ekene}} ‘non-visible side’ can also be used without the locative marker. For the latter one, this is even more common. The locative relational nouns are listed in \tabref{table:noun-stems-locative}.

\begin{table}[htbp]
\caption{Locative relational noun stems}

\begin{tabular}{ll}
\lsptoprule
Relational noun & Translation \cr
\midrule
\textit{-akene/-ekene} & non-visible side (behind, beside)\cr
\textit{-chuku} & side (next to, close to)\cr
\textit{-(i)ne} & top, place on top or above\cr
\textit{-upekÿ} & place under\cr
\lspbottomrule
\end{tabular}

\label{table:noun-stems-locative}
\end{table}

(\ref{ex:loci-2}) shows the use of the relational noun \textit{-upekÿ} ‘place under’. It was elicited from Miguel and refers to a pen I put under a bag.

\ea\label{ex:loci-2}
\begingl 
\glpreamble kaku chiupekÿye echÿu pusane\\
\gla kaku chi-upekÿ-yae echÿu pusane\\ 
\glb exist 3-place.under-\textsc{loc} \textsc{dem}b bag\\ 
\glft ‘it is under the bag’\\ 
\endgl
\trailingcitation{[mxx-e120505l-1]}
\xe

{\ref{ex:loci-3}) includes \textit{-chuku} ‘side’. It comes from an elicitation session with several playmobil toys and was produced by Juana.

\ea\label{ex:loci-3}
\begingl
\glpreamble kaku chichukuyae echÿu jente\\
\gla kaku chi-chuku-yae echÿu jente\\ 
\glb exist 3-side-\textsc{loc} \textsc{dem}b man\\ 
\glft ‘she is (standing) next to the man’\\ 
\endgl
\trailingcitation{[jrx-c151024lsf]}
\xe

The noun stem \textit{-(i)ne} ‘top’ is the only one of the relational nouns that can also be incorporated\is{incorporation} into active verb stems, see \sectref{sec:INC_ActiveVerbs}.
(\ref{ex:loci-4}) is one example of its use in a locative NP. It comes from Miguel who was describing the production of rice bread.

\ea\label{ex:loci-4}
\begingl
\glpreamble chetuku echÿu kesu tiyÿbapakubu chÿineyae echÿu masa\\
\gla chÿ-etuku echÿu kesu ti-yÿbapaku-bu chÿ-ine-yae echÿu masa\\
\glb 3-put \textsc{dem}b cheese 3i-grind-\textsc{mid} 3-top-\textsc{loc} \textsc{dem}b dough\\
\glft ‘she puts the grated cheese on top of the dough’
\endgl
\trailingcitation{[mxx-e120415ls.087]}
\xe

As has already been mentioned above, the noun \textit{-akene} or \textit{-ekene} ‘non-visible side’ is normally used without the locative marker. It can combine with a person marker or with the root \textit{(u)pu-} ‘other’, and is used if the referent in question is out of sight because something blocks the view. Two examples, both from Juana, are given here. 

(\ref{ex:loci-5}) comes from the creation story she told, a mixture of the tales of the bible and other elements. It is Jesus who hides away behind the door here. (\ref{ex:loci-6}) was elicited. It shows the use of the locative noun together with \textit{(u)pu-} ‘other’.

\ea\label{ex:loci-5}
\begingl
\glpreamble nauku chekene nuinekÿ chububuikubutu\\
\gla nauku chÿ-ekene nuinekÿ chÿ-ubu-buiku-bu-tu\\
\glb there 3-non.vis.side door 3-be.at-\textsc{cont}-\textsc{mid}-\textsc{iam}\\
\glft ‘there behind the door he was (hidden)’
\endgl
\trailingcitation{[jxx-n101013s-1.435]}
\xe

\ea\label{ex:loci-6}
\begingl
\glpreamble kaku upuakene \\
\gla kaku upu-akene \\
\glb exist other-non.vis.side\\
\glft ‘it is behind (the house)’
\endgl
\trailingcitation{[jxx-e191021e-2]}%el.
\xe

\is{relational noun|)}

Unlike in Baure \citep[cf.][81--82]{Admiraal2016}, body-part terminology is very restricted in the expression of spatial relations in Paunaka. Body-part nouns can be used only if the ground is animate\is{animacy} and a real possessor of the body part, as is the case in (\ref{ex:face-front}), elicited from Juana, in which I was pretending to look for my cell phone, which was lying in front of me. There is no semantic extension of body part terminology to inanimate referents.

\ea\label{ex:face-front}
\begingl
\glpreamble ¡kaku naka pibÿkeyae!\\
\gla kaku naka pi-bÿke-yae\\
\glb exist here 2\textsc{sg}-face-\textsc{loc}\\
\glft ‘it is here in front of you!’ (lit.: in your face)
\endgl
\trailingcitation{[jxx-p181104l-2]}%el.
\xe
\is{possession|)}

Last, there are also some locative nouns that are not possessed, \textit{anÿke} ‘up, above’, \textit{apuke} ‘ground, down’, and \textit{pÿkÿjÿe} ‘middle’, as well as the aforementioned \textit{nujekÿ} ‘inside’ and \textit{nekupai} ‘outside, yard’. They could actually also be adverbs,\is{adverb} \isi{word class} is not totally clear in this case (see discussion in \sectref{sec:LocativeAdverbs}). Most of the times, they do not occur in juxtaposition with another noun denoting the ground, i.e. they denote the ground themselves, and they usually do not take the locative marker with a few exceptions.

\is{locative marker|)}
\is{oblique|)}
\is{inflection|)}
To conclude this chapter, the next section provides some information about content of and word order inside the NP.

%not found on anÿke = up, above, but on apuke if this means ground and not down?



