%!TEX root = 3-P_Masterdokument.tex
%!TEX encoding = UTF-8 Unicode

\section{Middle voice}\label{sec:Middle_voice}\is{active verb|(}
\is{middle voice|(}

According to \citet[243]{Kemmer1993}, “[m]iddle marking is in general a morphosyntactic strategy for expressing an alternative conceptualization of an event in which aspects of the internal structure of the event that are less important from the point of view of the speaker are not made reference to in the utterance”.

In Paunaka, the middle marker is used to express \isi{reflexive} and \isi{anticausative} readings of an event. It only attaches to \isi{active verb} stems. There is also a number of deponent middle verbs.

The marker has two allomorphs whose usage depends on the preceding RS marker.\is{reality status} The allomorph \textit{-bu} occurs after default/\isi{realis} /u/ and \textit{-pu} after \isi{irrealis} /a/, see (\ref{ex:MID-real}) and (\ref{ex:MID-irr}) respectively.

(\ref{ex:MID-real}) comes from Juana, who is speaking about her daughter arriving at the airport too late – her sister, who had come to Spain to work as a nanny, had already been deported, because her visa was not valid.

\ea\label{ex:MID-real}
\begingl 
\glpreamble nÿmayu titupunubu nijinepÿi\\
\gla nÿmayu ti-tupunu-bu ni-jinepÿi\\ 
\glb just 3i-reach-\textsc{mid} 1\textsc{sg}-daughter\\ 
\glft ‘only then did may daughter arrive’\\ 
\endgl
 \trailingcitation{[jxx-p110923l-1.334]}
\xe

In (\ref{ex:MID-irr}), Juana states that her daughter never arrived at her home on Christmas, although she had promised to come.

\ea\label{ex:MID-irr}
\begingl 
\glpreamble kuina titupunapu nijinepÿi\\
\gla kuina ti-tupuna-pu ni-jinepÿi\\ 
\glb \textsc{neg} 3i-reach.\textsc{irr}-\textsc{mid} 1\textsc{sg}-daughter\\ 
\glft ‘my daughter did not arrive’\\ 
\endgl
 \trailingcitation{[jxx-p120430l-1.305]}
\xe


The “realis”\is{reality status|(} allomorph of the middle is homophonous with the \isi{discontinuous} marker \textit{-bu} (see \sectref{sec:Discontinuous}); however, they cannot be confounded, because the \isi{discontinuous} marker only shows up after irrealis-marked stems, and the middle marker would realise the “irrealis” allomorph in that case. Both morphemes can also be used in combination, consider (\ref{ex:MID-DSC}). Here, María C. tells us that she cannot cook anymore, because she does not see well anymore, and in addition, she lives with her daughter-in-law since her husband has passed away and is no longer responsible for cooking.

\ea\label{ex:MID-DSC}
\begingl 
\glpreamble kuinachu pueronÿina niyÿtikapubu\\
\gla kuina-chu puero-nÿ-ina ni-yÿtika-pu-bu\\ 
\glb \textsc{neg}-? can-1\textsc{sg}-\textsc{irr.nv} 1\textsc{sg}-set.on.fire.\textsc{irr}-\textsc{mid}-\textsc{dsc}\\ 
\glft ‘I cannot cook anymore’\\ 
\endgl
 \trailingcitation{[cux-c120410ls.106]}
\xe

The “irrealis” allomorph \textit{-pu} is homophonous with the realis allomorph of the \isi{dislocative} marker, but cannot be confounded because of the different RS\is{reality status|)} contexts in which the two markers show up. Those two markers do not occur together as far as I can tell.


As is typical for the middle voice, Paunaka has some deponent verbs \citep[cf.][22]{Kemmer1993} or media tantum \citep[cf.][98]{Klaiman1991}, i.e. verbs that only occur in middle form. These are listed in \tabref{table:Deponent_middles}.\footnote{Of course, there remains a possibility that these verbs have non-middle forms that I have not been able to elicit despite serious attempts to do so.} All deponent verbs are \isi{intransitive}, which is common for deponent verbs \citep[44]{Klaiman1991}.

%\emph{TO DO: check if chabu has an irrealis form!}

\begin{table}[htbp]
\caption{Deponent middle verbs}

\begin{tabular}{ll}
\lsptoprule
Verb & Gloss \cr
\midrule
\textit{-bÿtupaiku-bu} & fall \cr
\textit{-eku-bu} & laugh\cr
\textit{-kubiaku-bu} & be tired \cr
\textit{-kupunÿku-bu} & be full\cr
\textit{-kutiku-bu} & run\cr
\lspbottomrule
\end{tabular}

\label{table:Deponent_middles}
\end{table}

Other verbs have a non-middle form, but the middle is at least as frequent as the non-middle or even more frequent; they are listed in \tabref{table:Frequent_middles}. The verb \textit{-kubu} ‘bathe’, a typical middle verb of grooming \citep[cf.][54]{Kemmer1993}, seems to be totally lexicalised\is{lexicalisation} with the middle marker that is now considered part of the verb stem,\is{verbal stem} so that its \isi{irrealis} form is \textit{-kuba}; the same may be true for \textit{-kebu} ‘rain’ with the irrealis \textit{-keba} and for the defective verbs \textit{-ubu} ‘be, live’ and \textit{-chabu} ‘do’ that do not have irrealis forms.

\begin{table}[htbp]
\caption{High-frequency middle verbs}

\begin{tabular}{llll}
\lsptoprule
Middle verb & Gloss & Non-middle verb & Gloss\cr
\midrule
\textit{-bebeiku-bu} & lie & \textit{-beu}? & take away\cr
\textit{-benunuku-bu} & lie & \textit{-benu} & lie down\cr
\textit{-chemumuiku-bu} & stand & \textit{-chemu} & stand up\cr
\textit{-chujijiku-bu} & talk, converse & \textit{-chujiku} & speak, talk\cr
\textit{-pÿsisiku-bu} & be alone & \textit{-pÿsisi(si)ku}? & smoke, fume \cr
\textit{-tibubuiku-bu} & sit & \textit{-tibu} & sit down\cr
\textit{-tupunu-bu} & arrive & \textit{-tupunu} & reach\cr
\textit{-iyuyuiku-bu} & cry & \textit{-iyu} & cry\cr
\textit{-yÿtiku-bu} & cook & \textit{-yÿtiku} & set (pot) on fire\cr
\lspbottomrule
\end{tabular}

\label{table:Frequent_middles}
\end{table}

Most of the middle verbs in \tabref{table:Frequent_middles} have reduplicated\is{reduplication} root\is{verbal root} syllables. These verbs can also be used without the middle marker. However, forms without the middle marker occur less frequently and if asked, speakers state that the middle form sounds better. Except for the pairs \textit{-pÿsisiku-bu} ‘be alone’ – \textit{-pÿsisi(si)ku} ‘smoke, fume’ and \mbox{\textit{-bebeiku-bu}} ‘lie’ – \textit{-beu} ‘take away’ that do not have a transparent semantic relation and may be unrelated, the middle forms have in common that they tend to express events of longer duration than their non-middle relatives. Although this longer duration is among the meanings expressed by reduplication, the middle marker may be used with the same function here. This impression arises from a comparison with \isi{Baure}, which has a marker \textit{-wo} used to mark imperfective aspect among other things \citep[258]{Danielsen2007}. Strikingly, it is used on verbs expressing change in body posture with the same effect as in Paunaka: the change in body posture is expressed by a simple verb, and the resulting body posture state by the addition of the marker, but contrary to Paunaka, \isi{Baure} does not reduplicate any syllable of the verb stem \citep[cf.][260]{Danielsen2007}. The way the semantic pairs of change in body posture and static body posture are expressed in Paunaka also run counter to the expected distribution of the middle marker as supposed by \citet[55--56]{Kemmer1993}, i.e. the verbs expressing the change should be the ones that receive middle marking according to her analysis. However, \citet[45]{Klaiman1991} states that it is actually “physical states” that are expressed by the deponent middle verbs, and this is also reflected in the examples she gives, e.g. Fula (West Atlantic Niger-Congo) behaves like Paunaka in that the body posture verbs are middle-only verbs and the change in posture verbs are mostly active \citep[cf.][58]{Klaiman1991}. She further argues that “the middle, when in contrast with the active, cross-linguistically displays an association with various kinds of noneventuality*, e.g. with atelic, nonpunctual, and/or irrealis temporomodal categories of the verb” \citep[105]{Klaiman1991}. The atelic\is{telicity} and nonpunctual association also seems to play a role for at least some of the middle verbs in Paunaka, but RS\is{reality status} is totally independent from it.\footnote{With the active/middle pair \textit{-iyu}/\textit{-iyuyuiku-bu} ‘cry’ it is even the case that realis RS is associated with the middle form and irrealis with the active form.}

Most of the deponent or high-frequency middle verbs belong to one of the classes that \citet[]{Kemmer1993} mentions as being typical of middle marking. However, the opposite is not true: very few of the potential verbs with middle semantics are middle-marked in Paunaka. Apart from the semantic class of posture verbs, which are quite consistently associated with middle marking, there is no clustering of the middle forms among a certain semantic class. 

The reason to call \textit{-bu} a middle marker becomes more comprehensible if we leave aside the deponent and high frequency middle verbs and have a look at the functions the marker exhibits on active \isi{transitive} verbs.

On these verbs, the middle marker is used to express reflexive\is{reflexive|(} and anticausative or spontaneous meanings; both have been reported to be typical of middle voice \citep[cf.][1149--1150]{Shibatani2004}.

One example of the reflexive use of the middle marker follows. In (\ref{ex:MID-RFLX}) from Juana, the fox has made the jaguar believe that the way to get a piece of cheese from the water hole (which is not cheese after all but a reflection of the moon) is to tie oneself with a stone and jump in.

\ea\label{ex:MID-RFLX}
\begingl 
\glpreamble tititiukubu i chetuku mai\\
\gla ti-titiuku-bu i chÿ-etuku mai\\ 
\glb 3i-tie-\textsc{mid} and 3-put stone\\ 
\glft ‘he tied himself and put a stone (in the bonds)’\\ 
\endgl
 \trailingcitation{[jmx-n120429ls-x5.258-259]}
\xe

Besides this “direct reflexive” \citep[42]{Kemmer1993}, in which \isi{agent} and patient\is{patient/theme} referent are identical, Paunaka also uses the middle to express “indirect reflexives” \citep[74]{Kemmer1993}, i.e. the co-reference of \isi{agent} and beneficiary or \isi{recipient}, see (\ref{ex:MID-RFLX-2}), in which the \isi{agent} is the locus/\isi{recipient} of the action of putting the necklace on. The sentence comes from Juana who was thinking about an excursion to Cotoca.

\ea\label{ex:MID-RFLX-2}
\begingl 
\glpreamble betukapu eka bibite\\
\gla bi-etuka-pu eka bi-bite\\ 
\glb 1\textsc{pl}-put.\textsc{irr}-\textsc{mid} \textsc{dem}a 1\textsc{pl}-necklace\\ 
\glft ‘we will put on our necklaces’\\ 
\endgl
 \trailingcitation{[jxx-p120430l-2.608]}
\xe

Paunaka also has so-called “body-part reflexives” \citep[77]{Kemmer1993}, when the body part is incorporated\is{incorporation} into the verb or expressed by a \isi{classifier}. An example is (\ref{ex:body-RFLX}), elicited from Miguel.

\ea\label{ex:body-RFLX}
\begingl 
\glpreamble tikipubÿkeubu\\
\gla ti-kipu-bÿke-u-bu\\ 
\glb 3i-wash-face-\textsc{real}-\textsc{mid}\\ 
\glft ‘he washed his face’\\ 
\endgl
 \trailingcitation{[mdx-c120416ls.079]}
\xe
\is{reflexive|)}

The more important function of the middle marker, however, is the expression of anticausative\is{anticausative|(} and \isi{resultative}. As \citet[1682]{Kaufmann2007} states “[i]n the anticausative reading of the middle, the causer of the active form is unrealized. Here, the causer is not part of the interpretation of the middle form”.\footnote{According to \citet[1683]{Kaufmann2007}, the middle can also have a resultative reading, which is the stative variant of the anticausative according to \citet[1133]{HaspelmathMueller2004}. However, in Paunaka, a stative or \isi{resultative} reading is rather invoked by an extra \isi{iamitive} marker on the verb (see \sectref{sec:Iamitive}) and not part of the semantics of the middle marker per se.}

One example of an anticausative is (\ref{ex:Anticaus-MID}). It is from the story by Miguel, in which two men meet the devil in the woods. While one of the men hides in a tree, the other one gives the devil meat. When there is no meat left, which is expressed by the middle form of the verb \textit{buku} ‘finish’ here, the man is eaten by the devil, while the other one, who had hidden, can escape.

\ea\label{ex:Anticaus-MID}
\begingl 
\glpreamble tibukubutuji echÿu chÿeche\\
\gla ti-buku-bu-tu-ji echÿu chÿeche\\ 
\glb 3i-finish-\textsc{mid}-\textsc{iam}-\textsc{rprt} \textsc{dem}b meat\\ 
\glft ‘the meat was finished, it is said’, i.e. there was no meat left\\ 
\endgl
 \trailingcitation{[mxx-n101017s-1.044]}
\xe

María S. told the same story, but she decided to use the active form of the verb to express that the devil finished the meat, see (\ref{ex:no-anticaus}).

\ea\label{ex:no-anticaus}
\begingl 
\glpreamble chibukuji\\
\gla chi-buku-ji\\ 
\glb 3-finish-\textsc{rprt}\\ 
\glft ‘he finished it, it is said’\\ 
\endgl
 \trailingcitation{[rxx-n120511l-2.41]}
\xe

Another pair of active and anticausative middle verbs are presented in (\ref{ex:no-anticaus-2}) and (\ref{ex:anticaus-2}). (\ref{ex:no-anticaus-2}) has the active verb \textit{-bakaiku} ‘throw (away)’ and (\ref{ex:anticaus-2}) the middle form, which reads as ‘spill out’.

(\ref{ex:no-anticaus-2}) was elicited from José.

\ea\label{ex:no-anticaus-2}
\begingl 
\glpreamble nÿbakaika emuniki\\
\gla nÿ-bakaika emuniki\\ 
\glb 1\textsc{sg}-throw.away.\textsc{irr} embers\\ 
\glft ‘I throw away the embers’\\ 
\endgl
 \trailingcitation{[oxx-e120414ls-1a.003]}%semi-el.
\xe

(\ref{ex:anticaus-2}) comes from Juana telling the creation story. The silk floss tree had swallowed the whole supply of corn, but when Jesus felled it, the corn and everything else the tree had eaten spilled out.

\ea\label{ex:anticaus-2}
\begingl 
\glpreamble tibakaikubu amuke arusu\\
\gla ti-bakaiku-bu amuke arusu\\ 
\glb 3i-throw.away-\textsc{mid} corn rice\\ 
\glft ‘the corn and rice spilled out’\\ 
\endgl
 \trailingcitation{[jxx-n101013s-1.830]}
\xe

Another typical anticausative is expressed in the following example (\ref{ex:anticaus-3}), in which Juana paraphrases my reported reaction to strange sounds from my computer with the middle form \textit{-terereku-bu} ‘get frightened’.

\ea\label{ex:anticaus-3}
\begingl 
\glpreamble pitererekubu, kuina tamicha pue\\
\gla pi-terereku-bu kuina ti-a-micha pue\\ 
\glb 2\textsc{sg}-frighten-\textsc{mid} \textsc{neg} 3i-\textsc{irr}-good well\\ 
\glft ‘you got frightened, because it was not good, obviously’\\ 
\endgl
 \trailingcitation{[jxx-p120430l-1.007]}
\xe\is{anticausative|)}

Last but not least, the middle marker is also sometimes used to express a passive-like\is{passive} impersonal or general reading. This is probably due to influence from Spanish, which uses a reflexive construction\is{reflexive} with an equivalent function. Paunaka usually rather uses first person plural to make general statements. One example of an impersonal construction with a middle marker is given in (\ref{ex:mid-impers}), which is from a description of the preparation of chicha by Juana.

\ea\label{ex:mid-impers}
\begingl 
\glpreamble tetukubu kanela\\
\gla ti-etuku-bu kanela\\ 
\glb 3i-put-\textsc{mid} cinnamon\\ 
\glft ‘one adds cinnamon / cinnamon is added’\\ 
\endgl
 \trailingcitation{[jxx-e110923l-2.121]}
\xe

To sum up, in \isi{transitive}-\isi{intransitive} pairs, the middle marker expresses the notions of reflexivity\is{reflexive} and anticausativity,\is{anticausative} typically associated with middle marking. Among the deponent and high-frequency middle verbs, however, there seems to be a rather random assignment of a middle marker to one or a few verbs each from several semantic classes associated with middle marking, the only exception being posture verbs. This suggests that the middle is not very elaborate at the present stage of the language. Comparison with the closest relatives of Paunaka could shed light on whether there once was a more elaborate system that has been lost nowadays or it is a system that is just coming into existence.\is{active verb|)}

In the remainder of this work, the high frequency and deponent middle verbs are usually glossed as a unit (e.g. \textit{-tupunubu} ‘arrive’ instead of \textit{-tupunu-bu} ‘reach-\textsc{mid}’).\is{middle voice|)} Leaving behind middle voice now, the next section deals with TAME marking.