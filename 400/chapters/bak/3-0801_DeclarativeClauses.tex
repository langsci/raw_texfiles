%!TEX root = 3-P_Masterdokument.tex
%!TEX encoding = UTF-8 Unicode

\chapter{Simple clauses}\label{chap:SimpleClauses}

This chapter is about different kinds of simple clauses. With “simple clauses” I refer to main clauses that only contain a single \isi{finite verb} or non-verbal predicate.\is{non-verbal predication} Clause combination and complex predicates are described in Chapter \ref{sec:ComplexClauses}. The simple clauses described here relate to different types of speech acts: declarative, directive and interrogative. 

Declarative clauses are the topic of \sectref{sec:SimpleClauses} and \sectref{sec:NonVerbalPredication}, the difference being that the first is about those with verbal predicates and the latter about the ones with non-verbal predicates. Both sections include a discussion on the expression of arguments, word order and negation: in \sectref{sec:SimpleClauses} these topics form subsections, but in \sectref{sec:NonVerbalPredication} they are integrated into different subsections, which are ordered by semantic type. \sectref{sec:Imperative} is about imperatives and other directives, such as the prohibitive and hortative. \sectref{sec:Questions} deals with different types of interrogative clauses.

\section{Declarative clauses with verbal predicates}\label{sec:SimpleClauses}\is{declarative clause|(}

The verbal declarative clause minimally consists of an inflected verb.\is{finite verb} Core arguments\is{argument|(} are indexed on the verb,\is{person marking} except for third person objects\is{object} which are not always marked.\footnote{Remember that object markers are mainly reserved for SAP objects. Third person objects can be indexed on the verb by using the third person marker \textit{chÿ-}, which encodes 3>3 relationships, as opposed to \textit{ti-}, which only marks the third person \isi{subject}. Third person object indexing by using \textit{chÿ-} is obligatory with human objects\is{animacy} and optional with non-human objects. See \sectref{sec:NumberPersonVerbs} for more information on person marking.} NPs\is{noun phrase|(} can co-occur, i.e. they are conominal.\is{conomination|(} \sectref{sec:ExpressionSubjects} provides information on the expression of subjects and \sectref{sec:ExpressionObjects} on the expression of objects. There is no flagging of core arguments on nouns. Oblique NPs can be marked by the locative marker or prepositions; this is the topic of \sectref{sec:DeclClausesOBL}. Typically only one core argument is conominated, either a \isi{subject} or an \isi{object}, and this argument usually follows the verb. The most basic word orders\is{word order|(} are thus V, VS and VO. Obliques mostly follow the verb, and also the S or O argument, as far as it is conominated.

The preverbal slot is associated with highlighting. Both \isi{subject} and \isi{object} NPs can occur in this slot as well as obliques.\is{oblique}\is{noun phrase|)} We thus also find SV and OV orders. 

%\emph{TO DO: check preposed obliques: with subordination only? -> nein!}

In the rare cases that the \isi{subject} as well as the \isi{object} is conominated, the most frequent word orders are VOS and SVO. VSO is also possible; OVS, however, is highly exceptional.\is{word order|)} Information on possible word orders is provided in \sectref{sec:WordOrder}. It would certainly be worth to examine information and discourse structure of Paunaka to learn more about the conditions that trigger conomination of  participants and their position in the clause. I have some preliminary thoughts on this issue that I share in this section, but did not undertake a full analysis.

\subsection{Expression of subjects}\label{sec:ExpressionSubjects}\is{alignment|(}
\is{subject|(}

%contrast: chiejiku eka kabe tikutijikutu i eka janejane cheikukuiku, mox-a110920l-2 104

Subjects are obligatorily indexed on verbs by person markers\is{person marking|(} preceding the verb stem. The markers in the first and second person are the same for active intransitive, stative intransitive and transitive verbs. (\ref{ex:active-a}) has an active intransitive verb, (\ref{ex:stative-a}) has a stative intransitive verb and (\ref{ex:transitive-a}) has a transitive verb with a second person subject. In all of these examples, the  subject is marked on the verb with the index \textit{pi-}. For more examples see \sectref{sec:1_2Marking}.

With (\ref{ex:active-a}), Clara expressed her surprise that I bathed in the reservoir of Concepción by repetition of my statement. Some people are afraid of the reservoir, because there are piranhas there.

\ea\label{ex:active-a}
\begingl 
\glpreamble pikubu\\
\gla pi-kubu\\ 
\glb 2\textsc{sg}-bathe\\ 
\glft ‘you took a bath’\\ 
\endgl
\trailingcitation{[cux-c120414ls-1.223]}
\xe

%\ea\label{ex:active-a}
%\begingl 
%\glpreamble piniku\\
%\gla pi-niku\\ 
%\glb 2\textsc{sg}-eat\\ 
%\glft ‘you eat’\\ 
%\endgl
%\trailingcitation{[jxx-a120516l-a.477]}
%\xe

Prior to (\ref{ex:stative-a}), María C. had asked me whether I was not sad because of being in Bolivia without my family. I answered her that I was only a bit sad, and she repeated the statement as follows:

\ea\label{ex:stative-a}
\begingl 
\glpreamble sepitÿjiku pichÿnumi\\
\gla sepitÿ-jiku pi-chÿnumi\\ 
\glb small-\textsc{lim}1 2\textsc{sg}-be.sad\\ 
\glft ‘you are a little sad’\\ 
\endgl
\trailingcitation{[uxx-e120427l.052]}
\xe


(\ref{ex:transitive-a}) was directed to me, when Juana invited me to have a freezie. Freezies come in small plastic bags, which one can open by biting a little hole in one corner.

\ea\label{ex:transitive-a}
\begingl
\glpreamble aa nechÿu pinijabaka naka\\
\gla aa nechÿu pi-nijabaka naka\\
\glb \textsc{intj} \textsc{dem}c 2\textsc{sg}-bite.\textsc{irr} here\\
\glft ‘ah, there you can bite it (open), here’
\endgl
\trailingcitation{[jxx-e110923l-2.103]}
\xe

The third person subject marker \textit{ti-} occurs on \isi{intransitive} verbs and on \isi{transitive} verbs with SAP objects,\is{object} as well as with non-human\is{animacy} non-emphasised objects (see \sectref{sec:3Marking}). In order to mark 3>3 relationships, a subject/object marker \textit{chÿ-} is used. If reference is sufficiently clear, no subject NP needs to co-occur, as in (\ref{ex:make-chicha}), in which the subject referents, Juana’s grandparents being on their way to Moxos, are well established. The sentence describes what her grandparents did when they rested.

\ea\label{ex:make-chicha}
\begingl
\glpreamble tiyÿtipajikanube\\
\gla ti-yÿtipajika-nube\\
\glb 3i-make.chicha.\textsc{irr}-\textsc{pl}\\
\glft ‘they would make chicha’
\endgl
\trailingcitation{[jxx-p151016l-2.057]}
\xe
\is{person marking|)}

%\ea\label{ex:tired-rest}
%\begingl
%\glpreamble tikubiakubunube teumichunube\\
%\gla ti-kubiakubu-nube ti-eumichu-nube\\
%\glb 3i-be.tired-\textsc{pl} 3i-rest-\textsc{pl}\\
%\glft ‘when they got tired, they rested’\\
%\endgl
%\trailingcitation{[jxx-p151016l-2.025]}
%\xe

Subject NPs\is{noun phrase} can co-occur with the person indexes, but they are by no means required syntactically, i.e. they are conominals \citep[cf.][217]{Haspelmath2013}. They are never case-marked.

In (\ref{ex:piti-subj}), the second person pronoun conominates the second person index on the verb. It was produced by Juana, repeating my statement that it was me who went to visit Miguel, not him who visited me.

\ea\label{ex:piti-subj}
\begingl
\glpreamble aa piti piyunu nauku chubiuyae\\
\gla aa piti pi-yunu nauku chÿ-ubiu-yae\\
\glb \textsc{intj} 2\textsc{sg.prn} 2\textsc{sg}-go there 3-house-\textsc{loc}\\
\glft ‘ah, you went there to his house’
\endgl
\trailingcitation{[jxx-e110923l-1.028]}
\xe

In (\ref{ex:fish-subj}), the NP conominating the third person index includes a noun and a demonstrative. This sentence was elicited from Clara, when Swintha wanted to make a statement about the dried piranha we found at the shore of the reservoir in Concepción.

\ea\label{ex:fish-subj}
\begingl
\glpreamble teijuku echÿu jimu\\
\gla ti-eijuku echÿu jimu\\
\glb 3i-stink \textsc{dem}b fish\\
\glft ‘the fish stinks’
\endgl
\trailingcitation{[cux-c120414ls-2.111]}
\xe

% tipurtujaneu ÿbajane = se champaron los chanchos (al agua), jrx-c151001fls-9.63
\is{subject|)}

\subsection{Expression of objects}\label{sec:ExpressionObjects}
\is{object|(}

First and second person objects are obligatorily indexed on the \isi{verb}.\is{person marking|(} Object indexes follow the verb stem.\is{verbal stem} This is true for \isi{transitive} verbs, as in (\ref{ex:OBJ-foll}) and (\ref{ex:smoke-OBJ-smoke}), as well as \isi{ditransitive} verbs, as in (\ref{ex:give-OBJ}) and (\ref{ex:buy-for-you}). If the verb is \isi{ditransitive}, the indexed object has the semantic role of a \isi{recipient}.

The verb in (\ref{ex:OBJ-foll}) has a second person singular object marker (\textit{-pi}). Juana cites her own words here, repeating what she had said to her brother the day before. %For more examples with SAP object markers see \sectref{sec:1_2Marking}.

\ea\label{ex:OBJ-foll}
\begingl 
\glpreamble “nikichupapi tajaitu”\\
\gla ni-kichupa-pi tajaitu\\ 
\glb 1\textsc{sg}-wait.\textsc{irr}-2\textsc{sg} tomorrow\\ 
\glft ‘“I will expect you tomorrow”’\\ 
\endgl
\trailingcitation{[jxx-p120430l-1.127]}
\xe

In (\ref{ex:smoke-OBJ-smoke}), María S. states that smoking is bad. The verb \textit{-kupaku} ‘kill’ carries the first person plural object marker \textit{-bi}.

\ea\label{ex:smoke-OBJ-smoke}
\begingl
\glpreamble tikupakabi\\
\gla ti-kupaka-bi\\
\glb 3i-kill.\textsc{irr}-1\textsc{pl}\\
\glft ‘it (smoking) can kill us’
\endgl
\trailingcitation{[rxx-e120511l.385]}
\xe

%In (\ref{ex:god}), María C. expresses her Christian believe in our origin. The verb carries the first person plural object marker \textit{-bi}.
%
%\ea\label{ex:god}
%\begingl
%\glpreamble bia, chibu tetupaikubi naka apuke\\
%\gla bia chibu ti-etupaiku-bi naka apuke\\
%\glb God 3\textsc{top} 3i-put.down-1\textsc{pl} here ground\\
%\glft ‘God, he put us here into the world’\\
%\endgl
%\trailingcitation{[uxx-p110825l.111]}
%\xe

One example with a \isi{ditransitive} verb is (\ref{ex:give-OBJ}). It was produced by Juana in telling her brother what happened to the photo that Swintha had given her the day before. First, she had been telling this incident in Spanish, but repeated it in Paunaka on request. The verb in this example carries the first person singular object marker \textit{-ne}.

\ea\label{ex:give-OBJ}
\begingl 
\glpreamble ukuine tipunakune chifotone\\
\gla ukuine ti-punaku-ne chi-foto-ne\\ 
\glb yesterday 3i-give-1\textsc{sg} 3-photo-\textsc{possd}\\ 
\glft ‘yesterday she gave me her photo’\\ 
\endgl
\trailingcitation{[jmx-e090727s.041]}
\xe

(\ref{ex:buy-for-you}) was elicited in order to tell Clara that Federico bought something for her. The verb carries the second person singular marker \textit{-bi}.\footnote{The second person singular object marker has two allomorphs: \textit{-bi} is found after default/realis\is{realis} marking with /u/, \textit{-pi} (as in (\ref{ex:OBJ-foll}) above) after irrealis-marked\is{irrealis} morphemes in /a/.}


\ea\label{ex:buy-for-you}
\begingl
\glpreamble chiyÿseikinubi\\
\gla chi-yÿseik-inu-bi\\
\glb 3-buy-\textsc{ben}-2\textsc{sg}\\
\glft ‘he bought it for you’
\endgl
\trailingcitation{[cxx-e120410ls-2.006]}
\xe
\is{person marking|)}

Personal pronouns never conominate object indexes.\is{personal pronoun} In order to put more emphasis on an SAP object, a person-marked form of the preposition \textit{-tÿpi}\is{general oblique} can co-occur; however, this is very rare. The very same preposition is also used in the expression of some kinds of oblique objects, see \sectref{sec:DeclClausesOBL} below (and see also \sectref{sec:borrowed_verbs} for oblique objects in non-verbal predication).
One example in which \textit{-tÿpi} is used as a conominal expression for an object is given in (\ref{ex:OBJ-plusBEN}). The verb takes the person marker \textit{-ne} for the first person singular object. This marker is obligatory and cannot be omitted. The oblique preposition conominating the object follows the verb. The sentence comes from Miguel and is about Swintha not having told him the exact date of her return to Concepción and Santa Rita (after going back to Germany).

\ea\label{ex:OBJ-plusBEN}
\begingl 
\glpreamble kuinakuÿ tikechane nitÿpi\\
\gla kuina-kuÿ ti-kecha-ne ni-tÿpi\\ 
\glb \textsc{neg}-\textsc{incmp} 3i-say.\textsc{irr}-1\textsc{sg} 1\textsc{sg}-\textsc{obl}\\ 
\glft ‘she hasn’t told me, yet’\\ 
\endgl
\trailingcitation{[mxx-d110813s-2.052]}
\xe

%chuji- chujikunube telefenoyae tÿpi nijinepÿi o tÿpi nisinepÿinube, rxx-e181022le

Third person objects\is{person marking|(} are usually not indexed by a marker which follows the stem in declarative clauses (but see \sectref{sec:3_suffixes} for exceptions). The third person marker \textit{chÿ-}/\textit{chi-} can be used instead to express 3>3 relations with human\is{animacy} objects and with non-human objects that the speaker finds worth being explicitly marked (see detailed discussion in \sectref{sec:3Marking}). The \isi{plural} marker \textit{-nube} and the \isi{distributive} marker \textit{-jane} can be added to verbs to express plurality of human and non-human objects,\is{animacy} but since the same markers can also express plurality of third person subjects, the issue of which third person participant is \isi{plural} is not easily sorted out (see \sectref{sec:Verbs_3PL}).

(\ref{ex:bible}) has a human third person object which is expressed solely by use of the marker \textit{chÿ-}. All participants are sufficiently established in discourse by the preceding sentences (in Spanish), and therefore no NP needs to co-occur. The example stems from Juana telling Swintha about the creation of people and some animals and plants. It is interesting how the biblical creation story mixes with elements of non-Christian origin. Prior to this sentence, Juana had narrated that God formed María Eva out of mud as a future wife for Jesus, who did not want to marry a pigeon.

\ea\label{ex:bible}
\begingl
\glpreamble chetuku nauku nekupai\\
\gla chÿ-etuku nauku nekupai\\
\glb 3-put there outside\\
\glft ‘he (God) put her (María Eva) there outside’
\endgl
\trailingcitation{[jxx-n101013s-1.359]}
\xe
\is{person marking|)}

Non-human\is{animacy} third person objects are frequently expressed by NPs.\is{noun phrase} There is no case marking on the noun or any other constituent of the object NP. In many cases, there is no specific index on the verb either to cross-reference the object. This is the case in (\ref{ex:objNPfoll-1}), which was produced by Juana, when telling me how she raised her brother, feeding him with plantain. When he grew a bit older, he could eat some food. The verb carries the third person marker \textit{ti-}, i.e. only the subject is indexed here. 


\ea\label{ex:objNPfoll-1}
\begingl 
\glpreamble tinikumÿnÿ yÿtÿuku\\
\gla ti-niku-mÿnÿ yÿtÿuku\\ 
\glb 3i-eat-\textsc{dim} food\\ 
\glft ‘he ate some food’\\ 
\endgl
\trailingcitation{[jxx-p120430l-2.486]}
\xe

(\ref{ex:OBJ-follow-2}) is from the story about the lazy man. Before he goes to the wood, he prepares his machete, being supposed to make a field to grow food for his family. The object is indexed on the verb in this case, by making use of the third person marker \textit{chÿ-}. The conominal NP follows the verb.

\ea\label{ex:OBJ-follow-2}
\begingl 
\glpreamble chajÿikutuji chitÿmuepane\\
\gla chÿ-ajÿiku-tu-ji chi-tÿmuepa-ne\\
\glb 3-sharpen-\textsc{iam}-\textsc{rprt} 3-knife-\textsc{possd}\\ 
\glft ‘he sharpened his machete, it is said’\\ 
\endgl
\trailingcitation{[mox-n110920l.021]}
\xe

There are very few \isi{ditransitive} verbs in the corpus, (\ref{ex:OBJ-ditr}) offers one example. In this case, both third person objects, \isi{recipient} and theme,\is{patient/theme} are expressed by NPs,\is{noun phrase} the first containing only the demonstrative \textit{eka}, the second a demonstrative + noun. This sentence was produced by Juana, when we were discussing that her little grandson could or should learn Paunaka, since he showed interest in the language. While I insisted on it being necessary that Juana talks with him in Paunaka, she proposed the idea that he could learn it with the help of written material, referring to a sheet with some words and phrases in Paunaka that Swintha had handed over to Juana.

\ea\label{ex:OBJ-ditr}
\begingl
\glpreamble eka nipunaka echÿu ajumerku\\
\gla eka ni-punaka echÿu ajumerku\\
\glb \textsc{dem}a 1\textsc{sg}-give.\textsc{irr} \textsc{dem}b paper\\
\glft ‘this one, I will give him the paper’
\endgl
\trailingcitation{[jxx-e110923l-1.102]}
\xe

\is{object|)}
\is{alignment|)}
\is{conomination|)}


\subsection{Expression of obliques}\label{sec:DeclClausesOBL}
\is{oblique|(}

In this work, obliques are defined as per \citet[]{wals-84} as nominal constituents that modify a verb or clause. According to this definition, obliques are adjuncts, but the issue is not totally clear for Paunaka. Obliques are never required syntactically by any verb. They are neither indexed on the verb in verbal declarative clauses nor obligatorily expressed by an NP (or PP). Nonetheless, a few verbs highly favour the overt expression of an oblique, first and foremost the motion verb\is{motion predicate} \textit{-yunu} ‘go’, as exemplified in (\ref{ex:new23-OBL}) from María C., but also \textit{-etuku} ‘put’ and to a lesser extent \textit{-kuetea} ‘tell’. This is because these verbs semantically require a goal or \isi{addressee}. It could thus be argued that the obliques of these verbs are (optional) arguments.

\ea\label{ex:new23-OBL}
\begingl
\glpreamble tiyunu kampoyae\\
\gla ti-yunu kampo-yae\\
\glb 3i-go countryside-\textsc{loc}\\
\glft ‘she went to the countryside’
\endgl
\trailingcitation{[cux-c120510l-1.205]}
\xe

% = he went back home, mox-n110920l.035

It has been argued that the distinction between arguments and adjuncts is possibly not a crosslinguistic but a language-particular one \citep[]{Haspelmath2014}. Concerning Paunaka, there is no general difference in the expression of those kinds of obliques that are semantically entailed and other constituents that seem to be completely optional by the semantics of the verb, i.e. those that clearly qualify as adjuncts, and I do not know of any test that would unambiguously set apart arguments from adjuncts in Paunaka. This is why I decided not to distinguish them in the analysis for the time being.

Thus, obliques comprise spatial, temporal, benefactive,\is{beneficiary} instrumental, cause,\is{instrument/cause} and \isi{comitative} relations, and depending on the specific kind of relation, they can take the general locative marker\is{locative marker|(} \textit{-yae} (see \sectref{sec:Locative}), or they can be marked by a \isi{preposition} (see \sectref{sec:Adpositions}). Source\is{source} expressions are formed with the help of a preposition and can additionally take the locative marker. Obliques may also be completely unmarked, this is what we frequently find in the expression of goals in combination with the motion verb\is{motion predicate} \textit{-yunu} ‘go’.\footnote{This could actually be seen as a criterion for argument status. Whether or not a locative marker shows up seems to largely depend on the kind of \isi{noun} used as goal expression: toponyms\is{toponym} are likely to occur without locative marker, which does not come as a surprise \citep[cf.][291]{StolzAL2014}. In addition, the place nouns\is{place noun} \textit{uneku} ‘town’ and \textit{asaneti} ‘field’ often show up without locative marking in Paunaka, though the possessed form \textit{-sane} ‘field’ rather takes a locative marker. It remains to be checked how these nouns behave when combined with verbs other than \textit{-yunu} ‘go’.}\is{locative marker|)}
%Some goal expressions may also be formed with tÿpi + -yae!

I will start the description with pronominal expressions of obliques. There is an oblique \isi{topic pronoun} \textit{nebu} found with locative and temporal relations (see also \sectref{sec:FocPron}). It is often combined with deranked verbs, which is analysed here as indicating a cleft construction, a topic described in \sectref{sec:Clefts}. It may, however, also occur in declarative clause. \textit{Nebu} always precedes the verb.

(\ref{ex:nebu-obl-1}) is formed with the oblique pronoun \textit{nebu}. It comes from Juana’s account about her daughter who went to Spain but was deported for not having a valid visa. She was arrested together with other people and brought to a room, where they received some food. The room (or rather its location upstairs, on another floor) has been mentioned directly before.

\ea\label{ex:nebu-obl-1}
\begingl
\glpreamble nebu chupununube yÿtÿuku\\
\gla nebu chÿ-upunu-nube yÿtÿuku\\
\glb 3\textsc{obl.top.prn} 3-bring-\textsc{pl} food\\
\glft ‘there they brought (them) food’
\endgl
\trailingcitation{[jxx-p110923l-1.314]}
\xe

The context of the next example, which also contains \textit{nebu}, is as follows: María S. had described the quarter where her daughters live in Santa Cruz. I had been in this quarter once with Miguel to visit his daughter, who also lives there, so I told María S. that I knew it. She replied with (\ref{ex:nebu-obl-2}), in which \textit{nebu} refers to the quarter we had been talking about.

\ea\label{ex:nebu-obl-2}
\begingl
\glpreamble ja no ve nebu chubu nijinepÿinube\\
\gla ja {no ve} nebu chÿ-ubu ni-jinepÿi-nube\\
\glb \textsc{afm} {right} 3\textsc{obl.top.prn} 3-be 1\textsc{sg}-daughter-\textsc{pl}\\
\glft ‘ah, you know? there live my daughters’
\endgl
\trailingcitation{[rxx-e120511l.256]}
\xe

Locations and goals can be expressed by adding the locative marker\is{locative marker|(} \textit{-yae} to a noun, as in (\ref{ex:obl-yae-1}), which was produced by Miguel in telling me about the history of Santa Rita and his own personal history. After living in Naranjito for some time, he went to Santa Cruz and only came back to live in the Chiquitania again after 20 years.

\ea\label{ex:obl-yae-1}
\begingl
\glpreamble niyunu Santa Kruyae\\
\gla ni-yunu {Santa Kru}-yae\\
\glb 1\textsc{sg}-go {Santa Cruz}-\textsc{loc}\\
\glft ‘I went to Santa Cruz’
\endgl
\trailingcitation{[mxx-p110825l.074]}
\xe

(\ref{ex:obl-yae-2}) comes from a description by Juana of how to cook with a clay pot. 

\ea\label{ex:obl-yae-2}
\begingl
\glpreamble pijÿuka petukatu yÿkÿyae\\
\gla pi-jÿuka pi-etuka-tu yÿkÿ-yae\\
\glb 2\textsc{sg}-light.fire.\textsc{irr} 2\textsc{sg}-put.\textsc{irr}-\textsc{iam} fire-\textsc{loc}\\
\glft ‘you light fire, then you put it onto the fire’
\endgl
\trailingcitation{[jmx-d110918ls-1.009]}
\xe

While both locative-marked NPs in (\ref{ex:obl-yae-1}) and (\ref{ex:obl-yae-2}) above express goals, in the following example we find \textit{-yae} on a noun that expresses a static location. The example comes from Miguel’s telling of the story of the lazybones, who only swings on a liana (like in a hammock) and plays the flute in the woods instead of working.

\ea\label{ex:obl-yae-3}
\begingl
\glpreamble tebibiku echÿu jupipiyae\\
\gla ti-ebibiku echÿu jupipi-yae\\
\glb 3i-swing \textsc{dem}b liana.sp-\textsc{loc}\\
\glft ‘he swung on the liana’
\endgl
\trailingcitation{[mox-n110920l.067]}
\xe

In (\ref{ex:obl-unm}) we have a goal expression without locative marker. It was produced by Juana, who still lived in Santa Cruz at that time. She spoke about Federico.

\ea\label{ex:obl-unm}
\begingl
\glpreamble eka semana tiyuna Santa Rita\\
\gla eka semana tiyuna {Santa Rita}\\
\glb \textsc{dem}a week 3i-go.\textsc{irr} {Santa Rita}\\
\glft ‘this week he will go to Santa Rita’
\endgl
\trailingcitation{[jxx-p110923l-1.098]}
\xe

Another example of an unmarked oblique is (\ref{ex:obl-unm-2}), which is a commentary by Juana, when Miguel was telling the story about the fox and the jaguarundi. The story reaches its climax, the fox is drunk, the jaguarundi has fled to the woods, the dogs of the owner of the house they had broken in chase them. Apparently, Juana expects that the jaguarundi meets the fox again in the woods, which she expresses by this sentence. Miguel, however, goes on telling the story without such an encounter. It is possible that the adverb \textit{nauku} ‘there’ which is preposed to the locative NP has an influence on the omission of locative marking here, but it is not necessarily the case that \textit{-yae} is omitted if \textit{nauku} is preposed. Compare (\ref{ex:piti-subj}) above.

\ea\label{ex:obl-unm-2}
\begingl
\glpreamble titupu nauku kimenu\\
\gla ti-tupu nauku kimenu\\
\glb 3i-find there woods\\
\glft ‘he met him there in the woods’
\endgl
\trailingcitation{[jmx-n120429ls-x5.412]}
\xe

%depue Krara tiyunutu uneku, Kuana tikubipu uneku, depue tepajÿkunubetu uneku, rxx-p181101l-2.263-264
%
%tisukuejikuji chinabakÿyae = se cagó en su boca
%nisachu biyuna bisemaikupa takÿra nauku bibÿkupa chubiyaeyae = entramos en la casa
%titupunubuji kimenukÿyae = he arrived in the woods, mox-n110920l.025


%Non-exact locations can also be expressed by an NP with the preposition \textit{tÿpi}. -> no generelly only time, but look for: Turuxhi tÿpi Conce; tÿpi Cochabamba
Source\is{source|(} expressions are introduced by \textit{tukiu} ‘from’. The noun often takes the locative marker in this case, but not necessarily so.
(\ref{ex:obl-tukiu-1}) is an example which has both \textit{tukiu} and a locative-marked noun. It comes from the story about the creation of the world told by Juana. The main character is a very strong young man in this part of the story, which explains why specific trees and animals have specifically shaped (body) parts. In case of the silk floss tree, this is because it had swallowed all the crops that were meant by God for the people and animals to eat. The animals try to get back their food but fail to pull the silk floss tree out of the water, where it grows. Finally the strong young man helps them and they succeed.

\ea\label{ex:obl-tukiu-1}
\begingl
\glpreamble tukiu ÿneyae chetukunube echÿu yuke\\
\gla tukiu ÿne-yae chÿ-etuku-nube echÿu yuke\\
\glb from water-\textsc{loc} 3-put-\textsc{pl} \textsc{dem}b riverbank\\
\glft ‘from the water they put it (the silk floss tree) onto the riverbank’
\endgl
\trailingcitation{[jxx-n101013s-1.784]}
\xe

Another example of a source expression with \textit{tukiu} and the locative marker is (\ref{ex:obl-tukiu-2}). Miguel produced this sentence when speaking with Juan C. about their past. Miguel had talked about the load of work they had to do in \isi{Altavista} and Juan C. had just stated that they searched for another place to live, which is then presented as the reason for their moving away from Altavista by Miguel. 

\ea\label{ex:obl-tukiu-2}
\begingl
\glpreamble nechikue bibÿbÿsu tukiu Turuxhiyae\\
\gla nechikue bi-bÿbÿsu tukiu Turuxhi-yae\\
\glb therefore 1\textsc{pl}-come from Altavista-\textsc{loc}\\
\glft ‘therefore we came from Altavista’
\endgl
\trailingcitation{[mqx-p110826l.018]}
\xe

One example in which the source does not carry the locative marker is (\ref{ex:obl-tukiu-3}). It was produced by Miguel in speaking about Swintha.

\ea\label{ex:obl-tukiu-3}
\begingl
\glpreamble kapunu, titupunubu tukiu Alemania\\
\gla kapunu ti-tupunubu tukiu Alemania\\
\glb come 3i-arrive from Germany\\
\glft ‘she came, she arrived from Germany’
\endgl
\trailingcitation{[mxx-d110813s-2.028]}
\xe
\is{source|)}
\is{locative marker|)}

Temporal expressions can be introduced by \textit{tÿpi}. This can be seen in (\ref{ex:obl-tÿpi-1}), which was produced by Juan C., when he and Miguel were discussing the possibility of some rain in August.

\ea\label{ex:obl-tÿpi-1}
\begingl
\glpreamble tÿpi Santa Rosa repente tikeba pario\\
\gla tÿpi {Santa Rosa} repente ti-keba pario\\
\glb \textsc{obl} {Saint Rosa} maybe 3i-rain.\textsc{irr} some\\
\glft ‘around Saint Rosa(’s day) maybe it rains a bit’
\endgl
\trailingcitation{[mqx-p110826l.627]}
\xe

\textit{Tÿpi} can also be used to express periods of time, as in (\ref{ex:obl-tÿpi-2}), in which Juana talks about her plans to travel to Spain.

\ea\label{ex:obl-tÿpi-2}
\begingl
\glpreamble nauku niyuna tÿpi treschÿ kuje\\
\gla nauku ni-yuna tÿpi treschÿ kuje\\
\glb there 1\textsc{sg}-go.\textsc{irr} \textsc{obl} three month\\
\glft ‘I will go there for three months’
\endgl
\trailingcitation{[jxx-p110923l-1.260-261]}
\xe

In addition to that, \textit{tÿpi} can be used to encode benefectives\is{beneficiary} or recipients.\is{recipient} When \textit{tÿpi} is used in such a way, it can also occur without an NP and in these cases, it takes a person marker\is{person marking} as in (\ref{ex:obl-tÿpi-3}), which was elicited from Juana.

\ea\label{ex:obl-tÿpi-3}
\begingl
\glpreamble nikujemu chitÿpi\\
\gla ni-kujemu chi-tÿpi\\
\glb 1\textsc{sg}-be.angry 3-\textsc{obl}\\
\glft ‘I am angry with him’
\endgl
\trailingcitation{[jxx-e190210s-01]}
\xe

(\ref{ex:obl-tÿpi-4}) was produced by Miguel and directed towards María C. to tell her that we were leaving an invitation for a workshop on Paunaka with her, which the PDP team organised in 2011.

 \ea\label{ex:obl-tÿpi-4}
\begingl
\glpreamble binejika eka ajumerku pitÿpi\\
\gla bi-nejika eka ajumerku pi-tÿpi\\
\glb 1\textsc{pl}-leave.\textsc{irr} \textsc{dem}a paper 2\textsc{sg}-\textsc{obl}\\
\glft ‘we will leave this paper with you’
\endgl
\trailingcitation{[mux-c110810l.011]}
\xe

\textit{Tÿpi} can introduce purpose clauses (see \sectref{sec:PurposeClauses}), and it is also found together with NPs that express the aim,\is{aim/result} \is{purpose} or result of an action.

(\ref{ex:obl-tÿpi-6}) is from the creation story told by Juana. God has called María Eva in order to tell her to make linen for clothes, after she and Jesus had eaten the forbidden apple. The object of the clause is \textit{riensu} ‘linen’ and the PP that follows explains, what the linen is meant for, \textit{tÿpi pimÿuna} ‘for your future clothes’, the clothes people are supposed to wear from that point on.

\ea\label{ex:obl-tÿpi-6}
\begingl
\glpreamble jaje bana riensu tÿpi pimÿuna\\
\gla jaje bi-ana riensu tÿpi pi-mÿu-ina\\
\glb \textsc{hort} 1\textsc{pl}-make.\textsc{irr} linen \textsc{obl} 2\textsc{sg}-clothes-\textsc{irr.nv}\\
\glft ‘let’s make linen for your future clothes’
\endgl
\trailingcitation{[jxx-n101013s-1.501]}
\xe

The kinds of obliques described above, i.e. the locative-marked ones and the ones with the prepositions \textit{tukiu}\is{source} and \textit{(-)tÿpi}\is{general oblique} are quite common. More infrequently, we find also obliques with the semantic roles of instruments, causes\is{instrument/cause|(} or comitatives\is{comitative}. Instruments and causes are formed with the preposition \textit{-keuchi} and comitatives with \textit{-aj(i)echubu}. The first of those prepositions, \textit{-keuchi} is usually person-marked,\is{person marking|(} regardless of whether an NP follows. The \isi{comitative} preposition \textit{-aj(i)echubu} is always person-marked.\is{person marking|)}

In (\ref{ex:obl-keuchi-1}) Juana tells me about how she went fishing with her grandmother. The women fish with nets, while men fish with hooks. If the net caught a fish, they would take this fish out and kill it with a stick. The stick is marked as the instrument used for killing by \textit{-keuchi}.

\ea\label{ex:obl-keuchi-1}
\begingl
\glpreamble kue tituika bikupaka chikeuchi yÿkÿke\\
\gla kue ti-tuika bi-kupaka chi-keuchi yÿkÿke\\
\glb if 3i-hunt.\textsc{irr} 1\textsc{pl}-kill.\textsc{irr} 3-\textsc{ins} stick\\
\glft ‘if it (the net) caught (fish), we would kill them with a stick’
\endgl
\trailingcitation{[jxx-p120430l-1.073]}
\xe

An example in which \textit{-keuchi} marks a cause is (\ref{ex:obl-keuchi-2}), which comes from Miguel telling Alejo the \isi{frog story}. He produced this sentence when looking at the picture on which the beehive lies on the ground because the dog had jumped against it and made it fall. In this case no NP follows \textit{chikeuchi}, since it is sufficiently clear from the context who is responsible. 

\ea\label{ex:obl-keuchi-2}
\begingl
\glpreamble tibÿtupaikubutu chikeuchi\\
\gla ti-bÿtupaikubu-tu chi-keuchi\\
\glb 3i-fall-\textsc{iam} 3-\textsc{ins}\\
\glft ‘it (the beehive) fell down because of it (the dog)’
\endgl
\trailingcitation{[mtx-a110906l.104]}
\xe
\is{instrument/cause|)}

One example of \textit{-aj(i)echubu} is given in (\ref{ex:obl-com}). Clara answered María C.’s question about where my daughter was. I had taken her and my husband with me when I first came to Bolivia to work with the Paunaka people in 2011, but in 2012, I went alone.

\ea\label{ex:obl-com}
\begingl
\glpreamble chinejiku chajichubu chÿa\\
\gla chi-nejiku chÿ-ajechubu chÿ-a\\
\glb 3-leave 3-\textsc{com} 3-father\\
\glft ‘she left her (her daughter) with her father’
\endgl
\trailingcitation{[cux-120410ls.081]}
\xe

\is{oblique|)}
\is{argument|)}

In the following section, the information given up to here is brought together and different possible word orders are presented.

\subsection{Word order}\label{sec:WordOrder}
\is{word order|(}
\is{conomination|(}

Paunaka has a wide range of possible word orders regarding nominal expressions of arguments: VS, VO, SV, OV, VOS, VSO, SVO. If we consider the argument indexes on the verb, however, the order is rigid: subject indexes always precede the verb stem,\is{verbal stem} i.e. the order is s-V for \isi{intransitive} verbs. First and second person object indexes always follow the verb stem, yielding s-V-o. Third person objects are either indexed by a marker that encodes 3>3 relationships on verbs or remain unmarked, thus we have s+o-V or s-V argument orders on verbs with third person objects.\is{person marking}

It is common that only one core argument is conominated and the most basic word orders can thus be considered VS and VO. It is also very common that a clause contains nothing but a verb if subject and, if applicable, object participants are well-established in discourse. V and VS sentences are mainly found with \isi{intransitive} verbs and are related to topic\is{topic|(} continuity, topic establishment and topic change. VO order is typical for \isi{transitive} verbs. This is connected to the subject being an established topic\is{topic|)} and the object providing new information, i.e. having the role of \isi{focus} \citep[cf.][]{Lambrecht1994}.

For convenience, the word order types of the discussed sentences are placed above the examples in this section. Word orders of material that is not relevant for the discussion is given in parenthesis; this is usually other juxtaposed sentences that I did not want to omit, because intonation suggested that they closely belonged to the sentence I want to discuss or because I believe they are indicative for information structure.

Let us start with two examples that contain nothing more than the inflected verb. (\ref{ex:orderV-1}) represents the answer Juana gave to my (stammered) question after her relation to her grandchildren. The grandchildren are established as participants by my question, so they do not need to be conominated by an NP. They are encoded by the plural marker on the verb. The first person subject participant is expressed by the index preceding the verb stem.

\ea\label{ex:orderV-1}
\begingl
\glpreamble \textup{V:}\\nichaneikunube\\
\gla ni-chaneiku-nube\\
\glb 1\textsc{sg}-care.for-\textsc{pl}\\
\glft ‘I care for them’
\endgl
\trailingcitation{[jxx-p110923l-1.161]}
\xe
%event-reporting

Prior to (\ref{ex:orderV-2}), I had made a statement about José being the only one who stayed in the place where the whole family Supepí Yabeta used to live together. Thus, it is clear that reference is made to José, when María S. states that he is alone.

\ea\label{ex:orderV-2}
\begingl
\glpreamble \textup{V:}\\ tipÿsisikubu\\
\gla ti-pÿsisikubu\\
\glb 3i-be.alone\\
\glft ‘he is alone’
\endgl
\trailingcitation{[rxx-e120511l.187]}
\xe
%José is topical

If only one NP accompanies the verb, its unmarked position is that following the verb. However, this is not true for NPs containing personal or topic pronouns.\is{pronoun} These pronouns always precede the verb, see below. First, I give some examples with NPs containing nouns that follow the verb. (\ref{ex:SUBJ-follow-3}) and (\ref{ex:SUBJ-follow}) have VS and (\ref{ex:VO-1}) and (\ref{ex:VO-2}) VO order.

(\ref{ex:SUBJ-follow-3}) is an example with a subject following the verb. Although the subject participant of this clause, the jaguar, \textit{isini}, has been talked about by María S. in the previous clause, she decided to express it by an NP here, maybe because this is the highlight and also a kind of summary towards the end of the story of the fox and the jaguar.

\ea\label{ex:SUBJ-follow-3}
\begingl 
\glpreamble \textup{VS:}\\tipakutu isini\\
\gla ti-paku-tu isini\\ 
\glb 3i-die-\textsc{iam} jaguar\\
\glft ‘the jaguar died’\\ 
\endgl
\trailingcitation{[rxx-n120511l-1.040]}
\xe
%full example: tepakutu isini tijÿchÿichieku ÿne = the jaguar died, he drowned in the water -> deleted because not sure whether -e- in the verb can be considered an applicative for ÿne!

(\ref{ex:SUBJ-follow}) is another example with a subject NP following the verb. It is from the description of the \isi{frog story} by Juana. She had already mentioned the dog in the preceding clause, she even mentioned the very same event of the dog’s running. Repetition of the subject NP has two functions here. First, it creates more emphasis on the whole sentence, because Juana found it funny, and second, this sentence also provides a summary of what she had been telling before about the dog.

\ea\label{ex:SUBJ-follow}
\begingl 
\glpreamble \textup{VS:}\\tikutikubutu kabe\\
\gla ti-kutikubu-tu kabe\\ 
\glb 3i-run-\textsc{iam} dog\\ 
\glft ‘the dog is running’\\ 
\endgl
\trailingcitation{[jxx-a120516l-a.146]}
\xe

Objects\is{object} also frequently follow the verb. In (\ref{ex:VO-1}), the subject of the sentence, two men who go into the woods to hunt, is a well-established topic and does not need to be conominated. The object of this sentence, the collared peccaries that the men hunt, is new information and thus expressed by an NP. This sentence is part of the story about the two men who meet the devil in the woods that was told by María S.

\ea\label{ex:VO-1}
\begingl
\glpreamble \textup{VO:}\\tituikunubeji tijapÿ\\
\gla ti-tuiku-nube-ji tijapÿ\\
\glb 3i-hunt-\textsc{pl}-\textsc{rprt} collared.peccary\\
\glft ‘they hunted collared peccary, it is said’
\endgl
\trailingcitation{[rxx-n120511l-2.17]}
\xe

The next example comes from the same tale, but this time told by Miguel. The men have already met the devil and one of them gives him some of the meat. But the devil, still being hungry has demanded the heads of the pigs (since Miguel uses \textit{ÿba} ‘pig’ instead of \textit{tijapÿ} ‘collared peccary’ in his story). Thus the man gives him the heads in (\ref{ex:VO-2}). The verb carries the person marker \textit{chÿ-}. This person marker refers to the subject, the man, and to the object, the heads. Additionally, the object is expressed by the NP \textit{echÿu chichÿtijane} ‘their heads’, which follows the verb. The heads have been mentioned in the previous clauses, first as an object conominated by an NP, then as the subject of an existential clause without being overtly expressed. Both of these preceding clauses are formed as direct speech in the narrative. It may be the switch from subject of the existential clause to object of the verbal clause or the switch from direct speech to report that triggered use of an NP here, or – of course – both factors may have an influence.


\ea\label{ex:VO-2}
\begingl
\glpreamble \textup{VO:}\\chupunukuji echÿu chichÿtijane\\
\gla chÿ-upunu-uku-ji echÿu chi-chÿti-jane\\
\glb 3-bring-\textsc{add}-\textsc{rprt} \textsc{dem}b 3-head-\textsc{distr}\\
\glft ‘he also brought their heads (of the pigs)’
\endgl
\trailingcitation{[mxx-n101017s-1.050]}
\xe

There is one preverbal slot, which is used to indicate special discourse status. Subject\is{subject} as well as object\is{object} NPs can occur in this slot giving rise to SV and – more rarely – OV orders. 

NPs that precede the verb can have either topic\is{topic|(} or focus\is{focus|(} status.  As for topical NPs, the reason why they occur pre-verbally can be an indication of a change of topic\is{topicalisation} or re-activation of a non-active topic. However, not every change or re-activation of topic\is{topic|)} goes along with preverbal NP placement, and more research on information structure is certainly necessary to determine the exact conditions under which an NP can be preposed. Focus NPs may be preposed to indicate argument focus, i.e. the relation of the preposed NP to the rest of the proposition is new information or this information is highlighted \citep[cf.][228]{Lambrecht1994}

Personal and topic pronouns\is{pronoun} are only used for special emphasis and they always precede the verb,\is{focus|)} see (\ref{ex:Pron-prec}) and (\ref{ex:PRN-SV}) for a first person plural and a first person singular pronoun, respectively. Note that there is no personal pronoun for the third person, but the demonstrative\is{nominal demonstrative|(} \textit{echÿu} or the \isi{topic pronoun} \textit{chibu} can be used instead. The demonstrative can precede or follow the verb. (\ref{ex:echÿu-pron}) is an example with a demonstrative that accompanies a verb marked for a third person subject by the person marker \textit{ti-}, (\ref{ex:chibuchibu}) has a third person subject conominated by the topic pronoun \textit{chibu}.\is{nominal demonstrative|)}  %they usually do not replace the person marker on the verb unlike in other Arawakan languages like Nanti \citep[342]{Michael2008}

(\ref{ex:Pron-prec}) is narrated direct speech in the story about the man who loses the cows of his \textit{patrón}, finds them with a spirit and gets enchanted by that spirit. Towards the end of the story, he brings the cows to a village for the people there to eat. This is what he tells the people, before he leaves them to go back to the place of the spirit again. 

\ea\label{ex:Pron-prec}
\begingl 
\glpreamble \textup{SV:}\\ “biti biyunupunatu”\\
\gla biti bi-yunupuna-tu\\ 
\glb 1\textsc{pl.prn} 1\textsc{pl}-go.back.\textsc{irr}-\textsc{iam}\\ 
\glft ‘“we go back now”’\\ 
\endgl
\trailingcitation{[mxx-n151017l-1.92]}
\xe

(\ref{ex:PRN-SV}) was Miguel’s answer to Swintha’s question what he was doing while his wife was making rice bread. Actually, his answer that he was doing nothing but watching was a joke. He was only distracted at that very moment, answering our questions. 

\ea\label{ex:PRN-SV}
\begingl
\glpreamble \textup{SV:}\\nÿti nimumukujiku\\
\gla nÿti ni-imumuku-jiku\\
\glb 1\textsc{sg.prn} 1\textsc{sg}-look-\textsc{lim}1\\
\glft ‘I am only watching’
\endgl
\trailingcitation{[mxx-e120415ls.071]}
\xe


%(\ref{ex:SUBJ-PRON}) comes from an elicitation session. It was offered by María S. as another example with the verb \textit{-yÿbamukeiku} ‘peel (in mortar)’. 
%
%SV[VO]
%\ea\label{ex:SUBJ-PRON}
%\begingl 
%\glpreamble nÿti nisachu niyubamukeika arusu\\
%\gla nÿti ni-sachu ni-yÿbamukeika arusu\\ 
%\glb I 1\textsc{sg}-want 1\textsc{sg}-peel.in.mortar.\textsc{irr} rice\\ 
%\glft ‘I want to peel rice in mortar’\\ 
%\endgl
%\trailingcitation{[rxx-e141230s.201]}
%\xe

(\ref{ex:echÿu-pron}) comes from Miguel telling José the \isi{frog story}. This sentence is a repetition of his last utterance, which had the same verb, but the subject was expressed by the noun \textit{peÿ} ‘frog’ and the place from where it left, the big glass, was also expressed by an NP. Since the frog and the glass are thus sufficiently established in discourse, it is not necessary to repeat them again. Note that it is not uncommon to repeat propositions, this is usually done, when a discourse topic or a specific section about this topic comes to an end. Repetition also plays a role in back-chanelling.


\ea\label{ex:echÿu-pron}
\begingl 
\glpreamble \textup{VS:}\\mm, tibÿchÿutu echÿu\\
\gla mm ti-bÿchÿu-tu echÿu\\ 
\glb \textsc{intj} 3i-leave-\textsc{iam} \textsc{dem}b\\ 
\glft ‘mm, it left’ \\ 
\endgl
\trailingcitation{[mox-a110920l-2.026]}
\xe

In (\ref{ex:chibuchibu}), María C. expresses her faith in God. She uses the topic pronoun \textit{chibu} to refer back to God, who had been mentioned before by using an NP.

\ea\label{ex:chibuchibu}
\begingl
\glpreamble \textup{SV:}\\chibu tetupaikubi naka apuke\\
\gla chibu ti-etu-pai-ku-bi naka apuke\\
\glb 3\textsc{top.prn} 3i-put-\textsc{clf:}ground-\textsc{th}1-1\textsc{pl} here ground\\
\glft ‘he put us here on earth’
\endgl
\trailingcitation{[uxx-p110825l.111]}
\xe


Subject NPs containing a noun can also precede the verb for special emphasis, e.g. indicating contrast or a change of topic,\is{topic|(} but probably also for stylistic reasons.  Emphasis is not per se excluded if the subject follows the verb.
%New participants can be introduced into the discourse by using subject NPs, although this is mostly done either with object NPs or with subject NPs in presentational clauses including the non-verbal predicates \textit{kaku} ‘exist’ or \textit{kapunu} ‘come’. One example of introduction of a new participant is (\ref{ex:late-father-bb}), which is from Juana telling how her grandparents were deprived of their cows by \textit{karay}. Her father found out, where the cows were brought. He had not been mentioned before; however, the referent is clearly identifiable, since ‘father’ is, of course, an unambiguous kinship relation in most cases. The subject precedes the verb in this case.
%
%
%\ea\label{ex:late-father-bb}
%\begingl
%\glpreamble  \textup{SV, (non-verbal PRED S, VXO):}\\i eka nÿabane tanÿma tiyunu kapunumÿnÿ eka tenekubu jamuike baka\\
%\gla i eka nÿ-a-bane tanÿma ti-yunu kapunu-mÿnÿ eka ti-eneku-bu jamuike baka\\
%\glb and \textsc{dem}a 1\textsc{sg}-father-\textsc{rem} now 3i-go come-\textsc{dim} \textsc{dem}a 3i-leave-\textsc{mid} pampa cow\\
%\glft ‘and my late father went, he came there, the cows were left in the pampa’\\
%\endgl
%\trailingcitation{[jxx-e150925l-1.237]}
%\xe


% tipurtujaneu ÿbajane = se champaron los chanchos (al agua), jrx-c151001fls-9.63

%Another example is (\ref{ex:SUBJ-follow-2})
%
%\ea\label{ex:SUBJ-follow-2}
%\begingl 
%\glpreamble i tiyunu nijinepÿi chumu chipatrune\\
%\gla i tiyunu nijinepÿi chumu chipatrune\\ 
%\glb \\ 
%\glft \\ 
%\endgl
%\trailingcitation{[jxx-p110923l-1.328]}
%\xe

The next example, (\ref{ex:SUBJ-prec}), is a case in which the subject precedes the verb for contrastive topic. It is taken from the same recording as (\ref{ex:SUBJ-follow}). Previously, Juana talked about the dog, now she provides information about the boy, \textit{aitubuche}. This change triggered the use of a subject NP preceding the verb.

\ea\label{ex:SUBJ-prec}
\begingl 
\glpreamble  \textup{SV:}\\i eka aitubuche tipÿtapaikubutu\\
\gla i eka aitubuche ti-bÿtupaikubu-tu\\ 
\glb and \textsc{dem}a boy 3i-fall-\textsc{iam}\\ 
\glft ‘and the boy fell down’\\ 
\endgl
\trailingcitation{[jxx-a120516l-a.148]}
\xe
\is{topic|)}

An example of a contrastive \isi{focus} subject NP in preverbal position is (\ref{ex:SUBJ-prec-foc}), where Juana and I were discussing the consumption of frogs. It is not usual among the speakers of Paunaka to eat frogs, but Juana had heard that some species are tasty and that there are people who eat them, and she gives information about the presumed nationalities of frog eaters in the sentence.


\ea\label{ex:SUBJ-prec-foc}
\begingl 
\glpreamble  \textup{SV, SV:}\\eka japonesnube tinikunube los chino tinikunube\\
\gla eka japones-nube ti-niku-nube {los chino} ti-niku-nube\\ 
\glb \textsc{dem}a Japanese-\textsc{pl} 3i-eat-\textsc{pl} {the Chinese} 3i-eat-\textsc{pl}\\ 
\glft ‘the Japanese eat them (frogs), the Chinese eat them’\\ 
\endgl
\trailingcitation{[jxx-a120516l-a.482]}
\xe

%I chipujunepaikutu kupisaÿrÿ tiyunuji eka kupisaÿrÿ tiyayaumiji = y el zorro lo empujó (al tigre)  se fue el zorro contento, jmx-n120429ls-x5.278


OV order is less common than SV. It is used if speakers want to put special emphasis on the object. This is the case when speakers use the \isi{topic pronoun} \textit{chibu} as an object, as in (\ref{ex:chibu-OV-2}) from the same passage as (\ref{ex:objNPfoll-1}) above. Juana had just described that she fed her brother with plantain, then she states:

\ea\label{ex:chibu-OV-2}
\begingl
\glpreamble  \textup{OV:}\\chibu bekichumÿnÿ\\
\gla chibu bi-ekichu-mÿnÿ\\
\glb 3\textsc{top.prn} 1\textsc{pl}-invite-\textsc{dim}\\
\glft ‘this we gave (lit.: invited) him’
\endgl
\trailingcitation{[jxx-p120430l-2.482]}
\xe

(\ref{ex:chibu-OV}) comes from an elicitation context with María S. However, it was not requested as a direct translation, but rather originated from the elicited context (which included eating, conversing and going). As in the previous example, \textit{chibu} is the object of the clause and it precedes the verb.

\ea\label{ex:chibu-OV}
\begingl
\glpreamble  \textup{OV:}\\chibu bichujijikubu\\
\gla chibu bi-chujijiku-bu\\
\glb 3\textsc{top.prn} 1\textsc{pl}-talk-\textsc{mid}\\
\glft ‘this we talked about’
\endgl
\trailingcitation{[rxx-e181020le]}
\xe


Another example in which an object is emphasised and thus placed in preverbal position is given in (\getfullref{ex:tomato-OV.3}). It is a confirmation to my surprised reaction (\getfullref{ex:tomato-OV.2}) to Juana’s statement that frogs are prepared with tomatoes (\getfullref{ex:tomato-OV.1}). 

\ea\label{ex:tomato-OV}
  \ea\label{ex:tomato-OV.1}
 \begingl 
\glpreamble  \textup{(non-verbal PRED),VO:}\\\textup{j:} michaniki, tetuku tomate\\
\gla michaniki ti-etuku tomate\\ 
\glb tasty 3i-put tomato\\ 
\glft ‘it is tasty, they put tomatoes in (the cans with the frogs)’\\ 
\endgl
  \ex\label{ex:tomato-OV.2}
 \begingl 
\glpreamble \textup{l:} ¡aa, tomate!\\
\gla aa tomate\\ 
\glb \textsc{intj} tomato\\ 
\glft ‘ah, tomatoes!’\\ 
\endgl
  \ex\label{ex:tomato-OV.3}
 \begingl
\glpreamble  \textup{OV:}\\\textup{j:} ja, tomate tetuku\\
\gla ja tomate ti-etuku\\
\glb \textsc{afm} tomato 3i-put\\
\glft ‘yes, tomatoes they put in’
\endgl
\trailingcitation{[jxx-a120516l-a.468-470]}
\z
\xe

%echÿu timuÿji aparte chetukunube, jxx-p15016l-2.132 ebd. 206
%esekeÿ chetukunube jxx-p120430l-2.034
%kÿike tiyubajika betuicha pujukeke = maní molía y lo metemos al mote, rxx-p181101l-2.204

If both subject and object are conominated, we mostly find VOS or SVO order. There are also some examples of VSO in the corpus. As for OVS, however, it was only found in elicitation contexts, though it was not rejected as being completely wrong, given that the relations between the participants were sufficiently clear. I will provide examples for all orders. (\ref{ex:VOS-1}) and (\ref{ex:VOS}) have VOS order, (\ref{ex:SVO}) and (\ref{ex:SVO-2}) are examples for SVO, (\ref{ex:VSO}) and (\ref{ex:VSO-1}) show VSO order. An example from elicitation with OVS order is (\ref{ex:OVSfirst}).


(\ref{ex:VOS-1}) comes from María C. talking about Clara. Our visit at her place caused a delay in her work. 

\ea\label{ex:VOS-1}
\begingl
\glpreamble  \textup{VOS:}\\ tanaumÿnÿ pan de arro eka nipiji\\
\gla ti-anau-mÿnÿ {pan de arroz} eka ni-piji\\
\glb 3i-make-\textsc{dim} {rice bread} \textsc{dem}a 1\textsc{sg}-sibling\\
\glft ‘my sister makes rice bread’
\endgl
\trailingcitation{[cux-120410ls.227]}
\xe

%tanaumÿne pan de arro eka nipiji atrasau chitÿpi, cux-120410ls.227


%(\ref{ex:VOS-1111}) was produced by Juana, when she and her brother were telling the story about the fox and jaguar and subsequently about the fox and the jaguarundi. The jaguar is already dead at this point of the story, and this is the start of a new episode, in which the fox meets the jaguarundi, when he looks for chicken.

%\ea\label{ex:VOS-1111}
%\begingl
%\glpreamble  \textup{VOS:}\\tiyunukutu tisemaika takÿra kupisaÿrÿ\\
%\gla ti-yunuku-tu ti-semaika takÿra kupisaÿrÿ\\
%\glb 3i-go.on-\textsc{iam} 3i-search.\textsc{irr} chicken fox\\
%\glft ‘the fox went on in order to look for chicken’\\
%\endgl
%\trailingcitation{[jmx-n120429ls-x5.300]}
%\xe

(\ref{ex:VOS}) was elicited from Clara.

\ea\label{ex:VOS}
\begingl 
\glpreamble  \textup{VOS:}\\chibukutu chikebÿke kusiÿ\\
\gla chi-buku-tu chi-kebÿke kusiÿ\\ 
\glb 3-finish-\textsc{iam} 3-eye ant\\ 
\glft ‘the ants finished (i.e. ate) its eyes (of the dead dried fish that we found)’\\ 
\endgl
\trailingcitation{[cux-c120414ls-2.104]}
\xe


Generally, the preverbal position can be understood to convey emphasis on the participant. This is nicely illustrated by (\ref{ex:VOS-SVO}) below, in which María S. corrected herself when telling the story about the fox and the jaguar. She confused the main characters and first expressed – in VOS order – that the action of eating cheese was performed by the jaguar (\getfullref{ex:VOS-SVO.1}), but then corrected herself and using SVO order emphasised that it was the fox who was eating cheese. (\getfullref{ex:VOS-SVO.2}) is thus a \isi{focus} construction.


\ea\label{ex:VOS-SVO}
  \ea\label{ex:VOS-SVO.1}
 \begingl 
\glpreamble  \textup{VOS:}\\tinikukuikuji kesu isini\\
\gla ti-niku-kuiku-ji kesu isini\\ 
\glb 3i-eat-\textsc{cont}-\textsc{rprt} cheese jaguar\\ 
\glft ‘the jaguar was eating cheese, it is said’\\ 
\endgl
  \ex\label{ex:VOS-SVO.2}
 \begingl
\glpreamble   \textup{SVO:}\\jai kupisairÿ tinikuji kesu\\
\gla  jai kupisairÿ ti-niku-ji kesu\\
\glb \textsc{intj} fox 3i-eat-\textsc{rprt} cheese\\
\glft ‘hay, THE FOX ate cheese, it is said’
\endgl
\trailingcitation{[rxx-n120511l-1.026-027]}
\z
\xe

(\ref{ex:SVO}) is an example of SVO order. Juana and I had been looking at the \isi{frog story} and she had identified the bird that flies from the tree as a hawk. We go on with the next page, but Juana digresses from the task in order to tell me about the hawk and an experience with another bird of prey that stole her dog. The hawk has been mentioned shortly before, but it is not topical. In order to indicate that the topic changes to the hawk again, the NP is positioned before the verb.

\ea\label{ex:SVO}
\begingl 
\glpreamble  \textup{SVO:}\\eka sia tiniku takÿra\\
\gla eka sia ti-niku takÿra\\ 
\glb \textsc{dem}a bird.sp 3i-eat chicken\\ 
\glft ‘the hawk eats chicken’\\ 
\endgl
\trailingcitation{[jxx-a120516l-a.164]}
\xe

In (\ref{ex:SVO-2}) the object is a headless relative clause (see \sectref{sec:HeadlessRC}). It comes from the story about the two men who meet the devil in the woods as told by Miguel. The devil approaches shouting. It is important in the development of the story that only one of the men answers the devil and this importance is highlighted by placing the NP in preverbal position.

\ea\label{ex:SVO-2}
\begingl
\glpreamble  \textup{SVO:}\\i chinachÿ echÿu chikompanyerone chijakupu echÿu tiyÿbui\\
\gla i chinachÿ echÿu chi-kompanyero-ne chi-jakupu echÿu ti-yÿbui\\
\glb and one \textsc{dem}b 3-companion-\textsc{possd} 3-receive \textsc{dem}b 3i-shout\\
\glft ‘and one of the companions answered the one who shouted’
\endgl
\trailingcitation{[mxx-n101017s-1.021]}
\xe

%(\ref{ex:SVO-1}) also expresses a contrast. I had asked María S. whether she went to town to buy things, when she was a child. She denied, and I thought the reason was that her family grew everything they needed on their field. She shortly affirms this, but then encodes the real reason why she did not go to town: it was her siblings who went, she was still too young. This clause is connected to the preceding one, which is about sowing, with an adversative connective to indicate contrast and this contrast is also expressed by pre-verbal subject placement.
%
%\ea\label{ex:SVO-1}
%\begingl
%\glpreamble pero nipijijinubebane tiyununube, tiyÿseikunube xhabumÿnÿ\\
%\gla pero nipijijinubebane tiyununube tiyÿseikunube xhabumÿnÿ\\
%\glb \\
%\glft \\
%\endgl
%\trailingcitation{[rxx-p181101l-2.092]}
%\xe
%-> might be a  bi-clausal rather than mono-clausal sentence!


(\ref{ex:VSO}) is an example of VSO order. It comes from Juana telling the \isi{frog story} and describing the picture on which the dog is running away from the beehive that has fallen down and the bees chase it.

\ea\label{ex:VSO}
\begingl 
\glpreamble  \textup{VSO:}\\chinisapikutu jane eka kabe\\
\gla chi-nisapiku-tu jane eka kabe\\ 
\glb 3-sting-\textsc{iam} wasp \textsc{dem}a dog\\ 
\glft ‘the wasps sting the dog’\\ 
\endgl
\trailingcitation{[jxx-a120516l-a.112]}
\xe

%dxx-d120416s.056: chimumuku echa aitubuche echÿu chipeu eka ... = el joven está mirando el masi
%chipujunÿpaikujitu eka kupisaÿrÿ isini = the fox pushed the tiger, jmx-n120429ls-x5.260

(\ref{ex:VSO-1}) is another example of VSO order, where the subject is a proper name. It remains to be checked whether VSO order is (more) usual if subjects are expressed by proper names, there are very few examples of such constellations in the corpus. In any case, this sentence was produced by María S. in elicitation, she first used a first singular index, when I asked her to translate the sentence, but repeated it with her own name to be put like this in this work. Here you are, María.

\ea\label{ex:VSO-1}
\begingl
\glpreamble  \textup{VSO:}\\tanatu Maria yumaji\\
\gla ti-ana-tu Maria yumaji\\
\glb 3i-make.\textsc{irr}-\textsc{iam} María hammock\\
\glft ‘María is making a hammock’
\endgl
\trailingcitation{[rxx-e181022le]}
\xe

One of the very few examples of OVS orders was produced by Juana, when Swintha asked her to describe a photo that showed me with my two daughters lying on top of me. However, a lot of hesitation marks accompany this sentence, so it is very probable that it might be taken as a mistake or probably as \isi{left dislocation} of a topical object (the picture they had been looking at before also showed my daughters and me).


\ea\label{ex:OVSfirst}
\begingl
\glpreamble \textup{OVS:}\\ruschunubechÿ chakachu chÿenujinube\\
\gla ruschÿ-nube-chÿ chÿ-akachu chÿ-enu-ji-nube\\
\glb two-\textsc{pl}-3 3-lift 3-mother-\textsc{col}-\textsc{pl}\\
\glft ‘the two of them, her mother lifts them’
\endgl
\trailingcitation{[jxx-p141024s-1.21]}
\xe

A different sentence with OVS order was produced by me in elicitation with María S. I asked her whether it was correct and she confirmed it; however, when she repeated the sentence, her intonation indeed suggested that it was a case of \isi{left dislocation}. She would not accept such a sentence if both object and subject were animate,\is{animacy} i.e. when I tried to elicit OVS order with a cat being the subject and a mouse being the object of the verb \textit{-niku} ‘eat’.

\ea\label{ex:OVSsecond}
\begingl
\glpreamble \textup{OVS:}\\amuke, tebuku nÿa\\
\gla amuke ti-ebuku nÿ-a\\
\glb corn 3i-sow 1\textsc{sg}-father\\
\glft ‘corn, my father sowed it’
\endgl
\trailingcitation{[rxx-e181024l]}
\xe


%\ea\label{ex:OVS12345}
%\begingl 
%\glpreamble sesejinube cheikukuikunube kabejane\\
%\gla sesejinube ch-eikukuiku-nube kabe-jane\\ 
%\glb children 3-chase-\textsc{pl} dog-\textsc{distr}\\ 
%\glft ‘the dogs chase the children’\\ 
%\endgl
%\trailingcitation{[mrx-e150219s.056]}
%\xe


I have not found a single example of a \isi{ditransitive} verb being accompanied by three NPs\is{noun phrase} in the corpus, but there is one example with a transitive verb with an incorporated body part.\is{incorporation} The possessor of this body part is expressed by an NP, and subject and object of the verb are also conominated. Presupposed that the possessor is analysed as a raised object here,\is{possessor raising} word order is VOSO, with the theme object\is{patient/theme} of the verb being expressed first and the possessor of the incorporated body part last. The example comes from the story about the fox and the jaguar as told by María S. It occurs close to the end of the story, where the fox ties a stone on the hands of the jaguar and the latter jumps into the water, expecting to find cheese there, but it is only the reflection of the moon that he sees.


\ea\label{ex:VOSO}
\begingl
\glpreamble \textup{VOSO:}\\chirÿtÿnebuÿchuji mai echÿu kupisairÿ echÿu isini\\
\gla chi-rÿtÿ-ne-buÿ-chu-ji mai echÿu kupisairÿ echÿu isini\\
\glb 3-tie-top-hand-\textsc{th}2-\textsc{rprt} stone \textsc{dem}b fox \textsc{dem}b jaguar\\
\glft ‘the fox tied a stone on top of the jaguar’s hands, it is said’
\endgl
\trailingcitation{[rxx-n120511l-1.037]}
\xe
 
Two more examples with two conominated objects were produced by Juana. In (\ref{ex:VOO-1}), she tells that one of her daughters wants to give back money she borrowed from her sister in Spain, when the latter once comes to Bolivia. The whole sentence, which consists of three separate clauses, is given here. The topical subject of the first clause is the daughter who lives in Spain, but in the second clause, topic switches to the other daughter. The new topic is not expressed by an NP. It is thus not totally clear whether the NP \textit{nijinepÿi} ‘my daughter’ in the third clause refers to the subject or the object, since both participants are Juana’s daughters. I would opt for an analysis as an object, because the topical subject of this clause is the same as the one in the preceding clause, there is thus topic continuity and topical participants are usually not expressed by NPs. The translation of the example follows this analysis. The word order of the last clause is thus VOO. However, it is also possible that \textit{nijinepÿi} is a subject NP, a delayed indication of topic switch in the previous clause. The translation of the last clause would be ‘my daughter will give her the money’ in that case and word order VSO.

\ea\label{ex:VOO-1}
\begingl
\glpreamble \textup{(non-verbal PRED, VO), VOO:}\\i despue kue kapupunuina te chebÿpekupuna echÿu tÿmue, chipua nijinepÿi \\chitÿmuane\\
\gla i despue kue kapupunu-ina te chÿ-ebÿpeku-puna echÿu tÿmue chi-pua ni-jinepÿi chi-tÿmua-ne\\
\glb and afterwards if come.back-\textsc{irr.nv} \textsc{seq} 3-borrow.money-\textsc{am.prior.irr} \textsc{dem}b money 3-give.\textsc{irr} 1\textsc{pl}-daughter 3-money-\textsc{possd}\\
\glft ‘and later, when she comes back, then she goes to borrow money, and she will give my daughter her money’
\endgl
\trailingcitation{[jxx-p120430l-1.294]}
\xe

The other example consists of two juxtaposed clauses. Interestingly, the recipient object of the verb of the second clause appears to the left of the first (intransitive) verb, i.e. quite dislocated from the verb it belongs to. This can be considered a case of long-distance dependency in Paunaka. (\ref{ex:OVVO}) describes the same situation as (\ref{ex:nebu-obl-1}) above, but comes from another recording with Juana made in another year, when she told me the story again. It is about her daughter being arrested in the airport in Spain for not having a valid visa. She was brought to a room upstairs in the airport building and received some food. The structure of this sentence is OVVO, with the first O pertaining to the second V.

\ea\label{ex:OVVO}
\begingl
\glpreamble \textup{O(V)VO:}\\i eka nijinepÿimÿnÿ tipununubetu chipunakunube yÿtÿuku\\
\gla i eka ni-jinepÿi-mÿnÿ ti-punu-nube-tu chi-punaku-nube yÿtÿuku\\
\glb and \textsc{dem}a 1\textsc{sg}-daughter-\textsc{dim} 3i-go.up-\textsc{pl}-\textsc{iam} 3-give-\textsc{pl} food\\
\glft ‘and as for my daughter, they went up and gave her some food’
\endgl
\trailingcitation{[jxx-p120430l-1.213]}
\xe


As for obliques,\is{oblique} they usually occur to the right of the verb. VX orders, with X standing for oblique as in \citet[]{wals-84}, are most common. All other participants are usually well-established by the preceding clauses and thus do not have to be conominated by an NP. XV is also found, but considerably less common, and largely restricted to temporal and source expressions. I will only consider different kinds of locative obliques and a few recipients here, because there are few examples for the other kinds of obliques. 

(\ref{ex:VX-1}) is one example with VX order. The oblique is a PP with the preposition \textit{tukiu}, a source expression. The sentence was produced by Juana, when she told me about how her daughter was deported from Spain and arrived back to Bolivia.

\ea\label{ex:VX-1}
\begingl
\glpreamble \textup{VX:}\\tikubupaikunubetu tukiu labion\\
\gla ti-kubupaiku-nube-tu tukiu labion\\
\glb 3i-go.down-\textsc{pl}-\textsc{iam} from plane\\
\glft ‘they disembarked from the plane’
\endgl
\trailingcitation{[jxx-p120430l-1.266]}
\xe

An unmarked oblique with the semantic role of goal is found in (\ref{ex:VX-2}), which was a statement by María S., when I told her that we had been looking for her before.

\ea\label{ex:VX-2}
\begingl
\glpreamble \textup{VX:}\\niyunutu asaneti\\
\gla ni-yunu-tu asaneti\\
\glb 1\textsc{sg}-go-\textsc{iam} field\\
\glft ‘I had gone to the field’
\endgl
\trailingcitation{[mrx-c120509l.052]}
\xe

It is very common that a \isi{locative} adverb is placed before the PP. This is the case in (\ref{ex:VX-3}): first comes the verb, then the adverb \textit{nauku} ‘there’ and finally the locative-marked noun. Juana cites her own words here, which were directed to her daughter. 

\ea\label{ex:VX-3}
\begingl
\glpreamble \textup{VX:}\\niyuna nauku parkeyae\\
\gla ni-yuna nauku parke-yae\\
\glb 1\textsc{sg}-go.\textsc{irr} there park-\textsc{loc}\\
\glft ‘I will go to the park there’
\endgl
\trailingcitation{[jxx-p120430l-2.242]}
\xe

If there is an object in the clause, we predominantly find VOX order and only occasionally XVO. VOX is the most common order cross-linguistically for languages in which the object usually follows the verb \citep[]{wals-84}. It is uncommon that there is a conominal S argument in a sentence that contains an oblique.

(\ref{ex:VOX-1}) comes from Miguel telling José the \isi{frog story}. This is his description of the picture on which the deer throws the boy down the slope into the water.

\ea\label{ex:VOX-1}
\begingl
\glpreamble \textup{VOX:}\\chibikÿkÿnÿkutu echÿu aitubuchepÿimÿnÿ kÿpenukÿyae\\
\gla chi-bikÿkÿnÿku-tu echÿu aitubuchepÿi-mÿnÿ kÿpenukÿ-yae\\
\glb 3-throw.away-\textsc{iam} \textsc{dem}b boy-\textsc{dim} depth-\textsc{loc}\\
\glft ‘it throws the boy into the depth’
\endgl
\trailingcitation{[mox-a110920l-2.153]}
\xe

(\ref{ex:VOX-2}) has a first person plural benefactive oblique. It stems from Juana’s account about how they made the reservoir in Santa Rita. A lady came to Santa Rita and promised them to make the reservoir in exchange for clearing of a big piece of land for her for agricultural use. She is the one who brought them meat.


\ea\label{ex:VOX-2}
\begingl
\glpreamble \textup{VOX:}\\tupunu chÿeche bitÿpi\\
\gla ti-upunu chÿeche bi-tÿpi\\
\glb 3i-bring meat 1\textsc{pl}-\textsc{obl}\\
\glft ‘she brought us meat’
\endgl
\trailingcitation{[jxx-p120515l-2.098]}
\xe


%Two examples for the orders VXO and XVO follow. Both were produced by Juana. (\ref{ex:VXO}) comes from Juana’s first account about her grandparents journey to Moxos.
%
%\ea\label{ex:VXO}
%\begingl
%\glpreamble \textup{(V), VXO:}\\tiyunu tiyeseikunube Monkoxiyae baka\\
%\gla ti-yunu ti-yeseiku-nube Monkoxi-yae baka\\
%\glb 3i-go 3i-buy-\textsc{pl} Moxos-\textsc{loc} cow\\
%\glft ‘they went and bought cows in Moxos’\\
%\endgl
%\trailingcitation{[jxx-e150925l-1.254]}
%\xe
%

%(\ref{ex:XVO-1}) is an example with a temporal expression preceding the verb. It is completely unmarked, i.e. neither the locative marker nor a preposition marks its oblique status. The example comes from María S. who talked about her childhood. Food, and especially meat, was scarce at that time, but her mother cooked rice and corn stews regularly. \textit{Pujukeke} (or \textit{mote} in Spanish) is a kind of stew with corn. When meat is added, the dish is called \textit{patasca} in Spanish.
%
%XVO, XOV:
%\ea\label{ex:XVO-1}
%\begingl
%\glpreamble chÿnachÿ tijai tiyÿtikapumÿnÿ arusuji pujaine pujukekepupunukutu tiniku\\
%\gla chÿnachÿ tijai ti-yÿtikapu-mÿnÿ arusu-ji pu-jaine pujukeke-pupunuku-tu ti-niku\\
%\glb one day 3i-cook.\textsc{irr}-\textsc{dim} rice-\textsc{clf:}soft.mass other-day mote-\textsc{reg}-\textsc{iam} 3i-eat\\
%\glft ‘one day she would cook a rice stew, the other day she ate \textit{mote} again’\\
%\endgl
%\trailingcitation{[rxx-p181101l-2.250]}
%\xe

(\ref{ex:XVO-2}) is an example of XVO order. The oblique is a source expression with the preposition \textit{tukiu} ‘from’, the adverb \textit{naka} ‘here’, and a toponym, \textit{Santa Cruz}, which does not take the locative marker. The sentence was produced by Juana when she still lived in Santa Cruz. She told me that when she had lived in Concepción before, her daughters bought meat for her and sent it to her by bus in a styrofoam box. She could then cook and sell \textit{patasca} in Concepción.

\ea\label{ex:XVO-2}
\begingl
\glpreamble \textup{XVO:}\\tukiu naka Santa Cruz tiyÿseikunube chichÿti ÿba\\
\gla tukiu naka {Santa Cruz} ti-yÿseiku-nube chi-chÿti ÿba\\
\glb from here {Santa Cruz} 3i-buy-\textsc{pl} 3-head pig\\
\glft ‘from Santa Cruz here, they bought a pig’s head’
\endgl
\trailingcitation{[jxx-e110923l-2.156]}
\xe


%punachÿyae estansia tumeikuji = on another estate he stole, it is said, jxx-p120430l-2.066


There are even fewer verbal clauses in the corpus that contain an oblique and an NP that conominates the subject. I have found the orders SVX (\ref{ex:SVX}), VSX (\ref{ex:VSX}) and VXS (\ref{ex:VXS}), but cannot say which one is most neutral for lack of sufficient data. It is questionable whether one can speak of a neutral order at all for a type of sentence that is very uncommon.

(\ref{ex:SVX}) is from the account by María S. about how she grew up. It consists of two juxtaposed sentences, both with an unmarked oblique constituent following the respective verb, while the subject NPs precede the verbs.


\ea\label{ex:SVX}
\begingl
\glpreamble \textup{SVX, SVX:}\\depue Krara tiyunutu uneku Kuana tikubiupu uneku\\
\gla depue Krara ti-yunu-tu uneku Kuana ti-kubiu-pu uneku\\
\glb afterwards Clara 3i-go-\textsc{iam} town Juana 3i-have.house-\textsc{dloc} town\\
\glft ‘later Clara went to town, Juana got a house in town’
\endgl
\trailingcitation{[rxx-p181101l-2.263]}
\xe

%eti epajÿkutu nauku Naranjito, mqx-p110826l.434


When giving me a description of how to use palm fruit oil for hair care, Juana produced (\ref{ex:VSX}). The oblique NP is accompanied by the adverb \textit{naka} ‘here’ in this case.

\ea\label{ex:VSX}
\begingl
\glpreamble \textup{VSX:}\\tipajÿkutu echÿu aseite naka bichÿtiyae\\
\gla ti-pajÿku-tu echÿu aseite naka bi-chÿti-yae\\
\glb 3i-stay-\textsc{iam} \textsc{dem}b oil here 1\textsc{pl}-head-\textsc{loc}\\
\glft ‘the oil (of the palm fruit) stays here on our head’
\endgl
\trailingcitation{[jxx-d181102l.30]}
\xe


(\ref{ex:VXS}) is from Juana’s second account about her grandparents’ journey to Moxos and back home with the cows they bought there. It is a long journey and the grandparents had to rest on the way. They usually stayed in huts along the way and sometimes the huts also had an enclosure, where they kept their cows.

\ea\label{ex:VXS}
\begingl
\glpreamble \textup{VXS:}\\tibÿkupujaneji bakayayae baka\\
\gla ti-bÿkupu-jane-ji bakaya-yae baka\\
\glb 3i-enter-\textsc{distr}-\textsc{rprt} enclosure-\textsc{loc} cow\\
\glft ‘the cows went into the enclosure, it is said’
\endgl
\trailingcitation{[jxx-p151016l-2.166]}
\xe

%tipajÿku nauku España chimaretane, jxx-p120430l-1.272

Finally, it is also possible to have two conominated arguments plus an oblique. This is the case in (\ref{ex:obl-tÿpi-5}) and (\ref{ex:SVOX}). The constituent orders are SVXO and SVOX respectively and the oblique constituent is a benefactive in both cases. As for the question which of these orders is more common, I would opt for the second one, because the oblique usually follows the object NP in VOX clauses. However, I would not be able to prove this with data from the corpus, since there are simply not enough sentences in which we have conominal expressions of subject and object as well as an oblique.

(\ref{ex:obl-tÿpi-5}) comes from Miguel’s account about how he went to school. Since they had no paper, the pupils wrote on wooden boards. Miguel’s board was made by his father.


\ea\label{ex:obl-tÿpi-5}
\begingl
\glpreamble \textup{SVXO:}\\entonses kuineini taitaini tanau nitÿpi echÿu taurapamÿnÿ\\
\gla entonses kuineini taita-ini ti-anau ni-tÿpi echÿu taurapa-mÿnÿ\\
\glb thus deceased dad-\textsc{dec} 3i-make 1\textsc{sg}-\textsc{obl} \textsc{dem}b board-\textsc{dim}\\
\glft ‘so my late father made a small wooden board for me’
\endgl
\trailingcitation{[mxx-p181027l-1.023]}
\xe


A sentence with the order SVOX was elicited from Juana.


\ea\label{ex:SVOX}
\begingl
\glpreamble \textup{SVOX:}\\ eka nijinepÿi tiyÿseiku eka epuke tÿpi eka chipiji\\
\gla eka nij-inepÿi ti-yÿseiku eka epuke tÿpi eka chi-piji\\
\glb \textsc{dem}a 1\textsc{sg}-daughter 3i-buy \textsc{dem}a ground \textsc{obl} \textsc{dem}a 3-sibling\\
\glft ‘my daughter bought ground for her sister’
\endgl
\trailingcitation{[jxx-e191021e-2]}
\xe


%Another example shows the order SVXO; however, the S is detached from the rest of the clause by insertion of an interjection between subject and verb. This can thus be considered a case of left dislocation.
%
%\ea\label{ex:}
%\begingl
%\glpreamble i echÿu kayaraunu juu chumutu chÿestancianeye chipeunube baka\\
%\gla i echÿu kayaraunu juu chumutu chÿestancianeye chipeunube baka\\
%\glb \\
%\glft ‘and the \textit{karay}, huu, he took their cows to his estate’\\
%\endgl
%\trailingcitation{[jxx-e150925l-1.235]}
%\xe
%



% adverbial clauses: temporal, only juxtaposition: jmx-d110918ls-1.010

%chakachu chÿenu tiyunu chubu hospitalyae, jxx-p110923l-1.460


In summary, word order in Paunaka is quite flexible, but it is most common that the \isi{verb} comes first, and it also quite common that the object follows the verb directly. There is one pre-verbal slot, which may be filled with S, O or X to indicate special discourse function.

\is{conomination|)}
\is{word order|)}

The following section focuses on standard negation.




\subsection{Standard negation}\label{sec:Negation}
\is{negation|(}

This section is about “the basic way(s) a language has for negating declarative verbal main clauses” \citep[1]{Miestamo2005}. This has been called “standard negation”. Other types of negation are found in the individual sections about different non-verbal clauses (see \sectref{sec:NonVerbalPredication}) and in the section about negative imperatives (see \sectref{sec:Prohibitives}).\footnote{\label{fn:privative} As for morphological negation including a reflex of the famous Proto-Arawakan\is{Arawakan languages} \isi{privative} prefix \textit{*ma-}, this is not productive in Paunaka. The only two words I can think of containing a reflex of the privative prefix are \textit{mupÿinube} (\textit{mu-pÿi-nube} \textsc{priv}-body-\textsc{pl}) ‘devil’ (lit.: ‘the ones without body’) and \textit{mÿbanejiku} (\textit{mu-ÿ-bane-jiku} \textsc{priv}-be.long-\textsc{rem}-\textsc{lim}1) ‘close to, near’. In addition, there are some more words starting with \textit{mu} like \textit{mukÿe} ‘squash sp.’. They may or may not have once been derived\is{derivation} from other words with the privative prefix, but in any case they are not decomposable synchronically.}

Standard negation builds on the \isi{negative particle} \textit{kuina}, which is placed directly before the verb.\is{word order} This particle seems to include the \isi{non-verbal irrealis marker} \textit{-ina} attached to a stem or affix \textit{ku}. Note that a voiceless velar plosive is relatively common in standard negation of \isi{Arawakan languages} \citep[288]{Michael2014b}, and the \isi{Mojeño languages} have a prefix \textit{ku-} for irrealis negation (cf. Rose 2020, p.c.), i.e. the “doubly irrealis construction”.

In Paunaka’s standard negation, the verb necessarily has \isi{irrealis} RS given that a non-realised event is always non-factual. Standard negation thus shows a paradigmatic asymmetry as regards RS\is{reality status} \citep[96]{Miestamo2005}. There is no morphological “\isi{doubly irrealis construction}” in declarative sentences. This is defined as a construction that explicitly marks that there are (at least) two parameters that trigger irrealis RS, one of them being negation \citep[cf.][253]{Michael2014}. This may be worth mentioning explicitly, because closely related Trinitario,\is{Mojeño Trinitario} \isi{Terena} and Kinikinau as well as the more distantly related Kampan languages\is{Arawakan languages} all have more or less elaborate doubly irrealis contructions \citep[267--269]{Michael2014b}.%261-263

Consider (\ref{ex:neg-2}). The positive sentences in (\ref{ex:neg-2.1}) and (\ref{ex:neg-2.2}) differ from each other in RS,\is{reality status} with (\getfullref{ex:neg-2.1}) encoding a factual event by default/realis and (\getfullref{ex:neg-2.2}) a non-factual event by irrealis. When negated as in (\getfullref{ex:neg-2.3}), this distinction is neutralised.


\ea\label{ex:neg-2}
  \ea\label{ex:neg-2.1}
\begingl
\glpreamble niniku\\
\gla ni-niku\\
\glb 1\textsc{sg}-eat\\
\glft ‘I eat/ate it’
\endgl
  \ex\label{ex:neg-2.2}
\begingl
\glpreamble ninika\\
\gla ni-nika\\
\glb 1\textsc{sg}-eat.\textsc{irr}\\
\glft ‘I will/can/must eat it’
\endgl
  \ex\label{ex:neg-2.3}
\begingl
\glpreamble kuina ninika\\
\gla kuina ninika\\
\glb \textsc{neg} 1\textsc{sg}-eat.\textsc{irr}\\
\glft ‘I don’t/didn’t/can’t/couldn’t/won’t eat it’
\endgl
\z
\xe

Some more examples of negative declarative clauses follow, containing a stative intransitive verb  in (\ref{ex:neg-stat}), an active intransitive verb in (\ref{ex:neg-act}), a transitive verb in (\ref{ex:neg-trans}), and a ditransitive verb in (\ref{ex:neg-dit}).

(\ref{ex:neg-stat}) comes from María S. talking with me about snow in Germany.

\ea\label{ex:neg-stat}
\begingl
\glpreamble kuina tasÿeiyu\\
\gla kuina ti-a-sÿei-yu\\
\glb \textsc{neg} 3i-\textsc{irr}-be.cold-\textsc{ints}\\
\glft ‘it is not very cold’
\endgl
\trailingcitation{[rxx-e120511l.312]}
\xe

%(\ref{ex:neg-stat-2}) is a statement by Juana about her son-in-law, who is ill.
%
%\ea\label{ex:neg-stat-2}
%\begingl
%\glpreamble kuina tajimama\\
%\gla kuina ti-a-jimama\\
%\glb \textsc{neg} 3i-\textsc{irr}-be.strong\\
%\glft ‘he is not strong’\\
%\endgl
%\trailingcitation{[jxx-p110923l-1.053]}
%\xe

In (\ref{ex:neg-act}), Juana states that her daughter did not go (to the airport). In her opinion, her daughter should have picked up her sister there. The latter was supposed to work as a nanny in Spain, but was finally deported without having ever entered the country.

\ea\label{ex:neg-act}
\begingl
\glpreamble kuina tiyuna\\
\gla kuina ti-yuna\\
\glb \textsc{neg} 3i-go.\textsc{irr}\\
\glft ‘she didn’t go’
\endgl
\trailingcitation{[jxx-p110923l-1.312]}
\xe

(\ref{ex:neg-trans}) is part of the answer José gave, when Miguel asked him whether he knew the story about the lazy man.

\ea\label{ex:neg-trans}
\begingl
\glpreamble kuina nichupa micha\\
\gla kuina ni-chupa micha\\
\glb \textsc{neg} 1\textsc{sg}-know.\textsc{irr} good\\
\glft ‘I don’t know it well’
\endgl
\trailingcitation{[mox-n110920l.007]}
\xe

In (\ref{ex:neg-dit}), Juan C. speaks about the old times, when his \textit{patrón} refused to give him a pair of trousers that was supposed to be part of his compensation for working.

\ea\label{ex:neg-dit}
\begingl 
\glpreamble kuina tipunakane nikasuneina\\
\gla kuina ti-punaka-ne ni-kasune-ina\\ 
\glb \textsc{neg} 3i-give.\textsc{irr}-1\textsc{sg} 1\textsc{sg}-trousers-\textsc{irr.nv}\\ 
\glft ‘he didn’t give me my supposed trousers’\\ 
\endgl
\trailingcitation{[mqx-p110826l.458]}
\xe

Word order\is{word order} is largely the same as in positive sentences; however, conominals\is{conomination|(} are in general rarer. If present, they usually follow the negated verb. (\ref{ex:neg-VS}) is an example in which a conominated subject follows and (\ref{ex:neg-VO}) has a conominated object. In order to indicate special discourse status, a conominal argument can also precede the negator. This is the case in (\ref{ex:neg-SV}), where the subject precedes \textit{kuina} for contrastive \isi{focus}.

(\ref{ex:neg-VS}) comes from Miguel speaking about the old days.

\ea\label{ex:neg-VS}
\begingl
\glpreamble kuina chisiupuchanube eka patron\\
\gla  kuina chi-siupucha-nube eka patron\\
\glb \textsc{neg} 3-pay.\textsc{irr}-\textsc{pl} \textsc{dem}a patrón\\
\glft ‘the \textit{patrón} didn’t pay them’
\endgl
\trailingcitation{[mxx-p110825l.042]}
\xe

In (\ref{ex:neg-VO}), Juana tells her sister the reason why her ducklings died, when she was away for one week. 

\ea\label{ex:neg-VO}
\begingl
\glpreamble kuina chetukanube eka yÿtÿukumÿnÿ \\
\gla kuina chÿ-etuka-nube eka yÿtÿuku-mÿnÿ \\
\glb \textsc{neg} 3-put.\textsc{irr}-\textsc{pl} \textsc{dem}a food-\textsc{dim}\\
\glft ‘they didn’t give them food, poor ones’
\endgl
\trailingcitation{[jrx-c151001lsf-11.063]}
\xe
\is{conomination|)}

The context of (\ref{ex:neg-SV}) is that María S. complains that her chicken get stolen when she is away from her house.

\ea\label{ex:neg-SV}
\begingl
\glpreamble nÿti kuina nÿnika pero punachÿ tiniku\\
\gla nÿti kuina nÿ-nika pero punachÿ ti-niku\\
\glb 1\textsc{sg.prn} \textsc{neg} 1\textsc{sg}-eat.\textsc{irr} but other 3i-eat\\
\glft ‘I don’t eat them, but another one eats them’
\endgl
\trailingcitation{[rxx-e120511l.181]}
\xe

Some markers can be attached to the \isi{negative particle}, among them the \isi{additive} and those expressing TAME categories.\is{tense}\is{aspect}\is{modality}\is{evidentiality} However, all of these markers can also attach to the \isi{verb} with no difference in meaning.\footnote{Although when comparing the two examples in (\ref{ex:neg-add}) and (\ref{ex:neg-add-2}), it may look like the additive marker was a third-position clitic, this is not the case. There are dozens of examples in the corpus, where the additive occurs in other positions; see §\ref{sec:AffixesAndClitics} for a general discussion on this issue.}

Consider the following example, which has an additive marker. Prior to uttering this sentence, Juana had just told me that she did not speak Spanish, when she was a child, only Paunaka. She adds to this statement by (\ref{ex:neg-add}), telling me that she did not have any contact to \isi{Bésiro} either. 


\ea\label{ex:neg-add}
\begingl
\glpreamble i echÿu tiseteiku kuinauku nisama\\
\gla i echÿu tiseteiku kuina-uku ni-sama\\
\glb and \textsc{dem}b Bésiro \textsc{neg}-\textsc{add} 1\textsc{sg}-hear.\textsc{irr}\\
\glft ‘and Bésiro I didn’t hear either’
\endgl
\trailingcitation{[jxx-p120430l-1.028-030]}
\xe

The way of listing that preceded (\ref{ex:neg-add-2}) was very similar, but in this case the additive marker is attached to the verb: María C. stated that she did not know her grandparents, but only knew her mother and then added that she did not know her father either (because she was still very young when he passed away).

\ea\label{ex:neg-add-2}
\begingl
\glpreamble nÿa, kuina nichupuikuka nÿa\\
\gla  nÿ-a kuina ni-chupuiku-uka nÿ-a\\
\glb 1\textsc{sg}-father \textsc{neg} 1\textsc{sg}-know-\textsc{add.irr} 1\textsc{sg}-father\\
\glft ‘as for my father, I didn’t know my father either’
\endgl
\trailingcitation{[ump-p110815sf.148]}
\xe


As for TAME marking in negative sentences, there is one peculiarity: the \isi{prospective} does not occur in negative clauses, nor does the otherwise omnipresent iamitive \is{iamitive|(}.\footnote{The only exception is that the iamitive can be attached to the discontinuous\is{discontinuous|(} marker on the \isi{negative particle} itself, yielding \textit{kuinabutu} ‘not anymore now’.} Instead of this, the \isi{}discontinuous marker \textit{-bu} adopts one of the functions of the \isi{iamitive} indicating that a previously ongoing event is already finished. The discontinuous marker is only found in negative clauses and can be translated as ‘(not) anymore’. The verb is usually interpreted as encoding a state, even if this is not inherent in its semantics, see \sectref{sec:Discontinuous}. As for the other function of the iamitive, expressing that something is ongoing (telic verbs),\is{telicity} there is no way to form a corresponding negative sentence. Besides neutralisation of RS, this is the second asymmetry found between negative and positive declarative sentences.\is{iamitive|)}

One example of the discontinuous marker in a negative sentence is given in (\ref{ex:neg-dsc}).\footnote{The discontinuous marker actually occurs twice here, on the negative particle and on the verb, but this is not obligatory. It can also occur once, either on \textit{kuina} or on the verb, see \sectref{sec:Discontinuous}.} It was produced by Juana when she told me about how her brother passed away, thus ‘not speak anymore’ is stative in the sense that it does not refer to a momentary disruption of speaking, but to an irreversible state of weakness before his death. It is another brother of hers whom she cites here.

\ea\label{ex:neg-dsc}
\begingl
\glpreamble “nÿbÿsÿu kuinabu tichujikabu”, tikechu\\
\gla nÿ-bÿsÿu kuina-bu ti-chujika-bu ti-kechu\\
\glb 1\textsc{sg}-come \textsc{neg}-\textsc{dsc} 3i-speak.\textsc{irr}-\textsc{dsc} 3i-say\\
\glft ‘“when I came, he didn’t speak anymore”, he said’
\endgl
\trailingcitation{[jxx-p120430l-2.456]}
\xe
\is{discontinuous|)}
\is{negation|)}

The discussion of negation in verbal declarative clauses is completed at this point. The next section is dedicated to non-verbal predication, including both positive and negative non-verbal clauses. \is{declarative clause|)}



%kuina bichupuika eka nÿkÿiki, jxx-d110923l-2.06 no conocimos olla
%kuina chetukanube eka yÿtiÿukumÿnÿ eka yÿtÿuku chitÿpijane, jrx-c151001lsf-11.063 0 her ducks died, because they were not fed, when she was away i SCz

%SVO: nÿenu kuina tinikane yÿtÿuku, mxx-e160811sd.052 el.
%OV: echÿu chichupu echÿu chitareane kuina cheistaka, mxx-p181027l-1.076
%OV: ee chÿnajiku echÿu kuina kuina ni- nÿ- nÿ- nisumacha nechÿu matematica, mxx-p181027l-1.090

%“When no structural differences are found between the affirmative and the negative in addition to the negative marker(s) the structures are symmetric. When there are structural differences, i.e. asymmetry, between the affirmative and the negative in addition to the negative marker(s), the structures are asymmetric. ” \citep[51]{Miestamo2005}
%
%Negation:
%
%\ea\label{ex:no-ride}
%\begingl
%\glpreamble kuina tubuneikanube\\
%\gla kuina ti-ubuneika-nube\\
%\glb \textsc{neg} 3i-ride.\textsc{irr}-\textsc{pl}\\
%\glft ‘they didn’t ride on horse’\\
%\endgl
%\trailingcitation{[jxx-p151016l-2.040]}
%\xe








