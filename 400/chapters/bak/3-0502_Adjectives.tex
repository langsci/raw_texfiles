%!TEX root = 3-P_Masterdokument.tex
%!TEX encoding = UTF-8 Unicode

\section{Adjectives, numerals and quantifiers}\label{sec:AdjectivesNumerals}\is{adjective|(}

In this section, adjectives, numerals and quantifiers are described. The latter two are often subsumed under the former, but they differ from adjectives in some ways. 

There are only very few adjectives, since most property concepts are expressed by stative verbs\is{stative verb} (\sectref{sec:StativeVerbs}). Thus adjectives constitute a minor category in Paunaka. They express value, dimension, colour and shape. They are seldom used attributively in Paunaka, most of the time they occur as predicates. They are easily distinguished from verbs, since subject indexes do not precede the stem.\is{person marking|(} Indexes following the stem do not show up very frequently either, but this may be directly connected to adjectives having primarily third person referents. Occurrence of third person markers that follow the stem is very restricted in general (see \sectref{sec:3Marking}). The only adjective taking person markers from time to time is \textit{micha} ‘good’ and the adjectives derived from it. The person marker follows the adjective in this case, which is just what we expect in \isi{non-verbal predication}, see \sectref{sec:NonVerbalPredication}.\is{person marking|)} 

\textit{Micha} and its derivations are also easily distinguishable from nouns:\is{noun|(} when negated,\is{negation} a verbal form\is{verb} of the word occurs, which makes \textit{micha} suspicious of having been derived from a \isi{stative verb} originally.\footnote{Interestingly, \citet[313]{Michael2008} reports that in Nanti adjectives derived from verbs tend to occur in positive attributional clauses, while the corresponding verbs tend to occur in the negative counterparts.} 

As for other adjectives, a possibility to distinguish them from nouns is their behaviour when combining with classifiers\is{classifier|(} (see \sectref{sec:Classifiers}). If an adjective combines with a classifier, this classifier expresses a property (mostly shape) of the referent and thus it provides additional information about the referent. Nouns can also combine with classifiers, but the process resembles compounding and the product of the process is a new noun denoting a new referent. Adjectives are thus much more flexible in combining with a classifier than nouns, and they share this flexibility with the subgroup of descriptive stative verbs\is{stative verb} that can also take classifiers (\sectref{sec:StativeVerbs_CLF}). There are words, however, which are ambiguous as to whether they are nouns\is{noun|)} or should rather be analysed as adjectives.\is{classifier|)}

Numerals\is{numeral|(} are often defined as a subclass of adjectives, but there are features that distinguish them from adjectives in general. This is probably due to different functions of adjectives and numerals: “Whereas an adjective indicates a property of a noun, a numeral is not a property of the object itself but of a set of objects, often a nonce-property” \citep[770]{Greenberg2000}. In Paunaka, the most important difference is that numerals often occur attributively, while adjectives can in general occur attributively, but do not do so very frequently. Nonetheless, numerals can also be used as predicates. The only numeral of presumably Paunaka origin is \textit{chÿnachÿ} ‘one’, all numbers higher than ‘one’ have been borrowed\is{borrowing} from Spanish with different degrees of integration into Paunaka. The word for ‘other’, \textit{punachÿ} is very similar to \textit{chÿnachÿ} in the way it is composed. It is thus also treated in the section about numerals.\is{numeral|)} Both words probably contain the general \isi{classifier} \textit{-na}, which is also found on the adjectives \textit{(mu)temena} ‘big’ and \textit{kana} ‘this size’.

Quantifiers\is{quantifier} are also described in this chapter, although they were priorly analysed as adverbs because they do not frequently modify nouns, but are more often used predicatively or adverbially. However, adjectives do not often occur as nominal modifiers\is{modification} either. Semantically, quantifiers come close to numerals,\is{numeral} since both provide information about a quantity. The most frequently used quantifiers are loans\is{borrowing} from Bésiro.

In the NP,\is{noun phrase} both quantifiers\is{quantifier} and numerals\is{numeral} can only precede the noun,\is{word order} while adjectives can also follow it, although the latter case could possibly be described as a kind of \isi{modification} by a relative clause (see \sectref{sec:NP}).

\sectref{sec:Adjectives} describes adjectives in more detail, \sectref{sec:Numerals} is about numerals and the word for ‘other’ and \sectref{sec:QuantifyingAdverbs} discusses the quantifiers found in Paunaka.

\subsection{Adjectives}\label{sec:Adjectives}

There are only a few adjectives in Paunaka, thus \sectref{sec:ADJInventory} is dedicated to describing the ones that occur, while \sectref{sec:UsesADJ} examines the different usages of adjectives as predicates, attributes, adverbs, and secondary predicates.


\subsubsection{Inventory}\label{sec:ADJInventory}

The most important adjectives are \textit{micha} ‘good’, some derivations of \textit{micha}, \textit{(mu)temena} ‘big’ and the \isi{demonstrative adjective} \textit{kana} ‘this size’.

The most frequent among them is \textit{micha} ‘good’. One example with it is (\getref{ex:micha-1}), which is a statement by Miguel after having looked at a puzzle game that shows a story about a boy and his squirrel.\footnote{Note that the subordinating suffix \textit{-i} creates morphologically semi-nominal words, there is no subordination involved in this example, see also \sectref{sec:AdverbialModification}.}

\ea\label{ex:micha-1}
\begingl
\glpreamble michayu chÿnÿnÿikiu eka aitubuchepÿimÿnÿ\\
\gla micha-yu chÿ-nÿnÿik-i-u eka aitubuchepÿi-mÿnÿ\\
\glb good-\textsc{ints} 3-live-\textsc{subord}-\textsc{real} \textsc{dem}a boy-\textsc{dim}\\
\glft ‘the life of the little boy is very good’
\endgl
\trailingcitation{[mdx-c120416ls.191]}
\xe

\textit{Micha} usually does not take a person marker,\is{person marking|(} with one exception: when meeting another person, people use a greeting formula including \textit{micha} and a second person index. Literally, this is a question\is{polar question} about the condition of the other. The formulaic answer is without a person marker; see (\getref{ex:ADJ-micha-1}), which is a little lecture of this convention (in Spanish) that Juana gave to Swintha.

\ea\label{ex:ADJ-micha-1}
\begingl
\glpreamble yo primero “¿michabi?” y usted me contesta “micha”\\
\gla {yo primero} micha-bi {y usted me contesta} micha\\
\glb {I first} good-2\textsc{sg} {and you me answer} good\\
\glft ‘me first “¿michabi?” (= are you doing fine?) and you answer “micha” (= fine)’
\endgl
\trailingcitation{[jxx-n101013s-1.080-083]}
\xe
\is{person marking|)}

A number of other adjectives have been derived\is{derivation|(} from \textit{micha}; these are \textit{michana} ‘nice’, which probably includes the general \isi{classifier}, a further derivation \textit{michanabÿke} ‘beautiful, pretty, handsome’ used in reference to people, which additionally takes the noun \textit{-bÿke} ‘face’,\is{incorporation} \textit{michaniki} ‘delicious’, which probably includes the verb stem\is{verbal stem} \textit{-nik(u)} ‘eat’, and \textit{michamue} ‘of sunny weather, sky without clouds’, which presumably includes the same sequence \textit{-mu} that is also found in \textit{anÿmu} ‘sky’ (as opposed to \textit{anÿke} ‘up, above’). The proposed composition of these derived adjectives is found in (\getref{ex:michana-1}) to (\getref{ex:michamue-1}), interlinear glosses of these words are usually not given in this detail in the remainder of this work.

In (\getref{ex:michana-1}), Juana talks about the house of an acquaintance in Austria.

\ea\label{ex:michana-1}
\begingl
\glpreamble michana ubiae puru teka\\
\gla micha-na ubiae puru teka\\
\glb good-\textsc{clf:}general house mere brick\\
\glft ‘the house is nice, (it has) mere bricks’
\endgl
\trailingcitation{[jxx-p110923l-2.146]}
\xe

In (\getref{ex:pretty-1}), María S. corrects my use of \textit{michana} in reference to a baby.

\ea\label{ex:pretty-1}
\begingl
\glpreamble michanabÿke\\
\gla micha-na-bÿke\\
\glb good-\textsc{clf:}general-face\\
\glft ‘she is pretty’
\endgl
\trailingcitation{[rxx-e120511l.327]}
\xe

(\getref{ex:ADJ-Pred-2}) is a statement by María S. about the fish Juana is talking about.\footnote{As for the last \textit{-i} of the word, this might be an obsolete \isi{passive} suffix. There are two hints that point at that. First of all, \isi{Mojeño Trinitario} possibly has a passive suffix in the same position on the verb ‘be delicious’ (Rose 2021, p.c.), and second, I have found one verb form in the corpus that seems to include a suffix \textit{-i} to express a \isi{passive} or at least non-agentive reading: \textit{tisamitu} (\textit{ti-sam-i-tu} 3i-hear-?-\textsc{iam}) ‘one hears (it)’ (Span. \textit{se escucha}). A final vowel /i/ of unknown origin also occurs on the stative verb stem \textit{-ÿnai} ‘be tall’, where \textit{ÿ} is the root ‘long’ and \textit{na} the general \isi{classifier}, see §\ref{sec:StativeVerbs_long}, and on the non-decomposable stem \textit{-sÿei} ‘be cold’. Note also that the adjective \textit{micha} seems to have a verbal origin, see discussion below.}

\ea\label{ex:ADJ-Pred-2}
\begingl
\glpreamble ja, michaniki\\
\gla ja micha-nik-i\\
\glb \textsc{afm} good-eat-?\\
\glft ‘yes, it is delicious’
\endgl
\trailingcitation{[jrx-c151001lsf-11.010]}
\xe

(\getref{ex:michamue-1}) was elicited from Juana.

\ea\label{ex:michamue-1}
\begingl
\glpreamble michamue\\
\gla micha-mu-e\\
\glb good-\textsc{clf:}sky?-?\\
\glft ‘the weather (lit.: sky) is nice’
\endgl
\trailingcitation{[jcx-e090727s.127]}
\xe
\is{derivation|)}

A verbal form is preferred\is{verb|(} when these concepts are negated,\is{negation} see (\getref{ex:ADJ-micha-IRR-1}) and (\getref{ex:michana-2}). Since there are no lexical antonyms, this happens relatively frequently. I would suggest that \textit{micha} originated as a \isi{stative verb} in the first place, but in positive statements, person markers were lost at some point, which then led to a hybrid behaviour of the form.

In (\getref{ex:ADJ-micha-IRR-1}), Juana talks about her mother.

\ea\label{ex:ADJ-micha-IRR-1}
\begingl
\glpreamble i tanÿma kuina tamicha chiyuikia mimi\\
\gla i tanÿma kuina ti-a-micha chi-yuik-i-a mimi\\
\glb and now \textsc{neg} 3i-\textsc{irr}-good 3-walk-\textsc{subord}-\textsc{irr} mum\\
\glft ‘and by that time my mother couldn’t walk well (lit.: her walking was not good) anymore’
\endgl
\trailingcitation{[jxx-p120430l-2.499]}
\xe

(\getref{ex:michana-2}) is about the school building in Santa Rita, which was in a miserable state. A new building was thus constructed.

\ea\label{ex:michana-2}
\begingl 
\glpreamble kuina tamichana echÿu, tikebupu echÿu\\
\gla kuina ti-a-michana echÿu ti-kebu-pu echÿu\\ 
\glb \textsc{neg} 3i-\textsc{irr}-nice \textsc{dem}b 3i-rain-\textsc{dloc} \textsc{dem}b\\ 
\glft ‘it wasn’t good, it dripped in’\\ 
\endgl
\trailingcitation{[mxx-p110825l.089]}
\xe
\is{verb|)}

I have also found one example in which \textit{michanabÿke} takes a first person index preceding the stem in a positive sentence, another proof for the semi-verbal behaviour of these adjectives. In this example, (\getref{ex:pretty-2}), Juana cites the water spirit whom their grandparents met on their way back home from Moxos, where they had bought some cows. The water spirit wanted to lure away Juana’s grandfather from his wife by appearing to him at night and telling him his wife was ugly and she was beautiful. It is not clear to me why Juana used the reportive \textit{-ji} on the connective \textit{chijikiu} ‘however’, since I believe this word belongs to the quoted speech.

\ea\label{ex:pretty-2}
\begingl
\glpreamble “chÿjikiuji bien nimichanabÿke”, dice\\
\gla chÿjikiu-ji bien ni-michanabÿke dice\\
\glb however-\textsc{rprt} well 1\textsc{sg}-beautiful she.says\\
\glft ‘“however, I am very beautiful”‘ she says
\endgl
\trailingcitation{[jxx-p151016l-2.190]}
\xe

On the other hand, I have also found one non-verbal form of \textit{micha} with irrealis RS in the corpus, which is presented in (\getref{ex:ADJ-micha-IRR-2}). There are a few more examples with derived forms taking the non-verbal irrealis marker, like the one in (\getref{ex:goodweather-1}).

(\getref{ex:ADJ-micha-IRR-2}) was produced by María S. and directed to me to say farewell.

\ea\label{ex:ADJ-micha-IRR-2}
\begingl
\glpreamble ¡michaina pibÿbÿkupunia!\\
\gla micha-ina pi-bÿbÿkupun-i-a\\
\glb good-\textsc{irr.nv} 2\textsc{sg}-fly.back-\textsc{subord}-\textsc{irr}\\
\glft ‘may your flight back be good!’
\endgl
\trailingcitation{[rxx-e120511l.204]}
\xe

(\getref{ex:goodweather-1}) comes from Juana. There had been heavy rainfalls and the road where she lived was very muddy, but the forecast had announced that rain would stop for a while. Note that Juana incorrectly uses the incompletive marker \textit{-kuÿ} here instead of the discontinuous marker \textit{-bu}, but she corrected herself in the utterance that immediately followed.

\ea\label{ex:goodweather-1}
\begingl
\glpreamble michamuenatu te tajaitu kuinakuÿ tikeba\\
\gla michamue-ina-tu te tajaitu kuina-kuÿ ti-keba\\
\glb of.good.weather-\textsc{irr.nv}-\textsc{iam} \textsc{seq} tomorrow \textsc{neg}-\textsc{incmp} 3i-rain.\textsc{irr}\\
\glft ‘the weather will be nice now, tomorrow it won’t rain anymore’
\endgl
\trailingcitation{[jxx-p120515l-2.269]}
\xe

In summary, \textit{micha} and its derivations have a verbal\is{verb} and a non-verbal form. Choice is sensitive to \isi{reality status}, with \isi{realis} triggering a non-verbal and \isi{irrealis} a verbal realisation. However, the correlation is not perfect, there are a few counter-examples in the corpus.

A second relatively frequent adjective is \textit{temena}/\textit{mutemena} ‘big’. Both forms can be used interchangeably without any difference in meaning.\footnote{The first syllable \textit{mu} of the longer form \textit{mutemena} could appear to be related to the non-productive \isi{privative} marker. However, this would imply that some kind of negation is involved in the meaning of \textit{mutemena}. This is not the case: \textit{temena} and \textit{mutemena} have exactly the same meaning. The two Paunaka words containing the \isi{privative} marker are discussed in Footnote \ref{fn:privative} of \sectref{sec:Negation}.} Juana clearly prefers \textit{temena}, Miguel \textit{mutemena} and María S. seems to use both equally frequently. The last syllable \textit{na} probably goes back to the general \isi{classifier} \textit{-na} (see \sectref{sec:Classifiers}), it detaches when another classifier is added (see below).

(\getref{ex:temena-1}) is an example of the form \textit{mutemena} used by María S. in making jokes with Swintha. The other examples in this section all include the shorter form \textit{temena}.

\ea\label{ex:temena-1}
\begingl
\glpreamble aja, mutemena pichubatÿi\\
\gla aja, mutemena pi-chubatÿi\\
\glb \textsc{afm} big 2\textsc{sg}-buttocks\\
\glft ‘yes, your butt is big’
\endgl
\trailingcitation{[rxx-e121128s-4x.107]}
\xe

When the collective\is{collective|(} marker is attached to the adjective (as well as to other adjectives and verbs that end in \textit{na}), the last syllable is usually repeated. It could be the case that repetition intensifies the collective meaning, but this would not explain why it occurs with the general \isi{classifier} only, not with any other one.\footnote{There are a few cases of repetition on nouns taking the collective marker, though, see \sectref{sec:RDPL_Nouns}.} Thus repetition does not seem to indicate anything in this case (which is why it is just glossed as ‘\textsc{rep}’, i.e. ‘repetition’, here), it just comes automatically with the collective marker, as in (\getref{ex:ADJ-big-1}), where María S. talks about ripe fruits that are falling from the trees. 

\ea\label{ex:ADJ-big-1}
\begingl
\glpreamble tebakaupujanetu temenanajitu\\
\gla ti-ebakaupu-jane-tu temena-na-ji-tu\\
\glb 3i-fall.down-\textsc{distr} big-\textsc{rep}-\textsc{col}-\textsc{iam}\\
\glft ‘they (fruits) are falling down, they are big now’
\endgl
\trailingcitation{[rxx-e121128s-3.07]}
\xe
\is{collective|)}

\textit{(Mu)temena} can take classifiers\is{classifier} or combine with body-part terms\is{incorporation} and in this case \textit{-na} is dropped. (\getref{ex:ADJ-big-2}) contains a classifier and (\getref{ex:ADJ-big-3}) an inalienably possessed plant part. Both were elicited from Juana.

\ea\label{ex:ADJ-big-2}
\begingl
\glpreamble temekiji anibÿ\\
\gla teme-ki-ji anibÿ\\
\glb big-\textsc{clf:}spherical-\textsc{col} mosquito\\
\glft ‘the mosquitos are big’
\endgl
\trailingcitation{[jxx-e150925l-1.187]}
\xe

\ea\label{ex:ADJ-big-3}
\begingl
\glpreamble temepuneji\\
\gla teme-pune-ji\\
\glb big-leaf-\textsc{col}\\
\glft ‘big leaves’
\endgl
\trailingcitation{[jxx-e081025s-1.183]}
\xe

 \textit{Kana}\is{demonstrative adjective|(} ‘this size’ is a demonstrative adjective that is always accompanied by a gesture showing the size. One example is (\getref{ex:big-whip}), which comes from Miguel, who was speaking about the whip of his teacher back in the old days when he went to school in \isi{Altavista}.
 
 \ea\label{ex:big-whip}
\begingl
\glpreamble kaku echÿu asotera chija bitÿpi echÿu, kana echÿu, chimusuji eka baka\\
\gla kaku echÿu asotera chi-ija bi-tÿpi echÿu kana echÿu chi-musuji eka baka\\
\glb exist \textsc{dem}b whip 3-name 1\textsc{pl}-\textsc{obl} \textsc{dem}b this.size \textsc{dem}b 3-skin \textsc{dem}a cow\\
\glft ‘he had what we call an \textit{azotera} (= whip), it was of this size (showing with hands), made from cowhide’
\endgl
\trailingcitation{[mxx-p181027l-1.056-057]}
\xe
 
As is the case with \textit{(mu)temena}, the last syllable of \textit{kana} is repeated when a \isi{collective} marker is added, see (\getref{ex:ADJ-kana-1}) from Juana, where she speaks about some shells she needs for polishing a clay pot.

\ea\label{ex:ADJ-kana-1}
\begingl
\glpreamble kananaji micha sipÿ\\
\gla kana-na-ji micha sipÿ\\
\glb this.size-\textsc{rep}-\textsc{col} good shell\\
\glft ‘very big like this, the shells’
\endgl
\trailingcitation{[jmx-d110918ls-1.105]}
\xe\is{demonstrative adjective|)}

Apart from the adjectives presented up to here, there are a few more words that express qualities, and it is sometimes hard to decide whether they are nouns or adjectives. Consider the expressions of age such as \textit{chubui} ‘old man; old (male)’ and \textit{juberÿpu} ‘old woman; old (female)’ as well as \textit{sepitÿ} or \textit{chepitÿ} ‘child, offspring; small, little’. I treat \textit{chubui} and \textit{juberÿpu} as nouns in this grammar, and \textit{sepitÿ}/\textit{chepitÿ} as noun or adjective, depending on the context. I admit there is ambiguity in the decisions I made. There are simply no good criteria to arrive at a clear decision. Like adjectives, \textit{chubui}, \textit{juberÿpu} and \textit{sepitÿ}/\textit{chepitÿ} often occur predicatively. However, they also frequently head an NP.\is{head} In addition to these criteria, \textit{sepitÿ}/\textit{chepitÿ} can also take the limitative marker \textit{-jiku} and is then often used adverbially. In reference to non-singular participants, speakers may also use \textit{sese-ji} instead of \textit{sepitÿ}, which is probably derived\is{derivation} from the same root \textit{se}. This word, however, is usually realised as \textit{sesejinube} with the fixed meaning ‘children’, while other more adjective-like uses are very rare. Some examples that point to the words for ‘small’ being adjectives are given below.

In (\getref{ex:small1-1}), Juana uses \textit{sepitÿ} to contrast small and big clay pots.

\ea\label{ex:small1-1}
\begingl
\glpreamble sepitÿ i temena\\
\gla sepitÿ i temena\\
\glb small and big\\
\glft ‘small ones and big ones’
\endgl
\trailingcitation{[jxx-d110923l-2.35]}
\xe

(\getref{ex:small1-2}) is a statement by María S. about a tortoise.

\ea\label{ex:small1-2}
\begingl
\glpreamble sepitÿmÿnÿ chikebÿke\\
\gla sepitÿ-mÿnÿ chi-kebÿke\\
\glb small-\textsc{dim} 3-eye\\
\glft ‘its eyes are small’
\endgl
\trailingcitation{[rxx-e121128s-4x.039]}
\xe

In (\getref{ex:small2-1}), María S. makes use of \textit{sese-ji-} in speaking about some fruits that still have to ripen.

\ea\label{ex:small2-1}
\begingl
\glpreamble sesejikuÿmÿnÿ nikechubi\\
\gla sese-ji-kuÿ-mÿnÿ ni-kechu-bi\\
\glb small-\textsc{col}-\textsc{incmp}-\textsc{dim} 1\textsc{sg}-say-2\textsc{sg}\\
\glft ‘they are still very small as I said to you’
\endgl
\trailingcitation{[rxx-e121126s-3.29]}
\xe

(\getref{ex:small2-2}) was elicited from Juana and comes from the same context as (\getref{ex:ADJ-big-3}) above. It seems that the general \isi{classifier} shows up on this form together with an incorporated noun.\is{incorporation|(} Due to lack of more examples with classifiers or nouns being attached to the stem, I cannot make any judgements about this being grammatical or not. It seems strange though considering that other adjectives usually drop \textit{-na} when they combine with another \isi{classifier} or noun.

\ea\label{ex:small2-2}
\begingl
\glpreamble sesepunenaji\\
\gla sese-pune-na-ji\\
\glb small-leaf-\textsc{clf:}general?-\textsc{col}\\
\glft ‘very small leaves’
\endgl
\trailingcitation{[jxx-e081025s-1.187]}
\xe
\is{incorporation|)}

Other adjectives are \textit{enui} ‘green, not ripe, raw’ and the borrowed\is{borrowing} colour terms \textit{asuru} ‘blue’ and \textit{amariyo} ‘yellow’. All of them occur extremely infrequently in the corpus and none of them takes classifiers.\is{classifier} The reason to treat them as adjectives is a purely semantic one. The terms for ‘white’ \textit{-kipÿpa} and ‘black’ \textit{-pisÿ} are stative verbs. The term for ‘red’, \textit{tisi}, is most probably a verb, too (thus its form is actually \textit{ti-(i?)si} 3i-be.red). It is shorter than other colour terms, and it never occurs with reference to a first or second person in the corpus (and attempts to elicit such forms failed). However, \isi{Mojeño Trinitario} has a cognate form \textit{-itsi}, which is clearly a verb (Rose 2021, p.c.). \textit{Kachu-} ‘big and round’, possibly related to \textit{kana} ‘this size’, occurs twice in the corpus and takes a \isi{classifier} or inalienably possessed noun,\is{incorporation} the latter is the case in (\getref{ex:biground}) which comes from Juana who produced it in elicitation with some pictures.

\ea\label{ex:biground}
\begingl
\glpreamble isijibÿ, kachujibÿ eka tarupe\\
\gla isijibÿ kachu-jibÿ eka tarupe\\
\glb flower big.round-flower \textsc{dem}a flower.sp\\
\glft ‘a flower, the \textit{taropé} (\textit{Dorstenia brasiliensis}) has a big and round blossom’
\endgl
\trailingcitation{[jcx-e090727s.027]}
\xe

\subsubsection{Usage}\label{sec:UsesADJ}

Adjectives are used predicatively\is{attributive clause|(} most of the time, which is evident from the examples given above. If there is a noun in the sentence to which the property is predicated, the adjective usually comes first, then comes the NP. (\getref{ex:ADJ-Pred-3}) and (\getref{ex:ADJ-Pred-1}) provide two examples of this. Adjectives are also often the only constituent of a clause.

In (\getref{ex:ADJ-Pred-3}), Juana speaks about a bird of prey that once stole her dog.

\ea\label{ex:ADJ-Pred-3}
\begingl
\glpreamble temena echÿu sia\\
\gla temena echÿu sia\\
\glb big \textsc{dem}b hawk.sp\\
\glft ‘the hawk is big’
\endgl
\trailingcitation{[jxx-a120516l-a.206]}%non-el.
\xe

(\getref{ex:ADJ-Pred-1}) comes from a listing of different crops by María C.

\ea\label{ex:ADJ-Pred-1}
\begingl
\glpreamble michanikiyuku echÿu papayu\\
\gla michaniki-yu-uku echÿu papayu\\
\glb delicious-\textsc{ints}-\textsc{add} \textsc{dem}b papaya\\
\glft ‘papayas are delicious, too’
\endgl
\trailingcitation{[uxx-p110825l.193]}
\xe
\is{attributive clause|)}

Adjectives are seldom used attributively\is{modification|(} in free speech. One spontaneous example with an attributive adjective is (\getref{ex:ADJ-ATTR-1}). Juana talks about the making of pasture by the people from Santa Rita in exchange for the construction of their reservoir.

\ea\label{ex:ADJ-ATTR-1}
\begingl
\glpreamble i echÿu max temenanaji yÿkÿke kapunu makina, motosierra chibu\\
\gla i echÿu max temena-na-ji yÿkÿke kapunu makina motosierra chibu\\
\glb and \textsc{dem}b more big-\textsc{rep}-\textsc{col} tree come machine chain.saw 3\textsc{top.prn}\\
\glft ‘and [for] the biggest trees, a machine came, a chain saw more precisely’
\endgl
\trailingcitation{[jxx-p120515l-2.115]}
\xe

An example including the \isi{demonstrative adjective} \textit{kana} used as an attribute is (\getref{ex:ADJ-new-ATTR-1}). It comes from a correction session with María S. She first repeats the word her brother Miguel used in telling the story about the fox and the jaguarundi: \textit{karutemÿnÿ} ‘small club’. Probably because of the Spanish origin of this word (Span. \textit{garrote} ‘club’), she adds a Paunaka expression herself that was not used by Miguel in the original utterance, and this expression consists of a noun modified by \textit{kana}.\footnote{As for the second syllable \textit{ke} in yÿkÿkekemÿnÿ, it is actually not clear whether this is the \isi{classifier} as proposed by the glosses or repetition of the previous syllable. The noun \textit{yÿkÿke} has both meanings ‘tree’ and ‘stick’ (and also ‘wood’), and it is already derived\is{derivation} with the classifier \textit{-ke} for cylindrical items from \textit{yÿkÿ} ‘fire’, although this is probably not transparent for the speakers.}

\ea\label{ex:ADJ-new-ATTR-1}
\begingl
\glpreamble chisatÿkuji karutemÿnÿ kana yÿkÿkekemÿnÿ – ¡pa! – chikupakutu\\
\gla chi-satÿku-ji karute-mÿnÿ kana yÿkÿke-ke-mÿnÿ pa chi-kupaku-tu\\
\glb 3-cut-\textsc{rprt} club-\textsc{dim} this.size tree-\textsc{clf:}cylindrical-\textsc{dim} \textsc{idph} 3-kill-\textsc{iam}\\
\glft ‘he cut a small club, a small stick of this size, it is said, and – bang! – he killed him’
\endgl
\trailingcitation{[rxx-e150220s-2]}
\xe

There are a few more examples of nouns modified by an adjective that were produced in elicitation.\is{modification|)} 

The adjective \textit{micha} ‘good’ can also be used adverbially and translates as “well”, “nicely”, “really” or “a lot” in this case. The adjective usually follows the verb it modifies, as in (\getref{ex:ADJ-Adv-1}), but it can also precede it if emphasised, which is the case in (\getref{ex:ADJ-Adv-2}).\is{word order}

(\getref{ex:ADJ-Adv-1}) comes from Juan C. speaking about frogs.

\ea\label{ex:ADJ-Adv-1}
\begingl
\glpreamble pero yuti tikusuninechu micha peÿ\\
\gla pero yuti ti-kusuninechu micha peÿ\\
\glb but night 3i-sing good frog\\
\glft ‘but at night the frogs sing a lot’
\endgl
\trailingcitation{[mqx-p110826l.617-618]}
\xe

In (\getref{ex:ADJ-Adv-2}), Juana tells me who taught her Paunaka.

\ea\label{ex:ADJ-Adv-2}
\begingl
\glpreamble nÿuse – chibu micha timesumeikunÿ\\
\gla nÿ-use chibu micha ti-mesumeiku-nÿ\\
\glb 1\textsc{sg}-grandmother 3\textsc{top.prn} good 3i-teach-1\textsc{sg}\\
\glft ‘my grandmother – she is the one who taught me well’
\endgl
\trailingcitation{[jxx-p120430l-1.050-051]}
\xe

Adjectives are also sometimes found in depictive use, i.e. as a secondary predicate \citep[cf.][]{Schultze-Berndt2004}. This is the case in (\getref{ex:ADJ-Dep-1}), in which the adjective specifies a property of the object, which is not conominated, a corn cob that was not completely roasted yet when Swintha wanted to eat it. The warning comes from María S.

\ea\label{ex:ADJ-Dep-1}
\begingl
\glpreamble ¡masaini piniku enui!, painuepÿi\\
\gla masaini pi-niku enui p-a-inuepÿi\\
\glb \textsc{adm} 2\textsc{sg}-eat green 2\textsc{sg}-\textsc{irr}-have.wind\\
\glft ‘don’t eat it raw! You will have wind’
\endgl
\trailingcitation{[rxx-e150220s-1.25]}
\xe

\subsection{Numerals and ‘other’}\label{sec:Numerals}
\is{numeral|(}

There is only one numeral of supposedly Paunaka origin, \textit{chÿnachÿ} ‘one’, exemplified in (\getref{ex:one-3}) from Miguel, who talks about his experience in school.

\ea\label{ex:one-3}
\begingl
\glpreamble pasautu chÿnachÿ anyo nÿti nÿchupupaikutu echÿu nÿtareane\\
\gla pasau-tu chÿnachÿ anyo nÿti nÿ-chupu-paiku-tu echÿu nÿ-tarea-ne\\
\glb pass-\textsc{iam} one year 1\textsc{sg.prn} 1\textsc{sg}-know-\textsc{punct}-\textsc{iam} \textsc{dem}a 1\textsc{sg}-exercise-\textsc{possd}\\
\glft ‘one year passed and I had learned my exercises’
\endgl
\trailingcitation{[mxx-p181027l-1.087]}
\xe


The numeral consists of three parts. The first part, \textit{chÿ} seems to coincide with the third person marker\is{person marking} \textit{chÿ-}. The syllable \textit{na} may well go back to the default classifier\is{classifier|(} \textit{-na} given the fact that numerals in the related Bolivian Arawakan\is{Southern Arawakan} languages obligatorily take a classifier, and they have a classifier \textit{-no} or \textit{-na}, which serves as a default classifier \citep[147--148]{Terhart2016}. However, unlike in the related languages, the classifier \textit{-na} is lexicalised\is{lexicalisation} on the numeral, i.e. it never changes, regardless of which item is counted, and no other classifier can be elicited with the numeral. The final part of the numeral \textit{chÿnachÿ} might again be a third person marker \textit{-chÿ}, but with bleached semantics. Rose (2021, p.c.) suggests that the final syllable could be related to the \isi{limitative} marker \textit{-yÿchi}. It could also be the case that we are dealing with a restrictive marker \textit{-chÿ} cognate to  Trinitario\is{Mojeño Trinitario} \textit{-chu} here (Rose 2021, p.c.). A similar form \textit{-chu}/\textit{-chÿ}/\textit{-chÿu} sometimes occurs on question words\is{question word} (see \sectref{sec:ContentQuestions}), but is otherwise not productive in Paunaka. In any case, \textit{-chÿ} is usually omitted if other markers are attached to the numeral.\is{classifier|)}

In (\getref{ex:one-1}), Juana speaks about her daughter, who had fallen and badly injured her leg.

\ea\label{ex:one-1}
\begingl
\glpreamble eka chÿnachÿ kuje kuina puero tichema\\
\gla eka chÿnachÿ kuje kuina puero ti-chema\\
\glb \textsc{dem}a one month \textsc{neg} can 3i-stand.up.\textsc{irr}\\
\glft ‘she could not stand up for one month’
\endgl
\trailingcitation{[jxx-p110923l-1.474]}
\xe

(\getref{ex:one-2}) comes from Miguel’s story about the lazybones. Instead of making a field to nurture his family, he cuts off his own limbs in the end of the story, pretending they were \textit{cusi} palm fruits.

\ea\label{ex:one-2}
\begingl
\glpreamble chisatÿkujitu chinachÿ chijabu\\
\gla chi-satÿku-ji-tu chinachÿ chi-jabu\\
\glb 3-cut-\textsc{rprt}-\textsc{iam} one 3-leg\\
\glft ‘he cut off one of his legs, it is said’
\endgl
\trailingcitation{[mox-n110920l.097]}
\xe


The numeral is sometimes used like an indefinite article,\is{definiteness} which can be assumed is due to influence of Spanish, where the numeral \textit{uno} and the indefinite article \textit{un/una} are very similar (as is the case in many languages and directly connected to the fact that the indefinite article often derives from\is{grammaticalisation} the numeral).

(\getref{ex:one-4}) is the introductory sentence of the story about the lazybones told by Miguel.

\ea\label{ex:one-4}
\begingl
\glpreamble kakubaneji chÿnachÿ jente i tipÿkubai\\
\gla kaku-bane-ji chÿnachÿ jente i ti-pÿkubai\\
\glb exist-\textsc{rem}-\textsc{rprt} one man and 3i-be.lazy\\
\glft ‘once upon a time there was a man, it is said, and he was lazy’
\endgl
\trailingcitation{[mox-n110920l.011]}
\xe

The numeral can take the \isi{limitative} marker \textit{-jiku} and in that case, the person marker \textit{-chÿ} is detached.\is{person marking}

In (\getref{ex:one-5}), María C. asks me about my children.%\footnote{Note that she uses \textit{-checha} with the meaning of ‘daughter’ like in Bésiro, while for the other speakers of Paunaka, \textit{-checha} means ‘son’ or more generally ‘offspring’ (and also ‘egg’). \citet[48]{CarvalhoRose2018} reconstruct a noun \textit{*ʧiʧa} for Proto-Mojeño with the meaning ‘child’.}


\ea\label{ex:one-5}
\begingl
\glpreamble ¿chÿnajiku pichecha?\\
\gla chÿna-jiku pi-checha\\
\glb one-\textsc{lim}1 2\textsc{sg}-son\\
\glft ‘you have only one child?’
\endgl
\trailingcitation{[uxx-p110825l.242]}
\xe

(\getref{ex:one-6}) is from the story about the fox and the jaguarundi. The fox boasts about knowing 25 jumps, the jaguarundi has to admit to know only one (which saves him in the end, while the fox is killed).

\ea\label{ex:one-6}
\begingl
\glpreamble “kuina kakuina beintisinko nikeuchi, chÿnajiku”, tikechuji\\
\gla kuina kaku-ina beintisinko ni-keuchi chÿna-jiku ti-kechu-ji\\
\glb \textsc{neg} exist-\textsc{irr.nv} twenty-five 1\textsc{sg}-\textsc{ins} one-\textsc{lim}1 3i-say-\textsc{rprt}\\
\glft ‘“I don’t have 25, only one", he said, it is said’
\endgl
\trailingcitation{[jmx-n120429ls-x5.363]}
\xe

Finally, the number occurs in an exclamation equivalent to the English ‘oh Lord!’ or ‘good Lord!’ (Spanish ‘¡aiy señor!’), as in (\getref{ex:one-7}), which comes from Miguel. 

\ea\label{ex:one-7}
\begingl
\glpreamble ¡chÿnayue!\\
\gla chÿna-yu-e\\
\glb one-\textsc{ints}-2\textsc{pl}\\
\glft ‘good Lord!’ (lit.: ‘you (\textsc{pl}) dear one’)
\endgl
\trailingcitation{[rmx-e150922l.060]}%m
\xe

All numbers higher than ‘one’ have been borrowed\is{borrowing|(} from Spanish. Some of them also attach \textit{-chÿ}. This is obligatory with \textit{ruschÿ} ‘two’, and highly usual with \textit{treschÿ} ‘three’. From ‘four’ on, it gets less likely the higher the number in general,\footnote{Interestingly, a similar observation has been made for Trinitario\is{Mojeño Trinitario} \citep[23]{Rose2020}, and it also applies to \isi{Baure}, but it is a classifier which is more unlikely to be attached to higher numerals in those languages. Note that Trinitario and \isi{Baure} both have native numerals up to ‘three’ .} however, the numeral ‘twenty’ has also been found with \textit{-chÿ} once. The number ‘two’, \textit{ruschÿ}, is phonologically integrated into Paunaka, the Spanish word is \textit{dos} and /d/ changed to /r/ and /o/ to /u/ here. Other numerals are less integrated phonologically, e.g. the \isi{consonant cluster} in \textit{treschÿ} from Spanish \textit{tres} ‘three’ is not dissolved.\is{borrowing|)} 

(\getref{ex:two-2}) is an example of the numeral ‘two’ and (\getref{ex:three-1}) an example of ‘three’. In (\getref{ex:two-2}), Juana makes a statement about her daughter.

\ea\label{ex:two-2}
\begingl
\glpreamble kaku ruschÿ chilotene nauku\\
\gla kaku ruschÿ chi-lote-ne nauku\\
\glb exist two 3-plot-\textsc{possd} there\\
\glft ‘she has two plots there’
\endgl
\trailingcitation{[jxx-p110923l-1.421]}
\xe

(\getref{ex:three-1}) is also from Juana. She tells me about the duration of her grandson’s university studies here.

\ea\label{ex:three-1}
\begingl
\glpreamble sinko anyo tiyunuku treschÿ anyo\\
\gla sinko anyo ti-yunuku treschÿ anyo\\
\glb five year 3i-go.on three year\\
\glft ‘five years (in total), he goes on for three years’
\endgl
\trailingcitation{[jxx-p110923l-1.191]}
\xe


In (\getref{ex:two-1}), Miguel uses several numerals, ‘two’ and ‘three’ are realised with \textit{-chÿ}, but ‘four’ is not. His statement provides the answer to Swintha’s question how many baking trays of rice bread he baked together with his family. Note that the last verb is irregularly used without a subject index here.

\ea\label{ex:two-1}
\begingl
\glpreamble kuatru tipurtukabu jurnuye, pero ruschÿ banaiu entonses banaukupunuku punachÿ ruschÿ o treschÿ purtukupunuku\\
\gla kuatru ti-purtuka-bu jurnu-yae pero ruschÿ bi-ana-i-u entonses bi-anau-uku-punuku punachÿ ruschÿ o treschÿ purtuku-punuku\\
\glb four 3i-put.in.\textsc{irr}-\textsc{mid} oven-\textsc{loc} but two 1\textsc{pl}-make-\textsc{subord}-\textsc{real} thus 1\textsc{pl}-make-\textsc{add}-\textsc{reg} other two or three put.in-\textsc{reg}\\
\glft ‘four can be put into the oven, but having made two, then we made another two or three and put them in again’
\endgl
\trailingcitation{[mxx-e120415ls.096-097]}
\xe


On the other hand in (\getref{ex:four-1}), Juana uses the numeral ‘four’ with \textit{-chÿ}. The sentence refers to the picture in the end of the \isi{frog story}, where the boy finds his frog again, together with a frog lady and several little frogs.

\ea\label{ex:four-1}
\begingl
\glpreamble puru peÿjane kuatrochÿ chichecha\\
\gla puru peÿ-jane kuatruchÿ chi-checha\\
\glb mere frog-\textsc{distr} four 3-son \\
\glft ‘mere frogs, it has four children’
\endgl
\trailingcitation{[jxx-a120516l-a.435]}
\xe

Some TAME markers\is{tense}\is{aspect}\is{modality}\is{evidentiality} can attach to numerals, e.g. the \isi{iamitive} marker in (\getref{ex:two-3}), where María S. tells me about her situation.

\ea\label{ex:two-3}
\begingl
\glpreamble ruschÿtu anyo kuina nakuesanebu\\
\gla ruschÿ-tu anyo kuina nÿ-a-kuesane-bu\\
\glb two-\textsc{iam} year \textsc{neg} 1\textsc{sg}-\textsc{irr}-have.field-\textsc{dsc}\\
\glft ‘it’s already two years that I don’t have a field anymore’
\endgl
\trailingcitation{[rxx-e181017l.018]}
\xe

The numerals ‘two’ and ‘three’ can also take a person marker\is{person marking|(} in reference to humans, as in (\getref{ex:three-2}), or the \isi{plural} marker, as in (\getref{ex:three-3}). In the latter case, \textit{-chÿ} of \textit{ruschÿ} and \textit{treschÿ}, sometimes together with the final /s/ of the stem, weakens into [ʃ], [​ʂ​] or [​ʐ​], represented orthographically\is{orthography} as <xh> and <x> in the examples, and \textit{-chÿ} may then be attached if the numeral has third person reference. Thus in this case, the person marker \textit{-chÿ} seems to be involved. This is usually bound to the numeral being used predicatively, with a few counter-examples, where a numeral formed in this way is used attributively.

In (\getref{ex:three-2}), Juana speaks about the Supepí sisters who are still alive.

\ea\label{ex:three-2}
\begingl
\glpreamble i nÿti, Maria, Krara, tresxhexheikubimÿnÿ tanÿma\\
\gla i nÿti Maria Krara tresxhe-xheiku-bi-mÿnÿ tanÿma\\
\glb and 1\textsc{sg.prn} María Clara three-\textsc{cont}-1\textsc{pl}-\textsc{dim} now\\
\glft ‘and me, María, Clara, we are only three now’
\endgl
\trailingcitation{[jxx-p120430l-2.352-353]}
\xe

María C. was once severely injured by black magic. In (\getref{ex:two-predi}) she states how many frogs she had in her belly.

\ea\label{ex:two-predi}
\begingl
\glpreamble rusxenubechÿ \\
\gla rusxe-nube-chÿ\\
\glb two-\textsc{pl}-3\\
\glft ‘there were two of them’
\endgl
\trailingcitation{[ump-p110815sf.312]}
\xe

If the numeral is used attributively, the \isi{plural} marker can also be attached to it, but usually follows \textit{-chÿ} in that case and no sound change is involved, as in (\getref{ex:two-5}). The plural marker is not obligatory though.

In the following example, María S. speaks about her sister Juana, referring to the time when the family still lived more remote.

\ea\label{ex:two-5}
\begingl
\glpreamble kakutu ruschÿnube chichechajimÿnÿbane\\
\gla kaku-tu ruschÿ-nube chi-checha-ji-mÿnÿ-bane\\
\glb exist-\textsc{iam} two-\textsc{pl} 3-son-\textsc{col}-\textsc{dim}-\textsc{rem}\\
\glft ‘she already had two little children by that time long ago’
\endgl
\trailingcitation{[rxx-p181101l-2.107]}
\xe

(\getref{ex:two-4}) combines two possibilities. María S. first uses the numeral ‘two’ in an equative sentence\is{equative/proper inclusion clause} juxtaposed to a demonstrative. The demonstrative has the plural marker, the numeral does not. She then repeats the numeral as the sole predicate of a clause, and since this clause still refers to humans, the plural marker is attached to the numeral and the third person marker \textit{-chÿ} follows.\is{person marking|)} This is a statement about my children.

\ea\label{ex:two-4}
\begingl
\glpreamble ruschÿkena ekanube, rusxhunubechÿ\\
\gla ruschÿ-kena eka-nube rusxhu-nube-chÿ\\
\glb two-\textsc{uncert} \textsc{dem}a-\textsc{pl} two-\textsc{pl}-3\\
\glft ‘they are probably two, they are two’
\endgl
\trailingcitation{[rmx-e150922l.078]}
\xe

(\getref{ex:five-1}) is the only example with a numeral higher than ‘three’ that I have found taking a plural marker. It is the number ‘five’ used attributively, nonetheless, the plural marker comes first and then comes \textit{-chÿ} and finally an irrealis marker. The sentence comes from Miguel’s story about the cowherd who is enchanted by the spirit of the hill. The spirit first takes away the cows and hides them in his hill, but in the end of the story the cows are brought to a village for the people there to eat. 

\ea\label{ex:five-1}
\begingl 
\glpreamble “kapununubeina sinkonubechina jentenube ayaraunubeina bitÿpi eka bumia eka bakajane”\\
\gla kapunu-nube-ina sinko-nube-chi-ina jente-nube ayarau-nube-ina bi-tÿpi eka bi-um-i-a eka baka-jane\\ 
\glb come-\textsc{pl}-\textsc{irr.nv} five-\textsc{pl}-3-\textsc{irr.nv} man-\textsc{pl} help-\textsc{pl}-\textsc{irr.nv} 1\textsc{pl}-\textsc{obl} \textsc{dem}a 1\textsc{pl}-take-\textsc{subord}-\textsc{irr} \textsc{dem}a cow-\textsc{distr}\\ 
\glft ‘“... may five men come to help us take the cows”’\\ 
\endgl
\trailingcitation{[mxx-n151017l-1.78]}
\xe

Dates and times of the day are realised with Spanish numerals without attachment of \textit{-chÿ}. As for times of the day, the numeral is usually accompanied by the Spanish feminine article \textit{la(s)} and regarding ‘one o’clock’ and ‘two o‘clock’, the numerals \textit{una} and \textit{dos} are used rather than the Paunaka ones as in (\getref{ex:num-time-1}) and (\getref{ex:num-time-2}).

(\getref{ex:num-time-1}) is the answer of María S. to my question whether she had been to Concepción. Note that the verb does not carry the middle marker here, which is unusual, especially since there is no overt goal.

\ea\label{ex:num-time-1}
\begingl
\glpreamble hm, nitupunu la una\\
\gla hm ni-tupunu {la una}\\
\glb \textsc{afm} 1\textsc{sg}-reach {at one o’clock}\\
\glft ‘hm, I arrived at one o’clock’
\endgl
\trailingcitation{[rxx-e120511l.002]}
\xe

(\getref{ex:num-time-2}) comes from Juana telling me about the last things her brother did before he suddenly and unexpectedly died.

\newpage

\ea\label{ex:num-time-2}
\begingl
\glpreamble titupunubuji nauku las doskena\\
\gla ti-tupunubu-ji nauku {las dos}-kena\\
\glb 3i-arrive-\textsc{rprt} there {at two o’clock}-\textsc{uncert}\\
\glft ‘he arrived there at two o’clock maybe, it is said’
\endgl
\trailingcitation{[jxx-p120430l-2.404]}
\xe

The noun \textit{tose} ‘noon’ has possibly been borrowed\is{borrowing} from \isi{Bésiro}, although it presumably originates from the Spanish numeral \textit{doce} ‘twelve’. \textit{Tose} is exclusively used with reference to the midday, if the number is meant, speakers use \textit{dose},\footnote{Surprisingly, in Trinitario\is{Mojeño Trinitario} it is the other way round: \textit{te las doce} means ‘at noon’ and \textit{ntose} is used for counting (Rose 2021, p.c.).} compare (\getref{ex:twelve-1}) with the noun and (\getref{ex:twelve-2}) with the numeral.

(\getref{ex:twelve-1}) is a question by Clara directed to Swintha and me. We had been to Santa Rita that same day before visiting her and María C.

\ea\label{ex:twelve-1}
\begingl
\glpreamble ¿tose etupupunubu o kupeitu?\\
\gla tose e-tupupunu-bu o kupei-tu\\
\glb noon 2\textsc{pl}-arrive.back-\textsc{mid} or afternoon-\textsc{iam}\\
\glft ‘did you arrive back (from Santa Rita) at noon or in the afternoon?’
\endgl
\trailingcitation{[cux-c120414ls-2.332]}
\xe

(\getref{ex:twelve-2}) comes from Miguel telling the history of Santa Rita, which was founded after people were let free from forced labour in \isi{Altavista}.

\ea\label{ex:twelve-2}
\begingl 
\glpreamble kapunutu kuineini taitaini pero kapununube dose familia\\
\gla kapunu-tu kuineini taita-ini pero kapunu-nube dose familia\\ 
\glb come-\textsc{iam} deceased dad-\textsc{dec} but come-\textsc{pl} twelve family\\ 
\glft ‘my late father had come (here), but twelve families came (altogether)’\\ 
\endgl
\trailingcitation{[mxx-p110825l.056]}
\xe

The word for ‘other’ is \textit{punachÿ}. It resembles the numeral \textit{chÿnachÿ} in the way it is composed. The first syllable \textit{pu} is possibly related to the Proto-Arawakan numeral \textit{*pa-} ‘one’, which has developed into an impersonal \isi{pronoun} in some \isi{Arawakan languages} \citep[85]{Aikhenvald1999}.\is{numeral|)} Sometimes, Juana preposes a /u/ yielding \textit{upunachÿ}, but this is more frequent in the derived forms (see below).\footnote{The preposed /u/ is expected by the \isi{stress} pattern (see \sectref{sec:Stress}), and it also relates Paunaka to \isi{Mojeño Trinitario}, where the cognate form \textit{(‘)po-na} sometimes occurs with an initial glottal stop that corresponds to a syncopated vowel (Rose 2021, p.c.).}

\textit{Punachÿ} can be used as a modifier\is{modification} as in (\getref{ex:other-2}) or head an NP as in (\getref{ex:other-3}).\is{head} 

In (\getref{ex:other-2}), Juana tells me about her plans to move to another house in Santa Cruz together with the family of her daughter.

\ea\label{ex:other-2}
\begingl
\glpreamble repente bisemaika punachÿ kuarto nauku\\
\gla repente bi-semaika punachÿ kuarto nauku\\
\glb maybe 1\textsc{pl}-search.\textsc{irr} other room there\\
\glft ‘maybe we want to look for another room (i.e. house with one more room) there’
\endgl
\trailingcitation{[jxx-p120430l-1.355]}
\xe

%\ea\label{ex:other-1}
%\begingl
%\glpreamble i punachÿ kaku Espanya\\
%\gla i punachÿ kaku Espanya\\
%\glb and other exist Spain \\
%\glft ‘and there is another one in Spain’\\
%\endgl
%\trailingcitation{[jxx-e120516l-1.028]}
%\xe


Interestingly, in (\getref{ex:other-3}) María S. uses \textit{punachÿ} twice to contrast two men, where English (and also Spanish) would use the numeral ‘one’ in contrast with ‘other’. There are more examples in the corpus that point into a similar direction, but none is as clear as this one. It comes from the story about the two men who meet the devil in the woods. One of them interacts with the devil and is finally eaten, the other hides away on a tree and can escape in the end.

\ea\label{ex:other-3}
\begingl
\glpreamble echÿu punachÿ tipunu anÿke, mhm, i echÿu punachÿ kuina tipunaji ...\\
\gla echÿu punachÿ ti-punu anÿke mhm i echÿu punachÿ kuina ti-puna-ji\\
\glb \textsc{dem}a other 3i-go.up up \textsc{intj} and \textsc{dem}a other \textsc{neg} 3i-go.up.\textsc{irr}-\textsc{rprt}\\
\glft ‘one of them climbed up, mhm, and the other one didn’t climb up, it is said...’
\endgl
\trailingcitation{[rxx-n120511l-2.32-35]}
\xe


Like numerals, \textit{punachÿ} can attach some markers. The final \textit{-chÿ} is sometimes detached, but this happens relatively infrequently. Consider  (\getref{ex:other-4}) and (\getref{ex:other-5}), which come from Miguel and María S. respectively and were produced one after the other. In (\getref{ex:other-4}), the irrealis and the uncertainty marker are attached to the full form \textit{punachÿ}, in (\getref{ex:other-5}), the final \textit{-chÿ} is detached and replaced by the irrealis marker. Both sentences refer to my announced return to Bolivia. Note that in local Spanish, people use ‘other’ in combination with a temporal noun to refer to the next day, week, month or year, although this is not very precise and can sometimes also refer to the time unit following the next one.

\ea\label{ex:other-4}
\begingl
\glpreamble punachinakena anyo tibÿsÿupunuka\\
\gla punachÿ-ina-kena anyo ti-bÿsÿu-punuka\\
\glb other-\textsc{irr.nv}-\textsc{uncert} year 3i-come-\textsc{reg.irr}\\
\glft ‘maybe next year she will come back’
\endgl
\trailingcitation{[mrx-c120509l.125]}
\xe

\ea\label{ex:other-5}
\begingl
\glpreamble ¿puneina anyo pibÿsÿupunuka?\\
\gla puna-ina anyo pi-bÿsÿu-punuka\\
\glb other-\textsc{irr.nv} year 2\textsc{sg}-come-\textsc{reg.irr}\\
\glft ‘you will come back next year?’
\endgl
\trailingcitation{[mrx-c120509l.126]}
\xe


Unlike \textit{chÿnachÿ}, \textit{punachÿ} can detach the supposed \isi{classifier} \textit{-na} in three cases. First of all, it can combine with the \isi{verbal root} \textit{-jai} ‘be light, day’ followed by a syllable \textit{-ne}, which is probably the possessed marker (see \sectref{sec:Alienables}), yielding \textit{(u)pujaine} ‘the other day’. Second, it can also combine with the relational noun \textit{-akene} ‘non-visible side’ as \textit{(u)puakene} ‘other side’.\footnote{When combining with a person marker, this noun is also sometimes pronounced \textit{-ekene}, see \sectref{sec:Locative}.} Whether the initial /u/ occurs on both words seems to be bound to the rhythm of the whole sentence. Third, the \isi{distributive} marker is attached directly to the root, in this case the \isi{plural} marker usually follows the distributive (except for one example in the corpus), thus the form is \textit{pujanenube} ‘the others’. It is often used to refer to ‘all the others’, but not exclusively. While \textit{(u)puakene} is often found with a third person marker following it, \textit{(u)pujaine} and \textit{pujanenube} never take a third person marker. One example of each derived form is given below.\is{derivation|(}

In (\getref{ex:other-6}), Juana talks about one of her daughters building a house.

\ea\label{ex:other-6}
\begingl
\glpreamble ja’a puakenechÿ tanaunube chubiunubeina\\
\gla ja’a pu-akene-chÿ ti-anau-nube chÿ-ubiu-nube-ina\\
\glb \textsc{afm} other-non.vis.side-3 3i-make-\textsc{pl} 3-house-\textsc{pl}-\textsc{irr.nv}\\
\glft ‘yes, on the other side (of the street) they are making their future house’
\endgl
\trailingcitation{[jxx-p110923l-2.154]}
\xe

(\getref{ex:other-7}) comes from María S. telling me about the former times, or more precisely, the food her mother cooked in former times.

\ea\label{ex:other-7}
\begingl
\glpreamble chÿnachÿ tijai tiyÿtikapumÿnÿ arusuji pujaine pujukekepupunukutu tiniku\\
\gla chÿnachÿ tijai ti-yÿtikapu-mÿnÿ arusu-ji pu-jai-ne pujukeke-pupunuku-tu ti-niku\\
\glb one day 3i-cook.\textsc{irr}-\textsc{dim} rice-\textsc{clf:}soft.mass other-day-\textsc{possd} patasca-\textsc{reg}-\textsc{iam} 3i-eat\\
\glft ‘one day she would cook a rice stew, the other day she ate \textit{patasca} again’
\endgl
\trailingcitation{[rxx-p181101l-2.250]}
\xe

Finally in (\getref{ex:other-8}) from the same recording as the previous example, María S. explained me why she did not have friends when she was a child. She lived with her family a little remote, while other families already settled in the place where the village of Santa Rita is located until now.

\ea\label{ex:other-8}
\begingl
\glpreamble chijikiu pujanenube naka chubiunube\\
\gla chijikiu pu-jane-nube naka chÿ-ubiu-nube\\
\glb however other-\textsc{distr}-\textsc{pl} here 3-house-\textsc{pl}\\
\glft ‘in contrast, all the others had their houses here’
\endgl
\trailingcitation{[rxx-p181101l-2.119]}
\xe
\is{derivation|)}

\subsection{Quantifiers}\label{sec:QuantifyingAdverbs}
\is{quantifier|(}

Quantifiers provide information about the quantity of something. They can modify nouns, but only marginally. More often, they are used as heads of NPs,\is{head} as predicates (see also \sectref{sec:NP}) or they modify a \isi{verb}. \tabref{table:QuantifyingAdverbs} provides an overview of the quantifiers found in the corpus.

\begin{table}[htbp]
\caption[Quantifiers]{Quantifiers}

\begin{tabular}{lll}
\lsptoprule
Quantifier & Translation & Comment\cr
\midrule
\textit{chama} & much, many, a lot & loan from Bésiro \cr
\textit{pariki} & many, much, a lot  & \cr
\textit{pario} & some, something, to some degree & loan from Bésiro \cr
\textit{pasayu} & much, a lot & \cr
\textit{musume} & many & \cr
\textit{tumuyubu} & all & lexicalised from verb \cr
\lspbottomrule
\end{tabular}

\label{table:QuantifyingAdverbs}
\end{table}

The quantifier \textit{pario} has been borrowed\is{borrowing|(} from \isi{Bésiro} \citep[cf.][333]{FussRiester1986}, and \textit{pariki} seems to be related to it. However, I do not know whether the latter one is also used in Bésiro or has been derived from the former in Paunaka. While \textit{pario} encodes that something holds to some degree as in (\getref{ex:pario-ex}), \textit{pariki} tells us that something holds to a high degree as in (\getref{ex:pariki-ex}).\is{borrowing|)} Synonym with the latter is \textit{musume} as in (\getref{ex:musume-ex}), which is also sometimes given as \textit{musube}.

(\getref{ex:pario-ex}) comes from María C. who echoes a prior statement by Clara about her son. This kind of echoing is frequently used in conversation as a back-channelling device. 

\ea\label{ex:pario-ex}
\begingl
\glpreamble tichupumÿnÿ pario paunaka\\
\gla ti-chupu-mÿnÿ pario paunaka\\
\glb 3i-know-\textsc{dim} some Paunaka\\
\glft ‘he knows some Paunaka’
\endgl
\trailingcitation{[cux-c120414ls-2.269]}
\xe

In (\getref{ex:pariki-ex}), Juana makes a statement about the water reservoir of Santa Rita.

\ea\label{ex:pariki-ex}
\begingl
\glpreamble pariki jimu nechÿu\\
\gla pariki jimu nechÿu\\
\glb many fish \textsc{dem}c\\
\glft ‘there is a lot of fish’
\endgl
\trailingcitation{[jxx-p120515l-2.135]}
\xe

In (\getref{ex:musume-ex}), Juana talks about the viciousness of the \textit{karay} who, realising that the speakers’ grandparents had bought many cows, made plans to usurp them.

\ea\label{ex:musume-ex}
\begingl
\glpreamble chimunube musume, te tiyunukunubetu chibejiukunubetu chipeu baka\\
\gla chi-imu-nube musume te ti-yunuku-nube-tu chi-bejiuku-nube-tu chi-peu baka\\
\glb 3-see-\textsc{pl} many \textsc{seq} 3i-go.on-\textsc{pl}-\textsc{iam} 3-take.away-\textsc{pl}-\textsc{iam} 3-animal cow\\
\glft ‘they saw that there were many (cows), so they went to take away their cows’
\endgl
\trailingcitation{[jxx-e150925l-1.258]}%non-el.
\xe

While both \textit{pariki} and \textit{musume} are predominantly used for countable items, \textit{chama} is used with non-countable things, e.g. water. Compare (\getref{ex:chama-1}) to (\getref{ex:pariki-ex}) above. 

(\getref{ex:chama-1}) is also about a water reservoir, but this time about the one in \isi{Altavista}. The statement comes from Clara.

\ea\label{ex:chama-1}
\begingl
\glpreamble chama ÿne nechÿu\\
\gla chama ÿne nechÿu\\
\glb much water \textsc{dem}c\\
\glft ‘there is a lot of water there’
\endgl
\trailingcitation{[cux-c120414ls-1.207]}
\xe


María C. produced (\getref{ex:chama-verb}) when we were making fun about being drunk (resulting from Swintha asking Clara for the word for ‘be drunk’.)

\ea\label{ex:chama-verb}
\begingl
\glpreamble chama teukena\\
\gla chama ti-eu-kena\\
\glb much 3i-drink-\textsc{uncert}\\
\glft ‘maybe she has drunk a lot’
\endgl
\trailingcitation{[cux-c120414ls-1.056]}
\xe

However, \textit{pariki} is sometimes also found in connection with non-countable and \textit{chama} with countable things.

An alternative to \textit{chama} is \textit{pasayu} ‘much, a lot’, but it is used less often and majorly by María S. (\getref{ex:pasayu-n}), however, is an example that stems from the recordings made by Riester in the 1960s. Juan Ch. talks about the amount of work he is forced to do, in this case weeding in the peanut plantation:

\ea\label{ex:pasayu-n}
\begingl
\glpreamble ... pasayu chikeuchi bipatrunenube\\
\gla pasayu chi-keuchi bi-patrun-ne-nube\\
\glb much 3-\textsc{inst}\\
\glft ‘it is a lot because of our \textit{patrones}’
\endgl
\trailingcitation{[nxx-p630101g-2.26]}
\xe


(\getref{ex:pario-verb}) is an example of an adverbial use of a quantifier. It comes from Miguel telling José the \isi{frog story}. It refers to the picture on which the boy sees the beehive.

\ea\label{ex:pario-verb}
\begingl
\glpreamble i naka tiyuyuikutu pario eka aitubuchepÿimÿnÿ\\
\gla i naka ti-iyuyuiku-tu pario eka aitubuchepÿi-mÿnÿ\\
\glb and here 3i-cry-\textsc{iam} some \textsc{dem}a boy-\textsc{dim}\\
\glft ‘and here the little boy is crying a bit’
\endgl
\trailingcitation{[mox-a110920l-2.067]}
\xe


The quantifier \textit{tumuyubu} looks like a lexicalised\is{lexicalisation} middle verb\is{middle voice} (see \sectref{sec:Middle_voice}). It can possibly be decomposed as \textit{ti-umu-bu-yu} (3i-take-\textsc{mid}-\textsc{ints}) ‘much is taken’. Its meaning is ‘all, everything’. One example is (\getref{ex:tumuyubu-2}). A very small part of Juana’s speech is omitted from this example, since she only confirms a side question of me and then goes on with her utterance. She lists all the things a couple from Germany had brought to Concepción in order to sell them in their shop.

\ea\label{ex:tumuyubu-2}
\begingl
\glpreamble tupununube mÿiji (...) i tumuyubu tÿmuepa yubuti eka kuicha, tumuyubu\\
\gla ti-upunu-nube mÿiji i tumuyubu tÿmuepa yubuti eka kuicha tumuyubu\\
\glb 3i-bring-\textsc{pl} grass and all knife axe \textsc{dem}a spade all\\
\glft ‘they brought grass (seeds) (...) and everything, knifes, axes, spades, everything’
\endgl
\trailingcitation{[jxx-p120515l-2.036-039]}
\xe

If the referent is human, the \isi{plural} marker can be added to the quantifier. One example of this is given in (\getref{ex:tumuyubu-1}), where \textit{tumuyubunube} ‘all of them’ is the subject of the non-verbal motion predicate. This example also comes from Juana and refers to the people of Santa Rita who had all planned to come to Concepción to see the appearance of Evo Morales in the multi-purpose hall.

\ea\label{ex:tumuyubu-1}
\begingl
\glpreamble tumuyubunube kapununubeina\\
\gla tumuyubu-nube kapunu-nube-ina\\
\glb all-\textsc{pl} come-\textsc{pl}-\textsc{irr.nv}\\
\glft ‘they will all come’
\endgl
\trailingcitation{[jxx-p150920l.079]}
\xe

\is{quantifier|)}
\is{adjective|)}

%chaama = mucho: Sans 2013:63
% Bésiro poco = chímyantai, pequeño: chimyámantai, algunos: eanákiatai ubutúriki, harto: sɨrɨmána (Sans 2010:151)

This was the last example of this section, the following one deals with adverbs.




















