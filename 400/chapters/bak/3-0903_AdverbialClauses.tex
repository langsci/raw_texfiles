%!TEX root = 3-P_Masterdokument.tex
%!TEX encoding = UTF-8 Unicode

\section{Adverbial relations}\label{sec:AdverbialClauses}
\is{subordination|(}
\is{adverbial relation|(}

According to \citet[155]{Cristofaro2003}, one event is in an adverbial relation to another one if it describes the circumstances of the latter. Consider (\ref{ex:new23cause}) in which the second clause is introduced by \textit{porke} ‘because’ and provides the reason or cause for the first one. It was elicited from Clara.

\ea\label{ex:new23cause}
\begingl
\glpreamble kuina puero trabakuneina \textup{[}porke nikubÿu\textup{]}\\
\gla kuina puero trabaku-ne-ina porke ni-kubÿu\\
\glb \textsc{neg} can work-1\textsc{sg}-\textsc{irr.nv} because 1\textsc{sg}-be.drunk\\
\glft ‘I cannot work because I am drunk’
\endgl
\trailingcitation{[cux-c120414ls-1.062-063]}
\xe

The circumstances (such as cause in the example above) for an event are often expressed by a clause, which may be embedded\is{embedding} into the main clause or show greater or lesser independency. They may, however, also be encoded by derivational affixes \citep[192]{Lehmann1988}. Since this chapter is about complex \textit{clauses}, adverbial relations encoded by verbal morphology are not within its scope. 

Three general patterns\is{syndesis/asyndesis|(} can be observed in Paunaka,\is{verb|(} each of them showing a different degree of integration of the predicate encoding the adverbial relation into the main clause. First, the adverbial predicate can be part of a separate clause which is asyndetically or syndetically juxtaposed to the main clause and takes a balanced predicate.\is{finite verb} There is no doubt that we are dealing with two clauses in this case, a main clause and an adverbial clause (AC). I call these constructions asyndetic and syndetic subordination, respectively. Asyndetic subordination is the most covert means of expressing a subordinate relation; structurally, there is no difference to asyndetic \isi{coordination} (see \sectref{sec:AsyndeticCoordination}). Syndetic subordination offers the most elaborate means to encode which semantic type of adverbial relation is expressed by the AC by making use of different \isi{connective} words, e.g. \textit{porke} ‘because’ as in (\ref{ex:new23cause}) above.\is{syndesis/asyndesis|)} 

The second possibility to encode an adverbial relation is by a deranked predicate\is{deranked verb} that loses some of its verbal properties. It is more closely integrated into the main clause in this case, i.e. it is embedded. The highest degree of \isi{embedding} is accomplished when a \isi{preposition} is placed before the deranked verb thus showing its integration into the main clause as an oblique constituent.

A different kind of integration is found with purpose-of-motion predicates,\is{purpose|(} both marked (\isi{motion-cum-purpose construction}) and unmarked (\isi{serial verb construction}). The purpose verb can be argued to be a verbal goal \isi{argument} of the motion verb,\is{motion predicate} thus the whole construction is multi-verbal but monoclausal. Consequently, the question arises whether we can still speak of the purpose predicate and its arguments belonging to an adverbial \textit{clause}.\is{purpose|)} I leave this issue to be solved by others. In any case, this is the highest degree of integration we find in Paunaka, although unlike in deranking, the predicates maintain their verbal properties.\is{verb|)}

Of the semantic types of adverbial relations listed by \citet[156]{Cristofaro2003}, the following are expressed in complex adverbial clauses in Paunaka: temporal overlap, condition,\is{temporal overlap/condition} \isi{cause}, \isi{purpose}. Temporal anteriority and posteriority (i.e. before- and after-relations) are not commonly expressed by ACs. Instead of this, clauses marked for temporal sequence are used, which is analysed as a case of coordination (see \sectref{sec:SequentialCoordination}).\footnote{There are a few cases in the corpus of clauses expressing temporal anteriority and posteriority. They all make use of Spanish connectives,\is{connective}\is{borrowing} but since they occur very infrequently, I assume their use is not conventionalised (yet).} The different semantic types have different encoding possibilities, which are given in \tabref{table:AC-Types}. 

\begin{table}
\caption{Semantic types of adverbial clauses}

\begin{tabularx}{\textwidth}{lQQ}
\lsptoprule
Type of relation & Construction type & Comment \\
\midrule
Temporal overlap & asyndetic subordination & \\
& syndetic subordination & \\
& deranking & \\
Condition & asyndetic subordination & counterfactuals with \\
& syndetic subordination & frustrative marker\\
Cause & asyndetic subordination & \\
& syndetic subordination & \\
& deranking & with or without preposition\\
Purpose (non-motion) & asyndetic subordination & \\
 & syndetic subordination & \\
& deranking & with or without preposition \\
Apprehensional & syndetic subordination & \\
Purpose of motion & serial verb construction &\\
& motion-cum-purpose construction & \\
\lspbottomrule
\end{tabularx}

\label{table:AC-Types}
\end{table}


The remainder of this section is roughly organised by the degree of integration of the AC, evolving on a continuum from most independent to most dependent: all constructions in \sectref{sec:AdverbialJuxtaposition} have balanced verbs, the ones with deranked verbs are presented in \sectref{sec:SubordinateACs} and monoclausal multi-verb constructions are described in \sectref{sec:SVC_and_MCPC}. ACs are given in square brackets throughout this section. 

\subsection{Adverbial clauses with balanced verbs}\label{sec:AdverbialJuxtaposition}
\is{finite verb|(}
\is{juxtaposition|(}

The adverbial clauses (ACs) described in this section are connected to their main clauses at clause level. They are thus the ones that show least dependency on another clause. This is even truer for those clauses that are asyndetically juxtaposed: they could also appear completely independently. In clauses that are syndetically juxtaposed, there is a \isi{connective} to overtly show the kind of relationship towards the other clause, but apart from the connective itself, there is no dependency marking.\is{syndesis/asyndesis} ACs with balanced verbs thus have exactly the same syntactic structure as coordinate clauses.\is{coordination} Both clauses can only be distinguished by functional or pragmatic factors. The structure of a complex sentence consisting of a main clause and a juxtaposed adverbial clause is illustrated in \figref{fig:JuxtaposedACStructure}. 

\begin{figure}[!ht]


[[MC] [AC]]

[[MC] [co AC]]
\caption{Sentence structure of juxtaposed adverbial clauses}
\label{fig:JuxtaposedACStructure}

\end{figure}

%Adverbial clauses (ACs) express the circumstances under which the event of the main clause is realised \citep[155]{Cristofaro2003}. Different types of ACs can be distinguished by the kind of circumstances they express. 

The connectives\is{connective|(} that are used in syndetically juxtaposed ACs are summarised in \tabref{table:Connectives_AC}. In addition to the ones listed there, speakers sometimes also use other connectives from Spanish; however, given their low frequency, they are neglected here. ACs with a connective usually follow the main clause, with an exception being those introduced with \textit{kue} ‘if, when’, which can either precede or follow it. Purpose\is{purpose} clauses stick out, since the connective word used is a \isi{preposition}. Since this preposition can be combined with balanced verbs, we can assume that it is developing a parallel function as a connective in clause combining.

\begin{table}
\caption{Connectives in adverbial clauses}

\begin{tabularx}{\textwidth}{QQQ}
\lsptoprule
Connective & Translation & Clause type \\
\midrule
\textit{che(je)puine} & because & cause\\
\textit{kue} & if & condition\\
& when & temporal overlap\\
\textit{masa} & lest & apprehension\\
\textit{porke} & because & cause\\
\textit{tÿpi} & \textsc{obl} (preposition ‘for’) & purpose \\
\lspbottomrule
\end{tabularx}

\label{table:Connectives_AC}
\end{table}
\is{connective|)}


The remainder of this section is structured as follows: in \sectref{sec:AsyndeticSubordination} some clauses encoding adverbial relations without the use of a connective are presented. \sectref{sec:AC-kue} is about temporal and conditional clauses with the connective \textit{kue}, causal clauses with \textit{che(je)puine} and \textit{porke} are described in \sectref{sec:CauseConsequence}, purpose clauses with \textit{tÿpi} in \sectref{sec:PurposeClauses}, and finally apprehensional clauses with \textit{masa} are found in \sectref{sec:AprenhensionalClauses}. 

\subsubsection{Asyndetic subordination}\label{sec:AsyndeticSubordination}

In \sectref{sec:AsyndeticCoordination}, I have shown that clauses can be coordinated by simply juxtaposing them. The same holds for adverbial, i.e. semantically subordinate, relations. A clause that represents a \isi{cause}, \isi{purpose} or temporal precondition\is{temporal overlap/condition} for another, main clause can be juxtaposed to it, without any specific linking device. In some cases, TAME markers on the predicates of both clauses suggest a certain interpretation, but this is not always the case. We often find asyndetic juxtaposition if the events overlap in time or are separated by a short time interval, i.e. in “when-relations” \citep[159]{Cristofaro2003}. The AC provides the temporal setting for the main clause in this case \citep[cf.][155]{Cristofaro2003}. A few examples of temporal-overlap clauses follow.

(\ref{ex:cooked-down}) is from a description by Juana of how to use a clay pot for cooking. Irrealis RS is due to the explaining/habitual character of the text.

\ea\label{ex:cooked-down}
\begingl
\glpreamble \textup{[}taima\textup{]} petupaika apuke\\
\gla ti-a-ima pi-etupaika apuke\\
\glb 3i-\textsc{irr}-be.cooked 2\textsc{sg}-put.down.\textsc{irr} ground\\
\glft ‘when it (the food) is done, we can put it down (from the fire)’
\endgl
\trailingcitation{[jxx-d110923l-3.5]}
\xe

Another sentence that expresses a when-relation without any overt marking is (\ref{ex:chicken-steal}), in which Juana talks about her brother José and the problems he has because of living remote from the village of Santa Rita. 

\ea\label{ex:chicken-steal}
\begingl
\glpreamble \textup{[}pasaupu uneku\textup{]} tiyununube chumeikunube chipeu eka takÿra\\
\gla pasau-pu uneku ti-yunu-nube chÿ-umeiku-nube chi-peu eka takÿra\\
\glb pass-\textsc{dloc} town 3i-go-\textsc{pl} 3-steal-\textsc{pl} 3-animal \textsc{dem}a chicken\\
\glft ‘when he goes to the town, they (people from Santa Rita) go and steal his chicken’
\endgl
\trailingcitation{[jxx-p120515l-2.254]}
\xe

(\ref{ex:TempAdv-kue-2}) was offered by María S. as an answer to the question whether she smoked. Like in the examples above, no connective is involved and the relation between the clauses is only understood from the context.


\ea\label{ex:TempAdv-kue-2}
\begingl
\glpreamble nijibÿku \textup{[}niyunu asaneti\textup{]}\\
\gla ni-jibÿku ni-yunu asaneti\\
\glb 1\textsc{sg}-smoke 1\textsc{sg}-go field\\
\glft ‘I smoke, when I go to the field’
\endgl
\trailingcitation{[rxx-e120511l.390]}%non-el.
\xe

The \isi{incompletive} marker is often used if the events expressed by the clauses have \isi{past reference}. In that case \textit{-kuÿ} indicates that a certain state held at that time but does not hold anymore, and thus it sets the ground for the clause that follows. The clauses are otherwise completely unmarked for their relation to each other and \textit{-kuÿ} can also occur in contexts that do not involve subordination. In (\ref{ex:drank-before}), María C. makes a statement about her consumption of alcohol in former times (she was known for enjoying life, when she was younger): the state of being a girl or young woman holds in relation to the predicate of the main clause.

\ea\label{ex:drank-before}
\begingl
\glpreamble \textup{[}bane pimiyakuÿne\textup{]} neu\\
\gla bane pimiya-kuÿ-ne nÿ-eu\\
\glb \textsc{rem} girl-\textsc{incmp}-1\textsc{sg} 1\textsc{sg}-drink\\
\glft ‘long ago, when I was still a young woman, I drank’
\endgl
\trailingcitation{[cux-c120414ls-1.031]}
\xe

A similar possibility is offered by \isi{continuous} marking on one of the verbs. In this case, one event is marked as ongoing at the time another event, which has to be telic\is{telicity} and punctual, occurs. There are not many examples in the corpus of the use of the \isi{continuous} marker in such a way, but one is given in (\ref{ex:CONT-OV}), where in Juana’s story the fugitive criminal has just arrived at his wife’s house to eat and is then arrested by the police (or soldiers).

\ea\label{ex:CONT-OV}
\begingl
\glpreamble tinikukuikuji \textup{[}kapunukunube suntabunube\textup{]}\\
\gla ti-niku-kuiku-ji kapunu-uku-nube suntabu-nube\\
\glb 3i-eat-\textsc{cont}-\textsc{rprt} come-\textsc{add}-\textsc{pl} soldier-\textsc{pl}\\
\glft ‘he was eating, it is said, when the soldiers came, too’ (or: ‘while he was eating, the soldiers came, too’)
\endgl
\trailingcitation{[jxx-p120430l-2.151]}
\xe

Not only temporal relations can be expressed without overt linking. The following examples show causal\is{cause} (\ref{ex:cry-fear}) and purposive\is{purpose} (\ref{ex:coord-purp}) clauses that are asyndetically juxtaposed to their main clauses.

In (\ref{ex:cry-fear}), the second clause provides the reason for the statement made in the main clause, the boy crying. This sentence was produced by Miguel, when looking at the pictures of the \isi{frog story} and it refers to the picture on which the dog jumps against the tree and the boy holds his nose being bitten by a small rodent.

\ea\label{ex:cry-fear}
\begingl
\glpreamble entonses tiyuiyukubutu eka aitubuchepÿimÿnÿ \textup{[}tipiku\textup{]}\\
\gla entonses ti-iyuyuiku-bu-tu eka aitubuchepÿi-mÿnÿ ti-piku\\
\glb thus 3i-cry-\textsc{mid}-\textsc{iam} \textsc{dem}a boy-\textsc{dim} 3i-be.afraid\\
\glft ‘so the boy is crying, (because) he is afraid’
\endgl
\trailingcitation{[mox-a110920l-2.083]}
\xe


The following example, (\ref{ex:coord-purp}), consists of several clauses. The part that I want to discuss here is underlined. These are the two clauses that express “look for wood” and “make a small board” (which are both in a complement relation to an utterance predicate, but this is of no concern at the moment). The second of these clauses encodes the \isi{purpose} for the first one without being overtly marked for this relation. The example comes from Miguel’s description of how he learned to read, write and calculate and it cites his teacher who gave Miguel instructions how to obtain a board to write on.

\ea\label{ex:coord-purp}
\begingl
\glpreamble “pikechuchÿji echÿu pia \underline{tisemaika echÿu yÿkÿke \textup{[}tana taurapachumÿnÿ\textup{]}} i nebu pisuikia”, tikechu\\
\gla pi-kechu-chÿ-ji echÿu pi-a ti-semaika echÿu yÿkÿke ti-ana taurapachu-mÿnÿ i nebu pi-suik-i-a ti-kechu\\
\glb 2\textsc{sg}-say-3-\textsc{imp} \textsc{dem}b 2\textsc{sg}-father 3i-search.\textsc{irr} \textsc{dem}b wood 3i-make.\textsc{irr} board-\textsc{dim} and 3\textsc{obl.top.prn} 2\textsc{sg}-write-\textsc{subord}-\textsc{irr} 3i-say\\
\glft ‘“tell your father to look for wood to make a small board and on that one you can write”, he said’
\endgl
\trailingcitation{[mxx-p181027l-1.022]}
\xe

I want to emphasise again that the adverbial nature of linking between the clauses in the examples of this section is based on purely pragmatic factors and thus the translations into English with a connective may be a bit misleading. It would certainly be possible to translate most of the examples into English by using coordinated clauses, too. In the following sections, I will present examples in which the subordinate relationship of one event towards another is expressed overtly by using a connective that tells us about the exact nature of the relationship. %or by embedding of the subordinate predicate.

%aa, binuku echÿu merÿpune naka kuina tapitachÿ, ponemos la hoja de plátano para que no prenda, mxx-e120415ls.067


%nitupunubu nauku metu chima- chimakunubetu = when I arrived there, they had already buried him, jxx-p120430l-2.461
%nÿbÿsÿumÿnÿ naka nitibupu naka nichechapÿiyae, cux-c120410ls.015, vine aquí me siento aquí ande mi hijo
%tibururuka ÿne pijÿuka nÿkÿiki jxx-d110923l-3.2-3
%tibururukatu ÿne i pijuka, jxx-d110923l-3.4

\is{connective|(}
\subsubsection{Temporal overlap and conditional clauses}\label{sec:AC-kue}
\is{temporal overlap/condition|(}


Like many other languages \citep[cf.][161]{Cristofaro2003}, Paunaka encodes conditional and temporal-overlap relations by using the same connective, \textit{kue} ‘if, when’. A certain possibility to distinguish between temporal and conditional is offered by RS\is{reality status} marking of both clauses. According to \citet[160]{Cristofaro2003}, in conditional linking, both events are presented as non-factual. Thus we can deduce that conversely, if both clauses have \isi{realis} predicates, we are dealing with a temporal relation. However, this is rare in Paunaka, and there are only a few examples of all-realis temporal sentences. All of them express non-singular events, i.e. repeated or repeatable. Three examples with realis predicates in both clauses shall be given here, (\ref{ex:TempAdv-kue-1})–(\ref{ex:pot-cook}). 

(\ref{ex:TempAdv-kue-1}) encodes a \isi{habitual} event with present time reference. It is an answer to the question whether María S. smoked. Actually, it was a repetition of (\ref{ex:TempAdv-kue-2}) above, where the two predicates were asyndetically juxtaposed, and was produced to confirm what she had been saying before.

\ea\label{ex:TempAdv-kue-1}
\begingl
\glpreamble nijibÿku \textup{[}kue niyunu asaneti\textup{]}\\
\gla ni-jibÿku kue ni-yunu asaneti\\
\glb 1\textsc{sg}-smoke if 1\textsc{sg}-go field\\
\glft ‘I smoke, when I go to the field’
\endgl
\trailingcitation{[rxx-e120511l.391]}%non-el.
\xe

The next example also encodes repeated events. It was produced by María S. when we were sitting in her yard and suddenly some piglets turned up grunting and wanted to suckle.

\ea\label{ex:real-kue-1}
\begingl
\glpreamble \textup{[}kue tisabaikÿupunujane\textup{]} tibiyujaneutu\\
\gla kue ti-sabai-kÿupunu-jane ti-biyu-jane-u-tu\\
\glb if 3i-shout-\textsc{am.conc.cis}-\textsc{distr} 3i-be.thirsty-\textsc{distr}-\textsc{real}-\textsc{iam}\\
\glft ‘when they come grunting, they are thirsty’
\endgl
\trailingcitation{[rmx-e150922l.157]}
\xe

Finally, (\ref{ex:pot-cook}) comes from Juana telling me about the production of a pot in former times. Actually, it is rather exceptional that she uses realis predicates here, because past habitual\is{habitual|(} events are often encoded by use of irrealis. However, there are some inconsistencies in the relation of past habitual events and irrealis encoding anyway (see \sectref{sec:RS_TemporalReference}) and in this specific case, the surrounding sentences also have realis RS.\is{habitual|)} Note that in addition to having \textit{kue} in the antecedent clause, the consequent clause takes the coordinating \isi{sequential} connective \textit{te} (see \sectref{sec:SequentialCoordination}).


\ea\label{ex:pot-cook}
\begingl
\glpreamble \textup{[}kue timÿrakimÿnÿtu\textup{]} te beimachu\\
\gla kue ti-mÿra-ki-mÿnÿ-tu te bi-eimachu\\
\glb if 3i-dry-\textsc{clf:}spherical-\textsc{dim}-\textsc{iam} \textsc{seq} 1\textsc{pl}-cook.until.done\\
\glft ‘when it had dried, then we fired it’
\endgl
\trailingcitation{[jxx-d110923l-2.21]}
\xe

It is more common to find combinations of clauses, in which the main clause predicate has \isi{realis} RS and the predicate of the clause introduced by \textit{kue} has \isi{irrealis} RS, thus only the conditional clause is presented as non-factual, see (\ref{ex:kuekue})–(\ref{ex:money-rice}) below. This goes against the generalisation by \citet[160--161]{Cristofaro2003}. Consequently, if  both clauses have \isi{irrealis} predicates, this is due to reasons that have nothing to do with the conditional relation, e.g. future reference or negation. The following three examples, (\ref{ex:cond-1})–(\ref{ex:irr-kue-1}), have irrealis predicates in both clauses, the antecedent and the consequent.

The conditional clause in (\ref{ex:cond-1}) describes what Juana’s daughter would do in the improbable but possible case that her mother visited her in Spain. The event in the consequent clause has a (possible) singular occurrence – contrary to the examples presented above that represent repeated actions – and it has future reference. All predicates thus have irrealis RS in this case.

\ea\label{ex:cond-1}
\begingl
\glpreamble \textup{[}kue pibÿsÿa, mimi,\textup{]} nipabentecha nubiu, te biyunupunatu nauku\\
\gla kue pi-bÿsÿa mimi ni-pabentecha nÿ-ubiu te bi-yunupuna-tu nauku\\
\glb if 2\textsc{sg}-come.\textsc{irr} mum 1\textsc{sg}-sell.\textsc{irr} 1\textsc{sg}-house \textsc{seq} 1\textsc{pl}-go.back.\textsc{irr}-\textsc{iam} there\\
\glft ‘if you come, mum, I will sell my house, and then we go back there’
\endgl
\trailingcitation{[jxx-p110923l-1.432]}
\xe

(\ref{ex:cond-2}) is from the account by Miguel about how he learned to read and to calculate. He did not like the maths lessons in school, thus he was told that this could have negative consequences. (He finally learned to calculate, but only later, when he was a young man already.) Irrealis is due to negation here.

\ea\label{ex:cond-2}
\begingl
\glpreamble \textup{[}kue kuina pichupa echÿu matematika\textup{]}, kuina pueroina pana echÿu kuenta\\
\gla kue kuina pi-chupa echÿu matematika kuina puero-ina pi-ana echÿu kuenta\\
\glb if \textsc{neg} 2\textsc{sg}-know.\textsc{irr} \textsc{dem}b mathematics \textsc{neg} can-\textsc{irr} 2\textsc{sg}-make.\textsc{irr} \textsc{dem}b bill\\
\glft ‘if you don’t know mathematics, you cannot make bills’
\endgl
\trailingcitation{[mxx-p181027l-1.106]}
\xe

The next example was produced by María S. with the specific purpose to be presented to other people in this work. This is why she uses third person markers in reference to herself.  A singular future event is expressed here, the finishing of one specific hammock María S. was weaving at that time. Interpretation must be temporal here. It cannot be conditional, since there is no doubt that she would actually finish her hammock.

\ea\label{ex:irr-kue-1}
\begingl
\glpreamble \textup{[}kue cheanekatu yumaji\textup{]} chipabentecha\\
\gla kue chÿ-eaneka-tu yumaji chi-pabentecha\\
\glb if 3-finish.\textsc{irr}-\textsc{iam} hammock 3-sell.\textsc{irr}\\
\glft ‘when she has finished the hammock, she will sell it’
\endgl
\trailingcitation{[rxx-e181022le]}
\xe

The following examples all have irrealis in the antecedent clause and realis in the consequent clause. Like (\ref{ex:TempAdv-kue-1})–(\ref{ex:pot-cook}) above, the events encoded in the sentence are non-singular, repeated/repeatable. I am not entirely sure what sets these examples apart from the ones with all-realis predicates.\is{reality status} Possibly, the realisation of the event in the antecedent is less certain than in the examples above.

(\ref{ex:kuekue}) was produced by Juana telling me about a spirit that takes away the washed clothes by blowing them away. With the sentence, Juana sets the scene, describing the place where they wash, a hollow in a rock, which can only be used for this specific purpose after rainfalls.

\ea\label{ex:kuekue}
\begingl
\glpreamble i tejapÿku ÿne \textup{[}kue tikeba\textup{]} tejapÿku ÿne\\
\gla i ti-japÿku ÿne kue ti-keba ti-japÿku ÿne\\
\glb and 3i-fill water if 3i-rain.\textsc{irr} 3i-fill water\\
\glft ‘the water fills it there, when/if it rains, the water fills it’
\endgl
\trailingcitation{[jxx-p151020l-2]}
\xe

In (\ref{ex:south-hurt}), Juana speaks about pain in her breast. 

\ea\label{ex:south-hurt}
\begingl
\glpreamble \textup{[}kue kapunuina tisÿeipu\textup{]} max tikuti\\
\gla kue kapunu-ina tisÿeipu max ti-kuti\\
\glb if come-\textsc{irr.nv} south.wind more 3i-hurt\\
\glft ‘when/if south wind comes, it hurts more’
\endgl
\trailingcitation{[jxx-p120430l-1.326]}
\xe

The following example comes from a little tale by Miguel that explains why ants are happy and trees are sad when a boy is born.

\ea\label{ex:happy-born}
\begingl
\glpreamble eka kusiyÿ tiyayaumiji \textup{[}kue kakuina apuke eka aitubuchepÿimÿnÿ\textup{]}\\
\gla eka kusiyÿ ti-yayaumi-ji kue kaku-ina apuke eka aitubuchepÿi-mÿnÿ\\
\glb \textsc{dem}a ant 3i-be.happy-\textsc{rprt} if exist-\textsc{irr.nv} ground \textsc{dem}a boy-\textsc{dim}\\
\glft ‘the ant is happy, it is said, when/if a boy is born’
\endgl
\trailingcitation{[mxx-n120423lsf-X.12]}
\xe

(\ref{ex:money-rice}) was produced by María S. in a correction session.

\ea\label{ex:money-rice}
\begingl
\glpreamble \textup{[}kue kakuina nÿtÿmuane\textup{]} niyÿseikumÿnÿ arusu-muke\\
\gla kue kaku-ina nÿ-tÿmua-ne ni-yÿseiku-mÿnÿ arusu-muke\\
\glb if exist-\textsc{irr.nv} 1\textsc{sg}-money-\textsc{possd} 1\textsc{sg}-buy-\textsc{dim} rice-seed\\
\glft ‘when/if I have money, I buy peeled rice’
\endgl
\trailingcitation{[rxx-e121128s-3.28]}
\xe

Finally, there are also counterfactual conditional clauses.\is{counterfactuality|(} All predicates have \isi{irrealis} RS due to the non-factuality of these clauses. In addition, the frustrative\is{frustrative|(} or a related marker is attached to at least one of the predicates, i.e. it can show up in the antecedent clause, in the consequent clause or in both clauses. There are only a few counterfactual conditional sentences in the corpus.\footnote{There are more counterfactual clauses, but they are not necessarily linked to another clause in the specific way conditional sentences are formed, i.e. with an antecedent and a consequent clause.} 

In (\ref{ex:counter-1}), the predicate in the consequent clause has the frustrative marker. The sentence comes from María S. telling the story about the two men who meet the devil in order to explain me why one of the man did not want to follow the devil in the end of the story, when the devil has taken his friend with him.

\ea\label{ex:counter-1}
\begingl
\glpreamble aa tenikukaini \textup{[}kue cheibanea\textup{]}\\
\gla aa ti-niku-uka-ini kue chÿ-eibanea\\
\glb \textsc{intj} 3i-eat-\textsc{add.irr}-\textsc{frust} if 3-pursue.\textsc{irr}\\
\glft ‘ah, he would eat him as well if he pursued him’
\endgl
\trailingcitation{[rxx-n120511l-2.63-64]}
\xe

In (\ref{ex:chicha-invite}), the antecedent bears the \isi{optative} marker (which is composed of the intensifier \textit{-yu} and the frustrative \textit{-ini}, see \sectref{sec:FRUST-Optative}). This sentence was elicited from Miguel.

\ea\label{ex:chicha-invite}
\begingl
\glpreamble \textup{[}kue nanayuini pario aumue ukuine\textup{]} tanÿmakena nekichapi \\
\gla kue nÿ-ana-yuini pario aumue ukuine tanÿma-kena nÿ-ekicha-pi \\
\glb if 1\textsc{sg}-make.\textsc{irr}-\textsc{opt}1 some chicha yesterday now-\textsc{uncert} 1\textsc{sg}-invite.\textsc{irr}-2\textsc{sg}\\
\glft ‘if I had only made some chicha yesterday, I could invite you now’
\endgl
\trailingcitation{[mxx-e160811sd.438]}
\xe
%optative -yuini? If I had only made...


Finally in (\ref{ex:buy-here}), we find the frustrative marker on the predicates of both clauses. I requested this sentence from María S. as a translation of a corresponding Spanish one. I wanted to tell her that I would buy her hammock (instead of a hammock made by another woman in Santa Rita) if I stayed long enough for her to be able to finish it. Since she had only begun weaving, it was clear that I would travel back to Germany before she could finish it.

\ea\label{ex:buy-here}
\begingl
\glpreamble niyÿseikaini \textup{[}kue nÿtiukukuineini\textup{]}\\
\gla ni-yÿseika-ini kue nÿti-uku-kuÿ-ina-ini\\
\glb 1\textsc{sg}-buy.\textsc{irr}-\textsc{frust} if 1\textsc{sg.prn}-\textsc{prn.loc}-\textsc{incmp}-\textsc{irr.nv}-\textsc{frust}\\
\glft ‘I would buy it if I were still here’
\endgl
\trailingcitation{[rxx-e181022le]}
\xe
\is{frustrative|)}
\is{counterfactuality|)}
\is{temporal overlap/condition|)}

\subsubsection{Causal clauses}\label{sec:CauseConsequence}\is{cause|(}

Sometimes two events are in a relation of cause and consequence, thus one event is a prerequisite to the other. This can be expressed either by marking one event as a cause of another one (in a causal clause) or by marking one event as a consequence of the other (in a consecutive clause). Following \citet[38]{Cristofaro2003}, the latter is defined as a case of coordination and described in \sectref{sec:ConsecutiveCoordination}. This section is about causal clauses only.

The causal clause can be introduced by the connective \textit{che(je)puine} ‘because’. This connective is used by Miguel and María S. and it also appears in the recordings by Riester. The verb of a causal clause introduced by \textit{che(je)puine} is always balanced and its RS is not restricted.\is{reality status} The causal clause always follows the main clause; however, it can also introduce a separate \isi{intonation} unit, in which case the cause may be less tightly connected to the previous clause, but rather to the larger context.


(\ref{ex:because-1}) is an excerpt from the description of the \isi{frog story} told by Miguel. He describes the picture on which the dog and the boy lean in the window, with the dog’s head being stuck in the glass of the frog.

\ea\label{ex:because-1}
\begingl
\glpreamble tipikutu eka aitubuchepÿimÿnÿ \textup{[}chejepuinekena tubÿu chichÿti eka chipeu kabe naka\textup{]}\\
\gla ti-piku-tu eka aitubuchepÿi-mÿnÿ chejepuine-kena ti-ubÿu chi-chÿti eka chi-peu kabe naka\\
\glb 3i-be.afraid-\textsc{iam} \textsc{dem}a boy-\textsc{dim} because-\textsc{uncert} 3i-get.stuck 3-head \textsc{dem}a 3-animal dog here\\
\glft ‘the boy is afraid, maybe because the head of his dog is stuck here’
\endgl
\trailingcitation{[mox-a110920l-2.056-057]}
\xe

In (\ref{ex:because-2}), María S. answers my question whether she had a friend when she was a child. She negates and offers the reason: she could not have a friend, because her family lived remote, quite far away from other people, so there was little contact to other people.

\ea\label{ex:because-2}
\begingl
\glpreamble kuina niamigane \textup{[}chepuine tÿbane bubiu, nauku chukuyae Kose\textup{]}\\
\gla kuina ni-amiga-ne chepuine ti-ÿbane bi-ubiu nauku chi-chuku-yae Kose\\
\glb \textsc{neg} 1\textsc{sg}-friend-\textsc{possd} because 3i-be.far 1\textsc{pl}-house there 3-side-\textsc{loc} José\\
\glft ‘I didn’t have a friend, because our house was remote, there close to José(’s house)’
\endgl
\trailingcitation{[rxx-p181101l-2.115]}
\xe

(\ref{ex:because-3}) is from the same context as (\ref{ex:coord-purp}) above. Miguel had told me that when he was a child, people in Concepción would not sell paper to indigenous people, so in order to be able to write things down in school, pupils in \isi{Altavista} used wooden boards. Miguel evaluates this as a good thing, because in contrast to paper, a wooden board is easily re-usable.

\ea\label{ex:because-3}
\begingl
\glpreamble bueno pero michaubiyu nÿtÿpi echÿu taurapechumÿnÿ \textup{[}chejepuine eka kuina tibukapu\textup{]}\\
\gla bueno pero micha-u-bi-yu nÿ-tÿpi echÿu taurapechu-mÿnÿ chejepuine eka kuina ti-buka-pu\\
\glb well but good-?-1\textsc{pl}-\textsc{ints} 1\textsc{sg}-\textsc{obl} \textsc{dem}b board-\textsc{dim} because \textsc{dem}a \textsc{neg} 3i-finish.\textsc{irr}-\textsc{mid}\\
\glft ‘well, but for me the small board was good, because that one doesn’t finish’
\endgl
\trailingcitation{[mxx-p181027l-1.032-033]}
\xe

(\ref{ex:because-4}) seems to be an almost tautological statement by María S. and refers to the lack of knowledge about the dates of feast days in former times, when the families used to live more dispersed. The reason for this was the incapability of her family, i.e. they could not know which day was a feast day, because they did not have the means to recognise it, which could be a calendar or simply social interaction with other people if they had lived in a village.

\ea\label{ex:because-4}
\begingl
\glpreamble kuina bichupa \textup{[}chepuine kuina baichunabane\textup{]}\\
\gla kuina bi-chupa chepuine kuina bi-a-ichuna-bane\\
\glb \textsc{neg} 1\textsc{pl}-know.\textsc{irr} because \textsc{neg} 1\textsc{pl}-\textsc{irr}-be.capable-\textsc{rem}\\
\glft ‘we didn’t know it (which day was a feast day), because we were not capable in the old times’
\endgl
\trailingcitation{[rxx-p181101l-2.016]}
\xe

Finally, (\ref{ex:because-5}) was elicited from María S. in the same context as (\ref{ex:buy-here}) above and it refers to me not being able to buy a hammock from her, because I would leave Bolivia before she could finish it.
 
 \ea\label{ex:because-5}
\begingl
\glpreamble kuina puero niyÿseika \textup{[}chepuine niyunupunupuna nepukie\textup{]}\\
\gla kuina puero ni-yÿseika chepuine ni-yunupunu-puna nÿ-epukie\\
\glb \textsc{neg} can 1\textsc{sg}-buy.\textsc{irr} because 1\textsc{sg}-go.back-\textsc{am.prior.irr} 1\textsc{sg}-homeland\\
\glft ‘I can’t buy it, because I will go back to my country’
\endgl
\trailingcitation{[rxx-e181022le]}
\xe

Juana, Clara and María C. do not seem to use \textit{che(je)puine}. Instead, they resort to the Spanish causal conjunction \textit{porke} (Spanish: \textit{porque} ‘because’). The latter is also used by Miguel occasionally. (\ref{ex:porque-1:0903}) was produced by Juana, when I asked her which of the frogs on the picture in the end of \isi{frog story} she liked best. She chose one and gave an explanation why she did not prefer the other:

\ea\label{ex:porque-1:0903}
\begingl
\glpreamble eka punachÿ kuina pueroina micha \textup{[}porke mutemenayu i max chepitÿjiku chijabu\textup{]}\\
\gla eka punachÿ kuina puero-ina micha porke mutemena-yu i max chepitÿjiku chi-jabu\\
\glb \textsc{dem}a other \textsc{neg} can-\textsc{irr.nv} good because big-\textsc{ints} and more small 3-leg\\
\glft ‘the other one cannot (jump) well, because it is very big and its legs are shorter’
\endgl
\trailingcitation{[jxx-a120516l-a.529-531]}
\xe

(\ref{ex:porque-2}) comes from Miguel who stated that he was happy that the taxi driver who had brought us to San Miguelito was sitting with us and joining the talk.

\ea\label{ex:porque-2}
\begingl
\glpreamble chikuye pero nÿtiuku niyayaumi \textup{[}porke kaku eka bipiji naka\textup{]}\\
\gla chi-kuye pero nÿti-uku ni-yayaumi porke kaku eka bi-piji naka\\
\glb 3-be.like.this but 1\textsc{sg.prn}-\textsc{add} 1\textsc{sg}-be.happy because exist \textsc{dem}a 1\textsc{pl}-sibling here\\
\glft ‘it is like this, but I am also happy, because our brother is here’
\endgl
\trailingcitation{[mty-p110906l.208-209]}
\xe

%chejepuine eka echÿu aitubuchepÿi tijÿkatu tiyunaji tebitaka chisaneina entonses, mxx-n120423lsf-X.27-28

Finally, the last example in this section comes from Clara, who was chatting with María C. and told her about her plans to teach her daughters some Paunaka, because they were interested in learning it.

\ea\label{ex:porque-3}
\begingl
\glpreamble nisachu nimeisumeikanube nijinepÿinube \textup{[}porke tisachu tichujikanube\textup{]}\\
\gla ni-sachu ni-meisumeika-nube ni-jinepÿi-nube porke ti-sachu ti-chujika-nube\\
\glb 1\textsc{sg}-want 1\textsc{sg}-teach.\textsc{irr}-\textsc{pl} 1\textsc{sg}-daughter-\textsc{pl} because 3i-want 3i-speak.\textsc{irr}-\textsc{pl}\\
\glft ‘I want to teach it to my daughters, because they want to speak it’
\endgl
\trailingcitation{[cux-c120414ls-2.323-324]}
\xe
\is{cause|)}

\subsubsection{Purpose clauses}\label{sec:PurposeClauses}\is{purpose|(}
\is{general oblique|(}

In purpose clause linking, one event “(the main one) is performed with the goal of obtaining the realization of another one (the dependent one)” \citep[157]{Cristofaro2003}. Purpose clauses always follow the main clause. They can be introduced by \textit{tÿpi}, which is not strictly speaking a connective, but a \isi{preposition} that is used with obliques (see \sectref{sec:adp-tÿpi}). However, the predicate following \textit{tÿpi} in purpose clauses can be balanced, and thus should be described in this section. Purpose clauses with \textit{tÿpi} and balanced predicates are typical for Juana, but occasionally also found with other speakers. We can thus state that the preposition is developing a parallel function as connective. This parallels what we find with Spanish \textit{para} ‘for’, although in Spanish purpose clauses, the verb has to be an infinitive. In Paunaka, purpose clauses formed with a deranked verb can also additionally be marked by \textit{tÿpi}. Those clauses are described in \sectref{sec:EmbeddedAC_adp}.

The predicate of a purpose clause usually has \isi{irrealis} RS. It is not necessary that both clauses have the same subject, but there needs to be some involvement “at least in that there is an element of will on [the main clause performer’s] part towards such realization” \citep[157]{Cristofaro2003}. The subjects of the main and the purpose verb may thus be the same or different. Actually, in most examples I found, they are different, see (\ref{ex:purpose-1})–(\ref{ex:Purp-5}).

(\ref{ex:purpose-1}) comes from Juana’s account about how they made the reservoir in Santa Rita. The men prepared the ground by felling trees, and the women cooked for them.

\ea\label{ex:purpose-1}
\begingl
\glpreamble i biti metu biyÿtikatu nÿkÿiki \textup{[}tÿpi chinikanube\textup{]}\\
\gla i biti metu bi-yÿtika-tu nÿkÿiki tÿpi chi-nika-nube\\
\glb and 1\textsc{pl.prn} already 1\textsc{pl}-set.on.fire-\textsc{iam} pot \textsc{obl} 3-eat.\textsc{irr}-\textsc{pl}\\
\glft ‘and we already set the pots onto fire for them to eat it’
\endgl
\trailingcitation{[jxx-p120515l-2.185-186]}
\xe

From the same passage is (\ref{ex:purpose-2}), in which Juana describes that the men were supplied with chicha.

\ea\label{ex:purpose-2}
\begingl
\glpreamble i eka kaku chijinepÿinube chiyenu te, tumunube aumue \textup{[}tÿpi teanube nauku\textup{]}\\
\gla i eka kaku chi-jinepÿi-nube chi-yenu te ti-umu-nube aumue tÿpi ti-ea-nube nauku\\
\glb and \textsc{dem}a exist 3-daughter-\textsc{pl} 3-wife \textsc{seq} 3i-take-\textsc{pl} chicha \textsc{seq} 3i-drink.\textsc{irr}-\textsc{pl} there\\
\glft ‘and the ones who had daughters or a wife, they brought them chicha to drink there’
\endgl
\trailingcitation{[jxx-p120515l-2.182-184]}
\xe

In (\ref{ex:Purp-5}), Juana explains that she will give her daughter a post-pregnancy treatment: tie her belly to shrink it.

\ea\label{ex:Purp-5}
\begingl
\glpreamble nÿrÿtÿkabÿti chikÿ nijinepÿi \textup{[}tÿpi chirataka micha\textup{]}\\
\gla nÿ-rÿtÿka-bÿti chi-kÿ ni-jinepÿi tÿpi chi-rataka micha\\
\glb 1\textsc{sg}-tie-\textsc{prsp} 3-\textsc{clf:}bounded 1\textsc{sg}-daughter \textsc{obl} 3-press.\textsc{irr} good\\
\glft ‘I’m just going to tie my daughter’s belly so that it presses her well’
\endgl
\trailingcitation{[jxx-e120430l-2.1-2]}
\xe

In (\ref{ex:purpose-3}) the subjects are identical. The jar is the subject of the copula \textit{kaku} as well as of the middle verb \textit{tetukapu} ‘it is filled’. The example also comes from Juana, when she told me about the beautiful pottery they have in Cotoca, a city close to Santa Cruz and a popular destination for excursions.

\ea\label{ex:purpose-3}
\begingl
\glpreamble kaku yÿpijanemÿnÿ michananaji \textup{[}tÿpi tetukapu ÿne\textup{]}, tisÿeimumÿnÿ\\
\gla kaku yÿpi-jane-mÿnÿ michana-na-ji tÿpi ti-etuka-pu ÿne ti-sÿei-umu-mÿnÿ\\
\glb exist jar-\textsc{distr}-\textsc{dim} nice-\textsc{rep}-\textsc{col} \textsc{obl} 3i-put.\textsc{irr}-\textsc{mid} water 3i-be.cold-\textsc{clf:}liquid-\textsc{dim}\\
\glft ‘there are beautiful jars for being filled with water, the water stays cold’
\endgl
\trailingcitation{[jxx-p120430l-2.594-596]}
\xe


(\ref{ex:Purp-4}) is a negative purpose clause with \textit{tÿpi} produced by Juana describing the preparation of a medicine against cough to my colleague Lena.

\ea\label{ex:Purp-4}
\begingl
\glpreamble bea \textup{[}tÿpi kuina bijÿchikapu yutina\textup{]}\\
\gla bi-ea tÿpi kuina bi-jÿchikapu yuti-ina\\
\glb 1\textsc{sg}-drink.\textsc{irr} \textsc{obl} \textsc{neg} 1\textsc{pl}-cough.\textsc{irr} night-\textsc{irr.nv}\\
\glft ‘we drink it so that we won’t cough at night’
\endgl
\trailingcitation{[jxx-e191021e-2]}
\xe


Combination of \textit{tÿpi} with a negative clause like in the previous example is exceptional. Usually, \isi{apprehensional} clauses take a different connective, \textit{masa} ‘lest’. Apprehensional clauses are much rarer than purpose clauses. They are described in the following section. In addition, motion predicates\is{motion predicate} are never combined with purpose clauses introduced by \textit{tÿpi}. Speakers either use a serial verb or a motion-cum-purpose construction to encode these notions (see \sectref{sec:SVC_and_MCPC}). 


%
%Remarkable, in this regard, is the following example, in which all types of purpose marking show up
%
%\ea\label{ex:}
%\begingl
%\glpreamble biyunupuna nauku chubiunubeye echÿu piparentenenube [kapupunubeina sinkonubechina jentenube] [ayaraunubeina bitÿpi] [eka bumia eka bakajane] [tÿpi chinikanube nauku]\\
%\gla \\
%\glb \\
%\glft \\
%\endgl
%\trailingcitation{[]}
%\xe
\is{general oblique|)}
\is{purpose|)}

\subsubsection{Apprehensional clauses}\label{sec:AprenhensionalClauses}\is{apprehensional|(}

Like purpose clauses, apprehensional clauses also encode a kind of purpose, but negatively,\is{negation} i.e. the event expressed in the main clause is carried out to prevent the one in the adverbial clause. This\is{avertive|(} type of clauses has also been called “avertive clauses” \citep[e.g.][]{SchmidtkeBode2009}, but in following \citet[]{Kuteva2019} I reserve the term “avertive” for a modality marker with the meaning ‘almost’ (see \sectref{sec:FRUST-Avertive}) and use the term “apprehensional” for those clauses describing “an undesirable verb situation which is to be avoided” \citep[863]{Kuteva2019}.\is{avertive|)} Apprehensional clauses are introduced by \textit{masa} ‘lest’ and their predicates always have \isi{irrealis} RS. Apprehensional clauses are rare in the corpus, but three examples shall be given here nonetheless.

The context of (\ref{ex:masaApp-1}) is as follows: together with Juana, we had just found María S., who was making adobe bricks at a place in the woods. This place is a bit remote from the village, and we had been walking through the shrubbery before we met her there (when walking back to the village, María S. showed us a better way). María S. is making a joke about herself deliberately hiding from us in this place.\footnote{Remarkable in this example is that \textit{naka} ‘here’ apparently does not refer to the place of her current position, but to the village, where she lives. In closely related \isi{Baure}, the word \textit{ne’} ‘here’ can also refer to places quite far away from the current position of the speaker as long as she feels familiar with this place \citep[252--254]{Admiraal2016}. This possibly also holds for Paunaka. Alternatively, \textit{tukiu} refers to another source which is not expressed and \textit{naka} to the momentary location of the speaker.}

\ea\label{ex:masaApp-1}
\begingl
\glpreamble nÿjechika tukiu naka \textup{[}masa etupanÿ\textup{]}\\
\gla nÿ-jechika tukiu naka masa e-tupa-nÿ\\
\glb 1\textsc{sg}-hide.\textsc{irr} from here lest 2\textsc{pl}-find.\textsc{irr}-1\textsc{sg}\\
\glft ‘I wanted to hide from here lest you find me’
\endgl
\trailingcitation{[jrx-c151001fls-8.12]}
\xe

In (\ref{ex:masaApp-2}), Juana describes what her grandmother did to prevent her grandfather being enchanted by the female spirit of the water, at night, when they had physically already escaped her. 

\ea\label{ex:masaApp-2}
\begingl
\glpreamble chakiyeku chibuÿye \textup{[}masa chabikÿka\textup{]}\\
\gla chÿ-akiyeku chi-buÿ-yae masa chÿ-abikÿka\\
\glb 3-rub 3-hand-\textsc{loc} lest 3-grab.\textsc{irr}\\
\glft ‘she rubbed it (the tobacco) on his hand lest she could grab him’
\endgl
\trailingcitation{[jxx-p151016l-2.207-208]}
\xe

Finally, (\ref{ex:masaApp-3}) also comes from Juana. It was produced to exemplify the use of an expression that I had requested from her, ‘water a plant’.

\ea\label{ex:masaApp-3}
\begingl
\glpreamble puchuneka ÿne \textup{[}masa tepaka\textup{]}\\
\gla pi-uchu-ne-ka ÿne masa ti-paka\\
\glb 2\textsc{sg}-pour.liquid-top-\textsc{th}1.\textsc{irr} water lest 3i-die\\
\glft ‘water them (lit.: pour water on top) lest they die’
\endgl
\trailingcitation{[jxx-e151020l-1]}
\xe\is{apprehensional|)}

All ACs presented so far have balanced verbs.\is{finite verb|)} That a clause is in an adverbial relation to another one can also be signalled by deranking. This is the topic of the following section.


%mxx-e160811sd.404 -> apprehensional with nebutu!
\is{connective|)}
\is{juxtaposition|)}

\subsection{Adverbial relations expressed by deranking}\label{sec:SubordinateACs}
\is{deranked verb|(}

ACs can be built on deranked verbs (see \sectref{sec:Subordination-i}). An AC with a deranked verb cannot occur on its own.\footnote{But see \sectref{sec:AdverbialModification} for monoclausal constructions with deranked verbs.} It is thus more closely integrated into the main clause, i.e. embedded\is{embedding} in it, without being an \isi{argument} of the main clause. This is illustrated in \figref{fig:DerankedACStructure}.

\begin{figure}[!ht]


[MC [AC]]
\caption{Clause structure of adverbial clauses with a deranked verb}
\label{fig:DerankedACStructure}

\end{figure}

Integration into the main clause becomes most apparent if a \isi{preposition} is placed in front of the deranked verb, thus marking it as an \isi{oblique} constituent of the main clause. However, this is only occasionally found with verbs encoding \isi{purpose} and even less frequent with those encoding causal\is{cause} relations. If no preposition is placed before the deranked verb, the semantic type of relation towards the main clause is as unmarked as in asyndetically juxtaposed ACs (see \sectref{sec:AsyndeticSubordination}) and can only be deduced from the context.

ACs with deranked verbs usually follow the main clause with a few exceptions. Deranked verbs always index the \isi{subject}, they can also index an \isi{object} and this object can be conominated. Conomination\is{conomination} of subjects is exceptional though a few examples of this are found.


I will proceed as in the previous section and first illustrate the use of embedded clauses unmarked for their semantic connection to the main clause in \sectref{sec:EmbeddedAC_bare} and then describe the ones that combine with a preposition in \sectref{sec:EmbeddedAC_adp}, thus explicitly encoding the type of relation.

\subsubsection{Adverbial clauses with bare deranked verbs}\label{sec:EmbeddedAC_bare}

In ACs with “bare” deranked verbs, the deranked verb tells us that we are dealing with a subordinate relation. However, there is no information about the semantics of this relation, i.e. the exact nature of this relation has to be deduced from the context. We can then identify \isi{purpose}, causal\is{cause} and temporal relations, but there is sometimes a certain ambiguity involved.
The deranked verb can have realis or irrealis RS\is{reality status} for reasons that lie outside of the construction type and it can have the same or a different \isi{subject} than the main clause predicate. 
I will first consider some examples in which both predicates have the same subject and then discuss the ones with different subjects. Towards the end of the section, a few cases in which the deranked verb precedes the main clause predicate are presented, and finally, some cases in which the information provided by the subordinate verb doubles what is expressed on the main clause verb.

%tikechunÿtu bueno, kuina echÿu ajumerkuina pisuikia pero, mxx-p181027l-1.021

In (\ref{ex:walking-cane}), the main clause tells us about the existence of an entity, the walking cane, and the subordinate clause provides information about the \isi{purpose} of this walking cane. The sentence comes from Juana telling me about her encounter with two old ladies in Candelaria. It is a description of one of the ladies.

\ea\label{ex:walking-cane}
\begingl
\glpreamble  kaku chibastunemÿnetu, mhm, \textup{[}chiyuikiumÿnÿ\textup{]}\\
\gla  kaku chi-bastun-ne-mÿnÿ-tu mhm chi-yuik-i-u-mÿnÿ\\
\glb exist 3-walking.cane-\textsc{possd}-\textsc{dim}-\textsc{iam} \textsc{intj} 3-walk-\textsc{subord}-\textsc{real}-\textsc{dim}\\
\glft ‘she already had a cane, mhm, for walking’
\endgl
\trailingcitation{[jxx-p120515l-1.220-221]}
\xe

(\ref{ex:sub-i-3}) has a similar structure as (\ref{ex:walking-cane}) above. The main clause tells about an action of a participant, and the AC describes the reason of this action. It is from Miguel’s description of the \isi{frog story} to José. More precisely, it provides a description of the picture in which the boy has climbed the rock. 


\ea\label{ex:sub-i-3}
\begingl 
\glpreamble ja tipuna naka eka aitubuchepÿi \textup{[}chipikiuchi eka chumurkuku\textup{]}\\
\gla ja ti-puna naka eka aitubuchepÿi chi-pik-i-u-chi eka chumurkuku\\ 
\glb \textsc{intj} 3i-go.up.\textsc{irr} here \textsc{dem}a boy 3-be.afraid-\textsc{subord}-\textsc{real}-3 \textsc{dem}a tropical.screech.owl\\ 
\glft ‘ah, he is going to climb up here being afraid of the owl’\\ 
\endgl
\trailingcitation{[mox-a110920l-2.120-122]}
\xe

In (\ref{ex:scratch-bark-1}), Juana quite amusedly comments about her dog, which made a howling-barking sound when it scratched itself. In this example, the main clause verb is marked as continuous, but when Juana repeated the sentence for me just a moment later, she used a non-continuous verb form, the rest of the sentence being identical. %, see (\ref{ex:scratch-bark-2}).
The deranked verb either encodes the reason for barking or it simply expresses a temporal relation.

\ea\label{ex:scratch-bark-1}
\begingl
\glpreamble timajaikukuiku \textup{[}chibujakiubu\textup{]}\\
\gla ti-majaiku-kuiku chi-bujak-i-u-bu\\
\glb 3i-bark-\textsc{cont} 3-scratch-\textsc{subord}-\textsc{real}-\textsc{mid}\\
\glft ‘it barks scratching itself’
\endgl
\trailingcitation{[jxx-p120430l-1.479]}
\xe

%\ea\label{ex:scratch-bark-2}
%\begingl
%\glpreamble timajaiku \textup{[}chibujakiubu\textup{]}\\
%\gla ti-majaiku chi-bujak-i-u-bu\\
%\glb 3i-bark 3-scratch-\textsc{subord}-\textsc{real}-\textsc{mid}\\
%\glft ‘it barks scratching itself’\\
%\endgl
%\trailingcitation{[jxx-p120430l-1.480]}
%\xe

In (\ref{ex:sub-i-4}) and (\ref{ex:temcause}), the subjects differ; however, the singular agent of the main clause is also understood to be part of the plural subject of the deranked verb.

(\ref{ex:sub-i-4}) is another example from Miguel telling the \isi{frog story}, but on an occasion other than (\ref{ex:sub-i-3}) above.\footnote{Miguel told the story once to Alejo and once to José. In general, I use more examples from the second time he told the story in this work, because, already knowing what would happen, he told it more fluently and self-confidently.} He describes the picture in which the beehive lies on the ground after the dog has jumped against the tree. The deranked verb describes the \isi{purpose} of the main clause verb in this case.

\ea\label{ex:sub-i-4}
\begingl 
\glpreamble chibÿtupaiku echÿukena \textup{[}chinikianube ipitiumu\textup{]}\\
\gla chi-bÿtupaiku echÿu-kena chi-nik-i-a-nube ipiti-umu\\ 
\glb 3-make.fall \textsc{dem}b-\textsc{uncert} 3-eat-\textsc{subord}-\textsc{irr}-\textsc{pl} bee-\textsc{clf}:liquid\\ 
\glft ‘it seems that it makes it fall so that they can eat honey’\\ 
\endgl
\trailingcitation{[mtx-a110906l.093]}
\xe

(\ref{ex:temcause}) was produced by Juana who was telling about the old times before there was the reservoir in Santa Rita and they had to walk far to get water.

\ea\label{ex:temcause}
\begingl
\glpreamble tÿbaneyu ÿne tikuti nimupeki \textup{[}bejikiumÿnÿ ÿne\textup{]}\\
\gla ti-ÿbane-yu ÿne ti-kuti ni-mupeki bi-ejik-i-u-mÿnÿ ÿne\\
\glb 3i-be.far-\textsc{ints} water 3i-hurt 1\textsc{sg}-knee 1\textsc{pl}-take.away-\textsc{subord}-\textsc{real}-\textsc{dim} water\\
\glft ‘the water was very far away, my knee hurt, when we fetched water’
\endgl
\trailingcitation{[jxx-p120515l-2.005]}
\xe

(\ref{ex:sow}) and (\ref{ex:fall-fly}) have different subjects in main and subordinate clauses. However, the subject of the deranked verb is always affected by the event expressed by the main clause verb. Note that in (\ref{ex:fall-fly}), the subordinate clause precedes the main clause. A few more examples of that will follow below.


In (\ref{ex:sow}), there is a weather verb with a third person subject, and the deranked verb has a first person subject, i.e. Miguel, who gives the reason for him waiting for rain here.
%(\ref{ex:sow}) is another example with different subjects in main and purpose clause. Contrary to the statement by \citet[157]{Cristofaro2003} (see above), it is not the perfomer of the main clause that favours realisation of the purpose event. The subject of the weather verb in the main clause is non-volitional, but the subject of the purpose clause, Miguel who was speaking, has an interest in rain. This sentence could also be analysed as a consecutive one, but since it contains a deranked verb, I take it as an example of extended purpose clause.

\ea\label{ex:sow}
\begingl
\glpreamble repentekena tikeba pario \textup{[}nebukia\textup{]}\\
\gla repente-kena ti-keba pario ni-ebuk-i-a\\
\glb maybe-\textsc{uncert} 3i-rain.\textsc{irr} some 1\textsc{sg}-sow-\textsc{subord}-\textsc{irr} \\
\glft ‘maybe it rains a bit for me to sow’
\endgl
\trailingcitation{[mqx-p110826l.616]}
\xe

(\ref{ex:fall-fly}) has two third person subjects; however, one is singular and the other plural, and they refer to completely different participants. The subject of the deranked verb is the dog, and the subject of the main clause verb is the wasps. This is one of the few examples in which the clause with the deranked verb precedes the main clause. The sentence describes the same situation as (\ref{ex:sub-i-4}) above, but comes from the second occasion when Miguel told the \isi{frog story}.

\ea\label{ex:fall-fly}
\begingl
\glpreamble \textup{[}chibÿbÿtupaikiuchÿtu eka chubiu jane\textup{]} tibÿbÿkujanetu\\
\gla chi-bÿbÿtupaik-i-u-chÿ-tu eka chÿ-ubiu jane ti-bÿbÿku-jane-tu\\
\glb 3-make.fall-\textsc{subord}-\textsc{real}-3-\textsc{iam} \textsc{dem}a 3-house bee 3i-fly-\textsc{distr}-\textsc{iam}\\
\glft ‘it (the dog) having made the wasps’ nest fall, they (the wasps) fly’
\endgl
\trailingcitation{[mox-a110920l-2.080]}
\xe

Like (\ref{ex:fall-fly}), (\ref{ex:feast-town}) also has its subordinate clause preposed to the main clause. The subject of the deranked verb, \textit{piesta} ‘feast day’ is conominated here, which is very rare. The sentence stems from María S. telling me about her childhood and the lack of knowledge they had in the old days. I am not sure why María S. uses a reportive marker on the main clause verb here, either it is \isi{quotative} here – this is what I propose by the translation given – or it is used as a hearsay marker, the latter would suggest that María S. does not remember their sudden departure to town on a feast day herself and was only told by somebody else (maybe one of her older siblings).

\ea\label{ex:feast-town}
\begingl
\glpreamble \textup{[}kuyena chitupuniubu piesta\textup{]} repenteyÿchi biyunupatuji uneku\\
\gla kuyena chi-tupun-i-u-bu piesta repente-yÿchi bi-yunupa-tu-ji uneku\\
\glb like.this 3-reach-\textsc{subord}-\textsc{real}-\textsc{mid} feast.day suddenly-\textsc{lim}2 1\textsc{pl}-go.to-\textsc{iam}-\textsc{rprt} town\\
\glft ‘so when a feast day came, only that same day we said we would go to town’
\endgl
\trailingcitation{[rxx-p181101l-2.017]}
\xe


(\ref{ex:dig-dig}) is another example in which the deranked verb precedes the main clause. Actually, in this case the deranked verb is repeated as the start of a new utterance. In the previous utterance, this same subordinate verb was produced as a complement of a non-verbal predicate borrowed from Spanish. The sentence was produced by Miguel in telling the story about the fox and the jaguar and is about ongoing digging of the jaguar looking for the fox who has long escaped. The repeated verb of the main clause \textit{-teku} is atelic\is{telicity} and encodes digging, while the verb of the adverbial clause \textit{-seku} is telic with the meaning ‘dig a hole’.

\ea\label{ex:dig-dig}
\begingl
\glpreamble \textup{[}chisekiuchituji\textup{]} chitekuji chitekuji\\
\gla chi-sek-i-u-chi-tu-ji chi-teku-ji chi-teku-ji\\
\glb 3-dig.hole-\textsc{subord}-3-\textsc{iam}-\textsc{rprt} 3-dig-\textsc{rprt} 3-dig-\textsc{rprt}\\
\glft ‘digging the hole, it is said, he dug and dug, it is said’
\endgl
\trailingcitation{[jmx-n120429ls-x5.163]}
\xe

%i depue chikujikiunube, tiyunupunu chubiuyae
%and when they chased her off, she went back to her house, jxx-p120430l-2.107-108
%entonses chikujikiunube nipiji tiyunupunumÿne chubiuyae, same but 111 -> telic predicate -> after interpretation

The last two examples I want to present here were produced by Juana and in both cases, the main clause predicate is a stative verb with a concurrent associated motion\is{associated motion|(} marker. The deranked verb also expresses motion and thus seems to be unnecessary from a semantic point of view. Nonetheless, it is not uncommon in languages with the grammatical category of associated motion to combine motion predicates\is{motion predicate} with verbs including an associated motion marker \citep[128]{Rose2015}. This does not explain, however, why Juana decided to use a deranked verb in those cases, and it also seems to be possible to combine a balanced motion verb\is{finite verb} with a verb carrying an AM marker.

In (\ref{ex:stat-i-1}) Juana first uses a verb with the concurrent associated motion marker, combines it with a deranked verb encoding motion, and then uses a third verb, balanced again, which also takes the associated motion marker. With this sentence, she tells me how she arrived at her brother’s funeral, sad and all by herself. Actually, she produced a very similar sentence again a moment later, but then used a balanced verb juxtaposed to the one with the associated motion marker. The latter sentence is given in (\ref{ex:stat-noi-1}) for comparison. Thus, both options, i.e. combination of an AM verb with a deranked and a balanced verb, are possible.

\ea\label{ex:stat-i-1}
\begingl
\glpreamble nipÿsisikÿumÿnÿ \textup{[}niyuniu\textup{]} nimumukukukÿu\\
\gla ni-pÿsisi-kÿu-mÿnÿ ni-yun-i-u ni-imumuku-kukÿu\\
\glb 1\textsc{sg}-be.alone-\textsc{am.conc.tr}-\textsc{dim} 1\textsc{sg}-go-\textsc{subord}-\textsc{real} 1\textsc{sg}-look-\textsc{am.conc.tr}\\
\glft ‘I went alone, me going, I went looking around’
\endgl
\trailingcitation{[jxx-p120430l-2.248]}
\xe

\ea\label{ex:stat-noi-1}
\begingl
\glpreamble nipÿsisikÿu niyunu\\
\gla ni-pÿsisi-kÿu ni-yunu\\
\glb 1\textsc{sg}-be.alone-\textsc{am.conc.tr} 1\textsc{sg}-go\\
\glft ‘I went all by myself’
\endgl
\trailingcitation{[jxx-p120430l-2.250]}
\xe

(\ref{ex:hungry-go}) is very similar to (\ref{ex:stat-i-1}) in that there is a stative verb with the concurrent motion marker followed by a subordinate verb encoding motion. This example stems from the story about the fox and the jaguar. The jaguar did not succeed in eating the vulture. Thus he has to move on, still being hungry.

\ea\label{ex:hungry-go}
\begingl
\glpreamble tikunipapakÿu \textup{[}chiyuniu\textup{]}\\
\gla ti-kunipa-pakÿu chi-yun-i-u\\
\glb 3i-be.hungry-\textsc{am.conc.tr} 3-go-\textsc{subord}-\textsc{real}\\
\glft ‘hungry he went’
\endgl
\trailingcitation{[jmx-n120429ls-x5.222]}
\xe

\is{associated motion|)}
I will come back to these examples in \sectref{sec:AdverbialModification}.

%In (\ref{ex:cause-1}), Juana is telling me about the cows her grandparents had bought, and how they finally lost them, because some \textit{karay} came and took them away. This example is remarkable insofar as there is a stative verb \textit{-chÿnumi} ‘be sad’ that seems to take the subordinate marker. This is not entirely clear, because the verb stem ends in /i/, which then, I suppose, fuses with the subordinate marker. The last string is undoubtly the intensive marker \textit{-yu}, not a realis suffix \textit{-u}, since the /j/ is clearly audible.\footnote{Stative verbs are usually completely unmarked for realis RS, but one could assume that a realis suffix could show up here due to analogy with subordinate marking on active verbs.} %and stress is on the syllable \textit{mi}. 
%Evidence for subordinate marking is the person marker \textit{chÿ-} that does not usually appear on stative (and other intransitive) verbs. The subordinate stative verb is followed by a subordinate active verb, which offers either a second explanation for her grandmother’s illness or for her sadness. 
%
%\ea\label{ex:cause-1}
%\begingl
%\glpreamble eka yeye tikutiutu \textup{[}chÿchÿnumiyu\textup{]} \textup{[}chibejiukiunubechÿ chipeu baka\textup{]}\\
%\gla eka yeye ti-kutiu-tu chÿ-chÿnumi-i?-yu chi-bejiuk-i-u-nube-chÿ chi-peu baka\\
%\glb \textsc{dem}a granny 3i-be.ill-\textsc{iam} 3-be.sad-\textsc{subord}?-\textsc{ints} 3-take.away-\textsc{subord}-\textsc{real}-\textsc{pl}-3 3-animal cow \\
%\glft ‘my granny got ill, because of being sad, because they had taken away their cows’\\
%\endgl
%\trailingcitation{[jxx-e150925l-1.232-234]}
%\xe


\subsubsection{Deranked verbs combined with prepositions}\label{sec:EmbeddedAC_adp}
\is{preposition|(}
\is{general oblique|(}
\is{cause|(}
\is{instrument/cause|(}

Deranked verbs can be combined with the prepositions \textit{tÿpi} ‘\textsc{obl}’ and \textit{-keuchi} ‘\textsc{ins}’. This is the most overt signal of loss of verbal properties of the deranked verb and its \isi{embedding} into the main clause as an \isi{oblique}. However, the use of a preposition together with a deranked verb is rather rare in general and as for \textit{-keuchi}, this is only found in the speech of María S. It is possible that the use of prepositions together with the deranked verbs is influenced by Spanish, where the prepositions \textit{para} (\isi{purpose}) and \textit{por} (reason) are used together with a non-finite or subjunctive verb in order to express purpose and causal relations. However, in the case of \textit{tÿpi} the use as an overt marker for purpose clauses now extends to balanced verbs,\is{finite verb} too, see \sectref{sec:PurposeClauses} \citep[and also][142--143]{DanielsenTerhart2015}.
\is{instrument/cause|)}
\is{cause|)}

I will start with a few purpose clauses\is{purpose|(} that contain both \textit{tÿpi} and a deranked verb and then turn to causal clauses with \textit{-keuchi}.

(\ref{ex:Purp-3}), like (\ref{ex:feast-town}) in the previous section, is one of the few examples in which the subject of a deranked verb is conominated. The context of this sentence is Miguel speaking about a job he did in the past. Apparently, he wanted to explain the function of the railway sleepers he made by using a subordinate clause.

\ea\label{ex:Purp-3}
\begingl
\glpreamble banau echÿu durmientejane \textup{[}tÿpi chiyuikiu echÿu tren\textup{]}\\
\gla bi-anau echÿu durmiente-jane tÿpi chi-yuik-i-u echÿu tren\\
\glb 1\textsc{pl}-make \textsc{dem}b sleeper-\textsc{distr} \textsc{obl} 3-walk-\textsc{subord}-\textsc{real} \textsc{dem}b train\\
\glft ‘we made sleepers for the train to move’
\endgl
\trailingcitation{[mxx-p181027l-1.129]}
\xe


In (\ref{ex:sowing-tool}), Juana first uses the preposition together with the noun \textit{makina} ‘machine, tool’ and then with a clause that explains what the tool is good for, i.e. sowing rice and corn. The tool she speaks about eases sowing by making little holes in the ground and inserting the kernels. It was brought to Concepción and sold there by a lady from Germany and her husband.

\ea\label{ex:sowing-tool}
\begingl
\glpreamble i tÿpi echÿu makina kapunu \textup{[}tÿpi bebukia arusu bebukia amuke\textup{]}\\
\gla i tÿpi echÿu makina kapunu tÿpi bi-ebuk-i-a arusu bi-ebuk-i-a amuke\\
\glb and \textsc{obl} \textsc{dem}b machine come \textsc{obl} 1\textsc{pl}-sow-\textsc{subord}-\textsc{irr} rice 1\textsc{pl}-sow-\textsc{subord}-\textsc{irr} corn\\
\glft ‘and for this tool, she came, (the tool) for us to sow rice and sow corn’
\endgl
\trailingcitation{[jxx-p120515l-2.040]}
\xe

(\ref{ex:spindle-weave}) is also about the function of a tool, with the function being expressed by a clause with a deranked verb introduced by \textit{tÿpi}. It was produced by Clara as an answer to Swintha’s question about how a spindle is called in Paunaka.

\ea\label{ex:spindle-weave}
\begingl
\glpreamble echÿu turnu \textup{[}tÿpi bijÿkia\textup{]}\\
\gla echÿu turnu tÿpi bi-ijÿk-i-a\\
\glb \textsc{dem}b spindle \textsc{obl} 1\textsc{pl}-weave-\textsc{subord}-\textsc{irr}\\
\glft ‘it is (called) \textit{turnu}, (used) for weaving’
\endgl
\trailingcitation{[cux-120410ls.189-190]}
\xe

\is{purpose|)}
\is{general oblique|)}



\is{cause|(}
\is{instrument/cause|(}
The two examples found in the corpus in which the instrumental and causal preposition \textit{-keuchi} is used together with a deranked verb are given below. In both of them, the preposition takes a third person marker,\is{person marking} thus it does not agree\is{agreement} with the subject of the deranked verb, which is a first person plural in (\ref{ex:smoke-ill}) and a first person singular in (\ref{ex:hammock-buy}). Instead this third person marker seems to index the clause.

In (\ref{ex:smoke-ill}), the preposition is clearly used to encode a cause: we all know that we can get ill by smoking. The whole clause itself acts as a causal clause as signalled by the use of the connective \textit{ch(je)puine} ‘because’ (see \sectref{sec:CauseConsequence}) and represents an afterthought to the previous one, in which María evaluated the fact that I had stopped smoking as good.

\ea\label{ex:smoke-ill}
\begingl
\glpreamble chepuine bikutiu \textup{[}chikeuchi bijibÿkia\textup{]}\\
\gla chepuine bi-kutiu chi-keuchi bi-jibÿk-i-a\\
\glb because 1\textsc{pl}-be.ill 3-\textsc{ins} 1\textsc{pl}-smoke-\textsc{subord}-\textsc{irr}\\
\glft ‘because we get ill by smoking’
\endgl
\trailingcitation{[rxx-e120511l.384]}
\xe

In (\ref{ex:hammock-buy}), it seems that \textit{-keuchi} rather introduces a \isi{purpose} clause like \textit{tÿpi} does, but possibly María S. wanted to encode the reason for making and selling her hammock as a cause rather than a purpose, and this is what I try to convey with the translation given. More data is needed here in order to analyse and evaluate the use of \textit{-keuchi} together with a deranked verb. The sentence was produced when I wanted to elicit a question, but got the answer to the question instead.

\ea\label{ex:hammock-buy}
\begingl
\glpreamble micha, nana yumaji depue nipabentecha \textup{[}chikeuchi niyÿseikia amuke arusu\textup{]}\\
\gla micha nÿ-ana yumaji depue ni-pabentecha chi-keuchi ni-yÿseik-i-a amuke arusu\\
\glb good 1\textsc{sg}-make.\textsc{irr} hammock afterwards 1\textsc{sg}-sell.\textsc{irr} 3-\textsc{ins}  1\textsc{sg}-buy-\textsc{subord}-\textsc{irr} corn rice\\
\glft ‘I am fine, I will make my hammock and afterwards I will sell it, because I want to buy corn and rice’
\endgl
\trailingcitation{[rxx-e181022le]}
\xe

\is{instrument/cause|)}
\is{cause|)}


The use of a deranked verb together with a preposition is the most explicit sign of integration of the subordinate clause into the main clause.\is{preposition|)} The next section is also about integrating strategies on another level. In the cases discussed below, one verb is combined with another verb, both form a close nexus and the result is a single clause with the subordinate verb expressing a goal argument.
\is{deranked verb|)}

\subsection{Adverbial relations encoded by integrating strategies}\label{sec:SVC_and_MCPC}
\is{motion predicate|(}
\is{serial verb construction|(}
\is{motion-cum-purpose construction|(}

Paunaka has two multi-verb constructions that encode purpose relations: the serial verb construction (SVC) and a construction I call the motion-cum-purpose construction in this grammar (MCPC). Both are used to set an event (i.e. the one encoding purpose) in relation to a motion event. The difference between the two constructions is that the non-motion verb in a MCPC is marked for this relation, while it is unmarked in a SVC.

Consider (\ref{ex:SVC-MCP}) and (\ref{ex:SVC-1}). In (\ref{ex:SVC-MCP}) we have a MCPC: the \isi{dislocative} marker \textit{-pu} on the second predicate shows that it is related to the motion predicate. In (\ref{ex:SVC-1}) a similar event is expressed, but in this case, the second verb is completely unmarked. %The second verb in these constructions in put in square brackets, since it encodes the adverbial relation... -> oben
(\ref{ex:SVC-MCP}) was produced by María S. when telling me what she did that day. (\ref{ex:SVC-1}) was elicited from Juana. 


\ea\label{ex:SVC-MCP}
\begingl
\glpreamble niyunu nisane \textup{[}nisupu\textup{]}\\
\gla ni-yunu ni-sane ni-isu-pu\\
\glb 1\textsc{sg}-go 1\textsc{sg}-field 1\textsc{sg}-weed-\textsc{dloc}\\
\glft ‘I went to my field to weed’
\endgl
\trailingcitation{[rxx-e120511l.033]}
\xe

\ea\label{ex:SVC-1}
\begingl
\glpreamble piyuna nauku \textup{[}pisua\textup{]}\\
\gla pi-yuna nauku pi-isua\\
\glb 2\textsc{sg}-go.\textsc{irr} there 2\textsc{sg}-weed.\textsc{irr}\\
\glft ‘you go there to weed’
\endgl
\trailingcitation{[jxx-e191021e-2]}
\xe


The motion verbs in both of these constructions show a high degree of integration with another verb. They cannot be negated separately,\is{negation} and it can thus be claimed that the constructions consist of a single clause, not two separate ones. I will come back to this later in this section.

The serial verb construction is marginally used to encode simultaneous events, but the main function of both constructions is to express motion and purpose of this motion. Purpose of motion shows a certain analogy to nominal or adverbial expression of a goal. This is because by metonymical extension an action can come to stand for the location where it is carried out \citep[98]{SchmidtkeBode2009}.
% In one case, motion is the means to reach a location, in the other one motion is the means to make realisation of another event possible, i.e. to “reach” another event.

Compare (\ref{ex:go-field}) and (\ref{ex:go-dance}). In the first example, the goal is expressed by a noun, which takes the locative marker \textit{-yae}; in the second, the goal is a verb marked with the \isi{dislocative} suffix \textit{-pa} (with irrealis RS).\footnote{The noun expressing the goal can also be unmarked as in (\ref{ex:SVC-MCP}) above.} An NP or adverb expressing the goal can also co-occur with a verb expressing the purpose as has been shown in (\ref{ex:SVC-MCP}) and (\ref{ex:SVC-1}) above.

\ea\label{ex:go-field}
\begingl
\glpreamble niyuna nisaneyae\\
\gla ni-yuna ni-sane-yae\\
\glb 1\textsc{sg}-go.\textsc{irr} 1\textsc{sg}-field-\textsc{loc}\\
\glft ‘I will go to my field’
\endgl
\trailingcitation{[jxx-n101013s-1.652]}
\xe

\ea\label{ex:go-dance}
\begingl
\glpreamble niyuna \textup{[}nimuikupa\textup{]}\\
\gla ni-yuna ni-muiku-pa\\
\glb 1\textsc{sg}-go.\textsc{irr} 1\textsc{sg}-dance-\textsc{dloc.irr}\\
\glft ‘I will go to dance’
\endgl
\trailingcitation{[rxx-e181022le]}
\xe

%(jxx-e081025s-1.434): niyuna asaneti

As for the expression of simultaneous events in the SVC, there is only one specific verb series which is used at least by several, possibly by all speakers. In that case, the verb \textit{-yunu} ‘go’ combines with \textit{-eiku} ‘follow’ to indicate that the subject accompanies somebody else (‘go along with’). As is the case with those serial verbs encoding purpose of motion, this case of simultaneous event expression can be argued to encode a kind of goal, though not at a fixed location: the goal is the other person who is also moving.

We can thus argue that just like complement clauses\is{complement relation|(} are clausal expressions of an object of a verb,\footnote{This, at least, is a common definition of a complement clause. It is not sure, though, whether complement clauses can be analysed as arguments\is{argument} at all in Paunaka, see \sectref{sec:ComplementClauses}.} in the SVC and MCPC, a clause expresses the oblique (goal) argument of a motion verb. Since purpose has been traditionally defined as an \textit{adverbial} relation, purpose-of-motion constructions shall be discussed in this section, not as a special form of complement clause.\is{complement relation|)}

%This may ultimately the reason, why purpose of motion expressions are cross-linguistically frequently different from purpose expressions that do not involve motion.

Let us have a look at the characteristics of both constructions. As for the term “serial verb construction”, it has been applied to a wide array of multi-verb clauses, which has led to a certain ambiguity about which constructions constitute SVCs and which ones should be classified differently. In the spirit of \citet[]{Aikhenvald2006a, Aikhenvald2011,Aikhenvald2018}, the term is used to describe a technique to combine predicates that share at least one argument without morphosyntactically marking the relation between those predicates. According to this author, the construction encodes a single event and is monoclausal. As I have already stated in \sectref{sec:Coordination}, I do not have the means to check whether something is conceived as a single event or as multiple events, so I will not further pursue this issue here. As for monoclausality, this also holds for complement clauses\is{complement relation} in Paunaka, and indeed, we find complement clauses among the ones that are defined as SVCs in the aforementioned publications.

A relatively narrow definition of SVC that deliberately excludes complement clauses and some other constructions has been proposed by \citet[296]{Haspelmath2016}: “A serial verb construction is a monoclausal construction consisting of multiple independent verbs with no element linking them and with no predicate–argument relation between the verbs”. Since I have just stated that the purpose clause can be defined as a clausal \isi{argument} of the motion verb expressing the goal, the definition would possibly also exclude what I am just trying to define as a SVC here. However, if we replace “argument” by “core argument”, the definition works well for the purpose of this work.\footnote{The predicate-argument point of this definition is tricky in this case, because the complement clause\is{complement relation} cannot be defined as a proper \isi{argument} of the complement-taking verb either. Thus the definition does not rule out that complement clauses are SVCs in Paunaka. There are, however, some other features in which Paunaka’s complement clauses deviate from what has been proposed to constitute complementation SVCs, see \sectref{sec:ComplementClauses}.}  It should be mentioned though that the examples given by both \citet[]{Aikhenvald2006a,Aikhenvald2018} as well as \citet[]{Haspelmath2016} to illustrate SVCs often closely resemble the ones I have analysed as including asyndetic \isi{coordination} or asyndetic subordination (see \sectref{sec:AsyndeticCoordination} and \sectref{sec:AsyndeticSubordination} respectively), because I propose that they consist of two clauses.

A crucial point of the definition of the SVC is its being monoclausal – and this also holds for the MCPC of Paunaka. A test for monoclausality that is cross-linguistically applicable is scope of negation\is{negation|(} and place of the negator, i.e. “there is only one way to form the negation, usually with scope over all the verbs” \citep[299]{Haspelmath2016}. 

An example to illustrate the scope of negation over both predicates in a SVC is given in (\ref{ex:SVC-neg-1}). The negative particle \textit{kuina} precedes the motion verb \textit{-yunu}, and since this verb is negated, the verb encoding the purpose of the motion is negated, too. The sentence cannot be understood as ‘she did not go to the airport, (but) she took her’. In order to express such a meaning, two clauses would be necessary, clearly separated from each other by having at least different intonation contours, by the use of a connective or by repetition of \textit{kuina} uttered with falling \isi{intonation} and preceding the second predicate.\footnote{We have three clauses in this latter case; the first one is a negated independent one, the second one consists of the negator only and the third one is the positive clause that expresses some sort of contrast with the latter two being combined into a complex clause by asyndetic subordination, i.e. the the structure of such negative assertion corresponding to (\ref{ex:SVC-neg-1}) would be [\textit{kuina tiyuna la pistayae}] [[\textit{kuina}] [\textit{chibea}]] ‘she did not go to the airport. No, she will/can get her out’, where irrealis of the last predicate may be due to \isi{future reference} or possibility, but not due to negation. See Footnote \ref{fn:negation-coordination} in \sectref{sec:AsyndeticCoordination} for a real example of this linking strategy.}

The example comes from Juana in telling me how her daughter was deported from Spain for not having a valid visa. In Juana’s eyes, her other daughter could have prevented this.

\ea\label{ex:SVC-neg-1}
\begingl
\glpreamble kuina tiyuna la pistayae \textup{[}chibea\textup{]}\\
\gla kuina ti-yuna {la pista}-yae chi-bea\\
\glb \textsc{neg} 3i-go.\textsc{irr} {airport}-\textsc{loc} 3-take.away.\textsc{irr}\\
\glft ‘she didn’t go to the airport to get her out’
\endgl
\trailingcitation{[jxx-p110923l-1.296]}
\xe

In (\ref{ex:MCPC-neg-1}) a negative MCPC is shown. As in (\ref{ex:SVC-neg-1}) above, scope of the negator is over both predicates, thus the sentence cannot be read as ‘we don’t go on to the reservoir anymore, but we fetch water’. A different construction would be used in that case. Actually, Juana did express a similar kind of contrast in two sentences that immediately followed in her report. They are given in (\ref{ex:MCPC-follow}). This, although making use of material from Spanish (\textit{pa} is an abbreviation of \textit{para} ‘for’, \textit{si} comes from \textit{sí} ‘yes’), is a normal way to form a contrast.\footnote{For the use of a deranked verb following an adverb see \sectref{sec:AdverbialModification}.}


\ea\label{ex:MCPC-neg-1}
\begingl
\glpreamble teje kuina biyunukabu atajauyae \textup{[}bepa ÿne\textup{]}\\
\gla te-ja? kuina bi-yunu-uka-bu atajau-yae bi-be-pa ÿne\\
\glb \textsc{seq}-\textsc{emph}1? \textsc{neg} 1\textsc{pl}-go-\textsc{add.irr}-\textsc{dsc} reservoir-\textsc{loc} 1\textsc{pl}-take.away-\textsc{dloc.irr} water\\
\glft ‘thus we don’t go to the reservoir anymore either to fetch water’
\endgl
\trailingcitation{[jxx-p120515l-2.218]}
\xe

\ea\label{ex:MCPC-follow}
\begingl
\glpreamble \textup{[}pa bemusuika\textup{]} si nauku bemusuikia i \textup{[}tÿpi eka bea\textup{]} eka nechÿujikutu \\
\gla pa bi-emusuika si nauku bi-emusuik-i-a i tÿpi eka bi-ea eka nechÿu-jiku-tu\\
\glb for 1\textsc{pl}-wash.\textsc{irr} yes there 1\textsc{pl}-wash-\textsc{subord}-\textsc{irr} and \textsc{obl} \textsc{dem}a 1\textsc{pl}-drink.\textsc{irr} \textsc{dem}a \textsc{dem}c-\textsc{lim}1-\textsc{iam}\\
\glft ‘in order to wash, yes, there we wash and for drinking, this is just there (i.e. a water tank close to the rectory) now’
\endgl
\trailingcitation{[jxx-p120515l-2.219-220]}
\xe
\is{negation|)}


Both the SVC and the MCPC are exclusively used with motion verbs as first verbs, predominantly \textit{-yunu} ‘go’, marginally also with others. If the purpose of a non-motion event is to be expressed, a different construction has to be chosen. 

Consider (\ref{ex:Purp-2}) which has two purpose expressions: one is a MCPC and the other one a purpose clause that relates to the second (i.e. non-motion) verb of the MCPC. The purpose verb of the motion verb is marked by the irrealis \isi{dislocative} marker \textit{-pa}, while the other purpose verbs are deranked. The first part of the sentence, i.e. the motion and purpose-of-motion part, was elicited, but the non-motion purposive part was added to the sentence by María S. herself.

\ea\label{ex:Purp-2}
\begingl
\glpreamble niyunu asaneti \textup{[}nibÿkepaikupa\textup{]} \textup{[}nebukia amuke nebukia arusu\textup{]}\\
\gla ni-yunu asaneti ni-bÿkepaiku-pa nÿ-ebuk-i-a amuke nÿ-ebuk-i-a arusu \\
\glb 1\textsc{sg}-go field 1\textsc{sg}-clean.up.field-\textsc{dloc} 1\textsc{sg}-sow-\textsc{subord}-\textsc{irr} corn 1\textsc{sg}-sow-\textsc{subord}-\textsc{irr} rice\\
\glft ‘I went to my field to clean up the residues after fire clearing in order to sow corn and sow rice’
\endgl
\trailingcitation{[rxx-e181020le]}%el. or semi-el.
\xe

In both constructions, the motion verb and the purpose verb necessarily have the same subject. If the purpose verb has a different subject, a different construction has to be used. I have actually only found one example of this, which was elicited from Miguel and is given in (\ref{ex:Purp-DS}). He chose a \isi{deranked verb} to encode the purpose part.

\ea\label{ex:Purp-DS}
\begingl
\glpreamble nÿti niyuna \textup{[}chimukiachÿ\textup{]}\\
\gla nÿti ni-yuna chi-muk-i-a-chÿ\\
\glb 1\textsc{sg.prn} 1\textsc{sg}-go.\textsc{irr} 3-sleep-\textsc{subord}-\textsc{irr}-3\\
\glft ‘I go so that he can sleep’
\endgl
\trailingcitation{[mxx-e160811sd.311-312]}
\xe


Other features shared by the constructions are that both verbs are fully inflected, i.e. they take person\is{person marking} and RS marking.\is{reality status|(} In the case of SVC, the second verb looks exactly like an independent verb, while in a MCPC the \isi{dislocative} marker inflects for RS. If the motion verb has realis RS, the second verb may have realis or irrealis RS.\footnote{This is actually against the prediction by \citet[3]{Aikhenvald2018}, who claims that RS usually has scope over both verbs of a SVC.} If the motion verb has irrealis RS, the purpose verb necessarily also has irrealis RS.\is{reality status|)} The motion verb always precedes the purpose verb. In most cases, the motion verb and the purpose verb are contiguous \citep[cf.][37]{Aikhenvald2006}, but a noun or adverb expressing the goal can be placed between the two verbs. Only in MCPCs, a conominated subject can also interrupt the sequence of the two verbs. As has been mentioned above, only the SVC can marginally also be used to express simultaneous motion.

SVCs and MCPCs are largely interchangeable; they offer distinct means to express the same thing. They may also be combined. This is the case in (\ref{ex:SVC-MCPC}) and (\ref{ex:MCPC-SVC}). 

In (\ref{ex:SVC-MCPC}), there are two independent sentences that were uttered in a sequence (with a pause and laughter between them indicated by the use of the semicolon). First, María S. makes use of a SVC; the second sentence has a MCPC. Both refer to her pig that had been in her yard before, but at some point had suddenly disappeared. It is the answer about my question where the pig had gone. 

\ea\label{ex:SVC-MCPC}
\begingl
\glpreamble tiyunu \textup{[}tisemaiku yÿtie atajauyae\textup{]}; tiyunu \textup{[}tiyumachuikupu\textup{]}\\
\gla ti-yunu ti-semaiku yÿtie atajau-yae ti-yunu ti-yumachuiku-pu\\
\glb 3i-go 3i-search food reservoir-\textsc{loc} 3i-go 3i-root-\textsc{dloc}\\
\glft ‘it went to look for food at the reservoir; it went to root’
\endgl
\trailingcitation{[rxx-e181024l]}
\xe

In (\ref{ex:MCPC-SVC}) two purpose verbs of a motion predicate are coordinated; however, the first one, \textit{tinikupajane} ‘they eat’, has a \isi{dislocative} marker, and the second one, \textit{teajane} ‘they drink’, is unmarked.\is{coordination} For lack of data, I cannot say whether this is a general pattern in coordination of purpose verbs in this kind of construction. The sentence refers to Juana’s ducklings.

\ea\label{ex:MCPC-SVC}
\begingl
\glpreamble tiyunujane kosinayae \textup{[}tinikupajane teajane ÿne\textup{]}\\
\gla ti-yunu-jane kosina-yae ti-niku-pa-jane ti-ea-jane ÿne\\
\glb 3i-go-\textsc{distr} kitchen-\textsc{loc} 3i-eat-\textsc{dloc.irr}-\textsc{distr} 3i-drink.\textsc{irr}-\textsc{distr} water\\
\glft ‘they go into the kitchen to eat and drink water’
\endgl
\trailingcitation{[jxx-e150925l-1.116]}
\xe 




The characteristics of both constructions are summarised and contrasted in \tabref{table:AC-IntegratingConstructions}.

\begin{table}
\caption{Characteristics of serial verb construction and motion-cum-purpose construction in comparison}

\begin{tabular}{lll}
\lsptoprule
Feature & SVC & MCPC\\
\midrule
monoclausal & \ding{51} & \ding{51}\\
dependency marking & & \ding{51}\\
V1 = motion verb & \ding{51} & \ding{51}\\
V2 encodes purpose & \ding{51} & \ding{51} \\
V2 may encode simultaneous action/accompaniment & \ding{51} & \\
same subject & \ding{51} & \ding{51}\\
RS of V2 same as V1 or \textsc{irr} & \ding{51} & \ding{51}\\
goal between V1 and V2 & \ding{51} & \ding{51}\\
subject between V1 and V2 & & \ding{51}\\
\lspbottomrule
\end{tabular}

\label{table:AC-IntegratingConstructions}
\end{table}



There is a tendency that the action encoded by the purpose predicate in a MCPC is carried out in a specific place. The connection between action and place is non-accidental. The place does not need to be expressed overtly: it may be understood from the context or it may be conventionalised that a specific action is carried out in a specific place. The MCPC is thus often used with actions that are done habitually,\is{habitual} that belong to everyday-life of the speakers. There is no such tendency in serial verb constructions. 


For an illustration of the difference between the serial verb construction and the motion-cum-purpose construction, consider (\ref{ex:look-cows-1}) and (\ref{ex:look-cows-2}). Both examples are taken from the story about the cowherd whose cows are taken away by the \textit{pÿsi}, the spirit of the hill. In the SVC of (\ref{ex:look-cows-1}), the spirit asks the cowherd whether he wants to go and see his cows; the place, although it has been expressed in the previous sentence, is unimportant at first. However, when the cowherd has accepted the offer, in the invitation to actually go and see the cows, motion is necessarily directed towards a specific place, although this place is not overtly expressed in the MCPC of (\ref{ex:look-cows-2}).

\ea\label{ex:look-cows-1}
\begingl
\glpreamble ¿pisachu piyuna \textup{[}pimuajane\textup{]}?\\
\gla pi-sachu pi-yuna pi-imua-jane\\ 
\glb 2\textsc{sg}-want 2\textsc{sg}-go.\textsc{irr} 2\textsc{sg}-see.\textsc{irr}-\textsc{distr}\\ 
\glft ‘do you want to go and see them?’\\ 
\endgl
\trailingcitation{[mxx-n151017l-1.35]}
\xe


\ea\label{ex:look-cows-2}
\begingl
\glpreamble “¡jaje biyuna \textup{[}bimupajane echÿu bakajane\textup{]}!” tikechuji\\
\gla jaje bi-yuna bi-imu-pa-jane echÿu baka-jane ti-kechu-ji\\ 
\glb \textsc{hort} 1\textsc{pl}-go.\textsc{irr} 1\textsc{pl}-see-\textsc{dloc.irr}-\textsc{distr} \textsc{dem}b cow-\textsc{distr} 3i-say-\textsc{rprt}\\ 
\glft ‘“let’s go and see the cows!” he said, it is said’\\ 
\endgl
\trailingcitation{[mxx-n151017l-1.38]}
\xe


A similar contrast becomes apparent in the following two examples where (\ref{ex:fox-search-1}) is a SVC and (\ref{ex:fox-search-2}) a MCPC. The first of them was produced by Juana, when she and Miguel were telling the story about the fox and the jaguarundi. It is the beginning of this story after Miguel has completed another episode in which the fox and the jaguar interact. Juana sets the scene by stating that the fox went on from one place and was in search for chicken. This search is not carried out in a specific place, thus a serial verb construction is most appropriate.

\ea\label{ex:fox-search-1}
\begingl
\glpreamble tiyunukutu \textup{[}tisemaika takÿra\textup{]} kupisaÿrÿ\\
\gla ti-yunuku-tu ti-semaika takÿra kupisaÿrÿ\\
\glb 3i-go.on-\textsc{iam} 3i-search chicken fox\\
\glft ‘the fox went on in order to look for chicken’
\endgl
\trailingcitation{[jmx-n120429ls-x5.300]}
\xe

(\ref{ex:fox-search-2}) was produced by Miguel after he had taken the turn from Juana, since she did not remember how the story went on. This example represents direct speech of the fox who addresses the jaguarundi. In this case, the fox has made out a specific place to look for chicken, which is first encoded rather vaguely by \textit{nauku} ‘there’, but then expressed more specifically in the following juxtaposed clause which can be analysed as being coordinated to the first one expressing the purpose (thus both clauses together express the purpose of the motion predicate and the whole motion-cum-purpose construction is a complement of the verb \textit{-sachu} ‘want’).\footnote{The whole sentence is thus composed as follows:\\ \textit{\textup{[}nisachu \textup{[}biyuna \textup{[}\textup{[}bisemaikupa takÿra nauku\textup{]} \textup{[}bibÿkupa chubiaeyae\textup{]}\textup{]}\textup{]}\textup{]}}} Since there is a specific place where the search is carried out, a motion-cum-purpose construction is used.\footnote{Note that although the second predicate of the purpose clause, \textit{bibÿkupa} ‘we enter’, also ends in \textit{-pa}, this cannot be claimed to overtly indicate the connection to the motion predicate, because the dislocative marker is lexicalised on this verb, see \sectref{sec:PA}.}


\ea\label{ex:fox-search-2}
\begingl
\glpreamble “nisachu biyuna \textup{[}bisemaikupa takÿra nauku bibÿkupa chubiaeyae\textup{]}”\\
\gla ni-sachu bi-yuna bi-semaiku-pa takÿra nauku bi-bÿkupa chÿ-ubiae-yae\\
\glb 1\textsc{sg}-want 1\textsc{pl}-go.\textsc{irr} 1\textsc{pl}-search-\textsc{dloc.irr} chicken there 1\textsc{pl}-enter.\textsc{irr} 3-house-\textsc{loc}\\
\glft ‘“I want us to go to look for chicken there and go into the house”’
\endgl
\trailingcitation{[jmx-n120429ls-x5.321]}
\xe


The choice of one or the other construction may also partly depend on the purpose predicate. I have noticed that the verb \textit{-musuiku} ‘wash’ is more often used in a SVC, while the verb \textit{-isu} ‘weed’ is usually construed as purpose verb in a MCPC, although we can assume that both describe an action that is done habitually\is{habitual} in a specific place. This does not mean that the other construction is not possible or never used.

Backed by this definition and description of general characteristics of the two constructions, the following sections will provide a few more examples of both. I will proceed here as in the previous sections, from less to more overt marking of the adverbial relation between the two verbs. In \sectref{sec:SerialVerbs}, some examples including serial verbs will be given. Among them are the ones that encode purpose as well as the ones that encode simultaneous action. \sectref{sec:MotionCumPurpose}  shows some more examples of the MCPC. 

% tiyunu nÿuchiku tiyÿseikupu baka Monkoxÿyae, rxx-e181020le
\is{motion-cum-purpose construction|)}

\subsubsection{The serial verb construction}\label{sec:SerialVerbs}

The characteristics of the serial verb construction have been described in detail in \sectref{sec:SVC_and_MCPC} above. \figref{fig:SVC} provides a short summary.

\begin{figure}
%
%
A serial verb construction (SVC)... 
\begin{itemize}
\item is a monoclausal construction
\item in which a motion verb, usually \textit{-yunu} ‘go’, is combined with a second verb that either encodes purpose of motion or marginally a simultaneous action
\item both verbs are fully inflected and unmarked for dependency
\item they necessarily have the same subject
\item the RS of the second verb is either irrealis or equal to the RS of the motion verb
\item only an adverb or a noun referring to the goal of a motion verb can interrupt the sequence of motion verb and second verb
\end{itemize}
%
\caption{Characteristics of the SVC}
\label{fig:SVC}
\end{figure}

I will first give some examples of SVCs used to express purpose of motion. In the end of this section, I provide information about a second type of SVC, which encodes simultaneous motion. For most speakers, this is restricted to one verb in second position, i.e. \textit{-eiku} ‘follow’. Only Juana uses SVCs to encode simultaneous motion at least with a second verb \textit{-umu} ‘take’. Finally, I will discuss a few cases of possible SVCs with motion verbs other than \textit{-yunu} ‘go’ in first position.

In most cases that a SVC is used, the subject is topical\is{topic} or at least accessible, so that it needs not to be conominated,\is{conomination} but there are a few cases with conominated subjects, too. In this case, the subject either precedes the motion verb as in (\ref{ex:SVC-3}) or it follows the whole SVC with its objects, as has been shown in (\ref{ex:fox-search-1}).\is{word order} In (\ref{ex:SVC-3}) both verbs have realis RS. It is taken from the creation story as told by Juana.

\ea\label{ex:SVC-3}
\begingl
\glpreamble Maria Eva tiyunu \textup{[}tiyejiku ucheti\textup{]}\\
\gla {Maria Eva} ti-yunu ti-yejiku ucheti\\
\glb {María Eva} 3i-go 3i-tear.out chili\\
\glft ‘María Eva went to harvest chili’
\endgl
\trailingcitation{[jxx-n101013s-1.383]}
\xe

In contrast, in (\ref{ex:roast-leaf}) the second verb of the SVC has irrealis RS. There is a third verb juxtaposed, \textit{bipÿrupune} ‘we roasted leaves’, which again has realis RS and can be considered a separate clause asyndetically juxtaposed to the SVC. This sentence was produced by María S. in telling me about her past. It is about leaves of a wild plant that they collected and ate. 

\ea\label{ex:roast-leaf}
\begingl
\glpreamble biyunu \textup{[}biyejikamÿnÿ\textup{]} bipÿrupune\\
\gla bi-yunu bi-yejika-mÿnÿ bi-pÿru-pune\\
\glb 1\textsc{pl}-go 1\textsc{pl}-tear.out.\textsc{irr}-\textsc{dim} 1\textsc{pl}-burn-leaf\\
\glft ‘we went to harvest and roasted the leaves’
\endgl
\trailingcitation{[rxx-p181101l-2.223]}
\xe

(\ref{ex:SVC-4}) is a sentence elicited from Juana, which has future reference, thus both predicates of the SVC have irrealis RS. Like in (\ref{ex:roast-leaf}) above, there is a second clause, this time preceding the SVC. 

\ea\label{ex:SVC-4}
\begingl
\glpreamble nÿbÿsÿupupunuka naka te niyunatu \textup{[}nemusuika\textup{]}\\
\gla nÿ-bÿsÿu-pupunuka naka te ni-yuna-tu nÿ-emusuika\\
\glb 1\textsc{sg}-come-\textsc{reg.irr} here \textsc{seq} 1\textsc{sg}-go.\textsc{irr}-\textsc{iam} 1\textsc{sg}-wash.\textsc{irr}\\
\glft ‘when I come back here, then I go to wash’
\endgl
\trailingcitation{[jxx-e190210s-01]}
\xe

(\ref{ex:SVC-5}) is an example with negation. The negative particle precedes the verbs and has scope over both of them. This sentence was elicited from Juana and is highly complex. Besides the first clause including the SVC, the second one has a (double) complement construction. As can be seen, the complement verbs are also completely unmarked and thus the two constructions, SVC and complementation, look very similar (see \sectref{sec:Unmarked_CCs} for more information about complement clauses).


\ea\label{ex:SVC-5}
\begingl
\glpreamble kuina niyuna \textup{[}nichujijikabu\textup{]}, kuina nisacha nisamanube chichujijikabunube\\
\gla kuina ni-yuna ni-chujijika-bu kuina ni-sacha ni-sama-nube chi-chujijika-bu-nube\\
\glb \textsc{neg} 1\textsc{sg}-go.\textsc{irr} 1\textsc{sg}-talk.\textsc{irr}-\textsc{dsc} \textsc{neg} 1\textsc{sg}-want.\textsc{irr} 1\textsc{sg}-hear.\textsc{irr}-\textsc{pl} 3-talk.\textsc{irr}-\textsc{dsc}-\textsc{pl}\\
\glft ‘I don’t go to have a conversation anymore, I don’t want to hear them talking anymore’
\endgl
\trailingcitation{[jxx-e190210s-01]}
\xe

A request or command can also contain a SVC, as in (\ref{ex:SVC-2}), in which Miguel reports what his daughter told him that day, when he came to Santa Rita.

\ea\label{ex:SVC-2}
\begingl
\glpreamble “¡piyuna \textup{[}piririka echÿu kampana\textup{]}!”\\
\gla pi-yuna pi-ririka echÿu kampana\\
\glb 2\textsc{sg}-go.\textsc{irr} 2\textsc{sg}-knock.\textsc{irr} \textsc{dem}b bell\\
\glft ‘“go ring the bell!”’
\endgl
\trailingcitation{[mxx-n101017s-2.075-076]}
\xe

Up to here, all examples in this section had contiguous verbs, i.e. the verbs were adjacent with no constituents between them. An example with non-contiguous serial verb is given in (\ref{ex:SVC-nc}). The adverb \textit{nauku} ‘there’ is placed between the two verbs in this case. This sentence was elicited from María S.

\ea\label{ex:SVC-nc}
\begingl
\glpreamble niyuna nauku \textup{[}niyÿseichanube\textup{]}\\
\gla ni-yuna nauku ni-yÿseicha-nube\\
\glb 1\textsc{sg}-go.\textsc{irr} there 1\textsc{sg}-greet.\textsc{irr}-\textsc{pl}\\
\glft ‘I will go there to greet them’
\endgl
\trailingcitation{[rxx-e181022le]}
\xe

In all examples presented above, the non-motion verb encodes purpose of motion. In addition to this, there is one specific SVC in which the actions are realised simultaneously. In this case \textit{-yunu} is combined with \textit{-eiku} ‘follow, go behind’. The two verbs often, but not always, form one phonological word, with only one primary \isi{stress} (on the first syllable of the second grammatical word) and no pause in-between. They have a comitative reading in most cases, which is derived from the combination of the verbs’ semantics: ‘go somewhere following someone’ can be interpreted as ‘go somewhere with someone’. A \isi{comitative} reading of \textit{-eiku} is not possible, when no motion is implied, instead the preposition \textit{-ajechubu} is used in those cases (see \sectref{sec:adp-ajechubu}).\footnote{\label{fn:along}As for the requirement of independent usage of the verbs in a SVC, \textit{-eiku} alone is very rare, and it seems to begin to grammaticalise\is{grammaticalisation} into a \isi{preposition} with the meaning ‘along’. 
Consider (\ref{ex:tras}), where \textit{-eiku} does not agree\is{agreement} in person with the verb, but with the noun. This sentence was produced by Juana, when describing the search for water in former times.

\ea\label{ex:tras}
\begingl
\glpreamble cheiku chÿkÿ biseku epenue\\
\gla chÿ-eiku chÿkÿ bi-seku epenue\\
\glb 3-along? arroyo 1\textsc{pl}-dig.hole hole\\
\glft ‘along the arroyo we dug a hole’
\endgl
\trailingcitation{[jxx-p120515l-2.011]}
\xe

The related continuous verb \textit{-eikukuiku} ‘chase, follow’ is much more frequent though. A second related verb, \textit{-eibaneu} ‘pursue, follow, track’, is equally rare.\label{fn:eiku-adposition}}

(\ref{ex:eiku-1}) was elicited from María S. The verbs have a first person singular subject. The second person object of \textit{-eiku} is indexed on this verb. As in the case of purpose-of-motion SVCs, there is no sign of dependency on either of the two verbs.

\ea\label{ex:eiku-1}
\begingl
\glpreamble ukuine niyunu \textup{[}neikubi\textup{]}\\
\gla ukuine ni-yunu nÿ-eiku-bi\\
\glb yesterday 1\textsc{sg}-go 1\textsc{sg}-follow-2\textsc{sg}\\
\glft ‘yesterday I went with you’
\endgl
\trailingcitation{[rxx-e181031l-1]}
\xe

 All non-elicited examples have third person subjects and objects. One of them is (\ref{ex:eiku-2}), which was produced by Juana when telling me the story about her sister’s life. 

\ea\label{ex:eiku-2}
\begingl
\glpreamble i nipiji tiyunumÿnÿ \textup{[}cheiku chima\textup{]}\\
\gla i ni-piji ti-yunu-mÿnÿ chÿ-eiku chi-ima\\
\glb and 1\textsc{sg}-sibling 3i-go-\textsc{dim} 3-follow 3-husband\\
\glft ‘and my poor sister went with her husband’
\endgl
\trailingcitation{[jxx-p120430l-2.063]}
\xe

As is the case with all serial verbs, in \isi{negation}, the negative particle only occurs once, has scope over both verbs and precedes the motion verb. This can be seen in (\ref{ex:eiku-5}), which comes from the story told by María S. about the two men who meet the devil in the woods. While one man speaks with the devil and is finally taken away to be eaten, the other one hides in a tree, does not go with the devil and can escape in the end.

\ea\label{ex:eiku-5}
\begingl
\glpreamble nechikue kuina tiyuna \textup{[}cheika\textup{]}, tikupuiku tikutijikupunu chubiuyae\\
\gla nechikue kuina ti-yuna chÿ-eika ti-kupuiku ti-kutijikupunu chÿ-ubiu-yae\\
\glb therefore \textsc{neg} 3i-go.\textsc{irr} 3-follow.\textsc{irr} 3i-go.down 3i-flee.back 3-house-\textsc{loc}\\
\glft ‘that’s why he didn’t go with him, he climbed down (the tree) and fled back to his home’
\endgl
\trailingcitation{[rxx-n120511l-2.67]}
\xe

There are few examples of this type of SVC in the corpus, but they are found with various speakers. A construction which I have only found with Juana is the occasional combination of the verb \textit{-umu} ‘take’ with \textit{-yunu} to encode the simultaneous actions of somebody moving to a place taking along an item or another person. As is the case with the SVCs including \textit{-eiku}, use of \textit{-umu} also expresses accompaniment, but in a rather passive way. Two examples are given below.

(\ref{ex:go-take-1}) comes from Juana’s account about her sister’s life. It is her sister who is brought to the hospital by one of her sons.

\ea\label{ex:go-take-1}
\begingl
\glpreamble chakachu chÿenu tiyunu \textup{[}chumu hospitalyae\textup{]}\\
\gla chÿ-akachu chÿ-enu ti-yunu chÿ-umu hospital-yae\\
\glb 3-lift 3-mother 3i-go 3-take hospital-\textsc{loc}\\
\glft ‘he lifted his mother and went taking her to the hospital’
\endgl
\trailingcitation{[jxx-p110923l-1.460]}
\xe

(\ref{ex:go-take-2}) was produced by Juana, when telling us what to do with the loam she collected close to Santa Rita in order to make a clay pot.

\ea\label{ex:go-take-2}
\begingl
\glpreamble biyuna \textup{[}buma bubiuyae\textup{]}\\
\gla bi-yuna bi-uma bi-ubiu-yae\\
\glb 1\textsc{pl}-go.\textsc{irr} 1\textsc{pl}-take.\textsc{irr} 1\textsc{pl}-house-\textsc{loc}\\
\glft ‘we go taking it home’
\endgl
\trailingcitation{[jmx-d110918ls-2.04]}
\xe

Approaching the end of this section, I want to have a look at SVCs with motion predicates other than \textit{-yunu}. In general, other motion verbs do not usually enter into a SVC, but a few examples were found that look very similar to the ones presented up to here.

First of all,\is{suppletive imperative} if the imperative motion particle \textit{nabi} ‘go!’ is combined with another verb, this is usually unmarked. However, since \textit{nabi} is not a verb, this does by definition not count as a serial verb construction. One example shall be given nonetheless. It was elicited from Juana.

\ea\label{ex:SVC-nabi}
\begingl
\glpreamble ¡nabi pemusuika!\\
\gla nabi pi-emusuika\\
\glb go.\textsc{imp} 2\textsc{sg}-wash.\textsc{irr}\\
\glft ‘go and wash!’
\endgl
\trailingcitation{[jxx-e190210s-01]}
\xe

In (\ref{ex:come-look}), the verb \textit{-bÿsÿu} is used in combination with another verbal predicate, and in (\ref{ex:arrive-inaugurate}), the verb \textit{-tupunubu} ‘arrive’ combines with a non-verbal predicate\is{non-verbal predication} borrowed from Spanish, so this latter example does not strictly count as a serial \textit{verb} construction either. In both cases, there is no morphosyntactic marking of dependency.

(\ref{ex:come-look}) was produced by María C. when I first met her and tried to explain why I came to Santa Rita.

\ea\label{ex:come-look}
\begingl
\glpreamble pibÿsÿu naka \textup{[}pisamaiku paunaka\textup{]}\\
\gla pi-bÿsÿu naka pi-semaiku paunaka\\
\glb 2\textsc{sg}-come here 2\textsc{sg}-search Paunaka\\
\glft ‘you came here in search for Paunaka’
\endgl
\trailingcitation{[uxx-p110825l.028]}
\xe

(\ref{ex:arrive-inaugurate}) comes from Juana, when she told me that Evo Morales would come to visit Concepción.

\ea\label{ex:arrive-inaugurate}
\begingl
\glpreamble jaa titupunapu tajaitu \textup{[}nauburauna koliseo nauku\textup{]}\\
\gla jaa ti-tupunapu tajaitu nauburau-ina koliseo nauku\\
\glb \textsc{afm} 3i-arrive.\textsc{irr} tomorrow inaugurate-\textsc{irr.nv} multi-purpose.hall there\\
\glft ‘yes, he will come tomorrow to inaugurate the multi-purpose hall’
\endgl
\trailingcitation{[jxx-p150920l.013]}
\xe

Finally, in (\ref{ex:eiku-4}) we have the verb \textit{-muku} ‘sleep’ with the (possible) subsequent motion marker \textit{-nÿmu} in combination with \textit{-eiku} ‘follow’ as a second verb. It was produced by Juana, when I asked her about the verb form \textit{-mukunÿmu} ‘sleep and go(?)’ that she had used in a recording some days before. The subsequent motion marker is not productive in current Paunaka (see \sectref{sec:SubsequentMotion}). If the motion part of the verb with the AM marker\is{associated motion} becomes obscure, it does not make much sense to interpret \textit{-eiku} as encoding an activity anymore (‘they slept following the way’), and what is left is a locative or path interpretation. Sentences like that may have initiated the ongoing development of \textit{-eiku} into a \isi{preposition} (see Footnote \ref{fn:eiku-adposition} above).

\ea\label{ex:eiku-4}
\begingl
\glpreamble timukunÿmunube \textup{[}cheiku chenekÿ\textup{]}\\
\gla ti-muku-nÿmu-nube chÿ-eiku chenekÿ\\
\glb 3i-sleep-\textsc{am.subs}?-\textsc{pl} 3-follow way\\
\glft ‘they slept along the way’ (i.e ‘they slept and went following the way’)
\endgl
\trailingcitation{[jxx-p151016l-2.007]}
\xe

\is{serial verb construction|)}

 In the following section, a few more examples of the motion-cum-purpose construction shall be given.


\subsubsection{The motion-cum-purpose construction}\label{sec:MotionCumPurpose}
\is{motion-cum-purpose construction|(}
\is{dependency marking|(}

Since the motion-cum-purpose construction has already been described in detail in \sectref{sec:SVC_and_MCPC} above, \figref{fig:MCPC} only provides a short summary of its characteristics. The \isi{dislocative} suffix, which is used as a dependency marker in this construction, is described in \sectref{sec:PA}.

\begin{figure}[!ht]
%
%
A motion-cum-purpose construction (MCPC)... 
\begin{itemize}
\item is a monoclausal construction
\item in which a motion predicate, usually \textit{-yunu} ‘go’, is combined with a verb that encodes purpose of motion
\item the dislocative marker is suffixed on the purpose verb as an overt dependency marker
\item both verbs are fully inflected for person and they necessarily have the same subject
\item both verbs are fully inflected for RS; however, the place of RS marking on the dependent verb is the dislocative marker
\item the RS of the second verb is either irrealis or equal to the RS of the motion verb 
\item an adverb or a noun referring to the goal of a motion verb or a noun or pronoun conominating the subject can interrupt the sequence of motion predicate and purpose verb
\end{itemize}
\caption{Characteristics of the MCPC}
%
\label{fig:MCPC}
\end{figure}

\is{dependency marking|)}

%i depue tiyunutu jus- pus- suntabu chitekuputu kÿ- kapunutu kamion, jxx-p120430l-2.205

A few examples follow. First of all, consider (\ref{ex:pa-yunu}): the motion verb \textit{-yunu} ‘go’ comes first and the verb encoding the purpose follows. It takes the \isi{dislocative} marker, which is attached to the complete verb stem. RS is encoded on the \isi{dislocative} marker. It is necessarily irrealis in this case, since the motion verb also has irrealis RS. The example stems from one of the first recordings Swintha made with Juana, who produced this sentence to teach Swintha some words.

\ea\label{ex:pa-yunu}
\begingl 
\glpreamble biyuna \textup{[}bepuikupa\textup{]}\\
\gla bi-yuna bi-epuiku-pa\\ 
\glb 1\textsc{pl}-go.\textsc{irr} 1\textsc{pl}-fish-\textsc{dloc.irr}\\ 
\glft ‘we are going to fish’\\ 
\endgl
\trailingcitation{[jxx-e081025s-1.158]}
\xe


In the following example, we also have two verbs with irrealis RS. It comes from the story about the lazy man told by Miguel. Since the lazybones does not make a field to nurture his family, in the end he cuts off his limbs to give them food, pretending they were \textit{cusi} palm fruits. With (\ref{ex:mocc-2}) the lazy man invites his son to go to the woods to look for the presumed \textit{cusi} fruits.

\ea\label{ex:mocc-2}
\begingl
\glpreamble “biyuna \textup{[}bisemaikupa eka kÿsi\textup{]}”\\
\gla bi-yuna bi-semaiku-pa eka kÿsi\\
\glb 1\textsc{pl}-go.\textsc{irr} 1\textsc{pl}-search-\textsc{dloc.irr} \textsc{dem}a cusi\\
\glft ‘“we go to look for \textit{cusi} palm fruit”’
\endgl
\trailingcitation{[mox-n110920l.085]}
\xe


Like in other purposive constructions, the predicate expressing the purpose usually takes irrealis RS, but may take realis sometimes if the whole event has been realised, which is the case in (\ref{ex:mocc-3}).

In (\ref{ex:mocc-3}), we have two realis predicates and a subject that is placed between the motion and the purpose verb. This sentence was elicited from María S.

\ea\label{ex:mocc-3}
\begingl
\glpreamble tiyunu nÿuchiku \textup{[}tiyÿseikupu baka Monkoxÿyae\textup{]}\\
\gla ti-yunu nÿ-uchiku ti-yÿseiku-pu baka Monkoxÿ-yae\\
\glb 3i-go 1\textsc{sg}-grandfather 3i-buy-\textsc{dloc} cow Moxos-\textsc{loc}\\
\glft ‘my grandfather went to buy cows in Moxos’
\endgl
\trailingcitation{[rxx-e181020le]}
\xe

The subject can also follow the whole construction. This is the case in (\ref{ex:wash-sister}), which was also elicited from María S.

\ea\label{ex:wash-sister}
\begingl
\glpreamble tiyunu \textup{[}temusuikupu\textup{]} netine\\
\gla ti-yunu ti-emusuiku-pu nÿ-etine\\
\glb 3i-go 3i-wash-\textsc{dloc} 1\textsc{sg}-sister\\
\glft ‘my sister went to wash’
\endgl
\trailingcitation{[rxx-e181018le-a]}
\xe

An example, in which the RS of the motion verb is realis, but the purpose verb takes irrealis RS was elicited from Miguel:

\ea\label{ex:go-search-field}
\begingl
\glpreamble ukuine niyunu \textup{[}nisemaikupa juchubu nanaia nisaneina\textup{]}\\
\gla ukuine ni-yunu ni-semaiku-pa juchubu nÿ-ana-i-a ni-sane-ina\\
\glb yesterday 1\textsc{sg}-go 1\textsc{sg}-search-\textsc{dloc.irr} where 1\textsc{sg}-make-\textsc{subord}-\textsc{irr} 1\textsc{sg}-field-\textsc{irr.nv}\\
\glft ‘yesterday I went to look for somewhere to make my future field’
\endgl
\trailingcitation{[mxx-e160811sd.152]}
\xe


As is the case with the SVC (see \sectref{sec:SerialVerbs} above), the MCPC also sometimes builds on motion predicates other than \textit{-yunu}. It is sometimes found with the imperative\is{suppletive imperative} particle \textit{nabi} ‘go!’, marginally also with the \isi{manipulative verb} \textit{-bÿcheiku} ‘send so., make so. do’, but never with cislocative predicates.

One example with \textit{nabi} entering into a MCPC is given below. It was elicited from Juana. A few more examples with the MCPC can be found in \sectref{sec:PA}.

\ea\label{ex:nabi-MCPC}
\begingl
\glpreamble ¡nabi \textup{[}piyÿseikupa kanela\textup{]}! kuina kakuina\\
\gla nabi pi-yÿseiku-pa kanela kuina kaku-ina\\
\glb go.\textsc{imp} 2\textsc{sg}-buy-\textsc{dloc.irr} cinnamon \textsc{neg} exist-\textsc{irr.nv}\\
\glft ‘go and buy cinnamon! There is none’
\endgl
\trailingcitation{[jxx-e191021e-2]}
\xe
\is{motion-cum-purpose construction|)}

\subsubsection{Associated motion verbs combined with motion verbs}\label{sec:AM_MCPC}\is{associated motion|(}

As has been described in \sectref{sec:AssociatedMotion}, Paunaka has a number of associated motion (AM) markers that are attached to a verb to state that the event happens before, after or simultaneous with motion. This would be a prime example of an even more integrating strategy: encoding adverbial relations by verbal \isi{derivation}, thus decreasing syntactic and increasing morphological complexity.\footnote{The AM markers are actually analysed as an inflectional device in this grammar, yet relatively close to derivation, see \sectref{sec:AssociatedMotion}.}

A verb with an AM marker alone does already express an event including motion, but there are cases in which the verb with the AM marker is nonetheless combined with a motion verb. Two examples in which the speaker chose a deranked form for this motion verb have already been given as  (\ref{ex:stat-i-1}) and (\ref{ex:hungry-go}) in \sectref{sec:EmbeddedAC_bare}. The first was contrasted with (\ref{ex:stat-noi-1}), which has a verb marked for concurrent motion juxtaposed to a motion verb. There are a few more examples in the corpus, where verbs marked for AM combine with motion predicates, resulting in (almost?) tautological statements.

In (\ref{ex:come-fly-come}), the verb with the concurrent cislocative AM marker is surrounded by the non-verbal predicate \textit{kapunu}. This sentence was elicited from María S.

\ea\label{ex:come-fly-come}
\begingl
\glpreamble kapunu tibÿbÿkukukÿupunu kapunu\\
\gla kapunu ti-bÿbÿku-kukÿupunu kapunu\\
\glb come 3i-fly-\textsc{am.conc.cis} come\\
\glft ‘it comes, flying it comes, it comes’
\endgl
\trailingcitation{[rmx-e150922l.062]}
\xe

A similar example is (\ref{ex:go-field-talk}), which can be analysed as including a temporal clause juxtaposed to a main clause. This main clause has a verb with a translocative concurrent AM marker combined with the motion verb \textit{-yunu}. The information conveyed doubles. The sentence was produced by Juana when telling me about the past.

\ea\label{ex:go-field-talk}
\begingl
\glpreamble biyuna asaneti bichujikukukÿu biyunu asaneti\\
\gla bi-yuna asaneti bi-chujiku-kukÿu bi-yunu asaneti\\
\glb 1\textsc{pl}-go.\textsc{irr} field 1\textsc{pl}-speak-\textsc{am.conc.tr} 1\textsc{pl}-go field\\
\glft ‘when we went to the field, going talking we went to the field’
\endgl
\trailingcitation{[jxx-p120515l-1.168]}
\xe

I want to emphasise that not every verb with an AM marker is combined with a motion predicate. In many cases, the AM marker alone conveys the motion part of the meaning. However, it seems to be frequent cross-linguistically that motion verbs combine with verbs taking an AM marker \citep[128]{Rose2015}. It is well possible that the two possibilities, i.e. motion verbs combined with verbs taking AM markers and verbs taking AM markers alone, have different functions in discourse. This remains a topic for further studies.\is{associated motion|)}\is{motion predicate|)}\is{adverbial relation|)}

The following section is about complement relations.

