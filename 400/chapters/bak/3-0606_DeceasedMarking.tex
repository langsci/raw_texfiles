%!TEX root = 3-P_Masterdokument.tex
%!TEX encoding = UTF-8 Unicode
\section{Deceased marking}\label{sec:Deceased}\is{deceased marking|(}

This section is about different possibilities to express on a noun that a person is deceased. Three different markers can be used in this case, \textit{-ini}, \textit{-kue} and \textit{-bane}, which occur with different types of nouns. Only the last of them is also used with other parts of speech\is{word class} as a \isi{remote past} marker.

\tabref{table:DeceasedMarkers} provides a summary of the three forms used for deceased marking.

\begin{table}
\caption{Markers for ‘deceased’}

\begin{tabularx}{\textwidth}{llQ}
\lsptoprule
Marker & Gloss & Usage \cr
\midrule
\textit{-bane} & \textsc{rem} & general remote (past) marker, used for deceased marking on kinship terminology, then mainly on referential terms \cr
\textit{-ini} & \textsc{dec} & on kinship terminology, mainly on endearment/vocative terms \cr
\textit{-kue} & \textsc{dec.pn} & only found on proper names, possibly of Tupi-Guarani origin \cr
\lspbottomrule
\end{tabularx}

\label{table:DeceasedMarkers}
\end{table}

It is not uncommon among \isi{Arawakan languages} to mark on a human noun that the referent is deceased (\citealp[cf.][130, 276, 313]{Ramirez2001}; %Baniwa-Curripaco y Tariano, Piapoco, Achagua
\citealt[153, 157]{OlzaZubiri2004}; \citealt[115]{Danielsen2007}; \citealt[289]{Brandao2014}; \citealt[35]{Jorda2014}; \citealt[80, 81]{Rose2014a}; \citealt[356]{Mihas2015}), and this is also found in non-related languages in Amazonia\is{Amazonian language} (e.g. in Mosetén, see \citealt[75]{Sakel2004}, and Hup, see \citealt[353]{Epps2008}). 

As for \isi{Arawakan languages}, \citet[382]{Payne1991} states that “Wise (1988a) reconstructs one other classifier: \textit{*mini} meaning ‘dead, past, abandoned’, which in most northern languages retains a suffix similar to \textit{-mi}, and in most southern languages a suffix similar to \textit{-ni}. A fuller form was found in Maipure \textit{-mine} and Baré \textit{-amini}”. The Paunaka marker that relates to this is \textit{-ini}. This deceased marker attaches to kinship terms, sometimes to the referential forms, but mostly to the endearment/vocative forms\is{endearment|(} (see \tabref{table:Vocatives} in \sectref{sec:Inalienables}). (\ref{ex:deceased-1}) and (\ref{ex:deceased-2}) show its use on endearment forms. 

(\ref{ex:deceased-1}) comes from Juana who was talking about what she did with her grandmother in the old days.

\ea\label{ex:deceased-1}
\begingl
\glpreamble micha echÿu yeyeini\\
\gla micha echÿu yeye-ini\\
\glb good \textsc{dem}b granny-\textsc{dec}\\
\glft ‘my late granny was a good person’
\endgl
\trailingcitation{[jxx-p120430l-1.059]}
\xe

(\ref{ex:deceased-2}) is also from Juana. It comes from her account about their grandparents’ journey to Moxos to buy cows.

\ea\label{ex:deceased-2}
\begingl 
\glpreamble beintechÿ baka chiyÿseie chÿchÿini\\
\gla beintechÿ baka chi-yÿseie chÿchÿ-ini\\ 
\glb twenty cow 3-purchase grandpa-\textsc{dec}\\ 
\glft ‘it were twenty cows that my late grandpa bought’\\ 
\endgl
\trailingcitation{[jxx-p151016l-2.081-083]}
\xe
\is{endearment|)} 

%Old Baure had optative-ni (Magio 20), Terena also!
%-bane -> Old Baure Magio 26, cuparcari: 97; -ini for deceased Magio: 50, C. 103

 Paunaka’s deceased marker is \textit{-ini} is identical in form to the frustrative\is{frustrative|(} marker (see \sectref{sec:Frust_avertive_optatiev}). It is unclear to me at this stage of research whether \textit{-ini} should be described as one polysemous marker or as two homophonous markers. For the time being, I opt for an analysis of two homophonous markers with different glosses (\textsc{dec} for ‘deceased’, \textsc{frust} for ‘frustrative’).
According to the analysis of \citet[490--491]{Overall2017}, frustrative is often extended to the expression of discontinuous past, i.e. a past situation that was interrupted counter to the expectation of the speaker. Deceased people, obviously, belong to a discontinuous past.\is{past reference} Although the fact that people die may not be unexpected – at least if they are old –, the death of a beloved person causes pain and sorrow for the bereaved people, and frustrative is also connected to negative emotions (see \sectref{sec:Frust_avertive_optatiev}). However, even in the examples given by  \citet[490--492]{Overall2017}, it is in most cases not the frustrative alone that establishes a discontinuous past reading, but the frustrative together with another specialised marker.\footnote{It would go too far to explain this in detail here, but I have reasons to believe that deceased or remote past markers of two of the Arawakan examples cited by \citet[]{Overall2017} in his section on discontinuous past remained unrecognised by the author.}  %Overall’s examples (31) and (33) (p.491) also contain deceased or remote past markers that were not recognised by the authors. %Consider the words \textit{zatyokoenaene} and \textit{zeyenaene} of the Paresi that end in \textit{ene}, which is a deceased suffix according to Brandao, and the Ashéninka word has -ni, which is a remote past marker.

Comparing the form of the Paunaka deceased marker with other Arawakan languages \is{Arawakan languages|(} reveals that a lot of languages have a cognate form to express the meaning of ‘deceased’, and besides Paunaka, only the \isi{Mojeño languages} and \isi{Terena} have an identical (or homophonous) or similar form for frustrative marking.\footnote{In Trinitario \textit{-ini} is a general (discontinuous?) past marker, which occurs on proper names and human nouns to mark a person as deceased, but also on predicates, where it often conveys a counterfactual meaning together with irrealis \citep[cf.][80,81]{Rose2014a}. Ignaciano, apparently, has two markers \textit{-hini} (or \textit{-'ini}) for counterfactual (or frustrative) and \textit{-(i)ni} which marks (nominal?) past (\citealt[cf.][153, 157]{OlzaZubiri2004}; \citealt[35]{Jorda2014}). Baure’s marker\is{Baure} \textit{-in} ‘dead (family member)’ only marks people as deceased \citep[cf.][115]{Danielsen2007}, although in Old \isi{Baure}, the variety documented by the Jesuits in the 18th century, it was also used as a nominal past marker with non-human nouns (Danielsen 2020, p.c.). \isi{Terena} has two markers \textit{-ni} and \textit{-nini} which express a number of meanings related to frustrative (\citealt[cf.][55, 84]{ButlerEkdahl2014}; \citealt[7]{Butler2007}), but the deceased marker is \textit{-ikene} (de Carvalho 2017, p.c.). %26.9.2017
 The Northern Kampan languages have \textit{-ni} whose “basic function is to specify the ceased existence of a human entity", although in some languages it has extended functions and can even be used on verbs as a remote past marker \citep[793]{Mihas2017}. The frustrative markers have a totally different form. More distantly related Paresi has a particle \textit{ene} for nominal past, which is also used as a deceased marker \citep[289]{Brandao2014}, but another particle for frustrative.}\is{Arawakan languages|)}\is{frustrative|)}


In the examples (\ref{ex:deceased-1}) and (\ref{ex:deceased-2}), it is only the deceased marker on the noun that specifies that the people referred to have passed away. The deceased marker is also found on a noun \textit{kuineini} ‘deceased’, and often this noun precedes the noun referring to the deceased person. The deceased marker (or one of the other markers described below) can be added to the noun as in (\ref{ex:kuineini-ini}) or be left out as in (\ref{ex:kuineini-bare}). In that case, \textit{kuineini} is the only expression of ceased existence of the person in question.\footnote{In local Spanish, there is a convention to prepose the noun \textit{finado/-a} ‘deceased one’ before the noun referring to the deceased referent, which can be a kinship term or personal name.} 

(\ref{ex:kuineini-ini}) is from an account of Miguel about the history of Santa Rita. The founder of Santa Rita is the grandfather of the Supepí siblings and their father was among the twelve families that came to live in the village in the 1950s.

\ea\label{ex:kuineini-ini}
\begingl 
\glpreamble kapunutu kuineini taitaini pero kapununube dose familia\\
\gla kapunu-tu kuineini taita-ini pero kapunu-nube dose familia\\ 
\glb come-\textsc{iam} deceased dad-\textsc{dec} but come-\textsc{pl} twelve family\\ 
\glft ‘my late father had come (here), but twelve families came (altogether)’\\ 
\endgl
\trailingcitation{[mxx-p110825l.056]}
\xe

(\ref{ex:kuineini-bare}) is from a listing by María C. of people she knew who were killed by sorcery.

\ea\label{ex:kuineini-bare}
\begingl 
\glpreamble nechÿu kapunu kuineini kupare Tieko\\
\gla nechÿu kapunu kuineini kupare Tieko\\ 
\glb \textsc{dem}c come deceased fellow Diego\\ 
\glft ‘next came late fellow Diego’\\ 
\endgl
\trailingcitation{[ump-p110815sf.640]}
\xe

Alternatively to \textit{-ini}, the remote marker\is{remote past|(} \textit{-bane} (see \sectref{sec:RemotePast}) can also be used on human nouns to signal that the referent is deceased. It is mainly attached to referential kinship terms. These nouns occur with \textit{-bane} much more often than with \textit{-ini}. The noun \textit{-enu} ‘mother’ is even exclusively combined with \textit{-bane} in my corpus and never with \textit{-ini}. One example of \textit{-bane} on kinship terms is given in (\ref{ex:bane-deceased}), in which Juana tells me about the move of her late parents to \isi{Altavista}.

\ea\label{ex:bane-deceased}
\begingl 
\glpreamble te tiyununubetu tanÿma eka nÿabane nÿenubane te tijechikunubetu chukuyae patrun nauku Turuxhiyae\\
\gla te ti-yunu-nube-tu tanÿma eka nÿ-a-bane nÿ-enu-bane te ti-jechiku-nube-tu chi-chuku-yae patrun nauku Turuxhi-yae\\ 
\glb \textsc{seq} 3i-go-\textsc{pl}-\textsc{iam} now \textsc{dem}a 1\textsc{sg}-father-\textsc{rem} 1\textsc{sg}-mother-\textsc{rem} \textsc{seq} 3i-move-\textsc{pl}-\textsc{iam} 3-side-\textsc{loc} patrón there Altavista-\textsc{loc}\\ 
\glft ‘now my late father and my late mother went (away), they moved close to their \textit{patrón} there in Altavista’ \\ 
\endgl
\trailingcitation{[jxx-e150925l-1.248]}
\xe

Another example is (\ref{ex:new23-Trion}) from Juan C. telling about the various relocations in his life. I could not find out where Trion is or was, I suppose it is among the places that were renamed or abandoned.

\ea\label{ex:new23-Trion}
\begingl
\glpreamble kuineini niuchikubane tiyunu naka Trion\\
\gla kuineini ni-uchiku-bane ti-yunu naka Trion\\
\glb deceased 1\textsc{sg}-grandfather-\textsc{rem} 3i-go here Trion\\
\glft ‘my late grandfather went to Trion’
\endgl
\trailingcitation{[mqx-p110826l.440-442]}
\xe

In (\ref{ex:kuineini-ini-2}) María C. uses all three strategies described so far, \textit{-bane} on a referential kinship term, \textit{-ini} on an endearment form and an additional \textit{kuineini} preposed to it. She describes her exact kinship relation to the Supepí siblings here. 

\ea\label{ex:kuineini-ini-2}
\begingl 
\glpreamble ja chÿenujinube ekanube chipijibane kuineini mimini\\
\gla ja chÿ-enu-ji-nube eka-nube chi-piji-bane kuineini mimi-ini\\ 
\glb \textsc{afm} 3-mother-\textsc{col}-\textsc{pl} \textsc{dem}a-\textsc{pl} 3-sibling-\textsc{rem} deceased mum-\textsc{dec}\\ 
\glft ‘yes, their mother was the late sister of my late mother’\\ 
\endgl
\trailingcitation{[cux-c120410ls.124-125]}
\xe

As stated above, \textit{-bane} is also used as a general remote past marker (see \sectref{sec:RemotePast}). It is then mainly associated with predicates (or with the whole proposition). In some cases, use of \textit{-bane} with nouns can possibly be analysed as a nominal past\is{nominal tense|(} marker with the meaning ‘former, ex-, old’. In most cases, however, it is not possible to distinguish a predicative and a referential use of the marker. This is reminiscent of the overlap of predicative and referential use of the non-verbal irrealis marker (see \sectref{NominalRS}), but in the case of the remote marker, ambiguity is enhanced by the fact that it can float in the clause in predicative use: it mostly occurs on the predicate, but not always. 

Consider (\ref{ex:former-house}) and the two translations given. One suggests a predicative use of the marker, the other one a referential use. The example comes from a story by María S. about how the tortoise got its carapace: the tortoise does not want to leave her house in the story in order to pay homage to newborn Jesus. As a consequence, she is punished by having her house fixed on her back.

\ea\label{ex:former-house}
\begingl 
\glpreamble nechikue tepajÿku tanÿma chitapu chubiubane\\
\gla nechikue ti-epajÿku tanÿma chi-tapu chÿ-ubiu-bane\\ 
\glb therefore 3i-stay now 3-scales 3-house-\textsc{rem}\\ 
\glft ‘therefore her carapace stays now, which was her house before’\\or: ‘therefore her the carapace stays now, (which is) her former house’\\ 
\endgl
\trailingcitation{[rxx-n121128s.24]}
\xe

Another example, in which \textit{-bane} could be analysed as a nominal past marker is (\ref{ex:oldTuruxhi}). \isi{Altavista} does not exist anymore, at least not as a big estate dedicated to agriculture based on forced labour. Juana told me that her father had found two of the cows her grandparents had been deprived of by \textit{karay} somewhere in the pampa and took them to Altavista, where he lived at that time.

\ea\label{ex:oldTuruxhi}
\begingl
\glpreamble kapunu nÿabane te chumu nauku Turuxhiyaebane\\
\gla kapunu nÿ-a-bane te chÿ-umu nauku Turuxhi-yae-bane\\
\glb come 1\textsc{sg}-father-\textsc{rem} \textsc{seq} 3-take there Altavista-\textsc{loc}-\textsc{rem}\\
\glft ‘my late father came and took them to old Altavista’
\endgl
\trailingcitation{[jxx-e150925l-1.238]}
\xe
\is{remote past|)}
\is{nominal tense|)}

There is yet a third marker that can be used to signal the ceased existence of people. This marker, \textit{-kue}, is almost exclusively attached to proper names in my data, i.e. it does not occur with kinship terms or other human nouns, and proper names do not occur with any of the other two markers previously described (both statements with one exception each). It is thus glossed ‘\textsc{dec.pn}’, a deceased marker for proper names. The marker is probably of Tupi-Guarani origin, since it is very similar phonetically to the nominal past marker found in some of these languages: Guarayu has a past marker \textit{-kwer} (Bischoffberger 2017, p.c.), Guarasu uses \textit{-kwe/-we} as a “disconnected” marker, which marks nominal past among other things \citep[237--238]{RamirezAL2017}, in Bolivian Guaraní, the deceased marker is \textit{-gwe/-kwe} \citep[339]{Gustafson2014}, and Paraguayan Guaraní has a nominal (or referential) past marker \textit{-kue} \citep[34]{Nordhoff2004}. 

An example is given in (\ref{ex:kue-deceased}). It comes from María C. Note that she omits the third person marker here, something she does frequently.

\ea\label{ex:kue-deceased}
\begingl 
\glpreamble tupunubu kuineini Pernatokue\\
\gla tupunubu kuineini Pernato-kue\\ 
\glb arrive deceased Fernando-\textsc{dec.pn}\\ 
\glft ‘late Fernando arrived’\\ 
\endgl
\trailingcitation{[ump-p110815sf.412]}
\xe

I want to conclude this section with (\ref{ex:kue-deceased-2}), which comes from Juana. She was thinking about which tale she could tell us (when we had asked her to tell one). She considers telling one story her brother told her, who was a very good storyteller.\footnote{Remember that \textit{taita} literally means ‘dad’, but is used as a respectful form for all male people, see \sectref{sec:Inalienables}.}

\ea\label{ex:kue-deceased-2}
\begingl
\glpreamble chÿkueteabane taita Tubusiukue\\
\gla chÿ-kuetea-bane taita Tubusiu-kue\\
\glb 3-tell-\textsc{rem} dad Tiburcio-\textsc{dec.pn}\\
\glft ‘late Tiburcio told it in the old days’
\endgl
\trailingcitation{[jmx-n120429ls-x5.038]}
\xe
\is{deceased marking|)}

The following section is about diminutives, both marked on the noun and on other parts of speech.

