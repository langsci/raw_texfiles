\section{Reality status}\label{sec:RealityStatus}
\is{reality status|(}

While tense and aspect play a minor role in Paunaka (see \sectref{sec:AspectTense}), reality status (RS) is obligatorily expressed on every predicate, and in addition, there is \isi{nominal irrealis} marking on nouns.\footnote{This section is partly based on \citet[]{DanielsenTerhartSubm}, but focuses the discussion on the Paunaka language only.} In all cases of non-verbal reality status marking, \textit{-ina}\is{non-verbal irrealis marker} expresses irrealis and realis is unmarked. On verbs, irrealis is associated with \textit{a} and realis is marked by the absence of irrealis marking, but associated with \textit{u} on active verbs, as can be seen in (\ref{ex:new23-realirr}), in which the stem of the realis verb\is{verbal stem} in (\getfullref{ex:new23-realirr.1}) ends in \textit{u} and this vowel changes to \textit{a} in the irrealis form of (\getfullref{ex:new23-realirr.2}). A more detailed explanation is given below in \sectref{sec:VerbalRS}.

\ea\label{ex:new23-realirr}
  \ea\label{ex:new23-realirr.1}
\begingl
\glpreamble bimuku\\
\gla bi-muku\\
\glb 1\textsc{pl}-sleep\\
\glft ‘we sleep/slept’
\endgl
  \ex\label{ex:new23-realirr.2}
\begingl
\glpreamble bimuka\\
\gla bi-muka\\
\glb 1\textsc{pl}-sleep.\textsc{irr}\\
\glft ‘we will/must/may sleep’
\endgl
%\trailingcitation{[]}
\z
\xe

RS has been defined by \citet[56]{Elliott2000} as “a grammatical category […] with the binary distinction of realis and irrealis”, where the notion of realis marks a “perceived certainty of the factual reality of an event's taking place, while irrealis is used to identify that an event is perceived to exist only in an imagined or non-real world” \citep[67]{Elliott2000}. 
The validity of RS as a unifiable category has been doubted severely (see e.g. \citealt{Bybee1998}; \citealt{deHaan2012}); however, \citet[]{Michael2014,Michael2014a} has demonstrated that RS \textit{is} a significant category in Nanti, an Arawakan language\is{Arawakan languages} of the Kampan branch. \citet[]{DanielsenTerhartSubm} have shown that this is equally true for some of the \isi{Southern Arawakan} languages.

On predicates, \isi{realis} RS encodes non-hypothetical non-future events, while \isi{irrealis} expresses an array of functions. It is found on predicates with \isi{future reference}, in negative clauses,\is{negation} in hypothetical constructions, to mark imperative and polite directives,\is{directive speech act} obligation, possibility and ability, optatives and is also found on verbal complements of certain complement-taking verbs (see \sectref{sec:Unmarked_CCs}). In addition, it can express \isi{habitual} events in procedural texts and is sometimes, but not always, also used to express \isi{habitual} events in the past.\is{past reference} The system is thus consistent with the findings by \citet[]{Michael2014,Michael2014a} illustrated in \tabref{table:Michael_RS_semantics}, with two differences. First, in expressions of \isi{uncertainty} both RS can occur in Paunaka. There is a special uncertainty marker \textit{-kena} (see \sectref{sec:Epistemic_Mod}), so that there is no need to express uncertainty with an irrealis predicate. RS then rather provides information about the temporal reference or polarity of the event that is marked as uncertain.\is{uncertainty}\footnote{I believe that this is a mechanism that is found with any additional marker that covers some of the possible semantic functions of irrealis RS. For example, Trinitario\is{Mojeño Trinitario} and \isi{Terena} both have a special future marker, and in both languages verbs marked for future can take both RSs for reasons other than temporal reference: In Trinitario, future verbs take irrealis marking if there is negation of the event, in \isi{Terena} the choice of RS for the future verb reflects the certainty over the fulfilment of the event (cf. \citealt[262]{EkdahlGrimes1964}; \citealt[228--229]{Rose2014} and 2015 p.c.; \citealt{DanielsenTerhartSubm}). Another example is \isi{negation}. To my knowledge, no language marks negative polarity with irrealis only. The existence of one or more negative markers opens an opportunity to make finer distinctions, e.g. between negated realis (non-future, actual etc.) and negated irrealis (future, hypothetical, directive etc.). Some languages -- like Nanti -- make use of this possibility, others -- like Paunaka --  only partly make use of it, yet others may not use the possibility at all. A cross-linguistic comparison of how RS systems interact with other markers of (un)realness is a topic that awaits further research.\label{ftn:irrealis}}%"In some languages every finite clause must obligatorily be marked for either realis or irrealis, in much the same way as every finite clause in English or German carries marking for tense." (Elliott 2000:59).
 
\begin{table}
\caption[Semantic parameters to be considered in RS marking]{Semantic parameters to be considered in RS marking, adapted and adjusted from Michael (\citeyear[252, 266]{Michael2014} and \citeyear[189]{Michael2014a})}

\begin{tabularx}{\textwidth}{QlQ}
\lsptoprule
Semantic parameter & Realis marking & Irrealis marking \\
\midrule
temporal reference & non-future & future \\
\tablevspace
polarity & positive & negative \\
\tablevspace
hypotheticality & actual & hypothetical, (conditional), (counterfactual) \\
\tablevspace
factuality/epistemic modality & certainty & uncertainty \\
\tablevspace
speaker-oriented modality & --- & imperative, polite directive/ exhortative \\
\tablevspace
agent-oriented modality & --- & obligation, need \\
\tablevspace
prospectiveness & --- & purposive, prospective complement \\
\lspbottomrule
\end{tabularx}

\label{table:Michael_RS_semantics}
\end{table}


Second, habitual\is{habitual|(} is at least partly encoded by irrealis in Paunaka, while \citet[284]{Michael2014} argues that encoding of habitual vs. non-habitual belongs to another category, i.e. temporal definiteness. In Nanti, the argument goes, the “habitual construction crucially takes realis marking, in accord with our notional definition of realis and irrealis, since habitual constructions denote repeated realisation of some situation” \citep[271]{Michael2014}. Nanti has a specialised marker to encode habitual, and as I have argued, markers specialised for a certain kind of unrealness interact with RS and may thus have an influence on the encoding of it (see Footnote \ref{ftn:irrealis}). The question whether or not temporal definiteness should be considered a semantic parameter that triggers differential RS marking should therefore be addressed again in a cross-linguistic study of RS marking. Habitual is among the “additional semantic contexts where, in at least some languages, irrealis marking has saliency” as suggested by \citet[70]{Elliott2000}.\is{habitual|)}

Last but not least, in addition to the semantic contexts cited in the literature on RS, some speakers of Paunaka also produce irrealis verb forms when they are asked to translate Spanish verbs or clauses in elicitation sessions. Of course, the predicates in these translation tasks often refer to events that are not actual, not realised so that translation with an irrealis verb form is a logical consequence in some situations, at least for some of the speakers. %an example on how some people answer with irrealis forms in elicitation is: rxx-e120511l.052


The remainder of this section is dedicated to discussion of the formal expression of RS on verbs in \sectref{sec:VerbalRS} and the semantics of the RS system in \sectref{RS:parameters}. For marking RS on non-verbal predicates, see \sectref{sec:NonVerbalPredication} and for nominal irrealis see \sectref{NominalRS}. Paunaka’s RS marking on verbs is typical for a \isi{Southern Arawakan} language, both regarding the shape and behaviour of markers as well as the semantic parameters that trigger irrealis marking \citep[cf.][]{DanielsenTerhartSubm}.


\subsection{Form of the reality status markers}\label{sec:VerbalRS}

On verbs, irrealis is associated with the vowel \textit{a}. Stative verbs\is{stative verb|(}  take a prefix \textit{a-} that is inserted directly before the verb stem,\is{verbal stem} see the following example, where the verb in (\getfullref{ex:IRR-prefix.1}) is unmarked and the verb in (\getfullref{ex:IRR-prefix.2}) takes the irrealis prefix.

\ea\label{ex:IRR-prefix}
  \ea\label{ex:IRR-prefix.1}
 \begingl 
\glpreamble tikutiu\\
\gla ti-kutiu\\ 
\glb 3i-be.ill\\ 
\glft ‘he was ill’\\ 
% \trailingcitation{[mxx-e090728s-3.019]}
\endgl
  \ex\label{ex:IRR-prefix.2}
 \begingl
\glpreamble kuina takutiu\\
\gla kuina ti-a-kutiu\\
\glb \textsc{neg} 3i-\textsc{irr}-be.ill\\
\glft ‘he was not ill’\\
% \trailingcitation{[jxx-p120430l-2.437]}
\endgl
\z
\xe
\is{stative verb|)} 

Active verbs\is{active verb|(} take a suffix \textit{-a} which replaces the last vowel \textit{u} of the verb stem\is{verbal stem} in most cases as in the following example, where (\getfullref{ex:IRR-yunu.1}) shows an active verb with realis RS and (\getfullref{ex:IRR-yunu.2}) has the same verb with irrealis RS. I decided not to separate the irrealis marker from the stem or suffix by a dash in my analysis and not to overtly gloss realis RS at all for reasons that will become apparent in the discussion below. 

\ea\label{ex:IRR-yunu}
  \ea\label{ex:IRR-yunu.1}
 \begingl 
\glpreamble niyunu\\
\gla ni-yunu\\ 
\glb 1\textsc{sg}-go\\ 
\glft ‘I go, I went’\\ 
\endgl
  \ex\label{ex:IRR-yunu.2}
 \begingl
\glpreamble niyuna\\
\gla ni-yuna\\
\glb 1\textsc{sg}-go.\textsc{irr}\\
\glft ‘I will/can/may/must go’
\endgl
\z
\xe

\is{fusion|(}One peculiarity of RS marking is that a number of grammatical markers attract the irrealis suffix, i.e. the \isi{reciprocal}, \isi{continuous}, \isi{additive}, and associated motion \is{associated motion|(} (AM) and related markers (see \figref{fig:VerbTemplate-Active} in the introduction to this Chapter). This is also why the status of realis marking on active verbs is less clear. There is an association between realis and the final vowel \textit{u} of the verb stem and the markers mentioned above. Some of these markers can be inserted between the last vowel of the verb stem minus the \isi{thematic suffix} and the reality status marking, i.e. they replace the thematic suffix. The final vowel of these markers is then either realis \textit{u} or irrealis \textit{a}, see (\ref{ex:irr-no-th}), where this is exemplified with the prior motion marker.

\ea\label{ex:irr-no-th}
  \ea
\begingl
\glpreamble pinipunu\\
\gla pi-ni-punu\\
\glb 2\textsc{sg}-eat-\textsc{am.prior}\\
\glft ‘you went to eat’
\endgl
  \ex
\begingl
\glpreamble pinipuna\\
\gla pi-ni-puna\\
\glb 2\textsc{sg}-eat-\textsc{am.prior.irr}\\
\glft ‘you will go to eat’
\endgl
\z
\xe

However, instead of replacing the thematic suffix,\is{thematic suffix|(} all of the markers mentioned above can also follow the thematic suffix. Irrealis is only marked once in this case, i.e. on the marker that follows the thematic suffix. I will continue to exemplify this with the prior motion marker here. Thus in  (\ref{ex:real-irr-clash}), we have a verb with a stem ending in \textit{u} and an AM marker ending in \textit{a}, resulting in a clash between RS markings if \textit{u} were understood as a proper marker of realis:

\ea\label{ex:real-irr-clash}
\begingl 
\glpreamble pinikupuna\\
\gla ? pi-nik-u-pun-a\\ 
\glb ? 2\textsc{sg}-eat-\textsc{real}-\textsc{am.prior}-\textsc{irr}\\ 
\glft ‘you will go to eat’\\ 
\endgl
%\trailingcitation{}
\xe

It was cases like the one in (\ref{ex:real-irr-clash}) and the fact that stative verbs  do not have an \textit{u} that led \citet[227]{Rose2014} to conclude for Trinitario\is{Mojeño Trinitario} that the final \textit{o} (corresponding to Paunaka’s final \textit{u}) is phonological material of the stem that is deleted if an irrealis suffix is attached. The example above would thus be analysed as (\ref{ex:real-irr-trin}).

\ea\label{ex:real-irr-trin}
\begingl 
\glpreamble pinikupuna\\
\gla ? pi-niku-punu-a\\ 
\glb ? 2\textsc{sg}-eat-\textsc{am.prior}-\textsc{irr}\\ 
\glft ‘you will go to eat’\\ 
\endgl
%\trailingcitation{}
\xe
\is{associated motion|)}

However, in some cases, \textit{-u} seems to work as a proper marker of realis in Paunaka. There are three arguments in favour of such an analysis. First of all, the distributive\is{distributive|(} marker can be inserted in the slot of the thematic suffix instead of following it (see \sectref{sec:Verbs_3PL}). In this case, an RS suffix follows directly. Since we know that the form of the distributive marker is \textit{-jane}, there is no reason to assume that any default vowel is involved in this case, thus \textit{-u} must be a suffix. (\ref{ex:jane-real}), which comes from an elicitation session with María S., includes two verbs a distributive marker and a deleted thematic suffix. The first verb has irrealis RS, thus \textit{-a} is suffixed to \textit{-jane}; the second verb has realis RS signalled by the presence of a suffix \textit{-u} following \textit{-jane}.

\ea\label{ex:jane-real}
\begingl 
\glpreamble kuina chinijanea takÿra, bÿrÿsÿi si chinijaneu\\
\gla kuina chi-ni-jane-a takÿra bÿrÿsÿi si chi-ni-jane-u\\ 
\glb \textsc{neg} 3-eat-\textsc{distr}-\textsc{irr} chicken guava yes 3-eat-\textsc{distr}-\textsc{real}\\ 
\glft ‘the chickens don’t eat it, (but) guava, yes, they eat.’\\ 
\endgl
 \trailingcitation{[rxx-e121128s-3.24]}
\xe
\is{distributive|)}
\is{thematic suffix|)}

The other construction where \textit{-u} occurs as a separate suffix is the \isi{deranked verb} form with \textit{-i}, see \sectref{sec:Subordination-i}. (\ref{ex:subREAL}) offers one example. It comes from María C. speaking about a sorcerer.

\ea\label{ex:subREAL}
\begingl 
\glpreamble pariki chinikiu\\
\gla pariki chi-nik-i-u\\ 
\glb many 3-eat-\textsc{subord}-\textsc{real}\\ 
\glft ‘it was many that he ate’\\ 
\endgl
 \trailingcitation{[ump-p110815sf.597]}
\xe

Finally, the verb \textit{-anau} ‘make’ has an irrealis form that deletes the final vowel \textit{u}, see (\ref{ex:anaIRR}) with two sentences from Juana that both deal with the production of a clay pot (with (\getfullref{ex:anaIRR.2}) being uttered a few days earlier than (\getfullref{ex:anaIRR.1})).

\ea\label{ex:anaIRR}
  \ea\label{ex:anaIRR.1}
 \begingl 
\glpreamble pero ukuine nanau\\
\gla pero ukuine nÿ-anau\\ 
\glb but yesterday 1\textsc{sg}-make\\ 
\glft ‘but yesterday I (finally) made one’\\ 
\endgl
 \trailingcitation{[jxx-d110923l-1.15]}
  \ex\label{ex:anaIRR.2}
 \begingl
\glpreamble nana barereki\\
\gla nÿ-ana barereki\\
\glb 1\textsc{sg}-make.\textsc{irr} clay.pot\\
\glft ‘I will make a clay pot’
\endgl
 \trailingcitation{[jmx-d110918ls-1.022]}
\z
\xe

There is no neat way to capture the complexities of realis marking in active verbs in the glosses, since \textit{u} seems to be a hybrid of default vowel and realis marker proper.\footnote{\label{fn:VowelHarmony}A diachronic remark on irrealis marking \citep[cf. also][102--103]{DanielsenTerhartSubm}: It is perfectly possible that irrealis marking in all \isi{Southern Arawakan} languages was once expressed by changing every vowel /o/ of a verb or of an active verb to /a/. This kind of vowel harmony is still found in \isi{Terena} today \citep[35]{ButlerEkdahl2012} and it was reported for Old \isi{Baure}, the variety of \isi{Baure} documented by the Jesuits in the 18th century (and published by \citealt{AdamLeclerc1880}). Modern Trinitario\is{Mojeño Trinitario} has one verb that displays vowel harmony triggered by irrealis RS (Rose 2015, p.c.); however, there is no report of vowel harmony in irrealis verbs in the grammar by \citet[]{Marban1894} (i.e. for Old Mojeño).\is{Mojeño languages} Since the time Old \isi{Baure} and Old Mojeño were described by the Jesuits, \isi{Baure} has completely lost RS as a grammatical category. In Ignaciano,\is{Mojeño Ignaciano} RS has become largely invisible because the vowels /o/ and /a/ collapsed. In Trinitario\is{Mojeño Trinitario} /o/ may best be described as a default vowel. This is also a good analysis for the Paunaka vowel /u/ in some cases, while in other cases /u/ is reanalysed as a realis marker as the discussion above has shown. Only \isi{Terena} then seems to have maintained the old system.} My practical solution is as follows: I consider each and every verb stem to have two realisations, a default one and an irrealis one. The same is true for the \isi{reciprocal}, continuous, \isi{additive}, and AM and related markers.\is{associated motion} A verb has realis RS in the absence of any irrealis marking. The irrealis vowel \textit{a} is not considered a separate suffix because suffixes or – more generally speaking – markers starting with a vowel usually form a diphthong\is{vowel sequence} with a preceding vowel rather than deleting it (except when they attach to a diphthong, but then it is the first vowel of the marker that vanishes). I would thus consider irrealis marking to be a case of incipient fusion\is{fusion|)} in an otherwise strictly agglutinative language. The analysis adopted here differs in this point from the one proposed by \citet[227]{Rose2014} for Trinitario,\is{Mojeño Trinitario} which has a lot of vowel elision in general. In Paunaka, a verb including a \isi{thematic suffix} + AM marker\is{associated motion} would thus be analysed as follows:

\ea\label{ex:real-irr-no-clash}
\begingl 
\glpreamble pinikupuna\\
\gla pi-niku-puna\\ 
\glb 2\textsc{sg}-eat-\textsc{am.prior}.\textsc{irr}\\ 
\glft ‘you will go to eat’\\ 
\endgl
%\trailingcitation{}
\xe

However, in cases where \textit{a} can not be analysed as replacing a default vowel and in addition \textit{u} acts as a proper marker of realis, these vowels are glossed as separate suffixes as in (\ref{ex:jane-real}) and (\ref{ex:subREAL}) above.\is{active verb|)}

After this discussion of the form of the RS markers, the following section presents different semantic parameters that trigger either realis or irrealis marking.

\subsection{Semantic parameters of reality status}\label{RS:parameters}

As discussed in \sectref{sec:RealityStatus} above, there are several semantic parameters that trigger different RS marking. In this section, I will provide examples for each of them. \sectref{sec:RS_TemporalReference} is about the encoding of temporal reference by RS, and the influence of polarity in the choice of a realis or irrealis predicate is explained in \sectref{par:Polarity}. Hypotheticality is another factor involved, which is described in \sectref{sec:Hypotheticality}, while \sectref{par:IRR-factuality_epistemic_modality} to \sectref{sec:Agent-orientedModality} describe interaction of RS with three different kinds of modality. Finally, \sectref{sec:Prospectiveness} is dedicated to the choice of irrealis in relative future.

\subsubsection{Temporal reference}\label{sec:RS_TemporalReference}
\is{realis|(}
\is{past reference|(}

Realis is used with non-future events, as in (\ref{ex:REAL-PRES-2}) or (\ref{ex:REAL-PAST}). The former has present reference and the latter past reference, which is in this case also signalled by the adverb \textit{ukuine} ‘yesterday’.

In (\ref{ex:REAL-PRES-2}), Isidro describes a picture of a puzzle game, and so the action he sees is ongoing.

\ea\label{ex:REAL-PRES-2}
\begingl 
\glpreamble tikubijaku eka aitubuche\\
\gla ti-kubijaku eka aitubuche\\ 
\glb 3i-play \textsc{dem}a boy\\ 
\glft ‘the boy is playing’\\ 
\endgl
 \trailingcitation{[mdx-c120416ls.183]}
\xe

(\ref{ex:REAL-PAST}) comes from Juana, actually back-channelling my statement that I had met Fe\-de\-ri\-co the day before.

\ea\label{ex:REAL-PAST}
\begingl 
\glpreamble jaa, pitupu ukuine\\
\gla jaa pi-tupu ukuine\\ 
\glb \textsc{afm} 2\textsc{sg}-meet yesterday\\ 
\glft ‘ah, you met him yesterday’\\ 
\endgl
 \trailingcitation{[jxx-e120516l-1.058]}
\xe
\is{past reference|)}
\is{realis|)}


Irrealis\is{irrealis|(} is used for future events,\is{future reference|(} as in (\ref{ex:IRR-FUT}), where the exact temporal reference is again overtly expressed by \textit{tajaitu} ‘tomorrow’. The example comes from Miguel who was addressing María C.


\ea\label{ex:IRR-FUT}
\begingl
\glpreamble tajaitu nibÿsÿupunuka naka\\
\gla tajaitu ni-bÿsÿu-punuka naka\\
\glb tomorrow 1\textsc{sg}-come-\textsc{reg.irr} here\\
\glft ‘I will come back here tomorrow’
\endgl
 \trailingcitation{[mux-c110810l.140]}
\xe
\is{irrealis|)}

%\ea\label{ex:IRR-FUT-3}
%\begingl 
%\glpreamble tajaituyu niyuna\\
%\gla tajaitu-yu ni-yuna\\ 
%\glb tomorrow-\textsc{ints} 1\textsc{sg}-go.\textsc{irr}\\ 
%\glft ‘I will go tomorrow’\\ 
%\endgl
% \trailingcitation{[jxx-p120430l-2.459]}
%\xe

%\ea\label{ex:IRR-FUT-2}
%\begingl 
%\glpreamble tajaituji tikebupunuka\\
%\gla ti-a-jai-tu-ji ti-kebu-punuka\\ 
%\glb 3i-\textsc{irr}-be.light-\textsc{iam}-\textsc{rprt} 3i-rain-\textsc{reg.irr}\\ 
%\glft ‘they say that it will be raining again tomorrow’\\ 
%\endgl
% \trailingcitation{[jxx-e120516l-1.104]}
%\xe

However, most often, there are no lexical clues that signal future or past reference. RS together with linguistic and extralinguistic context suffices to establish the correct temporal setting. Consider (\ref{ex:IRR-FUT-3}), which has future reference signalled by irrealis RS, while (\ref{ex:REAL-PAST-CREA}) has past reference expressed with realis RS. Both examples come from Juana.

In (\ref{ex:IRR-FUT-3}) she tells me what she said to her daughter upon leaving the house.

\ea\label{ex:IRR-FUT-3}
\begingl 
\glpreamble niyuna nauku parkeyae\\
\gla ni-yuna nauku parke-yae\\ 
\glb 1\textsc{sg}-go.\textsc{irr} there park-\textsc{loc}\\ 
\glft ‘I will go to the park there’\\ 
\endgl
 \trailingcitation{[jxx-p120430l-2.242]}
\xe
\is{future reference|)}

(\ref{ex:REAL-PAST-CREA}) is part of her narration of the (biblical, yet syncretic) creation story.

\ea\label{ex:REAL-PAST-CREA}
\begingl 
\glpreamble kechue chibeu echÿu mansana\\
\gla kechue chi-beu echÿu mansana\\ 
\glb snake 3-take.away \textsc{dem}b apple\\ 
\glft ‘the snake took the apple off (the tree)’\\ 
\endgl
 \trailingcitation{[jxx-n101013s-1.410]}
\xe

In addition, irrealis\is{irrealis|(} is used to express habitual\is{habitual|(} actions in descriptions of how to process certain things. (\ref{ex:IRR-HAB}) is from a description by Miguel of the preparation of rice bread, and (\ref{ex:IRR-HAB-2}) from a description by Juana of the production and use of a clay pot. In this function habitual overlaps with obligation (see \sectref{sec:Agent-orientedModality}), since it is not clear at all whether the description is simply one of usual actions or includes an obligation insofar as one has to do it the way as described, because it is the right way of doing it.

\ea\label{ex:IRR-HAB}
\begingl 
\glpreamble primeru biyÿbapaka eka arusu\\
\gla primeru bi-yÿbapaka eka arusu\\ 
\glb first 1\textsc{pl}-grind.\textsc{irr} \textsc{dem}a rice\\ 
\glft ‘first, we (have to) grind the rice’\\ 
\endgl
 \trailingcitation{[mxx-d120411ls-1a.018]}
\xe

\ea\label{ex:IRR-HAB-2}
\begingl 
\glpreamble tibururuka, petuka chÿeche\\
\gla ti-bururuka pi-etuka chÿeche\\ 
\glb 3i-boil.\textsc{irr} 2\textsc{sg}-put.\textsc{irr} meat\\ 
\glft ‘when it boils, you put the meat in’\\ 
\endgl
 \trailingcitation{[jmx-d110918ls-1.010]}
\xe
\is{irrealis|)}

Realis,\is{realis|(} on the other hand, is used to describe general customs and habits, like the custom of eating something in (\ref{ex:REAL-CUST}). It comes from María C., who was listing names of crops for me to learn some vocabulary.

\ea\label{ex:REAL-CUST}
\begingl 
\glpreamble aa chermuya biniku, chÿi\\
\gla aa chermuya bi-niku chÿi\\ 
\glb \textsc{intj} cherimoya 1\textsc{pl}-eat fruit\\ 
\glft ‘ah, we eat cherimoya, it is a fruit’\\ 
\endgl
 \trailingcitation{[uxx-p110825l.190]}
\xe
\is{realis|)}

The difference between (\ref{ex:IRR-HAB}) and (\ref{ex:IRR-HAB-2}) as opposed to (\ref{ex:REAL-CUST}) is that the former two examples describe actions that are done habitually, when certain things are prepared. (\ref{ex:REAL-CUST}) on the other hand is not about the habitual eating of cherimoya fruits in their season, but it is used in this case to describe one property of the fruit, its edibility.

There may be certain differences here between speakers, or use of either realis or irrealis may depend on how concrete speakers imagine certain actions at the moment of speaking. While the description of how something is made or how it works is relatively abstracted from a real situation, personal habits are more concrete as in the following example from María S.

\ea
\begingl 
\glpreamble nijibÿku niyunu asaneti\\
\gla ni-jibÿku ni-yunu asaneti\\ 
\glb 1\textsc{sg}-smoke 1\textsc{sg}-go field\\ 
\glft ‘I smoke, when I go to the field’\\ 
\endgl
 \trailingcitation{[rxx-e120511l.390]}
\xe

Irrealis can also be used to describe habitual events in the past.\is{past reference} However, habitual past events are also often described with realis predicates and one speaker can even switch back and forth between realis and irrealis in accounts about the past. This may again be dependent on how much the speaker generalises the event or is remembering a concrete situation, even if the event happened habitually. This is hard to prove though, because we do not have direct access to their cognition. One example in which irrealis is used for an event in the past that occurred repeatedly is (\ref{ex:IRR-HAB-PAST}), which comes from an account about what Juana used to do with her late grandmother, when she was much younger.

\ea\label{ex:IRR-HAB-PAST}
\begingl 
\glpreamble puna tijai biyuna bepueikupa\\
\gla puna tijai bi-yuna bi-epueiku-pa\\ 
\glb other day 1\textsc{pl}-go.\textsc{irr} 1\textsc{pl}-fish-\textsc{dloc.irr}\\ 
\glft ‘another day we would go to fish’\\ 
\endgl
 \trailingcitation{[jxx-p120430l-1.060]}
\xe
\is{habitual|)}

\subsubsection{Polarity}\label{par:Polarity}

In Paunaka, all positive declarative\is{declarative clause} non-future clauses take \isi{realis} predicates. An example is (\ref{ex:REAL-POS}), where Juana describes a picture of the \isi{frog story}; for another example see (\ref{ex:REAL-PAST}) above.

\ea\label{ex:REAL-POS}
\begingl 
\glpreamble tiniku, teumÿnÿ ÿne\\
\gla ti-niku ti-eu-mÿnÿ ÿne\\ 
\glb 3i-eat 3i-drink-\textsc{dim} water\\ 
\glft ‘it (the dog) is eating, it is drinking water’\\ 
\endgl
 \trailingcitation{[jxx-a120516l-a.018]}
\xe

Negation,\is{negation|(} on the other hand, triggers \isi{irrealis} marking as in (\ref{ex:IRR-NEG}), which has a stative verb and therefore takes an irrealis prefix. Juana speaks about her son-in-law here, who is ill.

\ea\label{ex:IRR-NEG}
\begingl 
\glpreamble kuina tajimama\\
\gla kuina ti-a-jimama\\ 
\glb \textsc{neg} 3i-\textsc{irr}-be.strong\\ 
\glft ‘he is not strong’\\ 
\endgl
 \trailingcitation{[jxx-p110923l-1.053]}
\xe

(\ref{ex:IRR-NEG-2}) has a negated active verb. It is a statement by Clara who had asked María C. for a word in Paunaka that she did not remember. María C. did not remember either.

\ea\label{ex:IRR-NEG-2}
\begingl
\glpreamble kuina pichupabu\\
\gla kuina pi-chupa-bu\\
\glb \textsc{neg} 2\textsc{sg}-know.\textsc{irr}-\textsc{dsc}\\
\glft ‘you don’t know it anymore’
\endgl
 \trailingcitation{[cux-c120414ls-2.243]}
\xe

There is no general “doubly irrealis construction”\is{doubly irrealis construction|(}\footnote{A doubly irrealis construction is defined by \citet[271]{Michael2014} as a construction in which negation and another semantic parameter both trigger irrealis marking. Doubly irrealis constructions can be found in Nanti and other Kampan \isi{Arawakan languages}, but also in the Southern Arawakan languages \isi{Terena} and Trinitario, as well as in Ignaciano,\is{Mojeño languages} which has lost simple irrealis marking (\citealp[cf.][268]{EkdahlGrimes1964}; \citealt[132]{OlzaZubiri2004}; \citealt[279]{Michael2014}; \citealt[235]{Rose2014}; \citealt[108, 115]{DanielsenTerhartSubm}). There are several ways to express double irrealis: either with a special negative particle (Kampan Arawakan, \isi{Terena}) or with an affix on the verb (\isi{Mojeño languages}), but all doubly irrealis verbs have in common that they do not take the irrealis affix found in those constructions in which only one parameter triggers irrealis RS.} in declarative clauses,\is{declarative clause} though a doubly irrealis construction is possibly found in prohibitive clauses,\is{directive speech act!prohibitive}\is{negation!prohibitive} see \sectref{Par:Speaker_oriented_modality} below. Thus, a verb with \isi{future reference} \textit{and} negative polarity marks irrealis in a way identical to a verb that has future reference or negative polarity only. Paunaka is thus ambiguous to the extent that there is no way to mark on the predicate that there is negation and another semantic parameter triggering irrealis. (\ref{ex:DOB-IRR}) is an example of a predicate that takes irrealis for two reasons, negative polarity and future reference. It was elicited from María S.

\ea\label{ex:DOB-IRR}
\begingl 
\glpreamble tajaitu kuina nÿnika\\
\gla tajaitu kuina nÿ-nika\\ 
\glb tomorrow \textsc{neg} 1\textsc{sg}-eat.\textsc{irr}\\ 
\glft ‘I won’t eat tomorrow’\\ 
\endgl
 \trailingcitation{[rxx-e-151017l]}
\xe

There is also one kind of lexically expressed doubly irrealis construction, which regards negative abilitive sentences. While in positive statements, irrealis alone is sufficient to encode this \isi{modality}, in negative statements, speakers often add the non-verbal ability predicate \textit{puero} ‘can’, borrowed\is{borrowing}\is{knowledge/ability predicate} from the Spanish modal verb \textit{poder}, possibly via Besɨro (see \sectref{sec:borrowed_verbs}). This, however, is not mandatory. An example without \textit{puero} is given below. It was produced by Miguel making jokes with Swintha, who had asked him to give away his house:

\ea\label{ex:DOB-IRR-2}
\begingl
\glpreamble kuina nÿpunaka eka nubiu, kuina tamichana\\
\gla kuina nÿ-punaka eka nÿ-ubiu kuina ti-a-michana\\
\glb \textsc{neg} 1\textsc{sg}-give.\textsc{irr} \textsc{dem}a 1\textsc{sg}-house \textsc{neg} 3i-\textsc{irr}-be.nice\\
\glft ‘I can’t give away my house, it is not in a good condition’
\endgl
 \trailingcitation{[mxx-e110820ls.108]}
\xe

In (\ref{ex:DOB-IRR-3}), \textit{puero} is used in addition to the irrealis verb. This example comes from Juana talking about her ill son-in-law.

\ea\label{ex:DOB-IRR-3}
\begingl
\glpreamble kuina puero tiyuika\\
\gla kuina puero ti-yuika\\
\glb \textsc{neg} can 3i-walk.\textsc{irr}\\
\glft ‘he cannot walk’
\endgl
 \trailingcitation{[jxx-p110923l-1.048]}
\xe
\is{doubly irrealis construction|)}
\is{negation|)}

\subsubsection{Hypotheticality}\label{sec:Hypotheticality}

Most hypothetical predicates are found in conditional clauses\is{temporal overlap/condition|(} in my corpus (see also \sectref{sec:AC-kue}).
The antecedent clause often contains a \isi{connective} \textit{kue} ‘if, when’, but it can also be unmarked. The predicate of the antecedent clause is irrealis\is{irrealis|(} in most cases. The predicate of the consequent clause is also irrealis in deontic, hypothetical, future\is{future reference} or counterfactual constructions.\is{counterfactuality} Some of these cases are exemplified in (\ref{ex:COND-IRR-1}) and (\ref{ex:COND-IRR-2}). 
(\ref{ex:COND-IRR-1}) comes from Miguel and refers to the fact that María C.’s husband, being severely ill, cannot stay at home alone if she leaves for a day or so.

\ea\label{ex:COND-IRR-1}
\begingl 
\glpreamble kue piyuna tiyunauku echÿu\\
\gla kue pi-yuna ti-yuna-uku echÿu\\ 
\glb if 2\textsc{sg}-go.\textsc{irr} 3i-go.\textsc{irr}-\textsc{add} \textsc{dem}b\\ 
\glft ‘if you go, he has to go, too’\\ 
\endgl
 \trailingcitation{[mux-c110810l.042]}
\xe

In (\ref{ex:COND-IRR-2}), Juana cites what her daughter in Spain promised her.

\ea\label{ex:COND-IRR-2}
\begingl
\glpreamble kue pibÿsÿa, mimi nipabentecha nubiu, te biyunupunatu nauku\\
\gla kue pi-bÿsÿa mimi ni-pabentecha nÿ-ubiu te bi-yunupuna-tu nauku\\
\glb if 2\textsc{sg}-come.\textsc{irr} mum 1\textsc{sg}-sell.\textsc{irr} 1\textsc{sg}-house \textsc{seq} 1\textsc{pl}-go.back.\textsc{irr}-\textsc{iam} there\\
\glft ‘if you come, mum, I will sell my house, and then we go back there’
\endgl
 \trailingcitation{[jxx-p110923l-1.432]}
\xe
\is{irrealis|)}

However, if the consequence clause is realis,\is{realis|(} we are rather dealing with a temporal clause as in (\ref{ex:COND-REAL}), a statement by Juana about the arroyo close to Santa Rita.% jxx-a120516l-a.572ff. mxx-n120423lsf-X.01ff. rxx-e120511l.134 (but not temporal clause!), jxx-e150925l-1.212

\ea\label{ex:COND-REAL}
\begingl
\glpreamble kue tijinupatu tÿpi enero, juu tijapÿkutu\\
\gla kue ti-jinupa-tu tÿpi enero juu ti-japÿku-tu\\
\glb if 3i-flow.\textsc{irr}-\textsc{iam} \textsc{obl} January \textsc{intj} 3i-fill-\textsc{iam}\\
\glft ‘if water flows in January, huu, it fills’
\endgl
 \trailingcitation{[jxx-a120516l-a.572-573]}
\xe
\is{realis|)}
\is{temporal overlap/condition|)}

Counterfactual conditional predicates\is{counterfactuality|(} with past reference additionally take a \isi{frustrative} mark\-er (see also \sectref{sec:Frust_avertive_optatiev}). There is one personal account in which Juan tells how her daughter once travelled to Spain to help her sister care for her children. The whole journey turned out to be in vain because she was rejected at the airport, since she did not have a valid visa. Several predicates in this account are marked with irrealis and \isi{frustrative} because they refer to hypothetical and counterfactual ideas about what one daughter would have done in Spain, and what the other daughter should have done to prevent her sister from being deported. (\ref{ex:COUNT-FRUST-IRR}) is a remark about other people who were expelled, too.

\ea\label{ex:COUNT-FRUST-IRR}
\begingl 
\glpreamble i pujanenube tiyunanubeini trabakunubeina nauku \\
\gla i pu-jane-nube ti-yuna-nube-ini trabaku-nube-ina nauku \\ 
\glb and other-\textsc{distr}-\textsc{pl} 3i-go.\textsc{irr}-\textsc{pl}-\textsc{frust} work-\textsc{pl}-\textsc{irr.nv} there\\ 
\glft ‘and others would have gone to work there’\\ 
\endgl
 \trailingcitation{[jxx-p120430l-1.206]}
\xe
\is{counterfactuality|)}

\subsubsection{Factuality/epistemic modality}\label{par:IRR-factuality_epistemic_modality}\is{modality|(}

Epistemic modality can be expressed by using an irrealis\is{irrealis|(} predicate as in (\ref{ex:smoke-OBJ}), where María S. explains why smoking is bad.

\ea\label{ex:smoke-OBJ}
\begingl
\glpreamble tikupakabi\\
\gla ti-kupaka-bi\\
\glb 3i-kill.\textsc{irr}-1\textsc{pl}\\
\glft ‘it (smoking) can kill us’
\endgl
 \trailingcitation{[rxx-e120511l.385]}
\xe
\is{irrealis|)}

In addition, there are two epistemic markers  \textit{-kena} ‘\textsc{uncert}’ and \textit{-yenu} ‘\textsc{ded}’ (see \sectref{sec:Modality_Evidentiality}), which occur on both realis and irrealis verbs. RS reflects one of the other semantic parameters in those cases, e.g. hypotheticality, which is the case in (\ref{ex:SPEC-IRR}). For an epistemic marker on a realis verb see (\ref{ex:SPEC-REAL}), where realis is due to past reference.

(\ref{ex:SPEC-IRR}) comes from Miguel who had just discussed with Juana that they cannot write Paunaka.

\ea\label{ex:SPEC-IRR}
\begingl 
\glpreamble pero bichupakena timesumeikabitu \\
\gla pero bi-chupa-kena ti-mesumeika-bi-tu\\ 
\glb but 1\textsc{pl}-know.\textsc{irr}-\textsc{uncert} 3i-teach.\textsc{irr}-1\textsc{pl}-\textsc{iam}\\ 
\glft ‘but we may know it in the future, they will teach us now’\\ 
\endgl
 \trailingcitation{[jmx-e090727s.031]}
\xe

(\ref{ex:SPEC-REAL}) was produced by Juana and referred to Miguel, who was in Santa Cruz to demand financial support for his blind daughter.

\ea\label{ex:SPEC-REAL}
\begingl 
\glpreamble chisiupuchunubekena\\
\gla chi-siupuchu-nube-kena\\ 
\glb 3-pay-\textsc{pl}-\textsc{uncert}\\ 
\glft ‘they may have paid him (by now)’\\ 
\endgl
 \trailingcitation{[jxx-p120430l-1.084]}
\xe


\subsubsection{Speaker-oriented modality}\label{Par:Speaker_oriented_modality}
\is{irrealis|(}
The term speaker-oriented modality refers to “those cases in which the speaker gives someone an order or gives someone permission” \citep[31]{deHaan2005}. It comprises various directives as well as expressions of permission \citep[179]{Bybee_et_al1994}. The only directives\is{directive speech act|(} explicitly mentioned by \citet[252]{Michael2014} as taking irrealis marking are imperatives\is{imperative} and polite directives or exhortatives,\is{hortative} and speaker-oriented modality is never expressed by realis verbs according to his study.
I start with the constructions mentioned by \citet[252]{Michael2014} and then move on to other constructions that have been subsumed under the term speaker-oriented modality: prohibitives, hortatives, admonitives, and permissives \citep[cf.][179]{Bybee_et_al1994}.
%Michael: imperative, polite directive/ exhortative ; Bybee et al: Imperative, Prohibitive, optative, hortative, admonitive, permissive ; de Haan: directives (a term from Lyons 1977), imperatives (the command mood), prohibitions, optatives, admonitions (warnings), and permissions.

Imperatives\is{imperative|(} can simply be formed as second person irrealis predicates, as in (\ref{ex:IMP-IRR}). An imperative marker can be added, but the predicate remains irrealis (see \sectref{sec:MarkedImperatives}). Whether the imperative is to be understood as a command or polite directive mainly depends on \isi{intonation}.

(\ref{ex:IMP-IRR}) was produced by Juana who had given me some chicha.

\ea\label{ex:IMP-IRR}
\begingl 
\glpreamble ¡pea!\\
\gla p-ea\\ 
\glb 2\textsc{sg}-drink.\textsc{irr}\\ 
\glft ‘drink!’\\ 
\endgl
 \trailingcitation{[jxx-e120516l-1.044]}
\xe
\is{imperative|)}

There is a special particle \textit{jaje} ‘\textsc{hort}’ for hortatives.\is{hortative} This particle need not be accompanied by a predicate if it is clear from the context what the request is, but if there is an extra predicate, it has irrealis RS, as in (\ref{ex:HORT-IRR}). The example comes from Miguel’s story about the cowherd and the spirit of the hill. The latter had taken away the cows, but invites the cowherd to come with him to look for the cows here.

\ea\label{ex:HORT-IRR}
\begingl 
\glpreamble “¡jaje biyuna bimupajane echÿu bakajane!"\\
\gla jaje bi-yuna bi-imu-pa-jane echÿu baka-jane\\ 
\glb \textsc{hort} 1\textsc{pl}-go.\textsc{irr} 1\textsc{pl}-see-\textsc{dloc.irr}-\textsc{distr} \textsc{dem}b cow-\textsc{distr}\\ 
\glft ‘“let’s go and see the cows!”’\\ 
\endgl
 \trailingcitation{[mxx-n151017l-1.38]}
\xe

Note that in both imperative and \isi{hortative} constructions the event expressed by the predicate remains to be fulfilled, so that they may be understood as a subclass of temporally unrealised event.

Optative\is{optative} is expressed by \textit{-yuini} or \textit{-jÿti}. I have only found these markers in combination with irrealis predicates, but they are not very frequent in general (see \sectref{sec:FRUST-Optative}).
\is{irrealis|)}

There is no special permissive construction as far as I know that would be different from an imperative.

There is no general agreement on how to form prohibitives\is{directive speech act!prohibitive|(}\is{negation!prohibitive|(} among the speakers. While some of them simply use standard negation constructions, others employ constructions with a different negative\is{negative particle|(} particle.\footnote{Many \isi{Arawakan languages} have a prohibitive construction in which either the expression of negation differs from standard negation, e.g. by using a special prohibitive marker, or the remainder of the construction differs from the imperative construction, or both \citep[270--271]{Michael2014b}.} 
Two speakers, Juana and María C., use a special negative particle \textit{naka} ‘\textsc{proh}’\footnote{This marker resembles the \isi{Baure} general negative particle \textit{noka} \citep[cf.][338]{Danielsen2007}. } accompanied by a \isi{realis} predicate, but one of them also used \isi{irrealis} predicates in an elicitation session, when I asked for more examples with the prohibitive particle. Given the fact that the prohibitive construction with the realis predicate was not elicited and that there are doubly irrealis constructions\is{doubly irrealis construction} including negation and another semantic parameter in Nanti and \isi{Terena} that include a realis-marked predicate (see \sectref{par:Polarity} above), I believe that the construction with the realis predicate may be the conservative one and the use of irrealis an extension of its use in standard negation.

(\ref{ex:PROH-REAL}) is a prohibitive sentence elicited from María C.

\ea\label{ex:PROH-REAL}
\begingl 
\glpreamble ¡naka piyuyuikubu!, ticheneikabi piuse\\
\gla naka pi-iyuyuiku-bu ti-cheneika-bi pi-use\\ 
\glb \textsc{proh} 2\textsc{sg}-cry-\textsc{mid} 3i-care.for.\textsc{irr}-2\textsc{sg} 2\textsc{sg}-grandmother\\ 
\glft ‘don’t cry! your grandmother will take care of you.’\\ 
\endgl
 \trailingcitation{[uxx-c151002lf]}
\xe

The prohibitive negator \textit{naka} was verified by a third speaker, María S., who herself uses the negator \textit{masaini} in prohibitives, which is probably composed of the \isi{apprehensional} \isi{connective} \textit{masa} ‘lest’ and the \isi{frustrative} \textit{-ini}. \textit{Masaini} is also used in warnings, i.e. it has an admonitive \is{admonitive|(} flavour.\is{negative particle|)} In both cases, the predicate of \mbox{\textit{masaini}-}constructions is sometimes realis and sometimes irrealis in my corpus, similar to prohibitive constructions with \textit{naka}. As far as I can tell, this is free variation that may have to do with an extension of the use of irrealis in standard negation. I could not find any other semantic parameter that could have an influence on the choice of either realis or irrealis in these constructions, but note that there are only a few of them. Two examples of \textit{masaini} being combined with realis predicates are given below.

(\ref{ex:proh-masa-22-1}) was elicited from María S. and shows the prohibitive use of \textit{masaini} peculiar to this speaker.

\ea\label{ex:proh-masa-22-1}
\begingl
\glpreamble ¡masaini pekubu!\\
\gla masaini pi-ekubu\\
\glb \textsc{adm} 2\textsc{sg}-laugh\\
\glft ‘don’t laugh!’
\endgl
 \trailingcitation{[rxx-e150220s-1.06]}
\xe

(\ref{ex:ADM-REAL}) is rather a warning. It was elicited from Miguel. Just a moment earlier, he had used an irrealis predicate in almost the same sentence.

\ea\label{ex:ADM-REAL}
\begingl 
\glpreamble ¡masaini tinijabakubi kabe!\\
\gla masaini ti-nijabaku-bi kabe\\ 
\glb \textsc{adm} 3i-bite-2\textsc{sg} dog\\ 
\glft ‘be careful, the dog may bite you!’\\ 
\endgl
 \trailingcitation{[mrx-e150219s.149]}
\xe\is{admonitive|)} 

See \sectref{sec:Prohibitives} for more examples of negative imperatives and related constructions.\is{negation!prohibitive|)}\is{directive speech act!prohibitive|)}\is{directive speech act|)}

\subsubsection{Agent-oriented modality}\label{sec:Agent-orientedModality}
Agent-oriented modality refers to the expression of “those cases in which the \isi{agent} of a clause is influenced in some way in performing the action described in the clause” \citep[30]{deHaan2005} by “internal and external conditions” \citep[177]{Bybee_et_al1994}. According to \citet[252]{Michael2014}, obligation and need (called necessity by \citealt[177]{Bybee_et_al1994}) are expressed by irrealis predicates. In Paunaka, obligation and need may be expressed simply by an irrealis predicate,\is{irrealis|(} see (\ref{ex:OBL-IRR-1}) to (\ref{ex:OBL-IRR-3}), all of them coming from Juana.

In (\ref{ex:OBL-IRR-1}) she expresses her faith in God and duty as a Catholic.

\ea\label{ex:OBL-IRR-1}
\begingl 
\glpreamble baejumi micha bia bakukene\\
\gla bi-a-ejumi micha bia bi-a-kukene\\ 
\glb 1\textsc{pl}-\textsc{irr}-remember good God 1\textsc{pl}-\textsc{irr}-pray\\ 
\glft ‘we have to believe in God and pray’\\ 
\endgl
 \trailingcitation{[jxx-e150930lay-1]}
\xe


The following example was elicited from Juana, because I wanted to express my pity for some groups of children from different schools who had a sports competition in the blazing sun.

\ea\label{ex:OBL-IRR-4}
\begingl
\glpreamble takubijainube\\
\gla ti-a-kubijai-nube\\
\glb 3i-\textsc{irr}-play-\textsc{pl}\\
\glft ‘they have to play’
\endgl
 \trailingcitation{[jrx-e151019l-2]}
\xe

(\ref{ex:OBL-IRR-3}) comes from the creation story. The silk floss tree has swallowed the complete supply of corn, thus Jesus decides that it has to be felled to get the corn back.

\ea\label{ex:OBL-IRR-3}
\begingl
\glpreamble “bikutataka eka mupuÿ”\\
\gla bi-kutataka eka mupuÿ\\
\glb 1\textsc{pl}-fell.\textsc{irr} \textsc{dem}a silk.floss.tree\\
\glft ‘“we have to fell this silk floss tree”’
\endgl
 \trailingcitation{[jxx-n101013s-1.793-794]}
\xe

Additionally a Spanish loan phrase\is{borrowing} \textit{tiene ke} from \textit{tiene que} ‘must’, also used in \isi{Bésiro}, is occasionally used by some of the speakers. This makes the notion of obligation more explicit. Although it is a third person singular expression in Spanish (‘he/she/it has to’), it is not restricted to third person in Paunaka. \textit{Tiene ke} is always followed by an irrealis predicate as in the statement in (\ref{ex:OBL-IRR-2}), which comes from Clara who is speaking about washing a wound here.

\ea\label{ex:OBL-IRR-2}
\begingl 
\glpreamble tiene ke chikipucha xhabuji\\
\gla {tiene ke} chi-kipucha xhabu-ji\\ 
\glb must 3-wash.\textsc{irr} soap-\textsc{rprt}\\ 
\glft ‘she should wash it with soap, it is said’\\ 
\endgl
 \trailingcitation{[cux-120410ls.244]}
\xe

%max kapunuina mas bien niyuna, mox-n110920l.057

The other semantic notions that are subsumed under agent-oriented modality are ability, desire/willingness, and root possibility; the latter “is related to ability, but also takes external factors into account” \citep[31]{deHaan2005}. Desire is usually expressed by want-verb constructions in Paunaka, which are dealt with in \sectref{sec:Prospectiveness}. Ability and root possibility can indeed be expressed by irrealis marking, as is the case in (\ref{ex:ABIL-IRR-1}) taken from the narrative of the fox and the jaguar told by Miguel and Juana. The fox had just boasted about the many different jumps he knows, while the jaguarundi only knows one jump, but a very effective one, with which he can escape into trees, for instance, as Miguel states.

\ea\label{ex:ABIL-IRR-1}
\begingl
\glpreamble tikutijikatu chÿnajiku chijipuikiu, ¡bruj!\\
\gla ti-kutijika-tu chÿna-jiku chi-jipuik-i-u bruj\\
\glb 3i-flee.\textsc{irr}-\textsc{iam} one-\textsc{lim} 3-jump-\textsc{subord}-\textsc{real} \textsc{intj}\\
\glft ‘he can escape with only one jump, bruh!’
\endgl
 \trailingcitation{[jmx-n120429ls-x5.365]}
\xe
\is{irrealis|)}

In addition, there are also constructions containing the non-verbal predicate \textit{puero} ‘can’, a loan from Spanish, but mostly \textit{puero} is used when an ability or a possibility is negated\is{negation}\is{knowledge/ability predicate} (see \sectref{par:Polarity} above). Ability in the sense of capability can also be expressed by the stative verb \textit{-ichuna}\is{knowledge/ability predicate} ‘be capable, know’, which is followed by realis complements if no other factor (e.g. negation, hypotheticality) triggers irrealis marking. This is logical, since the notion of capability is expressed by an extra predicate (see discussion in Footnote \ref{ftn:irrealis}), so that the realis complement can express the “perceived certainty of the factual reality” \citep[67]{Elliott2000}. One example is (\ref{ex:ABIL-REAL}) from Juana, who is speaking about her children here.

\ea\label{ex:ABIL-REAL}
\begingl
\glpreamble tichunanube tubuejinube\\
\gla ti-ichuna-nube ti-ubueji-nube\\
\glb 3i-be.capable-\textsc{pl} 3i-swim-\textsc{pl}\\
\glft ‘they can swim’
\endgl
 \trailingcitation{[jxx-a120516l-a.570]}
\xe
\is{modality|)}

\subsubsection{Relative future}\label{sec:Prospectiveness}\is{future reference|(}

Relative future is used to refer to those cases in which two events are set into a relation, in which the first has past time reference\is{past reference} in relation to the moment of speaking, but the event encoded by the second predicate is not realised by the reference time defined by the first predicate. \citet[]{Michael2014} used the term “prospective construction”, but I prefer “relative future” because “prospective" was used to denote a specific type of aspect by \citet[64]{Comrie1976}. Relative future is inherent in purposive constructions as well as some types of complementation.

Irrealis\is{irrealis|(} is typically found on the \isi{purpose} verb in purposive constructions. There are various ways to form them (see \sectref{sec:PurposeClauses}, \sectref{sec:AprenhensionalClauses}, \sectref{sec:EmbeddedAC_bare}, and \sectref{sec:EmbeddedAC_adp}).

One example is given below. It comes from Juana who told me about the work the people of Santa Rita did in exchange for their reservoir. The women brought the men chicha.

\ea\label{ex:irr-purpi-1}
\begingl
\glpreamble tumunube aumue tÿpi teanube nauku\\
\gla ti-umu-nube aumue tÿpi ti-ea-nube nauku\\
\glb 3i-take-\textsc{pl} chicha \textsc{obl} 3i-drink.\textsc{irr}-\textsc{pl} there\\
\glft ‘they brought them chicha to drink’
\endgl
 \trailingcitation{[jxx-p120515l-2.183-184]}
\xe
\is{irrealis}

%
%\ea\label{ex:PURP-1}
%\begingl 
%\glpreamble biti bisu chebukiapu amuke\\
%\gla biti bi-isu ch-ebuk-i-a-pu amuke\\ 
%\glb we 1\textsc{pl}-weed 1\textsc{pl}-sow-\textsc{subord}-\textsc{irr}-\textsc{mid} corn\\ 
%\glft ‘we weed so that he can sow corn’\\ 
%\endgl
% \trailingcitation{[nxx-p630101g-1.089-090]}
%\xe
%
There are two construction types to express purpose of motion.\is{purpose|(} Both of them have in common that the purpose verb can be \isi{irrealis}, but also \isi{realis} if the motion verb also has realis RS. This is discussed in detail in \sectref{sec:SVC_and_MCPC}, but two examples of a motion-cum-purpose construction are given below. In (\ref{ex:PURP-pa}), the purpose verb is irrealis, in (\ref{ex:PURP-pu}) it is realis. It is not clear what triggers the choice of realis or irrealis complements\is{complement verb} in those cases. It might be the case that a different perspective is taken: while irrealis complements focus on the fact that the event is not completed by the reference time of the main verb, realis complements emphasise the fact that the whole event composed of the main and the \isi{complement verb} is over by the moment of speaking.

The context of (\ref{ex:PURP-pa}) is a description of the journey of Juana’s grandparents; the whole setting is in the past, but within this past there is a relative future, i.e. the acquisition of the cows.

\ea\label{ex:PURP-pa}
\begingl 
\glpreamble tiyununube Monkoxi tiyÿseikupanube chipeunube baka\\
\gla ti-yunu-nube Monkoxi ti-yÿseiku-pa-nube chi-peu-nube baka\\ 
\glb 3i-go-\textsc{pl} Moxos 3i-buy-\textsc{dloc.irr}-\textsc{pl} 3-animal-\textsc{pl} cow\\ 
\glft ‘they went to Moxos in order to buy cows’\\ 
\endgl
 \trailingcitation{[jxx-p151016l-2]}
\xe
\is{future reference|)}

On the other hand, in the recording from which (\ref{ex:PURP-pu}) is taken, a recording by Riester with Juan Ch., the general topic is a supply of food in the present to which the event of hunting in the past made a contribution. The RS of the \isi{complement verb} would probably have been different for a detailed description of the different sub-events that resulted from the hunting expedition.

\ea\label{ex:PURP-pu}
\begingl 
\glpreamble uchuine tijaikenekÿutu niyunu ninÿupu\\
\gla uchuine tijaikenekÿu-tu ni-yunu ni-nÿu-pu\\ 
\glb just.now dawn-\textsc{iam} 1\textsc{sg}-go 1\textsc{sg}-lie.in wait-\textsc{dloc}\\ 
\glft ‘today in the early morning I went to lie in wait (for animals)’\\ 
\endgl
 \trailingcitation{[nxx-a630101g-1.66]}
\xe
\is{purpose|)}

In complementation, some complement verbs\is{complement verb} obligatorily take irrealis RS, regardless of the RS of the main predicate. This is described in more detail in \sectref{sec:Unmarked_CCs}. One example is (\ref{ex:WANT-IRR}) with a desiderative verb with realis RS and the complement in irrealis. This is logical given that the event of the complement is hypothetical at the time of reference of the main verb. The sentence was elicited from José.

\ea\label{ex:WANT-IRR}
\begingl 
\glpreamble tisachu tinijabakabi echÿu kabe\\
\gla ti-sachu ti-nijabaka-bi echÿu kabe\\ 
\glb 3i-want 3i-bite.\textsc{irr}-2\textsc{sg} \textsc{dem}b dog\\ 
\glft ‘the dog wants to bite you’\\ 
\endgl
 \trailingcitation{[oxx-e120414ls-1a.134]}
\xe

\is{reality status|)}

In the following section, I will turn to a category that is still relatively unknown, although publications on this topic have been increasing during the last two decades. This is associated motion, the morphological expression of motion events together with a non-motion predicate.

