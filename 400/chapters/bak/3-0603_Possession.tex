%!TEX root = 3-P_Masterdokument.tex
%!TEX encoding = UTF-8 Unicode

\section{Possession}\label{sec:Possession}
\is{inflection|(}
\is{possession|(}
\is{person marking|(}

Possession is marked on the \isi{head} noun, i.e. the noun denoting the possessed. Possessed nouns take a person marker, which precedes the noun stem\is{nominal stem} and indexes the \is{possessor} as in (\ref{ex:new23-POSS}). 

\ea\label{ex:new23-POSS}
\begingl
\glpreamble pimuse\\
\gla pi-muse\\
\glb 2\textsc{sg}-mother.in.law\\
\glft ‘your (\textsc{sg}) mother-in-law’
\endgl
%\trailingcitation{[]}
\xe

The person markers are given in \tabref{table:POSS_Pref}.\is{possessor|(} They are identical to the ones that index subjects\is{subject} on verbs,\is{verb} with the only difference being that two third person markers are available for verbs, which are connected to differential object marking (see \sectref{sec:3Marking}), while nouns have only one third person marker. There is no \isi{gender} distinction in the third person, neither is the marker is specified for number. When a human\is{animacy} third person plural possessor is to be expressed, the \isi{plural} marker \textit{-nube} is added. The first person singular and the third person markers have allomorphs, both including a distinction between a high front and a high central vowel. The allomorphs with the front vowel predominantly occur when the following syllable contains an /i/ or /u/, the ones with the central vowel before syllables with /ÿ/, /ɛ/ and /a/, but this distribution is a tendency rather than an absolute rule. With only one exception to my knowledge, only the markers containing the central vowel can precede a vowel-initial syllable, although in most cases, the vowel of the person marker is deleted in such cases (see \sectref{section:Vowel_elision}).

\begin{table}
\caption{Person markers on possessed nouns}

\begin{tabular}{ll}
\lsptoprule
Person & Person marker \cr
\midrule
1\textsc{sg} & \textit{nÿ-/ni-}  \cr
2\textsc{sg} & \textit{pi- } \cr
3 & \textit{chÿ-/chi-}  \cr
1\textsc{pl} & \textit{bi-} \cr
2\textsc{pl} & \textit{e-} \cr
3\textsc{pl} & \textit{chÿ-/chi- ... -nube} \cr
\lspbottomrule
\end{tabular}

\label{table:POSS_Pref}
\end{table}
\is{person marking|)}

In most contexts with third person \isi{plural} possessors, there are also several possessed items, as in (\ref{ex:Possi-PL-1}). This becomes apparent from the context in which these nouns are used. One exception is the noun \textit{-ubiu} ‘house’. There is often only one house for various people, see (\ref{ex:Possi-PL}) .

\ea\label{ex:Possi-PL-1}
\begingl
\glpreamble chiyumaji – chiyumajinube\\
\gla chi-yumaji chi-yumaji-nube\\
\glb 3-hammock 3-hammock-\textsc{pl}\\
\glft ‘his/her hammock – their hammocks (or, less likely: their hammock)’
\endgl
%\trailingcitation{[]}
\xe

\ea\label{ex:Possi-PL}
\begingl
\glpreamble chubiu – chubiunube\\
\gla chi-ubiu chi-ubiu-nube\\
\glb 3-house 3-house-\textsc{pl}\\
\glft ‘his/her house – their house (or: their houses)’
\endgl
%\trailingcitation{[]}
\xe

If the possessed noun is non-human,\is{animacy} the \isi{plural} marker unambiguously relates to the possessor, but when it comes to possessed kin, it is often less clear whether the possessor or the possessed kin is pluralised. A detailed discussion about the ambiguity of \isi{plural} marking concerning possessed kin with third person possessors is postponed to \sectref{sec:Collective}.\is{possessor|)}

Nouns can be divided into three different classes according to how they interact with possession: there are inalienable, alienable and non-possessable nouns, see \sectref{sec:Inalienables}, \sectref{sec:Alienables} and \sectref{sec:Non-possessables} below. Roughly, the first of them must be possessed, the second ones can be possessed and the third ones cannot be possessed. This tripartite division is very typical for \is{Arawakan languages}  (\citealp[cf.][82]{Aikhenvald1999}; \citealt[]{Danielsen2014}) and in addition, the existence of a class of inalienable nouns is common in Amazonian languages\is{Amazonian language} in general \citep[88, 100]{Krasnoukhova2012}.


\subsection{Inalienable nouns}\label{sec:Inalienables}
\is{inalienability|(}

Inalienable nouns obligatorily express a possessor. This is related to the fact that in inalienable possession, there are “inextricable, essential or unchangeable relations between ‘possessor’ and ‘possessed’” \citep[4]{ChappelMcGregor1996}. 

According to \citet[572]{Nichols1988}, there is an implicational hierarchy among semantic groups that are conceived as inalienably possessed cross-linguistically. The hierarchy is given in \figref{fig:InalienabilityHierarchy}.

\begin{figure}[!ht]
\centering
Kin terms and/or body parts < Part-whole and/or spatial relations < Culturally basic possessed items (e.g. arrows, domestic animals)
\caption{Inalienability Hierarchy \citep[572]{Nichols1988}}
\label{fig:InalienabilityHierarchy}
\end{figure}

The semantic groups of nouns that are inalienably possessed in Paunaka are kinship terms, body parts, plant parts (to some extent), spatial relations, and some culturally basic items, so that Paunaka fully confirms the hierarchy.

All inalienable nouns obligatorily express the possessor by a person marker\is{person marking|(} preceding the possessed noun. An example with a kinship term is given in (\ref{ex:piati}). It was elicited from Juana.

\ea\label{ex:piati}
\begingl 
\glpreamble nimu piati ukuine\\
\gla ni-imu pi-ati ukuine\\ 
\glb 1\textsc{sg}-see 2\textsc{sg}-brother yesterday\\ 
\glft ‘I saw your brother yesterday’\\ 
\endgl
\trailingcitation{[jxx-e110923l-1.049]}
\xe
\is{person marking|)}

The noun \textit{-ati} ‘brother’ is used to refer to male siblings of females and the term \textit{-etine} ‘sister’ to refer to female siblings of males. If the reference is to a sibling of the same sex as the possessor, the noun \textit{-piji}, glossed here as ‘sibling’, is used, see (\ref{ex:piji-sibling}). The example also comes from Juana and is about her daughter who did not go to the airport to pick up her sister.

\ea\label{ex:piji-sibling}
\begingl 
\glpreamble kuina tiyuna chipiji\\
\gla kuina ti-yuna chi-piji\\ 
\glb \textsc{neg} 3i-go.\textsc{irr} 3-sibling\\ 
\glft ‘her sister didn’t go’\\ 
\endgl
\trailingcitation{[jxx-p110923l-1.299]}
\xe

Whether the referent is a male or a female person is only recoverable by the context.\footnote{As has already been stated in \sectref{section:Vowel_elision}, Paunaka does not make use of genderlects,\is{genderlect} with minimal exceptions.} An example with the kinship term \textit{-etine} ‘sister’ is given in (\ref{ex:netine}). It comes from Isidro, who greeted me.

\ea\label{ex:netine}
\begingl 
\glpreamble ¿michabi?, netine\\
\gla micha-bi nÿ-etine\\ 
\glb good-2\textsc{sg} 1\textsc{sg}-sister\\ 
\glft ‘how are you, my sister?’\\ 
\endgl
\trailingcitation{[mdx-c120416ls.008]}
\xe

The noun for ‘God’, \textit{bia}, is also a kinship term, it consists of the noun \textit{-a} ‘father’ and the first person plural marker\is{person marking|(} \textit{bi-}, so literally it means ‘our father’, though this seems to be intransparent to the speakers. In order to mark the difference between ‘our father’ and ‘God’, it looks as if the Paunaka re-analysed the noun stem \textit{-a} ‘father’ as \textit{-ÿa} as in (\ref{ex:God-father}). But this is only true for the first person plural, the second person singular is still \textit{pia} (\textit{pi-a} ‘your father’). If the stem were regular \textit{*-ÿa}, the second person singular should be \textit{*pÿa}.\footnote{But note that there is an additional vowel (or rather syllable) in the cognate form of \isi{Mojeño Trinitario} \textit{-iya}, and see also discussion in \sectref{section:Vowel_elision}.} In general, speakers rather avoid the first person plural form and speak of ‘my father’, \textit{nÿa} (\textit{nÿ-a}), instead. \textit{Bÿa} only showed up in elicitation. The same is true for the second person plural form, for which I could elicit \textit{ÿa}, but only after some contemplation about what the form could be.\is{person marking|)}

\ea\label{ex:God-father}
\begingl 
\glpreamble bia – bÿa\\
\gla bi-a bi-ÿa?\\ 
\glb 1\textsc{sg}-father 1\textsc{sg}-father\\ 
\glft ‘God – our father’\\ 
\endgl
\xe


Some kinship terms have suppletive endearment forms,\is{endearment|(}  which are free, non-possess\-able nouns used both as vocatives and referentials. The possessed forms on the other hand are never used as vocatives. \tabref{table:Vocatives} gives the forms.
The female ones, \textit{mimi} ‘mum’ and \textit{yeye} ‘granny’ are used a lot, the latter can be used in respectful reference to any older indigenous female person.\footnote{Non-indigenous women are called \textit{senyora} ‘Mrs, madame’ or \textit{senyorita} ‘Miss’ from Spanish \textit{señora} and \textit{señorita}, respectively.} \textit{Taita} ‘dad’ can also be used with non-kins, but occurs only rarely.

\begin{table}[htbp]
\caption{Kinship terminology with endearment forms}

\begin{tabular}{lll}
\lsptoprule
Kinship term & Endearment term & Translation\cr
\midrule
\textit{-enu} & \textit{mimi} & mother, mum \cr
\textit{-a} & \textit{taita} & father, dad \cr
\textit{-use} & \textit{yeye} & grandmother, granny \cr
\textit{-uchiku / -uma} & \textit{chÿchÿ} & grandfather, grandpa \cr
\lspbottomrule
\end{tabular}

\label{table:Vocatives}
\end{table}
\is{endearment|)} 

Most body-part terms are inalienably possessed \citep[]{TerhartDanielsenBODY}, one example is given in (\ref{ex:body-part-1}). It comes from María C. who was worried that her chicha would soon be finished.

\ea\label{ex:body-part-1}
\begingl 
\glpreamble tikuti nemua neamÿnÿ ÿne\\
\gla ti-kuti nÿ-emua nÿ-ea-mÿnÿ ÿne\\ 
\glb 3i-hurt 1\textsc{sg}-belly 1\textsc{sg}-drink.\textsc{irr}-\textsc{dim} water\\ 
\glft ‘my stomach hurts when I drink water’\\ 
\endgl
\trailingcitation{[ump-p110815sf.709-710]}
\xe

Plant parts are semantically closely related to body parts. These parts usually occur with a third person possessor \is{possessor|(} in Paunaka. The possessor is the plant in this case, which follows the possessed part, see (\ref{ex:leaves}) and (\ref{ex:branches}).\footnote{The other possibility is to use the parts as N2s in compounds, but compounding is not very productive in general in Paunaka, see \sectref{sec:Compounding}.}

\ea\label{ex:leaves}
\begingl 
\glpreamble chipuneji kÿjÿpi\\
\gla chi-pune-ji kÿjÿpi\\ 
\glb 3-leaf-\textsc{col} manioc\\ 
\glft ‘leaves of a manioc, manioc leaves’\\ 
\endgl
\trailingcitation{[nxx-a630101g-1.51]}
\xe

\ea\label{ex:branches}
\begingl 
\glpreamble chimusuji merÿ\\
\gla chi-musuji merÿ\\ 
\glb 3-skin plantain\\ 
\glft ‘banana peel’\\ 
\endgl
\trailingcitation{\citep[4]{Sell2019}}
\xe

A third person possessor is also chosen, when no possessor is lexically expressed, as in (\ref{ex:POSS-fruit}), which has \textit{chÿi} ‘its fruit’.\footnote{In the case of ‘leaf’, an additional \textit{e} is preposed to the plant part containing the third person possessor, for a reason unknown to me. Thus ‘leaf’ is normally realised as \textit{echÿpune} (\textit{e-chÿ-pune} ?-3-leaf), though the plural (i.e. \isi{collective}) form is \textit{chipuneji}, see (\ref{ex:leaves}).} María S. gives an explanation about a plant here.

\ea\label{ex:POSS-fruit}
\begingl 
\glpreamble takujibÿ eka te kanainatu chÿi te puero binika\\
\gla ti-a-kujibÿ eka te kana-ina-tu chÿ-i te puero bi-nika\\ 
\glb 3i-\textsc{irr}-have.flower \textsc{dem}a \textsc{seq} this.size-\textsc{irr.nv}-\textsc{iam} 3-fruit \textsc{seq} can 1\textsc{pl}-eat.\textsc{irr}\\ 
\glft ‘it blossoms and once its fruits have this size (showing with hands), we can eat them’\\ 
\endgl
\trailingcitation{[rxx-e121126s-3.16]}
\xe
\is{possessor|)}


Among the culturally basic items that are inalienably possessed in Paunaka are \textit{-sane} ‘field’, \textit{-mÿu} ‘clothes’, \textit{-etea} ‘language’, and also some parasites like \textit{-kane} ‘worm’ and \textit{-ine} ‘louse’. The word for ‘house’, \textit{-ubiu}, originated as a \isi{deranked verb}, but acquired characteristics of a noun like being able to combine with the locative marker (see \sectref{sec:Subordination-i}). It is also inalienably possessed. As for other nouns which typically fall into the class of inalienables, the word for ‘arrow’ is not remembered by the speakers and a word for ‘village’ does not exist.\footnote{There is, however, \textit{-epukie} ‘homeland, home’, which belongs to the inalienable class. In contrast, \textit{uneku} ‘town’ is usually not possessed, but may be used in a construction containing the general relational noun depending on the speaker, see \sectref{sec:Non-possessables} below.}

(\ref{ex:Npossi-1}) was produced by Isidro to answer Swintha’s question whether he would go to his field the next day.

\ea\label{ex:Npossi-1}
\begingl
\glpreamble tajaitu niyunupunuka nisaneyae\\
\gla tajaitu ni-yunu-punuka ni-sane-yae\\
\glb tomorrow 1\textsc{sg}-go-\textsc{reg.irr} 1\textsc{sg}-field-\textsc{loc}\\
\glft ‘tomorrow I will go to my field again’
\endgl
\trailingcitation{[dxx-d120416s.129]}
\xe

(\ref{ex:Npossi-2}) comes from Miguel who was looking at a little wooden toy figure and identified it as female.

\ea\label{ex:Npossi-2}
\begingl
\glpreamble chimÿu tÿnai entonses apimiyapÿimÿnÿ\\
\gla chi-mÿu ti-ÿnai entonses apimiyapÿi-mÿnÿ\\
\glb 3-clothes 3i-be.long thus girl-\textsc{dim}\\
\glft ‘its garment is long, so it is a girl’
\endgl
\trailingcitation{[mox-e110914l-1.049]}
\xe

Another semantic subclass of nouns that has been identified as a typical member of the inalienable class is spatial relations \citep[cf.][572]{Nichols1988}. There is indeed a small class of relational nouns\is{relational noun|(} in Paunaka, which can take the \isi{locative marker}. They express some specific spatial relations and are juxtaposed\is{juxtaposition} to the noun denoting the ground. The latter acts as the \isi{possessor} of the relational noun. There is also a number of free spatial nouns like \textit{anÿke} ‘up, above’, \textit{apuke} ‘ground, down’ and \textit{pÿkÿjÿe} ‘middle’ which are never possessed. Here is one example of a spatial relation that is inalienably possessed by the noun referring to the ground, more examples are given in \sectref{sec:Locative}.\footnote{This kind of construction is very similar to the one in Baure, see \citet[72--74]{Admiraal2016}.}

(\ref{ex:POSS-under}) comes from an elicitation session with Juana and María S. with playmobil toys.\footnote{Note that Juana sometimes uses \textit{-pÿtapaiku-bu} instead of \textit{-bÿtupaiku-bu} ‘fall’ for reasons unknown to me.}

\ea\label{ex:POSS-under}
\begingl
\glpreamble chÿupekÿ mura chipÿtapaikutu\\
\gla chÿ-upekÿ mura chi-bÿtupaiku-tu\\
\glb 3-place.under horse 3-make.fall-\textsc{iam}\\
\glft ‘it is under the horse, it throw it down’
\endgl
\trailingcitation{[jrx-c151024lsf]}
\xe
\is{relational noun|)}

As can be seen from the previous examples, (\ref{ex:leaves}), (\ref{ex:branches}), and (\ref{ex:Npossi-1}) to (\ref{ex:POSS-under}), if the possessor is expressed lexically, \isi{word order} in the NP\is{noun phrase} is always possessed – possessor.

Free nouns can be derived\is{derivation|(} from inalienables by addition of the suffix \textit{-ti}, see (\ref{ex:sane-asaneti}) and (\ref{ex:mukiji-mukitiji}). This is the only “non-possessed” suffix in the language.

\ea\label{ex:sane-asaneti}
\begingl 
\glpreamble nisane – asaneti\\
\gla ni-sane asane-ti\\ 
\glb 1\textsc{sg}-field field-\textsc{npossd}\\ 
\glft ‘my field – field’\\ 
\endgl
%\trailingcitation{[]}
\xe

\ea\label{ex:mukiji-mukitiji}
\begingl 
\glpreamble nimukiji – mukitiji\\
\gla ni-muki-ji muki-ti-ji\\ 
\glb 1\textsc{sg}-hair-\textsc{col} hair-\textsc{npossd}-\textsc{col}\\ 
\glft ‘my hair – hair’\\ 
\endgl
%\trailingcitation{[]}
\xe

%mukitiji = hair
%asaneti = field
%iti = blood, but lexicalised!
%yubuti = hacha??
%ucheti = chili ?
%upiti = abeja sp. ?? /ipiti
%
%also nÿti, piti etc.
Apart from \textit{asaneti} ‘field’, non-possessed forms of inalienable nouns are not very frequent in my data. The non-possessed suffix seems to be lexicalised\is{lexicalisation} on the alienably possessed noun \textit{yubuti} ‘axe’, i.e. it is not detached, when the noun is marked for possession.\is{derivation|)}

There is no prefix for an unspecified possessor.\footnote{A specialised prefix for an unspecified or indefinite possessor does exist in many \isi{Arawakan languages}, e.g. in closely related \isi{Baure}  \citep[119--120]{Danielsen2007} and more distantly related Tariana \citep[123]{Aikhenvald2003}.} %In the closely related language Baure, free nouns can be derived from body part terms with the prefix \textit{e-}, which marks an unspecified possessor, i.e. the body part is still possessed, but it is not clear by whom \citep[119--120]{Danielsen2007}.
Instead of deriving a non-possessed form or marking an unspecified possessor\is{possessor|(} on the inalienable noun, Paunaka speakers resort to a different strategy: a third person or a first person plural marker\is{person marking|(} can be used to express a more general reference of a noun. 

Plant parts, for instance, are never marked as non-possessed in the corpus. They take a third person marker by default, even when the part is detached from the plant. The same is true for the noun \textit{chÿeche} ‘meat’ (\textit{chÿ-eche} 3-flesh, lit.: ‘his/her/its flesh’), note that the same noun is used to refer to flesh as a body part, e.g. \textit{nÿeche} ‘my flesh’ (\textit{nÿ-eche} 1\textsc{sg}-flesh).

The use of a third person marker to indicate a general possessor was also the solution the PDP team chose for the production of a poster with body part terminology for the speaker community \citep[cf.][]{PDP2013}. Although it was generally agreed on by the speakers that this was correct, we later found out that for general reference to body parts, such as in school books, medical descriptions etc., people rather use the first person plural marker \textit{bi-},\footnote{Nouns with first person plural possessors are also found in some of the few entries for Paunaka vocabulary by d’Orbigny without being analysed as such (Danielsen 2021, p.c.).} consider (\ref{ex:Npossi-3}), which comes from Juana who was telling me about the medical use of the soursop.\footnote{I translate \textit{-kÿna} with ‘heart’ here, which is the core meaning of the noun, but it is also used for the stomach and the whole interior of the torso \citep[cf.][265]{TerhartDanielsenBODY}. Thus Juana could have also meant one of these concepts rather than precisely the heart.}

\ea\label{ex:Npossi-3}
\begingl
\glpreamble jaja upichai tÿpi bikÿna\\
\gla jaja upichai tÿpi bi-kÿna\\
\glb \textsc{afm} medicine \textsc{obl} 1\textsc{pl}-heart\\
\glft ‘yes, it is medicine for our heart’
\endgl
\trailingcitation{[jxx-e150925l-1.066]}
\xe
\is{person marking|)}
\is{possessor|)}
\is{inalienability|)}

\subsection{Alienable nouns}\label{sec:Alienables}
\is{alienability|(}

Alienable possession is “less permanent and inherent” than inalienable possession \citep[4]{ChappelMcGregor1996}. Nouns denoting manipulable objects usually belong to the class of alienable nouns and this is also true for Paunaka. Loans\is{borrowing} from Spanish (and less often Bésiro) are also typically alienable, even the ones denoting friends and kins – those latter ones usually show up in the derived, inalienable form (see below). Grammatically, alienability is reflected by the fact that these nouns are free forms in the first place, but can be marked as possessed.

There are two different sub-classes of alienably possessed nouns. A small number of nouns can be marked for possession directly, i.e. the only difference between a possessed and an unpossessed form is the presence of a person marker indexing the possessor.\is{person marking}\is{possessor|(} Some examples are given in \tabref{table:Inalienables1}.

\begin{table}
\caption{Alienable nouns that can be marked for possession directly}

\begin{tabular}{lll}
\lsptoprule
Non-possessed form & Possessed form (1\textsc{sg}) & Translation\cr
\midrule
\textit{kasune} & \textit{nikasune} & (my) trousers \cr
\textit{kuepia} & \textit{nikuepia} & (my) kidney\cr
\textit{nÿkÿiki} & \textit{ninÿkÿiki} & (my) pot \cr
\textit{pusane} & \textit{nipusane} & (my) bag \cr
\textit{yumaji} & \textit{niyumaji} & (my) hammock \cr
\lspbottomrule
\end{tabular}

\label{table:Inalienables1}
\end{table}

(\ref{ex:ali-1}) has a possessed form of \textit{-yumaji} ‘hammock’, which does not take the possessed suffix. Compare with (\ref{ex:ali-2}), in which the word occurs in non-possessed form.

(\ref{ex:ali-1}) comes from María C., who was afraid that her hammock would get wet because it was about to rain.

\ea\label{ex:ali-1}
\begingl
\glpreamble kaku niyumaji nekupai\\
\gla kaku ni-yumaji nekupai\\
\glb exist 1\textsc{sg}-hammock outside\\
\glft ‘my hammock is outside’
\endgl
\trailingcitation{[cux-120410ls.258]}
\xe


(\ref{ex:ali-2}) was a question by Juana directed to me.

\ea\label{ex:ali-2}
\begingl
\glpreamble ¿pisachu pibena yumaji?\\
\gla pi-sachu pi-bena yumaji\\
\glb 2\textsc{sg}-want 2\textsc{sg}-lie.down.\textsc{irr} hammock\\
\glft ‘do you want to lie down in the hammock?’
\endgl
\trailingcitation{[jxx-p150920l.017]}
\xe


The larger number of alienable nouns takes the suffix \textit{-ne} to derive\is{derivation|(} a possessable form. This form can be considered inalienable, since it obligatorily takes a person marker\is{person marking} for the possessor. Two nouns in \tabref{table:Inalienables1}, \textit{kasune} ‘trousers’ and \textit{pusane} ‘bag’ also end in \textit{-ne} in their non-possessed form.\footnote{In addition, there are also a few inalienable nouns that end in \textit{ne}, e.g. \textit{-etine} ‘sister (of a male person)’ and \textit{-machapene} ‘liver’.} Both are borrowed\is{borrowing} from \isi{Bésiro} or Proto-Chiquitano, \textit{pusane} derives from Proto-Chiquitano \textit{*/pɨtsaná-ʂɨ/} \citep[10]{Nikulin2019}, and \textit{kasune} from Bésiro \textit{<kasuná-x>}, which is itself a loan from Spanish \textit{calzón} ‘pants, underpants, shorts’ \citep[12]{Nikulin2019}. Thus in both cases the syllable \textit{ne} can be considered part of the root.

Nouns derived with \textit{-ne} take a person marker to index the possessor just like other inalienable nouns.\is{possessor|)} According to \citet[378]{Payne1991}, \textit{-ne} is the most common possessive suffix among the \isi{Arawakan languages}. It is the only one that Paunaka productively makes use of. This is mentioned here explicitly because some other Arawakan languages have more than one.

(\ref{ex:Alienables2-1}) and (\ref{ex:Alienables2-2}) show inalienable nouns in their non-possessed and possessed forms follow.%\footnote{There are different kinds of baskets of different materials and for different uses. \textit{Sÿki} is \textit{jasayé} in Spanish, it is the one that resembles a bag, \textit{panaku} corresponds to Spanish \textit{panacú}, this one can be carried on the back. The \textit{chupai}, Spanish \textit{quiboro}, is rather used to store away things.}

\ea\label{ex:Alienables2-1}
\begingl 
\glpreamble tÿmuepa – nitÿmuepane\\
\gla tÿmuepa ni-tÿmuepa-ne\\ 
\glb knife 1\textsc{sg}-knife-\textsc{possd}\\ 
\glft ‘knife – my knife’
\endgl
%\trailingcitation{[mox-n110920l.021]}
\xe

\ea\label{ex:Alienables2-2}
\begingl 
\glpreamble sÿki – nisÿkine\\
\gla sÿki ni-sÿki-ne\\ 
\glb basket 1\textsc{sg}-basket-\textsc{possd}\\ 
\glft ‘basket – my basket’\\ 
\endgl
%\trailingcitation{[cux-c120414ls-1.177]}
\xe

%yubuti - niyubutine (mxx-e181017l)
\is{derivation|)}

The overwhelming number of nouns belonging to the class of alienables are loans\is{borrowing} from Spanish denoting either objects or people with a kinship or other social relation to the possessor. (\ref{ex:Alienables2-3}) shows a borrowed kinship term, \textit{-kumare}\footnote{\textit{Kumare} comes from Spanish \textit{comadre} and is used to denote either the godmother of one’s child or godchild or a close friend of the same age group. It has several other phonetic realisations, e.g. \textit{kumade}.} with and without possession marking occurring in a single sentence. It comes from Juana who was talking about the trip to Europe.

\ea\label{ex:Alienables2-3}
\begingl 
\glpreamble echÿu nikumarene nauku Concecion, kumare Nacha, kuina tisacha tiyuna, tÿbaneyu\\
\gla echÿu ni-kumare-ne nauku Concecion kumare Nacha kuina ti-sacha ti-yuna ti-ÿbane-yu\\ 
\glb \textsc{dem}b 1\textsc{sg}-fellow-\textsc{possd} there Concepción fellow Nacha \textsc{neg} 3i-want.\textsc{irr} 3i-go.\textsc{irr} 3i-be.far-\textsc{ints}\\ 
\glft ‘my fellow there in Concepción, fellow Nacha, doesn’t want to go, because it is very far’\\ 
\endgl
\trailingcitation{[jxx-p120430l-1.175]}
\xe

(\ref{ex:Alienables2-4}) has the Spanish loan \textit{kama} from \textit{cama} ‘bed’, which denotes an object. It comes from Isidro who was describing a picture of a puzzle game.

\ea\label{ex:Alienables2-4}
\begingl 
\glpreamble timuku eka chikamaneyae\\
\gla ti-muku eka chi-kama-ne-yae\\ 
\glb 3i-sleep \textsc{dem}a 3-bed-\textsc{possd}-\textsc{loc}\\ 
\glft ‘he sleeps in his bed’\\ 
\endgl
\trailingcitation{[dxx-d120416s.002]}
\xe

If attributive verbs\is{attributive prefix} (see \sectref{sec:AttributiveVerbs}) are derived\is{derivation} from alienable nouns, the possessed form of the noun is used as in (\ref{ex:ali-3}), which comes from elicitation with María S.

\ea\label{ex:ali-3}
\begingl
\glpreamble nikupanakune\\
\gla ni-ku-panaku-ne\\
\glb 1\textsc{sg}-\textsc{attr}-basket-\textsc{possd}\\
\glft ‘I have a basket (on the back)’
\endgl
\trailingcitation{[rxx-e181020le]}
\xe
\is{alienability|)}
 
\subsection{Non-possessable nouns}\label{sec:Non-possessables}
\is{non-possessability|(}

Paunaka has a number of non-possessable nouns, which cannot take person markers.\is{person marking} Some of them can be possessed indirectly, they “require an additional grammatical element joining the two constituents” (i.e. the possessed and the possessor) \citep[58]{Krasnoukhova2012}. This is achieved by using a possessable noun including the \isi{possessor} marking in \isi{juxtaposition} to the non-possessable noun. The possessed noun always precedes the non-possessable one in this case, which can be seen in (\ref{ex:new23nonposs}) from Isidro describing a puzzle game on which a boy plays with a squirrel. 

\ea\label{ex:new23nonposs}
\begingl
\glpreamble chipeu mase\\
\gla chi-peu mase\\
\glb 3-animal squirrel\\
\glft ‘his squirrel’
\endgl
\trailingcitation{[mdx-c120416ls.177]}
\xe


\citet[82]{Aikhenvald1999} mentions that non-possessable nouns in \isi{Arawakan languages} “may include astronomical bodies, natural phenomena, harmful animals and personal names”. %\citet[58]{Krasnoukhova2012}: they “cannot occur with the possessor directly, and therefore require an additional grammatical element joining the two constituents”.
In Paunaka, at least some speakers allow a possessed form of astronomical bodies if used in a metaphorical context, e.g. in talking to one’s lover, see (\ref{ex:possessed-star}), which was elicited from Miguel.

\ea\label{ex:possessed-star}
\begingl 
\glpreamble nÿjaikene\\
\gla nÿ-jaike-ne\\ 
\glb 1\textsc{sg}-star-\textsc{possd}\\ 
\glft ‘my star’\\ 
\endgl
\trailingcitation{[mxx-e181017l]}
\xe

This shows that non-possessability is for semantic reasons and does not have to do with morphological properties of the noun in question. \citet[170]{Aikhenvald2012} explains that it is common sense among speakers of Amazonian languages\is{Amazonian language} that some items simply cannot be possessed, but the principles underlying are language and culture specific.

The most important group of non-possessable nouns in Paunaka is animals. Besides harmful animals as  predicted by \citet[82]{Aikhenvald1999}, this also includes pets, but not parasites like lice and worms, which are inalienably possessed. As has already been shown in (\ref{ex:new23nonposs}) above, the indirect strategy to express possession of animals includes the inalienable noun \textit{-peu} ‘domestic animal’ as a relational noun \is{relational noun|(}.\footnote{It is only given as ‘animal’ in the glosses of examples for the sake of brevity.} This noun carries the person marker\is{person marking} and the noun denoting the specific animal follows.\footnote{This type of construction has also been treated in the literature under the heading of genitive classifiers\is{classifier} (cf. \citealt[66]{Grinevald2000}; \citealt[283]{Campbell2012}) or possessive classifiers \citep[]{Fabre2014}. Languages with this kind of classifiers\is{classifier} often have more elaborate systems, but if a language in Amazonia has only one,\is{Amazonian language} it is usually the one for ‘domestic animal’ (Rose 2021, p.c.).} 

Another example is given in (\ref{ex:POSS-animal}). It comes from María S., who was complaining that her chicken get stolen when she leaves her house.

\ea\label{ex:POSS-animal}
\begingl 
\glpreamble kuina dejaunubeina nipeu takÿra\\
\gla kuina dejau-nube-ina ni-peu takÿra\\ 
\glb \textsc{neg} leave-\textsc{pl}-\textsc{irr.nv} 1\textsc{sg}-animal chicken\\ 
\glft ‘they don't leave my chicken alone’ (i.e. they steal them)\\ 
\endgl
\trailingcitation{[rxx-e120511l.179]}
\xe

(\ref{ex:aniposs-1}) comes from Miguel telling Alejo the \isi{frog story}. He describes the picture on which the boy stands on the stone. 

\ea\label{ex:aniposs-1}
\begingl
\glpreamble pero kapunuji echÿu chipeu kabemÿnÿ\\
\gla pero kapunu-ji echÿu chi-peu kabe-mÿnÿ\\
\glb but come-\textsc{rprt} \textsc{dem}b 3-animal dog-\textsc{dim}\\
\glft ‘but his dog is coming, it is said’
\endgl
\trailingcitation{[mtx-a110906l.147]}
\xe


This pattern, i.e. the expression of possession of an animal by \isi{juxtaposition} of a possessed noun ‘domestic animal’ and the name of the specific animal, is shared with \isi{Terena}, the \isi{Mojeño languages}, and \isi{Baure} (cf. \citealt[50]{ButlerEkdahl2012}; \citealt[51]{OlzaZubiri2004}; Rose 2021, p.c.; \citealt[123--124]{Danielsen2007}). In the Mojeño languages, however, the cognate nouns only denotes rideable animals. Trinitario\is{Mojeño Trinitario} uses a more general relational noun for possession of non-rideable animals \citep[cf.][79]{Rose2014}.\footnote{Ignaciano\is{Mojeño Ignaciano} on the other hand uses another specific noun for non-rideable domestic animals and the more general one for non-animals \citep[51]{OlzaZubiri2004}.} Paunaka also has a general relational noun \textit{-yae} (cognate to the Mojeño one). It is semantically unspecific and identical in form with the \isi{locative marker}. 

In (\ref{ex:possessed-town}) the relational noun occurs in juxtaposition to the noun \textit{uneku} ‘town’. According to Miguel, who provided this example in elicitation, this is the correct way to express the notion of one’s town. According to María S., however, ‘town’ cannot be possessed at all.

\ea\label{ex:possessed-town}
\begingl 
\glpreamble niyae uneku\\
\gla ni-yae uneku\\ 
\glb 1\textsc{sg}-\textsc{grn} town\\ 
\glft ‘my town’\\ 
\endgl
\trailingcitation{[mxx-e181017l]}
\xe

According to the analysis by \citet[]{Rose2019a} for the Trinitario\is{Mojeño Trinitario} equivalent of Paunaka’s \textit{-yae}, it first arose as a general relational noun before spreading to other contexts. As such, it has cognate forms in other \isi{Arawakan languages} (\citealt[14]{Rose2019a}, and consider also the “possessive \textit{-ya-}” of Tariana, \citealt[134]{Aikhenvald2003}). Note that \citet[150]{Danielsen2007} also considers that the \isi{Baure} locative marker \textit{-ye}, another cognate of Paunaka’s \textit{-yae}, could be a \isi{nominal root}. Since function and morphosyntactic contexts are relatively different in current Paunaka, I use two different glosses for \textit{-yae}. If it occurs in possession contexts together with a person marker,\is{person marking} I call it general relational noun, abbreviated \textsc{grn} in following \citet[]{Rose2019a}, in those contexts where it is attached to a noun, I call it a \isi{locative marker}, abbreviated \textsc{loc} (for locative marking see \sectref{sec:Locative}).

In current Paunaka, the general relational noun is found in genitive predication (see \sectref{sec:GenitiveBenfactivePreds}), but it can also occur in contexts of attributive possession. Possession of crops can be expressed in this way. I first came across this in one of the recordings  of the 1960s by Riester, see (\ref{ex:ATTRyae}). 

\ea\label{ex:ATTRyae}
\begingl 
\glpreamble akomoraupuna niyaemÿnÿ arusu\\
\gla akomorau-puna ni-yae-mÿnÿ arusu\\ 
\glb accommodate-\textsc{am.prior}.\textsc{irr} 1\textsc{sg}-\textsc{grn}-\textsc{dim} rice\\ 
\glft ‘I will go to store away my rice’\\ 
\endgl
\trailingcitation{[nxx-p630101g-1.006]}
\xe

Elicitation showed that crops can either be marked as possessed in this way or by using the alienable possession strategy explained above (see \sectref{sec:Alienables}). Which strategy is used for which crop may depend on frequency and speaker. (\ref{ex:ucheti23}) is another example of a crop that was spontaneously marked for possession by using the general relational noun \textit{-yae} in elicitation:

\ea\label{ex:ucheti23}
\begingl 
\glpreamble niyae ucheti\\
\gla ni-yae ucheti\\ 
\glb 1\textsc{sg}-\textsc{grn} chili\\ 
\glft ‘my chili’\\ 
\endgl
\trailingcitation{[rxx-e181018le]}
\xe

 %Besɨro besides a possessive classifier for domestic animals also has a possessive classifier for domesticated plants \citep[cf.][21--22]{Sans2013}.

Instead of using one of the relational nouns \textit{-peu} or \textit{-yae}, speakers may also use other possessable nouns in juxtaposition to the non-possessable ones. Consider (\ref{ex:water-possessed}), where the natural resource of water is marked as possessed by a preceding possessed noun that simultaneously acts as a measure term for the \isi{mass noun}. The noun \textit{tapiki} is borrowed from Spanish \textit{tapeque} or Proto-Chiquitano \textit{tapiki} ‘travel supplies’ \citep[cf.][9]{Nikulin2019}. This example also comes from elicitation with María S.

\ea\label{ex:water-possessed}
\begingl 
\glpreamble pitapikine ÿne\\
\gla pi-tapiki-ne ÿne\\ 
\glb 2\textsc{sg}-travel.supplies-\textsc{possd} water\\ 
\glft ‘your travel supplies of water’\\ 
\endgl
\trailingcitation{[rxx-e181018le]}
\xe

\is{relational noun|)}

\is{non-possessability|)}
\is{possession|)}

The next section is about number marking on nouns.

