%!TEX root = 3-P_Masterdokument.tex
%!TEX encoding = UTF-8 Unicode

\section{Imperatives and other directives}\label{sec:Imperative}
\is{directive speech act|(}

Imperatives are directive speech acts. They can express a “command, request, offer, advisory, or exhortation” \citep[277]{Koenig2007}. In Paunaka, imperatives usually build on active verbs.\is{active verb} They can be identical in structure to a \isi{declarative clause}. I call this type of imperatives “unmarked”, although they require \isi{irrealis} RS. Irrealis could occur in declarative clauses for other reasons. Unmarked imperatives inflect for person;\is{person marking} they either have second person singular or second person plural subjects.\is{subject} Objects\is{object} can be indexed on these imperatives and conominated objects\is{conomination} can occur, but they are never placed in \isi{focus} position preceding the verb.\is{word order}  Some examples of unmarked imperatives can be found in \sectref{sec:UnmarkedImperatives}. %“directive speech acts, i.e. orders and requests, but also invitations, the giving of advice, warnings, wishes, instructions, etc.” \citep[303]{Koenig2007}

Alternatively, imperatives can be marked by adding the suffix \textit{-ji} to the end of the verb. I cannot tell what is the exact difference to unmarked imperatives, but more emphasis seems to be involved. They are possibly only used to make requests and commands (i.e. no offers), but this remains to be verified. Emphatic imperatives are the topic of \sectref{sec:MarkedImperatives}. 

There are two suppletive motion imperatives:\is{motion predicate} \textit{nabi}/\textit{nabue} ‘go!’ and \textit{pana} ‘come!’, they are dealt with in \sectref{sec:SuppletiveImperatives}. 

Negative imperatives can look like negative declarative sentences\is{declarative clause} if they include the standard negation particle\is{negative particle} \textit{kuina} and an \isi{irrealis} predicate. However, they can also be formed with a \isi{realis} predicate and the prohibitive\is{directive speech act!prohibitive}\is{negation!prohibitive} particle \textit{naka} or the \isi{admonitive} particle \textit{masaini}. The latter is rather used in warnings. Different kinds of negative imperatives are described in \sectref{sec:Prohibitives}.

Hortatives are formed with the \isi{hortative} particle \textit{jaje}. This is the topic of \sectref{sec:Hortatives}.
%Optatives/jussives with \textit{-jÿti} and \textit{-yuini}



\subsection{Unmarked imperatives}\label{sec:UnmarkedImperatives}
\is{imperative|(}

Imperatives may be unmarked, but the verb usually has \isi{irrealis} RS (some possible exceptions are discussed in the end of this section). The \isi{verb} is not inflected\is{inflection} for TAME. Imperatives have second person singular or plural addressees,\is{addressee of imperative} with singular being more frequently found in the corpus. They take the same person indexes\is{person marking} that are also found in declarative sentences. (\ref{ex:imp-1})--(\ref{ex:imp-5}) have singular and (\ref{ex:imp-6})--(\ref{ex:imp-9}) plural addressees. Unmarked imperatives do not only express commands, but also requests or invitations/offers. It is \isi{intonation} alone that marks different degrees of politeness or friendliness, thus setting apart commands from all other possible uses.

By producing (\ref{ex:imp-1}), Juana offered me something to drink.

\ea\label{ex:imp-1}
\begingl
\glpreamble ¡pea!\\
\gla pi-ea\\
\glb 2\textsc{sg}-drink.\textsc{irr}\\
\glft ‘drink!’
\endgl
\trailingcitation{[jxx-p150920l.002]}
\xe

(\ref{ex:imp-2}) is an imperative including a goal argument. It also comes from Juana and was directed to Miguel to help her load the loam she had collected onto her head to carry it.

\ea\label{ex:imp-2}
\begingl
\glpreamble ¡petuka nitapukiyae!\\
\gla pi-etuka ni-tapuki-yae\\
\glb 2\textsc{sg}-put.\textsc{irr} 1\textsc{sg}-head-\textsc{loc}\\
\glft ‘put it on my head!’
\endgl
\trailingcitation{[jmx-d110918ls-1.110]}
\xe

In (\ref{ex:imp-3}), a first person singular object is indexed on the verb. The example comes from the story about the lazy man by Miguel. When he has finally cut off his limbs, pretending they were \textit{cusi} palm fruits, he requests his son to lift him and put him into the basket to be carried, since he cannot walk anymore without legs.

\ea\label{ex:imp-3}
\begingl
\glpreamble “bueno, ¡pakachane pipurtukane naka sÿkiyae!”\\
\gla bueno pi-akacha-ne pi-purtuka-ne naka sÿki-yae\\
\glb well 2\textsc{sg}-lift.\textsc{irr}-1\textsc{sg} 2\textsc{sg}-put.in.\textsc{irr}-1\textsc{sg} here basket-\textsc{loc}\\
\glft ‘“well, lift me and put me into the basket!”’
\endgl
\trailingcitation{[mox-n110920l.118]}
\xe

(\ref{ex:imp-4}) represents what Jesus tells the monkey in the creation story narrated by Juana. The background is that the monkey had stolen corn and hidden it in his mouth, although the corn was meant for the people to eat.

\ea\label{ex:imp-4}
\begingl
\glpreamble “¡piyuna pinika eka chÿi yÿkÿke!”\\
\gla pi-yuna pi-nika eka chÿi yÿkÿke\\
\glb 2\textsc{sg}-go.\textsc{irr} 2\textsc{sg}-eat.\textsc{irr} \textsc{dem}a fruit tree\\
\glft ‘“go and eat the fruit of the trees”’
\endgl
\trailingcitation{[jxx-n101013s-1.873]}
\xe

(\ref{ex:imp-5}) is an example which contains an associated motion marker. It was produced by Clara and directed to María C. to give her advice on how her children could learn some Paunaka. The first clause is a directive which does not build on an imperative clause, but rather makes use of the borrowed modal expression \textit{tiene ke} ‘must’. It does not inflect for person. The second clause, i.e. the direct speech complement, is an imperative.

\ea\label{ex:imp-5}
\begingl
\glpreamble pero pue tiene ke pikechachi: “¡pinipuna nichechapÿibi!”\\
\gla pero pue {tiene ke} pi-kecha-chi pi-ni-puna ni-chechapÿi-bi\\
\glb but well {must} 2\textsc{sg}-say.\textsc{irr}-3 2\textsc{sg}-eat-\textsc{am.prior.irr} 1\textsc{sg}-son-2\textsc{sg}\\
\glft ‘but, well, you have to tell him: “come and eat, my dear son!”’
\endgl
\trailingcitation{[cux-c120414ls-2.302]}
\xe

(\ref{ex:imp-6}) has a second person plural addressee, Miguel, Swintha and me. Juana and Miguel dug for loam, and the sentence is an exclamation by Juana, being excited about the quality of the loam she found.

\ea\label{ex:imp-6}
\begingl
\glpreamble ¡emua, micha michana muteji!\\
\gla e-imua micha michana muteji \\
\glb 2\textsc{pl}-see.\textsc{irr} good nice loam \\
\glft ‘look, the loam is good, beautiful!’
\endgl
\trailingcitation{[jmx-d110918ls-1.089]}
\xe

In (\ref{ex:imp-7}), Juana cites what their landlord said to her daughter. 

\ea\label{ex:imp-7}
\begingl
\glpreamble “¡esemaika juchubu ejecheka! porke kopaunatu nubiu”, tikechu\\
\gla e-semaika juchubu e-jecheka porke kopau-ina-tu nÿ-ubiu ti-kechu\\
\glb 2\textsc{pl}-search.\textsc{irr} where 2\textsc{pl}-move.\textsc{irr} because use-\textsc{irr.nv}-\textsc{iam} 1\textsc{sg}-house 3i-say\\
\glft ‘“look for where to move, because I want to use my house for myself!”, he said’
\endgl
\trailingcitation{[jxx-p120430l-1.397]}
\xe

(\ref{ex:imp-8}) shows that an adverb can precede the verb in an imperative.\is{word order} This sentence comes from the story about the cowherd and the spirit of the hill told by Miguel. When he has passed some time with the spirit, the cowherd finally brings the cows to a village with the help of some people. This is what the cowherd tells the people, before he actually releases the cows.

\ea\label{ex:imp-8}
\begingl 
\glpreamble “¡nakajiku ekichupupuikanÿ!”, tikechuchÿji\\
\gla naka-jiku e-kichupu-puika-nÿ ti-kechu-chÿ-ji\\ 
\glb here-\textsc{lim}1 2\textsc{pl}-wait-\textsc{cont}.\textsc{irr}-1\textsc{sg} 3i-say-3-\textsc{rprt}\\ 
\glft ‘“wait for me right here!”, he said to them, it is said’\\ 
\endgl
\trailingcitation{[mxx-n151017l-1.81]}
\xe

(\ref{ex:imp-9}) is a request of Juana’s sister to the policemen after she has been arrested for the deeds of her husband.

\ea\label{ex:imp-9}
\begingl
\glpreamble “¡epuninane nijinepÿimÿnÿ!"\\
\gla e-epun-ina-ne ni-jinepÿi-mÿnÿ\\
\glb 2\textsc{pl}-take-\textsc{ben.irr}-1\textsc{sg} 1\textsc{sg}-daughter-\textsc{dim}\\
\glft ‘“take my daughter to me!”’
\endgl
\trailingcitation{[jxx-p120430l-2.101]}
\xe

There are also examples in the corpus albeit very few, in which an imperative seems to be formed with a \isi{realis} verb. They all have in common that they are exclamations. It might thus be the case that realis is possible in those specific cases. Otherwise, these examples could also simply be taken as mistakes, considering (\ref{ex:imp-6}) above, which is an exclamation, too, but has an irrealis verb nonetheless. Or they are no imperatives at all, but rather a verbalisation of an ongoing action. Two examples are given below.

(\ref{ex:imp-exc-1}) was produced by María S, when she showed the flower of a plant to Swintha.

\ea\label{ex:imp-exc-1}
\begingl
\glpreamble ¡pimu! chibu eka chÿina\\
\gla pi-imu chibu eka chÿi-ina\\
\glb 2\textsc{sg}-see 3\textsc{top.prn} \textsc{dem}a fruit-\textsc{irr.nv}\\
\glft ‘look, this will be its fruit!’\\or: ‘you see, this will be its fruit!’
\endgl
\trailingcitation{[rxx-e121126s-3.17-18]}
\xe

(\ref{ex:imp-exc-2}) is from Riester’s recordings. It was produced by Juan Ch. as part of an introduction to his playing the flute. Note that the morphologically stative verb \textit{-kusabenu} ‘play flute’ (an attributive derivation, see \sectref{sec:AttributiveVerbs}), seems to include the subordinating suffix \textit{-i} here (see \sectref{sec:Subordination-i}). This is relatively uncommon, but happens from time to time.\is{deranked verb}

\ea\label{ex:imp-exc-2}
\begingl
\glpreamble ¡esamu kristianunube! nikusabenuiu baile suelto\\
\gla e-samu kristianu-nube ni-kusabenu-i-u {baile suelto}\\
\glb 2\textsc{pl}-hear person-\textsc{pl} 1\textsc{sg}-play.flute-\textsc{subord}-\textsc{real} {name.of.song}\\
\glft ‘people, listen to me playing \textit{baile suelto} by flute!’
\endgl
\trailingcitation{[nxx-a630101g-2.002-003]}
\xe

%nabi not always, but pana always, % ana für 2 pl??
\is{imperative|)}


\subsection{Emphatic imperatives}\label{sec:MarkedImperatives}
\is{emphatic imperative|(}\is{inflection|(}

Imperatives can be formed by adding the suffix \textit{-ji} to a \isi{verb} inflected for \isi{irrealis} and second person, i.e. emphatic imperatives take person indexes\is{person marking} and irrealis RS just like the unmarked ones.

To start with, consider (\ref{ex:eimp-1}). The verb has the second person singular marker, it has irrealis RS and it takes the imperative marker \textit{-ji}. The example was elicited from Juana and represents a command to a dog to bite a thief.

\ea\label{ex:eimp-1}
\begingl
\glpreamble ¡pinijabakaji!\\
\gla pi-nijabaka-ji\\
\glb 2\textsc{sg}-bite.\textsc{irr}-\textsc{imp}\\
\glft ‘bite him!’
\endgl
\trailingcitation{[jxx-e191021e-2]}
\xe

Actually, it began to dawn on me relatively late that this was indeed an imperative marker. I had taken it for some deictic element before, and thus in elicitation sessions, I tried to find out about dimensions of space rather than characteristics that set the marked imperatives apart from the unmarked ones. Thus, I can only share some observations here that remain to be checked.

First of all, looking at the examples with this marker, it seems that they only include requests and commands. Offers or invitations and suggestions are absent. However, this may be a coincidence, as there are also more requests among the unmarked imperatives than there are offers or suggestions.

Second, in an elicitation session, the gestures Juana made when using the word forms with \textit{-ji} were bigger and more encompassing, which suggests to me that more emphasis is involved. This is why I speak of an emphatic imperative.

Third, in the same elicitation session, the form \textit{¡pupuna!} was consistently translated by her with Spanish \textit{¡trae!} ‘bring!’, and \textit{¡pupunaji!} with \textit{¡traélo!} ‘bring it!’. All but one of the examples with marked imperatives indeed have a third person \isi{object}. The one exception has a first person singular object index and \textit{-ji} is added after that one. Unmarked imperatives can also have third person objects (e.g. (\ref{ex:imp-2}) and (\ref{ex:imp-4}) in \sectref{sec:UnmarkedImperatives} above), so the difference may ultimately not depend on the presence or absence of a third person object in the imperative clause, but rather the translations with or without an \isi{object} is the way people express the same difference in Spanish. This remains to be proved.

Fourth, I have only found one example of an emphatic imperative with a second person plural subject, but I suppose this is connected to the fact that imperatives with plural subjects are in general rarer than the ones with singular subjects.

Some more examples follow. (\ref{ex:eimp-2}) stems from the story about the lazybones told by Miguel. The man has just climbed a tree and cut off his arm in order to throw it down to his son, pretending it was a raceme of \textit{cusi} palm fruit.

\ea\label{ex:eimp-2}
\begingl
\glpreamble “¡pijakupaji eka kÿsi!” tikechu chichechapÿi\\
\gla pi-jakupa-ji eka kÿsi ti-kechu chi-chechapÿi\\
\glb 2\textsc{sg}-receive.\textsc{irr}-\textsc{imp} \textsc{dem}a cusi 3i-say 3-son\\
\glft ‘“take the \textit{cusi} fruit!” he said to his son’
\endgl
\trailingcitation{[mox-n110920l.100]}
\xe

(\ref{ex:eimp-3}) comes from Juana telling me how her grandparents bought cows in Moxos. On their way back home they were caught by heavy rainfalls and had to cross an arroyo, which had filled with water. In this situation, her grandfather can hardly reach the ground and in order to guide his wife through the water he says:

\ea\label{ex:eimp-3}
\begingl
\glpreamble “¡pabikÿkaji nitijÿe naka!”\\
\gla pi-abikÿka-ji ni-tijÿe naka\\
\glb 2\textsc{sg}-grab.\textsc{irr}-\textsc{imp} 1\textsc{sg}-belt here\\
\glft ‘“hold on to my belt here!”'
\endgl
\trailingcitation{[jxx-p151016l-2.141]}
\xe

%\ea\label{ex:eimp-62}
%\begingl
%\glpreamble ¡pibikajikaji pimaretane!\\
%\gla pi-bikajika-ji pi-mareta-ne\\
%\glb 2\textsc{sg}-throw.\textsc{irr}-\textsc{imp} 2\textsc{sg}-suitcase-\textsc{possd}\\
%\glft ‘"throw your suitcase away!”‘\\
%\endgl
%\trailingcitation{[jxx-p151016l-2.104]}
%\xe

(\ref{ex:eimp-5}) is from the creation story told by Juana and is a citation of the snake. It is the forbidden apple that María Eva is supposed to take to her husband.

\ea\label{ex:eimp-4}
\begingl
\glpreamble “¡pumaji nauku tÿpi pima!”\\
\gla pi-uma-ji nauku tÿpi pi-ima\\
\glb 2\textsc{sg}-take.\textsc{irr} there \textsc{obl} 2\textsc{sg}-husband\\
\glft ‘“take it there for your husband!”’
\endgl
\trailingcitation{[jxx-n101013s-1.413]}
\xe

In (\ref{ex:eimp-5}), Miguel requests of Alejo that he ask the taxi driver, who was joining the recording session, about his place of origin.

\ea\label{ex:eimp-5}
\begingl
\glpreamble ¡piyÿsebÿkeaji juchubu chubiu, juchubu eka kapuniuchÿ!\\
\gla pi-yÿsebÿkea-ji juchubu chÿ-ubiu juchubu eka kapun-i-u-chÿ\\
\glb 2\textsc{sg}-ask.\textsc{irr}-\textsc{imp} where 3-house where \textsc{dem}a come-\textsc{subord}-\textsc{real}-3\\
\glft ‘ask him where he lives, where he comes from!’
\endgl
\trailingcitation{[mty-p110906l.211-212]}
\xe

(\ref{ex:eimp-6}) is the only example I have found of an emphatic imperative that does not have a third person object. It is a first person singular object in this case, which is indexed on the verb. The imperative marker follows the object index. The example comes from Miguel telling the story about the fox and the jaguar. Since the vulture let the fox escape, the jaguar wants to punish and eat him. The vulture seemingly accepts his fate and tells the jaguar to pluck him except for his wings and throw him up into the air:
% so that he would fall right into his open mouth. The jaguar obeys, but instead of falling down, the vulture flies away, not without defecating into the mouth of the jaguar before.

\ea\label{ex:eimp-6}
\begingl
\glpreamble “entonses ¡pibikÿkaneji anÿke!”\\
\gla entonses pi-bikÿka-ne-ji anÿke\\
\glb thus 2\textsc{sg}-throw-1\textsc{sg}-\textsc{imp} up\\
\glft ‘“then throw me up!”’
\endgl
\trailingcitation{[jmx-n120429ls-x5.195]}
\xe

Finally, I also came across one occurrence of an emphatic imperative with a second person plural subject, given as (\ref{ex:eimp-7}) here. It comes from a conversation between Juana and Miguel, where the latter told his sister what a certain person had said to the people of Santa Rita.

\ea\label{ex:eimp-7}
\begingl
\glpreamble “¡anaji echÿu senta!”\\
\gla e-ana-ji echÿu senta\\
\glb 2\textsc{pl}-make.\textsc{irr}-\textsc{imp} \textsc{dem}b path\\
\glft ‘“make the path!”’
\endgl
\trailingcitation{[jmx-c120429ls-x5.063]}
\xe
\is{inflection|)}\is{emphatic imperative|)}

\subsection{Suppletive imperatives}\label{sec:SuppletiveImperatives}
\is{suppletive imperative|(}
\is{motion predicate|(}

There are two suppletive imperatives, \textit{nabi} ‘go!’ and \textit{pana} ‘come!’, the former being much more frequent. Both words can combine with a \isi{verb} as well as with a demonstrative adverb, \textit{nabi} has also been found in combination with locative-marked nouns.

It is not entirely clear how the suppletive imperatives are composed. As for \textit{nabi}, there may be a root \textit{na} that takes the second person singular marker \textit{-bi},\is{person marking} which is used to index objects on verbs and subjects on non-verbal predicates.\is{non-verbal predication} \textit{Pana} is identical to the second person singular irrealis form of the verb \textit{-anau} ‘make’. Alternatively, it might be related to the prior \isi{associated motion} marker \textit{-punu} (realis) / \textit{-puna} (irrealis).\footnote{Regarding active verbs, \isi{Terena}, Old \isi{Baure} (i.e. the \isi{Baure} variety documented by the Jesuits) and marginally \isi{Mojeño Trinitario} inflect for irrealis by changing every vowel /o/ to /a/ \citep[103]{DanielsenTerhartSubm}. The marker \textit{-punu} may be related to \textit{pana} in a similar fashion if we assume that the same kind of RS-triggered vowel harmony once existed in Paunaka.}

I will first present some examples with \textit{nabi}. In (\ref{ex:nabi-1}), it first stands alone in the first clause and is then combined with an adverb plus locative-marked noun in the second clause to indicate the goal of the motion action that is demanded here. The examples comes from Juana’s narration about her grandparents’ journey and is a citation of the water spirit talking with her grandfather at night, trying to lure him away from his wife.

\ea\label{ex:nabi-1}
\begingl
\glpreamble “¡nabi! ¡nabi nauku nubiuyae!” chikechuchÿji\\
\gla nabi nabi nauku nÿ-ubiu-yae chi-kechu-chÿ-ji\\
\glb go.\textsc{imp} go.\textsc{imp} there 1\textsc{sg}-house-\textsc{loc} 3-say-3-\textsc{rprt}\\
\glft ‘“go! go to my house there!” she said to him, it is said’
\endgl
\trailingcitation{[jxx-p151016l-2.195]}
\xe

(\ref{ex:nabi-2}) was elicited from María C. It is something one could say to a dog to chase it off.

\ea\label{ex:nabi-2}
\begingl
\glpreamble ¡nabi nekupaiyae!\\
\gla nabi nekupai-yae\\
\glb go.\textsc{imp} outside-\textsc{loc}\\
\glft ‘go out!’
\endgl
\trailingcitation{[uxx-e120427l.078]}
\xe

In (\ref{ex:nabi-3}), \textit{nabi} combines with a verb. It is a repetition of (\ref{ex:imp-4}) above, with which Juana got back to the storyline after summarising in Spanish a part of the creation story. Note that while she used a second person singular irrealis form of the verb \textit{-yunu} in the example above, she replaces it by the suppletive form \textit{nabi} here.

\ea\label{ex:nabi-3}
\begingl
\glpreamble “¡nabi pinika chÿi yÿkÿke!”\\
\gla nabi pi-nika chÿi yÿkÿke\\
\glb go.\textsc{imp} 2\textsc{sg}-eat.\textsc{irr} fruit tree\\
\glft ‘“go and eat the fruit of the trees!”’
\endgl
\trailingcitation{[jxx-n101013s-1.885]}
\xe

(\ref{ex:nabi-8}) also comes from Juana. She was telling Swintha about a very smart dog she once had. When she wanted to slaughter a chicken, she could point to one chicken and tell the dog to catch it. This is what she said to the dog:

\ea\label{ex:nabi-8}
\begingl
\glpreamble bikupaika takÿra ¡nabi peikukuika takÿra!\\
\gla bi-kupaika takÿra nabi pi-eikukuika takÿra\\
\glb 1\textsc{pl}-slaughter.\textsc{irr} chicken go.\textsc{imp} 2\textsc{sg}-chase.\textsc{irr} chicken\\
\glft ‘we are going to slaughter a chicken, go and chase the chicken!’
\endgl
\trailingcitation{[jxx-e191021e-2]}
\xe


For plural addressees,\is{addressee of imperative} Juana used \textit{nabue} a few times, which includes the second plural index \textit{-e} instead of singular \textit{-bi}. However, this has not been found with other speakers. Indeed, Juan C. once corrected himself with a verb with second person plural index, when he wanted to form an imperative with second person plural reference, see (\ref{ex:nabi-7}). Juana’s use of \textit{nabue} is exemplified in (\ref{ex:nabi-6}) below, which comes from elicitation.

\ea\label{ex:nabi-7}
\begingl
\glpreamble ¡nabi! ¡eyuna!\\
\gla nabi e-yuna\\
\glb go.\textsc{imp} 2\textsc{pl-go.\textsc{irr}}\\
\glft ‘go (\textsc{sg})! go (\textsc{pl})!’
\endgl
\trailingcitation{[mqx-p110826l.031]}
\xe

\ea\label{ex:nabi-6}
\begingl
\glpreamble nabue emusuika\\
\gla nabu-e e-musuika\\
\glb go.\textsc{imp}-2\textsc{pl} 2\textsc{pl}-wash.\textsc{irr}\\
\glft ‘go and wash!’
\endgl
\trailingcitation{[jxx-e081025s-1.535]}
\xe

The suppletive imperative \textit{pana} has only been found with singular addressees\is{addressee of imperative} in the corpus. It mostly combines with the adverb \textit{naka} ‘here’. One such case is (\ref{ex:pana-1}), where Juana told me how her sister María S. and her husband once had an encounter with a snake or water spirit in the reservoir of Santa Rita. This is what the husband exclaimed, when he noticed the snake:

\ea\label{ex:pana-1}
\begingl
\glpreamble “¡pana naka! ¡kechue echÿu!”\\
\gla pana naka kechue echÿu\\
\glb come.\textsc{imp} here snake \textsc{dem}b\\
\glft ‘“come here! that’s a snake!”’
\endgl
\trailingcitation{[jxx-p120515l-2.164]}
\xe

(\ref{ex:pana-2}) comes from Juana telling the creation story. After having fashioned her from mud, God requests María Eva to approach him in order to wed her to Jesus.

\ea\label{ex:pana-2}
\begingl
\glpreamble “Maria Eva, ¡pana naka!” chikechuchiji\\
\gla {Maria Eva} pana naka chi-kechu-chi-ji\\
\glb {María Eva} come.\textsc{imp} here 3-say-3-\textsc{rprt}\\
\glft ‘“María Eva, come here!” he said to her, it is said’
\endgl
\trailingcitation{[jxx-n101013s-1.364]}
\xe

In (\ref{ex:pana-3}), a verb follows the adverb. This example was elicited from Miguel.

\ea\label{ex:pana-3}
\begingl
\glpreamble ¡pana naka pitibua!\\
\gla pana naka pi-tibua\\
\glb come here 2\textsc{sg}-sit.down.\textsc{irr}\\
\glft ‘come here and sit down!’
\endgl
\trailingcitation{[mxx-e160811sd.221]}
\xe


If a verb follows \textit{nabi} or \textit{pana}, it is not unusual that this verb takes the prior motion\is{associated motion} marker. (\ref{ex:nabi-5}) and (\ref{ex:nabi-4}) exemplify this for \textit{nabi}, and (\ref{ex:pana-5}) and (\ref{ex:pana-4}) for \textit{pana}.

(\ref{ex:nabi-5}) comes from María C. who had told that she medicated herself with the bark of a tree. She knew about the use of the bark, because it was as if God had told her:

\ea\label{ex:nabi-5}
\begingl
\glpreamble ¡nabi parejipuna echÿu pichai!\\
\gla nabi pi-areji-puna echÿu pichai \\
\glb go.\textsc{imp} 2\textsc{sg}-rasp-\textsc{am.prior.irr} \textsc{dem}b medicine\\
\glft ‘go and rasp the medicine’
\endgl
\trailingcitation{[ump-p110815sf.371]}
\xe

(\ref{ex:nabi-4}) was elicited from María S., when a pig of hers was grunting very loudly, disturbing the recording we made. Note that \textit{-sabaiku} ‘grunt’ is a stative verb,\footnote{It is actually rare that stative verb combines the with prior motion marker, but this seems one of the cases, in which a morphologically stative verb is semantically active. Note that the related continuous verb form \textit{-sabaipaiku} ‘grunt’ is active.} so it takes an \isi{irrealis} prefix, and consequently, the associated motion marker occurs in its default/\isi{realis} form here.

\ea\label{ex:nabi-4}
\begingl
\glpreamble ¡nabi pasabaipunu max nauku!\\
\gla nabi pi-a-sabai-punu max nauku\\
\glb go.\textsc{imp} 2\textsc{sg}-\textsc{irr}-grunt-\textsc{am.prior} more there\\
\glft ‘go to grunt over there!’
\endgl
\trailingcitation{[rmx-e150922l.159]}
\xe

(\ref{ex:pana-5}) was elicited from Miguel. The verb follows the adverb \textit{naka} in this case.

\ea\label{ex:pana-5}
\begingl
\glpreamble ¡pana naka pimukupuna!\\
\gla pana naka pi-muku-puna\\
\glb come.\textsc{imp} here 2\textsc{sg}-sleep-\textsc{am.prior.irr}\\
\glft ‘come here to sleep!’
\endgl
\trailingcitation{[mxx-e160811sd.232]}
\xe

In (\ref{ex:pana-4}), \textit{pana} takes the \isi{prospective} marker \textit{-bÿti}. This example stems from the recordings by Riester, and I have found neither \textit{pana} nor \textit{nabi} taking any TAME markers in the recordings made from 2008 on. The example is one of the sentences Juan Ch. produces as a beginning of an imagined conversation with a visitor.

\ea\label{ex:pana-4}
\begingl
\glpreamble ¡panabÿti pitibupuna!\\
\gla pana-bÿti pi-tibu-puna\\
\glb come.\textsc{imp}-\textsc{prsp} 2\textsc{sg}-sit.down-\textsc{am.prior.irr}\\
\glft ‘come and sit down!’
\endgl
\trailingcitation{[nxx-p630101g-2.07]}
\xe

While the prior motion\is{associated motion} marker has been found with verbs accompanying both suppletive imperatives, the dislocative marker\is{dislocative|(} only occurs with verbs combining with \textit{nabi}. This is in accordance with their semantics. While \textit{-punu} is not specified for direction towards or away from a place, the dislocative marker has only been found in expressions of translocative motion (see \sectref{sec:punu} and \sectref{sec:PA}). If \textit{nabi} is combined with a verb taking the dislocative marker, this can be analysed as a case of \isi{motion-cum-purpose construction} (see \sectref{sec:MotionCumPurpose}). Two examples follow. Both were elicited from Juana.


\ea\label{ex:nabi-9}
\begingl
\glpreamble ¡nabi piyÿseikupa kanela! kuina kakuina\\
\gla nabi pi-yÿseiku-pa kanela kuina kaku-ina\\
\glb go.\textsc{imp} 2\textsc{sg}-buy-\textsc{dloc.irr} cinnamon \textsc{neg} exist-\textsc{irr.nv}\\
\glft ‘go and buy cinnamon! There isn’t any’
\endgl
\trailingcitation{[jxx-e190210s-01]}
\xe

\ea\label{ex:nabi-11}
\begingl
\glpreamble ¡nabue emusuikupa!\\
\gla nabu-e e-musuiku-pa\\
\glb go.\textsc{imp}-2\textsc{pl} 2\textsc{pl}-wash-\textsc{dloc.irr}\\
\glft ‘go and wash!’
\endgl
\trailingcitation{[jxx-e190210s-01]}
\xe
\is{dislocative|)}

%\ea\label{ex:nabi-10}
%\begingl
%\glpreamble “¡nabi epuikupa! temetapujiyu kÿpu”\\
%\gla nabi e-puiku-pa teme-tapu-ji-yu kÿpu\\
%\glb go.\textsc{imp} 2\textsc{pl}-fish-\textsc{dloc.irr} big-\textsc{shell}-\textsc{col}-\textsc{ints} sardine\\
%\glft ‘“go fishing, the sardines are very big!”’\\
%\endgl
%\trailingcitation{[jxx-e150925l-1.160]}
%\xe

\is{motion predicate|)}
\is{suppletive imperative|)}

\subsection{Negative imperatives}\label{sec:Prohibitives}
\is{directive speech act!prohibitive|(}\is{negation!prohibitive|(}

There are several ways to form a negative imperative. First of all, the negative particle\is{negative particle|(} \textit{kuina} can be used together with an irrealis verb. In this case, the negative imperative is identical to a negative declarative clause in structure. 

Second, it is also possible to use the specific negative particles \textit{naka} or \textit{masaini}. The first of them is used to form prohibitives, i.e. commands and requests not to do something. It is possibly related to the negative particle in \isi{Baure}, which is \textit{noka} \citep[cf.][338]{Danielsen2007}. As for \textit{masaini}, this is composed of the apprehensional \isi{connective} \textit{masa} (see \sectref{sec:Conjunctions}) and the \isi{frustrative} marker \textit{-ini} (see \sectref{sec:Frustrative}). María S. seems to use it in the same fashion as \textit{naka}, i.e. in prohibitives, but data from other speakers suggests that it is rather an \isi{admonitive} particle, i.e. it appears in warnings.\is{negative particle|)}

(\ref{ex:nimp-1}) to (\ref{ex:nimp-3}) are examples of negative imperatives with \textit{kuina}. As can be seen in (\ref{ex:nimp-1}), the negative particle precedes the irrealis verb and a declarative sentence would have exactly the same structure.\is{word order} This example was elicited from María S.

\ea\label{ex:nimp-1}
\begingl
\glpreamble ¡patÿkemiu nijinepÿi! ¡kuina piyuabu!\\
\gla pi-a-tÿkemiu ni-jinepÿi kuina pi-iyua-bu\\
\glb 2\textsc{sg}-be.quiet 1\textsc{sg}-daughter \textsc{neg} 2\textsc{sg}-cry.\textsc{irr}-\textsc{dsc}\\
\glft ‘be quiet, my daughter, don’t cry anymore!
\endgl
\trailingcitation{[mrx-e150219s.136]}
\xe

(\ref{ex:nimp-2}) comes from María C. Actually, I am not entirely sure whether this is really meant to be a negative imperative or rather a sentence with future reference (‘you won’t die!’). In any case, María C. tells what she said to her mother, when the latter was poisoned by a sorcerer.

\ea\label{ex:nimp-2}
\begingl
\glpreamble ¡kuina pipaka!\\
\gla kuina pi-paka\\
\glb \textsc{neg} 2\textsc{sg}-die.\textsc{irr}\\
\glft ‘don’t die!’
\endgl
\trailingcitation{[ump-p110815sf.465]}
\xe

The case is clearer in (\ref{ex:nimp-3}). This sentence was produced by Miguel and comes from the story about the fox and the jaguar. This is what the vulture says to the jaguar, when he is supposed to be punished for having let the fox escape. The vulture starts his utterance in Spanish (\textit{no} being the Spanish negative particle), but continues in Paunaka.

\ea\label{ex:nimp-3}
\begingl
\glpreamble entonses echÿu sÿmÿ tikechu: “no no no no, ¡kuina pinikanÿ!”\\
\gla entonses echÿu sÿmÿ ti-kechu {no no no no} kuina pi-nika-nÿ\\
\glb thus \textsc{dem}a vulture 3i-say {no no no no} \textsc{neg} 2\textsc{sg}-eat.\textsc{irr}-1\textsc{sg}\\
\glft ‘so the vulture said: “no, no, no, no, don’t eat me!”’
\endgl
\trailingcitation{[jmx-n120429ls-x5.180]}
\xe

In prohibitives, speakers can make use of \textit{naka}.\is{reality status|(} This negative particle is delimited to imperative contexts (commands, requests), so that no ambiguity arises. All examples of prohibitives with \textit{naka} were elicited. They can contain a \isi{realis} verb. Thus it seems that the fact that two parameters trigger \isi{irrealis} here, negation and imperative, is encoded by using the RS marking\is{reality status} that matches none of them.\is{doubly irrealis construction} 

When studying negation among \isi{Arawakan languages}, \citet[]{Michael2014b} %264
identified five types of possible prohibitive constructions. They are given in \tabref{table:ProhibitiveTypes}.
The distinguishing factors include how the negative expression differs from the one found in standard negation (column “expression of negation”) and how the rest of the prohibitive sentence differs from a positive imperative (column “prohibitive construction”).

\begin{table}[htbp]
\caption{Prohibitive construction types by \citet[270]{Michael2014b}}%264

\begin{tabular}{lll}
\lsptoprule
Prohibitive type & Prohibitive construction & Expression of negation \cr
\midrule
Type I & same as imperative & same as standard negation\cr
Type II & same as imperative & different from standard negation\cr
Type III & different from imperative & same as standard negation\cr
Type IV & different from imperative & different from standard negation\cr
Type V & \multicolumn{2}{c}{no distinct prohibitive construction} \cr
\lspbottomrule
 \end{tabular}

\label{table:ProhibitiveTypes}
\end{table}
 
According to this classification, the Paunaka prohibitive with \textit{naka} belongs to type IV: the negative particle is different from the one used in standard negation and the rest of the construction is different from the imperative. In his sample of 23 \isi{Arawakan languages}, only Kinikinau and Nanti share this specific behaviour with Paunaka \citep[271]{Michael2014b}.
\footnote{In Nanti, this construction is not delimited to prohibitives, but occurs in any context in which negation and another parameter trigger irrealis marking \citep[272--273]{Michael2014}.}

Consider (\ref{ex:prohib-1}), which comes from Juana. The prohibitive particle \textit{naka} is followed by a realis verb.

\ea\label{ex:prohib-1}
\begingl
\glpreamble ¡naka piyu!\\
\gla naka pi-iyu\\
\glb \textsc{prohib} 2\textsc{sg}-cry\\
\glft ‘don’t cry!’
\endgl
\trailingcitation{[jxx-e120430l-3a]}
\xe

In elicitation, Juana also produced some prohibitives with an irrealis verb like the one in (\ref{ex:prohib-2}); however, the seemingly more spontaneous uses (e.g. the first translation she gave in elicitation) all included realis predicates.

\ea\label{ex:prohib-2}
\begingl
\glpreamble ¡naka piyua!\\
\gla naka pi-iyua\\
\glb \textsc{prohib} 2\textsc{sg}-cry.\textsc{irr}\\
\glft ‘don’t cry!’
\endgl
\trailingcitation{[jxx-p150920l.041]}
\xe

Although María S. rather uses \textit{masaini} as a negative particle in prohibitives (see below), she confirms the use of \textit{naka}. (\ref{ex:prohib-3}) is an example of a prohibitive with \textit{naka} elicited from her.

\ea\label{ex:prohib-3}
\begingl
\glpreamble ¡naka pekubu!\\
\gla naka pi-ekubu\\
\glb \textsc{prohib} 2\textsc{sg}-laugh\\
\glft ‘don’t laugh!’
\endgl
\trailingcitation{[rxx-e150220s-1.08]}
\xe

(\ref{ex:prohib-4}) comes from Juana again.

\ea\label{ex:prohib-4}
\begingl
\glpreamble ¡naka pikupaiku ÿne! tisÿeimuyu\\
\gla naka pi-kupaiku ÿne ti-sÿei-umu-yu\\
\glb \textsc{prohib} 2\textsc{sg}-step.on water 3i-be.cold-\textsc{clf:}liquid-\textsc{ints}\\
\glft ‘don’t step in the water, it is very cold!’
\endgl
\trailingcitation{[jxx-e150925l-1.083-084]}
\xe

It is less clear which RS is required in clauses with \textit{masaini}.\is{admonitive|(}  All examples that follow were elicited, except for the last one, (\ref{ex:adm-7}), and they vary with regard to RS. 

María S. prefers to form prohibitives with \textit{masaini}, and this use was verified by Juana as a valid alternative to those with \textit{naka}. However, all of the examples produced by María S. can be read as warnings, the particle was translated with Spanish \textit{cuidado} ‘caution!, be careful!, watch out!’  by her and Miguel (in the session mrx-e150219s), and it is also warnings that Juana and Miguel use this particle for in (\ref{ex:adm-6}) and (\ref{ex:adm-7}). For this reason it is analysed as an admonitive particle here.

To start with, consider (\ref{ex:adm-2}) from María S. The admonitive particle precedes the realis verb here.

\ea\label{ex:adm-2}
\begingl
\glpreamble ¡masaini pijikupu!\\
\gla masaini pi-jikupu\\
\glb \textsc{adm} 2\textsc{sg}-swallow\\
\glft ‘don’t swallow it!’\\%or:‘be careful not to swallow it’??
\endgl
\trailingcitation{[rxx-e141230s.076]}
\xe

(\ref{ex:adm-2}) also has a realis verb. This is a warning directed to a child not to step on the table lest it topples over. The warning was originally produced in Spanish by María S. and translated to Paunaka by request of Swintha.

\ea\label{ex:adm-1}
\begingl 
\glpreamble ¡masaini pikupachu naka!\\
\gla masaini pi-kupachu naka\\ 
\glb \textsc{adm} 2\textsc{sg}-step.on here\\ 
\glft ‘don’t step on it here!’\\ 
\endgl
\trailingcitation{[mrx-e150219s.150]}
\xe

(\ref{ex:adm-3}) is another warning with a realis verb elicited from María S. to tell Swintha that she should not eat a corncob half-raw.

\ea\label{ex:adm-3}
\begingl
\glpreamble ¡masaini piniku enui! painuepÿi\\
\gla masaini pi-niku enui pi-a-inuepÿi\\
\glb \textsc{adm} 2\textsc{sg}-eat green 2\textsc{sg}-\textsc{irr}-have.wind\\
\glft ‘don’t eat it raw! You will have wind’
\endgl
\trailingcitation{[rxx-e150220s-1.25]}
\xe

In (\ref{ex:adm-4}), which is very similar to the previous example, María S. opted for an irrealis verb.

\ea\label{ex:adm-4}
\begingl
\glpreamble ¡masaini pinika! kuinakuÿ tayu\\
\gla masaini pi-nika kuina-kuÿ ti-a-yu\\
\glb \textsc{adm} 2\textsc{sg}-eat.\textsc{irr} \textsc{neg}-\textsc{incmp} 3i-\textsc{irr}-be.ripe\\
\glft ‘don’t eat it! It is not ripe yet’
\endgl
\trailingcitation{[rxx-e181022le]}
\xe

She also uses an irrealis verb in (\ref{ex:adm-5}). This warning was first uttered in Spanish and then translated. It was directed to a child (as (\ref{ex:adm-1}) above, which comes from the same session).

\ea\label{ex:adm-5}
\begingl
\glpreamble ¡masaini pakupuru!\\
\gla masaini pi-a-kupuru\\
\glb \textsc{adm} 2\textsc{sg}-\textsc{irr}-burn\\
\glft ‘don’t burn yourself!’\\or: ‘be careful, you will burn yourself!’
\endgl
\trailingcitation{[mrx-e150219s.147]}
\xe

In the very same elicitation session, Miguel used \textit{masaini} together with a verb with third person subject and irrealis RS.

\ea\label{ex:adm-6}
\begingl
\glpreamble ¡masaini, tinijabakapi!\\
\gla masaini tinijabakapi\\
\glb \textsc{adm} 3i-bite.\textsc{irr}-2\textsc{sg}\\
\glft ‘be careful, it may bite you!’
\endgl
\trailingcitation{[mrx-e150219s.148]}
\xe

When he repeated the sentence, the verb had realis RS, see (\ref{ex:adm-8}).

\ea\label{ex:adm-8}
\begingl
\glpreamble ¡masaini tinijabakubi kabe!\\
\gla masaini ti-nijabaku-bi kabe\\
\glb \textsc{adm} 3i-bite-2\textsc{sg} dog\\
\glft ‘be careful, the dog may bite you!’
\endgl
\trailingcitation{[mrx-e150219s.149]}
\xe
\is{reality status|)}

Finally, the only non-elicited example with \textit{masaini} is (\ref{ex:adm-7}) and comes from Juana. The particle seems to constitute a whole clause here, the following one starting with the Spanish conditional conjunction \textit{si} ‘if’. The example comes from the creation story and comprises the warning of God that Jesus and María Eva should not go into the garden and eat the apple.\footnote{I do not know what the function of \textit{ta} is. It shows up infrequently. When asked, Miguel once told me that \textit{ta kue} was simply a longer variant of \textit{kue} ‘if, when’; however, occurrence of \textit{ta} is not restricted to combinations with \textit{kue} as becomes apparent from this example.}

\ea\label{ex:adm-7}
\begingl
\glpreamble “ta masaini si eyuna uertiyayae kaku nauku mansana kaku ucheti”\\
\gla ta masaini si e-yuna uerta-yae kaku nauku mansana kaku ucheti\\
\glb ? \textsc{adm} if 2\textsc{pl}-go.\textsc{irr} garden-\textsc{loc} exist there apple exist chili\\
\glft ‘“be careful if you go into the garden, there are apples, there is chili”’
\endgl
\trailingcitation{[jxx-n101013s-1.371-373]}
\xe

\is{admonitive|)} 
\is{negation!prohibitive|)}\is{directive speech act!prohibitive|)}


\subsection{Hortatives}\label{sec:Hortatives}
\is{hortative|(}

Hortatives direct a command, request or invitation to a first person plural, i.e. they include the speaker. Hortatives are formed with the particle \textit{jaje}. This particle usually implies some motion\is{motion predicate} and can be translated as ‘let’s go!’. The case being like this, it can occur on its own (\ref{ex:hort-1}) or it can combine with adverbs (\ref{ex:hort-2}) or verbs (\ref{ex:hort-3}). Interestingly, Miguel also combines it with motion verbs (\ref{ex:hort-4}). 

%Just like imperatives, they can be completely unmarked. In this case, they take a first person plural subject index and irrealis RS.
%multiplicaubina, mxx-p181027l-1.150

(\ref{ex:hort-1}) comes from Miguel’s story about the cowherd and the spirit. The spirit has just told the man, who desperately searches for his cows, that he has taken them. He offers the man to go and have a look at them:

\ea\label{ex:hort-1}
\begingl
\glpreamble “¡jaje!” chikechuchÿji \\
\gla jaje chi-kechu-chÿ-ji\\
\glb \textsc{hort} 3-say-3-\textsc{rprt}\\
\glft ‘“let’s go!” he said to him, it is said
\endgl
\trailingcitation{[mxx-n151017l-1.36]}
\xe

(\ref{ex:hort-2}) is what Juana’s grandmother said to her husband when she realised that there was a water spirit in the arroyo they tried to cross with their cows on their way back from Moxos.

\ea\label{ex:hort-2}
\begingl
\glpreamble “¡jaje nauku anÿke!”\\
\gla jaje nauku anÿke\\
\glb \textsc{hort} there up\\
\glft ‘“let’s go up there!”’
\endgl
\trailingcitation{[jxx-p151016l-2.102]}
\xe

(\ref{ex:hort-3}) also comes from Juana, this example is from the creation story and a citation of God talking to María Eva after she has eaten the apple.

\ea\label{ex:hort-3}
\begingl
\glpreamble “¡pana naka! ¡jaje bana jiriensu tÿpi pimÿuna!”\\
\gla pana naka jaje bi-ana jiriensu tÿpi pi-mÿu-ina\\
\glb come.\textsc{imp} here \textsc{hort} 1\textsc{pl}-make.\textsc{irr} linen \textsc{obl} 2\textsc{sg}-clothes-\textsc{irr.nv}\\
\glft ‘“come here! Let’s go and make linen for your future clothes!”’
\endgl
\trailingcitation{[jxx-n101013s-1.503]}
\xe

In telling about his days in school, Miguel used (\ref{ex:hort-4}) to tell me what other children said to him, when they invited him to join school.

\ea\label{ex:hort-4}
\begingl
\glpreamble “¡jaje biyuna xhikuerayae!”\\
\gla jaje bi-yuna xhikuera-yae\\
\glb \textsc{hort} 1\textsc{pl}-go.\textsc{irr} school-\textsc{loc}\\
\glft ‘“let’s go to school!”’
\endgl
\trailingcitation{[mxx-p181027l-1.006]}
\xe

The hortative particle can take some morphology. It has been found with the \isi{iamitive}, the additive\is{additive} marker and the \isi{emphatic} marker \textit{-ja}. One example with the iamitive is given below. It was produced by Juana in order to teach me this expression. 

\ea\label{ex:hort-5}
\begingl
\glpreamble ¡jajetu! ¡tosetu!\\
\gla jaje-tu tose-tu\\
\glb \textsc{hort}-\textsc{iam} noon-\textsc{iam}\\
\glft ‘let’s go now! It is already noon!’
\endgl
\trailingcitation{[jxx-e110923l-1.084]}
\xe

%The hortative particle can combine with a verb that takes the dislocative marker. In that case, we are dealing with a motion-cum-purpose construction (see \sectref{sec:MotionCumPurpose}). This is the case in (\ref{ex:hort-6}), which comes from Juana and was produced on request by Swintha, because she had been telling her about freshwater snails in Spanish before.
%
%\ea\label{ex:hort-6}
%\begingl 
%\glpreamble ¡jaje binebÿkupa keyu binika!\\
%\gla jaje bi-nebÿku-pa keyu bi-nika\\ 
%\glb \textsc{hort} 1\textsc{pl}-collect-\textsc{dloc.irr} snail 1\textsc{pl}-eat.\textsc{irr}\\ 
%\glft ‘let's go to collect freshwater snails to eat’\\ 
%\endgl
%\trailingcitation{[jxx-e081025s-1.171]}
%\xe
%das einzige Beispiel so, das steht auch schon im AM-Kapitel
\is{hortative|)}
\is{directive speech act|)}

This was the last example in this section. The chapter on simple clauses is almost completed by now. The only sentence type missing being interrogative clauses. They are described in detail in the following section.
