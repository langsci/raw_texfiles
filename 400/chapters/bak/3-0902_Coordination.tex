%!TEX root = 3-P_Masterdokument.tex
%!TEX encoding = UTF-8 Unicode

\section{Coordinated clauses}\label{sec:Coordination}\is{coordination|(}

Coordination has been defined by \citet[34]{Haspelmath2004} as follows: “The term \textit{coordination} refers to syntactic constructions in which two or more units of the same type are combined into a larger unit and still have the same semantic relations with other surrounding elements”. As for the term “syntactic construction”, I consider only those clauses as coordinated that occur within the same intonation unit \citep[332]{Mithun1988}. Since I have not undertaken a full-fledged analysis of intonation patterns, one \isi{intonation} unit is defined mainly on the basis of pauses, i.e. in one intonation unit there is no pause between the two clauses. In addition, if intonation is not falling at the end of a clause, this is considered a further sign of connectedness to a second one. In contrast to subordination, all clauses in coordination are asserted.

Coordination of clauses is signalled by either asyndetic \isi{juxtaposition} or syndetic \isi{juxtaposition} with a \isi{connective} (see \sectref{sec:AsyndeticJuxtaposition} and \sectref{sec:SyndeticJuxtaposition} for a general discussion of the construction types). The structure of asyndetically and syndetically coordinated clauses is given in \figref{fig:CoordinationStructure}.\is{syndesis/asyndesis}

\begin{figure}[!ht]


[[MC] [MC]]

[[MC] co [MC]]
\caption{Sentence structure of juxtaposed coordinated clauses}
\label{fig:CoordinationStructure}

\end{figure}

As for asyndetic \isi{juxtaposition}, this is found mostly for \isi{conjunctive} and \isi{sequential} coordination. In syndetic \isi{juxtaposition}, connectives\is{connective|(} specify the kind of connection between the two clauses.\is{syndesis/asyndesis} The connectives used in coordination are given in \tabref{table:ConnectivesCoordination}. Apart from \textit{te} ‘then’ and \textit{nechikue} ‘therefore’, all of them have been borrowed\is{borrowing} from Spanish, which is not surprising considering the high degree of borrowability of connectives in general \citep[cf.][194]{Matras2009}.  All coordinating connectives not only combine clauses uttered within one intonation unit,\is{intonation|(} but also occur at the beginning of clauses that form their own intonation unit. The latter is analysed as discourse connection, not clause combining, i.e. the connective is used to link a piece of information to an entire episode, a feature which may just be typical for spoken mediality \citep[cf.][]{Chafe1988}.\is{connective|)} I will confine the discussion to coordination of clauses here, i.e. those uttered within one intonation unit, although the distinction between clause connection and discourse connection is certainly a gradual one.\is{intonation|)}\footnote{For a preliminary analysis of discourse connection see \citet[138--142]{DanielsenTerhart2015}.} Verbs in coordinated clauses are never deranked or marked for dependency -- at least not for the sake of signalling linking. RS\is{reality status|(} of both clauses is usually independent from each other with one exception: if the first clause encodes \isi{future reference} by having irrealis RS, a coordinated clause that expresses consequence\is{consecutive} or temporal succession\is{sequential} cannot have realis RS. This does not exclude the combination of irrealis and realis clauses for other reasons.\is{reality status|)}

\begin{table}
\caption{Connectives in coordinated clauses}

\begin{tabular}{lll}
\lsptoprule
Connective & Translation & Clause type \cr
\midrule
\textit{entonses} & thus & consecutive coordination\cr
\textit{i} & and & conjunction\cr
\textit{nechikue} & therefore & consecutive coordination\cr
\textit{o} & or & disjunction\cr
\textit{pero} & but & adversative coordination\cr
\textit{te} & ‘\textsc{seq}’ (‘then’) & sequential coordination\cr
\lspbottomrule
\end{tabular}

\label{table:ConnectivesCoordination}
\end{table}

The remainder of this section is organised as follows. \sectref{sec:AsyndeticCoordination} deals with asyndetic coordination, and the other sections deal with coordinated clauses that are linked by a connective: \sectref{sec:SequentialCoordination} is about sequential coordination including the connective \textit{te} ‘then’, \sectref{sec:ConjunctiveCoordination} deals with conjunctive coordination with the connective \textit{i} ‘and’, in \sectref{sec:DisjunctiveCoordination} disjunctive coordination with the connective \textit{o} ‘or’ is described, \sectref{sec:AdversativeCoordination} is about adversative coordination with \textit{pero} ‘but’ and finally, \sectref{sec:ConsecutiveCoordination} deals with consecutive coordination, a minor type of clause combining, with the connectives \textit{nechikue} ‘therefore’ and \textit{entonses} ‘thus’.

\subsection{Asyndetic coordination}\label{sec:AsyndeticCoordination}
\is{juxtaposition|(}

%tiberiukupunu tiyunupunu = she turned around and went home, mox-n110920l.077
%parikiyu chamayuchÿ aa año año trabakuyunubechi kuina kuina chisiupuchanube, mxx-p110825l.047

%nejujumi kuina nimuka yuti, jxx-p120515l-1.188
%tibÿkupunube kuina takÿraina = entraron y no había gallina, jmx-n120429ls-x5.326
%kuina nenaina echÿu gansojane tirÿrÿ chisikuji, jrx-c151001lsf-11.040


%kuina nÿnapÿka nÿbanenepuna = no vengo primero, vengo atrás, rxx-e141230s.167
%kuina tipajÿka naka kapunuina tukiu pasauna = no se queda aquí, va a venir de paseo (Federico, who lives in Conce), jxx-p110923l-1.122
%"aa nÿchÿnumi aa pensaikune kuina nitupa echÿu bakajane tijekupupuikutu", tikechu, mxx-n151017l-1.32

%nikutiutu nisuiu, kuina auantauneinabu nisua, cux-c120414ls-2.035-036

%niyunu nÿsupu kuina abansaunÿina tupirujiyu tipÿneji, rxx-e181022le
%nechikue kuina tikechunubebu kuina bichujijikabu kuina beteumichabu chimutu = y ya no decimos nada ya no le hablamos ya no le avisamos porque ya lo ha visto, nxx-p630101g-1.080
%bepaka tanÿma kuina taechunanube bijinepuinube ... bichechajinube? = nosotros nos murimos y ya no aprenden nuestro shijos y hijas, jxx-x110916.08
%kui nepakatu kuina taichunanube nimijÿnajinube niyÿsebÿkeunube kuina tisacha, jxx-x110916.48


Clauses can be coordinated by simply juxtaposing them. They occur within one \isi{intonation} unit, but there is no morphosyntactic signal of connection. These clauses are asyndetically coordinated,\footnote{Asyndetic juxtaposition is also found in coordination of NPs,\is{noun phrase} consider (\ref{ex:motherfather}):

\ea\label{ex:motherfather}
\begingl
\glpreamble nÿenu nÿa tebuku amuke\\
\gla nÿ-enu nÿ-a ti-ebuku amuke\\
\glb 1\textsc{sg}-mother 1\textsc{sg}-father 3i-sow corn\\
\glft ‘my mother and my father sowed corn’
\endgl
\trailingcitation{[rxx-p181101l-2.200]}
\xe}
as (\ref{ex:new23-asycor}), in which Juana tells me why her brother died.

\ea\label{ex:new23-asycor}
\begingl
\glpreamble ... kuina puero tinika kuina puero tea\\
\gla kuina puero ti-nika kuina puero ti-ea\\
\glb \textsc{neg} can 3i-eat.\textsc{irr} \textsc{neg} can 3i-drink.\textsc{irr}\\
\glft ‘he couldn’t eat and drink’
\endgl
\trailingcitation{[jxx-p120430l-2.370]}
\xe

According to \citet[335]{Mithun1988}, in asyndetic coordination clauses “conjoined with no intonation break typically describe subparts of what is conceived of as a single event. One clause typically sets the stage for the other by positioning a major participant. [...] By contrast, clauses separated by comma intonation typically represent conceptually distinct aspects of an action, event, or scene. The conjoined clauses most often describe sequential actions”. Alternatively, the latter can also “describe simultaneous aspects of a scene or state” \citep[336]{Mithun1988}. As for the notion of “single event”, this is hard to prove, because whether something is perceived as a single event or as multiple connected events is connected to cognitive processes, which I do not have access to.\footnote{\citet[]{Defina2016} describes an interesting approach, considering gestures co-occurring with serial verbs as indication of whether a sequence of verbs is considered as one or several events. This approach requires excellent video recordings, something that could be worth collecting in the future as long as there are still speakers available.} Or, as \citet[306]{Haspelmath2016} states, “there is no objective way of identifying a single event and distinguishing it from a set of several events”. For the time being, I do not make any distinctions between single and multiple events.

What we do find in coordination of clauses in Paunaka is that one clause sets the stage for another one by mentioning a participant. Both of these clauses typically have the same RS.\is{reality status}\footnote{I do not follow \citet[]{Mithun1988} in making a distinction between coordination of predicates and clauses, since a single predicate can be a full clause in Paunaka, see \sectref{sec:WordOrder}.} They usually share at least one participant and often but not always describe \isi{sequential} actions.

Consider (\ref{ex:lift-take}), where subject and object of both clauses are identical. A participant, \textit{echÿu jente} ‘the man’, is introduced as an object by a conominal NP in the first clause. The man is already well-established in the story, but not accessible as a possible object of the verb. This very same participant also acts as an object of the second clause, this time without being conominated. The subject of both clauses is topical and thus not conominated. The sentence stems from the tale told by María S. about the two men who meet the devil. The devil demands food and one man gives it to him. When the food is finished, the devil is still hungry, so he grabs the man and takes him with him in order to eat him.

\ea\label{ex:lift-take}
\begingl
\glpreamble chakachutuji echÿu jente chumu\\
\gla chÿ-akachu-tu-ji echÿu jente chÿ-umu\\
\glb 3-lift-\textsc{iam}-\textsc{rprt} \textsc{dem}b man 3-take\\
\glft ‘he picked up the man, it is said, and took him (with him)’
\endgl
\trailingcitation{[rxx-n120511l-2.57]}
\xe

In (\ref{ex:jump-fall}), the conominated object participant of the first verb, \textit{aitubuchepÿimÿnÿ} ‘little boy’, is a raised possessor\is{possessor raising} belonging to the incorporated\is{incorporation} body part \textit{-bÿke} ‘face’. It acts as the subject of the second, intransitive verb without being conominated again. The example stems from Miguel’s description of the \isi{frog story}, more precisely the picture in which the owl flies out of its hole in the tree and boy falls from the tree.

\ea\label{ex:jump-fall}
\begingl
\glpreamble i chijipubÿkejukukena eka aitubuchepÿimÿnÿ tibÿtupaikubu tukiu anÿke\\
\gla i chi-jipu-bÿke-ju-ku-kena eka aitubuchepÿi-mÿnÿ ti-bÿtupaikubu tukiu anÿke\\
\glb and 3-jump-face-?-\textsc{th}1-\textsc{uncert} \textsc{dem}a boy-\textsc{dim} 3i-fall from up\\
\glft ‘and it seems that it jumps into the face of the little boy and he falls down from above’
\endgl
\trailingcitation{[mox-a110920l-2.102]}
\xe

In (\ref{ex:lift-fish}), the subject of both verbs is identical. However, the participant \textit{chÿkuaji} ‘her fishing net’, which is introduced in the first clause, is not in a grammatical relation to the predicates of the second clause, two verbs in a motion-cum-purpose construction (see \sectref{sec:MotionCumPurpose}). However, the fishing net sets the ground semantically for the second clause, which is about fishing. Even more so, the verb \mbox{\textit{-epuiku}} ‘fish’ is only used for fishing with a net; when the Paunaka speak about fishing with a hook, they use a different verb. This example was produced by María C. to illustrate that her mother did not care for her, when she had a difficult birth and was weak by loss of blood.


\ea\label{ex:lift-fish}
\begingl
\glpreamble chakapaiku chÿkuaji tiyunu tepuikupu\\
\gla chÿ-akapaiku chÿ-kuaji ti-yunu ti-epuiku-pu\\
\glb 3-lift 3-net 3i-go 3i-fish-\textsc{dloc}\\
\glft ‘she picked up her net and went fishing’
\endgl
\trailingcitation{[ump-p110815sf.433]}
\xe

We also find asyndetic juxtaposition in coordination of clauses with SAP participants. This is the case in (\ref{ex:go-come}), in which both clauses have first person plural subjects. With this sentence, Miguel states that they went away from \isi{Altavista} and came to Santa Rita.

\ea\label{ex:go-come}
\begingl
\glpreamble biyunu bibÿsÿupunutu naka\\
\gla bi-yunu bi-bÿsÿupunu-tu naka\\
\glb 1\textsc{pl}-go 1\textsc{pl}-come-\textsc{iam} here\\
\glft ‘we went and then came here’
\endgl
\trailingcitation{[mqx-p110826l.380]}
\xe


Except for (\ref{ex:jump-fall}), in all examples presented up to here, the subjects of both coordinated clauses were identical. Another example of coordinated clauses with non-identical subjects is (\ref{ex:inscribe-stay}). The first person participant is introduced as the object of the first clause by being indexed on the verb with the marker \textit{-nÿ}, and then it acts as the subject of the intransitive verb in the second clause, being indexed by the marker \textit{ni-}. The sentence comes from Miguel’s account about how he acquired literacy and learned to calculate and it refers to the school.

\ea\label{ex:inscribe-stay}
\begingl
\glpreamble bueno, tisuichunÿtu echÿu profesor nipajÿkutu\\
\gla bueno ti-suichu-nÿ-tu echÿu profesor ni-pajÿku-tu\\
\glb well 3i-inscribe-1\textsc{sg}-\textsc{iam} \textsc{dem}b teacher 1\textsc{sg}-stay-\textsc{iam}\\
\glft ‘well, the teacher inscribed me and I stayed’
\endgl
\trailingcitation{[mxx-p181027l-1.019]}
\xe

The examples I have given up to this point resemble the ones cited by \citet{Mithun1988} to exemplify her category of “coordination by intonation”, i.e. asyndetic coordination.\is{intonation} However, many of them equally resemble the ones given by \citet{Aikhenvald2006,Aikhenvald2018} and \citet{Haspelmath2016} as  proof for another syntactic construction, namely the serial verb construction.\is{serial verb construction|(} This is because in all examples so far, at least one participant is shared, the verbs are often adjacent and they share the same RS.\is{reality status} I prefer the term asyndetic coordination over serial verb construction nonetheless. 

First of all, in an attempt to distinguish serial verbs from coordinated verbs, \citet[3,24,125-127]{Aikhenvald2018} proposes that the meaning changes if one inserts a connective\is{connective|(} between the verbs, i.e. serial verbs cannot be rephrased with a sequence of syndetically coordinated clauses. I have not tried this with the examples in this section, but my hypothesis is that nothing would change substantially if connectives were used. There are similar examples that do contain connectives (especially \isi{sequential} \textit{te} and \isi{conjunctive} \textit{i}, see \sectref{sec:SequentialCoordination} and \sectref{sec:ConjunctiveCoordination}).\is{serial verb construction|)} One factor in the choice between asyndetic and syndetic\is{syndesis/asyndesis} coordination I could identify is that the former is preferred if the time lapse between the events described by the two verbs is minimal as in (\ref{ex:lift-take}) above or also in (\ref{ex:snake-pull}) below. The situation described here is dangerous and thus requires quick and immediate action. The sentence comes from Juana, who was telling me how her sister María S. once met a snake (or water spirit), when she was bathing in the reservoir. Her husband saved her from being attacked.

\ea\label{ex:snake-pull}
\begingl
\glpreamble kapunu chima chijatÿku\\
\gla kapunu chÿ-ima chi-jatÿku\\
\glb come 3-husband 3-pull\\
\glft ‘her husband came and pulled her (out of the water)’
\endgl
\trailingcitation{[jxx-p120515l-2.168]}
\xe

In contrast, consider (\ref{ex:father-come}), which has the same predicates as (\ref{ex:lift-take}), but takes a \isi{sequential} connective \textit{te} (see \sectref{sec:SequentialCoordination}). It stems from Juana’s account about her grandparents, who lost the cows they had bought, because a \textit{karay} took them away. Apparently, Juana’s father found two of them in the pampa and brought them to \isi{Altavista}, where he lived at that time. The action of taking cows probably requires some preparation, and we can thus assume a bigger time lapse to the first action of coming.

\ea\label{ex:father-come}
\begingl
\glpreamble kapunu nÿabane te chumu nauku Turuxhiyaebane\\
\gla kapunu nÿ-a-bane te chÿ-umu nauku Turuxhi-yae-bane\\
\glb come 1\textsc{sg}-father-\textsc{rem} \textsc{seq} 3-take there Altavista-\textsc{loc}-\textsc{rem}\\
\glft ‘my late father came and took them to old Altavista’
\endgl
\trailingcitation{[jxx-e150925l-1.238]}
\xe

%kapunu punachÿ kayaraunu te chupunu chipeu baka, jxx-e150925l-1.250

This does, however, neither imply that a connective is impossible in expressing \isi{sequential} events with minimal time lapse nor that it is required when time lapse between two events is bigger.\is{connective|)}

The second reason why I prefer to analyse the examples in this section as cases of asyndetic coordination rather than as serial verbs\is{serial verb construction|(} is connected to grammatical integration: a crucial feature of serial verb constructions according to the definitions by \citet[296]{Haspelmath2016} and by \citet[1]{Aikhenvald2006} is that they are monoclausal.\is{negation|(} Consequently, they can only be negated in one way with negation having scope over both predicates \citep[299]{Haspelmath2016}. There are some construction types in Paunaka that fulfil this criterion, i.e. a serial verb construction with the motion verb \textit{-yunu} ‘go’ in which the second verb specifies the purpose of motion (see \sectref{sec:SerialVerbs}) and cases of complementation (see \sectref{sec:ComplementClauses}), but it is not clear whether the examples presented here could be defined as monoclausal by the criterion of (possible) negation having scope over both predicates. In general, clauses with negated predicates are seldom coordinated to other clauses in Paunaka, and if so, they usually have a \isi{connective} word.\is{serial verb construction|)}\footnote{\label{fn:negation-coordination}What we do find regularly is that first a negative clause is uttered and then the negator \textit{kuina} is repeated at the beginning of the next sentence. It is pronounced with a slightly falling \isi{intonation} and followed by a short pause. Juxtaposed to this negator, we find a positive clause, see (\ref{ex:neg-fn-1}). We can thus speak of one clause containing only the negator, which is coordinated to a positive clause. Word order would be identical if this was a single negative clause, but it would have a different intonation contour and no pause would occur between the negator and the following predicate. The example comes from Juana and is about the way chicha is served in Cotoca: in \textit{tutumas}, drinking vessels made from tree calebasses. 
\ea\label{ex:neg-fn-1}
\begingl
\glpreamble kuina tekikanube basoyae, kuina, japuyae \\
\gla kuina ti-ekika-nube basoyae kuina japu-yae\\
\glb \textsc{neg} 3i-serve.\textsc{irr}-\textsc{pl} glass-\textsc{loc} \textsc{neg} tutuma-\textsc{loc}\\
\glft ‘they don’t offer it in glasses, no, in \textit{tutumas}’
\endgl
\trailingcitation{[jxx-p120430l-2.571-574]}
\xe
%kuina tinikekÿkÿa chÿeche eka pat- musunube, kuina, trabajadorenube tinikujikanube eka ebijie, = he didn’t go providing the mozos with meat, no, the workers could only eat pututu, jxx-p120430l-2.026
This and some other patterns related to repetition of predicates can be considered bridging constructions,\is{bridging construction} though they do certainly not count as full-fledged tail-head linkage \citep[cf.][]{GuerinAiton2019}. Analysis of discourse structure remains a topic for future research.} Since negation often includes contrast, this calls for a less ambiguous strategy of clause combining. Nonetheless, there are a few examples of asyndetic coordination of one positive and one negative clause showing that negation only has scope over one predicate, i.e. the one that follows the negator directly.

If the negative clause comes first, both clauses often express the same circumstance with different wording. Consider (\ref{ex:neg-pos-1}) where it is very clear that negation only has scope over the first predicate. First of all, both predicates have different RS: the negative one is irrealis, the positive one realis. With this sentence María C. paraphrases that she was still very young when her father died and both clauses depict her age at the respective time. The first verb denoting speaking is negated, since the ability to speak is associated with older children. The second verb denoting crawling is associated with small children and thus it is not negated. \footnote{I do not know what the last sequence \textit{-kuyu} on \textit{nÿmajikukuyu} is, I suppose this was a slip and the speaker either wanted to use an incompletive marker \textit{-kuÿ} or derive a continuous verb with \textit{-CViku}. A third, more improbable option is that she reduplicated the last syllable of the stem for an unknown reason and added the intensifier \textit{-yu}. Incompletive marking is most probable, and this is what I propose in the analysis of the word.}

\ea\label{ex:neg-pos-1}
\begingl
\glpreamble kuina nichujikakuÿmÿnÿ nÿmajikukuyu\\
\gla kuina ni-chujika-kuÿ-mÿnÿ nÿ-majiku-kuÿ?\\
\glb \textsc{neg} 1\textsc{sg}-speak.\textsc{irr}-\textsc{incmp}-\textsc{dim} 1\textsc{sg}-crawl-\textsc{incmp}?\\
\glft ‘I did not speak yet, I was still crawling’
\endgl
\trailingcitation{[cux-c120414ls-2.278]}
\xe

%- with -uku = additive? I was also crawling a lot??

A similar example is (\ref{ex:neg-pos-2}), a comment by María S. about me coming to Bolivia without my daughters. She uses the two antonyms \textit{-upunu} ‘bring’ and \textit{-eneiku} ‘leave’ to express this fact, the first one being negated, the second one not. Thus, as in (\ref{ex:neg-pos-1}) above, negation has scope over the first predicate, but not over the second.

\ea\label{ex:neg-pos-2}
\begingl
\glpreamble kuina pupunanube peneikunubetu\\
\gla kuina pi-upuna-nube pi-eneiku-nube-tu\\
\glb \textsc{neg} 2\textsc{sg}-bring.\textsc{irr}-\textsc{pl} 2\textsc{sg}-leave-\textsc{pl}-\textsc{iam}\\
\glft ‘you didn’t bring them, you have left them’
\endgl
\trailingcitation{[rmx-e150922l.079]}
\xe

%así que kuina pichupa - pejekukubitu? mdx-c120416ls.107

It is also possible that the second clause is negated. This is the case in the following examples, which have in common that the first predicate has realis RS and the second one irrealis. While in (\ref{ex:asy-neg-1}), the negative clause provides a consequence\is{consecutive} of the first clause, in (\ref{ex:asy-neg-2}) and (\ref{ex:house-not-finish}) we find adversative coordination. This is rather unusual. Given that the cognitive demand of processing adversative clauses is supposedly higher \citep[cf.][304--305]{Matras1998}, speakers tend to use a \isi{connective} to overtly signal the relation. 

In (\ref{ex:asy-neg-1}), Miguel speculates why he does not hear the singing of frogs at night. Only the second verb is negated and additionally, the uncertainty marker attached to the first verb does not have scope over the second verb.

\ea\label{ex:asy-neg-1}
\begingl
\glpreamble nÿti nimukukena kuina nisama\\
\gla nÿti ni-muku-kena kuina ni-sama\\
\glb 1\textsc{sg.prn} 1\textsc{sg}-sleep-\textsc{uncert} \textsc{neg} 1\textsc{sg}-hear.\textsc{irr}\\
\glft ‘maybe I sleep, thus I don’t hear them (the frogs at night)’
\endgl
\trailingcitation{[mqx-p110826l.622]}
\xe

(\ref{ex:asy-neg-2}) was provided by Juana in an elicitation session.

\ea\label{ex:asy-neg-2}
\begingl
\glpreamble niniku kuina nikupunÿkapu\\
\gla ni-niku kuina ni-kupunÿkapu\\
\glb 1\textsc{sg}-eat \textsc{neg} 1\textsc{sg}-be.full.\textsc{irr}\\
\glft ‘I ate, but I am not full’
\endgl
\trailingcitation{[jxx-e150925l-1.015]}
\xe

(\ref{ex:house-not-finish}) comes from María S. and is about her sister Juana who had been to Concepción to work on her house.

\ea\label{ex:house-not-finish}
\begingl
\glpreamble tanaubu echÿu chubiu kuina tateane keuchi faltau plata\\
\gla ti-anau-bu echÿu chÿ-ubiu kuina ti-a-teane keuchi faltau plata\\
\glb 3i-make-\textsc{mid} \textsc{dem}a 3-house \textsc{neg} 3i-\textsc{irr}-finish \textsc{ins} lack money\\
\glft ‘she made her house, but didn’t finish because of lack of money’
\endgl
\trailingcitation{[rxx-e120511l.115]}
\xe

\is{negation|)}


I want to conclude this section with (\ref{ex:stay-stay}), an example that does not have a connective, but the second verb carries the additive marker\is{additive|(} \textit{-uku}, thus providing a strong sign of connectedness of these clauses. In this case, the clauses have different third person subjects but the same verb. The additive marker is not restricted to coordination contexts (see \sectref{sec:Additive}), thus this may still be considered as asyndetic rather than syndetic coordination. There are not many similar sentences in the corpus, so use of the additive marker cannot be considered a major coordination strategy.\is{additive|)}  The sentence was produced by María S. in telling me how all of her siblings dispersed to live in different places except for Miguel and her.

\ea\label{ex:stay-stay}
\begingl
\glpreamble jaja repue nÿtitu nÿpajÿku naka Miyel tipajÿkuku naka\\
\gla jaja repue nÿti-tu nÿ-pajÿku naka Miyel ti-pajÿku-uku naka\\
\glb \textsc{afm} afterwards 1\textsc{sg.prn}-\textsc{iam} 1\textsc{sg}-stay here Miguel 3i-stay-\textsc{add} here\\
\glft ‘yes, then I stayed here and Miguel also stayed here’
\endgl
\trailingcitation{[rxx-p181101l-2.267]}
\xe

The following sections deal with cases of coordination in which a connective overtly shows the kind of connection between the two clauses.

%(\ref{ex:cry-sad}) was uttered by Juana in telling me about her sister’s death.\footnote{The first part \textit{tukiu nechÿu} ‘from there’ is the attempt to get back to the storyline after some disruption in Spanish to scold her grandson.}
%
%\ea\label{ex:cry-sad}
%\begingl
%\glpreamble i tukiu nechÿu niyunu nauku nisachu niyua nichÿnumi\\
%\gla i tukiu nechÿu ni-yunu nauku ni-sachu ni-iyua ni-chÿnumi\\
%\glb and from \textsc{dem}c 1\textsc{sg}-go there 1\textsc{sg}-want 1\textsc{sg}-cry.\textsc{irr} 1\textsc{sg}-be.sad\\
%\glft ‘and to continue, I went there, I wanted to cry, I was sad’\\
%\endgl
%\trailingcitation{[jxx-p120430l-2.236]}
%\xe
%
%pero bichupakena timesuveikabitu = but we will know it, they teach us, we know, jmx-e090727s.031 -> cause clause?


\subsection{Sequential coordination}\label{sec:SequentialCoordination}\is{connective|(}
\is{sequential|(}

The connective \textit{te} is inserted between two clauses to express that the events happen in temporal succession, in a sequence. The events of both clauses are always semantically related. They belong to an overarching discourse topic. This means that \textit{te} is usually not chosen when the topic changes. Juana makes extensive use of this connective, while Miguel uses it only rarely. 

In (\ref{ex:SUCC-te-1}) the general discourse topic is the rest of Juana’s grandparents on their journey back from Moxos, where they had bought cows. The sentence describes two things the grandparents did when they rested. The choice of the connective \textit{te} not only signals that both actions belong together, but also that they happened in a certain sequence: first the cows were brought to the enclosure and locked in and only then the grandparents cooked. Irrealis RS in this sentence is due to this utterance reflecting a non-specific repeated (“habitual”) event, since the journey took several days, so several rests were necessary. All other clauses surrounding this example in the original text also have irrealis RS. 

\ea\label{ex:SUCC-te-1}
\begingl
\glpreamble chiratanÿkanube te tiyÿtikapunube\\
\gla chi-ratanÿka-nube te ti-yÿtikapu-nube\\
\glb 3-lock.in.\textsc{irr}-\textsc{pl} \textsc{seq} 3i-cook.\textsc{irr}-\textsc{pl}\\
\glft ‘they would lock them (i.e. their cows) in and then they would cook’
\endgl
\trailingcitation{[jxx-p151016l-2.058]}
\xe


(\ref{ex:SUCC-te-2}) was produced by María S. in an elicitation session. In this case \textit{te} connects two actions by chicken: eating and going. The first verb has realis RS, which together with the iamitive marker marks this action as completed (perfective reading). The second verb has irrealis RS, which signals that the action is not completed yet, but the iamitive marker tells us that the chicken are about to go (imperfective reading).\is{perfective/imperfective} The connection of the two events in temporal succession is additionally expressed by the connective \textit{te}. 

\ea\label{ex:SUCC-te-2}
\begingl
\glpreamble tinijaneutu takÿrajane te tiyunujaneatu\\
\gla ti-ni-jane-u-tu takÿra-jane te ti-yunu-jane-a-tu\\
\glb 3i-eat-\textsc{distr}-\textsc{real}-\textsc{iam} chicken-\textsc{distr} \textsc{seq} 3i-go-\textsc{distr}-\textsc{irr}-\textsc{iam}\\
\glft ‘the chicken have eaten and now they go’
\endgl
\trailingcitation{[rxx-e181022le]}
\xe

The next example, (\ref{ex:pot-select}) by Juana, stems from a description of how to make a clay pot. The first step after collecting the loam is grinding, then the pebbles turn up that have to be picked out in order to make a nice and smooth pot.

\ea\label{ex:pot-select}
\begingl
\glpreamble bibikÿkeka maichubaji te banatu nÿkÿikina\\
\gla bi-bikÿkeka maichuba-ji te bi-ana-tu nÿkÿiki-ina\\
\glb 1\textsc{pl}-select.\textsc{irr} pebble-\textsc{col} \textsc{seq} 1\textsc{pl}-make.\textsc{irr}-\textsc{iam} pot-\textsc{irr.nv}\\
\glft ‘we have to select the pebbles (from the ground loam) then we can make the pot’
\endgl
\trailingcitation{[jmx-d110918ls-1.005]}
\xe


In (\ref{ex:SUCC-te-1}) to (\ref{ex:pot-select}) the subjects of both coordinate clauses are the same. This is often the case given that both events belong to one discourse topic, but not necessarily so. The events may be connected in other ways than having the same subject, which can be seen in examples (\ref{ex:cooked-eat}) to (\ref{ex:no-chicha}).

In (\ref{ex:cooked-eat}) the patient\is{patient/theme} is shared, which is the subject of the first, stative intransitive verb and the object of the second, transitive verb. It is not conominated. The second sentence has a first person plural subject. %Using first person plural indexes is one way Paunaka speakers form impersonal reference. 
 The example comes from one of the descriptions about making the clay pot by Juana. Here she speaks about cooking food in the pot. The sentence is best translated to English by using a subordinate clause including ‘when’, but it is not subordinate in Paunaka.

\ea\label{ex:cooked-eat}
\begingl
\glpreamble taima te binikatu\\
\gla ti-a-ima te bi-nika-tu\\
\glb 3i-\textsc{irr}-be.cooked \textsc{seq} 1\textsc{pl}-eat.\textsc{irr}-\textsc{iam}\\
\glft ‘when it was done, we would eat it’
\endgl
\trailingcitation{[jxx-d110923l-2.25]}
\xe


In (\ref{ex:sunset}), the two clauses do not share any participant. The first clause has an environment verb whose subject is the sun, while the subject of the second verb is the lazybones, who also acts as a subject of all other surrounding sentences and can thus be said to be topical (in terms of both information structure and larger discourse organisation). RS marking is the opposite of (\ref{ex:SUCC-te-2}), the combination of irrealis and iamitive on the first verb shows that it has an imperfective reading, while the second verb with realis RS and iamitive has a perfective reading.\is{perfective/imperfective} The sentence was produced by Miguel in telling the story about the lazybones, who did not do anything else than swinging in his liana hammock the whole day, before going home.

\ea\label{ex:sunset}
\begingl
\glpreamble tibÿkupatuji sache te tiyunupunukutuji timukupuji\\
\gla ti-bÿkupa-tu-ji sache te ti-yunu-punuku-tu-ji ti-muku-pu-ji\\
\glb 3i-enter.\textsc{irr}-\textsc{iam}-\textsc{rprt} sun \textsc{seq} 3i-go-\textsc{reg}-\textsc{iam}-\textsc{rprt} 3i-sleep-\textsc{dloc}-\textsc{rprt}\\
\glft ‘at sunset (lit.: the sun was entering) he went home to sleep, it is said’
\endgl
\trailingcitation{[mox-n110920l.043]}
\xe


(\ref{ex:no-chicha}) from María C. includes three clauses which are coordinated by \textit{te}. The first and the second have the same subject, the corn, which is introduced by an NP in the first clause and conominated by a demonstrative in the second one. The third clause has a different subject, the speaker herself who is affected by the imminent shortage of corn, being already worried that soon she could not drink chicha anymore. The third clause does not share any participant with the other two, but is nonetheless semantically connected by the relation between corn and chicha, because chicha is made of corn.

\ea\label{ex:no-chicha}
\begingl
\glpreamble kakumÿnÿ amukemÿnÿ te tibukapu echÿu te kuinabu nea aumue\\
\gla kaku-mÿnÿ amuke-mÿnÿ te ti-buka-pu echÿu te kuina-bu nÿ-ea aumue\\
\glb exist-\textsc{dim} corn-\textsc{dim} \textsc{seq} 3i-finish.\textsc{irr}-\textsc{mid} \textsc{dem}b \textsc{seq} \textsc{neg}-\textsc{dsc} 1\textsc{sg}-drink.\textsc{irr} chicha\\
\glft ‘there is little corn and when it will be finished, then I cannot drink chicha anymore’
\endgl
\trailingcitation{[ump-p110815sf.693]}
\xe

As for the question to which of the clauses the connective \textit{te} belongs, there is no definite answer. There may be a short pause after \textit{te} and a drop in pitch, which suggests it belongs to the first clause. However, it also occurs after a short pause and then rather seems to be attached to the second one or there may be no pause at all and no other hint in \isi{intonation} that would point into one or the other direction. \textit{Te} occasionally combines with \textit{i} ‘and’, and we find both, \textit{te} preceding \textit{i} as in (\ref{ex:te-i}) and \textit{te} following \textit{i} as in (\ref{ex:i-te}).

In (\ref{ex:te-i}) there is a short pause after \textit{te}, and the next clause starts with \textit{i}.\footnote{For the combination of the adverb \textit{metu} ‘already’ with a deranked subordinate verb see \sectref{sec:AdverbialModification} where I explain this special construction. In short, although the first clause has a deranked verb, it is by no means subordinate to the clause that starts with the connective \textit{i}.} The example comes from Juana’s account about how her sister María S. was once attacked by a snake or water spirit.


\ea\label{ex:te-i}
\begingl
\glpreamble metu chikubiu te i chima nauku chimuji echÿu chikuye chichÿti ÿneyae\\
\gla metu chi-kub-i-u te i chi-ima nauku chi-imu-ji echÿu chi-kuye chi-chÿti ÿne-yae\\
\glb already 3-bathe-\textsc{subord}-\textsc{real} \textsc{seq} and 3-husband there 3-see-\textsc{rprt} \textsc{dem}b 3-be.like.this 3-head water-\textsc{loc}\\
\glft ‘she finished bathing and then her husband there saw something like a head in the water, it is said’
\endgl
\trailingcitation{[jxx-p120515l-2.154-155]}
\xe

In (\ref{ex:i-te}) from Miguel, the connectives \textit{i} and \textit{te} introduce the second clause after a burst of laughter. The example comes from the story about the fox and the jaguar and at this point in the story, the vulture has let the fox escape. The jaguar is angry with the vulture and wants to eat him. The vulture can convince the jaguar to pluck him except for the wings and throw him up with the promise to return from up there and fall right into his mouth. Instead of this, the vulture defecates into the jaguar’s mouth and escapes.

\ea\label{ex:i-te}
\begingl
\glpreamble tisukuejikuji chinabakÿyae i te tibÿbÿkutuji echÿu sÿmÿ tiyunu\\
\gla ti-suku-e-jiku-ji chi-naba-kÿ-yae i te ti-bÿbÿku-tu-ji echÿu sÿmÿ ti-yunu\\
\glb 3i-defecate-?-\textsc{lim}1-\textsc{rprt} 3-mouth.inside-\textsc{clf:}bounded-\textsc{loc} and \textsc{seq} 3i-fly-\textsc{iam}-\textsc{rprt} \textsc{dem}b vulture 3i-go\\
\glft ‘he shat into his open mouth, it is said, and then the vulture flew off, it is said, and went (i.e. escaped), it is said’
\endgl
\trailingcitation{[jmx-n120429ls-x5.209-211]}
\xe

If there is a whole sequence of more than two clauses, \textit{te} can occur on the last one only, as in (\ref{ex:tortoise-egg}), in which María S. describes how the tortoise lays its eggs. I am not sure why she changes from realis to irrealis RS here.

\ea\label{ex:tortoise-egg}
\begingl
\glpreamble tisuku kipÿ, tisekumÿnÿ epenue tisuka nechÿu, cheneikachÿ, chetunÿka echÿpune, depue tisuka nechÿu, te tiyunu\\
\gla ti-suku kipÿ ti-seku-mÿnÿ epenue ti-suka nechÿu chÿ-eneika-chÿ chÿ-etunÿka echÿpune depue ti-suka nechÿu te ti-yunu\\
\glb 3i-lay.egg tortoise 3i-dig.hole-\textsc{dim} hole 3i-lay.egg.\textsc{irr} \textsc{dem}c 3-leave.\textsc{irr}-3 3-put.around.\textsc{irr} leaf afterwards 3i-lay.egg.\textsc{irr} \textsc{dem}c \textsc{seq} 3i-go\\
\glft ‘the tortoise lays eggs, it digs a little hole to lay an egg there, it will leave it, it will put leaves around it, after that it will lay an egg there, then it goes’
\endgl
\trailingcitation{[rxx-e121128s-1.088]}
\xe

However, the sequential connective can also occur several times if more than two clauses are conjoined as in (\ref{ex:no-chicha}) above, or in (\ref{ex:flower-eat}), also produced by María S. in a correction session with Swintha, repeating what she had said on another occasion, but using slightly different wording.

\ea\label{ex:flower-eat}
\begingl
\glpreamble takujibÿ eka te kanainatu chÿi te puero binika\\
\gla ti-a-kujibÿ eka te kana-ina-tu chÿi te puero bi-nika\\
\glb 3i-\textsc{irr}-have.flower \textsc{dem}a \textsc{seq} this.size-\textsc{irr}-\textsc{iam} fruit \textsc{seq} can 1\textsc{pl}-eat.\textsc{irr}\\
\glft ‘it blossoms and once its fruits have this size (showing with hands), we can eat them’
\endgl
\trailingcitation{[rxx-e121126s-3.16]}
\xe

I want to conclude this section with an example produced by Juana, which, besides having a clause introduced by \textit{te}, shows how nicely Spanish material is integrated into Paunaka speech.\footnote{In this example \textit{sinko minutos} (from \textit{cinco minutos)} includes the Spanish plural marker \textit{-s} on the noun, but it takes the incompletive marker form Paunaka. \textit{Labion} can be considered a loan from \textit{el avión} ‘the plane’, possibly \textit{la avión}, with feminine gender in local Spanish.} It stems from the account about her daughter who went to Spain, but was deported, despite her sister, who already lived in Spain at that time, trying to avoid that. In Juana’s opinion, she just arrived at the airport too late.

\ea
\begingl
\glpreamble sinko minutoskuÿ te tibÿbÿkatu labion\\
\gla sinko minutos-kuÿ te ti-bÿbÿka-tu labion\\
\glb five minutes-\textsc{incmp} \textsc{seq} 3i-fly.\textsc{irr}-\textsc{iam} plane\\
\glft ‘it was still five minutes until the plane would take off’
\endgl
\trailingcitation{[jxx-p110923l-1.330]}
\xe
\is{sequential|)}

The next section is about the clauses conjoined by \textit{i} ‘and’.
 
\subsection{Conjunctive coordination}\label{sec:ConjunctiveCoordination}\is{conjunctive|(}

The conjunctive connective \textit{i} ‘and’ has been borrowed\is{borrowing} from Spanish \textit{y}. It hardly ever conjoins NPs,\is{noun phrase} although a few examples of NP connection with \textit{i} do occur in the corpus. The connective can conjoin clauses, and I got the impression that it is also often used to introduce a clause after change of discourse topic or turn-taking, but as I have mentioned before, information structure and the general organisation of discourse and conversation is beyond the scope of this work, so here I confine my analysis to connection of clauses.
% NP conjoining:
%hm i kakuku eka kabemÿne i eka peÿ, mtx-a110906l.004-006
%pÿsisikubutu Miyel, Kose i nÿti - i Krara - María, jxx-p120515l-2.234-236

As for the latter, \textit{i} does little more than connect. It can occur at the end of one \isi{intonation} unit or at the beginning of the second one or without any hint in intonation as to which clause it belongs to. The connected clauses can have the same subject or different ones. The predicates of both clauses are fully inflected. A few examples follow. In (\ref{ex:oven-fish}) and (\ref{ex:liana}), the subjects of both clauses are identical, and in (\ref{ex:and-1}) the subjects of the two coordinated clauses are different.

(\ref{ex:oven-fish}) was produced by Clara to tell us how she prepared a fish.

\ea\label{ex:oven-fish}
\begingl
\glpreamble betuku jurunuyae i biyÿbapaku\\
\gla bi-etuku jurunu-yae i bi-yÿbapaku\\
\glb 1\textsc{pl}-put oven-\textsc{loc} and 1\textsc{pl}-grind\\
\glft ‘we put it (the fish) into the oven and grind it’
\endgl
\trailingcitation{[cux-c120414ls-2.158]}
\xe


(\ref{ex:liana}) is from the story about the lazybones and describes what the lazybones does in the woods instead of making his field, the work he is supposed to do.

\ea\label{ex:liana}
\begingl
\glpreamble tebibikujiku kujipiyae i tikusabenunuiku chisabenu\\
\gla ti-ebibiku-jiku kujipi-yae i ti-kusabenu-nuiku chi-sabenu\\
\glb 3i-swing-\textsc{lim}1 liana.sp-\textsc{loc} and 3i-play.flute-\textsc{cont} 3-flute\\
\glft ‘he only swung on the liana and was playing his flute’
\endgl
\trailingcitation{[mox-n110920l.048-049]}
\xe

In (\ref{ex:and-1}), Miguel speaks about the past. The passage in his account from which the example is taken is about the time when debt bondage was forbidden and indigenous people were finally free. The sentence is about the unfair treatment of indigenous people prior to this. Note that the non-verbal predicate of the first clause irregularly takes a third person marker \textit{-chÿ} here.\is{person marking}

\ea\label{ex:and-1}
\begingl
\glpreamble chamayÿchi trabakuyunubechÿ i kuina chisiupuchanube eka patron\\
\gla chama-yÿchi trabaku-yu-nube-chÿ i kuina chi-siupucha-nube eka patrun\\
\glb much-\textsc{lim}2 work-\textsc{ints}-\textsc{pl}-3 and \textsc{neg} 3-pay.\textsc{irr}-\textsc{pl} \textsc{dem}a patrón\\
\glft ‘they worked a lot and the \textit{patrón} didn’t pay them’
\endgl
\trailingcitation{[mxx-p110825l.042]}
\xe


There is hardly ever any ellipsis (or gapping) of predicates in the second of the conjoined clauses, which may be due to the fact that speakers seldom use the same predicate in two conjoined clauses. But even if this is the case, the predicate is rather repeated than omitted. Two sentences follow to exemplify this. While in (\ref{ex:workwork}) both clauses have the same predicate but different subjects, in (\ref{ex:makemake}) the predicates and the subjects of both clauses are identical.

(\ref{ex:workwork}) is about Juana’s daughter in Spain and her husband who both worked and thus had trouble looking after their child.

\ea\label{ex:workwork}
\begingl
\glpreamble chima trabaku i nijinepÿi trabaku\\
\gla chÿ-ima trabaku i ni-jinepÿi trabaku\\
\glb 3-husband work and 1\textsc{sg}-daughter work\\
\glft ‘her husband worked and my daughter worked’
\endgl
\trailingcitation{[jxx-p110923l-1.359]}
\xe

(\ref{ex:makemake}) is from Miguel’s account about how Santa Rita was founded and grew and finally also got a chapel and a rectory.

\ea\label{ex:makemake}
\begingl
\glpreamble entonses tanaunube echÿu kapiya i repue tanaunube echÿu punachÿ, parokia\\
\gla entonses ti-anau-nube echÿu kapiya i repue ti-anau-nube echÿu punachÿ parokia\\
\glb thus 3i-make-\textsc{pl} \textsc{dem}b chapel and afterwards 3i-make-\textsc{pl} \textsc{dem}b other rectory\\
\glft ‘thus they made the chapel and afterwards they made the other one, the rectory’
\endgl
\trailingcitation{[mxx-p110825l.131]}
\xe

A few cases of gapping are found in Miguel’s telling of the \isi{frog story}. One of them is given below. It refers to the picture towards the end of the book, in which the boy leans over the log and the dog is standing on it. 

\ea\label{ex:ellipsis}
\begingl
\glpreamble tijipuikutuji chÿneyae echÿu yÿkÿke i echÿuku\\
\gla ti-jipuiku-tu-ji chÿ-ine-yae echÿu yÿkÿke i echÿu-uku\\
\glb 3i-jump-\textsc{iam}-\textsc{rprt} 3-on.top-\textsc{loc} \textsc{dem}b tree and \textsc{dem}b-\textsc{add}\\
\glft ‘it (the dog) has jumped on top of the log, it is said, and he (the boy), too’
\endgl
\trailingcitation{[mtx-a110906l.207-210]}
\xe 

Finally, the last example I want to present here has several conjoined clauses, most of them connected to each other with \textit{i}, but we also find the sequential connective \textit{te} in combination with \textit{i}, where \textit{te} occurs at the end of one intonation unit and \textit{i} introduces the next one. The last two predicates form a serial verb construction (see \sectref{sec:SerialVerbs}), although strictly speaking, the last predicate \textit{pretau} ‘borrow, lend’ is not a verb. It is borrowed from Spanish and integrated into Paunaka as a non-verbal predicate (see \sectref{sec:borrowed_verbs}). The sentence is about Juana’s grandson. She raised him, because his parents had to work.

\ea
\begingl
\glpreamble tijÿku i netuku xhikuera i tichunatu te i tiyunu kuarterayae pretau chiserbisione premilitar\\
\gla ti-jÿku i nÿ-etuku xhikuera i ti-ichuna-tu te i ti-yunu kuartera-yae pretau chi-serbisio-ne premilitar\\
\glb 3i-grow and 1\textsc{sg}-put school and 3i-be.capable-\textsc{iam} \textsc{seq} and 3i-go military.base-\textsc{loc} lend 3-service-\textsc{possd} pre-military\\
\glft ‘he grew and I put him into school and once he had acquired knowledge, then he went to the military base to do pre-military service’
\endgl
\trailingcitation{[jxx-p110923l-1.173-178]}
\xe


%punuku mane tinikutuji mane i tiyunupunukuji = the other morning he ate i the morning and he went again, mox-n110920l.045

%Ellipsis can only refer to ellipsis of predicates, since NPs do not have to co-occur with predicates, the arguments being indexed on the predicate
%with ellipsis: i nakauku kakuku peÿ naka i kabemÿne naka i echÿu aitubuchepÿimÿne timuku, mtx-a110906l.017-018 (another one in the same recording, also with kaku)
\is{conjunctive|)}

In the following section, the use of the much rarer connective \textit{o} ‘or’ is described.

\subsection{Disjunctive coordination}\label{sec:DisjunctiveCoordination}\is{disjunctive|(}

The disjunctive connective \textit{o} ‘or’ has been borrowed\is{borrowing} from Spanish, which has an identical disjunctive conjunction. In Paunaka, however, it is sometimes also pronounced \textit{a}. In contrast to clauses conjoined by \textit{i} ‘and’ (see \sectref{sec:ConjunctiveCoordination}), there is usually ellipsis of constituents in the second of the clauses connected by \textit{o} ‘or’. Some examples follow.

The context of (\ref{ex:or-1}) was Juan C. communicating his wish for educational material for Paunaka, containing drawings of different animals of the region together with their names in Paunaka.

\ea\label{ex:or-1}
\begingl
\glpreamble entonses kakuina chija naka anÿke o naka chÿupekÿye\\
\gla entonses kaku-ina chi-ija naka anÿke o naka chÿ-upekÿ-yae\\
\glb thus exist-\textsc{irr.nv} 3-name here up or here 3-place.under-\textsc{loc}\\
\glft ‘thus its name (of the animal) may be up here (on the page) or here under it (i.e. under the drawing)’
\endgl
\trailingcitation{[mqx-p110826l.663]}
\xe

(\ref{ex:farornear}) comes from an elicitation session on expression of spatial relations. I asked Miguel and Alejo to play a game with two identical sets of wooden toys. Miguel had to arrange his set of toys in the way Alejo arranged it without looking at it, just by asking him questions.

\ea\label{ex:farornear}
\begingl
\glpreamble ¿juchubu kaku echÿu ubiae, tÿbaneyu eka yÿkÿke o mÿbanejiku?\\
\gla juchubu kaku echÿu ubiae ti-ÿbane-yu eka yÿkÿke o mÿbane-jiku\\
\glb where exist \textsc{dem}b house 3i-be.far-\textsc{ints} \textsc{dem}a tree or close-\textsc{lim}\\
\glft ‘where is the house, far from the tree or close to it?’
\endgl
\trailingcitation{[mtx-e110915ls.57]}
\xe


(\ref{ex:or-1}) and (\ref{ex:farornear}) both have a non-verbal predicate, the copula \textit{kaku}. The following examples have verbal predicates.

Prior to the sentence in (\ref{ex:or-3}), Juana had expressed that she wanted Miguel to visit her again (he had been there the day before), because she could not remember the name of a bird, thus she hoped that either he would know it or she would remember it in his presence.

\ea\label{ex:or-3}
\begingl
\glpreamble echÿu chichupa o nÿtikena\\
\gla echÿu chi-chupa o nÿti-kena\\
\glb \textsc{dem}b 3-know.\textsc{irr} or 1\textsc{sg.prn}-\textsc{uncert}\\
\glft ‘either he would know it or maybe I would’
\endgl
\trailingcitation{[jxx-p120430l-1.094]}
\xe

(\ref{ex:or-2}) is a question that Clara directed to Swintha and me, when we visited her and María C. We were a bit tired, because we had already been to Santa Rita that day.

\ea\label{ex:or-2}
\begingl
\glpreamble ¿tose etupupunubu o kupeitu?\\
\gla tose e-tupupunubu o kupei-tu\\
\glb noon 2\textsc{pl}-arrive.back or afternoon-\textsc{iam}\\
\glft ‘did you arrive back (from Santa Rita) at noon or in the afternoon?’
\endgl
\trailingcitation{[cux-c120414ls-2.332]}
\xe

 The connective \textit{o} is also often used in self-correction after a false start, to correct the use of a word etc. This is the case in (\ref{ex:or-cor-1}) and (\ref{ex:or-cor-2}).
 
(\ref{ex:or-cor-1}) is the answer Clara gave Swintha who asked how to translate ‘have a head\-ache’ to Paunaka. Swintha used an infinitive in Spanish (\textit{doler la cabeza}). There is no infinitive in Paunaka, and the phrase is first translated by Clara with a sentence containing a third person subject. However, she then found it more appropriate to use a translation with first person reference and thus corrected herself, signalling this correction by the use of \textit{o}.

\ea\label{ex:or-cor-1}
\begingl
\glpreamble tikuti chichÿti, o tikuti nÿchÿti \\
\gla ti-kuti chi-chÿti o ti-kuti nÿ-chÿti \\
\glb 3i-hurt 3-head or 3i-hurt 1\textsc{sg}-head\\
\glft ‘his head aches or my head aches’
\endgl
\trailingcitation{[cux-c120414ls-1.007-008]}
\xe

In (\ref{ex:or-cor-2}), Miguel corrects a false start. Apparently, he first wanted to say something that applied to all pupils in his former class, back in the old days, when he went to school. Thus he used the first person plural marker \textit{bi-}, but corrected himself with a hesitation mark and \textit{o} to tell me something about a single third person referent (though an indefinite one). He reports that his teacher used to whip the child who did not know what they were supposed to have learned, and if several did not know, all children were punished.

\ea\label{ex:or-cor-2}
\begingl
\glpreamble kue bi- ee o punachÿ echÿu sesejinube uu tumuyubunube testaikechunubetu\\
\gla kue bi- ee o punachÿ echÿu sesejinube uu tumuyubu-nube ti-estaikechu-nube-tu\\
\glb if 1\textsc{pl}- er or other \textsc{dem}b children \textsc{intj} all-\textsc{pl} 3i-whip.all-\textsc{pl}-\textsc{iam}\\
\glft ‘if we – er – or another one of the children [didn’t know] uh he whipped them all’
\endgl
\trailingcitation{[mxx-p181027l-1.077]}
\xe 

Disjunction is not very frequent in the corpus\is{disjunctive|)} unlike adversative coordination with the connective \textit{pero} ‘but’, which is the topic of the next section.

\subsection{Adversative coordination}\label{sec:AdversativeCoordination}\is{adversative|(}

In adversative coordination, Paunaka speakers use the connective \textit{pero} ‘but’, which is borrowed\is{borrowing} from Spanish. It encodes contrast and/or the fact that something is contrary to the expectation of speaker or hearer. \textit{Pero} often occurs at the beginning of an \isi{intonation} unit in order to connect it to the previous discourse, and is in those cases probably best reflected by the English word ‘however’, but \textit{pero} is also found in clause combining in its narrower sense. Some examples follow.

In (\ref{ex:pero-2}), Juana tells me that she would like to travel to Europe sometime, but does not dare to. The adversative meaning is not only conveyed by use of the connective, but also by the frustrative marker on the predicate in the first clause.

\ea\label{ex:pero-2}
\begingl
\glpreamble nisachuini niyuna pero nipiku\\
\gla ni-sachu-ini ni-yuna pero ni-piku\\
\glb 1\textsc{sg}-want-\textsc{frust} 1\textsc{sg}-go.\textsc{irr} but 1\textsc{sg}-be.afraid\\
\glft ‘I would like to go, but I am afraid’
\endgl
\trailingcitation{[jxx-p110923l-1.403]}
\xe

The context of the following example is María S. telling me that when she still lived more remote from the village, her chicken got lost, because other people stole them. In contrast, in Santa Rita, chicken do not get lost, or better said they do, but the reason being herself killing them.

\ea\label{ex:pero-chicken}
\begingl
\glpreamble tijekupubu pero nÿti nikupaka, ninika\\
\gla ti-jekupu-bu pero nÿti ni-kupaka ni-nika\\
\glb 3i-lose-\textsc{mid} but 1\textsc{sg.prn} 1\textsc{sg}-kill.\textsc{irr} 1\textsc{sg}-eat.\textsc{irr}\\
\glft ‘they do get lost, but (since) I may kill them and eat them’
\endgl
\trailingcitation{[rxx-e120511l.184-185]}
\xe

%chikuye aa michachaikunÿ pero tanÿma tikuti nÿsÿikuke = aa I am doing well, but my knee hurts now, rxx-e181017l.008

Prior to (\ref{ex:pero-1}), María C. and Clara had claimed that they did not have anybody to talk with in Paunaka. According to María C., Clara’s daughters are not capable of learning it.

\ea\label{ex:pero-1}
\begingl
\glpreamble kaku pijinejinube pero kuina puero chitanube \\
\gla kaku pi-jine-ji-nube pero kuina puero chi-ita-nube \\
\glb exist 2\textsc{sg}-daughter-\textsc{col}-\textsc{pl} but \textsc{neg} can 3-master.\textsc{irr}-\textsc{pl}\\
\glft ‘you have daughters, but they can’t figure it out (to speak Paunaka)’
\endgl
\trailingcitation{[cux-c120414ls-2.265]}
\xe

(\ref{ex:howeverbut}) was produced by Miguel when telling the story of the fox and the jaguarundi. The fox has just met the jaguarundi and wants to go and steal some chicken with him. They find strong chicha instead of chicken and decide to get drunk.

\ea\label{ex:howeverbut}
\begingl
\glpreamble pero tibÿkupunube kuina takÿraina pero chimukunubeji echÿu barerekiji tijapÿkubu isipau\\
\gla pero ti-bÿkupu-nube kuina takÿra-ina pero chi-imu-uku-nube-ji echÿu barereki-ji ti-japÿku-bu isipau\\
\glb but 3i-enter-\textsc{pl} \textsc{neg} chicken-\textsc{irr.nv} but 3-see-\textsc{add}-\textsc{pl}-\textsc{rprt} \textsc{dem}b pot-\textsc{rprt} 3i-fill-\textsc{mid} strong.chicha\\
\glft ‘however, when they went in(to the house), there were no chicken, but they also saw a pot filled with strong chicha’
\endgl
\trailingcitation{[jmx-n120429ls-x5.325-328]}
\xe

In the older recordings by Riester, there is not a single occurrence of \textit{pero}. Instead of that, the connective \textit{masa} ‘lest’ seems to be used in adversative coordination. \textit{Masa} is also used nowadays by the speakers I worked with, but exclusively in apprehensional clauses (see \sectref{sec:AprenhensionalClauses}) and warnings (see \sectref{sec:Prohibitives}).\footnote{In Riester’s recordings \textit{masa} also occurs once in an apprehensional clause.} One example of \textit{masa} in adversative coordination follows. It was probably elicited by Riester.


\ea\label{ex:masa-but}
\begingl
\glpreamble ukuine niyunu kimenukÿyae nisemaikaini mukiankaini tÿpi nikineina nubiuyae masa kuina nitupa\\
\gla ukuine ni-yunu kimenu-kÿ-yae ni-semaika-ini mukianka-ini tÿpi ni-kene-ina nÿ-ubiu-yae masa kuina ni-tupa\\
\glb yesterday 1\textsc{sg}-go woods-\textsc{clf:}bounded-\textsc{loc} 1\textsc{sg}-search.\textsc{irr}-\textsc{frust} animal-\textsc{frust} \textsc{obl} eat-\textsc{nmlz}-\textsc{irr.nv} 1\textsc{sg}-house-\textsc{loc} but \textsc{neg} 1\textsc{sg}-find.\textsc{irr}\\
\glft ‘yesterday I went to the woods and looked for animals (in vain) to eat at home, but I didn’t find any’
\endgl
\trailingcitation{[nxx-a630101g-1.62]}
\xe
\is{adversative|)}
There is only one minor type of coordination left, consecutive coordination, which is described in the following section.

\subsection{Consecutive coordination}\label{sec:ConsecutiveCoordination}\is{consecutive|(}

Consecutive clauses are similar to causal clauses (see \sectref{sec:CauseConsequence}) in that both encode a cause and a consequence. Nonetheless, while causal clauses are considered to be a subclass of adverbial, i.e. subordinate clauses, consecutive clauses are asserted and should thus be considered a case of coordination \citep[cf.][38]{Cristofaro2003}.

Consecutive clauses often include the connectives \textit{nechikue} ‘therefore’ or \textit{entonses} ‘thus’; however, these are mostly used to connect an utterance to the previous discourse rather than to connect two clauses. The connectives often present the consequence of what has been expressed by a text unit larger than one clause, as a kind of summary and signal that this discourse unit comes to an end. They usually introduce a new \isi{intonation} unit, sometimes preceded by a pause, and this would then be a hint that the clause with the connective is independent from a previous one. 

Nonetheless, in the examples presented in this section, both clauses show a relatively high degree of connection by belonging to one intonation unit, so that we can assume that the connectives are used for clause combining here. 

(\ref{ex:nechikue-sad}) is from a little tale told by Miguel about the ants. According to this tale, the ants are happy when a boy is born, because he goes on trips, and when he eats his travel supplies, little crumbs fall down and can be eaten by the ants. On the other hand, the trees are sad when a boy is born, because once he has grown up, he will fell trees to make his field.

\ea\label{ex:nechikue-sad}
\begingl
\glpreamble tipakajane echÿu yÿkÿkejane nechikueji tichÿnumi\\
\gla ti-paka-jane echÿu yÿkÿke-jane nechikue-ji ti-chÿnumi\\
\glb 3i-die.\textsc{irr}-\textsc{distr} \textsc{dem}b tree-\textsc{distr} therefore-\textsc{rprt} 3i-be.sad\\
\glft ‘the trees will die, therefore they are sad, it is said’
\endgl
\trailingcitation{[mxx-n120423lsf-X.30]}
\xe

In (\ref{ex:nechikue-ue}) Juana cites her grandmother explaining to her husband that he sees is a spirit and not a real woman and this is the reason for her behaviour.

\ea\label{ex:nechikue-ue}
\begingl
\glpreamble “es ke seunube echÿu ue nechikue tisachu tumapi”\\
\gla{es ke} seunube echÿu ue nechikue ti-sachu ti-uma-pi\\
\glb {it is the case that} woman \textsc{dem}b water.spirit therefore 3i-want 3i-take.\textsc{irr}-2\textsc{sg}\\
\glft ‘“it’s because the woman is the water spirit, that’s why she wants to take you”’
\endgl
\trailingcitation{[jxx-p151016l-2.203]}
\xe

In (\ref{ex:entonses-girl}), Miguel deduces that a little wooden toy that I had brought to do some elicitation on locative expressions must be female, since it wears a dress. The main clause and the consequence clause are uttered in one intonation unit.

\ea\label{ex:entonses-girl}
\begingl
\glpreamble chimÿu tÿnai entonses apimiyapÿimÿnÿ\\
\gla chi-mÿu ti-ÿnai entonses apimiyapÿi-mÿnÿ\\
\glb 3-clothes 3i-be.long thus girl-\textsc{dim}\\
\glft ‘its garment is long, so it is a girl’
\endgl
\trailingcitation{[mox-e110914l-1.049]}
\xe

(\ref{ex:entonses-tree}) is from the same little tale as (\ref{ex:nechikue-sad}) above and directly precedes it. Here, Miguel explains what happens when the boy has grown up.

\ea\label{ex:entonses-tree}
\begingl
\glpreamble chejepuine echÿu aitubuchepÿi tijÿkatu, tiyunaji tebitaka chisaneina entonses \\chikeuchi echÿu yubuti chisatÿku yÿkÿkejane\\
\gla chejepuine echÿu aitubuchepÿi ti-jÿka-tu ti-yuna-ji ti-ebitaka chi-sane-ina entonses chi-keuchi echÿu yubuti chi-satÿku yÿkÿke-jane\\
\glb because \textsc{dem}b boy 3i-grow.\textsc{irr}-\textsc{iam} 3i-go.\textsc{irr}-\textsc{rprt} 3i-clear.\textsc{irr} 3-field-\textsc{irr.nv} thus 3-\textsc{ins} \textsc{dem}b axe 3-cut tree-\textsc{distr}\\
\glft ‘because once the boy has grown up, he will go and clear his future field, it is said, thus with an axe he fells the trees’
\endgl
\trailingcitation{[mxx-n120423lsf-X.27-29]}
\xe

\is{consecutive|)}
\is{connective|)}
\is{juxtaposition|)}

With this example, I complete the discussion of coordination.\is{coordination|)} In the following sections, cases of subordination in clause combining and complex clause formation are presented.

