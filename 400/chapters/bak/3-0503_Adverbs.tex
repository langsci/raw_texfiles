%!TEX root = 3-P_Masterdokument.tex
%!TEX encoding = UTF-8 Unicode

\section{Adverbs}\label{sec:Adverbs}\is{adverb|(}

Adverbs typically form a large heterogeneous class. They have been described as “a ‘catch-all’ category” that lumps together “[a]ny word with semantic content (i.e., other than grammatical particles) that is not clearly a noun, a verb, or an adjective” \citep[69]{Payne1997}. Adverbs typically modify, but unlike adjectives, they do not primarily modify nouns but verbs, adjectives,\is{adjective} other adverbs, nouns and full clauses \citep[715]{Evans2000}.

The adverbs of Paunaka are not marked for person\is{person marking} or for number and they do not take the locative marker \textit{-yae}, either. They cannot occur as arguments of a verb and they are never modified\is{modification} by a nominal demonstrative. This definition excludes some locative words which are often used adverbially, but are rather analysed as nouns due to their ability to occur with the locative marker or as an argument of a verb. %and also some temporal nouns and verbs: \textit{yuti} ‘night’, \textit{tijai} ‘day’.

\hspace*{-1.3pt}Adverbs can be subdivided into subclasses depending on their semantics. There are spatial (\sectref{sec:LocativeAdverbs}), temporal and aspectual (\sectref{sec:TemporalAspectualAdverbs}), and modal (\sectref{sec:ModalAdverbs}) adverbs. The different subclasses of adverbs can take different modal, temporal, aspectual and/or other markers.



\subsection{Locative adverbs}\label{sec:LocativeAdverbs}
\is{locative|(}

\largerpage
Locative adverbs provide clues about the spatial setting of an event. They definitely comprise the demonstrative adverbs\is{adverb!demonstrative adverb|(}\is{demonstrative!demonstrative adverb|(} \textit{naka} ‘here’ and \textit{nauku} ‘there’, i.e. not here, (far) away, see (\ref{ex:here-there}). What exactly is perceived as “here” varies, as it does in other languages, and largely depends on the situational “engagement area” of a speaker, i.e. the space that she is currently focused on by paying attention to this space and acting in it \citep[cf.][89]{Enfield2003}.\footnote{It should be mentioned here that the use of \textit{naka} ‘here’ and \textit{nauku} ‘there’ sometimes runs counter to my intuition. It remains to be analysed how Paunaka speakers conceptualise their here-sphere. See \citet[251--257]{Admiraal2016} for an analysis of “here” in closely related Baure.}  In addition, there is another demonstrative, \textit{nechÿu} glossed as ‘\textsc{dem}c’, with the approximate meaning of ‘in that place, there’, i.e. in an identifiable place of mostly medial distance. \textit{Nechÿu} has nominal and adverbial properties. This has been described in more detail in \sectref{sec:DemPron}. The locative adverbs are given in \tabref{table:SpatialWords}.

\begin{table}
\caption{Locative adverbs}

\begin{tabular}{lll}
\lsptoprule
Adverb & Translation & Comment\cr
\midrule
\textit{naka} &  here & demonstrative adverb \cr
\textit{nauku} &  there (distal) & demonstrative adverb \cr
\textit{nechÿu} & there (medial) & nominal/adverbial demonstrative\cr
\lspbottomrule
 \end{tabular}

\label{table:SpatialWords}
\end{table}

%\textit{anÿke} & up & noun or adverb\cr
%\textit{apuke} & down, on the ground & noun or adverb\cr
%\textit{mÿbane(jiku)} & close, near & adverb derived from a stative verb\cr
%\textit{pÿkÿjÿe} & in the middle & noun or adverb\cr

In (\ref{ex:here-there}), Juana uses two demonstrative adverbs. She speaks about a relocation within the city of Santa Cruz after her grandson finished his premilitary service.

\ea\label{ex:here-there}
\begingl
\glpreamble tukiu nauku tanÿma bibÿsÿu naka\\
\gla tukiu nauku tanÿma bi-bÿsÿu naka\\
\glb from there now 1\textsc{pl}-come here\\
\glft ‘from there we came here now’
\endgl
\trailingcitation{[jxx-p110923l-1.182]}
\xe


The demonstrative adverbs (including \textit{nechÿu}) can be used on their own or be accompanied by a noun phrase that specifies the exact location.

In (\ref{ex:naka-1}), the location meant by \textit{naka} is shown by a gesture, thus no NP co-occurs. \textit{Naka} refers to the belly of the María C. in this case, where a sorcerer had introduced a frog.

\ea\label{ex:naka-1}
\begingl
\glpreamble ani naka peÿ\\
\gla ani naka peÿ\\
\glb look here frog\\
\glft ‘look, here [I had] a frog’
\endgl
\trailingcitation{[ump-p110815sf.300]}
\xe

In (\ref{ex:naka-2}) María C. specifies what she means by \textit{naka} with a prepositional phrase. She had moved from Santa Rita to Concepción that year and was now living next to Clara. Note that she drops the second person singular person marker on the verb, which is typical for her speech, but not necessarily accepted by the other speakers.


\ea\label{ex:naka-2}
\begingl
\glpreamble simukunÿ naka tÿpi eka nipiji\\
\gla simuku-nÿ naka tÿpi eka ni-piji\\
\glb find-1\textsc{sg} here \textsc{obl} \textsc{dem}a 1\textsc{sg}-sibling\\
\glft ‘you have found me here at my sister’s’
\endgl
\trailingcitation{[cux-120410ls.007]}
\xe

In (\ref{ex:nauku-1}), the distal adverb \textit{nauku} is used. The exact location that it refers to is unknown, at least to me, the addressee. Specific location is not important in this context, the utterance is rather about Clara being away from her house, which is included in the here-location of that situation – the yard of María C.’s house next to Clara’s house.

\ea\label{ex:nauku-1}
\begingl
\glpreamble tiyunu nauku, mhm, trabaku\\
\gla ti-yunu nauku mhm trabaku\\
\glb 3i-go there \textsc{intj} work\\
\glft ‘she went there, mhm, in order to work’
\endgl
\trailingcitation{[uxx-e120427l.089]}
\xe

In (\ref{ex:nauku-2}) on the other hand, the exact location of \textit{nauku} is of importance, since this sentence is about origin. Thus an NP\is{noun phrase} is used together with the adverb. \textit{Nauku} conveys the additional information that the exact location, Santa Rita, is in a not-here sphere for María C., not in her engagement area anymore, since she had moved to Concepción to live with her son’s family after her husband had died.


\ea\label{ex:nauku-2}
\begingl
\glpreamble nÿti tukiu nauku Santa Rita\\
\gla nÿti tukiu nauku {Santa Rita}\\
\glb 1\textsc{sg.prn} from there {Santa Rita}\\
\glft ‘I am from Santa Rita there’
\endgl
\trailingcitation{[cux-120410ls.010]}
\xe


(\ref{ex:nechÿu-2}) provides an example of \textit{nechÿu}, more examples can be found in \sectref{sec:DemPron}. It comes from a narrative by Miguel, the exact location to which \textit{nechÿu} refers directly follows; a type of basket which can be carried. Into this basket, the limbs of the main character of the story are stored, which he had just cut off and given to his son pretending they were racemes of \textit{cusi} palm fruit.

\ea\label{ex:nechÿu-2}
\begingl
\glpreamble bueno, chikamureikuji echÿu chinachÿ chijabu punachÿ chijabu nechÿu sÿkikÿyae\\
\gla bueno chi-kamureiku-ji echÿu chinachÿ chi-jabu punachÿ chi-jabu nechÿu sÿki-kÿ-yae\\
\glb well 3-accommodate-\textsc{rprt} \textsc{dem}b one 3-leg other 3-leg \textsc{dem}c basket-\textsc{clf}:bounded-\textsc{loc}\\
\glft ‘well, he accommodated his one leg and his other leg there in the basket, it is said’
\endgl
\trailingcitation{[mox-n110920l.110]}
\xe
\is{adverb!demonstrative adverb|)}\is{demonstrative!demonstrative adverb|)}


As for the rest of the spatial words,\is{word class|(} it is not clear to me at the moment whether they should rather be classified as adverbs or nouns.\is{noun|(} The antonyms \textit{nujekÿ} ‘inside’ and \textit{nekupai} ‘outside, yard’ are mainly used adverbially. But both also occur with the locative marker occasionally, which might be a hint that they are nouns rather than adverbs. On the other hand, none of them is modified by nominal demonstratives in the corpus, which might be a hint that they are adverbs rather than nouns. The same holds for the two words that describe a horizontal axis, \textit{anÿke} ‘up, above’ and \textit{apuke} ‘ground, down’, the latter one occasionally takes a locative marker when it refers to the ground, but it has not been found on \textit{anÿke} in the corpus.\is{noun|)}

Thus if we consider only (\ref{ex:apuke-3}), \textit{apuke} could well be an adverb. It describes the direction of gaze of Juana’s daughter, when she was sitting in an airplane and only saw water below her.

\ea\label{ex:apuke-3}
\begingl
\glpreamble timumuku apuke\\
\gla ti-imumuku apuke\\
\glb 3i-look ground\\
\glft ‘she looked down’
\endgl
\trailingcitation{[jxx-p110923l-1.412]}
\xe

In (\ref{ex:apuke-1}), however, which comes from Miguel, \textit{apuke} takes a locative marker. It is used as a noun that denotes the ground as our known world, which is contrasted in the story with the world inside the hill, where the spirit lives.

\ea\label{ex:apuke-1}
\begingl
\glpreamble tibÿchÿupupunukuji naka apukeyae\\
\gla ti-bÿchÿu-pupunuku-ji naka apuke-yae\\
\glb 3i-leave-\textsc{reg}-\textsc{rprt} here ground-\textsc{loc}\\
\glft ‘he left to the ground here again, it is said (i.e. this world after having been inside the hill with the spirit)’
\endgl
\trailingcitation{[mxx-n151017l-1.58]}
\xe

\textit{Apuke} can even be the object of a verb, as in (\ref{ex:apuke-2}), which comes from Juana and is from her account about her grandparents’ journey to Moxos. On their way back home, they are surprised by heavy rainfalls and have to pass an arroyo that has filled with water.

\ea\label{ex:apuke-2}
\begingl
\glpreamble kuinaji chitupa apuke\\
\gla kuina-ji chi-tupa apuke\\
\glb \textsc{neg}-\textsc{rprt} 3-find.\textsc{irr} ground\\
\glft ‘she didn’t reach the ground (of the river with her feet), it is said’
\endgl
\trailingcitation{[jxx-p151016l-2.144]}
\xe

%\ea\label{ex:apuke-11}
%\begingl
%\glpreamble tiyupunutu pitikiri tukiu apuke\\
%\gla ti-yupunu-tu pitikiri tukiu apuke\\
%\glb 3i-come.out-\textsc{iam} cicada.sp from ground\\
%\glft ‘the cicada emerges from the ground’\\
%\endgl
%\trailingcitation{[rxx-e181021les.119]}
%\xe

%naka+ -tu, -ni, -ja{'a), -uku, -upunu, -jiku, -yÿchi, -kena, -pupunu, -mÿnÿ, -bane, kuina nakabu
%nauku + -uku, -bane, -jiku, -tu, -bÿti, -ni, -ja'a

%naka (795 hits): here, close, what is conceived as "here" varies like in English (depending on the contrast, what is perceived as not here)
%nechÿu (103 hits): something, which is easily identifiable because it has just been talked about or it is specified just after or it can be seen
%nauku (573 hits): more distant than nechÿu, including emotional/temporal/discourse distance

Thus, two kinds of overlaps concerning spatial words can be found. First, there is an overlap between nominal and adverbial function of the demonstrative adverb \textit{nechÿu}, and second, certain words with spatial semantics do not neatly fit into either of the categories of noun and adverb.
\is{word class|)}
\is{locative|)}



\subsection{Temporal and aspectual adverbs}\label{sec:TemporalAspectualAdverbs}\is{temporal/aspectual|(}

Temporal and aspectual adverbs comprise those words that provide information about the temporal setting of an event, its internal constituency or its relation to another event.

\tabref{table:TemporalAdverbs} lists the temporal and aspectual adverbs. In addition, there are a few more words referring to times of the day that are probably rather nouns than adverbs, although they are used adverbially most of the time. Among them are \textit{mane} ‘morning’, \textit{tose} ‘noon, midday’, \textit{kupei} ‘afternoon’ (also used with the meaning ‘late’ as its equivalent \textit{tarde} in Spanish), \textit{yuti} ‘night’. \textit{Mane}, \textit{kupei} and \textit{yuti} can be modified by a nominal demonstrative, but this happens rarely. \textit{Tose} occurs with the locative marker \textit{-yae} once in the corpus. The word for ‘day‘, \textit{tijai}, is a stative verb, and so are the words derived\is{derivation} from it, e.g. \textit{tijaikenekÿu} ‘dawn’, \textit{tajaitu} ‘tomorrow’ or \textit{tajaibÿti} with the idiomatic meaning ‘see you tomorrow/another day'.

\begin{table}
\caption{Temporal and aspectual adverbs}

\begin{tabularx}{\textwidth}{lQ}
\lsptoprule
Adverb & Translation \cr
\midrule
\textit{abane} & finally \cr
\textit{janeka} & never \cr
\textit{maneiku} & soon \cr
\textit{metu} & already \cr
\textit{nÿmayu} & just\cr
\textit{tanÿma} &  now  \cr
\textit{(u)chuine} & just now, recently, a few moments to hours ago, same day \cr
\textit{ukuine} & yesterday, a few days ago \cr
\textit{(u)kuinebu} & some day in the intermediate past, a few days, weeks or months ago\cr
\textit{upunuku} & again \cr
\lspbottomrule
 \end{tabularx}

\label{table:TemporalAdverbs}
\end{table}

%\textit{uchu} &  some day in an uncertain future \cr

%tanÿma: 265 hits = jetzt now
%uchuine: 26 hits = gerade eben just recently, a few moments ago
% uchu: some day, uncertain future


%janeka chikurabaka = que nadie lo quebre??, jmx-d110918ls-1.078
%mqx-p110826l.556

The\is{iamitive|(} adverb \textit{metu} ‘already’ is used very frequently and probably the source for the iamitive marker \textit{-tu} (see \sectref{sec:Iamitive}). Just like the iamitive marker,\is{iamitive|)} \textit{metu} can be used both when an event is completed, (\ref{ex:ADVmetu-1}), and when it is ongoing at the moment in question, (\ref{ex:ADVmetu-2}), with or without a connotation of earliness. Thus, depending on the specific context and the aktionsart of the predicate, ‘now’ is sometimes a more appropriate translation than ‘already’. \textit{Metu} always precedes the predicate.

(\ref{ex:ADVmetu-1}) was elicited from María S. and refers to an imagined pot.

\ea\label{ex:ADVmetu-1}
\begingl
\glpreamble nimutu metu terabajikutu\\
\gla ni-imu-tu metu ti-rabajiku-tu\\
\glb 1\textsc{sg}-see-\textsc{iam} already 3i-break-\textsc{iam}\\
\glft ‘when I saw it, it was already broken’
\endgl
\trailingcitation{[rxx-e181021les.222]}
\xe

(\ref{ex:ADVmetu-2}) is a statement by Juana about herself.

\ea\label{ex:ADVmetu-2}
\begingl
\glpreamble nÿti metu juberÿpunÿtu\\
\gla nÿti metu juberÿpu-nÿ-tu\\
\glb 1\textsc{sg.prn} already old.woman-1\textsc{sg}-\textsc{iam}\\
\glft ‘I am old already/now’
\endgl
\trailingcitation{[jxx-p110923l-1.205]}
\xe

Equally frequent is \textit{tanÿma} ‘now’. This adverb might have originated from a stative verb, the initial \textit{t} being a fixed third person marker, and \textit{-nÿma} an \isi{associated motion} marker (see \sectref{sec:SubsequentMotion}). However, the word is totally lexicalised. When a morpheme is added, especially \textit{-yu}, the first syllable is usually dropped today, resulting in a form \textit{nÿmayu}. In the recordings by Riester, we find the full form \textit{tanÿmayu}. \textit{Nÿmayu} is punctual and best translated as ‘just’.

(\ref{ex:ADVnow}) is a statement by María S. which refers to me having changed the position of my chair a bit because the sun was shining on me. When I was sitting in the shade again, she said:

\ea\label{ex:ADVnow}
\begingl
\glpreamble michatu tanÿma\\
\gla micha-tu tanÿma\\
\glb good-\textsc{iam} now\\
\glft ‘now it is good’
\endgl
\trailingcitation{[rxx-e181024l.019]}
\xe

(\ref{ex:ADVjust}) comes from Juana and refers to the work the people of Santa Rita did for a lady from Germany in exchange for the construction of the reservoir.

\ea\label{ex:ADVjust}
\begingl
\glpreamble juu nÿmayu tijaipai tiyununubetu techikanube\\
\gla juu nÿmayu ti-jai-pai ti-yunu-nube-tu ti-echika-nube\\
\glb \textsc{intj} just 3i-be.light-\textsc{clf:}ground 3i-go-\textsc{pl}-\textsc{iam} 3i-fell.tree.\textsc{irr}-\textsc{pl}\\
\glft ‘oh, when it just began to dawn (lit.: there was just light on the ground), they already went to fell trees’
\endgl
\trailingcitation{[jxx-p120515l-2.179]}
\xe

The adverb \textit{(u)chuine} is used if the point in time is not considered as “now” because the event expressed by the verb is completed, but it is still perceived as something in the very recent past, as in (\ref{ex:uchuine}). The word is only used to refer to points in time of the current day.\is{past reference} It may be derived\is{derivation} from the \isi{uncertain future} marker \textit{uchu} ‘some day (perhaps, in an uncertain future)’ (see \sectref{sec:UncertainFuture}), at the same time it resembles the adverb \textit{ukuine} ‘yesterday, some day in the recent past’, which also refers to a recent past, but less recent than the current day, see (\ref{ex:ukuine}). The translation with ‘yesterday’ is not very precise, since \textit{ukuine} can also refer to the day before yesterday or another day in the recent past. However, most of the times it is used to refer to the day preceding the current day.

(\ref{ex:uchuine}) is a statement by Juana about her daughter including \textit{uchuine}.

\ea\label{ex:uchuine}
\begingl
\glpreamble uchuine chÿnachÿtu ora tichujiku tukiu nauku\\
\gla uchuine chÿnachÿ-tu ora ti-chujiku tukiu nauku\\
\glb just.now one-\textsc{iam} hour 3i-speak from there\\
\glft ‘just an hour ago she talked [with me] from there (on telephone)’
\endgl
\trailingcitation{[jxx-p120430l-1.335-336]}
\xe

(\ref{ex:ukuine}) contains \textit{ukuine} and also comes from Juana, who tells me about the source of her knowledge about her brother’s health.

\ea\label{ex:ukuine}
\begingl
\glpreamble chikuetea chijinepÿi ukuine\\
\gla chi-kuetea chi-jinepÿi ukuine\\
\glb 3-tell 3-daughter yesterday\\
\glft ‘his daughter told it [to me] yesterday (or a few days ago)’
\endgl
\trailingcitation{[jxx-e150925l-1.124]}
\xe

In addition, \textit{(u)kuinebu} refers to points in time in an intermediate past\is{past reference} which may be some days ago, but also some weeks or months ago, as in (\ref{ex:ukuinebu}), which was recorded in April and the event that Juana is referring to happened some months ago: some friends of the daughter who live in Argentina came to visit her over New Year and brought some coffee as a present.

\ea\label{ex:ukuinebu}
\begingl
\glpreamble nauku Argentina tupununube uikuinebu\\
\gla nauku Argentina ti-upunu-nube uikuinebu\\
\glb there Argentina 3i-bring-\textsc{pl} some.time.ago\\
\glft ‘they brought it  from Argentina some time ago’
\endgl
\trailingcitation{[jxx-e120430l-4.28]}
\xe


The adverbs \textit{tanÿma} and \textit{(u)chuine} can be intensified by a marker \textit{-paiku/a}, which narrows the possible time frame.\footnote{This marker also shows up on verbs,\is{verb} but very infrequently. I suppose it fulfils the same function there.} (\ref{ex:right-now}) comes from Miguel who told us a story, while we sat to eat and relax a bit on our visit to \isi{Altavista}.

\ea\label{ex:right-now}
\begingl
\glpreamble ... tiyuna paseana nenabi biti tanÿmapaiku\\
\gla ti-yuna pasea-ina nena-bi biti tanÿma-paiku\\
\glb 3i-go.\textsc{irr} stroll-\textsc{irr.nv} like-1\textsc{pl} 1\textsc{pl.prn} now-\textsc{punct}\\
\glft ‘... he will go on a jaunt like we are doing right now’
\endgl
\trailingcitation{[mxx-n120423lsf-X.15]}
\xe

%tanÿmapaiku nÿpikeika, kuinakuÿ nÿbuka = ahorita estoy atiendo, todavía no he terminado, rxx-e181022le


In (\ref{ex:chuinepaiku}), María S. asks me about my arrival in Santa Rita that day.

\ea\label{ex:chuinepaiku}
\begingl
\glpreamble ¿chuinepaiku pitupunubu?\\
\gla chuine-paiku pi-tupunubu\\
\glb just.now-\textsc{punct} 2\textsc{sg}-arrive\\
\glft ‘you have just arrived?’
\endgl
\trailingcitation{[rxx-e120511l.005]}
\xe

The adverb \textit{maneiku} ‘soon’ is probably derived\is{derivation} from \textit{mane} ‘morning’. It is used very infrequently. One example is given in (\ref{ex:maneiku}), where Juana makes a statement about my departure to Germany being close in contrast to Swintha’s, who had plans to stay a bit longer.

\ea\label{ex:maneiku}
\begingl
\glpreamble asi max temprano piyunupuna, maneiku piyuna\\
\gla asi max temprano pi-yunupuna maneiku pi-yuna\\
\glb so more early 2\textsc{sg}-go.back.\textsc{irr} soon 2\textsc{sg}-go.\textsc{irr}\\
\glft ‘so you go back earlier, you go soon’
\endgl
\trailingcitation{[jxx-p120430l-2.635]}
\xe

\textit{Janeka} ‘never’ seldom shows up in the corpus, but one example from Juana is given below. It is not clear to me why she uses realis RS here. She speaks about the pot she was going to make with the clay she had just collected.

\ea\label{ex:janeka}
\begingl
\glpreamble janeka chikurabaku\\
\gla janeka chi-kurabaku\\
\glb never 3-break\\
\glft ‘nobody ever breaks it’ (lit.: ‘never [a non-specified agent] breaks it’)
\endgl
\trailingcitation{[jmx-d110918ls-1.078]}
\xe

%semantics of janeka not totally clear (n has: ti(bi)janeka)

Finally, there is \textit{abane}, which means ‘finally’. It expresses that something that has been expected is completed after a time span that is considered (too) long, as in (\ref{ex:abane-2-2}), where Juana’s daughter finally calls her boss to accompany her to the airport and speak to the people in charge to put in a good word for her sister who had arrived to Spain without a valid visa.

\ea\label{ex:abane-2-2}
\begingl
\glpreamble abane chichujiku te tiyununubetu\\
\gla abane chi-chujiku te ti-yunu-nube-tu\\
\glb finally 3-speak \textsc{seq} 3i-go-\textsc{pl}-\textsc{iam}\\
\glft ‘finally she spoke to him and they went (to the airport)’
\endgl
\trailingcitation{[jxx-p110923l-1.351]}
\xe
\is{temporal/aspectual|)}


\subsection{Modal adverbs}\label{sec:ModalAdverbs}\is{modal|(}

Modal meanings are seldom expressed by adverbs, since most information of this kind is marked on the predicate or more rarely other constituents of the clause (see \sectref{sec:Modality_Evidentiality}). However, a few modal adverbs are used in Paunaka and they are given in \tabref{table:ModalAdverbs}. 

\begin{table}
\caption{Modal adverbs}

\begin{tabular}{lll}
\lsptoprule
Adverb & Translation & Comment\cr
\midrule
\textit{nakayenetu} & almost & proximate\cr
\textit{nakayenetÿini} & almost & avertive\cr
\textit{repente} & maybe & loan from Spanish \cr
\lspbottomrule
 \end{tabular}

\label{table:ModalAdverbs}
\end{table}


\textit{Repente} is a loan\is{borrowing} from Spanish \textit{de repente} ‘maybe’ (also ‘suddenly’, which is also another meaning of the Paunaka word, but less frequently). It is more or less synonym with the \isi{uncertainty} marker \textit{(-)kena}, which expresses uncertainty (see \sectref{sec:ModalityUncertainty}). \textit{Kena} is phonologically bound to another word most of the time, and it can also attach to \textit{repente}. \textit{Repente} has some variants \textit{pente} and \textit{depente}, but they are less frequent than \textit{repente}. (\ref{ex:repente}) offers one example of the adverb, which comes from Miguel when talking with María C. about her husband, who was ill.

\ea\label{ex:repente}
\begingl
\glpreamble pero repentekena michaupunu punachina semana\\
\gla pero repente-kena micha-upunu punachÿ-ina semana\\
\glb but maybe-\textsc{uncert} good-\textsc{reg} other-\textsc{irr} week\\
\glft ‘but maybe he has recovered (lit.: is good again) next week’
\endgl
\trailingcitation{[mux-c110810l.037]}
\xe


The\is{avertive|(} adverb \textit{nakayenetu/nakayenetÿini} ‘almost’ is complex,\is{derivation} at least as its ending is concerned. It contains either the \isi{iamitive} marker \textit{-tu} or the avertive marker \textit{-tÿini} (see \sectref{sec:Iamitive} for iamitive aspect and \sectref{sec:FRUST-Avertive} for avertive modality). As for the rest of the word, \textit{naka} could derive from the demonstrative adverb (see \sectref{sec:LocativeAdverbs} above), and \textit{yene} could relate to the \isi{deductive} marker \textit{-yenu} (\sectref{sec:Epistemic_Mod}), but this is speculative. In any way, the difference between both variants of the adverb is that \textit{nakayenetu} is proximate, i.e. the event is temporally close and realisable, while \textit{nakayenetÿini} is avertive, i.e. the event was imminent but did not occur.


(\ref{ex:almost-prox}) is one example of the proximate use of the adverb. Juana answers my question here, when they were going to eat.

\ea\label{ex:almost-prox}
\begingl
\glpreamble nakayenetu toseina binika\\
\gla nakayenetu tuse-ina bi-nika\\
\glb almost noon-\textsc{irr.nv} 1\textsc{pl}-eat.\textsc{irr}\\
\glft ‘we will eat just before noon’
\endgl
\trailingcitation{[jxx-p110923l-2.099]}
\xe

(\ref{ex:almost-2}) gives one example of the avertive version of the adverb. Juana was searching for the Paunaka name for ‘deer’, there must have been one and it must have sounded similar to \textit{yÿnÿ} ‘jabiru’ (a bird: \textit{Jabiru mycteria}).

\ea\label{ex:almost-2}
\begingl
\glpreamble nakayenetÿini chija eka nikecha yÿnÿ\\
\gla nakayenetÿini chi-ija eka ni-kecha yÿnÿ\\
\glb almost 3-name \textsc{dem}a 1\textsc{sg}-say.\textsc{irr} jabiru\\
\glft ‘I would have almost said \textit{yÿnÿ} (i.e. jabiru) is its name’
\endgl
\trailingcitation{[jxx-a120516l-a.240-241]}
\xe
\is{avertive|)}
\is{modal|)}
\is{adverb|)}



The following section describes prepositions.



