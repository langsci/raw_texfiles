%!TEX root = 3-P_Masterdokument.tex
%!TEX encoding = UTF-8 Unicode
\section{Prepositions}\label{sec:Adpositions}
\is{preposition|(}
\is{oblique|(}

There are four words that can be analysed as prepositions, they are given in \tabref{table:Adpositions}. These prepositions are used to mark constituents as obliques (see \sectref{sec:DeclClausesOBL}). They encode an instrument, a causer,\is{instrument/cause} \isi{source}, benefactive,\is{beneficiary}\footnote{In the index at the end of this book, the semantic role “benefactive” is encoded as “beneficiary”, because the term “benefactive” is reserved for a suffix that changes the valency of a verb.} aim or result\is{aim/result}, \isi{recipient} or a \isi{comitative} relation.  

\begin{table}
\caption{Prepositions}

\begin{tabularx}{\textwidth}{llQ}
\lsptoprule
Preposition & Translation & Comment \cr
\midrule
\textit{-aj(i)echubu} & with & always person-marked, least grammaticalised preposition \cr
\textit{-keuchi} & by, with & usually with person marker, but possible without\cr
\textit{tukiu} & from & never person-marked \cr
\textit{(-)tÿpi }& \textsc{obl}, for & possible with and without person marker, also used for clause-linking \cr
\lspbottomrule
\end{tabularx}

\label{table:Adpositions}
\end{table}

Typical examples for the prepositions are given in (\getfullref{ex:new23-preps.1}) to (\getfullref{ex:new23-preps.4}).

\ea\label{ex:new23-preps}
  \ea\label{ex:new23-preps.1}
\begingl
\glpreamble chajechubu nÿenu\\
\gla chÿ-ajechubu nÿ-enu\\
\glb 3-\textsc{com} 1\textsc{sg}-mother\\
\glft ‘with my mother’
\endgl
\trailingcitation{[rxx-p181101l-2.044]}
  \ex\label{ex:new23-preps.2}
\begingl
\glpreamble chikeuchi yÿkÿke\\
\gla chi-keuchi yÿkÿke\\
\glb 3-\textsc{ins} stick\\
\glft ‘with a stick’
\endgl
\trailingcitation{[jxx-p120430l-1.073]}
  \ex\label{ex:new23-preps.3}
\begingl
\glpreamble tukiu Naranjito\\
\gla tukiu Naranjito\\
\glb from Naranjito\\
\glft ‘from Naranjito’
\endgl
\trailingcitation{[mxx-p110825l.183]}
  \ex\label{ex:new23-preps.4}
\begingl
\glpreamble chitÿpi echÿu patron\\
\gla chi-tÿpi echÿu patron\\
\glb 3-\textsc{obl} \textsc{dem}b patrón\\
\glft ‘for the \textit{patrón}’
\endgl
\trailingcitation{[mxx-p110825l.025]}
\z
\xe

As illustrated by these examples, prepositions are always placed before the NP\is{noun phrase|(} they are related to. Nonetheless, not all of them necessarily occur together with an NP: if they take a first or second person marker,\is{person marking|(} no NP shows up with the preposition, and if they take a third person marker, an NP is optional. The \isi{source} preposition \textit{tukiu} is never marked for person. Consequently, it always occurs together with a noun or an adverb. The \isi{comitative} preposition \textit{-aj(i)echubu} is always marked for person. The \isi{instrument/cause} preposition \textit{-keuchi} is normally person-marked and only rarely drops a third person index. As regards \textit{(-)tÿpi}, first and second person markers are obligatory, while the third person marker usually alternates with an NP, although sometimes there is both a third person index on the preposition and an NP\is{noun phrase|)} following it, especially if the referent is human.\is{person marking|)}\is{animacy}

In addition to the prepositions presented in this section, Paunaka makes use of a few words that express specific spatial relations to a referent. This is often achieved by using adpositions in other languages, but in Paunaka these words are better analysed as relational nouns,\is{relational noun|(} since they obligatorily take the locative marker to specify that a spatial relation is expressed (see \sectref{sec:Locative}). If a noun is juxtaposed to the relational noun, it can be regarded its \isi{possessor}. The possessor noun does not take the locative marker. In contrast to relational nouns, the locative marker is never attached to the prepositions, not even to \textit{tukiu}, which is used to mark \isi{source} expressions, i.e. a spatial relation. However, the noun that \textit{tukiu} relates to can take the \is{locative marker}. This is illustrated in (\ref{ex:adp-1}) and (\ref{ex:adp-2}). In (\ref{ex:adp-1}) the locative marker is attached to the noun following the preposition and the preposition itself is unmarked. In (\ref{ex:adp-2}), the locative marker is attached to the relational noun and its \isi{possessor} is unmarked.\is{relational noun|)}

(\ref{ex:adp-1}) was produced by Miguel in talking with Juan C. about their past. 

\ea\label{ex:adp-1}
\begingl
\glpreamble ja bibÿsÿupunu tukiu Turuxhiyae sinkuenta i dos\\
\gla ja bi-bÿsÿupunu tukiu Turuxhi-yae {sinkuenta i dos}\\
\glb \textsc{afm} 1\textsc{pl}-come from Altavista-\textsc{loc} {fifty-two}\\
\glft ‘well, we came from \isi{Altavista} in 52’
\endgl
\trailingcitation{[mqx-p110826l.055]}
\xe

(\ref{ex:adp-2}) also comes from Miguel and refers to the picture in the \isi{frog story} in which the dog stands on a log (and the boy leans over that log).

\ea\label{ex:adp-2}
\begingl
\glpreamble tijipuikutuji chÿineyae echÿu yÿkÿke\\
\gla ti-jipuiku-tu-ji chÿ-ine-yae echÿu yÿkÿke\\
\glb 3i-jump-\textsc{iam}-\textsc{rprt} 3-top-\textsc{loc} \textsc{dem}b tree\\
\glft ‘it has jumped on top of the log, it is said’
\endgl
\trailingcitation{[mtx-a110906l.207]}
\xe


The degree of \isi{grammaticalisation} of the prepositions differs with the verbal origin being still recognisable in the \isi{instrument/cause} preposition \textit{-keuchi} and the \isi{comitative} \textit{-aj(i)echubu}, while the sources of the prepositions \textit{(-)tÿpi} and \textit{tukiu} are not clear. This will be explained in more detail in the sections that follow. The source preposition \textit{tukiu} is described in \sectref{sec:adp-tukiu}, the many different functions of the general oblique marker \textit{(-)tÿpi} are illustrated in \sectref{sec:adp-tÿpi}, the instrument and cause preposition \textit{-keuchi} is described in \sectref{sec:adp-keuchi}, and finally \sectref{sec:adp-ajechubu} is dedicated to the least grammaticalised adpostion, the comitative marker \textit{-aj(i)echubu}.

\subsection{The source preposition}\label{sec:adp-tukiu}
\is{source|(}

The preposition \textit{tukiu} introduces source expressions and thus occurs a lot in sentences describing motion events. It never takes a person marker.\is{person marking} 

(\ref{ex:new23adp-tukiu}) was provided by María S. and referred to a present I gave her.

\ea\label{ex:new23adp-tukiu}
\begingl
\glpreamble tukiu Alemania pupunu\\
\gla tukiu Alemania pi-upunu\\
\glb from Germany 2\textsc{sg}-bring\\
\glft ‘you brought it from Germany’
\endgl
\trailingcitation{[rxx-e120511l.016]}
\xe

The initial \textit{t} of the preposition could be a hint that it derives\is{grammaticalisation} from a \isi{verb} inflected with the third person marker \textit{ti-}, \textit{u} we also find in the defective verb \textit{-ubu} ‘be, live’.\footnote{This verb probably contains the middle marker\is{middle voice} \textit{-bu}, thus its root is \textit{u}.} The sequence \textit{kiu} resembles what we find in deranked verbs, a \isi{thematic suffix} + subordinate marking, but then again, deranked verbs rather take the third person marker \textit{chÿ-} instead of \textit{ti-} (see \sectref{sec:Subordination-i}).\footnote{Note that Rose (2021, p.c.) proposes a link to the \isi{Mojeño Trinitario} verb \textit{os’o} ‘come from, be from’. This could possibly also explain the sequence \textit{kiu}. There is no related verb in Paunaka synchronically.}

\textit{Tukiu} is usually preposed to a noun or a locative adverb \is{adverb|(}, especially \textit{naka} ‘here’ and \textit{nauku} ‘there’. Locative adverb and noun also often combine to yield a more precise source expression. If there is a noun in the source expression, it can take the \isi{locative marker} \textit{-yae}, but this is not always the case. Adverbs never take this marker.\is{adverb|)}

In (\ref{ex:adp-tukiu-1}), the toponym\is{toponym|(} noun in the source expression bears the locative marker. The example is from María C.’s account about her life. She was once very ill and went to San Pedrito crawling, where they extracted two frogs from her belly. Note that she does not use the verb for ‘crawl’ in this sentence, but resorts to a collocation with a Spanish origin, \textit{kuatrupie} from \textit{a cuatro pies} ‘on hands and knees’, which is used adverbially.

\ea\label{ex:adp-tukiu-1}
\begingl
\glpreamble kuatrupie niyunu San Pedrito tukiu Arubeituyae\\
\gla {kuatrupie} ni-yunu {San Pedrito} tukiu Arubeitu-yae\\
\glb {on.hands.and.knees} 1\textsc{sg}-go {San Pedrito} from Arubeito-\textsc{loc}\\
\glft ‘on hands and knees I went to San Pedrito from Arubeito’
\endgl
\trailingcitation{[ump-p110815sf.303]}
\xe

(\ref{ex:adp-tukiu-3}) has a source expression in which the toponym (i.e. a noun) occurs without a locative marker. It was produced by Juana while sitting at her house in Concepción. She informed me about the plans of the inhabitants of Santa Rita to participate in the public appearance of Evo Morales in Concepción in 2015.

\ea\label{ex:adp-tukiu-3}
\begingl
\glpreamble aja las ocho naka kapununubeina titupunapunube tukiu Santa Rita\\
\gla aja {las ocho} naka kapunu-nube-ina ti-tupunapu-nube tukiu {Santa Rita}\\
\glb \textsc{intj} {at eight o’clock} here come-\textsc{pl}-\textsc{irr.nv} 3i-arrive.\textsc{irr}-\textsc{pl} from {Santa Rita}\\
\glft ‘yes, at eight o’clock they will come here, they will arrive from Santa Rita’
\endgl
\trailingcitation{[jxx-p150920l.078]}
\xe
\is{toponym}

In (\ref{ex:adp-tukiu-7}), the source is expressed with the adverb \textit{nauku} ‘there’. This sentence stems from Juana. I had been waiting for her at the zoo, when she had just arrived home from elsewhere.

\ea\label{ex:adp-tukiu-7}
\begingl
\glpreamble las dies nibÿsÿu tukiu nauku, nubiuyae\\
\gla {las dies} ni-bÿsÿu tukiu nauku nÿ-ubiu-yae\\
\glb {at ten o’clock} 1\textsc{sg}-come from there 1\textsc{sg}-house-\textsc{loc}\\
\glft ‘at ten I came home from there, to my house’
\endgl
\trailingcitation{[jxx-p110923l-2.043]}
\xe

In the source expression of (\ref{ex:adp-tukiu-4}), we have both the adverb \textit{nauku} and a complex locative expression containing the relational noun \textit{-chuku} ‘side’. The sentence comes from María S., who told me about her life. The family had once lived more remote from where Santa Rita is located today. Only José remained in this remote location, the other siblings moved away. 

\ea\label{ex:adp-tukiu-4}
\begingl
\glpreamble bijechikutu tukiu nauku chukuyae Kose\\
\gla bi-jechiku-tu tukiu nauku chi-chuku-yae Kose\\
\glb 1\textsc{pl}-move-\textsc{iam} from there 3-side-\textsc{loc} José\\
\glft ‘we moved from there close to José’s’
\endgl
\trailingcitation{[rxx-p181101l-2.257]}
\xe

In (\ref{ex:new23-adp-tukiu2}), Juana combines \textit{naka} with a toponym. She compares the coffee from Argentina she received as a present to the coffee from Bolivia here.

\ea\label{ex:new23-adp-tukiu2}
\begingl
\glpreamble max michaniki eka tukiu naka Bolivia\\
\gla max michaniki eka tukiu naka Bolivia\\
\glb more delicious \textsc{dem}a from here Bolivia\\
\glft ‘the one from here, from Bolivia is better’
\endgl
\trailingcitation{[jxx-e120430l-4.37]}
\xe

The source preposition is also often used to encode sources of non-motion events. It might be the case that this is due to influence of Spanish, which would also make use of the source preposition \textit{de} to encode these cases. Two examples of this use are given below, both were produced by Juana.

In the first of these examples, the source is a source of knowledge, not of motion along a path. It comes from Juana’s account about her encounter with two old Paunaka ladies, who first did not recognise that she understood them talking in Paunaka. In (\ref{ex:adp-tukiu-5}), Juana cites what one of the ladies said to her after Juana had introduced herself to them.

\ea\label{ex:adp-tukiu-5}
\begingl
\glpreamble “ja'a nichupuikubane pia tukiu Turuxhiyae”\\
\gla ja'a ni-chupuiku-bane pi-a tukiu Turuxhi-yae\\
\glb \textsc{afm} 1\textsc{sg}-know-\textsc{rem} 2\textsc{sg}-father from Altavista-\textsc{loc} \\
\glft ‘“yes, I know your father from \isi{Altavista} in the old days”’
\endgl
\trailingcitation{[jxx-p120515l-1.134]}
\xe

(\ref{ex:adp-tukiu-6}) is from the story about the fox and the jaguar. The jaguar has already drowned in a pond at this point of the story. Some months later, approximately in August, the fox comes back to speak with the skeleton of the jaguar. The pond has fallen dry by that time.

\ea\label{ex:adp-tukiu-6}
\begingl
\glpreamble tibukubutu echÿu ÿne tukiu nechÿu kurichiyae\\
\gla ti-buku-bu-tu echÿu ÿne tukiu nechÿu kurichi-yae\\
\glb 3i-finish-\textsc{mid}-\textsc{iam} \textsc{dem}b water from \textsc{dem}c pond-\textsc{loc}\\
\glft ‘the water had vanished (lit.: finished) from the pond’
\endgl
\trailingcitation{[jmx-n120429ls-x5.283]}
\xe

\is{source|)}

The following section is dedicated to the general oblique marker \textit{(-)tÿpi}, which is the most frequent preposition.

\subsection{The oblique preposition}\label{sec:adp-tÿpi}
\is{general oblique|(}

The preposition \textit{(-)tÿpi} is the one found with the largest array of different functions and is thus simply glossed ‘\textsc{obl}’ for ‘oblique’. It is often used with constituents that have the semantic role of a benefactive,\is{beneficiary} but also with aims or results,\is{aim/result} recipients,\is{recipient} temporal expressions, and sometimes also with goals. The preposition is also used to introduce purpose clauses. In general, \textit{(-)tÿpi} can be found in most of the contexts in which the speakers would use the prepositions \textit{para} or \textit{por} (both among other things mean ‘for’), when speaking Spanish.

As for its origin, there is a stative verb \textit{-tÿpina} ‘be straight, be correct’, but the similarity to the preposition may be coincidental. If the preposition is in some way related to the verb, the semantic connection has become opaque. 

In what follows, I give examples for the many different uses of \textit{(-)tÿpi}. Among the most frequent are the expression of benefactives\is{beneficiary|(} and results or aims.\is{aim/result} To illustrate these uses, compare the following two examples, in which \textit{(-)tÿpi} is used once to mark a benefactive constituent in (\ref{ex:adp-tÿpi-2}) and once to mark a result or aim in (\ref{ex:adp-tÿpi-3}). They were both produced by Juana in an elicitation session, but (\ref{ex:adp-tÿpi-2}) was requested as a translation of a Spanish sentence and (\ref{ex:adp-tÿpi-3}) was added by Juana herself.

\ea\label{ex:adp-tÿpi-2}
\begingl
\glpreamble nupunu eka merÿ pitÿpi\\
\gla nÿ-upunu eka merÿ pi-tÿpi\\
\glb 1\textsc{sg}-bring \textsc{dem}a plantain 2\textsc{sg}-\textsc{obl}\\
\glft ‘I brought these plantains for you’
\endgl
\trailingcitation{[jxx-e191021e-2]}
\xe

\ea\label{ex:adp-tÿpi-3}
\begingl
\glpreamble bupunu echÿu merÿ tÿpi masakujina\\
\gla bi-upunu echÿu merÿ tÿpi masaku-ji-ina\\
\glb 1\textsc{pl}-bring \textsc{dem}b plantain \textsc{obl} masaco-\textsc{clf:}soft.mass-\textsc{irr.nv}\\
\glft ‘we brought these plantains for \textit{masaco} (a dish made with plantains and cheese)’
\endgl
\trailingcitation{[jxx-e191021e-2]}
\xe

An example with two benefactives introduced by \textit{(-)tÿpi} is (\ref{ex:adp-tÿpi-11}), in which Miguel cites a monk who came to Santa Rita and helped the people there build a school for the children.

\ea\label{ex:adp-tÿpi-11}
\begingl
\glpreamble “nana ubiae nitÿpi xhikuera naka etÿpi”, tikechu\\
\gla nÿ-ana ubiae ni-tÿpi xhikuera naka e-tÿpi ti-kechu\\
\glb 1\textsc{sg}-make.\textsc{irr} house 1\textsc{sg}-\textsc{obl} school here 2\textsc{pl}-\textsc{obl} 3i-say\\
\glft ‘“I can make a house for myself and a school for you”, he said’
\endgl
\trailingcitation{[mxx-p110825l.110-111]}
\xe

The following example is an excerpt from a conversation between Miguel and Juan C. What we see here is that \textit{(-)tÿpi} can mark an NP as a benefactive oblique as in (\getfullref{ex:adp-tÿpi-1.1}), but it can also be used as a \isi{connective} that introduces purpose \is{purpose|(} clauses as in (\getfullref{ex:adp-tÿpi-1.2}). More examples for purpose clauses with \textit{(-)tÿpi} are found in \sectref{sec:PurposeClauses} and \sectref{sec:EmbeddedAC_adp}.
 Miguel had just told Juan C. that he thought about planting sweet potatoes and apparently Juan C. thought it was useless to plant them, because they are eaten by armadillos. He expressed this by stating that the sweet potatoes are meant for the armadillo, and Miguel takes up this joke. Both men were laughing.

\ea\label{ex:adp-tÿpi-1}
  \ea\label{ex:adp-tÿpi-1.1}
\begingl
\glpreamble \textup{q:} aa chibu tÿpi pÿrÿsÿsÿ\\
\gla aa chibu tÿpi pÿrÿsÿsÿ\\
\glb \textsc{intj} 3\textsc{top.prn} \textsc{obl} armadillo\\
\glft ‘ah, that is for the armadillo’
\endgl
  \ex\label{ex:adp-tÿpi-1.2}
\begingl
\glpreamble \textup{m:} tÿpi chinika pÿrÿsÿsÿ\\
\gla tÿpi chi-nika pÿrÿsÿsÿ\\
\glb \textsc{obl} 3-eat.\textsc{irr} armadillo\\
\glft ‘so that the armadillo can eat it’
\endgl
\trailingcitation{[mqx-p110826l.578-580]}
\z
\xe
\is{beneficiary|)}
\is{purpose|)}

\largerpage[-2]
The following two examples illustrate the use of \textit{(-)tÿpi} to encode results or aims of an action.\is{aim/result|(} (\ref{ex:adp-tÿpi-12}) was elicited from Juana in order to obtain more knowledge about the expression of causative relations, and the oblique phrase \textit{tÿpi upichai}, which encodes what the object of the verb, the cinnamon, is meant for, was added by Juana herself.

\ea\label{ex:adp-tÿpi-12}
\begingl
\glpreamble nÿbÿcheku tiyunu tiyeseikupa eka kanela tÿpi upichai\\
\gla nÿ-bÿcheku ti-yunu ti-yeseiku-pa eka kanela tÿpi upichai\\
\glb 1\textsc{sg}-order 3i-go 3i-buy-\textsc{dloc.irr} \textsc{dem}a cinnamon \textsc{obl} medicine\\
\glft ‘I sent her to go and buy cinnamon for the medicine’
\endgl
\trailingcitation{[jxx-e191021e-2]}
\xe

In (\ref{ex:adp-tÿpi-13}), María C. tells me what she still has at home to prepare food, as there was no meat anymore.

\ea\label{ex:adp-tÿpi-13}
\begingl
\glpreamble sekeÿ tÿpi yÿtÿuku\\
\gla esekeÿ tÿpi yÿtÿuku\\
\glb bean \textsc{obl} food\\
\glft ‘beans for food’
\endgl
\trailingcitation{[uxx-e120427l.203]}
\xe
\is{aim/result|)}

\textit{(-)Tÿpi} can be used for the expression of recipients\is{recipient} or addressees.\is{addressee} This is the case in the following example with the verb \textit{-kuetea} ‘tell’. Contrary to the verb \textit{-kechu} ‘say’, the addressee (or recipient of the information) cannot be indexed on this verb as an object. This is the beginning of Miguel’s narration of the \isi{frog story} while looking at the picture book together with Alejo.

\ea\label{ex:adp-tÿpi-4}
\begingl
\glpreamble nikuetea pitÿpi, Arejo, eka kakuji eka chinachÿ aitubuchepÿimÿnÿ\\
\gla ni-kuetea pi-tÿpi Arejo eka kaku-ji eka chinachÿ aitubuchepÿi-mÿnÿ\\
\glb 1\textsc{sg}-tell 2\textsc{sg}-\textsc{obl} Alejo \textsc{dem}a exist-\textsc{rprt} \textsc{dem}a one boy-\textsc{dim}\\
\glft ‘I tell you, Alejo, that there was this one boy, it is said’
\endgl
\trailingcitation{[mtx-a110906l.002]}
\xe

In (\ref{ex:adp-tÿpi-5}), we have a verb from Spanish that is integrated into Paunaka as a non-verbal predicate \textit{regalau} ‘give as a present’ (see \sectref{sec:borrowed_verbs} for this strategy to integrate borrowed verbs). While the semantically related verb \textit{-punaku} ‘give’ is \isi{ditransitive} and can index the \isi{recipient} as an object, non-verbal predicates in general cannot index any other argument than the subject.\is{non-verbal predication} Thus the \isi{recipient} is integrated into the clause as an oblique with \textit{-tÿpi}. Juana told me with this sentence that the coffee we were drinking was a present some friends of her daughter had brought from Argentina.

\ea\label{ex:adp-tÿpi-5}
\begingl
\glpreamble nauku Argentina tupununube uikuinebu i regalau nitÿpi\\
\gla nauku Argentina ti-upunu-nube uikuinebu i regalau ni-tÿpi\\
\glb there Argentina 3i-bring-\textsc{pl} some.time.ago and give.as.present 1\textsc{sg}-\textsc{obl}\\
\glft ‘they brought it from Argentina some time ago and gave it to me as a present’
\endgl
\trailingcitation{[jxx-e120430l-4.28-29]}
\xe

If a referent is affected by the existence of something or, even more importantly, by the non-existence, it can be added to the \isi{existential clause} with the help of \textit{-tÿpi} as in the following example, which comes from María S. in explaining why she has not finished making her hammock, yet.

\ea\label{ex:adp-tÿpi-9}
\begingl
\glpreamble kuina tiempoina nÿtÿpi\\
\gla kuina tiempo-ina nÿ-tÿpi\\
\glb \textsc{neg} time-\textsc{irr.nv} 1\textsc{sg}-\textsc{obl}\\
\glft ‘I didn’t have time’ (lit.: ‘there was no time for me’)
\endgl
\trailingcitation{[rxx-e181022le]}
\xe

The preposition can also be used in goal expressions, whenever there is no intention involved to actually reach the goal. This is the case in (\ref{ex:tÿpi-adp-6}), where María C. makes a statement about the distance between \isi{Altavista} and Concepción, starting the sentence in Spanish and finishing it in Paunaka.%\footnote{One league is approximately five kilometers.}

\ea\label{ex:tÿpi-adp-6}
\begingl
\glpreamble son unos cinco leguas tukiu Turuxhi tÿpi Conce\\
\gla {son unos cinco leguas} tukiu Turuxhi tÿpi Conce\\
\glb {it is approximately five leagues} from Altavista \textsc{obl} Concepción\\
\glft ‘it is approximately five leagues from Altavista to Concepción’
\endgl
\trailingcitation{[cux-c120414ls-1.159]}
\xe

If a quantity of something is set in relation to another entity, \textit{(-)tÿpi} can be placed between the two NPs, as in (\ref{ex:tÿpi-adp-7}), which was produced by Juana to tell me the price of the rent of a house her daughter had been looking at.

\ea\label{ex:tÿpi-adp-7}
\begingl
\glpreamble mil bolivianos tÿpi entero ubiae \\
\gla mil bolivianos tÿpi entero ubiae\\
\glb 1000 bolivianos \textsc{obl} whole house\\
\glft ‘1000 bolivianos for the whole house’
\endgl
\trailingcitation{[jxx-p120430l-1.368-369]}
\xe

Consequently, \textit{(-)tÿpi} is also used in statements or questions about age, as in (\ref{ex:adp-tÿpi-10}), which was elicited from Isidro to ask about the age of a baby.

\ea\label{ex:adp-tÿpi-10}
\begingl
\glpreamble ¿kajanetu kuje chitÿpi?\\
\gla kajane-tu kuje chi-tÿpi?\\
\glb how.many-\textsc{iam} month 3-\textsc{obl}\\
\glft ‘how many months old is he?
\endgl
\trailingcitation{[dxx-d120416s.071]}
\xe

To finish the discussion on \textit{(-)tÿpi}, I will present one last example in which \textit{tÿpi} is used with a temporal expression. (\ref{ex:tÿpi-adp-8}) was produced by Miguel who was citing what Swintha had told him the year before about the time of her return to Bolivia.

\ea\label{ex:tÿpi-adp-8}
\begingl
\glpreamble anyo pasau tikechu “nÿbÿsÿupuna tÿpi agustu”\\
\gla anyo pasau ti-kechu nÿ-bÿsÿupuna tÿpi agustu\\
\glb year past 3i-say 1\textsc{sg}-come.\textsc{irr} \textsc{obl} August\\
\glft ‘last year she said: “I will come in August”’
\endgl
\trailingcitation{[mxx-d110813s-2.057]}
\xe

%tikuchuakÿika tÿpi mayu = va a dar luz en mayo, rmx-c121126s.12
\is{general oblique|)}

The remaining two prepositions, \textit{-keuchi} and \textit{-aj(i)echubu}, are much less frequent.

\subsection{The instrument and cause preposition}\label{sec:adp-keuchi}
\is{instrument/cause|(}

The preposition \textit{-keuchi} introduces obliques with the semantic roles of instrument or cause. It is glossed as ‘\textsc{ins}’ for instrument in this work, irrespective of whether it encodes instruments or causes. The preposition is usually indexed for person,\is{person marking} regardless of whether an NP follows or not, but occasionally occurs without a person marker. 

(\ref{ex:adp-keuchi-1}) is an example of its use as an instrument preposition. The sentence comes from a description by Juana of how to make a clay pot.\footnote{The complete description is given in the appendix.} The collected loam has to dry and then it has to be ground with the help of a pestle. The pestle, \textit{yubauke}, is preceded by the preposition and thus cannot be mistaken to conominate the object of the verb (which is left unexpressed in this sentence).

\ea\label{ex:adp-keuchi-1}
\begingl
\glpreamble upujaine bitÿyajikatu chikeuchi yubauke\\
\gla upu-jai-ne bi-tÿyajika-tu chi-keuchi yubauke\\
\glb other-day-\textsc{possd} 1\textsc{pl}-grind.\textsc{irr}-\textsc{iam} 3-\textsc{ins} pestle\\
\glft ‘the next day we can grind it with a pestle’
\endgl
\trailingcitation{[jmx-d110918ls-2.07]}
\xe

As for its origin, \textit{-keuchi} is probably composed of the \isi{verbal root} \textit{-ke}, a default/realis suffix\is{realis} \textit{-u} and a third person marker\textit{-chi} (\textit{-ke-u-chi} -do?-\textsc{real}-3). The same \isi{verbal root} is also found in the verb \textit{-ke-chu} ‘say’, where \textit{-chu} is a thematic suffix with default/realis marking.\footnote{Note that there is an empty verb root \textit{-k(i)e} in \isi{Baure}, which is often used with the meanings ‘say’ and ‘do’ \citep[221--222]{Danielsen2007}. This root is probably related to the Paunaka root \textit{-ke} in the verb \textit{-kechu} ‘say’ and the instrumental/cause preposition \textit{-keuchi}.} The Paunaka preposition is grammaticalised\is{grammaticalisation} insofar as the third person marker is not detachable or replaceable by another person marker and the RS cannot be changed, i.e. it is not possible to replace the /u/ following the root by /a/ to form an \isi{irrealis} form.

Some more examples follow; besides (\ref{ex:adp-keuchi-1}) above, (\ref{ex:adp-keuchi-2}) also has an instrument oblique, while (\ref{ex:adp-keuchi-3}) to (\ref{ex:adp-keuchi-5}) have cause obliques marked by \textit{-keuchi} .


%%NOTE TO SELF: can chikeuchi be used without an NP, like a clause?

In (\ref{ex:adp-keuchi-2}), \textit{keuchi} is placed before the nouns referring to the instruments used in pottery. No person marker is attached to \textit{keuchi} in this case. The example comes from the very same description as (\ref{ex:adp-keuchi-1}) above and describes a further step in the production of the pot. The loam is rolled to coils, and then the coils are placed above each other and the loam is pulled up with the help of a shell and water. 

\ea\label{ex:adp-keuchi-2}
\begingl
\glpreamble i keuchi sipÿ ÿne naka bijatÿkatu anÿke\\
\gla i keuchi sipÿ ÿne naka bi-jatÿka-tu anÿke\\
\glb and \textsc{ins} shell water here 1\textsc{pl}-pull.\textsc{irr}-\textsc{iam} up\\
\glft ‘and with shell and water we pull it up here’
\endgl
\trailingcitation{[jmx-d110918ls-2.18-19]}
\xe

In (\ref{ex:adp-keuchi-3}), \textit{chikeuchi} marks the constituent \textit{sipau} ‘strong chicha’ as a cause for drunkenness. María C. produced this sentence when we were talking about the feast day of Santa Rita, and I asked her to show us once how to make chicha. 

\ea\label{ex:adp-keuchi-3}
\begingl
\glpreamble beamÿnÿ asipau bakubÿu chikeuchi sipau\\
\gla bi-ea-mÿnÿ isipau bi-a-kubÿu chi-keuchi isipau\\
\glb 1\textsc{pl}-drink.\textsc{irr}-\textsc{dim} strong.chicha 1\textsc{pl}-\textsc{irr}-be.drunk 3-\textsc{ins} strong.chicha\\
\glft ‘if we drink strong chicha, we get drunk by the strong chicha’
\endgl
\trailingcitation{[uxx-p110825l.296]}
\xe

In the last two examples presented here, the cause is a person. (\ref{ex:adp-keuchi-4}) was produced by Juan C. when speaking with Miguel about the bad old times, when they still lived in \isi{Altavista}. The \textit{patrón} “paid” in goods, but sometimes even denied that payment. He is thus the one to blame that Juan C. did not have any trousers to wear anymore and this is indicated by the use of \textit{chikeuchi} together with the noun referring to the \textit{patrón}.

\ea\label{ex:adp-keuchi-4}
\begingl
\glpreamble kuinabutu nÿkasuneina chikeuchi nÿpatrun\\
\gla kuina-bu-tu nÿ-kasune-ina chi-keuchi nÿ-patrun\\
\glb \textsc{neg}-\textsc{dsc}-\textsc{iam} 1\textsc{sg}-trousers-\textsc{irr.nv} 3-\textsc{ins} 1\textsc{sg}-patrón\\
\glft ‘I didn’t have any trousers anymore because of my \textit{patrón}’
\endgl
\trailingcitation{[mqx-p110826l.454]}
\xe

Finally, in (\ref{ex:adp-keuchi-5}), in combination with the existential \isi{copula} \textit{kaku}, \textit{-keuchi} refers to the \isi{possessor} in a \isi{possessive clause} (‘there is X caused by Y’ = ‘Y has X’).\footnote{This specific example is interesting because \textit{chÿeche} ‘meat’ already carries a semi-lexicalised\is{lexicalisation} (i.e. non-referential) third person marker\is{person marking} indexing the possessor (\textit{chÿ-eche} ‘its flesh’). Thus indexation of the first person possessor on the noun could be morphologically blocked in this case. There are, however, a few similar examples in the corpus that do not include a noun already marked for possession.} The possessor in this case is Juan Ch. himself who uttered this sentence in a recording session with Riester after having stated that he had hunted a gray brocket. The person marker is thus \textit{ni-} for first singular in this case and there is no NP following the preposition.

\ea\label{ex:adp-keuchi-5}
\begingl
\glpreamble tanÿmapaiku kaku chÿeche nikeuchi nubiuyae tÿpi chinachÿ semana\\
\gla tanÿma-paiku kaku chÿeche ni-keuchi nÿ-ubiu-yae tÿpi chinachÿ semana\\
\glb now-\textsc{punct} exist meat 1\textsc{sg}-\textsc{ins} 1\textsc{sg}-house-\textsc{loc} \textsc{obl} one week\\
\glft ‘right now I have meat for one week in my house’
\endgl
\trailingcitation{[nxx-a630101g-1.56]}
\xe
\is{instrument/cause|)}

The next section will discuss the status and use of the comitative preposition.



\subsection{The comitative preposition}\label{sec:adp-ajechubu}
\is{comitative|(}

The comitative role of a participant can be expressed by the preposition \textit{-aj(i)echu\-bu} as in (\ref{ex:new23-adp-com}), in which María S. makes a statement about some little dogs of hers on my request.

\ea\label{ex:new23-adp-com}
\begingl
\glpreamble tikubijai chajechubu chÿenu\\
\gla ti-kubijai chÿ-ajechubu chÿ-enu\\
\glb 3i-play 3-\textsc{com} 3-mother\\
\glft ‘they are playing with their mother’
\endgl
\trailingcitation{[rxx-e181101l-1]}
\xe

This is the least grammaticalised\is{grammaticalisation|(} preposition. It carries the middle marker\is{middle voice|(} \textit{-bu}, thus it is decomposable as \textit{-aj(i)echu-bu} and it inflects\is{inflection} for RS\is{reality status} like active verbs do, by changing the last /u/ of the stem to /a/. With \isi{irrealis} RS, the form of the middle marker changes to \textit{-pu}, which is just what we expect from middle verbs (see \sectref{sec:Middle_voice}). The irrealis form of the preposition is thus \textit{-aj(i)echapu}. Since middle verbs are notionally \isi{intransitive}, they take the third person marker \textit{ti-}, but this is not what we find with the comitative preposition. It always takes \textit{chÿ-}\is{person marking} as a third person marker and I take this as a sign that it has lost some of its verbal properties and can be considered a preposition, though not completely grammaticalised yet.\is{middle voice|)} In addition, on its way from \isi{verb} to preposition, the semantic role of the person indexed on \textit{-aj(i)echubu} must have changed from accompanee to companion. The accompanee is defined as the person who is accompanied, the companion as the person who accompanies in the terminology of \citet[]{Stolz2006}.\is{grammaticalisation|)}
The change of semantic roles will be explained in more detail towards the end of this section. There are also some cases in which the preposition seems to be used like a verb, but I will start the overview with some examples that point into the direction of \textit{-aj(i)echubu} being a preposition.

The preposition itself is never inflected for \isi{plural} in my corpus, but the verb in the sentence can take the plural marker if both participants, the subject of a verb and the comitative participant, are third persons. Accompanee and companion are indexed together on the verb in this case. This is found in (\ref{ex:adp-com-1}), where the third-person marked preposition \textit{chajechubu} is placed before a plural NP to mark the companion. The accompanee is not conominated in this sentence, but it is clear from the context that it is a single man. The example stems from Juana’s account about a criminal in-law of hers, who had to escape when people found out that he had stolen cows. Apparently, his brothers were involved in the theft, because they fled together.

\ea\label{ex:adp-com-1}
\begingl
\glpreamble chajechubu chipijijinube tikutijikunubeji kimenukÿ \\
\gla chÿ-ajechubu chi-piji-ji-nube ti-kutijiku-nube-ji kimenu-kÿ \\
\glb 3-\textsc{com} 3-sibling-\textsc{col}-\textsc{pl} 3i-flee-\textsc{pl}-\textsc{rprt} woods-\textsc{clf:}bounded\\
\glft ‘together with his brothers he fled to the woods, it is said’
\endgl
\trailingcitation{[jxx-p120430l-2.087]}
\xe

From the same recording is (\ref{ex:adp-com-5}). Like in (\ref{ex:adp-com-1}) above, the verb carries a plural marker (and a reciprocal marker) to indicate joint action of the accompanee and the companion. This sentence is about the fight of one of the sons with his criminal father.

\newpage
\ea\label{ex:adp-com-5}
\begingl
\glpreamble i chijikiu punachÿ chipiji teukukujinubetu chajechubu chÿa\\
\gla i chijikiu punachÿ chi-piji ti-eu-kuku-ji-nube-tu chÿ-ajechubu chÿ-a\\
\glb and however other 3-sibling 3i-fight-\textsc{rcpc}-\textsc{col}-\textsc{pl}-\textsc{iam} 3-\textsc{com} 3-father\\
\glft ‘and nonetheless the other brother fought with his father’
\endgl
\trailingcitation{[jxx-p120430l-2.196]}
\xe

In (\ref{ex:adp-com-2}), the preposition is marked for second person singular, i.e. we have a second person companion here. The verb, however, is indexed for first person singular, i.e. contrary to (\ref{ex:adp-com-1}) and (\ref{ex:adp-com-5}), the companion is not indexed on the verb together with the accompanee in this case. The sentence was produced by Clara, who was supposed to be baking bread with her daughter, but was sitting with us and chatting and had forgotten her duty until her daughter showed up and reminded her.

\ea\label{ex:adp-com-2}
\begingl
\glpreamble es que nitibubuiku ajiechubu naka dice\\
\gla {es que} ni-tibubuiku a-jiechubu naka {dice}\\
\glb {it is the case that} 1\textsc{sg}-sit 2\textsc{pl}-\textsc{com} here {she says}\\
\glft ‘it is because I am sitting here with you, she says’
\endgl
\trailingcitation{[cux-120410ls.222]}
\xe

(\ref{ex:adp-com-3}) is an example that was elicited from María S. Like in (\ref{ex:adp-com-2}) above, the second person singular companion is not indexed on the verb, instead the first person singular marker only indexes the accompanee.

\ea\label{ex:adp-com-3}
\begingl
\glpreamble nichujijikubu pajiechubu\\
\gla ni-chujijiku-bu pi-ajiechubu\\
\glb 1\textsc{sg}-talk-\textsc{mid} 2\textsc{sg}-\textsc{com}\\
\glft ‘I am talking with you’
\endgl
\trailingcitation{[rxx-e141230s.133]} %el.
\xe

In (\ref{ex:adp-com-4}), we have an irrealis form of the comitative preposition. Irrealis is due to future reference of the whole sentence. This is another example in which only the accompanee is realised as a subject of the predicate, which is of the non-verbal type and thus does not take any person marker to index a third person subject (but it could take a plural marker to indicate that there is a plural subject). The sentence was produced by Juana in telling me about a planned visit by her daughter.
\newpage

\ea\label{ex:adp-com-4}
\begingl
\glpreamble kapupunuina tukiu nauku chajechapu treschÿnube chichechapÿimÿnÿ\\
\gla kapupunu-ina tukiu nauku chÿ-ajechapu treschÿ-nube chi-chechapÿi-mÿnÿ\\
\glb come.back-\textsc{irr.nv} from there 3-\textsc{com} three-\textsc{pl} 3-son-\textsc{dim}\\
\glft ‘she will come from there with her three children’
\endgl
\trailingcitation{[jxx-p110923l-1.253-254]}
\xe

In all examples presented up to here, \textit{-aj(i)echubu} can be defined as a preposition, but there are also some examples in the corpus that point to verbal\is{verb|(} status of the word bound to the way person is marked.\is{person marking|(}\is{agreement|(} In these cases, \textit{-aj(i)echubu} takes the same person index that the (other) predicate of the clause has, i.e. it agrees in person/number. It is thus the accompanee that is indexed on \textit{-aj(i)echubu} and not the companion. The companion might even be indexed by a person marker; at least, this is what happens in the elicited example of (\ref{ex:adp-com-6}). Compare this example to (\ref{ex:adp-com-3}) above, which was elicited from the same speaker, María S., on another occasion. In (\ref{ex:adp-com-3}), the second person singular companion is indexed on \textit{-ajiechubu}, in (\ref{ex:adp-com-6}) \textit{-ajiechubu} carries the first person singular \isi{subject} index of the accompanee. If \textit{-aj(i)echubu} is defined as a verb in this case, we can state that the subject is identical to the one of the other verb in this sentence. In addition, \textit{-ajiechubu} carries an \isi{object} marker which indexes the companion.\is{person marking|)} If the subject of the verb \textit{-aj(i)echubu} is the accompanee, then the middle marker\is{middle voice|(} makes totally sense, since it often encodes anticausatives\is{anticausative} (see \sectref{sec:Middle_voice}), the meaning of the active verb \textit{-aj(i)echu} can thus be defined as ‘accompany’, and of the middle verb \textit{-aj(i)echubu} as ‘being accompanied’. In this case, it should be excluded that objects are indexed or that objects are present at all, but exactly this occurs in (\ref{ex:adp-com-6}),\is{middle voice|)} as well as in the other two examples that follow, where the object is a third person expressed by an NP. The grammatical relation of the companion with the verb is thus very unclear in these cases.\is{object}

\ea\label{ex:adp-com-6}
\begingl
\glpreamble nichujijikubu najiechububi\\
\gla ni-chujijiku-bu nÿ-ajiechu-bu-bi\\
\glb 1\textsc{sg}-talk-\textsc{mid} 1\textsc{sg}-accompany-\textsc{mid}-2\textsc{sg}\\
\glft ‘I am talking with you’
\endgl
\trailingcitation{[mrx-e150219s.006]}
\xe

\largerpage
Another example, in which the accompanee is indexed rather than the companion is (\ref{ex:adp-com-7}), which was elicited from José. The companion is expressed by an NP.\footnote{The dislocative marker on \textit{-yejiku} probably relates to a motion verb which was uttered before, or it is simply a repetition of the verb I used before with the dislocative marker in trying (and failing) to produce the sentence myself.}

\ea\label{ex:adp-com-7}
\begingl
\glpreamble biyejikupu arusu bajiechubu Miyel, bamichupu\\
\gla bi-yejiku-pu arusu bi-ajiechu-bu Miyel bi-amichupu\\
\glb 1\textsc{pl}-tear.out-\textsc{dloc} rice 1\textsc{pl}-accompany-\textsc{mid} Miguel 1\textsc{pl}-help\\
\glft ‘we went to harvest rice together with Miguel, we helped him’
\endgl
\trailingcitation{[oxx-e120414ls-1a.120]}
\xe

Given that we have a similar example in the recordings from the 1960s by Riester (including an NP that expresses the companion), I suspect that this is the way accompaniment was expressed in prior times: by a middle-marked\is{middle voice} verb that encodes accompaniment indexing the accompanee. Grammaticalisation\is{grammaticalisation} into a preposition and change of the role indexed on \textit{-aj(i)echubu} from accompanee to companion has probably happened recently and is motivated by the need to integrate the companion into the comitative expression. It is well possible that the way accompaniment is expressed in Spanish has played a role in this change.

The example from Riester’s recordings is given here as (\ref{ex:adp-com-8}). Juan Ch. talks about his life in Retiro here.

\ea\label{ex:adp-com-8}
\begingl
\glpreamble nÿti nipÿsisikubu naka najechubu netinemÿnÿ\\
\gla nÿti ni-pÿsisikubu naka nÿ-ajechu-bu nÿ-etine-mÿnÿ\\
\glb 1\textsc{sg.prn} 1\textsc{sg}-be.alone here 1\textsc{sg}-accompany-\textsc{mid} 1\textsc{sg}-sister-\textsc{dim}\\
\glft ‘I am alone here together with my sister’
\endgl
\trailingcitation{[nxx-p630101g-1.163-164]}
\xe

\is{agreement|)}
\is{verb|)}
\is{comitative|)}

In addition to the prepositions described up to here, some others from Spanish may be used, but this occurs very rarely suggesting that their use is not grammaticalised.\is{oblique|)}\is{preposition|)} Thus we can proceed to the next section, which is about connectives.
