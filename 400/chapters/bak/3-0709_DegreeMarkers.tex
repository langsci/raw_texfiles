%!TEX root = 3-P_Masterdokument.tex
%!TEX encoding = UTF-8 Unicode
\section{Degree markers}\label{sec:MiscellaneousMarkers}
\is{degree marker|(}

There is a small number of miscellaneous markers that have not been described up to here, because they do not fit neatly into any category that has been mentioned so far. They are possibly best analysed as \isi{focus} markers, but not in the sense that they simply mark the focus in a clause. They all add more specific information. Rose (2021, p.c.) proposes the term “degree markers”, which I will adopt in my description.

All of these markers occur with words of different classes.\is{word class} The intensifier is described in \sectref{sec:Intensifier}. It usually attaches to predicates and can be translated with ‘very’ in most of the cases. In \sectref{sec:Additive}, the additive marker is discussed, which comes close in function to the English adverb ‘also’. It expresses that an action or a participant is added to the discourse. \sectref{sec:Limitatives} is about the two limitative markers, which are similar in function to the adverbs ‘only’ and ‘just’. Finally, \sectref{sec:EmphMarker} provides some examples of the emphatic markers.

\subsection{The intensifier}\label{sec:Intensifier}
\is{intensifier|(}

The intensifier marker is \textit{-yu}. It only occurs on predicates, predominantly on adjectives and descriptive (stative) verbs, but also sometimes on active verbs. An example with \textit{-yu} on an adjective is given in (\getref{ex:yu-ADJ}), in which Juana states that she finds it very nice that she can speak a few words of Paunaka with her grandson.

\ea\label{ex:yu-ADJ}
\begingl
\glpreamble michanayu echÿu bichujikiuchÿ \\
\gla michana-yu echÿu bi-chujik-i-u-chÿ \\
\glb nice-\textsc{ints} \textsc{dem}b 1\textsc{pl}-speak-\textsc{subord}-\textsc{real}-3\\
\glft ‘what we speak is very nice’
\endgl
\trailingcitation{[jxx-x110916.31]}
\xe

(\getref{ex:yu-STAT}) offers an example with the stative verb \textit{-mÿra} ‘be dry’. This sentence was recorded when Juana and Miguel went to a place a bit outside of Santa Rita to dig for some clay for Juana for making a pot.

\ea\label{ex:yu-STAT}
\begingl
\glpreamble timÿrayu\\
\gla ti-mÿra-yu\\
\glb 3i-be.dry-\textsc{ints}\\
\glft ‘it (the clay) is very dry’
\endgl
\trailingcitation{[jmx-d110918ls-1.032]}
\xe

The intensifier also typically appears on statements about great distance, which is the case in (\getref{ex:yu-far}), where María S. explains that the reason why I do not visit them in Bolivia very frequently is the great distance of my home from theirs.

\ea\label{ex:yu-far}
\begingl
\glpreamble tÿbaneyu pubiu\\
\gla ti-ÿbane-yu pi-ubiu\\
\glb 3i-be.far-\textsc{ints} 2\textsc{sg}-house\\
\glft ‘your home is very far away’
\endgl
\trailingcitation{[rxx-e120511l.214]}
\xe


%tikutiyu sache, mxx-e090728s-2.50
%nikujemuyu
%nisuyu

The marker can occur on the noun \textit{chubui} ‘old man’, when it is used predicatively,\is{nominal predicate} as in (\getref{ex:chubuiyu}), which comes from Miguel who was talking with Juan C.

\ea\label{ex:chubuiyu}
\begingl
\glpreamble chikuyejachÿu eka chubuiyubitu\\
\gla chi-kuye-ja-chÿu eka chubui-yu-bi-tu\\
\glb 3-be.like.this-\textsc{emph}1-\textsc{dem}b \textsc{dem}a old.man-\textsc{ints}-1\textsc{pl}-\textsc{iam}\\
\glft ‘that’s it, we are very old now’
\endgl
\trailingcitation{[mqx-p110826l.343]}
\xe

In addition, the intensifier is occasionally added to active verbs, too, and then it is best translated as ‘much’ or ‘a lot’. An example is given in (\getref{ex:yu-ACT}). It comes from Juana.

\ea\label{ex:yu-ACT}
\begingl
\glpreamble eka nimuyene teuyu pichai\\
\gla eka ni-muyene t-eu-yu pichai\\
\glb \textsc{dem}a 1\textsc{sg}-son.in.law 3i-drink-\textsc{ints} medicine\\
\glft ‘my son-in-law takes a lot of medicine’
\endgl
\trailingcitation{[jxx-p110923l-1.050]}
\xe

In general, Paunaka speakers prefer to use the quantifiers\is{quantifier|(} \textit{chama} ‘much’ and \textit{pariki} ‘many’ in these contexts (see \sectref{sec:QuantifyingAdverbs}). Both of them can also add the intensifier resulting in the forms \textit{chamayu} ‘very much’ and \textit{parikiyu} ‘very many’, but in all examples including the intensive form of these quantifiers that I found in my corpus, they are used predicatively as in (\getref{ex:parikiyu}).

The next example shows the usage of \textit{-yu} on the quantifier \textit{pariki}. It is a statement by Miguel about the abundance of ticks that we noticed when we came back from a visit at José’s.

\ea\label{ex:parikiyu}
\begingl
\glpreamble parikiyu samuchujane\\
\gla pariki-yu samuchu-jane\\
\glb some-\textsc{ints} tick.sp-\textsc{distr}\\
\glft ‘there are a lot of ticks’
\endgl
\trailingcitation{[mrx-c120509l.148]}
\xe

(\getref{ex:silver-Potosi-2}) also comes from Miguel. He is talking about Potosí here, because the taxi driver who was sitting with us comes from this city.

\ea\label{ex:silver-Potosi-2}
\begingl
\glpreamble naukuji chamayuji tÿmue nauku Potosi\\
\gla nauku-ji chama-yu-ji tÿmue nauku Potosi\\
\glb there-\textsc{rprt} much-\textsc{ints}-\textsc{rprt} silver there Potosí\\
\glft ‘there is a lot of silver in Potosí, it is said’
\endgl
\trailingcitation{[mty-p110906l.229]}
\xe
\is{quantifier|)}

%\ea\label{ex:FirstINTS-3}
%\begingl
%\glpreamble michanikiyuku echÿu papayu\\
%\gla michaniki-yu-uku echÿu papayu\\
%\glb delicious-\textsc{ints}-\textsc{add} \textsc{dem}b papaya\\
%\glft ‘papaya is also very delicious’\\
%\endgl
%\trailingcitation{[uxx-p110825l.193]}
%\xe

%\ea\label{ex:FirstINTS-4}
%\begingl
%\glpreamble chubuiyubitu\\
%\gla chubui-yu-bi-tu\\
%\glb old.man-\textsc{ints}-1\textsc{pl}-\textsc{iam}\\
%\glft ‘we are already old men’\\
%\endgl
%\trailingcitation{[mqx-p110826l.343]}
%\xe

Last but not least, another use of the intensifier is to create a honorific or \isi{endearment} form (depending on how close the actual relation is) of a kinship term to express respect or affection towards the addressed person. The kinship terms in this construction can best be interpreted as predicates of a clause of address,\is{addressee} because they carry a person marker\is{person marking} as in \isi{non-verbal predication}. An example is given in (\getref{ex:yu-HON}). It comes from María C. who was addressing Miguel. She was talking about sorcerers and wanted to assert the truth of her statements.

\ea\label{ex:yu-HON}
\begingl
\glpreamble nichechapÿiyubi kuina nichujikayÿchi\\
\gla ni-chechapÿi-yu-bi kuina ni-chujika-yÿchi\\
\glb 1\textsc{sg}-son-\textsc{ints}-2\textsc{sg} \textsc{neg} 1\textsc{sg}-speak.\textsc{irr}-\textsc{lim}2\\
\glft ‘my dear son, I don’t speak for the sake of speaking (i.e. it is true, what I am saying)’
\endgl
\trailingcitation{[ump-p110815sf.500]}
\xe

The only exception to the intensifier appearing on the predicate is the lexicalised adverb \textit{nÿmayu} ‘just, only when’, which derives from \textit{tanÿma} ‘now’. An example is given below. It comes from Clara.

\ea
\begingl
\glpreamble aa nÿmayu bitupunubu\\
\gla aa nÿmayu bi-tupunubu\\
\glb \textsc{intj} just 1\textsc{pl}-arrive\\
\glft ‘ah, we just arrived’
\endgl
\trailingcitation{[cux-c120510l-1.278]}
\xe
\is{intensifier|)}

\subsection{The additive marker}\label{sec:Additive}\is{additive|(}

There is one additive marker \textit{-uku}, which can attach to predicates and some other constituents. It can be translated into English as ‘also, too, as well’ and as ‘neither’, when negated. Some examples follow.

Prior to (\getref{ex:ADD-verbal}), Juana  was speaking about her manioc that had just sprouted, and when I asked her what she had planted on her field, she told me that – in addition to the manioc – she had planted some plantain. The additive marker thus has scope over the object.

\ea\label{ex:ADD-verbal}
\begingl
\glpreamble nebukuku merÿ\\
\gla nÿ-ebuku-uku merÿ\\
\glb 1\textsc{sg}-sow-\textsc{add} plantain\\
\glft ‘I also planted plantain’
\endgl
\trailingcitation{[jxx-e110923l-2.063]}
\xe

Prior to the statement in (\getref{ex:ADD-nonverbal}), Miguel, Alejo and Polonia had just been talking about some old manors that do not exist anymore. They had mentioned Palmarito, La Embocada, and Retiro, and then Miguel adds that \isi{Altavista} is not inhabited anymore either.

\ea\label{ex:ADD-nonverbal}
\begingl
\glpreamble Turuxhiuku kuinabukutu jentenubeina\\
\gla Turuxhi-uku kuina-bu-uku-tu jente-nube-ina\\
\glb Altavista-\textsc{add} \textsc{neg}-\textsc{dsc}-\textsc{add}-\textsc{iam} man-\textsc{pl}-\textsc{irr.nv}\\
\glft ‘in Altavista, now there are no people anymore either’
\endgl
\trailingcitation{[mty-p110906l.170]}
\xe

The additive marker \textit{-uku} can be attached to irrealis\is{irrealis|(} verbs, as is the case in (\getref{ex:know-too}), where Miguel asks José whether he knows a story, too.

\ea\label{ex:know-too}
\begingl
\glpreamble ¿pitiuku pichupauku echÿu jente tipÿkubai?\\
\gla piti-uku pi-chupa-uku echÿu jente ti-pÿkubai\\
\glb 2\textsc{sg.prn}-\textsc{add} 2\textsc{sg}-know.\textsc{irr}-\textsc{add} \textsc{dem}b man 3i-be.lazy\\
\glft ‘do you know the one about the lazy man, too?
\endgl
\trailingcitation{[mox-n110920l.001]}
\xe

However, if the verb has a thematic suffix\is{thematic suffix|(}, speakers normally make use of the irrealis variant \textit{-uka}, which directly follows the thematic suffix. Irrealis is thus marked on the additive marker and not on the thematic suffix of the verb, similar to the AM markers\is{thematic suffix|)} (see \sectref{sec:RealityStatus} and \sectref{sec:AssociatedMotion}). One example is given below. It comes from María C.

\ea\label{ex:ADD-IRR-2}
\begingl
\glpreamble kuina nichupuikuka nÿa\\
\gla kuina ni-chupuiku-uka nÿ-a\\
\glb \textsc{neg} 1\textsc{sg}-know-\textsc{add.irr} 1\textsc{sg}-father\\
\glft ‘I didn’t know my father either’
\endgl
\trailingcitation{[ump-p110815sf.148]}
\is{irrealis|)}
\xe\is{additive|)} 

\subsection{The limitative markers}\label{sec:Limitatives}
\is{limitative|(}

There are two limitative markers, \textit{-jiku} and \textit{-yÿchi}, which translate as ‘only, just’, but just like the Spanish equivalent \textit{nomás} \citep[cf.][45--46]{Mendoza2015}, they can also be used in contexts that do not precisely or primarily mark delimitation. They are thus also often used to emphasise that something is the way it is expressed as against imagined possible objections of the interlocutor. Both markers attach to different parts of speech and both can have wide or narrow scope. In the latter case, the markers usually attach directly to the word over which they have scope, though a few counterexamples to this general tendency exist. According to the speakers,  \textit{-jiku} and \textit{-yÿchi} can be used interchangeably. However, there are a few words that habitually combine with \textit{-jiku} but not with \textit{-yÿchi}: \textit{mÿbanejiku} ‘close, near’ and \textit{sepitÿjiku} ‘small’, which usually occur with \textit{-jiku}, but are sometimes also used without it, and \textit{chinajiku} ‘only one, alone’, which derives from \textit{chÿnachÿ} ‘one’.

The following three examples show the usage of \textit{-jiku} ‘\textsc{lim}1’ with wide or proposition scope. 

The statement in (\getref{ex:jiku-2}) is made by Miguel, who had explained before that he is afraid of being operated on and he might just go blind.

\ea\label{ex:jiku-2}
\begingl
\glpreamble repentekena suturubÿkejikunÿ\\
\gla repente-kena suturubÿke-jiku-nÿ\\
\glb maybe-\textsc{uncert} blind-\textsc{lim}1-1\textsc{sg}\\
\glft ‘maybe I am just going blind’
\endgl
\trailingcitation{[mqx-p110826l.300]}
\xe

(\getref{ex:jiku-1}) is from a story about the fox and the jaguar. The jaguar has caught the vulture and wants to eat him as a punishment for letting the fox escape, but the vulture escapes due to the jaguar's stupidness and while flying up, he defecates into the jaguar’s mouth and the jaguar eats his excrement instead. Thus in this specific case, \textit{-jiku} could also have scope over the noun only, but chances are that to express such a meaning, the limitative marker would have been attached to the noun instead of the verb.

\ea\label{ex:jiku-1}
\begingl
\glpreamble chinikujikutu chisikuji\\
\gla chi-niku-jiku-tu chi-sikuji\\
\glb 3-eat-\textsc{lim}1-\textsc{iam} 3-excrement\\
\glft ‘now he only ate his excrement’
\endgl
\trailingcitation{[jmx-n120429ls-x5.213]}
\xe

Another case of \textit{-jiku} being used with proposition scope is (\getref{ex:jiku-4}) from Miguel, where he tells me about the old school building of Santa Rita whose thatched roof was rotting away:

\ea\label{ex:jiku-4}
\begingl
\glpreamble tibÿrujikutu echÿu sakiji\\
\gla ti-bÿru-jiku-tu echÿu sakiji\\
\glb 3i-be.rotten-\textsc{lim}1-\textsc{iam} \textsc{dem}b satintail.sp\\
\glft ‘the satintail (\textit{Imperata brasiliensis}) was just rotten’
\endgl
\trailingcitation{[mxx-p110825l.090]}
\xe

In (\getref{ex:jiku-3}), \textit{-jiku} has narrow scope over the predicate. The sentence comes from Juana and the machine she speaks of was brought to make the reservoir in Santa Rita.

\ea\label{ex:jiku-3}
\begingl
\glpreamble chupunu echÿu makina echÿu tikurumejikujiku\\
\gla chÿ-upunu echÿu makina echÿu ti-kurumejiku-jiku\\
\glb 3-bring \textsc{dem}b machine \textsc{dem}b 3i-pierce-\textsc{lim}1\\
\glft ‘he brought the machine that only drills’
\endgl
\trailingcitation{[jxx-p120515l-2.215]}
\xe

Another example of \textit{-jiku} exhibiting narrow scope is (\getref{ex:jiku-5}), where the marker is attached directly to the noun. María C. states here that she only knew her mother when she was a child, because her grandmother had already passed away and her father died when she was still very young.

\ea\label{ex:jiku-5}
\begingl
\glpreamble nÿenujiku nichupuiku\\
\gla nÿ-enu-jiku ni-chupuiku\\
\glb 1\textsc{sg}-mother-\textsc{lim}1 1\textsc{sg}-know\\
\glft ‘I knew only my mother'
\endgl
\trailingcitation{[ump-p110815sf.147]}
\xe

The following examples include \textit{-yÿchi} ‘\textsc{lim}2’, the first three of them having wide scope.

(\getref{ex:lim2-1}) is a comment by Juana about her feelings when speaking to her daughter in Spain on the telephone.

\ea\label{ex:lim2-1}
\begingl
\glpreamble nichÿnumiyÿchi kue tichujika\\
\gla ni-chÿnumi-yÿchi kue ti-chujika\\
\glb 1\textsc{sg}-be.sad-\textsc{lim}2 if 3i-speak.\textsc{irr}\\
\glft ‘I only get sad when she talks’
\endgl
\trailingcitation{[jxx-p120430l-1.307]}
\xe

(\getref{ex:lim2-4}) comes from Miguel telling me about how it came to be that he went to school. Some other children had invited him to the classes, but he did not tell his parents about his plans to attend:

\ea\label{ex:lim2-4}
\begingl
\glpreamble niyunuyÿchi\\
\gla ni-yunu-yÿchi\\
\glb 1\textsc{sg}-go-\textsc{lim}2\\
\glft ‘I just went’
\endgl
\trailingcitation{[mxx-p181027l-1.016]}
\xe

(\getref{ex:lim-2-3}) comes from María S. She produced this sentence as a contrast to what she had said before, that there is a small football field for children nowadays and that children in general play a lot. Thus, although the limitative marker is attached to the adverbial here, it has scope over the whole proposition.

\ea\label{ex:lim-2-3}
\begingl
\glpreamble maneyÿchi biyunu asaneti\\
\gla mane-yÿchi bi-yunu asaneti\\
\glb morning-\textsc{lim}2 1\textsc{pl}-go field\\
\glft ‘we just went to the field in the mornings’
\endgl
\trailingcitation{[rxx-p181101l-2.147]}
\xe

The following examples show the use of \textit{-yÿchi} with narrow scope. In (\getref{ex:lim2-2}), María S. explains to me that her field is close to her house in comparison with other people that have fields further away from the village.

\ea\label{ex:lim2-2}
\begingl
\glpreamble nÿti nakayÿchi nisane\\
\gla nÿti naka-yÿchi ni-sane\\
\glb 1\textsc{sg.prn} here-\textsc{lim}2 1\textsc{sg}-field\\
\glft ‘I have my field just here’
\endgl
\trailingcitation{[rxx-e120511l.399]}%non-el.
\xe


%\ea
%\begingl
%\glpreamble “¿chija ijÿupe?” ijÿupeyÿchi eka chija\\
%\gla chi-ija ijÿupe ijÿupe-yÿchi eka chija\\
%\glb \\
%\glft ‘what’s the name of the spindle?” it’s name is just spindle’\\
%\endgl
%\trailingcitation{[cux-120410ls.179]}
%\xe

(\getref{ex:lim2-5}) comes from María C. talking about sorcerers and the people they killed. She contrasts seeing the things that happened with just talking about them.

\ea\label{ex:lim2-5}
\begingl
\glpreamble nechukue nimu kuina nichujikayÿchi nÿatiyubi\\
\gla nechukue ni-imu kuina ni-chujika-yÿchi nÿ-ati-yu-bi\\
\glb therefore 1\textsc{sg}-see \textsc{neg} 1\textsc{sg}-talk.\textsc{irr}-\textsc{lim}2 1\textsc{sg}-brother-\textsc{ints}-2\textsc{sg}\\
\glft thus I saw it, I am not only talking (i.e. gossiping), my dear brother’
\endgl
\trailingcitation{[ump-p110815sf.498]}
\xe

One final example of \textit{-yÿchi} being used for emphasis is (\getref{ex:lim2-6}) from Juana, where she affirms a statement I had made when we were speaking about the former \textit{patrón} of Retiro after listening to the recordings Riester made with Juan Ch. in the 1960s.

\ea\label{ex:lim2-6}
\begingl
\glpreamble ja ¡pariki maruyÿchi!\\
\gla ja pariki maru-yÿchi\\
\glb \textsc{afm} many bad-\textsc{lim}2\\
\glft ‘yes, (he was) very bad!’
\endgl
\trailingcitation{[jxx-p120430l-2.023-024]}
\xe

\is{limitative|)}
This leads us to the topic of the next section.

\subsection{The emphatic markers}\label{sec:EmphMarker}
\is{emphatic|(}

There are two emphatic markers. One of them has a longer and a shorter allomorph. The longer one is \textit{-ja’a} and only occurs at the end of words. It was primarily used by Juan Ch., who attached it to demonstrative adverbs in the recordings from the 1960s. However, Juana sometimes attaches the long form to the \isi{mirative} particle \textit{jimu} ‘you see, you know, right?’ (see Footnote \ref{fn:mirative} in \sectref{sec:Frustrative}). The shorter allomorph \textit{-ja} can be followed by other markers and is more frequent today. 

The emphatic marker is used to emphasise, stress or particularly point out something. It is difficult to find a good translation into English. In the Spanish variety spoken in the region, the particle \textit{pues} (pronounced \textit{pue}) comes close. The emphatic marker can also occur as a separate phonological word (realised as \textit{ja}, \textit{ja’a}, \textit{jaja} and the like), usually at the beginning of an utterance. In this case, it marks affirmation and is glossed as such (‘\textsc{afm}’).

(\getref{ex:emphi-1}) is an example of the longer allomorph. It comes from Juan Ch., who had just previously mentioned that they should leave Retiro and let the woods take over that place.

\ea\label{ex:emphi-1}
\begingl
\glpreamble kue kuina repente bipakamÿnÿ nakaja’a\\
\gla kue kuina repente bi-paka-mÿnÿ naka-ja’a\\
\glb if \textsc{neg} maybe 1\textsc{pl}-die.\textsc{irr}-\textsc{dim} here-\textsc{emph}1\\
\glft ‘if not, we will maybe just die here’
\endgl
\trailingcitation{[nxx-p630101g-1.123]}
\xe

%ja nisamaikubija ukuinebu, cux-c120510l-1.113
%no- aa nakaku eka kamajayenu, mox-a110920l-2 032
%eka panajachÿu pario eka pisaneina kuina binikukeneina, mox-n110920l.015
%kenajakena kuina nichupa, mtx-e110915ls.56
%pipajÿkujachÿ nipÿsisikumÿne naka, jxx-p120515l-1.201

(\getref{ex:emphi-2}) comes from Juana’s account of her encounter with the two old Paunaka ladies in Candelaria. She reveals here that she is a speaker, too. Note that the additive marker is irregularly given as \textit{-kuku} here, probably because it cannot be properly integrated phonologically, since there are already two vowels preceding it.

\ea\label{ex:emphi-2}
\begingl
\glpreamble nichujikuja, yeye, jimu eka nÿenu paunaka nÿakuku paunaka\\
\gla ni-chujiku-ja yeye jimu eka nÿ-enu paunaka nÿ-a-uku? paunaka\\
\glb 1\textsc{sg}-speak-\textsc{emph}1 granny \textsc{mir} \textsc{dem}a 1\textsc{sg}-mother Paunaka 1\textsc{sg}-father-\textsc{add}? Paunaka\\
\glft ‘I really speak (Paunaka), granny, you know, my mother is Paunaka, my father is Paunaka, too’
\endgl
\trailingcitation{[jxx-p120515l-1.161]}
\xe

(\getref{ex:emphi-3}) comes from Miguel’s story about the lazybones. In the beginning of the story, his wife asks him to make a field to supply them with food.\footnote{The sequence \textit{chÿu} on the verb could also relate to a (largely unproductive) restrictive morpheme given the fact that \isi{Mojeño Trinitario} has a productive “restrictive clitic” with the form \textit{-chu} (Rose 2021, p.c.). There is another context in which a sequence \textit{chÿu}/\textit{chu}/\textit{chÿ} with possibly restrictive semantics may occur in Paunaka: question words,\is{question word} see \sectref{sec:ContentQuestions}.}

\ea\label{ex:emphi-3}
\begingl
\glpreamble “panajachÿu pario eka pisaneina kuina binikukeneina"\\
\gla pi-ana-ja-chÿu pario eka pi-sane-ina kuina bi-niku-kene-ina\\
\glb 2\textsc{sg}-make.\textsc{irr}-\textsc{emph}1-\textsc{dem}b some \textsc{dem}a 2\textsc{sg}-field-\textsc{irr.nv} \textsc{neg} 1\textsc{pl}-eat-\textsc{nmlz}-\textsc{irr.nv}\\
\glft ‘“make something for your field, we do not have any food”’
\endgl
\trailingcitation{[mox-n110920l.015]}
\xe

One example of an emphatic marker as an affirmative particle is (\getref{ex:emphi-4}). It is an affirmation by Juana of a statement I made before about a dog that was eating bread.

\ea\label{ex:emphi-4}
\begingl
\glpreamble ja tiniku yui\\
\gla ja ti-niku yui\\
\glb \textsc{afm} 3i-eat bread\\
\glft ‘yes, it eats bread’
\endgl
\trailingcitation{[jxx-e110923l-2.042]}
\xe

%jaja tisÿimu ÿne, jxx-p120430l-2.598


The second emphatic marker is \textit{-kene}.\is{focus|(} It is homophonous with the nominaliser\is{nominalisation} (see \sectref{sec:MorphologyNominalisation}), but is sometimes also realised as \textit{-kine}. It occurs rarely, which is why had not previously distinguished it from the (equally rare) nominaliser. However Rose (2021, p.c.) analyses the cognate and equally homophonous forms \textit{-giene} of Trinitario\is{Mojeño Trinitario} as two different morphemes, a nominaliser and an intensifier. Since an analysis that distinguishes two morphemes also makes sense in the Paunaka case, I follow her analysis here. 

Baure has no cognate form to my knowledge, but in \isi{Terena}, a particle \textit{kene}/\textit{keno} is described as a connective which functions as an adversative (‘but’), but is also used “to enumerate the phases of a process or the individuals of a group” \citep[77]{ButlerEkdahl2014}\footnote{“para enumerar as fases de um processo ou os indivíduos de um grupo”}, where it indicates a switch from one topic to the other, where the new topic is a new subject. This particle could be cognate with the Paunaka marker.

As for Paunaka, I have found \textit{-kene} being attached a few times to the verb \textit{-anau} ‘make’. In this case, it always indicates either focus on the \isi{subject} or \isi{topicalisation} of it.

An example is given in (\getref{ex:NMLZ-s1}). This sentence was produced by Juana who was talking with María S. and wanted to resume the topic of the conversation after some disruption by other people speaking about different things.

\ea\label{ex:NMLZ-s1}
\begingl
\glpreamble no che nanakene nikuaji nÿpuipuna\\
\gla {no che} nÿ-ana-kene ni-kuaji nÿ-pui-puna\\
\glb {\textsc{intj}} 1\textsc{sg}-make-\textsc{emph}2- 1\textsc{sg}-net 1\textsc{sg}-fish-\textsc{am.prior}\\
\glft ‘why, no, as for me, I make my net and go fishing’
\endgl
\trailingcitation{[jrx-c151001fls-9.54]}
\xe

If the verb including \textit{-kene} is combined with the \isi{topic pronoun} \textit{chibu}, there is \isi{subject} focus and the sentence resembles a \isi{cleft} construction (see \sectref{sec:Clefts}). (\getref{ex:NMLZ-s2}) is such a case.  María C. states here that she recognises who bakes the bread I was talking about (in order to take the turn in the conversation with me).

\ea\label{ex:NMLZ-s2}
\begingl
\glpreamble chibu chanaukine\\
\gla chibu chÿ-anau-kene\\
\glb 3\textsc{top.prn} 3-make-\textsc{emph}2\\
\glft ‘she is the one who makes it’
\endgl
\trailingcitation{[uxx-e120427l.127]}
\xe

Indication of subject focus is also the function of the marker in (\getref{ex:NMLZ-c1}). Thematically, this example is also about the \textit{patrón}, whom Juan Ch. cannot imagine working like the workers do. Apparently, the \textit{patrón} claimed that he could do it.

\ea\label{ex:NMLZ-c1}
\begingl
\glpreamble sachu chanaukene\\
\gla sachu chÿ-anau-kene\\
\glb want 3-make-\textsc{emph}2\\
\glft ‘HE wants to do it’
\endgl
\trailingcitation{[nxx-p630101g-1.094]}
\xe

Almost all of the other instances of \textit{-kene} I have found in the corpus also include the emphatic marker \textit{-ja} (see above) and \isi{iamitive} \textit{-tu} (see \sectref{sec:Iamitive}), resulting in a sequence \textit{-kenejatu}.\footnote{Alternatively, \textit{-kenejatu} can also be analysed as \textit{one} marker resulting from \isi{grammaticalisation} of the combination of three markers.} This construction does not seem to imply focus or topicalisation of the subject, which is why I decided to gloss \textit{-kene} as an emphatic rather than a focus marker in the end.\is{focus|)} The examples can usually be translated well into Spanish with a sentence containing the particle \textit{pues}, but their meaning is not easy to grasp in English. They usually contain a kind of affirmation of something that was previously said or implicit in the context and have a flavour of certainty or conviction about the truth of the statement. Additionally, they can also indicate that the discourse topic is definite and not considered worth discussing further.

Consider (\getref{ex:nmlz-strange}), in which we find the sequence \textit{-kene-ja-tu} twice: on the manner \isi{demonstrative verb} \textit{-kuye} ‘be like this’ and on the copula \textit{kaku}. Juana evaluates as being bad here the fact that there was strike (and the resulting impossibility of Miguel taking his daughter to the doctor). 

\ea\label{ex:nmlz-strange}
\begingl
\glpreamble chikuyekenejatu jaa kakukenejatu, kuina tamicha\\
\gla chi-kuye-kene-ja-tu jaa kaku-kene-ja-tu kuina ti-a-micha\\
\glb 3-be.like.this-\textsc{emph}2-\textsc{emph}1-\textsc{iam} \textsc{afm} exist-\textsc{emph}2-\textsc{emph}1-\textsc{iam} \textsc{neg} 3i-\textsc{irr}-good\\
\glft ‘that’s how it is after all, yes, there is (strike), this is not good’
\endgl
\trailingcitation{[jxx-e120516l-1.084]}
\xe

Another example is (\getref{ex:kenejatu}) below, which comes from María S. telling me and Swintha about her move from Santa Rita to Concepción.

\ea\label{ex:kenejatu}
\begingl
\glpreamble nikutiukenejatu nauku chetupunune nichechapÿi nitibuyechupuna\\
\gla ni-kutiu-kene-ja-tu nauku chÿ-etupunu-ne ni-chechapÿi ni-tibuyechu-puna\\
\glb 1\textsc{sg}-be.ill-\textsc{emph}2-\textsc{emph}1-\textsc{iam} there 3-bring.to.place-1\textsc{sg} 1\textsc{sg}-son 1\textsc{sg}-settle-\textsc{am.prior.irr}\\
\glft ‘I was ill over there after all, thus my son brought me (here) to come and settle down’
\endgl
\trailingcitation{[cux-120410ls.115]}
\xe

Most of the verbs including this specific sequence of markers are stative, but an example with an active verb is given in (\getref{ex:active-kene}). I deliberately do not give any context here, because the example stems from a conversation between Miguel and Juana about sensitive issues.

\ea\label{ex:active-kene}
\begingl
\glpreamble aa tipikunubekenejatu\\
\gla aa ti-piku-nube-kene-ja-tu\\
\glb \textsc{intj} 3i-be.afraid-\textsc{pl}-\textsc{emph}2-\textsc{emph}1-\textsc{iam}\\
\glft ‘well, they are afraid after all’
\endgl
\trailingcitation{[jmx-c120429ls-x5.138]}
\xe


Finally, I have found two examples in the corpus in which \textit{-kene} (on each occasion realised as \textit{-kine}) is preceded by \textit{-ji}, possibly the reportive marker.\is{evidentiality} One of them is given below as (\getref{ex:jikine}). It comes from Juana who summarises the end of the story about the fox and the tiger: The tiger has been convinced by the fox to tie his paws with a stone and be pushed into a pond to obtain some cheese there. In the end, it was never cheese but only the reflection of the moon that the poor tiger had seen in the water.

\ea\label{ex:jikine}
\begingl
\glpreamble tibÿtuekubu chepaji nauku echÿu kesu kuinajikine chibuina echÿu\\
\gla ti-bÿtu-e-ku-bu chÿ-epa-ji nauku echÿu kesu kuina-ji-kine chibu-ina echÿu\\
\glb 3i-fall-\textsc{clf:}water-\textsc{th}1-\textsc{mid} 3-take-\textsc{rprt} there \textsc{dem}b cheese \textsc{neg}-\textsc{rprt}-\textsc{emph}2 3\textsc{top.prn}-\textsc{irr.nv} \textsc{dem}b\\
\glft ‘he fell into the water wanting to take the cheese there, it is said, but it wasn’t this, it is said’
\endgl
\trailingcitation{[jmx-n120429ls-x5.263]}
\xe

%tÿnaikuikinejatu = it is still long, jmx-n120429ls-x5.218
\is{emphatic|)}
\is{degree marker|)}
\is{inflection|)}

With this example, the discussion of the verb, morphology on predicates, and related topics is complete.\is{verb|)} The remaining two chapters focus on the construction of clauses.





