%!TEX root = 3-P_Masterdokument.tex
%!TEX encoding = UTF-8 Unicode

\section{Relative relations}\label{sec:RelativeClauses}
\is{relative relation|(}

A relative clause (RC) “provides some kind of specification about a participant” of the main clause with the aim of identifying this participant \citep[195]{Cristofaro2003}. 

Paunaka uses different means to express this kind of specification. This depends partly on whether the RC is headed\is{head|(} or headless and partly on the role the relativised item has in the RC. 


A headed RC is one which relates to an NP\is{noun phrase} in the main clause. Most of these headed RCs are completely unmarked in Paunaka, i.e. they are asyndetically\is{syndesis/asyndesis} juxtaposed\is{juxtaposition} to the NP they modify. The modified\is{modification} NP is usually the last constituent in the main clause. This being the case, an RC is formally not distinguishable from an asyndetically coordinated clause\is{coordination} (see \sectref{sec:AsyndeticCoordination}). It is only a matter of semantic relation between the two clauses: if one clause modifies the NP\is{noun phrase} to which it is juxtaposed,\is{juxtaposition} it can be analysed as an RC, and if no modification is involved, we rather deal with coordination (or possibly with unmarked subordination including an adverbial relation). %Further, it is only possible to identify restrictive RCs on this basis, the ones that really help in identifying the participant by providing information about it. Non-restrictive RCs on the other hand, the ones that provide additional information, cannot be distinguished from coordinated clauses at all.

(\ref{ex:new23-headrc}) is an unmarked RC whose head is \textit{pedasitokena} ‘(possible) little bits and pieces’ (from Spanish \textit{pedacito} and including the uncertainty marker \textit{-kena}). The sentence comes from Miguel telling a tale about ants, which are happy when a man goes to his field and eats his provisions there.

\ea\label{ex:new23-headrc}
\begingl
\glpreamble kakukena echÿu pedasitokena \textup{[}tibÿkupu apuke\textup{]}\\
\gla kaku-kena echÿu pedasito-kena ti-bÿkupu apuke\\
\glb exist-\textsc{uncert} \textsc{dem}b piece.\textsc{dim}-\textsc{uncert} 3i-fall ground\\
\glft ‘maybe there are crumbles that fall down’
\endgl
\trailingcitation{[mxx-n120423lsf-X.19]}
\xe
\is{head|)}

Headless RCs are those in which no NP conominates\is{conomination} an \isi{argument} of the main clause – remember that Paunaka uses person indexes to encode arguments of a verb and that co-occurrence of NPs is optional. In this case, the RC becomes the conominal itself. In contrast to headed RCs, headless RCs are usually introduced by a demonstrative,\is{syndesis/asyndesis}\is{nominal demonstrative|(} predominantly \textit{echÿu} ‘\textsc{dem}b’, but also \textit{eka} ‘\textsc{dem}a’. These demonstratives act as relative pronouns.\is{pronoun}\footnote{See \sectref{sec:HeadlessRC} for argumentation why I consider the demonstrative to belong to the RC and not to the main clause.} The predicate of the RC is balanced,\is{finite verb} i.e. there is no sign of dependency on the predicate itself.

(\ref{ex:new23-headless}) illustrates this. There is no noun in the main clause which could be modified by the RCs here, and there is no sign of dependency on the verbs, but nonetheless the demonstrative \textit{eka} is placed before them. The sentences was produced by Juana, who was speaking about different breeds of chicken.

\ea\label{ex:new23-headless}
\begingl
\glpreamble kuina tinijanea \textup{[}eka tikipÿpanaji\textup{]}, \textup{[}eka tisina tipisÿna\textup{]} tinijaneu entero amuke\\
\gla kuina ti-ni-jane-a eka ti-kipÿpa-na-ji eka ti-si-na ti-pisÿ-na ti-ni-jane-u entero amuke\\
\glb \textsc{neg} 3i-eat-\textsc{distr}-\textsc{irr} \textsc{dem}a 3i-be.white-\textsc{clf:}general-\textsc{col} \textsc{dem}a 3i-be.red-\textsc{clf:}general 3i-be.black-\textsc{clf:}general 3i-eat-\textsc{distr}-\textsc{real} whole corn\\
\glft ‘the white ones don’t eat it, the red and black ones eat the whole corn kernels (i.e. without crushing them)’
\endgl
\trailingcitation{[jxx-e150925l-1.143-144]}
\xe\is{nominal demonstrative|)}

These two types of RCs just described can be considered the major ones. They have in common that they build on a balanced verb\is{finite verb} and primarily apply to relativisation of S\is{subject} and O\is{object} arguments whose role in the main clause is also S or O. For the sake of simplicity, roles in stative non-verbal predication are subsumed under the label S here. The role A is much rarer in both main and relative clauses. The shared argument may also function as a possessor in the main clause, but not in the RC. I have not found any example in which it has the role of an oblique in the main clause, and in the RC, this is bound to relativisation of the \isi{indefinite pronoun} \textit{juchubu} ‘somewhere’. 

In addition to the RCs with balanced verbs, there is a third type of RC which is also relatively frequent and builds on a \isi{deranked verb}. This type is majorly found with relativisation of obliques,\is{oblique} but also with objects.\is{object} Inside the main clause, the relativised item can have the same roles as the other kinds of RCs, and additionally oblique roles are found.

The roles of the relativised item in the main clause and in the RC are summarised in  \tabref{tab:RelativeClausesRoles}, where the first two types are subsumed under one, because they are similar as to which roles are accessible to relativisation.

\begin{table}
\begin{tabularx}{.8\textwidth}{QQl}
\lsptoprule
& Function in MC & Function in RC\\
\midrule
RC with balanced verb & S, O, A, POSS & S, O, A, X\\
RC with deranked verb & S, O, A, POSS, X & O, X\\
\lspbottomrule
\end{tabularx}

\caption{Roles of the relativised item in the main clause and the RCs}
\label{tab:RelativeClausesRoles}
\end{table}


The correspondence between type of predicate used and role of the relativised item in the RC is in line with the relative deranking hierarchy\is{deranked verb} in \figref{fig:RelativDeranking}, which claims that deranked verbs rather occur on the right-hand side of the hierarchy.

\begin{figure}

A, S > O > Indirect Object, Oblique

\caption{Relative Deranking Hierarchy after \citet[203]{Cristofaro2003}}
\label{fig:RelativDeranking}
\end{figure}

I proceed as in the previous sections from most unmarked to most marked clauses. Unmarked RCs are described in \sectref{sec:UnmarkedRC}. They are usually headed, but a few examples of unmarked headless RCs have been found as well. In \sectref{sec:HeadlessRC} those RCs introduced by a demonstrative are presented. Most of them are headless, but some headed ones have also shown up in the corpus. RCs with deranked predicates are described in \sectref{sec:RC-Subord}. 
Subsequently, a discussion on cleft constructions in \sectref{sec:Clefts} completes this section. All RCs are given in square brackets in this section.

\subsection{Unmarked relative clauses}\label{sec:UnmarkedRC}
\is{head|(}
\is{juxtaposition|(}

RCs can be completely unmarked in Paunaka, a characteristic it shares with Mojeño\is{Mojeño languages} (\citealp[cf.][596--597]{OlzaZubiri2004}; \citealt[92]{Rose2014a}). By unmarked I refer to the fact that there is no syntactic or morphological hint at all that sets these clauses apart from independent clauses, although the RC and the main clause are uttered in one intonation unit. This has been argued to be “an early form of embedding” \citep[6]{Givon2012}.\is{embedding} Unmarked RCs are usually headed. The head noun is the last constituent of the main clause and the RC follows. All RCs given in this section could alternatively also be analysed as asyndetically coordinated clauses,\is{coordination} but semantically they encode a relative relation, i.e. they help to identify a participant by providing some specification about this participant \citep[195]{Cristofaro2003}. However, there are cases in which it is not clear whether a juxtaposed clause is meant to provide specification or to develop the storyline. I will come back to these cases later in this section. I have not found any example in which an unmarked RC was independently negated,\is{negation} but cases of individual negation of marked RCs occur (see \sectref{sec:HeadlessRC} below). It might be the case that negation demands a more explicit strategy of RC marking.

To start with, consider (\ref{ex:headed-RC-MID}). The head noun \textit{jimu} ‘fish’ is the object of the main clause and its last constituent. The RC follows. It only consists of one verb \textit{tiyÿbapakubu} ‘it is ground’.\footnote{This notionally transitive verb is intransitivised by the middle marker in this case. The agent is defocused and the patient becomes the subject of the anticausative middle verb (see \sectref{sec:Middle_voice}).} The relativised noun is the subject of this verb, and the verb ascribes a property to it, the property of being ground. Possible reference is thus restricted to ground fish.

The sentence was produced by Clara, when asked what she did with fish after grinding them. Note that the middle marker on the verb for ‘cook’ is missing here for an unknown reason.

\ea\label{ex:headed-RC-MID}
\begingl
\glpreamble  \textup{O\textsubscript{MC}/S\textsubscript{RC}}\\biyÿtiku echÿu jimu \textup{[}tiyÿbapakubu\textup{]}\\
\gla bi-yÿtiku echÿu jimu ti-yÿbapaku-bu\\
\glb 1\textsc{pl}-set.on.fire \textsc{dem}b fish 3i-grind-\textsc{mid}\\
\glft ‘we cook the ground fish’
\endgl
\trailingcitation{[cux-c120414ls-2.168]}
\xe

In (\ref{ex:headed-RC-MID}), the relativised noun is the object  of the main clause and the subject of the RC. This is indicated by the index “\textup{O\textsubscript{MC}/S\textsubscript{RC}}” above the Paunaka sentence, where the letter with the MC subscript stands for its role in the main clause, the letter with RC subscript for its role in the RC. The relativised noun is not repeated in the RC nor is there any resumptive pronoun or other independent word relating to it. We can thus state that Paunaka uses a gapping strategy in RCs as far as one can speak of a “gap” at all in relation to a language that indexes its arguments on the verb. The verb in the RC is fully inflected\is{inflection} for its arguments and for RS.\is{reality status}

Backed by this description of the general characteristics of the unmarked RC, we can have a look at some more examples. An analysis favouring an RC over a coordinated clause is relatively clear whenever the clause ascribes some property to the head noun. Paunaka has few adjectives, and many properties are expressed by stative verbs that are juxtaposed to the noun they modify, thus in an RC. (\ref{ex:headed-RC-2}) and (\ref{ex:headed-RC-3}) are two examples of this. RCs containing stative verbs are often best translated by an adjective.

In (\ref{ex:headed-RC-2}) the stative verb \textit{timÿuji} ‘a soft mass is wet’ modifies the head noun \textit{bimÿu} ‘our clothes’.\footnote{The fact that noun and verb have a homophonuous stem \textit{-mÿu} is a coincidence.} The sentence comes from Juana telling the story about her grandparents’ journey to Moxos. On their way back, they have to cross an arroyo after rainfalls and get wet, so they have to change clothes as Juana’s grandmother prompts her husband:

\ea\label{ex:headed-RC-2}
\begingl
\glpreamble\textup{O\textsubscript{MC}/S\textsubscript{RC}}\\“kampiaubina bimÿu \textup{[}timÿuji\textup{]}” \\
\gla kampiau-bi-ina bi-mÿu ti-mÿu-ji\\
\glb change-1\textsc{pl}-\textsc{irr.nv} 1\textsc{pl}-clothes 3i-be.wet-\textsc{clf:}soft.mass\\
\glft ‘“let’s change the wet clothes”’
\endgl
\trailingcitation{[jxx-p151016l-2.128]}
\xe

In (\ref{ex:headed-RC-3}) the RC ascribes the property of being lazy to its head noun, the man. Miguel asks his brother José here whether he also knows the story about the lazy man. 

\ea\label{ex:headed-RC-3}
\begingl
\glpreamble \textup{O\textsubscript{MC}/S\textsubscript{RC}}\\¿pitiuku pichupauku echÿu jente \textup{[}tipÿkubai\textup{]}?\\
\gla piti-uku pi-chupa-uku echÿu jente ti-pÿkubai \\
\glb 2\textsc{sg.prn}-\textsc{add} 2\textsc{sg}-know.\textsc{irr}-\textsc{add} \textsc{dem}b man 3i-be.lazy\\
\glft ‘do you know the [one about] the lazy man, too?’
\endgl
\trailingcitation{[mox-n110920l.001]}
\xe

The question words \textit{chija} ‘what, who’ and \textit{juchubu} ‘where’ in their function as indefinite pronouns\is{indefinite pronoun} (see \sectref{sec:IndefinitePronouns}) are also usually followed by an unmarked RC.\footnote{\label{fn:chija}I have also found one example in which it seems that \textit{chija} ‘what, who’ is not the head but the relativiser, this being connected to the fact that there is another word \textit{punaina} ‘other (thing)’ that can be postulated to be the head of the RC. Due to lack of (more) data, this type will be neglected here. The sentence comes from the account by María S. about the past.

\ea\label{ex:RC-chija1}
\begingl
\glpreamble \textup{S\textsubscript{MC}/O\textsubscript{RC}}\\kuina punaina \textup{[}chija binika\textup{]}, chibu biniku\\
\gla kuina puna-ina chija bi-nika chibu bi-niku\\
\glb \textsc{neg} other-\textsc{irr} what 1\textsc{pl}-eat.\textsc{irr} 3\textsc{top.prn} 1\textsc{pl}-eat\\
\glft ‘there was nothing else that we could eat, this, we ate’
\endgl
\trailingcitation{[rxx-p181101l-2.247]}
\xe} (\ref{ex:RCchija3}) is from Juana’s telling of the \isi{frog story} and verbalises her assumption why the boy and the dog look behind the log at the end of the book.

\ea\label{ex:RCchija3}
\begingl
\glpreamble \textup{S\textsubscript{MC}/O\textsubscript{RC}}\\kakukena chija \textup{[}echÿu kabemÿnÿ tisÿikutu\textup{]}\\
\gla kaku-kena chija echÿu kabe-mÿnÿ ti-sÿiku-tu\\
\glb exist-\textsc{uncert} what \textsc{dem}b dog-\textsc{dim} 3i-smell-\textsc{iam}\\
\glft ‘there might be something that the dog has smelled’
\endgl
\trailingcitation{[jxx-a120516l-a.409]}
\xe

In (\ref{ex:RCwhere1}), Juana speaks about her grandchild.

\ea\label{ex:RCwhere1}
\begingl
\glpreamble \textup{O\textsubscript{MC}/X\textsubscript{RC}}\\... te tisemaikatu juchubu \textup{[}trabakuina\textup{]}\\
\gla te ti-semaika-tu juchubu trabaku-ina\\
\glb \textsc{seq} 3i-search.\textsc{irr}-\textsc{iam} where work-\textsc{irr.nv}\\
\glft ‘... then he will look for where to work’
\endgl
\trailingcitation{[jxx-p110923l-1.193]}
\xe

The RC of (\ref{ex:RCwhere1}) has a non-verbal predicate borrowed from Spanish as a predicate. The RC of (\ref{ex:headed-RC-N}) builds on a nominal predicate. This is the beginning of the story about the enchanted man told by Miguel.

\ea\label{ex:headed-RC-N}
\begingl
\glpreamble \textup{S\textsubscript{MC}/S\textsubscript{RC}}\\kakubaneji chÿnachÿ jente \textup{[}bakeronu\textup{]}\\
\gla kaku-bane-ji chÿnachÿ jente bakeronu\\
\glb exist-\textsc{rem}-\textsc{rprt} one man cowherd\\
\glft ‘once upon a time there was a man who was a cowherd, it is said’
\endgl
\trailingcitation{[mxx-n151017l-1.01]}
\xe

%(\ref{ex:headed-RC-3}) comes from pre-organisation of story-telling by Miguel. He first asks whether his brother José knows the story about the lazy man. When he actually starts to tell the story, he rather uses two coordinated main clauses in order to introduce and characterise the main character, see (\ref{ex:noRC}). The fact that Miguel uses the connective \textit{i} ‘and’ is decisive for analysing this example as consisting of two coordinated main clauses, otherwise it is very similar to (\ref{ex:headed-RC-1}) above and to (\ref{ex:headed-RC-3}).
%
%\ea\label{ex:noRC}
%\begingl
%\glpreamble kakubaneji chinachÿ jente i tipÿkubai\\
%\gla kaku-bane-ji chinachÿ jente i ti-pÿkubai\\
%\glb exist-\textsc{rem}-\textsc{rprt} one man and 3i-be.lazy\\
%\glft ‘once upon a time there was a man and he was lazy’\\
%\endgl
%\trailingcitation{[mox-n110920l.011]}
%\xe 

Whenever the verb in the RC does not describe a property per se, it is much harder to distinguish relativisation from \isi{coordination}. If the event described by the juxtaposed clause is prior to the one in the first clause, there is a great chance that the clause is produced to provide a specification of a participant of the other clause.

This is the case in (\ref{ex:ditr-RC}). At the moment Juana produced this sentence, the topic of her story was the journey back home of her grandparents who had been in Moxos to buy cows, and in this sentence, she describes what her grandparents did when they rested. The juxtaposed clause refers back to a prior point in the timeline, her grandparents’ stay in Moxos. After this short switch to the prior time point, Juana continues describing the activities of their grandparents during their rest. We can thus deduce that the juxtaposed clause was uttered with the sole aim to specify how and where a noun of the other clause, \textit{charke} ‘dried meat’, was obtained. It thus fits the general semantic profile of an RC and is best analysed as such.\footnote{Note that there is \isi{haplology} in the verb of the RC, so that the two morphemes \textit{-inu} ‘\textsc{ben}’ and \textit{-nube} ‘\textsc{pl}’ are fused into \textit{inube}.}

\ea\label{ex:ditr-RC}
\begingl
\glpreamble \textup{O\textsubscript{MC}/O\textsubscript{RC}}\\tinikukuikunubeji chitapikinenube i charki \textup{[}chipunakinube nauku tukiu Monkorÿye\textup{]}\\
\gla ti-niku-kuiku-nube-ji chi-tapiki-ne-nube i charki chi-punak-inu-nube nauku tukiu Monkoxÿ-yae\\
\glb 3i-eat-\textsc{cont}-\textsc{pl}-\textsc{rprt} 3-travel.supplies-\textsc{possd}-\textsc{pl} and dried.meat 3-give-\textsc{ben}-\textsc{pl} there from Moxos-\textsc{loc}\\
\glft ‘they were eating their travel supplies, it is said, and dried meat that they had given to them in Moxos’
\endgl
\trailingcitation{[jxx-p151016l-2.174]}
\xe

Similar switches to a point in time prior to the actual timeline of the discourse are found in (\ref{ex:headed-RC-OBJ1}) and (\ref{ex:headed-RC-OBJ2}). Both examples also stem from Juana.

In (\ref{ex:headed-RC-OBJ1}), she tells me about a bird whose name she just remembered.

\ea\label{ex:headed-RC-OBJ1}
\begingl
\glpreamble \textup{S\textsubscript{MC}/O\textsubscript{RC}}\\i kaku echÿu pisemÿnÿ \textup{[}nimumuku uchuine\textup{]}\\
\gla i kaku echÿu pise-mÿnÿ ni-imumuku uchuine\\
\glb and exist \textsc{dem}b bird-\textsc{dim} 1\textsc{sg}-look just.now\\
\glft ‘and there is this bird that I have just watched’
\endgl
\trailingcitation{[jxx-p120430l-1.100]}
\xe

(\ref{ex:headed-RC-OBJ2}) is from a description by Juana of how she makes a clay pot. The structure of the main clause is not entirely clear. It either misses a predicate that expresses possession or otherwise the \isi{personal pronoun} is somehow misplaced here or possibly best understood as a dislocated constituent in a relatively loose relation to the rest of the clause. In any case, the bringing of the loam precedes its presence at her house in Santa Cruz.

\ea\label{ex:headed-RC-OBJ2}
\begingl
\glpreamble \textup{O\textsubscript{MC} or S\textsubscript{MC}?/O\textsubscript{RC}}\\nÿti nauku nubiuyae echÿu muteji \textup{[}nupunu tukiu nauku\textup{]}\\
\gla nÿti nauku nÿ-ubiu-yae echÿu muteji nÿ-upunu tukiu nauku\\
\glb 1\textsc{sg.prn} there 1\textsc{sg}-house-\textsc{loc} \textsc{dem}b loam 1\textsc{sg}-bring from there\\
\glft ‘there in my house, I (have) the loam that I brought from there’
\endgl
\trailingcitation{[jxx-d110923l-1.01]}
\xe

If there is no such switch to a prior point in time, a juxtaposed clause can often be analysed as either an RC or an independent coordinated clause.

Consider (\ref{ex:headed-RC-1}), which is the first sentence from the story about two men who meet the devil in the woods as told by Miguel. The characters of the story, two men, are introduced with the non-verbal existential \isi{copula} \textit{kaku}, and the following clause provides information about what they do, i.e. they go hunting. It is not a specification of their character nor does it refer back to something the two men had done before. The opposite is true; the fact that they go hunting is important for the development of the story. On the other hand, since this is the first sentence in the story, it might well be the case that the second clause is produced to delimit the set of possible referents by specifying what the two men do, thus setting the scene for everything that follows. 

The index of the syntactic roles in main and relative clauses and the use of square brackets indicate the structure of the sentence if it is analysed as an RC. There are two translations though, one with an RC and the other one with a coordinated clause.

\ea\label{ex:headed-RC-1}
\begingl
\glpreamble \textup{S\textsubscript{MC}/S\textsubscript{RC}}\\kakubaneji ruschÿnubeji jente \textup{[}tiyununube tichubikupunube\textup{]}\\
\gla kaku-bane-ji ruschÿ-nube-ji jente ti-yunu-nube ti-chubiku-pu-nube\\
\glb exist-\textsc{rem}-\textsc{rprt} two-\textsc{pl}-\textsc{rprt} man 3i-go-\textsc{pl} 3i-stroll-\textsc{dloc}-\textsc{pl}\\
\glft ‘once upon a time, there were two men who went hunting (lit.: strolling around), it is said’\\or: ‘once upon a time, there were two men, it is said, and they went hunting’
\endgl
\trailingcitation{[mxx-n101017s-1.014]}
\xe

Considering the indexes of the syntactic roles I have provided in the examples above, we notice that there is a lot of relativisation out of S and O positions and that the relativised noun mostly either functions as the S or O of the RC.\is{object}\is{subject} In general, S and O seem to be more common in relativisation in nominative-accusative languages than A \citep[199]{Cristofaro2003}. I have found one example though, given here as (\ref{ex:headed-RC-A}), in which the relativised noun is the A of the main clause. Two coordinated RCs follow, where the relativised noun has the role of S in the first one and the possibly the A role in the second one. The restriction “possibly” relates to the fact that it is not clear which syntactic role a complement clause has in the main clause in Paunaka (see \sectref{sec:ComplementClauses}). The role A presupposes that a complement clause is an object of the main clause predicate.

The sentence, which is abbreviated here to only show the relevant details, comes from the speech Miguel gave at the workshop on Paunaka in 2011.

\ea\label{ex:headed-RC-A}
\begingl
\glpreamble \textup{A\textsubscript{MC}/S\textsubscript{RC}, A\textsubscript{RC}?}\\...titukanube eka beteapunuku eka bipijijinube \textup{[}kakunube naka, kuina chichupanube tichujikanube\textup{]}\\
\gla ti-itu-uka-nube eka bi-etea-punuku eka bi-piji-ji-nube kaku-nube naka kuina chi-chupa-nube ti-chujika-nube\\
\glb 3i-master-\textsc{add.irr}-\textsc{pl} \textsc{dem}a 1\textsc{pl}-language-\textsc{reg} \textsc{dem}a 1\textsc{pl}-sibling-\textsc{col}-\textsc{pl} exist-\textsc{pl} here \textsc{neg} 3-know.\textsc{irr}-\textsc{pl} 3i-speak.\textsc{irr}-\textsc{pl}\\
\glft ‘... our siblings, who are here and who do not know to speak, will also learn our language again’
\endgl
\trailingcitation{[mxx-x110917.18]}
\xe

As for the position of the RC, it is always postnominal. In the examples given up to this point, it also always followed the complete main clause. I have only found one example in which the RC does not follow the complete main clause. This example was elicited from Juana and is given in (\ref{ex:first-RC}). The RC directly follows its head noun here and is as unmarked as all other RCs given in this section.

\ea\label{ex:first-RC}
\begingl
\glpreamble \textup{S\textsubscript{MC}/S\textsubscript{RC}}\\eka nipeu kabe \textup{[}tipakutu\textup{]} timajaiku tÿbane\\
\gla eka ni-peu kabe ti-paku-tu ti-majaiku ti-ÿbane\\
\glb \textsc{dem}a 1\textsc{sg}-animal dog 3i-die-\textsc{iam} 3i-bark 3i-be.far\\
\glft ‘my dog that already died, used to bark’
\endgl
\trailingcitation{[jxx-a110923l.07]}
\xe
 
The strategy of using completely unmarked RCs is bound to the occurrence of a head noun in the main clause in general, but a few exceptions exist where we have an unmarked headless RC.\is{head|)} First of all, there is a small group of stative verbs\is{stative verb|(}  that are seldom used predicatively in main clauses. They are rather integrated into clauses by means of a headless RC. Among them are \textit{tijai} ‘day’ (\textit{ti-jai} 3i-be.light), \textit{tisÿeipu} ‘south wind; windy, cold, cloudy weather coming from the south’ (\textit{ti-sÿei-pu} 3i-be.cold-\textsc{dloc}), \textit{tijapÿ} ‘collared peccary’ (\textit{ti-japÿ} 3i-fill), \textit{tikubiku} ‘owner of a house’ (\textit{ti-ku-ubiku?} 3i-\textsc{attr}-reside?), \textit{tikupeuchÿ} ‘owner of animals’ (\textit{ti-ku-peu-chÿ} 3i-\textsc{attr}-animal-3), \textit{tikuyaechÿ} ‘owner’ (\mbox{\textit{ti-ku-yae-chÿ}} 3i-\textsc{attr}-\textsc{grn}-3), and a few more that are not decomposable anymore like \textit{tujubeiku} ‘wind’. Actually, every “noun”\is{noun} that begins with /t/ is suspicious of having originated from a stative verb in an RC.\is{stative verb|)} Normally, headless RCs are introduced by a demonstrative,\is{nominal demonstrative|(} see \sectref{sec:HeadlessRC}; however, some of these RCs have become so conventionalised as “noun phrases without nouns” \citep[cf.][]{Dryer2004}\is{noun phrase} that the demonstrative may also be dropped, as in (\ref{ex:RC-NP}). In this example the RC is the possessor of its head noun, i.e. it has a possessor role in the main clause. The possessor is postponed in this case, and it usually directly follows the possessed. The recording was made in May, and María S. and I were talking about coldness in Bolivia and in Germany. I told her that I didn’t like the cold, and she replied:

\ea\label{ex:RC-NP}
\begingl
\glpreamble \textup{POSS\textsubscript{MC}/S\textsubscript{RC}}\\chitiempone tanÿma \textup{[}tisÿeipu\textup{]}\\
\gla chi-tiempo-ne tanÿma ti-sÿei-pu\\
\glb 3-time-\textsc{possd} now 3i-be.cold-\textsc{dloc}\\
\glft ‘it is its season now, of the south wind’
\endgl
\trailingcitation{[rxx-e120511l.293]}
\xe
% POSS of MC, S of RC


Finally, the last example I want to present in this section goes even further in that it does not include a verb that belongs to the usual suspects that have been listed above. In this case, lack of a demonstrative may be due to indefiniteness of the relativised item. It is an exceptional case though. 

The sentence was produced by Clara, who replied to María C.’s question about who died. María C. obviously had not realised that we were talking about a dead dried piranha Swintha had found at the lakeside of the big reservoir in Concepción.

\ea\label{ex:RCiii}
\begingl
\glpreamble \textup{O\textsubscript{MC}/S\textsubscript{RC}}\\ÿmu tipaku, nauku chimunube \textup{[}tepakutu\textup{]}\\
\gla ÿmu ti-paku nauku chi-imu-nube ti-paku-tu\\
\glb piranha 3i-die there 3-see-\textsc{pl} 3i-die-\textsc{iam}\\
\glft ‘the PIRANHA died, they saw a dead one there’
\endgl
\trailingcitation{[cux-c120414ls-1.236]}
\xe\is{nominal demonstrative|)}

In the following section more examples of headless RCs are shown. They are usually marked by taking a demonstrative.


%In (\ref{ex:headed-RC-OBL}) the relativised noun has the role of an oblique in the RC. María S. compares her childhood to the life of children today. -> but is this an RC or two MCs??
%
%\ea\label{ex:headed-RC-OBL}
%\begingl
%\glpreamble kuina neneina tanÿma kaku kancha \textup{[}tikubijainube sesejinube\textup{]} bane kuina\\
%\gla kuina nena-ina tanÿma kaku kancha ti-kubijai-nube sesejinube bane kuina\\
%\glb \textsc{neg} like-\textsc{irr.nv} now exist small.football.area 3i-play-\textsc{pl} children before \textsc{neg}\\
%\glft ‘it is not like today that there is a small football area, where children play, before there was none’\\
%\endgl
%\trailingcitation{[rxx-p181101l-2.142]}
%\xe


%with resumptive adverb: pumane nauku kaku nauku chinachÿ posa mutemena, mox-n110920l.121

\subsection{Relative clauses introduced by demonstratives}\label{sec:HeadlessRC}\is{nominal demonstrative|(}

Relative clauses can be introduced by the demonstratives \textit{eka} or \textit{echÿu} (see \sectref{sec:DemPron}), with \textit{echÿu} being more frequent. This is sometimes found with headed RCs\is{head} as in (\ref{ex:headed-DEM}) and (\ref{ex:headed-DEM-2}), but most often, RCs introduced by a demonstrative are headless. The verb is usually completely unmarked (but see \sectref{sec:RC-Subord}). However, the use of a demonstrative can itself be considered a hint that the verb has some nominal properties, reflecting “at least a partial conversion to nominal type” \citep[232]{Andrews2007}. The relativised item can have the syntactic roles of S, O and also A in the RC, but A roles are rare. As for the function of the RC inside the main clause, S and O roles have been found. In drawing an analogy to headed RCs, I suspect that it is also possible that the RC is the A of its main clause, but this remains to be proved. 

The demonstrative belongs to the RC, not to the main clause (in which case it could be interpreted as the head \is{head|(} of the following RC). This becomes clear when considering the following example of a headed RC introduced by a demonstrative. It comes from the story about the cowherd who is enchanted by the spirit of the hill. The first part is an equative clause (see \sectref{sec:PropInclEquatAttr}) in which the topicalising pronoun \textit{chibu} and the NP \textit{eka bakajane} ‘the cows’ are equated. The equative clause is complete with these two constituents, and there is no slot for another argument either, so the demonstrative that follows the NP must belong to the RC.


\ea\label{ex:headed-DEM}
\begingl
\glpreamble \textup{S\textsubscript{MC}/O\textsubscript{RC}}\\chibu eka bakajane \textup{[}eka pisemaiku\textup{]}\\
\gla chibu eka baka-jane eka pi-semaiku\\
\glb3\textsc{top.prn} \textsc{dem}a cow-\textsc{distr} \textsc{dem}a 2\textsc{sg}-search\\
\glft ‘these are the cows that you were looking for’
\endgl
\trailingcitation{[mxx-n151017l-1.44]}
\xe
%rel n. = S/O

Another example with a headed RC introduced by a demonstrative is (\ref{ex:headed-DEM-2}). The main clause has a VO structure here and the RC is juxtaposed to specify which kind of machine Juana is talking about in her account about how the water reservoir in Santa Rita was made. An engineer had come and brought the big machine.

\ea\label{ex:headed-DEM-2}
\begingl
\glpreamble \textup{O\textsubscript{MC}/S\textsubscript{RC}}\\chupunu echÿu makina \textup{[}echÿu tikurumejikujiku\textup{]}\\
\gla chÿ-upunu echÿu makina echÿu ti-kurumejiku-jiku\\
\glb 3-bring \textsc{dem}b machine \textsc{dem}b 3i-pierce-\textsc{lim}1\\
\glft ‘he brought the machine that only drills’
\endgl
\trailingcitation{[jxx-p120515l-2.215]}
\xe
%rel n. = O/S
\is{head|)}

%i kaku punachÿ echÿu echÿu tike- chikechunubechÿ cabeza seco = and there is another one to which they say cabeza seco, jxx-a120516l-a.243 S,O
% porque echÿu chinajiku echÿu tinijabaijane micha, cux-c120414ls-2.070

Headless RCs are regularly introduced by a demonstrative. They constitute an NP\is{noun phrase} without a noun and thus can be analysed as conominal\is{conomination} arguments\is{argument} themselves. 

(\ref{ex:RC-S}) comes from the story about the lazy man told by Miguel. Once it is established that the story is about a lazybones, a relative clause with a stative verb ascribing the characteristic of being lazy to him is sufficient to refer to the main character of the story. No head is needed; the RC stands for the \isi{head} itself.

\ea\label{ex:RC-S}
\begingl
\glpreamble \textup{S\textsubscript{MC}/S\textsubscript{RC}}\\i punachÿ tijai takumurauchujitu \textup{[}echÿu tipÿkubai\textup{]}\\
\gla i punachÿ tijai ti-akumurauchu-ji-tu echÿu ti-pÿkubai\\
\glb and other day 3i-accomodate-\textsc{rprt}-\textsc{iam} \textsc{dem}b 3i-be.lazy\\
\glft ‘and the next day the lazybones prepared (i.e. he packed some things to take with him)’
\endgl
\trailingcitation{[mox-n110920l.082]}
\xe
%relativised entity = S of RC, S of MC

Another example with a stative verb is (\ref{ex:RC-S2}). In this case, the headless RC with the verb \textit{tikipÿpa} ‘it is white’ stands for white hair. The sentence was produced by Juana, when explaining me the effects that palm fruit oil has on their hair.

\ea\label{ex:RC-S2}
\begingl
\glpreamble \textup{S\textsubscript{MC}/S\textsubscript{RC}}\\i tejekupubu \textup{[}echÿu tikipÿpa\textup{]}\\
\gla  i ti-jekupu-bu echÿu ti-kipÿpa\\
\glb and 3i-lose-\textsc{mid} \textsc{dem}b 3i-be.white\\
\glft ‘and the white ones vanish’
\endgl
\trailingcitation{[jxx-d181102l.32]}
\xe

The verb in the headless RC does not need to be stative. (\ref{ex:RC-Oo}) has an active transitive verb and the role of the relativised item in the RC is O in this case. This sentence was elicited, so there is not much context.

\ea\label{ex:RC-Oo}
\begingl
\glpreamble \textup{O\textsubscript{MC}/O\textsubscript{RC}}\\numa \textup{[}eka niyÿseiku\textup{]}\\
\gla nÿ-uma eka ni-yÿseiku\\
\glb 1\textsc{sg}-take.\textsc{irr} \textsc{dem}a 1\textsc{sg}-buy\\
\glft ‘I take what I bought’
\endgl
\trailingcitation{[jxx-e191021e-2]}%el.
\xe
%relativised entity = O of RC, O of RC



Headless RCs, like any other argument (see \sectref{sec:WordOrder}), usually follow the verb, but they can also precede it.\is{word order} This is much more common for headless RCs than for the headed ones.


The RC in (\ref{ex:RC-comeshout}) has an intransitive active verb and it precedes the main clause verb. It comes from the story about the two men who meet the devil in the woods as told by Miguel. The one who arrives shouting is none other than the devil.

\ea\label{ex:RC-comeshout}
\begingl
\glpreamble \textup{S\textsubscript{RC}/S\textsubscript{MC}}\\\textup{[}echÿu tiyÿbuikÿupunu\textup{]} titupunubutu\\
\gla echÿu ti-yÿbui-kÿupunu ti-tupunubu-tu\\
\glb \textsc{dem}b 3i-shout-\textsc{am.conc.cis} 3i-arrive-\textsc{iam}\\
\glft ‘the one who came shouting arrived now’
\endgl
\trailingcitation{[mxx-n101017s-1.032]}
\xe

%bupuna echÿu tipitanÿikukunube tepuikanube = traemos esos que se abrazan a pecar, jrx-c151001fls-9.58
%tosetuji chisamuikunu- chisamunubetuji echÿu tiyÿbuikÿupununubetuji: "jia jÿa jia jia vamo vamo!" tiyÿbuikÿupununubetuji, mxx-n151017l-1.86
%kuina tinijanea eka tikipÿpanaji = los blancos (pollos) no lo comen (el maíz entero), jxx-e150925l-1.143

In (\ref{ex:RC-notripe}), the relativised item is the O of the RC. This example was elicited from María S. and is about eating unripe fruit.

\ea\label{ex:RC-notripe}
\begingl
\glpreamble \textup{O\textsubscript{RC}/S\textsubscript{MC}}\\\textup{[}echÿu pinika\textup{]} kuinakuÿ, nÿmayu tisachu tayu\\
\gla echÿu pi-nika kuina-kuÿ nÿmayu ti-sachu ti-a-yu\\
\glb \textsc{dem}b 2\textsc{sg}-eat.\textsc{irr} \textsc{neg}-\textsc{incmp} just 3i-want 3i-\textsc{irr}-be.ripe\\
\glft ‘the one you want to eat is not [ripe] yet, it is just about to get ripe’
\endgl
\trailingcitation{[rxx-e181022le]}
\xe

In (\ref{ex:RC-CLF}), the relativised item is the S of the RC and the RC constitutes the object of the main clause. Note that there is a \isi{classifier} on the verb in the RC. This classifier, \textit{-ji} ‘\textsc{clf:}soft.mass’, gives a hint about the identity of the relativised item, the clothes that have got wet.\footnote{See also (\ref{ex:headed-RC-2}) above, which the speaker had uttered shortly before, where the head noun is \textit{bimÿu} ‘our clothes’.} However, unlike in other languages \citep[cf.][]{Epps2012}, the classifier cannot be analysed as the \isi{head} of the RC, since classifiers\is{classifier} always attach at word level, not at clause level in Paunaka, i.e. its presence on the verb in the RC is solely due to requirements of the verb not of the construction.

\ea\label{ex:RC-CLF}
\begingl
\glpreamble \textup{S\textsubscript{RC}/O\textsubscript{MC}}\\\textup{[}echÿu timÿuji\textup{]} aparte chetukunube\\
\gla echÿu ti-mÿu-ji aparte chÿ-etuku-nube\\
\glb \textsc{dem}b 3i-be.wet-\textsc{clf:}soft.mass aside 3-put-\textsc{pl}\\
\glft ‘the wet things (i.e. clothes), they put aside’
\endgl
\trailingcitation{[jxx-p151016l-2.132]}
\xe
% relativised entity = S of RC, O of MC


(\ref{ex:RC-A}) is an example in which the relativised item is the A of the RC. Miguel speaks about the teacher he had at school here. He is one of the very few Paunaka speakers who can read and write. It was his own decision to go to school back in those days, when he lived with his family in \isi{Altavista}. Physical punishment was not unusual at that time. 

\ea\label{ex:RC-A}
\begingl
\glpreamble \textup{A\textsubscript{RC}/O\textsubscript{MC}}\\\textup{[}echÿu chichupu echÿu chitareane\textup{]} kuina cheistaka\\
\gla echÿu chi-chupu echÿu chi-tarea-ne kuina chÿ-eistaka\\
\glb \textsc{dem}b 3-know \textsc{dem}b 3-excercise-\textsc{possd} \textsc{neg} 3-whip\\
\glft ‘he did not whip the one who knew his schoolwork’
\endgl
\trailingcitation{[mxx-p181027l-1.076]}
\xe
%relativised entity = A of RC, O of MC

RCs introduced by a demonstrative can be individually negated.\is{negation} This is true for the headed ones as well as for the headless ones as can be seen in the following two examples.

(\ref{ex:RC-maths}) is an equative clause in which one entity consisting of the head \textit{chÿnajiku} ‘the only thing’ and its RC is equated to another entity.

The sentence was produced by Miguel who told me about his experience in school.

\ea\label{ex:RC-maths}
\begingl
\glpreamble \textup{S\textsubscript{MC}/O\textsubscript{RC}}\\chÿnajiku \textup{[}echÿu kuina nisumacha nechÿu\textup{]} matematica\\
\gla chÿna-jiku echÿu kuina ni-sumacha nechÿu matematica\\
\glb one-\textsc{lim}1 \textsc{dem}b \textsc{neg} 1\textsc{sg}-like.\textsc{irr} \textsc{dem}c mathematics\\
\glft ‘the only thing that I didn’t like there was maths’
\endgl
\trailingcitation{[mxx-p181027l-1.090]}
\xe

In (\ref{ex:RC-blind}), the RC stands on its own. It was produced by Juana for disambiguation between the different daughters of Miguel.

\ea\label{ex:RC-blind}
\begingl
\glpreamble \textup{S\textsubscript{RC}}\\jaa, \textup{[}echÿu kuina taimubÿkemÿnÿ\textup{]}\\
\gla jaa echÿu kuina ti-a-imubÿke-mÿnÿ\\
\glb \textsc{afm} \textsc{dem}b \textsc{neg} 3i-\textsc{irr}-see.well-\textsc{dim}\\
\glft ‘yes, the one who cannot see well (i.e. is blind)’
\endgl
\trailingcitation{[jxx-p120430l-1.087]}
\xe\is{nominal demonstrative|)}

In all of the examples above, the relativised items have S, O or A roles in the RCs. I have also found one example, given here as (\ref{ex:RC-oblob}), in which it is an \isi{oblique} (in both the MC and the RC). However, this was uttered with some hesitation by Juana, with \textit{te} being a marker of hesitation here rather than a connective for a subsequent event. She speaks about a place her grandparents reach on their journey back from Moxos. If I understood her right, the rubber workers’ going to this place long preceded the arrival of the grandparents; any other interpretation of the context would be odd. In addition, \textit{te} is surrounded by pauses and forms its own intonation unit.\footnote{For the sake of better readability, I have usually deleted all hesitation marks and false starts from the examples in this grammar. I maintain it here, because this example is somehow exceptional for encoding \isi{oblique} function and not being fully integrated in the intonational contour of the main clause.} If the relativised entity is an oblique of the RC, speakers seem to prefer a \isi{deranked verb} (but see the discussion in \sectref{sec:RC-Subord}). In any case, Juana speaks about a place to which her grandparents arrived on their journey back home from Moxos.

\ea\label{ex:RC-oblob}
\begingl
\glpreamble \textup{X\textsubscript{MC}/X\textsubscript{RC}}\\titupunubunube – te – \textup{[}eka komerunube tiyununubetu\textup{]}\\
\gla ti-tupunubu-nube te eka komeru-nube ti-yunu-nube-tu\\
\glb3i-arrive-\textsc{pl} \textsc{seq} \textsc{dem}a rubber.worker-\textsc{pl} 3i-go-\textsc{pl}-\textsc{iam} \\
\glft ‘they arrived – er – to where the rubber workers had gone’
\endgl
\trailingcitation{[jxx-p151016l-2.255-256]}
\xe
%relativised entity = OBL of RC, X of MC

Unmarked headed RCs and headless RCs introduced by a demonstrative are the most common types of RCs found in Paunaka. The next section is dedicated to a less common type, which includes a deranked verb.

\subsection{Relative clauses with deranked verbs}\label{sec:RC-Subord}
\is{deranked verb|(}
\is{oblique|(}

Some RCs build on a deranked verb (see \sectref{sec:TypesClauseCombining}). This is most typically found with RCs in which the relativised item is an oblique (X), but occasionally also with objects. Most RCs with deranked verbs are actually part of cleft constructions, but in this section some examples which do not contain clefts are discussed. For clefts containing RCs with deranked verbs see \sectref{sec:Clefts}.

(\ref{ex:hotel-RC}) is the result of Miguel and Juana looking for an expression for ‘hostel’ that does not contain a Spanish loan. Swintha had asked for a sentence to tell them that she would go back to her hostel. A pause occurred between the adverb \textit{nauku} and the subordinate verb, marked by a comma in the example. It is thus not entirely clear whether \textit{nauku} can be analysed as the \isi{head} of the RC or whether the RC should be analysed as headless and simply added to the main clause as an alternative, more explicit expression for the place. In any case, the relativised item has locative function in the RC.

\ea\label{ex:hotel-RC}
\begingl
\glpreamble  \textup{X\textsubscript{MC}/X\textsubscript{RC}}\\niyunupunatu nauku, \textup{[}nÿmukiu yuti\textup{]}\\
\gla ni-yunupuna-tu nauku nÿ-muk-i-u yuti\\
\glb 1\textsc{sg}-go.back.\textsc{irr}-\textsc{iam} there 1\textsc{sg}-sleep-\textsc{subord}-\textsc{real} night\\
\glft ‘now I’ll go back there, where I slept at night’
\endgl
\trailingcitation{[jmx-e090727s.381s]}
\xe
%head = nauku?

The next example comes from elicitation of body parts. Asked for the ear, María S. first gives the noun \textit{nÿchukape} ‘my ear’,\footnote{Actually, this is the outer part of the ear, the pinna, which is made explicit by use of the classifier \textit{-pe} ‘\textsc{clf:}flat’. The noun can also be used without the classifier: \textit{-chuka} ‘ear, pinna’. There are some hints that the distinction between outer body part/organ and inner body part/sense may have been more meaningful at some time in the past \citep[cf.][]{TerhartDanielsenBODY}.} then she explains the function of the body part with a deranked verb, which can be interpreted as a headless RC with the role of the relativised item being that of an instrumental, and finally repeats the noun. All of these three words constitute different intonation units, signalled by use of the comma in (\ref{ex:Ear-RC}), so that we can claim that the RC is not part of the main clause but in apposition to it.

\ea\label{ex:Ear-RC}
\begingl
\glpreamble  \textup{X\textsubscript{RC}}\\nÿchukape, \textup{[}nisamuikiu\textup{]}, nÿchukape\\
\gla nÿ-chuka-pe ni-samuik-i-u nÿ-chuka-pe\\
\glb 1\textsc{sg}-ear-\textsc{clf:}flat 1\textsc{sg}-listen-\textsc{subord}-\textsc{real} 1\textsc{sg}-ear-\textsc{clf:}flat\\
\glft ‘my ear, what I listen with, my ear’
\endgl
\trailingcitation{[rxx-e121128s-4x.078-079]}
\xe

The RC in (\ref{ex:RC-toco}), in addition to having a deranked verb, is introduced by a demonstrative. The relativised item is a locative oblique in the RC, and the RC is a possessor in the main clause. The example comes from elicitation and was produced by Miguel.

\ea\label{ex:RC-toco}
\begingl
\glpreamble  \textup{POSS\textsubscript{MC}/X\textsubscript{RC}}\\chija \textup{[}eka nitibuia\textup{]}\\
\gla chi-ija eka ni-tibu-i-a\\
\glb 3-name \textsc{dem}a 1\textsc{sg}-sit.down-\textsc{subord}-\textsc{irr}\\
\glft ‘it is the name of where I can sit down’
\endgl
\trailingcitation{[rmx-e150922l.126]}
\xe

An example with a negated RC is (\ref{ex:RC-kurichi}), which comes from Miguel telling the story about the cowherd and the spirit of the hill. The negative particle directly precedes the deranked verb and has only scope over the RC, not over the main clause.

\ea\label{ex:RC-kurichi}
\begingl 
\glpreamble \textup{S\textsubscript{MC}/X\textsubscript{RC}}\\pero nechÿuji estansiayae kakuji chÿnachÿ kurichi \textup{[}kuina tijbÿkiapu echÿu ÿne\textup{]}\\
\gla pero nechÿu-ji estansia-yae kaku-ji chÿnachÿ kurichi kuina ti-jibÿk-i-a-pu echÿu ÿne\\ 
\glb but \textsc{dem}c-\textsc{rprt} manor-\textsc{loc} exist-\textsc{rpt} one pond \textsc{neg} 3i-smoke-\textsc{subord}-\textsc{irr}-\textsc{mid} \textsc{dem}b water\\ 
\glft ‘but there, it is said, on the manor, it is said, there was a pond where the water never dried (i.e. evaporated)’\\ 
\endgl
\trailingcitation{[mxx-n151017l-1.06-07]}
\xe
\is{oblique|)}

Sometimes a deranked verb also occurs, when the relativised item has other functions than oblique or more precisely is an object.\is{object} This is the case in (\ref{ex:SUB-RC-OBJ-2}). It is not entirely clear to me what triggers the use of deranked verbs in these sentences. It is possible that for relativisation of objects in headless RCs, both strategies can be used, either with a demonstrative and a balanced verb\is{finite verb} or a deranked verb. 

Consider (\ref{ex:SUB-RC-OBJ-2}), which is an example with a \isi{ditransitive} verb. The sentence, which was elicited from Juana, was produced with some hesitation, as apparently the speaker had problems in choosing a word that would match the Spanish \textit{regalo} ‘present, gift’.


 %The hesitation in the utterance, the short break the speaker needs to think about how to best express what my colleague Lena had asked her to translate, may just be too much to integrate the RC into the intonation contour of the MC, and this is repaired by using a deranked verb. This may ultimately also be the reason why a deranked verb was used in (\ref{ex:hotel-RC}) and (\ref{ex:Ear-RC}). Analysing information structure and flow might help to better understand the use of deranked verbs, in general, but this is a topic for future research.

\ea\label{ex:SUB-RC-OBJ-2}
\begingl
\glpreamble \textup{O\textsubscript{MC}/O\textsubscript{RC}}\\numa eka \textup{[}nipunakiachÿ eka mimi\textup{]}\\
\gla nÿ-uma eka ni-punak-i-a-chÿ eka mimi\\
\glb 1\textsc{sg}-take.\textsc{irr} \textsc{dem}a 1\textsc{sg}-give-\textsc{subord}-\textsc{irr}-3 \textsc{dem}a mum\\
\glft ‘I’m going to take what I will give to mum’
\endgl
\trailingcitation{[jxx-e191021ls-2]}
\xe

A second example of an RC with a deranked verb in which the relativised item has the role of O is (\ref{ex:RC-cowcow}). It comes from Miguel’s story about the man and the spirit of the hill and is a citation of what the spirit says to the man.

\ea\label{ex:RC-cowcow}
\begingl 
\glpreamble  \textup{S\textsubscript{MC}/O\textsubscript{RC}}\\tikechuchÿji: “kaku naka nubiuyae kaku naka \textup{[}echÿu pisemaikiuchi\textup{]} echÿu bakajane kaku”\\
\gla ti-kechu-chÿ-ji kaku naka nÿ-ubiu-yae kaku naka echÿu pi-semaik-i-u-chi echÿu baka-jane kaku\\ 
\glb 3i-say-3-\textsc{rprt} exist here 1\textsc{sg}-house-\textsc{loc} exist here \textsc{dem}b 2\textsc{sg}-search-\textsc{subord}-\textsc{real}-3 \textsc{dem}b cow-\textsc{distr} exist\\ 
\glft ‘he said to him, it is said: “they are here in my house, here is what you are looking for, the cows, they are there”’\\ 
\endgl
\xe

% (\ref{ex:SUB-RC-O-1}) is an example of a headed RC with a deranked verb. The RC does not follow its head noun directly, the adverb \textit{naka} ‘here’ intervenes. Intonation suggests that the sentence is completed with \textit{naka}, and the RC is an afterthought, albeit attached without a pause. The example is from Miguels description of how he learnt to read and write. His teacher suggested that his father made a wooden plate for him to write on, since exercise books were only sold to \textit{karay}. Pencils on the other hand, indigenous people could buy. The clause could alternatively also be analysed as a cause clause (‘because we could buy them in town’, see {sec:CauseConsequence} for cause clauses).
%
%\ea\label{ex:SUB-RC-O-1}
%\begingl
%\glpreamble kaku lapi naka, \textup{[}biyÿseikia unekuyae\textup{]}\\
%\gla kaku lapi naka bi-yÿseik-i-a uneku-yae\\
%\glb exist pencil here 1\textsc{pl}-buy-\textsc{subord}-\textsc{irr} town-\textsc{loc}\\
%\glft ‘we had pencils here, which we could buy in town’\\
%\endgl
%\trailingcitation{[mxx-p181027l-1.026]}
%\xe


Finally, there are also some cases where it is not clear whether we better analyse them as RCs or as adverbial clauses\is{adverbial relation} (see \sectref{sec:SubordinateACs}). The following example could either be analysed as an adverbial clause\is{adverbial relation} expressing purpose or as a headed relative clause expressing an oblique. In case that we decide to analyse it as an RC, the head noun has the role of S in the main clause, and of an instrumental oblique in the RC. The example comes from Juana’s description of how to make a clay pot. Apparently, she had forgotten to collect a special stone in order to polish the pot.

\ea\label{ex:stone-lack}
\begingl
\glpreamble  \textup{S\textsubscript{MC}/X\textsubscript{RC}}\\paltau echÿu mai \textup{[}biyeneukiachÿ\textup{]}\\
\gla paltau echÿu mai bi-yeneuk-i-a-chÿ\\
\glb lack \textsc{dem}b stone 1\textsc{pl}-polish-\textsc{subord}-\textsc{irr}-3\\
\glft ‘a stone with which we can polish it was missing’\\ or: ‘a stone was missing in order to polish it’
\endgl
\trailingcitation{[jxx-d110923l-1.08-09]}
\xe


\subsection{Cleft constructions}\label{sec:Clefts}\is{cleft|(}

A cleft construction is usually defined as “a complex sentence structure consisting of a matrix clause headed by a copula and a relative or relative-like clause whose relativized argument is coindexed with the predicative argument of the copula. Taken together, the matrix and the relative express a logically simple proposition, which can also be expressed in the form of a single clause without a change in truth conditions” \citep[467]{Lambrecht2001}.\is{copula|(}

This definition is problematic for Paunaka insofar as the clauses analysed here as clefts do not contain a copula.\footnote{The copula \textit{kaku} is used to express location at a place or existence (in contrast to non-existence, which is usually expressed without copula), but it is not used to link non-verbal predicates of other kinds to their subjects (see \sectref{sec:NonVerbalPredication}).}\is{copula|)} They could thus only be defined as clefts by a broad definition, such as the one proposed by \citet[1]{PalancarVanhove2020} that clefts are “biclausal focus constructions”.\is{focus} This is also in line with what \citet[463]{Lambrecht2001} identifies as the primary, defining property of clefts, that they express “a single proposition via bi-clausal syntax”.

As for their biclausal status, let us consider one example: (\ref{ex:eqRC-a}) is from an account of María S. about her childhood. The family lived by subsistence farming, but food was sometimes scarce, so their mother also harvested leaves of wild plants that were mashed together with peanuts and then cooked in a stew.

\ea\label{ex:eqRC-a}
\begingl
\glpreamble nÿenubane \textup{[}echÿu chiyejiku\textup{]}\\
\gla nÿ-enu-bane echÿu chi-yejiku\\
\glb 1\textsc{sg}-mother-\textsc{rem} \textsc{dem}b 3-tear.out\\
\glft ‘it was my late mother who tore them out (i.e. harvested the leaves of a plant)’
\endgl
\trailingcitation{[rxx-p181101l-2.241]}
\xe
%relativised noun = A

\is{nominal demonstrative|(}The verbal predicate of the RC is \textit{chiyejiku} ‘she tears them out’. If this were a monoclausal sentence, the verb would have to be analysed as the predicate of the main clause, but in that case there would be two arguments preceding it: \textit{nÿenubane} ‘my late mother’ and the demonstrative \textit{echÿu}. Paunaka, however, has only one preverbal slot (see \sectref{sec:SimpleClauses}). SVO order\is{word order} is relatively frequent, OVS is (almost) absent and SOV is not allowed at all, and we must therefore assume that sentences like (\ref{ex:eqRC-a}) consist of two clauses, the matrix clause which only contains one noun in this case and a juxtaposed RC consisting of a demonstrative and a balanced verb,\is{finite verb} a pattern we usually find in headless RCs in which the relativised item has the role of S,O or A (see \sectref{sec:HeadlessRC}).\is{nominal demonstrative|)}

Clefts are often believed to be associated with focus,\is{focus} but \citet[5]{DelinOberlander2005} reject this, since the positions of given and new information in the cleft are not fixed (depending on the type of cleft). Looking at clefts in context, they identify four distinctive features of clefts: “uniqueness, stativising, presupposition, and separate information structure” \citep[1]{DelinOberlander2005}. All of these discourse features hold for the Paunaka cleft construction.

The cleft construction can be analysed as a subcategory of equative or proper-inclusion clauses\is{equative/proper inclusion clause} in Paunaka (see \sectref{sec:PropInclEquatAttr}), the difference being that instead of a noun they take an RC as predicate and this predicate usually follows rather than precedes the subject.\is{word order} It is thus clearly a stative construction. Uniqueness is also given, since the clefted constituent is interpreted to be the only one to which the predication of the RC applies. Thus, in (\ref{ex:eqRC-a}) above, it is only the mother who harvested the leaves. In (\ref{ex:eqRC-s}), the uniqueness is even expressed lexically. The example is about fish. Swintha had found a dried piranha at the shore of the big water reservoir of Concepción, and she showed it to Clara and María C., wondering which kind of piranha that was, since there are different classes of piranhas. However, Clara first claimed with (\ref{ex:eqRC-s}) that there was only one class of piranhas. (María C. subsequently corrected her and explained to us that there are indeed different classes of piranha, which have different colours.)

\ea\label{ex:eqRC-s}
\begingl
\glpreamble porke echÿu chinajiku \textup{[}echÿu tinijabaijane micha\textup{]} \\
\gla porke echÿu china-jiku echÿu ti-nijabai-jane micha\\
\glb because \textsc{dem}b one-\textsc{lim}1 \textsc{dem}b 3i-bite-\textsc{distr} good\\
\glft ‘because it is only these ones that bite hard’
\endgl
\trailingcitation{[cux-c120414ls-2.070]}
\xe

The order of clefted constituent and RC can also be reversed, but this is very rare. One example is (\ref{ex:eqRC-reverse}), which was produced by Miguel when talking with Juan C. about the past. The clefted constituent is used to identify their teacher, Mother Trinidad.

\ea\label{ex:eqRC-reverse}
\begingl
\glpreamble \textup{[}echÿu timesumeikubane naka turno unekuyae\textup{]} madre Trinidad, chibu\\
\gla echÿu ti-mesumeiku-bane naka turno uneku-yae {madre Trinidad} chibu\\
\glb \textsc{dem}b 3i-teach-\textsc{rem} here shift town-\textsc{loc} {Mother Trinidad} 3\textsc{top.prn}\\
\glft ‘the one who taught that school year here in town was Mother Trinidad, it was her’
\endgl
\trailingcitation{[mqx-p110826l.241-24]}
\xe


Presupposition in cleft clauses is about the predication of the RC being factitive, i.e. in (\ref{ex:eqRC-s}) it is presupposed that there are some fish that bite well, in (\ref{ex:eqRC-a}) there is someone who harvests leaves, and in (\ref{ex:eqRC-reverse}) there is someone who taught in school. 

As for the “separate information structure”, \citet[8]{DelinOberlander2005} propose that clefts always encode some new information, but never encode all new information. Their function then is “to make special and specific links with the preceding discourse”.

%mupÿinube echÿu kapunu, rxx-n120511l-2.26

% (\ref{ex:eqRC-a}) and (\ref{ex:eqRC-s}) have a cleft phrase that is focussed; however, cleft constructions very often contain topical cleft phrases. In those cases a topic pronoun is chosen.\footnote{It may seem strange that a topic pronoun is used to express a topic here, but the same pronoun is found in non-clefted structures and is clearly connected to focus then, see \sectref}

Just like in English it-clefts, the clefted constituent in Paunaka can encode old or new information. (\ref{ex:eqRC-a}) and (\ref{ex:eqRC-s}) both are examples of “topic clause clefts”,\is{topic} in which the clefted constituent encodes the new information. In (\ref{ex:eqRC-reverse}), what is new is rather the relation between the given information in the clefted constituent and the juxtaposed NP, which also denotes a known participant. This is an example of a “comment cleft clause” \citep[cf.][9]{DelinOberlander2005}. I will present some more examples of comment clause clefts below, in which the information in the clefted constituent is given. In this type of clefts, Paunaka speakers often make use of the topic pronoun\is{topic pronoun|(} \textit{chibu}.\footnote{The topic pronoun \textit{chibu} shows up in different kinds of constructions that all have to do with distinct information structure. It is not limited to marking \isi{topic}. See \sectref{sec:FocPron} for more information.} Just like in (\ref{ex:eqRC-reverse}), the new information is the relation of the given participant with a predication whose content may also be given or may be new. One example is (\ref{ex:eqRC-o}), which is about the scarcity of food.\footnote{A dish without meat is not a good dish in rural lowland Bolivia in general.} María C. had just listed names of different fruits and vegetables that they grow, which is what \textit{chibu} refers to.

\ea\label{ex:eqRC-o}
\begingl
\glpreamble chibu \textup{[}echÿu binika\textup{]}, kuina chÿecheina\\
\gla chibu echÿu bi-nika kuina chÿeche-ina\\
\glb 3\textsc{top.prn} \textsc{dem}b 1\textsc{pl}-eat.\textsc{irr} \textsc{neg} meat-\textsc{irr.nv}\\
\glft ‘this is what we can eat, there is no meat’
\endgl
\trailingcitation{[uxx-p110825l.197]}
\xe


Like in (\ref{ex:eqRC-o}), the clefted constituent in these “\textit{chibu} cleft constructions” is typically the \isi{object} of the relative clause, but there are also a few examples in which it is the \isi{subject}, as in (\ref{ex:cleft-Lena}). In this example, Miguel states that another linguist (me) would write about their language. The recording was made a few days before I actually arrived in Bolivia to work with the Paunaka speakers for the first time. Swintha and Federico had already told Miguel about my impending arrival, and he spread the news to María C.

\ea\label{ex:cleft-Lena}
\begingl
\glpreamble chibu \textup{[}echÿu tisuikatu echÿu betea paunaka\textup{]}\\
\gla chibu echÿu ti-suika-tu echÿu bi-etea paunaka\\
\glb 3\textsc{top.prn} \textsc{dem}b 3i-write.\textsc{irr}-\textsc{iam} \textsc{dem}b 1\textsc{pl}-language Paunaka\\
\glft ‘it is her who will write our language Paunaka’
\endgl
\trailingcitation{[ump-p110815sf.116]}
\xe

%check: jxx-p120515l-2.116


Similar sentences can also be formed with deranked verbs. This is the case if the relativised item has the role of an oblique in the RC. Just like the clefts built on RCs with balanced verbs, all of the examples that follow can be analysed as a kind of equative clause in which two obliques are equated.\is{equative/proper inclusion clause} The deranked verb is preceded by a demonstrative,\is{nominal demonstrative} so the sentence seems to exhibit the same biclausal structure as a cleft construction with a balanced verb. Another analogy is that just like the topic pronoun \textit{chibu}, the oblique topic pronoun \textit{nebu} can introduce these sentences. 

This is the case in (\ref{ex:tiger-cleft}), which comes from Juana. With this sentence, she provides the conclusion of Miguel’s story about the fox tricking the jaguar by making him believe that the reflection of the moon in the water was a big piece of cheese. The naive jaguar jumps into the water with a stone tied on his hands and consequently dies. In connection with \textit{nebu}, a deranked verb is used and in addition, this RC is also introduced by the demonstrative \textit{eka}.

\ea\label{ex:tiger-cleft}
\begingl
\glpreamble i nebu \textup{[}eka chÿpakiu isini\textup{]}\\
\gla i nebu eka chÿ-pak-i-u isini\\
\glb and 3\textsc{obl.top.prn} \textsc{dem}a 3-die-\textsc{subord}-\textsc{real} jaguar\\
\glft ‘and this is how the jaguar died’
\endgl
\trailingcitation{[jmx-n120429ls-x5.265]}
\xe
\is{topic pronoun|)}

(\ref{ex:deer-RC}) is an example in which two places are equated, the place the boy climbs and the place on the deer’s head. This sentence comes from Miguel in telling the \isi{frog story} to José. It was produced with some hesitation, though. As in (\ref{ex:tiger-cleft}) above, the deranked verb is introduced by a demonstrative.

\ea\label{ex:deer-RC}
\begingl
\glpreamble pero \textup{[}eka chipuniu naka\textup{]} chichÿtiyae echÿu cierbo\\
\gla eka chi-pun-i-u naka chi-chÿti-yae echÿu cierbo\\
\glb \textsc{dem}a 3-go.up-\textsc{subord}-\textsc{real} here 3-head-\textsc{loc} \textsc{dem}b deer\\
\glft ‘but where he climbs here is on the head of the deer’
\endgl
\trailingcitation{[mox-a110920l-2 124]}
\xe
%\textup{X\textsubscript{RC}/S\textsubscript{MC}}\\

There are some examples though which are similar to the ones just presented but lack a demonstrative.\is{nominal demonstrative} One example including \textit{nebu} is given in (\ref{ex:nebu-Cleft}) and one without \textit{nebu} in (\ref{ex:hammock-RC}).

In (\ref{ex:nebu-Cleft}), Juana explains that they used clay pots for cooking in the old times.

\ea\label{ex:nebu-Cleft}
\begingl
\glpreamble banau echÿu muteji nÿkÿiki, nebu \textup{[}biyÿtikiapu\textup{]}\\
\gla bi-anau echÿu muteji nÿkÿiki nebu bi-yÿtik-i-a-pu\\
\glb 1\textsc{pl}-make \textsc{dem}b loam pot 3\textsc{obl.top.prn} 1\textsc{pl}-set.on.fire-\textsc{subord}-\textsc{irr}-\textsc{mid}\\
\glft ‘we made clay pots, this is what we could cook with’
\endgl
\trailingcitation{[jxx-d110923l-2.20]}
\xe

(\ref{ex:hammock-RC}) is similar to (\ref{ex:deer-RC}) in that two places are equated, the place inside the hammock and the place where the subject sleeps best. This example was elicited from María S.

\ea\label{ex:hammock-RC}
\begingl
\glpreamble yumajikÿye \textup{[}maj mejor nÿmukiu\textup{]}\\
\gla yumaji-kÿ-yae {maj mejor} nÿ-muk-i-u\\
\glb hammock-\textsc{clf:}bounded-\textsc{loc} {best} 1\textsc{sg}-sleep-\textsc{subord}-\textsc{real}\\
\glft ‘in the hammock is where I best sleep’
\endgl
\trailingcitation{[rmx-e150922l.118]}
\xe
%\textup{S\textsubscript{MC}/X\textsubscript{RC}}\\


If we consider examples like (\ref{ex:nebu-Cleft}) and (\ref{ex:hammock-RC}), where there is no demonstrative\is{nominal demonstrative} to clearly separate the RC from its matrix clause, it becomes questionable whether we are still dealing with a biclausal structure at all. What they have in common with clefts is the equation of two items one of which contains a \isi{verb}. (\ref{ex:tiger-cleft}) and (\ref{ex:deer-RC}) could be analysed as clefts in analogy to the examples presented in the beginning of this section. (\ref{ex:nebu-Cleft}) and (\ref{ex:hammock-RC}) could then be analysed as clefts due to their similarity to (\ref{ex:tiger-cleft}) and (\ref{ex:deer-RC}), but it should be clear that we are moving away from what is a canonical cleft construction in Paunaka. There are constructions with deranked verbs that even move a little further towards monoclausality. I claim that they are indeed monoclausal. This is the topic of the next section.
\is{cleft|)}
\is{juxtaposition|)}
\is{relative relation|)}
\is{subordination|)}

\section{Deranked verbs in monoclausal constructions}\label{sec:AdverbialModification}
\is{focus|(}
\is{oblique|(}

In the construction described in this section, an adverb, a quantifier or a noun that has oblique function in the clause precedes a deranked verb.\is{word order} There is no other verb in the clause. This construction is mainly used to shift focus from the predicate to the adverbial expression.

Consider (\ref{ex:LOC-RC}), in which the locative adverb \textit{naka} ‘here’ precedes the verb marked by the subordinate suffix \textit{-i}. It comes from the account by María S. about how she grew up.

\ea\label{ex:LOC-RC}
\begingl
\glpreamble naka nÿjÿkiu\\
\gla naka nÿ-jÿk-i-u\\
\glb here 1\textsc{sg}-grow-\textsc{subord}-\textsc{real}\\
\glft ‘here I grew up’ (i.e. here was my growing up)
\endgl
\trailingcitation{[rxx-p181101l-2.007]}
\xe

This construction very much resembles clefts\is{cleft} with deranked verbs that contain the oblique pronoun\is{topic pronoun} or a locative-marked noun (see (\ref{ex:tiger-cleft})–(\ref{ex:hammock-RC}) in \sectref{sec:Clefts} above). The difference is that the examples in this section have preposed adverbs instead of the oblique topic pronoun or substantives not marked as obliques (e.g. by the locative marker \textit{-yae}) and there is no demonstrative\is{nominal demonstrative} before the deranked verb so that the hint that we are dealing with a biclausal structure is missing.  As I have stated before, this is a continuum, the examples presented towards the end of the last chapter already showed tendencies towards monoclausality and it is finally a matter of a decision where to draw the line.

Why do we have a deranked verb in a sentence like (\ref{ex:LOC-RC}) above? This has to do with information structure and with the kind of predication intended by the speaker. This construction has the same identifying features as \isi{cleft} constructions: “uniqueness, stativising, presupposition, and separate information structure” \citep[1]{DelinOberlander2005}. The aspect of stativising is particularly interesting. 

Prototypically, predication in a sentence is achieved by a verb, and verbs prototypically denote processes \citep[cf.][259]{Cristofaro2003}. By predicating one process after another a piece of discourse develops in time and space. However, sometimes predication is not meant to be about a process, but rather about properties or characteristics of this process. In this case, just like in subordination, “verb forms are not used in their prototypical cognitive or discourse function” \citep[257]{Cristofaro2003}. This is where deranked verbs come into play.

Deranked verbs have both verbal and nominal properties. They are considerably less finite than the verb forms we find in “normal” declarative sentences (see \sectref{sec:Subordination-i}). Nouns\is{noun} are more time-stable than verbs \citep[33]{Payne1997}, i.e. they are less dynamic, and they are cognitively processed as a whole rather than sequentially \citep[259]{Cristofaro2003}, just like the deranked verbs in the construction here. The process per se loses prominence in the predication. It is the relation between one entity and the process what predication is about. The process encoded by the deranked verb is thus not perceived dynamically but statively, and the development of discourse is stopped for a little while. The construction is then best analysed as a special form of equative clause,\is{equative/proper inclusion clause} in which a circumstance is related to a process as a whole.

A locative adverb is often combined with a deranked verb to relate a place to a sequence in the life cycle, i.e. to a process that is mostly of a longer duration and in any case meaningful for the development of the individual being. Two more examples are given below.

(\ref{ex:nauku-ld}) is a statement by Miguel about the place of birth of Alejo and Polonia.

\ea\label{ex:nauku-ld}
\begingl
\glpreamble naukubane ubupaikiu apuke\\
\gla nauku-bane e-ubupaik-i-u apuke\\
\glb there-\textsc{rem} 2\textsc{pl}-be.born-\textsc{subord}-\textsc{real} ground\\
\glft ‘there you were born’
\endgl
\trailingcitation{[mty-p110906l.035]}
\xe

(\ref{ex:naka-stay}) is also from Miguel and refers to the 1950s, when people first came to settle in Santa Rita. Miguel actually moved away later and only came back after 20 years or so, but the period of his first stay was of a certain duration and meaningful.

\ea\label{ex:naka-stay}
\begingl
\glpreamble nebutu naka bipajÿkiu pero kuinauku eka ÿneina bitÿpi\\
\gla nebu-tu naka bi-pajÿk-i-u pero kuina-uku eka ÿne-ina bi-tÿpi\\
\glb 3\textsc{obl.top.prn}-\textsc{iam} here 1\textsc{pl}-stay-\textsc{subord}-\textsc{real} but \textsc{neg}-\textsc{add} \textsc{dem}a water-\textsc{irr.nv} 1\textsc{pl}-\textsc{obl}\\
\glft ‘from that point on we stayed here, but there was no water for us either’
\endgl
\trailingcitation{[mxx-p110825l.060]}
\xe

(\ref{ex:Franka-walks}) is a statement by María S. about my daughter who was with me when I first came to work with Paunaka in 2011. The non-dynamic construal of this sentence encodes that my daughter not only walked around in Bolivia, but actually learnt to walk on her own, an important step in her development.

\ea\label{ex:Franka-walks}
\begingl
\glpreamble naka chiyuikiumÿne chijinepÿi Elena\\
\gla naka chi-yuik-i-u-mÿnÿ chi-jinepÿi Elena\\
\glb here 3-walk-\textsc{subord}-\textsc{real}-\textsc{dim} 3-daughter Lena\\
\glft ‘Lena’s daughter learnt to walk here’ (i.e. here was the walking of Lena’s daughter)
\endgl
\trailingcitation{[rxx-e121128s-1.069]}
\xe

If such kind of prolonged duration and meaningfulness of a place is not implied, i.e. the relation between the participant and the location is less meaningful, more coincidental or shorter, no deranked verb is used. The adverb usually follows the verb in this case, but can also sometimes precede it as in (\ref{ex:LOC-noRC-2}), which is given here for comparison.

Several factors distinguish this example from the ones given above. First of all, since the example comes from the description of the \isi{frog story}, \textit{naka} refers to a picture here, i.e. has a much more delimited reference than in the previous examples. Second, the fact that the boy is lying on the bed may be meaningful for the particular story or not. It is in any case nothing that supposedly has great influence on his life. If the speaker had wanted to express that this was the place, where the boy habitually lies, or that the boy was disabled so that he could not move away from this place, he would probably have chosen a deranked verb as in the previous examples, but the predication would have changed from describing a process he was identifying on the picture to a predication about a place. 

\ea\label{ex:LOC-noRC-2}
\begingl
\glpreamble naka tibebeikumÿnÿ eka aitubuchepÿimÿnÿ\\
\gla naka ti-bebeiku-mÿnÿ eka aitubuchepÿi-mÿnÿ\\
\glb here 3i-lie-\textsc{dim} \textsc{dem}a boy-\textsc{dim}\\
\glft ‘here the boy is lying (on the bed)’
\endgl
\trailingcitation{[mox-a110920l-2.034]}
\xe


It is not only locative adverbs that can be used together with a deranked verb to form a specific kind of predication. The aspectual adverb\is{temporal/aspectual} \textit{metu} ‘already, ready’ can be used in a similar fashion. Together with a \isi{finite verb}, the adverb expresses either that something is ongoing as in (\ref{ex:metu-speak}) or over as in (\ref{ex:metu-adv}). In these cases, it can be analysed as a modifier that provides some extra information about the processes encoded by the verbs.

(\ref{ex:metu-speak}) is a judgement about my competence in speaking Paunaka.

\ea\label{ex:metu-speak}
\begingl
\glpreamble kuina micha pero metu pichujikutu\\
\gla kuina micha pero metu pi-chujiku-tu\\
\glb \textsc{neg} good but already 2\textsc{sg}-speak-\textsc{iam}\\
\glft ‘(you) don’t do it well, but you already speak’
\endgl
\trailingcitation{[mrx-c120509l.068]}
\xe

(\ref{ex:metu-adv}) comes from Juana’s narration about how some of her siblings died. In this case, it refers to her brother Cristóbal, who died suddenly and unexpectedly. Juana wanted to attend the funeral, but arrived late.

\ea\label{ex:metu-adv}
\begingl
\glpreamble nitupunubu nauku, metu chimakunubetu\\
\gla ni-tupunubu nauku metu chi-maku-nube-tu\\
\glb 1\textsc{sg}-arrive there already 3-bury-\textsc{pl}-\textsc{iam}\\
\glft ‘when I arrived there, they had already buried him’
\endgl
\trailingcitation{[jxx-p120430l-2.461]}
\xe

When \textit{metu} is combined with a deranked verb, the status of the process is what the predication is about. In the examples I have found it only encodes notions of finished processes, not of ongoing ones. This is a direct consequence of the process being a stative whole rather than a dynamic process. Two examples follow.

(\ref{ex:finish-eating}) was elicited from María S.

\ea\label{ex:finish-eating}
\begingl
\glpreamble metu ninikiu, depue bichujijikubu\\
\gla metu ni-nik-i-u depue bi-chujijiku-bu\\
\glb already 1\textsc{sg}-eat-\textsc{subord}-\textsc{real} afterwards 1\textsc{pl}-talk-\textsc{mid}\\
\glft ‘I finished eating (i.e. my eating was over), afterwards we talked’
\endgl
\trailingcitation{[rxx-e181020le]}
\xe

In (\ref{ex:finish-adobe}), Juana asks María whether she and her family had finished making a number of adobe bricks.

\ea\label{ex:finish-adobe}
\begingl
\glpreamble ¿pero metu anaiu echÿu?\\
\gla pero metu e-ana-i-u echÿu\\
\glb but already 2\textsc{pl}-make-\textsc{subord}-\textsc{real} \textsc{dem}b\\
\glft ‘but have you finished doing this?’
\endgl
\trailingcitation{[jrx-c151001fls-9.69]}
\xe

It is not only adverbs that can be combined with a deranked verb, but also nouns. In a corresponding declarative sentence, they would usually have the syntactic roles of obliques, but occasionally also of objects.\is{object} Speakers usually translate these sentences to Spanish by use of a construction with a fronted or sometimes with a left-dislocated constituent. As for the position preceding the verb, Paunaka shares this feature with fronting and \isi{left dislocation}. Remember that the position preceding the verb\is{word order} is associated with special discourse status in Paunaka, but placing a constituent in this position does not necessarily entail that a deranked verb is used (see \sectref{sec:WordOrder}). Left dislocation is “motivated by the need to talk about (topic) or assert (focus) a referent which entertains a low degree of cognitive accessibility in the mind of the addressee” \citep[40]{Westbury2016},\footnote{Note that the approach by \citet[]{Westbury2016} towards left dislocation is different from others \citep[e.g.][]{Lambrecht2001a} in that it explicitly does not restrict \isi{left dislocation} to the function of \isi{topicalisation}, but also recognises a focusing function.} and this is also true for the Paunaka construction. However, by using a deranked verb, the speakers additionally construe the predication as stative, and this stativeness adds to the effect of emphasising the preposed constituent by de-emphasising the process.

Consider (\ref{ex:by-foot}), which comes from the account of her grandparents’ journey to Moxos by Juana. She had just started to tell the story, then she interrupted the storyline to produce this sentence, pronouncing it with a high pitch, which additionally set it apart from the previous utterance. By doing so she emphasised that her grandparents did all of the journey by foot. This kind of sentence has the same effect \isi{left dislocation} has in other languages: it “causes a disruption in the processing of the discourse” \citep[39]{Westbury2016}.

\ea\label{ex:by-foot}
\begingl
\glpreamble ¡epuke chiyuikiunube!\\
\gla epuke chi-yuik-i-u-nube\\
\glb ground 3-walk-\textsc{subord}-\textsc{real}-\textsc{pl}\\
\glft ‘by foot they went!’ (lit.: ‘[on] ground they walked!’)
\endgl
\trailingcitation{[jxx-p151016l-2.022]}
\xe

After the utterance in (\ref{ex:by-foot}), some disruption of the story followed, because I was asking for the difference between the forms \textit{epuke} and \textit{apuke} ‘ground, down’ (there is none according to Juana; it is just free variation). Juana then took up the storyline again by repeating the proposition of (\ref{ex:by-foot}), but this time with the oblique noun following the verb, which is thus not deranked, see (\ref{ex:by-foot-2}).

\ea\label{ex:by-foot-2}
\begingl
\glpreamble tiyuikunube epuke\\
\gla ti-yuiku-nube epuke\\
\glb 3i-walk-\textsc{pl} ground\\
\glft ‘they went by foot’
\endgl
\trailingcitation{[jxx-p151016l-2.028]}
\xe

Another example with a preposed locative expression is (\ref{ex:kurichi-search}), which comes from Miguel telling the story about the cowherd and the spirit of the hill. Here it is emphasised that the cowherd went far to look for the cows of his \textit{patrón}, until the other side of the pond or lake without finding them.

\ea\label{ex:kurichi-search}
\begingl 
\glpreamble ... punachÿ chÿakenechÿu kurichi chisemaikiujanechi bakajane\\
\gla punachÿ chÿ-akene-chÿu kurichi chi-semaik-i-u-jane-chi baka-jane\\ 
\glb  other 3-non.vis.side-\textsc{dem}b pond 3-search-\textsc{subord}-\textsc{real}-\textsc{distr}-3 cow-\textsc{distr}\\ 
\glft ‘... on the other side of the pond he searched for the cows’\\ 
\endgl
\trailingcitation{[mxx-n151017l-1.15]}
\xe


In the following example a temporal expression is preposed to a deranked verb. A few days after she first told the story about her grandparents, Juana told it again prompted by questions about some words that I had not understood. This time, she emphasised another aspect of the exhausting journey, the duration. The structure is the same as in (\ref{ex:by-foot}) and (\ref{ex:kurichi-search}) above: first comes the oblique, then the deranked verb follows, in this case accompanied by another oblique that modifies it.

\ea\label{ex:SUBORD-INTRO}
\begingl 
\glpreamble chinachÿkena kuje chiyuikiunube epuke\\
\gla chinachÿ-kena kuje chi-yuik-i-u-nube epuke\\ 
\glb one-\textsc{uncert} moon 3-walk-\textsc{subord}-\textsc{real}-\textsc{pl} ground\\ 
\glft ‘it was probably one month that they went by foot’\\ 
\endgl
\trailingcitation{[jxx-e150925l-1.196]}
\xe

%tijainube chichujikiu, he spoke every day, jxx-a120516l-a.238
\is{oblique|)}

(\ref{ex:inpack}) comes from Miguel’s story about the cowherd. He looks for his cows in the woods, but only finds their tracks. Note that the nominal modifier of the noun takes an associated motion marker\is{associated motion|(} here.

\ea\label{ex:inpack}
\begingl
\glpreamble pero echÿu chibujane tropakÿu chiyuikiujane \\
\gla pero echÿu chÿ-ibu-jane tropa-kÿu chi-yuik-i-u-jane \\ 
\glb but \textsc{dem}b 3-foot-\textsc{distr} pack-\textsc{am.conc.tr} 3-walk-\textsc{subord}-\textsc{real}-\textsc{distr} \\ 
\glft ‘but their feet (i.e. footprints) walk in a pack’\\ 
\endgl
\trailingcitation{[mxx-n151017l-1.19]}
\xe

If we consider (\ref{ex:inpack}) with the associated motion marker on the nominal modifier, a marker that usually attaches to verbs, the clauses that were analysed as encoding adverbial relations\is{adverbial relation} are not far. One of them is repeated here as  (\ref{ex:hungry-go-2}) (from (\ref{ex:hungry-go}) in \sectref{sec:EmbeddedAC_bare}, and consider also (\ref{ex:stat-i-1}) as well as (\ref{ex:scratch-bark-1}), which do not have associated motion markers but are otherwise very similar).


\ea\label{ex:hungry-go-2}
\begingl
\glpreamble tikunipapakÿu chiyuniu\\
\gla ti-kunipa-pakÿu chi-yun-i-u\\
\glb 3i-be.hungry-\textsc{am.conc.tr} 3-go-\textsc{subord}-\textsc{real}\\
\glft ‘hungry he went going’
\endgl
\trailingcitation{[jmx-n120429ls-x5.222]}
\xe

Here we have a biclausal structure again, with two verbal predicates, the finite one preceding the deranked one.\is{associated motion|)} This is where the circle closes. What almost all constructions with a deranked verb have in common is that something preceding the deranked verb is highlighted as if to signal: keep in mind what preceded me, this is (the) important information in this sentence.\is{focus|)} A thorough analysis of information structure may help to learn more about the omnipresent versatile marker \textit{-i}, but this is a topic for future work.\is{deranked verb|)}\is{complex sentence|)}

