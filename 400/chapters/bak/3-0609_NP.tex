%!TEX root = 3-P_Masterdokument.tex
%!TEX encoding = UTF-8 Unicode

\section{The NP}\label{sec:NP}
\is{noun phrase|(}
\is{modification|(}

The typical NP minimally consists of a noun or a \isi{pronoun}.\is{head} Nouns can optionally be modified, pronouns are never modified. NPs can also do without a noun or pronoun. In this case, only the “modifier” is present \citep[cf.][]{Dryer2004}. In this section, only those NPs consisting of a noun and a modifier are considered which form a single syntactic unit and act as an argument in the clause \citep[cf.][13]{Krasnoukhova2012}. There are no discontinuous NPs in Paunaka.

Modifiers are nominal demonstratives,\is{nominal demonstrative} adjectives,\is{adjective} numerals,\is{numeral} relative clauses,\is{relative relation} and other nouns. The status of quantifers\is{quantifier} as modifiers of nouns is not totally clear. Relative clauses are discussed in detail in §\ref{sec:RelativeClauses} and will thus not be considered here. \figref{fig:NP} shows the \isi{word order} in the NP.


\begin{figure}
\begin{tabularx}{\textwidth}{rQCCCCCCCCQl}
{${\Biggl [}$} & {(Q)} & {DEM} & {‘other’} & {NUM} & {ADJ\textsuperscript{1}} & {N} & [{DEM}  & {N\textsubscript{poss}}] & {N\textsubscript{type}} & {\multirow{2}{*}{\shortstack[c]{RC\\ (ADJ\textsuperscript{2})}}} & {${\Biggl ]}$}\\
 & & &  & & & & & & &  &\\
& & & & & & \textit{head} & & & & & \\
\end{tabularx}
\captionof{figure}{Word order in the NP}
\label{fig:NP}
\end{figure}



The following examples show those types of modifiers which precede the noun. (\ref{ex:NP-dem}) has a nominal demonstrative and a noun. It comes from Miguel’s narration of the story about the fox and the jaguar.

\ea\label{ex:NP-dem}
\begingl
\glpreamble \textup{demonstrative + noun:}\\tiyunutu echÿu kupisaÿrÿ\\
\gla ti-yunu-tu echÿu kupisaÿrÿ\\
\glb 3i-go-\textsc{iam} \textsc{dem}b fox\\
\glft ‘the fox had already gone’
\endgl
\trailingcitation{[jmx-n120429ls-x5.170]}
\xe

In (\ref{ex:NP-other-1}), the word for ‘other’ acts as a nominal modifier. Juana speaks about the plans of her landlord.

\ea\label{ex:NP-other-1}
\begingl
\glpreamble \textup{‘other’ + noun:}\\tana punachÿ kuartojane naka\\
\gla ti-ana punachÿ kuarto-jane naka\\
\glb 3i-make.\textsc{irr} other room-\textsc{distr} here\\
\glft ‘he wants to make other rooms here’
\endgl
\trailingcitation{[jxx-p120430l-1.393]}
\xe

(\ref{ex:NP-num-2}) exemplifies the use of a numeral as a modifier. María S. tells her husband here that I have three children (which is not true, I have only two, thus I corrected her, but this is not of importance for the use of the modifier). 

\ea\label{ex:NP-num-2}
\begingl
\glpreamble \textup{numeral + noun:}\\kakutu treschÿ chichechajinube chijinepÿinube\\
\gla kakutu treschÿ chi-checha-ji-nube chi-jinepÿi-nube\\
\glb exist-\textsc{iam} three 3-son-\textsc{col}-\textsc{pl} 3-daughter-\textsc{pl}\\
\glft ‘she has three sons and daughters by now’
\endgl
\trailingcitation{[rmx-e150922l.076]}
\xe

%kakiu nechÿu pario ubiyaenube, mqx-p110826l.182

In (\ref{ex:NP-adj}), there is an adjective \is{adjective|(} and a noun. The adjective in this example is \textit{kana} ‘this size’, a \isi{demonstrative adjective}, which is always accompanied by a gesture showing the size. It is the one that most often occupies the modifying position before the noun, other adjectives are rare in this position (and in modification in general). The example comes from Juana’s account about her grandparents’ journey from Moxos back home.

\ea\label{ex:NP-adj}
\begingl
\glpreamble \textup{adjective + noun:}\\tumunube kana boteyamÿnÿ aguardiente\\
\gla ti-umu-nube kana boteya-mÿnÿ aguardiente\\
\glb 3i-take-\textsc{pl} this.size bottle-\textsc{dim} liquor\\
\glft ‘they took a little bottle of this size of liquor'
\endgl
\trailingcitation{[jxx-p151016l-2.235]}
\xe\is{adjective|)}

Nominal demonstratives\is{nominal demonstrative|(} are the most frequent modifiers. Both Miguel and Juana use them a lot, María S. less so (see §\ref{sec:DemPron}). A demonstrative and a noun that are juxtaposed can also form a predication, but it is mostly the topic pronoun \textit{chibu} which is used in these cases (see §\ref{sec:FocPron}). In predication, there is usually a short pause between the demonstrative and the noun and the demonstrative is stressed, in an NP they form an intonational\is{intonation} unit and there is another predicate in the clause.\is{nominal demonstrative|)}

Apart from demonstratives, most of the pre-nominal modifiers do actually not frequently occur as modifiers. They are rather used as predicates or adverbs or they \isi{head} an NP themselves. This is because the modified referent is usually accessible and thus does not need to be repeated explicitly. This is in accord to what \citet[168]{Danielsen2007} found out for \isi{Baure}: “[m]odification within an NP is not very common Baure in general”. This also holds for Paunaka, to an even greater degree. 

Consider (\ref{ex:NP-other-2}). Juana first uses \textit{punachÿ} ‘other’ as a predicate here, but noting that she was not explicit enough adds a possessive clause in which \textit{punachÿ} functions as a modifier of a noun. She talks about one of her relatives here.

\ea\label{ex:NP-other-2}
\begingl
\glpreamble punachÿtu, kakutu punachÿ seunube eka chima\\
\gla punachÿ-tu kaku-tu punachÿ seunube eka chi-ima\\
\glb other-\textsc{iam} exist-\textsc{iam} other woman \textsc{dem}a 3-husband\\
\glft ‘it is another one now, her husband has another woman now’
\endgl
\trailingcitation{[jxx-p120430l-1.402]}
\xe

There are also a few clauses in which a quantifier\is{quantifier|(} seems to be used attributively, as in (\ref{ex:many-stories}), where Miguel thinks about which story he could tell us.\footnote{This example could also be analysed as containing a partitive NP, see below.}

\ea\label{ex:many-stories}
\begingl
\glpreamble bueno kaku chama echÿu kuento\\
\gla bueno kaku chama echÿu kuento\\
\glb well exist much \textsc{dem}b story\\
\glft ‘well, there are a lot of stories’
\endgl
\trailingcitation{[jmx-n120429ls-x5.048]}
\xe

A prime example of the attributive use of a quantifier is (\ref{ex:pario-ex-2}), but it is quite unique. It comes from María C. and is about her son.

\ea\label{ex:pario-ex-2}
\begingl
\glpreamble tichupumÿnÿ pario paunaka\\
\gla ti-chupu-mÿnÿ pario paunaka\\
\glb 3i-know-\textsc{dim} some Paunaka\\
\glft ‘he knows some Paunaka’
\endgl
\trailingcitation{[cux-c120414ls-2.269]}
\xe
\is{quantifier|)}

In general, NPs with ordinary pre-nominal modifiers (except for the demonstratives) are likely to occur in a non-verbal clause\is{non-verbal predication} including the \isi{copula} \textit{kaku}.

Modifiers that follow the noun can be nouns, adjectives\is{adjective|(} or relative clauses.\is{relative relation} Adjectives are probably best analysed as a subtype of relative clause, since they are generally rather used predicatively than attributively (see §\ref{sec:UsesADJ}). Note that headed RCs\is{head} including a verb are normally completely unmarked, just like adjectives following the noun.\is{adjective|)}

A nominal modifier of another noun often denotes a possessor.\is{possessor|(} As has been shown in §\ref{sec:Possession}, the possessor is indexed on the possessed, but if the possessor is a third person, a possessor noun can co-occur, as in (\ref{ex:possd-poss-1}), which comes from María S., who was repeating a statement of her brother for Swintha. My daughter learned to walk on her own, when we were in Bolivia together in 2011, an issue which is still remembered with pleasure.

\ea\label{ex:possd-poss-1}
\begingl
\glpreamble tiyuikutu chijinepÿimÿnÿ Elena\\
\gla ti-yuiku-tu chi-jinepÿi-mÿnÿ Elena\\
\glb 3i-walk-\textsc{iam} 3-daughter-\textsc{dim} Lena\\
\glft ‘Lena’s daughter is walking now’
\endgl
\trailingcitation{[rxx-e121128s-1.071]}
\xe

A kind of possessor (the figure) also follows the possessed ground in expressions of specific locative relations. In (\ref{ex:possd-poss-2}), Juana cites her brother who was about to depart to a visit at his other brother’s.\footnote{Note that combination of the third person marker \textit{chÿ-} with the \isi{relational noun} \textit{-chuku} ‘side’ is one of the very few cases where we often find \isi{haplology}, so that the possessed form is \textit{chuku} besides \textit{chichuku}.}

\ea\label{ex:possd-poss-2}
\begingl
\glpreamble “niyuna chukuyae Kujtin”\\
\gla ni-yuna chi-chuku-yae Kujtin\\
\glb 1\textsc{sg}-go.\textsc{irr} 3-side-\textsc{loc} Agustín\\
\glft ‘“I go to Agustín”’
\endgl
\trailingcitation{[jxx-p120430l-2.392]}
\xe

Occasionally, the possessor is itself modified by a demonstrative. (\ref{ex:possd-poss-3}) is an example of this. It comes from Miguel telling the story about the two men and the devil. It is the devil who eats up the heads and also all the rest of the meat the men had just hunted.

\ea\label{ex:possd-poss-3}
\begingl 
\glpreamble chijikupupuikutuji echÿu chichÿti echÿu ÿbajane\\
\gla chi-jikupu-puiku-tu-ji echÿu chi-chÿti echÿu ÿba-jane\\ 
\glb 3-swallow-\textsc{cont}-\textsc{iam}-\textsc{rprt} \textsc{dem}b 3-head \textsc{dem}b pig-\textsc{distr}\\ 
\glft ‘he was swallowing the heads of the pigs’\\ 
\endgl
\trailingcitation{[mxx-n101017s-1.052-053]} 
\xe
\is{possessor|)}

In the other kind of noun-noun combination, the modifier specifies the type. This can include combinations of a noun denoting a kind of measure term and the other one the thing which is measured, as in (\ref{ex:NP-1}), or an object and its material, as in (\ref{ex:NP-2}), both from Juana.

In (\ref{ex:NP-1}), the \isi{head} noun \textit{babetamÿnÿ} ‘little trough’ is the measure term, which is modified by \textit{ÿne} ‘water’, the item which comes in this measure. The example stems from Juana’s account about some gold in the woods, which is watched over by a spirit.

\ea\label{ex:NP-1}
\begingl
\glpreamble kaku ÿne kaku nena babetamÿnÿ ÿne\\
\gla kaku ÿne kaku nena babeta-mÿnÿ ÿne\\
\glb exist water exist like trough-\textsc{dim} water\\
\glft ‘there is water, there is what looks like a little trough of water’
\endgl
\trailingcitation{[jxx-p151020l-2]}
\xe

The \isi{head} noun in (\ref{ex:NP-2}) is \textit{yÿpi} ‘jar’ and it is modified by \textit{muteji} ‘loam, mud’ denoting the material of the jar. Juana is talking about the old days in Santa Rita here, before the reservoir and later the pump were constructed. They had to walk far with their clay jars to fetch water.

\ea\label{ex:NP-2}
\begingl
\glpreamble kuinakuÿ, puro eka yÿpi muteji\\
\gla kuina-kuÿ puro eka yÿpi muteji\\
\glb \textsc{neg}-\textsc{incmp} mere \textsc{dem}a jar loam\\
\glft ‘there were no (plastic canisters) yet, it was only with jars of clay’
\endgl
\trailingcitation{[jxx-p120515l-2.058]}
\xe

Possession of non-possessable nouns\is{non-possessability|(} is yet another kind of modification by a type noun: a possessed relational noun\is{relational noun|(} occurs together with a non-possessable noun, which encodes the type of thing that is possessed.  Animals cannot be possessed directly, the relational noun \textit{-peu} ‘domestic animal’ is needed if a possessive relationship to an animal shall be expressed (see §\ref{sec:Non-possessables}). One example is (\ref{ex:rel-possd-1}): the relational noun \textit{-peu} ‘domestic animal’ comes first, the animal denoting the type of possessed animal follows. The sentence comes from María C. who speaks about the lack of meat in her nutrition.

\ea\label{ex:rel-possd-1}
\begingl
\glpreamble kakuina bipeujanemÿnÿ ÿba bikupaka\\
\gla kaku-ina bi-peu-jane-mÿnÿ ÿba bi-kupaka\\ 
\glb exist-\textsc{irr.nv} 1\textsc{pl}-animal-\textsc{distr}-\textsc{dim} pig 1\textsc{pl}-kill.\textsc{irr}\\ 
\glft ‘if we had pigs, we would butcher them’\\ 
\endgl
\trailingcitation{[uxx-p110825l.200]}
\xe


Together with this type of relational noun, either the N\textsubscript{type} can co-occur, as in (\ref{ex:rel-possd-1}) above, or the \isi{possessor}, the N\textsubscript{poss}.\footnote{It is also possible to use the relational noun on its own without being modified.}  (\ref{ex:possd-poss-4}) has a relational noun modified by the noun denoting the possessor. In this case, the NP can actually be analysed as a predicate itself, as a relative clause specifying the preceding noun \textit{bakajane} ‘the cows’. I have found no example in which an NP of the type [N\textsubscript{rel} N\textsubscript{poss}] is an argument of a verbal clause. The example comes from Miguel who told me the story of the cowherd and the spirit of the hill.

\ea\label{ex:possd-poss-4}
\begingl
\glpreamble chikuirauchuji echÿu bakajane chipeujane chipatrune\\
\gla chi-kuirauchu-ji echÿu baka-jane chi-peu-jane chi-patrun-ne\\
\glb 3-care.for-\textsc{rprt} \textsc{dem}b cow-\textsc{distr} 3-animal-\textsc{distr} 3-patrón-\textsc{possd}\\
\glft ‘he looked for the cows, it is said, (which were) the animals of his \textit{patrón}’
\endgl
\trailingcitation{[mxx-n151017l-1.02]}
\xe

Only in one example in the corpus a relational noun is accompanied by both N\textsubscript{poss} and N\textsubscript{type}, in this order.\is{word order} It comes from Juana who was talking about her ducklings, which were not fed properly when she was away to Santa Cruz once. Apparently, only her son (or grandchild?) took some care, saying: 

\ea\label{ex:possd-poss-5}
\begingl
\glpreamble “tikunipajanemÿnÿ chipeujane mimi upuji”\\
\gla ti-kunipa-jane-mÿnÿ chi-peu-jane mimi upuji\\
\glb 3i-be.hungry-\textsc{distr}-\textsc{dim} 3-animal-\textsc{distr} mum duck\\
\glft ‘“the ducks of my mum are hungry”’
\endgl
\trailingcitation{[jrx-c151001lsf-11.067]}
\xe
\is{relational noun|)}
\is{non-possessability|)}

Inside an NP,\is{agreement|(} number, \isi{diminutive} and locative marking\is{locative marker} generally occur only once,\footnote{This does not come as a surprise. In the sample of \citet[]{Krasnoukhova2012}, only 16 out of 55 South American languages mark agreement in number with demonstratives\is{nominal demonstrative} as noun modifiers \citep[52]{Krasnoukhova2012}. Agreement in number is even rarer with modifying numerals\is{numeral} and adjectives\is{adjective} \citep[126, 163–164]{Krasnoukhova2012}.} and typically on the \isi{head} noun, see (\ref{ex:NP-other-1}) – (\ref{ex:NP-adj}), (\ref{ex:possd-poss-2}), (\ref{ex:NP-1}), (\ref{ex:rel-possd-1}), (\ref{ex:possd-poss-5}), but also sometimes on the modifier as in (\ref{ex:possd-poss-3}). (\ref{ex:NP-num}) is interesting in this regard, because the \isi{plural} marker is attached to the modifier and the \isi{diminutive} to the \isi{head} noun. The sentence was produced by María S. who spoke about her life in the old times and referred to her sister Juana here.

\ea\label{ex:NP-num}
\begingl
\glpreamble kakutu ruschÿnube chichechajimÿnÿbane\\
\gla kaku-tu ruschÿ-nube chi-checha-ji-mÿnÿ-bane\\
\glb exist-\textsc{iam} two-\textsc{pl} 3-son-\textsc{col}-\textsc{dim}-\textsc{rem}\\
\glft ‘she already had two little children by that time long ago’
\endgl
\trailingcitation{[rxx-p181101l-2.107]}
\xe

Plural\is{plural} marking may occur on both the modifier and the noun, but this is very rare. One example by Miguel of double plural marking – on \textit{punachÿ} ‘other’ and on \textit{krinko} ‘gringo’ – is given in (\ref{ex:double-pl}), where he tells Juan C. that some gringos told him not to let a \textit{patrón} take advantage of him.

\ea\label{ex:double-pl}
\begingl
\glpreamble tikechunenube echÿu punachÿnube krinkonube\\
\gla ti-kechu-ne-nube echÿu punachÿ-nube krinko-nube\\
\glb 3i-say-1\textsc{sg}-\textsc{pl} \textsc{dem}b other-\textsc{pl} gringo-\textsc{pl}\\
\glft ‘these other gringos told me’
\endgl
\trailingcitation{[mqx-p110826l.381]}
\xe\is{agreement|)}


There are usually not more than two modifiers in an NP, a demonstrative\is{nominal demonstrative} and another modifier, but a few other and partly more complex combinations have been found in the corpus. (\ref{ex:complNP-1}) has a numeral and an adjective. It was produced by Juana when looking at a picture book in order to teach some Paunaka phrases to her grandchild.

\ea\label{ex:complNP-1}
\begingl
\glpreamble sietechÿ sepitÿmÿnÿ kusu\\
\gla sietechÿ sepitÿ-mÿnÿ kusu\\
\glb seven small-\textsc{dim} mouse\\
\glft ‘seven little mice’
\endgl
\trailingcitation{[jxx-e081025s.050]}
\xe


(\ref{ex:complNP-2}) has a numeral that is used like an indefinite article by Miguel in this case, and an adjective which follows the noun. The sentence comes from the story about the lazy man who cuts off his limbs in the end to be thrown into the water by his son and rise as a comet.

\ea\label{ex:complNP-2}
\begingl
\glpreamble “pumane nauku kaku nauku chinachÿ posa mutemena”\\
\gla pi-uma-ne nauku kaku nauku chinachÿ posa mutemena\\
\glb 2\textsc{sg}-take.\textsc{irr}-1\textsc{sg} there exist there one well big\\
\glft ‘“take me there where there is a big well”’
\endgl
\trailingcitation{[mox-n110920l.121]}
\xe

In (\ref{ex:complNP-3}), the \isi{head} noun \textit{rasimo} ‘raceme’ is modified by the preceding demonstrative, by \textit{punachÿ} ‘other’ and by the following noun which specifies the type of raceme. The sentence comes from the same story as (\ref{ex:complNP-2}) above. The lazybones, sitting in the top of a \textit{cusi} palm tree, cuts off his limbs at this point of the story and drops them to the ground, telling his son to collect the supposed racemes of \textit{cusi}.

\ea\label{ex:complNP-3}
\begingl
\glpreamble “pijakupaji echÿu punachÿ rasimo kÿsi”\\
\gla pi-jakupa-ji echÿu punachÿ rasimo kÿsi\\
\glb 2\textsc{sg}-receive.\textsc{irr}-\textsc{imp} \textsc{dem}b other raceme cusi\\
\glft ‘“take this other raceme of \textit{cusi} palm fruit”’
\endgl
\trailingcitation{[mox-n110920l.105]}
\xe

Finally, it should also be mentioned that speakers sometimes use a \isi{numeral}, quantifier\is{quantifier} or the word for ‘other’ \textit{before} a demonstrative to form a partitive NP. This kind of NP has also been analysed for \isi{Baure} \citep[cf.][125]{Danielsen2007}, but it is not very frequent in Paunaka. (\ref{ex:NP-part-1}) is an example by Miguel who was narrating the story of the two men who meet the devil in the woods. The devil is the one who shouts.

\ea\label{ex:NP-part-1}
\begingl
\glpreamble i chinachÿ echÿu chikompanyerone chijakupu echÿu tiyÿbui\\
\gla i chinachÿ echÿu chi-kompanyero-ne chi-jakupu echÿu ti-yÿbui\\
\glb and one \textsc{dem}b 3-companion-\textsc{possd} 3-receive \textsc{dem}b 3i-shout\\
\glft ‘and one of the companions answered the one who shouted’
\endgl
\trailingcitation{[mxx-n101017s-1.021]}
\xe

(\ref{ex:NP-part-2}) is a short switch to Paunaka in Juana’s otherwise Spanish discourse. She was telling Swintha the creation story and switched back and forth between Paunaka and Spanish.

\ea\label{ex:NP-part-2}
\begingl
\glpreamble tumuyubu tumuyubu eka mukiankajanemÿnÿ\\
\gla tumuyubu tumuyubu eka mukianka-jane-mÿnÿ\\
\glb all all \textsc{dem}a animal-\textsc{distr}-\textsc{dim}\\
\glft ‘all, all of the animals’
\endgl
\trailingcitation{[jxx-n101013s-1.696]}
\xe
\is{modification|)}
\is{noun phrase|)}
\is{noun|)}

The following chapter discusses the verb and morphology associated with the verb (or other predicates) in more detail.




