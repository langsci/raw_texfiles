%!TEX root = 3-P_Masterdokument.tex
%!TEX encoding = UTF-8 Unicode

\chapter{The noun and the NP}\label{chapter:Nouns}
\is{noun|(}

This chapter is about nouns and NPs in their primary, referential use. Nouns can also act as predicates, which is described in \sectref{sec:NonVerbalPredication}.%\footnote{In non-verbal predication, nouns can also take TAME markers, which are described in \ref{sec:OperationsPredicates}.} 

First of all, the composition of simplex and complex noun stems is discussed in \sectref{sec:SimplexNouns} and \sectref{sec:ComplexNouns}. There are different processes for deriving\is{derivation} a complex noun from a \isi{nominal root}, among them repetition, \isi{compounding} and addition of classifiers.\is{classifier} Nominalisation\is{nominalisation} of verbs is a marginal strategy.

Three different classes of nouns can be distinguished by their interaction with \isi{possession} marking: inalienable, alienable and non-possessable nouns. Possession marking with these three types of nouns is the topic of \sectref{sec:Possession}. The second main division in nouns concerns human and non-human nouns\is{animacy} and their possibilities to combine with \isi{plural} and other number markers. This is described in \sectref{sec:NumberNouns}. There is no grammatical \isi{gender}.

Nominal irrealis is the topic of \sectref{NominalRS}. This process is comparable to the better-known \isi{nominal tense} marking, but in \isi{nominal irrealis} marking, the referent is non-existent but presupposed. “Deceased”\is{deceased marking} is a category marked on kinship terms and personal names in reference to people who have passed away (see \sectref{sec:Deceased}). This could be considered a case of specialised \isi{nominal tense}. The diminutive is discussed in \sectref{sec:Diminutives}. It occurs not only on nouns, but also on verbs and other parts of speech,\is{word class} but it most often relates to a referent, not to predication.

There is no core-case marking (flagging). A few prepositions\is{preposition} encode \isi{oblique} relations, and there is a general \isi{locative marker} \textit{-yae}. In order to express more specific spatial notions, special locative noun stems are used in \isi{juxtaposition} with a noun referring to the ground. This is the topic of \sectref{sec:Locative}. Information on the NP is given wherever it seems relevant to the topic discussed, but \sectref{sec:NP} summarises all this information and provides a unified description of the NP.

A schema of the noun including all markers that can attach to it in referential use is given in \figref{fig:NounTemplate}. The \isi{locative marker} occurs in two different slots. This is related to the fact that it precedes the diminutive\is{diminutive} marker and follows the \isi{plural} marker, while the diminutive always precedes the plural marker. Unfortunately, there are no examples in the corpus in which all three (diminutive, plural and locative) occur on the noun.\footnote{Examples with locative and plural or locative and diminutive are also rare.}

\begin{figure}[!ht]
\centering
%\includegraphics[width=\textwidth]{figures/NounTemplate.png}
\includegraphics[width=\textwidth]{figures/NounTemplate-new.pdf}
\caption{Template of a noun}
\label{fig:NounTemplate}
\end{figure}\is{agglutination}

\section{The simplex noun}\label{sec:SimplexNouns}
\is{nominal stem|(}

Compared to verbs, nouns have significantly less internal complexity. The great majority of simplex (i.e. non-derived non-composite) noun stems are di- or trisyllabic. In addition, there are a few mono- and tetrasyllabic stems. Some of the latter have a specific phonological structure (see \sectref{sec:WordStructure_subsection}). Table  \ref{table:noun-stems-simple} shows some non-possessed simplex noun stems with different numbers of syllables.

\begin{table}[htbp]
\caption{Simplex noun stems}

\begin{tabular}{lll}
\lsptoprule
& Noun stem & Gloss \cr
\midrule
Monosyllabic & \textit{mai} & stone\cr
& \textit{peÿ} & frog\cr
& \textit{yui} & bread\cr
Disyllabic & \textit{jimu} & fish\cr
& \textit{kuje} & moon\cr
& \textit{ÿku} & rain \cr
Trisyllabic & \textit{kimenu} & woods\cr
& \textit{kÿjÿpi} & manioc\cr
& \textit{uneku} & town\cr
Tetrasyllabic & \textit{ajumerku} & paper\cr
& \textit{kupisaÿrÿ} & fox\cr
& \textit{urupunu} & red brocket \cr
\lspbottomrule
\end{tabular}

\label{table:noun-stems-simple}
\end{table}

Among nouns, we find many loans from Spanish,\is{borrowing|(} especially nouns denoting objects and concepts that were introduced to the area by \textit{karay} in different points in time, such as \textit{baka} ‘cow’ from \textit{vaca}, \textit{arusu} ‘rice’ from \textit{arroz}, \textit{anyo} ‘year’ from \textit{año}, \textit{kupeta} ‘gun’ from \textit{escopeta}, etc. Some of them are phonologically more integrated than others, which may be a hint that they are older.

In addition, there are also a number of nouns borrowed from Bésiro\is{Bésiro|(}. When they are similar to Paunaka’s native nouns in phonemic and syllabic structure, they were mostly not recognised by me, since I do not speak Bésiro. As far as I can tell from checking a word list with 705 entries by \citet[]{Sans2010} and the vocabulary lists compiled by \citet[]{Pinto2010}, they are surprisingly low in number,\footnote{I did not check the complete dictionary by \citet[]{FussRiester1986}, which is of a different variety anyway.} but a more in-depth study may reveal that there are actually more loans.\footnote{For instance, I immediately identified a few more loans when reading the article by \citet[]{Nikulin2019} about Proto-Chiquitano.} A number of nouns with a Spanish origin must have entered Paunaka via Bésiro, easily detectable if they contain the sounds [ʂ] and [ʃ], which are not part of the phonemic inventory of native Paunaka words. One of these loans is \textit{remonixhi} ‘lemon’ from Bésiro \textit{nermónixhi} from Spanish \textit{limón}. The Bésiro noun contains the prefix \textit{n-}, which is preposed to nominal roots starting with a vowel\footnote{The Spanish word \textit{limón} does not start with a vowel. However, it is likely that the process of borrowing was as follows: Bésiro speakers replaced Spanish /l/, which is not part of the phonemic inventory of Bésiro, by /ɾ/. Since /ɾ/ does not usually occur word-initially, an epenthetic vowel was inserted or the syllable was metathesised and then the nasal prefix was attached (compare \citealt[]{Sans2010,Sans2013} for Bésiro phonology).} and the “general case” suffix \mbox{\textit{-xhi}} \citep[20]{Sans2013}. Both are typical for Bésiro nouns. While the prefix is detached in Paunaka – and the first two sounds metathesised again, reflecting the Spanish original –, the suffix is maintained. This cannot be considered a general pattern though, and there is only a small number of nouns with final \textit{xhi} (or similar sounds) in Paunaka. Most loanwords are integrated differently.\footnote{For instance, the noun ‘orange’, Spanish \textit{naranja}, is \textit{narankaxÿ} in Paunaka, with the [ʂ] sound suggesting that it was borrowed via Bésiro, too, while ‘tangerine’ is \textit{mantarina} from Spanish \textit{mandarina}.}\is{Bésiro|)}

Two animal terms that are widespread in the area are found in Paunaka as well, where they have the specific forms \textit{takÿra} ‘chicken, hen’ and \textit{kabe} ‘dog’. \textit{Merÿ} ‘plantain’ and \textit{patabi} ‘sugarcane’ are probably borrowed from Guarayu.\is{borrowing|)} Besides borrowing, new lexemes are created by word formation processes. This is the topic of the following section.

%loans from Spanish:jente, baka, arusu
%loans from Bésiro: aitubuche, mukianka, xhikuera, xhabu
%widespread terms takÿra and kabe
%Guarayo? merÿ, patabi

\section{The complex noun}\label{sec:ComplexNouns}

There are several processes that produce complex nouns in Paunaka, but none of them is very productive. A few nouns with repeated syllables are found (see \sectref{sec:RDPL_Nouns}). Compounding\is{compounding} is largely restricted to plant parts (this is the topic of \sectref{sec:Compounding}). Attachment of classifiers\is{classifier} to a noun or verb stem is similar to compounding (this is discussed in \sectref{sec:Nouns_CLF}). A different derivational pattern is found with some parts of body parts (see \sectref{sec:BodyPartofPartDerivation}). Finally, some nouns are derived from verbs with nominalisers (see \sectref{sec:MorphologyNominalisation}).

\subsection{Repetition}\label{sec:RDPL_Nouns}

Some noun stems contain repeated syllables, but this is hardly productive, and thus does not fall under the concept of \isi{reduplication} 
(\citealp[cf.][13]{Rubino2005}; \citealt[2]{GomezVoort2014}). \tabref{table:noun-stems-RDPL} shows some noun stems with repeated syllables. 

\begin{table}[htbp]
\caption{Repetition in noun stems}

\begin{tabular}{ll}
\lsptoprule
Noun stem & Gloss \cr
\midrule
\textit{barereki} & (clay) pot\cr
\textit{churupepe} & butterfly\cr
\textit{jupipi} & liana\cr
\textit{mimi} & mum (endearment form)\cr
\textit{pujukeke} & patasca (food)\cr
\textit{pÿrÿsÿsÿ} & armadillo sp.\cr
\textit{-sÿsÿ} & nose\cr
\textit{-tabubuji} & branches\cr
\textit{tupapana} & soursop \cr
\textit{yeye} & granny (endearment form), old lady \cr
\textit{yÿkÿkekeji} & branches\cr
\lspbottomrule
\end{tabular}

\label{table:noun-stems-RDPL}
\end{table}

%sÿrÿpÿtÿtÿ = picaflor ; sepitekÿrÿrÿ = picaflor; sumurukuku 'bird sp.'

Repetition of more than one syllable is extremely rare. One example with repetition of the last two syllables is \textit{pichikurakura} ‘thrush-like wren’, a bird species with the scientific name Cam\-py\-lo\-rhyn\-chus turdinus, which seems to be named after the sound it makes when singing. Another one is \textit{pasipasi} ‘sand fly sp.’, which might be onomatopoetic,\is{onomatopoeia} too, considering the buzzing sound these insects make. This is the complete list of nouns with more than one repeated syllable I found in the corpus. As for \textit{pasipasi}, it is also one of the few examples of full repetition. Most other cases of full repetition are disyllabic words, like the \isi{endearment} forms \textit{mimi} ‘mum, mother’ and \textit{yeye} ‘granny, grandmother, old lady’.

\largerpage
In most cases, there is no corresponding noun without the repeated syllable. However, corresponding to two words at the bottom of the table, \textit{-tabubuji} and \textit{yÿkÿkekeji}, there are also \textit{yÿkÿke} ‘tree, wood, stick’ and \textit{-tabu} ‘branch, twig’. Both words including the repetition express a multitude of branches and twigs in the crown of a tree. Repetition is probably triggered by the \isi{collective} marker in this case (see \sectref{sec:Collective}). There is also repetition of the general classifier on adjectives if the collective marker is added, see \sectref{sec:Adjectives}.


\subsection{Compounding}\label{sec:Compounding}\is{compounding|(}
\largerpage
Compounding is not very productive in Paunaka, the only exception being compounds of a plant name and a plant part, especially leaves of plants.\footnote{The noun or \isi{classifier} for fruits (\textit{-i}) is possibly too short to be recognised as a proper stem.\is{nominal stem} Miguel produced some compounds with \textit{-i} in elicitation, but his sister María S. did not and she even claimed that the compounds produced by her brother were incorrect. There is at least one generally agreed upon compound with \textit{-i}, \textit{ichÿi} ‘tree calabash (Crescentia cujete)’. The general word for ‘fruit’ is \textit{chÿi}, which is easily decomposable into the third person marker \textit{chÿ-} and the noun \textit{-i}.} The plant name occurs in N1 position and it modifies\is{modification} the plant part in N2 position. The nouns in N2 position are always inalienably possessed,\is{inalienability} as in (\ref{ex:plant-leaves}).

\ea\label{ex:plant-leaves}
  \ea\label{ex:plant-leaves.1}
\begingl
\glpreamble merÿpune\\
\gla merÿ-pune\\
\glb plantain-leaf\\
\glft ‘plantain leaf’
\endgl
  \ex\label{ex:plant-leaves.2}
\begingl
\glpreamble santiapune\\
\gla santia-pune\\
\glb watermelon-leaf\\
\glft ‘watermelon leaf’
\endgl
  \ex\label{ex:plant-leaves.3}
\begingl
\glpreamble amekaba\\
\gla ame-kaba\\
\glb palm.sp-palm.leaf\\
\glft ‘leaf of \textit{motacú} palm (\textit{Attalea princeps})’
\endgl
  \ex\label{ex:plant-leaves.4}
\begingl
\glpreamble kuyaekaba\\
\gla kuyae-kaba\\
\glb palm.sp-palm.leaf\\
\glft ‘leaf of \textit{totaí} palm (\textit{Acrocomia aculeata})’
\endgl
\z
\xe

In addition, some seeds of plants can be expressed by compounds, as in (\ref{ex:plant-seeds}).


\ea\label{ex:plant-seeds}
  \ea\label{ex:plant-seeds.1}
\begingl
\glpreamble kÿikemuke\\
\gla kÿike-muke\\
\glb peanut-seed\\
\glft ‘peanut seed’
\endgl
  \ex\label{ex:plant-seeds.2}
\begingl
\glpreamble eniyemuke\\
\gla eniye-muke\\
\glb achiote-seed\\
\glft ‘achiote seed’
\endgl
\trailingcitation{\citep[11, 19]{Sell2019}}
\z
\xe


As for other plant parts, speakers rather use complex NPs\is{noun phrase} with a possessed\is{possession} form of the plant part followed by the plant name, as in (\ref{ex:plant-part-NP}), which is from Juana’s telling of the \isi{frog story}.

\ea\label{ex:plant-part-NP}
\begingl
\glpreamble chÿabÿkÿkutu chitabu yÿkÿke\\
\gla chÿ-abÿkÿku-tu chi-tabu yÿkÿke\\
\glb 3-hold-\textsc{iam} 3-branch tree\\
\glft ‘he is holding to a branch of a tree’
\endgl
\trailingcitation{[jxx-a120516l-a.162]}
\xe

The word for ‘chicken egg’ is a compound, too (see (\ref{ex:chicken-egg})), but it can be considered a lexicalised compound,\is{lexicalisation} since other eggs of animals are rather expressed periphrastically, as in (\ref{ex:turtle-egg}). %The compound only denotes eggs, not little chicks. In reference to the latter, a diminutive marker is attached to the noun for ‘chicken’ yielding \textit{takÿramÿnÿ}.

\ea\label{ex:chicken-egg}
\begingl
\glpreamble takÿrachecha\\
\gla takÿra-checha\\
\glb chicken-son\\
\glft ‘chicken egg’
\endgl
\xe

\ea\label{ex:turtle-egg}
\begingl
\glpreamble chichecha kipÿ\\
\gla chi-checha kipÿ\\
\glb 3-son tortoise\\
\glft ‘tortoise egg’
\endgl
%\trailingcitation{[rxx-e121128s-1.086]}
\xe


There are many compounds consisting of a human noun,\is{endearment|(}  most of the times a kinship term, in N1 position, and the possessed noun \textit{-pÿi} ‘body’ in N2 position. In these constructions, \textit{-pÿi} does not alter the lexical meaning of the compound. It rather signals affection or sympathy for the N1, similar to a \isi{diminutive}.\footnote{Note that \textit{-pÿi} and the diminutive marker \textit{-mÿnÿ} are not mutually exclusive. Quite the contrary is true: Human noun compounds with \textit{-pÿi} in N2 position often take diminutive marking, too.} It is mainly used with human nouns denoting people of younger age than the speaker, although, when asked, one speaker claimed that it is also possible to use a compound with \textit{-pÿi} in reference to older people than oneself, like the own mother or father. Since the semantic connection of these compounds to the body-part term is totally opaque, \textit{-pÿi} could also be analysed as a derivational suffix.\footnote{Interestingly, in Apurinã, an Arawakan language\is{Arawakan languages} in Brazil, a similar process might be at work: \citet[15]{Pickering2009} provides a gloss ‘private, esteemed’ for a form \textit{mane}, which is translated as ‘body of’ by \citet[265]{Facundes2000}. The latter nevertheless disagrees with Pickering’s analyses stating that its only meaning is ‘body’. The form seems to occur as a fixed part of two lexemes, one kinship term and one other human noun.}
 In (\ref{ex:buddy-body}), some examples for human nouns composed with \textit{-pÿi} are given.\footnote{\textit{Aitubuche} ‘boy, young man’ is a loan from \isi{Bésiro}. I am actually not sure whether it is used in present-day Bésiro. It is not included in the word list by \citet[]{Sans2011} nor in the dictionary by \citet[]{FussRiester1986}, but \citet[479]{Adelaar2004} offer the form \textit{aɨtoboti} ‘his stepson’, which they found in historical data. This, I believe, is the source of the Paunaka noun.} In the remainder of this work, the forms are usually not decomposed in examples.

\ea\label{ex:buddy-body}
  \ea\label{ex:buddy-body.1}
\begingl
\glpreamble nichechapÿi\\
\gla ni-checha-pÿi\\
\glb 1\textsc{sg}-son-body\\
\glft ‘my son’
\endgl
  \ex\label{ex:buddy-body.2}
\begingl
\glpreamble aitubuchepÿi\\
\gla aitubuche-pÿi\\
\glb boy-body\\
\glft ‘boy’
\endgl
  \ex\label{ex:buddy-body.3}
\begingl
\glpreamble apimiyapÿi\\
\gla apimiya-pÿi\\
\glb girl-body\\
\glft ‘girl’
\endgl
  \ex\label{ex:buddy-body.4}
\begingl
\glpreamble nÿatipÿi\\
\gla nÿ-ati-pÿi\\
\glb 1\textsc{sg}-brother-body\\
\glft ‘my brother (of a woman)’
\endgl
\z
\xe

When the \isi{collective} marker is added, \textit{-pÿi} is detached from most forms, but it has completely lexicalised\is{lexicalisation} with the words \textit{-jinepÿi} ‘daughter’ and \textit{-sinepÿi} ‘grandchild’, i.e. these words are never found without \textit{-pÿi}. Consequently, they do not take the collective marker (see \sectref{sec:Collective}).
The first part of \textit{-jinepÿi}, \textit{*jine}, has cognates in other Southern Arawakan languages, the addition of \textit{-pÿi} is an innovation that is only found in Paunaka.% Baure \textit{-jin}, Kinikinao \textit{-ihine} \textit[146]{Souza2008}
\is{endearment|)} 

Finally, a lot of lexicalised\is{lexicalisation} compounds are found among body part terms, especially for bones and hair \citep[255]{TerhartDanielsenBODY}. While \textit{-jiyu} ‘body hair’ can also occur in non-compound forms,\footnote{The noun is inalienably possessed and thus underlies the conditions specified in \sectref{sec:Inalienables}.} the sequence \textit{-chupu/-chupea}, which is found in compounds denoting bones, is never used as a non-compound lexeme. 

In body part compounds, strikingly, the order of modifier\is{modification} and modified noun seems to be reversed, with N1 being the semantic \isi{head}. However, we cannot be sure, how these body parts were perceived originally, so, instead of ‘face hair’, the beard could have also been perceived as ‘hair face’, i.e. the part of the face with hair, so that the question which part is the semantic \isi{head} cannot be resolved here. (\ref{ex:jiyu}) lists a few examples for compounds with \textit{-jiyu} and (\ref{ex:chupu}) with \textit{-chupu}.

\ea\label{ex:jiyu}
  \ea\label{ex:jiyu.1}
\begingl
\glpreamble chijiyumama\\
\gla chi-jiyu-mama\\
\glb 3-hair-jaw\\
\glft ‘his jaw beard’
\endgl
  \ex\label{ex:jiyu.2}
\begingl
\glpreamble chijiyutaka\\
\gla chi-jiyu-taka\\
\glb 3-hair-armpit\\
\glft ‘his/her armpit hair’
\endgl
  \ex\label{ex:jiyu.3}
\begingl
\glpreamble chijiyunÿkÿ\\
\gla chi-jiyu-nÿkÿ\\
\glb 3-hair-mouth\\
\glft ‘his mustache’
\endgl
\z
\xe

\ea\label{ex:chupu}
  \ea\label{ex:chupu.1}
\begingl
\glpreamble chichupupiÿnÿ\\
\gla chi-chupu-piÿnÿ\\
\glb 3-bone-neck\\
\glft ‘his/her cervicals’
\endgl
  \ex\label{ex:chupu.2}
\begingl
\glpreamble chichuputÿi\\
\gla chi-chupu-tÿi\\
\glb 3-bone-anus\\
\glft ‘his/her tailbone’
\endgl
  \ex\label{ex:chupu.3}
\begingl
\glpreamble chichupukekÿ\\
\gla chi-chupu-kekÿ\\
\glb 3-bone-back.of.animal\\
\glft ‘his/her spine’
\endgl
\z
\xe

Both \textit{-jiyu} ‘hair’ and \textit{-chupu} ‘bone’ (in this specific case with the slightly different form \mbox{\textit{-chupea}}) can also be combined in the word for eyebrow, literally ‘hair bone face’ (or ‘hairy part of the boney part of the face’), see (\ref{ex:eyebrow}).

\ea\label{ex:eyebrow}
\begingl 
\glpreamble chijiyuchupeabÿke\\
\gla chi-jiyu-chupea-bÿke\\ 
\glb 3-hair-bone-face\\ 
\glft ‘his/her eyebrow’
\xe

The body part noun \textit{-bÿke} ‘face’ forms part of the exocentric compounds denoting cardinal directions \citep[266]{TerhartDanielsenBODY}. The expressions are given in (\ref{ex:cardinals}).\footnote{There is no noun \textit{*kuju}, but the form may be related to \textit{tujubeiku} ‘wind’, which is most probably a \isi{verb} structurally. Note that the wind usually comes from the North, unless it is cold wind from the South (\textit{tisÿeipu} ‘south wind, cold weather coming from the South’).}

\ea\label{ex:cardinals}
  \ea\label{ex:cardinals.1}
\begingl
\glpreamble manebÿke\\
\gla mane-bÿke\\
\glb morning-face\\
\glft ‘East’
\endgl
  \ex\label{ex:cardinals.2}
\begingl
\glpreamble kupeibÿke\\
\gla kupei-bÿke\\
\glb afternoon-face\\
\glft ‘West’
\endgl
  \ex\label{ex:cardinals.3}
\begingl
\glpreamble kuju-bÿke\\
\gla kuju-bÿke\\
\glb wind?-face\\
\glft ‘North’
\endgl
  \ex\label{ex:cardinals.4}
\begingl
\glpreamble tisÿeibÿke\\
\gla ti-sÿei-bÿke\\
\glb 3i-be.cold-face\\
\glft ‘South’
\endgl
\z
\xe
\is{compounding|)}

\subsection{Derivation of nouns with classifiers}\label{sec:Nouns_CLF}\is{derivation|(}\is{classifier|(}

Classifiers can derive nouns from other nouns and verbs.\is{verb} They are often completely lexicalised\is{lexicalisation|(} on the noun and cannot be detached. There is, for example, the word pair \textit{mutepa} ‘dust, earth’ and \textit{muteji} ‘loam, mud’. Both must derive from a stem \textit{*mute}, but there is no such noun (or verb) in the language – at least not synchronically.\is{lexicalisation|)} A list of all classifiers I could identify is given in \sectref{sec:Classifiers}.

(\ref{ex:noun-clf}) and (\ref{ex:verb-clf}) list words in which the derivational process is still transparent, because the stems\is{nominal stem} are also found without classifiers. The nouns in (\ref{ex:noun-clf}) are results of the combination of a noun with a classifier, while (\ref{ex:verb-clf}) lists two nouns which are derived from verb stems with the help of classifiers.

\ea\label{ex:noun-clf}
  \ea\label{ex:noun-clf.1}
\begingl
\glpreamble yÿkÿke\\
\gla yÿkÿ-ke\\
\glb fire-\textsc{clf:}cylindrical\\
\glft ‘tree, stick’
\endgl
  \ex\label{ex:noun-clf.2}
\begingl
\glpreamble chicheneumu\\
\gla chi-chene-umu\\
\glb 3-breast-\textsc{clf:}liquid\\
\glft ‘milk’
\endgl
  \ex\label{ex:noun-clf.3}
\begingl
\glpreamble yÿbapa jimupa\\
\gla yÿbapa jimu-pa\\
\glb flour fish-\textsc{clf:}particle\\
\glft ‘fish flour’
\endgl
  \ex\label{ex:noun-clf.4}
\begingl
\glpreamble kechuepi\\
\gla kechue-pi\\
\glb snake-\textsc{clf:}long.flexible\\
\glft ‘worm’
\endgl
\z
\xe

\ea\label{ex:verb-clf}
  \ea\label{ex:verb-clf.1}
\begingl
\glpreamble nijikupupi\\
\gla ni-jikupu-pi\\
\glb 1\textsc{sg}-swallow-\textsc{clf:}long.flexible\\
\glft ‘my gullet’
\endgl
  \ex\label{ex:verb-clf.2}
\begingl
\glpreamble chimukuji\\
\gla chi-muku-ji\\
\glb 3-sleep-\textsc{clf:}soft.mass\\
\glft ‘its nest’
\endgl
\z
\xe

There are also some idiosyncratic forms. The word \textit{ÿneumu} may be composed of \textit{ÿne} ‘water’ and \textit{-umu}, the classifier for liquids. Its meaning, however, is ‘inside the water’.\footnote{It could also include the same suffix found in \textit{anÿmu} ‘sky’ (opposed to \textit{anÿke} ‘up, above’), but then the first \textit{u} of \textit{ÿneumu} would be unexplained.} This word seems to be quite old, it was possibly already used in the Proto language of the Bolivian Arawakan\is{Southern Arawakan} languages, considering that we find a similar word in Old Mojeño,\is{Mojeño languages}\footnote{Old Mojeño is the variety of Mojeño that was documented by the Jesuits in the late 17th century in a grammar and catechism \citep[]{Marban1894}.} <uneamukû> ‘inside of the water’ \citep[94]{Marban1894}, although this one includes the boundedness classifier <-kû>  – whose cognate form \textit{-kÿ} is also found in the corpus on \textit{ÿneumu} once. The latter is a reflex of Proto-Arawakan\is{Arawakan languages} \textit{*-Vku} \citep[cf.][384]{Payne1991}. It expresses boundedness, i.e. nouns with this classifier are perceived as having boundaries, as a kind of “container”. It is a special case of classifier, since it is often used in locative expressions only, see \sectref{sec:Locative}, but also found lexicalised\is{lexicalisation} with some nouns outside of such contexts, e.g. the word \textit{chenekÿ} ‘way, path’ occurs with \textit{-kÿ} in locative and non-locative contexts.

I have claimed elsewhere \citep[178--180]{Terhart2016} that the two word formation processes of \isi{compounding} two noun stems and combination of a noun with a classifier can be seen as two ends of a continuum, where nouns have a more concrete lexical meaning and can typically occur on their own (or with a person marker if they are inalienably possessed), and classifiers have a broader meaning, mostly based on shape, and can never stand on their own, which makes them more reminiscent of derivational affixes,\is{affix} yet with a relatively concrete meaning.
\is{classifier|)}

\subsection{Derivation of parts of body parts}\label{sec:BodyPartofPartDerivation}

There is one derivational prefix \textit{ke-}, which attaches to a few body part terms to derive a part of this body part \citep[]{TerhartDanielsenBODY}. The process is not productive and restricted to the nouns listed in (\ref{ex:ke-deriv}).

\ea\label{ex:ke-deriv}
  \ea
 \begingl 
\glpreamble nibÿke – nikebÿke\\
\gla ni-bÿke ni-ke-bÿke\\ 
\glb 1\textsc{sg}-face 1\textsc{sg}-\textsc{der}-face\\ 
\glft ‘my face – my eye(s)’
  \ex
 \begingl
\glpreamble nibuÿ – nikebuÿ\\
\gla ni-buÿ ni-ke-buÿ\\ 
\glb 1\textsc{sg}-hand 1\textsc{sg}-\textsc{der}-hand\\ 
\glft ‘my hand(s) – my finger(s)’
  \ex
 \begingl
\glpreamble nibu – nikeibu\\
\gla ni-ibu ni-ke-ibu\\
\glb 1\textsc{sg}-foot 1\textsc{sg}-\textsc{der}-foot\\
\glft ‘my foot (feet) – my toe(s)’
\endgl
\z
\xe

The same derivation pattern seems to be at work in derivation of the word ‘tail’ from ‘wing’ as in (\ref{ex:ke-deriv-animal}), although the semantic relationship between those animal body parts is not the same as for the human body parts in (\ref{ex:ke-deriv}), since a tail is not a part of the wings. An animal does not even necessarily have to have wings in order to have a tail.

\ea\label{ex:ke-deriv-animal}
\begingl 
\glpreamble chisi – chikeisi\\
\gla chÿ-isi chi-ke-isi\\ 
\glb 3-wing 3-\textsc{der}-wing\\ 
\glft ‘its wing(s) – its tail’
\xe


\subsection{Nominalisation}\label{sec:MorphologyNominalisation}
\is{nominalisation|(}

A few nouns result from nominalisation with the suffix \textit{-kene}, they are listed in (\ref{ex:NomiNouns-1}). All of them are inalienably possessed\is{inalienability} and given here with the first person plural possessor. They are objective nouns \citep[cf.][]{ComrieThompson2007}, i.e. the patient\is{patient/theme} is nominalised.


\ea\label{ex:NomiNouns-1}
  \ea\label{ex:NomiNouns-1.1}
\begingl
\glpreamble bejumikene\\
\gla bi-ejumi-kene\\
\glb 1\textsc{pl}-remember-\textsc{nmlz}\\
\glft ‘our thoughts’
\endgl
  \ex\label{ex:NomiNouns-1.2}
\begingl
\glpreamble bupukene\\
\gla bi-upu(nu)-kene\\
\glb 1\textsc{pl}-bring-\textsc{nmlz}\\
\glft ‘our load’
\endgl
  \ex\label{ex:NomiNouns-1.3}
\begingl
\glpreamble bichabukene\\
\gla bi-chabu-kene\\
\glb 1\textsc{pl}-do-\textsc{nmlz}\\
\glft ‘our actions/deeds’
\endgl
\z
\xe

In addition, further nouns that may or may not contain the nominaliser are \textit{kuchepukene} ‘sorcerer’, \textit{-akene}/\textit{-ekene} ‘non-visible side’, in both cases no form without \textit{-kene} is known to me.\footnote{Rose (2021, p.c.) relates \textit{kuchepukene} ‘sorcerer’ to the noun for ‘bone’ as in \isi{Mojeño Trinitario}, where the form of the nominaliser is \textit{-giene} (while Ignaciano\is{Mojeño Ignaciano} has \textit{-kene} \citep[cf.][663--672]{OlzaZubiri2004}). ‘Bone’ is \textit{eupe} or \textit{-upeji} in Paunaka, thus the relation is not straightforward in this language. She further relates \textit{-akene} ‘non-visible side’ to a different etymon given the fact that there are several forms in Trinitario\is{Mojeño Trinitario} containing the sequence \textit{giene} that mean ‘follow’ or ‘behind’.} The word \textit{tijaikenekÿu} ‘dawn’ is derived from \textit{tijai} ‘it is light, day’, seemingly with \textit{-kene} and the translocative concurrent motion marker\is{associated motion} \textit{-kÿu}, but this does not make much sense to me, since this marker usually encodes motion away from the scene (‘the light that is going’?). Note that \textit{tijai} is formally a \isi{verb}, although used like a noun in most cases (see \sectref{sec:UnmarkedRC}).

One further example of a nominalised verb form used referentially (as S of a non-verbal predicate) has been found in the data collected by Riester: (\ref{ex:NMLZ-r3}), which is about scarcity of food due to a drought. 

\ea\label{ex:NMLZ-r3}
\begingl
\glpreamble nechikue sepitÿjiku tanÿma eka binikeneina\\
\gla nechikue sepitÿ-jiku tanÿma eka bi-ni-kene-ina\\
\glb therefore small-\textsc{lim}1 now \textsc{dem}a 1\textsc{pl}-eat-\textsc{nmlz}-\textsc{irr.nv}\\
\glft ‘therefore we have little (possible) food now’
\endgl
\trailingcitation{[nxx-a630101g-1.38-39]}
\xe

The form \textit{-nikene}\is{non-verbal irrealis marker|(} ‘food’ has not been found with the speakers I worked with. And there is even more to it: when working on parts of the recordings by Riester together with Miguel, Juana and María S., they repeated the nominalised verb \textit{binikeneina} as \textit{binikukeneina}, i.e. including a \isi{thematic suffix}, see \sectref{sec:ActiveVerbs_TH}. This is not trivial, since the thematic suffix is the place of \isi{reality status} marking on active verbs (see \sectref{sec:RealityStatus}), while nouns take a different irrealis marker (see \sectref{NominalRS} and \sectref{sec:NonVerbalPredication}).  %(e.g. in rxx-e181024l)
There were a few more instances in the corpus where speakers used \textit{-nikukene}, all of them taking the non-verbal irrealis marker \textit{-ina}. One of them is (\ref{ex:emphi-3nmlz}) which is taken from a story by Miguel. The wife of the main character, who is very lazy, asks him to make a field, because they do not have food.

\ea\label{ex:emphi-3nmlz}
\begingl
\glpreamble “panajachÿu pario eka pisaneina kuina binikukeneina"\\
\gla pi-ana-ja-chÿu pario eka pi-sane-ina kuina bi-niku-kene-ina\\
\glb 2\textsc{sg}-make.\textsc{irr}-\textsc{emph}1-\textsc{dem}b some \textsc{dem}a 2\textsc{sg}-field-\textsc{irr.nv} \textsc{neg} 1\textsc{pl}-eat-\textsc{nmlz}-\textsc{irr.nv}\\
\glft ‘“make something for your field, we do not have any food”’
\endgl
\trailingcitation{[mox-n110920l.015]}
\xe
\is{non-verbal irrealis marker|)}

In addition to the nominaliser, there is a homophonous \isi{emphatic} marker \textit{-kene}, which is equally rare (see \sectref{sec:EmphMarker}).

A few nouns seem to be derived from verbs by a suffix \textit{-e}. I have found only three examples, which are given in (\ref{ex:NomiNouns-2}), all of them with the first person plural possessor. I first thought they were deranked verbs (see \sectref{sec:Subordination-i}) with a somehow inarticulate RS suffix, but there is a difference: the subordinating suffix \textit{-i} comes after the thematic suffix, and here, the thematic suffix is detached, the /i/ is part of the verb stem. Just like the derivation with \textit{-kene}, this process does not seem to be productive in Paunaka considering the small number of words with this suffix in the corpus.\footnote{It is noticeable that all verb stems begin with \textit{yÿ}. As for \textit{-yÿtiku} ‘set on fire (to cook)’ and \mbox{\textit{-yÿtipajiku}} ‘make/cook chicha’, they are certainly related, the second one containing the \isi{classifier} \textit{-pa} for dusty things and the intensive aktionsart suffix \textit{-ji}. The verb \textit{-yÿseiku} ‘buy’, however, could be a loan from Guarayu (Danielsen 2021, p.c.).} 


\ea\label{ex:NomiNouns-2}
  \ea\label{ex:NomiNouns-2.1}
\begingl
\glpreamble biyÿtie\\
\gla bi-yÿti-e\\
\glb 1\textsc{pl}-set.on.fire-\textsc{nmlz}\\
\glft ‘our food’
\endgl
  \ex\label{ex:NomiNouns-2.2}
\begingl
\glpreamble biyÿtipajie\\
\gla bi-yÿtipaji-e\\
\glb 1\textsc{pl}-make.chicha-\textsc{nmlz}\\
\glft ‘our chicha (still being cooked)’
\endgl
  \ex\label{ex:NomiNouns-2.3}
\begingl
\glpreamble biyÿseie\\
\gla bi-yÿsei-e\\
\glb 1\textsc{pl}-buy-\textsc{nmlz}\\
\glft ‘our purchas’
\endgl
\z
\xe

We can conclude that nominalisation is a very rare process of word formation in Paunaka. This is because speakers rather use headless relative clauses\is{relative relation} in those contexts in which a nominalised verb would be expected in other languages.
\is{nominalisation|)}\is{derivation|)}\is{nominal stem|)}

We will leave the inner constituency of nouns now and have a look at nominal inflection, starting from the next section, which is about possession marking.




%!TEX root = 3-P_Masterdokument.tex
%!TEX encoding = UTF-8 Unicode

\section{Possession}\label{sec:Possession}
\is{inflection|(}
\is{possession|(}
\is{person marking|(}

Possession is marked on the \isi{head} noun, i.e. the noun denoting the possessed. Possessed nouns take a person marker, which precedes the noun stem\is{nominal stem} and indexes the \is{possessor} as in (\ref{ex:new23-POSS}). 

\ea\label{ex:new23-POSS}
\begingl
\glpreamble pimuse\\
\gla pi-muse\\
\glb 2\textsc{sg}-mother.in.law\\
\glft ‘your (\textsc{sg}) mother-in-law’
\endgl
%\trailingcitation{[]}
\xe

The person markers are given in \tabref{table:POSS_Pref}.\is{possessor|(} They are identical to the ones that index subjects\is{subject} on verbs,\is{verb} with the only difference being that two third person markers are available for verbs, which are connected to differential object marking (see \sectref{sec:3Marking}), while nouns have only one third person marker. There is no \isi{gender} distinction in the third person, neither is the marker is specified for number. When a human\is{animacy} third person plural possessor is to be expressed, the \isi{plural} marker \textit{-nube} is added. The first person singular and the third person markers have allomorphs, both including a distinction between a high front and a high central vowel. The allomorphs with the front vowel predominantly occur when the following syllable contains an /i/ or /u/, the ones with the central vowel before syllables with /ÿ/, /ɛ/ and /a/, but this distribution is a tendency rather than an absolute rule. With only one exception to my knowledge, only the markers containing the central vowel can precede a vowel-initial syllable, although in most cases, the vowel of the person marker is deleted in such cases (see \sectref{section:Vowel_elision}).

\begin{table}
\caption{Person markers on possessed nouns}

\begin{tabular}{ll}
\lsptoprule
Person & Person marker \cr
\midrule
1\textsc{sg} & \textit{nÿ-/ni-}  \cr
2\textsc{sg} & \textit{pi- } \cr
3 & \textit{chÿ-/chi-}  \cr
1\textsc{pl} & \textit{bi-} \cr
2\textsc{pl} & \textit{e-} \cr
3\textsc{pl} & \textit{chÿ-/chi- ... -nube} \cr
\lspbottomrule
\end{tabular}

\label{table:POSS_Pref}
\end{table}
\is{person marking|)}

In most contexts with third person \isi{plural} possessors, there are also several possessed items, as in (\ref{ex:Possi-PL-1}). This becomes apparent from the context in which these nouns are used. One exception is the noun \textit{-ubiu} ‘house’. There is often only one house for various people, see (\ref{ex:Possi-PL}) .

\ea\label{ex:Possi-PL-1}
\begingl
\glpreamble chiyumaji – chiyumajinube\\
\gla chi-yumaji chi-yumaji-nube\\
\glb 3-hammock 3-hammock-\textsc{pl}\\
\glft ‘his/her hammock – their hammocks (or, less likely: their hammock)’
\endgl
%\trailingcitation{[]}
\xe

\ea\label{ex:Possi-PL}
\begingl
\glpreamble chubiu – chubiunube\\
\gla chi-ubiu chi-ubiu-nube\\
\glb 3-house 3-house-\textsc{pl}\\
\glft ‘his/her house – their house (or: their houses)’
\endgl
%\trailingcitation{[]}
\xe

If the possessed noun is non-human,\is{animacy} the \isi{plural} marker unambiguously relates to the possessor, but when it comes to possessed kin, it is often less clear whether the possessor or the possessed kin is pluralised. A detailed discussion about the ambiguity of \isi{plural} marking concerning possessed kin with third person possessors is postponed to \sectref{sec:Collective}.\is{possessor|)}

Nouns can be divided into three different classes according to how they interact with possession: there are inalienable, alienable and non-possessable nouns, see \sectref{sec:Inalienables}, \sectref{sec:Alienables} and \sectref{sec:Non-possessables} below. Roughly, the first of them must be possessed, the second ones can be possessed and the third ones cannot be possessed. This tripartite division is very typical for \is{Arawakan languages}  (\citealp[cf.][82]{Aikhenvald1999}; \citealt[]{Danielsen2014}) and in addition, the existence of a class of inalienable nouns is common in Amazonian languages\is{Amazonian language} in general \citep[88, 100]{Krasnoukhova2012}.


\subsection{Inalienable nouns}\label{sec:Inalienables}
\is{inalienability|(}

Inalienable nouns obligatorily express a possessor. This is related to the fact that in inalienable possession, there are “inextricable, essential or unchangeable relations between ‘possessor’ and ‘possessed’” \citep[4]{ChappelMcGregor1996}. 

According to \citet[572]{Nichols1988}, there is an implicational hierarchy among semantic groups that are conceived as inalienably possessed cross-linguistically. The hierarchy is given in \figref{fig:InalienabilityHierarchy}.

\begin{figure}[!ht]
\centering
Kin terms and/or body parts < Part-whole and/or spatial relations < Culturally basic possessed items (e.g. arrows, domestic animals)
\caption{Inalienability Hierarchy \citep[572]{Nichols1988}}
\label{fig:InalienabilityHierarchy}
\end{figure}

The semantic groups of nouns that are inalienably possessed in Paunaka are kinship terms, body parts, plant parts (to some extent), spatial relations, and some culturally basic items, so that Paunaka fully confirms the hierarchy.

All inalienable nouns obligatorily express the possessor by a person marker\is{person marking|(} preceding the possessed noun. An example with a kinship term is given in (\ref{ex:piati}). It was elicited from Juana.

\ea\label{ex:piati}
\begingl 
\glpreamble nimu piati ukuine\\
\gla ni-imu pi-ati ukuine\\ 
\glb 1\textsc{sg}-see 2\textsc{sg}-brother yesterday\\ 
\glft ‘I saw your brother yesterday’
\trailingcitation{[jxx-e110923l-1.049]}
\xe
\is{person marking|)}

The noun \textit{-ati} ‘brother’ is used to refer to male siblings of females and the term \textit{-etine} ‘sister’ to refer to female siblings of males. If the reference is to a sibling of the same sex as the possessor, the noun \textit{-piji}, glossed here as ‘sibling’, is used, see (\ref{ex:piji-sibling}). The example also comes from Juana and is about her daughter who did not go to the airport to pick up her sister.

\ea\label{ex:piji-sibling}
\begingl 
\glpreamble kuina tiyuna chipiji\\
\gla kuina ti-yuna chi-piji\\ 
\glb \textsc{neg} 3i-go.\textsc{irr} 3-sibling\\ 
\glft ‘her sister didn’t go’
\trailingcitation{[jxx-p110923l-1.299]}
\xe

Whether the referent is a male or a female person is only recoverable by the context.\footnote{As has already been stated in \sectref{section:Vowel_elision}, Paunaka does not make use of genderlects,\is{genderlect} with minimal exceptions.} An example with the kinship term \textit{-etine} ‘sister’ is given in (\ref{ex:netine}). It comes from Isidro, who greeted me.

\ea\label{ex:netine}
\begingl 
\glpreamble ¿michabi?, netine\\
\gla micha-bi nÿ-etine\\ 
\glb good-2\textsc{sg} 1\textsc{sg}-sister\\ 
\glft ‘how are you, my sister?’
\trailingcitation{[mdx-c120416ls.008]}
\xe

\largerpage
The noun for ‘God’, \textit{bia}, is also a kinship term, it consists of the noun \textit{-a} ‘father’ and the first person plural marker\is{person marking|(} \textit{bi-}, so literally it means ‘our father’, though this seems to be intransparent to the speakers. In order to mark the difference between ‘our father’ and ‘God’, it looks as if the Paunaka re-analysed the noun stem \textit{-a} ‘father’ as \textit{-ÿa} as in (\ref{ex:God-father}). But this is only true for the first person plural, the second person singular is still \textit{pia} (\textit{pi-a} ‘your father’). If the stem were regular \textit{*-ÿa}, the second person singular should be \textit{*pÿa}.\footnote{But note that there is an additional vowel (or rather syllable) in the cognate form of \isi{Mojeño Trinitario} \textit{-iya}, and see also discussion in \sectref{section:Vowel_elision}.} In general, speakers rather avoid the first person plural form and speak of ‘my father’, \textit{nÿa} (\textit{nÿ-a}), instead. \textit{Bÿa} only showed up in elicitation. The same is true for the second person plural form, for which I could elicit \textit{ÿa}, but only after some contemplation about what the form could be.\is{person marking|)}

\ea\label{ex:God-father}
\begingl 
\glpreamble bia – bÿa\\
\gla bi-a bi-ÿa?\\ 
\glb 1\textsc{sg}-father 1\textsc{sg}-father\\ 
\glft ‘God – our father’
\xe


Some kinship terms have suppletive endearment forms,\is{endearment|(}  which are free, non-possess\-able nouns used both as vocatives and referentials. The possessed forms on the other hand are never used as vocatives. \tabref{table:Vocatives} gives the forms.
The female ones, \textit{mimi} ‘mum’ and \textit{yeye} ‘granny’ are used a lot, the latter can be used in respectful reference to any older indigenous female person.\footnote{Non-indigenous women are called \textit{senyora} ‘Mrs, madame’ or \textit{senyorita} ‘Miss’ from Spanish \textit{señora} and \textit{señorita}, respectively.} \textit{Taita} ‘dad’ can also be used with non-kins, but occurs only rarely.

\begin{table}[htbp]
\caption{Kinship terminology with endearment forms}

\begin{tabular}{lll}
\lsptoprule
Kinship term & Endearment term & Translation\cr
\midrule
\textit{-enu} & \textit{mimi} & mother, mum \cr
\textit{-a} & \textit{taita} & father, dad \cr
\textit{-use} & \textit{yeye} & grandmother, granny \cr
\textit{-uchiku / -uma} & \textit{chÿchÿ} & grandfather, grandpa \cr
\lspbottomrule
\end{tabular}

\label{table:Vocatives}
\end{table}
\is{endearment|)} 

Most body-part terms are inalienably possessed \citep[]{TerhartDanielsenBODY}, one example is given in (\ref{ex:body-part-1}). It comes from María C. who was worried that her chicha would soon be finished.

\ea\label{ex:body-part-1}
\begingl 
\glpreamble tikuti nemua neamÿnÿ ÿne\\
\gla ti-kuti nÿ-emua nÿ-ea-mÿnÿ ÿne\\ 
\glb 3i-hurt 1\textsc{sg}-belly 1\textsc{sg}-drink.\textsc{irr}-\textsc{dim} water\\ 
\glft ‘my stomach hurts when I drink water’
\trailingcitation{[ump-p110815sf.709-710]}
\xe

Plant parts are semantically closely related to body parts. These parts usually occur with a third person possessor \is{possessor|(} in Paunaka. The possessor is the plant in this case, which follows the possessed part, see (\ref{ex:leaves}) and (\ref{ex:branches}).\footnote{The other possibility is to use the parts as N2s in compounds, but compounding is not very productive in general in Paunaka, see \sectref{sec:Compounding}.}

\ea\label{ex:leaves}
\begingl 
\glpreamble chipuneji kÿjÿpi\\
\gla chi-pune-ji kÿjÿpi\\ 
\glb 3-leaf-\textsc{col} manioc\\ 
\glft ‘leaves of a manioc, manioc leaves’
\trailingcitation{[nxx-a630101g-1.51]}
\xe

\ea\label{ex:branches}
\begingl 
\glpreamble chimusuji merÿ\\
\gla chi-musuji merÿ\\ 
\glb 3-skin plantain\\ 
\glft ‘banana peel’
\trailingcitation{\citep[4]{Sell2019}}
\xe

A third person possessor is also chosen, when no possessor is lexically expressed, as in (\ref{ex:POSS-fruit}), which has \textit{chÿi} ‘its fruit’.\footnote{In the case of ‘leaf’, an additional \textit{e} is preposed to the plant part containing the third person possessor, for a reason unknown to me. Thus ‘leaf’ is normally realised as \textit{echÿpune} (\textit{e-chÿ-pune} ?-3-leaf), though the plural (i.e. \isi{collective}) form is \textit{chipuneji}, see (\ref{ex:leaves}).} María S. gives an explanation about a plant here.

\ea\label{ex:POSS-fruit}
\begingl 
\glpreamble takujibÿ eka te kanainatu chÿi te puero binika\\
\gla ti-a-kujibÿ eka te kana-ina-tu chÿ-i te puero bi-nika\\ 
\glb 3i-\textsc{irr}-have.flower \textsc{dem}a \textsc{seq} this.size-\textsc{irr.nv}-\textsc{iam} 3-fruit \textsc{seq} can 1\textsc{pl}-eat.\textsc{irr}\\ 
\glft ‘it blossoms and once its fruits have this size (showing with hands), we can eat them’
\trailingcitation{[rxx-e121126s-3.16]}
\xe
\is{possessor|)}


Among the culturally basic items that are inalienably possessed in Paunaka are \textit{-sane} ‘field’, \textit{-mÿu} ‘clothes’, \textit{-etea} ‘language’, and also some parasites like \textit{-kane} ‘worm’ and \textit{-ine} ‘louse’. The word for ‘house’, \textit{-ubiu}, originated as a \isi{deranked verb}, but acquired characteristics of a noun like being able to combine with the locative marker (see \sectref{sec:Subordination-i}). It is also inalienably possessed. As for other nouns which typically fall into the class of inalienables, the word for ‘arrow’ is not remembered by the speakers and a word for ‘village’ does not exist.\footnote{There is, however, \textit{-epukie} ‘homeland, home’, which belongs to the inalienable class. In contrast, \textit{uneku} ‘town’ is usually not possessed, but may be used in a construction containing the general relational noun depending on the speaker, see \sectref{sec:Non-possessables} below.}

(\ref{ex:Npossi-1}) was produced by Isidro to answer Swintha’s question whether he would go to his field the next day.

\ea\label{ex:Npossi-1}
\begingl
\glpreamble tajaitu niyunupunuka nisaneyae\\
\gla tajaitu ni-yunu-punuka ni-sane-yae\\
\glb tomorrow 1\textsc{sg}-go-\textsc{reg.irr} 1\textsc{sg}-field-\textsc{loc}\\
\glft ‘tomorrow I will go to my field again’
\endgl
\trailingcitation{[dxx-d120416s.129]}
\xe

(\ref{ex:Npossi-2}) comes from Miguel who was looking at a little wooden toy figure and identified it as female.

\ea\label{ex:Npossi-2}
\begingl
\glpreamble chimÿu tÿnai entonses apimiyapÿimÿnÿ\\
\gla chi-mÿu ti-ÿnai entonses apimiyapÿi-mÿnÿ\\
\glb 3-clothes 3i-be.long thus girl-\textsc{dim}\\
\glft ‘its garment is long, so it is a girl’
\endgl
\trailingcitation{[mox-e110914l-1.049]}
\xe

Another semantic subclass of nouns that has been identified as a typical member of the inalienable class is spatial relations \citep[cf.][572]{Nichols1988}. There is indeed a small class of relational nouns\is{relational noun|(} in Paunaka, which can take the \isi{locative marker}. They express some specific spatial relations and are juxtaposed\is{juxtaposition} to the noun denoting the ground. The latter acts as the \isi{possessor} of the relational noun. There is also a number of free spatial nouns like \textit{anÿke} ‘up, above’, \textit{apuke} ‘ground, down’ and \textit{pÿkÿjÿe} ‘middle’ which are never possessed. Here is one example of a spatial relation that is inalienably possessed by the noun referring to the ground, more examples are given in \sectref{sec:Locative}.\footnote{This kind of construction is very similar to the one in Baure, see \citet[72--74]{Admiraal2016}.}

(\ref{ex:POSS-under}) comes from an elicitation session with Juana and María S. with playmobil toys.\footnote{Note that Juana sometimes uses \textit{-pÿtapaiku-bu} instead of \textit{-bÿtupaiku-bu} ‘fall’ for reasons unknown to me.}

\ea\label{ex:POSS-under}
\begingl
\glpreamble chÿupekÿ mura chipÿtapaikutu\\
\gla chÿ-upekÿ mura chi-bÿtupaiku-tu\\
\glb 3-place.under horse 3-make.fall-\textsc{iam}\\
\glft ‘it is under the horse, it throw it down’
\endgl
\trailingcitation{[jrx-c151024lsf]}
\xe
\is{relational noun|)}

As can be seen from the previous examples, (\ref{ex:leaves}), (\ref{ex:branches}), and (\ref{ex:Npossi-1}) to (\ref{ex:POSS-under}), if the possessor is expressed lexically, \isi{word order} in the NP\is{noun phrase} is always possessed – possessor.

Free nouns can be derived\is{derivation|(} from inalienables by addition of the suffix \textit{-ti}, see (\ref{ex:sane-asaneti}) and (\ref{ex:mukiji-mukitiji}). This is the only “non-possessed” suffix in the language.

\ea\label{ex:sane-asaneti}
\begingl 
\glpreamble nisane – asaneti\\
\gla ni-sane asane-ti\\ 
\glb 1\textsc{sg}-field field-\textsc{npossd}\\ 
\glft ‘my field – field’
%\trailingcitation{[]}
\xe

\ea\label{ex:mukiji-mukitiji}
\begingl 
\glpreamble nimukiji – mukitiji\\
\gla ni-muki-ji muki-ti-ji\\ 
\glb 1\textsc{sg}-hair-\textsc{col} hair-\textsc{npossd}-\textsc{col}\\ 
\glft ‘my hair – hair’
%\trailingcitation{[]}
\xe

%mukitiji = hair
%asaneti = field
%iti = blood, but lexicalised!
%yubuti = hacha??
%ucheti = chili ?
%upiti = abeja sp. ?? /ipiti
%
%also nÿti, piti etc.
Apart from \textit{asaneti} ‘field’, non-possessed forms of inalienable nouns are not very frequent in my data. The non-possessed suffix seems to be lexicalised\is{lexicalisation} on the alienably possessed noun \textit{yubuti} ‘axe’, i.e. it is not detached, when the noun is marked for possession.\is{derivation|)}

There is no prefix for an unspecified possessor.\footnote{A specialised prefix for an unspecified or indefinite possessor does exist in many \isi{Arawakan languages}, e.g. in closely related \isi{Baure}  \citep[119--120]{Danielsen2007} and more distantly related Tariana \citep[123]{Aikhenvald2003}.} %In the closely related language Baure, free nouns can be derived from body part terms with the prefix \textit{e-}, which marks an unspecified possessor, i.e. the body part is still possessed, but it is not clear by whom \citep[119--120]{Danielsen2007}.
Instead of deriving a non-possessed form or marking an unspecified possessor\is{possessor|(} on the inalienable noun, Paunaka speakers resort to a different strategy: a third person or a first person plural marker\is{person marking|(} can be used to express a more general reference of a noun. 

Plant parts, for instance, are never marked as non-possessed in the corpus. They take a third person marker by default, even when the part is detached from the plant. The same is true for the noun \textit{chÿeche} ‘meat’ (\textit{chÿ-eche} 3-flesh, lit.: ‘his/her/its flesh’), note that the same noun is used to refer to flesh as a body part, e.g. \textit{nÿeche} ‘my flesh’ (\textit{nÿ-eche} 1\textsc{sg}-flesh).

The use of a third person marker to indicate a general possessor was also the solution the PDP team chose for the production of a poster with body part terminology for the speaker community \citep[cf.][]{PDP2013}. Although it was generally agreed on by the speakers that this was correct, we later found out that for general reference to body parts, such as in school books, medical descriptions etc., people rather use the first person plural marker \textit{bi-},\footnote{Nouns with first person plural possessors are also found in some of the few entries for Paunaka vocabulary by d’Orbigny without being analysed as such (Danielsen 2021, p.c.).} consider (\ref{ex:Npossi-3}), which comes from Juana who was telling me about the medical use of the soursop.\footnote{I translate \textit{-kÿna} with ‘heart’ here, which is the core meaning of the noun, but it is also used for the stomach and the whole interior of the torso \citep[cf.][265]{TerhartDanielsenBODY}. Thus Juana could have also meant one of these concepts rather than precisely the heart.}

\ea\label{ex:Npossi-3}
\begingl
\glpreamble jaja upichai tÿpi bikÿna\\
\gla jaja upichai tÿpi bi-kÿna\\
\glb \textsc{afm} medicine \textsc{obl} 1\textsc{pl}-heart\\
\glft ‘yes, it is medicine for our heart’
\endgl
\trailingcitation{[jxx-e150925l-1.066]}
\xe
\is{person marking|)}
\is{possessor|)}
\is{inalienability|)}

\subsection{Alienable nouns}\label{sec:Alienables}
\is{alienability|(}

Alienable possession is “less permanent and inherent” than inalienable possession \citep[4]{ChappelMcGregor1996}. Nouns denoting manipulable objects usually belong to the class of alienable nouns and this is also true for Paunaka. Loans\is{borrowing} from Spanish (and less often Bésiro) are also typically alienable, even the ones denoting friends and kins – those latter ones usually show up in the derived, inalienable form (see below). Grammatically, alienability is reflected by the fact that these nouns are free forms in the first place, but can be marked as possessed.

There are two different sub-classes of alienably possessed nouns. A small number of nouns can be marked for possession directly, i.e. the only difference between a possessed and an unpossessed form is the presence of a person marker indexing the possessor.\is{person marking}\is{possessor|(} Some examples are given in \tabref{table:Inalienables1}.

\begin{table}
\caption{Alienable nouns that can be marked for possession directly}

\begin{tabular}{lll}
\lsptoprule
Non-possessed form & Possessed form (1\textsc{sg}) & Translation\cr
\midrule
\textit{kasune} & \textit{nikasune} & (my) trousers \cr
\textit{kuepia} & \textit{nikuepia} & (my) kidney\cr
\textit{nÿkÿiki} & \textit{ninÿkÿiki} & (my) pot \cr
\textit{pusane} & \textit{nipusane} & (my) bag \cr
\textit{yumaji} & \textit{niyumaji} & (my) hammock \cr
\lspbottomrule
\end{tabular}

\label{table:Inalienables1}
\end{table}

(\ref{ex:ali-1}) has a possessed form of \textit{-yumaji} ‘hammock’, which does not take the possessed suffix. Compare with (\ref{ex:ali-2}), in which the word occurs in non-possessed form.

(\ref{ex:ali-1}) comes from María C., who was afraid that her hammock would get wet because it was about to rain.

\ea\label{ex:ali-1}
\begingl
\glpreamble kaku niyumaji nekupai\\
\gla kaku ni-yumaji nekupai\\
\glb exist 1\textsc{sg}-hammock outside\\
\glft ‘my hammock is outside’
\endgl
\trailingcitation{[cux-120410ls.258]}
\xe


(\ref{ex:ali-2}) was a question by Juana directed to me.

\ea\label{ex:ali-2}
\begingl
\glpreamble ¿pisachu pibena yumaji?\\
\gla pi-sachu pi-bena yumaji\\
\glb 2\textsc{sg}-want 2\textsc{sg}-lie.down.\textsc{irr} hammock\\
\glft ‘do you want to lie down in the hammock?’
\endgl
\trailingcitation{[jxx-p150920l.017]}
\xe


The larger number of alienable nouns takes the suffix \textit{-ne} to derive\is{derivation|(} a possessable form. This form can be considered inalienable, since it obligatorily takes a person marker\is{person marking} for the possessor. Two nouns in \tabref{table:Inalienables1}, \textit{kasune} ‘trousers’ and \textit{pusane} ‘bag’ also end in \textit{-ne} in their non-possessed form.\footnote{In addition, there are also a few inalienable nouns that end in \textit{ne}, e.g. \textit{-etine} ‘sister (of a male person)’ and \textit{-machapene} ‘liver’.} Both are borrowed\is{borrowing} from \isi{Bésiro} or Proto-Chiquitano, \textit{pusane} derives from Proto-Chiquitano \textit{*/pɨtsaná-ʂɨ/} \citep[10]{Nikulin2019}, and \textit{kasune} from Bésiro \textit{<kasuná-x>}, which is itself a loan from Spanish \textit{calzón} ‘pants, underpants, shorts’ \citep[12]{Nikulin2019}. Thus in both cases the syllable \textit{ne} can be considered part of the root.

Nouns derived with \textit{-ne} take a person marker to index the possessor just like other inalienable nouns.\is{possessor|)} According to \citet[378]{Payne1991}, \textit{-ne} is the most common possessive suffix among the \isi{Arawakan languages}. It is the only one that Paunaka productively makes use of. This is mentioned here explicitly because some other Arawakan languages have more than one.

(\ref{ex:Alienables2-1}) and (\ref{ex:Alienables2-2}) show inalienable nouns in their non-possessed and possessed forms follow.%\footnote{There are different kinds of baskets of different materials and for different uses. \textit{Sÿki} is \textit{jasayé} in Spanish, it is the one that resembles a bag, \textit{panaku} corresponds to Spanish \textit{panacú}, this one can be carried on the back. The \textit{chupai}, Spanish \textit{quiboro}, is rather used to store away things.}

\ea\label{ex:Alienables2-1}
\begingl 
\glpreamble tÿmuepa – nitÿmuepane\\
\gla tÿmuepa ni-tÿmuepa-ne\\ 
\glb knife 1\textsc{sg}-knife-\textsc{possd}\\ 
\glft ‘knife – my knife’
\endgl
%\trailingcitation{[mox-n110920l.021]}
\xe

\ea\label{ex:Alienables2-2}
\begingl 
\glpreamble sÿki – nisÿkine\\
\gla sÿki ni-sÿki-ne\\ 
\glb basket 1\textsc{sg}-basket-\textsc{possd}\\ 
\glft ‘basket – my basket’
%\trailingcitation{[cux-c120414ls-1.177]}
\xe

%yubuti - niyubutine (mxx-e181017l)
\is{derivation|)}

The overwhelming number of nouns belonging to the class of alienables are loans\is{borrowing} from Spanish denoting either objects or people with a kinship or other social relation to the possessor. (\ref{ex:Alienables2-3}) shows a borrowed kinship term, \textit{-kumare}\footnote{\textit{Kumare} comes from Spanish \textit{comadre} and is used to denote either the godmother of one’s child or godchild or a close friend of the same age group. It has several other phonetic realisations, e.g. \textit{kumade}.} with and without possession marking occurring in a single sentence. It comes from Juana who was talking about the trip to Europe.

\ea\label{ex:Alienables2-3}
\begingl 
\glpreamble echÿu nikumarene nauku Concecion, kumare Nacha, kuina tisacha tiyuna, tÿbaneyu\\
\gla echÿu ni-kumare-ne nauku Concecion kumare Nacha kuina ti-sacha ti-yuna ti-ÿbane-yu\\ 
\glb \textsc{dem}b 1\textsc{sg}-fellow-\textsc{possd} there Concepción fellow Nacha \textsc{neg} 3i-want.\textsc{irr} 3i-go.\textsc{irr} 3i-be.far-\textsc{ints}\\ 
\glft ‘my fellow there in Concepción, fellow Nacha, doesn’t want to go, because it is very far’
\trailingcitation{[jxx-p120430l-1.175]}
\xe

(\ref{ex:Alienables2-4}) has the Spanish loan \textit{kama} from \textit{cama} ‘bed’, which denotes an object. It comes from Isidro who was describing a picture of a puzzle game.

\ea\label{ex:Alienables2-4}
\begingl 
\glpreamble timuku eka chikamaneyae\\
\gla ti-muku eka chi-kama-ne-yae\\ 
\glb 3i-sleep \textsc{dem}a 3-bed-\textsc{possd}-\textsc{loc}\\ 
\glft ‘he sleeps in his bed’
\trailingcitation{[dxx-d120416s.002]}
\xe

If attributive verbs\is{attributive prefix} (see \sectref{sec:AttributiveVerbs}) are derived\is{derivation} from alienable nouns, the possessed form of the noun is used as in (\ref{ex:ali-3}), which comes from elicitation with María S.

\ea\label{ex:ali-3}
\begingl
\glpreamble nikupanakune\\
\gla ni-ku-panaku-ne\\
\glb 1\textsc{sg}-\textsc{attr}-basket-\textsc{possd}\\
\glft ‘I have a basket (on the back)’
\endgl
\trailingcitation{[rxx-e181020le]}
\xe
\is{alienability|)}
 
\subsection{Non-possessable nouns}\label{sec:Non-possessables}
\is{non-possessability|(}

Paunaka has a number of non-possessable nouns, which cannot take person markers.\is{person marking} Some of them can be possessed indirectly, they “require an additional grammatical element joining the two constituents” (i.e. the possessed and the possessor) \citep[58]{Krasnoukhova2012}. This is achieved by using a possessable noun including the \isi{possessor} marking in \isi{juxtaposition} to the non-possessable noun. The possessed noun always precedes the non-possessable one in this case, which can be seen in (\ref{ex:new23nonposs}) from Isidro describing a puzzle game on which a boy plays with a squirrel. 

\ea\label{ex:new23nonposs}
\begingl
\glpreamble chipeu mase\\
\gla chi-peu mase\\
\glb 3-animal squirrel\\
\glft ‘his squirrel’
\endgl
\trailingcitation{[mdx-c120416ls.177]}
\xe


\citet[82]{Aikhenvald1999} mentions that non-possessable nouns in \isi{Arawakan languages} “may include astronomical bodies, natural phenomena, harmful animals and personal names”. %\citet[58]{Krasnoukhova2012}: they “cannot occur with the possessor directly, and therefore require an additional grammatical element joining the two constituents”.
In Paunaka, at least some speakers allow a possessed form of astronomical bodies if used in a metaphorical context, e.g. in talking to one’s lover, see (\ref{ex:possessed-star}), which was elicited from Miguel.

\ea\label{ex:possessed-star}
\begingl 
\glpreamble nÿjaikene\\
\gla nÿ-jaike-ne\\ 
\glb 1\textsc{sg}-star-\textsc{possd}\\ 
\glft ‘my star’
\trailingcitation{[mxx-e181017l]}
\xe

This shows that non-possessability is for semantic reasons and does not have to do with morphological properties of the noun in question. \citet[170]{Aikhenvald2012} explains that it is common sense among speakers of Amazonian languages\is{Amazonian language} that some items simply cannot be possessed, but the principles underlying are language and culture specific.

The most important group of non-possessable nouns in Paunaka is animals. Besides harmful animals as  predicted by \citet[82]{Aikhenvald1999}, this also includes pets, but not parasites like lice and worms, which are inalienably possessed. As has already been shown in (\ref{ex:new23nonposs}) above, the indirect strategy to express possession of animals includes the inalienable noun \textit{-peu} ‘domestic animal’ as a relational noun \is{relational noun|(}.\footnote{It is only given as ‘animal’ in the glosses of examples for the sake of brevity.} This noun carries the person marker\is{person marking} and the noun denoting the specific animal follows.\footnote{This type of construction has also been treated in the literature under the heading of genitive classifiers\is{classifier} (cf. \citealt[66]{Grinevald2000}; \citealt[283]{Campbell2012}) or possessive classifiers \citep[]{Fabre2014}. Languages with this kind of classifiers\is{classifier} often have more elaborate systems, but if a language in Amazonia has only one,\is{Amazonian language} it is usually the one for ‘domestic animal’ (Rose 2021, p.c.).} 

Another example is given in (\ref{ex:POSS-animal}). It comes from María S., who was complaining that her chicken get stolen when she leaves her house.

\ea\label{ex:POSS-animal}
\begingl 
\glpreamble kuina dejaunubeina nipeu takÿra\\
\gla kuina dejau-nube-ina ni-peu takÿra\\ 
\glb \textsc{neg} leave-\textsc{pl}-\textsc{irr.nv} 1\textsc{sg}-animal chicken\\ 
\glft ‘they don't leave my chicken alone’ (i.e. they steal them)
\trailingcitation{[rxx-e120511l.179]}
\xe

(\ref{ex:aniposs-1}) comes from Miguel telling Alejo the \isi{frog story}. He describes the picture on which the boy stands on the stone. 

\ea\label{ex:aniposs-1}
\begingl
\glpreamble pero kapunuji echÿu chipeu kabemÿnÿ\\
\gla pero kapunu-ji echÿu chi-peu kabe-mÿnÿ\\
\glb but come-\textsc{rprt} \textsc{dem}b 3-animal dog-\textsc{dim}\\
\glft ‘but his dog is coming, it is said’
\endgl
\trailingcitation{[mtx-a110906l.147]}
\xe


This pattern, i.e. the expression of possession of an animal by \isi{juxtaposition} of a possessed noun ‘domestic animal’ and the name of the specific animal, is shared with \isi{Terena}, the \isi{Mojeño languages}, and \isi{Baure} (cf. \citealt[50]{ButlerEkdahl2012}; \citealt[51]{OlzaZubiri2004}; Rose 2021, p.c.; \citealt[123--124]{Danielsen2007}). In the Mojeño languages, however, the cognate nouns only denotes rideable animals. Trinitario\is{Mojeño Trinitario} uses a more general relational noun for possession of non-rideable animals \citep[cf.][79]{Rose2014}.\footnote{Ignaciano\is{Mojeño Ignaciano} on the other hand uses another specific noun for non-rideable domestic animals and the more general one for non-animals \citep[51]{OlzaZubiri2004}.} Paunaka also has a general relational noun \textit{-yae} (cognate to the Mojeño one). It is semantically unspecific and identical in form with the \isi{locative marker}. 

In (\ref{ex:possessed-town}) the relational noun occurs in juxtaposition to the noun \textit{uneku} ‘town’. According to Miguel, who provided this example in elicitation, this is the correct way to express the notion of one’s town. According to María S., however, ‘town’ cannot be possessed at all.

\ea\label{ex:possessed-town}
\begingl 
\glpreamble niyae uneku\\
\gla ni-yae uneku\\ 
\glb 1\textsc{sg}-\textsc{grn} town\\ 
\glft ‘my town’
\trailingcitation{[mxx-e181017l]}
\xe

According to the analysis by \citet[]{Rose2019a} for the Trinitario\is{Mojeño Trinitario} equivalent of Paunaka’s \textit{-yae}, it first arose as a general relational noun before spreading to other contexts. As such, it has cognate forms in other \isi{Arawakan languages} (\citealt[14]{Rose2019a}, and consider also the “possessive \textit{-ya-}” of Tariana, \citealt[134]{Aikhenvald2003}). Note that \citet[150]{Danielsen2007} also considers that the \isi{Baure} locative marker \textit{-ye}, another cognate of Paunaka’s \textit{-yae}, could be a \isi{nominal root}. Since function and morphosyntactic contexts are relatively different in current Paunaka, I use two different glosses for \textit{-yae}. If it occurs in possession contexts together with a person marker,\is{person marking} I call it general relational noun, abbreviated \textsc{grn} in following \citet[]{Rose2019a}, in those contexts where it is attached to a noun, I call it a \isi{locative marker}, abbreviated \textsc{loc} (for locative marking see \sectref{sec:Locative}).

\largerpage
In current Paunaka, the general relational noun is found in genitive predication (see \sectref{sec:GenitiveBenfactivePreds}), but it can also occur in contexts of attributive possession. Possession of crops can be expressed in this way. I first came across this in one of the recordings  of the 1960s by Riester, see (\ref{ex:ATTRyae}). 

\ea\label{ex:ATTRyae}
\begingl 
\glpreamble akomoraupuna niyaemÿnÿ arusu\\
\gla akomorau-puna ni-yae-mÿnÿ arusu\\ 
\glb accommodate-\textsc{am.prior}.\textsc{irr} 1\textsc{sg}-\textsc{grn}-\textsc{dim} rice\\ 
\glft ‘I will go to store away my rice’
\trailingcitation{[nxx-p630101g-1.006]}
\xe

Elicitation showed that crops can either be marked as possessed in this way or by using the alienable possession strategy explained above (see \sectref{sec:Alienables}). Which strategy is used for which crop may depend on frequency and speaker. (\ref{ex:ucheti23}) is another example of a crop that was spontaneously marked for possession by using the general relational noun \textit{-yae} in elicitation:

\ea\label{ex:ucheti23}
\begingl 
\glpreamble niyae ucheti\\
\gla ni-yae ucheti\\ 
\glb 1\textsc{sg}-\textsc{grn} chili\\ 
\glft ‘my chili’
\trailingcitation{[rxx-e181018le]}
\xe

 %Besɨro besides a possessive classifier for domestic animals also has a possessive classifier for domesticated plants \citep[cf.][21--22]{Sans2013}.

Instead of using one of the relational nouns \textit{-peu} or \textit{-yae}, speakers may also use other possessable nouns in juxtaposition to the non-possessable ones. Consider (\ref{ex:water-possessed}), where the natural resource of water is marked as possessed by a preceding possessed noun that simultaneously acts as a measure term for the \isi{mass noun}. The noun \textit{tapiki} is borrowed from Spanish \textit{tapeque} or Proto-Chiquitano \textit{tapiki} ‘travel supplies’ \citep[cf.][9]{Nikulin2019}. This example also comes from elicitation with María S.

\ea\label{ex:water-possessed}
\begingl 
\glpreamble pitapikine ÿne\\
\gla pi-tapiki-ne ÿne\\ 
\glb 2\textsc{sg}-travel.supplies-\textsc{possd} water\\ 
\glft ‘your travel supplies of water’
\trailingcitation{[rxx-e181018le]}
\xe

\is{relational noun|)}

\is{non-possessability|)}
\is{possession|)}

The next section is about number marking on nouns.

%!TEX root = 3-P_Masterdokument.tex
%!TEX encoding = UTF-8 Unicode

\section{Number}\label{sec:NumberNouns}

Plural marking is obligatory with human\is{animacy|(} referents. There is one \isi{plural} marker \textit{-nube}, which is largely restricted to human nouns (see \sectref{sec:NounsPL-nube}). In addition, the \isi{distributive} marker \textit{-jane} can be used to signal non-singularity of non-human referents, usually animate ones.\is{animacy|)} It is described in \sectref{sec:NounPL-jane}. The \isi{collective} marker \textit{-ji} is used with nouns of two different semantic classes: things which are little individuated, since they occur in masses or swarms and kinship terms with the plural marker (used for both plural kin and plural possessors) (see \sectref{sec:Collective}). Although distributives and collectives\isi{collective} are not part of the number system according to \citet[117, 119, 120]{Corbett2000}, they are certainly semantically related, since they also provide information about quantity. This is why they are all subsumed under the heading of “number” here. All three markers are also found on verbs\is{verb} (see \sectref{sec:Verbs_3PL}).

\subsection{The plural marker}\label{sec:NounsPL-nube}
\is{plural|(}

The plural marker \textit{-nube} is obligatory with non-singular human nouns.\is{animacy} An example of such a constellation is given in (\ref{ex:pimiyanube}), where the noun \textit{(a)pimiya} ‘girl, young woman’ takes this marker.
Juana is speaking about the production of traditional clay pots here.

\ea\label{ex:pimiyanube}
\begingl 
\glpreamble i tanÿma kuina tanabunube pimiyanube\\
\gla i tanÿma kuina ti-ana-bu-nube pimiya-nube\\ 
\glb and now \textsc{neg} 3i-make.\textsc{irr}-\textsc{dsc}-\textsc{pl} girl-\textsc{pl}\\ 
\glft ‘and today the young women don’t make them any more’
\trailingcitation{[jxx-p120430l-2.547]}
\xe

(\ref{ex:DEMa-noAGR}) comes from Miguel who was happy that Swintha knew a word he had forgotten because:

\ea\label{ex:DEMa-noAGR}
\begingl 
\glpreamble tiyÿsebÿkeunÿnube eka aitubuchepÿinube naka unekuyae\\
\gla ti-yÿsebÿkeu-nÿ-nube eka aitubuchepÿi-nube naka uneku-yae\\ 
\glb 3i-ask-1\textsc{sg}-\textsc{pl} \textsc{dem}a boy-\textsc{pl} here town-\textsc{loc}\\ 
\glft ‘the boys here in town asked me (about it)’
\trailingcitation{[mdx-c120416ls.121]}
\xe

The noun \textit{aitubuche} ‘boy, young man’ in (\ref{ex:DEMa-noAGR}) is a loan from Bésiro, and the plural marker can also be used with Spanish loans. An example is (\ref{ex:kristi-1}) with a plural-marked version of the noun \textit{kristianu} ‘person’, borrowed from the Spanish noun \textit{cristiano} ‘Christian person’. The sentence comes from the recordings of the 1960s with Juan Ch., who introduced his playing the flute with a few words.

\ea\label{ex:kristi-1}
\begingl 
\glpreamble ¡esamu!, kristianunube\\
\gla e-samu kristianu-nube\\ 
\glb 2\textsc{pl}-hear person-\textsc{pl}\\ 
\glft ‘listen, people!’
\trailingcitation{[nxx-a630101g-2.002]}
\xe

The Spanish word \textit{gente} ‘people’ is borrowed as a countable noun \textit{jente} ‘man’ into Paunaka. (\ref{ex:jentenube}) shows an occurrence of this noun with the plural marker. It comes from Juana telling about the work of the people of Santa Rita in exchange for the construction of their reservoir.

\ea\label{ex:jentenube}
\begingl 
\glpreamble tropanube eka jentenube trabakunube\\
\gla tropa-nube eka jente-nube trabaku-nube\\ 
\glb pack-\textsc{pl} \textsc{dem}a man-\textsc{pl} work-\textsc{pl}\\ 
\glft ‘the men worked in packs’
\trailingcitation{[jxx-p120515l-2.112]}
\xe

Two nouns contain the plural marker as lexicalised\is{lexicalisation|(} part of the stem.\is{nominal stem} One is \textit{mu\-pÿi\-nube} ‘devil’. The noun is composed of the privative\is{privative|(} marker \textit{mu-},\footnote{The privative marker is not productive anymore in Paunaka, but it can be traced back to Proto-Arawakan \textit{*ma-}. Many \isi{Arawakan languages} have (productive) reflexes of this prefix \citep[276]{Michael2014b}.\is{privative|)}} the body-part term \textit{-pÿi} ‘body’ and the plural marker, signifying thus ‘the ones without body’, a term that presumably goes back to pre-Christian belief in spirits. The other noun is \textit{seunube} ‘woman’. I cannot offer any explanation why the plural marker lexicalised with the root\is{nominal root} \textit{*seu}. The plural of \textit{seunube} ‘woman’ is \textit{seunubenube} ‘women’, see (\ref{ex:Plurili-1}). Juana counts the Supepí siblings here.

\ea\label{ex:Plurili-1}
\begingl
\glpreamble trexenubechÿ seunubenube i ruxhnubechÿ jentenube\\
\gla trexe-nube-chÿ seunube-nube i ruxh-nube-chÿ jente-nube\\
\glb three-\textsc{pl}-3 woman-\textsc{pl} and two-\textsc{pl}-3 men-\textsc{pl}\\
\glft ‘the women are three and the men are two’
\endgl
\trailingcitation{[jxx-p120515l-2.239]}
\xe
\is{lexicalisation|)}

There is also one plural-only noun: \textit{sesejinube} ‘children’. The corresponding singular forms are either \textit{sepitÿ} ‘small, child’ or the gender-specific \textit{(a)pimiya} ‘girl, young woman’ and \textit{aitubuche} ‘boy, young man’, see (\ref{ex:pimiyanube}) and (\ref{ex:DEMa-noAGR}) above.

Plural is usually marked on both the noun and the verb if the noun conominates\is{conomination} a subject or object index. There is thus a kind of \isi{agreement} in number between verb and noun, see (\ref{ex:pimiyanube}) and (\ref{ex:DEMa-noAGR}) above.

The plural marker \textit{-nube} does not occur on non-human nouns\is{animacy|(} with few exceptions. First, anthropomorphic characters in narratives can take the plural marker. However, in my data I only found this for verbs (see \sectref{sec:Verbs_3PL}).\footnote{The reason for this is that there is hardly any story in which two or more animals or other anthropomorphic characters would be of the same species or kind, so that there is not much possibility for plural marking on a noun referring to them (like ‘the foxes’). Of course, they could also be denominated by another noun that does not specify the species like ‘the friends’, but this is not the case in the stories I collected. There is one interesting example from a story told by María S. about how the tortoise obtained its carapace, in which there is a mismatch between plural marking on the verb and \isi{distributive} marking on the noun, see (\ref{ex:mismatch}); however, this cannot be generalised.

\ea\label{ex:mismatch}
\begingl
\glpreamble te chisamunubetuji eka ubechajane\\
\gla te chi-samu-nube-tu-ji eka ubecha-jane\\
\glb \textsc{seq} 3-hear-\textsc{pl}-\textsc{iam}-\textsc{rprt} \textsc{dem}a sheep-\textsc{distr}\\
\glft ‘then the sheep heard it, it is said’
\endgl
\trailingcitation{[rxx-n121128s.10]}
\xe}

Second, a few inanimate nouns occasionally take the plural marker. The noun \textit{anyo} ‘year’, a loan\is{borrowing} from Spanish \textit{año}, is such a case, which can be seen in (\ref{ex:anyonube}), where Juana talks about her mother who was ill for a long time.

\ea\label{ex:anyonube}
\begingl 
\glpreamble tibenunukubu yumaji anyonube\\
\gla ti-benunuku-bu yumaji anyo-nube\\ 
\glb 3i-lie-\textsc{mid} hammock year-\textsc{pl}\\ 
\glft ‘she lay in the hammock for years’
\trailingcitation{[jxx-p120430l-2.501]}
\xe

Furthermore, the noun \textit{ubiae} ‘house’ can take the plural marker when reference is to multiple houses, see (\ref{ex:ubiyaenube}). This noun is a special case, though, because it derives\is{derivation} from a verb (\textit{-ubu} ‘be, live’), thus the use of the plural marker may be a relict of subject number marking. Juan C. talks about his village, San Miguelito de la Cruz, in this example.

\ea\label{ex:ubiyaenube}
\begingl 
\glpreamble kakiu nechÿu pario ubiaenube\\
\gla kakiu nechÿu pario ubiae-nube\\ 
\glb exist.\textsc{subord}? \textsc{dem}c some house-\textsc{pl}\\ 
\glft ‘there are some houses’
\trailingcitation{[mqx-p110826l.182]}%kakiu
\xe

In addition, \textit{ubiae} can also take the \isi{distributive} marker \textit{-jane} as in (\ref{ex:ubiyaejane}), where Miguel talks with Alejo and Polonia about the current state of \isi{Altavista}.

\ea\label{ex:ubiyaejane}
\begingl 
\glpreamble i tanÿmatu echÿu ubiaejane kuinabutu\\
\gla i tanÿma-tu echÿu ubiae-jane kuina-bu-tu\\ 
\glb and now-\textsc{iam} \textsc{dem}b house-\textsc{distr} \textsc{neg}-\textsc{dsc}-\textsc{iam}\\ 
\glft ‘and now the houses do not exist anymore’
\trailingcitation{[mty-p110906l.200-201]}
\xe

There are not many occurrences of \textit{ubiae} with the plural marker in my corpus and even less with the \isi{distributive} marker, which is connected to the fact that non-human nouns do not have to be marked for number at all.\is{animacy|)} 

The nominal demonstratives can take the plural marker, when used pronominally, as in (\ref{ex:Plurili-2}), which was elicited from Miguel. If they modify the noun, there is usually no plural marking on the demonstratives, see also \sectref{sec:NP}.

\ea\label{ex:Plurili-2}
\begingl
\glpreamble echÿunube tichujijikubunube\\
\gla echÿu-nube ti-chujijiku-bu-nube\\
\glb \textsc{dem}b-\textsc{pl} 3i-talk-\textsc{mid}-\textsc{pl}\\
\glft ‘they are chatting’
\endgl
\trailingcitation{[mrx-e150219s.011]}
\xe
\is{plural|)}

\subsection{The distributive marker}\label{sec:NounPL-jane}\is{distributive|(}

The plural marker cannot be used with non-human nouns,\is{animacy} but there is another marker, \textit{-jane}, used mainly to express plurality of animals, as in (\ref{ex:kabejane}). It was produced by Miguel, but in the story he was telling, it is uttered by the jaguarundi, who warns his companion, the drunken fox, to stop singing lest he calls the attention of the dogs. While the fox and the jaguarundi are anthropomorphic characters and thus subject of plural marking, the dogs are not; they behave like dogs and they do not speak but bark. Use of a distributive form makes clear that there are several dogs that could harm them, thus marking the situation extremely dangerous.

\ea\label{ex:kabejane}
\begingl 
\glpreamble “¡tch xhhh, kaku kabejane naka, kaku kabejane naka!”\\
\gla tch xhhh kaku kabe-jane naka kaku kabe-jane naka\\ 
\glb \textsc{intj} \textsc{intj} exist dog-\textsc{distr} here exist dog-\textsc{distr} here\\ 
\glft ‘“shh, shhh, there are dogs around here, there are dogs around here!”’
\trailingcitation{[jmx-n120429ls-x5.381]}
\xe

The marker is called “distributive marker” in this grammar, although this term might be a bit misleading. According to \citet[112]{Corbett2000}, the primary function of distributive marking on nouns is to “spread (distribute) various entities over various locations or over various sorts (types)”. In current Paunaka, the function of the distributive is rather to express \textit{overtly} that there are various non-humans tokens,\is{animacy|(} since number of non-human entities does not have to be specified at all. I had priorly just glossed the marker as a non-human \isi{plural} until I noticed that there are a few cases in which \textit{-jane} occurs together with \textit{-nube}. In these cases, there would be a semantic mismatch if \textit{-jane} was analysed as a non-human plural marker.

I have found three such cases. First of all, there is a \isi{question word} \textit{kajane} ‘how many’ and a quantifying stative verb \textit{-kijane} ‘be many’, where the marker is a lexicalised\is{lexicalisation} part of the stem.\footnote{The root \textit{ka} of \textit{kajane} is probably a \isi{demonstrative} element to which \textit{-jane} is added, see \sectref{sec:DemPron}. The composition of \textit{-kijane} is opaque, the element \textit{*ki} could not be identified.} Both add the plural marker when they refer to quantities of humans. In addition, \textit{-jane} also shows up in the plural form \textit{pujane(nube)} ‘others’ of the singular form \textit{punachÿ} ‘other’. Distributives encode distinctiveness or individuation of referents \citep[116]{Corbett2000}, each member of a group is perceived individually in contrast to perceiving plurality as a unit. I suppose this may have once been the primary function of \textit{-jane}, and this is still well visible in the \isi{question word} \textit{kajane} ‘how many’. Asking for a number presupposes that each member of a group is counted individually. Nonetheless, the primary function of the distributive marker in current-day Paunaka is plural-marking of non-human referents.\is{plural} It is never attached to human nouns nor to verbs in reference to humans. Among the possible non-human referents, it is more commonly found with animate than with inanimate nouns and bigger, more individuated animals, like dogs, cows, and to a lesser extent pigs, are more likely to be marked by the distributive than smaller and less individuated animals like chicken and fish.\is{animacy|)}

In (\ref{ex:bakajane}), the distributive marker attaches to \textit{baka} ‘cow’. The example comes from Juana who was telling me about the journey of her grandparents back home from Moxos. They had bought cows there. It is a long way from Moxos to the Chiquitania, which the grandparents went by foot. They slept in huts or temporary shelters and let the cows in enclosures they found along the way.

\ea\label{ex:bakajane}
\begingl 
\glpreamble kaku eka bakayayae eka bakajane\\
\gla kaku eka bakaya-yae eka baka-jane\\ 
\glb exist \textsc{dem}a enclosure-\textsc{loc} \textsc{dem}a cow-\textsc{distr}\\ 
\glft ‘the cows were in the enclosure’
\trailingcitation{[jxx-p151016l-2.030]}
\xe

Contrary to the plural marker, \textit{-jane} usually occurs only once in a clause, either on the predicate or on the NP conominating\is{conomination} subject or object, with some exceptions. Which factors determine the choice of either predicate or NP\is{noun phrase} taking the distributive marker remains to be investigated.\footnote{It seems to be the case that distributive marking on the \isi{copula} \textit{kaku} ‘exist’ is generally avoided though not impossible.} Thus, there is usually no \isi{agreement} in \textit{-jane} between the NP and the predicate, although a few counterexamples exist. All examples in this section show the use on the NP.

%In most of the cases in which the distributive is used to signal non-singularity of animals, the NP is unmarked and \textit{-jane} occurs on the verb. However, sometimes \textit{-jane} is marked on the noun, but not on the verb as in (\ref{ex:no-agr-jane-3}).

 (\ref{ex:no-agr-jane-3}) is another example with dogs, it comes from Juana who was telling me about her own dogs.

\ea\label{ex:no-agr-jane-3}
\begingl 
\glpreamble tichaneikune eka kabejane\\
\gla ti-chaneiku-ne eka kabe-jane\\ 
\glb 3i-care.for-1\textsc{sg} \textsc{dem} dog-\textsc{distr}\\ 
\glft ‘the dogs protect me’
\trailingcitation{[jxx-e150925l-1.093]}
\xe

\largerpage
(\ref{ex:distri-3}) is an example of the distributive marker on the word for ‘pig’ and was elicited from Miguel.

\ea\label{ex:distri-3}
\begingl
\glpreamble tibÿjaneupuku ÿbajane\\
\gla ti-bÿ-jane-u-pu-uku ÿba-jane\\
\glb 3i-go.in-\textsc{distr}-\textsc{real}-\textsc{dloc}-\textsc{add} pig-\textsc{distr}\\
\glft ‘the pigs also go inside’
\endgl
\trailingcitation{[mrx-e150219s.102]}
\xe

In (\ref{ex:distri-1}), there are two inanimate\is{animacy} nouns with plural referents, the ‘stones’ and the ‘adobe bricks’; however, only the first one takes the distributive marker. The example comes from Miguel who told me about the construction of the school building in Santa Rita a long time ago.

\ea\label{ex:distri-1}
\begingl
\glpreamble entonses bisemaikutu echÿu maijane banautu echÿu arubi\\
\gla entonses bi-semaiku-tu echÿu mai-jane bi-anau-tu echÿu arubi\\
\glb thus 1\textsc{pl}-search-\textsc{iam} \textsc{dem}b stone-\textsc{distr} 1\textsc{pl}-make-\textsc{iam} \textsc{dem}b adobe\\
\glft ‘thus we looked for stones, we made adobe bricks’
\endgl
\trailingcitation{[mxx-p110825l.114]}
\xe

An example from Juana with two inanimate\is{animacy} distributive-marked nouns is (\ref{ex:distri-2}), in which the flower boxes they have in Cotoca, a small city famous for its ceramics, are compared to jars.

\ea\label{ex:distri-2}
\begingl
\glpreamble kaku echÿu maseterojane nena yÿpijanemÿnÿ\\
\gla kaku echÿu masetero-jane nena yÿpi-jane-mÿnÿ\\
\glb exist \textsc{dem}b flower.box-\textsc{distr} like jar-\textsc{distr}-\textsc{dim}\\
\glft ‘there are flower boxes that look like jars'
\endgl
\trailingcitation{[jxx-p120430l-2.616]}
\xe
\is{distributive|)}

\subsection{The collective marker}\label{sec:Collective}\is{collective|(}
The marker \textit{-ji}, which can best be interpreted as a collective marker, since it is found on a number of nouns that occur in uncountable collections or groups like \textit{-mukiji} ‘hair’ and  \textit{mÿiji} ‘grass’. It is used with certain plant parts perceived as a collection rather than as countable objects like \textit{yÿkÿkekeji} ‘branches, twigs’,\footnote{As noted in \sectref{sec:RDPL_Nouns}, the collective marker sometimes causes repetition of a preceding syllable for an unknown reason.} \textit{chipuneji} ‘leaves’ and \textit{chÿiji} ‘fruits’, and on names of small fish species that occur in swarms like \textit{kÿnupeji} ‘fish sp.’ and \textit{turukeji} ‘fish sp.’ (these fish are called \textit{cupacá} and \textit{tayoca} in local Spanish) as in (\ref{ex:turukeji}), where Clara describes the fish, which are small but fat.

\ea\label{ex:turukeji}
\begingl 
\glpreamble tisabananaji echÿu turukeji\\
\gla ti-sabana-na-ji echÿu turuke-ji\\ 
\glb 3i-be.fat-\textsc{rep}-\textsc{col} \textsc{dem}b fish.sp-\textsc{col}\\ 
\glft ‘the \textit{tayoca} fish are fat’
\trailingcitation{[cux-c120414ls-2.152]}
\xe

(\ref{ex:colli-1}) is a statement by Juana about her mother’s hair.

\ea\label{ex:colli-1}
\begingl
\glpreamble michana chimukijimÿnÿ nÿenubane\\
\gla michana chi-muki-ji-mÿnÿ nÿ-enu-bane\\
\glb nice 3-hair-\textsc{col}-\textsc{dim} 1\textsc{sg}-mother-\textsc{rem}\\
\glft ‘my late mother had beautiful hair’
\endgl
\trailingcitation{[jxx-d181102l.47]}
\xe


It is not always easy to distinguish the collective marker from one of homo\-nyms, especially the classifier\is{classifier|(} for soft masses (e.g. dough, mud), see \sectref{sec:Classifiers}. Both theoretically occur in different slots, see \figref{fig:NounTemplate} above,\footnote{This is the case e.g. in collective marked \textit{kÿnupeji}; the fish name is \textit{kÿnupe} in Paunaka %(\textit{Bujurquina oenolaemus}??
 with the classifier \textit{-pe} for flat things.} but since the nouns derived with the classifier denote (soft) masses, collective marking is not applicable to them. \textit{Muteji} ‘loam, mud’ is certainly a soft mass and \textit{-mukiji} a collection of ‘hair’, but what about \textit{-mÿuji} ‘clothes’ – is this a soft mass or a non-countable collection of individual pieces of garment? There may be a substantial semantic overlap in some cases.\footnote{Note also that \isi{Baure} has a similar marker \textit{-je} which was glossed ‘distributive’ by \citet[155--156]{Danielsen2007}, but has a collective function as well, resembling the Paunaka one (Danielsen 2021, p.c.). For the \isi{Mojeño languages}, on the other hand, the form \textit{-ji} was analysed as a classifier for amorphous items, applied among other things to “grass, leaves, small branches” \citep[17]{Rose2020}, i.e. items which I have analysed as including the collective marker. Nonetheless, Rose (2021, p.c.) confirms that a collective marker \textit{-ji} also exists in Trinitario\is{Mojeño Trinitario} and that some of the nouns priorly analysed to be built on the classifier may actually rather include the collective marker. The problem of distinguishing both markers remains for both languages, Paunaka and Trinitario.\is{Mojeño Trinitario}}\is{classifier|)}

The collective marker also shows up on kinship terms if the possessed kin or the possessor\is{possessor|(} is plural (or both).\is{plural|(} The first scenario is a case of regular plural marking on human nouns\is{animacy} (see \sectref{sec:NounsPL-nube} above), the other one relates to regular possessor marking, where addition of the plural marker to a noun bearing the third person marker\is{person marking} compensates for the non-existence of a specific third person plural person marker (see \sectref{sec:Possession}).

My hypothesis is that the collective was once used in addition to the plural marker if the possessed kin was the plural referent, while no collective marker showed up, when only the possessor was plural. There are two examples in my corpus that hint at this. In those examples, there is no collective marker, and in both, plural reference is to the possessor while the possessed kin is singular. The first one, (\ref{ex:Kin-no-ji-1}), was produced by Miguel in elicitation, the second one, (\ref{ex:Kin-no-ji-2}), occurred in spontaneous speech of María C.

\largerpage
\ea\label{ex:Kin-no-ji-1}
\begingl 
\glpreamble nÿti chÿenunube\\
\gla nÿti chÿ-enu-nube\\ 
\glb 1\textsc{sg.prn} 3-mother-\textsc{pl}\\ 
\glft ‘I am their mother’
\trailingcitation{[mxx-e090728s-3.081]}
\xe

\ea\label{ex:Kin-no-ji-2}
\begingl 
\glpreamble chibu chÿanube\\
\gla chibu chÿ-a-nube\\ 
\glb 3\textsc{top.prn} 3-father-\textsc{pl}\\ 
\glft ‘he is their father’
\trailingcitation{[cux-c120414ls-1.114]}
\xe

In most cases, however, the collective marker also occurs if there is a singular possessed kin and a plural possessor. (\ref{ex:Kin-ji-PL-1}) shows this. Like (\ref{ex:Kin-no-ji-2}), it was also produced by María C. in spontaneous speech and expresses exactly the same constellation: a plural possessor with a singular possessed kin (though with a different kinship term). Nonetheless, the collective marker is used together with the plural marker in this case.

\ea\label{ex:Kin-ji-PL-1}
\begingl 
\glpreamble chÿenujinube ekanube\\
\gla chÿ-enu-ji-nube eka-nube\\ 
\glb 3-mother-\textsc{col}-\textsc{pl} \textsc{dem}a-\textsc{pl}\\ 
\glft ‘the mother of them’
\trailingcitation{[cux-c120410ls.124]}
\xe

In an elicitation session with María S. about this topic (rxx-e151021l-1), she explicitly confirmed that the collective marker is used with both third person plural possessors and plural possessed kins. Even more so, omission of the marker leads to ungrammatical forms according to her. This was verified by the phrases she produced in the elicitation with some playmobil toys that represented mothers and daughters in different constellations.

(\ref{ex:Kin-ji-PL-2}) is another example with a third person plural possessor and a singular possessed kin. Both the plural and the collective marker are used again. It is a description by Juana of a photo on which my husband cradles both our daughters in his arms. 

\ea\label{ex:Kin-ji-PL-2}
\begingl 
\glpreamble i eka tanÿma chumu chÿajinube, chakachunube chÿajinube\\
\gla i eka tanÿma chÿ-umu chÿ-a-ji-nube ch-akachu-nube chÿ-a-ji-nube\\ 
\glb and \textsc{dem}a now 3-take 3-father-\textsc{col}-\textsc{pl} 3-lift-\textsc{pl} 3-father-\textsc{col}-\textsc{pl}\\ 
\glft ‘and (on) this one, now their father takes them, he lifts them’
\trailingcitation{[jxx-p141024s-1.26]}
\xe

The following examples have singular possessors and plural possessed kin. The plural marker thus relates to the possessed kin in these cases. (\ref{ex:POSS-ji-PL-1}) has a third person singular possessor, whereas in (\ref{ex:POSS-ji-PL-2}) the possessor is first person singular.

In (\ref{ex:POSS-ji-PL-1}), \textit{chipijijinube} ‘his brothers’ is the possessed noun with the singular possessor and plural possessed. The example comes from Juana telling me about the life of her sister.

\ea\label{ex:POSS-ji-PL-1}
\begingl
\glpreamble tikutijikuji chima, chajechubu chipijijinube tikutijikunubeji kimenukÿ \\
\gla ti-kutijiku-ji chi-ima, chÿ-ajechubu chi-piji-ji-nube ti-kutijiku-nube-ji kimenu-kÿ \\
\glb 3i-flee-\textsc{rprt} 3-husband 3-\textsc{com} 3-sibling-\textsc{col}-\textsc{pl} 3i-flee-\textsc{pl}-\textsc{rprt} woods-\textsc{clf:}bounded\\
\glft  ‘her husband fled, together with his brothers he fled to the woods, it is said’
\endgl
\trailingcitation{[jxx-p120430l-2.086-087]}
\xe

(\ref{ex:POSS-ji-PL-2}) is from María C.

\ea\label{ex:POSS-ji-PL-2}
\begingl 
\glpreamble nichechajinube kakunube uneku\\
\gla ni-checha-ji-nube kaku-nube uneku\\ 
\glb 1\textsc{sg}-son-\textsc{col}-\textsc{pl} exist-\textsc{pl} town\\ 
\glft ‘my children live in town’
\trailingcitation{[uxx-p110825l.075]}
\xe
\is{plural|)}
\is{possessor|)}


Plural kinship terms do not take the collective marker if they include the noun (or suffix) \textit{-pÿi}, which literally means ‘body’, but is rather used to express \isi{endearment} (see \sectref{sec:Compounding}). The nouns \textit{-jinepÿi} ‘daughter’ and \textit{-sinepÿi} ‘grandchild’ are lexicalised\is{lexicalisation} with \textit{-pÿi}, and since it is not detachable, these nouns cannot take the collective marker when pluralised. An example is (\ref{ex:daughter-PL}), where Clara speaks about her plans to spread the use of Paunaka.

\ea\label{ex:daughter-PL}
\begingl 
\glpreamble nisachu nimeisumeikanube nijinepÿinube\\
\gla ni-sachu ni-meisumeika-nube ni-jinepÿi-nube\\ 
\glb 1\textsc{sg}-want 1\textsc{sg}-teach.\textsc{irr}-\textsc{pl} 1\textsc{sg}-daughter-\textsc{pl}\\ 
\glft ‘I want to teach it to my daughters’
\trailingcitation{[cux-c120414ls-2.323]}
\xe

Other kinship terms can also take \textit{-pÿi}, but usually occur without it, when pluralised. Thus, in the singular we mostly find \textit{-chechapÿi} ‘son, child’, but in the plural, it is usually only \textit{-checha}. An example was already given in (\ref{ex:POSS-ji-PL-2}) above. There are a few exceptions though, where \textit{-chechapÿi} is pluralised without the collective marker, as in (\ref{ex:chechapÿi}), which comes from a story told by Miguel about a lazy man. His wife is angry with him, when she finds out that he did not do the work he was supposed to do, so she refuses to give him food.

\ea\label{ex:chechapÿi}
\begingl 
\glpreamble chÿnikujikutu chichechapÿinube\\
\gla chÿ-niku-jiku-tu chi-chechapÿi-nube\\ 
\glb 3-feed-\textsc{lim}1-\textsc{iam} 3-son-\textsc{pl}\\ 
\glft ‘she only gave food to her children’
\trailingcitation{[mox-n110920l.081]}
\xe

Strikingly, there is also one utterance in my corpus, given here as (\ref{ex:jinejinube}), in which detachment of \textit{-pÿi} in the \isi{plural} also applies to \textit{-jinepÿi} ‘daughter’, although \textit{*-jine} does not exist as an independent noun stem in Paunaka, contrary to \textit{-checha} ‘son, child, egg, offspring’. This shows that the underlying process of alternating \textit{-pÿi} and \textit{-ji} is transparent for the speakers. The sentence comes from María C. and is about the supposed incapability of Clara’s daughters to learn Paunaka.

\ea\label{ex:jinejinube}
\begingl 
\glpreamble kaku pijinejinube pero kuina puero chitanube\\
\gla kaku pi-jine-ji-nube pero kuina puero chi-ita-nube\\ 
\glb exist 2\textsc{sg}-daughter-\textsc{col}-\textsc{pl} but \textsc{neg} can 3-master.\textsc{irr}-\textsc{pl}\\ 
\glft ‘you have daughters, but they can’t figure it out’
\trailingcitation{[cux-c120414ls-2.265]}
\xe

Apart from kinship terminology, the collective marker also forms part of the plural-only word \textit{sesejinube} ‘children’.
\is{collective|)}

The following sections are dedicated to other kinds of inflectional morphology, nominal irrealis and deceased. Both of these categories provide information about the existence of an entity at reference or utterance time.

%!TEX root = 3-P_Masterdokument.tex
%!TEX encoding = UTF-8 Unicode

\section{Nominal irrealis}\label{NominalRS}
\is{nominal irrealis|(}

Nominal irrealis is a category that has not been widely described up to now. It is reminiscent of the better-known category of \isi{nominal tense} \citep[]{NordlingerSadler2004} (aka nominal temporal markers cf. \citealt[]{Tonhauser2008}) – a relatively widespread feature in South American languages (\citealt[158--163]{Aikhenvald2012}; \citealt[258]{Campbell2012}).

Nominal irrealis is marked by attaching \textit{-ina} to the noun in question. The very same marker also figures as an irrealis marker in \isi{non-verbal predication}. There is sometimes considerable overlap between both functions (see \sectref{sec:NonVerbalPredication}), but there are also enough cases in which nominal irrealis can well be distinguished from predication. These cases are described in this section.

In nominal irrealis, \textit{-ina} indicates that an entity is non-existent or that it has not come into existence yet. This is in compliance with the function it fulfils in predication (see \sectref{sec:RealityStatus}). Regarding syntax, I have found irrealis-marked objects\is{object} and obliques,\is{oblique} but no irrealis-marked subjects.

Consider (\ref{ex:IRR-OBJ-1}) from Juan C. Irrealis marking is due to the non-existence of a pair of trousers in the possession of the speaker here, because his \textit{patrón} refused to give him one. Before the land reform of 1952, many people in the Chiquitania worked in a debt-bondage relation on the haciendas of big landowners, \textit{patrones}, who were supposed to “pay” their workers in kind (see \sectref{sec:Century18-20}). Depending on the character or mood of the \textit{patrón}, people were often paid badly or not paid at all.

\ea\label{ex:IRR-OBJ-1}
\begingl 
\glpreamble kuina tipunakane nikasuneina\\
\gla kuina ti-punaka-ne ni-kasune-ina\\ 
\glb \textsc{neg} 3i-give.\textsc{irr}-1\textsc{sg} 1\textsc{sg}-trousers-\textsc{irr.nv}\\ 
\glft ‘he didn’t give me my supposed trousers’
\trailingcitation{[mqx-p110826l.458]}
\xe

Note that, if the pair of trousers in question existed, the object would not be marked for irrealis. Imagine, for example, a situation, in which a child wants to put on a pair of trousers, but the mother refuses to give it to her because it has just been washed and is drying, or because the child is supposed to wear that pair of trousers on Sunday service. In this situation, which was described to Miguel in elicitation, the statement of the child would be as in (\ref{ex:noIRR-OBJ-1}) and use of irrealis on the object would be incorrect.

\ea\label{ex:noIRR-OBJ-1}
\begingl 
\glpreamble nÿenu kuina tipunakane nikasune\\
\gla nÿ-enu kuina ti-punaka-ne ni-kasune\\ 
\glb 1\textsc{sg}-mother \textsc{neg} 3i-give.\textsc{irr}-1\textsc{sg} 1\textsc{sg}-trousers\\ 
\glft ‘my mother didn't give me my trousers’
\trailingcitation{[mxx-e160811sd.039]}
\xe

(\ref{ex:IRR-functive}) is from the personal account about Juana’s daughter who once wanted to emigrate to Spain to live with her sister as a nanny. The plan never worked out, this is why irrealis and frustrative is used on the predicate (see \sectref{sec:Frust_avertive_optatiev}). The irrealis on the oblique noun with the function of functive \citep[cf.][]{Creissels2014} complies with this reading: the job as an attendant was never accomplished despite the strong expectation of the people involved.

\ea\label{ex:IRR-functive}
\begingl 
\glpreamble i tiyunaini arsaroremÿnÿina tÿpi chisobrinonemÿnÿ\\
\gla i ti-yuna-ini arsarore-mÿnÿ-ina tÿpi chi-sobrino-ne-mÿnÿ\\ 
\glb and 3i-go.\textsc{irr}-\textsc{frust} attendant-\textsc{dim}-\textsc{irr.nv} \textsc{obl} 3-nephew-\textsc{possd}-\textsc{dim}\\ 
\glft ‘and she would have gone as an attendant of her nephew’
\trailingcitation{[jxx-p120430l-1.188]}
\xe

In (\ref{ex:IRR-transformative}), the field that is talked about by Miguel has not been made at all because the main character of the story is a lazybones who prefers swinging in his hammock and playing the flute to the hard physical work of wresting a field from the woods.

\ea\label{ex:IRR-transformative}
\begingl 
\glpreamble kuinaji tana pario chisaneina\\
\gla kuina-ji ti-ana pario chi-sane-ina\\ 
\glb \textsc{neg}-\textsc{rprt} 3i-make.\textsc{irr} some 3-field-\textsc{irr.nv}\\ 
\glft ‘he didn't do anything for his (supposed) field, it is said’
\trailingcitation{[mox-n110920l.012]}
\xe

The previous examples have shown the use of the nominal irrealis to express non-existence in absolute terms. Moreover, the non-existence of the entity marked with \textit{-ina} was contrary to the expectation of the people involved in all cases. This can be called the “negative use”\is{negation} of nominal irrealis. The other semantic context found to be expressed by nominal irrealis is \isi{future reference}. (\ref{ex:IRR-OBJ-2}) is from a story by Miguel about ants and trees and their relation to humans: Trees are sad, when a boy is born because once he grows up, he fells trees for making his field. The irrealis-marked object \textit{chisaneina} ‘his (future) field’ has not come into existence by the reference time of the clause, which is the birth of the boy.

\ea\label{ex:IRR-OBJ-2}
\begingl 
\glpreamble chejepuine echÿu aitubuchepÿi tijÿkatu, tiyunaji tebitaka chisaneina\\
\gla chejepuine echÿu aitubuchepÿi ti-jÿka-tu ti-yuna-ji ti-ebitaka chi-sane-ina\\ 
\glb because \textsc{dem}b boy 3i-grow.\textsc{irr}-\textsc{iam} 3i-go.\textsc{irr}-\textsc{rprt} 3i-clear.\textsc{irr} 3-field-\textsc{irr.nv}\\ 
\glft ‘because once the boy has grown up, he will go and clear his future field, it is said’
\trailingcitation{[mxx-n120423lsf-X.28]}
\xe

Irrealis nouns also occur in purposive expressions. In (\ref{ex:IRR-NOM-purposive}) the aim of the action is additionally marked by the oblique preposition \textit{tÿpi}. It comes from María S. 

\ea\label{ex:IRR-NOM-purposive}
\begingl 
\glpreamble niyunu niyÿbamukeikupu arusu tÿpi niyitÿina\\
\gla ni-yunu ni-yÿbamukeiku-pu arusu tÿpi ni-yÿti-ina\\ 
\glb 1\textsc{sg}-go 1\textsc{sg}-husk-\textsc{dloc} rice \textsc{obl} 1\textsc{sg}-food-\textsc{irr.nv}\\ 
\glft ‘I went to husk rice (in machine in town) for my (future, not yet made) food’
\trailingcitation{[rxx-e120511l.024-025]}
\xe

Nominal irrealis specifies the non-existence of the entity in question, be it absolute or only at reference time. It is not related to the semantics of RS marking of the predicate and should therefore be independent of RS marking of the predicate. Nonetheless, I have found only two examples of irrealis-marked objects in combination with a realis predicate in my corpus. Both come from Juana, and both are about houses.

In (\ref{ex:IRR-OBJ-REAL}), although the action of building the house is completed as signalled by the realis predicate and the general past setting of the story, the irrealis marker on the object is used to express that one of the main characters (Jesus in this case) does not live in the house by reference time, it is his future house,\is{future reference} the one where he is going to live after marrying his wife.

\ea\label{ex:IRR-OBJ-REAL}
\begingl 
\glpreamble tanau chubiuna\\
\gla ti-anau chÿ-ubiu-ina\\ 
\glb 3i-make 3-house-\textsc{irr.nv}\\ 
\glft ‘he made his house (where he was going to live)’
\trailingcitation{[jxx-n101013s-1.552]}
\xe

In (\ref{ex:nomirr-1}), Juana speaks about ongoing construction of the future house of one of her daughters.

\ea\label{ex:nomirr-1}
\begingl
\glpreamble ja’a puakenechÿ tanaunube chubiunubeina\\
\gla ja’a pu-akene-chÿ ti-anau-nube chÿ-ubiu-nube-ina\\
\glb \textsc{afm} other-non.vis.side-3 3i-make-\textsc{pl} 3-house-\textsc{pl}-\textsc{irr.nv}\\
\glft ‘yes, on the other side (of the street) they are making their future house’
\endgl
\trailingcitation{[jxx-p110923l-2.154]}
\xe


Last, non-verbal irrealis is also frequently found on temporal nouns or adverbs\is{temporal/aspectual} to trigger a future reading,\is{future reference} as in (\ref{ex:sabaruina}), which comes from Juana, who was talking about a visit of her brother at her other brother’s house. The latter was not at home.

\ea\label{ex:sabaruina}
\begingl 
\glpreamble sabaruina kapunuina\\
\gla sabaru-ina kapunu-ina\\ 
\glb saturday-\textsc{irr.nv} come-\textsc{irr.nv}\\ 
\glft ‘he will come on Saturday’
\trailingcitation{[jxx-p120430l-2.411]}
\xe

In complex NPs,\is{noun phrase} nominal irrealis is marked only once, i.e. it is not a feature of agreement between a noun and its modifier. This can be seen in (\ref{ex:punachina}), in which the irrealis marker only occurs on the modifier \textit{punachÿ} ‘other’, but not on the noun \textit{semana} ‘week’. The sentence comes from Miguel who was talking about Swintha here.

\ea\label{ex:punachina}
\begingl 
\glpreamble punachina semana tiyunupunukatu\\
\gla punachÿ-ina semana ti-yunu-punuka-tu\\ 
\glb other-\textsc{irr.nv} week 3i-go-\textsc{reg.irr}-\textsc{iam}\\ 
\glft ‘next week she will leave again’
\trailingcitation{[mxx-d110813s-2.043]}
\xe

The whole construction in (\ref{ex:punachina}) resembles the local Spanish expression \textit{la otra semana} ‘the other week’, which can refer either to the preceding or coming week. In Paunaka, the distinction is made by using either an irrealis-marked NP for the coming week or no irrealis marker for reference to the preceding week.

While nominal irrealis relates to the non-existence of an entity,\is{nominal irrealis|)} the markers described in the next section tell us about ceased existence, more precisely, about the fact that somebody is already deceased.
%!TEX root = 3-P_Masterdokument.tex
%!TEX encoding = UTF-8 Unicode
\section{Deceased marking}\label{sec:Deceased}\is{deceased marking|(}

This section is about different possibilities to express on a noun that a person is deceased. Three different markers can be used in this case, \textit{-ini}, \textit{-kue} and \textit{-bane}, which occur with different types of nouns. Only the last of them is also used with other parts of speech\is{word class} as a \isi{remote past} marker.

\tabref{table:DeceasedMarkers} provides a summary of the three forms used for deceased marking.

\begin{table}
\caption{Markers for ‘deceased’}

\begin{tabularx}{\textwidth}{llQ}
\lsptoprule
Marker & Gloss & Usage \cr
\midrule
\textit{-bane} & \textsc{rem} & general remote (past) marker, used for deceased marking on kinship terminology, then mainly on referential terms \cr
\textit{-ini} & \textsc{dec} & on kinship terminology, mainly on endearment/vocative terms \cr
\textit{-kue} & \textsc{dec.pn} & only found on proper names, possibly of Tupi-Guarani origin \cr
\lspbottomrule
\end{tabularx}

\label{table:DeceasedMarkers}
\end{table}

It is not uncommon among \isi{Arawakan languages} to mark on a human noun that the referent is deceased (\citealp[cf.][130, 276, 313]{Ramirez2001}; %Baniwa-Curripaco y Tariano, Piapoco, Achagua
\citealt[153, 157]{OlzaZubiri2004}; \citealt[115]{Danielsen2007}; \citealt[289]{Brandao2014}; \citealt[35]{Jorda2014}; \citealt[80, 81]{Rose2014a}; \citealt[356]{Mihas2015}), and this is also found in non-related languages in Amazonia\is{Amazonian language} (e.g. in Mosetén, see \citealt[75]{Sakel2004}, and Hup, see \citealt[353]{Epps2008}). 

As for \isi{Arawakan languages}, \citet[382]{Payne1991} states that “Wise (1988a) reconstructs one other classifier: \textit{*mini} meaning ‘dead, past, abandoned’, which in most northern languages retains a suffix similar to \textit{-mi}, and in most southern languages a suffix similar to \textit{-ni}. A fuller form was found in Maipure \textit{-mine} and Baré \textit{-amini}”. The Paunaka marker that relates to this is \textit{-ini}. This deceased marker attaches to kinship terms, sometimes to the referential forms, but mostly to the endearment/vocative forms\is{endearment|(} (see \tabref{table:Vocatives} in \sectref{sec:Inalienables}). (\ref{ex:deceased-1}) and (\ref{ex:deceased-2}) show its use on endearment forms. 

(\ref{ex:deceased-1}) comes from Juana who was talking about what she did with her grandmother in the old days.

\ea\label{ex:deceased-1}
\begingl
\glpreamble micha echÿu yeyeini\\
\gla micha echÿu yeye-ini\\
\glb good \textsc{dem}b granny-\textsc{dec}\\
\glft ‘my late granny was a good person’
\endgl
\trailingcitation{[jxx-p120430l-1.059]}
\xe

(\ref{ex:deceased-2}) is also from Juana. It comes from her account about their grandparents’ journey to Moxos to buy cows.

\ea\label{ex:deceased-2}
\begingl 
\glpreamble beintechÿ baka chiyÿseie chÿchÿini\\
\gla beintechÿ baka chi-yÿseie chÿchÿ-ini\\ 
\glb twenty cow 3-purchase grandpa-\textsc{dec}\\ 
\glft ‘it were twenty cows that my late grandpa bought’
\trailingcitation{[jxx-p151016l-2.081-083]}
\xe
\is{endearment|)} 

%Old Baure had optative-ni (Magio 20), Terena also!
%-bane -> Old Baure Magio 26, cuparcari: 97; -ini for deceased Magio: 50, C. 103

 Paunaka’s deceased marker is \textit{-ini} is identical in form to the frustrative\is{frustrative|(} marker (see \sectref{sec:Frust_avertive_optatiev}). It is unclear to me at this stage of research whether \textit{-ini} should be described as one polysemous marker or as two homophonous markers. For the time being, I opt for an analysis of two homophonous markers with different glosses (\textsc{dec} for ‘deceased’, \textsc{frust} for ‘frustrative’).
According to the analysis of \citet[490--491]{Overall2017}, frustrative is often extended to the expression of discontinuous past, i.e. a past situation that was interrupted counter to the expectation of the speaker. Deceased people, obviously, belong to a discontinuous past.\is{past reference} Although the fact that people die may not be unexpected – at least if they are old –, the death of a beloved person causes pain and sorrow for the bereaved people, and frustrative is also connected to negative emotions (see \sectref{sec:Frust_avertive_optatiev}). However, even in the examples given by  \citet[490--492]{Overall2017}, it is in most cases not the frustrative alone that establishes a discontinuous past reading, but the frustrative together with another specialised marker.\footnote{It would go too far to explain this in detail here, but I have reasons to believe that deceased or remote past markers of two of the Arawakan examples cited by \citet[]{Overall2017} in his section on discontinuous past remained unrecognised by the author.}  %Overall’s examples (31) and (33) (p.491) also contain deceased or remote past markers that were not recognised by the authors. %Consider the words \textit{zatyokoenaene} and \textit{zeyenaene} of the Paresi that end in \textit{ene}, which is a deceased suffix according to Brandao, and the Ashéninka word has -ni, which is a remote past marker.

Comparing the form of the Paunaka deceased marker with other Arawakan languages \is{Arawakan languages|(} reveals that a lot of languages have a cognate form to express the meaning of ‘deceased’, and besides Paunaka, only the \isi{Mojeño languages} and \isi{Terena} have an identical (or homophonous) or similar form for frustrative marking.\footnote{In Trinitario \textit{-ini} is a general (discontinuous?) past marker, which occurs on proper names and human nouns to mark a person as deceased, but also on predicates, where it often conveys a counterfactual meaning together with irrealis \citep[cf.][80,81]{Rose2014a}. Ignaciano, apparently, has two markers \textit{-hini} (or \textit{-'ini}) for counterfactual (or frustrative) and \textit{-(i)ni} which marks (nominal?) past (\citealt[cf.][153, 157]{OlzaZubiri2004}; \citealt[35]{Jorda2014}). Baure’s marker\is{Baure} \textit{-in} ‘dead (family member)’ only marks people as deceased \citep[cf.][115]{Danielsen2007}, although in Old \isi{Baure}, the variety documented by the Jesuits in the 18th century, it was also used as a nominal past marker with non-human nouns (Danielsen 2020, p.c.). \isi{Terena} has two markers \textit{-ni} and \textit{-nini} which express a number of meanings related to frustrative (\citealt[cf.][55, 84]{ButlerEkdahl2014}; \citealt[7]{Butler2007}), but the deceased marker is \textit{-ikene} (de Carvalho 2017, p.c.). %26.9.2017
 The Northern Kampan languages have \textit{-ni} whose “basic function is to specify the ceased existence of a human entity", although in some languages it has extended functions and can even be used on verbs as a remote past marker \citep[793]{Mihas2017}. The frustrative markers have a totally different form. More distantly related Paresi has a particle \textit{ene} for nominal past, which is also used as a deceased marker \citep[289]{Brandao2014}, but another particle for frustrative.}\is{Arawakan languages|)}\is{frustrative|)}


In the examples (\ref{ex:deceased-1}) and (\ref{ex:deceased-2}), it is only the deceased marker on the noun that specifies that the people referred to have passed away. The deceased marker is also found on a noun \textit{kuineini} ‘deceased’, and often this noun precedes the noun referring to the deceased person. The deceased marker (or one of the other markers described below) can be added to the noun as in (\ref{ex:kuineini-ini}) or be left out as in (\ref{ex:kuineini-bare}). In that case, \textit{kuineini} is the only expression of ceased existence of the person in question.\footnote{In local Spanish, there is a convention to prepose the noun \textit{finado/-a} ‘deceased one’ before the noun referring to the deceased referent, which can be a kinship term or personal name.} 

(\ref{ex:kuineini-ini}) is from an account of Miguel about the history of Santa Rita. The founder of Santa Rita is the grandfather of the Supepí siblings and their father was among the twelve families that came to live in the village in the 1950s.

\ea\label{ex:kuineini-ini}
\begingl 
\glpreamble kapunutu kuineini taitaini pero kapununube dose familia\\
\gla kapunu-tu kuineini taita-ini pero kapunu-nube dose familia\\ 
\glb come-\textsc{iam} deceased dad-\textsc{dec} but come-\textsc{pl} twelve family\\ 
\glft ‘my late father had come (here), but twelve families came (altogether)’
\trailingcitation{[mxx-p110825l.056]}
\xe

(\ref{ex:kuineini-bare}) is from a listing by María C. of people she knew who were killed by sorcery.

\ea\label{ex:kuineini-bare}
\begingl 
\glpreamble nechÿu kapunu kuineini kupare Tieko\\
\gla nechÿu kapunu kuineini kupare Tieko\\ 
\glb \textsc{dem}c come deceased fellow Diego\\ 
\glft ‘next came late fellow Diego’
\trailingcitation{[ump-p110815sf.640]}
\xe

Alternatively to \textit{-ini}, the remote marker\is{remote past|(} \textit{-bane} (see \sectref{sec:RemotePast}) can also be used on human nouns to signal that the referent is deceased. It is mainly attached to referential kinship terms. These nouns occur with \textit{-bane} much more often than with \textit{-ini}. The noun \textit{-enu} ‘mother’ is even exclusively combined with \textit{-bane} in my corpus and never with \textit{-ini}. One example of \textit{-bane} on kinship terms is given in (\ref{ex:bane-deceased}), in which Juana tells me about the move of her late parents to \isi{Altavista}.

\ea\label{ex:bane-deceased}
\begingl 
\glpreamble te tiyununubetu tanÿma eka nÿabane nÿenubane te tijechikunubetu chukuyae patrun nauku Turuxhiyae\\
\gla te ti-yunu-nube-tu tanÿma eka nÿ-a-bane nÿ-enu-bane te ti-jechiku-nube-tu chi-chuku-yae patrun nauku Turuxhi-yae\\ 
\glb \textsc{seq} 3i-go-\textsc{pl}-\textsc{iam} now \textsc{dem}a 1\textsc{sg}-father-\textsc{rem} 1\textsc{sg}-mother-\textsc{rem} \textsc{seq} 3i-move-\textsc{pl}-\textsc{iam} 3-side-\textsc{loc} patrón there Altavista-\textsc{loc}\\ 
\glft ‘now my late father and my late mother went (away), they moved close to their \textit{patrón} there in Altavista’
\trailingcitation{[jxx-e150925l-1.248]}
\xe

Another example is (\ref{ex:new23-Trion}) from Juan C. telling about the various relocations in his life. I could not find out where Trion is or was, I suppose it is among the places that were renamed or abandoned.

\ea\label{ex:new23-Trion}
\begingl
\glpreamble kuineini niuchikubane tiyunu naka Trion\\
\gla kuineini ni-uchiku-bane ti-yunu naka Trion\\
\glb deceased 1\textsc{sg}-grandfather-\textsc{rem} 3i-go here Trion\\
\glft ‘my late grandfather went to Trion’
\endgl
\trailingcitation{[mqx-p110826l.440-442]}
\xe

In (\ref{ex:kuineini-ini-2}) María C. uses all three strategies described so far, \textit{-bane} on a referential kinship term, \textit{-ini} on an endearment form and an additional \textit{kuineini} preposed to it. She describes her exact kinship relation to the Supepí siblings here. 

\ea\label{ex:kuineini-ini-2}
\begingl 
\glpreamble ja chÿenujinube ekanube chipijibane kuineini mimini\\
\gla ja chÿ-enu-ji-nube eka-nube chi-piji-bane kuineini mimi-ini\\ 
\glb \textsc{afm} 3-mother-\textsc{col}-\textsc{pl} \textsc{dem}a-\textsc{pl} 3-sibling-\textsc{rem} deceased mum-\textsc{dec}\\ 
\glft ‘yes, their mother was the late sister of my late mother’
\trailingcitation{[cux-c120410ls.124-125]}
\xe

As stated above, \textit{-bane} is also used as a general remote past marker (see \sectref{sec:RemotePast}). It is then mainly associated with predicates (or with the whole proposition). In some cases, use of \textit{-bane} with nouns can possibly be analysed as a nominal past\is{nominal tense|(} marker with the meaning ‘former, ex-, old’. In most cases, however, it is not possible to distinguish a predicative and a referential use of the marker. This is reminiscent of the overlap of predicative and referential use of the non-verbal irrealis marker (see \sectref{NominalRS}), but in the case of the remote marker, ambiguity is enhanced by the fact that it can float in the clause in predicative use: it mostly occurs on the predicate, but not always. 

Consider (\ref{ex:former-house}) and the two translations given. One suggests a predicative use of the marker, the other one a referential use. The example comes from a story by María S. about how the tortoise got its carapace: the tortoise does not want to leave her house in the story in order to pay homage to newborn Jesus. As a consequence, she is punished by having her house fixed on her back.

\ea\label{ex:former-house}
\begingl 
\glpreamble nechikue tepajÿku tanÿma chitapu chubiubane\\
\gla nechikue ti-epajÿku tanÿma chi-tapu chÿ-ubiu-bane\\ 
\glb therefore 3i-stay now 3-scales 3-house-\textsc{rem}\\ 
\glft ‘therefore her carapace stays now, which was her house before’\\or: ‘therefore her carapace stays now, (which is) her former house’
\trailingcitation{[rxx-n121128s.24]}
\xe

Another example, in which \textit{-bane} could be analysed as a nominal past marker is (\ref{ex:oldTuruxhi}). \isi{Altavista} does not exist anymore, at least not as a big estate dedicated to agriculture based on forced labour. Juana told me that her father had found two of the cows her grandparents had been deprived of by \textit{karay} somewhere in the pampa and took them to Altavista, where he lived at that time.

\ea\label{ex:oldTuruxhi}
\begingl
\glpreamble kapunu nÿabane te chumu nauku Turuxhiyaebane\\
\gla kapunu nÿ-a-bane te chÿ-umu nauku Turuxhi-yae-bane\\
\glb come 1\textsc{sg}-father-\textsc{rem} \textsc{seq} 3-take there Altavista-\textsc{loc}-\textsc{rem}\\
\glft ‘my late father came and took them to old Altavista’
\endgl
\trailingcitation{[jxx-e150925l-1.238]}
\xe
\is{remote past|)}
\is{nominal tense|)}

There is yet a third marker that can be used to signal the ceased existence of people. This marker, \textit{-kue}, is almost exclusively attached to proper names in my data, i.e. it does not occur with kinship terms or other human nouns, and proper names do not occur with any of the other two markers previously described (both statements with one exception each). It is thus glossed ‘\textsc{dec.pn}’, a deceased marker for proper names. The marker is probably of Tupi-Guarani origin, since it is very similar phonetically to the nominal past marker found in some of these languages: Guarayu has a past marker \textit{-kwer} (Bischoffberger 2017, p.c.), Guarasu uses \textit{-kwe/-we} as a “disconnected” marker, which marks nominal past among other things \citep[237--238]{RamirezAL2017}, in Bolivian Guaraní, the deceased marker is \textit{-gwe/-kwe} \citep[339]{Gustafson2014}, and Paraguayan Guaraní has a nominal (or referential) past marker \textit{-kue} \citep[34]{Nordhoff2004}. 

An example is given in (\ref{ex:kue-deceased}). It comes from María C. Note that she omits the third person marker here, something she does frequently.

\ea\label{ex:kue-deceased}
\begingl 
\glpreamble tupunubu kuineini Pernatokue\\
\gla tupunubu kuineini Pernato-kue\\ 
\glb arrive deceased Fernando-\textsc{dec.pn}\\ 
\glft ‘late Fernando arrived’
\trailingcitation{[ump-p110815sf.412]}
\xe

I want to conclude this section with (\ref{ex:kue-deceased-2}), which comes from Juana. She was thinking about which tale she could tell us (when we had asked her to tell one). She considers telling one story her brother told her, who was a very good storyteller.\footnote{Remember that \textit{taita} literally means ‘dad’, but is used as a respectful form for all male people, see \sectref{sec:Inalienables}.}

\ea\label{ex:kue-deceased-2}
\begingl
\glpreamble chÿkueteabane taita Tubusiukue\\
\gla chÿ-kuetea-bane taita Tubusiu-kue\\
\glb 3-tell-\textsc{rem} dad Tiburcio-\textsc{dec.pn}\\
\glft ‘late Tiburcio told it in the old days’
\endgl
\trailingcitation{[jmx-n120429ls-x5.038]}
\xe
\is{deceased marking|)}

The following section is about diminutives, both marked on the noun and on other parts of speech.

%!TEX root = 3-P_Masterdokument.tex
%!TEX encoding = UTF-8 Unicode



\section{Diminutive}\label{sec:Diminutives}\is{diminutive|(}\is{derivation|(}

Paunaka has one diminutive marker, \textit{-mÿnÿ}. It can occur on nouns and verbs,\is{verb} as in (\ref{ex:new23-dim}), which was elicited from María S. and shows both of this. The diminutive also occasionally occurs on words belonging to other classes. 

\ea\label{ex:new23-dim}
\begingl
\glpreamble tibebeikubumÿnÿ michimÿnÿ\\
\gla ti-bebeiku-bu-mÿnÿ michi-mÿnÿ\\
\glb 3i-lie-\textsc{mid}-\textsc{dim} cat-\textsc{dim}\\
\glft ‘the little cute cat is lying (on a chair)’
\endgl
\trailingcitation{[rxx-e181024l.066]}
\xe

\largerpage
The form of the diminutive marker, \textit{-mÿnÿ}, reminds me of the diminutive found in Guarayu. Guarayu has \textit{mini} among other diminutives, which can be added to verbs and adjectives, in addition to nouns \citep[13]{Hoeller1932a}. Paunaka’s diminutive marker is also similar to the Trinitario\is{Mojeño Trinitario} one \textit{-samini} and \citet[174--175]{Rose2018} proposed that both forms are cognates. Note however, that Paunaka’s /ɨ/ is a reflex of \textit{*u} of a common ancestor language, and the process of fronting did not take place in the Mojeño languages \citep[cf.][418]{deCarvalhoPAU}. If the forms are cognates, we have to assume an unconditioned shift from \textit{*i} to /ɨ/ in Paunaka. The same holds for the hypothesis that the Paunaka diminutive is related to the Guarayu form. In Guarasu, another Tupi-Guarani language closely related to Guarayu, there is a diminutive marker \textit{-m\'{ɨ}nɨ} \citep[437]{RamirezAL2017}, which is identical to the Paunaka one, but we do not know whether the languages were in contact.

Besides the basic meaning of smallness, the diminutive marker can also express emotional values like affection and compassion\is{endearment} and attenuation, often for reasons of politeness or modesty. This extension from the core meaning has been reported to occur very frequently cross-linguistically \citep[535, 558]{Jurafsky1996}. Even if the diminutive is not attached to a noun, but to an adjective\is{adjective} or a verb,\is{verb} it is associated with a noun (or its referent) in most cases. Sometimes it can also attenuate the verb’s meaning. It is, however, often impossible to distinguish diminutive notions belonging to the referent from purely predicative attenuation, because a small, modest or pitied referent usually causes little action. In many cases, I just do not know what the speaker exactly wanted to express with the diminutive. This is why I decided not to treat the diminutive in different chapters -- unlike other markers of \isi{transcategorial morphology} like the person and number markers, the non-verbal irrealis and the remote past marker, where the different functions are more easily distinguished (at least in some cases).\footnote{I have explained possible cases of overlap in the preceding sections.} To compensate for this, the subsections on diminutive marking are ordered by word class. First, examples of diminutives on nouns are given in \sectref{sec:Diminuitves_Nouns}, while \sectref{sec:Diminuitves_Verbs} discusses use of the marker on verbs. In \sectref{sec:Diminuitves_OtherPOS}, occurrences of \textit{-mÿnÿ} with other parts of speech are presented.

The use of diminutives is not only very common in Paunaka, but also in Bolivian Spanish, where it has largely the same functions \citep[cf.][38]{Mendoza2015}, but not the same distribution, i.e. it cannot be used on verbs.

%Other than Ignaciano \textit{-chicha}, the Paunaka diminutive \textit{-mÿnÿ} does NOT appear in the same slot as the classifier and possessed marker.

\subsection{Diminutives on nouns}\label{sec:Diminuitves_Nouns}
\largerpage
Sometimes, the diminutive marker \textit{-mÿnÿ} clearly expresses its core meaning of smallness. This is often the best interpretation, when it is added to inanimate nouns. An example is (\ref{ex:dim-small}) from Miguel. The woman, a character of a narrative, brings some food to her husband, who is supposed to be working in the woods. She uses a small pot for transportation, not one of those huge ones that are sometimes used for cooking. 

\ea\label{ex:dim-small}
\begingl 
\glpreamble tumuji nÿkÿikimÿnÿji yÿtÿuku\\
\gla ti-umu-ji nÿkÿiki-mÿnÿ-ji yÿtÿuku\\ 
\glb 3i-take-\textsc{rprt} pot-\textsc{dim}-\textsc{rprt} food\\ 
\glft ‘she took the small pot of food, it is said’
\trailingcitation{[mox-n110920l.058-059]}
\xe

Another example, in which smallness is the factor expressed by the diminutive is (\ref{ex:dimi-1}) from Juana, where she describes one of the last pictures of the \isi{frog story} including many little frogs.\footnote{The last \textit{-ji} of \textit{chÿenuji} which is glossed here as collective could also be the reportive marker.}

\ea\label{ex:dimi-1}
\begingl
\glpreamble aa peÿjanemÿnÿ cheikukukÿujanetuji chÿenuji\\
\gla aa peÿ-jane-mÿnÿ chÿ-eiku-kukÿu-jane-tu-ji chÿ-enu-ji\\
\glb \textsc{intj} frog-\textsc{distr}-\textsc{dim} 3-follow-\textsc{am.conc.tr}-\textsc{distr}-\textsc{iam}-\textsc{rprt} 3-mother-\textsc{col} \\
\glft ‘ah, the little frogs are following their mother, it is said’
\endgl
\trailingcitation{[jxx-a120516l-a.440]}
\xe

Whenever a diminutive is added to a noun denoting a child or a small animal, it is hard to say whether the speaker uses it only because the referent is small, or also to convey certain affection for the referent.\is{endearment}  Consider (\ref{ex:dim-small-aff}), which is from the same story as (\ref{ex:dim-small}) above. After the woman has discovered the deception of her husband, who had pretended he was making a field, the man decides to sacrifice himself by cutting off his limbs. He takes his little son with him, so that the latter can carry his limbless father and throw him into a well from where the lazy man rises as a comet.

\ea\label{ex:dim-small-aff}
\begingl 
\glpreamble chumuji chichechapÿimÿnÿ\\
\gla chÿ-umu-ji chi-chechapÿi-mÿnÿ\\ 
\glb 3-take-\textsc{rprt} 3-son-\textsc{dim}\\ 
\glft ‘he took his little son, it is said’
\trailingcitation{[mox-n110920l.089]}
\xe

The diminutive can also be used for a small amount of something. Consider example (\ref{ex:dim-small-amount}), in which María C. describes that she only has little corn left to prepare chicha, her preferred beverage, though it cannot be excluded that the speaker also uses the diminutive to express self-pity about that fact.

\ea\label{ex:dim-small-amount}
\begingl 
\glpreamble kakumÿnÿ amukemÿnÿ te tibukapu echÿu te kuinabu nea aumue\\
\gla kaku-mÿnÿ amuke-mÿnÿ te ti-buka-pu echÿu te kuina-bu nÿ-ea aumue\\ 
\glb exist-\textsc{dim} corn-\textsc{dim} \textsc{seq} 3i-finish.\textsc{irr}-\textsc{mid} \textsc{dem}b \textsc{seq} \textsc{neg}-\textsc{dsc} 1\textsc{sg}-drink.\textsc{irr} chicha\\ 
\glft ‘there is little corn and when it will be finished, then I cannot drink chicha anymore’
\trailingcitation{[ump-p110815sf.693]}
\xe

Finally, there are also cases, in which no smallness is involved and the only possible reading is one of emotional evaluation. This is the case in (\ref{ex:dim-pity-1}), where María C. feels pity for herself.

\ea\label{ex:dim-pity-1}
\begingl 
\glpreamble nÿti juberÿpunÿmÿnÿ\\
\gla nÿti juberÿpu-nÿ-mÿnÿ\\ 
\glb 1\textsc{sg.prn} old.woman-1\textsc{sg}-\textsc{dim}\\ 
\glft ‘poor me, I am an old woman’
\trailingcitation{[uxx-p110825l.038]}
\xe

Affection is not necessarily for the referent of the noun that bears the diminutive marker, but can also be for the \isi{possessor} of that noun. Preceding the cited clause in (\ref{ex:dim-pity-2}), Juana explained that the two old ladies she was talking about have passed away a long time ago. They had been old already when she first met them. Juana’s speech is full of diminutives in reference to the old ladies, and in (\ref{ex:dim-pity-2}), she adds one to a possessed item, the walking cane of one of the ladies.

\ea\label{ex:dim-pity-2}
\begingl 
\glpreamble kaku chibastunemÿnÿtu, mhm, chiyuikiumÿnÿ\\
\gla kaku chi-bastun-ne-mÿnÿ-tu mhm chi-yuik-i-u-mÿnÿ\\ 
\glb exist 3-cane-\textsc{possd}-\textsc{dim}-\textsc{iam} \textsc{intj} 3-walk-\textsc{subord}-\textsc{real}-\textsc{dim}\\ 
\glft ‘she already had a cane, mhm, for walking’
\trailingcitation{[jxx-p120515l-1.220-221]}
\xe

%other example: nisapatunemÿne, mqx-p110826l.514: Juan C. expresses his modesty in the wish for shoes that his patrón is supposed to give to him, it does not mean that the shoes are small or particularly nice -> not a good example though, because it is not a complete sentence



\subsection{Diminutives on verbs}\label{sec:Diminuitves_Verbs}
\is{verb|(}

The diminutive on verbs fulfils largely the same functions as on nouns and can also attenuate the meaning of the verb. As had been mentioned above, it is often hard to decide what exactly the speaker had in mind, when she used a diminutive.

(\ref{ex:dimi-2}) and (\ref{ex:dim-V-1}) are two sentences elicited from María S. and referring to a small chick of hers, which was given water by her grandchild. The diminutive expresses that the chick is small, that it is cute or that she feels empathy for it, or all of this together. In (\ref{ex:dimi-2}) the diminutive refers to the \isi{object} of the clause and in (\ref{ex:dim-V-1}) to the \isi{subject}.

\ea\label{ex:dimi-2}
\begingl
\glpreamble tekichamÿnÿ ÿne\\
\gla ti-ekicha-mÿnÿ ÿne\\
\glb 3i-invite.\textsc{irr}-\textsc{dim} water\\
\glft ‘she gives it water’
\endgl
\trailingcitation{[rmx-e150922l.051]}
\xe

\ea\label{ex:dim-V-1}
\begingl 
\glpreamble tibiyukumÿnÿ takÿra\\
\gla ti-biyuku-mÿnÿ takÿra\\ 
\glb 3i-be.thirsty-\textsc{dim} chicken\\ 
\glft ‘the chick is thirsty’
\trailingcitation{[rmx-e150922l.054]}
\xe

The attenuation of a verb’s meaning is prevalent in (\ref{ex:attenuation-dim}), where Juana tells me that on her grandparents’ journey back home from Moxos  the sun started to shine a bit again after heavy rainfalls.

\ea\label{ex:attenuation-dim}
\begingl 
\glpreamble tukiu nechÿu chikebiuji, las sinkotuji tijayekamÿnÿji sache\\
\gla tukiu nechÿu chi-keb-i-u-ji {las sinko}-tu-ji ti-jayeka-mÿnÿ-ji sache\\ 
\glb from \textsc{dem}c 3-rain-\textsc{subord}-\textsc{real}-\textsc{rprt} {at five o’clock}-\textsc{iam}-\textsc{rprt} 3i-shine.\textsc{irr}-\textsc{dim}-\textsc{rprt} sun\\ 
\glft ‘from then on it was raining, it is said, until at five the sun started to shine a bit’
\trailingcitation{[jxx-p151016l-2.122]}
\xe

Sometimes, a diminutive occurs on a verb to make an \isi{imperative} more polite, as is the case in (\ref{ex:imperative-dim}), where María C. tells Clara what we had said to her when she visited us in the hotel we stayed at.\footnote{Note that this recording has not been archived because it contains sensitive data.}

\ea\label{ex:imperative-dim}
\begingl
\glpreamble pibenamÿnÿ naka yumaji\\
\gla pi-bena-mÿnÿ naka yumaji\\
\glb 2\textsc{sg}-lie.down.\textsc{irr}-\textsc{dim} here hammock\\
\glft ‘lie down here in the hammock’
\endgl
\trailingcitation{[cux-c120510l-1.141]}
\xe

Attenuation can also be due to modesty, as in (\ref{ex:dim-modesty}), where Juana does not want to boast about her knowledge of Paunaka. It is her imagined or remembered answer in a remembered dialogue with the two old ladies also mentioned in (\ref{ex:dim-pity-2}) after they found out that she was a speaker of Paunaka.

\ea\label{ex:dim-modesty}
\begingl 
\glpreamble nichujikumÿnÿ, yeyeini kuina tichujikane kasteyano\\
\gla ni-chujiku-mÿnÿ yeye-ini kuina ti-chujika-ne kasteyano\\ 
\glb 1\textsc{sg}-speak-\textsc{dim} granny-\textsc{dec} \textsc{neg} 3i-speak-1\textsc{sg} Spanish\\ 
\glft ‘I speak it a little, my late grandmother didn’t speak Spanish with me’
\trailingcitation{[jxx-p120515l-1.166]}
\xe



%courtesy:
%\ea\label{ex:}
%\begingl 
%\glpreamble ¡peatumÿnÿ!\\
%\gla \\ 
%\glb \\ 
%\glft ‘drink! you can drink, now!’\\ 
%\endgl
%\trailingcitation{[jxx-p150920l.001]}
%\xe  --> mÿnÿ hier nach -tu, oder ist das bei Verben anders?

\is{verb|)}

\subsection{Diminutives on other parts of speech}\label{sec:Diminuitves_OtherPOS}

Diminutives can also occasionally be added to other parts of speech. They can attach to the few adjectives that exist in Paunaka, and infrequently also to pronouns and demonstratives (usually the nominal demonstratives, but one time in the corpus also to the demonstrative adverb \textit{naka} ‘here’).

Because of the emotional value of the diminutive, \textit{-mÿnÿ} can also be added to the adjective\is{adjective|(} \textit{(mu)temena} ‘big’. This is the case in (\ref{ex:dim-ADJ}). It is not clear, though, whether Juana uses the diminutive to express her pity for some, already grown, ducks that died in her absence, because none of her family members fed them, or to attenuate the meaning of the predicate as ‘big, but small’ = ‘biggish’.\footnote{Thanks to Swintha Danielsen for pointing out this second possibility.}


\ea\label{ex:dim-ADJ}
\begingl 
\glpreamble pero temenanajimÿnÿtu\\
\gla pero temena-na-ji-mÿnÿ-tu\\ 
\glb but big-\textsc{rep}-\textsc{col}-\textsc{dim}-\textsc{iam}\\ 
\glft ‘but they were already big, the poor ones’\\or: ‘but they were already biggish’
\trailingcitation{[jrx-c151001lsf-11.071]}
\xe
\is{adjective|)}

(\ref{ex:PersPron-3}) is a statement by María C. about herself, in which the diminutive is added to the first person singular pronoun,\is{personal pronoun} because she pities herself.

\ea\label{ex:PersPron-3}
\begingl
\glpreamble nÿtimÿnÿ baichane, kuina nÿana kuina nenuina\\
\gla nÿti-mÿnÿ baicha-ne kuina nÿ-a-ina kuina nÿ-enu-ina\\
\glb 1\textsc{sg.prn}-\textsc{dim} orphan-1\textsc{sg} \textsc{neg} 1\textsc{sg}-father-\textsc{irr.nv} \textsc{neg} 1\textsc{sg}-mother-\textsc{irr.nv}\\
\glft ‘poor me, I am an orphan, I don’t have a father, I don’t have a mother’
\endgl
\trailingcitation{[uxx-p110825l.071]}
\xe

In (\ref{ex:dimi-4}), a \isi{numeral} carries the diminutive for attenuation. The sentence comes from María S.

\ea\label{ex:dimi-4}
\begingl
\glpreamble chÿnamÿnÿchÿ nipeu ÿba\\
\gla chÿna-mÿnÿ-chÿ ni-peu ÿba\\
\glb one-\textsc{dim}-3 1\textsc{sg}-animal pig\\
\glft ‘I have a single pig’
\endgl
\trailingcitation{[rxx-e181024l.059]}
\xe

\is{derivation|)}


While the diminutive can thus attach to various parts of speech,\is{diminutive|)} the locative marker, which is described in the following section, exclusively occurs with nouns.


%!TEX root = 3-P_Masterdokument.tex
%!TEX encoding = UTF-8 Unicode

\section{Locative marking}\label{sec:Locative}
\is{oblique|(}
\is{locative marker|(}

There is one general locative marker, \textit{-yae}, which possibly originated as a relational noun\is{grammaticalisation} (cf. \citealt[]{Rose2019a} and see also \sectref{sec:Non-possessables}).\footnote{According to the analysis by \citet[]{Rose2019a}, this relational noun developed into a universal preposition in Trinitario.\is{Mojeño Trinitario} A related form \textit{-ye} also occurs in \isi{Baure} \citep[cf.][150]{Danielsen2007}. Paunaka shares with \isi{Baure} that the root is used as a locative marker on nouns, and it shares with Trinitario that it is used as a relational noun in possessive constructions.\is{possessive clause}} The marker attaches exclusively to nouns that express spatial relations in a clause, more precisely relations of place, goal, and source. It is not found on adverbs. In slow speech, the marker is pronounced \textit{-yae}, but it can be reduced to \textit{-ye} or \textit{-ya} in rapid speech. 

(\ref{ex:loc-place}) is an expression of a place, (\ref{ex:loc-goal}) presents the locative marker on a goal, and (\ref{ex:loc-source}) on a source expression. In (\ref{ex:loc-place}), Juana speaks about her daughter who had badly fallen down, was treated in hospital and still had to stay in bed afterwards.

\ea\label{ex:loc-place}
\begingl
\glpreamble \textup{place:}\\pero tibenunukubu chikamaneyae\\
\gla pero ti-benunuku-bu chi-kama-ne-yae\\
\glb but 3i-lie-\textsc{mid} 3-bed-\textsc{possd}-\textsc{loc}\\
\glft ‘but she kept lying in her bed’
\endgl
\trailingcitation{[jxx-p110923l-1.485]}
\xe

(\ref{ex:loc-goal}) was a conjecture of María S. about what her brother was going to do.

\ea\label{ex:loc-goal}
\begingl
\glpreamble \textup{goal:}\\tiyunakena chisaneyae\\
\gla ti-yuna-kena chi-sane-yae\\ 
\glb 3i-go.\textsc{irr}-\textsc{uncert} 3-field-\textsc{loc}\\ 
\glft ‘maybe he wants to go to his field’
\trailingcitation{[rxx-e120511l.348]}
\xe

(\ref{ex:loc-source}) was produced by Juana in telling me about the death of some of her siblings. She went to the funeral of her brother by public transportation, which is carried out by vans or small buses in Bolivia, called \textit{micros} in Spanish.

\ea\label{ex:loc-source}
\begingl
\glpreamble \textup{source:}\\nikupu tukiu mikroyae\\
\gla ni-kupu tukiu mikro-yae\\
\glb 1\textsc{sg}-go.down from microbus-\textsc{loc}\\
\glft ‘I got off the microbus’
\endgl
\trailingcitation{[jxx-p120430l-2.465]}
\xe

The locative marker may be dropped under certain conditions. Most importantly, if the spatial relation includes a \isi{toponym}, \textit{-yae} is often absent, see (\ref{ex:new23-loc1}); this is also true for the noun \textit{uneku} ‘town’ which is toponym-like\is{toponym}, since it usually refers to Concepción, as in (\ref{ex:new23-loc2}).

(\ref{ex:new23-loc1}) comes from Juana, who was telling me how hard it was to obtain water before the reservoir was made in Santa Rita.

\ea\label{ex:new23-loc1}
\begingl
\glpreamble bupunu ÿne Santa Rita\\
\gla bi-upunu ÿne {Santa Rita}\\
\glb 1\textsc{pl}-bring water {Santa Rita}\\
\glft ‘we brought water to Santa Rita’
\endgl
\trailingcitation{[jxx-p120515l-2.054]}
\xe

(\ref{ex:new23-loc2}) is a comment from María S. about her siblings who moved away from Santa Rita.

\ea\label{ex:new23-loc2}
\begingl
\glpreamble depue tepajÿkunubetu uneku\\
\gla depue ti-epajÿku-nube-tu uneku\\
\glb afterwards 3i-stay-\textsc{pl}-\textsc{iam} town\\
\glft ‘then they stayed in town’
\endgl
\trailingcitation{[rxx-p181101l-2.264]}
\xe

If the verb in a goal expression is \textit{-yunu} ‘go’, it is also not uncommon that the locative marker is missing on the noun denoting the goal, as in (\ref{ex:new23-loc3}) from Clara. She was explaining us that her daughter was baking bread alone at that moment, thus this sentence is about her other daughter’s physical presence in school, not the general enrollment in school.

\ea\label{ex:new23-loc3}
\begingl
\glpreamble punachÿ tiyunu xhikuera\\
\gla punachÿ ti-yunu xhikuera\\
\glb other 3i-go school\\
\glft ‘the other (sister) went to school’
\endgl
\trailingcitation{[cux-120410ls.220]}
\xe

In source expressions, the preposition \textit{tukiu}\is{source|(} is needed, and \textit{-yae} can be considered optional in this case, thus in (\ref{ex:new23-loc4}) from Juana, no locative marker is necessary on the noun. She spoke about her daughter (and other people) who had come back from Spain.

\ea\label{ex:new23-loc4}
\begingl
\glpreamble tikubupaikunubetu tukiu labion\\
\gla ti-kubupaiku-nube-tu tukiu labion\\
\glb 3i-go.down-\textsc{pl}-\textsc{iam} from plane\\
\glft ‘they disembarked from the plane’
\endgl
\trailingcitation{[jxx-p120430l-1.266]}
\xe
\is{source|)}

%nubupuna tukiu asaneti nubupuna ubiaeyae, rxx-e181020le

All of these examples would equally work well if the locative marker was attached to the noun. Furthermore, locative marking can also target human nouns, as is the case in (\ref{ex:loc-human}), which comes from Miguel who was talking about Swintha.

\ea\label{ex:loc-human}
\begingl
\glpreamble paseaubÿti nauku baurenyonubeyae\\
\gla paseau-bÿti nauku baurenyo-nube-yae\\
\glb stroll-\textsc{prsp} there Baure.person-\textsc{pl}-\textsc{loc}\\
\glft ‘she is going to travel to the Baure people’
\endgl
\trailingcitation{[mxx-d110813s-2.066]}
\xe

The marker \textit{-yae} can also figure as an instrumental marker in cases, in which the preposition \textit{en} ‘in’ would be used in Spanish, e.g. for motion by a vehicle. This is the case in (\ref{ex:loc-Span-1}), where Swintha and I were discussing our little excursion to \isi{Altavista} with María C. and Clara. Altavista is far away from Santa Rita if one has to walk, but nicely reachable by bike as Clara recognises here.

\ea\label{ex:loc-Span-1}
\begingl
\glpreamble pero un ratoyÿchi eyuna bisikletayae\\
\gla pero {un rato}-yÿchi e-yuna bisikleta-yae\\
\glb but {a while}-\textsc{lim}2 2\textsc{pl}-go.\textsc{irr} bicyle-\textsc{loc}\\
\glft ‘but it only takes you a little while if you go by bike’
\endgl
\trailingcitation{[cux-c120414ls-1.155]}
\xe

Other cases of semantic extension of locative marking that resemble the ones in Spanish are exemplified by (\ref{ex:loc-Span-2}) and (\ref{ex:loc-Span-3}). There is no extension to temporal expressions though.

In (\ref{ex:loc-Span-2}), María C. construes the inside of her head as the place containing knowledge.

\ea\label{ex:loc-Span-2}
\begingl
\glpreamble kakutu pario nÿchÿtiyaemÿnÿ pario\\
\gla kaku-tu pario nÿ-chÿti-yae-mÿnÿ pario\\
\glb exist-\textsc{iam} some 1\textsc{sg}-head-\textsc{loc}-\textsc{dim} some\\
\glft ‘I have a lot (knowledge) in my head, a lot’
\endgl
\trailingcitation{[uxx-p110825l.095]}
\xe

(\ref{ex:loc-Span-3}) comes from Juana and is about words in different languages.

\ea\label{ex:loc-Span-3}
\begingl
\glpreamble jaja, kastelyanoyae ciervo, pero naka neteayae kaku chijaini\\
\gla jaja kastelyano-yae ciervo pero naka nÿ-etea-yae kaku chi-ija-ini\\
\glb \textsc{afm} Spanish-\textsc{loc} ciervo but here 1\textsc{sg}-language-\textsc{loc} exist 3-name-\textsc{frust}\\
\glft ‘yes, it is \textit{ciervo} (=deer) in Spanish, but it has a name in my language (that I don’t remember)’
\endgl
\trailingcitation{[jxx-a120516l-a.231-233]}
\xe

Despite of these possible extensions, the locative marker is mainly applied to spatial relations of different kinds. In (\ref{ex:loc-on}), there is contact from above (‘on’), in (\ref{ex:loc-at}) the spatial relation is one of closeness and can be translated with ‘at’, and in (\ref{ex:loc-in}), the figure (i.e. the subject of the clause) is inside a location (‘in’).

(\ref{ex:loc-on}) was elicited from María S.

\ea\label{ex:loc-on}
\begingl
\glpreamble tibebeikubu siyayae\\
\gla ti-bebeiku-bu siya-yae\\
\glb 3i-lie-\textsc{mid} chair-\textsc{loc}\\
\glft ‘it (the cat) is lying on the chair’
\endgl
\trailingcitation{[rxx-e181024l]}%el.
\xe

(\ref{ex:loc-at}) comes from the same session. It referred to a pig which I had asked for, since it was suddenly not in the yard anymore.

\ea\label{ex:loc-at}
\begingl
\glpreamble tiyunu tisemaiku yÿtie atajauyae\\
\gla ti-yunu ti-semaiku yÿtie atajau-yae\\
\glb 3i-go 3i-search food water.reservoir-\textsc{loc}\\
\glft ‘it (the pig) went to look for food at the reservoir’
\endgl
\trailingcitation{[rxx-e181024l]}%non-el.
\xe


(\ref{ex:loc-in}) was elicited from Miguel.

\ea\label{ex:loc-in}
\begingl
\glpreamble kaku kÿjÿpi ubiaeye\\
\gla kaku kÿjÿpi ubiae-yae\\
\glb exist manioc house-\textsc{loc}\\
\glft ‘there is manioc in the house’
\endgl
\trailingcitation{[mxx-e160811sd.073]}%el.
\xe

The locative marker alone thus expresses the most expected spatial relations and its interpretation as ‘on’, ‘at’ or ‘in’ largely depends on the spatial dimensions of the noun denoting the ground (i.e. the location) and the habits or properties of the figure \citep[cf.][69]{Admiraal2016}. In order to be more specific or for the expression of unusual relations, speakers can make use of two different strategies: either a more precise locative noun is derived from the noun denoting the ground or a complex NP\is{noun phrase} is used which contains a possessed \isi{relational noun}\is{possession} and a possessor denoting the ground.

For the expression of complete containment, a “container” noun is derived\is{derivation|(} by adding the “bounded” classifier\is{classifier|(} \textit{-kÿ} or attaching the locative stem \textit{-j(ÿ)ekÿ} ‘inside’, which can most probably be classified as a \isi{nominal stem}. There are differences between the resulting nouns.

Not every noun can take the classifier \textit{-kÿ}, the majority are containers anyway. The difference is that without the classifier, they are perceived as manipulable objects, with \textit{-kÿ} they denote locations. The locative marker is usually added to the derived noun. An example is given in (\ref{ex:clf-loc-1}), which contains \textit{tachukÿyae} ‘inside the small pot’. It comes from Miguel describing the pictures of the \isi{frog story}.

\ea\label{ex:clf-loc-1}
\begingl
\glpreamble i naka chipurutukutu eka kabe chichÿti naka eka tachukÿyae\\
\gla i naka chi-purutuku-tu eka kabe chi-chÿti naka eka tachu-kÿ-yae\\
\glb and here 3-put.in-\textsc{iam} \textsc{dem}a dog 3-head here \textsc{dem}a small.pot-\textsc{clf:}bounded-\textsc{loc}\\
\glft ‘and here the dog has stuck its head into the small pot here’
\endgl
\trailingcitation{[mox-a110920l-2.052]}
\xe

Another example, which also comes from Miguel re-telling the \isi{frog story} (but on another occasion), is given in (\ref{ex:clf-loc-2}), and this time the classifier \textit{-kÿ} adds the important information that the action is performed in relation to the inside of the boot.

\ea\label{ex:clf-loc-2}
\begingl
\glpreamble chimumukuji chijachÿukena kakukena nauku botakÿyae\\
\gla ch-imumuku-ji chija-chÿu-kena kaku-kena nauku bota-kÿ-yae\\
\glb 3-look-\textsc{rprt} what-\textsc{dem}b-\textsc{uncert} exist-\textsc{uncert} there boot-\textsc{clf:}bounded-\textsc{loc}\\
\glft ‘he is looking what may be there in his boot, it is said’
\endgl
\trailingcitation{[mtx-a110906l.043-046]}
\xe

The noun \textit{kimenu} ‘woods’ is also frequently found with the classifier, when it conveys the idea of a place that somebody goes to or acts in. Unlike the other nouns with \textit{-kÿ}, it is often used without the locative marker. In (\ref{ex:loci-1}), however, \textit{-yae} is attached to the noun. The example comes from Miguel who told me and José the story about a lazy man.

\ea\label{ex:loci-1}
\begingl
\glpreamble titupunubuji kimenukÿyae tisemaikuji echÿu kujubipi\\
\gla ti-tupunubu-ji kimenu-kÿ-yae ti-semaiku-ji echÿu kujubipi\\
\glb 3i-arrive-\textsc{rprt} woods-\textsc{clf:}bounded-\textsc{loc} 3i-find-\textsc{rprt} \textsc{dem}b liana.sp\\
\glft ‘when he arrived in the woods, he found a liana, it is said’
\endgl
\trailingcitation{[mox-n110920l.025]}
\xe
\is{classifier|)}
\is{derivation|)}

More emphasis is attained through attachment of the locative stem \textit{-j(ÿ)ekÿ} ‘inside’, which is related to the free noun \textit{nujekÿ} ‘inside’.\footnote{\textit{Nujekÿ} ‘inside’ and its antonym \textit{nekupai} ‘outside, yard’ are never set in relation to another noun.} Its relation to \textit{-kÿ} is similar to the relation of ‘in(to)’ to ‘inside (of)’. Nominal compounds\is{compounding} with \textit{-j(ÿ)ekÿ} can be followed by the locative marker, as in (\ref{ex:inside-loc}), but this is not always the case, see (\ref{ex:inside-no-loc}). Both examples were elicited, (\ref{ex:inside-loc}) from Miguel and (\ref{ex:inside-no-loc}) from María S.

\ea\label{ex:inside-loc}
\begingl
\glpreamble nipurtuka jurnujÿekÿyae\\
\gla ni-purtuka jurnu-jÿekÿ-yae\\
\glb 1\textsc{sg}-put.in.\textsc{irr} oven-inside-\textsc{loc}\\
\glft ‘I will put it inside the oven’
\endgl
\trailingcitation{[mxx-e120415ls.105]}%el.
\xe

\ea\label{ex:inside-no-loc}
\begingl
\glpreamble tibÿkupu kabe kosinajÿekÿ\\
\gla ti-bÿkupu kabe kosina-jÿekÿ\\
\glb 3i-enter dog kitchen-inside\\
\glft ‘the dog goes into the kitchen’
\endgl
\trailingcitation{[rxx-e181021les.105]}%el.
\xe

The noun \textit{ÿne} ‘water’ does not combine with \textit{-j(ÿ)ekÿ}, maybe because it is not perceived as a container. There is a special expression for ‘inside of the water’, which is \textit{ÿneumu(kÿ)} (see \sectref{sec:Nouns_CLF}), while ‘above/on the water’ is \textit{ÿnemiuke}. These are unique non-productive derivations.\footnote{I could elicit \textit{mutejimiuke} ‘on/above the mud’, but no other word I tried to form accordingly was accepted by the speakers. There is another singular locative expression, \textit{mainekÿke} ‘on/above the stone’, which contains the relational noun \textit{-(i)ne} ‘top’ and seemingly also \textit{-kÿ}. Both words contain a suffix \textit{-ke}, which may be the one we find on nouns denoting places (see \sectref{sec:Classifiers}).}

For any relation other than “inside”,\is{possession|(} the other strategy mentioned above is used:\is{relational noun|(} an inalienably\is{inalienability} possessed locative noun stem\is{nominal stem} expressing the specific relation is juxtaposed\is{juxtaposition} to the noun denoting the ground which acts as a possessor. The locative marker is attached to the relational noun in this case. However, the nouns \textit{-upekÿ} ‘place under’ and \mbox{\textit{-akene/-ekene}} ‘non-visible side’ can also be used without the locative marker. For the latter one, this is even more common. The locative relational nouns are listed in \tabref{table:noun-stems-locative}.

\begin{table}[htbp]
\caption{Locative relational noun stems}

\begin{tabular}{ll}
\lsptoprule
Relational noun & Translation \cr
\midrule
\textit{-akene/-ekene} & non-visible side (behind, beside)\cr
\textit{-chuku} & side (next to, close to)\cr
\textit{-(i)ne} & top, place on top or above\cr
\textit{-upekÿ} & place under\cr
\lspbottomrule
\end{tabular}

\label{table:noun-stems-locative}
\end{table}

(\ref{ex:loci-2}) shows the use of the relational noun \textit{-upekÿ} ‘place under’. It was elicited from Miguel and refers to a pen I put under a bag.

\ea\label{ex:loci-2}
\begingl 
\glpreamble kaku chiupekÿye echÿu pusane\\
\gla kaku chi-upekÿ-yae echÿu pusane\\ 
\glb exist 3-place.under-\textsc{loc} \textsc{dem}b bag\\ 
\glft ‘it is under the bag’
\trailingcitation{[mxx-e120505l-1]}
\xe

{\ref{ex:loci-3}) includes \textit{-chuku} ‘side’. It comes from an elicitation session with several playmobil toys and was produced by Juana.

\ea\label{ex:loci-3}
\begingl
\glpreamble kaku chichukuyae echÿu jente\\
\gla kaku chi-chuku-yae echÿu jente\\ 
\glb exist 3-side-\textsc{loc} \textsc{dem}b man\\ 
\glft ‘she is (standing) next to the man’
\trailingcitation{[jrx-c151024lsf]}
\xe

The noun stem \textit{-(i)ne} ‘top’ is the only one of the relational nouns that can also be incorporated\is{incorporation} into active verb stems, see \sectref{sec:INC_ActiveVerbs}.
(\ref{ex:loci-4}) is one example of its use in a locative NP. It comes from Miguel who was describing the production of rice bread.

\ea\label{ex:loci-4}
\begingl
\glpreamble chetuku echÿu kesu tiyÿbapakubu chÿineyae echÿu masa\\
\gla chÿ-etuku echÿu kesu ti-yÿbapaku-bu chÿ-ine-yae echÿu masa\\
\glb 3-put \textsc{dem}b cheese 3i-grind-\textsc{mid} 3-top-\textsc{loc} \textsc{dem}b dough\\
\glft ‘she puts the grated cheese on top of the dough’
\endgl
\trailingcitation{[mxx-e120415ls.087]}
\xe

As has already been mentioned above, the noun \textit{-akene} or \textit{-ekene} ‘non-visible side’ is normally used without the locative marker. It can combine with a person marker or with the root \textit{(u)pu-} ‘other’, and is used if the referent in question is out of sight because something blocks the view. Two examples, both from Juana, are given here. 

(\ref{ex:loci-5}) comes from the creation story she told, a mixture of the tales of the bible and other elements. It is Jesus who hides away behind the door here. (\ref{ex:loci-6}) was elicited. It shows the use of the locative noun together with \textit{(u)pu-} ‘other’.

\ea\label{ex:loci-5}
\begingl
\glpreamble nauku chekene nuinekÿ chububuikubutu\\
\gla nauku chÿ-ekene nuinekÿ chÿ-ubu-buiku-bu-tu\\
\glb there 3-non.vis.side door 3-be.at-\textsc{cont}-\textsc{mid}-\textsc{iam}\\
\glft ‘there behind the door he was (hidden)’
\endgl
\trailingcitation{[jxx-n101013s-1.435]}
\xe

\ea\label{ex:loci-6}
\begingl
\glpreamble kaku upuakene \\
\gla kaku upu-akene \\
\glb exist other-non.vis.side\\
\glft ‘it is behind (the house)’
\endgl
\trailingcitation{[jxx-e191021e-2]}%el.
\xe

\is{relational noun|)}

Unlike in Baure \citep[cf.][81--82]{Admiraal2016}, body-part terminology is very restricted in the expression of spatial relations in Paunaka. Body-part nouns can be used only if the ground is animate\is{animacy} and a real possessor of the body part, as is the case in (\ref{ex:face-front}), elicited from Juana, in which I was pretending to look for my cell phone, which was lying in front of me. There is no semantic extension of body part terminology to inanimate referents.

\ea\label{ex:face-front}
\begingl
\glpreamble ¡kaku naka pibÿkeyae!\\
\gla kaku naka pi-bÿke-yae\\
\glb exist here 2\textsc{sg}-face-\textsc{loc}\\
\glft ‘it is here in front of you!’ (lit.: in your face)
\endgl
\trailingcitation{[jxx-p181104l-2]}%el.
\xe
\is{possession|)}

Last, there are also some locative nouns that are not possessed, \textit{anÿke} ‘up, above’, \textit{apuke} ‘ground, down’, and \textit{pÿkÿjÿe} ‘middle’, as well as the aforementioned \textit{nujekÿ} ‘inside’ and \textit{nekupai} ‘outside, yard’. They could actually also be adverbs,\is{adverb} \isi{word class} is not totally clear in this case (see discussion in \sectref{sec:LocativeAdverbs}). Most of the times, they do not occur in juxtaposition with another noun denoting the ground, i.e. they denote the ground themselves, and they usually do not take the locative marker with a few exceptions.

\is{locative marker|)}
\is{oblique|)}
\is{inflection|)}
To conclude this chapter, the next section provides some information about content of and word order inside the NP.

%not found on anÿke = up, above, but on apuke if this means ground and not down?



%!TEX root = 3-P_Masterdokument.tex
%!TEX encoding = UTF-8 Unicode

\section{The NP}\label{sec:NP}
\is{noun phrase|(}
\is{modification|(}

The typical NP minimally consists of a noun or a \isi{pronoun}.\is{head} Nouns can optionally be modified, pronouns are never modified. NPs can also do without a noun or pronoun. In this case, only the “modifier” is present \citep[cf.][]{Dryer2004}. In this section, only those NPs consisting of a noun and a modifier are considered which form a single syntactic unit and act as an argument in the clause \citep[cf.][13]{Krasnoukhova2012}. There are no discontinuous NPs in Paunaka.

Modifiers are nominal demonstratives,\is{nominal demonstrative} adjectives,\is{adjective} numerals,\is{numeral} relative clauses,\is{relative relation} and other nouns. The status of quantifers\is{quantifier} as modifiers of nouns is not totally clear. Relative clauses are discussed in detail in §\ref{sec:RelativeClauses} and will thus not be considered here. \figref{fig:NP} shows the \isi{word order} in the NP.


\begin{figure}
\begin{tabularx}{\textwidth}{rQCCCCCCCCQl}
{${\Biggl [}$} & {(Q)} & {DEM} & {‘other’} & {NUM} & {ADJ\textsuperscript{1}} & {N} & [{DEM}  & {N\textsubscript{poss}}] & {N\textsubscript{type}} & {\multirow{2}{*}{\shortstack[c]{RC\\ (ADJ\textsuperscript{2})}}} & {${\Biggl ]}$}\\
 & & &  & & & & & & &  &\\
& & & & & & \textit{head} & & & & & \\
\end{tabularx}
\captionof{figure}{Word order in the NP}
\label{fig:NP}
\end{figure}



The following examples show those types of modifiers which precede the noun. (\ref{ex:NP-dem}) has a nominal demonstrative and a noun. It comes from Miguel’s narration of the story about the fox and the jaguar.

\ea\label{ex:NP-dem}
\begingl
\glpreamble \textup{demonstrative + noun:}\\tiyunutu echÿu kupisaÿrÿ\\
\gla ti-yunu-tu echÿu kupisaÿrÿ\\
\glb 3i-go-\textsc{iam} \textsc{dem}b fox\\
\glft ‘the fox had already gone’
\endgl
\trailingcitation{[jmx-n120429ls-x5.170]}
\xe

In (\ref{ex:NP-other-1}), the word for ‘other’ acts as a nominal modifier. Juana speaks about the plans of her landlord.

\ea\label{ex:NP-other-1}
\begingl
\glpreamble \textup{‘other’ + noun:}\\tana punachÿ kuartojane naka\\
\gla ti-ana punachÿ kuarto-jane naka\\
\glb 3i-make.\textsc{irr} other room-\textsc{distr} here\\
\glft ‘he wants to make other rooms here’
\endgl
\trailingcitation{[jxx-p120430l-1.393]}
\xe

(\ref{ex:NP-num-2}) exemplifies the use of a numeral as a modifier. María S. tells her husband here that I have three children (which is not true, I have only two, thus I corrected her, but this is not of importance for the use of the modifier). 

\ea\label{ex:NP-num-2}
\begingl
\glpreamble \textup{numeral + noun:}\\kakutu treschÿ chichechajinube chijinepÿinube\\
\gla kakutu treschÿ chi-checha-ji-nube chi-jinepÿi-nube\\
\glb exist-\textsc{iam} three 3-son-\textsc{col}-\textsc{pl} 3-daughter-\textsc{pl}\\
\glft ‘she has three sons and daughters by now’
\endgl
\trailingcitation{[rmx-e150922l.076]}
\xe

%kakiu nechÿu pario ubiyaenube, mqx-p110826l.182

In (\ref{ex:NP-adj}), there is an adjective \is{adjective|(} and a noun. The adjective in this example is \textit{kana} ‘this size’, a \isi{demonstrative adjective}, which is always accompanied by a gesture showing the size. It is the one that most often occupies the modifying position before the noun, other adjectives are rare in this position (and in modification in general). The example comes from Juana’s account about her grandparents’ journey from Moxos back home.

\ea\label{ex:NP-adj}
\begingl
\glpreamble \textup{adjective + noun:}\\tumunube kana boteyamÿnÿ aguardiente\\
\gla ti-umu-nube kana boteya-mÿnÿ aguardiente\\
\glb 3i-take-\textsc{pl} this.size bottle-\textsc{dim} liquor\\
\glft ‘they took a little bottle of this size of liquor'
\endgl
\trailingcitation{[jxx-p151016l-2.235]}
\xe\is{adjective|)}

Nominal demonstratives\is{nominal demonstrative|(} are the most frequent modifiers. Both Miguel and Juana use them a lot, María S. less so (see §\ref{sec:DemPron}). A demonstrative and a noun that are juxtaposed can also form a predication, but it is mostly the topic pronoun \textit{chibu} which is used in these cases (see §\ref{sec:FocPron}). In predication, there is usually a short pause between the demonstrative and the noun and the demonstrative is stressed, in an NP they form an intonational\is{intonation} unit and there is another predicate in the clause.\is{nominal demonstrative|)}

Apart from demonstratives, most of the pre-nominal modifiers do actually not frequently occur as modifiers. They are rather used as predicates or adverbs or they \isi{head} an NP themselves. This is because the modified referent is usually accessible and thus does not need to be repeated explicitly. This is in accord to what \citet[168]{Danielsen2007} found out for \isi{Baure}: “[m]odification within an NP is not very common Baure in general”. This also holds for Paunaka, to an even greater degree. 

Consider (\ref{ex:NP-other-2}). Juana first uses \textit{punachÿ} ‘other’ as a predicate here, but noting that she was not explicit enough adds a possessive clause in which \textit{punachÿ} functions as a modifier of a noun. She talks about one of her relatives here.

\ea\label{ex:NP-other-2}
\begingl
\glpreamble punachÿtu, kakutu punachÿ seunube eka chima\\
\gla punachÿ-tu kaku-tu punachÿ seunube eka chi-ima\\
\glb other-\textsc{iam} exist-\textsc{iam} other woman \textsc{dem}a 3-husband\\
\glft ‘it is another one now, her husband has another woman now’
\endgl
\trailingcitation{[jxx-p120430l-1.402]}
\xe

There are also a few clauses in which a quantifier\is{quantifier|(} seems to be used attributively, as in (\ref{ex:many-stories}), where Miguel thinks about which story he could tell us.\footnote{This example could also be analysed as containing a partitive NP, see below.}

\ea\label{ex:many-stories}
\begingl
\glpreamble bueno kaku chama echÿu kuento\\
\gla bueno kaku chama echÿu kuento\\
\glb well exist much \textsc{dem}b story\\
\glft ‘well, there are a lot of stories’
\endgl
\trailingcitation{[jmx-n120429ls-x5.048]}
\xe

A prime example of the attributive use of a quantifier is (\ref{ex:pario-ex-2}), but it is quite unique. It comes from María C. and is about her son.

\ea\label{ex:pario-ex-2}
\begingl
\glpreamble tichupumÿnÿ pario paunaka\\
\gla ti-chupu-mÿnÿ pario paunaka\\
\glb 3i-know-\textsc{dim} some Paunaka\\
\glft ‘he knows some Paunaka’
\endgl
\trailingcitation{[cux-c120414ls-2.269]}
\xe
\is{quantifier|)}

In general, NPs with ordinary pre-nominal modifiers (except for the demonstratives) are likely to occur in a non-verbal clause\is{non-verbal predication} including the \isi{copula} \textit{kaku}.

Modifiers that follow the noun can be nouns, adjectives\is{adjective|(} or relative clauses.\is{relative relation} Adjectives are probably best analysed as a subtype of relative clause, since they are generally rather used predicatively than attributively (see §\ref{sec:UsesADJ}). Note that headed RCs\is{head} including a verb are normally completely unmarked, just like adjectives following the noun.\is{adjective|)}

A nominal modifier of another noun often denotes a possessor.\is{possessor|(} As has been shown in §\ref{sec:Possession}, the possessor is indexed on the possessed, but if the possessor is a third person, a possessor noun can co-occur, as in (\ref{ex:possd-poss-1}), which comes from María S., who was repeating a statement of her brother for Swintha. My daughter learned to walk on her own, when we were in Bolivia together in 2011, an issue which is still remembered with pleasure.

\ea\label{ex:possd-poss-1}
\begingl
\glpreamble tiyuikutu chijinepÿimÿnÿ Elena\\
\gla ti-yuiku-tu chi-jinepÿi-mÿnÿ Elena\\
\glb 3i-walk-\textsc{iam} 3-daughter-\textsc{dim} Lena\\
\glft ‘Lena’s daughter is walking now’
\endgl
\trailingcitation{[rxx-e121128s-1.071]}
\xe

A kind of possessor (the figure) also follows the possessed ground in expressions of specific locative relations. In (\ref{ex:possd-poss-2}), Juana cites her brother who was about to depart to a visit at his other brother’s.\footnote{Note that combination of the third person marker \textit{chÿ-} with the \isi{relational noun} \textit{-chuku} ‘side’ is one of the very few cases where we often find \isi{haplology}, so that the possessed form is \textit{chuku} besides \textit{chichuku}.}

\ea\label{ex:possd-poss-2}
\begingl
\glpreamble “niyuna chukuyae Kujtin”\\
\gla ni-yuna chi-chuku-yae Kujtin\\
\glb 1\textsc{sg}-go.\textsc{irr} 3-side-\textsc{loc} Agustín\\
\glft ‘“I go to Agustín”’
\endgl
\trailingcitation{[jxx-p120430l-2.392]}
\xe

Occasionally, the possessor is itself modified by a demonstrative. (\ref{ex:possd-poss-3}) is an example of this. It comes from Miguel telling the story about the two men and the devil. It is the devil who eats up the heads and also all the rest of the meat the men had just hunted.

\ea\label{ex:possd-poss-3}
\begingl 
\glpreamble chijikupupuikutuji echÿu chichÿti echÿu ÿbajane\\
\gla chi-jikupu-puiku-tu-ji echÿu chi-chÿti echÿu ÿba-jane\\ 
\glb 3-swallow-\textsc{cont}-\textsc{iam}-\textsc{rprt} \textsc{dem}b 3-head \textsc{dem}b pig-\textsc{distr}\\ 
\glft ‘he was swallowing the heads of the pigs’
\trailingcitation{[mxx-n101017s-1.052-053]} 
\xe
\is{possessor|)}

In the other kind of noun-noun combination, the modifier specifies the type. This can include combinations of a noun denoting a kind of measure term and the other one the thing which is measured, as in (\ref{ex:NP-1}), or an object and its material, as in (\ref{ex:NP-2}), both from Juana.

In (\ref{ex:NP-1}), the \isi{head} noun \textit{babetamÿnÿ} ‘little trough’ is the measure term, which is modified by \textit{ÿne} ‘water’, the item which comes in this measure. The example stems from Juana’s account about some gold in the woods, which is watched over by a spirit.

\ea\label{ex:NP-1}
\begingl
\glpreamble kaku ÿne kaku nena babetamÿnÿ ÿne\\
\gla kaku ÿne kaku nena babeta-mÿnÿ ÿne\\
\glb exist water exist like trough-\textsc{dim} water\\
\glft ‘there is water, there is what looks like a little trough of water’
\endgl
\trailingcitation{[jxx-p151020l-2]}
\xe

The \isi{head} noun in (\ref{ex:NP-2}) is \textit{yÿpi} ‘jar’ and it is modified by \textit{muteji} ‘loam, mud’ denoting the material of the jar. Juana is talking about the old days in Santa Rita here, before the reservoir and later the pump were constructed. They had to walk far with their clay jars to fetch water.

\ea\label{ex:NP-2}
\begingl
\glpreamble kuinakuÿ, puro eka yÿpi muteji\\
\gla kuina-kuÿ puro eka yÿpi muteji\\
\glb \textsc{neg}-\textsc{incmp} mere \textsc{dem}a jar loam\\
\glft ‘there were no (plastic canisters) yet, it was only with jars of clay’
\endgl
\trailingcitation{[jxx-p120515l-2.058]}
\xe

Possession of non-possessable nouns\is{non-possessability|(} is yet another kind of modification by a type noun: a possessed relational noun\is{relational noun|(} occurs together with a non-possessable noun, which encodes the type of thing that is possessed.  Animals cannot be possessed directly, the relational noun \textit{-peu} ‘domestic animal’ is needed if a possessive relationship to an animal shall be expressed (see §\ref{sec:Non-possessables}). One example is (\ref{ex:rel-possd-1}): the relational noun \textit{-peu} ‘domestic animal’ comes first, the animal denoting the type of possessed animal follows. The sentence comes from María C. who speaks about the lack of meat in her nutrition.

\ea\label{ex:rel-possd-1}
\begingl
\glpreamble kakuina bipeujanemÿnÿ ÿba bikupaka\\
\gla kaku-ina bi-peu-jane-mÿnÿ ÿba bi-kupaka\\ 
\glb exist-\textsc{irr.nv} 1\textsc{pl}-animal-\textsc{distr}-\textsc{dim} pig 1\textsc{pl}-kill.\textsc{irr}\\ 
\glft ‘if we had pigs, we would butcher them’
\trailingcitation{[uxx-p110825l.200]}
\xe


Together with this type of relational noun, either the N\textsubscript{type} can co-occur, as in (\ref{ex:rel-possd-1}) above, or the \isi{possessor}, the N\textsubscript{poss}.\footnote{It is also possible to use the relational noun on its own without being modified.}  (\ref{ex:possd-poss-4}) has a relational noun modified by the noun denoting the possessor. In this case, the NP can actually be analysed as a predicate itself, as a relative clause specifying the preceding noun \textit{bakajane} ‘the cows’. I have found no example in which an NP of the type [N\textsubscript{rel} N\textsubscript{poss}] is an argument of a verbal clause. The example comes from Miguel who told me the story of the cowherd and the spirit of the hill.

\ea\label{ex:possd-poss-4}
\begingl
\glpreamble chikuirauchuji echÿu bakajane chipeujane chipatrune\\
\gla chi-kuirauchu-ji echÿu baka-jane chi-peu-jane chi-patrun-ne\\
\glb 3-care.for-\textsc{rprt} \textsc{dem}b cow-\textsc{distr} 3-animal-\textsc{distr} 3-patrón-\textsc{possd}\\
\glft ‘he looked for the cows, it is said, (which were) the animals of his \textit{patrón}’
\endgl
\trailingcitation{[mxx-n151017l-1.02]}
\xe

Only in one example in the corpus a relational noun is accompanied by both N\textsubscript{poss} and N\textsubscript{type}, in this order.\is{word order} It comes from Juana who was talking about her ducklings, which were not fed properly when she was away to Santa Cruz once. Apparently, only her son (or grandchild?) took some care, saying: 

\ea\label{ex:possd-poss-5}
\begingl
\glpreamble “tikunipajanemÿnÿ chipeujane mimi upuji”\\
\gla ti-kunipa-jane-mÿnÿ chi-peu-jane mimi upuji\\
\glb 3i-be.hungry-\textsc{distr}-\textsc{dim} 3-animal-\textsc{distr} mum duck\\
\glft ‘“the ducks of my mum are hungry”’
\endgl
\trailingcitation{[jrx-c151001lsf-11.067]}
\xe
\is{relational noun|)}
\is{non-possessability|)}

Inside an NP,\is{agreement|(} number, \isi{diminutive} and locative marking\is{locative marker} generally occur only once,\footnote{This does not come as a surprise. In the sample of \citet[]{Krasnoukhova2012}, only 16 out of 55 South American languages mark agreement in number with demonstratives\is{nominal demonstrative} as noun modifiers \citep[52]{Krasnoukhova2012}. Agreement in number is even rarer with modifying numerals\is{numeral} and adjectives\is{adjective} \citep[126, 163–164]{Krasnoukhova2012}.} and typically on the \isi{head} noun, see (\ref{ex:NP-other-1}) – (\ref{ex:NP-adj}), (\ref{ex:possd-poss-2}), (\ref{ex:NP-1}), (\ref{ex:rel-possd-1}), (\ref{ex:possd-poss-5}), but also sometimes on the modifier as in (\ref{ex:possd-poss-3}). (\ref{ex:NP-num}) is interesting in this regard, because the \isi{plural} marker is attached to the modifier and the \isi{diminutive} to the \isi{head} noun. The sentence was produced by María S. who spoke about her life in the old times and referred to her sister Juana here.

\ea\label{ex:NP-num}
\begingl
\glpreamble kakutu ruschÿnube chichechajimÿnÿbane\\
\gla kaku-tu ruschÿ-nube chi-checha-ji-mÿnÿ-bane\\
\glb exist-\textsc{iam} two-\textsc{pl} 3-son-\textsc{col}-\textsc{dim}-\textsc{rem}\\
\glft ‘she already had two little children by that time long ago’
\endgl
\trailingcitation{[rxx-p181101l-2.107]}
\xe

Plural\is{plural} marking may occur on both the modifier and the noun, but this is very rare. One example by Miguel of double plural marking – on \textit{punachÿ} ‘other’ and on \textit{krinko} ‘gringo’ – is given in (\ref{ex:double-pl}), where he tells Juan C. that some gringos told him not to let a \textit{patrón} take advantage of him.

\ea\label{ex:double-pl}
\begingl
\glpreamble tikechunenube echÿu punachÿnube krinkonube\\
\gla ti-kechu-ne-nube echÿu punachÿ-nube krinko-nube\\
\glb 3i-say-1\textsc{sg}-\textsc{pl} \textsc{dem}b other-\textsc{pl} gringo-\textsc{pl}\\
\glft ‘these other gringos told me’
\endgl
\trailingcitation{[mqx-p110826l.381]}
\xe\is{agreement|)}


There are usually not more than two modifiers in an NP, a demonstrative\is{nominal demonstrative} and another modifier, but a few other and partly more complex combinations have been found in the corpus. (\ref{ex:complNP-1}) has a numeral and an adjective. It was produced by Juana when looking at a picture book in order to teach some Paunaka phrases to her grandchild.

\ea\label{ex:complNP-1}
\begingl
\glpreamble sietechÿ sepitÿmÿnÿ kusu\\
\gla sietechÿ sepitÿ-mÿnÿ kusu\\
\glb seven small-\textsc{dim} mouse\\
\glft ‘seven little mice’
\endgl
\trailingcitation{[jxx-e081025s.050]}
\xe


(\ref{ex:complNP-2}) has a numeral that is used like an indefinite article by Miguel in this case, and an adjective which follows the noun. The sentence comes from the story about the lazy man who cuts off his limbs in the end to be thrown into the water by his son and rise as a comet.

\ea\label{ex:complNP-2}
\begingl
\glpreamble “pumane nauku kaku nauku chinachÿ posa mutemena”\\
\gla pi-uma-ne nauku kaku nauku chinachÿ posa mutemena\\
\glb 2\textsc{sg}-take.\textsc{irr}-1\textsc{sg} there exist there one well big\\
\glft ‘“take me there where there is a big well”’
\endgl
\trailingcitation{[mox-n110920l.121]}
\xe

In (\ref{ex:complNP-3}), the \isi{head} noun \textit{rasimo} ‘raceme’ is modified by the preceding demonstrative, by \textit{punachÿ} ‘other’ and by the following noun which specifies the type of raceme. The sentence comes from the same story as (\ref{ex:complNP-2}) above. The lazybones, sitting in the top of a \textit{cusi} palm tree, cuts off his limbs at this point of the story and drops them to the ground, telling his son to collect the supposed racemes of \textit{cusi}.

\ea\label{ex:complNP-3}
\begingl
\glpreamble “pijakupaji echÿu punachÿ rasimo kÿsi”\\
\gla pi-jakupa-ji echÿu punachÿ rasimo kÿsi\\
\glb 2\textsc{sg}-receive.\textsc{irr}-\textsc{imp} \textsc{dem}b other raceme cusi\\
\glft ‘“take this other raceme of \textit{cusi} palm fruit”’
\endgl
\trailingcitation{[mox-n110920l.105]}
\xe

Finally, it should also be mentioned that speakers sometimes use a \isi{numeral}, quantifier\is{quantifier} or the word for ‘other’ \textit{before} a demonstrative to form a partitive NP. This kind of NP has also been analysed for \isi{Baure} \citep[cf.][125]{Danielsen2007}, but it is not very frequent in Paunaka. (\ref{ex:NP-part-1}) is an example by Miguel who was narrating the story of the two men who meet the devil in the woods. The devil is the one who shouts.

\ea\label{ex:NP-part-1}
\begingl
\glpreamble i chinachÿ echÿu chikompanyerone chijakupu echÿu tiyÿbui\\
\gla i chinachÿ echÿu chi-kompanyero-ne chi-jakupu echÿu ti-yÿbui\\
\glb and one \textsc{dem}b 3-companion-\textsc{possd} 3-receive \textsc{dem}b 3i-shout\\
\glft ‘and one of the companions answered the one who shouted’
\endgl
\trailingcitation{[mxx-n101017s-1.021]}
\xe

(\ref{ex:NP-part-2}) is a short switch to Paunaka in Juana’s otherwise Spanish discourse. She was telling Swintha the creation story and switched back and forth between Paunaka and Spanish.

\ea\label{ex:NP-part-2}
\begingl
\glpreamble tumuyubu tumuyubu eka mukiankajanemÿnÿ\\
\gla tumuyubu tumuyubu eka mukianka-jane-mÿnÿ\\
\glb all all \textsc{dem}a animal-\textsc{distr}-\textsc{dim}\\
\glft ‘all, all of the animals’
\endgl
\trailingcitation{[jxx-n101013s-1.696]}
\xe
\is{modification|)}
\is{noun phrase|)}
\is{noun|)}

The following chapter discusses the verb and morphology associated with the verb (or other predicates) in more detail.




