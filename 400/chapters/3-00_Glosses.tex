%!TEX root = Masterdokument.tex
%!TEX encoding = UTF-8 Unicode
\chapter*{Abbreviations and codes}\label{sec:glosses}


%\twocoloumn



 \renewcommand{\arraystretch}{1.2}
\begin{longtable}[l]{ll}
1\textsc{pl} & first person plural (\sectref{sec:Possession}, \sectref{sec:1_2Marking}, \sectref{sec:NonVerbalPredication})\\
1\textsc{sg} & first person singular (\sectref{sec:Possession}, \sectref{sec:1_2Marking}, \sectref{sec:NonVerbalPredication})\\
2\textsc{pl} & second person plural (\sectref{sec:Possession}, \sectref{sec:1_2Marking}, \sectref{sec:NonVerbalPredication})\\
2\textsc{sg} & second person singular (\sectref{sec:Possession}, \sectref{sec:1_2Marking}, \sectref{sec:NonVerbalPredication})\\
3 & third person (\sectref{sec:Possession}, \sectref{sec:3Marking})\\
3i & third person (\sectref{sec:3Marking})\\
A & subject of transitive verb (\sectref{sec:RelativeClauses})\\
AC & adverbial clause (\sectref{sec:AdverbialClauses})\\
\textsc{add} & additive (\sectref{sec:Additive})\\
ADJ & adjective (\sectref{sec:NP})\\
\textsc{adm} & admonitive (\sectref{sec:Prohibitives})\\
\textsc{afm} & affirmation (\sectref{sec:EmphMarker}, \sectref{sec:PolarQuestions})\\
AM & associated motion (\sectref{sec:AssociatedMotion})\\
\textsc{am.conc.cis} & concurrent cislocative motion (\sectref{sec:AMconcurrent})\\
\textsc{am.conc.tr} & concurrent translocative motion (\sectref{sec:AMconcurrent})\\
\textsc{am.prior} & prior motion (\sectref{sec:punu})\\
\textsc{am.subs} & subsequent motion (\sectref{sec:SubsequentMotion})\\
\textsc{avert} & avertive (\sectref{sec:Frust_avertive_optatiev})\\
\textsc{attr} & attributive (\sectref{sec:AttributiveVerbs})\\
\textsc{ben} & benefactive (\sectref{sec:Benefactive})\\
Bés. & Bésiro\\
C & consonant (\sectref{sec:SyllableStructure_subsection})\\
\textsc{caus} & causative (1 \& 2) (\sectref{sec:Causative})\\
CC & complement clause (\sectref{sec:ComplementClauses})\\
\textsc{clf} & classifier (\sectref{sec:Classifiers}, \sectref{sec:Nouns_CLF}, \sectref{sec:StativeVerbs_CLF}, \sectref{sec:CLF_ActiveVerbs})\\
\textsc{col} & collective (\sectref{sec:Collective}, \sectref{CollectiveVerbs})\\
\textsc{com} & comitative (\sectref{sec:adp-ajechubu})\\
\textsc{cont} & continuous (\sectref{sec:ActiveVerbs_RDPL}, \sectref{sec:ContinuousAspect})\\
\textsc{dec} & deceased (\sectref{sec:Deceased})\\
\textsc{dec.pn} & deceased for proper name (\sectref{sec:Deceased})\\
\textsc{ded} & deductive (\sectref{sec:ModalityDeductive})\\
\textsc{deict} & deictic marker (\sectref{sec:AMconcurrent})\\
\textsc{dem} & demonstrative (a, b \& c) (\sectref{sec:DemPron})\\
\textsc{der} & derivational affix (\sectref{sec:BodyPartofPartDerivation})\\
\textsc{dim} & diminutive (\sectref{sec:Diminutives})\\
\textsc{distr} & distributive (\sectref{sec:NounPL-jane}, \sectref{sec:Verbs_3PL})\\
\textsc{dloc} & dislocative (\sectref{sec:PA}, \sectref{sec:ConstructionTypePA}, \sectref{sec:SVC_and_MCPC}, \sectref{sec:MotionCumPurpose})\\
\textsc{dsc} & discontinuous (\sectref{sec:Discontinuous})\\
\textsc{emph} & emphatic (1 \& 2) (\sectref{sec:EmphMarker})\\
\textsc{ext} & extension applicative (\sectref{sec:EXTApplicative})\\
\textsc{frust} & frustrative (\sectref{sec:Frustrative})\\
\textsc{grn} & general relational noun (\sectref{sec:Non-possessables}, \sectref{sec:GenitiveBenfactivePreds})\\
\textsc{hon} & honorific \\%(\sectref{sec:Kapunu}, \sectref{sec:CC_Desideratives})
\textsc{hort} & hortative (\sectref{sec:Hortatives})\\
\textsc{iam} & iamitive (\sectref{sec:Iamitive})\\
\textsc{idph} & ideophone (\sectref{sec:UsesADJ})\\
\textsc{imp} & imperative (\sectref{sec:MarkedImperatives}, \sectref{sec:SuppletiveImperatives})\\
\textsc{incmp} & incompletive (\sectref{sec:Incompletive})\\
\textsc{ins} & instrumental (or cause) (\sectref{sec:adp-keuchi})\\
\textsc{intj} & interjection\\
\textsc{ints} & intensifier (\sectref{sec:Intensifier})\\
\textsc{intsv} & intensive (\sectref{sec:IntensiveAktionsart})\\
\textsc{irr} & irrealis (\sectref{sec:RealityStatus})\\
\textsc{irr.nv} & non-verbal irrealis (\sectref{NominalRS}, \sectref{sec:NonVerbalPredication})\\
\textsc{lim} & limitative (1 \& 2) (\sectref{sec:Limitatives})\\
\textsc{loc} & locative (\sectref{sec:Locative})\\
MC & main clause (Chapter \ref{sec:ComplexClauses})\\
MCPC & motion-cum-purpose construction (\sectref{sec:PA}, \sectref{sec:ConstructionTypePA}, \sectref{sec:SVC_and_MCPC}, \sectref{sec:MotionCumPurpose})\\
\textsc{mid} & middle voice (\sectref{sec:Middle_voice})\\
\textsc{mir} & mirative (\sectref{sec:Frustrative})\\
N & noun (\sectref{sec:NP})\\
\textsc{neg} & negation (\sectref{sec:Negation})\\
\textsc{nmlz} & nominaliser (\sectref{sec:MorphologyNominalisation}, \sectref{sec:SyntaxNominalisation})\\
NP & noun phrase (\sectref{sec:NP})\\
\textsc{npossd} & non-possessed (\sectref{sec:Inalienables})\\
NUM & numeral (\sectref{sec:NP})\\
O & object (\sectref{sec:WordOrder})\\
\textsc{obl} & oblique (\sectref{sec:adp-tÿpi})\\
\textsc{opt} & optative (1 \& 2) (\sectref{sec:FRUST-Optative})\\
%PBA & Proto Bolivian Arawakan (\sectref{})\\
PDP & Paunaka Documentation Project (\sectref{sec:Fieldwork})\\
\textsc{pl} & plural (\sectref{sec:Possession}, \sectref{sec:Verbs_3PL})\\
%\textsc{pn} & proper name\\
POSS & possessor (\sectref{sec:RelativeClauses})\\
\textsc{possd} & possessed (\sectref{sec:Alienables})\\
PRED & predicate (\sectref{sec:WordOrder})\\
\textsc{priv} & privative (\sectref{sec:Negation})\\
\textsc{prn} & pronoun (\sectref{chapter:Pronouns})\\
\textsc{proh} & prohibitive (\sectref{sec:Prohibitives})\\
\textsc{prsp} & prospective (\sectref{sec:Prospective})\\
\textsc{punct} & punctual (\sectref{sec:TemporalAspectualAdverbs})\\
Q & quantifier (\sectref{sec:NP})\\
RC & relative clause (\sectref{sec:RelativeClauses})\\
\textsc{rcpc} & reciprocal (\sectref{sec:RCPC})\\
\textsc{rdpl} & reduplication (\sectref{sec:StativeVerbs_RDPL}, \sectref{sec:ActiveVerbs_RDPL})\\
\textsc{real} & realis (\sectref{sec:RealityStatus})\\
\textsc{reg} & regressive (or repetitive) (\sectref{sec:Repetition})\\
\textsc{rel} & relativiser (\sectref{sec:Q_chija})\\
\textsc{rem} & remote (past or distance) (\sectref{sec:RemotePast})\\
\textsc{rep} & repetition (\sectref{sec:ADJInventory})\\
\textsc{rprt} & reportive (\sectref{sec:Evidentiality})\\
RS & reality status (\sectref{sec:RealityStatus})\\
S & subject (\sectref{sec:WordOrder})\\
S & subject of intransitive verb (\sectref{sec:RelativeClauses})\\
SAP & speech act participant\\
Span. & Spanish\\
\textsc{subord} & subordinate (\sectref{sec:Subordination-i})\\
SVC & serial verb construction (\sectref{sec:SVC_and_MCPC}, \sectref{sec:SerialVerbs})\\
TAME & tense, aspect, mood/modality, and evidentiality (\sectref{sec:OperationsPredicates})\\
\textsc{th} & thematic (1 \& 2) (\sectref{sec:ActiveVerbs_TH})\\
\textsc{top} & topic (\sectref{sec:FocPron}) (\sectref{sec:FocPron})\\
\textsc{uncert} & uncertainty (\sectref{sec:ModalityUncertainty})\\
\textsc{uncert.fut} & uncertain future (\sectref{sec:UncertainFuture})\\
V & verb (\sectref{sec:WordOrder})\\
V & vowel (\sectref{sec:SyllableStructure_subsection})\\
X & oblique (\sectref{sec:WordOrder})\\
\label{table:Glosses}
\end{longtable}
 \renewcommand{\arraystretch}{1}


%\onecolumn

Codes of sessions consist of a code for the consultant(s) followed by a dash, type of session, six-digit date (yymmdd), abbreviation of researcher(s), and in some cases a dash and an additional number if there was more than one session of the same type with the same speaker and researcher on the same day. For example, an abbreviation jmr-c120415lsf-2 reads as: second conversation between the speakers j, m, and r on the 15th of April 2012 with the researchers l, s, and f.

Speaker abbreviations are given in Table \ref{table:Speakers}.

\begin{table}[htbp]
\caption{Paunaka consultants’ codes}
\begin{tabular}{ll}
\lsptoprule
Code & Full name \cr
\midrule
c & Clara Supepí Yabeta \cr
d & Isidro Supepí Chijene\cr
j & Juana Supepí Yabeta\cr
m & Miguel Supepí Yabeta\cr
n & Juan Choma (or Chamo) (†)\cr
o & José Supepí Yabeta (†)\cr
p & Pedro Pinto Supepí\cr
q & Juan Cuasase Supepí\cr
r & María Supepí Yabeta\cr
t & Alejo Supayabe Pinto (†) \cr
u & María Cusase Choma\cr
y & Polonia Supayabe Pinto\cr
\lspbottomrule
\end{tabular}
\label{table:Speakers}
\end{table}%


The speaker position always has three digits. If there are fewer than three speakers, x is inserted as a placeholder. The different session types are listed in Table \ref{table:Session_abbreviations}. However, most recordings represent a mixture of several types, so that the type of session may actually not coincide with the type of information gathering of a certain word or sentence in an example. The origin and nature of the examples is thus explained in the text.

\begin{table}[htbp]
\caption[]{Session abbreviations}
\begin{tabularx}{\textwidth}{llQ}
\lsptoprule
Abbreviation & Meaning & Comment\cr
\midrule
a & artificial & frog stories or stories invented by the researchers\cr
c & conversation & conversation between two or more speakers\cr
d & description & procedural texts\cr
e & elicitation & includes conversation between researchers and speakers\cr
f & anthropological interview & only in Spanish \cr
h & historical & Paunaka history, only in Spanish\cr
n & narrative & stories\cr
p & personal narrative & events and deeds in one’s life or the life of family members\cr
r & rights & permission to store the recording in the archive\cr
s & song & songs, pieces of music\cr
\lspbottomrule
\end{tabularx}
\label{table:Session_abbreviations}
\end{table}

Table \ref{table:Researchers} shows the researchers’ abbreviations.

\begin{table}[htbp]
\caption[]{Researchers’ abbreviations}
\begin{tabular}{ll}
\lsptoprule
Code & Full name\cr
\midrule
e & Lena Sell\cr
f & Federico Villalta Rojas\cr
g & Jürgen Riester\cr
l & Lena Terhart\cr
s & Swintha Danielsen\cr
\lspbottomrule
\end{tabular}
\label{table:Researchers}
\end{table}

Following the session code, a number is provided, which is the annotation number the example had at the point in time, when I inserted it into the text. Please note that, while session codes are fixed, some example numbers may change. For example, I noticed while working on this book that some portions of some recordings were not transcribed and I would like to add the missing transcriptions in the future. This will entail re-numbering. In case that there is no annotation number, this is likely to be due to the fact that the recording was only transcribed in parts, thus no numbering applied to it. Again, this holds only for the moment in which I added the example to the description.
