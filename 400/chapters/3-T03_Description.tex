%!TEX root = 3-P_Masterdokument.tex
%!TEX encoding = UTF-8 Unicode

\section{Text 3: Making a clay pot}\label{text:ClayPot}

This is a description by Juana of how to make a clay pot. The recording can be found here:
\url{https://www.elararchive.org/uncategorized/IO_32c4fc3e-0cbd-4b49-adda-3c662fe4d8f3}.
The description was recorded on the 18th of September 2011. There are several similar recordings with the same topic, but this specific one was originally archived as jmx-d110918ls-2. Miguel intervenes once to help Juana out with a word, and this intervention is marked with a preposed “m”. The remainder of this text was produced by Juana alone. Spanish parts are not analysed, but a translation is provided.

The recording was made somewhere in the woods close to Santa Rita, where Juana and Miguel were digging for loam for Juana to make the pot. When she had collected enough loam, we asked her for a description of the production of the pot. 



\ea%[everyglpreamble=\it, exno={1.}, exnoformat=X]<ex:>
\begingl
\glpreamble buma eka bupukene\\
\gla bi-uma eka bi-upukene\\
\glb 1\textsc{pl}-take.\textsc{irr} \textsc{dem}a 1\textsc{pl}-load\\
\glft ‘we take our load’
\endgl
\xe
%verwendet als <ex:NMLZ-j1> in 8.1.5 Nominalisation

\ea%[everyglpreamble=\it, exno={2.}, exnoformat=X]<ex:>
\begingl 
\glpreamble betuku naka bichÿtiyae i biyunatu\\
\gla bi-etuku naka bi-chÿti-yae i bi-yuna-tu\\ 
\glb 1\textsc{pl}-put here 1\textsc{pl}-head-\textsc{loc} and 1\textsc{pl}-go.\textsc{irr}-\textsc{iam}\\ 
\glft ‘we put it here on our head and we can go (now)’\\ 
\endgl
\xe
%verwendet in Iamitive


\ea%[everyglpreamble=\it, exno={3.}, exnoformat=X]<ex:>
\begingl
\glpreamble biyuna buma bubiuyae\\
\gla bi-yuna bi-uma bi-ubiu-yae\\
\glb 1\textsc{pl}-go.\textsc{irr} 1\textsc{pl}-take.\textsc{irr} 1\textsc{pl}-house-\textsc{loc}\\
\glft ‘we go taking it home’
\endgl
\xe
%verwendet in Adverbial Clauses

\ea%[everyglpreamble=\it, exno={4.}, exnoformat=X]<ex:>
\begingl 
\glpreamble te betuka ÿne naka taurayae\\
\gla te bi-etuka ÿne naka taura-yae\\ 
\glb \textsc{seq} 1\textsc{pl}-put.\textsc{irr} water here table-\textsc{loc}\\ 
\glft ‘then we put water here on the table’\\ 
\endgl
\xe


\ea%[everyglpreamble=\it, exno={5.}, exnoformat=X]<ex:>
\begingl
\glpreamble upujaine bitÿyajikatu chikeuchi yubauke\\
\gla upu-jai-ne bi-tÿyajika-tu chi-keuchi yubauke\\
\glb other-day-\textsc{possd} 1\textsc{pl}-grind.\textsc{irr}-\textsc{iam} 3-\textsc{ins} pestle\\
\glft ‘the next day we can grind it with a pestle’
\endgl
\xe
%verwendet: <ex:other-7> in Numerals und in Instrument and cause preposition als <ex:adp-keuchi-2>

\ea%[everyglpreamble=\it, exno={6.}, exnoformat=X]<ex:>
\begingl 
\glpreamble bitÿyajika te bibeatu eka maikijacha\\
\gla bi-tÿyajika te bi-bea-tu eka mai-ki-jacha\\ 
\glb 1\textsc{pl}-grind.\textsc{irr} \textsc{seq} 1\textsc{pl}-take.away.\textsc{irr}-\textsc{iam} \textsc{dem}a stone-\textsc{clf:}spherical-?\\ 
\glft ‘we grind it and then we pick out the pebbles’\\ 
\endgl
\xe


\ea%[everyglpreamble=\it, exno={7.}, exnoformat=X]<ex:>
\begingl 
\glpreamble bibikÿka apuke kaku maijane\\
\gla bi-bikÿka apuke kaku mai-jane\\ 
\glb 1\textsc{pl}-throw.\textsc{irr} ground exist stone-\textsc{distr}\\ 
\glft ‘we throw them down if there are stones’\\ 
\endgl
\xe


\ea%[everyglpreamble=\it, exno={8.}, exnoformat=X]<ex:>
\begingl 
\glpreamble i chikuye\\
\gla i chikuye\\ 
\glb and like.this\\ 
\glft ‘and that’s it’\\ 
\endgl
\xe


\ea%[everyglpreamble=\it, exno={9.}, exnoformat=X]<ex:>
\begingl 
\glpreamble metuinatu bipapapuichajane\\
\gla metu-ina-tu bi-papapuicha-jane\\ 
\glb already-\textsc{irr.nv}-\textsc{iam} 1\textsc{pl}-make.ball?.\textsc{irr}-\textsc{distr}\\ 
\glft ‘now we can make the (clay) balls’\\ 
\endgl
\xe


\ea%[everyglpreamble=\it, exno={10.}, exnoformat=X]<ex:>
\begingl 
\glpreamble bipapapuichajane\\
\gla bi-papapuicha-jane\\ 
\glb 1\textsc{pl}-make.ball?.\textsc{irr}-\textsc{distr}\\ 
\glft ‘we make the balls’\\ 
\endgl
\xe


\ea%[everyglpreamble=\it, exno={11.}, exnoformat=X]<ex:>
\begingl 
\glpreamble i despues tauramÿnÿ banatu chitikejimÿnÿina\\
\gla i despues taura-mÿnÿ bi-ana-tu chi-tikeji-mÿnÿ-ina\\ 
\glb and afterwards board-\textsc{dim} 1\textsc{pl}-make.\textsc{irr}-\textsc{iam} 3-intestines-\textsc{dim}-\textsc{irr.nv}\\ 
\glft ‘and after that we can make the plaits (lit.: intestines) on a small board’\\ 
\endgl
\xe


\ea%[everyglpreamble=\it, exno={12.}, exnoformat=X]<ex:>
\begingl 
\glpreamble kana tÿpi tÿpi ee eka ¿cómo cómo te dijera?\\
\gla kana tÿpi tÿpi ee eka {cómo cómo te dijera}\\ 
\glb this.size \textsc{obl} \textsc{obl} \textsc{intj} \textsc{dem}a {how how could I tell you}\\ 
\glft ‘of this size in order to, in order to – er – how, how could I tell you?’\\ 
\endgl
\xe


\ea%[everyglpreamble=\it, exno={13.}, exnoformat=X]<ex:>
\begingl 
\glpreamble \textup{m:} -tibuiu\\
\gla tibu-i-u\\ 
\glb sit.down-\textsc{subord}-\textsc{real}\\ 
\glft ‘get plane’\\ 
\endgl
\xe

\ea%[everyglpreamble=\it, exno={14.}, exnoformat=X]<ex:>
\begingl 
\glpreamble chitibuia\\
\gla chi-tibu-i-a\\ 
\glb 3-sit.down-\textsc{subord}-\textsc{irr}\\ 
\glft ‘so that it gets plane’\\ 
\endgl
\xe

\ea%[everyglpreamble=\it, exno={15.}, exnoformat=X]<ex:>
\begingl 
\glpreamble i nebu tanaputu\\
\gla i nebu ti-ana-pu-tu\\ 
\glb and 3\textsc{obl.top.prn} 3i-make.\textsc{irr}-\textsc{mid}-\textsc{iam}\\ 
\glft ‘and from this it is made now’\\ 
\endgl
\xe

\ea%[everyglpreamble=\it, exno={16.}, exnoformat=X]<ex:>
\begingl
\glpreamble i keuchi sipÿ ÿne naka bijatÿkatu anÿke\\
\gla i keuchi sipÿ ÿne naka bi-jatÿka-tu anÿke\\
\glb and \textsc{ins} shell water here 1\textsc{pl}-pull.\textsc{irr}-\textsc{iam} up\\
\glft ‘and with shell and water we pull it up here’
\endgl
\xe
%verwendet in Adpositions

\ea%[everyglpreamble=\it, exno={17.}, exnoformat=X]<ex:>
\begingl 
\glpreamble i betukupunuka chitikejimÿnÿ te banaupupunuka\\
\gla i bi-etuku-punuka chi-tikeji-mÿnÿ te bi-anau-pupunuka\\ 
\glb and 1\textsc{pl}-put-\textsc{reg.irr} 3-intestines-\textsc{dim} \textsc{seq} 1\textsc{pl}-make-\textsc{reg.irr}\\ 
\glft ‘and again we put plaits and then we do it again’\\ 
\endgl
\xe


\ea%[everyglpreamble=\it, exno={18.}, exnoformat=X]<ex:>
\begingl 
\glpreamble te betukatuchi eka chimusunÿkÿina\\
\gla te bi-etuka-tu-chi eka chi-musunÿkÿ-ina\\ 
\glb \textsc{seq} 1\textsc{pl}-put.\textsc{irr}-\textsc{iam}-3 \textsc{dem}a 3-lip-\textsc{irr.nv}\\ 
\glft ‘then we put this for its handle (lit.: lip) onto it’\\ 
\endgl
\xe

\ea%[everyglpreamble=\it, exno={19.}, exnoformat=X]<ex:>
\begingl 
\glpreamble chimusunÿkÿina masa nebu eka ¿chija? bakachiachi\\
\gla chi-musunÿkÿ-ina masa nebu eka ¿chija? bi-akach-i-a-chi\\ 
\glb 3-lip-\textsc{irr.nv} lest 3\textsc{obl.top.prn} \textsc{dem}a what 1\textsc{pl}-lift-\textsc{subord}-\textsc{irr}-3\\ 
\glft ‘what is going to be its handle lest, from which – er – what was it? so that we can lift it’\\ 
\endgl
\xe


\ea%[everyglpreamble=\it, exno={20.}, exnoformat=X]<ex:>
\begingl 
\glpreamble betukatu\\
\gla bi-etuka-tu\\ 
\glb 1\textsc{pl}-put.\textsc{irr}-\textsc{iam}\\ 
\glft ‘we put it now’\\ 
\endgl
\xe

\newpage

\ea%[everyglpreamble=\it, exno={21.}, exnoformat=X]<ex:>
\begingl 
\glpreamble tamÿra kaku mai ¿chija? biyaneuka\\
\gla ti-a-mÿra kaku mai chija bi-yaneuka\\ 
\glb 3i-\textsc{irr}-be.dry exist stone what 1\textsc{pl}-polish.\textsc{irr}\\ 
\glft ‘when it is dry there is a stone – what was it? to polish it with’\\ 
\endgl
\xe

\ea%[nopreamble=true, exno={23.}, exnoformat=X]<ex:>
\begingl 
\gla {bruñimos adentro y afuera, piedra suave}\\
\\
\glft ‘we polish it from inside and outside, a soft stone’
\endgl
\xe

\ea%[everyglpreamble=\it, exno={24.}, exnoformat=X]<ex:>
\begingl 
\glpreamble ee chibu echÿu\\
\gla ee chibu echÿu\\ 
\glb \textsc{intj} 3\textsc{top.prn} \textsc{dem}b\\ 
\glft ‘er – this is it’\\ 
\endgl
\xe

\ea%[everyglpreamble=\it, exno={25.}, exnoformat=X]<ex:>
\begingl 
\glpreamble ¿chija? tajumeyemÿnÿ\\
\gla chija ti-a-jumeye-mÿnÿ\\ 
\glb what 3i-\textsc{irr}-be.smooth-\textsc{dim}\\ 
\glft ‘what was it? it gets smooth’\\ 
\endgl
\xe

\ea%[nopreamble=true, exno={26.}, exnoformat=X]<ex:>
\begingl 
\gla {listo eso sería}\\
\\
\glft ‘ready, that’s it’
\endgl
\xe

\ea%[everyglpreamble=\it, exno={27.}, exnoformat=X]<ex:>
\begingl 
\glpreamble \textup{l:} chapie\\
\gla chapie \\ 
\glb thanks\\ 
\glft ‘thank you’\\ 
\endgl
\xe
%
%\ea%[everyglpreamble=\it, exno={28.}, exnoformat=X]<ex:>
%\begingl 
%\glpreamble chapie\\
%\gla chapie \\ 
%\glb thanks\\ 
%\glft ‘thank you’\\ 
%\endgl
%\xe
%
%\ea%[everyglpreamble=\it, exno={29.}, exnoformat=X]<ex:>
%\begingl 
%\glpreamble ¿jaje?\\
%\gla jaje\\ 
%\glb \textsc{hort}\\ 
%\glft ‘shall we go?’\\ 
%\endgl
%\xe
%
%\ea%[everyglpreamble=\it, exno={30.}, exnoformat=X]<ex:>
%\begingl 
%\glpreamble ¡jaje!\\
%\gla jaje\\ 
%\glb \textsc{hort}\\ 
%\glft ‘let’s go!’\\ 
%\endgl
%\xe
%
%
%
