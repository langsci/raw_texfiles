%!TEX root = 3-P_Masterdokument.tex
%!TEX encoding = UTF-8 Unicode

\chapter{Introduction}

The aim of this work is to provide a detailed grammatical description of the Paunaka language (ISO 639-3: pnk, Glottocode: paun1241), a critically endangered \citep[cf.][6]{Krauss2007} \isi{Southern Arawakan} language spoken in the Chiquitania, a region in the lowlands of Eastern Bolivia, see Figure \ref{fig:MapBolivia}. 


\begin{figure}[t]
\includegraphics[width=\textwidth]{figures/Karte1.pdf}
\caption[Bolivia and the Chiquitania]{Bolivia and the Chiquitania \newline {\small (Map by Simone Faß)}}
\label{fig:MapBolivia}
\end{figure}

The data which forms the basis for this analysis was collected by me during four fieldwork trips (2011, 2012, 2015, 2018) and also by my colleagues Swintha Danielsen, Federico Villalta Rojas, and Lena Sell.


This chapter offers some general introductory information about the language, its speakers, and the research location. 
§\ref{sec:Site} starts with a description of the region and place where Paunaka is spoken. 
§\ref{sec:Speakers} offers some information about the speakers of the language 
and §\ref{sec:Fieldwork} about the fieldwork conditions, data collection and processing. 
Previous work on Paunaka is summarised in §\ref{sec:PreviousWork}. 
§\ref{sec:Name} discusses the name of the language.    
§\ref{sec:HistoricalBackground} provides a short overview about the history of the Paunaka people. 
§\ref{sec:Affiliation} is about the language’s affiliation.  
Finally, in §\ref{sec:StructureWork} the structure of this work is presented and some remarks about general decisions in selecting and representing topics and examples are made. 

\section{The site}\label{sec:Site}
Paunaka is spoken in the east of Bolivia, in a region called Chiquitania. Although this region belongs to the lowlands, it is not as flat as other parts of eastern Bolivia. Hills are a characteristic part of the landscape, as are cows grazing on the pastures.
Ecologically, the region is an intermediate zone between the humid forests of Amazonia in the north and the dry forests of the Chaco in the south. The rainy season usually lasts from November to January, while June, July, and August are very dry and rather cold months. 

As it shares many grammatical features with other Amazonian languages\is{Amazonian language} \citep[cf.][]{Aikhenvald2012}, Paunaka can be called an Amazonian language despite being spoken outside of the eco-region. The majority of its sister languages are indeed spoken in Amazonia.
The eight remaining speakers\is{consultants} of Paunaka\footnote{When I began working with the language in 2011, there were still eleven speakers. Three of them have passed away since then.} live in the small town Concepción or the nearby indigenous communities Santa Rita and San Miguelito de la Cruz, all belonging to the province Ñuflo de Chaves in the department Santa Cruz. In addition, a number of people who identify as “of Paunaka descent”  
(Span. \textit{de descendencia paunaka}) live in the indigenous villages of Monterito and Palmira, but they do not speak the language. In both Santa Rita and San Miguelito de la Cruz, Spanish and \isi{Bésiro} (ISO: cax, Glottocode: chiq1253), albeit to a lesser extent, are the main languages. Paunaka only has a low significance in communication, though not necessarily in self-identification. \citet[]{Adelaar2008} was the first to relate \isi{Bésiro} to the Jê family, and \citet[]{Nikulin2020} has proven its relation to Macro-Jê. The language is also known as Chiquitano. It is spelled \textit{bésɨro} with the letter <ɨ> in the language itself, but Spanish orthography uses <i>, a convention followed here to facilitate searchability in the PDF. Bésiro has been in contact with Paunaka for centuries and is mentioned frequently throughout this work.

The whole municipality of Concepción has approximately 20,000 inhabitants.\footnote{The census of 2012 gives the number of 18,800 inhabitants \citep[88]{INE2015}.} The town of Concepción is famous for its church, which was erected in the 18th century during the Jesuit era and restored between the 1970s and 1990s. It has been part of the \mbox{UNESCO} World Heritage site \textit{Jesuit Missions of the Chiquitos} since 1990 and regularly attracts a number of national and international tourists, especially during one of the festivals, e.g. the festival of Renaissance and Baroque music in April (where Concepción is only one of various venues) or the Orchid festival in October. 

From Concepción, it is a 30-minute ride by car to the communities. A dirt road leaves Concepción on the east side of the town and leads to Santa Rita, where it furcates. The northern road crosses Santa Rita and continues to San Miguelito de la Cruz, where it turns west, back into the direction of Concepción. The dirt road ends in a bigger asphalt road, the \textit{Carretera Hardeman – Colonia Pirai} that connects Concepción with the other towns of the Chiquitania, San Javier to the west (and further also Santa Cruz) and Santa Rosa de la Roca to the east (and further San Ignacio de Velasco). In order to go to San Miguelito de la Cruz from Concepción, taking this road takes less time and is more convenient (see Figure \ref{fig:MapConce}).

\begin{figure}[t]
\includegraphics[width=\textwidth]{figures/Karte3.pdf}
\caption[Concepción and the surrounding villages]{Concepción and the surrounding villages, where the Paunaka speakers live \newline {\small (Map by Simone Faß)}}
\label{fig:MapConce}
\end{figure}

Santa Rita is a community which had 305 people of different ethno-linguistic background belonging to 54 households in 2001 \citep[]{INE2001}.\footnote{The website of the INE that provided these numbers has changed and the numbers are no longer available. There is no information about the number of inhabitants in the communities in the newer census of 2012.} The houses have been relocated various times since its foundation in the 1950s, and the village is now centred around the school and the small football field. In former times, the families lived more separated from each other, and nowadays only one old couple still lives a bit outside of the village \citep[6]{Villalta2013}.\footnote{There was another man living outside of the village, but he passed away in 2020.} The latest innovation that substantially changed the appearance of the village was initiated by the villagers’ participation in a house-building project subsidised by the state in 2015. Standardised houses of two different models, all painted in white, were built everywhere in the village, replacing the old adobe-coloured houses of different sizes and styles that were constructed with wooden slats and traditionally thatched with palm leaves. The newly built houses have tiles on the floors, kitchens, and bathrooms with showers and water closets, so they are definitely a sign of increased comfort and prosperity in the village. Each household has had an installation for water on the site since 2013. Prior to this, water used to be extracted from the ground with the help of a big pump at the entrance of the village.

Traditionally, water was received from small wells, and a handful of elders still maintain and use their well. In the north, located at a lower part outside of the village, is a small water reservoir where people go to wash their clothes and bath.\footnote{Prior to the construction of the pump, they also fetched water for consumption there.}   Electricity is produced by a generator in the evenings for a few hours after the sun has set. During my stay in 2012, the only possibility to connect to a mobile network was to go to the highest point of the village at its entrance and try there; however, the attempt often failed. In 2015, there was a relatively stable mobile network everywhere in the village, and many people in Santa Rita now have cell phones. In addition, there is one public telephone that can also be called up. The people living next to the telephone are in charge of answering it and telling the people who have been called that they received a call. Santa Rita has a primary school, a church, and a small sale for basic food requirement such as eggs and oil.
The inhabitants of Santa Rita have their fields more or less close to the village. They grow manioc, rice, maize, plantain, peanuts, and other crops for subsistence. In addition, people hold chicken and ducks at their houses, and some have pigs, cows, or horses. Every household also has several dogs and sometimes cats. The community of Santa Rita is dedicated to forestal management, which gives them some extra income. A few elder women spin cotton, some of the younger women weave cotton scarves and hammocks, which are sold in a restaurant in Concepción. The owner of this restaurant, as well as the office of the mayor of Concepción and one NGO, organise visits to Santa Rita for tourists interested in an authentic indigenous village. On those visits, the inhabitants of Santa Rita dress in their traditional Chiquitano shirts and dresses (called \textit{tipoy}, a dress not only typical for the Chiquitania, but for the whole lowland of Bolivia), play music and dance. 
The community of Santa Rita possesses a small open lorry, and regular transport to and from Concepción was offered for a small fee that was used to maintain the lorry and pay the driver. The lorry was also used to offer individual services to the inhabitants of Santa Rita and other communities. In 2018, the lorry was out of work, because people had not been able to afford the necessary repairs. 

Settlement in San Miguelito de la Cruz still follows the more traditional lineage model, and the village has three parts corresponding to the three families that live there \citep[6]{Villalta2013}. In total, the village had 167 inhabitants belonging to 30 households in 2001 \citep[]{INE2001}. It is a little more isolated due to the fact that there is no regular transport service between San Miguelito de la Cruz and Concepción, and walking to town takes three hours for the older people. San Miguelito de la Cruz has its own water reservoir and a primary school. People also live on subsistence farming, just like they do in Santa Rita, and some women also weave scarves. My main consultants \is{consultants|(} live in Santa Rita and Concepción; thus I did not spend much time in San Miguelito de la Cruz.


\section{The speakers}\label{sec:Speakers}

When I started fieldwork in 2011, there were still eleven speakers of Paunaka. In the beginning of 2023, eight speakers remain with differing proficiencies in the language. The speakers whom I had the honour to meet and work with are (in alphabetical order, sorted by their first name): Alejo Supayabe Pinto (†), Clara Supepí Yabeta, Isidro Supepí Chijene, José Supepí Yabeta (†), Juan Cuasase Supepí, Juana Supepí Yabeta, María Cuasase Choma (†), María Supepí Yabeta, Miguel Supepí Yabeta, Pedro Pinto Supepí, and Polonia Supayabe Pinto. All speakers with the last names Supepí Yabeta are siblings. Note that \citet[167]{Villalta2022} counts a few more people who have limited, mostly passive knowledge. However, those people were never presented to me as “speakers” unlike the eleven people I mentioned above. If the people identified by \citet[]{Villalta2022} are added to the count, the number of speakers in 2011 rises to 17. In addition, a few people in other communities are remembered by the speakers to be Paunaka descendants or remember a few words \citep[195--197]{Villalta2022}. The oldest speaker was probably born in the 1920s, the youngest one in the 1950s. In addition, some children of the speakers are said to understand a few words of Paunaka; in the case of the Supepí family this is bound to their grandmother – the mother of the Supepí siblings – having spoken in Paunaka with them, while they replied in Spanish \citep[171]{Villalta2022}.\footnote{One daughter of Juana claimed to understand the language well, but she passed away in 2018 (Villalta 2019, p.c.).} I was told that the son of another speaker may even speak Paunaka a bit when he is drunk, as do a few other people who are usually not recognised as speakers. Grandchildren of the eleven speakers may know a few words or very simple phrases that they have been taught by the speakers.

All speakers are trilingual in Paunaka, Bésiro\is{Bésiro|(}, and Spanish, and none of them uses Paunaka on a regular basis. Thus, according to the classification by \citet[50]{GrinevaldBert2011}, they may all count as “semi-speakers” by the following definition: “the semi-speakers have not had and do not have regular conversation partners in the endangered language, and operate most of their sociolinguistic lives in the dominant language rather than the endangered language”. However, while they definitely lack regular conversational partners nowadays, this does not necessarily apply to the past. All siblings of the Supepí Yabeta family learned Paunaka as their first language and continued using the language with their mother until her death \citep[cf.][5, 145]{Villalta2022}. The situation is more complicated for the other speakers, some language shifts on personal level occurred, because children became orphans and were raised by their grandparents or adoptive parents \citep[133, 135, 138]{Villalta2022}, other people (partly) unlearned Paunaka because of the low prestige of the language combined with social pressure to speak Bésiro and later Spanish \citep[cf.][115]{Villalta2022}. 

In any case, lack of regular active usage of Paunaka certainly has an effect on it. Speakers were often desperately trying to remember a word in Paunaka, sometimes with success, sometimes without. Juana often wished for a relative of hers in such situations, Tiburcio, whom she remembers as a very proficient speaker. In other contexts, however, speakers may just use words of Spanish or Bésiro origin without this being a big issue. All speakers use a lot of loanwords.\is{borrowing} When two Paunaka speakers converse, they may switch back and forth between Paunaka and Spanish or Bésiro, depending on how fluent they are in each language. 

While the lexicon is certainly the domain that speakers themselves are most conscious about, we can also assume that the morphosyntax of Paunaka has changed under prolonged contact with Bésiro and Spanish.\is{Bésiro|)} In addition, there are some noticeable personal differences in the way speakers use the language. Such individual variation has been claimed to be typical for situations of language shift \citep[cf.][111--112]{PalosaariCampbell2011}. However, not having worked intensively with spoken language of a major language, I find it hard to determine how much individual variation is just “normal” and how much is due to the situation of the language. Noticeable in any case is that each speaker uses the nominal demonstratives\is{nominal demonstrative} very differently, especially my three main consultants, Juana, Miguel and María S. (see §\ref{sec:DemPron}).

I will provide a short characterisation of the five speakers I worked with most, including an assessment of the linguistic features that are characteristic for their individual speech:

Juana Supepí Yabeta, born in the 1940s, can be considered my main consultant. In 2011 and 2012, she still lived in Santa Cruz, but by 2015, she had moved back to Concepción. She has been a perfect consultant, because she is very talkative and whichever topic came up in conversation or elicitation served as a prompt to tell a story about something that once happened to her or one of her relatives. On the other hand, because she is just like this, elicitation could sometimes take long. Juana marks stressed\is{stress} syllables with a very high pitch. When listening to her speech, one could get the impression that Paunaka is a tonal language, but tone is not a distinctive feature in Paunaka. Other speakers do not produce this extremely high pitch on stressed syllables. Juana also shows some deviations regarding [a] and [ɛ], i.e. in some words, she produces an [ɛ], where other speakers use [a] without any change in meaning, e.g. [ɛpukɛ] instead of [apukɛ] for \textit{apuke} ‘ground, down’. I have certainly found a place in her heart: She treated me like another daughter of hers.

The same is true for Miguel Supepí Yabeta and his wife Lorenza Taseó, who live in Santa Rita. Miguel was also born in the 1930s. Having moved away from his family at the age of 14 approximately, Miguel had less exposure to Paunaka during his youth than some of his siblings. Sometimes, Paunaka words do not come quickly to his mind – this has become even more apparent as he got older – and he sometimes does not fully integrate borrowed verbs from Spanish like other speakers do. Nonetheless, Miguel is the best story-teller among the speakers, and he has a profound knowledge about the history of the Paunaka people in the 20th century. Of my main consultants, he is the only one who knows to read and write.

María Supepí Yabeta was born in the 1950s. She lives in Santa Rita. Since she speaks rapidly, I found it hard in the beginning to follow her. In 2018, I worked with her a lot and found out that she is the most patient and confident consultant in elicitation and able to provide explanations for certain phenomena and restrictions. More often than other speakers, she deletes thematic suffixes\is{thematic suffix} of active verbs and inserts certain other grammatical markers in this slot (e.g. the distributive and associated motion markers). Her use of the \isi{dislocative} distinguishes her speech from that of other speakers. María S. used to converse in Paunaka with Juana, when they went to their fields in former times and, thus, remained proficient in the language up to today. We have recently begun to exchange voice messages in Paunaka via WhatsApp from time to time.

María Cusase Choma, also known as María Lino by the other speakers,\footnote{Her father died when she was still a child, and the last name of her step father was Lino (Villalta 2019, p.c.). Nonetheless, she was raised by her grandmother.} was born in the 1930s and passed away in 2021. She used to live in Santa Rita in 2011, when I first met her. After her husband passed away, she moved to Concepción, where she lived with the family of her son. In 2012, she lived next to Clara’s house, which offered a good opportunity to get the two together for a conversation. María C. was the only of the Paunaka speakers more fluent in Bésiro than in Spanish. She often deleted word-initial V syllables,\is{syllable} e.g. she said \textit{sekeÿ} instead of \textit{esekeÿ} ‘bean’. This might be due to the influence of the \isi{Bésiro} genderlects.\is{genderlect}\footnote{In Bésiro, there is a difference between male and female speech, which is manifested in a variety of features \citep[cf.][]{Nikulin2019a}. Among those features is a distinction in some animal and other non-human animate nouns, where male speakers produce the word with an initial /o/ (also realised as [u]), and female speakers without the vowel, e.g. \textit{tipixh}♀︎ vs. \textit{otipixh}♂︎ ‘ant’, \textit{ɨgoj}♀︎ vs. \textit{oɨgoj}♂︎ ‘deer’, \textit{tangma’}♀︎ vs. \textit{utangma’}♂︎ ‘bird’ \citep[94--95]{Nikulin2019a}.} In addition, she also frequently omitted subject indexes\is{person marking} on verbs, especially the third-person prefix \textit{ti-}. This again may have been more common once.\footnote{In the recordings by Riester from the 1960s (see §\ref{sec:PreviousWork}), the subject indexes also seem to be dropped frequently. However, because of the low sound quality of the recordings, it is hard to make any definite statements, and the “omission” could also be a result of the microphone not having captured some unstressed syllables. In closely related \isi{Mojeño Ignaciano}, the \textit{ti-} prefix can be omitted before verb stems beginning with a consonant \citep[482]{OlzaZubiri2004}.} Unfortunately, I found it very hard to understand and transcribe María C.’s speech. First of all, she had already lost all her teeth, which made it harder to understand her acoustically, second she mixed a lot of Bésiro into her speech. While it is relatively easy for me to recognise and understand Spanish bits and chunks in Paunaka, it is almost impossible to recognise (as long as they are phonetically inconspicuous) and understand the Bésiro fragments, given that I do not speak the language.

Clara Supepí Yabeta was born in the 1950s and is the youngest speaker. She lives in Concepción and bakes bread for her living. I would have liked to work with her on a more regular basis, but she was always very busy, when I (or the other team members) wanted to visit her, so that we gave up at some point. There are a few recordings with her and María C. from 2012. Since I did not work with her on a regular basis, I cannot tell which features of her Paunaka are special.

The remaining speakers were not consulted frequently. Alejo and Polonia Supayabe Pinto both have\footnote{Or “had” in the case of Alejo. He already passed away in 2018, unfortunately.} an excellent passive command of the language but trouble speaking it actively. José Supepí Yabeta was a very good consultant for vocabulary, but did not produce any longer stretches of speech either. One reason for this may be that speaking the language painfully reminded him of his deceased parents, as he once commented, but he also lived alone for a long time, which may have contributed to the fact that he was no longer used to hearing or speaking Paunaka. José passed away in 2020. As for Juan Cuasase Supepí, I tried to visit him a few times in San Miguelito de la Cruz, but when I came there, he was usually not at home. Isidro Supepí Chijene, I met only once or twice. Pedro Pinto Supepí has (re)learned Paunaka as a second language and has a more passive than active command of it. He is the child of Paunaka parents, but was raised by adoptive parents in Bésiro.
\is{consultants|)}


\section{Fieldwork, methods, data collection}\label{sec:Fieldwork}

After having visited the Paunaka shortly in 2008, 2009, and 2010, Swintha Danielsen received a grant from the \textit{Endangered Languages Documentation Programme} (ELDP) to run the \textit{Paunaka Documentation Project} (PDP) for two years from 2011–2013. During these two years, I held a stipend as a PhD student in the project. The third principal investigator was Federico Villalta Rojas, who mainly did anthropological research. His work has been handed in as an MA thesis at PROEIB Andes, Cochabamba, in 2020 and was approved in 2022 after some revision \citep[]{Villalta2022}.

I spent six weeks in the field in August and September 2011, taking my family with me. The other two members of the core team were also present at that time, as was a friend of ours who had decided to accompany us to Bolivia. We stayed in a hostel in Concepción, from where we frequently went to Santa Rita and San Miguelito de la Cruz to visit the speakers and work with them. The reason to stay in the town rather than in one of the small villages was due to practical considerations. In Concepción, we had stable electricity, internet access at an internet post (and from 2012 on also via mobile broadband modem), mobile network, running water, and access to the hostel’s kitchen to prepare our own meals, things that we would have lacked in the villages. In addition, it would have been a major endeavour for the Paunaka to host and feed five adults and a 15-month-old child for over a month, and the villages were not difficult to reach.

In 2011, the members of the PDP conducted a workshop with the Paunaka people, where we introduced ourselves and the project, got an overview about the number of speakers, and discussed an \isi{orthography} that we could use in our transcriptions of the language. We collected words for each letter of the alphabet and invited the children present at the workshop to draw and paint illustrations of these words. Back in Germany, we produced an alphabet booklet with the collected words and illustrations of the children with the help of Julia Burda, a professional digital media designer, and printed it in Germany \citep[]{PDP2012}. The book was given to the speakers in 2012. Also present at the workshop was a member of the OICH (\textit{Organización Indígena Chiquitana} – Chiquitano Indigenous Organisation) and an anthropologist working in the Chiquitania. The OICH had major reservations against the project, because they feared that we aimed at establishing a new political movement and split the indigenous population that identifies as Chiquitano. In a conversation at the OICH office a few days later, we could, fortunately, dispel their concerns and gain their consent with the project.

Regarding the 30-minute car-ride from Concepción to the communities, we were lucky to meet a very friendly and reliable taxi driver, who not only became our favourite chauffeur, but also the person of trust for some Paunaka people to contact, whenever they needed a taxi to or from Santa Rita. As a speaker of Quechua, he even participated in one elicitation session about body parts, in which we collected words in Paunaka, Quechua, and German to everybody's enjoyment. In 2018, he had acquired a bigger car and was not longer available for these short trips. Fortunately, the people of Santa Rita had already found another trustworthy taxi driver.

I went to Santa Rita or San Miguelito de la Cruz every one to three days. I often went together with Swintha and/or Federico, and sometimes also with my husband, my daughter, and my friend. I worked a lot with Miguel in that year and with every other speaker I met.\is{consultants} Miguel introduced me to all of them. 
In addition, before returning back to Germany, I had the opportunity to meet Juana in Santa Cruz, where she used to live at that time.

In 2012, I went to Bolivia for almost seven weeks in April and May. I spent most of the time in Concepción, together with Swintha and Federico, with a short visit to Santa Cruz in the middle of my stay, where I met Juana regularly. I also worked a lot with Miguel, and with María C., Clara and María S. more sporadically.

In 2015, I spent another six weeks in Bolivia, financed with a grant from the DAAD (\textit{Deutscher Akademischer Austauschdienst} – German Academic Exchange Service). I devoted most of my time to work with Paunaka in Concepción. By that time, Juana had moved to Concepción, too, and I met her almost every day. I took some time off to visit the Guarayu speakers in Urubichá, which is where the project that I was employed in at that time had its research focus, as well as the Baure speakers in Baures, where I had done research for my M.A. thesis in 2008, and Santa Cruz for participating in a conference. I was in the fortunate situation that Swintha took her time to accompany me again.

My stay in 2018 took four weeks, with three weeks dedicated to the work with the Paunaka, and some days for travelling and attending a conference in Santa Cruz. Lena Sell, who was collecting data for her PhD, joined me during the first two weeks. Swintha stayed a few days in Concepción, too. Unfortunately, Juana was in Santa Cruz when I arrived in Concepción, and it took some time for her to get there, because there were street blockades. I worked a lot with her sister María S. and later, when Juana arrived, with both sisters, and also with Miguel.

I had originally planned to undertake another fieldwork trip to Bolivia in the first half of 2020 with the aim to focus on nominal demonstratives and complex sentences. However, I had been very busy with two jobs in 2019 and therefore not been able to prepare properly, and then the COVID-19 pandemic struck the world and made it impossible to travel. The analysis of demonstratives and complex sentences thus had to be limited to what I found in the \isi{corpus}, but I acknowledge that elicitation on these topics could certainly have clarified a few points and improved my work. This is something to continue in the future. 

My colleague Swintha Danielsen always spent much more time in Bolivia than I did, she travelled there almost every year and lived there from 2015–2021, so I was fortunate to put her in charge to do some elicitation with the speakers\is{consultants} with questions I had prepared before in 2014, 2015, 2016, and 2019 while I was in Germany. Lena Sell did the same for me in 2019 and Federico Villalta Rojas conducted a short interview with María S. on my request in 2020. Without their help, it would not have been possible to finish this work.

I recorded every session with a speaker,\is{corpus|(} most in audio with an Olympus LS11, and since 2018 with an Olympus LS-P2 digital recorder in wav format, some also with a digital video camera, and took notes in a notebook. Swintha made a lot of recordings as well, and we often also visited and recorded people together, especially if we wanted to film somebody. In 2018 and 2019, Lena S. added a few recordings to the corpus. Speakers were given a small fee for their participation in recordings.\is{consultants}
The approach was to record everything we could get hold of, turning on the recording device as soon as we met a speaker and only turning it off when we left. In a situation of advanced language shift, I think this approach is the most appropriate, since interesting linguistic constructions may pop up unexpectedly and never be repeated spontaneously. There is approximately 120 hours of audio and video material. However, due to the approach to record as much as possible, the recordings do not only include Paunaka but also Spanish conversations. The more anthropologically oriented research by Federico was conducted in Spanish exclusively, and his recordings are thus not counted in the 120 hours.

The recordings were put onto my computer, but only a part was transcribed and translated into Spanish using ELAN\footnote{Currently version 6 \citep[cf.][]{ELAN2020}.} given the vast amount of data. The transcription was done by primarily by me and secondarily by Swintha. Of the Paunaka speakers, nobody could help with this time-consuming task, unfortunately, and the younger people in the village were not interested. When I encountered unknown words and phrases, I wrote them down in order to clarify their meaning in the next session. I must admit that much of the transcription work was done back in Germany, so that the possibility to clarify passages was quite limited most of the time. For roughly 65 hours of recordings corresponding to 150 individual audio or video recordings, ELAN files were created. Many of them are only transcribed in parts. They make up the corpus that serves as the basis for the analysis presented here and includes some 50,000 transcribed words (tokens). In addition, some fieldnotes that have not been transcribed in ELAN and to a smaller part not even been recorded are considered as well. Many sessions remain that have not been transcribed at all up to today, thus there is still a wealth of unexplored data to be considered in the future. Only a very small part of the transcribed sessions were analysed in Toolbox.\footnote{See \url{https://software.sil.org/toolbox}, last checked 16-04-2021.} The elaborate search function of ELAN served extremely useful in locating examples for certain morphemes.

The content of the sessions with the speakers varied from pure elicitation, when I asked for translations of single sentences or repetitions of single words to find out about the stress patterns to spontaneous conversation between two or more speakers, but most of the time I recorded a mix of everything: a chat with the speaker in Spanish and/or Paunaka, a story, personal narrative or description of a procedure, clarification of words that I had not understood in a previous session, elicitation of sentences with grammatical constructions that I was interested in, etc. In addition, some sessions were dedicated to correct transcriptions. I used wooden and playmobil toys for elicitation of expressions for spatial relations, the \isi{frog story} by \citet[]{Mayer2003}, accompanied a speaker to his field to harvest rice, observed the baking of rice bread and cake, digging for clay to make a pot, and the production of adobe bricks. I learned a lot about the lives of my \isi{consultants}.

In addition to using the data collected by the larger team, we received approximately 20 minutes of recordings from the anthropologist Jürgen Riester, which he had made in the 1960s with a speaker\is{consultants|(} of Paunaka in Retiro (see §\ref{sec:PreviousWork}). Those recordings were transcribed with the help of two speakers, Miguel and Juana. I worked with them separately. Because of the poor audio quality of the recordings, it is often hard to understand what the speaker said, and it turned out that what Miguel and Juana understood often differed. I thus do not make much use of the data by Riester throughout this work, unless there was consent between Miguel, Juana and me about what we heard and identified. The speaker recorded by Riester was probably called Juan Choma. Riester himself introduces him as Juan Chamo on the magnetophone tape, but Juana claims that she had met the man in her youth and that his name was Juan Choma. While Choma is a common surname in the region, Chamo is unknown.\footnote{It should be noted, however, that Riester was an expert on Chiquitano culture, so that it seems strange that he would confuse a common family name with an uncommon one. Nonetheless, I follow my consultant on this issue and have memorised the speaker as Juan Choma for myself.}\is{consultants|)}\is{corpus|)}

I use the first names of the speakers\is{consultants} throughout this grammar to refer to them, with an abbreviation of their family name if two speakers have the same first name. This is not a sign of disrespect but rather of a deep feeling of familiarity with them. In addition, it would not have been practical to use the family name, given that there are five brothers and sisters among the speakers with the same family name, two pairs each whose first name starts with the same letter (M or J). The colleagues who accompanied me on my field trips are also referred to by their first name, unless mentioned in an academic context (i.e. reference to publications). In addition, every speaker received a one-letter code, which is given in Table \ref{table:Speakers} in the section on Abbreviations preceding this introductory chapter.

The materials (recordings, transcriptions, translations, etc.) were uploaded to a digital archive at ELAR\footnote{\url{https://www.elararchive.org}, last accessed 10-03-2023.} \citep[]{PDP_Archive2015}.

\section{Previous studies}\label{sec:PreviousWork}

There is almost no information about the Paunaka language in historical sources. The earliest account about the Paunaka by the Jesuit Lucas \citet[]{Matienzo_et_al2011} (see also §\ref{sec:Pre_early_colony}) does not include any linguistic material except for the name of a Paunaka village, \textit{Tesu}. However, we do not even know whether this was its name in Paunaka or in one of the other languages of the region. It does not have any meaning in the Paunaka language today.

There are a few words of Paunaka scattered in the publications by d’Orbigny (1835–1847),\ia{d’Orbigny, Alcide Dessalines} %1837, 1839, 1843, 1847 we say in LDB; d'orbigny kopiertes Buch S. 79 Paunaka
and \citet[308]{Cardus1886} offers 48 Paunaka words and phrases.

Paunaka shows up in some classifications of \isi{Arawakan languages} of the 20th century (e.g. \citealt[213]{Mason1950}; %includes Paunaka and Paikoneka. 
\citealt[142]{Loukotka1968}; %Paunaka and Paikoneka, both are given as extinct languages
\citealt[67]{Aikhenvald1999}), but without provision of any new data.

In the 1950s and 1960s, the anthropologist Jürgen Riester%\iai{Jürgen Riester}
, who worked among the Chiquitano, collected words of any indigenous language he encountered on his field trips. Among his notes are a few lists with (partly repeating) Paunaka words, summing up to roughly 200 entries. He also made three recordings of approximately 20 minutes in total with a Paunaka speaker in Retiro.

Paunaka was generally believed to be extinct until Lucrecia Villafañe %\iai{Lucrecia Villafañe}
met some speakers in 2005.\is{consultants} She collected some short texts and compiled a grammar sketch (three different versions ranging from eleven to 64 pages) and a word list with 519 entries that were never published. \citet[1]{Villafanen.d.} counted ten speakers all belonging to one family. Two of the speakers had already passed away when Swintha Danielsen undertook some short visits to the Paunaka people in 2008, 2009, and 2010, where she collected some audio material. In 2008, Danielsen also compiled a short booklet with texts taken from Villafañe’s data and a suggestion for an orthography.

In 2010, the Paunaka speaker Pedro Pinto compiled several trilingual thematic word lists for names of animals, trees, fruits, and weather phenomena, seven pages in total \citep[]{Pinto2010}. As he himself is not fluent in Paunaka, he sought the help of other speakers,\is{consultants} and the words in Bésiro outnumber the ones in Paunaka. An alphabet booklet was published in 2012 containing example words for each letter \citep[]{PDP2012}. In 2013, a poster with body part terms followed \citep[]{PDP2013}. A word list of cultivated plants was compiled by \citet[]{Sell2019} and published as a booklet.

Based on Danielsen’s and Villafañe’s data, Danielsen and I wrote a grammar sketch for the collection \textit{Lenguas de Bolivia}, which was finally published in 2014, but unfortunately presents only a preliminary analysis of the language \citep[cf.][]{DanielsenTerhart2014}. Other linguistic papers about or mentioning Paunaka by members of the PDP team include \citet[]{Danielsen2014} with a contribution of Arawakan data to SAILS, \citet[]{DanielsenTerhart2015} about borrowed clause combining patterns, \citet[]{DanielsenTerhartSubm} about reality status, \citet[]{Terhart_subm} about borrowing of verbs, and \citet[]{TerhartDanielsenBODY} about body part terminology.

Partly based on data published by \citet[]{DanielsenTerhart2014} and partly on vocabulary lists and information we offered on a personal basis, Paunaka phonology has been analysed in comparison with other Southern Arawakan languages by \citet[]{Jolkesky2016} and \citet[]{deCarvalhoPAU} to reconstruct forms of the Proto language. Paunaka is also included in the reconstruction of Proto Bolivian Arawakan (PBA) by \citet[]{RamirezFranca2019} with data that were collected by the authors themselves.

In addition, a few words or phrases of Paunaka provided personally have also been included in the analysis of Mojeño morphology by \citet[]{Rose2015a,Rose2018}.


\section{Name of the language}\label{sec:Name}
The name Paunaka has no meaning in the language itself. It may have originated as foreign appellation. The name includes the Bésiro\is{Bésiro|(} plural marker \textit{-ka}, and as a consequence, Bésiro speakers also use the term \textit{Paunáxɨ}, with the Bésiro singular suffix \textit{-xɨ}\footnote{Neither \citet[]{Sans2013}, nor \citet[]{Galeote2014} or \citet[]{Adelaar2004} provide a complete analysis of this suffix. \citet[8]{AdamHenry1880} state that it is a suffix of “absolute", i.e. unpossessed nouns, but this seems to have changed in modern Bésiro. It is certain, however, that the suffix only appears in the singular and is dropped when a diminutive or plural suffix is added to the stem. In addition, it plays a role in differentiating first and third person possessors in modern Bésiro \citep[20, 23]{Sans2013}.} to refer to one speaker, the language or the ethnic group of Paunaka \citep[cf.][9]{Villalta2013}. Paunaka people call themselves and their language \textit{Paunaka}; with \isi{stress} on the first syllable, when they speak Paunaka ([ˈpau.na.ka]), and stress on the penultimate syllable, when they speak Spanish ([pau.ˈna.ka]), according to regular stress placement in both languages. In addition, they also use the Paunaka word \textit{betea} ‘our language’ to refer to Paunaka while speaking it. In the literature, the Bésiro plural suffix is sometimes detached, and the language referred to as \textit{Pauna}.\is{Bésiro|)} It is also often written \textit{Paunaca}, following Spanish orthography. 

No other name for Paunaka is known, and the term \textit{paunacas} also figures as a designation for the ethnic group in the earliest known document that mentions the Paunaka people: Caballero’s report from 1707. %\iai{Lucas Caballero}

%As a close relation of the Paunaka language to the extinct Paikoneka (Paicone, Paiconeca) language was long proposed and the languages were often referred to jointly as Pauna-Paikone (Pauna-Paicone). This was still the case, when I started to work on Paunaka in 2011; however, by now classification has changed and a more close affiliation between Paunaka and Mojeño on the one hand and possibly Paikoneka and Baure on the other hand has been acknowledged. This topic will be taken up again in the following section. 


\section{Historical background}\label{sec:HistoricalBackground}
\is{history|(}

There is not much information about the history of the Paunaka. Most accounts are about the Chiquitano people. This term may include the Paunaka, as inhabitants of the Chiquitania, but most of the times it is used to refer to the speakers of \isi{Bésiro}, more widely known as Chiquitano (see below). 

The Paunaka were evangelised by the Jesuits in the early 18th century and resettled in mission towns together with other indigenous groups.\footnote{There is a vast number of publications about the history of the Jesuit missions in the Chiquitania, which I do not attempt to summarise here. \citet[]{TomichaCharupa2002} and \citet[]{MatienzoAL2011} offer extensive bibliographies.} Life in the missons has resulted in significant cultural changes among the indigenous people of the area \citep[7]{Tonelli2004}. The coerced coexistence of different ethnic groups led to a mixture of cultural traits, in addition to an adoption of the European religious culture of the Jesuits. The result was a new ethnic identity as \textit{Chiquitano}. Being Chiquitano means, on the one hand, having Chiquitano culture nowadays, a result of sharing the same experience or history since the time of the Jesuit missions.\footnote{According to \citet[247]{Martinez2015}, d'Orbigny was the first to equate the ethnonym \textit{Chiquitano} with the group of people inhabiting the area.}  On the other hand, the term \textit{Chiquitano} is equally often used to refer to the speakers of the Chiquitano language, who have called themselves \textit{Monkoka} and their language \textit{bésɨro} (Span. \textit{bésiro})\is{Bésiro|(} since the late 20th century \citep[cf.][1]{Sans2013}, though not in all parts of the Chiquitania \citep[cf.][478]{Adelaar2004}. Since \textit{Monkoka} and \textit{Bésiro} is used in and around Concepción, I make use of the terms throughout this work.\footnote{According to some speakers of Paunaka, the term \textit{bésɨro} was already used prior to the late 20th century. The word translates as ‘direct, correct, right’, and they perceived it as a claim of superiority over other indigenous languages, which were consequently denied the correctness of linguistic expression \citep[10]{Villalta2013}.} The ambiguity of the name Chiquitano for both a larger cultural and a smaller ethno-linguistic group makes it hard for any other ethno-linguistic group in the region to claim linguistic rights. The fact that apart from Bésiro there are other indigenous languages in the region is often either not known or, perhaps, simply ignored.\footnote{See e.g. \citet[86]{Tonelli2004}, who claims that the only remaining language in the Chiquitania is Bésiro with the exception of Ayoreo, whose speakers have a different historical background.} 
The Paunaka people identify as Chiquitano, but not as Monkoka \citep[8]{Villalta2013}. All of them indeed speak Bésiro, but they do not identify as Bésiro speakers \citep[11]{Villalta2013}.\is{Bésiro|)} Their ethnic identity may, therefore, best be captured by the term \textit{Chiquitano-Paunaka}.

The remainder of this section is dedicated to tracing back the history of the Paunaka people from pre-colonial times to now and provide a summary about the known facts.

\subsection{Pre-colony and early colonisation}\label{sec:Pre_early_colony}
It is not possible to exactly retrace the ethnic composition of the people inhabiting the Chiquitania before the colonial era – and even within the Jesuit missions~–, given that many names of groups appear in the sources without any additional information about the group. Most of them do not exist anymore. Different authors use different spellings; consequently, we cannot be sure, whether e.g. the \textit{Pisocas} that the Jesuit Caballero speaks of in his expedition report of 1707 \citep[54]{Matienzo_et_al2011} are the same as the \textit{Puizocas} that killed him in 1711 \citep[192]{MatienzoAL2011}.\footnote{Since the report by Caballero is the most important historical document regarding the Paunaka and the author of this document is known, I cite it by naming the original author as \citet[]{Matienzo_et_al2011}, while any other information taken from the historical sources compiled by \citet[]{MatienzoAL2011} gives reference to the editors. This is because some of the sources lack an identifiable date and/or author or have been composed by the editors on the basis of various versions of the documents.} In addition, we do not know much about the distinction between ethnic group and language, nor about possible different names given to identical groups.\footnote{For some recent attempts to shed a little more light on these issues see \citet[]{Nikulin2019} and \citet[]{RamirezFranca2019}.} 
It is relatively clear that people speaking languages of the Chiquito,\is{Bésiro|(}\footnote{I follow \citet[231]{TomichaCharupa2002} in using the term \textit{Chiquito} for the ethnic groups that spoke a linguistic variety related to current Bésiro prior to acculturation in the Jesuit missions in the 17th and 18th centuries. The term \textit{Chiquitano} is reserved for the group of people that have a common history in the Jesuit missions and share a common culture nowadays.} Zamucoan, Arawakan,\is{Arawakan languages} Chapacuran, Tupi-Guarani, and Bororoan linguistic families inhabited the area (cf. \citealt[255, 276-277]{TomichaCharupa2002}; \citealt[9]{Tonelli2004}), divided into a large number of different ethnic groups with 80-100 members each \citep[16]{Tonelli2004}. According to \citet[209, 363]{TomichaCharupa2002}, the speakers of different Chi\-qui\-to languages and dialects had a certain dominance in the Chiquitania due to their high number and elaborate warfare.\is{Bésiro|)}

If there are traces of the people nowadays, I have used the modern spelling of the ethnic groups including <k>, e.g. \textit{Paunaka}, \textit{Paikoneka}, \textit{Napeka}. If the groups are unknown today, I use the spelling as given in the sources, e.g. \textit{Manasica}, \textit{Puizoca}.

We owe the little information available about pre- and early colonial history of the Paunaka to the Jesuit Father Lucas Caballero, who wrote the “Diary and fourth relation of the fourth mission undertaken in the nation of the Manasicas and in the nation of the Paunacas, newly discovered, year 1707”, published by \citet[]{MatienzoAL2011}.\footnote{The complete title in Spanish reads: “Diario y cuarta relación de la cuarta misión hecha en la nación de los manasicas y en la nación de los paunacas nuevamente descubiertos, año de 1707. Con la noticia de los pueblos de las dos naciones, y se da de paso noticia de otras naciones”.}

Prior to the colonisation, the Paunaka people inhabited pampas north of today's San Javier that were rich in rivers, lagoons, and swamps. They mainly lived on fishing and poultry farming,\footnote{Caballero speaks of birds, hens, and ducks \citep[55]{Matienzo_et_al2011}. \citet[62]{Metraux1942} reports that the Mojeño people domesticated native ducks at the beginning of the 17th century, and it may well be that the Paunaka also kept native birds.} and their territory was north of the Tapacura people and west of the Manasica people, with a western to northern extension relative to the territory of the Manasica, who lived in forested areas \citep[54--56]{Matienzo_et_al2011}.\footnote{\citet[134]{Metraux1942} states that the \isi{Paikoneka} and Paunaka lived north of today's Concepción between the headwaters of Rio Blanco and Rio Verde between the meridian 62° and 61° west.} The Tapacura people were speakers of a Chapacuran language, and the Manasica most probably spoke a Chiquito dialect, but had different customs and traditions than the other Chiquito groups \citep[250]{TomichaCharupa2002}. It is also possible that the Manasica spoke a Chapacuran language. The names Manasica and Tapacura/Quimemoca\footnote{Note that \textit{Quimemoca} is the name \citet[]{Metraux1942} uses, while \citet[]{Matienzo_et_al2011} speaks of \textit{Quimomeca}.} could also have been used as synonyms by Caballero, given that he calls the Manasica a nation consisting of Tapacura and Quimemoca \citep[127]{Metraux1942}.

The many different ethnic groups in the area held mostly hostile relations to each other, if we believe Caballero’s report. The Paunaka, however, had an amicable relationship with their Tapacura neighbours \citep[55]{Matienzo_et_al2011}.

There were single entries of Spaniards in the Chiquitania in the early 16th century  without much impact on the indigenous people  \citep[cf.][25--29]{Tonelli2004}. Santa Cruz de la Sierra was founded in 1561 close to the current location of San José de Chiquitos. However, contacts between the Spanish people living in the town and indigenous people of the Chiquitania were long restricted to the ethnic groups living close to the town \citep[219--220]{TomichaCharupa2002}. A group of people called \textit{Paikono}, probably the \isi{Paikoneka},\footnote{While \textit{-ka} is the plural marker of modern \isi{Bésiro}, \textit{-ono} has the same function in the \isi{Mojeño languages}, which are closely related to Paunaka.} lived some 20 leagues away from the town. They are said to have been subjected by Ñuflo de Chaves, who entered the region in an expedition from 1557–1558 \citep[118, 121-122]{Metraux1942}. When Santa Cruz was re-located in 1621 to its current position further west, the influence of Spaniards on the indigenous population of the Chiquitania decreased \citep[48]{Tonelli2004}.%another re-location in 1595? cf. Metraux 1942:118

\hspace*{-3.2pt}The distribution of indigenous languages and groups changed completely when the Jesuits entered the region in the 17th and 18th century. Although Jesuits had been present in Santa Cruz since 1583 \citep[47]{Tonelli2004}, the establishment of mission towns only began in 1684 in the Moxos region further north \citep[47]{Tonelli2004}. In 1691, Jesuit missionaries also started to enter the Chiquitania with the aim to evangelise the indigenous people, build a path between the Jesuit mission of Tarija and the Paraguay river, %according to Tonelli 2004:71: re-open the path between Asunción and Santa Cruz la Vieja
 and secure the dominance of the Spanish colonisers over Portuguese \textit{bandeirantes}\footnote{The \textit{bandeirantes} were Portuguese fortune hunters who searched for precious metal and for indigenous people to enslave and sell them, for example to work in mines. To achieve this, they entered regions not previously colonised by Portuguese settlers and by doing this effectively enlarged the Portuguese (and later Brazilian) territory beyond the line of demarcation that divided the world into Spanish and Portuguese possession according to the treaty of Tordesillas from 1494. The \textit{bandeirantes} are also called \textit{mamelucos}.}, who entered the Chiquitania on slave-hunting expeditions  (\citealt[46]{APCOB_Saberes}; \citealt[528]{TomichaCharupa2002}; \citealt[66]{Tonelli2004}). %Because the missions offered a certain protection against slavery, some indigenous groups integrated voluntarily in the newly founded towns \citep[424--425]{Matienzo_et_al2011}.
In 1713, the Jesuit Order was put in charge of the supervision and protection of the whole Chi\-qui\-tania by the Spanish crown \citep[66]{Tonelli2004}. By 1766, the population in the ten mission towns that were founded by the Jesuits in the Chiquitania had risen to over 23,000 people \citep[77]{Tonelli2004}.%1767, 3.268 people in Concepción (Tonelli 2004:78).

The first mission town, or \textit{reduction} in the Jesuits’ terminology, was San Francisco Xavier de los Piñocas, nowadays known as San Francisco Javier or simply San Javier. It was founded by Father José de Arce among the Piñoca people and re-located several times until 1705/06. Father Lucas Caballero, who had come to the region from the Jesuit mission in Tarija, was in charge of some of the relocations \citep[529--531]{TomichaCharupa2002}.

From 1704 to 1711, Caballero undertook several journeys with the aim to evangelise the indigenous people of the region and find a place to found a new mission town \citep[536]{TomichaCharupa2002}. The fourth journey of 1707 was the one that finally led to the foundation of the two reductions Concepción and San Ignacio that consolidated later. It was also during this journey that Caballero met the Paunaka people. 

Caballero started his expedition from San Javier, accompanied by some Ma\-na\-si\-ca. He recorded the names of the different ethnic groups he came in contact with as \textit{sibacas, yurucure, quibiquicas, cosocas, moposicas, aruporecas, tapacuras, pichasicas, paunacas}, and \textit{bohococas}. We do not know exactly which languages these people spoke and how these languages were related, but we know that on an encounter in a Paunaka village between Tapacura, Pichasicas and Quimamacas, Aruporecas, and Paunaka people, they had difficulty understanding each other. Caballero describes the situation as follows:

\begin{quotation}
I greeted them, I spoke to them, and we all spoke, and nobody understood each other, because we spoke in three different languages, for convening pagans of three nations. It resembled the Babylonian confusion; I spoke two languages, but did not know the third. However, with some Paunaca words that I had written down the afternoon before, mixing in others of the Manasicas, and with the actions, I told them certain history, which frightened them and made them fear, showing well that they had understood me. \citep[85]{Matienzo_et_al2011}\footnote{Les saludé, les hablé, y todos hablábamos [51v], y nadie nos entendíamos, porque hablábamos en tres lenguas distintas, por concurrir gentiles de tres naciones. Parecía la confusión de Babilonia; yo hablaba las dos lenguas, pero no sabía la tercera. No obstante, con algunas palabras que la tarde antes había escrito de los paunacas, mezclando otras de los manasicas y las acciones, les conté cierta historia que les asustó y dio miedo, mostrando bien que me habían entendido.
(Translation: Lena Terhart)}\end{quotation}

It is noticeable that most names of ethnic groups mentioned by Caballero end in \textit{-ca} (in addition to the Spanish plural marker \textit{-s}). The Tao dialect of Chiquito had a plural marker \textit{-ca}, and it still exists in current \isi{Bésiro} with the spelling \mbox{\textit{-ka}} (cf. \citealt[21]{Sans2013}; \citealt[272]{Galeote2014}). The use of the plural marker on so many different indigenous groups shows how big the influence of Chiquito speakers was already at the beginning of the colonisation of the area.

\citet[67, 83 etc.]{Matienzo_et_al2011} reports of severe epidemics among the indigenous people encountered on his journey, especially those that he had visited and evangelised the years before. %indigenous people believe he and the cross is the reason for the illness, but C. explains it is a punishment for their failure in belief (e.g. p. 81, 86)
As a consequence, some villages were depopulated, some villages had been consolidated, and others were re-located, resulting in famine in some places \citep[77, 84, 87]{Matienzo_et_al2011}.

The Paunaka people were more numerous than the Manasica and they had many villages full of people \citep[55, 85]{Matienzo_et_al2011}.
\citet[76]{Matienzo_et_al2011} found one Paunaka village depopulated on his journey, and he was told of two villages called \textit{sepecas} and \textit{biyuricas}. The Tapacura people wanted to take him there instead of letting him enter more villages of their own ethnic group \citep[81]{Matienzo_et_al2011}.
The Paunaka village that Caballero finally visited was called \textit{Tesu} \citep[84]{Matienzo_et_al2011}. Paunaka people received the travellers on the way to the village with some food. They were lodged in a big house close to the ritual or ceremonial building, called “temple” by Caballero; despite the friendly welcome, Paunaka people had brought all children out of the village, lest the Jesuit could baptise them. His revenge was to put a cross in front of the “temple” and bash their figurines.
People of two other Paunaka villages came to invite Caballero – possibly in order to kill him as his travel companions warned him – but he decided to return to San Javier, because of the beginning rainy season, and to postpone his evangelisation expeditions to the following year \citep[85--86]{Matienzo_et_al2011}.


The map in Figure \ref{fig:Map1732}, dates from 1732, 25 years after the journey of Caballero. It shows the location of settlements of the Paunaka and some other ethnic groups, as well as the mission towns. The place of the old mission town of Concepción before its re-location (see §\ref{sec:Conce1708-1767}) can also be found on the map. Note, however, that Caballero reports that the Paunaka live north-west of the Manasica, and the Pisoca north-east of them \citep[54]{Matienzo_et_al2011}, where the map shows the Paunaka living east of the Manasica and the Picoca (probably the same Pisoca) west of them. Most of the ethnic groups mentioned by Caballero in 1707 are not shown on the map, and of the groups shown on the map, many are not mentioned by Caballero.

\begin{figure}
\centering
\includegraphics[width=\textwidth]{figures/Karte_komplett.pdf}
\caption[Map dating from 1732, which shows the location of the Paunaka]{Map dating from 1732 which shows the location of the Paunaka (\cite[]{Petroschi1732}, available at \textit{Wikimedia Commons})}
\label{fig:Map1732}
\end{figure}


\subsection{Concepción 1708–1767}\label{sec:Conce1708-1767}

The report of \citet[52]{Matienzo_et_al2011} about his fourth journey undertaken in 1707 tells about the foundation of two mission towns. Nonetheless, according to \citet[538]{TomichaCharupa2002} the first attempts to found Concepción already date to 1699, but they failed.
The two foundations refer to the "reductions" (Span. \textit{reducciones}) San Ignacio among the Boococa people, belonging to the Chiquito linguistic group, and Nuestra Señora de la Concepción among the Tapacura people. Concepción was situated at a big lagoon in an area inhabited by people of different ethnic and linguistic groups. However, it was later re-located and finally consolidated with San Ignacio, making Manasica and Boococa people live together. The final re-location took place in 1722 \citep[538--539]{TomichaCharupa2002}.

According to \citet[90]{MatienzoAL2011}, in 1707, there were 200 people living in Concepción and 330 in San Ignacio, but we do not know about the ethnic composition of the people.
According to research by \citet[284]{TomichaCharupa2002}, in the first years, there were different groups of Manasica, as well as Paunaka, Napeka, and Carabeca. %= Coraveca? Saraveka?
In 1715 and 1717, Cosirica and Cozoca people joined the mission.

Between 1730 and 1740, the ethnic groups of Puizoca (maybe the same Pisoca that Caballero had met before), \isi{Paikoneka}, \isi{Baure}, and Quiviquica were integrated into the reduction. In 1730, an expedition from Concepción with the aim to punish the Puizoca people, who had killed Caballero in 1711, led to the inclusion of 19 Paikoneka into the town \citep[166, 192-193]{MatienzoAL2011}, and in 1731, another 48 \isi{Paikoneka} and Baure were brought there \citep[206]{MatienzoAL2011}.

In 1745, speakers of Chiquito languages (Booca, Tubasi, Cusica, Purasi, Quimomeca, Yurucare, Zibaca) made up 46.9\% of the population of Concepción; Chapacuran people (Napeka, Kitemoka, Tapacura), 32.4\%; and \isi{Paikoneka} and Paunaka people, 19.2\%. The remaining 1.5\% of the people belonged to the Puizoca of unknown linguistic affiliation \citep[288]{TomichaCharupa2002}.

%- payconecas y paunacacas listed different in Matienzo:425

The missions were attractive for some groups of people, as they secured protection against the slave hunting expeditions of the Portuguese \textit{bandeirantes} from Brazil, who were in search of labourers for their mines and farms. This is why some people integrated voluntarily into the missions (\citealt[392]{TomichaCharupa2002}; \citealt[424, 425]{MatienzoAL2011}). Usually, inclusion into the missions was forced. The Jesuits – and sometimes also christianised indigenous people – frequently undertook expeditions to evangelise priorly uncontacted people and bring them to the mission towns (cf.\citealt[ch. VII]{TomichaCharupa2002}; \citealt[]{MatienzoAL2011}). The Jesuit priest Julian Knogler reports that people tended to flee if they found out that an evangelisation expedition approached them. Therefore, when they came close to a settlement, the expedition group split in order to surround the people and give no opportunity to escape. Then the priest entered the village to give a speech. If the group of people in the settlement was large enough, they occasionally fought the missionaries and their indigenous companions \citep[285]{Riester1970}. Whenever people could be captured on such expeditions, missionaries had to ensure that they would not flee or rebel on the way to the mission town:

\begin{quotation}
By the way, if we find what we look for on our journeys, that is faithless Indians, we make all efforts and joyfully return with them to the missions. On the way, being supplied with all necessary provisions, they have to be kept very kindly and affectionately, and also with great alertness and attention, lest they flee back or start an upheaval amongst the escort. \citep[Knogler 1767–1772 in][329]{Riester1970}\footnote{Übrigens an wir auf unserer Reisen finden was wür suchen, nemlich unglaubige Indianer, so geben wür alle mihe für gar wohl angewendet, und kehren mit ihnen freydig zuruk in die völ\-ker\-schafft. Auf dem weeg müssen sie mit denen nothwendigen lebensmitlen versehen, sehr gütig und liebreich gehalten werden, auch zugleich mit grosser wachbarkeit und obsorg, damit selbe nicht entweder zuruk flihen, oder unter der begleitschafft eine aufruhr anfangen. (Translation: Lena Terhart)}\end{quotation}


But despite the (initial or enduring) reservations of the indigenous people against their imposed integration into the mission towns, a new social, political, economic, and religious system was successfully established in the Chiquitania over the time \citep[65]{Tonelli2004}. The era of Jesuit rule still has a very positive image among most Bolivians nowadays.\footnote{This may be due to the fact that despite that the coercion into a paternalistic system must have been brutal, everything that came after the Jesuits’ era until the middle of the 20th century was even worse for the indigenous people.}

Everyday life of the inhabitants of the missions was fully structured with fixed times for work and religious activities like sermons and prayers \citep[509--510]{TomichaCharupa2002}. Children went to school, where they learnt to sing chorals, and sometimes even to read and write \citep[508]{TomichaCharupa2002}, and were under influence of the missionaries for most time of the day. Cattle were raised, and there were common fields administered by the church as well as private fields for subsistence of the families. Craftsmanship was taught by the Jesuits. Indigenous people worked three days a week for the church and three days a week for themselves and their families; Sunday was reserved for mass and prayer \citep[82--83]{Tonelli2004}. The contact between the indigenous people in the missions and secular Spaniards was reduced to a minimum by restricting trade to a manor close to San Javier, from where the goods were distributed among the mission towns \citep[80]{Tonelli2004}.
The towns were all built following the same schema with a big square and the church, school, and administrative buildings on one of its sides. This was the political and social heart of the mission \citep[81]{Tonelli2004}. The indigenous people were settled in different quarters by ethnic group, the quarters were called \textit{parcialidades}, a name that also came to denominate the group of people living in the quarter. The position of a \textit{parcialidad} inside a town was an indicator for its status: bigger, earlier integrated, and more trusted ethnic groups lived closer to the main square
(\citealt[49]{APCOB_Saberes}; \citealt[256]{TomichaCharupa2002}; \citealt[425]{MatienzoAL2011}).
%People were given position and titles in a hierarchy, both in religious tasks, as well as in trade \citep[82]{Tonelli2004}


In order to ease evangelisation of the indigenous people in the missions, the Jesuits chose one dominant language, the Chiquito language, as the lingua franca \citep[425]{MatienzoAL2011}. They wrote various grammars, dictionaries, and catechisms about or in the Chiquito language, composed music, wrote sermons, and translated religious texts \citep[235--238]{TomichaCharupa2002}. People of other ethno-linguistic backgrounds had to learn Chiquito, since all public religious life was communicated through this language. The choice of the linguistic variety was first at town level, this being similar to language policy in Moxos \citep[cf.][357]{Saito2009}, but a certain standardisation can be noticed. In Concepción, speakers of the Manasi dialect had been the most numerous Chiquito group in the early years of the town, but by the end of the Jesuit period, the Manasi dialect had been replaced by the Tao dialect \citep[256]{TomichaCharupa2002}. The linguistic diversity of the Chiquito varieties prior to the Jesuit missions is today reduced to two bigger dialectal groups with varieties determined by geographic factors \citep[cf.][268]{Galeote2014}.\footnote{See also \citet[]{Saito2009}, who describes the process of reducing a rich language variety to regional dialects in the Moxo missions.}
%also some dialects remained (TCh:294)
The massive presence of L2 speakers, both missionaries and indigenous people, must also doubtlessly have changed the original Chiquito varieties.

Despite the importance of the lingua franca, the practice to settle each ethnic group in a distinct \textit{parcialidad} also secured a certain continuity of different languages and cultures, at least in the private domain. The fact that more than one third of the total population of the Chiquitano missions were of non-Chiquito origin also suggests that not only the language was re-structured. Chi\-qui\-ta\-no culture and faith emerged from an interplay of Jesuit religious life with Chiquito culture, doubtlessly integrating cultural (and religious) traits of other ethno-linguistic groups as well \citep[278]{TomichaCharupa2002}.

The period of settlement in mission towns ended with the prohibition of the Jesuit Order by the Spanish crown in 1767 and their expulsion from Spanish territory. The last Jesuit missionaries left the Chiquitania in April 1768 \citep[55]{APCOB_Saberes}.


\subsection{Late 18th to 20th century}\label{sec:Century18-20}

After the expulsion of the Jesuits, the religious responsibility and administration of the mission towns was given to the diocese of Santa Cruz. The priests who were sent to the Chiquitania had the absolute power in the towns and soon started to exploit indigenous labour for their own ends. They appropriated goods and traded of agricultural products to enrich themselves. As a consequence, the priorly prospering mission towns started to decline in economic activity (\citealt[56]{APCOB_Saberes}; \citealt[93, 96, 109]{Tonelli2004}). There were also reports about sexual abuse \citep[105]{Tonelli2004}.

Complaints about the conduct of the priests were numerous \citep[cf.][ch. IV]{Tonelli2004}. In 1768, there were indigenous uprisings in Concepción and Santa Ana directed against the priests in charge \citep[102--103]{Tonelli2004}.
From 1778 on, the Chiquitania was slowly secularised, and in 1790 the separation of administration and church was codified by a governmental plan for Chiquitos. Nonetheless, economic decline continued, and the governors lined their own pockets with illicit trade \citep[106--116]{Tonelli2004}. 


\begin{sloppypar}
The first two decades of the 19th century were marked by fights between monarchists and those trying to establish independence from the crown in Bolivia \citep[cf.][ch. V]{Tonelli2004}. There were massive migrations of indigenous people between the different towns of the Chiquitania, and the area suffered a general decrease of inhabitants \citep[144, 169]{Tonelli2004}. The baptismal register in Concepción of 1800–1802 lists the following ethnic groups: \textit{bococa, napeca, yuracarica, quitemoca, paunaca, civaca, cucica, paiconeca, thapacuxaca, axupoca} \citep[9--10]{Villalta2012}.
\end{sloppypar}

In 1825, Bolivia became independent. But in spite of some attempts to grant the indigenous people more rights, their situation did not improve \citep[cf.][165--168]{Tonelli2004}. \textit{Karay}\footnote{\textit{Karay} is a Guaraní word for non-Guaraní people. It is also used by Paunaka speakers (in Spanish) for people belonging to the dominant non-indigenous society opposed to the indigenous population of Bolivia. By using the term \textit{karay} here, I try to avoid problematic terms such as ‘white people’ or ‘mestizos’, which are at best inaccurate. The Paunaka word for \textit{karay} is \textit{kayaraunu} referring to a male and \textit{senyurita} (from Span. \textit{señorita} ‘Miss’) for a female non-indigenous Bolivian person. Up to today, encounters with \textit{karay} and \textit{senyuritas} are often characterised by an inequality in social status. Both terms are thus often used in a pejorative sense by Paunaka speakers. It is not my aim to discredit all non-indigenous Bolivians, but I could not find any better, more neutral term. As long as inequal power relations between indigenous and non-indigenous people exist, neutral terms will be hard to find in general, I believe.}
 from Santa Cruz and elsewhere, encouraged by official decrees, came to the region in search of minerals and to establish big ranching and agriculture industries built on indigenous work  (\citealt[56]{APCOB_Saberes}; \citealt[172, 197, 199]{Tonelli2004}). In approximately 1850, the mission system that had managed working and living conditions of the indigenous people in the Chiquitania (e.g. division of labour into equal parts for the church/community and private ends, payment of taxes in kind etc.) was abandoned. Although this was based in the ideology to guarantee indigenous people equal rights, it laid ground for the system of \textit{em\-pa\-tro\-na\-mien\-to} introduced two decades later, in which every indigenous person was forced to have a \textit{patrón} (a \textit{karay} lord) \citep[191--192]{Tonelli2004}.

Several explorers of the 19th century report about the bad conditions and exploitations of indigenous people in the former missions:
%cite d'Orbigny Voyage tomo III: p. 48: expulsion of Jesuits and live in early 19th century

\begin{quotation}
The towns of Chiquitos, so rich before, so numerous and prospering, are now in a real misery. The industries that they had ceased to exist; some families have returned to the savage life; many are carried off by pox, others are brought to work in the rubber industry by deceit. \citep[299]{Cardus1886}\footnote{Los pueblos de Chiquitos, tan ricos antes, tan numerosos y florecientes, están ahora en una verdadera miseria. Las industrias que tenian no existen más; algunas familias se han vuelto á la vida salvaje; muchos sucumben por la viruela, otros son llevados por engaño á los trabajos de la goma. (Translation: Lena Terhart)}\end{quotation}

To be safe of the abuses of the \textit{karay} settlers, many people left the towns to live in their surroundings. Some Paunaka settled north and a little west of Concepción at the headwaters of the Río Blanco approximately 20 leagues from Concepción. Others lived in a settlement nine leagues away from Concepción, together with Napeka and people of other ethnic groups, where they raised some cattle and held contacts to other indigenous people who were more remote \citep[284--285]{Cardus1886}. The Paunaka resisted attempts to be integrated into the Franciscan missions of Guarayos; \citet[285]{Cardus1886} tells about an encounter with some people who were brought to the Ubaimini mission but disappeared after a few days.

%The population of the mission towns decreased from  89.000 in 1768 to 21.000 in 1772 \citep[92]{Tonelli2004}. -> not clear which missions: Paraguay, Chiquitania, Moxos, or all three of them?
Other indigenous people remained in the former missions; \citet[275]{Orbigny1839} %=\citep[189]{Orbigny1839_2}, see also Metraux 1942:135
estimates that 360 \isi{Paikoneka} and 250 Paunaka lived in Concepción, and 300 had fled to the woods.
  
In 1845, the languages  \textit{tapacuraca, napeca, paunaca, paiconeca, quitemoca, jurucariquia} and \textit{moncoca} were found in Concepción (Castelnau 1853 in \citealt[247]{Martinez2015}). It is noteworthy that a group called \textit{moncoca} is mentioned here for the first time. According to \citet[247]{Martinez2015}, the general language Chiquito was also called \textit{moncoca} in San Juan in 1845.
Between 1873 and 1884, among the baptised people of Concepción were \textit{napeca, moncoca, yurucariquia, quitemoca, paiconeca, paunaca}, and others that were called by the place of origin rather than by ethnic group, e.g. native of Baures \citep[9--10]{Villalta2012}. After the entries of 1884, we lose track of the \isi{Paikoneka}, who are not mentioned by any source anymore.\footnote{\citet[308]{Cardus1886} stated that \isi{Paikoneka} was spoken in Concepción, but it is unclear to me whether this was his own observation or taken from \citet[]{Orbigny1839_2}.} From 1898 on, the baptismal register of Concepción did not record the ethnic group of a person anymore; the year coincides with the first rubber boom \citep[9--10]{Villalta2012}.

The rubber boom had severe consequences for the demographic composition of Bolivia. Many indigenous people from all over the Bolivian lowlands were brought to work in the rubber industry, where they were enslaved and suffered famine and physical abuse \citep[cf.][]{Nordenskiold1923}. Life expectancy in the rubber production was of two or three years due to illness, work accidents, and attacks by wild animals like snakes and jaguars \citep[228]{Tonelli2004}.
Many indigenous people in the Chi\-qui\-ta\-nia were taken into possession by \textit{karay} settlers to work the lands that they had been granted in the vicinities of the former mission towns. A road from Santa Cruz  increased the influx of \textit{karay}. The rubber boom also increased the demand of food and stimulated the production of cattle and agricultural products on big estates like \isi{Altavista} (see below in this section). From 1874 on, all indigenous people in the Chi\-qui\-ta\-nia were incapacitated by the obligation to be under tutelage of a member of the dominant national society, a so-called \textit{patrón}. This system is called \textit{empatronamiento} \citep[57--58]{APCOB_Saberes}. 

%In 1915, the provine Ñuflo de Chávez was created with Concepción as its capital \citep[223bn ]{Tonelli2004}
In the early 20th century, a group of Paunaka lived on a territory approximately 15–20km east of today's Concepción. Some of them were hired by a man from Santa Cruz called Saturnino Saucedo to work in the construction of a road to the rubber tree habitat close to Puerto Alegre\footnote{It was impossible to identify the exact location of Puerto Alegre, but it must be somewhere north and east of Urubichá.} at the Río Negro.\footnote{Saucedo was also involved in the construction of the road from San Ignacio to Santa Cruz via Concepción and San Javier \citep[235]{Tonelli2004}.} There was food shortage and workers suffered famine. Other Paunaka went to work in the extraction of rubber (\citealt[11--12]{Villalta2012}; \citealt[1]{Villalta2013}). Saucedo used a common strategy of that time \citep[cf.][58]{APCOB_Saberes} and settled close to the Paunaka community, from where he began to take possession of the land where the Paunaka lived. He hired them to make a field for him, “paid” in goods and alcohol in advance and by doing so soon established a system of debt labour (Span. \textit{enganche}). He made the indigenous people build a big house for his family on an elevated part of the estate that he called \isi{Altavista}.\footnote{The place exists until today, but nowadays houses a centre for studies of the tropical dry forest.} In addition, Saucedo had another house in Concepción with some indigenous servants living there \citep[2]{Villalta2013}.
We know that there were some other estates, where Paunaka people used to live and work. When Riester did fieldwork among the Chiquitano in the 1960s, he also recorded some 20 minutes of Paunaka speech with a man called Juan Choma\is{consultants} (or Chamo, see §\ref{sec:Fieldwork}) on an estate called Retiro. Retiro ceased to exist, and we have very little information about the estate except for the recorded report of the speaker about his life circumstances.
In \isi{Altavista}, the indigenous people –  besides the approximately 30 Paunaka families (Villalta Rojas 2017, p.c.), there were also Napeka, Monkoka, and some Spanish-speaking Chiquitano – lived in a lower part of the estate close to a water reservoir. Every person, including the children, had a special task to fulfil for the \textit{patrón}; in addition, people worked on their own fields for subsistence on Sundays \citep[2]{Villalta2013}.
It was common at that time that the \textit{patrones} on such estates held the indigenous people in isolation from the external world in order to ease their dominance \citep[260]{Tonelli2004}.

Although, according to their own statement, the Paunaka were the most numerous group in \isi{Altavista}, the Monkoka\is{Bésiro|(} were socially dominant, manifested for example in the fact that the lead workers were Monkoka or at least spoke Bésiro \citep[2]{Villalta2013}. The Bésiro language was considered superior, and as a consequence, many Paunaka learned some Bésiro (and also Spanish) in addition to their first language.\is{Bésiro|)} 
The social relation between a \textit{patrón} and the indigenous people was of bondage and the children born on the estate automatically entered into the same relation as their parents \citep[4]{Villalta2013}.
Debt was hereditary and people could also be sold to other \textit{patrones}, who simply had to pay the debt for a person to take her into possession \citep[228]{Tonelli2004}. 
Indigenous workers were given certain goods for compensation – a portion of salt every week, soap, tobacco, some food supplies, a pair of trousers and a shirt (cf. recordings by Riester with Juan Choma in the 1960s). All items were put on a so-called bill (Span. \textit{cuenta}) for exorbitant prices that people could never pay back \citep[cf. also][261]{Tonelli2004}.
 
In the 1930s, some Paunaka people participated in the Chaco war, which lasted from 1932–1935 \citep[13]{Villalta2012}, but little is known about their fate.
One of the consequences of the Chaco war was the abandonment of the forced \textit{em\-pa\-tro\-na\-mien\-to} in 1939. The indigenous workers often simply did not return to the estates and founded their own communities (\citealt[61--62]{APCOB_Saberes}; \citealt[293--294]{Tonelli2004}). In other parts, like \isi{Altavista}, indigenous people remained on the estates, but their working conditions improved \citep[294]{Tonelli2004}. There was frequent migration between estates of different \textit{patrones}, because people were in search of the least bad living conditions. Saucedo’s son-in-law, Benigno Suárez, who took over \isi{Altavista} after Saucedo's death, was known as a clement \textit{patrón} who did not allow physical punishment, and is still remembered respectfully by the Paunaka \citep[4]{Villalta2013}.

In the 1940s, a teacher gave lessons to the indigenous children on the estate. Some Paunaka acquired basic skills in reading and writing, but the teacher forbade the children to speak indigenous languages at all under threat of punishment \citep[4]{Villalta2013}. In the 1950s, Hugo Suárez, Benigno Suárez’ son, took over the administration of \isi{Altavista}. He reintroduced physical punishment, and as a consequence, people started to leave \isi{Altavista} and settle in the vicinities to work for themselves without any \textit{patrón} \citep[5]{Villalta2013}. When the Bolivian agrarian reform of 1952/1953 prohibited the exploitation of indigenous people by debt labour, more people left \isi{Altavista}, driven away by the \textit{patrón}, as the Paunaka report, because of his fear to be obliged to pay for all the work the indigenous people had done on the estate in the previous decades.\footnote{The agrarian reform, which – among other things – introduced a fixed wage of 5 Bolivianos per day for the indigenous workers, certainly played a role in the decrease in significance of the estates and their commercial agricultural production. However, \citet[319, 332]{Tonelli2004} identifies the influx of better and cheaper products from Brazil (especially sugar and coffee) into the markets of Santa Cruz and the Chiquitania, the increased production of sugar in the area north of Santa Cruz, and a drought in the Chiquitania in the 1960s and 1970s as decisive factors.} In other parts of the Chiquitania, however, the system of \textit{empatronamiento} persisted until the 1970s \citep[300]{Tonelli2004}. Paunaka people founded the village of Santa Rita \citep[5]{Villalta2013}. Twelve families went with Hugo Suárez to the vicinity of Santa Rita to build a new estate, which was never fully established because Suárez got ill. They stayed in Santa Rita \citep[14]{Villalta2012}. 


Other Paunaka founded the village of Naranjito, which was abandoned in the 1960s, because the 1951 founded Apostolic vicariate of Ñuflo de Chávez claimed the territory for the church. Some of the people then founded San Miguelito de la Cruz, a village close to Santa Rita \citep[5]{Villalta2013}. Other people who had left \isi{Altavista} went to the big cities to try their luck \citep[14]{Villalta2012}. The dispersal of people accelerated the loss of the Paunaka language.



\subsection{Recent history and situation today}\label{sec:RecentHistory}

The villages Santa Rita and San Miguelito de la Cruz both belong to indigenous communities. In 1996, there were 314 such indigenous communities in the Chi\-qui\-ta\-nia \citep[337]{Tonelli2004}. They live in small villages in which land rights have been granted on a communal rather than personal basis (\citealt[333]{Tonelli2004}; \citealt[7]{Villalta2013}). The Paunaka people acquired title deeds in 1974 \citep[5]{Villalta2013}.
In the villages where the Paunaka live, people belonging to different ethno-linguistic groups (Paunaka, Napeka, Monkoka, and Spanish-speaking Chiquitano) established marital relationships to each other \citep[6]{Villalta2013}. In the 1970s, there was a massive work-related migration to other places, and many people never returned to the villages. This is also the time when the languages – bPaunaka, Napeka, and, to a lesser extent, Bésiro\is{Bésiro|(} - ceased to be transmitted to younger generations \citep[5--6]{Villalta2013}. Nowadays, all people in the villages speak Spanish, some speak Bésiro, and only a handful of elders still speak Paunaka.\is{Bésiro|)} The last speaker of Napeka passed away in 2011.
In the 1980s, the indigenous organisation CICC (\textit{Central Indígena de Comunidades de Concepción}) was founded to defend the rights of indigenous people \citep[7]{Villalta2013}.
The educational reform of 1994 established bilingual education in Bolivia – at least on paper. Some efforts were put into educational material in \isi{Bésiro}, but Paunaka was ignored \citep[8]{Villalta2013}.
In 2008, two Paunaka people from Santa Rita were invited to the standardisation workshop of the \isi{Bésiro} language, but they did not dare to demand linguistic rights for the language they primarily identify with because they felt shame for it. They had experienced all their life that their language was inferior to Spanish and to Bésiro and had not yet overcome the negative associations. In addition, none of the indigenous organisations of the region ever showed any interest in the Paunaka language \citep[9--10]{Villalta2013}. The combination of ignorance from official side and shame and insecurity on the side of the speakers\is{consultants} prevented the Paunaka language from being officially recognised. It was therefore not included in the new Bolivian constitution of 2009 under Evo Morales, which recognises 36 indigenous languages in addition to Spanish as official languages of the state. Paunaka kept being an “invisible language” \citep[cf.][3]{LangerHavinga2015}.
Among the people in Concepción (and elsewhere in Bolivia), the general knowledge about the Paunaka language used to be low. Many believed that the Paunaka people were a subgroup, a former \textit{parcialidad}, of Bésiro speakers and that Paunaka was a dialect of \isi{Bésiro}. This has been slowly changing during the last decade.

%\textit{Invisibility} describes a linguistic variety or variant that was not written down due to conscious or subconscious stigmatisation and hence remains invisible in a literal sense, to both contemporary speakers and future readers.
%
%\textit{Invisibilisation} is a process of implicit or explicit stigmatisation which prevents a linguistic variety from being written down.
%
%\citep[3]{LangerHavinga2015}
Some of the speakers’ children do understand Paunaka to some extent. However, their opinion about the language of their ancestors is not uniform: While some state that it would be nice to learn the language again (though practically impossible because of the workload, they say), others have a strong negative attitude towards the language and even scold their parent for speaking Paunaka \citep[10, 12-13]{Villalta2013}.

The interest of foreign linguists in their language increased the esteem of the Paunaka language among its speakers.\is{consultants} The alphabet booklet that the members of the documentation project produced as a result of a workshop with Paunaka speakers and descendants (see §\ref{sec:Fieldwork}) was received very positively, and children started to learn isolated words from the book. 
In 2011, we applied for an ISO~639-3 code for the Paunaka language at Ethnologue, and the language has received the code \textit{pnk} in 2012.\footnote{see: \url{http://www-01.sil.org/iso639-3/documentation.asp?id=pnk}, last checked 16-04-2021.} However, the language continued to be ignored by officials. Some of the Paunaka mentioned the Paunaka language as (one of) their mother tongue(s) in the official INE (\textit{Instituto Nacional de Estadística} – National Statistics Institute) census of 2012. Nevertheless, it still did not appear in any official sources following the survey. A letter of complaint including citations from the INE interviews that proved that people were speaking about Paunaka was sent to INE, signed by two members of the documentation project, two Paunaka speakers, and a teacher who worked at that time in Santa Rita and partly did the survey for the INE. It was dismissed with the response that Paunaka was not mentioned as a first language by none of the speakers; therefore no data concerning the language was obtained.
In 2018–2019, there were finally attempts from the IPELC (\textit{Instituto Plurinacional de Estudio de Lengua y Cultura} – Plurinational Institute of the Study of Language and Culture) to apply for the official recognition of Paunaka in the constitution at the Ministry of Decolonisation. This process was delayed due to problems occurring in the national elections in 2019 but taken up again in 2021.

If they succeed, an ILC (\textit{Instituto de Lengua y Cultura} – Institute of Language and Culture) will be installed for the Paunaka, thus offering the possibility to start education in and on Paunaka – at least theoretically, because with such a low number of speakers who are all elders and many of them illiterate, the establishment of education in Paunaka faces severe practical problems.
\is{history|)}

\section{Paunaka within the Arawakan family}\label{sec:Affiliation}
South America has a rich diversity of languages and language families, with up to 53 multi-member families and 58 isolates, yielding approximately 420 languages still being spoken \citep[59]{Campbell2012a}.

The Arawakan\is{Arawakan languages|(} family is the biggest language family of the continent, in geographic extension and – together with the Tupian family – also in number of languages still spoken. Arawakan languages are spoken as far north as Central America (Garífuna in Honduras, Guatemala, Nicaragua, and Belize) and as far south as Mato Grosso do Sul in Brazil (\isi{Terena}, Kinikinau), and they were formerly also present in northern Argentina and Paraguay. There are an estimated 40 Arawakan languages that are still spoken today (cf. \citealt[65]{Aikhenvald1999}; \citealt[71]{Campbell2012a}).

The name “Arawakan” itself is subject of ongoing discussion, because it was used by some scholars for a putative bigger grouping, including also languages of other families such as Arawán and Harakmbut. Therefore, the uncontroversial members of the linguistic family have sometimes been referred to as Maipur(e)an (e.g. \citealt[]{Kaufman1990}; \citealt[]{Payne1991}). In more recent publications, some scholars use the term “Arawak" to explicitly exclude candidates of a putative bigger group (e.g. \citealt{Aikhenvald1999}; \citealt[]{Michael2008}), others simply use “Arawakan” in analogy with other names of South American linguistic families, such as Tupian, Chapakuran etc., without any implications of inclusion of languages from other families (e.g. \citealt[]{Campbell2012a}; \citealt[]{Muysken_et_al2016}; \citealt[]{Hammarstroem2022}).  I follow the second approach and use the term “Arawakan” throughout this work.

A genealogical relationship between some languages of the family was already noticed in the 18th century by Father Gilij (cf. \citealt[363]{Payne1991}; \citealt[73]{Aikhenvald1999}), but the internal relations between the members of the family have only become clearer in the past 10–15 years. While both \citet[]{Aikhenvald1999} and \citet[]{Ramirez2001} still propose one major split (Northern vs. South-and-South-Western and Western vs. Eastern, respectively), the analyses based on phylogenetic methods that calculated the similarity between Arawakan languages based on lexical and grammatical features (cf.  \citealt[]{WalkerRibeiro2011}; \citealt[]{Danielsen_et_al2011}) suggests that there is little internal branching. 

In the newest proposal by \citet[3]{RamirezFranca2019}, the authors have identified twelve subgroups. Some of them are more closely related to each other, but there is no major split. They propose that there are 56 languages, 27 of them already extinct. However, some of the varieties identified as “languages” still have different sub-varieties, which have been classified as individual languages by others, often the researchers working on these varieties.

The position of Paunaka, as far as it has been mentioned at all in the classifications, is relatively uncontroversial: it belongs to the branch which was called \isi{Southern Arawakan} by \citet[]{Aikhenvald1999} and Bolivian subgroup by \citet[]{RamirezFranca2019}, although not all languages within this branch are spoken in Bolivia nowadays. The latter authors state that “[n]o branch of this family is as cohesive as the Bolivian subgroup” \citep[1]{RamirezFranca2019}. Figure \ref{fig:SouthernArawakanFamily} shows the position of Paunaka within this branch according to the analysis by \citet[3]{RamirezFranca2019}. Note, however, that except for Paunaka, all languages mentioned have subvarieties. Naming only those that are mentioned throughout this work, Mojeño\is{Mojeño languages} divides into Trinitario and Ignaciano, and Tereno into \isi{Terena} and Kinikinau, which are spoken in Brazil. Throughout this work, I use both terms, Southern Arawakan and Bolivian Arawakan, but due to the fact that I have more knowledge about the languages on the Bolivian side, the latter term excludes the Brazilian languages.\footnote{Thus, a statement that a phenomenon found in Paunaka is also found in other Bolivian Arawakan languages does not necessarily imply that this phenomenon does not exist in the other Southern Arawakan languages as well.} The branches most closely related to the Southern Arawakan languages are the Purus, Pre-Andine (or Kampan) and Pozuzo branches \citep[3]{RamirezFranca2019}.


\begin{figure}
% \includegraphics[width=\textwidth]{figures/Sprachfamilien-1}
\begin{forest} for tree={grow=east, delay={where content={}{shape=coordinate}{}}}, forked edges
  [{}
    [{Bolivian Arawakan}
        [{}
            [{}
                [{}
                    [Tereno,tier=lg]
                    [Mojeño,tier=lg]
                ]
                [Paunaka,tier=lg]
            ]
            [Baure,tier=lg]
        ]
    ]
  ]
\end{forest}

\caption{Bolivian or Southern Arawakan languages after \citet[3]{RamirezFranca2019}}
\label{fig:SouthernArawakanFamily}
\end{figure}



Before the 2020s, Paunaka was often placed together with Pai\-ko\-ne\-ka  (e.g. \citealt[67]{Kaufman2007}; \citealt[34--35]{Aikhenvald2012}) or even considered a subgroup or dialect of Paikoneka\is{Paikoneka|(} \citep[e.g. in the glossary of][438]{MatienzoAL2011}.
To my knowledge, it was \citet[188]{Orbigny1839_2}, who first proposed the connection between Paikoneka and Paunaka, which he believed belonged to one nation, although their languages differed phonologically to some extent. The words he listed for Paikoneka, however, point to a different direction. They strikingly resemble Baure.\is{Baure|(} Thus, \citet[31]{Ramirez2010} notes that Paikoneka is a variety of Baure, but without giving any evidence. \citet[]{Danielsen2013} analyses the available Paikoneka words and comes to the same conclusion. This is briefly repeated in \citet[225]{DanielsenTerhart2014} and was verified again by \citet[27]{Jolkesky2016}.

On the other hand, there is a famous quotation by \citet[160--161]{Hervas1800} that contradicts this assumption:

\begin{quotation}
I must point out that also it seems that there is an affinity between the languages \textit{baure} and \textit{paicone}, this is absolutely false and feigned. In order to dispel my doubts about this, I consulted Abbé Christobal Rodriguez, of big authority and potent memory, who has been the missionary of the \textit{baures} and \textit{paicones} that were in his protection in the mission and population of San Xavier with apostolic zeal for twenty years, and he told me that he did not find a shadow of affinity between the languages \textit{baure} and \textit{paicone}; he did not even ever hear about a single word that would be common amongst them; which is why when he began to catechise the \textit{paicones}, he had to make use of interpreters, whom he had to leave, because he found out that they betrayed him with the translation.\footnote{Debo advertir á v. que aunque se halla impreso que tienen afinidad las lenguas \textit{baure} y \textit{paicone}, v. no afirme darse tal afinidad, que es absolutamente falsa y fingida. Para salir de las dudas que yo sobre esto tenia, he consultado al señor Abate Don Christobal Rodriguez, de autoridad grande, y de portentosa memoria, el qual por veinte años, con apostólico zelo, ha sido misionero de los \textit{baures} y de los \textit{paicones}, que estaban á su cuidado en la mision y poblacion de San Xavier (20), y me ha dicho que no halló sombra alguna de afinidad entre las lenguas \textit{baure} y \textit{paicone}; ni jamas oyó decir que hubiera ni una solo palabra que fuera comun á ellas; por lo que al empezar á catequizar á los \textit{paicones}, aunque él sabia perfectamente la lengua \textit{baure}, le fué necesario valerse de intérpretes, que debió abandonar, porque halló que le burlaban con la interpretacion. (Translation: Lena Terhart)}
\end{quotation}

Thus \citet[33]{RamirezFranca2019} conclude that d’Orbigny himself may have been deceived when collecting Paikoneka words. \citet[]{Danielsen2020} found, however, an earlier citation by the very same Hervás y Panduro stating that Baure and Paikoneka are indeed related, though not mutually intelligible. There are several imaginable scenarios how Paikoneka could simultaneously be related and not be related to Baure, e.g. that two different ethnolinguistic groups were both called Paikoneka by a third group, or that part of the Paikoneka had shifted language. Due to lack of data, these scenarios remain purely speculative and the exact affiliation of Paikoneka will probably never be known.\is{Baure|)}

In San Javier, there is an indigenous organisation with the name \textit{Central Indígena Paiconeca de San Javier}, but the decision to include \textit{Paiconeca} in the name was an act of claiming indigenous identity by re-appropriation of the designation\footnote{The term \textit{paico} developed into an insult towards indigenous people \citep[104, fn. 31]{FussRiester1986}, thus re-appropriation.} and had little to do with the specific ethnic group, although it had been once present in San Javier (Villalta 2017, p.c.). Most indigenous people of today’s San Javier are descendants of people of the Piocoka \textit{parcialidad} \citep[7]{Villalta2013}.\footnote{Different ethnic groups were settled in different quarters in the Jesuit missions, each with its own internal administrative organisation. The quarters as well as the ethnic groups living there were called \textit{parcialidades}, see also §\ref{sec:Conce1708-1767}.}\is{Paikoneka|)}
\is{Arawakan languages|)}




\section{The structure of this work and general remarks}\label{sec:StructureWork}

After having provided basic information about the language, the site, the speakers and the fieldwork methods in this chapter, the remainder of this work divides into eight chapters. Chapter \ref{chap:Overview} provides a short overview of the phonology and most important grammatical features of Paunaka. Chapter \ref{chap:Phonology} describes the phonemes of the language, as well as its suprasegmental phonology and morphophonological processes. Chapter \ref{chap:Architecture} critically reviews the definitions of the notions of word, clitic and affix and explains the motivations behind what is considered a word and a grammatical marker in this grammar. In addition, it provides a short overview of word formation processes and a basic description of the parts of speech found in Paunaka. Chapter \ref{chap:MinorWordClasses} is dedicated to the minor word classes, which comprise pronouns, nominal demonstratives, adjectives, numerals and quantifiers, adverbs, prepositions, and connectives, the latter rather being a functional than a morphosyntactic category. Chapter \ref{chapter:Nouns} deals with the noun and the NP. The chapter includes a description of the composition of nouns and inflectional morphology that applies to the noun: possession, number, nominal irrealis, deceased, diminutive and locative marking. Subsequently, content and word order of the NP is considered. Chapter \ref{sec:Verbs} is about the verb and starts with a description of the composition of verb stems including several derivational patterns. Subsequently, the inflectional categories found exclusively on the verb and those primarily relating to predicates (verbal and non-verbal ones alike) are described: person marking, reality status, associated motion, middle voice, TAME, and degree markers. Chapter \ref{chap:SimpleClauses} discusses different types of simple clauses, declaratives, interrogatives and the ones expressing directives (imperative, prohibitive, hortative). It includes a description of different word orders found in Paunaka, a section on standard negation and a detailed analysis of non-verbal predication. Finally, Chapter \ref{sec:ComplexClauses} describes the Paunaka strategies of coordination and subordination of clauses. The appendix contains a narrative from Miguel, an excerpt from a conversation between María S. and Juana, and a procedural text by Juana.

It is a demanding task to document, analyse and describe a previously undescribed language, and there are certainly still very many topics to explore. This work focuses on a description of morphosyntax. A basic analysis of the phonology is also provided, but this topic could be examined in greater detail in the future. An area that was touched upon only superficially is information structure. This and the principles underlying the creation of greater portions of discourse offer an array of possibilities for future research. The grammar is not written in any particular theoretical framework, but is informed by language typology.

Emphasis in citations is maintained and reproduced in the same way as in the original source unless otherwise stated. For sake of better readability, I deleted false starts and most hesitation marks from the examples given in this work. Hesitation marks comprise \textit{¿chija?} ‘what?’, \textit{eka} ‘\textsc{dem}a’, \textit{te} ‘\textsc{seq}’, \textit{ee} ‘er’ and the like. It was not my aim to write about language contact, borrowing, code-switching or the like. As I have mentioned above, many words of Spanish origin are used in Paunaka. Consider (\getref{ex:kristianunube-1}) from Juana. Only two words of this utterance, \textit{kapunu} and \textit{tiyÿseikanube} do \textit{not} derive from Spanish.\footnote{\label{fn:yusei}In addition, the verb stem \textit{-yÿseiku} ‘buy’ is probably an old loan\is{borrowing} from Guarayu \textit{yusei} ‘want or wish sth. edible’ (Danielsen 2021, p.c.).}

\ea\label{ex:kristianunube-1}
\begingl 
\glpreamble i pente repentekena kapunu pasaunube kristianunube repente tiyÿseikanube\\
\gla i pente repente-kena kapunu pasau-nube kristianu-nube repente ti-yÿseika-nube\\
\glb and maybe maybe-\textsc{uncert} come pass-\textsc{pl} person-\textsc{pl} maybe 3i-buy.\textsc{irr}-\textsc{pl}\\ 
\glft ‘and maybe, maybe people come passing by and maybe they buy (some drinks)’\\
\endgl
\trailingcitation{[jxx-e110923l-2.11]}
\xe

I do not want to claim that (\getref{ex:kristianunube-1}) is a typical Paunaka sentence, but it is a normal Paunaka sentence, just one sentence among others, which can also contain more or even exclusively words of Paunaka origin. I did not exclude examples like (\getref{ex:kristianunube-1}) from the analysis and I did not try to find the “purest” Paunaka. Words with Spanish (and to a lesser extent Bésiro) origin are adapted to the \isi{orthography} used in this grammar and as far as they are phonetically adapted this is – of course – also reflected in the way the word is spelled.

One peculiarity of this work is that most examples are introduced by briefly providing the extralinguistic context. This is usually not done in grammatical descriptions. I started with this at a moment when I felt that context was necessary for understanding and then extended it further and further. Thus, the reader will not only learn about Paunaka, but also gain knowledge about the narratives and personal life stories of the speakers throughout this work. This work can thus hopefully not only contribute to visibilise the Paunaka language but also to visibilise its speakers. As a concession to the readers who are only interested in grammar, as far as extralinguistic context is not necessary for the explanation of certain grammatical features this information comes last in the text that introduces an example. I tried to be as consistent as possible with this order. I offer an array of different examples in this work, but some of them repeat. Nonetheless, numbering is consecutive, i.e. repeated examples receive a new number.

