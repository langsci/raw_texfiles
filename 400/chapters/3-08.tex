%!TEX root = 3-P_Masterdokument.tex
%!TEX encoding = UTF-8 Unicode

\chapter{Simple clauses}\label{chap:SimpleClauses}

This chapter is about different kinds of simple clauses. With “simple clauses” I refer to main clauses that only contain a single \isi{finite verb} or non-verbal predicate.\is{non-verbal predication} Clause combination and complex predicates are described in Chapter \ref{sec:ComplexClauses}. The simple clauses described here relate to different types of speech acts: declarative, directive and interrogative. 

Declarative clauses are the topic of \sectref{sec:SimpleClauses} and \sectref{sec:NonVerbalPredication}, the difference being that the first is about those with verbal predicates and the latter about the ones with non-verbal predicates. Both sections include a discussion on the expression of arguments, word order and negation: in \sectref{sec:SimpleClauses} these topics form subsections, but in \sectref{sec:NonVerbalPredication} they are integrated into different subsections, which are ordered by semantic type. \sectref{sec:Imperative} is about imperatives and other directives, such as the prohibitive and hortative. \sectref{sec:Questions} deals with different types of interrogative clauses.

\section{Declarative clauses with verbal predicates}\label{sec:SimpleClauses}\is{declarative clause|(}

The verbal declarative clause minimally consists of an inflected verb.\is{finite verb} Core arguments\is{argument|(} are indexed on the verb,\is{person marking} except for third person objects\is{object} which are not always marked.\footnote{Remember that object markers are mainly reserved for SAP objects. Third person objects can be indexed on the verb by using the third person marker \textit{chÿ-}, which encodes 3>3 relationships, as opposed to \textit{ti-}, which only marks the third person \isi{subject}. Third person object indexing by using \textit{chÿ-} is obligatory with human objects\is{animacy} and optional with non-human objects. See \sectref{sec:NumberPersonVerbs} for more information on person marking.} NPs\is{noun phrase|(} can co-occur, i.e. they are conominal.\is{conomination|(} \sectref{sec:ExpressionSubjects} provides information on the expression of subjects and \sectref{sec:ExpressionObjects} on the expression of objects. There is no flagging of core arguments on nouns. Oblique NPs can be marked by the locative marker or prepositions; this is the topic of \sectref{sec:DeclClausesOBL}. Typically only one core argument is conominated, either a \isi{subject} or an \isi{object}, and this argument usually follows the verb. The most basic word orders\is{word order|(} are thus V, VS and VO. Obliques mostly follow the verb, and also the S or O argument, as far as it is conominated.

The preverbal slot is associated with highlighting. Both \isi{subject} and \isi{object} NPs can occur in this slot as well as obliques.\is{oblique}\is{noun phrase|)} We thus also find SV and OV orders. 

%\emph{TO DO: check preposed obliques: with subordination only? -> nein!}

In the rare cases that the \isi{subject} as well as the \isi{object} is conominated, the most frequent word orders are VOS and SVO. VSO is also possible; OVS, however, is highly exceptional.\is{word order|)} Information on possible word orders is provided in \sectref{sec:WordOrder}. It would certainly be worth to examine information and discourse structure of Paunaka to learn more about the conditions that trigger conomination of  participants and their position in the clause. I have some preliminary thoughts on this issue that I share in this section, but did not undertake a full analysis.

\subsection{Expression of subjects}\label{sec:ExpressionSubjects}\is{alignment|(}
\is{subject|(}

%contrast: chiejiku eka kabe tikutijikutu i eka janejane cheikukuiku, mox-a110920l-2 104

Subjects are obligatorily indexed on verbs by person markers\is{person marking|(} preceding the verb stem. The markers in the first and second person are the same for active intransitive, stative intransitive and transitive verbs. (\ref{ex:active-a}) has an active intransitive verb, (\ref{ex:stative-a}) has a stative intransitive verb and (\ref{ex:transitive-a}) has a transitive verb with a second person subject. In all of these examples, the  subject is marked on the verb with the index \textit{pi-}. For more examples see \sectref{sec:1_2Marking}.

With (\ref{ex:active-a}), Clara expressed her surprise that I bathed in the reservoir of Concepción by repetition of my statement. Some people are afraid of the reservoir, because there are piranhas there.

\ea\label{ex:active-a}
\begingl 
\glpreamble pikubu\\
\gla pi-kubu\\ 
\glb 2\textsc{sg}-bathe\\ 
\glft ‘you took a bath’
\trailingcitation{[cux-c120414ls-1.223]}
\xe

%\ea\label{ex:active-a}
%\begingl 
%\glpreamble piniku\\
%\gla pi-niku\\ 
%\glb 2\textsc{sg}-eat\\ 
%\glft ‘you eat’\\ 
%\endgl
%\trailingcitation{[jxx-a120516l-a.477]}
%\xe

Prior to (\ref{ex:stative-a}), María C. had asked me whether I was not sad because of being in Bolivia without my family. I answered her that I was only a bit sad, and she repeated the statement as follows:

\ea\label{ex:stative-a}
\begingl 
\glpreamble sepitÿjiku pichÿnumi\\
\gla sepitÿ-jiku pi-chÿnumi\\ 
\glb small-\textsc{lim}1 2\textsc{sg}-be.sad\\ 
\glft ‘you are a little sad’
\trailingcitation{[uxx-e120427l.052]}
\xe


(\ref{ex:transitive-a}) was directed to me, when Juana invited me to have a freezie. Freezies come in small plastic bags, which one can open by biting a little hole in one corner.

\ea\label{ex:transitive-a}
\begingl
\glpreamble aa nechÿu pinijabaka naka\\
\gla aa nechÿu pi-nijabaka naka\\
\glb \textsc{intj} \textsc{dem}c 2\textsc{sg}-bite.\textsc{irr} here\\
\glft ‘ah, there you can bite it (open), here’
\endgl
\trailingcitation{[jxx-e110923l-2.103]}
\xe

The third person subject marker \textit{ti-} occurs on \isi{intransitive} verbs and on \isi{transitive} verbs with SAP objects,\is{object} as well as with non-human\is{animacy} non-emphasised objects (see \sectref{sec:3Marking}). In order to mark 3>3 relationships, a subject/object marker \textit{chÿ-} is used. If reference is sufficiently clear, no subject NP needs to co-occur, as in (\ref{ex:make-chicha}), in which the subject referents, Juana’s grandparents being on their way to Moxos, are well established. The sentence describes what her grandparents did when they rested.

\ea\label{ex:make-chicha}
\begingl
\glpreamble tiyÿtipajikanube\\
\gla ti-yÿtipajika-nube\\
\glb 3i-make.chicha.\textsc{irr}-\textsc{pl}\\
\glft ‘they would make chicha’
\endgl
\trailingcitation{[jxx-p151016l-2.057]}
\xe
\is{person marking|)}

%\ea\label{ex:tired-rest}
%\begingl
%\glpreamble tikubiakubunube teumichunube\\
%\gla ti-kubiakubu-nube ti-eumichu-nube\\
%\glb 3i-be.tired-\textsc{pl} 3i-rest-\textsc{pl}\\
%\glft ‘when they got tired, they rested’\\
%\endgl
%\trailingcitation{[jxx-p151016l-2.025]}
%\xe

Subject NPs\is{noun phrase} can co-occur with the person indexes, but they are by no means required syntactically, i.e. they are conominals \citep[cf.][217]{Haspelmath2013}. They are never case-marked.

In (\ref{ex:piti-subj}), the second person pronoun conominates the second person index on the verb. It was produced by Juana, repeating my statement that it was me who went to visit Miguel, not him who visited me.

\ea\label{ex:piti-subj}
\begingl
\glpreamble aa piti piyunu nauku chubiuyae\\
\gla aa piti pi-yunu nauku chÿ-ubiu-yae\\
\glb \textsc{intj} 2\textsc{sg.prn} 2\textsc{sg}-go there 3-house-\textsc{loc}\\
\glft ‘ah, you went there to his house’
\endgl
\trailingcitation{[jxx-e110923l-1.028]}
\xe

In (\ref{ex:fish-subj}), the NP conominating the third person index includes a noun and a demonstrative. This sentence was elicited from Clara, when Swintha wanted to make a statement about the dried piranha we found at the shore of the reservoir in Concepción.

\ea\label{ex:fish-subj}
\begingl
\glpreamble teijuku echÿu jimu\\
\gla ti-eijuku echÿu jimu\\
\glb 3i-stink \textsc{dem}b fish\\
\glft ‘the fish stinks’
\endgl
\trailingcitation{[cux-c120414ls-2.111]}
\xe

% tipurtujaneu ÿbajane = se champaron los chanchos (al agua), jrx-c151001fls-9.63
\is{subject|)}

\subsection{Expression of objects}\label{sec:ExpressionObjects}
\is{object|(}

First and second person objects are obligatorily indexed on the \isi{verb}.\is{person marking|(} Object indexes follow the verb stem.\is{verbal stem} This is true for \isi{transitive} verbs, as in (\ref{ex:OBJ-foll}) and (\ref{ex:smoke-OBJ-smoke}), as well as \isi{ditransitive} verbs, as in (\ref{ex:give-OBJ}) and (\ref{ex:buy-for-you}). If the verb is \isi{ditransitive}, the indexed object has the semantic role of a \isi{recipient}.

The verb in (\ref{ex:OBJ-foll}) has a second person singular object marker (\textit{-pi}). Juana cites her own words here, repeating what she had said to her brother the day before. %For more examples with SAP object markers see \sectref{sec:1_2Marking}.

\ea\label{ex:OBJ-foll}
\begingl 
\glpreamble “nikichupapi tajaitu”\\
\gla ni-kichupa-pi tajaitu\\ 
\glb 1\textsc{sg}-wait.\textsc{irr}-2\textsc{sg} tomorrow\\ 
\glft ‘“I will expect you tomorrow”’
\trailingcitation{[jxx-p120430l-1.127]}
\xe

In (\ref{ex:smoke-OBJ-smoke}), María S. states that smoking is bad. The verb \textit{-kupaku} ‘kill’ carries the first person plural object marker \textit{-bi}.

\ea\label{ex:smoke-OBJ-smoke}
\begingl
\glpreamble tikupakabi\\
\gla ti-kupaka-bi\\
\glb 3i-kill.\textsc{irr}-1\textsc{pl}\\
\glft ‘it (smoking) can kill us’
\endgl
\trailingcitation{[rxx-e120511l.385]}
\xe

%In (\ref{ex:god}), María C. expresses her Christian believe in our origin. The verb carries the first person plural object marker \textit{-bi}.
%
%\ea\label{ex:god}
%\begingl
%\glpreamble bia, chibu tetupaikubi naka apuke\\
%\gla bia chibu ti-etupaiku-bi naka apuke\\
%\glb God 3\textsc{top} 3i-put.down-1\textsc{pl} here ground\\
%\glft ‘God, he put us here into the world’\\
%\endgl
%\trailingcitation{[uxx-p110825l.111]}
%\xe

One example with a \isi{ditransitive} verb is (\ref{ex:give-OBJ}). It was produced by Juana in telling her brother what happened to the photo that Swintha had given her the day before. First, she had been telling this incident in Spanish, but repeated it in Paunaka on request. The verb in this example carries the first person singular object marker \textit{-ne}.

\ea\label{ex:give-OBJ}
\begingl 
\glpreamble ukuine tipunakune chifotone\\
\gla ukuine ti-punaku-ne chi-foto-ne\\ 
\glb yesterday 3i-give-1\textsc{sg} 3-photo-\textsc{possd}\\ 
\glft ‘yesterday she gave me her photo’
\trailingcitation{[jmx-e090727s.041]}
\xe

(\ref{ex:buy-for-you}) was elicited in order to tell Clara that Federico bought something for her. The verb carries the second person singular marker \textit{-bi}.\footnote{The second person singular object marker has two allomorphs: \textit{-bi} is found after default/realis\is{realis} marking with /u/, \textit{-pi} (as in (\ref{ex:OBJ-foll}) above) after irrealis-marked\is{irrealis} morphemes in /a/.}


\ea\label{ex:buy-for-you}
\begingl
\glpreamble chiyÿseikinubi\\
\gla chi-yÿseik-inu-bi\\
\glb 3-buy-\textsc{ben}-2\textsc{sg}\\
\glft ‘he bought it for you’
\endgl
\trailingcitation{[cxx-e120410ls-2.006]}
\xe
\is{person marking|)}

Personal pronouns never conominate object indexes.\is{personal pronoun} In order to put more emphasis on an SAP object, a person-marked form of the preposition \textit{-tÿpi}\is{general oblique} can co-occur; however, this is very rare. The very same preposition is also used in the expression of some kinds of oblique objects, see \sectref{sec:DeclClausesOBL} below (and see also \sectref{sec:borrowed_verbs} for oblique objects in non-verbal predication).
One example in which \textit{-tÿpi} is used as a conominal expression for an object is given in (\ref{ex:OBJ-plusBEN}). The verb takes the person marker \textit{-ne} for the first person singular object. This marker is obligatory and cannot be omitted. The oblique preposition conominating the object follows the verb. The sentence comes from Miguel and is about Swintha not having told him the exact date of her return to Concepción and Santa Rita (after going back to Germany).

\ea\label{ex:OBJ-plusBEN}
\begingl 
\glpreamble kuinakuÿ tikechane nitÿpi\\
\gla kuina-kuÿ ti-kecha-ne ni-tÿpi\\ 
\glb \textsc{neg}-\textsc{incmp} 3i-say.\textsc{irr}-1\textsc{sg} 1\textsc{sg}-\textsc{obl}\\ 
\glft ‘she hasn’t told me, yet’
\trailingcitation{[mxx-d110813s-2.052]}
\xe

%chuji- chujikunube telefenoyae tÿpi nijinepÿi o tÿpi nisinepÿinube, rxx-e181022le

Third person objects\is{person marking|(} are usually not indexed by a marker which follows the stem in declarative clauses (but see \sectref{sec:3_suffixes} for exceptions). The third person marker \textit{chÿ-}/\textit{chi-} can be used instead to express 3>3 relations with human\is{animacy} objects and with non-human objects that the speaker finds worth being explicitly marked (see detailed discussion in \sectref{sec:3Marking}). The \isi{plural} marker \textit{-nube} and the \isi{distributive} marker \textit{-jane} can be added to verbs to express plurality of human and non-human objects,\is{animacy} but since the same markers can also express plurality of third person subjects, the issue of which third person participant is \isi{plural} is not easily sorted out (see \sectref{sec:Verbs_3PL}).

(\ref{ex:bible}) has a human third person object which is expressed solely by use of the marker \textit{chÿ-}. All participants are sufficiently established in discourse by the preceding sentences (in Spanish), and therefore no NP needs to co-occur. The example stems from Juana telling Swintha about the creation of people and some animals and plants. It is interesting how the biblical creation story mixes with elements of non-Christian origin. Prior to this sentence, Juana had narrated that God formed María Eva out of mud as a future wife for Jesus, who did not want to marry a pigeon.

\ea\label{ex:bible}
\begingl
\glpreamble chetuku nauku nekupai\\
\gla chÿ-etuku nauku nekupai\\
\glb 3-put there outside\\
\glft ‘he (God) put her (María Eva) there outside’
\endgl
\trailingcitation{[jxx-n101013s-1.359]}
\xe
\is{person marking|)}

Non-human\is{animacy} third person objects are frequently expressed by NPs.\is{noun phrase} There is no case marking on the noun or any other constituent of the object NP. In many cases, there is no specific index on the verb either to cross-reference the object. This is the case in (\ref{ex:objNPfoll-1}), which was produced by Juana, when telling me how she raised her brother, feeding him with plantain. When he grew a bit older, he could eat some food. The verb carries the third person marker \textit{ti-}, i.e. only the subject is indexed here. 


\ea\label{ex:objNPfoll-1}
\begingl 
\glpreamble tinikumÿnÿ yÿtÿuku\\
\gla ti-niku-mÿnÿ yÿtÿuku\\ 
\glb 3i-eat-\textsc{dim} food\\ 
\glft ‘he ate some food’
\trailingcitation{[jxx-p120430l-2.486]}
\xe

(\ref{ex:OBJ-follow-2}) is from the story about the lazy man. Before he goes to the wood, he prepares his machete, being supposed to make a field to grow food for his family. The object is indexed on the verb in this case, by making use of the third person marker \textit{chÿ-}. The conominal NP follows the verb.

\ea\label{ex:OBJ-follow-2}
\begingl 
\glpreamble chajÿikutuji chitÿmuepane\\
\gla chÿ-ajÿiku-tu-ji chi-tÿmuepa-ne\\
\glb 3-sharpen-\textsc{iam}-\textsc{rprt} 3-knife-\textsc{possd}\\ 
\glft ‘he sharpened his machete, it is said’
\trailingcitation{[mox-n110920l.021]}
\xe

There are very few \isi{ditransitive} verbs in the corpus, (\ref{ex:OBJ-ditr}) offers one example. In this case, both third person objects, \isi{recipient} and theme,\is{patient/theme} are expressed by NPs,\is{noun phrase} the first containing only the demonstrative \textit{eka}, the second a demonstrative + noun. This sentence was produced by Juana, when we were discussing that her little grandson could or should learn Paunaka, since he showed interest in the language. While I insisted on it being necessary that Juana talks with him in Paunaka, she proposed the idea that he could learn it with the help of written material, referring to a sheet with some words and phrases in Paunaka that Swintha had handed over to Juana.

\ea\label{ex:OBJ-ditr}
\begingl
\glpreamble eka nipunaka echÿu ajumerku\\
\gla eka ni-punaka echÿu ajumerku\\
\glb \textsc{dem}a 1\textsc{sg}-give.\textsc{irr} \textsc{dem}b paper\\
\glft ‘this one, I will give him the paper’
\endgl
\trailingcitation{[jxx-e110923l-1.102]}
\xe

\is{object|)}
\is{alignment|)}
\is{conomination|)}


\subsection{Expression of obliques}\label{sec:DeclClausesOBL}
\is{oblique|(}

In this work, obliques are defined as per \citet[]{wals-84} as nominal constituents that modify a verb or clause. According to this definition, obliques are adjuncts, but the issue is not totally clear for Paunaka. Obliques are never required syntactically by any verb. They are neither indexed on the verb in verbal declarative clauses nor obligatorily expressed by an NP (or PP). Nonetheless, a few verbs highly favour the overt expression of an oblique, first and foremost the motion verb\is{motion predicate} \textit{-yunu} ‘go’, as exemplified in (\ref{ex:new23-OBL}) from María C., but also \textit{-etuku} ‘put’ and to a lesser extent \textit{-kuetea} ‘tell’. This is because these verbs semantically require a goal or \isi{addressee}. It could thus be argued that the obliques of these verbs are (optional) arguments.

\ea\label{ex:new23-OBL}
\begingl
\glpreamble tiyunu kampoyae\\
\gla ti-yunu kampo-yae\\
\glb 3i-go countryside-\textsc{loc}\\
\glft ‘she went to the countryside’
\endgl
\trailingcitation{[cux-c120510l-1.205]}
\xe

% = he went back home, mox-n110920l.035

It has been argued that the distinction between arguments and adjuncts is possibly not a crosslinguistic but a language-particular one \citep[]{Haspelmath2014}. Concerning Paunaka, there is no general difference in the expression of those kinds of obliques that are semantically entailed and other constituents that seem to be completely optional by the semantics of the verb, i.e. those that clearly qualify as adjuncts, and I do not know of any test that would unambiguously set apart arguments from adjuncts in Paunaka. This is why I decided not to distinguish them in the analysis for the time being.

Thus, obliques comprise spatial, temporal, benefactive,\is{beneficiary} instrumental, cause,\is{instrument/cause} and \isi{comitative} relations, and depending on the specific kind of relation, they can take the general locative marker\is{locative marker|(} \textit{-yae} (see \sectref{sec:Locative}), or they can be marked by a \isi{preposition} (see \sectref{sec:Adpositions}). Source\is{source} expressions are formed with the help of a preposition and can additionally take the locative marker. Obliques may also be completely unmarked, this is what we frequently find in the expression of goals in combination with the motion verb\is{motion predicate} \textit{-yunu} ‘go’.\footnote{This could actually be seen as a criterion for argument status. Whether or not a locative marker shows up seems to largely depend on the kind of \isi{noun} used as goal expression: toponyms\is{toponym} are likely to occur without locative marker, which does not come as a surprise \citep[cf.][291]{StolzAL2014}. In addition, the place nouns\is{place noun} \textit{uneku} ‘town’ and \textit{asaneti} ‘field’ often show up without locative marking in Paunaka, though the possessed form \textit{-sane} ‘field’ rather takes a locative marker. It remains to be checked how these nouns behave when combined with verbs other than \textit{-yunu} ‘go’.}\is{locative marker|)}
%Some goal expressions may also be formed with tÿpi + -yae!

I will start the description with pronominal expressions of obliques. There is an oblique \isi{topic pronoun} \textit{nebu} found with locative and temporal relations (see also \sectref{sec:FocPron}). It is often combined with deranked verbs, which is analysed here as indicating a cleft construction, a topic described in \sectref{sec:Clefts}. It may, however, also occur in declarative clause. \textit{Nebu} always precedes the verb.

(\ref{ex:nebu-obl-1}) is formed with the oblique pronoun \textit{nebu}. It comes from Juana’s account about her daughter who went to Spain but was deported for not having a valid visa. She was arrested together with other people and brought to a room, where they received some food. The room (or rather its location upstairs, on another floor) has been mentioned directly before.

\ea\label{ex:nebu-obl-1}
\begingl
\glpreamble nebu chupununube yÿtÿuku\\
\gla nebu chÿ-upunu-nube yÿtÿuku\\
\glb 3\textsc{obl.top.prn} 3-bring-\textsc{pl} food\\
\glft ‘there they brought (them) food’
\endgl
\trailingcitation{[jxx-p110923l-1.314]}
\xe

The context of the next example, which also contains \textit{nebu}, is as follows: María S. had described the quarter where her daughters live in Santa Cruz. I had been in this quarter once with Miguel to visit his daughter, who also lives there, so I told María S. that I knew it. She replied with (\ref{ex:nebu-obl-2}), in which \textit{nebu} refers to the quarter we had been talking about.

\ea\label{ex:nebu-obl-2}
\begingl
\glpreamble ja no ve nebu chubu nijinepÿinube\\
\gla ja {no ve} nebu chÿ-ubu ni-jinepÿi-nube\\
\glb \textsc{afm} {right} 3\textsc{obl.top.prn} 3-be 1\textsc{sg}-daughter-\textsc{pl}\\
\glft ‘ah, you know? there live my daughters’
\endgl
\trailingcitation{[rxx-e120511l.256]}
\xe

Locations and goals can be expressed by adding the locative marker\is{locative marker|(} \textit{-yae} to a noun, as in (\ref{ex:obl-yae-1}), which was produced by Miguel in telling me about the history of Santa Rita and his own personal history. After living in Naranjito for some time, he went to Santa Cruz and only came back to live in the Chiquitania again after 20 years.

\ea\label{ex:obl-yae-1}
\begingl
\glpreamble niyunu Santa Kruyae\\
\gla ni-yunu {Santa Kru}-yae\\
\glb 1\textsc{sg}-go {Santa Cruz}-\textsc{loc}\\
\glft ‘I went to Santa Cruz’
\endgl
\trailingcitation{[mxx-p110825l.074]}
\xe

(\ref{ex:obl-yae-2}) comes from a description by Juana of how to cook with a clay pot. 

\ea\label{ex:obl-yae-2}
\begingl
\glpreamble pijÿuka petukatu yÿkÿyae\\
\gla pi-jÿuka pi-etuka-tu yÿkÿ-yae\\
\glb 2\textsc{sg}-light.fire.\textsc{irr} 2\textsc{sg}-put.\textsc{irr}-\textsc{iam} fire-\textsc{loc}\\
\glft ‘you light fire, then you put it onto the fire’
\endgl
\trailingcitation{[jmx-d110918ls-1.009]}
\xe

While both locative-marked NPs in (\ref{ex:obl-yae-1}) and (\ref{ex:obl-yae-2}) above express goals, in the following example we find \textit{-yae} on a noun that expresses a static location. The example comes from Miguel’s telling of the story of the lazybones, who only swings on a liana (like in a hammock) and plays the flute in the woods instead of working.

\ea\label{ex:obl-yae-3}
\begingl
\glpreamble tebibiku echÿu jupipiyae\\
\gla ti-ebibiku echÿu jupipi-yae\\
\glb 3i-swing \textsc{dem}b liana.sp-\textsc{loc}\\
\glft ‘he swung on the liana’
\endgl
\trailingcitation{[mox-n110920l.067]}
\xe

In (\ref{ex:obl-unm}) we have a goal expression without locative marker. It was produced by Juana, who still lived in Santa Cruz at that time. She spoke about Federico.

\ea\label{ex:obl-unm}
\begingl
\glpreamble eka semana tiyuna Santa Rita\\
\gla eka semana tiyuna {Santa Rita}\\
\glb \textsc{dem}a week 3i-go.\textsc{irr} {Santa Rita}\\
\glft ‘this week he will go to Santa Rita’
\endgl
\trailingcitation{[jxx-p110923l-1.098]}
\xe

Another example of an unmarked oblique is (\ref{ex:obl-unm-2}), which is a commentary by Juana, when Miguel was telling the story about the fox and the jaguarundi. The story reaches its climax, the fox is drunk, the jaguarundi has fled to the woods, the dogs of the owner of the house they had broken in chase them. Apparently, Juana expects that the jaguarundi meets the fox again in the woods, which she expresses by this sentence. Miguel, however, goes on telling the story without such an encounter. It is possible that the adverb \textit{nauku} ‘there’ which is preposed to the locative NP has an influence on the omission of locative marking here, but it is not necessarily the case that \textit{-yae} is omitted if \textit{nauku} is preposed. Compare (\ref{ex:piti-subj}) above.

\ea\label{ex:obl-unm-2}
\begingl
\glpreamble titupu nauku kimenu\\
\gla ti-tupu nauku kimenu\\
\glb 3i-find there woods\\
\glft ‘he met him there in the woods’
\endgl
\trailingcitation{[jmx-n120429ls-x5.412]}
\xe

%depue Krara tiyunutu uneku, Kuana tikubipu uneku, depue tepajÿkunubetu uneku, rxx-p181101l-2.263-264
%
%tisukuejikuji chinabakÿyae = se cagó en su boca
%nisachu biyuna bisemaikupa takÿra nauku bibÿkupa chubiyaeyae = entramos en la casa
%titupunubuji kimenukÿyae = he arrived in the woods, mox-n110920l.025


%Non-exact locations can also be expressed by an NP with the preposition \textit{tÿpi}. -> no generelly only time, but look for: Turuxhi tÿpi Conce; tÿpi Cochabamba
Source\is{source|(} expressions are introduced by \textit{tukiu} ‘from’. The noun often takes the locative marker in this case, but not necessarily so.
(\ref{ex:obl-tukiu-1}) is an example which has both \textit{tukiu} and a locative-marked noun. It comes from the story about the creation of the world told by Juana. The main character is a very strong young man in this part of the story, which explains why specific trees and animals have specifically shaped (body) parts. In case of the silk floss tree, this is because it had swallowed all the crops that were meant by God for the people and animals to eat. The animals try to get back their food but fail to pull the silk floss tree out of the water, where it grows. Finally the strong young man helps them and they succeed.

\ea\label{ex:obl-tukiu-1}
\begingl
\glpreamble tukiu ÿneyae chetukunube echÿu yuke\\
\gla tukiu ÿne-yae chÿ-etuku-nube echÿu yuke\\
\glb from water-\textsc{loc} 3-put-\textsc{pl} \textsc{dem}b riverbank\\
\glft ‘from the water they put it (the silk floss tree) onto the riverbank’
\endgl
\trailingcitation{[jxx-n101013s-1.784]}
\xe

Another example of a source expression with \textit{tukiu} and the locative marker is (\ref{ex:obl-tukiu-2}). Miguel produced this sentence when speaking with Juan C. about their past. Miguel had talked about the load of work they had to do in \isi{Altavista} and Juan C. had just stated that they searched for another place to live, which is then presented as the reason for their moving away from Altavista by Miguel. 

\ea\label{ex:obl-tukiu-2}
\begingl
\glpreamble nechikue bibÿbÿsu tukiu Turuxhiyae\\
\gla nechikue bi-bÿbÿsu tukiu Turuxhi-yae\\
\glb therefore 1\textsc{pl}-come from Altavista-\textsc{loc}\\
\glft ‘therefore we came from Altavista’
\endgl
\trailingcitation{[mqx-p110826l.018]}
\xe

One example in which the source does not carry the locative marker is (\ref{ex:obl-tukiu-3}). It was produced by Miguel in speaking about Swintha.

\ea\label{ex:obl-tukiu-3}
\begingl
\glpreamble kapunu, titupunubu tukiu Alemania\\
\gla kapunu ti-tupunubu tukiu Alemania\\
\glb come 3i-arrive from Germany\\
\glft ‘she came, she arrived from Germany’
\endgl
\trailingcitation{[mxx-d110813s-2.028]}
\xe
\is{source|)}
\is{locative marker|)}

Temporal expressions can be introduced by \textit{tÿpi}. This can be seen in (\ref{ex:obl-tÿpi-1}), which was produced by Juan C., when he and Miguel were discussing the possibility of some rain in August.

\ea\label{ex:obl-tÿpi-1}
\begingl
\glpreamble tÿpi Santa Rosa repente tikeba pario\\
\gla tÿpi {Santa Rosa} repente ti-keba pario\\
\glb \textsc{obl} {Saint Rosa} maybe 3i-rain.\textsc{irr} some\\
\glft ‘around Saint Rosa(’s day) maybe it rains a bit’
\endgl
\trailingcitation{[mqx-p110826l.627]}
\xe

\textit{Tÿpi} can also be used to express periods of time, as in (\ref{ex:obl-tÿpi-2}), in which Juana talks about her plans to travel to Spain.

\ea\label{ex:obl-tÿpi-2}
\begingl
\glpreamble nauku niyuna tÿpi treschÿ kuje\\
\gla nauku ni-yuna tÿpi treschÿ kuje\\
\glb there 1\textsc{sg}-go.\textsc{irr} \textsc{obl} three month\\
\glft ‘I will go there for three months’
\endgl
\trailingcitation{[jxx-p110923l-1.260-261]}
\xe

In addition to that, \textit{tÿpi} can be used to encode benefectives\is{beneficiary} or recipients.\is{recipient} When \textit{tÿpi} is used in such a way, it can also occur without an NP and in these cases, it takes a person marker\is{person marking} as in (\ref{ex:obl-tÿpi-3}), which was elicited from Juana.

\ea\label{ex:obl-tÿpi-3}
\begingl
\glpreamble nikujemu chitÿpi\\
\gla ni-kujemu chi-tÿpi\\
\glb 1\textsc{sg}-be.angry 3-\textsc{obl}\\
\glft ‘I am angry with him’
\endgl
\trailingcitation{[jxx-e190210s-01]}
\xe

(\ref{ex:obl-tÿpi-4}) was produced by Miguel and directed towards María C. to tell her that we were leaving an invitation for a workshop on Paunaka with her, which the PDP team organised in 2011.

 \ea\label{ex:obl-tÿpi-4}
\begingl
\glpreamble binejika eka ajumerku pitÿpi\\
\gla bi-nejika eka ajumerku pi-tÿpi\\
\glb 1\textsc{pl}-leave.\textsc{irr} \textsc{dem}a paper 2\textsc{sg}-\textsc{obl}\\
\glft ‘we will leave this paper with you’
\endgl
\trailingcitation{[mux-c110810l.011]}
\xe

\textit{Tÿpi} can introduce purpose clauses (see \sectref{sec:PurposeClauses}), and it is also found together with NPs that express the aim,\is{aim/result} \is{purpose} or result of an action.

(\ref{ex:obl-tÿpi-6}) is from the creation story told by Juana. God has called María Eva in order to tell her to make linen for clothes, after she and Jesus had eaten the forbidden apple. The object of the clause is \textit{riensu} ‘linen’ and the PP that follows explains, what the linen is meant for, \textit{tÿpi pimÿuna} ‘for your future clothes’, the clothes people are supposed to wear from that point on.

\ea\label{ex:obl-tÿpi-6}
\begingl
\glpreamble jaje bana riensu tÿpi pimÿuna\\
\gla jaje bi-ana riensu tÿpi pi-mÿu-ina\\
\glb \textsc{hort} 1\textsc{pl}-make.\textsc{irr} linen \textsc{obl} 2\textsc{sg}-clothes-\textsc{irr.nv}\\
\glft ‘let’s make linen for your future clothes’
\endgl
\trailingcitation{[jxx-n101013s-1.501]}
\xe

The kinds of obliques described above, i.e. the locative-marked ones and the ones with the prepositions \textit{tukiu}\is{source} and \textit{(-)tÿpi}\is{general oblique} are quite common. More infrequently, we find also obliques with the semantic roles of instruments, causes\is{instrument/cause|(} or comitatives\is{comitative}. Instruments and causes are formed with the preposition \textit{-keuchi} and comitatives with \textit{-aj(i)echubu}. The first of those prepositions, \textit{-keuchi} is usually person-marked,\is{person marking|(} regardless of whether an NP follows. The \isi{comitative} preposition \textit{-aj(i)echubu} is always person-marked.\is{person marking|)}

In (\ref{ex:obl-keuchi-1}) Juana tells me about how she went fishing with her grandmother. The women fish with nets, while men fish with hooks. If the net caught a fish, they would take this fish out and kill it with a stick. The stick is marked as the instrument used for killing by \textit{-keuchi}.

\ea\label{ex:obl-keuchi-1}
\begingl
\glpreamble kue tituika bikupaka chikeuchi yÿkÿke\\
\gla kue ti-tuika bi-kupaka chi-keuchi yÿkÿke\\
\glb if 3i-hunt.\textsc{irr} 1\textsc{pl}-kill.\textsc{irr} 3-\textsc{ins} stick\\
\glft ‘if it (the net) caught (fish), we would kill them with a stick’
\endgl
\trailingcitation{[jxx-p120430l-1.073]}
\xe

An example in which \textit{-keuchi} marks a cause is (\ref{ex:obl-keuchi-2}), which comes from Miguel telling Alejo the \isi{frog story}. He produced this sentence when looking at the picture on which the beehive lies on the ground because the dog had jumped against it and made it fall. In this case no NP follows \textit{chikeuchi}, since it is sufficiently clear from the context who is responsible. 

\ea\label{ex:obl-keuchi-2}
\begingl
\glpreamble tibÿtupaikubutu chikeuchi\\
\gla ti-bÿtupaikubu-tu chi-keuchi\\
\glb 3i-fall-\textsc{iam} 3-\textsc{ins}\\
\glft ‘it (the beehive) fell down because of it (the dog)’
\endgl
\trailingcitation{[mtx-a110906l.104]}
\xe
\is{instrument/cause|)}

One example of \textit{-aj(i)echubu} is given in (\ref{ex:obl-com}). Clara answered María C.’s question about where my daughter was. I had taken her and my husband with me when I first came to Bolivia to work with the Paunaka people in 2011, but in 2012, I went alone.

\ea\label{ex:obl-com}
\begingl
\glpreamble chinejiku chajichubu chÿa\\
\gla chi-nejiku chÿ-ajechubu chÿ-a\\
\glb 3-leave 3-\textsc{com} 3-father\\
\glft ‘she left her (her daughter) with her father’
\endgl
\trailingcitation{[cux-120410ls.081]}
\xe

\is{oblique|)}
\is{argument|)}

In the following section, the information given up to here is brought together and different possible word orders are presented.

\subsection{Word order}\label{sec:WordOrder}
\is{word order|(}
\is{conomination|(}

Paunaka has a wide range of possible word orders regarding nominal expressions of arguments: VS, VO, SV, OV, VOS, VSO, SVO. If we consider the argument indexes on the verb, however, the order is rigid: subject indexes always precede the verb stem,\is{verbal stem} i.e. the order is s-V for \isi{intransitive} verbs. First and second person object indexes always follow the verb stem, yielding s-V-o. Third person objects are either indexed by a marker that encodes 3>3 relationships on verbs or remain unmarked, thus we have s+o-V or s-V argument orders on verbs with third person objects.\is{person marking}

It is common that only one core argument is conominated and the most basic word orders can thus be considered VS and VO. It is also very common that a clause contains nothing but a verb if subject and, if applicable, object participants are well-established in discourse. V and VS sentences are mainly found with \isi{intransitive} verbs and are related to topic\is{topic|(} continuity, topic establishment and topic change. VO order is typical for \isi{transitive} verbs. This is connected to the subject being an established topic\is{topic|)} and the object providing new information, i.e. having the role of \isi{focus} \citep[cf.][]{Lambrecht1994}.

For convenience, the word order types of the discussed sentences are placed above the examples in this section. Word orders of material that is not relevant for the discussion is given in parenthesis; this is usually other juxtaposed sentences that I did not want to omit, because intonation suggested that they closely belonged to the sentence I want to discuss or because I believe they are indicative for information structure.

Let us start with two examples that contain nothing more than the inflected verb. (\ref{ex:orderV-1}) represents the answer Juana gave to my (stammered) question after her relation to her grandchildren. The grandchildren are established as participants by my question, so they do not need to be conominated by an NP. They are encoded by the plural marker on the verb. The first person subject participant is expressed by the index preceding the verb stem.

\ea\label{ex:orderV-1}
\begingl
\glpreamble \textup{V:}\\nichaneikunube\\
\gla ni-chaneiku-nube\\
\glb 1\textsc{sg}-care.for-\textsc{pl}\\
\glft ‘I care for them’
\endgl
\trailingcitation{[jxx-p110923l-1.161]}
\xe
%event-reporting

Prior to (\ref{ex:orderV-2}), I had made a statement about José being the only one who stayed in the place where the whole family Supepí Yabeta used to live together. Thus, it is clear that reference is made to José, when María S. states that he is alone.

\ea\label{ex:orderV-2}
\begingl
\glpreamble \textup{V:}\\ tipÿsisikubu\\
\gla ti-pÿsisikubu\\
\glb 3i-be.alone\\
\glft ‘he is alone’
\endgl
\trailingcitation{[rxx-e120511l.187]}
\xe
%José is topical

If only one NP accompanies the verb, its unmarked position is that following the verb. However, this is not true for NPs containing personal or topic pronouns.\is{pronoun} These pronouns always precede the verb, see below. First, I give some examples with NPs containing nouns that follow the verb. (\ref{ex:SUBJ-follow-3}) and (\ref{ex:SUBJ-follow}) have VS and (\ref{ex:VO-1}) and (\ref{ex:VO-2}) VO order.

(\ref{ex:SUBJ-follow-3}) is an example with a subject following the verb. Although the subject participant of this clause, the jaguar, \textit{isini}, has been talked about by María S. in the previous clause, she decided to express it by an NP here, maybe because this is the highlight and also a kind of summary towards the end of the story of the fox and the jaguar.

\ea\label{ex:SUBJ-follow-3}
\begingl 
\glpreamble \textup{VS:}\\tipakutu isini\\
\gla ti-paku-tu isini\\ 
\glb 3i-die-\textsc{iam} jaguar\\
\glft ‘the jaguar died’
\trailingcitation{[rxx-n120511l-1.040]}
\xe
%full example: tepakutu isini tijÿchÿichieku ÿne = the jaguar died, he drowned in the water -> deleted because not sure whether -e- in the verb can be considered an applicative for ÿne!

(\ref{ex:SUBJ-follow}) is another example with a subject NP following the verb. It is from the description of the \isi{frog story} by Juana. She had already mentioned the dog in the preceding clause, she even mentioned the very same event of the dog’s running. Repetition of the subject NP has two functions here. First, it creates more emphasis on the whole sentence, because Juana found it funny, and second, this sentence also provides a summary of what she had been telling before about the dog.

\ea\label{ex:SUBJ-follow}
\begingl 
\glpreamble \textup{VS:}\\tikutikubutu kabe\\
\gla ti-kutikubu-tu kabe\\ 
\glb 3i-run-\textsc{iam} dog\\ 
\glft ‘the dog is running’
\trailingcitation{[jxx-a120516l-a.146]}
\xe

Objects\is{object} also frequently follow the verb. In (\ref{ex:VO-1}), the subject of the sentence, two men who go into the woods to hunt, is a well-established topic and does not need to be conominated. The object of this sentence, the collared peccaries that the men hunt, is new information and thus expressed by an NP. This sentence is part of the story about the two men who meet the devil in the woods that was told by María S.

\ea\label{ex:VO-1}
\begingl
\glpreamble \textup{VO:}\\tituikunubeji tijapÿ\\
\gla ti-tuiku-nube-ji tijapÿ\\
\glb 3i-hunt-\textsc{pl}-\textsc{rprt} collared.peccary\\
\glft ‘they hunted collared peccary, it is said’
\endgl
\trailingcitation{[rxx-n120511l-2.17]}
\xe

The next example comes from the same tale, but this time told by Miguel. The men have already met the devil and one of them gives him some of the meat. But the devil, still being hungry has demanded the heads of the pigs (since Miguel uses \textit{ÿba} ‘pig’ instead of \textit{tijapÿ} ‘collared peccary’ in his story). Thus the man gives him the heads in (\ref{ex:VO-2}). The verb carries the person marker \textit{chÿ-}. This person marker refers to the subject, the man, and to the object, the heads. Additionally, the object is expressed by the NP \textit{echÿu chichÿtijane} ‘their heads’, which follows the verb. The heads have been mentioned in the previous clauses, first as an object conominated by an NP, then as the subject of an existential clause without being overtly expressed. Both of these preceding clauses are formed as direct speech in the narrative. It may be the switch from subject of the existential clause to object of the verbal clause or the switch from direct speech to report that triggered use of an NP here, or – of course – both factors may have an influence.


\ea\label{ex:VO-2}
\begingl
\glpreamble \textup{VO:}\\chupunukuji echÿu chichÿtijane\\
\gla chÿ-upunu-uku-ji echÿu chi-chÿti-jane\\
\glb 3-bring-\textsc{add}-\textsc{rprt} \textsc{dem}b 3-head-\textsc{distr}\\
\glft ‘he also brought their heads (of the pigs)’
\endgl
\trailingcitation{[mxx-n101017s-1.050]}
\xe

There is one preverbal slot, which is used to indicate special discourse status. Subject\is{subject} as well as object\is{object} NPs can occur in this slot giving rise to SV and – more rarely – OV orders. 

NPs that precede the verb can have either topic\is{topic|(} or focus\is{focus|(} status.  As for topical NPs, the reason why they occur pre-verbally can be an indication of a change of topic\is{topicalisation} or re-activation of a non-active topic. However, not every change or re-activation of topic\is{topic|)} goes along with preverbal NP placement, and more research on information structure is certainly necessary to determine the exact conditions under which an NP can be preposed. Focus NPs may be preposed to indicate argument focus, i.e. the relation of the preposed NP to the rest of the proposition is new information or this information is highlighted \citep[cf.][228]{Lambrecht1994}

Personal and topic pronouns\is{pronoun} are only used for special emphasis and they always precede the verb,\is{focus|)} see (\ref{ex:Pron-prec}) and (\ref{ex:PRN-SV}) for a first person plural and a first person singular pronoun, respectively. Note that there is no personal pronoun for the third person, but the demonstrative\is{nominal demonstrative|(} \textit{echÿu} or the \isi{topic pronoun} \textit{chibu} can be used instead. The demonstrative can precede or follow the verb. (\ref{ex:echÿu-pron}) is an example with a demonstrative that accompanies a verb marked for a third person subject by the person marker \textit{ti-}, (\ref{ex:chibuchibu}) has a third person subject conominated by the topic pronoun \textit{chibu}.\is{nominal demonstrative|)}  %they usually do not replace the person marker on the verb unlike in other Arawakan languages like Nanti \citep[342]{Michael2008}

(\ref{ex:Pron-prec}) is narrated direct speech in the story about the man who loses the cows of his \textit{patrón}, finds them with a spirit and gets enchanted by that spirit. Towards the end of the story, he brings the cows to a village for the people there to eat. This is what he tells the people, before he leaves them to go back to the place of the spirit again. 

\ea\label{ex:Pron-prec}
\begingl 
\glpreamble \textup{SV:}\\ “biti biyunupunatu”\\
\gla biti bi-yunupuna-tu\\ 
\glb 1\textsc{pl.prn} 1\textsc{pl}-go.back.\textsc{irr}-\textsc{iam}\\ 
\glft ‘“we go back now”’
\trailingcitation{[mxx-n151017l-1.92]}
\xe

\largerpage
(\ref{ex:PRN-SV}) was Miguel’s answer to Swintha’s question what he was doing while his wife was making rice bread. Actually, his answer that he was doing nothing but watching was a joke. He was only distracted at that very moment, answering our questions. 

\ea\label{ex:PRN-SV}
\begingl
\glpreamble \textup{SV:}\\nÿti nimumukujiku\\
\gla nÿti ni-imumuku-jiku\\
\glb 1\textsc{sg.prn} 1\textsc{sg}-look-\textsc{lim}1\\
\glft ‘I am only watching’
\endgl
\trailingcitation{[mxx-e120415ls.071]}
\xe


%(\ref{ex:SUBJ-PRON}) comes from an elicitation session. It was offered by María S. as another example with the verb \textit{-yÿbamukeiku} ‘peel (in mortar)’. 
%
%SV[VO]
%\ea\label{ex:SUBJ-PRON}
%\begingl 
%\glpreamble nÿti nisachu niyubamukeika arusu\\
%\gla nÿti ni-sachu ni-yÿbamukeika arusu\\ 
%\glb I 1\textsc{sg}-want 1\textsc{sg}-peel.in.mortar.\textsc{irr} rice\\ 
%\glft ‘I want to peel rice in mortar’\\ 
%\endgl
%\trailingcitation{[rxx-e141230s.201]}
%\xe

(\ref{ex:echÿu-pron}) comes from Miguel telling José the \isi{frog story}. This sentence is a repetition of his last utterance, which had the same verb, but the subject was expressed by the noun \textit{peÿ} ‘frog’ and the place from where it left, the big glass, was also expressed by an NP. Since the frog and the glass are thus sufficiently established in discourse, it is not necessary to repeat them again. Note that it is not uncommon to repeat propositions, this is usually done, when a discourse topic or a specific section about this topic comes to an end. Repetition also plays a role in back-chanelling.


\ea\label{ex:echÿu-pron}
\begingl 
\glpreamble \textup{VS:}\\mm, tibÿchÿutu echÿu\\
\gla mm ti-bÿchÿu-tu echÿu\\ 
\glb \textsc{intj} 3i-leave-\textsc{iam} \textsc{dem}b\\ 
\glft ‘mm, it left’
\trailingcitation{[mox-a110920l-2.026]}
\xe

In (\ref{ex:chibuchibu}), María C. expresses her faith in God. She uses the topic pronoun \textit{chibu} to refer back to God, who had been mentioned before by using an NP.

\ea\label{ex:chibuchibu}
\begingl
\glpreamble \textup{SV:}\\chibu tetupaikubi naka apuke\\
\gla chibu ti-etu-pai-ku-bi naka apuke\\
\glb 3\textsc{top.prn} 3i-put-\textsc{clf:}ground-\textsc{th}1-1\textsc{pl} here ground\\
\glft ‘he put us here on earth’
\endgl
\trailingcitation{[uxx-p110825l.111]}
\xe


Subject NPs containing a noun can also precede the verb for special emphasis, e.g. indicating contrast or a change of topic,\is{topic|(} but probably also for stylistic reasons.  Emphasis is not per se excluded if the subject follows the verb.
%New participants can be introduced into the discourse by using subject NPs, although this is mostly done either with object NPs or with subject NPs in presentational clauses including the non-verbal predicates \textit{kaku} ‘exist’ or \textit{kapunu} ‘come’. One example of introduction of a new participant is (\ref{ex:late-father-bb}), which is from Juana telling how her grandparents were deprived of their cows by \textit{karay}. Her father found out, where the cows were brought. He had not been mentioned before; however, the referent is clearly identifiable, since ‘father’ is, of course, an unambiguous kinship relation in most cases. The subject precedes the verb in this case.
%
%
%\ea\label{ex:late-father-bb}
%\begingl
%\glpreamble  \textup{SV, (non-verbal PRED S, VXO):}\\i eka nÿabane tanÿma tiyunu kapunumÿnÿ eka tenekubu jamuike baka\\
%\gla i eka nÿ-a-bane tanÿma ti-yunu kapunu-mÿnÿ eka ti-eneku-bu jamuike baka\\
%\glb and \textsc{dem}a 1\textsc{sg}-father-\textsc{rem} now 3i-go come-\textsc{dim} \textsc{dem}a 3i-leave-\textsc{mid} pampa cow\\
%\glft ‘and my late father went, he came there, the cows were left in the pampa’\\
%\endgl
%\trailingcitation{[jxx-e150925l-1.237]}
%\xe


% tipurtujaneu ÿbajane = se champaron los chanchos (al agua), jrx-c151001fls-9.63

%Another example is (\ref{ex:SUBJ-follow-2})
%
%\ea\label{ex:SUBJ-follow-2}
%\begingl 
%\glpreamble i tiyunu nijinepÿi chumu chipatrune\\
%\gla i tiyunu nijinepÿi chumu chipatrune\\ 
%\glb \\ 
%\glft \\ 
%\endgl
%\trailingcitation{[jxx-p110923l-1.328]}
%\xe

The next example, (\ref{ex:SUBJ-prec}), is a case in which the subject precedes the verb for contrastive topic. It is taken from the same recording as (\ref{ex:SUBJ-follow}). Previously, Juana talked about the dog, now she provides information about the boy, \textit{aitubuche}. This change triggered the use of a subject NP preceding the verb.

\ea\label{ex:SUBJ-prec}
\begingl 
\glpreamble  \textup{SV:}\\i eka aitubuche tipÿtapaikubutu\\
\gla i eka aitubuche ti-bÿtupaikubu-tu\\ 
\glb and \textsc{dem}a boy 3i-fall-\textsc{iam}\\ 
\glft ‘and the boy fell down’
\trailingcitation{[jxx-a120516l-a.148]}
\xe
\is{topic|)}

An example of a contrastive \isi{focus} subject NP in preverbal position is (\ref{ex:SUBJ-prec-foc}), where Juana and I were discussing the consumption of frogs. It is not usual among the speakers of Paunaka to eat frogs, but Juana had heard that some species are tasty and that there are people who eat them, and she gives information about the presumed nationalities of frog eaters in the sentence.


\ea\label{ex:SUBJ-prec-foc}
\begingl 
\glpreamble  \textup{SV, SV:}\\eka japonesnube tinikunube los chino tinikunube\\
\gla eka japones-nube ti-niku-nube {los chino} ti-niku-nube\\ 
\glb \textsc{dem}a Japanese-\textsc{pl} 3i-eat-\textsc{pl} {the Chinese} 3i-eat-\textsc{pl}\\ 
\glft ‘the Japanese eat them (frogs), the Chinese eat them’
\trailingcitation{[jxx-a120516l-a.482]}
\xe

%I chipujunepaikutu kupisaÿrÿ tiyunuji eka kupisaÿrÿ tiyayaumiji = y el zorro lo empujó (al tigre)  se fue el zorro contento, jmx-n120429ls-x5.278


OV order is less common than SV. It is used if speakers want to put special emphasis on the object. This is the case when speakers use the \isi{topic pronoun} \textit{chibu} as an object, as in (\ref{ex:chibu-OV-2}) from the same passage as (\ref{ex:objNPfoll-1}) above. Juana had just described that she fed her brother with plantain, then she states:

\ea\label{ex:chibu-OV-2}
\begingl
\glpreamble  \textup{OV:}\\chibu bekichumÿnÿ\\
\gla chibu bi-ekichu-mÿnÿ\\
\glb 3\textsc{top.prn} 1\textsc{pl}-invite-\textsc{dim}\\
\glft ‘this we gave (lit.: invited) him’
\endgl
\trailingcitation{[jxx-p120430l-2.482]}
\xe

(\ref{ex:chibu-OV}) comes from an elicitation context with María S. However, it was not requested as a direct translation, but rather originated from the elicited context (which included eating, conversing and going). As in the previous example, \textit{chibu} is the object of the clause and it precedes the verb.

\ea\label{ex:chibu-OV}
\begingl
\glpreamble  \textup{OV:}\\chibu bichujijikubu\\
\gla chibu bi-chujijiku-bu\\
\glb 3\textsc{top.prn} 1\textsc{pl}-talk-\textsc{mid}\\
\glft ‘this we talked about’
\endgl
\trailingcitation{[rxx-e181020le]}
\xe


Another example in which an object is emphasised and thus placed in preverbal position is given in (\getfullref{ex:tomato-OV.3}). It is a confirmation to my surprised reaction (\getfullref{ex:tomato-OV.2}) to Juana’s statement that frogs are prepared with tomatoes (\getfullref{ex:tomato-OV.1}). 

\ea\label{ex:tomato-OV}
  \ea\label{ex:tomato-OV.1}
 \begingl 
\glpreamble  \textup{(non-verbal PRED),VO:}\\\textup{j:} michaniki, tetuku tomate\\
\gla michaniki ti-etuku tomate\\ 
\glb tasty 3i-put tomato\\ 
\glft ‘it is tasty, they put tomatoes in (the cans with the frogs)’
  \ex\label{ex:tomato-OV.2}
 \begingl 
\glpreamble \textup{l:} ¡aa, tomate!\\
\gla aa tomate\\ 
\glb \textsc{intj} tomato\\ 
\glft ‘ah, tomatoes!’
  \ex\label{ex:tomato-OV.3}
 \begingl
\glpreamble  \textup{OV:}\\\textup{j:} ja, tomate tetuku\\
\gla ja tomate ti-etuku\\
\glb \textsc{afm} tomato 3i-put\\
\glft ‘yes, tomatoes they put in’
\endgl
\trailingcitation{[jxx-a120516l-a.468-470]}
\z
\xe

%echÿu timuÿji aparte chetukunube, jxx-p15016l-2.132 ebd. 206
%esekeÿ chetukunube jxx-p120430l-2.034
%kÿike tiyubajika betuicha pujukeke = maní molía y lo metemos al mote, rxx-p181101l-2.204

If both subject and object are conominated, we mostly find VOS or SVO order. There are also some examples of VSO in the corpus. As for OVS, however, it was only found in elicitation contexts, though it was not rejected as being completely wrong, given that the relations between the participants were sufficiently clear. I will provide examples for all orders. (\ref{ex:VOS-1}) and (\ref{ex:VOS}) have VOS order, (\ref{ex:SVO}) and (\ref{ex:SVO-2}) are examples for SVO, (\ref{ex:VSO}) and (\ref{ex:VSO-1}) show VSO order. An example from elicitation with OVS order is (\ref{ex:OVSfirst}).


(\ref{ex:VOS-1}) comes from María C. talking about Clara. Our visit at her place caused a delay in her work. 

\ea\label{ex:VOS-1}
\begingl
\glpreamble  \textup{VOS:}\\ tanaumÿnÿ pan de arro eka nipiji\\
\gla ti-anau-mÿnÿ {pan de arroz} eka ni-piji\\
\glb 3i-make-\textsc{dim} {rice bread} \textsc{dem}a 1\textsc{sg}-sibling\\
\glft ‘my sister makes rice bread’
\endgl
\trailingcitation{[cux-120410ls.227]}
\xe

%tanaumÿne pan de arro eka nipiji atrasau chitÿpi, cux-120410ls.227


%(\ref{ex:VOS-1111}) was produced by Juana, when she and her brother were telling the story about the fox and jaguar and subsequently about the fox and the jaguarundi. The jaguar is already dead at this point of the story, and this is the start of a new episode, in which the fox meets the jaguarundi, when he looks for chicken.

%\ea\label{ex:VOS-1111}
%\begingl
%\glpreamble  \textup{VOS:}\\tiyunukutu tisemaika takÿra kupisaÿrÿ\\
%\gla ti-yunuku-tu ti-semaika takÿra kupisaÿrÿ\\
%\glb 3i-go.on-\textsc{iam} 3i-search.\textsc{irr} chicken fox\\
%\glft ‘the fox went on in order to look for chicken’\\
%\endgl
%\trailingcitation{[jmx-n120429ls-x5.300]}
%\xe

(\ref{ex:VOS}) was elicited from Clara.

\ea\label{ex:VOS}
\begingl 
\glpreamble  \textup{VOS:}\\chibukutu chikebÿke kusiÿ\\
\gla chi-buku-tu chi-kebÿke kusiÿ\\ 
\glb 3-finish-\textsc{iam} 3-eye ant\\ 
\glft ‘the ants finished (i.e. ate) its eyes (of the dead dried fish that we found)’
\trailingcitation{[cux-c120414ls-2.104]}
\xe


Generally, the preverbal position can be understood to convey emphasis on the participant. This is nicely illustrated by (\ref{ex:VOS-SVO}) below, in which María S. corrected herself when telling the story about the fox and the jaguar. She confused the main characters and first expressed – in VOS order – that the action of eating cheese was performed by the jaguar (\getfullref{ex:VOS-SVO.1}), but then corrected herself and using SVO order emphasised that it was the fox who was eating cheese. (\getfullref{ex:VOS-SVO.2}) is thus a \isi{focus} construction.


\ea\label{ex:VOS-SVO}
  \ea\label{ex:VOS-SVO.1}
 \begingl 
\glpreamble  \textup{VOS:}\\tinikukuikuji kesu isini\\
\gla ti-niku-kuiku-ji kesu isini\\ 
\glb 3i-eat-\textsc{cont}-\textsc{rprt} cheese jaguar\\ 
\glft ‘the jaguar was eating cheese, it is said’
  \ex\label{ex:VOS-SVO.2}
 \begingl
\glpreamble   \textup{SVO:}\\jai kupisairÿ tinikuji kesu\\
\gla  jai kupisairÿ ti-niku-ji kesu\\
\glb \textsc{intj} fox 3i-eat-\textsc{rprt} cheese\\
\glft ‘hay, THE FOX ate cheese, it is said’
\endgl
\trailingcitation{[rxx-n120511l-1.026-027]}
\z
\xe

(\ref{ex:SVO}) is an example of SVO order. Juana and I had been looking at the \isi{frog story} and she had identified the bird that flies from the tree as a hawk. We go on with the next page, but Juana digresses from the task in order to tell me about the hawk and an experience with another bird of prey that stole her dog. The hawk has been mentioned shortly before, but it is not topical. In order to indicate that the topic changes to the hawk again, the NP is positioned before the verb.

\ea\label{ex:SVO}
\begingl 
\glpreamble  \textup{SVO:}\\eka sia tiniku takÿra\\
\gla eka sia ti-niku takÿra\\ 
\glb \textsc{dem}a bird.sp 3i-eat chicken\\ 
\glft ‘the hawk eats chicken’
\trailingcitation{[jxx-a120516l-a.164]}
\xe

In (\ref{ex:SVO-2}) the object is a headless relative clause (see \sectref{sec:HeadlessRC}). It comes from the story about the two men who meet the devil in the woods as told by Miguel. The devil approaches shouting. It is important in the development of the story that only one of the men answers the devil and this importance is highlighted by placing the NP in preverbal position.

\ea\label{ex:SVO-2}
\begingl
\glpreamble  \textup{SVO:}\\i chinachÿ echÿu chikompanyerone chijakupu echÿu tiyÿbui\\
\gla i chinachÿ echÿu chi-kompanyero-ne chi-jakupu echÿu ti-yÿbui\\
\glb and one \textsc{dem}b 3-companion-\textsc{possd} 3-receive \textsc{dem}b 3i-shout\\
\glft ‘and one of the companions answered the one who shouted’
\endgl
\trailingcitation{[mxx-n101017s-1.021]}
\xe

%(\ref{ex:SVO-1}) also expresses a contrast. I had asked María S. whether she went to town to buy things, when she was a child. She denied, and I thought the reason was that her family grew everything they needed on their field. She shortly affirms this, but then encodes the real reason why she did not go to town: it was her siblings who went, she was still too young. This clause is connected to the preceding one, which is about sowing, with an adversative connective to indicate contrast and this contrast is also expressed by pre-verbal subject placement.
%
%\ea\label{ex:SVO-1}
%\begingl
%\glpreamble pero nipijijinubebane tiyununube, tiyÿseikunube xhabumÿnÿ\\
%\gla pero nipijijinubebane tiyununube tiyÿseikunube xhabumÿnÿ\\
%\glb \\
%\glft \\
%\endgl
%\trailingcitation{[rxx-p181101l-2.092]}
%\xe
%-> might be a  bi-clausal rather than mono-clausal sentence!


(\ref{ex:VSO}) is an example of VSO order. It comes from Juana telling the \isi{frog story} and describing the picture on which the dog is running away from the beehive that has fallen down and the bees chase it.

\ea\label{ex:VSO}
\begingl 
\glpreamble  \textup{VSO:}\\chinisapikutu jane eka kabe\\
\gla chi-nisapiku-tu jane eka kabe\\ 
\glb 3-sting-\textsc{iam} wasp \textsc{dem}a dog\\ 
\glft ‘the wasps sting the dog’
\trailingcitation{[jxx-a120516l-a.112]}
\xe

%dxx-d120416s.056: chimumuku echa aitubuche echÿu chipeu eka ... = el joven está mirando el masi
%chipujunÿpaikujitu eka kupisaÿrÿ isini = the fox pushed the tiger, jmx-n120429ls-x5.260

(\ref{ex:VSO-1}) is another example of VSO order, where the subject is a proper name. It remains to be checked whether VSO order is (more) usual if subjects are expressed by proper names, there are very few examples of such constellations in the corpus. In any case, this sentence was produced by María S. in elicitation, she first used a first singular index, when I asked her to translate the sentence, but repeated it with her own name to be put like this in this work. Here you are, María.

\ea\label{ex:VSO-1}
\begingl
\glpreamble  \textup{VSO:}\\tanatu Maria yumaji\\
\gla ti-ana-tu Maria yumaji\\
\glb 3i-make.\textsc{irr}-\textsc{iam} María hammock\\
\glft ‘María is making a hammock’
\endgl
\trailingcitation{[rxx-e181022le]}
\xe

One of the very few examples of OVS orders was produced by Juana, when Swintha asked her to describe a photo that showed me with my two daughters lying on top of me. However, a lot of hesitation marks accompany this sentence, so it is very probable that it might be taken as a mistake or probably as \isi{left dislocation} of a topical object (the picture they had been looking at before also showed my daughters and me).


\ea\label{ex:OVSfirst}
\begingl
\glpreamble \textup{OVS:}\\ruschunubechÿ chakachu chÿenujinube\\
\gla ruschÿ-nube-chÿ chÿ-akachu chÿ-enu-ji-nube\\
\glb two-\textsc{pl}-3 3-lift 3-mother-\textsc{col}-\textsc{pl}\\
\glft ‘the two of them, her mother lifts them’
\endgl
\trailingcitation{[jxx-p141024s-1.21]}
\xe

A different sentence with OVS order was produced by me in elicitation with María S. I asked her whether it was correct and she confirmed it; however, when she repeated the sentence, her intonation indeed suggested that it was a case of \isi{left dislocation}. She would not accept such a sentence if both object and subject were animate,\is{animacy} i.e. when I tried to elicit OVS order with a cat being the subject and a mouse being the object of the verb \textit{-niku} ‘eat’.

\ea\label{ex:OVSsecond}
\begingl
\glpreamble \textup{OVS:}\\amuke, tebuku nÿa\\
\gla amuke ti-ebuku nÿ-a\\
\glb corn 3i-sow 1\textsc{sg}-father\\
\glft ‘corn, my father sowed it’
\endgl
\trailingcitation{[rxx-e181024l]}
\xe


%\ea\label{ex:OVS12345}
%\begingl 
%\glpreamble sesejinube cheikukuikunube kabejane\\
%\gla sesejinube ch-eikukuiku-nube kabe-jane\\ 
%\glb children 3-chase-\textsc{pl} dog-\textsc{distr}\\ 
%\glft ‘the dogs chase the children’\\ 
%\endgl
%\trailingcitation{[mrx-e150219s.056]}
%\xe


I have not found a single example of a \isi{ditransitive} verb being accompanied by three NPs\is{noun phrase} in the corpus, but there is one example with a transitive verb with an incorporated body part.\is{incorporation} The possessor of this body part is expressed by an NP, and subject and object of the verb are also conominated. Presupposed that the possessor is analysed as a raised object here,\is{possessor raising} word order is VOSO, with the theme object\is{patient/theme} of the verb being expressed first and the possessor of the incorporated body part last. The example comes from the story about the fox and the jaguar as told by María S. It occurs close to the end of the story, where the fox ties a stone on the hands of the jaguar and the latter jumps into the water, expecting to find cheese there, but it is only the reflection of the moon that he sees.


\ea\label{ex:VOSO}
\begingl
\glpreamble \textup{VOSO:}\\chirÿtÿnebuÿchuji mai echÿu kupisairÿ echÿu isini\\
\gla chi-rÿtÿ-ne-buÿ-chu-ji mai echÿu kupisairÿ echÿu isini\\
\glb 3-tie-top-hand-\textsc{th}2-\textsc{rprt} stone \textsc{dem}b fox \textsc{dem}b jaguar\\
\glft ‘the fox tied a stone on top of the jaguar’s hands, it is said’
\endgl
\trailingcitation{[rxx-n120511l-1.037]}
\xe
 
Two more examples with two conominated objects were produced by Juana. In (\ref{ex:VOO-1}), she tells that one of her daughters wants to give back money she borrowed from her sister in Spain, when the latter once comes to Bolivia. The whole sentence, which consists of three separate clauses, is given here. The topical subject of the first clause is the daughter who lives in Spain, but in the second clause, topic switches to the other daughter. The new topic is not expressed by an NP. It is thus not totally clear whether the NP \textit{nijinepÿi} ‘my daughter’ in the third clause refers to the subject or the object, since both participants are Juana’s daughters. I would opt for an analysis as an object, because the topical subject of this clause is the same as the one in the preceding clause, there is thus topic continuity and topical participants are usually not expressed by NPs. The translation of the example follows this analysis. The word order of the last clause is thus VOO. However, it is also possible that \textit{nijinepÿi} is a subject NP, a delayed indication of topic switch in the previous clause. The translation of the last clause would be ‘my daughter will give her the money’ in that case and word order VSO.

\ea\label{ex:VOO-1}
\begingl
\glpreamble \textup{(non-verbal PRED, VO), VOO:}\\i despue kue kapupunuina te chebÿpekupuna echÿu tÿmue, chipua nijinepÿi \\chitÿmuane\\
\gla i despue kue kapupunu-ina te chÿ-ebÿpeku-puna echÿu tÿmue chi-pua ni-jinepÿi chi-tÿmua-ne\\
\glb and afterwards if come.back-\textsc{irr.nv} \textsc{seq} 3-borrow.money-\textsc{am.prior.irr} \textsc{dem}b money 3-give.\textsc{irr} 1\textsc{pl}-daughter 3-money-\textsc{possd}\\
\glft ‘and later, when she comes back, then she goes to borrow money, and she will give my daughter her money’
\endgl
\trailingcitation{[jxx-p120430l-1.294]}
\xe

The other example consists of two juxtaposed clauses. Interestingly, the recipient object of the verb of the second clause appears to the left of the first (intransitive) verb, i.e. quite dislocated from the verb it belongs to. This can be considered a case of long-distance dependency in Paunaka. (\ref{ex:OVVO}) describes the same situation as (\ref{ex:nebu-obl-1}) above, but comes from another recording with Juana made in another year, when she told me the story again. It is about her daughter being arrested in the airport in Spain for not having a valid visa. She was brought to a room upstairs in the airport building and received some food. The structure of this sentence is OVVO, with the first O pertaining to the second V.

\ea\label{ex:OVVO}
\begingl
\glpreamble \textup{O(V)VO:}\\i eka nijinepÿimÿnÿ tipununubetu chipunakunube yÿtÿuku\\
\gla i eka ni-jinepÿi-mÿnÿ ti-punu-nube-tu chi-punaku-nube yÿtÿuku\\
\glb and \textsc{dem}a 1\textsc{sg}-daughter-\textsc{dim} 3i-go.up-\textsc{pl}-\textsc{iam} 3-give-\textsc{pl} food\\
\glft ‘and as for my daughter, they went up and gave her some food’
\endgl
\trailingcitation{[jxx-p120430l-1.213]}
\xe


As for obliques,\is{oblique} they usually occur to the right of the verb. VX orders, with X standing for oblique as in \citet[]{wals-84}, are most common. All other participants are usually well-established by the preceding clauses and thus do not have to be conominated by an NP. XV is also found, but considerably less common, and largely restricted to temporal and source expressions. I will only consider different kinds of locative obliques and a few recipients here, because there are few examples for the other kinds of obliques. 

(\ref{ex:VX-1}) is one example with VX order. The oblique is a PP with the preposition \textit{tukiu}, a source expression. The sentence was produced by Juana, when she told me about how her daughter was deported from Spain and arrived back to Bolivia.

\ea\label{ex:VX-1}
\begingl
\glpreamble \textup{VX:}\\tikubupaikunubetu tukiu labion\\
\gla ti-kubupaiku-nube-tu tukiu labion\\
\glb 3i-go.down-\textsc{pl}-\textsc{iam} from plane\\
\glft ‘they disembarked from the plane’
\endgl
\trailingcitation{[jxx-p120430l-1.266]}
\xe

An unmarked oblique with the semantic role of goal is found in (\ref{ex:VX-2}), which was a statement by María S., when I told her that we had been looking for her before.

\ea\label{ex:VX-2}
\begingl
\glpreamble \textup{VX:}\\niyunutu asaneti\\
\gla ni-yunu-tu asaneti\\
\glb 1\textsc{sg}-go-\textsc{iam} field\\
\glft ‘I had gone to the field’
\endgl
\trailingcitation{[mrx-c120509l.052]}
\xe

It is very common that a \isi{locative} adverb is placed before the PP. This is the case in (\ref{ex:VX-3}): first comes the verb, then the adverb \textit{nauku} ‘there’ and finally the locative-marked noun. Juana cites her own words here, which were directed to her daughter. 

\ea\label{ex:VX-3}
\begingl
\glpreamble \textup{VX:}\\niyuna nauku parkeyae\\
\gla ni-yuna nauku parke-yae\\
\glb 1\textsc{sg}-go.\textsc{irr} there park-\textsc{loc}\\
\glft ‘I will go to the park there’
\endgl
\trailingcitation{[jxx-p120430l-2.242]}
\xe

If there is an object in the clause, we predominantly find VOX order and only occasionally XVO. VOX is the most common order cross-linguistically for languages in which the object usually follows the verb \citep[]{wals-84}. It is uncommon that there is a conominal S argument in a sentence that contains an oblique.

(\ref{ex:VOX-1}) comes from Miguel telling José the \isi{frog story}. This is his description of the picture on which the deer throws the boy down the slope into the water.

\ea\label{ex:VOX-1}
\begingl
\glpreamble \textup{VOX:}\\chibikÿkÿnÿkutu echÿu aitubuchepÿimÿnÿ kÿpenukÿyae\\
\gla chi-bikÿkÿnÿku-tu echÿu aitubuchepÿi-mÿnÿ kÿpenukÿ-yae\\
\glb 3-throw.away-\textsc{iam} \textsc{dem}b boy-\textsc{dim} depth-\textsc{loc}\\
\glft ‘it throws the boy into the depth’
\endgl
\trailingcitation{[mox-a110920l-2.153]}
\xe

(\ref{ex:VOX-2}) has a first person plural benefactive oblique. It stems from Juana’s account about how they made the reservoir in Santa Rita. A lady came to Santa Rita and promised them to make the reservoir in exchange for clearing of a big piece of land for her for agricultural use. She is the one who brought them meat.


\ea\label{ex:VOX-2}
\begingl
\glpreamble \textup{VOX:}\\tupunu chÿeche bitÿpi\\
\gla ti-upunu chÿeche bi-tÿpi\\
\glb 3i-bring meat 1\textsc{pl}-\textsc{obl}\\
\glft ‘she brought us meat’
\endgl
\trailingcitation{[jxx-p120515l-2.098]}
\xe


%Two examples for the orders VXO and XVO follow. Both were produced by Juana. (\ref{ex:VXO}) comes from Juana’s first account about her grandparents journey to Moxos.
%
%\ea\label{ex:VXO}
%\begingl
%\glpreamble \textup{(V), VXO:}\\tiyunu tiyeseikunube Monkoxiyae baka\\
%\gla ti-yunu ti-yeseiku-nube Monkoxi-yae baka\\
%\glb 3i-go 3i-buy-\textsc{pl} Moxos-\textsc{loc} cow\\
%\glft ‘they went and bought cows in Moxos’\\
%\endgl
%\trailingcitation{[jxx-e150925l-1.254]}
%\xe
%

%(\ref{ex:XVO-1}) is an example with a temporal expression preceding the verb. It is completely unmarked, i.e. neither the locative marker nor a preposition marks its oblique status. The example comes from María S. who talked about her childhood. Food, and especially meat, was scarce at that time, but her mother cooked rice and corn stews regularly. \textit{Pujukeke} (or \textit{mote} in Spanish) is a kind of stew with corn. When meat is added, the dish is called \textit{patasca} in Spanish.
%
%XVO, XOV:
%\ea\label{ex:XVO-1}
%\begingl
%\glpreamble chÿnachÿ tijai tiyÿtikapumÿnÿ arusuji pujaine pujukekepupunukutu tiniku\\
%\gla chÿnachÿ tijai ti-yÿtikapu-mÿnÿ arusu-ji pu-jaine pujukeke-pupunuku-tu ti-niku\\
%\glb one day 3i-cook.\textsc{irr}-\textsc{dim} rice-\textsc{clf:}soft.mass other-day mote-\textsc{reg}-\textsc{iam} 3i-eat\\
%\glft ‘one day she would cook a rice stew, the other day she ate \textit{mote} again’\\
%\endgl
%\trailingcitation{[rxx-p181101l-2.250]}
%\xe

(\ref{ex:XVO-2}) is an example of XVO order. The oblique is a source expression with the preposition \textit{tukiu} ‘from’, the adverb \textit{naka} ‘here’, and a toponym, \textit{Santa Cruz}, which does not take the locative marker. The sentence was produced by Juana when she still lived in Santa Cruz. She told me that when she had lived in Concepción before, her daughters bought meat for her and sent it to her by bus in a styrofoam box. She could then cook and sell \textit{patasca} in Concepción.

\ea\label{ex:XVO-2}
\begingl
\glpreamble \textup{XVO:}\\tukiu naka Santa Cruz tiyÿseikunube chichÿti ÿba\\
\gla tukiu naka {Santa Cruz} ti-yÿseiku-nube chi-chÿti ÿba\\
\glb from here {Santa Cruz} 3i-buy-\textsc{pl} 3-head pig\\
\glft ‘from Santa Cruz here, they bought a pig’s head’
\endgl
\trailingcitation{[jxx-e110923l-2.156]}
\xe


%punachÿyae estansia tumeikuji = on another estate he stole, it is said, jxx-p120430l-2.066


There are even fewer verbal clauses in the corpus that contain an oblique and an NP that conominates the subject. I have found the orders SVX (\ref{ex:SVX}), VSX (\ref{ex:VSX}) and VXS (\ref{ex:VXS}), but cannot say which one is most neutral for lack of sufficient data. It is questionable whether one can speak of a neutral order at all for a type of sentence that is very uncommon.

(\ref{ex:SVX}) is from the account by María S. about how she grew up. It consists of two juxtaposed sentences, both with an unmarked oblique constituent following the respective verb, while the subject NPs precede the verbs.


\ea\label{ex:SVX}
\begingl
\glpreamble \textup{SVX, SVX:}\\depue Krara tiyunutu uneku Kuana tikubiupu uneku\\
\gla depue Krara ti-yunu-tu uneku Kuana ti-kubiu-pu uneku\\
\glb afterwards Clara 3i-go-\textsc{iam} town Juana 3i-have.house-\textsc{dloc} town\\
\glft ‘later Clara went to town, Juana got a house in town’
\endgl
\trailingcitation{[rxx-p181101l-2.263]}
\xe

%eti epajÿkutu nauku Naranjito, mqx-p110826l.434


When giving me a description of how to use palm fruit oil for hair care, Juana produced (\ref{ex:VSX}). The oblique NP is accompanied by the adverb \textit{naka} ‘here’ in this case.

\ea\label{ex:VSX}
\begingl
\glpreamble \textup{VSX:}\\tipajÿkutu echÿu aseite naka bichÿtiyae\\
\gla ti-pajÿku-tu echÿu aseite naka bi-chÿti-yae\\
\glb 3i-stay-\textsc{iam} \textsc{dem}b oil here 1\textsc{pl}-head-\textsc{loc}\\
\glft ‘the oil (of the palm fruit) stays here on our head’
\endgl
\trailingcitation{[jxx-d181102l.30]}
\xe


(\ref{ex:VXS}) is from Juana’s second account about her grandparents’ journey to Moxos and back home with the cows they bought there. It is a long journey and the grandparents had to rest on the way. They usually stayed in huts along the way and sometimes the huts also had an enclosure, where they kept their cows.

\ea\label{ex:VXS}
\begingl
\glpreamble \textup{VXS:}\\tibÿkupujaneji bakayayae baka\\
\gla ti-bÿkupu-jane-ji bakaya-yae baka\\
\glb 3i-enter-\textsc{distr}-\textsc{rprt} enclosure-\textsc{loc} cow\\
\glft ‘the cows went into the enclosure, it is said’
\endgl
\trailingcitation{[jxx-p151016l-2.166]}
\xe

%tipajÿku nauku España chimaretane, jxx-p120430l-1.272

Finally, it is also possible to have two conominated arguments plus an oblique. This is the case in (\ref{ex:obl-tÿpi-5}) and (\ref{ex:SVOX}). The constituent orders are SVXO and SVOX respectively and the oblique constituent is a benefactive in both cases. As for the question which of these orders is more common, I would opt for the second one, because the oblique usually follows the object NP in VOX clauses. However, I would not be able to prove this with data from the corpus, since there are simply not enough sentences in which we have conominal expressions of subject and object as well as an oblique.

(\ref{ex:obl-tÿpi-5}) comes from Miguel’s account about how he went to school. Since they had no paper, the pupils wrote on wooden boards. Miguel’s board was made by his father.


\ea\label{ex:obl-tÿpi-5}
\begingl
\glpreamble \textup{SVXO:}\\entonses kuineini taitaini tanau nitÿpi echÿu taurapamÿnÿ\\
\gla entonses kuineini taita-ini ti-anau ni-tÿpi echÿu taurapa-mÿnÿ\\
\glb thus deceased dad-\textsc{dec} 3i-make 1\textsc{sg}-\textsc{obl} \textsc{dem}b board-\textsc{dim}\\
\glft ‘so my late father made a small wooden board for me’
\endgl
\trailingcitation{[mxx-p181027l-1.023]}
\xe


A sentence with the order SVOX was elicited from Juana.


\ea\label{ex:SVOX}
\begingl
\glpreamble \textup{SVOX:}\\ eka nijinepÿi tiyÿseiku eka epuke tÿpi eka chipiji\\
\gla eka nij-inepÿi ti-yÿseiku eka epuke tÿpi eka chi-piji\\
\glb \textsc{dem}a 1\textsc{sg}-daughter 3i-buy \textsc{dem}a ground \textsc{obl} \textsc{dem}a 3-sibling\\
\glft ‘my daughter bought ground for her sister’
\endgl
\trailingcitation{[jxx-e191021e-2]}
\xe


%Another example shows the order SVXO; however, the S is detached from the rest of the clause by insertion of an interjection between subject and verb. This can thus be considered a case of left dislocation.
%
%\ea\label{ex:}
%\begingl
%\glpreamble i echÿu kayaraunu juu chumutu chÿestancianeye chipeunube baka\\
%\gla i echÿu kayaraunu juu chumutu chÿestancianeye chipeunube baka\\
%\glb \\
%\glft ‘and the \textit{karay}, huu, he took their cows to his estate’\\
%\endgl
%\trailingcitation{[jxx-e150925l-1.235]}
%\xe
%



% adverbial clauses: temporal, only juxtaposition: jmx-d110918ls-1.010

%chakachu chÿenu tiyunu chubu hospitalyae, jxx-p110923l-1.460


In summary, word order in Paunaka is quite flexible, but it is most common that the \isi{verb} comes first, and it also quite common that the object follows the verb directly. There is one pre-verbal slot, which may be filled with S, O or X to indicate special discourse function.

\is{conomination|)}
\is{word order|)}

The following section focuses on standard negation.




\subsection{Standard negation}\label{sec:Negation}
\is{negation|(}

This section is about “the basic way(s) a language has for negating declarative verbal main clauses” \citep[1]{Miestamo2005}. This has been called “standard negation”. Other types of negation are found in the individual sections about different non-verbal clauses (see \sectref{sec:NonVerbalPredication}) and in the section about negative imperatives (see \sectref{sec:Prohibitives}).\footnote{\label{fn:privative} As for morphological negation including a reflex of the famous Proto-Arawakan\is{Arawakan languages} \isi{privative} prefix \textit{*ma-}, this is not productive in Paunaka. The only two words I can think of containing a reflex of the privative prefix are \textit{mupÿinube} (\textit{mu-pÿi-nube} \textsc{priv}-body-\textsc{pl}) ‘devil’ (lit.: ‘the ones without body’) and \textit{mÿbanejiku} (\textit{mu-ÿ-bane-jiku} \textsc{priv}-be.long-\textsc{rem}-\textsc{lim}1) ‘close to, near’. In addition, there are some more words starting with \textit{mu} like \textit{mukÿe} ‘squash sp.’. They may or may not have once been derived\is{derivation} from other words with the privative prefix, but in any case they are not decomposable synchronically.}

Standard negation builds on the \isi{negative particle} \textit{kuina}, which is placed directly before the verb.\is{word order} This particle seems to include the \isi{non-verbal irrealis marker} \textit{-ina} attached to a stem or affix \textit{ku}. Note that a voiceless velar plosive is relatively common in standard negation of \isi{Arawakan languages} \citep[288]{Michael2014b}, and the \isi{Mojeño languages} have a prefix \textit{ku-} for irrealis negation (cf. Rose 2020, p.c.), i.e. the “doubly irrealis construction”.

In Paunaka’s standard negation, the verb necessarily has \isi{irrealis} RS given that a non-realised event is always non-factual. Standard negation thus shows a paradigmatic asymmetry as regards RS\is{reality status} \citep[96]{Miestamo2005}. There is no morphological “\isi{doubly irrealis construction}” in declarative sentences. This is defined as a construction that explicitly marks that there are (at least) two parameters that trigger irrealis RS, one of them being negation \citep[cf.][253]{Michael2014}. This may be worth mentioning explicitly, because closely related Trinitario,\is{Mojeño Trinitario} \isi{Terena} and Kinikinau as well as the more distantly related Kampan languages\is{Arawakan languages} all have more or less elaborate doubly irrealis contructions \citep[267--269]{Michael2014b}.%261-263

Consider (\ref{ex:neg-2}). The positive sentences in (\ref{ex:neg-2.1}) and (\ref{ex:neg-2.2}) differ from each other in RS,\is{reality status} with (\getfullref{ex:neg-2.1}) encoding a factual event by default/realis and (\getfullref{ex:neg-2.2}) a non-factual event by irrealis. When negated as in (\getfullref{ex:neg-2.3}), this distinction is neutralised.


\ea\label{ex:neg-2}
  \ea\label{ex:neg-2.1}
\begingl
\glpreamble niniku\\
\gla ni-niku\\
\glb 1\textsc{sg}-eat\\
\glft ‘I eat/ate it’
\endgl
  \ex\label{ex:neg-2.2}
\begingl
\glpreamble ninika\\
\gla ni-nika\\
\glb 1\textsc{sg}-eat.\textsc{irr}\\
\glft ‘I will/can/must eat it’
\endgl
  \ex\label{ex:neg-2.3}
\begingl
\glpreamble kuina ninika\\
\gla kuina ninika\\
\glb \textsc{neg} 1\textsc{sg}-eat.\textsc{irr}\\
\glft ‘I don’t/didn’t/can’t/couldn’t/won’t eat it’
\endgl
\z
\xe

Some more examples of negative declarative clauses follow, containing a stative intransitive verb  in (\ref{ex:neg-stat}), an active intransitive verb in (\ref{ex:neg-act}), a transitive verb in (\ref{ex:neg-trans}), and a ditransitive verb in (\ref{ex:neg-dit}).

(\ref{ex:neg-stat}) comes from María S. talking with me about snow in Germany.

\ea\label{ex:neg-stat}
\begingl
\glpreamble kuina tasÿeiyu\\
\gla kuina ti-a-sÿei-yu\\
\glb \textsc{neg} 3i-\textsc{irr}-be.cold-\textsc{ints}\\
\glft ‘it is not very cold’
\endgl
\trailingcitation{[rxx-e120511l.312]}
\xe

%(\ref{ex:neg-stat-2}) is a statement by Juana about her son-in-law, who is ill.
%
%\ea\label{ex:neg-stat-2}
%\begingl
%\glpreamble kuina tajimama\\
%\gla kuina ti-a-jimama\\
%\glb \textsc{neg} 3i-\textsc{irr}-be.strong\\
%\glft ‘he is not strong’\\
%\endgl
%\trailingcitation{[jxx-p110923l-1.053]}
%\xe

In (\ref{ex:neg-act}), Juana states that her daughter did not go (to the airport). In her opinion, her daughter should have picked up her sister there. The latter was supposed to work as a nanny in Spain, but was finally deported without having ever entered the country.

\ea\label{ex:neg-act}
\begingl
\glpreamble kuina tiyuna\\
\gla kuina ti-yuna\\
\glb \textsc{neg} 3i-go.\textsc{irr}\\
\glft ‘she didn’t go’
\endgl
\trailingcitation{[jxx-p110923l-1.312]}
\xe

(\ref{ex:neg-trans}) is part of the answer José gave, when Miguel asked him whether he knew the story about the lazy man.

\ea\label{ex:neg-trans}
\begingl
\glpreamble kuina nichupa micha\\
\gla kuina ni-chupa micha\\
\glb \textsc{neg} 1\textsc{sg}-know.\textsc{irr} good\\
\glft ‘I don’t know it well’
\endgl
\trailingcitation{[mox-n110920l.007]}
\xe

In (\ref{ex:neg-dit}), Juan C. speaks about the old times, when his \textit{patrón} refused to give him a pair of trousers that was supposed to be part of his compensation for working.

\ea\label{ex:neg-dit}
\begingl 
\glpreamble kuina tipunakane nikasuneina\\
\gla kuina ti-punaka-ne ni-kasune-ina\\ 
\glb \textsc{neg} 3i-give.\textsc{irr}-1\textsc{sg} 1\textsc{sg}-trousers-\textsc{irr.nv}\\ 
\glft ‘he didn’t give me my supposed trousers’
\trailingcitation{[mqx-p110826l.458]}
\xe

Word order\is{word order} is largely the same as in positive sentences; however, conominals\is{conomination|(} are in general rarer. If present, they usually follow the negated verb. (\ref{ex:neg-VS}) is an example in which a conominated subject follows and (\ref{ex:neg-VO}) has a conominated object. In order to indicate special discourse status, a conominal argument can also precede the negator. This is the case in (\ref{ex:neg-SV}), where the subject precedes \textit{kuina} for contrastive \isi{focus}.

(\ref{ex:neg-VS}) comes from Miguel speaking about the old days.

\ea\label{ex:neg-VS}
\begingl
\glpreamble kuina chisiupuchanube eka patron\\
\gla  kuina chi-siupucha-nube eka patron\\
\glb \textsc{neg} 3-pay.\textsc{irr}-\textsc{pl} \textsc{dem}a patrón\\
\glft ‘the \textit{patrón} didn’t pay them’
\endgl
\trailingcitation{[mxx-p110825l.042]}
\xe

In (\ref{ex:neg-VO}), Juana tells her sister the reason why her ducklings died, when she was away for one week. 

\ea\label{ex:neg-VO}
\begingl
\glpreamble kuina chetukanube eka yÿtÿukumÿnÿ \\
\gla kuina chÿ-etuka-nube eka yÿtÿuku-mÿnÿ \\
\glb \textsc{neg} 3-put.\textsc{irr}-\textsc{pl} \textsc{dem}a food-\textsc{dim}\\
\glft ‘they didn’t give them food, poor ones’
\endgl
\trailingcitation{[jrx-c151001lsf-11.063]}
\xe
\is{conomination|)}

The context of (\ref{ex:neg-SV}) is that María S. complains that her chicken get stolen when she is away from her house.

\ea\label{ex:neg-SV}
\begingl
\glpreamble nÿti kuina nÿnika pero punachÿ tiniku\\
\gla nÿti kuina nÿ-nika pero punachÿ ti-niku\\
\glb 1\textsc{sg.prn} \textsc{neg} 1\textsc{sg}-eat.\textsc{irr} but other 3i-eat\\
\glft ‘I don’t eat them, but another one eats them’
\endgl
\trailingcitation{[rxx-e120511l.181]}
\xe

Some markers can be attached to the \isi{negative particle}, among them the \isi{additive} and those expressing TAME categories.\is{tense}\is{aspect}\is{modality}\is{evidentiality} However, all of these markers can also attach to the \isi{verb} with no difference in meaning.\footnote{Although when comparing the two examples in (\ref{ex:neg-add}) and (\ref{ex:neg-add-2}), it may look like the additive marker was a third-position clitic, this is not the case. There are dozens of examples in the corpus, where the additive occurs in other positions; see §\ref{sec:AffixesAndClitics} for a general discussion on this issue.}

Consider the following example, which has an additive marker. Prior to uttering this sentence, Juana had just told me that she did not speak Spanish, when she was a child, only Paunaka. She adds to this statement by (\ref{ex:neg-add}), telling me that she did not have any contact to \isi{Bésiro} either. 


\ea\label{ex:neg-add}
\begingl
\glpreamble i echÿu tiseteiku kuinauku nisama\\
\gla i echÿu tiseteiku kuina-uku ni-sama\\
\glb and \textsc{dem}b Bésiro \textsc{neg}-\textsc{add} 1\textsc{sg}-hear.\textsc{irr}\\
\glft ‘and Bésiro I didn’t hear either’
\endgl
\trailingcitation{[jxx-p120430l-1.028-030]}
\xe

The way of listing that preceded (\ref{ex:neg-add-2}) was very similar, but in this case the additive marker is attached to the verb: María C. stated that she did not know her grandparents, but only knew her mother and then added that she did not know her father either (because she was still very young when he passed away).

\ea\label{ex:neg-add-2}
\begingl
\glpreamble nÿa, kuina nichupuikuka nÿa\\
\gla  nÿ-a kuina ni-chupuiku-uka nÿ-a\\
\glb 1\textsc{sg}-father \textsc{neg} 1\textsc{sg}-know-\textsc{add.irr} 1\textsc{sg}-father\\
\glft ‘as for my father, I didn’t know my father either’
\endgl
\trailingcitation{[ump-p110815sf.148]}
\xe


As for TAME marking in negative sentences, there is one peculiarity: the \isi{prospective} does not occur in negative clauses, nor does the otherwise omnipres\-ent iamitive \is{iamitive|(}.\footnote{The only exception is that the iamitive can be attached to the discontinuous\is{discontinuous|(} marker on the \isi{negative particle} itself, yielding \textit{kuinabutu} ‘not anymore now’.} Instead of this, the \isi{}discontinuous marker \textit{-bu} adopts one of the functions of the \isi{iamitive} indicating that a previously ongoing event is already finished. The discontinuous marker is only found in negative clauses and can be translated as ‘(not) anymore’. The verb is usually interpreted as encoding a state, even if this is not inherent in its semantics, see \sectref{sec:Discontinuous}. As for the other function of the iamitive, expressing that something is ongoing (telic verbs),\is{telicity} there is no way to form a corresponding negative sentence. Besides neutralisation of RS, this is the second asymmetry found between negative and positive declarative sentences.\is{iamitive|)}

One example of the discontinuous marker in a negative sentence is given in (\ref{ex:neg-dsc}).\footnote{The discontinuous marker actually occurs twice here, on the negative particle and on the verb, but this is not obligatory. It can also occur once, either on \textit{kuina} or on the verb, see \sectref{sec:Discontinuous}.} It was produced by Juana when she told me about how her brother passed away, thus ‘not speak anymore’ is stative in the sense that it does not refer to a momentary disruption of speaking, but to an irreversible state of weakness before his death. It is another brother of hers whom she cites here.

\ea\label{ex:neg-dsc}
\begingl
\glpreamble “nÿbÿsÿu kuinabu tichujikabu”, tikechu\\
\gla nÿ-bÿsÿu kuina-bu ti-chujika-bu ti-kechu\\
\glb 1\textsc{sg}-come \textsc{neg}-\textsc{dsc} 3i-speak.\textsc{irr}-\textsc{dsc} 3i-say\\
\glft ‘“when I came, he didn’t speak anymore”, he said’
\endgl
\trailingcitation{[jxx-p120430l-2.456]}
\xe
\is{discontinuous|)}
\is{negation|)}

The discussion of negation in verbal declarative clauses is completed at this point. The next section is dedicated to non-verbal predication, including both positive and negative non-verbal clauses. \is{declarative clause|)}



%kuina bichupuika eka nÿkÿiki, jxx-d110923l-2.06 no conocimos olla
%kuina chetukanube eka yÿtiÿukumÿnÿ eka yÿtÿuku chitÿpijane, jrx-c151001lsf-11.063 0 her ducks died, because they were not fed, when she was away i SCz

%SVO: nÿenu kuina tinikane yÿtÿuku, mxx-e160811sd.052 el.
%OV: echÿu chichupu echÿu chitareane kuina cheistaka, mxx-p181027l-1.076
%OV: ee chÿnajiku echÿu kuina kuina ni- nÿ- nÿ- nisumacha nechÿu matematica, mxx-p181027l-1.090

%“When no structural differences are found between the affirmative and the negative in addition to the negative marker(s) the structures are symmetric. When there are structural differences, i.e. asymmetry, between the affirmative and the negative in addition to the negative marker(s), the structures are asymmetric. ” \citep[51]{Miestamo2005}
%
%Negation:
%
%\ea\label{ex:no-ride}
%\begingl
%\glpreamble kuina tubuneikanube\\
%\gla kuina ti-ubuneika-nube\\
%\glb \textsc{neg} 3i-ride.\textsc{irr}-\textsc{pl}\\
%\glft ‘they didn’t ride on horse’\\
%\endgl
%\trailingcitation{[jxx-p151016l-2.040]}
%\xe








%!TEX root = 3-P_Masterdokument.tex
%!TEX encoding = UTF-8 Unicode

\section{Non-verbal predication}\label{sec:NonVerbalPredication}
\is{declarative clause|(}
\is{non-verbal predication|(}

There are many clauses with non-verbal predicates. They belong to different semantic types of non-verbal predication with partly different construction types. As for the latter we find \isi{juxtaposition} of the predicate and the \isi{subject} NP,\is{noun phrase} usage of the non-verbal \isi{copula} \textit{kaku} and other strategies. The semantic types of non-verbal predication comprise the ones typically found in the literature: equative, proper inclusion,
attributive, location, existential and possessive, to use the terminology by \citet[ch. 6]{Payne1997}.\footnote{\citet[6]{Overall2018} use the term “identification” instead of “equative” and “categorisation” instead of “proper inclusion”. They further deviate from \citet[]{Payne1997} in that they split the attributive type in two subtypes encoding permanent versus temporary property, but this is not of concern for Paunaka. See also \citet[]{Dryer2007} for yet another similar classification.} In addition, some other semantic types are found in Paunaka, which have been called “minor types”: genitive and benefactive,\is{genitive/benefactive clause} quantification, similative \is{similative clause}  \citep[246--249]{Dryer2007},\footnote{The author actually uses the term “simulative”, while “similative” is the one proposed by \citet[]{HaspelmathBuchholz1998}.} and also locomotion \citep[113]{Payne1997}, the latter being restricted to third person cislocative motion.\is{motion predicate} Strikingly, verbs are often borrowed\is{borrowing} from Spanish as non-verbal predicates, too. \tabref{table:OverviewNV-Pred} shows how semantic types and construction types correlate. All semantic types marked by an asterisk can also be expressed by a verbal strategy. 

\begin{table}[htbp] 
\caption{Semantic types and construction types in non-verbal predication}

\begin{tabular}{ll}
\lsptoprule
Semantic type & Construction type\cr
\midrule
equative & juxtaposition\cr
proper inclusion & juxtaposition\cr
attributive & juxtaposition\cr
quantification & juxtaposition\cr
genitive/benefactive & juxtaposition \cr
location* & juxtaposition / copula\cr
existential & copula\cr
possessive*  & copula\cr
similative  & other \cr
(3rd person) locomotion* & other\cr
borrowed verbs* & other\cr
\lspbottomrule
\end{tabular}

\label{table:OverviewNV-Pred}
\end{table}


Many semantic types in the table are assigned to the juxtaposition construction and a few to the one including a \isi{copula}; however, this is a simplification of the issue. As for \isi{juxtaposition}, a \isi{subject} NP\is{noun phrase} does not necessarily co-occur with the predicate just as in verbal predication. In non-verbal predication this can be related to the subject being topical\is{topic} and/or the predicate taking person indexes. As for the location,\is{locative clause} existential,\is{existential clause} and possessive\is{possessive clause} non-verbal types, they include a copula in positive clauses, but do not need a \isi{copula} in negative clauses\is{negation} where its use is often related to emphasis.

Before describing the different constructions in more detail, I want to provide a short illustration of the properties in which non-verbal predicates differ from verbal ones. Two factors are involved, \isi{reality status} and \isi{person marking}.\is{inflection|(}

\is{non-verbal irrealis marker|(}
\hspace*{-.3pt}Realis is completely unmarked in non-verbal predication, but irrealis is marked. It is triggered by the same parameters that are also relevant for irrealis marking on verbs minus the imperative (see \sectref{RS:parameters}), but there is a specific irrealis marker \textit{-ina} that only occurs on non-verbal words.

Consider (\ref{ex:NV-irr-1}), in which the \isi{nominal predicate} is negated and thus has irrealis RS. The example was produced by María S. and referred to some puppies in her yard which had apparently been brought to Santa Rita although they were still sucklings, thus they were condemned to die.

\ea\label{ex:NV-irr-1}
\begingl
\glpreamble kuina chÿenuina tukiu uneku\\
\gla kuina chÿ-enu-ina tukiu uneku\\
\glb \textsc{neg} 3-mother-\textsc{irr.nv} from town\\
\glft ‘they don’t have a mother, they are from town’
\endgl
\trailingcitation{[rxx-e120511l.363]}
\xe

The non-verbal irrealis marker is the very same morpheme that is also used as a \isi{nominal irrealis} marker (see \sectref{NominalRS}). There is sometimes substantial functional overlap between both, as in (\ref{ex:NV-IRR-cuento}), where the nominal predicate has \isi{future reference}, but could also well be analysed as a nominal future.

The example is taken from a syncretistic creation story told by Juana. Jesus is about to marry María Eva and God tells them:

\ea\label{ex:NV-IRR-cuento}
\begingl 
\glpreamble “eka pimaina i eka piyenuina”\\
\gla eka pi-ima-ina i eka pi-yenu-ina\\ 
\glb \textsc{dem}a 2\textsc{sg}-husband-\textsc{irr.nv} and \textsc{dem}a 2\textsc{sg}-wife-\textsc{irr.nv}\\ 
\glft ‘“this one will be your husband and this one will be your wife”’\\or:‘“this is your future husband and this is your future wife”’
\trailingcitation{[jxx-n101013s-1.368-369]}
\xe

Irrealis marking is applicable to distinguish most cases of verbal and non-ver\-bal predication. However, some words do not inflect for irrealis, e.g. the demonstrative adverbs.\is{non-verbal irrealis marker|)}

The second criterion\is{alignment|(} to distinguish both kinds of predicates is the position of the \isi{subject} indexes.\is{person marking|(}\is{argument|(} It has been shown in \sectref{sec:NumberPersonVerbs} (and also in \sectref{sec:ExpressionSubjects} and \sectref{sec:DeclClausesOBL}) that verbs index subjects with person markers preceding the stem and objects with person markers following the stem. Non-verbal predicates, however, index subjects with person markers following the stem, while the position preceding the stem is retained for possessors. The person markers are identical to the ones used on verbs.\footnote{This type of \isi{split-S marking} dependent on parts of speech\is{word class} has been described in detail for \isi{Baure} in contrast with another Arawakan language\is{Arawakan languages} in which the split depends on other factors \citep[cf.][]{Danielsen_Granadillo2008}.}\is{alignment|)}

The subject indexes on non-verbal predicates are summarised in \tabref{table:NVP_Person}. Compare to \tabref{table:VerbsPerson_all} in \sectref{sec:NumberPersonVerbs}, which summarises the person indexes used with verbs.

\begin{table}[htbp] 
\caption{Subject indexes on non-verbal predicates}

\begin{tabular}{ll}
\lsptoprule
 Person & Index \cr
\midrule
1\textsc{sg} & \textit{-ne / -nÿ} \cr
2\textsc{sg} & \textit{-bi / -pi} \cr
3\textsc{sg} & / \cr
1\textsc{pl} & \textit{-bi} \cr
2\textsc{pl} & \textit{-e} \cr
3\textsc{pl} & (\textit{-nube}) \cr
3\textsc{distr} & (\textit{-jane}) \cr
\lspbottomrule
\end{tabular}

\label{table:NVP_Person}
\end{table}

(\ref{ex:NV-pers-1}) is an example in which a person marker indexes the subject on a \isi{nominal predicate}. It comes from Isidro who was talking about getting old and grey with Miguel and contrasted this to my age.

\ea\label{ex:NV-pers-1}
\begingl
\glpreamble pimiyakuÿbi\\
\gla pimiya-kuÿ-bi\\
\glb girl-\textsc{incmp}-2\textsc{sg}\\
\glft ‘you are still young’
\endgl
\trailingcitation{[mdx-c120416ls.152]}
\xe

%nisimubi, pisimune

Position of the subject index is not always applicable as an indicator for non-verbal predication. The use of third person markers that follow the stem is very restricted on verbs (see \sectref{sec:3_suffixes}), and they do not occur on non-verbal predicates, so that there is no subject index for the third person.\footnote{A \isi{plural} or \isi{distributive} marker can be used, though, if the subject is a third person plural, but it is not always possible to distinguish referential and predicative use, i.e. it is arguable whether \textit{-nube} and \textit{-jane} relate to subject marking or rather to number marking of a noun or nominal demonstrative. Regarding the locomotion predicate\is{motion predicate} \textit{kapunu} ‘come’ and borrowed verbs\is{borrowing} integrated as non-verbal predicates, these markers relate to subjects.} In addition, indexing the subject is sometimes optional, e.g. in equative clauses,\is{equative/proper inclusion clause} and some words generally do not take person markers, e.g. the demonstrative adverbs.
\is{person marking|)}\is{inflection|)}

As for the copula \textit{kaku},\is{copula|(} it only relates to third-person referents. The same is true for the non-verbal locomotion predicate\is{motion predicate} \textit{kapunu} ‘come’. The form of both of them includes a first syllable \textit{ka} which, as I have argued in \sectref{sec:DemPron}, could be the same deictic root that we find in the demonstratives\is{demonstrative} \textit{eka} ‘\textsc{dem}a’ and \textit{naka} ‘here’. This would explain their restriction to the third person. However, as for the copula, there is also a suffix \textit{-uku} used exclusively with personal pronouns\is{personal pronoun} (i.e. with first and second person reference) in non-verbal locative predication, see \sectref{sec:LocativePredicates} below.\is{copula|)} This suffix might well be related to the final syllable \textit{ku} of the copula \textit{kaku}. The expression of locomotion, however, uses a verbal strategy whenever there is an SAP subject.\is{argument|)}

The remainder of this section is structured by semantic and construction type. Equative, proper inclusion\is{equative/proper inclusion clause} and attributive clauses\is{attributive clause} overlap to a large degree and are thus described together in \sectref{sec:PropInclEquatAttr}. Predication of quantification is the topic of \sectref{sec:Quantification}. Genitive and benefactive clauses are discussed in \sectref{sec:GenitiveBenfactivePreds}. All of these semantic types are formed by \isi{juxtaposition} of predicate and \isi{subject}. Predication of location can also be expressed by \isi{juxtaposition} or with the help of a \isi{copula}. Due to restriction of the \isi{copula} to third person referents, there are different strategies for SAP. This is explained in detail in \sectref{sec:LocativePredicates}. The existential construction is described in \sectref{sec:Existentials}, and possessive clauses, which can be considered to belong to the existential construction, in \sectref{sec:PossessiveClauses}. \sectref{sec:SimilativePreds} is about the similative construction and \sectref{sec:Kapunu} describes the use of the non-verbal third person cislocative predicate. Finally, \sectref{sec:borrowed_verbs} shows how verbs from Spanish are integrated into Paunaka as non-verbal predicates.



\subsection{Equative, proper inclusion and attributive}\label{sec:PropInclEquatAttr}
\is{juxtaposition|(}
\is{equative/proper inclusion clause|(}

The two semantic types of equation and proper inclusion usually include nouns\is{noun} that serve as predicates,\is{nominal predicate} the difference among them being that equation expresses that the \isi{subject} of the clause is an entity that is identical to the entity specified by the predicate, while proper inclusion encodes that the \isi{subject} is member of a class which is specified by the predicate \citep[114]{Payne1997}. There are some languages, in which both types are encoded differently, but the languages in South and Central America commonly use the same construction for both types \citep[7]{Overall2018}. Since Paunaka does not obligatorily mark \isi{definiteness} on NPs, some sentences are ambiguous as to the question whether they represent equation or proper inclusion, and one example of this ambiguity is presented in (\ref{ex:ambi-eqprop}) below.

The attributive type\is{attributive clause|(} is bound to adjectives\is{adjective|(} having the role of predicates. However, in Paunaka most property concepts are expressed by stative verbs. Only a few words can felicitously be defined as adjectives. In addition, some properties, especially age, are predicated by nouns\is{noun} (see \sectref{sec:Adjectives}). This is where proper inclusion and attributive predication semantically overlap. In addition, attributive clauses are cross-linguistically often also identical to proper inclusion and equative clauses in structure \citep[120]{Payne1997}, and this is also the case in Paunaka.\is{adjective|)} 

There are three possibilities how to encode all three types: first, the \isi{subject} NP\is{noun phrase} and the predicate are juxtaposed, second, the subject is indexed on the predicate, and third, subject NP and predicate are juxtaposed AND the subject is indexed on the predicate. (\ref{ex:Equative-1}), which stems from elicitation with Miguel, shows all of these possibilities. 
 
\ea\label{ex:Equative-1}
  \ea
 \begingl 
\glpreamble piti nÿa \\
\gla piti nÿ-a\\ 
\glb 2\textsc{sg.prn} 1\textsc{sg}-father\\ 
\glft ‘you are my father’
  \ex
 \begingl 
\glpreamble nÿabi\\
\gla nÿ-a-bi\\ 
\glb 1\textsc{sg}-father-2\textsc{sg}\\ 
\glft ‘you are my father’
  \ex
 \begingl
\glpreamble piti nÿabi\\
\gla piti nÿ-a-bi\\
\glb 2\textsc{sg.prn} 1\textsc{sg}-father-2\textsc{sg}\\
\glft ‘you are my father’
\endgl
\trailingcitation{[mxx-e090728s-3.088-090]}
\z
\xe

If speakers choose the juxtaposition construction,\is{word order|(} the predicate usually precedes the \isi{subject} NP unless the subject is expressed by a \isi{pronoun}. In the latter case, the pronoun precedes the predicate. This is also the preferred word order in Trinitario’s\is{Mojeño Trinitario} non-verbal clauses \citep[75]{Rose2018}, and it mirrors the one found in verbal clauses, where non-emphasised subjects usually follow the verb, but not if they are pronominal (see \sectref{sec:WordOrder}).\is{word order|)} A few examples with the order predicate -- subject NP follow. (\ref{ex:equa-1}) and (\ref{ex:equa-2}) are equative clauses, (\ref{ex:propi-1}) is an example of proper inclusion, and (\ref{ex:attri-1}) and (\ref{ex:attri-2}) have adjectival predicates and are thus attributive clauses.\is{attributive clause|)}

(\ref{ex:equa-1}) comes from Juana who interrupted her speech, when she recognised the wasp close to her.

\ea\label{ex:equa-1}
\begingl
\glpreamble ¡aij jane echÿu!\\
\gla aij jane echÿu\\
\glb \textsc{intj} wasp \textsc{dem}b\\
\glft ‘aiy, this is a wasp!’
\endgl
\trailingcitation{[jxx-p120430l-2.478]}
\xe

(\ref{ex:equa-2}) is a similar example from Miguel, who first thought the bed of the boy in the \isi{frog story} was a church.

\ea\label{ex:equa-2}
\begingl
\glpreamble chubiukena bia eka naka\\
\gla chÿ-ubiu-kena bia eka naka\\
\glb 3-house-\textsc{uncert} God \textsc{dem}a here\\
\glft ‘this one here might be a church’
\endgl
\trailingcitation{[mox-a110920l-2.019]}
\xe

In (\ref{ex:propi-1}), Juana tells me that the water spirit whom their grandparents met on their journey back home from Moxos was a woman, a fact that becomes important a bit later in the story.

\ea\label{ex:propi-1}
\begingl
\glpreamble i seunube echÿu ue\\
\gla i seunube echÿu ue\\
\glb and woman \textsc{dem}b water.spirit\\
\glft ‘and the water spirit was a woman’
\endgl
\trailingcitation{[jxx-p151016l-2.157]}
\xe
\is{equative/proper inclusion clause|)}

\is{attributive clause|(}
(\ref{ex:attri-1}) was produced by Juana as a confirmation of what I had said before. It is about the house of her daughter in Santa Cruz.

\ea\label{ex:attri-1}
\begingl
\glpreamble ja temena ubiyae\\
\gla ja temena ubiyae\\
\glb \textsc{afm} big house\\
\glft ‘yes, the house is big’
\endgl
\trailingcitation{[jxx-p120430l-1.414]}
\xe

(\ref{ex:attri-2}) was a statement by María C. about her favourite drink.

\ea\label{ex:attri-2}
\begingl
\glpreamble michaniki aumue\\
\gla michaniki aumue\\
\glb delicious chicha\\
\glft ‘chicha tastes good’
\endgl
\trailingcitation{[uxx-p110825l.257]}
\xe
\is{attributive clause|)}

\is{equative/proper inclusion clause|(}
If the information given in the sentence is about the name of somebody, we actually find both orders:\is{word order} predicate – subject NP as in (\ref{ex:Equative2}) and subject NP – predicate as in (\ref{ex:name-1}). This may be related to the structure of the corresponding sentence in Spanish. Both examples come from Juana, the first one is about her father, the second about one of her daughters.

\ea\label{ex:Equative2}
\begingl 
\glpreamble Kwachu chija\\
\gla Kwachu chi-ija\\ 
\glb Juan 3-name\\ 
\glft ‘his name was Juan’
\trailingcitation{[jxx-p120515l-1.125]}
\xe

\ea\label{ex:name-1}
\begingl
\glpreamble chija Gladys\\
\gla chi-ija Gladys\\
\glb 3-name Gladys\\
\glft ‘her name is Gladys’
\endgl
\trailingcitation{[jxx-p110923l-2.059]}
\xe

There are a few more examples from Juana in which the order of predicate and subject is reversed, all of them either include proper names or contrastive topics\is{topic} as in (\ref{ex:propi-rev}). Note, however, that irrealis marking is on the subject (\textit{punachina}) in this case although it is related to the predicate (\textit{jente}).

\ea\label{ex:propi-rev}
\begingl
\glpreamble mm rusxhunubechÿ chichechajimÿnÿnube, kana, punachina jente\\
\gla mm rusxhu-nube-chÿ chi-checha-ji-mÿnÿ-nube kana punachÿ-ina jente\\
\glb \textsc{intj} two-\textsc{pl}-3 3-son-\textsc{col}-\textsc{dim}-\textsc{pl} this.size other-\textsc{irr.nv} man\\
\glft ‘mhm, she has two children, [one is] of this size (showing with hands) and the other one will be a man (soon)’
\endgl
\trailingcitation{[jxx-p110923l-1.241]}
\xe

The following examples show that a pronominal subject precedes the non-verbal predicate. (\ref{ex:EqPoss-nube}) was elicited from Miguel.

\ea\label{ex:EqPoss-nube}
\begingl 
\glpreamble nÿti chÿenunube\\
\gla nÿti chÿ-enu-nube\\ 
\glb 1\textsc{sg.prn} 3-mother-\textsc{pl}\\ 
\glft ‘I am their mother’
\trailingcitation{[mxx-e090728s-3.081]}
\xe

(\ref{ex:Prn-eq-1}) comes from María narrating the story of the fox and the jaguar. The fox makes the jaguar believe that the reflection of the moon in the water was a wheel of cheese.

\ea\label{ex:Prn-eq-1}
\begingl
\glpreamble “chibu echÿu kesu”\\
\gla chibu echÿu kesu\\
\glb 3\textsc{top.prn} \textsc{dem}b cheese\\
\glft ‘“this is the cheese”’
\endgl
\trailingcitation{[rxx-n120511l-1.037]}
\xe

The non-verbal predicate can also be a prepositional phrase as in (\ref{ex:EqClause-2}), which is a statement by Clara about her origin.

\ea\label{ex:EqClause-2}
\begingl 
\glpreamble nÿti tukiu nauku Santa Rita\\
\gla nÿti tukiu nauku {Santa Rita}\\ 
\glb 1\textsc{sg.prn} from there {Santa Rita}\\ 
\glft ‘I am from Santa Rita’
\trailingcitation{[cxx-e121130s.011]}
\xe

(\ref{ex:PropIncl-1}) and (\ref{ex:propi-2}) are two additional examples in which juxtaposition is accompanied by subject marking on the predicate. In (\ref{ex:PropIncl-1}), María C. talks about herself.

\ea\label{ex:PropIncl-1}
\begingl 
\glpreamble nÿti juberÿpunÿmÿnÿ\\
\gla nÿti juberÿpu-nÿ-mÿnÿ\\ 
\glb 1\textsc{sg.prn} old.woman-1\textsc{sg}-\textsc{dim}\\ 
\glft ‘poor me, I am an old woman’
\trailingcitation{[uxx-p110825l.038]}
\xe

(\ref{ex:propi-2}) is a confirmation of a sentence I had produced. It comes from María S.

\ea\label{ex:propi-2}
\begingl
\glpreamble biti paunakabi\\
\gla biti paunaka-bi\\
\glb 1\textsc{pl.prn} Paunaka-1\textsc{pl}\\
\glft ‘we are Paunakas’
\endgl
\trailingcitation{[rmx-e150922l.103]}
\xe

There are also some examples in which the person marker alone is used as subject expression.\is{person marking} Since there is no third person marker to index subjects of non-verbal predicates, third person singular subjects can be completely unmarked in non-verbal predication as in (\ref{ex:equa-3}).

(\ref{ex:ambi-eqprop}) comes from Juana who told me about her duty in \isi{Altavista}. Depending on the noun’s \isi{definiteness}, this sentence could be an example of proper inclusion (if it is indefinite) or as equative (if it is definite). NPs are not obligatorily marked for definiteness, so that only context or general knowledge can be used to distinguish both. In this case, I do not know whether Juana was the only cook or one among several (I guess the latter was the case, but this is speculation). 

\ea\label{ex:ambi-eqprop}
\begingl
\glpreamble asta nÿti niyunu, kosinerunÿ\\
\gla asta nÿti ni-yunu kosineru-nÿ\\
\glb until 1\textsc{sg.prn} 1\textsc{sg}-go cook-1\textsc{sg}\\
\glft ‘even I went, I was a/the cook’
\endgl
\trailingcitation{[jxx-p120515l-2.085-086]}
\xe

(\ref{ex:attri-3}) is another example with an adjective serving as predicate. It is the adjective with which one usually answers small-talk questions for one’s health and condition, but in this case María S.’  statement is not about herself but about me (in helping me formulate an adequate answer).

\ea\label{ex:attri-3}
\begingl
\glpreamble michachaikubi\\
\gla micha-chaiku-bi\\
\glb good-\textsc{cont}-2\textsc{sg}\\
\glft ‘you are fine’
\endgl
\trailingcitation{[rmx-e150922l.016]}
\xe

(\ref{ex:equa-3}) comes from Juana who provides some additional information about a woman she was talking about.

\ea\label{ex:equa-3}
\begingl
\glpreamble chikomarne Miyel\\
\gla chi-komar-ne Miyel\\
\glb 3-fellow-\textsc{possd} Miguel\\
\glft ‘she is Miguel’s fellow’
\endgl
\trailingcitation{[jxx-p120430l-2.342]}
\xe


Negation\is{negation|(} is rarely found among the semantic types described in this section, most of the examples I found in the corpus were elicited. If the subject is an SAP, the same negative particle we find in standard negation is used. Consider (\ref{ex:not-mother-1}) and (\ref{ex:not-mother-2}) which were elicited from Miguel and Juana respectively.

\ea\label{ex:not-mother-1}
\begingl
\glpreamble piti kuina nÿenuina\\
\gla piti kuina nÿ-enu-ina\\
\glb 2\textsc{sg.prn} \textsc{neg} 1\textsc{sg}-mother-\textsc{irr.nv}\\
\glft ‘you are not my mother’
\endgl
\trailingcitation{[rmx-e150922l.100]}
\xe

\ea\label{ex:not-mother-2}
\begingl
\glpreamble kuina nÿenubina\\
\gla kuina nÿ-enu-bi-ina\\
\glb \textsc{neg} 1\textsc{sg}-mother-2\textsc{sg}-\textsc{irr.nv}\\
\glft ‘you are not my mother’
\endgl
\trailingcitation{[jxx-p150920l.052]}
\xe

If the subject is a third person, there is a different strategy: a negative third person pronoun\is{pronoun!negative pronoun}\is{negation!negative pronoun} is used, composed of the third person index \textit{chÿ-} and the \isi{non-verbal irrealis marker} \textit{-ina}. It thus resembles both the \isi{topic pronoun} \textit{chibu} and the negative particle \textit{kuina}. Two examples are given below.

(\ref{ex:neg-eq-1}) comes from elicitation with María S.

\ea\label{ex:neg-eq-1}
\begingl
\glpreamble chÿina nÿenuina\\
\gla chÿina nÿ-enu-ina\\
\glb 3\textsc{neg.prn} 1\textsc{sg}-mother-\textsc{irr.nv}\\
\glft ‘she is not my mother’
\endgl
\trailingcitation{[rmx-e150922l.099]}
\xe

(\ref{ex:neg-eq-2}) stems from Miguel’s description of the \isi{frog story}. It refers to the picture on which the boy realises that what he was holding were not branches of a bush, but a deer’s antler.

\ea\label{ex:neg-eq-2}
\begingl
\glpreamble chÿinatu kÿkejina\\
\gla chÿina-tu yÿkÿke-ji-ina\\
\glb 3\textsc{neg.prn}-\textsc{iam} stick-\textsc{col}-\textsc{irr.nv}\\
\glft ‘now it wasn’t branches’
\endgl
\trailingcitation{[mox-a110920l-2.129]}
\xe

I have also found a few examples in which \textit{kuina} is used with third person referents in non-verbal predication. One of them is (\ref{ex:neg-eq-4}) from elicitation with Juana. She was asked to translate ‘he is not my brother’ but instead she gave the form for a sibling of the same sex, i.e. a sister in this case.

\ea\label{ex:neg-eq-4}
\begingl
\glpreamble kuina nipijina\\
\gla kuina ni-piji-ina\\
\glb \textsc{neg} 1\textsc{sg}-sibling-\textsc{irr.nv}\\
\glft ‘she is not my sister’
\endgl
\trailingcitation{[jxx-p150920l.053]}
\xe
\is{negation|)}
\is{equative/proper inclusion clause|)}

\subsection{Quantification}\label{sec:Quantification}
\is{quantification clause|(}

Predication of a quantity is achieved in the same way as equative, proper inclusion\is{equative/proper inclusion clause} and attributive predication, the only difference being that the predicate is a \isi{numeral} or quantifier\is{quantifier|(} in this case. However, since it seems to be a type rarely mentioned in the literature \citep[cf.][61]{Rose2018}, it deserves being treated in a bit more detail here. 

As is the case with the other types described above, a subject NP can be juxtaposed to the numeral or quantifier or a person marker\is{person marking} can index the subject.

(\ref{ex:QuantP-2}) was elicited from Juana.

\ea\label{ex:QuantP-2}
\begingl 
\glpreamble musumenubetu chimajinubetu\\
\gla musume-nube-tu chi-ima-ji-nube-tu\\ 
\glb many-\textsc{pl}-\textsc{iam} 3-husband-\textsc{col}-\textsc{pl}-\textsc{iam}\\ 
\glft ‘she has had many husbands (lit.: her husbands are many already)’
\trailingcitation{[jmx-e090727s.076]}
\xe
\is{quantifier|)}

(\ref{ex:QuantP-5}) also comes from Juana who speaks about the duration of her grandson’s studies at university.

\ea\label{ex:QuantP-5}
\begingl 
\glpreamble ruschÿtu anyo\\
\gla ruschÿ-tu anyo\\ 
\glb two-\textsc{iam} year\\ 
\glft ‘it is two years now (that he is in university)’
\trailingcitation{[jxx-p110923l-1.185]}
\xe

(\ref{ex:QuantP-3}) is a statement by Miguel after we came back from visiting José. Miguel was bitten by many ticks, while I only suffered a few tick bites.

\ea\label{ex:QuantP-3}
\begingl 
\glpreamble parikiyu samuchujane\\
\gla pariki-yu samuchu-jane\\ 
\glb many-\textsc{ints} tick.sp-\textsc{pl.nh}\\ 
\glft ‘there are a lot of ticks’
\trailingcitation{[mrx-c120509l.148]}
\xe

(\ref{ex:QuantP-1}) is an example in which the subject is indexed on the numeral. It is a statement by Juana about the number of the Supepí sisters, not counting the ones who had already passed away.

\ea\label{ex:QuantP-1}
\begingl
\glpreamble i nÿti, Maria, Krara, tresxhecheikubimÿnÿ tanÿma\\
\gla i nÿti Maria Krara tresxhe-cheiku-bi-mÿnÿ tanÿma\\
\glb and 1\textsc{sg.prn} María Clara three-\textsc{cont}-1\textsc{pl}-\textsc{dim} now\\
\glft ‘and me, María, Clara, we are only three now’
\endgl
\trailingcitation{[jxx-p120430l-2.352-353]}
\xe

One peculiarity of numerals\is{numeral|(} acting as predicates is that after a \isi{plural} marker, they can take a third person marker\is{person marking} following the stem. This marker is usually part of the numeral, but it undergoes lenition if the plural marker or some other morpheme (as in (\ref{ex:QuantP-1})) is added. A third person marker is then attached. Consider (\ref{ex:quant-2-1}) in which María S. first uses the numeral in juxtaposition to a nominal demonstrative and then again, in repetition of the predication on with plural and third person marker attached to it. By doing so, she corrected her own priorly uttered statement that I had three children.

\ea\label{ex:quant-2-1}
\begingl
\glpreamble ruschÿkena ekanube rusxhunubechÿ\\
\gla ruschÿ-kena eka-nube rusxhu-nube-chÿ\\
\glb two-\textsc{uncert} \textsc{dem}a-\textsc{pl} two-\textsc{pl}-3\\
\glft ‘they are probably two, they are two’
\endgl
\trailingcitation{[rmx-e150922l.078]}
\xe

(\ref{ex:three-3}) is from Juana and has a similar context as (\ref{ex:QuantP-1}) above, only that this time the sentence is about third person subjects and she counts the men in. Sadly to say, one of them has passed away since then.

\ea\label{ex:three-3}
\begingl
\glpreamble trexenubechÿ seunubenube i ruxhnubechÿ jentenube\\
\gla trexe-nube-chÿ seunube-nube i ruxh-nube-chÿ jente-nube\\
\glb three-\textsc{pl}-3 woman-\textsc{pl} and two-\textsc{pl}-3 men-\textsc{pl}\\
\glft ‘the women are three and the men are two’
\endgl
\trailingcitation{[jxx-p120515l-2.239]}
\xe

Finally, (\ref{ex:only-child}) shows a quantification clause with the numeral \textit{chÿnachÿ} ‘one’ to which the limitative marker \textit{-jiku} is attached. It is a question by María C. about the number of my children, when I first came to Santa Rita in 2011.

\ea\label{ex:only-child}
\begingl
\glpreamble ¿chÿnajiku pichecha?, kuina punachÿina\\
\gla chÿna-jiku pi-checha kuina punachÿ-ina\\
\glb one-\textsc{lim}1 2\textsc{sg}-son \textsc{neg} other-\textsc{irr.nv}\\
\glft ‘you have only one child?, there is no other’
\endgl
\trailingcitation{[uxx-p110825l.242-244]}
\xe
\is{numeral|)}
\is{quantification clause|)}


\subsection{Genitive and benefactive predication}\label{sec:GenitiveBenfactivePreds}\is{genitive/benefactive clause|(}

Two further minor types in non-verbal predication have been called genitive and benefactive by \citet[248]{Dryer2007}.

Genitive predication is different from possessive predication in that the existence of an item is presupposed, and the information conveyed is its relation to a possessor, while in possessive clauses,\is{possessive clause} the possessor is presupposed and the predication is about relating an item to it. Structurally, genitive predication is a subtype of equation or proper inclusion.\is{equative/proper inclusion clause} It only differs from them in that its focus is the possessive relation rather than identification of any kind.

There are a few examples in the corpus which build on the general relational noun\is{relational noun|(} \textit{-yae} as a predicate, which may be extended by a possessor NP if it has a third person possessor. A subject NP can be juxtaposed, but is omitted most of the times, since the subject is usually topical.\is{topic} The subject is never indexed on the predicate, because it always has an inanimate third person referent. All of these examples clearly focus on possessive relations rather than on identification.

(\ref{ex:GenP-2}) was elicited from Clara.

\ea\label{ex:GenP-2}
\begingl
\glpreamble niyae echÿu lote\\
\gla ni-yae echÿu lote\\
\glb 1\textsc{sg}-\textsc{grn} \textsc{dem}b plot\\
\glft ‘the plot is mine’
\endgl
\trailingcitation{[cux-c120414ls-1.104]}
\xe

(\ref{ex:GenP-3}) comes from Juana who talked about the different names a specific manor has had during the decades. Retiro was the place where Juan Ch., the consultant of Riester, used to live.

\ea\label{ex:GenP-3}
\begingl
\glpreamble aa Retiro estansiane chiyaebane mm Aurerio Castedo\\
\gla aa Retiro estansia-ne chi-yae-bane mm {Aurerio Castedo}\\
\glb \textsc{intj} Retiro manor-\textsc{possd}? 3-\textsc{grn}-\textsc{rem} \textsc{intj} {Aurelio Castedo}\\
\glft ‘ah as for the manor Retiro, it was mm Aurelio Castedo’s’
\endgl
\trailingcitation{[jxx-p120430l-2.019]}
\xe

(\ref{ex:GenP-1}) was elicited from Juana.

\ea\label{ex:GenP-1}
\begingl 
\glpreamble kuina niyaena, chiyae nima\\
\gla kuina ni-yae-ina chi-yae ni-ima\\ 
\glb \textsc{neg} 1\textsc{sg}-\textsc{grn}-\textsc{irr} 3-\textsc{grn} 1\textsc{sg}-husband\\ 
\glft ‘it is not mine (the sombrero), it is my husband's‘
\trailingcitation{[jxx-e081025s-1.123]}
\xe

(\ref{ex:GenP-4}) comes from Miguel and refers to something that was mine. I cannot say what it was in retrospect, because there is no video-recording.

\ea\label{ex:GenP-4}
\begingl
\glpreamble eka piyae\\
\gla eka pi-yae\\
\glb \textsc{dem}a 2\textsc{sg}-\textsc{grn}\\
\glft ‘this is yours’
\endgl
\trailingcitation{[mrx-c120509l.030]}
\xe

\is{relational noun|)}

Benefactive predicates are built on the \isi{general oblique} preposition \textit{tÿpi}. Just like genitive predication, it can be considered a subtype of the equative or proper inclusion type.\is{equative/proper inclusion clause} Unlike in equative or proper inclusion, the subject usually precedes the predicate,\is{word order} which may be due to influence of Spanish word order (but see (\ref{ex:BenP-3}) where the subject follows due to emphasis on an exclusive benefactive relation).

(\ref{ex:BenP-1}) and (\ref{ex:BenP-2}) were both produced by Juana in elicitation.

\ea\label{ex:BenP-1}
\begingl 
\glpreamble eka pitÿpi\\
\gla eka pi-tÿpi\\ 
\glb \textsc{dem}a 2\textsc{sg}-\textsc{obl}\\ 
\glft ‘this is for you’
\trailingcitation{[jmx-e090727s.067]}
\xe

\ea\label{ex:BenP-2}
\begingl
\glpreamble eka punachÿ tÿpi piati\\
\gla eka punachÿ tÿpi pi-ati\\
\glb \textsc{dem}a other \textsc{obl} 2\textsc{sg}-brother\\
\glft ‘the other one is for your brother’
\endgl
\trailingcitation{[jmx-e090727s.063]}
\xe

An example with a negated benefactive predicate is (\ref{ex:BenP-4}), which comes from Juana telling the story about the origin of some plants. This story mixes with the biblical creation story, so it is actually Jesus who speaks here, telling a monkey that the corn is not meant for it.

\ea\label{ex:BenP-4}
\begingl
\glpreamble “kuina pitÿpina”\\
\gla kuina pi-tÿpi-ina\\
\glb \textsc{neg} 2\textsc{sg}-\textsc{obl}-\textsc{irr.nv}\\
\glft ‘“it is not for you”’
\endgl
\trailingcitation{[jxx-n101013s-1.872]}
\xe

The long example of (\ref{ex:BenP-3}) comes from Miguel and has several benefactive predications in a row. He explains here why the pupils in \isi{Altavista} had wooden plates to write on in the old days: because paper was reserved for the \textit{karay}.

\ea\label{ex:BenP-3}
\begingl
\glpreamble kaku pero kuina, chitÿpijiku eka kayaraunube echÿu ajumerku kuadernu, \\chitÿpijikunube, chitÿpi eka jentenube naka o komunidades kuina\\
\gla kaku pero kuina chi-tÿpi-jiku eka kayarau-nube echÿu ajumerku kuadernu chi-tÿpi-jiku-nube chi-tÿpi eka jente-nube naka o komunidades kuina\\
\glb exist but \textsc{neg} 3-\textsc{obl}-\textsc{lim}1 \textsc{dem}a karay-\textsc{pl} \textsc{dem}b paper notebook 3-\textsc{obl}-\textsc{lim}1-\textsc{pl} 3-\textsc{obl} \textsc{dem}a man-\textsc{pl} here or communities \textsc{neg}\\
\glft ‘there was, but no, the paper and notebooks were only for the \textit{karay}, only for them, not for the people here or the communities’
\endgl
\trailingcitation{[mxx-p181027l-1.027-029]}
\xe

Calling this type of non-verbal predication “benefactive” suggests that the beneficiary is human or at least animate,\is{animacy} but we also find constructions of this type in which we have inanimate “beneficiaries”. Again, these examples resemble Spanish resemble Spanish word order very much, except that a copula is missing. Two examples follow.

(\ref{ex:BenP-5}) comes from the listing of several plants by María C. to teach me some vocabulary. I could not find out which tree \textit{kupaju} refers to.

\ea\label{ex:BenP-5}
\begingl
\glpreamble kupajumÿnÿ tÿpi bubiu\\
\gla kupaju-mÿnÿ tÿpi bi-ubiu\\
\glb tree.sp-\textsc{dim} \textsc{obl} 1\textsc{pl}-house\\
\glft ‘the \textit{kupaju} wood is for our houses’
\endgl
\trailingcitation{[uxx-p110825l.229]}
\xe

(\ref{ex:BenP-6}) was produced by Juana, when she told me about a house that they considered renting.

\ea\label{ex:BenP-6}
\begingl
\glpreamble ... pero mil bolivianos tÿpi entero ubiae\\
\gla pero mil bolivianos tÿpi entero ubiae\\
\glb but 1000 bolivianos \textsc{obl} whole house\\
\glft ‘...but it is 1000 bolivianos for the whole house’
\endgl
\trailingcitation{[jxx-p120430l-1.365-369]}
\xe

%kaku +
%-mÿnÿ
%-nube
%-jane
%-tu
%-kuÿ
%-ina
%-kena
%-ini
%-yenu
%-ji
%-jiku
%-yu
%-uku
%-kuiku?
\is{genitive/benefactive clause}
\is{juxtaposition|)}

\subsection{Location}\label{sec:LocativePredicates}
\is{locative clause|(}

Paunaka has a \isi{verb} to express location, \textit{-ubu} ‘be, live’,\is{copula|(} but it is mainly used with temporally stable locations like the place where somebody lives. In reference to temporary locations, speakers often prefer a non-verbal strategy. 

If the \isi{subject} of a locative predication is a third person, the copula \textit{kaku} ‘exist’ can be used. Just like the gloss suggests, \textit{kaku} is also found in existential predication.\is{existential clause|(} Indeed, there is semantic overlap between locative and existential predication. According to \citet[9]{Creissels2014a}, both provide different perspectives on how to encode the relationship between a figure and a ground, with locative predication tracking the figure and existential predication tracking the ground. 

In Paunaka, this distinction can be reflected in a different word order\is{word order|(} of the two constructions. In locative predication the locative expression directly follows the copula. If there is a conominal subject,\is{conomination} the latter can precede the copula as in (\ref{ex:LocP-1}) or follow the locative phrase as in (\ref{ex:LocP-3}). In existential predication, however, it is the subject that directly follows the copula. A locative expression can occur in these clauses but is not mandatory.\is{word order|)}\is{existential clause|)}

(\ref{ex:LocP-1}) comes from Juana who was speaking about several people in her social network, and told me where they lived.\footnote{This would actually be a context in which the verb \textit{-ubu} ‘be, live’ could well be used. I do not know why Juana preferred the copula here, but I suspect it has to do with the copula being much more frequent than the verb, thus gradually being extended to contexts of permanent location, too.}

\ea\label{ex:LocP-1}
\begingl 
\glpreamble i nikumarne kaku nauku Conceyae\\
\gla i ni-kumare-ne kaku nauku Conce-yae\\ 
\glb and 1\textsc{sg}-fellow-\textsc{possd} exist there Concepción-\textsc{loc}\\ 
\glft ‘and my fellow is there in Concepción’
\trailingcitation{[jxx-p110923l-2.133]}
\xe

(\ref{ex:LocP-3}) comes from the story about the enchanted cowherd told by Miguel. The spirit of the hill had taken away the cows to his world in the hill. When the cowherd finds out, he informs his wife, and she replies:

\ea\label{ex:LocP-3}
\begingl
\glpreamble “kakutu chiyikikiyae echÿu bakajane kakunubetu nauku”\\
\gla kaku-tu chiyikiki-yae echÿu baka-jane kaku-nube-tu nauku\\
\glb exist-\textsc{iam} hill-\textsc{loc} \textsc{dem}b cow-\textsc{distr} exist-\textsc{pl}-\textsc{iam} there\\
\glft ‘“the cows are in the hill now, they are there now”’
\endgl
\trailingcitation{[mxx-n151017l-1.64]}
\xe

I deliberately stated above that the difference between locative and existential predication\is{existential clause|(} \textit{can} be reflected in different \isi{word order}, because this is not always the case. Due to information structure, the locative expression can also sometimes precede the predicate. This is the case in (\ref{ex:loc-exi-1}), in which Juana connects to my statement that Federico was in Buenos Aires with the following:

\ea\label{ex:loc-exi-1}
\begingl
\glpreamble aa Buenos Aires, aa, nauku kaku nijinepÿi\\
\gla aa {Buenos Aires} aa nauku kaku ni-jinepÿi\\
\glb \textsc{intj} {Buenos Aires} \textsc{intj} there exist 1\textsc{sg}-daughter\\
\glft ‘ah Buenos Aires, ah, my daughter is there’
\endgl
\trailingcitation{[jxx-p110923l-1.104-107]}
\xe

In this case, it is only the context and general knowledge that helps to distinguish locative from existential\is{existential clause|)} and also possessive predication.\is{possessive clause} With general knowledge I refer to the fact that Juana most probably presupposed that I know she has a daughter, so that an existential or possessive reading is excluded. For the sake of simplicity and because it is often not totally clear what exactly a speaker had in mind when producing a sentence, I will not consider more examples like (\ref{ex:loc-exi-1}) here.

In locative predication, it is common that the \isi{subject} is topical\is{topic} and thus not conominated, as in the following examples. (\ref{ex:LocP-2}) comes from Juana and is about my cell phone. The first person possessor on the noun is related to the fact that this sentence was produced to correct my pronunciation.

\ea\label{ex:LocP-2}
\begingl 
\glpreamble kaku nipusaneyae\\
\gla kaku ni-pusane-yae \\ 
\glb exist 1\textsc{sg}-bag-\textsc{loc}\\ 
\glft ‘it is in my bag’
\trailingcitation{[jxx-p110923l-2.040]}
\xe

(\ref{ex:LocP-5}) was produced by María S. and is about Juana. The reason for her being in Santa Cruz was given before: she cares for her grandchildren and cooks.

\ea\label{ex:LocP-5}
\begingl
\glpreamble nechikue kaku nauku Santa Cruz\\
\gla nechikue kaku nauku {Santa Cruz}\\
\glb therefore exist there {Santa Cruz}\\
\glft ‘that’s why she is there in Santa Cruz’
\endgl
\trailingcitation{[rxx-e120511l.120]}
\xe

(\ref{ex:LocP-6}) comes from an elicitation session with wooden toy figures. It is Miguel’s answer to Alejo’s question where the wooden toy was.

\ea\label{ex:LocP-6}
\begingl
\glpreamble hm, kaku naka mÿbanejiku eka ubiae\\
\gla hm kaku naka mÿbane-jiku eka ubiae\\
\glb \textsc{intj} exist here close-\textsc{lim}1 \textsc{dem}a house\\
\glft ‘hm, it is here, close to the house’
\endgl
\trailingcitation{[mtx-e110915ls.47]}
\xe

So far, all examples were about the location of a third person referent. SAP referents cannot combine with the copula \textit{kaku}.\footnote{A probable explanation for this incompatibility is provided in \sectref{sec:Kapunu} below. Note that there is one counter-example in the corpus, which comes from Juana.} There are two alternative ways to predicate location of a first or second person: first, the locative copular suffix \textit{-uku} can be attached to a \isi{personal pronoun}. This suffix is exclusively found with personal pronouns. Second, the personal pronoun and the locative expression can be juxtaposed\is{juxtaposition} without any further marking of the relation between them. In any case, the personal pronoun comes first and the locative expression follows.\is{word order}

(\ref{ex:LocP-7}) to (\ref{ex:LocP-8}) illustrate the use of the locative suffix, and (\ref{ex:LocP1sg}) and (\ref{ex:LocP2sg}) the juxtaposition strategy.

(\ref{ex:LocP-7}) comes from the creation story as told by Juana. God has just asked Jesus where he was, thus the latter answers:

\ea\label{ex:LocP-7}
\begingl
\glpreamble “nÿtiuku naka”\\
\gla nÿti-uku naka\\
\glb 1\textsc{sg.prn}-\textsc{prn.loc} here\\
\glft ‘“I am here”’
\endgl
\trailingcitation{[jxx-n101013s-1.467]}
\xe

(\ref{ex:LocP-bitiuku}) comes from the recordings by Riester and is about Juan Ch. and his sister being the only ones of their family in Retiro.

\ea\label{ex:LocP-bitiuku}
\begingl 
\glpreamble rusxujikubinube bitiuku nakaja\\
\gla rusxu-jiku-bi-nube biti-uku naka-ja\\ 
\glb two-\textsc{lim}1-1\textsc{pl}-\textsc{pl} 1\textsc{pl.prn}-\textsc{prn.loc} here-\textsc{emph}1\\ 
\glft ‘only the two of us are here’
\trailingcitation{[nxx-p630101g-1.165]}
\xe

(\ref{ex:LocP-8}) was elicited from Juana. Note, however, that María S. does not accept the combination of the locative-marked pronoun\is{personal pronoun} with the adverb \textit{nauku}. According to her, it can only combine with the proximate \textit{naka} ‘here’.

\ea\label{ex:LocP-8}
\begingl
\glpreamble pitiuku nauku\\
\gla piti-uku nauku\\
\glb 2\textsc{sg.prn}-\textsc{prn.loc} there\\
\glft ‘you were there’
\endgl
\trailingcitation{[jxx-p150920l.108]}
\xe

(\ref{ex:LocP1sg}) comes from Isidro talking with Swintha. He contrasts his state of being at the place of conversation (here) in the first clause with his wife being alone on their field, which is expressed in the second clause with a verbal predicate.

\ea\label{ex:LocP1sg}
\begingl 
\glpreamble nÿti naka, tipÿisisikubu nauku\\
\gla nÿti naka ti-pÿisisikubu nauku\\ 
\glb 1\textsc{sg.prn} here 3i-be.alone there\\ 
\glft ‘I am here, she is alone there’
\trailingcitation{[dxx-d120416s.167]}
\xe

(\ref{ex:LocP2sg}) was produced by Juana, re-narrating what happened when we wanted to meet, but she came late. When she just left home, I had already arrived at the zoo, so I had to wait for her for quite some time there.

\ea\label{ex:LocP2sg}
\begingl 
\glpreamble i piti nauku zoolojikayae\\
\gla i piti nauku zoolojika-yae\\ 
\glb and 2\textsc{sg.prn} there zoo-\textsc{loc}\\ 
\glft ‘and you were there at the zoo’
\trailingcitation{[jxx-p110923l-2.044]}
\xe

Negation\is{negation|(} is achieved by the same negative particle \textit{kuina} that we also find in verbal clauses (see \sectref{sec:Negation}). In negated clauses, the copula takes the \isi{non-verbal irrealis marker} as in (\ref{ex:LocP-9}) or is omitted as in (\ref{ex:LocP-10}), with the latter being less common. All examples of negated locative predication I found in the corpus have third person referents.

(\ref{ex:LocP-9}) comes from Juana reporting what Miguel’s daughter had said when she asked her about her father.

\ea\label{ex:LocP-9}
\begingl
\glpreamble “kuina kakuina, tiyunu Santa Kuru”\\
\gla kuina kaku-ina ti-yunu {Santa Kuru}\\
\glb \textsc{neg} exist-\textsc{irr.nv} 3i-go {Santa Cruz}\\
\glft ‘“he is not here, he went to Santa Cruz”'
\endgl
\trailingcitation{[jxx-e150925l-1.126]}
\xe

(\ref{ex:LocP-10}) was produced by Miguel in telling the \isi{frog story} and it refers to the frog, which has left its glass.

\ea\label{ex:LocP-10}
\begingl
\glpreamble kuinabutu naka\\
\gla kuina-bu-tu naka\\
\glb \textsc{neg}-\textsc{dsc}-\textsc{iam} here\\
\glft ‘it is not here anymore’
\endgl
\trailingcitation{[mox-a110920l-2.039]}
\xe
\is{negation|)}

(\ref{ex:LocP-11}) comes from Miguel. It is a description of the first picture of the \isi{frog story}. He uses a locative clause first to introduce the boy into the discourse: the adverb \textit{naka} ‘here’ directly follows the copula.  Then he uses two existential clauses to introduce two additional referents, the dog and the glass. In these latter cases, the subjects follow the copula and the adverbs come last in the clause. Existential predication is the topic of the next section.

\ea\label{ex:LocP-11}
\begingl
\glpreamble kaku naka eka sepitÿmÿnÿ, kaku kabemÿnÿ naka, kakuku eka tachumÿnÿkena eka naka\\
\gla kaku naka eka sepitÿ-mÿnÿ kaku kabe-mÿnÿ naka kaku-uku eka tachu-mÿnÿ-kena eka naka\\
\glb exist here \textsc{dem}a child-\textsc{dim} exist dog-\textsc{dim} here exist-\textsc{add} \textsc{dem}a small.pot-\textsc{dim}-\textsc{uncert} \textsc{dem}a here\\
\glft ‘the boy is here, here is a little dog and here is what I suppose is a small pot’
\endgl
\trailingcitation{[mox-a110920l-2.006-007]}
\xe
\is{locative clause|)}

\subsection{Existentials}\label{sec:Existentials}
\is{existential clause|(}

Existential and locative clauses\is{locative clause} overlap in that both often express a spatial relation between a figure and a ground. Both prototypically encode “\textit{episodic} spatial relationships between a \textit{concrete} entity conceived as \textit{movable} (the figure) and another concrete entity (the ground) conceived as occupying a fixed position in the space, or at least as being \textit{less easily movable} than the figure” \citep[10]{Creissels2014a}.
 Existential clauses, however, provide a different perspective on the spatial relation as locative clauses do,\is{locative clause} i.e. a perspective from the ground, not the figure \citep[9, 18]{Creissels2014a}.
%“selection of the ground as the perspectival center in clauses encoding figure-ground relationships” (18)
This is why they do not serve as “adequate answers to questions about the location of an entity, but can be used to identify an entity present at a certain location” \citep[2]{Creissels2014a}.

(\ref{ex:Exi-1}) and (\ref{ex:Exi-2}) are examples for prototypical existential clauses resembling the ones presented by \citet[]{Creissels2014a}: they have an indefinite referent,\is{definiteness} which is the moveable figure on a relatively fixed ground. Both examples were elicited, (\ref{ex:Exi-1}) comes from María S., (\ref{ex:Exi-2}) from Juana.

\ea\label{ex:Exi-1}
\begingl
\glpreamble kaku jike nikusepineyae\\
\gla kaku jike ni-kusepi-ne-yae\\
\glb exist fly 1\textsc{sg}-thread-\textsc{possd}-\textsc{loc}\\
\glft ‘there is a fly on my thread’
\endgl
\trailingcitation{[rxx-e181024l.092]}
\xe

\ea\label{ex:Exi-2}
\begingl
\glpreamble kaku ÿne chÿupekÿyae keyu\\
\gla kaku ÿne chÿ-upekÿ-yae keyu\\
\glb exist water 3-place.under-\textsc{loc} snail\\
\glft ‘there is water under the snail’
\endgl
\trailingcitation{[jcx-e090727s.035]}
\xe


Thus an existential construction exists in Paunaka, and in many cases it can be distinguished from locative predication\is{locative clause} by placement of the subject directly following the copula.\is{word order}\footnote{Note, however, that manipulation of word order does not count as “a dedicated existential construction” but is rather analysed as an equivalent by \citet[19]{Creissels2014a}.} This, however, is not the only ambit of this construction type, and probably even not its main one. Consider (\ref{ex:Exi-33}), which shows that the existential construction is not restricted to indefinite referents\is{definiteness|(} in Paunaka.\footnote{Note that although indefiniteness is often explicitly or implicitly involved in the definition of an existential construction, \citet[4]{Creissels2014a} has shown that in some languages existential constructions are not restricted to indefinite subjects.} It was produced by María C. when it was about to rain. The hammock is mentioned here for the first time. This is probably the reason why María C. chose an existential construction rather than a locative one with the spatial expression following the copula directly.

\ea\label{ex:Exi-33}
\begingl
\glpreamble kaku niyumaji nekupai\\
\gla kaku ni-yumaji nekupai\\
\glb exist 1\textsc{sg}-hammock outside\\
\glft ‘my hammock is outside (i.e. there is my hammock outside)’
\endgl
\trailingcitation{[cux-120410ls.258]}
\xe

A second example with a definite subject is (\ref{ex:Exi-3}) from Juana. The cows she is speaking about were already well established in the story, which was about her grandparents buying cows in Moxos. However, they had not been mentioned for some time. The existential construction is thus used here to re-establish the cows as a topic. The location mentioned in this sentence is an enclosure by a hut where Juana’s grandparents slept on their way home, so it can be considered a temporary, episodic location rather than a permanent one.

\ea\label{ex:Exi-3}
\begingl
\glpreamble i kaku baka bakayayae\\
\gla i kaku baka bakaya-yae\\
\glb and exist cow enclosure-\textsc{loc}\\
\glft ‘and the cows were in the enclosure (i.e. there were the cows in the enclosure)’
\endgl
\trailingcitation{[jxx-p151016l-2.185]}
\xe


The Paunaka existential construction also relates to two types that \citet[]{Creissels2014} explicitly distinguishes from existential predication, although he recognises that they are related and in some languages encoded by the same construction.

First, the Paunaka construction serves a presentative function, i.e. the introduction of participants into the discourse \citep[cf.][15]{Creissels2014a}. Actually, the presentative and the “prototypical” existential construction both introduce an indefinite referent into the discourse and thus only differ in presence or absence of a locative expression in the clause. It is because of this presentative type that existential predication partly overlaps with possessive predication\is{possessive clause} in Paunaka (see \sectref{sec:PossessiveClauses}). 

Second, the existential construction in Paunaka also encodes “habitual presence of an entity at some place” \citet[14]{Creissels2014a}. The actual place does not have to be overtly expressed in this case if it is identifiable from the context or identical to the deictic centre. This is to say that if I speak of existential predication (or an existential clause or construction) in this work, this includes also presentatives as well as expressions of \isi{habitual} presence. My usage of the term is thus more conform with the broader definition of existential predication given by \citet[123--125]{Payne1997} or \citet[240--244]{Dryer2007}.  

A last word about the notion of “subject”\is{subject} is necessary. I agree with \citet[9]{Overall2018} who state that “[t]he indefinite participant introduced in the existential construction often lacks some of the grammatical properties of a prototypical subject, but even so, there is usually no other argument available as a candidate to be the subject”.

With this general background in mind, I turn to a few more examples of existential predication in Paunaka now.

With (\ref{ex:exist-1}), Juana introduced the arroyo close to Santa Rita into the discourse as a place where the young people go swimming. This is another example of a definite subject being introduced by an existential clause.

\ea\label{ex:exist-1}
\begingl 
\glpreamble nauku Santa Ritayae kaku echÿu chÿkÿ\\
\gla nauku {Santa Rita}-yae kaku echÿu chÿkÿ\\ 
\glb there {Santa Rita}-\textsc{loc} exist \textsc{dem}b arroyo\\ 
\glft ‘there in Santa Rita, there is this arroyo’
\trailingcitation{[jxx-a120516l-a.571]}
\xe
\is{definiteness|)}

(\ref{ex:Exi-4}) is a typical beginning of a story by Miguel. A participant is introduced into the discourse here. There is no locative expression, but the whole story is posited in the \isi{remote past} by the remote marker \textit{-bane} being attached to the copula.

\ea\label{ex:Exi-4}
\begingl
\glpreamble kakubaneji chÿnachÿ jente i tipÿkubai\\
\gla kaku-bane-ji chÿnachÿ jente i ti-pÿkubai\\
\glb exist-\textsc{rem}-\textsc{rprt} one man and 3i-be.lazy\\
\glft ‘once upon a time there was a man, it is said, and he was lazy’
\endgl
\trailingcitation{[mox-n110920l.011]}
\xe

The copula can also take the \isi{iamitive} marker to contrast the state of existence of a referent with the time prior to this existence. This is the case in (\ref{ex:Exi-5}) from Miguel. He was talking about the history of Santa Rita and his own personal history and had just abbreviated his more detailed account by simply telling me that several years turned by until:

\ea\label{ex:Exi-5}
\begingl
\glpreamble i kakutu echÿu nuebo presidente de Bolivia\\
\gla i kaku-tu echÿu {nuebo presidente de Bolivia}\\
\glb and exist-\textsc{iam} \textsc{dem}b {new president of Bolivia}\\
\glft ‘and then there was this new president of Bolivia’
\endgl
\trailingcitation{[mxx-p110825l.035]}
\xe

(\ref{ex:exist-tu}) comes from Juana and is an example of habitual presence. The subject of this clause is complex. It is an equative clause\is{equative/proper inclusion clause} with two NPs in juxtaposition, which both mean ‘pot’. The difference is that \textit{nÿkÿiki} is a Paunaka word, and because of its native origin it is here associated with traditional (clay) pots. \textit{Uyetaki} is a loan from Spanish \textit{olleta} ‘pot’ with the \isi{classifier} for round objects \textit{-ki} attached to it. The Spanish loan is associated with modern pots made of aluminium.

\ea\label{ex:exist-tu}
\begingl 
\glpreamble metu kakutu eka nÿkÿiki uyetaki\\
\gla metu kaku-tu eka nÿkÿiki uyeta-ki\\ 
\glb already exist-\textsc{iam} \textsc{dem}a pot aluminium.pot-\textsc{clf}:spherical\\ 
\glft ‘now there are these modern pots (i.e. the pots that are aluminium pots)’
\trailingcitation{[jxx-d110923l-2.41]}
\xe

Another sentence representing habitual presence is (\ref{ex:Exi-6}) from María S. Although a location is specified here, this sentence is not understood as encoding an episodic spatial relation, because the general context was that the Supepí siblings had left their old house after their father passed away, and it was only their mother who stayed in the old house, permanently.

\ea\label{ex:Exi-6}
\begingl
\glpreamble depue kakukuÿbane nÿenubane primero nubiu nauku\\
\gla depue kaku-kuÿ-bane nÿ-enu-bane primero nÿ-ubiu nauku\\
\glb afterwards exist-\textsc{incm}-\textsc{rem} 1\textsc{sg}-mother-\textsc{rem} first 1\textsc{sg}-house there\\
\glft ‘afterwards there was still my late mother in my first house there long time ago’
\endgl
\trailingcitation{[rxx-e120511l.172]}
\xe

I have found but one example in the corpus, in which an existential predication is about a non-third-person referent. It is given in (\ref{ex:Exi-7}). Just like locative clauses, this sentence is realised without a copula. It comes from María S. telling me that it was only her family that lived in the specific place they used to live before most of the siblings moved to Santa Rita, to Concepción or elsewhere.

\ea\label{ex:Exi-7}
\begingl
\glpreamble kuina, bitiyÿchi nauku\\
\gla kuina biti-yÿchi nauku\\
\glb \textsc{neg} 1\textsc{pl.prn}-\textsc{lim}2 there\\
\glft ‘no, it was only us there’
\endgl
\trailingcitation{[rxx-p181101l-2.130]}
\xe

Negative existential clauses\is{negation|(} require the negative particle \textit{kuina}. They can be formed with or without a copula. The copula usually shows up if the \isi{subject} is not conominated,\is{conomination} as in (\ref{ex:neg-exist-cond}) or in negative answers. The non-verbal irrealis marker is always attached to negated \textit{kaku} in this case.

(\ref{ex:neg-exist-cond}) was elicited from Miguel and referred to the fact that Federico had bought some food for our picnic in \isi{Altavista}.

\ea\label{ex:neg-exist-cond}
\begingl 
\glpreamble kue kuina tiyÿseika, kuina kakuina naka\\
\gla kue kuina ti-yÿseika kuina kaku-ina naka\\ 
\glb if \textsc{neg} 3i-buy.\textsc{irr} \textsc{neg} exist-\textsc{irr.nv} here\\ 
\glft ‘if he hadn’t bought it, there wouldn’t be anything (to eat) here’
\trailingcitation{[mxx-n120423lsf-X.45]}
\xe

Otherwise, the use of the copula in negative clauses is rare, although it does occur sometimes as in (\ref{ex:kuina-kakuina-1}). This example was elicited from Miguel, but it seems to over-emphasise the non-existence a bit.

\ea\label{ex:kuina-kakuina-1}
\begingl 
\glpreamble kuina kakuina menonitanube\\
\gla kuina kaku-ina menonita-nube\\ 
\glb \textsc{neg} exist-\textsc{irr.nv} Menonite-\textsc{pl}\\ 
\glft ‘there are no Menonites (in Beni)’
\trailingcitation{[jmx-e090727s.357]}
\xe

Usually, the negative existential clause can do without a copula, as in examples (\ref{ex:Exi-8}) and (\ref{ex:Exi-9}).

(\ref{ex:Exi-8}) comes from Juana, who told me about the circumstances of the encounter of María S. and her husband with a snake (or water spirit) in the reservoir of Santa Rita.

\ea\label{ex:Exi-8}
\begingl
\glpreamble tipÿsisikubunube kuina kristianunubeina\\
\gla ti-pÿsisikubu-nube kuina kristianu-nube-ina\\
\glb 3i-be.alone-\textsc{pl} \textsc{neg} person-\textsc{pl}-\textsc{irr.nv}\\
\glft ‘they were alone, there were no people’
\endgl
\trailingcitation{[jxx-p120515l-2.145]}
\xe

\largerpage[-1]
(\ref{ex:Exi-9}) is a summary of the climax of the story about the jaguar and the fox narrated by María S. The fox had made the jaguar believe that the reflection of the moon in the water was a wheel of cheese and the jaguar had drowned in trying to get hold of the suspected cheese.

\ea\label{ex:Exi-9}
\begingl
\glpreamble kuina kesuina, kujejiku\\
\gla kuina kesu-ina kuje-jiku\\
\glb \textsc{neg} cheese-\textsc{irr.nv} moon-\textsc{lim}1\\
\glft ‘there wasn’t any cheese, it was only the moon’
\endgl
\trailingcitation{[rxx-n120511l-1.044]}
\xe
\is{negation|)}

\subsection{Possessive clauses}\label{sec:PossessiveClauses}
\is{possessive clause|(}

The non-verbal possessive clause is a type of existential clause (see \sectref{sec:Existentials} above). Just like in the latter, in positive possessive clauses, there is a copula directly followed by the subject, while in negative clauses,\is{negation} the copula can be omitted.\is{word order} The difference to existential clauses is that the subject is marked as possessed in some way. A locative expression is not required in possessive clauses, but as we have just seen, a locative expression does not necessarily occur in existential clauses either. The main reason to describe possessive clauses in its own section here is that unlike existence, possession can also be expressed by a verbal strategy which builds on a \isi{verb} composed of the \isi{attributive prefix} \textit{ku-} and a \isi{nominal stem} (see \sectref{sec:AttributiveVerbs} for the verbal expression of possession).

(\ref{ex:Possi-1}) to (\ref{ex:PossP-3}) show non-verbal possessive predication build on an inalienably possessed noun in positive clauses.

(\ref{ex:Possi-1}) comes from the creation story as told by Juana. It is Jesus who had a field.

\ea\label{ex:Possi-1}
\begingl
\glpreamble kaku chisane\\
\gla kaku chi-sane\\
\glb exist 3-field\\
\glft ‘he had a field’ (lit.: ‘there was his field’)
\endgl
\trailingcitation{[jxx-n101013s-1.555]}
\xe

(\ref{ex:Possi-2}) was produced by Clara who was trying to remember the name of the fish that bites.

\ea\label{ex:Possi-2}
\begingl
\glpreamble kaku chija echÿu\\
\gla kaku chi-ija echÿu\\
\glb exist 3-name \textsc{dem}b\\
\glft ‘it has a name’
\endgl
\trailingcitation{[cux-c120414ls-1.217]}
\xe

With (\ref{ex:PossP-3}), María C. made a judgement about the capacity of Clara’s daughters to learn Paunaka.

\newpage
\ea\label{ex:PossP-3}
\begingl 
\glpreamble kaku pijinejinube pero kuina puero chitanube\\
\gla kaku pi-jine-ji-nube pero kuina puero chi-ita-nube\\ 
\glb exist 2\textsc{sg}-daughter-\textsc{col}-\textsc{pl} but \textsc{neg} can 3-master.\textsc{irr}-\textsc{pl}\\ 
\glft ‘you have daughters, but they can't figure it out (to speak Paunaka)’
\trailingcitation{[cux-c120414ls-2.265]}
\xe
% maybe this is an exitential??


The inalienably possessed noun may also be derived by the possessed marker as in (\ref{ex:Possi-3}). This is often the case with Spanish loans. The example comes from María S. telling the story about how the tortoise got its carapace. She did not want to leave her house to welcome new-born Jesus, because she had a shop in that house.

\ea\label{ex:Possi-3}
\begingl
\glpreamble pimua, kaku chibentane\\
\gla pi-imua kaku chi-benta-ne\\
\glb 2\textsc{sg}-see.\textsc{irr} exist 3-shop-\textsc{possd}\\
\glft ‘you see, she had a shop’
\endgl
\trailingcitation{[rxx-n121128s.17]}
\xe

%The example was produced by Juana to provide some information about an old lady she once met in Candelaria.

%\ea\label{ex:Possi-33}
%\begingl
%\glpreamble kaku chibastunemÿnÿtu\\
%\gla kaku chi-bastun-ne-mÿnÿ-tu\\
%\glb exist 3-walking.cane-\textsc{possd}-\textsc{dim}-\textsc{iam}\\
%\glft ‘she already had a walking cane’\\
%\endgl
%\trailingcitation{[jxx-p120515l-1.220]}
%\xe

(\ref{ex:Possi-4}) is an example of a negative possessive clause without copula. It is a statement by María C. about being all alone, without any siblings.

\ea\label{ex:Possi-4}
\begingl
\glpreamble kuina nÿatimÿnÿina nipijina \\
\gla kuina nÿ-ati-mÿnÿ-ina ni-piji-ina\\
\glb \textsc{neg} 1\textsc{sg}-brother-\textsc{dim}-\textsc{irr.nv} 1\textsc{sg}-sibling-\textsc{irr.nv}\\
\glft ‘I don’t have a brother or sister’
\endgl
\trailingcitation{[uxx-p110825l.074]}
\xe

The possessive clause can contain a locative expression as in (\ref{ex:PossP-1}) and (\ref{ex:PossP-2}).

(\ref{ex:PossP-1}) was elicited from Juana.

\ea\label{ex:PossP-1}
\begingl 
\glpreamble ¿kaku pubiu nauku pisaneyae?\\
\gla kaku pi-ubiu nauku pi-sane-yae\\ 
\glb exist 2\textsc{sg}-house there 2\textsc{sg}-field-\textsc{loc}\\ 
\glft ‘do you have a house at your field?’
\trailingcitation{[jmx-e090727s.352]}
\xe

(\ref{ex:PossP-2}) also comes from Juana who was making a statement about her daughter here.

\ea\label{ex:PossP-2}
\begingl 
\glpreamble kaku ruschÿ chilotene nauku\\
\gla kaku ruschÿ chi-lote-ne nauku\\ 
\glb exist two 3-plot-\textsc{possd} there\\ 
\glft ‘she has two plots there’
\trailingcitation{[jxx-p110923l-1.421]}
\xe

Instead of marking the possession directly on the noun, the \isi{possessor} can also be expressed by a person-marked\is{person marking} preposition directly following the possessed entity. The causal and instrumental\is{instrument/cause} preposition \textit{-keuchi} is used if the possessed is a concrete object, the \isi{general oblique} preposition \textit{-tÿpi} for possession of more abstract entities, usually some temporal units. In the latter case, it is arguable whether the clause can be analysed as a possessive clause at all or rather counts as existential, depending on the question whether temporal units can be considered as being possessed. This, however, is a philosophical rather than a linguistic question, because there is no difference in structure between existential and possessive clauses anyway.

(\ref{ex:PossP-keuchi-1}) is particularly interesting, because it contains a kind of \isi{secondary possession} of an inalienably possessed noun. This noun, \textit{chÿeche} ‘meat’, has a third person marker\is{person marking} by default if used to denote meat as an edible good (\textit{chÿ-eche} 3-flesh). The third person marker can be replaced by an SAP person marker in reference to the flesh of the body (e.g. \textit{nÿ-eche} ‘my flesh’). Since there is already a person marker on the noun denoting ‘meat’, attachment of a second possessor marker is blocked and another way of expressing the possessor is needed. In possessive predication, this is achieved by using the preposition \textit{-keuchi} which carries the person marker of the possessor. The example comes from Juan Ch. who was recorded by Riester and speaks about consequences of a successful hunting expedition.

\ea\label{ex:PossP-keuchi-1}
\begingl 
\glpreamble tanÿmapaiku kaku chÿeche nikeuchi nubiuyae tÿpi chÿnachÿ semana\\
\gla tanÿma-paiku kaku chÿeche ni-keuchi nÿ-ubiu-yae tÿpi chÿnachÿ semana\\ 
\glb now-\textsc{punct} exist meat 1\textsc{sg}-\textsc{ins} 1\textsc{sg}-house-\textsc{loc} \textsc{obl} one week\\ 
\glft ‘right now I have meat for one week in my house’
\trailingcitation{[nxx-a630101g-1.56]}
\xe

(\ref{ex:PossP-keuchi-2}) shows the same construction. In this case, I think it might also be possible to derive an inalienably possessed noun,\footnote{Animals are in general not directly possessable with a few exceptions concerning parasites. In the case of shells, however, I can imagine that a possessed form could be derived at least if reference is not to the mussel as an animal but to its shell as is the case in (\ref{ex:PossP-keuchi-2}). This remains to be verified.} but the possessive relation between a shell and a possessor is not a permanent one, unlike the relation to family members, fields or walking canes. Thus a construction with \textit{-keuchi} is preferred. The sentence comes from Miguel who was asking Juana about a special kind of shell which they use to polish pottery before burning.

\ea\label{ex:PossP-keuchi-2}
\begingl 
\glpreamble ¿pero kaku nauku sipÿ pikeuchi?\\
\gla pero kaku nauku sipÿ pi-keuchi\\ 
\glb but exist there shell 2\textsc{sg}-\textsc{ins}\\ 
\glft ‘but do you have shells (for polishing clay) there?’
\trailingcitation{[jmx-d110918ls-1.098]}
\xe

(\ref{ex:Poss-no-banana}) is a negative possessive clause including \textit{-keuchi}. It comes from María C. who said this to me regretfully, because I had told her that my little daughter wanted a plantain.\footnote{Actually I had intended to tell María C. that my daugther \textit{liked} the plantains, when she was crawling around pointing to things and uttering one-word clauses. Apparently, she found some plantains particularly interesting.}

\ea\label{ex:Poss-no-banana}
\begingl 
\glpreamble kuinachu merÿna nikeuchi\\
\gla kuina-chÿu? merÿ-ina ni-keuchi\\ 
\glb \textsc{neg}-\textsc{dem}b? plantain-\textsc{irr.nv} 1\textsc{sg}-\textsc{ins}\\ 
\glft ‘I don’t have plantains’
\trailingcitation{[uxx-p110825l.173]}
\xe

Turning to the use of \textit{-tÿpi} in possessive predication now, consider (\ref{ex:Possi-5}). It was elicited from Isidro and is about the age of an invented person.

\ea\label{ex:Possi-5}
\begingl 
\glpreamble metu kakutu nobenta anyo chitÿpi\\
\gla metu kaku-tu nobenta anyo chi-tÿpi\\ 
\glb already exist-\textsc{iam} ninety year 3-\textsc{obl}\\ 
\glft ‘she was already 90 years old’
\trailingcitation{[dxx-d120416s.203]}
\xe

(\ref{ex:Possi-6}) is from the recordings made by Riester with Juan Ch.

\ea\label{ex:Possi-6}
\begingl
\glpreamble tanÿma uchuini kaku tiempo nitÿpi\\
\gla tanÿma uchuine? kaku tiempo ni-tÿpi\\
\glb now just.now? exist time-\textsc{irr.nv} 1\textsc{sg}-\textsc{obl}\\
\glft ‘now I have time’
\endgl
\trailingcitation{[nxx-p630101g-1.012]}
\xe

(\ref{ex:Possi-7}) is a negated version of (\ref{ex:Possi-6}) and comes from María S. who provides the reason why she has not finished knotting a hammock.

\ea\label{ex:Possi-7}
\begingl
\glpreamble kuina tiempoina nÿtÿpi\\
\gla kuina tiempo-ina nÿ-tÿpi\\
\glb \textsc{neg} time-\textsc{irr.nv} 1\textsc{sg}-\textsc{obl}\\
\glft ‘I didn’t have time’
\endgl
\trailingcitation{[rxx-e181022le]}
\xe

Another negative possessive clause including \textit{-tÿpi} comes from Juan C. who was talking about the past with Miguel and stated here that he had a very hard life.

\ea\label{ex:Exi-10}
\begingl
\glpreamble tijainube tijainube kuina ruminkuina nitÿpi\\
\gla tijai-nube tijai-nube kuina ruminku-ina ni-tÿpi\\
\glb day-\textsc{pl} day-\textsc{pl} \textsc{neg} Sunday-\textsc{irr.nv} 1\textsc{sg}-\textsc{obl}\\
\glft ‘every single day (I worked), there was no Sunday for me’
\endgl
\trailingcitation{[mqx-p110826l.467]}
\xe

There are also some cases of formally existential clauses, i.e. clauses that do not include any marking of possession, but imply a possessive relation nonetheless. One example is given in (\ref{ex:Possi-8}), where existence of rice implies possession of rice. It comes from Miguel.

\ea\label{ex:Possi-8}
\begingl
\glpreamble kue kaku arusu banau pan de arroz\\
\gla kue kaku arusu bi-anau {pan de arroz}\\
\glb if exist rice 1\textsc{pl}-make {rice bread}\\
\glft ‘when there is rice, we make rice bread’
\endgl
\trailingcitation{[mxx-d120411ls-1a.042]}
\xe

\is{possessive clause|)}
\is{existential clause|)}
\is{copula|)}

\subsection{Similative and related construction}\label{sec:SimilativePreds}
\is{similative clause|(}

The similative construction is relatively simple in Paunaka. It includes a comparee, a standard marker and a standard \citep[cf.][]{HaspelmathBuchholz1998}, that is, there are clauses of the type ‘X is like Y’ in Paunaka, in which \textit{X} is the comparee, \textit{like} the standard marker and \textit{Y} the standard. The standard marker is \textit{nena} ‘like, resemble, be like’ in Paunaka. Specific parameters of comparison are not included in the cosntruction. Thus sentences equivalent to ‘She is as old as me’, which I will call “equality sentences”, do not exist.\footnote{\citet[277]{HaspelmathBuchholz1998} use the term “equative construction”, but this term is already applied to a different construction here (see \sectref{sec:PropInclEquatAttr} above).} Such concepts are rather expressed by more than one clause, which do not convey exactly the same meaning.\footnote{This is also in line with what \citet[]{Rose2019c} found for Trinitario.\is{Mojeño Trinitario}} There are a few cases that resemble sentences like ‘She jumps like a frog’. This is the kind of sentence that has been described under the realm of “similative construction” by \citet[277]{HaspelmathBuchholz1998}, I thus apply the term a bit differently here. I believe though that such concepts are also often expressed by two clauses. Spanish influence may play a role here in that originally biclausal structures are re-interpreted as monoclausal, because \textit{nena} is used as a translational equivalent of \textit{como} ‘like’, \textit{parece} ‘it seems, resembles’ and \textit{igual que} ‘equal to’.\footnote{Because the question will probably arise at this point as to how comparative constructions look like in Paunaka, and since there is no other place in the grammar, where they would be described, I give a very short summary here: Comparative constructions include the adverb \textit{max} ‘more, most’, a loan\is{borrowing} from Spanish \textit{más} which has the same meaning. The adverb is placed before the word expressing a property or quality. This may be an adjective, a noun or a stative verb. Comparative constructions never include a standard, i.e. the item that is surpassed. This is rather deduced from the context. (\ref{ex:comp-adj}) is an example with an adjective and thus also a case of non-verbal predication. It comes from Juana who was conversing with María S.

\ea\label{ex:comp-adj}
\begingl
\glpreamble amukeyu max michaniki\\
\gla amukeyu max michaniki\\
\glb soft.corn more delicious\\
\glft ‘soft corn is more delicious’
\endgl
\trailingcitation{[jrx-c151001lsf-11.184]}
\xe}

Depending on topicality,\is{topic} it is possible that the comparee or the standard are not overtly expressed in the similative clause, but if there is a standard NP, it follows the standard marker \textit{nena} ‘like, resemble, be like’ directly.\is{word order} I think it is generally not possible to index a subject directly, but there are a few counter-examples in the corpus. It is common though that the \isi{additive} marker is added to \textit{nena}, and in this case, a subject index can follow the additive marker (see below in this section).

(\ref{ex:nena-2}) is an example in which both comparee and standard are expressed by NPs. It comes from María C. who uses a \isi{Bésiro} word to refer to a specific tree with dark, blood-like resin.

\ea\label{ex:nena-2}
\begingl 
\glpreamble echÿu tokoxhirx nena iti\\
\gla echÿu tokoxhirx nena iti\\ 
\glb \textsc{dem}b tree.sp like blood\\ 
\glft ‘the (resin of the) \textit{tokoxhirxh} tree is like blood’
\trailingcitation{[ump-p110815sf.366]}
\xe

The comparee can also follow the standard as in (\ref{ex:nena-1}), which is from the data collected by Riester in the 1960s. Juan Ch. compares the \textit{pututu} soup with chicha here, i.e. the soup is not well garnished.

\ea\label{ex:nena-1}
\begingl 
\glpreamble nenayu aumue bijiemÿnÿjini\\
\gla nena-yu aumue abijie-mÿnÿ-ji-ini\\ 
\glb like-\textsc{ints} chicha pututu-\textsc{dim}-\textsc{rprt}-\textsc{frust}\\ 
\glft ‘the so-called \textit{pututu} soup is (thin) like chicha’
\trailingcitation{[nxx-p630101g-2.58]}
\xe

(\ref{ex:nena-3}) comes from Juana and is a description of the spirit of the water, with whom her grandparents had an unpleasant encounter on their way back home from Moxos, where they had bought cows.

\ea\label{ex:nena-3}
\begingl
\glpreamble kananaji chikebÿke, nenayuji kuje chibÿke\\
\gla kanana-ji chi-kebÿke nena-yu-ji kuje chi-bÿke\\
\glb this.size-\textsc{rprt} 3-eye like-\textsc{ints}-\textsc{rprt} moon 3-face\\
\glft ‘she had big eyes, her face was like the moon, it is said’
\endgl
\trailingcitation{[jxx-p151016l-2.091]}
\xe

In (\ref{ex:nena-4}), there are two juxtaposed clauses. The comparee is expressed in the first clause, a possessive one. The similative clause follows, the comparee is not repeated. The fact that the parameter is the age has to be deduced from the context. The sentence comes from Juana who was talking about her relatives.

\ea\label{ex:nena-4}
\begingl
\glpreamble i kaku echÿu chichechapÿi nena eka nisinepÿi\\
\gla i kaku echÿu chi-chechapÿi nena eka ni-sinepÿi\\
\glb and exist \textsc{dem}b 3-son like \textsc{dem}a 1\textsc{sg}-grandchild\\
\glft ‘and she had a son, who was like my grandson (in age)’
\endgl
\trailingcitation{[jxx-p120430l-2.163]}
\xe

In (\ref{ex:nena-6}), the comparee is incorporated\is{incorporation} into the verb that precedes the similative clause. The sentence comes from Juana who reported what the old lady she met in Candelaria long ago said when some chicha dripped on her face.

\ea\label{ex:nena-6}
\begingl
\glpreamble “nijirebÿketu nenayu chÿbÿke iyu”\\
\gla ni-jire-bÿke-tu nena-yu chÿ-bÿke iyu\\
\glb 1\textsc{sg}-wrinkle-face-\textsc{iam} like-\textsc{ints} 3-face monkey\\
\glft ‘“my face wrinkled, it looks like the face of a monkey”’
\endgl
\trailingcitation{[jxx-p120515l-1.075]}
\xe

On the other hand, in (\ref{ex:nena-5}), it is the standard which is not expressed. The sentence comes from a conversation between María S. and Juana. They were just talking about keeping ducks and Juana had mentioned that ducks are dirty, because they just squat and defecate everywhere and their excrements are liquid like diarrhea. María S. adds to this:

\ea\label{ex:nena-5}
\begingl
\glpreamble kuina nenaina echÿu gansojane tirÿrÿ chisikuji\\
\gla kuina nena-ina echÿu ganso-jane ti-rÿrÿ chi-sikuji\\
\glb \textsc{neg} like-\textsc{irr.nv} \textsc{dem}b goose-\textsc{distr} 3i-be.hard 3-excrement\\
\glft ‘the geese are not like them, their poo is hard’
\endgl
\trailingcitation{[jrx-c151001lsf-11.040]}
\xe

(\ref{ex:nena-6}) and (\ref{ex:nena-5}) come close to what was originally defined as the similative construction \citep[cf.][277]{HaspelmathBuchholz1998}. I have found one example which comes even closer. It was produced by María S. when I was eliciting examples with the associated motion marker. Apparently, she found the idea of simultaneously moving and eating quite funny, saying:

\ea\label{ex:nena-7}
\begingl
\glpreamble ninikukukÿu nena mura\\
\gla ni-niku-kukÿu nena mura\\
\glb 1\textsc{sg}-eat-\textsc{am.conc.tr} like horse\\
\glft ‘I walk eating like a horse’
\endgl
\trailingcitation{[rmx-e150922l.066]}
\xe

It is not clear to me whether (\ref{ex:nena-7}) is still a biclausal sentence or can be considered a monoclausal one. What becomes apparent in any case is that there is no subject marker on \textit{nena}, although the comparee is a first person. Consider also (\ref{ex:nena-8}), which comes from Miguel who addressed Juana. The latter had just loaded a big bag full of loam onto her head.

\ea\label{ex:nena-8}
\begingl
\glpreamble nenayu mutuÿ\\
\gla nena-yu mutuÿ\\
\glb like-\textsc{ints} termite\\
\glft ‘you look like a termite’
\endgl
\trailingcitation{[jmx-d110918ls-1.112]}
\xe

There are, however, also a few examples in the corpus with a subject marker added to the standard marker.\is{person marking} One of them is (\ref{ex:nena-9}), which comes from Miguel, when he was telling the story about the ants that are happy, when a boy is born, because when he is on a trip, he drops little crumbs of food that they can eat. The boy’s being on a trip is compared to our situation, because we were currently on a trip to \isi{Altavista}. Note that the verb \textit{-chubiku} ‘stroll’ is mostly used to denote hunting trips, which is probably why Miguel felt the need to specify what he wanted to say by use of a Spanish loan \textit{pasea} ‘stroll’.

\ea\label{ex:nena-9}
\begingl
\glpreamble tiyuna tichubikupa tiyuna paseana nenabi biti tanÿmapaiku\\
\gla ti-yuna ti-chubiku-pa ti-yuna pasea-ina nena-bi biti tanÿma-paiku\\
\glb 3i-go.\textsc{irr} 3i-stroll-\textsc{dloc.irr} 3i-go.\textsc{irr} stroll-\textsc{irr.nv} like-1\textsc{pl} 1\textsc{pl.prn} now-\textsc{punct}\\
\glft ‘he will go on a hunting trip, he will go on a jaunt like we are doing right now’
\endgl
\trailingcitation{[mxx-n120423lsf-X.14-15]}
\xe

It is common to add the \isi{additive} marker \textit{-uku} to the standard marker and then attach a person marker.\is{person marking} In (\ref{ex:nena-10}), Miguel uses the standard marker in this way to make a comparative statement to what Juan C. had said before. Both of them suffered lack of water in former times.

\ea\label{ex:nena-10}
\begingl
\glpreamble nenaukubi nauku Santa Rita kuina ÿneina\\
\gla nena-uku-bi nauku {Santa Rita} kuina ÿne-ina\\
\glb like-\textsc{add}-1\textsc{pl} there {Santa Rita} \textsc{neg} water-\textsc{irr.nv}\\
\glft ‘we didn’t have water either in Santa Rita’ (lit.: ‘like us, too, there in Santa Rita was no water’)
\endgl
\trailingcitation{[mqx-p110826l.103]}
\xe

In (\ref{ex:nena-12}), Juana compares her own state of being full to mine. I had just said before that I was ready with eating.

\ea\label{ex:nena-12}
\begingl 
\glpreamble nenaukunÿ metu\\
\gla nena-uku-nÿ metu\\ 
\glb like-\textsc{add}-1\textsc{sg} already\\ 
\glft ‘me, too, I am finished’
\trailingcitation{[jxx-p120515l-2.262]}
\xe

One last example with \textit{nenauku} follows, including the complete statement. This is the closest possible equivalent to equality sentences in other languages. Juana speaks about how much her foster child and her daughter love her.

\ea\label{ex:nena-11}
\begingl
\glpreamble tesabichunÿ micha nimijÿna, \textup{(pause)} nijinepÿi Gladys nenauku, tisumachune micha\\
\gla ti-esabichu-nÿ micha ni-mijÿna ni-jinepÿi Gladys nena-uku ti-sumachu-ne micha\\
\glb 3i-estimate-1\textsc{sg} good 1\textsc{sg}-foster.child 1\textsc{sg}-daughter Gladys like-\textsc{add} 3i-want-1\textsc{sg} good\\
\glft ‘my foster child estimates me a lot, (pause) my daughter Gladys, too, she likes me a lot’
\endgl
\trailingcitation{[jxx-p110923l-1.212-214]}
\xe
\is{similative clause|)}

\subsection{Locomotion of third person}\label{sec:Kapunu}
\is{motion predicate|(}
\is{non-verbal motion clause|(}

Cislocative locomotion of third person participants is usually expressed with a non-verbal strategy in Paunaka. It builds on the word \textit{kapunu} ‘come’. There is also a \isi{verb} \textit{-bÿsÿu} ‘come’, but it is hardly ever used with a third person subject. Consider (\ref{ex:new23-come}), which clearly shows that \textit{kapunu} is not a verb. There is no person index on the predicate and the irrealis marker is \textit{-ina}.\is{non-verbal irrealis marker} The sentence was produced by Juana on my first visit to hers in 2015.

\ea\label{ex:new23-come}
\begingl
\glpreamble tajaitu kapunuina Maria\\
\gla tajaitu kapunu-ina Maria\\
\glb tomorrow come-\textsc{irr} María\\
\glft ‘María will come tomorrow’
\endgl
\trailingcitation{[jxx-p150920l.009]}
\xe

Non-verbal predication includes stativity, so it may sound strange that the volitional action of motion is expressed non-verbally. There is, however, a connection between locomotion and stativity and this is expressed in some way in several languages of very different language families around the world \citep[]{Payne2008}. The similarity derives from locomotion predicates encoding a change of place, situation or scene which is analogous to the change of state encoded by other stative predicates \citep[249]{Payne2008}. \citet[57, 113]{Payne1997} further states that locomotion may even be expressed non-verbally in some languages.

This does still not explain why non-verbal expression of locomotion is restricted to cislocative motion of third person participants in Paunaka. A look at closely related Trinitario\is{Mojeño Trinitario|(} sheds some light on this issue. Mojeño Trinitario has a non-verbal predication type called “motion-presentationals” by \citet[68]{Rose2018a}. This construction is used to introduce new participants into the discourse, just like the existential construction does,\is{existential clause} but with an additional notion of movement onto the scene. That is, while the existential construction can be often translated by ‘there/here is ...’, the motion-presentational construction expresses meanings like ‘there/here comes ...’. Both constructions are based on a personal or demonstrative pronoun in Trinitario to which a suffix (existential or motion copula) is added. 

As I have argued in \sectref{sec:DemPron} and \sectref{sec:NonVerbalPredication} above, the first syllable \textit{ka} of \textit{kapunu} is most probably a \is{demonstrative} root with third person reference. The rest is the \isi{associated motion} marker \textit{-punu} that encodes prior motion to and away from the deictic centre on non-motion verbs, but has exclusively cislocative semantics if combined with motion verbs (see \sectref{sec:punu}). Unlike in Trinitario, this marker never combines with personal pronouns\is{personal pronoun} in Paunaka, thus there are no non-verbal expressions for motion of first or second persons.

I would suggest that just like in the Trinitario case,\is{Mojeño Trinitario|)} a construction with \textit{kapunu} was once used to introduce new participants into the discourse only, but at some point, use of the predicate became independent from the presentational function. It thus developed into the default cislocative motion predicate for third person referents. This means that nowadays \textit{kapunu} can also occur in questions,\is{interrogative clause} it can be negated etc.\is{negation}

\newpage
Consider (\ref{ex:kapunu-3}). The subject is not conominated here and it is not the place of arrival but of precedence that is of importance here. This shows that this is not a presentational construction anymore. The sentence was elicited from Juana.

\ea\label{ex:kapunu-3}
\begingl 
\glpreamble kapununube tukiu tÿbane\\
\gla kapunu-nube tukiu ti-ÿbane\\ 
\glb come-\textsc{pl} from 3i-be.far\\ 
\glft ‘they came from far away’
\trailingcitation{[jmx-e090727s.320]}
\xe

In (\ref{ex:kapunu-new}), \textit{kapunu} is part of the antecedent clause of a conditional sentence. Thus no presentation is implied here. The sentence also comes from Juana, who was talking about a possible visit of her daughter to hers.

\ea\label{ex:kapunu-new}
\begingl
\glpreamble kue kapunuina parauna kuatruchÿ kuje\\
\gla kue kapunu-ina parau-ina kuatruchÿ kuje\\
\glb if come-\textsc{irr.nv} stop-\textsc{irr.nv} four month\\
\glft ‘if she comes, she stays four months’
\endgl
\trailingcitation{[jxx-p110923l-1.425]}
\xe

Nonetheless, there are also cases in which a presentational function is notable as in (\ref{ex:kapunu-2}), a statement by Clara about the weather.

\ea\label{ex:kapunu-2}
\begingl 
\glpreamble mm, kapunu ÿku\\
\gla mm kapunu ÿku\\ 
\glb mh come rain\\ 
\glft ‘mh, rain is coming’
\trailingcitation{[cux-120410ls.257]}
\xe

 
Some more examples follow. (\ref{ex:kapunu-4}) comes from Miguel’s account about the history of Santa Rita.

\ea\label{ex:kapunu-4}
\begingl
\glpreamble i depueskuku, kuina naejumibu chijakena anyokena, kapunu padre Xeinaldo\\
\gla i depues-uku? kuina nÿ-a-ejumi-bu chija-kena anyo-kena kapunu {padre Xeinaldo}\\
\glb and afterwards-\textsc{add}? \textsc{neg} 1\textsc{sg}-\textsc{irr}-remember-\textsc{dsc} what-\textsc{uncert} year-\textsc{uncert} come {Father Reinaldo}\\
\glft ‘and also afterwards, I don’t remember anymore in which year, Father Reinaldo came’
\endgl
\trailingcitation{[mxx-p110825l.150-151]}
\xe

(\ref{ex:kapunu-new-2}) also comes from Miguel, who was conversing with Juana.\footnote{Don is a
respectful form of address in Spanish, which is used a lot in the region.}

\ea\label{ex:kapunu-new-2}
\begingl
\glpreamble rumingo kapunu unekoyae echÿu don Mario\\
\gla rumingo kapunu uneku-yae echÿu don Mario\\
\glb Sunday come town-\textsc{loc} \textsc{dem}b \textsc{hon} Mario\\
\glft ‘on Sunday, don Mario came to town’
\endgl
\trailingcitation{[jmx-c120429ls-x5.141]}
\xe

Unlike the copula \textit{kaku}, \textit{kapunu} is never omitted in negated sentences.\is{negation|(} One example is given in (\ref{ex:kapunuIRR-2}). It comes from Juana who was disappointed that her daughter did not visit her over Christmas.

\ea\label{ex:kapunuIRR-2}
\begingl 
\glpreamble kuina kapunuina nijinepÿi\\
\gla kuina kapunu-ina ni-jinepÿi\\ 
\glb \textsc{neg} come-\textsc{irr.nv} 1\textsc{sg}-daughter\\ 
\glft ‘my daughter didn’t come’
\trailingcitation{[jxx-p120430l-1.317]}
\xe
\is{negation|)}

Just like other motion predicates (see \sectref{sec:Repetition}), \textit{kapunu} has a regressive\is{regressive/repetitive} \isi{derivation}, which is \textit{kapupunu} ‘come back’. This is illustrated by (\ref{ex:kapupunu}), which comes from María S. and is about me.\footnote{Actually, I \textit{did} come back, but only later, when my second child was born and had grown a little.} 

\ea\label{ex:kapupunu}
\begingl 
\glpreamble tichÿunumi kuina kapupunuinabu naka\\
\gla ti-chÿnumi kuina kapupunu-ina-bu naka\\ 
\glb 3i-be.sad \textsc{neg} come.back-\textsc{irr}-\textsc{dsc} here\\ 
\glft ‘she is sad, because she doesn’t come back here anymore’
\endgl
\trailingcitation{[rxx-e121128s-1.020]}
\xe
\is{non-verbal motion clause|)}
\is{motion predicate|)}

The existence of a non-verbal predicate with active semantics may have played a role in non-verbal integration of verbs borrowed from Spanish into Paunaka. This is the topic of the following section.


\subsection{Borrowed verbs}\label{sec:borrowed_verbs}\is{borrowing|(}
\is{non-verbal clause with borrowed verbs|(}
Although verbs borrowed from Spanish can be verbalised and then be used just like normal active verbs\is{active verb} in Paunaka (see \sectref{sec:ActiveVerbs_TH}), this is not the preferred pattern.\footnote{This section is based on \citet[]{Terhart_subm}, but provides some additional examples.} Speakers rather rely on integrating borrowed verbs as non-verbal predicates. No light verb is needed in order to accommodate these non-verbal predicates.\footnote{A light verb is a verb with relatively general semantics, such as ‘do’, which is used as a kind of auxiliary together with an uninflected form of the borrowed verb \citep[102]{Wohlgemuth2009}.} They are rather treated as if they were nouns or adjectives (see \sectref{sec:PropInclEquatAttr} above), i.e. they take person markers that follow the predicate to index the \isi{subject}\is{person marking|(} and the \isi{non-verbal irrealis marker} \textit{-ina} in contexts that demand irrealis RS.

This can be seen in (\ref{ex:PTCP-Person}), where the borrowed form \textit{komorau}, from Spanish \textit{acomodar} ‘accomodate, arrange’, takes a second person singular marker following the predicate to index the subject and the non-verbal irrealis marker for future reference. It comes from Juana and refers to me packing my stuff shortly before I would fly back to Germany.

\ea\label{ex:PTCP-Person}
\begingl 
\glpreamble metu komoraubinatu\\
\gla metu komorau-bi-ina-tu\\ 
\glb already accommodate-2\textsc{sg}-\textsc{irr.nv}-\textsc{iam}\\ 
\glft ‘you are already going to arrange (your stuff)’
\trailingcitation{[jxx-p120515l-2.275]}
\xe
\is{person marking|)}

In most cases the input form, i.e. the form of the original verb which is borrowed \citep[cf.][]{Wohlgemuth2009}, is based on a Spanish past participle in \textit{-ado}, which is pronounced [ao̪] in Eastern Bolivia. Some examples are listed in \tabref{table:ptcps_ado}.\footnote{In addition, borrowed participles are sometimes also used adverbially like in Spanish, consider (\ref{ex:Borri-fn}) which has \textit{purau} from \textit{apurar(se)} ‘hurry up’ : 

\ea\label{ex:Borri-fn}
\begingl
\glpreamble purau tikubu\\
\gla purau ti-kubu\\
\glb hurry 3i-bath\\
\glft ‘she bathed quickly’
\endgl
\trailingcitation{[jxx-p120515l-2.152]}
\xe}

\begin{table}
\caption{Paunaka loans of past participles in \textit{-ado}}

\begin{tabularx}{\textwidth}{lllQ}
\lsptoprule
{Spanish infinitive} & {Spanish participle} & {Paunaka loan} & {Translation}\cr
\midrule
\textit{apostar} & \textit{apostado} & \textit{apostau} & bet\cr
\textit{ayudar} & \textit{ayudado} & \textit{ayurau} & help\cr
\textit{comenzar} & \textit{comenzado} & \textit{komensau} & begin\cr
\textit{ganar} & \textit{ganado} & \textit{kanau} & win\cr
\textit{mandar} & \textit{mandado} & \textit{mandau} & send\cr
\textit{multiplicar} & \textit{multiplicado} & \textit{multiplikau} & multiply\cr
\textit{olvidar} & \textit{olvidado} & \textit{arbidau/arbirau} & forget\cr
\textit{pasar} & \textit{pasado} & \textit{pasau} & pass by; happen, pass\cr
\textit{regalar} & \textit{regalado} & \textit{regalau} & give as present\cr
\lspbottomrule
\end{tabularx}

\label{table:ptcps_ado}
\end{table}

For verbs that do not provide participles in \textit{-ado}, i.e. the ones with an infinitive ending in \textit{er} and \textit{ir}, there exist different strategies. Either a past participle ending in \textit{-ido} is the input form -- Paunaka then borrows a form ending in \textit{-iru} or \textit{-iu} --, or a reduced infinitive is borrowed. The reduced infinitive is the Spanish infinitive minus the final \textit{r}, a form that is often borrowed by American languages in contact with Romance languages \citep[170]{Wohlgemuth2009}. In addition, there are some minor strategies encountered with only one or two verbs, e.g. the predicate \textit{trabaku} ‘work’ seems to be derived from the noun \textit{trabajo} ‘work’. It can also be used nominally in Paunaka (e.g. a possessed form can be derived\is{derivation} by addition of the possessed marker \textit{-ne}).

In \citet[8]{Terhart_subm}, I developed the hypothesis that the predicate \textit{kompirau} ‘share, invite’ has evolved from the Spanish verb \textit{compartir} ‘share’, whose participle is \textit{compartido}. In my argumentation, speakers would have metathesised the vowels of the last two syllables of the Spanish infinitive in order to arrive at the form \textit{kompirau}, yielding first *\textit{compirtar} and then a participle *\textit{compirtado} with the preferred ending in \textit{-ado}. However, as Nikulin (2020, p.c.) has pointed out, the input verb is possibly not \textit{compartir} ‘share’ but \textit{convidar} ‘invite’ with the participle \textit{convidado}, thus no metathesis is involved. I deem it possible that both input verbs merged in the Paunaka predicate. This would explain the sequence /mp/ in \textit{kompirau} as well as the fact that both meanings ‘share’ and ‘invite’ can be realised by this form. \textit{Kompirau} is verbalised in most cases, but one example in which it is used non-verbally is (\ref{ex:Borri-1}). Note that the predicate irregularly takes a third person marker in this case. This is usually excluded in non-verbal predication. The example comes from Miguel telling the story about the cowherd and the spirit of hill. After the spirit has taken away the cows of the man, the latter agrees to reside with the spirit in his world. Towards the end of the story, the spirit suggests that the cows can be given to the people of a village. In order to bring the cows there, the cowherd needs some help.

\ea\label{ex:Borri-1}
\begingl 
\glpreamble tupunuji kompirauchituji sinko jentenube\\
\gla ti-upunu-ji kompirau-chi-tu-ji sinko jente-nube\\ 
\glb 3i-bring-\textsc{rprt} share-3-\textsc{iam}-\textsc{rprt} five man-\textsc{pl}\\ 
\glft ‘he brought five men to share (the workload), it is said’
\trailingcitation{[mxx-n151017l-1.81]}
\xe

Another interesting feature of borrowed non-verbal predicates is their possibility to be used transitively.\is{transitivity} The \isi{object} is expressed by an NP in this case. An example is (\ref{ex:Borri-4}), which comes from Miguel’s account about how he learned to calculate. It was a young man doing military service together with Miguel who taught him, but first of all, Miguel had to register for military service:

\ea\label{ex:Borri-4}
\begingl
\glpreamble konsegiunÿtu echÿu niribretane\\
\gla konsegiu-nÿ-tu echÿu ni-ribreta-ne\\
\glb obtain-1\textsc{sg}-\textsc{iam} \textsc{dem}b 1\textsc{sg}-military.registration.document-\textsc{possd}\\
\glft ‘I obtained my military registration document’
\endgl
\trailingcitation{[mxx-p181027l-1.114]}
\xe


In (\ref{ex:PTCP-OBJ-HUM}), a theme object\is{patient/theme} is expressed by an NP and a recipient\is{recipient|(} participant is additionally added to the clause with the help of the oblique preposition \textit{-tÿpi}.\is{general oblique} This example comes from María S. and is about her plans to write back to me after I had sent her greetings via Swintha. She actually produced this sentence in Spanish first and translated it on request.

\ea\label{ex:PTCP-OBJ-HUM}
\begingl 
\glpreamble mandaubina karta chitÿpiuku\\
\gla mandau-bi-ina karta chi-tÿpi-uku\\ 
\glb send-1\textsc{pl}-\textsc{irr.nv} letter 3-\textsc{obl}-\textsc{add}\\ 
\glft ‘we will send her a letter, too’
\trailingcitation{[rxx-e121128s-1.115]}
\xe


(\ref{ex:Borri-2}) has another recipient participant that is encoded with the oblique preposition.\is{general oblique} This is a sentence by Juana about some coffee from Argentina which some friends of her daughter had given her. Note that the borrowed verb does not take a plural marker here, which is unusual, since there is a plural subject (which is clear from the context).

\ea\label{ex:Borri-2}
\begingl
\glpreamble regalau nitÿpi\\
\gla regalau ni-tÿpi\\
\glb give.as.present 1\textsc{sg}-\textsc{obl}\\
\glft ‘they gave it to me as a present’
\endgl
\trailingcitation{[jxx-e120430l-4.29]}
\xe
\is{recipient|)}

Borrowed non-verbal predicates can be used in complex clauses. In (\ref{ex:PTCP-complement}), \textit{trabaku} ‘work’ is the complement of a desiderative verb, in (\ref{ex:PTCP-OBJ}) we have a construction that resembles the serial verb construction but with a non-verbal predicate as the second predicate (i.e. a serial predicate construction).

\ea\label{ex:PTCP-complement}
\begingl 
\glpreamble nijinepÿi kuina tisacha trabakuneina\\
\gla ni-jinepÿi kuina ti-sacha trabaku-ne-ina\\ 
\glb 1\textsc{sg}-daughter \textsc{neg} 3i-want work-1\textsc{sg}-\textsc{irr.nv}\\ 
\glft ‘my daughter doesn't want me to work’
\trailingcitation{[jxx-n101013s-1.193-194]}
\xe

\ea\label{ex:PTCP-OBJ}
\begingl 
\glpreamble eka semana niyuna kontratauneina chÿnachÿ makina\\
\gla eka semana ni-yuna kontratau-ne-ina chÿnachÿ makina\\ 
\glb \textsc{dem}a week 1\textsc{sg}-go.\textsc{irr} engage-1\textsc{sg}-\textsc{irr.nv} one machine\\ 
\glft ‘this week I will hire a machine’
\trailingcitation{[jxx-p120515l-2.106]}
\xe


\hspace*{-2pt}There is also one modal non-verbal predicate\is{modality|(} borrowed from the Spanish modal verb \textit{poder} ‘can, be able to’. In Paunaka, its form is \textit{puero},\is{knowledge/ability predicate|(} and it has possibly been borrowed via \isi{Bésiro}, which has a noun \textit{puéru} ‘possibility’ and a verb \textit{puérux} ‘can, be able to’, which is derived from that noun \citep[cf.][]{Sans2011}.

\textit{Puero} is exceptional insofar as that it usually does not take subject indexes,\is{person marking} although a few cases with a first person singular marker do occur. If used together with another predicate, it is also not necessarily marked for irrealis in irrealis contexts.\is{non-verbal irrealis marker} If used alone (e.g. as an answer to a question), it does take the irrealis marker in these contexts. \textit{Puero} is primarily used in negative contexts,\is{negation|(} since irrealis alone is sufficient to indicate a permissive or abilitive reading in Paunaka. In negation of a permissive or abilitive constructions, however, two factors trigger irrealis marking, so that speakers may feel the need to be more explicit about the modal meaning. This is reminiscent of the \isi{doubly irrealis construction} found in other Arawakan languages \citep[cf.][271]{Michael2014}, the difference being that in Paunaka, the fact that two parameters trigger irrealis is expressed by a lexical rather than morphological means. Two examples of \textit{puero} follow, one with and the other one without irrealis marking on \textit{puero}. Both come from María S.

(\ref{ex:Borri-6}) was elicitated. It refers to an imagined old man.

\ea\label{ex:Borri-6}
\begingl
\glpreamble kuina pueroina tiyuna asaneti\\
\gla kuina puero-ina ti-yuna asaneti\\
\glb \textsc{neg} can-\textsc{irr.nv} 3i-go.\textsc{irr} field\\
\glft ‘he cannot go to the field’
\endgl
\trailingcitation{[rxx-e181022le]}
\xe

(\ref{ex:Borri-5}) is a statement by María S. about herself. She had a bad knee by that time.

\ea\label{ex:Borri-5}
\begingl
\glpreamble kuina puero niyuika kasi\\
\gla kuina puero ni-yuika kasi\\
\glb \textsc{neg} can 1\textsc{sg}-walk.\textsc{irr} almost\\
\glft ‘I almost cannot walk’
\endgl
\trailingcitation{[rxx-e181017l.011]}
\xe
\is{negation|)}
\is{knowledge/ability predicate|)}

In addition, there is also \textit{tiene ke} ‘must’ (from Span. \textit{tiene que} ‘he/she/it has to’), but this one is used very infrequently. It also occurs in \isi{Bésiro} \citep[cf. Bésiro text examples in][47--70]{Sans2013}. One example is (\ref{ex:must-borr-1}) from Clara, who refers to the excursion to \isi{Altavista} which Swintha, Federico and I had planned.

\ea\label{ex:must-borr-1}
\begingl
\glpreamble pero esachu eyuna tiene ke tiyunakena Miyel\\
\gla pero e-sachu e-yuna {tiene ke} ti-yuna-kena Miyel\\
\glb but 2\textsc{pl}-want 2\textsc{pl}-go.\textsc{irr} must 3i-go.\textsc{irr}-\textsc{uncert} Miguel\\
\glft ‘but if you want to go, Miguel probably has to go as well’
\endgl
\trailingcitation{[cux-c120414ls-1.139]}
\xe
\is{modality|)}

The use of borrowed verbs as non-verbal predicates is surprising, because it links the encoding of events and actions to non-verbal predication, although this is usually closely connected to verbal predication (e.g. \citealt[189, 244]{Langacker1987}; \citealt[140, 142]{Frawley1992}; \citealt[82--83]{VanValinLaPolla1997}; \citealt[52]{Givon2001}).

Considering native structures only, non-verbal predication in Paunaka covers stative relationships, but the insertion of borrowed Spanish verbs has extended the semantic scope to include also active relationships. It might be the case though that prior to this, the non-verbal predicate \textit{kapunu} ‘come’ was already used with active semantics, thus facilitating the integration of borrowed verbs in a similar way. Furthermore, the integration of borrowed verbs as non-verbal predicates might be an areal feature. Consider the case of Bésiro.\is{Bésiro|(} In this language, verbs obligatorily take prefixes to index the subject and they take enclitics to index objects. Nominal and adjectival predicates take enclitics to index a subject, and so do some borrowed verbs. Between the input form and the enclitic, a suffix \textit{-bo} is inserted (Sans 2012, p.c.). Note, however, that Bésiro borrows reduced infinitives instead of participles, and some borrowed verbs seem to be verbalised rather than used non-verbally \citep[cf. Bésiro texts in][47--70]{Sans2013}. It remains unclear, for the time being, how frequent the borrowing of verbs as non-verbal predicates is in Bésiro.\is{Bésiro|)}

As for the borrowing of participles, this input form could also have been preferred by speakers of the Chapacuran language Kitemoka \citep[cf. ex. KIT1 739 in][96]{Wienold2012}. With only one example of a borrowed Spanish verb in the Kitemoka corpus and the little knowledge about Kitemoka in general, we cannot, unfortunately, make any statements about verbal or non-verbal character of the borrowed item.\is{borrowing|)}
\is{non-verbal clause with borrowed verbs|)}
\is{non-verbal predication|)}

While this chapter has focused on different kinds of declarative clauses up to here,\is{declarative clause|)} the remaining two sections are dedicated to other speech acts: directives and interrogatives. The next section starts with a discussion of imperatives and other kinds of directives.


%!TEX root = 3-P_Masterdokument.tex
%!TEX encoding = UTF-8 Unicode

\section{Imperatives and other directives}\label{sec:Imperative}
\is{directive speech act|(}

Imperatives are directive speech acts. They can express a “command, request, offer, advisory, or exhortation” \citep[277]{Koenig2007}. In Paunaka, imperatives usually build on active verbs.\is{active verb} They can be identical in structure to a \isi{declarative clause}. I call this type of imperatives “unmarked”, although they require \isi{irrealis} RS. Irrealis could occur in declarative clauses for other reasons. Unmarked imperatives inflect for person;\is{person marking} they either have second person singular or second person plural subjects.\is{subject} Objects\is{object} can be indexed on these imperatives and conominated objects\is{conomination} can occur, but they are never placed in \isi{focus} position preceding the verb.\is{word order}  Some examples of unmarked imperatives can be found in \sectref{sec:UnmarkedImperatives}. %“directive speech acts, i.e. orders and requests, but also invitations, the giving of advice, warnings, wishes, instructions, etc.” \citep[303]{Koenig2007}

Alternatively, imperatives can be marked by adding the suffix \textit{-ji} to the end of the verb. I cannot tell what is the exact difference to unmarked imperatives, but more emphasis seems to be involved. They are possibly only used to make requests and commands (i.e. no offers), but this remains to be verified. Emphatic imperatives are the topic of \sectref{sec:MarkedImperatives}. 

There are two suppletive motion imperatives:\is{motion predicate} \textit{nabi}/\textit{nabue} ‘go!’ and \textit{pana} ‘come!’, they are dealt with in \sectref{sec:SuppletiveImperatives}. 

Negative imperatives can look like negative declarative sentences\is{declarative clause} if they include the standard negation particle\is{negative particle} \textit{kuina} and an \isi{irrealis} predicate. However, they can also be formed with a \isi{realis} predicate and the prohibitive\is{directive speech act!prohibitive}\is{negation!prohibitive} particle \textit{naka} or the \isi{admonitive} particle \textit{masaini}. The latter is rather used in warnings. Different kinds of negative imperatives are described in \sectref{sec:Prohibitives}.

Hortatives are formed with the \isi{hortative} particle \textit{jaje}. This is the topic of \sectref{sec:Hortatives}.
%Optatives/jussives with \textit{-jÿti} and \textit{-yuini}



\subsection{Unmarked imperatives}\label{sec:UnmarkedImperatives}
\is{imperative|(}

Imperatives may be unmarked, but the verb usually has \isi{irrealis} RS (some possible exceptions are discussed in the end of this section). The \isi{verb} is not inflected\is{inflection} for TAME. Imperatives have second person singular or plural addressees,\is{addressee of imperative} with singular being more frequently found in the corpus. They take the same person indexes\is{person marking} that are also found in declarative sentences. (\ref{ex:imp-1})--(\ref{ex:imp-5}) have singular and (\ref{ex:imp-6})--(\ref{ex:imp-9}) plural addressees. Unmarked imperatives do not only express commands, but also requests or invitations/offers. It is \isi{intonation} alone that marks different degrees of politeness or friendliness, thus setting apart commands from all other possible uses.

By producing (\ref{ex:imp-1}), Juana offered me something to drink.

\ea\label{ex:imp-1}
\begingl
\glpreamble ¡pea!\\
\gla pi-ea\\
\glb 2\textsc{sg}-drink.\textsc{irr}\\
\glft ‘drink!’
\endgl
\trailingcitation{[jxx-p150920l.002]}
\xe

(\ref{ex:imp-2}) is an imperative including a goal argument. It also comes from Juana and was directed to Miguel to help her load the loam she had collected onto her head to carry it.

\ea\label{ex:imp-2}
\begingl
\glpreamble ¡petuka nitapukiyae!\\
\gla pi-etuka ni-tapuki-yae\\
\glb 2\textsc{sg}-put.\textsc{irr} 1\textsc{sg}-head-\textsc{loc}\\
\glft ‘put it on my head!’
\endgl
\trailingcitation{[jmx-d110918ls-1.110]}
\xe

In (\ref{ex:imp-3}), a first person singular object is indexed on the verb. The example comes from the story about the lazy man by Miguel. When he has finally cut off his limbs, pretending they were \textit{cusi} palm fruits, he requests his son to lift him and put him into the basket to be carried, since he cannot walk anymore without legs.

\ea\label{ex:imp-3}
\begingl
\glpreamble “bueno, ¡pakachane pipurtukane naka sÿkiyae!”\\
\gla bueno pi-akacha-ne pi-purtuka-ne naka sÿki-yae\\
\glb well 2\textsc{sg}-lift.\textsc{irr}-1\textsc{sg} 2\textsc{sg}-put.in.\textsc{irr}-1\textsc{sg} here basket-\textsc{loc}\\
\glft ‘“well, lift me and put me into the basket!”’
\endgl
\trailingcitation{[mox-n110920l.118]}
\xe

(\ref{ex:imp-4}) represents what Jesus tells the monkey in the creation story narrated by Juana. The background is that the monkey had stolen corn and hidden it in his mouth, although the corn was meant for the people to eat.

\ea\label{ex:imp-4}
\begingl
\glpreamble “¡piyuna pinika eka chÿi yÿkÿke!”\\
\gla pi-yuna pi-nika eka chÿi yÿkÿke\\
\glb 2\textsc{sg}-go.\textsc{irr} 2\textsc{sg}-eat.\textsc{irr} \textsc{dem}a fruit tree\\
\glft ‘“go and eat the fruit of the trees”’
\endgl
\trailingcitation{[jxx-n101013s-1.873]}
\xe

(\ref{ex:imp-5}) is an example which contains an associated motion marker. It was produced by Clara and directed to María C. to give her advice on how her children could learn some Paunaka. The first clause is a directive which does not build on an imperative clause, but rather makes use of the borrowed modal expression \textit{tiene ke} ‘must’. It does not inflect for person. The second clause, i.e. the direct speech complement, is an imperative.

\ea\label{ex:imp-5}
\begingl
\glpreamble pero pue tiene ke pikechachi: “¡pinipuna nichechapÿibi!”\\
\gla pero pue {tiene ke} pi-kecha-chi pi-ni-puna ni-chechapÿi-bi\\
\glb but well {must} 2\textsc{sg}-say.\textsc{irr}-3 2\textsc{sg}-eat-\textsc{am.prior.irr} 1\textsc{sg}-son-2\textsc{sg}\\
\glft ‘but, well, you have to tell him: “come and eat, my dear son!”’
\endgl
\trailingcitation{[cux-c120414ls-2.302]}
\xe

(\ref{ex:imp-6}) has a second person plural addressee, Miguel, Swintha and me. Juana and Miguel dug for loam, and the sentence is an exclamation by Juana, being excited about the quality of the loam she found.

\ea\label{ex:imp-6}
\begingl
\glpreamble ¡emua, micha michana muteji!\\
\gla e-imua micha michana muteji \\
\glb 2\textsc{pl}-see.\textsc{irr} good nice loam \\
\glft ‘look, the loam is good, beautiful!’
\endgl
\trailingcitation{[jmx-d110918ls-1.089]}
\xe

In (\ref{ex:imp-7}), Juana cites what their landlord said to her daughter. 

\ea\label{ex:imp-7}
\begingl
\glpreamble “¡esemaika juchubu ejecheka! porke kopaunatu nubiu”, tikechu\\
\gla e-semaika juchubu e-jecheka porke kopau-ina-tu nÿ-ubiu ti-kechu\\
\glb 2\textsc{pl}-search.\textsc{irr} where 2\textsc{pl}-move.\textsc{irr} because use-\textsc{irr.nv}-\textsc{iam} 1\textsc{sg}-house 3i-say\\
\glft ‘“look for where to move, because I want to use my house for myself!”, he said’
\endgl
\trailingcitation{[jxx-p120430l-1.397]}
\xe

(\ref{ex:imp-8}) shows that an adverb can precede the verb in an imperative.\is{word order} This sentence comes from the story about the cowherd and the spirit of the hill told by Miguel. When he has passed some time with the spirit, the cowherd finally brings the cows to a village with the help of some people. This is what the cowherd tells the people, before he actually releases the cows.

\ea\label{ex:imp-8}
\begingl 
\glpreamble “¡nakajiku ekichupupuikanÿ!”, tikechuchÿji\\
\gla naka-jiku e-kichupu-puika-nÿ ti-kechu-chÿ-ji\\ 
\glb here-\textsc{lim}1 2\textsc{pl}-wait-\textsc{cont}.\textsc{irr}-1\textsc{sg} 3i-say-3-\textsc{rprt}\\ 
\glft ‘“wait for me right here!”, he said to them, it is said’
\trailingcitation{[mxx-n151017l-1.81]}
\xe

(\ref{ex:imp-9}) is a request of Juana’s sister to the policemen after she has been arrested for the deeds of her husband.

\ea\label{ex:imp-9}
\begingl
\glpreamble “¡epuninane nijinepÿimÿnÿ!"\\
\gla e-epun-ina-ne ni-jinepÿi-mÿnÿ\\
\glb 2\textsc{pl}-take-\textsc{ben.irr}-1\textsc{sg} 1\textsc{sg}-daughter-\textsc{dim}\\
\glft ‘“take my daughter to me!”’
\endgl
\trailingcitation{[jxx-p120430l-2.101]}
\xe

There are also examples in the corpus albeit very few, in which an imperative seems to be formed with a \isi{realis} verb. They all have in common that they are exclamations. It might thus be the case that realis is possible in those specific cases. Otherwise, these examples could also simply be taken as mistakes, considering (\ref{ex:imp-6}) above, which is an exclamation, too, but has an irrealis verb nonetheless. Or they are no imperatives at all, but rather a verbalisation of an ongoing action. Two examples are given below.

(\ref{ex:imp-exc-1}) was produced by María S, when she showed the flower of a plant to Swintha.

\ea\label{ex:imp-exc-1}
\begingl
\glpreamble ¡pimu! chibu eka chÿina\\
\gla pi-imu chibu eka chÿi-ina\\
\glb 2\textsc{sg}-see 3\textsc{top.prn} \textsc{dem}a fruit-\textsc{irr.nv}\\
\glft ‘look, this will be its fruit!’\\or: ‘you see, this will be its fruit!’
\endgl
\trailingcitation{[rxx-e121126s-3.17-18]}
\xe

(\ref{ex:imp-exc-2}) is from Riester’s recordings. It was produced by Juan Ch. as part of an introduction to his playing the flute. Note that the morphologically stative verb \textit{-kusabenu} ‘play flute’ (an attributive derivation, see \sectref{sec:AttributiveVerbs}), seems to include the subordinating suffix \textit{-i} here (see \sectref{sec:Subordination-i}). This is relatively uncommon, but happens from time to time.\is{deranked verb}

\ea\label{ex:imp-exc-2}
\begingl
\glpreamble ¡esamu kristianunube! nikusabenuiu baile suelto\\
\gla e-samu kristianu-nube ni-kusabenu-i-u {baile suelto}\\
\glb 2\textsc{pl}-hear person-\textsc{pl} 1\textsc{sg}-play.flute-\textsc{subord}-\textsc{real} {name.of.song}\\
\glft ‘people, listen to me playing \textit{baile suelto} by flute!’
\endgl
\trailingcitation{[nxx-a630101g-2.002-003]}
\xe

%nabi not always, but pana always, % ana für 2 pl??
\is{imperative|)}


\subsection{Emphatic imperatives}\label{sec:MarkedImperatives}
\is{emphatic imperative|(}\is{inflection|(}

Imperatives can be formed by adding the suffix \textit{-ji} to a \isi{verb} inflected for \isi{irrealis} and second person, i.e. emphatic imperatives take person indexes\is{person marking} and irrealis RS just like the unmarked ones.

To start with, consider (\ref{ex:eimp-1}). The verb has the second person singular marker, it has irrealis RS and it takes the imperative marker \textit{-ji}. The example was elicited from Juana and represents a command to a dog to bite a thief.

\ea\label{ex:eimp-1}
\begingl
\glpreamble ¡pinijabakaji!\\
\gla pi-nijabaka-ji\\
\glb 2\textsc{sg}-bite.\textsc{irr}-\textsc{imp}\\
\glft ‘bite him!’
\endgl
\trailingcitation{[jxx-e191021e-2]}
\xe

Actually, it began to dawn on me relatively late that this was indeed an imperative marker. I had taken it for some deictic element before, and thus in elicitation sessions, I tried to find out about dimensions of space rather than characteristics that set the marked imperatives apart from the unmarked ones. Thus, I can only share some observations here that remain to be checked.

First of all, looking at the examples with this marker, it seems that they only include requests and commands. Offers or invitations and suggestions are absent. However, this may be a coincidence, as there are also more requests among the unmarked imperatives than there are offers or suggestions.

Second, in an elicitation session, the gestures Juana made when using the word forms with \textit{-ji} were bigger and more encompassing, which suggests to me that more emphasis is involved. This is why I speak of an emphatic imperative.

Third, in the same elicitation session, the form \textit{¡pupuna!} was consistently translated by her with Spanish \textit{¡trae!} ‘bring!’, and \textit{¡pupunaji!} with \textit{¡traélo!} ‘bring it!’. All but one of the examples with marked imperatives indeed have a third person \isi{object}. The one exception has a first person singular object index and \textit{-ji} is added after that one. Unmarked imperatives can also have third person objects (e.g. (\ref{ex:imp-2}) and (\ref{ex:imp-4}) in \sectref{sec:UnmarkedImperatives} above), so the difference may ultimately not depend on the presence or absence of a third person object in the imperative clause, but rather the translations with or without an \isi{object} is the way people express the same difference in Spanish. This remains to be proved.

Fourth, I have only found one example of an emphatic imperative with a second person plural subject, but I suppose this is connected to the fact that imperatives with plural subjects are in general rarer than the ones with singular subjects.

Some more examples follow. (\ref{ex:eimp-2}) stems from the story about the lazybones told by Miguel. The man has just climbed a tree and cut off his arm in order to throw it down to his son, pretending it was a raceme of \textit{cusi} palm fruit.

\ea\label{ex:eimp-2}
\begingl
\glpreamble “¡pijakupaji eka kÿsi!” tikechu chichechapÿi\\
\gla pi-jakupa-ji eka kÿsi ti-kechu chi-chechapÿi\\
\glb 2\textsc{sg}-receive.\textsc{irr}-\textsc{imp} \textsc{dem}a cusi 3i-say 3-son\\
\glft ‘“take the \textit{cusi} fruit!” he said to his son’
\endgl
\trailingcitation{[mox-n110920l.100]}
\xe

\hspace*{-2.3pt}(\ref{ex:eimp-3}) comes from Juana telling me how her grandparents bought cows in Moxos. On their way back home they were caught by heavy rainfalls and had to cross an arroyo, which had filled with water. In this situation, her grandfather can hardly reach the ground and in order to guide his wife through the water he says:

\newpage
\ea\label{ex:eimp-3}
\begingl
\glpreamble “¡pabikÿkaji nitijÿe naka!”\\
\gla pi-abikÿka-ji ni-tijÿe naka\\
\glb 2\textsc{sg}-grab.\textsc{irr}-\textsc{imp} 1\textsc{sg}-belt here\\
\glft ‘“hold on to my belt here!”'
\endgl
\trailingcitation{[jxx-p151016l-2.141]}
\xe

%\ea\label{ex:eimp-62}
%\begingl
%\glpreamble ¡pibikajikaji pimaretane!\\
%\gla pi-bikajika-ji pi-mareta-ne\\
%\glb 2\textsc{sg}-throw.\textsc{irr}-\textsc{imp} 2\textsc{sg}-suitcase-\textsc{possd}\\
%\glft ‘"throw your suitcase away!”‘\\
%\endgl
%\trailingcitation{[jxx-p151016l-2.104]}
%\xe

(\ref{ex:eimp-5}) is from the creation story told by Juana and is a citation of the snake. It is the forbidden apple that María Eva is supposed to take to her husband.

\ea\label{ex:eimp-4}
\begingl
\glpreamble “¡pumaji nauku tÿpi pima!”\\
\gla pi-uma-ji nauku tÿpi pi-ima\\
\glb 2\textsc{sg}-take.\textsc{irr} there \textsc{obl} 2\textsc{sg}-husband\\
\glft ‘“take it there for your husband!”’
\endgl
\trailingcitation{[jxx-n101013s-1.413]}
\xe

In (\ref{ex:eimp-5}), Miguel requests of Alejo that he ask the taxi driver, who was joining the recording session, about his place of origin.

\ea\label{ex:eimp-5}
\begingl
\glpreamble ¡piyÿsebÿkeaji juchubu chubiu, juchubu eka kapuniuchÿ!\\
\gla pi-yÿsebÿkea-ji juchubu chÿ-ubiu juchubu eka kapun-i-u-chÿ\\
\glb 2\textsc{sg}-ask.\textsc{irr}-\textsc{imp} where 3-house where \textsc{dem}a come-\textsc{subord}-\textsc{real}-3\\
\glft ‘ask him where he lives, where he comes from!’
\endgl
\trailingcitation{[mty-p110906l.211-212]}
\xe

(\ref{ex:eimp-6}) is the only example I have found of an emphatic imperative that does not have a third person object. It is a first person singular object in this case, which is indexed on the verb. The imperative marker follows the object index. The example comes from Miguel telling the story about the fox and the jaguar. Since the vulture let the fox escape, the jaguar wants to punish and eat him. The vulture seemingly accepts his fate and tells the jaguar to pluck him except for his wings and throw him up into the air:
% so that he would fall right into his open mouth. The jaguar obeys, but instead of falling down, the vulture flies away, not without defecating into the mouth of the jaguar before.

\ea\label{ex:eimp-6}
\begingl
\glpreamble “entonses ¡pibikÿkaneji anÿke!”\\
\gla entonses pi-bikÿka-ne-ji anÿke\\
\glb thus 2\textsc{sg}-throw-1\textsc{sg}-\textsc{imp} up\\
\glft ‘“then throw me up!”’
\endgl
\trailingcitation{[jmx-n120429ls-x5.195]}
\xe

Finally, I also came across one occurrence of an emphatic imperative with a second person plural subject, given as (\ref{ex:eimp-7}) here. It comes from a conversation between Juana and Miguel, where the latter told his sister what a certain person had said to the people of Santa Rita.

\ea\label{ex:eimp-7}
\begingl
\glpreamble “¡anaji echÿu senta!”\\
\gla e-ana-ji echÿu senta\\
\glb 2\textsc{pl}-make.\textsc{irr}-\textsc{imp} \textsc{dem}b path\\
\glft ‘“make the path!”’
\endgl
\trailingcitation{[jmx-c120429ls-x5.063]}
\xe
\is{inflection|)}\is{emphatic imperative|)}

\subsection{Suppletive imperatives}\label{sec:SuppletiveImperatives}
\is{suppletive imperative|(}
\is{motion predicate|(}

There are two suppletive imperatives, \textit{nabi} ‘go!’ and \textit{pana} ‘come!’, the former being much more frequent. Both words can combine with a \isi{verb} as well as with a demonstrative adverb, \textit{nabi} has also been found in combination with locative-marked nouns.

It is not entirely clear how the suppletive imperatives are composed. As for \textit{nabi}, there may be a root \textit{na} that takes the second person singular marker \textit{-bi},\is{person marking} which is used to index objects on verbs and subjects on non-verbal predicates.\is{non-verbal predication} \textit{Pana} is identical to the second person singular irrealis form of the verb \textit{-anau} ‘make’. Alternatively, it might be related to the prior \isi{associated motion} marker \textit{-punu} (realis) / \textit{-puna} (irrealis).\footnote{Regarding active verbs, \isi{Terena}, Old \isi{Baure} (i.e. the \isi{Baure} variety documented by the Jesuits) and marginally \isi{Mojeño Trinitario} inflect for irrealis by changing every vowel /o/ to /a/ \citep[103]{DanielsenTerhartSubm}. The marker \textit{-punu} may be related to \textit{pana} in a similar fashion if we assume that the same kind of RS-triggered vowel harmony once existed in Paunaka.}

I will first present some examples with \textit{nabi}. In (\ref{ex:nabi-1}), it first stands alone in the first clause and is then combined with an adverb plus locative-marked noun in the second clause to indicate the goal of the motion action that is demanded here. The examples comes from Juana’s narration about her grandparents’ journey and is a citation of the water spirit talking with her grandfather at night, trying to lure him away from his wife.

\ea\label{ex:nabi-1}
\begingl
\glpreamble “¡nabi! ¡nabi nauku nubiuyae!” chikechuchÿji\\
\gla nabi nabi nauku nÿ-ubiu-yae chi-kechu-chÿ-ji\\
\glb go.\textsc{imp} go.\textsc{imp} there 1\textsc{sg}-house-\textsc{loc} 3-say-3-\textsc{rprt}\\
\glft ‘“go! go to my house there!” she said to him, it is said’
\endgl
\trailingcitation{[jxx-p151016l-2.195]}
\xe

(\ref{ex:nabi-2}) was elicited from María C. It is something one could say to a dog to chase it off.

\largerpage
\ea\label{ex:nabi-2}
\begingl
\glpreamble ¡nabi nekupaiyae!\\
\gla nabi nekupai-yae\\
\glb go.\textsc{imp} outside-\textsc{loc}\\
\glft ‘go out!’
\endgl
\trailingcitation{[uxx-e120427l.078]}
\xe

In (\ref{ex:nabi-3}), \textit{nabi} combines with a verb. It is a repetition of (\ref{ex:imp-4}) above, with which Juana got back to the storyline after summarising in Spanish a part of the creation story. Note that while she used a second person singular irrealis form of the verb \textit{-yunu} in the example above, she replaces it by the suppletive form \textit{nabi} here.

\ea\label{ex:nabi-3}
\begingl
\glpreamble “¡nabi pinika chÿi yÿkÿke!”\\
\gla nabi pi-nika chÿi yÿkÿke\\
\glb go.\textsc{imp} 2\textsc{sg}-eat.\textsc{irr} fruit tree\\
\glft ‘“go and eat the fruit of the trees!”’
\endgl
\trailingcitation{[jxx-n101013s-1.885]}
\xe

(\ref{ex:nabi-8}) also comes from Juana. She was telling Swintha about a very smart dog she once had. When she wanted to slaughter a chicken, she could point to one chicken and tell the dog to catch it. This is what she said to the dog:

\ea\label{ex:nabi-8}
\begingl
\glpreamble bikupaika takÿra ¡nabi peikukuika takÿra!\\
\gla bi-kupaika takÿra nabi pi-eikukuika takÿra\\
\glb 1\textsc{pl}-slaughter.\textsc{irr} chicken go.\textsc{imp} 2\textsc{sg}-chase.\textsc{irr} chicken\\
\glft ‘we are going to slaughter a chicken, go and chase the chicken!’
\endgl
\trailingcitation{[jxx-e191021e-2]}
\xe


For plural addressees,\is{addressee of imperative} Juana used \textit{nabue} a few times, which includes the second plural index \textit{-e} instead of singular \textit{-bi}. However, this has not been found with other speakers. Indeed, Juan C. once corrected himself with a verb with second person plural index, when he wanted to form an imperative with second person plural reference, see (\ref{ex:nabi-7}). Juana’s use of \textit{nabue} is exemplified in (\ref{ex:nabi-6}) below, which comes from elicitation.

\ea\label{ex:nabi-7}
\begingl
\glpreamble ¡nabi! ¡eyuna!\\
\gla nabi e-yuna\\
\glb go.\textsc{imp} 2\textsc{pl-go.\textsc{irr}}\\
\glft ‘go (\textsc{sg})! go (\textsc{pl})!’
\endgl
\trailingcitation{[mqx-p110826l.031]}
\xe

\ea\label{ex:nabi-6}
\begingl
\glpreamble nabue emusuika\\
\gla nabu-e e-musuika\\
\glb go.\textsc{imp}-2\textsc{pl} 2\textsc{pl}-wash.\textsc{irr}\\
\glft ‘go and wash!’
\endgl
\trailingcitation{[jxx-e081025s-1.535]}
\xe

The suppletive imperative \textit{pana} has only been found with singular addressees\is{addressee of imperative} in the corpus. It mostly combines with the adverb \textit{naka} ‘here’. One such case is (\ref{ex:pana-1}), where Juana told me how her sister María S. and her husband once had an encounter with a snake or water spirit in the reservoir of Santa Rita. This is what the husband exclaimed, when he noticed the snake:

\ea\label{ex:pana-1}
\begingl
\glpreamble “¡pana naka! ¡kechue echÿu!”\\
\gla pana naka kechue echÿu\\
\glb come.\textsc{imp} here snake \textsc{dem}b\\
\glft ‘“come here! that’s a snake!”’
\endgl
\trailingcitation{[jxx-p120515l-2.164]}
\xe

(\ref{ex:pana-2}) comes from Juana telling the creation story. After having fashioned her from mud, God requests María Eva to approach him in order to wed her to Jesus.

\ea\label{ex:pana-2}
\begingl
\glpreamble “Maria Eva, ¡pana naka!” chikechuchiji\\
\gla {Maria Eva} pana naka chi-kechu-chi-ji\\
\glb {María Eva} come.\textsc{imp} here 3-say-3-\textsc{rprt}\\
\glft ‘“María Eva, come here!” he said to her, it is said’
\endgl
\trailingcitation{[jxx-n101013s-1.364]}
\xe

In (\ref{ex:pana-3}), a verb follows the adverb. This example was elicited from Miguel.

\ea\label{ex:pana-3}
\begingl
\glpreamble ¡pana naka pitibua!\\
\gla pana naka pi-tibua\\
\glb come here 2\textsc{sg}-sit.down.\textsc{irr}\\
\glft ‘come here and sit down!’
\endgl
\trailingcitation{[mxx-e160811sd.221]}
\xe


If a verb follows \textit{nabi} or \textit{pana}, it is not unusual that this verb takes the prior motion\is{associated motion} marker. (\ref{ex:nabi-5}) and (\ref{ex:nabi-4}) exemplify this for \textit{nabi}, and (\ref{ex:pana-5}) and (\ref{ex:pana-4}) for \textit{pana}.

(\ref{ex:nabi-5}) comes from María C. who had told that she medicated herself with the bark of a tree. She knew about the use of the bark, because it was as if God had told her:

\ea\label{ex:nabi-5}
\begingl
\glpreamble ¡nabi parejipuna echÿu pichai!\\
\gla nabi pi-areji-puna echÿu pichai \\
\glb go.\textsc{imp} 2\textsc{sg}-rasp-\textsc{am.prior.irr} \textsc{dem}b medicine\\
\glft ‘go and rasp the medicine’
\endgl
\trailingcitation{[ump-p110815sf.371]}
\xe

(\ref{ex:nabi-4}) was elicited from María S., when a pig of hers was grunting very loudly, disturbing the recording we made. Note that \textit{-sabaiku} ‘grunt’ is a stative verb,\footnote{It is actually rare that stative verb combines the with prior motion marker, but this seems one of the cases, in which a morphologically stative verb is semantically active. Note that the related continuous verb form \textit{-sabaipaiku} ‘grunt’ is active.} so it takes an \isi{irrealis} prefix, and consequently, the associated motion marker occurs in its default/\isi{realis} form here.

\ea\label{ex:nabi-4}
\begingl
\glpreamble ¡nabi pasabaipunu max nauku!\\
\gla nabi pi-a-sabai-punu max nauku\\
\glb go.\textsc{imp} 2\textsc{sg}-\textsc{irr}-grunt-\textsc{am.prior} more there\\
\glft ‘go to grunt over there!’
\endgl
\trailingcitation{[rmx-e150922l.159]}
\xe

(\ref{ex:pana-5}) was elicited from Miguel. The verb follows the adverb \textit{naka} in this case.

\ea\label{ex:pana-5}
\begingl
\glpreamble ¡pana naka pimukupuna!\\
\gla pana naka pi-muku-puna\\
\glb come.\textsc{imp} here 2\textsc{sg}-sleep-\textsc{am.prior.irr}\\
\glft ‘come here to sleep!’
\endgl
\trailingcitation{[mxx-e160811sd.232]}
\xe

In (\ref{ex:pana-4}), \textit{pana} takes the \isi{prospective} marker \textit{-bÿti}. This example stems from the recordings by Riester, and I have found neither \textit{pana} nor \textit{nabi} taking any TAME markers in the recordings made from 2008 on. The example is one of the sentences Juan Ch. produces as a beginning of an imagined conversation with a visitor.

\ea\label{ex:pana-4}
\begingl
\glpreamble ¡panabÿti pitibupuna!\\
\gla pana-bÿti pi-tibu-puna\\
\glb come.\textsc{imp}-\textsc{prsp} 2\textsc{sg}-sit.down-\textsc{am.prior.irr}\\
\glft ‘come and sit down!’
\endgl
\trailingcitation{[nxx-p630101g-2.07]}
\xe

While the prior motion\is{associated motion} marker has been found with verbs accompanying both suppletive imperatives, the dislocative marker\is{dislocative|(} only occurs with verbs combining with \textit{nabi}. This is in accordance with their semantics. While \textit{-punu} is not specified for direction towards or away from a place, the dislocative marker has only been found in expressions of translocative motion (see \sectref{sec:punu} and \sectref{sec:PA}). If \textit{nabi} is combined with a verb taking the dislocative marker, this can be analysed as a case of \isi{motion-cum-purpose construction} (see \sectref{sec:MotionCumPurpose}). Two examples follow. Both were elicited from Juana.


\ea\label{ex:nabi-9}
\begingl
\glpreamble ¡nabi piyÿseikupa kanela! kuina kakuina\\
\gla nabi pi-yÿseiku-pa kanela kuina kaku-ina\\
\glb go.\textsc{imp} 2\textsc{sg}-buy-\textsc{dloc.irr} cinnamon \textsc{neg} exist-\textsc{irr.nv}\\
\glft ‘go and buy cinnamon! There isn’t any’
\endgl
\trailingcitation{[jxx-e190210s-01]}
\xe

\ea\label{ex:nabi-11}
\begingl
\glpreamble ¡nabue emusuikupa!\\
\gla nabu-e e-musuiku-pa\\
\glb go.\textsc{imp}-2\textsc{pl} 2\textsc{pl}-wash-\textsc{dloc.irr}\\
\glft ‘go and wash!’
\endgl
\trailingcitation{[jxx-e190210s-01]}
\xe
\is{dislocative|)}

%\ea\label{ex:nabi-10}
%\begingl
%\glpreamble “¡nabi epuikupa! temetapujiyu kÿpu”\\
%\gla nabi e-puiku-pa teme-tapu-ji-yu kÿpu\\
%\glb go.\textsc{imp} 2\textsc{pl}-fish-\textsc{dloc.irr} big-\textsc{shell}-\textsc{col}-\textsc{ints} sardine\\
%\glft ‘“go fishing, the sardines are very big!”’\\
%\endgl
%\trailingcitation{[jxx-e150925l-1.160]}
%\xe

\is{motion predicate|)}
\is{suppletive imperative|)}

\subsection{Negative imperatives}\label{sec:Prohibitives}
\is{directive speech act!prohibitive|(}\is{negation!prohibitive|(}

There are several ways to form a negative imperative. First of all, the negative particle\is{negative particle|(} \textit{kuina} can be used together with an irrealis verb. In this case, the negative imperative is identical to a negative declarative clause in structure. 

Second, it is also possible to use the specific negative particles \textit{naka} or \textit{masaini}. The first of them is used to form prohibitives, i.e. commands and requests not to do something. It is possibly related to the negative particle in \isi{Baure}, which is \textit{noka} \citep[cf.][338]{Danielsen2007}. As for \textit{masaini}, this is composed of the apprehensional \isi{connective} \textit{masa} (see \sectref{sec:Conjunctions}) and the \isi{frustrative} marker \textit{-ini} (see \sectref{sec:Frustrative}). María S. seems to use it in the same fashion as \textit{naka}, i.e. in prohibitives, but data from other speakers suggests that it is rather an \isi{admonitive} particle, i.e. it appears in warnings.\is{negative particle|)}

(\ref{ex:nimp-1}) to (\ref{ex:nimp-3}) are examples of negative imperatives with \textit{kuina}. As can be seen in (\ref{ex:nimp-1}), the negative particle precedes the irrealis verb and a declarative sentence would have exactly the same structure.\is{word order} This example was elicited from María S.

\ea\label{ex:nimp-1}
\begingl
\glpreamble ¡patÿkemiu nijinepÿi! ¡kuina piyuabu!\\
\gla pi-a-tÿkemiu ni-jinepÿi kuina pi-iyua-bu\\
\glb 2\textsc{sg}-\textsc{irr}-be.quiet 1\textsc{sg}-daughter \textsc{neg} 2\textsc{sg}-cry.\textsc{irr}-\textsc{dsc}\\
\glft ‘be quiet, my daughter, don’t cry anymore!
\endgl
\trailingcitation{[mrx-e150219s.136]}
\xe

(\ref{ex:nimp-2}) comes from María C. Actually, I am not entirely sure whether this is really meant to be a negative imperative or rather a sentence with future reference (‘you won’t die!’). In any case, María C. tells what she said to her mother, when the latter was poisoned by a sorcerer.

\ea\label{ex:nimp-2}
\begingl
\glpreamble ¡kuina pipaka!\\
\gla kuina pi-paka\\
\glb \textsc{neg} 2\textsc{sg}-die.\textsc{irr}\\
\glft ‘don’t die!’
\endgl
\trailingcitation{[ump-p110815sf.465]}
\xe

The case is clearer in (\ref{ex:nimp-3}). This sentence was produced by Miguel and comes from the story about the fox and the jaguar. This is what the vulture says to the jaguar, when he is supposed to be punished for having let the fox escape. The vulture starts his utterance in Spanish (\textit{no} being the Spanish negative particle), but continues in Paunaka.

\newpage
\ea\label{ex:nimp-3}
\begingl
\glpreamble entonses echÿu sÿmÿ tikechu: “no no no no, ¡kuina pinikanÿ!”\\
\gla entonses echÿu sÿmÿ ti-kechu {no no no no} kuina pi-nika-nÿ\\
\glb thus \textsc{dem}a vulture 3i-say {no no no no} \textsc{neg} 2\textsc{sg}-eat.\textsc{irr}-1\textsc{sg}\\
\glft ‘so the vulture said: “no, no, no, no, don’t eat me!”’
\endgl
\trailingcitation{[jmx-n120429ls-x5.180]}
\xe

In prohibitives, speakers can make use of \textit{naka}.\is{reality status|(} This negative particle is delimited to imperative contexts (commands, requests), so that no ambiguity arises. All examples of prohibitives with \textit{naka} were elicited. They can contain a \isi{realis} verb. Thus it seems that the fact that two parameters trigger \isi{irrealis} here, negation and imperative, is encoded by using the RS marking\is{reality status} that matches none of them.\is{doubly irrealis construction} 

When studying negation among \isi{Arawakan languages}, \citet[]{Michael2014b} %264
identified five types of possible prohibitive constructions. They are given in \tabref{table:ProhibitiveTypes}.
The distinguishing factors include how the negative expression differs from the one found in standard negation (column “expression of negation”) and how the rest of the prohibitive sentence differs from a positive imperative (column “prohibitive construction”).

\begin{table}[htbp]
\caption{Prohibitive construction types by \citet[270]{Michael2014b}}%264

\fittable{
\begin{tabular}{lll}
\lsptoprule
Prohibitive type & Prohibitive construction & Expression of negation \cr
\midrule
Type I & same as imperative & same as standard negation\cr
Type II & same as imperative & different from standard negation\cr
Type III & different from imperative & same as standard negation\cr
Type IV & different from imperative & different from standard negation\cr
Type V & \multicolumn{2}{c}{no distinct prohibitive construction} \cr
\lspbottomrule
 \end{tabular}
}
\label{table:ProhibitiveTypes}
\end{table}
 
According to this classification, the Paunaka prohibitive with \textit{naka} belongs to type IV: the negative particle is different from the one used in standard negation and the rest of the construction is different from the imperative. In his sample of 23 \isi{Arawakan languages}, only Kinikinau and Nanti share this specific behaviour with Paunaka \citep[271]{Michael2014b}.
\footnote{In Nanti, this construction is not delimited to prohibitives, but occurs in any context in which negation and another parameter trigger irrealis marking \citep[272--273]{Michael2014}.}

Consider (\ref{ex:prohib-1}), which comes from Juana. The prohibitive particle \textit{naka} is followed by a realis verb.

\ea\label{ex:prohib-1}
\begingl
\glpreamble ¡naka piyu!\\
\gla naka pi-iyu\\
\glb \textsc{prohib} 2\textsc{sg}-cry\\
\glft ‘don’t cry!’
\endgl
\trailingcitation{[jxx-e120430l-3a]}
\xe

In elicitation, Juana also produced some prohibitives with an irrealis verb like the one in (\ref{ex:prohib-2}); however, the seemingly more spontaneous uses (e.g. the first translation she gave in elicitation) all included realis predicates.

\ea\label{ex:prohib-2}
\begingl
\glpreamble ¡naka piyua!\\
\gla naka pi-iyua\\
\glb \textsc{prohib} 2\textsc{sg}-cry.\textsc{irr}\\
\glft ‘don’t cry!’
\endgl
\trailingcitation{[jxx-p150920l.041]}
\xe

Although María S. rather uses \textit{masaini} as a negative particle in prohibitives (see below), she confirms the use of \textit{naka}. (\ref{ex:prohib-3}) is an example of a prohibitive with \textit{naka} elicited from her.

\ea\label{ex:prohib-3}
\begingl
\glpreamble ¡naka pekubu!\\
\gla naka pi-ekubu\\
\glb \textsc{prohib} 2\textsc{sg}-laugh\\
\glft ‘don’t laugh!’
\endgl
\trailingcitation{[rxx-e150220s-1.08]}
\xe

(\ref{ex:prohib-4}) comes from Juana again.

\ea\label{ex:prohib-4}
\begingl
\glpreamble ¡naka pikupaiku ÿne! tisÿeimuyu\\
\gla naka pi-kupaiku ÿne ti-sÿei-umu-yu\\
\glb \textsc{prohib} 2\textsc{sg}-step.on water 3i-be.cold-\textsc{clf:}liquid-\textsc{ints}\\
\glft ‘don’t step in the water, it is very cold!’
\endgl
\trailingcitation{[jxx-e150925l-1.083-084]}
\xe

It is less clear which RS is required in clauses with \textit{masaini}.\is{admonitive|(}  All examples that follow were elicited, except for the last one, (\ref{ex:adm-7}), and they vary with regard to RS. 

María S. prefers to form prohibitives with \textit{masaini}, and this use was verified by Juana as a valid alternative to those with \textit{naka}. However, all of the examples produced by María S. can be read as warnings, the particle was translated with Spanish \textit{cuidado} ‘caution!, be careful!, watch out!’  by her and Miguel (in the session mrx-e150219s), and it is also warnings that Juana and Miguel use this particle for in (\ref{ex:adm-6}) and (\ref{ex:adm-7}). For this reason it is analysed as an admonitive particle here.

To start with, consider (\ref{ex:adm-2}) from María S. The admonitive particle precedes the realis verb here.

\ea\label{ex:adm-2}
\begingl
\glpreamble ¡masaini pijikupu!\\
\gla masaini pi-jikupu\\
\glb \textsc{adm} 2\textsc{sg}-swallow\\
\glft ‘don’t swallow it!’\\%or:‘be careful not to swallow it’??
\endgl
\trailingcitation{[rxx-e141230s.076]}
\xe

(\ref{ex:adm-2}) also has a realis verb. This is a warning directed to a child not to step on the table lest it topples over. The warning was originally produced in Spanish by María S. and translated to Paunaka by request of Swintha.

\ea\label{ex:adm-1}
\begingl 
\glpreamble ¡masaini pikupachu naka!\\
\gla masaini pi-kupachu naka\\ 
\glb \textsc{adm} 2\textsc{sg}-step.on here\\ 
\glft ‘don’t step on it here!’
\trailingcitation{[mrx-e150219s.150]}
\xe

(\ref{ex:adm-3}) is another warning with a realis verb elicited from María S. to tell Swintha that she should not eat a corncob half-raw.

\ea\label{ex:adm-3}
\begingl
\glpreamble ¡masaini piniku enui! painuepÿi\\
\gla masaini pi-niku enui pi-a-inuepÿi\\
\glb \textsc{adm} 2\textsc{sg}-eat green 2\textsc{sg}-\textsc{irr}-have.wind\\
\glft ‘don’t eat it raw! You will have wind’
\endgl
\trailingcitation{[rxx-e150220s-1.25]}
\xe

In (\ref{ex:adm-4}), which is very similar to the previous example, María S. opted for an irrealis verb.

\ea\label{ex:adm-4}
\begingl
\glpreamble ¡masaini pinika! kuinakuÿ tayu\\
\gla masaini pi-nika kuina-kuÿ ti-a-yu\\
\glb \textsc{adm} 2\textsc{sg}-eat.\textsc{irr} \textsc{neg}-\textsc{incmp} 3i-\textsc{irr}-be.ripe\\
\glft ‘don’t eat it! It is not ripe yet’
\endgl
\trailingcitation{[rxx-e181022le]}
\xe

She also uses an irrealis verb in (\ref{ex:adm-5}). This warning was first uttered in Spanish and then translated. It was directed to a child (as (\ref{ex:adm-1}) above, which comes from the same session).

\ea\label{ex:adm-5}
\begingl
\glpreamble ¡masaini pakupuru!\\
\gla masaini pi-a-kupuru\\
\glb \textsc{adm} 2\textsc{sg}-\textsc{irr}-burn\\
\glft ‘don’t burn yourself!’\\or: ‘be careful, you will burn yourself!’
\endgl
\trailingcitation{[mrx-e150219s.147]}
\xe

In the very same elicitation session, Miguel used \textit{masaini} together with a verb with third person subject and irrealis RS.

\ea\label{ex:adm-6}
\begingl
\glpreamble ¡masaini, tinijabakapi!\\
\gla masaini tinijabakapi\\
\glb \textsc{adm} 3i-bite.\textsc{irr}-2\textsc{sg}\\
\glft ‘be careful, it may bite you!’
\endgl
\trailingcitation{[mrx-e150219s.148]}
\xe

When he repeated the sentence, the verb had realis RS, see (\ref{ex:adm-8}).

\ea\label{ex:adm-8}
\begingl
\glpreamble ¡masaini tinijabakubi kabe!\\
\gla masaini ti-nijabaku-bi kabe\\
\glb \textsc{adm} 3i-bite-2\textsc{sg} dog\\
\glft ‘be careful, the dog may bite you!’
\endgl
\trailingcitation{[mrx-e150219s.149]}
\xe
\is{reality status|)}

Finally, the only non-elicited example with \textit{masaini} is (\ref{ex:adm-7}) and comes from Juana. The particle seems to constitute a whole clause here, the following one starting with the Spanish conditional conjunction \textit{si} ‘if’. The example comes from the creation story and comprises the warning of God that Jesus and María Eva should not go into the garden and eat the apple.\footnote{I do not know what the function of \textit{ta} is. It shows up infrequently. When asked, Miguel once told me that \textit{ta kue} was simply a longer variant of \textit{kue} ‘if, when’; however, occurrence of \textit{ta} is not restricted to combinations with \textit{kue} as becomes apparent from this example.}

\ea\label{ex:adm-7}
\begingl
\glpreamble “ta masaini si eyuna uertiyayae kaku nauku mansana kaku ucheti”\\
\gla ta masaini si e-yuna uerta-yae kaku nauku mansana kaku ucheti\\
\glb ? \textsc{adm} if 2\textsc{pl}-go.\textsc{irr} garden-\textsc{loc} exist there apple exist chili\\
\glft ‘“be careful if you go into the garden, there are apples, there is chili”’
\endgl
\trailingcitation{[jxx-n101013s-1.371-373]}
\xe

\is{admonitive|)} 
\is{negation!prohibitive|)}\is{directive speech act!prohibitive|)}


\subsection{Hortatives}\label{sec:Hortatives}
\is{hortative|(}

Hortatives direct a command, request or invitation to a first person plural, i.e. they include the speaker. Hortatives are formed with the particle \textit{jaje}. This particle usually implies some motion\is{motion predicate} and can be translated as ‘let’s go!’. The case being like this, it can occur on its own (\ref{ex:hort-1}) or it can combine with adverbs (\ref{ex:hort-2}) or verbs (\ref{ex:hort-3}). Interestingly, Miguel also combines it with motion verbs (\ref{ex:hort-4}). 

%Just like imperatives, they can be completely unmarked. In this case, they take a first person plural subject index and irrealis RS.
%multiplicaubina, mxx-p181027l-1.150

(\ref{ex:hort-1}) comes from Miguel’s story about the cowherd and the spirit. The spirit has just told the man, who desperately searches for his cows, that he has taken them. He offers the man to go and have a look at them:

\ea\label{ex:hort-1}
\begingl
\glpreamble “¡jaje!” chikechuchÿji \\
\gla jaje chi-kechu-chÿ-ji\\
\glb \textsc{hort} 3-say-3-\textsc{rprt}\\
\glft ‘“let’s go!” he said to him, it is said
\endgl
\trailingcitation{[mxx-n151017l-1.36]}
\xe

(\ref{ex:hort-2}) is what Juana’s grandmother said to her husband when she realised that there was a water spirit in the arroyo they tried to cross with their cows on their way back from Moxos.

\ea\label{ex:hort-2}
\begingl
\glpreamble “¡jaje nauku anÿke!”\\
\gla jaje nauku anÿke\\
\glb \textsc{hort} there up\\
\glft ‘“let’s go up there!”’
\endgl
\trailingcitation{[jxx-p151016l-2.102]}
\xe

(\ref{ex:hort-3}) also comes from Juana, this example is from the creation story and a citation of God talking to María Eva after she has eaten the apple.

\ea\label{ex:hort-3}
\begingl
\glpreamble “¡pana naka! ¡jaje bana jiriensu tÿpi pimÿuna!”\\
\gla pana naka jaje bi-ana jiriensu tÿpi pi-mÿu-ina\\
\glb come.\textsc{imp} here \textsc{hort} 1\textsc{pl}-make.\textsc{irr} linen \textsc{obl} 2\textsc{sg}-clothes-\textsc{irr.nv}\\
\glft ‘“come here! Let’s go and make linen for your future clothes!”’
\endgl
\trailingcitation{[jxx-n101013s-1.503]}
\xe

In telling about his days in school, Miguel used (\ref{ex:hort-4}) to tell me what other children said to him, when they invited him to join school.

\ea\label{ex:hort-4}
\begingl
\glpreamble “¡jaje biyuna xhikuerayae!”\\
\gla jaje bi-yuna xhikuera-yae\\
\glb \textsc{hort} 1\textsc{pl}-go.\textsc{irr} school-\textsc{loc}\\
\glft ‘“let’s go to school!”’
\endgl
\trailingcitation{[mxx-p181027l-1.006]}
\xe

The hortative particle can take some morphology. It has been found with the \isi{iamitive}, the additive\is{additive} marker and the \isi{emphatic} marker \textit{-ja}. One example with the iamitive is given below. It was produced by Juana in order to teach me this expression. 

\ea\label{ex:hort-5}
\begingl
\glpreamble ¡jajetu! ¡tosetu!\\
\gla jaje-tu tose-tu\\
\glb \textsc{hort}-\textsc{iam} noon-\textsc{iam}\\
\glft ‘let’s go now! It is already noon!’
\endgl
\trailingcitation{[jxx-e110923l-1.084]}
\xe

%The hortative particle can combine with a verb that takes the dislocative marker. In that case, we are dealing with a motion-cum-purpose construction (see \sectref{sec:MotionCumPurpose}). This is the case in (\ref{ex:hort-6}), which comes from Juana and was produced on request by Swintha, because she had been telling her about freshwater snails in Spanish before.
%
%\ea\label{ex:hort-6}
%\begingl 
%\glpreamble ¡jaje binebÿkupa keyu binika!\\
%\gla jaje bi-nebÿku-pa keyu bi-nika\\ 
%\glb \textsc{hort} 1\textsc{pl}-collect-\textsc{dloc.irr} snail 1\textsc{pl}-eat.\textsc{irr}\\ 
%\glft ‘let's go to collect freshwater snails to eat’\\ 
%\endgl
%\trailingcitation{[jxx-e081025s-1.171]}
%\xe
%das einzige Beispiel so, das steht auch schon im AM-Kapitel
\is{hortative|)}
\is{directive speech act|)}

This was the last example in this section. The chapter on simple clauses is almost completed by now. The only sentence type missing being interrogative clauses. They are described in detail in the following section.
%!TEX root = 3-P_Masterdokument.tex
%!TEX encoding = UTF-8 Unicode

\section{Interrogative clauses}\label{sec:Questions}
\is{interrogative clause|(}

While declarative clauses typically assert information, the main function of interrogative clauses is to request information \citep[294]{Payne1997}. Two main types of interrogative clauses can be distinguished. Polar questions seek an affirmation or negation\is{negation} of information already given in the question \citep[291]{Koenig2007}. They can be distinguished from declarative sentences by intonation in Paunaka. Content questions seek information about a participant in an event or some circumstances in the event. They build on a question word. Question words always precede the verb, i.e. they are placed in \isi{focus} position\is{word order} (see \sectref{sec:WordOrder}).

In this section, I first describe polar questions in \sectref{sec:PolarQuestions} and then turn to content questions in \sectref{sec:ContentQuestions}.


\subsection{Polar questions}\label{sec:PolarQuestions}
\is{polar question|(}

The only feature that distinguishes polar questions from declarative sentences is intonation.\is{intonation|(} In polar questions, pitch rises towards the end of the utterance, sometimes considerably, sometimes only slightly.

%special intonation pattern most frequent type of building polar questions (König & Siemund: 292), usually with rising intonation (ebd.)

To start with, consider (\ref{ex:Q1}). The question was produced by Juana and directed to me. It was uttered with a very high pitch towards the end.

\ea\label{ex:Q1}
\begingl
\glpreamble ¿pisachu pinika yÿtÿuku?\\
\gla pi-sachu pi-nika yÿtÿuku\\
\glb 2\textsc{sg}-want 2\textsc{sg}-eat.\textsc{irr} food\\
\glft ‘do you want to eat some food?’
\endgl
\trailingcitation{[jxx-d110923l-2.45]}
\xe

\figref{fig:pitch-Q} shows the pitch analysis for (\ref{ex:Q1}), for which I used Praat.\footnote{Developed by Paul Boersma and David Weenink, Phonetic Sciences, University of Amsterdam, see: https:\\www.fon.hum.uva.nl/praat/ Accessed 2021-04-16}

\begin{figure}

\includegraphics[width=\textwidth]{figures/QuestionPitch.pdf}
\caption{Pitch analysis of the question \textit{¿pisachu pinika yÿtÿuku?}}
\label{fig:pitch-Q}

\end{figure}

\is{intonation|)}

A question can consist of a single verb as in (\ref{ex:Q2}), where María S. asked me whether I had met her sister Clara earlier that day.

\largerpage
\ea\label{ex:Q2}
\begingl
\glpreamble ¿pisimuku?\\
\gla pi-simuku\\
\glb 2\textsc{sg}-find\\
\glft ‘did you meet her?’
\endgl
\trailingcitation{[rxx-e120511l.083]}
\xe

%pichupuiku nauku?, rxx-e120511l.255

(\ref{ex:Q3}) is a question about a third person. It refers to Federico who had rented an apartment in Concepción.

\ea\label{ex:Q3}
\begingl
\glpreamble ¿eka Federico tepajÿka nauku?\\
\gla eka Federico ti-pajÿka nauku\\
\glb \textsc{dem}a Federico 3i-stay.\textsc{irr} there\\
\glft ‘Federico will stay there?’
\endgl
\trailingcitation{[jxx-p110923l-1.089]}
\xe

A polar question can also be formed with a non-verbal predicate.\is{non-verbal predication} The greeting formula in Paunaka is actually a question for one’s condition and builds on the adjective \textit{micha} ‘good’. (\ref{ex:Q7}) is an example, where Juana produced this formula to teach Swintha.

\ea\label{ex:Q7}
\begingl
\glpreamble ¿michabi?\\
\gla micha-bi\\
\glb good-2\textsc{sg}\\
\glft ‘how are you?’ (lit.: ‘are you well?’)
\endgl
\trailingcitation{[jxx-n101013s-1.081]}
\xe

It is not really expected to provide information about ones condition when one is asked the question in (\ref{ex:Q7}). In order to ask for the condition of somebody, speakers use a slightly different wording attaching the \isi{continuous} marker to the adjective, see (\ref{ex:Q8}).\footnote{I cannot provide a recorded source for this question, since I simply do not have a recording of it. Greeting usually took place before I started my recording device.}

\ea\label{ex:Q8}
\begingl
\glpreamble ¿michachaikubi?\\
\gla micha-chaiku-bi\\
\glb good-\textsc{cont}-2\textsc{sg}\\
\glft ‘how are you?’
\endgl
%\trailingcitation{[]}
\xe

The question in (\ref{ex:Q4}) includes the non-verbal existential copula \textit{kaku}. In this specific case, the question rather expresses surprise than a request for information, since the information has been given before; Miguel had already told María S. that we were attacked by little ticks on the way to José’s house. 

\ea\label{ex:Q4}
\begingl
\glpreamble ¿kakutu samuchu?\\
\gla kaku-tu samuchu\\
\glb exist-\textsc{iam} tick\\
\glft ‘there are ticks already?’
\endgl
\trailingcitation{[mrx-c120509l.149]}
\xe

A polar question can also include a negative particle, and in that case it usually includes some greater amount of previous knowledge or some presupposition as  in the following examples.

(\ref{ex:Q5}) was elicited from María S. for the purpose that I could ask her about her knee, which I knew had hurt the days before.

\ea\label{ex:Q5}
\begingl
\glpreamble ¿kuina takutibu pisÿikuke?\\
\gla kuina ti-a-kuti-bu pi-sÿikuke\\
\glb \textsc{neg} 3i-\textsc{irr}-hurt-\textsc{dsc} 2\textsc{sg}-knee\\
\glft ‘doesn’t your knee hurt anymore?
\endgl
\trailingcitation{[rxx-e181022le]}
\xe

(\ref{ex:Q6}) is a negative question\is{negation} produced by Juana. She was talking about Cotoca, a town in the vicinity of Santa Cruz, and it was probably my reaction to what she had just said before that let her suppose that I had never visited the place.

\ea\label{ex:Q6}
\begingl
\glpreamble ¿kuina piyuna?\\
\gla kuina pi-yuna\\
\glb \textsc{neg} 2\textsc{sg}-go.\textsc{irr}\\
\glft ‘you haven’t gone there?’
\endgl
\trailingcitation{[jxx-p120430l-2.551]}
\xe

When answering\is{answer to question|(} a question, the predicate is usually repeated, i.e. Paunaka speakers use “verb-echo answers”\is{verb} \citep[3]{Holmberg2016}. There is an affirmative particle in Paunaka, which is \textit{jaa}, \textit{ja’a} or the like, but it does not suffice as an answer. 

One example of a question-answer pair is (\ref{ex:QA-2}), where María C. asks whether Pedro knows the man she was speaking about, a sorcerer, and Pedro affirms that he knows him.


\ea\label{ex:QA-2}
  \ea\label{ex:QA-2.1}
\begingl
\glpreamble \textup{u:} ¿pichupuiku?\\
\gla pi-chupuiku\\
\glb 2\textsc{sg}-know\\
\glft ‘do you know him?’
\endgl
  \ex\label{ex:QA-2.2}
\begingl
\glpreamble \textup{p:} nichupuiku\\
\gla ni-chupuiku\\
\glb 1\textsc{sg}-know\\
\glft ‘I know him (i.e. yes)’
\endgl
\trailingcitation{[ump-p110815sf.553-554]}
\z
\xe


Another question-answer pair is given in (\ref{ex:QA-1}). The question was asked by Miguel, when he helped Juana digging for loam for her clay pot in the vicinity of Santa Rita. The shell he asks for is used to pull up and smoothen the clay rolls, \textit{nauku} ‘there’ refers to Juana’s house in Santa Cruz, where she lived at that time.

\ea\label{ex:QA-1}
  \ea
\begingl
\glpreamble \textup{m:} ¿pero kaku nauku sipÿ pikeuchi?\\
\gla pero kaku nauku sipÿ pi-keuchi\\
\glb but exist there shell 2\textsc{sg}-\textsc{ins}\\
\glft ‘but do you have shells there?’ (lit.: ‘are there shells with regard to you?’)
\endgl
  \ex
\begingl
\glpreamble \textup{j:} kaku\\
\gla kaku\\
\glb exist\\
\glft ‘there are’
\endgl
\trailingcitation{[jmx-d110918ls-1.098-099]}
\z
\xe

In (\ref{ex:QA-3}) the affirmative particle accompanies the repeated verb in the answer. This example stems from Miguel’s narration about his time in school. It is a citation of a question of his teacher and the answer of one pupil.

\ea\label{ex:QA-3}
  \ea
\begingl
\glpreamble “¿pichuputu eka pitareane?”\\
\gla pi-chupu-tu eka pi-tarea-ne\\
\glb 2\textsc{sg}-know-\textsc{iam} \textsc{dem}a 2\textsc{sg}-excercise-\textsc{possd}\\
\glft ‘“do you know your exercise now?”'
\endgl
  \ex
\begingl
\glpreamble “jaa, nÿchuputu”\\
\gla jaa nÿ-chupu-tu\\
\glb \textsc{afm} 1\textsc{sg}-know-\textsc{iam}\\
\glft ‘“yes, I know it”’
\endgl
\trailingcitation{[mxx-p181027l-1.047]}
\z
\xe

In an answer to a negative question,\is{negation|(} the negative particle occurs together with the predicate to confirm the negative alternative, i.e. Paunaka exhibits a polarity-based system \citep[cf.][140]{Holmberg2016}. An example is (\ref{ex:QA-new23}) from Juana who reproduced what her brother asked her when their other brother had passed away.

\ea\label{ex:QA-new23}
  \ea
\begingl
\glpreamble “¿kuina pisama eka mensaje?”\\
\gla kuina pi-sama eka mensaje\\
\glb \textsc{neg} 2\textsc{sg}-hear.\textsc{irr} \textsc{dem}a message\\
\glft ‘“haven’t you heard (i.e. received) the message?”'
\endgl
  \ex
\begingl
\glpreamble “kuina nisama”\\
\gla kuina ni-sama\\
\glb \textsc{neg} 2\textsc{sg}-hear.\textsc{irr}\\
\glft ‘“no, I haven’t heard it”’
\endgl
\trailingcitation{[jxx-p120430l-2.266-267]}
\z
\xe

In this case, it also seems to be possible to omit the verb and only use the negative particle, as in (\ref{ex:QA-new23-2}) from elicitation with Juana. However, all examples I have stem from elicitation or imagined or remembered dialogues reported by a single speaker, I have not found a single example of a question – answer pair including a negative question that comes from natural conversation between two Paunaka speakers. Thus it remains to be checked whether they employ the same patterns in real conversation.

\ea\label{ex:QA-new23-2}
  \ea
\begingl
\glpreamble ¿kuina pekicha?\\
\gla kuina pi-ekicha\\
\glb \textsc{neg} 2\textsc{sg}-invite.\textsc{irr} \\
\glft ‘didn’t you give him anything (to eat or drink)?'
\endgl
  \ex
\begingl
\glpreamble kuina\\
\gla kuina\\
\glb \textsc{neg}\\
\glft ‘no’
\endgl
\trailingcitation{[jmx-e090727s.179]}
\z
\xe
\is{negation|)}

A negative question can be answered positively by repeating the predicate, which usually has \isi{realis} RS in the answer (if irrealis is not demanded by another factor, e.g. future time reference). This is the case in (\ref{ex:QA-4}), which is a little imagined conversation by  Juana and was triggered by me asking questions about lurking (in trying to make sense of the dislocative marker). This made Juana think about hunting and catching animals.

\ea\label{ex:QA-4}
  \ea
\begingl
\glpreamble ¿kuina tituika? \\
\gla kuina ti-tuika\\
\glb \textsc{neg} 3i-hunt.\textsc{irr}\\
\glft ‘he didn’t catch any (animals)?
\endgl
  \ex
\begingl
\glpreamble tituiku, unya\\
\gla ti-tuiku unya\\
\glb 3i-hunt gray.brocket\\
\glft ‘he caught one, a gray brocket’
\endgl
\trailingcitation{[jxx-e110923l-1.059-060]}
\z
\xe
\is{answer to question|)}

%pikubiakubu max nÿmayu bi- etupunubu tukiu nauku Santa Rita? cux-c120414ls-2.330, 
%
%
%¿pisachu pinika yÿtÿuku? tekomperauchapi eka nijinepÿi jxx-d110923l-2.45 -> gute, deutliche Fragekontur
%¿ee pisachu pinika eka mutu?, jxx-p120430l-2.640 -> viel weniger deutlich!
%
%¿pisachu pibena yumaji? ¿pisachu pebikapu?
%Quieres mecerte (en la hamaca)? Jxx-p150920l.017-018
%
%eka Federico tepajÿka nauku? jxx-p110923l-1.089
%
%
%
%
%
%¿pisachu pea kape?, jxx-e120430l-3a
\is{polar question|)}


\subsection{Content questions}\label{sec:ContentQuestions}
\is{content question|(}

Content questions “receive answers that provide the kind of information specified by the interrogative word” \citep[291]{Koenig2007}.\is{answer to question} Content questions build on question words\is{question word|(} in Paunaka, which are given in \tabref{table:QuestionWords}. The question words are always placed in the first position, i.e. they occupy the position of the sentence that is associated with emphasis.

\begin{table}
\caption{Question words}
\begin{tabularx}{\textwidth}{l>{\raggedright\arraybackslash}p{2.2cm}QQ}
\lsptoprule
Question word & Translation & Category & Source\\
\midrule
\textit{chija} & what, who, whom & subject, object, action, identity & \textit{chi-ija} 3-name ‘his/her/its name’?\\
\textit{(chi)kuyena} & how, why & manner, reason & based on manner verb \textit{-kuye} ‘be like this’\\
\textit{kena} & what about & generic & uncertainty marker \textit{kena}\\
\textit{juchubu} & where, when & location, time & \textit{j-u(-)chu-bu}? \\
\textit{(u)kajane} & how many & quantity & with distributive marker \textit{-jane} \\
\lspbottomrule
 \end{tabularx}
\label{table:QuestionWords}
\end{table}

Question words often, but not always, combine with \isi{focus} expressions. This may be a special form of the transitive verb including a third person marker\is{person marking} following the verb stem, a relative clause\is{relative relation} or a \isi{deranked verb}, see \sectref{sec:3_suffixes}, \sectref{sec:Clefts} and \sectref{sec:Subordination-i} for more information about these constructions. The different question words have different possibilities of combining with one or the other of them. In addition, “plain” finite verbs\is{finite verb} can also be used in questions.

Sometimes, question words take \textit{-chÿ}, \textit{-chÿu} or \textit{-chu}. It is not entirely clear to me, what this form is, and whether it is always the same marker only pronounced differently. As for \textit{-chÿ}, it is found on \textit{chija} and \textit{(u)kajane}. This might be the third person marker.\is{person marking} The form \textit{-chÿu}, which may be a cliticised\is{clitic} form of the demonstrative\is{nominal demonstrative} \textit{echÿu}, occurs on \textit{chija}, \textit{juchubu} and \textit{(chi)kuyena}, \textit{-chu} is only found on \textit{juchubu}. I gloss \textit{-chÿ} as ‘3’, i.e. as third person marker on \textit{(u)kajane}, every other occurrence of \textit{-chÿ}, \textit{-chÿu} or \textit{-chu} is glossed as ‘\textsc{dem}b?’, regardless of its actual pronunciation.\footnote{\isi{Mojeño Trinitario} has a “restrictive clitic” \textit{-chu} (Rose 2021, p.c.). Thus in the future, thorough comparison with this language could shed light on the precise function of the Paunaka form(s) found on question words.}\is{question word|)}

The remainder of this section is structured as follows: \sectref{sec:Q_chija} is about questions for subject and object participants as well as actions, \sectref{sec:Q_juchubu} deals with questions about locations and points in time. In \sectref{sec:Q_chikuyena}, questions for reason and manner are presented, and \sectref{sec:Q_kajane} is about requesting quantities. Finally, \sectref{sec:Q_kena} describes some very general questions based on the uncertainty marker. 

\subsubsection{Questions for persons and things}\label{sec:Q_chija}

The question word \is{question word|(} \textit{chija} ‘what, who’ is used to form different kinds of questions for a referent: it can be used to request for human and non-human entities, animate and non-animate alike. It can be combined with a verbal or non-verbal predicate\is{non-verbal predication} and the requested participant can be a \isi{subject} or an \isi{object} of a verbal clause, one of the constituents of an existential\is{existential clause} or equative clause\is{equative/proper inclusion clause} or an action.

The question word itself could possibly derive from the noun \textit{-ija} ‘name’ with a third person possessor, but in any case it is totally grammaticalised\is{grammaticalisation} and may thus be called an interrogative \isi{pronoun}.\footnote{In addition, \textit{chija} also serves as an indefinite pronoun, see \sectref{sec:IndefinitePronouns}.} This becomes apparent in questions about the name of somebody, in which the word form \textit{chija} doubles, as in (\ref{ex:what-name}), which was produced by María C. to obtain some information about my family.

\ea\label{ex:what-name}
\begingl
\glpreamble ¿chija chija penu?\\
\gla chija chi-ija pi-enu\\
\glb what 3-name 2\textsc{sg}-mother\\
\glft ‘what is your mother’s name?’
\endgl
\trailingcitation{[uxx-p110825l.144]}
\xe

Sometimes, \textit{-chÿu} or \textit{-chÿ} is attached to \textit{chija}, which could be a cliticised nominal demonstrative (\textit{echÿu}). Like the question word itself, attachment of \textit{-chÿu} seems to be relatively grammaticalised,\is{grammaticalisation} because a free demonstrative can co-occur, as in (\ref{ex:what-this}), which represents Juana’s reaction as reported by herself, when she was offered frogs to eat.

\ea\label{ex:what-this}
\begingl
\glpreamble ¿chijachÿu echÿu?\\
\gla chija-chÿu echÿu\\
\glb what-\textsc{dem}b? \textsc{dem}b\\
\glft ‘what is this?’
\endgl
\trailingcitation{[jxx-a120516l-a.479]}
\xe
\is{question word|)}

When asking for an action, speakers make use of the verb \textit{-chabu} ‘do’, which almost exclusively occurs in questions.\footnote{The verb may also be nominalised \textit{-chabukene} ‘deeds, behaviour’, but is not used as a verbal predicate in declarative sentences.} 

The question in (\ref{ex:what-do-1}) belongs to the repertoire of exchange of pleasantries, when meeting each other, like (\ref{ex:Q7}) and (\ref{ex:Q8}) above. It was produced by Isidro when meeting Swintha.

\ea\label{ex:what-do-1}
\begingl
\glpreamble ¿chija pichabu?\\
\gla chija pi-chabu\\
\glb what 2\textsc{sg}-do\\
\glft ‘what are you doing?’
\endgl
\trailingcitation{[mdx-c120416ls.005]}
\xe

The verb can also be used to request what others are doing as in (\ref{ex:what-do-2}), in which María S. asks about her brother.

\ea\label{ex:what-do-2}
\begingl
\glpreamble ¿chija chichabu Miyel?\\
\gla chija chi-chabu Miyel\\
\glb what 3-do Miguel\\
\glft ‘what is Miguel doing?’
\endgl
\trailingcitation{[rxx-e120511l.337]}
\xe

Apart from \textit{-chabu} ‘do’, \textit{chija} is not often combined with a plain \isi{finite verb}, but a few examples occurred nonetheless. Two are given here. 

(\ref{ex:what-eat}) is a question the jaguar asks the fox, when he finds him eating cheese in the story told by María S.

\ea\label{ex:what-eat}
\begingl
\glpreamble “¿chija piniku?”\\
\gla chija pi-niku\\
\glb what 2\textsc{sg}-eat\\
\glft ‘“what are you eating?“’
\endgl
\trailingcitation{[rxx-n120511l-1.031]}
\xe

In (\ref{ex:who-die}), the uncertainty marker on the question word tells us that we are dealing with a rhetorical question, that the one who asks does not expect the addressee to know the answer. Juana reports here what she asked her brother, when he told her that a family member had died, but could not say whom it was, because the message he received to tell him about the death was not clear.

\ea\label{ex:who-die}
\begingl
\glpreamble ¿chijakena tepaku?\\
\gla chija-kena ti-paku\\
\glb what-\textsc{uncert} 3i-die\\
\glft ‘who may have died?’
\endgl
\trailingcitation{[jxx-p120430l-2.285]}
\xe


If the verb is \isi{transitive}, a third person marker\is{person marking} can follow the verb stem, a construction reserved to express argument \isi{focus}. This is the case in the following questions.

(\ref{ex:who-break}) was elicited from María S.

\ea\label{ex:who-break}
\begingl
\glpreamble ¿chija tikurabajikuchÿ nÿnikÿiki?\\
\gla chija ti-kurabajiku-chÿ nÿ-nikÿiki\\
\glb what 3i-break-3 1\textsc{sg}-pot\\
\glft ‘who broke my pot?’
\endgl
\trailingcitation{[rxx-e181024l]}
\xe

(\ref{ex:who-take}) comes from Juana telling the story about how the silk floss tree obtained its big belly-like trunk: it swallowed all of Jesus’ corn, who asks for the fate of this supply of corn here.

\ea\label{ex:who-take}
\begingl
\glpreamble “¿chija tumuchÿ?”\\
\gla chija ti-umu-chÿ\\
\glb what 3i-take-3\\
\glft ‘“who took it?”’
\endgl
\trailingcitation{[jxx-n101013s-1.663]}
\xe

Questions for a possessor \is{possessor|(} are a subtype of this kind of questions, since they all build on a verb.\is{attributive prefix|(} The verb is composed of the attributive prefix \textit{ku-} (see \sectref{sec:AttributiveVerbs}) and either \textit{-peu} ‘animal’, as in (\ref{ex:Q-own-1}), or \textit{-yae} ‘\textsc{grn}’, as in (\ref{ex:Q-own-2}) and usually takes the third person marker\is{person marking} \textit{-chÿ}. The examples were elicited from María S. Both \textit{-peu} and \textit{-yae} also play a role in possession marking of non-possessable nouns (see \sectref{sec:Non-possessables}).

\ea\label{ex:Q-own-1}
\begingl
\glpreamble ¿chija tikupeuchÿ ÿba?\\
\gla chija ti-kupeu-chÿ ÿba\\
\glb what 3i-have.animal-3 pig\\
\glft ‘whose pig is this?’
\endgl
\trailingcitation{[rxx-e201231f.08]}
\xe

\ea\label{ex:Q-own-2}
\begingl
\glpreamble ¿chijakena tikuyaechÿkenatu San Jorge?\\
\gla chija-kena ti-kuyae-chÿ-kena-tu {San Jorge}\\
\glb what-\textsc{uncert} 3i-own-3-\textsc{uncert}-\textsc{iam} {San Jorge}\\
\glft ‘who may be the owner of (the estate) San Jorge now?’
\endgl
\trailingcitation{[rxx-e201231f.34]}
\xe

The usage of an attributive verb based on either \textit{-peu} or \textit{-yae} can be considered the main strategy of asking for a possessor, but alternatively, any other noun can also be added to the attributive prefix to derive\is{derivation} a verb of possession, as in (\ref{ex:Q-own-3}), which was also elicited from María S.

\ea\label{ex:Q-own-3}
\begingl
\glpreamble ¿chijakena tikumÿubanechÿ eka mÿuji?\\
\gla chija-kena ti-kumÿu-bane-chÿ eka mÿu-ji\\
\glb what-\textsc{uncert} 3i-have.garment-\textsc{rem}-3 \textsc{dem}a clothes-\textsc{clf:}soft.mass\\
\glft ‘whose garment may this have been before?’
\endgl
\trailingcitation{[rxx-e201231f.44]}
\xe
\is{possessor|)}
\is{attributive prefix|)}

Finally, \textit{chija} often combines with a relative clause,\is{relative relation} especially when \textit{-kena} is attached. In many relative clauses, the \isi{verb} is totally unmarked, but they can be recognised by being introduced with a demonstrative (see \sectref{sec:HeadlessRC}). Combination of \textit{chija} with a relative clause can be considered a subtype of cleft\is{cleft|(} construction (see \sectref{sec:Clefts}) and reflects the structure of the corresponding question in Spanish: \textit{¿qué/quién será que...?} ‘what/who could it be that...?’.

In (\ref{ex:Q-cleft-1}), Juana starts telling the story about the fox and the jaguarundi, but interrupts herself, because she does not remember which animal the fox met.

\ea\label{ex:Q-cleft-1}
\begingl
\glpreamble i chitupukuku echÿu ¿chijachÿukena echÿu chitupuku?\\
\gla i chi-tupuku-uku echÿu chija-chÿu-kena echÿu chi-tupuku\\
\glb and 3-meet-\textsc{add} \textsc{dem}b what-\textsc{dem}b?-\textsc{uncert} \textsc{dem}b 3-meet\\
\glft ‘and he also met the, what was it that he met?’
\endgl
\trailingcitation{[jmx-n120429ls-x5.301-302]}
\xe

(\ref{ex:Q-cleft-2}) was elicited from María S. The relative clause builds on a non-verbal predicate borrowed from Spanish.

\ea\label{ex:Q-cleft-2}
\begingl
\glpreamble ¿chijakena eka pasau chitÿpi? kuina kapunuinabu \\
\gla chija-kena eka pasau chi-tÿpi kuina kapunu-ina-bu\\
\glb what-\textsc{uncert} \textsc{dem}a pass 3-\textsc{obl} \textsc{neg} come-\textsc{irr.nv}-\textsc{dsc}\\
\glft ‘what may have happened to him that he doesn’t come anymore?’
\endgl
\trailingcitation{[rxx-e181022le]}
\xe

(\ref{ex:Q-cleft-3}) was elicited from Juana. Actually, she was asked to translate “why are the cows afraid?”, but she formed the question differently.

\ea\label{ex:Q-cleft-3}
\begingl
\glpreamble tipikujane eka bakajane, ¿chijakena eka cheikukuikujane?\\
\gla ti-piku-jane eka baka-jane chija-kena eka chÿ-eikukuiku-jane\\
\glb 3i-be.afraid-\textsc{distr} \textsc{dem}a cow-\textsc{distr} what-\textsc{uncert} \textsc{dem}a 3-chase-\textsc{distr}\\
\glft ‘the cows are afraid, what may it be that chases them?’
\endgl
\trailingcitation{[jxx-a110923l.18]}
\xe

Occasionally, in questions with a relative clause,\is{relative relation} the Spanish relativiser \textit{ke} (Span. \textit{que}) shows up instead of a demonstrative, as in (\ref{ex:Q-cleft-4}). This question was asked by Juana, when we had requested a story from her and Miguel.

\ea\label{ex:Q-cleft-4}
\begingl
\glpreamble ¿chijakena ke bakueteachikena?\\
\gla chija-kena ke bi-a-kuetea-chi-kena\\
\glb what-\textsc{uncert} \textsc{rel} 1\textsc{pl}-\textsc{irr}-tell-3-\textsc{uncert}\\
\glft ‘what can we tell her?’
\endgl
\trailingcitation{[jmx-n120429ls-x5.046]}
\xe
\is{cleft|)}

\subsubsection{Questions for locations and time}\label{sec:Q_juchubu}

Questions for location and points in time are formed with the question word\is{question word|(} \textit{juchubu}. The composition of the word is quite opaque, it may be decomposed into \textit{j-u-chu-bu}, with an existential or locative root \textit{-u}, also found in the defective verb \textit{-ubu} ‘be, live’, the thematic suffix \textit{-chu} and the middle marker \textit{-bu}, so it possibly goes back to a verb denoting existence or location at a place. It may also be related to the \isi{uncertain future} particle \textit{uchu} (see \sectref{sec:UncertainFuture}).  As for the first part \textit{j-}, this prefix is also found on the \isi{mirative} particle \textit{jimu} ‘you see, you know, right?’ (see Footnote \ref{fn:mirative} in \sectref{sec:Frust_avertive_optatiev}), but it is not productive in any way. The middle marker\is{middle voice} \textit{-bu} in \textit{juchubu} is dropped in a few examples and sometimes \textit{-chÿu} or \textit{-chu} is attached to the question word. Like \textit{chija} \textit{juchubu} can also be used as an indefinite pronoun, see \sectref{sec:IndefinitePronouns}.

In questions for location, the question word \textit{juchubu} is usually combined with the \isi{copula} \textit{kaku} or the defective verb \textit{-ubu} ‘be, live’.  The first two examples are formed with the non-verbal existential copula \textit{kaku}.\is{question word|)}

(\ref{ex:where-1}) was produced by Miguel in an elicitation session, in which he and Alejo had two identical sets of wooden toys. The arrangement of toys Alejo saw was given and Miguel was supposed to arrange his set of wooden toys in an identical way by asking questions.

\newpage
\ea\label{ex:where-1}
\begingl
\glpreamble ¿juchubu kaku echÿu yÿkÿke?\\
\gla juchubu kaku echÿu yÿkÿke?\\
\glb where exist \textsc{dem}b tree\\
\glft ‘where is the tree?’
\endgl
\trailingcitation{[mtx-e110915ls.19]}
\xe

In (\ref{ex:where-2}), which comes from the story about the fox and the jaguar told by María S., the jaguar asks the fox, where he had obtained the cheese he was eating. There is no NP denoting the cheese, reference is sufficiently clear from the context.

\ea\label{ex:where-2}
\begingl
\glpreamble “¿juchubu kaku?”\\
\gla juchubu kaku\\
\glb where exist\\
\glft “where is some?”
\endgl
\trailingcitation{[rxx-n120511l-1.033]}
\xe

(\ref{ex:where-3}) and (\ref{ex:where-4}) are examples of the use of the verb \textit{-ubu} in a question for a location. In (\ref{ex:where-3}), the location of a third person participant is requested. The verb thus takes a third person marker, more precisely \textit{chÿ-}, since this verb is never found with \textit{ti-}. The example comes from Juana’s story about how the floss silk tree obtained its big trunk. The question is asked by Jesus in the story in order to obtain information about his supply of corn (see also (\ref{ex:who-take}) above).

\ea\label{ex:where-3}
\begingl
\glpreamble “¿juchubu chubu neumuka?”\\
\gla juchubu chÿ-ubu neumuka\\
\glb where 3-be supply\\
\glft ‘“where is the supply (of corn)?”’
\endgl
\trailingcitation{[jxx-n101013s-1.659]}
\xe

In (\ref{ex:where-4}), there is second person reference. It is Miguel’s translation of the book title of the \isi{frog story} by \citet[]{Mayer2003} “Frog, where are you?".

\ea\label{ex:where-4}
\begingl
\glpreamble ¿juchubu pubu peÿyubi?\\
\gla juchubu pi-ubu peÿ-yu-bi\\
\glb where 2\textsc{sg}-be frog-\textsc{ints}-2\textsc{sg}\\
\glft ‘where are you, dear frog?’
\endgl
\trailingcitation{[mox-a110920l-2.197]}
\xe

It seems that any other verb except for \textit{-ubu} usually occurs in deranked form\is{deranked verb} when combined with \textit{juchubu}, although a few exceptions of this are found.

(\ref{ex:where-5}) is such an exception. It has a dynamic finite verb and was directed to José whom Miguel and Swintha met, when they were just on the way to his house to visit him.

\ea\label{ex:where-5}
\begingl
\glpreamble ¿juchubu piyuna?\\
\gla juchubu pi-yuna\\
\glb where 2\textsc{sg}-go.\textsc{irr}\\
\glft ‘where are you going?’
\endgl
\trailingcitation{[mox-c110926s-1.132]}
\xe

In contrast, the following questions are built on a deranked verb. The deranked form of a verb contains the “subordinate” suffix \textit{-i}. This form occurs in several contexts of subordination but not exclusively, see \sectref{sec:Subordination-i} and \sectref{sec:AdverbialModification}.

(\ref{ex:where-6}) was produced by Juana to ask me about the route of my flight back to Germany.

\ea\label{ex:where-6}
\begingl
\glpreamble ¿juchubu piyunia tukiu naka, Argentina?\\
\gla juchubu pi-yun-i-a tukiu naka Argentina\\
\glb where 2\textsc{sg}-go-\textsc{subord}-\textsc{irr} from here Argentina\\
\glft ‘where will you go from here, to Argentina?’
\endgl
\trailingcitation{[jxx-e120516l-1.111]}
\xe

(\ref{ex:where-7}) is from the story about the two men and the devil. Having eaten all meat the men hunted, the devil is still hungry and asks for the pigs’ heads.

\ea\label{ex:where-7}
\begingl
\glpreamble “¿juchubu ebikÿjikiuchÿ echÿu chichÿtijane ÿba?”\\
\gla juchubu e-bikÿjik-i-u-chÿ echÿu chi-chÿti-jane ÿba\\
\glb where 2\textsc{pl}-throw.away-\textsc{subord}-\textsc{real}-3 \textsc{dem}b 3-head-\textsc{distr} pig\\
\glft ‘“where did you throw the pigs’ heads?”’
\endgl
\trailingcitation{[mxx-n101017s-1.046-048]}
\xe

(\ref{ex:where-8}) was translated on request in an elicitation session with Miguel and Juana. Note that Miguel attaches \textit{-chÿu} to the question word (\getfullref{ex:where-8.1}), while Juana does not (\getfullref{ex:where-8.2}). It is not clear whether there is a difference in meaning.

\ea\label{ex:where-8}
  \ea\label{ex:where-8.1}
\begingl
\glpreamble \textup{m:} ¿juchubuchÿu pimukiu?\\
\gla juchubu-chÿu pi-muk-i-u\\
\glb where-\textsc{dem}b? 2\textsc{sg}-sleep-\textsc{subord}-\textsc{real}\\
\glft ‘where do you sleep?’
\endgl
  \ex\label{ex:where-8.2}
\begingl
\glpreamble \textup{j:} ¿juchubu pimukiu? \\
\gla juchubu pi-muk-i-u\\
\glb where 2\textsc{sg}-sleep-\textsc{subord}-\textsc{real}\\
\glft ‘where do you sleep?’
\endgl
\trailingcitation{[jmx-e090727s.362-363]}
\z
\xe

Finally, (\ref{ex:where-9}) also has \textit{-chÿu} on the question word but is combined with a finite verb. The example comes from the recordings made by Riester and reflects the hopelessness of Juan Ch. who knows he is treated badly by his \textit{patrón}, but sees no alternative to staying with him nonetheless.

\ea\label{ex:where-9}
\begingl
\glpreamble ¿juchubuchÿukena biyuna?\\
\gla juchubu-chÿu-kena bi-yuna\\
\glb where-\textsc{dem}b?-\textsc{uncert} 1\textsc{pl}-go.\textsc{irr}\\
\glft ‘where could we go?’
\endgl
\trailingcitation{[nxx-p630101g-1.176]}
\xe

Requesting a location can be considered the primary function of \textit{juchubu}, but it may also be used to ask for a point in time. In the latter case, \textit{juchubu} is usually combined with a word with temporal meaning. In (\ref{ex:when-1}), this is the word \textit{tijai} ‘day’. In this case, the shorter form \textit{juchu} is used, which lacks the middle marker. This form occurs infrequently in the corpus without any notable functional or semantic difference to \textit{juchubu}. The question was asked by María C. in seeking information about which day the workshop on Paunaka would be held.

\ea\label{ex:when-1}
\begingl
\glpreamble ¿juchu tijai?\\
\gla juchu tijai\\
\glb where day\\
\glft ‘what day?’
\endgl
\trailingcitation{[mux-c110810l.015]}
\xe

%juchubu tijai pitupunubu, María S., rmx-e150922l.002

In (\ref{ex:when-2}), \textit{juchubu} combines with \textit{uchu}, a particle denoting an uncertain and in most cases remote future (see \sectref{sec:UncertainFuture}). Juana asked this question, when she was telling me about Cotoca and felt like going there together with me.

\ea\label{ex:when-2}
\begingl
\glpreamble ¿juchubukena uchukena biyuna nauku?\\
\gla juchubu-kena uchu-kena bi-yuna nauku\\
\glb where-\textsc{uncert} \textsc{uncert.fut}-\textsc{uncert} 1\textsc{pl}-go.\textsc{irr} there\\
\glft ‘when may we go there?’
\endgl
\trailingcitation{[jxx-p120430l-2.557]}
\xe

(\ref{ex:when-3}) was produced by Juana in an elicitation session in an imagined beginning of a conversation. Note that the adverb \textit{uchuine} ‘just now', which denotes a point in time some time ago on the same day, has a continuous marker attached here and a third person marker which follows the stem and thus resembles a verb. I do not know why this is the case.

\ea\label{ex:when-3}
\begingl
\glpreamble naka, ¿juchubu chuineneikuchÿ pibÿsÿu?\\
\gla naka juchubu uchuine-neiku-chÿ pi-bÿsÿu\\
\glb here where just.now-\textsc{cont}-3 2\textsc{sg}-come\\
\glft ‘how much time has passed since you came here?’
\endgl
\trailingcitation{[jxx-e150925l-1.038]}
\xe

\hspace*{-5.8pt}Finally, sometimes there is no temporal expression in combination with \textit{juchubu}; usually, this is the case when it is clear enough from the context or due to combination with the verb that a point in time is requested instead of a location. This is the case in the last example in this section, which comes from elicitation with Miguel and has the structure of a \isi{cleft} construction.

\ea\label{ex:when-4}
\begingl
\glpreamble ¿juchubukena echÿu pibÿsÿupupunuka?\\
\gla juchubu-kena echÿu pi-bÿsÿu-pupunuka\\
\glb where-\textsc{uncert} \textsc{dem}b 2\textsc{sg}-come-\textsc{reg.irr}\\
\glft ‘when is it that you come back?’
\endgl
\trailingcitation{[mxx-e090728s-1.48]}
\xe

\subsubsection{Questions for manner and cause}\label{sec:Q_chikuyena}

Questions for manner and reason are formed with \textit{(chi)kuyena}.\is{question word|(} Reason may be an extension of manner, as the overlap resembles the overlap between the instrumental and causal\is{instrument/cause} function of the preposition \textit{-keuchi} (see \sectref{sec:adp-keuchi}). The question word derives from the manner verb \textit{-kuye} ‘be like this’,\is{demonstrative verb} which always takes the third person marker \textit{chi-} in my corpus, even though it is a stative verb by position of irrealis marking (i.e. its irrealis form is \textit{chakuye} – see also \sectref{sec:TransitiveStativeV} –, but this one does not occur in questions). When used as a question word, \textit{-na} is added to \textit{chikuye} yielding \textit{chikuyena}.\footnote{\textit{Chikuyena} or \textit{kuyena} is also occasionally found in non-interrogative contexts, but \textit{chikuye} (or \textit{kuye}) without \textit{-na} is never used as a question word.} It is not clear what kind of suffix this is; it could be the general \isi{classifier} \textit{-na}, which we also find with some adjectives (see \sectref{sec:Adjectives}) or – less probably given the verbal origin – the non-verbal irrealis marker \textit{-ina}. The third person marker is sometimes dropped, thus we also find \textit{kuyena} in questions. 


The question word most often combines with a verb that is introduced by the demonstrative \textit{eka},\is{nominal demonstrative} a construction that we also sometimes find in complementation\is{complement relation} (see \sectref{sec:CC_CMPL}). The verb may be finite\is{finite verb} or deranked.\is{deranked verb} Sometimes, however, no demonstrative is included. \is{question word|)} I will start with some examples that show the use of the question word in requesting manner and then turn to some examples which illustrate its use in questions for reason.

(\ref{ex:Qkuyena-1}) was elicited from María S. It has a deranked verb which is not introduced by a demonstrative.


\ea\label{ex:Qkuyena-1}
\begingl
\glpreamble ¿kuyena panaiuchi yumaji?\\
\gla kuyena pi-ana-i-u-chi yumaji\\
\glb how 2\textsc{sg}-make-\textsc{subord}-\textsc{real}-3 hammock\\
\glft ‘how do you make the hammock?’
\endgl
\trailingcitation{[rxx-e181022le]}
\xe

A deranked verb combined with a demonstrative is found in (\ref{ex:Qkuyena-3}) from Juana telling the story about the fox and the jaguar. This is a question the hungry jaguar asks the fox when he finds him eating cheese.

\ea\label{ex:Qkuyena-3}
\begingl
\glpreamble “¿chikuyena eka pitiuchi eka?” \\
\gla chikuyena eka pi-it-i-u-chi eka\\
\glb how \textsc{dem}a 2\textsc{sg}-master-\textsc{subord}-\textsc{real}-3 \textsc{dem}a\\
\glft ‘“how did you get this?”’
\endgl
\trailingcitation{[jmx-n120429ls-x5.245-246]}
\xe

(\ref{ex:Qkuyena-3}) has a finite verb introduced with \textit{eka}. The sentence produced by Miguel, when he helped Juana dig for loam at a place close to Santa Rita. The remote marker \textit{-bane} on the verb indicates that he knows that she has known this place long before.

\ea\label{ex:Qkuyena-2}
\begingl
\glpreamble ¿chikuyena eka pitupubanechÿ eka muteji? \\
\gla chikuyena eka pi-tupu-bane-chÿ eka muteji\\
\glb how \textsc{dem}a 2\textsc{sg}-find-\textsc{rem}-3 \textsc{dem}a loam\\
\glft ‘how did you once find the loam?’
\endgl
\trailingcitation{[jmx-d110918ls-1.013]}
\xe

(\ref{ex:Qkuyena-4}) could theoretically also be analysed as a request for manner, but it is a rhetorical question in this case, so no answer is expected. In this example, Juana uses finite verbs that are not introduced by \textit{eka}. She told me about a place in the woods, where they wash in a big hollow rock. Apparently, the place is watched over by a spirit, because the clothes are often blown away and disappear in the woods, which are difficult to access with all the plants growing there.

\ea\label{ex:Qkuyena-4}
\begingl
\glpreamble kimenu nauku ¿kuyena biyuna bisemaika bimÿu nauku?\\
\gla kimenu nauku kuyena bi-yuna bi-semaika bi-mÿu nauku?\\
\glb woods there how 1\textsc{pl}-go.\textsc{irr} 1\textsc{pl}-search.\textsc{irr} 1\textsc{pl}-clothes there\\
\glft ‘there are the woods, how could we go and look for our clothes there?’
\endgl
\trailingcitation{[jxx-p151020l-2]}
\xe

The next example clearly does not request manner, because it is about the non-realisation of an action. Here, \textit{-chÿ} is attached to the question word; however, this is less frequent with \textit{(chi)kuyena} than with \textit{chija} or \textit{juchubu}, see \sectref{sec:Q_chija} and \sectref{sec:Q_juchubu} above. Like in (\ref{ex:Qkuyena-4}) above, a finite verb is used, this time in combination with the demonstrative. In Juana’s report, she said this sentence to her daughter, when the latter did not pick up her other daughter at the airport in Spain. Her other daughter was finally deported and had to fly back to Bolivia.

\ea\label{ex:Qkuyena-5}
\begingl
\glpreamble ¿chikuyenachÿ eka kuina piyuna?\\
\gla chikuyena-chÿu eka kuina pi-yuna\\
\glb how-\textsc{dem}b? \textsc{dem}a \textsc{neg} 2\textsc{sg}-go.\textsc{irr}\\
\glft ‘how could you NOT go?’\\or: ‘why didn’t you go?’
\endgl
\trailingcitation{[jxx-p110923l-1.308]}
\xe

(\ref{ex:Qkuyena-7}) allows a manner and reason reading. The question has no verb, but an equative clause is attached to the question word. It was produced by Clara and refers to Swintha’s dreadlocks.

\ea\label{ex:Qkuyena-7}
\begingl
\glpreamble ¿kuyenakena eka chimukiji eka?\\
\gla kuyena-kena eka chi-muki-ji eka\\
\glb how-\textsc{uncert} \textsc{dem}a 3-hair-\textsc{col} \textsc{dem}a\\
\glft ‘how can her hair be like this?’\\or: ‘why is her hair like this?’
\endgl
\trailingcitation{[cux-c120414ls-2.345]}
\xe

The last two examples in this section rather request reason than manner and both make use of deranked verbs.

(\ref{ex:Qkuyena-6}) comes from the story about the fox and the jaguar as told by Miguel. This is what the jaguar asks the vulture, when he discovers that the vulture, who was supposed to watch over the fox, let him escape.

\ea\label{ex:Qkuyena-6}
\begingl
\glpreamble “¿chikuyena eka pikujikiuchi eka kupisaÿrÿ?”\\
\gla chikuyena eka pi-kujik-i-u-chi eka kupisaÿrÿ\\
\glb how \textsc{dem}a 2\textsc{sg}-let.go-\textsc{subord}-\textsc{real}-3 \textsc{dem}a fox\\
\glft ‘“why did you let the fox go?”’
\endgl
\trailingcitation{[jmx-n120429ls-x5.179]}
\xe


Finally, (\ref{ex:Qkuyena-8}) was elicited from Miguel to be able to ask him questions about the process of baking rice bread that we had filmed.

\ea\label{ex:Qkuyena-8}
\begingl
\glpreamble ¿chikuyena eka penukiuchÿ echÿu merÿpune naka latakÿye?\\
\gla chikuyena eka pi-nuk-i-u-chÿ echÿu merÿ-pune naka lata-kÿ-yae\\
\glb how \textsc{dem}a 2\textsc{sg}-put-\textsc{subord}-\textsc{real}-3 \textsc{dem}b plantain-leave here metal.sheet-\textsc{clf:}bounded-\textsc{loc}\\
\glft ‘why do you put plantain leaves on the baking tray?’
\endgl
\trailingcitation{[mxx-e120415ls.058]}
\xe


\subsubsection{Questions for quantities}\label{sec:Q_kajane}

Questions for quantities build on the question word\is{question word|(} \textit{kajane} ‘how many’. It is composed of a root \textit{ka-} and the \isi{distributive} marker \textit{-jane}. As for \textit{ka-}, this is possibly the same root we find in the copula \textit{kaku} and the non-verbal motion predicate \textit{kapunu} ‘come’. This root might be related to the demonstrative \textit{eka} (see \sectref{sec:DemPron}). In some cases, \textit{-chÿ} is added to the question word. Unlike the similar sequence added to other question words, this is never pronounced \textit{-chÿu}, so that there is no reason to believe that it could be a demonstrative. Consequently \textit{-chÿ} is glossed as a third person marker here.\is{person marking}
Sometimes \textit{u-} is placed before the question word and in that case, the marker \textit{-chÿ} always follows, yielding \textit{ukajanechÿ}.\footnote{The word patterns with \textit{punachÿ} ‘other’, which is also sometimes given as \textit{upunachÿ}. When \textit{u-} is added, those words have the regular \isi{iambic} pattern of polysyllabic words (see \sectref{sec:IambicPattern}), while \textit{kajane} and \textit{punachÿ} are both irregularly stressed\is{stress} on the first syllable. This hints at \textit{u} having been a fixed part of the root once.}\is{question word|)}

There are not many examples in the corpus which stem from one of the speakers (there are some more examples that were produced by the researchers though). All but one refer to quantities of countable things.

The only example in which the quantity of a solid object is requested is (\ref{ex:qquan-1}), which was elicited from Miguel, when Swintha wanted to ask him how many baking trays of rice bread he had baked.

\ea\label{ex:qquan-1}
\begingl
\glpreamble ¿kajane latajane?\\
\gla kajane lata-jane\\
\glb how.many metal.sheet-\textsc{distr}\\
\glft ‘how many baking trays?’
\endgl
\trailingcitation{[mxx-e120415ls.093]}
\xe

The question word is used to request age, as in (\ref{ex:qquan-2}), which was elicited from Isidro.

\ea\label{ex:qquan-2}
\begingl
\glpreamble ¿kajane anyu pitÿpi?\\
\gla kajane anyo pi-tÿpi\\
\glb how.many year 2\textsc{sg}-\textsc{obl}\\
\glft ‘how old are you?’
\endgl
\trailingcitation{[dxx-d120416s.073]}
\xe

The same question was translated by María S. using \textit{ukajanechÿ}, see (\ref{ex:qquan-22}). It is not clear whether there is any difference to (\ref{ex:qquan-2}).

\ea\label{ex:qquan-22}
\begingl
\glpreamble ¿ukajanechÿtu anyo pitÿpi?\\
\gla ukajane-chÿ-tu anyo pi-tÿpi?\\
\glb how.many-3-\textsc{iam} year 2\textsc{sg}-\textsc{obl}\\
\glft ‘how old are you?’
\endgl
\trailingcitation{[rxx-e151017l]}
\xe

(\ref{ex:qquan-3}) comes from the story about the fox and the jaguarundi. The fox asks how many different jumps the jaguarundi knows and the latter answers him that he knows only one, while the fox brags about knowing twenty-five. Nonetheless, it is this one jump that saves the jaguarundi later; he escapes on a tree, while the fox is killed by dogs.

\ea\label{ex:qquan-3}
\begingl
\glpreamble “¿kajane eka piyae lanse?” tikechuchÿji\\
\gla kajane eka pi-yae lanse ti-kechu-chÿ-ji\\
\glb how.many \textsc{dem}a 2\textsc{sg}-\textsc{grn} jump 3i-say-3-\textsc{rprt}\\
\glft ‘“how many jumps do you know?” he said to him, it is said’
\endgl
\trailingcitation{[jmx-n120429ls-x5.355]}
\xe

The one example in which \textit{kajane} rather refers to a non-countable noun is given in (\ref{ex:qquan-4}).  Although the semantics of the \isi{distributive} marker rather seems to impede requesting quantity of a \isi{mass noun}, this example suggests that it is possible. This is ultimately a matter of semantic extension.
Juana asks for the quantity of money that I had to pay for my flight to Germany.

\ea\label{ex:qquan-4}
\begingl
\glpreamble ¿kajanechÿ eka tÿmue?\\
\gla kajane-chÿ eka tÿmue\\
\glb how.many-3 \textsc{dem}a money\\
\glft ‘how much money did you pay?’
\endgl
\trailingcitation{[jxx-p120430l-1.157]}
\xe

\subsubsection{Questions of the ‘what about’ type}\label{sec:Q_kena}\is{uncertainty|(}

The uncertainty marker \textit{kena} is not precisely a question word,\is{question word|(} but can be used to form very general questions. Just like the question word in the other kinds of content questions, \textit{kena} is placed in \isi{focus} position. It is usually followed by an NP\is{noun phrase} or sometimes by a relative clause,\is{relative relation} but not directly by a verb.\is{word order} \textit{Kena} signals that there is some uncertainty about a referent. This kind of question can be translated to English with ‘what about X?’. Depending on the context, \textit{kena} can be used to request the identity of someone, her disposition, health, activity etc.\is{question word|)} Some examples follow.

(\ref{ex:name-father}) was produced by María C. to follow up on questions about my parents. Shortly before, she had asked about the name of my mother, see (\ref{ex:what-name}), so it is clear that the information she is seeking is a name here, too.

\ea\label{ex:name-father}
\begingl
\glpreamble ¿kena pia?\\
\gla kena pi-a\\
\glb \textsc{uncert} 2\textsc{sg}-father\\
\glft ‘what about your father?’
\endgl
\trailingcitation{[uxx-p110825l.152]}
\xe

(\ref{ex:qkena-1}) is the counter question to (\ref{ex:qquan-3}). The jaguarundi has replied to the fox, telling him that he knows one jump, and now he wants to know about the repertoire of jumps of the fox.

\ea\label{ex:qkena-1}
\begingl
\glpreamble “¿i kenabi?” chikechuchÿji echÿu tisepiu\\
\gla i kena-bi chi-kechu-chÿ-ji echÿu tisepiu\\
\glb and \textsc{uncert}-2\textsc{sg} 3-say-3-\textsc{rprt} \textsc{dem}b jaguarundi\\
\glft ‘“and what about you?”, said the jaguarundi to him, it is said’
\endgl
\trailingcitation{[jmx-n120429ls-x5.360-361]}
\xe

(\ref{ex:qkena-2}) was produced by Juana when turning a page of the book with the \isi{frog story} and having a first look at the picture.

\ea\label{ex:qkena-2}
\begingl
\glpreamble ¿kena naka?\\
\gla kena naka\\
\glb \textsc{uncert} here\\
\glft ‘what do we have here?’
\endgl
\trailingcitation{[jxx-a120516l-a.033]}
\xe

In (\ref{ex:qkena-3}), Juana ponders about her daughter in Argentina, whom she has not seen in a while.

\ea\label{ex:qkena-3}
\begingl
\glpreamble ¿kenaja nijinepÿi?\\
\gla kena-ja ni-jinepÿi\\
\glb \textsc{uncert}-\textsc{emph}1 1\textsc{sg}-daughter\\
\glft ‘how may my daughter be doing?’
\endgl
\trailingcitation{[jxx-e120516l-1.022]}
\xe

Finally, (\ref{ex:Syn-REL-rep}) is an example in which \textit{kena} combines with a relative clause. María C. asks Miguel about a leaflet with information about the Paunaka Documentation Project, when Miguel had told her that she would receive another one.

\ea\label{ex:Syn-REL-rep}
\begingl
\glpreamble ¿i kena echÿu chinejiku ukuinebu?\\
\gla i kena echÿu chi-nejiku ukuinebu\\
\glb and \textsc{uncert} \textsc{dem}b 3-leave some.time.ago\\
\glft ‘and what about the one he left some days ago?’
\endgl
\trailingcitation{[mux-c110810l.131]}
\xe
\is{uncertainty|)}
\is{content question|)}
\is{interrogative clause|)}



This is the end of the description of simple clauses. The chapter that follows is about different kinds of combinations of clauses and predicates.
