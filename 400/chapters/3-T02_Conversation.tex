%!TEX root = 3-P_Masterdokument.tex
%!TEX encoding = UTF-8 Unicode

%is it possible to insert a page break here?

\section{Text 2: Conversation by Juana and María S.}

This is an excerpt of a conversation between Juana and María S. on the 1st of October 2015.  The Spanish parts of the conversation are not transcribed but maintained on the recording, which can be found here:
\url{https://www.elararchive.org/uncategorized/IO_09aeed25-1ddc-4716-ba3b-41d31379f15c}.
The original, slightly longer recording (audio and video) is archived as jrx-c151001fls-9.

María S. was making adobe bricks at a little pond close to Santa Rita, and some relatives of her were fishing. There were several people around, so the conversation switched to Spanish in-between to include them. The Spanish parts of the conversation are left out here.


\ea%[everyglpreamble=\it, exno={1.}, exnoformat=X]<ex:>
\begingl 
\glpreamble \textup{j:} ta mi- te ¿chija?\\
\gla ta mi- te chija\\ 
\glb ? look? \textsc{seq} what\\ 
\glft ‘look, what is that?’\\ 
\endgl
\xe

\ea%[everyglpreamble=\it, exno={2.}, exnoformat=X]<ex:>
\begingl 
\glpreamble \textup{j:} bucheejane\\
\gla buchee-jane\\ 
\glb catfish.sp-\textsc{distr}\\ 
\glft ‘catfish’ (\textit{Hoplosternum littorale})\\ 
\endgl
\xe

\ea%[everyglpreamble=\it, exno={3.}, exnoformat=X]<ex:>
\begingl 
\glpreamble \textup{r:} mba nauku te tikusachunube kananaji ineejane\\
\gla mba nauku te ti-kusachu-nube kana-na-ji inee-jane\\ 
\glb \textsc{intj} there \textsc{seq} 3i-angle-\textsc{pl} this.size-\textsc{rep}-\textsc{col} wolf.fish\\ 
\glft ‘well, there, where they fish, the wolf fish are that big (showing size with the hands)’\\ 
\endgl
\xe


\ea%[everyglpreamble=\it, exno={4.}, exnoformat=X]<ex:>
\begingl 
\glpreamble \textup{j:} no che\\
\gla {no che}\\ 
\glb \textsc{intj}\\ 
\glft ‘no way’\\ 
\endgl
\xe


\ea%[everyglpreamble=\it, exno={5.}, exnoformat=X]<ex:>
\begingl 
\glpreamble \textup{j:} i kakukena uepitu kaku kujibÿe ekani eka chi- ja\\
\gla i kaku-kena ue-pi-tu kaku kujibÿe eka-ni eka chi- ja\\ 
\glb and exist-\textsc{uncert} water.spirit-\textsc{clf:}long.flexible-\textsc{iam} exist caiman \textsc{dem}a-\textsc{deict} \textsc{dem}a 3- \textsc{afm}\\ 
\glft ‘there may be a spirit now, there are caimans here, the... yes’\\ 
\endgl
\xe


\ea%[everyglpreamble=\it, exno={6.}, exnoformat=X]<ex:>
\begingl 
\glpreamble \textup{r:} kaku kujibÿe, si kaku kujibÿekenatu\\
\gla kaku kujibÿe si kaku kujibÿe-kena-tu\\ 
\glb exist caiman yes exist caiman-\textsc{uncert}-\textsc{iam}\\ 
\glft ‘there are caimans, yes, there may be caimans now’\\ 
\endgl
\xe

\ea%[everyglpreamble=\it, exno={7.}, exnoformat=X]<ex:>
\begingl 
\glpreamble \textup{j:} temena pue\\
\gla temena pue\\ 
\glb big well\\ 
\glft ‘well, big’\\ 
\endgl
\xe

\ea%[everyglpreamble=\it, exno={8.}, exnoformat=X]<ex:>
\begingl 
\glpreamble \textup{r:} nechikue kaku chimukuji\\
\gla nechikue kaku chi-mukuji\\ 
\glb therefore exist 3-nest\\ 
\glft ‘so they have a nest’\\ 
\endgl
\xe


\ea%[everyglpreamble=\it, exno={9.}, exnoformat=X]<ex:>
\begingl 
\glpreamble \textup{j:} nechikue kaku chimukujitu\\
\gla nechikue kaku chi-mukuji-tu\\ 
\glb therefore exist 3-nest-\textsc{iam}\\ 
\glft ‘thus they have a nest now’\\ 
\endgl
\xe

\ea%[everyglpreamble=\it, exno={11.}, exnoformat=X]<ex:>
\begingl 
\glpreamble \textup{r:} mhm\\
\gla mhm\\ 
\glb \textsc{intj}\\ 
\glft ‘mhm’\\ 
\endgl
\xe

\ea%[everyglpreamble=\it, exno={10.}, exnoformat=X]<ex:>
\begingl 
\glpreamble \textup{j:} pero tijarÿjikapu echÿu\\
\gla pero ti-jarÿ-ji-ka-pu echÿu\\ 
\glb but 3i-drag-\textsc{clf:}soft.mass-\textsc{th}1\textsc{.irr}-\textsc{mid} \textsc{dem}b\\ 
\glft ‘but the weed has been dragged (to the bank)’\\ 
\endgl
\xe

\ea%[everyglpreamble=\it, exno={11.}, exnoformat=X]<ex:>
\begingl 
\glpreamble \textup{r:} mba, pero tÿpenu\\
\gla mba pero ti-ÿpenu\\ 
\glb \textsc{intj} but 3i-be.deep\\ 
\glft ‘well, but it is deep’\\ 
\endgl
\xe

\ea%[everyglpreamble=\it, exno={12.}, exnoformat=X]<ex:>
\begingl 
\glpreamble \textup{j:} tÿpenutu ¿pero muteji?\\
\gla t-ÿpenu-tu pero muteji\\ 
\glb 3i-be.deep-\textsc{iam} but loam\\ 
\glft ‘it is deep, but is it muddy?’\\ 
\endgl
\xe

\ea%[everyglpreamble=\it, exno={13.}, exnoformat=X]<ex:>
\begingl 
\glpreamble \textup{r:} muteji pero tÿpenuku no puedo, tisabeichuiyÿkukukÿa\\
\gla muteji pero t-ÿpenu-uku {no puedo} ti-sabeichuiyÿku-kukÿa\\ 
\glb loam but 3i-be.deep-\textsc{add} {I can’t} 3i-be.shallow-\textsc{am.conc.tr}.\textsc{irr}\\ 
\glft ‘muddy, but also deep, I can’t, they only walk in the shallow part near the bank’\\ 
\endgl
\xe



\ea%[everyglpreamble=\it, exno={14.}, exnoformat=X]<ex:>
\begingl 
\glpreamble \textup{j:} aa claro pue bepuikia bikumuyu jajaa\\
\gla aa {claro pue} bi-epuik-i-a bi-kumuyu  jajaa\\ 
\glb \textsc{intj} {obviously}  1\textsc{pl}-fish-\textsc{subord}-\textsc{irr} 1\textsc{pl}-sweat \textsc{afm}\\ 
\glft ‘ah, obviously, fishing, we sweat, yes’ (the second verb is acoustically not well intelligible)\\ 
\endgl
\xe


\ea%[everyglpreamble=\it, exno={15.}, exnoformat=X]<ex:>
\begingl 
\glpreamble \textup{j:} pero bijatÿjika nenabane Turuxhiyae biyunu nakabi ÿne\\
\gla pero bi-jatÿ-ji-ka nena-bane Turuxhi-yae bi-yunu naka-bi ÿne\\ 
\glb but 1\textsc{pl}-pull-\textsc{clf:}soft.mass-\textsc{th}1\textsc{.irr} like-\textsc{rem} Altavista-\textsc{loc} 1\textsc{pl}-go here-1\textsc{pl} water\\ 
\glft ‘but if we throw the weed (to the bank), it is like before in Altavista, when we went, the water was up to here (showing with hands)’\\ 
\endgl
\xe

\ea%[everyglpreamble=\it, exno={16.}, exnoformat=X]<ex:>
\begingl 
\glpreamble \textup{j:} tÿpenu\\
\gla ti-ÿpenu\\ 
\glb 3i-be.deep\\ 
\glft ‘it was deep’\\ 
\endgl
\xe

\ea%[everyglpreamble=\it, exno={17.}, exnoformat=X]<ex:>
\begingl 
\glpreamble \textup{r:} tÿpenu ÿne\\
\gla ti-ÿpenu ÿne\\ 
\glb 3i-be.deep water\\ 
\glft ‘the water was deep’\\ 
\endgl
\xe

\ea%[everyglpreamble=\it, exno={18.}, exnoformat=X]<ex:>
\begingl 
\glpreamble \textup{j:} ¿tiyunutu echÿu pisejane?\\
\gla ti-yunu-tu echÿu pise-jane\\ 
\glb 3i-go-\textsc{iam} \textsc{dem}b bird-\textsc{distr}\\ 
\glft ‘have the birds gone?’ (referring to some birds they were watching earlier)\\ 
\endgl
\xe

\ea%[everyglpreamble=\it, exno={19.}, exnoformat=X]<ex:>
\begingl 
\glpreamble \textup{j:}  kuina tiyunajane\\
\gla  kuina ti-yuna-jane\\ 
\glb \textsc{neg} 3i-go.\textsc{irr}-\textsc{distr}\\ 
\glft ‘they haven’t gone’\\ 
\endgl
\xe

\ea%[everyglpreamble=\it, exno={20.}, exnoformat=X]<ex:>
 \textup{r:} (unintelligable)
\xe


\ea%[everyglpreamble=\it, exno={21.}, exnoformat=X]<ex:>
\begingl 
\glpreamble \textup{j:} kena tapakare\\
\gla kena tapakare\\ 
\glb \textsc{uncert} southern.screamer\\ 
\glft ‘maybe they are southern screamers’ (\textit{Chauna torquata})\\ 
\endgl
\xe

\ea%[everyglpreamble=\it, exno={22.}, exnoformat=X]<ex:>
\begingl 
\glpreamble \textup{r:} teijaneukene\\
\gla ti-eijaneu-kene\\ 
\glb 3i-stink-\textsc{emph}2\\ 
\glft ‘they stink’\\ 
\endgl
\xe

\ea%[everyglpreamble=\it, exno={23.}, exnoformat=X]<ex:>
\begingl 
\glpreamble \textup{j:} ¿michanikikena echÿu?\\
\gla michaniki-kena echÿu\\ 
\glb delicious-\textsc{uncert} \textsc{dem}b\\ 
\glft ‘do they taste good?’\\ 
\endgl
\xe

\ea%[everyglpreamble=\it, exno={24.}, exnoformat=X]<ex:>
\begingl 
\glpreamble \textup{j:}  ¿binika?\\
\gla bi-nika\\ 
\glb 1\textsc{pl}-eat.\textsc{irr}\\ 
\glft ‘can we eat them?’\\ 
\endgl
\xe

\ea%[everyglpreamble=\it, exno={25.}, exnoformat=X]<ex:>
\begingl 
\glpreamble \textup{r:} kena si, nena chijimemeji\\
\gla kena si nena chi-jimemeji\\ 
\glb \textsc{uncert} yes like 3-lung\\ 
\glft ‘maybe yes, it is like lung’\\ 
\endgl
\xe

\ea%[everyglpreamble=\it, exno={25.}, exnoformat=X]<ex:>
\begingl 
\glpreamble \textup{l:} ¿pinika?\\
\gla pi-nika\\ 
\glb 2\textsc{sg}-eat.\textsc{irr}\\ 
\glft ‘you can eat them?’\\ 
\endgl
\xe

\ea%[everyglpreamble=\it, exno={25.}, exnoformat=X]<ex:>
\begingl 
\glpreamble \textup{j:} ja'a\\
\gla ja'a\\ 
\glb \textsc{afm}\\ 
\glft ‘yes’\\ 
\endgl
\xe

(...) conversation in Spanish

\ea%[everyglpreamble=\it, exno={25.}, exnoformat=X]<ex:>
\begingl 
\glpreamble \textup{l:} kuina kuina enikatu\\
\gla kuina kuina enikatu\\ 
\glb \textsc{neg} \textsc{neg} 2\textsc{sg}-eat.\textsc{irr}-\textsc{iam}\\ 
\glft ‘you haven’t eaten them’\\ 
\endgl
\xe

\ea%[everyglpreamble=\it, exno={25.}, exnoformat=X]<ex:>
\begingl 
\glpreamble \textup{j:} kuina\\
\gla kuina\\ 
\glb \textsc{neg}\\ 
\glft ‘no’\\ 
\endgl
\xe

(...) conversation in Spanish


\ea%[everyglpreamble=\it, exno={26.}, exnoformat=X]<ex:>
\begingl 
\glpreamble \textup{j:} aiy michanayu, michanikijane tibÿ- te temenanaji ineejane\\
\gla aiy michana-yu michaniki-jane tibÿ- te temena-na-ji inee-jane\\ 
\glb \textsc{intj} nice-\textsc{ints} delicious-\textsc{distr} ? \textsc{seq} big-\textsc{rep}-\textsc{col} wolf.fish-\textsc{distr}\\ 
\glft ‘aiy, the big wolf fish are very nice, very delicious’\\ 
\endgl
\xe


\ea%[everyglpreamble=\it, exno={27.}, exnoformat=X]<ex:>
\begingl 
\glpreamble \textup{r:} tÿ- temenanaji micha\\
\gla tÿ- temena-na-ji micha\\ 
\glb ? big-\textsc{rep}-\textsc{col} good\\ 
\glft ‘they are very big’\\ 
\endgl
\xe

\ea%[everyglpreamble=\it, exno={28.}, exnoformat=X]<ex:>
\begingl 
\glpreamble \textup{j:} chÿnajikumÿnÿ biyÿtikapu\\
\gla chÿna-jiku-mÿnÿ bi-yÿtikapu\\ 
\glb one-\textsc{lim}-\textsc{dim} 1\textsc{pl}-cook\\ 
\glft ‘we can cook a single one’\\ 
\endgl
\xe

\ea%[everyglpreamble=\it, exno={29.}, exnoformat=X]<ex:>
\begingl 
\glpreamble \textup{r:} mba\\
\gla mba\\ 
\glb \textsc{intj}\\ 
\glft ‘well’\\ 
\endgl
\xe


\ea%[everyglpreamble=\it, exno={30.}, exnoformat=X]<ex:>
\begingl 
\glpreamble \textup{j:} tirakueji micha biyÿtie\\
\gla ti-rakueji micha bi-yÿtie\\ 
\glb 3i-be.full.of.meat good 1\textsc{pl}-food\\ 
\glft ‘and our food is well-fed with meat’\\ 
\endgl
\xe

\ea%[everyglpreamble=\it, exno={31.}, exnoformat=X]<ex:>
\begingl 
\glpreamble \textup{r:} mhm\\
\gla mhm\\ 
\glb \textsc{intj}\\ 
\glft ‘mhm’\\ 
\endgl
\xe

\ea%[everyglpreamble=\it, exno={32.}, exnoformat=X]<ex:>
\begingl 
\glpreamble \textup{j:} ruschÿ nana nÿkuaji\\
\gla ruschÿ nÿ-ana nÿ-kuaji\\ 
\glb two 1\textsc{sg}-make 1\textsc{pl}-net\\ 
\glft ‘two I will make, my nets’\\ 
\endgl
\xe

\ea%[everyglpreamble=\it, exno={33.}, exnoformat=X]<ex:>
 \textup{r:} (unintelligable)
\xe


\ea%[everyglpreamble=\it, exno={34.}, exnoformat=X]<ex:>
\begingl 
\glpreamble \textup{j:} ¿kenaja tibepitÿtÿpaiku nauku?\\
\gla kena-ja ti-bepitÿtÿ-pai-ku nauku\\ 
\glb \textsc{uncert}-\textsc{emph}1 3i-collect-\textsc{clf:}ground-\textsc{th}1 there\\ 
\glft ‘what may it be that he is collecting there?’\\ 
\endgl
\xe

\ea%[everyglpreamble=\it, exno={35.}, exnoformat=X]<ex:>
\begingl 
\glpreamble \textup{r:} chitÿiku\\
\gla chi-tÿiku\\ 
\glb 3-catch\\ 
\glft ‘he caught it’\\ 
\endgl
\xe

(...) conversation in Spanish


\ea%[everyglpreamble=\it, exno={36.}, exnoformat=X]<ex:>
\begingl 
\glpreamble \textup{j:} no che nanakene nikuaji nÿpuipuna\\
\gla {no che} nÿ-ana-kene ni-kuaji nÿ-pui-puna\\
\glb {\textsc{intj}} 1\textsc{sg}-make-\textsc{emph}2- 1\textsc{sg}-net 1\textsc{sg}-fish-\textsc{am.prior}\\
\glft ‘why, no, as for me, I make my net and go fishing’
\endgl
\xe
%verwendet<ex:NMLZ-s1>

\ea%[everyglpreamble=\it, exno={37.}, exnoformat=X]<ex:>
\begingl 
\glpreamble \textup{r:} mba\\
\gla mba\\ 
\glb \textsc{intj}\\ 
\glft ‘well’\\ 
\endgl
\xe


\ea%[everyglpreamble=\it, exno={38.}, exnoformat=X]<ex:>
\begingl 
\glpreamble \textup{j:} kuatruchÿ kuaji, aa pitÿika\\
\gla kuatruchÿ kuaji aa pi-tÿika\\ 
\glb four net \textsc{intj} 2\textsc{sg}-catch.\textsc{irr}\\ 
\glft ‘four nets, ah, you can catch something’\\ 
\endgl
\xe


\ea%[everyglpreamble=\it, exno={39.}, exnoformat=X]<ex:>
\begingl 
\glpreamble \textup{r:} mba, chikuyekenejeatu\\
\gla mba chi-kuye-kene-ja-tu\\ 
\glb \textsc{intj} 3-be.like.this-\textsc{emph}2-\textsc{emph}1-\textsc{iam}\\ 
\glft ‘well, it can be like that’\\ 
\endgl
\xe

\ea%[everyglpreamble=\it, exno={40.}, exnoformat=X]<ex:>
\begingl 
\glpreamble \textup{r:}  bupuna echÿu tipitanÿikukunube tepuikanube\\
\gla bi-upuna echÿu ti-pitanÿi-kuku-nube ti-epuika-nube\\
\glb 1\textsc{pl}-bring.\textsc{irr} \textsc{dem}b 3i-embrace-\textsc{rcpc}-\textsc{pl} 3i-fish.\textsc{irr}-\textsc{pl}\\
\glft ‘let’s bring the ones that embrace each other so that they fish’
\endgl
\xe
%verwendet<ex:RCPC-2>

\ea%[everyglpreamble=\it, exno={41.}, exnoformat=X]<ex:>
\begingl 
\glpreamble \textup{j:} jaja’a tijarÿjika\\
\gla jaja’a ti-jarÿ-ji-ka\\ 
\glb \textsc{afm} 3i-drag-\textsc{clf:}soft.mass-\textsc{th}1\textsc{.irr}\\ 
\glft ‘yes, to drag the weed (out of the water onto the bank)’\\ 
\endgl
\xe

(...) conversation in Spanish


\ea%[everyglpreamble=\it, exno={42.}, exnoformat=X]<ex:>
\begingl 
\glpreamble \textup{j:} siabikena jimu aiy\\
\gla siabi-kena jimu aiy\\ 
\glb pity-\textsc{uncert} fish \textsc{intj}\\ 
\glft ‘what a pity for the fish, aiy’\\ 
\endgl
\xe


\ea%[everyglpreamble=\it, exno={43.}, exnoformat=X]<ex:>
\begingl 
\glpreamble \textup{r:} mba kaku jimu naka\\
\gla mba kaku jimu naka\\ 
\glb \textsc{intj} exist fish here\\ 
\glft ‘well, there are fish here’\\ 
\endgl
\xe

\ea%[everyglpreamble=\it, exno={44.}, exnoformat=X]<ex:>
\begingl 
\glpreamble \textup{j:} nibÿrupekaini kÿnupe\\
\gla ni-bÿru-pe-ka-ini kÿnupe\\
\glb 1\textsc{sg}-suck-\textsc{clf:}flat-\textsc{th}1\textsc{.irr}-\textsc{frust} fish.sp\\
\glft ‘I would suck the (juice out of the) \textit{cupacá} fish’
\endgl
\xe
%verwendet<ex:CLF-OBJ>,<ex:FRUST-COUNT-2>

\ea%[everyglpreamble=\it, exno={45.}, exnoformat=X]<ex:>
\begingl 
\glpreamble \textup{r:} asi kuina taÿpenu nauku las kue tipuru- tipurtujaneu ÿbajane\\
\gla asi kuina ti-a-ÿpenu nauku las kue ti-puru- ti-purtu-jane-u ÿba-jane\\ 
\glb so \textsc{neg} 3-\textsc{irr}-be.deep there the if 3i-put- 3i-put.in-\textsc{distr}-\textsc{real} pig-\textsc{distr}\\ 
\glft ‘so, it is not deep there if the pigs enter’\\ 
\endgl
\xe

\ea%[everyglpreamble=\it, exno={46.}, exnoformat=X]<ex:>
\begingl 
\glpreamble \textup{j:} aja kuina taÿpenuyenu nechÿu\\
\gla aja kuina ti-a-ÿpenu-yenu nechÿu\\
\glb \textsc{intj} \textsc{neg} 3i-\textsc{irr}-be.deep-\textsc{ded} \textsc{dem}c\\ 
\glft ‘aha, so it must be the case that it is not deep there’\\ 
\endgl
\xe
%verwendet<ex:yenu-2>


\ea%[everyglpreamble=\it, exno={47.}, exnoformat=X]<ex:>
\begingl 
\glpreamble \textup{r:} pero kaku kube echÿu ni- \textup{(unintelligable)}\\
\gla pero kaku kube echÿu ni-\\ 
\glb but exist thorn \textsc{dem}b 1\textsc{sg}\\ 
\glft ‘but there are thorns’\\ 
\endgl
\xe

\ea%[everyglpreamble=\it, exno={48.}, exnoformat=X]<ex:>
\begingl 
\glpreamble \textup{j:} entonses chÿneyajiku echÿu\\
\gla entonses chÿ-ne-yae-jiku echÿu\\ 
\glb thus 3-top-\textsc{loc}-\textsc{lim}1 \textsc{dem}b\\ 
\glft ‘so it (the weed) is only on top (of the water)’\\ 
\endgl
\xe


\ea%[everyglpreamble=\it, exno={48.}, exnoformat=X]<ex:>
\begingl 
\glpreamble \textup{x:} ¿michabi?\\
\gla micha-bi\\ 
\glb good-1\textsc{pl}\\ 
\glft ‘how are you?’\\ 
\endgl
\xe

\ea%[everyglpreamble=\it, exno={48.}, exnoformat=X]<ex:>
\begingl 
\gla \textup{j:} eso es\\
\\ 
\glft ‘that’s it’\\ 
\endgl
\xe

\ea%[everyglpreamble=\it, exno={48.}, exnoformat=X]<ex:>
\begingl 
\glpreamble \textup{j:} ¿michabi?\\
\gla micha-bi\\ 
\glb good-1\textsc{pl}\\ 
\glft ‘how are you?’\\ 
\endgl
\xe


\ea%[everyglpreamble=\it, exno={49.}, exnoformat=X]<ex:>
\begingl
\glpreamble \textup{j:} ¿pero metu anaiu echÿu?\\
\gla pero metu e-ana-i-u echÿu\\
\glb but already 2\textsc{pl}-make-\textsc{subord}-\textsc{real} \textsc{dem}b\\
\glft ‘but have you finished doing this?’
\endgl
\xe
%verwendet<ex:finish-adobe>


\ea%[everyglpreamble=\it, exno={50.}, exnoformat=X]<ex:>
\begingl
\glpreamble \textup{r:} kuina metu, pue kupeitu\\
\gla kuina metu pue kupei-tu\\
\glb \textsc{neg} already well afternoon-\textsc{iam}\\
\glft ‘it is not finished, well in the afternoon (we continue)’
\endgl
\xe


\ea%[everyglpreamble=\it, exno={51.}, exnoformat=X]<ex:>
\begingl
\glpreamble \textup{j:}  ja, kupeitukenejatu\\
\gla  ja kupei-tu-kene-ja-tu\\
\glb \textsc{afm} afternoon-\textsc{iam}-\textsc{emph}2-\textsc{emph}1-\textsc{iam}\\
\glft ‘ah, in the afternoon’
\endgl
\xe



\ea%[everyglpreamble=\it, exno={52.}, exnoformat=X]<ex:>
\begingl
\glpreamble \textup{j:} ¡jajepupunuku!\\
\gla jaje-pupunuku\\
\glb \textsc{hort}-\textsc{reg}\\
\glft ‘let’s go back again!’
\endgl
\xe


\ea%[everyglpreamble=\it, exno={53.}, exnoformat=X]<ex:>
\begingl
\glpreamble \textup{r:}  pajÿkanube tikusachanube\\
\gla pajÿka-nube ti-kusacha-nube\\
\glb stay.\textsc{irr}-\textsc{pl} 3i-angle.\textsc{irr}-\textsc{pl}\\
\glft ‘they will stay to fish’
\endgl
\xe


\ea%[everyglpreamble=\it, exno={54.}, exnoformat=X]<ex:>
\begingl
\glpreamble \textup{j:} ja tipajÿku\\
\gla ja ti-pajÿku\\
\glb \textsc{afm} 3i-stay\\
\glft ‘ah, he stays’
\endgl
\xe


(...) conversation in Spanish

\ea%[everyglpreamble=\it, exno={55.}, exnoformat=X]<ex:>
\begingl
\glpreamble \textup{s:} ¿chija eka paunaka?\\
\gla chija eka paunaka\\
\glb what \textsc{dem}a paunaka\\
\glft ‘what is this in Paunaka?’
\endgl
\xe


\ea%[everyglpreamble=\it, exno={56.}, exnoformat=X]<ex:>
\begingl
\glpreamble \textup{r:} ineejane\\
\gla inee-jane\\
\glb wolf.fish-\textsc{distr}\\
\glft ‘wolf fish’
\endgl
\xe


\ea%[everyglpreamble=\it, exno={57.}, exnoformat=X]<ex:>
\begingl
\glpreamble \textup{s:} aa, ineejane, ¿kaku ineejane?\\
\gla aa inee-jane kaku inee-jane \\
\glb \textsc{intj} wolf.fish-\textsc{distr} exist wolf.fish-\textsc{distr}\\
\glft ‘ah wolf fish, are there wolf fish?’
\endgl
\xe


\ea%[everyglpreamble=\it, exno={58.}, exnoformat=X]<ex:>
\begingl
\glpreamble \textup{r:} kaku ineejane, si\\
\gla kaku inee-jane si\\
\glb exist wolf.fish-\textsc{distr} yes\\
\glft ‘there are wolf fish, yes’
\endgl
\xe


\ea%[everyglpreamble=\it, exno={59.}, exnoformat=X]<ex:>
\begingl
\glpreamble \textup{s:} temena\\
\gla temena\\
\glb big\\
\glft ‘they are big’
\endgl
\xe



\ea%[everyglpreamble=\it, exno={60.}, exnoformat=X]<ex:>
\begingl
\glpreamble \textup{j:} a buchee naukuni\\
\gla a buchee nauku-ni\\
\glb or catfish there-\textsc{deict}\\
\glft ‘or there are catfish’
\endgl
\xe

\ea%[everyglpreamble=\it, exno={61.}, exnoformat=X]<ex:>
\begingl
\glpreamble \textup{r:} hm tijipupuku\\
\gla hm ti-jipupuku\\
\glb \textsc{intj} 3i-jump\\
\glft ‘hm, they jump’
\endgl
\xe


\ea%[everyglpreamble=\it, exno={62.}, exnoformat=X]<ex:>
\begingl
\glpreamble \textup{j:} jipupupaijaneu jipupupai-\\
\gla jipupupai-jane-u jipupupai\\
\glb jump-\textsc{distr}-\textsc{real} jump\\
\glft ‘they jump and jump’
\endgl
\xe


\ea%[everyglpreamble=\it, exno={63.}, exnoformat=X]<ex:>
\begingl
\glpreamble \textup{l:} ¿chikeuchi eka?\\
\gla chi-keuchi eka\\
\glb 3-\textsc{ins} \textsc{dem}a\\
\glft ‘with this?’
\endgl
\xe



\ea%[everyglpreamble=\it, exno={64.}, exnoformat=X]<ex:>
\begingl
\glpreamble \textup{r:}  ja’a keuchi chÿajinube\\
\gla  ja’a keuchi chÿajinube\\
\glb \textsc{afm} \textsc{ins} 3-father-\textsc{col}-\textsc{pl}\\
\glft ‘yes, with (lit.: because of) their father’
\endgl
\xe


\ea%[everyglpreamble=\it, exno={65.}, exnoformat=X]<ex:>
\begingl
\glpreamble \textup{r:} chikeuchi chikÿku, chechajinube\\
\gla chi-keuchi chi-kÿku checha-ji-nube\\
\glb 3-\textsc{ins} 3-uncle son-\textsc{col}-\textsc{pl}\\
\glft ‘with their uncle, of the children’
\endgl
\xe



\ea%[everyglpreamble=\it, exno={66.}, exnoformat=X]<ex:>
\begingl
\glpreamble \textup{j:} chichechajinube\\
\gla chi-checha-ji-nube\\
\glb 3-son-\textsc{col}-\textsc{pl}\\
\glft ‘they are his children’
\endgl
\xe


\ea%[everyglpreamble=\it, exno={67.}, exnoformat=X]<ex:>
\begingl
\glpreamble \textup{r:} chibu tikusachu\\
\gla chibu ti-kusachu\\
\glb  3\textsc{top.prn} 3i-angle\\
\glft ‘he is the one who is angling’
\endgl
\xe


\ea%[everyglpreamble=\it, exno={68.}, exnoformat=X]<ex:>
\begingl
\glpreamble \textup{s:} ¿piti piti kuina pi-?\\
\gla piti piti kuina pi-\\
\glb 2\textsc{sg.prn} 2\textsc{sg.prn} \textsc{neg} 2\textsc{sg}-\\
\glft ‘you don’t...?’
\endgl
\xe


\ea%[everyglpreamble=\it, exno={69.}, exnoformat=X]<ex:>
\begingl
\glpreamble \textup{r:} kuina, niyunupunatu nÿbenupuna nikubiakubu\\
\gla kuina ni-yunupuna-tu nÿ-benu-puna ni-kubiakubu\\
\glb \textsc{neg} 1\textsc{sg}-go.back.\textsc{irr}-\textsc{iam} 1\textsc{sg}-lie.down-\textsc{am.prior.irr} 1\textsc{sg}-be.tired\\
\glft ‘no, I will go back now to lie down, I am tired’
\endgl
\xe

laughing

\ea%[everyglpreamble=\it, exno={70.}, exnoformat=X]<ex:>
\begingl
\glpreamble \textup{j:} niyunupunatu nikubiakubu\\
\gla ni-yunupuna-tu ni-kubiakubu\\
\glb 1\textsc{sg}-go.back.\textsc{irr}-\textsc{iam} 1\textsc{sg}-be.tired\\
\glft ‘I will go back now, I am tired’
\endgl
\xe


(...) conversation in Spanish

\ea%[everyglpreamble=\it, exno={71.}, exnoformat=X]<ex:>
\begingl
\glpreamble \textup{r:} i kapunu, ¿van ir ya?\\
\gla i kapunu {van ir ya}\\
\glb and come {will you go now}\\
\glft ‘and he came, will you go now?’
\endgl
\xe

\ea%[everyglpreamble=\it, exno={71.}, exnoformat=X]<ex:>
\begingl
\glpreamble \textup{j:} ¡jaje!\\
\gla jaje\\
\glb \textsc{hort}\\
\glft ‘let’s go!’
\endgl
\xe

\ea%[everyglpreamble=\it, exno={72.}, exnoformat=X]<ex:>
\begingl
\glpreamble \textup{j:} ¡jaje!\\
\gla jaje\\
\glb \textsc{hort}\\
\glft ‘let’s go!’
\endgl
\xe

