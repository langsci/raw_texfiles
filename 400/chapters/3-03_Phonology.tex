%!TEX root = 3-P_Masterdokument.tex
%!TEX encoding = UTF-8 Unicode

\chapter{Phonology}\label{chap:Phonology}

\TabPositions{1cm,2cm,3cm,4cm,5cm,6cm,7cm,8cm,9cm}
This chapter provides a description of the segmental and suprasegmental phonology of Paunaka. The phonemic inventory is presented in \sectref{sec:PhonemicInventory}. In \sectref{Nasalisation}, I describe rhinoglottophilia and vowel nasalisation. \sectref{sec:Orthography} deals with the orthography chosen to transcribe Paunaka throughout this work. 
The morphophonological processes of vowel elision and vowel assimilation at morpheme boundaries, as well as the very restricted process of haplology is the topic of \sectref{sec:PhonProcesses}. \sectref{SyllableStructure} deals with possible syllable types and the minimality requirements of a Paunaka word. \sectref{sec:Stress} is about stress and describes the two rhythmic patterns found in the language. Finally, in \sectref{sec:Intonation}, I share some preliminary observations about intonational patterns.


\section{Phonemic inventory} \label{sec:PhonemicInventory}
\is{segmental phonology|(}
The phonemic inventory of Paunaka is relatively simple with twelve consonants, which are described in \sectref{Consonants}, and five vowels presented in \sectref{Vowels}. Each vowel can follow any other one, e.g. when two morphemes are adjoined; vowel sequences are described in \sectref{Diphthongs}. Section \ref{SoundsLoans} deals with the realisation of sounds in loan words.


\subsection{Consonants}
\label{Consonants}
\is{consonant|(}
Paunaka has twelve contrasting consonants plus another two that only occur in loans.\is{borrowing|(} The consonants are presented in Table~\ref{table:consonants}. All of them occur word- and stem-initially and -internally, with two exceptions. The palatal nasal /ɲ/ is only found in a very small number of words of assumed Paunaka origin and in a number of Spanish loans.\is{borrowing|)} The flap /ɾ/ occurs very infrequently, too. All consonants can precede all vowels, except for the glide /j/, which does not precede /i/. As for /ɲ/, it cannot be said with certainty whether there are any distributional constraints, because there are not enough words containing the nasal to make any firm statement. There is hardly any allophonic variation regarding the realisation of consonants, except for the realisation of the bilabial fricative /β/, which can be [β], [v], [w], or [b] depending on the following vowel. The glottal stop can only appear on the emphatic/affirmative marker,\is{emphatic} which is realised as [hã] or [hãʔã]. It is thus not considered phonemic here.

The consonant inventory is typical for a \isi{Southern Arawakan} language. It also fits the pattern of the greater Arawakan phonemic profile presented by \citet[76]{Aikhenvald1999}. Most noticeable in comparison with other \isi{Arawakan languages} is the absence of the postalveolar fricative /ʃ/ in native words. This sound does occur, but only in loans from \isi{Bésiro} (see \sectref{SoundsLoans}). As already mentioned above, the palatal nasal /ɲ/ occurs very infrequently (see \sectref{Nasals}), which is atypical for Arawakan languages in general, but not for the ones spoken in Bolivia. There are no voiced, aspirated, or palatalised plosives. 

\begin{table}
\caption{Consonant inventory}
\fittable{
\begin{tabular}{lccccccc}
\lsptoprule
& Bilabial & Alveolar & Postalveolar & Retroflex & Palatal & Velar & Glottal \\
\midrule
Plosive & p & t & & & & k & \\

Nasal & m & n & & & ɲ & & \\

Flap &  &  ɾ  &  &  & & & \\

Fricative & β &  s & (ʃ)  & (ʂ) & & & h \\

Affricate &  &  ʧ &  &  & & & \\

Approximant &  &  &  &  & j & & \\
\lspbottomrule
\end{tabular}
}
\label{table:consonants}
\end{table}

%\FloatBarrier

\subsubsection{The plosives} \label{Plosives}
There are three plosives: bilabial /p/, alveolar /t/, and velar /k/. All of them are voiceless and have no voiced allophones.


 The bilabial plosive /p/ is presented in (\getfullref{ex:p.1}) in word-initial position and in (\getfullref{ex:p.2}) in word-internal position.

\ea\label{ex:p}
% %\vtop{\labels\halign{\tl #\hfil\tab \tspace[textoffset]#\hfil\tab\tab \tspace[dimb]#\hfil\\
\ea\label{ex:p.1}   \tab[pi.ˈmu.ko] \tab  /pimuku/ \tab  ‘you (\textsc{sg}) sleep’\\
     \tab{[}pa.ˈta.vi] \tab /pataβi/ \tab ‘sugarcane’\\
     \tab{[}ˈpɛi] \tab\tab /pɛi/ \tab\tab ‘agouti’\\
\ex\label{ex:p.2}    \tab [ˈka.pu.nʊ] \tab  /kapunu/ \tab  ‘he/she/it comes’\\
     \tab{[}ti.ˈpa.ku] \tab /tipaku/ \tab ‘he/she/it dies’\\
     \tab{[}ˈtɨ.pi] \tab /tɨ.pi/ \tab\tab ‘\textsc{obl}’\\%}}
\z
\xe



The alveolar plosive /t/ can occur in word-initial position, as in (\getfullref{ex:t.1}), and in word-internal position as in (\getfullref{ex:t.2}).

\ea\label{ex:t}
% %\vtop{\labels\halign{\tl #\hfil\tab \tspace[textoffset]#\hfil\tab\tab \tspace[dimb]#\hfil\\
\ea\label{ex:t.1}   \tab [ˈti.si] \tab\tab  /tisi/ \tab\tab  ‘it is red’\\
     \tab{[}ta.ˈkɨ.ɾa] \tab /takɨɾa/ \tab ‘chicken’\\
     \tab{[}tɨ.ˈβa.nɛ] \tab /tɨβanɛ/ \tab ‘it is far’\\
\ex\label{ex:t.2}  \tab [ˈnɨ.ti] \tab  /nɨti/ \tab\tab  ‘1\textsc{sg.prn} (‘I’)’\\
     \tab{[}ni.ˈtu.pu] \tab /nitupu/ \tab ‘I find’\\
     \tab{[}pi.ˈti.wa] \tab /pitiβua/ \tab ‘sit down!’\\%}}
\z
\xe
    
The alveolar plosive contrasts with the bilabial plosive, which is shown by the minimal pairs in (\getref{ex:t-p}).

\ea\label{ex:t-p}
    %\vtop{\labels\halign{\tl #\hfil\tab \tspace[textoffset]#\hfil\tab\tab \tspace[dimb]#\hfil\\
\ea\label{ex:t-p.1} \tab  [ti.ˈju.nu] \tab /tijunu/ \tab ‘he/she/it goes’\\
   \tab {[}pi.ˈju.nu] \tab /pijunu/ \tab ‘you (\textsc{sg}) go’\\
\ex\label{ex:t-p.2}  \tab [ˈti.si] \tab\tab /tisi/ \tab\tab ‘it is red’ \\
     \tab {[}ˈpi.sɛ] \tab /pisɛ/ \tab\tab ‘bird’\\
\ex\label{ex:t-p.3}  \tab [mu.ˈtu.ɨ] \tab /mutuɨ/ \tab ‘termite’ \\
      \tab{[}mu.ˈpuɨ] \tab /mupuɨ/ \tab ‘silk floss tree’\\%}}
\z
\xe    
%nÿjatÿku <-> nÿjapÿku



The velar plosive /k/ is shown in (\getfullref{ex:k.1}) in word-initial position and in (\getfullref{ex:k.2}) in word-internal position.

\ea\label{ex:k}
    %\vtop{\labels\halign{\tl #\hfil\tab \tspace[textoffset]#\hfil\tab\tab \tspace[dimb]#\hfil\\
\ea\label{ex:k.1} \tab[ˈka.və] \tab  /kaβɛ/ \tab  ‘dog’\\
     \tab{[}ˈkwi.na] \tab /kuina/ \tab ‘\textsc{neg}’\\
     \tab{[}ki.ˈmɛ.no] \tab /kimɛnu/ \tab ‘woods’\\
\ex\label{ex:k.2}   \tab[ˈna.ka] \tab  /naka/ \tab  ‘here’\\
     \tab{[}ni.ˈkɛ.ʧu] \tab /nikɛʧu/ \tab ‘I say’\\
     \tab{[}pi.ˈni.kʊ] \tab /piniku/ \tab ‘you (\textsc{sg}) eat’\\%}}
\z
\xe
    
The velar plosive contrasts with the alveolar plosive. This can be seen in (\getref{ex:k-t}).

\ea\label{ex:k-t}
    %\vtop{\labels\halign{\tl #\hfil\tab \tspace[textoffset]#\hfil\tab\tab \tspace[dimb]#\hfil\\
\ea      \tab[ˈkɨ.pu] \tab\tab /kɨpu/ \tab ‘sardine’\\
     \tab{[}ˈtɨ.pi] \tab\tab /tɨpi/ \tab\tab ‘\textsc{obl}’\\
\ex  \tab[ni.ˈku.pu] \tab\tab /nikupu/ \tab ‘I go down’\\
     \tab{[}ni.ˈtu.pu] \tab\tab /nitupu/ \tab ‘I find’\\
\ex  \tab[nə.ˈni.ku.ku] \tab /nɨnikuku/ \tab ‘I eat, too’\\
    \tab{[}nə.ˈni.ku.tu] \tab /nɨnikutu/ \tab ‘I already ate’\\%}}
\z
\xe
%nichupa <-> nichuka
    

The plosives are represented by <p>, <t>, and <k>, respectively, throughout this work.

%nijikupu =yo trago <-> nijipuku = yo salto

\subsubsection{The nasals} \label{Nasals}
There are three nasals: bilabial /m/, alveolar /n/, and palatal /ɲ/. While /m/ and /n/ are fully productive, /ɲ/ only occurs in two words of presumable Paunaka origin.

The bilabial nasal /m/ is presented in (\getfullref{ex:m.1}) in word-initial position and in (\getfullref{ex:m.2}) in word-internal position.
 
\ea\label{ex:m}
    %\vtop{\labels\halign{\tl #\hfil\tab \tspace[textoffset]#\hfil\tab\tab \tspace[dimb]#\hfil\\
\ea\label{ex:m.1}    \tab[ˈma.nɛ] \tab  /manɛ/ \tab  ‘morning’\\
     \tab{[}ˈmi.ʧa] \tab /miʧa/ \tab ‘good’\\
     \tab{[}mu.ˈtɛ.pa] \tab /mutɛpa/ \tab ‘dust, earth’\\
\ex\label{ex:m.2}   \tab[ˈsɨ.mɨ] \tab  /sɨmɨ/ \tab  ‘vulture’\\
      \tab{[}ˈa.mɛ] \tab /amɛ/ \tab\tab ‘palm sp.’ (\textit{Attalea princeps})\\  %Span. motacú
     \tab{[}ni.ˈmu.ku] \tab /nimuku/ \tab ‘I sleep’\\%}}
\z
\xe 
 
 The bilabial nasal contrasts with the bilabial plosive. This is shown in (\getref{ex:m-p}).

\ea\label{ex:m-p}
    %\vtop{\labels\halign{\tl #\hfil\tab \tspace[textoffset]#\hfil\tab\tab \tspace[dimb]#\hfil\\
\ea   \tab[ˈmai] \tab /mai/ \tab\tab ‘stone’\\
     \tab{[}ˈpai] \tab\tab /pai/ \tab\tab ‘priest, parson’\\
\ex  \tab[a.ˈmu.kɛ] \tab /amukɛ/ \tab ‘corn’\\
    \tab{[}a.ˈpu.kɛ] \tab /apukɛ/ \tab ‘ground’\\
\ex  \tab[pi.ˈmui.kʊ] \tab /pimuiku/ \tab ‘you (\textsc{sg}) dance’\\
    \tab{[}pǝ.ˈpui.kʊ] \tab /pɛpuiku/ \tab ‘you (\textsc{sg}) fish’\\%}}
\z
\xe
 
The alveolar nasal /n/ occurs in word-initial position, as in (\getfullref{ex:n.1}), and in word-internal position, as in (\getfullref{ex:n.2}).

\ea\label{ex:n}
    %\vtop{\labels\halign{\tl #\hfil\tab \tspace[textoffset]#\hfil\tab\tab \tspace[dimb]#\hfil\\
\ea\label{ex:n.1}     \tab[ˈnɛ.na] \tab  /nɛna/ \tab  ‘like, similar’\\
     \tab{[}ˈni.hã] \tab /niha/ \tab ‘my name’\\
     \tab{[}ˈnui.nə.kɨ] \tab /nuinɛkɨ/ \tab ‘door’\\
\ex\label{ex:n.2}      \tab[pi.ˈni.ku] \tab  /piniku/ \tab  ‘you (\textsc{sg}) eat’\\
      \tab{[}ɛ.ˈsɛ.nu] \tab /ɛsɛnu/ \tab ‘female’ \\
     \tab{[}a.ˈnɨ.kɛ] \tab /anɨkɛ/ \tab ‘up’\\ %}}
\z
\xe 
    
The alveolar nasal contrasts with the alveolar plosive. Some minimal pairs are presented in (\getref{ex:n-t}).

\ea\label{ex:n-t}
    %\vtop{\labels\halign{\tl #\hfil\tab \tspace[textoffset]#\hfil\tab\tab \tspace[dimb]#\hfil\\
\ea    \tab[ni.ˈsa.ʧu] \tab /nisaʧu/ \tab ‘I want’\\
     \tab{[}ti.ˈsa.ʧu] \tab /tisaʧu/ \tab ‘he/she/it wants’\\
\ex     \tab[ti.ˈju.nʊ] \tab /tijunu/ \tab ‘he/she/it goes’\\
     \tab{[}ˈti.ju.tʊ] \tab /tijutu/ \tab ‘it is ripe now’\\
\ex     \tab[ʧɨ.ˈsa.nɛ] \tab /ʧɨsanɛ/ \tab ‘his/her field’\\
     \tab{[}ʧɨ.ˈsa.tə.ko] \tab /ʧɨsatɨku/ \tab ‘he/she cuts’\\%}}
\z
\xe

The alveolar nasal contrasts with the bilabial nasal. There are no exact minimal pairs in my corpus to demonstrate this contrast, but some near minimal pairs do occur. They are presented in (\getref{ex:n-m}) below.

\ea\label{ex:n-m}
    %\vtop{\labels\halign{\tl #\hfil\tab \tspace[textoffset]#\hfil\tab\tab \tspace[dimb]#\hfil\\
\ea     \tab[ˈɨ.nɛ] \tab\tab /ɨnɛ/ \tab\tab ‘water’\\
     \tab{[}ˈɨ.mu] \tab /ɨmu/ \tab\tab ‘piranha’\\
\ex     \tab[ˈni.nə] \tab /ninɛ/ \tab ‘my flea’\\
     \tab{[}ˈmi.mi] \tab /mimi/ \tab ‘mum’\\
\ex     \tab[ˈni.ʧɛu] \tab /niʧɛu/ \tab ‘my feathers’\\
     \tab{[}ˈmi.ʧi] \tab /miʧi/ \tab ‘cat’\\%}}
\z
\xe
   
The palatal nasal /ɲ/ is not fully productive. It only occurs in a few words, some of which are from Spanish and \isi{Bésiro}.\is{borrowing|(} Examples are shown in (\getref{ex:ñ}).


\ea\label{ex:ñ} %\vtop{\halign{%
%#\hfil\tab\tab \qquad #\hfil \\
    [ku.ˈɲũə̃] \tab  /kuɲuɨ/ \tab  ‘tapeti’ (\textit{Sylvilagus brasiliensis})\\
    {[}ˈu.ɲa] \tab  /uɲa/ \tab\tab  ‘gray brocket’ (\textit{Mazama gouazoubira}) \\
     {[}ˈa.ɲo] \tab  /aɲu/ \tab\tab  ‘year’ (from Span. \textit{año}) \\
      {[}u.ˈɲa.ka] \tab  /uɲaka/ \tab  ‘Southern three-banded armadillo’ (\textit{Tolypeutes} \\
     \tab \tab \tab \tab  \textit{matacus}, from Bés. \textit{nuñakax}) \\%}}
 \xe
     
There are no minimal pairs in my corpus to demonstrate a phonemic difference between /ɲ/ and /n/ or /ɲ/ and /j/, but according to the speakers replacement of /ɲ/ by /n/ or /j/ in these words produces nonsense words, thus:


\ea\label{ex:ñ-Sternchen}
\ea     \tab*{[}ku.ˈnuɨ] \\
\tab*{[}ku.ˈjuɨ]
\ex     \tab*{[}ˈu.na] \\
\tab*{[}ˈu.ja]
\z
\xe



An infrequent occurrence such as the one of /ɲ/ may suggest that the sound has been borrowed into the language together with some words, as is the case for the Spanish loans containing the palatal nasal like \textit{anyo} ‘year’ (/aɲo/, Span. \textit{año}). However, the closely related \isi{Mojeño languages} have similar forms for the ‘gray brocket’ – \textit{ona} in Trinitario \citep[32]{Gill1993} and \textit{ana} in Ignaciano \citep[67]{OttOtt1983}.\footnote{\label{fn:arapana}Note, however, that \citet[67]{OttOtt1983} give ‘red brocket’ (\textit{Mazama americana}, Span. \textit{guaso}) as a translation. They gloss the gray brocket as \textit{arapana}, which has a cognate form \textit{urupunu} ‘red brocket’ in Paunaka. There seem to be some semantic shifts involved here.} \isi{Baure} has \textit{nor}. The words may well be cognates, which speaks against the hypothesis that the phoneme was borrowed together with the word.\is{borrowing|)}

In both \isi{Mojeño languages}, Trinitario and Ignaciano, the palatal nasal is rather infrequent, too, but there are more words containing the sound than in Paunaka (\citealp[cf.][]{OttOtt1983}), and it has been claimed that the nasal formed part of the consonant inventory of Proto-Mojeño \citep[13]{CarvalhoRose2018}. \isi{Baure} does not have phonemic /ɲ/.
 %The palatal nasal /ɲ/ is phonemic in the neighbouring languages Bésiro \citep[69]{Sans2010}, Napeka, Kitemoka \citep[58, 61]{Wienold2012}, and Guarayu (-> Dictionaria) and in Spanish, too, which may have reinforced its status as a separate phoneme in Paunaka.

The nasals are spelled as <m>, <n>, and <ny>, respectively, throughout this work.

\subsubsection{The flap}\label{phonology_r}
Paunaka has one flap, the alveolar /ɾ/. Word-initially, it is found very rarely with most of the words being well-integrated loans\is{borrowing|(} from Spanish that originally had initial /d/. The restriction on word-initial /ɾ/ is widespread among the \isi{Arawakan languages} \citep[787]{Mihas2017} and interestingly, it also holds for Bésiro\is{Bésiro|(} where /ɾ/ is restricted to a few Spanish loans word-initially \citep[cf.][61]{Sans2010}. In my corpus, the only word of non-Spanish origin starting with the flap is the noun \textit{rupinu} ‘banana sp.’, which is presented in (\getfullref{ex:r.1}). The same noun (though with a final retroflex [ʂ]: \textit{rupinux}) is used in the local variety of Bésiro spoken in Concepción, but probably not native to this language either considering its own restriction on the sound /ɾ/.\is{Bésiro|)}

Strikingly, \citet[415]{deCarvalhoPAU} found out that Paunaka lost /ɾ/ in inherited words and thus concludes that all words containing this consonant must have been borrowed. However, there are two words containing the flap in Paunaka that have cognates in the other Bolivian Arawakan languages: \textit{urupunu} ‘red brocket’ (\textit{Mazama americana}) and \textit{ajumerku} ‘paper’. As for \textit{urupunu}, the cognate forms are \textit{arapana} in Ignaciano\is{Mojeño Ignaciano}\footnote{See Footnote \ref{fn:arapana} above for the translation given by \citet[]{OttOtt1983}.} and \textit{ropo} in Trinitario,\is{Mojeño Trinitario} while \textit{ajumerku} seems to be related to \isi{Baure} \textit{-ajmer}/\textit{jamerok}, Ignaciano \textit{ajumerucu} and Trinitario \textit{‘jiumeruko}.\footnote{Thanks to Françoise Rose (2021, p.c.) %\iai{Françoise Rose}
for providing the Trinitario words.}\is{borrowing|)}

(\getfullref{ex:r.1}) presents some words that begin with /ɾ/, including Spanish loans. There are some verb stems\is{verbal stem} that begin with /ɾ/ like the onomatopoeic\is{onomatopoeia} verb \textit{\=/ramuku} ‘thunder’, and /ɾ/ is also found stem-internally. Some examples are presented in (\getfullref{ex:r.2}).
%Ignaciano: guineo = apu; plátano = cáerena
%Trinitario: guineo = apu; plátano = queeno


\ea\label{ex:r}
    %\vtop{\labels\halign{\tl #\hfil& \tspace[textoffset]#\hfil&& \tspace[dimb]#\hfil\\
\ea\label{ex:r.1}    \tab[ɾu.ˈpi.no]\tab\tab /ɾupinu/\tab ‘banana sp.’\\
   \tab{[}ˈɾus.ʧɨ]\tab\tab /ɾusʧɨ/\tab ‘two’ (from Span. \textit{dos})\\
    \tab{[}ɾu.ˈmi.ku]\tab\tab /ɾumiku/\tab ‘Sunday’ (from Span. \textit{domingo})\\
\ex\label{ex:r.2}     \tab[ti.ˈɾa.mu.kʊ]\tab /tiɾamuku/\tab ‘it thunders’\\
    \tab{[}nə.ˈhã.ɾə]\tab\tab /nɨhaɾɛ/\tab ‘my namesake’\\
    \tab{[}nə.ˈma.ɾɨ.ku]\tab /nɨmaɾɨku/\tab ‘I cut’\\%}}
\z
 \xe

/ɾ/ contrasts with all other consonants. (\getref{ex:r-t}) lists some minimal pairs with the alveolar /t/.

\ea\label{ex:r-t}
    %\vtop{\labels\halign{\tl #\hfil& \tspace[textoffset]#\hfil&& \tspace[dimb]#\hfil\\
\ea     \tab[ti.ˈɾɨ.ɾɨ.ku]\tab /tɨɾɨɾɨku/\tab ‘it burns’\\
    \tab{[}ti.ˈɾɨ.tɨ.ku]\tab /tɨɾɨtɨku/\tab ‘he/she ties’\\
\ex    \tab[ˈʧɨ.ɾi]\tab\tab /ʧɨɾi/\tab\tab ‘parrot sp.’\\
    \tab{[}ni.ˈʧɨ.ti]\tab /niʧɨti/\tab ‘my head’\\
\ex    \tab[ˈmɛ.ɾɨ]\tab /mɛɾɨ/\tab \tab ‘plantain’\\
    \tab{[}ˈmɛ.tu]\tab /mɛtu/\tab ‘already, now’\\%}}
\z
\xe    

(\getref{ex:r-n}) presents the contrast between /ɾ/ and /n/.
    
\ea\label{ex:r-n}
    %\vtop{\labels\halign{\tl #\hfil& \tspace[textoffset]#\hfil&& \tspace[dimb]#\hfil\\
\ea     \tab[ka.ˈpu.ɾu]\tab /kapuru/\tab ‘catfish sp.’ (\textit{Callichthys spp.}) \\
    \tab{[}ˈka.pu.nu]\tab /kapunu/\tab ‘come’\\
\ex    \tab[pi.ˈɾi.ɾi.ku]\tab /piɾiɾiku/\tab ‘you (\textsc{sg}) knock’\\
    \tab{[}pi.ˈni.ko]\tab /piniku/\tab ‘you (\textsc{sg}) eat’\\
\ex    \tab[ni.ˈhã.ɾɛ]\tab /nihaɾɛ/\tab ‘my namesake’\\
    \tab{[}ˈhã.nɛ]\tab /hanɛ/\tab\tab ‘wasp’\\%}}
\z
\xe     
   
The orthographic representation of the flap is <r> throughout this work.

\subsubsection{The fricatives} \label{par:Fricatives}
There are three contrasting fricatives: the bilabial /β/, the alveolar /s/, and the glottal /h/. All of them can occur word- and stem-initially and -internally. 

The bilabial fricative /β/ \label{par:BilabialFricative} has four allophones [β], [b], [v], and [w]. Paunaka shares this kind of allophony with the other Bolivian Arawakan\is{Southern Arawakan} languages: the \isi{Baure} phoneme is analysed as /v/ with the allophones [v], [β], [b], and [w] \citep[43]{Danielsen2007}, the \isi{Mojeño Ignaciano} phoneme as /β/ with the allophones [β] and [w] \citep[6]{OttOtt1983}, and the Trinitario\is{Mojeño Trinitario} phoneme as /w/ with the allophones [w], [β], [v], and [u̯] \citep[10]{Rose2021}.

In Paunaka, the \isi{vowel} that follows the phoneme detemines the choice of the allophone. [v] is found before the front vowels /i/ and /ɛ/, and [w] before the mid vowel /ɨ/ as well as in the word \textit{-ubiu} ‘house’. The sequences /βu/ and /βa/ are pronounces with either [β] or, more rarely, [b]. The choice of [β] or [b] is dependent on the individual speaker. %, but it may also be influenced by the height of other surrounding vowels. 
 Occasionally, [w] may also be heard before /u/ and /a/ and the sequence /βu/ may be reduced to [w] in rapid speech.
(\getfullref{ex:b-1}) to (\getfullref{ex:b-3}) show all allophones in word-initial and word-internal position. (\getfullref{ex:b-1}) presents examples with the allophone [v].
 


\ea\label{ex:b-1}
    %\vtop{\labels\halign{\tl #\hfil& \tspace[textoffset]#\hfil&& \tspace[dimb]#\hfil\\
\ea    \tab[ˈvi.ti]\tab\tab\tab /βiti/\tab\tab\tab ‘we’\\
    \tab{[}vi.su.ˈma.ʧu]\tab /βisumaʧu/\tab\tab ‘we like’\\
    \tab{[}ˈvɛ.no]\tab\tab /βɛnu/\tab\tab ‘virgin’\\
    \tab{[}ˈvɛ.tu.ku]\tab\tab /βɛtuku/\tab\tab ‘we bring’\\
\ex    \tab[ˈna.vi]\tab\tab /naβi/\tab\tab\tab ‘go!’\\
   \tab{[}ni.ˌku.vi.ˈa.ku.βo]\tab /nikuβiakuβu/\tab ‘I am tired’\\
   \tab{[}ˈka.və]\tab\tab /kaβɛ/\tab\tab\tab ‘dog’\\
   \tab{[}ti.ˈju.nu.ˌnu.vɛ]\tab /tijununuβɛ/\tab\tab ‘they go’\\%}}
\z
\xe

(\getfullref{ex:b-2}) presents examples with the allophone [w].
    
\ea\label{ex:b-2}
    %\vtop{\labels\halign{\tl #\hfil& \tspace[textoffset]#\hfil&& \tspace[dimb]#\hfil\\
\ea    \tab[wɨ.ˈɾɨ.sɨi]\tab /βɨɾɨsɨi/\tab ‘guava’\\
\ex    \tab[ʧi.ˈwɨ.kɛ]\tab /ʧiβɨkɛ/\tab ‘his/her face’\\
   \tab{[}ni.ˈku.wɨ.u]\tab /nikuβɨu/\tab ‘I am drunk’\\
   \tab{[}ˈnu.wiu]\tab /nuβiu/\tab ‘my house’\\
    \tab{[}a.ˈni.wɨ]\tab /aniβɨ/\tab ‘mosquito’\\%}}
\z
\xe
%bibÿsÿpunu, esijibÿ

(\getfullref{ex:b-3}) shows the distribution of the allophones [β] and [b].

\ea\label{ex:b-3}
    %\vtop{\labels\halign{\tl #\hfil& \tspace[textoffset]#\hfil&& \tspace[dimb]#\hfil\\
\ea    \tab[βu.ˈmɛi.ku]\tab /βumɛiku/\tab ‘we steal’\\
   \tab{[}ˈbu.pu.nu]\tab /βupunu/\tab ‘we bring’\\
   \tab{[}βa.ˈmi.ʧu]\tab /βamiʧu/\tab ‘we help’\\
   \tab{[}ˈba.nau]\tab /βanau/\tab ‘we make’\\
\ex    \tab[ˈʧi.bʊ]\tab /ʧiβu/\tab\tab ‘3.\textsc{top.prn}’\\
   \tab{[}ˈhũ.ʧu.βu]\tab /huʧuβu/\tab ‘where’\\
   \tab{[}tɨ.ˈβa.nɛ]\tab /tɨβanɛ/\tab ‘it is far’\\
   \tab{[}ti.ˈkɛ.βa]\tab /tikɛβa/\tab ‘it rains’\\%}}
\z
\xe

The phoneme /β/ contrasts with the bilabial stop /p/, as can be seen in (\getref{ex:b-p}).
 
 \ea\label{ex:b-p}
    %\vtop{\labels\halign{\tl #\hfil& \tspace[textoffset]#\hfil&& \tspace[dimb]#\hfil\\
\ea    \tab[ˈvi.ti]\tab\tab /βiti/\tab\tab ‘we’\\
   \tab{[}ˈpi.ti]\tab\tab /piti/\tab\tab ‘you (\textsc{sg})’\\
\ex    \tab[ni.ˈku.βu]\tab /nikuβu/\tab ‘I bathe’\\
   \tab{[}ni.ˈku.pu]\tab /nikupu/\tab ‘I go down’\\
\ex    \tab[ni.ˈhã.bo]\tab /nihaβu/\tab ‘my thigh’\\
   \tab{[}ˈhã.pu]\tab /hapu/\tab ‘\textit{mate} calabash’\\
%     }}
\z
    \xe

(\getref{ex:b-m}) shows the contrast between /β/ and /m/.
 
 \ea\label{ex:b-m}
    %\vtop{\labels\halign{\tl #\hfil& \tspace[textoffset]#\hfil&& \tspace[dimb]#\hfil\\
\ea    \tab[ˈʧi.βu]\tab /ʧiβu/\tab\tab ‘3\textsc{top.prn}’\\
   \tab{[}ˈʧi.mu]\tab /ʧimu/\tab ‘he/she/it sees him/her/it’\\
\ex    \tab[nɛ.ˈβu.ko]\tab /nɛβuku/\tab ‘I sow, I plant’\\
   \tab{[}ni.ˈmu.ko]\tab /nimuku/\tab ‘I sleep’\\
\ex    \tab[tɨ.ˈβa.nɛ]\tab /tɨβanɛ/\tab ‘far’\\
   \tab{[}ˈmã.nɛ]\tab /manɛ/\tab ‘early morning’\\%}}
\z
\xe
 

The alveolar fricative /s/ occurs word-initially and word-internally; see (\getfullref{ex:s.1}) and (\getfullref{ex:s.2}), respectively.

\ea\label{ex:s}
    %\vtop{\labels\halign{\tl #\hfil& \tspace[textoffset]#\hfil&& \tspace[dimb]#\hfil\\
\ea\label{ex:s.1}    \tab[ˈsɨ.ki]\tab\tab /sɨki/\tab\tab ‘basket’\\
   \tab{[}si.ˈma.pa]\tab /simapa/\tab ‘ash’\\
   \tab{[}ˈsa.ʧɛ]\tab /saʧɛ/\tab\tab ‘sun’\\
\ex\label{ex:s.2}    \tab[ni.ˈsa.ʧu]\tab /nisaʧu/\tab ‘I want’\\
   \tab{[}ku.ˈsɛ.pi]\tab /kusɛpi/\tab ‘thread’\\
   \tab{[}ʧɨ.ˈkɛi.si]\tab /ʧɨkɛisi/\tab ‘its tail’\\%}}
\z
\xe

The alveolar fricative contrasts with all the other consonants, e.g. with the alveolar /ɾ/ in (\getref{ex:s-r}).
  
  \ea\label{ex:s-r}
    %\vtop{\labels\halign{\tl #\hfil& \tspace[textoffset]#\hfil&& \tspace[dimb]#\hfil\\
\ea    \tab[ti.ˈsa.mu.ku]\tab /tisamuku/\tab ‘he/she/it listens’\\
    \tab{[}ti.ˈɾa.mu.ku]\tab /tiɾamuku/\tab ‘it thunders’\\
\ex   \tab[ti.ˈwɨ.sɨu]\tab\tab /tiβɨsɨu/\tab ‘he/she/it arrives’\\
    \tab{[}ti.ˈwɨ.ɾu]\tab\tab /tiβɨɾu/\tab ‘it rots’\\%}}
\z
\xe

(\getref{ex:s-t}) presents the contrast between /s/ and /t/.
 
  \ea\label{ex:s-t}
    %\vtop{\labels\halign{\tl #\hfil& \tspace[textoffset]#\hfil&& \tspace[dimb]#\hfil\\
\ea    \tab[ni.ˈmu.su.hĩ]\tab /nimusuhi/\tab ‘my skin’\\
    \tab{[}ˈmu.tu]\tab\tab /mutu/\tab ‘armadillo’\\
\ex    \tab[ti.ˈsui.kʊ]\tab\tab /tisuiku/\tab ‘he/she writes’\\
    \tab{[}ti.ˈtui.kʊ]\tab\tab /tituiku/\tab ‘he/she/it hunts’\\
\ex    \tab[ˈpi.sɛ]\tab\tab /pisɛ/\tab\tab ‘bird’\\
    \tab{[}ˈpi.ti]\tab\tab\tab /piti/\tab\tab ‘you (\textsc{sg})’ \\%}} %or: bite = bat
\z
    \xe

The contrast between /s/ and /n/ is shown in (\getref{ex:s-n}).

  \ea\label{ex:s-n}
    %\vtop{\labels\halign{\tl #\hfil& \tspace[textoffset]#\hfil&& \tspace[dimb]#\hfil\\
\ea    \tab[pi.ˈsi.ka]\tab /pisika/\tab ‘your (\textsc{sg}) arm’\\
   \tab{[}pi.ˈni.ka]\tab /pinika/\tab ‘you (\textsc{sg}) eat (\textsc{irr})’\\
\ex   \tab[ˈsɨ.ki]\tab\tab /sɨki/\tab\tab ‘basket’\\
    \tab{[}a.ˈnɨ.kǝ]\tab /anɨkɛ/\tab ‘up’\\%}}
\z
\xe 
    %penu = your mother - pisu = you weed
    
(\getref{ex:s-b}) shows some minimal pairs including the fricative /β/.    

  \ea\label{ex:s-b}
    %\vtop{\labels\halign{\tl #\hfil& \tspace[textoffset]#\hfil&& \tspace[dimb]#\hfil\\
\ea    \tab[ˈsi.a]\tab\tab /sia/\tab\tab ‘falcon sp.’\\
    \tab{[}ˈvi.a]\tab\tab /βia/\tab\tab ‘God’ (lit.: our father) \\
\ex   \tab[ni.ˈsa.nɛ]\tab /nisanɛ/\tab ‘my field’\\
    \tab{[}tɨ.ˈβa.nɛ]\tab /tɨβanɛ/\tab ‘far’\\
\ex  \tab[ˈni.su]\tab /nisu/\tab\tab ‘I weed’\\
   \tab{[}ˈnɛ.βu]\tab /nɛβu/\tab ‘3\textsc{obl.top.prn}’\\%}}
\z
 \xe


\is{borrowing|(}
The  postalveolar fricative /ʃ/ and the retroflex fricative /ʂ/ were not originally phonemes of Paunaka, but have entered the language via Bésiro.\is{Bésiro|(} However, they seem not be considered as “foreign” by the speakers, since the words containing these sounds are not considered foreign. The retroflex sound has two allophones, voiceless [ʂ] and voiced [ʐ] in Paunaka.

The sounds /s/ and /ɾ/ of Spanish words are sometimes replaced by /ʂ/ when integrated into Paunaka speech. As for /ɾ/, this sound is also often pronounced as a postalveolar ([ʒ]) or retroflex ([ʐ]) sound in the speakers’ Spanish,\footnote{\citet[32]{Mendoza2015} describes the sound as [z] for all varieties of Bolivian Spanish, but at least the Paunaka speakers tend to pronounce a postalveolar or retroflex sound instead.} but a syllable-final /s/ tends to be [h] in Eastern Bolivian Spanish. 

In any case, all of the words that contain one of the fricatives are loans from other languages, either Bésiro or Spanish or Spanish via Bésiro (see \sectref{SoundsLoans}), but not necessarily identified as such by the speakers.\is{Bésiro|)}

Examples for the postalveolar fricative can be found in (\getref{ex:xh}) and for the retroflex fricative in (\getref{ex:x}).

\ea\label{ex:xh}% %\vtop{\halign{%
%#\hfil\tab\tab \qquad #\hfil \\
    [tu.ˈɾu.ʃi] \tab  /tuɾuʃi/ \tab  ‘\isi{Altavista}’ (toponym)\\
    {[}ˈʃa.bu] \tab  /ʃabu/ \tab  ‘soap’ \\
     {[}ʃi.ˈkwɛ.ɾa]\tab  /ʃikuɛɾa/ \tab  ‘school’ \\
     {[}ɾi.ˈmo.nɛ.ʃi]\tab  /ɾimunɛʃi/ \tab  ‘lemon’ \\
%      }}
 \xe
 
 \ea\label{ex:x}% %\vtop{\halign{%
%#\hfil\tab\tab \qquad #\hfil \\
    [ˈmaʂ] \tab\tab  /maʂ/ \tab\tab  ‘more’ \\
    {[}ˈʐi.o] \tab\tab\tab  /ʂiu/ \tab\tab  ‘river’  \\
     {[}na.ˈɾaŋ.ka.ʂi] \tab  /naɾankaʂi/ \tab  ‘orange’ \\%}}
 \xe
 
%obixhpo / ubixhpu = Bischoff
%rusxenube
%runexeina
%Monkoxinube
%paxkane
%xiu

\is{borrowing|)}

The glottal fricative /h/ has an effect on following vowels\is{vowel} that is perceived as \isi{nasalisation}. This phenomenon has been called \isi{rhinoglottophilia} and is further described in \sectref{Nasalisation}. The glottal fricative can occur at the beginning and in the middle of words, as can be seen in (\getfullref{ex:h.1}) and (\getfullref{ex:h.2}) respectively.


\ea\label{ex:h}
%     %\vtop{\labels\halign{\tl #\hfil& \tspace[textoffset]#\hfil&& \tspace[dimb]#\hfil\\
\ea\label{ex:h.1}    \tab[ˈhãi.kɛ]\tab /haikɛ/\tab ‘star’\\
   \tab{[}ˈhĩ.mu]\tab /himu/\tab ‘fish’\\
   \tab{[}ˈhũ.ʧu.βʊ]\tab /huʧuβu/\tab ‘where’\\
\ex\label{ex:h.2} \tab[ti.pa.ˈhɨ̃.ku]\tab /tipahɨku/\tab ‘he/she/it stays’\\
   \tab{[}ni.ˈhã.βo]\tab /nihaβu/\tab ‘my thigh’\\
   \tab{[}ni.ˈhɛ̃.pɛ.nǝ]\tab /nihɛpɛnɛ/\tab ‘my breast’\\ %}}
\z
\xe

The glottal fricative contrasts with all other consonants, including the fricative /s/, see (\getref{ex:h-s}). 

\ea\label{ex:h-s}
%     %\vtop{\labels\halign{\tl #\hfil& \tspace[textoffset]#\hfil&& \tspace[dimb]#\hfil\\
\ea     \tab[ˈhã.nɛ]\tab\tab /hanɛ/\tab\tab ‘wasp’\\
    \tab{[}ni.ˈsa.nɛ]\tab\tab /nisanɛ/\tab ‘my field’\\
\ex   \tab[ni.ˈhĩ.nɛ.pɨi]\tab\tab /nihinɛpɨi/\tab ‘my daughter’\\
   \tab{[}ni.ˈsi.nɛ.pɨi]\tab\tab /nisinɛpɨi/\tab ‘my grandchild’\\
\ex   \tab[nɨ.ˈhɛ̃.ku.pu]\tab /nɨhɛkupu/\tab ‘I forget’\\
    \tab{[}nɨ.ˈsɛ.ku.pu]\tab /nɨsɛkupu/\tab ‘I resemble’\\%}}
\z
\xe
 
Some minimal pairs with /β/ are shown in (\getref{ex:h-b}).
    
\ea\label{ex:h-b}
%     %\vtop{\labels\halign{\tl #\hfil& \tspace[textoffset]#\hfil&& \tspace[dimb]#\hfil\\
\ea    \tab[ˈhĩ.mu]\tab /himu/\tab ‘fish’\\
    \tab{[}ˈvi.mu]\tab /βimu/\tab ‘we see’\\
\ex   \tab[ˈhã.nɛ]\tab /hanɛ/\tab\tab ‘wasp’\\
    \tab{[}tɨ.ˈβa.nɛ]\tab /tɨβanɛ/\tab ‘it is far’\\
\ex    \tab[ti.ˈkɛ.ʧu.hĩ]\tab /tikɛʧuhi/\tab ‘he/she says, it is said’\\
   \tab{[}ti.ˈkɛ.ʧu.vi]\tab /tikɛʧuβi/\tab ‘he/she says to you (\textsc{sg})/us’\\%}}
\z
\xe
 
The contrast between /h/ and the velar plosive /k/ is presented in (\getref{ex:h-k}). 
 
    
\ea\label{ex:h-k}
%     %\vtop{\labels\halign{\tl #\hfil& \tspace[textoffset]#\hfil&& \tspace[dimb]#\hfil\\
\ea    \tab[nə.ˈhɨ̃.ku]\tab\tab /nɨhɨku/\tab ‘I grow’\\
    \tab{[}nə.ˈkɨ.ku]\tab\tab /nɨkɨku/\tab ‘my uncle’\\
\ex    \tab[ˈhã.pu]\tab\tab /hapu/\tab ‘\textit{mate} calabash’\\
    \tab{[}ˈka.pu.no]\tab\tab /kapunu/\tab ‘he/she/it comes’\\
\ex    \tab[ti.ˈni.ka.nə]\tab\tab /tinikanɨ/\tab ‘he/she/it eats me (\textsc{irr})’\\
    \tab{[}ti.ni.hã.ˈnɛu]\tab /tinihanɛu/\tab ‘they (non-human) eat’\\%}}
\z
\xe

Throughout this work, the spelling of the fricatives is <b> for /β/ and <s> for /s/.  Analogous to \isi{Bésiro}, <xh> is the spelling for /ʃ/ and <x> for /ʂ/. Due to the orthographic conventions of Spanish, <j> represents /h/. 

\subsubsection{The affricate}
There is one affricate, the voiceless postalveolar /ʧ/. It appears word-initially as well as word-internally, as shown in (\getref{ex:ch}).

\ea\label{ex:ch}
    %\vtop{\labels\halign{\tl #\hfil\tab \tspace[textoffset]#\hfil\tab\tab \tspace[dimb]#\hfil\\
\ea     \tab[ˈʧɨ.ɾi] \tab /ʧɨɾi/ \tab\tab ‘parrot sp.’\\
     \tab{[}ʧi.ˈku.jɛ] \tab /ʧikujɛ/ \tab ‘it is like this’\\
     \tab{[}ˈʧi.ma] \tab /ʧima/ \tab ‘her husband’\\
\ex     \tab[kɛ.ˈʧu.ɛ] \tab /kɛʧuɛ/ \tab ‘snake’\\
     \tab{[}ˈmi.ʧa] \tab /miʧa/ \tab ‘good’\\
     \tab{[}ti.ˈʧɛ.mo] \tab /tiʧɛmu/ \tab ‘he/she/it gets up’\\%}}
\z
\xe

/ʧ/ contrasts with the alveolar plosive as well as with the alveolar fricative. (\getref{ex:ch-t}) presents some minimal pairs with /t/.

\ea\label{ex:ch-t}
    %\vtop{\labels\halign{\tl #\hfil\tab \tspace[textoffset]#\hfil\tab\tab \tspace[dimb]#\hfil\\
\ea     \tab[ni.ˈʧu.pu] \tab /niʧupu/ \tab ‘I know’\\
     \tab{[}ni.ˈtu.pu] \tab /nitupu/ \tab ‘I find’\\
\ex    \tab[ˈʧi.ma] \tab /ʧima/ \tab ‘her husband’\\
    \tab{[}ˈti.ma] \tab /tima/ \tab ‘it is cooked’\\
 \ex  \tab[ˈʧi.hã] \tab /ʧiha/ \tab ‘his/her/its name’\\
     \tab{[}ti.ˈhãi] \tab /tihai/ \tab ‘day’\\%}}
\z
    \xe

(\getref{ex:ch-s}) shows the contrast between /ʧ/ and /s/.
    
\ea\label{ex:ch-s}
    %\vtop{\labels\halign{\tl #\hfil\tab \tspace[textoffset]#\hfil\tab\tab \tspace[dimb]#\hfil\\
\ea     \tab[ˈʧɨ.kɨ] \tab /ʧɨkɨ/ \tab\tab ‘arroyo’\\
     \tab{[}ˈsɨ.ki] \tab /sɨki/ \tab\tab ‘basket’\\
\ex  \tab[ˈʧɨ.ʧɨ] \tab /ʧɨʧɨ/ \tab\tab ‘grandpa’\\
     \tab{[}nɨ.ˈsɨ.sɨ] \tab /nɨsɨsɨ/ \tab ‘my nose’\\
\ex  \tab[ˈʧa.ma] \tab /ʧama/ \tab ‘much’\\
     \tab{[}ni.ˈsa.ma] \tab /nisama/ \tab ‘I hear (\textsc{irr})’\\%}}
\z
\xe    %chÿi = its fruit, tisÿi = it is cold

The affricate is represented as <ch> throughout this work.

\subsubsection{The approximant}
Paunaka has one approximant /j/. It can occur word-initially and -internally, as shown by the following examples.

\ea\label{ex:j}
    %\vtop{\labels\halign{\tl #\hfil\tab \tspace[textoffset]#\hfil\tab\tab \tspace[dimb]#\hfil\\
\ea     \tab[ju.ˈma.hĩ] \tab\tab /jumahi/ \tab ‘hammock’\\
     \tab{[}ˈjɨ.pi] \tab\tab /jɨpi/ \tab\tab ‘jar’\\
     \tab{[}ˈjɛ.jǝ] \tab\tab /jɛjɛ/ \tab\tab ‘granny’\\
\ex     \tab[ni.ja.ˈjau.mi] \tab /nijajaumi/ \tab ‘I am happy’\\
     \tab{[}ni.ˈju.nu] \tab\tab /nijinu/ \tab ‘I go’\\
     \tab{[}ˈku.jə.pa] \tab\tab /kujɛpa/ \tab ‘salt’\\%}}
\z
\xe   
    % &{[}ni.jɨ.ˈβa.hĩ.ku] & /nijɨβahiku/ & ‘I grind’\\
%iyu, apimiya, chikuye, chebÿya (manteca), -jiyupÿi (vello), keyu (caracól), kiye? enjabonar, -kiyuraki = brain, ?kubijayeku = play or kubijaiku?, kuyae = totaí, kuyajijiku = laugh, kuyechemu = levantar, kuyeneu = visitar, kuyepa = sal, kuyeririku = fire (v), kuyuine = make bread, -mubukeyu = enfrente, -muyene = yerno, tumuyubu = todo

/j/ does not usually precede /i/ in Paunaka. /j/ could therefore be argued to be an allophone of /i/ in rising diphthongs.\is{vowel sequence} However, onset-less diphthongs are only found word-initially (see \sectref{SyllableStructure}).  

Additional support for the assumption that /j/ is a separate phoneme comes from comparison with the \isi{Mojeño languages} and \isi{Baure}, where the phonemic status of /j/ is clearer (\citealp[cf.][63]{Rose2014a}; \citealt[48--49]{Danielsen2007}). Some cognates are listed in \tabref{table:/j/}:\footnote{Data were taken from \citet[]{Gill1993} for Trinitario and \citet[]{OttOtt1983} for Ignaciano, the Baure words were looked up in the unpublished Toolbox dictionary produced within the Baure Documentation Project. A Spanish-Baure version of the dictionary is currently being prepared by the \textit{Instituto de Lengua y Cultura Baure} and will possibly be published in 2024 (Danielsen 2023, p.c.).}


\begin{table}
\caption{Cognate stems with /j/ (<y>) in Paunaka, Trinitario, Ignacio, and Baure}

\begin{tabularx}{\textwidth}{XXXlQ}
\lsptoprule
Paunaka & Trinitario & Ignaciano & Baure & Translation \\
\midrule
\textit{-yenu} & \textit{-yeno} & \textit{-yena} & \textit{-eyon} & wife \\

\textit{-iyu} & \textit{-iyo'o} & \textit{-íyaha} & \textit{-ya} & cry \\

\textit{yÿpi} & \textit{yupi}  &  \textit{yupi}  & \textit{jopi} & jar \\

\textit{yuti} & \textit{yoti} & \textit{yati} & \textit{yotoe'}; \textsc{clf}: \textit{-yiti} & night \\

\textit{yÿkÿ} & \textit{yucu} & \textit{yucu} & \textit{yaki} & fire \\

\textit{kuyae} & \textit{kiara}  & \textit{cayara} &  \textit{koyoroeawok} & palm sp. (\textit{Acrocomia aculeata}) \\

\textit{-yunu} & \textit{-yono}  & \textit{-yana} &  \textit{-yon} & go, walk \\
\lspbottomrule
\end{tabularx}

\label{table:/j/}
\end{table}

/j/ contrasts with all the other consonants. Two minimal pairs with the glottal fricative /h/ are presented in (\getref{ex:j-h}).

\ea\label{ex:j-h}
    %\vtop{\labels\halign{\tl #\hfil\tab \tspace[textoffset]#\hfil\tab\tab \tspace[dimb]#\hfil\\
\ea     \tab[ni.ˈjɨ.kʊ] \tab\tab /nijɨku/ \tab ‘I shoot’\\
     \tab{[}ni.ˈhɨ̃.kʊ] \tab\tab /nihɨku/ \tab ‘I grow’\\
\ex     \tab[ni.kʊ.ˈjui.nɛ] \tab /nikujuinɛ/ \tab ‘I make bread’\\
     \tab{[}ku.ˈhũ.ɛ] \tab\tab /kuhuɛ/ \tab ‘cotton’\\%}}
\z
   \xe

The approximant is spelled <y> throughout this work.
\is{consonant|)}

\subsection{Vowels}\label{Vowels}
\is{vowel|(}

There are five phonemic vowels, which are presented in \tabref{table:vowels}. This set of vowels is typical for an Arawakan language\is{Arawakan languages}, but unlike a number of other Arawakan languages, Paunaka has no contrastive long vowels. The same is true for closely related \isi{Baure} and \isi{Mojeño Ignaciano} \citep[cf.][76, 78]{Aikhenvald1999}.

\begin{table}
\caption{Vowel inventory}

\begin{tabular}{lccc}
\lsptoprule
 & front & central & back \\
\midrule high & i & ɨ & u \\
 mid & ɛ & & \\
 low & & a & \\
\lspbottomrule
\end{tabular}

\label{table:vowels}
\end{table}%


All vowels can be nasalised,\is{nasalisation} when they are surrounded by nasal consonants.\is{consonant} In addition there is \isi{rhinoglottophilia}, which makes vowels sound as if they were nasalised after /h/ (see \sectref{Nasalisation}). However, nasality is not a contrastive feature for vowels.

Most surprising is the existence of a phonemic high central /ɨ/, which does occur in some \isi{Arawakan languages} \citep[cf.][78]{Aikhenvald1999}, but is not phonemic in the other \isi{Southern Arawakan} languages.\footnote{Compare vowel charts or orthography lists in \citet[33]{Danielsen2007} for Baure, \citet[9]{OlzaZubiri2004} for Ignaciano, \citet[12--13]{Rose2021} for Trinitario, \citet[9]{ButlerEkdahl2012} for \isi{Terena}, \citet[63]{Souza2008} for Kinikinau.}
 However, it has been shown that /ɨ/ results from fronting of the back vowel *u of a presumed proto-language, while /u/ derives from *o (\citealt[]{deCarvalhoPAU}; \citealp[]{RamirezFranca2019}).

\subsubsection{The high vowels} \label{paragraph:HighVowels}

The high vowel /i/ can occur word-initially, -internally, and -finally. When unstressed, it is sometimes pronounced as [ɪ], but there is certain variation among speakers. Especially María C. tends to pronounce the high vowels /i/ and /u/ lower than other speakers do. Some examples with the high vowel /i/ are given in (\getref{ex:i}).

\ea\label{ex:i}
    %\vtop{\labels\halign{\tl #\hfil\tab \tspace[textoffset]#\hfil\tab\tab \tspace[dimb]#\hfil\\
\ea     \tab[ˈi.ju] \tab\tab /iju/ \tab\tab ‘monkey’\\
     \tab{[}i.ˈsi.ni] \tab /isini/ \tab ‘jaguar’\\  %may end in ɪ
     \tab{[}ɪ.ˈhɨ̃ũ.pɛ] \tab /ihɨupɛ/ \tab ‘spindle’\\
\ex     \tab[ki.ˈmɛ.nu] \tab /kimenu/ \tab ‘woods’\\
     \tab{[}ni.ˈʧu.ka] \tab /niʧuka/ \tab ‘my ear’\\
     \tab{[}ˈpi.ma] \tab /pima/ \tab ‘your husband’\\
\ex  \tab[ju.ˈma.hĩ] \tab /jumahi/ \tab ‘hammock’\\
     \tab{[}ˈjɨ.pi] \tab /jɨpi/ \tab\tab ‘jar’\\
     \tab{[}u.ˈʧɛ.ti] \tab /uʧɛti/ \tab ‘chili’\\%}}
\z
\xe

The high vowel /u/ can occur word-initially, -internally, and -finally. When it appears in unstressed syllables, and especially in word-final syllables, it is frequently pronounced lower, i.e. [ʊ] or [o], although it may well be [u], too. The pronunciation depends on speech rate, the place of the word inside an utterance, and also on the speaker. (\getref{ex:u}) shows the occurrence of /u/ in different positions in the word.

\ea\label{ex:u}
    %\vtop{\labels\halign{\tl #\hfil\tab \tspace[textoffset]#\hfil\tab\tab \tspace[dimb]#\hfil\\
\ea     \tab[u.ˈpu.hĩ] \tab\tab /upuhi/ \tab ‘duck’\\ 
     \tab{[}ˈu.ɛ] \tab\tab\tab /uɛ/ \tab\tab ‘rainbow, water spirit’\\
     \tab{[}u.ˈba.ra.mo] \tab /ubaramu/ \tab ‘spider monkey’\\
\ex  \tab[ˈnu.pu.nʊ] \tab\tab /nupunu/ \tab ‘I bring’\\
     \tab{[}ti.ˈku.ti] \tab\tab /tikuti/ \tab ‘it hurts’\\
     \tab{[}ti.kɛ.bu.ˈɾi.ku] \tab /tikɛbuɾiku/ \tab ‘he/she removes grains’\\
\ex  \tab[ˈhã.pu] \tab\tab /hapu/ \tab ‘\textit{mate} calabash’\\
     \tab{[}a.ˈnɨ.mu] \tab\tab /anɨmu/ \tab ‘sky’\\
     \tab{[}o.ˈɾu.pu.nu] \tab /uɾupunu/ \tab ‘red brocket’\\%}}
\z
\xe


The minimal pairs in (\getref{ex:u-i}) show that the high vowels /u/ and /i/ fully contrast.

  \ea\label{ex:u-i}
    %\vtop{\labels\halign{\tl #\hfil\tab \tspace[textoffset]#\hfil\tab\tab \tspace[dimb]#\hfil\\
\ea     \tab[ˈu.ti] \tab\tab\tab /uti/ \tab\tab\tab ‘plant sp.’ (\textit{Urera \\ \tab\tab\tab\tab\tab\tab\tab caracasana})\\     %Span.: pica pica
     \tab{[}ˈi.ti] \tab\tab\tab /iti/ \tab\tab\tab ‘blood’\\
\ex     \tab[ˈnu.ma] \tab\tab /numa/ \tab\tab ‘I take (\textsc{irr})’\\
     \tab{[}ˈni.ma] \tab\tab /nima/ \tab\tab ‘my husband’\\
\ex      \tab[ti.ˌwu.ɾʊ.ˈɾu.ku] \tab /tiβuɾuɾuku/ \tab ‘it boils’\\
     \tab{[}ti.ˌwu.ɾɪ.ˈɾi.ku] \tab /tiβuɾiɾiku/ \tab\tab ‘it falls out (hair)’\\%}}
\z
\xe
%chibu  vs. chubu = he/she lives

The high vowel /ɨ/ is presented below in word-initial, -medial and -final position.

\ea\label{ex:ÿ}
    %\vtop{\labels\halign{\tl #\hfil\tab \tspace[textoffset]#\hfil\tab\tab \tspace[dimb]#\hfil\\
\ea     \tab[ˈɨ.ku] \tab /ɨku/ \tab\tab ‘rain’\\
     \tab{[}ˈɨ.βa] \tab\tab /ɨβa/ \tab\tab ‘pig’\\
     \tab{[}ˈɨ.nɛ] \tab\tab /ɨnɛ/ \tab\tab ‘water’\\
\ex  \tab[nɪ.ˈjɨ.ti.ku] \tab /nijɨtiku/ \tab ‘I set (a pot) on the fire’\\
     \tab{[}ˈsɨ.ki] \tab /sɨki/ \tab\tab ‘basket’\\
     \tab{[}mu.ˈkɨ.ɛ] \tab /mukɨɛ/ \tab ‘squash’\\
\ex  \tab[a.ˈni.βɨ] \tab /aniβɨ/ \tab ‘mosquito’\\
     \tab{[}ni.ˈkɛ.pɨ] \tab /nikɛpɨ/ \tab ‘my back’\\
     \tab{[}ˈsɨ.mɨ] \tab /sɨmɨ/ \tab ‘vulture’\\%}}
\z
 \xe

/ɨ/ is less stable than the other high vowels. It is often hard to distinguish it from /u/ and /ɛ/, because it is often pronounced more like [ə] or [ʊ]. Thus, a speaker may pronounce the word /kusiɨ/ (\textit{kusiÿ} ‘ant’)  like [kusiʊ] in rapid speech, and only when she utters the word very carefully and slowly, it is [kusiɨ]. This may be a hint that the fronting from *u to /ɨ/ has not been totally completed.
The confusion with /ɛ/ is due to the fact that both vowels may be reduced to [ə] in unstressed syllables. /ɨ/ has a tendency towards \isi{nasalisation}, so a nasalised schwa is more likely to be an instance of /ɨ/ than of  /ɛ/. 

Some near minimal pairs with /i/ are presented in (\getref{ex:ÿ-i}).

  \ea\label{ex:ÿ-i}
    %\vtop{\labels\halign{\tl #\hfil\tab \tspace[textoffset]#\hfil\tab\tab \tspace[dimb]#\hfil\\
\ea     \tab[nɨ.ˈhɨ̃.ku] \tab\tab /nɨhɨku/ \tab ‘I grow’\\
     \tab{[}ˈni.hĩ.ku] \tab\tab /nihiku/ \tab ‘I spin’\\
\ex  \tab[ˈɨ.mu] \tab\tab /ɨmu/ \tab\tab ‘piranha’\\
    \tab{[}ˈhĩ.mu] \tab\tab /himu/ \tab ‘fish’\\
\ex     \tab[ti.ˈhã.pɨ.ku] \tab\tab /tihapɨku/ \tab ‘he/she fills’\\
     \tab{[}ti.ˌhã.pi.ˈpi.ku] \tab /tihapipiku/ \tab ‘it lightens’\\%}}
\z
\xe
   %ÿne vs nine 'my flea’ 

(\getref{ex:ÿ-u}) shows the contrast between /ɨ/ and /u/.


  \ea\label{ex:ÿ-u}
    %\vtop{\labels\halign{\tl #\hfil\tab \tspace[textoffset]#\hfil\tab\tab \tspace[dimb]#\hfil\\
\ea     \tab[ˈɨ.ʧɨ] \tab\tab /ɨʧɨ/ \tab\tab ‘capybara’\\
     \tab{[}ˈu.ʧu] \tab /uʧu/ \tab\tab ‘\textsc{uncert.fut}’\\
\ex     \tab[ˈjɨ.nɨ] \tab /jɨnɨ/ \tab\tab ‘jabiru’\\
     \tab{[}ˈju.nu] \tab /junu/ \tab ‘tree sp.’\\
\ex      \tab[ˈjɨ.pi] \tab /jɨpi/ \tab\tab ‘jar’\\
    \tab{[}ˈju.pu] \tab /ju.pu/ \tab ‘paca’\\%}}
\z
\xe
%nijÿku 'I grow’ vs. nujiku ‘I suckle’    
%nÿjÿchu = I make fire vs. juchubu = where

The orthographic representation of the high vowel /i/ is <i>, /ɨ/ is spelled <ÿ>, and /u/ is given as <u> and in loans\is{borrowing} also as <o> throughout this work.
    
\subsubsection{The mid vowel} \label{paragraph:MidVowel}

There is one mid vowel in Paunaka, which is /ɛ/. It can occur in all positions in a word. When unstressed, it may centralise and be realised as [ə]. (\getref{ex:e}) presents some words containing the mid vowel in different positions in the word.

\ea\label{ex:e}
    %\vtop{\labels\halign{\tl #\hfil\tab \tspace[textoffset]#\hfil\tab\tab \tspace[dimb]#\hfil\\
\ea     \tab[ˈɛ.ka] \tab /ɛka/ \tab\tab ‘\textsc{dem}a’\\
     \tab{[}ɛ.ˈmu.ni.ki] \tab /ɛmuniki/ \tab ‘ember’\\
     \tab{[}ɛ.ˈsɛ.kɛɨ] \tab /ɛsɛkɛɨ/ \tab ‘bean’\\
\ex  \tab[a.ˈsa.nɛ.ti] \tab /asanɛti/ \tab ‘field’\\
     \tab{[}ˈnɛ.tu.ku] \tab /nɛtuku/ \tab ‘I put’\\
    \tab{[}ˈtɛ.mɛ.na] \tab /tɛmɛna/ \tab ‘big’\\
\ex  \tab[i.ˈti.ɛ] \tab /itiɛ/ \tab\tab ‘eel’\\
     \tab{[}ˈma.nǝ] \tab /manɛ/ \tab ‘morning’\\
     \tab{[}a.ˈmu.kǝ] \tab /amukɛ/ \tab ‘corn’\\%}}
\z
\xe

Since /ɨ/ has an identical allophone [ə] (see above),  it is often hard to tell whether a sound corresponds to /ɛ/ or /ɨ/. Nevertheless, in stressed syllables there is a clear audible difference between /ɨ/ and /ɛ/, as can be seen from the following examples.

  \ea\label{ex:e-ÿ}
    %\vtop{\labels\halign{\tl #\hfil\tab \tspace[textoffset]#\hfil\tab\tab \tspace[dimb]#\hfil\\
\ea     \tab[ˈɨ.mu] \tab /ɨmu/ \tab\tab ‘piranha’\\
    \tab{[}ˈɛ.mu] \tab /ɛmu/ \tab ‘you (\textsc{pl}) see’\\
\ex     \tab[ˈɨ.ku] \tab /ɨku/ \tab\tab ‘rain’\\
      \tab[ˈɛ.ka] \tab /ɛka/ \tab\tab ‘\textsc{dem}a’\\
\ex      \tab[ˈɨ.ʧɨ] \tab\tab /ɨʧɨ/ \tab\tab ‘capybara’\\
     \tab{[}ˈɛ.ʧɨu] \tab /ɛʧɨu/ \tab\tab ‘\textsc{dem}b’\\%}}
\z
 \xe
 %chÿchÿ vs chÿeche
 
The mid vowel also contrasts with the other high vowels. (\getref{ex:e-i}) presents some examples for the contrast with /i/.

  \ea\label{ex:e-i}
    %\vtop{\labels\halign{\tl #\hfil\tab \tspace[textoffset]#\hfil\tab\tab \tspace[dimb]#\hfil\\
\ea      \tab[ˈɛ.ti] \tab\tab\tab /ɛti/ \tab\tab ‘you (\textsc{pl})’\\
     \tab{[}ˈi.ti] \tab\tab\tab /iti/ \tab\tab ‘blood’\\
\ex    \tab[ˈvi.tə] \tab\tab /βitɛ/ \tab\tab ‘bat’\\
    \tab{[}ˈvi.ti] \tab\tab /βiti/ \tab\tab ‘we’\\
\ex    \tab[nə.ˈhɛ̃.ku.pu] \tab /nɨhɛkupu/ \tab ‘I forget’\\
     \tab{[}ni.ˈhĩ.ku.pu] \tab /nihikupu/ \tab ‘I swallow’\\%}}
\z
\xe


(\getref{ex:e-u}) shows that /ɛ/ and /u/ are fully contrastive.  
    
\ea\label{ex:e-u}
	%\vtop{\labels\halign{\tl #\hfil\tab \tspace[textoffset]#\hfil\tab\tab \tspace[dimb]#\hfil\\
\ea 	\tab{[}ˈɛ.ʧɨu] \tab /ɛʧɨu/ \tab\tab ‘\textsc{dem}b’\\
 	\tab{[}ˈu.ʧʊ] \tab /uʧu/ \tab\tab ‘\textsc{uncert.fut}’\\
\ex 	\tab[ˈmɛ.tu] \tab /mɛtu/ \tab ‘already’\\
 	\tab{[}ˈmũ.tu] \tab /mutu/ \tab ‘armadillo’\\
\ex   \tab[ni.ˈjɛ.nu] \tab /nijɛnu/ \tab ‘my wife’\\
 	\tab{[}ni.ˈju.nu] \tab /nijunu/ \tab ‘I go’\\%}}
\z
\xe
	
The mid vowel is written <e> throughout this work.

\subsubsection{The low vowel} \label{paragraph:LowVowel}

There is one low vowel, the central /a/, which can occur at the beginning, in the middle or at the end of words. This is shown  in (\getfullref{ex:a.1}) to (\getfullref{ex:a.3}).

\ea\label{ex:a}
    %\vtop{\labels\halign{\tl #\hfil\tab \tspace[textoffset]#\hfil\tab\tab \tspace[dimb]#\hfil\\
\ea\label{ex:a.1}     \tab[a.ˈni.βɨ] \tab /aniβɨ/ \tab ‘mosquito’\\
     \tab{[}ˈa.mɛ] \tab /amɛ/ \tab\tab ‘palm sp.’ (Attalea princeps)\\
     \tab{[}a.ˈsa.nǝ.ti] \tab /asanɛti/ \tab ‘field’\\
\ex\label{ex:a.2}    \tab[ˈnɨa.ti] \tab /nɨati/ \tab ‘my brother (of female)’\\
     \tab{[}ta.ˈnɨ.ma] \tab /tanɨma/ \tab ‘now’\\
    \tab{[}ˈka.ku] \tab /kaku/ \tab ‘exist’\\
\ex\label{ex:a.3}      \tab[nɨ.ˈna.βa] \tab /nɨnaβa/ \tab ‘my mouth (inside)’\\
     \tab{[}ˈmi.ʧa] \tab /miʧa/ \tab ‘good’\\
     \tab{[}ni.ˈsi.ka] \tab /nisika/ \tab ‘my arm’\\%}}
\z
\xe

/a/ contrasts with all other vowels. For some minimal pairs with the high vowel /i/,  see (\getref{ex:a-i}) below. 

\ea\label{ex:a-i}
	%\vtop{\labels\halign{\tl #\hfil\tab \tspace[textoffset]#\hfil\tab\tab \tspace[dimb]#\hfil\\
\ea 	\tab[ˈmi.ʧa] \tab\tab /miʧa/ \tab ‘good’\\
 	\tab{[}ˈmi.ʧi] \tab\tab /miʧi/ \tab ‘cat’\\
\ex 	\tab[ni.ˈhã.ku.pu] \tab /nihakupu/ \tab ‘I receive’\\
 	\tab{[}ni.ˈhĩ.ku.pu] \tab /nihikupu/ \tab ‘I swallow’\\
\ex 	\tab[ni.ˈpa.ko] \tab\tab /nipaku/ \tab ‘I die’\\
 	\tab{[}ni.ˈpi.ko] \tab\tab /nipiku/ \tab ‘I am afraid’\\%}}
\z
\xe

In (\getref{ex:a-ÿ}), some examples for the contrast between /a/ and /ɨ/ are given.

\ea\label{ex:a-ÿ}
	%\vtop{\labels\halign{\tl #\hfil\tab \tspace[textoffset]#\hfil\tab\tab \tspace[dimb]#\hfil\\
\ea 	\tab[ˈki.pa] \tab /kipa/ \tab ‘rhea’\\
 	\tab{[}ˈki.pɨ] \tab /kipɨ/ \tab\tab ‘turtle’\\
\ex 	\tab{[}ni.ˈhã.ka] \tab /nihaka/ \tab ‘my molar tooth’\\
 	\tab{[}nə.ˈhɨ̃.ka] \tab /nɨhɨka/ \tab ‘I grow (\textsc{irr})’\\
\ex 	\tab[ˈka.ku] \tab /kaku/ \tab ‘exist’\\
 	\tab{[}nɨ.ˈkɨ.ku] \tab /nɨkɨku/ \tab ‘my uncle’\\%}}
\z
\xe	

(\getref{ex:a-u}) shows that /a/ contrasts with /u/.
	
\ea\label{ex:a-u}
       %\vtop{\labels\halign{\tl #\hfil\tab \tspace[textoffset]#\hfil\tab\tab \tspace[dimb]#\hfil\\
\ea 	\tab[ni.ˈpa.ku] \tab /nipaku/ \tab ‘I die’\\
 	\tab{[}ni.ˈpu.ku] \tab /nipuku/ \tab ‘my forehead’\\
\ex 	\tab[ni.ˈni.ka] \tab /ninika/ \tab ‘I eat (\textsc{irr})’\\
 	\tab{[}ni.ˈni.ko] \tab /niniku/ \tab ‘I eat (\textsc{real})’\\
\ex 	\tab[ˈni.ma] \tab /nima/ \tab ‘my husband’\\
 	\tab{[}ˈni.mʊ] \tab /nimu/ \tab ‘I see’\\%}}
\z
\xe

There is also a contrast with the mid vowel /ɛ/, as shown in (\getref{ex:a-e}). However, there are a few words that have variants with either /a/ or /ɛ/ without any difference in meaning according to the speakers, e.g. \textit{apuke} [a.ˈpu.kɛ] ‘ground, down’ has an alternative form as [ɛ.ˈpu.kɛ], the verb stem \textit{-semaiku} [-sɛ.ˈmai.ku] can also be realised as [-sa.ˈmai.ku]. 	

	
\ea\label{ex:a-e}
       %\vtop{\labels\halign{\tl #\hfil\tab \tspace[textoffset]#\hfil\tab\tab \tspace[dimb]#\hfil\\
\ea 	\tab[ˈna.na] \tab /nana/ \tab ‘I make (\textsc{irr})’\\
 	\tab{[}ˈnɛ.na] \tab /nɛna/ \tab ‘like, similar’\\
\ex 	\tab{[}ˈa.ʧu] \tab /aʧu/ \tab\tab ‘fish sp.’\\
 	\tab{[}ˈɛ.ʧɨu] \tab /ɛʧɨu/ \tab\tab ‘\textsc{dem}b’\\
\ex 	\tab[ˈka.vɛ] \tab /kaβɛ/ \tab ‘dog’\\
 	\tab{[}ni.ˈkɛ.vɛ] \tab /nikɛβɛ/ \tab ‘my tooth’\\%}}
\z
\xe	
    
The low vowel is represented as <a> throughout this work.
    
\subsection{Vowel sequences} \label{Diphthongs}
\is{vowel sequence|(}
There are a lot of vowel sequences in Paunaka, and it is not always clear whether such a sequence is better analysed as a diphthong or hiatus. If we compare the other Bolivian Arawakan\is{Southern Arawakan} languages and also take the reconstructed proto-lan\-guage into account (\citealt[cf.][]{deCarvalhoPAU}; \citealt[]{RamirezFranca2019}), it becomes clear that many vowel sequences result from the deletion of consonants in unstressed syllables. \isi{Mojeño Trinitario}, for instance, often has either /ɾ/ or /ʔ/ and sometimes other consonants between two vowels, where Paunaka has a sequence of two vowels (Rose 2014, p.c.); see \tabref{table:PauTriniComparison-VV} for a few examples. The fact that some of these sequences have fused into a diphthong and others rather belong to two syllables may then reflect different time depths or degrees of conventionalisation of consonant deletion. I consider a sequence of two vowels a diphthong, if it sounds short and the whole sequence is either stressed (most of the times) or unstressed. Two vowels in hiatus sound longer, and stress is assigned to only one vowel, but not to the other (if stress falls on one of the syllables at all). However, at a higher speech rate, a hiatus may also sound like a diphthong. It would therefore be worth doing a phonetic analysis of vowel sequences in the future, and the analysis given here can only be considered as tentative and preliminary.

\begin{table}
\caption{Cognate stems: Paunaka has a vowel sequence where Trinitario has /ɾ/ or /ʔ/}

\begin{tabular}{lll}
\lsptoprule
Paunaka & Trinitario & Gloss \\
\midrule
\textit{mai} & \textit{mari} & stone\\
\textit{sabae} & \textit{saware} & tobacco\\
\textit{ue} & \textit{‘o‘e} & rainbow\\
\textit{pei} & \textit{pe‘i} & agouti\\
\lspbottomrule
\end{tabular}

\label{table:PauTriniComparison-VV}
\end{table}

%FRANÇOISE: krigre in MT from kVrikere (kÿike)	
% FRANÇOISE: MT viya (bia)


The clearest example of a hiatus is the word \textit{ue}, a word that is exceptionally composed of only two V syllables.

\ea\label{ex:w-2}%\vtop{\halign{%
%#\hfil&& \qquad #\hfil \\
[ˈu.ɛ]\tab /uɛ/\tab ‘rainbow, water spirit’\\%}}
\xe

There are more words with a final /ɛ/ preceded by another vowel, and the two vowels usually belong to two different syllables. The nouns\is{noun} ending in an /ɛ/ syllable mostly refer to animals or plants (or their fruits), so the final /ɛ/ may go back to an old nominal suffix or classifier, which is not transparent anymore.\footnote{The \isi{Mojeño languages} both have a classifier \textit{-‘e}/\textit{-he} and \isi{Baure} has \textit{-e}, but the list of items that fall into this class show no correspondences to the Paunaka nouns that have a final /Vɛ/ sequence, except for the fact that one kind of squash is classified by \textit{-e} in \isi{Baure} (cf. \citealt[232--235]{OlzaZubiri2004}; \citealt[464]{Rose2019b}; \citealt[141--142]{Terhart2016}).} There are even two cases in which the final /ɛ/ is preceded by a syllable with a nucleus /ɛ/, and in this case the last syllable has a long vowel. The \isi{Mojeño languages} often have \textit{-re} in plant names, where Paunaka has \textit{-e} (Sell 2021, p.c.).

\ea %\vtop{\halign{%
%#\hfil\tab\tab \qquad #\hfil \\
	{[}i.ˈti.ɛ] \tab /itiɛ/ \tab\tab ‘eel’\\
	{[}mu.ˈkɨ.ɛ] \tab /mukɨɛ/ \tab ‘squash’\\
	{[}kɛ.ˈʧu.ɛ] \tab /kɛʧuɛ/ \tab ‘snake’\\
	{[}sa.ˈβa.ɛ] \tab /saβaɛ/ \tab ‘tobacco’\\
	{[}i.ˈnɛ:] \tab /inɛɛ/ \tab\tab ‘fish sp.’ (\textit{Hoplias malabaricus}) \\
	{[}mu.ˈkɛ:] \tab /mukɛɛ/ \tab ‘rodent sp.’ (Span. \textit{cujuchi}) \\%}}
\xe
%kujue, tÿmue, epenue, a/ebijie, buchee, ketÿe ‘nigua’, kipitae ‘tábano’

An exception is the word \textit{chapie} [ʧa.ˈpi̯e] ‘thank you’, a loan\is{borrowing} from \isi{Bésiro}, which is disyllabic with a diphthong /iɛ̯/\footnote{Note that diphthongs are marked by an inverted breve in this section only.} in the second syllable.	
	
Another vowel sequences that is often found in hiatus is /ia/.  

\ea %\vtop{\halign{%
%#\hfil\tab\tab \qquad #\hfil \\
	{[}ˈvi.a] \tab\tab\tab /βia/ \tab\tab\tab ‘God’ (lit.: our father) \\
	{[}ˈkwɛ.pi.a] \tab\tab /kuɛpia/ \tab\tab ‘kidney’\\
	{[}ni.ˌku.vi.ˈa.ku.βu] \tab /nikuβiakuβu/ \tab ‘I am tired’\\%}}
\xe


The exception is again a \isi{Bésiro} loan;\is{borrowing} the word \textit{mukianka} [mu.ˈkʲaŋ.ka] ‘animal’ which has a palatalised [k].

Most likely to be fused into diphthongs are the sequences /ai/, /ɛi/, /ui/, /ɨi/, /au/, /ɛu/, and some instances of /iu/, /ua/, and /uɛ/.

Here are some words with the diphthong /ai̯/.

\ea %\vtop{\halign{%
%#\hfil\tab\tab \qquad #\hfil \\
	{[}ˈmai̯] \tab /mai/ \tab\tab ‘stone’\\
	{[}ti.ˈhãĩ̯] \tab /tihai/ \tab ‘day’\\
	{[}ˈtɨ.nai̯] \tab /tɨnai/ \tab ‘long’\\%}}
	\xe
		
The following words all have a diphthong /ɛi̯/.
	
\ea %\vtop{\halign{%
%#\hfil\tab\tab \qquad #\hfil \\
	{[}ku.ˈpɛi̯] \tab /kupɛi/ \tab ‘afternoon’\\
	{[}ʧu.ˈmɛi̯.ku] \tab /ʧumɛiku/ \tab ‘he/she/it steals it’\\
	{[}ˈtɛi̯.hũ.ku] \tab /tɛihuku/ \tab ‘he/she/it stinks’\\%}}
	\xe
%\tab{[}ma.ˈnɛi̯.ku] \tab /manɛiku/ \tab ‘soon’\\

The diphthong /ui̯/ is articulated with a semi-vowel after /k/. Here are some examples of the diphthong.

\ea %\vtop{\halign{%
%#\hfil\tab\tab \qquad #\hfil \\
        {[}ˈkwi.na] \tab\tab /kuina/ \tab\tab ‘\textsc{neg}’\\
	{[}vɛ.mu.ˈsui̯.ka] \tab /βɛmusuika/ \tab ‘we wash (\textsc{irr})’\\
	{[}ˈʧu.bui̯] \tab\tab /ʧuβui/ \tab\tab ‘old man’\\%}}
	\xe	
	
	
Some words containing /ɨi̯/ are given below.	

\ea %\vtop{\halign{%
%#\hfil\tab\tab \qquad #\hfil \\
        {[}ni.ˈpɨi̯] \tab /nipɨi/ \tab ‘my body’\\
	{[}ˈkɨi̯.kɛ] \tab /kɨikɛ/ \tab ‘peanut’\\
        {[}nɨ.ˈkɨi̯.ki] \tab /nɨkɨiki/ \tab ‘pot’\\%}}
	\xe
	%titÿiku or tituiku? ‘chase’; pisÿikuke 'your knee’
	
	
Here are some examples for the diphthong /au̯/:\footnote{\textit{Chicha} is a beverage made of corn in most cases, but also of manioc, peanuts etc. It is widespread in South America. If it ferments, it is called ‘strong chicha’ (Pau. \textit{isipau}, Span. \textit{chicha fuerte}).}
	
\ea %\vtop{\halign{%
%#\hfil\tab\tab \qquad #\hfil \\
        {[}ˈau̯.moɛ̯] \tab /aumuɛ/ \tab ‘chicha’\\
	{[}i.ˈsi.pau̯] \tab /isipau/ \tab ‘strong (fermented) chicha’\\
	{[}ˈna.nau̯] \tab /nanau/ \tab ‘I make’\\%}}
	\xe

The words below all have a diphthong /ɛu̯/.

\ea %\vtop{\halign{%
%#\hfil\tab\tab \qquad #\hfil \\
        {[}ˈsɛu̯.nu.vɛ] \tab /sɛunuβɛ/ \tab ‘woman’\\
	{[}ni.ˈpɛu̯] \tab /nipɛu/ \tab ‘my animal’\\
	{[}mu.ˈvɛo̯] \tab /muβɛu/ \tab ‘bird sp.’ (\textit{Columbina picui}) \\%}}
	\xe
	
%FRANÇOISE: < seno-nove ?; \\ pero in MI
	
There are only a few words containing the diphthong /aɨ/, two of which are given below.

\ea %\vtop{\halign{%
%#\hfil\tab\tab \qquad #\hfil \\
        {[}ku.pi.ˈsaɨ̯.ɾɨ] \tab /kupisaɨɾɨ/ \tab ‘fox’\\
        {[}ʧu.ˈmu.taɨ̯] \tab\tab /ʧumutaɨ/ \tab ‘stool’\\%}}
	\xe
	

The sequence /iu/ often occurs in hiatus, especially in subordinate verbs,\is{deranked verb} where \textit{-i} is the subordinating suffix, but in rapid speech the suffix may also only palatalise\is{palatalisation} the preceding \isi{consonant}, which is mostly /k/. Nevertheless, there are some instances where the sequence /iu̯/ is a diphthong, most often after /s/.

\ea %\vtop{\halign{%
%#\hfil\tab\tab \qquad #\hfil \\
        {[}ˈti.siu̯] \tab\tab /tisiu/ \tab ‘puma’\\
	{[}mu.si.ˈsiu̯.pa] \tab /musisiupa/ \tab ‘sand’\\
	{[}ti.vɛ.ˈɾiu̯.ku] \tab /tiβɛɾiuku/ \tab ‘he/she/it returns’\\%}}
	\xe	
%isiupe 'seed’
	
There are not many words with the diphthong /ua̯/ in the corpus, but some can be found. As with /ui̯/, it is articulated with a semi-vowel after /k/. Some examples are given below.

\ea %\vtop{\halign{%
%#\hfil\tab\tab \qquad #\hfil \\
        {[}ˈkwa.hĩ] \tab /kuahi/ \tab ‘fishing net’\\
	{[}ˈnɛ.moa̯] \tab /nɛmua/ \tab ‘my belly’\\
	{[}ˈpua̯.kɛ.nɛ] \tab /puakɛnɛ/ \tab ‘other side’\\ %}}
	\xe
	
As with /ui̯/ and /ua̯/, the diphthong /uɛ̯/ is articulated with a semi-vowel after /k/. The semi-vowel also occurs in the sequence /suɛ̯/ (which is thus pronounced [swɛ]). Some words with the diphthong are given below.

\ea %\vtop{\halign{%
%#\hfil\tab\tab \qquad #\hfil \\
        {[}ˈkwɛ.pi] \tab /kuɛpi/ \tab ‘sweet potato’\\
	{[}ɛ.ˈpɛ.nuɛ̯] \tab /ɛpɛnuɛ/ \tab ‘hole’\\
	{[}ku.ˈswɛ.nu] \tab /kusuɛnu/ \tab ‘rabbit’\\%}}
	\xe	

%what about eÿ in esekeÿ???

In addition, there is also one triphthong, which only shows up in one word, to my knowledge:

\ea %\vtop{\halign{%
%#\hfil\tab\tab \qquad #\hfil \\
        {[}ti.ˈsɨɛi] \tab /tsɨɛ/ \tab ‘it is cold’\\%}}
	\xe	

\is{vowel sequence|)}
\is{vowel|)}
\is{segmental phonology|)}

\subsection{Phonological adaption of loanwords} \label{SoundsLoans}\is{borrowing|(}
%Guarayu: mberɨ -> merÿ; patavɨi -> patabi 'sugar cane’, yusei -> yÿseiku, 

This section provides information about how the sounds of words with a foreign origin – as far as recognisable – are adapted.\is{consonants in loans|(}
Spanish and Bésiro are the two major sources for loans.\footnote{I am grateful to Pierric Sans for sharing his field dictionary of Bésiro with me \citep[]{Sans2011}.} In addition, a few words are strikingly similar to the ones found in Guarayu, a Tupi-Guarani language, today spoken in an area north of the Chiquitania. Not every loan is recognised as such by the speakers of Paunaka.\footnote{Quite telling in this regard was elicitation of plant names together with my colleague Lena Sell in 2018: the speakers gave a name for a plant, which was clearly of Bésiro\is{Bésiro|(} origin, since it contained the sound /ʂ/. When I asked what the name was in Bésiro, however, the speakers replied that they did not know.\is{Bésiro|)}} As for words with a Spanish origin, there are some regular correspondences between Spanish and Paunaka sounds. They comprise /x/ and /g/→ /k/, /d/ and /l/ → /ɾ/, and /f/ → /β/ or /p/. \tabref{table:SpanishSounds} presents some examples of these correspondences, but there are also be words that retain the original sounds. This may partly have to do with how well words containing the sounds are integrated into the language and can be considered part of the Paunaka lexicon. Well-integrated loans tend to have a fixed phonological form. If speakers spontaneously resort to Spanish and Bésiro words in conversation, the degree of adaption to Paunaka phonology may be lesser or greater from occasion to occasion (i.e. it is not the case that some speakers are more likely to adapt words phonologically than others). This also applies some words they use regularly, i.e. where a certain integration into the lexicon can be assumed. To give but one example, the word \textit{kanela} ‘cinnamon’ from Spanish \textit{canela} always retains /l/ in Juana’s speech although she uses this word relatively frequently in the corpus.

\begin{table}
\caption{Replacement of Spanish sounds}
\small
\begin{tabularx}{\textwidth}{lllllQ}
\lsptoprule
Pattern  & \multicolumn{2}{l}{Spanish} & \multicolumn{2}{l}{Paunaka} & Translation\\
\midrule 
/x/ → /k/ & \textit{trabajo} & [tɾa.ˈβa.xo] & \textit{trabaku} & [tɾa.ˈβa.ku] & work\\
& \textit{Juana} & [ˈxua.na] & \textit{Kuana} & [ˈkwa.na] & Juana (proper name)\\
 /g/ → /k/ & \textit{guineo} & [gi.ˈnɛo] & \textit{kineu} & [ki.ˈnɛu] & banana sp.\\
& \textit{domingo} & [do.ˈmiŋ.go] & \textit{ruminko} & [ɾu.ˈmiŋ.ko] & Sunday\\
  /d/ → /ɾ/  & \textit{dos} & [ˈdoh] & \textit{ruschÿ} & [ˈɾus.ʧɨ] & two\\
& \textit{después} & [dɛhˈpuɛh] & \textit{repue} & [ɾɛ.ˈpuɛ] & after\\
 /l/ →  /ɾ/  &  \textit{mula} & [ˈmu.la] & \textit{mura} & [ˈmu.ɾa] & mule (Pau. also: horse)\\
& \textit{pelota} &  [pɛ.ˈlo.ta] & \textit{peruta} & [pɛ.ˈɾu.ta] & ball\\
 /f/ →  /β/ & \textit{foto} & [ˈfo.to] & \textit{boto} & [ˈβo.to] & photo\\
& \textit{Federico} & [fɛ.dɛˈɾi.ko] & \textit{Federico} & [βɛ.ɾɛ.ˈɾi.ko] & Federico (proper name)\\
  /f/ →  /p/ & \textit{fiesta} & [ˈfiɛs.ta] & \textit{piesta} & [ˈpiɛs.ta] & feast, party\\
& \textit{faltado} & [fal.ˈtao] & \textit{paltau} & [pal.ˈtau] & be missing (Span.: missed, lacked)\\
 \lspbottomrule
\end{tabularx}

\label{table:SpanishSounds}
\end{table}

\is{vowels in loans|(}
The vowel /u/ of Paunaka has an allophone [o] (see \sectref{paragraph:HighVowels}), so that borrowed words containing an /o/ (precisely [o̞] in Spanish) may be pronounced either with an [u] or and [o], e.g. the Spanish word \textit{año} ‘year’ may be [ˈa.ɲu] or [ˈa.ɲo] in Paunaka.
\is{vowels in loans|)}

Since I do not speak Bésiro, it is much harder for me to detect possible loans as long as they have the same syllabic and phonemic structure as native Paunaka words. Clearly noticeable are all words containing the retroflex fricative [ʂ] or the postalveolar fricative [ʃ], which are retained in Paunaka loans.\footnote{But note that [ʂ] is often voiced and pronounced as [ʐ].} The sounds are allophones in Bésiro \is{Bésiro|(} according to \citet[70]{Sans2010}. Examples have been given in \sectref{par:Fricatives}. The phonemic inventory of Bésiro further differs from the Paunaka because it contains a palatal plosive /c/ and a phonemic /o/. I did not encounter instances of [c] in Paunaka, and as for Bésiro /o/, it may be retained in Paunaka or be pronounced as [u], similar to the loans from Spanish (e.g. we find both [ˈto.sɛ] and [ˈtu.sɛ] ‘noon’, originally from Span. \textit{doce} ‘twelve’ but probably borrowed via Bésiro).

Some words with a Spanish origin must have entered the Paunaka lexicon via Bésiro. Such words have a postalveolar [ʃ] in Paunaka and in Bésiro, but not in Spanish.\is{Bésiro|)} Three examples are provided in \tabref{table:BesiroSpanishLoans}. Note that while the sound is retained in Paunaka, \isi{syllable} structure (see \sectref{SyllableStructure}) or \isi{stress} assignment (see \sectref{sec:Stress}) may change to match the Paunaka system.\footnote{The \textit{n-} in \isi{Bésiro} \textit{nixhkuéra} is a nominal prefix to prevent vowel-initial nouns \citep[cf.][20]{Sans2013}. It was presumably not interpreted as part of the lexeme by the Paunaka borrowers and thus deleted. The case of \textit{rimonexhi} ‘lemon’ is also peculiar, since the fricative is retroflex in Bésiro, but postalveolar in Paunaka.}


\begin{table}
\caption{[ʃ] in loans from \isi{Bésiro} with a Spanish origin}

\begin{tabularx}{\textwidth}{QQQl}
\lsptoprule
Spanish & Bésiro & Paunaka & Translation\\
\midrule 
\textit{escuela}  

[ɛs.ˈkwɛ.la] & \textit{nixhkuéra} 

[niʃ.ˈkwɛ.ɾa] & \textit{xhikuera} 

[ʃi.ˈkwɛ.ɾa] & school\\
\tablevspace
 \textit{jabón}  
 
 [xa.ˈβon] & \textit{xhabú} 
 
 [ʃa.ˈβu]  &  \textit{xhabu} 
 
 [ˈʃa.βu] & soap\\
\tablevspace
 \textit{limón} 
 
 [li.ˈmon] &  \textit{rimonexi} 
 
 [ɾi.ˈmo.nɛ.ʂɨ] &  \textit{rimonexhi} 
 
 [ɾi.ˈmo.nɛ.ʃɨ] & lemon\\
\lspbottomrule
\end{tabularx}

\label{table:BesiroSpanishLoans}
\end{table}


Paunaka speakers also frequently pronounce a word-initial /ɾ/ of a Spanish loan as [ʂ] or [ʐ], but this is also done in the local variety of Spanish and thus no adaption takes place (see also \sectref{par:Fricatives} above). Peculiar, however, is that a syllable-final /s/ of words with Spanish origin can be pronounced as [ʂ] or [ʐ] in Paunaka. It remains unclear whether Paunaka speakers replace /s/ in this case, because the word is borrowed from Bésiro, not from Spanish, or to make it sound less Spanish-like, in general. At least for the loan \textit{max} [maʐ] ‘more’ from Spanish \textit{más} [mas], there is a similar loan in \isi{Bésiro}, too, which is pronounced [mãŋ͡ʂ] \citep[98]{Sans2010}.
\is{consonants in loans|)}

Finally, there are some words of Guarayu origin that are worth mentioning. I already mentioned in \sectref{sec:StructureWork} (Footnote \ref{fn:yusei}) the Paunaka verb \textit{-yÿseiku} ([yɨ.ˈsɛi.ku]) ‘buy’ could well derive from Guarayu \textit{-yusei} ‘want or wish sth. edible’. If this is correct, the word must have entered the language before the shift from /u/ to /ɨ/ took place in Paunaka (see \sectref{Vowels}). The other two loans I could detect are \textit{merÿ} ([ˈmɛ.ɾɨ]) from \textit{mberɨ} ‘plantain’ and \textit{patabi} ([pa.ˈta.vi]) from \textit{patavɨi} ‘sugar cane’. The first of these, i.e. \textit{merÿ}, actually already includes the high central vowel in the donor language. The prenasalised stop of the original language is replaced by the nasal in Paunaka. In the second loan \textit{patabi}, the final diphthong is resolved. Both words are otherwise unsuspicious.

\hspace*{-1.9pt}We know that the Paunaka language has also been in close contact with Napeka (of the Chapacuran family); however, since there are almost no data obtainable about this language, possible loans (and other kinds of influences) cannot be detected for the time being.\is{borrowing|)}

\section{Rhinoglottophilia and nasalisation} \label{Nasalisation}
\is{rhinoglottophilia|(}
Paunaka has rhinoglottophilia, defined originally by \citet{Matisoff1975} as an interaction between glottality and nasality which causes nasalisation on vowels following [h] or [ʔ] in various languages over the world.\is{consonant|(}
Phoneticians later found out that it is not precisely nasalisation that is caused by “high airflow segments like voiceless fricatives and aspirated stops”. Instead, the “coupling between the oral and the subglottal cavities” that appears on vowels in rhinoglottophilia has the effect that vowels are perceived as nasal, although they are not nasal \citep[240]{OhalaOhala1993}.

Rhinoglottophilia can be found in a number of \isi{Arawakan languages}, like Paresi \citep[cf.][85]{daSilva2009}, Nanti \citep[cf.][231]{Michael2008}, Iñapari \citep[cf.][9]{Parker1999}, Warekena \citep[cf.][401]{Aikhenvald1998}, Kurripako \citep[cf.][75]{Granadillo2006} and Yucuna \citep[cf.][8]{SchauerSchauer1978}, among others, and also in some other Amazonian languages\is{Amazonian language} that are not genetically related \citep[cf.][116]{Aikhenvald2012}. It is not found in the other Bolivian Arawakan languages as far as I can tell. % not in Baure, Trinitario, Kinikinau, Wapixana
In Paunaka, rhinoglottophilia affects all vowels\is{vowel} and diphthongs following /h/. Those vowels are marked with a tilde throughout this chapter. Some examples are presented in (\getref{ex:Rhinoglottophilia}) below. More examples can be found in \sectref{par:Fricatives}.


\ea\label{ex:Rhinoglottophilia}
%\vtop{\halign{%
%#\hfil\tab\tab \qquad #\hfil \\
       [ˈhã.nɛ] \tab /hanɛ/ \tab ‘wasp’\\
       {[}ˈhĩ.mu] \tab /himu/ \tab ‘fish’\\
	{[}ɛ.ˈhũi] \tab /ɛhui/ \tab ‘cock’\\%}}
	\xe
\is{rhinoglottophilia|)}
	
\is{nasalisation|(}
Vowels\is{vowel} can be nasalised if they are surrounded by nasal consonants.\is{consonant|)} However, this seems to be optional. I have not encountered any case in which nasalisation of a vowel distinctive. Some examples of nasalised vowels are given in (\getref{ex:NasalVs}).

\ea\label{ex:NasalVs}
%\vtop{\halign{%
%#\hfil&& \qquad #\hfil \\
       [ˈnũĩ.nɛ.kɨ]\tab /nuinɛkɨ/\tab ‘door’\\
       {[}ˈmɨ̃ũ.hĩ]\tab /mɨuhi/\tab ‘clothes’\\
       {[}-ˈmɨ̃.nə]\tab /mɨnɨ/\tab\tab ‘\textsc{dim}’\\%}}
	\xe
\is{nasalisation|)}	

\section{Orthography} \label{sec:Orthography}
\is{consonant|(}
\is{orthography|(}

Paunaka is not a written language. Indeed, most of the remaining speakers are illiterate. When the Paunaka Documentation Project started in 2011, there was no orthography that people had agreed upon. Nonetheless, there are some entries of Paunaka vocabulary by \citet[319]{Cardus1886} and some more by d’Orbigny %\iai{Alcide Dessalines d’Orbigny} 
scattered through his works, there are several unpublished word lists composed by Riester in the 1950s and 1960s, and there is even an unpublished grammar sketch compiled by Villafañe (in three different versions). None of the solutions how to write Paunaka was very consistent or practical. Therefore, during my first field trip in 2011, the members of the PDP conducted a workshop with the speakers and their families, in which an orthography was established for the production of an alphabet booklet \citep[cf.][]{PDP2012}.\footnote{Unfortunately, we had not noticed prior to the workshop that most participants could not read or write. Due to the fact that many speakers were illiterate, there was hardly any discussion on possible orthographic alternatives, and when people were asked to vote for one or the other alternative, most of them simply repeated what the person before had said.}  

The result is an orthography which is based on the Spanish one, but differs from it in some important details, like having <ny> instead of <ñ> and <k> instead of <c, qu>. It is therefore easy to read, but still shows the difference with Spanish.  The choice of the letter <ÿ> over <ɨ>  manifests Paunaka’s difference from \isi{Bésiro}. At the same time it is easy to type on Spanish keyboards, which have a key for diaeresis (¨). In comparison, the <ɨ> is prone to be typed as <i>, simply because people do not know how to get the barred i into their texts (Interview with Guarayu speakers, 05.05.2012).\footnote{Time has passed since then, and Guarayu speakers currently use special keyboards created for computers and cell phones or the sign + (Danielsen 2020, p.c.).} Decisions on orthography were made back in 2011, and ten years later, Paunaka people hope to get their language officially recognised (see \sectref{sec:RecentHistory}). It may thus well be the case that an official orthography will be created in the future that differs from the one used throughout this work. Time will tell.

The graphemes and digraphs used in the Paunaka alphabet throughout this work are the following: a, b, ch, e, i, j, k, m, n, ny, (o), p, r, s, t, u, x, xh, y, ÿ, (’). \tabref{table:Orthography} shows how the phonemes of Paunaka are represented orthographically in this grammar, comparing it to previous attempts to transcribe the sounds of the language. As for punctuation, the Spanish system is used, including reversed question (¿) and exclamation (¡) marks to indicate the beginning of a question or exclamation.\is{interrogative clause}

\begin{table}
\caption{Orthography}

\begin{tabularx}{\textwidth}{llQQQQ}
\lsptoprule  
 Sound & PDP & \citealt{Orbigny1839} & \citealt{Cardus1886} & \mbox{\citealt{Riester1955}}\newline\citeyear*{Riester1965, Riesters.d.} & \citealt{Villafanen.d.} \\
\midrule
/i/ & i & i & i & i & i \\
/ɨ/ & ÿ & u? & i, e & ǐ, u, e, i, ö & ë, ï, ü\\
/u/ & u, (o) & u, o & o, u & u, o & u, o\\
/ɛ/ & e & e & e & e, ä & e\\
/a/ & a & a & a & a & a\\
/p/ & p & p & p & p & p \\
/t/ & t & t & t & t & t\\
/k/ & k & c, qu & c, qu & k & k\\
/ʔ/ & (') & & h & h & (')\\
/m/ & m & m & m & m & m \\
/n/ & n & n & n & n & n \\
/ɲ/ & ny & & & ñ & ñ\\
/ɾ/ & r & & & r & r \\
/β/ & b & & b, v & b, v, hu & b \\
/s/ & s & c? & s & s, z & s \\
/h/ & j & y, j? & j, h & h, j & j \\
/ʧ/ & ch & & ch & sh, š, ć, č & ch \\
/j/ & y & & y & j, y & y, i\\
/ʃ/ & xh & & & & sh \\
/ʂ/ & x & & & & \\
\lspbottomrule
\end{tabularx}

\label{table:Orthography}
\end{table}
\is{orthography|)}
\is{consonant|)}
	
\section{Morphophonological processes} \label{sec:PhonProcesses}

There are some morphophonological processes, i.e. phonological processes that occur at morpheme boundaries. First of all, Paunaka has vowel \isi{elision} in person markers that attach to vowel-initial stems (see \sectref{section:Vowel_elision}), a process typical for \isi{Arawakan languages}. In addition, it also often deletes the initial vowel of the \isi{non-verbal irrealis marker}, when it follows a stem that ends in a diphthong.\is{vowel sequence} If the last vowel of the stem is /a/ and the irrealis marker \textit{-ina} is attached, both vowels can fuse into a diphthong /ɛi/, see \sectref{par:Vassimilation}. There is also a small number of cases of haplology, which are presented in \sectref{sec:Haplology}.

\subsection{Vowel elision} \label{section:Vowel_elision}
\is{vowel|(}
\is{elision|(}

Vowel elision is a cross-linguistically widespread process of hiatus resolution. Paunaka has vowel elision at some morpheme boundaries, precisely when a) a person marker\is{person marking|(} precedes a vowel-initial nominal\is{nominal stem} or verbal stem\is{verbal stem|(} and b) a morpheme with initial /i/ is attached to a stem that ends in a diphthong.\is{vowel sequence}  
The scenario described in a) is common in the Arawakan family.\is{Arawakan languages} In those cases either the vowel of the person marker or the first vowel of the stem is deleted \citep[385]{Payne1991}. In Paunaka, it is the vowel of the person marker that is deleted. \tabref{table:elision1} gives a (simplified) verb paradigm showing the elision of the vowel of the person markers.


\begin{table}[t]
\caption{Vowel elision on person markers}

\begin{tabular}{lllll}
\lsptoprule
\multicolumn{2}{l}{Person marker} & Verb stem & Inflected verb & Translation\\
\midrule
1\textsc{sg} & \textit{nÿ-} & \multirow{6}{*}{\textit{-ebuku} ‘sow’} & \textit{nebuku} & I sow\\
2\textsc{sg} & \textit{pi-} &  & \textit{pebuku} & you sow\\
3\textsc{sg} & \textit{ti-} &  & \textit{tebuku} & he/she sows\\
1\textsc{pl} & \textit{bi-} &  & \textit{bebuku} & we sow\\
2\textsc{pl} & \textit{e-} &  & \textit{ebuku} & you sow\\
3\textsc{pl} & \textit{ti-…-nube} &  & \textit{tebukunube} &  they sow\\
\lspbottomrule
\end{tabular}

\label{table:elision1}
\end{table}
	
The process applies to person markers in combination with verb stems starting with any vowel except for \textit{ÿ}, because \textit{ÿ}-initial verbal stems do not exist as far as documented in the corpus.\is{verbal stem|)} 

As for nouns,\is{noun|(} the picture gets blurry. Most noun stems\is{nominal stem|(} with an initial \textit{e} also cause elision of the vowel of the person markers as is shown in Table  \ref{table:noun-elision}, exemplified here with the first person singular marker \textit{nÿ-}.\footnote{Some noun stems are obligatorily bound to a person marker or possibly to another noun stem. They are called inalienably possessed nouns in this grammar, see \sectref{sec:Possession}.} The table additionally lists a noun with initial \textit{a} causing the elision that I also found in the corpus.


\begin{table}[t]
\caption{Vowel-initial nouns with vowel elision on person marker}

\begin{tabular}{lllll}
\lsptoprule
1\textsc{sg} & & Noun stem & Possessed noun & Translation\\
\midrule
	\textit{nÿ-}  & + & \textit{-arusu-ne} &  \textit{narusune} & my rice\\
	\textit{nÿ-}  & + & \textit{-emua} &  \textit{nemua} & my belly\\
	\textit{nÿ-} & + & \textit{-etea} & \textit{netea} & my language \\
	\textit{nÿ-}  & + & \textit{-etine} &  \textit{netine} & my sister (of ♂︎) \\
	\textit{nÿ-}  & + & \textit{-eumuka} & \textit{neumuka} & my corn supply\\
\lspbottomrule
\end{tabular}

\label{table:noun-elision}
\end{table}


Nevertheless, there are also noun stems that do not cause vowel elision. In those cases the rule does not seem to apply. Alternatively, one could also assume that the stem starts with a diphthong\is{vowel sequence} and the first vowel of the diphthong is deleted or fused with the vowel of the person marker. This may be true for some but not all of the stems as comparison to the words reconstructed for Proto-Mojeño suggests.\is{Mojeño languages} In \tabref{tab:no-elision} all nouns known to me that do not cause vowel elision on the person marker are presented, here exemplified with the first person marker \textit{nÿ-}. For comparison, I also add the reconstructed Proto-Mojeño stems taken from \citet[]{CarvalhoRose2018}. In one case, the Proto-Mojeño noun for ‘bone’, a possible alternative form \textit{*-Vyope} provided by Rose (2021, p.c.) is added.\footnote{As for Proto-Bolivian-Arawakan,\is{Southern Arawakan} only one of the words has been reconstructed: \textit{*-aʧʊ} ‘grandfather’ \citep[cf.][65]{RamirezFranca2019}.} 


\begin{table}
\caption{Vowel-initial nouns with no vowel elision on person marker}

\begin{tabularx}{\textwidth}{llllQQ}
\lsptoprule
\multicolumn{2}{l}{Person marker} & Noun stem & Possessed noun & Translation & Proto-Mojeño\\
\midrule
	\textit{nÿ-}  & + & \textit{-use} &  \textit{nÿuse} & my grandmother & *-otse\\
	\textit{nÿ-}  & + & \textit{-uchiku} &  \textit{nÿuchiku} & my grandfather (of ♂︎) & *-oʧuko\\
	\textit{ni-} & + & \textit{-uma} & \textit{niuma} & my grandfather (of ♀) & \\
	\textit{nÿ-}  & + & \textit{-ukÿ} &  \textit{nÿukÿ} & my lower leg & \\
	\textit{nÿ-}  & + & \textit{-upeji} & \textit{nÿupeji} & my bone & *-opera / *-Vjope\\
	\textit{nÿ-}  & + & \textit{-upekÿ} &  \textit{nÿupekÿ} & under me & \\
	\textit{nÿ-} & + & \textit{-a} &  \textit{nÿa} & my father & *-ija\\
	\textit{nÿ-} & + & \textit{-ati} &  \textit{nÿati} & my brother (of ♀) & \\
	\textit{nÿ-} & + & \textit{-enu} & \textit{nÿenu} & my mother & *-eno\\
	\textit{nÿ-} & + & \textit{-eche} & \textit{nÿeche} & my flesh & *-eʧe\\
	\textit{chÿ-} & + & \textit{-i} & \textit{chÿi} & (its) fruit & \\
\lspbottomrule
\end{tabularx}

\label{tab:no-elision}
\end{table}


Regarding \textit{nÿupeji} ‘my bone’, \textit{nÿa} ‘my father’ and \textit{chÿi} ‘(its) fruit’, it might actually be the case that another vowel is involved if we consider Proto-Mojeño\is{Mojeño languages} and the fact that many vowel sequences in Paunaka seem to be the result of consonant deletion, see \sectref{Diphthongs}. For the last form ‘fruit’, no Proto-Mojeño word has been reconstructed, but the Trinitario\is{Mojeño Trinitario} word is \textit{-o’i} (Rose 2021, p.c.). Regarding the stem \textit{-upeji} ‘bone’, there is indeed a related free noun in Paunaka that starts with a diphthong: \textit{eupe} ‘bone’. However, we also know that a prefix \textit{e-} is frequently used in closely related \isi{Baure} to derive free from inalienably possessed nouns (cf. \citealt[119]{Danielsen2007}). The \textit{e} in \textit{eupe} could therefore also be an obsolete derivational\is{derivation} prefix lexicalised on this noun. In any case, if we are dealing with vowel elision here, it exceptionally applies to the first vowel of the diphthong of the noun stem unlike in other cases, where the vowel of the person marker is deleted. Regarding \textit{nÿa} and \textit{chÿi}, the preservation of the vowel of the person marker could also be due to minimal word requirements (see \sectref{SyllableStructure}), without necessarily involving a diphthong. 

I have no explanation for the retention of the vowel of the person marker on \textit{nÿeche} ‘my flesh’. In the case of \textit{nÿenu} ‘my mother’, the vowel of the first person singular and third person marker is maintained, but not the one of the second person singular and first person plural marker, i.e. ‘your mother’ is \textit{penu}.

Most of the vowel-initial stems that do not cause vowel elision on the person marker start with /u/ and the list in \tabref{tab:no-elision} is exhaustive for noun stems starting with /u/ as found in the corpus, so bound nouns with initial /u/ never seem to cause vowel elision on the person marker. Possible exceptions are \mbox{\textit{-ubiu}} ‘house’, but this is structurally a (highly lexicalised)\is{lexicalisation} subordinate verb,\is{deranked verb} and \textit{-upukene} ‘load’, which is a nominalised verb.\is{nominalisation} In both cases, the vowel of the person marker is deleted.\is{nominal stem|)}\is{noun|)} The vowel is not preserved with the few verb stems\is{verbal stem} starting with /u/ as is shown in \tabref{table:elision2}.

\begin{table}
\caption{Vowel elision caused by verb stems with initial /u/}

\begin{tabular}{lllll}
\lsptoprule
1\textsc{sg} & & Verb stem & Inflected verb & Translation\\
\midrule
	\textit{nÿ-}  & + & \textit{-umu} &  \textit{numu} & I take\\
	\textit{nÿ-} & + & \textit{-umeiku} &  \textit{numeiku} & I steal \\
	\textit{nÿ-} & + & \textit{-upunu} &  \textit{nupunu} & I bring \\
	\textit{nÿ-} & + & \textit{-ububuiku-bu} & \textit{nububuikubu} & I am (in a place)\\
\lspbottomrule
\end{tabular}

\label{table:elision2}
\end{table}

\is{person marking|)}

The second kind of vowel elision applies to the \isi{non-verbal irrealis marker} \textit{-ina} (see \sectref{NominalRS}). When this marker is attached to a stem that ends in a diphthong\is{vowel sequence} or in two adjacent V syllables (see discussion in \sectref{Diphthongs}), the initial vowel of the irrealis marker is lost. This is often the case with borrowed participles from Spanish (see \sectref{sec:borrowed_verbs}), but also with a few other words, see \tabref{table:elision-ina}. The same could be true for the \isi{frustrative} and deceased markers,\is{deceased marking} which both have the form \textit{-ini}, but I have not found any examples in the corpus.

\begin{table}
\caption{Vowel elision caused by final diphthong or VV sequence}

\begin{tabularx}{\textwidth}{lXlll}
\lsptoprule
 Word & & Irrealis marker & Inflected stem & Translation\\
\midrule
	\textit{aumue} & + &\textit{-ina} & \textit{aumuena} & chicha (\textsc{irr})\\
	\textit{nÿ-a}  & + & \textit{-ina} &  \textit{nÿana} & my father (\textsc{irr})\\
	\textit{ni-yae}  & + & \textit{-ina} &  \textit{niyaena} & my, mine (\textsc{irr}) \\
	\textit{arbirau} & + & \textit{-ina} &  \textit{arbirauna} & he/she/it forgets (\textsc{irr})\\
	\textit{organisau} & + & \textit{-ina} & \textit{organisauna} & he/she/it organises (\textsc{irr})\\
\lspbottomrule
\end{tabularx}

\label{table:elision-ina}
\end{table}

There is yet another type of vowel elision, which is apparently not morpho\-phon\-ological, because it does not affect vowels at morpheme boundaries.\footnote{There is possibly even a third type, but this is even more restricted (see \sectref{SyllableStructure}).} Even though it is very restricted, it will be described here. Vowel elision applies to a few words with initial /a/ or /u/ before the plosives /p/ or /k/. Some lexemes that may appear with or without an initial vowel are shown in \tabref{tab:word-initialElision}. Stress\is{stress} is on the same syllable regardless of whether the initial vowel is there or not; it is placed on the second syllable of the words containing the initial vowel, and on the first syllable if this vowel is absent. As for the word \textit{(a)pimiya}, the differences could be gender-related, since Miguel and José use predominantly \textit{apimiya}, Juana and María C. predominantly \textit{pimiya}. However, Isidro and Alejo also tend to use \textit{pimiya} rather than \textit{apimiya} %(4 and 8 vs. 0 tokens in the corpus, respectively),
and María S. and Juana have also been found using \textit{apimiya}, although very rarely. %there are at least 2 tokens for each of them
 Thus this question cannot be settled. Gender-related differences (genderlects) are not a pervasive feature in Paunaka, except for a few peculiarities in kinship terms.\footnote{There is one word, \textit{-piji} for a sibling of the same sex, the brother of a woman is \textit{-ati} and the sister of a man \textit{-etine}. In addition, the grandfather of a woman is \textit{-uma} and of a man \textit{-uchiku}. However, these terms describe gender differences of the ego term of the kinship relation (i.e. the possessor); it is not the case that female and male speakers use different words or grammatical markers to refer to the same relationship (unlike in Trinitario,\is{Mojeño Trinitario} where this happens marginally, cf. \citealt[]{Rose2013}).} However, genderlects are found in \isi{Bésiro}, where male speakers produce some nouns with an initial vowel, while women use the form without initial vowel \citep[]{Nikulin2019a}. The nouns in \tabref{tab:word-initialElision} could have a Bésiro origin, but if so, they have changed in meaning and/or structure: \textit{(u)pichai} ‘medicine’, could thus derive from \textit{pichara} ‘poison’, and \textit{apimiya} ‘girl, young women’ shows some similarity with \textit{kupíkimia} ‘girl’ \citep[cf.][]{Sans2011}, although this could also be a coincidence. 
 
The variation between \textit{upichai} / \textit{pichai} and \textit{uchenekÿ} / \textit{chenekÿ} is found with several speakers. \textit{Upunachÿ} with the initial /u/ has only been found in the corpus when it was produced by Juana, and does not occur very frequently, \textit{ukajanechÿ} was used by Juana and María S.


\begin{table}
\caption{Word-initial vowel elision}

\begin{tabularx}{\textwidth}{QQl}
\lsptoprule
With initial vowel & Without initial vowel & Translation\\
\midrule
	\textit{\textbf{a}pimiya} & \textit{pimiya} & girl, young woman\\
	\textit{\textbf{u}chenekÿ} &  \textit{chenekÿ} & way, path, street\\
	\textit{\textbf{u}pichai} &  \textit{pichai} & medicine\\
	\textit{\textbf{u}punachÿ} & \textit{punachÿ} & other\\
	\textit{\textbf{u}kajanechÿ} & \textit{kajanechÿ} & how many\\
	
\lspbottomrule
\end{tabularx}

\label{tab:word-initialElision}
\end{table}


It should also be mentioned that María C. frequently omits the first vowels of vowel-initial words, e.g. for \textit{isipau} (/isipau/) ‘fermented chicha’ she says  \textit{sipau} (/sipau/). This, however, may indeed be due to interference with genderlects in \isi{Bésiro} (see \sectref{sec:Speakers}). 
\is{elision|)}

\subsection{Vowel assimilation} \label{par:Vassimilation}\is{assimilation|(} 

When the \isi{non-verbal irrealis marker} \textit{-ina} is attached to a stem with a final /a/, both vowels fuse into a diphthong,\is{vowel sequence|(} which is mostly [ɛi] and only sometimes [ai]. Some examples with [ɛi] are given in (\getfullref{ex:Vass.1}) and (\getfullref{ex:Vass.2}), the latter examples being loans from Spanish.

\ea\label{ex:Vass}
%\vtop{\labels\halign{\tl #\hfil\tab \tspace[textoffset]#\hfil\tab\tab \tspace[dimb]#\hfil\\
\ea\label{ex:Vass.1} 	\tab{/}nɛna/ \tab +\tab /ina/\tab =\tab /nɛnɛina/\tab ‘like (\textsc{irr})’\\
	\tab{/}puna/ \tab +\tab /ina/\tab =\tab /punɛina/\tab ‘other (\textsc{irr})’\\
\ex\label{ex:Vass.2} 	\tab{/}kuɛsta/ \tab +\tab /ina/\tab =\tab /kuɛstɛina/\tab ‘difficult (\textsc{irr})’\\
	\tab{/}kapija/ \tab +\tab /ina/\tab =\tab /kapijɛina/\tab ‘chapel (\textsc{irr})’\\
% 	}}
\z
\xe


Note that this assimilation usually does not take place when the deceased\is{deceased marking} or the \isi{frustrative} marker, which are both \textit{-ini}, attach after /a/. The diphthong\ is [ai] in that case. Two examples are given in (\getref{ex:no-Vassi})

\ea\label{ex:no-Vassi}
%\vtop{\labels\halign{\tl #\hfil\tab \tspace[textoffset]#\hfil\tab\tab \tspace[dimb]#\hfil\\
\ea 	\tab{/}taita/ \tab +\tab /ini/\tab =\tab /taitaini/\tab ‘(my) late dad’\\
\ex 	\tab{/}nijuna/ \tab +\tab /ini/\tab =\tab /nijunaini/\tab ‘I would go’\\
     \tab{/}βiβɨsɨa/ \tab +\tab /ini/\tab =\tab /βiβɨsɨaini/\tab ‘we would come’\\
% 	}}
\z
\xe
\is{vowel sequence|)}

  % \tab{/}pipahɨka/ \tab +\tab /ini/\tab =\tab /pipahɨkaini/\tab ‘you (\textsc{sg}) would stay’\\

%nakeini jxx-a120516l-a.194
%capilleina mxx-p110825l 138
%kuesteina cux-c120410ls
%neneina cux-x120414ls-2.
%puneina mrx-c120509l
%but: nenaini, perutaina
\is{assimilation|)} 
\is{vowel|)} 

\subsection{Haplology}\label{sec:Haplology}\is{haplology|(}

Haplology is a process by which adjacent identical or very similar syllables\is{syllable} are avoided. 
There are only very few cases of haplology in Paunaka, but they will be mentioned here: The first person plural\is{person marking} form of the verb \textit{-beu} ‘take way’ may be realised as \textit{beu} instead of \textit{bibeu}. The third person marker \textit{chÿ-} is sometimes omitted on the noun \textit{-chuku} ‘side’, so we find \textit{chukuyae} besides \textit{chÿchukuyae} ‘close to him/her/it, at his/her/its side’. 

If a grammatical marker that ends in \textit{nu} is followed by the \isi{plural} marker \textit{-nube}, the two syllables \textit{nu} may fuse. Thus for instance the regressive marker\is{regressive/repetitive} \textit{-pupunu} plus plural marker \textit{-nube} can fuse into \textit{-pupunube}, as in (\getref{ex:haplology}).

\ea\label{ex:haplology}
\begingl
\glpreamble tibÿsÿupupunube\\
\gla ti-bÿsÿu-pupunu-nube\\
\glb 3i-come-\textsc{reg}-\textsc{pl}\\
\glft ‘they came back’
\trailingcitation{[mqx-p110826l.084]}
\endgl
\xe

As for the prior motion marker \textit{-punu}\is{associated motion} followed by \textit{-nube}, it is hard to interpret whether there are cases of haplology, since there is also a \isi{dislocative} marker \textit{-pu}, and both can occur in similar contexts, see \sectref{sec:AssociatedMotion}.
\is{haplology|)}

\section{Syllable and word structure} \label{SyllableStructure}

In this section, the structure of possible syllables is explained in \sectref{sec:SyllableStructure_subsection}. In \sectref{sec:WordStructure_subsection}, I describe the minimal size requirements for a word.

\subsection{Structure of the syllable}\label{sec:SyllableStructure_subsection}
\is{syllable|(}
The syllable structure is {(}C)V(V), with CV being the most frequent type. In addition, there are some loans\is{borrowing} with closed syllables. VV̯ syllables without an onset only occur word-initially and are very restricted. Syllables that only consist of V occur word-initially and word-finally. Whether they also occur in the middle of words depends on how vowel sequences are analysed, a question that could not be solved in this work (see \sectref{Diphthongs}).

\largerpage
Some examples of the different syllable types are given in \tabref{table:SyllableStructure}. 




Closed syllables are found in some loans.\is{borrowing|(} There are a few examples with a nasal coda \isi{consonant}, and two with /s/.\footnote{Note that Bésiro\is{Bésiro|(} has a similar restriction on coda consonants, which must be [s], [ʂ], and [ʃ] \citep[cf.][53]{Sans2010}. What is interpreted as a nasal coda consonant in Paunaka is a prenasalised onset of a new syllable in Bésiro.\is{Bésiro|)} According to \citet[93]{Sans2010}, Bésiro obstruents have prenasalised allophones when preceded by a nasal vowel. Paunaka does not have prenasalised obstruents in nasal environments, and loans with a sequence {(}C)VC$_{nasal}$C$_{plosive}$V must therefore be interpreted as containing a closed syllable with a nasal coda consonant followed by an open syllable ({(}C)VC$_{nasal}$.C$_{plosive}$V).} Interestingly, while the Spanish sound /s/ is frequently reduced to [h] in coda position in local Spanish, this does not hold for the forms borrowed into Paunaka, which have [s]. \tabref{table:SyllableStructureLoans} presents some loans containing closed syllables.\footnote{As for the last word in the table, the \isi{Bésiro} citation form has the singular marker \textit{-xɨ}, whereas the Paunaka singular form is lexicalised\is{lexicalisation} with the Bésiro plural marker \textit{-ka}. See also \sectref{sec:Name} for the occurrence of the plural marker in the name of the language.}\clearpage

\begin{table}
\caption{Syllable structures}

\begin{tabular}{llll}
\lsptoprule Syllable type & Example &  & Translation\\
\midrule V & [\textbf{a}.ˈpu.kɛ] & /apukɛ/ & ground \\
& [mu.ˈkɨ.\textbf{ɛ}] & /mukɨɛ/ & squash \\
 VV̯ & [ˈ\textbf{au}.moɛ] & /aumue/ & chicha \\
& [ˈ\textbf{ɛu}.pɛ] & /ɛupɛ/ & bone \\
 CV & [ˈ\textbf{jɨ}.ki] & /jɨki/ & fire\\
& [i.ˈsi.\textbf{ni}] & /isini/ & jaguar \\
 CVV̯ & [ˈ\textbf{hãĩ}.kɛ] & /haikɛ/ & star \\
& [hã.ˈ\textbf{mui}.kɛ] & /hamuikɛ/ & countryside \\
\lspbottomrule
\end{tabular}

\label{table:SyllableStructure}
\end{table}%


\begin{table}
\caption{Closed syllables in loanwords}

\begin{tabularx}{\textwidth}{llQl}
\lsptoprule 
Source & Source & Phonemic representation & Translation\\
language & word &  in Paunaka & \\
\midrule
Spanish & \textit{dos} & /ɾus.ʧɨ/  & two\\
& \textit{tres} & /tɾɛs.ʧɨ/ & three\\
& \textit{gente} & /hɛn.tɛ/ & man (Span. ‘people’)\\
& \textit{de repente} & /ɾɛ.pɛn.tɛ/ & maybe\\
Bésiro & \textit{numukiánxɨ} & /mu.kian.ka/ & animal\\
\lspbottomrule
\end{tabularx}

\label{table:SyllableStructureLoans}
\end{table}
\is{borrowing|)}   

In addition, there are two words with a CVC syllable that most probably resulted from vowel elsion\is{elision|(}. Vowel elision is usually restricted to morpheme boundaries in Paunaka (see \sectref{section:Vowel_elision}), but more pervasively found in Trinitario\is{Mojeño Trinitario} and \isi{Baure}.\footnote{In Baure, vowel elision applies to the vowels /o/ and to a lesser degree /i/. They are deleted when they are unstressed, mostly word-finally, but also in the middle of words. Elision in \isi{Baure} depends to some degree on the speech rate (\citealp[cf.][53--54]{Danielsen2007}; \citealt{BaptistaWallin1968}). In Trinitario, elision applies to all metrically weak vowels except the final one \citep[cf.][1]{Rose2019}. In both languages, it has the effect that closed syllables arise in the spoken language, although there are no closed syllables underlyingly. One lexical word has different surface forms, depending on speech rate, the presence of certain morphemes and its position in a phonological phrase.} The two words are given below.

\ea\label{ex:CVCnative}
 %\vtop{\halign{%
%#\hfil\tab\tab \qquad #\hfil \\
    [a.hũ.ˈmɛɾ.ku] \tab /ahumɛɾku/ \tab\tab ‘paper’\\
    {[}nɨ.ˈpuɾ.tu.ku] \tab /nɨpuɾutuku/ \tab ‘I put into’\\%}}
    \xe

In both cases, deletion has affected a vowel between /ɾ/ and a plosive, most likely the vowel /u/.\footnote{This is somehow surprising, since /ɾ/ was usually deleted in Paunaka words, see \sectref{phonology_r} and \citet[]{deCarvalhoPAU}, leaving behind quite a few vowel sequences,\is{vowel sequence} as discussed in \sectref{Diphthongs}. In \isi{Mojeño Trinitario}, rhythmic syncope would predict elision of exactly those vowels (Rose 2021, p.c.).} The first example, the noun \textit{ajumerku} ‘paper’ (pronounced [a.hũ.ˈmɛɾ.ku]) apparently has cognate forms in \isi{Baure} and Mojeño\is{Mojeño languages} (Baure: \textit{jame\-rok} and \textit{-ajmer}, Ignaciano \textit{ajumeruca} cf. \citealt[885]{OlzaZubiri2004}, Trinitario \textit{‘jiumeruko} (Rose 2021, p.c.)). However, this word is peculiar in the sense that both Paunaka and Ignaciano\is{Mojeño Ignaciano} have /u/ in the second syllable. While /u/ is a reflex of *o in Paunaka \citep[]{deCarvalhoPAU}, the reflex of this vowel is /a/ in Ignaciano \citep[]{CarvalhoRose2018}. Nonetheless, since this word is analysable in Mojeño (a noun derived from a verb, Rose 2021, p.c.), it is most probably a cognate form. \textit{Ajumerku} never occurs with an additional [u] in Paunaka (thus */ahumɛɾuku/).
%Ignaciano: aju-me-ru-ka write-CLF-DERIV-N.POSD

The second word with the alleged deleted /u/ is the verb \textit{-purtuku} ‘put into’. In careful speech, however, a very short [u] is audible between the flap and the plosive, so that it may surface as [-pu.ɾu.ˈtu.ku] (thus the phonemic form is /puɾutuku/). The verb is also sometimes pronounced [-ˈpu.tu.ku], with deletion of the whole syllable.
When the \isi{classifier} \textit{-e} ‘to, in, into water’ (see \sectref{sec:CLF_ActiveVerbs}) is inserted in the verb stem, the /u/ also shows up; \textit{-purutueku} ‘immerse in water’.\is{elision|)} 

Some loans\is{borrowing|(} from Spanish have syllables with complex onsets\is{consonant} consisting of a plosive and the flap, see \tabref{table:ComplexOnsets}.

\begin{table}
\caption{Complex onsets in loans}

\begin{tabularx}{\textwidth}{QQQQ}
\lsptoprule Spanish word & Paunaka word & Translation \\
\midrule
\textit{plato} [ˈpla.to] & \textit{pratuchÿ} [ˈpɾa.tu.ʧɨ] & plate\\
\textit{trabajo} [tɾa.ˈβa.xo] & \textit{trabaku} [tɾa.ˈba.ku] & work\\
\textit{patrón} [pa.ˈtɾon]& \textit{patrun} [pa.ˈtɾun] & lord, employer, boss\\
\textit{Clara} [ˈkla.ɾa] & \textit{Krara} [ˈkɾa.ɾa] & Clara (proper name)\\
\lspbottomrule
\end{tabularx}

\label{table:ComplexOnsets}
\end{table}

However, complex onsets can also be dissolved by insertion of a vowel, as in (\getref{ex:NoComplexOnsets}).

\ea\label{ex:NoComplexOnsets} %\vtop{\halign{%
%#\hfil\tab\tab \qquad #\hfil \\
    [ˈsan.ta ˈku.ɾu] \tab /santa kuɾu/ \tab ‘Santa Cruz’ (city name) \\%}}
    \xe 

This last example may also be pronounced [ˈsan.ta ˈkɾu].
 \is{borrowing|)}
 \is{syllable|)}
 
\subsection{Prosodic structure of the word}\label{sec:WordStructure_subsection} 
In Paunaka, all content words are at least bimoraic, and most of the bimoraic words are disyllabic. There are only a few monosyllabic words, and they all have a heavy \isi{syllable} with an onset consonant and a diphthong.\is{vowel sequence} Some examples are given in (\getref{ex:CVV}). There are some monomoraic verb roots,\is{verbal root} but they obligatorily combine with person and reality status markers, so that they never appear as monomoraic forms in actual speech. There are also a few monomoraic function words.

\ea\label{ex:CVV}
 %\vtop{\halign{%
%#\hfil&& \qquad #\hfil \\
    [ˈmai]	\tab /mai/	\tab\tab ‘stone’\\
    {[}ˈpɛi]	\tab\tab /pɛi/	\tab\tab ‘agouti’  \\
    {[}ˈʧɨi]	\tab\tab /ʧɨi/	\tab\tab ‘fruit’  \\
    {[}ˈjui]	\tab\tab /jui/	\tab\tab ‘bread’  \\%}}
    \xe
 

There is a large number of disyllabic and trisyllabic content words. Disyllabic words of the structure CV.V are very rare, and there is also only one word, to my knowledge, which is composed of just two V syllables (see also \sectref{sec:Stress}). (\getref{ex:CV-V}) lists two words with the structure (C)V.V.
     
      
 \ea\label{ex:CV-V} %\vtop{\halign{%
%#\hfil&& \qquad #\hfil \\
    {[}ˈu.ɛ]\tab\tab /uɛ/\tab\tab ‘rainbow, spirit’\\
    {[}ˈnɨ.a]\tab\tab /nɨa/\tab\tab ‘my father’\\
%     }}
    \xe
    
%\emph{TO DO: check whether it is siya o sia ‘hawk sp.’ (chuhubi)! If sia, add!, if not, also delete from example further up (minimal pairs), check also pue ‘sloth sp.’ (perico)}

Tetrasyllabic nouns,\is{noun} especially names of animals, often, but not always, contain repeated syllables, have the same vowel in all four syllables or have a specific pattern (C)V$_{1}$CV$_{2}$CV$_{2}$CV$_{1}$. Examples are given in \tabref{table:TetrasyllabicStructures}.



\begin{table}
\caption{Structure of some tetrasyllabic nouns}

\begin{tabular}{lll}
\lsptoprule 
Pattern & Example & Gloss\\
\midrule
\multirow[t]{3}{*}{Repeated syllables} & \textit{pujukeke} & \textit{patasca}, a stew\\
& \textit{churupepe} & butterfly\\
& \textit{barereki} & pot\\
\tablevspace
\multirow[t]{2}{*}{Same V in all syllables} & \textit{pÿrÿsÿsÿ} & armadillo sp.\\
& \textit{urupunu} & red brocket\\
\tablevspace
\multirow[t]{3}{*}{(C)V$_{1}$CV$_{2}$CV$_{2}$CV$_{1}$} & \textit{apimiya} & girl\\
& \textit{ubaramu} & spider monkey\\
& \textit{pichuruki} & dragonfly\\
\lspbottomrule
\end{tabular}

\label{table:TetrasyllabicStructures}
\end{table}
%
%also: sÿrÿpÿtÿtÿ = picaflor ; sepitekÿrÿrÿ = picaflor; sumurukuku 'bird sp.', but the last one is a loan from Spanish; simukimua = peni (Span.), ipitinio = melero; ??kuchÿmeno =carachupa, kusuripure = oso hormiguero, pujukeke ‘patasca’, apimiya ‘girl’

Paunaka words can be very long. This is especially true for verbs\is{verb} that can undergo several derivational processes like reduplication of roots or stems, insertion of classifiers, noun incorporation, and several inflectional processes. (\getref{ex:12syllables}) gives a verb with twelve syllables. Usually, however, verbs are much shorter.


\ea\label{ex:12syllables}
\textit{tijatÿtÿkeikukukÿubunubeji}\\
{[ti.ˌhã.tɨ.ˌtɨ.ˈkɛi.ku.ku.ˌkɨu.bu.ˌnu.vɛ.hĩ]}\\
‘they went pulling themselves up with the help of sticks, it is said’ \trailingcitation{[jxx-p151016l-2]}
\xe


\section{Word stress} \label{sec:Stress}
\is{stress|(}

A stressed syllable in Paunaka is generally louder and longer than an unstressed one and has a higher pitch; thus it has features that are cross-linguistically wide\-spread to mark stress \citep[cf.][195]{Kager2007}. There is remarkable variation between the speakers here; Juana pronounces stressed syllables with such a high pitch that her Paunaka sounds like a tonal language at first impression. She does not produce such a high pitch on accented syllables when speaking Spanish.
However, there are only two words that are distinguished solely by stress (see (\getref{ex:chujiku}) in \sectref{sec:IambicPattern}). Thus, the definition by \citet[230]{Yip2007} of a tonal language as “one in which an indication of pitch enters into the lexical realisation of at least some morphemes” does not hold for Paunaka.

Stress assignment is best explained on the basis of metrical patterns \citep[cf.][]{Hayes1995}.\footnote{Thanks to Françoise Rose for pointing this out to me and for her very helpful comments on this section.}  It is similar to the stress patterns found in the \isi{Mojeño languages} \citep{CarvalhoRose2018,Rose2019} and usually follows an iambic\is{iambic|(} pattern with left-to-right parsing, which is sensitive to morae (\sectref{sec:IambicPattern}). The last \isi{syllable} of a word is always extrametrical, i.e. it is “invisible” for stress parsing. Morae are organised into feet, and primary stress is usually found on the last foot of the word, sometimes on the last foot that belongs to the word \textit{stem}, depending on how many markers are attached. Secondary stress is on every other foot.
Bimoraic words have a trochaic\isi{trochaic|(} pattern for stress assignment (\sectref{sec:TrochaicPattern}). When grammatical markers are attached to the word stem, the metrical pattern of the stem is maintained, i.e. words with iambic stems follow an iambic pattern, words with trochaic stems have a trochaic pattern.c\is{trochaic|)}

\subsection{Iambic pattern}\label{sec:IambicPattern}

The iambic pattern found with most words that have a stem with three or more morae is presented in \tabref{table:IambicMorae}. It holds for both inflected and uninflected words, with a few exceptions.\footnote{Note that uninflected words are almost exclusively nouns, since verb stems never show up without inflection.} I  only present uninflected or minimally inflected words here, but the same patterns apply to words with more inflectional markers.

\begin{table}
\caption{Iambic pattern}

\begin{tabular}{lllllllll}
\lsptoprule 
Number of morae & \multicolumn{7}{l}{Morae, feet and stress assignment}\\
\midrule
odd & μ & μ & ... & μ & μ & μ &\\
& (. & x) & ... & (. & x) & .& \\
&& & & & x & & \\
&& & & & &\\
even & μ & μ & ... & μ & μ & μ & μ \\
& (. & x) & ... & (. & x) & . & . &\\
&& & & & x & & \\
\lspbottomrule
\end{tabular}

\label{table:IambicMorae}
\end{table}



Some words with three syllables are given below. They are all uninflected or minimally inflected.\footnote{By minimally inflected, I mean that these words consist of roots with obligatory morphology only, i.e. I do not give bare word stems, e.g. for verbs or inalienably possessed nouns.} 


\ea\label{ex:TrisyllabicWords}
 \textit{Trisyllabic words with rhythmic pattern} (. x) .\\
% \par\nobreak\medskip \quad\vbox{\halign{%
%#\hfil\tab\tab \hskip3em #\hfil\\
    {[}ta.ˈkɨ.ra] \tab  ‘hen’\\
    {[}ʧi.ˈkɛ.pɨ] \tab  ‘his/her back’\\
    {[}pi.ˈni.ku] \tab  ‘you (\textsc{sg}) eat’\\
    {[}ta.ˈnɨ.ma] \tab  ‘now’  \\%}}
    \xe
     %{[}pi.ˈsa.nɛ] \tab /pisanɛ/ \tab ‘your field’\\

Tetrasyllabic words also usually have a iambic parse, see (\getref{ex:Tetra}):

\ea\label{ex:Tetra}
 \textit{Tetrasyllabic words with iambic pattern} (. x) . .\\
% \par\nobreak\medskip \quad\vbox{\halign{%
%#\hfil\tab\tab \hskip3em #\hfil\\
    {[}ʧu.ˈɾu.pɛ.pɛ] \tab ‘butterfly’\\
    {[}ni.ˈʧɨ.nu.mi]  \tab ‘I am sad’\\
    {[}tɨ.ˈβɨ̃.βɨ̃.ka] \tab\tab ‘it flies (\textsc{irr})’\\
    {[}mu.ˈtɛ.mɛ.na] \tab ‘big’\\%}}
    \xe


Words with five syllables are given in (\getref{ex:Penta}):

\ea\label{ex:Penta}
 \textit{Pentasyllabic words with iambic pattern} (. x) (. x) .\\
% \par\nobreak\medskip \quad\vbox{\halign{%
%#\hfil\tab\tab \hskip3em #\hfil\\
   {[}ʧi.ˌhĩ.ku.ˈpu.pi] \tab ‘his/her oesophagus’\\
   {[}pi.ˌpɨ.su.ˈsi.ka] \tab ‘your (\textsc{sg}) elbow’\\
   {[}ti.ˌbu.ɾu.ˈɾu.ka]  \tab ‘it boils (\textsc{irr})’\\
   {[}pi.ˌjɨ.ti.ˈka.pu]  \tab ‘you (\textsc{sg}) cook (\textsc{irr})’\\%}}
\xe

Finally, some hexasyllabic verbs are shown in (\getref{ex:Hexa}). There are no uninflected or minimally inflected hexasyllabic nouns to my knowledge.

\ea\label{ex:Hexa}
 \textit{Hexasyllabic words with iambic pattern} (. x) (. x) . .\\
% \par\nobreak\medskip \quad\vbox{\halign{%
%#\hfil\tab\tab \hskip3em #\hfil\\
   {[}ti.ˌku.pa.ˈnɛ.hĩ.ku] \tab\tab  ‘he/she/it steps on’\\
   {[}ti.ˌpɨ.si.ˈsi.ku.βu] \tab\tab   ‘he/she/it is alone’\\
   {[}ni.ˌku.ɾu.ˈmɛ.hĩ.ku] \tab  ‘I pierce’\\
   {[}pi.ˌjɨ.ti.ˈpa.hĩ.ku] \tab\tab  ‘you (\textsc{sg}) make chicha’\\
%    }}
   \xe
   
That parsing is sensitive to morae and not to syllables becomes apparent when considering words that contain diphthongs.\is{vowel sequence}\footnote{But remember that many diphthongs result from consonant deletion (see \sectref{Diphthongs}); a diphthong may thus still count as if it was two CV syllables.} 
The following examples in (\getref{ex:DisyllabicDiphthongIambs}) thus all follow an iambic pattern.

\ea\label{ex:DisyllabicDiphthongIambs}
  \textit{Disyllabic, trimoraic words with iambic pattern} (. x) .\\
% \par\nobreak\medskip \quad\vbox{\halign{%
%#\hfil\tab\tab \hskip3em #\hfil\\
     {[}ɛ.ˈhũi] \tab ‘cock’\\
     {[}ku.ˈpɛi] \tab ‘afternoon’\\
     {[}ni.ˈmɨ̃u] \tab ‘my clothes’\\%}}
    \xe   
    
    
The same holds for longer words that contain a diphthong,\is{vowel sequence|(} most notably in words with four syllables.\footnote{This has to do with the location of diphthongs in words. With regard to tri- and pentasyllabic words, an iambic parse by syllable or mora would result in the same stressed syllable.} Primary stress falls on the fourth mora, a vowel that fuses into a diphthong with the preceding one. Secondary stress is deleted due to stress clash.

\ea\label{ex:TetraPenV}
 \textit{Tetrasyllabic, pentamoraic words with iambic pattern} (. x) (. x) .\\
% \par\nobreak\medskip \quad\vbox{\halign{%
%#\hfil\tab\tab \hskip3em #\hfil\\
    {[}pi.sa.ˈmui.ku] \tab ‘you (\textsc{sg}) listen’\\
    {[}ti.ma.ˈhãĩ.ku] \tab ‘it barks’\\
    {[}ni.ja.ˈjau.mi] \tab ‘I am happy’\\
     {[}ka.ja.ˈɾau.nu] \tab ‘\textit{karay}, man from Santa Cruz’ (pejorative)\\%}}
     \xe

     \is{vowel sequence|)}

A few words seemingly do not respect the iambic parse. There may be different reasons, and not all of them can be explained with my current knowledge. First of all, a morphophonemic rule deletes vowels from person markers preceding a stem that begins with a vowel (see \sectref{section:Vowel_elision}). The surface form of those words that are affected by this rule has one \isi{syllable} less than the underlying form. Iambic parse holds for these words, but only for the underlying form, which tells us that stress assignment is prior to vowel deletion of the person marker, as shown by the examples in \tabref{table:VEL-Iambs}.

\begin{table}
\caption{Words with iambic pattern in the underlying form}

\begin{tabularx}{\textwidth}{QlllQ}
\lsptoprule
 Morphological parse & Underlying form &  & Surface form & Translation\\
\midrule
nɨ-ɛtuku & nɨ.ˈɛ.tu.ku & → & [ˈnɛ.tu.ku] & ‘I put’\\
pi-upunu & pi.ˈu.pu.nu & → & [ˈpu.pu.nu] & ‘you bring’\\
    βi-ɛhĩku &  vi.ˈɛ.hĩ.ku &  → & [ˈvɛ.hĩ.ku] & ‘we transport’\\
    ʧɨ-imumuku & ʧɨ.ˌi.mu.ˈmu.ku & → & [ˌʧi.mu.ˈmu.ku] & ‘she looks at her’\\

\lspbottomrule
\end{tabularx}

\label{table:VEL-Iambs}
\end{table}

     
Second, reduplicated\is{reduplication} or repeated syllables may be extrametrical, but there are also cases in which they are metrical (see examples above), so that this analysis is weak (and the first word given in (\getref{ex:ViolaRED}) does not fit this analysis at all). Below are some examples with repetition that do not follow an iambic pattern.
 
 \ea\label{ex:ViolaRED}
 \textit{Words with repeated syllables violating iambic pattern}\\
% \par\nobreak\medskip \quad\vbox{\halign{%
%#\hfil\tab\tab \hskip3em #\hfil\\
     {[}pu.hũ.ˈkɛ.kɛ] \tab ‘patasca’ (a stew)\\
   {[}ti.ˌku.ja.hĩ.ˈhĩ.ku] \tab ‘he/she laughs’\\
   {[}ti.ˌja.pi.pi.ˈpi.ku] \tab ‘it wags its tail’\\%}}
\xe


Third, some verbs\is{verbal stem} that contain a syllable [hĩ] or [hɨ̃] also violate the iambic pattern, since this syllable seems to attract stress. 

\ea\label{ex:VIOLhi}
 \textit{Verbs with a sequence /hi/ or /hɨ/ violating iambic pattern}\\
% \par\nobreak\medskip \quad\vbox{\halign{%
%#\hfil\tab\tab \hskip3em #\hfil\\
     {[}ˌvi.nɛ.ˈhĩ.ka] \tab ‘we leave it (\textsc{irr})’\\
     {[}tu.ˈhĩ.ku] \tab\tab ‘it suckles’\\
    {[}ˌti.pa.ˈhɨ̃.ka] \tab ‘he/she/it stays (\textsc{irr})’\\
    {[}ˌvi.mi.ˈhɨ̃.ku] \tab  ‘we raise someone’\\
%     }}
    \xe

%but ti.'ye.ji.ku 'arrancar, cosechar'    
%nÿ.pu.jÿ.ka 'I will cook it'
% ?narejiku 'lo raspo'

However, this is not always the case, i.e. there are verbs containing [hĩ] or [hɨ̃] that have the normal iambic parse, e.g. [ni.ˈje.hĩ.ku]  ‘I tear out, harvest’. There is even one case in which two words containing the \isi{syllable} [hĩ] are distinguished by stress placement, as shown in (\getref{ex:chujiku}).

\ea\label{ex:chujiku} %\vtop{\halign{%
%#\hfil\tab\tab \qquad #\hfil \\
    [ni.ˈʧu.hĩ.ku]  \tab ‘I speak’\\
  {[}ˌni.ʧu.ˈhĩ.ku] \tab ‘I harvest’\\
%     }}
    \xe

It could be the case that all verb stems\is{verbal stem} in (\getref{ex:ViolaRED}) to (\getref{ex:chujiku}) that do not follow the iambic pattern actually start with an /i/, which is then merged with the person marker,\footnote{Since all person markers\is{person marking} contain or may contain the vowel /i/ and verbs never show up without a person marker, it is impossible to tell whether a stem begins with this vowel. As for \textit{-imumuku} ‘look’ in \tabref{table:VEL-Iambs}, we only know that there is an initial vowel because there is a related stative verb \textit{-imubÿke}, and since stative verbs are marked for \isi{irrealis} by a \isi{prefix} (see \sectref{sec:VerbalRS}). The initial /i/ shows up on the surface form of the irrealis verb.} so that the iambic pattern holds for an underlying form, but not for the surface form. However this analysis is excluded for {[}tu.ˈhĩ.ku] in (\getref{ex:VIOLhi}), which should have the same stress pattern as the verbs in \tabref{table:VEL-Iambs}.

% 'ti.siu, and nísu, nísua; nímu, nímua; níyu, níyua also behave strangely!
\is{iambic|)}

\subsection{Trochaic pattern}\label{sec:TrochaicPattern}
\is{trochaic|(}

The trochaic pattern (x .) is found on bimoraic words. This holds for disyllabic ones (\getref{ex:DisyllabicWords}) as well as well as monosyllabic ones (\getref{ex:MonosyllabicWords}). 

\ea\label{ex:DisyllabicWords}
 \textit{Disyllabic words with trochaic pattern} (x .)\\
% \par\nobreak\medskip \quad\vbox{\halign{%
%#\hfil\tab\tab \hskip3em #\hfil\\
    {[}ˈku.su] \tab ‘mouse’\\
    {[}ˈɨ.nɛ] \tab\tab ‘water’\\
   {[}ˈmi.mi] \tab ‘(my) mum’ \\
    {[}ˈnɨ.a] \tab ‘my father’ \\ %}}
\xe
%sache

\ea\label{ex:MonosyllabicWords}
 \textit{Monosyllabic words with trochaic pattern} (x .)\\
% \par\nobreak\medskip \quad\vbox{\halign{%
%#\hfil\tab\tab \hskip3em #\hfil\\
    {[}ˈpai] \tab\tab ‘priest’  \\
    {[}ˈpɛi] \tab\tab ‘agouti’  \\
    {[}ˈmai] \tab ‘stone’\\
    {[}ˈjui] \tab\tab ‘bread’\\ %}}
\xe

That we are dealing with a trochaic pattern and not with an iambic one with degenerate feet becomes apparent when we attach grammatical markers to these word stems: the parse remains trochaic, as shown in \tabref{table:TrochMarkers}. However, there is some variation concerning primary stress placement: it sometimes remains on the word stem and sometimes shifts to the marker to keep with the rule that stress is on the last foot of the word. This latter option seems to be preferred if the \isi{locative marker} \textit{-yae} is involved. If more than one marker is added, stress is also more likely to fall on the last foot instead of the stem. In other cases, primary stress often remains on the stem, but with possible variations depending on the position of the word inside an utterance, a topic that deserves further investigation.



\begin{table}
\caption{Trochaic pattern on words with added markers}

\begin{tabularx}{\textwidth}{l@{}llCll}
\lsptoprule
Word stem &  & Marker & & Surface form & Translation\\
\midrule
{[}ˈku.su] & + & [mɨ.nɨ] & → & [ˈku.su.ˌmɨ.nɨ] & ‘little mouse’\\
 {[}ˈɨ.nɛ] & + & [ja.ɛ] & → & [ˌɨ.nɛ.ˈja.ɛ] & ‘in(to) the water’\\
{[}ˈmai] & + & [ha.nɛ] & → & [ˈmai.ha.nɛ] & ‘stones’\\
{[}ˈhɛn.tɛ] & + & [nu.vɛ] & → & [ˈhɛn.tɛ.ˌnu.vɛ] & ‘men’\\
\lspbottomrule
\end{tabularx}

\label{table:TrochMarkers}
\end{table}



A marker with an initial vowel fuses with the last vowel of the stem, but underlyingly, the parse is still trochaic: there are four morae, two belonging to the lexical stem and two to the grammatical marker. Primary stress is usually assigned to the first vowel of the grammatical marker in this case, unlike grammatical markers that start with a consonant and often receive secondary stress only, as was shown above. The first vowel of the vowel-initial marker is also its first mora. When this vowel fuses with the preceding one of the lexical stem, stress falls on the diphthong (or single vowel in case the final vowel of the lexical word and the first vowel of the marker are identical), thus on the second syllable of the word. Secondary stress should be on the first syllable of the word then, but it is deleted to prevent stress clash. See \tabref{table:TrocheesSuf-2} for examples.

\begin{table}
\caption{Words with trochaic pattern in the underlying form}

\fittable{
\begin{tabular}{l@{}lll@{}lll@{}}
\lsptoprule
Word stem &  & Marker & Underlying form &  & Surface form & Translation\\
\midrule
    {[}ˈku.su] & + & [i.na] & ˌku.su.ˈi.na & → & [ku.ˈsui.na] & ‘mouse (\textsc{irr})’\\
    {[}ˈɨ.nɛ] & + & [i.na] & ˌɨ.nɛ.ˈi.na & → & [ɨ.ˈnɛi.na] & ‘water (\textsc{irr})’\\
    {[}ˈjɛ.jɛ] & + & [i.ni] & ˌjɛ.jɛ.ˈi.ni & → & [jɛ.ˈjɛi.ni] & ‘(my) late granny’\\
    {[}ˈmi.mi] & + & [i.ni] & ˌmi.mi.ˈi.ni & → & [mi.ˈmi.ni] & ‘(my) late mum’\\
\lspbottomrule
\end{tabular}
}

\label{table:TrocheesSuf-2}
\end{table}


%Dieser Vorlauf hier gibt Beispiele, die mit a, b, c durchnummeriert sind aus:  
% \ex [textoffset=1em,dima=1em,dimb=1em]<ex:...>
% %\vtop{\labels\halign{\tl #\hfil& \tspace[textoffset]#\hfil&& \tspace[dimb]#\hfil\\
\is{trochaic|)}
\is{stress|)}

\section{Intonation} \label{sec:Intonation}
\is{intonation|(}
For the time being, I cannot present an in-depth study of intonational patterns in Paunaka. Nevertheless, there are some observations that I consider worth mentioning here.

Polar questions\is{polar question} can be identified by a rising tone at the end of the utterance. 
Declarative sentences\is{declarative clause} usually have a falling tone and a pitch accent on the stress-bearing syllable of the word in \isi{focus}, which is often the last content word of the utterance.

 \ea\label{ex:intonation_simple}

        \textit{nÿti nichuna nijiku}  \\
        {[}ˈnɨ.ti ni.ˈʧu.na \textbf{ˈni}.hĩ.ku] \\
        ‘I know how to spin’ \trailingcitation{[jxx-p120515l-1.059]}
        \xe
        
Speakers may, however, emphasise the last \isi{syllable} of an utterance with \isi{stress}, a lengthening of the \isi{vowel} and greater loudness. It is not clear to me what the intended effect is. I observed that María C., who uses this emphasis most often, employs it when complaining about her health, the lack of interest of the young generation in speaking Paunaka, etc. 

\ea\label{ex:intonation_long}

     \textit{nemusuikamÿnÿ kuinabu naimubÿkemÿnÿ}\\ %naimubukemÿne
    {[}nə.mu.ˈsui.ka.ˌmɨ.nə ˈkwi.na.wo nai.ˌmu.βʊ.ˌkə.mʊ.ˈ\textbf{nəː}] \\
    ‘I wash a little bit, I can't see [well] anymore’  \trailingcitation{[cux-c120410ls.107]}
    \xe

María C. is the oldest speaker. The other speakers do not frequently make use of this intonational pattern. But interestingly, in a personal narrative about two old Paunaka ladies, whom she met when she was a young woman (jxx-p120515l-1), Juana also often emphasises final syllables of utterances when she reproduces the old ladies’ speech. It may thus be a speech style that speakers used to adopt in old age, but which has gone out of usage among most of the remaining speakers.

It may also signal the end of a turn and offer the conversational partner the possibility to take the turn. In the old recordings of Riester, the lengthening of final syllables is almost absent, but that may be due to the monologue character of the interviews, with a researcher unable to understand what the speaker was saying. Stressing of utterance-final syllables appears only in one file (nxx-a630101g-3), which is about the intended or imagined theft of a young woman and directed to an imaginary addressee, who is included into the discourse by usage of first plural. Another characteristic that is more prominent in the speech of Juan Ch. is a reduced speed towards the end of the utterance. The word with the intonational pitch is even more often the last one of an utterance.
The monologic character of many of the recordings that make up my \isi{corpus} may also have an influence on the rarer occurrence of emphasised utterance-final syllables. Most often there is only one speaker present together with one or two interviewers, who certainly got more acquainted with the language over time, but cannot replace a fully proficient Paunaka speaker as a conversational partner.

A fine-grained study of Paunaka discourse style could certainly shed light on the function of this and other intonational patterns.
\is{intonation|)}





