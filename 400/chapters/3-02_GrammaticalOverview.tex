%!TEX root = 3-P_Masterdokument.tex
%!TEX encoding = UTF-8 Unicode

\chapter{Grammatical overview}\label{chap:Overview}

This chapter has the purpose to provide a short overview about the most important grammatical structures of Paunaka. It roughly follows the order in which the topics are presented in the more complete description throughout the following Chapters \ref{chap:Phonology} to \ref{sec:ComplexClauses}. The examples presented here are taken from the more detailed grammatical description, but may be abbreviated and simplified, and do thus not have a reference to the \isi{corpus}. Contrary to the remainder of this work, no information is provided to embed the examples in their extralinguistic context.


\section{Phonology}\label{sec:O_Phonology}
\is{segmental phonology|(}
Paunaka has twelve phonemic consonants.\is{consonant|(} All of them are given in \tabref{table:O_consonants}, including their orthographic\is{orthography} representation in <>. Furthermore, two additional consonants only occur in loans\is{borrowing} from \isi{Bésiro}. This fact is not recognised by the speakers themselves, and these consonants may be considered phonemic, too. They are given in parenthesis in the table.

\begin{table}[htbp] 
\caption{Consonant inventory with orthographic representation}
\fittable{
\begin{tabular}{lccccccc}
\lsptoprule
& Bilabial & Alveolar & Postalveolar & Retroflex & Palatal & Velar & Glottal \\
\midrule
Plosive & p <p> & t <t> & & & & k <k> & \\

Nasal & m <m> & n <n> & & & ɲ <ny> & & \\

Flap &  &  ɾ <r> &  &  & & & \\

Fricative & β <b> &  s <s> & (ʃ) <xh> & (ʂ) <x> & & & h <j> \\

Affricate &  &  ʧ <ch> &  &  & & & \\

Approximant &  &  &  &  & j <y> & & \\
\lspbottomrule
\end{tabular}
}
\label{table:O_consonants}
\end{table}
\is{consonant|)}

There are five contrasting vowels\is{vowel|(} that are shown in \tabref{table:O_vowels} including their orthographic representation in <>.\is{orthography}

\begin{table}[htbp]
\caption{Vowel inventory with orthographic representation}

\begin{tabular}{lccc}
\lsptoprule
 & front & central & back \\
\midrule high & i <i> & ɨ <ÿ> & u <u> \\
 mid & ɛ <e> & & \\
 low & & a <a> & \\
\lspbottomrule
\end{tabular}

\label{table:O_vowels}
\end{table}%
\is{vowel|)}

There is only little allophonic variation. Most importantly, the fricative /β/ is realised as [v] before the front vowels /i/ and /ɛ/, and as [w] before /ɨ/. If /u/ or /a/ follows, it is usually pronounced [β] but may also be [b] or [w].

Among the vowels,\is{vowel} it is most noticeable that /ɨ/ goes back to *u of a proto-language, while /u/ derives from *o (\citealt[]{deCarvalhoPAU}; \citealp[]{RamirezFranca2019}). The original quality of these vowel is still sometimes noticeable in rapidly uttered unstressed syllables, where /ɨ/ can have a tendency towards [ʊ] and /u/ towards [o].
\is{segmental phonology|)}

Phonological processes include \isi{rhinoglottophilia} caused by /h/ and \isi{nasalisation}, which may be caused by nasal consonants\is{consonant} on the following vowels.\is{vowel} Haplology\is{haplology} is a very marginal process.

Syllables\is{syllable} can have the structure (C)V(V). Closed syllables are only found in loans\is{borrowing|(} with, as far as I know, only two exceptions of native words including closed syllables. Consonant clusters\is{consonant cluster} also only occur in loans.\is{borrowing|)} There are many vowel sequences,\is{vowel sequence} which often result from diachronic consonant deletion. Some of these sequences are realized as diphthongs, while others are realized as two separate syllables.

The most important morphophonological rule in Paunaka is widespread in the Arawakan family\is{Arawakan languages} \citep[cf.][385]{Payne1991}: the deletion of vowels\is{elision} of the person markers\is{person marking} before a vowel-initial stem as shown in (\getref{ex:Sketch-VE-1}) for verbs. Note that there are no verb stems\is{verbal stem} with initial /ɨ/.

\renewcommand{\exfont}{\normalsize\upshape}
\TabPositions{1cm,2cm,3cm,4cm,5cm,6cm,7cm,8cm,9cm}
\ea\label{ex:Sketch-VE-1}
% \vtop{\halign{%
% #\hfil&\tab \qquad #\hfil \\
      /nɨ/ ‘1\textsc{sg}’ \tab + \tab /ɛpunu/ \tab → \tab /nɛpunu/  \tab ‘I take’ \\
       /pi/ ‘2\textsc{sg}’ \tab + \tab /ihiku/ \tab → \tab /pihiku/ \tab ‘you spin (thread)’ \\
       /ti/ ‘3i’ \tab + \tab /upunu/ \tab → \tab /tupunu/ \tab ‘she brings’ \\
       /chɨ/ ‘3’ \tab + \tab /imu/ \tab\tab → \tab /chimu/ \tab ‘she sees her’ \\
       /bi/ ‘1\textsc{pl}’ \tab + \tab /anau/ \tab → \tab /banau/ \tab ‘we make’ \\ %-akachu
       /e/ ‘2\textsc{pl}’  \tab + \tab /ichuna/ \tab → \tab /ichuna/ \tab ‘you know’\\
\xe
As for the combination of person markers with nouns, however, vowel elision is only caused by a few vowel-initial noun stems,\is{nominal stem} most of them starting with /ɛ/.

Another kind of vowel \isi{elision} triggers the initial /i/ of the \isi{non-verbal irrealis marker} \textit{-ina} after a diphthong. The very same marker can also cause assimilation\is{assimilation} of a final /a/ to /ɛ/, see (\getref{ex:Sketch-Phon-ina}) for examples of both processes.


\ea\label{ex:Sketch-Phon-ina}
%%\vtop{\labels\halign{\tl #\hfil\tab \tspace[textoffset]#\hfil\tab\tab \tspace[dimb]#\hfil\\
\ea\label{ex:Sketch-Phon-ina.1}	\tab{/}aumuɛ/ ‘chicha’ + /ina/ \tab → \tab /aumuɛna/ \tab ‘chicha (\textsc{irr})’\\
 \tab {/}arbirau/ ‘forget’ + /ina/ \tab → \tab  /arbirauna/ \tab ‘she forgets (\textsc{irr})’\\
 \ex\label{ex:Sketch-Phon-ina.2}\tab{/}kapija/ ‘chapel’ + /ina/ \tab → \tab /kapijɛina/ \tab ‘chapel (\textsc{irr})’\\
	\tab{/}puna/ ‘other’ + /ina/ \tab → \tab /punɛina/ \tab ‘other (\textsc{irr})’\\
\z
\xe

 
Lexical words have at least two morae. Stress\is{stress|(} assignment follows metrical patterns with left-to-right parsing \citep[cf.][]{Hayes1995}. Morae are organised into feet, the last \isi{syllable} is always extrametrical. Bimoraic words have \isi{trochaic} stress assignment, as can be seen in (\getref{ex:Sketch-Troch}). Words with any other number of morae have \isi{iambic} stress assignment, this is illustrated in (\getref{ex:Sketch-Iamb}). This unusual pattern is also found in closely related \isi{Mojeño Trinitario} \citep[]{Rose2019}. 


\ea\label{ex:Sketch-Troch}
% \vtop{\labels\halign{\tl #\hfil\tab \tspace[textoffset]#\hfil&\tab \tspace[dimb]#\hfil\\
\ea\label{ex:Sketch-Troch.1}  
\tab   {[}ˈu.ɛ] \tab\tab ‘water spirit, rainbow’\\
  \tab  {[}ˈɨ.nɛ] \tab\tab ‘water’\\
  \tab {[}ˈku.su] \tab ‘mouse’\\
\ex\label{ex:Sketch-Troch.2} \tab   {[}ˈpɛi] \tab\tab ‘agouti’\\
  \tab  {[}ˈjui] \tab\tab ‘bread’\\
  \z
\xe
%sache




\ea\label{ex:Sketch-Iamb}
% \vtop{\labels\halign{\tl #\hfil\tab \tspace[textoffset]#\hfil&\tab \tspace[dimb]#\hfil\\
\ea\label{ex:Sketch-Iamb.1} 
\tab     {[}ni.ˈmɨ̃u] \tab\tab\tab ‘my clothes’\\
 \tab    {[}ku.ˈpɛi] \tab\tab\tab ‘afternoon’\\
\ex\label{ex:Sketch-Iamb.2} 
\tab     {[}ta.ˈkɨ.ra] \tab\tab\tab  ‘hen’ \\
 \tab      {[}pi.ˈni.ku] \tab\tab\tab  ‘you (\textsc{sg}) eat’\\
\ex\label{ex:Sketch-Iamb.3} 
\tab     {[}ʧu.ˈɾu.pɛ.pɛ] \tab\tab ‘butterfly’\\
 \tab     {[}mu.ˈtɛ.mɛ.na] \tab\tab ‘big’\\
 \ex\label{ex:Sketch-Iamb.4} 
 \tab     {[}ʧi.ˌhĩ.ku.ˈpu.pi] \tab\tab ‘his/her oesophagus’\\
 \tab      {[}ti.ˌbu.ɾu.ˈɾu.ka]  \tab\tab ‘it boils (\textsc{irr})’\\
 \ex\label{ex:Sketch-Iamb.5} 
 \tab      {[}ti.ma.ˈhãĩ.ku] \tab\tab ‘it barks’\\
 \tab   {[}ni.ja.ˈjau.mi] \tab\tab ‘I am happy’\\
 \ex\label{ex:Sketch-Iamb.6} 
 \tab     {[}ti.ˌpɨ.si.ˈsi.ku.βu] \tab\tab   ‘he/she/it is alone’\\
 \tab  {[}ni.ˌku.ɾu.ˈmɛ.hĩ.ku] \tab  ‘I pierce’\\
 \z
    \xe   
    
All words in (\getref{ex:Sketch-Iamb}) are minimally grammatical, i.e. stems that require a person marker (verbs and inalienable nouns) are given with a person marker. However, the bimoraic words in (\getref{ex:Sketch-Troch}) do not require any further grammatical markers. That they follow a \isi{trochaic} pattern – instead of having default assignment of stress on the first mora due to a degenerate foot with an underlying iambic pattern – becomes clear when grammatical markers follow these stems: the pattern remains \isi{trochaic}, as can be seen in (\getref{ex:Sketch-Troch-Markers}). If grammatical markers follow a stem (regardless of whether the word exhibits a \isi{trochaic} or \isi{iambic} pattern), primary stress can either shift to the last foot or remain on the stem.\footnote{This partly depends on the markers, i.e. the \isi{locative marker} is prone to attract stress, and it partly depends on the length of the word, i.e. longer words are more likely to show stress shift to a non-stem \isi{syllable}. Pragmatic features such as the word in question providing highlighted information or backgrounded information may also play a role, this has not been investigated, yet.}

\ea\label{ex:Sketch-Troch-Markers}
% \vtop{\halign{%
% #\hfil&& \qquad #\hfil \\
    {[}ˌɨ.nɛ.ˈja.ɛ] \tab\tab ‘in the water’\\
    {[}ˈku.su.ˌhã.nɛ] \tab ‘mice’\\
    {[}ˈjui.ˌmɨ.nɨ] \tab\tab ‘small bread’\\
%     }
\xe
\is{stress|)}

A more comprehensive account of Paunaka’s phonology is given in \chapref{chap:Phonology}.
\renewcommand{\exfont}{\normalsize\itshape}

\section{General remarks on morphology and word classes}\label{sec:O_GeneralRemarksMorphology}

Paunaka is a head-marking\is{head} – or indexing in the terms of \citet[][]{Haspelmath2019} – polysynthetic and agglutinating\footnote{The classification of Paunaka as a polysynthetic and agglutinating language presupposes an analysis in which markers with promiscuous attachment to various parts of speech can be subject to morphological rather than syntactic rules, see further below, and \chapref{chap:Architecture} for a discussion of this issue.}  language\is{agglutination} with incipient \isi{fusion} concerning \is{reality status} marking on verbs (see §\ref{sec:O_Verbs} below or §\ref{sec:RealityStatus} for a detailed discussion of this).  However, although words can consist of a number of morphemes, they are often relatively simple with only a few categories being marked obligatorily. The language exhibits “transcategorial morphology”\is{transcategorial morphology} \citep[73]{Rose2014a} as usual in \isi{Arawakan languages} \citep[cf.][13]{Overall2018}, i.e. most grammatical markers can occur with words of various word classes. Indeed, it is rather an exception than the rule that a certain inflectional marker\is{inflection} is restricted to a specific part of speech.\is{word class} Most markers could thus be defined as “clitics”\is{clitic|(} according to the criterion of their promiscuous attachment. However, since the markers in question do not entirely fulfil other characteristics of clitics that have been proposed in the literature (e.g. by \citealt{ZwickyPullum1983}; \citealt{Sadock1991}; \citealt{Aikhenvald2003b}; \citealt{SpencerLuis2012}), I decided not to use the term “clitic” in this grammar and not to use the equal sign in interlinear glosses. Instead, I make use of the term “marker”, which includes clitics and affixes,\is{affix} and only use a dash to indicate morpheme boundaries. There is not a single marker, to my knowledge, that is dependent on a specific syntactic position inside the clause, but some markers, especially those encoding TAME\is{tense}\is{aspect}\is{modality}\is{evidentiality} and degree,\is{degree marker} may float in the clause without difference in meaning.\is{clitic|)}  More information about this topic can be found in \chapref{chap:Architecture}. As for the organisation of this grammar, transcategorial markers\is{transcategorial morphology} are described in the chapters where they are most likely to be expected, e.g. the diminutive is introduced in the chapter on nouns, TAME markers are described in the chapter on verbs, and person and number marking is a topic in both of these chapters.

Person markers\is{person marking|(} are the most important representatives of \isi{transcategorial morphology}. Depending on their position preceding or following the stem, they encode possessors\is{possessor} and subjects\is{subject} on nouns,\is{noun} and subjects\is{subject} and objects\is{object} on verbs\is{verb} (following nominative-accusative \isi{alignment}).\footnote{Paunaka thus exhibits \isi{split-S marking}, dependent on part of speech;\is{word class} \isi{subject} markers precede verb stems but follow stems of other parts of speech.} (\getref{ex:Sketch-Pers-N}) shows the use of person markers on nouns and (\getref{ex:Sketch-Pers-V}) on verbs, exemplified with the first person plural marker \textit{bi-}/\textit{-bi}.

\ea\label{ex:Sketch-Pers-N}
  \ea
\begingl
\glpreamble bibite\\
\gla bi-bite\\
\glb 1\textsc{pl}-necklace\\
\glft ‘our necklaces’\\
\endgl
  \ex
\begingl
\glpreamble paunakabi\\
\gla paunaka-bi\\
\glb Paunaka-1\textsc{pl}\\
\glft ‘we are Paunaka’\\
\endgl
%\trailingcitation{[]}
\z
\xe

\ea\label{ex:Sketch-Pers-V}
  \ea\label{ex:Sketch-Pers-V.1}
\begingl
\glpreamble biyunu\\
\gla bi-yunu\\
\glb 1\textsc{pl}-go\\
\glft ‘we go’\\
\endgl
  \ex\label{ex:Sketch-Pers-V.2}
\begingl
\glpreamble tikupakubi\\
\gla ti-kupaku-bi\\
\glb 3i-kill-1\textsc{pl}\\
\glft ‘it kills us’\\
\endgl
%\trailingcitation{[]}
\z
\xe

The full paradigm of person markers is given in \tabref{table:Sketch-PersonMarkers}.

\begin{table}
\caption{Person markers}

\begin{tabular}{lll}
\lsptoprule
Gloss & Form preceding stem & Form following stem \\
\midrule
1\textsc{sg} & \textit{nÿ- / ni-} & \textit{-nÿ / -ne} \\
1\textsc{pl} & \textit{bi-} & \textit{-bi} \\
2\textsc{sg} & \textit{pi-} & \textit{-bi / -pi} \\
2\textsc{pl} & \textit{e- / a- / i-} & \textit{-e} \\
3 & \textit{chÿ- / chi-} & \textit{(-chÿ)}\\
3i & \textit{ti-} & \\
\lspbottomrule
\end{tabular}

\label{table:Sketch-PersonMarkers}
\end{table}

There are some peculiarities regarding the encoding of the third person. The marker \textit{chÿ-} (or \textit{chi-}) indexes a \isi{possessor} on nouns\is{noun} just like the other person markers do, but on verbs\is{verb} it is used to encode 3>3 relationships. In the latter case, \textit{chÿ-} is obligatory if the object \is{object|(} is human and optional with non-human objects.\is{animacy} For any other relation including a verb with third person \isi{subject}, the marker \textit{ti-} is used, which only occurs with verbs. The “i” in the gloss “3i” is used to distinguish the two third person markers, it does not carry any meaning. There is no suffix corresponding to \textit{ti-}. The form \textit{-chÿ} is given in parenthesis in the table because it only occurs in a few fixed constructions, e.g. with speech verbs\is{speech verb} and in a specific \isi{focus} construction. It does not index subjects in non-verbal predication nor does it usually occur as an object marker on verbs outside the constructions mentioned. Consider (\getref{ex:Sketch-3-N}) for the use of \textit{chÿ-} as possessor marker on a noun and (\getref{ex:Sketch-3-V}) for its use on a verb, where it jointly indexes third person subject and object. See (\getref{ex:Sketch-3i-V}) for the use of the other third person marker \textit{ti-} on an \isi{intransitive} verb. For its use on a \isi{transitive} verb see (\getfullref{ex:Sketch-Pers-V.2}) above. Notice that there is no grammatical \isi{gender} in Paunaka. If the gender of a referent could not be determined by the context of an utterance, generic female gender is usually used in translations of examples.

\ea\label{ex:Sketch-3-N}
\begingl
\glpreamble chiyenu\\
\gla chi-yenu\\
\glb 3-wife\\
\glft ‘his wife’\\
\endgl
%\trailingcitation{[]}
\xe

\ea\label{ex:Sketch-3-V}
\begingl
\glpreamble chumu\\
\gla chÿ-umu\\
\glb 3-take\\
\glft ‘she takes her’\\
\endgl
%\trailingcitation{[]}
\xe

\ea\label{ex:Sketch-3i-V}
\begingl
\glpreamble timuku\\
\gla ti-muku\\
\glb 3i-sleep\\
\glft ‘she sleeps’\\
\endgl
%\trailingcitation{[]}
\xe
\is{object|)}

The third person markers do not distinguish number, but the \isi{plural} marker \textit{-nube} is added to the stem obligatorily for human referents (i.e. human third person possessors of nouns, human third person subjects or objects of verbs). The \isi{distributive} marker \textit{-jane} can optionally be added for non-human non-singular referents.\is{person marking|)}

Nouns\is{noun} and verbs\is{verb} are the most important word classes\is{word class} and are summarised in the following two sections of this overview chapter. In addition, there are minor word classes. Among those minor classes are pronouns\is{pronoun} and nominal demonstratives.\is{nominal demonstrative|(} There is no third person \isi{personal pronoun}. The difference between the two nominal demonstratives could not be determined, which is why they are glossed as ‘\textsc{dem}a’ and ‘\textsc{dem}b’ respectively (‘\textsc{dem}c’ can be analysed as an oblique nominal demonstrative or an adverbial demonstrative\is{adverb!demonstrative adverb}\is{demonstrative!demonstrative adverb}).\is{nominal demonstrative|)}

Adjectives\is{adjective} also play a minor role in Paunaka. There are only few of them and they are mainly used predicatively. Some of them behave like nouns when \isi{realis}, but their \isi{irrealis} counterparts are stative verbs.\is{stative verb} Among the numerals,\is{numeral} only \textit{chÿnachÿ} ‘one’ is of presumable Paunaka origin, all others are borrowed from Spanish,\is{borrowing|(} with the numbers ‘two’ and ‘three’ being more integrated into the language than the higher ones. In addition to numerals, there is also a number of quantifiers,\is{quantifier} some of them borrowed from \isi{Bésiro}.\is{borrowing|)} Adverbs\is{adverb} can have \isi{locative}, temporal and aspectual,\is{temporal/aspectual} and \isi{modal} meanings.

There are four prepositions\is{preposition|(}; \textit{(-)tÿpi} is a \isi{general oblique} marker, the others signal \isi{source} (\textit{tukiu}), instrument or cause \is{instrument/cause} (\textit{-keuchi}), and \isi{comitative} relation (\textit{-ajie\-chu\-bu}). The source preposition never receives person marking. The oblique preposition takes person markers obligatorily for first and second person obliques, first and second person pronouns cannot co-occur. The preposition also takes person markers for third person obliques if no NP follows. However, if the preposition is combined with an NP, person marking is optional; see (\getref{ex:Sketch-PREP}) for a case in which both person marking and NP are present.\is{general oblique} The \isi{instrument/cause} and \isi{comitative} prepositions are always person-marked regardless of whether an NP follows. Person markers\is{person marking} stem from the same set used on nouns and verbs, which has been presented in \tabref{table:Sketch-PersonMarkers} above.

\ea\label{ex:Sketch-PREP}
\begingl
\glpreamble berajane chitÿpi benu\\
\gla bera-jane chi-tÿpi benu\\
\glb candle-\textsc{distr} 3-\textsc{obl} virgin\\
\glft ‘candles for the virgin’\\
\endgl
%\trailingcitation{[]}
\xe
\is{preposition|)}

Among the coordinating\is{coordination} connectives\is{connective}, most are borrowed\is{borrowing|(} from Spanish, and among the subordinating connectives, we also find some loans.\is{borrowing|)} At least some members of the quantifiers,\is{quantifier} adverbs,\is{adverb} connectives, and prepositions\is{preposition} have grammaticalised\is{grammaticalisation} from verbs with their origin still being recognisable. 

A more profound description of minor word classes is given in \chapref{chap:MinorWordClasses}.


\section{Nouns and the NP}\label{sec:O_Nouns}
\is{noun|(}

The most important morphological processes on nouns are \isi{possession} and number marking. In addition, there is \isi{nominal irrealis}, deceased,\is{deceased marking} and \isi{diminutive} marking. Nouns are not marked for core cases, or, in other words, there is no flagging \citep[cf.][]{Haspelmath2019}. There is one general \isi{locative marker}, which encodes the most prototypical spatial relation in a given situation. Inside the NP,\is{noun phrase} demonstratives\is{nominal demonstrative} are the most frequent modifiers\is{modification} of nouns. In addition, numerals,\is{numeral} adjectives,\is{adjective} nouns and relative clauses\is{relative relation} may modify a noun. There are no articles.

Paunaka distinguishes three classes of nouns by how they interact with possession\is{possession|(} marking: there are inalienable, alienable, and non-possessable nouns. Inalienable\is{inalienability} nouns always require a person marker\is{person marking|(} to index the possessor, but some of them may derive\is{derivation} a non-possessed form by a suffix \textit{-ti}. An inalienable noun and the non-possessed form derived from it are given in (\getref{ex:Sketch-Inal}).

\ea\label{ex:Sketch-Inal}
\begingl
\glpreamble nimukiji – mukitiji\\
\gla ni-muki-ji muki-ti-ji\\
\glb 1\textsc{sg}-hair-\textsc{col} hair-\textsc{nposs}-\textsc{col}\\
\glft ‘my hair – hair’\\
\endgl
%\trailingcitation{[]}
\xe

Alienable\is{alienability} nouns are not possessed in their basic form, but can take possessor marking. Some of them require an additional suffix \textit{-ne} in their possessed form, others do not, as is exemplified by the two nouns in (\getref{ex:Sketch-Al}). 

\ea\label{ex:Sketch-Al}
  \ea
\begingl
\glpreamble sÿki – nÿsÿkine\\
\gla sÿki nÿ-sÿki-ne\\
\glb basket 1\textsc{sg}-basket-\textsc{possd}\\
\glft ‘basket – my basket’\\
\endgl
  \ex
\begingl
\glpreamble yumaji – niyumaji\\
\gla yumaji ni-yumaji\\
\glb hammock 1\textsc{sg}-hammock\\
\glft ‘hammock – my hammock’\\
\endgl
%\trailingcitation{[]}
\z
\xe


Non-possessable\is{non-possessability} nouns cannot take a person marker,\is{person marking|)} however, if semantically possible, possession can be expressed nonetheless by use of a possessable \isi{relational noun} in \isi{juxtaposition} with the non-possessable one, see (\getref{ex:Sketch-Nonposs}). The relational noun always precedes the non-possessable noun.

\ea\label{ex:Sketch-Nonposs}
\begingl
\glpreamble nipeu kabe\\
\gla ni-peu kabe\\
\glb 1\textsc{sg}-animal dog\\
\glft ‘my dog’\\
\endgl
%\trailingcitation{[]}
\xe

\hspace*{-4.2pt}Animals constitute the most important semantic type of non-possessable nouns, thus \textit{-peu} ‘domestic animal’ is very frequently found as a relational noun. Another one is the general relational noun (\textsc{grn}) \textit{-yae}, which is very unspecific in meaning and has cognates in related languages. The same form \textit{-yae} is also used as a \isi{locative marker} (see below). For all kinds of possession marking applies that first and second person possessors are solely indexed by the person marker. If the possessor is a third person, a noun specifying the possessor can additionally follow the possessed noun. Regarding the non-possessables, there is usually either a \isi{relational noun} + possessor noun or a relational noun + non-possessable noun only.
\is{possession|)}

Non-singularity can be expressed by a plural,\is{plural|(} distributive\is{distributive|(} or \isi{collective} marker. All of them are also found on verbs and other parts of speech. Regarding nouns,\is{animacy|(} the plural marker is mainly added to human ones and is obligatory there. There is also a small number of inanimate nouns that optionally take the plural marker, many of them borrowed\is{borrowing} from Spanish, such as \textit{anyo} ‘year’. The second marker is \textit{-jane} and it is called “distributive marker” in this work because of its supposed origin as a real distributive marker. Nonetheless, today it almost exclusively acts as a plural marker for non-human referents. The distributive marker is always optional and mainly used with nouns referring to mammals, but also with other animate and inanimate nouns. (\getref{ex:Sketch-PL}) shows the plural marker on a human and the distributive marker on a non-human noun.

\ea\label{ex:Sketch-PL}
  \ea
\begingl
\glpreamble aitubuchepÿinube\\
\gla aitubuchepÿi-nube\\
\glb boy-\textsc{pl}\\
\glft ‘boys’\\
\endgl
  \ex
\begingl
\glpreamble kabejane\\
\gla kabe-jane\\
\glb dog-\textsc{distr}\\
\glft ‘dogs’\\
\endgl
%\trailingcitation{[]}
\z
\xe
\is{animacy|)}


\hspace*{-1.1pt}Both plural and distributive marker can also complement a third person marker\is{person marking} on nouns so that there may be ambiguity as to whether the possessor or the possessed is the non-singular referent, see (\getref{ex:Sketch-PL-Ambig}).\is{distributive|)}\is{plural|)}

\ea\label{ex:Sketch-PL-Ambig}
\begingl
\glpreamble chijinepÿinube\\
\gla chi-jinepÿi-nube\\
\glb 3-daughter-\textsc{pl}\\
\glft ‘her daughters’\\or: ‘their daughter’\\or: ‘their daughters’\\
\endgl
%\trailingcitation{[]}
\xe

The third non-singularity marker has the form \textit{-ji} and is analysed as a \isi{collective}. It attaches to nouns referring to items that come in uncountable groups like hair (see (\getref{ex:Sketch-Inal}) above) or swarms like some fish. In addition, the marker is also added to kinship terms and a few other human nouns\is{animacy} where it precedes the \isi{plural} marker. In this respect, (\getref{ex:Sketch-PL-Ambig}) above is an exception and (\getref{ex:Sketch-COL-PL}) more typical:

\ea\label{ex:Sketch-COL-PL}
\begingl
\glpreamble nichechajinube\\
\gla ni-checha-ji-nube\\
\glb 1\textsc{sg}-son-\textsc{col}-\textsc{pl}\\
\glft ‘my children’\\
\endgl
%\trailingcitation{[]}
\xe

Reality status\is{reality status|(} is \textit{the} inflectional\is{inflection} category for verbs (see §\ref{sec:O_Verbs} below), and it is also found on nouns.\is{nominal irrealis|(} There is a separate irrealis marker \textit{-ina}\is{non-verbal irrealis marker} for all words that do not belong to the class of verbs. This marker plays a role in \isi{non-verbal predication}, but it can also be used referentially on nouns. In this case, it either indicates that something did not come into existence despite the strong expectation that it would (i.e. negative reference) as in (\getref{ex:Sketch-NIRR-neg}) or that it has not come into existence yet (i.e. \isi{future reference}) as in (\getref{ex:Sketch-NIRR-fut}).

\ea\label{ex:Sketch-NIRR-neg}
\begingl
\glpreamble nikasuneina\\
\gla ni-kasune-ina\\
\glb 1\textsc{sg}-trousers-\textsc{irr.nv}\\
\glft ‘my supposed pair of trousers (that I should have received, but did not receive)’\\
\endgl
%\trailingcitation{[]}
\xe

\ea\label{ex:Sketch-NIRR-fut}
\begingl
\glpreamble chubiunubeina\\
\gla chÿ-ubiu-nube-ina\\
\glb 3-house-\textsc{pl}-\textsc{irr.nv}\\
\glft ‘their future house’\\
\endgl
%\trailingcitation{[]}
\xe
\is{nominal irrealis|)}
\is{reality status|)}

Three markers are used to indicate that a person has passed away.\is{deceased marking|(} The general remote\is{remote past} (past) marker \textit{-bane}, which is also used to posit an event in the remote past, mainly occurs with referential kinship terms, the marker \textit{-ini} is predominantly used with the \isi{endearment} forms, and \textit{-kue} with proper names:

\ea\label{ex:Sketch-Deceased}
  \ea
\begingl
\glpreamble nÿabane\\
\gla nÿ-a-bane\\
\glb 1\textsc{sg}-father-\textsc{rem}\\
\glft ‘my late father’\\
\endgl
  \ex
\begingl
\glpreamble chÿchÿini\\
\gla chÿchÿ-ini\\
\glb grandpa-\textsc{dec}\\
\glft ‘late grandpa’\\
\endgl
  \ex
\begingl
\glpreamble Tubusiukue\\
\gla Tubusiu-kue\\
\glb Tiburcio-\textsc{dec.pn}\\
\glft ‘late Tiburcio’\\
\endgl
%\trailingcitation{[]}
\z
\xe
\is{deceased marking|)}

The diminutive marker\is{diminutive|(} is \textit{-mÿnÿ}. It can indicate smallness of a nominal referent as can be seen in (\getref{ex:Sketch-Dim-N}), but also empathy, positive affection, modesty or self-pity. If used with a verbal\is{verb} or non-verbal predicate,\is{non-verbal predication} it can apply these values to a subject or object referent\is{argument} or attenuate the meaning of the word itself, which is often hard to distinguish. If the noun in question is possessed, the diminutive can also relate to the possessor instead of the possessed as in (\getref{ex:Sketch-Dim-N-2}).

\ea\label{ex:Sketch-Dim-N}
\begingl
\glpreamble peÿjanemÿnÿ\\
\gla peÿ-jane-mÿnÿ\\
\glb frog-\textsc{distr}-\textsc{dim}\\
\glft ‘little frogs, baby frogs’\\
\endgl
%\trailingcitation{[]}
\xe

\ea\label{ex:Sketch-Dim-N-2}
\begingl
\glpreamble chibastunemÿnÿ\\
\gla chi-bastun-ne-mÿnÿ\\
\glb 3-cane-\textsc{possd}-\textsc{dim}\\
\glft ‘her walking cane (of the nice old lady)’\\
\endgl
%\trailingcitation{[]}
\xe
\is{diminutive|)}

Paunaka has one locative marker\is{locative marker|(}. It has the same form \textit{-yae} (in rapid speech often \textit{-ye} or \textit{-ya}) as the general \isi{relational noun} used in \isi{possession} marking of non-possessable nouns (see above). The \isi{locative} marker is used with expressions of place and goal, as in (\getref{ex:Sketch-LOC-PlGo}). It can also occur in source expressions, but these require a \isi{source} preposition and the locative marker can be considered optional. The marker is often absent from toponyms.\is{toponym} 

\ea\label{ex:Sketch-LOC-PlGo}
  \ea
\begingl
\glpreamble kaku pisaneyae\\
\gla kaku pi-sane-yae\\
\glb exist 2\textsc{sg}-field-\textit{loc}\\
\glft ‘it is on your field’\\
\endgl
  \ex
\begingl
\glpreamble piyunu pisaneyae\\
\gla pi-yunu pi-sane-yae\\
\glb 2\textsc{sg}-go 2\textsc{sg}-field\\
\glft ‘you go to your field’\\
\endgl
%\trailingcitation{[]}
\z
\xe


The locative marker can be used with different configurations of ground and figure and always expresses the most prototypical or expected one (i.e. it can be translated as ‘in’, ‘at’, ‘on’ depending on the nature and constitution of figure and ground).

In order to be more precise, speakers make use of two general patterns. For any relation that is about containment, they add the \isi{classifier} \textit{-kÿ} ‘\textsc{clf:}bounded’ or the (probably nominal) locative stem \textit{-j(ÿ)ekÿ} ‘inside’ to the noun. These may then be followed by the locative marker, see (\getref{ex:Sketch-inside-Loc}).

\ea\label{ex:Sketch-inside-Loc}
  \ea
\begingl
\glpreamble sÿkikÿyae\\
\gla sÿki-kÿ-yae\\
\glb basket-\textsc{clf:}bounded-\textsc{loc}\\
\glft ‘in the basket’\\
\endgl
  \ex
\begingl
\glpreamble chiyikijÿekÿyae\\
\gla chiyiki-jÿekÿ-yae\\
\glb hill-inside-\textsc{loc}\\
\glft ‘inside of the hill’\\
\endgl
%\trailingcitation{[]}
\z
\xe

The other possibility is to use one of four spatial relational nouns\is{relational noun|(} in \isi{juxtaposition} to the noun denoting the ground. These relational nouns are given in \tabref{table:Sketch-noun-stems-locative}. All of them are inalienably possessed,\is{inalienability} i.e. they require a person marker. Except for \textit{-akene/\--ekene} ‘non-visible side’, they usually take the locative marker, one example being (\getref{ex:Sketch-under}).

\begin{table}[htbp]
\caption{Locative relational noun stems}

\begin{tabular}{ll}
\lsptoprule
Relational noun & Translation \\
\midrule
\textit{-akene/-ekene} & non-visible side (behind, beside)\\
\textit{-chuku} & side (next to, close to)\\
\textit{-(i)ne} & top, place on top or above\\
\textit{-upekÿ} & place under\\
\lspbottomrule
\end{tabular}

\label{table:Sketch-noun-stems-locative}
\end{table}

\ea\label{ex:Sketch-under}
\begingl
\glpreamble chÿupekÿye echÿu ame\\
\gla chÿ-upekÿ-yae echÿu ame\\
\glb 3-place.under-\textsc{loc} \textsc{dem}b motacú\\
\glft ‘under a \textit{motacú} palm’\\
\endgl
%\trailingcitation{[]}
\xe
\is{relational noun|)}
\is{locative marker|)}
\is{noun|)}

\is{noun phrase|(}
The \isi{word order} inside an NP is summarised in \figref{fig:Sketch-NP}. Except for demonstratives,\is{nominal demonstrative} modifiers\is{modification} of nouns are not very frequent in Paunaka. The status of quantifiers acting as modifiers is not totally clear from the data, which is why “Q” is given in parentheses here. As for ADJ\textsuperscript{2}, an adjective following a noun can possibly best be analysed as a relative clause, so that this is given in parentheses, too.


\begin{figure}
\begin{tabularx}{\textwidth}{rQCCCCCCCCQl}
{${\Biggl [}$} & {(Q)} & {DEM} & {‘other’} & {NUM} & {ADJ\textsuperscript{1}} & {N} & [{DEM}  & {N\textsubscript{poss}}] & {N\textsubscript{type}} & {\multirow{2}{*}{\shortstack[c]{RC\\ (ADJ\textsuperscript{2})}}} & {${\Biggl ]}$}\\
 & & &  & & & & & & &  &\\
& & & & & & \textit{head} & & & & & \\
\end{tabularx}
\captionof{figure}{Word order in the NP}
\label{fig:Sketch-NP}
\end{figure}

\is{noun phrase|)}

More precise information on nominal morphology and the NP can be found in \chapref{chapter:Nouns}.

\section{Verbs}\label{sec:O_Verbs}
\is{verb|(}

Verbs can be grouped into two main classes: stative\is{stative verb|(} and active.\is{active verb|(}  Both index the \isi{subject} by a person marker\is{person marking} preceding the stem, unlike other Arawakan languages \citep[cf.][86]{Aikhenvald1999}, but the two classes are easily distinguished by a different slot for irrealis\is{irrealis|(} marking. While a \isi{prefix} \textit{a-} precedes the stem of a stative verb, \textit{a} always follows the verb stem in active verbs, see (\getref{ex:Sketch-Stat}) for an irrealis stative and (\getref{ex:Sketch-Act}) for an irrealis active verb. Note that there is incipient \isi{fusion} of irrealis marking with a restricted number of other markers on active verbs, thus irrealis \textit{a} is usually not given as a separate marker \textit{-a} throughout the grammar.

\newpage
\ea\label{ex:Sketch-Stat}
\begingl
\glpreamble kuina tajimama\\
\gla kuina ti-a-jimama\\
\glb \textsc{neg} 3i-\textsc{irr}-be.strong\\
\glft ‘he is not strong’\\
\endgl
%\trailingcitation{[]}
\xe

\ea\label{ex:Sketch-Act}
\begingl
\glpreamble kuina tiyuna\\
\gla kuina ti-yuna\\
\glb \textsc{neg} 3i-go.\textsc{irr}\\
\glft ‘she didn’t go’\\
\endgl
%\trailingcitation{[]}
\xe
\is{irrealis|)}

Stative and active verb stems\is{verbal stem} have partly different possibilities of \isi{derivation}. Many active verb stems end in a thematic suffix\is{thematic suffix|(} \textit{-ku} or \textit{-chu} (which may be deleted or replaced by another marker under some circumstances), while stative verbs never take a thematic suffix.\is{thematic suffix|)} However, processes like \isi{reduplication} and insertion of classifiers\is{classifier} or \isi{incorporation} are found with both stative and active verbs. Among the latter, many are ambitransitive,\is{ambitransitivity} i.e. they can be used transitively and intransitively. In addition, \isi{causative}, \isi{benefactive}, and \isi{reciprocal} derivations\is{derivation} can alter the \isi{valency} of an active verb, but these processes are not very productive and/or frequent.\is{active verb|)}\is{stative verb|)}

The most important inflectional\is{inflection} processes on verbs are person (and number)\is{person marking|(} marking as well as \isi{reality status}. Both are obligatory. Person and number marking is achieved by the same set of person and number markers also found on nouns, in addition, there is a third person \isi{prefix} \textit{ti-} that only occurs on verbs, see \tabref{table:Sketch-PersonMarkers} in §\ref{sec:O_GeneralRemarksMorphology} above for the forms of the person markers.

For indexing first and second persons holds that markers preceding the verb stem index subjects\is{subject} and markers following the stem index objects\is{object} as in (\getref{ex:Sketch-PersVerb}).

\ea\label{ex:Sketch-PersVerb}
\begingl
\glpreamble nikichupapi\\
\gla ni-kichupa-pi\\
\glb 1\textsc{sg}-wait.\textsc{irr}-2\textsc{sg}\\
\glft ‘I will wait for you’\\
\endgl
%\trailingcitation{[]}
\xe

As regards the third person, there are two different markers which both precede the stem. Among them, \textit{chÿ-} can only be used, if there is a third person \isi{subject} and a third person \isi{object}. In this case, it is obligatory with human objects\is{animacy} and optional with non-human objects. (\getref{ex:Sketch-3transHUM}) shows the use of \textit{chÿ-} with a verb having a third person subject and a third person human object.

\newpage
\ea\label{ex:Sketch-3transHUM}
\begingl
\glpreamble chakachu chÿenu\\
\gla chÿ-akachu chÿ-enu\\
\glb 3-lift 3-mother\\
\glft ‘he lifted his mother’\\
\endgl
%\trailingcitation{[]}
\xe

The other marker, \textit{ti-}, is used in all other cases, e.g. with \isi{intransitive} verbs and \isi{transitive} verbs with an SAP \isi{object}. It can also occur if the third person object is non-human.\is{animacy} (\getref{ex:Sketch-3i}) shows different uses of \textit{ti-}: with an intransitive verb, with a first person object, and with a third person non-human object.

\ea\label{ex:Sketch-3i}
  \ea
\begingl
\glpreamble titupunubu\\
\gla ti-tupunubu\\
\glb 3i-arrive\\
\glft ‘she arrived’\\
\endgl
  \ex
\begingl
\glpreamble tinijabakunÿ\\
\gla ti-nijabaku-nÿ\\
\glb 3i-bite-1\textsc{sg}\\
\glft ‘it bites me’\\
\endgl
  \ex
\begingl
\glpreamble tiniku yui\\
\gla ti-niku yui\\
\glb 3i-eat bread\\
\glft ‘it eats bread’\\
\endgl
%\trailingcitation{[]}
\z
\xe

In addition to \textit{ti-} and \textit{chÿ-}, which both precede the verb stem, there is also \textit{-chÿ}, which follows stems of verbs (and words of other classes), but is restricted to a few specific contexts, most importantly the \isi{speech verb} \textit{-kechu} ‘say’ that introduces or closes up reported speech.

The marker \textit{chÿ-} is glossed as ‘3’ and not as ‘3<3’ in order to use one and the same gloss for its occurrence on nouns and on verbs. In order to clearly distinguish \textit{ti-}, I decided to use a different gloss ‘3i’. The letter ‘i’ is thus indeed meaningless but derives from the very first assumption that this was a person marker used with intransitive verbs only (which proved to be incorrect).

There is no dedicated third person plural marker.\is{plural|(} In order to express that one of the third person participants is non-singular, the plural marker \textit{-nube}, \isi{distributive} marker \-\textit{-jane} and \isi{collective} marker \textit{-ji} can be used. The first one, \textit{-nube}, is obligatory with human\is{animacy} third person plural participants, regardless of their status as \isi{subject} or \isi{object} of the verb. If only subject or only object has a third person referent, it is quite clear to whom the marker refers, see (\getref{ex:Sketch-3PLSUBJ}) with a third person plural subject and (\getref{ex:Sketch-3PLOBJ}) with a third person plural object. 

\ea\label{ex:Sketch-3PLSUBJ}
\begingl
\glpreamble tiyÿsebÿkeunÿnube\\
\gla ti-yÿsebÿkeu-nÿ-nube\\
\glb 3i-ask-1\textsc{sg}-\textsc{pl}\\
\glft ‘they asked me’\\
\endgl
%\trailingcitation{[]}
\xe

\ea\label{ex:Sketch-3PLOBJ}
\begingl
\glpreamble peneikunubetu\\
\gla pi-eneiku-nube-tu\\
\glb 2\textsc{sg}-leave-\textsc{pl}-\textsc{iam}\\
\glft ‘you have left them now’\\
\endgl
%\trailingcitation{[]}
\xe


However, if both subject and object have third person referents, the assignment of the plural marker is ambiguous, and only context can clarify which of the participants has plural number, as in (\getref{ex:Sketch-3PLAMB}).

\ea\label{ex:Sketch-3PLAMB}
\begingl
\glpreamble chimunube\\
\gla chi-imu-nube\\
\glb 3-see-\textsc{pl}\\
\glft ‘they see her’\\or: ‘she sees them’\\or: ‘they see them’\\
\endgl
%\trailingcitation{[]}
\xe
\is{plural|)}

The same issue also holds for the \isi{distributive} marker \textit{-jane} with the difference that this one is used for non-human referents and is always optional. It usually only occurs on a verb if the referent is animate\is{animacy|(} (with a few exceptions), while it may well occur on inanimate nouns\is{noun} – although the chance that it is attached to nouns is also higher for animate than for inanimate referents.\is{animacy|)}

Finally, the \isi{collective} marker \textit{-ji} occurs almost exclusively with stative verbs\is{stative verb|(} that encode properties. It thus always refers to the \isi{subject} participant, which has to be an entity that occurs in masses or in a swarm.\is{person marking|)}

Apart from person and number marking, the second obligatory category in Paunaka’s verbal morphology is reality status \is{reality status|(}. As has been stated above, there are differences between stative and active verbs in this regard, actually place of irrealis marking is the (most) decisive factor in distinguishing these two classes.

Stative verbs are unmarked in realis and receive a prefix \textit{a-} in irrealis, see (\getref{ex:Sketch-StatRS}).

\ea\label{ex:Sketch-StatRS}
  \ea
\begingl
\glpreamble tiyutu\\
\gla ti-yu-tu\\
\glb 3i-be.ripe-\textsc{iam}\\
\glft ‘it is ripe’\\
\endgl
  \ex
\begingl
\glpreamble tayutu\\
\gla ti-a-yu-tu\\
\glb 3i-\textsc{irr}-be.ripe-\textsc{iam}\\
\glft ‘it will be ripe’\\
\endgl
%\trailingcitation{[]}
\z
\xe\is{stative verb|)}

The issue is more complicated as regards active verbs.\is{active verb|(} Usually, active verbs end in \textit{u} in realis and in \textit{a} in irrealis as in (\getref{ex:Sketch-ActRS}).

\ea\label{ex:Sketch-ActRS}
  \ea
\begingl
\glpreamble niniku\\
\gla ni-niku\\
\glb 1\textsc{sg}-eat\\
\glft ‘I eat/ate’\\
\endgl
  \ex
\begingl
\glpreamble ninika\\
\gla ni-nika\\
\glb 1\textsc{sg}-eat.\textsc{irr}\\
\glft ‘I will/must/may eat’\\
\endgl
%\trailingcitation{[]}
\z
\xe

However, there are a few markers that shift the place of irrealis marking, which would cause an ugly mismatch between realis and irrealis marking if I considered \textit{u} and \textit{a} to be proper suffixes\is{suffix|(} (\textit{-u} and \textit{-a}). There are several possible analyses to solve this issue, all of them with some advantages and disadvantages (described in detail in §\ref{sec:VerbalRS}). My final decision was to consider \textit{u} as the default ending of the verb stem\is{verbal stem|(} and certain markers and \textit{a} as an irrealis marker fused with the stem or the markers in question. This works well in most cases, but there are also some occasions where \textit{u} and \textit{a} do indeed occur as independent suffixes and are thus glossed as such as when following the \isi{distributive} marker that has been inserted in the slot of the \isi{thematic suffix} (\textit{-ku} or \textit{-ka} respectively) of the active verb stem, see (\getref{ex:Sketch-jane-RS}).\is{active verb|(}\is{suffix|)}

\ea\label{ex:Sketch-jane-RS}
  \ea
\begingl
\glpreamble tinijaneu\\
\gla ti-ni-jane-u\\
\glb 3i-eat-\textsc{distr}-\textsc{real}\\
\glft ‘they eat/ate’\\
\endgl
  \ex
\begingl
\glpreamble tinijanea\\
\gla ti-ni-jane-a\\
\glb 3i-eat-\textsc{distr}-\textsc{irr}\\
\glft ‘they will/must/may eat’\\
\z
\endgl
%\trailingcitation{[]}
\xe
\is{verbal stem|)}

As regards semantics, Paunaka’s reality status system can be considered canonical. It follows the theoretical outline proposed by \citet[]{Elliott2000} and \citet[]{Michael2014} to a large degree. Realis is found in the expression of factual, realised events, irrealis in non-factual, non-realised events, which comprises future time reference,\is{future reference} \isi{negation}, hypotheticality, epistemic \isi{modality}, and speaker- and agent-oriented \isi{modality}. Some of these parameters are expressed in the possible translations of (\getref{ex:Sketch-ActRS}) and (\getref{ex:Sketch-jane-RS}) above, irrealis triggered by negation was exemplified in (\getref{ex:Sketch-Stat}) and (\getref{ex:Sketch-Act}).

Reality status interacts with the semantics of other markers and some com\-ple\-ment-taking verbs, i.e. if a specific notion that belongs to the realm of non-factiveness is already expressed by other material, reality status may be used in some cases to convey information about other notions of (un)realness. This is the case in the two examples given in (\getref{ex:Sketch-kena-RS}), where the \isi{modality} marker \textit{-kena} provides information about the notion of \isi{uncertainty} and reality status thus indicates past\is{past reference} and future time reference,\is{future reference} respectively.

\ea\label{ex:Sketch-kena-RS}
  \ea
\begingl
\glpreamble teukena\\
\gla ti-eu-kena\\
\glb 3i-drink-\textsc{uncert}\\
\glft ‘maybe he has drunk’\\
\endgl
  \ex
\begingl
\glpreamble tikebakena\\
\gla ti-keba-kena\\
\glb 3i-rain.\textsc{irr}-\textsc{uncert}\\
\glft ‘maybe it is going to rain’\\
\endgl
%\trailingcitation{[]}
\z
\xe

Having said this, it important to know that Paunaka does not exhibit a \isi{doubly irrealis construction} in the context of standard \isi{negation} unlike the related Kampan languages \citep[cf.][]{Michael2014,Michael2014a}, that is with the exception of prohibitives,\is{directive speech act!prohibitive}\is{negation!prohibitive} all negated verbs inflect for irrealis in Paunaka.
\is{reality status|)}

Associated motion\is{associated motion|(} is a grammatical category that encodes motion in relation to the event expressed by the (non-motion) verb. It is relatively widespread in the languages of South America \citep[cf.][]{Guillaume2016}. Three markers can definitely be classified as associated motion markers in Paunaka: \textit{-kÿu} encoding concurrent translocative motion, \textit{-kÿupunu} encoding concurrent cislocative motion, and \textit{-punu} encoding prior motion. In addition, a fourth marker \textit{-nÿmu} possibly encodes subsequent motion, but is not used productively anymore by the speakers. All of them are shown in (\getref{ex:Sketch-AM}).

\newpage
\ea\label{ex:Sketch-AM}
  \ea
\begingl
\glpreamble ninikukukÿu\\
\gla ni-niku-kukÿu\\
\glb 1\textsc{sg}-eat-\textsc{am.con.tr}\\
\glft ‘I go eating’\\
\endgl
  \ex
\begingl
\glpreamble pipÿsisikÿupunu\\
\gla pi-pÿsisi-kÿupunu\\
\glb 2\textsc{sg}-be.alone-\textsc{am.conc.cis}\\
\glft ‘you came alone’\\
\endgl
  \ex
\begingl
\glpreamble chebÿpekupuna\\
\gla chÿ-ebÿpeku-puna\\
\glb 3-borrow.money-\textsc{am.prior.irr}\\
\glft ‘she will go and borrow money’\\
\endgl
  \ex
\begingl
\glpreamble timukunÿmunube\\
\gla ti-muku-nÿmu-nube\\
\glb 3i-sleep-\textsc{am.subs?}-\textsc{pl}\\
\glft ‘they slept (and went?)’\\
\endgl
\z
\xe

Related to the category of associated motion\is{associated motion|)} is the \isi{dislocative} marker, which does not include motion by itself. It can add a path component to a non-motion verb, but most importantly marks the purpose verb in the \isi{motion-cum-purpose construction} as in (\getref{ex:Sketch-MCPC}).

\ea\label{ex:Sketch-MCPC}
\begingl
\glpreamble niyunu ninÿupu\\
\gla ni-yunu ni-nÿu-pu\\
\glb 1\textsc{sg}-go 1\textsc{sg}-lie.in.wait-\textsc{dloc}\\
\glft ‘I went to lie in wait (for animals)’\\
\endgl
%\trailingcitation{[]}
\xe

Another related morpheme is the regressive and repetitive\is{regressive/repetitive} marker \textit{-punuku} (and its several allomorphs). It clearly derives\is{derivation} from the prior motion marker \textit{-punu}\is{associated motion} and expresses motion back to a point of origin on motion verbs\is{motion predicate} and repeated action on non-motion verbs, as can be seen in (\getref{ex:Sketch-regressive}).

\ea\label{ex:Sketch-regressive}
  \ea
\begingl
\glpreamble tiyunupunuka Alemania\\
\gla ti-yunu-punuka Alemania\\
\glb 3i-go-\textsc{reg.irr} Germany\\
\glft ‘she will go back to Germany’\\
\endgl
\newpage
  \ex
\begingl
\glpreamble beupupunuka\\
\gla bi-eu-pupunuka\\
\glb 1\textsc{pl}-drink-\textsc{reg.irr}\\
\glft ‘let’s drink again!'\\
\endgl
%\trailingcitation{[]}
\z
\xe

Paunaka has a middle\is{middle voice|(} marker which has the form \textit{-bu} after \isi{realis} and \textit{-pu} after \isi{irrealis} marking. There is a number of deponent middle verbs, i.e. verbs that never occur without the middle marker, and an even higher number of verbs that have a non-middle form, but whose middle form is at least as frequent as the non-middle form. These are contexts where \isi{lexicalisation} is at work. In other contexts, the middle marker usually expresses either anticausativity\is{anticausative} or reflexivity\is{reflexive} (including “direct”, “indirect” and “body-part” reflexives, \citealp[cf.][]{Kemmer1993}) as is typical for middle marking, see (\getref{ex:Sketch-Middle})

\ea\label{ex:Sketch-Middle}
  \ea
\begingl
\glpreamble tijekupubu\\
\gla ti-jekupu-bu\\
\glb 3i-lose-\textsc{mid}\\
\glft ‘they get lost’\\
\endgl
  \ex
\begingl
\glpreamble netukikapu\\
\gla nÿ-etu-ki-ka-pu\\
\glb 1\textsc{sg}-put-\textsc{clf:}spherical-\textsc{th}1.\textsc{irr}-\textsc{mid}\\
\glft ‘I'm going to put it on my head’\\
\endgl
%\trailingcitation{[]}
\z
\xe
\is{middle voice|)}

Paunaka has five different \isi{aspect}, two \isi{tense}, six \isi{modality}, and one \isi{evidentiality} markers. All of them do not only attach to verbs, but to words of other classes as well, with the exception of the \isi{uncertain future} marker, which is a free particle that is not phonologically bound to a preceding word at all.

\tabref{table:Sketch-TAMEmarkers} gives an overview about the TAME markers.\is{tense}\is{aspect}\is{modality}\is{evidentiality}

\begin{table}[htbp]
\caption{TAME markers}
\small
\begin{tabularx}{\textwidth}{llllQ}
\lsptoprule
Category & Name & Marker & Gloss & Rough translation \\
\midrule
Aspect & Iamitive (perfect) & \textit{-tu} & \textsc{iam} & already, now \\
 & Discontinuous & \textit{-bu} & \textsc{dsc} & (not) anymore \\
 & Incompletive & \textit{-kuÿ} & \textsc{incmp} & still, (not) yet \\
 & Prospective & \textit{-bÿti} & \textsc{prsp} & be about to, starting, first \\
 & Continuous & \textit{-CViku} & \textsc{cont} & be ongoing\\
Tense & Remote (past) & \textit{(-)bane} & \textsc{rem} & long ago, away \\
 & Uncertain future & \textit{uchu} & \textsc{uncert.fut} & one day\\
Modality & Frustrative & \textit{-ini} & \textsc{frust} & in vain, would X\\
 & Avertive & \textit{-tÿini} & \textsc{avert} & almost\\
 & Optative & \textit{-yuini} & \textsc{opt}1 & hopefully, if only, may\\
 & Optative & \textit{-jÿti} & \textsc{opt}2 & hopefully, if only, may\\
 & Uncertainty & \textit{(-)kena} & \textsc{uncert} & maybe\\
 & Deductive & \textit{-yenu} & \textsc{ded} & must be X\\
Evidentiality & Reportive & \textit{-ji} & \textsc{rprt} & it is said\\
\lspbottomrule
\end{tabularx}

\label{table:Sketch-TAMEmarkers}
\end{table}

All aspect\is{aspect|(} markers interact with event boundaries. Among them, the iamitive\is{iamitive|(} is the most frequent. Iamitive is a gram type related to the better-known perfect, but with differences concerning the interaction with \isi{aktionsart} \citep[cf.][]{Olsson2013}. In Paunaka, the iamitive expresses on stative verbs\is{stative verb|(} that the state is the result of a previous process that has reached its endpoint.\is{resultative} On active telic\is{telicity|(} verbs,\is{active verb|(} it triggers the state after the final boundary. On atelic verbs, however, it introduces a boundary, which may be the initial or final boundary of the event, and then refers to the time after this boundary, thus evoking an ongoing or completed interpretation depending on the context. Consider (\getref{ex:Sketch-IAM}) with a stative, an active telic and an active atelic verb together with the possible translations for illustration.\is{active verb|)}\is{telicity|)}\is{stative verb|)}

\newpage
\ea\label{ex:Sketch-IAM}
  \ea
\begingl
\glpreamble tiyutu\\
\gla ti-yu-tu\\
\glb 3i-be.ripe-\textsc{iam}\\
\glft ‘it is ripe (now/already)’\\
\endgl
  \ex
\begingl
\glpreamble tipakutu\\
\gla ti-paku-tu\\
\glb 3i-die-\textsc{iam}\\
\glft ‘she died/is dead’\\
\endgl
  \ex
\begingl
\glpreamble tikutijikutu\\
\gla ti-kutijiku-tu\\
\glb 3i-flee-\textsc{iam}\\
\glft ‘she is fleeing’\\or: ‘she has escaped’\\
\endgl
%\trailingcitation{[]}
\z
\xe

Iamitive does not occur on negated verbs.\is{negation}\is{iamitive|)} Instead of this, the \isi{discontinuous} marker \textit{-bu} with the meaning ‘(not) anymore‘ occurs in negative clauses like (\getref{ex:Sketch-DSC}).

\newpage
\ea\label{ex:Sketch-DSC}
\begingl
\glpreamble kuina nakuesanebu\\
\gla kuina nÿ-a-kuesane-bu\\
\glb \textsc{neg} 1\textsc{sg}-\textsc{irr}-have.field-\textsc{dsc}\\
\glft ‘I don’t have a field anymore’\\
\endgl
%\trailingcitation{[]}
\xe

The \isi{incompletive} marker \textit{-kuÿ} occurs in positive and negative clauses, where it exhibits the meanings ‘still’ and ‘(not) yet’ respectively, see (\getref{ex:Sketch-INCMP}).

\ea\label{ex:Sketch-INCMP}
  \ea
\begingl
\glpreamble tujikukuÿjaneyu\\
\gla ti-ujiku-kuÿ-jane-yu\\
\glb 3i-suckle-\textsc{incmp}-\textsc{distr}-\textsc{ints}\\
\glft ‘they still suckle a lot’\\
\endgl
  \ex
\begingl
\glpreamble kuina nichujikakuÿmÿnÿ\\
\gla kuina ni-chujika-kuÿ-mÿnÿ\\
\glb \textsc{neg} 1\textsc{sg}-speak.\textsc{irr}-\textsc{incmp}-\textsc{dim}\\
\glft ‘I did not speak yet’\\
\endgl
%\trailingcitation{[]}
\z
\xe 

The \isi{prospective} marker \textit{-bÿti} indicates that an event is imminent. Use of the marker often implies that an expected event follows the imminent event, as in (\getref{ex:Sketch-PRSP}), and it is thus also often used for temporal ordering, where it marks the first of a row of events.

\ea\label{ex:Sketch-PRSP}
\begingl
\glpreamble nipunakabÿti merÿ\\
\gla ni-punaka-bÿti merÿ\\
\glb 1\textsc{sg}-give.\textsc{irr}-\textsc{prsp} plantain\\
\glft ‘I’m just going to give her plantains (and then we can sit together and work)’\\
\endgl
%\trailingcitation{[]}
\xe

Finally, the \isi{continuous} marker consists of a reduplicated\is{reduplication} CV syllable followed by the sequence \textit{iku}. It operates in a transition zone between \isi{derivation} and \isi{inflection} dependent partly on whether it attaches to the \isi{verbal root} or to a stem.\is{verbal stem} (\getref{ex:Sketch-CONT}) is an example of its inflection-like use.

\ea\label{ex:Sketch-CONT}
\begingl
\glpreamble timajaikukuiku\\
\gla ti-majaiku-kuiku\\
\glb 3i-bark-\textsc{cont}\\
\glft ‘it is barking’\\
\endgl
%\trailingcitation{[]}
\xe
\is{aspect|)}

There are considerably less tense \is{tense|(} markers: \isi{remote past} is marked by \textit{(-)bane}, which is phonologically attached to other words most of the time, but may also occur as a free particle. If attached to human\is{animacy} nouns,\is{noun} it usually signals the deceased\is{deceased marking} state of the person in question (see §\ref{sec:O_Nouns}), otherwise it signals remote past reference of the clause. Uncertain future\is{uncertain future} is expressed by the free particle \textit{uchu}, which occurs very infrequently. (\getref{ex:Sketch-Tense}) provides illustration for the use of the tense markers.

\ea\label{ex:Sketch-Tense}
  \ea
\begingl
\glpreamble timesumeikunÿbane\\
\gla ti-mesumeiku-nÿ-bane\\
\glb 3i-teach-1\textsc{sg}-\textsc{rem}\\
\glft ‘she taught me long ago’\\
\endgl
  \ex
\begingl
\glpreamble nikichupapi uchu\\
\gla ni-kichupa-pi uchu\\
\glb 1\textsc{sg}-wait.\textsc{irr}-2\textsc{sg} \textsc{uncert.fut}\\
\glft ‘I will wait for you (whenever you may come back)’\\
\endgl
%\trailingcitation{[]}
\z
\xe
\is{tense|)}

Among the several modality\is{modality|(} markers, \isi{frustrative} is used in its canonic function to indicate that an action was carried out in vain, but even more often to express \isi{counterfactuality}. It also frequently attaches to the verb \textit{-sachu} ‘want’, if a wish was not or cannot be fulfilled, which is shown in (\getref{ex:Sketch-FRUST}).

\ea\label{ex:Sketch-FRUST}
\begingl
\glpreamble pisachuini pinikanÿ\\
\gla pi-sachu-ini pi-nika-nÿ\\
\glb 2\textsc{sg}-want-\textsc{frust} 2\textsc{sg}-eat.\textsc{irr}-1\textsc{sg}\\
\glft ‘you wanted to eat me (but did not succeed)’\\
\endgl
%\trailingcitation{[]}
\xe

%ti-niku-uka-ini = he would eat him as well (if he could)
%ni-yÿseika-ini = I would buy it (if I was still here)

The \isi{avertive} marker is very infrequent. It expresses that an event almost happened:

\ea\label{ex:Sketch-AVERT}
\begingl
\glpreamble ti-kupaka-ne-tÿini\\
\gla ti-kupaka-ne-tÿini\\
\glb 3i-kill.\textsc{irr}-1\textsc{sg}-\textsc{avert}\\
\glft ‘they (the mosquitos) almost killed me’\\
\endgl
%\trailingcitation{[]}
\xe

The two \isi{optative} markers \textit{-yuini} and \textit{-jÿti} occur even more seldom, almost all of the few examples I have in the corpus are elicited. On the contrary, the \isi{uncertainty} marker \textit{(-)kena}, which – similar to the remote marker remote past \textit{(-)bane} – is usually phonologically bound to another word but can also occur as a free particle, is very frequent. Two illustrative examples have already been given in (\getref{ex:Sketch-kena-RS}) above. Finally, there is \textit{-yenu} among the modality markers to express that a finding is based on a deduction,\is{deductive} as in (\getref{ex:Sketch-DED}). Thus, in contrast to clauses containing \textit{-kena}, which are speculative, \textit{-yenu} signals that there is something that makes the speaker believe that the proposition is true, although the reason does not need to be verbalised.

\ea\label{ex:Sketch-DED}
\begingl
\glpreamble chisamuyenu paunaka\\
\gla chi-samu-yenu paunaka\\
\glb 3-hear-\textsc{ded} Paunaka\\
\glft ‘she must understand Paunaka’\\
\endgl
%\trailingcitation{[]}
\xe
\is{modality|)}

As for the category of evidentiality \is{evidentiality|(}, Paunaka possesses one reportive marker \textit{-ji} that indicates second-hand information, i.e. what the speaker is talking about was not experienced by herself, as in (\getref{ex:Sketch-RPRT}). 

\ea\label{ex:Sketch-RPRT}
\begingl
\glpreamble chisatÿkujitu chinachÿ chijabu\\
\gla chi-satÿku-ji-tu chinachÿ chi-jabu\\
\glb 3-cut-\textsc{rprt}-\textsc{iam} one 3-leg\\
\glft ‘he cut off one of his legs, it is said’\\
\endgl
%\trailingcitation{[]}
\xe

The reportive marker occurs a lot in narratives and in these narratives, it abounds on speech verbs\is{speech verb} introducing or closing up reported speech, presumably to explicitly mark the reproduced speech as not being self-experienced. Furthermore, \textit{-ji} is occasionally used as a \isi{quotative} marker, too.\is{evidentiality|)}

Paunaka has a number of degree markers\is{degree marker|(} which attach to predicates and other constituents of the clause. \tabref{table:Sketch-degreemarkers} provides an overview of the forms.

\begin{table}[htbp]
\caption{Degree markers}

\begin{tabular}{llll}
\lsptoprule
Name & Marker & Gloss & Rough translation \\
\midrule
Intensifier & \textit{-yu} & \textsc{ints} & very\\
Additive & \textit{-uku} & \textsc{add} & also, too\\
Limitative & \textit{-jiku} & \textsc{lim}1 & only\\
Limitative & \textit{-yÿchi} & \textsc{lim}2 & just\\
Emphatic & \textit{-ja} / \textit{-ja’a} & \textsc{emph}1 & really\\
Emphatic & \textit{-kene} & \textsc{emph}2 & indeed\\
\lspbottomrule
\end{tabular}

\label{table:Sketch-degreemarkers}
\end{table}

The \isi{intensifier} states that a proposition holds to a large degree. It most often occurs with stative predicates, i.e. stative verbs\is{stative verb} as in (\getref{ex:Sketch-INTS}) or non-verbal predicates.\is{non-verbal predication}

\ea\label{ex:Sketch-INTS}
\begingl
\glpreamble tÿbaneyu\\
\gla ti-ÿbane-yu\\
\glb 3i-be.far-\textsc{ints}\\
\glft ‘it is very far’\\
\endgl
%\trailingcitation{[]}
\xe

The \isi{additive} marker signals an addition of a participant (as in (\getref{ex:Sketch-ADD})) or an action.

\ea\label{ex:Sketch-ADD}
\begingl
\glpreamble tiyunauku echÿu\\
\gla ti-yuna-uku echÿu\\
\glb 3i-go.\textsc{irr}-\textsc{add} \textsc{dem}b\\
\glft ‘he has to go, too’\\
\endgl
%\trailingcitation{[]}
\xe

There are two \isi{limitative} markers \textit{-jiku} and \textit{-yÿchi}, which both express delimitation or the fact that an event occurs without more ado, as can be seen in (\getref{ex:Sketch-LIM}).

\ea\label{ex:Sketch-LIM}
  \ea
\begingl
\glpreamble tebibikujiku kujipiyae\\
\gla ti-ebibiku-jiku kujipi-yae\\
\glb 3i-swing-\textsc{lim}1 liana.sp-\textsc{loc}\\
\glft ‘he only swung on the liana’\\
\endgl
  \ex
\begingl
\glpreamble niyunuyÿchi\\
\gla ni-yunu-yÿchi\\
\glb 1\textsc{sg}-go-\textsc{lim}2\\
\glft ‘I just went’\\
\endgl
%\trailingcitation{[]}
\z
\xe

There are also two emphatic markers.\is{emphatic} While \textit{-ja} (or \textit{-ja’a}) emphasises, stresses or particularly points out something, the rare marker \textit{-kene} is mostly used to establish \isi{subject} \isi{focus} or \isi{topicalisation}. Both markers can also occur together as in (\getref{ex:Sketch-EMPH}). In this case, the iamitive marker always follows as well.

\ea\label{ex:Sketch-EMPH}
\begingl
\glpreamble tipikunubekenejatu\\
\gla ti-piku-nube-kene-ja-tu\\
\glb 3i-be.afraid-\textsc{pl}-\textsc{emph}2-\textsc{emph}1-\textsc{iam}\\
\glft ‘they are afraid after all’\\
\endgl
%\trailingcitation{[]}
\xe
\is{degree marker|)}

More detailed information about morphology found with verbs is given in \chapref{sec:Verbs}.
\is{verb|)}

\section{Simple clauses}\label{sec:O_SimpleClauses}

Core arguments\is{argument} are indexed\is{person marking} on the \isi{verb} in Paunaka following nom\-i\-na\-tive-ac\-cusative \isi{alignment}, with some restrictions concerning non-human third person objects, see §\ref{sec:O_Verbs} above. Subject and object NPs are not required syntactically,\is{noun phrase} i.e. they are optional and can thus be considered conominals\is{conomination} \citep[cf.][]{Haspelmath2013}. They are not case-marked. Word order\is{word order|(} is quite flexible, but it is most common that the \isi{verb} precedes any other constituent, and objects\is{object} usually follow the verb directly. In most of the cases, there is only one conominal or none, thus the most common word orders are VS, VO and V. If both, subject and object are conominated,\is{conomination} we predominantly find VOS and SVO order.\is{subject}\is{object} Obliques\is{oblique} (X) usually follow the verb and object. There is one pre-verbal slot, which may be filled with S, O or X to indicate a special discourse status, either \isi{focus} or contrastive or changed \isi{topic}.\is{word order|)}

Standard negation\is{negation|(} is achieved by use of the \isi{negative particle} \textit{kuina}, which is placed before the predicate. Negated predicates always have \isi{irrealis} reality status as in (\getref{ex:Sketch-Negation}).

\ea\label{ex:Sketch-Negation}
\begingl
\glpreamble kuina nitupa echÿu bakajane\\
\gla kuina ni-tupa echÿu baka-jane\\
\glb \textsc{neg} 1\textsc{sg}-find.\textsc{irr} \textsc{dem}b cow-\textsc{distr}\\
\glft ‘I don’t find the cows’\\
\endgl
%\trailingcitation{[]}
\xe
\is{negation|)}

Non-verbal predication\is{non-verbal predication|(} is quite frequent in Paunaka. Different semantic types of non-verbal predication correlate with different construction types, among them \isi{juxtaposition} of predicate and subject, use of the non-verbal third person \isi{copula} \textit{kaku} and others. Non-verbal predicates can largely inflect\is{inflection} for the same categories as verbal predicates and largely use the same markers. The most important feature to set them apart from verbal predicates is the different locus for subject marking\is{person marking} (if applicable) – following the stem instead of preceding it – and a different irrealis marker\is{non-verbal irrealis marker} \textit{-ina}, see (\getref{ex:Sketch-NVPRED}) for both features.

\ea\label{ex:Sketch-NVPRED}
\begingl
\glpreamble kuina nÿenubina\\
\gla kuina nÿ-enu-bi-ina\\
\glb \textsc{neg} 1\textsc{sg}-mother-2\textsc{sg}-\textsc{irr.nv}\\
\glft ‘you are not my mother’\\
\endgl
%\trailingcitation{[jxx-p150920l.052]}
\xe

The topic of non-verbal predication in Paunaka is particularly interesting, because the most common expression for cislocative motion of a third person is a non-verbal predicate (\textit{kapunu} ‘come’),\is{motion predicate} and many verbs borrowed\is{borrowing} from Spanish are integrated into the language as non-verbal predicates, i.e. in clauses without verbs, see (\getref{ex:Sketch-NVPRED2}) for examples. Both of this is cross-linguistically relatively uncommon.

\ea\label{ex:Sketch-NVPRED2}
  \ea
\begingl
\glpreamble kapununube dose familia\\
\gla kapunu-nube dose familia\\
\glb come-\textsc{pl} twelve family\\
\glft ‘twelve families came’\\
\endgl
  \ex
\begingl
\glpreamble komoraubinatu\\
\gla komorau-bi-ina-tu\\
\glb accomodate-2\textsc{sg}-\textsc{irr.nv}-\textsc{iam}\\
\glft ‘you have to arrange (your stuff) now’\\
\endgl
%\trailingcitation{[]}
\z
\xe
\is{non-verbal predication|)}

Imperatives\is{directive speech act}\is{imperative} are most commonly construed by using an \isi{irrealis} verb with a second person subject index see (\getref{ex:Sketch-IMP}). Imperatives do not have any TAME marking. Intonation\is{intonation} and context set them apart from declarative clauses.

\ea\label{ex:Sketch-IMP}
\begingl
\glpreamble ¡pajanaba!\\
\gla pi-a-ja-naba\\
\glb 2\textsc{sg}-\textsc{irr}-open-mouth.inside\\
\glft ‘open your mouth!’\\
\endgl
%\trailingcitation{[jmx-n120429ls-x5.197]}
\xe

In addition, an imperative suffix \textit{-ji} may be attached to the verb to form an \isi{emphatic imperative}. There are several possibilities to form negative imperatives or prohibitives.\is{directive speech act!prohibitive}\is{negation!prohibitive} Speakers sometimes simply use the particle\is{negative particle} \textit{kuina} also found in standard negation together with an \isi{irrealis} verb, but there are also two other particles, prohibitive \textit{naka} and admonitive \textit{masaini}, that occur in negative imperatives. Both have been found with realis-\is{realis} and irrealis-marked\is{irrealis} predicates in the corpus in partly very similar contexts. Hortatives\is{hortative} build on the particle \textit{jaje}, which may be followed by a \isi{verb} with a first person plural subject.

As for interrogative clauses,\is{interrogative clause} polar questions\is{polar question} are solely set apart from declarative sentences by \isi{intonation}, while content questions build on different question words: \textit{chija} is used to ask for a \isi{subject}, \isi{object}, action, or identity, \textit{juchu(bu)} can be used to ask for location or time, \textit{(chi)kuyena} asks for manner or reason, and \textit{(u)kajane} for quantity. In addition, the \isi{uncertainty} marker \textit{kena} (in its free form) can be used to form generic questions that can best be translated with ‘what about X?’. The \isi{question word} is usually the first constituent of an \isi{interrogative clause}. Verbs in content questions\is{content question} may be finite\is{finite verb} as the one in (\getref{ex:Sketch-Qcont}), but there are also cases in which the question word combines with a relative clauses\is{relative relation} or deranked verbs\is{deranked verb} (see §\ref{sec:O_ComplexSentences} below for relative clauses and deranked verbs).

\ea\label{ex:Sketch-Qcont}
\begingl
\glpreamble ¿chija pichabubuikubu?\\
\gla chija pi-chabu-buiku-bu\\
\glb what 2\textsc{sg}-do-\textsc{mid}\\
\glft ‘what are you doing?’\\
\endgl
%\trailingcitation{[]}
\xe

A more complete description of simple clauses can be found in \chapref{chap:SimpleClauses}.

\section{Complex sentences}\label{sec:O_ComplexSentences}
\is{complex sentence|(}

A sentence may be complex because it contains several clauses or several predicates, i.e. Paunaka exhibits biclausal and monoclausal constructions that contain more than one predicate. There are different semantic types of complex sentences, which include \isi{coordination} on the one hand and \isi{subordination}, or more precisely adverbial,\is{adverbial relation} complement,\is{complement relation} and relative relations\is{relative relation} on the other hand. Construction types comprise asyndetic and syndetic\is{syndesis/asyndesis|(} \isi{juxtaposition} including finite verbs,\is{finite verb} \isi{dependency marking} on a verb or deranking of a verb,\is{deranked verb} with the latter type signifying the loss of verbal and gain of nominal characteristics. Deranked verbs contain the marker \textit{-i} ‘\textsc{subord}’ directly following the last consonant of the verb stem.\is{verbal stem} As for subordinate relations, the reality status \is{reality status|(} of the subordinate verb can be predetermined by the semantic type or by the construction type of some complex sentence. There is no neat correspondence between semantic type, construction type, predetermination of RS,\is{reality status|)} and bi- or monoclausality.

Coordination presupposes biclausality.\is{coordination|(} There are asyndetically and syndetically coordinated clauses,\is{syndesis/asyndesis} the latter type includes a \isi{connective}, e.g. the adversative connective \textit{pero} in (\getref{ex:Sketch-COORD-syn}), while the former does without one, see (\getref{ex:Sketch-COORD-as}). Connectives\is{connective} are free forms which occur between the coordinated clauses.

\ea\label{ex:Sketch-COORD-syn}
\begingl
\glpreamble nÿti kuina nÿnika pero punachÿ tiniku\\
\gla nÿti kuina nÿ-nika pero punachÿ ti-niku\\
\glb 1\textsc{sg.prn} \textsc{neg} 1\textsc{sg}-eat.\textsc{irr} but other 3i-eat\\
\glft ‘I don’t eat them, but another one eats them’\\
\endgl
%\trailingcitation{[rxx-e120511l.181]}
\xe

\ea\label{ex:Sketch-COORD-as}
\begingl
\glpreamble biyunu bibÿsÿupunutu naka\\
\gla bi-yunu bi-bÿsÿupunu-tu naka\\
\glb 1\textsc{pl}-go 1\textsc{pl}-come-\textsc{iam} here\\
\glft ‘we went and then came here’\\
\endgl
%\trailingcitation{[mqx-p110826l.380]}
\xe
\is{coordination|)}

In order to express adverbial\is{adverbial relation|(} relations, all of the construction types are found.  Nonetheless, \isi{dependency marking} is only found on purpose verbs in the specific \isi{motion-cum-purpose construction}, see (\getref{ex:Sketch-MCPC}) above. Adverbial relations are expressed in mono- and biclausal constructions. An example of a biclausal construction with a syndetically juxtaposed adverbial clause is (\getref{ex:Sketch-ADV-syn}), which includes the connective \textit{kue} ‘if, when’, while (\getref{ex:Sketch-ADV-der}) is an example of a \isi{purpose} clause including the \isi{deranked verb} \textit{chinikianube}.

\ea\label{ex:Sketch-ADV-syn}
\begingl
\glpreamble kue kaku arusu banau pan de arroz\\
\gla kue kaku arusu bi-anau {pan de arroz}\\
\glb if exist rice 1\textsc{pl}-make {rice bread}\\
\glft ‘when there is rice, we make rice bread’\\
\endgl
%\trailingcitation{[mxx-d120411ls-1a.042]}
\xe

\ea\label{ex:Sketch-ADV-der}
\begingl 
\glpreamble chibÿtupaiku echÿukena chinikianube ipitiumu\\
\gla chi-bÿtupaiku echÿu-kena chi-nik-i-a-nube ipiti-umu\\ 
\glb 3-make.fall \textsc{dem}b-\textsc{uncert} 3-eat-\textsc{subord}-\textsc{irr}-\textsc{pl} bee-\textsc{clf}:liquid\\ 
\glft ‘it seems that it makes it fall so that they can eat honey’
\endgl
%\trailingcitation{[mtx-a110906l.093]}
\xe
\is{adverbial relation|)}

Complement relations\is{complement relation|(} are usually expressed in a monoclausal construction, in which the \isi{complement verb} is asyndetically juxtaposed\is{juxtaposition} to the complement-taking verb, as in (\getref{ex:Sketch-COMPL}). In addition, there are minor types. One builds on a deranked complement verb,\is{deranked verb} the other one is a syndetic construction with the demonstrative\is{nominal demonstrative} \textit{eka} being used as a \isi{complementiser}. The latter type might be biclausal, but there are actually not enough examples in the corpus to test this hypothesis.

\ea\label{ex:Sketch-COMPL}
\begingl
\glpreamble tisachu tumapi\\
\gla ti-sachu ti-uma-pi\\
\glb 3i-want 3i-take.irr-2sg\\
\glft ‘she wants to take you’\\
\endgl
%\trailingcitation{[jxx-p151016l-2.203]}
\xe
\is{complement relation|)}

Relative relations\is{relative relation|(} are expressed by asyndetic or syndetic \isi{juxtaposition} with the latter building on the nominal demonstratives,\is{nominal demonstrative} mainly \textit{echÿu} but also \textit{eka} as a relativiser. Asyndetic \isi{juxtaposition} is predominantly found with headed\is{head} and syndetic \isi{juxtaposition} with headless relative clauses, see (\getref{ex:Sketch-headed-RC}) and (\getref{ex:Sketch-headless-RC}) respectively. Relative relations can also be expressed with the help of a \isi{deranked verb}, this is typically done if the relativised entity has the role of an \isi{oblique} inside the relative clause, but may also occur if its role is that of an \isi{object}. The relative relation is always expressed in a separate clause, i.e. we are dealing with a biclausal construction.

\ea\label{ex:Sketch-headed-RC}
\begingl
\glpreamble i kaku echÿu pisemÿnÿ nimumuku uchuine\\
\gla i kaku echÿu pise-mÿnÿ ni-imumuku uchuine\\
\glb and exist \textsc{dem}b bird-\textsc{dim} 1\textsc{sg}-look just.now\\
\glft ‘and there is this bird that I have just watched’ \\
\endgl
%\trailingcitation{[jxx-p120430l-1.100]}
\xe

\ea\label{ex:Sketch-headless-RC}
\begingl
\glpreamble echÿu timÿuji aparte chetukunube\\
\gla echÿu ti-mÿu-ji aparte chÿ-etuku-nube\\
\glb \textsc{dem}b 3i-be.wet-\textsc{clf:}soft.mass aside 3-put-\textsc{pl}\\
\glft ‘the wet things (i.e. clothes), they put aside’\\
\endgl
\trailingcitation{[jxx-p151016l-2.132]}
\xe
\is{relative relation|)}
\is{syndesis/asyndesis|)}

\chapref{sec:ComplexClauses} deals with complex sentences. Finally, deranked verbs,\is{deranked verb} which are typically associated with subordinate clauses, can also occur in constructions that more closely resemble main clauses. This is described in detail in §\ref{sec:AdverbialModification}.\is{complex sentence|)}
