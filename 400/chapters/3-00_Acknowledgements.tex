%!TEX root = 3-P_Masterdokument.tex
%!TEX encoding = UTF-8 Unicode
\chapter*{Acknowledgements}\label{sec:thanks}

This grammar is a revised version of my PhD thesis, which was submitted to the Europa-Universität Flensburg in May 2021, successfully defended in December 2021, and subsequently published digitally via the university’s library as \citet{Terhart2022}. In 2022, the thesis was awarded with the annual research prize for the best dissertation by the Europa-Universität Flensburg.

Many people have contributed to this work in various ways. First and foremost, I want to express my deepest gratitude to the speakers of Paunaka, especially Juana Supepí Yabeta, Miguel Supepí Yabeta and María Supepí Yabeta, but also (in alphabetical order) María Cuasase Choma (†), Juan Cuasase Supepí, Pedro Pinto Supepí, Alejo Supayabe Pinto (†), Polonia Supayabe Pinto, Isidro Supepí Chijene, Clara Supepí Yabeta, and José Supepí Yabeta (†). \textit{Chapie tÿpi emesumeikiu etea, paunaka. Nejujumiyue, nejujumi micha bichujijikiu, bikuyajijikiu, paseaubi, binikumÿnÿ yÿtÿuku, beumÿnÿ aumue. Nenayu niparientenenube.}

It has been a pleasure to work in the Paunaka Documentation Project, continuing the research even after grants finished. My warmest thanks go to the colleagues of my core team, Swintha Danielsen and Federico Villalta Rojas. It was great to work with you! The larger Bolivian team also includes Femmy Admiraal, Franziska Riedel, Lena Sell, and for a shorter period also Julia Bischoffberger, with whom I feel deeply connected.

I would like to thank the supervisors of my thesis, Eva Gugenberger and Françoise Rose, who provided very helpful remarks on this work. Especially the comparison with Mojeño Trinitario was very revealing and contributed a lot to improving my analysis. However, the greatest conceivable support I received from my unofficial supervisor, Swintha Danielsen, who guided me through the entire process of grammar-writing, commented on every single chapter, and was always willing and interested to discuss individual aspects of the grammar. Others came and went, she has always been there. 

Michael Rießler was involved unofficially for a short period. Jürgen Riester (†), Balthasar Bickel, and Barbara Stiebels helped by providing important signatures for obtaining grants. I would like to thank them all.

I am grateful for the financial support I received from the Endangered Languages Documentation Programme (ELDP) by obtaining a grant from 2011–2013 (MDP0217). My fieldwork in 2015 was financed by a travel grant from the German Academic Exchange Service (DAAD).

I am delighted that my grammar is being published in a series of high-quality grammars by Language Science Press. Many thanks for that to Martin Haspelmath, to Sebastian Nordhoff for his help with typesetting, and to Yanru Lu for transferring some of my amateur graphics into much nicer vector graphics. 

A lot of people found typos, misplaced commas, duplicate words, incomplete sentences and infelicitous wordings, among them Swintha Danielsen, David Terhart, Françoise Rose, and numerous volunteer proofreaders who support the idea of Language Science Press. Thanks for that! I am also grateful to Simone Faß, "die Visuelle Übersetzerin”, for drawing the maps.

It has been a long time that I have been working on this grammar, and I have had wonderful colleagues during this time in Leipzig, Flensburg, Bräist/Bredstedt, and elsewhere. Thanks to all of them! I am also thankful to all my friends and my extended family. Thanks to Sopir (†) and Nala for purring. Thanks to my parents, Rita Terhart and Klaus Oenning-Terhart, for their love and support.

And last but far from least, I want to express my gratitude and deepest love to my husband David Terhart and my wonderful daughters Franka and Anouk. Thank you for being there. You are the best!



