%!TEX root = 3-P_Masterdokument.tex
%!TEX encoding = UTF-8 Unicode


\chapter{Texts}

\section{Text 1: Story of the cowherd}\label{sec:CowherdStory}


This story was told by Miguel on the 17th of October 2015 and in parts corrected together with him later on, which is why the text given here deviates from what is actually said in the recording in parts.

The recording containing the story as well as a transcription can be found at the following address:
\url{https://www.elararchive.org/uncategorized/IO_9a1d3066-aaac-4781-b4a4-15ee93899e60}.
The original recording is archived under the name mxx-n151017l-1, and also contains a Spanish version of the story.

The story's main character is a cowherd. There is a drought and he loses the cows of his \textit{patrón}. After searching for them desperately, the spirit of the hill appears to him and tells him that he has taken the cows (the reason for that being that he felt pity with the thin cows, but this was only told in the Spanish version of the story). Subsequently, the man moves with his family to the spirit’s place inside the hill. After some time, they decide to give the cows to the people of a neighbouring village. The villagers help driving the cows, but the cowherd’s family remains invisible to them. The villagers promise to use the suet of the cows to make candles for the virgin.


\ea%[everyglpreamble=\it, exno={1.}, exnoformat=X]<ex:>
\begingl 
\glpreamble kakubaneji chÿnachÿ jente bakeronu\\
\gla kaku-bane-ji chÿnachÿ jente bakeronu\\ 
\glb exist-\textsc{rem}-\textsc{rprt} one man cowherd\\ 
\glft ‘once upon a time, there was a man who was a cowherd, it is said’\\ 
\endgl
\xe

\ea%[everyglpreamble=\it, exno={2.}, exnoformat=X]<ex:>
\begingl 
\glpreamble chikuirauchuji echÿu bakajane chipeujane chipatrune\\
\gla chi-kuirauchu-ji echÿu baka-jane chi-peu-jane chi-patrun-ne\\
\glb 3-care.for-\textsc{rprt} \textsc{dem}b cow-\textsc{distr} 3-animal-\textsc{distr} 3-patrón-\textsc{possd}\\
\glft ‘he looked for the cows, it is said, (which were) the animals of his \textit{patrón}’
\endgl
\xe
%used!

\newpage
\ea%[everyglpreamble=\it, exno={3.}, exnoformat=X]<ex:>
\begingl 
\glpreamble trabakukuikuji pasau, ¿kuyenakena?, pero pasau tiempotu trabakuyÿchi\\
\gla trabaku-kuiku-ji pasau kuyena-kena pero pasau tiempo-tu trabaku-yÿchi\\ 
\glb work-\textsc{cont}-\textsc{rprt} pass how-\textsc{uncert} but pass time-\textsc{iam} work-\textsc{lim}2\\ 
\glft ‘while he was working, it is said (time) passed, how may it have been? but time passed by and he only worked’ \\ 
\endgl
\xe

\ea%[everyglpreamble=\it, exno={4.}, exnoformat=X]<ex:>
\begingl 
\glpreamble titupunubutu eka sekía kuinaji chinikatu kuinakena mÿijinatu chinikajane eka bakajane\\
\gla ti-tupunubu-tu eka sekía kuina-ji chi-nika-tu kuina-kena mÿiji-ina-tu chi-nika-jane eka baka-jane \\ 
\glb 3i-arrive-\textsc{iam} \textsc{dem}a drought \textsc{neg}-\textsc{rprt} 3-feed.\textsc{irr}-\textsc{iam} \textsc{neg}-\textsc{uncert} grass-\textsc{irr.nv}-\textsc{iam} 3-feed.\textsc{irr}-\textsc{distr} \textsc{dem}a cow-\textsc{distr}\\ 
\glft ‘a drought came, he could not feed them, it is said, supposedly there was no grass anymore to feed the cows’\\ 
\endgl
\xe

\ea%[everyglpreamble=\it, exno={5.}, exnoformat=X]<ex:>
\begingl 
\glpreamble tiyuikunubetu kimenukÿyae\\
\gla ti-yuiku-nube-tu kimenu-kÿ-yae\\ 
\glb 3i-walk-\textsc{pl}-\textsc{iam} woods-\textsc{clf:}bounded-\textsc{loc}\\ 
\glft ‘they walked in the woods’\\ 
\endgl
\xe

\ea%[everyglpreamble=\it, exno={6.}, exnoformat=X]<ex:>
\begingl 
\glpreamble pero nechÿuji estansiayae kakuji chÿnachÿ kurichi kuina tijibÿkiapu echÿu ÿne\\
\gla pero nechÿu-ji estansia-yae kaku-ji chÿnachÿ kurichi kuina ti-jibÿk-i-a-pu echÿu ÿne\\ 
\glb but \textsc{dem}c-\textsc{rprt} manor-\textsc{loc} exist-\textsc{rpt} one pond \textsc{neg} 3i-smoke-\textsc{subord}-\textsc{irr}-\textsc{mid} \textsc{dem}b water\\ 
\glft ‘but there, it is said, on the manor, it is said, there was a pond where the water never dried (i.e. evaporated)’\\ 
\endgl
\xe

\ea%[everyglpreamble=\it, exno={7.}, exnoformat=X]<ex:>
\begingl 
\glpreamble entonses echÿu bakeronuji, pasauji treschÿ tijainube tiyunu chisemaikunube echÿu \\bakajane\\
\gla entonses echÿu bakeronu-ji pasau-ji treschÿ tijai-nube ti-yunu chi-semaiku-nube echÿu baka-jane\\ 
\glb thus \textsc{dem}b cowherd-\textsc{rprt} pass-\textsc{rprt} three day-\textsc{pl} 3i-go 3-search-\textsc{pl} \textsc{dem}b cow-\textsc{distr}\\ 
\glft ‘so the cowherd, three days passed, it is said, and he went and searched for the cows’\\ 
\endgl
\xe

\ea%[everyglpreamble=\it, exno={8.}, exnoformat=X]<ex:>
\begingl 
\glpreamble i kuina chÿtupa\\
\gla i kuina chÿ-tupa\\ 
\glb and \textsc{neg} 3-find.\textsc{irr}\\ 
\glft ‘and he didn’t find them’\\ 
\endgl
\xe

\ea%[everyglpreamble=\it, exno={9.}, exnoformat=X]<ex:>
\begingl 
\glpreamble titupajane pero kaku echÿu chibu chikuinejane tiyununube titupunubunube nechÿu \\kurichiyae\\
\gla ti-tupa-jane pero kaku echÿu chÿ-ibu chi-kuine-jane ti-yunu-nube ti-tupunubu-nube nechÿu kurichi-yae\\ 
\glb 3i-find.\textsc{irr}-\textsc{distr} but exist \textsc{dem}b 3-foot 3-footprint-\textsc{distr} 3i-go-\textsc{pl} 3-arrive-\textsc{pl} \textsc{dem}c pond-\textsc{loc}\\ 
\glft ‘he (did not) find them, but there were footprints, they went and arrived at the pond’\\ 
\endgl
\xe


\ea%[everyglpreamble=\it, exno={10.}, exnoformat=X]<ex:>
\begingl 
\glpreamble nebujikutu echÿu chibujane eka bakajane, tijekupubu\\
\gla nebu-jiku-tu echÿu chÿ-ibu-jane eka baka-jane ti-jekupu-bu\\ 
\glb 3\textsc{obl.top.prn}-\textsc{lim}1-\textsc{iam} \textsc{dem}b 3-foot-\textsc{distr} \textsc{dem}a cow-\textsc{distr} 3i-lose-\textsc{mid}\\ 
\glft ‘only up to there were the feet of the cows, they vanished’\\ 
\endgl
\xe

\ea%[everyglpreamble=\it, exno={11.}, exnoformat=X]<ex:>
\begingl 
\glpreamble tibÿkupukenajanena nechÿu ÿneumuyae\\
\gla ti-bÿkupu-kena-jane-na nechÿu ÿne-umu-yae\\ 
\glb 3i-enter-\textsc{uncert}-\textsc{distr}-? \textsc{dem}c water-\textsc{clf:}liquid-\textsc{loc}\\ 
\glft ‘apparently they had entered the water’\\ 
\endgl
\xe

\ea%[everyglpreamble=\it, exno={12.}, exnoformat=X]<ex:>
\begingl 
\glpreamble i echÿu bakeronutu te chisemaikiuchÿji, punachÿ chÿakenechÿu kurichi \\chisemaikiujanechi bakajane\\
\gla i echÿu bakeronu-tu te chi-semaik-i-u-chÿ-ji punachÿ chÿ-akene-chÿu kurichi chi-semaik-i-u-jane-chi baka-jane\\ 
\glb and \textsc{dem}b cowherd-\textsc{iam} \textsc{seq} 3-search-\textsc{subord}-\textsc{real}-3-\textsc{rprt} other 3-non.vis.side-\textsc{dem}b pond 3-search-\textsc{subord}-\textsc{real}-\textsc{distr}-3 cow-\textsc{distr}\\ 
\glft ‘and the cowherd then in looking for them, it is said, on the other side of the pond he searched for the cows’\\ 
\endgl
\xe
%verwendet


\ea%[everyglpreamble=\it, exno={13.}, exnoformat=X]<ex:>
\begingl 
\glpreamble repentekena pasaujane echÿu kurichi, pero kuinaji chitupa echÿu chibujane\\
\gla repente-kena pasau-jane echÿu kurichi pero kuina-ji chi-tupa echÿu chÿ-ibu-jane\\ 
\glb maybe-\textsc{uncert} pass-\textsc{distr} \textsc{dem}b pond but \textsc{neg}-\textsc{rprt} 3-find.\textsc{irr} \textsc{dem}b 3-foot-\textsc{distr}\\ 
\glft ‘in case that they had passed the pond, but he did not find their feet, it is said’\\ 
\endgl
\xe

\ea%[everyglpreamble=\it, exno={14.}, exnoformat=X]<ex:>
\begingl 
\glpreamble tichÿnumituji pasau trestu tijainube chikechuchÿji chiyenu:\\
\gla ti-chÿnumi-tu-ji pasau tres-tu tijai-nube chi-kechu-chÿ-ji chi-yenu:\\ 
\glb 3i-be.sad-\textsc{iam}-\textsc{rprt} pass three-\textsc{iam} day-\textsc{pl} 3-say-3-\textsc{rprt} 3-wife\\ 
\glft ‘he was sad, when three days had passed, he said to his wife, it is said:’\\ 
\endgl
\xe

\ea%[everyglpreamble=\it, exno={15.}, exnoformat=X]<ex:>
\begingl 
\glpreamble “kuina nitupa echÿu bakajane, ¿juchubukena chiyunujane?\\
\gla kuina ni-tupa echÿu baka-jane juchubu-kena chi-yunu-jane\\ 
\glb \textsc{neg} 1\textsc{sg}-find.\textsc{irr} \textsc{dem}b cow-\textsc{distr} where-\textsc{uncert} 3-go-\textsc{distr}\\ 
\glft ‘“I don’t find the cows, where may they have gone?’\\ 
\endgl
\xe

\ea%[everyglpreamble=\it, exno={16.}, exnoformat=X]<ex:>
\begingl 
\glpreamble pero echÿu chibujane tropakÿu chiyuikiujane \\
\gla pero echÿu chÿ-ibu-jane tropa-kÿu chi-yuik-i-u-jane \\ 
\glb but \textsc{dem}b 3-foot-\textsc{distr} pack-\textsc{am.conc.tr} 3-walk-\textsc{subord}-\textsc{real}-\textsc{distr} \\ 
\glft ‘but their feet walk in a pack’\\ 
\endgl
\xe

\ea%[everyglpreamble=\it, exno={17.}, exnoformat=X]<ex:>
\begingl 
\glpreamble pero titupunubu echÿu kurichiyae nebujiku titupunubutu, kuina pasaujaneina\\
\gla pero ti-tupunubu echÿu kurichi-yae nebu-jiku ti-tupunubu-tu kuina pasau-jane-ina\\ 
\glb but 3i-arrive \textsc{dem}b pond-\textsc{loc} 3\textsc{obl.top.prn}-\textsc{lim}1 3-arrive-\textsc{iam} \textsc{neg} pass-\textsc{distr}-\textsc{irr.nv}\\ 
\glft ‘but they arrive at the pond, they arrive only there, they do not pass”’\\ 
\endgl
\xe

\ea%[everyglpreamble=\it, exno={18.}, exnoformat=X]<ex:>
\begingl 
\glpreamble chÿkechuchÿji echÿu chiyenu\\
\gla chÿ-kechu-chÿ-ji echÿu chi-yenu\\ 
\glb 3-say-3-\textsc{rprt} \textsc{dem}b 3-wife\\ 
\glft ‘he said to his wife, it is said’\\ 
\endgl
\xe

\ea%[everyglpreamble=\it, exno={19.}, exnoformat=X]<ex:>
\begingl 
\glpreamble “takujemukena echÿu bipatrune kue kuina etupa echÿu bakajane”\\
\gla ti-a-kujemu-kena echÿu bi-patrun-ne kue kuina e-tupa echÿu baka-jane\\ 
\glb 3i-\textsc{irr}-be.angry-\textsc{uncert} \textsc{dem}b 1\textsc{pl}-patrón-\textsc{possd} if \textsc{neg} 2\textsc{pl}-find.\textsc{irr} \textsc{dem}b cow-\textsc{distr}\\ 
\glft ‘“our \textit{patrón} will probably be angry, if you don’t find the cows”’\\ 
\endgl
\xe

\ea%[everyglpreamble=\it, exno={20.}, exnoformat=X]<ex:>
\begingl 
\glpreamble tikechuchÿ chiyenu\\
\gla ti-kechu-chÿ chi-yenu\\ 
\glb 3i-say-3 3-wife\\ 
\glft ‘his wife said’\\ 
\endgl
\xe


\ea%[everyglpreamble=\it, exno={21.}, exnoformat=X]<ex:>
\begingl 
\glpreamble punachÿ tijai tiyunupunukuji\\
\gla punachÿ tijai ti-yunu-punuku-ji\\ 
\glb other day 3i-go-\textsc{reg}-\textsc{rprt}\\ 
\glft ‘the other day, he went again, it is said’\\ 
\endgl
\xe

\ea%[everyglpreamble=\it, exno={22.}, exnoformat=X]<ex:>
\begingl 
\glpreamble kabaritoji tusetu tikutiyu sache tichemumuikubuji chiupekÿye echÿu ameji\\
\gla kabarito-ji tuse-tu ti-kuti-yu sache ti-chemumuiku-bu-ji chi-upekÿ-yae echÿu ame-ji\\ 
\glb exactly-\textsc{rprt} noon-\textsc{iam} 3i-hurt-\textsc{ints} sun 3i-stand-\textsc{mid}-\textsc{rprt} 3-place.under-\textsc{loc} \textsc{dem}b motacú-\textsc{rprt}\\ 
\glft ‘exactly at noon, it is said, when the sun was very strong, he stood under a \textit{motacú} palm, it is said’\\ 
\endgl
\xe

\ea%[everyglpreamble=\it, exno={23.}, exnoformat=X]<ex:>
\begingl 
\glpreamble reskansaujiku repenteyÿchiji titabipaiku echÿu pÿsi\\
\gla reskansau-jiku repente-yÿchi-ji ti-tabipaiku echÿu pÿsi\\ 
\glb rest-\textsc{lim}1 suddenly-\textsc{lim}2-\textsc{rprt} 3i-appear \textsc{dem}a spirit.of.hill\\ 
\glft ‘he just rested, when all of a sudden the spirit of the hill appeared’\\ 
\endgl
\xe


\ea%[everyglpreamble=\it, exno={24.}, exnoformat=X]<ex:>
\begingl
\glpreamble pÿsi chija bitÿpi echÿu tikubiunube naka chiyikikeyae, pÿsi echÿu mupÿinube\\
\gla pÿsi chi-ija bi-tÿpi echÿu ti-ku-ubiu-nube naka chiyikike-yae pÿsi echÿu mupÿinube\\
\glb pÿsi 3-name 1\textsc{pl}-\textsc{obl} \textsc{dem}b 3i-\textsc{attr}-house-\textsc{pl} here hill-\textsc{loc} pÿsi \textsc{dem}b devil\\
\glft ‘\textit{pÿsi} we call the ones who have their houses in the hills (i.e. the owners of the hills), \textit{pÿsi} is the devil’
\endgl
\xe
%verwendet

\ea%[everyglpreamble=\it, exno={25.}, exnoformat=X]<ex:>
\begingl 
\glpreamble titabipaikuji chikechuchÿ: “¿chija pichabubuikubu?”\\
\gla ti-tabipaiku-ji chi-kechu-chÿ chija pi-chabubuiku-bu? \\ 
\glb 3i-appear-\textsc{rprt} 3-say-3 what 2\textsc{sg}-do-\textsc{mid}\\ 
\glft ‘he appeared, it is said, and said: “what are you doing?”’\\ 
\endgl
\xe

\ea%[everyglpreamble=\it, exno={26.}, exnoformat=X]<ex:>
\begingl 
\glpreamble chiyÿseichupunu\\
\gla chi-yÿseichu-punu\\ 
\glb 3-ask-\textsc{am.prior}\\ 
\glft ‘he came and asked him’\\ 
\endgl
\xe

\ea%[everyglpreamble=\it, exno={27.}, exnoformat=X]<ex:>
\begingl 
\glpreamble “¿michabi?” – “micha”\\
\gla micha-bi micha\\ 
\glb good-2\textsc{sg} good\\ 
\glft ‘“how are you?” – “fine”’\\ 
\endgl
\xe


\ea%[everyglpreamble=\it, exno={28.}, exnoformat=X]<ex:>
\begingl 
\glpreamble “¿chija pichabubuikubu?”\\
\gla  chija pi-chabubuiku-bu\\ 
\glb  what 2\textsc{sg}-do-\textsc{mid}\\ 
\glft ‘“what are you doing?”’\\ 
\endgl
\xe

\ea%[everyglpreamble=\it, exno={29.}, exnoformat=X]<ex:>
\begingl 
\glpreamble “aa, nÿchÿnumi, aa, pensaikunÿ kuina nitupa echÿu bakajane tijekupupuikutu”, tikechuji\\
\gla aa nÿ-chÿnumi aa pensai-ku-nÿ kuina ni-tupa echÿu baka-jane ti-jekupu-puiku-tu ti-kechu-ji\\ 
\glb \textsc{intj} 1\textsc{sg}-be.sad \textsc{intj} think-?-1\textsc{sg} \textsc{neg} 1\textsc{sg}-find.\textsc{irr} \textsc{dem}b cow-\textsc{distr} 3i-lose-\textsc{cont}-\textsc{iam} 3i-say-\textsc{rprt}\\ 
\glft ‘“ah, I am sad, ah, I am thinking that I don’t find the cows, they are lost”, he said, it is said’\\ 
\endgl
\xe

\ea%[everyglpreamble=\it, exno={30.}, exnoformat=X]<ex:>
\begingl 
\glpreamble “aa, chikuye”\\
\gla aa chi-kuye\\ 
\glb \textsc{intj} 3-be.like.this\\ 
\glft ‘“ah, that’s it”’\\ 
\endgl
\xe

\ea%[everyglpreamble=\it, exno={31.}, exnoformat=X]<ex:>
\begingl 
\glpreamble tikechuchÿji: “kaku naka nubiuyae kaku naka echÿu pisemaikiuchi echÿu bakajane kaku\\
\gla ti-kechu-chÿ-ji kaku naka nÿ-ubiu-yae kaku naka echÿu pi-semaik-i-u-chi echÿu baka-jane kaku\\ 
\glb 3i-say-3-\textsc{rprt} exist here 1\textsc{sg}-house-\textsc{loc} exist here \textsc{dem}b 2\textsc{sg}-search-\textsc{subord}-\textsc{real}-3 \textsc{dem}b cow-\textsc{distr} exist\\ 
\glft ‘he said to him, it is said: “they are here in my house, here is what you are looking for, the cows, they are there’\\ 
\endgl
\xe
%verwendet

\ea%[everyglpreamble=\it, exno={32.}, exnoformat=X]<ex:>
\begingl 
\glpreamble ¿pisachu piyuna pimuajane?\\
\gla pi-sachu pi-yuna pi-imua-jane\\ 
\glb 2\textsc{sg}-want 2\textsc{sg}-go.\textsc{irr} 2\textsc{sg}-see.\textsc{irr}-\textsc{distr}\\ 
\glft ‘do you want to go and see them?’\\ 
\endgl
\xe
%verwendet


\ea%[everyglpreamble=\it, exno={33.}, exnoformat=X]<ex:>
\begingl 
\glpreamble ¡jaje!” chikechuchÿji\\
\gla jaje chi-kechu-chÿ-ji\\ 
\glb \textsc{hort} 3-say-3-\textsc{rprt}\\ 
\glft ‘let’s go!” he said to him, it is said’\\ 
\endgl
\xe

\newpage

\ea%[everyglpreamble=\it, exno={34.}, exnoformat=X]<ex:>
\begingl 
\glpreamble “¡jaje biyuna bimupajane echÿu bakajane!” tikechuji\\
\gla jaje bi-yuna bi-imu-pa-jane echÿu baka-jane ti-kechu-ji\\ 
\glb \textsc{hort} 1\textsc{pl}-go.\textsc{irr} 1\textsc{pl}-see-\textsc{dloc.irr}-\textsc{distr} \textsc{dem}b cow-\textsc{distr} 3i-say-\textsc{rprt}\\ 
\glft ‘“let’s go and see the cows!” he said, it is said’\\ 
\endgl
\xe
%zweimal verwendet

\ea%[everyglpreamble=\it, exno={35.}, exnoformat=X]<ex:>
\begingl 
\glpreamble entonses echÿu bakeronuji tikechuji: “¡jajejachÿutu!”\\
\gla entonses echÿu bakeronu-ji ti-kechu-ji jaje-ja-chÿu-tu\\ 
\glb thus \textsc{dem}b cowherd-\textsc{rprt} 3i-say-\textsc{rprt} \textsc{hort}-\textsc{emph}1-\textsc{dem}b?-\textsc{iam}\\ 
\glft ‘so the cowherd said, it is said: “let’s go, then!”\\ 
\endgl
\xe

\ea%[everyglpreamble=\it, exno={36.}, exnoformat=X]<ex:>
\begingl 
\glpreamble tiyunuji, tiyuiku tiyuikunubeji echÿu ÿneyae repenteyÿchi kakutu punachÿukukena apukeji\\
\gla ti-yunu-ji ti-yuiku ti-yuiku-nube-ji echÿu ÿne-yae repente-yÿchi kaku-tu punachÿ-uku-kena apukeji\\ 
\glb 3i-go-\textsc{rprt} 3i-walk 3i-walk-\textsc{pl}-\textsc{rprt} \textsc{dem}b water-\textsc{loc} suddenly-\textsc{lim}2 exist-\textsc{iam} other-\textsc{add}-\textsc{uncert} ground-\textsc{rprt}\\ 
\glft ‘he went, it is said, he walked, they walked, it is said, into the water, suddenly there was apparently another world, it is said’\\ 
\endgl
\xe

\ea%[everyglpreamble=\it, exno={37.}, exnoformat=X]<ex:>
\begingl 
\glpreamble tibÿkupunubetuji chÿyÿkikeyae\\
\gla ti-bÿkupu-nube-tu-ji chÿyÿkike-yae\\ 
\glb 3i-enter-\textsc{pl}-\textsc{iam}-\textsc{rprt} hill-\textsc{loc}\\ 
\glft ‘they entered the hill, it is said’\\ 
\endgl
\xe

\ea%[everyglpreamble=\it, exno={38.}, exnoformat=X]<ex:>
\begingl 
\glpreamble nebutukena nebuji chimukiuchÿtuji echÿu bakajane chÿchupuikujane\\
\gla nebu-tu-kena nebu-ji chi-imuk-i-u-chÿ-tu-ji echÿu baka-jane chÿ-chupuiku-jane\\ 
\glb 3\textsc{obl.top.prn}-\textsc{iam}-\textsc{uncert} 3\textsc{obl.top.prn}-\textsc{rprt} 3-see-\textsc{subord}-\textsc{real}-3-\textsc{iam}-\textsc{rprt} \textsc{dem}b cow-\textsc{distr} 3-know-\textsc{distr}\\ 
\glft ‘it was probably there, it is said, that he saw the cows, he recognised them’\\ 
\endgl
\xe

\ea%[everyglpreamble=\it, exno={39.}, exnoformat=X]<ex:>
\begingl
\glpreamble “chibu eka bakajane eka pisemaiku”, tikechuchÿji echÿu pÿsi\\
\gla chibu eka baka-jane eka pi-semaiku ti-kechu-chÿ-ji echÿu pÿsi\\
\glb 3\textsc{top.prn} \textsc{dem}a cow-\textsc{distr} \textsc{dem}a 2\textsc{sg}-search 3i-say-3-\textsc{rprt} \textsc{dem}b spirit.of.hill\\
\glft ‘“these are the cows that you were looking for”, the spirit of the hill said to him, it is said’
\endgl
\xe
%verwendet: <ex:headed-DEM>

\ea%[everyglpreamble=\it, exno={40.}, exnoformat=X]<ex:>
\begingl 
\glpreamble “aa, chibu eka nisemaikutu, eka eka eka”, chichupuikiuchÿ echÿu bakajane\\
\gla aa chibu eka ni-semaiku-tu eka eka eka chi-chupuik-i-u-chÿ echÿu baka-jane\\ 
\glb \textsc{intj} 3\textsc{top.prn} \textsc{dem}a 1\textsc{sg}-search-\textsc{iam} \textsc{dem}a \textsc{dem}a \textsc{dem}a 3-know-\textsc{subord}-\textsc{real}-3 \textsc{dem}b cow-\textsc{distr}\\ 
\glft ‘“ah, this is what I was searching for, that one, that one, that one”, recognising the cows’\\ 
\endgl
\xe


\ea%[everyglpreamble=\it, exno={41.}, exnoformat=X]<ex:>
\begingl 
\glpreamble “¿chikuyena?” pensaikui tanÿma\\
\gla chikuyena pensai-kui tanÿma\\ 
\glb how think-\textsc{cont}? now\\ 
\glft ‘“how?”, he was thinking now’\\ 
\endgl
\xe

\ea%[everyglpreamble=\it, exno={42.}, exnoformat=X]<ex:>
\begingl 
\glpreamble “tanÿma kuina puero pupuna echÿu bakajane\\
\gla tanÿma kuina puero pi-upuna echÿu baka-jane\\ 
\glb now \textsc{neg} can 2\textsc{sg}-take.\textsc{irr} \textsc{dem}b cow-\textsc{distr}\\ 
\glft ‘“you can’t take the cows now’\\ 
\endgl
\xe

\ea%[everyglpreamble=\it, exno={43.}, exnoformat=X]<ex:>
\begingl 
\glpreamble mas bien pibÿsÿpuna naka pipajÿka”, chikechuchÿji\\
\gla {mas bien} pi-bÿsÿpuna naka pi-pajÿka chi-kechu-chÿ-ji\\ 
\glb {better} 2\textsc{sg}-come.\textsc{irr} here 2\textsc{sg}-stay.\textsc{irr} 3-say-3-\textsc{rprt}\\ 
\glft ‘you’d better come here and stay”, he said to him, it is said’\\ 
\endgl
\xe

\ea%[everyglpreamble=\it, exno={44.}, exnoformat=X]<ex:>
\begingl 
\glpreamble pensaikituji echÿu bakeronu i repueji ta te chijakuputuji chikechuchi: “bueno, nÿbÿsÿakena”\\
\gla pensai-ki-tu-ji echÿu bakeronu i repue-ji ta te chi-jakupu-tu-ji chi-kechu-chi bueno nÿ-bÿsÿa-kena\\ 
\glb think-?-\textsc{iam}-\textsc{rprt} \textsc{dem}b cowherd and afterwards-\textsc{rprt} ? \textsc{seq} 3-accept-\textsc{iam}-\textsc{rprt} 3-say-3 well 1\textsc{sg}-come.\textsc{irr}-\textsc{uncert}\\ 
\glft ‘the cowherd thought (about it), it is said, and then he accepted, he said: “well, I may come”’\\ 
\endgl
\xe

\ea%[everyglpreamble=\it, exno={45.}, exnoformat=X]<ex:>
\begingl 
\glpreamble “bueno, piyunupuna pubiuyae entonses pupuna tumuyubu echÿu pichechajinube, pupuna piyenu\\
\gla bueno pi-yunupuna pi-ubiu-yae entonses pi-upuna tumuyubu echÿu pi-checha-ji-nube pi-upuna pi-yenu\\ 
\glb well 2\textsc{sg}-go.back.\textsc{irr} 2\textsc{sg}-house-\textsc{loc} thus 2\textsc{sg}-bring.\textsc{irr} all \textsc{dem}b 2\textsc{sg}-son-\textsc{col}-\textsc{pl} 2\textsc{sg}-bring.\textsc{irr} 2\textsc{sg}-wife\\ 
\glft ‘“well, go back to your house, so bring all your children, bring your wife’
\endgl
\xe

\ea%[everyglpreamble=\it, exno={46.}, exnoformat=X]<ex:>
\begingl 
\glpreamble epajÿkatu naka, kuina sufriubina naka\\
\gla e-pajÿka-tu naka kuina sufriu-bi-ina naka\\ 
\glb 2\textsc{pl}-stay.\textsc{irr}-\textsc{iam} here \textsc{neg} suffer-1\textsc{pl}-\textsc{irr.nv} here\\ 
\glft ‘you can stay here now, we will not suffer here’\\ 
\endgl
\xe


\ea%[everyglpreamble=\it, exno={47.}, exnoformat=X]<ex:>
\begingl 
\glpreamble kaku mukiankajane kakutu naka bakajane”\\
\gla kaku mukianka-jane kaku-tu naka baka-jane\\ 
\glb exist animal-\textsc{distr} exist-\textsc{iam} here cow-\textsc{distr}\\ 
\glft ‘here are animals, the cows are here now”’\\ 
\endgl
\xe

\ea%[everyglpreamble=\it, exno={48.}, exnoformat=X]<ex:>
\begingl 
\glpreamble bueno, tiyunupunutuji echÿu bakeronu\\
\gla bueno ti-yunupunu-tu-ji echÿu bakeronu\\ 
\glb well 3i-go.back-\textsc{iam}-\textsc{rprt} \textsc{dem}b cowherd\\ 
\glft ‘well, the cowherd went back, it is said’\\ 
\endgl
\xe

\ea%[everyglpreamble=\it, exno={49.}, exnoformat=X]<ex:>
\begingl 
\glpreamble tibÿchÿupupunukuji naka apukeyae i tiyunupunu chubiuyae titupupunubuji \\chikechuchituji chiyenu:\\
\gla ti-bÿchÿu-pupunuku-ji naka apuke-yae i ti-yunupunu chÿ-ubiu-yae ti-tupupunubu-ji chi-kechu-chi-tu-ji chi-yenu\\ 
\glb 3i-leave-\textsc{reg}-\textsc{rprt} here ground-\textsc{loc} and 3i-go.back 3-house-\textsc{loc} 3i-arrive.\textsc{reg}-\textsc{rprt} 3-say-3-\textsc{iam}-\textsc{rprt} 3-wife\\ 
\glft ‘he left to the ground here again, it is said, and went back to his house, when he arrived back, he said to his wife:’\\ 
\endgl
\xe
%verwendet: <ex:apuke-1>


\ea%[everyglpreamble=\it, exno={50.}, exnoformat=X]<ex:>
\begingl 
\glpreamble “nÿtuputu echÿu bakajane”\\
\gla nÿ-tupu-tu echÿu baka-jane\\ 
\glb 1\textsc{sg}-find-\textsc{iam} \textsc{dem}b cow-\textsc{distr}\\ 
\glft ‘“I have found the cows”’\\ 
\endgl
\xe

\ea%[everyglpreamble=\it, exno={51.}, exnoformat=X]<ex:>
\begingl 
\glpreamble “¡pituputu!” – “jaa, nÿtuputu\\
\gla pi-tupu-tu jaa nÿ-tupu-tu\\ 
\glb 2\textsc{sg}-find-\textsc{iam} \textsc{afm} 1\textsc{sg}-find-\textsc{iam}\\ 
\glft ‘“you have found them!” – “yes, I have found them’\\ 
\endgl
\xe

\ea%[everyglpreamble=\it, exno={52.}, exnoformat=X]<ex:>
\begingl 
\glpreamble kakujanetu nauku chiyikikeyae” tikechuchÿji\\
\gla kaku-jane-tu nauku chiyikike-yae ti-kechu-chÿ-ji\\ 
\glb exist-\textsc{distr}-\textsc{iam} there hill-\textsc{loc} 3i-say-3-\textsc{rprt}\\ 
\glft ‘they are there now, in the hill”, he said to her, it is said’\\ 
\endgl
\xe

\ea%[everyglpreamble=\it, exno={53.}, exnoformat=X]<ex:>
\begingl 
\glpreamble “aiy”, chikechituji echÿu chiyenu, “kakutu chiyikikiyae echÿu bakajane kakunubetu nauku”\\
\gla aiy chi-ke-chi-tu-ji echÿu chi-yenu kaku-tu chiyikiki-yae echÿu baka-jane kaku-nube-tu nauku\\ 
\glb \textsc{intj} 3-say-3-\textsc{iam}-\textsc{rprt} \textsc{dem}b 3-wife exist-\textsc{iam} hill-\textsc{loc} \textsc{dem}b cow-\textsc{distr} exist-\textsc{pl}-\textsc{iam} there\\ 
\glft ‘“aiy”, said his wife, “the cows are in the hill now, they are there now”’\\ 
\endgl
\xe
%verwendet<ex:LocP-3>


\ea%[everyglpreamble=\it, exno={54.}, exnoformat=X]<ex:>
\begingl 
\glpreamble “echÿu pÿsi tikechunÿ mas bien i biyuna nauku chiyikikiyae\\
\gla echÿu pÿsi ti-kechu-nÿ {mas bien} i bi-yuna nauku chiyikiki-yae\\ 
\glb \textsc{dem}b spirit.of.hill 3i-say-1\textsc{sg} better and 1\textsc{pl}-go.\textsc{irr} there hill-\textsc{loc}\\ 
\glft ‘“the spirit of the hill said to me that we’d better go there to the hill’\\ 
\endgl
\xe

\ea%[everyglpreamble=\it, exno={55.}, exnoformat=X]<ex:>
\begingl 
\glpreamble chikijÿekÿyae bipajÿkatu nauku tikechunÿ\\
\gla chiki-jÿekÿ-yae bi-pajÿka-tu nauku ti-kechu-1\textsc{sg}\\ 
\glb hill-inside-\textsc{loc} 1\textsc{pl}-stay.\textsc{irr}-\textsc{iam} there 3i-say\\ 
\glft ‘we can stay there inside of the hill, he said to me\\ 
\endgl
\xe

\ea%[everyglpreamble=\it, exno={56.}, exnoformat=X]<ex:>
\begingl 
\glpreamble ¿pisachukena piyuna?”\\
\gla pi-sachu-kena pi-yuna\\ 
\glb 2\textsc{sg}-want-\textsc{uncert} 2\textsc{sg}-go.\textsc{irr}\\ 
\glft ‘do you want to go, perhaps?”’\\ 
\endgl
\xe

\ea%[everyglpreamble=\it, exno={57.}, exnoformat=X]<ex:>
\begingl 
\glpreamble “¡jajejachÿu!” tikechutuji\\
\gla jaje-ja-chÿu” ti-kechu-tu-ji\\ 
\glb \textsc{hort}-\textsc{emph}1-\textsc{dem}b? 3i-say-\textsc{iam}-\textsc{rprt}\\ 
\glft ‘“let’s go, then”, she said, it is said’\\ 
\endgl
\xe

\ea%[everyglpreamble=\it, exno={58.}, exnoformat=X]<ex:>
\begingl 
\glpreamble komoraunubetuji echÿu chichechajinube tiyununubeji\\
\gla komorau-nube-tu-ji echÿu chi-checha-ji-nube ti-yunu-nube\\ 
\glb accomodate-\textsc{pl}-\textsc{iam}-\textsc{rprt} \textsc{dem}b 3-son-\textsc{col}-\textsc{pl} 3i-go-\textsc{pl}-\textsc{rprt}\\ 
\glft ‘they arranged everything as regards their children and they went, it is said’\\ 
\endgl
\xe

\newpage
\ea%[everyglpreamble=\it, exno={59.}, exnoformat=X]<ex:>
\begingl 
\glpreamble titupunubunubeji nechÿu kurichiyae, “¡jaje ÿneneumukÿyae!”\\
\gla ti-tupunubu-nube-ji nechÿu kurichi-yae jaje ÿne-ne-umu-kÿ-yae ti-kechu-chi\\ 
\glb 3i-arrive-\textsc{pl}-\textsc{rprt} \textsc{dem}c pond-\textsc{loc} \textsc{hort} water-\textsc{rep}-\textsc{clf:}liquid-\textsc{clf:}bounded-\textsc{loc}\\ 
\glft ‘they arrived there at the pond, “let’s go inside the water”’\\ 
\endgl
\xe

\ea%[everyglpreamble=\it, exno={60.}, exnoformat=X]<ex:>
\begingl 
\glpreamble tibÿkupunubetuji ÿneumuyae i nebu nechÿukena nuinekÿ chÿtÿpi echÿu pÿsi\\
\gla ti-bÿkupu-nube-tu-ji ÿne-umu-yae i nebu nechÿu-kena nuinekÿ chÿ-tÿpi echÿu pÿsi\\ 
\glb 3i-enter-\textsc{pl}-\textsc{iam}-\textsc{rprt} water-\textsc{clf:}liquid-\textsc{loc} and 3\textsc{obl.top.prn} \textsc{dem}c-\textsc{uncert} door 3-\textsc{obl} \textsc{dem}b spirit.of.hill\\ 
\glft ‘they went into the water, it is said, and there was probably a door for the spirit of the hill’\\ 
\endgl
\xe

\ea%[everyglpreamble=\it, exno={61.}, exnoformat=X]<ex:>
\begingl 
\glpreamble repenteyÿchi kakunubetuji chiyikikeyae chiyikijÿekÿyae\\
\gla repente-yÿchi kaku-nube-tu-ji chiyikike-yae chiyiki-jÿekÿ-yae\\ 
\glb suddenly-\textsc{lim}2 exist-\textsc{pl}-\textsc{iam}-\textsc{rprt} hill-\textsc{loc} hill-inside-\textsc{loc}\\ 
\glft ‘suddenly they were in the hill, it is said, inside of the hill’\\ 
\endgl
\xe

\ea%[everyglpreamble=\it, exno={62.}, exnoformat=X]<ex:>
\begingl 
\glpreamble tipajÿkunubetu\\
\gla ti-pajÿku-nube-tu\\ 
\glb 3i-stay-\textsc{pl}-\textsc{iam}\\ 
\glft ‘they stayed’\\ 
\endgl
\xe

\ea%[everyglpreamble=\it, exno={63.}, exnoformat=X]<ex:>
\begingl 
\glpreamble pasauji chÿnachÿtu anyo tikechuchÿji echÿu pÿsi echÿu bakeronu ja:\\
\gla pasau-ji chÿnachÿ-tu anyo ti-kechu-chÿ-ji echÿu pÿsi echÿu bakeronu ja\\ 
\glb pass-\textsc{rprt} one-\textsc{iam} year 3i-say-3-\textsc{rprt} \textsc{dem}b spirit.of.hill \textsc{dem}b cowherd \textsc{emph}1?\\ 
\glft ‘after one year had passed by, it is said, the spirit of the hill said to the cowherd:’\\ 
\endgl
\xe

\ea%[everyglpreamble=\it, exno={64.}, exnoformat=X]<ex:>
\begingl 
\glpreamble “biyunupuna nauku chubiunubeye echÿu piparientenenube\\
\gla bi-yunupuna nauku chÿ-ubiu-nube-yae echÿu pi-pariente-ne-nube\\ 
\glb 1\textsc{pl}-go.back.\textsc{irr} there 3-house-\textsc{pl}-\textsc{loc} \textsc{dem}b 2\textsc{sg}-relative-\textsc{possd}-\textsc{pl}\\ 
\glft ‘“we will go back to the houses of your relatives’\\ 
\endgl
\xe

\ea%[everyglpreamble=\it, exno={65.}, exnoformat=X]<ex:>
\begingl 
\glpreamble kapununubeina sinkonubechina jentenube ayaraunubeina bitÿpi eka bumia eka bakajane\\
\gla kapunu-nube-ina sinko-nube-chi-ina jente-nube ayarau-nube-ina bi-tÿpi eka bi-um-i-a eka baka-jane\\ 
\glb come-\textsc{pl}-\textsc{irr.nv} five-\textsc{pl}-3-\textsc{irr.nv} man-\textsc{pl} help-\textsc{pl}-\textsc{irr.nv} 1\textsc{pl}-\textsc{obl} \textsc{dem}a 1\textsc{pl}-take-\textsc{subord}-\textsc{irr} \textsc{dem}a cow-\textsc{distr}\\ 
\glft ‘may five men come to help us take the cows’\\ 
\endgl
\xe
%verwendet<ex:five-1>


\ea%[everyglpreamble=\it, exno={66.}, exnoformat=X]<ex:>
\begingl 
\glpreamble tÿpi chinikanube nauku i echÿu sebo tÿpi beraina chitÿpi eka benu” chikechuchÿji\\
\gla tÿpi chi-nika-nube nauku i echÿu sebo tÿpi bera-ina chi-tÿpi eka benu” chi-kechu-chÿ-ji\\ 
\glb \textsc{obl} 3-eat.\textsc{irr}-\textsc{pl} there and \textsc{dem}b suet \textsc{obl} candle-\textsc{ir.nv} 3-\textsc{obl} \textsc{dem}a virgin 3-say-3-\textsc{rprt}\\ 
\glft ‘so that they eat them there and the suet is for candles for the virgin”, he said to him, it is said’\\ 
\endgl
\xe

\ea%[everyglpreamble=\it, exno={67.}, exnoformat=X]<ex:>
\begingl 
\glpreamble tiyunuji echÿu jente bakeronu, tupunuji kompirauchituji sinko jentenube chumuji\\
\gla ti-yunu-ji echÿu jente bakeronu ti-upunu-ji kompirau-chi-tu-ji sinko jente-nube chÿ-umu-ji\\ 
\glb 3i-go-\textsc{rprt} \textsc{dem}b man cowherd 3i-bring-\textsc{rprt} share-3-\textsc{iam}-\textsc{rprt} five man-\textsc{pl} 3-take-\textsc{rprt}\\ 
\glft ‘the man who was a cowherd went, it is said, he brought five men to share (the workload), it is said, he took them, it is said’\\ 
\endgl
\xe
%verwendet<ex:Borri-1>

\ea%[everyglpreamble=\it, exno={68.}, exnoformat=X]<ex:>
\begingl 
\glpreamble “nakajiku ekichupupuikanÿ”, tikechuchÿji, “epuna anÿke\\
\gla naka-jiku e-kichupu-puika-nÿ ti-kechu-chÿ-ji e-puna anÿke\\ 
\glb here-\textsc{lim}1 2\textsc{pl}-wait-\textsc{cont.irr}-1\textsc{sg} 3i-say-3-\textsc{rprt} 2\textsc{pl}-go.up.\textsc{irr} up\\ 
\glft ‘“wait for me right here”, he said to them, it is said, “go up’\\ 
\endgl
\xe
%verwendet<ex:imp-8>

\ea%[everyglpreamble=\it, exno={69.}, exnoformat=X]<ex:>
\begingl 
\glpreamble echÿu bakajane tikujemujane”\\
\gla echÿu baka-jane ti-kujemu-jane ti-kechu-ji\\ 
\glb \textsc{dem}b cow-\textsc{distr} 3i-be.angry-\textsc{distr}\\ 
\glft ‘the cows are wild”’\\ 
\endgl
\xe

\ea%[everyglpreamble=\it, exno={70.}, exnoformat=X]<ex:>
\begingl 
\glpreamble bueno, tipununubeji anÿke, tikichupupuikunubeji\\
\gla  bueno ti-punu-nube-ji anÿke ti-kichupu-puiku-nube-ji\\ 
\glb well 3i-go.up-\textsc{rprt} up 3i-wait-\textsc{cont}-\textsc{pl}-\textsc{rprt}\\ 
\glft ‘well, they went up, they waited, it is said’\\ 
\endgl
\xe

\ea%[everyglpreamble=\it, exno={71.}, exnoformat=X]<ex:>
\begingl 
\glpreamble tosetuji chisamunubetuji echÿu tiyÿbuikÿupununubetuji “¡jia jÿa jia jia vamo vamo!” tiyÿbuikÿupununubetuji\\
\gla tose-tu-ji chi-samu-nube-tu-ji echÿu ti-yÿbui-kÿupunu-nube-tu-ji {jia jÿa jia jia vamo vamo} ti-yÿbui-kÿupunu-nube-tu-ji\\ 
\glb noon-\textsc{iam}-\textsc{rprt} 3-hear-\textsc{pl}-\textsc{iam}-\textsc{rprt} \textsc{dem}b 3i-shout-\textsc{am.conc.cis}-\textsc{pl}-\textsc{iam}-\textsc{rprt} {sound.of.wrangling} 3i-shout-\textsc{am.conc.cis}-\textsc{pl}-\textsc{iam}-\textsc{rprt}\\ 
\glft ‘when it turned twelve, it is said, they heard the ones who came shouting “hia hɨa hia hia let’s go, let’s go!”, they came shouting, it is said’\\ 
\endgl
\xe
%verwendet<ex:kÿupunu-non.mot-2>

\ea%[everyglpreamble=\it, exno={72.}, exnoformat=X]<ex:>
\begingl 
\glpreamble i repue ya tibÿchecheikutuji echÿu bakajane\\
\gla i repue ya ti-bÿchecheiku-tu-ji echÿu baka-jane\\ 
\glb and afterwards already 3i-come.out.of.water-\textsc{iam}-\textsc{rprt} \textsc{dem}b cow-\textsc{distr}\\ 
\glft ‘and then the cows already came out of the water, it is said’\\ 
\endgl
\xe

\ea%[everyglpreamble=\it, exno={73.}, exnoformat=X]<ex:>
\begingl 
\glpreamble pero tumuikubutuji tibÿchecheikutuji pero echÿu bakeronunubeji echÿu tiyÿbuikÿupunu kuinaji chimuanube\\
\gla pero ti-umuiku-bu-tu-ji ti-bÿchecheiku-tu-ji pero echÿu bakeronu-nube-ji echÿu ti-yÿbui-kÿupunu kuinaji chi-imua-nube\\ 
\glb but 3i-all?-\textsc{mid}-\textsc{iam}-\textsc{rprt} 3i-come.out.of.water-\textsc{iam}-\textsc{rprt} but \textsc{dem}b cowherd-\textsc{pl}-\textsc{rprt} \textsc{dem}b 3i-shout-\textsc{am.conc.cis} \textsc{neg}-\textsc{rprt} 3-see.\textsc{irr}-\textsc{pl}\\ 
\glft ‘but all of them came out of the water, it is said, but they did not see the cowherds who came shouting, it is said’\\ 
\endgl
\xe

%--> tumuikubutuji = tumuyubutuji? = ser todos? también posible bumuikubu? = somos todos?

\ea%[everyglpreamble=\it, exno={74.}, exnoformat=X]<ex:>
\begingl 
\glpreamble titupunubuji chumunubetuji bakajane i tipunubutuji titupunubunube nechÿu chubiunubeyaetuji\\
\gla ti-tupunubu-ji chÿ-umu-nube-tu-ji baka-jane i ti-punu-bu-tu-ji ti-tupunubu-nube nechÿu chÿ-ubiu-nube-yae-tu-ji\\ 
\glb 3i-arrive-\textsc{rprt} 3-take-\textsc{pl}-\textsc{iam}-\textsc{rprt} cow-\textsc{distr} and 3i-go.up-\textsc{mid}-\textsc{iam}-\textsc{rprt} 3i-arrive-\textsc{pl} \textsc{dem}c 3-house-\textsc{pl}-\textsc{loc}-\textsc{iam}-\textsc{rprt}\\ 
\glft ‘they arrived, it is said, they took the cows and went up, they already arrived there at their houses, it is said’\\ 
\endgl
\xe

\newpage

\ea%[everyglpreamble=\it, exno={75.}, exnoformat=X]<ex:>
\begingl 
\glpreamble chipresunenube bakajane nechÿu koral\\
\gla chi-presu-ne-nube baka-jane nechÿu koral\\ 
\glb 3-captive-\textsc{possd}-\textsc{pl} cow-\textsc{distr} \textsc{dem}c enclosure\\ 
\glft ‘they closed the cows in in the enclosure’\\ 
\endgl
\xe

\ea%[everyglpreamble=\it, exno={76.}, exnoformat=X]<ex:>
\begingl 
\glpreamble entonses tikechunube: “bueno, eka bakajane enika, enikia tumuyubu pero echÿu sebo ana puro berajane chitÿpi eka benu”, tikechunubechÿ\\
\gla entonses ti-kechu-nube bueno eka baka-jane e-nika e-nik-i-a tumuyubu pero echÿu sebo e-ana puro bera-jane chi-tÿpi eka benu ti-kechu-nube-chÿ\\ 
\glb thus 3i-say-\textsc{pl} well \textsc{dem}a cow-\textsc{distr} 2\textsc{pl}-eat.\textsc{irr} 2\textsc{pl}-eat-\textsc{subord}-\textsc{irr} all but \textsc{dem}b suet 2\textsc{pl}-make.\textsc{irr} mere candle-\textsc{distr} 3-\textsc{obl} \textsc{dem}a virgin 3i-say-\textsc{pl}-3\\ 
\glft ‘then he said: “well, you can eat the cows, everything is meant for eating, but with the suet, you shall only make candles for the virgin”, he said to them’\\ 
\endgl
\xe

\ea%[everyglpreamble=\it, exno={77.}, exnoformat=X]<ex:>
\begingl 
\glpreamble “biti biyunupunatu”\\
\gla biti bi-yunupuna-tu\\ 
\glb 1\textsc{pl.prn} 1\textsc{pl}-go.back.\textsc{irr}-\textsc{iam}\\ 
\glft ‘“we will go back now”’\\ 
\endgl
\xe

\ea%[everyglpreamble=\it, exno={78.}, exnoformat=X]<ex:>
\begingl 
\glpreamble bueno tiyunupununubetuji\\
\gla bueno ti-yunupunu-nube-tu-ji\\ 
\glb well 3i-go.back-\textsc{pl}-\textsc{iam}-\textsc{rprt}\\ 
\glft ‘well, they went back, it is said’\\ 
\endgl
\xe

\ea%[everyglpreamble=\it, exno={79.}, exnoformat=X]<ex:>
\begingl 
\glpreamble echÿutu echÿu jentenube nechÿu chubiunubeyaeji tikupaikunubetuji baka chinikianube nena tumuyubuji tijainube\\
\gla echÿu-tu echÿu jente-nube nechÿu chÿ-ubiu-nube-yae-ji ti-kupaiku-nube-tu-ji baka chi-nik-i-a-nube nena tumuyubu-ji tijai-nube\\ 
\glb\textsc{dem}b-\textsc{iam} \textsc{dem} man-\textsc{pl} \textsc{dem}c 3-house-\textsc{pl}-\textsc{loc}-\textsc{rprt} 3i-slaughter-\textsc{pl}-\textsc{iam}-\textsc{rprt} cow 3-eat-\textsc{subord}-\textsc{irr}-\textsc{pl} like all-\textsc{rprt} day-\textsc{pl}\\ 
\glft ‘that was it, the men over there in their houses slaughtered cows to eat, it seems every day, it is said’\\ 
\endgl
\xe

\newpage
\ea%[everyglpreamble=\it, exno={80.}, exnoformat=X]<ex:>
\begingl 
\glpreamble asta ke chÿbukunubeji i repue ya chanaunubetuji echÿu berajane chitÿpi benu, chubiuye bia\\
\gla {asta ke} chÿ-buku-nube-ji i repue ya chÿ-anau-nube-tu-ji echÿu bera-jane chi-tÿpi benu chÿ-ubiu-yae bia\\ 
\glb {until} 3-finish-\textsc{pl}-\textsc{rprt} and afterwards already 3-make-\textsc{pl}-\textsc{iam}-\textsc{rprt} \textsc{dem}b candle-\textsc{distr} 3-\textsc{obl} virgin 3-house-\textsc{loc} god\\ 
\glft ‘until they finished them, it is said, and then they made candles for the virgin in the church, it is said’\\ 
\endgl
\xe

\ea%[everyglpreamble=\it, exno={81.}, exnoformat=X]<ex:>
\begingl 
\glpreamble chibu echÿuji pasau chitÿpi echÿu bakeronu\\
\gla chibu echÿu-ji pasau chi-tÿpi echÿu bakeronu\\ 
\glb 3\textsc{top.prn} \textsc{dem}b-\textsc{rprt} pass 3-\textsc{obl} \textsc{dem}b cowherd\\ 
\glft ‘this is what is said to have happened to the cowherd’\\ 
\endgl
\xe
