\usepackage{langsci-optional}
\usepackage{langsci-gb4e}
\usepackage{langsci-lgr}

\usepackage{listings}
\lstset{basicstyle=\ttfamily,tabsize=2,breaklines=true}

\usepackage{pdflscape}% for rotated pages
\usepackage{afterpage} % um den Text rund um die rotated page besser zu verteilen


\usepackage{multicol}%für zweispaltige Seiten


\usepackage{amssymb}
% \usepackage{expex}
% \gathertags % write tags to external tag file, braucht man für ExPex, damit Referenzen auf Beispiele immer funktionieren
% \usepackage{eptexf}%für die römische Nummerierung von Beispielen in Fußnoten, wie auch:
% \usepackage{epltxfn}%für die römische Nummerierung von Beispielen in Fußnoten



\usepackage{longtable} %allows tables to stretch over several pages and footnotes work properly!!!
\usepackage{tabularx} %wie tabulary, nur angeblich auch mit Fußnoten möglich
% \newenvironment{tabulary}{\begin{tabularx}}{\end{tabularx}}
%\usepackage{ltablex} % commented out because tables do not float properly
% \usepackage{tabulary} %reguliert die Weite der Tabellen (sodass sie nicht über den Seitenrand ragen)
\usepackage{multirow} %zum vertikalen Zellenzusammenfügen in Tabellen
\usepackage{appendix} % Dafür, dass Appendix im Inhaltsverzeichnis auftaucht

\usepackage{subscript}%für tiefgestellten Text!
\usepackage{tabto}
\usepackage[linguistics,edges]{forest}

\usetikzlibrary{tikzmark}
\usetikzlibrary{decorations.pathreplacing}

\usepackage{langsci-branding}

