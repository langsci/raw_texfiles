%%%%%%%%%%%%%%%%%%%%%%%%%%%%%%%%%%%%%%%%%%%%%%%%%%%%
%%%                                              %%%
%%%                 Metadata                     %%%
%%%          fill in as appropriate              %%%
%%%                                              %%%
%%%%%%%%%%%%%%%%%%%%%%%%%%%%%%%%%%%%%%%%%%%%%%%%%%%%

\renewcommand{\title}{How mobile robots can self-organize a vocabulary}  %look no further, you can change those things right here.
\newcommand{\subtitle}{}
\newcommand{\BackTitle}{\title} % Change if BackTitle != Title
\newcommand{\BackBody}{One of the hardest problems in science is the symbol grounding problem, a question that has intrigued philosophers and linguists for more than a century. With the rise of artificial intelligence, the question has become very actual, especially within the field of robotics. The problem is that an agent, be it a robot or a human, perceives the world in analogue signals. Yet humans have the ability to categorise the world in symbols that they, for instance, may use for language.\\This book presents a series of experiments in which two robots try to solve the symbol grounding problem. The experiments are based on the language game paradigm, and involve real mobile robots that are able to develop a grounded lexicon about the objects that they can detect in their world. Crucially, neither the lexicon nor the ontology of the robots has been preprogrammed, so the experiments demonstrate how a population of embodied language users can develop their own vocabularies from scratch.}
%\dedication{Change dedication in localmetadata.tex}
%\typesetter{Change typesetter in localmetadata.tex}
%\proofreader{Change proofreaders in localmetadata.tex}
\renewcommand{\author}{Paul Vogt}
\newlength{\csspine} 
\newlength{\bodspine}
\setlength{\csspine}{16.1878264mm} % Please calculate: Total Page Number (excluding cover, usually (Total Page - 3)) * 0.0572008 mm for Createspace
\setlength{\bodspine}{40mm} % Please calculate: Total Page Number (excluding cover) * BODFACTOR + BODABS for Books on Demand
\newcommand{\lsISBN}{978-3-944675-43-5}                     
\newcommand{\lsSeries}{cmle} % use lowercase acronym, e.g. sidl, eotms, tgdi
\newcommand{\lsSeriesNumber}{2} %will be assigned when the book enters the proofreading stage
\newcommand{\lsURL}{http://langsci-press.org/catalog/book/51} % contact the coordinator for the right number