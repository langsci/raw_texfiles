\documentclass[output=paper,colorlinks,citecolor=brown]{langscibook}
\ChapterDOI{10.5281/zenodo.13383799}
\author{Ivan Yuen\orcid{0000-0002-3238-0402}\affiliation{Saarland University} and Bistra Andreeva\orcid{0000-0003-2774-1346}\affiliation{Saarland University} and Omnia Ibrahim\orcid{0000-0002-3649-7376}\affiliation{Saarland University} and Bernd M{\"o}bius\orcid{0000-0003-3065-9984}\affiliation{Saarland University}}

\title[Word-final syllable duration in German polysyllabic words]{Prosodic factors do not always suppress discourse or surprisal factors on word-final syllable duration in German polysyllabic words}
\abstract{
Predictability is known to influence acoustic duration \citep[e.g.,][]{Ibrahim2022} and prosodic factors such as accenting and boundary-related lengthening have been postulated to account for this effect \citep[e.g.,][]{Aylett2004}. However, it has also been shown that other factors such as information status or speech styles could contribute to acoustic duration \citep[e.g.][]{Baker2009}. This raises the question as to whether acoustic duration is primarily subject to the influence of prosody that reflects linguistic structure including predictability. The current study addressed this question by examining the acoustic duration of word-final syllables in polysyllabic words in DIRNDL, a German radio broadcast corpus \citep[e.g.][]{Eckart2012}. We analysed polysyllabic words followed by an intermediate phrase or an intonational phrase boundary, with or without accenting, and with given or new information status. Our results indicate that the acoustic duration of the word-final syllable was subject to the effect of prosodic boundary for long host words, in line with \citet{Aylett2004}; however, we also observed additional effects of information status, log surprisal and accenting for short host words, in line with \citet{Baker2009}. These results suggest that acoustic duration is subject to the influence of prosodic (e.g., boundary and accenting) and linguistic factors (e.g., information status and surprisal), and that the primacy of prosodic factors impacting on acoustic duration is further constrained by some intrinsic durational constraints, for example word length.}

\IfFileExists{../localcommands.tex}{
  \addbibresource{../localbibliography.bib}
   \usepackage{langsci-optional}
\usepackage{langsci-gb4e}
\usepackage{langsci-lgr}

\usepackage{listings}
\lstset{basicstyle=\ttfamily,tabsize=2,breaklines=true}

%added by author
% \usepackage{tipa}
\usepackage{multirow}
\graphicspath{{figures/}}
\usepackage{langsci-branding}

   
\newcommand{\sent}{\enumsentence}
\newcommand{\sents}{\eenumsentence}
\let\citeasnoun\citet

\renewcommand{\lsCoverTitleFont}[1]{\sffamily\addfontfeatures{Scale=MatchUppercase}\fontsize{44pt}{16mm}\selectfont #1}
  
   %% hyphenation points for line breaks
%% Normally, automatic hyphenation in LaTeX is very good
%% If a word is mis-hyphenated, add it to this file
%%
%% add information to TeX file before \begin{document} with:
%% %% hyphenation points for line breaks
%% Normally, automatic hyphenation in LaTeX is very good
%% If a word is mis-hyphenated, add it to this file
%%
%% add information to TeX file before \begin{document} with:
%% %% hyphenation points for line breaks
%% Normally, automatic hyphenation in LaTeX is very good
%% If a word is mis-hyphenated, add it to this file
%%
%% add information to TeX file before \begin{document} with:
%% \include{localhyphenation}
\hyphenation{
affri-ca-te
affri-ca-tes
an-no-tated
com-ple-ments
com-po-si-tio-na-li-ty
non-com-po-si-tio-na-li-ty
Gon-zá-lez
out-side
Ri-chárd
se-man-tics
STREU-SLE
Tie-de-mann
}
\hyphenation{
affri-ca-te
affri-ca-tes
an-no-tated
com-ple-ments
com-po-si-tio-na-li-ty
non-com-po-si-tio-na-li-ty
Gon-zá-lez
out-side
Ri-chárd
se-man-tics
STREU-SLE
Tie-de-mann
}
\hyphenation{
affri-ca-te
affri-ca-tes
an-no-tated
com-ple-ments
com-po-si-tio-na-li-ty
non-com-po-si-tio-na-li-ty
Gon-zá-lez
out-side
Ri-chárd
se-man-tics
STREU-SLE
Tie-de-mann
}
%   \boolfalse{bookcompile}
%   \togglepaper[23]%%chapternumber
}{}

\begin{document}
\SetupAffiliations{mark style=none}
\maketitle

\section{Introduction}
Information-theoretic measures have been used to account for variabilities in word length \citep[e.g.,][]{Piantadosi2011}, phrase duration \citep[e.g.,][]{Arnon2013}, word duration \citep[e.g.,][]{Baker2009, Seyfarth2014}, syllable duration \citep[e.g.,][]{vanSon2003, Aylett2004}, vowel spectra \citep[e.g.,][]{Aylett2006, Brandt2021}, vowel and consonant dispersion \citep[e.g.,][]{Malisz2018}, and lenition \citep[e.g.,][]{Cohen2017}. These studies suggest that speakers’ choice of phonetic forms is guided by informativity-based considerations, which include frequency \citep[e.g.,][]{Gahl2008, Arnon2013}, or contextual predictability \citep[e.g.,][]{Aylett2004, Aylett2006, Baker2009, Seyfarth2014, Piantadosi2011}.

Adopting the information-theoretic perspective, \citet{Aylett2004} postulated the smooth signal redundancy hypothesis (SSRH) to explain the acoustic variability of duration in English. According to this hypothesis, prosody directly affects speech acoustics through assignment of prosodic prominence or boundary. For the sake of robust optimal communication, this prosodic influence is inversely related to the influence from language predictability/redundancy factors in order to maintain smooth transmission of information (i.e., to avoid any abrupt surge or dip in information density). While language predictability will induce short acoustic duration, prominence will induce long acoustic duration. For instance, \textit{nine} in \textit{a stitch in time saves nine} will have shorter duration than the same word in \textit{the winning number is nine}, because nine is more predictable and less prominent in the former than the latter. To test their hypothesis, they analysed syllable duration in the HCRC Map Task Corpus \citep{Anderson1991}, and found a significant inverse relationship between the language predictability factors (e.g., log word frequency, syllabic trigram probability and word mention) and syllable duration. They also observed significant influences on syllable duration from a range of prosodic factors (e.g., lexical stress, phrasal stress, different types of prosodic boundary). Further regression analysis revealed that the model with prosodic prominence structure accounted for most of the variance in the syllable duration, with little unique significant contribution from the language predictability factors. On such basis, they argued that prosody absorbs the effects from language predictability to influence speech acoustics. In other words, prosody mediates language predictability. 

However, other studies showed that predictability can directly influence duration, rather than be mediated through prosodic prominence structure. For instance, \citet{Baker2009} examined two predictability factors on reduction (i.e., word duration) in two speech styles: plain vs. clear. Plain style was defined as one in which a speaker hypo-articulates because listeners do not have difficulty in perceiving one’s speech; and clear style as one in which the speaker hyper-articulates because listeners might have difficulty in perceiving one’s speech. The two predictability factors were first vs. second mention, and word frequency. As expected, word duration in clear speech is longer than that in plain speech for first and second mentions. Similarly, word duration is longer for first mention than second mention in both speech styles. These patterns remained, irrespective of the presence vs. absence of a prosodic break around the measured targets, or the presence vs. absence of accenting on the target stimuli. On the whole, word frequency is negatively correlated with first-mention word duration in plain and clear speech (short duration for high frequency words). However, this frequency-induced reduction effect is exaggerated on second mention duration in plain speech, not clear speech.  \citet{Baker2009} found a significant positive correlation between word frequency and second mention in plain speech (suggesting high frequency words undergo more second mention reduction), but not in clear speech. These findings then support the idea that other non-prosody factors such as discourse structure, lexicon-based frequency (predictability) and speech styles contribute to acoustic realization, with the implication that the connection between predictability and acoustics can be direct, in addition to being mediated through prosody.

Consistent with \citet{Baker2009}, a recent study reported independent effects of syllable-based surprisal (one type of predictability) and Lombard speech style (i.e., speech produced in a noisy environment) on syllable duration in German lab-speech \citep{Ibrahim2022}. Further evidence for the effect of surprisal was observed on word-final syllable duration in German preceding an intonational phrase (IP) boundary \citep{Andreeva2020}. Based on the analysis of the DIRNDL corpus \citep{Eckart2012}, the authors showed that the duration of a German word-final syllable with high surprisal was longer than that with low surprisal. Critically, they found an interaction between surprisal and the strength of an IP boundary, with the effect of surprisal being more pronounced in the presence of a strong IP boundary. The presence of such interaction indicates that both prosodic boundary and surprisal contributed to the acoustic duration of the word-final syllable in German. Interestingly, the duration of a word-final syllable preceding a strong IP boundary was shorter than that preceding a weak IP boundary. 

However, this study did not differentiate between monosyllabic and polysyllabic words preceding an IP boundary. While the location of lexical stress does not vary for monosyllabic words, this cannot be said for polysyllabic words. The number of syllables may then be confounded with pitch accenting (associated with lexical stress) in influencing the measured syllable duration. Pitch accenting is typically associated with focus or information status \citep[e.g.,][]{Cooper1985, Cruttenden1993, Cruttenden2006}, although the distribution and use of accenting can be language-specific  \citep[e.g.,][]{Swerts2002}. For instance, Swerts and colleagues reported Dutch speakers accenting new and contrastive information, but not given information. \citet{Cruttenden1993} showed that speakers of English de-accent repeated or old information; however, this tendency will be attenuated in the presence of a contrast in the discourse. As such, this raises further questions as to whether accenting or information status, or both, contribute to the measured duration and whether or not such effect(s) will interact with surprisal. Besides, it remains to be seen whether the effect of surprisal continues to be observed for other types of prosodic boundary, say intermediate phrase (ip) boundary.

To better understand how discourse-based structure (i.e., information structure), language predictability (i.e., surprisal), prosody (i.e., presence vs. absence of accenting or prosodic boundary types), and/or their interactions might account for the acoustic variability of duration, the current study used broadcast data from an annotated German corpus (DIRNDL) to examine any effects of information status (an aspect of discourse-based information structure) and syllable-based surprisal on word-final syllable durations adjacent to an intonational or an intermediate phrase boundary.

Given the previous finding from \citet{Aylett2004}, we expected prosody (through boundary-related lengthening or accenting) to largely account for the acoustic duration of word-final syllables, mediating any effects of surprisal or discourse (e.g., information status). However, according to observations from \citet{Baker2009}, \citet{Andreeva2020} and \citet{Ibrahim2022}, we also expected prosodic, surprisal and discourse factors (or their interactions) to contribute to the acoustic duration. 

 \section{Method}
We extracted polysyllabic words from the DIRNDL corpus to empirically test whether information status and syllable-based surprisal moderate word-final syllable duration in two prosodic boundary types in German. The DIRNDL corpus (Discourse Information Radio News Database for Linguistic analysis) consists of 5 hours of audio news recordings in German from 9 speakers (5M, 4F) with prosodic annotations for pitch accent types and boundaries according to the GToBI(S) framework \citep{Mayer1995}. The accompanying written scripts were annotated for information status (see \cite{Eckart2012} for details of corpus construction and segmentation).

 \subsection{Data selection criteria}
A total of 3716 polysyllabic words were identified to occur before either an intermediate phrase (as denoted by ip) or an intonational phrase (as denoted by IP) boundary in the DIRNDL corpus. The word-final syllable constitutes the target syllable because it occurs immediately adjacent to a prosodic boundary. As not all of the identified polysyllabic words before a prosodic boundary were annotated for lexical information status, the data set was further trimmed to include only those with a clearly specified information status. Note that information status was grouped into two levels: given vs. new. Items annotated as “accessible” in the DIRNDL corpus were classified as “given” in the current study. This procedure reduced the data set to a total of 2907 items for statistical analysis. Information related to the host word containing the target syllable were extracted from the DIRNDL corpus: speaker identity, speaker gender, identity of the orthographically transcribed host word, phonemic transcription of the target syllable, pitch accent type (if present) for the host word, prosodic boundary (i.e., intermediate or intonational), and lexical information status. 

\subsection{Language modelling}
\largerpage[2]
We estimated the syllable-based surprisal measure in the current study from language models based on the deWaC (deutsches Web as Corpus) corpus \citep{Baroni2009}. The corpus is a collection of web-crawled data containing about 1.7 billion word tokens and 8 million word types from a diverse range of genres such as newspaper articles and chat messages. The corpus was first pre-processed and normalized using German Festival \citep{Mohler2000}. This procedure consisted of removing unnecessary/irrelevant/duplicate document information, for example, web-specific structures such as HTML structures or long lists. After pre-processing, the normalized corpus was divided into a training set (80\%) and a test set (20\%). Syllable-based trigram language models including word boundary as a unit were trained on the training set using the SRILM toolkit \citep{Stolcke2002}. All language models underwent Witten-Bell smoothing \citep{Witten1991}. The best-performing trained language model was then used as the default to calculate the conditional probability of a syllable, given the preceding context, i.e.,

\begin{equation}
S(\text{unit}_{i}) = - \log_{2} P(\text{unit}_{i} | \text{context})
\end{equation}

where $S$ = surprisal and $P$ = probability \citep{Hale2016}. The context consisted of two units\slash states: syllable or\slash and word boundary. The conditional probability constituted the syllable-based surprisal measure for the target syllables.

\subsection{Analysis}
 Prior to the main analysis, we first checked the estimated surprisal values of the target syllables in the polysyllabic words preceding the two prosodic boundaries and observed two patterns: (a) target syllables preceding an intonational phrase boundary (IP) had overall higher surprisal values than those preceding an intermediate phrase boundary (ip) when the host word contained 2 or 3 syllables, (b) target syllables preceding an intermediate phrase boundary had overall higher surprisal values than those preceding an intonational phrase boundary when the host word contained 4 to 8 syllables. In (a) higher surprisal values were associated with an intonation phrase boundary (IP); whereas in (b) higher surprisal values were associated with an intermediate phrase boundary (ip) (see \figref{fig1}). 
 Because of this, we divided the full data set of 2907 polysyllabic words into two separate data sets to de-confound the effect of surprisal from that of prosodic boundary: 2317 words with no more than 3 syllables and 590 words with no more than 8 syllables (but at least 4 syllables). The former was referred to as \textit{short words}, and the latter as \textit{long words} hereafter. A custom Python script was then used to extract durations of the word-final target syllables from the DIRNDL corpus.

\begin{figure}
\includegraphics[width=.66\textwidth]{8_Fig1.pdf}
\caption{Mean syllable-based log surprisal values of the final syllables according to the number of syllables in the polysyllabic words and prosodic boundary}
\label{fig1}
\end{figure}

The duration of the word-final syllable was the dependent variable. Predictors included prosodic boundary type (intermediate vs. intonational), information status (given vs. new), word length (short vs. long), presence vs. absence of a pitch accent, log surprisal of the word-final syllable, prosodic boundary * information status interaction, presence vs. absence of a pitch accent * information status interaction, log surprisal * prosodic boundary interaction, and log surprisal * information status interaction. Random factors included speaker identity and syllable identity.

\section{Results}
Linear mixed effects models were then fit to the dependent variable, namely word-final syllable duration, using the lme4 package \citep{Bates2015} in R \citep{R2022}. Multiple random structures were first constructed and compared, using AIC  (Alkaike Information Criterion) to determine the optimal random structure as the baseline model. The baseline random structure included by-speaker and by-item intercepts, with prosodic boundary by-speaker slope. Predictors were then included in the baseline model to construct simple and interactive models, which were compared using AIC in order to determine the optimal predictive model. In case of singularity, the complexity of a model structure was reduced to minimize overfitting. Statistical significance of the predictors in the optimal model was then evaluated using the anova() function with Satterthwaite to approximate degrees of freedom. The same procedures were followed in all analyses below. The factors included prosodic boundary, information status, log surprisal of the word-final syllable, presence vs. absence of a pitch accent for the host word, word length, and their interactions. 
The omnibus analysis revealed significant effects of prosodic boundary, presence vs. absence of a pitch accent, log surprisal, with a significant 2-way prosodic boundary * word length interaction and a significant 3-way prosodic boundary * information status * word length interaction (\tabref{table:Table1}). To better understand the 3-way interaction, we analysed short and long words separately.

\begin{table}
\begin{tabular}{l S[table-format=2.1] c S[table-format=<1.4{***}]}
\lsptoprule
{Factors} & {F} & {df} & $p$\\ \midrule
Prosodic boundary (PB)  & 17   & 1 & <0.001{***}\\
Information status (IS) & 2.6  & 1 & .11\\
Log surprisal (S)       & 46.9 & 1 & <0.0001{***}\\
Word length (WL)        & 1.61 & 1 & .21\\
Presence vs. absence    & 77.9 & 1 & <0.0001{***}\\
\quad of pitch accent (PA)    & \\
PB * IS      & 1.9 & 1 & .17\\
PB * WL      & 4.9 & 1 & 0.03{*}\\
IS * WL      & .5  & 1 & .5\\
IS * PA      & .5  & 1 & .5\\
PB * S       & 3   & 1 & .08\\
IS * S       & 1.1 & 1 & .3\\
PB * IS * WL & 7.5 & 1 & 0.006{**}\\
\lspbottomrule
\end{tabular}
\caption{Statistical results of linear mixed effects modelling on word-final syllable durations in all polysyllabic words. \textit{The model: {\textasciitilde} prosodic boundary * information status * word length + presence vs. absence of accent * information status + prosodic boundary * log surprisal + information status * log surprisal + (prosodic boundary {\textbar} speaker) + (prosodic boundary {\textbar} syllable item)}}
\label{table:Table1}
\end{table}

\subsection{Final syllable duration in short words}
\figref{fig2} illustrates the word-final syllable duration in accented or unaccented host words preceding an intermediate or intonational phrase boundary, labelled as having the information status of either “given” or “new”. The word-final syllable duration was increasingly longer when the prosodic boundary immediately following the host word became stronger (i.e., intermediate vs. intonational phrase) and this pattern was magnified when the host word was accented, as well as when the host word was labelled as “new” information. \tabref{table:2} summarizes the mean and standard deviation of word-final syllable durations in the 2 prosodic boundary (i.e. intermediate vs. intonational) $\times$ 2 information status (i.e. given vs. new) $\times$ 2 accenting (i.e. with vs. without) conditions.

\begin{figure}
\includegraphics[width=.66\textwidth]{8_Fig2.pdf}
\caption{Mean word-final syllable durations in short words, with the following intermediate (ip) or intonational phrase (IP) boundary, with given or new information status, and with or without accent, with +/- 1 SD}
\label{fig2}
\end{figure}

\begin{table}
\begin{tabularx}{0.8\textwidth}{QQQQQ}
\lsptoprule
{A}ccenting & {I}nformation status & {P}rosodic boundary & {N}o. of items & {M}ean in ms ({SD})\\ \hline
 No & Given & ip & 230 & 225 (72)\\
  &  & IP & 166 & 247 (74)\\
  & New & ip & 724 & 222 (73)\\
  & & IP & 736 & 259 (94)\\
 Yes & Given & ip & 64 & 296 (57)\\
  & & IP & 40 & 354 (96)\\
  & New & ip & 173 & 335 (80)\\
  & & IP & 184 & 403 (99)\\
\lspbottomrule
\end{tabularx}
\caption{Mean (SD) word-final syllable durations in short words, with the following intermediate (ip) or intonational (IP) phrase boundary, with given or new information status, and with or without accenting}
\label{table:2}
\end{table}

Linear mixed effects models were then fit to the dependent variable: word-final syllable duration in short words. The factors in the optimal predictive model included prosodic boundary, information status, log surprisal of the word-final syllable, presence vs. absence of a pitch accent for the host word, and prosodic boundary * information status interaction. 

The structure of the final model was {\textasciitilde} prosodic boundary * information status + log suprisal + presence vs. absence of pitch accent + (prosodic boundary {\textbar} speaker) + (prosodic boundary {\textbar} syllable item), with significant main effects of prosodic boundary, information status, log surprisal, presence vs. absence of pitch accent and the significant prosodic boundary * information status interaction (\tabref{table:3}).

\begin{table}
\begin{tabular}{l S[table-format=2.1] c S[table-format=<1.4{***}]}
\lsptoprule
 {F}actors & {F} & df & {$p$}\\ \midrule
 Prosodic boundary (PB)  & 18.3 & 1 & .001{**}\\
 Information status (IS) & 4.3 & 1 & .04{*}\\
 Log surprisal (S)       & 64.5 & 1 & <.0001{***}\\
 Presence vs. absence  & 93.3 & 1 & <.0001{***}\\
 \quad of pitch accent (PA) & \\
 PB * IS & 17.7 & 1 & <.0001{***}\\
\lspbottomrule
\end{tabular}
\caption{Statistical results of linear mixed effects modelling on word-final syllable duration in all short words. \textit{The model: {\textasciitilde} prosodic boundary * information status + log suprisal + presence vs. absence of pitch accent + (prosodic boundary {\textbar} speaker) + (prosodic boundary {\textbar} syllable item)}}
\label{table:3}
\end{table}

As expected, the word-final syllable duration was longer preceding an intonational phrase than an intermediate phrase boundary (prosodic boundary effect). It was longer when the host word contained “new” rather than “given” information (information status effect). It was longer when the host word was accented as opposed to unaccented (presence vs. absence of a pitch accent effect).  It was also longer when the word-final syllable had high log surprisal (log surprisal effect). However, the significant prosodic boundary * information status interaction suggests that the effect of information status on word-final syllable duration was magnified when the immediately adjacent boundary constitutes an intonational phrase. These results were more in line with the predictions from \citet{Baker2009} than those from \citet{Aylett2004}. Counter to our expectations, log surprisal did not interact with information status or prosodic boundary.

To further investigate this, we sub-divided our data into words with vs. without accenting and separately analysed them. The optimal model for the data with accenting included predictors: prosodic boundary, information status, log surprisal, prosodic boundary * information status interaction, information status * log surprisal interaction, and prosodic boundary * log surprisal interaction. The model structure was {\textasciitilde} prosodic boundary * information status + prosodic boundary * log surprisal + information status * log surprisal + (prosodic boundary {\textbar} speaker) + (prosodic boundary {\textbar} syllable item), with log surprisal and information status * log surprisal interaction reaching statistical significance (\tabref{table:4}). Unlike the analysis of all short words, we observed the effect of log surprisal and the log surprisal * information status interaction.

\begin{table}
\begin{tabular}{l S[table-format=2.1] c S[table-format=<1.4{***}]}
\lsptoprule
 {F}actors & {F} & df & {$p$}\\ \midrule
 Prosodic boundary (PB) & .9 & 1 & .34\\
 Information status (IS) & .7 & 1 & .4\\
 Log surprisal (S) & 20.4 & 1 & <.0001{***}\\
 PB * IS & 2.5 & 1 & .12\\
 S * PB & 1.4 & 1 & .24\\
 S * IS & 3.9 & 1 & .05{*}\\
\lspbottomrule
\end{tabular}
\caption{Statistical results of linear mixed effects modelling on word-final syllable duration in short words with accenting. \textit{The model: {\textasciitilde} prosodic boundary * information status + prosodic boundary * log surprisal + information status * log surprisal + (prosodic boundary {\textbar} speaker) + (prosodic boundary {\textbar} syllable item)}}
\label{table:4}
\end{table}

The optimal model for the data without accenting included the following predictors: prosodic boundary, information status, log surprisal, prosodic boundary * log surprisal interaction, prosodic boundary * information status interaction and information status * log surprisal interaction. The structure of the final model for the data without accenting was {\textasciitilde} prosodic boundary * information status + prosodic boundary * log surprisal + information status * log surprisal + (prosodic boundary {\textbar} speaker) + (prosodic boundary {\textbar} syllable item), with significant effects of prosodic boundary, information status, log surprisal and the prosodic boundary * information status interaction (\tabref{table:5}). These results were consistent with the results in the analysis of all short words, suggesting that the overall pattern might be driven primarily by the data without accenting (which contained more items overall).

\begin{table}
\begin{tabular}{l S[table-format=2.1] c S[table-format=<1.4{***}]}
\lsptoprule
 {F}actors & {F} & df & {$p$}\\ \midrule
 Prosodic boundary (PB) & 25.9 & 1 & <.001{***}\\
 Information status (IS) & 5.9 & 1 & .02{*}\\
 Log surprisal (S) & 40.2 & 1 & <.0001{***}\\
 PB * IS & 13.3 & 1 & <.001{***}\\
\lspbottomrule
\end{tabular}
\caption{Statistical results of linear mixed effects modelling on word-final syllable duration in short words without accenting. \textit{The model: {\textasciitilde} prosodic boundary * information status + prosodic boundary * log surprisal + information status * log surprisal + (prosodic boundary {\textbar} speaker) + (prosodic boundary {\textbar} syllable item)}}
\label{table:5}
\end{table}

The interaction of log surprisal and information status on word-final syllable duration is illustrated in \figref{fig3} for accented words and the lack of interaction in \figref{fig4} for unaccented words. In Figures~\ref{fig3} and \ref{fig4}, the word-final syllable duration was lengthened when the log surprisal value was high, as reflected in the positive correlation. However, in \figref{fig3}, the slope between the word-final syllable durations and the log surprisal values was conspicuously steeper when host words contained new rather than given information. That is, the effect of surprisal on a word-final syllable duration was attenuated when the host word contained “given” information and accenting. Unlike \figref{fig3}, the differences in the slope between given vs. new information in \figref{fig4} were less obvious.

\begin{figure}
\includegraphics[width=.66\textwidth]{8_Fig3.pdf}
\caption{Scatterplots relating word-final syllable durations on the y-axis to syllable-based log suprisal values on the x-axis in accented short host words labelled with given or new information status}
\label{fig3}
\end{figure}

\begin{figure}
\includegraphics[width=.66\textwidth]{8_Fig4.pdf}
\caption{Scatterplots relating word-final syllable durations on the y-axis to syllable-based log suprisal values on the x-axis in unaccented short host words labelled with given or new information status}
\label{fig4}
\end{figure}

 \subsection{Final syllable duration in long words}
A total of 590 long words were included in this analysis. \figref{fig5} illustrates the patterns of word-final syllable duration preceding two different prosodic boundaries, with either given or new information status, and with or without accenting. Generally, the word-final syllable duration with new information status was longer than that with given information status. This held for unaccented words preceding an intermediate phrase or an intonational phrase boundary; however, the pattern was not as consistent for accented words. \tabref{table:6} summarizes the mean duration with SD.

\begin{figure}
\includegraphics[width=.66\textwidth]{8_Fig5.pdf}
\caption{Mean word-final syllable durations in long words, followed by an intermediate (ip) or intonational phrase (IP) boundary, with given or new information status, and with or without accenting, with +/− 1 SD}
\label{fig5}
\end{figure}

\begin{table}
\begin{tabularx}{0.8\textwidth}{QQQQQ}
\lsptoprule
{A}ccenting & {I}nformation status & {P}rosodic boundary & {N}o. of items & {M}ean in ms ({SD})\\ \hline
 No & Given & ip & 64 & 193 (63)\\
  &  & IP & 60 & 202 (61)\\
  & New & ip & 232 & 206 (57)\\
  & & IP & 213 & 221 (63)\\
 Yes & Given & ip & 4 & 286 (56)\\
  & & IP & 2 & 260 (13)\\
  & New & ip & 10 & 278 (55)\\
  & & IP & 5 & 272 (46)\\
\lspbottomrule
\end{tabularx}
\caption{Mean (SD) word-final syllable duration in long words, followed by an intermediate (ip) or an intonational (IP) phrase boundary, with either given or new information status, and with or without accenting}
\label{table:6}
\end{table}

The optimal model included prosodic boundary, information status, log surprisal, presence vs. absence of pitch accent and prosodic boundary * information status interaction as predictors. The model structure of the optimal model was {\textasciitilde} prosodic boundary * information status + log surprisal + presence vs. absence of pitch accent + (prosodic boundary {\textbar} speaker) + (1 {\textbar} syllable item), with the significant effect of prosodic boundary (\tabref{table:7}).

\begin{table}[t]
\begin{tabular}{l S[table-format=2.1] c S[table-format=1.3{**}]}
\lsptoprule
 {F}actors & {F} & df & {$p$}\\\midrule
 Prosodic boundary (PB) & 15.3 & 1 & .002{**}\\
 Information status (IS) & 2.3 & 1 & .13\\
 Log surprisal (S) & .05 & 1 & .82\\
 Presence vs. absence  & 2.5 & 1 & .11\\
 \quad of pitch accent (PA) & \\
 PB * IS & 2.2 & 1 & .14\\
\lspbottomrule
\end{tabular}
\caption{Statistical results of linear mixed effects modelling on the final syllable duration in long words. \textit{The model: {\textasciitilde} prosodic boundary * information status + log surprisal + presence vs. absence of pitch accent + (prosodic boundary {\textbar} speaker) + (1 {\textbar} syllable item)}}
\label{table:7}
\end{table}

Counter to the prediction from \citet{Baker2009}, neither the effect of log surprisal nor its interaction with other predictors were observed on the word-final syllable duration (as exemplified by the lack of log surprisal * information status interaction in \figref{fig6}). These results are more in line with the predictions from \citet{Aylett2004}.

\begin{figure}
\includegraphics[width=.66\textwidth]{8_Fig6.pdf}
\caption{Scatterplots relating word-final syllable durations on the y-axis to syllable-based log surprisal values on the x-axis in all long words with given or new information status}
\label{fig6}
\end{figure}

\section{Discussion}
The goal of the current investigation was to empirically test whether discourse factors such as information status, prosodic factors such as prosodic boundary type, accenting, and surprisal (or their interactions) would contribute to the acoustic realization of the word-final syllable duration in polysyllabic words. We expected the word-final syllable duration with new information status to be longer than that with given information status \citep[e.g.,][]{Fowler1987, Lam2010}. We also expected the word-final syllable duration to be longer when followed by an intonational phrase boundary rather than an intermediate phrase boundary \citep[e.g.,][]{Wightman1992}. We also expected word-final syllable duration in an accented word to be longer than that in an unaccented word \citep[e.g.,][]{Turk1999}. We further hypothesized that the word-final syllable duration with high log surprisal would be longer than that with low log surprisal \citep[e.g.,][]{Ibrahim2022}. We postulated that prosodic factors would largely account for the acoustic duration without unique contribution from other factors, in line with \citet{Aylett2004}; however, other factors in addition to the prosodic factors might contribute to the measured duration, in line with \citet{Baker2009}. 

Our overall results on short polysyllabic words are consistent with the effects of boundary-related lengthening, accentual lengthening, surprisal and information status on the acoustic realization of syllable duration in previous studies. These results are more in line with \citet{Baker2009} rather than \citet{Aylett2004}, because of additional contributions (including interactions) from prosodic and language predictability factors. However, our overall results on \textit{long} polysyllabic words are more in line with \citet{Aylett2004} rather than \citet{Baker2009}, because we observed only the prosodic boundary effect. Perhaps, polysyllabic shortening constrains the extent to which these various factors can modify syllable duration (i.e., a word length constraint). Despite such a constraint, the durational adjustment is primarily attributable to prosodic boundary type. This suggests a strong tendency for duration to maintain information about a major prosodic boundary. But this interpretation has to be taken with caution because of the low statistical power arising from a relatively smaller data set containing long words (a total of 590 items) than that containing short words (a total of 2317 items). 

Depending on whether or not host words were accented, different results were revealed in the data set containing short words. When host words were accented, only the effect of log surprisal and the log surprisal * information status interaction reached statistical significance. However, when host words were unaccented, prosodic boundary, information status, log surprisal and prosodic boundary * information status interaction significantly predicted the word-final syllable durations. As the syllable duration in an unaccented word is shorter than that in an accented word (as reflected in the different y-intercepts in \figref{fig4} vs. \figref{fig3}), the former might have more room than the latter to accommodate durational increases from multiple sources, resulting in more reported effects and interactions. This interpretation may account for the different results between the unaccented vs. accented words on the assumption that there is an upper duration limit, which seems to be the case, because the majority of the data for word-final syllable duration fell below 600ms (\figref{fig3} and \figref{fig4}). An alternative interpretation may be related to the statistical power of the relatively small sample size for accented words (461 items) to detect multiple effects as compared to that for unaccented words (1856 items).

Recall that one of our questions is whether information status might interact with the log surprisal effect, presence vs. absence of accenting effect and/or the prosodic boundary effect. Since information status did not have any effect on long words, our discussion focuses on short words. In short \textit{accented} words, information status interacts with log surprisal. The interaction occurs because the effect of log surprisal on word-final syllable duration was stronger for words with new information status than those with given information status. However, no such interaction was observed in \textit{unaccented} words. In that case, information status interacts with prosodic boundary instead. This interaction is due to the effect of prosodic boundary on pre-boundary syllable duration which is stronger for words with new information than those with given information. In other words, information status additionally exaggerates the effect of log surprisal in accented words, and the effect of prosodic boundary in unaccented words. These results suggest that information status (a discourse-based factor) cannot be subsumed under surprisal (a language predictability factor) or prosodic boundary (a prosodic factor). 

These observations based on the corpus data raise further questions as to whether a speaker will differentially weigh these factors (language predictability, information structure and prosody) according to speech styles, e.g., scripted vs. spontaneous or formal vs. informal speech. Speech styles could affect how utterances will be structured, because a speaker might adopt different production planning strategies to cope with time pressure for example. It is possible that less scripted styles might induce more pause breaks. Since these structural considerations can also affect speech acoustics \citep[e.g.,][]{Watson2004}, it remains to explore in future questions as to how other acoustic cues such as pause might relate to pre-boundary syllable duration.

 \section{Conclusion}
Our overall results showed that information status and surprisal do not encode the same type of linguistic information and that neither information status nor surprisal are redundant. Both can influence the acoustic realization of word-final syllable durations. Critically, our results showed that prosodic factors such as prosodic boundary type could largely account for the acoustic duration as predicted by \citet{Aylett2004} for long words on the one hand, but that factors other than prosody also contribute to the measured acoustic duration as predicted by \citet{Baker2009} for short words.  In other words, information status as a discourse factor can interact with language predictability and prosodic factors to influence the measured acoustic syllable duration, but these interactions are subject to some duration constraint(s) arising from word length. 

\section*{Acknowledgements}
This research was funded by the Deutsche Forschungsgemeinschaft (DFG, German Research Foundation), Project-ID 232722074 (SFB 1102 Information Density and Linguistic Encoding).

\printbibliography[heading=subbibliography]
\end{document}
