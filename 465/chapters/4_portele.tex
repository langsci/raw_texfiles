\documentclass[output=paper,colorlinks,citecolor=brown]{langscibook}
\ChapterDOI{10.5281/zenodo.13383789}
\title[Choosing referential expressions and their order]{Choosing referential expressions and their order: Accessibility or Uniform Information Density?} 
\author{Yvonne Portele\orcid{0000-0001-8823-4335}\affiliation{Goethe University Frankfurt} and Markus Bader\orcid{0000-0002-9765-8970}\affiliation{Goethe University Frankfurt}}

\abstract{Choosing referential expressions as well as fixing word order are among the central tasks of speakers when producing language.
\textit{Accessibility} accounts focus on the fact that the accessibility status of referents involved in the current discourse is known to influence both these tasks.
\textit{Uniform Information Density} (UID) accounts highlight the role of information transmission when producing language, with consequences for choosing referential forms as well as word order. In the current article, we compare accessibility based and UID accounts and evaluate them based on linguistic findings obtained in our language production experiments on German.
In line with the fact that neither one of the accounts claims to account for all aspects of referential as well as word order choices, we found that both accounts offer relevant insights. Whereas the pattern found for referential choices seems to be explained more straightforwardly in terms of accessibility compared to UID accounts, we identify helpful aspects offered by UID accounts in the domain of syntactic options. We suggest to include both perspectives to gain a more comprehensive picture in future production studies.}


\IfFileExists{../localcommands.tex}{
   \addbibresource{../localbibliography.bib}
   \usepackage{langsci-optional}
\usepackage{langsci-gb4e}
\usepackage{langsci-lgr}

\usepackage{listings}
\lstset{basicstyle=\ttfamily,tabsize=2,breaklines=true}

%added by author
% \usepackage{tipa}
\usepackage{multirow}
\graphicspath{{figures/}}
\usepackage{langsci-branding}

   
\newcommand{\sent}{\enumsentence}
\newcommand{\sents}{\eenumsentence}
\let\citeasnoun\citet

\renewcommand{\lsCoverTitleFont}[1]{\sffamily\addfontfeatures{Scale=MatchUppercase}\fontsize{44pt}{16mm}\selectfont #1}
  
   %% hyphenation points for line breaks
%% Normally, automatic hyphenation in LaTeX is very good
%% If a word is mis-hyphenated, add it to this file
%%
%% add information to TeX file before \begin{document} with:
%% %% hyphenation points for line breaks
%% Normally, automatic hyphenation in LaTeX is very good
%% If a word is mis-hyphenated, add it to this file
%%
%% add information to TeX file before \begin{document} with:
%% %% hyphenation points for line breaks
%% Normally, automatic hyphenation in LaTeX is very good
%% If a word is mis-hyphenated, add it to this file
%%
%% add information to TeX file before \begin{document} with:
%% \include{localhyphenation}
\hyphenation{
affri-ca-te
affri-ca-tes
an-no-tated
com-ple-ments
com-po-si-tio-na-li-ty
non-com-po-si-tio-na-li-ty
Gon-zá-lez
out-side
Ri-chárd
se-man-tics
STREU-SLE
Tie-de-mann
}
\hyphenation{
affri-ca-te
affri-ca-tes
an-no-tated
com-ple-ments
com-po-si-tio-na-li-ty
non-com-po-si-tio-na-li-ty
Gon-zá-lez
out-side
Ri-chárd
se-man-tics
STREU-SLE
Tie-de-mann
}
\hyphenation{
affri-ca-te
affri-ca-tes
an-no-tated
com-ple-ments
com-po-si-tio-na-li-ty
non-com-po-si-tio-na-li-ty
Gon-zá-lez
out-side
Ri-chárd
se-man-tics
STREU-SLE
Tie-de-mann
}
   \boolfalse{bookcompile}
   \togglepaper[23]%%chapternumber
}{}


\begin{document}
\maketitle

%-------------------------------------------------------------------%
%---------------------------- Section ------------------------------%
%-------------------------------------------------------------------%
\section{Introduction}

Suppose that a speaker of German wants to inform some interlocutors that a journalist saw a former teacher in a café and that the journalist greeted the teacher warmly. For the first part of this message, the speaker will probably use indefinite NPs for introducing the two referents in a sentence starting with the subject NP, as shown in (\ref{ex:introduction-example-s1}).

\ea \label{ex:introduction-example}
 \Subject{Ein} \Subject{Journalist} sah \Object{einen} \Object{ehemaligen} \Object{Lehrer} in einem Café.\\
\enquote*{A journalist saw a former teacher {in a} café.}
\label{ex:introduction-example-s1}
\z

For the second part of the message, the speaker then has a whole range of options available, some of which are shown in  (\ref{ex:introduction-example-s2}). 

\ea \label{ex:introduction-example-s2}
\ea \gll \Subject{Er/Dieser/Der Journalist} hat \Object{ihn/diesen/den Lehrer} herzlich begrüßt\label{ex:SO}\\
          {he/this-one/{the journalist}.\NOM} has {him/this-one/{the teacher}.\ACC} warmly greeted\\
     \glt \enquote*{He/The journalist greeted him/the teacher warmly.}
\ex \gll \Object{Ihn/Diesen/Den Lehrer} hat \Subject{er/dieser/der Journalist} herzlich begrüßt\label{ex:OS}\\
          {him/this-one/{the teacher}.\ACC} has {he/this-one/{the journalist}.\NOM} warmly greeted\\
     \glt \enquote*{He/The journalist greeted him/the teacher warmly.}
\ex \gll \Object{Er/Dieser/Der Lehrer} wurde von \Subject{ihm/diesem/dem Journalist} herzlich begrüßt\label{ex:pas}\\
          {he/this-one/{the teacher}.\NOM} was by him/this-one/{the journalist} warmly greeted\\
     \glt \enquote*{He/The teacher was greeted by him/the journalist warmly.}
\z
\z

Referents already introduced into the discourse can be realized by a variety of referential expressions differing in terms of explicitness, including pronouns (\textit{er} `he'/\textit{ihn} `him'), demonstratives (\textit{dieser} `this one.\NOM'/\textit{diesen} `this one.\ACC') and definite NPs (\textit{der}/\textit{den} \textit{Lehrer}/\textit{Journalist} `the.\NOM/the.\ACC\ teacher/journalist'), among others. With respect to the linear order and the syntactic functions of the two referents, the speaker also has several options, defined in terms of the linear position of the subject (sentence-initial or not) and voice (active or passive). Thus, as shown in (\ref{ex:introduction-example-s2}), the speaker has a choice, among others, between active subject-before-object (SO) sentences (\ref{ex:SO}), active object-before-subject (OS) sentences (\ref{ex:OS}), and passive sentences (\ref{ex:pas}).%
\footnote{When we talk of passive clauses in the following, we always mean passive clauses with a \textit{by}-phrase and the linear order subject before \textit{by}-phrase. The \textit{by}-phrase can also precede the subject, but passive clauses with this order are exceedingly rare in German; see \citet[Table 1]{Bader::Meng-18} for corpus evidence.}

Within the subfield of psycholinguistics concerned with language production, choosing referential expressions and determining word order and voice are usually considered separate topics that are investigated independently from each other. Despite this, a single notion has turned out to play a dominant role in both areas -- the notion of conceptual accessibility. Conceptual accessibility refers to the activation of referents in short- and long-term memory.%
\footnote{Conceptual accessibility contrasts with lexical or lemma accessibility, which refers to the activation status of lexical entries in the mental lexicon (see \citealt{Ferreira::Dell-00}). The latter type of accessibility can also affect the order of elements in a sentence, but for the sake of brevity, we only consider conceptual accessibility in the following.}
Referents that are in the focus of attention of speaker or hearer are highly activated and therefore easily accessible whereas referents that are outside of the focus of attention are more difficult to access. With regard to the choice of referential expressions, less explicit expressions (e.g., pronouns) are typically used for more accessible referents whereas less accessible referents are referred to by more explicit expressions (e.g., proper names or definite NPs), as captured in various referential hierarchies (e.g., \citealt{Gundel::al-93, Ariel-01}). With regard to the choice of word order, it has often been found that more accessible referents are produced before less accessible referents (e.g., \citealt{Bock::Warren-85, McDonald::al-93, Ferreira-94, Prat-Sala::Branigan-00}), which indicates that referents are produced in the order in which they become available in memory.

Accessibility-based accounts do not claim that accessibility is the only factor governing the choice of referential expressions and the choice of word order. For example, production experiments investigating various languages have repeatedly found that speakers produce almost exclusively active subject-initial sentences when the subject referent is animate and the object referent inanimate (\textit{The student bought the book}), in accordance with the claim that animate referents are more accessible than inanimate referents. When the subject referent is inanimate and the object referent is animate, the percentages of passives clauses increase (\textit{The student was impressed by the book}), as expected given that an animate object is realized as a sentence-initial subject in a passive clause. Importantly, however, even in this case, participants produce a substantial number of active sentences (\textit{The book impressed the student}), sometimes even outnumbering the production of passive sentences (e.g., \citealt{McDonald::al-93, Ferreira-94}). Thus, almost all sentences are produced in the active voice when active is favored by animacy, whereas many, but far from all, sentences are produced in the passive voice when passive is favored by animacy. This asymmetry is usually explained with recourse to the fact that active sentences are structurally less complex and much more frequent than passive sentences. Active sentences are therefore produced by default, whereas non-active sentences are produced only when favored by sufficiently strong reasons (see the principle of Plan Reuse proposed in \citealt{MacDonald-13-How-language-production}).

The last fifteen years have seen the rise of information theoretic approaches to language production in which the distribution of information takes the role that the notion of accessibility has in accessibility-based accounts. The information associated with a linguistic unit is defined in terms of the unit's probability -- the information conveyed by a unit is the higher the less probable the unit is. The formal definition of information, also called surprisal, is given in (\ref{ex::surprisal}) (from \citealt{Crocker::al-16}). Surprisal is therefore inversely related to predictability: Lower surprisal equals higher predictability and vice versa.
\begin{equation}
\text{Surprisal}(\text{unit}_i) = \log_2\dfrac{1}{P(\text{unit}_i\mid \text{context})}\label{ex::surprisal}
\end{equation}

The variable \textit{context} in formula (\ref{ex::surprisal}) is to be understood in a wide sense. For example, when considering the surprisal associated with each word in a sentence, the context includes the words preceding the current word within the sentence, but also the linguistic and non-linguistic context preceding the current sentence.

For language comprehension, surprisal has been claimed to capture the word-by-word complexity of human parsing (\citealt{Levy-08-Expectation-based-syntactic-comprehension}, see \citealt{Levy-13} for critical discussion). In particular, when a word in a sentence is extremely surprising, as for example in the case of the disambiguating word of a garden-path sentence, processing breakdown may result. For smooth communication, the speaker should therefore avoid sentences with extreme information peaks. On the other hand, the speaker should also avoid sentences containing words with extremely low information value, otherwise the hearer's resources are wasted. Taken together, the speaker should strive for sentences in which information takes neither extremely low nor extremely high values. This idea is captured in the Hypothesis of Uniform Information Density (UID) given in (\ref{ex:UID}) (see also \citealt{Fenk-Oczlon-89} and \citealt{Levy::Jaeger-06}).

\ea Uniform Information Density (UID, \citealt[25]{Jaeger-10})\label{ex:UID}\\
Within the bounds defined by grammar, speakers prefer utterances that distribute information uniformly across the signal (information density). Where speakers have a choice between several variants to encode their message, they prefer the variant with more uniform information density (ceteris paribus).
\z

Models based on UID are also called \textit{rational} models
(e.g., \citealt{Levy::Jaeger-06, Arnold::Zerkle-19, Orita::al-21}, see also \citealt{Frank::Goodman-12} for the \textit{Rational Speech Act Model}). The rationale behind this term is that speakers try to optimize information transmission to be efficient. Efficiency is usually driven by two main principles: the goal to be informative and and the goal to reduce speech cost. With regard to the choice of referential expressions, speakers can be more efficient by using shorter words and phrases (e.g., a pronoun compared to a definite description) to refer to expected referents. Assuming that a referent becomes more probable (in context) with higher accessibility, UID predicts shorter word forms to be chosen for more accessible referents (e.g., \citealt{Tily::Piantadosi-09}).
With regard to the choice of word order, UID predicts that speakers should use word order options that allow for a more uniform distribution of information compared to other options. 
Depending on the differing accessibility statuses of the referents involved in a sentence, different options, for example in terms of the order of grammatical functions or the (non-)inclusion of optional elements within the sentence (e.g., \citealt{Jaeger-10}), might be favored in terms of UID.
Like accessibility-based accounts, UID-based accounts of language production do not claim that UID is the only factor governing the choice between competing variants. 


To sum up so far, accessibility and UID have been proposed as alternative overarching influences on speakers' choices of referential expressions and word order. Since there is general agreement that  speakers' choices can only be explained in a multifactorial way, language production can well be governed by accessibility and by UID, although not necessarily to the same extent. Thus, depending on which of these two notions is considered most important, we get either accessibility-based or information-based accounts.%
\footnote{Our distinction between accessibility-based accounts and information-based accounts corresponds to \citeauthor{Arnold::Zerkle-19}'s (2019) distinction between \emph{pragmatic selection models} and \emph{rational models}.}
In this paper, we discuss a range of recent findings from our lab that bear on the question of what is more important -- accessibility or UID. \sectref{sec:referential-expressions} is devoted to the choice of referential expressions and \sectref{sec:word-order} to the issue of word order and voice. The paper ends with a summary and an outlook into future research in \sectref{sec:4_discussion}.


%-------------------------------------------------------------------%
%---------------------------- Section ------------------------------%
%-------------------------------------------------------------------%
\section{Referential expressions}\label{sec:referential-expressions}

\noindent A vast literature on the production of referential expressions has shown that speakers' and writers' choices of referential forms are influenced by the accessibility of the various referents that a message is about.%
\footnote{Instead of accessibility, the terms \textit{discourse} or \textit{cognitive status} as well as \textit{salience} and \textit{prominence} are often used in the literature more or less interchangeably. We use \textit{accessibility} as an umbrella term for these different notions.}
As pointed out in the introduction, the underlying idea of accessibility accounts is that accessibility is reflected by a particular referential expression chosen by the speaker which in turn helps the listener to identify the respective referent -- based on the chosen form; the more accessible a referent, the more reduced the chosen referential form (i.e., pronouns or null forms).

To investigate referential form choices, participants are usually presented with story continuation tasks. They are shown sentences or sentence fragments and are asked to complete the sentence or write a natural continuation to the story. In the following, we review some of the linguistic factors identified to contribute to a referent's accessibility and consequently influencing the choice of referential expressions using story continuation tasks. In line with the literature, we focus on the most studied phenomenon within this domain -- the choice of pronouns instead of definite descriptions or proper names. We will then turn to a controversial factor in the choice of referential expressions -- a referent's predictability. After a short discussion of the role of predictability for accessibility and UID accounts as well as previous results, we will continue by reinspecting some of our own data with regard to the question of predictability effects on pronoun choice.

\subsection{Factors governing the choice of pronouns}

Several studies have shown that speakers are more likely to use pronouns for referents mentioned recently (e.g., in the same or previous sentence) compared to referents that have been mentioned at an earlier time (e.g., \citealt{Givon-83, Ariel-90-book, Arnold-98-thesis}). Furthermore, there is considerable evidence that pronominalization rates are higher when referents were mentioned last in sentence-initial subject position compared to sentence-final object position (e.g., \citealt{Stevenson::al-94, Arnold-01, Fukumura::van_Gompel-10}). 

\citet{Stevenson::al-94} used different linguistic contexts (\ref{ex:s94}) to investigate the choice of referential expressions in written production. They found that in all conditions, pronouns were produced more often for the first-mentioned subject referent, whereas names were preferred for the second mentioned referents.

\ea \label{ex:s94} 
    \ea John seized the comic from Bill. \ldots \label{ex:s94-a} \hfill{(goal-source verb)}
    \ex Joseph hit Patrick. \ldots  \label{ex:s94-b} \hfill{(agent-patient verb)}
    \ex Ken admired Geoff. \ldots  \label{ex:s94-c} \hfill{(experiencer-stimulus verb)}
    \ex Simon ran towards Richard. \ldots  \label{ex:s94-d} \hfill{(agent-goal verb)}
\z\z

\citet[141]{Arnold-01} used three-sentence stories including source-goal (\ref{ex:arnold01-a}) or goal-source verbs (\ref{ex:arnold01-b}).

\ea \ea There was so much food for Thanksgiving, we didn’t even eat half of it.
        Everyone got to take some food home. Lisa gave the leftover pie to Brendan. \ldots \label{ex:arnold01-a}
    \ex I hate getting sick. It always seems like everyone gets sick as soon as it’s vacation. 
        Marguerite caught a cold from Eduardo two days before Christmas. \ldots \label{ex:arnold01-b}
\z\z

 In an oral continuation task, she found that in their continuations, participants used pronouns more frequently for subject referents compared to object referents. This finding was replicated by \citet{Fukumura::van_Gompel-10} for stimulus-experiencer (\ref{ex:fg10-a}) and experiencer-stimulus verbs (\ref{ex:fg10-b}) in written production.

\ea\label{ex:fg10} \ea Gary scared Anna after the long discussion ended in a row. This was because\ldots \label{ex:fg10-a}
    \ex Gary feared Anna after the long discussion ended in a row. This was because\ldots \label{ex:fg10-b}
\z\z 


\citet{Rohde::Kehler-14} proposed that the main determinant of pronominalization is a referent's topic status. Following classic definitions of the information-structural concept of sentence topic, they assume that the topic of a sentence is the entity the sentence makes a statement \textit{about} (e.g., \citealt{Reinhart-81, Lambrecht-96-book}). Note that in English, grammatical subject and information-structural topic are highly correlated and therefore usually confounded in linguistic materials, which might also hold for the previous studies investigating pronominalization rates mentioned above.
To tease apart the influence of the subject versus topic status on the choice of referential forms, they presented participants with active and passive sentences, followed by a blank line (the continuation prompt).

\ea \ea Amanda amazed Brittany. \ldots \label{ex:rk14-a}
    \ex Brittany was amazed by Amanda. \ldots \label{ex:rk14-b}
\z\z

The comparison of active and passive sentences is based on the assumption that whereas the subject is the default topic in active sentences such as (\ref{ex:rk14-a}), the passive structure (\ref{ex:rk14-b}) establishes the subject as topic more strongly -- promoting the non-agent to the subject position. If topichood instead of subject status is the main force boosting a referent's accessibility and consequently the choice to pronominalize this referent, pronoun rates in continuations should be higher in passive compared to active contexts. This is indeed what \citet{Rohde::Kehler-14} found with a written continuation task. Subjects of passive sentences were pronominalized more often than subjects of active sentences, whereas non-subject referents were pronominalized at a similar rate for the two structures.

\subsection{The predictability controversy}

A longstanding controversy with regard to the use of referential expressions concerns the question of whether the \textit{predictability} of a referent has an influence on the choice of a referential expression. Several studies have shown that certain verbs establish strong expectations with regard to re-mentioning one of the referents involved in the event denoted by the verb. Implicit causality biases are among the best studied phenomena in the literature on thematic role expectations (e.g., \citealt{Garvey::Caramazza-74, Au-86, Bott::Solstad-14}).

Following sentence fragments such as (\ref{ex:imp}) (see also example (\ref{ex:s94-c}) of \citealt{Stevenson::al-94}, and example (\ref{ex:fg10}) of \citealt{Fukumura::van_Gompel-10}), participants systematically re-mention Kathy (the previous subject; e.g., \textit{\ldots because she/Kathy was so helpful}) when completing fragments such as (\ref{ex:imp}), whereas they re-mention Tom (the previous object; e.g., \textit{\ldots because he/Tom was so helpful}) when completing fragments such as (\ref{ex:adm}) in continuation tasks (see also \citealt{Holler::Suckow-16, Bittner-19, Portele::Bader-20} for German).

\ea \ea Kathy impressed Tom because\ldots\label{ex:imp}
     \ex Kathy admired Tom because\ldots\label{ex:adm}
\z\z

The re-mentioning preferences have been ascribed to the stimulus of the event being regarded as causing the psychological state of the experiencer. People are more likely to refer to the implicit cause of the event, especially in the context of explicit causal discourse connectives such as \textit{because}.
Similarly, continuation experiments investigating transfer of possession contexts such as (\ref{ex:top}) have shown that participants start their completions by re-mentioning the goal -- Kathy in (\ref{ex:get}), Tom in (\ref{ex:give}) -- more often than referring back to the source (e.g., \citealt{Stevenson::al-94, Arnold-01, Kehler::Rohde-13}).

\ea\label{ex:top} \ea Kathy got a present from Tom. \ldots\label{ex:get}
    \ex Kathy gave a present to Tom. \ldots\label{ex:give}
\z\z

The prevalence of certain thematic arguments to be mentioned again due to verb meaning and discourse connectives has been subsumed under the notion of \textit{semantic biases}. Of interest for the discussion at hand is the fact that several researchers have ascribed the preference for a certain referent to be mentioned again, its higher predictability, to being more accessible than the alternative referent. 
Assuming that predictability affects the discourse status of referents, we might expect there to be influences on the choice of referential expressions. There is, however, mixed evidence from previous work, leading to an inconclusive pattern. Whereas thematic preferences in terms of likelihood of reference have been replicated for both implicit causality as well as transfer of possession verbs, studies have found differing results with regard to the choice of referential forms.
Studies involving transfer of possession verbs provide evidence that the choice to pronominalize can indeed be influenced by predictability or likelihood of reference. \citet{Arnold-01} found that participants are more likely to pronominalize goal characters compared to source characters. This finding was also replicated by \citet{Rosa::Arnold-17}. In implicit causality contexts, however, several studies did not find effects of the more likely referent to be mentioned again on the choice of referential expressions (\citealt{Stevenson::al-94, Kehler::al-08, Fukumura::van_Gompel-10}), but in recent work, \citet{Weatherford::Arnold-21} could reveal influences of semantic predictability on the choice of referential forms in implicit causality contexts. They found that participants were more likely to pronominalize stimulus referents compared to experiencer referents, but this effect was limited to object referents. 
The mixed results found in terms of predictability effects on pronoun production have been ascribed to the different verb types investigated (transfer of possession vs. implicit causality verbs) as well as to task variation (sentence continuation tasks vs. production tasks encouraging stronger discourse representations).

The question whether predictability influences the production of referential expressions is of utmost importance for the current discussion, since this is exactly what UID accounts (see Introduction) predict. The higher the predictability of a certain word, the less information the respective word carries. \citet[48]{Jaeger-10}, for example, states: \enquote{Speakers should be more likely to produce pronouns (e.g., \textit{she}) instead of full noun phrases (e.g., \textit{the girl}) when reference to the expression's referent is probable in that context}. 

In their information theoretic study, \citet{Tily::Piantadosi-09} had participants guess upcoming referents in authentic contexts. In a second analysis, they investigated whether writers were influenced by the predictability of the referents (participants' guesses) when choosing referential forms. The authors found that pronouns were indeed used more often when the referent was predictable, suggesting an influence of predictability on the choice of referential expressions in line with UID accounts. \citet{Tily::Piantadosi-09} conclude that \enquote{[p]ronouns in particular provide language with context-dependent code that allows more predictable nouns to be referenced with a shorter word. These results align with recent production theories such as Uniform Information Density (\citealt{Genzel::Charniak-02, Jaeger-06-thesis, Levy::Jaeger-06})}.

In a recent study, \citet{Orita::al-21} compared an accessibility based model with an informativity driven model of referential choices. The accessibility based model (called the \textit{topicality} model by \citeauthor{Orita::al-21}) was created under the assumption that the topichood of a referent is the main determinant of choosing referential expressions (as discussed above). The informativity model (called the \textit{rational} model by \citeauthor{Orita::al-21}) was based on the assumption that referential choices reflect the amount of information words carry in discourse, on speakers' speech cost, and on the predictability of referents. The results of their simulations suggest that both the referent's topichood status as well as word informativity are important factors in the choice of referential forms.\footnote{The speech cost factor of preferring shorter word forms, on the other hand, did not turn out to be an influential factor to the choice of referential form.}

To investigate the question whether referent predictability affects the choice of referential expressions, we returned to a free sentence continuation study we conducted some years ago, investigating the interpretation and production of pronouns in German. The respective study was published in \citet{Bader::Portele-19-The-interpretation-of}.


%-------------------------------------------------------------------%
%---------------------------- Section ------------------------------%
%-------------------------------------------------------------------%
\subsection{A new look at Experiment 3 of \citet{Bader::Portele-19-The-interpretation-of}}\label{sec:exp-BP19}

Experiment 3 of \citet{Bader::Portele-19-The-interpretation-of} used a free sentence continuation task in order to obtain the production data necessary to test the Bayesian theory of pronoun resolution proposed by \citet{Kehler::al-08} (see also \citealt{Kehler::Rohde-13}). According to this theory, pronoun resolution is based on two production probabilities -- the probability of which referent to mention next and the probability to use a pronoun for a given referent. These two probabilities, which are assumed by the Bayesian theory to be independent of each other, are combined using Bayes formula to predict the most likely referent of a referentially ambiguous pronoun. Although Experiment 3 was run without having UID in mind, the two probabilities needed to apply the Bayesian theory are exactly the two probabilities needed to test the prediction based on UID accounts that pronoun production is influenced by referent predictability. Free sentence continuation allows us to measure surprisal/predictability values by analyzing which of the referents mentioned in the previous context was taken up again in the written continuation. By looking at the rate of different referential expressions used, we can measure pronoun rates for the respective referents. The predictions based on information theoretic models, such as UID, is that we should find higher pronoun rates for predictable/less surprising referents. Importantly, the experiment that we discuss used verbs without strong semantic biases with regard to which referent is mentioned next. This avoids potential pitfalls due to (strong) semantic biases of differing verb types (e.g., psych verbs, transfer-of-possession verbs). These biases have been discussed as one of the reasons for the mixed results in terms of predictability effects on pronoun production. 

The materials for this experiment were a German adaption of sentences used in a study conducted by \citet{Kaiser::Trueswell-08} in Finnish. A complete item is shown in \tabref{table-material-exp3}.\footnote{We limit ourselves to one of the four conditions of the original experiment.} Experimental items consisted of three context sentences followed by a blank line, the free continuation prompt. The first context sentence (C1 in \tabref{table-material-exp3}) constituted a scene-setting sentence. This sentence always introduced a female character in the form of a proper name. The second context sentence (C2 in \tabref{table-material-exp3}) re-mentioned the female character in the form of a personal pronoun and introduced a male character by using an indefinite NP. The third context sentence (C3 in \tabref{table-material-exp3}) re-mentioned this male character by using a definite NP and introduced a second male character in the form of an indefinite NP. A relative clause modified the indefinite NP object. The last context sentence was followed by a blank line, the prompt eliciting participants' continuations.

\begin{table}
\caption{\label{table-material-exp3}Sample stimulus for Experiment 1 (from \citealt{Bader::Portele-19-The-interpretation-of}).}
\begin{tabularx}{\textwidth}{lQ}
\lsptoprule
C1 & Sabine (C\Down{1}) war am Sonntag im Zirkus.\\
   & `Sabine (C\Down{1}) visited a circus on Sunday.'\\
C2 & Bevor die Aufführung begann, hatte sie schon \Subject{einen Clown} (C\Down{2}) herumlaufen sehen.\\
   & `Before the show began, she saw \Subject{a clown} (C\Down{2}) walking around.'\\
C3 & \Subject{Der Clown} (C\Down{2}) umarmte \Object{einen Mann} (C\Down{3}), der ganz wirre Haare hatte.\\
   & `\Subject{The clown} (C\Down{2}) hugged \Object{a man} (C\Down{3}) with completely tousled hair.'\\
\lspbottomrule
\end{tabularx}
\end{table}

Forty-four students of the Goethe University Frankfurt read 16 experimental and 24 filler contexts and wrote a sensible continuation sentence for each context. Participants were asked to provide a complete continuation sentence, but there were no further restrictions in terms of form or content of the continuation.
For the continuations given by participants, we scored which characters were mentioned again (female referent, first male NP, second male NP) and which referential expression was chosen to refer back to the referent (pronoun, proper name, definite NP). For continuations containing more than one of the human referents, the first referent given in the continuation was counted. Exemplary continuations together with their respective scoring information are shown in (\ref{ex:cont}).


\ea\label{ex:cont} \ea Re-mentioned: female character  -- Form: pronoun\\
        \textit{Sie hoffte, dass der Clown sie in Ruhe ließ}.
    \glt \enquote*{She hoped the clown would leave her alone.}
    \ex Re-mentioned: female character  -- Form: proper name\\
        \textit{Sabine war froh, dass es nicht sie getroffen hatte}.
    \glt \enquote*{Sabine was glad it was not her.}
    \ex Re-mentioned: first male NP  -- Form: definite NP\\
        \textit{Der Clown hatte ebenso wirre Haare}.
    \glt \enquote*{The clown also had tousled hair.}    
    \ex Re-mentioned: first male NP  -- Form: pronoun\\
        \textit{Er war später im Programm sehr lustig}.
    \glt \enquote*{He was very funny later on in the show.}
    \ex Re-mentioned: second male NP  -- Form: definite NP\\
        \textit{Der Mann saß in der ersten Reihe und freute sich über jeden Auftritt}.
    \glt \enquote*{The man was sitting in the front row and was happy about every appearance.}  
    \ex Re-mentioned: second male NP  -- Form: pronoun\\
        \textit{Er war der Bruder des Clowns}.
    \glt \enquote*{He was the clown's brother.} 
\z\z



To estimate referent predictability/surprisal, we looked at the re-mention rates of the three human characters. Percentages of referents from the context occurring as the only or first referent in participants’ continuations and the derived surprisal levels are shown in \tabref{table-results-exp1-P(referent)}.
The female character (introduced in the first context sentence) was taken up again in 53\% of the continuations provided by participants, followed by the second male character (introduced in context sentence three) with 23\% and the first male character (introduced in the second context sentence) with 13\% of references. We think that the majority of continuations refers back to the female character because she is interpreted as the discourse topic of the narrational contexts (e.g., \citealt{Asher-04}). Based on these percentages, the female character is the most predictable or least surprising referent, followed by the second male character with a medium predictability/surprisal level, whereas the first male character is the least predictable/most surprising character.
If the use of pronouns is indeed driven by information theoretic considerations as discussed above, pronoun rates should be inversely related to the surprisal levels of the referents. We would therefore expect to find the highest pronominalization rate for the female character, whereas the first male character should be pronominalized least often and the pronoun rate for the second male character should fall between these two.


\begin{table}
\caption{\label{table-results-exp1-P(referent)}Percentages of referents from the context occurring as the only or first referent in participants' continuations}
\fittable{
\begin{tabular}{lcr}
\lsptoprule
Referent &  \% Re-mention & \\\midrule
C\Down{1} (Female NP)    &  53   & $\rightarrow$ high predictability/low surprisal \\
\Subject{C\Down{2} (1. male NP)}   &  13   & $\rightarrow$ low predictability/high surprisal\\
\Object{C\Down{3} (2. male NP)}   &  23   & $\rightarrow$ medium predictability/medium surprisal\\
\lspbottomrule
\end{tabular}
}
\end{table}

\begin{table}
\caption{\label{table-results-exp1-P(referent)Form}Percentages of referents from the context occurring as the only or first referent in participants' continuations and percentages of referential forms for each referent}
\begin{tabular}{llccc}
\lsptoprule
                         &                   & \multicolumn{3}{c}{Referential expression}\\\cmidrule{3-5}
                         &                   & Pro- & Proper  & Definite\\
  Referent               &   \% Re-mention    & noun & name    & NP \\\midrule
C\Down{1} (Female NP)    &  53 $\rightarrow$ low surprisal             & 26  & 74 &     \\
\Subject{C\Down{2} (1. male NP)}  &  13 $\rightarrow$ high surprisal   & 67  & -- & 33  \\
\Object{C\Down{3} (2. male NP)}   &  23 $\rightarrow$ medium surprisal & 11  & -- & 72  \\
\lspbottomrule
\end{tabular}
\end{table}

\tabref{table-results-exp1-P(referent)Form} shows percentages of referential forms for each referent. Contrary to expectations based on UID accounts to pronominalization, the highest pronoun rate (67\% pronouns vs. 33\% definite NPs) was found for reference to the first male character (the subject of the preceding clause), the most surprising or least expected referent based on re-mention rates. The second highest pronoun rate (26\%) was found for the least surprising referent (the female character). Note, however, that overall the female character was re-mentioned more often with a proper name (74\%) compared to a personal pronoun. The lowest pronoun rate was found for the medium expected\slash surprising second male character (11\% vs. 72\% definite NPs).


%---------------------------- Subsection ------------------------------%
%-------------------------------------------------------------------%
\subsection{Discussion}
In order to shed further light on the question whether referent predictability guides the choice of referential expressions, we looked at data from a former experiment (Experiment 3 of \citealt{Bader::Portele-19-The-interpretation-of}) that used a sentence continuation task to measure referent predictability (Who is mentioned again in the continuation?) as well as pronominalization rates (How often did participants use a pronoun to talk about the referent mentioned again?). UID accounts (e.g., \citealt{Fenk-Oczlon-89, Levy::Jaeger-06, Tily::Piantadosi-09}) predict that pronoun rates should be higher for predictable and thus less surprising referents. However, the re-mention ranking found in our continuation results, which was used to derive surprisal levels, was not mirrored by participants' choice of referential expressions. Pronoun use was not highest for the least surprising referent, as might be expected under an information-theoretic account of referring expressions. To the contrary, we found the highest pronominalization rate for the most surprising\slash less expected referent to be taken up again in the continuation. This referent, however, was the (subject) topic of the previous sentence. In line with previous studies (e.g., \citealt{Fukumura::van_Gompel-10, Rohde::Kehler-14}) we therefore suggest that accessibility based accounts are better suited to account for the referential pattern found in our study.

\largerpage
A potential argument against this conclusion might be the fact that we used contexts involving alternative referents or \textit{competitors}. In our materials, the contexts introduced one female character and two male characters. Whereas the choice of the feminine third person singular pronoun (\textit{sie} `she') should not be influenced by the other two male referents, the choice of using a potentially ambiguous masculine third person singular pronoun (\textit{er} `he') might be influenced by the fact that there are two characters of the same gender participating in the context, therefore establishing two potential referents for one and the same pronoun.
Both accessibility as well as UID accounts of the choice of referential expressions acknowledge a role of potential competitors in contexts. Within accessibility accounts, \textit{competition} (e.g., \citealt{Ariel-90-book}) or \textit{interference} (e.g., \citealt{Givon-84-book}) are taken to influence a referent's accessibility status. In contexts involving more than one potential referent for the pronoun, speakers should therefore use more explicit expressions (i.e., definite NPs or proper names). This prediction was borne out in studies investigating the production of referential expressions in English (e.g., \citealt{Arnold::Griffin-07}).
Within UID accounts, contextual competitors are taken into account since the informativity of a referential expression, the personal pronoun in this case, decreases with an increase of contextual competitors (see also \citealt{Orita::al-21} for a formalization of competitors within an information theoretic model evaluation). 

We are well aware of the fact that there are potential competitors present in our experimental materials. However, we want to bring up two thoughts for further discussion. The highest pronoun rate we found in our data was for re-mentioning one of the two male characters, which was the most surprising referent based on the re-mention rates. If the presence of potential competitors for the use of a specific referential expression (in this case the male personal pronoun) was an influential factor in the current study, it is not clear why we should find the overall highest pronoun rate (67\% pronouns vs. 33\% definite NPs ) for one of the two male characters, since the use of a potentially ambiguous pronoun still leads to a decrease in terms of informativity of the pronoun. On the other hand, if the highest pronoun rate found in this study is already influenced by the fact that there is a competitor present in the context in the sense that without this competitor, we might even expect higher pronoun rates for this referent, this does not alter the fact that we are still talking about the referent having the lowest predictability value to be taken up again in the continuation. Taking competitors into account therefore does not remove the argument against UID accounts.

Furthermore, we should also focus on the referent with the lowest surprisal level, or in other words, the highest predictability -- the female referent. We have attributed the highest re-mention rate for the female referent to the fact that she establishes the discourse topic in the contexts. The female character is introduced in the first (scene-setting) context sentence. She is re-mentioned in the form of a personal pronoun in the second context sentence (which also introduced the first male character) and picked up most often in the continuation following the third context sentence (that mentioned both male characters). However, when looking at the referential expressions chosen to refer to the female character in the continuations, we again do not see a prevalence of pronouns to refer back to this character. Whereas pronouns were chosen in 26\% of cases, the majority of cases (74\%) is constituted of proper names. Although this pattern might be explained in terms of (non-)recency of the respective referent (there is one sentence intervening between the last mention of the female character and the continuation) or stylistic reasons, this is not what we would expect based on information theoretic considerations. This is at least what holds for the singular pronoun. However, note that the feminine third person singular pronoun \textit{sie} (`she') is identical in form with the German third person plural pronoun. The personal pronoun could in principle refer to, for example, the two male characters mentioned in the sentence preceding the continuation prompt. When extending the notion of potential competitors to referents differing in number, one might therefore argue that an avoidance of the personal pronoun might be due to preventing a potential ambiguity. Since participants were asked to write complete sentences, this potential ambiguity, however, would be limited to the pronoun itself in most cases, since the finite verb in second position would disambiguate the pronoun immediately.




%-------------------------------------------------------------------%
%---------------------------- Section ------------------------------%
%-------------------------------------------------------------------%
\section{Word order}\label{sec:word-order}

We now turn to the question of how language producers determine in which order they mention the referents of a message. Since its introduction by K. Bock and colleagues (e.g., \citealt{Bock::Warren-85}), the notion of conceptual accessibility has played a dominant role with regard to this question. The accessibility of referents depends on their inherent features, like animacy, and on features temporarily derived from the context in which the referents appear, like givenness or contextual salience. Both inherent and derived accessibility have been shown to influence how speakers order the various referents of a message (e.g., \citealt{McDonald::al-93, Ferreira-94, Prat-Sala::Branigan-00}) -- the most accessible referent tends to start the sentence, subject to independent constraints like Plan Reuse \citep{MacDonald-13-How-language-production}.

The question of how speakers choose the order of words has also received some attention within the UID framework. For ease of reference, we repeat the UID hypothesis in (\ref{ex:UID-repeated}). 

\ea Uniform Information Density (UID, \citealt[25]{Jaeger-10})\label{ex:UID-repeated}\\
Within the bounds defined by grammar, speakers prefer utterances that distribute information uniformly across the signal (information density). Where speakers have a choice between several variants to encode their message, they prefer the variant with more uniform information density (ceteris paribus).
\z

When encoding a given message, speakers can typically choose from a range of different syntactic structures that vary along several dimensions -- besides having different word order, the variants can differ in terms of optional elements (e.g., the complementizer \textit{that}) and in terms of elements that can be produced in more or less reduced forms (e.g., proper names versus pronouns). In contrast to phenomena involving optional and/or reduced items, we are not aware of much research applying the UID approach to the issue of word order variation. Furthermore, the research we are aware of (\citealt{Maurits::al-10, Collins-14, Jain::al-18, Rubio-Fernandez::al-21}) has yielded mixed results, sometimes supporting the UID approach to word order and sometimes contradicting it.

In the following, we focus on German main clauses with an agent and a patient argument. Such clauses can have an active SO structure, in which case the agent argument precedes the patient argument. Alternatively, the patient argument can be moved in front of the agent argument either by producing a passive clause or by producing an active OS clause. Depending on the particular structure, information will be more or less evenly distributed. In the corpus study of \citet[162]{Hoberg-81-book}, about 58\% of all German main clauses started with the subject and only about 3.5\% with an object, the rest being made up of sentences starting with an adverbial or a predicate nominal. According to these data, encountering an object as the first phrase of a sentence should be surprising whereas encountering a subject should be unremarkable. Of course, the surprisal value of a phrase is not only a function of its syntactic function. As defined in (\ref{ex::surprisal}), the surprisal value of a linguistic unit is inversely related to its predictability in context. For a phrase in sentence-initial position, the preceding sentences constitute the context; for later parts of a sentence, the context is made up of the preceding sentences and the initial part of the sentence. 

Consider first the sentence-initial phrase, which for German main clauses is the phrase occupying the so-called prefield. The probability that either the agent or the patient will be mentioned next -- the so-called next-mention bias -- depends on semantic and pragmatic properties of the preceding context. For example, if the last context sentence contains an implicit causality verb, chances are high that the stimulus will be mentioned next whereas with verbs that are not associated with strong semantic biases, which argument is mentioned next is less predictable a priori. However, whatever referent of the preceding context is selected to start the next sentence (if any is selected, because starting a sentence with an adverbial is also common), it will not be surprising if the referent occurs with the syntactic function of subject whereas it is highly surprising if it occurs as object. 

With regard to the surprisal values of elements following the sentence-initial phrase, general statements do not seem to be possible, not least because of the verb-second nature of German. In a declarative main clause, the initial phrase is necessarily followed by the finite verb in German in verb-second position, but the identity and thus the predictability of the finite verb can vary widely, even when taking into account the first phrase. For illustration, consider the example in (\ref{ex:surprisal-after-prefield}). Here, the sentence-initial position is occupied by the subject. Besides predicting that a finite verb form comes next, a sentence-initial subject imposes few constraints.

\newpage

\ea\label{ex:surprisal-after-prefield} 
    Der Bussard \ldots\ \enquote*{The buzzard}
    \ea \gll {\ldots} \textbf{attackierte} den Läufer / \textbf{hat} den Läufer attackiert.\\
        {} attacked the runner / has the runner attacked\\
    \glt \enquote*{\ldots\ attacked the runner/has attacked the runner.}\label{ex::surprisal-after-prefield-active}
    \ex \gll {\ldots} \textbf{wurde} von dem Läufer erschreckt. / \textbf{ist} von dem Läufer erschreckt worden.\\
        {} was by the runner frightened / is by the runner frightened been\\
    \glt \enquote*{\ldots\ was frightened by the runner / has been frightened by the runner.}\label{ex::surprisal-after-prefield-passive}
    \ex \gll {\ldots} \textbf{hat} ein großes Nest / \textbf{ist} ziemlich gefährlich / \textbf{wurde} immer lauter.\\
        {} has a large nest / is rather dangerous / became always louder\\
    \glt \enquote*{\ldots\ has a large nest / is rather dangerous / became louder and louder.}\label{ex::surprisal-after-prefield-copula}
\z\z

On the one hand, the sentence can be an active sentence with a transitive verb, as in (\ref{ex::surprisal-after-prefield-active}). In this case, the lexical verb serves as finite verb itself, or it appears in a non-finite form, for example, as a participle in which case the finite verb is a form of the auxiliary \LingCite{haben} (\enquote*{have}). Any particular main verb has a low predictability (because there are so many different main verbs) unless there are strong contextual constrains. A perfect auxiliary, in contrast, is much more predictable and thus less surprising for the simple reason that perfect tense is a frequent tense form and \LingCite{haben} is the perfect auxiliary of a very large number of verbs. Note furthermore that the predictability of the following object also strongly depends on the finite verb in verb-second position. A transitive main verb in verb-second position makes an upcoming object predictable and thus less surprising whereas an auxiliary in verb-second position is also compatible with intransitive verbs, making the appearance of an object less predictable. As shown in (\ref{ex::surprisal-after-prefield-passive}), a subject in sentence-initial position can also start a passive clause, which can be signaled by a passive auxiliary as finite verb. In a composite tense form, however, the passive auxiliary appears clause-finally and a form of the perfect auxiliary \LingCite{sein} (\enquote*{be}) occupies the verb-second position. Since this is also the perfect auxiliary for a certain subset of intransitive verbs, it is only mildly predictive of a passive clause. The situation is even further complicated by the fact that all auxiliaries also have other uses -- as main verb or as copula verb, as illustrated in (\ref{ex::surprisal-after-prefield-copula}). 

In sum, although it seems impossible to make general statements about the distribution of information across sentences, it seems safe to conclude that information is typically  relatively uniformly distributed when a sentence starts with the subject whereas a sentence-initial object leads to a considerable information peak at the sentence beginning. Thus, all else being equal, a speaker adhering to UID will prefer to produce subject-initial sentences. This still leaves open the choice between a sentence in the active voice and a sentence in the passive voice. Whether the fact that active SO sentences are much more common than passive sentences also follows from UID must be considered an open question. If not, an independent principle like Plan Reuse would have to be invoked, too.

The question now is whether conditions favoring patient-initial orders remove or at least reduce the disadvantage of patient-initial sentences with regard to the distribution of information. In principle at least, this seems to be the case. One recurrent finding in word order research has been that animate referents tend to precede inanimate referents. This is compatible with UID accounts given that animate referents (and human referents in particular) are typically more predictable than inanimate referents, both at the conceptual and at the lexical level. A further recurrent finding, namely that contextually more salient referents tend to occur earlier than contextually less salient referents, is also compatible with UID. If a referent is contextually more salient, it is more likely that the referent is mentioned again, that is, the predictability of the referent increases, making it more likely that the referent will occur in initial position. 

To sum up so far, basic findings concerning the choice between SO and OS order can -- at least in principle -- be explained both in terms of accessibility and UID. In the rest of this section, we discuss a range of more specific findings from our lab concerning word order in German main clauses with two arguments. These findings show, among others, that the relationship between topichood and word order is more complicated than sketched above: Contrary to common assumptions, topics precede non-topical referents only under specific conditions. We first present these findings and afterwards discuss how they may be accounted for in terms of accessibility or in terms of UID. 


%-------------------------------------------------------------------%
\subsection{Producing agent-patient sentences in context}

In order to further explore the role of accessibility for the choice of referential expressions and the choice of word order, we have run a series of picture description experiments in the last few years. For reasons of space, we concentrate here on pictures depicting an event in which an animate agent acts on an animate patient, for example, a doctor examining a teacher. In two experiments reported in \citet{Bader::al-17}, pictures were preceded by contexts consisting of a sentence introducing an agent and a patient and a question which varied across experiments. In Experiment 2 of \citet{Bader::al-17}, the question was a wide-focus one. In Experiment 3 of \citet{Bader::al-17}, it was a narrow-focus question asking for further information about the patient argument, thereby establishing the patient as topic of the following picture description. For the picture of a doctor examining a teacher, the two kind of questions are shown in (\ref{ex:PD-simple-contexts}).

\ea \label{ex:PD-simple-contexts}
  \ea  Wide-focus question\\
     Hier geht es um einen Arzt und einen Klavierlehrer. Was ist zu sehen?\label{ex:PD-simple-contexts-wide}\\
     \enquote*{This picture involves a doctor and a piano teacher. What can be seen?}
  \ex Narrow-focus question with patient as topic\label{ex:PD-simple-contexts-narrow}\\
      Hier geht es um einen Arzt und einen Klavierlehrer. 
      Was lässt sich über den Klavierlehrer sagen?\\
     \enquote*{This picture involves a doctor and a piano teacher. What can one say about the piano teacher?}
\z\z

With regard to word order, these two experiments revealed a striking effect of the context question: With a wide focus question, almost all sentences were produced with SO order (\textit{Der Arzt hat den Lehrer untersucht.} `The doctor examined the teacher.'). With a narrow focus question that established the patient as topic, the large majority of sentences were passive sentences with the patient argument in initial position (\textit{Der Lehrer wurde von dem Arzt untersucht.} `The teacher was examined by the doctor.'). 

The contexts in (\ref{ex:PD-simple-contexts}) are admittedly somewhat artificial and situations where contexts of this kind are produced may be rare (although not impossible). We therefore ran a further set of experiments in which pictures were embedded in an evolving story. To this end, the contexts preceding the pictures consisted of a header introducing either the agent or the patient as the topic followed by three sentences. The contexts thus formed the beginning of a story that was continued by the event depicted in the picture. In Experiment 2 of \citet{Bader::Portele-23-Talk-AMLaP}, the picture was preceded by an agent or a patient context as shown in (\ref{ex:PD05-contexts}).

\ea \label{ex:PD05-contexts}
  \ea  
     Topic = {Agent}\\
     Der beste Arzt\\
     In unserem Viertel gab es {einen sehr guten Arzt.} \\ 
     {Dieser Arzt} konnte fast immer helfen. \\
     Einmal musste {er} {einen scheinbar schwerhörigen} {Klavierlehrer} behandeln.\\
     \enquote*{The best doctor -- {A very good doctor} was practicing in our quarter.
               {This doctor} could help almost always.
               Once {he} had to treat {a seemingly hearing-impaired piano teacher.}}
  \ex  
     Topic = Patient\\
     Sorgen eines Klavierlehrers\\
     In unserem Viertel gab es {einen guten Klavierlehrer}.\\ 
     {Dieser Klavierlehrer} hatte eine Zeit lang Probleme beim Hören.\\
     Einst suchte {er} {einen angesehenen Ohrenarzt} auf.\\
     \enquote*{Sorrows of a piano teacher -- {A good piano teacher} was living in our quarter.
               {This piano teacher} was having hearing problems for quite a while.
                Once {he} visited {a respected ear specialist}.}
\z\z

24 students of the Goethe University Frankfurt, who participated for course credit, read each context and then described the picture using the verb shown above the picture. Participants' descriptions were digitally recorded and transcribed. In this experiment, participants produced almost only sentences with SO order, but topichood still had a significant effect. When the patient was the topic, the percentage of SO sentences was only slightly below 100\%. When the agent was the topic, the rate of SO sentences was about 90\%, which was significantly lower than the almost 100\% percent in the condition with a patient topic. Thus, with an agent as topic, about 10\% patient-initial sentences were produced, which means that when patient-initial sentences were produced, they had the non-topic in first and the topic in second position. A final finding of this experiment was that the object in OS sentences was realized as a demonstrative pronoun in most cases.

A comparison of the findings for question contexts as in (\ref{ex:PD-simple-contexts}) and story contexts as in (\ref{ex:PD05-contexts}) reveals two puzzling differences. First, the experiment in which a question set the patient as topic (\LingCite{What can one say about the piano teacher?}) found a high rate of patient-initial sentences. With a story-like context, in contrast, almost only agent-initial sentences, that is, SO-sentences, were produced when the patient was the topic whereas a small number of patient-initial sentences was observed when the agent was the topic. Second, whereas the patient-initial sentences in the experiment with a context question were produced as passive sentences in most cases, in the experiment with story-like context mainly OS sentences were observed. In the next two sections we discuss how far accessibility (\sectref{section-word-order-accessibility}) and  UID (\sectref{section-word-order-UID}) take us in accounting for these differences.


%-------------------------------------------------------------------%
\subsection{Accessibility-based accounts of word order in German main clauses\label{section-word-order-accessibility}}

The finding of mainly agent-initial sentences following a wide-focus question (see \ref{ex:PD-simple-contexts-wide}) and mainly patient-initial sentences following a narrow-focus question with the patient as topic (see \ref{ex:PD-simple-contexts-narrow}) is clearly compatible with accessibility-based accounts of grammatical encoding. First, prior research has shown that the momentary accessibility of a referent increases when it is assigned a thematic role that is high on the thematic hierarchy (e.g., \citealt{Ferreira-94}). The accessibility of an agent argument is therefore higher than the accessibility of a patient argument so that the agent argument will precede the patient argument when both arguments are on a par in other respects, like animacy and givenness. This was the case in the experiment with a wide-focus question. In the experiment with a narrow-focus question, in contrast, the question explicitly established the patient referent as topic. Given that in this experiment most sentences started with the patient topic, we can conclude that the topic-setting question increased the accessibility level of the patient to a larger degree than the accessibility increase brought about by being the agent. The only finding which accessibility does not account for is that the patient was almost always inserted in sentence-initial position by means of passivization whereas OS sentences, which allow fronting the patient without a concomitant change in voice, were almost never produced. 

Accounting for the findings of the experiment in which the picture was preceded by a story-like context is less straightforward in terms of accessibility. First, when the patient was the topic, almost only agent-initial SO sentences were produced. This is in stark contrast to the finding of a majority of patient-initial passive sentences when the patient was explicitly established as topic by means of a question. Second, when patient-initial OS sentences were produced at all, they were produced in the condition with the agent as topic, that is, when the agent was more accessible than the patient.

As stressed several times before, there is general agreement that determining word order during language production is subject to a multitude of factors. By itself, the fact that some findings cannot be accounted for in terms of accessibility is therefore not worrisome. The question of course is whether there are plausible and independently motivated factors that can explain those findings not accounted for by accessibility.

Consider first the finding that the patient was fronted when a question explicitly established it as topic but not when it was implicitly established as topic by the preceding context. With regard to this finding, we can follow \citet{Rohde::Kehler-14} and assume that the boost in accessibility brought about by topicalization varies with the degree of explicitness of establishing the topic. Independent reasons also account for the additional finding that the patient was preposed mainly by means of passivization when it was explicitly set as topic by the preceding question. As proposed in \citet{Bader-20-Objects-in-the}, this finding follows from the preference of topics to be realized as subjects. When the patient argument is the topic, passivization achieves the preferred association of topichood and subjecthood.

Consider finally the finding that a small number of OS sentences was observed when the agent was the topic and the object was realized as a demonstrative. Research has found that demonstrative objects are especially prone to occur sentence-initially (e.g., \citealt{Bader::Portele-21}). This preference can tentatively be ascribed to a violable constraint requiring demonstrative NPs to occur as close to their antecedent as possible. In the case of a demonstrative object, this can be achieved by fronting the object, which decreases the distance between object and antecedent. Since demonstratives are typically confined to non-topical referents, it also follows that OS sentences with a demonstrative object only occur when the agent is the topic and the patient (= the object) is not the topic.

%-------------------------------------------------------------------%
\subsection{UID-based accounts of word order in German main clauses\label{section-word-order-UID}}

We now consider how the UID account fares with regard to the findings from picture description summarized above. As discussed before, subject-initial sentences exhibit a more uniform information density than object-initial sentences. This is in agreement with the finding that participants basically produced only SO sentences in the experiment with a wide-focus question, where the context introduced both arguments without making one of them more salient or prominent than the other. When a narrow-focus question established the patient as topic, it seems plausible to assume that the patient becomes more predictable. After all, when one is asked to provide information about a particular referent, it is highly likely that the answer will contain a reference to this referent. Because topics are preferentially realized as subject in sentence-initial position, a passive structure is the optimal choice because it allows the sentence to start with a subject. Thus, like the accessibility account, the UID account has no difficulty accounting for word order in contexts with wide or narrow focus question. 

Consider next the finding that following a story-like context with the patient introduced as topic, participants produced almost only SO sentences, that is, sentences with the non-topical agent in first position. This seems to be in conflict with the high number of patient-initial passive clauses when the patient was established as topic by means of a narrow-focus question. However, under a UID perspective, there is not necessarily a conflict between these two results. Under a UID perspective, what counts is the predictability of referents. With an explicit topic-setting question, the predictability of the topic referent is high, as argued above. Whether the same is true without an explicit topic-setting question is not so clear. Although it is often claimed that continuing with the topic referent has a privileged status (see, for example, the preferred transition types of Centering Theory, \citealt{Grosz::al-95}), free continuation experiments do not in general confirm this claim. To determine whether the observed preference for agent before patient is compatible with UID, it is therefore necessary to determine experimentally which referent is more likely to be re-mentioned and thus more predictable -- the topic or the non-topic. Running the necessary free continuation experiments must be left as a task for future research.

The final finding to consider is that active sentences with OS order were rarely produced and if so, only in contexts where the object was not the topic and the object was realized by a demonstrative pronoun. Does the observed association between the form of the object and its position follow from UID? Since sentence-initial objects are unexpected, a more explicit form should be chosen for them. Assuming the simplified referential hierarchy ``def NP $>$ demonstrative pronoun $>$ personal pronoun'' (modified from \citealt[313]{Kaiser::Fedele-19}), definite NPs are most explicit and personal pronouns least explicit, with demonstrative pronouns in between. The inverse correlation between explicitness and predictability therefore explains the rareness of OS sentences when the object is realized as a personal pronoun. On the other hand, it remains unclear why demonstrative pronoun objects show a preference for OS order whereas definite NPs do not, although the latter are more explicit and should thus be especially appropriate for the highly surprising sentence-initial objects. Thus, like the accessibility account, the UID account needs an independent explanation for this finding. As in the case of the accessibility account, a constraint favoring short distances between demonstratives and their antecedents could be hypothesized for this purpose.

%-------------------------------------------------------------------%
%---------------------------- Section ------------------------------%
%-------------------------------------------------------------------%
\section{General discussion}\label{sec:4_discussion}

In this article, we discussed two central notions within the domain of choosing referential expressions and word order during language production: accessibility and UID. Accessibility accounts focus on the differing accessibility status of referents involved in the current discourse. Accessible referents are highly activated and more easily retrieved from memory. When choosing referential expressions, less explicit referential forms (e.g., pronouns) are chosen for more accessible referents. When determining word order, more accessible referents are produced earlier than less accessible referents. UID accounts, on the other hand, focus on information transmission. The underlying assumption is that speakers prefer choices leading to uniform information density. When choosing referential expressions, speakers are expected to choose shorter expressions (e.g., pronouns) for more predictable referents, since the higher the predictability of a word, the less information it carries. With regard to word order, UID prefers orders in which constituents are neither highly predictably nor highly unpredictable.

Note that neither accessibility nor UID accounts claim that the respective factors constitute the sole determinants of referential forms as well as word order during language production.
In order to adjudicate between accessibility and UID accounts, we (re-)evaluated some data that we gathered in language production experiments over the last few years. 
To evaluate the question whether referent predictability influences the choice to pronominalize, we revisited Experiment 3 of \citet{Bader::Portele-19-The-interpretation-of}. The sentence continuation task used in this experiment allowed us to derive surprisal/predictability values based on data on which referent was mentioned next, following a context that introduced several referents. By looking at the rates of different expressions chosen to refer back to this referent, we were able to investigate the question whether more predictable referents were pronominalized more often compared to less predictable referents, as expected within UID accounts. The surprisal levels derived for referents were, however, not reflected by participants' referential expressions. The highest pronominalization rate was indeed found for the most surprising referent. In the materials presented, this referent was the topic/subject of the preceding sentence. Thus, for the data at hand, accessibility accounts for pronominalization, claiming that distance and grammatical function/topic status are the main force behind pronoun choice (e.g., \citealt{Fukumura::van_Gompel-10, Rohde::Kehler-14}), are superior to UID accounts.  

With regard to word order, our discussion was more open-ended. We presented a range of findings of an ongoing series of picture description experiments. Most of the results turned out to be compatible with both accessibility accounts and with UID accounts. For some findings, however, additional assumptions were necessary for accessibility based-accounts. One question was whether these additional assumptions could be avoided under an UID perspective. The answer to this question was mixed. On the one hand, the presence or absence of topic effects on word order may follow when looking at predictability instead of topichood. On the other hand, the association between referential form and word order (OS with demonstrative object) turned out to be difficult to reconcile with UID.

It's in the nature of things that the factors discussed in this article are not exhaustive. Previous research within both domains, for example, highlights communicative aspects in terms of speaker vs. listener, potential ambiguity avoidance, audience-design, etc. 
In the experiments discussed here, we looked at written data gained in monologue lab settings. Comparing these results to patterns gained by using different, more interactive tasks as well as within different linguistic contexts are just a few of the many options to continue this line of work.
We hope that the current article helps to identify some fruitful aspects for the investigation of accessibility vs. UID accounts to language production, thereby offering a starting point for future research. 
Including and comparing considerations from both accounts will certainly be of crucial value for investigations within the domain of language production.

\section*{Acknowledgements}
The authors would like to thank the organizers of the workshop \textit{Discourse obligates} at the 2022 annual meeting of the German Linguistic Society, the audience at the workshop, as well as two anonymous reviewers for helpful discussions and comments. 

%\section*{Contributions}
%John Doe contributed to conceptualization, methodology, and validation.
%Jane Doe contributed to the writing of the original draft, review, and editing.



\sloppy
\printbibliography[heading=subbibliography,notkeyword=this]
\end{document}
