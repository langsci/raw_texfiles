\documentclass[output=paper,colorlinks,citecolor=brown]{langscibook}
\ChapterDOI{10.5281/zenodo.13383785}
\author{Radim Lacina\orcid{0000-0002-7534-6204}\affiliation{Osnabrück University} and Patrick Sturt\orcid{0000-0002-2055-6933}\affiliation{University of Edinburgh} and Nicole Gotzner\orcid{0000-0002-9584-4518}\affiliation{Osnabrück University}}

%\ORCIDs{}

\title{The comprehension of broad focus: Probing alternatives to verb phrases}

\abstract{Recent research has shown that comprehenders represent alternative meanings to single focused words online \citep[for a review, see][]{gotzner2019life}, consistent with Rooth's (\citeyear{rooth1992}) formal semantic account. However, focus can also take a scope over whole phrases such as the VP \textit{read the manuscript}. We examined whether in these cases, too,  alternatives are represented by testing for an interference effect of the particle \textit{only}, which necessarily evokes alternatives. Using the probe recognition task, we first tested unmentioned alternatives to the constituent parts of VPs, to object nouns (Experiment 1, \textit{letter} for \textit{manuscript}) and verbs (Experiment 2, \textit{wrote} for \textit{read}). In Experiment 3, we tested alternatives to the whole phrase (\textit{wrote the letter}). In all experiments, alternative probes were processed slower than unrelated ones. We found varying evidence of the interference effects of \textit{only} with noun, verb and whole-phrase alternatives. Overall, this study does not provide support for the generalisation of the effects of \textit{only} to larger units. 
Since our study was the first to use the probe recognition task with phrase-sized constituents, we discuss the methodological implications of our work -- we found a relatedness effect for whole phrases and this shows that the probe recognition task can be used to test the representation of larger constituents.}


\IfFileExists{../localcommands.tex}{
   \addbibresource{../2_lacina.bib}
   % add all extra packages you need to load to this file

\usepackage{tabularx,multicol}
\usepackage{url}
\urlstyle{same}

\usepackage{listings}
\lstset{basicstyle=\ttfamily,tabsize=2,breaklines=true}

\usepackage{langsci-basic}
\usepackage{langsci-optional}
\usepackage{langsci-lgr}
\usepackage{langsci-osl}
% \usepackage{./langsci/styles/langsci-lgr}
% \usepackage{./langsci/styles/langsci-osl}
% \usepackage{langsci-gb4e}

\usepackage{tikz}
\usetikzlibrary{patterns,calc}
\pgfdeclarepatternformonly{south east lines}{\pgfqpoint{-0pt}{-0pt}}{\pgfqpoint{3pt}{3pt}}{\pgfqpoint{3pt}{3pt}}{
    \pgfsetlinewidth{0.6pt}
    \pgfpathmoveto{\pgfqpoint{0pt}{3pt}}
    \pgfpathlineto{\pgfqpoint{3pt}{0pt}}
    \pgfpathmoveto{\pgfqpoint{.2pt}{-.2pt}}
    \pgfpathlineto{\pgfqpoint{-.2pt}{.2pt}}
    \pgfpathmoveto{\pgfqpoint{3.2pt}{2.8pt}}
    \pgfpathlineto{\pgfqpoint{2.8pt}{3.2pt}}
    \pgfusepath{stroke}}
    
\usepackage{stmaryrd}
\usepackage{wasysym}
\usepackage{multirow}
\usepackage{caption}
\usepackage{subcaption}
\usepackage{mathrsfs}
\usepackage{qtree}

\usepackage{linguex}


   %pminos do not split footnotes
% \interfootnotelinepenalty=10000 %Footnote in Laporte chapters has to be split SN


%\DeclareIndexNameFormat{default}{%
%\nameparts{#1}%
%\usebibmacro{index:name}%
%{\index[names]}%
%{\namepartfamily}%
%{\namepartgiveni}%
% {}% L1
% {}% L2
%{\namepartprefix}% generates spurious space L3
%{\namepartsuffix}% generates spurious space L4
%}

%  {\DeclareIndexNameFormat{default}{%
%     \usebibmacro{index:name}{\index[names]}{#1}{#3}{#5}{#7}}}

%\DeclareIndexNameFormat{default}{%
%  \usebibmacro{index:name}{\sindex[nom]}{#1}{#3}{#5}{#7}}

%\DeclareIndexNameFormat{default}{%
%  \usebibmacro{index:name}{\sindex[person]}{#1}{#3}{#5}{#7}}
%\DeclareIndexNameFormat{default}{%
%\nameparts{#1} \usebibmacro{index:name}{\sindex[person]]}{\namepartfamily}{‌​\namepartgiven}{\nam‌​epartprefix}{\namepa‌​rtsuffix}}

%\newcommand{\smiley}{:)}

%\renewbibmacro*{index:name}[5]{%
%\usebibmacro{index:entry}{#1}%
%{\iffieldundef{usera}{}{\thefield{usera}\actualoperator}\mkbibindexname{#2}{#3}{#4}{#5}}}

% \newcommand{\noop}[1]{}

%remove for final
%\overfullrule=1mm

\newcommand{\tobi}[2]}}
\renewcommand{\S}[1]{\tobi{#1}{\textsc{*}}}

% this volume references
% puts: [this volume]
% already defined: \citetv
%\newcommand{\citepv}[1]{(\citeauthor{#1} \citeyear*{#1} [this volume])}
\newcommand{\citealtv}[1]{\citeauthor{#1} \citeyear*{#1} [this volume]}

%parentheses around example number
\newcommand{\pref}[1]{(\ref{#1})}

% in-text examples

\newcommand{\lnex}[1]{\textit{#1}} %target lang word
\newcommand{\lnlit}[1]{(lit.: `#1')} %literal reading
\newcommand{\lnlat}[1]{(#1)} % latinization
\newcommand{\lntrans}[1]{`#1'} %translation
\newcommand{\lnexl}[2]%
{\lnex{#1}{} \lnlat{#2}} % ex with latinization
\newcommand{\lnexlat}[3]{\lnex{#1}{} \lnlat{#2}{} \lntrans{#3}} % ex with latinization and tranl.

%ch01
\newcommand{\co}[1]{\mbox{\textbf{#1}}}

%ch09

\newcommand{\cyrbulg}[1]{\begin{otherlanguage*}{bulgarian}#1\end{otherlanguage*}}


%ch10
\newcommand{\nlp}{{\small NLP}}
\newcommand{\mwe}{{\small MWE}}
\newcommand{\rae}{{\small RAE}}
\newcommand{\lvc}{{\small LVC}}
\newcommand{\pos}{{\small P}o{\small S}}
%\newcommand{\todo}[1]{ \textcolor{red}{#1} }

%\renewcommand{\labelenumi}{\theenumi}
%\ainamefmt{{vv}{ll}{, ff}{, jj}} % fullname

\newcommand{\biberror}[1]{{\color{red}#1}}

\newcommand{\osenovaitem}{--~}
   %% hyphenation points for line breaks
%% Normally, automatic hyphenation in LaTeX is very good
%% If a word is mis-hyphenated, add it to this file
%%
%% add information to TeX file before \begin{document} with:
%% %% hyphenation points for line breaks
%% Normally, automatic hyphenation in LaTeX is very good
%% If a word is mis-hyphenated, add it to this file
%%
%% add information to TeX file before \begin{document} with:
%% %% hyphenation points for line breaks
%% Normally, automatic hyphenation in LaTeX is very good
%% If a word is mis-hyphenated, add it to this file
%%
%% add information to TeX file before \begin{document} with:
%% \include{localhyphenation}
\hyphenation{
    Beck-man
    Ngu-yen
    back-chan-nel
    back-chan-nels
    mo-not-o-nous
    ste-reo-typ-i-cal
}

\hyphenation{
    Beck-man
    Ngu-yen
    back-chan-nel
    back-chan-nels
    mo-not-o-nous
    ste-reo-typ-i-cal
}

\hyphenation{
    Beck-man
    Ngu-yen
    back-chan-nel
    back-chan-nels
    mo-not-o-nous
    ste-reo-typ-i-cal
}

   %\boolfalse{bookcompile}
   %\togglepaper[23]%%chapternumber
}{}

\begin{document}
\maketitle

\section{Introduction} 

Speakers’ utterances often contain presupposed parts as well as those that are emphasised as new or contrastive. This information must be rapidly integrated within an ever-evolving model of the discourse \citep{johnson1983mental}. Focus is a key category within this information structure, which has been likened, as a level of linguistic structure, to the ``packaging" of truth-conditional content that makes it fit within different contexts \citep{Chafe76}. Alternative meanings that speakers could have used but did not have been proposed as crucial to understanding focus \citep{Krifka2008}. Recent research using psycholingustic methods has found that when faced with individual focused words, comprehenders actually consider contextually related alternatives to those words in their minds in real time \citep{braun2010role,husband2016role,gotzner2019life}. In the current research, we aim at expanding this investigation to cases of focus on larger units, i.e., situations where whole phrases as opposed to single words are focused, and test whether comprehenders also represent alternatives to these more complex elements in the course of online processing.

Firstly, we cover the theoretic treatment of the phenomenon of focus to provide a base for our understanding of meaning alternatives. Next, we review the experimental evidence regarding the activation, selection, and representation of alternatives in real-time processing. Then, we discuss the distinction between narrow and broad focus. From this, we derive the motivation and design of our three experiments, which follow in the next section. Finally, we discuss our results both in relation to the current psycholinguistic investigations of focus but also to some of the theoretical debates.

\subsection{Focus and its theoretical explanations}
Let us now turn to the theoretical approaches that have been proposed to account for focus. Many researchers have broadly associated it with the ``new" or ``contrastive" parts of utterances in opposition to the known and presupposed parts \citep{Halliday1967Intonation,jackendoff1972semantic,sgall-hajicova-benesova73,SgallPanevova1986}. Here, we will mostly zoom in on an approach that sees meaning alternatives as crucial, as this is the basis of the current study. Additionally, we will also discuss a recent attempt to ground focus within a general framework of pragmatic reasoning, namely the Rational Speech Act model.


\subsubsection{The Roothian approach}
An influential semantic approach to focus claims that its main function is to introduce alternative meanings that could replace the focused element in the discourse \citep{Krifka2008}. This theory, proposed by \citet{rooth1985}, states that focus generates an additional level of meaning, which consists of a set of propositions derived by replacing the focused element with contextually plausible alternatives of the same semantic type. Take the following example:

\ea\label{ex:ex1} In the monastery, [Jane]\textsubscript{F} read the manuscript.
\z

According to Roothian alternative semantics \citep{rooth1992}, the focused subject noun \textit{Jane} 
in (\ref{ex:ex1}) gives rise to alternatives of the same semantic type as the focused element, here let us assume type \textit{e}. This formal set containing alternatives is then subject to a further process, namely contextual restriction. Suppose that there is a world containing four individuals -- Jane, William, George, and Whiskers the cat. The focus value of sentence (\ref{ex:ex1}) is then said to be the set of propositions that are derived by systematically replacing the focused element with its contextually appropriate alternatives. In our example, the set would be \{Jane, William, George\}. Even though Whiskers is also of type \textit{e}, it cannot, being a cat, read anything and thus is not contextually appropriate in (\ref{ex:ex1}). Equally, should the word \textit{manuscript} be focused in (\ref{ex:ex1}), alternatives such as \textit{letter} or \textit{scroll}, i.e. words or concepts that could be the objects of \textit{read}, would be generated.

One of the strengths of Rooth's (\citeyear{rooth1992}) theory is that it can also account for various focus-related effects in addition to bare focus. This includes the behaviour of focus-sensitive particles such as \textit{only}, \textit{also}, or \textit{even}. For example, if inserted into (1) before \textit{Jane}, \textit{only} would assert that the proposition expressed in (\ref{ex:ex1}) is true for the focused element, but that none of the propositions obtained by replacing \textit{Jane} with its alternatives are. Therefore, the communicated content is that neither William nor George read the manuscript. The analysis of \textit{only} is that it is a particle that takes a proposition to combine with and asserts that no other contextually relevant proposition is true \citep{fintel1997bare}. Focus is thus seen as one of the many phenomena in language where alternative utterances play a crucial role (see \citealt{gotzner2022meaning} for a review).

\subsubsection{Explanations within the RSA framework}

Another explanation of the phenomenon of focus in language is provided by the Rational Speech Act framework \citep{goodman2016pragmatic,franke2016probabilistic}, which has become a standard modelling tool for a variety of semantic and pragmatic phenomena in recent years. This framework provides a general model of speaker-hearer interactions and human cognition. Within this approach, non-literal meaning is modelled as successive Bayesian calculations on the side of the hearer about the intended meaning of the speaker's utterances. Hearers are said to compute probability distributions over the possible worlds in which any given utterance would be true. Crucially for the purposes of giving an account for focus, the RSA approach incorporates the variable of utterance cost into the computation that the hearer conducts. Within the RSA framework, pragmatic enrichment occurs due to some utterances being more costly to produce than others.

\citet{bergen2015strategic} attempt to combine the RSA approach with noisy channel theories \citep{stevens2016focus,stevens2019noise}. These noisy channel approaches see language as essentially a solution to the problem of transferring information between individuals in a situation where some of it can be lost or distorted \citep{ShannonWeaver1949}. They start their account by appealing to the prosodic marking of focus in languages such as English or German. What this prosodic prominence is said to achieve is to lower the probability that the word in question is misheard by the listener and potentially misinterpreted for one of its plausible alternatives, these may be previously given in the context or constructed by the comprehender. Given a word, this prominence also carries a certain cost on the part of the speaker. Within the RSA framework then, \citet{stevens2016focus} argues, the way focus alternatives are introduced is by means of the following chain of iterative Bayesian reasoning. In essence, upon hearing a word with prosodic prominence marking focus, the listener reasons that the speaker must have intended to expend extra production costs with the goal of improving the chances that the correct word would make it across the noisy channel of communication to the listener. If this were not done, the listener might take the word to be not the intended one, but one of its contextually plausible alternatives. Therefore, it must be that these other alternatives are in fact false according to the beliefs of the speaker given that they were willing to give the extra cost associated with the prominence. In this way then, the listener arrives at the conclusion that a prosodically prominent focused word carries with it an exhaustivity implicature, i.e. that its alternatives are false.

This is how the RSA model proposes to explain the phenomenon of focus alternatives. However, as has been perceptively pointed out by an anonymous reviewer of this chapter, there are several issues with this proposal that ought to be acknowledged. Whether or not something is a plausible alternative to a focused word is dependent, unsurprisingly, on the words' meaning. However, similarity in meaning does not, in the majority of cases, necessitate similarity in the phonological form. For example, should \textit{sheep} be focused in a particular sentence, its alternatives would presumably be \textit{goat} or \textit{cow}, words very different from \textit{sheep} phonologically and not likely to be misheard. \textit{Ship} on the other hand would in most cases not be a plausible alternative, yet this word is much more likely to be confused with \textit{sheep}.

Notice that the account is not mutually exclusive with that of \citet{rooth1992}. Rather, it embeds the effects of focus within a computational model of pragmatic reasoning. It could therefore be seen as complementary to the formal semantic theories that were presented earlier in the introduction. There are however differences, since, arguably, the RSA-based account relies on the notion of prosody playing a crucial role in the computation of focus alternatives (this is at least the case in Germanic languages). It could be argued that where there is no prosodic prominence in the marking of focus, alternatives should not be entertained by comprehenders. We will return to the role of prosody and the associated RSA-based explanation when discussing both the predictions and results of the current study.

\subsubsection{Processing focus}
What we have discussed so far was concerned with the interpretations and the associated formalisations of sentences with focused elements. We will now turn to how these interpretations are arrived at in the minds of comprehenders when perceiving and parsing such sentences in real time. This is the algorithmic level of analysis of cognitive phenomena \citep{marr2010vision} argued to be distinct from formal analyses, yet complementary to them and capable of influencing them \citep{love2015algorithmic}.

As far as online comprehension is concerned, both focus
and focus particles have been linked to distinctive effects with research suggesting that focused information is processed “more deeply” \citep{sturt2004linguistic,ward2007linguistic}, that focus enhances anaphor integration \citep{klin2004readers,sanford2009enhancement}, that it affects ellipsis processing \citep{frazier2007focus,carlson2015clefting}, and that it exerts an influence on parsing \citep{filik2005parsing}. Focus particles have been found to affect syntactic attachment \citep{carlson2022focus} and allow comprehenders to predict upcoming contrasts \citep{Carlson2013}.

Crucially for the purposes of the current study, the past decade has seen evidence suggesting that Rooth's (\citeyear{rooth1992}) semantic approach can also be applied to the online processing of focus by comprehenders (for an overview, see \citealt{gotzner2019life}). This was arrived at by examining the patterns of the activation, selection, and representation of alternatives in comprehenders when exposed to either spoken or written stimuli with focus marked either prosodically or syntactically. When we speak of \textit{activation} or \textit{representation}, what we wish these terms to refer to are phenomena within the real-time processing of language.

Many experimental paradigms have been used in focus alternative research. However, two in particular have been implemented extensively, namely the lexical decision \citep{braun2010role,byram2011focus,husband2016role,gotzner2016impact,yan2019priming,yan2022role} and probe recognition tasks \citep{gotzner2016impact,gotzner2017role,jordens2020role,spalek2019neurocognitive}. Below, we review some of the studies using these two tasks that have been instrumental in establishing the processing reality of Roothian-style alternatives.

\citet{husband2016role} examined whether focus alternatives are activated in processing and what the time-course of this activation is. In their study, English speakers heard sentences in which a particular word (here \textit{sculptor}) was pronounced either with contrastive (L+H*) or non-contrastive (H*) prosody:

\ea\label{ex:ex2} The museum thrilled the \textit{sculptor} when they called about his work.
\z

Contrastively (\textit{painter}) and noncontrastively (\textit{statue}) associated probe words were subsequently presented. In their two experiments, they found that when the probe word was presented immediately after the prime word (0ms SOA) both contrastively and non-contrastively associated words were activated (in contrast to unrelated words)\footnote{The researchers also presented non-words as probes in their filler items to balance the ratio of yes-no responses.}. However, when there was a 750ms delay relative to the prime word, only the contrastively associated probes remained activated. \citet{husband2016role} measured this activation by means of a lexical decision task where their participants had to judge whether a given probe was an existent word of English or not. This pattern suggested that comprehenders were generating sets of focus alternatives in real time, since conditional upon prosodic contrastive focus marking, only those words that could replace the focused element were activated. It also appears that in the course of processing focus, the mechanism first activates broadly associated words and only afterwards selects the final set of focus alternatives.

Regarding focus particles, it has been found that they have additional effects on the processing of alternatives. \citet{gotzner2016impact} conducted a study, in which their German participants had the task of indicating whether a given probe word appeared anywhere in discourses such as these: 

\ea\label{ex:ex3} In the fruit bowl, there are peaches, cherries, and bananas.
\ex\label{ex:ex4} I bet Carsten has eaten cherries and bananas.
\ex\label{ex:ex5} No, he \_/only/even ate the [peaches]\textsubscript{F}.
\z

In this study, the noun was always spoken with a contrastive pitch accent (L+H*). The final sentence also included either one of two focus particles, \textit{only} or \textit{even}, or neither was present. They used the probe recognition task, in which participants have to judge whether a given word appeared anywhere in the previously presented stimulus, while their accuracy and response times are measured. This task has been argued to tap into the mental representation of the discourse \citep{gernsbacher1995cataphoric} as opposed to the immediate activation within the lexical-semantic network, which is often studied with the lexical decision task \citep{meyer1971facilitation}. Their results revealed that the presence of focus particles interfered with the recognition of mentioned alternatives (\textit{cherries}) as well as the rejection of unmentioned yet plausible alternatives (\textit{melons}). Both types of probes were reacted to even more slowly (compared to unrelated words) when either \textit{only} or \textit{even} was present. This additional slowdown caused by focus particles was later shown to be specific to contrastive alternatives by \citet{gotzner2017role}, who found that when general associates of focused nouns were presented as probes, no interference effects of focus particles were observed. 

\citet{gotzner2017alternative} interprets these results as an indication of a competition process taking place between the mentioned alternatives (\textit{cherries}, \textit{bananas}), the unmentioned alternatives (\textit{melons}) and the focused element (\textit{peaches}). Comprehenders are said to create a place holder upon processing a focus sensitive particle \citep{gotzner2017alternative}, since in order to successfully parse the exhaustive or additive meaning conveyed by the particle, alternatives to the focused element must be represented. This place holder then is sensitive to what could replace it. Since both the mentioned and unmentioned alternatives match the place holder, interference occurs. The results of both \citet{gotzner2016impact} and \citet{gotzner2017role} show that focus particles create additional effects compared to bare intonational focus and that they are specific to alternatives appropriate within the utterance's context. As such, these focus particle interference effects can be said to be litmus test of the representation of focus alternatives in cases where focus itself is not manipulated.

Further studies have replicated the effects exhibited by focus alternatives either in languages other than English (e.g. see \citealt{braun2010role}, for Dutch; \citealt{jordens2020role}, for German; \citealt{yan2019priming,yan2022role}, for Mandarin Chinese; \citealt{Calhoun_et_al_2022}, for Samoan; \citealt{tjuka2020foxes}, for Vietnamese; \citealt{kaldi2021hungarian}, for Hungarian; \citealt{Lacina_Czech_2023}, for Czech) or using different paradigms (see \citealt{kim2015context}; \citealt{braun2019not} for eye-tracking), and have examined how focus influences the memory recall of alternatives \citep{fraundorf2010recognition,lee2017effects,norberg2021memory}. What the evidence converges on is that focus alternatives indeed play an important role during online comprehension, in addition to being a valuable tool in the domain of formal semantic theory.

\subsubsection{Broad focus}
However, the alternative-based account of the processing of focus has so far only been tested on cases of narrow focus, that is on sentences where a single word is focused. It has long been known that focus scope can vary and that it can encompass a constituent larger than a single word \citep{selkirk1995sentence,gussenhoven1999discreteness,erteschik2007information}. For example, a whole VP including the verb and its direct object may be focused. Consider the contrast between the two following sentences:

\ea\label{ex:ex6} In the monastery, Jane read the [manuscript]\textsubscript{F}.
\ex\label{ex:ex7} In the monastery, Jane [read the manuscript]\textsubscript{F}.
\z

Cases such as (\ref{ex:ex7}) are known as broad focus and contrast with narrow focus constructions seen in sentence (\ref{ex:ex6}). This phenomenon is readily accommodated under Rooth's (\citeyear{rooth1992}) formal account. The principle of substituting elements of the same semantic type is maintained. The only difference from narrow focus is the type being substituted. Let us take (\ref{ex:ex7}) as an example. Here, the phrase \textit{read the manuscript} is in focus and its meaning is standardly analysed as being type <\textit{e,t}> \citep{kratzer1998semantics}. According to \citet{rooth1992}, its alternatives should also be of this type. The set of alternatives for (\ref{ex:ex7}) might then consist of, for example, {λ.x [x sealed the letter]; λ.x [x wrote the scroll]}.

One feature of broad focus that ought to be mentioned is that these structures are often ambiguous with narrow focus ones. This is the phenomenon of focus projection \citep{rochemont1986focus,von1986some,selkirk1995sentence}, where in languages such as English, prosodic prominence marking focus is placed at the right edge of the broad-focused phrase. As a consequence, it has been claimed that this is prosodically indistinguishable from narrow focus on the right-most element \citep{ladd2008intonational}. This debate has not been conclusively settled with some evidence suggesting that speakers tend to give larger prenuclear prominence to verbs under broad focus compared to narrow focus situations (e.g. \citealt{breen2010acoustic}). What is known, however, is that the disambiguation between broad and narrow focus can be made by means of the preceding context \citep{Buring2007}, for example by an explicit question-under-discussion \citep{roberts2012information} or by means of what is given in the context, according to the Givenness Principle \citep{schwarzschild1999givenness}.


\subsubsection{The question of the processing of broad focus}

Broad focus is arguably crucial to our understanding of how alternatives operate. Given the alternatives' more complex structure, new questions arise both on the level of formal analysis \citep[e.g.][]{fox2011characterization} and processing regarding their generation and representation. However, no study to-date has tested whether the comprehension results presented in the previous section obtain for these broad focus cases as well. While there has been a processing study on broad focus, namely that of \citet{bishop2017focus}, it only examined the above-mentioned role of the prenuclear accent on the verb in marking broad focus and did not test either verb or whole-phrase alternatives.

This question is crucial for the study of how people process information structure and incorporate focused elements into their evolving interpretation of meaning. We believe this to be the case, given that there is a clear prediction from Rooth's (\citeyear{rooth1992}) formal semantic theory -- broad focus is semantically analogous to narrow focus and therefore, comprehenders ought to exhibit the same patterns of the representation of alternatives that research has shown for narrow focus when they encounter broad focus. In essence, we are asking whether the Roothian-inspired processing theory truly generalises to all the cases that the formal theory of \citet{rooth1992} applies to. Below, we report our attempt to test this prediction by means of three experiments.

\subsection{The current study} 
In the current study, we aimed to test whether the alternative-based approach to the processing of focus generalises to cases of broad focus. We put forward the following hypothesis:

\ea Alternative Representation Hypothesis\\
Comprehenders create representations of contextually appropriate alternatives to the focused element concordant with its semantic type.
\z

As far as our account of the processing of focus is concerned, the predictions of this Roothian-inspired (\citeyear{rooth1992}) processing approach, with the central claim that alternatives are being entertained in the minds of comprehenders in real-time processing, are clear. In case a comprehender encounters a sentence in which the whole VP is focused, they ought to activate, select, and represent alternatives to this larger constituent. Likewise, the constituent parts of these alternatives ought to give rise to an enhanced activation and representation of their constituent parts. In the case of focused VPs with transitive verbs, these would be both the verbs and nouns within the alternative phrase.

As for what the RSA-based approaches might give us for predictions in the case of broad focus, these depend on the question of whether broad and narrow focus are prosodically distinct. Should only the right-most element receive prominence, then it could be argued that alternatives should only arise for this element and not for the whole VP. Therefore, if we do not see evidence for VP-level alternatives in processing, this could be seen as consistent with the RSA approach. Should, however, the two have different prosodic profiles, the RSA model could incorporate whole VP alternatives.

We conducted three probe recognition experiments aimed at testing the Alternative Representation Hypothesis. As our starting point, we took the interference effect of focus particles that has been identified by the research of \citet{gotzner2016impact}. Should the previous results obtained for narrow focus fully generalise to cases of broad focus, we would expect the same interference pattern. We reasoned that if this effect were to be found when rejecting unmentioned alternatives to focused VPs, this would constitute evidence in favour of these alternatives being activated, selected, and represented by comprehenders. We also predicted that the alternatives would exhibit the interference effect of \textit{only}, this should also be the case for their constituent parts. We thus predicted that if the Alternative Representation Hypothesis is true, we ought to see \textit{only} interfering with alternatives to both nouns, verbs as well as whole phrases. Remember that this is because the interference effect of focus particles has been interpreted in the literature as a sign of the additional unmentioned alternatives being activated and competing for selection \citep{gotzner2016impact}. Should we not find this interference, this would go against the straightforward generalisability of the Roothian-inspired (\citeyear{rooth1992}) processing approach to broad focus.

We constructed discourses designed to elicit a broad focus interpretation in the final sentence presented. Given that we were working with texts, no explicit prosody could potentially distinguish between different focus structures. To this end, we used context sentences preceding the critical one. These context sentences first set up an assertion concerning a particular event. Then, a sentence, contrasting both in the action performed and in the patient affected, follows, and either includes the focus particle \textit{only} or not. What is then probed are unmentioned yet plausible alternatives to broadly focused phrases. 

Below, we report three probe recognition experiments. In all, the stimuli were presented in the rapid serial visual presentation mode (RSVP). The results of \citet{byram2011focus} showed that written stimuli can induce focus alternative effects when \textit{only} is present. The experiments were web-based; the first two were hosted on the IbexFarm platform \citep{DrummondA2013}, whilst Experiment 3 was conducted using PCIbex \citep{ZehrSchwarz2018}. The first two experiments probed alternatives (i.e. tested the speed of rejection of unmentioned plausible alternative words compared to unrelated ones) to the constituent nouns (Experiment 1) and verbs (Experiment 2) within focused VPs, whilst Experiment 3 tested whole phrase alternatives to focused VPs.

\section{Experiments 1 \& 2} 

In Experiment 1, we aimed to test alternatives to the nouns within the focused phrase (e.g. \textit{letter} for \textit{manuscript}). Experiment 2 tested verbal alternatives (\textit{wrote} for \textit{read}). This means that we presented participants with stimuli, in which we set up a broad-focused phrase. This phrase was then either bare or preceded by the focus particle \textit{only}. Detecting the established interference of \textit{only} when plausible alternatives to either the noun or verb are presented to comprehenders would be evidence for their representation in real-time focus processing.

Both experiments were set up according to a 2 $\times$ 2 factorial design. Firstly, we manipulated particle presence, i.e. whether the focus particle \textit{only} was present in the critical sentence of the stimuli, i.e. in the sentence where a broad-focused phrase was present. The second factor was alternative status, in which we manipulated the type of probes that participants were reacting to. Either alternative or unrelated probes were shown. This means words that could plausibly replace the corresponding part of the focused phrase within the given context or words whose meaning is contextually incompatible with this replacement.

Given the discussion of our hypothesis and general predictions, we expect to detect the following effects. Firstly, we predict a main effect of the alternative status manipulation, since a general relatedness effect has been reported by many studies that used the probe recognition task with focus alternatives \citep{gotzner2016impact,jordens2020role}. We predict that related probes will be rejected more slowly compared to unrelated ones. Critically, we predict an interaction between particle presence and alternative status. We expect \textit{only} to interact with the plausible unmentioned alternatives and not with unrelated probes and this to be evidenced in the response time measure. If seen, this would be evidence of an interference effect caused by the focus particle \textit{only} and thus consequently, an indication that the constituent part of the alternative is being represented by comprehenders within the mechanisms responsible for processing focus.

\subsection{Method}

\subsubsection{Participants}

62 (mean age 22.9) native English speaker participants were recruited to take part in Experiment 1. The participants did not receive any monetary compensation for their time. In Experiment 2, 60 (mean age 32.9) native English speakers took part. These were recruited on the Prolific platform and received £5 for their time. Full demographic information can be found on an OSF project entry (\url{https://osf.io/uvbdr/}).

\subsubsection{Materials}

We constructed 40 experimental items based broadly on the stimuli used by \citet{gotzner2016impact}. They can be found on the OSF project page mentioned above. Each item consisted of four sentences which introduced a context together with an agent, then alternative actions that the agent could perform, an assertion of one of them and then finally a negation of this action and a correction. Take the following example:\footnote{During the review process, it has been pointed out to us that some of our items might have the issue that the third sentence contains a presupposition of the other alternative and that the final sentence denies it. However, since what we are testing are unmentioned alternatives that are not presupposed, we do not believe this to be a detrimental issue of the design.}

\ea\label{ex:ex8} Harry is a butcher.
\ex\label{ex:ex9} At the butcher's shop, Harry could smoke and carve the ham and the brisket.
\ex\label{ex:ex10} Harry surely carved the brisket.
\ex\label{ex:ex11} No, he \_/only smoked the ham.
\z

The items were constructed in such a way to induce a broadly focused interpretation of the VP in the final sentence. There, we also manipulated the presence or absence of the focus particle \textit{only}, i.e. the particle presence manipulation. Next, the items differed in the probe words that were presented after the final sentence. These were either plausible yet unmentioned alternatives or unrelated words. We also included mentioned alternatives in the second context sentence (\ref{ex:ex9}). These were constructed in such a way to elicit in the mental model of the comprehenders the possible actions that the agent of the scenario could take with the aim that any permutation of the conjoined verbs and nouns would be included. In Experiment 1, we probed alternatives to the object noun (\textit{ham}) within the VP (alternative: \textit{sirloin}; unrelated: \textit{mastiff}). In Experiment 2, the verbs (\textit{smoked}) within those phrases had their alternatives tested (alternative: \textit{salted}; unrelated: \textit{distanced}). This was the alternative status manipulation. These probe words were controlled for letter length and the log-frequency of word forms in the British National Corpus. We also conducted a latent semantic analysis \citep{landauer1997latent} to measure the degree of association between the probe words and the focused ones with the goal of maximising it for alternative probes and minimising it for unrelated ones. The descriptive statistics and models computed for these purposes can be accessed on the OSF project entry. 

Since in the current study, the correct answer to the probe recognition task was always \textit{no} for the experimental items, we created a set of 80 filler items, out of which 60 required the participant to answer \textit{yes}.

\subsubsection{Procedure}

Every trial consisted of four sentences presented in the RSVP mode. Words appeared individually on the screen for 300ms followed by 100ms of a blank screen. When the final word was reached, 2000ms of a blank screen followed. Next, a probe word appeared in capital letters together with a forced choice of \textit{yes} or \textit{no} that was made by pressing \textit{j} for the former and \textit{k} for the latter. In the instructions, participants were told that their task was to indicate whether the given probe word appeared anywhere in the preceding four sentences. They were told that ``any form of the word" would count as the word having appeared. For example, if the form \textit{dogs} was found in the stimulus and the probe word was \textit{dog}, the correct response was to be \textit{yes}. The participants were instructed to answer as quickly as possible. There was no timeout for their answers and no feedback was given. Before the experiment proper, the participants were given example items with their correct responses and completed a practice part. Each participant saw 40 experimental items and 80 fillers with no two experimental trials in immediate succession organised in a list according to the Latin Square design.

We collected participants' responses (\textit{yes} or \textit{no}) as well as the associated response times. These were measured from the moment of the appearance of the probe on the screen.

\subsubsection{Analysis}

We fitted Bayesian hierarchical models to the log-transformed response time data. Only those trials where participants correctly rejected the probe were included in the analysis. We used the \texttt{brms} package \citep{Burkner2017} in the R programming language \citep{RCoreTeam}. We included the factors of alternative status, particle presence and their interaction, as well as mean-centred ordinal trial position as fixed effects and the three-way interaction between the factors. The factors of alternative status and particle presence were sum coded. In our random effects structure, we included the full structure justified by the design of the experiment \citep{barr2013random}, i.e. random intercepts for both participants and items as well as random slopes for both. For the full specification of the models used here, consult the preregistration entry for Experiment 3 on OSF (https://osf.io/cf36w). 

Below, we report the posterior distributions of the sizes of the main effects and interactions given the data and the priors, along with their 95\% credible intervals (CrI). In cases where the credible interval of the posterior distributions of the size of an effect does not include zero, we will consider this to be compelling evidence for the hypothesis that the size of the effect is different from zero \citep{franke2019bayesian}. 

\subsection{Results}
Firstly, we report the observed response times of the correct rejections of unmentioned probes in \figref{fig:fig1} (Experiment 1) and \ref{fig:fig2} (Experiment 2). The data are divided by block, i.e. into the first and second half of trials for each participant. The data are reported following outlier removal. This was done to show the variability of data observed in Experiment 1, but not in Experiment 2.

\begin{figure}[p]
\includegraphics[height=.4\textheight]{2_Exp1_clean_block_plot.pdf}
\caption{Mean response times and standard errors of correct rejections after outlier removal by condition and block in Experiment 1 (Nouns)}
\label{fig:fig1}
\end{figure}

\begin{figure}[p]
\includegraphics[height=.4\textheight]{2_Exp2_clean_block_plot.pdf}
\caption{Mean response times and standard errors of correct rejections after outlier removal by condition and block in Experiment 2 (Verbs)}
\label{fig:fig2}
\end{figure}

The Bayesian models fitted to the data produced posterior distributions of the parameters given both the priors and the data (see Figures  \ref{fig:fig3} and \ref{fig:fig4}). In both Experiment 1 and 2, the models fitted to the response time data provided compelling evidence for the main effect of alternative status being larger than zero (Exp1: $\beta = 0.082$, CrI [0.055, 0.110]; Exp 2: $\beta = 0.104$, CrI [0.070, 0.139]). This means that alternative probes were being rejected more slowly compared to unrelated ones. In neither experiment did we see compelling evidence for the main effect of particle presence, since in both experiments, the credible intervals included zero (Exp1: $\beta = -0.008$, CrI [$-0.023, 0.007$]; Exp 2: $\beta = -0.006$, CrI [$-0.028, 0.015$]). As for the interaction of alternative status and particle presence, the posterior distributions also included zero within their 95\% CrIs (Exp1: $\beta = 0.001$, CrI [$-0.013, 0.015$]; Exp 2: $\beta = 0.0004$, CrI [$-0.023, 0.022$]). Where the results differed between the experiments was in the three-way interaction of the two manipulations and centred trial order. In Experiment 1, there is compelling evidence for this three-way interaction effect being larger than zero ($\beta = 0.0012$, CrI [0.0001, 0.0024]). This means that there was more interference at the beginning of the experiment and that the effect seemed to have been evolving over the course of the experimental session. In Experiment 2, however, the model did not compellingly show this three-way interaction ($\beta = 0.0007$, CrI [$-0.0007, 0.0022$]).

\begin{figure}[p]
\includegraphics[height=.4\textheight]{2_Exp3_main_grey.pdf}
\caption{Posterior probabilities and 95\% CrIs for the parameters of interest in Experiments 1 and 2}
\label{fig:fig3}
\end{figure}

\begin{figure}[p]
\includegraphics[height=.4\textheight]{2_Exp1_2_threeway_grey.pdf}
\caption{Posterior probabilities and 95\% CrIs for the three-way interaction between particle presence, alternative status, and trial position in Experiments 1 and 2}
\label{fig:fig4}
\end{figure}

We find some evidence that replicates the previously identified interference effects of \textit{only} in the case of noun alternatives (Experiment 1). These were not extended to the case of verbs (Experiment 2), for which we saw no evidence of interference. Therefore, these results are on the whole inconsistent with our predictions, since we expected both the noun and verb constituent parts of focused VPs to show the interference of \textit{only}. However, Experiments 1 and 2 cannot provide a conclusive falsification of the hypothesis. This is due to it being possible that the constituent parts of focused VPs could be represented differently depending on their word class and that they could give rise to different patterns of interactions with focus particles while whole VP alternatives are being represented. In other words, the interference pattern of \textit{only} might still hold for the whole phrases while being present in the alternatives to only some constituent parts of the focused phrases. Therefore, in order to fully test the hypothesis, we conducted Experiment 3.

\section{Experiment 3}

Since Experiments 1 and 2 only tested probes that were alternatives to the constituent parts of focused VPs rather than to the VPs as a whole, we ran Experiment 3, in which we examined the representation of whole alternative phrases. This experiment was pre-registered and the time-stamped entry can be found on the OSF platform (\url{https://osf.io/cf36w}). The design and predictions were the same as those in Experiments 1 and 2.

\subsection{Method}

\subsubsection{Participants}

121 participants aged 20 to 30 (mean age 24.6) recruited on the Prolific platform (£3.13 in compensation) took part in the experiment (see OSF for more information). From this set, we excluded one participant for not being a native speaker of English.

\subsubsection{Materials}

The materials used in Experiment 3 had the same structure as the first two experiments. However, we reduced the number of items to 24 in order to lessen the load on participants and shorten the length of the experimental session. Take the following example of a critical sentence:

\ea\label{ex:ex12} Lily is a tailor.
\ex\label{ex:ex13} At the workshop, Lily could sew and stitch the shirt and the skirt.
\ex\label{ex:ex14} Lily surely stitched the skirt.
\ex\label{ex:ex15} No, she \_/only sewed the shirt.
\z

We probed either whole alternative phrases (\textit{knitted the scarf}) or unrelated ones (\textit{published the study}). Following the creation of the probes for this experiment by combining the noun and verb probes from the first two experiments, we ran a naturalness rating study with this set of stimuli. We found a statistically significant difference in naturalness ratings between the alternative and unrelated probes with the latter being rated as less natural. Believing this to be a potential confound for the main experiment, we replaced the lowest rated probes with highly rated filler items that were a part of the rating study. Finally, we conducted a new set of analyses of naturalness ratings, log-frequencies and letter-length of the nouns and verbs, and the LSA measures of relatedness of the noun probes to the object noun and verb probes to the main transitive verbs found in the focused phrase and no confound was found (see the OSF entry for more information).

\subsubsection{Procedure}

The procedure was nearly identical to that of Experiment 2 with only the response keys changed to \textit{y} for \textit{yes} and \textit{n} for \textit{no}. 

\subsubsection{Analysis}

Firstly, two items (14 and 15) were excluded from further analyses, since an error in the construction of their probes was discovered after data collection.

As per our pre-registration, the log-transformed response times of correct trials were analysed by a Bayesian hierarchical model with trial order, alternative status, particle presence, the two-way interaction of alternative status and particle presence, and the three-way interaction of trial order, alternative status and particle presence as fixed effects. The full specification of the model can be viewed on the OSF platform.

\subsection{Results}

The reader can see the observed response times for Experiment 3 summarised in \figref{fig:fig5}. We report the means of correct rejections after outlier removal by condition and block.

\begin{figure}
\includegraphics[height=.4\textheight]{2_Exp3_clean_block_plot.pdf}
\caption{Mean response times and standard errors of correct rejections after outlier removal by condition and block in Experiment 3 (Phrases)}
\label{fig:fig5}
\end{figure}

In \figref{fig:fig6}, we display the distributions of the posterior probabilities of the parameters of interest from the Bayesian model fitted to the log-transformed response time data. These are the main effects of particle presence and alternatives status, and their interaction. The model showed compelling evidence that alternative probes are rejected substantially more slowly compared to unrelated probes ($\beta = 0.088$, CrI [0.048, 0.126]). On the other hand, the maximum likelihood estimate for the size of the effect of the presence of \textit{only} is close to zero ($\beta = 0.0006$, CrI [$-0.0146, 0.0156$]). Crucially for our hypothesis, the model together with the data and given our priors does not provide us with compelling evidence to assert that the effect of the interaction between alternative status and particle presence is different from zero ($\beta = 0.00$, CrI [$-0.01, 0.02$]). Neither was there evidence (see \figref{fig:fig7}) for a three-way interaction of particle presence, alternative status, and centred trial order ($\beta = -0.0001$, CrI [$-0.0021, 0.0019$]).

These results are again not in line with our predictions, since they do not give us evidence for \textit{only} causing further interference in the rejection of whole alternative probes. Neither do we see this effect emerging through an interaction with trial order.

\begin{figure}[p]
\includegraphics[height=.4\textheight]{2_Exp3_main_grey.pdf}
\caption{Posterior probabilities and 95\% CrIs for the parameters of interest in Experiment 3}
\label{fig:fig6}
\end{figure}

\begin{figure}[p]
\includegraphics[height=.4\textheight]{2_Exp3_threeway_grey.pdf}
\caption{Posterior probabilities and 95\% CrIs for the three-way interaction between particle presence, alternative status, and trial position in Experiment 3}
\label{fig:fig7}
\end{figure}
\clearpage

\section{General discussion}

\subsection{Results summary}

We conducted three probe recognition experiments aimed at testing whether the Alternative Representation approach to the processing of focus based on Rooth’s (\citeyear{rooth1992}) semantic theory, the Alternative Representation Hypothesis, can be generalised to cases other than narrow focus, namely broadly focused VPs. We investigated this by means of using the interference effect of \textit{only} on the rejection of unmentioned alternatives \citep{gotzner2016impact,gotzner2017role} as a litmus test of whether focus alternatives were being activated, selected and represented by comprehenders.

Experiments 1 and 2, which targeted alternatives to the constituent nouns and verbs respectively, revealed a pattern that was not predicted. In neither experiment did we observe the simple interaction of our two manipulations, alternative status and particle presence. However, in the case of Experiment 1 only, the model and the data provided compelling evidence for a three-way interaction of the factors with trial order. We take this to be indicative of an interference effect, which additionally seems to be evolving over the course of the experimental session. Therefore, we have evidence for the interference by the presence of \textit{only} for alternatives to nouns, but not for verbs. Finally, Experiment 3 with entire phrases also did not provide compelling evidence for the presence of the predicted interference effect. Neither was there any evidence for a three-way interaction.


\subsection{Implications for the Alternative Representation account}

The results of our experiment are mixed and overall speak against the straightforward extrapolation of previous research and the Alternative Representation model to cases of broad focus. This is because this model predicted that the interference effects of \textit{only} would be present in broad focus. This would be equally the case in the experiment which tested constituent parts of the alternatives, as well as when whole phrases would be tested. The current research tentatively suggests that our credence in the presence of this interference effect on the rejection of plausible alternatives ought to be lowered in cases of broad focus. While the current results fail to conclusively support the Alternative Representation Hypothesis, it remains nevertheless plausible that comprehenders represent alternatives to broadly focused phrases without the emergence of the interference pattern of focus sensitive particles identified in the cases of narrow focus. 

We take the fact that in all three experiments, especially in Experiment 3, the main effect of relatedness was observed to be indirect evidence in favour of our comprehenders representing phrasal alternatives. That we observed inhibition in the case of phrases suggests that comprehenders were in fact representing them -- adopting the view of \citet{gotzner2016impact} for single nouns, this might possibly have been due to competition with the mentioned alternatives. This evidence would have been more conclusive had the interaction with the focus particle \textit{only} been present. Since it remains possible that the behaviour of focus particles in processing differs between narrow and broad focus, a direct test of the presence of alternatives in broad focus constructions would give us even more credence.
We put forward ways in which this could be done by explicitly manipulating focus in \sectref{future.directions}.

What requires discussion is the fact that as far as Experiments 1 and 2 are concerned, the interference was observed only in the case of alternatives to the constituent nouns within broadly focused phrases and that this effect was modulated by trial order in the case of Experiment 1.


The first plausible explanation assumes that alternatives to broadly focused phrases are being represented by comprehenders and postulates that it is due to the peculiar processing features of nouns that these effects only occur while not being present in either verbs or whole phrases. In other words, the source of the divergence may be sought in the general processing profile that has been claimed to exist between nouns and verbs. For instance, \citet{soloukhina2018investigating} studied the comprehension of verbs and nouns both in healthy individuals as well as in people with aphasia. They found differences in reaction times between verbs and nouns in a matching picture task in both populations, with verbs being reacted to more slowly.


The second explanation lies in the interpretation given by our participants to the presented stimuli. Even though the discourses were designed to elicit broad focus interpretations in the final sentence of each item, an interpretation with narrow focus on the direct object might be possible due to the phenomenon of focus projection \citep{selkirk1995sentence,gussenhoven1999discreteness} mentioned in the introduction, which refers to the direct object receiving focus prosody both in cases of narrow and broad focus, making the scope of focus ambiguous, according to some researchers. Furthermore, for reasons of comparability, we modelled our stimuli on the study of \citet{gotzner2016impact}, which uses corrective focus. While corrective and contrastive uses of focus have been equated in the literature \citep[e.g.,][]{zimmermann2008contrastive}, the use of negation might have biased our participants towards a narrow focus interpretation. Our results could therefore be due to participants in fact interpreting the final sentences as encoding narrow focus on the noun.


The influence of trial order could also be accommodated within this explanation, since while comprehenders might have started with the broad focus interpretation, the presentation of single noun probes might have caused them to change their preferred reading of the sentences towards narrow focus. This would be consistent with the interference effect being absent both in the case of probing whole phrases as well as in probing the verbal constituents in Experiment 2 and only present in Experiment 1. Since verb alternatives could have therefore been completely bypassed from the comprehenders’ processing of our stimuli, they would not be expected to be interfered with by \textit{only}. The interaction found in Experiment 1 (nouns) would therefore be consistent with the rest of the literature on the effects of \textit{only} in the processing of narrow focus \citep{gotzner2016impact,gotzner2017role}.


There are several potential reasons for the lack of an interaction effect in verbs. One could be that NPs and VPs are differently restricted. While the object NPs are faced with selectional verb restrictions, this is not the case for the whole VP, which is only restricted by (non-linguistic) context. This could have been one of the sources of the difference found between nouns and verbs in Experiments 1 and 2.
Another reason for why interference effects were not found in the rejection of verbal probes might have been the greater recency of the direct object noun, as one reviewer pointed out. Likewise, processing strategies could have influenced the results. We believe these issues ought to be addressed in further research by the employment of a cross-modal presentation method with auditory stimuli and textual probes. This would also address the issue of the lack of explicit prosody that makes it possible for comprehenders to assign implicit prosody to different elements and obscure the results.


Overall, the above-mentioned methodological issues mean that as far as the implications for theories of alternative representation in comprehension go, caution is to be taken. However, the presented evidence suggests that the probe recognition methodology can be extended to larger constituents and provides some initial evidence that comprehenders represent alternatives to broad-focused phrases.

\subsection{Implications for the RSA-based approaches}

The above mentioned explanation of the pattern of our results warrants further discussion in connection to the approach taken by the proponents of the RSA framework to focus. Given the phenomenon of focus projection, the distinction between broad and narrow focus in English is often not possible to make, given that only the rightmost element receives prosodic marking. In the case of our stimuli, had they been presented auditorily, this prosodic marking would be on the direct object noun. Therefore, prosodic prominence is arguably realised only on the noun and not on the verb (under the assumption that broad and narrow focus are prosodically indistinguishable). Assuming that this translates to implicit prosody employed by our participants when reading our sentences, this could be seen as supporting a view of focus being exclusively tied to prosodic prominence via the mechanism proposed within the RSA framework \citep{bergen2015strategic,stevens2016focus,stevens2019noise}. 


The reasoning is as follows -- given the ambiguity of the scope of focus, listeners do not perceive a difference in the expended effort on part of the speaker on anything but the focused object noun. Consequently, the pragmatic reasoning employed by the listener would be the same as in the cases of narrow focus. This would then be consistent with our findings, which showed the predicted pattern only in the case of alternatives to the nouns within broad-focused phrases. This, however, would also lead to the RSA approach needing to be amended to be able to deal with the effects associated with broad focus, such as its felicity given certain QUDs.


All in all, the results of the current study are not decisive with regards to the RSA approach to focus. We have highlighted here how our experiments may be interpreted in light of these proposals, yet we are also of the opinion that this theory needs to be spelled out more for researchers to be able to test it properly.


\subsection{Future directions}\label{future.directions}

The possibility that the comprehenders in this study interpreted the stimuli with narrow focus necessitates further studies that would use either explicit questions-under-discussion to elicit broad focus or unambiguous broad focus constructions such as pseudoclefts:

\ea\label{ex:ex16} What Jane did was read the manuscript.
\z

If seen, differential patterns of representation or activation between appropriate and inappropriate alternatives dependent on the type of the clefted constituent would provide evidence in favour of the Roothian-inspired Alternative Representation approach.

The observed differences between the noun and verb alternatives also highlight a gap in the literature. The research on the comprehension of focus has mostly examined focused nouns, leaving other word classes understudied (but see \citealt{ito2008anticipatory}, \citealt{fraundorf2010recognition}, \citealt{kurumada2014or}). To our knowledge, there has not yet been a study examining the processing of alternatives with narrow focus on either finite verbs or infinitives. Furthermore, the issue of how narrowly focused verbs differ in their processing profile from narrowly focused nouns remains to be examined.


\subsection{Methodological implications}

Moving finally onto further implications of this research, the unambiguous evidence for alternative probes in all three experiments being rejected more slowly compared to unrelated ones is, we believe, methodologically significant. While the probe recognition task has been conducted with either single words as probes (e.g. \citealt{cowan2013does}) or with entire sentences (e.g. \citealt{radvansky2005age}), there has been, to our knowledge, no research that used phrases in the probe recognition task. That we observed the main effect of alternative status in Experiment 3 suggests that probing whole phrasal constructions in the probe recognition paradigm can be done. The pattern of observed results is in line with what has been observed in this task in the situation when single words are used for probes. We observed an interference of semantically associated phrases on the rejection of unmentioned probes in the expected direction and thus, our research constitutes the first piece of evidence in favour of this extension of the probe recognition. This then opens novel possibilities for the use of this paradigm in psycholinguistic research. 

\section{Conclusion}
This study reported the first attempt to test the generalisability of the Alternative Representation theory of the processing of focus based on the semantic approach of \citet{rooth1992} to larger focused phrases. Regarding constituent parts, the results of Experiments 1 and 2 were contrary to our predictions. We have some evidence that focus particles operate on noun alternatives, but we do not have evidence that they operate on verb alternatives. Finally, Experiment 3 showed that while whole-phrase alternatives were being rejected more slowly, there was no compelling evidence in favour of the interference effect of \textit{only}. This main effect provides some initial evidence that comprehenders represent phrasal alternatives. Yet overall, the results do not allow us to support the generalisability of the Alternative Representation model to cases of broad focus as the relatedness effect cannot be solely attributed to focus in our design. On the methodological side, this paper provides the contribution of the evidence in favour of using the probe recognition task with constituents larger than a single word yet smaller than a whole sentence. 

\section*{Acknowledgements}
Firstly, we thank Hannah Rohde for her advice in the initial planning stages of this research. We would like to thank Henrik Discher for his help with the creation of the stimuli for Experiment 3. Next, we are very grateful for the feedback E. Matthew Husband provided to us on an earlier draft of this article. We are also thankful to the audiences of AMLaP2021, HSP2022, and DGfS2022 conferences for their valuable comments and suggestions. Finally, we would like to thank the two anonymous reviewers of this chapter and Ingo Reich, one of the editors of this volume, who helped us improve it substantially. The first and third author were  supported  by  the  Deutsche  Forschungsgemeinschaft (DFG,  German Research  Foundation)  as  part  of  the  Emmy Noether  project  awarded  to  Nicole  Gotzner (Grant  Nr.  GO 3378/1-1). 


%\section*{Contributions}
%Radim Lacina contributed to the conceptualisation, methodology, statistical analysis, writing of the original draft, revisions, and editing.
%Patrick Sturt contributed to the conceptualisation, methodology, and editing.
%Nicole Gotzner contributed to the conceptualisation, methodology, statistical analysis, and editing.

{\sloppy\printbibliography[heading=subbibliography,notkeyword=this]}
\end{document}
