\addchap{Preface}

\begin{refsection}
%This preface has no abstract and no authors, but cites \citet{Nordhoff2018} nevertheless.
%content goes here




% \textbf{what this Introduction to the book should contain, IMO:
% - mention recent growing interest in MWE;\\
% - highlight the importance of lexicons for the linguistic description of MWEs and for their use in applications;\\
% - aim of the book;\\
% - overview of the chapters
% }


%Research  on  
Multiword Expressions  (MWEs) have received growing attention    
from scholars in various disciplines, from theoretical to applied linguistics and psycholinguistics and from lexicography for human users to Human Language Technology. In this respect, linguists seek to account for their properties and to define typologies thereof; in applied linguistics, MWEs of various kinds pose issues for language learning and teaching; issues relative to the acquisition, and processing of MWEs, as well as the way they are stored in the mental lexicon constitute the focus of attention in psycholinguistic research, whereas lexicographers are well aware of the importance of their presence in dictionaries \citep{Evert} and strive to define optimal representation formats tailored to meet the needs of humans and machines alike. Computational linguists on the other hand are concerned with MWE processing, primarily with their identification and discovery in corpora, as well as with their cross-lingual equivalence, even though MWEs might be of importance in other downstream tasks too. Given the inherent idiosyncrasies of MWEs, all these tasks are considered problematic. 

%From a purely computational perspective, MWE discovery and identification is still problematic due to the inherent idiosyncrasies of MWEs; and despite the ever-increasing effort to develop corpora of considerable size as well as tools and language models of all kinds, there is still room for improvement.

%MWE processing, defined as either MWE discovery or MWE identification \citep{constant-etal-2017-survey}, is still problematic due to their idiosyncratic nature. 
MWE identification and discovery are seen as the two facets of MWE processing \citep{constant-etal-2017-survey} and lexical resources of all sorts remain at the heart of both: the former could be made easier given a resource lexicon containing them, while the latter could contribute to the enhancement of such a resource \citep{RamischHab}. Consequently, \citet{savary-etal-2019-without} proposed the deployment of MWE-related lexical resources as a possible solution for improving MWE processing; therefore, despite the ever-increasing effort to develop corpora of considerable size as well as language models of all kinds, MWE lexica are still needed.


% Our starting point was previous research undertaken within the PARSEME COST action (ref) on the state-of-the-art their representation of MWEs in lexical resources (ref). 
% The MWE Workshop series organised and endorsed by SIGLEX-MWE




% \section{Multiword Expression processing}
% MWE expression identification and discovery

%\section{MWEs in Language Resources: open issues}




An important open issue in the literature dedicated to this topic is the representation of MWEs in lexical resources. The time when mere lists of MWEs were considered lexicons has passed, and rich descriptions of MWEs are being created or enriched, with special attention paid to their idiosyncrasies at various linguistic levels (lexical, morphological, syntactic, and semantic). 

This volume contains chapters that paint the current landscape of MWE representations in lexical resources from the perspectives of their robust identification and computational processing. Both large-size general lexica and smaller MWE-centred ones are included, with special focus on the representation decisions and mechanisms that facilitate their usage in NLP tasks. The presentations go beyond the morpho-syntactic description of MWEs, into their semantics. These chapters confirm that no common technical solution to the problem of MWE lexical representation exists, as already pointed out in the literature \citep{lichte-etal}.

One challenge in representing MWEs in lexical resources is ensuring that the variability along with extra features required by the different types of MWEs can be captured efficiently. In this respect, recommendations for representing MWEs in mono- and multilingual computational lexicons have been proposed; these focus mainly on the syntactic and semantic properties of support verbs and noun compounds and their proper encoding \citep{calzolari_etal_2002,copestake_etal_2002}.

The interest in developing MWE lexicons results either in those that are MWE-dedicated (see the chapters authored by Skoumalová et al., Markantonatou et al. and Leseva et al.) or in those that are MWE-aware (see Osenova and Simov's contribution and Giouli et al.'s one). Though most of the time the focus is on a language's MWE system, there is also concern for language varieties (see Markantonatou et al.).

All chapters are circumscribed by the NLP domain, with the exception of Tiedemann et al.'s work in which language learning and teaching is the field of interest. The NLP-oriented chapters are concerned with facilitating the processing of texts containing MWEs, while the latter aims at improving learners' fluency by promoting a better understanding of MWE's degree of compositionality and properly handling this approach in teaching materials. However, compositionality, as a key characteristic of MWEs, is a challenge not only for machines, but also for human users, be they language learners, who are the target of Tiedemann et al.'s experiments, or native speakers, as reported in the chapter authored by Schulte im Walde.

There are languages for which language resources have been created over a long period and it is high time they were interconnected to better exploit their potential synergy. Osenova and Simov use the catena representation to this end, while Chiarcos et al. present a solution for standardized formatting of resources, namely the Linked (Open) Data paradigm, which can also help overcome resource scarcity of languages by complementing linguistic information in one resource with information from one or more other resources.

%\textbf{Most lexical resources dedicated to MWEs give an account of their lexical, morphological, and syntactic idiosyncrasies (Shudo et al., 2011; Zaninello and Nissim, 2010; Odijk, 2013). One influential paradigm is that proposed by the Dutch Electronic Lexicon of Multiword Expressions (DuELME), which is a computational lexical resource that describes the syntactic and lexical idiosyncrasies of MWEs (Gregoire, 2010).}

A resource such as WordNet \citep{miller,Fellbaum:98} has the advantage of encoding the meaning of MWEs in a relational manner: on the one hand, they participate in a synonymy relation at the level of synsets (MWEs may be part of a synset alongside either simple words or other MWEs); on the other hand, such synsets are themselves interlinked with other synsets by means of semantic relations. However, a set of one or more specific relations for linking MWEs to meanings of the component words, as proposed by \citet{Osherson2010}, has not been defined yet. On the other hand, the existence of aligned wordnets\footnote{The word \textit{wordnet} is used to refer to a ``lexical knowledge base for a given language, modeled after the principles of Princeton WordNet'' (see \url{http://www.dblab.upatras.gr/balkanet/journal/20_BalkaNetGlossary.pdf}). The form \textit{Wordnet} is used for a particular such resource, e.g., the Bulgarian Wordnet or the Romanian Wordnet; the form \textit{WordNet} is used only for the trademarked Princeton WordNet (see \url{https://wordnet.princeton.edu/}).} for tens of languages offers easy access to MWEs in other languages and can serve as material for multi- and cross-lingual studies, as illustrated by Leseva et al.'s chapter.

Being concerned with the mapping of meaning to form via the theory of Frame Semantics \citep{fillmore_1976, fillmore_1977, fillmore_1982}, the FrameNet lexical database \citep{baker_etal_1998} seeks to account for the semantics of lexical units by assigning them to semantic frames whereas the valences or combinatorial possibilities of each item are revealed from semantically and syntactically annotated sentences from which reliable information can be obtained. In this volume, Giouli et al. make use of FrameNet mechanisms for representing the semantics of MWEs in the light of their valences and the lexicon-corpus interface.

The development of MWE lexicons is intended both for automatic exploitation in NLP and for human usage. With respect to the former, the mere computational format of these resources shows that developers are aware of the need for automatic language processing, while a concern for standardization is proof of the language engineers' need to access such linguistic knowledge. However, tools for manual retrieval of MWEs from lexicons and even from corpora have been created and one of them is presented by Odijk et al. in this volume. 

Hana   Skoumalová,   Marie  Kopřivová,  Vladimír   Petkevič,    Tomáš   Jelínek, 
Ale-xandr Rosen, Pavel Vondřička, and Milena Hnátková 
present LEMUR, a MWE lexicon for Czech. The paper is an attempt to innovatively capture MWEs in Czech so that they can be annotated and searched for in large corpora, thus allowing the user to make effective use of them. Detailed properties concerning both the MWE as a whole and its components are included; for example, for MWEs, the types of idiomaticity (morphological, syntactic, semantic and statistical) are distinguished. At the same time, the entries are designed in such a way that the considerable variability of MWEs in the corpus texts (fragments, varied word order, syntactic modification, etc.) can be captured as well as possible, i.e. to include as many uses of variable MWEs as possible in the search. The MWEs annotated in the corpus are also linked to the corresponding entries in the database, where detailed searchable properties of the MWEs are available to the user, including their meaning, traditional linguistic categorization, typical examples, etc. Linking the corpus to the database allows the user to work with the current language and, for example, to determine the frequency of occurrence of individual MWEs in the corpus. Linking this database further with other lexicographic resources is a natural next step.

Stella   Markantonatou,  Nikolaos T. Kokkas,  Panagiotis  G. Krimpas,   Ana O. 
Chi-ril, Dimitrios Karamatskos, Nikolaos Valeontis, and George Pavlidis present the challenges involved in collecting and representing MWEs for non-standardized language varieties, the focus being on Pomak, an endangered, non-standardized language variety of the East South Slavic dialect continuum. The chapter describes an openly available, online dataset of Pomak verbal MWEs, which were collected via fieldwork. The resource was developed with IDION, a web-based environment for the documentation of a wide range of syntactic, semantic, and stylistic properties of the expressions. Translations and usage examples of the Pomak expressions are provided along with a syntactic analysis in the Universal Dependencies framework. In the collected data both light verb constructions and idioms have been observed. %A considerable number of the collected expressions have a literal equivalent in Modern Greek which is one of the languages in close contact with Pomak (the other being Bulgarian and Turkish).

%The chapter ``A Uniform Multilingual Approach to the Description of Multiword Expressions'', by 
Svetlozara Leseva, Verginica Barbu Mititelu, Ivelina Stoyanova, and Mihaela Cristescu describe an empirically devised framework for the creation of 
%a 
linked bilingual 
%(Bulgarian  and  Romanian)  
computational  
%lexicon  
lexicons of MWEs. The framework is applied to a bilingual (Bulgarian and Romanian) lexicon of verbal   MWEs, which aims at providing a comprehensive description of their features in each of the languages under study. The MWEs, derived from the Bulgarian and the Romanian Wordnet, represent counterparts or translation equivalents of each other; while they are described according to the common principles and features adopted, the data in each language constitute a self-contained monolingual lexicon which may be developed independently. The description of each monolingual lexicon entry includes technical details necessary for cross-lingual linking and a rich linguistic description, on multiple levels. 
%: technical details allowing the cross-lingual linking; lemma and grammatical features; syntactic description including the MWE’s internal structure and morphosyntactic features and restrictions, its valence frame, relevant word order restrictions and types of intervening external elements; semantic description including a gloss, semantic relations, stylistic information, etc.; derivational information in terms of links to MWE derivatives. 
The work illustrates the applicability of a uniform description of MWEs to two languages from different families in a way that accounts for linguistic similarities and specificities. The resource can be enhanced to cover other levels and features of linguistic description, as well as expanded towards other languages.

%Similarly to dependency representation, t
Petya Osenova and Kiril Simov model MWEs in the framework of integrated lexical resources that would facilitate various NLP tasks. They use the notion of catena, an alternative to representing the structure of MWEs in lexicons, %The chapter ``Representation of Multiword Expressions in the Bulgarian Integrated Lexicon for Language Technology'' focuses on the modelling of  
%These resources are as follows: an inflectional dictionary for Bulgarian, a valency dictionary for Bulgarian and a wordnet for Bulgarian. In all of them either the MWEs were not properly represented, or they were represented, but in different ways. 
for the unified encoding of the grammatical, lexical and semantic information. This kind of approach is tree-oriented, thus providing better possibilities for handling idiosyncrasies in comparison to the static methods. The tree representations follow the ideology of Universal Dependencies. MWE lexical entries have a layered structure, with a complexity modelled with respect to two important features of MWEs: discontinuity and fixedness.
%The contributions of the paper include: introducing the layered structure of the MWE lexical entry; tuning the catena-based formalization to the complex structure of integrated linguistic information; modelling the complexity of the entry with respect to the MWE classification features discontinuity and fixedness. 
%Six types of MWEs are discussed: a) fixed, continuous, b) fixed, discontinuous, c) semi-fixed, continuous, d) semi-fixed, discontinuous, e) flexible, continuous and f) flexible, discontinuous. 
%The outlined work is a bottom-top effort that would gradually cover specific lemmas, meanings and cases. A unified representation of MWE types in a number of lexical resources would facilitate various NLP tasks, such as parsing, lexical induction, and machine translation.

One challenge while encoding MWEs for Natural Language Understanding applications is the representation of their semantics. 
%The chapter `A FrameNet approach to providing deep semantics for Multiword Expressions' by Giouli et al. 
Voula Giouli, Vera Pilitsidou, and Hephestion Christopoulos present a frame-based lexical resource for Modern Greek and the encoding of nominal and verbal MWEs in it. To better account for the deep semantics of these complex predicates, their argument structure (or valency) is identified and their lexical-semantic description is provided by means of assigning them to a frame and identifying their Frame Elements. Lexicon development is based on corpus evidence and the annotation performed. The authors discuss the difficulties encountered due to the nature of these complex predicates. They also discuss on the basis of discrepancies observed between single- and multiword lexical units assumed under the same frame in terms of Frame Elements assignment and syntactic realization. 
 
%The chapter `Multiword expressions, Collocations and the OntoLex Vocabulary' by Chiarcos et al. 
Christian Chiarcos, Maxim Ionov, Elena-Simona Apostol, Katerina Gkirtzou, Besim Kabashi, Anas Fahad Khan, and Ciprian-Octavian Truică set out the challenges of modeling MWEs within linked data lexicons and demonstrate how OntoLex-Lemon, a de facto community standard for modelling and publishing lexical resources on the Semantic Web, can effectively address them. Their chapter can serve as a guide for users grappling with the complexities of MWE data modeling in linked data lexicons. The reader is presented diverse strategies for modeling MWEs via the different modules of OntoLex-Lemon, both individually and in combination. The aim is to match specific modeling strategies with particular use cases. This chapter not only presents recommendations, but also furnishes practical examples drawing from real-world use cases, % that illustrate how to model multiword expressions using OntoLex. To summarise the article's contents in more detail: we (1) look at different kinds of multiword expressions; (2) give detailed descriptions of relevant modules of OntoLex-Lemon— OntoLex-Decomp, OntoLex-Morph, OntoLex-FrAC—in terms of their support for modeling multiword expressions; (3) identify and describe different use cases and modeling preferences relevant to different kinds language resources and applications and show that the OntoLex modules are sufficiently expressive to address their modelling needs. Finally, (4) 
at the same time featuring a comparative analysis of OntoLex and other pre-RDF vocabularies, exploring the advantages and disadvantages of the former for existing tools and potential downstream applications in modeling MWEs.

%The chapter `MWE-finder for Dutch' by Odijk et al. 
Jan Odijk, Martin Kroon, Sheean Spoel, Ben Bonfil, and Tijmen Baarda present MWE-Finder, an application that enables a user to search for MWEs in large Dutch text corpora. To cope with the discontinuity of MWE components, with their word order variation, 
%Components of many MWEs in Dutch can occur in multiple forms, need not be adjacent, and can occur in multiple orders (such MWEs are called ‘flexible’). Searching for such flexible MWEs is difficult and cannot be done reliably with most search applications. What is needed is a 
the search engine takes into account the MWE grammatical configuration. 
%MWE-Finder is embedded in GrETEL, a treebank search application for Dutch and is intended as a tool for linguistic and lexicographic research into MWEs. 
%A user can enter an example of a MWE, after which the system searches for sentences in which the MWE occurs. The example must be in a specific canonical form. The MWE can also be selected from a list of approximately 10k canonical forms for Dutch MWEs that MWE Finder offers. The nature of this canonical form for MWEs is described and justified.  
Searches are made possible by using a canonical     form, which is an implicit hypothesis on the properties of the MWE with regard to form variation, modification, and determination. 
To this end, the DUtch CAnonicalised Multiword Expressions lexical resource (DUCAME) is used. %Given a canonical form, MWE Finder fully automatically generates three queries that generalize from the example MWE. The chapter describes in detail how the queries for a MWE are derived from the canonical form. The three queries are called the MWE query, the near-miss query, and the major lemma query. These queries are increasingly less specific, so that the results of the near-miss query form a superset of the results of the MWE query, and the results of the major lemma query form a superset of the near-miss query. The user has the option to leave out the results of a more specific query, which makes it possible to efficiently determine unexpected forms, modifiers, determiners, and grammatical structure. 
The chapter presents an overview of DUCAME, demonstrates the user interface, describes the redesign of the back-end needed for dealing with large text corpora, and illustrates the application for a specific MWE example showing how unexpected form variations, modifications, and determinations, as well as a variant of the MWE are found.

The development of computational models of compositionality typically goes hand in hand with the creation of reliable lexical resources as gold standards for formative intrinsic evaluation. Even though datasets of noun compounds with ratings on compositionality across languages have been developed for many languages, work that looks into whether and how much both the gold standards and the prediction models vary according to the properties of the targets within the lexical resources is still scarce. 
%The chapter `Creating and Investigating Feature-based Compositionality Ratings for Noun Compounds' by 
In her chapter, Sabine Schulte im Walde suggests a novel route to assess the interactions of compound and constituent properties concerning the degrees of compositionality of the compounds  while focusing on English and German noun compounds. %Our contributions are two-fold: (1) a 
A novel collection of compositionality ratings for 
%1,099 
German noun compounds is proposed, where human judges were asked to provide compound and constituent properties 
%(such as paraphrases, meaning contributions, hypernymy relations, and concreteness)
before judging the compositionality.
Also, a series of analyses on rating distributions and interactions with compound and constituent properties for the novel collection, as well as existing gold standard resources in English and German are made and discussed. The author recommends assessing computational models not only on the full dataset, but also on subsets of targets with coherent task-relevant properties.

Fluency in a (new) language comes from mastering the vocabulary and semantics, the rules for inflecting and combining words in phrases and sentences, the pragmatic factors, the cultural knowledge, but, to the same extent, from knowledge about the word combination possibilities \citep{RamischHab}. 
Therese Lindström Tiedemann, David Alfter, Yousuf Ali Mohammed, Daniela Piipponen, Beatrice Silén, and Elena Volodina present part of a new resource, the Swedish L2 profile. %that is based on their prior work on MWEs in L2 Swedish. % in their chapter entitled `Multiword expressions in Swedish as a second language: taxonomy, annotation, and Initial Results'. 
%Their earlier studies are summarised and extended with new qualitative and quantitative analyses. The Swedish L2 profile 
It provides access to MWEs which can be filtered according to type and the level in the Common European Framework of Reference (CEFR) and includes receptive and productive statistics of usage in corpora, as well as links to the empirical data upon which the resource has been built. This makes the resource useful for research, teaching and technical developments. The experiments presented in the chapter show that the receptive difficulty of MWEs is evaluated similarly by experts and non-experts, while their level of compositionality or transparency influence their ranking on the CEFR scale.
%Lindström Tiedemann et al. show that MWEs were well annotated automatically according to the knowledge-based method that they used. Manual categorisation by type also worked well, but annotation according to compositionality proved problematic. Still, compositionality or transparency is shown to be a possible reason that MWEs were seen as more or less difficult. Experts and non-experts agreed well in their relative rankings of MWEs in a crowdsourcing experiment, and the easiest items showed particularly high agreement and were also usually from the easiest levels in coursebooks, which relates well to the fact that lower CEFR levels are more clearly linked to specific topics and themes, whereas higher levels have a focus on various types of specialisation which is likely to make the lexis more varied. The authors also show that receptive levels based on coursebooks can be problematic when expressions are present in few books at one level. Furthermore, there is a clear increase in MWE lemmas at the group level and MWE lemma percentages become very similar in productive and receptive data at C1-level. Further studies are needed in relation to what this means for individual learner texts which are often fairly short and unlikely to contain many MWEs.

After more than two decades since MWEs were initially discussed in the literature of Natural Language Processing (NLP), there are still open issues of all sorts, starting with the very definition of a MWE, as readers will also notice in the chapters of this volume. It was beyond our scope to have a common understanding of this concept, as all phenomena covered are related to a certain extent and it is relevant to see how their descriptions can be leveraged with mutual benefits.

%\noindent This is a sample preface with author(s). 
%\citet{Nordhoff2018} is useful for compiling bibliographies.

{\sloppy\printbibliography[heading=subbibliography]}
\end{refsection}

