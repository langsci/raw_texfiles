\subsection{Coordinate Clauses}
\label{sec:coordinateclauses}
Fully independent verbal clauses may be conjoined with the conjunctions \textit{daw, asta, daw dili,} or \textit{o}. In \sectref{sec:daw} through \sectref{sec:o} we describe and illustrate each of these conjunctions with multi-clause examples from the corpus. In \sectref{sec:culminativeuse} we illustrate the \textit{culminative} usage of irrealis modality in clause coordination. 

Omission of arguments in discourse is common when the referents of the omitted elements can easily be recovered from the context. In particular, in coordinate clauses, there do not seem to be any strictly syntactic constraints on omission of arguments, or on which arguments must be coreferential. There is a tendency for coordinate clauses to share absolutive arguments, but this is not a rigid requirement, as can be seen in the following examples. In the examples in the following four sections, we present the conjunction and any overt coreferential arguments in bold.

\subsubsection{\textit{Daw} ‘and’}
\label{sec:daw}
As discussed in \sectref{sec:finiteadverbialclauses}, \textit{daw} may introduce adverbial time clauses. As such we have glossed it as ‘when’. It also functions as a general conjunction that stands between units of equal syntactic rank, e.g., two nouns, two referring phrases, two adverbs, or two fully inflected clauses. In this usage, it may be glossed as ‘and’.  If two clauses coordinated with \textit{daw} share an  argument, the second reference may or may not be omitted. In example \REF{ex:hefellover}, the absolutive, \textit{kanen an}, is not omitted in the second clause, though it would be fully grammatical if the second reference to the absolutive were omitted:
\ea
\label{ex:hefellover}
Dayon kon \textbf{kanen} \textbf{i} salamat ta Ginuo \textbf{daw} natumba \textbf{kanen} \textbf{an} … \\
\vspace{4pt}
\gll Dayon kon \textbf{kanen} \textbf{i} salamat ta Ginuo \textbf{daw} na-tumba \textbf{kanen} \textbf{an} … \\
right.away \textsc{hsy} 3s\textsc{abs} \textsc{def.n} thank \textsc{nabs} Lord and \textsc{a.hap.r}-fall.over 3s\textsc{abs} \textsc{def.m} \\
\glt ‘Right away he thanked the Lord and he fell over …’ [CBWN-C-21 4.8] 
\z

Example \REF{ex:drinkvitamins} omits mention of the coreferential absolutive in the second clause:

\ea
\label{ex:drinkvitamins}
Daw bagnes en isya na nanay kinang\c{l}an \textbf{kanen} magkaan ta gulay \textbf{daw} mag-inem ta bitamina … \\
\vspace{4pt} \gll Daw bagnes en isya na nanay kinang\c{l}an \textbf{kanen} mag-kaan ta gulay \textbf{daw} mag-{}-inem ta bitamina … \\
when pregnant \textsc{cm} one \textsc{lk} mother should 3s\textsc{abs} \textsc{i.ir}-eat \textsc{nabs} vegetables and \textsc{i.ir}-drink \textsc{nabs} vitamins \\
\glt ‘When a mother is pregnant she should eat vegetables and take (lit. drink) vitamins …’ [LBOP-C-03 11.3]
\z

If there are two or more distinct participants in two coordinate clauses, all are normally retained. The actor is usually not omitted unless the clause is in the peak of a narrative. Coreferential undergoers are more likely to be omitted in the second clause when they are absolutives. In example \REF{ex:onekilo}, the absolutive (\textit{sidda} 'fish') is omitted in two clauses because it is set up as the undergoer in the previous clause:

\ea
\label{ex:onekilo}
Yan p\c{l}a \textbf{sidda} \textbf{na} \textbf{sikad} \textbf{bakod}. Pada\c{l}a danen ta baybay \textbf{daw} baligya singko isya kilo. \\
\vspace{4pt} \gll Yan p\c{l}a \textbf{sidda} \textbf{na} \textbf{sikad} \textbf{bakod}. Pa-da\c{l}a danen ta baybay \textbf{daw} ...-baligya singko isya kilo. \\
\textsc{d2abs} surprise fish \textsc{lk} very big \textsc{t.r}-carry 3p{erg} \textsc{nabs} beach and \textsc{t.r}-sell five one kilogram \\
\glt ‘That surprise was a fish that was very big. They took (it) to the beach and sold (it) for five (pesos for) one kilo.’ [DBOE-C-05 1.5-6]
\z

Undergoers are less-often omitted when they are non-absolutive. This is because non-absolutive undergoers tend to be non-topical in the discourse, and therefore not as easily recovered as participants expressed in the absolutive. Example \REF{ex:attheschool} illustrates a retained non-absolutive undergoer, \textit{ti} `\textsc{d1nabs}', referring back to  \textit{ba\c{l}on} `packed lunch':

\ea
\label{ex:attheschool}
Gada\c{l}a \textbf{a} nang \textbf{ba\c{l}on ko} \textbf{daw} naan \textbf{a} nang gakaan \textbf{ti} iskwilahan i. \\
\vspace{4pt} \gll Ga-da\c{l}a \textbf{a} nang \textbf{ba\c{l}on ko} \textbf{daw} naan \textbf{a} nang ga-kaan \textbf{ti} iskwila-an i. \\
\textsc{i.r}-carry 1s\textsc{abs} only/just packed.lunch 1s\textsc{gen} and \textsc{spat.def} 1s\textsc{abs} only/just \textsc{i.r}-eat \textsc{d1nabs} study-\textsc{nr} \textsc{def.n}\\
\glt ‘I carried my packed lunch and I just ate (it) at the school.’ [DBON-C-07 2.1]
\z

Example \REF{ex:andateit} illustrates two omitted absolutive undergoers, and one omitted non-absolutive undergoer. The Actor, \textit{nay} `1p\textsc{incl.erg}', is also omitted in the second and third clauses:

\ea
\label{ex:andateit}
Gani	dayon		\textbf{nay} anien 			\textbf{daw} lutuon		\textbf{daw}	magkaan. \\
\vspace{4pt}
\gll Gani	dayon		\textbf{nay} ani-en			\textbf{daw} luto-en	\textbf{daw}	mag-kaan\\
So 	right.away	1p\textsc{incl.erg} harvest-\textsc{t.ir} 	and cook-\textsc{t.ir} 		 and	\textsc{i.ir}-eat \\
\glt `So, right away we harvest (previously mentioned coconut, corn, sorghum) and cook (it) and eat (it).' [SFOE-T-06 5.10]
\z

In example \REF{ex:shewateredit} the absolutive undergoer, \textit{niog} `coconut (palm), is omitted in the second clause:

\ea
\label{ex:shewateredit}
Gani, papalangga \textbf{din} man \textbf{yi} \textbf{na} \textbf{niog} \textbf{daw} adlaw-adlaw \textbf{din} pabunyagan. \\
\vspace{4pt} \gll Gani, pa-palangga \textbf{din} man \textbf{yi} \textbf{na} \textbf{niog} \textbf{daw} adlaw-adlaw \textbf{din} pa-bunyag-an. \\
so \textsc{t.r}-have.affection 3s\textsc{erg} also \textsc{d1abs} \textsc{lk} coconut.palm and \textsc{red}-day 3s\textsc{erg}   \textsc{t.r}-irrigate-\textsc{apl} \\
\glt ‘So, she had affection for this coconut tree and every day she watered (it).’ [REFERENCE?]
\z

Example \REF{ex:topofthetree} illustrates two intransitive clauses coordinated with \textit{daw}:
 
\ea
\label{ex:topofthetree}
Gakatay en \textbf{kuti} \textbf{i} \textbf{daw} galayog man \textbf{manok} \textbf{i} naan punta ta ugbos ta kaoy. \\
\vspace{4pt} \gll Ga-katay en \textbf{kuti} \textbf{i} \textbf{daw} ga-layog man \textbf{manok} \textbf{i} naan punta ta ugbos ta kaoy. \\
\textsc{i.r}-climb \textsc{cm} cat \textsc{def.n} and \textsc{i.r}-fly also chicken \textsc{def.i} \textsc{spat.def} going.to \textsc{nabs} top \textsc{gen} tree \\
\glt ‘The cat climbed and also the chicken flew going to the very top of the tree.’ [CBWN-C-18 7.11]
\z

Examples \REF{ex:alsotheyhelped} and \REF{ex:wearingasuit} each illustrate a grammatically transitive and an intransitive clause in a coordinate construction with different actors and different absolutives. 

\ea
\label{ex:alsotheyhelped}
Dayon \textbf{ko} pilak mga kaoy an \textbf{daw} gatabang man \textbf{danen} \textbf{an}. \\
\vspace{4pt} \gll Dayon \textbf{ko} ...-pilak mga kaoy an \textbf{daw} ga-tabang man \textbf{danen} \textbf{an}. \\
right.away 1s\textsc{erg} \textsc{t.r}-throw.away \textsc{pl} wood \textsc{def.m} and \textsc{i.r}-help also 3p\textsc{abs} \textsc{def.m} \\
\glt `Right away I threw away the wood and also they helped.’ [CBWN-C-11 4.27]
\z

\ea
\label{ex:wearingasuit}
Pag-u\c{l}og din ya, tag-iya ya ta b\c{l}angay, dayon \textbf{kanen} ya tugpa \textbf{daw} peseb \textbf{din} bata ya na Pedro na nu\c{l}og na gaamirikana kanen an. \\
\gll Pag-{}-u\c{l}og din ya, tag-iya ya ta b\c{l}angay, dayon \textbf{kanen} ya ...-tugpa \textbf{daw} pa-eseb \textbf{din} bata ya na Pedro na na-u\c{l}og na ga-amirikana kanen an. \\
\textsc{nr.act}-fall 3s\textsc{erg} \textsc{def.f} owner \textsc{def.f}  \textsc{gen} two.masted.boat right.away 3s\textsc{abs} \textsc{def.f} \textsc{i.r}-jump and \textsc{t.r}-dive.to.get 3s\textsc{erg} child \textsc{def.f} \textsc{lk} Pedro \textsc{lk} \textsc{a.hap.r}-fall \textsc{lk} \textsc{i.r}-suit 3s\textsc{abs} \textsc{def.m} \\
\glt ‘When he fell, as for the owner of the two-masted boat, right away he jumped (into the sea) and dove underwater (to get) the child Pedro who had fallen (and) was wearing a suit.’ [PCON-C-01 3.11]
\z

Example \REF{ex:gotsomecookedrice} illustrates three clauses conjoined with \textit{daw}: 

\ea
\label{ex:gotsomecookedrice}
Tapos kay kaan, listo \textbf{kay} eman lisinsya na manaw \textbf{daw} nanay ta barkada ko gakamang ta tinapaan na sidda na deen \textbf{nay} muli naan ta Sintro \textbf{daw} bai na duma nay, gakamang man ta kan-en … \\
\vspace{4pt} \gll Tapos kay ...-kaan, listo \textbf{kay} eman ...-lisinsya na m-panaw \textbf{daw} nanay ta barkada ko ga-kamang ta t<in>apa-an na sidda na da\c{l}a-en \textbf{nay} m-uli naan ta Sintro \textbf{daw} bai na duma nay, ga-kamang man ta kan-en … \\
after 1p\textsc{exc.abs} \textsc{i.r}-eat promptly  1p\textsc{exc.abs} again.as.before \textsc{i.r}-ask.permission \textsc{lk} \textsc{i.v.ir}-leave/walk and mother \textsc{gen} friend 1s\textsc{gen} \textsc{i.r}-get \textsc{nabs} <\textsc{nr.res}>smoke-\textsc{apl} \textsc{lk} fish \textsc{lk} carry-\textsc{t.ir} 2p\textsc{excl.erg} \textsc{i.v.r}-go.home \textsc{spat.def} \textsc{nabs} Central  and woman \textsc{lk} companion 2p\textsc{excl.gen} \textsc{i.r}-get also \textsc{nabs} cooked.rice \\
\glt ‘After we ate, promptly we again as before requested permission to leave  and the mother of my friend got some smoked fish which we would take to Central and as for the woman our companion, she got some cooked rice …’ [CBWN-C-11 3.3]
\z

\subsubsection{\textit{Asta} ‘until’}
\label{sec:asta}
\textit{Asta} (a Spanish word that means ‘until’) usually means ‘until a certain time, place or event.’  As a conjunction between clauses, it can have a resultative sense--the situation expressed in clause B is a result of the situation expressed in clause A:

\ea
Pelles en angin an. Darko ba\c{l}ed \textbf{asta} en mga layag ni ubos en ta gisi. \\
\vspace{4pt} \gll Pelles en angin an. Darko ba\c{l}ed \textbf{asta} en mga layag ni ubos en ta gisi. \\
strong.wind \textsc{cm} wind/air \textsc{def.m} big.\textsc{pl} wave until \textsc{cm} \textsc{pl} sail \textsc{d1abs} all \textsc{cm} \textsc{nabs} tear \\
\glt ‘The wind was strong. The waves were big and (as a result) as for the sails, these were all torn.’ [PCON-C-01 2.16]
\z
 
\ea
Pabatangan no ta tellek saging i a \textbf{asta} nang en na tama tellek lawa ko i. \\
\vspace{4pt} \gll Pa-batang-an no ta tellek saging i a \textbf{asta} nang en na tama tellek lawa ko i. \\
\textsc{t.r}-put-\textsc{apl} 2s\textsc{erg} \textsc{nabs} thorn banana \textsc{def.n} \textsc{inj} until only/just \textsc{cm} \textsc{lk} many thorn body 1s\textsc{gen} \textsc{def.n} \\
\glt ‘You put thorns on (the trunk of) the banana plant and (the result is) my body has lots of thorns.’ [CBWN-C-16 9.4]
\z

\subsubsection{\textit{daw dili} `if not/and not'}
\label{sec:dawdili}
\textit{Daw dili} as a fixed expression usually means ‘if not’, ‘and not’, or ‘but rather’ depending on the context. However, sometimes it simply coordinates independent clauses, with a conjunction or disjunction sense. We consider example \REF{ex:usetwomastedboats} to illustrate clause coordination even though the verb (\textit{mabyai} ‘travel’) is omitted in the second clause due to coreferentiality:

\ea
\label{ex:usetwomastedboats}
Pagamit nang pa \textbf{ta} \textbf{mga} \textbf{ittaw} unti, daw mabyai naan Iloilo \textbf{daw} \textbf{dili} gani naan ta minland Palawan, pagamit \textbf{danen} b\c{l}angay. \\
\vspace{4pt} \gll Pa-gamit nang pa \textbf{ta} \textbf{mga} \textbf{ittaw} unti, daw mabyai naan Iloilo \textbf{daw} \textbf{dili} gani naan ta minland Palawan, pa-gamit \textbf{danen} b\c{l}angay. \\
\textsc{t.r}-use only/just still \textsc{nabs} \textsc{pl} person here and travel \textsc{spat.def} Iloilo and \textsc{neg} truly \textsc{spat.def} \textsc{nabs} mainland Palawan \textsc{t.r}-use 3p\textsc{erg} two.masted.boat \\
\glt ‘People here still use, when traveling to Iloilo if not (or) to mainland Palawan, they use two-masted boats.’ 
\z

\ea
… kalabanan ta mga mamy daw dady ubos may ubra ta upisina ta darko na mga kumpanya paryo abi manigir ka \textbf{daw} \textbf{dili} gaubra ka ta Municipyo. \\
\vspace{4pt} \gll … kalabanan ta mga mamy daw dady ubos may ubra ta upisina ta darko na mga kumpanya paryo abi manigir ka \textbf{daw} \textbf{dili} ga-ubra ka ta Municipyo. \\
{} most \textsc{nabs} \textsc{pl} mom and dad all \textsc{ext.in} work \textsc{nabs} office \textsc{gen} big \textsc{lk} \textsc{pl} company like for.example manager 2s\textsc{abs} and \textsc{neg} \textsc{i.r}-work 2s\textsc{abs} \textsc{nabs} town.hall \\
\glt ‘… most of the moms and dads all have work in offices of big companies like for example you are a manager if not (or) you work in the town hall.’ (RZWE-J-01 15.3)
\z

\subsubsection{o – disjunction ‘or’}
\label{sec:o}
The Kagayanen word \textit{o} is from Spanish ‘or’. It usually expresses alternatives between two conjuncts of equal syntactic status. As a conjunction between clauses, it sometimes presents a reiteration or paraphrase of an idea, as in \REF{ex:tookcareofthem}:
 
\ea
\label{ex:tookcareofthem}
Ta pugya na timpo \textbf{kabaw} \textbf{i} \textbf{daw} \textbf{baka} isya nang ta istaran \textbf{o} isya nang ta tag-iya na gasagod \textbf{ki} \textbf{danen}. \\
\vspace{4pt} \gll Ta pugya na timpo \textbf{kabaw} \textbf{i} \textbf{daw} \textbf{baka} isya nang ta istar-an \textbf{o} isya nang ta tag-iya na ga-sagod \textbf{ki} \textbf{danen}. \\
\textsc{nabs} long \textsc{lk} time water.buffalo \textsc{def.n} an cow one only/just \textsc{nabs} live-\textsc{nr} or one only/just \textsc{nabs} owner \textsc{lk} \textsc{i.r}-care.for \textsc{obl.p} 3p \\
\glt ‘A long time ago the water buffalo and the cow had only one place to live or had only one owner who took care of them.’ [CBWN-C-25 2.1]
\z

\subsubsection{Culminative use of irrealis modality in clause coordination}
\label{sec:culminativeuse}
In \chapref{chap:verbstructure}, \sectref{sec:intransitiveirrealis} we briefly discussed what we describe as the \textit{culminative} use of irrealis modality in narrative chains of events. In this section we will provide additional examples and discussion. Recall that irrealis modality is one of the inflectional values in Kagayanen. Verbs in irrealis modality can be considered fully inflected, and therefore fully finite. However, when a clause terminates a narrative chain of events, it may appear in irrealis modality, even though semantically--according to the content of the narrative-the event is presented as an accomplished fact. In this section we will present the irrealis marked verbs in bold.

Example \REF{ex:weslepttherealso} is from a long narrative in which the narrator and company climb a very high mountain. Example \REF{ex:weslepttherealso} describes what they did when they finally arrived at a house on the mountain. The use of irrealis modality in the last two verbs highlights the fact that this is the culmination of an arduous journey.

\ea
\label{ex:weslepttherealso}
Naan kay dya anay gadayon, \textbf{magpuay} kay daw naan kay man \textbf{magtunuga}. \\
\vspace{4pt} \gll Naan kay dya anay ga-dayon, \textbf{mag-puay} kay daw naan kay man \textbf{mag-tunuga}. \\
\textsc{spat.def} 1p\textsc{excl.abs} \textsc{d4loc} for.awhile \textsc{i.r}-stay \textsc{i.ir}-rest 1p\textsc{excl.abs} and \textsc{spat.def} 1p\textsc{excl.abs} also \textsc{i.ir}-sleep \\
\glt ‘There we stayed for a while, we rested and we slept there also.’ [PCON-C-01 6.3]
\z

Examples \REF{ex:key:2} and \REF{ex:key:3} are both from a long sad narrative. Example \REF{ex:key:2} is the final sentence of a long episode in which a father tricked his child to go with him to the mountains where the father killed the child and buried her. The paragraph following this one is about the mother of the child and what she did when she found their child missing. 

\ea
Pagtapos din tampek ta lungag ya, dayon kanen uli daw \textbf{magdapa-dapa} isab. \\
\vspace{4pt} \gll Pag-tapos din tampek ta lungag ya, dayon kanen ...-uli daw \textbf{mag-dapa-dapa} isab. \\
\textsc{nr.act}-finish 3s\textsc{gen} pack.soil \textsc{nabs} hole \textsc{def.f} right.away 3s\textsc{abs} \textsc{i.r}-return.home and \textsc{i.ir}-lie.flat again\\
\glt ‘After he packed (soil) in the hole, right away he went home and kept on lying flat again. [PBWN-C-12 7.1]
\z

Example \REF{ex:tothetown} describes what happened when the police arrested the man for murder. Again, this excerpt terminates an episode, though it is not the end of the story: 

\ea
\label{ex:tothetown}
Pag-abot danen naan ta kapitto na bukid, dayon din man tudlo daw indi din dapit palebbenga. Pagkita danen ta lebbengan ya, gamandar dayon mga pulis an ki kanen na kutkuton din iya na bata. Pagkita danen, dayon danen pati na kanen matuod nakapatay ta bugtong danen na bata. U\c{l}a en maimo iya na sawa daw dili sigi nang en aga\c{l} daw sigi nang man en na pababawi. Gani, patampekan isab danen daw \textbf{muli} \textbf{naan ta banwa}. \\
\vspace{4pt} \gll Pag-abot danen naan ta ka-pitto na bukid, dayon din man tudlo daw indi din dapit
pa-lebbeng-a. Pag-kita danen ta lebbeng-an ya, ga-mandar dayon mga pulis an ki kanen na
kutkot-en din iya na bata. Pag-kita danen, dayon danen pati na kanen matuod naka-patay ta
bugtong danen na bata. U\c{l}a en ma-imo iya na sawa daw dili sigi nang en aga\c{l} daw sigi nang man en na pa-ba~bawi. Gani, pa-tampek-an isab danen daw \textbf{m-uli} \textbf{naan}
\textbf{ta} \textbf{banwa}. \\
\textsc{nr.act}-arrive 3p\textsc{abs} \textsc{spat.def} \textsc{nabs} \textsc{ord}-seven \textsc{lk} mountain right.away 3s\textsc{erg} also point and direction 3s\textsc{erg} direction
dig-\textsc{t.ir} 3s\textsc{erg} 3s\textsc{gen} \textsc{lk} child \textsc{nr.act}-see 3p\textsc{gen} right.away 3p\textsc{erg}  believe \textsc{lk} 3p\textsc{abs} truly \textsc{i.hap.r}-die
\textsc{t.r}-bury-\textsc{xc} \textsc{nr.act}-see 3p\textsc{gen} \textsc{nabs} bury-\textsc{nr} \textsc{def.f} \textsc{i.r}-command immediately \textsc{pl} police \textsc{def.m} \textsc{obl.p} 3s \textsc{lk}
\textsc{nabs} only 3p\textsc{gen} \textsc{lk} child \textsc{neg} \textsc{cm} \textsc{i.r}-do 3s\textsc{gen} \textsc{lk} spouse and \textsc{neg} continue only/just \textsc{cm} \textsc{i.ir}-cry and continue only/just also
\textsc{cm} \textsc{lk} \textsc{t.r}-\textsc{red}~revive then \textsc{t.r}-pack.soil-\textsc{apl} again 3p\textsc{erg} and \textsc{i.v.ir}-return.home \textsc{spat.def} \textsc{nabs} town \\
\glt ‘When they (the police with the father who killed his child) arrived on the seventh mountain, he (the father) pointed out in what direction he buried (the child). When they saw the grave, the police immediately commanded him that he dig up his child. When they saw (the child), right away they believed that he truly had killed their one and only child. There was nothing his spouse could do, except to keep on crying and keep on being revived. So, they packed (soil on it) again and \textbf{went home to the town}.’ [PBWN-C-12 12.2]  
\z

Example \REF{ex:andthenbathed} is from a different narrative in which a water buffalo and a cow have a discussion of how they will go to the river and swim without their owner knowing it. Example \REF{ex:andthenbathed} describes what they do when they arrive at the river. The next paragraph gives more details of what they did when they were swimming. 

\ea
\label{ex:andthenbathed}
Pag-abot danen naan ta suba, palubbas danen iran na bayo daw pabatang danen naan ta kilid ta suba daw \textbf{maglangoy}.\\
\vspace{4pt} \gll Pag-{}-abot danen naan ta suba, pa-lubbas danen iran na bayo daw pa-batang danen naan ta kilid ta suba daw \textbf{mag-langoy}.\\
\textsc{nr.act}-arrive 3p\textsc{abs} \textsc{spat.def} \textsc{nabs} river \textsc{t.r}-remove 3p\textsc{erg} 3p\textsc{gen} \textsc{lk} clothes and \textsc{t.r}-put 3p\textsc{erg} \textsc{spat.def} \textsc{nabs} bank \textsc{lk} river and \textsc{i.ir}-swim \\
\glt ‘When they arrived at the river, they took off their clothes and put (them) on the bank of the river and then bathed.’ [CBWN-C-25 5.4] 
\z

There is also a tendency for irrealis modality to be used at points of high tension or episodic climax of a narrative. Since such points tend to be characterized by multiple closely linked events, this usage often overlaps with the culminative usage. A future discourse study is needed to elucidate the factors that contribute to the use of irrealis modality to express realis events at certain points in narrative discourse. Example \REF{ex:andkilledhim} illustrates irrealis modality in the second conjunct of two conjoined clauses at a point of high tension in a story:

\ea
\label{ex:andkilledhim}
Lugar na gatago kanen i nakita din iya na magulang na galebbeng naan ta b\c{l}awan daw padakep ta mga ittaw magulang ya daw \textbf{gapuson} daw \textbf{patayen}. \\
\vspace{4pt} \gll lugar na ga-tago kanen i na-kita din iya na magulang an ga-lebbeng naan ta  b\c{l}awan daw pa-dakep ta mga ittaw magulang ya daw \textbf{gapos-en} daw \textbf{patay-en}. \\
then \textsc{lk} \textsc{i.r}-hide 3s\textsc{abs} \textsc{def.n} \textsc{a.hap.r}-see 3s\textsc{erg} 3s\textsc{gen} \textsc{lk} older.sibling \textsc{def.m} \textsc{i.r}-bury \textsc{spat.def} \textsc{nabs} gold and \textsc{t.r}-capture \textsc{nabs} \textsc{pl} person older.sibling \textsc{def.f} and tie.up-\textsc{t.ir} and die-\textsc{t.ir} \\
\glt ‘Then when he was hiding, he saw his older sibling (brother) being buried in gold and the people captured the older sibling and \textbf{tied (him) up} and \textbf{killed (him)}.’ [CBWN-C-22 12.4]
\z
