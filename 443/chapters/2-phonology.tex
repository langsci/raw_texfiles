\chapter{Phonology}\label{chap:phonology}
\section{Introduction}\label{introduction-2}
In Kagayanen there are twenty \isi{phonemes} including sixteen \isi{consonants} and four \isi{vowels}. In this chapter, we justify the phonemic status of each phoneme with minimal or near minimal pairs (\sectref{bkm:Ref360478146} and \sectref{bkm:Ref360478150}). Then we describe the syllable structures (\sectref{bkm:Ref360478178}) and stress patterns (\sectref{bkm:Ref360478190}). In \sectref{bkm:Ref360478227} we describe and exemplify five major phonological and morphophonemic rules. Finally, in \sectref{bkm:Ref360478279} we discuss several characteristics of relaxed speech, and in \sectref{bkm:Ref360478305} we describe orthographic conventions we have adopted in this grammar.
\section{Consonants}
\label{bkm:Ref360478146}\label{sec:consonants}
\is{consonants|(}

The consonant phonemes of Kagayanen are classified as seven plosives\is{consonants!plosives}, three nasals\is{consonants!nasals}, one fricative\is{consonants!fricative}, a flap\is{consonants!flap} and four approximants\is{consonants!approximants}, as illustrated in \tabref{tab2.1}.

\begin{table}
\caption{Consonants of Kagayanen}
\label{tab2.1}
\begin{tabularx}{\textwidth}{Qccccc}
\lsptoprule 
& \textbf{Bilabial} & \textbf{Dental/Alveolar}& \textbf{Palatal} & \textbf{Velar} & \textbf{Glottal} \\
\midrule
Plosives (vl) & p & t &  & k & ʔ \\
Plosives (vd) & b & d &  & ɡ & \\
Nasals & m & n &  & ŋ & \\
Flap &  & ɾ &  &  & \\
Fricative &  & s &  & & \\
Lateral approximant &  & l &  & & \\
Other approximants & w & ð̞ & j &  & \\
\lspbottomrule
\end{tabularx}
\end{table}

There is no phonemic contrast between dental and alveolar consonants. However, consonants in the dental/alveolar column vary between these two places of articulation in the following way: /t/, /d/, /n/, /s/, and /1/ tend to be alveolar, /ɾ/ is always alveolar, and /ð̞/ is clearly (inter)dental.

 Example \REF{ex:sneeze} illustrates the contrasts between phonetically similar plosives, and between the glottal stop and zero. All plosives, both voiceless and voiced, are unaspirated and in final position are unreleased.
\ea
\label{ex:sneeze}
\begin{tabbing}
\hspace{1.2cm} \= \kill
/p/ - /b/ \> [paʔán] ‘bait’, [baʔán] ‘sneeze’ \\
\> [lápaw] ‘to rise above/go over’, [labáw] ‘surpass/superior’ \\
\> [sipɪ́t] ‘to clip/clamp’, [sibɪ́t] ‘safety pin’ \\
\> [kɤ̞́pkɤ̞p] ‘to cling on’, [kɤ̞́bkɤ̞b] ‘immature coconut’ \\
/t/ - /d/ \> [taʔán] ‘maybe’, [daʔán] ‘old’ \\
\>   [putɪ́ʔ] ‘white’, [pudɪ́ɾ] ‘right/authority’ \\
\>   [ɡatʊ́s] ‘one hundred’, [badʊ́j] ‘unattractive appearance’ \\
\>   [pátjʊ] ‘cemetery’, [mádjʊʔ] ‘far’ \\
\>   [ð̞ɤ̞́ttɤ̞m] ‘to be hungry’, [lɤ̞́ddaŋ] ‘to sink’ \\
\>   [ʔɤ̞́ɡɤ̞t] ‘to be angry/annoyed’, [ʔɤ̞́ɡɤ̞d] ‘to work hard’ \\
/k/ - /ɡ/ \> [kaʔán] ‘to eat’, [ɡaʔán] ‘light weight’ \\
\>   [líkɪd] ‘to roll up’, [líɡɪd] ‘to roll around’ \\
\>   [tʊ́ktʊk] ‘to knock’, [tʊ́ɡtʊɡ] ‘music’ \\
/k/ - /ʔ/ \> [káɡaw] ‘germ’, [ʔáɡaw] ‘grab/snatch’ \\
\>   [sakáj] ‘to ride’, [saʔáɡ] ‘floor’ \\
\>   [bátak] ‘to pull upwards’, [bátaʔ] ‘child’ \\
/ɡ/ - /ʔ/ \> [ɡɤ̞́ð̞ɤ̞t] ‘to cut’, [ʔɤ̞́ð̞ɤ̞t] ‘divider between rooms’ \\
\>   [saɡád] ‘skillful’, [saʔád] ‘vow/promise’ \\
\>   [báʊɡ] ‘to feed animals’, [báʊʔ] ‘odor/bad smell’ \\
/ʔ/-/${\emptyset}$/ \> [tɤ̞́ʔɤ̞b]‘steam therapy for sinus or colds’, [tɤ̞́ɤ̞d] ‘conscientious’ \\
\>    [kánʔɤ̞n] ‘cooked rice’, [kánɤ̞n] ‘she/he/it’ \\
\>   [báɡaʔ] ‘lungs’, [báɡa] ‘embers’
\end{tabbing}
\z

 There is only one phonemic fricative\is{consonants!fricative}, /s/. \citet[21--22]{harmon1977} lists two fricatives /s/ and /h/. However, there are good reasons for not including /h/ as a phoneme. We do agree with Harmon that there is an [h] sound in exaggerated slow speech between contiguous vowels in words such as [báʊʔ]/[báhʊʔ] ‘odor/bad smell’ or [dáɪk]/[dáhɪk] ‘to crawl on stomach’. However, it is not heard in relaxed speech. In slow exaggerated speech the [h] is mostly heard in words borrowed from other languages, such as \isi{Hiligaynon} and \isi{Kinaray-a}, in which an [h] sound appears in intervocalic position. If speakers are asked to repeat one of these words, they will say it slowly breaking it up into syllables with the [h] in intervocalic position and beginning a new syllable. \citet[23]{harmon1977} proposes that [h] was present in Proto-Manobo and Proto-Kagayanen but is being lost in Kagayanen.

\is{consonants!obstruent distribution|(}The plosives and fricative occur syllable initially, syllable finally, and as geminates\is{consonants!geminate}. In syllable onsets, clusters consisting of an obstruent and a flap or an approximant occur. The phoneme /d/ has limited distribution, usually not occurring in intervocalic position in native words. (See exceptions above in \REF{ex:sneeze}.) In loan words, such as [káda] ‘each’, [ʔidád] ‘age’ and [kúdak] ‘picture’, /d/ may occur intervocalically. The /ʔ/ is exceptional in that it does not occur as a geminate and also does not occur in syllable-onset \isi{consonant clusters} (\tabref{tab2.2}).

\begin{table}[b]
\caption{Obstruent distribution}
\label{tab2.2} 
\begin{tabularx}{\textwidth}{lQQQQ}
\lsptoprule 
 & Syllable initial & Syllable final & Consonant cluster & Geminate \\
\midrule
 p & pá.naw \newline ‘to~walk/go’
 \newline 
 sá.pɤ̞k~\newline ‘thick, gooey consistency’ &
sɤ̞́p.sɤ̞p~\newline ‘to~sip/ suck’
\newline
sa.ð̞áp~\newline ‘to catch with hands’
\newline 
sá.ð̞ap \newline ‘drag net’
\newline sá.ð̞ɤ̞p \newline ‘sunset or moonset’ &
pð̞á.paʔ \newline ‘palm frond’
\newline 
prú.tas \newline ‘fruit’ &
táp.paw \newline ‘superficial’
\newline 
dʊ́p.pa \newline ‘arm~span’ \\
\tablevspace
 t & tɤ̞́l.lɤ̞k \newline ‘thorn’
\newline 
ʔí.taʔ \newline ‘soft’ &
kʊ́t.kʊt \newline ‘to dig a hole’
\newline
ʔí.bɪt \newline ‘to hold’ &
tð̞ú.nʊn \newline ‘wild pig’
\newline
tjá.ɾʊk \newline ‘sailboat mast’ &
ʔɪ́t.taw \newline ‘person’
\newline
lát.tɪk~\newline ‘to~crack’ \\
\tablevspace
 k & ku.jʊ́ɡ \newline ‘to accompany’
\newline
la.kás \newline ‘to return home from far’ &
lák.baj \newline ‘to skip over’
\newline 
ð̞ɤ̞́ɡ.sɤ̞k \newline ‘mashed’ &
kð̞á.maj \newline ‘sugar’
\newline
sʊ́k.klɪp \newline ‘to partially insert’
\newline
kwa.ɾɪ́s.ma \newline ‘dry season’ &
bák.kɤ̞n \newline ‘diaper’
\newline
ták.kɪʔ \newline ‘tokay gecko’ \\
\midrule
\end{tabularx}
\end{table}

\begin{table}[p]
\caption*{Table 2.2: Obstruent distribution (cont.)}
\begin{tabularx}{\textwidth}{lQQQQ}
\midrule
 & Syllable Initial & Syllable Final & Consonant cluster & Geminate \\
\midrule
{}ʔ & ʔa.kɤ̞́ʔ \newline ‘my/mine’
\newline
la.ʔʊ́ɡ \newline ‘to wander’ &
bá.baʔ \newline ‘mouth/short (things)’
\newline
bá.ɡaʔ \newline ‘lungs’ & & \\
\tablevspace
b & ba.ð̞ɤ̞́d \newline ‘wave of water’
\newline
tú.bʊʔ \newline ‘to grow’ &
táb.ʔaŋ \newline ‘tasteless/no salt’
\newline
ʔɤ̞́.sɤ̞b \newline ‘to dive/submerge’ &
bð̞á.ŋaj \newline ‘two-masted boat’
\newline
bwin \newline ‘to reduce s.t.’ &
ʔʊ́b.baɡ \newline ‘swollen’
\newline
lɤ̞́b.baɡ \newline ‘to swell/expand’ \\
\tablevspace
d & da.jád \newline ‘good’
\newline
dʊ́ð̞.dʊð̞ \newline ‘kapok tree’ &
sʊ́d.laj \newline ‘comb’
\newline
ʔɤ̞́.ŋɤ̞d \newline ‘intentionally’ &
dð̞á.ɡa \newline ‘single girl’
\newline
djá.nʊʔ  \newline ‘straight’ &
lɤ̞́d.daŋ \newline ‘to sink’
\newline
sɪ́d.daʔ \newline ‘fish’ \\
\tablevspace
ɡ & ɡɤ̞́d.dan \newline ‘stairs’
\newline
ð̞á.ɡʊk \newline ‘to snore’ &
sʊ́ɡ.ba \newline ‘to grill’
\newline
ʔi.sɤ̞́ɡ \newline ‘brave’ &
ɡjá.nap \newline ‘to crawl’
\newline
ɡwa \newline ‘outside’ &
nɤ̞́ɡ.ɡaʔ \newline ‘to lie down’
\newline
láɡ.ɡʊŋ \newline ‘low pitch sound’ \\
\tablevspace
s & sɪd.ʔá.nan \newline ‘fish, meat, or vegetable dish’
\newline
lí.sɤ̞n \newline ‘round’ &
más.ki \newline ‘even though’
\newline
lɤ̞́m.mɤ̞s \newline ‘to drown’ &
swá.ð̞ɤ̞m \newline ‘chickenpox’
\newline
sð̞áʔɪt \newline ‘spearpoint’ &
pɤ̞́s.saʔ \newline ‘to crush’
\newline
lás.sʊŋ \newline ‘mortar used with pestle’ \\
\lspbottomrule
\end{tabularx}
\end{table}
\is{consonants!obstruent distribution|)}

\newpage
There is a tendency for consonants excluding /ɾ/, /ʔ/ and the non-lateral approximants, /w/, /ð̞/, and /j/, to geminate when preceded by /ɤ̞/ and followed by /ɤ̞/ or sometimes /a/. \citet[30]{harmon1977} asserts that there is a diachronic development that caused a ``non-glide" consonant preceded by [ɤ̞] and followed by any vowel or a liquid consonant\is{consonants!liquids} to become geminate. However, according to Harmon, this development is inconsistent, and has failed to become a regular synchronic rule in all environments. We find that there are no words in the data with geminate consonants preceded by [ɤ̞] and followed by a liquid consonant or the high vowels. There are a few words with geminate plosive consonants that have vowels other than [ɤ̞] preceding and a liquid following. Even words with non-geminate consonant preceded by [ɤ̞] and followed by a liquid consonant or high vowel were not found except for one exception -- [ʔɤ̞́nlɤ̞k] ’to withdraw out of view’, though for some speakers this word is pronounced as [ʔɪ́nlɤ̞k]. These patterns are illustrated in \tabref{tab2.3}.

\is{consonants!gemination|(}
\begin{table}
\caption{Consonant gemination after /ɤ/}
\label{tab2.3}
\begin{tabularx}{\textwidth}{QQ}
\lsptoprule 
Geminate & Not Geminate \\
\midrule
{}kɤ̞́ppɤ̞s ‘to shrink’
\newline
{}sɤ̞̀ppaʔɤ̞́n ‘a cold/mucus’ & 
{}dɤ̞́pɤ̞t ‘diligent’ \\
\tablevspace
{}ɡɤ̞́ttɤ̞k ‘abdomen’
\newline
{}lɤ̞́ttab ‘wild plants’&
{}ɡɤ̞tɤ̞́k ‘tight’ \\
\tablevspace
{}bɤ̞́kkɤ̞ɡ ‘bone’
\newline
{}tɤ̞́kkad ‘for feet to touch something' & 
{}tɤ̞kɤ̞́b ‘for a predator to bite or attack’ \\
\tablevspace
{}lɤ̞́bbɤ̞ŋ ‘to bury’
\newline
lɤ̞́bbag  ‘to swell/expand’ &
{}bð̞ɤ̞́bɤ̞d ‘to wrap around’ \\
\tablevspace
{}pɤ̞́ddɤ̞s ‘stinging feeling’
\newline
{}ɡɤ̞́ddan ‘stairs/ladder’ &
{}kɤ̞́dɤ̞t ‘to be shriveled up’ \\
\tablevspace
{}tɤ̞́ɡɡɤ̞b‘to be overloaded’
\newline
{}bɤ̞́ɡɡas ‘milled rice’ & 
ʔɤ̞́ɡɤ̞d ‘to work hard’
\newline
{}pɤ̞́ɡaʔ ‘to squeeze/wring out’ \\
\tablevspace
{}lɤ̞́mmɤ̞s‘to drown’ & 
{}kɤ̞́mɤ̞s ‘to squeeze in the hand’
\newline
{}ʔɤ̞́man ‘again as before’ \\
\tablevspace
{}ʔɤ̞́nnɤ̞b‘to be soaked, saturated, spread throughout ’
\newline
{}ʔɤ̞́nnas‘low tide’ &
tɤ̞nɤ̞́ŋ ‘to stop/cease’
\newline
{}tɤ̞́naʔan ‘reason’
\newline
{}ʔɤ̞́nlɤ̞k ’to withdraw out of view’ \\
\tablevspace
{}tɤ̞́ŋŋɤ̞d ‘beside/next to’
\newline
{}tɤ̞́ŋŋa ‘middle’ & 
tɤ̞ŋɤ̞́d ‘because/about’
\newline
{}tɤ̞ŋáʔ ‘half/to divide’ \\
\tablevspace
{}dɤ̞́ssɤ̞n ‘hard’
\newline
{}lɤ̞́ŋŋɤssà ‘blood’ & 
dɤ̞́sɤ̞k ‘to compress the contents’ \\
\tablevspace
{}sɤ̞́llɤ̞ɡ ‘current/fast flow’
\newline
{}tɤ̞́llan ‘to be constipated’ &
sɤ̞lɤ̞́d ‘within a time span’ \\
\tablevspace
{}pɤ̞́ð̞ð̞ɤ̞ɡ\footnote{There are only one or two words with a [ɤ̞ð̞ð̞] sequence, so this may be treated as an exception rather than a regular pattern. Other than [pɤ̞́ð̞ð̞ɤ̞ɡ] ‘to threaten’, the only other word we have encountered that includes this sequence is [ʔɤ̞́ð̞ð̞aŋ] ‘barrier’, but this word is also pronounced [ʔɤ̞́llaŋ] by some speakers.} ‘to threaten’ &
pɤ̞ð̞ɤ̞́ɡ ‘to be constantly moving' \\
\tablevspace
& [bɤ̞́ðag] ‘to separate’ \\
\lspbottomrule
\end{tabularx}
\end{table}

This pattern does not work across morpheme boundaries as in [ɡɤ̞́ð̞ɤ̞t] ‘to cut’+ [-ɤ̞n] \rightarrow [ɡɤ̞ð̞ɤ̞́tɤ̞n] ‘will cut’, not *[ɡɤ̞ð̞ɤ̞́ttɤ̞n].
\is{consonants!gemination|)}

The three nasals\is{consonants!nasals|(}
 are also contrastive, as illustrated below:

\ea
\begin{tabbing}
\hspace{1.3cm} \= \kill
/m/ - /n/ \> [máma] ‘male’, [nánaʔ] ‘pus’ \\
\>     [ʔámaj] ‘father’, [ʔánaj] ‘termite’ \\
\>     [ʔítɤ̞m] ‘rain clouds’, [kítɤ̞n] 1\textsc{p.inc.abs} \\
/m/ - /ŋ/ \> [mɤ̞́tmɤ̞t] ‘grudge’, [ŋɤ̞tŋɤ̞́t] ‘throbbing pain’ \\
\>     [samáj] ‘striped pattern’, [saŋáj] ‘having same name’ \\
\>   [kɤ̞́mkɤ̞m] ‘to form a fist’, [kɤ̞́ŋkɤ̞ŋ] ‘to hold in the arms’ \\
/n/ - /ŋ/  \>  [nálɪʔ] ‘abruptly’, [ŋáli] ‘to dig up root crops’ \\
\>     [ʔánaj] ‘termite’, [ʔáŋaj] ‘fitting/becoming’ \\
\>     [tában] ‘to run away with’, [tábaŋ] ‘to help’
\end{tabbing}
\z

Similar to obstruents, the nasals occur syllable initially, syllable finally, as geminates, and as the first consonant in syllable-onset clusters where the second consonant is the interdental approximant /ð̞/ (\tabref{tab2.4}). The nasal /ŋ/ is an exception in that it does not occur in a consonant cluster except as a geminate consonant or across syllable boundaries.

\is{consonants!nasal distribution|(}
\begin{table}
\caption{Nasal distribution}
\label{tab2.4}
\begin{tabularx}{\textwidth}{lQQQQ}
\lsptoprule
& Syllable initial & Syllable final & Consonant cluster & Geminate \\
\midrule
m & má.kɤ̞t \newline ‘to start a fire’
\newline
má.ma \newline ‘male’ &
sám.pɤ̞t \newline ‘to arrive briefly’
\newline
ʔá.jam \newline ‘dog’ &
mð̞áŋ.kaw \newline ‘long’
\newline
mð̞ú.tʊʔ \newline ‘provisions for a trip’ &
da.ɡám.mʊ \newline ‘dream’
\newline
lʊ́m.maj \newline ‘love potion’ \\
\tablevspace
n & ní.ʲɪʔ \newline ‘to gather shellfish’
\newline
lá.na \newline ‘coconut oil’ &
sán.dad \newline ‘to stub toe’
\newline
ʔá.pɪn \newline ‘protective cloth’ &
nð̞á.ʔʊ thirst’
\newline
njá.ʔan \newline ‘definite spatial marker \newline ‘ (archaic and regional) &
bán.naw \newline ‘to resurrect’
\newline
bɪ́n.ni \newline ‘seed grain’ \\
\tablevspace
ŋ & ŋí.taʔ \newline ‘to look for something’
\newline
ʔa.ŋád \newline ‘to look up at’ &
sáŋ.ɡɪʔ \newline ‘a leaf, branch or fruit breaks off’
\newline
sí.lʊŋ  \newline ‘area under and around a house’ &
ka.máŋ.jan \newline ‘incense’
\newline
bʊ́ŋ.jʊd \newline ‘hill’ &
máŋ.ŋʊd \newline ‘younger sibling’
\newline
tɤ̞́ŋ.ŋaʔ \newline ‘middle’ \\
\lspbottomrule
\end{tabularx}
\end{table}
\is{consonants!nasal distribution|)}
\is{consonants!nasals|)}

\is{consonants!liquids|(}
There are two liquids: /l/, a voiced alveolar lateral, and /ɾ/, an alveolar flap that varies with a trill. \citet[16--17]{harmon1977} states that Kagayanen has an l-colored lateral that is an allophone of /l/, but we consider this sound to be an interdental approximant phoneme /ð̞/, following \citet{olson2010}. The articulation, distribution and other information on this sound is discussed below.

The examples in \REF{groan} illustrate the contrast among /l/, /ð̞/ and /ɾ/. The interdental approximant has been included in the contrasts and distribution tables since historically it probably was an allophone of /l/ (\citealt{olson2008}:9).
\ea
\label{groan}
\begin{tabbing}
\hspace{1.3cm} \= \hspace{.3cm} \= \kill
/l/ - /ð̞/ \> [láɡɡaj] ‘to tie together w/ rattan’, [ð̞áɡaj] ‘to groan’ \\
\>   [lálaʔ] ‘bad spirit/ghost’, [ð̞áð̞a] ‘to weave’ \\
\>   [kámbal] ‘twins’, [ʔámbað̞] ‘to say’ \\
/l/ - /ɾ/ \> [lʊ́mbʊj] ‘java plum’, [ɾʊ́mbʊ] ‘to move in a certain direction’ \\
\>   [talíŋa] ‘ear’, [taɾɪ́ŋ] ‘to lose the way’, \\
\>   [málaʔ] to change clothes’, [máɾa] ‘dry’ \\
\>   [kʊ́ndal] ‘silk banana’, [ʔándaɾ]’to start an engine’ \\
\end{tabbing}
\newpage
\begin{tabbing}
\hspace{1.3cm} \= \hspace{.3cm} \= \kill
/ð̞/ - /ɾ/ \> [ð̞ábʊ] ‘to capsize’, [ɾábas] ‘to cut down weeds with a machete’ \\
\>   [ʔáð̞ɤ̞ŋ] ‘mole’, [ʔáɾɤ̞m] ‘bruise’ \\
\>   [ʔuð̞ás] ‘sweat’, [ʔóɾas] ‘hour/time’ \\
\>   [sáð̞aʔ] ‘to strain  s.t.’, [saɾá] ‘door/shutter’ \\
\>   [kasáð̞] ‘wedding’, [ʔúsaɾ] ‘to use’ \\
/ð̞/ - /l/ - /ɾ/ \> \> [páð̞a] ‘to erase’, [pála] ‘shovel’, [páɾa] ‘for/purpose’ \\
\>   [ʔisjáð̞] ‘blanket’, [ʔupisjál] ‘official’, [pasjáɾ] ‘to go visiting’
\end{tabbing}
\z

\begin{table}[b]
\caption{Liquids and interdental approximant distribution}
\label{tab2.5}
\begin{tabularx}{\textwidth}{lQQQQQ}
\lsptoprule
& Syllable initial & Syllable final & Consonant cluster & Geminate \\
\midrule
l & lɤ̞́.ɡɤ̞d ‘to rub’
\newline
ɡi.lɤ̞́k ‘angry’ &
pɤ̞́l.pɤ̞l ‘slow to learn’
\newline
ka.wɪ́l ‘fish hook’ &
bli.ŋɪ́ʔ ‘crooked’
\newline
tam.blí.lʊ ‘cowrie shellfish’ &
pɤ̞́l.lɤ̞s  ‘strong wind’
\newline
tál.lʊɡ ‘egg’ \\
\tablevspace
ð̞ & ð̞ɤ̞́ɡ.ð̞ɤ̞ɡ ‘flame’
\newline
ʔá.ð̞ʊʔ ‘shadow’ &
bɤ̞́ð̞.bɤ̞ð̞ ‘body hair/fur/feathers’
\newline
dak.mɤ̞́ð̞ ‘thick’ &
bð̞á.ŋaj ‘two-masted boat’
\newline
sap.ð̞á ‘rough feel’ &
ʔáð̞.ð̞ʊ ‘pestle’
\newline
báð̞.ð̞ɤ̞ŋ ‘to have impaired body and brain functions as if drunk’ \\
\tablevspace
ɾ & ɾá.bas ‘to cut down weeds with a machete’
\newline
má.ɾa ‘dry’ &
dáɾ.kʊ ‘large PL’
\newline
tú.kaɾ ‘to play music’
\newline
tí.nɪɾ ‘to stay somewhere’ &
ɡɾá.pʊn ‘bottle/ jar’ &
[bárriʊ] ‘Barrio’ (a name for Barangay Santa Cruz in Cagayancillo) \\
\lspbottomrule
\end{tabularx}
\end{table}

 The liquids occur syllable initially, syllable finally, as the second consonant in consonant clusters and as geminate consonants, as illustrated in \tabref{tab2.5}. However, in native words /ɾ/ only occurs in intervocalic position and syllable finally. \citet[21]{harmon1977} says that /ɾ/ is the reflex of intervocalic Proto-Austronesian *d or *j. With loan words /ɾ/ occurs syllable initially, syllable finally, and as the second consonant in a consonant cluster within a syllable. Only one word was found that has a geminate /r/; [bárriʲʊ] which is another name commonly used to refer to Santa Cruz, a neighborhood north of the main part of Cagayancillo. “Barrio”, with a trilled [ɾ] sound, is the Spanish word for “neighborhood”, and was until 1975 the general word in the Philippines. However in 1975 the Philippine government passed a law that changed \textit{barrio} to \textit{barangay} as the basic unit of government for the whole country, though some people continued to use \textit{barrio}. On the main island of Cagayancillo “Barrio” uniquely refers to the community of Barangay Santa Cruz.


\is{consonants!liquids|)}

\is{consonants!approximants|(}
There are three voiced non-lateral approximants in Kagayanen: /w/ bilabial, /ð̞/ interdental and /j/ palatal (see \tabref{tab2.1}). \citet[27]{harmon1977} suggests that the interdental approximant is a glide, citing the fact that it varies freely with /j/ in two words; [ʔi.sð̞á]/ [ʔi.sja\'{} ] ‘one’ and [dð̞á.nʊʔ]/[djá.nʊʔ] ‘smooth.’ Another word has been found with this free variation: [dð̞ú.nan]/\linebreak{}[djú.nan] ‘boundary’. However, the data in (\ref{groan-2}) below show phonemic contrasts among /w/, /ð̞/ and /j/.

\ea
\label{groan-2}
\begin{tabbing}
\hspace{1.3cm} \= \kill

/w/ – /ð̞/ \> [wáɡaj] ‘to spill/flow out’, [ð̞áɡaj] ‘to groan’ \\
\> {}[búwan] ‘stove’, [búð̞an] ‘moon’ \\
\> {}[ʔámbaw] ‘mouse’ , [ʔámbað̞] ‘to say’ \\
/w/ – /j/ \> [wáɡak] ‘ragged clothes’, [jáɡak] ‘to cry’ \\
\>{}[láwa] ‘body’, [lája] ‘cast net’ \\
\> {}[ʔáɡaw] ‘to grab/snatch’, [ʔáɡaj] ‘to hang up to dry’ \\
/j/ – /ð̞/ \> [jɤ̞́ɡjɤ̞ɡ] ‘to shake’, [ð̞ɤ̞́ɡð̞ɤ̞ɡ] ‘to flame’ \\
\> {}[dájaɁ] ‘to deceive’, [dáð̞a] ‘to carry’ \\
\> {}[ʔáɡaj] ‘to hang up to dry’, [ʔáɡað̞] ‘to cry’
\end{tabbing}
\z

Even though the interdental approximant /ð̞/ contrasts with the lateral as well as the other approximants in Kagayanen, /ð̞/ has limited distribution in that it does not occur next to /i/ except for one word [káð̞ð̞iŋ] ‘hardened stuff after cooking coconut milk’, and this is only for some speakers.

\is{consonants!interdental approximant|(}
\citet{olson2010} describe the articulation of the interdental approximant as a voiced oral consonant pronounced with unrounded lips and egressive pulmonic air. They also say that there is non-turbulent air flow and that, unlike a lateral, the tongue blade remains relaxed throughout the articulation. For this reason, they do not describe the sound as a \textit{lateral} approximant, even though perceptually it has a ``lateral" sound. 
\citet{olson2008} also say that when the interdental approximant occurs next to /u/ there is little tongue tip protrusion between the teeth, and when it is next to /ɤ̞/ there is some protrusion. Finally, the greatest protrusion is when it is next to /a/. Jacqueline Huggins in personal communication comes to a similar conclusion, saying that there is slight tongue raising when the interdental approximant occurs near a high-central vowel /ɪ/ and mid-central vowel /ɤ̞/ (cf [ʔɤ̞́.ð̞ɤ̞s] `borrow/lend'). Huggins further explains that the interdental approximant has two articulators: the tongue tip and the tongue sides. The sides of the tongue brush across the sides of the upper molars, and the tongue tip protrudes slightly. However, when it is contiguous to the low-central vowel [a] the articulation of the interdental approximant is a little different because the sides of the tongue brush across the sides of the lower molars not touching the upper teeth, and the tongue tip either pokes slightly against the back of the lower lip or slightly protrudes from the mouth. When the interdental approximant occurs next to /u/ the tongue is level and brushes across the upper front teeth, not the molars, and the tongue tip protrudes slightly. 
For the purposes of this chapter, we will not represent these nuances of pronunciation in our transcriptions. Rather we will represent the interdental approximant consistently with the symbol [ð̞]. The orthography currently in use does not distinguish this sound from the lateral [l]. In subsequent chapters of this grammar we will make the distinction by using the grapheme \textit{ļ} to represent the interdental approximant, so as to preserve this important distinction, while minimally deviating from the standard orthography.
\is{consonants!interdental approximant|)}

\is{consonants!glides|(}
There is contrast between /j/ and /i/ and also between /w/ and /u/. The difference between the /j/ and /i/ and between /w/ and /u/ is that the /j/ and /w/ do not occur in syllable nuclei nor can they stand alone in their own syllables while /i/ and /u/ occur in syllable nuclei and can stand alone as a syllable [V].


\ea
/j/ - /i/    [taj] ‘father’, [tá.ɪɁ] ‘to sew’; [naj] ‘we (excl)’, [ná.i] ‘this’ \\
/w/ - /u/  [bat.wáʔ] ‘to appear’, [bà.tu.án] ‘rocky ground’ \\
{}     [kaw] ‘you plural’, [ká.ʊn] ‘box/drawer’ \\
{}     [ka.ʊ́n] ‘you singular \textsc{abs}’
\is{consonants!glides|)}
\z

The non-lateral approximants occur syllable initially, syllable finally, as the second consonant of a consonant cluster and as a geminate. However, as discussed earlier,  the non-lateral approximants do not occur as geminates when preceded by /ɤ̞/ (see \tabref{tab2.3}). The only exceptions are the words [pɤ̞jjɤ̞́t] ‘to stutter’, [pɤ̞́jjɤ̞ŋ] ‘to squint’, and [pɤ̞́ð̞ð̞ɤ̞ɡ] ‘to threaten', and some speakers’ pronunciation of ‘barrier’ as [ʔɤ̞́ð̞ð̞aŋ], though this word is also pronounced [ʔɤ̞́llaŋ] by other speakers.

\begin{table}
\caption{Non-lateral approximant distribution}
\label{tab2.6}
\begin{tabularx}{\textwidth}{lQQQQ}
\lsptoprule
& Syllable initial & Syllable final & Consonant cluster & Geminate \\
\midrule
w & wa.ð̞ɤ̞́ŋ ’face’
\newline
ʔá.wak \newline ‘waist’ &
láw.ʔaj \newline ‘disgusting’
\newline
bð̞á.ŋaw \newline ‘rainbow’ &
ɡwá.pa \newline ‘attractive’
\newline
sáb.waɡ \newline ‘to sow seed by scattering' &
sʊ́w.wað̞ \newline ‘slacks’
\newline
táw.waj \newline ‘inner peace’ \\
\tablevspace
ð̞ & ð̞á.ɡʊ \newline ‘earthworm’
\newline
ʔa.ð̞áb \newline ‘to graze’ &
báð̞.ʔɤ̞s \newline ‘spoiled food’
\newline
ʔá.bɤ̞ð̞ \newline ‘tube-like wrap around skirt’ &
bð̞á.ɡɤ̞n \newline ‘vine’
\newline
sab.ð̞áj \newline ‘to drape over’ &
ʔáð̞.ð̞ʊ \newline ‘pestle’
\newline
wáð̞.ð̞ʊ \newline ‘eight’ \\
\tablevspace
j & já.pʊn \newline ‘supper’
\newline
da.jád \newline ‘good’ &
sáj.lʊ \newline ‘to move somewhere or to transfer something’
\newline
pú.ʷaj \newline ‘to rest’ &
ɡjá.nap \newline ‘to creep along’
\newline
tad.jáw \newline ‘earthenware water jar’ &
ɡáj.jaɁ \newline ‘ashamed/em-barrassed’
\newline
pɤ̞́j.jɤ̞ŋ \newline ‘to squint’ \\
\lspbottomrule
\end{tabularx}
\end{table}
\is{consonants!approximants|)}
\is{consonants|)}

\newpage
\section{Vowels} \is{vowels|(}
\label{bkm:Ref360478150}
\begin{table}[b]
\caption{Phonemic vowels of Kagayanen}
\label{tab2.7}
\begin{tabular}{lccc}
\lsptoprule
 & Front unrounded & Central~unrounded & Back rounded \\
\midrule
Close & i
  &  & u \\
Mid &  & ɤ̞ & \\
Open &  & a & \\
\lspbottomrule
\end{tabular}
\end{table}


There are four phonemic vowels in Kagayanen: /i/, /u/, /a/ and /ɤ̞/, as displayed in \tabref{tab2.7}. The phoneme /i/ has four allophones; [i] close front, [ɪ] near close near-front, [e] close-mid front, and [e̞] mid front. In open syllables in native words [i] occurs and in open syllables in loan words, [i] varies with [e] especially in the first or last syllables of the word. [ri.ɡá.lʊ] or [re.ɡá.lʊ] ‘gift’ and [lí.bri] or [lí.bre] ‘free’.


In closed syllables in native words [ɪ] occurs and in closed syllables in loan words [ɪ] varies with [e̞].

\ea
Native words \\
{}[sí.li] ‘to change/exchange’ \\
{}[bɪ́n.ni] ‘seed grain’ \\
{}[ká.bɪɡ] ‘to hold-back/keep/retain’
\ex
Loan words \\
{}[sɪ́l.jʊ] or [sé̞l.jʊ] ‘seal’ \\
{}[bɪn.dí.sjʊn] or [be̞n.dí.syʊn] ‘to bless’ \\
{}[sɪŋ.kwɪ́n.ta] or [sɪŋ.kwé̞n.ta] ‘fifty’ \\
{}[dɪ̀s.pi.dí.da] or [dè̞s.pi.dí.da] ‘going-away party’ \\
{}[ʔɪ̀s.pi.ɾí.tʊ] or [ʔè̞s.pi.ɾí.tʊ] ‘spirit’ \\
{}[ɾɪk.lá.mʊ] or [ɾe̞k.lámʊ] ‘to complain’ \\
{}[ʔɪs.má.ɡɤ̞l] or [ʔe̞s.má.ɡɤ̞l] ‘flip-flop' \\
{}[níɡ.ɾos] or [né̞ɡ.ɾos] ‘Negros' (the name of an island in east central Philippines) \\
{}[pɪb.ɾɪ́.ɾʊ] or [pe̞b.ɾɪ́.ɾʊ] ‘February’
\z

\citet[24]{harmon1977} suggests that /i/ is realized as [ɪ] or [ɛ] before consonant clusters in unstressed syllables, without giving specific examples. Possibly what she was hearing is what we mention above that in loan words in closed syllables, [ɪ] varies with [e̞], since [e̞] and [ɛ] are very similar in pronunciation.

The phoneme /u/ has two allophones [u] and [ʊ]. The complementary distribution of these allophones is somewhat analogous to that of [i] and [ɪ]. The allophone [u] occurs in open non-final syllables and in stressed open final syllables, while [ʊ] occurs in closed syllables and in unstressed open final syllables.

The phonemes /ɤ̞/ and /a/ do not have allophones. The /ɤ̞/ is realized as [ɤ̞] and /a/ as [a].

\largerpage
The data in example (\ref{cocks}) illustrate contrasts among the four vowel phonemes.
\ea
\label{cocks}
\begin{tabbing}
\hspace{1.3cm} \= \kill
/i/ - /u/ \> [bílaŋ] ‘to count/consider’, [búlaŋ] ‘to fight cocks’ \\
\>{}[báti] ‘to beat in mixing together’, [bátʊ] ‘to stone’ \\
/i/ - /ɤ̞/ \> [Ɂíɡɤ̞d] ‘jealous’, [Ɂɤ̞́ɡɤ̞d] ‘to work hard’ \\
\> {}[taɾɪ́ŋ] ‘to lose the way’, [taɾɤ̞́ŋ] ‘righteous/good character’ \\
/i/ - /a/ \> [ɡíba] ‘to sit on lap’, [ɡábaʔ] ‘consequences for disrespect’ \\
\> {}[dilɪ́Ɂ] \textsc{neg.irr}, [diláʔ] ‘tongue’ \\
/u/ - /ɤ̞/ \> [Ɂumán] ‘to finish’, [Ɂɤ̞mán] ‘again as before’ \\
\> {}[láʊɡ] ‘connected/penetrated’, [láɤ̞ɡ] ‘to joke’ \\
/u/- /a/ \> [Ɂúbay] ‘to guide’, [Ɂábay] ‘side-by-side/bridesmaid/groomsman’ \\
\>{}[libʊ́t] ‘go around/surround’, [libát] ‘cross-eyed’ \\
\> {}[buɁʊ́ŋ]‘for glass to break’, [buɁáŋ] ‘crazy’ \\
/ɤ̞/ - /a/ \> [bɤ̞́taŋ] ‘thing’, [bátaŋ] ‘driftwood’ \\
\> {}[pílɤ̞k] ‘eyelash’, [pílak] ‘silver’
\end{tabbing}
\z

\is{transitional glides|(}
There is a transitional glide when two vowels occur contiguously in a word only in the penultimate and ultimate syllables and the first vowel is stressed. The transitional glide [ʲ] occurs between the stressed vowel /i/ and the second vowel /i/, /a/, or /u/, for example, [níʲɪɁ] ‘to gather shellfish’, [míʲaɡ] ‘to agree/want’, and [ɡíʲʊ] ‘to move/motion.’ When the stress is on the second of two contiguous vowels, there is no transitional glide: contrast [lúʷa] ‘tears’ with [luáɁ] ‘spit out’.

The transitional glide [ʷ] occurs between the stressed vowel /u/ and the second vowel /i/ or /a/, but no words were found with [úʷu], for example, [búʷi] ‘to get free’, [lúʷaɡ] ‘ladle.’ When two vowels occur contiguously in a word and the second vowel is stressed, there is no transitional glide. Some words have the full glides /w/ and /j/ between vowels but these are not transitional because the words are always pronounced with a full glide [líju] *[líu] ‘other side’ and [lúwas] *[lúas] ‘to save'. The transitional glides also occur in words with more than two syllables. When a three-syllable word has stress on the antepenultimate there are no /íʲV/ or /úʷV/ sequences because the stress must be on the penultimate /i/ or /u/. Examples for three syllable words with transitional glides are [taɡ.ʔí.ʲa] ‘owner’, [pa.ɾí.ʲʊ] ‘same’, and [pa.ɾí.ʲas] ‘to be like’. However, three-syllable words with stress on the antepenultimate cannot take the transitional glide [bíb.li.a] *[bíb.li.ʲa] ‘Bible’ and [sá.ɾi.ʊ] *[sá.ɾi.ʲʊ] ‘prayer group’.

With four or more syllables the primary stress occurs on the penultimate and the secondary stress on the second syllable to the left of that. Examples include [kà.pɪt.tú.ʷan] ‘seventy’ and [dù.a.dú.ʷa] ‘to waver/doubt’.

Transitional glides occur across morpheme boundaries when suffixes are\linebreak{}added, as in [síli] ‘to change clothes’ plus [an] \textsc{apl} becomes [silíʲan] and [bátʊ] ‘stone’ plus [ɤ̞n] \textsc{t.ir} becomes [batúʷʊn]. When the enclitic definite markers that begin with vowels [i] ‘near speaker’ and [an] ‘nearby in the same area’ occur after a word that ends in a vowel, a transitional glide also occurs, as in [maní] ‘peanut’ plus [i] or [an] becomes [maníʲi] or [maníʲan] and [ʔáð̞̞ð̞̞ʊ] ‘pestle’ plus [i] or [an] becomes [ʔað̞̞ð̞̞úʷi] or [ʔáð̞̞ð̞̞úʷan].

Kagayanen has words that have cognates in Bisayan languages, such as \isi{Hiligaynon} and \isi{Cebuano}. Where a Bisayan word includes the sequence VhV, the cognate word in Kagayanen has no /h/, but it may emerge in exaggerated speech, for example, [biáɡ] ‘to take captive’, [buát] ‘make’, [búʊk] ‘hair’ and [lúʊd] ‘kneel’ may be pronounced [biháɡ], [buhát], [búhʊk] and [lúhʊd]. Kagayanen also has the contrasting sequence VɁV, glottal stop between vowels, with stress on either syllable: [láɁɪn] ‘bad’ and [laɁɪ́n] ‘different.’

When a two-syllable word has two contiguous vowels with stress on the second vowel, optionally the first vowel /i/ may become the glide /j/ and the first vowel /u/ may become the glide /w/, thus shortening the word to only one syllable.

\ea
i+a [miád] or [mjad] ‘good/kind’ \\
i+u [niʊ́ɡ] or [njʊɡ] ‘coconut’, [ɡiʊ́] or [ɡjʊ] ‘to move/motion’ \\
u+a [buát] or [bwat] ‘to make’, [ɡuáɁ] or [ɡwaɁ] ‘to go out’ \\
u+i [buín] or [bwɪn] ‘to decrease something’
\is{transitional glides|)}
\is{vowel clusters|(}
\z

Vowels occur in the nuclei of syllables in any position in the word. They occur inter-consonantally, in vowel clusters, and word finally. The co-occurrence possibilities of vowels with intervening consonants are illustrated in  \tabref{tab2.8}. All possible vowel-vowel sequences are illustrated in \tabref{tab2.9}.

\begin{table}
\caption{Vowel coocurrences with intervening consonants}
\label{tab2.8}
\begin{tabularx}{\textwidth}{QQQQQ}
\lsptoprule
Before consonant &  \multicolumn{4}{c}{After consonant} \\
\midrule
& a & ɤ̞ & i & u \\
\midrule
a & Ɂábaj ‘side-by-side’ & katɤ̞́ð̞ ‘itchy’ & ɡalís ‘scrape /scratch’ & kabaɁʊ́ŋ ‘coffin’ \\
ɤ̞ & dɤ̞́ɡaɁ ‘juice’ & tɤ̞ɡɤ̞́tɤ̞Ɂ ‘small house lizard’ &  & \\
i & píɁaŋ ‘lame’ & mítɤ̞m ‘black’ & bíbɪɡ ‘lips’ & libʊ́t ‘go around/surround’ \\
u & dúð̞ak ‘animal fight’ &  & pulíŋ ‘speck in eye’ & kʊ́tkʊt ‘to dig’ \\ 
\lspbottomrule
\end{tabularx}
\end{table}

\begin{table}
\caption{Vowel sequences with no intervening consonant}
\label{tab2.9}
\begin{tabularx}{\textwidth}{lQQQQ}
\lsptoprule
First vowel & \multicolumn{4}{c}{Second vowel} 
 \\
\midrule
& a & ɤ̞ & ɪ & ʊ \\
\midrule
a & láas ‘untamed chicken’ & náɤ̞t ‘noun classifier for long narrow shape’ & dáɪk ‘to crawl on stomach’ & láʊd ‘deep sea’ \\
ɤ̞ &  & bɤ̞ɤ̞́ð̞ ‘to be satisfied’ &  & \\
i & miád ‘good/kind’ &  & ɁíʲɪɁ ‘urine’ & sáɾiʊ ‘group prayer and singing’ \\
u & púʷaj ‘to rest’ & & buɪ́Ɂ ‘alive/living/to live’ & dúʊð̞ ‘to lift something for another to carry’ \\
\lspbottomrule
\end{tabularx}
\end{table}
\is{vowels|)}

\section{Syllable structure} \is{syllable structure|(}
\label{bkm:Ref360478178}
A syllable consists of an obligatory vocalic nucleus, an optional onset and an optional coda. A word-initial syllable has an obligatory onset, but elsewhere it is optional. The onset can be either a single consonant or a consonant cluster consisting of either a stop, fricative or nasal as the first consonant and an approximant as the second consonant. If the first consonant of the cluster is a stop it most frequently is either labial or alveolar. Though velar stops may occur in consonant clusters in native words, they more frequently occur in loan words such as [klási] ‘class/kind’, [kláɾʊ] ‘clear’, and [ɡɾípʊ] ‘faucet.’ As detailed above, the velar nasal and glottal stop do not occur in consonant clusters. The second consonant is most commonly /l/, /ð̞/, /w/, or /j/. The phoneme /ɾ/ occurs in consonant clusters only in loan words.

The optional coda of a syllable is a single consonant. Thus the syllable types in Kagayanen are V, VC, CV, CVC, CCV and CCVC. However, V and VC cannot occur word initially. The most frequently occurring syllable structure is CV and the least frequently occurring ones are CCV and CCVC, which mostly occur in word initial position. Example (\ref{born}) lists a few words with each possible syllable type:

\ea
\label{born}
\begin{tabbing}
\hspace{1.3cm} \= \kill
V  \> [bá.i] ‘woman’, [ná.i] ‘this’, [ta.tá.ʊ] ‘doll’, [ta.ú] ‘to be born’ \\
VC  \> [tá.ap] ‘to winnow’, [pá.ɪd] ‘to wipe’, [ná.ɤ̞t] ‘strand’, \\
 \> [ba.ʊ́ɡ] ‘to bend’,
 [lʊ.ás] ‘except for’ \\
CV \>  [ka.ð̞át] ‘rope’, [bí.lɪn] ‘to leave behind’, [sɤ̞́.jɤ̞p] ‘to suck in’, \\
 \> [Ɂɪn.jú] 2\textsc{p.gen} \\
CVC \>  [lám.baj] ‘to pass by’, [sɪ́d.sɪd] ‘hem’, [pɤ̞́d.dɤ̞ŋ] ‘to close eyes’, \\
 \> [Ɂi.ɡʊ́t] ‘to tie’ \\
CCV \>  pð̞á.paʔ ‘palm frond’, [ɡwá.pa] ‘attractive’, [tam.bð̞á.li] ‘wax \\
  \>  gourd’, [pwí.kan] ‘sea turtle’, [bð̞ɤ̞́.bɤ̞d] ‘to wrap around’, \\ 
 \> [dð̞ú.nan]   ‘boundary’, [pɾú.tas] ‘fruit’ \\
CCVC \>  [tð̞áɡ.sa] ‘seldom’, [bð̞át.ʔaŋ] ‘hip’, [pjʊ́ŋ.kʊt] ‘fetal position’, \\ 
\> [blɪs.kád] ‘opposite/upside-down’, [nák.kð̞ɤ̞ʔ] ‘hiccup’ \\
 \> [bð̞ʊ́n.bʊn]   ‘sandbar’
\end{tabbing}
\z

\is{vowel deletion, diachronic process|(}
\citet[11]{harmon1977} suggests that many of the consonant clusters within syllables have come into Kagayanen through vowel deletion or in borrowed words. As an example of vowel deletion, she gives the Kagayanen word \textit{bð̞áŋaw} ‘rainbow’ which reconstructs as the Proto-Philippine word *\textit{balaŋaw}. As an example of borrowing she gives the Kagayanen word \textit{lámbɾɪ} ‘wire’ which comes from the Spanish word \textit{alambre.} \is{vowel deletion, diachronic process|)}
\is{syllable structure|)}

\section{Stress} \is{stress|(}
\label{bkm:Ref360478190}
Stress in Kagayanen is contrastive, and consists of lengthening of a syllable nucleus. Rise in voice pitch and intensity does not always fall on the stressed syllable, as these are high level phenomena involving intonation at the phrase or clause level. Isolated words can be pronounced with very different pitch and intensity from the same words used in communicative contexts. Because of these factors, it has been very difficult to mark words consistently for stress outside of elicitation of individual words. Stress occurs on any of the syllables of two and three syllable words. With four and five syllable words, usually there is primary stress on the penultimate and secondary stress on the second syllable to the left of the penultimate.
\tabref{tab2.10} and \tabref{tab2.11} illustrate contrasting stress patterns in two and three syllable words. Examples (\ref{4syllable}) and (\ref{5syllable}) just illustrate how stress occurs in each type.

\begin{table}
\caption{Stress on two syllable words}
\label{tab2.10}
\begin{tabularx}{\textwidth}{Xl}
\lsptoprule
Stress on first syllable & Stress on second syllable \\
\midrule
{}[Ɂá.naj] ‘termite’ & [Ɂa.náj] ‘first/a while/please’ \\
{}[sá.ð̞̞ap] ’drag net’ & [sa.ð̞áp] ‘to catch in hands’ \\
{}[bú.ð̞aʔ] ‘bubbles’ & [bu.láʔ] ‘to tell a lie’ \\
{}[pɤ̞́l.pɤ̞l] ‘slow to learn’ & [pɤl.pɤ̞́l] ‘to patch a hole or tear’ \\
{}[Ɂí.law] ‘light’ & [Ɂi.láw] ‘uncooked/unripe’ \\
{}[ɡá.lɪŋ] ‘to grind/mill’ & [ɡa.lɪ́ŋ] ‘however’ \\
{}[bú.ð̞ak] ‘flower’ & [bu.ð̞ák] ‘cotton’ \\
{}[bá.kak] ’to tell lies’ & [ba.kák] ‘to be delighted’ \\
{}[lú.aɡ] ‘ladle’ & [lu.áɡ] ‘to look/watch’ \\
{}[ká.ʊn] ‘box/drawer’ & [ka.ʊ́n] ‘you singular \textsc{abs}’ \\
{}[Ɂá.bɤ̞ð̞] ‘tube-like wrap around skirt’ & [Ɂa.bɤ̞́ð̞] ‘dull blade/tool’ \\
\lspbottomrule
\end{tabularx}
\end{table}

\begin{table}
\caption{Stress on three syllable words}
\label{tab2.11}
\begin{tabularx}{\textwidth}{QQQ}
\lsptoprule
Primary stress on first
Syllable & Primary Stress on second syllable & Primary stress on third syllable \\
\midrule
{}[pá.Ɂa.ɡì] ‘means’
\newline
{}[bá.ka.làn] ‘wooden structure’ 
\newline
{}[kí.si.mì] ‘ceiling’
\newline
{}[bí.bi.ɾʊ̀n] ‘baby bottle’
\newline
{}[bɤ̞́k.kɤ̞s.sàn] ‘snake’
\newline
{}[lɤ̞́ŋ.ŋɤ̞s.sà] ‘blood’
\newline
{}[ɡʊ́ɾ.mi.tì] ‘boat’s crew’
\newline
{}[lú.ŋaj.ŋàj] ‘to have no strength’
& 
{}[ki.náŋ.lan] ‘to need’
\newline
{}[bʊm.bð̞á.nan] ‘breastbone or wishbone of fowl’
\newline
{}[ku.lín.tas] ‘necklace’
\newline
{}[si.lɪ́ŋ.sɪŋ] ‘shoot of a plant’
\newline
{}[sa.ð̞ɤ̞́.pan] ‘west’
\newline
{}[ban.dɤ̞́.ɾan] ‘numb’
\newline
{}[kʊs.tʊ́m.bɾi] ‘custom’
\newline
{}[bu.bʊ́ŋ.aj] ‘fringe’
&
{}[kʊ̀l.ti.báɾ] ‘to cultivate’
\newline
{}[bù.ku.Ɂán] ‘woven box’
\newline
{}[pà.ɾa.ní] ‘to come close’
\newline
{}[ɡì.bi.Ɂí] ‘yesterday’
\newline
{}[bà.na.ɡɤ̞́ŋ] ‘organ pipe coral’
\newline
{}[bà.ta.nɤ̞́ɡ] ‘half cooked rice’
\newline
{}[bù.bu.Ɂú] ‘tortoise’
\newline
{}[bà.ð̞a.ɾʊ́l] ‘kite’ \\
\lspbottomrule
\end{tabularx}
\end{table}

\newpage

\ea
\label{4syllable}
Stress on four syllable words: \\
{}[kà.li.bú.tan] ‘earth’ \\
{}[kà.li.mú.taw] ‘pupil of the eye’ \\
{}[kà.li.báŋ.baŋ] ‘butterfly’ \\
{}[kà.tɤ̞ŋ.dá.nan] ‘position/role/duty of authority and responsibility’ \\
{}[pà.si.lá.bɤ̞t] ‘to get involved in something one should not’ \\
{}[bà.ð̞aj-bá.ð̞aj] ‘playhouse’ \\
{}[kà.la.mʊ́n.dɪŋ] ‘calamansi’ (a small citrus fruit) \\
{}[sà.lam.pá.ti] ‘dove’ \\
{}[kà.ɪm.tá.ŋan] ‘situation/condition’ \\
{}[kà.maŋ.ŋú.ɾan] ‘youngest sibling’ \\
{}[kà.ma.tá.jan] ‘death’ \\
{}[kà.lɪl.já.wan] ‘Philippine black-naped Oriole’
\ex
\label{5syllable}
Stress on five syllable words: \\
{}[ka.Ɂù.ɡa.lí.ŋɤ̞n] ‘self’ \\
{}[ka.Ɂù.ɡa.lí.an] ‘tradition’ \\
{}[sa.làm.ba.bá.ɡa] ‘centipede’ \\
{}[ta.ɡì.pu.sú.ʊn] ‘heart’ \\
{}[ka.mà.tu.ú.ɾan] ‘truth’ \\
{}[ka.mà.ɡu.lá.ŋan] ‘oldest sibling’
\z

When a suffix or enclitic is added to a penultimately stressed root, the stress moves to the right one syllable.

\ea
{}   [ɡʊ́n.tɪŋ] ‘scissors’ + [-ɤ̞n] \rightarrow [ɡʊn.tí.ŋɤ̞n] ‘will cut with scissors’ \\
{}   [ɡʊ́n.tɪŋ] + [i] definite marker \rightarrow [ɡʊn.tɪ́ŋ.ᵑi] ‘the scissors’ \\
{}   [ɡʊ́n.tɪŋ] + [ku] \rightarrow [ɡʊn.tɪ́ŋ ku] ‘my scissors’ (two words) \\
{}   [ɡʊ́n.tɪŋ] + [ku] + [i] \rightarrow [ɡʊntɪ́ŋ ku ʷi] ‘my scissors’ (three words)
\z

However, with prefixation the stress is not affected.

\ea
{}[ɡa-] + [ɡʊ́n.tɪŋ] \rightarrow [ɡa.ɡʊ́n.tɪŋ] ‘is cutting with scissors’ \\
{}[maɡ-] + [pa-] + [ɡʊ́n.tɪŋ] \rightarrow [màɡ.pa.ɡʊ́n.tɪŋ] ‘will have one’s (hair) cut’
\is{stress|)}
\z

\section{Major phonological and morphophonemic processes}
\label{bkm:Ref360478227}
\is{phonological processes|(}
\is{morphophonemics|(}
\subsection{Rhoticization}\label{sec:rhoticization}
\is{rhoticization|(}
The voiced alveolar stop /d/ becomes a flap [ɾ] in intervocalic position when a vowel-initial suffix is added to a /d/-final root. This is consistent with the observation in \sectref{bkm:Ref360478146} above that in native words [d] does not occur intervocalically within a root, while [ɾ] usually occurs intervocalically.



\ea
/d/-final roots plus vowel-initial suffix: \\
{}[dajád] ‘good’ + [-an] \rightarrow [dajáɾan] ‘will feel something is good’ \\
{}[sɪ́nɁad] ‘to cook grain’ + [-ɤ̞n] \rightarrow [sɪnɁáɾɤ̞n] ‘will cook grain’ \\
{}[kuɡʊ́d] ‘to grate coconut’ + [-ɤ̞n] \rightarrow [kuɡúɾʊn] ‘will grate’ \\
{}[Ɂípɪd] ‘to fold’ + [-ɤ̞n] \rightarrow [Ɂipíɾɤ̞n]/[píɾɤ̞n] ‘will fold’ \\
{}[súɡɪd] ‘to tell’ + [-an] \rightarrow [suɡíɾan] ‘will tell’
\z

This process does not occur with prefixation, as shown in \REF{ex:better}:

\ea
\label{ex:better}
{}[dajád] ‘good’ + [pa-] \rightarrow [pàdajád] ‘to make something better’ \\
{}[dúma] ‘companion’ + [ɡa-] \rightarrow [ɡadúma] ‘to accompany’
\z

Possibly this process occurs with suffixes because the suffix has no stress and the /d/ occurs after a stress. With prefixes, on the other hand, the stress occurs after the /d/. \is{rhoticization|)}

\subsection{Vowel rounding harmony}\label{vowel-harmony}
\is{vowel harmony|(}
When the suffix [-ɤ̞n] is added to a root that has /u/ in the final syllable, whether there is a final consonant or not, then the [ɤ̞] of the suffix becomes [ʊ]. This is consistent with the earlier observation that the sequence /u/C/ɤ̞/ does not occur in Kagayanen.


\ea
{}[Ɂanʊ́d] ‘to drift’ + [-ɤ̞n] \rightarrow [Ɂanúɾʊn] ‘will drift’ \\
{}[pánnʊɁ] ‘to fill up’+ [-ɤ̞n] \rightarrow [pannúɁʊn] ‘will fill up’ \\
{}[Ɂútʊd] ‘to cut’ + [-ɤ̞n] \rightarrow [Ɂutúɾʊn] ‘will cut’
\is{vowel harmony|)}
\z

\subsection{Consonant omission}\label{consonant-deletion}
\is{consonant omission|(}

When the intransitive, irrealis affix [m-] is prefixed to a root with an initial stop, fricative, or lateral, then the initial consonant of the root is omitted. No words with nasals or /r/ in initial position are found that permit prefixation of [m-]. Rather the intransitive, irrealis forms of such words are formed with the prefix [maɡ-] (see \chapref{chap:verbstructure}, \sectref{sec:intransitiveirrealis}). \citet[29]{harmon1977} hypothesizes that the irrealis [m-] is actually an infix, and since the sequence of a consonant plus a nasal is not permitted in Kagayanen, when [-m-] occurs, the initial consonant is omitted.


\ea
{}[m-] + [pálɪt] ‘to buy’  \rightarrow [málɪt] ‘will buy’ \\
{}[m-] + [tʊ́dlʊɁ] ‘to teach’ \rightarrow [mʊ́dlʊɁ] ‘will teach’ \\
{}[m-] + [tɤ̞́ɡbɤ̞ŋ] ‘to go downhill’ \rightarrow [mɤ̞́ɡbɤ̞ŋ] ‘will go down hill’ \\
{}[m-] + [kámaŋ] ‘to get/take’  \rightarrow [mámaŋ] ‘will get/take’ \\
{}[m-] + [Ɂútʊd] ‘to cut’  \rightarrow [mútʊd] ‘will cut’ \\
{}[m-] + [bálɪk] ‘to return’ \rightarrow [málɪk] ‘will return’ \\
{}[m-] + [dð̞áɡan] ‘to run’  \rightarrow [mð̞áɡan] ‘will run’ \\
{}[m-] + [ɡwáɁ] ‘to go out’ \rightarrow [mwaɁ] ‘will go out’ \\
{}[m-] + [pjʊ́ŋkʊt] ‘to assume a fetal position’ \rightarrow [mjʊ́ŋkʊt] ‘will assume a fetal position’ \\
{}[m-] + [lábɤ̞t] ‘to participate’ \rightarrow [mábɤ̞t] ‘will participate’ \\
{}[m-] + [sʊ́dlaj] ‘to comb’ \rightarrow [mʊ́dlaj] ‘will comb’ \\
{}[m-] + [sɪ́nɁad] ‘to cook grain’ \rightarrow [mɪ́nɁad] ‘will cook grain’
\z

Root-intial approximants /ð̞/, /w/ and /j/ are not omitted when [m-] is prefixed. This may be considered counterevidence to Harmon’s hypothesis that [m-] is an infix. The older generation uses this more than the younger generation, who tend to use [maɡ-] or [ma-] instead of [m-] as the intransitive, irrealis forms of approximant-initial roots.

\ea
{}[m-] + [ð̞ábʊ] ‘to capsize’  \rightarrow [mð̞ábʊ] ‘will capsize’ \\
{}[m-] + [wasák] ‘to scatter’ \rightarrow [mwasák] ‘will scatter’ \\
{}[m-] + [jápʊn] ‘supper’ \rightarrow [mjápʊn] ‘will eat supper’
\is{consonant omission|)}
\z

\subsection{Consonant gemination before enclitic demonstrative determiners}
\label{sec:gemination} \is{consonants!gemination|(}

There are three enclitic demonstrative determiners [i], [an] and [jaɁ] that indicate definiteness, location in space and time, and staging of participants (see \chapref{chap:referringexpressions}, \sectref{sec:definiteness}). When any of the enclitic demonstrative determiners follows a nominal that ends in a non-glottal consonant, the final consonant of the nominal is geminated and the second component of the geminate occurs as a transitional consonant. With a glottal-final or [ɾ]-final nominal, the enclitics are added without gemination. This rule must ``precede" (or take precedence over) rhoticization, since when a root ends in a syllable closed by /d/, the /d/ is not rhoticized before the vowel-initial enclitics, [i] or [an], though this is an environment for rhoticization for vowel-initial suffixes (see \sectref{sec:rhoticization}). 

\newpage
When the enclitic [jaɁ], \textsc{def.f}, occurs after a liquid- or approximant{}-final nominal there is no geminate transitional consonant.

When the enclitics [i] and [an] occur after an [i]-final nominal, then a transitional [ʲ] is inserted before [i] and [an]. When the enclitic [i] occurs after an [a]-final nominal, then a transitional [ʲ] is inserted before [i], but not with the enclitic [an]. With [u]{}-final nominals, [ʷ] is inserted as a transition before [i] and [an]. No nominal was found with [ɤ̞] finally. With vowel-final nominals, there is no transition before the enclitic [jaɁ].

\tabref{tab2.12} presents a representative list of examples of gemination before the enclitic demonstrative determiners.


\begin{table}
\caption{Gemination and transition consonants before enclitic demonstrative determiners}
\label{tab2.12}
\begin{tabularx}{\textwidth}{lQQQQ}
\lsptoprule
Final C/V & Root & Enclitic [i] & Enclitic [an] & Enclitic [jaɁ] \\
\midrule
p & [Ɂátɤ̞p] ‘roof’ & [Ɂatɤ̞́pᵖi] & [Ɂatɤ̞́pᵖan] & [Ɂatɤ̞́pᵖjaɁ] \\
t & [kað̞át] ‘rope’ & [kað̞átᵗi] & [kað̞átᵗan] & [kað̞átᵗjaɁ] \\
k & [píjak] ‘chick’ & [pijákᵏi] & [pijákᵏan] & [pijákᵏjaɁ] \\
Ɂ & [bájʊɁ] ‘clothes/shirt’ & [bajúɁi] & [bajúɁan] & [bajúɁjaɁ] \\
b & [kajáb] ‘hand fan’ & [kajábᵇi] & [kajábᵇan] & [kajábᵇjaɁ] \\
d & [sɤ̞́llɤ̞d] ‘room’ & [sɤ̞llɤ̞́dᵈi] & [sɤ̞llɤ̞́dᵈan] & [sɤ̞llɤ̞́dᵈjaɁ] \\
ɡ & [tállʊɡ] ‘egg’ & [tallʊ́ɡᵍi] & [tallʊ́ɡᵍan] & [tallʊ́ɡᵍjaɁ] \\
s & [bákð̞ɤ̞s] ‘belt’ & [bakð̞ɤ̞́sˢi] & [bakð̞ɤ̞́sˢan] & [bakð̞ɤ̞́sˢjaɁ] \\
m & [Ɂájam] ‘dog’ & [Ɂajámᵐi] & [Ɂajámᵐan] & [ɁajámᵐjaɁ] \\
n & [dánɤ̞n] ‘they’ & [danɤ̞́nⁿi] & [danɤ̞́nⁿan] & [danɤ̞́nⁿjaɁ] \\
ŋ & [dɪ́ŋdɪŋ] ‘wall’ & [dɪŋdɪ́ŋᵑi] & [dɪŋdɪ́ŋᵑan] & [dɪŋdɪ́ŋᵑjaɁ] \\
l & [kawɪ́l] ‘fish hook’ & [kawɪ́lˡi] & [kawɪ́lˡan] & [kawɪ́ljaɁ] \\
ɾ & [túkaɾ] ‘play music’ & [tukáɾi] & [tʊkáɾan] & [tukáɾjaɁ] \\
ð̞ & [baɡʊð̞] ‘coconut shell’ & [baɡʊð̞\textsuperscript{ð̞}i] & [baɡʊð̞\textsuperscript{ð̞}an] & [baɡʊð̞jaɁ] \\
w & [Ɂámbaw] ‘rat’ & [Ɂambáwʷi] & [Ɂambáwʷan] & [ɁambáwjaɁ] \\
j & [báð̞aj] ‘house’ & [bað̞ájʲi] & [bað̞ájʲan] & [bað̞ájjaɁ] \\
a & [máta] ‘eye’ & [matáʲi] & [matáan] & [matájaɁ] \\
i & [bái] ‘woman’ & [baíʲi] & [baíʲan] & [baíjaɁ] \\
u & [básʊ] ‘glass’ & [basúʷi] & [basúʷan] & [basújaɁ] \\
\lspbottomrule
\end{tabularx}
\end{table}

Enclitic pronouns also undergo this process. With second person pronouns, singular [kaʊ́n] and plural [kjʊ], the enclitic /i/ does not occur and the enclitic [jaɁ] is rare. Only [an] occurs and it varies as follows: [kaʊ́nⁿan/kaʊ́nⁿjan] and [kjúʷan/kjúʲan]. \tabref{tab2.13} illustrates these patterns.
\begin{table}
\caption{Pronouns with enclitics}
\label{tab2.13}
\begin{tabularx}{\textwidth}{l@{~}lQl}
\lsptoprule
Pronoun & Enclitic [ɪ] & \mbox{Enclitic [an]} & Enclitic [jaɁ] \\
\midrule
{}[ku] 1s, \textsc{erg}/\textsc{gen} & [kuʷi] & [kuʷan] & [kujaɁ] \\
\tablevspace
{}[nu] 2s, \textsc{erg}/\textsc{gen} & [nuʷi] & [nuʷan] & [nujaɁ] \\
\tablevspace
{}[dɪn] 3s, \textsc{erg}/\textsc{gen} & [dɪnⁿi] & [dɪnⁿan] & [dɪnⁿjaɁ] \\
\tablevspace
{}[ta] 1\textsc{p.incl}, \textsc{erg}/\textsc{gen} & [tai] & [taan] & [tajaɁ] \\
\tablevspace
{}[naj] 1\textsc{p.excl}, \textsc{erg}/\textsc{gen} & [najʲi] & [najʲan] & [najjaɁ] \\
\tablevspace
{}[nju] 2\textsc{p.erg/gen} & [njuʷi] & [njuʷan] & [njujaɁ] \\
\tablevspace
{}[dánɤ̞n]/[dáɤ̞n] 3\textsc{p.erg/gen} & none & none & none \\
\tablevspace
{}[jákɤ̞n] 1\textsc{s.abs} & [jakɤ̞́nⁿi] & none & none \\
\tablevspace
{}[kaʊn] 2\textsc{s.abs} & none & [kaʊnⁿan]{\slash}\mbox{[kaʊnⁿjan]} & none \\
\tablevspace
{}[kánɤ̞n] 3\textsc{s.abs} & [kanɤ̞́nⁿi] & [kanɤ̞́nⁿan] & [kanɤ̞́nⁿjaɁ] \\
\tablevspace
{}[kítɤ̞n] 1\textsc{p.incl.abs} & [kɪtɤ̞́nⁿi] & none & none \\
\tablevspace
{}[kámi] 1\textsc{p.excl.abs} & [kamíʲi] & none & none \\
\tablevspace
{}[kju] 2\textsc{p.abs} & none & [kjúʷan]/ [kjúʲan] & none \\
\tablevspace
{}[dánɤn] 3\textsc{p.abs} & [danɤ̞́nⁿi] & [danɤ̞́nⁿan] & [danɤ̞́nⁿjaɁ] \\
\tablevspace
{}[Ɂákɤ̞Ɂ] 1\textsc{s.gen} (preposed) & [Ɂakɤ̞́Ɂi] & [Ɂakɤ̞́Ɂan] & [Ɂakɤ̞́ɁjaɁ] \\
\tablevspace
{}[Ɂímʊ] 2\textsc{s.gen} (preposed) & [Ɂimúʷi] & [Ɂimúʷan] & [ɁimújaɁ] \\
\tablevspace
{}[Ɂíja] 3\textsc{s.gen} (preposed) & [Ɂijáʲi] & [Ɂijáan] & [ɁijájaɁ] \\
\tablevspace
{}[Ɂátɤ̞Ɂ] 1\textsc{p.gen.incl} (preposed) & [Ɂatɤ̞́Ɂi] & [Ɂatɤ̞́Ɂan] & [Ɂatɤ̞́ɁjaɁ] \\
\tablevspace
{}[Ɂámɤ̞Ɂ] 1\textsc{p.gen.excl} (preposed) & [Ɂamɤ̞́Ɂi] & [Ɂamɤ̞́Ɂan] & [Ɂamɤ̞́ɁjaɁ] \\
\tablevspace
{}[Ɂɪnjú] 2\textsc{p.gen} (preposed) [Ɂɪnjúʷi] & [Ɂɪnjúʷan] & [ɁɪnjújaɁ] \\
\tablevspace
{}[Ɂiɾán] 3\textsc{p.gen} (preposed) & [Ɂiɾánⁿi] & [Ɂɪɾánⁿan] & [ɁiɾánⁿjaɁ] \\
\lspbottomrule
\end{tabularx}
\end{table}
\is{consonants!gemination|)}
\subsection{Nasal assimilation and consonant deletion}\label{nasal-assimilation} \is{nasal assimilation|(}

When a nasal-final prefix occurs, such as N- ‘pluraction’ and [paŋ-] ‘action nominalizer’, the nasal usually assimilates to the point of articulation of the initial consonant of the root, and the initial consonant is omitted. If the initial consonant is bilabial, then the nasal becomes [m]. If the initial consonant is dental, then the nasal becomes [n]. The root-initial [d] is not omitted in some roots when prefixed. When [d] is not omitted, sometimes the nasal assimilates and sometimes not. For some speakers, the initial [l] in roots is optionally omitted after the prefix, but other speakers do not accept this variation. When [l] remains, then the nasal optionally assimilates. If the initial consonant of the root is [k], [ʔ], [ɡ] or a nasal, then the final nasal of the prefix remains [ŋ], and the root-initial consonant is omitted. Root-initial [ɡ], [m] and [n] are not omitted. When the initial consonant of the root is a non-lateral approximant, it remains and the nasal of the prefix is [ŋ].
\begin{table}
\caption{Nasal-final prefixes and consonant omission}
\label{tab2.14}
\begin{tabular}
    {
    >{\RaggedRight\arraybackslash}p{70pt}
    >{\RaggedRight\arraybackslash}p{60pt}
    >{\RaggedRight\arraybackslash}p{56pt}
    >{\RaggedRight\arraybackslash}p{56pt}
    >{\RaggedRight\arraybackslash}p{54pt}
    }
\lsptoprule
Root & ma-N- & na-N- & ɡa-N- & paŋ- \\
\midrule
{}[pánaɁ] ‘arrow’ & [mamánaɁ]  & [namánaɁ] & [ɡamánaɁ] & none \\
{}[táɪɁ] ‘to sew’ & [manáɪɁ] & [nanáɪɁ] & [ɡanáɪɁ] & none \\
{}[káʊj] ‘tree’ & [maŋáʊj] & [naŋáʊj]  & [ɡaŋáʊj] & [paŋáʊj] \\
{}[Ɂúma] ‘field’ & [maŋúma] & [naŋúma]  & [ɡaŋúma]  & none \\
{}[bátɁaw] ‘torch/light’ & [mamátɁaw] & [namátɁaw]  & [ɡamátɁaw] & none \\
{}[dáɡat] ‘sea/ ocean’ & [manáɡat] & [nanáɡat]  & [ɡanáɡat]  & none \\
{}[dájaɁ] ‘to deceive’ & [mandájaɁ]/ [maŋdájaɁ]  & [nandájaɁ]/ [naŋdájaɁ]  & none & [pandájaɁ]/ {}[paŋdájaɁ] \\
{}[ɡʊ́ntɪŋ] ‘scissors’ & [maŋɡʊ́ntɪŋ] & [naŋɡʊ́ntɪŋ] & none & [paŋɡʊ́ntɪŋ] \\
{}[lisʊk] ‘to bud’ & [manlisʊk]/ [maŋlisʊ́k] & [nanlisʊk/ [naŋlisʊk]  & [ɡanlisʊk/ 
ɡaŋlisʊk]  & none \\
{}[sáɪd] ‘drag net’ & [manáɪd] & [nanáɪd] & [ɡanáɪd] & none \\
{}[málaɁ] ‘to change clothes’ & [maŋmálaɁ] & [naŋmálaɁ] & [ɡaŋmálaɁ]  & none \\
{}[nákð̞aɁ] ‘to click tongue’ & [maŋnákð̞aɁ]  & [naŋnákð̞aɁ]  & [ɡaŋnákð̞aɁ] & none \\
{}[ŋáddʊð̞] ‘puberty’ & [maŋáddʊð̞] & [naŋáddʊð̞] & [ɡaŋáddʊð̞] & none \\
{}[ð̞áð̞a] ‘to get to know someone’ & [maŋð̞áð̞a]  & [naŋð̞áð̞a]  & [ɡaŋð̞áð̞a]  & none \\
{}[wáslɪɡ] ‘to throw away’ & [maŋwáslɪɡ] & [naŋwáslɪɡ]  & none & none \\
{}[jáɡak] ‘to cry’ & [maŋjáɡak] & [naŋjáɡak] & [ɡaŋjáɡak]  & none \\
\lspbottomrule
\end{tabular}
\end{table}
\is{nasal assimilation|)}
\is{morphophonemics|)}
\is{phonological processes|)}

\newpage
\section{Relaxed speech}
\label{bkm:Ref360478279} \is{relaxed speech|(}
In relaxed speech, intervocalic /d/ may become [ɾ]. For example, there is a commonly used phrase [mánaw káɾi] ‘come here’ which comes from [mánaw ká di]; \textit{di} is a locational demonstrative. Also in relaxed speech, when a suffix is added to a verb that begins with a glottal stop, then the glottal stop and following vowel can optionally be omitted.


\ea
{}[Ɂátaɡ] ‘to give’+ [-an] \rightarrow [Ɂatáɡan/táɡan] ‘will give’ \\
{}[Ɂípɪd] ‘to fold’ + [-ɤ̞n] \rightarrow [Ɂipíɾɤ̞n/píɾɤ̞n] ‘will fold’ \\
{}[Ɂɤ̞́ð̞ɤ̞s] ‘to borrow’ + [-ɤ̞n] \rightarrow [Ɂɤ̞ð̞ɤ̞́sɤ̞n/ð̞ɤ̞́sɤ̞n] ‘will borrow’ \\
{}[Ɂútad] ‘to step on’+ [-an] \rightarrow [Ɂutáɾan/táɾan] ‘will step on’
\z

Note that when [Ɂɤ̞́ð̞ɤ̞s] ‘to borrow’ takes a suffix and the initial syllable is omitted then the interdental optionally becomes the lateral /l/.

\ea
{}[Ɂɤ̞́ð̞ɤ̞s]  + [-an] \rightarrow [lɤ̞́sɤ̞n]  or [ð̞ɤ̞́sɤ̞n]
\is{relaxed speech|)}
\z

\section{Orthography and transcription conventions} \is{orthography|(}
\label{bkm:Ref360478305}\is{transcription conventions|(}
Throughout the remainder of this grammar, Kagayanen examples are written following orthography conventions commonly used by Kagayanen writers, with a few exceptions as explained in this section. In 2003, SIL-Philippines approved an orthography for literature published in Kagayanen under the auspices of SIL-Philippines. Most Kagayanen publications to date employ that system. The examples of Kagayanen in the rest of the chapters follow the conventions of this standard orthography with the exception of the interdental approximant [ð̞], which we indicate with \textit{ļ}, as mentioned above and further discussed in this section.

\hspace*{-1pt}As is common for working orthographies in the Philippines, word-initial, word-final and intervocalic glottal stop is not written. Thus, [ʔáɡi] ‘to pass’ is written \textit{aɡi}, [lútʊʔ] ‘dark/bright color’ is written \textit{luto}, and [labáʔab] ‘lukewarm’ is written \textit{labaab}. Only the glottal stop that is word medial after a consonant and before a vowel is  symbolized with a hyphen. For example [lʊ́jʔa] ‘ginger’ is written as \textit{luy-a}.

The nasal [ŋ] is written \textit{ng}, for example, [bɪ́ŋbɪŋ] ‘to carry in a bucket like container’ is written orthographically as \textit{bingbing}.

Primary school teachers agree that it is important, at least for beginning readers, to indicate the difference between the phonemically distinct lateral and interdental approximants (see \sectref{bkm:Ref360478146}), even though they are not distinguished in the SIL orthography. Various alternatives have been suggested for indicating the interdental approximant, including underscoring and capitalization. However, initial orthography testing suggests that writing the interdental approximant with a lowered comma or cedilla is optimal for making the contrast clear, while maintaining maximal similarity with the common orthography. In this grammar we will indicate the interdental approximant /ð̞/ as \textit{ļ} (upper case \textit{\c{L}}) following recommendations of several Department of Education (DepEd) teachers. Thus, for example, [kð̞áʊt] ‘kind of small bean’ is written \textit{kļaot}.

The approximant /j/ is written as \textit{y}, so [jábɁʊk] ‘dust’ is written \textit{yab-ok.}

All allophones of /i/ are written with \textit{i} and all the allophones of /u/ with \textit{u} except when /u/ occurs in the final syllable of a word, In which case it is written as \textit{o}. The vowel /ɤ̞/ is written as \textit{e}.

A geminate transitional consonant before an enclitic will not be written. A space is written before the enclitic. Thus, [ʔikámᵐan] ‘the mat’ is written as \textit{ikam an} and [ʔikámkuʷan] ‘my mat’ is written as \textit{ikam ko an.}

In the second line of data in numbered examples, a morphologically analyzed version of the standard orthography is employed. In this line, an en-dash (--) indicates \isi{concatenative morpheme boundaries}, and a simple dash (-) indicates a glottal stop, as per the preceding discussion. Thus when a glottal-stop initial root, such as \textit{anod} [ʔanʊ́d] `drift' takes a consonant-final prefix, such as \textit{pag}- `action nominalizer', both an en-dash (for the morpheme boundary), and a simple dash (for the glottal stop) appear: \textit{pag--{}-anod}. Other conventions regarding morpheme boundaries include angle brackets surrounding infixes (e.g. \textit{<in>} `resultative nominalizer'), and the tilde character (\sim{}) indicating boundaries between reduplicate and reduplicand (e.g., \textit{anod\sim{}-anod} `keep on drifting', \textit{luto\sim{}luto} 'keep on cooking'). 

Other features of the second line of data in examples include the use of a zero symbol (\emptyset{}) to represent morphemes with no phonetic realization, and the ellipsis character (...) to indicate conversational omission of affixes (see \chapref{chap:verbstructure}, \sectref{sec:omissionofaffixes}). These two phenomena are seldom confusable in context, though both involve the absense of phonological material. 
\is{transcription conventions|)}
\is{orthography|)}
