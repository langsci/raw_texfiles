\chapter{Semantically motivated verb classes}
\label{chap:verbclasses-2}
\section{Introduction}
\label{sec:introduction-9}

In this chapter we describe a few semantically motivated verb classes not mentioned in \chapref{chap:verbclasses-1}. In many cases, these classes have characteristics of the major classes mentioned in \chapref{chap:verbclasses-1}, but all exhibit some distinctive morphosyntactic properties that warrant their separate treatment. Since this chapter takes a “semantics prior” approach, the section headings describe particularly salient situation types, and the content of each section describes the constructions that characteristically express such situations.

\section{Meteorological situations}
\label{sec:meteorologicalsituations}

Meteorological situations are those which describe weather and other natural conditions. They typically have no actor or undergoer participants, and so technically are not semantically transitive\is{semantic transitivity}\is{transitivity!semantic} or intransitive. In Kagayanen, predicates describing many meterological and other natural phenomena do not require a distinct absolutive argument. Bare roots in such situations are usually inflected with affixes from the intransitive, dynamic set:

\ea
Isya  na  mapon  \textbf{guran}   ta  dakmeļ. \\\smallskip
\gll Isya  na  mapon  \textbf{ga-uran}   ta  dakmeļ. \\
one  \textsc{lk} afternoon  \textsc{i.r}-rain  \textsc{nabs}  thick \\
\glt ‘One afternoon it \textbf{was raining} heavily (lit. thickly).’ [JCWN-L-33 27.1]
\z
\ea
Kabay  pa  na  \textbf{mag-uran}. \\\smallskip
\gll Kabay  pa  na  \textbf{mag-uran}. \\
may.it.be  \textsc{inc}  \textsc{cm}  \textsc{i.ir}-rain \\
\glt ‘May it (be that it will) \textbf{rain}.’ (I hope it will rain.)
\z

\newpage
\ea
Mag-alas  dos  kagaen,  gulpi  nang  \textbf{galinaw}. \\\smallskip
\gll Mag-alas  dos  kagaen,  gulpi  nang  \textbf{ga-linaw}. \\
\textsc{i.ir}-o’clock  two  before.daybreak  suddenly  just  \textsc{i.r}-calm \\
\glt ‘It was almost two o’clock before dawn just suddenly the (sea) \textbf{became calm}.’ [VAWN-T-18 5.3]
\z
\ea
Naambaļ  ta  manakem  daw  \textbf{magtaliktik}  gani  mian  angin  an, mian ta mait,  \textbf{magbagyo}  gid. \\\smallskip
\gll Na-ambaļ  ta  manakem  daw  \textbf{mag-taliktik}  gani  mian  angin  an, mian ta mait,  \textbf{mag-bagyo}  gid. \\
\textsc{a.hap.r}-say  \textsc{nabs}  older  if/when  \textsc{i.ir}-light.rain  truly  northeast.wind  wind \textsc{def.m} northeast.wind \textsc{nabs} season  \textsc{i.ir}-typhoon  \textsc{int} \\
\glt ‘The older ones have said when \textbf{it rains lightly} the wind is the northeast wind, the northeast wind season, it \textbf{will really typhoon}.’ [PCON-C-01 2.6]
\z

Happenstantial inflections are normally disallowed with bare roots expressing meteorological phenomena:

\ea
    \ea[*] {
    Nadlaw	en	anduni. \\\smallskip
\gll Na-adlaw	en	anduni. \\
    \textsc{a.hap.r}-day/sun	\textsc{cm}	now/today \\
    \glt (‘It is sunny now/today.’) 
    }
    \ex[*]{
    Madlaw en kani. \\\smallskip
\gll Ma-adlaw	en	kani. \\
    \textsc{a.hap.ir}-day/sun	\textsc{cm}	later \\
    \glt (`It will be sunny later.’)
    }
    \z
\z

Roots describing meteorological phenomena may take applicative morphology, in which case a typically animate affected participant occurs in the absolutive. The applicative stem is then inflected in happenstantial modality. This is a kind of \textit{experiential} construction in which the experiencer of the situation is the undergoer  (exs. \ref{ex:travelling}-\ref{ex:seaweed}):


\ea
\label{ex:travelling}
\textbf{Nalinawan}  kami  i  ta  ame  na  pagbyai. \\\smallskip
\gll \textbf{Na-linaw-an}  kami  i  ta  ame  na  pag-byai. \\
\textsc{a.hap.r}-calm-\textsc{apl}  1\textsc{p.excl.abs}  \textsc{def.n}  \textsc{nabs}  1\textsc{p.excl.gen}  \textsc{lk}  \textsc{nr.act}-travel \\
\glt ‘(It) \textbf{happened to become calm} on us during our traveling.’
\z
\ea
\textbf{Naadlawan}  kay  en  na  munta  uma. \\\smallskip
\gll \textbf{Na-adlaw-an}  kay  en  na  munta  uma. \\
\textsc{a.hap.r}-adlaw-\textsc{apl}  1\textsc{p.excl.abs}  \textsc{cm}  \textsc{lk}  toward  field \\
\glt ‘(It) \textbf{became sunny} on us when going to the field.’
\z
\ea
\textbf{Naadlawan}  a,  tapos  \textbf{nauranan}  man.  Busa, yaken  nalagnat  aren. \\\smallskip
\gll \textbf{Na-adlaw-an}  a,  tapos  \textbf{na-uran-an}  man.  Busa, yaken  na-lagnat  aren. \\
\textsc{a.hap.r}-adlaw-\textsc{apl}  1\textsc{s.abs}  afterwards  \textsc{a.hap.r}-rain-\textsc{apl}  also  so
1\textsc{s.abs}  \textsc{a.hap.r}-fever  1\textsc{s.abs} \\
\glt `It was sunny on me, afterwards it rained (on me) also. So, I had become feverish.’ [MBON-T-07 2.12]
\z
% \ea
% \label{ex:yesterday}
% \textbf{Gapauran}  a  nang  gibii. \\\smallskip
% \gll \textbf{Ga-pa-uran}  a  nang  gibii. \\
% \textsc{i.r}-\textsc{caus}-rain  1\textsc{s.abs}  only/just  yesterday \\
% \glt ‘I just \textbf{let it rain} on me yesterday.’ (Instead of going in the house he kept trying to spear a wild pig.) [RCON-L-01 3.9]
% \z

While the absolutive in an applicative meteorological construction is typically human, example \REF{ex:seaweed} illustrates that it need not be. In this example \textit{guso} ‘agar seaweed’ is the absolutive: 
\ea
\label{ex:seaweed}
Guso  i  daw  \textbf{maadlawan}  ta  darwa  nang  na  oras ta  takas  gayem-ek. \\\smallskip
\gll Guso  i  daw  \textbf{ma-adlaw-an}  ta  darwa  nang  na  oras ta  takas  ga-yem-ek. \\
agar.seaweed  \textsc{def.n}  if/when  \textsc{a.hap.ir}-day/sun-\textsc{apl}  \textsc{nabs}  two  only/just  \textsc{lk}  hours \textsc{nabs}  up.hill  \textsc{i.r}-soft \\
\glt `The agar seaweed when in the sunshine for just two hours on the shore it becomes soft.’ [MEWE-T-01 8.2]
\z

Example \REF{bkm:Ref149029627} illustrates the root \textit{uran} ‘rain’ in a dynamic transitive construction, with causative and applicative affixes. Note that the absolutive is the sun-dried corn: 

\ea 
\label{bkm:Ref149029627}
Papauranan  ko  nang  binaļad  ya  na  mais. \\\smallskip
\gll Pa-pa-uran-an  ko  nang  b<in>aļad  ya  na  mais. \\
\textsc{t.r-caus}-rain-\textsc{apl}  1\textsc{s.erg}  only/just  <\textsc{nr.res}>dry.in.sun  \textsc{def.f}  \textsc{lk}  corn \\
\glt ‘I just let it rain on the sun-dried corn.’
\z  

Example \REF{bkm:Ref149030144} illustrates a kind of reflexive construction. This is not an experiential construction (i.e., there is no applicative marker indicating the absolutive is the experiencer of the situation), but the implication is that the speaker “let it rain” on himself while hunting (see \chapref{chap:voice}, \sectref{sec:causatives} for this reflexive use of the detransitive causative \textit{ga-pa-} or \textit{mag-pa-}):

\ea 
\label{bkm:Ref149030144}
\textbf{Gapauran}  a  nang  gibii. \\\smallskip
\gll \textbf{Ga-pa-uran}  a  nang  gibii. \\
\textsc{i.r}-\textsc{caus}-rain  1\textsc{s.abs}  only/just  yesterday \\
\glt ‘I just \textbf{let it rain} (on me) yesterday.’ (Instead of going inside he kept trying to spear a wild pig.) [RCON-L-01 3.9]
\z

\section{Utterance}
\label{sec:utterance}

Situations involving speech (or other types of utterance, such as shouting, whispering, asking or replying) are similar to situations of transfer (Class VII as identified in \chapref{chap:verbclasses-1}) in that the content of the utterance (what is said) is presented as a theme that is “transferred” from the Speaker to the Addressee. In Kagayanen predicates that depict utterance situations, the actor is always the speaker, but either the theme or the recipient may be expressed in the absolutive case, as will be described below.

With the root \textit{ambaļ} ‘to say’ the addressee(s) does not have to be explicitly mentioned if it is obvious in the context. The quote can be either indirect \REF{bkm:Ref361243181} or direct \REF{bkm:Ref402322324}. Notice also that in the first part of the sentence in example \REF{bkm:Ref402322324} there is no quote. The quote is in the previous sentence in the text.

\ea
\label{bkm:Ref361243181}
Detransitive (Actor = absolutive), indirect quote: \\
Miyirkules  sellem  \textbf{gambaļ}  a  ta  sawa  ko   na  maglangoy aren  inta  unduni. \\\smallskip
\gll Miyirkules  sellem  \textbf{ga-ambaļ}  a  ta  sawa  ko   na  mag-langoy aren  inta  unduni. \\
Wednesday  morning  \textsc{i.r}-say  1\textsc{s.abs}  \textsc{nabs}  spouse  2\textsc{s.gen}
\textsc{lk}  \textsc{i.ir}-bathe 1\textsc{s.abs}  \textsc{opt}  now/today \\
\glt ‘Wednesday morning I said to my husband that I should bathe now.’ [EMWN-T-05 4.6]
\z
\ea
\label{bkm:Ref402322324}
Detransitive (Actor = absolutive), direct quote: \\
Na  \textbf{gambaļ}  ake  na  papa  gatuļo  iya  na  luwa  daw \textbf{ambaļ}  din,  “Malipayen  aren  daw  makita  ko    ake  na  mga  kabataan  daw  dayad  iran  na  lawa." \\\smallskip
\gll Na  \textbf{ga-mbaļ}  ake  na  papa  ga-tuļo  iya  na  luwa  daw \textbf{ …ambaļ}  din,  “Ma-lipay-en  aren  daw  ma-kita  ko    ake  na  mga  ka-bata-an  daw  dayad  iran  na  lawa." \\
\textsc{lk}  \textsc{i.r}-say  1\textsc{s.gen}  \textsc{lk}  papa  \textsc{i.r}-drip  3\textsc{s.gen}  \textsc{lk}  tears  and
\textsc{t.r}-say  3\textsc{s.erg}  \textsc{ar}-joy-\textsc{ar}  1\textsc{s.abs}  if/when  \textsc{a.hap.ir}-see
1\textsc{s.erg}  1\textsc{s.gen}  \textsc{lk}  \textsc{pl}  \textsc{nr}-child-\textsc{nr}  and  good  3\textsc{p.gen}  \textsc{lk}  body \\
\glt `When my papa was saying (this), his tears were dripping and he said, “I am joyful when I see my children and their bodies are well.”’ [CBWE-C-05 5.5]
\z  

\ea
Transitive (Theme = absolutive): \\
Dason  \textbf{pambaļ}  nanay  ko,  “Man-o  kon  uļa  luag   sinayaw   Bantayan?" \\\smallskip
\gll Dason  \textbf{pa-ambaļ}  nanay  ko,  “Man-o  kon  uļa  luag   s<in>ayaw   Bantayan?" \\
next  \textsc{t.r}-say  mother  1\textsc{s.gen}  why  \textsc{hsy}  \textsc{neg.r}  watch
<\textsc{nr.res}>dance Bantayan \\
\glt `Next my mother said, “Why are (you) not watching the dance in Bantayan?”’ [BMON-C-06 2.29]
\z

\ea
Transitive (Theme = absolutive), Deontic modality: \\
\textbf{Iambaļ}  no  ki  kanen  daw  ino  ubraen  din. \\\smallskip
\gll \textbf{I-ambaļ}  no  ki  kanen  daw  ino  ubra-en  din. \\
\textsc{t.deon}-say  2\textsc{s.erg}  \textsc{obl.p}  3s  if/when  what  work-\textsc{t.ir}  3\textsc{s.gen} \\
\glt ‘You should say to him/her whatever s/he will work on.’
\z

\ea
Applicative (Recipient = absolutive): \\
\textbf{Pambaļan}  kay  mama  na  mangamuyo  kay  nang kon.\\\smallskip
\gll \textbf{Pa-ambaļ-an}  kay  mama  na  m-pangamuyo  kay  nang kon.\\
\textsc{t.r}-say-\textsc{apl}  1\textsc{p.excl.abs}  mother  \textsc{lk}  \textsc{i.ir}-pray  1\textsc{p.excl.abs}  only/just \textsc{hsy} \\
\glt ‘Mother said to us that we will just pray.’ [VAWN-T-18 4.6]
\z

Occasionally, the recipient may be expressed as the absolutive with no applicative marking. In that case, the implication is the speech is more direct or frank. For example, \REF{bkm:Ref402322426} is about a fairy telling a person to go with him to the world of the dead. This is an unusual and momentous direct request, and so the applicative is not used. This emphasizes the usually direct nature of this statement. Example \REF{bkm:Ref122516350} is about a woman telling her mother to regret or repent of her sins, which also is an unusually frank thing for anyone to say to their parents.

\ea
\label{bkm:Ref402322426}
\textbf{Pambaļ}  din  yaken  i  na  muyog  a  mamasyar. \\\smallskip
\gll \textbf{Pa-ambaļ}  din  yaken  i  na  m-kuyog  a  ma-ng-pasyar. \\
\textsc{t.r}-say  3\textsc{s.erg}  1\textsc{s.abs}  \textsc{def.n}  \textsc{lk}  \textsc{i.v.ir}-go.with  1\textsc{s.abs}  \textsc{a.hap.ir-pl}-visit \\
\glt ‘S/he told me that I will go with (him) to go visiting.’ [EDON-J-01 1.7]
\z
\ea
\label{bkm:Ref122516350}
\textbf{Pambaļ}  ko  kanen  an,  “Maneļseļ  ka   ta  imo  na  saļa.\\\smallskip
\gll \textbf{Pa-ambaļ}  ko  kanen  an,  “Ma-neļseļ  ka   ta  imo  na  saļa."\\
\textsc{t.r}-say  1\textsc{s.erg}  3\textsc{s.abs}  \textsc{def.m}  \textsc{a.hap.ir}-regret/repent  2\textsc{s.abs} \textsc{nabs}  2\textsc{s.gen}  \textsc{lk}  sin/mistake \\
\glt ‘I told her (directly), “You may regret/repent of your sins.”' [BMON-C-06 4.11]
\z

The transitive, deontic mood prefix \textit{i-} may occur with a Beneficiary as absolutive. This may be considered an applicative usage of the \textit{i-} prefix:

\ea
\textbf{Iambaļ}  a  no  naan  ta  mistro  na  absint  a. \\\smallskip
\gll \textbf{I-ambaļ}  a  no  naan  ta  mistro  na  absint  a. \\
\textsc{t.deon}-say  1\textsc{s.abs}  2\textsc{s.erg}  \textsc{spat.def}  \textsc{nabs}  teacher  \textsc{lk}  absent  1\textsc{s.abs} \\
\glt ‘(You must) \textbf{say} for me to the teacher that I am absent.’
\z

Additional examples of utterance predicates from the texts follow:

\ea
Isya  na  sellem  nanampara  Pedro  i  ta  Mamaan. Nasugat  din  manakem  na i.  \textbf{Painsaan}  Pedro  daw  indi  punta?  \textbf{Ambaļ}  din  nangita  ta  layaen  darwa  danen magsawa.  \textbf{Gasugid}  nang  man  a  Pedro  i  na  kanen manampara.   \textbf{Insaan}  din  daw  kino  duma  din. \textbf{Sabat}  Pedro,  subuok  nang  kanen. \textbf{Insa}  ta  manakem  isab, “Uļa  ka  adlek?”  \textbf{Sabat}  Pedro,  “Ino  adlekan  ko  a?  Daw  may magpadlek,  dili  katilaw  yan  ta  pana  i.”\\\smallskip
\gll Isya  na  sellem  na-ng-tampara  Pedro  i  ta  Mamaan. Na-sugat  din  manakem  na i.  \textbf{Pa-insa-an}  Pedro  daw  indi  punta?  \textbf{ …-Ambaļ}  din  na-ngita  ta  laya-en  darwa  danen mag-sawa.  \textbf{Ga-sugid}  nang  man  a  Pedro  i  na  kanen ma-ng-tampara.  \textbf{ …-Insa-an}  din  daw  kino  duma  din. \textbf{ …-Sabat}  Pedro,  subuok  nang  kanen. \textbf{ …-Insa}  ta  manakem  isab, “Uļa  ka  adlek?”  \textbf{ …-Sabat}  Pedro,  “Ino  adlek-an  ko  a?  Daw  may mag-pa-adlek,  dili  ka-tilaw  yan  ta  pana  i.”\\
one  \textsc{lk}  morning  \textsc{a.hap.r}-\textsc{pl}-goggles  Pedro  \textsc{def.n}  \textsc{nabs}  Mamaan \textsc{a.hap.r}-meet  3\textsc{s.erg}  older  \textsc{lk} \textsc{d1adj}  \textsc{t.r}-ask-\textsc{apl}  Pedro if/when  where going  \textsc{t.r}-say  3\textsc{s.erg}  \textsc{a.hap.r}-search  \textsc{nabs}   cast.net-\textsc{t.ir}  two  3\textsc{p.erg} \textsc{rel}-spouse  \textsc{i.r}-tell  only/just  also  \textsc{inj}  Pedro  \textsc{def.n}  \textsc{lk}  3\textsc{s.abs} \textsc{a.hap.ir}-\textsc{pl}-goggles \textsc{t.r}-ask-\textsc{apl}  3\textsc{s.erg}  if/when  who  companion  3\textsc{s.gen}
\textsc{t.r}-answer  Pedro  one.piece  only/just  3\textsc{s.abs}  \textsc{t.r}-ask   \textsc{nabs}  older  again
\textsc{neg.r}  2\textsc{s.abs}  afraid  \textsc{t.r}-answer  Pedro  what  afraid-\textsc{apl}  1\textsc{s.erg}  \textsc{inj}  if/when  \textsc{ext.in} \textsc{i.ir}-\textsc{caus}-afraid  \textsc{neg.ir}  \textsc{i.exm}-try/taste  \textsc{d2abs}  \textsc{nabs}  spear  \textsc{def.n} \\
\glt `One day Pedro went spear fishing at Mamaan. He happened to meet this older one. Pedro \textbf{asked} (him) where he is going. He \textbf{said} that both his wife and he have been searching for something to cast net fish. Pedro just \textbf{told} him that he is going spear fishing. He \textbf{asked} him who is his companion. Pedro \textbf{answered} that he is only one. The older one \textbf{asked} again, “Aren’t you afraid?” Pedro \textbf{answered}, “What will I be afraid of? If there is something to cause me fear won’t that one experience the spear.”' [JCWN-T-26 5:1-10]
\z
\ea
Ti,  \textbf{naisturyaan}  din  arey  din  na  danen  an Pedro  daw  Maria   dali  nang  magkasaļ. \\\smallskip
\gll Ti,  \textbf{na-isturya-an}  din  arey  din  na  danen  an Pedro  daw  Maria   dali  nang  mag-kasaļ. \\
so  \textsc{a.hap.r}-converse/talk-\textsc{apl}  3\textsc{s.erg}  friend  3\textsc{s.gen}  \textsc{lk}  3\textsc{p.abs}  \textsc{def.m} Pedro  and  Maria  soon  only/just  \textsc{i.ir}-wedding \\
\glt `So, he had (previously) \textbf{conversed} with his friend that they, Pedro and Maria, soon will be wed.’ (This is a flashback in the story.) [LGON-L-01 11.2]
\z
\section{Perception}
\label{sec:perception}

Situations involving perception (as well as emotion and cognition described below) are semantically “somewhat transitive” in that there are two necessary participants involved: an experiencer and a source of the experience, sometimes refered to as a stimulus. For example, in an event of visual perception, the experiencer is the person or animal that sees something, and the stimulus is the thing seen. These situations are different from prototypical transitive situations in that there is little or no activity that is “transferred” from one participant to the other. In the words of \citet[164]{croft2001} the experiencer “attends to” the stimulus, but does not act on it in any way. The stimulus does not undergo a change as a result of being perceived. In some cases the experiencer does act intentionally, as in \textit{He listened to the radio}, or \textit{I looked at the horizon}. English expresses the difference between intentional and unintentional perception largely via distinct verb roots, though sometimes the same root can be used either way, for example, \textit{He smelled the roses} can mean he intentionally sniffed them, or he happened to smell them as he walked by. In Kagayanen this difference is largely expressed via Happenstantial versus Dynamic verbal inflections.

Non-volitional perception, such as seeing, hearing, and smelling something unintentionally, is usually inflected in experiencer modality. This seems to be the norm for roots that express peception. Recall that for non-volitional situations, experiencer modality is neutral with respect to grammatical transitivity – non-volitional verbs inflected in experiencer modality may occur in transitive or intransitive argument structure frames. This is understandable insofar as perception situations are semantically “somewhat transitive”, as mentioned above.

Intentional perception, such as watching, listening and actively tasting, is typically expressed in dynamic modality. Intentional or unintentional Experiential situations may be expressed grammatically in the various “voices” relevant to transitive situations in general, including detransitive (actor voice), and applicative (locative voice) constructions. These are just some of the patterns in which speakers use and adjust roots to express the precise meanings needed, according to specific communicative situations. The following examples from the corpus are of perception predicates that illustrate their various inflectional possibilities and their semantic effects.

We will begin with examples of the common root that expresses visual perception \textit{kita} ‘to see’. Predicates headed by this root take happenstantial affixes, unless they are the basis of causative or reciprocal stems.

\ea
\textbf{Nakita}  ko  danen  dya  na  sigi  kaan. \\\smallskip
\gll \textbf{Na-kita}  ko  danen  dya  na  sigi  kaan. \\
\textsc{a.hap.r}-see  1\textsc{s.erg}  3\textsc{s.abs}  \textsc{d4loc}  \textsc{lk}  continually  eat \\
\glt ‘I saw them there continually eating.’ [EFOB-J-03 2.9]
\z
\ea
\textbf{Nakita}  ko  na  may  anen  dya  na  baked  na  bangkilan, … \\\smallskip
\gll \textbf{Na-kita}  ko  na  may  anen  dya  na  baked  na  bangkilan  … \\
\textsc{a.hap.r}-see  1\textsc{s.erg}  \textsc{lk}  \textsc{ext.in}  \textsc{ext.g}  \textsc{d4loc}  \textsc{lk}  big  \textsc{lk}  tusked.one \\
\glt ‘I saw that there was some big-tusked wild pig  …’ [RCON-L-01 3.11]
\z
\ea
\textbf{Nakita}  din  na  gaitem  inog  an  na  lumboy. \\\smallskip
\gll \textbf{Na-kita}  din  na  ga-item  inog  an  na  lumboy. \\
\textsc{a.hap.r}-see  3\textsc{s.erg}  \textsc{lk}  \textsc{i.r}-black  ripe \textsc{def.m}  \textsc{lk}  Java.plum \\
\glt ‘He saw that the ripe Java plums were becoming black.’ [JCWN-L-38 7.3]
\z

\newpage
\ea
Maski  \textbf{makita}  din  ambaw,  kanen  nang  \textbf{magtan-aw}. \\\smallskip
\gll Maski  \textbf{ma-kita}  din  ambaw,  kanen  nang  \textbf{mag-tan-aw}. \\
even.if  \textsc{a.hap.ir}-see  3\textsc{s.erg}   mouse/rat  3\textsc{s.abs}  only/just  \textsc{i.ir}-look.at \\
\glt ‘Even if s/he (the cat) sees a mouse or rat, s/he will just look at it.’ [JCOO-C-10 8.1]
\z
\ea
Iling  din  makaļa  din  daw  \textbf{makita}  din.\\\smallskip
\gll Iling  din  ma-kaļa  din  daw  \textbf{ma-kita}  din.\\
say  3\textsc{s.erg}  \textsc{a.hap.ir}-know/recognize  3\textsc{s.erg}  if/when  \textsc{a.hap.ir}-see  3\textsc{s.erg}\\
\glt ‘He said that he will recognize him if he sees him.' [JCWN-T-26 4.13]
\z

\ea
May  isya  na  manakem  na  ngaran  din  Tatay  Pedro  na gasakay ta  isya  na  lunday  na  manampara. \textbf{Nakita}  din  ake  na  mangngod  na  nalemmes. \\\smallskip
\gll May  isya  na  manakem  na  ngaran  din  Tatay  Pedro  na ga-sakay ta  isya  na  lunday  na  ma-ng-tampara. \textbf{Na-kita}  din  ake  na  mangngod  na  na-lemmes. \\
\textsc{ext.in}  one  \textsc{lk}  older  \textsc{lk}  name  3\textsc{s.gen}  Father  Pedro  \textsc{lk}
\textsc{i.r}-ride   \textsc{nabs}  one  \textsc{lk}  outrigger.canoe  \textsc{lk}  \textsc{a.hap.ir-pl}-goggles
\textsc{a.hap.r}-see  3\textsc{s.erg}
1\textsc{s.gen} \textsc{lk}  younger.sibling  \textsc{lk}  \textsc{a.hap.r}-drown \\
\glt `There was an older person whose name was Father Pedro who was riding an outrigger canoe going fishing. He saw my younger sibling who had drowned.’ [LCWN-T-01 2.8-9]
\z

\ea
\textit{kita} ‘to see’ in detransitive frames: \\
Miron  na  mama  \textbf{kakita}  ti  na  bai. \\\smallskip
\gll Miron  na  mama  \textbf{ka-kita}  ti  na  bai. \\
single.man  \textsc{lk}  man  \textsc{i.exm}-see  \textsc{d1nabs}  \textsc{lk}  woman \\
\glt ‘The single man found (was enabled to see) this woman.’ [JCWN-T-25 2.18]
\z
\ea
\textbf{Kakita}  a  unso  ittaw  na  bakod. \\\smallskip
\gll \textbf{Ka-kita}  a  unso  ittaw  na  bakod. \\
\textsc{i.exm}-see  1\textsc{s.abs}  \textsc{d4loc.pr}  person  \textsc{lk}  big \\
\glt ‘I encountered (was enabled to see) right there a big person.’ [RCON-L-02 3.5]
\z

\newpage
\ea 
Piro  yi  na  manakem  \textbf{nakakita}  ki  danen  daw  diristso  din umaw  danen  an… \\\smallskip
\gll Piro  yi  na  manakem  \textbf{naka-kita}  ki  danen  daw  diristso  din  …-umaw  danen  an  … \\
but  \textsc{d}1\textsc{adj}  \textsc{lk}  older  \textsc{i.hap.r}-see  \textsc{obl.p}  3\textsc{p.abs}  and  straightway  3\textsc{s.erg}
\textsc{t.r}-call  3\textsc{p.abs}  \textsc{def.m} \\
\glt ‘But this older man saw them and straightway called them  …’ [PBWN-C-13 4.3]
\z  

\ea 
Paryo  ta  ayep  na  kanding,  pugya  uļa  gid  kanding  banwa  i, piro  namatian  no  daw  \textbf{makakita}  ka  ta  tai  din. \\\smallskip
\gll Paryo  ta  ayep  na  kanding,  pugya  uļa  gid  kanding  banwa  i, piro  na-mati-an  no  daw  \textbf{maka-kita}  ka  ta  tai  din. \\
same  \textsc{nabs}  animal  \textsc{lk}  goat  long.ago  \textsc{neg.r}  \textsc{int}  goat  town/country  \textsc{def.n} but  \textsc{a.hap.r}-hear-\textsc{apl}  2\textsc{s.erg}  and  \textsc{i.hap.ir}-see  2\textsc{s.abs}  \textsc{nabs}  scat  3\textsc{s.gen} \\
\glt ‘Like the animal goat, long ago the town really had no goats, but you heard them and you can see their scat.’ (This is a story about a place on Cagayancillo where amazing things happen.) [EFWE-T-05 3.8]
\z

\textit{Makakita} can also just mean to have the internal ability to see, as when a blind person is healed. \textit{Nakakita}, \textit{nakita}, and \textit{makita} are not used in this way. 

\ea 
Sikad  en  kon  na  kalipay  danen  tak  buļag  ya  \textbf{makakita}  en daw  buktot  ya  tanos  en.\\\smallskip
\gll Sikad  en  kon  na  ka-lipay  danen  tak  buļag  ya  \textbf{maka-kita}  en daw  buktot  ya  tanos  en.\\
very  \textsc{cm}  \textsc{hsy}  \textsc{lk}  \textsc{nr}-joy  3\textsc{s.gen}  because  blind  \textsc{def.f}  \textsc{i.hap.r}-see  \textsc{cm} and  hunchback  \textsc{def.f}  straight  \textsc{cm} \\
\glt ‘Their joy was great because the blind one can see and the hunchback is straight.’ [CBWN-C-15 5.18]
\z  

The root \textit{kita} with the meaning of ‘see’ only occurs in happenstantial modality. The causative stem, \textit{pakita}, may occur in dynamic modality with the meaning of ‘to show’. The reciprocal stem, \textit{kitaay} ‘to meet together’, may also appear in dynamic modality. There are thirty-eight examples of stems based on \textit{kita} functioning as predicates in dynamic modality in the corpus, twelve of which are causatives and twenty-six of which are reciprocals. Examples \REF{bkm:Ref121468636} and \REF{bkm:Ref121468640} illustrate the reciprocal usage:

\ea 
\label{bkm:Ref121468636}
\textbf{Gakitaay}  kay  isab  ta  mga  mama  nay  ya  na  duma. Lain  man  iran  baļay  daen. \\\smallskip
\gll \textbf{Ga-kita-ay}  kay  isab  ta  mga  mama  nay  ya  na  duma. Lain  man  iran  baļay  daen. \\
\textsc{i.r}-see-\textsc{rec}  1\textsc{p.excl.abs} again  \textsc{nabs}  \textsc{pl}  man 1\textsc{p.excl.gen}  \textsc{def.f}  \textsc{lk} companion different  also  3\textsc{p.gen}  house  3\textsc{p.gen} \\
\glt ‘We (the women) met together again with our male companions. As for them, their house was different (from ours).’ [EMWN-T-09  6.1]
\z 

\ea 
\label{bkm:Ref121468640}
Ta  isya  na  adlaw  \textbf{gakitaay}  Pwikan  i  daw Umang  ta  isya  na  baybay.\\
\gll Ta  isya  na  adlaw  \textbf{ga-kita-ay}  Pwikan  i  daw Umang  ta  isya  na  baybay.\\
\textsc{nabs}  one  \textsc{lk}  day  \textsc{i.r}-see-\textsc{rec}  tortoise  \textsc{def.n}  and
hermit.crab  \textsc{nabs}  one  \textsc{lk}  beach\\
\glt ‘One day Tortoise and Hermit Crab met each other on a beach.’ [DBWN-T-26 2.1]
\z  

Examples \REF{ex:hard-working} and \REF{ex:giveyou} illustrate the causative stem \textit{pakita} 'to show/demonstrate'  in dynamic modality:

\ea
\label{ex:hard-working}
Kyo na mga ginikanan, \textbf{pakita} nyo man ta batasan nyo na sigkat kaw … \\\smallskip
\gll Kyo	na	mga	ginikanan, \emptyset{}-\textbf{pa-kita}	nyo	man	ta	batasan	nyo	na	sigkat	kaw … \\
2\textsc{p.abs}	\textsc{lk}	\textsc{pl}	parent	\textsc{t.ir}-\textsc{caus}-see	2\textsc{p.erg}	also	\textsc{nabs}	behavior	2\textsc{p.gen}	\textsc{lk}	hard.working	2\textsc{p.abs} \\
\glt ‘You parents, also show (to your children) your behavior that you are being hard-working  …’ [RZWE-J-01 16.7]
\z
\ea
\label{ex:giveyou}
Isya pa may atag kanen ki kyo piro uļa din \textbf{pakita} ki kami  daw ino atag din ki kyo. \\\smallskip
\gll Isya	pa	may	atag	kanen	ki	kyo,	piro	uļa	din	\textbf{ …-pa-kita}	ki	kami	daw	ino	atag din	ki	kyo. \\
one	\textsc{inc}	\textsc{ext.in}	give	3\textsc{s.abs}	\textsc{obl.p}	2p	but	\textsc{neg.r}	3\textsc{s.erg}	\textsc{t.r-caus}-see	\textsc{obl.p}	1\textsc{p.excl}	if/when	what	give
3\textsc{s.erg}	\textsc{obl.p}	2p \\
\glt ‘One more (thing), he has something to give to you, but he did not show (it) to us what he will give you.’   [BCWL-T-10 2.5]
\z

Predicates describing intentional visual perception tend to take dynamic affixes with no stem-forming morphology required, for example, \textit{luag} ‘to watch’ and \textit{tan}{}-aw ‘to look at’. These verbs are sometimes interchangeable.

\ea
\textbf{Galuag}  kay  ta  kaoy  an  na  gakuwa  don. \\\smallskip
\gll \textbf{Ga-luag}  kay  ta  kaoy  an  na  ga-kuwa  don. \\
\textsc{i.r}-watch  1\textsc{p.excl.abs}  \textsc{nabs}  tree  \textsc{def.m}  \textsc{lk}  \textsc{i.r}-whatchmacallit  \textsc{d3loc} \\
\glt ‘We \textbf{were watching} the tree that was whatsamacallit there.’ [DBON-C-06 2.20]
\z
\ea
\textbf{Muag}  gid  yan  kani  ta  damag  di  daw  \textbf{manilag} di kani  nya  na  mga  ittaw. \\\smallskip
\gll \textbf{M-luag}  gid  yan  kani  ta  damag  di  daw  \textbf{m-panilag} di kani  nya  na  mga  ittaw. \\
\textsc{i.v.ir}-watch  \textsc{int}  \textsc{d2abs}  later  \textsc{nabs}  wake  \textsc{d1loc}  and  \textsc{i.v.ir}-observe
\textsc{d1loc} later  \textsc{d4adj}  \textsc{lk}  \textsc{pl} person \\
\glt `That one \textbf{will really watch/attend} later the wake here and those people \textbf{will be observing} here.’ [CBWN-C-22 13.9]
\z

\ea
\textbf{Paluag}  ko  daw  ano  na  sistema  tan  na  uļa natapat. \\\smallskip
\gll \textbf{Pa-luag}  ko  daw  ano  na  sistema  tan  na  uļa na-tapat. \\
\textsc{t.r}-watch  1\textsc{s.abs}  if/when  what  \textsc{lk}  system  \textsc{d3nabs}  \textsc{lk}  \textsc{neg.r} \textsc{a.hap.r}-mortally.wounded \\
\glt `I \textbf{inspected} it (the spear) (to see) what is its position that it (the wild pig) was not mortally wounded.’ (The speaker was spear hunting a wild pig and he hit the pig but the spear fell out of the pig and so he wanted to know why the pig was not mortally wounded.) [RCON-L-01 8.5]
\z

\ea
Ayep  na  sipit-sipit  tiset  daw  \textbf{luagen}  apang   may  abilidad. \\\smallskip
\gll Ayep  na  sipit-sipit  tiset  daw  \textbf{luag-en}  apang   may  abilidad. \\
animal  \textsc{lk}  scorpion  small  if/when  watch-\textsc{t.ir}  but  \textsc{ext.in}  ability \\
\glt ‘As for the animal scorpion it is small when looking at it but it has ability.’ [EFWE-T-04 26.1]
\z

\ea
Pag-uļog  ko  ta  pundo   ya,  \textbf{tan-awen}  ko,  may  bwaya  plang. \\\smallskip
\gll Pag-uļog  ko  ta  pundo   ya,  \textbf{tan-aw-en}  ko,  may  bwaya  plang. \\
\textsc{nr.act}-fall  1\textsc{s.gen}  \textsc{nabs}  anchor  \textsc{def.f}  look.at-\textsc{t.ir}  1\textsc{s.erg}  \textsc{ext.in}  crocodile  \textsc{surp} \\
\glt ‘When I dropped the anchor, when I looked at it, there was unexpectedly/surprisingly a crocodile!’ [SFWN-T-03 2.9].
\z

\ea
Masilaw  \textbf{tan-awen}  daw  kilat  nagakidlap. \\\smallskip
\gll Ma-silaw  \textbf{tan-aw-en}  daw  kilat  naga-kidlap.\footnotemark{} \\
\textsc{a.hap.ir}-blinding.bright  look.at-\textsc{t.ir}  if/when  lightning  \textsc{i.r}-flash \\
\footnotetext{The prefix \textit{naga}- in this example is code switching from \isi{Hiligaynon} (a Bisayan language). The Kagayanen form would be \textit{ga}-.}
\glt ‘Looking at lightning flashing is blinding bright.’ [JCWO-T-30 75.1]
\z

\ea
\textbf{Patan-aw}  ko  tiyo  ko  pario  nadlek. \\\smallskip
\gll \textbf{Pa-tan-aw}  ko  tiyo  ko  pario  na-adlek. \\
\textsc{t.r}-look.at  1\textsc{s.erg}  uncle  1\textsc{s.gen}  same  \textsc{a.hap.r}-afraid \\
\glt ‘I looked at my uncle who seemed afraid.’ [PMWN-T-02 2.9]
\z

\ea
\textbf{Patan-aw}  din;  \textbf{pasil-ing}  din  kon.  ``Bļawan  man  ni  a!" \\\smallskip
\gll \textbf{Pa-tan-aw}  din;  \textbf{pa-sil-ing}  din  kon.  Bļawan  man  ni  a! \\
\textsc{t.r}-look.at  3\textsc{s.erg}  \textsc{t.r}-peek.inside  3\textsc{s.erg}  \textsc{hsy}  gold  \textsc{emph}  \textsc{d1abs}  \textsc{inj} \\
\glt ‘He looked at it (the hole in the tree), (it is said). He peeked inside (and said) ``This really is gold!"’ [MBON-T-04 3.15]
\z

When the happenstantial affixes occur on these roots they express abilitative, optative, or opportunitive modality, or perfect aspect.

\ea
Opportunitive: \\
\textbf{Maluag}  ko  kani  sinayaw   an  daw  uļa  en  ubra. \\\smallskip
\gll \textbf{Ma-luag}  ko  kani  s<in>ayaw   an  daw  uļa  en  ubra. \\
\textsc{a.hap.ir}-watch  1\textsc{s.erg} later  <\textsc{nr.res}>dance  \textsc{def.m}  if/when  \textsc{neg.r} \textsc{cm}  work \\
\glt ‘I will \textbf{get to watch} the dance later if there is no more work.’
\z

\ea
\textbf{Nakaluag}  aren  ta  yon  na  sini. \\\smallskip
\gll \textbf{Naka-luag}  aren  ta  yon  na  sini. \\
\textsc{i.hap.r}-watch  1\textsc{s.abs}  \textsc{nabs}  \textsc{d3adj}  \textsc{lk}  movie \\
\glt ‘I already \textbf{had the opportunity to watch} that movie.’
\z

\newpage
\ea
Basi	\textbf{makaluag}	a	ta	sinayaw	kani	daw	uļa	ubra. \\\smallskip
\gll Basi	\textbf{maka-luag}	a	ta	s<in>ayaw	kani	daw	uļa	ubra. \\
perhaps  \textsc{i.hap.ir}-watch  1\textsc{s.abs}  \textsc{nabs}  \textsc{<nr.res>}dance  later  if/when  \textsc{neg.r} work \\
\glt ‘Perhaps I \textbf{will have the opportunity to watch} the dance later if there is no work.’
\z

\ea
\textbf{Matan-aw}  ko  kanen  an  na  marga. \\\smallskip
\gll \textbf{Ma-tan-aw}  ko  kanen  an  na  m-larga. \\
\textsc{a.hap.ir}-look.at  1\textsc{s.erg}  3\textsc{s.abs}  \textsc{def.m}  \textsc{lk}  \textsc{i.v.ir}-depart \\
\glt ‘I \textbf{will  have the opportunity to look at} him/her departing on a trip.’
\z

\ea
\textbf{Nakatan-aw}  aren  ta  yon  na  sini. \\\smallskip
\gll \textbf{Naka-tan-aw}  aren  ta  yon  na  sini. \\
\textsc{i.hap.r}-look.at  1\textsc{s.abs}  \textsc{nabs}  \textsc{d3adj}  \textsc{lk}  movie \\
\glt ‘I \textbf{had the opportunity to watch} that movie.’
\z

\ea
Abilitative: \\
Gapayong  kay  nang  aged  na  \textbf{makatan-aw}  kay  nang. \\\smallskip
\gll Ga-payong  kay  nang  aged  na  \textbf{maka-tan-aw}  kay  nang. \\
\textsc{i.r}-umbrela  1\textsc{p.excl.abs}  only/just  so.that  \textsc{lk}  \textsc{i.hap.ir}-look.at  1\textsc{p.excl.abs}  only/just \\
\glt ‘We just used an umbrella so that we \textbf{would be able to watch} the (show).’ [YBWL-T-03 5.9]
\z

\ea
Perfect aspect: \\
\textbf{Naluag}  nay  en  yon  na  sini. \\\smallskip
\gll \textbf{Na-luag}  nay  en  yon  na  sini. \\
\textsc{a.hap.r}-watch  1\textsc{p.excl.erg}  \textsc{cm}  \textsc{d3adj}  \textsc{lk}  movie \\
\glt ‘We \textbf{have already watched} that movie.’
\z

\ea
Pagsekeb  danen  uļa  danen  \textbf{natan-aw}  gantangan  ya daw  may  gapilit  na  bļawan. \\\smallskip
\gll Pag-sekeb  danen  uļa  danen  \textbf{na-tan-aw}  gantang-an  ya daw  may  ga-pilit  na  bļawan. \\
\textsc{nr.act}-volume.measure  3\textsc{p.gen}  \textsc{neg.r}  3\textsc{p.erg}  \textsc{a.hap.r}-look.at  3.quart-\textsc{nr}  \textsc{def.f}
if/when  \textsc{ext.in}  \textsc{i.r}-stick.on  \textsc{lk}  gold \\
\glt `When they measured (how much was the gold), they \textbf{had not looked at} the measurer if some gold was sticking (inside the measuring device).’ [CBWN-C-22 7.1]
\z

The root \textit{mati} ‘to hear/listen’ is of Class V, subclass Vb of non-volitional semantically transitive\is{semantic transitivity}\is{transitivity!semantic} roots, and Class VIII, volitional transitive roots, depending on the context. Both of these classes are those that take the applicative -\textit{an} in their basic transitive form (see \chapref{chap:verbclasses-1}, \sectref{sec:overviewofeightverbclasses}). In happenstantial modality with the applicative, the meaning tends to be unintentional: ‘to hear’ (\ref{bkm:Ref121468755} and \ref{bkm:Ref121468759}), and in dynamic modalities the meaning is more intentional: ‘to listen’ (\ref{bkm:Ref121468840} and \ref{bkm:Ref121468843}):

\ea
Transitive, happenstantial (with obligatory applicative): \\
\label{bkm:Ref121468755}
Pag-uļa  nang  lugay  \textbf{namatian}  nay  en   daguno  ta  iruplano. \\\smallskip
\gll Pag-uļa  nang  lugay  \textbf{na-mati-an}  nay  en   daguno  ta  iruplano. \\
\textsc{nr.act}-\textsc{neg.r}  only/just  long.time  \textsc{a.hap.r}-hear-\textsc{apl}  1\textsc{p.excl.erg}  \textsc{cm} drone.sound  \textsc{nabs}  airplane \\
\glt `Not long afterwards, we \textbf{heard} the drone of the airplane.’ [JCWN-T-22 7.2]
\z

\ea
\label{bkm:Ref121468759}
Na  nuga  kay  en  mga  alas  unsi  ta  kilem, \textbf{matian}   ko  umaw  ta  isya  na  mama  na  ingkantado. \\\smallskip
\gll Na  nuga\footnotemark  kay  en  mga  alas  unsi  ta  kilem, \textsc{ …-}\textbf{mati-an}\footnotemark   ko  umaw  ta  isya  na  mama  na  ingkantado. \\
\textsc{lk}  sleep  1\textsc{p.excl.abs}  \textsc{cm}  \textsc{pl}  o’clock  eleven  \textsc{nabs}  night \textsc{a.hap.r}-hear-\textsc{apl}  1\textsc{s.erg}  call  \textsc{nabs}  one  \textsc{lk}  male  \textsc{lk}  fairy \\
\footnotetext[2]{\textit{Nuga} in this example is the shortened form of \textit{tanuga}, which may also be pronounced \textit{tinuga} and \textit{tunuga}.}
\footnotetext[3]{See \chapref{chap:verbstructure}, \sectref{sec:omissionofprefixes} for description of conversational omission of inflectional prefixes.}
\glt `When we were asleep around eleven o’clock at night, I heard a call of a male fairy.’ [EDON-J-01 1.6]
\z

\ea
Transitive, dynamic (with obligatory applicative): \\
\label{bkm:Ref121468840}
\textbf{Pamatian}  no  ta  gaļaļa  an  ta  ikam  paryo   ta  isya  na  ittaw  na  gatype. \\\smallskip
\gll \textbf{Pa-mati-an}  no  ta  ga-ļaļa  an  ta  ikam  paryo   ta  isya  na  ittaw  na  ga-type. \\
\textsc{t.r}-hear-\textsc{apl}  2\textsc{s.erg}  \textsc{nabs}  \textsc{i.r}-weave  \textsc{def.m}  \textsc{nabs}  mat  same \textsc{nabs}  one  \textsc{lk}  person  \textsc{lk}  \textsc{i.r}-type \\
\glt  ‘When you \textbf{listen} to someone weaving a mat, it is the same as one person typing.’ [DBWE-T-18 6.1]
\z

\ea
\label{bkm:Ref121468843}
Lain  man  iran  na  ambaļ  na  daw  \textbf{matian}   no  matawa  ka  gid. \\\smallskip
\gll Lain  man  iran  na  ambaļ  na  daw  \emptyset{}-\textbf{mati-an}   no  ma-tawa  ka  gid. \\
dfferent  too  3\textsc{p.gen}  \textsc{lk}  say  \textsc{lk}  if/when  \textsc{t.ir}-hear-\textsc{apl} 2\textsc{s.erg}  \textsc{a.hap.ir}-laugh  2\textsc{s.abs}  \textsc{int} \\
\glt ‘Their language is so different that if you \textbf{listen} to it, you might really laugh.’ [VAWN-T-15 6.5]
\z

The root \textit{mati} may also appear in detransitive constructions, with a similar distinction between happenstantial and dynamic inflections:

\ea
Detransitive, happenstantial: \\
Kami  i  \textbf{nakamati}  kay  ta  isya  na  singgit   ta  isya  na  bai  na  ame  man  na  katagsa. \\\smallskip
\gll Kami  i  \textbf{naka-mati}  kay  ta  isya  na  singgit   ta  isya  na  bai  na  ame  man  na  katagsa. \\
1\textsc{p.excl.abs}  \textsc{def.n}  \textsc{i.hap.r}-hear  1\textsc{p.excl.abs}  \textsc{nabs}  one  \textsc{lk}  shout \textsc{nabs}  one  \textsc{lk}  woman  \textsc{lk}  1\textsc{p.excl.gen}  too  \textsc{lk}  cousin \\
\glt `As for us, we \textbf{happened to hear} one shout of a woman who was our cousin too.’ [MBON-T-02 2.3]
\z

\ea
Ta  pagpangaoy,  \textbf{nakamati}  kay  ta  kagi  ta   isya  na  yupan  na  yupan  i  na  gaambaļ  na  “Kikikiyaw  kikikiyaw.” \\\smallskip
\gll Ta  pag-pangaoy,  \textbf{naka-mati}  kay  ta  kagi  ta   isya  na  yupan  na  yupan  i  na  ga-ambaļ  na  “Kikikiyaw  kikikiyaw.” \\
\textsc{nabs}  \textsc{nr.act}-gather.firewood  \textsc{i.hap.r}-hear  1\textsc{p.excl.abs}  \textsc{nabs}  voice  \textsc{nabs} one  \textsc{lk}  bird  \textsc{lk}  bird  \textsc{def.n}  \textsc{lk}  \textsc{i.r}-say  \textsc{lk}  kikikiyaw  kikikiyaw \\
\glt `While gathering firewood, we \textbf{happened to hear} the voice of a bird which is the bird that says, “Kikikiyaw kikikiyaw.”' (This is the Philippine Oriole.) [MEWN-T-02 2.2]
\z

\ea
Basi  \textbf{makamati}  kanen  kani  daw  sugid  ko  ki  kaon. \\\smallskip
\gll Basi  \textbf{maka-mati}  kanen  kani  daw  sugid  ko  ki  kaon. \\
perhaps  \textsc{i.hap.ir}-hear  3\textsc{s.abs}  later  if/whe  tell  1\textsc{s.erg}  \textsc{obl.p}  2s \\
\glt ‘Perhaps s/he \textbf{will be able to hear} later if I tell you.’
\z

\ea
Detransitive, dynamic: \\
Kami  darwa  kay  Pedro  \textbf{gamati}  ta  drama  iran  Maria. \\\smallskip
\gll Kami  darwa  kay  Pedro  \textbf{ga-mati}  ta  drama  iran  Maria. \\
1\textsc{p.excl.abs}  two  1\textsc{p.excl.abs}  Pedro  \textsc{i.r}-hear  \textsc{nabs}  drama  3\textsc{p.gen}  Maria \\
\glt ‘Us two, Pedro and I, \textbf{listened to} a drama at Maria’s place.’ [RZWN-T-02 2.2]
\z

\ea
\label{bkm:Ref123365713}
\textbf{Magmati}  kay  ta  laygay  Mayor  naan  plasa. \\\smallskip
\gll \textbf{Mag-mati}  kay  ta  laygay  Mayor  naan  plasa. \\
\textsc{i.ir}-hear  1\textsc{p.excl.abs}  \textsc{nabs}  speech  Mayor  \textsc{spat.def}  plaza \\
\glt ‘We \textbf{will listen to} the Mayor’s speech on the plaza.’
\z

As with other perception verbs, happenstantial modality can express abilitative, optative, or opportunitive meanings: ‘may hear/listen’, ‘might hear/listen’, ‘possibly hear/listen’, ‘get to hear/listen’:

\ea
Happenstantial, opportunitive meaning, ‘get to hear’: \\
Daw  iling  man  tan  mga  ittaw  na  \textbf{nakamati}  ta  isturya na mananem  ki  ta  kaoy,  mananem  matuod piro  dili  terek  ta  iran  na  nakem  na  pagtanem. \\\smallskip
\gll Daw  iling  man  tan  mga  ittaw  na  \textbf{naka-mati}  ta  isturya na ma-ng-tanem  ki  ta  kaoy,  ma-ng-tanem  matuod piro  dili  terek  ta  iran  na  nakem  na  pag-tanem. \\
if/when  like  too  \textsc{d3nabs}  \textsc{pl}  person  \textsc{lk}  \textsc{i.hap.r}-hear  \textsc{nabs}  story \textsc{lk} \textsc{a.hap.ir-pl}-plant  1\textsc{p.incl.abs}  \textsc{nabs}  tree  \textsc{a.hap.ir-pl}-plant  true but  \textsc{neg.ir}  straight  \textsc{nabs}  3\textsc{p.gen}  \textsc{lk}  inner.self  \textsc{lk}  \textsc{nr.act}-plant \\
\glt `If the people who \textbf{got to hear} the story ``let’s plant trees" are like that too, (they) will truly plant but (it will) not be straight from their inner selves planting.’ (Straight from the inner self is an idiom for sincerity.) [SFOB-L-01 3.7]
\z

\ea
Yi  en  isturya  ta  tallo  i  buok  na  sidda  na  inyo  na   \textbf{namatian}  anduni. \\\smallskip
\gll Yi  en  isturya  ta  tallo  i  buok  na  sidda  na  inyo  na   \textbf{na-mati-an}  anduni. \\
\textsc{d1abs}  \textsc{cm}  story  \textsc{nabs}  three  \textsc{def.n}  piece  \textsc{lk}  fish  \textsc{lk}  2\textsc{p.gen}  \textsc{lk}
\textsc{a.hap.r}-hear-\textsc{apl}  now/today. \\
\glt `This is the story of the three fish which you \textbf{got to hear/had the opportunity to hear} now.’ [JCON-L-07 21.5]
\z

\ea
Transitive, Abilitative meaning, ‘able to hear’: \\
Ta,  \textbf{namatian}  din  kon  dya  sali  ta   gudgumod  mga  ittaw  an  ta  daļem  ya  ta  lungag. \\\smallskip
\gll Ta,  \textbf{na-mati-an}  din  kon  dya  sali  ta   gud-gumod  mga  ittaw  an  ta  daļem  ya  ta  lungag. \\
so  \textsc{a.hap.r}-hear-\textsc{apl}  3\textsc{s.erg}  \textsc{hsy}  \textsc{d4loc}  continually  \textsc{nabs}
\textsc{red}-mumble  \textsc{pl}  person  \textsc{def.m}  \textsc{nabs}  deep  \textsc{def.f}  \textsc{nabs}  hole \\
\glt `So, he could hear there they say people in the deep hole continually mumbling.’ [MBON-T-04 2.7]
\z

\ea
Keseg  ta  pagtangkeb  \textbf{mamatian}  ta  bilog  na  banwa. \\\smallskip
\gll Keseg  ta  pag-tangkeb  \textbf{ma-mati-an}  ta  bilog  na  banwa. \\
strength  \textsc{nabs}  \textsc{nr.act}-close  \textsc{a.hap.ir}-hear-\textsc{apl}  \textsc{nabs}  whole  \textsc{lk}  town \\
\glt ‘The strength of the closing can be heard in the whole town.’ (This is a story about two big rocks that close together when an official or government employee in the town dies.) [EFWE-T-05 4.4]
\z

The root \textit{mati} is unusual in that it allows a detransitive construction in the \textit{na}{}- and \textit{ma}{}- happenstantial forms (recall that for volitional roots, detransitive happenstantials require \textit{naka}{}- or \textit{maka}{}-). In such situations, the intentional meaning of the basic root is understood ‘to listen’, though the usual abilitative/opportunitive meanings are also part of the semantics of these constructions.

\ea
Detransitive \textit{na-} and \textit{ma}{}- intentional: \\
\textbf{Namati}  a  ta  balita  ya  kabii  kilem. \\\smallskip
\gll \textbf{Na-mati}  a  ta  balita  ya  kabii  kilem. \\
\textsc{a.hap.r}-hear  1\textsc{s.abs}  \textsc{nabs}  news  \textsc{def.f}  yesterday  night. \\
\glt ‘I \textbf{had the opportunity to hear/listen} to the news last night.’
\z

\ea
Manong  megbeng  ka  tak  \textbf{mamati}  ka anay   ta  miting    ta  PTA.\\\smallskip
\gll Manong  m-tegbeng  ka  tak  \textbf{ma-mati}  ka anay   ta  miting    ta  PTA.\\
older.brother  \textsc{i.v.ir}-go.downhill  2\textsc{s.abs}  because  \textsc{a.hap.ir}-hear  2\textsc{s.abs} first/for.a.while  \textsc{nabs}  meeting  \textsc{nabs}  PTA \\
\glt `Older brother, you will go downhill because you will \textbf{have the opppotunity to attend} the meeting of the PTA.’ [EDWN-T-04 2.2]
\z

\ea
\textbf{Mamati}  kay  ta  laygay  Mayor  naan  plasa. \\\smallskip
\gll \textbf{Ma-mati}  kay  ta  laygay  Mayor  naan  plasa. \\
\textsc{a.hap.ir}-hear  1\textsc{p.excl.abs}  \textsc{nabs}  speech  Mayor  \textsc{spat.def}  plaza \\
\glt ‘We \textbf{will get to listen to} the Mayor’s speech on the plaza.’ (c.f. \REF{bkm:Ref123365713} above)
\z

The excerpt in \REF{bkm:Ref121310939} illustrates four uses of the root \textit{mati}. First, \textit{namatian} is the plain happenstantial (unintentional) usage. Second, \textit{kamati} represents the external motivation usage. Third, \textit{gamati} is a dynamic (intentional) usage, and fourth, \textit{namatii} is the exclamatory form of the happenstantial usage (see \chapref{chap:stemformingprocesses}, \sectref{sec:exclamatory} on the exclamatory suffixes).

\ea
\label{bkm:Ref121310939}
Ta  ame  na  pag-isturya  may  \textbf{namatian}  \textbf{ake}  \textbf{na}   mangngod.  Ambaļ  din,  “Ma,  \textbf{kamati}  kaw  ta  kanta naan  ta  tyarok  galin?  \textbf{Gamati}  kay. Galineng  kay   gid  anay.  Naan  nay  dya  \textbf{namatii}  a kanta  na,  “Aliluya,  aliluya.”\\\smallskip
\gll Ta  ame  na  pag-isturya  may  \textbf{na-mati-an}  \textbf{ake}  \textbf{na}   mangngod.   …-Ambaļ  din,  “Ma,  \textbf{ka-mati}  kaw  ta  kanta naan  ta  tyarok  ga-alin?  \textbf{Ga-mati}  kay.  Ga-lineng  kay   gid  anay.  Naan  nay  dya  \textbf{na-mati-i}  a kanta  na,  “Aliluya,  aliluya.”\\
\textsc{nabs}  1\textsc{p.excl.gen}  \textsc{lk}  \textsc{nr.act}-story  \textsc{ext.in}  \textsc{a.hap.r}-hear-\textsc{apl}  1\textsc{s.gen}  \textsc{lk}
younger.sibling  \textsc{t.r}-say  3\textsc{s.erg}  Mother  \textsc{i.exm}-hear  2\textsc{p.abs}  \textsc{nabs}  song/sing \textsc{spat.def}  \textsc{nabs}  mast.pole  \textsc{i.r}-from  \textsc{i.r}-hear  1\textsc{p.excl.abs}  \textsc{i.r}-peace
1\textsc{p.excl.abs} \textsc{int}  first/for.a.while  \textsc{spat.def}  1\textsc{p.excl.erg}  \textsc{d4loc}  \textsc{a.hap.r}-hear-\textsc{xc.apl}  \textsc{inj}
  song/sing  \textsc{lk}  halleluja  halleluja \\
\glt `While we were talking, my younger sibling \textbf{heard} something. He said, “Mother, \textbf{can you hear} the singing coming from the mast pole?” \textbf{We listened}. We really became silent for a while. There we (surprisingly) \textbf{really heard} singing, “Halleluja halleluja.”' [VAWN-T-18 5.5-9]
\z

There are two roots that depict situations involving the sense of smell. These are  \textit{singngot} and \textit{plamao}. These roots are synonymous, but \textit{plamao} is a borrowing from \isi{Hiligaynon}. The root \textit{singngot} is more common, occuring four times in the corpus as an inflected verb, while \textit{plamao} occurs twice. Though these numbers are small, they accurately represent our impression of the relative frequency of these roots in conversation. With the dynamic affixes the meaning of both roots is intentional ‘to smell/sniff something’. With the happenstantial affixes the meaning is ‘to smell something unintentionally’. As with other sensory roots, the experiencer is the actor and the stimulus, if present, is the undergoer. The root \textit{plamao} always takes the applicative in its transitive uses (Class Vb for unintentional uses, and Class VIII for intentional uses). For \textit{singngot}, the applicative is optional (see further below).

\newpage
\ea
Transitive, dynamic with applicative: \\
\textbf{Paplamauan}  din  buļak  an. \\\smallskip
\gll \textbf{Pa-plamao-an}  din  buļak  an. \\
\textsc{t.r}-smell-\textsc{apl}  3\textsc{s.erg}  flower \textsc{def.m} \\
\glt ’S/he \textbf{sniffed} the flower.’
\z

\ea
\textbf{Pasingngutan}  din  anay  mga  pagkaan  an  bag-o  din kan-enen. \\\smallskip
\gll \textbf{Pa-singngot-an}  din  anay  mga  pagkaan  an  bag-o  din kan-en-en. \\
\textsc{t.r}-smell-\textsc{apl}  3\textsc{s.erg}  first/for.a.while  \textsc{pl}  food  \textsc{def.m}  before  3\textsc{s.erg} cooked.rice-\textsc{t.r} \\
\glt ‘S/he \textbf{sniffed} first the food before eating it.’
\z

\ea Transitive, dynamic without applicative: \\
\textbf{Pasingngot} ta mga bata ragbi an. \\\smallskip
\gll \textbf{Pa-singngot} ta mga bata ragbi an.\\
\textsc{t.r}-smell \textsc{erg} \textsc{pl} child rugbi \textsc{def.m} \\
\glt `The children sniffed the rugby.' \\\smallskip
*Paplamao ta bata ragbi an.
\z

The difference between \textit{pasingngutan} and \textit{pasingngot} is that without the applicative, \textit{pasingngot} implies sniffing up close, for example, holding something up to the nose and deeply sniffing or inhaling. With the applicative, \textit{pasingngutan} just implies normal sniffing the air and smelling something.

\ea
Detransitive, dynamic: \\
\textbf{Gaplamao}  en  mga  ittaw  i  daw  ambaļ  danen,  “May bui  man  i  na  ittaw  di  en  a!” \\\smallskip
\gll \textbf{Ga-plamao}  en  mga  ittaw  i  daw   …-ambaļ  danen,  “May bui  man  i  na  ittaw  di  en  a!” \\
\textsc{i.r}-smell  \textsc{cm}  \textsc{pl}  person  \textsc{def.n}  and  \textsc{t.r}-say  3\textsc{p.erg}  \textsc{ext.in}
live  \textsc{emph}  \textsc{att}  \textsc{lk}  person  \textsc{d1loc}  \textsc{cm}  \textsc{inj} \\
\glt `The people \textbf{sniffed} and they said, “There’s really a live person here now!”’ [CBWN-C-22 10.3]
\z

\ea
\textbf{Gasingngot}  bata  an  ta  ragbi. \\\smallskip
\gll \textbf{Ga-singngot}  bata  an  ta  ragbi. \\
\textsc{i.r}-smell  child  \textsc{def.m}  \textsc{nabs}  rugby \\
\glt ‘The child \textbf{was sniffing} some rugby.’
\z

\ea
Detransitive, happenstantial: \\
Tak  daw  naan  a  agi  ta  dapit  abagat  i \textbf{makasingngot}  kanen  ta  bao  ko. \\\smallskip
\gll Tak  daw  naan  a  agi  ta  dapit  abagat  i \textbf{maka-singngot}  kanen  ta  bao  ko. \\
because  if/when  \textsc{spat.def}  1\textsc{s.abs}  pass  \textsc{nabs}  direction  southwest  \textsc{def.n}
\textsc{i.hap.ir}-smell  3\textsc{s.abs}  \textsc{nabs}  odor  1\textsc{s.gen} \\
\glt `Because if I pass in the direction of south he \textbf{can smell} my odor.’ (This is a story about the speaker hunting a wild pig.) [RCON-L-01 2.6]
\z

\ea
Detransitive, happenstantial, external motivation: \\
\textbf{Kaplamao}  en  mga  ittaw  i  ta  ammot  ta  luto  ya. \\\smallskip
\gll \textbf{Ka-plamao}  en  mga  ittaw  i  ta  ammot  ta  luto  ya. \\
\textsc{i.exm}-smell  \textsc{cm}  \textsc{pl}  person  \textsc{def.n}  \textsc{nabs}  fragance  \textsc{nabs}  cook  \textsc{def.f} \\
\glt  ‘The people \textbf{were able to smell} the fragance of the cooking.’
\z 

As mentioned above, \textit{plamao} always takes the applicative suffix in a transitive frame, while \textit{singngot} may or may not occur with an applicative. In happenstantial modality, When the applicative does appear, the meaning is simple non-volitional ‘to sense something by smelling’, whereas without the applicative, the sense is more direct sniffing (as above in dynamic modality), in perfect aspect or abilitative/opportunitive:

\ea
Transitive, happenstantial, applicative = non-volitional ‘to sense by smelling’: \\
\textbf{Naplamauan}  din  inog  an  na  nangka. \\\smallskip
\gll \textbf{Na-plamau-an}  din  inog  an  na  nangka. \\
\textsc{a.hap.r}-smell-\textsc{apl}  3\textsc{s.erg}  ripe  \textsc{def.m}  \textsc{lk}  jackfruit \\
\glt ‘S/he \textbf{smelled} the ripe jackfruit.’ \\\smallskip
*Naplamo din inog an na nangka.  *Maplamao din inog an na Nangka.
\z

\ea
\textbf{Maplamauan}  ka  tatay  no  kani  na  bao  ka  ta  inemen. \\\smallskip
\gll \textbf{Ma-plamau-an}  ka  tatay  no  kani  na  bao  ka  ta  inem-en. \\
\textsc{a.hap.r}-smell-\textsc{apl}  2\textsc{s.abs}  father  2\textsc{s.gen}  later  \textsc{lk}  odor  2\textsc{s.abs}  \textsc{nabs}  drink-\textsc{nr} \\
\glt ’Your father \textbf{will smell} you later that you are smelly of alcohol.’
\z

\newpage
\ea
\textbf{Nasingngutan}  din  bao  ya  ta  durian  na  pakaan a  iya  na  arey. \\\smallskip
\gll \textbf{Na-singngut-an}  din  bao  ya  ta  durian  na  pa-kaan a  iya  na  arey. \\
\textsc{a.hap.r}-smell-\textsc{apl}  3\textsc{s.erg}  odor  \textsc{def.f}  \textsc{nabs}  durian.fruit  \textsc{lk}  \textsc{t.r}-eat \textsc{nabs}   3\textsc{s.gen}  \textsc{lk}  friend \\
\glt ‘S/he \textbf{smelled} the odor of the durian fruit that his/her friend was eating.’
\z

\ea
Transitive, happenstantial, no applicative = direct sniffing some specific thing, abilitative/opportunitive: \\
\textbf{Nasingngot}  ko  bao  ta  durian. \\\smallskip
\gll \textbf{Na-singngot}  ko  bao  ta  durian. \\
\textsc{a.hap.r}-smell  1\textsc{s.erg}  odor  \textsc{nabs}  durian.fruit \\
\glt ‘I \textbf{could sniff and smell} the odor of the durian fruit.'
\z

\ea
\textbf{Masingngot}  ko  bao  ta  durian. \\\smallskip
\gll \textbf{Ma-singngot}  ko  bao  ta  durian. \\
\textsc{a.hap.ir}-smell  1\textsc{s.erg}  odor  \textsc{nabs}  durian.fruit \\
\glt ‘I \textbf{will sniff and smell} the odor of the durian fruit.'
\z

\ea
Daw  mag-inog  en  gani  sikad  ammot  na  maski   adyo  ka  pa \textbf{nasingngot}  no  an. \\\smallskip
\gll Daw  mag-inog  en  gani  sikad  ammot  na  maski   adyo  ka  pa \textbf{na-singngot}  no  an. \\
if/when  \textsc{i.ir}-ripe  \textsc{cm}  truly  very  fragant  \textsc{lk}  even.if  far  2\textsc{s.abs}  \textsc{inc} \textsc{a.hap.r}-smell  2\textsc{s.erg}  \textsc{def.m} \\
\glt `When it (jackfruit) is becoming ripe, truly it is very fragrant that even if you are still far away it is what you \textbf{can sniff and smell}.’ [EFWE-T-04 3.5]
\z

\ea
Detransitive, happenstantial, hypothetical/polite meaning: \\
\textbf{Maplamao}  ka  ta  White Flower  aged  mamang lipeng uļo  no. \\\smallskip
\gll \textbf{Ma-plamao}  ka  ta  White Flower  aged  ma-kamang lipeng uļo  no. \\
\textsc{i.hap.ir}-smell  2\textsc{s.abs}  \textsc{nabs}  White Flower  so.that  \textsc{a.hap.ir}-get/remove  dizzy head  2\textsc{s.gen} \\
\glt ‘You \textbf{should sniff} the White Flower so that the dizziness of your head will be removed.’
\z

\newpage
\ea
\textbf{Masingngot}  ka  ta  White Flower  aged  mamang lipeng uļo  no. \\\smallskip
\gll \textbf{Ma-singngot}  ka  ta  White Flower  aged  ma-kamang lipeng uļo  no. \\
\textsc{i.hap.ir}-smell  2\textsc{s.abs}  \textsc{nabs}  White Flower  so.that \textsc{a.hap.ir}-get/remove  dizzy head  2\textsc{s.gen} \\
\glt `You \textbf{should sniff} the White Flower so that the dizziness of your head will be removed.’\\\smallskip
*Nasingngot ka ta White Flower.    *Naplamao ka ta White Flower.
\z

There is one root, \textit{lasa}, that depicts situations of tasting. It usually occurs with happenstantial affixes with a unintentional meaning to sense the taste of something. Like \textit{mati} ‘to hear/listen’, it belongs to Class V, subclass Vb, which is the class of non-volitional transitive roots that take the applicative -\textit{an} in their basic, transitive form. The actor is the experiencer, expressed in the ergative case, and the undergoer is the stimulus, expressed in the absolutive. There are no examples of this inflected form in the corpus, though it does occur in conversation.

\ea
\textbf{Nalasaan}  ko  luto  din  na  masin. \\\smallskip
\gll \textbf{Na-lasa-an}  ko  luto  din  na  masin. \\
\textsc{a.hap.r}-taste-\textsc{apl}  1\textsc{s.erg}  cook  3\textsc{s.gen}  \textsc{lk}  salty \\
\glt ‘I tasted his/her cooking that (it is) salty.’
\z

\ea
\textbf{Kalasa}  a  ta  tab-ang  tak  may  trangkaso  a. \\\smallskip
\gll \textbf{Ka-lasa}  a  ta  tab-ang  tak  may  trangkaso  a. \\
\textsc{i.exm}-taste  1\textsc{s.abs}  \textsc{nabs}  lacking.salt  because  \textsc{ext.in}  flu  1\textsc{s.abs} \\
\glt ‘Everything tastes bland to me because I have the flu.’ (lit. ‘I can taste lacking-salt because I have the flu.’)
\z

The common root that expresses the sense of touch is \textit{batyag} ‘to feel’. There are sixty (60) examples of \textit{batyag} used as an inflected verb in the corpus. The meanings of this root in its various possible forms are listed in \REF{ex:meaningsofbatyag}, along with their frequencies in the corpus. Corpus and conversational examples follow:

\ea
\label{ex:meaningsofbatyag}
\begin{enumerate}
\item Transitive frame, happenstantial, with applicative (\textit{nabatyagan/mabatyagan}): non-volitional ‘to physically, emotionally or internally sense or feel something in a permanent, deep, inward way’. This is the most frequent usage of \textit{batyag} in the corpus (30 examples of \textit{nabatyagan} and 8 examples of \textit{mabatyagan}). % Old Usage 2, new usage 1
\item Transitive frame, happenstantial, no applicative (\textit{nabatyag/mabatyag}): Simple non-volitional ‘to physically, emotionally or internally sense or feel something in a momentary, passing way’ (1 example of \textit{nabatyag} and 1 example of \textit{mabatyag} used in this way in the corpus). % Old Usage 1, new usage 2
\item Detransitive frame, happenstantial (\textit{nakabatyag/makabatyag}): Abilitative/opportunitive ‘to be able to physically, emotionally or internally sense or feel something’ (no examples of \textit{nakabatyag} and 3 examples of \textit{makabatyag} in the corpus). % Usage 3
\item Detransitive frame, external motivation (\textit{kabatyag}): Enablement ‘to be enabled to notice something’ (5 examples). % Usage 4
\item Detransitive frame, happenstantial (\textit{nabatyag/mabatyag}): Abilitative/opportunitive ‘to be able to take notice of something’, ‘happen to take notice of something’, stimulus often a complement clause (2 examples of \textit{nabatyag}, 1 example of \textit{mabatyag} used this way in the corpus). % Usage 5
\item Transitive frame, dynamic, no applicative (\textit{batyagen} only; Realis, \textit{pabatyag}, is ungrammatical for all speakers consulted): To take notice of something previously talked about (no examples in the corpus). % Usage 6 
\item Transitive frame, dynamic, applicative (\textit{pabatyagan/batyagan}): Volitional ‘to take notice/make note of something’ (no examples of \textit{pabatyagan}, 3 examples of \textit{batyagan}). % Usage 7
\item Detransitive frame, dynamic, no applicative (gabatyag/magbatyag): Volitional ‘to take notice/make note of something’, stimulus often a complement clause’ (6 examples of \textit{gabatyag}, no examples of \textit{magbatyag}). % Usage 8
\end{enumerate}
\z

When the verb \textit{batyag} is used in a transitive frame, the experiencer is in the ergative case and the stimulus, the item or sensation felt, is absolutive as in \REF{ex:becomingverybig} through \REF{bkm:Ref123389246}, and \REF{bkm:Ref122549938}. In an intransitive frame, the experiencer is absolutive and the stimulus, if present, is oblique (several examples from \ref{ex:willrun} through \ref{ex:asthmaattack}). Of the sixty examples of \textit{batyag} used as a inflected predicate in the corpus, thirty-eight are in happenstantial modality plus the applicative \nobreakdash-\textit{an} or its exclamatory equivalent -\textit{i} (two examples). Twenty are expressed in dynamic modality (usages 6 and 7). There are no examples of \textit{batyag} with the non-applicative exclamatory suffix -\textit{a}.  

\newpage
Examples \REF{ex:becomingverybig} and \REF{bkm:Ref122532854} illustrate the difference in meaning between the first two usages of \textit{batyag}. Both are from the same first person personal experience recounted by a fisherman whose boat overturned in rough seas. While fishing he noticed that he felt hungry, and uses \textit{nabatyag} \REF{ex:becomingverybig} (usage 2). Then, after surviving the accident and coming ashore, he was feeling intense hunger and tiredness and used \textit{nabatyagan}, shown in example \REF{bkm:Ref122532854} (usage 1): 

\ea
\label{ex:becomingverybig}
Happenstantial, transitive, realis non-applicative (usage 2): \\
Piro \textbf{nabatyag} ko en ļettem daw kaadlek tak baļed gadarko na gadarko. \\\smallskip
\gll Piro \textbf{na-batyag} ko en ļettem daw ka-adlek tak baļed ga-darko na ga-darko. \\
but \textsc{a.hap.r}-feel 1\textsc{s.erg} \textsc{cm} hunger and \textsc{nr}-fear because wave \textsc{i.r}-big.\textsc{pl} \textsc{lk} \textsc{i.r}-big.\textsc{pl} \\
\glt ‘But I felt hungry and afraid because the waves were becoming very big.’ [EFWN-T-11 9.2]
\z

\ea
Happenstantial, transitive, realis applicative (usage 1): \\
\label{bkm:Ref122532854}
Naan  aren  ta  baybay  Peek  \textbf{nabatyagan}  ko  en ļettem  daw  kakapoy  ta  ake  na  lawa. \\\smallskip
\gll Naan  aren  ta  baybay  Peek  \textbf{na-batyag-an}  ko  en ļettem  daw  ka-kapoy  ta  ake  na  lawa. \\
\textsc{spat.def}  1\textsc{s.abs}  \textsc{nabs}  beach  Peek  \textsc{a.hap.r}-feel-\textsc{apl}  1\textsc{s.erg}  \textsc{cm}
hunger  and  \textsc{nr}-tired  \textsc{nabs}  1\textsc{s.gen}  \textsc{lk}  body \\
\glt `When I was on the beach of Peek I \textbf{felt} hunger and exhaustion of my body.’ [EFWN-T-11 15.1]
\z

Example \REF{ex:wholebody} illustrates the applicative stem \textit{batyagan} in happenstantial irrealis  modality (usage 1):

\ea
\label{ex:wholebody}
Sakit  ta  kalingking  \textbf{mabatyagan}  ta  bilog  na  lawa. \\\smallskip
\gll Sakit  ta  kalingking  \textbf{ma-batyag-an}  ta  bilog  na  lawa. \\
pain  \textsc{nabs}  little.finger  \textsc{a.hap.ir}-feel-\textsc{apl}  \textsc{nabs}  whole  \textsc{lk}  body \\
\glt ‘Pain in the little finger \textbf{is felt} in the whole body.’  [VAWV-J-21 11.1]
\z

Examples \REF{bkm:Ref121399507} and \REF{bkm:Ref123389246} illustrate the applicative stem \textit{batyagan} inflected in happenstantial, realis modality, with a complement clause as the stimulus (usage 1):

\newpage

\ea
\label{bkm:Ref121399507}
Uļa  nang  lugay  \textbf{nabatyagan}  ko  na  gabaliskad en  dyip  na  sakayan  nay. \\\smallskip
\gll Uļa  nang  lugay  \textbf{na-batyag-an}  ko  na  ga-baliskad en  dyip  na  sakay-an  nay. \\
\textsc{neg.r}  only/just  long.time  \textsc{a.hap.r}-feel-\textsc{apl}  1\textsc{s.erg}  \textsc{lk}  \textsc{i.r}-upside.down
\textsc{cm}   jeep  \textsc{lk}  ride-\textsc{nr}  1\textsc{p.excl.gen} \\
\glt `Not a long time I \textbf{felt} that the jeep that we were riding was turning upside down.’ [PMWN-T-02  2.10]
\z

\ea
\label{bkm:Ref123389246}
Piro  \textbf{nabatyagan}  din  man  na  nagayya  kanen  i ta  nubya  din  … \\\smallskip
\gll Piro  \textbf{na-batyag-an}  din  man  na  na-gayya  kanen  i ta  nubya  din  … \\
but  \textsc{a.hap.r}-feel-\textsc{apl}  3\textsc{s.erg}  too  \textsc{lk}  \textsc{a.hap.r}-embarrass  3\textsc{s.abs}  \textsc{def.n}
\textsc{nabs}  girlfriend  3\textsc{s.gen} \\
\glt `But he \textbf{felt} too that he was embarrassed to his girlfriend  …’ [CBWN-C-12 3.9]
\z

The third usage of \textit{batyag} is in a detransitive frame in happenstantial modality. In the  context of example \REF{ex:willrun}, the meaning is ‘to be able to sense’: \\

\ea 
\label{ex:willrun}
Isip  ko  daw  bangkilan  ya  yo  bunuon  ko  na  marani a  pa  \textbf{makabatyag}  en  dumlaga  i.  Daw  \textbf{makabatyag} nan  mļagan  en  danen  an. \\\smallskip
\gll Isip  ko  daw  bangkil-an  ya  yo  buno-en  ko  na  m-parani a  pa  \textbf{maka-batyag}  en  dumlaga  i.  Daw  \textbf{maka-batyag} nan  m-dļagan  en  danen  an. \\
think  1\textsc{s.erg}  if/when  tusk-\textsc{nr}  \textsc{def.f}  \textsc{d}4\textsc{abs}  stab-\textsc{t.ir}  1\textsc{s.erg}  \textsc{lk}  \textsc{i.v.ir}-go.close
1\textsc{s.abs}  \textsc{inc}  \textsc{i.hap.ir}-feel  \textsc{cm}  gilt  \textsc{def.n}  if/when  \textsc{i.hap.ir}-feel
\textsc{d}3\textsc{abs}  \textsc{i.v.ir}-run  \textsc{cm}  3\textsc{p.abs}  \textsc{def.m} \\
\glt ‘I thought that if I stab the tusked wild pig when I get closer, the gilt wild pig \textbf{will sense} (I am there). If that one \textbf{senses} (me) they (the wild pigs there nearby) will run.’ [RCON-L-01 3.12-13]
\z  

% In the context of example \REF{ex:poundinggrain} (from conversation), a detransitive construction is used because the stimulus is a complement clause, and the actor of the complement (the one who pounds grain) must be the same as the absolutive of the main clause: 

\ea 
\label{ex:poundinggrain}
\textbf{Nakabatyag}  a  ta  ļettem  daw  kapoy  ta  asod. \\\smallskip
\gll \textbf{Naka-batyag}  a  ta  ļettem  daw  kapoy  ta  asod. \\
\textsc{i.hap.ir}-feel  1\textsc{s.abs}  \textsc{nasbs}  hungry  and  tired  \textsc{nabs}  pound.grain \\
\glt ‘I felt hunger and tiredness from pounding grain.’
\z  

The root \textit{batyag} is also used in the detransitive, external motivation form (usage 4):

\ea
Pagnegga ko, ambaļ ko ta sawa ko na daw kino \textbf{kabatyag} ta uras pukawen a nang. \\\smallskip
\gll Pag-negga ko,  …-ambaļ ko ta sawa ko na daw kino \textbf{ka-batyag} ta uras pukaw-en a nang. \\
\textsc{nr.act}-lie.down 1\textsc{s.gen} \textsc{t.r}-say 1\textsc{s.erg} \textsc{nabs} spouse 1\textsc{s.gen} \textsc{lk} if/when who \textsc{i.exm}-notice/feel \textsc{nabs} time/hour wake.up-\textsc{t.ir} 1\textsc{s.abs} only/just \\
\glt `When I lay down, I said to my spouse, ``Whoever notices the time, just wake me up."’  (The author was planning on going night fishing early in the morning before dawn and so he wanted to be woken up.) [JCWN-L-31 2.5]
\z

\ea
 …kis-a  \textbf{kabatyag}   a  man  ta  kapung-aw  ki  kyo. \\\smallskip
\gll  … kis-a  \textbf{ka-batyag}   a  man  ta  ka-pung-aw  ki  kyo. \\
{} sometimes  \textsc{i.exm}-feel  1\textsc{s.abs}  \textsc{emph}  \textsc{nabs}  \textsc{nr}-lonely  \textsc{obl.p}  2p \\
\glt ‘Sometmes I \textbf{feel} loneliness for you.’ [VBWL-T-07 3.3]
\z

The meaning of ‘to take notice’ can be inferred when \textit{batyag} is inflected with the intransitive happenstantial external motivation \textit{ka}{}-. In this context, the stimulus may be Referring Expression or a complement clause. In \REF{bkm:Ref121399400} the stimulus is the nominalization \textit{kakulian} ‘difficulty’:

\ea
\label{bkm:Ref121399400}
Tanan-tanan  ki  \textbf{kabatyag}  ta  kakulian  ta  Cagayancillo … \\\smallskip
\gll Tanan-tanan  ki  \textbf{ka-batyag}  ta  ka-kuli-an  ta  Cagayancillo … \\
\textsc{red}-all  1\textsc{p.incl.abs}  \textsc{i.exm}-feel/notice  \textsc{nabs}  \textsc{nr}-difficult-\textsc{nr}  \textsc{nabs}  Cagayancillo \\
\glt ‘Completely all of us \textbf{can take notice} of the great  difficulty on Cagayancillo  …’ [FDOE-T-01 3.3]
\z


When the root \textit{batyag} occurs in detransitive happenstantial modality with \textit{na}- or \textit{ma}-, the meaning is ‘to happen to take notice’ (usage 5). Three such examples occur in the corpus, including the following:

\ea
\label{bkm:Ref122533197}
Isya  na  sakay  danen  \textbf{nabatyag}  ta  yon  na  masakit. \\\smallskip
\gll Isya  na  sakay  danen  \textbf{na-batyag}  ta  yon  na  masakit. \\
one  \textsc{lk}  ride  3\textsc{p.gen}  \textsc{a.hap.r}-notice  \textsc{nabs}  \textsc{d3adj}  \textsc{lk}  sick \\
\glt  ‘A rider of theirs (on their boat) \textbf{took notice} of being sick.’ [JCWN-T-21 3.3]
\z

The last three usages of \textit{batyag} imply taking notice of something intentionally. As such, it is inflected with dynamic modality. As mentioned above, transitive realis inflection with no applicative is ungrammatical (*\textit{pabatyag}). The irrealis counterpart is acceptable, though rare \REF{ex:noticethepains} and \REF{ex:noticethechild} (usage 6). There are no clear examples of this usage in the corpus: 

\ea
\label{ex:noticethepains}
Daw manakem ka en, \textbf{batyagen} no mga sakit an. \\\smallskip
\gll Daw manakem ka en, \textbf{batyag=en} no mga sakit an. \\
If/when	older 2\textsc{s.abs} \textsc{cm} feel/notice-\textsc{t.ir}	2\textsc{s.erg} \textsc{pl} pain \textsc{def.m} \\
\glt `When you are older, you will notice the pains.'
\z

\ea
\label{ex:noticethechild}
Batyagen no kani \textbf{bata an}. Basi mabugtaw kanen an. \\
\gll Batyag-en no kani \textbf{bata an}. Basi ma-bugtaw kanen an. \\
feel-\textsc{t.ir} 2\textsc{s.erg} later child \textsc{def.m} maybe \textsc{a.hap.ir}-wake.up 3\textsc{s.abs} \textsc{def.m} \\
\glt `Notice the child. Maybe (s/he) will wake up later.'
\z

There are also no examples of the applicative used in dynamic, realis modality with this root in the corpus. Example \REF{bkm:Ref122549938} is from conversation (usage 7):

\ea
\label{bkm:Ref122549938}
\textbf{Pabatyagan}  a  din  daw  manaw  a. \\\smallskip
\gll \textbf{Pa-batyag-an}  a  din  daw  m-panaw  a. \\
\textsc{t.r}-noticel-\textsc{apl}  1\textsc{s.abs}  3\textsc{s.erg}  if/when  \textsc{i.v.ir}-go/walk  1\textsc{s.abs} \\
\glt ‘S/he \textbf{took notice} of me whenever I leave.’
\z

\ea
\label{ex:areinjected}
Dynamic, detransitive (usage 8): \\
Mga  ittaw  na  uļa  pa  \textbf{gabatyag}  ta  masakit  yon  nang paindyiksyunan. \\\smallskip
\gll Mga  ittaw  na  uļa  pa  \textbf{ga-batyag}  ta  masakit  yon  nang pa-indyiksyon-an. \\
\textsc{pl}  person  \textsc{lk}  \textsc{neg.r}  \textsc{inc}  \textsc{i.r}-notice  \textsc{nabs}  sickness  \textsc{d3abs}  only/just \textsc{t.r}-inject-\textsc{apl} \\
\glt `The people who are not yet \textbf{taking notice} of the sickness, those only are injected.’ (This is about a cholera epidemic on Cagayancillo in 1987.)  [JCWN-T-21 8.7]
\z

In \REF{ex:withthebamboo}, the actor was pierced in the stomach by three bamboo sticks, but he was so fat he could not see them. He thought he had a stomach ache:
\ea
\label{ex:withthebamboo}
Piro  na  suntok  tallo  ya  na  kwayan  uļa  gid  \textbf{gabatyag}. \\\smallskip
\gll Piro  na  suntok  tallo  ya  na  kwayan  uļa  gid  \textbf{ga-batyag}. \\
but  \textsc{lk}  pierce  three  \textsc{def.f}  \textsc{lk}  bamboo  \textsc{neg.r}  \textsc{int}  \textsc{i.r}-notice \\
\glt ‘But when the three bamboos pierced (him), (he) did not \textbf{notice them}.’ [PBON-T-01 3.7]
\z

\ea
\label{ex:asthmaattack}
Pirmi  a  nang  \textbf{gabatyag}  ta  kiien  daw  paryo a  man  ta  kindisen. \\\smallskip
\gll Pirmi  a  nang  \textbf{ga-batyag}  ta  ka-ii-en  daw  paryo a  man  ta  ka-indis-en. \\
always  1\textsc{s.abs}  only/just  \textsc{i.r}-notice  \textsc{nabs}  \textsc{nr}-urine-\textsc{nr}  and  same 1\textsc{s.abs}  too  \textsc{nabs}  \textsc{nr}-defecate-\textsc{nr} \\
\glt `I always was \textbf{noticing} the urge to urinate and it was like I had the urge to defecate too.’ (This is the story of a man having a serious asthma attack.) [JCWN-T-22 3.10]
\z


\section{Emotion}
\label{sec:emotion}

Many situations of emotion require only one participant, an experiencer, someone who experiences the emotion, for example, \textit{I’m happy}, \textit{they are angry}. Often, however, there is also a stimulus of the emotion that plays a central role in the situation, as in \REF{ex:weswim}:

\ea
\label{ex:weswim}
\begin{tabbing}
\hspace{3.5cm} \= \kill 
\textit{She is angry at me}. \> Stimulus = \textit{me} \\
\textit{We enjoy swimming}. \> Stimulus = \textit{we swim} \\
\textit{I’m happy to be here}. \> Stimulus = \textit{I am here}
\end{tabbing}
\z

In Kagayanen emotion predicates are different from predicates of perception and cognition in that the experiencer is not always presented as an actor. Often the experiencer is the absolutive case undergoer of an intransitive construction. If there is a stimulus, it is presented in the oblique case. In this respect emotion situations are treated similarly to non-volitional, undergoer-oriented, intransitive situations.

As with other non-volitional situations, predicates that express emotion most commonly occur in  happenstantial modality (examples \ref{ex:joyful} through \ref{ex:saidthegiant}):

\ea
\label{ex:joyful}
Intransitive, happenstantial, realis: \\
Daw  \textbf{nalipay}  kay  gid. \\\smallskip
\gll Daw  \textbf{na-lipay}  kay  gid. \\
and  \textsc{a.hap.r}-joy  1\textsc{p.excl.abs}  \textsc{int} \\
\glt ‘And we \textbf{were really joyful}.’ [SBWL-C-01 5.4]
\z
\ea
\label{ex:bumpedinto}
Intransitive, happenstantial, realis with oblique stimulus: \\
\textbf{Nadlek}  a  ta  bļangay  ya  na  masanggaan a. \\\smallskip
\gll \textbf{Na-adlek}  a  ta  bļangay  ya  na  ma-sangga-an a. \\
na-adlek \\
\textsc{a.hap.r}-fear  1\textsc{s.abs}  \textsc{nabs}  2.mast.boat  \textsc{def.f}  \textsc{lk}  \textsc{a.hap.ir}-bump.into-\textsc{apl} 1\textsc{s.abs} \\
\glt ‘I was afraid of the two-masted boat that I would be bumped into.’ [JCON-L-07 8.3]
\z
\ea
\label{ex:saidthegiant}
Intransitive, happenstantial, irrealis with oblique stimulus: \\
Piro  bisan  ino  ka  pa  kabakod  na  ittaw  ka  dili  a \textbf{madlek}  ki  kaon  tak  ake  na  baļay  ni,”  ambaļ  ta ganti  ya.\\\smallskip
\gll Piro  bisan  ino  ka  pa  ka-bakod  na  ittaw  ka  dili  a \textbf{ma-adlek}  ki  kaon  tak  ake  na  baļay  ni,”   …ambaļ  ta ganti  ya.\\
but  any  what  2\textsc{s.abs}  \textsc{inc}  \textsc{nr}-big  \textsc{lk}  person  2\textsc{s.abs}  \textsc{neg.ir}  1\textsc{s.abs}
\textsc{a.hap.ir}-fear  \textsc{obl.p}  2s  because  1\textsc{s.gen}  \textsc{lk}  house  \textsc{d1abs}  \textsc{t.r}-say  \textsc{nabs} giant  \textsc{def.f} \\
\glt ‘“But no matter how much bigger a person you are, I \textbf{will not be afraid} of you because this is my house,” said the giant.’ [CBWN-C-10 7.11]
\z

Dynamic modality in an intransitive frame usually expresses an \textsc{inchoative}\is{inchoative} sense – entering into the emotional state described by the root (examples \ref{ex:mybreathing} through \ref{ex:afraidtoomuch}). 

\ea
\label{ex:mybreathing}
Intransitive, dynamic, realis  (inchoative): \\
 … uļa a pa gapanaik ta lunday tak \textbf{gakulba} pa ake na ginawa. \\\smallskip
\gll  … uļa a pa ga-panaik ta lunday tak \textbf{ga}-\textbf{kulba} pa ake na ginawa.\footnotemark{} \\
{} \textsc{neg.r}    1\textsc{s.abs} \textsc{inc} \textsc{i.ir}-go.up \textsc{nabs} outrigger.canoe because \textsc{i.r}-frightened \textsc{inc} 1\textsc{s.gen} \textsc{nabs} breath \\
\footnotetext{The breath is a euphemistic expression referring to a person affected by an adverse condition. For example, the following statement may be used to describe a death: \textit{Nabugto iya na ginawa}, literally `his/her breath broke'. When someone is not feeling well, one might say:  \textit{Lain iya na ginawa}, literally `his/her breath is bad'.}
\glt ` … I did not yet get up in the outrigger canoe because I was feeling frightened (lit. My breath was becoming frightened).' [EDWN-T-05 1.21]
\z

\newpage
\ea
Intransitive, dynamic, irrealis  (inchoative): \\
\label{ex:weatherisbad}
\textbf{Magkulba}  ka  daw  lain  timpo. \\\smallskip
\gll \textbf{Mag-kulba}  ka  daw  lain  timpo. \\
\textsc{i.ir}-frighten  2\textsc{s.abs}  if/when  bad/different  weather/seaon \\
\glt ‘You will \textbf{become frightened} when the weather is bad.’ \\
\z

\ea
\label{bkm:Ref122550199} \label{ex:afraidtoomuch}
Dili  ka  en  \textbf{mag-adlek}  ta  miad. \\\smallskip
\gll Dili  ka  en  \textbf{mag-adlek}  ta  miad. \\
\textsc{neg.ir}  2\textsc{s.abs}  \textsc{cm}  \textsc{i.ir}-fear  \textsc{nabs}  well \\
\glt ‘Do not \textbf{become afraid} too much.’ (In the context the author says that whatever you hear do not let yourself be startled. But first watch and investigate it. Do not become afraid.) [JCON –L-07 22.3]
\z

In a transitive or detransitive frame, emotion predicates in dynamic modality express the sense of treating the absolutive participant with the named emotion (examples \ref{ex:scoldedme} through \ref{ex:notcorrect}). Finally, when appearing in a transitive frame (either in happenstantial or dynamic modalities), emotion predicates strongly tend to take the applicative -\textit{an} suffix (examples \ref{ex:scoldedme} and \ref{ex:withfear}). This is reasonable since the target of an emotion is not a prototypical Patient---it is only invisibly affected by the emotional reaction directed toward it:

\ea
Transitive, dynamic, realis, with obligatory applicative: \\
\label{ex:scoldedme}
Pagilekan a din. \\\smallskip
\gll Pa-gilek-an a din. \\
\textsc{t.r}-angry-\textsc{apl} \\
\glt ‘S/he scolded me.’
\z

\ea
\label{ex:withfear}
Paadlekan kanen ta mga ittaw. \\\smallskip
\gll Pa-adlek-an kanen ta mga ittaw. \\
\textsc{t.r}-fear-\textsc{apl} 3\textsc{s.abs} \textsc{nabs} \textsc{pl} person \\
\glt ‘People treat him/her with fear.’
\z

\ea
\label{ex:myboss}
Detransitive dynamic realis: \\
Ta gagilek a i ta amo ko ya. \\\smallskip
\gll Ta ga-gilek a i ta amo ko ya. \\
\textsc{inj} \textsc{i.r}-angry 1\textsc{s.abs} \textsc{def.n} \textsc{nabs} boss 1\textsc{s.gen} \textsc{def.f} \\
\glt ‘So I became angry with my boss.’ [CBWN-C-18 3.3] 
\z

\newpage
\ea
\label{ex:notcorrect}
\textbf{Gagilek}  a  man  ta  ake  na  tatay  tak  dili  igo  iya na  pagdumala  ta  mga  ittawan  din. \\\smallskip
\gll \textbf{Ga-gilek}  a  man  ta  ake  na  tatay  tak  dili  igo  iya na  pag-dumala  ta  mga  ittaw-an  din. \\
\textsc{i.r}-angry  1\textsc{s.abs}  also  \textsc{nabs}  1\textsc{s.gen}  \textsc{lk}  father  because  \textsc{neg.ir}  correct  3\textsc{s.gen} \textsc{lk}  \textsc{nr.act}-rule  \textsc{nabs}  \textsc{pl}  person-\textsc{nr}  3\textsc{s.gen} \\
\glt `I \textbf{became angry} also at my father because his ruling of his people was not correct.’ [BEWN-T-01 5.7]
\z

The only emotion verb that takes dynamic, transitive inflection with no applicative morphology is \textit{palangga} ‘to have affection for a family member’. Therefore this verb is not grammatically in the class of emotion verbs, though semantically it does seem to express an emotion. It acts like common inherently transitive roots in that the experiencer is presented as the actor, and the stimulus as the undergoer.

\ea
Transitive, dynamic realis: \\
\textbf{Papalangga}  gid  ta  mag-asawa  iran  i  na  bata.  Bisan  ino na  mga  ampangan  an  paatag  danen  ta  iran  na  bata. \\\smallskip
\gll \textbf{Pa-palangga}  gid  ta  mag-asawa  iran  i  na  bata.  Bisan  ino na  mga  ampang-an  an  pa-atag  danen  ta  iran  na  bata. \\
\textsc{t.r}-have.affection  \textsc{int}  \textsc{nabs}  \textsc{rel}-spouse  3\textsc{p.gen}  \textsc{def.n}  \textsc{lk}  child  even  what
\textsc{lk}  \textsc{pl}  play-\textsc{nr}  \textsc{def.m}  \textsc{t.r}-give  3\textsc{p.erg}  \textsc{nabs}  3\textsc{p.gen}  \textsc{lk}  child \\
\glt `The couple really \textbf{had affection for} their child. Whatever toys, they gave (them) to their child.’ [CBWN-C-14 2.6-7]
\z

\ea
Transitive, dynamic irrealis: \\
Gani  una  ta  tanan  ki  kyo  mga  atong  daw  mga  akay \textbf{palanggaen}  ta  Ginuo  … \\\smallskip
\gll Gani  una  ta  tanan  ki  kyo  mga  atong  daw  mga  akay \textbf{palangga-en}  ta  Ginuo … \\
so  first  \textsc{nabs}  all  \textsc{obl.p}  2p  \textsc{pl}  boy  and  \textsc{pl}  girl
have.affection-\textsc{t.ir}  1\textsc{p.incl.abs}  Lord \\
\glt `First of all for you boys and girls, let us \textbf{have affection for} the Lord  …’ [EFOB-C-01 2.4]
\z

When \textit{adlek} ‘to be afraid’, and several other emotion roots, occur in an applicative construction, the experiencer is in the ergative case (when present) and the stimulus in the absolutive. There are no examples of the root \textit{adlek} in this usage in the corpus, but they do occur in conversation:

\ea
Transitive, happenstantial, applicative: \\
\textbf{Nadlekan}  kanen  ta  mga  ittaw. \\\smallskip
\gll \textbf{Na-adlek-an}  kanen  ta  mga  ittaw. \\
\textsc{a.hap.r}-fear-\textsc{apl}  3\textsc{s.abs}  \textsc{nabs}  \textsc{pl}  ittaw \\
\glt ‘People \textbf{are afraid} of him/her.’
\z

\ea
\textbf{Nadlekan}  gid  kanen  tak  sikad  magilek. \\\smallskip
\gll \textbf{Na-adlek-an}  gid  kanen  tak  sikad  ma-gilek. \\
\textsc{a.hap.r}-fear-\textsc{apl}  \textsc{int}  3\textsc{s.abs}  because  very  \textsc{a.hap.ir}-angry \\
\glt ‘S/he is \textbf{really feared} because s/he always gets angry.’
\z

Roots of the \textit{adlek} class can also occur in dynamic modality with an applicative affix. Again, the experiencer of the emotion is ergative and the stimulus is absolutive. The meaning of \textit{adlek} in this construction is ‘to act with fear toward X’ or ‘to treat X in a fearful way’. The following are examples of such roots from conversations:

\ea
Contrast between dynamic, and happenstantial transitive: \\
Yi  na  manakem  \textbf{paadlekan}  gid  ta  mga  ittaw … \\\smallskip
\gll Yi  na  manakem  \textbf{pa-adlek-an}  gid  ta  mga  ittaw … \\
\textsc{d1adj}  \textsc{lk}  older  \textsc{t.r}-fear-\textsc{apl}  \textsc{int}  \textsc{nabs}  \textsc{pl}  person \\
\glt ‘This older person people \textbf{treat with fear} …’ [JCWN-T-26 3.6]
\z

\ea
    \ea
    \textbf{Naangepan}  ta  nakem  ta  bata  amay  din  an. \\\smallskip
\gll \textbf{Na-angep-an}  ta  nakem  ta  bata  amay  din  an. \\
    \textsc{a.hap.r}-fond.of-\textsc{apl}  \textsc{nabs}  inner.self  \textsc{nabs}  child  father  3\textsc{s.gen}  \textsc{def.m} \\
    \glt ‘The inner self of the child \textbf{is fond of} (or favors) his/her father.’ \\
    \ex
    \textbf{Paangepan}  ta  nakem  ta  bata  amay  din  an. \\\smallskip
\gll \textbf{Pa-angep-an}  ta  nakem  ta  bata  amay  din  an. \\
    \textsc{t.r}-fond.of-\textsc{apl}  \textsc{nabs}  inner.self  \textsc{nabs}  child  father  3\textsc{s.gen}  \textsc{def.m} \\
    \glt ‘The inner self of the child intentionally makes himself/herself fond of (or favor) his/her father.’
    \z
\z

\newpage
\ea
    \ea 
    \textbf{Nainggitan}  kanen  ta  duma  na  mga  maistro tak  kanen  nang  kapasar  ta  board  exam. \\\smallskip
\gll \textbf{Na-inggit-an}  kanen  ta  duma  na  mga  maistro tak  kanen  nang  ka-pasar  ta  board  exam. \\
    \textsc{a.hap.r}-jealous-\textsc{apl}  3\textsc{s.abs}  \textsc{nabs}  companion  \textsc{lk}  \textsc{pl}  teacher
    because  3\textsc{s.abs} just \textsc{i.exm}-pass  \textsc{nabs}  board  exam \\
    \glt ‘The other teachers \textbf{are jealous of} him/her because only s/he was able to pass the board exam.’ \\
    \ex
    \textbf{Painggitan}  din  bai  an  tak  mas  gwapa  ki  kanen. \\\smallskip
\gll \textbf{Pa-inggit-an}  din  bai  an  tak  mas  gwapa  ki  kanen. \\
    \textsc{t.r}-jealous-\textsc{apl}  3\textsc{s.erg}  woman  \textsc{def.m}  because  more  attractive  \textsc{obl.p}  3s \\
    \glt ‘She \textbf{acts in a jealous way} to the woman because (she is) more attractive than her.’
    \z
\z

\ea
    \ea
    \textbf{Naimunan}  din  arey  din  an   tak  sigi  duma  ta  nubyo  din. \\\smallskip
\gll \textbf{Na-imon-an}  din  arey  din  an   tak  sigi  duma  ta  nubyo  din. \\
    \textsc{a.hap.r}-envious-\textsc{apl}  3\textsc{s.erg} friend  3\textsc{s.gen}  \textsc{def.m}
    because  continually  with  \textsc{nabs}  boyfriend  3\textsc{s.gen} \\
    \glt ‘She \textbf{is envious of} her friend because (she) keeps accompanying her boyfriend.’ \\
    \ex
    \textbf{Paimunan}  din  bai  ya  tak  sigi  isturya  daw  nubyo  din. \\\smallskip
\gll \textbf{Pa-imon-an}  din  bai  ya  tak  sigi  isturya  daw  nubyo  din. \\
    \textsc{t.r}-envious-\textsc{apl}  3\textsc{s.erg}  woman  \textsc{def.f} because  continually  talk  and  boyfriend  3\textsc{s.gen} \\
    \glt ‘She \textbf{acts enviously toward} the woman because (she) and her boyfriend keep on talking.’
    \z
\z

\ea
    \ea
    \textbf{Naneļseļan}  ko  en  tanan  na  saļa  ko. \\\smallskip
\gll \textbf{Na-neļseļ-an}  ko  en  tanan  na  saļa  ko. \\
    \textsc{a.hap.r}-regret-\textsc{apl}  1\textsc{s.erg}  \textsc{cm}  all  \textsc{lk}  sin  1\textsc{s.gen} \\
    \glt ‘I \textbf{feel regret} for all my sins.’ (This is more about how one feels rather than what they actually do.) \\
    \ex
    \textbf{Paneļseļan}  ko  en  tanan  na  saļa  ko. \\\smallskip
\gll \textbf{Pa-neļseļ-an}  ko  en  tanan  na  saļa  ko. \\
    \textsc{t.r}-regret-\textsc{apl}  1\textsc{s.erg}  \textsc{cm}  all  \textsc{lk}  sin  1\textsc{s.gen} \\
    \glt ‘I \textbf{repented of} all my sins.’ (This implies that they will not do it again.)
    \z
\z
\ea
    \ea
    \textbf{Nagilekan}  kay  ta  ame  na  maistra. \\\smallskip
\gll \textbf{Na-gilek-an}  kay  ta  ame  na  maistra. \\
    \textsc{a.hap.r}-angry-\textsc{apl}  1\textsc{p.excl.abs}  \textsc{nabs}  1\textsc{p.excl.gen}  \textsc{lk}  teacher \\
    \glt ‘Our teacher \textbf{was angry with} us.’ \\
    \ex
    \textbf{Pagilekan}  kay  ta  ame  na  maistra. \\\smallskip
\gll \textbf{Pa-gilek-an}  kay  ta  ame  na  maistra. \\
    \textsc{t.r}-angry-\textsc{apl}  1\textsc{p.excl.abs}  \textsc{nabs}  1\textsc{p.excl.gen}  \textsc{lk}  teacher \\
    \glt ‘Our teacher \textbf{scolded} us.’
    \z
\z
\ea
    \ea
    \textbf{Naluuyan}  kanen  ta  mga  ittaw. \\\smallskip
\gll \textbf{Na-luoy-an}  kanen  ta  mga  ittaw. \\
    \textsc{a.hap.r}-pity-\textsc{apl}  3\textsc{p.abs}  \textsc{nabs}  \textsc{pl}  person \\
    \glt ‘People \textbf{have pity on} him/her.’ \\
    \ex
    \textbf{Paluuyan}  kanen  ta  mga  ittaw. \\\smallskip
\gll \textbf{Pa-luoy-an}  kanen  ta  mga  ittaw. \\
    \textsc{t.r}-pity-\textsc{apl}  3\textsc{p.abs}  \textsc{nabs}  \textsc{pl}  person \\
    \glt ‘People \textbf{act with pity} on him/her.’
    \z
\z

The root \textit{demet} expresses the transitive notions of ‘to begrudge X’, ‘to hate X’, or ‘to be angry with X’. There are no examples of this root in the corpus.

\ea 
Transitive, happenstantial plus obligatory applicative (stimulus = absolutive): \\
\textbf{Nademetan}  a  gid  ta  mga  ayam  daw  maglambay a  naan  ta  silong  danen. \\\smallskip
\gll \textbf{Na-demet-an}  a  gid  ta  mga  ayam  daw  mag-lambay a  naan  ta  silong  danen. \\
\textsc{a.hap.r}-brgrudge/hate-\textsc{apl}  1\textsc{s.abs}  \textsc{int}  \textsc{nabs}  \textsc{pl}  dog  if/when  \textsc{i.ir}-pass.by
1\textsc{s.abs}  \textsc{spat.def}  \textsc{nabs}  yard  3\textsc{p.gen} \\
\glt `The dogs really \textbf{get angry at} me when I go past their yard.’ 
\z  

\newpage
\ea 
Transitive, dynamic plus obligatory applicative (stimulus = absolutive): \\
\textbf{Pademetan}  kanen  ta  mga  ittaw  naan  ta  ame na  lugar.\\\smallskip
\gll \textbf{Pa-demet-an}  kanen  ta  mga  ittaw  naan  ta  ame na  lugar.\\
\textsc{t.r}-grudge/hate-\textsc{apl}  3\textsc{s.abs}  \textsc{nabs}  \textsc{pl}  person  \textsc{spat.def}  \textsc{nabs}  1\textsc{p.excl.gen} \textsc{lk}  place \\
\glt ‘People \textbf{begrudge/hate} him/her in our place.’
\z

Other emotion roots, including \textit{kulba} ‘to feel frightened/nervous’, \textit{sebe} ‘to feel sad’, \textit{sadya} ‘to enjoy/have fun’, \textit{lipay} ’to be joyful’, and \textit{imon} ‘to be envious because another gets more attention’ also take the applicative as their basic form (Class VIII). With these verbs, the experiencer is absolutive and the stimulus is in an adverbial or reason clause.

\ea
Intransitive, happenstantial, stimulus is an adverbial clause: \\
\textbf{Nakulbaan}  a  gid  tak  sigi  eyeg-eyeg iruplano  an  daw  maagi  ta  sikad  dakmeļ  na  panganod. \\\smallskip
\gll \textbf{Na-kulba-an}  a  gid  tak  sigi  eyeg-eyeg iruplano  an  daw  ma-agi  ta  sikad  dakmeļ  na  panganod. \\
\textsc{a.hap.r}-frighten-\textsc{apl}  1\textsc{s.abs}  \textsc{int}  because  continually  \textsc{red}-shake
airplane  \textsc{def.m}  if/when  \textsc{a.hap.ir}-pass  \textsc{nabs}  very  thick  \textsc{lk}  cloud \\
\glt `I \textbf{was really frightened} because the airplane kept shaking when passing through very thick clouds.’ [VAWN-T-15 2.4]
\z

\ea
Intransitive, dynamic (inchoative): \\
\textbf{Magkulba}  ka  daw  lain  timpo. \\\smallskip
\gll \textbf{Mag-kulba}  ka  daw  lain  timpo. \\
\textsc{i.ir}-frighten  2\textsc{s.abs}  if/when  bad/different  weather/seaon \\
\glt ‘You will \textbf{become frightened} when the weather is bad.’ \\\smallskip
*Nakulba a gid. (‘I am really frightened.’)
\z

The emotions of sadness, happiness and joy are expressed by roots which frequently occur with the stem-forming prefixes \textit{ng}{}- ‘pluraction’ and \textit{ka}{}- ‘verbalizer’, resulting in the prefix  complex \textit{nga-}. These roots are also of the class that takes the adjectivizing prefix \textit{ma}- when functioning as a modifier in a Referring Phrase, or in a non-verbal predicate. \tabref{tab:occurencesofrootsofsadnessandjoyinthecorpus} gives the frequencies of these three roots in the corpus, according to the stem-forming affixes they take. \\

\begin{table} [H]
\caption{Occurences of roots of sadness and joy in the corpus}
\label{tab:occurencesofrootsofsadnessandjoyinthecorpus}
\begin{tabular}
    {p{3cm}
    >{\RaggedRight\arraybackslash}p{3cm}
    >{\RaggedRight\arraybackslash}p{2.8cm}
    >{\RaggedRight\arraybackslash}p{2cm}
    }
\lsptoprule
& Inflected verbs without \textit{ng}- \footnote{Forms inflected with prefixes borrowed from other languages (e.g. \textit{naga}{}- or \textit{gina}{}-) are not included in the totals for any of the cells in this table.} & Inflected verbs with \textit{ng}{}- ‘\textsc{pl}+\textsc{vr’} & \textit{ma-} \newline ‘adjectivizer’ \\
\midrule
\textit{sebe}\newline ‘to be sad’ & 8  (no dynamic inflections. 7 with applicatives.) & 13 (3 dynamic) & 6 \\
\tablevspace
\textit{lipay}\newline ‘to be happy/joyful’ & 25 (no dynamic inflections. None with applicatives.) & 8 (2 dynamic) & 36 \\
\tablevspace
\textit{sadya}\newline ‘to enjoy/rejoice’ & 17 (no dynamic inflections. All with applicatives.) & 0 & 3 \\
\lspbottomrule
\end{tabular}
\end{table}

With the bare root \textit{sebe}, ‘to feel sad’, the only possible inflections are happenstantial modality. It usually occurs in intransitive frames in an applicative construction \REF{ex:feltsad}:

\ea
\label{ex:feltsad}
\textit{Sebe} in intransitive, happenstantial, realis form (4 examples in the corpus): \\
Ansaan  a  din  daw  \textbf{nasebean}  a  kon. \\\smallskip
\gll  …-Ansa-an  a  din  daw  \textbf{na-sebe-an}  a  kon. \\
\textsc{t.r}-ask-\textsc{apl}  1\textsc{s.abs}  3\textsc{s.erg}  if/when  \textsc{a.hap.r}-sad-\textsc{apl}  1\textsc{s.abs}  \textsc{hsy} \\
\glt ‘She asked me if \textbf{I felt sad}, it is said.’ [CBWN-C-24 3.2]
\z

The pluractional stem, \textit{ngasebe}, may appear with dynamic or happenstantial inflections.  It appears in intransitive frames with no applicative when the experiencer is plural. It may also have a more active meaning ‘to be sorrowing’. 

\ea
\textit{Ngasebe} in Intransitive, happenstantial, realis (2 examples in the corpus): \\
Labi  na  gid  ki  kami  na  mga  ginikanan   sikad  gid kapung-aw daw  \textbf{nangasebe}  ta  imo  na  pag-alin.\\\smallskip
\gll Labi  na  gid  ki  kami  na  mga  ginikanan   sikad  gid ka-pung-aw daw  \textbf{na-ng-ka-sebe}  ta  imo  na  pag-alin.\\
especially  \textsc{lk}  \textsc{int}  \textsc{obl.p}  1\textsc{p.excl}  \textsc{lk}  \textsc{pl}  parent  very  \textsc{int}
\textsc{nr}-lonely
and  \textsc{a.hap.r}-\textsc{pl}-\textsc{vr}-sad  \textsc{nabs}  2\textsc{s.gen}  \textsc{lk} \textsc{nr.act}-from \\
\glt `Especially for us parents (we) were really very lonely and \textbf{were sorrowing} when you left.’ [NEWL-T-04 5.5]
\z

\ea
Ngasebe in Intransitive, happenstantial, irrealis (1 example in the corpus): \\
Dayon	ambaļ	Bubuo	i	na,	“Dey,	dili	ka	mangasebe." \\\smallskip
\gll Dayon	 …-ambaļ	Bubuo	i	na,	“Dey,	dili	ka	ma-ng-ka-sebe." \\
right.away	\textsc{i.r}-say	tortoise	\textsc{def.n}	\textsc{lk}	friend	\textsc{neg.ir}	2\textsc{s.abs}	\textsc{a.hap.ir}-\textsc{pl}-\textsc{vblz}-sad \\
\glt ‘Right away Tortoise said, “Friend, do not sorrow."' [CBWN-C-16 6.12]
\z

\ea
\textit{Ngasebe}- in intransitive, dynamic, realis (4 examples in the corpus): \\
\label{bkm:Ref123572729}
Man-o  tak  \textbf{gangasebe}  ka? \\\smallskip
\gll Man-o  tak  \textbf{ga-ng-ka-sebe}  ka? \\
why  because  \textsc{i.r}-\textsc{pl}-\textsc{vblz}-sad  2\textsc{s.abs} \\
\glt ‘Why are you \textbf{sorrowing}?’ [CBWN-C-16 6.6]
\z

Example \REF{bkm:Ref123572729} illustrates that the pluraction prefix \textit{ng}{}- is not strictly “agreement” with a plural experiencer, since the experiencer in this example is singular. Although pluraction for stems expressing sadness and joy overwhelmingly occur with plural experiencers, this is not a grammatical restriction. The sense in example \REF{bkm:Ref123572729} is that the actor is sorrowing over and over again, over a long period of time.

The root \textit{lipay} ‘to feel joy’ also often occurs with the intransitive pluraction prefix, though it may take happenstantial or dynamic inflections (\textit{na}-, \textit{ma}- or \textit{ga}-, examples \ref{ex:receivethelord} through \ref{ex:yourparents}). It also occurs in happenstantial modality applicative and non-applicative constructions (see \tabref{tab:occurencesofrootsofsadnessandjoyinthecorpus} above, and several more examples in this section). It occurs in dynamic modality only in combination with the intransitive pluraction prefix \REF{ex:childarrived}. In happenstantial modality without the applicative, \textit{lipay} expresses temporary joy because of some specific external stimulus (examples \ref{ex:receivethelord} through \ref{ex:veryjoyful}). For example, if a child is given candy, her emotion may be described with \textit{nalipay}. But the joy does not last long. The applicative stem, \textit{lipayan}, describes a deeper, more fulfilling joy. An example of this is when a person finishes college and gets a certificate. The emotion experienced by that person and their family can be described as \textit{nalipayan} \REF{ex:yourparents}.

\ea
\label{ex:receivethelord}
Happenstantial, realis (20 examples in the corpus): \\
\textbf{Nalipay}  a  gid  tenged  ki  Maria  daw  paagi  ta  iya na  laygay  na  nļaman  ko  daw  ino  na pagbaton  ta  Ginuo. \\\smallskip
\gll \textbf{Na-lipay}  a  gid  tenged  ki  Maria  daw  paagi  ta  iya na  laygay  na  na-aļam-an  ko  daw  ino  na pag-baton  ta  Ginuo. \\
\textsc{a.hap.r}-joy  1\textsc{s.abs}  \textsc{int}  because  \textsc{obl.p}  Maria  and  by.means.of   \textsc{nabs}  3\textsc{s.gen}
\textsc{lk}   advice  \textsc{lk}  \textsc{a.hap.r}-know-\textsc{apl}  1\textsc{s.erg}  if/when  what  \textsc{lk}
\textsc{nr.act}-receive  \textsc{nabs}  Lord \\
\glt `I \textbf{am really joyful} because of Maria and by means of her advice I know how to receive the Lord.’ [CBWN-C-23 5.6]
\z
\ea
 … daw  \textbf{nalipay}  kay  gid  tak gapadaļa ka  ta suļat  ki  kami  daw  nļaman  nay imo  na  kaimtangan. \\\smallskip
\gll  … daw  \textbf{na-lipay}  kay  gid  tak ga-pa-daļa ka  ta suļat  ki  kami  daw  na-aļam-an  nay imo  na  kaimtangan. \\
{} and  \textsc{a.hap.r}-joy  1\textsc{p.excl.abs}  \textsc{int}  because  \textsc{i.r}-\textsc{caus}-take/carry
2\textsc{s.abs}  \textsc{nabs} write  \textsc{obl.p}  2\textsc{p.excl}  and  \textsc{a.hap.r}-know-\textsc{apl}  1\textsc{p.excl.erg} 2\textsc{s.gen}  \textsc{lk}  situation \\
\glt `… and we \textbf{are really joyful} because you sent a letter to us and we know your situation or circumstances.’ [SBWL-C-02  4.1]
\z
\ea
\label{ex:togetherwitheachother}
\textbf{Nalipay}  mga  gauturay  na  gakitaay. \\\smallskip
\gll \textbf{Na-lipay}  mga  ga-utod-ay  na  ga-kita-ay. \\
\textsc{a.hap.r}-joy  \textsc{pl}  \textsc{i.r}-sibling-\textsc{rec}  \textsc{lk}  \textsc{i.r}-see-\textsc{rec} \\
\glt ‘The relatives are joyful when meeting together with each other.’ [JCWN-L-33 6.1]
\z

\ea
\label{ex:lookingatme}
Happenstantial, irrealis (5 examples in the corpus): \\
 … para  dili  danen  masebean  ta  pagtan-aw,  kundi \textbf{malipay}  ta  pagtan-aw  ki  yaken … \\\smallskip
\gll  … para  dili  danen  ma-sebe-an  ta  pag-tan-aw,  kundi\footnotemark{} \textbf{ma-lipay}  ta  pag-tan-aw  ki  yaken … \\
 {} \textsc{purp}  \textsc{neg.ir}   3\textsc{s.abs} \textsc{a.hap.ir}-sad-\textsc{apl}  \textsc{nabs}  \textsc{nr.act}-look.at  but.rather
\textsc{ha.hap.ir}-joy  \textsc{nabs}  \textsc{nr.act}-look.at  \textsc{obl.p}  1s \\
\footnotetext{This is code switching. In CGC this word would be \textit{daw dili}.}
\glt ` … in order that they will not be sad in looking at (me), rather they will be joyful in looking at me …’ (The speaker wanted to finish his college for this reason.) [JBON-J-01 4.9]
\z

\ea
\label{ex:veryjoyful}
Daw  kyo  nadayaran,  kami  \textbf{malipay}  man. \\\smallskip
\gll Daw  kyo  na-dayad-an,  kami  \textbf{ma-lipay}  man. \\
if/when  2\textsc{p.abs}  \textsc{a.hap.r}-good-\textsc{apl}  1\textsc{p.excl.abs}  \textsc{a.hap.ir}-joy  also \\
\glt ‘If you consider (the program) good, we \textbf{will be very joyful}.’ [EMWO-L-10 33.1]
\z

The happenstantial applicative form expresses a deeper, more enduring joy: 

\ea
\label{ex:yourparents}
Isya	pa	ta	uryan	na	adlaw	daw	makatapos	ka	en	dili nang	kaon	\textbf{malipayan}	daw	di,	asta	man	imo	na mga	ginikanan 	daw	makita	ka	danen	na	nakatapos ta	imo	na	kurso. \\\smallskip
\gll Isya	pa	ta	uryan	na	adlaw	daw	maka-tapos	ka	en	dili nang	kaon	\textbf{ma-lipay-an}	daw	di,	asta	man	imo	na mga	ginikanan 	daw	ma-kita	ka	danen	na	naka-tapos ta	imo	na	kurso. \\
one	\textsc{inc}	\textsc{nabs}	late	\textsc{lk}	sun/day	if/when	\textsc{i.hap.r}-finish	2\textsc{s.abs}	\textsc{cm}	\textsc{neg.ir} only/just	2\textsc{s.abs}	\textsc{a.hap.ir}-joy-\textsc{apl}	if/when	\textsc{neg.ir}	including	also	2\textsc{s.gen}	\textsc{lk} \textsc{pl}	parent	if/when	\textsc{a.hap.ir}-see	2\textsc{s.abs}	3\textsc{p.erg}	\textsc{lk}	\textsc{i.hap.r}-finish \textsc{nabs} 	2\textsc{s.gen}	\textsc{lk}	course \\
\glt ‘One more (thing), on a later day if you can finish (your college course), not only will you be innerwardly joyful but rather, your parents when they see you having finished already your course.’ [YBWL-T-02 5.5]
\z

\ea
\label{ex:childarrived}
\textit{ngalipay} in dynamic, realis (no examples in the corpus): \\
\textbf{Gangalipay}  kay  tak  gabot  bata  nay. \\\smallskip
\gll \textbf{Ga-ng-ka-lipay}  kay  tak  ga-abot  bata  nay. \\
\textsc{i.r-pl-vr}-rejoice  1\textsc{p.incl.abs}  because  \textsc{i.r}-arrive  child  1\textsc{p.excl.gen} \\
\glt ‘We \textbf{are rejoicing} because our child arrived.’
\z

\ea
\textit{ngalipay} in happenstantial, realis (2 examples in the corpus): \\ 
\textbf{Nangalipay}   ki  sanglit  kiten  i bilang  gaprudukto ki  en  ta  kabellayan  ta  ate  na  mga  apo  una  sa  na  mga  adlaw. \\\smallskip
\gll \textbf{Na-ng-ka-lipay}   ki  sanglit  kiten  i bilang  ga-prudukto ki  en  ta  ka-bellay-an  ta  ate  na  mga  apo  una  sa  na  mga  adlaw. \\
\textsc{a.hap.r-pl-vr}-rejoice  1\textsc{p.incl.abs}  because  1\textsc{p.incl.abs}  \textsc{def.n}
consider  \textsc{i.r}-product
1\textsc{p.incl.abs}  \textsc{cm}  \textsc{nabs}  \textsc{nr}-difficulty-\textsc{nr}  \textsc{nabs}  1\textsc{p.incl.gen}  \textsc{lk}  \textsc{pl}  ancestor first  \textsc{d4nabs}  \textsc{lk}  \textsc{pl} sun/day \\
\glt `We have \textbf{rejoiced} because as for us we are considered we are producing from the difficulties of our ancestors in those first days.’ [ROOB-T-01 9.20]
\z
\ea
\textit{ngalipay} in happenstantial, irrealis (4 examples in corpus): \\
 … tak  gusto  pa  gid  daen  na  \textbf{mangalipay}. \\\smallskip
\gll  … tak  gusto  pa  gid  daen  na  \textbf{ma-ng-ka-lipay}. \\
{} because  want  \textsc{inc}  \textsc{int}  3\textsc{p.abs}  \textsc{lk}  \textsc{a.hap.ir-pl-vr}-rejoice \\
\glt ‘ … because they really still wanted to \textbf{make merry}.’ (The older people wanted to keep going to programs and dances because they still wanted to make merry.) [RZWE-J-01 20.4]
\z

\ea
\textbf{Mangalipay}  ki  na  para  nang  ki  kiten. \\\smallskip
\gll \textbf{ma-ng-ka-lipay}  ki  na  para  nang  ki  kiten. \\
\textsc{a.hap.ir-pl-vr}-rejoice  1\textsc{p.incl.abs}  \textsc{lk}  \textsc{purp}  only/just  \textsc{obl.p}  1\textsc{p.incl} \\
\glt ‘Let’s \textbf{rejoice} just for us.’  [ROOB-T-01 9.15]
\z

As with the other verbs in this section, the root \textit{sadya}, ‘to enjoy/have fun/make merry’, often takes the pluractional prefix. The pluractional form, \textit{ngasadya}, occurs overwhelmingly with plural experiencers, but this is probably due to the cultural value of being with others when rejoicing or having fun.

\ea
\textbf{Gangasadya}  danen  an  tak  daag  danen  ampang. \\\smallskip
\gll \textbf{Ga-ng-ka-sadya}  danen  an  tak  daag  danen  ampang. \\
\textsc{i.r}-\textsc{pl}-\textsc{vr}-make.merry  3\textsc{p.abs}  \textsc{def.m}  because  win  3\textsc{p.abs}  play \\
‘They \textbf{are celebrating} because they won a/the game.’
\z

\largerpage
\ea
Transitive, happenstantial with obligatory applicative: \\
\textbf{Nasadyaan}  kay  man  tak  gatingeb kay  pa  di, … \\\smallskip
\gll \textbf{Na-sadya-an}  kay  man  tak  ga-tingeb kay  pa  di, … \\
\textsc{a.hap.r}-enjoy/fun-\textsc{apl}  1\textsc{p.excl.abs}  \textsc{emph}  because  \textsc{i.r}-gather.together
1p\textsc{s.excl.abs}  \textsc{inc}  \textsc{d2loc} \\
\glt `We are so \textbf{enjoying/having fun} because we are gathered together here …’ [MBWL-C-08 3.2]
\z

\section{Cognition}
\label{sec:cognition}

The semantic roles in a situation of cogition (internal mental activity) are a Cognizer and a stimulus. The Cognizer is the person or animal that thinks, knows, understands, believes, remembers, forgets, etc., and the stimulus is the entity or idea that is known, understood, believed, etc. Predicates that express cognition in Kagayanen pattern like transitive predicates with the Cognizer as the actor and the stimulus as the undergoer. As with many other situation types discussed in this section, when a cognition predicate takes dynamic inflections, it expresses an intentional meaning, and when it takes happenstantial inflection it may have two meanings depending on the context; 1. simple unintentionally, and 2. abilitative, optative, or opportunitive meanings: ‘may/might/can/possibly’. Exceptions to this generalization are the Class V roots \textit{masmas} ‘to notice’ and \textit{aļam} ‘to know’. These only express unintentional meanings, and consequently only take happenstantial inflection. These two special roots are discussed in some detail later in this section. The following examples illustrate the root \textit{isip} ‘to think’ in its various forms.

\ea
Transitive dynamic: \\
Nya  na  bisan  gabellay  a  uļa  ko  gid  \textbf{paisip}  nan tak  bakod  pagpalangga  ko  an  ki  kyo. \\\smallskip
\gll Nya  na  bisan  ga-bellay  a  uļa  ko  gid  \textbf{pa-isip}  nan tak  bakod  pag-palangga  ko  an  ki  kyo. \\
\textsc{d4abs}  \textsc{lk}  even.though  \textsc{i.r}-tired  1\textsc{s.abs}  \textsc{neg.r}  1\textsc{s.erg}  \textsc{int}  \textsc{t.r}-think  \textsc{d2abs}
because  big  \textsc{nr.act}-have.affection  1\textsc{s.gen}  \textsc{def.m}  \textsc{obl.p}  2p \\
\glt `That is the reason that even though I am working till being tired, I really do not think about that because my affection for you is big.’ [ICWL-T-05 5.5]
\z

\ea
Transitive dynamic with applicative: \\
Daw  \textbf{isipan}  man  kani  danen  na  mananem  ta guso di  salamat.\\\smallskip
\gll Daw  \emptyset{}-\textbf{isip-an}  man  kani  danen  na  ma-ng-tanem  ta guso di  salamat.\\
if/when  \textsc{t.ir}-think-\textsc{apl}  \textsc{emph}  later  3\textsc{p.erg}  \textsc{lk}  \textsc{a.hap.ir}-\textsc{pl}-plant  \textsc{nabs}  seaweed \textsc{rq}  thanks \\
\glt `If later they \textbf{will think about} planting agar seaweed, won’t that be (reason for giving) thanks.’ (The \textit{di} here is the rhetorical question marker.) [RZWE-J-01 18.13]
\z
\ea
Detransitive dynamic: \\
Uļa  \textbf{gaisip}  ta  kayaran  ta  ittaw  daw  dili   kwarta  nang  iran  na  isip. \\\smallskip
\gll Uļa  \textbf{ga-isip}  ta  kayaran  ta  ittaw  daw  dili   kwarta  nang  iran  na  isip. \\
\textsc{neg.r}  \textsc{i.r}-think  \textsc{nabs}  well-being  \textsc{nabs}  person  if/when  \textsc{neg.ir} money  only  3\textsc{p.gen}  \textsc{lk}  think \\
\glt `(They) are not thinking of the well-being of people, but rather money only is what they think about.’ [MOOE-C-01 186.1]
\z
\ea
Transitive happenstantial: \\
Piro  yon  man  \textbf{naisip}  Pedro.  Yon  man  \textbf{naisip}  danen. \\\smallskip
\gll Piro  yon  man  \textbf{na-isip}  Pedro.  Yon  man  \textbf{na-isip}  danen. \\
but  \textsc{d3abs}  also  \textsc{a.hap.r}-think  Pedro  \textsc{d3abs}  also  \textsc{a.hap.r}-think  3\textsc{p.erg} \\
\glt ‘But that really was what Pedro thought on. That really was what they thought on.’ [TTOB-L-03 7.6]
\z

\ea
Detranstive happenstantial: \\
\textbf{Nakaisip}  a  na  magbaksyon  naan  ta  Manila. \\\smallskip
\gll \textbf{Naka-isip}  a  na  mag-baksyon  naan  ta  Manila. \\
\textsc{i.hap.r}-think  1\textsc{s.abs}  \textsc{lk}  \textsc{i.ir}-vacation  \textsc{spat.def}  \textsc{nabs}  Manila \\
\glt ‘I had thought that I will vacation in Manila.’
\z

\ea
Mga  ittaw  \textbf{nakaisip}   na  sulbaren  prublima  ta  pagkaan. \\\smallskip
\gll Mga  ittaw  \textbf{naka-isip}   na  sulbar-en  prublima  ta  pagkaan. \\
\textsc{pl}   person  \textsc{i.hap.r}-think  \textsc{lk}  solve-\textsc{t.ir}  problem  \textsc{nabs}  food \\
\glt ‘People were able to think to solve the problem of food.’ (This story is about periods of crisis when boats with needed supplies could not reach Cagayancillo.) [JCWE-L-32 3.5]
\z

\ea
Detransitive external motivation: \\
\textbf{Kaisip}  a  kani  daw  ino  buaten  ko daw  dili  sakit  uļo  ko  i. \\\smallskip
\gll \textbf{Ka-isip}  a  kani  daw  ino  buat-en  ko daw  dili  sakit  uļo  ko  i. \\
\textsc{i.exm}-think  1\textsc{s.abs}  later  if/when  what  do/make-\textsc{t.ir}  1\textsc{s.erg}
if/when  \textsc{neg.ir}  pain  head  1\textsc{s.gen}  \textsc{def.n} \\
\glt `I can think later of what I will do if my head does not hurt.’
\z

\newpage
\ea
Transitive happenstantial with applicative, with complement clause as stimulus: \\
\textbf{Naisipan}  ta  magkumpari  na  batangan  ta  kwarta tengnged  i  ta  galon. \\\smallskip
\gll \textbf{Na-isip-an}  ta  mag-kumpari  na  \emptyset{}-batang-an  ta  kwarta tengnged  i  ta  galon. \\
\textsc{i.hap.r}-think-\textsc{apl}  \textsc{nabs}  \textsc{rel}-fellow.godfather  \textsc{lk}  \textsc{t.ir}-put-\textsc{apl}  \textsc{nabs}  money
next.to  \textsc{def.n}  \textsc{nabs}  gallon \\
\glt `The fellow godfathers \textbf{thought} to put money next to the gallon.’ [DBWN-T-33 2.14]
\z

\ea
Transitive, dynamic with complement clause as stimulus: \\
 … \textbf{isipen}  ta  na  itanem  ta. \\\smallskip
\gll  … \textbf{isip-en}  ta  na  i-tanem  ta. \\
{} think-\textsc{t.ir}  1\textsc{p.incl.erg}  \textsc{lk}  \textsc{t.deon}-plant  1\textsc{p.incl.erg} \\
\glt ‘ … let’s think that we should plant (something).’ [ROOB-T-01 8.4]
\z

\ea
\textbf{Paisip}  ko  na  dayad  gid  baļay  danen  na  bag-o  tukod. \\\smallskip
\gll \textbf{Pa-isip}  ko  na  dayad  gid  baļay  danen  na  bag-o  tukod. \\
\textsc{t.r}-think  1\textsc{s.erg}  \textsc{lk}  good  \textsc{int}  house  3\textsc{p.gen}  \textsc{lk}  new  build \\
\glt ‘\textbf{I think} that their house that they newly built is really nice.’
\z

\ea
Detransitive, dynamic with complement clause as stimulus: \\
Ta,  ta  uryan  na  mga  adlaw  danen  i  \textbf{gaisip}  en  a  na magpangabui  man  iran  ta  Cagayancillo  i. \\\smallskip
\gll Ta,  ta  uryan  na  mga  adlaw  danen  i  \textbf{ga-isip}  en  a  na mag-pangabui  man  iran  ta  Cagayancillo  i. \\
so  \textsc{nabs}  late  \textsc{lk}  \textsc{pl}  sun/day  3\textsc{p.abs}  \textsc{def.n}  \textsc{i.r}-think  \textsc{cm}  \textsc{inj}  \textsc{lk}
\textsc{i.ir}-live  \textsc{emph}  3\textsc{p.gen}  \textsc{nabs}  Cagayancillo  \textsc{def.n} \\
\glt `So, at a latter time they \textbf{were thinking} that they would live on Cagayancillo.’ [PTOE-T-01 8.1]
\z
\ea
Kaysan  man  dili  a  makatunuga  daw  kilem tak  \textbf{gaisip}  a  tenged  ta  ame  na  pangabui. \\\smallskip
\gll Kaysan  man  dili  a  maka-tunuga  daw  kilem tak  \textbf{ga-isip}  a  tenged  ta  ame  na  pangabui. \\
sometimes  \textsc{emph}  \textsc{neg.ir}  1\textsc{s.abs}  \textsc{i.hap.ir}-sleep  if/when  night
because  \textsc{i.r}-think  1\textsc{s.abs}  about  \textsc{nabs}  1\textsc{p.excl.gen}  \textsc{lk}  living \\
\glt `Sometimes I can’t even sleep when it is night because I \textbf{am thinking} about our living.’ [AFWL-L-01 7.1]
\z

\ea
Pag-adyo-adyo  tise  en  \textbf{gaisip}  en  danen  na  meļes ta  duyan. \\\smallskip
\gll Pag-adyo-adyo  tise  en  \textbf{ga-isip}  en  danen  na  m-eļes ta  duyan. \\
\textsc{nr.act}-\textsc{red}-far  small  \textsc{cm}  \textsc{i.r}-think  \textsc{cm}  3\textsc{p.abs}  \textsc{lk} \textsc{i.v.ir}-borrow \textsc{nabs}  hammock \\
\glt `When a little further, they \textbf{were thinking} that (they) will borrow a hamock.’ [JCWN-T-22 5.11]
\z

\ea
Transitive, dynamic, applicative, with complement clause: \\
\textbf{Paisipan}  ko  pa  daw  paalinen  ko  kanen  an daw  dili. \\\smallskip
\gll \textbf{Pa-isip-an}  ko  pa  daw  pa-alin-en  ko  kanen  an daw  dili. \\
\textsc{t.r}-think-\textsc{apl}  1\textsc{s.erg}  \textsc{inc}  if/when  \textsc{caus}-from-\textsc{t.ir}  1\textsc{s.gen}  3\textsc{s.abs}  \textsc{def.m} if/when  \textsc{neg.ir} \\
\glt `I \textbf{was still considering} if I will make him/her leave or not.’
\z

\ea
 … dili  nyo  gilekan  inyo  na  bata  tak  \textbf{isipan}  danen,   bata  pa  danen  i. \\\smallskip
\gll  … dili  nyo  gilek-an  inyo  na  bata  tak  \emptyset{}-\textbf{isip-an}  danen,   bata  pa  danen  i. \\
{} \textsc{neg.ir}  2\textsc{p.erg}  angry-\textsc{apl}  2\textsc{p.gen}  \textsc{lk}  child  because  \textsc{t.ir}-think-\textsc{apl}  3\textsc{p.abs}
child  \textsc{inc}  3\textsc{p.abs}  \textsc{def.n} \\
\glt ‘ … do not get angry with your children because, \textbf{think about} them, they are still young.’ [RZWE-J-01 8.4]
\z
\ea
Transitive, happenstantial, applicative, with complement clause: \\
\textbf{Naisipan}  din  en  na  buatan  din  na  para makabaļes  kanen  ki  Pedro  ya. \\\smallskip
\gll \textbf{Na-isip-an}  din  en  na  \emptyset{}-buat-an  din  na  para maka-baļes  kanen  ki  Pedro  ya. \\
\textsc{a.hap.r}-think-\textsc{apl}  3\textsc{s.erg}  \textsc{cm}  \textsc{lk}  \textsc{t.ir}-do/make-\textsc{apl}  3\textsc{s.erg}  \textsc{lk}  for
\textsc{i.hap.ir}-revenge  3\textsc{s.abs}  \textsc{obl.p}  Pedro  \textsc{def.f} \\
\glt ‘She \textbf{thought} that she will do (something) to him to avenge Pedro.' [LGON-L-01 14.2]
\z

The root \textit{aļam} ‘to know’ is unusual in a number of ways. It is a Class Vb root, which means it is a non-volitional, basically transitive root that requires an applicative affix to appear in a transitive frame. The non-applicative forms (\textit{naļam} and \textit{maļam}) are always detransitive, because the experiencer must be absolutive, and the stimulus non-absolutive. Like other Class Vb verbs, it only occurs with happenstantial inflection (unless the causative prefix is employed).

The only inflections allowed for the applicative stem \textit{aļaman} are \textit{na}- and \textit{ma}-, in which case the root-initial \textit{a} drops out, leaving the transitive inflected forms \textit{nļaman} and \textit{mļaman}. Without the applicative suffix, the detransitive forms are \textit{naļam} and \textit{maļam}.  This is similar to \textit{mati} `hear', \textit{batyag} `feel' above, and \textit{lipat} `forget', \textit{tingala} `be amazed/wonder', and  \textit{kullaw} `ponder' below in that the detransitive takes either \textit{na}-/\textit{ma}- or \textit{naka}-/\textit{maka}-.


This root is also unique in that sometimes speaker or writers add an \textit{n}- to the root. That is to say, \textit{nakanaļam} and \textit{makanaļam} are interchangeable with \textit{nakaaļam} and \textit{makaaļam} respectively, with no distinction in meaning:

\ea
Nakaaļam kanen	ta	daļan	munta	ta	uma nay. \\
Nakanaļam kanen	ta	daļan	munta	ta	uma nay. \\\smallskip
\gll Naka-(n)aļam	kanen	ta	daļan	munta	ta	uma nay. \\
\textsc{i.hap.r}-know	3\textsc{s.abs}	\textsc{nabs}	road	going	\textsc{nabs}	field	1\textsc{p.excl.gen} \\
\glt ‘S/he knows the road to our field.’
\z

\ea
Uļa	may	makaaļam ta	isturya ko. \\
Uļa	may	makanaļam ta isturya ko. \\\smallskip
\gll Uļa	may	maka-(n)aļam ta	isturya ko. \\
\textsc{neg.r}		\textsc{ext.in}	\textsc{i.hap.ir}-know	\textsc{nabs}	story 1\textsc{s.gen} \\
\glt ‘There is no one who can know my story.’
\z

In terms of semantics, when \textit{aļam} occurs in a detransitive construction, it usually means ‘to know how to do something’ and in a transitive construction with the applicative it means ‘to know something’ (facts or head knowledge). This is a strong tendency, but not always true. The morphological causative, \textit{paļam}, means ‘to ask permission, or to inform/tell/let someone know something’. There is considerable variation in how this root is used.
 
The following are a few examples of \textit{aļam} from the corpus.

\ea
Detransitive, happenstantial with complement clause ‘to know how to do something’: \\
\textbf{Naļam}  en  danen  i  magbasa,  \textbf{naļam}  en  magsuma, \textbf{naļam}  en  magsulbar  ta  prublima,  \textbf{naļam}  en mananem,  \textbf{naļam}  en  ta  tanan  na  ubra … \\\smallskip
\gll \textbf{Na-aļam}  en  danen  i  mag-basa,  \textbf{na-aļam}  en  mag-suma, \textbf{na-aļam}  en  mag-sulbar  ta  prublima,  \textbf{na-aļam}  en ma-ng-tanem,  \textbf{na-aļam}  en  ta  tanan  na  ubra … \\
\textsc{a.hap.r}-know  \textsc{cm}  3\textsc{p.abs}  \textsc{def.n}  \textsc{i.ir}-read  \textsc{a.hap.r}-know  \textsc{cm}  \textsc{i.ir}-add \textsc{a.hap.r}-know  \textsc{cm}  \textsc{i.ir}-solve  \textsc{nabs}  problem  \textsc{a.hap.r}-know  \textsc{cm} \textsc{a.hap.ir-pl}-plant  \textsc{a.hap.r}-know  \textsc{cm}  \textsc{nabs}  all  \textsc{lk}  work \\
\glt `They already know how to read, already know how to add, already know how to solve problems, already know how to plant, know how to do all the work …’ [SPOB-L-01 8.3]
\z

\ea
Detransitive, happenstantial with stimulus RP and complement clause ‘to know a fact’: \\
Tama  na  ittaw  \textbf{naļam}  gid  ta  ake  na  inagian … \\\smallskip
\gll Tama  na  ittaw  \textbf{na-aļam}  gid  ta  ake  na  <in>agi-an … \\
 many  \textsc{lk}  person  \textsc{a.hap.r}-know  \textsc{int}  \textsc{nabs}  1\textsc{s.gen}  \textsc{lk}  <\textsc{nr.res}>pass-\textsc{apl} \\
\glt ‘Many people really knew about my experience …’ [VAWN-T-16 6.2]
\z

\ea
Uļa  gid  \textbf{naļam}  Pwikan  i  na  paluko  nang kanen i  ta  Umang …\\\smallskip
\gll Uļa  gid  \textbf{na-aļam}  Pwikan  i  na  pa-luko  nang kanen i  ta  Umang …\\
\textsc{neg.r}  \textsc{int}  \textsc{a.hap.r}-know  sea.turtle  \textsc{def.n}  \textsc{lk}  \textsc{t.r}-trick  only/just 3\textsc{s.abs}
\textsc{def.n}  \textsc{nabs}  hermit.crab \\
\glt `The Sea Turtle did not know that Hermit Crab tricked him/her …' [DBWN-T-26 10.1]
\z

Example \REF{ex:knowhowtoswim} is from a story about planting agar seaweed, which is a major income crop on Cagayancillo. The paragraph describes how all members of the family are involved in the process.

\ea
\label{ex:knowhowtoswim}
Basta	na	darko	en	mga	bataan	an	a	pwidi	iran	maļam  man	mag-igot,	maļam	man	magluoy. \\\smallskip
\gll Basta	na	darko	en	mga	bata-an	an	a	pwidi	iran	ma-aļam  man	mag-igot,	ma-aļam	man	mag-luoy. \\
just.so.that	\textsc{lk}	big.\textsc{pl}	\textsc{cm}	\textsc{pl}	child-\textsc{nr}	\textsc{def.m}	\textsc{inj}	can	3\textsc{p.gen} \textsc{a.hap.ir}-know  also	\textsc{i.ir}-tie	\textsc{a.hap.ir}-know also \textsc{i.ir}-swim \\
\glt ‘Just so that the children who are big (in contrast with smaller children) can know how to tie (agar seaweed on long ropes to grow in the sea), know how to swim (in the sea to stake the long ropes with seaweed tied on them in the sea).'  [ETOP-C-08 5.2]
\z

Example \REF{ex:privileges} is from a story about a long ago tyrant leader. Because the people were not taught or were lacking in knowing of how to deal with this in a legal way, they did not know their rights and privileges, so he was able to do whatever he wanted to them:

\ea
\label{ex:privileges}
Mga	ittaw	an	dili	maļam	ta	iran	na	kayaran	daw pribliyo … \\\smallskip
\gll Mga	ittaw	an	dili	ma-aļam	ta	iran	na	ka-ayad-an	daw pribliyo … \\
\textsc{pl}	person	\textsc{def.m}	\textsc{neg.ir}	\textsc{a.hap.ir}-know	\textsc{nabs}	3\textsc{p.gen}	\textsc{lk}	\textsc{nr}-well-\textsc{nr}	and privilege \\
\glt ‘The people cannot know what is good for them (lit. their goodness) and privileges.’ [BEWN-T-01 2.16]
\z

\ea
Transitive, happenstantial, with obligatory applicative = `to know a fact': \\
Uļa  kay  nang  en  galeges  tak  \textbf{nļaman} nay  na  dili  gid  miyag  ame  na  mga  ginikanan. \\\smallskip
\gll Uļa  kay  nang  en  ga-leges  tak  \textbf{na-aļam-an} nay  na  dili  gid  miyag  ame  na  mga  ginikanan. \\
\textsc{neg.r}  1\textsc{p.excl.abs}  only/just  \textsc{cm}  \textsc{i.r}-force  because  \textsc{a.hap.r}-know-\textsc{apl}
1\textsc{p.excl.erg}   \textsc{lk}  \textsc{neg.ir}  \textsc{int}   agree  1\textsc{p.excl.gen}  \textsc{lk}  \textsc{pl}  parents \\
\glt `We just did not force (our parents to let us go to college) because we knew that our parents really do not agree.’ [CBWE-C-05 4.3]
\z

\ea
Unso	a	nang	tunuga	aged	mļaman	ko	daw	matuod gid	sugid	nyo	an	na	maļbaļ	danen	mari. \\\smallskip
\gll Unso	a	nang	tunuga	aged	ma-aļam-an	ko	daw	ma-tuod gid	sugid	nyo	an	na	maļbaļ	danen	mari. \\
\textsc{d4loc.pr}	1\textsc{s.abs}	only/just	sleep	so.that	\textsc{a.hap.ir}-know-\textsc{apl}	1\textsc{s.erg}	if/when	\textsc{adj}-true	\textsc{int}	 tell
2\textsc{p.gen}	\textsc{def.m}	\textsc{lk}	witch	3\textsc{p.abs}	godmother \\
\glt ‘Right there I will just sleep	so that I can know if what you said is true, that (my) godmother and her companions are witches.’ [CBWN-C-13 4.4]
\z

Examples \REF{ex:toelementaryschool} and \REF{ex:thelanguageofkagayanen} illustrate the causative stem \textit{paļam} with the dynamic inflections.

\ea
\label{ex:toelementaryschool}
Causative, dynamic, transitive inflection: \\
Kaappat	na	papaļam	ta	bandilyo	daw	oras	ta	selled iskwila 	mga	istudyanti	daw	nagaiskwila	ta	ilimintiriya. \\\smallskip

\gll Ka-appat	na	pa-pa-aļam	ta	bandilyo	daw	oras	ta	selled iskwila 	mga	istudyanti	daw	naga-iskwila\footnotemark{}	ta	ilimintiriya. \\
\textsc{nr}-four	\textsc{lk}	\textsc{t.r}-\textsc{caus}-know	\textsc{nabs} 	public.announce	if/when	hour.time	\textsc{nabs}	enter	school \textsc{pl}	student	and	\textsc{i.r}-school	\textsc{nabs}	elementary\\
\glt ‘The fourth (thing) the public announcement lets (people) know is when students will enter school and going to elementary school.’ [VAOE-J-01 7.1]\footnotetext{The form \textit{naga-} in this example is code switching from \isi{Hiligaynon}. The Kagayanen form would be \textit{ga-}.}
\z

\ea
\label{ex:thelanguageofkagayanen}
Causative, dynamic, intransitive inflection (detransitive construction): \\
Ta	pitsa	traynta	uno	ta	Uktubri	sellem,	gapaļam	a ki	danen	tatay	daw	nanay	na,	“Tay,	Nay,	miling	a	ta	Manila tak	magtudlo	a	kon	ki	Pedro	ta	ambaļ	ta	na	Kagay-anen. \\\smallskip
\gll Ta	pitsa	traynta	uno	ta	Uktubri	sellem,	ga-pa-aļam	a ki	danen	tatay	daw	nanay	na,	“Tay,	Nay,	m-iling	a	ta	Manila tak	mag-tudlo	a	kon	ki	Pedro	ta	ambaļ	ta	na	Kagay-anen. \\
\textsc{nabs}	date	thirty	one	\textsc{nabs}	October	morning	\textsc{i.r-caus}-know	1\textsc{s.abs}
\textsc{obl.p}	3p	father	and	mother	\textsc{lk}	Dad	Mom	\textsc{i.v.ir}-go	1\textsc{s.abs}	\textsc{nabs}	Manila
because	\textsc{i.ir}-teach	1\textsc{s.abs}	\textsc{hsy}	\textsc{obl.p}	Pedro	\textsc{nabs}		say	1\textsc{p.incl.gen}	\textsc{lk}	Kagay-anen \\
\glt ‘On the date of thirty one of October morning I let them know father and mother, “Dad, Mom, I will go to Manila because (I) will teach Pedro our language Kagayanen.”’ [RCON-L-03 4.1]
\z

Finally, example \REF{ex:writeinkagayanen} illustrates the adjectival stem \textit{maļam} based on the root \textit{aļam}. This stem, then, may be inflected with dynamic inflections with an inchoative sense. A literal translation of this example might be ‘before you become proficient/knowledgeable writing in other languages  …’.

\ea
\label{ex:writeinkagayanen}
Kinangļan  bag-o  kaw  \textbf{magmaļam}   magsuļat  ta  duma bilang makasuļat  kaw  anay  ta  Kagayanen.\\\smallskip
\gll Kinangļan  bag-o  kaw  \textbf{mag-ma-aļam}   mag-suļat  ta  duma bilang maka-suļat  kaw  anay  ta  Kagayanen.\\
necessary  before  2\textsc{p.abs}  \textsc{i.ir}-\textsc{adj}-know  \textsc{i.ir}-write  \textsc{nabs}  other
example \textsc{i.hap.ir}-write  2\textsc{p.abs}  first/while  \textsc{nabs}  Kagayanen \\
\glt `(It is) necessary that before you learn how to write in other (languages), for example, you are able to write first in Kagayanen  …’ [JCOE-T-06 1.4]
\z

The roots \textit{demdem} ‘to remember’ and \textit{intindi} ‘to understand’ function much like \textit{isip} ‘to think’ above. In the dynamic forms, the  meaning is intentional and sometimes expresses ideas such as ‘to be concerned about’ or ‘to care for another’. These roots require the applicative in the transitive, happenstantial forms.

\ea
Transitive dynamic: \\
Pirmi  ka  din  na  \textbf{pademdem}  daw  ino  en  kaimtangan no  an  unsan  Manila  an. \\\smallskip
\gll Pirmi  ka  din  na  \textbf{pa-demdem}  daw  ino  en  kaimtangan no  an  unsan  Manila  an. \\
always  2\textsc{s.abs}  3\textsc{s.erg}  \textsc{lk}  \textsc{t.r}-remember  if/when  what  \textsc{cm}   situation
2\textsc{s.gen}  \textsc{def.m}  \textsc{d2loc.pr}  Manila  \textsc{def.m} \\
\glt `He always \textbf{remembers} you/has you in mind, what your situation is right there in Manila.’ [PMWL-T-06 2.3]
\z

\ea
Dapat  no  \textbf{demdemen}  ni  ake  na  paglaygay ki  kaon. \\\smallskip
\gll Dapat  no  \textbf{demdem-en}  ni  ake  na  pag-laygay ki  kaon. \\
must  2\textsc{s.erg}  remember-\textsc{t.ir}  \textsc{d1abs}  1\textsc{s.gen}  \textsc{lk}  \textsc{nr.act}-speech/advice
\textsc{obl.p}  2s \\
\glt `You must \textbf{remember}/keep in mind this my advice to you.’ [NEWL-T-03 5.6]
\z

\ea
Detransitive dynamic: \\
Asta  mga  kabataan  ta  napatay  ya  sigi  na  agaļ tak  \textbf{gademdem}  danen  ta  tatay  danen. \\\smallskip
\gll Asta  mga  ka-bata-an  ta  na-patay  ya  sigi  na  agaļ tak  \textbf{ga-demdem}  danen  ta  tatay  danen. \\
until  \textsc{pl}  \textsc{nr}-child-\textsc{nr}  \textsc{nabs}  \textsc{a.hap.r}-die  \textsc{def.f}  continually  \textsc{lk}  cry
because  \textsc{i.r}-remember  3\textsc{p.abs}  \textsc{nabs}  father  3\textsc{p.gen} \\
\glt `Including the children of the one who died kept on crying because they \textbf{remember/keep in their minds} their father.’ [AMWN-T-01 2.15]
\z

\ea
Transitive dynamic with applicative: \\
A  bilang  \textbf{pademdeman}  ko  nang  a  pitsa  trisi  ta buļan  ta  Eniro  1976  a  kaadlawan  ta  ame  na bata  na  maguļang  ta  ame  na  pamilya  na  ngaran  din  Pedro. \\\smallskip
\gll A  bilang  \textbf{pa-demdem-an}  ko  nang  a  pitsa  trisi  ta buļan  ta  Eniro  1976  a  ka-adlaw-an  ta  ame  na bata  na  ma-guļang  ta  ame  na  pamilya  na  ngaran  din  Pedro. \\
\textsc{inj}  as  \textsc{t.r}-remember-\textsc{apl}  1\textsc{s.erg}  only/just  \textsc{inj}  date  thirteen  \textsc{nabs}
month/moon  \textsc{nabs}  January  \textsc{1976}  \textsc{inj}  \textsc{nr}-sun/day-\textsc{nr}  \textsc{nabs}  1\textsc{p.excl.gen}  \textsc{lk} child  \textsc{lk}  \textsc{adj}-old  \textsc{nabs}  1\textsc{p.excl.gen}  \textsc{lk}  family  \textsc{lk}  name  3\textsc{s.gen}  Pedro \\
\glt `Well, as I \textbf{brought to mind} the date of thirteen in the month of January, 1976, the birthday (lit. special day) of our child who is the older one in our family whose name is Pedro.’ [EFWN-T-10 1.1]
\z

\ea
Transitive happenstantial with obligatory applicative: \\
\textbf{Nademdeman}  ko  na  lain  man  daw  manaw  na may  gamasakit. \\\smallskip
\gll \textbf{Na-demdem-an}  ko  na  lain  man  daw  m-panaw  na may  ga-masakit. \\
\textsc{a.hap.r}-remember-\textsc{apl}  1\textsc{s.erg}  \textsc{lk}  bad  \textsc{emph}  if/when  \textsc{i.v.ir}-walk/go  \textsc{lk}
\textsc{ext.in}  \textsc{i.r}-sick \\
\glt ‘I remember that it is bad when leaving somewhere when there is someone who became sick.’ [RCON-L-03 8.4] \\\smallskip
\textit{*nademdem}
\z

The following are examples of \textit{intindi} ‘to understand’ (from Spanish \textit{entender} ‘to understand’) from the corpus. The only example in the corpus of the detransitive dynamic form of \textit{intindi} is a reciprocal (\ref{bkm:Ref123631997}). However, other examples do occur in conversation (\ref{bkm:Ref123632049}). The reciprocal can mean either to understand each other or to get along together well:

\ea 
\label{bkm:Ref123631997}
Detransitive dynamic: \\
Uļa  kaw  nang  \textbf{gaintindiay}. \\\smallskip
\gll Uļa  kaw  nang  \textbf{ga-intindi-ay}. \\
\textsc{neg.r}  2\textsc{p.abs}  just  \textsc{i.r}-understand-\textsc{rec} \\
\glt ‘You just didn’t \textbf{understand/get along with each other}.’ [EFWN-T-09 54.2]
\z

\newpage
\ea
\label{bkm:Ref123632049}
\textbf{Gaintindi}  man  kanen  ki  kami  na  naan  kay  dya gatunuga. \\\smallskip
\gll \textbf{Ga-intindi}  man  kanen  ki  kami  na  naan  kay  dya ga-tunuga. \\
\textsc{i.r}-understand  \textsc{emph}  3\textsc{s.abs}  \textsc{obl.p}  1\textsc{p.excl}  \textsc{lk}  \textsc{spat.def}  1\textsc{p.excl.abs}  \textsc{d4loc} \textsc{i.r}-sleep \\
\glt `S/he \textbf{was concerned for/took care of} us when we slept there.’
\z

\ea
Transitive dynamic: \\
\textbf{Paintindi}  ko  gid  ta  usto  ambaļ  no  an. \\\smallskip
\gll \textbf{Pa-intindi}  ko  gid  ta  usto  ambaļ  no  an. \\
\textsc{t.r}-understand  1\textsc{s.erg}  \textsc{int}  \textsc{nabs}  well.properly  say  2\textsc{s.gen}  \textsc{def.m} \\
\glt ‘I \textbf{intentionally understood} well what you said.’
\z
\ea
Transitive dynamic with applicative: \\
\textbf{Paintindian}  ko  gid  ta  usto  ambaļ  no  an. \\\smallskip
\gll \textbf{Pa-intindi-an}  ko  gid  ta  usto  ambaļ  no  an. \\
\textsc{t.r}-understand-\textsc{apl}  1\textsc{s.erg}  \textsc{int}  \textsc{nabs}  well.properly  say  2\textsc{s.gen}  \textsc{def.m} \\
\glt ‘I \textbf{intentionally understood} well what you said.’
\z

\ea
Detransitive happenstantial: \\
Kanen  nang  \textbf{nakaintindi}  daw  ino  ambaļ  din. \\\smallskip
\gll Kanen  nang  \textbf{naka-intindi}  daw  ino  ambaļ  din. \\
3\textsc{s.abs}  only/just  \textsc{i.hap.r}-understand  if/when  what  say  3\textsc{s.gen} \\
\glt ‘S/he only was \textbf{able to understand} what s/he said.’
\z

\ea
Detransitive, external motivation:
\textbf{Kaintindi}  a  ta  Cebuano. \\\smallskip
\gll \textbf{Ka-intindi}  a  ta  Cebuano. \\
\textsc{i.exm}-understand  1\textsc{s.abs}  \textsc{nabs}  Cebuano \\
\glt ‘I can \textbf{understand} Cebuano.’
\z

\ea
Transitive happenstantial with obligatory applicative (both \textit{masmas} ‘notice’ and \textit{intindi} ‘understand’): \\
Naan  dya  danen  namasmasan  na  iran  na  bata  gakinangļan ta  iran  na  pagtamed  daw  pagpalangga  daw  naan  dya danen  \textbf{naintindian}  na  iran  na  bata  uļa  gakinangļan ta  tama  na  kwarta … \\\smallskip
\gll Naan  dya  danen  na-masmas-an  na  iran  na  bata  ga-kinangļan ta  iran  na  pag-tamed  daw  pag-palangga  daw  naan  dya danen  \textbf{na-intindi-an}  na  iran  na  bata  uļa  ga-kinangļan ta  tama  na  kwarta … \\
\textsc{spat.def}  \textsc{d4loc}  3\textsc{p.erg}  \textsc{a.hap.r}-notice-\textsc{apl}  \textsc{lk}  3\textsc{p.gen}  \textsc{lk}  child  \textsc{i.r}-need  \textsc{nabs}  3\textsc{p.gen}  \textsc{lk}  \textsc{nr.act}-attention  and  \textsc{nr.act}-affection  and  \textsc{spat.def}  \textsc{d4loc}
3\textsc{p.erg}  \textsc{a.hap.r}-understand-\textsc{apl}  \textsc{lk}  3\textsc{p.gen}  \textsc{lk}  child  \textsc{neg.r}  \textsc{i.r}-need
\textsc{nabs}  many  \textsc{lk}  money \\
\glt `There they noticed that their child needs their attention and affection and there they \textbf{understood} that their child does not need lots of money …’ [YBWN-T-07 2.8] \\\smallskip
\textit{*naintindi}
\z

The root \textit{lipat} ‘to forget’ patterns like emotion predicates in that it takes the applicative in its basic transitive form. In intransitive constructions, the experiencer is absolutive and the stimulus is non-absolutive. In detransitive constructions the meaning is ‘to become forgetful of something’ and in transitive constructions the meaning is ‘to forget something’ or ‘forget to do something’. In detranstiive happenstantial constructions, \textit{lipat} can take either \textit{na-/ma}- or \textit{naka-/maka}- which is similar to \textit{mati} `hear', \textit{batyag} `feel', \textit{aļam} `know', \textit{tingala} `be amazed/ wonder', and \textit{kullaw} `ponder'. In transitive constructions, the experiencer is ergative and the stimulus is absolutive.

\ea
Detransitive dynamic: \\
\textbf{Galipat}  en  kanen  ta  mga  ambaļen  din. \\\smallskip
\gll \textbf{Ga-lipat}  en  kanen  ta  mga  ambaļ-en  din. \\
\textsc{i.r}-forget  \textsc{cm}  3\textsc{s.abs}  \textsc{nabs}  \textsc{pl}  say-\textsc{t.ir}  3\textsc{s.gen} \\
\glt ‘S/he \textbf{is forgetful} now of what s/he will say.’ (This is not a one-time event, but is an ongoing condition.)
\z

\ea
… bilang  yon  kon  sumpaan  danen  na  dili  danen  \textbf{maglipat} ta  kada  isya  ki  danen. \\\smallskip
\gll  … bilang  yon  kon  sumpa-an  danen  na  dili  danen  \textbf{mag-lipat} ta  kada  isya  ki  danen. \\
{} as  \textsc{d3abs}  \textsc{hsy}  vow-\textsc{apl}  3\textsc{p.erg}  \textsc{lk}  \textsc{neg.ir}  3\textsc{p.abs}  \textsc{i.ir}-forget \textsc{nabs}  each  one  \textsc{obl.p}  3p \\
\glt ` … as what they vowed that they \textbf{will not forget/put out of mind} each other.’ [CBWN-C-14 5.6] 
\z

\newpage
\ea
Transitive dynamic with obligatory applicative: \\
\textbf{Palipatan}  din  ambaļ  ya  ki  kanen  na  may  miting  kisyem. \\\smallskip
\gll \textbf{Pa-lipat-an}  din  ambaļ  ya  ki  kanen  na  may  miting  kisyem \\
\textsc{t.r}-forget-\textsc{apl}  3\textsc{s.erg}  say  \textsc{def.f}  \textsc{obl.p}  3s  \textsc{lk}  \textsc{ext.in}  meeting  tomorrow \\
\glt ‘S/he \textbf{forgot/put out of mind} what was said to him/her that there is a meeting tomorrow.’
\z

\ea
\textbf{Lipatan}  no  a  mga  bisyo, … \\\smallskip
\gll \emptyset{}-\textbf{Lipat-an}  no  a  mga  bisyo, … \\
\textsc{t.ir}-forget-\textsc{apl}  2\textsc{s.erg}  \textsc{inj}  \textsc{pl}  vice \\
\glt ‘\textbf{Forget/put out of mind} the vices …’ [JCOB-L-02 13.14]
\z

\ea
Detransitive happenstantial: \\
\textbf{Nakalipat}  aren  daw  kino  ka. \\\smallskip
\gll \textbf{Naka-lipat}  aren  daw  kino  ka. \\
\textsc{i.hap.r}-forget  1\textsc{s.abs}  if/when  who  2\textsc{s.abs} \\
\glt ‘I \textbf{happened to forget} who you are.’ \\
\z
\ea
Na  danen  bisan  nyaan  ta  duma  na  lugar  uļa  danen \textbf{nalipat}  ta  ate  na  ambaļ. \\\smallskip
\gll Na  danen  bisan  nyaan  ta  duma  na  lugar  uļa  danen \textbf{na-lipat}  ta  ate  na  ambaļ. \\
\textsc{lk}  3\textsc{p.abs}  even.though  \textsc{spat.def}  \textsc{nabs}  other  \textsc{lk}  place  \textsc{neg.r}  3\textsc{p.abs}
\textsc{a.hap.r}-forget  \textsc{nabs}  1\textsc{p.incl.gen}  \textsc{lk}  say \\
\glt `When they are even in different places they did not \textbf{happen to forget} our language.’ [SFOB-L-02 4.3]
\z

\ea
Transitive happenstantial: \\ 
\textbf{Nalipat}  ko  na  deen  bayo  no  ya  na nabilin  no  naan  baļay  nay. \\\smallskip
\gll \textbf{Na-lipat}  ko  na  dala-en  bayo  no  ya  na na-bilin  no  naan  baļay  nay. \\
\textsc{a.hap.r}-forget  1\textsc{s.erg}  \textsc{lk}  take/carry-\textsc{t.ir}  clothes  2\textsc{s.gen}  \textsc{def.f}  \textsc{lk}
\textsc{a.hap.r}-leave.behind  2\textsc{s.erg}  \textsc{spat.def}  house  1\textsc{p.excl.gen} \\
\glt ‘I forgot to bring your clothes that you left at our house.’
\z

\newpage
\ea
Transitive happenstantial with applicative: \\
\textbf{Nalipatan}  nay  naan  ta  sakayan  isya  na  bag daw  malita. \\\smallskip
\gll \textbf{Na-lipat-an}  nay  naan  ta  sakay-an  isya  na  bag daw  malita. \\
\textsc{a.hap.r}-forget-\textsc{apl}  1\textsc{p.excl.erg}  \textsc{spat.def}  \textsc{nabs}  ride-\textsc{nr}  one  \textsc{lk}  bag and  suitcase. \\
\glt `We \textbf{forgot} one bag and a suitcase on the bus.’ [AFWL-L-01 9.7]
\z

The root \textit{tingaļa} ‘to wonder/be amazed’ belongs to both Class Vb and Class VIII. It is much like roots depicting transitive situations. The detransitive form can take \textit{na-/ma}- or \textit{naka-/maka}-, like \textit{mati} `hear', \textit{batyag} `feel', \textit{aļam} `know', \textit{lipat} `forget' and \textit{kullaw} `ponder.'
\ea
Detransitive dynamic: \\
\textbf{Gatingaļa}  kon  iya  na  bļai  tak  iran  na  ayam sigi  aļok. \\\smallskip
\gll \textbf{Ga-tingaļa}  kon  iya  na  bļai  tak  iran  na  ayam sigi  aļok. \\
\textsc{i.r}-amaze/wonder  \textsc{hsy}  3\textsc{s.gen}  \textsc{lk}  child.in.law’s.parent  bcause  3\textsc{p.gen}  \textsc{lk}  dog  continually bark \\
\glt `Her child’s spouse’s parent \textbf{was wondering} because their dog keeps on barking.’ [MBON-T-05 3.3]
\z
\ea
Transitive dynamic with obligatory applicative: \\
\textbf{Patingaļaan}  danen  tatos  na  daļa  ko  tak bag-o  nang  danen  nakita. \\\smallskip
\gll \textbf{Pa-tingaļa-an}  danen  tatos  na  daļa  ko  tak bag-o  nang  danen  na-kita. \\
\textsc{t.r}-amaze/wonder-\textsc{apl}  3\textsc{p.erg}  coconut.crab  \textsc{lk}  take/carry  1\textsc{s.erg}  because
new/first  ony/just  3\textsc{p.erg}  \textsc{a.hap.r}-see \\
\glt `They \textbf{were amazed/wondered} about the coconut crab that I carried because they just newly saw it.’
\z
\ea
External motivation: \\
\textbf{Katingaļa}  a  tak  ubos  sidda  ya. \\\smallskip
\gll \textbf{Ka-tingaļa}  a  tak  ubos  sidda  ya. \\
\textsc{i.exm}-amaze/wonder  1\textsc{s.abs}  because  used.up  fish  \textsc{def.f} \\
\glt ‘I \textbf{was amazed/wondered} because the fish was gone or used up.’
\z

\newpage
\ea
Detransitive happenstantial: \\
Pag-abot	ko	dya	nakatingaļa	a	gid	tak	sikad dayad	na	lugar. \\\smallskip
\gll Pag-abot	ko	dya	naka-tingaļa	a	gid	tak	sikad dayad	na	lugar. \\
\textsc{nr.act}-arrive	1\textsc{s.abs}	\textsc{d4loc}	\textsc{i.hap.r}-amaze/wonder	1\textsc{s.abs}	\textsc{int}	because	very good	\textsc{lk}	place \\
\glt ‘When I arrived there, I was very amazed because it was a very good place.’
\z

\ea
\textbf{Natingaļa}  a  na  uļa  di  tsinalas  ko  ya. \\\smallskip
\gll \textbf{Na-tingaļa}  a  na  uļa  di  tsinalas  ko  ya. \\
\textsc{a.hap.r}-amaze/wonder  1\textsc{s.abs}  \textsc{lk}  \textsc{neg.r}  \textsc{d1loc}  sandals  1\textsc{s.gen}  \textsc{def.f} \\
\glt `I \textbf{had been amazed} that my sandals were not here.’
\z
\ea
Transitive happenstantial with obligatory applicative: \\
Yon  \textbf{natingaļaan}  ta  mga  ittaw  na  uļa  i ennasan. Naan  dya  kanen  i  nanglaya.\\\smallskip
\gll Yon  \textbf{na-tingaļa-an}  ta  mga  ittaw  na  uļa  i ennas-an. Naan  dya  kanen  i  na-ng-laya.\\
\textsc{d3abs}  \textsc{a.hap.r}-amazed/wonder-\textsc{apl}  \textsc{nabs}  \textsc{pl}  person  \textsc{lk}  \textsc{neg.r}  \textsc{att}
low.tide-\textsc{nr}
\textsc{spat.def}  \textsc{d4loc}  3\textsc{s.abs}  \textsc{def.n}  \textsc{a.hap.r}-\textsc{pl}-cast.net \\
\glt `That is what people \textbf{were wondering about} that there is no tidal flats. There he was cast net fishing.’ [VAWN-T-17 2.10]
\z

\largerpage
The verb \textit{kaļa} ‘to recognize/know someone or something’ usually occurs with happenstantial affixes in the corpus.

\ea
Detransitive dynamic (this is an unusual construction for this root, and is not accepted by all speakers): \\
Daw  ittaw  na  \textbf{gakaļa}  ta  iya  na  mga  saļa  daw  ganeļseļ ta  iya  na  mga  saļa  daw  gabaton  ta  Ginuo,  yon  kon  masalbar. \\\smallskip
\gll Daw  ittaw  na  \textbf{ga-kaļa}  ta  iya  na  mga  saļa  daw  ga-neļseļ ta  iya  na  mga  saļa  daw  ga-baton  ta  Ginuo,  yon  kon  ma-salbar. \\
if/when  person  \textsc{lk}  \textsc{i.r}-know/recognize  \textsc{nabs}  3\textsc{s.gen}  \textsc{lk}  \textsc{pl}  sin  and  \textsc{i.r}-regret
\textsc{nabs}  3\textsc{s.gen}  \textsc{lk}  \textsc{pl}  sin  and  \textsc{i.r}-reeive  \textsc{nabs}  Lord  \textsc{d3abs}  \textsc{hsy}  \textsc{a.hap.ir}-save \\
\glt `If a person \textbf{knows/recognizes} his/her sins and regrets/repents of his/her sins and receives the Lord, that one, they say, will be saved.’ [CBWN-C-23 4.3] \\\smallskip
*Gakaļa a ki kanen
\zlast\clearpage

\ea
Transitive dynamic (no examples in the corpus, though they do occur, rarely, in conversation): \\
\textbf{Pakaļa}  ko  gid  daw  kino  gapanaw  ya. \\\smallskip
\gll \textbf{Pa-kaļa}  ko  gid  daw  kino  ga-panaw  ya. \\
\textsc{t.r}-know/recognize  1\textsc{s.erg}  \textsc{int}  if/when  who  \textsc{i.r}-walk/go  \textsc{def.f} \\
\glt ‘I \textbf{intentionally recognized} who was the one walking.’
\z

\ea
\textbf{Kaļaen}  no  daw  kino  gapanaw  ya. \\\smallskip
\gll \textbf{Kaļa-en}  no  daw  kino  ga-panaw  ya. \\
know/recognize-\textsc{t.ir}  2\textsc{s.erg}  if/when  who  \textsc{i.r}-walk  \textsc{def.f} \\
\glt ‘You will intentionally recognize who is going.’
\z

\ea
Detransitive happenstantial: \\
\textbf{Nakakaļa}  a  din  bisan  lugay  uļa  gakitaay. \\\smallskip
\gll \textbf{Naka-kaļa}  a  din  bisan  lugay  uļa  ga-kita-ay. \\
\textsc{i.hap.r}-know/recognize  1\textsc{s.abs}  3\textsc{s.erg}  even.though  long.time  \textsc{neg.r}  \textsc{i.r}-see-\textsc{rec} \\
\glt ‘S/he recognized me even though we did not see each other for a long time.’
\z
\ea
Transitive happenstantial: \\
\textbf{Nakaļa}  din  nang  bayo  na  pasuot  ko  na  iya arey  ko. \\\smallskip
\gll \textbf{Na-kaļa}  din  nang  bayo  na  pa-suot  ko  na  iya arey  ko. \\
\textsc{a.hap.r}-know/recognize  3\textsc{s.erg}  only/just  clothes  \textsc{lk}  \textsc{t.r}-wear  1\textsc{s.erg}  \textsc{lk}  3\textsc{s.gen}
friend 1\textsc{s.gen} \\
\glt `S/he just \textbf{recognized} the clothes I was wearing as belonging to my friend.’
\z
\ea
\textbf{Nakaļa}  a  din. \\\smallskip
\gll \textbf{Na-kaļa}  a  din. \\
\textsc{a.hap.r}-know/recognize  1\textsc{s.abs}  3\textsc{s.erg} \\
\glt ‘S/he \textbf{recognized} me.’
\z
\ea
Unso  \textbf{nakaļa}  nay  isya  na  manugpalit ta  guso  na  anduni  may  basak  en  ta  Cavili  na  isla. \\\smallskip
\gll Unso  \textbf{na-kaļa}  nay  isya  na  manug-palit ta  guso  na  anduni  may  basak  en  ta  Cavili  na  isla. \\
\textsc{d4loc.pr}  \textsc{a.hap.r}-know/recognize  1\textsc{p.excl.erg}  one  \textsc{lk}  \textsc{nr.occ}-buy
\textsc{nabs}  seaweed  \textsc{lk}  now/today  \textsc{ext.in}  soil/land  \textsc{cm}  \textsc{nabs}  Cavili   \textsc{lk}  island \\
\glt `There we knew one buyer of agar seaweed who now has land on Cavili island.’ [DBWN-T-23 6.3]
\z
\ea
Transitive dynamic and happenstantial forms of \textit{kaļa} with the applicative are ungrammatical: \\
\textit{*Pakaļaan. *kaļaan. *nakaļaan. *makaļaan.}
\z

The verb \textit{pati} ‘to believe/obey’ may occur with transitive dynamic and happenstantial affixation with experiencer as ergative and stimulus as absolutive (exs. \ref{ex:mustbeshown}-\ref{ex:thenextyear}). It also may occur in detransitive dynamic frames with the experiencer as absolutive (\ref{ex:notbecomesick}). It does not occur in detransitive happenstantial constructions \REF{ex:nakapati}.

\ea
\label{ex:mustbeshown}
Transitive dynamic: \\
Uļa  man  don  nakita  daw  kino  na  mga  ittaw,  basta \textbf{papati}  na  yon  na  mga  lugar  may  mga  ittaw  don na  dapat  ipakita  gid  dya  bata  an. \\\smallskip
\gll Uļa  man  don  na-kita  daw  kino  na  mga  ittaw,  basta \textbf{pa-pati}  na  yon  na  mga  lugar  may  mga  ittaw  don na  dapat  i-pa-kita  gid  dya  bata  an. \\
\textsc{neg.r}  \textsc{emph}  \textsc{d3loc}  \textsc{a.hap.r}-see  if/when  who  \textsc{lk}  \textsc{pl}  person  just.so.that
\textsc{t.r}-believe/obey  \textsc{lk}  \textsc{d3adj}  \textsc{lk}  \textsc{pl}  place  \textsc{ext.in}  \textsc{pl}  person \textsc{d3loc}
\textsc{lk}  must  \textsc{t.deon}-\textsc{caus}-see  \textsc{int}  \textsc{d4loc}  child  \textsc{def.m} \\
\glt `Whoever the people are not seen there, just so that it \textbf{is believed} that in those places there are people there for whom the child must be shown.’ [JCWE-T-15 5.6]
\z
\ea
Miad  pa  daw  \textbf{papati}  ta  gina  danen  ya  papa ko  daw  mama  na  dili  ki  manaw  tak  Birnis  daw istasyon. \\\smallskip
\gll Miad  pa  daw  \textbf{pa-pati}  ta  gina  danen  ya  papa ko  daw  mama  na  dili  ki  m-panaw  tak  Birnis  daw istasyon. \\
good  \textsc{inc}  if/when  \textsc{t.r}-believe/obey  1\textsc{p.excl.erg}  earlier  3\textsc{p.abs}  \textsc{def.f}  papa
1\textsc{s.gen}  and  mama  \textsc{lk}  \textsc{neg.ir}  1\textsc{p.incl.abs}  \textsc{i.v.ir}-go/walk  because  Friday  and
holy.week \\
\glt `It was better if we \textbf{obeyed} earlier my papa and mama that we will not leave because it is Friday and holy week.’ [CBWN-C-11 4.22]
\z

\ea
Transitive dynamic with applicative (no examples in the corpus): \\
\textbf{Papatian}  din  ambaļ  ko  ya  ki  kanen. \\\smallskip
\gll \textbf{Pa-pati-an}  din  ambaļ  ko  ya  ki  kanen. \\
\textsc{t.r}-believe/obey-\textsc{apl}  3\textsc{s.erg}  say  1\textsc{s.erg}  \textsc{def.f}  \textsc{obl.p}  3s \\
\glt ‘S/he \textbf{believed/obeyed} what I said to him/her.’
\z

\ea
\textbf{Papatian}  a  din  daw  ino  ambaļ  ko. \\\smallskip
\gll \textbf{Pa-pati-an}  a  din  daw  ino  ambaļ  ko. \\
\textsc{t.r}-believe/obey-\textsc{apl}  1\textsc{s.abs}  3\textsc{s.erg}  sif/when  what  say  1\textsc{s.erg} \\
\glt ‘S/he \textbf{believed/obeyed} me whatever I say.’
\z

\ea
\textbf{Papatian}   din  na  yaken  i  mindeg  mayor  sunod na  taon. \\\smallskip
\gll \textbf{Pa-pati-an}   din  na  yaken  i  m-tindeg  mayor  sunod na  taon. \\
\textsc{t.r}-believe/obey-\textsc{apl}  3\textsc{s.erg}  \textsc{lk}  1\textsc{s.abs}  \textsc{def.n}  \textsc{i.v.ir}-stand  mayor  next \textsc{lk}  year \\
\glt ‘S/he \textbf{believed} that I will stand (candidate for) mayor in the next year.’
\z

\ea
Transitive happenstantial: \\
\textbf{Napati}  din  ambaļ  ko  ki  kanen. \\\smallskip
\gll \textbf{Na-pati}  din  ambaļ  ko  ki  kanen. \\
\textsc{a.hap.r}-believe/obey  3\textsc{s.erg}  say  1\textsc{s.erg}  \textsc{obl.p}  3s \\
\glt ‘S/he \textbf{had believed/obeyed} what I said to him/her.’
\z


\ea
\textbf{Napati}  din  na  yaken  i  mindeg  mayor  sunod na  taon.\\\smallskip
\gll \textbf{Na-pati}  din  na  yaken  i  m-tindeg  mayor  sunod na  taon.\\
\textsc{a.hap.r}-believe/obey  3\textsc{s.erg}  \textsc{lk}  1\textsc{s.abs}  \textsc{def.n}  \textsc{i.v.ir}-stand  mayor  next
\textsc{lk}  year \\
\glt ‘S/he \textbf{had believed} that I will stand (candidate for) mayor in the next year.’
\z

\ea
\textbf{Napati}  a  din  na  kanen  katagsa  ko. \\\smallskip
\gll \textbf{Na-pati}  a  din  na  kanen  katagsa  ko. \\
\textsc{a.hap.r}-believe/obey  1\textsc{s.abs}  3\textsc{s.erg}  \textsc{lk}  3\textsc{s.abs}  cousin  1\textsc{s.gen} \\
\glt ‘S/he had believed/obeyed me that s/he was my cousin.’
\z
\newpage
\ea
Transitive happenstantial with applicative (no examples in the corpus):
\textbf{Napatian}  din  ambaļ  ko  ya  ki  kanen. \\\smallskip
\gll \textbf{Na-pati-an}  din  ambaļ  ko  ya  ki  kanen. \\
\textsc{a.hap.r}-believe/obey-\textsc{apl}  3\textsc{s.erg}  say  1\textsc{s.erg}  \textsc{def.f}  \textsc{obl.p}  3s \\
\glt ‘S/he \textbf{had believed/obeyed} what I said to him/her.’
\z

\ea
\textbf{Napatian}  a  din  daw  ino  ambaļ  ko. \\\smallskip
\gll \textbf{Na-pati-an}  a  din  daw  ino  ambaļ  ko. \\
\textsc{a.hap.r}-believe/obey-\textsc{apl}  1\textsc{s.abs}  3\textsc{s.erg}  if/when  what  say  1\textsc{s.erg} \\
\glt ‘S/he \textbf{had believed/obeyed} me whatever I say.’
\z

\ea
\label{ex:thenextyear}
\textbf{Napatian}  din  na  yaken  i  mindeg  mayor sunod  na  taon. \\\smallskip
\gll \textbf{Na-pati-an}  din  na  yaken  i  m-tindeg  mayor sunod  na  taon. \\
\textsc{a.hap.r}-believe/obey-\textsc{apl}  3\textsc{s.erg}  \textsc{lk}  1\textsc{s.abs}  \textsc{def.n}  \textsc{i.v.ir}-stand  mayor next  \textsc{lk}  year \\
\glt `S/he \textbf{had believed} that I will stand (candidate for) mayor in the next year.’
\z

\ea
\label{ex:notbecomesick}
Detransitive dynamic: \\
Daw  isya  na  baļay  maimuan  en  ta  mikaw, \textbf{gapati}  kay  na  mga  ittaw  ta  uyi  na  baļay dili  en  magmasakit. \\\smallskip
\gll Daw  isya  na  baļay  ma-imu-an  en  ta  mikaw, \textbf{ga-pati}  kay  na  mga  ittaw  ta  u-yi  na  baļay dili  en  mag-masakit. \\
if/when  one  \textsc{lk}  house  \textsc{a.hap.ir}-do/make-\textsc{apl}  \textsc{cm}  \textsc{nabs}  food.offering
\textsc{i.r}-believe/obey  1\textsc{p.lexcl.abs}  \textsc{lk}  \textsc{pl}  person  \textsc{nabs}  \textsc{emph-d1adj}  \textsc{lk}  house
\textsc{neg.ir}  \textsc{cm}  \textsc{i.ir}-sick \\
\glt `If one house a food offering is done for it, we believe that the people of \textsc{this} house will not become sick.’ [CCWE-T-01 4.1]
\z

\ea
\label{ex:nakapati}
Detransitive happenstantial is ungrammatical: \\
\textit{*nakapati}
\z

The verb \textit{masmas} ‘to notice something’ only occurs with happenstantial affixes and has only an unintentional meaning. It can occur with the intransitive happenstantial affixes \textit{naka}- and \textit{maka}-, in which case the Cognizer is absolutive and the stimulus as non-absolutive. In an applicative construction, the Cognizer is ergative and the stimulus absolutive.

\ea
All dynamic forms of \textit{masmas} are ungrammatical: \\
\textit{*gamasmas *pamasmas  *pamasmasan  *masmasen}
\z

\ea
Detransitive happenstantial: \\
Nakamasmas	aren	en	daw	indi	baļay	danen. \\\smallskip
\gll Naka-masmas	aren	en	daw	indi	baļay	danen. \\
\textsc{i.hap.r}-notice	1\textsc{s.abs}	\textsc{cm}	if/when	where	house	3\textsc{p.gen} \\
\glt `I noticed where their house was.'
\z

\ea
External motivation: \\
\textbf{Kamasmas}  aren  daw  indi  dapit  baļay  danen. \\\smallskip
\gll \textbf{Ka-masmas}  aren  daw  indi  dapit  baļay  danen. \\
\textsc{i.exm}-notice  1\textsc{s.abs}  if/when  where  direction  house  3\textsc{p.gen} \\
\glt ‘I was \textbf{able to notice} where (lit. whatever direction) their house is.’
\z

\ea
Transitive happenstantial with obligatory applicative: \\
Naan  din  ya  en  \textbf{namasmasan}   na  pļaas  i  sikad  gwapo na  mama. \\\smallskip
\gll Naan  din  ya  en  \textbf{na-masmas-an}   na  pļaas  i  sikad  gwapo na  mama. \\
\textsc{spat.def}  3\textsc{s.erg}  \textsc{att}  \textsc{cm}  \textsc{a.hap.r}-notice-\textsc{apl}  \textsc{lk}  iguana  \textsc{def.n}  very  handsome
\textsc{lk}  man \\
\glt `It was there that she \textbf{noticed} that the lizard was a very handsome man.’ [NEWN-T-07 2.13]
\z

The word \textit{salig} ‘to think wrongly’ usually occurs as a bare root, and so may be thought of either as a “defective” verb, or something falling in between the verb and adverb categories. This is in contrast to the verbal root \textit{sálig} ‘to trust’ (prominence on first syllable). In example \REF{bkm:Ref123650588} it is clear that \textit{salig} 'to think wrongly' is functioning as a transitive predicate with an ergative Cognizer and a complement clause as the absolutive stimulus, though it carries no inflection. 

\ea
\label{bkm:Ref123650588}
\textbf{Salíg}   din  daw  galaeg-laeg  a  nang. \\\smallskip
\gll \textbf{Salíg}   din  daw  ga-laeg-laeg  a  nang. \\
think.wrongly  3\textsc{s.erg}  if/when  \textsc{i.r}-\textsc{red}-joke  1\textsc{s.abs}  only/just \\
\glt ‘She \textbf{thought wrongly} that I was just joking.’ [SLWN-C-01 8.7]
\z

The distinct root \textit{sálig} ‘to trust’ is a more prototypical verbal root. It occurs in many of the Inflectional forms common to Cognition predicates.

\ea
Detransitive dynamic: \\
Apang  na  \textbf{gasálig}  kay  gid  na  paagi  ta  imo  na  pagtabang adlaw-adlaw,  mga  prublima  nay  dili  mabeg-at   bagaen. \\\smallskip
\gll Apang  na  \textbf{ga-sálig}  kay  gid  na  paagi  ta  imo  na  pag-tabang adlaw-adlaw,  mga  prublima  nay  dili  ma-beg-at   baga-en. \\
but  \textsc{lk}  \textsc{i.r}-trust  1\textsc{p.excl.abs}  \textsc{int}  \textsc{lk}  by.means  \textsc{nabs}  2\textsc{s.gen}  \textsc{lk}  \textsc{nr.act}-help
\textsc{red}-sun/day  \textsc{pl}  problem  1\textsc{p.excl.gen}  \textsc{neg.ir}  \textsc{adj}-heavy  shoulder-\textsc{t.ir} \\
\glt `But we are really trusting that by the means of your help everyday, our problems will not be burdensome to shoulder.’ (This is part of a prayer.)  [RNWB-L-01 1.6]
\z

\ea
Transitive dynamic: \\
\textbf{Pasalig}  ko  ki  kanen  tanan  na  ubra. \\\smallskip
\gll \textbf{Pa-salig}  ko  ki  kanen  tanan  na  ubra. \\
\textsc{t.r}-trust  1\textsc{s.erg}  \textsc{obl.p}  3s  all  \textsc{lk}  work \\
\glt ‘I \textbf{trusted} to him/her all the work.’
\z

\ea
Transitive dynamic with applicative: \\
\textbf{Pasaligan}  ko  kanen  an  ta  tanan  na  ubra. \\\smallskip
\gll \textbf{Pa-salig-an}  ko  kanen  an  ta  tanan  na  ubra. \\
\textsc{t.r}-trust-\textsc{apl}  1\textsc{s.erg}  3\textsc{s.abs}  \textsc{def.m}  \textsc{nabs}  all  \textsc{lk}  work \\
\glt ‘I \textbf{trusted} him/her with all the work.’
\z

The root \textit{tuo} ‘to believe in someone’ only occurs in detransitive frames, with dynamic affixes. In the texts the apparently dynamic transitive applicative form \textit{patuuan} is used as a noun referring to ‘something believed in’, but not as a verbal predicate. It is also ungrammatical with happenstantial Inflections:

\ea
Piro  kalabanan  ta  Kagayanen  naan  ta  Manila  \textbf{gatuo} ta  Dios  o  gasirbi  ta  Dios … \\\smallskip
\gll Piro  ka-laban-an  ta  Kagayanen  naan  ta  Manila  \textbf{ga-tuo} ta  Dios  o  ga-sirbi  ta  Dios … \\
but  \textsc{nr}-most-\textsc{nr}  \textsc{nabs}  Kagayanen  \textsc{spat.def}  \textsc{nabs}  Manila  \textsc{i.r}-believe.in
\textsc{nabs}  God  or  \textsc{i.r}-serve  \textsc{nabs}  God \\
\glt `But almost all of the Kagayanens in Manila \textbf{believe in} God or serve God …’ [BCWL-T-11 2.4]
\z

The following are examples of the verb \textit{kullaw} ‘to ponder over, wonder about, try to figure out’ (Class Vb).

\ea
Detransitive dynamic: \\
Nanay  din  i  \textbf{gakullaw}  man-o  tak  uļa  pa  guli  iya  na bata. \\\smallskip
\gll Nanay  din  i  \textbf{ga-kullaw}  man-o  tak  uļa  pa  ga-uli  iya  na bata. \\
mother  3\textsc{s.gen}  \textsc{def.n}  \textsc{i.r}-ponder  why  because  \textsc{ineg.r}  \textsc{inc}  \textsc{i.r}-go.home  3\textsc{s.gen}  \textsc{lk} child \\
\glt `Her mother was ponderng why her child did not yet come home.’ [VAWN-T-20 4.1]
\z
\ea
Detransitive happenstantial \textit{na}{}- with experiencer as absolutive: \\
\textbf{Nakullaw}  a  gid  na  gatambek  ka.  Bagnes  ka  pla. \\\smallskip
\gll \textbf{Na-kullaw}  a  gid  na  ga-tambek  ka.  Bagnes  ka  pla. \\
\textsc{a.hap.r}-ponder  1\textsc{s.abs}  \textsc{int}  \textsc{lk}  \textsc{i.r}-fat  2\textsc{s.abs}  pregnant  2\textsc{s.abs}  \textsc{surp} \\
\glt ‘I had \textbf{pondered} that you became fat. You are pregnant!’
\z
\ea
Transitive happenstantial with applicative: \\
  … \textbf{nakullawan}  man  en  ta  ittaw  na  lain  en man  iya  na  pang-uļag. \\\smallskip
\gll  … \textbf{na-kullaw-an}  man  en  ta  ittaw  na  lain  en man  iya  na  pang-uļag. \\
{} \textsc{a.hap.r}-ponder-\textsc{apl}  \textsc{emph}  \textsc{cm}  \textsc{nabs}  person  \textsc{lk}  different  \textsc{cm} \textsc{emph}  3\textsc{s.gen}  \textsc{lk}  movement \\
\glt ` … the people \textbf{pondered} that her movements were different.’ [JCWN-T-26 21.1]
\z

\ea
Transitive dynamic with and without applicative and transitive happenstantial are all ungrammatical: \\
\textit{*pakullawan ko *pakullaw ko  *nakullaw ko *makullaw ko}  
\z

The root \textit{tuļtoļ} ‘to know the way’ takes both happenstantial and dynamic affixes. In happenstantial it takes transitive, transitive plus applicative and detransitive. But in dynamic modality, it can only take the realis transitive.

\ea
Transitive dynamic: \\
Patuļtoļ	din	gid	daļan	ya. \\\smallskip
\gll Pa-tuļtoļ	din	gid	daļan	ya. \\
\textsc{t.r}-know.the.way	3\textsc{s.erg}	\textsc{int}	road	\textsc{def.f} \\
\glt ’S/he knows the way to the road.’
\z 

\ea
Transitive dynamic with applicative: \\
a. *\textit{Patuļtuļan din gid daw indya daļan ya}. \\\smallskip
Detransitive dynamic: \\
b. \textit{*Gatuļtoļ a daw indya daļan na munta iran}. \\
\z

\ea
Transitive happenstantial: \\
\textbf{Natuļtoļ}	din	en	daļan	ya	na	munta 	baļay	danen. \\\smallskip
\gll \textbf{Na-tuļtoļ}	din	en	daļan	ya	na	munta 	baļay	danen. \\
\textsc{a.hap.r}-know.the.way	3\textsc{s.erg}	\textsc{cm}	road	\textsc{def.f}	\textsc{lk}	going	house	3\textsc{p.gen} \\
\glt ’S/he knows the way to the road that goies to their house.’
\z

\ea
Transitive happenstantial with applicative: \\
\textbf{Natuļtuļan}	din	baļay	nay	an. \\\smallskip
\gll \textbf{na-tuļtoļ-an}	din	baļay	nay	an. \\
\textsc{a.hap.r}-know.the.way-\textsc{apl} 3\textsc{s.erg} house	1\textsc{p.excl.gen}	\textsc{def.m} \\
\glt ‘S/he knows the way to our house.’
\z

\ea
Detransitive happenstantial: \\
Kino	nakatuļtoļ	dya	ta	baļay	danen? \\\smallskip
\gll Kino	naka-tuļtoļ	dya	ta	baļay	danen? \\
who		\textsc{i.hap.r}-know.the.way	\textsc{d4loc}	\textsc{nabs}	house	3\textsc{p.gen} \\
\glt ‘Who knows the way to their house?’
\z

\ea
Kanen	nang	nakatuļtoļ	ta	baļay	ya. \\\smallskip
\gll Kanen	nang	naka-tuļtoļ	ta	baļay	ya. \\
3\textsc{s.abs}	only/just	\textsc{i.hap.r}-know.the.way	\textsc{nabs}	house	\textsc{def.f} \\
\glt ‘S/he only knows the way to the house.’
\z

\ea
External motivation: \\
\textbf{Katuļtoļ}	a	ta	baļay	danen	tak	sigi	a	iling	dya. \\\smallskip
\gll \textbf{Ka-tuļtoļ}	a	ta	baļay	danen	tak	sigi	a	iling	dya. \\
\textsc{i.exm}-know.the.way	1\textsc{s.abs}	\textsc{nabs}	house	3\textsc{p.gen}	because	continually	1\textsc{s.abs}	go	\textsc{d4loc} \\
\glt ‘I am able to know the way to their house because I keep on going there.’
\z

\newpage
\ea
Piro	uļa	kay	kami	ame	gakwa diskuan	tak	dili	kay	katuļtoļ	ta	diskuan naan	Iloilo. \\\smallskip
\gll Piro	uļa	kay	kami	ame	ga-kwa disko-an	tak	dili	kay	ka-tuļtoļ	ta	disko-an naan	Iloilo. \\
but		\textsc{neg.r}	1\textsc{p.excl.abs}	1\textsc{p.excl.abs}	1\textsc{p.excl.gen}	\textsc{i.r}-whatchmacallit
disco-\textsc{nr}	because	\textsc{neg.ir}	1\textsc{p.excl.abs}	\textsc{i.exm}-know.the.way	\textsc{nabs}	disco-\textsc{nr} \textsc{spat.def}	Iloilo \\
\glt ‘We did not whatchmacallit (go) to the disco because we did not know the way to the disco in Iloilo.’ [DBON-C-07 2.9]
\z

\section{Situations of needing and wanting}
\label{sec:needingwanting}

There are at least four roots that often occur as predicates without inflectional affixes. All of these express situations that can be characterized as wanting, needing or liking. These are the following:

\ea
\begin{tabbing}
\hspace{2cm} \= \kill
\textit{liag} \> ‘to like/want’ \\
\textit{gusto} \> ‘to like/want’ \\
\textit{ilig} \>  'to be fond of/prone to' \\   
\textit{kinangļan} \> ‘to need’
\end{tabbing}
\z

For the most part when these roots serve as the main predicating word, the argument structure frame is transitive, with an ergative experiencer, and an absolutive stimulus (the thing wanted, liked or needed). For all these concepts, the stimulus is either a referential expression or a complement clause.

\ea
Tigbasen  no  inay  tak  \textbf{liag}  no  bataan  din  an. \\\smallskip
\gll Tigbas-en  no  inay  tak  \textbf{liag}  no  bata-an  din  an. \\
chop-\textsc{t.ir}  2\textsc{s.erg}  mother  because  want  2\textsc{s.erg}  child-\textsc{nr}  3\textsc{s.gen}  \textsc{def.m} \\
\glt ‘Chop the mother because you \textbf{want} her children.’ (This is a riddle and the answer is a banana plant.) [TROR-T-01 9.1l]
\z

\ea
\textbf{Liag}  ko  na  makita  ko  a  mga  Kagayanen  na  ta  pila adlaw  makadilanto  ta  pagpangabui  tenged  ta  idukasyon. \\\smallskip
\gll \textbf{Liag}  ko  na  ma-kita  ko  a  mga  Kagayanen  na  ta  pila adlaw  maka-dilanto  ta  pag-pangabui  tenged  ta  idukasyon. \\
want  1\textsc{s.erg}  \textsc{lk}  \textsc{a.hap.ir}-see  1\textsc{s.erg}  \textsc{inj}  \textsc{pl}  Kagayanen  \textsc{lk}  \textsc{nabs}  few
sun/day  \textsc{i.hap.ir}-successful  \textsc{nabs}  \textsc{nr.act}-living  because  \textsc{nabs}  education \\
\glt `I \textbf{want} to see soon that Kagayanens can become successful in living because of education.’ [JCWB-T-12 12.1]
\z

As with other complement taking predicates, pronouns in the complement clause may be omitted when coreferential with the actor/experiencer \REF{bkm:Ref122109526}.

The roots \textit{gusto} and \textit{liag} seem to be synonymous. Further research may reveal some slight semantic differences. \textit{Gusto} is a common word in the Philippines that derives from the Spanish verb \textit{gustar} `to please'. 

\ea
Ee,  \textbf{gusto}  ko  ino  na  kundisyon. \\\smallskip
\gll Ee,  \textbf{gusto}  ko  ino  na  kundisyon. \\
yes  want  1\textsc{s.erg}  what  \textsc{lk}  condition. \\
\glt ‘Yes, I want whatever the condition.’ (This means something one has to do to get what they want, no matter what.) [CBWN-C-17 3.13]
\z

\ea
 … tak  nļaman  nyo  daw  may  pangita isya  na  ittaw  \textbf{gusto}  din  mamang  kanen  ta  iya  na  pirsinto. \\\smallskip
\gll  … tak  na-aļam-an  nyo  daw  may  pa-ng-ngita isya  na  ittaw  \textbf{gusto}  din  m-kamang  kanen  ta  iya  na  pirsinto. \\
{} because  \textsc{a.hap.r}-know-\textsc{apl}  2\textsc{p.erg}  if/when  \textsc{ext.in}  \textsc{t.ir-pl}-look.for
one  \textsc{lk}  person  want  3\textsc{s.erg}  \textsc{i.v.ir}-get  3\textsc{s.abs}  \textsc{nabs}  3\textsc{s.gen}  \textsc{lk}  percent \\
\glt ` …because you know that if a person is making a living (lit. looking for something), he wants that he will get his percentage.’ [ETON-C-06 2.1]
\z

Example \REF{ex:beobeyedimmediately} illustrates the root \textit{gusto} with a complement clause as the absolutive:

\ea
\label{ex:beobeyedimmediately}
Daw mugo ta iya na duma gusto din na pagsugo din patien dayon. \\\smallskip
\gll Daw m-sugo ta iya na duma gusto din na pag-sugo din pati-en dayon. \\
if/when \textsc{i.v.ir}-order	\textsc{nabs}	3\textsc{s.gen}	\textsc{lk}	companions	want
3\textsc{s.erg}	\textsc{lk} 	\textsc{nr.act}-order	3\textsc{s.gen}	obey/believe-\textsc{t.ir}
immediately \\
\glt 'If/When (he) orders his companions (to do something), he wants his order to be obeyed immediately.' [CBWN-C-12]
\z

The root \textit{liag} has approximately the same range of usages as \textit{gusto}, though \textit{gusto} is more frequent in the corpus:

\ea
\label{bkm:Ref122109526}
\textbf{Liag}  din  na  magkaan ta  sidda  na  dilis. \\\smallskip
\gll \textbf{Liag}  din  na  mag-kaan ta  sidda  na  dilis. \\
want  3\textsc{s.erg}  \textsc{lk}  \textsc{i.ir}-eat \textsc{nabs}  fish  \textsc{lk}  anchovies. \\
\glt ‘She \textbf{wanted} to eat anchovies.’ [VAOE-J-05 2.4]
\z

The root \textit{kinangļan} is the most common means of expressing the idea of ‘to need'.

\ea
Na  daw  \textbf{kinangļan}  ko  mga  upisyalis,  tallo  bagting yon  umaw  ko  ki  danen. \\\smallskip
\gll Na  daw  \textbf{kinangļan}  ko  mga  upisyalis,  tallo  bagting yon  umaw  ko  ki  danen. \\
\textsc{lk}  if/when  need  1\textsc{s.erg}  \textsc{pl}  officials  three  bell.ring
\textsc{d3abs}  call  1\textsc{s.erg}  \textsc{obl.p}  3p \\
\glt `Then when I need officials, three bell rings, that is my calling them.’ [JCWN-T-21 14.8]
\z
\ea
Ambaļ  din  \textbf{kinangļan}  kon  danen  na  miling  a  ta  Manila tak  magtudlo  a  ta  ambaļ  ta  na  Kagay-anen. \\\smallskip
\gll Ambaļ  din  \textbf{kinangļan}  kon  danen  na  m-iling  a  ta  Manila tak  mag-tudlo  a  ta  ambaļ  ta  na  Kagay-anen." \\
say  3\textsc{s.erg}  need  \textsc{hsy}  3\textsc{p.erg}  \textsc{lk}  \textsc{i.v.ir}-go  1\textsc{s.abs}  \textsc{nabs}  Manila
because  \textsc{i.ir}-teach  1\textsc{s.abs}  \textsc{nabs}  say  1\textsc{p.incl.gen}  \textsc{lk}  Kagayanen \\
\glt `She said that they need that I go to Manila because I will teach our language Kagayanen.’ [RCON-L-03 5.4]
\z

\section{Situations of naming and forming: Secondary predicates}
\label{sec:secondarypredicates}

There are a few verbs that may take a “stripped” noun complement preceded by the linker \textit{na}. A stripped noun is one that is devoid of any indications of case marking or definiteness (see \cite{miner1986}). In general these constructions express situations of naming and forming, and can be represented in the schematic given in \REF{ex:secondarypredicatestemplate}, where SN is the stripped noun:

\ea
\label{ex:secondarypredicatestemplate}
VERB-TR (ERG) ABS na SN
\z

The Verb in this construction only occurs in transitive (TR) forms. Often, however, there is no overt ergative, especially with the inflected verbs \textit{piling}, \textit{pangaranan}, \textit{paumaw}, and \textit{pambaļ} all of which have meanings related to naming.

We consider the \textit{na}+stripped noun component of this construction to be a secondary predicate because the semantic structure involves two predicates: X did something to Y (primary predicate), and Y is/becomes something (secondary predicate). The secondary predicate may be a result of the primary predicate, or a depiction presented in the primary predicate. These are sometimes called \textit{resultative secondary predicates}\is{resultative secondary predicates} and \textit{depictive secondary predicates}\is{depictive secondary predicates} respectively \citep{rothstein2012}.

The following are the verbs that are most commonly found in this construction: 

\ea
\begin{tabbing}
\hspace{1.5cm} \= \hspace{2cm} \= \kill  
\textit{buat} \> \textnormal{‘to make/do’} \>	\textnormal{Resultative} \\
\textit{imo} \> \textnormal{‘to make/do’} \>	\textnormal{Resultative} \\
\textit{ngaran} \> \textnormal{‘to name’} \>	\textnormal{Resultative or depictive} \\
\textit{pili} \> \textnormal{‘to choose’}	\>	\textnormal{Resultative} \\
\textit{umaw} \> \textnormal{‘to call’} 	\>	\textnormal{Depictive} \\
\textit{ambaļ} \> \textnormal{‘to say’} \>		\textnormal{Depictive} \\
\textit{iling} \> \textnormal{‘to say’} \> \textnormal{Depictive} \\
\end{tabbing}
\z

The following examples illustrate this construction from conversations and the corpus. The ungrammatical examples illustrate the impossibility of this construction in a detransitive frame:

\ea
\textit{buat} ‘to make/do’, resultative: \\
Pabuat	din	pantad	an	na	bula. \\\smallskip
\gll Pa-buat	din	pantad	an	na	bula. \\
\textsc{t.r}-make/do	3\textsc{s.erg}	sand	\textsc{def.m}	\textsc{lk}	ball \\
\glt ‘S/he made the sand into a ball.’ \\\smallskip
\textit{*Gabuat kanen ta pantad na bula.}
\z

\ea
Gamasakit	an	pasugo	na	muli	ta	Sintro	tak	iskwilaan ta	High School	pabuat	na	Imirginsi	Hospital. \\\smallskip
\gll Ga-masakit	an	pa-sugo	na	m-uli	ta	Sintro	tak	iskwila-an ta	High School	pa-buat	na	Imirginsi	Hospital. \\
\textsc{i.r}-sick	\textsc{def.m}	\textsc{t.r}-order	\textsc{lk}	\textsc{i.v.ir}-go.home	\textsc{nabs}	Central	because	school-\textsc{nr} \textsc{nabs}	High School	\textsc{t.r}-make/do	\textsc{lk}	emergency	hospital \\
\glt ‘The ones who are sick were ordered to go to Central because the school building of the high school was made into an emergency hospital.’ [JCWN-T-21 8.8]
\z

Example \REF{ex:ricecake} illustrates that the secondary predicate, \textit{na puto}, may precede the absolutive argument, \textit{beggas an} in this example:

\ea
\label{ex:ricecake}
\textit{imo} ‘to make/do’, resultative: \\
Paimo	din	na	puto	beggas	an. \\\smallskip
\gll Pa-imo	din	na	puto	beggas	an. \\
\textsc{t.r}-make/do	3\textsc{s.erg}	\textsc{lk}	rice.cake 	uncooked.rice	\textsc{def.m} \\
\glt ‘S/he made the rice into rice cake.’ \\\smallskip
\textit{*Gaimo kanen ta beggas na puto.}
\z

\ea
Kanen	paimo	na	manugdrawing	ta	mga	tanem	daw	mga	ayep	daw	yupan. \\\smallskip
\gll Kanen	pa-imo	na	manug-drawing	ta	mga	tanem	daw	mga	ayep	daw	yupan. \\
3\textsc{s.abs}	\textsc{t.r}-make/do	\textsc{lk}	\textsc{nr}-drawing	\textsc{nabs}	\textsc{pl}	plant	and	\textsc{pl}	animal	and	bird \\
\glt ‘As for him, (he) was made a drawer (i.e., one who draws) of plants and animals and birds.’ [JCWB-T-12 5.13] \\\smallskip
*\textit{Gaimo danen ki kanen na manugdrawing ta mga tanem daw mga ayep daw yupan}. \\
\z

\ea
\textit{piling} ‘to call X something’, depictive: \\
Uwi	na	simbaan	i	piling	na	simbaan	datas. \\\smallskip
\gll U-wi	na	simba-an	i	pa-iling	na	simba-an	datas. \\
\textsc{emph-d1adj}	\textsc{lk}	worship-\textsc{nr}	\textsc{def.n}	\textsc{t.r}-say	\textsc{lk}	worship-\textsc{nr}	high \\
\glt ‘This very church is called church on high.’ [MEWN-T-03 2.12]
\z

\ea
Uwi	na	simbaan	i	piling	nay	na	simbaan	datas. \\\smallskip
\gll U-wi	na	simba-an	i	pa-iling	nay	na	simba-an	datas. \\
\textsc{emph-d1adj}	\textsc{lk}	worship-\textsc{nr}	\textsc{def.n}	\textsc{t.r}-say	1\textsc{p.excl.erg}	\textsc{lk}	worship-NR	high \\
\glt ‘This very church we called church on high.’ \\\smallskip
*\textit{Giling kami ta yi na simbaan na simbaan datas}.
\z

\ea
\textit{pangaranan} ‘to name X something’, resultative: \\
Pangaranan	din	bata	din	na	Pedro. \\\smallskip
\gll Pa-ngaran-an	din	bata	din	na	Pedro. \\
\textsc{t.r}-name-\textsc{apl}	3\textsc{s.erg}	child	3\textsc{s.gen}	\textsc{lk}	Pedro \\
\glt ‘S/he named his/her child Pedro.’ \\\smallskip
*\textit{Gangaran kanen ta bata din na Pedro}.
\z

\ea
Ambaļ	ta	nanay	para	dili	din	kon	malipatan	iya	na bata	na	Pinang,	miad	pa	kon	na	pangaranan	din	na	pinya yi	na	tanem	na	gaanggid	ta	ngaran	ta	iya	na	bata	na	napatay. \\\smallskip
\gll Ambaļ	ta	nanay	para	dili	din	kon	ma-lipat-an	iya	na bata	na	Pinang,	miad	pa	kon	na	pa-ngaran-an	din	na	pinya yi	na	tanem	na	ga-anggid	ta	ngaran	ta	iya	na	bata	na	na-patay. \\
say	\textsc{nabs}	mother	purpose	\textsc{neg.ir}	3\textsc{s.erg}	\textsc{hsy}	\textsc{a.hap.ir}-forget-\textsc{apl}	3\textsc{s.gen}	\textsc{lk} child	\textsc{lk}	Pinang	good	\textsc{emph}	\textsc{hsy}	\textsc{lk}	\textsc{t.r}-name-\textsc{apl}	3\textsc{s.erg}	\textsc{lk}	pineapple \textsc{d1adj}	\textsc{lk}	plant	\textsc{lk}	\textsc{i.r}-similar	\textsc{nabs}	name	\textsc{nabs}	3\textsc{s.gen}	\textsc{lk}	child	\textsc{lk}	\textsc{a.hap.r}-dead \\
\glt ‘The mother said that in order not to forget her child Pinang, it is better that she names this plant pinya that is similar to the name of her child who died.’ (This is a story about where the name \textit{pinya} ‘pineapple’ came from.) [VAWN-T-20 5.5]
\z

\ea
\textit{umaw} ‘to call X something’, depictive: \\
Paumaw	nay	iran	i	na	sakayan	na	paraw. \\\smallskip
\gll Pa-umaw	nay	iran	i	na	sakay-an	na	paraw. \\
\textsc{t.r}-call	1\textsc{p.excl.erg}	3\textsc{p.gen}	\textsc{def.n}	\textsc{lk}	ride-\textsc{nr}	\textsc{lk}	outrigger.canoe \\
\glt ‘We call what they rode an outrigger canoe.’ \\\smallskip
*Gaumaw kanen ta iran na sakayan na paraw.
\z

\ea
Iran	i	na	sakayan	paumaw	nay	na	paraw. \\\smallskip
\gll Iran	i	na	sakayan	pa-umaw	nay	na	paraw. \\
3\textsc{p.gen}	\textsc{def.n}	\textsc{lk}	ride-\textsc{nr}	\textsc{t.r}-call	1\textsc{p.excl.erg}	\textsc{lk}	outrigger.canoe \\
\glt ‘What they rode we call outrigger canoe.’ [CBWN-C-12 2.7]
\z

\ea
Naan	ta	baybay	ta	yi	na	lugar	may	bakod	na	kaoy	na	gagabong iya	na	mga	daon	daw	yi	na	kaoy	paumaw	nay	na	Buton. \\\smallskip
\gll Naan	ta	baybay	ta	yi	na	lugar	may	bakod	na	kaoy	na	ga-gabong iya	na	mga	daon daw	yi	na	kaoy	pa-umaw	nay	na	Buton. \\
\textsc{spat.def}	\textsc{nabs}	beach	\textsc{nabs}	\textsc{d1adj}	\textsc{lk}	place	\textsc{ext.in}	big	\textsc{lk}	tree	\textsc{lk}	\textsc{i.r}-thick.foliage
3\textsc{s.gen}	\textsc{lk}	\textsc{pl}	leaf	and	\textsc{d1adj}	\textsc{lk}	tree	\textsc{t.r}-call	1\textsc{p.excl.erg}	\textsc{lk}	Botong \\
\glt ‘On the beach of this place there is a big tree that has thick leaves, and this tree we call (it) Buton (English Botong or Barringtonia Asiatica).’ [CBWN-C-20 2.5]
\z

\ea
\textit{pambaļ} ‘to call X something’, depictive: \\
Apang,	daw	isya	na	ittaw	nakaangken	ta	paambaļ	na	moral values, makaambaļ	ki	na	yi	na	ittaw	nakapanubli	ta	pambaļ ta	na	panublien	ta	kultura	na	Kagayanen. \\\smallskip
\gll Apang,	daw	isya	na	ittaw	naka-angken	ta	pa-ambaļ	na	moral values, maka-ambaļ	ki	na	yi	na	ittaw	naka-panubli	ta	pa-ambaļ ta	na	panubli-en	ta	kultura	na	Kagayanen. \\
but	if/when	one	\textsc{lk}	person	\textsc{i.hap.r}-acquire	\textsc{nabs}	\textsc{t.r}-say	\textsc{lk}	moral values \textsc{i.hap.ir}-say	1\textsc{p.incl.abs}	\textsc{lk}	\textsc{d1adj}	\textsc{lk}	person	\textsc{i.hap.r}-inherit	\textsc{nabs}	\textsc{t.r}-say 1\textsc{p.incl.erg}	\textsc{lk}	inherit-\textsc{nr}	\textsc{nabs}	culture	\textsc{lk}	Kagayanen \\
\glt ‘But, when one person is able to acquire what is called moral values, we can say that this person has been able to inherit what we call the inheritance of the Kagayanen culture.’ [JCOB-L-02 4.4]
\z

\ea
\textit{buto} ‘to vote/elect X something’, resultative: \\
Pabuto	danen	Maria	an	na	mayor. \\\smallskip
\gll Pa-buto	danen	Maria	an	na	mayor. \\
\textsc{t.r}-vote	3\textsc{p.erg}	Maria	\textsc{def.m}	\textsc{lk}	mayor \\
\glt ‘They voted Maria mayor.’ \\\smallskip
*\textit{Gabuto danen an ki Maria na mayor}.
\z

Example \REF{ex:leaderoftheirgroup} illustrates that the stripped noun complement may accept certain kinds of modification, in this case a genitive phrase meaning ‘of their group’. However, case markers or determiners do not appear.

\ea
\label{ex:leaderoftheirgroup}
Papili	a	danen	na	pangulo	ta	iran	na	grupo. \\\smallskip
\gll Pa-pili	a	danen	na	pangulo	ta	iran	na	grupo. \\
\textsc{t.r}-choose	1\textsc{s.abs}	3\textsc{p.erg}	\textsc{lk}	leader	\textsc{nabs}	3\textsc{p.gen}	\textsc{lk}	group \\
\glt ‘They chose me as leader of their group.’
\z

It is worth noting here that with some roots a full complement clause may occur in the same role in a sentence as a secondary predicate \REF{ex:tobecomemayor-1}. In this case, a detransitive construction is possible \REF{ex:tobecomemayor-2}:  

\ea
\label{ex:tobecomemayor-1}
Pabuto	danen	Maria	an	na	mangin 	mayor. \\\smallskip
\gll Pa-buto	danen	Maria	an	na	mangin 	mayor. \\
\textsc{t.r}-vote	3\textsc{p.erg}	Maria	\textsc{def.m}	\textsc{lk}	\textsc{ir}+become\footnotemark{}	mayor \\
\footnotetext{The verb \textit{mangin/nangin} `become' is irregular. It only occurs in these two forms \textit{mangin} `\textsc{ir}{+}become' and \textit{nangin} `\textsc{r}+became'.}
\glt ‘They voted Maria to become mayor.’
\z

\ea
\label{ex:tobecomemayor-2}
Gabuto	danen	an	ki	Maria	na	mangin	mayor. \\\smallskip
\gll Ga-buto	danen	an	ki	Maria	na	mangin	mayor. \\
\textsc{i.r}-vote	3\textsc{p.abs}	\textsc{def.m}	\textsc{obl.p}	Maria	\textsc{lk}	\textsc{ir}+become	mayor \\
\glt ‘They voted Maria to become mayor.’
\z

Such constructions are discussed in more detail in \chapref{chap:clausecombining} on clause combining. It is also worth noting that the element following \textit{na} in a secondary predicate is always understood to be a nominal, rather than a property concept word. Ideas expressed in attributive secondary predicates in other languages, such as \textit{she painted the fence white}, or \textit{that made him happy} are expressed in other ways in Kagayanen. 

\section{Ideophonic verbs}
\label{sec:ideophonicverbs}

In this section we discuss a few special verb roots that we designate as “ideophonic verbs.” These are roots for which the sound imitates the meaning in some way. These are regular roots that can and usually do take verbal affixes. They are not simply interjections or onomatopoeic expressions. Here is a non-exhaustive list of ideophonic verbs in Kagayanen, followed by a few examples from the corpus.

\ea
\begin{tabbing}
\hspace{2.4cm} \= \kill
\textit{dagbeng}  \>   The loud thud of a person or a tree that falls, the sound \\
\> of thunder. \\
\textit{dagkoļ} \>  The sound of things hitting the floor especially if \\
\> one is downstairs and hears something fall upstairs. \\
\textit{daguok}  \>  The loud crunching sound of something hitting \\
\> something else like the sound of waves hitting cliffs. \\ \> Also the crunching sound of someone eating crunchy \\ \> food like potato chips. \\
\textit{kaļas}  \>      A rustling sound when dry leaves are blown by the  wind \\
\> or when dried buri and pandan leaves rub together \\
\> when woven. Also waves of the sea, or rain on grass. \\
\textit{kaļkaroļ} \>    The clanging sound of pots and pans. \\
\textit{lagamak}  \>   The sound of tearing cloth, the sound of trees falling \\
\> or wood creeking. \\
\textit{laggong}   \>  A low pitch sound. \\
\textit{lagpak}   \>  The slapping sound of shoes on the ground, or when \\
\>jumping into water or other things. \\
\textit{lagpi}    \>   To slap lightly. \\
\textit{lagpok}   \>  To explode, the sound of an explosion. \\
\textit{lagtok/lagatok} \>  The snap, crackle, pop sound as fingernails weaving \\
\> a mat with \textit{pandan} and \textit{buri} leaves. \\
\textit{lagtik/lagatik}   \>  A popping or snapping sound—a softer and higher pitch \\
\> sound than \textit{lagtok}. \\
\textit{ļaļļa}  \>     For the tongue to stick out or the sound of delirious \\
\> moaning, crying or screaming. \\
\textit{legemek}   \>  The crunchy sound of chewing when eating hard or \\ \> crispy foods. \\
\textit{legsek}    \>   To smash or crush something. \\
\textit{pakpak}   \>  To slap or thump on something with open hand or to \\
\> crush with a heavy object like a stone. \\
\textit{pakpak} \>   Wings of bird or airplane. The sound of wings. \\
\textit{pukpok}  \>   To pound on something with an object like a hammer. \\
\textit{sagaid}   \>  To rub against something. \\
\textit{sagilit}    \>   A buzzing or humming sound. \\
\textit{tagabļong} \>    A high small ping sound. \\
\textit{taktak}      \> To shake out something, or to spread it out. \\
\textit{taliktik}   \>   Light rain, drizzle, sprinkling rain. \\
\textit{tampa}    \>   To slap the face but not hard. \\
\textit{tampaling}  \>  To slap in the face hard with a smacking sound. \\
\textit{tuktok}    \>  To tap, knock or peck. \\
\textit{wawwaw}   \>  To whisper. This is how it sounds to others when \\
\> people are whispering.
\end{tabbing}
\z

The following are a few examples of ideophonic verbs in context:

\newpage
\ea
\textbf{Patuktok} ta manok ame na binaļad na ummay. \\\smallskip
\gll \textbf{Pa-tuktok} ta manok ame na b<in>aļad na ummay. \\
\textsc{t.r}-knock/peck	\textsc{nabs}	chicken	1\textsc{p.excl.gen}	\textsc{lk}	<\textsc{nr.res}>sun.dry	\textsc{lk}	unmilled.rice \\
\glt `The chicken pecked our sun-dried unmilled rice.'
\z

\ea
Uļa  lugay  may  \textbf{gatuktok}  ta  kawil  ko. \\\smallskip
\gll Uļa  lugay  may  \textbf{ga-tuktok}  ta  kawil  ko. \\
\textsc{neg.r}   long.time  \textsc{ext.in}  \textsc{i.r}-knock/peck  \textsc{nabs}  fishhook  1\textsc{s.gen} \\
\glt ‘Not a long time some (fish) knocked on my fishhook.’ [JCWN-L-33 23.1]
\z

\ea
Yaken  matay.  Daw  may  \textbf{gadagbeng}  na  nuļog,  isipen no tak  yon nan  niog  na  naipo ko. \\\smallskip
\gll Yaken  m-katay.  Daw  may  \textbf{ga-dagbeng}  na  na-uļog,  isip-en no tak  yon nan  niog  na  na-ipo  ko.\\
1\textsc{s.abs}  \textsc{i.v.r}-climb  if/when  \textsc{ext.in}  \textsc{i.r}-thud  \textsc{lk}  \textsc{a.hap.r}-fall  think-\textsc{t.ir} 2\textsc{s.erg} because  \textsc{d3abs} \textsc{d3pr}  coconut  \textsc{lk}  \textsc{a.hap.r}-pick  1\textsc{s.erg} \\
\glt `As for me I will climb (the tree). If some things fall with a thud, count (them) because those very ones are the coconuts I have picked.’ (In this story a kyphotic person climbs a tree to pick some coconuts and a blind person waits below and counts the thuds. But instead of coconuts falling it is the kyphotic who keeps on falling out of the tree.) [CBWN-C-15]
\z 
\ea
Pagleged  ta  patik  ta  labbot  ta  salod  ga\textbf{kaļas} na  ga\textbf{sagaid}   ta  labbot  ta  salod.  Paggabot ta  patik  ga\textbf{tagabļong}.  Dayon  en  patik  ta  paļapa  na  may  lagtik daw  may  laggong  na  ga\textbf{sagilit}    tak    duma  an  na  patik may  lagtik-lagtik. \\\smallskip
\gll Pag-leged  ta  patik  ta  labbot  ta  salod  ga-\textbf{kaļas} na  ga-\textbf{sagaid}   ta  labbot  ta  salod.  Pag-gabot ta  patik  ga-\textbf{tagabļong}.  Dayon  en  patik  ta  paļapa  na  may  lagtik daw  may  laggong  na  ga\textbf{sagilit}    tak    duma  an  na  patik may  lagtik-lagtik. \\
\textsc{nr.act}-rub.on  \textsc{nabs}  stick  \textsc{nabs}  bottom  \textsc{nabs}  collecting.container  \textsc{i.r}-rustling \textsc{lk}  \textsc{i.r}-rub.against  \textsc{nabs}  bottom  \textsc{nabs}  collecting.container  \textsc{nr.act}-pull.out
\textsc{nabs}  stick  \textsc{i.r}-ping.sound  right.away  \textsc{cm}  strick  \textsc{nabs}  frond  \textsc{lk}  \textsc{ext.in}  snap/pop
and  \textsc{ext.in}  low.sound  \textsc{lk}  \textsc{i.r}-buzz/hum  because  some  \textsc{def.m}  \textsc{lk}  stick
\textsc{ext.in}  \textsc{red}-snap/pop \\
\glt `When rubbing the stick on the bottom of the collecting container (for coconut sap) it makes a rustling sound when rubbing against the bottom of the collecting container. When pulling out the stick (from the container) it makes a ping sound (the sap dripping off the stick back into the container is what pings). Right away the stick strikes the coconut frond with a popping or snapping sound and with a low buzzing/humming sound because some strikes have continuing popping or snapping.' (This is a description of how it sounds when someone is up in a coconut tree getting the sap that has collected into containers for making coconut wine.) [JCWE-T-14 7.2-6]
\z

\ea
Daw  baļed  maka\textbf{tampa}  ta  pambot  ga\textbf{lagpak}  matian  no.   Daw  naan  ta  pangpang  may  anen  na  ga\textbf{daguok}  daw  ga\textbf{kaļas}. Daw  katupa  ta  lungag  o  gasikleb  na  bato  paryo  gid  ta \textbf{lagpok}.  Daw  magbasya  baļed  ta  baybay  ga\textbf{kaļas}  na  inay-inay nang ta tise. \\\smallskip
\gll Daw  baļed  maka-\textbf{tampa}  ta  pambot  ga-\textbf{lagpak}   …-mati-an  no.   Daw  naan  ta  pangpang  may  anen  na  ga-\textbf{daguok}  daw  ga-\textbf{kaļas}. Daw  ka-tupa  ta  lungag  o  ga-sikleb  na  bato  paryo  gid  ta \textbf{lagpok}.  Daw  mag-basya  baļed  ta  baybay  ga-\textbf{kaļas}  na  inay-inay nang ta tise. \\
if/when  wave  \textsc{i.hap.ir}-slap  \textsc{nabs}  motor.boat  \textsc{i.r}-clap  \textsc{a.hap.ir}-hear-\textsc{apl}  2\textsc{s.erg} if/when  \textsc{spat.def}  \textsc{nabs}  cliff  \textsc{ext.in}  \textsc{ext.g}  \textsc{lk}  \textsc{i.r}-crunch  and  \textsc{i.r}-rustling if/when  \textsc{i.exm}-land.on  \textsc{nabs}  hole  or \textsc{i.r}-hollowed.out  \textsc{lk}  rock  like  \textsc{int}  \textsc{nabs} explode  if/when  \textsc{i.ir}-splash  wave  \textsc{nabs}  beach  \textsc{i.r}-rustling  \textsc{lk}  \textsc{red}-slow only/just  \textsc{nabs}  little \\
\glt `If waves lightly slap the motor boat, what you hear is a clapping sound. If (waves hit) the cliffs there are crunching and rustling sounds. If (they) land in a hole and hollowed out rock it is like exploding sounds. If the waves splash on the beach (they) make rustling sounds that slowly become softer.’ [JCWE-T-14 11.2-6]
\z

\ea
Lugar  na  abrian  din  ga\textbf{lagpok}. \\\smallskip
\gll Lugar  na   …-abri-an  din  ga-\textbf{lagpok}. \\
then  \textsc{lk}  \textsc{t.r}-open-\textsc{apl}  3\textsc{erg}  \textsc{i.r}-explode \\
\glt ‘Then when he opened it (the grenade) it exploded.’ [MBON-T-07 14.4]
\z

\ea
Ta,  mga  ittaw  an,  tak  nakita  danen  na  buļan  ya, naduwad  en  tak  palam-ed  ta  bekkessan  na  bakod, mag\textbf{pukpok}  danen  an  ta  mga  lata,  mga  drum  o  daw  ano  man  na  makaatag  ta  sikad  sagbak … \\\smallskip
\gll Ta,  mga  ittaw  an,  tak  na-kita  danen  na  buļan  ya, na-duwad  en  tak  pa-lam-ed  ta  bekkessan  na  bakod, mag-\textbf{pukpok}  danen  an  ta  mga  lata,  mga  drum  o  daw  ano\footnotemark{}  man  na  maka-atag  ta  sikad  sagbak … \\
so  \textsc{pl}  person  \textsc{def.m}  because  \textsc{a.hap.r}-see  3\textsc{p.erg}  \textsc{lk}.  moon/month  \textsc{def.f}
\textsc{a.hap.r}-lose  \textsc{cm}  because  \textsc{t.r}-swallow  \textsc{nabs}  snake  \textsc{lk}  big
\textsc{i.ir}-beat.on  3\textsc{p.abs}  \textsc{def.m}  \textsc{nabs}  \textsc{pl}  can  \textsc{pl}  drum  or  and  what  also
\textsc{lk}  \textsc{i.hap.ir}-give  \textsc{nabs}  very  noise \\
\footnotetext{The word \textit{ano} is a \isi{Tagalog} word meaning ‘what’.}
\glt `So, the people, because they saw that as for the moon (it) happened to disappear because the big snake swallowed it, they will \textbf{beat} on cans, drums and whatever else that can give out very noisy (sound) …’ [JCOE-C-03 2.3]
\z
\ea
Pagpawa  isab  ta  buļan  an  miyag  ambaļen  tan  na bekkessan  ya  gadļagan  en  tak  naadlek  ta  sagbak  ya na  pabuat  ta  mga  ittaw  na  pa\textbf{pukpok}  danen  mga  lata  an daw  mga  drum. \\\smallskip
\gll Pag-pawa  isab  ta  buļan  an  miyag  ambaļ-en  tan  na bekkessan  ya  ga-dļagan  en  tak  na-adlek  ta  sagbak  ya na  pa-buat  ta  mga  ittaw  na  pa-\textbf{pukpok}  danen  mga  lata  an daw  mga  drum. \\
\textsc{nr.act}-bright  again  \textsc{nabs}  moon  \textsc{def.m}  want  say-\textsc{t.ir}  \textsc{d3nabs}  \textsc{lk} snake  \textsc{def.f}  \textsc{i.r}-run  \textsc{cm}  because  \textsc{a.hap.r}-afraid  \textsc{nabs}  noise  \textsc{def.f} \textsc{lk}  \textsc{t.r}-make  \textsc{nabs}  \textsc{pl}  person  \textsc{lk}  \textsc{t.r}-beat  3\textsc{p.erg}  \textsc{pl}  can  \textsc{def.m} and  \textsc{pl}  drum \\
\glt `When the moon becomes bright again, that means to say that the snake already ran away because (s/he) is afraid of the noise that the people made when they beat on cans and drums.  [JCOE-C-03 2.5]
\z
\ea
Sabat  man  ta  bubuo  ya,  “Miad  gani  daw  \textbf{pakpaken}  no tudtod  ko  i  aged  magļapad.” \\\smallskip
\gll Sabat  man  ta  bubuo  ya,  “Miad  gani  daw  \textbf{pakpak-en}  no tudtod  ko  i  aged  mag-ļapad.” \\
reply  also  \textsc{nabs}  tortoise  \textsc{def.f}  good  truly  if/when  pound.on-\textsc{t.ir}  \textsc{2serg}
back  1\textsc{s.gen}  \textsc{def.n}  so.that  \textsc{i.ir}-wide \\
\glt `The tortoise answered, “It is truly good if you pound on my back such that (result) (it) will become wide.”' [CBWN-C-16 9.15]
\z

% \begin{verbatim}%%move bib entries to  localbibliography.bib
% \end{verbatim}warning
