\chapter{Stem-forming morphological processes}
\label{chap:stemformingprocesses}
\section{Introduction}
\label{sec:introduction-7}

As mentioned in \chapref{chap:1} and the introduction to \chapref{chap:verbstructure}, there are two broad groups of morphological processes that apply to verbs in Kagayanen. These are \isi{inflectional processes} and \isi{stem-forming processes}. Inflectional processes are described in \chapref{chap:verbstructure}. They are different from stem-forming processes in that they define a paradigm (see \chapref{chap:verbstructure}, \tabref{tab:inflectionalaffixes}). A \isi{paradigm}\is{paradigmatic} is a set of word forms that fully represents coherent grammatical categories, with each form expressing a value for each category. We identified the two main categories relevant to inflection in Kagayanen as transitivity and modality. Every verb functioning as a independent predicate must involve one and only one inflection.\footnote{Sometimes inflectional affixes may be omitted in face-to-face conversation, when the speaker’s intention is obvious. This omission of affixes is addressed directly in \chapref{chap:verbstructure}, \sectref{sec:omissionofaffixes} and mentioned at various points throughout this grammar. In all cases, the omitted forms may freely be supplied upon reflection. This omission of inflectional affixes is very different from zero as a member of the inflectional system, and the two phenomena can seldom be confused in actual communication.} They are like the different clothing outfits that one person may wear in different situations-- one outfit for school, another for the beach, another around home, and so on. Verbs with different inflections are not different words-- they are the same word with different “clothes”.

Stem-forming processes (prefixes, suffixes, and reduplications), on the other hand, do not constitute a paradigm for three reasons. First, they are all “optional”; a root alone may be inflected directly. Second, stem-forming processes do not represent a coherent semantic dimension, but rather are a mixed bag in terms of their functions. For the most part, their meanings do not inherently exclude one another. Third, there may be more than one stem-forming process involved in the formation of a given stem. Finally, stems formed from a root plus stem-forming morphology still need inflection in order to be integrated coherently into discourse as verbal predicates. Words with different stem-forming affixes are like different people who are members of the same “family”. They resemble one another because they all come from the same ancestor, or “root”, but they each have their own personality and particular jobs they are good at. Stem-forming morphological processes provide a rich resource for multiplying the expressive potential of Kagayanen predicates.

This chapter is about stem-forming morphological processes. Bare verb roots depict very general “scenes” \is{scenes}in the minds of users. A root may take inflection directly, or it may combine with one or more stem-forming processes to allow speakers to craft nuanced and varied conceptual \isi{scenes}. Stem-forming morphology adds elements such as causative, applicative, pluraction, associative, reciprocal, and iterative to adjust the meaning of the verb to more precisely depict the specific scene the speaker intends. Some stem-forming processes seem to have little to no meaning effects, but are simply required to allow the stem to take further affixation, or their effects may be different depending on the root they attach to or other contextual factors. In this chapter we will first discuss our decision to treat -\textit{en/-on} as an inflectional suffix, and -\textit{an} ‘applicative’ as a stem forming suffix (\sectref{bkm:Ref123018118}). Following this, we discuss seven stem- forming processes that commonly occur in Kagayanen verbs. There are undoubtedly others that are not as productive or common. These we leave for future research.

\section{The “one-suffix” constraint}
\label{bkm:Ref123018118}
Four of the stem-forming processes in Kagayanen are expressed by suffixes on the verb root: -\textit{an} ‘applicative’, -\textit{ay/-anay} ‘reciprocal’, -\textit{a} ‘exclamatory’ and \textit{{}-i} ‘exclamatory applicative’.  As discussed in \chapref{chap:verbstructure}, one inflectional verbal affix is also a suffix. Interestingly, however, the stem-forming suffixes may not occur with one another, nor with the transitive irrealis suffix -\textit{en/-on}. In other words, there seems to be a constraint that at most one suffix may occur on an inflected verb form, regardless of whether it is -\textit{en/-on} or one of the stem-forming group. Nevertheless, we do not consider the five suffixes to constitute a paradigm because they do not exhaustively represent a coherent grammatical category or set of categories. They are not all, for example, different values in one category, like aspect, tense, or modality. There are, however, logical semantically-based reasons why certain suffixes do not co-ocur. For example, the exclamatory suffixes \textit{{}-a} and -\textit{i} exclude one another because of their meanings; -\textit{a} is non-applicative and -\textit{i} is applicative. Also reciprocal constructions are inherently intransitive, so they are not compatible with transitive morphological marking, including applicatives. That leaves applicative plus transitive irrealis as the only possible combination of meanings expressed by verb suffixes.

When the semantically conceivable and common combination of applicative and transitive irrealis occurs, the applicative -\textit{an} “overrides” -\textit{en}, leaving “zero” as the only marker of irrealis modality. As we will see in \chapref{chap:verbclasses-1}, for one subclass of verbs, the regular transitive irrealis inflection is also zero (Class VII), so there is independent evidence for a meaningful zero functioning as the transitive irrealis inflection. One could say that \textit{{}-an} changes the subclass of a stem from one that takes \textit{{}-en/-on} (Class VI) into one that takes \emptyset{}- as the transitive irrealis inflection (Class VII).

The alternative to this approach is to consider zero to always be the marker of grammatical transitivity and irrealis modality, and -\textit{en/-on} and \nobreakdash-\textit{an} to code different “voices” (patient voice, and locative/benefactive voice respectively). The major problem with this analysis is that -\textit{en/-on} never appears in realis modality, whereas -\textit{an} does. Therefore, under this second approach, one must assert that the transitive realis forms (\textit{pa}{}- and \textit{na}{}-) override, or replace -\textit{en/-on}. So under either approach, -\textit{en} must be replaced by something else, either -\textit{an} in applicative contexts, or \textit{pa}{}-/\textit{na}{}- in realis contexts. In this grammar we prefer the first of these approaches as we find it results in a more consistent and simpler overall view of the entire system of verb morphology.

In the following sections we will discuss several stem-forming morphological processes, and provide examples from the corpus whenever possible.

\section{Root reduplication}
\label{sec:rootreduplication-verbs}

Root reduplication was described in \chapref{chap:referringexpressions}, \sectref{sec:rootreduplication} for referential roots (or “nouns”), and in \chapref{chap:modification}, \sectref{sec:rootreduplication-derivation} for property concept roots (or “adjectives”). The same two processes, complete or partial root reduplication, occur with predicating roots (or “verbs”). The meaning of a partially or completely reduplicated verbal root may be repetitive, continuous, not really doing X, sort of doing X, or pretending to do X. The context usually distinguishes among these possibilities. The `not really doing,' `kind of doing', or `pretending to do' meanings are similar to the meanings associated with reduplicated nouns and adjectives. There is only one example in the texts of the `not really doing' sense (\ref{bkm:Ref361055761} below), but this usage does occur frequently in everyday conversation. For example, the root \textit{buat} ‘do’/‘make’ when reduplicated \textit{buat-buat} can mean ‘keep doing/making something’ or it can mean ‘make something up’ or ‘make believe’ (as in example \ref{bkm:Ref361055761}). The root \textit{suļat} ‘to write’, when reduplicated \textit{suļat-suļat} or \textit{sutsuļat} may mean ‘continuously write’, in other contexts ‘write repeatedly’, and in still other contexts ‘scribble’ or ‘pretend to write’. Verbs denoting punctual changes in state, such as \textit{buong} ‘to break’ or punctual changes in location, such as \textit{balik} ‘return’, cannot express the ‘continuously’ meaning when reduplicated because this would be pragmatically odd---something can’t continuously break or return. Root reduplication for these verbs can only mean to ‘break repeatedly’, ‘return repeatedly’ or conceivably ‘sort of break’, ‘pretend to return’ (though these uses are not attested in the corpus, and are difficult for speakers to use or imagine outside of a specific communicative context). Occasionally there is a slight difference in meaning between complete and partial reduplication for the same root, but there is no consistent pattern. For example, the root \textit{sunod} means ‘to follow’. \textit{Sunod-sunod} means `to mimic', whereas \textit{sudsunod} means `to follow', as in someone’s footsteps. Also, \textit{kaan-kaan} is the general reduplicated form ‘to keep on eating’ (from \textit{kaan} ‘eat’) while \textit{kankaan} means `to graze', that is, animals eating with their heads down.

Example \REF{ex:following} lists some reduplicated roots that occur as verbal Predicates in the text corpus. Note that for some roots, there is no partial reduplication form:

\ea
\label{ex:following}
\begin{tabbing}
\hspace{3.6cm} \= \hspace{4.3cm} \= \kill
\textbf{Root} \> \textbf{Complete} \> \textbf{Partial} \\
sunod ‘to follow’ \> sunod-sunod ‘mimic’ \> sudsunod \\
\> \> ‘keep following’ \\
balik ‘to return \> balik-balik \> bakbalik \\
sabat ‘to answer’ \> sabat-sabat \> satsabat \\
kaan ‘to eat’ \> kaan-kaan ‘keep eating’ \> kankaan \\
\> \> ‘graze (animals)’ \\
layog ‘to fly’ \> layog-layog \> laglayog \\
tagad ‘to wait’ \> tagad-tagad \> tadtagad \\
geļet ‘to cut/slice’ \> geļet-geļet \> getgeļet \\
suļat ‘to write \> suļat-suļat \> sutsuļat \\
singngot ‘to smell’ \> singngot-singngot \> sitsingngot \\
tawa ‘to laugh’ \> tawa-tawa \> tatawa \\
basa ‘to read’ \> basa-basa \\
panaw ‘to go/walk’ \> panaw-panaw \\
insa ‘to ask’ \> insa-insa \\
atag ‘to give’ \> atag-atag \\
sayaw ‘to dance’ \> sayaw-sayaw \\
tabang ‘to help’ \> tabang-tabang
\end{tabbing}
\z

The following are a few examples from the corpus of constructions that involve verb-root reduplication. Because of the varying semantics expressed by root reduplication, we consistently gloss it as \textsc{red,} rather than glossing it differently for each usage.

\ea
\label{bkm:Ref361055761}
Uyi  na  isturya  matuod,  dili  ko  nang  \textbf{buat-buat}  na  isturya. \\\smallskip

\gll U-yi  na  isturya  matuod,  dili  ko  nang  \textbf{buat\sim{}buat}  na  isturya. \\
\textsc{emph-d}1\textsc{adj}  \textsc{lk}  story  true  \textsc{neg.ir} 1\textsc{s.erg} only  \textsc{red}\sim{}make  \textsc{lk}  story \\
\glt ‘This story is true, (it is) not just my made up story.’ [BCWN-C-04 9.14]
\z
\ea
Ta  iran  na  \textbf{pagsabat-sabatay}  gaambaļ  Pawikan … \\\smallskip
\gll Ta  iran  na  \textbf{pag-sabat\sim{}sabat-ay}  ga-ambaļ  Pawikan … \\
\textsc{nabs}  3\textsc{p.gen}  \textsc{lk}  \textsc{nr.act-red}\sim{}reply-\textsc{rec}  \textsc{i.r}-say  Sea.turtle \\
\glt ‘While they repeatedly were discussing with each other, Sea Turtle said …’ [DBWN-T-26 2.6]
\z
\ea
Subļa  man  isya  duminggo  na  \textbf{panaw-panaw}  din  ta lungag  na  laog  ta  baybay  ta  Mantubang. \\\smallskip
\gll Subļa  man  isya  duminggo  na  \textbf{panaw\sim{}panaw}  din  ta lungag  na  laog  ta  baybay  ta  Mantubang. \\
more.than  \textsc{emph}  one  week  \textsc{lk}  \textsc{red}\sim{}go/walk  3\textsc{s.gen}  \textsc{nabs}
hole/cave  \textsc{lk} connect  \textsc{nabs} beach  \textsc{nabs} Mantubang \\
\glt `(It was) for more than one week that she was continuously walking in the cave that connects to the beach at Mantubang.’ [JCWN-T-25 2.14]
\z
\ea
\textbf{Gamungko-mungko}  kay  pa  dya. \\\smallskip

\gll \textbf{ga-mungko\sim{}ng-pungko}\footnotemark{}  kay  pa  dya. \\
\textsc{i.r}-\textsc{red}\sim{}\textsc{pl}-sit    1\textsc{p.excl.abs} \textsc{inc}  \textsc{d}4\textsc{loc} \\
\footnotetext{This example incidentally illustrates the layered nature of stem-forming processes. The root is \textit{pungko} `to sit', to which the stem-forming pluraction prefix \textit{ng}- is added, yielding the intermediate stem \textit{mungko}. This stem then becomes the basis for full reduplication.}
\glt ‘We kept sitting still there.’ [DBON-C-06 3.2]
\z
\ea
\textbf{Gatatawa}  kalibangbang  tak  buksol  takong  an. \\\smallskip
\gll \textbf{Ga-ta\sim{}tawa}  kalibangbang  tak  buksol  takong  an. \\
\textsc{i.r-red}\sim{}laugh  butterfly  because  lump  forehead  \textsc{def.m} \\
\glt ‘Butterfly fish kept laughing because the forehead (of the parrot fish) had a lump.’ [JCWO-L-28 27.1]
\z

\newpage
\ea
Gaampang  danen  daw  \textbf{galagas-lagasay}  danen  naan  ta  suba … \\\smallskip
\gll Ga-ampang  danen  daw  \textbf{ga-lagas\sim{}lagas-ay}  danen  naan  ta  suba … \\
\textsc{i.r}-play  3\textsc{p.abs}  and  \textsc{i.r-red}\sim{}chase-\textsc{rec}  3\textsc{p.abs}  \textsc{spat.def}  \textsc{nabs}  river \\
\glt ‘They were playing and they kept (playfully) chasing each other in the river ….’ [CBWN 6.2]
\z
\section{Pluractional \textit{ng-} /ŋ/}
\label{sec:pluraction}

A certain kind of plural action may be expressed by a stem-forming prefix with a basic form /ŋ/ (spelled \textit{ng-}) that assimilates in place of articulation to the initial consonant of the root, unless that consonant is the glottal stop. Any root-initial consonant, including glottal stop, is subsequently dropped. Pluraction only occurs in grammatically intransitive constructions, and only when the construction involves multiple actors,  expresses a coherent group of iterations, or is distributed over items that normally occur in groups, such as hair falling out, leaves blowing away, plants growing or being planted, one person doing something habitually, as a profession, and so on:

\ea
\label{bkm:Ref447198515}
Ubra  ta  ake  na  amay  \textbf{manampara}  nang  daw makwa  ta  guso  \textbf{mananem},  \textbf{manguma}. \\\smallskip
\gll Ubra  ta  ake  na  amay  \textbf{ma-ng-tampara}  nang  daw ma-kwa  ta  guso  \textbf{ma-ng-tanem},  \textbf{ma-ng-uma}. \\
work  \textsc{nabs}  1\textsc{s.gen}  \textsc{lk}  father  \textsc{a.hap.ir-pl}-goggles  only/just  and
\textsc{a.hap.ir}-whachamacallit  \textsc{nabs}  seaweed  \textsc{a.hap.ir-pl}-plant  \textsc{a.hap.ir-pl}-field \\
\glt `The work of my father is to spear fish, to whatchamacallit, to plant agar-agar seaweed, to farm.’ [BMON-C-03 1.5]
\z

The verb \textit{tanem} ‘to plant’, as in examples \REF{bkm:Ref447198515} and \REF{bkm:Ref444336178}, often takes the pluraction prefix. This is because planting necessarily involves multiple, repetitive actions. Thus, even if one person is planting, the pluraction prefix may be used (example \ref{bkm:Ref447198515}). A verb may optionally take the pluraction prefix when several actors perform the same intransitive action individually (examples \ref{bkm:Ref444336178} and \ref{bkm:Ref394988295}):

\newpage
\ea
\label{bkm:Ref444336178}
Tapos  duma  an  \textbf{gananem}  ta  guso  tak  uļa  iskwila. \\\smallskip
\gll Tapos  duma  an  \textbf{ga-ng-tanem}  ta  guso  tak  uļa  iskwila. \\
then  other  \textsc{def.m}  \textsc{i.r-pl}-plant  \textsc{nabs}  seaweed  because  \textsc{neg.r}  school \\
\glt ‘Then the other (students) were planting agar-agar seaweed because there is no school (in session).’ [VAOE-J-02 5.5]
\z
\ea
\label{bkm:Ref394988295}
Mga  nangisda  na  mga  paraw  na  tagaduma  na  lugar \textbf{nanguli}  ta  napatayan  man  en  iran. \\\smallskip
\gll Mga  na-ng-isda\footnotemark{}  na  mga  paraw  na  taga-duma  na  lugar \textbf{na-ng-uli}  ta  na-patay-an  man  en  iran. \\
\textsc{pl}  \textsc{a.hap.r-pl-}fish  \textsc{lk}  \textsc{pl}  boat  \textsc{lk}  \textsc{orig}-other  \textsc{lk}  place
\textsc{a.hap.r-pl-}go.home  \textsc{nabs}  \textsc{a.hap.r}-dead-\textsc{apl}  too  \textsc{cm}  3\textsc{p.gen} \\
\footnotetext{The word \textit{isda} ‘fish’ is both \isi{Hiligaynon} and \isi{Tagalog}. The word meaning ‘fish’ in Kagayanen is \textit{sidda}.}
\glt `Fishing boats from other places had gone home since one of them had died too on them on their (boats).’ [JCWN-T-21 13.5]
\z

Since pluractionality is only expressed in grammatically intransitive constructions, it does not occur with the transitive, realis \textit{pa}{}-. When it occurs with the ambitransitive happenstantial prefixes, \textit{na}{}- and \textit{ma}{}-, the construction must be understood as grammatically intransitive (see examples \ref{bkm:Ref447198515} and \ref{bkm:Ref444336178} above).

In example \REF{bkm:Ref394988618} the pluraction prefix is used on the verb \textit{utang} ‘to borrow money’ because this fish was in the habit of making debts, and in the discourse context he enabled himself to make many debts again by changing his appearance:

\ea
\label{bkm:Ref394988618}
Sunod  i  eman   na  mga  adlaw  sidda  na  utang-utang \textbf{nangutang}  isab. \\\smallskip
\gll Sunod  i  eman   na  mga  adlaw  sidda  na  utang\sim{}utang \textbf{na-ng-utang}  isab. \\
next  \textsc{def.n}  again.as.before  \textsc{lk} \textsc{pl}  day/sun  fish  \textsc{lk}  \textsc{red}\sim{}borrow.money \textsc{a.hap.r-pl}-borrow.money again \\
\glt `The next days again, as before, the fish (called) always-debting was able again to make debts.’ [EMWN-T-08 3.1]
\z

Example \REF{bkm:Ref397352255} illustrates pluraction in a semantically transitive\is{semantic transitivity}\is{transitivity!semantic} situation. However, it is clear that this is a grammatically intransitive clause because the Undergoer, “trees”, is marked as non-absolutive. The actor is not mentioned, but is understood to be an absolutive argument mentioned in the previous clause. This is a detransitive (or “actor voice” construction) as described in \chapref{chap:voice}, \sectref{sec:actorvoice}. Note also that the plural marker \textit{mga} does not occur preceding \textit{kaoy} ‘tree/trees’. This is because no particular trees are being mentioned, but just the idea that trees (any trees) can now be planted. This is called a \textit{generic}\is{generic!} plural:

\ea
\label{bkm:Ref397352255}
… basi  en  na  \textbf{mananem}  nang  ta  kaoy  en. \\\smallskip
\gll … basi  en  na  \textbf{ma-ng-tanem}  nang  ta  kaoy  en. \\
  {}  perhaps  \textsc{cm}  \textsc{lk}  \textsc{a.hap.ir-pl-}plant  only/just  \textsc{nabs}  tree  \textsc{cm} \\
\glt ‘… perhaps now (we) will be able to/might just plant trees.’ [ROOB-T-01 12.1]
\z

Pluraction is not synonymous with plural actor. For example, the verb \textit{sayaw}, ‘to dance’, does not occur with the pluraction prefix, even if several actors are intended. The reason is that dancing is considered one action that several people may do together, whereas the pluraction prefix is only appropriate when there are multiple distinct actions. For example, in \REF{bkm:Ref394988295} the boats are all individually going home. They are scattered over a wide area, and may be going home at different times, and to different places. This is not just one act of “going home”, but rather several individual acts. The apparent optionality of the pluraction prefix may be due to speakers’ varying construals of scenes as being one act by many actors, or multiple individual acts.

\section{Causative \textit{pa-}}
\label{sec:morphologicalcausative}

The stem-forming prefix \textit{pa}{}- adds a causative sense to any verb root. The result is an inherently transitive stem that can then be inflected with any of the inflectional affixes discussed in \chapref{chap:verbstructure}. Such morphological causatives are discussed from a function-first perspective, along with other types of causatives, in \chapref{chap:voice}, \sectref{sec:causatives}. Examples \REF{bkm:Ref397062399} and \REF{bkm:Ref444336543} illustrate dynamic non-causative and corresponding causative constructions:

\ea
\label{bkm:Ref397062399}
    \ea
    Inherently intransitive root \\
    Gii  kanen  an. \\\smallskip
 \gll Ga-ii  kanen  an. \\
    \textsc{i.r}-urine  3\textsc{s.abs}  \textsc{def.m} \\
    \glt ‘S/he urinated/was/is urinating.'
    \ex
    Causative, inherently transitive stem \\
    Paiien  kanen  ta  duktor. \\\smallskip
 \gll Pa-ii-en  kanen  ta  duktor. \\
    \textsc{caus}-urine-\textsc{t.ir}  3\textsc{s.abs}  \textsc{nabs}  doctor \\
    \glt ‘The doctor will have him/her urinate.’
    \z
\z
\ea
    \ea
    \label{bkm:Ref444336543} 
    Inherently transitive root \\
    \textbf{Lutuon}  ta  nanay  din  suman  an  kani. \\\smallskip
 \gll  \textbf{Luto-en}  ta  nanay  din  suman  an  kani. \\
    cook-\textsc{t.ir}  \textsc{nabs}  mother  3\textsc{sgen}  sticky.rice.cake  \textsc{def.m} later \\
    \glt ‘His/her mother will cook the sticky rice cake later.'
    \ex
    \label{bkm:Ref395074775}
    Causative, inherently ditransitive stem \\
    \textbf{Palutuon}  kanen  ta  nanay  din  ta  suman  kani. \\\smallskip
 \gll \textbf{Pa-luto-en}  kanen  ta  nanay  din  ta  suman  kani. \\
    \textsc{caus}-cook-\textsc{t.ir}  3\textsc{s.abs}  \textsc{nabs}  mother  3\textsc{s.gen}  \textsc{nabs}  sticky.rice.cake  later \\
    \glt ‘His/her mother will have him/her cook the sticky rice cake later.
    \z
\z
\ea 
    \ea
    Inherently intransitive root \\
    Tapos,  \textbf{guli}  kay  naan  ta  baļay  na  istaran nay. \\\smallskip
 \gll Tapos,  \textbf{ga-uli}  kay  naan  ta  baļay  na  istar-an nay. \\
    then  \textsc{i.r}-go.home  1\textsc{p.excl.abs}  \textsc{spat.def} \textsc{nabs}  house  \textsc{lk}  live-\textsc{nr} 1\textsc{p.excl.gen} \\
    \glt ‘Then, we went home to the house we were living in.’ [AGWN-L-01 3.14]
    \ex
    Causative, inherently transitive stem \\
    \textbf{Puli}  kay  din  naan  ta  baļay na istaran  nay. \\\smallskip
 \gll \textbf{...-pa-uli}  kay  din  naan  ta  baļay na istar-an  nay. \\
    \textsc{t.r-caus}-go.home  1\textsc{p.excl.abs} 3\textsc{s.erg}  \textsc{spat.def}  \textsc{nabs}  house
    \textsc{lk} live-\textsc{nr} 1\textsc{p.excl.gen} \\
    \glt `S/he sent us home to the house we were living in’, or ‘s/he let us go home to the house we were living in.’
    \z
\z

Example \REF{bkm:Ref122944930} illustrates the causative (inherently transitive) stem \textit{pauli} ‘send home’ in a detransitive (or “actor voice”) construction:

\ea
\label{bkm:Ref122944930}
Kanen  ya  \textbf{gapauli}  ki  kami  naan  ta  baļay na  istaran  nay. \\\smallskip
\gll Kanen  ya  \textbf{ga-pa-uli}\footnotemark{}  ki  kami  naan  ta baļay na  istar-an  nay. \\
3\textsc{s.abs}  \textsc{def.f}  \textsc{i.r}-\textsc{caus}-go.home  \textsc{obl.p}  1\textsc{p.excl}  \textsc{spat.def}  \textsc{nabs}  house
\textsc{lk}  live-\textsc{nr} 1\textsc{p.excl.gen} \\
\footnotetext{The detransitive (actor voice) construction is often used when an ergative element needs to occur in the absolutive form for purposes of relativization, or, as in this elicited example based on example \REF{bkm:Ref395074775}, when the causer is fronted for contrastive/focal prominence (see \chapref{chap:voice}, \sectref{sec:causatives}).}
\glt `S/he is the one who sent us home to the house we were living in.' / ‘S/he is the one who let us go home to the house we were living in.’
\z

The following are a few corpus examples of non-causative and causative forms of the same verb:
\ea
Intransitive root \\
Daw  \textbf{melled}  ka  ta  Malacañang  an  bayad  ka  dyis  pisos. \\\smallskip
\gll Daw  \textbf{m-selled}  ka  ta  Malacañang  an  bayad  ka  dyis  pisos. \\
if/when  \textsc{i.v.ir}-go.inside  2\textsc{s.abs}  \textsc{nabs}  Malacañang  \textsc{def.m}  pay  2\textsc{s.abs} ten  pesos \\
\glt ‘When you go inside Malacañang (the presidential palace), you pay ten pesos.’ [BMON-C-05 3.18]
\z
\ea
Causative, inherently transitive stem \\
 … dayon  kon  \textbf{paselleren}  Maria  an  ta  sawa  din. \\\smallskip
\gll … dayon  kon  \textbf{pa-selled-en}  Maria  an  ta  sawa  din. \\
 {}    right.away  \textsc{hsy}  \textsc{caus}-go.inside-\textsc{t.ir}  Maria  \textsc{def.m}  \textsc{nabs} spouse  3\textsc{s.gen} \\
\glt ‘… right away they said her husband causes Maria to go inside (an inner room in their store).’ [EMWN-T-06 6.6]
\z
\ea
Transitive root \\
Bali  P800  nang  atag  din  ki  yaken  tak  \textbf{pautang} din  pa. \\\smallskip
\gll Bali  P800  nang  ...-atag  din  ki  yaken  tak  \textbf{pa-utang} din  pa. \\
amount  800.pesos  only/just  \textsc{t.r}-give  3\textsc{s.erg}  \textsc{obl.p}  1s  because  \textsc{t.r}-borrow.money
3\textsc{s.erg}  \textsc{inc} \\
\glt `S/he only gave me the amount of 800 pesos because s/he still borrowed (some of my money).’ (Someone had borrowed 1,400 pesos from the author of this letter and paid back only 800. So the balance of 600 was considered still a debt.) [PBWL-T-07 3.11]
\z
\ea
Causative, inherently ditransitive stem \\
Yan  kabangdanan  na  duma  ya  \textbf{papautang}  ko ta  mga  ittaw  na  gaorder  ki  kami. \\\smallskip
\gll Yan  kabangdanan  na  duma  ya  \textbf{pa-pa-utang}  ko ta  mga  ittaw  na  ga-order  ki  kami. \\
\textsc{d}2\textsc{abs} reason \textsc{lk} some \textsc{def.f}  \textsc{t.r}-\textsc{caus}-borrow.money  1\textsc{s.erg} \textsc{nabs}  \textsc{pl} person  \textsc{lk}  \textsc{i.r}-order  \textsc{obl.p}  1\textsc{p.excl} \\
\glt `That is why I let the people who ordered from us borrow some (of the money).’  (People asked the author of this sentence to buy some things for  them. But after he bought their ‘order’ they did not want to pay and wanted to consider it a debt.) [AFWL-L-01 8.18]
\z
\ea
\label{bkm:Ref362342951}
Transitive root \\
\textbf{Patumar}  ko  antibiotic  ya. \\\smallskip
\gll \textbf{Pa-tumar}  ko  antibiotic  ya. \\
\textsc{t.r}-take.meds  1\textsc{s.erg}  antibiotics  \textsc{def.f} \\
\glt ‘I took the antibiotics.’
\z
\ea
\label{bkm:Ref122954287}
Causative, inherently transitive stem \\
\textbf{Patumar}  a  ta  midwife  ta  antibiotic... \\\smallskip
\gll \textbf{...-pa-tumar}  a  ta  midwife  ta  antibiotic... \\
\textsc{t.r-caus}-take.meds  1\textsc{s.abs}  \textsc{nabs}  midwife  \textsc{nabs}  antibiotic \\
\glt ‘The midwife had me take some antibiotics ...’ [VAWN-T-16 4.7]
\z

Although the verb forms are identical in examples \REF{bkm:Ref362342951} and \REF{bkm:Ref122954287}, \REF{bkm:Ref122954287} is clearly the causative because of the construction in which it appears. The midwife is the ergative case agent of cause, and the speaker is the absolutive case causee. As a non-causative this would have to mean ‘The midwife took me (as medicine)’, which is pragmatically nonsensical. Furthermore, the verb form \textit{papatumar}, with both \textit{pa}{}- prefixes present, may replace the verb in \REF{bkm:Ref122954287} but not in \REF{bkm:Ref362342951}. When both prefixes occur, the effect is to make the causative notion more forceful or more direct. The causer is the one initiating and forcing the causee to perform the action either with words, threats of punishment or physical force. The causee does not have a choice to perform the action or not.

With a causative of a transitive situation, either the causee \REF{bkm:Ref122954287} or the patient \REF{bkm:Ref389142348} can occur in the absolutive case. When the patient (the antibiotic) is absolutive, the interpretation is much more likely to be permissive.

\ea
\label{bkm:Ref389142348}
\textbf{Patumar}  ta  midwife  ki   yaken  antibiotic  ya. \\\smallskip
\gll \textbf{...-pa-tumar}  ta  midwife  ki   yaken  antibiotic  ya. \\
\textsc{t.r-caus}-take.meds  \textsc{nabs}  midwife  \textsc{obl.p}  1s  antibiotic  \textsc{def.f} \\
\glt ‘The midwife let me take the antibiotics.’
\z

Examples \REF{bkm:Ref396488032} and \REF{bkm:Ref501351430} illustrate causatives of non-volitional intransitive roots, \textit{anod} ‘drift’ and \textit{tubo} ‘grow’ respectively. They both take intransitive prefixes because the constructions are detransitive:\footnote{In example \REF{bkm:Ref501351430}, the forms \textit{magripir} and \textit{magpatubo} may also be considered nominalizations ‘repairing’ and ‘causing to grow’ respectively.}

\ea
\label{bkm:Ref396488032}
Gani  kumo  en  uļa  layag  \textbf{gapaanod-anod}  nang  en. \\\smallskip
\gll Gani  kumo  en  uļa  layag  \textbf{ga-pa-anod\sim{}-anod}  nang  en. \\
so  because  \textsc{cm}  \textsc{neg.r}  sail  \textsc{i.r-caus-red}\sim{}drift  only/just  \textsc{cm} \\
\glt ‘So because of no more sail (they) just allowed (the boat) to keep drifting.’ [VPWN-T-06 3.2]
\z
\ea
\label{bkm:Ref501351430}
 … bellay  magripir  daw  \textbf{magpatubo}  en  eman  ta  mga  buļak… \\\smallskip
\gll … bellay  mag-ripir  daw  \textbf{mag-pa-tubo}  en  eman  ta  mga  buļak … \\
   {} difficult  \textsc{i.ir}-repair  and  \textsc{i.ir-caus}-grow  \textsc{cm}  again.as.before  \textsc{nabs}  \textsc{pl}  flower \\
\glt ‘… (it is) difficult to repair (the fence) and to get the flowers to grow again …’ [JCOB-L-01 1.2]
\z

Some non-volitional intransitive roots, including “adjectives” (roots expressing property concepts) and nouns, may be used in transitive causative frames with no causative prefix. These roots, then, can be considered “P-labile”, as defined by \citet{haspelmath1993}. Example \REF{bkm:Ref395080546} illustrates the transitive construction with the property concept roots \textit{talawis} ‘pointed’ and \textit{mļangkaw/ļangkaw} ‘long’. These examples are clearly causative because they are grammatically transitive, as evidenced by the presence of the transitive irrealis suffix \nobreakdash-\textit{en}, and an ergative argument, which in both cases expresses a causal agent.

\ea
\label{bkm:Ref395080546}
Mag-agos  ka  ta  tallo  na  kawayan  na  \textbf{talawisen}  no  gid na  \textbf{mļangkawen}  no … \\\smallskip
\gll Mag--agos  ka  ta  tallo  na  kawayan  na  \textbf{talawis-en}  no  gid na  \textbf{mļangkaw-en}  no ... \\
\textsc{i.ir}-strip  2\textsc{s.abs}  \textsc{nabs}  three  \textsc{lk}  bamboo  \textsc{lk}  pointed-\textsc{t.ir} 2\textsc{s.erg} \textsc{int}
\textsc{lk}  long-\textsc{t.ir}  2\textsc{s.erg} \\
\glt `Strip three bamboo sticks making (them) really pointed while making (them) long …’ [PBON-T-01 2.31]
\z

These roots can also occur with the overt causative prefix. Example \REF{bkm:Ref395081078} illustrates the root \textit{mļangkaw/ļangkaw} with the causative prefix and transitive irrealis suffix:

\ea
\label{bkm:Ref395081078}
Bag-o  ko  \textbf{paļangkawen}  ake  na  suļat,  liag  ko  anay paabot  ko  a  miad  na  adlaw  ki  kaon. \\\smallskip
\gll Bag-o  ko  \textbf{pa-ļangkaw-en}  ake  na  suļat,  liag  ko  anay …-pa-abot  ko  a  miad  na  adlaw  ki  kaon. \\
before  1\textsc{s.erg}  \textsc{caus}-long-\textsc{t.ir}  1\textsc{s.gen}  \textsc{lk}  letter  want  1\textsc{s.erg}  first/for.awhile
\textsc{t.ir-caus}-arrive  1\textsc{s.erg}  \textsc{inj}  good  \textsc{lk}  day/sun  \textsc{obl.p}  2s \\
\glt `Before I make my letter long, I want to first wish you a good day.’ (This is a common way to begin a letter in Kagayanen, literally: ... I cause a good day to reach you.’) [EMWL-T-04 5.1]
\z

Example \REF{bkm:Ref395110780} illustrates the non-volitional intransitive verb \textit{tunaw} ‘melt/dissolve’ in its basic usage.

\ea
\label{bkm:Ref395110780}
\textbf{Natunaw}  tubuyong  an. \\\smallskip
\gll \textbf{Na-tunaw}  tubuyong  an. \\
\textsc{a.hap.r}-dissolve  manioc.flour  \textsc{def.m} \\
\glt ‘This manioc flour has dissolved.’
\z

In \REF{bkm:Ref395110927} and \REF{bkm:Ref122956811}, this same root is used transitively with transitive irrealis and transitive realis affixation respectively. The effect is a causative interpretation (`cause to dissolve'), even though the causative prefix does not appear.

\ea
\label{bkm:Ref395110927}
\textbf{Tunawen}  no  anay  tubuyong  an  para  pangmiroļ. \\\smallskip
\gll \textbf{Tunaw-en}  no  anay  tubuyong  an  para  pang-miroļ. \\
dissolve-\textsc{t.ir}  2\textsc{s.erg}  first/for.awhile  manioc.flour  \textsc{def.m}  for  \textsc{inst}-starch \\
\glt ‘Dissolve please the manioc flour for use as starch.’
\z
\ea
\label{bkm:Ref122956811}
\textbf{Patunaw}  din  tubuyong  an  para  pangmiroļ. \\\smallskip
\gll \textbf{Pa-tunaw}  din  tubuyong  an  para  pang-miroļ. \\
\textsc{t.r}-dissolve  3\textsc{s.erg}  manioc.flour  \textsc{def.m}  for  \textsc{inst}-starch \\
\glt ‘S/he dissolved the manioc flour for use as starch.’
\z

There are at least two pieces of evidence that show that the \textit{pa}{}- prefix in example \REF{bkm:Ref122956811} is the transitive, realis \textit{pa}{}- and not the causative. First, the transitive irrealis form is used in example \REF{bkm:Ref395110927} with the same set of argument roles (Causer-ergative, patient-absolutive). Second, this verb may also take the causative prefix, in which case the effect is an indirect causative of a causative:

\ea
\label{bkm:Ref395126311}
\textbf{Patunawen}  no  anay  ki  kanen  tubuyong  an. \\\smallskip
\gll \textbf{Pa-tunaw-en}  no  anay  ki  kanen  tubuyong  an. \\
\textsc{caus}-dissolve-\textsc{t.ir}  2\textsc{s.erg} first/for.awhile \textsc{obl.p}  3s  manioc.flour   \textsc{def.m} \\
\glt ‘Please have him/her dissolve the manioc flour.’ (e.g., ‘Cause him/her to cause the manioc flour to dissolve.’)
\z
\ea
\label{bkm:Ref122957467}
\textbf{Papatunaw}  din  ki  yaken  tubuyong  an. \\\smallskip
\gll \textbf{Pa-pa-tunaw}  din  ki  yaken  tubuyong  an. \\
\textsc{t.r-caus}-dissolve  3\textsc{s.erg}  \textsc{obl.p}  1s  manioc.flour   \textsc{def.m} \\
\glt ‘S/he had me dissolve the manioc flour.’ (e.g. ‘S/he caused me to cause the manioc flour to dissolve.)
\z

In these examples the roles are Causer=ergative, patient=absolutive, causee=ob\-lique. In other words, examples \REF{bkm:Ref395126311} and \REF{bkm:Ref122957467} are morphological causatives of the lexical (patient-preserving) causative roots illustrated in \REF{bkm:Ref395110927} and \REF{bkm:Ref122956811} respectively.

Some causative verb stems can occur in detransitive constructions to express a kind of reflexivity. The meaning is ‘to cause or let oneself be \textsc{verbed}’. For example:

\ea
Gagi  a  pa  ta  Narra  tak  \textbf{gapabuļong} a  pa. \\\smallskip
\gll Ga-agi  a  pa  ta  Narra  tak  \textbf{ga-pa-buļong} a  pa. \\
\textsc{i.r}-pass  1\textsc{s.abs}  \textsc{inc}  \textsc{nabs}  Narra  because  \textsc{i.r-caus}-medicine  1\textsc{s.abs}  \textsc{inc} \\
\glt ‘I just went to Narra because I got (myself) checked up.’ [RCON-L-03 10.2]
\z

\newpage
\ea
Uļa  gid  kanen  \textbf{gapaawid}  ki  danen. \\\smallskip
\gll Uļa  gid  kanen  \textbf{ga-pa-awid}  ki  danen. \\
\textsc{neg.r}  \textsc{int}  \textsc{3}\textsc{s.abs}  \textsc{i.r-caus-}hold.back  \textsc{obl.p}  3p \\
\glt ‘He did not let himself be held back by them.’ [PBWN-C-1 21.11]
\z

Essentially all verbs that describe events that one can have done to oneself can be used in this way. These include the following:

\ea
\label{ex:haircut}
\begin{tabbing}
\hspace{3.5cm} \= \kill
gunting ‘scissors’ \> Gapagunting kanen an. \\
\>‘S/he got her/himself a haircut.’ \\
lipay ‘joy’  \>  Gapalipay kanen an. \\
\>‘S/he let/made her/himself to rejoice.’ \\
gwapa ‘attractive(f)’ \> Gapagwapa kanen an. \\ 
\>‘She made herself attractive.’ \\
gwapo ‘attractive(m)’ \> Gapagwapo kanen an. \\
\>‘He made himself attractive.’ \\
duktor ‘doctor’ \> Gapaduktor kanen an. \\
\>‘S/he got her/himself checked up.’ \\
gulpi ‘sudden’ \> Gapagulpi kanen an. \\
\>‘S/he let her/himself act suddenly.’ \\
luko ‘to trick’ \> Gapaluko kanen an. \\
\>‘S/he let her/himself be tricked.’ \\
libeg ‘worry’  \>  Gapalibeg kanen an. \\
\>‘S/he let/made her/himself be worried.’ \\
bellay ‘tire/difficult’	\> Gapabellay kanen an. \\
\>‘S/he let/made her/himself be tired.’ \\
salamat ‘thanks’ \> Gapasalamat kanen an. \\ 
\>‘S/he let/made her/himself give thanks.’ \\
datas ‘high’ \>   Gapadatas kanen an. \\
\>‘S/he let/made her/himself high.’ (i.e. boasted) \\
inrol ‘to enroll’ \> Gapainroll kanen an. \\ 
\>‘S/he enrolled her/himself.’ \\
kasaļ ‘wedding’ \> Gapakasaļ kanen an. \\
\>‘S/he let/had her/himself be wed.' \\
indyiksyon ‘injection’ \> Gapaindyiksyon kanen an \\ 
\>‘S/he let/had her/himself be injected.’ \\
sagod ‘take care of’ \> Gapasagod kanen an. \\
\>‘S/he let/had her/himself be taken care of.’
\end{tabbing}
\z
\section{Applicative \textit{-an}}
\label{bkm:Ref394919927} \label{sec:applicative-an}\is{applicative morphology|(}

The applicative suffix -\textit{an} creates transitive stems from inherently intransitive, transitive and ditransitive verb roots. Usually, when the applicative occurs, a participant that is not a prototypical patient of the scene\is{scenes} described by the verb root occurs as an Absolutive, and the other arguments become non-absolutive. The applicative -\textit{an} may co-occur with any prefix consistent with grammatical transitivity (see \chapref{chap:verbstructure}, \tabref{tab:inflectionalaffixes}). They never occur with the transitive, irrealis suffix -\textit{en/-on}. Applicatives may only be used in intransitive frames in certain experiential predicates (see \chapref{chap:verbclasses-2}). Example \REF{bkm:Ref395170873} illustrates a volitional intransitive verb, while \REF{bkm:Ref123019919} illustrates the same verb in an applicative construction:

\ea
\label{bkm:Ref395170873}
Ta  buļan  na  Abril  tallo  kay  na  magbarkada \textbf{giling} kay  naan  ta  Puļo. \\\smallskip
\gll Ta  buļan  na  Abril  tallo  kay  na  mag-barkada \textbf{ga-iling} kay  naan  ta  Puļo. \\
\textsc{nabs} month  \textsc{lk}  April  three  1\textsc{p.excl.abs}  \textsc{lk}  \textsc{rel}-friends \textsc{i.r}-go
  1\textsc{p.excl.abs}  \textsc{spat.def}  \textsc{nabs}  Puļo \\
\glt `In the month of April we three friends went to Pulo (island).’\footnote{The word \textit{puļo} in Kagayanen means island. It is also a proper name for a small island that is part of the Cagayan island group. In this example, the name of the small island is intended.}  [CBWN-C-11 2.1]
\z
\ea
\label{bkm:Ref123019919}
Tapos,  bai  i  gariport  naan  ta  pulis  daw \textbf{pailingan}  danen  yo  na  lungag … \\\smallskip
\gll Tapos,  bai  i  ga-riport  naan  ta  pulis  daw \textbf{pa-iling-an}  danen  yo  na  lungag … \\
then  woman  \textsc{def.n}  \textsc{i.r}-report \textsc{spat.def}  \textsc{nabs}  police  and \textsc{t.r}-go-\textsc{apl}  3\textsc{p.erg}  \textsc{d}4\textsc{abs}  \textsc{lk}  hole/cave \\
\glt `Then the woman reported to the police and they went to that cave …’ [MBON-C-01 6.1]
\z

In example \REF{bkm:Ref123019919}, the applicative suffix creates a grammatically transitive verb from the inherently intransitive root \textit{iling} ‘go’. The new stem, -\textit{ilingan}, may be paraphrased ‘go to X’. Note that the ergative pronoun \textit{danen} refers to the police and the woman, while the destination, \textit{yo na lungag}, ‘that cave’, is in the absolutive case (marked by the lack of a prenominal particle). The applicative is used here because of the high salience of the cave where they had been told they can find the body of the woman’s child who had been murdered.\footnote{In \chapref{chap:voice}, this construction is described from a semantic perspective as “locative voice.” Other voices are also expressed via the applicative construction, as will be seen in this section, and in \chapref{chap:voice}.}

Example \REF{bkm:Ref395170605} is a non-volitional intransitive construction, while \REF{bkm:Ref395170609} is the applicative equivalent:

\ea
\label{bkm:Ref395170605} 
Sakon  ya  \textbf{nadagsa}  naan  ta  Pasig  River. \\\smallskip
\gll Sakon  ya  \textbf{na-dagsa}  naan  ta  Pasig  River. \\
whale  \textsc{def.f}  \textsc{a.hap.r}-wash.ashore  \textsc{spat.def}  \textsc{nabs}  Pasig  River \\
\glt ‘The whale happened to wash ashore in the Pasig River.’ [CBWN-T-27 3.1]
\z
\ea
\label{bkm:Ref395170609}
\textbf{Nadagsaan}  ta  sakon  Pasig  River  ya. \\\smallskip
\gll \textbf{Na-dagsa-an}  ta  sakon  Pasig  River  ya. \\
\textsc{a.hap.r}-wash.ashore-\textsc{apl}  \textsc{nabs}  whale  Pasig  River  \textsc{def.f} \\
\glt ‘The whale happened to wash ashore in the Pasig River.’
\z

In example \REF{bkm:Ref395170609}, the Pasig River lacks the non-absolutive prenominal case marker \textit{ta}. This indicates that it is in the absolutive case. Meanwhile the whale is no longer absolutive, but is now marked with \textit{ta}. We may say that in this case, the non-absolutive role is ergative. Unfortunately, there is no natural way to reflect the difference in meaning expressed by this “advancement” to absolutive status of a locative participant in the English free translation, literally: `The whale washed-ashore-on the Pasig river,' or `the whale affected the Pasig river by washing ashore,' or `The Pasig river was washed upon by the whale.'

In example \REF{bkm:Ref395171864}, the inherently transitive verb \textit{lebbeng} ‘bury’ is used in a grammatically transitive, non-applicative clause with an Actor in the ergative role, Undergoer (patient) in the absolutive role and a location, \textit{ta} \textit{patyo}, ‘in the cemetery’, in an oblique role:

% \ea
% \label{bkm:Ref395171864}
% Lugar  na  gatago  kanen  i  nakita  din  iya  na  maguļang \textbf{galebbeng} naan ta bļawan... \\\smallskip
%  \gll Lugar  na  ga-tago  kanen  i  na-kita  din  iya  na  maguļang \textbf{ga-lebbeng} naan ta bļawan... \\
% then  \textsc{comp}  \textsc{i.r}-hide  3\textsc{s.abs}  \textsc{def.n}  \textsc{a.hap.r}-see  3\textsc{s.erg} 3\textsc{s.gen}  \textsc{lk} elder.sibling \textsc{i.r}-bury \textsc{spat.def} \textsc{nabs} gold \\
% \glt 'Then when he hid, he saw his older sibling being buried in the gold ...' (This is a story of two brothers. The younger showed the elder where he found much gold. But the older brother wanted to get as much of it as possible and so he was not careful and kept filling up his sack. But the younger one left and hid outside watching. The big piles of gold fell on the older one and buried him. The people who hid the gold then came and killed him.) [CBWN-C-22 12.4]
% \z
\ea
\label{bkm:Ref395171864}
Ta  dason  na  adlaw  ake  na  lola  \textbf{palebbeng} nay  ta  patyo. \\\smallskip
\gll Ta  dason  na  adlaw  ake  na  lola  \textbf{pa-lebbeng} nay  ta  patyo. \\
\textsc{nabs} next  \textsc{lk}  day/sun  1\textsc{s.gen}  \textsc{lk}  grandmother  \textsc{t.r}-bury
1\textsc{p.excl.erg}  \textsc{nabs}  cemetery \\
\glt `The next day we buried my grandmother in the cemetery.’ [MLWN-T-01 2.4]
\z

In example \REF{bkm:Ref395171866}, the same verb is used with the applicative suffix. In this case the cemetery is the absolutive, while the patient argument, \textit{ta mga patay} ‘the dead ones’ is in a non-absolutive role. The actor is ergative in both examples:

\newpage
\ea
\label{bkm:Ref395171866}
\textbf{Palebbengan}  nay  patyo  an  ta  mga  patay. \\\smallskip
\gll \textbf{Pa-lebbeng-an}  nay  patyo  an  ta  mga  patay. \\
\textsc{t.r}-bury-\textsc{apl}  1\textsc{p.excl.erg}  cemetery  \textsc{def.m}  \textsc{nabs}  \textsc{pl}  dead \\
\glt ‘We buried the dead in the cemetery.’
\z

Ditransitive verbs can also appear in applicative constructions. The effect is to express a recipient or location (depending on the verb) in the absolutive role (see \chapref{chap:voice} on voice). Example \REF{bkm:Ref396506191} illustrates the verb \textit{atag} ‘give’ in a non-applicative construction, with the recipient, \textit{iran na bata} ‘their child’, in an oblique role, marked by the prenominal case marker \textit{ta}:

\ea
\label{bkm:Ref396506191}
Bisan  ino  na  mga  ampangan  an  \textbf{paatag}  danen  ta  iran  na  bata. \\\smallskip
\gll Bisan  ino  na  mga  ampang-an  an  \textbf{pa-atag}  danen  ta  iran  na  bata. \\
any  what  \textsc{lk}  \textsc{pl}  play-\textsc{nr}  \textsc{def.m}  \textsc{t.r}-give  3\textsc{p.erg}  \textsc{nabs}  3\textsc{p.gen}  \textsc{lk}  child \\
\glt ‘Whatever playthings, they gave (them) to their child.’ [CBWN-C-14 2.6]
\z

Example \REF{bkm:Ref395799832} illustrates the same verb in an applicative construction. Note that the recipient, \textit{mama} ‘man’, is absolutive, while the theme, the flower, appears in a non-absolutive, oblique role:

\ea
\label{bkm:Ref395799832}
Tapos,  \textbf{paatagan}  din  mama  ya  ta  buļak … \\\smallskip
\gll Tapos,  \textbf{pa-atag-an}  din  mama  ya  ta  buļak … \\
then  \textsc{t.r}-give-\textsc{apl}  3\textsc{s.erg}  man  \textsc{def.f}  \textsc{nabs}  flower \\
\glt ‘Then she gave the man a flower …' [CBWN-C-14 4.6]
\z

The applicative suffix also has a \isi{partitive} use with some inherently transitive verbs. Example \REF{bkm:Ref123020654} illustrates the verb \textit{panno} ‘fill’ in a non-applicative construction. In this example, the shirt is the absolutive, while the agent, ‘he’ is ergative:

\ea
\label{bkm:Ref123020654}
\textbf{Papanno}  din  bayo  din  ya  ta  bļawan … \\\smallskip
\gll \textbf{Pa-panno}  din  bayo  din  ya  ta  bļawan … \\
\textsc{t.r}-fill  3\textsc{s.erg}  shirt  3\textsc{s.gen}  \textsc{def.f}  \textsc{nabs}  gold \\
\glt ‘He filled up his shirt with gold …’ [CBWN-C-22 4.23]
\z

Example \REF{bkm:Ref395171874} illustrates the same verb in an applicative construction. Note, however, that the argument structure is the same. The shirt is still absolutive, and the agent is still ergative:

\ea
\label{bkm:Ref395171874}
\textbf{Papannuan}  din  bayo  din  ya  ta  bļawan. \\\smallskip
\gll \textbf{Pa-panno-an}  din  bayo  din  ya  ta  bļawan. \\
\textsc{t.r}-fill-\textsc{apl}  3\textsc{s.erg}  shirt  3\textsc{s.gen}  \textsc{def.f}  \textsc{nabs}  gold \\
\glt ‘He partially filled his shirt with gold.’
\z

Example \REF{bkm:Ref395206755} illustrates the same construction in the happenstantial mode:

\ea
\label{bkm:Ref395206755}
\textbf{Napannuan}  din  bayo  din  ya  ta  bļawan. \\\smallskip
\gll \textbf{Na-panno-an}  din  bayo  din  ya  ta  bļawan. \\
\textsc{a}.\textsc{hap.r}-fill-\textsc{apl}  3\textsc{s.erg}  shirt  3\textsc{s.gen}  \textsc{def.f}  \textsc{nabs}  gold \\
\glt ‘He was able to partially fill his shirt with gold.’
\z

Examples \REF{bkm:Ref395171874} and \REF{bkm:Ref395206755} illustrate the partitive sense of the applicative with the verb \textit{panno} ‘fill’, which arguably takes a location (the shirt in this case) as its absolutive argument in the unmarked form. However, the applicative has this effect for many other verbs that are inherently transitive. Consider the following:

\ea
    \ea Absolutive = patient (roasted fish) \\
    \label{bkm:Ref396998462}
    \textbf{Pakaan}  din  sinugba  ya  na  sidda  naan  ta pinggan  ko. \\\smallskip
 \gll \textbf{Pa-kaan}  din  s<in>ugba  ya  na  sidda  naan  ta pinggan  ko. \\
    \textsc{t.r}-eat  3\textsc{s.erg}  <\textsc{nr.res}>-roast  \textsc{def.f}  \textsc{lk} fish  \textsc{spat.def}  \textsc{nabs}
    dish  1\textsc{s.gen} \\
    \glt ‘S/he ate the roasted fish in my dish.’
    \ex
    \label{ex:inmydish-1}
    Absolutive = location (dish) \\
    \textbf{Pakan-anan}  din  pinggan  ko  an  ta  sinugba  ya na  sidda. \\\smallskip
 \gll \textbf{Pa-kan-en-an}\footnotemark{}  din  pinggan  ko  an  ta  s<in>ugba  ya na  sidda. \\
    \textsc{t.r}-cooked.rice-\textsc{apl}  3\textsc{s.erg}  dish  1\textsc{s.gen}  \textsc{def.m}  \textsc{nabs} <\textsc{nr.res}>-roast  \textsc{def.f}   \textsc{lk}  fish \\
    \footnotetext{As mentioned elsewhere in this grammar, the root meaning `eat' is suppletive. When no suffix occurs, the root is \textit{kaan}, but when any suffix occurs, the root is \textit{kan-en}, which is the noun referring to `cooked rice'. This root becomes \textit{kan-an} when the suffix contains the vowel \textit{a}, namely the suffixes -\textit{an} or -\textit{a}.}
    \glt ‘S/he ate roasted fish in my dish.' 
    \newpage
    \ex
    \label{ex:inmydish-2}
    Absolutive = Partitive patient (some of the roasted fish) \\
    \textbf{Pakan-anan}  din  sinugba  ya  na  sidda naan  ta pinggan  ko. \\\smallskip
 \gll \textbf{Pa-kan-en-an}  din  s<in>ugba  ya  na  sidda naan  ta pinggan  ko. \\
    \textsc{t.r}-cooked.rice-\textsc{apl}  3\textsc{s.erg}  <\textsc{nr.res}>roast  \textsc{def.f}  \textsc{lk} fish
    \textsc{spat.def}  \textsc{nabs} dish  1\textsc{s.gen} \\
    \glt `S/he ate some of the roasted fish in my dish.’
    \z
\z
Example \REF{bkm:Ref396998462} illustrates the verb \textit{kaan} ‘eat’ in its basic transitive argument structure with the agent expressed with an ergative pronoun and the patient, the roasted fish, with no prenominal marker, indicating it is absolutive. Example \REF{ex:inmydish-1} is a standard applicative in which the verb carries the applicative suffix, the roasted fish has been marked as non-absolutive, and the dish is now in the absolutive role. Finally, example \REF{ex:inmydish-2} illustrates the partitive use of the applicative construction. The case frame in \REF{ex:inmydish-2} is the same as in \REF{bkm:Ref396998462}. The only difference in structure is the presence of the applicative suffix. The semantic effect is to assert that only some of the roasted fish was consumed.

One way of understanding this usage is that, since the applicative has the default function of indicating that the absolutive is a location, recipient or some other non-patient semantic role, the use of the applicative when the absolutive is a patient expresses that it is somehow a less than prototypical patient. One reasonable interpretation of this is that the patient is not completely affected by the action of the verb. Another way of understanding this usage is that \REF{bkm:Ref396998462}b is the applicative of \REF{bkm:Ref396998462}a, while \REF{bkm:Ref396998462}c is the applicative of the detransitive of \REF{bkm:Ref396998462}b, given in \REF{bkm:Ref460337201}:

\ea
\label{bkm:Ref460337201}
\textbf{Gakaan}  kanen  ta  sinugba  ya  na  sidda  naan  ta pinggan  ko. \\\smallskip
\gll \textbf{Ga-kaan}  kanen  ta  s<in>ugba  ya  na  sidda  naan  ta pinggan  ko. \\
\textsc{i.r}-eat  3\textsc{s.abs}  \textsc{nabs}  <\textsc{nr.res}>-roast  \textsc{def.f}  \textsc{lk} fish  \textsc{spat.def} \textsc{nabs} dish  1\textsc{s.gen} \\
\glt ‘S/he ate roasted fish in my dish.’
\z

Finally, there are some verbs for which the applicative suffix is obligatory in the basic transitive construction. For the most part, these are verbs for which the absolutive is only superficially or invisibly affected by the activity, and therefore is not a prototypical patient. These include \textit{tabang} ‘to help’, \textit{bantay} ‘to watch/ guard’, and \textit{amblig} ‘to care for.’ A long, but not inclusive, list of such verbs appears in \tabref{tab:rootsthattaketwodistincttransitiveirrealisinflections-1}, in \chapref{chap:morphosyntacticallydefinedverbclasses}, \sectref{sec:volitionaltransitiveroots}, where they are designated as ``Class VIII" roots. The functions of the applicative construction are discussed in more detail in \chapref{chap:voice}, \sectref{sec:applicativevoice}, in the context of a general discussion of voice.
\is{applicative morphology|)}
\section{Reciprocal -\textit{ay} and \textit{-anay}}
\label{sec:reciprocals}\is{reciprocal morphology|(}

The suffixes -\textit{ay} and \textit{{}-anay} are \isi{morphological reciprocals}. These suffixes usually indicate that two agents are acting on each other, though they also have an \isi{associative} or \isi{mutual action} use, as well as an idiomatic emphatic usage described in \chapref{chap:voice}, \sectref{sec:reflexivesandreciprocals}. Examples \REF{ex:reciprocal-1} and \REF{ex:reciprocal-2} illustrate the two reciprocal suffixes:

\ea
\label{ex:reciprocal-1}
\textbf{Galumbaay}  danen  an  ta  dļagan  naan  ta  plasa. \\\smallskip
\gll \textbf{Ga-lumba-ay}  danen  an  ta  dļagan  naan  ta  plasa. \\
\textsc{i.r}-race-\textsc{rec}  3\textsc{p.abs}  \textsc{def.m}  \textsc{nabs}  run  \textsc{spat.def}  \textsc{nabs}  plaza \\
\glt ‘They raced each other running on the plaza.’
\z
\ea
\label{ex:reciprocal-2}
\textbf{Galumbaanay}  danen  an  ta  dļagan  naan  ta  plasa. \\\smallskip
\gll \textbf{Ga-lumba-anay}  danen  an  ta  dļagan  naan  ta  plasa. \\
\textsc{i.r}-race-\textsc{rec}  3\textsc{p.abs}  \textsc{def.m}  \textsc{nabs}  run  \textsc{spat.def}  \textsc{nabs}  plaza \\
\glt ‘They raced each other running on the plaza.’
\z

The reciprocals reduce the transitivity of the clause, therefore verbs in either reciprocal form are always grammatically intransitive. They do not appear with the causative in the corpus for this study, and it is hard to conceive of a situation in which some participants might `cause (each other) to do something (to each other)' as opposed to just doing it to each other. All semantically compatible roots may appear with -\textit{ay}, and most may also appear with -\textit{anay}. 

The following are a few examples of reciprocal constructions from the corpus. Example \REF{ex:netfishing} illustrates the suffix -\textit{ay} on the root \textit{isturya}, `to talk':
\ea
\label{ex:netfishing}
\textbf{Gaisturyaay}  danen  na  maggata-gata. \\\smallskip
\gll \textbf{Ga-isturya-ay}  danen  na  mag-gatagata.\footnotemark \\
\textsc{i.r-}talk-\textsc{rec}  3\textsc{p.abs}  \textsc{lk}  \textsc{i.ir}-fish.with.nets \\
\footnotetext{The root \textit{gata-gata} meaning `to fish with nets' is spelled with the hyphen as though it involves reduplication, but there is no root *\textit{gata} that we are aware of. We have removed the hyphen in the analysis line because the hyphen in the text line is purely conventional, and does not indicate a morpheme boundary.}
\glt ‘They talked together about going net fishing.’ [JPWN-L-01 3.2]
\z

Example \REF{ex:jackfruit} illustrates the same root with the suffix -\textit{anay}. Native speaker assessments is that there is no difference in meaning between the two suffixes in this case.
\ea
\label{ex:jackfruit}
Na  \textbf{gaisturyaanay}  mama  ya  daw  darwa  na  magsawa namatian  ta  nangka  kagi  ta  iya  na  tatay … \\\smallskip
\gll Na  \textbf{ga-isturya-anay}  mama  ya  daw  darwa  na mag-sawa na-mati-an  ta  nangka  kagi  ta  iya  na  tatay … \\
\textsc{lk}  \textsc{i.r}-talk-\textsc{rec}  man  \textsc{def.f} and  two  \textsc{lk}  \textsc{rel}-spouse
\textsc{a.hap.r}-hear-\textsc{apl}  \textsc{nabs}  jackfruit  voice  \textsc{nabs}  3\textsc{s.gen}  \textsc{lk}  father \\
\glt `When the man and the couple were talking together, the jackfruit heard the voice of his father …’ (This is a story about a child that was turned into a jackfruit.) [YBWN-T-01 2.13]
\z

\hspace*{-3.9pt}The following two examples describe two similar situations from a single telling of the same story.

\ea
\label{ex:raceeachother-1}
Ti,	paglinya	an	ta	Pwikan	an,	danen	an	gaabay	en	ta linya	ya	na	anda	en	na	maglumbaanay. \\\smallskip
\gll Ti,	pag-linya	an	ta	Pwikan	an,	danen	an	ga-abay	en	ta linya	ya	na	anda	en	na	mag-lumba-anay. \\
so	\textsc{nr.act}-line	\textsc{def.m}	\textsc{nabs}	sea.turtle	\textsc{def.m}	3\textsc{p.abs}	\textsc{def.m}	\textsc{i.r}-side.by.side	\textsc{cm}	\textsc{nabs} line	\textsc{def.f}	\textsc{def.m}	ready	\textsc{cm}	\textsc{lk}	\textsc{i.ir}-race-\textsc{rec} \\
\glt `So, when Sea Turtle made a line (on the ground), they were side-by-side at the line ready  to race each other.' [JCON-L-08 30.6]
\z
\ea
\label{ex:raceeachother-2}
Pagkitaay	danen	na	tama	en	na	mga	umang,	pasugiran	en	na kanen	an	daw	Pwikan	kisyem	sellem	 maglumbaay	dlagan. \\
\gll Pag-kita-ay	danen	na	tama	en	na	mga	umang,	pa-sugid-an	en	na kanen an	daw	Pwikan	kisyem	sellem	 mag-lumba-ay	dlagan. \\
\textsc{nr.act}-see-\textsc{rec}	3\textsc{p.abs}	\textsc{lk}	many	\textsc{cm}	\textsc{lk}	\textsc{pl}	hermit.crab	\textsc{t.r}-tell-\textsc{apl} \textsc{cm}	\textsc{lk} 3\textsc{s.abs}	\textsc{def.m}	and	sea.turtle	tomorrow	morning	\textsc{i.ir}-race-\textsc{rec}	run \\
\glt ‘When he (Hermit crab) and many hermit crabs met together, (he) told (the many hermit crabs) that he and Sea Turtle tomorrow morning will race each other.’ [JCON-L-08 19.4].
\z

For some verbs, the only reciprocal suffix allowed is -\textit{ay}:
\ea
\textbf{Magtabangay}  ki  ta  pag-ubra. \\\smallskip
\gll \textbf{Mag-tabang-ay}  ki  ta  pag--ubra. \\
\textsc{i.ir}-help-\textsc{rec}  1\textsc{p.incl.abs}  \textsc{nabs}  \textsc{nr.act}-work \\
\glt ‘Let’s help each other in working.’ \\\smallskip
* Magtabanganay ki ta pag-ubra.
\z
\ea
Nangusmuran  a  din  na  \textbf{gasugatay}  kay  gina. \\\smallskip
\gll Na-ngusmod-an  a  din  na  \textbf{ga-sugat-ay}  kay  gina. \\
\textsc{a.hap.r}-frown-\textsc{apl}  1\textsc{s.abs}  3\textsc{s.erg}  \textsc{lk}  \textsc{i.r}-meet-\textsc{rec}  1\textsc{p.excl.abs}  earlier \\
\glt ‘S/he frowned at me when we met each other earlier.’ \\\smallskip
* gasugatanay
\z

With some other verbs, such as \textit{ambaļ} ‘say’ and \textit{palangga} ‘love’, -\textit{anay} asserts a more binding, intense, united, or closer feeling or relationship than -\textit{ay} does. For example, \textit{mag-ambaļanay} means ‘to make an agreement’ such as families meeting together to talk about the wedding of the children (example \ref{bkm:Ref414723411}). \textit{Mag-ambaļay}, on the other hand, means to speak to each other again after some time of not speaking (example \ref{bkm:Ref414723505}).

\ea
\label{bkm:Ref414723411} \label{bkm:Ref414723505}
\textbf{Mag-ambaļanay}  danen  an  parti  ta  kasaļ. \\\smallskip
\gll \textbf{Mag--ambaļ-anay}  danen  an  parti  ta  kasaļ. \\
\textsc{i.ir}-say-\textsc{rec}  3\textsc{p.abs}  \textsc{def.m}  about  \textsc{nabs}  wedding \\
\glt ‘They will make an agreement with each other about the wedding.’
\z
\ea
\label{ex:nolongerangry}
\textbf{Gaambaļay}  en  danen  an  tak  uļa  en  danen  gilek. \\\smallskip
\gll \textbf{Ga-ambaļ-ay}  en  danen  an  tak  uļa  en  danen  gilek. \\
\textsc{i.r}-say-\textsc{rec}  \textsc{cm}  3\textsc{p.abs}  \textsc{def.m}  because  \textsc{neg.r}  \textsc{cm}  3\textsc{p.abs}  angry \\
\glt ‘They are now talking to each other because they are no longer angry.’
\z

\hspace*{-.6pt}The inflected verb \textit{gapalanggaay} means to ‘love each other’ (example \ref{bkm:Ref414723576}), while \textit{gapalanggaanay} means to ‘seriously love each other from one’s heart’ (example \ref{bkm:Ref414723659}).

\ea
\label{bkm:Ref414723576} \label{bkm:Ref414723659}
\textbf{Gapalanggaay}  ki  ta  kada  isya. \\\smallskip
\gll \textbf{Ga-palangga-ay}  ki  ta  kada  isya. \\
\textsc{i.r}-love-\textsc{rec}  1\textsc{p.incl.abs}  \textsc{nabs}  each  one \\
\glt ‘We love each other.’
\z
\ea
Malipayen  a  daw  makita  ko  ake  na  mga kabataan  na  \textbf{gapalanggaanay}  daw  gataod,  uļa  \textbf{gaisaay}, gatabangay,  uļa  gabatok  ta  iran  na  mga  maguļang daw  iran  na  mga  ginikanan. \\\smallskip
\gll Ma-lipay-en  a  daw  ma-kita  ko  ake  na  mga ka-bata-an  na  \textbf{ga-palangga-anay}  daw  ga-taod,  uļa  \textbf{ga-isa-ay}, ga-tabang-ay,  uļa  ga-batok  ta  iran  na  mga  maguļang daw  iran  na  mga  ginikanan. \\
\textsc{adj}-joy-\textsc{adj} 1\textsc{s.abs}  if/when  \textsc{a.hap.ir}-see  1\textsc{s.erg} 1\textsc{s.gen}  \textsc{lk}  \textsc{pl}
\textsc{nr}-children-\textsc{nr}  \textsc{lk}  \textsc{i.r}-love-\textsc{rec}  and  \textsc{i.r}-respect  \textsc{neg.r}  \textsc{i.r}-selfish-\textsc{rec} \textsc{i.r}-help-\textsc{rec}  \textsc{neg.r}  \textsc{i.r}-rebel  \textsc{nabs}  3\textsc{p.gen}  \textsc{lk}  \textsc{pl}  older.sibling and  3\textsc{p.gen}  \textsc{lk}  \textsc{pl} parent \\
\glt `I am joyful when I see my children love each other (from the heart) and give respect, not being selfish toward each other, helping each other, not rebelling against their older siblings and their parents.’ [CBWE-C-05 5.7]
\z

Since there is no consistent pattern of meaning variation between the two forms \nobreakdash-\textit{ay} and \nobreakdash-\textit{anay} when both are possible, we consider them to be partially autonomous allomorphs, and gloss them both as \textsc{rec}. It should also be noted that there is no reasonable sense in which -\textit{anay} may be considered a combination of the applicative suffix -\textit{an} plus -\textit{ay}.  First, there are no other situations in which two suffixes simultaneously appear on one verb. Second, examples with -\textit{anay} have the same, intransitive, argument structure as those with -\textit{ay}. Since the applicative construction normally changes the argument structure of the clause by making it transitive (see \ref{bkm:Ref394919927}), an applicative analysis of -\textit{anay} is excluded. Additional associative and a special emphatic usage of reciprocal morphology are discussed in \chapref{chap:voice}, \sectref{sec:reflexivesandreciprocals}.
\is{reciprocal morphology|)}

\section{Exclamatory -\textit{a} and -\textit{i}}
\label{sec:exclamatory}\is{exclamatory morphology|(}

What we are calling the \isi{exclamatory} suffixes are always “optional”, but may only be used when certain substantive words or phrases precede a grammatically transitive verb in realis modality. In this circumstance, any enclitic pronouns or adverbs in the clause must also appear before the verb. We will refer to the contexts in which exclamatory suffixes appear as “candidate contexts”. In other words, they are situations in which these suffixes may occur, but are not required. They are simply precluded in all other contexts. The following are candidate contexts:

\begin{enumerate}
\item When the interrogative adverbs \textit{indi} ‘where’ or \textit{kan-o} ‘when’ appear before a grammatically transitive, realis verb. Recall that these interrogative words do not reference core verbal arguments. Rather they reference oblique adverbial adjuncts (see \chapref{chap:modification}, \sectref{sec:adverbs}).
\item When any locational or temporal adverb, such as \textit{anduni} ‘now,’ or \textit{naan} ‘spatial definite marker’, appears before a grammatically transitive realis verb.
\item When the realis negative \textit{uļa} appears before a grammatically transitive realis verb.
\end{enumerate}


The function of both exclamatory suffixes is to express a more “excited”, “animated” or “surprised” modality. They are used most frequently in episodic peaks of narrative discourse. For this reason, in this grammar free English translations of examples that contain exclamatory suffixes are followed by the exclamation point (!).

In general (see further below for some unusual usages), \textit{{}-i} is the exclamatory form of the applicative \textit{{}-an} suffix, while \textit{{}-a} is the exclamatory form for other grammatically transitive verbs. It co-occurs with the realis transitive inflections \textit{pa}{}- and \textit{na}{}-. It is clear that both -\textit{i} and \textit{-a} occur only on grammatically transitive verbs because, they never co-occur with the inflectional prefixes that mark grammatical intransitivity, \textit{ga}{}- and (\textit{naka}{}- (see \chapref{chap:verbstructure}, \tabref{tab:inflectionalaffixes}). Furthermore, when they co-occur with the “ambitransitive” happenstantial inflectional prefix \textit{na}{}- the argument structure is always transitive. They also never occur in irrealis modality.

The following are some examples of the basic usages of these suffixes. The a examples below illustrate non-exclamatory applicatives, while the b examples illustrate exclamatory applicative constructions. The c examples illustrate the ungrammaticality of the -\textit{i} suffix outside of the candidate contexts:

\ea Preposed adverb \\
    \ea
    Anduni  kay  nang  naabutan  ta  pelles. \\\smallskip
 \gll Anduni  kay  nang  na-abot-an  ta  pelles. \\
    now/today  1\textsc{p.excl.abs}  only/just  \textsc{a.hap.r}-arrive-\textsc{apl}  \textsc{nabs}  strong.wind \\
    \glt ‘Just now a strong wind came upon us.’
    \newpage
    \ex
    Anduni  kay  nang  \textbf{naabuti}  ta  pelles. \\\smallskip
 \gll Anduni  kay  nang  \textbf{na-abot-i}  ta  pelles. \\
     now/today  1\textsc{p.excl.abs}  only/just  \textsc{a.hap.r}-arrive-\textsc{xc.apl}  \textsc{nabs}  strong.wind \\
    \glt ‘Just now a strong wind came upon us!’
    \ex
    *Naabuti kay ta pelles!
    \z
\z
\ea Realis negative \textit{uļa} \\
    \ea
    Uļa  danen  napatayan  ta  baboy. \\\smallskip
 \gll Uļa  danen  na-patay-an  ta  baboy. \\
    \textsc{neg.r}  3\textsc{p.abs}  \textsc{a.hap.r}-die-\textsc{apl}  \textsc{nabs}  pig \\
    \glt ‘The pig did not die on them.’
    \ex
    Uļa  danen  \textbf{napatayi}  ta  baboy. \\\smallskip
 \gll Uļa  danen  \textbf{na-patay-i}  ta  baboy. \\
    \textsc{neg.r}  3\textsc{p.abs}  \textsc{a.hap.r}-die-\textsc{xc.apl}  \textsc{nabs}  pig \\
    \glt ‘The pig did not die on them!'
    \ex
    *Napatayi danen ta baboy!
    \z
\z

\ea Q-word \\
    \ea
    Kan-o  danen  an  napatayan  ta  baboy? \\\smallskip
 \gll Kan-o  danen  an  na-patay-an  ta  baboy? \\
    when  3\textsc{p.abs}  \textsc{def.m} \textsc{a.hap.r}-die-\textsc{apl}  \textsc{nabs} pig \\
    \glt ‘When did the pig die on them?’
    \ex
    Kan-o  danen  an  \textbf{napatayi}  ta  baboy? \\\smallskip
 \gll Kan-o  danen  an  \textbf{na-patay-i}  ta  baboy? \\
    when  3\textsc{p.abs}  \textsc{def.m}  \textsc{a.hap.r}-die-\textsc{xc.apl}  \textsc{nabs}  pig \\
    \glt ‘When did the pig die on them?!’
    \ex
    *Napatayi danen an ta baboy?!
    \z
\z

In the next set of examples, the a. examples illustrate plain non-applicatives, the b. examples illustrate corresponding exclamatory constructions, and the c. examples illustrate the ungrammaticality of the -\textit{a} suffix outside of the candidate contexts:

\newpage
\ea Preposed adverb \\
     \ea 
     Anduni  ko  nang  narwad  rilo  ya. \\\smallskip
\gll Anduni  ko  nang  na-duwad  rilo  ya. \\
        now/today  1\textsc{s.erg}  only/just  \textsc{a.hap.r}-lose  watch  \textsc{def.f} \\
    \glt ‘I just now happened to lose the watch.’
    \ex
    Anduni  ko  nang  \textbf{narwara}  rilo  ya. \\\smallskip
 \gll Anduni  ko  nang  \textbf{na-duwad-a}  rilo  ya. \\
     now/today  1\textsc{s.erg}  only/just  \textsc{a.hap.r}-lose-\textsc{xc}  watch  \textsc{def.f} \\
    \glt ‘I just now happened to lose the watch!’
    \ex
    *Narwara ko rilo ya!
    \z
\z
\ea
Realis negative \\
    \ea
    Uļa  ko  parwad  rilo  na  atag  din. \\\smallskip
 \gll Uļa  ko  pa-duwad  rilo  na  atag  din. \\
    \textsc{neg.r}  1\textsc{s.erg}  \textsc{t.r}-lose  watch \textsc{lk}  give  3\textsc{s.erg} \\
    \glt ‘I did not lose the watch s/he gave (me).’
    \ex
    Uļa  ko  \textbf{parwara}  rilo  na  atag  din. \\\smallskip
 \gll Uļa  ko  \textbf{pa-duwad-a}  rilo  na  atag  din. \\
    \textsc{neg.r}  1\textsc{s.erg}  \textsc{t.r}-lose-\textsc{xc}  watch \textsc{lk}  give  3\textsc{s.erg} \\
    \glt ‘I did not let the watch s/he gave (me) get lost!’
    \ex
    *Parwara ko rilo na atag din!
    \z
\z
\ea Q-word \\
    \ea
    Kan-o  no  narwad  rilo  ya? \\\smallskip
 \gll Kan-o  no  na-duwad  rilo  ya? \\
    when  2\textsc{s.erg}  2\textsc{s.a.hap.r}-lose  watch  \textsc{def.f} \\
    \glt `When did you happen to lose the watch?’
    \ex
    Kan-o  no  \textbf{narwara}  rilo  ya? \\\smallskip
 \gll Kan-o  no  \textbf{na-duwad-a}  rilo  ya? \\
    when  2\textsc{s.erg}  \textsc{a.hap.r}-lose-\textsc{xc}  watch  \textsc{def.f} \\
    \glt ‘When did you happen to lose the watch?!’
    \ex
    *Narwara no rilo ya!
    \z
\z

The following are some additional examples of exclamatory suffixes from the corpus.

\ea
\label{bkm:Ref395799694}
Indi  no  imo  \textbf{kamanga}  mga  bļawan  an  Pedro? \\\smallskip
\gll Indi  no  imo  \textbf{...-kamang-a}  mga  bļawan  an  Pedro? \\
where  2\textsc{s.erg}  \textsc{emph} \textsc{t.r}-get-\textsc{xc}    \textsc{pl}  gold  \textsc{def.m}  Pedro \\
\glt ‘Where did you get the gold, Pedro?!’ (The older brother was so surprised that the younger brother had gold, since they were very poor, and he wanted some too!) [CBWN-C-22 8.3]
\z
Example \REF{ex:seeds} is from a folk story in which the seeds of a certain squash were coins. So when the owners sold the squash, they kept all the money (seeds) for themselves. The narrator of this story was feeling the excitement of finding money in squash.
\ea
\label{ex:seeds}
… uļa  \textbf{patagi}  danen  ta  liso. \\\smallskip
\gll … uļa  \textbf{pa-atag-i}  danen  ta  liso. \\
\textsc{neg.r}  \textsc{t.r}-give-\textsc{xc.apl} 3\textsc{p.abs}  \textsc{nabs}  seed \\
\glt ‘(They) did not give them the seeds!’ [AION-C-01 11.3]
\z

In example \REF{ex:seeds} the absolutive argument \textit{danen} refers to  the recipients, the ones to whom the squash was sold. As illustrated in example \REF{bkm:Ref395799832}, when the recipient is the absolutive of this verb, it always occurs with an applicative suffix. Otherwise the theme is the absolutive, and in this example the theme, \textit{liso} ‘seed’, is marked as non-absolutive. Therefore, this is clearly an applicative construction.

Example \REF{bkm:Ref123024886} is from a story about a house fire. The speaker gets her younger siblings and hands them to a man in the yard to take care of while she is trying to help save some of the house. The use of the exclamatory suffix on the verb underscores the desperation of the situation. 

\ea
\label{bkm:Ref123024886}
Bag-o,  anen  ko  dya  \textbf{padawata}  ake  na  mga  mangngod. \\\smallskip
\gll Bag-o,  anen  ko  dya  \textbf{pa-dawat-a}  ake  na  mga  mangngod. \\
before  \textsc{ext.g} 1\textsc{s.erg}  \textsc{d}4\textsc{loc}  \textsc{t.r}-hand.to-\textsc{xc}  1\textsc{s.gen}  \textsc{lk}  \textsc{pl} younger.sibling \\
\glt ‘Then, there I handed my younger siblings off (to the man in the yard)!’ [RZWN-T-02 4.14]
\z

Finally, in example \REF{ex:getangry} the protagonist of the story and his brother climbed a tree and his brother fell. Surprisingly, when their mother was told of the event, she did not get angry.
\ea
\label{ex:getangry}
Uļa  man  kanen  \textbf{pagileki}  ta  nanay  din. \\\smallskip
\gll Uļa  man  kanen  \textbf{pa-gilek-i}  ta  nanay  din. \\
\textsc{neg.r} too  3\textsc{s.abs}  \textsc{t.r}-angry-\textsc{xc.apl}  \textsc{nabs}  mother  3\textsc{s.gen} \\
\glt ‘His mother did not get angry at him!’ [JCWN-L-38 16.8]
\is{exclamatory morphology|)}
\z
