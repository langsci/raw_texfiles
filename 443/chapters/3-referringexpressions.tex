\chapter{Referring expressions}
\label{chap:referringexpressions}
\section{Introduction}
\label{sec:introduction-3}

Referring\is{reference} to participants and props in discourse “scenes”\is{discourse scenes}\is{scenes} is one major task that people accomplish using language. Other major tasks include describing the situations that participants and props are involved in (\isi{predication}), and specifying the properties of the participants and props (\isi{modification}). In this chapter we will discuss linguistic structures in Kagayanen that fulfill a referring function. These will sometimes be described collectively as Referring Expressions (REs). In \chapref{chap:modification} we discuss modification, in \chapref{chap:non-verbalclauses} we describe non-verbal predication, and in \chapref{chap:verbstructure} we describe verbal predication.

In Kagayanen, as in most Western Austronesian languages, labels such as ``noun" and ``verb" are not lexical tags that mechanistically attach to individual entries apart from any communicative context. Rather, substantive roots\footnote{As opposed to purely grammatical particles such as adverbial enclitics and prenominal case markers.} are simply form + meaning composites that may be used in any way that speakers find helpful in communication. In other words, substantive roots in Kagayanen are for the most part \is{precategoriality} precategorial \citep{himmelmann2008}. 

While we acknowledge that roots are generally precategorial out of context, because of their meanings some roots are \textit{more likely} than others to function referentially, and others are \textit{more likely} to function predicatively. Our use of the term ``noun" in this grammar can be thought of as a shortcut for ``a root that, because of its semantics, is very likely to be used in a referential way in communication." Other forms and constructions that more rigidly serve this function include pronouns and referential phrases. The latter are what are traditionally referred to as ``noun phrases". 

Prototypical \isi{nouns} evoke concepts that are relatively “time-stable” \citep[55]{givon2001-1}, and which potentially have continuing presence on the \isi{discourse stage} \citep{hopper1984-2, du_bois1985}. Nouns\is{nouns} in Kagayanen exhibit no \isi{inflectional morphology}; \isi{case}, \isi{number} and \isi{definiteness} are expressed analytically by separate words, as discussed in \sectref{bkm:Ref117344615} below. However, there are several interesting and complex \isi{stem-forming processes} that create nouns from any root. \is{roots} These processes are collectively referred to as \textit{nominalization} (or ``noun-forming processes"). In \sectref{sec:nounforming} \is{nominalization}we catalog the systematic nominalization processes that we have observed in the corpus. In \sectref{sec:pronouns}, we describe four types of \isi{pronouns}: personal pronouns, interrogative pronouns, indefinite pronouns, and demonstrative pronouns. Finally, in \sectref{sec:referringphrases} we describe the general structure of \isi{Referring Phrases}, meaning groups of words that collectively refer to participants or props in discourse.

\section{Nouns}
\label{bkm:Ref117344615}\is{nouns|(}
We recognize three grammatically distinct classes of lexical nouns in Kagayanen: 1) \textit{common count nouns}\is{common count nouns} (``\isi{common nouns}" for short), 2) \textit{personal names}\is{personal names} (proper nouns referring to specific people), and 3) \textit{common mass nouns}\is{common mass nouns} (also known simply as ``\isi{mass nouns}" or ``\isi{non-count nouns}"). The structures of Referring Phrases headed by all \isi{common nouns} versus personal names are distinct, as described in \sectref{sec:basicreferringphrases} and \sectref{sec:referringphrasesheadedbypersonalnames}. The structures of Referring Phrases headed by nominalized clauses are also distinct. However, since nominalized clauses are not lexical items, we do not treat them as a sub-class of nouns. Their distinct morphosyntactic properties are described in \sectref{sec:referringphrasesheadedbynominalizedclauses}.

In this section we will give some preliminary examples of \isi{common nouns}, \isi{personal names}, and \isi{mass nouns}.

\is{common nouns|(}Prototypical common nouns are words that refer to things, people, animals, and other concrete items. They include such words as \textit{baļay} ‘house’, \textit{bata} ‘child’, \textit{umang} ‘hermit crab’, and \textit{suba} ‘river.’\is{common nouns|(}

Personal names\is{personal names|(}, on the other hand, refer to particular individuals, for example, \textit{Juan}, \textit{Neneng}, \textit{Bibing}, \textit{Gulang}, \textit{Angngod}, \textit{Dongdong}, and \textit{Bebot}. These nouns are grammatically distinct from common nouns in that they take a different set of \isi{prenominal case markers}, and can only be modified in certain limited ways. \tabref{tab:prenominalcasemarkers} illustrates the different prenominal markers used for common nouns and personal names. In \sectref{sec:referringphrasesheadedbypersonalnames} special morphosyntactic properties of personal names are described in more detail. Personal names are always distinguished in writing by the use of an initial capital letter.

\begin{table}
\caption{Prenominal case markers}
\label{tab:prenominalcasemarkers}
\begin{tabularx}{\textwidth}{XXll}
\lsptoprule
& Absolutive & Ergative/Genitive & Oblique \\
\midrule
Common nouns & \emptyset{} & ta  & ta  \\
 Personal names & \emptyset{} & \emptyset{} & ki \\
 Pronouns & \emptyset{} & \emptyset{} & ki \\
\lspbottomrule
\end{tabularx}
\end{table}

It should be noted that personal names are not the same as “proper nouns”\is{proper nouns} as defined in other language traditions. For example, names of specific places, languages, or people groups, such as \textit{Kagayanen} (the language or the people), \textit{Filipino} and \textit{Maynila} ‘Manila’,  are treated grammatically as common nouns in Kagayanen, for example, they take the \isi{prenominal case markers} associated with common nouns and not personal names. Nevertheless, because of tradition, these nouns are also written with initial capital letters.

Some \isi{kinship terms} may be used either as common nouns or as personal names, for example \textit{ta nanay} ‘to (someone’s) mother’ or \textit{ki Nanay} ‘to Mother’, \textit{ta manang} ‘to (someone’s) older sister’ or \textit{ki Manang} ‘to Older Sister’. The same is true of most other kinship terms for immediate family members, such as \textit{manong} ‘older brother’, \textit{tatay} ‘father’, \textit{lola} ‘grandmother,’ \textit{lolo} ‘grandfather,’ though not \textit{katagsa} ‘cousin.’

\ea
\textit{Nanay} ‘mother’ and \textit{tatay} ‘father’ used as common nouns: \\
Mabot  a  ta  baļay  nay  gangita a  \textbf{ta}  \textbf{nanay}  \textbf{ko}  daw  \textbf{tatay}  \textbf{ko}. \\\smallskip
 \gll M-abot  a  ta  baļay  nay  ga-ngita a  \textbf{ta}  \textbf{nanay}  \textbf{ko}  daw  \textbf{tatay}  \textbf{ko}. \\
\textsc{i.v.ir}-arrive  1\textsc{s.abs}  \textsc{nabs}  house  1\textsc{p.excl.gen}  \textsc{i.r}-look.for 1\textsc{s.abs} \textsc{nabs}  mother  1\textsc{s.gen} and  father  1\textsc{s.gen} \\
\glt `Arriving at our house, I was looking for my mother and my father.’ [BMON-C-04 1.4]
\z
\ea
\textit{Nanay} ‘mother’ used as a personal name: \\
Lugar isya  i  na  pirates  a  gainsa  \textbf{ki}  \textbf{Nanay}  ko  daw indi  tatay  ko. \\\smallskip
 \gll Lugar\footnotemark{}  isya  i  na  pirates  a  ga-insa  \textbf{ki}  \textbf{Nanay}  ko  daw indi  tatay  ko. \\
then  one  \textsc{def.n}  \textsc{lk}  pirates  \textsc{ctr}  \textsc{i.r}-ask  \textsc{obl.p} mother  1\textsc{s.gen}  if/when  where  father  1\textsc{s.gen} \\
\footnotetext{This word is from Spanish meaning ‘place,’ but in Kagayanen, it can also function as a discourse particle with the sense of ‘then,’ as in this example. There is an indigenous Kagayanen word \textit{lugay} meaning ‘long-time’ that is also used in this way. Some Kagayanen speakers prefer \textit{lugar} to \textit{lugay} as a discourse particle.}
\glt `Then one of the pirates asked Mother where is my father.’ [BCWN-C-04 8.1]
\is{personal names|)}
\z



\textit{Mass nouns}\is{mass nouns|(}, also known as \textit{non-count nouns}\is{non-count nouns} 
include most liquids, such as \textit{agas} ‘kerosine’, \textit{bino} ‘wine’, \textit{waig} ‘water’, \textit{gatas} ‘milk’, \textit{uļas} ‘sweat’, \textit{luwa} ‘tears’, \textit{dagat} ‘sea/ocean’, \textit{uran} ‘rain’, \textit{lengngessa} ‘blood’, etc. They also include gases, such as \textit{angin} ‘air/wind’, \textit{aso} ‘smoke’, \textit{tun-og} ‘dew/mist/fog’, most particle nouns, such as \textit{asin} ‘salt’, \textit{kļamay} ‘sugar’, \textit{beggas} ‘milled rice’, \textit{ummay} ‘unmilled rice’, \textit{kan-en} ‘cooked rice’, etc. Also included are many abstract nouns, including \textit{swirti} ‘luck’, \textit{kanļaman} ‘knowledge/wisdom’, \textit{laem} ‘hope’, \textit{andem} ‘ambition’, and collective nouns such as \textit{daļa} ‘stuff/things’, \textit{kwarta} ‘money’, \textit{pagkaan}, ‘food’, \textit{basura} ‘garbage’, \textit{sagbet}‘weeds/trash’, \textit{bļawan} ‘gold’, and \textit{pilak} ‘silver’.

Grammatically, mass nouns take the same \isi{prenominal case markers} as the ordinary common nouns. They also can occur in a noun phrase with the demonstrative adjectives, demonstrative determiners, genitive pronouns, adjectives, and some non-numeral quantifiers, such as \textit{tanan} ‘all’, \textit{tama} ‘many/much’, and \textit{sise nang} ‘just a little bit.’ However, mass nouns cannot be modified directly with numerals, noun classifiers, plural adjectives such as \textit{darko} ‘plural big’ and \textit{derse/derset} ‘plural small’. They cannot be referred to with the interrogative/relative pronoun \textit{pila} ‘how many’ or the quantifier \textit{pila nang} ‘just a few.’ To specify a quantity of a mass noun concept, one must mention a container, for example, \textit{tallo na sako ta beggas} ‘three sacks of rice’, or a measurement \textit{tallo na kilo ta beggas} ‘three kilograms of rice’ (see \sectref{sec:interrogativepronouns}). Mass nouns do not normally occur with the plural marker \textit{mga}, but may in certain contexts, as in examples \REF{ex:food} and \REF{ex:here}:
\ea
\label{ex:food}
kasklasi	na	mga	pagkaan \\\smallskip
 \gll kas-klasi	na	mga	pag-kaan \\
\textsc{red}-type \textsc{lk}	\textsc{pl} \textsc{nr.act}-eat \\
\glt ‘different kinds of food.’
\z

\ea
\label{ex:here}
    \ea
    \label{ex:here-a}
    Tama	na	mga	beggas	tinda	di. \\\smallskip
 \gll Tama	na	mga	beggas	tinda	di. \\
    many \textsc{lk} \textsc{pl} milled.rice	sell \textsc{d1loc} \\
    \glt ‘Much rice is sold here.’
    \ex
    \label{ex:here-b}
    Tama	na	mga	waig	di. \\\smallskip
 \gll Tama	na	mga	waig	di. \\
    many	\textsc{lk} \textsc{pl}	water	\textsc{d1loc} \\
    \glt ‘Much water is here.’
    \z
\z

Example \REF{ex:here-a} would be used in a situation where many different people are selling rice in the market. Example \REF{ex:here-b} asserts there are many instances of water, such as rivers, wells, waterfalls, and so on.

\is{mass nouns|)}
\is{definiteness|)}
\is{nouns|)}
\section{Noun-forming processes}\is{noun-forming processes|(}\is{nominalization|(}
\label{sec:nounforming}

In Kagayanen, as in many other Philippine languages, most inflected verb forms can function as Heads \is{heads, syntactic} or Modifiers within Referential Phrases (see \chapref{chap:verbstructure}, \sectref{bkm:Ref58564378} on inflected verb forms\is{inflected verbs}). In this sense, almost any verb form is potentially a nominalization. However, the reverse is not the case; there are certain affix patterns that create stems which may only be used as nouns. Some other affix patterns create stems which are commonly used as nouns, and their use as verbs is uncommon. Finally there are noun-forming processes that are identical to verbal inflectional processes in form, but which create noun stems whose meanings are unrelated to the analogous forms used as verbs. In this section, we describe noun-forming processes that do not have exact counterparts in verbal inflection, making note of any similarities with verb morphology as may exist. In the following subsections we emphasize the \textit{productive}\is{productive} uses of these nominalization processes. There are other nominal forms built with these processes for which the meanings cannot be predicted according to any consistent pattern. Several such forms are noted in the subsections below, though a thorough catalog of these \textit{lexicalizations}\is{lexicalization} is more appropriate for a dictionary than a grammatical description. There also exists much variation among Kagayanen speakers in how these nominalization processes are used. We have tried to account for this variation where possible, but there are undoubtedly variations in form and meaning that we have overlooked.

\subsection{“Zero” deverbalization}
\label{sec:zerodeverbalization}

\is{zero deverbalization|(}Inherently verbal roots may function as action nominalizations with no overt morphology. We refer to this phenomenon as \textit{zero deverbalization}\is{zero deverbalization}. In the following examples, the deverbal nouns are presented in bold:

\ea
Nakita  danen  na  galaway  ta  \textbf{tunuga}. \\\smallskip
 \gll Na-kita  danen  na  ga-laway  ta  \textbf{tunuga}. \\
\textsc{a.hap.r}-see  3\textsc{p.abs}  \textsc{lk}  \textsc{i.r}-salivate  \textsc{nabs}  sleep \\
\glt ‘They saw that (surgeon fish) was salivating in (his) \textbf{sleep}.’ [JCWN-T-27 6.2]
\z
\ea
Gaketkeret  waleng  ta  \textbf{tawa}. \\\smallskip
 \gll Ga-ket-keret  waleng  ta  \textbf{tawa}. \\
\textsc{i.r-red}-wrinkle  face  \textsc{nabs}  laugh \\
\glt ‘The face is wrinkled up with \textbf{laughing}.’ [CBOE-C-01 4.2]
\z
\ea
Tuduan  din  gid  en  \textbf{dļagan}  \textbf{din}  \textbf{an}. \\\smallskip
 \gll ...-tudo-an  din  gid  en  \textbf{dļagan}  \textbf{din}  \textbf{an}. \\
\textsc{t.r}-put.all.effort-\textsc{apl}  3\textsc{s.erg}  \textsc{int}  \textsc{cm}  run  3\textsc{s.gen}  \textsc{def.m} \\
\glt ‘He really put all effort into \textbf{his} \textbf{running}.’ [DBWN-T-26 9.1]
\is{zero deverbalization|)}
\z

\subsection{Root reduplication – diminutive, similitive, collective}
\label{sec:rootreduplication}

\is{root reduplication|(}
Some nominal roots can undergo partial and/or complete reduplication. For roots meaning X, reduplicated forms refer to things that are ‘a small X’, ‘not a real X’, ‘pretend X’, ‘pattern of X’, ‘something that looks like or is used in the same way as X’, ‘some sort of X’, or ‘a collection of X’. These processes are formally identical, and semantically comparable to root reduplication in the verb, as discussed in \chapref{chap:stemformingprocesses}, \sectref{sec:rootreduplication-verbs}. Complete reduplication applies to roots of any syllable structure, and any historical origin. Partial reduplication mostly applies to Austronesian roots which begin with a stop consonant, /l/ or /ļ/. Additionally, there are two types of partial reduplication:

\textbf{Partial} \textbf{reduplication} \textbf{pattern} \textbf{1:}\is{partial reduplication|(}\is{reduplication!partial|(} The first C(C)V sequence plus the root-final consonant is reduplicated. This is the most common pattern (see below for examples).

\textbf{Partial} \textbf{reduplication} \textbf{pattern} \textbf{2:} The first C(C)VC sequence is reduplicated.

It seems that roots in which the consonant following the first vowel is a velar (\textit{ng} or \textit{g}) are more likely to follow pattern 2, but this is not consistent. Many such roots do not permit partial reduplication at all.

\is{full reduplication|(}\is{reduplication!full|(}The following is a representative list of nouns with their full and partial reduplicative forms, along with their common meanings. This list is divided between roots that follow partial reduplication pattern 1 (example \ref{ex:pattern}), those that follow partial reduplication pattern 2 (example \ref{ex:boat}), those that do not allow partial reduplication (example \ref{ex:shell}), and a few nouns that do not seem to undergo reduplication at all (example \ref{ex:barrier}). For roots that allow both full and partial reduplication, the meaning of the reduplicative forms is usually the same (though there is some variation from speaker to speaker as to how these forms are used):

\ea 
\label{ex:pattern}
\begin{tabbing}
kaon ‘chest/box/drawer’ \hspace{.1cm} \= Complete redup. \hspace{.1cm} \= \kill
\textbf{Root}  \>    \textbf{Complete redup.} \> \textbf{Partial redup. (pattern 1)} \\
buļak ‘flower'\>buļak-buļak\>bukbuļak ‘floral pattern’ \\
kaon ‘chest/box/drawer’ \> kaon-kaon  \>    kankaon ‘dresser/desk’ \\
puon ‘trunk/stem’ \> puon-puon  \>    punpuon ‘different kinds of \\
 \> \> trunks/stems’, `some kind of \\
 \> \>                trunk/stem’ \\
ļaod ‘deep ocean’ \>  ļaod-ļaod \>      ļadļaod ‘somewhat deep ocean’ \\
payag ‘nipa house’ \> payag-payag  \>   pagpayag ‘small rest house’ \\
saag ‘floor’  \>  saag-saag   \>   sagsaag ‘sort of a floor’ \\
kaoy ‘wood’/’tree’ \> kaoy-kaoy \>     kaykaoy ‘many trees’ \\
bagoļ ‘coconut shell’ \>  bagoļ-bagoļ  \>    baļbagoļ ‘many coconut shells’ \\ \> \> / ‘skull’ \\
laas ‘undomesticated' \>  laas-laas \>        laslaas ‘sort of undomesticated \\
 \> \>                 animal’ \\
uran ‘rain’  \>  uran-uran \> un-uran  ‘rainy season’ / \\
 \> ‘lots of rain' \> `keep on raining’ \\
\end{tabbing}
\z
\ea 
\label{ex:boat}
\begin{tabbing}
kaon ‘chest/box/drawer’ \hspace{.1cm} \= Complete redup. \hspace{.1cm} \= \kill
\textbf{Root}  \>    \textbf{Complete redup.} \> \textbf{Partial redup. (pattern 2)} \\
bļangay ‘two-masted boat’ \>  bļangay-bļangay  \>  bļangbļangay / banbļangay \\
\> \> ‘many two-masted boats’ / \\
\> \> ‘toy boat’ \\
baļay ‘house’ \> baļay-baļay \>  baļbaļay ‘bamboo resthouse’ / \\       \> \> ‘playhouse’ / ‘dollhouse’ \\
klasi ‘kind, type’ \>  klasi-klasi \>       klasklasi / kasklasi ‘various \\
\> \> kinds/types’ \\
bļagaw ‘blond color’ \> bļagaw-bļagaw  \>    bļagbļagaw ‘little blonde’ \\
buļan `month/moon' \> buļan-buļan \> bunbuļan `every month \\
minuto `minute' \> minuto-minuto \> minu-minuto `every minute' \\
taon `year' \> taon-taon \> tantaon `every year' \\
segundo `second' \> *segundo-segundo \\
duminggo `week' \> *duminggo-duminggo
\end{tabbing}
\z
\ea
\label{ex:shell}
\begin{tabbing}
kaon ‘chest/box/drawer’ \hspace{.1cm}  \=  \kill
\textbf{Root}  \>    \textbf{Complete redup. (No partial redup.)} \\
mangkok ‘bowl’ \>  mangkok-mangkok ‘toy bowl’ or ‘a kind of shell’ \\
pandan ‘screw pine plant’ \>  pandan-pandan ‘many screw pine plants’ \\
pinggan ‘plate, dish’ \>  pinggan-pinggan   ‘toy plate/dish’ \\
sungay ‘horn’ \>  sungay-sungay ‘many horns’ \\
amo ‘monkey’ \>  amo-amo ‘toy monkey’ \\
abeļ ‘tube skirt’ \>  abeļ-abeļ ‘many tube skirts’ \\
adlaw ‘day, sun’ \>  adlaw-adlaw ‘every day’ \\
oras `hour/time' \> oras-oras `every hour' \\
alad ‘offering' \>  alad-alad ‘many offerings or small offering’ \\
ikam ‘mat’ \>    ikam-ikam ‘small toy mat’ \\
ikog ‘tail’  \>   ikog-ikog ‘small tail’ \\
ilaw ‘light’  \>  ilaw-ilaw ‘many lights' \\
imbis ‘fish scales’ \>  imbis-imbis ‘many fish scales’ \\
ubra ‘work’ \>  ubra-ubra ‘much work’ \\
ubos ‘all included’ \>  ubos-ubos ‘completely all included’ \\
uling ‘charcooal’ \>  uling-uling ‘lots of  charcoal’ \\
kabaw ‘carabao’ \> kabaw-kabaw ‘small carabao’ \\
dļaga ‘single woman’ \>  dļaga-dļaga ‘younger single girl’ \\
dļunan ‘boundary’ \>  dļunan-dļunan ‘many boundaries’ \\
bļakaw ‘a kind of spirit’ \> bļakaw-bļakaw ‘many spirits’ \\
bļanak ‘mullet fish’ \>  bļanak-bļanak ‘many mullet fish’ \\
bļubog ‘gruel, porridge’ \>  bļubog-bļubog ‘a little gruel’ \\
bļumbon ‘sandbar’ \>  bļumbon-bļumbon ‘many sandbars', `sm. sandbar’ \\
bļutot ‘boil, abscess’ \> bļutot-bļutot ‘many boils/abcesses’, `small boil' \\
kļamay ‘sugar’ \>  kļamay-kļamay ‘much sugar’ \\
plano ‘plan’ \>  plano-plano ‘many plans’ \\
pļaas ‘monitor lizard’  \> pļaas-pļaas ‘many monitor lizards’ \\
tļaba ‘barnacle, kind of shell’ \\
\> tļaba-tļaba ‘many such shells’
\end{tabbing}
\z
\ea
\label{ex:barrier}
\begin{tabbing}
sļait ‘point or barb of a spear’ \hspace{.15cm}\= \kill
\textbf{Root} \> \textbf{Neither complete nor partial reduplication} \\
bļabag ‘barrier’ \> *bļabag-bļabag *bļagbļabag \\
bļagen ‘vine or creeping plant’ \> *bļagen-bļagen *bļagbļagen *bļanbļagen \\
bļaran ‘drying place’ \> *bļaran-bļaran *bļanbļaran \\
bļatong ‘string bean’ \> *bļatong{}-bļatong *bļangbļatong \\
bļawan ‘gold’ \> *bļawan-bļawan *bļanbļawan *bļawbļawan \\
bļaen ‘a kind of basket’ \> *bļaen-bļaen *bļanbļaen \\
pļapa ‘coconut frond’ \> *pļapa-pļapa *pļappļapa \\
bļukan ‘arm’ \> *bļukan-bļukan *bļunbļukan \\
klamutaw ‘pupil, iris’ \> *klamutaw-klamutaw *klawklamutaw \\
kļim-an ‘fifty’ \> *kļim-an-kļim-an *kļankļim-an \\
kļubay ‘bottle gourd’ \> *kļubay-kļubay *kļuykļubay *kļukļubay \\
mļuto ‘provision for a trip’ \> *mļuto{}-mļuto *mļuɁmļuto \\
plantsa ‘clothes iron’ \> *plantsa-plantsa *planplantsa \\
sļait ‘point or barb of a spear’ \> *sļaɪt-sļaɪt *sļatsļaɪt \\
ismagel ‘flipflop’ \> *ismagel-ismagel *ismalismagel \\
tļunon ‘wild pig’ \> *tļunon-tļunon *tļuntļunon *tļutļunon \\
tļunay ‘splinter, thorn or  \> *tļunay-tļunay *tļuytļunay *tļuntļunay \\
anything stuck in the skin’ \> *tļutļunay
\end{tabbing}
\z

On verbs and adjectives this morphological process may also have comparable semantic effects (e.g., ‘\textsc{verb} iteratively/continuously’, ‘pretend to \textsc{verb}’, ‘\textsc{verb} randomly,’ ‘sort of \textsc{adj}’). These are described for verbs in \chapref{chap:stemformingprocesses}, \sectref{sec:rootreduplication-verbs}, and for adjectives in \chapref{chap:modification}, \sectref{sec:rootreduplication-derivation}.
\is{reduplication!full|)}\is{full reduplication|)}
\is{reduplication!partial|)}\is{partial reduplication|)}
\is{root reduplication|)}

\subsection{\textit{ka-}: abstract quality}
\label{sec:ka}

With some adjectives and stative verbs the prefix \textit{ka}{}- alone derives a noun that refers to the \isi{abstract quality} expressed by the root. For example:

\ea
\label{ex:ka}
\begin{tabbing}
\hspace{6cm}  \= \kill
\textbf{Root} \> \textbf{Derived form} \\
dayad ‘good/pretty’ (adj.)  \>   kadayad ‘goodness/prettiness’ \\
bellay ‘difficult’ (adj.) \> kabellay ‘difficulty’ \\
baskeg/biskeg ‘strong’ (adj.) \> kabaskeg/kabiskeg ‘strength’ \\
sakit ‘painful’ (adj.) \> kasakit ‘pain’ \\
kapoy ‘tired’ (adj.) \> kakapoy ‘tiredness’ \\
pawa ‘bright’ (adj.) \> kapawa ‘brightness’ \\
lineng ‘quiet/peaceful’ (adj.) \> kalineng ‘quiet/peace’ \\
tawway ‘inwardly peaceful’ \> katawway ‘inner peacefulness’ \\
linaw ‘calm weather or sea' \> kalinaw ‘calmness’ \\
bui ‘live’ (v.) \> kabui ‘life’ \\
lipay ‘feel joyful’ (v.) \> kalipay ‘joy’ \\
sadya ‘enjoy/have fun’ (v.) \> kasadya ‘enjoyment/fun’ \\
sebe ‘feel sad’ (v.) \> kasebe ‘sadness’ \\
lised ‘feel distress’ (v.) \> kalised ‘distress’ \\
ļettem ‘hungry’ (v.) \> kaļettem ‘hunger’ \\
dalok ‘gluttonous’ (adj.) \> kadalok ‘gluttony’ \\
adlek ‘feel afraid’ (v.) \>  kaadlek/kadlek ‘fear’ \\
gilek ‘angry’ (v.) \> kagilek ‘anger’ \\
lain ‘bad’ (adj.) \> kalain ‘badness’
\end{tabbing}
\z

Examples \REF{bkm:Ref447113409} and \REF{bkm:Ref441302833} illustrate \textit{ka}{}- nominalizations used as nouns in context:

\ea
\label{bkm:Ref447113409}
Daw  ino  na  \textbf{kadayad}  mga  dikurasyon  an  na  mga  paniin. \\\smallskip
 \gll Daw  ino  na  \textbf{ka-dayad}  mga  dikurasyon  an  na  mga  paniin. \\
if/when  what  \textsc{lk}  \textsc{nr}-good \textsc{pl}  decoration  \textsc{def.m}  \textsc{lk}  \textsc{pl}  shells \\
\glt ‘What \textbf{beauty} of the shell decorations!’ [DBWN-T-23 4.5]
\z

\newpage
\ea
\label{bkm:Ref441302833}
Tagan  kay  ta  \textbf{kabaskeg}  ta ame  na  mga  lawa para  mag-ubra  kay... \\\smallskip
 \gll \emptyset{}-atag-an  kay  ta  \textbf{ka-baskeg}  ta ame  na  mga  lawa para  mag--ubra  kay... \\
\textsc{t.ir}-give-\textsc{apl}  1\textsc{p.excl.abs}  \textsc{nabs}  \textsc{nr}-strong  \textsc{nabs}  1\textsc{p.excl.gen}  \textsc{lk}  \textsc{pl}  body for \textsc{i.ir}-work 1\textsc{p.excl.abs} \\
\glt ‘Give us \textbf{strength} for our bodies in order for us to work ...' [ETOB-C-02 1.7]
\z

\subsection{\textit{ka-...-én/-ón}: abstract quality/dimension}
\label{sec:ka-en}

The affix combination \textit{ka}{}-...-\textit{én/-ón} nominalizes \isi{dimensional adjectives} and a few nouns and verbs. It is the normal way of forming abstract nouns of dimension in space and time:
\ea
\label{bkm:Ref503507725}
\begin{tabbing}
\hspace{3.8cm} \= \kill
\textbf{Root} \> \textbf{Derived form} \\
wayang ‘wide space’ \> kawàyangén ‘inside width’ \\
ļapad ‘wide’ \> kaļàparén ‘outside width’ \\
datas ‘high’ \> kadàtasén ‘height’ \\
ļangkaw ‘long’ \> kaļàngkawén ‘length’ \\
daļem ‘deep’ \> kadàļemén ‘depth’. \\
madyo/adyo ‘far’ \> kamàdyuón/kaàdyuón ‘distance’ \\
bakod/baked ‘big’ \> kabàkurón/kabàkerén ‘size’ \\
darko ‘big (plural)’ \> kadàrkuón ‘size (plural)’ \\
derse/derset ‘small (pl.)’ \> `smallness' (plural)’ \\
layog ‘tall’ \> kalàyugón ‘height’ (usually people) \\
baaļ ‘thick’ \> kabàaļén ‘thickness’ (diameter) \\
dakmeļ ‘thick’ \> kadàkmeļén ‘thickness’ (of flat objects and rain) \\
lugay ‘long time’ \> kalùgayén ‘length of time’ \\
lawɪg ‘long time’ \> kalàwigén ‘length of a long time’ \\
lipo `short' \> kalìpuón `shortness' \\
baba `short' \> kabàbaén `shortness' \\
gamay `narrow' \> kagàmayén `narrowness'
\end{tabbing}
\z

\textit{Ka}{}-...-\textit{én} can also occur with other kinds of adjectives. The result is an abstract noun referring to the quality expressed by the adjective (similar to that of \textit{ka}{}- above), but perhaps with more intensity. The words listed in \REF{ex:oldage} with \textit{ka-} ... \textit{{}-én} usually describe the consistent character or habits of a person. This is in contrast to the use of \textit{ka-} alone and \textit{pagka-} described below. For example, \textit{tamad} ‘lazy’ becomes \textit{katámad} ‘normal laziness’, \textit{katàmarén} ‘characteristic or habitual laziness’ and \textit{pagkatámad} ‘laziness in the moment’ (see \sectref{sec:ka} above and \sectref{sec:pagka} below).
\ea
\label{ex:oldage}
\begin{tabbing}
\hspace{4.5cm} \= \kill
\textbf{Root} \> \textbf{Derived form} \\
manakem ‘old’ \> kamànakemén ‘old age’ \\
dasig ‘fast’ \> kadàsigén ‘speed’ \\
daļas ‘fast’ \> kadàļasén ‘speed’ \\
beg-at ‘heavy’ \> kabèg-atén ‘weight’ \\
tama ‘many’ \> katàmaén ‘amount’ \\
tambek ‘fat’ \> katàmbekén ‘fatness/obesity’ \\
niwang ‘skinny’ \> kanìwangén ‘skinniness’ \\
dayaw ‘beautiful’ \> kadàyawén ‘beauty’ \\
minog ‘red’ \> kamìnugón ‘redness’ \\
inog ‘ripe’ \> kaìnugón ‘ripeness’ \\
dayad ‘good/pretty’ \> kadàyarén ‘goodness/prettiness’ \\
bellay ‘difficult’ \>  kabèllayén ‘difficulty’ \\
biskeg ‘strong’ \>  kabìskegén ‘strength’ \\
sakit ‘painful’ \>  kasàkitén ‘painfulness’ \\
lineng ‘quiet/peaceful’ \>  kalìnengén ‘quietness/peacefulness’ \\
tawway ‘inwardly peaceful’ \>  katàwwayén ‘inner peacefulness’ \\
lipay ‘feel joyful’ \>  kalìpayén ‘joyfulness’ \\
sadya ‘enjoy/have fun’ \>  kasàdyaén ‘enjoyment’ \\
sebe ‘feel sad’ \>  kasèbeén ‘sadness’ \\
lised ‘feel distress’ \>  kalìserén ‘distress’ \\
ļettem ‘hungry’ \>  kaļèttemén ‘hunger’ \\
dalok ‘gluttonous’ \>  kadàlukón ‘gluttony’ \\
gilek ‘angry’ \>  kagìlekén ‘anger’ \\
tamad ‘lazy’ \> katàmarén ‘laziness’ \\
uļa ‘nothing’ \> kaùļaén ‘having nothing/destitution’ \\
pubri ‘poor’ \> kapùbrién ‘poverty’ \\
beet ‘kind/good/considerate’ \> kebèetén ‘kindness/consideration’
\end{tabbing}
\z

Examples \REF{bkm:Ref116398555}{}-\REF{bkm:Ref116398718} illustrate some \textit{ka}{}-...-\textit{én} nominalizations in context:

\ea
\label{bkm:Ref116398555}
Piro  ta  \textbf{kasàkeén}  ta  iran  na  ubra  uļa  danen  namasmasi  na  iran  na  bata  may  kalain  ta  beet  ki  danen. \\\smallskip
 \gll Piro  ta  \textbf{ka-s}à\textbf{ke-én}  ta  iran  na  ubra  uļa  danen  na-masmas--i  na  iran  na  bata  may  ka-lain  ta  beet  ki  danen. \\
but  \textsc{nabs}  \textsc{nr}-busy-\textsc{nr}  \textsc{nabs}  3\textsc{p.gen}  \textsc{lk}  work  \textsc{neg.r}  3\textsc{p.erg}  \textsc{a.hap.r}-notice-\textsc{xc.apl} \textsc{lk}  \textsc{3}\textsc{p.gen}  \textsc{lk}  child  \textsc{ext.in}  \textsc{nr}-bad  \textsc{nabs}  inner.being  \textsc{obl.p}  3p \\
\glt `But in the \textbf{business} of their work they did not notice that their child had bad feelings towards them.’ [YBWN-T-07 2.5) \\
\z
\ea
Ame  na  lugar  sikad  lineng  kag  uyi may  \textbf{kaļangkawén}  na  katursi  kilumitros. \\\smallskip
 \gll Ame  na  lugar  sikad  lineng  kag\footnotemark{}  u-yi may  \textbf{ka-ļangkaw-én}  na  katursi  kilumitros. \\
1\textsc{p.excl.gen}  \textsc{lk}  place  very  peaceful  and  \textsc{emph-d1abs} \textsc{ext.in}  \textsc{nr}-long-\textsc{nr}  \textsc{lk}  fourteen  kilometers \\
\footnotetext{The word \textit{kag} is code mixing from Ilonggo (\isi{Hiligaynon}). The normal Kagayanen conjunction is \textit{daw}.}
\glt `Our place is very peaceful and has the \textbf{length} of fourteen kilometers.’ [EFWL-T-07 11.1]
\z
\ea
\label{bkm:Ref116398718}
\textbf{Katamarén}  din  paryo  ta  iya  na  amay. \\\smallskip
 \gll \textbf{ka-tamad-én}  din  paryo  ta  iya  na  amay. \\
\textsc{nr}-lazy-\textsc{nr}  3\textsc{s.gen}  same  \textsc{nabs}  3\textsc{s.gen}  \textsc{lk}  father \\
\glt ‘His/her \textbf{laziness} is like his/her father.’
\z

\subsection{{}-\textit{an}, and \textit{ka-...-an}: location}
\label{sec:ka-an}

The suffix -\textit{an} may occur on some roots to form a noun that refers to the place where something exists or where some action occurs. The combination \textit{ka}{}-...-\textit{an} refers to groups of many things, multiple locations, or extreme/absolute locations. Unlike the nominalizing suffixes -\textit{én/-ón} in Kagayanen and -\textit{án} in other Philippine languages, \textit{{}-an} used in a nominalization in Kagayanen is never stressed or lengthened:

\ea
\label{ex:grove}
\begin{tabbing}
\hspace{4.5cm} \= \kill
\textbf{Root} \> \textbf{Derived form} \\
nióg or nyog ‘coconut’ \>  nyúgan ‘coconut grove’ \\
\>  kanyúgan ‘coconut groves’ \\
káoy ‘wood, tree, stick’ \> kaúyan ‘place where trees grow’ \\
\>   kàkaúyan ‘place of many trees / forest’ \\
gúso ‘agar-seaweed’ \>  gusúan ‘place to grow agar-seaweed’ \\
\>   *kagusuan \\
\end{tabbing}
\begin{tabbing}
\hspace{4.5cm} \= \kill
báļay ‘house’ \> baļáyan ‘place of a house’ \\
\>  kàbaļáyan ‘place of houses/residential area’ \\
tapós ‘finish’ \> tapúsan ‘the end’ \\
\>  kàtapúsan ‘the absolute end’ \\
tumán ‘fulfill/obey’ \> tumánan ‘end point or boundary’ \\
\>  kàtumánan ‘absolute end point or boundary’ \\
umpísa ‘start’ \> umpisáan beginning place’ \\
\>   kaùmpisáan ‘absolute beginning place’ \\
taném ‘plant’ \> tanéman ‘garden/place for planting’ \\
\>  *kataneman \\
báto ‘rock/stone’ \> batúan ‘place with many rocks, little soil’ \\
\>   *kabatuan \\
dúngka ‘to dock a boat’ \> dungkáan ‘place to dock a boat’ \\
\>   *kadungkaan \\
pundó ‘anchor’ \> pundúan ‘place of anchor’ \\
\>   *kapunduan \\
paráda ‘to park a vehicle’ \>  pàradáan ‘place for parking / parking lot’ \\
\>   *kaparadaan \\
sagbét ‘weeds/trash’ \> sagbétan ‘place with lots of weeds/trash’ \\
\>   *kasagbetan \\
úna ‘first’ \> unáan ‘place on ahead’ \\
\>   kàunáan ‘absolutely first place’ \\
kumbíra ‘feast’ \> kùmbiráan ‘place of a feast’ \\
\>   *kakumbiraan \\
tindá ‘sell’ \>  tindáan ‘store/place for selling’ \\
\>   *katindaan \\
dawís ‘point of an island’ \>  dawísan ‘place where an island has a point’ \\
\>   *kadawisan \\
lébbeng ‘bury’ \> lebbéngan ‘grave/burial site’ \\
\>   *kalebbengan \\
príso ‘prisoner’ \> prisúan ‘prison’ \\
\>   *kaprisuan \\
iskwíla ‘go to school' \> ìskwiláan ‘school building’ \\
\>   *kaiskwilaan \\
símba ‘to worship’ \> simbáan ‘church building’ \\
\>   *kasimbaan \\
sakáy ‘ride’ \>  sakáyan ‘vehicle’ \\
\>   *kasakayan \\
\end{tabbing}
\begin{tabbing}
\hspace{4.5cm} \= \kill
lánding ‘to land’ \> landíngan ‘airstrip’ \\
\>   *kalandɪngan \\
tapá ‘smoke cure on a fire’ \> tapáan ‘place for smoke curing on a fire’ \\
\>   *katapaan \\
énnas ‘low tide’ \> ennásan ‘tidal flats’ \\
\>   *kaennasan \\
búkid ‘mountain’ \> bukíran ‘mountain range’ \\
\>   kàbukíran ‘mountain ranges’ \\
úgsak ‘to put inside’ \>  ugsákan ‘container’ \\
\>   *kaugsakan \\
sangát ‘to put away/  \> sangátan ‘thing/rack for putting s.t. \\
\hspace{2cm}hook onto' \> away’ \\
\>   *kasangatan \\
púngko ‘to sit’ \> pungkuan ‘chair’ \\
\>   *kapungkuan \\
saļá ‘to strain wet things’ \>  saļáan ‘strainer for wet things’ \\
\>   *kasaļaan \\
ayág ‘to sift dry things’ \>  ayágan / yágan ‘sifter for dry things’ \\
\>   *kaayagan \\
ámpang ‘to play’ \> ampángan ‘thing to play with’ \\
\>   kàampángan ‘group of playmates’ \\
basák ‘soil/ground’ \> basákan ‘fields/lands owned by s.o.’ \\
\>   *kabasakan \\
bákkaw ‘mangrove tree’ \> bákkawan ‘place of  many mangrove trees’ \\
\>   *kabakkawan \\
tengég ‘bark of mangrove’ \> ténggan ‘place of mangroves’ \\
\>   katénggan ‘mangrove swamp’ \\
lamísa ‘one specific table’ \> làmisáan ‘tables in general or many tables’ \\
\>    *kalamisaan \\
gastós ‘expenses’ \> gastúsan ‘many expenses to pay’ \\
\>   *kagastusan \\
utód ‘sibling’ \> *uturan \\
\>   kàutúran ‘many relatives’ \\
báta ‘child’ \> batáan ‘many children usually \\
\> having same parents’ \\
\>  kàbatáan ‘many children usually \\
\>having different parents’
\end{tabbing}
\z

\largerpage
The following examples illustrate some of these constructions in context. Example \REF{bkm:Ref419298360} illustrates the form \textit{tumanan} used as a noun:

\ea
\label{bkm:Ref419298360}
Piro  pag-abot  din  ta  \textbf{tumanan}  ta  lumba  danen nakita  din  na  naan  en  dya  kauna umang  an  na  gatagad  kanen. \\\smallskip
 \gll Piro  pag--abot  din  ta  \textbf{tuman-an}  ta  lumba  danen na-kita  din  na  naan  en  dya  ka-una umang  an  na  ga-tagad  kanen.\footnotemark{} \\
but  \textsc{nr.act}-arrive  3\textsc{s.gen}  \textsc{nabs}  extent\textsc{-nr}  \textsc{nabs}  race  3\textsc{p.gen}
\textsc{a.hap.r}-see  3\textsc{s.erg}  \textsc{lk}  \textsc{spat.def}  \textsc{cm}  \textsc{d4loc}  \textsc{i.exm}-first
hermit.crab  \textsc{def.m}  \textsc{lk}  \textsc{i.r-}wait  3s \\
\footnotetext{In this example the oblique marker \textit{ki} has dropped out before \textit{kanen}. This is common in spoken Kagayanen and it can happen, though less commonly, in written discourse, as in this example.}
\glt `But, when he arrived at the \textbf{finish} \textbf{line} of their race, he saw that the hermit crab was there first waiting for him.’ [DBWN-T-26 9.2]
\z

Example \REF{bkm:Ref106949019} illustrates the form \textit{kaguļangan} ‘mature (land)’, based on a prop\-er\-ty-concept root\is{property-concept roots (adjectives)}\is{adjectives}, and \textit{kakauyan} ‘forest’ based on a nominal root both functioning as nouns:

\ea
\label{bkm:Ref106949019}
Amo  i  isya  na  ayep  naan  gaistar  ta  \textbf{kaguļangan} o  \textbf{kakauyan}. \\\smallskip
 \gll Amo  i  isya  na  ayep  naan  ga-istar  ta  \textbf{ka-guļang-an} o  \textbf{ka-kauy-an}. \\
monkey  \textsc{def.n}  one  \textsc{lk}  animal  \textsc{spat.def}  \textsc{i.r}-live  \textsc{nabs}  \textsc{nr}-mature-\textsc{nr} or  \textsc{nr}-tree-\textsc{nr} \\
\glt \textsc{‘}Monkeys are one (kind of) animal living in \textbf{oldest} \textbf{(virgin} \textbf{forests)} or \textbf{places} \textbf{with} \textbf{many} \textbf{trees}.’ [NEWE-T-012.1]
\z

Example \REF{bkm:Ref106949274} illustrates the form \textit{batuan} ‘rocky place’, functioning as a location:

\ea
\label{bkm:Ref106949274}
Niog  nai  derse  nang  pļapa  din  tak  naan gatubo  ta  \textbf{batuan}. \\\smallskip
 \gll Niog  nai  derse  nang  pļapa  din  tak  naan ga-tubo  ta  \textbf{bato-an}. \\
coconut  \textsc{d}1\textsc{abs}  small.\textsc{pl}  only  frond  3\textsc{s.gen}  because  \textsc{spat.def} \textsc{i.r}-grow  \textsc{nabs}  rock-\textsc{nr} \\
\glt ‘This coconut tree, its fronds are small only because it grows in a \textbf{rocky} \textbf{place}.’ [JCWN-L-31 13.6]
\z

Example \REF{bkm:Ref106949527} illustrates the verbal root \textit{ugsak} ‘put inside’ nominalized with -\textit{an}, to refer to a location.

\ea
\label{bkm:Ref106949527}
Gakamang  a  ta  kaļaw  daw  \textbf{ugsakan}  ta ummay  na  ega. \\\smallskip
 \gll Ga-kamang  a  ta  kaļaw  daw  \textbf{ugsak-an}  ta ummay  na  ega. \\
\textsc{i.r}-get  1\textsc{s.abs}  \textsc{nabs}  winnowing.basket  and  put.inside-\textsc{nr}  \textsc{nabs} unmilled.rice \textsc{lk} dry \\
\glt `I got a winnowing basket and a \textbf{container} of unmilled rice that was dry.’ [JCWE-T-13 2.3]
\z

Finally, example \REF{bkm:Ref106950024} illustrates the noun \textit{bata} ‘child’ with -\textit{an} meaning ‘group of children usually having the same parents’, while \REF{bkm:Ref106950027} illustrates the same root with \textit{ka-…-an} meaning ‘group of many children’, or ‘youth’, usually having different parents:
\ea
\label{bkm:Ref106950024}
May  bata  daen  appat  buok,  darwa  bai,  darwa  man  mama.  Ta,  tak ubos  en  may  sawa  \textbf{bataan}  daen,  darwa  nang  en  bilin  an  baļay  en. \\\smallskip
 \gll May  bata  daen  appat  buok,  darwa  bai,  darwa  man  mama.  Ta,  tak ubos  en  may  sawa  \textbf{bata-an}  daen,  darwa  nang  en  bilin  an  baļay  en. \\
\textsc{ext.in}  child  3\textsc{p.abs}  four  piece  two  female  two   too  male  so  because all  \textsc{cm}  \textsc{ext.in}  spouse  child-\textsc{nr}  3\textsc{p.gen}  two  only  \textsc{cm}  left  \textsc{spat.def}  house  \textsc{cm} \\
\glt `They have four children, two female, two male too. So, because all their \textbf{children} have spouses, just (they) two only are left at the house.’ [CBON-C-02 1.4]
\z
\ea
\label{bkm:Ref106950027}
Anduni  mga  bag-o  tubo  na  mga  \textbf{kabataan}  uļa  gataod  ta iran  na  isigkaittaw. \\\smallskip
 \gll Anduni  mga  bag-o  tubo  na  mga  \textbf{ka-bata-an}  uļa  ga-taod  ta iran  na  isigka-ittaw.\footnotemark{} \\
now/today  \textsc{pl}  new  grow  \textsc{lk}  \textsc{pl} \textsc{nr}-child-\textsc{nr} \textsc{neg.r}  \textsc{i.r}-respect  \textsc{nabs} 3\textsc{p.gen}  \textsc{lk}  fellow-people \\
\footnotetext{The prefix \textit{isiga}{}- is a borrowing from \isi{Hiligaynon}.}
\glt `Now/today, the new generation of \textbf{children} do not respect their fellow people.’ [BCWL-T-12 2.3]
\z

There are other nouns that seem to have the \textit{ka}{}- nominalizer and either -\textit{en} or -\textit{an} suffix but are not easily divided into morphemes. These may have been borrowed from other languages (notably Hiligaynon) or are simply lexicalized. A selection of such words is given in \REF{bkm:Ref51161274}. An example of an irregular formation is given in \ref{ex:kindness}:

\ea
\label{bkm:Ref51161274}
kanļáman ‘knowledge’ \\
kàimtángan ‘situation/condition’ \\
kàsugtánan ‘agreement’ \\
kàtebtéban ‘forever/eternity’ \\
kàtengdánan ‘authority/high position/power’ \\
kaùgalíngen ‘oneself’ \\
kàmatáyen ‘death’ \\
kanáyen ‘aunts and uncles’ \\
kàsanágen ‘dawn/daybreak’ \\   
kabèbet-én ‘the kindness/will/behavior of a person’\footnotemark
\z
\footnotetext{This word seems to be based on the root \textit{beet} `good/kind/considerate'. However, it involves morphophonological changes that are not regular in Kagayanen (irregular reduplication and vowel dropping). Furthermore, the meaning is not compositional given the meaning of the Kagayanen root. These facts are evidence that it is a borrowing from some other language.}

\ea
\label{ex:kindness}
Yon  baked  na  utang  na  \textbf{kabèbetén}  ta  taga-Magsaysay. \\\smallskip
 \gll Yon  baked  na  utang  na  \textbf{ka-beet-én}  ta  taga--Magsaysay. \\
\textsc{d3abs}  big  \textsc{lk}  debt  \textsc{lk}  \textsc{nr-}kind-\textsc{nr}  \textsc{nabs}  \textsc{orig}-Magsaysay \\
\glt ‘That (is the reason for having) a big debt for the \textbf{kindness} of the ones from Magsaysay.’ (After the speaker's boat capsized some children from Barangay Magsaysay helped him carry his boat back to where he lives.) [EFWN-T-10 6.2] 
\z

The affix combination \textit{ka}{}-...-\textit{an} when occurring on property concept roots derives an abstract noun with superlative meaning. For example, \textit{datas} ‘high/tall’ \rightarrow {} \textit{kadatasan} ‘the absolute highest one’; \textit{delem} ‘dark’ \rightarrow {} \textit{kadeleman} ‘the absolute greatest darkness’.

The suffix -\textit{an} component of this nominalizing process functions as an applicative suffix and \textit{ka}{}- functions as a happenstantial external motivation prefix in verbal morphology (see \chapref{chap:verbstructure} \sectref{sec:modality} and \sectref{sec:externalmotivation}). Though these forms are obviously related to the nominalizers described above, we choose to gloss them distinctly when they function as nominalizers, as opposed to inflectional verb affixes. Situations are rare in which there is uncertainty or ambiguity as to how these affixes are functioning.


\subsection{\textit{-én/-ón}: future patient noun}
\label{sec:en}

Some verb and noun roots can be nominalized with the suffix -\textit{én}, which then attracts the primary stress of the word. This form is normally pronounced -\textit{ón} when the final vowel of the root is /u/. The result is a noun that is understood to refer to something that is “destined” to undergo a future change of state. For example, the verb \textit{búnak}, ‘to launder’, is normally stressed on the first syllable. When it occurs with the future patient nominalizer, the primary stress shifts to the final syllable \textit{bùnakén} ‘thing(s) to be laundered.’ Though the segments of this suffix are identical to the main allomorph of the transitive, irrealis inflection (see \chapref{chap:verbstructure}, \sectref{sec:transitiveirrealis}), this is clearly a different suffix for two reasons. First, as just mentioned, the future patient nominalizer attracts word stress to itself. In contrast, the transitive irrealis -\textit{en} remains unstressed, drawing primary word stress to the penultimate syllable, as do all other verbal suffixes. Second, some verbs, including \textit{bunak}, never take -\textit{en} as the transitive irrealis form, though they may take the future patient nominalizer.

In the examples in \REF{ex:paidfor}, secondary and primary stresses are indicated, although neither is represented in the standard orthography:

\ea
\label{ex:paidfor}
\begin{tabbing}
\hspace{4cm} \= \kill
\textbf{Root} \>  \textbf{Nominalized form} \\
bayád ‘to pay’ \> bàyarén ‘thing(s) to be paid for’ \\
búnak ‘to launder’ \> bùnakén ‘clothes to be laundered’ \\
láswa ‘to cook fish stew’ \> làswaén ‘vegetable(s) to be cooked for fish stew’ \\
lamón ‘to weed’ \>  làmunón ‘weeds and grass to be weeded’ \\
lúto ‘to cook’ \> lùtuón ‘thing(s) to be cooked’ \\
pálit ‘to buy’ \> pàlitén ‘thing(s) to be bought’ \\
ínay ‘mother’ \> ìnayén ‘female animal to be bred’
\end{tabbing}
\z

Future patient nouns may also be derived from nominal roots (examples in \ref{bkm:Ref117346657}).

\ea
\label{bkm:Ref117346657}
\begin{tabbing}
\hspace{4cm} \= \kill
gúlay ‘vegetables’ \> gùlayén ‘vegetable(s) to be cooked’ \\
lub{}-óng ‘cooked cassava’ \> lùb{}-ungón ‘cassava for cooking’ \\
kasáļ ‘wedding’ \> kàsaļén ‘bride’ / ‘bridegroom’ / \\
\> ‘bride and groom’ (someone to be married)
\end{tabbing}
\z

The suffix -\textit{én}/-\textit{ón} can also derive adjectives as described in \chapref{chap:modification}, \sectref{bkm:Ref117512242}. Examples \REF{bkm:Ref116398789} and \REF{bkm:Ref116398793} illustrate some future patient nominalizations from the corpus:

\ea
\label{bkm:Ref116398789}
Gatanem  kay  ta  mga  saging  daw  mga  \textbf{laswaén}. \\\smallskip
 \gll Ga-tanem  kay  ta  mga  saging  daw  mga  \textbf{laswa-én}. \\
\textsc{i.r}-plant  1\textsc{p.excl.abs}  \textsc{nabs}  \textsc{pl}  banana  and  \textsc{pl}  cook.fish.stew-\textsc{nr} \\
\glt ‘We planted bananas and \textbf{fish-stew-vegetables}.’ [JCWN-T-24 2.6)
\z
\ea
\label{bkm:Ref116398793}
\textbf{Gulayén}  ta  Cagayancillo  klasi-klasi  na  mga  prutas  tampayas puso  ta  saging,  sitaw,  bļatong,  taļong,  kļubay,  klabasa,  nangka  daw  duma  pa  na  \textbf{gulayén}. \\\smallskip
 \gll \textbf{Gulay-én}  ta  Cagayancillo  klasi--klasi  na  mga  prutas  tampayas puso  ta  saging,  sitaw,  bļatong,  taļong,  kļubay,  klabasa\footnotemark{},  nangka  daw  duma  pa  na  \textbf{gulay-én}. \\
vegetables-\textsc{nr}  \textsc{nabs}  Cagayancillo  \textsc{red}-kind  \textsc{lk}  \textsc{pl}  fruit  papaya heart  \textsc{nabs}  banana  yardlong.bean  string.bean  eggplant  bottle.gourd summer.squash  jackfruit  and  other  \textsc{inc}  \textsc{lk}  vegetable-\textsc{nr} \\
\footnotetext{The full form of this word is \textit{kalabasa} (from Spanish). However, sometimes unstressed vowels are dropped, especially in longer words. In this case, the writer intentionally wrote \textit{klabasa}, so we have left it in this form.}
\glt `\textbf{Cooking} \textbf{vegetables} on Cagayancillo, (are) kinds of fruits, papaya, flower (lit. heart) of banana, yardlong beans, string beans, eggplant, bottle gourd, summer squash, jackfruit, and other \textbf{cooking} \textbf{vegetables}.’ [CNWE-L-01 2.3]
\z

\subsection{\textit{manug-/manig}-: occupations}
\label{sec:manug}

When the prefix \textit{manug}{}- or \textit{manig-} occurs on a verbal, nominal or property concept root, the meaning becomes the person whose occupation is the activity expressed by the verbal root, who uses the thing referred to with the nominal root in his/her occupation, or who causes the property referred to with the property concept root to come about. For example:

\ea
\begin{tabbing}
\hspace{3.5cm} \= \kill
\textbf{Root} \> \textbf{Derived form} \\
luto ‘to cook’ (v.) \> manugluto/manigluto ‘one who cooks \\
 \> as an occupation’ \\
buļong ‘medicine’ (n.) \> manugbuļong/manigbuļong ‘healer’ \\
limpyo ‘clean’ (adj.) \> manuglimpyo/maniglimpyo ‘cleaner / \\
\> one whose occupation is to clean’.
\end{tabbing}
\z

The variation between \textit{manug}{}- and \textit{manig}{}- depends on the individual. It seems \textit{manig}{}- is more Kagayanen and \textit{manug}{}- is a borrowing from \isi{Hiligaynon}. No distinction in meaning has been found. Neither \textit{manug}{}- nor \textit{manig}{}- are comparable to inflectional verbal affixes or affix combinations (see \chapref{chap:verbstructure}).

\subsection{\textit{ma- + ng-} + first syllable reduplication: occupations}
\label{sec:ma-ng}

The prefix combination \textit{mang}{}- plus \textit{first syllable reduplication}\is{first syllable reduplication}\is{reduplication!first syllable} is similar to the meaning of \textit{manug-/manig}, though \textit{manug}{}-/\textit{manig}{}- is more productive, occurring more often and with a wider range of roots. The prefixal component of this nominalizing process, \textit{mang}{}-, is identical in form and morphophonological behavior to the happenstantial irrealis \textit{ma}{}- plus the pluraction prefix \textit{ng}{}- on verbs (see \chapref{chap:verbstructure}, \sectref{sec:modality}). Semantically this parallelism makes sense because the happenstantial irrealis (‘able to do’/ ‘might do’) combined with pluraction (‘do many times’) is a reasonable way to refer to an activity someone habitually engages in as a occupation. There is no corresponding realis version (*\textit{nang-}) of this process.

What we are calling ‘first syllable reduplication’ \is{first syllable reduplication}here is distinct from \isi{root reduplication} described earlier, in that it never involves complete reduplication of the root and it never includes the last consonant of the root. Furthermore, in a derivational model it must occur \textit{after} prefixation of \textit{ng-}, since the assimilated nasal reduplicates, as though it were the first consonant of the root. Finally, the meaning of this process does not include the ‘sort of’, ‘pretend’ or ‘randomly’ meanings associated with root reduplication. There are no other meanings that we are aware of for this type of \isi{reduplication}.

\is{first syllable reduplication|(}Note that the \textit{ng-} in \textit{ma+ng}- is different from the \textit{ng} in \textit{pang}{}- (below) for two reasons: 1. a root-initial glottal stop always elides with \textit{ma+ng-} but not with \textit{pang}{}-. 2. \textit{ng}- in \textit{ma+ng-} is “replacive” – it replaces, and assimilates to, the first consonant of the root. The \textit{ng} in \textit{pang-} assimilates (sometimes) but does not replace the root-initial consonant.

\ea
\label{ex:net}
\begin{tabbing}
\hspace{3cm} \= \hspace{4cm} \= \kill
Root \> ma- + ng- form  \>  Reduplication \\
\textbf{p-initial}: \\
\textit{pukot}  \> \textit{mamukot} \> \textit{mamumukot} \\
‘drift net’ \>  ‘to fish w/ drift net’ \> ‘drift net fisherman’ \\
\textit{pana} \> \textit{mamana}   \>  \textit{mamamana} \\
‘spear’ \> ‘to spearfish’ \> ‘spear fisherman’ \\
\textbf{s/d/t-initial:} \\
\textit{said}  \> \textit{manaid} \> \textit{mananaid} \\
‘drag net’ \> ‘to fish w/ \textit{said}’ \> ‘\textit{said} fisherman’ \\
\textit{saļap}\footnotemark{} \> \textit{manaļap}  \> \textit{mananaļap} \\
`drag net’ \> ‘to fish w/ \textit{saļap}’ \> ‘\textit{saļap} fisherman’ \\
\textit{sangget} \> \textit{manangget} \> \textit{mananangget} \\
‘sickle for \>  ‘to collect coconut sap' \>  ‘collector of coconut sap' \\
collecting coconut sap \\
\textit{dagat} \> \textit{managat} \> \textit{mananagat} \\
‘sea’ \>  ‘to fish (generic)’ \>  ‘fisherman’ (gen.) \\
\textit{tai} \> \textit{manai} \> \textit{mananai} \\
‘to sew’ \>  ‘to sew a lot’ \>  ‘taylor/seamstress’ \\
\textit{tudlo} \> \textit{manudlo} \> \textit{manunudlo} \\
‘to teach’ \>  ‘to teach’ \>  ‘teacher’ \\
\textit{takaw} \> \textit{manakaw} \> \textit{mananakaw} \\
‘to steal’ \>  ‘to steal’ \> ‘one who steals’ \\
\textbf{k-initial:} \\
\textit{kaoy} \> \textit{mangaoy} \> \textit{mangangaoy} \\
‘tree/wood’ \>  ‘to gather wood’ \> ‘one who gathers wood’ \\
\textit{kawil} \> \textit{mangawil} \> \textit{mangangawil} \\
‘fish hook’ \>  ‘to fish w/ hook and line’ \>  ‘hook and line fisherman’ \\
\textbf{l-initial:} \\
\textit{laya} \> \textit{manlaya} \> \textit{manlalaya} \\
‘cast net’ \>  ‘to fish w/ cast net’ \>    ‘cast net fisherman’ \\
\textbf{Glottal-stop initial} (in these cases, the initial glottal stop drops out): \\
\textit{Ɂayam} \> \textit{mangayam} \>\textit{ mangangayam} \\
‘dog’ \>  ‘to hunt’  \>   ‘hunter’ \\
\textit{Ɂuma} \> \textit{mangúma} \>   \textit{mangungúma} \\
‘field’ \>  ‘to farm’ \>  ‘farmer’
\end{tabbing}
\footnotetext{\textit{Said} and \textit{saļap} are types of drag nets that differ in size and shape. In this grammar, both are glossed as `drag net'.}
\is{first syllable reduplication|)}
\z

\subsection{\textit{<in>}: resultative}
\label{sec:in}
The nominalizing infix \textit{<in>} occurs after the first consonant of the root.\footnote{As discussed in Chapter 2, in Kagayanen all words begin with consonants. Words that begin with vowel letters in the orthography actually start with an unwritten glottal stop, which is a full consonant in Kagayanen. So, for example, the verb \textit{ambaļ} has an initial unwritten glottal stop. The infix \textit{<in>} occurs after the initial glottal stop to form the word spelled \textit{inambaļ}.} The resulting noun usually refers to something that is affected by the action or process described by the root (examples \ref{bkm:Ref419299508} and \ref{bkm:Ref419299510}). In this sense, \textit{<in>} is analogous to a past-participle marker. Unlike \isi{Tagalog} and many other Philippine languages, \textit{<in>} alone has no function as a Transitivity/Modality marker on inflected verbs. Rather, in Kagayanen the corresponding \isi{inflectional} form is the transitive realis \textit{pa}{}- (see \chapref{chap:verbstructure}, \sectref{sec:transitiverealis}):

\ea
\label{bkm:Ref419299508}
May  baļon  a  na  \textbf{linub-ong}. \\\smallskip
 \gll May  baļon  a  na  \textbf{l<in>ub-ong}. \\
\textsc{ext.in}  provisions  1\textsc{s.abs}  \textsc{lk}  <\textsc{nr.res}>cooked.cassava \\
\glt ‘I have some \textbf{steamed} \textbf{cassava} provisions.’ [EFWN-T-10 3.6]
\z
\ea
\label{bkm:Ref419299510}
Daw  luto  en  iran  na  \textbf{sinin-ad} magtumpok  danen  daw magkaan. \\\smallskip
 \gll Daw  luto  en  iran  na  \textbf{s<in>in-ad} mag-tumpok  danen  daw mag-kaan. \\
if/when  cook  \textsc{cm}  3\textsc{p.gen}  \textsc{lk}  <\textsc{nr.res}>cook.grain
\textsc{i.ir}-gather  3\textsc{s.abs}  and \textsc{i.ir-}eat \\
\glt ‘When \textbf{what} \textbf{was} \textbf{cooked} is already cooked (completely) they gather together and eat.’ [VAOE-J-07 3.2]
\z

Like all Kagayanen nouns, nominalizations formed with the infix \textit{<in>} can function as modifiers (as in \ref{bkm:Ref419299508}). For example, \textit{baļad} ‘to sun dry’ with \textit{<in>} becomes \textit{binaļad} ‘sun dried thing’. This form may occur in the position of a modifier in Referring Phrases, e.g., \textit{binaļad na mais} ‘sun dried corn’. Examples \REF{bkm:Ref419378020} and \REF{bkm:Ref51233639} are corpus examples illustrating this usage with the roots \textit{sugba} ‘to roast’ and \textit{begkes} ‘to bundle’ respectively:

\ea
\label{bkm:Ref419378020}
Pag-abot  nay  dya,  listo  kay  eman kaan tak  may  mga  \textbf{sinugba}  na  sidda. \\\smallskip
 \gll Pag--abot  nay  dya,  listo  kay  eman kaan tak  may  mga  \textbf{s<in>inugba}  na  sidda. \\
\textsc{nr.act}-arrive  1\textsc{p.excl.gen}  \textsc{d}4\textsc{loc}  quickly  1\textsc{p.excl.abs}  again.as.before eat because  \textsc{ext.in}  \textsc{pl}  <\textsc{nr.res}>roast  \textsc{lk}  fish \\
\glt `When we arrived there, quickly we ate again because there were some \textbf{roasted} fish.’ [CBWN-C-11 3.2]
\z
\ea
\label{bkm:Ref51233639}
Masimba  anay  danen  an  daw  magprusisyon  na  daļa  a santos  na  Santo  Ninyo  palibot  ta  banwa  na  gadaļa ta \textbf{binegkes}  na  tanitang. \\\smallskip
 \gll Ma-simba  anay  danen  an  daw  mag-prusisyon  na  daļa  a santos  na  Santo  Ninyo  pa-libot  ta  banwa  na  ga-daļa ta \textbf{b<in>egkes}  na  tanitang. \\
\textsc{a.hap.ir}-worship  first  3\textsc{p.abs}  \textsc{def.m}  and  \textsc{i.ir}-procession  \textsc{lk}  carry  \textsc{inj} statue  \textsc{lk}  Holy  Child  \textsc{t.r}-around  \textsc{nabs}  town/country  \textsc{lk}  \textsc{i.r}-carry \textsc{nabs}
 <\textsc{nr.res}>bundle  \textsc{lk} dried.coconut.frond \\
\glt `They will first go to church and go in a procession carrying the statue, the Holy Child, taking (it) around the town carrying \textbf{bundled} dried coconut fronds.’ [VAOE-J-09 2.2]
\z

With verb roots involving the creation of something, \textit{<in>} forms a \textit{factitive} \is{factitive}nominalization (referring to a thing that comes into existence as a result of the action):

\ea
\textbf{Tinukod}  din  na  baļay  nagubba  ta  bagyo. \\\smallskip
 \gll \textbf{T<in>ukod}  din  na  baļay  na-gubba  ta  bagyo. \\
\textsc{<nr.res>}build  3\textsc{s.gen} \textsc{lk}  house  \textsc{a.hap.r}-ruin  \textsc{nabs}  typhoon \\
\glt ‘The house he \textbf{built} (his built house) was ruined by the typhoon.’
\z
\ea
Daw  timpo  kurisma  iran  na  \textbf{kinutkot}  na  waig  dili en  magamit  tak  mangngod  nang  ta  dagat  na  sikad  masin. \\\smallskip
 \gll Daw  timpo  kurisma  iran  na  \textbf{k<in>utkot}  na  waig  dili en  ma-gamit  tak  mangngod  nang  ta  dagat  na  sikad  masin. \\
if/when  season  dry.season  3\textsc{p.gen} \textsc{lk}  <\textsc{nr.res}>dig  \textsc{lk} water  \textsc{neg.ir}  \textsc{cm} \textsc{a.hap.r}-use  because  younger.sibling  only  \textsc{nabs}  sea  \textsc{lk}  very  salty \\
\glt ‘When (it is) dry season, their \textbf{dug} well water cannot be used because (it is) a younger sibling of the sea which is very salty.’ (The word \textit{waig} literally means ‘water,’ but it is also used to mean a ‘well of water.’ The idiom \textit{mangngod ta dagat} means salty water.) [VPWE-T-01 2:4]
\z

Noun roots referring to some kind of container, weight or measure with the infix \textit{<in>} can function as nouns or as noun modifiers. For example, \textit{sako} ‘sack’ with \textit{<in>} becomes \textit{sinako} ‘by the sack’. This form can occur as the head of a Referring Phrase or a modifier within a Referring Phrase, e.g., \textit{sinako na beggas} meaning ‘rice by the sack’, or literally ‘sacked rice’.
\ea
\begin{tabbing}
\hspace{3cm} \= \hspace{4cm} \= \kill
\textbf{Root} \> \textbf{Noun} \> \textbf{Referring Phrase} \\
lata ‘can’ \> linata ‘by cans’ \> linata na sardinas ‘canned sardines’ \\
sako ‘sack’ \> sinako ‘by sacks’ \> sinako na kamúti ‘cassava by sack’ \\
kilo ‘kilogram’ \> kinilo ‘by kilogram’ \> kinilo na sidda ‘fish by kilogram’ \\
gantang ‘ganta’ \> ginantang ‘by ganta’\footnotemark{} \> ginantang na mais ‘corn by ganta’ \\
tabid ‘basket’ \> tinabid ‘by basket’ \> tinabid na kaļaot ‘beans by basket’ \\
geļet ‘cut’ \>  gineļet ‘by slice’ \> gineļet na karni ‘meat by slice’ \\
galon ‘gallon’ \> ginalon ‘by gallon’ \> ginalon na tuba ‘coconut wine by \\
         \> \> gallon’ \\
begkes ‘bundle’ \> binegkes ‘by bundle’ \> binegkes na kaoy ‘wood by bundle’ \\
pungpong ‘cluster’ \> pinongpong ‘by cluster’ \> pinungpong na bunga ‘fruits in \\
     \> \>   clusters’ \\
rulyo ‘roll’ \> rinulyo ‘by roll’ \> rinulyo na kaļat ‘rope by roll’ \\
mitro ‘meter’ \> minitro ‘by meter’ \> minitro na tila ‘cloth by meter’ \\
putos ‘wrapping’ \> pinutos ‘by packet’ \> pinutos na suman ‘sweet sticky \\
 \>  \>      rice by packet’
\end{tabbing}
\footnotetext{A \textit{ganta} is an indigenous Philippine volume measurement equaling about 3 liters.}
\z
\ea
Kita  din  ittaw  na  gapas-an  kon  ta mga  \textbf{sinako}  na  tama. \\\smallskip
 \gll Kita  din  ittaw  na  ga-pas-an  kon  ta mga  \textbf{s<in>ako}  na  tama. \\
see  3\textsc{s.erg}  person  \textsc{lk}  \textsc{i.r-}carry.on.shoulder  \textsc{hsy}  \textsc{nabs} \textsc{pl}  <\textsc{nr.res}>sack  \textsc{lk}  many \\
\glt `Reportedly he saw people who were carrying much of something in \textbf{sacks} on their shoulders.’ [MBON-T-04 2.14]
\z

When \textit{<in>} occurs on nominal roots referring to units of time, the meaning is plural, with (\ref{bkm:Ref419299808} and \ref{bkm:Ref419299809}) or without (\ref{bkm:Ref441664725}) the \isi{plural} marker \textit{mga}:

\ea
\label{bkm:Ref419299808}
Ta  \textbf{mga}  \textbf{binuļan}  na  gubra  mga  mama  ta  simbaan  daw  iskwilaan  malised  na  uļa  angay  pangabui  ta  mga  ittaw. \\\smallskip
 \gll Ta  \textbf{mga}  \textbf{b<in>uļan}  na  ga-ubra  mga  mama  ta  simba-an  daw  iskwila-an  ma-lised  na  uļa  angay  pangabui  ta  mga  ittaw. \\
\textsc{nabs} \textsc{pl}  <\textsc{nr.res}>month  \textsc{lk}  \textsc{i.r-}work  \textsc{pl}  man  \textsc{nabs}  worship-\textsc{nr}  and go.to.school-\textsc{nr}  \textsc{adj-}distress  \textsc{lk}  \textsc{neg.r}  fitting  living  \textsc{nabs}  \textsc{pl}  person \\
\glt ‘During the \textbf{months} when the men were working on the church and school (it was) distressing (and) not fitting for the lives of people.' [JCWN-T-20 13.1]
\z
\ea
\label{bkm:Ref419299809}
Ta  una  sa  na  \textbf{mga}  \textbf{inadlaw}  panliton  i  pasunod ta  maid-id  ta  mga  ittaw  ta  banwa  ta  i. \\\smallskip
 \gll Ta  una  sa  na  \textbf{mga}  \textbf{<in>adlaw}  panliton  i  pa-sunod ta  maid-id  ta  mga  ittaw  ta  banwa  ta  i. \\
\textsc{nabs}  first  \textsc{d}4\textsc{nabs}  \textsc{lk}  \textsc{pl}  \textsc{<}\textsc{nr.res}>day  baby.ritual  \textsc{def.n}  \textsc{t.r}-follow \textsc{nabs}  well  \textsc{nabs}  \textsc{pl}  person  \textsc{nabs}  town/country  1\textsc{p.incl.gen}  \textsc{def.n} \\
\glt `In the first (i.e. early) \textbf{days} the ritual of presenting babies to ancestor spirits the people in our town followed (it) well.’ [JCWE-T-15 2.1]
\z
\ea
\label{bkm:Ref441664725}
Piro  duma  ya  na  galambay  ya  na  \textbf{tinaon}  dili  pa  gid  segeng  na  pag-ataki  ta  apo. \\\smallskip
 \gll Piro  duma  ya  na  ga-lambay  ya  na  \textbf{t<in>aon}  dili  pa  gid  segeng  na  pag--ataki  ta  apo. \\
but  some  \textsc{def.f}  \textsc{lk}  \textsc{i.r}-pass.by  \textsc{def.f}  \textsc{lk}  \textsc{<}\textsc{nr.res}>year  \textsc{neg.ir}  \textsc{inc}  \textsc{int}  extreme  \textsc{lk}  \textsc{nr.act}-attack  \textsc{nabs}  asthma \\
\glt `But other past \textbf{years} it was not yet really an extreme attack of asthma.’ [JCWN-T-22 2.2]
\z

The infix \textit{<in>} may combine with the applicative \textit{{}-an}. Only verb roots which require or allow the applicative \textit{\nobreakdash-an} may be nominalized in this way (see \chapref{chap:stemformingprocesses}, \sectref{sec:applicative-an}). The interpretation of the resulting nominalization depends largely on the nature of the verb root, and is generally consistent with the use of the applicative suffix on Inflected Verbs. For example, as discussed in \chapref{chap:stemformingprocesses}, \sectref{sec:applicative-an}, some transitive verbs require \textit{{}-an} in their normal active-transitive use as main verbs. For this same set of roots, \textit{<in>...-an} forms a nominalization that refers to the Patient of the verb (\ref{bkm:Ref419300230}-\ref{bkm:Ref419300233}). The stress on such nominalizations is regular---the main stress falls on the penultimate syllable and the secondary stress two syllables to the left.

\ea
\label{bkm:Ref419300230}
\textbf{Tinanuran}  ko  ya  na  bata  nuļog ta geddan. \\\smallskip
 \gll \textbf{T<in>anur-an}  ko  ya  na  bata  na-uļog ta geddan. \\
<\textsc{nr.res}>care.for-\textsc{apl}  1\textsc{s.gen}  \textsc{def.f}  \textsc{lk}  child  \textsc{a.hap.r}-fall \textsc{nabs} stairs \\
\glt `\textbf{The} child I \textbf{cared} \textbf{for} fell down the stairs.’ (lit. `my cared for child')
\z
\ea
\label{ex:sidda}
Kaan  ta  ayam  \textbf{inugasan}  ko  na  sidda. \\\smallskip
 \gll \emptyset{}-Kaan  ta  ayam  \textbf{<in>ugas-an}  ko  na  sidda. \\
\textsc{t.r}-eat  \textsc{nabs}  dog  <\textsc{nr.res}>wash-\textsc{apl}  1\textsc{s.gen}  \textsc{lk}  fish \\
\glt ‘A/The dog ate the fish I \textbf{washed}.’ (lit. `my washed fish')
\z
\ea
\label{bkm:Ref419300233}
… sigi  kay  pa  isturya  tenged  ta \textbf{inagian}. \\\smallskip
 \gll … sigi  kay  pa  isturya  tenged  ta \textbf{<in>agi-an}. \\
{} continue  1\textsc{p.excl.abs}  \textsc{inc}  talk  about  \textsc{nabs}  <\textsc{nr.res}>pass-\textsc{apl} \\
\glt ‘...we continued to talk about our \textbf{experiences}.’ [CBWN-C-11 4.41]
\z

Some roots can take either \textit{<in>} alone or \textit{<in>}...-\textit{an} to form a Patient nominalization (\ref{bkm:Ref419300396} and \ref{bkm:Ref447637695}). With \textit{<in>} alone the patient is presented as more precise and individuated and in some situations is more directly acted upon than the corresponding form with \textit{<in>}...-\textit{an}. This meaning difference is comparable to the difference between \textit{pa}- and \textit{pa-…-an} for this same class of roots used as inflected verbs (see \chapref{chap:verbstructure}, \sectref{bkm:Ref448124627}).

\ea
\label{bkm:Ref419300396}
\textbf{Pinetpet}  din  na  kaoy  kamang  ta  duma  na  ittaw. \\\smallskip
 \gll \textbf{P<in>etpet}  din  na  kaoy  kamang  ta  duma  na  ittaw. \\
<\textsc{nr.res}>chop  3\textsc{s.gen}  \textsc{lk}  tree/wood  get  \textsc{nabs}  other  \textsc{lk}  person \\
\glt ‘His/her \textbf{chopped} wood another person took (it).’
\z
\ea
\label{bkm:Ref447637695}
\textbf{Pinetpetan}  din  na  kaoy  nabilin  din  naan  uma  nay. \\\smallskip
 \gll \textbf{P<in>etpetan}  din  na  kaoy  nabilin  din  naan\footnotemark{}  uma  nay. \\
<\textsc{nr.res}>chop-\textsc{apl}  3\textsc{s.gen}  \textsc{lk}  tree/wood  \textsc{a.hap.r}-left  3\textsc{s.erg} \textsc{spat.def}  field 1\textsc{p.excl.gen} \\
\footnotetext{In conversation the non-absolutive case particle \textit{ta} often drops out after \textit{naan,} the spatial demonstrative determiner. In this example, the field is clearly an oblique element, since \textit{kaoy} ‘wood’ is the absolutive undergoer of the action. We will not note this dropping of \textit{ta} from now on, unless it bears on the discussion at hand.}
\glt ‘Some of his/her \textbf{chopped} wood s/he happened to leave behind in our field.’
\z

On other verbs, \textit{<in>} plus -\textit{an} forms a Location or Recipient nominalization. Again, this usage is consistent with the function of -\textit{an} as an applicative marker (see \chapref{chap:stemformingprocesses}, \sectref{sec:applicative-an}). Examples \REF{bkm:Ref419300326} and \REF{bkm:Ref419300323} illustrate Location nominalizations (the place where something occurs):
\ea
\label{bkm:Ref419300326}
Kada  kilem  may pagwa  kanen  an  daw indi  na  mga  \textbf{binaylian}. \\\smallskip
 \gll Kada  kilem  may pa-gwa\footnotemark{}  kanen  an  daw indi  na  mga  \textbf{b<in>ayli-an}. \\
each  night  \textsc{ext.in}  \textsc{caus}-out  3\textsc{s.abs}  \textsc{def.m}  if/when where  \textsc{lk}  \textsc{pl} <\textsc{nr.res}>slow.dance-\textsc{apl} \\
\footnotetext{\textit{Pagwa} is a verb stem formed from the root \textit{gwa} meaning ‘out’ plus the causative prefix. This stem means ‘to cause to come out’ or ‘to show something to someone’. Like all verb stems, \textit{pagwa} can function as a noun (see \sectref{sec:zerodeverbalization}). In this instance, it means ‘a show’ or ‘presentation’.}
\glt `Each night he had a presentation wherever there were \textbf{places} \textbf{for} \textbf{dances}.’ [MBON-T-07 7.6] 
\z

\newpage
\ea
\label{bkm:Ref419300323}
\textbf{Inalinan} din na lugar Cagay-an. \\\smallskip
 \gll \textbf{<In>alin-an}\footnotemark{} din na lugar Cagay-an.\footnotemark{} \\
\textsc{<}\textsc{nr.res}>from-\textsc{apl}  3\textsc{s.erg}  \textsc{lk}  place  Cagay-an \\
\footnotetext[13]{The root \textit{alin} functions as an oblique preposition with the meaning ‘from.’ It also functions as a verb, as in example \REF{bkm:Ref419300323}, meaning ‘come from’, or ‘be from.’}
\footnotetext{\textit{Cagay-an} is the official name and spelling (as in this written text) of the main island in the Municipality of Cagayancillo.}
\glt ‘The place s/he \textbf{came} \textbf{from} is Cagay-an.’ [VPWE-T-01.1]
\z

In example \REF{bkm:Ref419380070}, \textit{<in>} with the verb \textit{tudlo} ‘teach’ becomes \textit{tinudlo} referring to what is being taught, which in this example is a dance. In example \REF{bkm:Ref419444251}, \textit{<in>}...-\textit{an} with \textit{tudlo} ‘teach’ becomes \textit{tinudluan} referring to the recipient of the action, that is, the one taught, which in this example are children.

\ea
\label{bkm:Ref419380070}
\textbf{Tinudlo}  din  an  na  sayaw  dili  dayad. \\\smallskip
 \gll \textbf{T<in>udlo}  din  an  na  sayaw  dili\footnotemark{}  dayad. \\
<\textsc{nr.res}>teach  3\textsc{s.gen}  \textsc{def.m}  \textsc{lk}  dance  \textsc{neg.ir}  good \\
\footnotetext{We are calling \textit{dili} the irrealis negative because it is used to negate main verbal clauses that are irrealis in modality.  However, it is also used as a constituent negator, as in this example. So it occurs with descriptive modifiers such as not hot, not cold, not tall, not fat, not short, not skinny, not kind, and so on. It is also used for negating one argument of a clause, as in ‘it was not he who saw you.’ The usages of the two negators \textit{dili} ‘irrealis negative’ and \textit{uļa} ‘realis negative’ are discussed in \chapref{chap:pragmaticallymarkedstructures}, \sectref{sec:negatives}.}
\glt ‘The dance he taught was not good.’ (lit. ‘His/her taught dance ...)
\z
\ea
\label{bkm:Ref419444251}
Yi  na  \textbf{tinudluan}  ko  na  mga  bata kan-o,  gatudlo  man  anduni. \\\smallskip
 \gll Yi  na  \textbf{t<in>udlu-an}  ko  na  mga  bata kan-o,  ga-tudlo  man  anduni. \\
\textsc{d}1\textsc{abs}  \textsc{lk}  <\textsc{nr.res}>teach-\textsc{apl}  1\textsc{s.gen}  \textsc{lk}  \textsc{pl}  child previously  \textsc{i.r-}teach  also now/today \\
\glt `These children I taught previously, are teaching also now.’ (lit. ‘My taught children ...”)
\z

\subsection{\textit{pang}-: instrumental}
\label{sec:pang}

The prefix \textit{pang}{}- with some roots forms a noun describing an instrument used to accomplish the act described by a verbal root, or used for the referent of a more nominal root. This prefix should not be confused with the prefix \textit{pang}- that occurs on cardinal numbers to express the ordinal version, or the transitive realis \textit{pa}{}- plus pluraction \textit{ng}{}- for the following reason: the \textit{ng}{}- of the pluraction and ordinal prefixes assimilates with and replaces the root-initial consonant, while the \textit{ng} of instrumental \textit{pang}{}- sometimes assimilates to the following consonant, but does not replace it (see \chapref{chap:stemformingprocesses}, \sectref{sec:pluraction} on the stem-forming pluraction prefix \textit{ng-} and \chapref{chap:modification}, \sectref{sec:ordinalnumbers} on the ordinal prefix \textit{pang}-). The following are examples of nouns formed with instrumental \textit{pang-}:

\ea
\begin{tabbing}
\hspace{3.1cm} \= \kill
\textbf{Root} \> \textbf{Noun stem }\\
suļat ‘write’ \>  pangsuļat/pansuļat ‘s.t. used in writing (e.g., pen)’ \\
kasaļ ‘wedding’ \> pangkasaļ ‘s.t. used at a wedding (e.g. wedding   clothes)’ \\
ugas ‘wash’ \>  pang-ugas\footnotemark{} ‘something used in washing’ \\
tanuga ‘sleep’ \>  pangtanuga/pantanuga ‘s.t. used in sleep (e.g.   pajamas)’ \\
gastos ‘expenses’ \> panggastos ‘something used in paying expenses’ \\
bisita ‘visitor’ \> pangbisita/pambisita ‘something used for visitors’ \\
trapo ‘wipe off’ \> pangtrapo/pantrapo ‘something used in wiping off’ \\
panaw ‘walk/go’ \> pangpanaw/pampanaw ‘s.t. worn for going out’ \\
baļay ‘house’ \> pangbaļay/pambaļay ‘s.t. worn around the house’ \\
simba ‘worship’ \> pangsimba/*pansimba ‘s.t. used for going to church’ \\
inrol ‘enroll’ \> pang{}-inrol ‘something used for enrollment’ \\
palit ‘to buy' \> pangpalit/pampalit ‘something used in buying things’ \\
gulay ‘vegetables’ \> panggulay ‘s.t. used for cooking vegetables’ \\
iskwila ‘school’ \> pang-iskwila/pangskwila ‘something used for school’ \\
adlaw{}-adlaw ‘daily’ \> pang{}-adlaw{}-adlaw ‘something used daily’
\end{tabbing}
\footnotetext{When \textit{pang}{}- occurs on roots beginning with glottal stop (orthographically vowel-initial), the glottal stop is pronounced only in careful speech. In relaxed speech, it is always elided. We retain the dash in the orthography to reflect the careful speech form.}
\z

The following are some examples of this prefix from the corpus:

\ea
Isya  pa,  dili  a  kaiskwila  daw  uļa  a  \textbf{pangpliti}. \\\smallskip
 \gll Isya  pa,  dili  a  ka-iskwila  daw  uļa  a  \textbf{pang-pliti}. \\
one  \textsc{inc}  \textsc{neg.ir}  1\textsc{s.abs}  \textsc{i.exm}-school  if/when  \textsc{neg.r}  1\textsc{s.abs}  \textsc{inst}-fare \\
\glt \textsc{‘}One more (thing), I can’t go to school if I have nothing \textbf{for} \textbf{fare}.’ [PBWL-T-06 5.3]
\z

\largerpage
\ea
Sikad  sebe  paggira  tak  uyi  gid  gamiten  na  \textbf{pangbaļay}. \\\smallskip
 \gll Sikad  sebe  pag-gira  tak  u-yi  gid  gamit-en  na  \textbf{pang-baļay}. \\
very  sad  \textsc{nr.act}-war  because  \textsc{emph}-\textsc{d}1\textsc{abs}  \textsc{int}  use-\textsc{t.ir}  \textsc{lk}  \textsc{inst}-house \\
\glt ‘War was very sad, because these (things being discussed) were \textbf{for} \textbf{house} use.’ [JCWN-T-24 3:13]
\z
\ea
Ubra  Nanay  daw  mapon  eman  gasin-ad ta  \textbf{pangyapon}… \\\smallskip
 \gll Ubra  Nanay  daw  mapon  eman  ga-sin-ad ta  \textbf{pang-yapon}… \\
work  mother  if/when  afternoon  again.as.before  \textsc{i.r}-cook.rice \textsc{nabs}  \textsc{inst}-supper \\
\glt `The work of Mother when it’s afternoon again is cooking rice which is used \textbf{for} \textbf{supper}.’ [ICOE-C-01 4.3]
\z

Example \REF{bkm:Ref447638230} is a riddle, with two possible answers:

\ea
\label{bkm:Ref447638230}
Daw  bata  pa  \textbf{panggulay}.  Daw  manakem  en  \textbf{pangbaļay}. Patong. /   Tambo. \\\smallskip
 \gll Daw  bata  pa  \textbf{pang-gulay}.  Daw  manakem  en  \textbf{pang-baļay}. Patong. /   Tambo. \\
if/when  young  \textsc{inc}  \textsc{inst}-vegetable  if/when  older  \textsc{cm}  \textsc{inst}-house rattan / bamboo.shoot \\
\glt `When still young (it is) \textbf{used} \textbf{for} \textbf{vegetable}. When older (it is) \textbf{used} \textbf{for} \textbf{the} \textbf{house}. Rattan. / Bamboo shoot.’ [MRWR-T-01 21.1 and 21.2]
\z

\subsection{\textit{pag}-: action nominalization}
\label{sec:pag}

The prefix \textit{pag}{}- forms an action nominalization, or “verbal noun”, from a verbal root. The nominalized verb can then function as a Referring Expression in a clause or the head of various kinds of dependent clauses. Below are examples of nominalized verbs with \textit{pag}{}- in Referring Expressions. The functions of action nominalizations in clause combining are discussed in \chapref{chap:clausecombining}, \sectref{sec:nominalizationsasdependentclauses}.

\ea
Nasadyaan  kay  man  ta  ame  na  \textbf{pag-iskursyon}. \\\smallskip
 \gll Na-sadya-an  kay  man  ta  ame  na  \textbf{pag--iskursyon}. \\
\textsc{a.hap.r}-enjoy-\textsc{apl}  1\textsc{p.excl.abs}  too  \textsc{nabs}  1\textsc{p.excl.gen}  \textsc{lk}  \textsc{nr.act}-outting \\
\glt ‘We enjoyed too our \textbf{outing}…’ [VAWN-T-15 6.17]
\z
\ea
Paistudyuan  danen  \textbf{pag-ambaļ}.  Paistudyuan  danen  \textbf{pagsuļat}, \textbf{pagbasa}  kag  tanan-tanan  en  danen. \\\smallskip
 \gll Pa-istudyu-an  danen  \textbf{pag--ambaļ}.  Pa-istudyu-an  danen  \textbf{pag-suļat}, \textbf{pag-basa}  kag\footnotemark{}  tanan-tanan  en  danen. \\
\textsc{t.r}-study-\textsc{apl}  3\textsc{p.erg}  \textsc{nr.act}-say  \textsc{t.r}-study-\textsc{apl}  3\textsc{p.erg}  \textsc{nr.act}-write \textsc{nr.act}-read  and  \textsc{red}-all  \textsc{cm}  3\textsc{p.erg} \\
\footnotetext{The conjunction \textit{kag} is from \isi{Hiligaynon}. It is a common substitute for the Kagayanen \textit{daw}.}
\glt `They studied \textbf{speaking}. They studied \textbf{writing}, \textbf{reading} and absolutely everything they (studied).’ [JCOE-T-06 7.4-5]
\z  
\ea
Basi  daw  ta  \textbf{pag-intindi}  danen,  iran  na  litiral  na \textbf{pag}-intindi. \\\smallskip
 \gll Basi  daw  ta  \textbf{pag--intindi}  danen,  iran  na  litiral  na \textbf{pag-}-intindi. \\
maybe  if/when  \textsc{nabs}  \textsc{nr.act}-understand  3\textsc{p.gen}  3\textsc{p.gen}  \textsc{lk} literal  \textsc{lk} \textsc{nr.act}-understand \\
\glt `Maybe in their \textbf{understanding}, theirs is a literal \textbf{understanding}.’ [TTOB-L-03 6.28]
\z
\ea
\textbf{Pagsuļat}  i  ta  Kagayanen  paryo  nang  man  ta  \textbf{pagsuļat} ta  Pilipino. \\\smallskip
 \gll \textbf{Pag-suļat}  i  ta  Kagayanen  paryo  nang  man  ta  \textbf{pag-suļat} ta  Pilipino. \\
\textsc{nr.act}-write  \textsc{def.n}  \textsc{nabs}  Kagayanen  same  only  \textsc{emph}  \textsc{nabs}  \textsc{nr.act}-write \textsc{nabs}  Filipino \\
\glt `\textbf{Writing} Kagayanen is just the same as \textbf{writing} Filipino.’ [COE-T-06 6.1]
\z
\ea
Nakita  ko  \textbf{pag-igo}  ya.  Ambaļ  ko,  “Karni  en!” \\\smallskip
 \gll Na-kita  ko  \textbf{pag--igo}  ya.  Ambaļ  ko,  “Karni  en!” \\
\textsc{a.hap.r}-see  1\textsc{s.erg}  \textsc{nr.act}-hit  \textsc{def.f}  say  1\textsc{s.erg}  meat  \textsc{cm} \\
\glt ‘I saw the \textbf{hitting} (the spear hitting a wild pig). I said, “(We have) meat now!”' [RCON-L-01 4.1] \\
\z
\ea
Dayad  \textbf{pagtan-aw}  ko  tak  adlaw  en. \\\smallskip
 \gll Dayad  \textbf{pag-tan-aw}  ko  tak  adlaw  en. \\
good  \textsc{nr.act}-look  1\textsc{s.gen}  because  sun/day  \textsc{cm} \\
\glt ‘My \textbf{view} is good because (it is) daylight now.’ [RCON-L-01 3.7]
\z
\ea
Ta  uyo  na  natabo,  uļa  a naduwari  ta  disisyon  daw  \textbf{pag-andem} ta  Dios. \\\smallskip
 \gll Ta  u-yo  na  na-tabo,  uļa  a na-duwad-i  ta  disisyon  daw  \textbf{pag--andem} ta  Dios. \\
\textsc{nabs}  \textsc{emph}-\textsc{d}4\textsc{abs}  \textsc{lk}  \textsc{a.hap.r}-happen  \textsc{neg.r}  1\textsc{s.abs} \textsc{a.hap.r}-lose/disappear-\textsc{xc.apl} \textsc{nabs} decision  and  \textsc{nr.act}-be.ambitious \textsc{nabs}  God \\
\glt `During that happening, I did not lose my decision and \textbf{ambition} (to serve) God.’ (\textit{Andem} ‘ambition’ includes passion, dedication, faithfulness, and strong desire to serve.) [EFWN-T-11 13.7]
\z

Some action nominalizations also employ the \textit{-én} nominalizer or the -\textit{an} appicative suffixes as in the following examples.
\ea
Daw  kanen  man  gaplano  ta  mga  \textbf{pag-ampangén}  daw  may mga  \textbf{paggwaén}  ta  ame  na  munisipyo. \\\smallskip
 \gll Daw  kanen  man  ga-plano  ta  mga  \textbf{pag--ampang-én}  daw  may mga  \textbf{pag-gwa-én}  ta  ame  na  munisipyo. \\
and  3\textsc{s.abs}  also  \textsc{i.r}-plan  \textsc{nabs}  \textsc{pl}  \textsc{nr.act}-play-\textsc{nr} and \textsc{ext.in} \textsc{pl}  \textsc{nr.act}-out-\textsc{nr}  \textsc{nabs}  1\textsc{p.incl.gen}  \textsc{lk}  municipality \\
\glt ‘And as for him also (he) planned the \textbf{games} and he had \textbf{shows} in our municipality.' [ICWE-T-02 2.9]
\z
\ea
Lugay  en  na  \textbf{pag-isturyaan}  daw  alin  ta  tallog  o  piyak  nauna. \\\smallskip
 \gll Lugay  en  na  \textbf{pag--isturya-an}  daw  alin  ta  tallog  o  piyak  na-una. \\
long.time  \textsc{cm}  \textsc{lk}  \textsc{nr.act}-story-\textsc{apl}  if/when  from  \textsc{nabs}  egg  or  chick  \textsc{a.hap.r}-first \\
\glt ‘It is a long conversation about whether an egg or a chick was first.’ [ESWO-L-02 33.1]
\z

Finally, there are some examples of \textit{pag}{}- that form nouns, but not strictly action nominals. We consider these to be lexicalizations:

\ea
\begin{tabbing}
\hspace{5cm} \= \kill
Lexicalizations formed with \textit{pag}{}-: \\
kaan ‘eat’ \> pagkaan ‘food’ \\
pati ‘to believe/obey’ \> pagpati ‘belief’ \\
pangabui ‘life/living’ \> pagpangabui ‘livelihood’ \\
pari ‘godfather or pal/buddy’ \> pagpari ‘godfathership’ or ‘male friendship’ \\
arey ‘friend'  \>  pag{}-arey ‘friendship’ \\
utod ‘sibling’ \> pag{}-utod ‘siblingship’ \\
beet ‘kind/well-behaved' \> pagbeet ‘will, desire, wish’ \\
inner self’ \\
daļa ‘carry’  \>  pagdļeen/pagdaļaen ‘administration/ \\
  \> government’
\end{tabbing}
\z
\subsection{\textit{pagka}-: abstract qualities}
\label{sec:pagka}

The form \textit{pagka}{}- can be thought of as a complex consisting of the action nominalizer \textit{pag}{}- plus the verbalizer \textit{ka}{}-. However, its effect is not exactly the expected “act of VERBing” based on a nominal root. It forms nouns that refer to the quality, character, or behavior of a person or thing. When added to a property concept root, it normally means the characteristic associated with that property temporarily exhibited by the modified noun.

\ea
\begin{tabbing}
\hspace{4cm} \= \kill
\textit{pagka}{}- on property concept roots: \\
beet ‘kind/well behaved \> pagkabeet ‘being kind/well behaved’ \\
tamad ‘lazy’ \> pagkatamad ‘being lazy/laziness’ \\
sutɪl ‘onery/stubborn’ \> pagkasutil ‘being stubborn/stubborness’ \\
bellay ‘difficult’ \> pagkabellay ‘being difficult/difficulty’ \\
mingaw ‘lonely’ \> pagkamingaw ‘being lonely/loneliness’ \\
bugal ‘proud’ \> pagkabugal ‘being prideful/pride’ \\
dayaw ‘beautiful’\>  pagkadayaw ‘being beautiful/beauty’ \\
minog ‘red’ \>   pagkaminog ‘tint of red’ / ‘reddish' \\
dayad ‘good’ \> pagkadayad ‘being good/goodness’ \\
\> (inanimate things) \\
miad ‘good/kind’ \> pagkamiad ‘being kind/goodness/kindness’ \\
\> (animate beings)
\end{tabbing}
\z

The following are some examples of this basic usage of \textit{pagka}{}- from the text corpus:

\ea
Daw  a  \textbf{pagkabugal}  i,  yi  en  gaparani  en ki  kaon,  yon  a  gaparani  ki  kaon  kagayyaan. \\\smallskip
 \gll Daw  a  \textbf{pag-ka-bugal}  i,  yi  en  ga-parani  en ki  kaon,  yon  a  ga-parani  ki  kaon  ka-gayya-an. \\
if/when  \textsc{inj}  \textsc{nr.act-nr}-pride  \textsc{def.n}  \textsc{d1abs}  \textsc{cm}  \textsc{i.r-}near  \textsc{cm} \textsc{obl.p}  2s  \textsc{d3abs}  \textsc{inj}  \textsc{i.r}-near  \textsc{obl.p}  2s  \textsc{nr}-shame-\textsc{apl} \\
\glt `If \textbf{being} \textbf{prideful} comes close to you, then that which comes to you is absolute shamefulness. (lit. If \textbf{being} \textbf{prideful}, this is what comes near to you, that coming near to you is very shameful.)’ [JCOB-L-02 6.5]
\z
\ea
… daw  sikad  gid  en  \textbf{pagkamaal}  ta  mga  palitén  di. \\\smallskip
 \gll … daw  sikad  gid  en  \textbf{pag-ka-maal}  ta  mga  palit-én  di. \\
{} and  very  \textsc{int}  \textsc{cm}  \textsc{nr.act-nr}-expensive  \textsc{nabs}  \textsc{pl}  buy-\textsc{nr}  \textsc{d}1\textsc{loc} \\
\glt ‘and things to buy here are really very \textbf{expensive’} (lit. And \textbf{being} \textbf{expensive} of the things here to buy is really very much.)’ [PBWL-T-10 7.3]
\z
\ea
Lengngessa  ta  baka  may  \textbf{pagkasapek}  daw  malangsa. \\\smallskip
 \gll Lengngessa  ta  baka  may  \textbf{pag-ka-sapek}  daw  ma-langsa. \\
blood  \textsc{nabs}  cow  \textsc{ext.in}  \textsc{nr.act-nr}-thick  and  \textsc{abs}-fishy.smell \\
\glt ‘Blood of cows has a \textbf{thick} consistency and is fishy smelling (lit. Blood of cows has some \textbf{being} \textbf{thickness} and fishy smell.).' [JCWE-T-14 17.4]
\z

On nominal roots that refer to people by their title or occupation, forms with \textit{pagka}{}- describe abstract qualities associated with the root. Most of the roots in \REF{ex:captain} are borrowings from Spanish or English. The forms listed here are the normal Kagayanen spellings, which in most cases reflect the pronunciation. The word \textit{midwife} ‘midwife’ is an exception in that it is spelled and pronounced in the English way, suggesting that it is probably a more recent borrowing.:

\ea
\label{ex:captain}
\begin{tabbing}
\hspace{4.5cm} \= \kill
\textit{pagka}{}- on occupation nouns: \\
nars ‘nurse’  \>  pagkanars ‘having the position of nurse’ \\
duktor ‘doctor’ \> pagkaduktor ‘… position  of doctor’ \\
kapitan ‘captain’ \> pagkakapitan ‘… position of captain’. \\
maistro ‘male teacher’ \> pagkamaistro ‘… position of male teacher’ \\
maistra ‘female teacher’ \> pagkamaistra ‘… position of female teacher’ \\
mayor ‘mayor’ \> pagkamayor ‘… position of mayor’ \\
pari ‘priest’  \>  pagkapari ‘… position of priest’ \\
pastor ‘pastor’ \> pagkapastor ‘… position of pastor’ \\
dļaga ‘single woman’ \> pagkadļaga ‘… position of single woman’ \\
miron ‘single man’ \> pagkamiron ‘… position of single man’ \\
midwife ‘midwife’ \> pagkamidwife ‘… position of midwife’ \\
ugwangan ‘parents-in-law’ \> pagkaugwangan ‘… pos. of parents-in-law \\
ittaw ‘person’ \> pagkaittaw ‘… position of being \\
 \> a person or personhood’
\end{tabbing}
\z

On roots describing times or activities, \textit{pagka}{}- creates an action nominalization that refers to the time after something else happens. In this usage, \textit{pagka}{}- contrasts with \textit{pag}{}-, which indicates simultaneous or sequential events that are closely related to the previous action (see \chapref{chap:clausecombining}, \sectref{sec:nominalizationsasdependentclauses}). Examples \REF{bkm:Ref116460168} and \REF{bkm:Ref116460171} illustrate some \textit{pagka-} nominalizations in contrast with \textit{pag}{}- nominalizations:

\newpage
\ea
\label{bkm:Ref116460168}
\begin{tabbing}
\hspace{4.1cm} \= \kill
\textit{pagka}{}- vs. \textit{pag}{}- on nouns describing times: \\
sellem ‘morning’ \> pagkasellem ‘after it becomes morning’ \\
\>  pagsellem ‘when it is morning’ \\
kilem ‘night’ \> pagkakilem ‘after it becomes night’ \\
\>   pagkilem ‘when it is night’ \\
Lunis ‘Monday’ \> pagkalunis ‘after it became Monday’ \\
\>       paglunis ‘when it is Monday’ \\
kisyem ‘tomorrow’ \> pagkakisyem ‘after it became tomorrow’ \\
\>  pagkisyem ‘when it is/was tomorrow’ \\
pawa buļan ‘bright moon’ \> pagkapawa ta buļan ‘after the moon \\
\> becomes bright’ \\
\>   pagpawa ta buļan ‘when the moon is bright’ \\
sibo buļan ‘full moon’ \> pagkasibo ta buļan ‘after the moon is full’ \\
\>      pagsibo ta buļan ‘when the moon is full
\end{tabbing}
\z
\ea
\label{bkm:Ref116460171}
\begin{tabbing}
\hspace{4.3cm} \= \kill
\textit{pagka}{}- vs. \textit{pag}{}- on nouns describing activities: \\
panaog ‘to go.down.stairs’ \> pagkapanaong ‘after going down stairs’ \\
\>       pagpanaong ‘when going down stairs.’ \\
kita ‘to see’ \>   pagkakita ‘after seeing’ \\
\>       pagkita ‘when seeing’ \\
patay ‘to die, dead, corpse’ \> pagkapatay ‘after dying’ \\
\>       pagpatay ‘when dying’ \\
mati ‘to hear’ \> pagkamati ‘after hearing’ \\
\>        pagmati ‘when hearing’
\end{tabbing}
\z

The following examples illustrate these usages of \textit{pagka-} in the corpus:

\ea
\label{bkm:Ref447656201}
\textbf{Pagkasellem}  eman,  gapanaw  kay  isab na miling  ta  isya  na  ristaurant. \\\smallskip
 \gll \textbf{Pag-ka-sellem}  eman,  ga-panaw  kay  isab\footnotemark{}  na m-iling  ta  isya  na  ristaurant. \\
\textsc{nr.act-nr}-morning  again.as.before \textsc{i.r}-walk/leave  1\textsc{p.incl.abs}  again  \textsc{lk} \textsc{i.v.ir}-go  \textsc{nabs}  one  \textsc{lk}  restaurant \\
\footnotetext{There are two words glossed as ‘again’ in this example. \textit{Eman} is the marked member of the pair, and refers to the continuation of an action that was interrupted. \textit{Isab} expresses no particular sense of continuation. In the context of example \REF{bkm:Ref447656201}, \textit{eman} would have been semantically appropriate in both positions, however we speculate that having two \textit{eman}s in a row would have sounded odd, so the speaker chose to use \textit{isab} in the second instance, which is also semantically appropriate.}
\glt `\textbf{After it became morning} again as before, we left again going to a restaurant.’ [AGWN-L-01 3.9]
\z
\ea
\textbf{Pagkapanaog}  ta  sakay  nay  na  duma, kami  galarga  eman   na  miling  en   ta  Nusa. \\\smallskip
 \gll \textbf{Pag-ka-panaog}  ta  sakay  nay  na  duma, kami  ga-larga  eman   na  m-iling  en   ta  Nusa. \\
\textsc{nr.act-nr}-go.down.stairs  \textsc{nabs}  ride  1\textsc{p.excl.gen}  \textsc{lk}  companion
1\textsc{p.excl,abs}  \textsc{i.r}-depart  again.as.before  \textsc{lk}  \textsc{i.v.r}-go  \textsc{cm}  \textsc{nabs}  Nusa \\
\glt ‘\textbf{After} \textbf{our} \textbf{fellow} \textbf{passengers} \textbf{went-down} \textbf{(disembarked)}, as-for-us we departed again as before now going to Nusa.’ [SOWN-L-01 1.3]
\z

There is an idiomatic use of \textit{pagka}{}- on adjective roots with the existential \textit{may}, in which the meaning is ‘kind of X’ or ‘sort of X’.  Compare \REF{ex:rough} and \REF{ex:roughness}:
\ea 
    \ea 
      \label{ex:rough}
        Karni  ta  baka  sapļa. \\\smallskip
 \gll Karni  ta  baka  sapļa. \\
          meat  \textsc{nabs}  cow  rough \\
        \glt ‘Cow meat is rough.’
    \ex
        \label{ex:roughness}
        Karni  ta  baka  \textbf{may}    \textbf{pagka sapļa}. \\\smallskip
 \gll Karni  ta  baka  \textbf{may}    \textbf{pag-ka-sapļa} \\
        meat  \textsc{nabs}  cow  \textsc{ext.in}  \textsc{nr.act-nr-}rough \\
        \glt ‘Cow meat is kind of rough.’ (lit.  Cow meat has some roughness.)
    \z
\z

Finally, there are some nominalizations with \textit{pagka}{}- that do not follow any of the above patterns. These we consider to be lexicalizations rather than productive constructions.

\ea
\begin{tabbing}
\hspace{4cm} \= \hspace{3cm} \= \kill
Some idiomatic uses of \textit{pagka}{}-: \\ 
betang/batang ‘to put’ \> betang ‘thing’ \> pagkabetang ‘possessions’ \\
patay ‘to die, to kill, corpse’ \> \>   pagkapatay ‘death/killing’ \\
tao ‘to be born’ \> \>  pagkatao ‘birth’
\end{tabbing}
\z
\subsection{\textit{mag}-: human relationships}
\label{sec:mag}

The affix \textit{mag}{}- with nouns that indicate a human relationship, such as kinship terms and words such as \textit{arey} ‘friend’, result in a noun that refers to particular persons who are in that relationship. Some of these nouns express reciprocal and others non-reciprocal relationships. For example, \textit{mag}{}- with \textit{utod} ‘sibling’ becomes \textit{mag-utod} ‘siblings (to each other)’ (\ref{bkm:Ref420509881}); \textit{mag-} with \textit{arey} ‘friend’ becomes \textit{mag-arey} meaning ‘friends (to each other)’, (\ref{bkm:Ref420509947}). These are reciprocal relationships, since each member of the relationship is equal---sibling-to-sibling, or friend-to-friend. An example of a non-reciprocal relationship is \textit{mag}{}- + \textit{inay} ‘mother’ yielding \textit{mag-inay} ‘mother and child(ren)’:

\ea
\label{bkm:Ref420509881}
Darwa  i  na  \textbf{mag-utod}  gasakay  ta  lunday … \\\smallskip
 \gll Darwa  i  na  \textbf{mag--utod}  ga-sakay  ta  lunday … \\
two  \textsc{def.n}  \textsc{lk}  \textsc{rel}-sibling  \textsc{i.r-}ride  \textsc{nabs}  outrigger.canoe \\
\glt ‘The two \textbf{siblings} rode the outrigger canoe.’ [EMWN-T-06 4.2]
\z
\ea
\label{bkm:Ref420509947}
Isya  na  adlaw,  darwa  i  na  \textbf{mag-arey},  Umang  i daw Pwikan,  namasyar  naan  ta  baybay. \\\smallskip
 \gll Isya  na  adlaw,  darwa  i  na  \textbf{mag--arey},  Umang  i daw Pwikan,  na-ŋ-pasyar  naan  ta  baybay. \\
one  \textsc{lk}  day  two  \textsc{def.n}  \textsc{lk}  \textsc{rel}-friend  hermit.crab  \textsc{def.n} and sea.turtle  \textsc{a.hap.r}-\textsc{pl}-visit  \textsc{spat.def}  \textsc{nabs}  beach \\
\glt `One day, the two \textbf{friends}, Hermit Crab and Sea Turtle, went strolling on the beach.’ [JCON-L-08 3.1]
\z

The following is a list of all the roots we are aware of that occur in this construction, with their meanings with \textit{mag-}. We have divided these roots into those which describe reciprocal relationships and those which describe non-reciprocal relationships:

\ea
\begin{tabbing}
\hspace{4.3cm}    \=  \kill
\textbf{Reciprocal roots} \> \textbf{Forms derived with \textit{mag-} }\\
pamɪlya ‘family’ \> magpamɪlya ‘members of a family’ \\
utod ‘sibling’ \> mag{}-utod ‘siblings/relatives’ \\
sawa ‘spouse’ \> magsawa ‘a married couple’ \\
katagsa ‘cousin’ \> magkatagsa ‘cousins’ \\
parinti ‘relatives’ \> magparinti ‘relatives’ \\
bayaw ‘sibling-in-law’ \> magbayaw ‘siblings-in-law’ \\
bļai ‘fellow parent-in-law’ \> magbļai ‘both sets of parents-in-law’ \\
mari ‘godmother/friend’ \> magmari ‘co-godmothers’ or ‘female pals’ \\
pari ‘godfather/friend’ \> magpari ‘co-godfathers’ or ‘male pals’ \\
barkada ‘pal’ \> magbarkada ‘pals’ \\
klasmit ‘classmate’ \> magklasmit ‘fellow classmates’ \\
kasimanwa ‘town-mate’ \> magkasimanwa ‘fellow town-mates’ \\
upismat ‘officemate' \> mag-upismat ‘fellow office mates’ \\
arey ‘friend’ \> mag{}-arey ‘friends’
\end{tabbing}
\z

\newpage
\ea
\begin{tabbing}
\hspace{4.3cm}    \=  \kill
\textbf{Non-reciprocal roots} \> \textbf{Forms derived with \textit{mag-}} \\
bata ‘child’ \> magbata ‘parent(s) and child(ren)’ \\
inay ‘mother’ \> mag-inay ‘mother and child(ren)’ \\
nanay ‘mom/mother’ \> magnanay ‘mother and child(ren)’ \\
amay ‘father’ \> mag{}-amay ‘father and child(ren)’ \\
tatay ‘dad/father’ \> magtatay ‘father and child(ren)’ \\
anti ‘aunt’ \> mag{}-anti ‘aunt and niece(s)/nephew(s)’ \\
tiya ‘aunt’ \> magtiya ‘aunt and niece(s)/nephew(s)’ \\
ankil ‘uncle’ \> mag{}-ankil ‘uncle and niece(s)/nephew(s)’ \\
tiyo ‘uncle’ \> magtiyo ‘uncle and niece(s)/nephew(s)’ \\
apo ‘ancestor, grandparent’ \>  mag{}-apo ‘grandparent(s) and grandchild(ren)’ \\
lulo ‘grandfather’ \> maglulo ‘grandfather and grandchild(ren)’ \\
lula ‘grandmother’ \> maglula ‘grandmother and grandchild(ren)’ \\
ugwangan ‘parent-in\textit{{}-}law’ \> mag{}-ugwangan ‘parent-in-law \& child-in-law’ \\
magad ‘child-in-law’ \> magmagad ‘parent-in-law \& child-in-law’ \\
ninang ‘godmother’ \> magninang ‘godmother and godchild’ \\
ninong ‘godfather’ \> magninong ‘godfather and godchild’
\end{tabbing}
\z

Roots that would seem to meet the semantic profile of nouns available for this usage of \textit{mag}{}-, but which do not allow it include the following. All of these, it seems, describe non-reciprocal relationships:
\ea
\begin{tabbing}
\hspace{5cm}    \=  \kill
ginikanan ‘parent(s)’ \> *magginikanan \\
gamangken ‘niece/nephew’ \> *maggamangken \\
apo ta buoļ ‘great grandchild’ \> *mag{}-apo ta buoļ \\
manak ‘step child’ \> *magmanak \\
anak ‘godchild’ \> *mag{}-anak
\end{tabbing}
\z

The form \textit{mag-} also functions as an intransitive, irrealis marker on inflected verbs (see \chapref{chap:verbstructure}, \sectref{sec:intransitiveirrealis}). However, we can discern no semantic relationship between the nominalizing and verbal inflectional uses.

\subsection{\textit{king}-: style of X}
\label{sec:king}

The prefix \textit{king-} is added to a nominal root to form a new noun that refers to the ‘style’ or ‘manner’ of the root noun. This is not a common prefix. Example \REF{bkm:Ref116461310} below is the only instance of this prefix in the corpus. The others are all from conversation, or are elicited. Words formed with \textit{king-} are not adjectives by the definition given in \chapref{chap:modification}, \sectref{sec:adjectives}, because they do not occur with degree adverbs such as \textit{sikad} ‘very’:

\ea
\begin{tabbing}
manakem   ‘older person’\hspace{.5cm} \= kingmanakem   ‘the older people’s style/ways’ \\
Kagayanen   ‘Kagayanen’ \> kingkagayanen   ‘the Kagayanen style/ways’ \\
umang   ‘hermit crab’ \>  king{}-umang  ‘the hermit crab’s style/ways’ \\
manggaranen  ‘rich people’ \> kingmanggaranen  ‘the rich people’s style/way’
\end{tabbing}
\z
\ea
\textbf{Kingmanakem}  na  sayaw  nan. \\\smallskip

\gll \textbf{King-manakem}  na  sayaw  nan. \\
style/way-older  \textsc{lk}  dance \textsc{d}3\textsc{abs} \\
\glt ‘That is an older people’s style of dance.' \\
*Sikad kingmanakem na sayaw nan. (‘That is really an older person dance style.’)
\z
\ea
\label{bkm:Ref116461310}
Patugtug  an  ta  banda  mga  tukar  an \textbf{kingmanakem}  en  pugya  a.\\\smallskip
\gll Pa-tugtug  an  ta  banda  mga  tukar  an \textbf{king-manakem}  en  pugya  a. \\
\textsc{t.r}-play.music  \textsc{def.m}  \textsc{nabs}  band  \textsc{pl}  music  \textsc{def.m} style/way-older  \textsc{cm} long.ago  \textsc{inj} \\
\glt `The music the band played was the older people’s style of music of long ago.’ [PBON-T-01 6.17]
\z

\subsection{\textit{tag-}: season for X}
\label{sec:tag}

The prefix \textit{tag}{}- occurs on some roots to refer to a certain season of the year or a season of some activity.

\ea
\begin{tabbing}
\hspace{4.5cm} \= \kill
\textbf{Root} \> \textbf{Derived form} \\
adlaw ‘day/sun’ \>  tag{}-adlaw ‘hot dry season’ \\
uran ‘rain’ \> tag{}-uran ‘rainy season’ \\
ani ‘harvest’ \> tag{}-ani ‘harvest season’ \\
tanem ‘plant’ \> tagtanem ‘planting season’ \\
nubig ‘haul water’ \> tagnubɪg ‘season to haul water’ \\
daon ‘leaf’ \> tagdaon ‘season of (growing) leaves’ \\
lagas ‘fall apart/out/down’ \> taglagas ‘season of falling leaves’ \\
ļettem ‘hungry’ \> tagļettem ‘hunger season’ \\
linaw ‘calm sea’ \> taglinaw ‘calm sea season’ \\
pelles ‘strong wind’ \> tagpelles ‘strong wind season’ \\
ngaoy ‘to gather wood’ \> tagngaoy ‘season to gather wood’ \\
tignaw ‘cold’ \> tagtignaw ‘cold season’
\end{tabbing}
\z

Only one word \textit{adlaw} ‘sun/day’ can occur with either \textit{tag}{}- or \textit{pang}{}- to mean ‘summer.’

\ea
tag{}-adlaw ‘summer’ \\
pang{}-adlaw ‘summer’
\z
\subsection{\textit{taga}-: place of origin}
\label{sec:taga}

The prefix \textit{taga}{}- forms a noun or modifier that refers to a person who is originally from a location specified in the root. Example \REF{bkm:Ref447688060} is a predicate nominal consisting of a predicate meaning “someone originally from Cagayancillo”, followed by the single Argument:

\ea
\label{bkm:Ref447688060}
\textbf{Taga-Cagayancillo}  yaken  i. \\\smallskip
 \gll \textbf{Taga-Cagayancillo}  yaken  i. \\
\textsc{orig}-Cagayancillo  1\textsc{s.abs}  \textsc{def.n} \\
\glt ‘I am \textbf{originally} \textbf{from} \textbf{Cagayancillo}.’ [BMON-C-05 1.2]
\z
Some writers use a hyphen following \textit{taga}-, especially when it occurs before a capitalized place name, such as Cagayancillo (\ref{bkm:Ref447688060}) or Negros \REF{ex:negros}. However, this is not consistent \REF{ex:philippines}.

\ea
\label{ex:negros}
Naan  ta  baryo  Sta. Cruz  may  isya  na  ittaw  na  \textbf{taga-Negros} na  nangasawa  don. \\\smallskip
 \gll Naan  ta  baryo  Sta. Cruz  may  isya  na  ittaw  na  \textbf{taga-Negros} na  na-ngasawa  don. \\
\textsc{spat.def}  \textsc{nabs}  barrio  Sta. Cruz  \textsc{ext.in}  one  \textsc{lk}  person  \textsc{lk}  \textsc{orig}-Negros
\textsc{lk}  \textsc{a.hap.r}-get.married  \textsc{d3loc} \\
\glt `In the community of Sta. Cruz there was a person who \textbf{originated} \textbf{from} \textbf{Negros} who was married there.’ [JCWN-T-20 16.4]
\z

\ea
\label{ex:philippines}
Gailing  di  mga  \textbf{taga-duma}  an  na  banwa  lawas  ta  Pilipinas. \\\smallskip
 \gll Ga-iling  di  mga  \textbf{taga-duma}  an  na  banwa  lawas  ta  Pilipinas. \\
\textsc{i.r}-go  \textsc{d}1\textsc{loc}  \textsc{pl}  \textsc{orig}-other  \textsc{def.m}  \textsc{lk}  town/country  outside  \textsc{nabs}  Philippines \\
\glt ‘\textbf{Ones} \textbf{from} \textbf{other} \textbf{places} outside the Philipines came here.’ [ICWN-T-04 3.2]
\z

The use of the hyphen following \textit{taga}- in written material reflects speakers' intuitions that this form functions as a pro-clitic as well as a nominal stem-forming prefix. For example, it attaches to the first word in an RP:
\ea
    \ea
    \label{ex:otherplaces}
    ... taga-duma na lugar ... \\\smallskip

    \gll ... taga-duma na lugar ... \\
     { } \textsc{orig}-other \textsc{lk} place \\
    \glt `... (people) from other places ...' [JCWN-T-21 13.5] \\
    \ex 
    \label{ex:andnusa}
    Taga-Mampyo  daw  Nusa ... \\
    \gll Taga-Mampyo  daw  Nusa{}... \\
    \textsc{orig}-Mampyo(place) and Nusa(place) \\
    \glt `(People) from Mampyo and Nusa ...' [JCWN-T-21 10.1]
    \z
\z
In example \REF{ex:otherplaces}, \textit{taga}- nominalizes the entire RP \textit{duma na lugar} `other place(s)', and not just the word \textit{duma} `other'. In \REF{ex:andnusa} \textit{taga}- appears only once before the first conjunct of two conjoined nouns, but its meaning applies equally to both nouns. 

\subsection{\textit{isig(ka)-}, \textit{masig(ka)-} and \textit{kasig(ka)-}: fellow person}
\label{sec:isig}

There are three roots we know of that take the prefixes \textit{isig(ka)-} \textit{masig(ka)-} and \textit{kasig(ka)} to form a noun that means ‘fellow N’, where N refers to a group of people. These may be a result of influence from \isi{Hiligaynon}. There do not seem to be any differences in meaning among these forms. Their use is  idiolectal. 

\ea
\label{bkm:Ref106967184} \label{ex:isigka}
    \ea 
    \label{ex:isigka-a}
    \gll \textit{isigka-ittaw} / \textit{masigka-ittaw} / kasigka-ittaw \\
        fellow-person \\
    \glt ‘fellow person/human’ \\
    \ex 
    \label{ex:isigka-b}
    \gll \textit{isigka-paryo} / \textit{masigka-paryo} / kasigka-paryo \\
    fellow-same \\
    \glt ‘fellow one like (you/us)’ \\
    \ex
    \label{ex:isigka-c}
    \gll \textit{isig-Kagayanen} / \textit{masig-Kagayanen} / kasig-Kagayanen \\
    fellow-Kagayanen \\
    \glt ‘fellow Kagayanen’
    \z
\z

As mentioned above (\sectref{sec:ka}), \textit{ka-} itself can be a nominalizer. The fact that the only root that seems to take these prefixes without \textit{ka}{}- happens to have \textit{ka} as its initial syllable \REF{ex:isigka-c} suggests that the \textit{ka}- component of the prefixes is the nominalizer, and the word \textit{Kagayanen} is already a nominalization. Alternatively, the repetition of \textit{ka} may be suppressed because of haplology (suppression of repetitive syllables): *\textit{isigkakagayanen / *masigkakagayanen}. 

\hspace*{-.7pt}In addition, the word \textit{kasimanwa} ‘townmate’ or ‘countrymate’ may be a weakly suppletive formation from \textit{kasig}{}- plus  \textit{banwa} ‘town/country.’

Though the usual use of \textit{isigka}{}-, \textit{masigka}{}-, and \textit{kasigka}{}- is with a word referring to a person, there is one example in the corpus that describes a fellow cat. This is evidence that at least \textit{masigka-} is productive, and not just a component of certain lexicalized expressions:

\ea
Daw  kanen  paryo  ta  gaumaw  ta  \textbf{masigkakuti} rin  mataning  daw  malagday  iya  na   kagi paryo  ta  ittaw  na mangayo  ta  tabang. \\\smallskip
 \gll Daw  kanen  paryo  ta  ga-umaw  ta  \textbf{masigka-kuti} rin\footnotemark{}   ma-taning  daw  ma-lagday  iya  na   kagi paryo  ta  ittaw  na ma-ngayo  ta  tabang. \\
if/when  3\textsc{s.abs}  same  \textsc{nabs}  \textsc{i.r}-call  \textsc{nabs}  fellow-cat 3\textsc{s.gen}  \textsc{adj}-high.pitch  and  \textsc{adj}-slow  3\textsc{s.gen}  \textsc{lk}  voice \textsc{same}  \textsc{nabs}  person  \textsc{lk} \textsc{a.hap.ir}-request   \textsc{nabs}  help \\
\footnotetext{This writer habitually replaces the pronoun \textit{din} 3\textsc{s.erg/gen} with \textit{rin} when following a word ending in a vowel, though most Kagayanen speakers do not do this. It may be a result of influence from \isi{Tagalog}.}
\glt `When s/he (a cat) is like calling to \textbf{his/her} \textbf{fellow} \textbf{cat} his/her voice is shrill and slow like a person requesting help.’ [JCWE-T-14 13.3]
\z
\subsection{{}-\textit{anen}: having the characteristic of X}
\label{sec:anen}

The suffix -\textit{anen} is an uncommon and non-productive suffix that designates a person or characteristic associated with the root to which it attaches. It is possibly a reflex of the common Bisayan suffix -\textit{anon}. It occurs in only six words in Kagayanen, all of which were probably borrowed from \isi{Hiligaynon}. These are illustrated in \REF{bkm:Ref118704475}:

\ea
\label{bkm:Ref118704475}
\begin{tabbing}
\hspace{3cm} \= \kill
\textbf{Root}  \>  \textbf{Derived form }\\
manggad ‘wealth’ \>  manggranen ‘wealthy’ \\
dengeg ‘dignity’ \>  dengganen ‘honorable’ \\
aļang  ‘not sure’ \>  aļanganen ‘hesitant behavior’ \\
uli ‘return home’ \>  uļianen ‘childlike behavior of older people’\footnotemark{} \\
ambaļ ‘say’ \>  ambaļanen ‘saying/expression/words’ \\
sugid ‘tell’ \>  sugranen ‘conversation’
\end{tabbing}
\footnotetext{This is an idiom referring to the tendency for some adults to return to childlike behavior in their later years.}
\is{nominalization|)}\is{noun-forming processes|)}
\z

\section{Pronouns}
\label{sec:pronouns}

\is{pronouns|(}In this section we describe four types of pronouns in Kagayanen: personal pronouns and enclitics (\sectref{sec:personalpronouns}), interrogative pronouns (\sectref{sec:interrogativepronouns}), indefinite pronouns (\sectref{sec:indefinitepronouns}), and demonstrative pronouns together with related forms (\sectref{sec:deicticpronouns}).

\subsection{Personal pronouns and enclitics}
\label{sec:personalpronouns}
\is{personal pronouns|(}\is{pronouns!personal|(}
The personal pronouns in Kagayanen can be divided into \textit{free personal pronouns}\is{personal pronouns}\is{pronouns!personal}, which have the same distribution as Referring Phrases, and \textit{anaphoric enclitics}\is{anaphoric enclitics}, which are bound to a preceding host. For convenience, we will often refer to these sets collectively as “personal pronouns”. When necessary to make the distinction, we will use the terms “free pronouns” and “enclitic pronouns”. All of these forms vary for case, number, and person. The 3\textsuperscript{rd} person pronouns may refer to people and animate beings. Rarely do personal pronouns refer to inanimate things, though sometimes the 3\textsuperscript{rd} person genitive pronouns are used for things in relation to their parts, such as \textit{iya na puon} (3\textsc{s.gen lk} trunk) ‘its trunk’, referring to a tree. First person plural has two forms: \textit{inclusive}\is{inclusive}, which includes speaker, addressee and possibly some other person(s), and \textit{exclusive}\is{exclusive}, which includes the speaker and some other person(s), but excludes the addressee.

Consistent with the syntactic typology of the language as a whole, the genitive and ergative case pronouns are identical. However, we gloss them distinctly depending on their usages. The actor in a transitive, independent clause is glossed as ergative. All other uses of this pronoun set, including the “possessor” in a Referring Phrase, arguments of nominalized clauses, and a contrastive usage described below, are glossed as “genitive”. We justify these terms in  \sectref{sec:grammaticalrelations} and show that these are structurally distinct grammatical relations, although the pronoun forms are the same.
\tabref{tab:personalpronouns} illustrates the personal pronouns of Kagayanen. The first two columns display the enclitics, while the other columns display the free (or “emphatic/prominent”) pronouns. As can be seen in \tabref{tab:personalpronouns}, the third person singular absolutive enclitic and free/emphatic absolutive pronoun set are the same. All the third person plural forms are the same except for the free ergative/genitive case. Because of this, third person pronouns are not as useful as first and second person pronouns to exemplify the various usages of the set of personal pronouns. In addition, the distinct “long genitive” free pronouns are sociolinguistically marked, and may be archaic. There are no 2\textsuperscript{nd} or 3\textsuperscript{rd} person long genitive forms that are recognized by any speakers we have consulted. The 1\textsuperscript{st} person long genitive forms listed in \tabref{tab:personalpronouns} do occur in the corpus and are heard in conversation. Our impression is that they are used mostly by older individuals, though understood by everyone. Their use seems to imply an extra degree of contrastiveness.

\begin{table}
\caption{Personal ronouns of Kagayanen}
\label{tab:personalpronouns}
\begin{tabularx}{\textwidth}{>{\scshape}llX llll}
\lsptoprule
& \multicolumn{2}{c}{Enclitics} &  \multicolumn{4}{c}{Free/contrastive pronouns} \\
\cmidrule(r){1-3}\cmidrule(l){4-7}
& Abs & Erg/Gen & Gen & Long Gen (archaic) & Abs & Obl \\
\midrule
1sg & a/aren\footnotemark & ko & ake & kende & yaken & ki yaken \\
2sg & ka & no & imo & {}-- & kaon & ki kaon \\
3sg & kanen & din & iya & {}-- & kanen & ki kanen \\
1exc & kay & nay & ame & mende & kami & ki kami \\
1inc & ki & ta & ate & tende & kiten & ki kiten \\
2pl & kaw & nyo & inyo & {}-- & kyo & ki kyo \\
3pl & danen\footnotemark & danen & iran & {}-- & danen & ki danen \\
\lspbottomrule
\end{tabularx}
\end{table}
\footnotetext[21]{\textit{Aren} as an alternate 1\textsc{s.abs} enclitic may be a blend of \textit{a} plus the completive aspectual enclitic \textit{en}. \textit{Aren} alone does seem to imply a completive sense. Ease of pronunciation could have led to the insertion of the \textit{r}, and paradigm pressure from several other pronominal forms ending in -\textit{en} could have facilitated this co-lexicalization. However, the completive enclitic \textit{en} may follow \textit{aren}, so \textit{aren}, if it is a blend, has clearly become lexicalized---26 of the 128 examples of \textit{aren} in the corpus are followed by \textit{en}. Therefore, in this grammar we simply treat \textit{aren} as an alternate 1\textsc{s.abs} enclitic. A discourse study is needed to determine the usages of \textit{a} versus \textit{aren}.}
\footnotetext{Sometimes \textit{danen} is pronounced \textit{daen}. We have observed no consistent pattern to its usage.}
In addition to these enclitic and free pronouns, there are two composite personal pronouns used when first person acts on second person---\textit{ta kaw} is used when both arguments are singular (\ref{bkm:Ref419448603}) and \textit{ta kyo} is used when the 1\textsuperscript{st} person is singular and the 2\textsuperscript{nd} person is plural (\ref{bkm:Ref419448605}). There are no other situations in which composite pronouns are used.

The following examples illustrate a few of the free personal pronouns and enclitics in context:
\ea
\label{bkm:Ref419465634}
Absolutive anaphoric enclitic \\
Tagan  \textbf{a}   din  ta  sid-anan. \\\smallskip
 \gll \emptyset{}-atag-an\footnotemark{}  \textbf{a}   din  ta  sid-anan. \\
\textsc{t.ir}-give\textsc{-apl}  1\textsc{s.abs}  3\textsc{s.erg}  \textsc{nabs}  viand \\
\footnotetext{The root for this verb is \textit{atag}. In conversation, the initial vowel normally drops out when a suffix occurs, though this is optional. In example \REF{bkm:Ref420429206}, below, the initial vowel may not drop out because the verb carries no suffix.}
\glt ‘S/he will give \textbf{me} some of the viand.’
\z
\ea
\label{bkm:Ref419465636}
Absolutive free pronoun \\
Tagan  din  \textbf{yaken}  i  ta  sid-anan. \\\smallskip
 \gll \emptyset{}-atag-an  din  \textbf{yaken}  i  ta  sid-anan. \\
\textsc{t.ir}-give\textsc{-apl}  3\textsc{s.erg}  1\textsc{s.abs}  \textsc{def.n}  \textsc{nabs}  viand \\
\glt ‘S/he will give \textbf{ME} some viand.’ (Emphasis on ME.)
\z

Examples \REF{bkm:Ref419465634} and \REF{bkm:Ref419465636} are identical except that the absolutive case Recipient is expressed with the emphatic pronoun plus demonstrative determiner in \REF{bkm:Ref419465636}.  Emphatic pronouns have the same distributional possibilities as full Referring Phrases, while enclitics follow the first major constituent of the immediate phrase. The demonstrative determiner \textit{i} may be “optional” in this and the following examples, especially in relaxed speech. However, speakers agree that it “sounds better” with the demonstrative determiner:

\ea
Fronted absolutive free pronoun \\
\textbf{Yaken}  \textbf{i}   tagan  din  ta  sid-anan. \\\smallskip
 \gll \textbf{Yaken}  \textbf{i}   \emptyset{}-atag-an  din  ta  sid-anan. \\
1\textsc{s.abs}  \textsc{def.n} \textsc{t.ir}-give\textsc{-apl}  3\textsc{s.erg}  \textsc{nabs}  viand \\
\glt ‘\textbf{I} \textbf{am} \textbf{the} \textbf{one} to whom s/he will give some viand.’
\z

\ea
Ergative anaphoric enclitic \\
Tagan  \textbf{ko}  kanen  an  ta  sid-anan. \\\smallskip
 \gll \emptyset{}-atag-an  \textbf{ko}  kanen  an  ta  sid-anan. \\
\textsc{t.ir}-give\textsc{-apl}  1\textsc{s.erg}  3\textsc{s.abs}  \textsc{def}.\textsc{m}  \textsc{nabs}  viand \\
\glt ‘\textbf{I} will give him/her some of the viand.’
\z

\ea
Genitive anaphoric enclitic \\
Tagan  ko  kanen  an  ta  sid-anan \textbf{ko}. \\\smallskip
 \gll \emptyset{}-atag-an  ko  kanen  an  ta  sid-anan \textbf{ko}. \\
\textsc{t.ir}-give-\textsc{apl}  1\textsc{s.erg}  3\textsc{s.abs}  \textsc{def.m}  \textsc{nabs}  viand  1\textsc{s.gen} \\
\glt ‘I should/must give him/her some of \textbf{my} viand.’
\z

\ea
Genitive free pronoun \\
Tagan  ko  kanen  an  ta  \textbf{ake}  i  na  sid-anan. \\\smallskip
 \gll \emptyset{}-atag-an  ko  kanen  an  ta  \textbf{ake}  i  na  sid-anan. \\
\textsc{t.ir}-give\textsc{-apl}  \textsc{1serg}  3\textsc{s.abs}  \textsc{def.m}  \textsc{nabs}  1\textsc{s.gen}  \textsc{def.n}  \textsc{lk}  viand \\
\glt ‘I will give him/her some of \textbf{my} \textbf{own} (sometimes contrastive – “my very own” rather than someone else’s) viand.’
\z

\ea
Long genitive pronoun \\
Tagan   ko  kanen  an  ta  \textbf{kende}  i  na  sid-anan. \\\smallskip
 \gll \emptyset{}-\textit{Tagan}   ko  kanen  an  ta  \textbf{kende}  i  na  sid-anan. \\
\textsc{t.ir}-give\textsc{-apl}  1\textsc{s.erg}  3\textsc{s.abs}  \textsc{def.m}  \textsc{nabs}  1\textsc{s.gen.ctr}  \textsc{def.n}  \textsc{lk}  viand \\
\glt ‘I will give him/her some of \textbf{my} \textbf{own} (sometimes contrastive – “my very own” rather than someone else’s) viand.’
\z

A free genitive pronoun may recapitulate the absolutive argument. This usage is clearly contrastive:
\ea
\label{bkm:Ref419117602}
Gaatag  kay   \textbf{ame}  ta  sid-anan  ki  kanen. \\\smallskip
 \gll Ga-atag  kay   \textbf{ame}  ta  sid-anan  ki  kanen. \\
\textsc{i.r-}give  1\textsc{p.excl.abs}  1\textsc{p.excl.gen}  \textsc{nabs}  viand  \textsc{obl.p}  3s \\
\glt ‘\textbf{We} \textbf{ourselves} (rather than someone else) gave some viand to him’.
\z

In example \REF{bkm:Ref419117602}, the verb appears in the intransitive form, so the actor \textit{kay} is the absolutive argument. Therefore \textit{ame} is not ergative; rather it occurs after the absolutive pronoun of the same person and number to express strong contrast. Example \REF{bkm:Ref419174393} is similar, but with a first person singular actor. In relaxed speech, the \textit{a} 1\textsc{s.abs} pronoun may elide before \textit{ake}.
\ea
\label{bkm:Ref419174393}
Gaatag   a  \textbf{ake}  ta  sid-anan  ki  kanen. \\\smallskip
 \gll Ga-atag   a  \textbf{ake}  ta  sid-anan  ki  kanen. \\
\textsc{i.r}-give  1\textsc{s.abs} 1\textsc{s.gen}  \textsc{nabs}  viand  \textsc{obl.p}  3s \\
\glt ‘I myself (rather than someone else) gave some viand to him/her.’
\z

\ea
\label{bkm:Ref420429206}  
Free pronoun in Oblique role (see also examples \ref{bkm:Ref419117602} and \ref{bkm:Ref419174393} above) \\
Paatag   din   sid-anan  din   an \textbf{ki}  \textbf{yaken}. \\\smallskip
 \gll Pa-atag   din   sid-anan  din   an \textbf{ki}  \textbf{yaken}. \\
\textsc{t.r}-give  3\textsc{s.erg}  viand  3\textsc{s.gen}  \textsc{def.m}  \textsc{obl.p}  1s \\
\glt ‘S/he gave his/her viand to \textbf{me}.’
\z

\ea
\label{bkm:Ref419448603}
Composite personal pronouns \\
Pangamuyuan  \textbf{ta} \textbf{kaw}  nang  na  kabay makakita  ka   man  ta  ubra  na  dayad. \\\smallskip
 \gll Pangamuyu-an  \textbf{ta} \textbf{kaw}  nang  na  kabay maka-kita  ka   man  ta  ubra  na  dayad. \\
pray\textsc{-apl}  1\textsc{s.erg} 2\textsc{s.abs}  just  \textsc{lk}  may.it.be
\textsc{i.hap.ir}-see  2\textsc{s.abs}  too  \textsc{nabs}  work  \textsc{lk}  good \\
\glt `\textbf{I} will just pray for \textbf{you} that may it be you are able to find good work.’ [DBWL-T-20 8.8]
\z
\ea
\label{bkm:Ref419448605}
Atagan \textbf{ta kyo}  en  ta  tanan  na  pagsuļaten ta  Kinagayanen \\\smallskip
 \gll \emptyset{}-Atag-an  \textbf{ta} \textbf{kyo}  en  ta  tanan  na  pag-suļat-en\footnotemark{} ta  K<in>agayanen\footnotemark{} \\
\textsc{t.ir-}give-\textsc{apl}  1\textsc{s.erg} 2\textsc{p.abs}  \textsc{cm}  \textsc{nabs} all \textsc{lk}  \textsc{nr.act}-write\textsc{-t.ir}
\textsc{nabs}  <\textsc{nr.res}>Kagayanen \\
\footnotetext[24]{The form \textit{pagsuļaten} is a variation of \textit{pagsuļat} ‘writing.’ Some speakers prefer one over the other. There is no difference in meaning or syntactic function between the two as far as we can tell.}
\footnotetext{The use of the infix <\textit{in}> is common in the Philippines to describe the speech, mannerisms or customs of a group of people named in the root, for example \textit{tinagalog}, ‘in the Tagalog way’, \textit{inamerikano} ‘in the American way’ and so on. The use of \textit{kinagayanen} is rare in the corpus for this study, occurring only in two texts, both by well-educated individuals. Most speakers use the term \textit{Kagayanen} for both the language and the people.}
\glt `\textbf{I} will give \textbf{you} (pl) all the ways to write in the Kagayanen language.’ [JCOE-T-06 18.3]
\is{pronouns!personal|)}\is{personal pronouns|)}
\z

\subsection{Interrogative pronouns}
\label{sec:interrogativepronouns}
\is{interrogative pronouns|(}\is{pronouns!interrogative|(}
There are three kinds of interrogative words that differently affect the syntax of a clause in questions. These are interrogative pronouns, interrogative adverbs, and interrogative adjectives (see \chapref{chap:pragmaticallymarkedstructures}, \sectref{sec:questionwordquestions} on the formation of question-word questions for a full description of the uses of these words). In this section we will limit the discussion to the interrogative pronouns.

There are three interrogative pronouns \textit{kino} ‘who’, \textit{ino} ‘what’, and \textit{pila} ‘how many’. We consider these to be distinct from the interrogative adverbs \textit{kan-o} ‘when’, \textit{indi} ‘where’, \textit{indya} ‘which’, and \textit{man-o} ‘why’. The pronouns are verbal arguments and are always in the absolutive role, whereas the adverbs refer to oblique or adverbial elements, and are not linked to the verb via the system of transitivity and voice. The interrogative pronouns usually occur sentence initially. The semantic macro-role of the questioned element is expressed via verbal affixation (transitivity plus applicative marking). The following examples illustrate the interrogative pronoun \textit{kino} ‘who’ in context:

\ea
Non-verbal clause \\
\textbf{Kino}  tag-iya  ta  niog  na  pasanggetan  no? \\\smallskip
 \gll \textbf{Kino}  tag-iya  ta  niog  na  pa-sangget-an  no? \\
who  owner  \textsc{nabs}  coconut  \textsc{lk}  \textsc{t.r}-get.sa\textsc{p.-apl}  2\textsc{s.erg} \\
\glt ‘\textbf{Who} is the owner of the coconut tree that you are getting coconut sap from?’ [RDOI-T-01 21.1]
\z
\ea
Intransitive verbal clause \\ 
O,  \textbf{kino}  magmandar  na  mag-umpisa  dļagan? \\\smallskip
 \gll O,  \textbf{kino}  mag-mandar  na  mag--umpisa  dļagan? \\
oh  who  \textsc{i.ir}-command  \textsc{lk}  \textsc{i.ir}-begin  run \\
\glt `\textbf{Who} will command to begin the running?’ [JCON-L-08 31.7]
\z

\ea
Transitive verbal clause \\
\textbf{Kino}  paanggat  no  ta  iskursyon? \\\smallskip
 \gll \textbf{Kino}  pa-anggat  no  ta  iskursyon? \\
who  \textsc{t.r}-ask.to.go.along  2\textsc{s.erg}  \textsc{nabs}  outing \\
\glt ‘\textbf{Whom} did you ask to go along on the outing?’
\z
\ea
Transitive applicative clause \\
\textbf{Kino}  pasugiran  no  parti   ta  natabo  ya? \\\smallskip
 \gll \textbf{Kino}  pa-sugid-an  no  parti   ta  na-tabo  ya? \\
who  \textsc{t.r}-tell-\textsc{apl}  2\textsc{s.erg}  about  \textsc{nabs}  \textsc{a.hap.r}-happen  \textsc{def.f} \\
\glt `\textbf{Whom} did you tell about what happened?’
\z

The following examples illustrate the interrogative pronoun \textit{ino} in context:

\ea
Non-verbal clause \\
\textbf{Ino}  ubra  ta  mananggiti? \\\smallskip
 \gll \textbf{Ino}  ubra  ta  mananggiti? \\
what  work  \textsc{nabs}  gatherer.of.coconut.sap \\
\glt `\textbf{What} is the work of the one who gathers coconut sap?’ [RDOI-T-01 2.1]
\z
\ea
Transitive clause \\
\textbf{Ino}  pakaan  no? \\\smallskip
 \gll \textbf{Ino}  pa-kaan  no? \\
what  \textsc{t.}\textsc{r}-eat  2\textsc{s.erg} \\
\glt `\textbf{What} did you eat?’
\z

\ea
Transitive (instrumental) applicative clause \\
\textbf{Ino}  igeļet  no  ta  sidda? \\\smallskip
 \gll \textbf{Ino}  i-geļet  no  ta  sidda? \\
what  \textsc{apl2}-cut  2\textsc{s.erg}  \textsc{nabs}  fish \\
\glt `\textbf{What} will you use to cut up fish?’
\z

In questioning someone’s name, the interrogative pronouns \textit{kino} ‘who’ and \textit{ino} ‘what’ are interchangeable for some Kagayanen speakers, though \textit{ino} is probably more common, especially in communities outside of Cagayancillo:
\ea
\textbf{Ino}  ngaran  no? / \textbf{Kino}  ngaran  no? \\\smallskip
 \gll \textbf{Ino/Kino}  ngaran  no? \\
what/Who  name  2\textsc{s.gen} \\
\glt \textbf{‘What/who} is your name?’
\z

The interrogative pronoun \textit{pila} ‘how many’ or \textit{pila buok} ‘how many pieces’ is used only for count nouns. In order to ask about how much rice (a mass noun), one must mention a container, ‘how many sacks of rice’, or a measurement such as  ‘how many kilograms of rice’. For how much something costs, \textit{pila bayad} is used meaning ‘how many (pesos) is the payment?'

\ea
Non-verbal clause \\
\textbf{Pila}  kaw  adlaw  dya? \\\smallskip
 \gll \textbf{Pila}  kaw  adlaw  dya? \\
how.many  2\textsc{p.abs}  day/sun  \textsc{d}4\textsc{loc} \\
\glt ‘\textbf{How} \textbf{many} days are you there?' [EMWN-T-09 3.3]
\z
\ea
\textbf{Pila}  (\textbf{buok})  bata  no  an? \\\smallskip
 \gll \textbf{Pila}  (\textbf{buok)}  bata  no  an? \\
how.many  piece  child 2\textsc{s.gen}  \textsc{def.f} \\
\glt ‘\textbf{How} \textbf{many} are your children?’
\z
\ea
\textbf{Pila}  kilo  beggas  paliten  no? \\\smallskip
 \gll \textbf{Pila}  kilo  beggas  palit-en  no? \\
how.many  kilogram  milled.rice  buy-\textsc{t.ir}  2\textsc{s.erg} \\
\glt `\textbf{How} \textbf{many} kilograms of rice will you buy?' \\\smallskip

*Pila beggas paliten no?
\z

\ea
\textbf{Pila}  bayad  ta  tinapay  an? \\\smallskip
 \gll \textbf{Pila}  bayad  ta  tinapay  an? \\
how.many  pay  \textsc{nabs}  bread  \textsc{def.m} \\
\glt `\textbf{How} \textbf{much} is the payment of the bread?’
\z

\ea
Transitive clause \\
\textbf{Pila}  nakamang  no? \\\smallskip
 \gll \textbf{Pila}  na-kamang  no? \\
how.many  \textsc{a.hap.r}-get  2\textsc{s.erg} \\
\glt `\textbf{How} \textbf{many} did you get?
\is{pronouns!interrogative|)}\is{interrogative pronouns|)}
\z

\subsection{Indefinite pronouns}
\label{sec:indefinitepronouns}
\is{indefinite pronouns|(}\is{pronouns!indefinite|(}
The indefinite pronouns are formed with either \textit{bisan} ‘any’ or \textit{daw} ‘if/when’ plus one of the interrogative pronouns: \textit{kino} ‘who’, \textit{ino} ‘what’, \textit{pila} ‘how much/many’ or interrogative adverbs \textit{kan-o} ‘when’, and \textit{indi} ‘where’, but not with \textit{man-o} ‘why’. To say ‘any time’ \textit{kan-o} is not usually used with \textit{bisan}---there is only one example in the corpus.  Instead \textit{bisan} ‘any’ is followed by \textit{ino} ‘what’ and then a time word like \textit{oras} ‘hour/time’, \textit{adlaw} ‘sun/day’, \textit{duminggo} ‘week’, \textit{buļan} ‘moon/month’ or \textit{taon} ‘year’. An optional modifier and linker can follow \textit{bisan} plus the interrogative pronoun or adverb but the modifier is obligatory with \textit{daw} plus interrogative pronoun or adverb.

The indefinite pronouns with \textit{bisan} or \textit{daw} plus an interrogative pronoun, but not interrogative adverbs, occur as arguments of the verb with the regular case forms – zero for absolutive case, \textit{ta} for non-absolutive case, or \textit{naan (ta)} or \textit{ta} alone for any locational or spatial oblique role. The indefinite pronouns with \textit{daw} ‘if/when’ only occur in the absolutive case while indefinite pronouns with \textit{bisan} ‘any’ can occur either in the absolutive or non-absolutive case. The indefinite pronouns do not occur with the demonstrative determiners \textit{i}, \textit{an}, and \textit{ya}, since these determiners conflict with the indefiniteness or non-specificity of these pronouns.

\begin{table}
\label{indefinitepronouns}
\caption{Indefinite pronouns and adverbs}
\begin{tabularx}{\textwidth}{Xl}
\lsptoprule
\multicolumn{2}{c}{Indefinite pronouns} \\
\midrule
bisan kino    ‘anyone’ & daw kino  ‘whoever’ \\
bisan ino    ‘anything’ & daw ino  ‘whatever’ \\
bisan pila    ‘any number’ & daw pila   ‘whatever number’ \\
\midrule
\multicolumn{2}{c}{{Indefinite adverbs}} \\
\midrule
bisan ino na oras  ‘any time’  & daw ino na oras   ‘whatever time’ \\
bisan kan-o `any time' & daw kan-o  ‘whenever’ \\
bisan indi    ‘anywhere’ & daw indi  ‘wherever’ \\
bisan ino na paagi  ‘any means/way’ & daw ino na paagi ‘whatever way’ \\
\lspbottomrule
\end{tabularx}
\end{table}
The following are some examples of indefinite pronouns in context:
\ea
Indefinite pronouns in absolutive case \\
\textbf{Bisan}  \textbf{kino}  nang,  daw  dili  magpati,  paumaw  din  ta iya  na  upisina  daw  silutan  din  na  uļa  gid  kaluoy. \\\smallskip
 \gll \textbf{Bisan}  \textbf{kino}  nang,  daw  dili  mag-pati,  pa-umaw  din  ta iya  na  upisina  daw  \emptyset{}-silot-an  din  na  uļa  gid  ka-luoy. \\
any  who  just  if/when  \textsc{neg.ir}  \textsc{i.ir}-believe/obey  \textsc{t.r}-call  3\textsc{s.erg}  \textsc{nabs} 3\textsc{s.gen}  \textsc{lk}  office  and \textsc{t.ir}-punish\textsc{-apl}  3\textsc{s.erg}  \textsc{lk}  \textsc{neg}  \textsc{int}  \textsc{nr}-mercy \\
\glt `Just \textbf{anyone}, if (s/he) will not obey (him), he called to his office and punished without mercy.’ [BEWN-T-01 2.15]
\z
\ea
Paatagan  ko  ta  pagkaan  \textbf{daw}  \textbf{kino}  na  naļettem. \\\smallskip
 \gll Pa-atag-an  ko  ta  pagkaan  \textbf{daw}  \textbf{kino}  na  na-ļettem. \\
\textsc{t.r}-give\textsc{-apl}  1\textsc{s.erg}  \textsc{nabs}  food  if/when  who  \textsc{lk}  \textsc{a.hap.r}-hungry \\
\glt ‘I gave food to \textbf{whomever} was hungry.’
\z
\ea
Pedro,  \textbf{daw}  \textbf{kino}  gani  na  gapasilong  ta imo  na  bubong  o  ta  imo  na  atep,  batunon  no ta  bug-os  ta  imo  na  tagipusuon. \\\smallskip
 \gll Pedro,  \textbf{daw}  \textbf{kino}  gani  na  ga-pa-silong  ta imo  na  bubong  o  ta  imo  na  atep,  baton-en  no ta  bug-os  ta  imo  na  tagipusuon. \\
Pedro  if/when  who  \textsc{atn}  \textsc{lk}  \textsc{i.r-caus}-yard/shelter  \textsc{nabs} 2\textsc{s.gen}  \textsc{lk}  roof\footnotemark{}  or  \textsc{nabs}  2\textsc{s.gen}  \textsc{lk}  roof  receive\textsc{-t.ir}  2\textsc{s.erg} \textsc{nabs}  completely  \textsc{nabs}  2\textsc{s.gen}  \textsc{lk}  heart \\
\footnotetext{The words \textit{atep} and \textit{bubong} are used interchangeably as ‘roof’ with no difference in meaning. Sometimes \textit{atep} is the general word, and \textit{bubong} refers to the ‘the very highest peak of the roof.’ In \isi{Tagalog}, \textit{atip} is a grass or palm roof and \textit{bubong} is any kind of roof, but we have noted no such distinction in everyday Kagayanen speech (though perhaps professional house builders would).}
\glt `Pedro, \textbf{whoever} takes shelter under your roof, receive (him/her) completely with your heart.’ [JCOB-L-02 9.1]
\z
\ea
Papalit  din  \textbf{bisan}  \textbf{ino}  (na  gusto  din). \\\smallskip
 \gll Pa-palit  din  \textbf{bisan}  \textbf{ino}  (na  gusto  din). \\
\textsc{t.r}-buy  3\textsc{s.erg}  any  what  \textsc{lk}  want  3\textsc{s.erg} \\
\glt ‘S/he bought \textbf{anything} he/she wanted.’
\z
\ea
Ambaļ  din  en  \textbf{bisan}  \textbf{ino}  pa  kon  ayuon  din  atag  din ... \\\smallskip
 \gll Ambaļ  din  en  \textbf{bisan}  \textbf{ino}  pa  kon  ayo-en  din  atag  din ... \\
say  3\textsc{s.erg}  \textsc{cm}  any  what  \textsc{inc}  \textsc{hsy}  request\textsc{-t.ir}  3\textsc{s.erg}  give  3\textsc{s.erg} \\
\glt `He (the king) had already said that \textbf{anything} he (another person) asks for he (the king) will give ...’ [PBON-T-01 4.2]
\z
\ea
\textbf{Daw}  \textbf{ino}  (na)  atag  ki  kaon,  salamatan  no. \\\smallskip
 \gll \textbf{Daw}  \textbf{ino}  (na)  \emptyset{}-atag  ki  kaon,  \emptyset{}-salamat-an  no. \\
if/when  what  \textsc{lk}  \textsc{t.ir-}give  \textsc{obl.p}  \textsc{2s}  \textsc{t.ir}-thank\textsc{-apl}  2\textsc{s.erg} \\
\glt ‘\textbf{Whatever} is given to you, give thanks (for it).’
\z
\ea
Papalit  din  \textbf{bisan}  \textbf{pila}  (na  gusto  din). \\\smallskip
 \gll Pa-palit  din  \textbf{bisan}  \textbf{pila}  (na  gusto  din). \\
\textsc{t.r}-buy  3\textsc{s.erg}  any  how.many  \textsc{lk}  want  3\textsc{s.erg} \\
\glt ‘S/he bought \textbf{any} \textbf{amount/at} \textbf{any} \textbf{price} (he/she wanted).’ (This sentence can be understood in either of the two ways indicated in the free translation.)
\z
\ea
Iatag  ko  \textbf{daw}  \textbf{pila}  (na)  ayuon  no. \\\smallskip
 \gll I-atag  ko  \textbf{daw}  \textbf{pila}  (na)  ayo-en  no. \\
\textsc{t.deon}-give  1\textsc{s.erg}  if/when  how.many  \textsc{lk}  request\textsc{-t.ir}  2\textsc{s.erg} \\
\glt ‘I will give (you) \textbf{whatever} \textbf{amount} you request.’
\z
\ea
Indefinite pronouns in non-absolutive case \\
Gaatag  a  ta  pagkaan  ta  \textbf{bisan}  \textbf{kino}  (na  ittaw). \\\smallskip
 \gll Ga-atag  a  ta  pagkaan  ta  \textbf{bisan}  \textbf{kino}  (na  ittaw). \\
\textsc{i.r-}give  1\textsc{s.abs}  \textsc{nabs}  food  \textsc{nabs}  any  who  \textsc{lk}  people \\
\glt ‘I gave food to \textbf{any} \textbf{person}.’ \\\smallskip

*Gaatag a ta pagkaan ta daw kino (na ittaw). \\
*Gaatag a ta pagkaan daw kino (na ittaw). \\
*Gaatag a ta pagkaan ta kino (na ittaw).
\z
\ea
Gapalit  kanen  ta  \textbf{bisan}  \textbf{ino}  (na  gusto  din). \\\smallskip
 \gll Ga-palit  kanen  ta  \textbf{bisan}  \textbf{ino}  (na  gusto  din). \\
\textsc{i.r}-buy  3\textsc{s.abs}  \textsc{nabs}  any  what  \textsc{lk}  want  3\textsc{s.erg} \\
\glt ‘S/he bought \textbf{anything} (s/he wants).’ \\\smallskip

*Gapalit kanen ta \textbf{daw} \textbf{ino} (na gusto din). \\
*Gapalit kanen \textbf{daw ino} (na gusto din). \\
*Gapalit kanen ta \textbf{ino} (na gusto din).
\z
\ea
Gapalit  kanen  ta  \textbf{bisan}  \textbf{pila}  (na  gusto  din). \\\smallskip
 \gll Ga-palit  kanen  ta  \textbf{bisan}  \textbf{pila}  (na  gusto  din). \\
\textsc{i.r}-buy  3\textsc{s.abs}  \textsc{nabs}  any  how.many  \textsc{lk}  want  3\textsc{s.erg} \\
\glt ‘S/he bought \textbf{any} \textbf{amount} (s/he wants).’ \\\smallskip

*Gapalit kanen ta \textbf{daw} \textbf{pila} (na gusto din). \\
*Gapalit kanen \textbf{daw pila} (na gusto din). \\
*Gapalit kanen ta \textbf{pila} (na gusto din).
\z
\ea
Indefinite adverbs \\
Umawen  a  no  nang  \textbf{bisan}  \textbf{ino}  \textbf{na}  \textbf{oras}. \\\smallskip
 \gll Umaw-en  a  no  nang  \textbf{bisan}  \textbf{ino}  \textbf{na}  \textbf{oras}. \\
call\textsc{-t.ir}  1\textsc{s.abs}  2\textsc{s.erg}  just  any  what  \textsc{lk}  time/hour \\
\glt ‘Just call me \textbf{anytime}.’
\z
\ea
Dili	ko	mabayad-bayaran	\textbf{bisan}	\textbf{kan-o}. \\\smallskip
 \gll Dili	ko	ma-bayad\~bayad-an	\textbf{bisan}	\textbf{kan-o}. \\
\textsc{neg.ir}	1\textsc{s.erg}	\textsc{a.hap.ir-red}-pay-\textsc{APL}	any	when \\
\glt ‘I am not able to pay (my debt of gratitude) at any time.’ [EFWN-T-11 17.8] 
\z
\ea
Buaten  ko  \textbf{bisan}  \textbf{ino}  \textbf{na}  \textbf{paagi}  aged  miling  a  dya. \\\smallskip
 \gll Buat-en  ko  \textbf{bisan}  \textbf{ino}  \textbf{na}  \textbf{paagi}  aged  m-iling  a  dya. \\
do\textsc{-t.ir}  1\textsc{s.erg}  any  what  \textsc{lk}  means  so.that  \textsc{i.v.ir-}go  1\textsc{s.abs}  \textsc{d4loc} \\
\glt ‘I will do it \textbf{by} \textbf{any} \textbf{means} so that I will go there.’
\z
\ea
Patudluan  kay  din  \textbf{daw}  \textbf{ino}  \textbf{na}  \textbf{paagi} ta  pagtanem  ta  guso. \\\smallskip
 \gll Pa-tudlo-an  kay  din  \textbf{daw}  \textbf{ino}  \textbf{na}  \textbf{paagi} ta  pag-tanem  ta  guso. \\
\textsc{t.r}-teach\textsc{-apl}  1\textsc{p.excl.abs}  3\textsc{s.erg}  if/when  what  \textsc{lk}  way/means
\textsc{nabs}  \textsc{nr.act}-plant  \textsc{nabs}  agar.seaweed \\
\glt `S/he taught us \textbf{whatever} \textbf{way} to plant agar-agar seaweed.’
\z
\ea
Uļa  a  naļam  \textbf{daw}  \textbf{ino}  \textbf{oras}  muli  tatay  ko. \\\smallskip
 \gll Uļa  a  na-aļam  \textbf{daw}  \textbf{ino}  \textbf{oras}  m-uli  tatay  ko. \\
\textsc{neg.r}  \textsc{1s.abs}  \textsc{a.hap.r-}know  if/when what  time/hour  \textsc{i.v.r}-go.home father  1\textsc{s.gen} \\
\glt ‘I do not know \textbf{whatever} \textbf{time/hour} my father will come home.’
\z
\ea
\textbf{Daw}  \textbf{kan-o}  kani  uli  sawa  no  ya  liso  ta  kalabasa  na  darwa  nai  tanem  ta  kuron  na  busļot. \\\smallskip
 \gll \textbf{Daw}  \textbf{kan-o}  kani  uli  sawa  no  ya  liso  ta  kalabasa  na  darwa  nai  tanem  ta  kuron  na  busļot. \\
if/when  when  later  go.home  spouse  2\textsc{s.gen}  \textsc{def.f}  seed  \textsc{nabs} summer.squash  \textsc{lk}  two  \textsc{d}1\textsc{abs}  plant  \textsc{nabs} clay.pot  \textsc{lk}  hole \\
\glt \textbf{‘Whenever} your spouse comes home later, plant these two squash seeds in a clay pot with a hole.’ [AION-C-01 9.4]
\z
\ea
Dayon  en  garay  na  paumaw  mga  maligno  ta  \textbf{bisan}  \textbf{indi} nang  na lugar  na  marani... \\\smallskip
 \gll Dayon  en  garay  na  pa-umaw  mga  maligno  ta  \textbf{bisan}  \textbf{indi} nang  na lugar  na  ma-rani... \\
right.away  \textsc{cm}  chant  \textsc{lk}  \textsc{t.r}-call  \textsc{pl}  evil  \textsc{nabs}  any  where just  \textsc{lk}
place  \textsc{lk}  \textsc{i.ir-}approach \\
\glt `Then right away (they) chant calling the evil (ones) \textbf{from} \textbf{anywhere} to come close ...’ [JCWE-T-17 3.5]
\is{pronouns!indefinite|)}\is{indefinite pronouns|)}
\z

\subsection{Demonstrative pronouns and related forms}
\label{sec:deicticpronouns}
\is{demonstrative pronouns|(}\is{pronouns!demonstrative|(}
Kagayanen employs a rich and complex system of spatial deixis involving four spatial distinctions, a general/precise distinction, and different pronominal, adjectival, enclitic and adverbial forms. In this section, we describe the differences among these various forms, and provide examples of their usages. Although not all of these are, strictly speaking, pronouns, they all function in the domain of participant reference, and many are clearly related to pronouns etymologically and functionally. For this reason it is reasonable to treat them all in the current section.

We begin with the distinction between two types of free demonstrative forms, which we have termed \textit{demonstrative pronouns}\is{demonstrative pronouns}\is{pronouns!demonstrative} and \textit{demonstrative adjectives}\is{demonstrative adjectives}\is{adjectives!demonstrative}, respectively. These terms characterize in a general way the main usages of these forms, though both types have a distribution that transcends these general terms. Demonstrative adjectives are further divided between pre-Head demonstrative adjectives, and post-Head demonstrative adjectives. This difference is discussed in the following subsection.

Though the functions of demonstrative pronouns and pre-Head demonstrative adjectives seem to overlap, they are in fact mutually exclusive. These categories may be distinguished, as outlined in \tabref{tab:demonstrativepronounsadjectives}.

\begin{table}
\caption{Demonstrative pronouns and adjectives}
\label{tab:demonstrativepronounsadjectives}
\begin{tabularx}{\textwidth}{Qp{3cm}Q}
\lsptoprule
& Demonstrative\newline  pronouns & Pre-head demonstrative adjectives \\
\midrule
Replace a Referring Phrase in post-verbal position & YES & NO \\
\tablevspace
Replace a Referring Phrase in pre-verbal position & NO & YES (absolutive only) \\
\tablevspace
Occur inside a Referring Phrase in pre-head Modifier position & NO & YES \\
\tablevspace
Occur in Absolutive/non-absolutive pairs & YES & NO \\
\lspbottomrule
\end{tabularx}
\end{table}

Notice that both demonstrative pronouns and pre-Head demonstrative adjectives may replace a Referring Phrase, however the circumstances under which this may occur are very different for the two sets. The examples below illustrate these usages.

The forms of the demonstrative pronouns are displayed in \tabref{tab:demonstrativepronouns}, in which the rows D1, D2, and so on, represent values in the spatial deixis system. D1 demonstratives refer to something near the speaker, D2 near the addressee, D3 somewhere in the area around speaker and addressee, and D4 far away and out of sight. The “general” demonstratives are the default, unmarked forms, that may express varying degrees of deictic precision. The “precise” demonstratives express a very restricted deictic range. This distinction may be approximated in English demonstratives by the distinction between “this X” and “this very X”; “that Y” and “that very Y.”

\begin{table}
\caption{Demonstrative pronouns}
\label{tab:demonstrativepronouns}
\begin{tabularx}{\textwidth}{Xl@{\qquad\qquad}l@{\qquad\qquad}l}
\lsptoprule
& Abs & \multicolumn{2}{c}{Non-abs} \\
\cmidrule{3-4}
&  & General & Precise \\
\midrule
D1 near speaker & ni & ti & ti ni \\
D2 near hearer & nyan & san & san nyan \\
D3 near both & nan & tan & tan nan \\
D4 far away & nya & sa & sa nya \\
\lspbottomrule
\end{tabularx}
\end{table}
As shown in \tabref{tab:demonstrativepronouns}, absolutive demonstrative pronouns do not exhibit a  general/precise distinction. None of the demonstrative pronouns reflect a singular/plural distinction, so they are translated as singular in the English free translations unless the original context implies plurality. The following are a few examples of absolutive demonstrative pronouns (bolded) in context:

\newpage
\ea
Nļettem  gid  man  taan  \textbf{ni}   tak  uļa  kay kan-en. \\\smallskip
 \gll Na-ļettem  gid  man  taan  \textbf{ni}   tak  uļa  kay kan-en. \\
\textsc{a.hap.r}-hungry  \textsc{int}  \textsc{emph}  maybe  \textsc{d1pr}  because  \textsc{neg.r} 1\textsc{p.excl.abs} cooked.rice \\
\glt ‘This (one) is perhaps really hungry because we have no cooked rice.’ [AION-C-01 4.5]
\z
\ea
Piro,  Mam,  papalit  nay  \textbf{ni}  tak  naluoy kay  ki  danen ... \\\smallskip
 \gll Piro,  Mam,  pa-palit  nay  \textbf{ni}  tak  na-luoy kay  ki  danen ... \\
but  Ma’am  \textsc{t.}\textsc{r}-buy  1\textsc{p.excl.erg}  \textsc{d1pr}  because  \textsc{a.hap.r}-pity 1\textsc{p.excl.abs}  \textsc{obl.p}  3p \\
\glt `But, Ma’am, we bought \textbf{these} because we pitied them.’ (things other people asked them to buy when they went to the town/city) [AFWL-L-01 8.12]
\z
\ea
Sayang  nang  Manang  tak  nagustuan  no  \textbf{nya}. \\\smallskip
 \gll Sayang  nang  Manang  tak  na-gustu-an  no  \textbf{nya}. \\
too.bad  only  older.sister  because  \textsc{a.hap.r}-want\textsc{-apl}  2\textsc{s.erg}  \textsc{d4abs} \\
\glt ‘Too bad, Manang, because you wanted \textbf{that}.’ (A huge coral that had been broken up, so no longer exists) [MAWL-C-03 4.6]
\z

The non-absolutive demonstrative pronouns function in ergative (\ref{bkm:Ref419468709} and \ref{bkm:Ref419525739}), or oblique (\ref{bkm:Ref420432602} and \ref{bkm:Ref420432613}) roles. Similarly, they function within Referring Phrases (RPs) in the genitive role (\ref{bkm:Ref116482624} and \ref{bkm:Ref420432688}):

\newpage
\ea
\label{bkm:Ref419468709}
Non-absolutive demonstrative pronoun in the ergative role: \\
Ambaļ  \textbf{ti},   “Marga  kon  en.” \\\smallskip
 \gll Ambaļ  \textbf{ti},   “M-larga  kon  en.” \\
say  \textsc{d1nabs}  \textsc{i.v.ir}-depart  \textsc{hsy}  \textsc{cm} \\
\glt ‘\textbf{This} \textbf{one} said, “(The boat) will depart now".’
\z
\ea
\label{bkm:Ref419525739}
Inday  daw  ino  paliten  \textbf{sa}. \\\smallskip
 \gll Inday  daw  ino  palit-en  \textbf{sa}. \\
don’t.know  if/when  what  buy\textsc{-t.ir}  \textsc{d4nabs} \\
\glt ‘(I) don’t know what \textbf{that} \textbf{one} will buy.’
\z
\ea
\label{bkm:Ref420432602}
Non-absolutive demonstrative pronoun in oblique roles \\
Uyi  na  ubra  Kagayanen  nang  gabuat  \textbf{ti}. \\\smallskip
 \gll Uyi  na  ubra  Kagayanen  nang  ga-buat  \textbf{ti}. \\
\textsc{emph.d1abs}  \textsc{lk}  \textsc{do/}work  Kagayanen  only  \textsc{i.r}-do/make  \textsc{d1nabs} \\
\glt ‘This work (special weave of mats) only Kagayanens do/make \textbf{this}.’ [BCWE-T-09 2.15]
\z
\ea
\label{bkm:Ref420432613}
Tagan  ki  \textbf{sa}. \\\smallskip
 \gll \emptyset{}-atag-an  ki  \textbf{sa}. \\
\textsc{t.ir}-give\textsc{-apl}  1\textsc{p.incl.abs}  \textsc{d4nabs} \\
\glt ‘Give us some of \textbf{that}.’ [TTOB-L-03 7.11]
\z
\ea
\label{bkm:Ref116482624}
Non-absolutive demonstrative pronouns in a genitive role \\
Lasa \textbf{tan}  sikad  na  tam-is… \\\smallskip
 \gll Lasa \textbf{tan}  sikad  na  tam-is… \\
taste  \textsc{d3nabs}  very  \textsc{lk}  sweet \\
\glt ‘The taste \textbf{of} \textbf{that} is very sweet…’ (This text is describing a pineapple.) [MEWE-T-01 2.2]
\z
\ea
\label{bkm:Ref420432688}
Lain  man  nyan,  bao  \textbf{san}  na  agas  a. \\\smallskip
 \gll Lain  man  nyan,  bao  \textbf{san}  na  agas  a. \\
bad  \textsc{emph}  \textsc{d2pr}  odor  \textsc{d2nabs}  \textsc{lk}  kerosene  \textsc{inj} \\
\glt ‘That is really bad, the odor \textbf{of} \textbf{that} kerosene.’ [RZWN-T-02 4.10]
\z

\newpage
\ea
Non-absolutive demonstrative pronoun in adjunct roles \\
Ta,  di  lasga  nang  \textbf{sa}  magpulos. \\\smallskip
 \gll Ta,  di  lasga  nang  \textbf{sa}  mag-pulos. \\
so  \textsc{rq}  ant  only  \textsc{d4nabs}  \textsc{i,ir}-value \\
\glt ‘So, only ants will get value \textbf{from} \textbf{that}?’ (This is about an unnecessary killing of a bird.) [MEWN-T-02 5.9]
\z
\ea
Daw  bui  pa  inta  Tatay  ya  pwidi  kanen  makaisturya  \textbf{san}… \\\smallskip
 \gll Daw  bui  pa  inta  Tatay  ya  pwidi  kanen  maka-isturya  \textbf{san}… \\
if/when  alive  \textsc{inc}  \textsc{opt}  father  \textsc{def.f}  can  3\textsc{s.abs}  \textsc{i.hap.ir}-talk/tell.story  \textsc{d2nabs} \\
\glt ‘If Father were still alive he could tell the story \textbf{about} \textbf{that}…’ [BBOE-C 02 1.7]
\z
\ea
Non-absolutive demonstrative pronoun, precise form \\
Magsekeb  ki  \textbf{tan} \textbf{nan}. \\\smallskip
 \gll Mag-sekeb  ki  \textbf{tan} \textbf{nan}. \\
\textsc{i.ir-}measure  1\textsc{p.incl.abs}  \textsc{d3nabs} \textsc{pr} \\
\glt ‘Let’s measure \textbf{that} \textbf{very} \textbf{one}.’
\z
\ea
Dayon  din  gabot  ta  iya  na   sundang  naan  ta  iya  na awak daw  insaan  darwa  na  manakem  daw  ino  engeran \textbf{ti} \textbf{ni}. \\\smallskip
 \gll Dayon  din  gabot  ta  iya  na   sundang  naan  ta  iya  na awak daw  \emptyset{}-insa-an\footnotemark{}  darwa  na  manakem  daw  ino  enged-an \textbf{ti} \textbf{ni}. \\
right.away  3\textsc{s.erg}  draw.out  \textsc{nabs}  3\textsc{s.gen}  \textsc{lk}  machete  \textsc{spat.def}  \textsc{nabs}  3\textsc{s.gen}  \textsc{lk}
waist and  \textsc{t.ir}-ask-\textsc{apl}  two  \textsc{lk}  older  if/when  what  purpose-\textsc{nr} \textsc{d1nabs} \textsc{d1.pr} \\
\footnotetext{Irrealis modality is often used at points of high tension in a story such as this and/or in the final conjunct of a closely linked series of realis events (see \chapref{chap:clausecombining}, \sectref{sec:culminativeuse} . This is a place of high tension in the story. where irrealis is also used because he is prepared to attack with his machete. Irrealis is also used with events of high tension and peaks or climax of narratives.}
\glt ‘Then he immediately drew out his machete on his waist and asked the older couple what was the reason for \textbf{this} \textbf{very} \textbf{one}.’ (The one asking this question heard a jackfruit say, ‘Dad.’) [YBWN-T-01 5.8]
\z

\newpage
\ea
Daw  dili  ka  makadaļa  tanan  \textbf{ti} \textbf{ni} bunaļen  ka  man  ta  isya  pasong. \\\smallskip
 \gll Daw  dili  ka  maka-daļa  tanan  \textbf{ti} \textbf{ni} bunaļ-en  ka  man  ta  isya  pasong. \\
if/when  \textsc{neg.ir}  2\textsc{s.abs}  \textsc{i.hap.ir}-take  all  \textsc{d1nabs} d1pr whi\textsc{p.-t.ir}  \textsc{2sabs}  too  \textsc{nabs}  one  50.kilo.sack \\
\glt `If you cannot bring all of \textbf{these} \textbf{very} \textbf{things}, you will be whipped with fifty lashes.’ (The literal Kagayanen says 50-kilo-sack. So \textit{sack} is used as a euphemism here meaning fifty lashes.) [JCWN-T-20 12.3]
\z

Demonstrative adjectives are similar to demonstrative pronouns, except that they usually occur as constituents of Referring Phrases rather than replacements for them. \tabref{tab:demonstrativeadjectives} illustrates the forms of demonstrative adjectives.

\begin{table}
\caption{Demonstrative adjectives}
\label{tab:demonstrativeadjectives}
\begin{tabularx}{\textwidth}{llllQl}
\lsptoprule
& \multicolumn{4}{c}{Pre-Head} & Post-Head \\
\cmidrule(r){2-5}
& General & Emphatic & Precise & Precise Emphatic \\
\midrule
D1 & yi/wi & uyi/uwi & yi ni & uyi ni/uwi ni & nai \\
D2 & yan & yan & yan nyan & yan nyan & nayan \\
D3 & yon & uyon & yon nan & uyon nan & naya \\
D4 & yo & uyo & yo nya & uyo nya & nayo \\
\lspbottomrule
\end{tabularx}
\end{table}

A pre-Head demonstrative adjective may occur alone as an absolutive Referring Expression only when it occurs in pre-predicate position (see examples \ref{bkm:Ref420931966} and \ref{bkm:Ref420931983} below). The usual function of demonstrative adjectives, however, is as pre-Head modifiers of RPs in any case role, in any allowable position.
\ea
\label{bkm:Ref420931966}
\textbf{Uyi}  gid  una  na  suwa  ta  Cagayan  i. \\\smallskip
 \gll \textbf{U-yi}  gid  una  na  suwa  ta  Cagayan  i. \\
\textsc{emph}-\textsc{d}1\textsc{abs}  \textsc{int}  first  \textsc{lk}  citrus.fruit  \textsc{nabs}  Cagayan  \textsc{def.n} \\
\glt ‘\textbf{This} really is the first citrus fruit on Cagayancillo.’ [MBON-T-06 6.2]
\z
\ea
\label{bkm:Ref420931983}
\textbf{Yon}  masugid  ko  ki  kyo. \\\smallskip
 \gll \textbf{Yon}  ma-sugid  ko  ki  kyo. \\
\textsc{d3abs}  \textsc{a.hap.r}-tell  1\textsc{s.erg}  \textsc{obl.p}  2\textsc{p.abs} \\
\glt ‘\textbf{That} is what I can tell you.’ [MEWE-T-0 2.6]
\z

\newpage
\ea
\textbf{Yon}  na  mga  kaoy  dapat  ta  itanem. \\\smallskip
 \gll \textbf{Yon}  na  mga  kaoy  dapat  ta  i-tanem. \\
\textsc{d3abs}  \textsc{lk}  \textsc{pl}  tree  must  1\textsc{p.incl.erg}  \textsc{t.deon}-plant \\
\glt ‘\textbf{Those} trees we must plant.’ [ROOB-T-01 8.10]
\z
\ea
Gatunuga  kay  ta  isya  na  kilem  ta  balay  \textbf{nai}. \\\smallskip
 \gll Ga-tunuga  kay  ta  isya  na  kilem  ta  balay  \textbf{nai}. \\
\textsc{i.r}-sleep  1\textsc{p.excl.abs}  \textsc{nabs}  one  \textsc{lk}  night  \textsc{nabs}  house  \textsc{d}1\textsc{abs} \\
\glt ‘We slept one night in \textbf{this} house.' [AGWN-L-01 6.2]
\z
\ea
\textbf{Yi} \textbf{ni}  na  pulo  nalibutan  ta  puti  na  pantad. \\\smallskip
 \gll \textbf{Yi} \textbf{ni}  na  pulo  na-libut-an  ta  puti  na  pantad. \\
\textsc{d1abs} \textsc{d1pr} \textsc{lk}  island  \textsc{a.hap.r}-surround-\textsc{apl}  \textsc{nabs} ‘white  \textsc{lk}   sand \\
\glt ‘\textbf{This} \textbf{very} island is surrounded by white sand.’ [VAWL-T-14 5.9]
\z
\ea
\textbf{Yi}  na  ittaw  \textbf{ni}  naļam  mag-ubra  ta  lamisaan daw pungkuan  na  folding. \\\smallskip
 \gll \textbf{Yi}  na  ittaw  \textbf{ni}  na-aļam  mag--ubra  ta  lamisa-an daw pungko-an  na  folding. \\
\textsc{d1abs}  \textsc{lk}  person  \textsc{d1pr}  \textsc{a.hap.r}-know  \textsc{i.ir}-do/work  \textsc{nabs}  table-\textsc{nr} and
sit-\textsc{nr}  \textsc{lk}  folding \\
\glt `\textbf{This} \textbf{very} person knows how to make tables and chairs that are folding.’ [DBWN-T-23 9.9]
\is{pronouns!demonstrative|)}\is{demonstrative pronouns|)}
\is{pronouns|)}
\z

\section{Referring Phrases}
\label{sec:referringphrases}
\is{Referring Phrases|(}
In this section we describe syntactic constructions that allow speakers to refer to participants in discourse. Traditionally, these would be termed “Noun Phrases”. However, we consciously adopt the more functional label “Referring Phrases” (RP) for two reasons:

\begin{itemize}
\item The Head of this construction type may be a noun, adjective, numeral, verb, precategorial root, or a clause, and the Head may itself be syntactically complex (see \sectref{sec:referringphrasesheadedbynominalizedclauses} and \sectref{sec:rpscontainingrelativeclauses} below). Therefore the structure is not always headed by something that can independently be identified as a “noun”.
\item
Similarly, the major dependent elements within RPs are not limited to particular lexical or syntactic categories. Rather, they are best thought of first as functional positions (Determination, Modification, Quantification and Deixis) that may be filled by various lexical or phrasal elements as suits the communicative needs of speakers.
\end{itemize}

Since most of the elements of these constructions are functionally characterized, it makes sense to refer to the structure itself in terms of its prototypical function – namely to refer to participants in discourse. The term “Referring Phrase” or “Reference Phrase” \citep{vanvalin2006} is also standard in some emerging linguistic theories. Our use of the term is generally consistent with such theories, but does not imply that our analysis as a whole follows any particular theoretical framework.

We recognize three types of Referring Phrases in Kagayanen: basic RPs, RPs headed by personal names, and RPs headed by nominalized clauses. The structural “templates” for these types of phrases will be discussed and exemplified in \sectref{sec:basicreferringphrases}, \sectref{sec:referringphrasesheadedbypersonalnames}, and \sectref{sec:referringphrasesheadedbynominalizedclauses} respectively. In \sectref{sec:rpscontainingrelativeclauses} the special characteristics of RPs containing relative clauses are discussed. Finally, in \sectref{sec:caseinreferringexpressions}, \sectref{sec:number}, and \sectref{sec:definiteness}, the different kinds of “inflectional” information expressed syntactically in the Referring Phrase will be discussed from a functional perspective.

\subsection{Basic Referring Phrases}
\label{sec:basicreferringphrases}

The following is a rough template of the Kagayanen basic Referring Phrase. There is potential complexity within some of the elements displayed in \REF{bkm:Ref340905081}, as well as some co-occurrence restrictions and possible variations in order. These will be discussed in the following paragraphs and subsections.

\ea
\label{bkm:Ref340905081} \label{ex:rptemplate}
RP = CM (PL) (\{MOD\textsubscript{1} ({Є), DEM\} LK) (PL) HEAD (Є (LK MOD}\textsubscript{2})) (RP\textsubscript{GEN})
\z

In this template, CM refers to a set of case marking particles – zero for absolutive Referring Phrases and \textit{ta} for non-absolutive Referring Phrases headed by common nouns (RPs headed by personal names are discussed in \sectref{sec:referringphrasesheadedbypersonalnames}). The PL element before MOD\textsubscript{1} indicates indefinite plural (\ref{bkm:Ref363459525}) while PL before the head is unspecified\textsubscript{} for definiteness (\ref{bkm:Ref414967349} and \ref{bkm:Ref414967351}). PL refers to the plural/collective particle pronounced [maŋá] but spelled \textit{mga} following common practice in the Philippines for languages with similar particles (e.g., Tagalog [máŋa]). Both plural markers are optional even when the head is semantically plural (see example \ref{bkm:Ref363459636}).

MOD\textsubscript{1} and MOD\textsubscript{2} are two positions where potentially complex modifying elements may occur. The symbol {Є} indicates an enclitic complex consisting of a pronominal genitive enclitic and/or a demonstrative determiner (see below for a discussion of these categories). 

When Є occurs, it is in the second position in MOD\textsubscript{1}. Thus, if MOD\textsubscript{1} consists of a single word, the Є complex immediately follows that word (see examples \ref{bkm:Ref447702060}{}--\ref{bkm:Ref447702062}). When MOD\textsubscript{1} is itself complex (e.g., a Modifier Phrase (see \chapref{chap:modification}) or a relative clause (\sectref{sec:rpscontainingrelativeclauses}), Є usually occurs after the first element. It is only when the modifier phrase includes an intensity adverb such as \textit{sikad} ‘very’ initially that Є does not occur in second position. Rather, the  intensity adverb and the adjective occur together and the genitive enclitic follows them. But when no MOD\textsubscript{1} is present, both elements of the optional enclitic (Є) complex follow the head.

As mentioned above, Є consists of a genitive enclitic pronoun and/or a demonstrative determiner. Interestingly, the genitive enclitic precedes the demonstrative determiner for all categories except 3rd person plural, \textit{danen}. This may be because \textit{danen} is the only multisyllabic genitive enclitic: 
\ea
MOD \hspace{12pt} Є \hspace{8pt} LK  HEAD \\
\glll bakod \textbf{ko} \textbf{i} na baļay. \textnormal{‘my big  house (near me)’} \\
\textit{bakod} \textbf{\textit{no}} \textbf{\textit{an}} \textit{na} \textit{baļay} {‘your big house (near you)’}	 \\ 
\textit{bakod} \textbf{\textit{din}} \textbf{\textit{ya}} \textit{na} \textit{baļay}  {‘her/his big house (far away)’} \\
\glt etc.
\z

\ea
But: \\
MOD \hspace{12pt}   Є \hspace{22pt}  LK  HEAD \\
\glll bakod \textbf{i} \textbf{danen} na baļay. \textnormal{‘their big  house (near me)’} \\
\textit{bakod} \textbf{\textit{an}} \textbf{\textit{danen}} \textit{na} \textit{baļay}. {‘their big house (near you)’} \\ 
\textit{bakod} \textbf{\textit{ya}} \textbf{\textit{danen}} \textit{na} \textit{baļay} {‘their big house (far away)’} \\
\glt {}
\z

As indicated by the curly brackets (\{ \}) in \REF{bkm:Ref340905081}, MOD\textsubscript{1} plus the optional enclitic complex is in complementary distribution with a demonstrative adjective; both are “optional” and both are followed by the “linker” (LK) \textit{na}. RP\textsubscript{gen} refers to another Referring Phrase functioning as a “possessor” of the Head.

\tabref{tab:rptemplate} presents some examples of basic Referring Phrases (RPs headed by common nouns) with varying configurations of elements.

\begin{sidewaystable}
\caption{Basic Referring Phrases in Kagayanen}
\label{tab:rptemplate}
\fittable{
\begin{tabular} {
    p{.8cm}   %CM
    p{.7cm}   %PL
    p{3cm}    %MOD/LK/DEM
    p{.7cm}   %PL
    p{1.4cm}  %Head
    p{1.6cm}  %Є
    p{1.6cm}  %LK MOD
    p{2cm}    %RPgen
    >{\RaggedRight\arraybackslash}p{4.8cm}}          %Translation
\lsptoprule
CM & PL & 
$\displaystyle \biggr \lbrace {\textrm{MOD\textsubscript{1}  Є  LK} \atop \textrm{DEM}} \biggr \rbrace$
 & PL & HEAD & {Є} & LK MOD\textsubscript{2} & RP\textsubscript{GEN} & \\
\midrule
0 &  &  & mga & baļay
house & i &  &  & ‘the houses’ (absolutive, close to speaker) \\
\midrule
ta \newline
\textsc{nabs} &  &  & mga & baļay & i &  &  & ‘the houses’ (non-absolutive, close to speaker) \\
\midrule
0 &  & bakod an na \newline
big \hspace{14pt}Є \hspace{1pt}  \textsc{lk} &  & baļay &  &  &  & ‘the big house’ (absolutive, not far away) \\
\midrule
0 &  &  &  & baļay & an \newline
 & na bakod \newline
\textsc{lk} big &  & ‘the big house’ (absolutive, not far away) \\
\midrule
0 &  &  &  & baļay & an danen \newline
|\hspace{12pt} {Є} \hspace{10pt} | & na bakod \newline
 &  & ‘their big house’ (absolutive, not far away) \\
\midrule
0 &  & bakod an danen na \newline 
big \hspace{12pt}|\hspace{12pt} {Є} \hspace{10pt} | \textsc{lk} &  & baļay &  &  &  & ‘their big house’ (absolutive, not far away) \\
\midrule
0 &  &  &  & baļay &  &  & ta \hspace{12pt} mama \newline
\textsc{nabs} man & ‘the man’s house’ (absolutive) \\
\midrule
ta \newline
\textsc{nabs} &  & bakod na & mga & baļay &  & na daan \newline
\textsc{lk} old &  & ‘the big old houses’ (non-absolutive) \\
\midrule
0 &  & datas gid ya na \newline
high \hspace{1pt} \textsc{int} {Є}  \textsc{lk} &  & bungyod \newline
hill &  &  &  & ‘the truly high hill' (absolutive) \\
\midrule
0 &  & uyi na &  & baļay &  &  &  & ‘this house’ (absolutive) \\
\midrule
0 & mga & bakod na &  & baļay &  &  &  & ‘some big houses’ (absolutive) \\
\lspbottomrule
\end{tabular}
}
\end{sidewaystable}
The MOD\textsubscript{1} position may contain Modifier Phrases (\chapref{chap:modification}, \sectref{sec:modifierphrases}), Numeral Phrases (\chapref{chap:modification}, \sectref{bkm:Ref418107822}), quantifiers and free genitive pronouns (\sectref{sec:personalpronouns}), and nominalized verbs or “short” relative clauses (those having no more than one overt argument, which must be pronominal). The MOD\textsubscript{2} position may also contain Modifier Phrases and relative clauses, but less commonly numerals or quantifiers. “Long” relative clauses (those having one overt full RP argument, or more than one overt argument, either pronominal or full RP), may only occur in MOD\textsubscript{2}. When two modifiers simultaneously modify the same head, usually one is in MOD\textsubscript{1} and the other in MOD\textsubscript{2} or there are two MOD\textsubscript{2} positions (see examples \ref{bkm:Ref363459636}{}-\ref{bkm:Ref447702340}). RPs with more than two modifiers modifying a single head are not common but do occur. Usually there can be only one MOD\textsubscript{1}, but there can be up to three MOD\textsubscript{2} elements. When a RP has three modifiers then the linker in one of the linker positions is likely to drop out. Occasionally two MOD\textsubscript{1} elements occur, in which case the first must be a numeral and the second an adjective (\ref{bkm:Ref363460123}), but this also is not common.

Functionally, modifiers in MOD\textsubscript{1} tend to be non-restrictive, meaning they add important detail about a referent that is sufficiently identified even without the modifier. Modifiers in MOD\textsubscript{2}, on the other hand, tend to be restrictive, meaning they delimit the scope of the RP to a particular referent. A full discourse study of the functions of modifiers in these two positions would be necessary to confirm or refute this observation.

The following are a few additional examples of basic RPs from the corpus, with labels indicating the functions of the individual parts, as referenced in the template in \REF{bkm:Ref340905081}:

\ea
\label{bkm:Ref363459525}
ta  mga  bakod  na  baļay \\\smallskip

CM  \hspace{6pt} PL \hspace{4pt}  MOD\textsubscript{1} \quad    Head \\
\gll ta  mga  bakod  na  baļay\footnotemark{} \\
\textsc{nabs}  \textsc{pl}  big  \textsc{lk}  house \\
\footnotetext{The property concept words \textit{bakod} ‘big’ and \textit{sise/tiset} ‘small’ may refer to singular or plural items. There are also specialized plural forms of these words---\textit{darko} ‘big.\textsc{pl}’ (see the following three examples) and \textit{derset/derse} ‘small.\textsc{pl}’. This is discussed in more detail in \sectref{sec:number} on number in the Referring Phrase.}
\glt ‘some big houses’ [BMON-C-05 13.9]
\z
\ea
\label{bkm:Ref414967349}
ta  biskeg  na  angin  daw  darko  na  mga  baļed \\\smallskip

CM  \hspace{4pt} MOD\textsubscript{1} \hspace{12pt}  Head \hspace{32pt}   MOD\textsubscript{1} 
\hspace{10pt}   PL  \hspace{2pt} Head \\
\gll ta  biskeg  na  angin  daw  darko  na  mga  baļed \\
\textsc{nabs}  strong  \textsc{lk}  wind/air  and  big.\textsc{pl}  \textsc{lk}  \textsc{pl}  wave \\
\glt ‘strong wind and big waves’ [VPWE-T-01]
\z
\ea
\label{bkm:Ref414967351}
ta  mga  darko  na   mga  kaoy \\\smallskip

CM \hspace{6pt} PL \hspace{4pt} MOD\textsubscript{1} \hspace{8pt}  PL \hspace{4pt} Head \\
\gll ta  mga  darko  na   mga  kaoy \\
\textsc{nabs}  \textsc{pl}  big.\textsc{pl}  \textsc{lk}  \textsc{pl}  tree \\
\glt ‘some big trees’ [JCWE-T-15 4.4]
\z
\ea
\label{bkm:Ref363459636}
\emptyset{}  darko  na  daon  na  ļangkaw  na  nipis \\\smallskip

CM  MOD\textsubscript{1} \hspace{10pt}   Head \hspace{8pt}   MOD\textsubscript{2} \hspace{18pt}   MOD\textsubscript{2} \\
\gll \emptyset{}  darko  na  daon  na  ļangkaw  na  nipis \\
\textsc{abs}  big.\textsc{pl}  \textsc{lk}  leaf  \textsc{lk}  long  \textsc{lk}  thin \\
\glt ‘big long thin leaves’ [JCWE-T-14 5.2]
\z
\ea
bakod  na  baļay  na  sikad  dayad \\\smallskip

CM  MOD\textsubscript{1} \hspace{9pt}   Head \hspace{13pt}   MOD\textsubscript{2} \\
\gll \emptyset{}  bakod  na  baļay  na  sikad  dayad \\
\textsc{abs}  big  \textsc{lk}  house  \textsc{lk}  very  good \\
\glt ‘a/the big, very nice house’ [PBWN-C-12 14.6]
\z
\ea
\emptyset{}  daon  na  ļangkaw  na  grin \\\smallskip

CM  Head \hspace{8pt}   MOD\textsubscript{2} \hspace{18pt}   MOD\textsubscript{2} \\
\gll \emptyset{}  daon  na  ļangkaw  na  grin \\
\textsc{abs} leaf  \textsc{lk}  long  \textsc{lk}  green \\
\glt ‘a/the long green leaf’ [DBOE-C-04 1.1]
\z
\ea
ta  isya  na  kaoy  na  bakod  na  may  lungag  naan  ta  puon \\\smallskip

CM \hspace{4pt} MOD\textsubscript{1} \hspace{2pt}   Head \hspace{6pt}   MOD\textsubscript{2} \hspace{8pt} Relative clause \\
\gll ta  isya  na  kaoy  na  bakod  na  may  lungag  naan  ta  puon \\
\textsc{nabs}  one  \textsc{lk}  tree  \textsc{lk}  big  \textsc{lk}  \textsc{ext.in}  hole  \textsc{spat.def}  \textsc{nabs}  trunk \\
\glt ‘one big tree having a hole in its trunk’ [CBWN-C-22 4.3]
\z
\ea
\label{bkm:Ref447702340}
\emptyset{}  sise  na  lugar  na  nalibutan  ta  dagat \\\smallskip

CM  MOD\textsubscript{1} \hspace{6pt} Head \hspace{12pt}   Relative Clause \\
\gll \emptyset{}  sise  na  lugar  na  na-libut-an  ta  dagat \\
\textsc{abs} small  \textsc{lk}  place  \textsc{lk}  \textsc{a.hap.r}-around-\textsc{apl}  \textsc{nabs}  sea \\
\glt ‘a/the small place surrounded by the sea’ [VAWL-T-14 8.15]
\z
\ea
\emptyset{}  yi  na  bai  na  sikad  gwapa \\\smallskip

CM  DEM \hspace{13pt}   Head \hspace{22pt}   MOD\textsubscript{2} \\
\gll \emptyset{}  yi  na  bai  na  sikad  gwapa \\
\textsc{abs} \textsc{d}1\textsc{abs}  \textsc{lk}  woman  \textsc{lk}  very  attractive \\
\glt ‘this very attractive woman’ [PBWN-C-12 19.2]
\z
\ea
\label{bkm:Ref363460123}
ta  pitto  na  mga  maiseg  na  mga  mama  na  mag-panguļo  ta  grupo \\\smallskip

CM \hspace{4pt} MOD\textsubscript{1} \hspace{8pt} PL \hspace{4pt} MOD\textsubscript{1}  \hspace{29pt}  PL \hspace{7pt} Head \hspace{14pt}   MOD\textsubscript{2} (Relative cl.) \\
\gll ta  pitto  na  mga  ma-iseg  na  mga  mama  na  mag-panguļo  ta  grupo \\
\textsc{nabs}  seven  \textsc{lk}  \textsc{pl}  \textsc{abs}-brave  \textsc{lk}  \textsc{pl}  male  \textsc{lk}  \textsc{i.ir}-leader  \textsc{nabs}  group \\
\glt ‘seven brave men who are leaders of groups’ [JCWN-T-20 16.7]
\z
\ea
\label{bkm:Ref447702060}
ta  isya  na  bai  na  ame  man  na  katagsa \\\smallskip

CM \hspace{4pt} MOD\textsubscript{1} \hspace{1pt} Head \hspace{19pt} MOD\textsubscript{2} (Relative Clause) \\
\gll ta  isya  na  bai  na  ame  man  na  katagsa \\
\textsc{nabs}  one  \textsc{lk}  woman  \textsc{lk}  1\textsc{p.excl.gen}  too  \textsc{lk}  cousin \\
\glt ‘one woman who is also our cousin’ [RZWN-T-02 2.3]
\z
\ea
tallo  ko  nang  na  mga  mangngod  na  mga  derset \\\smallskip

CM  MOD\textsubscript{1}  {Є}  \hspace{50pt}     PL \hspace{4pt}  Head \hspace{58pt}   MOD\textsubscript{2} (Relative cl.) \\
\gll \emptyset{} tallo  ko  nang  na  mga  mangngod  na  mga  derset \\
\textsc{abs} three  1\textsc{s.gen}  only  \textsc{lk}  \textsc{pl}  younger.sibling  \textsc{lk}  \textsc{pl}  small.\textsc{pl} \\
\glt ‘just my three younger siblings who are small (ones)’ [ETON-C-07 1.3]
\z
\ea
ta  mga  mama  nay  ya  na  duma \\\smallskip

CM \hspace{4pt} PL \hspace{6pt} MOD\textsubscript{1} {Є} \hspace{80pt}       Head \\
\gll ta  mga  mama  nay  ya  na  duma \\
\textsc{nabs} \textsc{pl}  male  1\textsc{p.excl.gen}  \textsc{def.f}  \textsc{lk}  companion \\
\glt ‘our male companions’ [EMWN-T-09 6.1]
\z
\ea
\label{bkm:Ref447702062}
Darwa  ko  ya  na  mga  utod \\\smallskip

CM  MOD\textsubscript{1}  {Є}  \hspace{58pt}    PL \hspace{3pt} Head \\
\gll \emptyset{} darwa  ko  ya  na  mga  utod \\
\textsc{abs} two  1\textsc{s.gen}  \textsc{def.f}  \textsc{lk}  \textsc{pl}  sibling \\
\glt ‘my two siblings’ [PBWN-C-13 8.3]
\z
The RP\textsubscript{gen} position may be filled by any genitive case RP, bolded in examples \REF{bkm:Ref447702935}-\REF{bkm:Ref363473619}. Only one expression of a possessor/genitive element modifying the head may occur in any RP. In other words, if a full RP\textsubscript{gen} occurs, enclitic genitives are precluded.

\ea
\label{bkm:Ref447702935}
lieg  ya  \textbf{ta}  \textbf{bata}  \textbf{din} \\\smallskip
 \gll lieg  ya  \textbf{ta}  \textbf{bata}  \textbf{din} \\
neck  \textsc{def.f}  \textsc{nabs}  child  3\textsc{s.gen} \\
\glt ‘the neck \textbf{of} \textbf{his} \textbf{child}’ [PBWN-C-12 6.8]
\z
\ea
ta  isya  na  bakod  na  bunga  \textbf{ta}  \textbf{isya}  \textbf{na}  \textbf{buļak} \\\smallskip
 \gll ta  isya  na  bakod  na  bunga  \textbf{ta}  \textbf{isya}  \textbf{na}  \textbf{buļak} \\
\textsc{nabs}  one  \textsc{lk}  big  \textsc{lk}  fruit  \textsc{nabs}  one  \textsc{lk}  flower \\
\glt ‘one big fruit \textbf{of} \textbf{one} \textbf{flower} \textbf{(plant)}’ [VPWE-T-02 4.1]
\z
\ea
\label{bkm:Ref363473619}
ta  mga  ginikanan  man  \textbf{ta}  \textbf{mga}  \textbf{bag-o}  \textbf{tao}  \textbf{na}  \textbf{bata} \\\smallskip
 \gll ta  mga  ginikanan  man  \textbf{ta}  \textbf{mga}  \textbf{bag-o}  \textbf{tao}  \textbf{na}  \textbf{bata} \\
\textsc{nabs}  \textsc{pl}  parent  too  \textsc{nabs}  \textsc{pl}  new  born  \textsc{lk}  child \\
\glt ‘also parents \textbf{of} \textbf{newborn} \textbf{children}’ [JCWE-T-15 2.2]
\z

MOD\textsubscript{2} does not co-occur with RP\textsubscript{gen} in the corpus for this study. This is an argument that RP\textsubscript{gen} may be considered one possible instantiation of MOD\textsubscript{2}. However, because MOD\textsubscript{2} + RP\textsubscript{gen} constructions are (marginally) acceptable to some native speakers, and because the kinds of information in the MOD\textsubscript{2} and RP\textsubscript{gen} positions are semantically so distinct, we have chosen to treat the two positions as distinct.

Finally, free genitive pronouns can occur in MOD\textsubscript{1} position (\ref{bkm:Ref447707029}{}-\ref{bkm:Ref343925554}).


\ea
\label{bkm:Ref447707029}
\textbf{ake}  \textbf{ya}  na  duma \\\smallskip
 \gll \textbf{ake}  \textbf{ya}  na  duma \\
1\textsc{s.gen}  \textsc{def.f}  \textsc{lk}  companion \\
\glt ‘\textbf{my} companion’ [MFWN-T-04 3.5]
\z
\ea
\textbf{iran}  \textbf{nang}  na  kaugalingen \\\smallskip
 \gll \textbf{iran}  \textbf{nang}  na  kaugalingen \\
3\textsc{p.gen}  only  \textsc{lk}  self \\
\glt ‘only themselves’ [MOOE-C-01 187.1]
\z

Pronominal genitives in MOD\textsubscript{1} often express a contrastive or emphatic sense, as in the second conjunct (RP2) of example \REF{bkm:Ref343925554}:

\ea
\label{bkm:Ref343925554}
bayo    ta  iya  na  mga  bataan  daw    iya  man  na  bayo \\\smallskip

[Head  [\hspace{36pt}         MOD\textsubscript{2} \hspace{46pt}  ]]\textsubscript{RP1} \hspace{6pt} [[MOD\textsubscript{1}] \hspace{22pt}         Head ]\textsubscript{RP2} \\
\gll bayo    ta  iya  na  mga  bata-an  daw    iya  man  na  bayo \\
clothes    \textsc{nabs}  3\textsc{s.gen}  \textsc{lk}  \textsc{PL}  child-\textsc{nr}  and    3\textsc{s.gen}  too  \textsc{lk}  clothes \\
\glt ‘clothes of her children and HER clothes too’ [ECOE-C-01]
\z

Demonstrative adjectives can occur in MOD\textsubscript{1} \REF{bkm:Ref343578428}-\REF{singlewoman}, or in MOD\textsubscript{2} position \REF{bkm:Ref360182922}-\REF{bkm:Ref363547504}. In the following examples, the RP containing a demonstrative is bolded in the text and in its free translation:

\newpage
\ea
\label{bkm:Ref343578428}
\textbf{Yi}  \textbf{na}  \textbf{manakem}  lipo  na  tise,  piro  tama  nadlek  ki kanen. \\\smallskip
 \gll \textbf{Yi}  \textbf{na} \textbf{manakem}  lipo  na  tise,  piro  tama  na-adlek  ki kanen. \\
\textsc{d}1\textsc{abs}  \textsc{lk}  older  short  \textsc{lk}  small  but  many \textsc{a.hap.r}-afraid  \textsc{obl.p} 3s \\
\glt ‘\textbf{This} \textbf{older} \textbf{person} is short and small, but many are afraid of her.’ [MBON-C-02 2.8]
\z
\ea
Piro  \textbf{yi}  \textbf{na}  \textbf{mga}  \textbf{kabaw}  sikad  niwang…. \\\smallskip
 \gll Piro  \textbf{yi}  \textbf{na}  \textbf{mga}  \textbf{kabaw}  sikad  niwang…. \\
but  \textsc{d}1\textsc{abs}  \textsc{lk}  \textsc{pl}  water.buffalo  very  skinny \\
\glt ‘But \textbf{these} \textbf{water} \textbf{buffalo} were very skinny…’ [PBWN-C-13 10.3]
\z
\ea
Tapos,  dayon  uli  mama  i  daw  itanem  \textbf{yon}  \textbf{na}  \textbf{buļak} naan  ta  may  tumbuan  din. \\\smallskip
 \gll Tapos,  dayon  uli  mama  i  daw  i-tanem  \textbf{yon}  \textbf{na}  \textbf{buļak} naan  ta  may  tumbuan  din. \\
then  right.away  go.home  man  \textsc{def.n}  and  \textsc{t.deon}-plant  \textsc{d}3\textsc{abs}  \textsc{lk}  flower \textsc{spat.def}  \textsc{nabs}  \textsc{ext.in}  window  3\textsc{s.gen} \\
\glt `Then the man went home immediately and  planted \textbf{that} \textbf{flower} where his window was.’ [CBWN-C-13 5.7]
\z
\ea
Tak  uļa  a  gapati  ta  \textbf{yon}  \textbf{na}  \textbf{mga}  \textbf{ambaļ,} gagwa  a  gid. \\\smallskip
 \gll Tak  uļa  a  ga-pati  ta  \textbf{yon}  \textbf{na}  \textbf{mga}  \textbf{ambaļ,} ga-gwa  a  gid. \\
because  \textsc{neg.r}  1\textsc{s.abs}  \textsc{i.r}-believe/obey  \textsc{nabs}  \textsc{d}3\textsc{abs}  \textsc{lk}  \textsc{PL}  say
\textsc{i.r-}out  1\textsc{s.abs}  \textsc{int} \\
\glt `Because I did not believe/obey \textbf{those} \textbf{words}, I really went out.’ [VAWN-T-16 2.8]
\z
\ea
\label{singlewoman}
… dļaga  ya  gaangep  en  nakem  din  ta  \textbf{yo}  \textbf{na} \textbf{mama}  daw  galiagay  danen. \\\smallskip
 \gll … dļaga  ya  ga-angep  en  nakem  din  ta  \textbf{yo}  \textbf{na} \textbf{mama}  daw  ga-liag-ay  danen. \\
{} single.woman  \textsc{def.f}  \textsc{i.r}-favor  \textsc{cm}  innerself  3\textsc{s.gen}  \textsc{nabs}  \textsc{d}4\textsc{abs}  \textsc{lk} man  and  \textsc{i.r}-like-\textsc{rec}  3\textsc{p.abs} \\
\glt `... the single woman her inner self began to favor \textbf{that} \textbf{man} and they began to like each other.’ [CBWN-C-13 5.4]
\z

When a demonstrative occurs in MOD\textsubscript{1}, an intensity, clausal or aspectual adverb such as \textit{man} ‘also’ may occur after the demonstrative \REF{bkm:Ref342913658}.\footnote{We gloss the adverb \textit{man} as ‘also’, which adequately captures its main function. However, depending on the context it may express the sense of ‘again’ or ‘same’ (as in example \ref{bkm:Ref342913658}). In other contexts, it may evoke excitement or strong feelings. The same form functions as a conjunction joining two RPs, two clauses or even higher level units. In that usage, it expresses none of the emotional overtones that accompany its adverbial usages.}  Other words, including the enclitic demonstrative determiner, numerals, adjectives, free genitive pronouns or enclitic genitive pronouns do not co-occur with demonstratives in the same Referring Phrase (compare \ref{bkm:Ref447708043} and the ungrammatical or extremely awkward examples in \ref{bkm:Ref116485548}):
\ea
\label{bkm:Ref342913658}
Ta  \textbf{yo}  \textbf{man}  \textbf{na}  \textbf{adlaw}  sikad  kay  sadya. \\\smallskip
 \gll Ta  \textbf{yo}  \textbf{man}  \textbf{na}  \textbf{adlaw}  sikad  kay  sadya. \\
\textsc{nabs}  \textsc{d}4\textsc{abs}  too  \textsc{lk}  day  very  1\textsc{p.excl.abs}  fun/enjoy \\
\glt ‘On \textbf{that} \textbf{same} \textbf{day} we were very happy.’ [EMWN-T-09 9.2]
\z
\ea
\label{bkm:Ref447708043}
Yi  man  na  bayo \\\smallskip
 \gll Yi  man  na  bayo \\
\textsc{d}41\textsc{abs}  too  \textsc{lk}  clothes/shirt \\
\glt  ‘this shirt too’
\z
\ea
\label{bkm:Ref116485548}
\ea {\itshape *yi na bayo i           } (‘this shirt’) \\
\ex {\itshape *yi na darwa na bayo    }(‘these two shirts’) \\
\ex {\itshape *yi na bayo na darwa    }(‘these two shirts’) \\
\ex {\itshape *yi ko na bayo          } (‘this my shirt’) \\
\ex {\itshape *yi na ake na bayo      }(‘this MY shirt’) \\
\ex {\itshape *yi na bayo ko          } (‘this my shirt‘) \\
\ex {\itshape *yi na sise na bayo     }  (‘this small shirt') \\
\ex {\itshape *yi na bayo na sise     }  (‘this small shirt') \\
\ex {\itshape *yi na sise na bayo ko i}    (‘this my small shirt') \\
\ex {\itshape *yi na sise ko i na bayo}    (‘this my small shirt')
\z
\z

\largerpage
A demonstrative may occur in MOD\textsubscript{2} (\ref{bkm:Ref360182922}-\ref{bkm:Ref363547504}), rather than MOD\textsubscript{1}. An enclitic genitive pronoun, mostly first person singular \textit{ko}, and a postposed adjective may co-occur in the same Referring Phrase (\ref{bkm:Ref447709205} and \ref{bkm:Ref363547504}), but not a numeral, preposed adjective, free genitive pronoun or an enclitic demonstrative determiner (\ref{bkm:Ref447709215}). The linker \textit{na} does not occur between the postposed demonstrative and the head since the postposed demonstratives are contractions of the linker \textit{na} plus the basic demonstrative.

\ea
\label{bkm:Ref360182922}
Manakem  \textbf{naya}  kateļen. \\\smallskip
 \gll Manakem  \textbf{naya}  kateļ-en. \\
older  \textsc{d3abs}  itchy-\textsc{abs} \\
\glt ‘\textbf{That} older one was very itchy with sores.’ [AION-C-01 5.4]
\z
\ea
\label{bkm:Ref447709205}
Gumangken  ko  \textbf{nai}  gapadugang  gid  ta  abilidad  din… \\\smallskip
 \gll Gumangken  ko  \textbf{nai}  ga-pa-dugang  gid  ta  abilidad  din… \\
niece/nephew  1\textsc{s.gen}  \textsc{d1abs}  \textsc{i.r}-\textsc{caus}-add  \textsc{int}  \textsc{nabs}  ability  3\textsc{s.gen} \\
\glt `\textbf{This} nephew of mine really added to his ability…’ [JCWB-T-12 5.4]
\z
\ea
\label{bkm:Ref363547504}
mangngod  ko  \textbf{nai}  na  sise \\\smallskip
 \gll mangngod  ko  \textbf{nai}  na  sise \\
younger.sibling  1\textsc{s.gen}  \textsc{d1abs}  \textsc{lk}  small \\
\glt ‘\textbf{this} small younger sibling of mine’
\z
\ea
\label{bkm:Ref447709215}
a.  *darwa ko na mangngod nai       (‘these my two younger siblings’) \\
b.  *sise ko na mangngod nai        (‘this small my younger sibling’) \\
c.  *ake na mangngod nai        (‘this MY younger sibling’) \\
d.  *mangngod ko i nai        (‘this my younger sibling’)
\z

\subsection{Referring Phrases headed by personal names}
\label{sec:referringphrasesheadedbypersonalnames}

Personal names (proper names referring to human beings) require a different RP template than basic RPs, as follows:

\ea
RP\textsubscript{name} = CM\textsubscript{pn} (PRO\textsubscript{abs}) Head ({Є})
\z

The oblique Case Marker for personal names (CM\textsubscript{pn}) is \textit{ki}. The absolutive, ergative and genitive cases are unmarked. \textit{ki} contrasts with \textit{ta}, which marks all non-absolutive functions of common nouns (see \sectref{sec:basicreferringphrases}). The most obvious syntactic difference between basic RPs and RPs headed by personal names is that the latter do not allow modification or full RP possession. As in basic RPs, Є may contain either or both possessive and demonstrative determiners. However, personal names may only be followed by an enclitic possessor in the Є complex if they are being treated as common nouns. For example, one may say \textit{Maria ko} `my Maria' to distinguish one's own child named Maria from all the other Marias in the community. A plural absolutive pronoun optionally precedes the head is to refer to to the person named in addition to someone else. \tabref{tab:rppersonalnames} illustrates some personal RPs.

\begin{table}
\caption{Referring Phrases headed by personal names}
\label{tab:rppersonalnames}
\begin{tabularx}{\textwidth}{llllQ}
\lsptoprule
CM\textsubscript{PN} & PRO\textsubscript{ABS} & Head & {Є} & \\
\midrule
ki &  & Maria &  & ‘Maria’ (oblique) \\
ki & danen & Maria &  & ‘Maria and him/her/them’ (oblique) \\
\emptyset & kyo & Maria &  & ‘Maria and you/y’all’ (\textsc{abs/erg/gen})) \\
\emptyset &  & Maria & ya & ‘that Maria’ (far from speaker and hearer, \textsc{abs/erg/gen}) \\
\lspbottomrule
\end{tabularx}
\end{table}

 The following are some examples from the corpus. In these examples we include a ``zero" marker for the absolutive case marker in the RPs being illustrated, though this is not our practice throughout the grammar. 
\ea
Bantayan  no  \textbf{Pedro}  \textbf{an}  tak  manaw a  na  mamugon  ta  pagkaan  ta. \\\smallskip
 \gll \emptyset{}-Bantay-an  no   \emptyset {} \textbf{Pedro}  \textbf{an}  tak  m-anaw a  na  ma-mugon  ta  pagkaan  ta. \\
\textsc{t.ir}-watch-\textsc{apl}  2\textsc{s.erg}  \textsc{abs}  Pedro  \textsc{def.m}  because  \textsc{i.ir}-walk/go 1\textsc{s.abs}  \textsc{lk}  \textsc{a.hap.ir}-day.worker  \textsc{nabs}  food  1\textsc{p.incl.gen} \\
\glt `Watch \textbf{Pedro} (while I’m away) because I am leaving to work for our food.’ [MBON-C-01 2.3]
\z
\ea
Nabaton  nay imo  na  suļat  na  padaļa  no \textbf{ki}  \textbf{Maria}  ta  pitsa  23  ta  Oct. \\\smallskip
 \gll Na-baton  nay \emptyset{} imo  na  suļat  na  \emptyset{}-pa-daļa  no \textbf{ki}  \textbf{Maria}  ta  pitsa  23  ta  Oct. \\
\textsc{a.hap.r}-receive  1\textsc{p.excl.erg} \textsc{abs} 2\textsc{s.gen}  \textsc{lk}  letter  \textsc{lk}  \textsc{t.r-caus-}carry  2\textsc{s.erg} \textsc{obl.p}  Maria  \textsc{nabs}  date  23  \textsc{nabs}  October \\
\glt `We received your letter that you sent \textbf{with} \textbf{Maria} on the date of the 23\textsuperscript{rd} in October.’ [PBWL-C-04 3.1]
\z
\ea
\label{bkm:Ref53060834}
Kamusta  nang  en  \textbf{ki}  \textbf{danen}  \textbf{Maria}. \\\smallskip
 \gll Kamusta  nang  en  \textbf{ki}  \textbf{danen}  \textbf{Maria}. \\
greeting  only  \textsc{cm}  \textsc{obl.p}  \textsc{3p}  Maria \\
\glt ‘Greetings \textbf{to} \textbf{Maria} \textbf{and} \textbf{companions}.’ (The word \textit{nang} ‘only’ in this example indicates politeness.) [PBWL-C-05 6.1]
\z


\largerpage
\ea
\label{bkm:Ref343578640}
Na abot  kay  en  naan  ta  baļay, gatingaļa  en  \textbf{danen}  \textbf{Papa}  tak  ubos  kay  ta  basa. \\\smallskip
 \gll Na\footnotemark{}  \emptyset{}-abot  kay  en  naan  ta  baļay, ga-tingaļa  en  \textbf{danen}  \textbf{Papa}  tak  ubos  kay  ta  basa. \\
\textsc{lk}  \textsc{i.r-}arrive  1\textsc{p.excl.abs}  \textsc{cm}  \textsc{spat.def}  \textsc{nabs}  house \textsc{i.r-}wonder  \textsc{cm}  3\textsc{p.abs}  Papa  because  all  1\textsc{p.excl.abs}  \textsc{nabs}  wet \\
\footnotetext{This use of the linker to introduce dependent clauses is discussed in \chapref{chap:clausecombining}.}
\glt `When we arrived at the house, \textbf{Papa} \textbf{and} \textbf{companions} were wondering because all of us were wet.’ CBWN-C-10 4.42
\z
\subsection{Referring Phrases headed by nominalized clauses}
\label{sec:referringphrasesheadedbynominalizedclauses}

Referring Phrases may be headed by nominalized verbs or clauses. The template for such Referring Phrases is the same as that of Basic Referring Phrases (see \sectref{sec:basicreferringphrases}), with two exceptions: the Head position may itself be syntactically complex, and the final RP\textsubscript{gen} may be described as RP\textsubscript{gen/obl}. These differences will be discussed below.

We divide this section into two parts: Nominalized clauses functioning as participant nominalizations (“headless relative clauses”), and those functioning as action nominalizations, meaning those that refer to the action itself, rather than any participants in the action. Selected examples are given in \tabref{tab:rpsheadedbynominalizedclauses}. Additional examples from the corpus are given below.

\begin{sidewaystable}
\caption{Referring Phrases headed by nominalized clauses}
\label{tab:rpsheadedbynominalizedclauses}
\fittable {
\begin{tabular} {
    p{.8cm}     %CM
    p{2.6cm}    %MOD/LK/DEM
    p{.7cm}     %PL
    >{\RaggedRight\arraybackslash}p{3cm}    %Head
    p{1.5cm}    %Є
    >{\RaggedRight\arraybackslash}p{1.6cm}    %LK MOD
    p{1.9cm}    %RPgen
    >{\RaggedRight\arraybackslash}p{5cm}}          %Translation
\lsptoprule
CM & 
$\displaystyle \biggr \lbrace {\textrm{MOD\textsubscript{1}  Є  LK} \atop \textrm{DEM}} \biggr \rbrace$
 & PL & Head & {Є} & LK MOD\textsubscript{2} & RP\textsubscript{GEN} & \\
\midrule
 0 &  & mga & lutu-ón \newline
 cook-\textsc{nr} & an danen \newline
                        &  &  & ‘the things they will cook’ (absolutive, not far away) \\
\midrule
 0 &  & mga & pa-palit \newline \textsc{t.r}-buy & nay an & na brato \newline
 \textsc{lk} cheap & & ‘the inexpensive things we bought’ (absolutive, not far away) \\
\midrule
0 &  & mga & ka-kita \newline
\textsc{i.exm}-see & an &  & ki \hspace{14pt}   kami \newline
\textsc{obl.p} \textsc{1}\textsc{p.exc} & ‘the ones who saw us’ (absolutive, not far away) \\
\midrule
0 &  & mga & ga-patay \newline
\textsc{i.r}-die &  &  & ta      amay  din  & ‘the ones who killed his father’ \\
\midrule
0 & tanan na \newline
all \hspace{12pt} \textsc{lk} &  & ga-ani \newline
\textsc{i.r}-harvest &  &  &  & ‘all the ones harvesting/who harvested’ (absolutive) \\
\midrule
ta \newline \textsc{nabs} & iran \hspace{6pt} na \newline 3\textsc{p.gen} \textsc{lk} &  & na-ani \newline
\textsc{a.hap.r}-harvest &  &  & & ‘their harvested  things’ (non-absolutive) \\
\midrule
0 &  & mga & dapat \hspace{4pt} imes-én \newline
should prepare-\textsc{nr} &  &  &  & ‘the things that should be prepared’ (absolutive) \\
\midrule
0 & tama \hspace{2pt} din \hspace{2pt} na \newline
many 3\textsc{s.gen} \textsc{lk} &  & imes-én \newline
prepare-\textsc{nr} &  &  &  & ‘his/her many things to prepare’ (absolutive) \\
\lspbottomrule
 \end{tabular}
    }
\end{sidewaystable}

\begin{sidewaystable}
\caption{Referring Phrases headed by nominalized clauses (cont.)}
\fittable {
\begin{tabular} {
    p{.8cm}     %CM
    p{2.6cm}    %MOD/LK/DEM
    p{.7cm}     %PL
    >{\RaggedRight\arraybackslash}p{3cm}    %Head
    p{1.3cm}    %Є
    >{\RaggedRight\arraybackslash}p{1.8cm}    %LK MOD
    p{2.6cm}    %RPgen
    >{\RaggedRight\arraybackslash}p{5cm}}          %Translation
\lsptoprule
CM & 
$\displaystyle \biggr \lbrace {\textrm{MOD\textsubscript{1}  Є  LK} \atop \textrm{DEM}} \biggr \rbrace$
 & PL & Head & {Є} & LK MOD\textsubscript{2} & RP\textsubscript{GEN} & \\
\midrule
0 &  &  & imes-én \newline
prepare-\textsc{nr} & din \newline
3\textsc{s.gen} & na tama \newline
\textsc{lk} many & & ‘his/her many things to prepare’ (absolutive) \\
\midrule
0 & una an na \newline
first {Є} \textsc{lk} &  & pag-pati \newline
\textsc{nr.act-}believe &  &  & ta mga ittaw\newline
& ‘the first belief of the people’ (absolutive) \\
\midrule
ta
\textsc{nabs} & dayad na \newline
 good \textsc{lk} &  & pag-ambaļ \newline
\textsc{nr.act}-say &  &  &ta English\newline
\textsc{nabs} & ‘good speaking of English’ (non-absolutive) \\
\midrule
0 & iran \hspace{8pt} na \newline
3\textsc{p.gen} \textsc{lk} &  & pag-pangamuyo \newline
\textsc{nr.act}-pray &  &  & ki \hspace{12pt} yaken \newline
\textsc{obl.p} 1s & ‘their praying for me’ (absolutive) \\
\midrule
0 & iya \hspace{10pt} na \newline
3\textsc{s.gen} \textsc{lk} &  & pag-tratar \newline
\textsc{nr.act}-treat &  &  & ki \hspace{12pt} yaken \newline
\textsc{obl.p} 1s & ‘his/her way of treating me’ (absolutive) \\
\midrule
0 &  &  & pag-dako \newline
\textsc{nr.act}-big & din an \newline
&  &  & ‘his/her growing up’ (absolutive) \\
\midrule
0 & primiro ko na \newline
first 1\textsc{s.gen} \textsc{lk} &  & pag-sakay \newline
\textsc{nr.act}-ride &  &  & ta iruplano \newline
\textsc{nabs} plane & ‘my first riding of an airplane’ (absolutive) \\
\midrule
0 &  &  & pag-kanta \newline
\textsc{nr.act}-sing & din & na dayad \newline
\textsc{lk} good &  & `his/her good singing' (absolutive) \\
\midrule
0 &  & mga & pag-basa \newline
\textsc{nr.act}-read &  &  & ta mga baluta \newline & ‘readings of the ballots' (absolutive) \\
\midrule
0 &  &  & pag-isab \newline
\textsc{nr.act}-again & ko \newline 1\textsc{s.gen} & na dili igo ta ambaļ ta na \newline
 Kagayanen & & ‘my changing what is not correct in our Kagayanen language' (absolutive) \\
\lspbottomrule
 \end{tabular}
    }
\end{sidewaystable}
\footnotetext{The phrase \textit{brato nang} may be considered a lexicalized modifier meaning “inexpensive”, though the word \textit{brato} (from Spanish \textit{barato}) alone may also express this meaning in certain contexts.}

\largerpage
\subsubsection{Participant nominalizations}
\label{sec:participantnominalizations}

The head of a Referring Phrase can be a verb root inflected with certain Transitivity/Modality (TM) affixes (see \chapref{chap:verbstructure}, \sectref{sec:verbinflection} on verb inflection), to form \textit{participant nominalizations}\is{participant nominalizations} \citep[225]{payne1997}. In all such nominalizations, the resulting noun refers to the participant of the activity that would be the absolutive of the verb, given the TM affixes present. Example \REF{bkm:Ref117494864} lists participant nominalizations based on the root \textit{luto} “cook”. In-context examples from the corpus are given further below.


\ea
\label{bkm:Ref117494864}
\begin{tabbing}·
\textbf{TM Value} \hspace{1cm}\= \textbf{Nominalization} \= \textbf{Meaning} \\
 ga- \textsc{i.r} \> galuto   \> ‘someone who cooks’ \\
mag- \textsc{i.ir} \> magluto \> ‘someone who will cook’ \\
na- \textsc{a.hap.r} \> naluto \>   ‘something already cooked’ \\
\> \>‘something accidentally cooked’ \\
\> \> ‘something that got cooked’ \\
\emptyset{}-...-an\> lutuan    \> ‘thing in which something is cooked’/ \\
\textsc{t.r-...-apl}   \> \>‘place of cooking’
\end{tabbing}
\z

Transitive realis and irrealis participant nominalizations are formed with the dedicated nominalizers (\textit{{}<in>} and \textit{{}-én} respectively) rather than with Inflected Verb forms (see \sectref{sec:in} and \sectref{sec:en}). 

Since participant nominalizations may have arguments, some linguists may describe them as (headless) relative clauses (see \chapref{chap:clausecombining}, \sectref{sec:relativeclauses}). The ambitransitive affixes (\textit{na}{}- and \textit{ma}-) create a noun that refers to the S or O argument of the verb (\textit{na}- twice in example \ref{bkm:Ref447709957} and once in example \ref{bkm:Ref447710237}). Intransitive affixes create a noun that refers to the S (\textit{ga}{}- in examples \ref{bkm:Ref447709957}, \ref{bkm:Ref417973067}, \ref{bkm:Ref417973212}, and \textit{mag}{}- in \ref{bkm:Ref447710237}). Finally, the applicative plus transitive affixation forms a noun that refers to an applied O argument (\ref{bkm:Ref363550595}).
\ea
\label{bkm:Ref447709957}
Daw  oras  ta  lungi ta  ummay  o  mais,  \textbf{tanan} \textbf{na} \textbf{naani} dļeen  ta  baļay  tag-iya  daw  magtipon  \textbf{tanan}  \textbf{na}  \textbf{mga}  \textbf{gaani} daw  magsin-ad  ta  iran  na  \textbf{naani}. \\\smallskip
 \gll Daw  oras  ta  lungi\footnotemark{}  ta  ummay  o  mais,  \textbf{tanan} \textbf{na} \textbf{na-ani} daļa-en  ta  baļay  tag-iya  daw  mag-tipon  \textbf{tanan}  \textbf{na}  \textbf{mga}  \textbf{ga-ani} daw  mag-sin-ad  ta  iran  na  \textbf{na-ani}. \\
if/when  time/hour  \textsc{nabs}  harvest  \textsc{nabs}  unmilled.rice  or  corn  all  \textsc{lk} \textsc{a.hap.r}-harvest take-\textsc{t.ir}  \textsc{nabs}  house  owner  and  \textsc{i.ir-}gather  all  \textsc{lk}  \textsc{pl} \textsc{i.r-}harvest
and  \textsc{i.ir-}cook.grain  \textsc{nabs}  3\textsc{p.gen}  \textsc{lk}  \textsc{a.hap.r}-harvest \\
\footnotetext{The verb \textit{lungi} refers to collecting ears of corn, which doesn’t really collocate with \textit{ummay} ‘unmilled rice’. It seems the author made a mistake here, and first said \textit{lungi ta ummay} ‘rice’ and then repaired by adding \textit{o mais} ‘or corn.’ Rice does not grow in Cagayancillo, so corn is the staple crop.}
\glt `When it’s time to harvest rice or corn, \textbf{all} \textbf{that} \textbf{was} \textbf{harvested} is taken to the house of the owner and \textbf{all} \textbf{the} \textbf{ones} \textbf{who} \textbf{were} \textbf{harvesting} will gather and will cook some of \textbf{their} \textbf{harvested} \textbf{(grain/crop)}.’ [VAOE-J-07 3.1]
\z

\ea
\label{bkm:Ref417973067}
Pag-abot  nay  \textbf{ta}  \textbf{gaubra}  \textbf{ta}  \textbf{karsada}  dayon  din angad  operator  ya  daw  ambaļen  ta  Visaya  na  ambaļ... \\\smallskip
 \gll Pag--abot  nay  \textbf{ta}  \textbf{ga-ubra}  \textbf{ta}  \textbf{karsada}  dayon  din angad  operator  ya  daw  ambaļ-en  ta  Visaya  na  ambaļ... \\
\textsc{nr.act}-arrive  1\textsc{p.excl.gen}  \textsc{nabs}  \textsc{i.r}-work  \textsc{nabs}  road  right.away  3\textsc{s.erg}
look.up  operator  \textsc{def.f}  and  say-\textsc{t.ir}  \textsc{nabs}  Visayan  \textsc{lk}  say \\
\glt `When we arrived at \textbf{the} \textbf{one} \textbf{working} \textbf{on} \textbf{the} \textbf{road}, then he looked up at the operator (of the construction machinery) and said in the Visayan language…’ [EFWN-T-09 3.1]
\z
\ea
\label{bkm:Ref447710237}
Uļa  en  \textbf{magdamag}  \textbf{ta}  \textbf{mga}  \textbf{napatay}. \\\smallskip
 \gll Uļa  en  \textbf{mag-damag}  \textbf{ta}  \textbf{mga}  \textbf{na-patay} \\
\textsc{neg.r}  \textsc{cm}  \textsc{i.ir}-wake  \textsc{nabs}  \textsc{PL}  \textsc{a.hap.r}-die \\
\glt ‘There was \textbf{no} \textbf{one} \textbf{who} \textbf{will} \textbf{attend} \textbf{the} \textbf{wake} of \textbf{the} \textbf{ones} \textbf{who} \textbf{had} \textbf{died}.’ [JCWN-T-21 12.15]
\z
\ea
\label{bkm:Ref417973212}
Daw  mga  pamilya  \textbf{ta}  \textbf{napatay}  \textbf{an}  gasake  ta  imes \textbf{ta}  \textbf{mga}  \textbf{ubraén}  paryas  ta  kaoy  na  buaten  na  kabaong. \\\smallskip
 \gll Daw  mga  pamilya  \textbf{ta}  \textbf{na-patay}  \textbf{an}  ga-sake  ta  imes \textbf{ta}  \textbf{mga}  \textbf{ubra-én}  paryas  ta  kaoy  na  buat-en  na  kabaong. \\
and  \textsc{pl}  family  \textsc{nabs}  \textsc{a.hap.r}\textsc{-}die  \textsc{d}\textsc{ef.m}  \textsc{i.r}-busy  \textsc{nabs}  prepare
\textsc{nabs}  \textsc{PL}  work-\textsc{nr}  like  \textsc{nabs}  wood  \textsc{lk}  make-\textsc{t.ir}  \textsc{lk}  coffin \\
\glt `And the family \textbf{of} \textbf{the} \textbf{one} \textbf{who} \textbf{died} are busy preparing \textbf{for} \textbf{the} \textbf{work} \textbf{that} \textbf{will} \textbf{be} \textbf{done,} such as the wood to make a coffin.’ [CBWE-C-05 3.1]
\z
\ea
Mga  bai  an  a  gaimes  ta  kan-enén  ta \textbf{mga}  \textbf{ittaw}  \textbf{na}  \textbf{gatabang}  \textbf{ta}  \textbf{ubra}  daw  kan-enén  \textbf{ta}  \textbf{mga}  \textbf{gaambag}. \\\smallskip
 \gll Mga  bai  an  a  ga-imes  ta  kan-en-én  ta \textbf{mga}  \textbf{ittaw}  \textbf{na}  \textbf{ga-tabang}  \textbf{ta}  \textbf{ubra}  daw  kan-en-én  \textbf{ta}  \textbf{mga}  \textbf{ga-ambag}. \\
\textsc{pl}  woman  \textsc{def.m}  \textsc{ctr}  \textsc{i.r-}prepare  \textsc{nabs}  cooked.rice-\textsc{nr}  \textsc{nabs} \textsc{pl}  person  \textsc{lk}  \textsc{i.r}-help  \textsc{nabs}  work  and  cooked.rice-\textsc{nr}  \textsc{nabs}  \textsc{pl}  \textsc{i.r}-donate \\
\glt `The women prepare the food to be eaten by \textbf{the} \textbf{people} \textbf{who} \textbf{help} \textbf{in} \textbf{the} \textbf{work} and the food to be eaten by \textbf{the} \textbf{ones} \textbf{who} \textbf{donate}.’ [CBWE-C-05 4.1]
\z
\ea
\label{bkm:Ref363550595}
\textbf{Uļa-uļaan}  \textbf{an}  pirmi  tampek  ta    pantad. \\\smallskip
 \gll \textbf{Uļa\~uļa-an}  \textbf{an}  pirmi  tampek  ta    pantad. \\
\textsc{red}\~spill/drain/pour-\textsc{apl}  \textsc{def.m}  always  pack.on  \textsc{nabs}  sand \\
\glt ‘Sand was always packed on \textbf{where} \textbf{liquid} \textbf{drains} \textbf{out}.’ [JCWN-T-21 12.20]
\z
\subsubsection{Action nominalizations}
\label{sec:actionnominalizations}

As discussed in \sectref{sec:pag}, the prefix \textit{pag}{}- forms action nominalizations. The resulting form does not refer to a participant in the action, but to the action itself. As with participant nominalizations described in the previous section, action nominalizations may have arguments, and thus may be considered non-finite nominalized or headless relative clauses:

\ea
Yon  \textbf{una}  \textbf{an}  \textbf{na}  \textbf{pagpati}  \textbf{ta}  \textbf{mga}  \textbf{ittaw}. \\\smallskip
 \gll Yon  \textbf{una}  \textbf{an}  \textbf{na}  \textbf{pag-pati}  \textbf{ta}  \textbf{mga}  \textbf{ittaw}. \\
\textsc{d}3\textsc{abs}  first  \textsc{def.m}  \textsc{lk}  \textsc{nr.act}-believe  \textsc{nabs}  \textsc{pl}  person \\
\glt ‘That was \textbf{the} \textbf{first} \textbf{belief} \textbf{of} \textbf{people}.’ [JCOE-C- 4.1]
\z
\ea
Daw  tudluan  ta  \textbf{dayad}  \textbf{na}  \textbf{pag-ambaļ}  \textbf{ta}  \textbf{Inglis}. \\\smallskip
 \gll Daw  \emptyset{}-tudlo-an  ta  \textbf{dayad}  \textbf{na}  \textbf{pag--ambaļ}  \textbf{ta}  \textbf{Inglis}. \\
and  \textsc{t.ir}-teach-\textsc{apl}  \textsc{nabs}  good  \textsc{lk} \textsc{nr.act-}say  \textsc{nabs}  English \\
\glt ‘And teach (them) \textbf{good} \textbf{speaking} \textbf{in} \textbf{English}.’ [LBOP-C-03 3.2]
\z
\ea
… daw  namatian  man  ta  Dios  \textbf{iran}  \textbf{na} \textbf{pagpangamuyo}. \\\smallskip
 \gll … daw  na-mati-an  man  ta  Dios  \textbf{iran}  \textbf{na} \textbf{pag-pangamuyo}. \\
{} and  \textsc{a.hap.r}-hear-\textsc{apl}  \textsc{emph}  \textsc{nabs}  God  3\textsc{p.gen}  \textsc{lk}  \textsc{nr.act}-pray \\
\glt ‘… and God heard \textbf{their} \textbf{praying}.’ [CBWN-C-13 2.4]
\z
\ea
 … \textbf{iya}  \textbf{na}  \textbf{pagtrato}  \textbf{ki}  \textbf{yaken}  paryo  ta  bata  din. \\\smallskip
 \gll … \textbf{iya}  \textbf{na}  \textbf{pag-trato}  \textbf{ki}  \textbf{yaken}  paryo  ta  bata  din. \\
{} 3\textsc{s.gen}  \textsc{lk}  \textsc{nr.act-}treat  \textsc{obl.p}  1s  same  \textsc{nabs}  child  3\textsc{s.gen} \\
\glt ‘… \textbf{her} \textbf{way} \textbf{of} \textbf{treating} \textbf{me} was like the same as her child.’ [DDWN-C-01 7.3]
\z
\ea
Pamikawan  bata  an  aged  magdayad  \textbf{pagdako}  \textbf{din}  \textbf{an}. \\\smallskip
 \gll Pa-mikaw-an  bata  an  aged  mag-dayad  \textbf{pag-dako}  \textbf{din}  \textbf{an}. \\
\textsc{t.r}-food.offering-\textsc{apl}  child  \textsc{def.m}  so.that  \textsc{i.ir}-good  \textsc{nr.act}-big  3\textsc{s.gen}  \textsc{def.m} \\
\glt ‘A child has a food offering done (for him/her) so that \textbf{his/her} \textbf{growing} \textbf{up} will be good…’ [JCWE-T-16 2.4]
\z


\ea
Nļaman  nyo  yi  gid  \textbf{primiro}  \textbf{ko}  \textbf{na}  \textbf{pagsakay} \textbf{ta}  \textbf{iruplano}. \\\smallskip
 \gll Na-aļam-an  nyo  yi  gid  \textbf{primiro}  \textbf{ko}  \textbf{na}  \textbf{pag-sakay} \textbf{ta}  \textbf{iruplano}. \\
\textsc{a.hap.r}-know-\textsc{apl}  2\textsc{p.erg}  \textsc{d1abs}  \textsc{int}  first  1\textsc{s.gen}  \textsc{lk}  \textsc{nr.act}-ride \textsc{nabs}  airplane \\
\glt `You know that this is \textbf{my} \textbf{first} \textbf{riding} \textbf{in} \textbf{an} \textbf{airplane}.’ [VAWN-T-15 2.3]
\z

\subsection{Referring Phrases containing relative clauses}
\label{sec:rpscontainingrelativeclauses}

MOD\textsubscript{1} may contain a relative clause consisting only of a verb and up to one pronominal argument. Such “short” relative clauses may also appear in MOD\textsubscript{2} (\ref{bkm:Ref360181429}). In the following examples, the short relative clauses are given in brackets. Additional discussion and examples of relative clauses is provided in \chapref{chap:clausecombining}, \sectref{sec:relativeclauses}:

\ea
\textbf{pausog}  \textbf{din}    na  ittaw \\\smallskip
 \gll [ \textbf{pa-usog}  \textbf{din}    ]  na  ittaw \\
{} \textsc{caus}-stomach.ache  3\textsc{s.gen} {}  \textsc{lk}  person \\
\glt ‘the person \textbf{whom} \textbf{s/he} \textbf{caused} \textbf{to} \textbf{have} \textbf{a} \textbf{stomach} \textbf{ache’} [VAOE-J-04]
\z
\ea
ta \textbf{istaran}  \textbf{nay}  na  baļay \\\smallskip
 \gll ta       [ \textbf{istar-an}  \textbf{nay}           ]  na  baļay \\
\textsc{nabs} {} live-\textsc{nr}  1\textsc{p.excl.gen} {}  \textsc{lk}  house \\
\glt ‘the house \textbf{that} \textbf{we} \textbf{live} \textbf{in}’ [DBWN-T-22 4.5]
\z
\ea
ta  \textbf{nadakep}  \textbf{danen} na  mga  Kagayanen \\\smallskip
 \gll ta  [ \textbf{na-dakep}  \textbf{danen} ]  na  mga  Kagayanen \\
\textsc{nabs} {} \textsc{a.hap.r}\textsc{-}catch  3\textsc{p.gen} {} \textsc{lk}  \textsc{pl}  Kagayanen \\
\glt ‘the Kagayanens \textbf{whom} \textbf{they} \textbf{were} \textbf{able} \textbf{to} \textbf{capture}’ [JCWN-T-13 5.1]
\z
\ea
\label{bkm:Ref360181429}
\textbf{iatag}  \textbf{ki}  \textbf{kyo}   na  mga  bagay  na \textbf{kinangļanen}  \textbf{nyo} \\\smallskip
 \gll [ \textbf{i-atag}  \textbf{ki}  \textbf{kyo} ]  na  mga  bagay  na  [ \textbf{kinangļan-en}  \textbf{nyo}{ }] \\
{} \textsc{t.deon}-give  \textsc{obl.p}  2\textsc{p.abs}  {} \textsc{lk}  \textsc{pl}  thing  \textsc{lk}     {} need-\textsc{t.ir}  2\textsc{p.gen} {} \\
\glt ‘things \textbf{to} \textbf{give} \textbf{you} \textbf{that} \textbf{you} \textbf{need’} [ETOB-C-01 1.3]
\z

\largerpage
As in basic RPs, a plural marker \textit{mga} may occur before the relative clause in MOD\textsubscript{1} indicating indefinite plural (\ref{bkm:Ref447714036} and \ref{bkm:Ref421109060}).


\ea
\label{bkm:Ref447714036}
Tenged  pila  nang  pa  na  mga  ittaw  gaistar  an  dya  yon  na  asta  nang  pangappat na  grado  \textbf{mga}  \textbf{gaiskwila}  \textbf{an}   na  mga  kabataan. \\\smallskip
 \gll Tenged  pila  nang  pa  na  mga  ittaw  ga-istar  an  dya  yon  na  asta  nang  pang-appat  na  grado  \textbf{mga}  [ \textbf{ga-iskwila}  \textbf{an}   ]  na  mga  ka-bata-an. \\
because  few  only  \textsc{inc}  \textsc{lk}  \textsc{pl}  person  \textsc{i.r}-live  \textsc{spat.def}  \textsc{d}4\textsc{loc}  \textsc{d3abs}  \textsc{lk} until  only  \textsc{ord}-four  \textsc{lk}  grade  \textsc{pl}  {}  \textsc{i.r-}school  \textsc{def.m} {} \textsc{lk}  \textsc{pl}  \textsc{nr}-child-\textsc{nr} \\
\glt `Because still only a few people live there, that is the reason that the children \textbf{who} \textbf{are} \textbf{going} \textbf{to} \textbf{school} are only up to fourth grade.’ [DDWN-C-01 3.5]
\z
\ea
\label{bkm:Ref421109060}
Tapos  papabasa  a  din  ta  \textbf{mga}  \textbf{natapos}  \textbf{an}  \textbf{danen} na  ubra \\\smallskip
 \gll Tapos  pa-pa-basa  a  din  ta  \textbf{mga}  [ \textbf{na-tapos}  \textbf{an}  \textbf{danen} ] na  ubra \\
then  \textsc{t.r-caus}-read  1\textsc{s.abs}  \textsc{3}\textsc{s.erg}  \textsc{nabs}  \textsc{PL} {} \textsc{a.hap.r}\textsc{-}finish  \textsc{def.m}  3\textsc{s.erg}
{} \textsc{lk}  work \\
\glt `Then she had me read the tasks \textbf{they} \textbf{already} \textbf{finished}.’ [DDWN-C-01 8.4]
\z

Longer relative clauses---those containing at least one full RP---must occur in MOD\textsubscript{2}. In example \REF{bkm:Ref117497845}, the relative clause (bracketed) contains the RP \textit{ta ummay na inog} ‘ripe rice’ and so it must follow the Head, \textit{baboy an} ‘the pig’:

\ea
\label{bkm:Ref117497845}
Na  yaken  gapanaog  nakilipan    ko baboy  an  \textbf{na}  \textbf{gadļagan}  \textbf{munta}  \textbf{ta}  \textbf{ummay}  \textbf{na}  \textbf{inog}. \\\smallskip
 \gll Na  yaken  ga-panaog  na-kilip-an    ko baboy  an  \textbf{na}  [ \textbf{ga-dļagan}  \textbf{munta}  \textbf{ta}  \textbf{ummay}  \textbf{na}  \textbf{inog} ]. \\
\textsc{lk}  1\textsc{s.abs}  \textsc{i.r}-go.down  \textsc{a.hap.r}-glimpse-\textsc{apl}  1\textsc{s.erg} pig  \textsc{def.m}  \textsc{lk}  {} \textsc{i.r}-run  going  \textsc{nabs}  unmilled.rice  \textsc{lk}  ripe {} \\
\glt `When I went down I saw out of the corner of my eye the pig \textbf{that} \textbf{was} \textbf{going} \textbf{to} \textbf{the} \textbf{ripe} \textbf{rice}.’ [RCON-L-01 1.1.]
\z
\ea
Tudluan  ta  pagpasalamat  ta  Ginuo  ta mga  betang  \textbf{na}  \textbf{paatag}  \textbf{din}  \textbf{ki}  \textbf{kiten}. \\\smallskip
 \gll \emptyset{}-Tudlo-an  ta  pag-pa-salamat  ta  Ginuo  ta mga  betang  \textbf{na}  [ \textbf{pa-atag}  \textbf{din}  \textbf{ki}  \textbf{kiten}{ }]. \\
\textsc{t.ir}-teach-\textsc{apl}  \textsc{nabs}  \textsc{nr.act-caus}-thanks  \textsc{nabs}  Lord \textsc{nabs} \textsc{pl}  things  \textsc{lk} {}  \textsc{t.r}-give  3\textsc{s.erg}  \textsc{obl.p}  1\textsc{p.incl} {} \\
\glt `Teach them (small children) to give thanks to the Lord for  the things \textbf{that} \textbf{he} \textbf{gives} \textbf{us}.’ [ETOP-C-10 1.2]
\z
\ea
… paumaw  kanen  ta  pirates  daw  painsaan  daw  indi gabatang  makina  ya na \textbf{patago}  \textbf{ta}  \textbf{nanay}  \textbf{din}.  \\\smallskip
 \gll … pa-umaw  kanen  ta  pirates  daw  pa-insa-an  daw  indi ga-batang  makina  ya  na  [ \textbf{pa-tago}  \textbf{ta}  \textbf{nanay}  \textbf{din}{ }]. \\
{} \textsc{t.r}-call  3\textsc{s.abs}  \textsc{nabs}  pirates  and  \textsc{t.r}-ask-\textsc{apl}  if/when  where \textsc{i.r}-put  motor  \textsc{def.f} \textsc{lk} {} \textsc{t.r}-hide  \textsc{nabs}  mother  3\textsc{s.gen} {} \\
\glt `… the pirates called him and asked where the motor \textbf{that} \textbf{his} \textbf{mother} \textbf{hid} \textbf{was} \textbf{placed}.' [BCWN-C-04 7.1]
\z 

\subsection{Case in Referring Expressions}
\label{sec:caseinreferringexpressions}

\is{case|(}There are three grammatical values that comprise the category of “case” in Kagayanen Referring Phrases. These are \textit{absolutive}\is{absolutive}, \textit{ergative/genitive}\is{genitive}\is{ergative} and \textit{oblique}\is{oblique}. These cases are expressed by \isi{prenominal case markers} (CM), or zero. The reason we consider these forms to express case is that, in their usual usages, they reflect a grammatical relationship between the RP and the main predicator in their clause. A traditional grammar would say that the predicator (usually a verb) \textit{governs}\is{government, syntactic} the appearance of these prenominal markers. For our purposes, we do not take a stand on whether the inflected Verb\is{inflected verbs} governs the prenominal case marker (including zero), or whether the choice of nominal category as expressed by prenominal case markers governs the form of the verb---the directionality of the relationship is not crucial. What is crucial is that particular arrangements of categories expressed by \isi{prenominal case markers} correlate with particular verb forms. Thus we say that there is a structural link or “grammatical relation” \is{grammatical relations}between the two. In \chapref{chap:voice} we discuss the system of organizing grammatical relations in more detail. In this section we describe the case forms, and illustrate their usages from the text corpus.

\is{prenominal case markers|(}For basic Referring Phrases (those not headed by personal names), ergative/genitive and oblique cases are isomorphic (identical in form), expressed by the prenominal case marker \textit{ta}, while \isi{absolutive} case is zero-marked. For RPs headed by personal names (names referring to people), absolutive and ergative/genitive are isomorphic (zero-marked), and oblique case is marked by the prenominal case marker \textit{ki}.  Justification for these terms is provided in more detail in \chapref{chap:voice}, \sectref{sec:grammaticalrelations}. Personal and demonstrative pronouns also appear in absolutive, and non-absolutive forms (see \sectref{sec:pronouns}, \tabref{tab:personalpronouns}). Personal pronouns functioning in Oblique roles are the same as the absolutive pronouns, but are preceded by the oblique marker \textit{ki} (see \tabref{tab:personalpronouns} in \sectref{sec:personalpronouns} for the complete paradigm of personal pronouns). \tabref{tab:prenominalcasemarkers} above displays the prenominal case markers, repeated in \tabref{tab:prenominalcasemarkingparticles} for convenience.

\begin{table}
\caption{Prenominal case markers}
\label{tab:prenominalcasemarkingparticles}
\begin{tabularx}{\textwidth}{XXll}
\lsptoprule
& Absolutive & Ergative/Genitive & Oblique \\
\midrule
Common nouns & \emptyset{} & ta  & ta  \\
 Personal names & \emptyset{} & \emptyset{} & ki \\
 Pronouns & \emptyset{} & \emptyset{} & ki \\
\lspbottomrule
\end{tabularx}
\end{table}

The following examples illustrate the uses of the three cases in various contexts:

\ea
\textit{ta} marking non-absolutive common nouns \\
Daw  dili  pa  gusto  \textbf{ta}  \textbf{mga}  \textbf{ginikanan}, mama  i  masirbi  anay  \textbf{ta}  \textbf{baļay}  \textbf{ta}  \textbf{bai}. \\\smallskip
 \gll Daw  dili  pa  gusto  \textbf{ta}  \textbf{mga}  \textbf{ginikanan}, mama  i  ma-sirbi  anay  \textbf{ta}  \textbf{baļay}  \textbf{ta}  \textbf{bai}. \\
if/when  \textsc{neg.ir}  \textsc{inc}  want  \textsc{nabs}  \textsc{pl}  parent male  \textsc{def.n}  \textsc{a.hap.ir}-serve  first  \textsc{nabs}  house  \textsc{nabs}  woman \\
\glt `If \textbf{the parents} (RP ergative) still do not want (their children to marry each other), the man will serve in \textbf{the house} (RP oblique) \textbf{of the woman} (RP genitive).’ [DBOE-C-01 1.6]
\z
\ea
… gaimes  \textbf{ta}  \textbf{sidda}  sawa  din  an  na  bai. \\\smallskip
 \gll … ga-imes  \textbf{ta}  \textbf{sidda}  sawa  din  an  na  bai. \\
{} \textsc{i.r}-prepare  \textsc{nabs}  fish  spouse  3\textsc{s.gen}  \textsc{def.m}  \textsc{lk}  woman \\
\glt ‘(When he came home having fished already,) his wife prepared \textbf{the/some} \textbf{fish} (RP oblique).’ [CBON-C-02  2.3] \\
\z
\ea
Pasaway  a  \textbf{ta}  \textbf{bai}  na  duma  nay… \\\smallskip
 \gll Pa-saway  a  \textbf{ta}  \textbf{bai}  na  duma  nay… \\
\textsc{t.r}-correct  1\textsc{s.abs}  \textsc{nabs}  woman  \textsc{lk}  companion  1\textsc{p.excl.gen} \\
\glt ‘\textbf{The} \textbf{woman} (RP ergative) who was our companion corrected me…’ [CBWN-C-10 4.14]
\z
\ea
\textit{ki} marking oblique personal names \\
Nabaton  nay  imo  na  suļat  na  padaļa no  \textbf{ki}  \textbf{Maria}  ta  pitsa  23  ta  Oct. \\\smallskip
 \gll Na-baton  nay  imo  na  suļat  na  \emptyset{}-pa-daļa no  \textbf{ki}  \textbf{Maria}  ta  pitsa  23  ta  Oct.\footnotemark{} \\
\textsc{a.hap.r}-receive  \textsc{1}\textsc{p.excl}.\textsc{erg}  2\textsc{gen}  \textsc{lk}  letter  \textsc{lk} \textsc{t.r-caus-}carry \textsc{2serg}  \textsc{obl.p}  Maria  \textsc{nabs}  date  23  \textsc{nabs}  October \\
\footnotetext{This example is from a letter, and reflects what the writer actually wrote, for example, “23 ta Oct.”.} 
\glt `We received your letter that you sent \textbf{with} \textbf{Maria} (RP oblique) on the date of the 23\textsuperscript{rd} in October.' [PBWL-C-04 3.1]
\z

\newpage
\ea
\textit{ki} marking third person oblique pronoun \\
… danen  i  Papa  daw  Mama  gatagad  \textbf{ki} \textbf{kanen} na  uļa  pa kuļi. \\\smallskip
 \gll … danen  i  Papa  daw  Mama  ga-tagad  \textbf{ki} \textbf{kanen} na  uļa  pa ka-uļi. \\
{} 3\textsc{p.abs}  \textsc{def.n}  Papa  and  Mama  \textsc{i.r}-wait  \textsc{obl.p}  \textsc{3s} \textsc{lk}  \textsc{neg.r} \textsc{inc} \textsc{i.exm}-go.home \\
\glt `Papa and Mama were waiting \textbf{for} \textbf{him} (RP oblique) when he had not yet come home.’ [CBWN-C-18 4.1]
\z

The absolutive case is unmarked for common nouns and personal names. Since both the actor (\isi{ergative}) personal name and the \isi{absolutive} undergoer are unmarked, when both occur in one clause, their roles are distinguished only by order (\ref{bkm:Ref329345305})---the Actor is always first and the Undergoer second.
\ea
Zero marking on absolutive actor common noun \\
Lugay  gaambaļ \textbf{mga}  \textbf{ittaw}  \textbf{an},  “Sunog!  Sunog!” \\\smallskip
 \gll Lugay  ga-ambaļ  \emptyset{} \textbf{mga}  \textbf{ittaw}  \textbf{an},  “Sunog!  Sunog!” \\
long.time  \textsc{i.r}-say  \textsc{abs} \textsc{pl}  person  \textsc{def.m}  burn  burn \\
\glt ‘Then \textbf{the} \textbf{people} said, “Fire! Fire!”' [RZWN-T-02 2.5]
\z
\ea
Zero marking on absolutive undergoer common noun \\
Pagkita  ko  ta  ittaw  ya,  paanggat  ko  \textbf{bata}  \textbf{i}  ta  tag-iya  ta  lansa  na manaog. \\\smallskip
 \gll Pag-kita  ko  ta  ittaw  ya,  pa-anggat  ko  \emptyset{} \textbf{bata}  \textbf{i}  ta  tag-iya  ta  lansa  na m-panaog. \\
\textsc{nr.act}-see  \textsc{1}\textsc{s.gen}  \textsc{nabs}  person  \textsc{def.m}  \textsc{t.r}-ask.to.accompany  1\textsc{s.erg} \textsc{abs} child  \textsc{def.n}  \textsc{nabs}  owner  \textsc{nabs}  launch  \textsc{lk}  \textsc{i.v.ir}-go.down \\
\glt `When/after I saw the person, I asked \textbf{the} \textbf{child} of the owner of the motorized launch to accompany me to go down (off the boat).’ [DBWN-T-23 9.2]
\z
\ea
Zero marking on absolutive location/recipient common noun (applicative construction) \\
Tapos  paatagan  din \textbf{mama}  \textbf{ya}  ta  buļak … \\\smallskip
 \gll Tapos  pa-atag-an  din \emptyset{} \textbf{mama}  \textbf{ya}  ta  buļak … \\
then \textsc{t.r}-give-\textsc{apl}  3\textsc{s.erg} \textsc{abs} man  \textsc{def.f}  \textsc{nabs}  flower \\
\glt `Then she gave \textbf{the man} a flower.'
\z

\newpage
\ea
Zero marking on absolutive location common noun (applicative construction) \\
… daw  pailingan  danen \textbf{yo}  \textbf{na}  \textbf{lungag} …. \\\smallskip
 \gll … daw  pa-iling-an  danen \emptyset{} \textbf{yo}  \textbf{na}  \textbf{lungag} …. \\
{} and \textsc{t.r}-go-\textsc{apl}  3\textsc{p.erg} \textsc{abs} \textsc{d4abs}  \textsc{lk} hole \\
\glt ‘…and they went to \textbf{that} \textbf{hole}…’ [ETON-C-07 5.1] 
\z
\ea
Zero marking on absolutive actor personal name \\
Mugtaw  kaw  tak  guli \textbf{Maria}  \textbf{i}. \\\smallskip
 \gll M-bugtaw  kaw  tak  ga-uli \emptyset{} \textbf{Maria}  \textbf{i}. \\
\textsc{i.v.ir}-wake.up  \textsc{2}\textsc{s.abs}  because  \textsc{i.r}-go.home \textsc{abs.p} Maria  \textsc{def.n} \\
\glt ‘Wake up because \textbf{Maria} came home.’ [CBWN-C-19 5.2]
\z
\ea
Zero marking on absolutive undergoer personal name \\
Nuļog \textbf{Pedro}  \textbf{ya.} \\\smallskip
 \gll Na-uļog \emptyset{} \textbf{Pedro}  \textbf{ya.} \\
\textsc{a.hap.r}-fall \textsc{abs.p} Pedro  \textsc{def.f} \\
\glt ‘\textbf{Pedro} fell (off a carabao).’ [RCON-L-02 2.15]
\z
\ea
… pakapkap  din \textbf{Maria}  \textbf{an}. \\\smallskip
 \gll … pa-kapkap  din \emptyset{} \textbf{Maria}  \textbf{an}. \\
{} \textsc{t.r}-feel 3\textsc{s.erg} \textsc{abs.p} Maria  \textsc{def.m} \\
\glt ‘… he felt \textbf{Maria}.’ [LGON-L-01 7.12]
\z
\ea
Zero marking on absolutive location personal name (applicative construction) \\
… na  nakamang  din  bayo  ta  Pedro,  pasuot  din daw  ilingan  dya \textbf{Maria}  \textbf{i}  ta  selled  din  ya. \\\smallskip
 \gll … na  na-kamang  din  bayo  ta  Pedro,  pa-suot  din daw  ...-iling-an\footnotemark{}  dya  \emptyset{} \textbf{Maria}  \textbf{i}  ta  selled  din  ya. \\
{} when  \textsc{a.hap.r}\textsc{-}get  3\textsc{s.erg} clothes \textsc{nabs}  Pedro  \textsc{t.r}-wear  3\textsc{s.erg} and \textsc{t.r}-go-\textsc{apl}  \textsc{d4loc} \textsc{abs.p} Maria  \textsc{def.n}  \textsc{nabs}  inside  3\textsc{s.gen}  \textsc{def.f} \\
\footnotetext{Sometimes after the conjunction \textit{daw} the prefix of the following verb drops off. In this example the verb \textit{pa-iling-an}, which is realis, becomes \textit{iling-an} which looks like the irrealis form of the same verb. The initial vowel also may drop off, as in this example, leaving \textit{lingan} as the surface form. This tendency to drop affixes is described in \chapref{chap:verbstructure}, \sectref{sec:omissionofprefixes}.}
\glt `… when he had gotten the clothes of Pedro, he put them on and went there to \textbf{Maria} in her room.’ [LGON-L-01 7.3]
\z
\ea
Zero marking on ergative personal name \\
… daw  papayapunan  man \textbf{Manang}  kanen  an. \\\smallskip
 \gll … daw  pa-pa-yapun-an  man \emptyset{} \textbf{Manang}  kanen  an. \\
{} and  \textsc{t.r-caus}-supper-\textsc{apl}  too \textsc{erg.p} older.sister  \textsc{3}\textsc{s.abs}  \textsc{def.m} \\
\glt ‘… and \textbf{Manang} fed him supper.’ [CBWN-C-18 3.1]
\z
\ea
Zero marking on genitive personal name \\
Pagsabat  ko  namatian  ko  kagi \textbf{Manang} … na  gainsa  ki  yaken  daw  kamusta  a  kon. \\\smallskip
 \gll Pag-sabat  ko  na-mati-an  ko  kagi \emptyset{} \textbf{Manang} … na  ga-insa  ki  yaken  daw  kamusta  a  kon. \\
\textsc{nr.act}-answer  \textsc{1}\textsc{s.erg} \textsc{a.hap.r}-hear-\textsc{apl}  1\textsc{s.erg} voice  \textsc{gen.p} older.sister {} \textsc{lk}  \textsc{i.r}-ask  \textsc{obl.p}  1s if/when  greet  1\textsc{s.abs}  \textsc{hsy} \\
\glt `When I answered (the telephone) I heard the voice \textbf{of} \textbf{Manang} … asking me how I am.’ [CBWN-C-22 2.8]
\z
\ea
\label{bkm:Ref329345305}
Zero marking on \isi{ergative} and \isi{absolutive}  \isi{personal names} \\
Pasugo \textbf{Manang} \textbf{Pedro}  an  na  mandok. \\\smallskip
 \gll Pa-sugo \emptyset{} \textbf{Manang} \emptyset{} \textbf{Pedro}  an  na  m-sandok. \\
\textsc{t.r}-order \textsc{erg.p} older.sister \textsc{abs.p} Pedro  \textsc{def.m}  \textsc{lk}  \textsc{i.v.ir}-carry.water \\
\glt ‘\textbf{Older-sister} ordered \textbf{Pedro} to carry water.’
\z

Though there are no case markers preceding either of the core arguments in \REF{bkm:Ref329345305}, the interpretation is unambiguous as to who is ordering whom to carry water. First, there is a strong preference for ergative arguments to precede other arguments, especially when proper names are involved. Second, the demonstrative determiner \textit{an}, though not required, reinforces the idea that Pedro is the absolutive argument, therefore the Patient/Causee. Third, pragmatically a younger person (Pedro) could not conceivably order his older sister to do something.
\is{prenominal case markers|)}

Personal pronouns\is{personal pronouns} and \isi{enclitics} also distinguish all three cases. The demonstratives\is{demonstrative pronouns} only distinguish absolutive and non-absolutive cases. See \sectref{sec:pronouns} for a description and exemplification of the cases of the various types of pronouns.\is{case|)}

\subsection{Number}
\label{sec:number}

\is{number|(}Referring Phrases are optionally marked for \isi{plural} depending on what type of noun functions as the Head. Common nouns are optionally marked for plural with the particle \textit{mga}. Example \REF{ex:bata} describes a marriage arranged by the parents. Note that the fathers and mothers are marked as plural with \textit{mga} in the first clause, but the fathers and mothers are not marked as plural in the second clause. In this context, it is understood that each child has a mother and a father and these are the ones referred to in the second clause. Including \textit{mga} in the second instance of “fathers and mothers” would seem redundant.

\ea
\label{ex:bata}
Plural marker with common noun \\
… \textbf{mga}  \textbf{tatay}  \textbf{daw}  \textbf{nanay}  may  kasugtanan  na  darwa  na  bata magsawaay  danen  daw  may  idad  en  na  usto. Piro,  \textbf{tatay}  \textbf{daw}  \textbf{nanay}  ta  \textbf{mga}  \textbf{bata}  na  darwa  may  kundisyunis. \\\smallskip
 \gll … \textbf{mga}  \textbf{tatay}  \textbf{daw}  \textbf{nanay}  may  kasugtanan  na  darwa  na  bata mag-sawa-ay  danen  daw  may  idad  en  na  usto. Piro,  \textbf{tatay}  \textbf{daw}  \textbf{nanay}  ta  \textbf{mga}  \textbf{bata}  na  darwa  may  kundisyunis. \\
{} \textsc{pl} father  and  mother  \textsc{ext.in}  agreeement  \textsc{lk} two  \textsc{lk}  child \textsc{i.ir}-spouse-\textsc{rec}  3\textsc{p.abs}  if/when  \textsc{ext.in}  age  cm  \textsc{lk}  proper/sufficient but  father  and  mother  \textsc{nabs}  \textsc{pl}  child  \textsc{lk}  two  \textsc{ext.in}  qualifications \\
\glt `\textbf{Fathers} \textbf{and} \textbf{mother}s have an agreement that two children will marry each other when the age is sufficient. But \textbf{the} \textbf{fathers} \textbf{and} \textbf{mothers} of \textbf{the} \textbf{two} \textbf{children} had qualifications.’ [HCOE-J-02 3.2-3]
\z

In conversation the plural marker is often left out, especially when plurality is obvious. For example, in \REF{bkm:Ref447720182} the RP \textit{ittaw i na malain} ‘people who are bad’ has no plural marker but it is obvious that this is plural from the 3\textsuperscript{rd} person plural pronoun \textit{danen} after the verb referring to the same people.
\ea
\label{bkm:Ref447720182}
Tapos  na  gakilem  en  \textbf{ittaw}  \textbf{i}  \textbf{na}  \textbf{malain} gapanaik  \textbf{danen}  ta  isya  na  baļay … \\\smallskip
 \gll Tapos  na  ga-kilem  en  \textbf{ittaw}  \textbf{i}  \textbf{na}  \textbf{ma-lain} ga-panaik  \textbf{danen}  ta  isya  na  baļay … \\
then  \textsc{lk}  \textsc{i.r}-night  \textsc{cm}  person  \textsc{def.n}  \textsc{lk}  \textsc{abs}-bad\footnotemark{}
\textsc{i.r}-go.up  3\textsc{p.abs}  \textsc{nabs}  one  \textsc{lk}  house \\
\footnotetext{The \textit{ma-} prefix here is identical to the irrealis happenstantial prefix. However, this is not the happenstantial prefix. See the discussion in \chapref{chap:modification}, \sectref{sec:ma-derivation} on the use of the prefix \textit{ma}{}- to mark property concept words that refer to permanent or inherent properties.}
\glt `Then when it became night \textbf{the} \textbf{people} \textbf{who} \textbf{are} \textbf{bad} they went up to one house…' [BCWN-C-02  3.5]
\z 

Personal names can be marked for plural with plural pronouns, in which case the RP refers to the person named and some group of associates, for example, \textit{danen Maria} `Maria and them/family/companions/associates', \textit{danen Papa} `Dad and them/family/companions/associates' (see \sectref{sec:referringphrasesheadedbypersonalnames}, examples \ref{bkm:Ref53060834} and \ref{bkm:Ref343578640}). Referring Phrases can also be pluralized with numerals, quantifiers, collective nouns, and with two adjectives that have specialized plural forms – \textit{sise/siset} ‘small’, \textit{derse/derset} ‘small.\textsc{pl}’ (\ref{bkm:Ref447720514}), and \textit{bakod} ‘big’,  \textit{darko} ‘big.\textsc{pl}’ (examples \ref{bkm:Ref447720545} and \ref{bkm:Ref447720547}). Collections or groups can also be specified morphologically, for example, \textit{bata} ‘child’, \textit{bataan} ‘group of children usually with the same parents’ \textit{kabataan} ‘groups or many children usually having different parents’ (see \sectref{sec:ka-an} above). The plural marker \textit{mga} is optional when there is a plural form of a numeral, quantifier or adjective:

\ea
\label{bkm:Ref447720514}
Plural form of an adjective \\
Tapos  lungagan  ta  \textbf{derset}  \textbf{na}  \textbf{lungag}  para  makasingngaw daw  magsikaļ  waig  an. \\\smallskip
 \gll Tapos  \emptyset{}-lungag-an  ta  \textbf{derset}  \textbf{na}  \textbf{lungag}  para  maka-singngaw daw  mag-sikaļ  waig  an. \\
then  \textsc{t.ir}-hole-\textsc{apl}  \textsc{nabs} small.\textsc{pl}  \textsc{lk} hole  \textsc{purp}  \textsc{i.caus.hap.ir}-evaporate if/when  \textsc{i.ir}-boil  water  \textsc{def.m} \\
\glt `Then make \textbf{small} \textbf{holes} (in the can) for the purpose of the water evaporating out when boiling.’ [ETOP-C-11 1.7]
\z
\ea
\label{bkm:Ref447720545}
 … dili  pa  kuļi  tenged  ta  biskeg  na  angin  daw \textbf{darko}  \textbf{na}  \textbf{mga}  baļed. \\\smallskip
 \gll … dili  pa  ka-uļi  tenged  ta  biskeg  na  angin  daw \textbf{darko}  \textbf{na}  \textbf{mga}  baļed. \\
{} \textsc{neg.ir}  \textsc{inc}  \textsc{i.exm}-go.home  because  \textsc{nabs}  strong  \textsc{lk}  wind  and big.\textsc{pl}  \textsc{lk}  \textsc{pl}  wave \\
\glt `… (they) cannot go home because of strong wind and \textbf{big} \textbf{waves}.’ [VPWE-T-01 2.8]
\z
\ea
\label{bkm:Ref447720547}
Unsi  bilog  na  \textbf{darko}  \textbf{na}  \textbf{sidda}  igo  nang  layungan ta  appat  na  bataan. \\\smallskip
 \gll Unsi  bilog  na  \textbf{darko}  \textbf{na}  \textbf{sidda}  igo  nang  \emptyset{}-layung-an ta  appat  na  bata-an. \\
eleven  round  \textsc{lk}  big.\textsc{pl}  \textsc{lk}  fish  right  just  \textsc{t.ir}-carry.on.pole-\textsc{apl} \textsc{nabs}  four  \textsc{lk}  child-\textsc{nr} \\
\glt `Eleven \textbf{big} \textbf{fish} is just right for four children to carry between them (on a pole over the shoulders).’ [EFWN-T-11 16.3]
\z

\newpage
\ea
Numeral \\
\textbf{Sampuļo}  \textbf{daw}  \textbf{annem}  \textbf{na}  \textbf{bļangay}  naan  man  dya. \\\smallskip
 \gll \textbf{Sampuļo}  \textbf{daw}  \textbf{annem}  \textbf{na}  \textbf{bļangay}  naan  man  dya. \\
ten  and  six  \textsc{lk}  2.mast.boat  \textsc{spat.def} too  \textsc{d}4\textsc{loc} \\
\glt ‘\textbf{Ten} \textbf{and} \textbf{six} \textbf{(sixteen)} \textbf{two-masted boats} were there too.’ [VAWN-T-18 3.2]
\z
\ea
Quantifier \\
May  bandilyo  na  \textbf{tanan}  \textbf{na}  \textbf{ittaw}  magpaindyiksyon naan  ta  Health Center. \\\smallskip
 \gll May  bandilyo  na  \textbf{tanan}  \textbf{na}  \textbf{ittaw}  mag-pa-indyiksyon naan  ta  Health Center. \\
\textsc{ext.in} public.announcement  \textsc{lk}  all  \textsc{lk}  person  \textsc{i.ir}-\textsc{caus}-injection
\textsc{spat.def}  \textsc{nabs} Health Center \\
\glt `There was a public announcement that \textbf{all} \textbf{people} will have themselves injected at the Health Center.’ [JCWN-T-21 7.2]
\z
\ea
Collective noun \\
Daw  may  saļa  ka  na  dili  matuman \textbf{iran}  \textbf{na}  \textbf{kasuguan}  may   bunaļ … \\\smallskip
 \gll Daw  may  saļa  ka  na  dili  ma-tuman \textbf{iran}  \textbf{na}  \textbf{ka-sugu-an}  may   bunaļ … \\
if/when  \textsc{ext.in}  mistake/wrong  2\textsc{s.abs}  \textsc{lk}  \textsc{neg.ir}  \textsc{a.hap.ir}-fulfill/obey
3\textsc{p.gen}  \textsc{lk}  \textsc{nr}-command-\textsc{nr}  \textsc{ext.in}  whip \\
\glt `If you have mistake/wrong that does not fulfill \textbf{their} \textbf{commands} there will be a whipping…’ [BBON-C-03 1.6]
\z
\ea
\textbf{Kaluluan}  nay  ya  gapinsar  daw  ino  iran  na pagbuaten \\\smallskip
 \gll \textbf{Ka-lulo-an}  nay  ya  ga-pinsar  daw  ino  iran  na pag-buat-en \\
\textsc{nr}-lulo-\textsc{nr} 1\textsc{p.excl.gen}  \textsc{def.f}  \textsc{i.r}-think  if/when  what 3\textsc{p.gen}  \textsc{lk} \textsc{nr.act}-do/make-\textsc{t.ir} \\
\glt `Our ancestors thought about what they will do…’ [EMWN-T-07 3.3]
\is{number|)}
\z

\subsection{Definiteness: The demonstrative determiners}
\label{sec:definiteness}
\is{definiteness|(}
Kagayanen has three demonstrative determiners \textit{i}, \textit{an}, and \textit{ya} that occur in the RP as part of the enclitic complex we designate as “Є” (see the template in example \REF{ex:rptemplate}, and \tabref{tab:rptemplate}). If the Head is a common noun, the determiners indicate that it is definite, in a specific Kagayanen sense,\footnote{Terms such as ``definite" and ``identifiable" as they are commonly used for Indo-European languages do not correspond directly to the pragmatic function of these determiners in Kagayanen. \citet{pebleyenclitic1999} describes their functions in detail from a discourse perspective. For the purposes of this grammar, however, we consider ``definite" to reasonably evoke the pragmatic sense that these determiners impart.} and specify its location in space and/or time.  The determiner \textit{i} `\textsc{def.n}’ indicates that the referent is close to the speaker or current in time, \textit{an} `\textsc{def.m}’ indicates it is nearby in the area and/or further away in time, and \textit{ya} `\textsc{def.f}’ indicates that it is far away, out of sight or more distant in time.
Absolutive nouns frequently occur with a determiner, but not obligatorily. RPs in all other roles can also occur with these forms.  In other words, these are not case markers (contra \citealt{harmon1977}). The demonstrative determiners can also occur on genitive and oblique common nouns (\ref{ex:prison}-\ref{ex:night}). Any common noun that occurs clause finally preferentially takes a determiner (see example \ref{bkm:Ref447721085} and many others below). The ergative element is usually already introduced into the story and so does not need to be marked with a definite marker. However, sometimes a definite marker does occur in an ergative RP (\ref{ex:woman}-\ref{ex:snake}). There are also discourse functions for \textit{i}, \textit{an}, and \textit{ya} that involve staging of participants \citep{pebleyenclitic1999}. They cannot occur with other demonstrative elements in the same RP.

In addition to their usages in RPs headed by common nouns, the demonstrative determiners also occur when the Head is an absolutive proper name or absolutive free personal pronoun. When the RP head is a proper name or personal pronoun, the three determiners do not indicate definiteness since these are already definite. Rather they indicate only the location in space and time (see examples \ref{bkm:Ref447721085}, and \ref{bkm:Ref447721088}-\ref{bkm:Ref447721090}).

Phonologically, the demonstrative determiners are enclitics, and thus exhibit initially geminate transition consonants in certain environments (see \chapref{chap:phonology}, \sectref{sec:gemination}).  \citet{pebleyenclitic1999} provides extensive discussion of the discourse functions of the demonstrative determiners in Kagayanen. The following are a few contrasting examples of demonstrative determiners from the corpus:

\largerpage
\ea
Demonstrative determiner on absolutive common noun \\
Gapanaw  kon  en  \textbf{bai}  \textbf{ya}  na  magmugon. \\\smallskip
 \gll Ga-panaw  kon  en  \textbf{bai}  \textbf{ya}  na  mag-mugon. \\
\textsc{i.r}-walk/go  \textsc{hsy} \textsc{inc}  woman  \textsc{def.f}  \textsc{lk}  \textsc{i.ir}-day.worker \\
\glt ‘\textbf{The} \textbf{woman} left to work at a daily wage job, I’m told.’ [MBON-C-01 2.5]
\z

\newpage
\ea
\label{bkm:Ref447721085}
Daw  iling  sa  di  umawen  \textbf{bai}  \textbf{ya}  daw  \textbf{mama}  \textbf{ya}? \\\smallskip
 \gll Daw  iling  sa  di  umaw-en  \textbf{bai}  \textbf{ya}  daw  \textbf{mama}  \textbf{ya}? \\
if/when  like  \textsc{d4nabs}  \textsc{rq}  call-\textsc{t.ir}  woman\textsc{def.f}  and  man  \textsc{def.f} \\
\glt ‘If it is like that, shouldn’t you call both \textbf{the} \textbf{woman} and \textbf{the} \textbf{man}?’  [BGON-L-01 8.1]
\z
\ea
\label{ex:outriggercanoe}
Patakked	daen	\textbf{tampara}	\textbf{an}	ta	iran	na	mata	daw	magtugpa	ta	lunday. \\\smallskip
 \gll Pa-takked	daen	\textbf{tampara}	\textbf{an}	ta	iran	na	mata	daw	mag-tugpa	ta	lunday. \\
\textsc{t.r}-attach	3\textsc{p.erg}	goggles	\textsc{def.m}	\textsc{nabs}	3\textsc{p.gen}	\textsc{lk}	eye	and		\textsc{i.ir}-jump.down
\textsc{nabs}	outrigger.canoe \\
\glt ‘They put \textbf{the goggles} on their eyes and jumped out of the outrigger canoe.’ [JPWN-L-01 4.3]
\z

\ea
\label{bkm:Ref343005737} \label{ex:woman}
Demonstrative determiner on ergative common noun \\
Piro  uļa  ki  kaisturyaay  tak  ambaļ \textbf{ta}  \textbf{bai}  \textbf{ya}  dili  kon  kaļa  kaon  daw  Manang. \\\smallskip
 \gll Piro  uļa  ki  ka-isturya-ay  tak  ambaļ \textbf{ta}  \textbf{bai}  \textbf{ya}  dili  kon  kaļa  kaon  daw  Manang. \\
but  \textsc{neg.r}  1\textsc{p.incl.abs}  \textsc{i.exm}-talk-\textsc{rec}  because  say
\textsc{nabs}  woman  \textsc{def.f}  \textsc{neg.ir}  \textsc{hsy}  know  \textsc{2}\textsc{s.abs}  and  older.sister \\
\glt `But we did not talk with each other because \textbf{the} \textbf{woman} said (she) did not know you and Older sister.’ [CBWl-C-07 3.2]
\z

\ea
\label{ex:hermitcrab}
Ambaļ	\textbf{ta}	\textbf{mga}	\textbf{umang}	\textbf{an}	na	duma,	"Ino	ka	man	i	imo?” \\\smallskip
 \gll Ambaļ	\textbf{ta}	\textbf{mga}	\textbf{umang}	\textbf{an}	na	duma,	"Ino	ka	man	i	imo?” \\
say	\textsc{nabs}	\textsc{pl}	hermit.crab	\textsc{def.m}	\textsc{lk}	other	what	2\textsc{s.abs}	\textsc{emph}	\textsc{att}	\textsc{emph}	 \\
\glt ‘\textbf{The other hermit crabs} said, “What are you really?”' (This emphatic question is similar to English “Who do you think you are?”)[JCON-L-08 19.5] 
\z
\ea
\label{bkm:Ref447721088}
\label{ex:snake}
… tak  uļa  kon  nalam-ed  ta  miad  \textbf{ta}  \textbf{bekkessan}  \textbf{ya}. \\\smallskip
 \gll … tak  uļa  kon  na-lam-ed  ta  miad  \textbf{ta}  \textbf{bekkessan}  \textbf{ya}. \\
{} because  \textsc{neg.r} \textsc{hsy} \textsc{a.hap.r}-swallow  \textsc{nabs}  good  \textsc{nabs}  snake  \textsc{def.f} \\
\glt ‘… because, they say, \textbf{the} \textbf{snake} did not swallow (it) well/completely.’ (This text is about the belief that a lunar eclipse is cause by a big snake swallowing the moon.) [JCOE-C-03 4.6b]
\z
\ea
On non-absolutive Undergoer, common nouns \\
Daw	uļa	don	daen	tatay	daw	nanay	din,	kanen gasikaso \textbf{ta}	\textbf{mga} \textbf{mangngod}	\textbf{din} \textbf{an}… \\\smallskip
 \gll Daw	uļa	don	daen	tatay	daw	nanay	din,	kanen ga-asikaso \textbf{ta}	\textbf{mga} \textbf{mangngod}	\textbf{din} \textbf{an}… \\
if/when	\textsc{neg.r}	\textsc{d}3\textsc{loc}	3\textsc{p.abs}	father	and	mother	3\textsc{s.gen}	3\textsc{s.abs} \textsc{i.r}-take.care \textsc{nabs}	\textsc{pl} younger.sibling	3\textsc{s.gen} \textsc{def.m} \\
\glt ‘When her father and mother are not there, she takes care of \textbf{her younger siblings}.’ [RZWE-J-01 9.9]
\z
\ea
\label{bkm:Ref447721537}
\label{ex:prison}
On genitive common noun \\
Gani,  papriso  amay  \textbf{ta}  \textbf{bata}  \textbf{ya}. \\\smallskip
 \gll Gani,  pa-priso  amay  \textbf{ta}  \textbf{bata}  \textbf{ya}. \\
so  \textsc{t.r}-prisoner  father  \textsc{nabs}  child  \textsc{def.f} \\
\glt ‘So, the father \textbf{of} \textbf{the} \textbf{child} was put in prison.’ [PBWN-C-12 12.6]
\z
\ea \label{bkm:Ref447721090}
On oblique common noun \\
Gasinggit  a  \textbf{ta}  \textbf{duma}  \textbf{ko}  \textbf{ya}. \\\smallskip
 \gll Ga-singgit  a  \textbf{ta}  \textbf{duma}  \textbf{ko}  \textbf{ya}. \\
\textsc{i.r}-shout  \textsc{1sabs}  \textsc{nabs}  companion  \textsc{1sgen  def.f} \\
\glt ‘I shouted \textbf{to} \textbf{my} \textbf{companion}.’ [DBON-C-08  2.8]
\z
\ea
… isya  a  nang  naan  \textbf{ta}  \textbf{baļay}  \textbf{i}. \\\smallskip
 \gll … isya  a  nang  naan  \textbf{ta}  \textbf{baļay}  \textbf{i}. \\
   {} one  1\textsc{s.abs}  only  \textsc{spat.def}  \textsc{nabs}  house  \textsc{def.n} \\
\glt ‘… I am/was the only one \textbf{in} \textbf{the} \textbf{house}.’ [CBWN-C-22 2.10]
\z

\ea
Ensan  ka  nang  en  Mari  ii  naan  \textbf{ta}  \textbf{kusina}  \textbf{an}. \\\smallskip
 \gll Ensan  ka  nang  en  Mari  ii  naan  \textbf{ta}  \textbf{kusina}  \textbf{an}. \\
\textsc{d}2\textsc{loc.pr}  2\textsc{s.abs}  only/just  \textsc{cm}  friend  urine  \textsc{spat.def}  \textsc{nabs}  kitchen  \textsc{def.m} \\
\glt ‘… there you just urinate in the kitchen.’ (Since the floors are made of bamboo with spaces between, then one can urinate in the kitchen. This was when the restrooms were outhouses and so some would urinate in the kitchen at night, or when it was not convenient to go to the outhouse.) [CBWN-C-13 6.10]
\z
\ea
Kalibangbang  ya  dayon  man  a  suklip  naan  \textbf{ta}  \textbf{pagang}  \textbf{ya}. Adlek  man  kanen  an.  Indangan  ya  a  dayon  man  seddep naan  \textbf{ta}  \textbf{bato}  \textbf{ya}. \\\smallskip
 \gll Kalibangbang  ya  dayon  man  a  suklip  naan  \textbf{ta}  \textbf{pagang}  \textbf{ya}. Adlek  man  kanen  an.  Indangan  ya  a  dayon  man  seddep naan  \textbf{ta}  \textbf{bato}  \textbf{ya}. \\
butterfly    \textsc{def.f}  right.away  also  \textsc{ctr}  insert  \textsc{spat.def}  \textsc{nabs}  coral  \textsc{def.f}
afraid  also  3\textsc{s.abs}  \textsc{def.m}  Surgeonfish  \textsc{def.f}  \textsc{ctr}  right.away  also  go.under
\textsc{spat.def}  \textsc{nabs}  rock  \textsc{def.f} \\
\glt `Butterfly fish immediately went between the coral. S/he was also afraid. Surgeonfish immediately went \textbf{under the rock}.’ [JCON-L-07 4.10-12]
\z
\ea
\label{ex:night}
Ibalik	no	kallot	an	\textbf{ta}	\textbf{banggay}	\textbf{an}	daw	betangan eman	ta	dagat	ta	appat	na	kilem. \\\smallskip
 \gll I-balik	no	kallot	an	\textbf{ta}	\textbf{banggay}	\textbf{an}	daw	betang-an eman	ta	dagat	ta	appat	na	kilem. \\
\textsc{t.deon}-return	2\textsc{s.erg}	wild.cassava	\textsc{def.m}	\textsc{nabs}	trough	\textsc{def.m}	and		put-\textsc{apl}
again.as.before	\textsc{nabs}	sea	\textsc{nabs}	four	\textsc{lk}	night \\
\glt ‘You should return the wild cassava \textbf{to the trough} and put it again as before in seawater for four nights.’ [JCWE-L-32 5.14]
\z
\ea
On non-absolutive instruments \\
Dakepen	ta kaw	daw	pakpaken	ko	tudtod	no	an	\textbf{ta}	\textbf{bato}	\textbf{i}. \\\smallskip
 \gll Dakep-en	ta kaw	daw	pakpak-en	ko	tudtod	no	an	\textbf{ta}	\textbf{bato}	\textbf{i}. \\
catch-\textsc{t.ir}	1\textsc{s.erg} 2\textsc{s.abs}	and	pound.on-\textsc{t.ir}	1\textsc{s.erg}	back	2\textsc{s.gen}	\textsc{def.m}	\textsc{nabs}	rock	\textsc{def.n} \\
\glt `I will catch you and pound on your back \textbf{with the rock}.’ [CBWN-C-16 9.14]
\z
\ea
On proper names \\
Bantayan	no	\textbf{Pedro}	\textbf{an}	tak	manaw	a	na	mamugon	ta	pagkaan. \\\smallskip
 \gll Bantay-an	no	\textbf{Pedro}	\textbf{an}	tak	m-panaw	a	na	ma-mugon	ta	pag-kaan. \\
watch-\textsc{apl}	2\textsc{s.erg}	Pedro	\textsc{def.m}	because	\textsc{i.v.ir}-go/walk	1\textsc{s.abs}	\textsc{lk} \textsc{a.hap.ir}-day.worker \textsc{nabs}	\textsc{nr.act}-eat \\
\glt ‘Watch \textbf{Pedro} because I am leaving to do day labor for food.’ [MBON-C-01 2.3]
\z
\ea
\textbf{Manong}	\textbf{ya}	patay	en. \\\smallskip
 \gll \textbf{Manong}	\textbf{ya}	patay	en. \\
older.brother	\textsc{def.f}	dead	\textsc{cm} \\
\glt ‘\textbf{Older brother} is dead now.’ [CBWN-C-22 13.2]
\z
\ea
On free absolutive personal pronouns \\
Tangkisan	\textbf{kanen}	\textbf{ya}	ta	ame	na	duma. \\\smallskip

\gll ...-Tangkis-an	\textbf{kanen}	\textbf{ya}	ta	ame	na	duma. \\
\textsc{t.r}-grin-\textsc{apl}	3s\textsc{abs}	\textsc{def.f}	\textsc{nabs}	1p\textsc{excl.gen}	\textsc{lk}	companion \\
\glt ‘Our companion grinned \textbf{at him}.’ [NFWN-T-01 2.14]
\z
\ea
Pasugo	\textbf{yaken}	\textbf{i}	ta	ake	na	nanay	mangaoy	para sunggan. \\\smallskip
 \gll Pa-sugo	\textbf{yaken}	\textbf{i}	ta	ake	na	nanay	ma-ŋ-kaoy	para sunggan. \\
\textsc{t.r}-order	1\textsc{s.abs}	\textsc{def.n}	\textsc{nabs}	1\textsc{s.gen}	\textsc{lk}	mother \textsc{a.hap.ir}-\textsc{pl}-wood	for firewood \\
\glt ‘My mother ordered \textbf{me} to gather wood for firewood.’ [NFWN-T-01 2.2]
\z 

The demonstrative determiners are homophonous with the intensity and attitude adverbs (see \chapref{chap:modification}, \sectref{sec:secondpositionadverbs} on second-position adverbs). However, the adverbs are intoned differently. Also, the adverbs distribute on the clause level, rather than the RP level. The determiner \textit{an} happens to be homophonous with the shortened form of \textit{naan} the definite location marker:
\ea
\textbf{An}	kay	dya	gigma. \\\smallskip
 \gll \textbf{An}	kay	dya	ga-igma. \\
\textsc{spat.def} 	1\textsc{p.excl.abs}	\textsc{d}4\textsc{loc}	\textsc{i.r}-lunch \\
\glt ‘\textbf{There} we had lunch.’ [BMON-C-05 8.6]
\z
\ea
Tabangan	a	no	tak	yaken	i	pila	adlaw	en	\textbf{an}	gettek ta	sidda	i. \\\smallskip
 \gll \emptyset{}-Tabang-an	a	no	tak	yaken	i	pila	adlaw	en	\textbf{an}	gettek ta	sidda	i. \\
\textsc{t.ir}-help-\textsc{apl}	1\textsc{s.abs}	2\textsc{s.erg}	because	1\textsc{s.abs}	\textsc{def.n}	few	sun/day	\textsc{cm}	\textsc{spat.def}	stomach \textsc{nabs}	fish	\textsc{def.n} \\
\glt ‘Help me because as for me for few days I was \textbf{there in} the stomach of the fish.’ [CBON-T-04 6.9]
\z
\is{Referring Phrases|)}

% \begin{verbatim}%%move bib entries to  localbibliography.bib
% \end{verbatim}warning
