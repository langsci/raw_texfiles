\chapter{Pragmatically marked structures}
\label{chap:pragmaticallymarkedstructures}

\section{Introduction}
\label{sec:introduction-11}

Up to this point we have been concerned primarily with Kagayanen constructions whose prototypical function is to express affirmative assertions in compact information packages (see, e.g., \citealt{chafe1976} on information packaging in discourse). In this chapter we will describe several constructions that deviate from this prototype in one way or another, and so may be characterized as “pragmatically marked”\is{pragmatically marked constructions} \citep[261--305]{payne1997}. First we will describe various types of \isi{negation} (\sectref{bkm:Ref445900201}). Then we will discuss \isi{non-declarative speech acts} (\sectref{bkm:Ref445900286}), that is, constructions whose functions are to elicit information (\isi{interrogative constructions}, \sectref{bkm:Ref445900289} and \sectref{bkm:Ref445900292}), and those whose primary function is to encourage or direct other people to act (\isi{imperative constructionss}, \sectref{bkm:Ref445900361}). Next, we will discuss several pragmatically marked construction types that are used to ascribe different types of special prominence to parts of a clause (\sectref{bkm:Ref445900433}). These include \isi{argument fronting} (\sectref{bkm:Ref445900436}), special \isi{discourse particles} (\sectref{bkm:Ref118460437}), and \isi{cleft constructions} (\sectref{sec:cleftconstructions}). Finally, we discuss certain \isi{interjections} (\sectref{bkm:Ref113971883}) and the general pro-form \textit{kwa} ‘thingamajig’ (\sectref{bkm:Ref113971891}).

\section{Negative constructions}
\label{bkm:Ref445900201} \label{sec:negatives}
\is{negation|(}
There are two particles in Kagayanen that express clausal negation\is{negation!clausal}\is{clausal negation}. \textit{Uļa} negates existential clauses and clauses in realis modalities (\sectref{bkm:Ref80510883}). \textit{Dili} negates non-ver\-bal clauses other than existentials, individual words, and clauses in irrealis modalities (\sectref{bkm:Ref446684276}). These particles may be considered \isi{adjunct adverbs}, as discussed in \chapref{chap:modification}, \sectref{sec:adjunctadverbs}. Like all adjunct adverbs, both negative particles attract \isi{second-position enclitics} and \isi{clitic pronouns}.

\subsection{Realis negative: \textit{uļa}}
\label{bkm:Ref80510883} \is{negation!realis|(}
The adjunct adverb \textit{uļa} negates verbal clauses in realis moods (examples \ref{bkm:Ref446045407}--\ref{bkm:Ref373140065}). Like other preverbal adjunct adverbs, \textit{uļa} attracts second position adverbs and enclitics. In examples \REF{bkm:Ref230081554} and \REF{bkm:Ref80941165}, the ergative enclitic pronouns \textit{din} and \textit{ko} follow \textit{uļa.} In example \REF{bkm:Ref80941209} both \textit{din} and the referential expression \textit{kami i} follow \textit{uļa}:

\ea
\label{bkm:Ref230081554}\label{bkm:Ref446045407}
Mangngod  din  ya  a  \textbf{uļa}  din  gid  tagi. \\\smallskip
\gll Mangngod  din  ya  a  \textbf{uļa}  din  gid  ...-atag-i. \\
younger.sibling  3\textsc{s.erg}  \textsc{def.f}  \textsc{ctr}  \textsc{neg.r}  3\textsc{s.erg}  \textsc{int}  \textsc{t.r}-give-\textsc{xc.apl} \\
\glt ‘His younger sibling he did not give (him anything)!’ [RBWN-T-02 2. 8]
\z
\ea
\label{bkm:Ref80941165}
… nademdeman  ko  inay  daw  amay  ko  na \textbf{uļa}  ko  pamatian  iran  na  ambaļ. \\\smallskip
\gll … na-demdem-an  ko  inay  daw  amay  ko  na \textbf{uļa}  ko  pa-mati-an  iran  na  ambaļ. \\
 {} \textsc{a.hap.r}-remember-\textsc{apl}  1\textsc{s.erg}  mother  and  father  1\textsc{s.gen}  \textsc{lk}
\textsc{neg.r}  1\textsc{s.erg}  \textsc{t.r}-hear-\textsc{apl}  3\textsc{p.gen}  \textsc{lk}  say \\
\glt ‘… I remembered my mother and father, that I did not obey what they said.’ [PMWN-T-02  2.13]
\z
\ea
\label{bkm:Ref80941209}
Salamat  nang  man  ta  Dios  tak  \textbf{uļa}  din  kami  i  patandega. \\\smallskip
\gll Salamat  nang  man  ta  Dios  tak  \textbf{uļa}  din  kami  i  pa-tandeg-a. \\
thank  only  \textsc{emph}  \textsc{nabs}  God  because  \textsc{neg.r}  3\textsc{s.erg}  1\textsc{p.excl.abs}  \textsc{def.n}  \textsc{t.r}-touch-\textsc{xc} \\
\glt ‘Thanks to God also because he did not touch us!’ (This is a text about an attack by pirates.) [BCWN-C-04 5.4]
\z

Examples \REF{bkm:Ref80941234} through \REF{bkm:Ref373140065} illustrate various \isi{second-position adverbs} following \textit{uļa}:

\ea
\label{bkm:Ref80941234}
Duma  an  \textbf{uļa}  man  gatanem  ta  guso. \\\smallskip
\gll Duma  an  \textbf{uļa}  man  ga-tanem  ta  guso. \\
some  \textsc{def.m}  \textsc{neg.r}  \textsc{emph}  \textsc{t.r}-plant  \textsc{nabs}  seaweed \\
\glt ‘Some do not really plant agar-agar seaweed.’ [ETOP-C-08 3.12]
\z
\ea
… daw  \textbf{uļa}  din  en  nakita  yo  na  mga  kabaw daw  mga  lengngessa. \\\smallskip
\gll … daw  \textbf{uļa}  din  en  na-kita  yo  na  mga  kabaw daw  mga  lengngessa \\
{} and  \textsc{neg.r}  3\textsc{s.erg}  \textsc{cm}  \textsc{a.hap.r}-see  \textsc{d4abs}  \textsc{lk}  \textsc{pl}  water.buffalo
and  \textsc{pl}  blood \\
\glt ‘…and he did not see those water buffalo and blood.’ [PBWN-C-13 11.5]
\z
\ea
Tapos  asta  anduni  \textbf{uļa}  pa  man  danen  pakamang  iran  na  mga  order tak  \textbf{uļa}  kon  danen   kwarta… \\\smallskip
\gll Tapos  asta  anduni  \textbf{uļa}  pa  man  danen  pa-kamang  iran  na  mga  order tak  \textbf{uļa}  kon  danen   kwarta … \\
then  until  now/today  \textsc{neg.r}  \textsc{inc}  \textsc{emph}  3\textsc{p.erg}  \textsc{t.r}-get  3\textsc{p.erg}  \textsc{lk}  \textsc{pl}  order
because  \textsc{neg.r}  \textsc{hsy}  3\textsc{p.abs}  money \\
\glt ‘Then even until now/today they did not yet get what they ordered because they say they have no money …’ [AFWL-L-01 5.13]
\z

\ea
Tama  a  man  nailingan  na  mga  lugar  na  \textbf{uļa}  ko  pa nailingan. \\\smallskip
\gll Tama  a  man  na-iling-an  na  mga  lugar  na  \textbf{uļa}  ko  pa na-iling-an. \\
many 1\textsc{s.abs}  \textsc{emph}  \textsc{a.hap.r}-go-\textsc{apl}  \textsc{lk}  \textsc{pl}  place  \textsc{lk}  \textsc{neg.r} 1\textsc{s.erg}  \textsc{inc} \textsc{a.hap.r-}go\textsc{-apl} \\
\glt ‘There are many places I have gone (lit. I have many gone-to places) that I had not yet gone to.’ [SLWN-C-01 10.3]
\z


When a noun or pronoun is replaced with the existential \textit{may} (see \chapref{chap:non-verbalclauses}, \sectref{sec:existentialconstructions}), the clause contains both \textit{uļa} and \textit{may}:

\ea
\label{bkm:Ref373140065}
Pila  kon  bisis  tilipunuan  danen  piro  \textbf{uļa}  \textbf{may}  \textbf{gasabat}. \\\smallskip
\gll Pila  kon  bisis  tilipuno-an  danen  piro  \textbf{uļa}  \textbf{may}  \textbf{ga-sabat}. \\
few  \textsc{hsy}  times  telephone-\textsc{apl}  3\textsc{p.erg}  but  \textsc{neg.r}  \textsc{ext.in}  \textsc{i.r}-answer \\
\glt ‘A few times they telephoned (him/her) but no one answered.’ [PBWL-T-07 3.3]
\z

\textit{Uļa} replaces the existential particles \textit{may} and \textit{anen} (see \chapref{chap:non-verbalclauses}, \sectref{sec:existentialconstructions}) in negative existential clauses:

\ea
\textbf{Uļa}  kan-en  gid  tak  \textbf{uļa}  ittaw  dya  i. \\\smallskip
\gll \textbf{Uļa}  kan-en  gid  tak  \textbf{uļa}  ittaw  dya  i. \\
\textsc{neg.r}  cooked.rice  \textsc{int}  because  \textsc{neg.r}  person  \textsc{d}4\textsc{loc}  \textsc{att} \\
\glt ‘There is really no cooked rice because there are no people there.’ [MOOE-C-01 5]
\is{negation!realis|)}
\z
\subsection{ Irrealis negative: \textit{dili}}
\label{bkm:Ref446684276} \is{negation!irrealis|(}
The form \textit{dili} negates all types of irrealis clauses (examples \ref{bkm:Ref373243062}{}-\ref{bkm:Ref446617639}):

\ea
\label{bkm:Ref373243062}
Daw  balikiren  ko  kanen  ya,  \textbf{dili}  ko  man  nya makita. \\\smallskip
\gll daw  balikid-en  ko  kanen  ya,  \textbf{dili}  ko  man  nya ma-kita \\
if/when  look.back-\textsc{t.ir}  1\textsc{s.erg}  3\textsc{s.abs}  \textsc{def.f}  \textsc{neg.ir}  1\textsc{s.erg}  also  \textsc{d}4\textsc{abs} \textsc{a.hap.ir}-see \\
\glt ‘If I look back at him/her, I will not be able to see that one.’ [JCON-T-08 42.4]
\z
\ea
\textbf{Dili}  kay  gid  magsagbak,  Sir. \\\smallskip
\gll \textbf{dili}  kay  gid  mag-sagbak,  sir \\
\textsc{neg.ir}  1\textsc{p.excl.abs}  \textsc{int}  \textsc{i.ir}-noisy  sir \\
\glt ‘We really will not be noisy, Sir.’ [SFOB-L-01 3.4]
\z
\ea
Piro  daw  kapatay  ka  man  \textbf{dili}  a  makatabang  ki  kaon. \\\smallskip
\gll Piro  daw  ka-patay  ka  man  \textbf{dili}  a  maka-tabang  ki  kaon.' \\
but  if/when  \textsc{i.exm}-kill  2\textsc{s.abs}  also  \textsc{neg.ir}  1\textsc{s.abs}  \textsc{i.hap.ir}-help  \textsc{obl.p}  2s \\
\glt ‘But if you happened to kill someone, I cannot help you.’ [RBON-T-01 2.7]
\z
\ea
Nyan  en  ambaļ  ko  lecture  ko  ta  mga  bataan  ko na  \textbf{dili}  nyo  gid  labien  duma  na  ittaw  daw  labien nyo  gid  utod  nyo. \\\smallskip
\gll Nyan  en  ambaļ  ko  lecture  ko  ta  mga  bata-an  ko na  \textbf{dili}  nyo  gid  labi-en  duma  na  ittaw  daw  labi-en nyo  gid  utod  nyo. \\
\textsc{d}2\textsc{abs}  \textsc{cm}  say  1\textsc{s.erg}  lecture  1\textsc{s.erg}  \textsc{nabs}  \textsc{pl}  child-\textsc{nr}  1\textsc{s.ge} \textsc{lk}  \textsc{neg.ir}  2\textsc{p.erg}  \textsc{int}  favor-\textsc{t.ir}  other  \textsc{lk}  person  if/when  favor-\textsc{t.ir} 2\textsc{p.erg}  \textsc{int}  sibling  2\textsc{p.gen} \\
\glt `That is what I say, I lecture to my children that you do not ever favor other people and you really favor your own sibling(s).’ [RBON-T-01 7.1]
\z
\ea
Tama  na  mga  ittaw  na  gambaļ  na  \textbf{dili}  gid  kon  madayon ame  na  kasaļ  tak  basi  \textbf{dili}  a  kon  kapanaw. \\\smallskip
\gll Tama  na  mga  ittaw  na  ga-ambaļ  na  \textbf{dili}  gid  kon  ma-dayon ame  na  kasaļ  tak  basi  \textbf{dili}  a  kon  ka-panaw. \\
many  \textsc{lk}  \textsc{pl}  person  \textsc{lk}  \textsc{i.r}-say  \textsc{lk}  \textsc{neg.ir}  \textsc{int}  \textsc{hsy}  \textsc{a.hap.ir}-continue
1\textsc{p.excl.gen}  \textsc{lk}  wedding  because  maybe  \textsc{neg.ir}  1\textsc{s.abs}  \textsc{hsy}  \textsc{i.exm}-go/walk \\
\glt `Many people said that our wedding should not really continue because maybe I could not walk.’ [VAWN-T-163.4]
\z
\ea
\label{bkm:Ref446617639}
Ambaļ  ko,  “Daw  may  tirador  ka  dyan  \textbf{dili}  no  tiraduron tak  yan,  yupan  i,  yan  gatabyang  ta  ate  na  uma  daw tanto  na  danen  i  makakaan  ta  luod.” \\\smallskip
\gll Ambaļ  ko,  “Daw  may  tirador  ka  dyan  \textbf{dili}  no  tirador-en tak  yan,  yupan  i,  yan  ga-tabyang  ta  ate  na  uma  daw tanto  na  danen  i  maka-kaan  ta  luod.” \\
say  1\textsc{s.erg}  if/when  \textsc{ext.in}  slingshot  2\textsc{s.abs}  \textsc{d}2\textsc{loc}  \textsc{neg.ir}  2\textsc{s.erg}  slingshot-\textsc{t.ir}
because  \textsc{d}2\textsc{abs}  bird  \textsc{def.n}  \textsc{d}2\textsc{abs}  \textsc{i.r}-help  \textsc{nabs}  1\textsc{p.incl.gen}  \textsc{lk}  field  and reason.why  \textsc{lk}  3\textsc{p.abs}  \textsc{def.n}  \textsc{i.hap.ir}-eat  \textsc{nabs}  worm \\
\glt `I said, “If you have a slingshot there, do not shoot (it) with the slingshot because that one, the bird, that one helps our fields and for the reason they can eat worms.”' [MEWN-T-02 3.2]
\z

The form \textit{dili} also negates predicate nominal and other \isi{non-verbal predicates}, except existentials (see the functions of \textit{uļa} above). With non-verbal clauses the negative marker occurs before the Predicate, in both Predicate+Argument (exs. \ref{bkm:Ref80941751} through \ref{bkm:Ref80941878}) and Argument+Predicate structures (exs. \ref{bkm:Ref113700006} and \ref{bkm:Ref80941881}).

\ea
\label{bkm:Ref80941751}
{}... \textbf{dili}  man  kon  pelles  angin  an. \\\smallskip
\gll ... \textbf{dili}  man  kon  pelles  angin  an. \\
{} \textsc{neg.ir}  also  \textsc{hsy}  strong  wind  \textsc{def.m} \\
\glt ‘ ... the wind was not even strong, they say.’ [EMWN-T-06 5.3]
\z
\ea
\textbf{Dili}  maklaro  agi  din  ya  daw  indi  gagi. \\\smallskip
\gll \textbf{Dili}  ma-klaro  agi  din  ya  daw  indi  ga-agi. \\
\textsc{neg.ir}  \textsc{adj}-clear  path  3\textsc{s.gen}  \textsc{def.f}  if/when  where  \textsc{i.r}-path \\
\glt ‘His path is not clear where he went.’ [RCON-L-01 5.5]
\z
\ea
\label{bkm:Ref80941878}
\textbf{Dili}  keļeng  buok  din  an. \\\smallskip
\gll \textbf{Dili}  keļeng  buok  din  an. \\
\textsc{neg.ir}  curly  shair  3\textsc{s.gen}  \textsc{def.m} \\
\glt ‘Her hair is not curly.’  [CBOE-C-01 1.13]
\z
\ea
\label{bkm:Ref113700006}
Isya  din  na  mari  a  maļbaļ  daw  isya  ya  a  \textbf{dili}  maļbaļ. \\\smallskip
\gll Isya  din  na  mari  a  maļbaļ  daw  isya  ya  a  \textbf{dili}  maļbaļ. \\
one  \textsc{3sgen}  \textsc{lk}  godmother  \textsc{inj}  witch  and  one  \textsc{def.f}  \textsc{inj}  \textsc{neg.ir}  witch \\
\glt ‘One of her godmothers is a witch and the (other) one is not a witch.’ [CBWN-C-13 3.2]
\z
\ea
\label{bkm:Ref80941881}
Paryo  ta  makay  ta  lunday,  “manaog”  \textbf{dili}  man  igo. \\\smallskip
\gll Paryo  ta  m-sakay  ta  lunday,  “m-panaog” \textbf{dili}  man  igo. \\
like  \textsc{nabs}  \textsc{i.v.ir}-ride  \textsc{nabs}  outrigger.canoe  \textsc{i.v.ir}-go.down.stairs \textsc{neg.ir}  also  right \\
\glt ‘Like riding an outrigger canoe, “\textit{manaog}” (going down stairs) is not right.’ (This is about the words \textit{makay ta lunday}, which do not collocate with \textit{manaog} ‘go down stairs’ because the verb should be \textit{mawas} ‘disembark’.) [RCON-L-03 21.12] 
\z

Like other adjunct adverbs, \textit{dili} attracts enclitic pronouns \REF{bkm:Ref446051860} and second-position adverbs\is{enclitics}(\ref{bkm:Ref373229895}).

\ea
\label{bkm:Ref446051860}
Dayon  a  ambaļ,  “Pedro,  \textbf{dili}  \textbf{a}  lain  na  ittaw.” \\\smallskip
\gll Dayon  a  ambaļ,  “Pedro,  \textbf{dili}  \textbf{a}  lain  na  ittaw.” \\
right.away  1\textsc{s.abs}  say  Pedro  \textsc{neg.ir}  1\textsc{s.abs}  different  \textsc{lk}  person \\
\glt ‘Right away I said, “Pedro, I am not a different/another person.”' [EFWN-T-11 15.5]
\z
\ea
\label{bkm:Ref373229895}
… \textbf{dili}  \textbf{pa}  yaken  sagad. \\\smallskip
\gll … \textbf{dili}  \textbf{pa}  yaken  sagad. \\
{} \textsc{neg.ir}  \textsc{inc}  1\textsc{s.abs}  skillful \\
\glt ‘… I am not yet skillful.’ [MAWL-C-03 8.1]
\is{negation!irrealis|)}
\z
\subsection{Constituent negation}
\label{sec:constituentnegation} \is{negation!constituent|(}\is{constituent negation|(}
\textit{Constituent negation} is a clause type in which a constituent of a clause is negated, rather than the clause itself. Examples in English include:

\ea
No-one arrived. \\
Not many people attended the party. \\
She planted no trees. \\
None of our students received an award.
\z

Examples of constituent negation in Kagayanen can always be interpreted as cleft constructions (see  \sectref{sec:cleftconstructions}). This is because Kagayanen employs no copula in \isi{predicate nominal constructions}. Thus a sentence like “She planted no trees” can always be understood literally as “No trees are what she planted”. Not surprisingly, then, \textit{dili} is used in such contexts, since clefts are a kind of predicate nominal construction:

\ea
Yi  \textbf{dili}  nyog  tanem  din.  Iya  a  mangga. \\\smallskip
\gll Yi  \textbf{dili}  nyog  tanem  din.  Iya  a  mangga. \\
\textsc{d1abs}  \textsc{neg.ir}  coconut  plant  3\textsc{s.erg}  3\textsc{s.gen}  \textsc{ctr}  mango \\
\glt ‘These ones (the parents), what they planted was not coconut. As for them, (what they planted was) mango.’\footnote{In this narrative, the speaker uses singular pronouns (\textit{din}, \textit{iya}) to refer to two people, parents, who acted together. This is a common discourse usage.} [SFOB-L-01 5.2]
\z
\ea
Anduni  uļa  gapati  mga  ittaw  tak  \textbf{dili}  Apo  Kagiyaw gibit  ta  angin  i daw  dili,  ate  na  Maal  na Dios. \\\smallskip
\gll Anduni  uļa  ga-pati  mga  ittaw  tak  \textbf{dili}  Apo  Kagiyaw ga-ibit  ta  angin  i daw  dili,\footnotemark{}  ate  na  Maal  na Dios. \\
now/today  \textsc{neg.r}  \textsc{i.r}-believe  \textsc{pl}  person  because  \textsc{neg.ir}  Apo  Kagiyaw
\textsc{i.r}-hold  \textsc{nabs}  wind/air  \textsc{def.n}
if/when  \textsc{neg.ir}  1\textsc{p.incl.erg}  \textsc{lk}  beloved  \textsc{lk} God \\
\footnotetext{\textit{Daw dili} is an idiomatic expression translated freely in English as “but rather.” The following elliptical clause is affirmative in every respect.}
\glt `Now/today people do not believe (it) because it is not Ancestor Kagiyaw holding the wind but rather, our Beloved God (holds the wind).’ [VAWN-T-17 5.2]
\z
\ea
Dili  magtakaw  ta  \textbf{dili}  imo  na  ampangan. \\\smallskip
\gll Dili  mag-takaw  ta  \textbf{dili}  imo  na  ampang-an. \\
\textsc{neg.ir}  \textsc{i.ir}-steal  \textsc{nabs}  \textsc{neg.ir}  2\textsc{s.gen}  \textsc{lk}  play-\textsc{nr} \\
\glt ‘Do not steal what is not your play thing.’ [LBOP-C-04 1.4]
\z
\ea
Pakaan  danen  iran  \textbf{dili}  iling  na  beggas  daw  \textbf{dili} bunga  ta  kaoy  na  mga  kan-enen  na  dili  makilo. \\\smallskip
\gll Pa-kaan  danen  iran  \textbf{dili}  iling  na  beggas  daw  \textbf{dili} bunga  ta  kaoy  na  mga  kan-en-en  na  dili  ma-kilo. \\
\textsc{t.r}-eat  3\textsc{p.erg}  3\textsc{p.gen}  \textsc{neg.ir}  like  \textsc{lk}  milled.rice  if/when  \textsc{neg.ir}
fruit  \textsc{nabs}  tree  \textsc{lk}  \textsc{pl}  cooked.rice-\textsc{t.ir}  \textsc{lk}  \textsc{neg.ir}  \textsc{a.hap.ir}-kilogram \\
\glt `What they ate was not like rice but rather, fruit of trees which was food that cannot be weighed.’ [PTOE-T-01 3.5]
\z

The following is a riddle containing two negative predicate nominal constructions:

\ea
\textbf{Dili}  ittaw.  \textbf{Dili}  ayep.  Naļam  mambaļ  ta  maskin  ino. Radyo. \\\smallskip
\gll \textbf{Dili}  ittaw.  \textbf{Dili}  ayep.  Na-aļam  m-ambaļ  ta  maskin  ino. Radyo. \\
\textsc{neg.ir}  person  \textsc{neg.ir}  animal  \textsc{a.hap.r}-know  \textsc{i.v.ir}-say  \textsc{nabs}  even  what radio \\
\glt ‘(It is) not a person. (It is) not an animal. (It) knows how to say anything. Radio.’ [MRWR-T-01 22.1]
\z

Adverbs may be negated, even if the main predicate is not negated. This is also a context for the use of \textit{dili}:

\ea
Tapos  uļa  din  tuturi  lampraan  tak  ambaļ  ta  mga manakem  \textbf{dili}  kon  anay  tuturan  lampraan  daw    narem. \\\smallskip
\gll Tapos  uļa  din  ...-tutod-i  lampraan  tak  ambaļ  ta  mga manakem  \textbf{dili}  kon  anay  \emptyset{}-tutod-an  lampraan  daw narem. \\
then  \textsc{neg.r}  3\textsc{s.erg}  \textsc{t.r}-light-\textsc{xc.apl}  lamp  because  say  \textsc{nabs}  \textsc{pl}
older  \textsc{neg.ir}  \textsc{hsy}  first/for.a.while  \textsc{t.ir}-light-\textsc{apl}  lamp  if/when sleep.paralysis \\
\glt ‘Then he did not light the lamp because the older people say do not first light the lamp if/when (someone) has sleep paralysis.' [ETON-C-07 4.4]
\z

In this example, \textit{dili} in the second clause negates the adverb \textit{anay} ‘first/for awhile’. The older people do say to light the lamp when someone has sleep paralysis, but you are not supposed to light the lamp first. You must wake the person up first, then light the lamp.

\ea
\textbf{Dili}  nay  gid  pirmi  buat  na  iling  tan. \\\smallskip
\gll \textbf{Dili}  nay  gid  pirmi  buat  na  iling  tan. \\
\textsc{neg.ir}  1\textsc{p.excl.erg}  \textsc{int}  always  make/do  \textsc{lk}  like  \textsc{d3nabs} \\
\glt ‘We really do not always do like that.’ (elicited – a variation of example \ref{bkm:Ref118463113})
\z

\ea
\textbf{Dili}  ko  enged  naisturbuan  ka  ta  imo  na  pagtunuga… \\\smallskip
\gll \textbf{Dili}  ko  enged  na-isturbo-an  ka  ta  imo  na  pag-tunuga… \\
\textsc{neg.ir}  1\textsc{s.erg}  intentionally  \textsc{a.hap.r}-disturb-\textsc{apl}  2\textsc{s.abs}  \textsc{nabs}  2\textsc{s.gen}  \textsc{lk}  \textsc{nr.act}-sleep \\
\glt ‘I did not intentionally disturb you in your sleeping….’ (The mouse passed by the lion waking him  up, though the mouse didn't intend to disturb him.) [BCWN-T-05 1:4]
\z
\ea
… ingkantado  i  \textbf{dili}  gulpi  mag-atag  ta  masakit. \\\smallskip
\gll … ingkantado  i  \textbf{dili}  gulpi  mag-atag  ta  masakit. \\
{} fairy  \textsc{def.n}  \textsc{neg.ir}  suddenly  \textsc{i.ir}-give  \textsc{nabs}  sick \\
\glt ‘… fairies do not suddenly give sickness.’ (It is believed that fairies can give sickness but it is gradual not sudden) [CBWE-T-07 4.1]
\z
\ea
\label{bkm:Ref118463113}
Yon  buat  nay  kaysan  piro  \textbf{dili}  gid  man  pirmi. \\\smallskip
\gll Yon  buat  nay  kaysan  piro  \textbf{dili}  gid  man  pirmi. \\
\textsc{d}3\textsc{abs}  do/make  1\textsc{p.excl.erg}  sometimes  but  \textsc{neg.ir}  \textsc{int}  \textsc{emph}  always \\
\glt ‘That is what we do sometimes but not really always.’ [EMWE-T-01 3.6]
\z

A time phrase may also be negated with \textit{dili}. This is similar to constituent negation in that such clauses may always be considered a kind of predicate nominal. In the following examples, the negated time phrase is bolded in the English free translation:

\ea
Gambaļ  danen  na  pagtakaw  tak  beet  ambaļen  dili  pa pwidi  lungien  tak  ilaw  pa  o  \textbf{dili}  pa  oras  ta  lungi. \\\smallskip
\gll Ga-ambaļ  danen  na  pag-takaw  tak  beet  ambaļ-en  dili  pa pwidi  lungi-en  tak  ilaw  pa  o  \textbf{dili}  pa  oras  ta  lungi. \\
\textsc{i.r}-say  3\textsc{p.abs}  \textsc{lk}  \textsc{nr.act}-steal  because  means  say-\textsc{t.ir}  \textsc{neg.ir}  \textsc{inc}
can  harvest-\textsc{t.ir}  because  unripe  \textsc{inc}  or  \textsc{neg.ir}  \textsc{inc}  time/hour  \textsc{nabs}  harvest \\
\glt `They said ‘stealing’ because it means to say that (the corn) can’t yet be harvested because (it is) not yet ripe or \textbf{(it is) not yet time for harvesting corn}.’ [VAOE-J-07 4.5]
\z
\ea
Magbata  kanen  ta  \textbf{dili}  iya  na  oras. \\\smallskip
\gll Mag-bata  kanen  ta  \textbf{dili}  iya  na  oras. \\
\textsc{i.ir}-child  3\textsc{s.abs}  \textsc{nabs}  \textsc{neg.ir}  3\textsc{s.gen}  \textsc{lk}  time/hour \\
\glt ‘She gave birth when \textbf{(it was) not yet her time}.’ [VAOE-J-05 1.3]
\z
\ea
Piro  \textbf{dili}  man  tanan-tanan  na  oras  makailing  ka  na danen  an  miad  iran  na  isip  daw  dili  kaysan masinawayen  gid  man. \\\smallskip
\gll Piro  \textbf{dili}  man  tanan-tanan  na  oras  maka-iling  ka  na danen  an  miad  iran  na  isip  daw  dili  kaysan  ma-s<in>away-en  gid  man. \\
but  \textsc{neg.ir}  also  \textsc{red}-all  \textsc{lk}  time/hour  \textsc{i.hap.ir}-say  2\textsc{s.abs}  \textsc{lk}
3\textsc{p.abs}  \textsc{def.m}  good/kind  3\textsc{p.gen}  \textsc{lk}  think  if/when  \textsc{neg.ir}  sometimes 
\textsc{adj}-\textsc<{nr.res}>correct-\textsc{adj}  \textsc{int}  also \\
\glt `But \textbf{not all the time} can you say that as for them their thinking is good/kind but rather sometimes (their character) really is to be corrected.’ [EMWE-T-01 3.2]
\z

Prepositional phrases (bolded in the free translations) may also be negated with \textit{dili}:

\ea
\label{bkm:Ref373356719}
A  yan,  \textbf{dili}  nang  para  ki  kiten  anduni. \\\smallskip
\gll A  yan,  \textbf{dili}  nang  para  ki  kiten  anduni. \\
\textsc{inj}  \textsc{d}2\textsc{abs}  \textsc{neg.ir}  only  for  \textsc{obl.p}  1\textsc{p.incl}  now/today \\
\glt ‘(Expression of dismay) as for that, \textbf{(it is) not just for us now/today}.’ [EFOB-C-01 6.8]
\z
\ea
Magtanem  ki  ta  kaoy  na  \textbf{dili}  para  nang  ki  kiten  anduni. \\\smallskip
\gll Mag-tanem  ki  ta  kaoy  na  \textbf{dili}  para  nang  ki  kiten  anduni. \\
\textsc{i.ir}-plant  1\textsc{p.incl.abs}  \textsc{nabs}  tree  \textsc{lk}  \textsc{neg.ir}  for  only  \textsc{obl.p}  1\textsc{p.incl}  now/today \\
\glt ‘Let’s plant trees \textbf{not just for us now/today}.’ (elicited, variation of \ref{bkm:Ref373356719})
\z

It is also possible to negate one constituent within a noun phrase. For example, in \REF{bkm:Ref446156832}, only the modifier \textit{segeng} ‘extreme’ is negated. The assertion is that the Actor indeed had asthma attacks in former years but they weren’t as extreme as the current one.

\ea
\label{bkm:Ref446156832}
Piro  duma  ya  na  galambay  ya  na  tinaon \textbf{dili}  pa  gid segeng  na  pag-ataki  ta  apo. \\\smallskip
\gll Piro  duma  ya  na  ga-lambay  ya  na  t<in>aon \textbf{dili}  pa  gid segeng  na  pag--ataki  ta  apo. \\
but  other  \textsc{def.f}  \textsc{lk}  \textsc{i.r}-pass.by  \textsc{def.f}  \textsc{lk}  <\textsc{<nr.res>}year
\textsc{neg.ir}  \textsc{inc}  \textsc{int}
extreme  \textsc{lk}  \textsc{nr.act}-attack  \textsc{nabs}  asthma \\
\glt `But in other past years (they were) not yet very severe attacks of asthma.’ [JCWN-T-22 2.2]
\z
\ea
Sakit  mata  minog  na  daon  i  na  giting-giting  na  \textbf{dili}  gid  ļapad. \\\smallskip
\gll Sakit  mata  minog  na  daon  i  na  giting\sim giting  na  \textbf{dili}  gid  ļapad. \\
pain  eye  red  \textsc{lk}  leaf  \textsc{def.n}  \textsc{lk}  scalloped  \textsc{lk}  \textsc{neg.ir}  \textsc{int}  wide \\
\glt ‘For painful eyes, (use) the red leaf that is scalloped, that is not very wide.’ [DBOE-C-04 11.1]
\z

One constituent in an attributive clause can also be negated with \textit{dili}. In example \REF{bkm:Ref446664253} the meaning is that the governing is very bad:

\ea
\label{bkm:Ref446664253}
Iya  na  panggubirno  ta  mga  ittaw  sikad  gid  na  \textbf{dili}  usto. \\\smallskip
\gll Iya  na  pang-gubirno  ta  mga  ittaw  sikad  gid  na  \textbf{dili}  usto. \\
3\textsc{s.gen}  \textsc{lk}  \textsc{nr-}governor  \textsc{nabs}  \textsc{pl}  person  very  \textsc{int}  \textsc{lk}  \textsc{neg.ir}  well/right \\
\glt ‘His governing of people was very much not right.’ [BEWN-T-01 2.8]
\z

Example \REF{bkm:Ref373243525} is a variation of \REF{bkm:Ref446664253}, in which \textit{dili} is placed at the beginning of the predicate modifier. In this case, the meaning is ‘not very good’, that is, not as bad as in example \REF{bkm:Ref446664253}.

\ea
\label{bkm:Ref373243525}
Iya  na  panggubirno  ta  mga  ittaw  \textbf{dili}  sikad  gid  na  usto. \\\smallskip
\gll Iya  na  pang-gubirno  ta  mga  ittaw  \textbf{dili}  sikad  gid  na  usto. \\
3\textsc{s.gen}  \textsc{lk}  \textsc{nr-}governor  \textsc{nabs}  \textsc{pl}  person  \textsc{neg.ir}  very  \textsc{int}  \textsc{lk}  well/right \\
\glt ‘His governing of people was not really very right.’
\z

Negated hypothetical assertions and assertions that are negative for all times are also often expressed with the predicating verb in an action nominal form with \textit{pag}{}- (see \chapref{chap:referringexpressions}, \sectref{sec:pag}) and sometimes with the nominalizing suffix -\textit{én} or the applicative suffix -\textit{an}.

\ea
Daw  gaumaw  isya  na  manakem  \textbf{dili}  \textbf{pagsabatén}  na  ee tak  magilek. \\\smallskip
\gll Daw  ga-umaw  isya  na  manakem  \textbf{dili}  \textbf{pag-sabat-én}  na  ee tak  ma-gilek. \\
if/when  \textsc{i.r}-call  one  \textsc{lk}  older  \textsc{neg.ir}  \textsc{nr.act}-answer-\textsc{nr}  \textsc{lk}  yes
because  \textsc{a.hap.ir}-angry \\
\glt `When an older person calls (you) do not answer ‘yes’, because  (s/he) will become angry.’ (lit. (Your) answering is not “yes”) [LBOP-C-04 1.6]
\z

\newpage

\ea
\label{bkm:Ref118464902}
… Gapangako  ki  na  dili  ta  \textbf{pagtakkarén} ta  basak ate  na  bata. \\\smallskip
\gll … Ga-pangako  ki  na  dili  ta  \textbf{pag-takkad-én} ta  basak ate  na  bata. \\
{} \textsc{i.r}-promise  1\textsc{p.incl.abs}  \textsc{lk}  \textsc{neg.ir}  1\textsc{p.lincl.gen}  \textsc{nr.act}-step.on-\textsc{nr}
\textsc{nabs}  ground 1\textsc{p.incl.gen}  \textsc{lk}  child \\
\glt `’…We promised that we will \textbf{not} \textbf{let} \textbf{our} \textbf{child} \textbf{step} on the ground.’ [CBWN-C-14 2.8]
\z
\ea
Pamikawan  bata  an  aged  magdayad  pagdako  din  an, dili  maagian  ta  lain  na  lawa,  \textbf{dili}  \textbf{pag-ikawan} ta  iya  na  pangabui  daw  dili  mapintasan  ta  duma. \\\smallskip
\gll Pa-mikaw-an  bata  an  aged  mag-dayad  pag-dako  din  an, dili  ma-agi-an  ta  lain  na  lawa,  \textbf{dili}  \textbf{pag-ikaw-an} ta  iya  na  pangabui  daw  dili  ma-pintas-an  ta  duma. \\
\textsc{t.r}-food.sacrifice-\textsc{apl}  child  \textsc{def.m}  so.that  \textsc{i.ir}-good  \textsc{nr.act}-large  3\textsc{s.gen}  \textsc{def.m}
\textsc{neg.ir}  \textsc{a.hap.ir}-pass-\textsc{apl}  \textsc{nabs}  bad  \textsc{lk}  body  \textsc{neg.ir}  \textsc{nr.act}-curse-\textsc{apl}
\textsc{nabs}  3\textsc{s.gen}  \textsc{lk}  living  if/when  \textsc{neg.ir}  \textsc{a.hap.ir}-hurt-\textsc{apl}  \textsc{nabs}  other \\
\glt `Food sacrifice is done for the child (first born son) so that his/her becoming big will be good, (he) cannot experience a bad body (sickness), (his) life \textbf{will} \textbf{not} \textbf{ever} \textbf{be} \textbf{cursed} and others cannot hurt (him).’ [JCWE-T-16 2.4]
\z

\ea
… daw  gasumpa  kanen  na  \textbf{dili}  \textbf{din}  \textbf{kon}  \textbf{paglipatan}  yo na  dļaga. \\\smallskip
\gll … daw  ga-sumpa  kanen  na  \textbf{dili}  \textbf{din}  \textbf{kon}  \textbf{pag-lipat-an}  yo na  dļaga. \\
{} and  \textsc{i.r}-vow  3\textsc{s.abs}  \textsc{lk}  \textsc{neg.ir}  3\textsc{s.gen}  \textsc{hsy}  \textsc{nr.act}-forget-\textsc{apl}  \textsc{d4abs}
\textsc{lk}  single.woman \\
\glt `… and he vowed that \textbf{he} \textbf{will} \textbf{not} \textbf{ever} \textbf{forget} that single woman.’ [CBWN-C-14 5.9]
\z

Example \REF{bkm:Ref446680856} expresses approximately the same meaning, but with a negated verb inflected in irrealis mood rather than an action nominalization. The same usage appears twice in \REF{bkm:Ref118464902} above (unhighlighted), in irrealis happenstantial modalities:

\newpage
\ea
\label{bkm:Ref446680856}
… daw  gasumpa  kanen  na  \textbf{dili}  \textbf{din}  \textbf{kon}  \textbf{lipatan}  yo na  dļaga. \\\smallskip
\gll … daw  ga-sumpa  kanen  na  \textbf{dili}  \textbf{din}  \textbf{kon}  \emptyset{}-\textbf{lipat-an}  yo na  dļaga. \\
{} and  \textsc{i.r}-vow  3\textsc{s.abs}  \textsc{lk}  \textsc{neg.ir}  3\textsc{s.gen}  \textsc{hsy}  \textsc{t.ir}-forget-\textsc{apl}  \textsc{d4abs}
\textsc{lk}  single.woman \\
\glt `… and he vowed that \textbf{he} \textbf{will} \textbf{not} \textbf{forget} that single woman.’
\z

\ea
Manang  \textbf{dili}  \textbf{no}  \textbf{lipatan}  ake  na  penpal. \\\smallskip
\gll Manang  \textbf{dili}  \textbf{no}  \emptyset{}-\textbf{lipat-an}  ake  na  penpal. \\
older.sister  \textsc{neg.ir}  2\textsc{s.gen}  \textsc{t.ir}-forget- \textsc{apl}  1\textsc{s.gen}  \textsc{lk}  pen.pal \\
\glt ‘Older sister \textbf{do} \textbf{not} \textbf{forget} my pen pal.’ (This does not mean to always not forget her pen pal but it means to not forget to get a pen pal for her now.) [BCWL-C-01 3.25]
\z

Some stative predicates only occur with \textit{dili}; they cannot occur with \textit{uļa} even in realis moods. Such predicates include \textit{gusto} ‘to want/like’ followed by a complement clause, \textit{kinangļan} ‘necessary’ with a complement clause, \textit{dapat} ‘must’ with complement clauses, and others:

\ea
Piro  \textbf{dili}  \textbf{ko}  \textbf{gusto}  na  magsawa  ki  kaon… \\\smallskip
\gll Piro  \textbf{dili}  \textbf{ko}  \textbf{gusto}  na  mag-sawa  ki  kaon… \\
but  \textsc{neg.ir}  1\textsc{s.erg}  want/like  \textsc{lk}  \textsc{i.ir}-spouse  \textsc{obl.p}  2s \\
\glt ‘But \textbf{I} \textbf{do} \textbf{not} \textbf{want} to marry you ...’ [CBWN-C-17 3.21] \\\smallskip
*Piro uļa ko gusto na magsawa ki kaon.
\z

\ea
\textbf{Dili}  \textbf{kinangļan}  na  kiten  magtanem  unduni  daan… \\\smallskip
\gll \textbf{Dili}  \textbf{kinangļan}  na  kiten  mag-tanem  unduni  daan… \\
\textsc{neg.ir}  necessary  \textsc{lk} 1\textsc{p.incl.abs}  \textsc{i.ir}-plant  now/today  ahead.of.time \\
\glt ‘\textbf{It} \textbf{is} \textbf{not} \textbf{necessary} that we plant now/today ahead of time…’ [ROOB-T-01 10.17] \\\smallskip
*Uļa kinangļan na kiten magtanem unduni daan...
\z
\ea
\textbf{Dili}  \textbf{ki}  \textbf{dapat}  na  magbeļag… \\\smallskip
\gll \textbf{Dili}  \textbf{ki}  \textbf{dapat}  na  mag-beļag… \\
\textsc{neg.ir}  1\textsc{p.incl.abs}  must  \textsc{lk}  \textsc{i.ir}-separate \\
\glt `\textbf{We} \textbf{must} \textbf{not} separate…’ [EMWN-T-09 5.16] \\\smallskip
*Uļa ki dapat na magbelag tak.
\is{negation!constituent|)}\is{constituent negation|)}
\z
\subsection{Rhetorical confirmation, \textit{di ba}}
\label{bkm:Ref113630514}\label{sec:rhetoricalconfirmation} \is{rhetorical confirmation|(}
The expression \textit{di ba} (a short form of \textit{dili} ‘\textsc{neg.ir’} plus the common Philippine question particle \textit{ba}) can occur clause-initially to indicate rhetorical confirmation (exs. \ref{bkm:Ref441346257}{}-\ref{bkm:Ref441346278}). As mentioned in \sectref{sec:interjections}, in Kagayanen \textit{ba} alone is a marker of rhetorical questions\is{rhetorical questions}, irony, or sarcasm. By using \textit{di ba} the speaker asserts something that s/he thinks is true or maybe not completely sure about and wants the addressee to consider it carefully or to confirm it as true. The assumed response to such expressions is “yes.” \textit{Di ba} may also occur in second position, or clause finally with approximately the same effect – a full discourse study would be needed to fully understand any nuances associated with the different possible positions for \textit{di ba}. Examples of tag questions with \textit{di ba} are given below in \sectref{sec:yesnoquestions}.

\ea
\label{bkm:Ref441346257}
\textbf{Di ba},  gambaļ  ka  na  dumaan  a  no? \\\smallskip
\gll \textbf{Di ba},  ga-ambaļ  ka  na  \emptyset{}-duma-an  a  no? \\
isn't.it.right  \textsc{i.r}-say  2\textsc{s.abs}  \textsc{lk}  \textsc{t.ir}-accompany-\textsc{apl}  1\textsc{s.abs}  2\textsc{s.erg} \\
\glt ‘Isn’t it right, you said you will accompany me?’
\z
\ea
Piro  \textbf{di ba},  ta  pagtabas  daw  ta  paglimpyo  kinangļan  gid na  listo  ka… \\\smallskip
\gll Piro  \textbf{di ba},  ta  pag-tabas  daw  ta  pag-limpyo  kinangļan  gid na  listo  ka… \\
but  isn't.it.right  \textsc{nabs}  \textsc{nr.act}-cut  and  \textsc{nabs}  \textsc{nr.act}-clean  necessary  \textsc{int}
\textsc{lk}  alert   2\textsc{s.abs} \\
\glt ‘But isn’t it right, when cutting (weeds) and in cleaning it is necessary that you are alert/aware…’[VBWE-T-04 2.4]
\z
\ea
\label{bkm:Ref441346278}
\textbf{Di ba},  pugya  yaken  i,  sawa  a  no? \\\smallskip
\gll \textbf{Di ba},  pugya  yaken  i,  sawa  a  no? \\
isn't.it.right  long.ago 1\textsc{s.abs}  \textsc{def.n}  spouse  1\textsc{s.abs}  2\textsc{s.gen} \\
\glt ‘Isn’t it right, long ago as for me, I am your spouse.’ (edited for clarity) [PBON-T-01 2.23] \\\smallskip
Also:
\textit{\textbf{Di} \textbf{ba}, pugya yaken i sawa a no?} \\
\textit{Pugya, \textbf{di ba}, yaken i, sawa a no?} \\
\textit{Pugya yaken i, sawa a no, \textbf{di ba}?}
\z

\newpage
\ea
Duma  na  ambaļ  na  Kagayanen,  \textbf{di}  \textbf{ba,}  dili  masuļat? \\\smallskip
\gll Duma  na  ambaļ  na  Kagayanen,  \textbf{di ba,}  dili  ma-suļat? \\
some  \textsc{lk}  word  \textsc{lk}  Kagayanen  isn't.it.right  \textsc{neg.ir}  \textsc{a.hap.ir}-write \\
\glt ‘Some Kagayanen words, isn’t it right, cannot be written?’ [MOOE-C-01 28.2]’ \\\smallskip
\textit{*Duma na ambaļ na Kagayanen dili masuļat \textbf{di} \textbf{ba}?}
\is{rhetorical confirmation|)}
\is{negation|)}
\z
\section{Non-declarative speech acts}
\label{bkm:Ref445900286}
The term “declarative” in traditional grammar refers to clause types that speakers use to express information. Often the term “declarative mood” will be found in the literature. In this grammar, and in linguistics in general, this sense of the word “declarative” does not describe a mood at all (see \chapref{chap:verbstructure}, \sectref{sec:modality} on the expression of modality in Kagayanen). Rather, in the tradition of speech act theory, the term \textit{assertion}\is{assertion:speech act type!} most closely approximates the traditional notion of declarative mood. An assertion is a kind of speech act speakers perform when they want to express information to other people (see e.g. \citealt{searle1969}). This is a very common task that people perform using language, therefore assertions are usually the most common construction types in any language. Up to now, most of the examples and discussion in this grammar have dealt with declarative assertions. In this section we will describe some grammatical construction types that are used primarily to accomplish other, non-declarative, speech acts.  These speech acts include requesting information (\textit{interrogative constructions}\is{interrogative constructions}), and requesting or commanding action (\textit{imperative constructions}\is{imperative constructions}).

\subsection{Polar (yes/no) questions}
\label{bkm:Ref445900289} \label{sec:yesnoquestions}
Polar, or “yes/no”, questions are interrogative clauses for which the expected answer is either ‘yes’ or ‘no’. In Kagayanen, polar questions have the same syntax as assertions. The only difference between an assertion and a yes/no question is that assertions typically have falling intonation at the end of the sentence and yes/no questions have rising intonation at the end. This rising intonation is indicated in the orthography, and in this grammar, with a question mark:

\ea
Pagtapos,  guli  kanen  i  daw  painsaan  ta manakem  i  na,  “\textbf{Guli}  \textbf{ka}  \textbf{man}  \textbf{dayon}?” \\\smallskip
\gll Pag-tapos,  ga-uli  kanen  i  daw  pa-insa-an  ta manakem  i  na,  “\textbf{Ga-uli}  \textbf{ka}  \textbf{man}  \textbf{dayon}?” \\
\textsc{nr.act}-finish  \textsc{i.r}-go.home  3\textsc{s.abs}  \textsc{def.n}  and  \textsc{t.r}-ask-\textsc{apl}  \textsc{nabs}
older  \textsc{def.n}  \textsc{lk}  \textsc{i.r}-go.home  2\textsc{s.abs}  \textsc{emph}  immediately \\
\glt `Afterwards, he went home and the older person asked (him), “Did  you come home immmediately?”' (The older man was surprised that he came back home so soon.) [NEWN-T-07 6.3] 
\z

Yes/no questions can be identified also by the context. The following are some additional examples from the corpus.

\ea
Isya  man  na  gatindir  ta  yi  na  palumba,  paambaļan  din, “Pwikan,  \textbf{anda}  \textbf{ka}  \textbf{en?}”  Sabat  ta  Pwikan,  “Ee.”  Dason  painsaan man  Umang  i,  “Umang,  \textbf{anda}  \textbf{ka}  \textbf{en}  \textbf{ta}  \textbf{inyo}  \textbf{na} \textbf{palumbaanay?}”  Sabat  man  ta  Umang,  “Ee,  anda  aren  en.” \\\smallskip
\gll Isya  man  na  ga-tindir  ta  yi  na  pa-lumba,  pa-ambaļ-an  din, “Pwikan,  \textbf{anda}  \textbf{ka}  \textbf{en?}”  Sabat  ta  Pwikan,  “Ee.”  Dason  pa-insa-an man  Umang  i,  “Umang,  \textbf{anda}  \textbf{ka}  \textbf{en}  \textbf{ta}  \textbf{inyo}  \textbf{na} \textbf{pa-lumba-anay?}”  Sabat  man  ta  Umang,  “Ee,  anda  aren  en.” \\
one  also  \textsc{lk}  \textsc{i.r}-attend  \textsc{nabs}  \textsc{d}1\textsc{adj}  \textsc{lk}  \textsc{caus}-race  \textsc{t.r}-say-\textsc{apl}  3\textsc{s.erg}
sea.turtle  ready  2\textsc{s.abs}  \textsc{cm}  answer  \textsc{nabs}  sea.turtle  yes  next  \textsc{t.r}-ask-\textsc{apl}
also  hermit.crab  \textsc{def.n}  hermit.crab  ready 2\textsc{s.abs}  \textsc{cm}  \textsc{nabs}  2\textsc{gen}  \textsc{lk}
\textsc{caus}-race-\textsc{rec}  answer  also  \textsc{nabs}  hermit.crab  yes  ready  1\textsc{s.abs}  \textsc{cm} \\
\glt `One who attended this race, he said, “Sea Turtle, are you ready?” The Sea Turtle answered, “Yes.” Next the Hermit Crab was also asked, “Hermit Crab, are you ready for  your racing each other?” Hermit Crab answered, “Yes, I am now ready.”' [DBWN-T-26 4.6]
\z
\ea
\textbf{Kakita}  \textbf{ka}  \textbf{ta}  \textbf{umang?}  \textbf{Kakita}  \textbf{ka}  \textbf{ta}  \textbf{pwikan}? \\\smallskip
\gll \textbf{Ka-kita}  \textbf{ka}  \textbf{ta}  \textbf{umang?}  \textbf{Ka-kita}  \textbf{ka}  \textbf{ta}  \textbf{pwikan}? \\
\textsc{i.exm}-see  2\textsc{s.abs}  \textsc{nabs}  hermit.crab  \textsc{i.exm}-see  2\textsc{s.abs}  \textsc{nabs}  sea.turtle \\
\glt ‘Did you happen to see a hermit crab? Did you happen to see a  sea turtle?' [JCON-L-08 2.4-5]
\z

Sometimes polar questions include the tag \textit{di ba} mentioned above in \sectref{sec:rhetoricalconfirmation}. As a tag, it expresses a genuine question, whereas in other positions (clause initially or second position) the question is more likely to be sarcastic or ironic. \textit{Tag questions}\is{tag questions} function to remind people of something they already know, to help them to consider something carefully, and to ask for confirmation when the speaker is not quite sure.

\ea
May  ummay  ki,  \textbf{di ba?} \\\smallskip
\gll May  ummay  ki,  \textbf{di ba?} \\
\textsc{ext.in}  unmilled.rice  1\textsc{p.incl.abs}  isn’t.it.right \\
\glt ‘We have rice, isn’t it right?’ [JCOE-T-06 8.21]
\z
\ea
Uļa  a  gid  kasaļa  ta  ambaļ  ko  ya,  \textbf{di ba}? \\\smallskip
\gll Uļa  a  gid  ka-saļa  ta  ambaļ  ko  ya,  \textbf{di ba}? \\
\textsc{neg.r}  1\textsc{s.abs}  \textsc{int}  \textsc{iexm}-sin/wrong  \textsc{nabs}  say  1\textsc{s.gen}  \textsc{def.f}  isn’t.it.right \\
\glt ‘I really was not wrong in what I said, isn’t that right”' [EMWN-T-09 7.3]
\z

Sometimes yes/no questions can be accompanied by \textit{ino} ‘what’ at the beginning of the question, even though it does not refer to anything in the clause.

\ea
Ambaļ  ta  iya  na  kumpari,  “\textbf{Ino}  dey,  sigi  ki  nang en  ta  ate  i  na  lugar?” \\\smallskip
\gll Ambaļ  ta  iya  na  kumpari,  “\textbf{Ino}  dey,  sigi  ki  nang en  ta  ate  i  na  lugar?” \\
say  \textsc{nabs}  3\textsc{s.gen}  \textsc{lk}  his.child’s.godfather  what  friend  continue  1\textsc{p.incl.abs}  only
\textsc{cm}  \textsc{nabs}  1\textsc{p.incl.gen}  \textsc{def.n}  \textsc{lk}  place \\
\glt `The godfather of his child said, “What friend, will we continue only to be in our place?”’ (He wanted to go to  another place to hunt for shellfish.) [DBWN-T-33 2.2]
\z
\subsection{Question-word questions }
\label{bkm:Ref445900292} \label{sec:questionwordquestions}

Questions that expect a more elaborate response than simply an affirmation or disaffirmation are called \textit{question-word} \textit{questions}\is{question-word questions !}, \textit{content} \textit{questions}\is{content questions!}, \textit{information} \textit{questions} or \textit{WH- questions}\is{information questions!}. In this grammar we will use the term “question-word questions .”

The question words in Kagayanen consist of three interrogative pronouns, and four interrogative adverbs. The three interrogative pronouns are \textit{kino} ‘who’, \textit{ino} ‘what’, and \textit{pila} ‘how many.’ These only occur in the absolutive case, and are described in the context of a general discussion of pronouns in \chapref{chap:referringexpressions}, \sectref{sec:interrogativepronouns}. The four interrogative adverbs are \textit{kan-o} ‘when’, \textit{indi} ‘where’, \textit{indya} ‘which’, and \textit{man-o tak} ‘why’. The interrogative pronouns and adverbs can also be used as modifiers within RPs (see \ref{ex:atthewindow} and \ref{ex:restroom}). In this section we will discuss all of these forms and their functions in the formation of question-word questions .

\newpage
\ea
\textit{kino} ‘who?’ \\
Ambaļ  gid  ta  Pwikan  an,  “O  \textbf{kino}  magmandar  na  mag-umpisa  dļagan. \\\smallskip
\gll Ambaļ  gid  ta  Pwikan  an,  “O  \textbf{kino}  mag-mandar  na  mag-umpisa  dļagan. \\
say  \textsc{int}  \textsc{nabs}  sea.turtle  \textsc{def.m}  oh  who  \textsc{i.ir}-command  \textsc{lk}  \textsc{i.ir}-begin  run \\
\glt ‘The Sea turtle said, “Oh, who will command to start running?”' [JCON-T-08 31.7]
\z
\ea
\textbf{Kino}  pasud-o  no  ya  gibii? \\\smallskip
\gll \textbf{Kino}  pa-sud-o  no  ya  gibii? \\
who  \textsc{t.r}-visit  2\textsc{s.erg}  \textsc{def.f}  yesterday \\
\glt ‘Whom did you visit  yesterday?’
\z
\ea
\textbf{Kino}  patagan  no  ta  kwarta? \\\smallskip
\gll \textbf{Kino}  pa-atag-an  no  ta  kwarta? \\
who  \textsc{t.r}-give-\textsc{apl}  2\textsc{s.erg}  \textsc{nabs}  money \\
\glt ‘Whom did you give money to?'
\z
\ea
\textbf{Ano},   \textbf{kino}  gid  gauļa  ta  imo  binaļad  na  dawa? \\\smallskip
\gll \textbf{Ano,}   \textbf{kino}  gid  ga-uļa  ta  imo  b<in>aļad  na  dawa? \\
what    who  \textsc{int}  \textsc{i.r}-spill  \textsc{nabs}  2\textsc{p.gen}  <\textsc{nr.res}>dry.in.sun  \textsc{lk}  lentil \\
\glt ‘What, who really spilled your sun-dried lentils?' [TPWN-J-01 9.11]
\z
\ea
Pedro, \textbf{kino}  nļaman  no  na  mga  istudyanti  unti? \\\smallskip
\gll Pedro,     \textbf{kino}  na-aļam-an  no  na  mga  istudyanti  unti? \\
Pedro  who \textsc{a.hap.r}-know-\textsc{apl}  2\textsc{s.erg}  \textsc{lk}  \textsc{pl}  student  \textsc{d}1\textsc{loc.pr} \\
\glt ‘Pedro, who do you know who are students here?' [EFWN-T-11 15.12]
\z
\ea
\textit{Ino}  ‘what?’ \\
Ta  pag-abot  danen  unti,  \textbf{ino}  natabo  pugya  daw  asta anduni  a? \\\smallskip
\gll Ta  pag--abot  danen  unti,  \textbf{ino}  na-tabo  pugya  daw  asta anduni  a? \\
\textsc{nabs}  \textsc{nr.act}-arrive  3\textsc{s.gen}  \textsc{d}1\textsc{loc.pr}  what  \textsc{a.hap.r}-happen  long.ago  and  until
now/today  \textsc{inj} \\
\glt `When they arrived here, what happened long ago and until now?’  [PTOE-T-01 14.1]
\z
\ea
\textbf{Ino}  kani  tsinilasen  no  daw  mawas? \\\smallskip
\gll \textbf{Ino}  kani  tsinilas-en  no  daw  m-kawas? \\
what  later  sandals-\textsc{t.ir}  2\textsc{s.erg}  if/when  \textsc{i.v.ir}-disembark \\
\glt ‘What later will you wear as sandals when disembarking?’ [EFOB-C-01 4. 37]
\z
\ea
\textbf{Ino}   maambaļ  ta  ake  na  mga  ginikanan  daw  mga  nakakaļa ki  yaken? \\\smallskip
\gll \textbf{Ino}   ma-ambaļ  ta  ake  na  mga  ginikanan  daw  mga  naka-kaļa ki  yaken? \\
what  \textsc{a.hap.ir}-say  \textsc{nabs}  1\textsc{s.gen}  \textsc{lk}  \textsc{pl}  parent  and  \textsc{pl}  \textsc{i.hap.r}-know
\textsc{obl.p}  1s \\
\glt `What possibly will my parents and the ones who know me say?’ (In the context in the previous sentence a frog asked a princess to marry him and this is her response) [CBWN-C-17 5.12]
\z
\ea
\textit{Pila} ‘how many?’ \\
\textbf{Pila}  baton  no  isya  duminggo  daw  manangget ka? \\\smallskip
\gll \textbf{Pila}  baton  no  isya  duminggo  daw  ma-ng-sangget ka? \\
how.many  receive  2\textsc{s.erg}  one  week  if/when  \textsc{a.hap.ir-pl}-sickle
2\textsc{s.abs} \\
\glt ‘How many (pesos) do you receive in one week when you gather coconut sap?’ [RDOI-T-01 13.1]
\z
\ea
\textbf{Pila}  buok  paliten  no  na  sidda? \\\smallskip
\gll \textbf{Pila}  buok  palit-en  no  na  sidda? \\
how.many  piece  buy-\textsc{t.ir}  2\textsc{s.erg}  \textsc{lk}  fish \\
\glt ‘How many fish will you buy?’ \\\smallskip
Also: \textit{Pila buok na sidda paliten no?}
\z

\ea
\textit{kan-o} ‘when?’ \\
Insa  iya  ta  isya-isya  ki  kami,  “\textbf{Kan-o}  ki  isab   balik  di?” \\\smallskip
\gll Insa  iya  ta  isya\sim isya  ki  kami,  “\textbf{Kan-o}  ki  isab  balik  di?” \\
ask  3\textsc{s.gen}  \textsc{nabs}  \textsc{red}\sim{}one  \textsc{obl.p}  1\textsc{p.excl}  when  1\textsc{p.incl.abs}  again  return  \textsc{d}1\textsc{loc} \\
\glt ‘She asked each one of us, “When will we return here again?”' [EMWN-T-09 11.3]
\z
\ea
\textbf{Kan-o}  no   padaļa  suļat  ya  ki  kami? \\\smallskip
\gll \textbf{Kan-o}  no   ...-pa-daļa  suļat  ya  ki  kami? \\
when  2\textsc{s.erg}  \textsc{t.r}-\textsc{caus}-carry/bring  letter  \textsc{def.f}  \textsc{obl.p}  1\textsc{p.excl} \\
\glt ‘When did you send the letter to us?
\z
\ea
\textit{indi} ‘where’ \\
\textbf{Indi}  ka  gatago? \\\smallskip
\gll \textbf{Indi}  ka  ga-tago? \\
where  2\textsc{s.abs}  \textsc{i.r}-hide \\
\glt ‘Where did you hide?’ [JCON-L-07 5.7]
\z
\ea
\textbf{Indi}  no  imo  kamanga  mga  bļawan  an,  Pedro? \\\smallskip
\gll \textbf{Indi}  no  imo  ...=kamang-a  mga  bļawan  an,  Pedro? \\
where  2\textsc{s.erg}  \textsc{emph}  \textsc{t.r}-get-\textsc{xc}  \textsc{pl}  gold  \textsc{def.m}  Pedro \\
\glt ‘Where did you really get the gold, Pedro?' [ CBWN-C-22 8.3]
\z

Normally \textit{indi} means ‘where?’ (“which location”) and \textit{indya} means ‘which?’ But \textit{indya} alone may express the idea of ‘where’ also, since it may be a contraction of \textit{indi} and \textit{dya} ‘where there?’ In example \REF{bkm:Ref113884250}, \textit{indya} is appropriately translated as ‘where’, though the semantic similarity between ‘where’ and ‘which’ is apparent in this case:

\ea
\label{bkm:Ref113884250}
Manong,  \textbf{indya}  en  ittaw  ya  na  papatay  no? \\\smallskip
\gll Manong,  \textbf{indya}  en  ittaw  ya  na  pa-patay  no? \\
older.brother  where  \textsc{cm}  person  \textsc{def.f}  \textsc{lk}  \textsc{t.r}-die  2\textsc{p.erg} \\
\glt ‘Older brother, where now is the person you killed?’ [RBWN-T-02 4.1]
\z

\ea
\textit{indya} ‘which’ \\
\textbf{Indya}  di  paliten  no   na  bayo? \\\smallskip
\gll \textbf{Indya}  di  palit-en  no   na  bayo? \\
which  \textsc{d}1\textsc{loc}  buy-\textsc{t.ir}  2\textsc{s.erg}  \textsc{lk}  clothes \\
\glt ‘Which here are the clothes that you will buy?’
\z
\ea
\textbf{Indya}  ti  darwa  i  na  bayo  paliten  no? \\\smallskip
\gll \textbf{Indya}  ti  darwa  i  na  bayo  palit-en  no? \\
which  \textsc{d1nabs}  two  \textsc{def.n}  \textsc{lk}  clothes  buy-\textsc{t.ir}  2\textsc{s.erg} \\
\glt ‘Which of these two clothes will you buy?’ \\\smallskip
*\textit{Indi ti darwa i na bayo paliten no}?
\z
\ea
\textit{man-o tak} and \textit{man-o} ‘why’ \\
\textbf{Man-o}  \textbf{tak}  gambaļ  nangka  i? \\\smallskip
\gll \textbf{Man-o}  \textbf{tak}  ga=ambaļ  nangka  i? \\
why  because  \textsc{i.r}-say  jackfruit  \textsc{def.n} \\
\glt ‘Why is the jackfruit speaking?’ [YBWN-T-01 5.9]
\z
\ea
\textbf{Man-o?}  Daw  danen  i  maan  naan  nang  galabbod ta  iran  na  mimi,  uļa  gadiritso  ta  iran  na  gettek? \\\smallskip
\gll \textbf{Man-o?}  Daw  danen  i  m-kaan  naan  nang  ga-labbod ta  iran  na  mimi,  uļa  ga-diritso  ta  iran  na  gettek? \\
why  if/when  3\textsc{p.abs}  \textsc{def.n}  \textsc{i.v.ir}-eat  \textsc{spat.def}  only  \textsc{i.r}-expand/fill.up
\textsc{nabs}  3\textsc{p.gen}  \textsc{lk}  cheek  \textsc{neg.r}  \textsc{i.r}-straight  \textsc{nabs}  3\textsc{p.gen}  \textsc{lk}  stomach \\
\glt ‘Why? When they eat (food) it just expands/fills up their cheeks, not going straight to their stomachs.’ (This is  about monkeys) [NEWE-T-01 2.6]
\z

Some adverbs may occur between \textit{man-o} and \textit{tak} as in the following example.

\ea
\textbf{Man-o}  \textbf{kon}  \textbf{tak}  gina  gaselled  a  gagaļ  a  kon? \\\smallskip
\gll \textbf{Man-o}  \textbf{kon}  \textbf{tak}  gina  ga-selled  a  ga-agaļ  a  kon? \\
why  \textsc{hsy}  because  earlier  \textsc{i.r-}inside  1\textsc{s.abs}  \textsc{i.r}-cry  1\textsc{s.abs}  \textsc{hsy} \\
\glt ‘Why  did I go inside earlier crying?’ [BMON-C-06 3.15]
\z

Often \textit{man-o tak} or simply \textit{man-o} are used in scolding, as a kind of rhetorical question:

\ea
Pagilekan  Pedro  i  ta  nanay  din  na  ambaļ  ta  nanay  din, “\textbf{Man-o}  \textbf{tak}  patudlo  no  imo  makina  ya  na  patago  ko.” \\\smallskip
\gll Pa-gilek-an  Pedro  i  ta  nanay  din  na  ambaļ  ta  nanay  din, “\textbf{Man-o}  \textbf{tak}  pa-tudlo  no  imo  makina  ya  na  pa-tago  ko.” \\
\textsc{t.r}-angry-\textsc{apl}  Pedro  \textsc{def.n}  \textsc{nabs}  mother  3\textsc{s.gen}  \textsc{lk}  say  \textsc{nabs}  mother  3\textsc{s.gen}
why  because  \textsc{t.r}-teach/point  2\textsc{s.erg}  \textsc{emph}  machine  \textsc{def.f}  \textsc{lk}  \textsc{t.r}-hide  1\textsc{s.erg} \\
\glt `His mother scolded Pedro when his mother said, “Why did you point out (where) the machine is that I hid?”' [BCWN-C-04 7.5]
\z

The interrogative pronouns can be used as modifiers in questions with the \textit{na} linker and a head noun, as in examples \REF{bkm:Ref113885505} and \REF{bkm:Ref113885509} {}- \textit{kino na manakem} ‘who older person’ and \textit{ino na lugar} ‘what place.’ In conversation the linker may drop as in example \REF{bkm:Ref113885512} \textit{ino oras} ‘what time’:

\ea
\label{bkm:Ref113885505}\label{ex:atthewindow}
\textbf{Kino}  \textbf{na}  \textbf{manakem}  naan  dya  gatindeg  an  ta  tumbuan  an? \\\smallskip
\gll \textbf{Kino}  \textbf{na}  \textbf{manakem}  naan  dya  ga-tindeg  an  ta  tumbuan  an? \\
who  \textsc{lk}  older  \textsc{spat.def}   \textsc{d}4\textsc{loc}  \textsc{i.r}-stand  \textsc{spat.def}  \textsc{nabs}  window  \textsc{def.m} \\
\glt ‘Who is the older person standing there at the window?' [ BGON-L-01 2.4]
\z
\ea
\label{bkm:Ref113885509}\label{ex:restroom}
\textbf{Ino}  \textbf{na}  \textbf{lugar}  ginabetangan  ta  tanan  nang  pakaan  ta?  Kasilyas. \\\smallskip
\gll \textbf{Ino}  \textbf{na}  \textbf{lugar}  gina-betang-an\footnotemark{}  ta  tanan  nang  pa-kaan  ta?  Kasilyas. \\
what  \textsc{lk}  place  \textsc{t.r}-put-\textsc{apl}  \textsc{nabs}  all  only  \textsc{t.r}-eat  1\textsc{p.incl.erg}  restroom \\
\footnotetext{The prefix \textit{gina}- in this example is code switching from \isi{Hiligaynon}.}
\glt ‘What is the place where all we eat is put? Restroom.’ (This is a riddle with the answer). [SFWR-L-05 11.1]
\z
\ea
\label{bkm:Ref113885512}
Ambaļ  gid  daen,  “\textbf{Ino}  \textbf{oras}  mabot  kaw  di?” \\\smallskip
\gll Ambaļ  gid  daen,  “\textbf{Ino}  \textbf{oras}  m-abot  kaw  di?” \\
say  \textsc{int}  3\textsc{p.erg}  what  time/hour  \textsc{i.v.ir}-arrive  2\textsc{p.abs}  \textsc{d}1\textsc{loc} \\
\glt ‘They enthusiastically said, “What time will you arrive here?”' [BGON-L-01 3.15] \\\smallskip
Also: Ino na oras mabot kaw di?
\z

The adverbs \textit{kan-o}, \textit{indi}, \textit{man-o} and \textit{man-o tak} may not be used as modifiers:

\ea
*\textit{Kan-o na oras} ‘When’ \\
*\textit{Indi na lugar} ‘Where’ \\
*\textit{Man-o tak na} ... ‘Why’ \\
*\textit{Man-o na} ... ‘Why’
\z
\ea
\textbf{Pila}  \textbf{na}  \textbf{karton}  pakot  nyo? \\\smallskip
\gll \textbf{Pila}  \textbf{na}  \textbf{karton}  pa-akot  nyo? \\
how.many  \textsc{lk}  box  \textsc{t.r}-carry.many  2\textsc{p.erg} \\
\glt ‘How many boxes did you carry?’
\z
\ea
\textbf{Indya}  \textbf{na}  \textbf{bayo}  gusto  no? \\\smallskip
\gll \textbf{Indya}  \textbf{na}  \textbf{bayo}  gusto  no? \\
which  \textsc{lk}  clothes  want  2\textsc{s.erg} \\
\glt ‘Which clothes do you want?’ \\
\z

The interrogative pronoun \textit{ino} ‘what’ can occur with verbal inflection (no other interrogative pronoun or adverb can be used in this way): \textit{gino/gaino} ‘what did X do or what is X doing’, \textit{mag-ino} ‘what will X do,’ \textit{paino/pino} ‘what did someone do to X,’ \textit{inuon/nuon}\footnote{The form \textit{inuon} is sometimes shortened to \textit{nuon}. See example \REF{bkm:Ref118798909} below.} ‘what will someone do to X', \textit{nino/naino} ‘what happened to X or what was someone able to do to X,’ \textit{maino} ‘what will happen to X or what will someone be able to do to X’:

\ea
Dayon  prani  isya  na  bata  na  ginsa,  “Sir,  \textbf{nino}  ka?” \\\smallskip
\gll Dayon  prani  isya  na  bata  na  ga-insa,  “Sir,  \textbf{na-ino}  ka?” \\
right.away  come.close  one  \textsc{lk}  child  \textsc{lk}  \textsc{i.r}-ask  Sir  \textsc{a.hap.r}-what  2\textsc{s.abs} \\
\glt ‘Right away one child came close asking, “Sir, what happened to you?”' [EFWN-T-11 15.8]
\z
\ea
\textbf{Gino}  a  pwikan  an  daw  manaw?  Gadaik. \textbf{Gino}  a  umang  an  daw  manaw?  Gagyanap. \\\smallskip
\gll \textbf{Ga-ino}  a  pwikan  an  daw  m-panaw?  Ga-daik. \textbf{Ga-ino}  a  umang  an  daw  m-panaw?  Ga-gyanap. \\
\textsc{i.r}-what  \textsc{inj}  sea.turtle  \textsc{def.m}  if/when  \textsc{i.v.ir}-go/walk  \textsc{i.r}-crawl.on.stomach
\textsc{i.r}-what  \textsc{inj}  hermit.crab  \textsc{def.m}  if/when  \textsc{i.v.ir}-go/walk  \textsc{i.r}-crawl \\
\glt `What does the sea turtle do when moving? Crawls on stomach. What does the hermit crab do when moving? Crawls.’ [JCON-L-08 2.6-7]
\z
\ea
\textbf{Pino}  no  mga  bata  an  na  gagaļ? \\\smallskip
\gll \textbf{Pa-ino}  no  mga  bata  an  na  ga-agaļ? \\
\textsc{t.r}-what  2\textsc{s.erg}  \textsc{pl}  child  \textsc{def.m}  \textsc{lk}  \textsc{i.r}-cry \\
\glt ‘What did you do to the children who are crying?’
\z
\ea
\textbf{Pino}  na  pagluto  din  ta  lub-ong? \\\smallskip
\gll \textbf{Pa-ino}  na  pag-luto  din  ta  lub-ong? \\
\textsc{t.r}-what  \textsc{lk}  \textsc{nr.act}-cook  3\textsc{s.erg}  \textsc{nabs}  steamed.cassava \\
\glt ‘How did s/he cook the steamed cassava?’
\z

\textit{Ino} with optional \textit{na} linker and \textit{pag}{}- action nominalized verb can be used to ask how something is done.

\ea
\textbf{Ino}  \textbf{pagkamang}  ko  ta  tļunon? \\\smallskip
\gll \textbf{Ino}  \textbf{pag-kamang}  ko  ta  tļunon? \\
what  \textsc{nr.act}-get  1\textsc{s.erg}  \textsc{nabs}  wild.pig \\
\glt ‘How did I get a wild pig?' [ RCON-L-01 1.1] \\\smallskip
\textit{Also:} Ino na pagkamang ko ta tļunon?
\z

\ea
Bag-o  pagkatapos,  \textbf{ino}  \textbf{na}  \textbf{pag-ubra}  ta  kaļaw, ta  lunday,  buti,  bļangay,  lansa…? \\\smallskip
\gll Bag-o  pagka-tapos,  \textbf{ino}  \textbf{na}  \textbf{pag-ubra}  ta  kaļaw, ta  lunday,  buti,  bļangay,  lansa…? \\
next  \textsc{nr.act-nr}-finish  what  \textsc{lk}  \textsc{nr.act}-make/work  \textsc{nabs}  winnowing.basket
\textsc{nabs} outrigger.canoe  rowboat  2.masted.boat  launch \\
\glt `Next, afterwards, how to make a winnowing basket, outrigger canoe, rowboat, 2 masted boat, launch…?’ [JCOE-T-06 8.5]
\z

There is one content question construction that does not involve a question word. There is only one example of this construction in the corpus:

\ea
Ngaran  tan  masakit  no  an? \\\smallskip
\gll Ngaran  tan  masakit  no  an? \\
name  \textsc{d3nabs}  sickness  2\textsc{s.gen}  \textsc{def.m} \\
\glt ‘What is (the name of) your sickness?’ (lit. ‘name of that your sickness’) [ETWC-C-03 3.1]
\z

Grammatically, this is a simple RP (Referring Phrase) meaning “name of your sickness.” However, because of intonation and context, it is obviously intended to be a question. This may be an example of omission of a question word, as in interview questions in English: “Name? Address? Place of birth?”

\subsection{Imperatives} 
\label{bkm:Ref445900361}
\textit{Imperative construction} is the term linguists use to refer to a type of sentence that is uniquely adapted to the speech act of directing someone to do something. While most languages have special forms for this type of construction, Kagayanen does not. Rather than having a special imperative construction, Kagayanen speakers use the plain declarative irrealis construction with a second person Actor. The examples in \REF{bkm:Ref113955378} illustrate some simple imperatives. Note that some of these are grammatically transitive and some are grammatically intransitive. Thus, the second person Actor may be absolutive or ergative. Following \REF{bkm:Ref113955378} are some examples from the corpus:

\ea
\label{bkm:Ref113955378}
\begin{tabbing}
\hspace{4cm} \= \kill
\textit{Mandok ka ta waig}. \> ‘Fetch water.’ \\
\textit{Min-ad ka}. \> ‘Cook rice.’ \\
\textit{Millig ka ta silong}. \> ‘Sweep the yard.’ \\
\textit{Silligan no silong ya}. \>‘Sweep the yard.’ \\
\textit{Maglamon ka ta silong}. \> ‘Weed the yard.’ \\
\textit{Muli ka dya en}. \> ‘Go home there now.’ \\
\textit{Manaw ka en}. \> ‘Go now.’ \\
\textit{Mampang ka dya en}. \> ‘Play there now.’ \\
\textit{Nisnisan no kaldiro an}. \> ‘Scour the pot.’ \\
\textit{Magnisnis ka ta kaldiro}. \> ‘Scour the pot(s).’ \\
\textit{Baugan no baboy ya}. \> ‘Feed the pig.’ \\
\textit{Magbaog ka en ta baboy}. \> ‘Feed the pig now.’ \\
\textit{Maan ka en}. \> ‘Eat now.’ \\
\textit{Kan-enen no en sidda i}. \> ‘Eat the fish.’
\end{tabbing}
\z
\ea
\textbf{Ilingan}  no  tatay  ya  na  tabangan  a  iyaw  ta  baboy  i. \\\smallskip
\gll \emptyset{}-\textbf{Iling-an}  no  tatay  ya  na  \emptyset{}-tabang-an  a  iyaw  ta  baboy  i. \\
\textsc{t.ir}-go-\textsc{apl}  2\textsc{s.erg}  father  \textsc{def.f}  \textsc{lk}  \textsc{t.ir}-help-\textsc{apl}  1\textsc{s.abs}  slaughter  \textsc{nabs}  pig  \textsc{def.n} \\
\glt `“Go to (my) father (and ask him) to help me slaughter the pig.”' [RCON-L-01 10.1]
\z
\ea
Liksyon  an  na  nakamang  ta  don  na  daw  ano gani namatian  no  an  na  mga  balita,  dili  kaw magpagulpi-gulpi.  \textbf{Lagen} no  gid  anay.  Dili  ka \textbf{mag-adlek}  ta  miad.  \textbf{Usisaen}  no  gid  anay  aged na  uļa  may  matabo  ki  kaon. \\\smallskip
\gll Liksyon  an  na  na-kamang  ta  don  na  daw  ano\footnotemark{}  gani na-mati-an  no  an  na  mga  balita,  dili  kaw mag-pa-gulpi-gulpi.  \textbf{luag-en}\footnotemark{}  no  gid  anay.  Dili  ka \textbf{mag-adlek}  ta  miad.  \textbf{Usisa-en}  no  gid  anay  aged na  uļa  may  ma-tabo  ki  kaon. \\
lesson  \textsc{def.m}  \textsc{lk}  \textsc{a.hap.r}-get  1\textsc{p.incl.erg}  \textsc{d}3\textsc{loc}  \textsc{lk}  if/when  what  truly
\textsc{a.hap.r}-hear-\textsc{apl}  2\textsc{s.erg}  \textsc{def.m}  \textsc{lk}  \textsc{pl}  news  \textsc{neg.ir}  2\textsc{p.abs}
\textsc{i.ir}-\textsc{caus}-\textsc{red}-sudden  watch-\textsc{t.ir}  2\textsc{s.erg}  \textsc{int}  first/a.while  \textsc{neg.ir}  2\textsc{s.abs}
\textsc{i.ir}-afraid  \textsc{nabs}  well/good  investigate-\textsc{t.ir}  2\textsc{s.erg}  \textsc{int}  first/a.while  so.that
\textsc{lk}  \textsc{neg.r}  \textsc{ext.in}  \textsc{a}\textsc{.hap.r}-happen  \textsc{obl.p}  2s \\
\footnotetext[3]{This use of \textit{ano} in place of the Kagayanen \textit{ino} is code switching from \isi{Hiligaynon} or \isi{Tagalog}.}
\footnotetext{This is a very common word. The root is clearly \textit{luag} ‘to watch’, but the irrealis transitive form is \textit{lagen} rather than the expected *\textit{luagen}.}
\glt `The lesson that we can get from that, is that whatever news you hear, do not be startled. Look (at it) first. Do not be overly afraid. Investigate (it) first so that there will not be something that will happen to you.’ [JCON-L-07 22.3-6]
\z
\ea 
\label{bkm:Ref113956604}
\textbf{Manaw}  kaw  di  tak  \textbf{mangamuyo}  ki. \\\smallskip
\gll \textbf{m-panaw}  kaw  di  tak  \textbf{m-pangamuyo}  ki. \\
\textsc{i.v.ir}-walk/go  2\textsc{p.abs}  \textsc{d}1\textsc{loc}  because  \textsc{i.v.ir}-pray  1\textsc{p.incl.abs} \\
\glt ‘Come here because let’s pray.’ (lit. You will come here because we will pray) [JCWN-T-22 3.1]
\z

An irrealis construction with first person inclusive Actor may also be interpreted as \textit{cohortative mood}\is{cohortative mood} (“lets VERB” in English), which is a kind of imperative. See the second verb in \REF{bkm:Ref113956604} and the following:

\ea 
\textbf{Muli}  \textbf{ki}  en. \\\smallskip
\gll \textit{\textbf{M-uli}  \textbf{ki}  en}. \\
\textsc{i.v.ir}-go.home  1\textsc{p.incl.abs}  \textsc{cm} \\
\glt ‘Let’s go home.’ [PEWN-T-01 2.8]
\z

As with all verbal irrealis assertions, the negator \textit{dili} is used for negative imperatives:

\ea 
… daw  \textbf{dili}  kaw  mag-ubra  ta  usto  tak  manakem  kaw  en. \\\smallskip
\gll … daw  \textbf{dili}  kaw  mag-ubra  ta  usto  tak  manakem  kaw  en. \\
{} and  \textsc{neg.ir}  2\textsc{p.abs}  \textsc{i.ir}-make/work  \textsc{nabs}  well/right  because  older  2\textsc{p.abs}  \textsc{cm} \\
\glt ‘…and do not work too much because you are older now.’ [BCWL-T-10 3.6]
\z

 As in most speech communities, Kagayanen speakers employ various strategies for making requests for action more polite or indirect than direct commands. For polite commands \textit{pwidi} ‘may/can/possible,’ \textit{maimo} ‘possible to do/make’ and \textit{anay} ‘first/for a while’ are used:

\ea 
Daw  \textbf{pwidi}  \textbf{nang}  na  dili  kaw  en  mag-iling-iling  ki  yaken di  tak  uļa  aren  en  iatag  ki  kyo  na  mga  bagay na  kinangļanen  nyo. \\\smallskip
\gll Daw  \textbf{pwidi}  \textbf{nang}  na  dili  kaw  en  mag-iling-iling  ki  yaken di  tak  uļa  aren  en  i-atag  ki  kyo  na  mga  bagay na  kinangļan-en  nyo. \\
if/when  can  only  \textsc{lk}  \textsc{neg.ir}  2\textsc{p.abs}  \textsc{cm}  \textsc{i.ir}-\textsc{red}-go  \textsc{obl.p}  1s
\textsc{d}1\textsc{loc}  because  \textsc{neg.ir}  1\textsc{s.abs}  \textsc{cm}  \textsc{t.deon}-give  \textsc{obl.p}  2p  \textsc{lk}  \textsc{pl}  thing
\textsc{lk}  need-\textsc{t.ir}  2\textsc{p.gen} \\
\glt `If it is possible, do not keep coming to me here because I do not have things that you need that I should give to you.’ [ETOB-C-01 1.3]
\z
\ea 
\label{bkm:Ref113971261}
Daw  \textbf{maimo}  nang  inta,  magtinir  ka  \textbf{anay}  mga darwa  adlaw…. \\\smallskip
\gll Daw  \textbf{ma-imo}  nang  inta,  mag-tinir  ka  \textbf{anay}  mga darwa  adlaw…. \\
if/when  \textsc{a.hap.ir}-make/do  only/just  \textsc{opt}  \textsc{i.ir}-stay  2\textsc{s.abs}  first/for.a.while  \textsc{pl} two  day/sun \\
\glt ‘If it is possible, stay (here) for a while about two days…’ [PMWL-T-07 2.3]
\z
\ea 
Dayon  ko  ambaļ  arey  ko  i  na  Pedro  na,  “Dili  ka \textbf{anay}  magpetpet  ta  kaoy. \\\smallskip
\gll Dayon  ko  ambaļ  arey  ko  i  na  Pedro  na,  “Dili  ka \textbf{anay}  mag-petpet  ta  kaoy. \\
right.away  1\textsc{s.erg}  say  friend  1\textsc{s.gen}  \textsc{def.n}  \textsc{lk}  Pedro  \textsc{lk}  \textsc{neg.ir}  2\textsc{s.abs}
first/a.while  \textsc{i.ir}-chop  \textsc{nabs}  tree \\
\glt `Right away I said to my friend Pedro, “Do not chop down trees for now.” ' [MEWN-T-02 2.4]
\z

Sometimes first person plural inclusive pronouns are used to soften a command:

\ea 
Magtanem  \textbf{ki}  ta  kassoy. \\\smallskip
\gll Mag-tanem  \textbf{ki}  ta  kassoy. \\
\textsc{i.ir}-plant  1\textsc{p.incl.abs}  \textsc{nabs}  cashew \\
\glt ‘Let’s plant cashew.’ [MCOB-C-01 12.9]
\z
\ea
Gani,  mga  utod  impurtanti  matuod  na  mag-ambaļ  \textbf{ki}  ta  Kagayanen bisan  indi  ki  nang. \\\smallskip
\gll Gani,  mga  utod  impurtanti  matuod  na  mag-ambaļ  \textbf{ki}  ta  Kagayanen bisan  indi  ki  nang. \\
so  \textsc{pl}  sibling  important  true  \textsc{lk}  \textsc{i.ir}-say  1\textsc{p.incl.abs}  \textsc{nabs}  Kagayanen
any  where  1\textsc{p.incl.abs}  only/just \\
\glt `So, my siblings, it is truly important that we speak Kagayanen wherever we are.’ [TTOB-J-01 10.1]
\z
\ea 
Pagtapos  daen  i  inem,  iling  ko,  “Ta  dayon  \textbf{ki}  en. A  miling  \textbf{ki}  unso  en  baļay  ya  ta  inay  ta  bai  ya. \\\smallskip
\gll Pag-tapos  daen  i  inem,  iling  ko,  “Ta  dayon  \textbf{ki}  en. A  m-iling  \textbf{ki}  unso  en  baļay  ya  ta  inay  ta  bai  ya. \\
\textsc{nr.act}-finish  3\textsc{p.gen}  \textsc{def.n}  drink  say  1\textsc{s.erg}  so  continue  1\textsc{p.incl.abs}  \textsc{cm}
ah  \textsc{i.v.ir}-go  1\textsc{p.incl.abs}  \textsc{d}4\textsc{loc.pr}  \textsc{cm}  housse  \textsc{def.f}  \textsc{nabs}  mother  \textsc{nabs}  woman  \textsc{def.f} \\
\glt `After they drank, I said, “So, let’s go ahead now. Let’s go there to the house of the mother of the woman.”' [BGON-L-01 2.25]
\z
The adverbs \textit{dapat} ‘must’, \textit{kinangļan} ‘necessary’, and \textit{inta} ‘wish/ought/should’ add more force to a command. See example \ref{bkm:Ref113971261} above. In addition, the deontic transitive prefix \textit{i-} on the verb can add more force, as in \REF{bkm:Ref118711236}. The following are additional examples of the use of these adverbs in imperatives. from the corpus:

\ea 
\label{bkm:Ref118711236}
Yon  na  mga  kaoy  \textbf{dapat}  ta  itanem. \\\smallskip
\gll Yon  na  mga  kaoy  \textbf{dapat}  ta  i-tanem. \\
\textsc{d}3\textsc{adj}  \textsc{lk}  \textsc{pl}  tree  should  1\textsc{p.incl.erg}  \textsc{t.deon}-plant \\
\glt ‘Those trees we must plant.’ [ROOB-T-01 8.10]
\z
\ea
\textbf{Kinangļan}  na  kiten  magtanem  anduni  daan... \\\smallskip
\gll \textbf{Kinangļan}  na  kiten  mag-tanem  anduni  daan... \\
need  \textsc{lk}  1\textsc{p.incl.abs}  \textsc{i.ir}-plant  now/today  ahead/immediately \\
\glt ‘(It is) necessary that we plant (trees) now/today ahead of time….’ [ROOB-T-01 9.17]
\z
\ea
Dili  nyo  \textbf{inta}  pintasan. \\\smallskip
\gll Dili  nyo  \textbf{inta}  pintas-an. \\
\textsc{neg.ir}  2\textsc{s.erg}  \textsc{opt}  harm-\textsc{apl} \\
\glt ‘You should not harm (it).’ [MEWN-T-02 5.6]
\z
\section{Special focus constructions}
\label{bkm:Ref445900433}
All languages provide special grammatical constructions, particles, or affixes that draw attention to particular parts of communicative acts. For example, English places elements at the beginning of a clause when they merit various kinds of focus according to the communicative intentions of the speaker. For example, “Coffee I like”, “It’s coffee I like”, “Coffee is what I like”, “As for coffee, I like it” are all pragmatically marked variations on the basic assertion “I like coffee.” In this section we describe three ways of ascribing special pragmatic statuses to elements in a clause. These are constituent order variation (\sectref{bkm:Ref445900436}), grammatical particles (\sectref{bkm:Ref118460437}), and cleft constructions (\sectref{bkm:Ref122603009}). We have done our best to characterize the discourse functions of these constructions as they appear in the corpus, but a full discourse study is still needed to precisely evaluate the pragmatic effects of each of them.


\subsection{Constituent order variation}
\label{bkm:Ref445900436}\label{sec:constituentordervariation}

The basic unmarked constituent order in Kagayanen intransitive clauses is V ABS, that is, a predicating word (usually a verb) followed by an absolutive RP referring to the single argument (S) of the intransitive clause. There are two basic orders of core arguments in transitive clauses, depending on the relative position of the arguments on the following saliency hierarchy:


\ea 
Pronoun\hspace{.5cm}    Full RP \\
1~>~2~>~3~>  {\textbar}  human~>~animate~>~inanimate
\z

Of the two core arguments, A and O (see \chapref{chap:voice}, \sectref{sec:grammaticalrelations}), whichever one is higher (further to the left) on this hierarchy usually occurs first. This may be considered a \textit{word order direct/inverse construction} \is{word order direct/inverse construction}(see \citealt{payne1994} on \isi{Cebuano}). The direct form occurs when A and O are equal on this hierarchy (example \ref{bkm:Ref113973502}), or when A outranks O (example \ref{bkm:Ref113973505}). The inverse form occurs when O outranks A (example \ref{bkm:Ref113973507}):

\ea 
\label{bkm:Ref113973502}
\hspace{1.1cm} V\hspace{2cm}          A\hspace{2.1cm}          O \\
… daw  patigbas  man  \textbf{ta}  \textbf{amay}  \textbf{ta}  \textbf{bata}  \textbf{manakem}  \textbf{ya}  \textbf{na}  \textbf{bai}. \\\smallskip
\gll … daw  pa-tigbas  man  \textbf{ta}  \textbf{amay}  \textbf{ta}  \textbf{bata}  \textbf{manakem}  \textbf{ya}  \textbf{na}  \textbf{bai}. \\
{} and  \textsc{t.r}-chop  also  \textsc{erg}  father  \textsc{nabs}  child  older.person  \textsc{def.f}  \textsc{lk}  woman \\
\glt ‘… and the father of the child chopped the older woman.’ [YBWN-T-01 5.11]
\z

When O outranks A on the hierarchy, the O appears first. In example \REF{bkm:Ref113973507} the 1\textsc{s.abs} pronoun precedes the full RP referring to the ergative constituent:

\ea 
\label{bkm:Ref113973507}
\hspace{4.4cm}V\hspace{.4cm} O\hspace{.1cm} A \\
Na  gapanaw  kay  ta  daļan,  kagat  \textbf{a}  \textbf{ta}  \textbf{sitsipit}. \\\smallskip
\gll Na  ga-panaw  kay  ta  daļan,  kagat  \textbf{a}  \textbf{ta}  \textbf{sitsipit}. \\
\textsc{lk}  \textsc{i.r}-walk/go  1\textsc{p.excl.abs}  \textsc{nabs}  road  bite  1\textsc{s.abs}  \textsc{erg}  scorpion \\
\glt ‘When we were walking on the road, a/the scorpion bit (stung) me.’ [LSWN-T-01 2.8]
\z

\largerpage%longdistance
There are no examples in the corpus in which both A and O are third person pronouns, and such examples are difficult for speakers to contextualize. However, A and O can both be pronouns if one is a different person category than the other. In such cases, the pronoun that is higher on the hierarchy occurs first, whether it is A or O. In examples \REF{bkm:Ref113973505} and \REF{bkm:Ref113973510} the first person pronoun occurs first, though it is A in \REF{bkm:Ref113973505}, and O in \REF{bkm:Ref113973510}:

\ea 
\label{bkm:Ref113973505}
\hspace{.4cm}V\hspace{.8cm}A\hspace{.3cm}O \\
… nakita  \textbf{ko}  \textbf{danen}  dya  na  sigi  kaan. \\\smallskip
\gll … na-kita  \textbf{ko}  \textbf{danen}  dya  na  sigi  kaan. \\
{} \textsc{a.hap.r}-see  1\textsc{s.erg}  3\textsc{p.abs}  \textsc{d}4\textsc{loc}  \textsc{lk}  continue  eat. \\
\glt ‘…I saw them there continuing to eat.’ [RCON-L-01 2.9]
\z
\ea
\label{bkm:Ref113973510}
\hspace{.4cm}V\hspace{1.4cm}O\hspace{.1cm}A \\
… pakamang  \textbf{a}  \textbf{danen}  na  magduma  ta  iran  na  bata na  gaiskwila  man  ta  Iloilo. \\\smallskip
\gll … pa-kamang  \textbf{a}  \textbf{danen}  na  mag-duma  ta  iran  na  bata na  ga-iskwila  man  ta  Iloilo. \\
{} \textsc{t.r}-get  1\textsc{s.abs}  3\textsc{p.erg}  \textsc{lk}  \textsc{i.ir}-companion  \textsc{nabs}  3\textsc{p.gen}  \textsc{lk}  child
\textsc{lk}  \textsc{i.r}-school  also  \textsc{nabs}  Iloilo \\
\glt `… they  got me to companion their child who was attending school in Iloilo.’ [DBWN-T-21 4.6]
\z 

There are no examples in the corpus of VOA order occurring when both core arguments are full RPs. However, such examples do occur in conversation, and speakers agree they are grammatically correct. A full discourse study of the function of VOA order in this circumstance would require a much larger corpus, since it is so rare. The following are perfectly grammatical elicited examples. Note that “my child” \REF{bkm:Ref113974691} and “our house” \REF{bkm:Ref113974693}, though both in the grammatical function of O (undergoer), are likely to be more highly salient than the corresponding A arguments by virtue of being definite and possessed by the speaker. This may explain why these examples are easy to imagine, but again, a full discourse study would be needed to confirm or disconfirm this hypothesis.

\ea 
\label{bkm:Ref113974691}
\hspace{.2cm}V\hspace{1.5cm}O\hspace{1.8cm}A \\
Napattikan  \textbf{bata}  \textbf{ko}  \textbf{an}  \textbf{ta}  \textbf{sitsipit}. \\\smallskip
\gll Na-pattik-an  \textbf{bata}  \textbf{ko}  \textbf{an}  \textbf{ta}  \textbf{sitsipit}. \\
\textsc{a.hap.r}-sting-\textsc{apl}  child  1\textsc{s.gen}  \textsc{def.m}  \textsc{nabs}  scorpion \\
\glt ‘A/The scorpion stung my child.’
\z
\ea
\label{bkm:Ref113974693}
\hspace{.2cm}V\hspace{1.2cm}O\hspace{2cm}A \\
Nasunog  \textbf{baļay  nay  ya}  \textbf{ta}  \textbf{apoy}. \\\smallskip
\gll Na-sunog  \textbf{baļay}  \textbf{nay}  \textbf{ya}  \textbf{ta}  \textbf{apoy}. \\
\textsc{a.hap.r}-burn  house  1\textsc{p.excl.gen}  \textsc{def.f}  \textsc{nabs}  fire \\
\glt ‘A/the fire burned our house.’ or ‘Our house was burned with fire.’
\z

Though the above generalizations hold in the majority of situations, it is not uncommon for these patterns to vary for special emphasis or when the speaker is in some heightened emotional state. For example, when someone is frightened or excited by the content of the utterance, a marked word order may occur. With the marked word order, free emphatic absolutive pronouns occur instead of enclitic pronouns (see \chapref{chap:referringexpressions}, \sectref{sec:personalpronouns}). In example \REF{bkm:Ref447287658} the speaker was describing a very fearful time when in a dream a fairy was forcing her to go away with him. It is believed that such dreams can result in one’s death.

\ea 

\label{bkm:Ref447287658}
\hspace{.1cm}V\hspace{.8cm}   A\hspace{.5cm}      O \\
Pambaļ  \textbf{din}  \textbf{yaken}  \textbf{i}  na  muyog  a. \\\smallskip
\gll Pa-ambaļ  \textbf{din}  \textbf{yaken}  \textbf{i}  na  m-kuyog  a. \\
\textsc{t.r}-say  3\textsc{s.erg}  1\textsc{s.abs}  \textsc{def.n}  \textsc{lk}  \textsc{i.v.ir}-go.with  1\textsc{s.abs} \\
\glt ‘He told me that I will go with (him).’ [EDON-J-01 1.7]
\z

\largerpage
Since this is a clause in which the O argument (the speaker) is higher in the salience hierarchy than the Actor (the fairy) the unmarked word order would be VOA, with the enclitic absolutive enclitic pronoun \textit{a}, instead of the free pronoun \textit{yaken} referring to the speaker:

\ea 
\hspace{.1cm}V\hspace{.7cm}   O\hspace{.2cm}A\hspace{.5cm} (Expected, common order) \\
Pambaļ  \textbf{a  din}  na  muyog  a.    \\\smallskip
\gll Pa-ambaļ  \textbf{a}  \textbf{din}  na  m-kuyog  a. \\
\textsc{t.r}-say  1\textsc{s.abs}  3\textsc{s.erg}  \textsc{lk}  \textsc{i.v.ir}-go.with  1\textsc{s.abs} \\
\glt ‘He said to me that I will go with (him).’
\z

The use of the emphatic pronoun, and the unusual word order in \REF{bkm:Ref447287658} signals that the speaker was strongly affected emotionally by this frightening event.

Because the basic constituent order is predicate-initial, the pre-predicate position is a very powerful signal of special pragmatic importance. \citet[89]{pebleyfunctions1999} describe the functions of the pre-predicate element in expository discourse as follows:

\begin{modquote}
Fronted RPs mark sentences that: 1)~introduce new themes, 2)~signal changes in theme, 3)~summarize themes, 4)~present results or reasons that are peaks, 5)~signal selective focus, and 6)~signal contrastive focus.
\end{modquote}

In this section we summarize the findings of \citet{pebleyfunctions1999} using new examples from the corpus for the present study. The same or similar observations can be made regarding the functions of pre-predicate position in genres other than expository as well.

% The primary function of fronted RPs in Kagayanen expository discourse is to mark the three kinds of information that are most important for developing an expository text, namely themes, result, and contrast. 

The primary function of fronted RPs in Kagayanen expository discourse is to mark the kinds of information that are most important for developing an expository text, namely global and lower level themes,\footnote{Not to be confused with “Theme” as a semantic role as discussed in other parts of this grammar.} result, and contrast \citep{pebleyfunctions1999}. A “global theme”, or the main idea being explained in an expository text, is usually introduced in a fronted RP in the first paragraph. In a text explaining the traditional belief about lunar eclipses \REF{bkm:Ref372792644}, the global theme is introduced in the first fronted RP in the first sentence of the paragraph. It is stated again in the second  part of the same sentence:

\ea 
\label{bkm:Ref372792644}
\textbf{Kaanlao},  pagpati  ta  mga  ittaw  di  ta  Cagayancillo na  daw  \textbf{buļan  an},  kaan  ta  lao,  kaanlao, \textbf{bakod}  \textbf{kon}  \textbf{an} \textbf{bekkessan}  palam-ed  din  buļan  an. \\\smallskip
Fronted RP1 \\
\gll \textbf{Kaanlao},  pagpati  ta  mga  ittaw  di  ta  Cagayancillo\\
lunar.eclipse \textsc{nr.act}-believe \textsc{nabs} \textsc{pl}  person \textsc{d}1\textsc{loc}  \textsc{nabs} Cagayancillo\\\smallskip

\hspace{1.8cm}Fronted RP2 \\
\gll na  daw  \textbf{buļan}  \textbf{an},  kaan  ta  lao,  kaanlao, \\
\textsc{lk}  if/when  moon/month  \textsc{def.m}  eat  \textsc{nabs}  sky.snake  lunar.eclipse \\\smallskip
 {}

Fronted RP3 \\
\gll \textbf{bakod}  \textbf{kon}  \textbf{an}  \textbf{bekkessan}  pa-lam-ed  din  buļan  an. \\
big  \textsc{hsy}  \textsc{def.m}  snake  \textsc{t.r}-swallow  3\textsc{s.erg}  moon/month  \textsc{def.m} \\
\glt `\textbf{Lunar eclipse}, a belief of people here on Cagayancillo is that when \textbf{as} \textbf{for} \textbf{the} \textbf{moon}, the sky snake eats (it), lunar eclipse, \textbf{as for the big snake} s/he swallowed the moon.’ [JCOE-C-03 2.1]
\z

Notice that the sentence-initial fronted RP and its repeated form later in the sentence are not  arguments of any clause in the sentence. There are, however, two fronted RPs that are arguments of a clause. The verb \textit{pati} ‘believe’ takes a long complement clause which is made  up of a conditional clause and a main clause. Both the conditional clause and the main clause have fronted RPs. In the conditional clause, the O argument, \textit{buļan} ‘moon’, is fronted to the initial position of its clause. In the main clause the A argument, \textit{bakod an na bekkessan} ‘the big snake’, is fronted to a position following the fronted RP that states the global theme. In this text, the moon and the snake are participants that have important roles in the global theme.

In the same text two sentences later \REF{bkm:Ref113975815}, a lower-level theme is introduced-- people trying to frighten the snake (which is eating the moon). The theme, partially represented by the phrase \textit{mga ittaw} ‘people’, is expressed in the fronted RP. Notice that the fronted RP is an argument of the immediately following dependent clause and of the main clause.
\ea 
\label{bkm:Ref372884211}
\label{bkm:Ref113975815}
Ta,  \textbf{mga}  \textbf{ittaw}  \textbf{an},  tak  nakita  danen  na  \textbf{buļan}  \textbf{ya} naduwad  en  tak  palam-ed  ta  bekkessan  na  bakod, magpukpok  danen  an  ta  mga  lata,  mga  drum  o  daw  ano man na  makaatag  ta  sikad  sagbak  aged  na  \textbf{bekkessan}  \textbf{an}, maadlek ilua  din  buļan  ya,  ig  \textbf{bekkessan}  \textbf{an},  mļagan  ya  en. \\\smallskip
\hspace{.5cm} Fronted RP1 \hspace{5.2cm}                Fronted RP2 \\
\gll Ta,  \textbf{mga}  \textbf{ittaw}  \textbf{an},  tak  na-kita  danen  na  \textbf{buļan}  \textbf{ya} na-duwad  en  tak  pa-lam-ed  ta  bekkessan  na  bakod, mag-pukpok  danen  an  ta  mga  lata,  mga  drum  o  daw  ano.\footnotemark{} \\
so	\textsc{pl}	person	\textsc{def.m}	because	\textsc{a.hap.r}-see	3\textsc{p.erg}	\textsc{lk}	moon/month	\textsc{def.f} \textsc{a.hap.r}-lose	\textsc{cm}	because	\textsc{t.r}-swallow	\textsc{nabs}	snake	\textsc{lk}	big \textsc{i.ir}-beat	3\textsc{p.abs}	\textsc{def.m}	\textsc{nabs}	\textsc{pl}	can	\textsc{pl}	drum	or	if/when	what \\\smallskip
\hspace{7.8cm} Fronted RP3 \\
\gll   man na  maka-atag  ta  sikad  sagbak  aged  na  \textbf{bekkessan}  \textbf{an}, \\
also \textsc{lk}  \textsc{i.hap.ir}-give  \textsc{nabs}  very  noise  so.that  \textsc{lk}  snake  \textsc{def.m} \\
\gll ma-adlek i-lua  din  buļan  ya,  ig\footnotemark{} \\
\textsc{a.hap.ir}-afraid \textsc{t.deon}-spit.out  3\textsc{s.erg}  moon/month  \textsc{def.f}  and \\\smallskip
 {}

Fronted RP4 \\
\gll \textbf{bekkessan}  \textbf{an},  m-dļagan  ya  en. \\
snake  \textsc{def.m}  \textsc{i.v.ir}-run  \textsc{att}  \textsc{cm} \\
\footnotetext{The word \textit{ig} is a Cuyunon word meaning ‘and’ and is used sometimes in more formal types of oral speeches.}
\footnotetext{The word \textit{ano} is a \isi{Tagalog} word meaning ‘what’.}
\newpage
\glt `So, \textbf{the people}, because they saw that \textbf{as for the moon} (it) happened to disappear because the big snake swallowed it, they will beat on cans, drums and whatever else that can give out very noisy (sound) so that when \textbf{the} \textbf{snake} will be afraid, s/he will have to spit out the moon and \textbf{the} \textbf{snake} will run away.’ [JCOE-C-03 2.3]
\z

Fronted RPs also mark sentences that signal a change of theme. An expository text of any significant length usually has more than one lower-level theme. When the speaker changes from one lower-level theme to another, the first mention of theme 2 occurs in a fronted RP, and this usually marks the beginning of a new paragraph. Theme 2 is not necessarily a brand new theme, but may be a previously mentioned theme that is being reintroduced. For example, the following three examples are from a text in which the global theme is the speaker’s family, introduced with a series of fronted RPs in the first paragraph. The second paragraph introduces the speaker and her husband as a secondary theme \REF{bkm:Ref372873939}, with the fronted pronoun \textit{kami}. In the third paragraph, the first-born daughter is introduced as another secondary theme, with the fronted RP \textit{panganay na bata} ‘firstborn child’ \REF{bkm:Ref122614214}. Finally, in the fourth paragraph the speaker returns to the secondary theme of herself and her husband \REF{bkm:Ref372873998}, using another fronted RP \textit{kami na magsawa} ‘my husband and I’:

\ea 
\label{bkm:Ref372873939}
Second paragraph, second sentence: \\
\textbf{Kami},  may  kabataan  na  limma  buok. \\\smallskip
Fronted RP \\
\gll \textbf{Kami},  may  ka-bata-an  na  limma  buok. \\
1\textsc{p.excl.abs}  \textsc{ext.in}  \textsc{nr-}child-\textsc{nr}   \textsc{lk}  five  piece \\
\glt ‘\textbf{As} \textbf{for} \textbf{us}, we have five children.’ [CDWE-T-01 2.2]
\z
\ea
\label{bkm:Ref122614214}
Third paragraph, first sentence: \\
\textbf{A}  \textbf{panganay}  \textbf{na}  \textbf{bata},  bai,  yaan  nagaiskwila  ta  Talaga Elementary School. \\\smallskip
\hspace{.5cm}Fronted RP \\
\gll A  \textbf{panganay}  \textbf{na}  \textbf{bata},  bai,  yaan  naga-iskwila\footnotemark{}  ta  Talaga Elementary School. \\
\textsc{inj}  first.born  \textsc{lk}  child  female  \textsc{spat.def}  \textsc{i.r}-school  \textsc{nabs}  Talaga Elementary School \\
\footnotetext{The prefix \textit{naga}- is a borrowing from Ilongo.}
\glt `\textbf{The} \textbf{first} \textbf{born}, a girl, she went to school at Talaga Elementary School.’[CDWE-T-01 3.1]
\z
\ea 
\label{bkm:Ref372873998}
Fourth paragraph, first sentence: \\
Umpisa  ta  pag-iskwila  ta  ame  na  kabataan,  \textbf{kami}  \textbf{na}  \textbf{magsawa}  nagasagod  kay  ta  baboy. \\\smallskip
\gll Umpisa  ta  pag-iskwila  ta  ame  na  ka-bata-an, \\
begin \textsc{nabs} \textsc{nr.act}-school \textsc{nabs} 1\textsc{p.excl.gen} \textsc{lk} \textsc{nr}-child-\textsc{nr} \\\smallskip
 {}

Fronted RP \\
\gll \textbf{kami}  \textbf{na}  \textbf{mag-sawa}  naga-sagod  kay  ta  baboy. \\
1\textsc{p.excl.abs}  \textsc{lk}  \textsc{rel}-spouse  \textsc{i.r}-raise  1\textsc{p.excl.abs}  \textsc{nabs}  pig \\
\glt `Beginning when our children went to school, \textbf{my} \textbf{husband} \textbf{and} \textbf{I}, we raised  pigs.’ [CDWE-T-01 4.1]
\z
Fronted RPs also mark sentences that summarize themes. These summaries occur at the ends of paragraphs, sections, and texts. Example \REF{bkm:Ref372876159} is from the eclipse text. It is the last sentence in a paragraph explaining the belief that a pregnant woman should not look at an eclipse. Here the deictic \textit{yon} ‘that’ is the fronted RP.

\ea 
\label{bkm:Ref372876159}
\textbf{Yon},  isya  man  na  pagpati  ta  mga  inay  na  gabagnes. \\\smallskip
Fronted RP \\
\gll \textbf{Yon},  isya  man  na  pag-pati  ta  mga  inay  na  ga-bagnes. \\
\textsc{d}3\textsc{abs}  one  also  \textsc{lk}  \textsc{nr.act}-believe  \textsc{nabs}  \textsc{pl}  mother  \textsc{lk}  \textsc{i.r}-pregnant \\
\glt ‘\textbf{That}, it is another belief of mothers who are pregnant.’ [JCOE-C-03 5.2]
\z

One might argue that \REF{bkm:Ref372876159} is a cleft construction; however, fronted RP constructions and cleft constructions differ in two ways. First, fronted RPs are always followed by a phonological pause (as indicated by the comma in \ref{bkm:Ref372876159}), while head RPs of cleft constructions are never followed by a pause. Second, in cleft constructions, the head RP is the absolutive of a following nominalized clause (see \sectref{sec:cleftconstructions} below),  whereas the clause that follows a fronted RP is not necessarily nominalized. The following are some additional examples of fronted RPs from the corpus.
\ea 

Ta  daw  may  sabid  ittaw  an  na  nagamasakit  na gainay-inay  ta  luya,  yon  na  masakit  nan  o  sabid naan  galin  ta  mga  ingkantado,  tak \textbf{ingkantado}  \textbf{i},  dili  gulpi  mag-atag  ta  masakit. \\\smallskip
\gll Ta  daw  may  sabid  ittaw  an  na  naga-masakit\footnotemark{}  na ga-inay\sim inay  ta  luya,  yon  na  masakit  nan  o  sabid naan  ga-alin  ta  mga  ingkantado, \\
so  if/when  \textsc{ext.in}  sickness.from.spirits  person  \textsc{def.m}  \textsc{lk}  \textsc{i.r}-sick  \textsc{lk}
\textsc{i.r}-\textsc{red}\sim slow  \textsc{nabs}  weak  \textsc{d}3\textsc{abs}  \textsc{lk}  sick  \textsc{d}3\textsc{abs.pr}  or  sickness.from.spirits \textsc{spat.def}  \textsc{i.r}-from  \textsc{nabs}  \textsc{pl}  fairy \\\smallskip
 {}

\hspace{1.4cm}Fronted RP \\
\gll tak \textbf{ingkantado}  \textbf{i},  dili  gulpi  mag-atag  ta  masakit. \\
because fairy  \textsc{def.n}  \textsc{neg.ir}  suddenly  \textsc{i.ir}-give  \textsc{nabs} sick \\
\footnotetext{The prefix \textit{naga-} in this example is another borrowing from Ilongo. The form in Kagayanen would be simply \textit{ga}{}-.}
\glt `So if the sick person has a sickness from a spirit in which he gradually becomes weak, that very sickness or sickness from a spirit, it came from fairies because \textbf{the} \textbf{fairy}, it does not give sickness suddenly.’ [CBWE-T-07 4.1]
\z

Finally, fronted RPs mark sentences that signal contrastive focus, another type of pragmatic focus. For this type of focus, one item is selected from a presupposed set in which the correct values are restricted \citep{dik1981}. In the eclipse text, the speaker talks about the beliefs of the older generation concerning the eclipse. Then in \REF{bkm:Ref372890039}, he contrasts the old generation with the new generation of young people who do not hold those beliefs. The new generation is the selected item and occurs in a fronted RP which is followed by a second fronted RP \textit{mga ittaw} ‘the people’ which also refers to the same people, the new generation.

\ea 
\label{bkm:Ref372890039}
Piro  anduni  ta  \textbf{mga}  \textbf{bag-ongtubo,}  \textbf{mga}  \textbf{ittaw}  \textbf{i}  uļa  en  gapati ta  iling  tan. \\\smallskip
\hspace{3.4cm}Fronted RP1\hspace{1.3cm}  Fronted RP2 \\
\gll Piro  anduni  ta  \textbf{mga}  \textbf{bagongtubo,}  \textbf{mga}  \textbf{ittaw}  \textbf{i}  uļa  en  ga-pati ta  iling  tan. \\
but  now/today  \textsc{nabs}  \textsc{pl}  new.generation  \textsc{pl}  person  \textsc{def.n}  \textsc{neg.r}  \textsc{cm}  \textsc{i.ir}-believe
\textsc{nabs}  like  \textsc{d}3\textsc{nabs} \\
\glt `But now regarding \textbf{the} \textbf{new} \textbf{generation}, \textbf{the} \textbf{people}, they do not believe (in things) like that.’ [JCOE-C-03 4.2]
\z

In example \REF{bkm:Ref122614892} the “other boats” are contrasted with \textit{kami} the first person plural pronoun, both of which are fronted.

\ea 
\label{bkm:Ref122614892}
\textbf{Duma}  \textbf{ya}  \textbf{na}  \textbf{bļangay}  ubos  balik  ta  Cagayancillo. \textbf{Kami}  \textbf{i}  nang  gadiritso  ta  Anini-y. \\\smallskip
Fronted RP1 \\
\gll \textbf{Duma}  \textbf{ya}  \textbf{na}  \textbf{bļangay}  ubos  balik  ta  Cagayancillo. \\
other  \textsc{def.f}  \textsc{lk}  2.masted.boat  all  return  \textsc{nabs}  Cagayancillo \\\smallskip
\newpage
Fronted RP2 \\
\gll \textbf{Kami}  \textbf{i}  nang  ga-diritso  ta  Anini-y. \\
1\textsc{p.excl.abs}  \textsc{def.n}  only  \textsc{i.r}-straight  \textsc{nabs}  Anini-y \\
\glt `\textbf{The} \textbf{other} \textbf{two-masted} \textbf{boats} all returned to Cagayancillo. \textbf{We} only went straight to Anini-y.’ [VAWN-T-18 6.4]
\z

Question words in information questions occur preverbally (see \sectref{bkm:Ref445900292}).  In answers to such questions, the focused element occurs in either the normal word order or preverbally. In the following example the answer has two clauses; the first is a nonverbal clause and the second is a verbal clause. In the nonverbal clause the topic \textit{gadaag ya} ‘the one who won’ is fronted before the comment \textit{umang} ‘hermit crab’ which is also the focus item. The verbal clause has normal word order.

\ea 
Q: \textbf{Kino}  gadaag  ya  ta  lumba  ya? \\
A: O,  gadaag  ya  \textbf{umang}.  Gadaag  \textbf{umang}  ya. \\\smallskip

\gll Q: \textbf{Kino}  ga-daag  ya  ta  lumba  ya? \\
{} who  \textsc{i.r}-win  \textsc{def.f}  \textsc{nabs}  race  \textsc{def.f} \\

\gll A: O,  ga-daag  ya  \textbf{umang}. Ga-daag umang ya. \\
{} oh  \textsc{i.r}-win  \textsc{def.f}  hermit.crab \textsc{i.r}-win hermit.crab \textsc{def.f} \\

\glt Q: `\textbf{Who} was the one who won the race?' \\
A: 'Oh, the one who won was \textbf{hermit} \textbf{crab}. \textbf{Hermit} \textbf{crab} won.’ [JCON-T-08 55.3-4]
\z
\ea 
Q: \textbf{Man-e}  \textbf{tak}  gadaag  umang  ya? \\
A: Gadaag  umang  ya  \textbf{tak}  \textbf{pausar}  \textbf{din}  \textbf{utok}  \textbf{din}  \textbf{an}. \\\smallskip
\gll Q: \textbf{Man-e}  \textbf{tak}  ga-daag  umang  ya? \\
{} why  because  \textsc{i.r}-win  hermit.crab  \textsc{def.f} \\

\gll A: Ga-daag  umang  ya  \textbf{tak}  \textbf{pa-usar}  \textbf{din}  \textbf{utok}  \textbf{din}  \textbf{an}. \\
{} \textsc{i.r}-win  hermit.crab  \textsc{def.f}  because  \textsc{t.r}-use  3\textsc{s.erg}  brain  3\textsc{s.gen}  \textsc{def.m} \\

\glt Q: `\textbf{Why} did hermit crab win?' \\
A: 'Hermit crab won \textbf{because} \textbf{he} \textbf{used} \textbf{his} \textbf{brain}.' [JCON-T-08 55.5-6]
\z

\ea 
Painsaan  din  isab  gasukot  ya  ki  kanen, “\textbf{Ino}  baba  ta  sidda  ya  na  gautang  ki  kaon?” Ambaļ  ta  gasukot,  “\textbf{Mļangkaw}  iya  na  baba.” \\\smallskip
\gll Pa-insa-an  din  isab  ga-sukot  ya  ki  kanen, “\textbf{Ino}  baba  ta  sidda  ya  na  ga-utang  ki  kaon?” ambaļ  ta  ga-sukot,  “\textbf{Mļangkaw}  iya  na  baba.” \\
\textsc{t.r}-ask-\textsc{apl}  3\textsc{s.erg}  again  \textsc{i.r}-collect.payment  \textsc{def.f}  \textsc{obl.p}  3s what  mouth  \textsc{nabs}  fish  \textsc{def.f}  \textsc{lk}  \textsc{i.r}-debt  \textsc{obl.p}  2s say  \textsc{nabs}  \textsc{i.r}-collect.payment  long  3\textsc{s.gen}  \textsc{lk}  mouth \\
\glt `He asked again the one who was collecting money from him, “\textbf{What} was the mouth of the fish who was borrowing money from you?” The one collecting the payment said, “His mouth was \textbf{long}.”' [EMWN-T-08 3.4-5]
\z

\subsection{Discourse particles}
\label{bkm:Ref118460437} \label{sec:discourseparticles}
Discourse particles are small words that are used by speakers to guide the addressee through the structure of the discourse. For example, they may signal points of transition (e.g., paragraph and episode boundaries), points of heightened tension, climax, and contrast. They also may suggest how ideas are related to one another. In English, words such as \textit{therefore}, \textit{so} and \textit{OK} may be considered discourse particles. Here we discuss only three common discourse particles in Kagayanen, \textit{a}, \textit{gani}, and \textit{ta}. There are undoubtedly others, but a full discourse study is needed to elucidate all of their usages. Some of the words we have described as adjunct adverbs in \chapref{chap:modification}, \sectref{sec:adjunctadverbs} may also qualify as discourse particles. As with most distinctions proposed by linguists, the differences among discourse particles, adjunct adverbs, and interjections may not be absolute.

When the particle \textit{a} occurs after a noun or noun phrase, it indicates contrast \REF{bkm:Ref372900695}. For this reason, we call it a \is{contrast particle}contrast particle. However, when it occurs at the end of a sentence, it indicates truth value focus \REF{bkm:Ref372900724}. It also appears utterance initially to mark special discourse functions, as discussed in \sectref{bkm:Ref113971883}.

\ea 
\label{bkm:Ref372900695}
Buli  i  daw  imo  na  bitan  paryo  ta  tama  pantad   na gasapļa  daw  gabiskeg  bitan  no  tak  may  bekkeg  na  gaduma. \textbf{Pandan}  \textbf{i}  \textbf{a}  paryo  ta  isya  na  lima  na  uļa  ubra  na  daw bitan  no  sikad  na  yem-ek  daw  mapino. \\\smallskip
\gll Buli  i  daw  imo  na  \emptyset{}-ibit-an  paryo  ta  tama  pantad   na ga-sapļa  daw  ga-biskeg  \emptyset{}-ibit-an  no  tak  may  bekkeg  na  ga-duma. \textbf{Pandan}  \textbf{i}  \textbf{a}  paryo  ta  isya  na  lima  na  uļa  ubra  na  daw \emptyset{}-ibit-an  no  sikad  na  yem-ek  daw  ma-pino. \\
buri  \textsc{def.n}  if/when  2\textsc{s.gen}  \textsc{lk}  \textsc{t.ir}-hold-\textsc{apl}  same  \textsc{nabs}  many  sand  \textsc{lk} \textsc{i.r}-rough  and  \textsc{i.r}-strong  \textsc{t.ir}-hold-\textsc{apl}  2\textsc{s.erg}  because  \textsc{ext.in}  bone  \textsc{lk}  \textsc{i.r}-with pandan  \textsc{def.n}  \textsc{ctr}  same  \textsc{nabs}  one  \textsc{lk}  hand  \textsc{lk}  \textsc{neg.r}  work  \textsc{lk}  if/when \textsc{t.ir}-hold-\textsc{apl} 2\textsc{s.erg}  very  \textsc{lk}  soft  and  \textsc{adj}-fine \\
\glt `Buri (leaf), if you feel it (it is) like much sand that is rough, and hard is what you feel because there is a rib with it. Pandan (leaf), (it is) like a hand that has no work, that if you feel it (it is) very soft and fine.’ [DBWE-T-18 8.1]
\z
\ea 
\label{bkm:Ref372900724}
Lain  man  nyan,  bao  san  na  agas  \textbf{a}. \\\smallskip
\gll Lain  man  nyan,  bao  san  na  agas  \textbf{a}. \\
bad  \textsc{emph}  \textsc{d}2\textsc{abs}  odor  \textsc{d}2\textsc{nabs}  \textsc{lk}  kerosine  truly \\
\glt ‘That is truly very bad, the odor of that kerosene.’ [RZWN-T-02 4.10]
\z

Examples \REF{bkm:Ref441496729} and \REF{bkm:Ref441496733} compare two rituals. The contrast particle \textit{a} is homophonous with the first person singular pronoun and two interjections (see \sectref{bkm:Ref113971883} on interjections). The difference is that the contrast particle occurs after the noun phrase it is contrasting, and has a level intonation that is held longer than for ordinary words.

\ea 
\label{bkm:Ref441496729}
\textit{a} Contrast \\
Duļot  i  \textbf{a}  isya  na  buaten  ta  isya  na  surano daw  may  mag-umaw  ki  kanen  na  gamasakit. \\\smallskip
\gll Duļot  i  \textbf{a}  isya  na  buat-en  ta  isya  na  surano daw  may  mag-umaw  ki  kanen  na  ga-masakit. \\
food.offering  \textsc{def.m}  \textsc{ctr}  one  \textsc{lk}  make/do-\textsc{t.ir}  \textsc{nabs}  one  \textsc{lk}  healer
if/when  \textsc{ext.in}  \textsc{i.ir}-call  \textsc{obl.p}  \textsc{3s}  \textsc{lk}  \textsc{i.r}-sick \\
\glt `\textit{Duļot} food offering is one thing a healer does when someone who is sick calls him/her.’ [VAOE-J-04 1.1]’
\z

\ea 
\label{bkm:Ref441496733}
Mikaw  i  \textbf{a}  iya  man  ta  surano  na  ubra  ni sise  nang  dipirinsya  ta  duļot. \\\smallskip
\gll Mikaw  i  \textbf{a}  iya  man  ta  surano  na  ubra  ni sise  nang  dipirinsya  ta  duļot. \\
food.offering  \textsc{def.n}  \textsc{ctr}  3\textsc{s.gen}  also  \textsc{nabs}  healer  \textsc{lk}  work  \textsc{d1abs}
little  only  difference  \textsc{nabs}  food.offering \\
\glt `\textit{Mikaw} food offering, this is also the work of the healer, with only a little difference from \textit{duļot} food offering.’ [VAOE-J-04 2.1]’
\z

The word \textit{gani} when it occurs in second position indicates truth value focus of the whole sentence  %(CITATION NEEDED)
. It is usually used to counter what another person assumes. Example \REF{bkm:Ref373133996} is part of a conversation about where the ancestors of the Kagayanen people originated. The one telling the story says that some people came to Cagayancillo on a boat from another island in the south called Cagayan de Sulu. The one who asks the question below assumes they were going to a certain destination. The one telling the story corrects that wrong assumption by saying they were just fleeing from Cagayan de Sulu island.

\ea 
\label{bkm:Ref373133996}
Piro  indi  danen  inta  punta? Galayas  \textbf{gani}  danen  Cagayan  Sulu  tak  adlek  na  tulien. \\\smallskip
\gll Piro  indi  danen  inta  punta? Ga-layas  \textbf{gani}  danen  Cagayan  Sulu  tak  adlek  na  tuli-en. \\
but  where  3\textsc{p.abs}  \textsc{opt}  going \textsc{i.r}-flee  truly  3\textsc{p.abs}  Cagayan  Sulu  because  afraid  \textsc{lk}  circumcision-\textsc{t.ir} \\
\glt ‘But, where were they going? They were truly fleeing from Cagayan Sulu because (they) were afraid to be circumcised.’ [MOOE-C-01 16-17]
\z

The form \textit{ta} in sentence-initial position, followed by a pause often expresses a conclusion to what has been said earlier. It sometimes follows a flashback, background information, digression or quotation, and brings the text back to the main point of the story or conversation.

\ea 
\label{bkm:Ref445900445}
\textit{ta} ‘so’ (return to main topic of discussion after a quote) \\
\textbf{Ta,}  gapanaw  en  darwa  i  na  mag-utod. \\\smallskip
\gll \textbf{Ta,}  ga-panaw  en  darwa  i  na  mag-utod. \\
so  \textsc{i.r}-walk/go \textsc{cm}  two  \textsc{def.n}  \textsc{lk}  \textsc{rel-}sibling \\
\glt ‘So, the two siblings left.’  (This follows a conversation between the  two siblings.) [CBWN-C-22 9.1]
\z
\ea
\textbf{Ta,}  gadiritso  kay  ame  so  Barrio. \\\smallskip
\gll \textbf{Ta,}  ga-diritso  kay  ame  so  Barrio \\
so  \textsc{i.r}-straight  1\textsc{p.excl.abs} 1\textsc{p.excl.gen} \textsc{d4abs.pr} Barrio \\
\glt So, we ourselves went  straight there to Baryo.’ (Previous to  this sentence was a digression about  some people fishing with dynamite.) [DBON-C-09 2.9] 
\z
\ea
\textbf{Ta},  utod  din  ya  pasikway  din. \\\smallskip
\gll \textbf{Ta},  utod  din  ya  pa-sikway  din. \\
so  sibling  3\textsc{s.gen}  \textsc{def.f}  \textsc{t.r}-reject  3\textsc{s.erg} \\
\glt ‘So, his brother rejected (him).’ [RBON-T-01 1.12]
\z

\subsection{Cleft constructions}
\label{bkm:Ref122603009}\label{sec:cleftconstructions}

A cleft construction in Kagayanen consists of a clause with a noun phrase fronted before the verb and no resumptive pronoun left \textit{in situ}. This same construction can function either as topicalization or as focal prominence construction (\ref{bkm:Ref329967612}-\ref{bkm:Ref113976722}).

\ea 
\label{bkm:Ref329967612}
Pedro  ya  gaatag  ki  yaken  ta  kwarta. \\\smallskip
\gll Pedro  ya  ga-atag  ki  yaken  ta  kwarta. \\
Pedro  \textsc{def.f}  \textsc{i.r}-give  \textsc{obl.p}  1s  \textsc{nabs}  money \\
\glt ‘Pedro is the one who gave me money.' Or 'It was Pedro who gave me money.’
\z
\ea
Kwarta  i  paatag  din  ki  yaken. \\\smallskip
\gll Kwarta  i  pa-atag  din  ki  yaken. \\
money  \textsc{def.n}  \textsc{t.r}-give  3\textsc{s.erg}  \textsc{obl.p}  1s \\
\glt ‘The money is what s/he gave me.' Or 'It was the money that s/he gave me.’
\z
\ea
\label{bkm:Ref113976722}
Yaken  i  paatagan  din   ta  kwarta. \\\smallskip
\gll Yaken  i  pa-atag-an  din   ta  kwarta. \\
1\textsc{s.abs}  \textsc{def.n}  \textsc{t.r}-give-\textsc{apl}  3\textsc{s.erg}  \textsc{nabs}  money \\
\glt ‘I was the one s/he gave money to.’ Or ‘It was me whom s/he gave money to.’
\z

A different kind of cleft construction consists of a nominalized clause followed by a RP as in \REF{bkm:Ref329967818}. The nominalized clause has the regular verbal affixes and a demonstrative determiner \textit{i}, \textit{an}, or \textit{ya} after the verb. The RP has no determiner. When the verb in the nominalized clause is inflected as intransitive, the meaning is ‘the one who did the action’. The examples in \REF{bkm:Ref329968346} show that the determiner cannot occur in other positions than after the verb, including at the end of the nominalized clause (\ref{bkm:Ref329968346}a, b, c), and the RP must occur in clause-final position (\ref{bkm:Ref329968346}d, e).

\ea 
\label{bkm:Ref329967818}
Gaatag  an  ki  yaken  ta  kwarta  Pedro. \\\smallskip
Nominalized clause\hspace{2cm}              RP \\
\gll Ga-atag  an  ki  yaken  ta  kwarta  Pedro. \\
\textsc{i.r}-give  \textsc{def.m}  \textsc{obl}  1s  \textsc{nabs}  money  Pedro \\
\glt ‘The one who gave me money was Pedro.’
\z
\ea 
\label{bkm:Ref329968346}
    \ea[*]{
    \label{ex:takwartapedro}
    Gaatag ki yaken an ta kwarta Pedro. \\
    }
    \ex[*]{
    \label{ex:takwartaanpedro}
    Gaatag ki yaken ta kwarta an Pedro. \\
    }
    \ex[*]{
    \label{ex:takwartapedroi}
    Gaatag an ki yaken ta kwarta Pedro i. \\
    }
    \ex[*]{
    \label{ex:kiyakentakwarta}
    Pedro gaatag an ki yaken ta kwarta. \\
    }
    \ex[*]{
    \label{ex:kiyakentakwarta-2}
    Pedro an gaatag an ki yaken ta kwarta.
    }
    \z
\z

When the verbal inflection is transitive, the verb is followed by a clitic ergative pronoun indicating the Actor and the demonstrative determiner occurs after the clitic pronoun. Any oblique phrases occur after the determiner (exs. \ref{bkm:Ref329968477} and \ref{bkm:Ref329968698}). The determiner cannot occur at the end of the nominalized clause as in (\ref{bkm:Ref235676492}a, b) and (\ref{bkm:Ref329968619}a, b). After the nominalized clause is the RP that cannot take a determiner. It also cannot occur before the nominalized clause (exs. \ref{ex:kiyaken} and \ref{ex:takwarta}).
\ea 
\label{bkm:Ref329968477}
Patag  din  an  ki  yaken  kwarta. \\\smallskip
Nominalized clause     \hspace{2.5cm} RP \\
\gll Pa-atag  din  an  ki  yaken  kwarta. \\
\textsc{t.r}-give  3\textsc{s.erg}  \textsc{def.m}  \textsc{obl.p}  1s  money \\
\glt ‘What s/he gave to me was money.’
\z

\ea
\label{bkm:Ref235676492}
    \ea[*]{
    \label{ex:ankwarta}
    Paatag din ki yaken an kwarta. \\
    }
    \ex[*]{
    \label{ex:kwartai}
    Paatag din an ki yaken kwarta i. \\
    }
    \ex[*]{
    \label{ex:kiyaken}
    Kwarta paatag din an ki yaken. \\
    }
    \z
\z
\ea 
\label{bkm:Ref329968698}
Paatagan  din  an  ta  kwarta  yaken. \\\smallskip
Nominalized clause\hspace{3.6cm}RP \\
\gll Pa-atag-an  din  an  ta  kwarta  yaken. \\
\textsc{t.r}-give-\textsc{apl}  3\textsc{s.erg}  \textsc{def.m}  \textsc{nabs}  money  1\textsc{s.abs} \\
\glt ‘The one s/he gave money to was me.’
\z
\ea 
\label{bkm:Ref329968619}
    \ea[*]{
    \label{ex:anyaken}
    Paatagan din ta kwarta an yaken. \\
    }
    \ex[*]{
    \label{ex:yakeni}
    Paatagan din an ta kwarta yaken i. \\
    }
    \ex[*]{
    \label{ex:takwarta}
    Yaken paatagan din an ta kwarta.
    }
    \z
\z

Cleft constructions with a demonstrative initially usually occur as a summary sentence for a paragraph, section or text.

\newpage
\ea 
\textbf{Yon}  na  isturya  na  inagian  ko  na  sise  a  pa. \\\smallskip
\gll \textbf{Yon}  na  isturya  na  <in>agi-an  ko  na  sise  a  pa. \\
\textsc{d}3\textsc{abs}  \textsc{lk}  story  \textsc{lk}  <\textsc{nr.res}>pass-\textsc{nr}  1\textsc{s.gen}  \textsc{lk}  small  1\textsc{s.abs}  \textsc{inc} \\
\glt ‘\textbf{That} was the story which was my experience when I was small.’ [BMON-C-02 1.17]
\z

\ea 
Galebbeng  an  ta  mga  patay  mga  seed  nang  na  mga  utod. \\\smallskip
\gll Ga-lebbeng  an  ta  mga  patay  mga  seed  nang  na  mga  utod. \\
\textsc{i.r}-bury    \textsc{def.m}  \textsc{nabs}  \textsc{pl}  dead  \textsc{pl}  close  only  \textsc{lk}  \textsc{pl}  sibling \\
\glt ‘The ones burying the dead are only the close relatives.’ [JCWN-T-21 13.7]
\z

Another way clefts are structured is with a definite referring phrase followed by another referring phrase, which is the normal way of constructing predicate nominal constructions  (see \chapref{chap:non-verbalclauses}, \sectref{sec:predicatenominals}).

\ea 
Sakayan  an  di  batil,    bļangay  daw  lansa. \\\smallskip
[\hspace{1.1cm}RP1\hspace{1.1cm}]\hspace{.5cm}   [\hspace{2.6cm}RP2\hspace{2.6cm}] \\
\gll Sakay-an  an  di  batil,    bļangay  daw  lansa. \\
ride-\textsc{nr}  \textsc{def.n}  \textsc{d}1\textsc{loc}  1.masted.boat  2.masted.boat  and  launch \\
\glt ‘The transportation here is one-masted boats, two-masted boats and launches.’ [SAWE-T-01 2.4]
\z
\ea 
Patugtog  an  ta  banda  mga  tukar  an  kingmanakem  en  pugya  a. \\\smallskip
[\hspace{1cm}RP1\hspace{1cm}]\hspace{.4cm} [\hspace{3.5cm}RP2\hspace{3.5cm}] \\\smallskip
\gll Pa-tugtog  an  ta  banda  mga  tukar  an  king-manakem  en  pugya  a. \\
\textsc{t.r}-play.music  \textsc{def.m}  \textsc{nabs}  band  \textsc{pl}  music  \textsc{def.n}  style-older  \textsc{cm}  long.ago  truly \\
\glt ‘What the band played was music of the style of older people long ago truly.’ [PBON-T-01 6.17 edited for naturalness]
\z

\ea 
Sakit  en  an  namatian  ko  ya. \\\smallskip
\gll Sakit  en  an  na-mati-an  ko  ya. \\
pain  \textsc{cm}  \textsc{def.n}  \textsc{a.hap.r}-hear-\textsc{apl}  1\textsc{s.erg}  \textsc{def.f} \\
\glt ‘What is hurtful now is what I hear.’ (In the context others were laughing and so the speaker felt emotional pain that they were laughing at him.) [JCON-L-07 10.3]
\z
\ea 
Tape  nang  gaambaļ  an  piro  sikad  dayad  man. \\\smallskip
\gll Tape  nang  ga-ambaļ  an  piro  sikad  dayad  man. \\
tape  only/just  \textsc{i.r}-say  \textsc{def.m}  but  very  good  also \\
\glt ‘Just a tape was what was speaking, but it was very good also.’ [BMON-C-05 16.4]
\z
\ea 
 Di  mag-ubra  ya  bali  mga  Kagayanen.  Mag-ubra  ya  ta resorts  mga  Kagayanen,  piro  gamanage  ya  a  Club Noah. \\\smallskip
\gll Di  mag-ubra  ya  bali  mga  Kagayanen.  Mag--ubra  ya  ta resorts  mga  Kagayanen,  piro  ga-manage  ya  a  Club Noah. \\
\textsc{inj.rq}  \textsc{i.ir}-work/do  \textsc{def.f}  amounts.to  \textsc{pl}  Kagayanen  \textsc{i.ir}-work/do  \textsc{def.f}  \textsc{nabs}
resorts  \textsc{pl}  Kagayanen  but  \textsc{i.r}-manage  \textsc{def.f}  \textsc{ctr}  Club Noah \\
\glt `What else would it be, the ones who will work turn out to be (lit. amount to) Kagayanens. The ones who will work at the resorts are Kagayanens, but the ones who manage are Club Noah.’ [MOOE-C-01 234.1-2]
\z
\ea 
Kano  nang  bui  ya... \\\smallskip
\gll Kano  nang  bui  ya... \\
American  only  live  \textsc{def.f} \\
\glt ‘Only the American man is the one who is alive.’ (This is about a plane crash off the shores of the island and there was only one survivor.) [MOOE-C-01 145.1]
\z
\ea 
Bula  kon  an  lisen.  Ugsak  ta  bula  ya  buļak.  Ugsak  ya  ta buļak  ya  ilo.  Ugsak  ya  ta  ilo  ya  waig.  Suwa. \\\smallskip
\gll Bula  kon  an  lisen.  Ugsak  ta  bula  ya  buļak.  Ugsak  ya  ta buļak  ya  ilo.  Ugasak  ya  ta  ilo  ya  waig.  Suwa. \\
ball  \textsc{hsy}  \textsc{def.n}  round  inside  \textsc{nabs}  ball  \textsc{def.f}  cotton.  inside  \textsc{def.f}  \textsc{nabs}
cotton  \textsc{def.f}  thread  inside  \textsc{def.f}  \textsc{nabs}  thred  \textsc{def.f}  water  orange \\
\glt `The ball is round. What is inside the ball is cotton. What is inside the cotton is thread. What is inside the thread is water. Citrus fruit.’ [MRWR-T-01 9.3]
\z

\newpage
\ea 
Pari  din  ya  yon  palabi  din. \\\smallskip
\gll Pari  din  ya  yon  pa-labi  din. \\
friend  3\textsc{s.gen}  \textsc{def.f}  \textsc{d}3\textsc{abs}  \textsc{t.r}-favor  3\textsc{s.erg} \\
\glt ‘It was his friend, that one he favored.’[RBON-T-01 1.13]
\z
\ea 
Ittaw  ya  na  galabyog  gasinggit,  “Apo  Kagiyaw,  anen  pagkaan no  i  en." \\\smallskip
\gll Ittaw  ya  na  ga-labyog  ga-singgit,  “Apo  Kagiyaw,  anen  pagkaan no  i  en." \\
person  \textsc{def.f}  \textsc{lk}  \textsc{i.r}-throw  \textsc{i.r}-shouts  Ancestor  Kagiyaw  \textsc{ext.g}  food
2\textsc{s.gen}  \textsc{def.n}  \textsc{cn} \\
\glt `The person who throws, shouts, “Ancestor Kagiyaw, here is your food.”' [VAWN-T-17 3.3]
\z
\ea 
Yo  waig  din  ya  en  na  palaga  ya  duma  ta tangļad  ya,  yo  inemén. \\\smallskip
\gll Yo  waig  din  ya  en  na  pa-laga  ya  duma  ta tangļad  ya,  yo  inem-én. \\
\textsc{d}4\textsc{abs}  water  3\textsc{s.gen}  \textsc{def.f}  \textsc{cm}  \textsc{lk}  \textsc{t.r}-boil  \textsc{def.f}  with  \textsc{nabs}
lemon.grass  \textsc{def.f}  \textsc{d}4\textsc{abs}  drink-\textsc{nr} \\
\glt ‘That, its water which is the one boiled with the lemon grass, that is what is to be drunk.’ [DBOE-C-04 9.3]
\z
\ea 
Uļa  a  naļam  lain  pa  ya  gian  ko  ya  en.\\\smallskip
\gll Uļa  a  na-aļam  lain  pa  ya  agi-an  ko  ya  en.\\
\textsc{neg.r}  1\textsc{s.abs}  \textsc{a.hap.r}-know  diffferent  \textsc{inc}  \textsc{def.f}  pass-\textsc{nr}  1\textsc{s.gen}  \textsc{def.f}  \textsc{cm}\\
\glt ‘I did not know that what was different was my path.’ [DBON-C-08 2.6]
\z
\ea 
Iya  ya  na  kamatayen  granada. \\\smallskip
\gll Iya  ya  na  kamatayen  granada. \\
3\textsc{s.gen}  \textsc{def.f}  \textsc{lk}  death  grenade \\
\glt ‘His death was a grenade.’ [MBON-T-07 14.2]
\z
\ea 
Tanem  din  ya  40  nang  puon  na  sandia. \\\smallskip
\gll Tanem  din  ya  40  nang  puon  na  sandia. \\
plant  3\textsc{s.gen}  \textsc{def.f}  40  only/just  stem  \textsc{lk}  watermelon \\
\glt ‘What he planted was just 40 plants of watermelon.’ [SFOB-L-01 7.5]
%Since this is oral, should we write out "40" in whatever system the speaker used? What did the speaker actually say?
\z
\section{Interjections}
\label{bkm:Ref113971883}\label{sec:interjections}
Interjections are expressive words that aren't really part of the structure of any particular sentence, yet express something about the speaker’s emotional response or commitment to the ideas presented in nearby sentences. In English, words such as \textit{wow!}, \textit{man!}, \textit{oh boy!} and some taboo words count as interjections. Kagayanen speakers use interjections frequently to enliven and add “spice” to conversations, stories and other types of discourse. There are so many interjections, including ideophones and taboo words, that they could constitute the subject matter for an entire book. In this section, we will limit the discussion to a few common interjections found in the corpus for this study and often heard in conversation.

Most interjections in Kagayanen occur between sentences, though they can constitute a sentence (predication) in themselves and they can occur within a sentence. The following is a list of interjections that occur in the corpus and some that only occur in conversation, not in the corpus. Examples \REF{bkm:Ref114565664} through \REF{bkm:Ref114565677} illustrate these interjections in context.

\ea 
\begin{tabbing}
\hspace{1.7cm} \= \kill 
\textit{a} \> utterance-initial, falling intonation, and a pause to draw \\
\> attention to a new
   topic or to something the speaker wants \\ 
\> the addressee to pay special attention to  (\ref{bkm:Ref118531220}-\ref{bkm:Ref118532820}). \\
\textit{a} \> utterance-final, rising intonation--veridical: ‘truly’, ‘really’, \\
\> emphasizing the truth of the statement \REF{bkm:Ref118532820} second instance \\
\> of \textit{a}, and \REF{bkm:Ref118532950}. \\
\textit{aaw} \> utterance-initial--interest and excitement ‘oh really’, or some \\
\> new thought comes to mind or the speaker is reminded \\
\> of something \REF{bkm:Ref329594720}. \\
\textit{aba} \> Utterance-initial, with rising and falling intonation, followed \\
\> by a pause--‘wow’, wonderment, amazement, or excitement. \\
\> Often accompanied by  an alveolar/palatal click \REF{bkm:Ref118535175}. \\
\textit{ara} \> Expression of disagreement or dislike of something \REF{bkm:Ref118535248}. \\
\textit{aro} \> Surprise \REF{bkm:Ref118535339}. \\
\textit{ay} \> Surprise (\ref{bkm:Ref329594818}-\ref{bkm:Ref118535037}). \\
\textit{aroy/arey} \> Sympathy, ‘oh no’ (\ref{bkm:Ref329594828}-\ref{bkm:Ref118535410}). \\
\textit{ba} \> Irony, sarcasm \REF{bkm:Ref118730619}. \\
\textit{di} \> Utterance-initial, a rhetorical question--‘what else?’ It \\
\> usually occurs after digressions or quotations to indicate \\
\> return to the main topic. It can express sarcasm, \\
\> disapproval, disgust, disagreement or objection \REF{bkm:Ref118730658}. \\
\textit{dukwa} \> Politeness, tentativity (\ref{bkm:Ref118730750}-\ref{bkm:Ref118730755}). \\
\textit{i} \> Utterance-final, veridical (high likelihood of truth), \\
\> sympathy (\ref{bkm:Ref118731059}-\ref{bkm:Ref118731030}). \\
\textit{inday} \> Utterance-intial--doubt ‘I don’t know’ \REF{bkm:Ref114565677}. \\
\textit{oy} \> Attention getter--‘Hey!’ \REF{bkm:Ref118731240}. \\
\textit{ta} \> Mild rebuke--‘well of course’, ‘that’s right, ‘it’s true’, \\
\> (\ref{bkm:Ref329594868}-\ref{bkm:Ref118798909}). \\
tongue \> Excitement, joy, enjoyment. This is not transcribed in the \\
click \> corpus but occurs often in conversation. The Kagayanen \\
\> word is \textit{nakļa} ‘to  click the tongue.’
\end{tabbing}
\z 
\ea
\label{bkm:Ref118531220}\label{bkm:Ref114565664}
\textit{a} Drawing attention to new topic: \\
\textbf{A},  prublima  nang  unti  ame  i  uļa  mga  sakayan na  para  ta  basak… \\\smallskip
\gll \textbf{A},  prublima  nang  unti  ame  i  uļa  mga  sakay-an na  para  ta  basak… \\
Well,  problem  only/just  \textsc{d}1\textsc{loc.pr}  1\textsc{p.excl.gen}  \textsc{def.n}  \textsc{neg.r}  \textsc{pl}  ride-\textsc{nr}
\textsc{lk}  for  \textsc{nabs}  ground \\
\glt ‘Well, our only problem here is there is no transportation for the ground (instead of just boats for the sea).’ (This is a new topic because previous to this sentence the text was about how nice the island is and how many people come to see it.) [DBWL-T-19 9.5]
\z
\ea 
\textit{a} Drawing attention to something special: \\
\textbf{A},   mainay  na  pagsabat  nagapaadyo  ta  kagilek, piro,  \textbf{a},  mabiskeg  an  na  mga  ambaļanen  nagasegyet  ta  kagilek. \\\smallskip
\gll \textbf{A},   ma-inay  na  pag-sabat  naga-pa-adyo  ta  ka-gilek, piro,   \textbf{a},   ma-biskeg  an  na  mga  ambaļ-anen  naga-segyet  ta  ka-gilek. \\
take.note  \textsc{adj}-slow  \textsc{lk}  \textsc{nr.act}-answer  \textsc{i.r}-\textsc{caus}-far  \textsc{nabs}  \textsc{nr}-angry
but  take.note  \textsc{adj}-strong  \textsc{def.m}  \textsc{lk}  \textsc{pl}  say-\textsc{nr}   \textsc{i.r}-intice  \textsc{nabs}  \textsc{nr}-angry \\
\glt `Take note, slow answers drive anger far away, but take note, strong words entice anger.’ [JCOB-L-02 85]
\z
\ea 
\label{bkm:Ref118532820}
\textit{a} Dismay, counter expectation: \\
\textbf{A},  yaken  pa  nuļog  a. \\\smallskip
\gll \textbf{A},  yaken  pa  na-uļog  a. \\
well  1\textsc{s.abs}  \textsc{inc}  \textsc{a.hap.r}-fall  truly \\
\glt ‘Well, I even fell truly.’ (In the context, he was trying to stab a fish when fishing but instead of stabbing it he fell.) [EFWN-T-10 2.15]
\z
\ea
\label{bkm:Ref118532950}
\textit{a} Emphasis on truth, ‘truly’: \\
Malit  ki  nang  kiten  i  tenga  kilo  tak  maal sidda  an  Manila  \textbf{a}. \\\smallskip
\gll Malit  ki  nang  kiten  i  tenga  kilo  tak  maal sidda  naan  Manila  \textbf{a}. \\
\textsc{i.v.r}-buy  1\textsc{p.incl.abs}  only/just  1\textsc{p.incl.abs}  \textsc{def.n}  half  kilogram  because  expensive fish  \textsc{spat.def}  Manila  truly \\
\glt `Let’s just buy half a kilogram, because fish is expensive in Manila truly.’ [BMON-C-05 10.14]
\z

The interjections \textit{aaw} ‘oh really’, \textit{aba} ‘wow’, and \textit{ara} ‘that’s wrong’ do not occur in the corpus. The following examples are from conversations:

\ea 
\label{bkm:Ref329594720}
\textit{aaw} Excitement: \\
\textbf{Aaw}  muag  ki  kani  ta  prugrama! \\\smallskip
\gll \textbf{Aaw}  m-luag  ki  kani  ta  prugrama! \\
oh.really  \textsc{i.v.ir}-watch  1\textsc{p.incl.abs}  later  \textsc{nabs}  program \\
\glt ‘Oh really, let’s watch the program later!’ \\
\z
\ea
\label{bkm:Ref118535175}
\textit{aba} Wonderment/amazement: \\
\textbf{Aba},  gabot  gwapa  i  na  bai! \\\smallskip
\gll \textbf{Aba},  ga-abot  gwapa  i  na  bai! \\
wow  \textsc{i.r}-arrive  attractive  \textsc{def.n}  \textsc{lk}  woman \\
\glt ‘Wow, this attractive woman is/was arriving!’
\z
\ea
\label{bkm:Ref118535248}
\textit{ara} Disagreement: \\
\textbf{Ara},  man-o  yaken  i  pabangdanan  no  na  ganakaw ta  kwarta  no? \\\smallskip
\gll \textbf{Ara},  man-o  yaken  i  pa-bangdan-an  no  na  ga-nakaw ta  kwarta  no? \\
that’s.wrong  why  1\textsc{s.abs}  \textsc{def.n}  \textsc{t.r}-blame-\textsc{apl}  2\textsc{s.erg}  \textsc{lk}  \textsc{i.r}-steal \textsc{nabs}  money  2\textsc{s.gen} \\
\glt ‘That’s wrong, why am I the one you blame that (I) stole your money?’
\z
\ea 
\label{bkm:Ref118535339}
\textit{aro} Surprise: \\
\textbf{Aro},  sugat  din  baked  na  kapri. \\\smallskip
\gll \textbf{Aro},  sugat  din  baked  na  kapri. \\
\textsc{surp}  meet  3\textsc{s.erg}  big  \textsc{lk}  spirit.giant \\
\glt ‘Oh, she met a big spirit giant.’ [CBON-T-03 3.2]
\z
\ea
\label{bkm:Ref329594818}
\textit{ay} Surprise: \\
\textbf{Ay},  bugnaw  man  san  lima  no  yan  a. \\\smallskip
\gll \textbf{Ay},  bugnaw  man  san  lima  no  yan  a. \\
\textsc{surp}  cold  \textsc{emph}  \textsc{d}2\textsc{nabs}  hand  2\textsc{s.gen}  \textsc{def.m}  truly \\
\glt ‘Oh, your hand is so cold truly.’
\z
\ea
\textbf{Ay}  sus  uļa  swirti  na  mga  gubat.  Danen  ya  ubos  ta  tumba. \\\smallskip
\gll \textbf{Ay}  sus  uļa  swirti  na  mga  gubat.  Danen  ya  ubos  ta  tumba. \\
\textsc{surp}   \textsc{inj}  \textsc{neg.r}  luck  \textsc{lk}  \textsc{pl}  raider  3\textsc{p.nabs}  \textsc{def.f}  all  \textsc{nabs}  fall.over \\
\glt ‘Oh, there were no lucky raiders. They all fell over.’ (This is a story about raiders who attacked the people on Cagayancillo long ago. It is surprising that the raiders fell over and were killed by the Kagayanens because the expected thing is that the raiders would kill the Kagayanens since they had many weapons.) [EMWN-T-07 4.3]
\z
\ea
\label{bkm:Ref118535037}
Na  iran  an  na  masigKagayanen  dya  naabutan  ta  malised o  malain  na  mga  betang,  \textbf{ay},  uļa  en  danen  an  en  pagbalikid. \\\smallskip
\gll Na  iran  an  na  masig-Kagayanen  dya  na-abot-an  ta  ma-lised o  ma-lain  na  mga  betang,  \textbf{ay},  uļa  en  danen  an  en  pag-balikid. \\
\textsc{lk}  3\textsc{p.gen}  \textsc{def.m}  \textsc{lk}  fellow-Kagayanen  \textsc{d}4\textsc{loc}  \textsc{a.hap.r}-arrive-\textsc{apl}  \textsc{nabs}  \textsc{adj}-distress or  \textsc{adj}-bad  \textsc{lk}  \textsc{pl}  thing  \textsc{surp}  \textsc{neg.r}  \textsc{cm}  3\textsc{p.gen}  \textsc{def.m}  \textsc{cm}  \textsc{nr.act}-look.back \\
\glt `When their fellow Kagayanens there come upon distressful or bad things, \textbf{oh}, they don’t ever look back.’ (Look back here implies to pay attention and to help. It is a surprise because it is cultural for people to always want to help each other.) [JCOB-L-02 12.4]
\z
\ea 
\textit{aroy} Sympathy: \\
\textbf{Aroy},  pirdien  a  gid  kani  ta  umang! \\\smallskip
\gll \textbf{Aroy},  pirdi-en  a  gid  kani  ta  umang! \\
too.bad  defeat-\textsc{t.ir}  1\textsc{s.abs}  \textsc{int}  later  \textsc{nabs}  hermit.crab \\
\glt ‘Too bad, hermit crab will really defeat me later!’\label{bkm:Ref329594828} [JCON-L-08 43.6]
\z

\newpage
\ea
\label{bkm:Ref118535410}
Pagsangga  ta  baked  ya  na  manunggoļ,  \textbf{aroy},  sikad  batyag din  na  sakit  naan  ta  takong  din  an. \\\smallskip
\gll Pag-sangga  ta  baked  ya  na  manunggoļ,  \textbf{aroy},  sikad  batyag din  na  sakit  naan  ta  takong  din  an. \\
\textsc{nr.act}-bump.into  \textsc{nabs}  big  \textsc{def.f}  \textsc{lk}  limestone  too.bad  very  feel
3\textsc{s.erg}  \textsc{lk}  pain  \textsc{spat.def}  \textsc{nabs}  forehead  3\textsc{s.gen}  \textsc{def.m} \\
\glt `After bumping into a big limestone, \textit{oh no!} the pain he felt on his forehead was much!’ [JCON-L-07 4.6]
\z

\ea 
\label{bkm:Ref118730619}
\textit{ba} Irony and sarcasm: \\
Bellay   \textbf{ba}  bag-o  utod  buok  din  an. \\\smallskip
\gll Bellay   \textbf{ba}  bag-o  utod  buok  din  an. \\
difficult  \textsc{irn}  newly  cut  hair  3\textsc{s.gen}  \textsc{def.m} \\
\glt ‘Oh so difficult, his/her hair is newly cut.’ 
\z

In \REF{bkm:Ref118730619} the speaker is referring to the hearer’s hair. This is clearly irony, since it is not difficult to get a haircut, but rather is a good thing. Many times in irony the third person is used in place of second person.
\ea
\label{bkm:Ref118730658}
\textit{di} Disapproval/objection: \\
\textbf{Di}  pinsaran  ta  duma  ya  na  pigado  en  na  uļa  en  pagkaan… \\\smallskip
\gll \textbf{Di}  \emptyset{}-pinsar-an  ta  duma  ya  na  pigado  en  na  uļa  en  pagkaan… \\
\textsc{inj/rq}  \textsc{t.ir}-think-\textsc{apl}  \textsc{nabs}  some  \textsc{def.f}  \textsc{lk}  difficult  \textsc{cm}  \textsc{lk}  \textsc{neg.r}  \textsc{cm}  food \\
\glt ‘What else, tsk, some think that (it is) difficult now not having food…' (This occurs after an explanation about how the land is rocky and the portions small and so making a living from it is difficult.) [MOOE-C-01 43.7]
\z

The form \textit{dukwa} ‘tentatively sometime’, is used as a polite way to request something or suggest something should happen at some time. Example \REF{bkm:Ref480870985} is a typical expression used when taking leave of others. One does not want to sound too anxious to leave, and \textit{dukwa} adds an appropriate degree of tentativeness to the proposition.

\ea 
\label{bkm:Ref118730750}\label{bkm:Ref480870985}
\textbf{Dukwa}  a  nang  en  muli. \\\smallskip
\gll
\textbf{Dukwa}  a  nang  en  m-uli. \\
possibly.some.time  1\textsc{s.abs}  only/just  \textsc{cm}  \textsc{i.v.ir}-go.home \\
\glt ‘Possibly some time I will just go home.’
\z
\ea
Sigi  kaw  taan  \textbf{dukwa}  tangkis  daw  magkitaay  kaw. \\\smallskip
\gll Sigi  kaw  taan  \textbf{dukwa}  tangkis  daw  mag-kita-ay  kaw. \\
continually  2\textsc{p.abs}  maybe  possibly.some.time  grin  if/when  \textsc{i.ir}-see-\textsc{rec}    2\textsc{p.abs} \\
\glt ‘You will maybe continually possibly be grinning \textbf{some} \textbf{time} when you see each other. (This is a letter to a person far away. The letter writer is imagining the addressee seeing people she had not seen for a while.) [PBWL-C-05 4.2]
\z
\ea
\label{bkm:Ref118730755}
Isya  \textbf{dukwa}  dakmeļ  na  libro  basaen  ta  duma  na  ittaw \\\smallskip
\gll Isya  \textbf{dukwa}  dakmeļ  na  libro  basa-en  ta  duma  na  {ittaw ...} \\
one  possibly.some.time  thick  \textsc{lk}  book  read-\textsc{t.ir}  \textsc{nabs}  some  \textsc{lk}  person {}\\
\glt ‘\textbf{Possibly} \textbf{some} \textbf{time} one thick book is what some people will read ….’ (This is part of a speech introducing a Kagayanen writing contest describing all the different things one can write about and trying to encourage the people to write stories in Kagayanen.) [SFOE-T-06 2.6]
\z

There is an interjection \textit{i} that occurs at the end of a unit. It seems to add a sense of veridicality (high likelihood of truth) or sympathy. This utterance-final \textit{i} is different in function from the definite demonstrative marker inside an RP. It is also quite different in function from the second position \textit{i} that indicates speaker's attitude. A full analysis of the use of this \textit{i} in discourse is needed to fully understand its communicative function. Here we provide a few examples from the corpus.

\ea 
\label{bkm:Ref118731059}
\textit{i} Strong assertion of truth: \\
Uyi  Pedro  \textbf{i},  iran  na  baļay  nasunog  \textbf{i}. \\\smallskip
\gll U-yi  Pedro  \textbf{i},  iran  na  baļay  na-sunog  \textbf{i}. \\
\textsc{emph-d1abs}  Pedro  \textsc{def.n}  3\textsc{p.gen}  \textsc{lk}  house  \textsc{a.hap.r}-burn  \textsc{inj} \\
\glt ‘This Pedro, their house was the one burned.’ (The house burned and the name Pedro was already mentioned.) [RZWN-T-02 3.2]
\z


\newpage

\ea
\textit{i} Sympathy: \\
Nabatyagan  ko  man  daw  ino  kasebe  ki  kaon  \textbf{i}. \\\smallskip
\gll Na-batyag-an  ko  man  daw  ino  ka-sebe  ki  kaon  \textbf{i}. \\
\textsc{a.hap.r}-feel-\textsc{apl}  1\textsc{s.erg}  \textsc{emph}  and  what  \textsc{nr}-sadness  \textsc{obl.p}  2\textsc{s.abs}  \textsc{inj} \\
\glt ‘I really feel how much sadness (it is) for you.’ (The addressee did not pass the classes to be a doctor.) [VBWL-T-08 2.2]
\z
\ea
Yaken  i  isab  nuļog  \textbf{i}. \\\smallskip
\gll Yaken  i  isab  na-uļog  \textbf{i}. \\
1sabs  \textsc{def.n} again  \textsc{a.hap.r}{}-fall  \textsc{inj} \\
\glt ‘I fell again.’  [CBWN-C-15 5.7]
\z
\ea
Mos,  Birnis  pa  man  \textbf{i}. \\\smallskip
\gll Mos,  Birnis  pa  man  \textbf{i}. \\
let’s-go  Friday  even  \textsc{emph}  \textsc{inj} \\
\glt ‘Let’s go, after all it is Friday.'  (Because Friday is believed to be a taboo day they expected something bad to happen to their relative who had gone out and so they went out to find him and make sure he was okay.) [CBWN-C-19 5.14]
\z
\ea
\label{bkm:Ref118731030}
Padumaan  a  din  \textbf{i}. \\\smallskip
\gll Pa-duma-an  a  din  \textbf{i}. \\
\textsc{t.r}-other-\textsc{apl}  1\textsc{s.abs}  3\textsc{s.erg}  \textsc{inj} \\
\glt ‘He was accompanying me.’ (This is about an unseen person who was already mentioned by another person, the speaker’s wife who had seen him.  His wife was worried about the speaker’s safety when he had gone night fishing and just arrived home. So the speaker tells her that the unseen person was with him and helped him get fish when he was fishing.) [JCWN-L-31 26.2]
\z
\ea
\label{bkm:Ref114565677}
\textit{inday} Uncertainty, ‘I don’t know’: \\
\textbf{Inday!}  Uļa  a  naļam  daw  indya  Pedro  ya. \\\smallskip
\gll \textbf{Inday!}  Uļa  a  na-aļam  daw  indya  Pedro  ya. \\
I.don’t.know  \textsc{neg.r}  1\textsc{s.abs}  \textsc{a.hap.r}-know  if/when  where  Pedro  \textsc{def.f} \\
\glt ‘I don’t know! I do not know where Pedro is.’
\z

\newpage

\ea
\label{bkm:Ref118731240}
\textit{oy} Attention getter: \\
\textbf{Oy},  indi  ka  galin  imo? \\\smallskip
\gll \textbf{Oy},  indi  ka  ga-alin  imo? \\
hey  where  2\textsc{s.abs}  \textsc{i.r}-from  \textsc{emph} \\
\glt ‘Hey, where did you come from?’
\z

The word \textit{ta} in clause-initial position functions both as an interjection expressing a mild rebuke, and as a discourse particle expressing a return to the main thread of the text or conversation. The second usage is described and illustrated in \sectref{bkm:Ref118460437}. Examples \REF{bkm:Ref118798805} and \REF{bkm:Ref118798909} illustrate three instances of \textit{ta} as an interjection expressing a mild rebuke:

\ea 
\label{bkm:Ref118798805}\label{bkm:Ref329594868}
\textit{ta} Mild rebuke: \\
\textbf{Ta},  daw  gapati  ka  gina,  di,  uļa  ka  inta nabunaļ.  \textbf{Ta,}  isaben  no  pa? \\\smallskip
\gll \textbf{Ta},  daw  ga-pati  ka  gina,  di,  uļa  ka  inta na-bunaļ.  \textbf{Ta,}  isab-en  no  pa? \\
so  if/when  \textsc{i.r}-listen/obey  2\textsc{s.abs}  earlier  \textsc{inj/rq}  \textsc{neg.r}  2\textsc{s.ab}\textsc{s}  \textsc{opt}
\textsc{a.hap.r}-spank  so  again-\textsc{t.ir}  2si\textsc{erg}  \textsc{inc} \\
\glt `\textbf{So}, if you had obeyed earlier, what else, you would not have been spanked. \textbf{So}, will you do it again?' [EMWE-T-01 1.11-12]
\z
\ea 
\label{bkm:Ref118798909}
\textbf{Ta},  nuon  no  ya,  patay  ya  en. \\\smallskip
\gll \textbf{Ta},  ino-en  no  ya,  patay  ya  en. \\
so  what-\textsc{t.ir}  2\textsc{s.erg}  \textsc{att}  dead  \textsc{def.f}  \textsc{cm} \\
\glt ‘So, what will you do (with that), it is dead now?’ [MEWN-T-02 5.7]
\z

\section{The general pro-form \textit{kwa}}
\label{bkm:Ref113971891}
The word \textit{kwa} ‘what-do-you-call-it’ can take the place of noun roots \REF{bkm:Ref329764949} or verb roots \REF{bkm:Ref441497341}, when the speaker cannot immediately think of the correct word. Also, it is often used in contexts when what is said is sensitive or embarrassing to the addressee. Thus \textit{kwa} may be considered a marker of a kind of indirection.

\ea 
\label{bkm:Ref329764949}
Pagubba  danen  en  \textbf{kwa}  ya,  kaoy  na  mga  darko. \\\smallskip
\gll Pa-gubba  danen  en  \textbf{kwa}  ya,  kaoy  na  mga  darko. \\
\textsc{t.r}-ruin  3\textsc{p.erg}  \textsc{cm}  what.do.you.call.it  \textsc{def.f}  tree  \textsc{lk}  \textsc{pl}  big.\textsc{pl} \\
\glt ‘They destroyed the what-do-you-call-it the trees that were large.’ [PTOE-T-01 4.13]
\z
\ea
\label{bkm:Ref441497341}
Uļa  a  \textbf{gakwa}  ta  swildo  na  bakod. \\\smallskip
\gll Uļa  a  \textbf{ga-kwa}  ta  swildo  na  bakod. \\
\textsc{neg.r}  1\textsc{s.abs}  \textsc{i.r}-what.do.you.call.it  \textsc{nabs}  wage  \textsc{lk}  big \\
\glt ‘I am/was not what-do-you-call-it (wanting) a big salary.’ (This probably means that s/he is not trying to get a big salary, is not focusing on that.) [RCON-L-03 6.8]
\z

% \begin{verbatim}%%move bib entries to  localbibliography.bib
% \end{verbatim}warning

% % \setcounter{page}{445}
